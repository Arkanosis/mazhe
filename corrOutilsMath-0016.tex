% This is part of Exercices et corrigés de CdI-1
% Copyright (c) 2011
%   Laurent Claessens
% See the file fdl-1.3.txt for copying conditions.

\begin{corrige}{OutilsMath-0016}

	\begin{enumerate}
		\item
			Il s'agit de trouver n'importe quel vecteur dont le produit scalaire avec $\begin{pmatrix}
				1	\\ 
				2	\\ 
				3	
			\end{pmatrix}$ est nul. Par exemple le vecteur
			\begin{equation}
				v=\begin{pmatrix}
					1	\\ 
						-1/2	\\ 
					0	
				\end{pmatrix}
			\end{equation}
			fait l'affaire.
		\item
			Un vecteur $(x,y,z)$ est perpendiculaire à $(1,1,1)$ si le produit scalaire est nul. Dans ce cas, l'équation est
			\begin{equation}
				\begin{pmatrix}
					x	\\ 
					y	\\ 
					z	
				\end{pmatrix}\cdot
				\begin{pmatrix}
					1	\\ 
					1	\\ 
					1	
				\end{pmatrix}=x+y+z=0.
			\end{equation}
			Cela est un plan. C'est le plan perpendiculaire au vecteur $(1,1,1)$.
	\end{enumerate}

\end{corrige}
