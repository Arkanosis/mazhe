% This is part of Exercices et corrigés de CdI-1
% Copyright (c) 2011,2014
%   Laurent Claessens
% See the file fdl-1.3.txt for copying conditions.

\begin{corrige}{OutilsMath-0048}

    La première chose à faire est de calculer la dérivée de $f$ :
    \begin{equation}
        f'(x)=-\sin(x)\sin\big( x+\frac{ \pi }{ 4 } \big)+\cos(x)\cos\big( x+\frac{ \pi }{ 4 } \big).
    \end{equation}
    Le coefficient directeur de la droite que nous cherchons est donc donné par
    \begin{equation}
        f'\left( \frac{ \pi }{ 2 } \right)=-\sin\frac{ \pi }{2}\sin\frac{ 3\pi }{ 4 }+\cos\frac{ \pi }{2}\cos\frac{ 3\pi }{ 4 }=-\frac{ \sqrt{2} }{2}.
    \end{equation}
    L'équation de la droite est donc de la forme
    \begin{equation}
        y=-\frac{ \sqrt{2} }{2}x+b
    \end{equation}
    où la constante $b$ est encore à fixer. Nous fixons cette constante en imposant
    \begin{equation}
        y\left( \frac{ \pi }{2} \right)=f\left( \frac{ \pi }{2} \right).
    \end{equation}
    Étant donné que $f(\pi/2)=0$, nous devons poser
    \begin{equation}
        b=\frac{ \sqrt{2}\pi }{ 4 } 
    \end{equation}
    et par conséquent
    \begin{equation}
        y=-\frac{ \sqrt{2} }{2}x+\frac{ \sqrt{2}\pi }{ 4 }.
    \end{equation}

\end{corrige}
