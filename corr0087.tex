% This is part of Exercices et corrigés de CdI-1
% Copyright (c) 2011
%   Laurent Claessens
% See the file fdl-1.3.txt for copying conditions.

\begin{corrige}{0087}

\begin{enumerate}
\item Soit $\epsilon > 0$, on choisit $\delta = \frac\epsilon L$, on a ce qu'il faut.
\item On note que $\abs{a x + b - (a y + b)} = \abs{a}\abs{x-y}$. La plus petite constante est donc $L = \abs a$.
\item
  \begin{enumerate}
  \item On note que $\abs{\abs z - \abs {z^\prime}} \leq
    \abs{z-z^\prime}$ donc $L = 1$ convient. Ce choix est minimal
    (prendre $z = 2$ et $z^\prime = 1$).
  \item On note que $\abs{\bar z - \bar{z^\prime}} =
    \abs{\bar{z-z^\prime}} = \abs{z-z^\prime}$ donc $L = 1$ convient
    et est minimal.
  \item On note que $\abs{\Re z - \Re z^\prime} = \abs{\Re
      (z-z^\prime)} \leq \abs{z-z^\prime}$. Dès lors $L = 1$ convient
    et est minimal (prendre $z = 2$ et $z^\prime = 1$).
  \item Idem (prendre $z=2i$ et $z^\prime = i$).
  \item Cela découle directement de l'inégalité \eqref{eq:ex13} utilisée pour montrer la continuité.
\end{enumerate}

\end{enumerate}

\end{corrige}
