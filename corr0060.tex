% This is part of Exercices et corrigés de CdI-1
% Copyright (c) 2011
%   Laurent Claessens
% See the file fdl-1.3.txt for copying conditions.

\begin{corrige}{0060}

Le fait que l'ensemble des solutions soit un espace vectoriel découle de la linéarité de l'opération de dérivation.

Calculons maintenant les dérivées par rapport à $x$ et à $t$ $F(x,t)=\phi(x+ct)+\psi(x-ct)$. Cela est un exercice de dérivation partielle usuelle :
\begin{equation}
	\begin{aligned}[]
		\frac{ \partial F }{ \partial x }(x,t)&=\phi'(x+ct)+\psi(x-ct),\\
		\frac{ \partial^2F  }{ \partial x^2 }(x,t)&=\phi''(x+ct)+\psi''(x-ct).
	\end{aligned}
\end{equation}
De la même manière, mais en tenant compte du signe et du coefficient $c$, nous trouvons
\begin{equation}
	\begin{aligned}[]
		\frac{ \partial F }{ \partial t }(x,t)&=c\phi'(x+ct)-c\psi(x-ct),\\
		\frac{ \partial^2F  }{ \partial t^2 }(x,t)&=c^2\phi''(x+ct)+c^2\psi''(x-ct).
	\end{aligned}
\end{equation}
En comparant, nous trouvons directement que $\partial^2_tF=c^2\partial^2_xF$.

Posons
\begin{equation}
	G(\xi,\eta)=F\big(\frac{ 1 }{2}(\xi+\eta),\frac{1}{ 2c }(\xi-\eta) \big),
\end{equation}
tandis que, pour simplifier la notation, nous notons $A=\big( \frac{ 1 }{2}(\xi+\eta),\frac{1}{ 2c }(\xi-\eta)\big)$. Toujours exercice de dérivation partielle :
\begin{equation}
	\frac{ \partial G }{ \partial \xi }(\xi,\eta)=\frac{ \partial F }{ \partial x }(A)\cdot\frac{ 1 }{2}+\frac{ \partial F }{ \partial t }(A)\cdot\frac{1}{ 2c },
\end{equation}
et puis
\begin{equation}
	\frac{ \partial^2G  }{ \partial\eta\partial\xi }(\xi,\eta)=\frac{1}{ 4 }\frac{ \partial^2F  }{ \partial x^2 }(A)-\frac{1}{ 4c^2 }\frac{ \partial^2F  }{ \partial t^2 }(A).
\end{equation}
L'équation d'onde est donc en fait exactement
\begin{equation}		\label{EqOndeNouvVariable}
	\frac{ \partial^2G  }{ \partial \eta\partial\xi }=0.
\end{equation}
En particulier, $\partial_{\xi}G$ est une constante par rapport à $\eta$, nous écrivons donc 
\begin{equation}
	\frac{ \partial G }{ \partial \xi }=a(\xi).
\end{equation}
Étant donné que $G$ est $C^2$, nous savons que $a$ est $C^1$; par conséquent, cette fonction admet une primitive $C^2$ que nous notons $\phi$. L'équation  d'onde dans les nouvelles variables \eqref{EqOndeNouvVariable} a pour solution
\begin{equation}
	G(\xi,\eta)=\phi(\xi)+C
\end{equation}
où $C$ est une constante \emph{par rapport à $\xi$}, c'est à dire que $C$ peut dépendre de $\eta$. Au final,
\begin{equation}
		G(\xi,\eta)=\phi(\xi)+\psi(\eta),
\end{equation}
ce qu'il fallait démontrer.


\end{corrige}
