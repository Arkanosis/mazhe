% This is part of Mes notes de mathématique
% Copyright (c) 2012-2013
%   Laurent Claessens
% See the file fdl-1.3.txt for copying conditions.

%+++++++++++++++++++++++++++++++++++++++++++++++++++++++++++++++++++++++++++++++++++++++++++++++++++++++++++++++++++++++++++
\section{Dérivabilité au sens complexe}
%+++++++++++++++++++++++++++++++++++++++++++++++++++++++++++++++++++++++++++++++++++++++++++++++++++++++++++++++++++++++++++

Dans cette partie, nous désignons par \( \Omega\) un ouvert de \( \eC\). Une fonction \( f\colon \Omega\to \eC\) est $\eC$-dérivable si la limite
\begin{equation}
    \lim_{h\to 0} \frac{ f(a+h)-f(a) }{ h }
\end{equation}
existe. Dans ce cas, cette limite est la dérivée de \( f\).

Nous identifions \( \eR^2\) à \( \eC\) par l'application \( \varphi\colon \eR^2\to \eC\) l'application \( \varphi(x,y)=x+iy\).

\begin{lemma}
    Une application \( A\colon \eC\to \eC\) est \( \eC\)-linéaire si et seulement si sa matrice en tant qu'application \( \eR^2\to \eR^2\) est la de forme
    \begin{equation}
        \begin{pmatrix}
            \alpha    &   \beta    \\ 
            -\beta    &   \alpha    
        \end{pmatrix}.
    \end{equation}
\end{lemma}

\begin{proposition}
    Une fonction \( f\colon \eC\to \eC\) est $\eC$-dérivable au point \( z_0=x_0+iy_0\) si et seulement si la fonction \( F=\varphi^{-1}\circ f\circ \varphi\) est différentiable en \( (x_0,y_0)\) et si la matrice de \( dF\) est de la forme
    \begin{equation}
        dF=\begin{pmatrix}
            \alpha    &   \beta    \\ 
            -\beta    &   \alpha    
        \end{pmatrix},
    \end{equation}
    c'est à dire si \( dF\) fournit une application \( \eC\)-linéaire.
\end{proposition}

\begin{proof}
    Nous décomposons \( f\) en parties réelles et imaginaires :
    \begin{equation}
        f(x+iy)=P(x,y)+iQ(x,y)
    \end{equation}
    où \( P\) et \( Q\) sont des fonctions réelles. La jacobienne de \( F\) est la matrice
    \begin{equation}
        \begin{pmatrix}
            \frac{ \partial P }{ \partial x }    &   \frac{ \partial P }{ \partial y }    \\ 
            \frac{ \partial Q }{ \partial x }    &   \frac{ \partial Q }{ \partial y }    
        \end{pmatrix},
    \end{equation}
    et la condition dont nous parlons s'écrit comme le système
    \begin{subequations}
        \begin{numcases}{}
            \frac{ \partial P }{ \partial x }=\frac{ \partial Q }{ \partial y }\\
            \frac{ \partial P }{ \partial y }=-\frac{ \partial Q }{ \partial x}.
        \end{numcases}
    \end{subequations}
    Si \( F\) est différentiable en \( (x_0,y_0)\) alors nous avons
    \begin{equation}        \label{EqwlVfiR}
        F\big( (x_0,y_0)+(h,k) \big)=F(x_0,y_0)+dF_{(x_0,y_0)}\begin{pmatrix}
            h    \\ 
            k    
        \end{pmatrix}+s(| h |+| k |)
    \end{equation}
    où \( s\) est une fonction vérifiant \( \lim_{t\to 0} \frac{ s(t) }{ t }=0\). Soit
    \begin{equation}
        dF_{(x_0,y_0)}=\begin{pmatrix}
            \alpha    &   \beta    \\ 
            -\beta    &   \alpha    
        \end{pmatrix}.
    \end{equation}
    Si nous posons \( \sigma=\alpha-i\beta\) et \( w=h+ik\), l'équation \eqref{EqwlVfiR} s'écrit dans \( \eC\) sous la forme
    \begin{equation}        \label{EqYFmoiM}
        f(z_0+w)=f(z_0)+\sigma w+s(|w|),
    \end{equation}
    ce qui implique que \( f\) est $\eC$-dérivable en \( z_0\).

    Supposons maintenant que \( f\) soit $\eC$-dérivable en \( z_0\). Alors nous avons
    \begin{equation}
        f'(z_0)=\lim_{w\to 0} \frac{ f(z_0+w)-f(z_0) }{ w }=\sigma\in \eC,
    \end{equation}
    ce qui se récrit sous la forme
    \begin{equation}
        \lim_{w\to 0} \frac{ f(z_0+w)-f(z_0)-\sigma w }{ w }=0.
    \end{equation}
    Si nous posons \( z_0=x_0+iy_0\), \( w=h+ik\) et \( \sigma=\alpha-i\beta\) nous avons
    \begin{equation}
        \lim_{(h,k)\to (0,0)} \left| \frac{ F\big( (x_0,y_0)+(h,k) \big)-F(x_0,y_0)-\begin{pmatrix}
            \alpha    &   \beta    \\ 
            -\beta    &   \alpha    
        \end{pmatrix}\begin{pmatrix}
            h    \\ 
            k    
        \end{pmatrix}}{ | w | } \right| =0,
    \end{equation}
    ce qui signifie que \( F\) est différentiable et que sa différentielle est la matrice
    \begin{equation}    \label{EqMLtbLD}
       \begin{pmatrix}
           \alpha &   \beta    \\ 
           -\beta &   \alpha    
       \end{pmatrix}.
    \end{equation}
\end{proof}

La matrice \eqref{EqMLtbLD} est, vue dans \( \eR^2\), la matrice de multiplication dans \( \eC\) par \( \alpha-i\beta=f'(z_0)\). En d'autre termes, dans \( \eC\) nous avons
\begin{equation}
    df_{z_0}z=f'(z_0)z,
\end{equation}
et en particulier la différentielle est donnée par
\begin{equation}        \label{EqPropZOkfmO}
    df_{z_0}=f'(z_0)dz.
\end{equation}

Notons que la formule \eqref{EqYFmoiM} donne un \defe{développement limité}{développement!limité!fonction holomorphe} pour les fonctions holomorphes. Si \( f\) est holomorphe en \( z_0\) alors si \( z\) est dans un voisinage de \( z_0\), il existe une fonction \( s\colon \eR\to \eC\) telle que \( \lim_{t\to 0} s(t)/t=0\) et 
\begin{equation}    \label{EqptwBFG}
    f(z)=f(z_0)+f'(z_0)(z-z_0)+s(| z-z_0 |).
\end{equation}

Nous introduisons les opérateurs\nomenclature[Y]{\( \partial_z\),\( \partial_{\bar z}\)}{dérivées partielles d'une fonction complexe}
\begin{subequations}
    \begin{align}
        \frac{ \partial  }{ \partial z }=\partial=\frac{ 1 }{2}\left( \frac{ \partial  }{ \partial x }-i\frac{ \partial  }{ \partial y } \right)\\
        \frac{ \partial  }{ \partial \bar z }=\bar\partial=\frac{ 1 }{2}\left( \frac{ \partial  }{ \partial x }+i\frac{ \partial  }{ \partial y } \right)
    \end{align}
\end{subequations}
Si \( f\) est une fonction $\eC$-dérivable représentée par la fonction \( F=P+iQ\), les équations de Cauchy-Schwartz signifient que \( \Delta P=\Delta Q=0\), c'est à dire que la fonction \( f\) a des composantes harmoniques.


%+++++++++++++++++++++++++++++++++++++++++++++++++++++++++++++++++++++++++++++++++++++++++++++++++++++++++++++++++++++++++++
\section{Fonctions holomorphes}
%+++++++++++++++++++++++++++++++++++++++++++++++++++++++++++++++++++++++++++++++++++++++++++++++++++++++++++++++++++++++++++
\label{SecoLNvnO}

\begin{definition}  \label{DefMMpjJZ}
    Soit \( \Omega\) un ouvert dans \( \eC\). Une fonction \( f\colon \Omega\to \eC\) est \defe{holomorphe}{holomorphe}\index{fonction!holomorphe} si elle est \( C^1\) et \( \eC\)-dérivable sur \( \Omega\). 
\end{definition}

\begin{proposition}
    Une fonction \( f\colon \Omega\to \eC\) est $\eC$-dérivable en \( a\in\Omega\) si et seulement si elle est différentiable en \( a\) et si \( df_a\) est une similitude.
\end{proposition}

\begin{theorem} \label{ThokwIQwg}
    Si \( f\in C^1(\Omega)\) alors nous avons équivalence des faits suivants :
    \begin{enumerate}
        \item
            \( f\) est holomorphe sur \( \Omega\),
        \item
            \( f\) vérifie \( \partial_{\bar z}f=0\).
    \end{enumerate}
\end{theorem}
%TODO : une preuve.

\begin{lemma}       \label{LemtpEOmi}
    Si \( g\) est une fonction continue dans un ouvert \( \Omega\subset \eC\) et si \( g\) admet une primitive complexe sur \( \Omega\) alors 
    \begin{equation}
        \int_{\gamma}g(z)dz=0
    \end{equation}
    pour tout chemin fermé \( \gamma\) de classe \( C^1\) contenu dans \( \Omega\).
\end{lemma}

\begin{proof}
    Nommons \( G\) une primitive de \( g\). Par définition,
    \begin{subequations}
        \begin{align}
            \int_{\gamma}g&=\int_{\gamma}G'\\
            &=\int_0^1G'\big( \gamma(t) \big)\gamma'(t)dt\\
            &=\int_0^1 (G\circ g\gamma)'(t)dt\\
            &=G(\gamma(1))-G\big( \gamma(0) \big)\\
            &=0
        \end{align}
    \end{subequations}
    parce que le chemin est fermé : \( \gamma(0)=\gamma(1)\).
\end{proof}

\begin{lemma}[Goursat\cite{Holomorphieus}]  \label{LemwbwbUR}
    Soit \( \Omega\) un ouvert dans \( \eC\) et \( f\) une fonction continue sur \( \Omega\), holomorphe sur \( \Omega\) moins éventuellement un point (nommé \( z_1\in\Omega\)). Soit \( T\), un triangle\footnote{Nous considérons ici le triangle «plein».} fermé inclus à \( \Omega\). Alors nous avons
    \begin{equation}
        \int_{\partial T}f(z)dz=0.
    \end{equation}
\end{lemma}

\begin{proof}
    Nous notons \( \gamma=\partial T\). Dans la suite nous allons définir une suite de triangles \( T^{(n)}\) et nous noterons \( \gamma_n=\partial T^{(n)}\) avec une orientation que nous allons expliquer. Pour commencer nous posons \( T^{(0)}=T\) et \( \gamma_0=\partial T^{(0)}\).

    Nous considérons le cas \( z_1\notin T\), et nous posons
    \begin{equation}
        c=l(\gamma)^{-2}| \int_{\gamma}f |.
    \end{equation}
    Notre objectif est de montrer que \( c=0\). Soit \( A,B,C\) les trois sommes du triangle; nous divisons le triangle de la façons suivante. D'abord nous considérons les points \( A',B,C'\) respectivement milieux de \( BC\), \( AC\) et \( AB\). En traçant le triangle \( A'B'C'\), nous construisons quatre triangles que nous nommons \( T^{(0)}_i\). Le théorème de Thalès assure que le périmètre de chacun des quatre triangles est la moitié du périmètre du grand triangle \( T\).

    Sur \( T\) nous choisissons l'orientation \( ABC\). De façon à être «compatible», nous choisissons les orientations \( AC'B'\), \( BA'C'\) et \( A'CB'\). La somme de ces trois triangles donne \( T\) plus le triangle \( A'C'B'\). Par conséquent nous choisissons sur le triangle central l'orientation (inverse) \( AB'C'\) de façon à avoir
    \begin{equation}
        \int_{\gamma}f=\sum_{i=1}^4\int_{\partial T^{(0)}_i}f.
    \end{equation}
    Cela implique que pour au moins un des quatre triangles (disons \( T^{(0)}_k\) pour fixer les idées) nous ayons
    \begin{equation}
        \int_{\partial T^{(0)}_k}f\geq \frac{1}{ 4 }\int_{\partial T^{(0)}}f
    \end{equation}
    Nous notons \( T^{(1)}\) ce triangle. Comme noté précédemment nous avons
    \begin{equation}
        l(\partial T^{(1)})=\frac{ 1 }{2}l(\partial T^{(0)}),
    \end{equation}
    et donc
    \begin{equation}
        l(\gamma_1)^{-2}| \int_{\gamma_1} |f=4l(\gamma_0)^{-2}| \int_{\gamma_1}f |\geq 4l(\gamma_0)^{-2}\frac{1}{ 4 }| \int_{\gamma_0}f |=c.
    \end{equation}
    En répétant le procédé nous construisons une suite de triangles \( T^{(n)}\) qui satisfont toujours
    \begin{equation}
        l(\partial T^{(n)})=\frac{1}{ 2^n }l(\partial T^{(0)}).
    \end{equation}
    Ces triangles forment une suite de fermés emboités dont le diamètre tend vers zéro. Leur intersection contient donc exactement un point (lemme \ref{LemdCOMQM}) que nous nommons \( z_0\) (et qui appartient évidemment à \( \Omega\)). Étant donné que \( f\) est holomorphe nous utilisons le développement limité \eqref{EqptwBFG} autour de \( z_0\) :
    \begin{equation}
        f(z)=f(z_0)+f'(z_0)(z-z_0)+s(| z-z_0 |)(z-z_0)
    \end{equation}
    avec \( \lim_{t\to 0} s(t)=0\). Nous posons \( g(z)=f(z_0)+f'(z_0)(z-z_0)\) et nous considérons \( \epsilon>0\). Soit \( \alpha>0\) tel que
    \begin{equation}
        | f(z)-g(z) |<\epsilon| z-z_0 |
    \end{equation}
    pour tout \( | z-z_0 |<\alpha\). Le \( \alpha\) à choisir pour obtenir cet effet est celui qui donne \( s(| z-z_0 |)<\epsilon\). Soit \( N\in \eN\) tel que \( l(\gamma_n)<\alpha\) pour tout \( n>N\). D'autre part, deux points dans un triangle sont toujours à distance moindre que la longueur d'un côté, donc pour tout \( z\in T^{(n)}\) nous avons \( | z-z_0 |<\alpha\) et par conséquent pour tout \( z\) dans \( T^{(n)}\) nous avons
    \begin{equation}
        | f(z)-g(z) |<\epsilon| z-z_0 |.
    \end{equation}
    Notons que la fonction \( g\) est une dérivée : c'est la dérivée de la fonction
    \begin{equation}
        G(z)=zf(z_0)+\frac{ 1 }{2}f'(z_0)(z-z_0)^2.
    \end{equation}
    Par conséquent nous avons
    \begin{equation}
        \int_{\gamma_n}g=0
    \end{equation}
    par le lemme \ref{LemtpEOmi}. Nous avons donc
    \begin{subequations}
        \begin{align}
            | \int_{\gamma_n}f |&=|\int_{\gamma_n}(f-g)|\\
            &\leq l(\gamma_n)\max\{ | f(z)-g(z) |\tq z\in T^{(n)} \}\\
            &\leq \epsilon l(\gamma_n)^2,
        \end{align}
    \end{subequations}
    et par conséquent
    \begin{equation}
        c\leq l(\gamma_n)^{-2}| \int_{\gamma_n}f |\leq \epsilon,
    \end{equation}
    ce qui signifie que \( c=0\) parce que \( \epsilon\) est arbitraire. Nous avons donc prouvé le lemme de Goursat dans le cas où le point de non holomorphie \( z_1\) est en dehors de \( T\).

    Si \( z_1\) est sur un côté, disons sur le côté \( AB\), alors nous considérons un vecteur \( v\in \eC\) tel que \( T_{\epsilon}=T+\epsilon v\) ne contienne \( z_1\) pour aucun \( \epsilon\). Le vecteur \( v=z_1-C\) fait par exemple l'affaire. En vertu du point précédent nous avons
    \begin{equation}
        \int_{\partial T_{\epsilon}}f=0
    \end{equation}
    pour tout \( \epsilon>0\). Étant donné que la fonction \( f\) est continue (y compris en \( z_1\)), l'intégrale sur \( \partial T\) est également nulle.

    Si maintenant le point \( z_1\) est à l'intérieur de \( T\) nous décomposons \( T\) en trois triangles ayant \( z_1\) comme sommet commun. Si nous considérons les orientations \( Az_1C\), \( ABz_1\) et \( BCz_1\), alors nous avons
    \begin{equation}
        \int_Tf=\int_{Az_1C}f+\int_{ABz_1}f+\int_{BCz_1}f,
    \end{equation}
    alors que par le point précédent les trois intégrales du membre de droite sont nulles.
\end{proof}

\begin{proposition}[\cite{Holomorphieus}]   \label{PrpopwQSbJg}
    Soit \( \Omega\) un ouvert étoilé et \( f\) une fonction holomorphe sur \( \Omega\) sauf éventuellement en un point \( z_1\) où \( f\) est seulement continue. Alors si \( \gamma\) est un chemin fermé dans \( \Omega\), nous avons
    \begin{equation}
        \int_{\gamma}f=0.
    \end{equation}
\end{proposition}

\begin{definition}
    Une fonction \( f\colon \Omega\to \eC\) est \defe{$\eC$-analytique}{analytique!au sens complexe} sur \( \Omega\) si pour tout \( z_0\in\Omega\), il existe une suite complexe \( (c_n)\) et \( r>0\) tels que
    \begin{equation}
        f(z)=\sum_{n=0}^{\infty} c_n(z-z_0)^n
    \end{equation}
    pour tout \( z\in B(z_0,r)\).
\end{definition}


\begin{proposition}
    Une application \( f\colon \Omega\to \eC\) est $C$-dérivable sur \( \Omega\) si et seulement si elle est différentiable et
    \begin{subequations}        \label{EqmblExI}
        \begin{numcases}{}
            \frac{ \partial u }{ \partial x }=\frac{ \partial v }{ \partial y }\\
            \frac{ \partial u }{ \partial y }=-\frac{ \partial v }{ \partial x }
        \end{numcases}
    \end{subequations}
    où \( f(x+iy)=u(x,y)+iv(x,y)\).
\end{proposition}
Les équations \eqref{EqmblExI} sont les équations de \defe{Cauchy-Riemann}{Cauchy-Riemann}.

\begin{proof}
    La différentielle de \( f\colon \eR^2\to \eR^2\) est donnée par la matrice
    \begin{equation}        \label{EQwtagsz}
        T=\begin{pmatrix}
            \partial_xu(a)    &   \partial_yu(a)    \\ 
            \partial_xv(a)    &   \partial_yv(a)    
        \end{pmatrix}.
    \end{equation}
    Cette matrice est une similitude si et seulement si les équations de Cauchy-Riemann sont satisfaites. En effet si \( 1=\begin{pmatrix}
        1    \\ 
        0    
    \end{pmatrix}\) et \( i=\begin{pmatrix}
        0    \\ 
        1    
    \end{pmatrix}\), la matrice \( T\) est une similitude (écrivons \( \alpha+i\beta\) son coefficient) si
    \begin{subequations}
        \begin{numcases}{}
            T(1)=\alpha+i\beta\\
            T(i)=-\beta+i\alpha,
        \end{numcases}
    \end{subequations}
    c'est à dire
    \begin{equation}
        T=\begin{pmatrix}
            \alpha    &   -\beta    \\ 
           \beta    &   \alpha    
        \end{pmatrix}.
    \end{equation}
    Identifier cette matrice à \eqref{EQwtagsz} fournit le résultat annoncé.
\end{proof}

\begin{proposition}
    Une fonction \( f\colon \Omega\to \eC\) est $C$-dérivable si et seulement si elle est différentiable et \( df_a\) est une similitude.
\end{proposition}


\begin{proposition}     \label{PropRZCKeO}
    Si \( f(z)=\sum_na_nz^n\) a pour rayon de convergence \( R\), alors \( f\) est $\eC$-dérivable et nous pouvons dériver terme à terme dans la boule ouverte \( B(0,R)\).
\end{proposition}

\begin{proof}
    Cela est exactement la proposition \ref{ProptzOIuG}.
\end{proof}

%---------------------------------------------------------------------------------------------------------------------------
\subsection{Théorème de Cauchy}
%---------------------------------------------------------------------------------------------------------------------------

Cette sous-section veut prouver le théorème de Cauchy. Comme d'habitude, une référence qui ne peut pas rater est \cite{Holomorphieus}.

\begin{theorem}[formule de Cauchy]    \label{ThoUHztQe}
    Soit \( \Omega\) ouvert dans \( \eC\), \( z_0\in \Omega\) et \( f\), une fonction holomorphe sur \( \Omega\). Soit \( r>0\) tel que \( B(z_0,r)\subset \Omega\). Alors pour tout \( z\in B(z_0,r)\) nous avons
    \begin{equation}    \label{EqPzUABM}
        f(z)=\frac{1}{ 2\pi i }\int_{\partial B(z_0,r)}\frac{ f(\omega) }{ \omega-z }d\omega.
    \end{equation}
\end{theorem}
\index{formule!de Cauchy}
\index{Cauchy!formule}

\begin{proof}
    Soit \( z\in B(z_0,r)\) et considérons la fonction
    \begin{equation}
        g(\omega)=\begin{cases}
            \frac{ f(\omega)-f(z) }{ \omega-z }    &   \text{si \( \omega\neq z\)}\\
            f'(z)    &    \text{si \( \omega=z\)}.
        \end{cases}
    \end{equation}
    Cette fonction est holomorphe sur \( B(z_0,r)\setminus\{ z \}\). Étant holomorphe sur \( B(z_0,r)\setminus\{ z \}\) et continue en \( z\), elle vérifie la proposition \ref{PrpopwQSbJg} et nous avons
    \begin{equation}
        \int_{\gamma}g=0
    \end{equation}
    où \( \gamma\) est le cercle de centre \( z_0\) et de rayon \( r\). Nous avons donc
    \begin{equation}
        0=\int_{\gamma}\frac{ f(\omega) }{ \omega-z }-\int_{\gamma}\frac{ f(z) }{ \omega-z },
    \end{equation}
    et ayant déjà calculé la seconde intégrale dans l'exemple \ref{ExradygL} nous en déduisons
    \begin{equation}
        \int_{\gamma}\frac{ f(\omega) }{ \omega-z }d\omega=2\pi if(z),
    \end{equation}
    ce qu'il fallait.
\end{proof}

\begin{theorem}     \label{ThomcPOdd}
    Soit \( \Omega\) ouvert dans \( \eC\) et \( f\), holomorphe sur \( \Omega\). Soient encore \( z_0\in \Omega\) et \( r_0\) tel que \( B(z_0,r_0)\subset \Omega\). Alors sur \( B(z_0,r_0)\), la fonction \( f\) s'écrit
    \begin{equation}
        f(z)=\sum_{n=0}^{\infty}a_n(z-z_0)^n.
    \end{equation}
    De plus nous avons
    \begin{equation}
        a_n=\frac{ f^{(n)}(z_0) }{ n! }=\frac{1}{ 2\pi i }\int_{\gamma}\frac{ f(\omega) }{ (\omega-z_0)^{n+1} }d\omega
    \end{equation}
    où \( \gamma=\partial B(z_0,r)\) avec \( | z-z_0 |<r<r_0\).

    En particulier \( f\) est infiniment dérivable.
\end{theorem}

\begin{proof}
    Soit \( r>0\) tel que \( | z-z_0 |<r<r_0\). La formule de Cauchy (théorème \ref{ThoUHztQe}) nous dit que
    \begin{equation}
        f(z)=\frac{1}{ 2\pi i }\int_{\gamma}\frac{ f(\omega)}{ \omega-z }d\omega
    \end{equation}
    où \( \gamma=\partial B(z_0,r)\). Nous pouvons paramétrer ce chemin par \( \omega=z_0+r e^{i\theta}\) et \( \theta\in \mathopen[ 0 , 2\pi \mathclose]\). Nous avons
    \begin{subequations}
        \begin{align}
            f(z)&=\frac{1}{ 2\pi i }\int_0^{2\pi}\frac{ f(z_0+r e^{i\theta}) }{ z_0+r e^{i\theta}-z }ri e^{i\theta}d\theta\\
            &=\frac{1}{ 2\pi }\int_0^{2\pi}\frac{ f(z_0+r e^{i\theta}) }{ 1- e^{-i\theta}(z-z_0)/r }d\theta.
        \end{align}
    \end{subequations}
    Nous pouvons développer l'intégrante en puissance de \( (z-z_0)\) en utilisant la formule \ref{EqVmuaqT}. Ici le rôle de \( x\) est tenu par
    \begin{equation}
        e^{-i\theta}(z-z_0)/r
    \end{equation}
    dont le module est bien plus petit que \( 1\), par hypothèse sur \( r\). Nous avons donc
    \begin{equation}
        f(z)=\frac{1}{ 2\pi }\int_0^{2\pi}\sum_{n=0}^{\infty}f(z_0+r e^{i\theta}) e^{-in\theta}r^{-n}(z-z_0)^nd\theta.
    \end{equation}
    L'art est maintenant de permuter la somme et l'intégrale. Pour cela nous remarquons que ce qui se trouve dans la somme est majoré en module par
    \begin{equation}        \label{EqbykTLD}
        M\left| \frac{ z-z_0 }{ r } \right|^n
    \end{equation}
    où \( M\) est le maximum de \( | f |\) sur \( \gamma\). La borne \eqref{EqbykTLD} ne dépend pas de \( \theta\); par conséquent la convergence de la somme est uniforme en \( \theta\) par le critère de Weierstrass (théorème \ref{ThoCritWeierstrass}). Le théorème \ref{ThoCciOlZ} s'applique\footnote{Étant donné que nous savions déjà que la somme était une fonction intégrable, nous sommes loin d'avoir utilisé toute la puissance du théorème.} et nous pouvons permuter la somme avec l'intégrale.

    Ce que nous trouvons est que
    \begin{equation}
        f(z)=\sum_{n=0}^{\infty}a_n(z-z_0)^n
    \end{equation}
    où
    \begin{equation}
        a_n=\frac{1}{ 2\pi }\int_0^{2\pi}f(z_0+r e^{i\theta}) e^{-in\theta}r^{-n}d\theta=\frac{1}{ 2\pi i }\int_{\gamma}\frac{ f(\omega) }{ (\omega-z_0)^{n+1} }.
    \end{equation}
    Cette formule est valable pour \( | z-z_0 |<r\). Sur cette boule, la fonction est donc une série entière Le théorème de Taylor \ref{ThoTGPtDj} nous permet donc d'affirmer que \( f\) est partout infiniment continument dérivable (parce que en chaque point on a un voisinage sur lequel c'est vrai), et d'identifier les coefficients (qui, eux, ne sont valables que localement) sous la forme
    \begin{equation}
        a_n=\frac{ f^{(n)}(z_0) }{ n! }.
    \end{equation}
\end{proof}

\begin{corollary}       \label{CorwfHtJu}
    Soit \( f\) une fonction continue sur un ouvert \( \Omega\) telle que pour toute boule \( B(a,r)\) contenue dans \( \Omega\), nous ayons
    \begin{equation}
        f(a)=\frac{1}{ 2\pi i }\int_{\partial B(a,r)}\frac{ f(\xi) }{ \xi-a }d\xi.
    \end{equation}
    Alors \( f\) est holomorphe.
\end{corollary}

\begin{proof}
    Il suffit de recopier la démonstration du théorème \ref{ThomcPOdd} pour savoir que \( f\) se développe en série de puissances et est donc en particulier dérivable.
\end{proof}

\begin{proposition}\label{PropZOkfmO}
    Une fonction continue \( f\) est holomorphe si et seulement si la \( 1\)-forme différentielle \( f(z)dz\) est localement exacte.
\end{proposition}

\begin{proof}
    Si \( f\) est holomorphe, alors nous avons vu que \( f\) était différentiable et que \( df_{z}=f(z)dz\) par la formule \ref{EqPropZOkfmO}.

    Dans le sens inverse, supposons que \( f(z)dz\) est localement exacte, et soit \( F\) telle que \( dF=f(z)dz\). Ce que nous allons faire est montrer que la dérivée de \( F\) existe et vaut \( f\). En effet, la définition de la différentielle nous dit que
    \begin{equation}
        \lim_{h\to 0} \left| \frac{ F(z+h)-F(z)-dF_z(h) }{ h } \right| =0.
    \end{equation}
    La limite vaut évidemment encore zéro si nous enlevons les modules :
    \begin{subequations}
        \begin{align}
            0&=\lim_{h\to 0} \frac{ F(z+h)-F(z)-f(z)h }{ h }\\
            &=\lim_{h\to 0} \frac{ F(z+h)-F(z) }{ h }-f(z).
        \end{align}
    \end{subequations}
    Donc \( F'=f\). Cela montre que \( F\) est dérivable et donc holomorphe. En conséquence du théorème \ref{ThoUHztQe}, \( F\) est infiniment dérivable et \( f\) l'est alors aussi. La fonction \( f\) est donc holomorphe\footnote{Dire que la dérivée d'une fonction holomorphe est holomorphe est un raisonnement classique.}.
\end{proof}

%--------------------------------------------------------------------------------------------------------------------------- 
\subsection{Lacets, indice et homotopie}
%---------------------------------------------------------------------------------------------------------------------------

\begin{definition}
    Soit \( \gamma\) un chemin fermé\footnote{Par abus de langage, nous désignerons par \( \gamma\) à la fois le chemin et son image.} dans \( \eC\). L'\defe{indice}{indice!d'une courbe dans $\eC$} de la courbe \( \gamma\) est la fonction
    \begin{equation}
        \begin{aligned}
            \Ind_{\gamma}\colon \eC\setminus \gamma&\to \eZ \\
            z&\mapsto \frac{1}{ 2\pi i }\int_{\gamma}\frac{ d\omega }{ \omega-z }. 
        \end{aligned}
    \end{equation}
    Un chemin continu et fermé (au sens \( \gamma(1)=\gamma(0)\)) est un \defe{lacet}{lacet}.
\end{definition}

\begin{definition}  \label{DefECnFJQp}
    Si \( \gamma_1\) et \( \gamma_2\) sont deux lacets en \( x_0\in X\) (un espace topologique), une \defe{équivalence d'homotopie}{équivalence!homotopie} est une application \( f\colon \mathopen[ 0 , 1 \mathclose]\times \mathopen[ 0 , 1 \mathclose]\to X\) telle que
    \begin{enumerate}
        \item
            \( f(0,t)=\gamma_1(t)\) pour tout \( t\);
        \item
            \( f(1,t)=\gamma_1(t)\) pour tout \( t\);
        \item
            pour chaque \( t\in \mathopen[ 0 , 1 \mathclose]\), l'application \( s\mapsto f(s,t)\) est continue;
        \item
            pour chaque \( s\in \mathopen[ 0 , 1 \mathclose]\), l'application \( t\mapsto f(s,t)\) est un lacet basé en \( x_0\).
    \end{enumerate}
\end{definition}

\begin{theorem}     \label{ThoDYQQXZ}
    \begin{enumerate}
        \item
            La fonction \( \Ind_{\gamma}\) est continue et prend des valeurs entières.
        \item
            La fonction indice est constante sur chaque composante connexe de \( \eC\setminus \gamma\) et est nulle sur la composante non bornée.
    \end{enumerate}
\end{theorem}
%TODO : une preuve. Si cette preuve ne demande pas vraiment d'analyse complexe, alors on peut la mettre plus haut et éventuellement remettre le théorème de Brouwer \ref{ThoLVViheK} à sa place.
Le second point est en partie la proposition \ref{PropHSjJcIr}.
\index{connexité!indice d'une courbe}

\begin{example} \label{ExradygL}
    Si \( \gamma\) est un cercle de centre \( z_0\in \eC\) et de rayon \( r\), alors 
    \begin{equation}
        \Ind_{\gamma}(z)=\begin{cases}
            2\pi i    &   \text{si \( z\in B(z_0,r)\)}\\
            0    &    \text{sinon}.
        \end{cases}
    \end{equation}
    La seconde ligne provient directement du théorème \ref{ThoDYQQXZ}. Pour la première, le cercle \( \gamma\) se paramètre par
    \begin{equation}
        \gamma(\theta)=z_0+r e^{i\theta},
    \end{equation}
    et l'intégrale vaut
    \begin{equation}
        \int_{\gamma}\frac{ d\omega }{ \omega-z_0 }=\int_0^{2\pi}\frac{1}{ r e^{i\theta} }ir e^{i\theta}d\theta=2\pi i.
    \end{equation}
    L'indice de ce chemin va évidemment jouer un rôle particulier dans la suite.
\end{example}

\begin{theorem}[Cauchy, version homotopique\cite{ADEyNiz}]
    Soit \( \Omega\) un ouvert de \( \eC\) et \( f\) une fonction holomorphe sur \( \Omega\). Si \( \gamma_1\) et \( \gamma_2\) sont deux lacets homotopes de classe \( C^1\) dans \( \Omega\), alors
    \begin{equation}
        \int_{\gamma_1}f(z)dz=\int_{\gamma_2}f(z)dz.
    \end{equation}
\end{theorem}

\begin{corollary}[\cite{ADEyNiz}]   \label{CorGZXzuZR}
    Soit \( a\in \eC\) ainsi que deux chemins \( \gamma_1\) et \( \gamma_2\) homotopes dans \( \eC\setminus\{ a \}\). Alors \( \Int(\gamma_1,a)=\Ind(\gamma_2,a)\).
\end{corollary}
Il y a aussi des choses sur l'indice dans \cite{Holomorphieus}.

%--------------------------------------------------------------------------------------------------------------------------- 
\subsection{Théorème de Brouwer}
%---------------------------------------------------------------------------------------------------------------------------

\begin{theorem}[Brouwer en dimension \( 2\)\cite{KXjFWKA}]     \label{ThoLVViheK}
    Soit \( \mB\) la boule unité fermée de \( \eR^2\). Alors toute application continue de \( \mB\) dans elle-même admet un point fixe.
\end{theorem}
\index{théorème!Brouwer!dimension \( 2\)}
\index{connexité!utilisation!Brouwer}
\index{théorème!point fixe!Brouwer}

\begin{proof}
    Supposons que la fonction \( f\in C^0(\mB,\mB)\) n'admette pas de points fixes sur \( \mB=\overline{ B(0,1) }\). Pour \( x\in \mB\) nous notons \( g(x)\) l'intersection entre \( \partial \mB\) et la demi-droite allant de \( f(x)\) vers \( x\). C'est bien parce que \( f\) n'a pas de points fixes que \( g\) est bien définie.

    En reprenant le même début de la preuve de la proposition \ref{PropDRpYwv} nous savons que la fonction
    \begin{equation}
        \begin{aligned}
            g\colon \overline{ B(0,1) }&\to \partial B(0,1) \\
            x&\mapsto \lambda(x)\big( x-f(x) \big)+f(x) 
        \end{aligned}
    \end{equation}
    est continue. De plus \( g(x)=x\) sur \( \partial B(0,1)\). Nous allons montrer qu'une telle fonction\footnote{Qui est nommée \emph{rétraction} de la sphère sur elle-même.} ne peut pas exister.

    Pour \( s\in\mathopen[ 0 , 1 \mathclose]\) nous paramétrons le cercle \( \partial B(0,s)\) par
    \begin{equation}
        \begin{aligned}
            x_s\colon \mathopen[ 0 , 1 \mathclose]&\to \partial B(0,1) \\
            t&\mapsto \big( s\cos(2\pi t),s\sin(2\pi t) \big). 
        \end{aligned}
    \end{equation}
    Ensuite nous considérons les chemins
    \begin{equation}
        \begin{aligned}
            \gamma_s\colon \mathopen[ 0 , 1 \mathclose]&\to \partial B(0,s) \\
            t&\mapsto g\circ x_s. 
        \end{aligned}
    \end{equation}
    L'application \( \gamma_s\) est continue et \( \gamma_s(0)=\gamma_s(1)\). Les chemins \( \gamma_s\) sont des lacets; nous nous intéressons maintenant à l'indice au point \( 0\) de \( \gamma_0\) et \( \gamma_1\). D'une part \( \gamma_0(t)=g(0)\) (lacet constant) et \( \gamma_1(t)= e^{2i\pi t}\) (parce que \( g(x)=x\) sur le bord). Nous avons donc
    \begin{equation}
        \Ind_{\gamma_0}(0)=\frac{1}{ 2\pi i }\Ind_{\gamma_0}\frac{ d\omega }{ \omega }=\frac{1}{ 2\pi i }\int_0^1\frac{ \gamma_0'(t) }{ \gamma_0(t) }dt=0,
    \end{equation}
    alors que
    \begin{equation}
        \Ind_{\gamma_1}(0)=\frac{1}{ 2\pi i }\int_0^1\frac{ 2i\pi e^{2i\pi t} }{  e^{2i\pi t} }dt=1.
    \end{equation}
    
    Nous considérons l'homotopie 
    \begin{equation}
        \begin{aligned}
            \gamma\colon \mathopen[ 0 , 1 \mathclose]\times \mathopen[ 0 , 1 \mathclose]&\to \overline{ B(0,1) } \\
            (s,t)&\mapsto \gamma_s(t)=(g\circ x_s)(t). 
        \end{aligned}
    \end{equation}
    Nous avons \( g(0)\neq 0\) parce que \( g\) prend ses valeurs sur le bord. Vu que c'est une équivalence d'homotopie\footnote{Définition \ref{DefECnFJQp}} entre \( \gamma_1\) et \( \gamma_2\), les indices devraient être égaux par le corollaire \ref{CorGZXzuZR}.
\end{proof}

%---------------------------------------------------------------------------------------------------------------------------
\subsection{Principe des zéros isolés}
%---------------------------------------------------------------------------------------------------------------------------

\begin{theorem}[Principe des zéros isolés \cite{Holomorphieus}]     \label{ThoukDPBX}
    Soit \( f\) une fonction holomorphe et \( a\), une zéro non isolé de \( f\). Alors \( f\) est nulle sur un voisinage de \( a\).
\end{theorem}
\index{principe!zéros isolés}

\begin{proof}
    Nous écrivons \( f\) sous la forme d'une série entière autour de \( a\) :
    \begin{equation}        \label{EqgrvfVl}
        f(z)=\sum_{n=0}^{\infty}c_n(z-a)^n
    \end{equation}
    valable sur une boule \( B(a,r)\). Soit \( c_m\) le premier coefficient non nul (si il n'existe pas c'est que \( f\) est nulle sur tout \( B(a,r) \) et alors le théorème est prouvé). Nous avons alors
    \begin{equation}
        f(z)=c_m(z-a)^m\big( 1+\sum_{k=1}^{\infty}d_k(z-a)^k \big)
    \end{equation}
    avec \( d_k=c_{m-k}\). Le rayon de convergence de la série \( \sum_k d_k(z-a)^k\) est le même que celui de \eqref{EqgrvfVl} parce que la suite \( d_kr^{m+k}\) reste bornée (critère d'Abel, lemme \ref{LemmbWnFI}). Si nous posons
    \begin{equation}
        g(z)=1+\sum_{k=1}^{\infty}d_k(z-a)^k,
    \end{equation}
    alors \( g\) est une fonction continue et \( g(a)=1\). De plus 
    \begin{equation}
        f(z)=c_m(z-a)^mg(z).
    \end{equation}

    Soit une suite \( (z_n)\) de zéros de \( f\) convergent vers \( a\). Étant donné que \( g\) est continue, nous devrions avoir \( \lim_{k\to\infty}g(z_k)=g(a)=1\), mais si \( f(z_k=0)\) avec \( z_k\neq a\), alors \( g(z_k)=0\). Cela est un paradoxe qui nous permet de conclure que si la suite \( z_n\) existe bien, alors \( f\) est identiquement nulle sur un voisinage, c'est à dire que tous les \( c_n\) sont nuls.
\end{proof}

\begin{corollary}
    Soit \( f\) une fonction holomorphe sur un ouvert connexe \( \Omega\). Si \( f\) s'annule sur un un ouvert (non vide) de \( \Omega\), alors \( f\) s'annule sur tout \( \Omega\).
\end{corollary}

\begin{proof}
    soit 
    \begin{equation}
        N=\{ z\in \Omega\tq f=0\text{ sur un ouvert autour de $z$} \}.
    \end{equation}
    Le fait que \( N\) soit ouvert est évident à partir de sa définition. Nous allons montrer que \( N\) est également fermé dans \( \Omega\), et donc conclure que \( N=\Omega\). Soit \( (z_n)\) une suite dans \( N\) convergente vers \( z\in \Omega\). Étant donné que \( f(z_n)=0\) et que \( f\) est continue, nous avons
    \begin{equation}
        f(z)=\lim_{n\to \infty} f(z_n)=0,
    \end{equation}
    ce qui fait de \( z\) un zéro non isolé de \( f\). Par conséquent le principe des zéros isolés (théorème \ref{ThoukDPBX}) nous enseigne que \( f\) s'annule dans un voisinage autour de \( z\), c'est à dire que \( z\in N\). L'ensemble \( N\) est donc fermé.
\end{proof}

%---------------------------------------------------------------------------------------------------------------------------
\subsection{Prolongement}
%---------------------------------------------------------------------------------------------------------------------------

\begin{proposition}
    Soit \( \Omega\), un ouvert de \( \eC\) et \( f\colon \Omega\to \eC\) une fonction holomorphe sur \( \Omega\setminus\{ a \}\) (\( a\in \Omega\)). Nous supposons qu'il existe \( r>0\) tel que \( f\) est bornée sur \( B(a,r)\cap\Omega\). Alors \( f\) se prolonge en une fonction holomorphe sur \( \Omega\).
\end{proposition}

\begin{proof}
    Nous définissons la fonction \( g\colon \Omega\to \eC\) par
    \begin{equation}
        g(z)=\begin{cases}
            (z-a)f(z)    &   \text{si \( z\neq a\)}\\
            0    &    \text{si \( z=a\)}.
        \end{cases}
    \end{equation}
    Sur \( \Omega\setminus\{ a \}\), la fonction \( g\) est holomorphe (produit de fonctions holomorphes), et elle est continue en \( a\). Par conséquent elle est holomorphe sur \( \Omega\). Nous la développons en série entière sur une boule \( B(a,r)\) :
    \begin{equation}
        g(z)=\sum_{n=0}^{\infty}c_n(z-a)^n.
    \end{equation}
    Nous avons \( g(a)=c_0=0\). Nous posons
    \begin{equation}
        \varphi(z)=\sum_{n=0}^{\infty}c_{n+1}(z-a)^n.
    \end{equation}
    Si \( z\neq a\), alors \( \varphi(z)=f(a)\) parce que \( \varphi(z)=g(z)/(z-a)\). Mais \( \varphi\) est continue en \( a\), et donc holomorphe en \( a\).

    La fonction \( \varphi\) est par conséquent un prolongement holomorphe de \( f\) en \( a\).
\end{proof}

%---------------------------------------------------------------------------------------------------------------------------
\subsection{Théorème de Runge}
%---------------------------------------------------------------------------------------------------------------------------

Le théorème que nous allons prouver n'est en réalité qu'une partie de ce qui est usuellement appelle le théorème de Runge.
\begin{theorem}[Théorème de Runge]\index{théorème!Runge}     \label{ThoMvMCci}
    Soit \( K\), un compact de \( \eC\) tel que \( \complement K\) soit connexe. Si \( a\in \complement K\) alors la fonction 
    \begin{equation}
        \varphi_a(z)=\frac{1}{ z-a }
    \end{equation}
    est limite uniforme de polynômes sur \( K\).
\end{theorem}
\index{connexité!théorème de Runge}
\index{approximation!polynômiale}

\begin{proof}
    Nous considérons \( P(K)\), l'adhérence des polynômes sur \( K\) pour la norme uniforme (sur \( K\)). Nous devons montrer que pour tout \( a\in \complement K\), la fonction \( \varphi_a\) est dans \( P(K)\). Pour cela nous considérons l'ensemble
    \begin{equation}
        A=\{ a\in\complement K\tq \varphi_a\in P(K) \}
    \end{equation}
    et nous allons montrer qu'il est à la fois non vide, ouvert et fermé dans le connexe \( \complement K\).

    Je répète : nous allons prouver l'ouverture et la fermeture \emph{pour la topologie de \( \complement K\)}. Nous n'allons pas prouver que \( A\) est un ouvert de \( \eC\). Ce qui sera par conséquent prouvé est que \( A=\complement K\).

    \begin{subproof}
    \item[Non vide] Soit \( R=\sup_{z\in K}| z |\) et \( a\in \complement K\) tel que \( | a |>R\). Nous avons
        \begin{equation}
                \varphi_a(z)=\frac{1}{ a }\frac{1}{ \frac{ z }{ a }-1 }
                =-\frac{1}{ a }\frac{1}{ 1-\frac{ z }{ a } }
                =-\frac{1}{ a }\sum_{k=0}^{\infty}\left( \frac{ z }{ a } \right)^k
                =\sum_{k=0}^{\infty}\frac{ z^k }{ a^{k+1} }.
        \end{equation}
        Ici la convergence de la série et sa limite sont assurées par le fait que \( | z/a |<1\) par choix de \( R\) et \( a\). La suite de polynômes
        \begin{equation}
            P_n(z)=\sum_{k=0}^n\frac{ z^k }{ a^{k+1} }
        \end{equation}
        converge uniformément sur \( B(0,R)\) et en particulier sur \( K\). Donc \( P_n\to \varphi_a\).

    \item[Fermé] 
            
        Nous allons montrer que la fermeture de \( A\) (dans \( \complement K\)) est inclue dans \( A\), et donc qu'elle est égale à \( A\) et donc que \( A\) est fermé. Par le lemme \ref{LemkUYkQt}, la fermeture de \( A\) dans \( \complement K\) est l'ensemble \( \bar A\cap\complement K\) où \( \bar A\) est la fermeture de \( A\) au sens usuel.

        Bref, soit \( a\in \bar A\cap\complement K\), et montrons que \( \varphi_a\in \overline{ P(K) }\). Vu que \( P(K)\) est déjà une fermeture, nous aurons en fait \( \varphi_a\in P(K)\) et donc \( a\in A\), ce qui signifierait que \( \bar A\cap\complement A=A\) et donc que \( A\) est fermé.

        Au travail.

        Soit \( (a_n)\in A\) une suite convergente vers \( a\). Soit aussi \( d=d(a,K)\); on a \( d>0\) parce que \( K\) est compact et \( a\) est hors de \( a\) alors le complémentaire de \( K\) est ouvert. Nous choisissons en plus la suite \( a_n\) pour avoir \( | a_n-a |<\frac{ d }{2}\); au pire on prend la queue de suite. Soit \( z\in K\); nous avons
        \begin{equation}    \label{EqYHWQhI}
            | \varphi_{a_n}(z)-\varphi_a(z) |=\left| \frac{1}{ z-a_n }-\frac{1}{ z-a } \right| =  \left| \frac{ a_n-a }{ (z-a_n)(z-a) } \right|.
        \end{equation}
        Vu que \( a_n\in B(a,\frac{ d }{2})\) et que \( z\in K\) et \( d=d(a,K)\) nous avons \( | a_n-z |\geq \frac{ d }{2}\); et aussi \( | a-z |\geq \frac{ d }{2}\). Nous pouvons donc majorer \eqref{EqYHWQhI} par
        \begin{equation}
            | \varphi_{a_n}(z)-\varphi_a(z) |\leq 2\frac{ | a_n-a | }{ d^2 }.
        \end{equation}
        Donc nous avons
        \begin{equation}
            \| \varphi_a-\varphi_{a_n} \|_K\leq 2\frac{ | a_n-a | }{ d^2 }\to 0
        \end{equation}
        où la norme \( \| . \|_K\) est la norme supremum sur \( K\). Donc \( a\in \overline{ P(K) }=P(K)\) et \( A\) est fermé.

    \item[Ouvert] Vu que \( K\) est compact, il est fermé et donc \( \complement K\) est ouvert. Par conséquent, ainsi que précisé dans l'exemple \ref{ExloeyoR}, les ouverts de \( \complement K\) sont les ouverts de \( \eC\) contenus dans \( \complement K\). Afin de prouver que \( A\) est ouvert, nous prenons  \( a\in A\) et nous cherchons une boule (au sens de \( \eC\)) autour de \( a\) qui serait incluse dans \( A\).

        Soit donc \( h\in \eC\) «petit» dans un sens que nous allons préciser plus tard. Encore une fois nous posons \( d=d(a,K)\). Nous avons
        \begin{equation}        \label{EqgBSxFB}
            \varphi_{a+h}(z)=\frac{1}{ z-a-h }=\frac{1}{ z-a }\frac{1}{ 1-\frac{ h }{ z-a } }=\sum_{k=0}^{\infty}\frac{ h^k }{ (z-a)^{k+1} }.
        \end{equation}
        Déjà ici nous demandons \( h<\sup_{z\in K}| z-a |\). Puisque \( | z-a |>d\), nous avons alors
        \begin{equation}
            | \varphi_{a+h}(z) |\leq \sum_{k=0}^{\infty}\frac{ h^k }{ d^{k+1} }<\infty.
        \end{equation}
        Cela pour dire que la somme à droite de \eqref{EqgBSxFB} converge bien pourvu que \( h\) soit bien petit. Nous pouvons donc poursuivre :
        \begin{equation}    \label{EqTSSdttylSDX}
            \varphi_{a+h}(z)=\sum_{k=0}^{\infty}\frac{ h^k }{ (z-a)^{k+1} }=\sum_{k=0}^{\infty}h^k\varphi_a(z)^{k+1}.
        \end{equation}
        Nous montrons maintenant que la convergence de la somme \eqref{EqTSSdttylSDX} est en réalité uniforme en \( z\). En effet
        \begin{subequations}
            \begin{align}
                \big| \varphi_{a+h}(z)-\sum_{k=0}^Nh^k\varphi_a(z)^{k+1} \big|&=\big| \sum_{k=N+1}^{\infty}h^k\varphi_a(z)^{k+1} \big|\\
                &\leq\sum_{k=N+1}^{\infty}| h |^k| \varphi_a(z) |^{k+1}.
            \end{align}
        \end{subequations}
        Étant donné que \( \varphi_a\) est continue sur le compact \( K\), elle y est majorée en module; on peut même être plus précis :
        \begin{equation}
            |\varphi_a(z)|=\frac{1}{ | z-a | }\leq \frac{1}{ d }.
        \end{equation}
        Nous pouvons donc écrire
        \begin{equation}
            \big| \varphi_{a+h}(z)-\sum_{k=0}^Nh^k\varphi_a(z)^{k+1} \big|\leq\frac{1}{ d }\sum_{k=N+1}^{\infty}\left| \frac{ h }{ d } \right|^k.
        \end{equation}
        Étant donné que la somme \( \sum_{k=0}^{\infty}| h/d |^k\) converge, la limite \( N\to \infty\) est nulle et nous avons
        \begin{equation}
            \lim_{N\to \infty} \| \varphi_{a+h}-\sum_{k=0}^Nh^k\varphi_a^{k+1} \|_K=0.
        \end{equation}
        Pour avoir \( \varphi_{a+h}\in P(K)\), il faut encore savoir si les fonctions \( \varphi_a^{k}\) sont dans \( P(K)\) pour tout \( k\). Dans ce cas pour chaque \( N\) la somme sera encore dans \( P(K)\) et \( \varphi_{a+h}\) sera limite uniforme d'éléments de \( P(K)\).

        Par hypothèse, \( \varphi_a\in P(K)\); soit \( P_n\) une suite de polynômes qui converge uniformément vers \( \varphi_a\). Nous allons montrer qu'alors la suite de polynômes \( P_n^k\) converge uniformément vers \( \varphi_a^k\). Soit \( n\) tel que \( \| P_n-\varphi_a \|_{K}\leq \epsilon\) et utilisons le produit remarquable\index{produit remarquable}
        \begin{equation}
            a^k-b^k=(a-b)\sum_{i=0}^{k-1}a^ib^{k-1-i}
        \end{equation}
        pour obtenir
        \begin{equation}
            | P_n(z)^k-\varphi_a(z)^k |\leq | P_n(z)-\varphi_a(z) |\sum_{i=0}^{k-1}| P_n(z)^i\varphi_a(z)^{k-1-i} |.
        \end{equation}
        Vu que \( P_n\) et \( \varphi_a\) sont continues sur le compact \( K\), on peut majorer la somme par une constante \( M\), et il restera
        \begin{equation}
            | P_n(z)^k-\varphi_a(z)^k |\leq M | P_n(z)-\varphi_a(z) |,
        \end{equation}
        ou encore
        \begin{equation}
            \| P_n^k-\varphi_a^k \|\leq M\epsilon.
        \end{equation}
        Cela prouve que \( \varphi_a^{k}\in P(K)\) et donc que \( \varphi_{a+h}\) est limite uniforme (sur \( K\)) d'éléments de \( P(K)\) et donc fait partie de \( P(K)\) lui aussi.

        Ceci achève de prouver que \( A\) est ouvert dans \( \complement K\).
    \item[Conclusion]

        L'ensemble \( A\) est non vide, ouvert et fermé dans \( \complement K\), donc il est égal à \( \complement K\). Le théorème est ainsi démontré.
    \end{subproof}
\end{proof}

%+++++++++++++++++++++++++++++++++++++++++++++++++++++++++++++++++++++++++++++++++++++++++++++++++++++++++++++++++++++++++++
\section{Intégrales de fonctions holomorphes}
%+++++++++++++++++++++++++++++++++++++++++++++++++++++++++++++++++++++++++++++++++++++++++++++++++++++++++++++++++++++++++++

\begin{lemma}       \label{LemNAnweA}
    Soit \( f\) une fonction holomorphe sur \( B(z_0,r_0)\). Pour tout \( z\in B(z_0,r)\) (avec \( r<r_0\)) nous avons
    \begin{equation}
        | f'(z) |\leq \frac{ r }{ \big( r-| z-z_0 | \big)^2 }\max\big\{ f(z_0+r e^{i\theta}) \big\}_{\theta\in \eR}.
    \end{equation}
\end{lemma}

\begin{proof}
    Par translation nous pouvons supposer que \( z_0=0\). Étant donné que \( f\) est holomorphe, elle admet un développement en séries entières
    \begin{equation}
        f(z)=\sum_{n=0}^{\infty}a_nz^n
    \end{equation}
    et nous notons \( M=\max\{ f(z)\tq z\in \overline{ B(0,r) } \}\). Nous avons\cite{Holomorphieus} \( r^n| a_n |\leq M\). Par conséquent
    \begin{subequations}
        \begin{align}
            | f'(z) |&=\left| \sum_{n=1}^{\infty}na_nz^{n-1} \right| \\
            &\leq\frac{1}{ r }\sum r^n| a_n |n\left( \frac{ | z | }{ r } \right)^{n-1}\\
            &<\frac{ M }{ r }\sum n\left( \frac{ | z | }{ r } \right)^{n-1}
        \end{align}
    \end{subequations}
    À ce point nous devons utiliser la série de l'exemple \ref{ExGxzLlP}. Nous avons alors
    \begin{equation}
        | f'(z) |\leq \frac{ M }{ r }\frac{ 1 }{ \left( 1-\frac{ | z | }{ r } \right)^2 }=\frac{ Mr }{ (r-| z |)^2 }.
    \end{equation}
\end{proof}

\begin{theorem}[Holomorphie sous l'intégrale\cite{Holomorphieus}] \label{ThopCLOVN}
    Soit un espace mesuré \( (\Omega,\mu)\), un ouvert \( A\) dans \( \eC\) et une fonction \( f\colon A\times \Omega\to \eC\). Nous voulons étudier la fonction
    \begin{equation}
        F(z)=\int_{\Omega}f(z,\omega)d\mu(\omega)
    \end{equation}
    pour tout \( z\in A\). Nous supposons que
    \begin{enumerate}
        \item
            la fonction \( f(.,\omega)\) est holomorphe sur \( A\) pour chaque \( \omega\).
        \item
            La fonction \( f(z,.)\) est mesurable sur \( (\Omega,\mu)\).
        \item
            Pour tout compact \( K\subset A\), il existe une fonction \( g_K\colon \Omega\to \eR\) telle que \( | f(z,\omega) |\leq g_K(\omega)\) et telle que
            \begin{equation}
                \int_{\Omega}g_K(\omega)d\mu(\omega)
            \end{equation}
            existe.
    \end{enumerate}
    Alors la fonction \( F\) est holomorphe et
    \begin{equation}
        F'(z)=\int_{\Omega}\frac{ \partial f }{ \partial z }(z,\omega)d\mu(\omega).
    \end{equation}
\end{theorem}

\begin{proof}
    Soient \( z_0\in A\) et \( r>0\) tels que \( K=\overline{ B(z_0,r) }\subset A\). Pour chaque \( \omega\in \Omega\) nous considérons la fonction
    \begin{equation}
        \begin{aligned}
            f_{\omega}\colon \overline{ B(z_0,r) }&\to \eC \\
            z&\mapsto f(z,\omega). 
        \end{aligned}
    \end{equation}
    Étant donné que \( \overline{ B(z_0,r) }\) est compacte, la fonction \( | f_{\omega} |\) est majorée par un nombre que nous notons \( f_K(\omega)\) qui est indépendant de \( z\) (pour autant que $z\in K$). Nous désignons par \( S(z_0,r)\) la frontière de la boule \( B(z_0,r)\). Étant donné que la majoration est valable sur \( \overline{ B(z_0,r) }\), nous avons en particulier
    \begin{equation}
        | f_{\omega}(z) |\leq f_K(\omega)
    \end{equation}
    pour tout \( z\in S\). En utilisant la lemme \ref{LemNAnweA} nous avons
    \begin{subequations}
        \begin{align}
            | f'_{\omega}(z) |&\leq \frac{ r }{ (r-| z-z_0 |)^2 }\max\{ f(z_0+r e^{i\theta}) \}_{\theta\in \eR}\\
            &\leq \frac{ rf_K(\omega) }{ (r-| z-z_0 |)^2 }.
        \end{align}
    \end{subequations}
    Cette majoration est valable pour tout \( z\in B(z_0,r)\). Si nous supposons de plus que \( z\in B(z_0,r/2)\)  nous avons
    \begin{equation}
        | f'(z) |\leq \frac{ rf_K(\omega) }{ \left( r-\frac{ r }{2} \right)^2 }=\frac{ 4 }{ r }f_K(\omega).
    \end{equation}
    Étant donné que la boule \( B(z_0,r/2)\) est convexe, la fonction \( f_{\omega}\) est Lipschitz et pour tout \( h\in \eC\) tel que \( | h |<r/2\) nous avons
    \begin{equation}
        \left| \frac{ f_{\omega}(z_0+h)-f_{\omega}(z_0) }{ h } \right| \leq \frac{ 4f_K(\omega) }{ r }.
    \end{equation}
    Soit maintenant une suite \( (h_n)\) qui converge vers \( 0\) dans \( \eC\). Nous considérons la suite de fonctions correspondantes
    \begin{equation}
        g_n(\omega)=\frac{ f(z_0+h_n,\omega)-f(z_0,\omega) }{ h_n }.
    \end{equation}
    Cette suite de fonction vérifie la convergence ponctuelle
    \begin{equation}
        g_n(\omega)\to\frac{ \partial f }{ \partial z }(z_0,\omega).
    \end{equation}
    De plus \( g_n\) est une fonction (de \( \omega\)) dominée par \( \frac{ 4f_K }{ r }\) qui est intégrable. Par conséquent le théorème de la convergence dominée nous indique que
    \begin{equation}
        \int_{\Omega}g_n(\omega)d\mu(\omega)\to \int_{\Omega}\frac{ \partial f }{ \partial z }(z_0,\omega)d\mu(\omega),
    \end{equation}
    tandis que
    \begin{equation}
        F'(z)=\lim_{n\to \infty} \frac{ F(z_0+h_n)-F(z_0) }{ h_n }=\lim_{n\to \infty} \int_{\Omega}g_N(\omega)d\mu(\omega).
    \end{equation}
\end{proof}

\begin{corollary}       \label{CorNxTjEj}
    Si \( f\) est une fonction holomorphe sur l'ouvert \( \Omega\) contenant la fermeture de la boule \( B(z_0,r)\), alors pour tout \( z\) dans \( B(z_0,\rho)\) (\( \rho<r\)) les dérivées de \( f\) s'expriment pas la formule suivante :
    \begin{equation}
        f^{(k)}(z)=\frac{1}{ 2\pi i }\int_{\partial B(z_0,r)}\frac{ f(\omega) }{ (\omega-z)^{k+1} }d\omega.
    \end{equation}
\end{corollary}
\index{compacité}

\begin{proof}
    Nous appliquons le théorème \ref{ThopCLOVN} à la fonction
    \begin{equation}
        g(z,\omega)=\frac{ f(\omega) }{ \omega-z }
    \end{equation}
    avec \( \Omega=\partial B(z_0,r)\) et \( A=B(z_0,\rho)\) avec \( \rho<r\). Étant donné que \( f\) est holomorphe, elle est continue et donc bornée sur tout compact \( K\subset A\) par une constante \( M\) (qui dépend du compact choisi).  D'autre part, nous avons toujours \( | \omega-z |>r-\rho\) et donc
    \begin{equation}
        | g(z,\omega) |\leq \frac{ M }{ r-\rho }.
    \end{equation}
    La fonction constante \( g_K=\frac{ M }{ r-\rho }\) est évidemment intégrable. Le théorème conclu que \( f\) est holomorphe (cela, nous le savions déjà\footnote{et cela fournit une preuve alternative à la réciproque du théorème de Cauchy : une fonction continue qui vérifie la formule de Cauchy est holomorphe.}), et
    \begin{equation}
        f'(z)=\frac{1}{ 2i\pi }\int_{\partial B}\frac{ f(\omega) }{ (\omega-z)^2 }d\omega.
    \end{equation}
    Un peu de récurrence montre maintenant que
    \begin{equation}
        f^{(k)}(z)=\frac{1}{ 2i\pi }\int_{\partial B(z_0,r)}\frac{ f(\omega) }{ (\omega-z)^{k+1} }d\omega.
    \end{equation}
\end{proof}

\begin{definition}
    Une \defe{mesure de Radon}{mesure!de Radon} sur un compact \(  K\) de \( \eC\) est une forme linéaire continue sur \( C(K)\). Si \( \mu\) est une mesure de Radon, on définit la \defe{transformée de Cauchy}{transformée!de Cauchy} de \( \mu\) par 
    \begin{equation}
        \begin{aligned}
            \hat \mu\colon \eC\setminus K&\to \eC \\
            z&\mapsto -\frac{1}{ \pi }\mu\left( \frac{1}{ \xi-z } \right). 
        \end{aligned}
    \end{equation}
\end{definition}

\begin{theorem}     \label{ThoJVNTzn}
    Si \( \mu\) est une mesure de Radon sur \( K\) alors \( \hat \mu\) est infiniment \( \eC\)-dérivable sur \( \Omega=\eC\setminus K\) et nous avons
    \begin{equation}
        \hat\mu^{(n)}(z)=-\frac{ n! }{ \pi }\mu\left( \frac{1}{ (\xi-z)^{n+1} } \right).
    \end{equation}
\end{theorem}

\begin{lemma}
    Si \( f\) est holomorphe sur \( \Omega\) et si \( B\) est une boule fermée dans \( \Omega\) alors pour tout \( z\in \Int(B)\) nous avons
    \begin{equation}
        f^{(n)}(z)=\frac{ n! }{ 2i\pi }\int_{\partial B}\frac{ f(\xi) }{ (\xi-z)^{n+1} }d\xi.
    \end{equation}
\end{lemma}

\begin{proof}
    Appliquer le théorème \ref{ThoJVNTzn} à la mesure de Radon
    \begin{equation}
        \mu(\phi)=\int_{\partial B}\phi(\xi)d\xi.
    \end{equation}
\end{proof}

\begin{lemma}
    Si \( f\) est holomorphe sur \( \Omega\) et si \( B\) est une boule fermée dans \( \Omega\) alors pour tout \( z\) dans l'intérieur de \( B\) nous avons
    \begin{equation}
        f^{(n)}(z)=\frac{ n! }{ 2i\pi }\int_{\partial B}\frac{ f(\xi) }{ (\xi-z)^{n+1} }d\xi.
    \end{equation}
\end{lemma}

\begin{theorem}
    Si \( f\) est une fonction holomorphe sur le disque ouvert \( B(z_0,R)\) alors
    \begin{equation}
        f(z)=\sum_{n=0}^{\infty}\frac{ f^{(n)}(z_0) }{ n! }(z-z_0)^n
    \end{equation}
    et cette série converge uniformément sur tout compact.
\end{theorem}

\begin{proof}
    Sans perte de généralité nous supposons que \( z_0=0\). La formule de Cauchy fournit
    \begin{equation}
        f(z)=\frac{1}{ 2\pi i }\int_{\partial B}\frac{ f(\xi) }{ \xi-z }d\xi=\frac{1}{ 2\pi i }\int_{\partial B}\frac{ f(\xi) }{ 1-(z/\xi) }\frac{ d\xi }{ \xi }.
    \end{equation}
    Nous utilisons la série géométrique
    \begin{equation}
        \frac{1}{ 1-(z/\xi) }=\sum_{n=0}^{\infty}\left( \frac{ z }{ \xi } \right)^n,
    \end{equation}
    nous avons
    \begin{subequations}        \label{EqXSgZGw}
        \begin{align}
            f(z)&=\frac{1}{ 2\pi i }\sum_{n=0}^{\infty}\int_{\partial B}\frac{ z^nf(\xi) }{ \xi^{n+1} }\\
            &=\sum_{n=0}^{\infty}\left( \frac{1}{ 2\pi i }\int_{\partial B}\frac{ f(\xi) }{ \xi^{n+1} } \right)z^n.
        \end{align}
    \end{subequations}
    Nous devons maintenant montrer que ce qui se trouve dans la grande parenthèse vaut \( f^{(n)}(0)/n!\). Nous utilisons le théorème de Radon \ref{ThoJVNTzn} à la mesure
    \begin{equation}
        \mu(\phi)=\int_{\partial B}\phi(\xi)d\xi.
    \end{equation}
    La transformée de Cauchy est
    \begin{equation}        \label{EqTzkmeL}
        \hat \mu(z)=-\frac{1}{ \pi }\mu\left( \frac{1}{ \xi-z } \right)=-\frac{1}{ \pi }\int_{\partial B}\frac{1}{ \xi-z }d\xi,
    \end{equation}
    et le théorème assure que
    \begin{equation}
        \hat\mu^{(n)}(z)=-\frac{ n! }{ \pi }\mu\left( \frac{1}{ (\xi-z)^{n+1} } \right)=-\frac{ n! }{ \pi }\int_{\partial B}\frac{ 1 }{ (\xi-z)^{n+1} }d\xi.
    \end{equation}
    En comparant la formule \eqref{EqTzkmeL} avec la formule de Cauchy nous voyons que \( \hat\mu(z)=-2i f(z)\). Par conséquent
    \begin{equation}
        f^{(n)}(z)=-\frac{1}{ 2i }\hat\mu^{(n)}(z)=\frac{ n! }{ 2\pi i }\int_{\partial B}\frac{1}{ (\xi-z)^{n+1} }d\xi,
    \end{equation}
    et
    \begin{equation}
        f^{(n)}(0)=\frac{ n! }{ 2\pi i }\int_{\partial B}\frac{1}{ \xi^{n+1} }d\xi.
    \end{equation}
\end{proof}
% TODO : justifier la permutation entre la somme et l'intégrale.

\begin{proposition}[Morera \cite{NEBgfg}]   \label{ThoRckxes}
    Soit \( \Omega\) ouvert dans \( \eC\) et \( f\) continue. Si
    \begin{equation}
        \int_{\partial T}f=0
    \end{equation}
    pour tout triangle (plein) \( T\) contenu dans \( \Omega\), alors \( f\) est holomorphe sur \( \Omega\).
\end{proposition}

\begin{proof}
    Il est suffisant de prouver que \( f\) est holomorphe sur toute boule ouverte \( B(a,r)\) inclue dans \( \Omega\). Nous posons, pour tout \( z\in B(a,r)\),
    \begin{equation}
        F(z)=\int_{[p,z]}f,
    \end{equation}
    et nous considérons le chemin triangulaire \( a\to z\to z+h\to a\) où \( h\in \eC\) est choisit assez petit pour que \( z+h\in B(a,r)\). L'intégrale sur le triangle étant nulle, nous avons
    \begin{equation}
        0=\int_{a\to z}f+\int_{z\to z+h}f+\int_{z+h\to a}f,
    \end{equation}
    c'est à dire
    \begin{equation}
        F(z+h)-F(z)=\int_{z\to z+h}f.
    \end{equation}
    En paramétrant le chemin par \( z+th\) avec \( t\in\mathopen[ 0 , 1 \mathclose]\), et en tenant compte de la remarque \ref{RemiqswPd},
    \begin{subequations}
        \begin{align}
            F'(z)&=\lim_{h\to 0} \frac{ F(z+h)-F(z) }{ h }\\
            &=\lim_{h\to 0} \frac{1}{ h }\int_0^1f(z+th)hdt,
        \end{align}
    \end{subequations}
    ce qui prouve que \( F\) est dérivable et \( F'=f\). Par définition (\ref{DefMMpjJZ}), \( F\) est holomorphe, et donc \( C^{\infty}\) par le théorème \ref{ThomcPOdd}. Du coup \( f\) est également \(  C^{\infty}\) et donc en particulier holomorphe.
\end{proof}

%+++++++++++++++++++++++++++++++++++++++++++++++++++++++++++++++++++++++++++++++++++++++++++++++++++++++++++++++++++++++++++
\section{Conditions équivalentes à l'holomorphie}
%+++++++++++++++++++++++++++++++++++++++++++++++++++++++++++++++++++++++++++++++++++++++++++++++++++++++++++++++++++++++++++

Nous nous proposons de lister les conditions que nous avons vues être équivalentes à l'holomorphie.

\begin{theorem}
    Soit \( \Omega\) un ouvert de \( \eC\) et \( f\colon \Omega\to \eC\) une fonction continue. Les conditions suivantes sont équivalentes.
    \begin{enumerate}
        \item   \label{ItemOtPcTb}
            \( f\) est holomorphe.
        \item   \label{ItemHWRnxx}
            Pour tout triangle (plein) \( T\) contenu dans \( \Omega\), \( \int_Tf=0\).
        \item   \label{ItempBBPVv}
            \( f\) est dérivable.
        \item   \label{ItemmLhzbB}
            \( f\) est \(  C^{\infty}\)
        \item   \label{ItemCCrSrLj}
            \( \frac{ \partial f }{ \partial \bar z }=0\)
        \item   \label{ItemEvxRSn}
            La \( 1\)-forme différentielle \( f(z)dz\) est localement exacte.
        \item   \label{ItemVSCHtY}
            Pour toute boule \( B(a,r)\) contenue dans \( \Omega\) nous avons
            \begin{equation}
                f(a)=\frac{1}{ 2\pi i }\int_{\partial B(a,r)}\frac{ f(z) }{ z-a }dz.
            \end{equation}
    \end{enumerate}
\end{theorem}
% TODO : il faudrait rajouter les équations de Cauchy-Riemann.

\begin{proof}
    \ref{ItemOtPcTb} implique \ref{ItemHWRnxx} est le lemme de Goursat \ref{LemwbwbUR}. \ref{ItemHWRnxx} implique \ref{ItemOtPcTb} est le théorème de Morera \ref{ThoRckxes}.

    \ref{ItempBBPVv} est la définition de l'holomorphie.

    \ref{ItemmLhzbB} implique \ref{ItemOtPcTb} est un a fortiori sur la définition. \ref{ItemOtPcTb} implique \ref{ItemmLhzbB} est contenu dans le théorème de développement en série entière \ref{ThoUHztQe}.

    L'équivalence entre \ref{ItemCCrSrLj} et l'holomorphie est le théorème \ref{ThokwIQwg}.

    L'équivalence entre \ref{ItemEvxRSn} et \ref{ItemOtPcTb} est la proposition \ref{PropZOkfmO}.

    L'équivalence entre \ref{ItemOtPcTb} et \ref{ItemVSCHtY} est d'une part le théorème \ref{ThomcPOdd} et d'autre part le corollaire \ref{CorwfHtJu}.
\end{proof}


%+++++++++++++++++++++++++++++++++++++++++++++++++++++++++++++++++++++++++++++++++++++++++++++++++++++++++++++++++++++++++++ 
\section{Singularités, pôles et méromorphe}
%+++++++++++++++++++++++++++++++++++++++++++++++++++++++++++++++++++++++++++++++++++++++++++++++++++++++++++++++++++++++++++

\begin{definition}
    Si \( f\) est holomorphe sur un ouvert \( \Omega\), alors une \defe{singularité}{singularité} de \( f\) est un point isolé du bord de \( \Omega\).
    \begin{enumerate}
        \item
            La singularité est \defe{effaçable}{singularité!effaçable} si la fonction \( f\) s'y prolonge en une fonction holomorphe.
        \item
            La singularité \( Z\) est un \defe{pôle}{singularité!pôle} d'ordre \( k\) de \( f\) si elle n'est pas effaçable et si la fonction \( z\mapsto (z-Z)^kf(z)\) se prolonge en une fonction holomorphe en \( Z\).
    \end{enumerate}
\end{definition}

\begin{example}
    La fonction 
    \begin{equation}
        z\mapsto \frac{ \sin(z) }{ z }
    \end{equation}
    n'est pas définie en \( z=0\), mais elle s'y prolonge en une fonction continue en posant \( f(0)=1\).
\end{example}
%TODO : dans \eC je ne sais pas si c'est facile à montrer. De toutes façons, il faudrait déjà définir le sinus.

\begin{proposition}
    Une singularité de \( f\) est un pôle si et seulement si
    \begin{equation}
        \lim_{z\to Z}f(z)=\infty.
    \end{equation}
\end{proposition}

\begin{theorem}[Prolongement de Riemann]
    Soit \( f\colon \Omega\to \eC\) et une singularité \( Z\) de \( f\). Nous avons équivalence de 
    \begin{enumerate}
        \item
            la singularité \( Z\) est effaçable;
        \item
            \( f\) possède un prolongement continue en \( Z\);
        \item
            il existe un voisinage épointé de \( Z\) sur lequel \( f\) est bornée;
        \item
            \( \lim_{z\to Z}(z-Z)f(z)=0\).
    \end{enumerate}
\end{theorem}
\index{théorème!prolongement de Riemann}

\begin{definition}
    Une fonction \defe{méromorphe}{fonction!méromorphe} est une fonction holomorphe sur \( \eC\) sauf éventuellement sur un ensemble de points isolés dont chacun est un pôle ou une singularité effaçable.
\end{definition}

\begin{proposition} \label{PropPUZTQKl}
    Soit \( \Omega\) un ouvert de \( \eC\) et une suite de fonctions \( f_n\colon \Omega\to \eC\) telles que pour tout compact \( K\) de \( \Omega\) il existe \( N_K\geq 0\) tel que
    \begin{enumerate}
        \item
            \( f_n\) n'a pas de pôles dans \( K\) dès que \( n\geq N_K\);
        \item
            la série \( \sum_{n\geq N_K}f_n\) converge uniformément sur \( K\).
    \end{enumerate}
    Alors
    \begin{enumerate}
        \item
            La fonction 
            \begin{equation}
                f(z)=\sum_{n=0}^{\infty}f_n(z)
            \end{equation}
            est méromorphe sur \( \Omega\) et ses pôles sont l'union de ceux des \( f_n\).
        \item
            Nous pouvons permuter la somme et la dérivée :
            \begin{equation}
                f'(z)=\sum_{n=0}^{\infty}f'_n(z).   
            \end{equation}
    \end{enumerate}
\end{proposition}

%+++++++++++++++++++++++++++++++++++++++++++++++++++++++++++++++++++++++++++++++++++++++++++++++++++++++++++++++++++++++++++ 
\section{Fonctions d'Euler}
%+++++++++++++++++++++++++++++++++++++++++++++++++++++++++++++++++++++++++++++++++++++++++++++++++++++++++++++++++++++++++++

\begin{theorem}[Prolongement méromorphe de la fonction \( \Gamma\) d'Euler\cite{KXjFWKA}]
    Nous considérons la formule
    \begin{equation}
        \Gamma(z)=\int_0^{\infty} e^{-t}t^{z-1}dt.
    \end{equation}
    Alors
    \begin{enumerate}
        \item
            Cette formule définit une fonction holomorphe sur 
            \begin{equation}
                \mP=\{ z\in \eC\tq \Re(z)>0 \}.
            \end{equation}
        \item
            La fonction \( \Gamma\colon \mP\to \eC\) admet un unique prolongement méromorphe sur \( \eC\), lequel a des pôles sur les entiers négatifs.
    \end{enumerate}
\end{theorem}

\begin{proof}
    \begin{subproof}
        \item[Holomorphie sous l'intégrale]

            Pour étudier l'holomorphie de la fonction \( \Gamma\) sur \( \mP\) nous utilisons le théorème \ref{ThopCLOVN}.

            Nous considérons la fonction
            \begin{equation}
                \begin{aligned}
                    g\colon \mP\times \eR^+&\to \eC \\
                    (z,t)&\mapsto  e^{-t}z^{z-1}
                \end{aligned}
            \end{equation}
            et nous commençons par montrer que c'est holomorphe en \( z\) pour chaque \( t>0\) fixé. Nous le vérifions par le critère de \( \partial_{\bar zf=0}\)\footnote{Théorème \ref{ThokwIQwg}.} et en nous souvenant que \( t^i= e^{\ln(t^i)}= e^{i\ln(t)}\). Nous obtenons rapidement que
            \begin{equation}
                \frac{ \partial g }{ \partial \bar z }=0.
            \end{equation}

            Le fait que la fonction \( t\mapsto g(z,t)\) soit mesurable pour tout \( z\) est d'accord.

            Et enfin soit \( K\) compact dans \( \mP\). Il faut trouver une fonction \( g_K(t)\) intégrable sur \( \mathopen[ 0 , \infty [\) telle que pour tout \( z\in K\) et \( t\in\mathopen[ 0 , \infty [\) nous ayons \( | f(z,t)\leq g(t) |\). Pour cela nous majorons séparément les parties \( t\in\mathopen] 0 , 1 \mathclose[\) et \( t\geq 1\). 

            Soit donc \( K\) compact dans \( \mP\); nous posons \( M=\max_{z\in K}\Re(z)\) et \( \epsilon=\min_{z\in K}\Re(z)\).

            Si \( t\in \mathopen] 0 , 1 \mathclose[\) alors nous avons 
            \begin{equation}
                e^{-t}t^{z-1}= e^{-t} e^{(z-1)\ln(t)},
            \end{equation}
            de telle façon à que que
            \begin{subequations}
                \begin{align}
                    |  e^{-t}t^{z-1} |&\leq|  e^{(x-1+iy)\ln(t)} |\\
                    &=|   e^{(\Re(z)-1)\ln(t)} |\\
                    &=| t^{\Re(z)-1} |\\
                    &\leq | t^{\epsilon-1} |\\
                    &=\frac{1}{ t^{1-\epsilon} }.
                \end{align}
            \end{subequations}
            Cette dernière fonction est intégrable sur \( \mathopen] 0 , 1 \mathclose[\).

            Nous considérons maintenant \( t\geq 1\). Dans ce cas nous avons
            \begin{equation}
                |  e^{-t}z^{z-1} |= e^{-t}t^{\Re(z)-1}\leq  e^{-t}t^{M-1}.
            \end{equation}
            Cette dernière fonction est un produit d'une exponentielle décroissante avec un polynôme. C'est donc intégrable entre \( 1\) et l'infini.

            La fonction \( g_K\) que nous considérons est donc
            \begin{equation}
                g_K(t)=\begin{cases}
                    \frac{1}{ t^{1-\epsilon} }    &   \text{si \( t<1\)}\\
                    \text{borné}    &    \text{si \( 1\leq t\leq b\)}\\
                    e^{-t}t^{M-1}    &    \text{si \( t>b\)}.
                \end{cases}
            \end{equation}
            Cela est une fonction intégrable sur \( \mathopen] 0    \infty ,  \mathclose[\) et qui majore \( f\) uniformément en \( z\) sur le compact \( K\) de \( \mP\). Le théorème \ref{ThopCLOVN} nous permet donc de conclure que
            \begin{equation}
                \Gamma(z)=\int_0^{\infty}f(z,t)dt
            \end{equation}
            est holomorphe en \( z\) sur \( \mP\) et que
            \begin{equation}
                \Gamma'(z)=\int_0^{\infty}\frac{ \partial f }{ \partial z }(z,t)dt.
            \end{equation}
            
        \item[En deux morceaux] Nous passons maintenant à la seconde partie du théorème. Pour \( z\in \mP\) nous coupons l'intégrale en deux :
            \begin{equation}
                \Gamma(z)=\int_0^1 e^{-t}t^{z-1}dt+\int_1^{\infty} e^{-t}t^{z-1}dt
            \end{equation}
            
        \item[Première partie] Nous commençons par parler de la première partie : \( \int_0^1 e^{-t}t^{z-1}dt\) dans laquelle nous voulons utiliser le développement en série de l'exponentielle \(  e^{-t}\). Nous devons donc traiter
            \begin{equation}
                \int_0^1\sum_{n=0}^{\infty}\frac{ (-1)^n }{ n! }t^{n+z-1}dt.
            \end{equation}
            Nous allons permuter la somme avec l'intégrale à l'aide du théorème de Fubini \ref{ThoFubinioYLtPI} en posant la fonction
            \begin{equation}
                g(n,t)=\frac{ (-1)^n }{ n! }t^{n+z-1}
            \end{equation}
            et en considérant le produit entre la mesure de Lebesgue sur \( \eC\) et la mesure de comptage sur \( \eN\), c'est à dire que nous étudions
            \begin{equation}
                \int_0^1\int_{\eN}g(n,t)dndt.
            \end{equation}
            Pour permuter il suffit de prouver que \( | g |\) est intégrable pour la mesure produit, c'est à dire que
            \begin{equation}
                \int_0^1\int_{\eN}\left| \frac{ (-1)^n }{ n! }t^{n+z-1} \right| <\infty.
            \end{equation}
            Nous avons \( | t^z=t^{\Re(z)} |\), donc
            \begin{equation}
                \sum_{n=0}^{\infty}\left| \frac{ t^{n+z-1} }{ n! } \right| =t^{\Re(z)-1}\sum_{n=0}^{\infty}\frac{ t^n }{ n! }=t^{\Re(z)-1} e^{t}.
            \end{equation}
            Étant donné que nous avons fixé \( z\in\mP\), nous avons \( \Re(z)-1>-1\) et donc \( t^{\Re(z)-1}\) est intégrable entre \( 0\) et \( 1\).
            %TODO : il faudrait prouver et citer ici le coup du 1/x^alpha qui est intégrable ou non.
            La partie \(  e^{t}\) se majore sur \( \mathopen[ 0 , 1 \mathclose]\) par une constante quelconque. Nous avons donc payé le droit d'inverser la somme et l'intégrale :
            \begin{equation}
                \int_0^1 e^{-t}t^{z-1}dt=\sum_{n=0}^{\infty}\int_0^1\frac{ (-1)^n }{ n! }t^{n+z-1}dt=\sum_{n=0}^{\infty}\frac{ (-1)^n }{ n! }[t^{n+z}]_0^1=\sum_{n=0}^{\infty}\frac{ (-1)^n }{ n!(n+z) }.
            \end{equation}
            Nous avons donc l'intéressante formule suivante, valable pour tout \( z\in\mP\) :
            \begin{equation}
                \Gamma(z)=\sum_{n=0}^{\infty}\frac{ (-1)^n }{ n!(n-z) }+\int_1^{\infty} e^{-t}t^{z-1}dt.
            \end{equation}
            
        \item[Prolongation de la première partie] Nous voudrions montrer maintenant que la fonction
            \begin{equation}
                \sum_{n=0}^{\infty}\frac{ (-1)^n }{ n!(n-z) }
            \end{equation}
            est méromorphe sur \( \eC\) avec des pôles en les entiers négatifs. Pour cela nous considérons la suite de fonctions
            \begin{equation}
                f_n(z)=\frac{ (-1)^n }{ n!(z+n) }
            \end{equation}
            et nous allons utiliser la proposition \ref{PropPUZTQKl}. Si \( n\geq 0\), la fonction \( f_n\) est méromorphe sur \( \eC\) avec un pôle simple en \( z=-n\). Soit \( K\) compact de \( \eC\) et \( N_K\) tel que \( K\subset\overline{ B(0,N_K) }\). Pour \( n\geq N_K+1\), la fonction \( f_n\) n'a pas de pôles dans \( K\) et de plus pour tout \( z\in K\) nous avons
            \begin{equation}
                | z+n |=| z-(-z) |\geq\big| n-| z | \big|\geq n-| z |\geq n-N_K,
            \end{equation}
            et par conséquent
            \begin{equation}
                | f_n(z) |\leq \frac{1}{ n!(n-N) },
            \end{equation}
            ou pour le dire de façon plus snob :
            \begin{equation}
                \| f_n \|_{\infty,K}\leq \frac{1}{ n!(n-N) },
            \end{equation}
            dont la série converge. Cela signifie que la série \( \sum_{n\geq N}f_n\) converge normalement\footnote{Définition \ref{DefQDrDqek}.} sur \( K\), donc la fonction
            \begin{equation}
                f(z)=\sum_{n=0}^{\infty}f_n(z)
            \end{equation}
            est une fonction méromorphe dont les pôles sont ceux des \( f_n\), c'est à dire les entiers négatifs (proposition \ref{PropPUZTQKl}).

        \item[La seconde partie] 
            
            Nous allons à présent prouver que la fonction
            \begin{equation}
                g(z)=\int_1^{\infty} e^{-t}t^{z-1}dt
            \end{equation}
            est holomorphe sur \( \eC\). Pour cela nous considérons la fonction de deux variables \( f(z,t)= e^{-t}t^{z-1}\) et nous utilisons le théorème d'holomorphie sous l'intégrale \ref{ThopCLOVN}. D'abord pour \( z_0\) fixé dans \( \eC\) nous avons 
            \begin{equation}
                \int_1^{\infty}|  e^{-t}t^{z_0-1} |\leq \int_1^{\infty} e^{-t}t^{\Re(z_0)-1}dt,
            \end{equation}
            donc l'intégrale converge parce que c'est polynôme contre exponentielle. Par ailleurs pour chaque \( t_0\) fixé sur \( \mathopen[ 0 , \infty [\), la fonction \( z\mapsto  e^{-t_0}t_0^{z-1}\) est holomorphe sur \( \eC\) comme en témoigne le calcul suivant :
                \begin{equation}
                    \frac{ 1 }{2}\left( \frac{ \partial  }{ \partial x }+i\frac{ \partial  }{ \partial y } \right)t_0^{x+iy-1}=0.
                \end{equation}
                Et enfin si \( K\) est compact dans \( \eC\) nous avons
                \begin{equation}
                    | f(z,t) |=|  e^{-t}t^{z-1} |= e^{-t}| t^{\Re(z)-1} |\leq  e^{-t}t^{M-1}
                \end{equation}
                où \( M=\max_{z\in K}\Re(z)\). Nous en déduisons que la fonction
                \begin{equation}
                    z\mapsto\int_1^{\infty} e^{-t}t^{z-1}dt
                \end{equation}
                est une fonction holomorphe sur \( \eC\).

            \item[Conclusion]

                Au final nous avons prouvé que la fonction \( \Gamma\) d'Euler admet le prolongement méromorphe sur \( \eC\) donné par
                \begin{equation}
                    \Gamma(z)=\sum_{n=0}^{\infty}\frac{ (-1)^n }{ n!(z+n) }+\int_1^{\infty} e^{-t}t^{z-1}dt.
                \end{equation}
    \end{subproof}
\end{proof}

%--------------------------------------------------------------------------------------------------------------------------- 
\subsection{Euler et factorielle}
%---------------------------------------------------------------------------------------------------------------------------

\begin{proposition}
    Nous avons la formule \( \Gamma(n)=(n-1)!\) pour tout \( n\in \eN\).
\end{proposition}

\begin{proof}
    Nous partons de la formule 
    \begin{equation}
        \Gamma(n)=\int_0^{\infty} e^{-t}t^{n-1}dt
    \end{equation}
    que nous intégrons par partie en posant
    \begin{equation}
        \begin{aligned}[]
            u&=t^{n-1}&u'&=(n-1)t^{n-1}\\
            v&= e^{-t}&v'&=- e^{-t}.
        \end{aligned}
    \end{equation}
    Les termes au bord s'annulent (ici il y a un passage à la limite qui n'est pas écrit) et nous trouvons
    \begin{equation}
        \Gamma(n)=\int_0^{\infty}(n-1) e^{-t}t^{n-2}dt=(n-1)\Gamma(n-1).
    \end{equation}
    
    Pour conclure il suffit de remarquer que
    \begin{equation}
        \Gamma(1)=\int_0^{\infty}=-[ e^{-t}]_0^{\infty}=1.
    \end{equation}
\end{proof}

%+++++++++++++++++++++++++++++++++++++++++++++++++++++++++++++++++++++++++++++++++++++++++++++++++++++++++++++++++++++++++++ 
\section{Partition d'un entier en parts fixées}
%+++++++++++++++++++++++++++++++++++++++++++++++++++++++++++++++++++++++++++++++++++++++++++++++++++++++++++++++++++++++++++
\index{partition!d'un entier en parts fixées}

\begin{proposition}[\cite{KXjFWKA}]     \label{PropWUFpuBR}
    Soient \( a_1,\ldots, a_k\in \eN^*\) des entiers premiers entre eux dans leur ensemble. Pour \( n\geq 1\) nous posons
    \begin{equation}
        u_n=\Card\left\{  (x_1,\ldots, x_k)\in \eN^*\tq \sum_{i=1}^ka_ix_i=n \right\},
    \end{equation}
    et \( u_0=1\).

    Alors nous avons l'équivalence de suite (pour \( n\to \infty\)) :
    \begin{equation}
        u_n\sim\frac{1}{ a_1\ldots a_k }\frac{ n^{k-1} }{ (k-1)! }.
    \end{equation}
\end{proposition}
\index{série!entière!utilisation}
\index{série!génératrice d'une suite!utilisation}
\index{anneaux!de séries formelles!utilisation}
\index{corps!des fractions rationnelles!utilisation}

\begin{proof}
    Pour chacun des \( i\in\{ 1,\ldots, k \}\) nous considérons la série entière
    \begin{equation}
        \sum_{x=0}^{\infty}z^{xa_i}=\sum_k(z^{a_i})^x.
    \end{equation}
    Étant donné que \( | z^{a_i} |<1\) si et seulement si \( | z<1 |\), cette série a un rayon de convergence égal à \( 1\). Nous allons calculer le produit de Cauchy de ces \( k\) séries, en nous souvenant que le théorème \ref{ThokPTXYC} nous assure que la série résultante aura un rayon de convergence au moins égal à \( 1\) et vaudra le produit des différentes séries.

    Le coefficient de \( z^n\) dans cette série vaut
    \begin{equation}
        \sum_{\substack{x\in \eN^k\\\sum x_ia_i=n}}1=u_n 
    \end{equation}
    parce que dans chacune des séries, le coefficient de tous les \( z^{xa_i}\) est \( 1\). Nous définissons la fonction
    \begin{equation}    \label{EqKTRNFSl}
        f(z)=\sum_{n=0}^{\infty}u_nz^n=\prod_{i=1}^k\left( \sum_{x=0}^{\infty}z^{xa_i} \right)=\prod_{i=1}^k\frac{1}{ 1-z^{a_i} }.
    \end{equation}
    La fonction \( f\) existe sur \( | z |<1\) parce que nous venons de voir qu'elle peut  s'exprimer comme un produit de Cauchy; et la dernière égalité est simplement la somme de la série harmonique. D'autre part la fonction \( f\) est la série génératrice de la suite \( (u_n)\).

    Nous sommes en présence d'une fonction ayant des pôles aux racines \( a_1\),\ldots, \( a_k\)\Ieme de l'unité. Étant donné que \( 1\) est une racine de l'unité de tous les ordres, le pôle en \( z=1\) est de multiplicité \( k\). Les autres pôles sont de multiplicité strictement inférieure; en effet soit \( \omega\in \eC\) tel que \( \omega^{a_i}=1\) pour tout \( i\). Alors Bezout\footnote{Théorème \ref{ThoBuNjam}.} nous donne des entiers \( v_i\in \eZ\) tels que \( \sum_iv_ia_i=1\). Alors nous avons
%TODO : il faudrait citer ici un théorème de Bezout pour les ensembles de nombres premiers dans leur ensemble; le théorème cité ici n'est pas suffisant.
    \begin{equation}
        \omega=\omega^{\sum_{v_ia_i}}=\prod_{i=1}^k(\omega^{a_i})^{v_i}=1.
    \end{equation}
    Donc nous voyons que \( 1\) est le seul à être racine de tous les ordres en même temps. Nous notons
    \begin{equation}
        P=\{ \omega_1,\ldots, \omega_p \}
    \end{equation}
    l'ensemble des pôles avec \( \omega_1=1\). Par ailleurs la fonction \( f\) est une fraction rationnelle dont nous connaissons les racines du dénominateur (ce sont les \( \omega_i\)) et à peu près leurs ordres. Nous utilisons le truc de la décomposition en fractions simples
%TODO : après avoir fait la décomposition en fractions simples, il faut mettre une référence ici.
    en séparant le terme de puissance \( k\) qui n'existe que pour la racine \( \omega_1=1\) :
    \begin{equation}    \label{EqDLTJaYr}
        f(z)=\frac{ \alpha }{ (1-z)^k }+\sum_{i=1}^p\sum_{j=1}^{k-1}\frac{ c_{ij} }{ (\omega_i-z)^j }.
    \end{equation}
    Ce développement est valable pour tout \( | z |<1\). Nous considérons maintenant \( \omega\in P\) et \( j\in \eN\) et nous étudions la fonction
    \begin{equation}
        g(z)=\frac{1}{ \omega-z }.
    \end{equation}
    Un rapide calcul (par exemple par récurrence) montre que
    \begin{equation}    \label{EqEJLDIFJ}
        g^{(k)}(z)=\frac{ k! }{ (\omega-z)^{k+1} },
    \end{equation}
    et étant donné que \( | \omega |=1\) nous pouvons écrire la série
    \begin{equation}
        \frac{1}{ \omega-z }=\sum_{k=0}^{\infty}\frac{ z^k }{ \omega^{k+1} },
    \end{equation}
    valable pour \( | z |<1\). Ce qui nous intéresse, c'est d'exprimer une série pour \( 1/(\omega-z)^j\); et voyant \eqref{EqEJLDIFJ}, nous voyons qu'il suffit de calculer les dérivées de la série de \( g\). Nous dériver terme à terme à l'intérieur du rayon de convergence. Avec quelque abus d'écriture, et en utilisant la bête formule \eqref{EqSOFdwhw} nous avons\quext{À ce niveau j'ai pas exactement le même coefficient binomial que dans \cite{KXjFWKA}, mais je n'exclus absolument pas que ce soit moi qui me trompe. Écrivez-moi si vous pouvez infirmer ou confirmer l'erreur. Quoi qu'il en soit, cela ne change pas le résultat asymptotique que nous cherchons.}
    \begin{subequations}
        \begin{align}
            \frac{1}{ (\omega-z)^j }&=\frac{ g^{(j-1)}(z) }{ (j-1)! }\\
            &=\frac{1}{ (j-1)! }\left( \frac{1}{ \omega-z } \right)^{(j-1)}\\
            &=\frac{1}{ (j-1)! }\sum_{k=0}^{\infty}\frac{1}{ \omega^{k+1} }(z^k)^{(j-1)}\\
            &=\sum_{k=j-^{\infty}}\frac{1}{ (j-1)! }\frac{1}{ \omega^{k+1} }\frac{ k! }{ (k-j+1)! }z^{k-j+1}\\
            &=\sum_{n=0}^{\infty}\frac{1}{ \omega^{n+j} }\frac{ (n+j-1)! }{ n!(j-1)! }z^n\\
            &=\sum_{n=0}^{\infty}\frac{1}{ \omega^{n+j} }{n+j-1\choose n}z^n.
        \end{align}
    \end{subequations}
    Nous pouvons utiliser cela pour récrire la formule \eqref{EqDLTJaYr} de façon considérablement plus compliquée :
    \begin{equation}
            f(z)=\alpha\sum_{n=0}^{\infty}{n+j-1\choose n}z^n
            +\sum_{i=1}^p\sum_{j=1}^{k-1}\sum_{n=0}^{\infty}c_{ij}{n+j-1\choose n}\frac{ z^n }{ \omega_i^{n+j} }.
    \end{equation}
    Mais nous savons que ce \( f\) est la série génératrice de la suite \( (u_n)\) et que nous pouvons donc utiliser la formule \eqref{EqNGhVCpP} pour exprimer les nombres \( u_l\) : \( u_l\) est simplement le coefficient de \( z^l\) divisé par \( l!\). C'est à dire
    \begin{equation}
        u_l=\alpha{l+k-1\choose l}+\sum_{i=1}^p\sum_{j=1}^{k-1}c_{ij}{l+j-1\choose l}\frac{1}{ \omega_i^{l+j} }.
    \end{equation}
    Notre boulot est d'examiner le comportement de cela lorsque \( l\to\infty\), c'est à dire regarder quels sont les puissances de \( l\) en présence. Notons que 

    En ce qui concerne le premier terme, la puissance dominante dans le coefficient binomial est \( l^{k-1}\). Dans les autres termes\footnote{Attention : les termes \( i=1\) ont \( \omega_1=1\) et il n'est donc pas possible de conclure simplement en disant que \( \omega_i^{l-j}\to 0\) pour \( l\to \infty\); bien que cela soit vrai pour tous les \( i\neq 1\).}, c'est \( l^{j-1}\) qui est de degré moins grand. Donc le comportement de \( u_l\) en terme de \( l\) est
    \begin{equation}
        u_l\sim \alpha\frac{ l^{k-1} }{ (k-1)! }.
    \end{equation}
    Il nous reste à voir ce que vaut \( \alpha\). Pour ce la nous repartons de l'expression \ref{EqKTRNFSl} que nous écrivons sous la forme
    \begin{equation}
        (1-z)^kf(z)=\prod_{i=1}^{k}\frac{ 1-z }{ 1-z^{a_i} }.
    \end{equation}
    Nous reconnaissons l'inverse d'une somme harmonique partielle :
    \begin{equation}    \label{EqTIAxvHE}
        (1-z)^kf(z)=\prod_{i=1}^k\frac{1}{ 1+z+z^2+\ldots +z^{a_i-1} }.
    \end{equation}
    Par ailleurs, nous ne savons pas si \( f(1)\) existe parce que son rayon de convergence n'est que de \( 1\); et nous savons même qu'elle n'existe pas (parce que ce serait la somme des \( u_n\)). Mais nous savons aussi que le pôle de plus grande multiplicité de \( f\) est en \( z=1\) et est de multiplicité \( k\). Donc \( (1-z)^kf(z)\) devrait converger pour \( z\to 1\). Pour tout \( | z<1 |\) nous avons
    \begin{equation}
        (1-z)^kf(z)=\alpha+\sum_{i=1}^p\sum_{j=1}^{k-1}c_{ij}\frac{ (1-z)^k }{ (\omega_i-z)^j }.
    \end{equation}
    Lorsque \( z\to 1\), tous les termes des sommes tendent vers zéro, y compris ceux avec \( i=1\) parce que \( j<k\). Il reste donc
    \begin{equation}
        \lim_{z\to 0} (1-z)^kf(z)=\alpha.
    \end{equation}
    En calculant la même limite avec \eqref{EqTIAxvHE} nous trouvons
    \begin{equation}
        \lim_{z\to 1}(1-z)^kf(z)=\lim_{z\to 1}\prod_{i=1}^k\frac{1}{ 1+z+z^2+\ldots +z^{a_i-1} }=\frac{1}{ a_1\ldots a_k }.
    \end{equation}
    Donc
    \begin{equation}
        \alpha=\frac{1}{ a_1\ldots a_k },
    \end{equation}
    et le résultat est prouvé.
    
\end{proof}

%+++++++++++++++++++++++++++++++++++++++++++++++++++++++++++++++++++++++++++++++++++++++++++++++++++++++++++++++++++++++++++
\section{Exponentielle complexe}
%+++++++++++++++++++++++++++++++++++++++++++++++++++++++++++++++++++++++++++++++++++++++++++++++++++++++++++++++++++++++++++

\begin{definition}  \label{DefJilXoM}
    Soit \( z=x+iy\in \eC\). Nous définissons l'\defe{exponentielle}{exponentielle!complexe} de \( z\) par
    \begin{equation}
        \begin{aligned}
            \exp\colon \eC&\to \eC \\
            z&\mapsto \sum_{n=0}^{\infty}\frac{ z^n }{ n! }. 
        \end{aligned}
    \end{equation}
\end{definition}
Le rayon de convergence de cette somme est infini.

\begin{proposition}[\cite{RomainBoilEnt}]     \label{PropdDjisy}
    Quelque propriétés de l'exponentielle.
    \begin{enumerate}
        \item
            Le fonction \( \exp\) est continue.
        \item
            Nous avons la formule \(  e^{z+w}= e^{z}+e^w\) pour tout \( z,w\in \eC\).
        \item
            \( (e^z)^{-1}= e^{-z}\)
        \item
            \( (\exp(z))^n=\exp(nz)\).
    \end{enumerate}
\end{proposition}

\begin{proof}
    L'exponentielle est continue parce qu'elle est la somme d'une série entière de rayon de convergence infini (proposition \ref{PropUEMoNF}).

    Les séries \( \exp(z)\) et \( \exp(w)\) ayant un rayon de convergence infini nous pouvons utiliser le produit de Cauchy (théorème \ref{ThokPTXYC}) :
    \begin{subequations}
        \begin{align}
            e^{z} e^{w}&=\sum_{n=0}^{\infty}\left( \sum_{i+j=n}\frac{ z^iw^j }{ i!j! } \right)\\
            &=\sum_{n=0}^{\infty}\left( \sum_{i=0}^n\frac{ z^iw^{n-i} }{ i!(n-i)! } \right)\\
            &=\sum_{n=0}^{\infty}\frac{1}{ n! }\sum_{i=0}^{n}{n\choose i}z^iw^{n-i}\\
            &=\sum_{n=0}^{\infty}\frac{1}{ n! }(z+w)^{n}\\
            &=\exp(z+w).
        \end{align}
    \end{subequations}
    Nous avons utilisé la formule du binôme (proposition \ref{PropBinomFExOiL}).

    Les autres propriétés énoncées sont des corollaires :
    \begin{equation}
        e^{z} e^{-z}= e^{0}=1.
    \end{equation}
\end{proof}

\begin{proposition}
    Si \( z=x+iy\in \eC\) alors
    \begin{equation}
        e^{x+iy}= e^{x}\big( \cos(y)+i\sin(y) \big).
    \end{equation}
\end{proposition}

\begin{proof}
    Par la proposition \ref{PropdDjisy} nous savons que \(  e^{x+iy}= e^{x} e^{iy}\). Nous devons donc seulement étudier \(  e^{iy}\). Nous avons
    \begin{subequations}
        \begin{align}
            e^{iy}&=\sum_{n=0}^{\infty}\frac{ (iy)^n }{ n! }\\
            &=\sum_{n=0}^{\infty}(-1)^n\frac{ y^{2n} }{ (2n)! }+i\sum_{n=0}^{\infty}(-1)^n\frac{ y^{2n+1} }{ (2n+1)! }\\
            &=\cos(y)+i\sin(y).
        \end{align}
    \end{subequations}
    Nous avons utilisé le fait que \( i^{2n}=(-1)^n\) et \( i^{2n+1}=i(-1)^n\).
\end{proof}

\begin{proposition}
    Soit \( z\in\eC\) fixé. La fonction
    \begin{equation}
        \begin{aligned}
            E\colon \eR&\to \eC \\
            t&\mapsto  e^{tz} 
        \end{aligned}
    \end{equation}
    est  \(  C^{\infty}\), sa dérivée est 
    \begin{equation}
        E'(t)=z e^{tz}.
    \end{equation}
    La fonction \( E\) est développable en série entière (voir définition \ref{DefwmRzKh}) sur \( \eR\) en \( t=0\) et
    \begin{equation}
        e^{tz}=\sum_{n=0}^{\infty}\frac{ z^n }{ n! }t^n.
    \end{equation}
\end{proposition}

\begin{proof}
    Nous fixons \( z\in \eC\). Par définition \ref{DefJilXoM}, la série suivante est \(  e^{tz}\) :
    \begin{equation}
        f(t)=\sum_{n=0}^{\infty}\frac{ z^n }{ n! }t^n.
    \end{equation}
    Cette série a un rayon de convergence infini et la fonction \( f\) est donc \(  C^{\infty}\) sur \( \eR\). Nous pouvons la dériver terme à terme :
    \begin{subequations}
        \begin{align}
            f'(t)&=\sum_{n=1}^{\infty}\frac{ z^n }{ n! }nt^{n-1}\\
            &=z\sum_{n=1}^{\infty}\frac{ z^{n-1} }{ (n-1)! }t^{n-1}\\
            &=z e^{tz}.
        \end{align}
    \end{subequations}
\end{proof}

\begin{theorem}
    La fonction exponentielle vérifie les propriétés suivantes.
    \begin{enumerate}
        \item
            \( \exp\) est holomorphe.
        \item
            \( (e^z)'=e^z\).
        \item
            L'exponentielle est développable en série entière,
            \begin{equation}
                e^z=\sum_{n=0}^{\infty}\frac{ z^n }{ n! }
            \end{equation}
            et la série converge normalement sur tout compact de \( \eC\).
    \end{enumerate}
\end{theorem}

\begin{proof}
    En tant que application \( E\colon \eR^2\to \eC\), la fonction
    \begin{equation}
        E(x,y)=e^x(\cos y+i\sin y)
    \end{equation}
    est \( C^{\infty}\). De plus nous avons
    \begin{subequations}
        \begin{align}
            \frac{ \partial E }{ \partial x }(x,y)= e^{x+iy}=E(x,y)\\
            \frac{ \partial E }{ \partial y }(x,y)=iE(x,y),
        \end{align}
    \end{subequations}
    et par conséquent la fonction \( E\) vérifie les équations de Cauchy-Riemann.

    Si \( r\) est fixé, par le critère d'Abel appliqué à la suite \(r/n!\) nous savons que la série \( \sum z^n/n!\) converge normalement sur le compact \( B(0,r)\).
\end{proof}

%---------------------------------------------------------------------------------------------------------------------------
\subsection{Intégrale de Fresnel}
%---------------------------------------------------------------------------------------------------------------------------

Nous allons calculer l'\defe{intégrale de Fresnel}{intégrale!Fresnel}\index{Fresnel!intégrale}
\begin{equation}
    \int_0^{\infty} e^{-ix^2}dx=\frac{ \sqrt{\pi} }{ 2 } e^{-i\pi/4}
\end{equation}
%TODO : mettre cette référence vers Wikipédia dans la bibliographie.
en suivant la démarche présentée par \wikipedia{fr}{Intégrale_de_fresnel}{wikipédia}. Nous commençons par prouver que l'intégrale est convergente en nous contentant de justifier la convergence de
\begin{equation}
    \int_0^{\infty}\sin(x^2)dx.
\end{equation}
Pour tout \( a>0\), l'intégrale \( \int_0^a\sin(x^2)dx\) ne pose pas de problèmes. En tenant compte du lemme \ref{LemTHBSEs}, nous devons donc seulement calculer
\begin{equation}
    \lim_{b\to \infty}\int_a^b\sin(x^2)dx
\end{equation}
où \( a\) est une constante strictement positive. Nous effectuons une intégration par partie en posant
\begin{subequations}
    \begin{align}
        u&=\frac{1}{ x }&   u'&=-\frac{1}{ x^2 }\\
        v'&=x\sin(x)    & v&=\frac{ 1-\cos(x) }{2}.
    \end{align}
\end{subequations}
Notons que la primitive \( v\) a été choisie pour avoir \( v(0)=0\). Nous avons
\begin{equation}    \label{EqOdeKye}
    \int_a^b\sin(x^2)dx=\left[ \frac{ 1-\cos(x^2) }{ 2x } \right]_a^b-\int_a^b\frac{ \cos(x^2)-1 }{ 2x^2 }dx
\end{equation}
Pour le premier terme nous avons
\begin{equation}
    \lim_{b\to \infty}\left[ \frac{ 1-\cos(x^2) }{ 2x } \right]_a^b=\lim_{b\to \infty}\frac{ 1-\cos(b^2) }{ 2b }-\frac{ 1-\cos(a^2) }{ 2a }=-\frac{ 1-\cos(a^2) }{ 2a }.
\end{equation}
C'est borné. Pour le second terme de \eqref{EqOdeKye}, la fonction
\begin{equation}
    \frac{ \cos(x^2)-1 }{ 2x^2 }
\end{equation}
est majorée par la fonction \( 1/x^2\) qui est intégrable entre \( a\) et \( \infty\).


Nous allons calculer l'intégrale demandée en passant par la fonction
\begin{equation}
    f(x)= e^{-z^2}
\end{equation}
définie sur le plan complexe. Nous l'intégrons sur le chemin \( \gamma=\gamma_1+\gamma_2-\gamma_3\) indiqué à la figure \ref{LabelFigCheminFresnel}.
\newcommand{\CaptionFigCheminFresnel}{Chemin d'intégration pour l'intégrale de Fresnel}
\input{Fig_CheminFresnel.pstricks}
Ces chemins sont donnés par
\begin{equation}
    \begin{aligned}
        \gamma_1\colon \mathopen[ 0 , R \mathclose]&\to \eC \\
        t&\mapsto t, 
    \end{aligned}
\end{equation}
\begin{equation}
    \begin{aligned}
        \gamma_2\colon \mathopen[ 0 , \frac{ \pi }{ 4 } \mathclose]&\to \eC \\
        t&\mapsto R e^{it}, 
    \end{aligned}
\end{equation}
\begin{equation}
    \begin{aligned}
        \gamma_3\colon \mathopen[ 0 , R \mathclose]&\to \eC \\
        t&\mapsto t e^{i\pi/4}. 
    \end{aligned}
\end{equation}
Tout d'abord la fonction \( f\) est bien holomorphe par le critère du théorème \ref{ThokwIQwg}. Le calcul de \( \frac{ \partial f }{ \partial \bar z }\) se fait simplement en posant \( f(x,y)= e^{-(x+iy)^2}\). Le calcul est usuel :
\begin{verbatim}
----------------------------------------------------------------------
| Sage Version 4.8, Release Date: 2012-01-20                         |
| Type notebook() for the GUI, and license() for information.        |
----------------------------------------------------------------------
sage: f(x,y)=exp(-(x+I*y)**2)
sage: A=f.diff(x)+I*f.diff(y)
sage: A.simplify_full()
(x, y) |--> 0
\end{verbatim}
Nous avons donc
\begin{equation}    \label{EqfaoRgU}
    0=\int_{\gamma}f=\underbrace{\int_0^R e^{-t^2}dt}_{I_1(R)}+\underbrace{\int_0^{\pi/4} e^{-R^2 e^{2it}}Ri e^{it}dt}_{I_2(R)}+\underbrace{\int_0^R e^{-t^2 e^{i\pi/2}} e^{i\pi/4}dt}_{I_3(R)}.
\end{equation}
L'intégrale est nulle pour tout \( R\) en vertu de la proposition \ref{PrpopwQSbJg}. L'intégrale \( I_1\) est une gaussienne et nous avons
\begin{equation}
    \lim_{R\to\infty}I_1(R)=\frac{ \sqrt{\pi} }{ 2 }
\end{equation}
par l'exemple \ref{ExrgMIni}. Nous montrons maintenant que \( \lim_{R\to\infty}| I_2(R) |=0\)\footnote{Dans \cite{FresnelDavidS}, ce fait est démontré via le lemme de Jordan. Nous donnons ici une démonstration moins technologique.}. D'abord nous majorons en prenant la norme puis nous effectuons le changement de variables \( u=2t\) :
\begin{subequations}
    \begin{align}
        | I_2(R) |&\leq \int_{0}^{\pi/4}R e^{-R^2\cos(2t)}dt\\
        &=\frac{ R }{ 2 }\int_0^{\pi/2} e^{-R^2\cos(u)}du.
    \end{align}
\end{subequations}
Nous savons que le graphe du cosinus est concave : il reste au-dessus de la droite que joint \( (0,1)\) à \( (\frac{ \pi }{2},0)\). Du coup \( \cos(u)\geq 1-\frac{ 2 }{ \pi }u\) et par conséquent
\begin{equation}
        e^{-R^2\cos(u)}\leq  e^{-R^2(1-\frac{ 2 }{ \pi }u)}= e^{R^2(\frac{ 2 }{ \pi }u-1)}.
\end{equation}
Nous effectuons l'intégrale
\begin{subequations}
    \begin{align}
        | I_2(R) |&\leq \frac{ R }{2}\int_0^{\pi/2} e^{-R^2} e^{\frac{ 2R^2 }{ \pi }u}du\\
        &=\frac{ R }{2} e^{-R^2}\left[ \frac{ \pi }{ 2R^2 } e^{2R^2 u/\pi} \right]_0^{\pi/2}\\
        &=\frac{ \pi }{ 4R }-\frac{ \pi e^{-R^2} }{ 4R },
    \end{align}
\end{subequations}
et nous avons bien \( \lim_{R\to\infty}| I_2(R) |=0\). Nous passons à la troisième intégrale. En tenant compte que \(  e^{i\pi/2}=i\), nous avons
\begin{subequations}
    \begin{align}
        I_3(R)&=-\int_0^R e^{-\gamma_3(t)^2} e^{i\pi/4}dt\\
        &=-\frac{ 1+i }{ \sqrt{2} }\int_0^R e^{-t^2} e^{2i\pi/4}\\
        &=-\frac{ 1+i }{ \sqrt{2} }\int_0^R e^{-it^2}.
    \end{align}
\end{subequations}
En passant à la limite \( R\to 0 \), de l'équation \eqref{EqfaoRgU} il ne reste que
\begin{equation}
    0=\frac{ \sqrt{2} }{2}-\frac{ 1+i }{ \sqrt{2} }\int_0^{\infty} e^{-it^2}dt,
\end{equation}
ce qui signifie que
\begin{equation}
    \int_0^{\infty} e^{-it^2}dt=\frac{ \sqrt{2\pi} }{ 2(1+i) }=\frac{ \sqrt{\pi} }{2} e^{-i\pi/4}.
\end{equation}

%+++++++++++++++++++++++++++++++++++++++++++++++++++++++++++++++++++++++++++++++++++++++++++++++++++++++++++++++++++++++++++
\section{Théorème de Weierstrass}
%+++++++++++++++++++++++++++++++++++++++++++++++++++++++++++++++++++++++++++++++++++++++++++++++++++++++++++++++++++++++++++

\begin{theorem}[Théorème de Weierstrass\cite{uTyBDj}]       \label{ThoArYtQO}
    Soit \( (f_n)\) une suite de fonctions holomorphes sur un ouvert \( \Omega\) de \( \eC\) que nous supposons converger uniformément sur tout compact vers \( f\). Alors \( f\) est holomorphe sur \( \Omega\) et pour tout \( k\) nous avons
    \begin{equation}
        f^{(k)}_n\to f^{(k)}
    \end{equation}
    uniformément sur tout compact.

    Dit en peu de mots, la limite uniforme d'une suite de fonctions holomorphes est holomorphe, et on peut permuter la limite avec la dérivation.
\end{theorem}
\index{compacité}
\index{suite!de fonctions intégrables}
\index{fonction!définie par une intégrale}
\index{fonction!holomorphe}
\index{limite!inversion}
\index{limite!de fonctions holomorphes}

\begin{proof}
    Chacune des fonctions \( f_n\) étant holomorphes, si \( a\in \Omega\) et \( r\) est tel que \( B(a,r)\subset \Omega\), nous avons par la formule de Cauchy \ref{ThoUHztQe} :
    \begin{equation}
        f_n(z)=\frac{1}{ 2\pi i }\int_{\partial B(a,r)}\frac{ f_n(\xi) }{ \xi-z }d\xi
    \end{equation}
    pour tout \( z\) dans un boule \( B(a,\rho)\) incluse dans \( B(a,r)\). Étant donné que le cercle \( \partial B\) est compact, elle y est majorée par une constante \( M\). Montrons que de plus nous pouvons choisir \( M\) de telle façon à avoir \( | f_n(\xi) |\leq M\) pour tout \( n\) et tout \( z\) en même temps. D'abord nous utilisons la continuité de la limite \( f\) sur le compact \( \partial B \) pour poser \( A=\max_{z\in\partial B}| f(z) |\). Ensuite nous considérons un \( \epsilon>0\) et \( N\) tel que \( |\ f_n-f \|_{\partial B}\leq \epsilon\) pour tout \( n\geq N\). Nous savons maintenant que
    \begin{equation}
        \{ | f_n(\xi) |\tq n \geq N,\xi\in\partial B \}
    \end{equation}
    est majoré par \( A+\epsilon\). Nous posons enfin
    \begin{equation}
        B=\max_{n\leq N}\max_{\xi\in\partial B}| f_n(z) |,
    \end{equation}
    et alors le nombre \( M=\max\{ A+\epsilon,B \}\) majore \( | f_n(\xi) |\) pour tout \( n\) et tout \( \xi\in\partial B\).
    
    De plus pour tout \( \xi\in\partial B\) et pour tout \( z\) dans la petite boule, nous avons \( | \xi-z |>r-\rho\), donc  la fonction dans l'intégrale est majorée par une constante ne dépendant ni de \( n\) ni de \( \xi\). Nous pouvons donc permuter l'intégrale et la limite sur \( n\) :
    \begin{equation}
        f(z)=\frac{1}{ 2i\pi }\int_{\partial B}\frac{ f(\xi) }{ \xi-z }.
    \end{equation}
    Cela implique que la fonction \( f\) est holomorphe par le corollaire \ref{CorwfHtJu}.

    Nous voudrions maintenant parler des dérivées des \( f_n\) et de \( f\). Pour cela nous voulons permuter l'intégrale et les dérivées, ce qui est fait au corollaire \ref{CorNxTjEj} :
    \begin{equation}
        f_n^{(k)}=\frac{1}{ 2\pi i }\int_{\partial B(z_0,r)}\frac{ f(\omega) }{ (\omega-z)^{k+1} }d\omega.
    \end{equation}
    Nous voulons la convergence sur tout compact contenu dans l'ouvert \( \Omega\). Pour ce faire, nous allons considérer un compact \( K\subset \Omega\) et prouver la convergence uniforme dans toute boule de la forme \( B(z_0,r)\) avec \( z_0\in K\) et \( B(z_0,r)\subset \Omega\). Pour chaque tel couple \( (z_0,r)\), nous aurons un \( N_{(z_0,r)}\in \eN\) tel que si \( n\geq N_{(z_0,r)}\),
    \begin{equation}
        \| f_n^{(k)}-f^{(k)} \|_{B(z_0,r)}\leq \epsilon.
    \end{equation}
    Vu que ces boules \( B(z_0,r)\) forment un recouvrement de \( K\) par des ouverts, nous pouvons en retirer un sous-recouvrement fini et prendre, comme \( N\), le maximum des \( N_{(z_0,r)}\) correspondants. Pour ce \( N\) nous aurons
    \begin{equation}
        \| f_n^{(k)}-f^{(k)} \|_K\leq \epsilon.
    \end{equation}
    Au travail !

    Pour \( z\in B(z_0,r)\) nous considérons \( r'>r\) tel que \( B(z_0,r')\subset \Omega\) et nous avons
    \begin{subequations}
        \begin{align}
            | f^{(k)}_n(z)-f^{(k)}(z) |&=\left| \frac{1}{ 2\pi i }\int_{\partial B(z_0,r')}\frac{ f_n(\xi)-f(\xi) }{ (\xi-z)^{k+1} }d\xi \right| \\
            &\leq\frac{1}{ 2\pi }\int_{\partial B(z_0,r')}\frac{ | f_n(\xi)-f(\xi) | }{ | r-r' |^{k+1} }d\xi.
        \end{align}
    \end{subequations}
    Nous avons pris ce \( r'\) de telle manière à ce que \( | \xi-z |\) soit borné par le bas par \( | r-r' |\); sinon la majoration que nous venons de faire ne marche pas. Étant donné que \( f_n\to f\) uniformément, nous pouvons considérer \( n\) assez grand pour que le numérateur soit plus petit que \( \epsilon\) indépendamment de \( \xi\) et de \( z\). Donc pour un \( n\) assez grand,
    \begin{equation}
        | f^{(k)}_n(z)-f^{(k)}(z) |\leq \frac{ \epsilon }{ 2\pi }\frac{ 2\pi r' }{ | r-r' |^{k+1} }
    \end{equation}
    pour tout \( z\in B(z_0,r)\). Donc nous avons convergence uniforme \( f_n^{(k)}\to f^{(k)}\) sur cette boule. Par l'argument de compacité donné plus haut, nous avons la convergence uniforme sur tout compact.
\end{proof}

%+++++++++++++++++++++++++++++++++++++++++++++++++++++++++++++++++++++++++++++++++++++++++++++++++++++++++++++++++++++++++++ 
\section{Théorème de Montel}
%+++++++++++++++++++++++++++++++++++++++++++++++++++++++++++++++++++++++++++++++++++++++++++++++++++++++++++++++++++++++++++

\begin{theorem}[Montel\cite{KXjFWKA}]   \label{ThoXLyCzol}
    Soit \( \Omega\) un ouvert de \( \eC\) et \( \mF\) une famille de fonctions holomorphes sur \( \Omega\), uniformément bornée sur tout compact de \( \Omega\). Alors de toute suite dans \( \mF\) nous pouvons extraire une sous-suite convergeant uniformément sur tout compact de \( \Omega\).
\end{theorem}
\index{théorème!Montel}
\index{compacité!utilisation!théorème de Montel}
\index{suite!de fonctions!théorème de Montel}
\index{fonction!holomorphe!théorème de Montel}

\begin{proof}

    \begin{subproof}
    \item[Un ensemble équicontinu]

        Nous commençons par prendre une suite de compacts dans \( \Omega\) comme dans le lemme \ref{LemGDeZlOo}, et une suite \( \delta_n\) de réels strictement positifs tels que
        \begin{equation}
            B(z,2\delta_n)\subset K_{n+1}
        \end{equation}
        pour tout \( z\in K_n\). Soient \( x,y\in K_n\) tels que \( | x-y |<\delta_n\); nous notons \( \partial B(x,2\delta_n)\) le cercle de rayon \( 2\delta_n\) autour de \( x\), parcouru dans le sens positif. La formule de Cauchy \ref{EqPzUABM} nous donne
        \begin{equation}
                f(x)-f(y)=\frac{1}{ 2\pi i }\int_{\partial B}\left( \frac{ f(\xi) }{ \xi-x }-\frac{ f(\xi) }{ \xi-y } \right)d\xi
                =\frac{ x-y }{ 2\pi i }\int_{\partial B}\frac{ f(\xi) }{ (\xi-x)(\xi-y) }d\xi
        \end{equation}
        Nous majorons ça par
        \begin{equation}
            \big| f(x)-f(y) \big|\leq\frac{ | x-y | }{ 2\pi }\int_{\partial B}\frac{ | f(\xi) | }{ 2\delta_n^2 }d\xi\leq \frac{ | x-y | }{ \delta_n }M_n.
        \end{equation}
        Justifications :
        \begin{itemize}
            \item 
                \( | \xi-x |=2\delta_n\) et \( | \xi-y |\geq \delta_n\) parce que \( \xi\) est au mieux sur le rayon passant par \( x\) et \( y\).
            \item
                \( | f(\xi) |\leq M_n\) où \( M_n\) est la borne uniforme de \( \mF\) sur le compact \( K_n\). 
            \item
                Nous avons aussi fini par calculer l'intégrale dans laquelle il ne restait plus rien, ça a donné la circonférence du cercle de rayon \( 2\delta_n\).
        \end{itemize}
        Jusqu'à présent nous avons prouvé que l'ensemble
        \begin{equation}
            \mF_n=\{ f|_{K_n}\tq f\in\mF \}
        \end{equation}
        est équicontinu. Il est aussi équiborné par hypothèse.

    \item[Application du théorème d'Ascoli]

        L'ensemble \( \mF_n\) vérifie les hypothèses du théorème d'Ascoli \ref{ThoKRbtpah}. Donc l'ensemble \( \mF_n\) est relativement compact dans \( C(K_n,\eC)\) pour la norme uniforme. Autrement dit l'ensemble \( \bar\mF\) est compact et si nous avons une suite de fonctions dans \( \mF_n\), il existe une sous-suite convergeant dans \( \bar\mF_n\), c'est à dire uniformément. Autrement dit il existe une fonction strictement croissante \( \varphi\colon \eN\to \eN\) telle que la suite \( k\mapsto f_{\varphi(k)}\) converge uniformément sur \( K_n\). La limite n'est cependant pas spécialement dans \( \mF_n\).

    \item[L'argument diagonal]

        La suite \( k\mapsto f_{\varphi_1\circ\ldots\varphi_k(k)}\) converge uniformément sur tous les \( K_n\). Si \( K\) est un compact de \( \Omega\), alors les petites propriétés sympas du lemme \ref{LemGDeZlOo} nous disent que \( K\subset \Int(K_m)\) pour un certain \( m\). Ladite suite convergeant uniformément sur \( K_m\), elle converge uniformément sur \( K\) et nous avons montré la convergence uniforme sur tout compact de \( \Omega\).

    \end{subproof}
\end{proof}

\begin{corollary}[\cite{KXjFWKA}]
    Soit \( \Omega\) un ouvert connexe borné de \( \eC\) et \( a\in \Omega\). Soit \( f\) holomorphe sur \( \Omega\) telle que \( f(a)=a\) et \( | f'(a) |<1\).

    Alors de \( (f^n)\) on peut extraire une sous-suite convergeant uniformément sur tout compact de \( \Omega\) vers la fonction constante \( a\).
\end{corollary}
\index{prolongement!analytique!utilisation}

\begin{proof}
    Nous considérons un voisinage de \( a\) inclus à \( \Omega\); sachant que \( | f(a) |<1\), nous trouvons un voisinage encore plus petit de \( a\) sur lequel \( | f'(z) |<1\).  Soit donc \( r\) tel que \( \overline{ B(a,r) }\subset \Omega\) et tel que \( | f'(z) |<1\) sur \( \overline{ B(a,r) }\). Étant donné que \( f'(z)\) est continue sur le compact \( \overline{ B(a,r) }\), nous en prenons le maximum \( \lambda\) (qui est strictement inférieur à \( 1\)) et nous avons au final
    \begin{equation}
        | f'(z) |\leq \lambda< 1
    \end{equation}
    pour tout \( z\in \overline{ B(a,r) }\). Le théorème des accroissements finis \ref{val_medio_2} nous dit que
    \begin{equation}
        \big| f(z)-a \big|\leq \lambda| z-a |
    \end{equation}
    pour tout \( z\in\overline{ B(a,r) }\). C'est ici que nous utilisons l'hypothèse de convexité de \( \Omega\). Nous montrons alors par récurrence que 
    \begin{equation}    \label{EqIQUzKpg}
        \big| f^n(z)-a \big|\leq \lambda^n| z-a |\leq \lambda^nr\leq r.
    \end{equation}
    L'ensemble \( A=\{ f^n\tq n\geq 1 \}\) est donc uniformément borné sur \( \overline{ B(a,r) }\) par \( a+r\). Autre manière de le dire : pour tout \( z\in\overline{ B(a,r) }\) nous avons
    \begin{equation}
        f^n(z)\in\overline{ B(a,r) }.
    \end{equation}
    La suite \( (f^n)\) est donc uniformément bornée sur tout compact de \( B(a,r)\). Le théorème de Montel \ref{ThoXLyCzol} nous indique que l'on peut extraire une sous-suite convergente uniformément sur tout compact. Au vu de \eqref{EqIQUzKpg} cette convergence ne peut avoir lieu que vers une fonction \( g\) qui vaut la constante \( a\) sur \( B(a,r)\).

    D'autre par la fonction \( g\) est holomorphe en tant que limite uniforme de fonctions holomorphes, théorème \ref{ThoArYtQO}. Or une fonction holomorphe constante sur un ouvert est constante sur tout son domaine d'holomorphie (principe d'extension analytique, théorème \ref{ThoAVBCewB}).
\end{proof}


%+++++++++++++++++++++++++++++++++++++++++++++++++++++++++++++++++++++++++++++++++++++++++++++++++++++++++++++++++++++++++++
\section{Espaces de Bergman}
%+++++++++++++++++++++++++++++++++++++++++++++++++++++++++++++++++++++++++++++++++++++++++++++++++++++++++++++++++++++++++++

Source : \cite{ytMOpe}.

Soit \( \Omega\) un borné dans \( \eC\) et \( D\) le disque unité ouvert de \( \eC\).

\begin{definition}
    L'\defe{espace de Bergman}{espace!de Bergman}\index{Bergman (espace)} sur \( \Omega\), noté \( A^2(\Omega)\)\nomenclature[Y]{\( A^2(\Omega)\)}{espace de Bergman} est l'espace des fonctions holomorphes sur \( \Omega\) qui sont en même temps dans \( L^2(\Omega)\).
\end{definition}
Nous mettons sur \( A^2(\Omega)\) le produit scalaire usuel hérité de \( L^2\) :
\begin{equation}
    \langle f, g\rangle =\int_{\Omega}f(z)\overline{ g(z) }dz.
\end{equation}

\begin{lemma}   \label{LemIZxKfB}
    Soit \( K\subset \Omega\) un compact et \( f\in A^2(\Omega)\). Alors
    \begin{equation}
        \max_{z\in K}| f(z) |\leq \frac{1}{ \sqrt{\pi} }\frac{1}{ d(K,\partial \Omega) }\| f \|_2.
    \end{equation}
\end{lemma}

\begin{proof}
    Soient \( a\in \Omega\) et \( r>0\) tels que \( B(a,r)\subset\Omega\). Nous considérons aussi \( \rho\leq r\). La formule de Cauchy \eqref{EqPzUABM} nous donne   
    \begin{equation}
        f(a)=\frac{1}{ 2\pi i }\int_{B(a,\rho)}\frac{ f(\xi) }{ \xi-a }f\xi=\frac{1}{ 2\pi }\int_0^{2\pi}f(a+\rho e^{i\theta})d\theta
    \end{equation}
    où nous avons utilisé le chemin \( \gamma(\theta)=a+\rho e^{i\theta}\), \( \gamma'(\theta)=i\rho e^{i\theta}\) et \( \rho=| \xi-a |\). Maintenant une astuce est d'écrire
    \begin{equation}
        \frac{ r^2 }{2}f(a)=\int_0^rf(a)\rho d\rho,
    \end{equation}
    et d'y substituer la valeur de \( f(a)\) que nous venons de calculer :
    \begin{subequations}
        \begin{align}
            \frac{ r^2 }{2}f(a)&=\int_0^r\frac{1}{ 2\pi }\int_0^{2\pi}f(a+\rho e^{i\theta})d\theta\rho d\rho\\
            &=\frac{1}{ 2\pi }\int_{B(a,r)}f(z)dz   &   \text{passage aux polaires}\\
            &=\frac{1}{ 2\pi }\langle 1, f\rangle_B   &   \text{produit scalaire sur \( B(a,r)\)}\\
            &\leq\frac{1}{ 2\pi }\sqrt{\langle 1, 1\rangle_B\langle f, f\rangle_B }\\
        \end{align}
    \end{subequations}
    Nous avons donc
    \begin{equation}
        r^2f(a)\leq \frac{1}{ \pi }\sqrt{\langle 1, 1\rangle_B\langle f, f\rangle_B},
    \end{equation}
    et donc
    \begin{equation}
        \pi r^2 f(a)\leq \sqrt{\pi r^2}\| f \|_2,
    \end{equation}
    parce que \( \langle f, f\rangle_B\leq \| f \|_2^2\). En effet le produit scalaire \( \| . \|_2\) est donné par une intégrale sur \( \Omega\) alors que \( B(a,r)\subset \Omega\) et que la fonction qu'on y intègre est positive (c'est \( | f(z) |^2\)). En simplifiant,
    \begin{equation}
        f(a)\leq \frac{1}{ \sqrt{\pi}r }\| f \|_2.
    \end{equation}
    Mais \( r\) a été choisit pour avoir \( B(a,r)\subset\Omega\), donc \( r\leq d(a,\partial \Omega)\) et
    \begin{equation}
        | f(a) |\leq \frac{1}{ d(a,\partial\Omega)\sqrt{\pi} }\| f \|_2.
    \end{equation}
    
    Maintenant si nous prenons \( a\in K\), nous avons encore la minoration \( d(a,\partial K)\leq d(a,\partial \Omega)\) et donc
    \begin{equation}
        | f(a) |\leq\frac{1}{ d(a,\partial K)\sqrt{\pi} }\| f \|_2.
    \end{equation}

\end{proof}

\begin{theorem}
    Soit \( \Omega\) un ouvert de \( \eC\).
    \begin{enumerate}
        \item
            L'espace \( A^2(\Omega)\) est un espace de Hilbert.
        \item
            Si \( D\) est la boule unité dans \( \eC\), une base hilbertienne de \( A^2(D)\) est donnée par les fonctions
            \begin{equation}
                e_n(z)=\sqrt{\frac{ n+1 }{ \pi }}z^n
            \end{equation}
            pour \( n\geq 0\).
    \end{enumerate}
\end{theorem}

\begin{proof}
    Nous commençons par montrer que \( A^2(\Omega)\) est complet. Pour cela nous considérons une suite de Cauchy \( (f_n)\) dans \( A^2(\Omega)\) et un compact \( K\subset \Omega\). Nous savons par le lemme \ref{LemIZxKfB} que
    \begin{equation}
        \max_{z\in K}\big| f_n(z)-f_m(z) \big|\leq \frac{1}{ \sqrt{\pi}d(K,\partial\Omega) }\| f_n-f_m \|_2.
    \end{equation}
    Donc \( f_n\) converge uniformément sur \( K\). Par le théorème de Weierstrass \ref{ThoArYtQO}, la fonction \( f\) est holomorphe. Il existe donc une fonction holomorphe \( f\) qui est limite uniforme sur tout compact de \( \Omega\) de la suite \( (f_n)\).

    Mais \( L^2(\Omega)\) étant complet, la suite \( (f_n)\) a une limite \( g\in L^2(\Omega)\). Ce que nous voudrions faire est prouver que \( f=g\). Notons que tel quel, ce n'est pas vrai parce que \( f\) est une vraie fonction alors que \( g\) est une classe. Ce que nous enseigne la proposition \ref{PropWoywYG} est qu'il existe une sous-suite (qu'on note \( (g_n)\)) qui converge vers \( g\) presque partout. Dans cette dernière phrase, \( g_n\) et \( g\) sont de vraies fonctions, des représentants des classes dans \( L^2\).

    Nous déduisons que \( f=g\) presque partout (ici \( f\) et \( g\) sont les fonctions) parce que la sous-suite converge uniformément vers \( f\) en même temps que presque partout vers \( g\). Donc \( f=g\) dans \( L^2(\Omega)\) (ici \( f\) et \( g\) sont les classes). Donc \( f\in L^2(\Omega)\) et l'espace \( A^2(\Omega)\) est de Hilbert.

    Il nous faut encore prouver que \( (e_n)_{n\geq 0}\) est une base orthonormale. En ce qui concerne les produits scalaires,
    \begin{subequations}
        \begin{align}
            \langle e_m, e_n\rangle &=\sqrt{\frac{ (m+1)(n+1) }{ \pi }}\int_Dz^n\overline{ z^m }dz\\
            &=\sqrt{\frac{ (m+1)(n+1) }{ \pi^2 }}\int_0^1\rho\,d\rho\int_0^{2\pi}d\theta \rho^{m+n} e^{i\theta(n-m)}\\
            &=\sqrt{\frac{ (m+1)(n+1) }{ \pi^2 }}\frac{1}{ m+n+2 }\underbrace{\int_{0}^{2\pi} e^{i\theta(n-m)}d\theta}_{2\pi \delta_{mn}}\\
            &=\sqrt{\frac{ (n+1)^2 }{ \pi^2 }}\frac{1}{ 2n+2 }2\pi \delta_{nm}\\
            &=\delta_{nm}.
        \end{align}
    \end{subequations}
    Donc les fonctions données sont bien orthonormales. Nous devons montrer qu'elles sont denses dans \( A^2(D)\). Soit \( f\in A^2(D)\) et \( c_n(f)=\langle f, e_n\rangle \); nous allons montrer que
    \begin{equation}
        \| f \|_2^2=\sum_{n=0}^{\infty}| \langle f, e_n\rangle  |^2,
    \end{equation}
    parce que le point \ref{ItemQGwoIx} du théorème \ref{ThoyAjoqP} nous indique que ce sera suffisant pour avoir une base hilbertienne.

    Étant donné que \( f\) est holomorphe sur \( D\), le théorème \ref{ThoUHztQe} nous développe \( f\) en série entière :
    \begin{equation}    \label{EqObkbPK}
        f(z)=\sum_{k=0}^{\infty}a_kz^k.
    \end{equation}
    En permutant la somme avec le produit scalaire,
    \begin{equation}
        c_n(f)=\int_Df(z)\bar e_n(z)=\sqrt{\frac{ n+1 }{ \pi }}\int_Df(z)\bar z^ndz.
    \end{equation}
    Afin de profiter de la convergence uniforme de la série \eqref{EqObkbPK} à l'intérieur de \( D\), nous allons exprimer l'intégrale sur \( D\) comme une intégrale sur \( | z |<r\) en faisant tendre \( r\) vers \( 1\) (par le bas). Pour ce faire nous considérons les fonctions
    \begin{equation}
        g_k(z)=\begin{cases}
            f(z)\bar z^n    &   \text{si \( | z |<1-1/k\)}\\
            0    &    \text{sinon.}
        \end{cases}
    \end{equation}
    Ces fonctions sont intégrables sur \( D\) et dominées par \( f(z)\bar z^n\) qui est intégrable sans dépendre de \( k\). Mais nous avons évidemment \( g_k(z)\to f(z)\bar z^n\). Le théorème de la convergence dominée permet alors de permuter l'intégrale et la limite \( k\to \infty\). Cela nous permet d'écrire
    \begin{equation}
        c_n(f)=\sqrt{\frac{ n+1 }{ \pi }}\lim_{r\to 1^-}\int_{| z |<r}\bar z^nf(z)dz=\sqrt{\frac{ n+1 }{ \pi }}\lim_{r\to 1^-}\int_{| z |<r}\sum_{k=0}^{\infty}a_kz^k\bar z^n.
    \end{equation}
    Par la convergence uniforme de la série entière \emph{à l'intérieur} du disque \( D\) nous pouvons permuter l'intégrale et la somme (proposition \ref{PropfeFQWr}) :
    \begin{equation}
        c_n(f)=\sqrt{\frac{ n+1 }{ \pi }}\lim_{r\to 1^-}\sum_{k=0}^{\infty}a_k\int_{| z |<r}z^k\bar z^ndz.
    \end{equation}
    L'intégrale proprement dite est vite calculée et vaut
    \begin{equation}
        \int_{| z |<1}\bar z^nz^kdz=\frac{ \pi r^{2n+2} }{ n+1 }\delta_{kn}.
    \end{equation}
    Nous pouvons donc continuer le calcul de \( c_n(f)\) en effectuant la somme sur \( k\) qui se réduit à changer \( k\) en \( n\) puis en effectuant la limite :
    \begin{equation}
        c_n(f)=\sqrt{\frac{ n+1 }{ \pi }}\lim_{r\to 1^-}\sum_ka_k\frac{ \pi r^{2n+2} }{ n+1 }\delta_{kn}=\sqrt{\frac{ \pi }{ n+1 }}a_n.
    \end{equation}
    
    Nous effectuons le même genre de calculs pour évaluer \( \| f \|^2_2\) :
    \begin{subequations}
        \begin{align}
            \| f \|_2^2&=\int_D| f(z) |^2dz\\
            &=\lim_{r\to 1^-}\int_{| z |<r}f(z)\sum_{k=0}^{\infty}\bar a_k\bar z_kdz\\
            &=\lim_{r\to 1^-}\sum_{k=0}^{\infty}\bar a_k\int_{| z |<r}f(z)\bar z^kdz&\text{permuter \( \sum\) et \( \int\)}\\
            &=\lim_{r\to 1^-}\sum_{k=0}^{\infty}\bar a_ka_k\frac{ \pi r^{2k+2} }{ k+1 }&\text{intégrale déjà faite}.
        \end{align}
    \end{subequations}
    Mais nous savons déjà que \( c_n(f)=\sqrt{\pi/(n+1)}\), donc ce qui est dans la somme est \( \pi\bar a_ka_k/(n+1)=| c_k(f) |^2\). Nous avons donc
    \begin{equation}
        \| f \|^2_2=\lim_{r\to 1^-}\sum_{k=0}^{\infty}| c_k(f) |^2 r^{2k+2}.
    \end{equation}
    La fonction (de \( r\)) constante \( | c_k(f) |^2\) domine \( | c_k(f)r^{2k+2} |\) tout en ayant une somme (sur \( k\)) qui converge; en effet la proposition \ref{PropHKqVHj} nous indique que \( \sum_j| c_k(f) |^2\leq \| f \|_2^2\). Le théorème de la convergence dominée nous permet d'inverser la limite et la somme pour obtenir le résultat attendu :
    \begin{equation}
        \| f \|_2^2=\sum_{k=0}^{\infty}| c_k(f) |^2.
    \end{equation}
\end{proof}
