% This is part of Mes notes de mathématique
% Copyright (c) 2012-2013
%   Laurent Claessens
% See the file fdl-1.3.txt for copying conditions.

%+++++++++++++++++++++++++++++++++++++++++++++++++++++++++++++++++++++++++++++++++++++++++++++++++++++++++++++++++++++++++++
\section{Dérivabilité au sens complexe}
%+++++++++++++++++++++++++++++++++++++++++++++++++++++++++++++++++++++++++++++++++++++++++++++++++++++++++++++++++++++++++++

Dans cette partie, nous désignons par \( \Omega\) un ouvert de \( \eC\). Une fonction \( f\colon \Omega\to \eC\) est $\eC$-dérivable si la limite
\begin{equation}
    \lim_{h\to 0} \frac{ f(a+h)-f(a) }{ h }
\end{equation}
existe. Dans ce cas, cette limite est la dérivée de \( f\).

Nous identifions \( \eR^2\) à \( \eC\) par l'application \( \varphi\colon \eR^2\to \eC\) l'application \( \varphi(x,y)=x+iy\).

\begin{lemma}
    Une application \( A\colon \eC\to \eC\) est \( \eC\)-linéaire si et seulement si sa matrice en tant qu'application \( \eR^2\to \eR^2\) est la de forme
    \begin{equation}
        \begin{pmatrix}
            \alpha    &   \beta    \\ 
            -\beta    &   \alpha    
        \end{pmatrix}.
    \end{equation}
\end{lemma}

\begin{proposition}
    Une fonction \( f\colon \eC\to \eC\) est $\eC$-dérivable au point \( z_0=x_0+iy_0\) si et seulement si la fonction \( F=\varphi^{-1}\circ f\circ \varphi\) est différentiable en \( (x_0,y_0)\) et si la matrice de \( dF\) est de la forme
    \begin{equation}
        dF=\begin{pmatrix}
            \alpha    &   \beta    \\ 
            -\beta    &   \alpha    
        \end{pmatrix},
    \end{equation}
    c'est à dire si \( dF\) fournit une application \( \eC\)-linéaire.
\end{proposition}

\begin{proof}
    Nous décomposons \( f\) en parties réelles et imaginaires :
    \begin{equation}
        f(x+iy)=P(x,y)+iQ(x,y)
    \end{equation}
    où \( P\) et \( Q\) sont des fonctions réelles. La jacobienne de \( F\) est la matrice
    \begin{equation}
        \begin{pmatrix}
            \frac{ \partial P }{ \partial x }    &   \frac{ \partial P }{ \partial y }    \\ 
            \frac{ \partial Q }{ \partial x }    &   \frac{ \partial Q }{ \partial y }    
        \end{pmatrix},
    \end{equation}
    et la condition dont nous parlons s'écrit comme le système
    \begin{subequations}
        \begin{numcases}{}
            \frac{ \partial P }{ \partial x }=\frac{ \partial Q }{ \partial y }\\
            \frac{ \partial P }{ \partial y }=-\frac{ \partial Q }{ \partial x}.
        \end{numcases}
    \end{subequations}
    Si \( F\) est différentiable en \( (x_0,y_0)\) alors nous avons
    \begin{equation}        \label{EqwlVfiR}
        F\big( (x_0,y_0)+(h,k) \big)=F(x_0,y_0)+dF_{(x_0,y_0)}\begin{pmatrix}
            h    \\ 
            k    
        \end{pmatrix}+s(| h |+| k |)
    \end{equation}
    où \( s\) est une fonction vérifiant \( \lim_{t\to 0} \frac{ s(t) }{ t }=0\). Soit
    \begin{equation}
        dF_{(x_0,y_0)}=\begin{pmatrix}
            \alpha    &   \beta    \\ 
            -\beta    &   \alpha    
        \end{pmatrix}.
    \end{equation}
    Si nous posons \( \sigma=\alpha-i\beta\) et \( w=h+ik\), l'équation \eqref{EqwlVfiR} s'écrit dans \( \eC\) sous la forme
    \begin{equation}        \label{EqYFmoiM}
        f(z_0+w)=f(z_0)+\sigma w+s(|w|),
    \end{equation}
    ce qui implique que \( f\) est $\eC$-dérivable en \( z_0\).

    Supposons maintenant que \( f\) soit $\eC$-dérivable en \( z_0\). Alors nous avons
    \begin{equation}
        f'(z_0)=\lim_{w\to 0} \frac{ f(z_0+w)-f(z_0) }{ w }=\sigma\in \eC,
    \end{equation}
    ce qui se récrit sous la forme
    \begin{equation}
        \lim_{w\to 0} \frac{ f(z_0+w)-f(z_0)-\sigma w }{ w }=0.
    \end{equation}
    Si nous posons \( z_0=x_0+iy_0\), \( w=h+ik\) et \( \sigma=\alpha-i\beta\) nous avons
    \begin{equation}
        \lim_{(h,k)\to (0,0)} \left| \frac{ F\big( (x_0,y_0)+(h,k) \big)-F(x_0,y_0)-\begin{pmatrix}
            \alpha    &   \beta    \\ 
            -\beta    &   \alpha    
        \end{pmatrix}\begin{pmatrix}
            h    \\ 
            k    
        \end{pmatrix}}{ | w | } \right| =0,
    \end{equation}
    ce qui signifie que \( F\) est différentiable et que sa différentielle est la matrice
    \begin{equation}    \label{EqMLtbLD}
       \begin{pmatrix}
           \alpha &   \beta    \\ 
           -\beta &   \alpha    
       \end{pmatrix}.
    \end{equation}
\end{proof}

La matrice \eqref{EqMLtbLD} est, vue dans \( \eR^2\), la matrice de multiplication dans \( \eC\) par \( \alpha-i\beta=f'(z_0)\). En d'autre termes, dans \( \eC\) nous avons
\begin{equation}
    df_{z_0}z=f'(z_0)z,
\end{equation}
et en particulier la différentielle est donnée par
\begin{equation}        \label{EqPropZOkfmO}
    df_{z_0}=f'(z_0)dz.
\end{equation}

Notons que la formule \eqref{EqYFmoiM} donne un \defe{développement limité}{développement!limité!fonction holomorphe} pour les fonctions holomorphes. Si \( f\) est holomorphe en \( z_0\) alors si \( z\) est dans un voisinage de \( z_0\), il existe une fonction \( s\colon \eR\to \eC\) telle que \( \lim_{t\to 0} s(t)/t=0\) et 
\begin{equation}    \label{EqptwBFG}
    f(z)=f(z_0)+f'(z_0)(z-z_0)+s(| z-z_0 |).
\end{equation}

Nous introduisons les opérateurs\nomenclature[Y]{\( \partial_z\),\( \partial_{\bar z}\)}{dérivées partielles d'une fonction complexe}
\begin{subequations}
    \begin{align}
        \frac{ \partial  }{ \partial z }=\partial=\frac{ 1 }{2}\left( \frac{ \partial  }{ \partial x }-i\frac{ \partial  }{ \partial y } \right)\\
        \frac{ \partial  }{ \partial \bar z }=\bar\partial=\frac{ 1 }{2}\left( \frac{ \partial  }{ \partial x }+i\frac{ \partial  }{ \partial y } \right)
    \end{align}
\end{subequations}
Si \( f\) est une fonction $\eC$-dérivable représentée par la fonction \( F=P+iQ\), les équations de Cauchy-Schwartz signifient que \( \Delta P=\Delta Q=0\), c'est à dire que la fonction \( f\) a des composantes harmoniques.


%+++++++++++++++++++++++++++++++++++++++++++++++++++++++++++++++++++++++++++++++++++++++++++++++++++++++++++++++++++++++++++
\section{Fonctions holomorphes}
%+++++++++++++++++++++++++++++++++++++++++++++++++++++++++++++++++++++++++++++++++++++++++++++++++++++++++++++++++++++++++++
\label{SecoLNvnO}

\begin{definition}  \label{DefMMpjJZ}
    Soit \( \Omega\) un ouvert dans \( \eC\). Une fonction \( f\colon \Omega\to \eC\) est \defe{holomorphe}{holomorphe}\index{fonction!holomorphe} si elle est \( C^1\) et \( \eC\)-dérivable sur \( \Omega\). 
\end{definition}

\begin{proposition}
    Une fonction \( f\colon \Omega\to \eC\) est $\eC$-dérivable en \( a\in\Omega\) si et seulement si elle est différentiable en \( a\) et si \( df_a\) est une similitude.
\end{proposition}

\begin{theorem} \label{ThokwIQwg}
    Si \( f\in C^1(\Omega)\) alors nous avons équivalence des faits suivants :
    \begin{enumerate}
        \item
            \( f\) est holomorphe sur \( \Omega\),
        \item
            \( f\) vérifie \( \partial_{\bar z}f=0\).
    \end{enumerate}
\end{theorem}
%TODO : une preuve.

\begin{lemma}       \label{LemtpEOmi}
    Si \( g\) est une fonction continue dans un ouvert \( \Omega\subset \eC\) et si \( g\) admet une primitive complexe sur \( \Omega\) alors 
    \begin{equation}
        \int_{\gamma}g(z)dz=0
    \end{equation}
    pour tout chemin fermé \( \gamma\) de classe \( C^1\) contenu dans \( \Omega\).
\end{lemma}

\begin{proof}
    Nommons \( G\) une primitive de \( g\). Par définition,
    \begin{subequations}
        \begin{align}
            \int_{\gamma}g&=\int_{\gamma}G'\\
            &=\int_0^1G'\big( \gamma(t) \big)\gamma'(t)dt\\
            &=\int_0^1 (G\circ g\gamma)'(t)dt\\
            &=G(\gamma(1))-G\big( \gamma(0) \big)\\
            &=0
        \end{align}
    \end{subequations}
    parce que le chemin est fermé : \( \gamma(0)=\gamma(1)\).
\end{proof}

\begin{lemma}[Goursat\cite{Holomorphieus}]  \label{LemwbwbUR}
    Soit \( \Omega\) un ouvert dans \( \eC\) et \( f\) une fonction continue sur \( \Omega\), holomorphe sur \( \Omega\) moins éventuellement un point (nommé \( z_1\in\Omega\)). Soit \( T\), un triangle\footnote{Nous considérons ici le triangle «plein».} fermé inclus à \( \Omega\). Alors nous avons
    \begin{equation}
        \int_{\partial T}f(z)dz=0.
    \end{equation}
\end{lemma}

\begin{proof}
    Nous notons \( \gamma=\partial T\). Dans la suite nous allons définir une suite de triangles \( T^{(n)}\) et nous noterons \( \gamma_n=\partial T^{(n)}\) avec une orientation que nous allons expliquer. Pour commencer nous posons \( T^{(0)}=T\) et \( \gamma_0=\partial T^{(0)}\).

    Nous considérons le cas \( z_1\notin T\), et nous posons
    \begin{equation}
        c=l(\gamma)^{-2}| \int_{\gamma}f |.
    \end{equation}
    Notre objectif est de montrer que \( c=0\). Soit \( A,B,C\) les trois sommes du triangle; nous divisons le triangle de la façons suivante. D'abord nous considérons les points \( A',B,C'\) respectivement milieux de \( BC\), \( AC\) et \( AB\). En traçant le triangle \( A'B'C'\), nous construisons quatre triangles que nous nommons \( T^{(0)}_i\). Le théorème de Thalès assure que le périmètre de chacun des quatre triangles est la moitié du périmètre du grand triangle \( T\).

    Sur \( T\) nous choisissons l'orientation \( ABC\). De façon à être «compatible», nous choisissons les orientations \( AC'B'\), \( BA'C'\) et \( A'CB'\). La somme de ces trois triangles donne \( T\) plus le triangle \( A'C'B'\). Par conséquent nous choisissons sur le triangle central l'orientation (inverse) \( AB'C'\) de façon à avoir
    \begin{equation}
        \int_{\gamma}f=\sum_{i=1}^4\int_{\partial T^{(0)}_i}f.
    \end{equation}
    Cela implique que pour au moins un des quatre triangles (disons \( T^{(0)}_k\) pour fixer les idées) nous ayons
    \begin{equation}
        \int_{\partial T^{(0)}_k}f\geq \frac{1}{ 4 }\int_{\partial T^{(0)}}f
    \end{equation}
    Nous notons \( T^{(1)}\) ce triangle. Comme noté précédemment nous avons
    \begin{equation}
        l(\partial T^{(1)})=\frac{ 1 }{2}l(\partial T^{(0)}),
    \end{equation}
    et donc
    \begin{equation}
        l(\gamma_1)^{-2}| \int_{\gamma_1} |f=4l(\gamma_0)^{-2}| \int_{\gamma_1}f |\geq 4l(\gamma_0)^{-2}\frac{1}{ 4 }| \int_{\gamma_0}f |=c.
    \end{equation}
    En répétant le procédé nous construisons une suite de triangles \( T^{(n)}\) qui satisfont toujours
    \begin{equation}
        l(\partial T^{(n)})=\frac{1}{ 2^n }l(\partial T^{(0)}).
    \end{equation}
    Ces triangles forment une suite de fermés emboités dont le diamètre tend vers zéro. Leur intersection contient donc exactement un point (lemme \ref{LemdCOMQM}) que nous nommons \( z_0\) (et qui appartient évidemment à \( \Omega\)). Étant donné que \( f\) est holomorphe nous utilisons le développement limité \eqref{EqptwBFG} autour de \( z_0\) :
    \begin{equation}
        f(z)=f(z_0)+f'(z_0)(z-z_0)+s(| z-z_0 |)(z-z_0)
    \end{equation}
    avec \( \lim_{t\to 0} s(t)=0\). Nous posons \( g(z)=f(z_0)+f'(z_0)(z-z_0)\) et nous considérons \( \epsilon>0\). Soit \( \alpha>0\) tel que
    \begin{equation}
        | f(z)-g(z) |<\epsilon| z-z_0 |
    \end{equation}
    pour tout \( | z-z_0 |<\alpha\). Le \( \alpha\) à choisir pour obtenir cet effet est celui qui donne \( s(| z-z_0 |)<\epsilon\). Soit \( N\in \eN\) tel que \( l(\gamma_n)<\alpha\) pour tout \( n>N\). D'autre part, deux points dans un triangle sont toujours à distance moindre que la longueur d'un côté, donc pour tout \( z\in T^{(n)}\) nous avons \( | z-z_0 |<\alpha\) et par conséquent pour tout \( z\) dans \( T^{(n)}\) nous avons
    \begin{equation}
        | f(z)-g(z) |<\epsilon| z-z_0 |.
    \end{equation}
    Notons que la fonction \( g\) est une dérivée : c'est la dérivée de la fonction
    \begin{equation}
        G(z)=zf(z_0)+\frac{ 1 }{2}f'(z_0)(z-z_0)^2.
    \end{equation}
    Par conséquent nous avons
    \begin{equation}
        \int_{\gamma_n}g=0
    \end{equation}
    par le lemme \ref{LemtpEOmi}. Nous avons donc
    \begin{subequations}
        \begin{align}
            | \int_{\gamma_n}f |&=|\int_{\gamma_n}(f-g)|\\
            &\leq l(\gamma_n)\max\{ | f(z)-g(z) |\tq z\in T^{(n)} \}\\
            &\leq \epsilon l(\gamma_n)^2,
        \end{align}
    \end{subequations}
    et par conséquent
    \begin{equation}
        c\leq l(\gamma_n)^{-2}| \int_{\gamma_n}f |\leq \epsilon,
    \end{equation}
    ce qui signifie que \( c=0\) parce que \( \epsilon\) est arbitraire. Nous avons donc prouvé le lemme de Goursat dans le cas où le point de non holomorphie \( z_1\) est en dehors de \( T\).

    Si \( z_1\) est sur un côté, disons sur le côté \( AB\), alors nous considérons un vecteur \( v\in \eC\) tel que \( T_{\epsilon}=T+\epsilon v\) ne contienne \( z_1\) pour aucun \( \epsilon\). Le vecteur \( v=z_1-C\) fait par exemple l'affaire. En vertu du point précédent nous avons
    \begin{equation}
        \int_{\partial T_{\epsilon}}f=0
    \end{equation}
    pour tout \( \epsilon>0\). Étant donné que la fonction \( f\) est continue (y compris en \( z_1\)), l'intégrale sur \( \partial T\) est également nulle.

    Si maintenant le point \( z_1\) est à l'intérieur de \( T\) nous décomposons \( T\) en trois triangles ayant \( z_1\) comme sommet commun. Si nous considérons les orientations \( Az_1C\), \( ABz_1\) et \( BCz_1\), alors nous avons
    \begin{equation}
        \int_Tf=\int_{Az_1C}f+\int_{ABz_1}f+\int_{BCz_1}f,
    \end{equation}
    alors que par le point précédent les trois intégrales du membre de droite sont nulles.
\end{proof}

\begin{proposition}[\cite{Holomorphieus}]   \label{PrpopwQSbJg}
    Soit \( \Omega\) un ouvert étoilé et \( f\) une fonction holomorphe sur \( \Omega\) sauf éventuellement en un point \( z_1\) où \( f\) est seulement continue. Alors si \( \gamma\) est un chemin fermé dans \( \Omega\), nous avons
    \begin{equation}
        \int_{\gamma}f=0.
    \end{equation}
\end{proposition}

\begin{definition}
    Une fonction \( f\colon \Omega\to \eC\) est \defe{$\eC$-analytique}{analytique!au sens complexe} sur \( \Omega\) si pour tout \( z_0\in\Omega\), il existe une suite complexe \( (c_n)\) et \( r>0\) tels que
    \begin{equation}
        f(z)=\sum_{n=0}^{\infty} c_n(z-z_0)^n
    \end{equation}
    pour tout \( z\in B(z_0,r)\).
\end{definition}


\begin{proposition}
    Une application \( f\colon \Omega\to \eC\) est $C$-dérivable sur \( \Omega\) si et seulement si elle est différentiable et
    \begin{subequations}        \label{EqmblExI}
        \begin{numcases}{}
            \frac{ \partial u }{ \partial x }=\frac{ \partial v }{ \partial y }\\
            \frac{ \partial u }{ \partial y }=-\frac{ \partial v }{ \partial x }
        \end{numcases}
    \end{subequations}
    où \( f(x+iy)=u(x,y)+iv(x,y)\).
\end{proposition}
Les équations \eqref{EqmblExI} sont les équations de \defe{Cauchy-Riemann}{Cauchy-Riemann}.

\begin{proof}
    La différentielle de \( f\colon \eR^2\to \eR^2\) est donnée par la matrice
    \begin{equation}        \label{EQwtagsz}
        T=\begin{pmatrix}
            \partial_xu(a)    &   \partial_yu(a)    \\ 
            \partial_xv(a)    &   \partial_yv(a)    
        \end{pmatrix}.
    \end{equation}
    Cette matrice est une similitude si et seulement si les équations de Cauchy-Riemann sont satisfaites. En effet si \( 1=\begin{pmatrix}
        1    \\ 
        0    
    \end{pmatrix}\) et \( i=\begin{pmatrix}
        0    \\ 
        1    
    \end{pmatrix}\), la matrice \( T\) est une similitude (écrivons \( \alpha+i\beta\) son coefficient) si
    \begin{subequations}
        \begin{numcases}{}
            T(1)=\alpha+i\beta\\
            T(i)=-\beta+i\alpha,
        \end{numcases}
    \end{subequations}
    c'est à dire
    \begin{equation}
        T=\begin{pmatrix}
            \alpha    &   -\beta    \\ 
           \beta    &   \alpha    
        \end{pmatrix}.
    \end{equation}
    Identifier cette matrice à \eqref{EQwtagsz} fournit le résultat annoncé.
\end{proof}

\begin{proposition}
    Une fonction \( f\colon \Omega\to \eC\) est $C$-dérivable si et seulement si elle est différentiable et \( df_a\) est une similitude.
\end{proposition}


\begin{proposition}     \label{PropRZCKeO}
    Si \( f(z)=\sum_na_nz^n\) a pour rayon de convergence \( R\), alors \( f\) est $\eC$-dérivable et nous pouvons dériver terme à terme dans la boule ouverte \( B(0,R)\).
\end{proposition}

\begin{proof}
    Cela est exactement la proposition \ref{ProptzOIuG}.
\end{proof}

%---------------------------------------------------------------------------------------------------------------------------
\subsection{Théorème de Cauchy}
%---------------------------------------------------------------------------------------------------------------------------

Cette sous-section veut prouver le théorème de Cauchy. Comme d'habitude, une référence qui ne peut pas rater est \cite{Holomorphieus}.

\begin{theorem}[formule de Cauchy]    \label{ThoUHztQe}
    Soit \( \Omega\) ouvert dans \( \eC\), \( z_0\in \Omega\) et \( f\), une fonction holomorphe sur \( \Omega\). Soit \( r>0\) tel que \( B(z_0,r)\subset \Omega\). Alors pour tout \( z\in B(z_0,r)\) nous avons
    \begin{equation}    \label{EqPzUABM}
        f(z)=\frac{1}{ 2\pi i }\int_{\partial B(z_0,r)}\frac{ f(\omega) }{ \omega-z }d\omega.
    \end{equation}
\end{theorem}
\index{formule!de Cauchy}
\index{Cauchy!formule}

\begin{proof}
    Soit \( z\in B(z_0,r)\) et considérons la fonction
    \begin{equation}
        g(\omega)=\begin{cases}
            \frac{ f(\omega)-f(z) }{ \omega-z }    &   \text{si \( \omega\neq z\)}\\
            f'(z)    &    \text{si \( \omega=z\)}.
        \end{cases}
    \end{equation}
    Cette fonction est holomorphe sur \( B(z_0,r)\setminus\{ z \}\). Étant holomorphe sur \( B(z_0,r)\setminus\{ z \}\) et continue en \( z\), elle vérifie la proposition \ref{PrpopwQSbJg} et nous avons
    \begin{equation}
        \int_{\gamma}g=0
    \end{equation}
    où \( \gamma\) est le cercle de centre \( z_0\) et de rayon \( r\). Nous avons donc
    \begin{equation}
        0=\int_{\gamma}\frac{ f(\omega) }{ \omega-z }-\int_{\gamma}\frac{ f(z) }{ \omega-z },
    \end{equation}
    et ayant déjà calculé la seconde intégrale dans l'exemple \ref{ExradygL} nous en déduisons
    \begin{equation}
        \int_{\gamma}\frac{ f(\omega) }{ \omega-z }d\omega=2\pi if(z),
    \end{equation}
    ce qu'il fallait.
\end{proof}

\begin{theorem}     \label{ThomcPOdd}
    Soit \( \Omega\) ouvert dans \( \eC\) et \( f\), holomorphe sur \( \Omega\). Soient encore \( z_0\in \Omega\) et \( r_0\) tel que \( B(z_0,r_0)\subset \Omega\). Alors sur \( B(z_0,r_0)\), la fonction \( f\) s'écrit
    \begin{equation}
        f(z)=\sum_{n=0}^{\infty}a_n(z-z_0)^n.
    \end{equation}
    De plus nous avons
    \begin{equation}
        a_n=\frac{ f^{(n)}(z_0) }{ n! }=\frac{1}{ 2\pi i }\int_{\gamma}\frac{ f(\omega) }{ (\omega-z_0)^{n+1} }d\omega
    \end{equation}
    où \( \gamma=\partial B(z_0,r)\) avec \( | z-z_0 |<r<r_0\).

    En particulier \( f\) est infiniment dérivable.
\end{theorem}

\begin{proof}
    Soit \( r>0\) tel que \( | z-z_0 |<r<r_0\). La formule de Cauchy (théorème \ref{ThoUHztQe}) nous dit que
    \begin{equation}
        f(z)=\frac{1}{ 2\pi i }\int_{\gamma}\frac{ f(\omega)}{ \omega-z }d\omega
    \end{equation}
    où \( \gamma=\partial B(z_0,r)\). Nous pouvons paramétrer ce chemin par \( \omega=z_0+r e^{i\theta}\) et \( \theta\in \mathopen[ 0 , 2\pi \mathclose]\). Nous avons
    \begin{subequations}
        \begin{align}
            f(z)&=\frac{1}{ 2\pi i }\int_0^{2\pi}\frac{ f(z_0+r e^{i\theta}) }{ z_0+r e^{i\theta}-z }ri e^{i\theta}d\theta\\
            &=\frac{1}{ 2\pi }\int_0^{2\pi}\frac{ f(z_0+r e^{i\theta}) }{ 1- e^{-i\theta}(z-z_0)/r }d\theta.
        \end{align}
    \end{subequations}
    Nous pouvons développer l'intégrante en puissance de \( (z-z_0)\) en utilisant la formule \ref{EqVmuaqT}. Ici le rôle de \( x\) est tenu par
    \begin{equation}
        e^{-i\theta}(z-z_0)/r
    \end{equation}
    dont le module est bien plus petit que \( 1\), par hypothèse sur \( r\). Nous avons donc
    \begin{equation}
        f(z)=\frac{1}{ 2\pi }\int_0^{2\pi}\sum_{n=0}^{\infty}f(z_0+r e^{i\theta}) e^{-in\theta}r^{-n}(z-z_0)^nd\theta.
    \end{equation}
    L'art est maintenant de permuter la somme et l'intégrale. Pour cela nous remarquons que ce qui se trouve dans la somme est majoré en module par
    \begin{equation}        \label{EqbykTLD}
        M\left| \frac{ z-z_0 }{ r } \right|^n
    \end{equation}
    où \( M\) est le maximum de \( | f |\) sur \( \gamma\). La borne \eqref{EqbykTLD} ne dépend pas de \( \theta\); par conséquent la convergence de la somme est uniforme en \( \theta\) par le critère de Weierstrass (théorème \ref{ThoCritWeierstrass}). Le théorème \ref{ThoCciOlZ} s'applique\footnote{Étant donné que nous savions déjà que la somme était une fonction intégrable, nous sommes loin d'avoir utilisé toute la puissance du théorème.} et nous pouvons permuter la somme avec l'intégrale.

    Ce que nous trouvons est que
    \begin{equation}
        f(z)=\sum_{n=0}^{\infty}a_n(z-z_0)^n
    \end{equation}
    où
    \begin{equation}
        a_n=\frac{1}{ 2\pi }\int_0^{2\pi}f(z_0+r e^{i\theta}) e^{-in\theta}r^{-n}d\theta=\frac{1}{ 2\pi i }\int_{\gamma}\frac{ f(\omega) }{ (\omega-z_0)^{n+1} }.
    \end{equation}
    Cette formule est valable pour \( | z-z_0 |<r\). Sur cette boule, la fonction est donc une série entière Le théorème de Taylor \ref{ThoTGPtDj} nous permet donc d'affirmer que \( f\) est partout infiniment continument dérivable (parce que en chaque point on a un voisinage sur lequel c'est vrai), et d'identifier les coefficients (qui, eux, ne sont valables que localement) sous la forme
    \begin{equation}
        a_n=\frac{ f^{(n)}(z_0) }{ n! }.
    \end{equation}
\end{proof}

\begin{corollary}       \label{CorwfHtJu}
    Soit \( f\) une fonction continue sur un ouvert \( \Omega\) telle que pour toute boule \( B(a,r)\) contenue dans \( \Omega\), nous ayons
    \begin{equation}
        f(a)=\frac{1}{ 2\pi i }\int_{\partial B(a,r)}\frac{ f(\xi) }{ \xi-a }d\xi.
    \end{equation}
    Alors \( f\) est holomorphe.
\end{corollary}

\begin{proof}
    Il suffit de recopier la démonstration du théorème \ref{ThomcPOdd} pour savoir que \( f\) se développe en série de puissances et est donc en particulier dérivable.
\end{proof}

\begin{proposition}\label{PropZOkfmO}
    Une fonction continue \( f\) est holomorphe si et seulement si la \( 1\)-forme différentielle \( f(z)dz\) est localement exacte.
\end{proposition}

\begin{proof}
    Si \( f\) est holomorphe, alors nous avons vu que \( f\) était différentiable et que \( df_{z}=f(z)dz\) par la formule \ref{EqPropZOkfmO}.

    Dans le sens inverse, supposons que \( f(z)dz\) est localement exacte, et soit \( F\) telle que \( dF=f(z)dz\). Ce que nous allons faire est montrer que la dérivée de \( F\) existe et vaut \( f\). En effet, la définition de la différentielle nous dit que
    \begin{equation}
        \lim_{h\to 0} \left| \frac{ F(z+h)-F(z)-dF_z(h) }{ h } \right| =0.
    \end{equation}
    La limite vaut évidemment encore zéro si nous enlevons les modules :
    \begin{subequations}
        \begin{align}
            0&=\lim_{h\to 0} \frac{ F(z+h)-F(z)-f(z)h }{ h }\\
            &=\lim_{h\to 0} \frac{ F(z+h)-F(z) }{ h }-f(z).
        \end{align}
    \end{subequations}
    Donc \( F'=f\). Cela montre que \( F\) est dérivable et donc holomorphe. En conséquence du théorème \ref{ThoUHztQe}, \( F\) est infiniment dérivable et \( f\) l'est alors aussi. La fonction \( f\) est donc holomorphe\footnote{Dire que la dérivée d'une fonction holomorphe est holomorphe est un raisonnement classique.}.
\end{proof}

%--------------------------------------------------------------------------------------------------------------------------- 
\subsection{Lacets, indice et homotopie}
%---------------------------------------------------------------------------------------------------------------------------

\begin{definition}
    Soit \( \gamma\) un chemin fermé\footnote{Par abus de langage, nous désignerons par \( \gamma\) à la fois le chemin et son image.} dans \( \eC\). L'\defe{indice}{indice!d'une courbe dans $\eC$} de la courbe \( \gamma\) est la fonction
    \begin{equation}
        \begin{aligned}
            \Ind_{\gamma}\colon \eC\setminus \gamma&\to \eZ \\
            z&\mapsto \frac{1}{ 2\pi i }\int_{\gamma}\frac{ d\omega }{ \omega-z }. 
        \end{aligned}
    \end{equation}
    Un chemin continu et fermé (au sens \( \gamma(1)=\gamma(0)\)) est un \defe{lacet}{lacet}.
\end{definition}

\begin{definition}  \label{DefECnFJQp}
    Si \( \gamma_1\) et \( \gamma_2\) sont deux lacets en \( x_0\in X\) (un espace topologique), une \defe{équivalence d'homotopie}{équivalence!homotopie} est une application \( f\colon \mathopen[ 0 , 1 \mathclose]\times \mathopen[ 0 , 1 \mathclose]\to X\) telle que
    \begin{enumerate}
        \item
            \( f(0,t)=\gamma_1(t)\) pour tout \( t\);
        \item
            \( f(1,t)=\gamma_1(t)\) pour tout \( t\);
        \item
            pour chaque \( t\in \mathopen[ 0 , 1 \mathclose]\), l'application \( s\mapsto f(s,t)\) est continue;
        \item
            pour chaque \( s\in \mathopen[ 0 , 1 \mathclose]\), l'application \( t\mapsto f(s,t)\) est un lacet basé en \( x_0\).
    \end{enumerate}
\end{definition}

\begin{theorem}     \label{ThoDYQQXZ}
    \begin{enumerate}
        \item
            La fonction \( \Ind_{\gamma}\) est continue et prend des valeurs entières.
        \item
            La fonction indice est constante sur chaque composante connexe de \( \eC\setminus \gamma\) et est nulle sur la composante non bornée.
    \end{enumerate}
\end{theorem}
%TODO : une preuve. Si cette preuve ne demande pas vraiment d'analyse complexe, alors on peut la mettre plus haut et éventuellement remettre le théorème de Brouwer \ref{ThoLVViheK} à sa place.
Le second point est en partie la proposition \ref{PropHSjJcIr}.
\index{connexité!indice d'une courbe}

\begin{example} \label{ExradygL}
    Si \( \gamma\) est un cercle de centre \( z_0\in \eC\) et de rayon \( r\), alors 
    \begin{equation}
        \Ind_{\gamma}(z)=\begin{cases}
            2\pi i    &   \text{si \( z\in B(z_0,r)\)}\\
            0    &    \text{sinon}.
        \end{cases}
    \end{equation}
    La seconde ligne provient directement du théorème \ref{ThoDYQQXZ}. Pour la première, le cercle \( \gamma\) se paramètre par
    \begin{equation}
        \gamma(\theta)=z_0+r e^{i\theta},
    \end{equation}
    et l'intégrale vaut
    \begin{equation}
        \int_{\gamma}\frac{ d\omega }{ \omega-z_0 }=\int_0^{2\pi}\frac{1}{ r e^{i\theta} }ir e^{i\theta}d\theta=2\pi i.
    \end{equation}
    L'indice de ce chemin va évidemment jouer un rôle particulier dans la suite.
\end{example}

\begin{theorem}[Cauchy, version homotopique\cite{ADEyNiz}]
    Soit \( \Omega\) un ouvert de \( \eC\) et \( f\) une fonction holomorphe sur \( \Omega\). Si \( \gamma_1\) et \( \gamma_2\) sont deux lacets homotopes de classe \( C^1\) dans \( \Omega\), alors
    \begin{equation}
        \int_{\gamma_1}f(z)dz=\int_{\gamma_2}f(z)dz.
    \end{equation}
\end{theorem}

\begin{corollary}[\cite{ADEyNiz}]   \label{CorGZXzuZR}
    Soit \( a\in \eC\) ainsi que deux chemins \( \gamma_1\) et \( \gamma_2\) homotopes dans \( \eC\setminus\{ a \}\). Alors \( \Int(\gamma_1,a)=\Ind(\gamma_2,a)\).
\end{corollary}
Il y a aussi des choses sur l'indice dans \cite{Holomorphieus}.

%--------------------------------------------------------------------------------------------------------------------------- 
\subsection{Théorème de Brouwer en dimension \texorpdfstring{$ 2$}{2}}
%---------------------------------------------------------------------------------------------------------------------------
Pour d'autres versions du théorème de Brouwer, voir la section \ref{SecZCCmMnQ}.

\begin{theorem}[Brouwer en dimension \( 2\)\cite{KXjFWKA}]     \label{ThoLVViheK}
    Soit \( \mB\) la boule unité fermée de \( \eR^2\). Alors toute application continue de \( \mB\) dans elle-même admet un point fixe.
\end{theorem}
\index{théorème!Brouwer!dimension \( 2\)}
\index{connexité!utilisation!Brouwer}
\index{théorème!point fixe!Brouwer}

\begin{proof}
    Supposons que la fonction \( f\in C^0(\mB,\mB)\) n'admette pas de points fixes sur \( \mB=\overline{ B(0,1) }\). Pour \( x\in \mB\) nous notons \( g(x)\) l'intersection entre \( \partial \mB\) et la demi-droite allant de \( f(x)\) vers \( x\). C'est bien parce que \( f\) n'a pas de points fixes que \( g\) est bien définie.

    En reprenant le même début de la preuve de la proposition \ref{PropDRpYwv} nous savons que la fonction
    \begin{equation}
        \begin{aligned}
            g\colon \overline{ B(0,1) }&\to \partial B(0,1) \\
            x&\mapsto \lambda(x)\big( x-f(x) \big)+f(x) 
        \end{aligned}
    \end{equation}
    est continue. De plus \( g(x)=x\) sur \( \partial B(0,1)\). Nous allons montrer qu'une telle fonction\footnote{Qui est nommée \emph{rétraction} de la sphère sur elle-même.} ne peut pas exister.

    Pour \( s\in\mathopen[ 0 , 1 \mathclose]\) nous paramétrons le cercle \( \partial B(0,s)\) par
    \begin{equation}
        \begin{aligned}
            x_s\colon \mathopen[ 0 , 1 \mathclose]&\to \partial B(0,1) \\
            t&\mapsto \big( s\cos(2\pi t),s\sin(2\pi t) \big). 
        \end{aligned}
    \end{equation}
    Ensuite nous considérons les chemins
    \begin{equation}
        \begin{aligned}
            \gamma_s\colon \mathopen[ 0 , 1 \mathclose]&\to \partial B(0,s) \\
            t&\mapsto g\circ x_s. 
        \end{aligned}
    \end{equation}
    L'application \( \gamma_s\) est continue et \( \gamma_s(0)=\gamma_s(1)\). Les chemins \( \gamma_s\) sont des lacets; nous nous intéressons maintenant à l'indice au point \( 0\) de \( \gamma_0\) et \( \gamma_1\). D'une part \( \gamma_0(t)=g(0)\) (lacet constant) et \( \gamma_1(t)= e^{2i\pi t}\) (parce que \( g(x)=x\) sur le bord). Nous avons donc
    \begin{equation}
        \Ind_{\gamma_0}(0)=\frac{1}{ 2\pi i }\Ind_{\gamma_0}\frac{ d\omega }{ \omega }=\frac{1}{ 2\pi i }\int_0^1\frac{ \gamma_0'(t) }{ \gamma_0(t) }dt=0,
    \end{equation}
    alors que
    \begin{equation}
        \Ind_{\gamma_1}(0)=\frac{1}{ 2\pi i }\int_0^1\frac{ 2i\pi e^{2i\pi t} }{  e^{2i\pi t} }dt=1.
    \end{equation}
    
    Nous considérons l'homotopie 
    \begin{equation}
        \begin{aligned}
            \gamma\colon \mathopen[ 0 , 1 \mathclose]\times \mathopen[ 0 , 1 \mathclose]&\to \overline{ B(0,1) } \\
            (s,t)&\mapsto \gamma_s(t)=(g\circ x_s)(t). 
        \end{aligned}
    \end{equation}
    Nous avons \( g(0)\neq 0\) parce que \( g\) prend ses valeurs sur le bord. Vu que c'est une équivalence d'homotopie\footnote{Définition \ref{DefECnFJQp}} entre \( \gamma_1\) et \( \gamma_2\), les indices devraient être égaux par le corollaire \ref{CorGZXzuZR}.
\end{proof}

%---------------------------------------------------------------------------------------------------------------------------
\subsection{Principe des zéros isolés}
%---------------------------------------------------------------------------------------------------------------------------

\begin{theorem}[Principe des zéros isolés \cite{Holomorphieus}]     \label{ThoukDPBX}
    Soit \( f\) une fonction holomorphe et \( a\), une zéro non isolé de \( f\). Alors \( f\) est nulle sur un voisinage de \( a\).
\end{theorem}
\index{principe!zéros isolés}

\begin{proof}
    Nous écrivons \( f\) sous la forme d'une série entière autour de \( a\) :
    \begin{equation}        \label{EqgrvfVl}
        f(z)=\sum_{n=0}^{\infty}c_n(z-a)^n
    \end{equation}
    valable sur une boule \( B(a,r)\). Soit \( c_m\) le premier coefficient non nul (si il n'existe pas c'est que \( f\) est nulle sur tout \( B(a,r) \) et alors le théorème est prouvé). Nous avons alors
    \begin{equation}
        f(z)=c_m(z-a)^m\big( 1+\sum_{k=1}^{\infty}d_k(z-a)^k \big)
    \end{equation}
    avec \( d_k=c_{m-k}\). Le rayon de convergence de la série \( \sum_k d_k(z-a)^k\) est le même que celui de \eqref{EqgrvfVl} parce que la suite \( d_kr^{m+k}\) reste bornée (critère d'Abel, lemme \ref{LemmbWnFI}). Si nous posons
    \begin{equation}
        g(z)=1+\sum_{k=1}^{\infty}d_k(z-a)^k,
    \end{equation}
    alors \( g\) est une fonction continue et \( g(a)=1\). De plus 
    \begin{equation}
        f(z)=c_m(z-a)^mg(z).
    \end{equation}

    Soit une suite \( (z_n)\) de zéros de \( f\) convergent vers \( a\). Étant donné que \( g\) est continue, nous devrions avoir \( \lim_{k\to\infty}g(z_k)=g(a)=1\), mais si \( f(z_k=0)\) avec \( z_k\neq a\), alors \( g(z_k)=0\). Cela est un paradoxe qui nous permet de conclure que si la suite \( z_n\) existe bien, alors \( f\) est identiquement nulle sur un voisinage, c'est à dire que tous les \( c_n\) sont nuls.
\end{proof}

\begin{corollary}
    Soit \( f\) une fonction holomorphe sur un ouvert connexe \( \Omega\). Si \( f\) s'annule sur un un ouvert (non vide) de \( \Omega\), alors \( f\) s'annule sur tout \( \Omega\).
\end{corollary}

\begin{proof}
    soit 
    \begin{equation}
        N=\{ z\in \Omega\tq f=0\text{ sur un ouvert autour de $z$} \}.
    \end{equation}
    Le fait que \( N\) soit ouvert est évident à partir de sa définition. Nous allons montrer que \( N\) est également fermé dans \( \Omega\), et donc conclure que \( N=\Omega\). Soit \( (z_n)\) une suite dans \( N\) convergente vers \( z\in \Omega\). Étant donné que \( f(z_n)=0\) et que \( f\) est continue, nous avons
    \begin{equation}
        f(z)=\lim_{n\to \infty} f(z_n)=0,
    \end{equation}
    ce qui fait de \( z\) un zéro non isolé de \( f\). Par conséquent le principe des zéros isolés (théorème \ref{ThoukDPBX}) nous enseigne que \( f\) s'annule dans un voisinage autour de \( z\), c'est à dire que \( z\in N\). L'ensemble \( N\) est donc fermé.
\end{proof}

%---------------------------------------------------------------------------------------------------------------------------
\subsection{Prolongement}
%---------------------------------------------------------------------------------------------------------------------------

\begin{proposition}
    Soit \( \Omega\), un ouvert de \( \eC\) et \( f\colon \Omega\to \eC\) une fonction holomorphe sur \( \Omega\setminus\{ a \}\) (\( a\in \Omega\)). Nous supposons qu'il existe \( r>0\) tel que \( f\) est bornée sur \( B(a,r)\cap\Omega\). Alors \( f\) se prolonge en une fonction holomorphe sur \( \Omega\).
\end{proposition}

\begin{proof}
    Nous définissons la fonction \( g\colon \Omega\to \eC\) par
    \begin{equation}
        g(z)=\begin{cases}
            (z-a)f(z)    &   \text{si \( z\neq a\)}\\
            0    &    \text{si \( z=a\)}.
        \end{cases}
    \end{equation}
    Sur \( \Omega\setminus\{ a \}\), la fonction \( g\) est holomorphe (produit de fonctions holomorphes), et elle est continue en \( a\). Par conséquent elle est holomorphe sur \( \Omega\). Nous la développons en série entière sur une boule \( B(a,r)\) :
    \begin{equation}
        g(z)=\sum_{n=0}^{\infty}c_n(z-a)^n.
    \end{equation}
    Nous avons \( g(a)=c_0=0\). Nous posons
    \begin{equation}
        \varphi(z)=\sum_{n=0}^{\infty}c_{n+1}(z-a)^n.
    \end{equation}
    Si \( z\neq a\), alors \( \varphi(z)=f(a)\) parce que \( \varphi(z)=g(z)/(z-a)\). Mais \( \varphi\) est continue en \( a\), et donc holomorphe en \( a\).

    La fonction \( \varphi\) est par conséquent un prolongement holomorphe de \( f\) en \( a\).
\end{proof}

%---------------------------------------------------------------------------------------------------------------------------
\subsection{Théorème de Runge}
%---------------------------------------------------------------------------------------------------------------------------

Le théorème que nous allons prouver n'est en réalité qu'une partie de ce qui est usuellement appelle le théorème de Runge.
\begin{theorem}[Théorème de Runge]\index{théorème!Runge}     \label{ThoMvMCci}
    Soit \( K\), un compact de \( \eC\) tel que \( \complement K\) soit connexe. Si \( a\in \complement K\) alors la fonction 
    \begin{equation}
        \varphi_a(z)=\frac{1}{ z-a }
    \end{equation}
    est limite uniforme de polynômes sur \( K\).
\end{theorem}
\index{connexité!théorème de Runge}
\index{approximation!polynômiale}

\begin{proof}
    Nous considérons \( P(K)\), l'adhérence des polynômes sur \( K\) pour la norme uniforme (sur \( K\)). Nous devons montrer que pour tout \( a\in \complement K\), la fonction \( \varphi_a\) est dans \( P(K)\). Pour cela nous considérons l'ensemble
    \begin{equation}
        A=\{ a\in\complement K\tq \varphi_a\in P(K) \}
    \end{equation}
    et nous allons montrer qu'il est à la fois non vide, ouvert et fermé dans le connexe \( \complement K\).

    Je répète : nous allons prouver l'ouverture et la fermeture \emph{pour la topologie de \( \complement K\)}. Nous n'allons pas prouver que \( A\) est un ouvert de \( \eC\). Ce qui sera par conséquent prouvé est que \( A=\complement K\).

    \begin{subproof}
    \item[Non vide] Soit \( R=\sup_{z\in K}| z |\) et \( a\in \complement K\) tel que \( | a |>R\). Nous avons
        \begin{equation}
                \varphi_a(z)=\frac{1}{ a }\frac{1}{ \frac{ z }{ a }-1 }
                =-\frac{1}{ a }\frac{1}{ 1-\frac{ z }{ a } }
                =-\frac{1}{ a }\sum_{k=0}^{\infty}\left( \frac{ z }{ a } \right)^k
                =\sum_{k=0}^{\infty}\frac{ z^k }{ a^{k+1} }.
        \end{equation}
        Ici la convergence de la série et sa limite sont assurées par le fait que \( | z/a |<1\) par choix de \( R\) et \( a\). La suite de polynômes
        \begin{equation}
            P_n(z)=\sum_{k=0}^n\frac{ z^k }{ a^{k+1} }
        \end{equation}
        converge uniformément sur \( B(0,R)\) et en particulier sur \( K\). Donc \( P_n\to \varphi_a\).

    \item[Fermé] 
            
        Nous allons montrer que la fermeture de \( A\) (dans \( \complement K\)) est inclue dans \( A\), et donc qu'elle est égale à \( A\) et donc que \( A\) est fermé. Par le lemme \ref{LemkUYkQt}, la fermeture de \( A\) dans \( \complement K\) est l'ensemble \( \bar A\cap\complement K\) où \( \bar A\) est la fermeture de \( A\) au sens usuel.

        Bref, soit \( a\in \bar A\cap\complement K\), et montrons que \( \varphi_a\in \overline{ P(K) }\). Vu que \( P(K)\) est déjà une fermeture, nous aurons en fait \( \varphi_a\in P(K)\) et donc \( a\in A\), ce qui signifierait que \( \bar A\cap\complement A=A\) et donc que \( A\) est fermé.

        Au travail.

        Soit \( (a_n)\in A\) une suite convergente vers \( a\). Soit aussi \( d=d(a,K)\); on a \( d>0\) parce que \( K\) est compact et \( a\) est hors de \( a\) alors le complémentaire de \( K\) est ouvert. Nous choisissons en plus la suite \( a_n\) pour avoir \( | a_n-a |<\frac{ d }{2}\); au pire on prend la queue de suite. Soit \( z\in K\); nous avons
        \begin{equation}    \label{EqYHWQhI}
            | \varphi_{a_n}(z)-\varphi_a(z) |=\left| \frac{1}{ z-a_n }-\frac{1}{ z-a } \right| =  \left| \frac{ a_n-a }{ (z-a_n)(z-a) } \right|.
        \end{equation}
        Vu que \( a_n\in B(a,\frac{ d }{2})\) et que \( z\in K\) et \( d=d(a,K)\) nous avons \( | a_n-z |\geq \frac{ d }{2}\); et aussi \( | a-z |\geq \frac{ d }{2}\). Nous pouvons donc majorer \eqref{EqYHWQhI} par
        \begin{equation}
            | \varphi_{a_n}(z)-\varphi_a(z) |\leq 2\frac{ | a_n-a | }{ d^2 }.
        \end{equation}
        Donc nous avons
        \begin{equation}
            \| \varphi_a-\varphi_{a_n} \|_K\leq 2\frac{ | a_n-a | }{ d^2 }\to 0
        \end{equation}
        où la norme \( \| . \|_K\) est la norme supremum sur \( K\). Donc \( a\in \overline{ P(K) }=P(K)\) et \( A\) est fermé.

    \item[Ouvert] Vu que \( K\) est compact, il est fermé et donc \( \complement K\) est ouvert. Par conséquent, ainsi que précisé dans l'exemple \ref{ExloeyoR}, les ouverts de \( \complement K\) sont les ouverts de \( \eC\) contenus dans \( \complement K\). Afin de prouver que \( A\) est ouvert, nous prenons  \( a\in A\) et nous cherchons une boule (au sens de \( \eC\)) autour de \( a\) qui serait incluse dans \( A\).

        Soit donc \( h\in \eC\) «petit» dans un sens que nous allons préciser plus tard. Encore une fois nous posons \( d=d(a,K)\). Nous avons
        \begin{equation}        \label{EqgBSxFB}
            \varphi_{a+h}(z)=\frac{1}{ z-a-h }=\frac{1}{ z-a }\frac{1}{ 1-\frac{ h }{ z-a } }=\sum_{k=0}^{\infty}\frac{ h^k }{ (z-a)^{k+1} }.
        \end{equation}
        Déjà ici nous demandons \( h<\sup_{z\in K}| z-a |\). Puisque \( | z-a |>d\), nous avons alors
        \begin{equation}
            | \varphi_{a+h}(z) |\leq \sum_{k=0}^{\infty}\frac{ h^k }{ d^{k+1} }<\infty.
        \end{equation}
        Cela pour dire que la somme à droite de \eqref{EqgBSxFB} converge bien pourvu que \( h\) soit bien petit. Nous pouvons donc poursuivre :
        \begin{equation}    \label{EqTSSdttylSDX}
            \varphi_{a+h}(z)=\sum_{k=0}^{\infty}\frac{ h^k }{ (z-a)^{k+1} }=\sum_{k=0}^{\infty}h^k\varphi_a(z)^{k+1}.
        \end{equation}
        Nous montrons maintenant que la convergence de la somme \eqref{EqTSSdttylSDX} est en réalité uniforme en \( z\). En effet
        \begin{subequations}
            \begin{align}
                \big| \varphi_{a+h}(z)-\sum_{k=0}^Nh^k\varphi_a(z)^{k+1} \big|&=\big| \sum_{k=N+1}^{\infty}h^k\varphi_a(z)^{k+1} \big|\\
                &\leq\sum_{k=N+1}^{\infty}| h |^k| \varphi_a(z) |^{k+1}.
            \end{align}
        \end{subequations}
        Étant donné que \( \varphi_a\) est continue sur le compact \( K\), elle y est majorée en module; on peut même être plus précis :
        \begin{equation}
            |\varphi_a(z)|=\frac{1}{ | z-a | }\leq \frac{1}{ d }.
        \end{equation}
        Nous pouvons donc écrire
        \begin{equation}
            \big| \varphi_{a+h}(z)-\sum_{k=0}^Nh^k\varphi_a(z)^{k+1} \big|\leq\frac{1}{ d }\sum_{k=N+1}^{\infty}\left| \frac{ h }{ d } \right|^k.
        \end{equation}
        Étant donné que la somme \( \sum_{k=0}^{\infty}| h/d |^k\) converge, la limite \( N\to \infty\) est nulle et nous avons
        \begin{equation}
            \lim_{N\to \infty} \| \varphi_{a+h}-\sum_{k=0}^Nh^k\varphi_a^{k+1} \|_K=0.
        \end{equation}
        Pour avoir \( \varphi_{a+h}\in P(K)\), il faut encore savoir si les fonctions \( \varphi_a^{k}\) sont dans \( P(K)\) pour tout \( k\). Dans ce cas pour chaque \( N\) la somme sera encore dans \( P(K)\) et \( \varphi_{a+h}\) sera limite uniforme d'éléments de \( P(K)\).

        Par hypothèse, \( \varphi_a\in P(K)\); soit \( P_n\) une suite de polynômes qui converge uniformément vers \( \varphi_a\). Nous allons montrer qu'alors la suite de polynômes \( P_n^k\) converge uniformément vers \( \varphi_a^k\). Soit \( n\) tel que \( \| P_n-\varphi_a \|_{K}\leq \epsilon\) et utilisons le produit remarquable\index{produit remarquable}
        \begin{equation}
            a^k-b^k=(a-b)\sum_{i=0}^{k-1}a^ib^{k-1-i}
        \end{equation}
        pour obtenir
        \begin{equation}
            | P_n(z)^k-\varphi_a(z)^k |\leq | P_n(z)-\varphi_a(z) |\sum_{i=0}^{k-1}| P_n(z)^i\varphi_a(z)^{k-1-i} |.
        \end{equation}
        Vu que \( P_n\) et \( \varphi_a\) sont continues sur le compact \( K\), on peut majorer la somme par une constante \( M\), et il restera
        \begin{equation}
            | P_n(z)^k-\varphi_a(z)^k |\leq M | P_n(z)-\varphi_a(z) |,
        \end{equation}
        ou encore
        \begin{equation}
            \| P_n^k-\varphi_a^k \|\leq M\epsilon.
        \end{equation}
        Cela prouve que \( \varphi_a^{k}\in P(K)\) et donc que \( \varphi_{a+h}\) est limite uniforme (sur \( K\)) d'éléments de \( P(K)\) et donc fait partie de \( P(K)\) lui aussi.

        Ceci achève de prouver que \( A\) est ouvert dans \( \complement K\).
    \item[Conclusion]

        L'ensemble \( A\) est non vide, ouvert et fermé dans \( \complement K\), donc il est égal à \( \complement K\). Le théorème est ainsi démontré.
    \end{subproof}
\end{proof}

%+++++++++++++++++++++++++++++++++++++++++++++++++++++++++++++++++++++++++++++++++++++++++++++++++++++++++++++++++++++++++++
\section{Intégrales de fonctions holomorphes}
%+++++++++++++++++++++++++++++++++++++++++++++++++++++++++++++++++++++++++++++++++++++++++++++++++++++++++++++++++++++++++++

\begin{lemma}       \label{LemNAnweA}
    Soit \( f\) une fonction holomorphe sur \( B(z_0,r_0)\). Pour tout \( z\in B(z_0,r)\) (avec \( r<r_0\)) nous avons
    \begin{equation}
        | f'(z) |\leq \frac{ r }{ \big( r-| z-z_0 | \big)^2 }\max\big\{ f(z_0+r e^{i\theta}) \big\}_{\theta\in \eR}.
    \end{equation}
\end{lemma}

\begin{proof}
    Par translation nous pouvons supposer que \( z_0=0\). Étant donné que \( f\) est holomorphe, elle admet un développement en séries entières
    \begin{equation}
        f(z)=\sum_{n=0}^{\infty}a_nz^n
    \end{equation}
    et nous notons \( M=\max\{ f(z)\tq z\in \overline{ B(0,r) } \}\). Nous avons\cite{Holomorphieus} \( r^n| a_n |\leq M\). Par conséquent
    \begin{subequations}
        \begin{align}
            | f'(z) |&=\left| \sum_{n=1}^{\infty}na_nz^{n-1} \right| \\
            &\leq\frac{1}{ r }\sum r^n| a_n |n\left( \frac{ | z | }{ r } \right)^{n-1}\\
            &<\frac{ M }{ r }\sum n\left( \frac{ | z | }{ r } \right)^{n-1}
        \end{align}
    \end{subequations}
    À ce point nous devons utiliser la série de l'exemple \ref{ExGxzLlP}. Nous avons alors
    \begin{equation}
        | f'(z) |\leq \frac{ M }{ r }\frac{ 1 }{ \left( 1-\frac{ | z | }{ r } \right)^2 }=\frac{ Mr }{ (r-| z |)^2 }.
    \end{equation}
\end{proof}

\begin{theorem}[Holomorphie sous l'intégrale\cite{Holomorphieus}] \label{ThopCLOVN}
    Soit un espace mesuré \( (\Omega,\mu)\), un ouvert \( A\) dans \( \eC\) et une fonction \( f\colon A\times \Omega\to \eC\). Nous voulons étudier la fonction
    \begin{equation}
        F(z)=\int_{\Omega}f(z,\omega)d\mu(\omega)
    \end{equation}
    pour tout \( z\in A\). Nous supposons que
    \begin{enumerate}
        \item
            la fonction \( f(.,\omega)\) est holomorphe sur \( A\) pour chaque \( \omega\).
        \item
            La fonction \( f(z,.)\) est mesurable sur \( (\Omega,\mu)\).
        \item
            Pour tout compact \( K\subset A\), il existe une fonction \( g_K\colon \Omega\to \eR\) telle que \( | f(z,\omega) |\leq g_K(\omega)\) et telle que
            \begin{equation}
                \int_{\Omega}g_K(\omega)d\mu(\omega)
            \end{equation}
            existe.
    \end{enumerate}
    Alors la fonction \( F\) est holomorphe et
    \begin{equation}
        F'(z)=\int_{\Omega}\frac{ \partial f }{ \partial z }(z,\omega)d\mu(\omega).
    \end{equation}
\end{theorem}

\begin{proof}
    Soient \( z_0\in A\) et \( r>0\) tels que \( K=\overline{ B(z_0,r) }\subset A\). Pour chaque \( \omega\in \Omega\) nous considérons la fonction
    \begin{equation}
        \begin{aligned}
            f_{\omega}\colon \overline{ B(z_0,r) }&\to \eC \\
            z&\mapsto f(z,\omega). 
        \end{aligned}
    \end{equation}
    Étant donné que \( \overline{ B(z_0,r) }\) est compacte, la fonction \( | f_{\omega} |\) est majorée par un nombre que nous notons \( f_K(\omega)\) qui est indépendant de \( z\) (pour autant que $z\in K$). Nous désignons par \( S(z_0,r)\) la frontière de la boule \( B(z_0,r)\). Étant donné que la majoration est valable sur \( \overline{ B(z_0,r) }\), nous avons en particulier
    \begin{equation}
        | f_{\omega}(z) |\leq f_K(\omega)
    \end{equation}
    pour tout \( z\in S\). En utilisant la lemme \ref{LemNAnweA} nous avons
    \begin{subequations}
        \begin{align}
            | f'_{\omega}(z) |&\leq \frac{ r }{ (r-| z-z_0 |)^2 }\max\{ f(z_0+r e^{i\theta}) \}_{\theta\in \eR}\\
            &\leq \frac{ rf_K(\omega) }{ (r-| z-z_0 |)^2 }.
        \end{align}
    \end{subequations}
    Cette majoration est valable pour tout \( z\in B(z_0,r)\). Si nous supposons de plus que \( z\in B(z_0,r/2)\)  nous avons
    \begin{equation}
        | f'(z) |\leq \frac{ rf_K(\omega) }{ \left( r-\frac{ r }{2} \right)^2 }=\frac{ 4 }{ r }f_K(\omega).
    \end{equation}
    Étant donné que la boule \( B(z_0,r/2)\) est convexe, la fonction \( f_{\omega}\) est Lipschitz et pour tout \( h\in \eC\) tel que \( | h |<r/2\) nous avons
    \begin{equation}
        \left| \frac{ f_{\omega}(z_0+h)-f_{\omega}(z_0) }{ h } \right| \leq \frac{ 4f_K(\omega) }{ r }.
    \end{equation}
    Soit maintenant une suite \( (h_n)\) qui converge vers \( 0\) dans \( \eC\). Nous considérons la suite de fonctions correspondantes
    \begin{equation}
        g_n(\omega)=\frac{ f(z_0+h_n,\omega)-f(z_0,\omega) }{ h_n }.
    \end{equation}
    Cette suite de fonction vérifie la convergence ponctuelle
    \begin{equation}
        g_n(\omega)\to\frac{ \partial f }{ \partial z }(z_0,\omega).
    \end{equation}
    De plus \( g_n\) est une fonction (de \( \omega\)) dominée par \( \frac{ 4f_K }{ r }\) qui est intégrable. Par conséquent le théorème de la convergence dominée nous indique que
    \begin{equation}
        \int_{\Omega}g_n(\omega)d\mu(\omega)\to \int_{\Omega}\frac{ \partial f }{ \partial z }(z_0,\omega)d\mu(\omega),
    \end{equation}
    tandis que
    \begin{equation}
        F'(z)=\lim_{n\to \infty} \frac{ F(z_0+h_n)-F(z_0) }{ h_n }=\lim_{n\to \infty} \int_{\Omega}g_N(\omega)d\mu(\omega).
    \end{equation}
\end{proof}

\begin{corollary}       \label{CorNxTjEj}
    Si \( f\) est une fonction holomorphe sur l'ouvert \( \Omega\) contenant la fermeture de la boule \( B(z_0,r)\), alors pour tout \( z\) dans \( B(z_0,\rho)\) (\( \rho<r\)) les dérivées de \( f\) s'expriment pas la formule suivante :
    \begin{equation}
        f^{(k)}(z)=\frac{1}{ 2\pi i }\int_{\partial B(z_0,r)}\frac{ f(\omega) }{ (\omega-z)^{k+1} }d\omega.
    \end{equation}
\end{corollary}
\index{compacité}

\begin{proof}
    Nous appliquons le théorème \ref{ThopCLOVN} à la fonction
    \begin{equation}
        g(z,\omega)=\frac{ f(\omega) }{ \omega-z }
    \end{equation}
    avec \( \Omega=\partial B(z_0,r)\) et \( A=B(z_0,\rho)\) avec \( \rho<r\). Étant donné que \( f\) est holomorphe, elle est continue et donc bornée sur tout compact \( K\subset A\) par une constante \( M\) (qui dépend du compact choisi).  D'autre part, nous avons toujours \( | \omega-z |>r-\rho\) et donc
    \begin{equation}
        | g(z,\omega) |\leq \frac{ M }{ r-\rho }.
    \end{equation}
    La fonction constante \( g_K=\frac{ M }{ r-\rho }\) est évidemment intégrable. Le théorème conclu que \( f\) est holomorphe (cela, nous le savions déjà\footnote{et cela fournit une preuve alternative à la réciproque du théorème de Cauchy : une fonction continue qui vérifie la formule de Cauchy est holomorphe.}), et
    \begin{equation}
        f'(z)=\frac{1}{ 2i\pi }\int_{\partial B}\frac{ f(\omega) }{ (\omega-z)^2 }d\omega.
    \end{equation}
    Un peu de récurrence montre maintenant que
    \begin{equation}
        f^{(k)}(z)=\frac{1}{ 2i\pi }\int_{\partial B(z_0,r)}\frac{ f(\omega) }{ (\omega-z)^{k+1} }d\omega.
    \end{equation}
\end{proof}

\begin{definition}
    Une \defe{mesure de Radon}{mesure!de Radon} sur un compact \(  K\) de \( \eC\) est une forme linéaire continue sur \( C(K)\). Si \( \mu\) est une mesure de Radon, on définit la \defe{transformée de Cauchy}{transformée!de Cauchy} de \( \mu\) par 
    \begin{equation}
        \begin{aligned}
            \hat \mu\colon \eC\setminus K&\to \eC \\
            z&\mapsto -\frac{1}{ \pi }\mu\left( \frac{1}{ \xi-z } \right). 
        \end{aligned}
    \end{equation}
\end{definition}

\begin{theorem}     \label{ThoJVNTzn}
    Si \( \mu\) est une mesure de Radon sur \( K\) alors \( \hat \mu\) est infiniment \( \eC\)-dérivable sur \( \Omega=\eC\setminus K\) et nous avons
    \begin{equation}
        \hat\mu^{(n)}(z)=-\frac{ n! }{ \pi }\mu\left( \frac{1}{ (\xi-z)^{n+1} } \right).
    \end{equation}
\end{theorem}

\begin{lemma}
    Si \( f\) est holomorphe sur \( \Omega\) et si \( B\) est une boule fermée dans \( \Omega\) alors pour tout \( z\in \Int(B)\) nous avons
    \begin{equation}
        f^{(n)}(z)=\frac{ n! }{ 2i\pi }\int_{\partial B}\frac{ f(\xi) }{ (\xi-z)^{n+1} }d\xi.
    \end{equation}
\end{lemma}

\begin{proof}
    Appliquer le théorème \ref{ThoJVNTzn} à la mesure de Radon
    \begin{equation}
        \mu(\phi)=\int_{\partial B}\phi(\xi)d\xi.
    \end{equation}
\end{proof}

\begin{lemma}
    Si \( f\) est holomorphe sur \( \Omega\) et si \( B\) est une boule fermée dans \( \Omega\) alors pour tout \( z\) dans l'intérieur de \( B\) nous avons
    \begin{equation}
        f^{(n)}(z)=\frac{ n! }{ 2i\pi }\int_{\partial B}\frac{ f(\xi) }{ (\xi-z)^{n+1} }d\xi.
    \end{equation}
\end{lemma}

\begin{theorem}
    Si \( f\) est une fonction holomorphe sur le disque ouvert \( B(z_0,R)\) alors
    \begin{equation}
        f(z)=\sum_{n=0}^{\infty}\frac{ f^{(n)}(z_0) }{ n! }(z-z_0)^n
    \end{equation}
    et cette série converge uniformément sur tout compact.
\end{theorem}

\begin{proof}
    Sans perte de généralité nous supposons que \( z_0=0\). La formule de Cauchy fournit
    \begin{equation}
        f(z)=\frac{1}{ 2\pi i }\int_{\partial B}\frac{ f(\xi) }{ \xi-z }d\xi=\frac{1}{ 2\pi i }\int_{\partial B}\frac{ f(\xi) }{ 1-(z/\xi) }\frac{ d\xi }{ \xi }.
    \end{equation}
    Nous utilisons la série géométrique
    \begin{equation}
        \frac{1}{ 1-(z/\xi) }=\sum_{n=0}^{\infty}\left( \frac{ z }{ \xi } \right)^n,
    \end{equation}
    nous avons
    \begin{subequations}        \label{EqXSgZGw}
        \begin{align}
            f(z)&=\frac{1}{ 2\pi i }\sum_{n=0}^{\infty}\int_{\partial B}\frac{ z^nf(\xi) }{ \xi^{n+1} }\\
            &=\sum_{n=0}^{\infty}\left( \frac{1}{ 2\pi i }\int_{\partial B}\frac{ f(\xi) }{ \xi^{n+1} } \right)z^n.
        \end{align}
    \end{subequations}
    Nous devons maintenant montrer que ce qui se trouve dans la grande parenthèse vaut \( f^{(n)}(0)/n!\). Nous utilisons le théorème de Radon \ref{ThoJVNTzn} à la mesure
    \begin{equation}
        \mu(\phi)=\int_{\partial B}\phi(\xi)d\xi.
    \end{equation}
    La transformée de Cauchy est
    \begin{equation}        \label{EqTzkmeL}
        \hat \mu(z)=-\frac{1}{ \pi }\mu\left( \frac{1}{ \xi-z } \right)=-\frac{1}{ \pi }\int_{\partial B}\frac{1}{ \xi-z }d\xi,
    \end{equation}
    et le théorème assure que
    \begin{equation}
        \hat\mu^{(n)}(z)=-\frac{ n! }{ \pi }\mu\left( \frac{1}{ (\xi-z)^{n+1} } \right)=-\frac{ n! }{ \pi }\int_{\partial B}\frac{ 1 }{ (\xi-z)^{n+1} }d\xi.
    \end{equation}
    En comparant la formule \eqref{EqTzkmeL} avec la formule de Cauchy nous voyons que \( \hat\mu(z)=-2i f(z)\). Par conséquent
    \begin{equation}
        f^{(n)}(z)=-\frac{1}{ 2i }\hat\mu^{(n)}(z)=\frac{ n! }{ 2\pi i }\int_{\partial B}\frac{1}{ (\xi-z)^{n+1} }d\xi,
    \end{equation}
    et
    \begin{equation}
        f^{(n)}(0)=\frac{ n! }{ 2\pi i }\int_{\partial B}\frac{1}{ \xi^{n+1} }d\xi.
    \end{equation}
\end{proof}
% TODO : justifier la permutation entre la somme et l'intégrale.

\begin{proposition}[Morera \cite{NEBgfg}]   \label{ThoRckxes}
    Soit \( \Omega\) ouvert dans \( \eC\) et \( f\) continue. Si
    \begin{equation}
        \int_{\partial T}f=0
    \end{equation}
    pour tout triangle (plein) \( T\) contenu dans \( \Omega\), alors \( f\) est holomorphe sur \( \Omega\).
\end{proposition}

\begin{proof}
    Il est suffisant de prouver que \( f\) est holomorphe sur toute boule ouverte \( B(a,r)\) inclue dans \( \Omega\). Nous posons, pour tout \( z\in B(a,r)\),
    \begin{equation}
        F(z)=\int_{[p,z]}f,
    \end{equation}
    et nous considérons le chemin triangulaire \( a\to z\to z+h\to a\) où \( h\in \eC\) est choisit assez petit pour que \( z+h\in B(a,r)\). L'intégrale sur le triangle étant nulle, nous avons
    \begin{equation}
        0=\int_{a\to z}f+\int_{z\to z+h}f+\int_{z+h\to a}f,
    \end{equation}
    c'est à dire
    \begin{equation}
        F(z+h)-F(z)=\int_{z\to z+h}f.
    \end{equation}
    En paramétrant le chemin par \( z+th\) avec \( t\in\mathopen[ 0 , 1 \mathclose]\), et en tenant compte de la remarque \ref{RemiqswPd},
    \begin{subequations}
        \begin{align}
            F'(z)&=\lim_{h\to 0} \frac{ F(z+h)-F(z) }{ h }\\
            &=\lim_{h\to 0} \frac{1}{ h }\int_0^1f(z+th)hdt,
        \end{align}
    \end{subequations}
    ce qui prouve que \( F\) est dérivable et \( F'=f\). Par définition (\ref{DefMMpjJZ}), \( F\) est holomorphe, et donc \( C^{\infty}\) par le théorème \ref{ThomcPOdd}. Du coup \( f\) est également \(  C^{\infty}\) et donc en particulier holomorphe.
\end{proof}

%+++++++++++++++++++++++++++++++++++++++++++++++++++++++++++++++++++++++++++++++++++++++++++++++++++++++++++++++++++++++++++
\section{Conditions équivalentes à l'holomorphie}
%+++++++++++++++++++++++++++++++++++++++++++++++++++++++++++++++++++++++++++++++++++++++++++++++++++++++++++++++++++++++++++

Nous nous proposons de lister les conditions que nous avons vues être équivalentes à l'holomorphie.

\begin{theorem}
    Soit \( \Omega\) un ouvert de \( \eC\) et \( f\colon \Omega\to \eC\) une fonction continue. Les conditions suivantes sont équivalentes.
    \begin{enumerate}
        \item   \label{ItemOtPcTb}
            \( f\) est holomorphe.
        \item   \label{ItemHWRnxx}
            Pour tout triangle (plein) \( T\) contenu dans \( \Omega\), \( \int_Tf=0\).
        \item   \label{ItempBBPVv}
            \( f\) est dérivable.
        \item   \label{ItemmLhzbB}
            \( f\) est \(  C^{\infty}\)
        \item   \label{ItemCCrSrLj}
            \( \frac{ \partial f }{ \partial \bar z }=0\)
        \item   \label{ItemEvxRSn}
            La \( 1\)-forme différentielle \( f(z)dz\) est localement exacte.
        \item   \label{ItemVSCHtY}
            Pour toute boule \( B(a,r)\) contenue dans \( \Omega\) nous avons
            \begin{equation}
                f(a)=\frac{1}{ 2\pi i }\int_{\partial B(a,r)}\frac{ f(z) }{ z-a }dz.
            \end{equation}
    \end{enumerate}
\end{theorem}
% TODO : il faudrait rajouter les équations de Cauchy-Riemann.

\begin{proof}
    \ref{ItemOtPcTb} implique \ref{ItemHWRnxx} est le lemme de Goursat \ref{LemwbwbUR}. \ref{ItemHWRnxx} implique \ref{ItemOtPcTb} est le théorème de Morera \ref{ThoRckxes}.

    \ref{ItempBBPVv} est la définition de l'holomorphie.

    \ref{ItemmLhzbB} implique \ref{ItemOtPcTb} est un a fortiori sur la définition. \ref{ItemOtPcTb} implique \ref{ItemmLhzbB} est contenu dans le théorème de développement en série entière \ref{ThoUHztQe}.

    L'équivalence entre \ref{ItemCCrSrLj} et l'holomorphie est le théorème \ref{ThokwIQwg}.

    L'équivalence entre \ref{ItemEvxRSn} et \ref{ItemOtPcTb} est la proposition \ref{PropZOkfmO}.

    L'équivalence entre \ref{ItemOtPcTb} et \ref{ItemVSCHtY} est d'une part le théorème \ref{ThomcPOdd} et d'autre part le corollaire \ref{CorwfHtJu}.
\end{proof}
