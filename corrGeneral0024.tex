% This is part of the Exercices et corrigés de mathématique générale.
% Copyright (C) 2009
%   Laurent Claessens
% See the file fdl-1.3.txt for copying conditions.
\begin{corrige}{0024}

Trouver les intersections entre les deux courbes est aisé : il suffit de résoudre le petit système
\begin{subequations}
\begin{numcases}{}
	x^2+y^2=4\\
	x^2+y^2=4x,
\end{numcases}
\end{subequations}
dont on déduit immédiatement que $4=4x$, et donc que $x=1$. Ensuite, $y=\pm\sqrt{3}$. Les deux courbes s'intersectent donc en $(1,\pm\sqrt{3})$. Nous allons calculer la surface au-dessus de l'axe, et puis multiplier par deux. Avant que la courbe $x^2+y^2=4x$ ne quitte le cercle, c'est à dire entre $0$ et $1$, la surface est donné par cette courbe. Après, entre $1$ et $2$, elle est donné par le cercle lui-même. Un dessin est donné à la figure \ref{LabelFigCercleRacine}.
\newcommand{\CaptionFigCercleRacine}{L'aire à calculer pour l'exercice \ref{exo0024}.}
\input{Fig_CercleRacine.pstricks}

Nous décomposons la surface à calculer en plusieurs intégrales à effectuer :
\begin{equation}
	\begin{aligned}[]
		S_1&=\int_0^1\sqrt{4x-x^2}dx,&S_2&=\int_1^2\sqrt{4-x^2}dx,
	\end{aligned}
\end{equation}
et la surface totale à faire vaut
\begin{equation}
	S=2(S_1+S_2).
\end{equation}
L'intégrale $S_2$ se règle en posant $u=x/2$, ce qui amène à $\sqrt{4-4u^2}=2\sqrt{1-u^2}$. L'intégrale de cette fonction se calcule en posant $u=\cos(v)$, et en utilisant le fait que $\sqrt{1-\cos^2(v)}=\sin(v)$. Ce que nous avons est donc
\begin{equation}
	\int \sqrt{1-u^2}du=\int\sqrt{1-\cos^2(v)}\big(-\sin(v)\big)dv=-\int\sin^2(v).
\end{equation}
La dernière intégrale s'effectue en utilisant la formule de trigonométrie $2\sin^2\frac{ x }{ 2 }=1-\cos(x)$.


L'intégrale $S_1$ est plus problématique. Il faut faire apparaître quelque chose de la forme $u^2+B$. Cherchons donc $A$ et $B$ tels que
\begin{equation}
	4x-x^2=-(x+A)^2+B.
\end{equation}
La solution est $A=-2$ et $B=4$, de telle façon à ce que la première intégrale soit
\begin{equation}
	S_1=\int_0^1\sqrt{ -(x-2)^2-4}=\int_{-2}^{-1}\sqrt{-u^2+4}=S_2
\end{equation}
en posant le changement de variable $u=x-2$.

Note : en remarquant que la seconde courbe est en réalité aussi un cercle de même rayon, centré en $(0,2)$, on peut considérablement simplifier le calcul parce que les surfaces rouges et cyan sont en réalité les mêmes. Cela se voit dans le calcul : au final la difficulté est levée en faisant un changement de variable $u=x-2$ (ce qui est une translation), et on retombe sur la même intégrale que la simple.

\end{corrige}
