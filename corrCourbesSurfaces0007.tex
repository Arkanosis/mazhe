\begin{corrige}{CourbesSurfaces0007}

	\begin{enumerate}
		\item
			Le théorème \ref{ThoLongueurIntegrale} nous renseigne qu'un arc paramétré compact $C^1$ est automatiquement rectifiable. Dans notre cas, la fonction $\gamma$ est effectivement de classe $C^1$ parce que ses composantes sont des produit de $r(\theta)$ et d'une fonction trigonométrique.
		\item
			En suivant la formule \eqref{EqLongGammalInt} du même théorème \ref{ThoLongueurIntegrale}, nous avons $l(\gamma)=\int_{\theta_1}^{\theta_2}\| \gamma'(t) \|dt$. Il reste simplement à calculer la norme :
			\begin{equation}
				\gamma'(t)=\big( r'(t)\cos(t)-r(t)\sin(t),r'(t)\sin(t)+r(t)\cos(t) \big).
			\end{equation}
			Nous trouvons alors
			\begin{equation}
				\| \gamma'(t) \|=\sqrt{ r(t)^2+r'(t)^2 },
			\end{equation}
			et par conséquent la formule annoncée.
		\item
			En utilisant la formule, nous devons calculer l'intégrale
			\begin{equation}
				\begin{aligned}[]
					l&=\int_0^{\pi}\sqrt{\cos^2(\theta)+4\cos^2(\theta)\sin^2(\theta)}d\theta\\
					&=2\int_0^{\pi/2}\cos(\theta)\sqrt{1+4\sin^2(\theta)}d\theta\\
					&=2\int_0^1\sqrt{1+3u^2}du\\
                    &=2\Big[ \frac{ 1 }{2}\sqrt(4x^2 + 1)x + \frac{1}{ 4 }\arcsinh(2x)\Big]_0^1\\
				\end{aligned}
			\end{equation}
			où nous avons utilisé le changement de variable $u=\sin(\theta)$. La dernière intégrale se calcule en utilisant la formule \eqref{EqTrucsIntsqrtAplusu}. Le résultat est
			\begin{equation}
                l=\sqrt{5}+\frac{ \arcsinh(2) }{2}.
			\end{equation}
		\item
			En utilisant la même formule \eqref{EqFormDemExotLpola}, nous tombons sur l'intégrale
			\begin{equation}
				l=\sqrt{2}\int_{-\pi}^{\pi}\sqrt{1-\cos(\theta)}d\theta.
			\end{equation}
			Nous utilisons la formule $1-\cos(\theta)=2\sin^2\frac{ \theta }{2}$ et nous effectuons le changement de variables $u=\theta/2$ :
			\begin{equation}
				l=4\int_{-\pi/2}^{\pi/2}| \sin(u) |du.
			\end{equation}
			Attention à ne pas oublier la valeur absolue : sans elle l'intégrale est nulle, ce qui serait absurde pour une longueur. Au final nous avons $l=8$. 
			
	\end{enumerate}

\end{corrige}
