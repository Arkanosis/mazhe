%+++++++++++++++++++++++++++++++++++++++++++++++++++++++++++++++++++++++++++++++++++++++++++++++++++++++++++++++++++++++++++
					\section{Modules over von~Neumann algebras}
%+++++++++++++++++++++++++++++++++++++++++++++++++++++++++++++++++++++++++++++++++++++++++++++++++++++++++++++++++++++++++++
\label{SecOverModVNalgDim}

Let $X$ be a topological space and denote by $\pi$ the group $\pi_1(X)$. We denote by $\tilde X$ the universal covering of $X$, and the group $\pi$ acts on $\tilde X$ by the monodromy action defined in subsection \ref{sssMonodromyact}. We denote by $C_p(X)$ the space of $p$-chains over $X$ and $C_p(\tilde X)$ the one of $\tilde X$ from which we build the complexes
\[ 
  \xymatrix{%
   \ldots \ar[r]^b	&	C_p(X)\ar[r]^{b}	&C_{p-1}(X)\ar[r]^{b}	&\ldots	
}
\]
and
\[ 
  \xymatrix{%
   \ldots \ar[r]^b	&	C_p(\tilde X)\ar[r]^{b}	&C_{p-1}(\tilde X)\ar[r]^{b}	&\ldots	
}
\]
with $b\circ b=0$. The chain space $C_p(\tilde X)$ is not only a group, but a module over $\eZ[\pi]$.

We are now going to look at the tensor product von~Neumann algebra
\begin{equation}
	M_p(\tilde X)=M(\pi)\otimes_{\eZ[\pi]}C_p(\tilde X).
\end{equation}
In the right hand side, $M(\pi)$ is the von~Neumann algebra described in subsection \ref{sssOnePartCaseMG}, which is generated by the operators $\mU_g\in\oB\big( l^2(\pi) \big)$ defined by $\mU_g(\varphi)(g)=\varphi(h^{-1}g)$, and the tensor product is the one defined in equation \eqref{EqdefAtensRB}.

We can then look at the complex of $M(\pi)$-modules
\[ 
  \xymatrix{%
   \ldots \ar[r]^b	&	M_{p+1}(\tilde X)\ar[r]^{b}	&M_p(\tilde X) \ar[r]^{b}	&M_{p-1}(\tilde X)\ar[r]^{b}	&\ldots	
}
\]
and look at the corresponding Betty numbers which are integers numbers associated with any topological space.


%---------------------------------------------------------------------------------------------------------------------------
					\subsection{Modular conjugation and factor of type \texorpdfstring{$II_1$}{II1}}
%---------------------------------------------------------------------------------------------------------------------------

The results presented here partially come from \cite{Wassermann}, section 9 and \cite{JonesSunder}, proposition 2.2.6.

A \defe{module}{module!over a von~Neumann algebra} over a von~Neumann algebra $M$ is an Hilbert space\footnote{Hilbert space are generally denoted by $\hH$, while modules are denoted by $\modE$ or $\modF$. Here, the notation will depend on what aspect we are focusing on.} $\modE$ for which there exists a weakly continuous map $\pi_r\colon M\to \oB(\modE)$ such that $\pi_r(T)=\big( \pi_r(T) \big)^*$ and $\pi_r(ST)=\pi_r(S)\pi_r(T)$. If $\modE$ and $\modF$ are two modules over $M$, a map $a\colon \modE\to \modF$ is \defe{linear}{linear map between modules} if 
\begin{equation}
	a(S\xi)=Sa(\xi)
\end{equation}
for every $S\in M$ and $\xi\in\modE$. We denote by\footnote{This is what \cite{JonesSunder} denote by $_M\mathcal{L}(\hH)$} $\oL_M(\modE,\modF)$\nomenclature[A]{$\oL_M(\modE,\modF)$}{Space of linear maps between the modules $\modE$ and $\modF$} the set of $M$-linear maps from $\modE$ to $\modF$. When $\modE=\modF$, we write $\oL_M(\modE)$. By definition of linearity, we have $\oL_M(\modE)=\pi(M)'$. The same notion exists for linearity at right, and in the general case, we denote by $\oL_{M,N}(\modE,\modF)$, the set of $M$-left-linear and $N$-right-linear maps from $\modE$ to $\modF$.



 We say that the modules $\modE$ and $\modF$ are \defe{isomorphic}{isomorphism!of modules} if there exists an unitary $M$-linear map between them.

When $M$ is a type $II_1$ factor, we denote $\hH_1=L^2(M)$ and $\hH_{\infty}=\hH_1\otimes l^2$.
\begin{proposition}			\label{PropTypeIIProjhHinfty}\index{factor!of type $II_1$}
Let $M$ be a type $II_1$-factor and $\modE$ be a separable $M$-module. There exists a projection $P\in \eM_{\infty}(M)$ such that $\modE\simeq \hH_{\infty}P$. All such projections are Murray-von Neumann equivalent.
\end{proposition}
\begin{proof}
No proof, explanations can be found in \cite{JonesSunder}, theorem 2.2.2.
\end{proof}

When $\modE$ is a separable $M$-module ($M$ is a type $II_1$ factor), we define
\begin{equation}
	\dim_M\modE=\tr p
\end{equation}
where $p\in\oP\big( \eM_{\infty}(M) \big)$ is the projection such that $\big( L^2(M)\otimes l^2 \big)p\simeq \modE$ and whose existence is given by proposition \ref{PropTypeIIProjhHinfty}.

Let $M$ be a factor of type $II_1$ with the unique trace $\tau$ given by proposition \ref{PropExistenceTrace}. We denote by $L^2(M,\tau)$ the Hilbert space of its GNS representation, and $\pi_{\tau}$ the representation. Since $\tau$ is unique, we will simply denote them by $L^2(M)$ and $\pi$. We denote by $\Omega\in L^2(M)$ a cyclic vector. If $\pi(T)=0$, we have
\[ 
	\tau(T^*T)=\| \pi(T)\Omega \|^2=0,
\]
so that $T=0$ and the representation is faithful. We can thus identify $T\in M$ with $\pi(T)\in \oB\big( L^2(M) \big)$ and we have $M\subset\oB\big( L^2(M) \big)$. The basics properties of the GNS construction say also that $\overline{ M\Omega }=L^2(M)$ and $\tau(T)=\langle T\Omega, \Omega\rangle $ for every $T$ in $M$. 

If $T\Omega=0$, then $0=\| T\Omega \|^2=\tau(T^*T)$, so that $T=0$. That means that $\Omega$ is separating for~$M$.

\begin{lemma}
If $M\subseteq\oB(\hH)$ is a von~Neumann algebra, a vector in $\hH$ is cyclic for $M$ if and only if it is separating for $M'$.
\end{lemma}
\begin{proof}
A proof is given in \cite{JonesSunder}.
\end{proof}

Now, for every $\xi\in L^2(M)$, we define the operators $\pi_l$ and $\pi_r$ by
\begin{subequations}
\begin{align}
	\pi_l(\xi)(T'\Omega)&= T'\xi\\
	\pi_r(\xi)(T\Omega) &= T\xi
\end{align}
\end{subequations}
for $T'\in M'$ and $T\in M$. The domains are $\dom\big( \pi_l(\xi) \big)=M'\Omega$ and $\dom\big( \pi_r(\xi) \big)=M\Omega$. These operations are well defined because $\Omega$ is cyclic and separating for $M$ and $M'$.

If $\pi_l(\xi)$ extends to a bounded operator on $L^2(M)$, one says that $\xi$ is \defe{left bounded}{left!bounded vector}\index{right!bounded vector}., and the (necessarily unique) extension is still denoted by $\pi_l(\xi)$. We do the same for $\pi_r$. Let now the map
\begin{equation}
\begin{aligned}
 J\colon M\Omega&\to M\Omega \\ 
   T\Omega&\mapsto T^*\Omega, 
\end{aligned}
\end{equation}
this is a conjugate linear isometry, so that it extends to an anti-unitary involution.
\begin{equation}
	J\colon L^2(M)\to L^2(M)
\end{equation}
That map $J$ is the \defe{modular conjugation}{modular!conjugation} operator for the factor $M$ of type $II_1$.

The algebra $JMJ$ acts on $L^2(M)$ as well as on $L^2(JMJ)$. If $\xi\in L^2(M)$, the action is
\begin{equation}		\label{EqActJMJLdM}
	(JSJ)\xi=\xi S^*.
\end{equation}
That one is inspired by the fact that $JSJT=J(ST^*)=TS^*$. Since $J^2=1$, the action of $JMJ$ on $L^2(JMJ)$ is given by
\begin{equation}		\label{EqActJMJLdJMJ}
	(JSJ)(JTJ)=J(ST)J.
\end{equation}

Let $M$ be a type $II_1$ factor. We denote $\hH_1(M)=L^2(M)$\nomenclature{$\hH_1=L^2(M)$}{A completion of a von~Neumann algebra}, or simply by $\hH_1$ when no confusion is possible, and $\hH_{\infty}=L^2(M)\otimes l^2$\nomenclature{$\hH_{\infty}$}{$\hH_1\otimes l^2$}. The algebra $M$ acts on $\hH_1$ by
\begin{equation}
	\pi_1(S)\xi=S\xi,
\end{equation}
and $M$ acts on $\hH_{\infty}$ by
\begin{equation}
	\pi_{\infty}(S)=\pi_1(s)\otimes\id_{l^2}.
\end{equation}
We also note $\eM_{\infty}(M)=M\otimes\oB(l^2)$, that has to be understood as infinite matrices with elements in $M$. It is a type $II_{\infty}$ factor, and the trace is a follows. If $p\in\eM_{\infty}(M)=(p_{ij})$, then
\begin{equation}					\label{EqTraceMinfinuM}
	\tr(p)=\sum_{i=1}^{\infty}\tr_M(p_{ii}).
\end{equation}
With that formula, when $q\in\oP(l^2)$ has rank $1$, then $\tr(1_m\otimes q)=1$. Let $q\in\eM_n(M)$ and consider the element of $\eM_{\infty}(M)$ given by the following:
\begin{equation}
	\begin{pmatrix}
  q	&	0	\\ 
  0	&	0	
\end{pmatrix}.
\end{equation}
It's trace, in $\eM_{\infty}(M)$, is $\sum_{i=1}^n\tr(q_{ii})$, but the trace of $q$ in $\eM_n(M)$ is $\frac{1}{ n }\sum_{i-1}^n\tr(q_{ii})$. The normalization is not the same in $\eM_n(M)$ and in $\eM_{\infty}(M)$.

We denote by $e_{11}$, the element of $\eM_n(M)$ given by
\begin{equation}
	e_{11}=
\begin{pmatrix}
  1	&		\\ 
  	&	0	
\end{pmatrix}.
\end{equation}
We have $\tr(e_{11})=1/n$, and $\pi_e(e_{11})\in\oP\Big( \pi_r\big( \eM_n(M) \big) \Big)$.

\begin{lemma}
	The $\oL_M\big( L^2(M) \big)$-modules $\oL_M\big(L^2(M)\big)$ and $L^2(M)$ are isomorphic.
\end{lemma}

\begin{proof}
	The result comes from the fact that an element of $\oL_M\big(L^2(M)\big)$ is uniquely defined by its value at $1\in M$.
\end{proof}
As a consequence of that lemma, we have
\begin{equation}
	\dim_{\oL_M\big( L^2(M) \big)}L^2(M)=1,
\end{equation}
because
\begin{equation}
	\Big( \oL_M\big( L^2(M) \big)\otimes l^2 \Big)p=L^2(M)
\end{equation}
when $p=\id\otimes e_{11}$, whose trace is\footnote{it is not $1/n$ or anything like that because of the remark we did bellow the definition \eqref{EqTraceMinfinuM}.} $1$.

\begin{lemma}		\label{LemLJMJequalLM}
	We have the isomorphism
	\begin{equation}
		L^2(JMJ)\simeq L^2(M)
	\end{equation}
as $JMJ$-module.
\end{lemma}

\begin{proof}
	Let us prove that the map which extend the following is an isomorphism:
	\begin{equation}
		\begin{aligned}
		 \psi\colon L^2(JMJ)&\to L^2(M) \\ 
		   JSJ&\mapsto S^* 
		\end{aligned}
	\end{equation}
	So we have to prove that for every $a\in JMJ$ and $\xi\in L^2(JMJ)$, we have $\psi(a\xi)=a\psi(\xi)$. Using the actions \eqref{EqActJMJLdM} and \eqref{EqActJMJLdJMJ}, if $a=JSJ$ and $\xi=JTJ$, we find
	\begin{equation}
		\psi(a\xi)=\psi(JSJJTJ)=\psi(JSTJ)=T^*S^*, 
	\end{equation}
	while
	\begin{equation}
		a\psi(\xi)=(JSJ)\psi(JTJ)=(JSJ)T^*=T^*S^*.
	\end{equation}
	That prove that $L^2(M)\simeq L^2(JMJ)$ as $JMJ$-modules.
\end{proof}

\begin{lemma}
	We have
	\begin{equation}		\label{EqoLLdpireununmodE}
		\oL_M\big( L^2(M) \big)=\oL_{M,\pi_r(e_{11})}(\modE_n)
	\end{equation}
	where $\modE_n=L^2(M)\oplus \cdots\oplus L^2(M)$ ($n$ terms).
\end{lemma}

\begin{proof}
An element in the right hand side of \eqref{EqoLLdpireununmodE} is a right-linear map with respect to $\pi_r(e_{11})$, i.e. a map $f\colon \modE\to \modE$,
\begin{equation}
	f\big( (\xi_1,\ldots,\xi_n)\pi_r(e_{11}) \big)=f(\xi_1,\ldots,\xi_n)\pi_r(e_{11}).
\end{equation}
Since the left hand side only depends on $\xi_1$, the right hand side  shows that $f(\xi_1,\cdots,\xi_n)$ only depends on $\xi_1$. Now, the right hand side takes its values in $L^2(M)$, so that the left hand side shows that $f$ takes its values in $L^2(M)$.
\end{proof}

The following is the proposition 2.2.6 in \cite{JonesSunder}.
\begin{proposition}	\label{PropDimIIun}
	Let $\hH$ be a separable Hilbert space and $M\subseteq\oB(\hH)$, a factor of type $II_1$. We have
	\begin{enumerate}

		\item\label{ItemiPropDimIIun} for every $d\in[0,1]$, there exists a $M$-module $\modE_d$ such that $\dim_M\modE_d=d$.
		
		\item\label{ItemiiPropDimIIun} There exists an unique $d\in[0,\infty]$ such that $\hH\simeq\modE_d$ as $M$-module.

		\item\label{ItemiiiPropDimIIun} The algebra $M'$ is a factor of type $II_1$ if and only if $\dim_M\hH<\infty$.

		\item\label{ItemivPropDimIIun} $\dim_M L^2(M)=1$.
 
		\item\label{ItemvPropDimIIun} If $\{ \modF_n \}_{n\in\eN}$ is a set of separable $M$-modules, then
			\begin{equation}
				\dim_M\big( \oplus_n\modF_n \big)=\sum_M\dim_M\modF_n.
			\end{equation}
		\item\label{ItemviPropDimIIun} If $\dim_M\hH<\infty$ and if $P'\in\oP(M')$, then 
			\begin{equation}
				\dim_{PMP}(P\hH)=\big( \tr_M(P) \big)^{-1}\dim_M\hH.				
			\end{equation}
	
		\item\label{ItemviiPropDimIIun} If $P$ is a projection in $M$, then 
			\begin{equation}
				\dim_{PMP}(P\hH)=\big( \tr_M(P) \big)^{-1}\dim_M\hH.
			\end{equation}

		\item\label{ItemviiiPropDimIIun}  We have
		\begin{equation}				\label{EqDimMMprimeprodun}
			\dim_{M'}\hH=(\dim_M\hH)^{-1}
		\end{equation}
		if $M'$ is a factor of type $II_1$
	\end{enumerate}
\end{proposition}

\begin{proof}

	For \ref{ItemiPropDimIIun}, let begin with $d=n\in \eN$, and define
	\begin{equation}
		\modE_n=\underbrace{  L^2(M)\oplus\ldots\oplus L^2(M)   }_{\text{$n$ times}}
	\end{equation}
	which can be seen as an element of $\eM_{1\times n}\big( L^2(M) \big)$. This is a $M$-$\eM_n(M)$-bimodule, for the matrix multiplication. What we have is to compute $\dim_{M}\hH_n=\tr p$ where $p\in\oP\big( \eM_{\infty}(M) \big)$ is the projection such that $\big( L^2(M)\otimes l^2 \big)P=\hH_n$. 
	
	Intuitively, $P$ is the projection onto the first $n$ component of the space of infinite vertical matrices. Indeed, the picture is that $\hH_n=\eM_{1\times n}\big( L^2(M) \big)$ while $\hH_{\infty}=\eM_{1\times \infty}\big( L^2(M) \big)$. We are searching for $p\in\oP(\eM_{\infty}(M))$ such that 
	\begin{equation}
		\big( L^2(M)\otimes l^2 \big)p=L^2(M)\oplus\ldots\oplus L^2(M).
	\end{equation}
	The answer is $p=\id_{M}\otimes \pr_n$ where $\pr_n$ stands for the projection onto the first $n$ components. Using formula \eqref{EqTraceMinfinuM}, we conclude that $\dim_M\modE_n=n$. 

	Let us now consider $d\in[0,\infty[$, and an integer $n\geq d$. Since $\eM_n(M)$ is a factor of type $II_1$, proposition \ref{PropFactIIunttedim} provides a projection $q\in\oP\big( \eM_m(M) \big)$ such that $\tr_{\eM_n(M)}(q)=d/n$. Now we look at
	\begin{equation}
		\modE_d=\modE_nq.
	\end{equation}
	In order to compute $\dim_M(\modE_d)$, we have to find $r\in\oP\big( \eM_{\infty}(M) \big)$ such that
	\begin{equation}
		\big( L^2(M)\otimes l^2 \big)r\simeq \big( L^2(M)\oplus\ldots\oplus L^2(M) \big)q.
	\end{equation}
	The answer is
	\begin{equation}
		r=
	\begin{pmatrix}
	  q	&	0	\\ 
	  0	&	0	
	\end{pmatrix}
	\end{equation}
	whose trace is 
	\begin{equation}
		\tr_{\eM_{\infty}(M)}(r)=\sum_{i=1}^{\infty}\tr_M(r_{ii})=\sum_{i=1}^n\tr_M(q_{ii})=n\tr_{\eM_n(M)}(q),
	\end{equation}
	because of the discussion bellow the definition \eqref{EqTraceMinfinuM}. This concludes the proof of \ref{ItemiPropDimIIun}.
	
	We pass to the proof of \ref{ItemviiiPropDimIIun}. Since $\dim_M\hH<\infty$, we know that $\hH$ is a factor of type $II_1$ by \ref{ItemiiiPropDimIIun}. We can thus suppose that $\hH=\modE_d$ for some $d$. One knows that $M'=JMJ$, so that $\dim_{M'}\modE_d=\dim_{JMJ}\modE_d$. 
	
	Let us first work with $d=1$. In order to compute $\dim_{JMJ}\big( L^2(M) \big)$, we are searching for $p\in\oP\big(\eM_{\infty}(JMJ))$ such that $\big( L^2(JMJ)\otimes l^2 \big)p\simeq L^2(M)$ as $JMJ$-modules. From lemma \ref{LemLJMJequalLM}, $P=e_{11}$ works.
	
	Let us now study the case $d=n\in\eN$.
	
\end{proof}

\begin{corollary}
	When $M$ is a factor of type $II_1$, two $M$-modules are isomorphic if and only if they have same dimension.
\end{corollary}
This is a direct consequence of point \ref{ItemiiPropDimIIun} in proposition \ref{PropDimIIun}.


%---------------------------------------------------------------------------------------------------------------------------
					\subsection{Dimension}
%---------------------------------------------------------------------------------------------------------------------------

We are going to associate any $M=M(\pi)$-module with a dimension in $[0,1]$ such that for every projection $P\in M$, the dimension of the submodule $M(\pi)P$ is equal to $\dim\big( M(\pi)P \big)=\tr(P)$. We will in fact give a dimension to every module over a von~Neumann algebra accepting a trace. For that, we use theory developed in section \ref{SecModUnitalAnneau}.

\begin{theorem}			\label{ThofgsurMFSubEEClSplits}
If $M$ is a finite von~Neumann algebra and if $E$ is a finitely generated module over $M$, and if $F$ is any submodule, then the quotient map $E\to E/\Cl_E(F)$ splits and furthermore the quotient space $E/\bar E$ is projective and finitely generated.
\end{theorem}

\begin{probleme}
Has one to add the assumption that $F$ is projective ?
\end{probleme}
\begin{proof}
Later.
\end{proof}

As consequence of the this theorem, $F\simeq \bar E\oplus(\text{finite projective module})$. In particular,
\begin{equation}
	F\simeq \Cl_F(0)\oplus  (\text{finitely generated projective module})
\end{equation}

\begin{theorem}		\label{ThoPropDimiM}
Let $M$ be a finite von~Neumann algebra with a normal faithful tracial state $\varphi$. There is an unique function 
\[ 
	\dim_{\varphi}\colon \{ \text{$M$-modules} \}\to [0,\infty]
\]
such that
\begin{enumerate}
\item $F_1\simeq F_2$ implies $\dim_{\varphi}F_1=\dim_{\varphi}F_2$,
\item (normalisation) if $F$ is a finitely generated projective module, then the value of $\dim_{\varphi}(F)$ coincides with the definition \eqref{DefDimFunctModule}, in particular when $E\simeq M^nP$ is finitely generated and projective, we have $\dim E=\sum_i\varphi(P_{ii})$,
\item (continuity) if $E$ is any submodule on a finitely generated module $F$, then $\dim_{\varphi}(E)=\dim_{\varphi}\big( \Cl_F(E) \big)$\footnote{From an analytic point of view, that condition is not usual, as can be seen on the example $\dim\eQ=0$ while $\dim\eR=1$.},
\item\label{ItemCofinalitySuiteExact} (cofinality) given an exact sequence
\[ 
	\xymatrix{%
   0\ar[r] 	&F_1\ar[r]	&F\ar[r]	&F_2 \ar[r]	&0,
}
\]
then $\dim_{\varphi}(F)=\dim_{\varphi}(F_1)+\dim_{\varphi}(F_2)$.
\end{enumerate}
\end{theorem}

\begin{proof}
later.
\end{proof}

Notice that, since $P=P^*P$ for every projection, we have $\dim E\geq 0$.

We will prove the following
\begin{proposition}			\label{PropDimClEgalDim}
If $E$ is a finitely generated projective module over $M$, we have
\begin{equation}
	\Dim(F)=\Dim\big( \Cl_E(F) \big)
\end{equation}
if $F$ is any submodule of $E$.
\end{proposition}

\begin{probleme}
One has to check that is it actually proved somewhere in the next pages.
\end{probleme}

We will not often explicitly write the trace $\varphi$ and simply write $\dim(E)$. However, we will sometimes precise the ring over which we consider the module and denote by $\dim_M(E)$\nomenclature{$\dim_M(E)$}{The dimension of $E$ as module over $M$} the dimension of $E$ as module over $M$.

%///////////////////////////////////////////////////////////////////////////////////////////////////////////////////////////
					\subsubsection{Examples}		\label{subsubsecExemDimMMMod}
%///////////////////////////////////////////////////////////////////////////////////////////////////////////////////////////

Let $M$ be a von~Neumann algebra with a normal faithful tracial state $\varphi$, and let us give some examples of modules over $M$.

First, $M$ itself is a free module of dimension $1$\footnote{In fact that dimension is $\tr(\mtu)$ which is usually normalised at $1$, see definition \eqref{EqPreDefDimModuleRA} and bellow.} over $M$. If $P$ is a projection, then $MP$ is a projective module\label{PgMPprojModule} and $\dim_{\varphi}(MP)=\varphi(P)$. The fact that $MP$ is a projective module comes from the direct sum decomposition $M=MP\oplus M(\mtu-P)$ and the third characterisation if projective module in proposition \ref{PropEquivProjModule}. Notice that $M$ is a free module over $M$ because $M=M\mtu$.

\begin{proposition}		\label{PropMTprojpourtoutT}
For every element $T\in M$, the set $MT$ is a projective $M$-module.
\end{proposition}

\begin{proof}
Let $P$ be the projection onto $\overline{ \Image(T) }$, so that $P=\lim_{n\to\infty}(TT^*)^{1/n}$ as already seen. We are going to prove that $MT\simeq MP$ as $M$-module. For we define $\Phi\colon MT\to MP$ by $\Phi(ST)=SP$. In order to see that this is a good definition, suppose $S_1T=S_2T$, then $(S_1-S_2)T=0$. But we have
\begin{align}
	ST&=0&\Rightarrow &&S|_{\Image(T)}&=0&\Rightarrow &&S|_{\overline{ \Image(T) }}&=0&\Rightarrow	&&S|_{\Image(P)}&=0,
\end{align}
so that $SP=0$. It shows that $\Phi$ is a well defined $M$-module isomorphism between $MT$ and $MP$.
\end{proof}

By polar decomposition, we have $MT=M| T |$, so we can suppose $T\geq 0$ without loss of generality. We have the exact sequence
\[ 
	\xymatrix{%
   0\ar[r] 	&MT\ar[r]^{\Phi}	&MP\ar[r]	&MP/MT \ar[r]	&0,	
}
\]
so that by property \ref{ItemCofinalitySuiteExact} of theorem \ref{ThoPropDimiM}, we have $\dim_{\varphi}(MT)=\dim_{\varphi}(MP)+\dim_{\varphi}(MP/MT)$. Since the modules $MP$ and $MT$ are isomorphic, that means that 
\[ 
	\dim_{\varphi}(MP/MT)=0.
\]

As another example, take projections $P_m$ with $\varphi(P_m)=2^{-m}$, and let $F$ be the infinite algebraic direct sum $MP_1\oplus MP_2\oplus\ldots$. By cofinality, it has at least the dimension of each of its submodule. In other words, we have
\[ 
	\dim F=\sup_n\dim(MP_1\oplus\ldots\oplus MP_n)=1.
\]

Since we can embed $F$ into $M$, we also get $\dim(M/F)=0$.

%///////////////////////////////////////////////////////////////////////////////////////////////////////////////////////////
					\subsubsection{Summary}
%///////////////////////////////////////////////////////////////////////////////////////////////////////////////////////////

We want to define, for each module, a dimension such that
\begin{itemize}
\item if $E$ is a projective left module with $E\simeq M^nP$ (see proposition \ref{PropFGPRkP}) and $P=P^2\in\eM_n(M)$,
\[ 
	\dim(E)=\sum \varphi(P_{ii}),
\]
and does not depend on the choices.
\item If $E$ is not a finitely generated projective module, then
\begin{equation}		\label{DefDimAvecGrandD}
	\Dim(E)=\sup\big\{  \dim(F)\text{ with $F$ finite projective submodule of $E$} \big\}.
\end{equation}
\end{itemize}

%+++++++++++++++++++++++++++++++++++++++++++++++++++++++++++++++++++++++++++++++++++++++++++++++++++++++++++++++++++++++++++
					\section{Position of submodules}
%+++++++++++++++++++++++++++++++++++++++++++++++++++++++++++++++++++++++++++++++++++++++++++++++++++++++++++++++++++++++++++

\begin{proposition}
Let $M$ be a von~Neumann algebra (not specially with trace). Every finitely generated submodule of a finite generated projective module is projective.
\end{proposition}

\begin{proof}
Passing to the matrix algebra (lemmas \ref{LemEprojEEEfproj} and \ref{LemFGenEEEsingleGen}), one can assume that the module and the submodule are in fact singly generated.

A singly generated module over $ M$ has the form $E=MT\subseteq M$ for some element $T\in M$. We saw during the proof of proposition \ref{PropMTprojpourtoutT} that $MT\simeq MP$ where $P$ is the projection onto $\overline{ \Image(T) }$. The fact that $MP$ is a projective module was already argued on page \pageref{PgMPprojModule}.
\end{proof}

\begin{proposition}			\label{PropEfgpFssmodQuotProj}
If $E$ is a finite projective module over $M$ and $F$ is a submodule of $E$, then $E/\Cl_E(F)$ is projective.
\end{proposition}

\begin{proof}
Once again, using the matrix trick, one can suppose that $E$ is singly generated by an element $e$. A left module map $\varphi\colon E\to M$ such that $\varphi(F)=0$ is determined by the value of $\varphi(e)\in M$. The element $\varphi(e)$ is an operator on $\hH$ and we define $P_{\varphi}$ as the projection onto $\overline{ \Image\big( \varphi(e) \big) }$.

Now we define the module map
\begin{equation}
\begin{aligned}
\tilde{\varphi} \colon E&\to M \\ 
   \tilde{\varphi}(Te)&\mapsto TP_{\varphi}. 
\end{aligned}
\end{equation}
This is well defined. Indeed if $Se=0$, then $\varphi(Se)=S\varphi(e)=0$, which implies that $SP_{\varphi}=0$. For the same reason, $\ker(\varphi)=\ker(\tilde{\varphi})$. We define 
\begin{equation}
	Q=\bigvee_{\varphi}P_{\varphi},
\end{equation}
and 
\begin{equation}
\begin{aligned}
 \psi\colon E&\to M \\ 
   \psi(Te)&\mapsto TQ 
\end{aligned}
\end{equation}
which is well defined because $TQ=0$ if and only if $TP\varphi =0$ for every $\varphi$, since the operator $T$ is bounded\footnote{If $T$ is not bounded, it can get bigger and bigger on the range of $\varphi_k$ when $k$ goes to infinity, so that the limit of $T\bigvee_{\varphi\in\alpha}$ is not zero when $\alpha$ gets bigger.}.
\begin{probleme}
The explanation in the footnote is unclear; it has to be expressed in terms of nets.
\end{probleme}
By construction, $\ker(\psi)=\Cl_E(F)$, while $\Image(\psi)=MQ$, so that
\begin{equation}
	E/\ker(\psi)\simeq MQ,
\end{equation}
while the latter is projective.
\end{proof}

\begin{corollary}		\label{CorEfgpFssIsom}
When $E$ is a finite projective module and $F$ a submodule, we have an isomorphism
\[ 
	E=\Cl_E(F)\oplus E/\Cl_E(F)
\]
as direct sum of modules.
\end{corollary}

\begin{proof}
If $\xi\in E$, the class of $\xi$ in $E/\Cl_E(F)$ is
\begin{equation}
	[\xi]=\{ \xi+\eta\tq \eta\in\Cl_E(F) \}.
\end{equation}
Let us choose a representative $[\xi]_0$ in each of the classes\quext{This uses the famous axiom; it would be possible to do otherwise, isn't ?}. The following is the module isomorphism we are searching for
\begin{equation}
\begin{aligned}
 \psi\colon E&\to \Cl_E(F)\oplus E/\Cl_E(F) \\ 
   \xi&\mapsto  \eta\oplus [\xi]_0 
\end{aligned}
\end{equation}
where $\eta$ is the unique element of $\Cl_E(F)$ such that $[\xi]_0+\eta=\xi$.
\end{proof}

When $E$ is a finite projective module over $M$, we say that $\Cl_E(0)$ is the \defe{torsion submodule}{torsion!submodule} of $E$.

Let us see an example. Let 
\begin{equation}
	E=M/MT 
\end{equation}
with $T\geq 0$ and $\overline{ \Image(T) }=\hH$. We claim that the torsion submodule of $E$ is $E$ itself. A module map $\varphi\colon E\to M$ is a module map $M\to M$ composed with a projection, or in other words a module map $\varphi\colon M\to M$ such that $\varphi(T)=0$. Since $\varphi$ is a module map, its fulfils
\begin{equation}
	\varphi(S)=\varphi(S\mtu)=S\varphi(\mtu)=SX
\end{equation}
for a certain $X\in M$ such that $TX=0$. Thus we have $\overline{ \Image(X) }\subset\ker(T)$, but since $T\geq 0$ (in particular $T$ is self-adjoint) and $\overline{ \Image(T) }=\hH$, we have $\ker(T)=0$, and we conclude that $X=0$, so that $\varphi=0$. The question is to know if $E=M/MT$ has a finitely generated submodule or not. Let $P$ be a projective submodule of $M/MT$; by the lifting property \eqref{EqLiftPropProjModules} we have a map $\tilde \lambda$ such that the following commutes
\begin{equation}
\xymatrix{%
 									&  M \ar@{->>}[d]^{\displaystyle\lambda}\\
   P \ar[r]\ar@{.>}[ru]							& M/MT
}
\end{equation}
where the arrow from $P$ to $M/MT$ is injective. The image of $P$ by $\tilde\lambda$ is a submodule $MX$ that has to be injectively\quext{How to say the fact to be injective in English ? Does the word \emph{injectively} exist ?} projected in $M/MT$, so that $MX\cap MT=\emptyset$. Notice that it is not possible when $\hH$ is finite dimensional because $T$ is invertible (from the fact that the closure of its image is the whole space), so that $MT=M$.

Now we suppose that $X$ is positive. This is done without loss of generality because from polar decomposition, for every $X\in M$, there exists a positive $Y$ such that $MX=MY$.

Since $T$ is positive, we can consider the spectral theorem \ref{ThoSpectralTho} and the isomorphism
\begin{equation}
\begin{aligned}
 \theta\colon C^*(T,\cun)&\to C\big( \Spec(T) \big). 
\end{aligned}
\end{equation}
We define the following function on $\Spec(T)$
\begin{equation}
	f_{\epsilon}=
			\begin{cases}
					0				&\text{if $x<\epsilon$}\\
					\frac{ 1 }{ (\theta T)(x) }	&\text{if $x\geq\epsilon$,}
			\end{cases}
\end{equation}
and then one defines the operator $S_{\epsilon}=\theta^{-1}(f_{\epsilon})\in C^*(T,\cun)$. Let us prove that $P_{\epsilon}=S\epsilon T$ is a projection. We have $P_{\epsilon}^2=S_{\epsilon} TS_{\epsilon}T$. Take an orthonormal basis of $\hH$ of eigenvectors of $T$ and let's call $\lambda_i$ the eigenvalues: $Te_i=\lambda_ie_i$. If $\lambda_k<\epsilon$, then $S_{\epsilon}e_i=0$ and of course $S_{\epsilon}TS_{\epsilon}e_i=S_{\epsilon}e_i$. Otherwise, we have
\begin{equation}
	S_{\epsilon}TS_{\epsilon}e_i=\frac{1}{ \lambda_i }S_{\epsilon}T e_i =S_{\epsilon}e_i.
\end{equation}
That proves that $P_{\epsilon}^2=P_{\epsilon}$, and so that this is a projection. We obviously also have $P_{\epsilon}\to \mtu$ in $MT$.

Similarly one can define $Q$, the projection onto $\overline{ \Image(X) }$ and we have projections $Q_{\epsilon}\to Q$ with $Q_{\epsilon}=Y_{\epsilon} X$ for some $Y\epsilon\in M$ and $Q_{\epsilon}\in MX$. Now take $\epsilon_1$ and $\epsilon_2$ and look at the projection
\begin{equation}
	Q_{\epsilon_1}\vee P_{\epsilon_2}
\end{equation}
onto $\Image(Q_{\epsilon_1})\cap\Image(P_{\epsilon_2})$. The latter intersection is in fact $0$ because $A_{\epsilon_1}\vee P_{\epsilon_2}\in MQ_{\epsilon_1}\cap MP_{\epsilon_2}=MX\cap MT=0$. Using lemma \ref{LemDimSupDeuxProjs}, we get
\begin{equation}
	\dim\mtu\geq \dim(Q_{\epsilon_1}\vee P_{\epsilon_2})=\dim(Q_{\epsilon_1})+\dim(P_{\epsilon_2}).
\end{equation}
Recall that the trace used to define the dimension has to be normal, so that the dimension function is continuous in such a way that taking the limit $\epsilon_1\to 0$ and $\epsilon_2\to 0$ provides the expected result
\begin{equation}
	\dim(\mtu)\geq \dim(Q)+\dim(\mtu),
\end{equation}
from which one deduce that $\dim Q=0$ and therefore $X=0$. This finish the proof that the module $E=M/MT$ with a positive $T$ and $\overline{ \Image(T) }=\hH$ has no projective submodules.

\begin{proposition}
If $E$ is a finitely generated module over $M$, then the torsion submodule does not contains non zero projective finite submodules.
\end{proposition}

\begin{proof}
Later.
\end{proof}


\begin{lemma}			\label{LemFClosEF}
If $E$ is a finitely generated submodule of a finitely generated projective module $E$, then $F$ is projective and $F\simeq \Cl_E(F)$.
\end{lemma}

\begin{proof}
We assume as usual that $E$ and $F$ are singly generated. The singly generated projective module $E$ reads $E=MP$ while the general form of a singly generated submodule is $F=MT$ for a positive $T$ which vanishes on $P^{\perp}\hH$ and $\Image(T)\subseteq\Image(P)$. We already proved that $MT\simeq MQ$ where $Q$ is the projection over $\Image(T)$.

We have $\Cl_{MP}(MT)=MQ$ because of the direct sum decomposition $MP=MQ\oplus M(P-Q)$ from which we can build an homomorphism $MP\to M$ which vanishes on $MT$, namely the projection  because $MT\subseteq MQ$.
\begin{probleme}
I do not understand one single word about the latter justification $:($
\end{probleme}
\end{proof}

\begin{corollary}		\label{Corfgfgdilleqdim}
If $F$ is a finitely generated submodule of a finitely generated projective module $E$, then $\dim(F)\leq\dim(E)$.
\end{corollary}

\begin{proof}
By lemma \ref{LemFClosEF}, we have $\dim(F)\simeq\dim\big( \Cl_E(F) \big)$ while we know that $E=\Cl_E(F)\oplus E/\Cl_E(F)$. The latter makes that $\dim E$ is given by the trace of two projections:
\begin{equation}
	\dim E=\tr P_{\Cl_E(F)}+\tr P_{E/\Cl_E(F)}.
\end{equation}
\end{proof}

\begin{corollary}
If $E$ is a finitely projective module over a von~Neumann algebra with a trace, then $\dim(E)=\Dim(E)$. 
\end{corollary}

\begin{proof}
By very definition, $\Dim(E)\geq\dim(E)$ because $E$ itself belongs to the set on which the supremum is taken in the definition \eqref{DefDimAvecGrandD} of $\Dim$. The corollary \ref{Corfgfgdilleqdim} provides the opposite inequality.
\end{proof}

\begin{proposition}		\label{PropProjFiniDimCldim}
If $E$ is projective finitely generated and if $F\subseteq E$, then $\dim\big( \Cl_E(F) \big)=\Dim(F)$.
\end{proposition}

\begin{proof}
The case where $F$ is finitely generated is already done by lemma \ref{LemFClosEF}. For the general case, suppose that the proposition does not hold. In this case, the lemma \ref{LemFClosEF} yields
\begin{equation}
	\sup\{ \dim\big( \Cl_E(H) \big)\tq \text{$H$ is finite projective} \}<\dim\big( \Cl_E(F) \big).
\end{equation}
On the one hand, by corollary \ref{CorEfgpFssIsom}, the module $\Cl_E(H)$ is a direct summand of $\Cl_E(F)$ which is itself (by the same result) a direct summand of $E$. On the other hand, being a finitely generated module, it reads $E=M^nP$ for some projection $P$ by proposition \ref{PropFGPRkP}. Combining both, we have a projection $P_H\leq P$ such that $\Cl_E(H)=M^nP_H$.

Now the finitely generated projective submodules of $F$ form a directed system. For every finitely generated projective submodules $H_1$ and $H_2$ of $F$, there exists at least $H_3=H_1\oplus H_2$ which contains $H_1$ and $H_2$. Thus the set of $P_H$ is directed too and one can look at $\bigvee_HP_H$, the smalest projection whose range contains the range of all of the $P_H$.

We have $\bigvee_HP_H<P$ because if not, using normality of the trace,
\begin{equation}
	\dim\big( \bigvee_HP_H \big)=\sup_H\dim(P_H)<P.
\end{equation}
Now let $Q=P-\bigvee_HP_H$. If $F$ is some sunmodule of $E$, one has
\begin{equation}
\begin{split}
	F&=\bigcup\{ H\subseteq F\tq \text{$H$ is finitely generated} \}\\
	&=\bigcup\{ H\subseteq F\tq \text{$H$ is finitely generated and projective} \}
\end{split}
\end{equation}
because $H\subseteq F\subseteq E$ which is finitely generated projective. Such a $H$ has the form $M^nP_H$, sp $F\subseteq M^n\big( \bigvee_HP_H \big)$. Notice that this $F$ lies in the kernel of the non zero map
\begin{equation}
\begin{aligned}
 \Cl_E(F)=M^nP&\to M^nQ \\ 
   V&\mapsto VQ. 
\end{aligned}
\end{equation}
Indeed, since $\bigvee_HP_H<P$, we have $\big( \bigvee_HP_H \big)P=\bigvee_HP_H$, so that for every $T\in M^n$ we have $T\bigvee_HP_H\big( P-\bigvee_HP_H \big)=0$. By looking at the complement of $VQ$, one has a nonzero homomorphism $ \Cl_E(F)\to M$ which vanishes on $F$. That contradicts the definition of the closure.

\end{proof}

The three point in this demonstration that use the von~Neumann algebra blackground (and not only general module theory) are the following.
\begin{itemize}
\item First we used normality of the trace to commute the dimension with the suppremum,
\item and second, we used continuity of $\bigvee$ with respect to the dimension.
\end{itemize}

%---------------------------------------------------------------------------------------------------------------------------
					\subsection{Summary}
%---------------------------------------------------------------------------------------------------------------------------

We have two dimension functions $\dim$ and $\Dim$ such that
\begin{enumerate}
\item $\dim E=\Dim E$ whenever $E$ is a finitely generated projective module,
\item for every finitely generated module $E$ and every submodule $F$, the module $E/\Cl_E(F)$ is finitely generated and projective,
\begin{probleme}
Check if one does not need the assumption that $E$ is projective too.
\end{probleme}
\item If $F\subseteq E$ and if $E$ is a finitely generated projective module, then $\Dim(F)=\Dim\big( \Cl_E(F) \big)$.
\end{enumerate}
From now we do no more use the ``von~Neumann algebra'' assumption. Instead we suppose to have a ring $R$ and two dimensions functions satisfying these three properties.

%---------------------------------------------------------------------------------------------------------------------------
					\subsection{Properties of the dimension function}
%---------------------------------------------------------------------------------------------------------------------------

\begin{proposition}
If $E$ is the union of a directed system of submodules $E_{\alpha}$, then $\Dim(E)=\sup \Dim E_{\alpha}$.
\end{proposition}

\begin{proof}
A finitely generated projective submodule in $E$ is generated by $n$ elements, each of them being contained in some $E_{\alpha_1}$, $E_{\alpha_2},\ldots$ By definition of a directed set, the union of all the so defined $E_{\alpha_i}$ is contained in a $E_{\beta}$. Thus every finitely generated projective submodule in $E$ is of the form $E_{\beta}$.
\end{proof}

\begin{lemma}			\label{LemHinjectifHdimdim}
If one has a projective module map $\rho\colon H_1\to H_2$, then $\Dim(H_1)\geq\Dim(H_2)$.
\end{lemma}

\begin{proof}
Let $F$ be any projective module for which there exists an inclusion $\iota\colon F\to H_2$. That map can be lifted because $F$ is projective. So among all the submodules of $H_1$, there is the one which is the image of $F$ be the lifted map. That one of course contains $H_2$ itself. Thus $H_2$ is a submodule of $H_1$ and the supremum defining the dimension in $H_1$ is automatically bigger or equal to the one defining the dimension of $H_2$.
\end{proof}

\begin{proposition}
If
\begin{equation}
	\xymatrix{%
   0\ar[r] 	&E_0\ar[r]^{}	&E_1\ar[r]^{p}	&E_2 \ar[r]	&0	
}
\end{equation}
is a short exact sequence of modules, then $\dim(E_1)=\dim(E_0)+\dim(E_2)$.
\end{proposition}

\begin{proof}
Using last proposition, we can assume that all of $E_0$, $E_1$ and $E_2$ are finitely generated. Indeed when a module is not finitely generated, it is still the union of the directed system of all its finitely generated submodules.

Let $F$ be a finitely generated projective submodule of $E_2$, we have the exact sequence
\begin{equation}
	\xymatrix{%
   0\ar[r] 	&E_0\ar[r]	&p^{-1}(F)\ar[r]	&F \ar[r]	&0.
}
\end{equation}
The module $F$ being projective, we have $p^{-1}(F)\simeq F\oplus E_0$. We do not know if the dimension function is additive with respect to direct sum, but by definition of a supremum, we have the inequality $\Dim(E_0)+\Dim(F)\leq \Dim\big( p^{-1}(F) \big)$, and the chain
\begin{equation}
	\Dim(E_0)+\Dim(F)\leq \Dim\big( p^{-1}(F) \big)\leq\Dim(E_1)
\end{equation}
which in turn provides the inequalities
\begin{equation}
	\Dim(E_0)+\Dim(E_2)\leq\Dim(E_1).
\end{equation}
For the reverse inequality, let $F$ be a finitely generated projective submodule of $E_1$, and consider the following exact sequence of finitely generated projective module
\begin{equation}
	\xymatrix{%
   0\ar[r] 	&\Cl_F(F\cap E_0)\ar[r]	&F\ar[r]	&F/\Cl_F(F\cap E_0) \ar[r]	&0	
}
\end{equation}
The fact that $F/\Cl_F(F\cap E_0)$ is projective is proposition \ref{PropEfgpFssmodQuotProj}. Since $F/\Cl_F(F\cap E_0)$ is at most a subset of $F$ which is finitely generated, it has to be finitely generated too. We also know by corollary \ref{CorEfgpFssIsom} that $F$ splits into
\begin{equation}
	F\simeq \Cl_F(F\cap E_0)\oplus F/\Cl_F(F\cap E_0)
\end{equation}
which is a direct sum of finitely generated projective module, so that we can use the definition of dimension with traces (which sums up over direct sum) instead of the one with supremum. Thus we have
\begin{equation}
	\dim(E)=\dim\big( \Cl_F(F\cap E_0) \big)+\dim\big( F/\Cl_F(F\cap E_0) \big).
\end{equation}
The module $F$ being projective and finitely generated, proposition \ref{PropProjFiniDimCldim} allows us to replace $\dim\big( \Cl_F(F\cap E_0) \big)$ by $\Dim(F\cap E_0)$ and write
\begin{equation}
	\dim(E)=\Dim(F\cap E_0)+\Dim\big( F/\Cl_F(F\cap E_0) \big)\leq \Dim(E_0)+\Dim(F/(F\cap E_0))
\end{equation}
where we also used lemma \ref{LemHinjectifHdimdim}. Since $F/(F\cap E_0)$ is a quotient of $E_2$, we have $\Dim(F/(F\cap E_0))\leq\Dim(E_2)$, and taking the supremum over all suitable $F$, we find the result
\begin{equation}
	\dim(E_1)\leq \Dim(E_0)+\Dim(E_2).
\end{equation}
\end{proof}

\begin{proposition}
If $E$ is a finitely generated module and $F\subseteq E$, then $\Dim(F)=\Dim\big(\Cl_E(F)\big)$.
\end{proposition}
\begin{proof}
Hint: by the proposition, one can assume that $E$ is actually projective.
\end{proof}
\begin{probleme}
 That has to be completed.
\end{probleme}
