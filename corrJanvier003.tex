\begin{corrige}{Janvier003}

On sait que $z$ est solution, c'est-à-dire
\begin{equation*}
  a_n z^n + \ldots + a_1 z + a_0 = 0
\end{equation*}
donc en prenant la conjugué des deux membres de l'équation, on obtient
\begin{equation*}
  \overline{a_n z^n + \ldots + a_1 z + a_0} = \bar 0
\end{equation*}
et, comme $\overline{z_1 z_2} = \overline{z_1} \overline{z_2}$ et
$\overline{z_1 + z_2} = \overline{z_1} + \overline{z_2}$ pour tout
$z_1, z_2 \in \eC$, on en déduit
\begin{equation*}
    \bar a_n\bar z^n + \ldots + \bar a_1\bar z +\bar a_0 = \bar 0
\end{equation*}
c'est-à-dire, comme $\bar a = a$ pour tout $a \in \eR$,
\begin{equation*}
  a_n \bar z^n + \ldots + a_1\bar z + a_0 = 0  
\end{equation*}
ce qui montre que $\bar z$ est solution de l'équation.


\end{corrige}
% This is part of the Exercices et corrigés de mathématique générale.
% Copyright (C) 2009
%   Laurent Claessens
% See the file fdl-1.3.txt for copying conditions.
