% This is part of Mes notes de mathématique
% Copyright (c) 2012-2013
%   Laurent Claessens, Carlotta Donadello
% See the file fdl-1.3.txt for copying conditions.

%+++++++++++++++++++++++++++++++++++++++++++++++++++++++++++++++++++++++++++++++++++++++++++++++++++++++++++++++++++++++++++
					\section{Topologie en général}
%+++++++++++++++++++++++++++++++++++++++++++++++++++++++++++++++++++++++++++++++++++++++++++++++++++++++++++++++++++++++++++

\begin{definition}		\label{DefTopologieGene}
Soit $E$, un ensemble et $\mT$, une partie de l'ensemble de ses parties qui vérifie les propriétés suivantes
\begin{enumerate}

\item
les ensembles $\emptyset$ et $E$ sont dans $\mT$,

\item
Si $I$ est n'importe quel ensemble et si pour tout $i\in I$, nous avons un élément $\mO_i\in\mT$, alors $\cup_{i\in I}\mO_i\in\mT$,

\item
Si $J$ est un ensemble fini et si pour tout $j\in J$, nous avons un élément $\mO_j\in\mT$, alors $\cap_{j\in J}\mO_j\in\mT$.

\end{enumerate}
Les deux dernières propriétés s'énoncent en disant que toute réunions d'éléments de $\mT$ est un élément de $\mT$ et que toute intersection \emph{finie} d'éléments de $\mT$ est un élément de $\mT$.

Un tel choix $\mT$ de sous-ensembles de $E$ est une  \defe{\href{http://fr.wikipedia.org/wiki/Espace_topologique}{topologie}}{topologie} sur $E$, et les éléments de $\mT$ sont appelés des \defe{ouverts}{ouvert}. Nous disons que un sous ensemble $A$ de $E$ est \defe{fermé}{fermé} si son complémentaire, $A^c$ est ouvert.
\end{definition}

\begin{definition}
    Un \defe{homéomorphisme}{homéomorphisme} est une application bijective continue entre deux espaces topologiques dont la réciproque est continue. Deux espaces topologiques homéomorphes sont dits \defe{isomorphes}{isomorphisme!d'espaces topologiques}.
\end{definition}

\begin{definition}
    Si deux points distincts ont toujours deux voisinages distincts, nous disons que l'espace est \defe{séparé}{séparé} ou \defe{Hausdorff}{Hausdorff}.
\end{definition}

Dès que nous avons une topologie, nous avons une notion de convergence de suite : nous disons qu'une suite $x_n$ d'éléments de $E$ \defe{converge}{convergence!en topologie} vers l'élément $x$ de $E$ si pour tout ouvert $\mO$ contenant $x$, il existe un $K$ tel que $k>K$ implique $x_k\in\mO$. Cette définition est exactement celle donnée pour la convergence de suites dans $\eR^n$, à part que nous avons remplacé le mot \og boule\fg{} par \og ouvert\fg.

Dans un espace topologique, nous avons une caractérisation très importante des ouverts.
\begin{theorem}		\label{ThoPartieOUvpartouv}
Une partie $A$ de $E$ est ouverte si et seulement si pour tout $x\in A$, il existe un ouvert autour de $x$ contenu dans $A$.
\end{theorem}

\begin{proof}
Le sens direct est évident : $A$ lui-même est un ouvert autour de $x\in A$, qui est inclus à $A$.

Pour le sens inverse, pour chaque $x\in A$, nous considérons l'ensemble $\mO_x\subset A$, un ouvert autour de $x$. Nous avons que
\begin{equation}	\label{EqAUniondesOx}
	A=\bigcup_{x\in A}\mO_x.
\end{equation}
En effet $A\subset\cup_{x\in A}\mO_x$ parce que tous les éléments de $A$ sont dans un des $\mO_x$, par construction. D'autre part, $\cup_{x\in A}\mO_x\subset A$ parce que chacun des $\mO_x$ est compris dans $A$.

L'union du membre de droite de \eqref{EqAUniondesOx} est une union d'ouverts et est donc un ouvert. Cela prouve que $A$ est un ouvert.

\end{proof}

Une utilisation typique de ce théorème est faite à l'exercice \ref{exo0083}.

%+++++++++++++++++++++++++++++++++++++++++++++++++++++++++++++++++++++++++++++++++++++++++++++++++++++++++++++++++++++++++++ 
\section{Fonctions continues}
%+++++++++++++++++++++++++++++++++++++++++++++++++++++++++++++++++++++++++++++++++++++++++++++++++++++++++++++++++++++++++++

\begin{proposition}		\label{PropFnContParSuite}
	Soit une fonction $f\colon X\to Y$ et $a\in X$. La fonction $f$ est continue en $a$ si et seulement si pour toute suite $(x_n)$ dans $X\setminus\{ a \}$ convergente vers $a$, nous avons $\lim f(x_n)=f(a)$.
\end{proposition}
%TODO : une preuve.

%+++++++++++++++++++++++++++++++++++++++++++++++++++++++++++++++++++++++++++++++++++++++++++++++++++++++++++++++++++++++++++
\section{Topologie induite}
%+++++++++++++++++++++++++++++++++++++++++++++++++++++++++++++++++++++++++++++++++++++++++++++++++++++++++++++++++++++++++++

Soit \( X\) un espace topologique et \( A\subset X\). L'ensemble \( A\) devient un espace topologique en lui-même par la \defe{topologie induite}{topologie!induite} de \( X\). Un ouvert de \( A\) est un ensemble de la forme \( A\cap\mO\) où \( \mO\) est un ouvert de \( X\).

\begin{lemma}       \label{LemkUYkQt}
    Si \( B\subset A\) alors la fermeture de \( B\) pour la topologie de \( A\) (induite de \( X\)) que nous noterons \( \tilde B\) est 
    \begin{equation}
        \tilde B=\bar B\cap A
    \end{equation}
    où \( \bar B\) est la fermeture de \( B\) pour la topologie de \( X\).
\end{lemma}

\begin{proof}
    Si \( a\in \bar B\cap A\), un ouvert de \( A\) autour de \( a\) est un ensemble de la forme \( \mO\cap A\) où \( \mO\) est un ouvert de \( X\). Vu que \( a\in\bar B\), l'ensemble \( \mO\) intersecte \( B\) et donc \( (\mO\cap A)\cap B\neq \emptyset\). Donc \( a\) est bien dans l'adhérence de \( B\) au sens de la topologie de \( A\).

    Pour l'inclusion inverse, soit \( a\in \tilde  B\), et montrons que \( a\in \bar B\cap A\). Par définition \( a\in A\), parce que \( \tilde B\) est une fermeture dans l'espace topologique \( A\). Il faut donc seulement montrer que \( a\in\bar B\). Soit donc \( \mO\) un ouvert de \( X\) contenant \( a\); par hypothèse \( \mO\cap A\) intersecte \( B\) (parce que tout ouvert de \( A\) contenant \( a\) intersecte \( B\)). Donc \( \mO\) intersecte \( B\). Cela signifie que tout ouvert (de \( X\)) contenant \( a\) intersecte \( B\), ou encore que \( a\in \bar B\).
\end{proof}

\begin{example} \label{ExloeyoR}
    Si \( A\) est un ouvert de \( X\), on pourrait croire que la topologie induite n'a rien de spécial. Il est vrai que \( B\) sera ouvert dans \( A\) si et seulement si il est ouvert dans \( X\), mais des choses se passent quand même. Prenons \( X=\eR\) et \( A=\mathopen] 0 , 1 \mathclose[\). Si \( B=\mathopen] \frac{ 1 }{2} , 1 \mathclose[ \), alors la fermeture de \( B\) dans \( A\) sera \( \tilde B=\mathopen[ \frac{ 1 }{2} , 1 [\) et non \( \mathopen[ \frac{ 1 }{2} , 1 \mathclose]\) comme on l'aurait dans \( \eR\).
\end{example}

%+++++++++++++++++++++++++++++++++++++++++++++++++++++++++++++++++++++++++++++++++++++++++++++++++++++++++++++++++++++++++++ 
\section{Compacité}
%+++++++++++++++++++++++++++++++++++++++++++++++++++++++++++++++++++++++++++++++++++++++++++++++++++++++++++++++++++++++++++

\begin{definition}
  Une partie $A$ d'un espace topologique est \defe{compacte}{compact} si il vérifie la propriété de Borel-Lebesgue : pour tout recouvrement de $A$ par des ouverts (c'est-à-dire une collection d'ouverts dont la réunion contient $A$) on peut tirer un recouvrement fini.
\end{definition}
\begin{remark}
    Certaines sources (dont \wikipedia{fr}{Compacité_(mathématiques)}{wikipédia}) disent que pour être compact il faut aussi être \defe{séparable}{séparable} (c'est à dire que deux points distincts ont des voisinages distincts). Pour ces sources, un espace qui ne vérifie que la propriété de Borel-Lebesgue est alors dit \defe{quasi-compact}{quasi-compact}\index{compact!quasi}.
\end{remark}

\begin{definition}
    Une famille \( \mA\) de parties de \( X\) a la \defe{propriété d'intersection finie non vide}{propriété d'intersection non vide} si tout sous-ensemble fini de \( \mA\) a une intersection non vide.
\end{definition}

\begin{proposition}\label{PropXKUMiCj}
    Soit \( X\) un espace topologique et \( K\subset X\). Les propriétés suivantes sont équivalentes :
    \begin{enumerate}
        \item\label{ItemXYmGHFai}
            \( K\) est compact.
        \item\label{ItemXYmGHFaii}
            Si \( \{ F_i \}\) est une famille de fermés telle que \( K\bigcap_{i\in I}F_i=\emptyset\), alors il existe un sous-ensemble fini \( A\) de \( I\) tel que \( K\bigcap_{i\in A}F_i=\emptyset\).
        \item\label{ItemXYmGHFaiii}
            Si \( \{ F_i \}_{i\in I}\) est une famille de fermés telle que \( K\bigcap_{i\in A}F_i\neq\emptyset\) pour tout choix de \( A\) fini dans \( I\), alors l'intersection complète est non vide : \( K\bigcap_{i\in I}F_i\neq\emptyset\).
        \item\label{ItemXYmGHFaiv}
            Toute famille ayant la propriété d'intersection finie non vide a une intersection non vide.
    \end{enumerate}
\end{proposition}

\begin{proof}
    Les propriétés \ref{ItemXYmGHFaiii} et \ref{ItemXYmGHFaii} sont équivalentes par contraposition. De plus le point \ref{ItemXYmGHFaiv} est une simple reformulation en français de la propriété \ref{ItemXYmGHFaiii}.

    Prouvons \ref{ItemXYmGHFai} \( \Rightarrow\) \ref{ItemXYmGHFaii}. Soit \( \{ F_i \}_{i\in I}\) une famille de fermés tels que \( K\bigcap_{i\in I}F_i=\emptyset\). Les complémentaires \( \mO_i\) de \( F_i\) dans \( X\) recouvrent \( K\) et donc on peut en extraire un sous-recouvrement fini :
    \begin{equation}
        K\subset\bigcup_{i\in A}\mO_i
    \end{equation}
    pour un certain sous-ensemble fini \( A\) de \( I\). Pour ce même choix \( A\), nous avons alors aussi
    \begin{equation}
        K\bigcap_{i\in A}F_i=\emptyset.
    \end{equation}

    L'implication \ref{ItemXYmGHFaii} \( \Rightarrow\) \ref{ItemXYmGHFai} est la même histoire.
\end{proof}


\begin{lemma}       \label{LemooynkH}
    Soit \( A_n\) une suite décroissante de fermés dans un compact \( K\). Alors
    \begin{equation}
        C=\bigcap_{n\in \eN}A_n
    \end{equation}
    est non vide. 
\end{lemma}

\begin{proof}
    Soit \( (x_n)\) une suite dans \( K\) telle que \( x_n\in A_n\). La suite étant contenue dans \( A_1\), elle possède une sous suite \( (y_n=x_{\sigma_1(n)})\) convergente dont la limite est dans \( A_1\). Une queue de la suite \( y_n\) est dans \( A_2\) et nous considérons donc une sous suite convergente dans \( A_2\) donnée par
    \begin{equation}
        z_n=y_{\sigma_2(n)}=x_{\sigma_1\sigma_2(n)}.
    \end{equation}
    En continuant ainsi nous construisons une suite convergente dans \( A_k\). Nous considérons enfin la suite
    \begin{equation}
        y_n=x_{\sigma_1\ldots \sigma_n(n)}.
    \end{equation}
    Pour tout \( k\), une queue de cette suite est une sous suite de \( x_{\sigma_1\ldots \sigma_k(n)}\) et par conséquent cette suite converge dans \( A_k\). La limite de cette suite est donc dans l'intersection demandée.
\end{proof}

\begin{remark}
    Cette propriété est fausse pour les ouverts. Par exemple
    \begin{equation}
        \bigcap_{n>1}\mathopen] 0 , \frac{1}{ n } \mathclose[=\emptyset.
    \end{equation}
\end{remark}


\begin{theorem}     \label{ThoImCompCotComp}
L'image d'un compact par une fonction continue est un compact
\end{theorem}

\begin{proof}
    Soit $K\subset X$, un ensemble compact, et regardons $f(K)$; en particulier, nous considérons $\Omega$, un recouvrement de $f(K)$ par des ouverts. Nous avons que
    \begin{equation}
        f(K)\subseteq\bigcup_{\mO\in\Omega}\mO.
    \end{equation}
    Par construction, nous avons aussi
    \begin{equation}
        K\subseteq\bigcup_{\mO\in\Omega}f^{-1}(\mO),
    \end{equation}
    en effet, si $x\in K$, alors $f(x)$ est dans un des ouverts de $\Omega$, disons $f(x)\in \mO_0$, et évidemment, $x\in f^{-1}(\mO)$.  Les $f^{-1}(\mO)$ recouvrent le compact $K$, et donc on peut en choisir un sous-recouvrement fini, c'est à dire un choix de $\{ f^{-1}(\mO_1),\ldots,f^{-1}(\mO_n) \}$ tels que
    \begin{equation}
        K\subseteq \bigcup_{i=1}^nf^{-1}(\mO_i).
    \end{equation}
    Dans ce cas, nous avons que
    \begin{equation}
        f(K)\subseteq\bigcup_{i=1}^n\mO_i,
    \end{equation}
    ce qui prouve la compacité de $f(K)$.
\end{proof}

\begin{theorem}[Tykhonov]\index{théorème!Tykhonov}\label{ThoFWXsQOZ}
    Un produit quelconque d'espaces métriques non vides est compact si et seulement si chacun de ses facteurs est compact.
\end{theorem}
Nous n'allons donner la preuve que dans le cas d'un produit dénombrable, dans le théorème \ref{ThoCDhbZbf}.

%+++++++++++++++++++++++++++++++++++++++++++++++++++++++++++++++++++++++++++++++++++++++++++++++++++++++++++++++++++++++++++ 
\section{Connexité}
%+++++++++++++++++++++++++++++++++++++++++++++++++++++++++++++++++++++++++++++++++++++++++++++++++++++++++++++++++++++++++++

Dès qu'un ensemble est muni d'une métrique, nous pouvons définir les boules ouvertes, les voisinages et les sous-ensembles ouverts. Dès que l'on a identifié les sous-ensemble ouverts de $E$, nous disons que $E$ devient un \defe{espace topologique}{espace topologique}. Nous allons maintenant un pas plus loin.

\begin{definition}
 Lorsque $E$ est un espace topologique, nous disons qu'un sous-ensemble $A$ est \defe{non connexe}{connexe} quand on peut trouver des ouverts $O_1$ et $O_2$ tels que
\begin{equation}    \label{EqDefnnCon}
  A=(A\cap O_1)\cup (A\cap O_2),
\end{equation}
et tels que $A\cap O_1\neq\emptyset$, et $A\cap O_2\neq\emptyset$.
Si un sous-ensemble n'est pas non-connexe, alors on dit qu'il est connexe.
\end{definition}
Une autre façon d'exprimer la condition \eqref{EqDefnnCon} est de dire que $A$ n'est pas connexe quand il est contenu dans la réunion de deux ouverts disjoints qui intersectent tous les deux $A$.

\begin{proposition}\label{PropGWMVzqb}
    L'image d'un ensemble connexe par une fonction continue est connexe.
\end{proposition}

\begin{proof}
    Soit \( f\colon X\to Y\) une application continue entre deux espaces topologiques, et \( E\) une partie connexe de \( X\). Nous devons montrer que \( f(E)\) est connexe dans \( Y\).

    Par l'absurde nous considérons \( A\) et \( B\), deux ouverts de \( Y\) disjoints recouvrant \( f(E)\). Étant donné que \( f\) est continue, les ensembles \( f^{-1}(A)\) et \( f^{-1}(B)\) sont ouverts dans \( X\). De plus ces deux ensembles recouvrent \( E\).

    Si \( x\) est un élément de \( f^{-1}(A)\cap f^{-1}(B)\), alors \( f(x)\in A\cap B\), ce qui est impossible parce que nous avons supposé que \( A\) et \( B\) étaient disjoints. Par conséquent \( f^{-1}(A)\) et \( f^{-1}(B)\) sont deux ouverts disjoints recouvrant \( E\). Contradiction avec la connexité de \( E\). Nous concluons que \( f(E)\) est connexe.
\end{proof}

%+++++++++++++++++++++++++++++++++++++++++++++++++++++++++++++++++++++++++++++++++++++++++++++++++++++++++++++++++++++++++++ 
\section{Un peu de topologie sur $\eR$}
%+++++++++++++++++++++++++++++++++++++++++++++++++++++++++++++++++++++++++++++++++++++++++++++++++++++++++++++++++++++++++++

Afin de pouvoir étudier la topologie des espaces métriques, il faut savoir quelque propriétés des réels parce que nous allons étudier la fonction distance qui est une fonction continue à valeurs dans les réels.

%--------------------------------------------------------------------------------------------------------------------------- 
\subsection{Critère de Cauchy}
%---------------------------------------------------------------------------------------------------------------------------

\begin{definition}
    Une suite \( (x_n)\) dans \( \eR\) est \defe{de Cauchy}{suite!de Cauchy}\index{Cauchy!suite} si pour tout \( \epsilon>0\), il existe \( N\in \eN\) tel que \( m,n>N\) implique \( | x_n-x_m |<\epsilon\).
\end{definition}

Toute suite de Cauchy est évidemment bornée et donc continue dans un compact.

\begin{proposition}
    Une suite dans \( \eR\) est convergente si et seulement si elle est de Cauchy.
\end{proposition}

\begin{proof}
    Nous commençons par prouver que toute suite de Cauchy est convergente. Nous considérons les ensembles fermés emboités
    \begin{equation}
        A_n=\overline{ x_i\tq i>n }.
    \end{equation}
    En vertu du lemme \ref{LemooynkH}
\end{proof}
<++>

%--------------------------------------------------------------------------------------------------------------------------- 
\subsection{Maximum, supremum et compagnie}
%---------------------------------------------------------------------------------------------------------------------------


Ce n'est un secret pour personne que $\eR$ est un \href{http://fr.wikipedia.org/wiki/Relation_d'ordre}{ensemble ordonné} : il y a des éléments plus grands que d'autres, et mieux : à chaque fois que je prends deux éléments différents dans $\eR$, il y en a un des deux qui est plus grand que l'autre. Il n'y a pas d'\emph{ex æquo} dans $\eR$.

  Si je regarde l'intervalle $I=[0,1]$, je peux même dire que $10$ est plus grand que tous les éléments de $I$. Nous disons que $10$ est un \emph{majorant} de $[0,1]$. La définition est la suivante.
\begin{definition}
Lorsque $A$ est un sous-ensemble de $\eR$, on dit que $s$ est un \defe{majorant}{majorant} de $A$ si $s$ est plus grand que tous les éléments de $A$. En d'autres termes, si
\[
  \forall x\in A,\,s\geq x.
\]
\end{definition}
Je me permet d'insister sur le fait que l'inégalité n'est pas stricte. Ainsi, $1$ est un majorant de $[0,1]$. Dès qu'un ensemble a un majorant, il en a plein. Si $s$ majore l'ensemble $A$, alors évident $s+1$, $s+4$, $s+\pi^2$ majorent également $A$.

\begin{example}
Une petite galerie d'exemples de majorants.
\begin{itemize}
\item L'intervalle fermé $[4,8]$ admet entre autres $8$ et $130$ comme majorants,
\item l'intervalle ouvert $]4,8[$ admet également $8$ et $130$ comme majorants,
\item $7$ n'est pas un majorant de $[1,5]\cup]8,32]$,
\item $10/10$ majore les côtes qu'on peut obtenir à une interrogation,
\item l'intervalle $[4,\infty[$ n'a pas de majorants.
\end{itemize}
\end{example}

\begin{definition}
Le \defe{supremum}{supremum} d'un ensemble est le plus petit majorant. En d'autres terme, $s$ est un supremum de $A$ si tout nombre plus petit que $s$ ne majore pas $A$, ou encore,
\[
  \forall x<s,\exists y\in A\text{ tel que } y>x.
\]
Nous disons que $M$ est un \defe{maximum}{maximum} de $A$ si $M$ est un supremum \emph{et} $M\in A$.
\end{definition}
Quand $s$ est un supremum de $A$, ça veut dire que le moindre pas vers la gauche que l'on fait à partir de $s$ (c'est à dire le moindre $\epsilon$), et on tombe dans $A$, ou tout au moins, il existe des éléments de $A$ qui sont plus grand que $s-\epsilon$.

\subsubsection{\ldots et quelque exemples}
%//////////////////////

En matières de notations, le maximum de l'ensemble $A$ est noté $\max A$, le supremum est noté $\sup A$. Le minimum et l'infimum sont notés $\min A$ et $\inf A$.

\begin{example}
Exemples de différence entre majorant, supremum et maximum.
\begin{itemize}
\item Le nombre $10$ est un supremum, majorant et maximum de l'intervalle fermé $[0,10]$,
\item Le nombre $10$ est un majorant et un supremum, mais pas un maximum de l'intervalle ouvert $]0,10[$,
\item Le nombre $136$ est un majorant, mais ni un maximum ni un supremum de l'intervalle $[0,10]$.
\end{itemize}
\end{example}

En utilisant les notations concises, ces différents cas s'écrivent ainsi :
\begin{align*}
10&=\max[0,10]=\sup[0,10]	& 10&=\sup[0,10[
\end{align*}


\begin{example}
Si on dit que un pont s'effondre à partir d'une charge de $10$ tonnes, alors $10$ tonnes est un \emph{supremum} des charges que le pont peut supporter : si on met $9,999999$ tonnes dessus, il tient encore le coup, mais si on ajoute un gramme, alors il s'effondre (on sort de l'ensemble des charges acceptables).
\end{example}

\begin{example}
Si on dit qu'un pont résiste jusqu'à $10$ tonnes, alors $10$ tonnes est un \emph{maximum} de la charge acceptable. Sur ce pont-ci, on peut ajouter le dernier gramme. Mais à partir de là, le moindre truc qu'on ajoute, il s'effondre.
\end{example}

Maintenant il est important de se rendre compte d'une chose : un ensemble ne peut avoir qu'un seul maximum et supremum. Jusqu'à présent nous avons toujours dit \emph{un} supremum. À partir de maintenant nous pouvons dire \emph{le} supremum. La preuve de cela est assez simple.
\begin{proposition}
Si $A$ est un sous-ensemble de $\eR$ admettant un supremum, alors il n'a qu'un seul supremum; et si il accepte un maximum, il n'en accepte un seul, et le maximum est égal au supremum.
\end{proposition}

\begin{proof}
Commençons par l'affirmation concernant le supremum. Supposons que $x$ et $y$ soient tous les deux suprema différents de $A$. Étant donné que $x\neq y$, nous pouvons supposer que $x<y$, et donc, par définition du fait que $y$ est un supremum, il existe un élément de $A$ qui est plus grand que $x$. Cela contredit le fait que $x$ soit supremum. En conclusion, il ne peut pas y avoir deux suprema différents pour un même ensemble.

Étant donné qu'un maximum est un supremum, il ne peut pas y avoir deux maxima différents vu qu'il ne peut pas y avoir deux suprema différents.
\end{proof}


Lorsque vous lisez que la charge maximale d'un camion est de \unit{2.5}{\ton}, est-ce que cela veut dire que vous pouvez y mettre \unit{2.5}{\ton}, mais qui si un oiseau se pose dessus, le camion s'effondre ? Ou bien est-ce que cela signifie qu'à \unit{2.5}{\ton} le camion s'écroule, mais que toute charge inférieure est valable ?

C'est à cette rude question que nous allons nous attaquer maintenant.

\begin{definition}
Soit une partie $A$ de $\eR$. Nous disons qu'un nombre $M$ est un \defe{majorant}{majorant} de $A$ si $M$ est plus grand ou égal que tous les éléments de $A$, c'est à dire si
\begin{equation}
	\forall a\in A,\, M\geq a.
\end{equation}
Un \defe{minorant}{minorant} de $A$ est un nombre $m$ tel que 
\begin{equation}
	\forall a\in A,\, m\leq a.
\end{equation}
\end{definition}

\begin{definition}		\label{DefSupeA}
Soit $A$ une partie majorée de $\eR$. Le \defe{supremum}{supremum} de $A$ est le plus petit des majorants, c'est à dire le nombre $M$ tel que
\begin{enumerate}
	\item
		$M\geq x$ pour tout $x\in A$,
	\item
		pour tout $\varepsilon$, le nombre $M-\varepsilon$ n'est pas un majorant de $a$, c'est à dire qu'il existe un élément $x\in A$ tel que $x>M-\varepsilon$.
\end{enumerate}
Nous notons $\sup A$ le supremum de $A$.

De la même façon, \defe{l'infimum}{infimum} de $A$, noté $\inf A$, est le plus grand de ses minorants. 
\end{definition}
Par convention, si la partie n'est pas bornée vers le haut, nous dirons que son supremum n'existe pas, ou bien qu'il est égal à $+\infty$, suivant les contextes. Pour votre culture générale, sachez toutefois que $\infty\notin\eR$.

La définition est justifiée par le lemme \ref{LemInfUnique} et la proposition \ref{PropBorneSupInf}. Le premier montre que si $A$ possède un infimum, alors il est unique, tandis que le second montre que toute partie majorée de $\eR$ accepter un supremum, et que toute partie minorée accepte un infimum.
\begin{lemma}		\label{LemInfUnique}
	Soit $A$ une partie de $\eR$. Supposons que $m_1$ et $m_2$ soient deux nombres qui vérifient les propriétés de l'infimum de $A$. Alors $m_1=m_2$.
\end{lemma}

\begin{proof}
	Si $_1\neq m_2$, nous pouvons supposer $m_2>m_1$. Dans ce cas, étant donné que $m_1$ est un infimum, $m_2$ ne peut pas minorer $A$, et donc ne peut pas être un infimum.
\end{proof}

\begin{proposition}		\label{PropBorneSupInf}
	Tout sous-ensemble de $\eR$ borné vers le bas possède un infimum; tout sous-ensemble de $\eR$ borné vers le haut possède un supremum.
\end{proposition}

La preuve qui suit est proche de celle donnée par Wikipédia  dans l'article \wikipedia{en}{http://en.wikipedia.org/wiki/Least_upper_bound_principle}{Least uppert bound principle}.

\begin{proof}
	Soit $A$, une partie de $\eR$. Nous allons trouver son infimum en suivant une méthode de dichotomie. Pour cela nous allons construire trois suites en même temps de la façon suivante. D'abord nous choisissons un point $x_0$ de $A$ et un point $x_1$ qui minore $A$ (qui existe par hypothèse) :
	\begin{equation}
		\begin{aligned}[]
			x_0&\text{ est un élément de $A$},\\
			x_1&\text{ est un minorant de $A$},\\
			a_0&=x_0\\
			b_0&=x_1\\
			b_1&=x_1.
		\end{aligned}
	\end{equation}
	Ensuite, nous faisons la récurrence suivante :
	\begin{equation}
		\begin{aligned}[]
			x_{n+1}&=\frac{ a_n+b_n }{2},\\
			a_{n+1}&=\begin{cases}
				a_{n}	&	\text{si $x_{n+1}$ minore $A$}\\
				x_{n+1}	&	 \text{sinon},
			\end{cases}\\
			b_{n+1}&=\begin{cases}
				x_{n+1}	&	\text{si $x_{n+1}$ minore $A$}\\
				b_n	&	 \text{sinon}.
			\end{cases}
		\end{aligned}
	\end{equation}
    Nous allons montrer que \( a_n\) et \( (b_n)\) sont des suites convergentes de même limite et que cette limite est l'infimum de \( A\).

	Soit $n\in\eN$; il y a deux possibilités. Soit $a_n=a_{n-1}$ et $b_n=x_n$, soit $a_n=x_n$ et $b_n=b_{n-1}$. Supposons que nous soyons dans le premier cas (le second se traite de façon similaire). Alors nous avons
	\begin{equation}
		\begin{aligned}[]
			| a_n-b_n |&=| a_{n-1}-x_n |\\
			&=\left| a_{n-1}-\frac{ a_{n-1}+b_{n-1} }{2} \right| \\
			&=\frac{ 1 }{2}| a_{n-1}-b_{n-1} |,
		\end{aligned}
	\end{equation}
	ce qui prouve que $| a_n-b_n |\to 0$. Nous montrons maintenant que la suite \( (a_n)\) est de Cauchy. En effet nous avons
    \begin{equation}
        | a_n-a_{n-1} |=\begin{cases}
          0\\
          \left| \frac{ a_n -b_n}{ 2} \right|   
      \end{cases}\leq \frac{1}{ 2n }.
    \end{equation}
    Il en est de même pour la suite \( (b_n)\). Ce sont deux suites de Cauchy (donc convergentes) qui convergent vers la même limite. Soit \( \ell\) cette limite.
    
	Le nombre $\ell$ minore $A$. En effet si $a\in A$ est plus petit que $\ell$, les éléments $b_n$ tels que $| b_n-\ell |<| a-\ell |$ ne peuvent pas minorer $A$. D'autre part, pour tout $\epsilon$, le nombre $\ell+\epsilon$ ne peut pas minorer $A$. En effet, $\ell$ est la limite de la suite décroissante $(a_n)$, donc il existe $a_n$ entre $\ell$ et $\ell+\epsilon$. Mais $a_n$ ne minore pas $A$, donc $\ell+\epsilon$ ne minore pas non plus $A$.

	Nous avons prouvé que toute partie minorée de $\eR$ possède un infimum. La preuve que toute partie majorée possède un supremum se fait de la même façon.
	
\end{proof}


\begin{definition}
	Si le supremum d'un ensemble appartient à l'ensemble, nous l'appelons \defe{maximum}{maximum}. De la même façon si l'infimum d'un ensemble appartient à l'ensemble, nous disons que c'est le \defe{minimum}{minimum}.
\end{definition}

\begin{example}
	Pour les intervalles, ces notions sont simples : les bornes de l'intervalle sont les supremum et infimum, et ce sont des minima et maxima si l'intervalle est fermé. Le nombre $53$ est un majorant.
	\begin{enumerate}
		\item
			$A=\mathopen[ 1 , 2 \mathclose]$. Tous les nombres plus petits ou égaux à $1$ sont minorants, $1$ est infimum et minimum. Le nombre $2$ est un majorant, le maximum et le supremum.
		\item
			$B=\mathopen] 3 , \pi \mathclose[$. Le nombre $\pi$ est le supremum et est un majorant, mais n'est pas le maximum (parce que $\pi\notin B$). L'ensemble $B$ n'a pas de maximum. Bien entendu, $-1000$ est un minorant.
	\end{enumerate}
\end{example}

Il existe évidement de nombreux exemples plus vicieux.

\begin{example}
	Prenons $E=\{ \frac{1}{ n }\tq n\in\eN_0 \}$, dont les premiers points sont indiqués sur la figure \ref{LabelFigSuiteUnSurn}. Cet ensemble est constitué des nombres $1$, $\frac{ 1 }{2}$, $\frac{1}{ 3 }$, \ldots Le plus grand d'entre eux est $1$ parce que tous les nombres de la forme $\frac{1}{ n }$ avec $n\geq 1$ sont plus petits ou égaux à $1$. Le nombre $1$ est donc maximum de $E$.

	L'ensemble $E$ n'a par contre pas de minimum parce que tout élément de $E$ s'écrit $\frac{1}{ n }$ pour un certain $n$ et est plus grand que $\frac{1}{ n+1 }$ qui est également dans $E$.

	Prouvons que zéro est l'infimum de $E$. D'abord, tous les éléments de $E$ sont strictement positifs, donc zéro est certainement un minorant de $E$. Ensuite, nous savons que pour tout $\varepsilon>0$, il existe un $n$ tel que $\frac{1}{ n }$ est plus petit que $\varepsilon$. L'ensemble $E$ possède donc un élément plus petit que $0+\varepsilon$, et zéro est bien l'infimum.
\end{example}

\newcommand{\CaptionFigSuiteUnSurn}{Les premiers points du type $x_n=1/n$.}
\input{Fig_SuiteUnSurn.pstricks}

L'exemple suivant est une source classique d'erreurs en ce qui concerne l'infimum. Il sera à relire après avoir vu la définition de limite (définition \ref{DefLimiteSuiteNum}).

\begin{example}
	Les premiers points de l'ensemble $F=\{ \frac{ (-1)^n }{ n }\tq n\in\eN_0 \}$ sont représentés à la figure \ref{LabelFigSuiteInverseAlterne}. Bien que (comme nous le verrons plus tard) la limite de la suite $x_n=(-1)^n/n$ soit zéro, il n'est pas correct de dire que zéro est l'infimum de l'ensemble $F$. Le dessin, au contraire, montre bien que $-1$ est le minium (aucun point est plus bas que $-1$), tandis que le maximum est $1/2$.

	Nous reviendrons avec cet exemple dans la suite. Pour l'instant, ayez bien en tête que zéro n'est rien de spécial pour l'ensemble $F$ en ce qui concerne les notions de maximum, minimum et compagnie.
\end{example}
\newcommand{\CaptionFigSuiteInverseAlterne}{Les quelque premiers points du type $(-1)^n/n$.}
\input{Fig_SuiteInverseAlterne.pstricks}

%+++++++++++++++++++++++++++++++++++++++++++++++++++++++++++++++++++++++++++++++++++++++++++++++++++++++++++++++++++++++++++ 
\section{Espaces métriques}
%+++++++++++++++++++++++++++++++++++++++++++++++++++++++++++++++++++++++++++++++++++++++++++++++++++++++++++++++++++++++++++

Si $E$ est un ensemble, une \defe{distance}{distance} sur $E$ est une application $d\colon E\times E\to \eR$ telle que pour tout $x,y\in E$,
\begin{enumerate}

\item
$d(x,y)\geq 0$

\item
$d(x,y)=0$ si et seulement si $x=y$,

\item
$d(x,y)=d(y,x)$

\item
$d(x,y)\leq d(x,z)+d(z,y)$.

\end{enumerate}
La dernière condition est l'\defe{inégalité triangulaire}{inégalité!triangulaire}. Le couple $(E,d)$ d'un ensemble et d'une métrique est un \defe{espace métrique}{espace!métrique}.

Dès que l'ensemble $E$ est muni d'une distance, nous définissons une topologie en disant que les boules
\begin{equation}
	B(x,r)=\{ y\in E\tq d(x,y)<r \}
\end{equation}
sont ouvertes.



Nous allons présenter maintenant les bases de la topologie sur des espaces métriques en prenant $\eR$ et $\eR^2$ comme exemple principaux. La topologie est un des fondements de la mathématique et est une prolongation de la théorie des ensembles. Nous n'en trouvons hélas pas beaucoup d'application directes en physique.

Si $E$ est un ensemble quelconque, nous disons qu'une \defe{distance}{distance} sur $E$ est une fonction $d\colon E\times E\to \eR^+$ telle que
\begin{description}
\item[Symétrie] $d(x,y)=d(y,x)$,
\item[Séparation] $d(x,y)=0$ ssi $x=y$. Insistons sur le fait que dans tous les cas, nous devons avoir $d(x,y)\geq 0$,
\item[Inégalité triangulaire] $d(x,z)\leq d(x,y)+d(y,z)$
\end{description}
pour tout $x$, $y$, $z\in E$. Un ensemble muni d'une loi de distance s'appelle un \href{http://fr.wikipedia.org/wiki/Espace_métrique}{espace métrique}.

Le premier exemple d'espace métrique que nous connaissons est $\eR$ muni de la distance usuelle ente deux nombres :
\begin{equation}
d(x,y)=| y-x |.
\end{equation}
Je me permet de faire remarquer la valeur absolue.

\begin{exercice}
Que penser de la formule $d(x,y)=y-x$ pour définir une distance sur $\eR$ ?
\end{exercice}

À partir de là, nous définissons la notion de \defe{boule ouverte}{boule!ouverte} sur l'ensemble $E$ centrée au point $x$ et de rayon $r>0$ comme
\[
  B(x,r)=\{ y\in\eR\tq d(x,y)< r \}.
\]
La \defe{boule fermée}{boule!fermée} centrée en $x$ et de rayon $r>0$ est définie par
\[
  \bar B(x,r)=\{ y\in\eR\tq d(x,y)\leq r \}.
\]
La différence est que dans la première l'inégalité est stricte.

\begin{theorem}     \label{ThoBoulOuvVois}
Une boule ouverte contient une boule ouverte autour de chacun de ses points.
\end{theorem}

\begin{proof}
Prenons $y\in B(x,r)$, et prouvons que la boule $B(y,r-d(x,y))$ est contenue dans $B(x,r)$. Première chose : $r-d(x,y)>0$ parce que $y$ est dans la boule ouverté centrée en $x$ et de rayon $r$. Pour prouver que  $B(y,r-d(x,y))\subset B(x,r)$, prenons un point dans le premier ensemble et montrons qu'il est dans le second ensemble.

Soit donc $z\in B\big(y,r-d(x,y)\big)$ et testons $d(x,z)$ que nous voudrions être plus petit que~$r$. Et, miracle, il l'est parce que
\begin{align*}
  d(x,z)    &\leq d(x,y)+d(y,z)&\text{inégalité triangulaire}\\
        &<d(x,y)+\big(r-d(x,y)\big)&\text{$z\in B\big(y,r-d(x,y)\big)$}\\
        &=r.
\end{align*}
Remarquez que la première inégalité n'est pas stricte, tandis que la seconde est stricte. Nous avons donc bien $d(x,z)<r$ (strictement) comme le demandé pour que $z$ soit dans la boule \emph{ouverte} de centre $x$ et de rayon $r$.
\end{proof}

\begin{theorem}[Théorème \wikipedia{fr}{Théorème_des_fermés_emboités}{des fermés emboîtés}\cite{OIywOjl}]
    Soit \( (E,d)\) un espace métrique. Il est complet si et seulement si toute suite décroissante de fermés non vides dont le diamètre tend vers zéro a une intersection qui se réduit à un seul point.
\end{theorem}

\begin{proof}
    \begin{subproof}
    \item[Condition suffisante]

        Soit \( \{ F_n \}_{n\in \eN}\) une telle suite de fermés emboités. Si nous choisissons des points \( x_n\in F_n\), nous obtenons une suite \( (x_n)\) de Cauchy et qui est par conséquent convergente vu que l'espace est par hypothèse complet. De plus, pour chaque \( N\geq n\), la queue de suite \( (x_n)_{n\geq N}\) est contenue dans \( F_N\) et donc converge vers un élément de \( F_N\) (parce que ce dernier est fermé). Donc la limite de \( (x_n)\) est dans \( \bigcap_{n\in \eN}F_n\).

        De plus cette intersection a diamètre nul parce que le diamètre de \( \bigcap_{n\in \eN}F_n\) est majoré par tous les diamètres des \( F_n\), lesquels sont arbitrairement petits par hypothèse. Donc l'intersection est réduite a un point.

    \item[Condition nécessaire]

        Soit une suite de Cauchy \( (x_n)\). Nous considérons les ensembles
        \begin{equation}
            F_n=\overline{ \{ x_i\tq i\geq n \} }.
        \end{equation}
        Le fait que la suite soit de Cauchy implique que \( \diam(F_n)\to 0\). Par hypothèse, nous avons alors
        \begin{equation}
            \bigcap_{n\in \eN}F_n=\{ a \}.
        \end{equation}
        Pour s'assurer que \( a\) est bien la limite de \( (x_n)\), il suffit de remarquer que
        \begin{equation}
            d(x_n,a)\leq \diam F_n\to 0.
        \end{equation}
    \end{subproof}
\end{proof}

%--------------------------------------------------------------------------------------------------------------------------- 
\subsection{Compacité}
%---------------------------------------------------------------------------------------------------------------------------

\begin{lemma}[de Lebesgue\cite{JBRzHwn}]    \label{LemQFXOWyx}
    Soit \( (X,d)\) un espace métrique tel que toute suite ait une sous-suite convergente à l'intérieur de l'espace. Si \( \{ V_i \}\) est un recouvrement par des ouverts de \( X\), alors il existe \( \epsilon\) tel que pour tout \( x\in X\), nous ayons \( B(x,\epsilon)\subset V_i\) pour un certain \( i\).
\end{lemma}

\begin{proof}
    Par l'absurde, nous supposons que pour tout \( n\), il existe un \( x_n\in X\) tel que la boule \( B(x_n,\frac{1}{ n })\) n'est contenue dans aucun des \( V_i\). Ce des \( x_n\) nous extrayons une sous-suite convergente (que nous nommons encore \( (x_n)\)) et nous posons \( x_n\to x\). Pour \( n\) assez grand (\( \frac{1}{ n }<\epsilon\)) nous avons \( x_n\in B(x,\epsilon)\), donc tous les \( x_n\) suivants sont dans le \( V_i\) qui contient \( x\).
\end{proof}

\begin{lemma}[\cite{JBRzHwn}]   \label{LemMGQqgDG}
    Soit \( (X,d)\) un espace métrique tel que toute suite possède une sous-suite convergente. Pour tout \( \epsilon>0\), il existe un ensemble fini \( \{ x_i \}_{i\in I}\) tel que les boules \( B(x_i,\epsilon)\) recouvrent \( X\).
\end{lemma}

\begin{proof}
    Soit par l'absurde un \( \epsilon>0\) contredisant le lemme. Il n'existe pas d'ensemble finis autour des points duquel les boules de taille \( \epsilon\) recouvrent \( X\).

    Nous construisons par récurrence une suite ne possédant pas de sous-suites convergente. Le premier terme, \( x_0\) est pris arbitrairement dans \( X\). Ensuite si nous en avons \( N\) termes, nous savons que les boules de rayon \( \epsilon\) et centrées en les points \( \{ x_i \}_{i=1,\ldots, N}\) ne recouvrent pas \( X\). Donc nous prenons \( x_{N+1}\) hors de l'union de ces boules.

    Ainsi nous avons une suite \( (x_n)\) dont tous les termes sont à distance plus grande que \( \epsilon\) les uns des autres. Une telle suite ne peut pas contenir de sous-suite convergente. Contradiction.
\end{proof}

\begin{theorem}[Bolzano-Weierstrass\cite{JBRzHwn}]\index{théorème!Bolzano-Weierstrass}\index{compacité}\label{ThoBWFTXAZNH}
    Un espace métrique est compact si et seulement si toute suite admet une sous-suite qui converge à l'intérieur de l'espace.
\end{theorem}

\begin{proof}
   Soit \( X\) un espace métrique compact et \( (x_n)\) une suite dans \( X\). Nous considérons la suite de fermés emboités
   \begin{equation}
       X_n=\overline{ \{ x_k\tq k>n \} }.
   \end{equation}
   Ce sont des fermés ayant la propriété d'intersection finie non vide, et donc la proposition \ref{PropXKUMiCj} nous dit qu'ils ont une intersection non vide. Un élément de cette intersection est automatiquement un point d'accumulation de la suite.

   Nous passons à l'autre sens. Nous supposons que toute suite dans \( X\) contient une sous-suite convergente, et nous considérons \( \{ V_i \}_{i\in I}\), un recouvrement de \( X\) par des ouverts. Par le lemme \ref{LemQFXOWyx}, nous considérons un \( \epsilon\) tel que pour tout \( x\), il existe un \( i\in I\) avec \( B(x,\epsilon)\subset V_i\). Par le lemme \ref{LemMGQqgDG}, nous considérons un ensemble fini \( \{ y_i \}_{i\in A}\) tel que le boules \( B(y_i,\epsilon)\) recouvrent \( X\).

   Par construction, chacune de ces boules \( B(y_i,\epsilon)\) est contenue dans un des ouverts \( V_i\). Nous sélectionnons donc parmi les \( V_i\) le nombre fini qu'il faut pour recouvrir les \( B(y_i,\epsilon)\) et donc pour recouvrir \( X\).
\end{proof}

Le théorème de Bolzano–Weierstrass \ref{ThoBWFTXAZNH} a l'importante conséquence suivante.
\begin{theorem}[Weierstrass]		\label{ThoWeirstrassRn}
	Une fonction continue à valeurs réelles définie sur un compact est bornée et atteint ses bornes.
\end{theorem}

\begin{proof}
	Soit $K$ une partie compacte et $f\colon K\to \eR$ une fonction continue. Nous désignons par $A$ l'ensemble des valeurs prises par $f$ sur $K$ :
	\begin{equation}
		A=f(K)=\{ f(x)\tq x\in K \}.
	\end{equation}
	Nous considérons le supremum $M=\sup A=\sup_{x\in K}f(x)$ avec la convention comme quoi si $A$ n'est pas borné supérieurement, nous posons $M=\infty$ (voir définition \ref{DefSupeA}).

	Nous allons maintenant construire une suite $(x_n)$ de deux façons différentes suivant que $M=\infty$ ou non.
	\begin{enumerate}
		\item
			Si $M=\infty$, nous choisissons, pour chaque $n\in\eN$, un $x_n\in K$ tel que $f(x_n)>n$. Cela est certainement possible parce que si $A$ n'est pas borné, nous pouvons y trouver des nombres aussi grands que nous voulons.
		\item
			Si $M<\infty$, nous savons que pour tout $\varepsilon$, il existe un $y\in A$ tel que $y>M-\varepsilon$. Pour chaque $n$, nous choisissons donc $x_n\in K$ tel que $f(x_n)>M-\frac{1}{ n }$.
	\end{enumerate}
    Quel que soit le cas dans lequel nous sommes, la suite $(x_n)$ est une suite dans $K$ qui est compact, et donc nous pouvons en extraire une sous-suite convergente à l'intérieur de \( K\) par le théorème de Bolzano-Weierstrass\ref{ThoBWFTXAZNH}. Afin d'alléger la notation, nous allons noter $(x_n)$ la sous-suite convergente. Nous avons donc 
	\begin{equation}
		x_n\to x\in K.
	\end{equation}
	Par la proposition \ref{PropFnContParSuite}, nous avons que $f$ prend en \( x\) la valeur
	\begin{equation}
		f(x)=\lim_{n\to \infty} f(x_n).
	\end{equation}
	Donc $f(x)<\infty$. Évidement, si nous avions été dans le cas où $M=\infty$, la suite $x_n$ aurait été choisie pour avoir $f(x_n)>n$ et donc il n'aurait pas été possible d'avoir $\lim_{n\to \infty} f(x_n)<\infty$. Nous en concluons que $M<\infty$, et donc que $f$ est bornée sur $K$.

	Afin de prouver que $f$ atteint sa borne, c'est à dire que $M\in A$, nous considérons les inégalités
	\begin{equation}
		M-\frac{1}{ n }<f(x_n)\leq M.
	\end{equation}
	En passant à la limite $n\to \infty$, ces inégalités deviennent
	\begin{equation}
		M\leq f(x)\leq M,
	\end{equation}
	et donc $f(x)=M$, ce qui prouve que $f$ atteint sa borne $M$ au point $x\in K$.
\end{proof}

\begin{lemma}   \label{LemnAeACf}
    Si \( K\) est compact et si \( F\) est fermé dans \( K\), alors \( F\) est compact.
\end{lemma}

\begin{proof}
    Nous allons utiliser la caractérisation de la compacité en termes de suites donnée par le théorème de Bolzano-Weierstrass \ref{ThoBWFTXAZNH}. Soit \( (x_n)\) une suite dans \( F\); par la compacité de \( K\) nous pouvons considérer une sous suite \( (y_n)\) qui converge dans \( K\) (proposition \ref{ThoBWFTXAZNH}). Étant donné que \( (y_n)\) est une suite convergente contenue dans \( F\) et étant donné que \( F\) est fermé, la limite est dans \( F\), ce qui prouve que \( (x_n)\) possède une sous suite convergente dans $F$ et par conséquent que \( F\) est compact.
\end{proof}

\begin{lemma}   \label{LemKIcAbic}
    Si \( K\) est un compact et \( F\) un fermé disjoint de \( K\), alors \( d(K,F)>0\).
\end{lemma}

\begin{proof}
    Le fonction 
    \begin{equation}
        \begin{aligned}
             K&\to \eR \\
            x&\mapsto d(x,F) 
        \end{aligned}
    \end{equation}
    est une fonction continue sur \( K\), et donc atteint son minimum par le théorème de Weierstrass \ref{ThoWeirstrassRn}. Soit \( x_0\in K\) un point de \( K\) qui réalise ce minimum. Si \( d(x_0,F)=0\), alors on aurait une suite \( (x_n)\) dans \( F\) qui convergerait vers \( x_0\), mais \( F\) étant fermé cela signifierait que \( x_0\) serait dans \( F\), ce qui contredirait l'hypothèse que \( F\) et \( K\) sont disjoints.
\end{proof}

\begin{proposition}[\cite{JBRzHwn}]
    une isométrie d'un espace métrique compact sur lui-même est une bijection.
\end{proposition}

\begin{proof}
    Soit \( X\) un espace métrique compact et \( f\colon X\to X\) une isométrie. Le fait que \( f\) soit injective est obligatoire (sinon il y a des images dont la distance est nulle). Il faut montrer que \( f\) est surjective.

    Soit \( x\in X\) hors de \( f(X)\). Le lemme \ref{LemKIcAbic} appliqué au fermé \( \{ x \}\) et au compact \( f(K)\) donne un \( r>0\) tel que
    \begin{equation}
        d\big( x,f(K)\big)>r.
    \end{equation}
    Soit la suite \( u_n=f^n(x)\); c'est une suite dans \( K\) et possède donc une sous-suite convergente (Bolzano-Weierstrass\ref{ThoBWFTXAZNH}) que l'on nomme \( (y_n)\). Vu que \( f\) est une isométrie,
    \begin{equation}
        d(y_{n},y_{n+1})=d(x,y_m)>r
    \end{equation}
    pour un certain \( m\leq n+1\). Ce la signifie que pour tout \( n\), nous avons \( d(y_n,y_{n+1})>r\), ce qui contredit le fait que la suite \( (y_n)\) converge.
\end{proof}

\begin{proposition} \label{PropLHWACDU}
    Soit \( (X,d)\) un espace métrique compact et \( (u_n)\) une suite de \( X\) telle que
    \begin{equation}
        \lim_{n\to \infty} d(u_n,u_{n+1})=0.
    \end{equation}
    Alors l'ensemble des valeurs d'adhérence de \( (u_n)\) est connexe.
\end{proposition}
<++>

%--------------------------------------------------------------------------------------------------------------------------- 
\subsection{Produit dénombrables d'espaces métriques}
%---------------------------------------------------------------------------------------------------------------------------

\begin{definition}
    Soient \( (E_n,d_n)\) des espaces métriques. Sur l'ensemble produit \( E=\prod_{i=1}^{\infty}E_i\) nous définissons la métrique
    \begin{equation}
        d(x,y)=\sum_{i=1}^{\infty}\frac{1}{ 2^i }d'_n(x_i,y_i)
    \end{equation}
    où \( d'_i=\min(d_i,1)\).
\end{definition}
On peut montrer que ce \( d\) est bien une distance et que \( (E,d)\) est alors bien un espace métrique.
%TODO: le faire.

\begin{theorem}\label{ThoCDhbZbf}
    Un produit dénombrable d'espaces métriques non vides est compact si et seulement si chacun de ses facteurs est compact.
\end{theorem}
Note : ce résultat est encore valable pour un produit quelconque, c'est le théorème de Tykhonov \ref{ThoFWXsQOZ}.
