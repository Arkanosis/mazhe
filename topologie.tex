% This is part of Mes notes de mathématique
% Copyright (c) 2012
%   Laurent Claessens
% See the file fdl-1.3.txt for copying conditions.

%+++++++++++++++++++++++++++++++++++++++++++++++++++++++++++++++++++++++++++++++++++++++++++++++++++++++++++++++++++++++++++
					\section{Topologie en général}
%+++++++++++++++++++++++++++++++++++++++++++++++++++++++++++++++++++++++++++++++++++++++++++++++++++++++++++++++++++++++++++

\begin{definition}		\label{DefTopologieGene}
Soit $E$, un ensemble et $\mT$, une partie de l'ensemble de ses parties qui vérifie les propriétés suivantes
\begin{enumerate}

\item
les ensembles $\emptyset$ et $E$ sont dans $\mT$,

\item
Si $I$ est n'importe quel ensemble et si pour tout $i\in I$, nous avons un élément $\mO_i\in\mT$, alors $\cup_{i\in I}\mO_i\in\mT$,

\item
Si $J$ est un ensemble fini et si pour tout $j\in J$, nous avons un élément $\mO_j\in\mT$, alors $\cap_{j\in J}\mO_j\in\mT$.

\end{enumerate}
Les deux dernières propriétés s'énoncent en disant que toute réunions d'éléments de $\mT$ est un élément de $\mT$ et que toute intersection \emph{finie} d'éléments de $\mT$ est un élément de $\mT$.

Un tel choix $\mT$ de sous-ensembles de $E$ est une  \defe{\href{http://fr.wikipedia.org/wiki/Espace_topologique}{topologie}}{topologie} sur $E$, et les éléments de $\mT$ sont appelés des \defe{ouverts}{ouvert}. Nous disons que un sous ensemble $A$ de $E$ est \defe{fermé}{fermé} si son complémentaire, $A^c$ est ouvert.
\end{definition}

Dès que nous avons une topologie, nous avons une notion de convergence de suite : nous disons qu'une suite $x_n$ d'éléments de $E$ \defe{converge}{convergence!en topologie} vers l'élément $x$ de $E$ si pour tout ouvert $\mO$ contenant $x$, il existe un $K$ tel que $k>K$ implique $x_k\in\mO$. Cette définition est exactement celle donnée pour la convergence de suites dans $\eR^n$, à part que nous avons remplacé le mot \og boule\fg{} par \og ouvert\fg.

Dans un espace topologique, nous avons une caractérisation très importante des ouverts.
\begin{theorem}		\label{ThoPartieOUvpartouv}
Une partie $A$ de $E$ est ouverte si et seulement si pour tout $x\in A$, il existe un ouvert autour de $x$ contenu dans $A$.
\end{theorem}

\begin{proof}
Le sens direct est évident : $A$ lui-même est un ouvert autour de $x\in A$, qui est inclus à $A$.

Pour le sens inverse, pour chaque $x\in A$, nous considérons l'ensemble $\mO_x\subset A$, un ouvert autour de $x$. Nous avons que
\begin{equation}	\label{EqAUniondesOx}
	A=\bigcup_{x\in A}\mO_x.
\end{equation}
En effet $A\subset\cup_{x\in A}\mO_x$ parce que tous les éléments de $A$ sont dans un des $\mO_x$, par construction. D'autre part, $\cup_{x\in A}\mO_x\subset A$ parce que chacun des $\mO_x$ est compris dans $A$.

L'union du membre de droite de \eqref{EqAUniondesOx} est une union d'ouverts et est donc un ouvert. Cela prouve que $A$ est un ouvert.

\end{proof}

Une utilisation typique de ce théorème est faite à l'exercice \ref{exo0083}.

%%%%%%%%%%%%%%%%%%%%%%%%%%
%
   \section{Topologie dans \texorpdfstring{$\eR^n$}{Rn}}
%
%%%%%%%%%%%%%%%%%%%%%%%%



Dans cette section, nous travaillons dans l'espace $\eR^n$ pour un certain naturel $n$. Nous y définissons la notion d'ouvert et de fermé, qui sont la base de la topologie générale. Notons que ces définitions n'ont de sens que relativement à l'espace ambiant, aussi un ouvert de $\eR$ ne sera en général pas un ouvert de $\eR^2$~: d'une part, il n'y a pas d'inclusion canonique de $\eR$ dans $\eR^2$ (les ouverts du second ne sont même pas des sous-ensembles du premier) et, d'autre part, les définitions se basent sur la notion de boule de $\eR^n$ qui dépend évidemment de la valeur de $n$ (une boule dans $\eR$ est un intervalle, dans $\eR^2$ c'est un disque, etc.)

%---------------------------------------------------------------------------------------------------------------------------
					\subsection{Ouverts et fermés}
%---------------------------------------------------------------------------------------------------------------------------

\begin{definition}
	La \defe{boule ouverte}{Boule!ouverte} de centre $x_0 \in \eR^n$ et de rayon $r \in
	\eR^+$ est définie par
	\begin{equation}
		B(x_0,r) = \{ x \in \eR^n \tq \norme{x - x_0} < r \},
	\end{equation}
	tandis que la \defe{boule fermée}{Boule!fermée} de centre $x_0$ et de rayon $r$ est
	\begin{equation}
		\bar B(x_0,r) = \{ x \in \eR^n \tq \norme{x - x_0} \leq r \};
	\end{equation}
	la différence est que l'inégalité dans la première est stricte.
\end{definition}

%---------------------------------------------------------------------------------------------------------------------------
					\subsection{Intérieur, adhérence et frontière}
%---------------------------------------------------------------------------------------------------------------------------

\begin{definition}
  Soit $A \subset \eR^n$ et $x \in \eR^n$. Le point $x$ est \defe{intérieur}{intérieur} à $A$ si il existe une boule autour de $x$ complètement contenue dans $A$. L'ensemble des points intérieurs à $A$ est noté $\interieur A$ ou $\mathring A$, de sorte qu'on a précisément
  \begin{equation*}
    x \in \interieur A \iffdefn  \exists \epsilon > 0 \tq
    B(x,\epsilon) \subset A.
  \end{equation*}
\end{definition}


\begin{definition}
Le point $x$ est dans l'\defe{adhérence}{adhérence} de $A$ si toute boule autour de $x$ intersecte $A$. L'ensemble de ces points est noté $\adh A$ ou $\bar A$, et on a donc de manière plus précise
\begin{equation}
	x \in \adh A \iffdefn \forall \epsilon > 0, B(x,\epsilon) \cap A \neq \emptyset
\end{equation}
\end{definition}

\begin{proposition}
Pour $A \subset \eR^n$, nous avons
\begin{equation*}
	\interieur A \subseteq A  \subseteq \adh A
\end{equation*}
\end{proposition}

\begin{definition}
  La \defe{frontière}{frontière} ou le \defe{bord}{bord} de $A$ est défini par $\partial A = \adh A \setminus \interieur A$. L'ensemble $A$ est un \defe{ouvert}{ouvert} si $A = \interieur A$, et c'est un \defe{fermé}{fermé} si $A = \adh A$.
\end{definition}

On vérifiera que les notations et les dénominations sont cohérentes en
prouvant la proposition suivante.
\begin{proposition}Pour $\epsilon > 0$,
  \begin{enumerate}
  \item l'adhérence de $B(x,\epsilon)$ est $\bar B(x,\epsilon)$,
  \item l'intérieur de $\bar B(x,\epsilon)$ est $B(x,\epsilon)$,
  \item la boule ouverte $B(x,\epsilon)$ est un ouvert,
  \item la boule fermée $\bar B(x,\epsilon)$ est un fermé.
  \end{enumerate}
\end{proposition}

Nous avons également les liens suivants entre intérieur, adhérence,
ouvert, fermé et passage au complémentaire (noté ${}^c$)~:
\begin{proposition}
Si $A \subset \eR^n$ et $A^c = \eR^n\setminus A$, nous
  avons
  \begin{enumerate}
  \item $(\interieur A)^c = \adh (A^c)$ et $(\adh A)^c = \interieur
    (A^c)$,
  \item $A$ est ouvert si et seulement si $A^c$ est fermé,
  \item $\interieur A$ est le plus grand ouvert contenu dans $A$,
  \item $\adh A$ est le plus petit fermé contenant $A$,
    % \item
  \end{enumerate}
\end{proposition}

%---------------------------------------------------------------------------------------------------------------------------
					\subsection{Bornés et compacts}
%---------------------------------------------------------------------------------------------------------------------------


\begin{definition}
  Un sous ensemble $A \subset \eR^n$ est \defe{borné}{borné} si il existe une boule de $\eR^n$ contenant $A$.
\end{definition}

\begin{proposition}
  Toute réunion finie d'ensembles bornés est un ensemble borné. Toute
  partie d'un ensemble borné est un ensemble borné.
\end{proposition}

\begin{definition}
  La partie $A \subset \eR^n$ est \defe{compacte}{compact} si et seulement si, pour tout
  recouvrement de $A$ par des ouverts (c'est-à-dire une collection
  d'ouverts dont la réunion contient $A$) on peut tirer un
  recouvrement fini.
\end{definition}

% En particulier, si on recouvre $A$ par l'ensemble des boules
% $B(x,1)$ où $x$ parcourt $A$ (de sorte que tout point de $A$ est
% dans \og sa\fg{} boule, et donc la réunion des boules recouvre bien
% $A$), on doit pouvoir en tirer un recouvrement fini, c'est-à-dire
% des boules $B(x_1,1), B(x_2,1), \ldots, B(x_k,1)$ (avec $k$ un
% naturel) dont la réunion contient $A$.

\begin{proposition}
Une partie de $\eR^n$ est compacte si et seulement si elle est fermée et bornée.
\end{proposition}

%---------------------------------------------------------------------------------------------------------------------------
					\subsection{Connexité}
%---------------------------------------------------------------------------------------------------------------------------

\begin{definition}
  Le sous ensemble $A \subset \eR^n$ est \defe{connexe par arcs}{Connexe!par arc} si pour tout $x, y \in
  A$, il existe un chemin\footnote{Attention : ici quand on dit \emph{chemin}, on demande que l'application soit continue. Dans de nombreux cours de géométrie différentielle, on demande $ C^{\infty}$. Il faut s'adapter au contexte.} contenu dans $A$ les reliant, c'est-à-dire
  une application continue
  \begin{equation*}
    \gamma : [0,1] \to \eR^n \tq \gamma(0) = x~\text{et}~\gamma(1) = y
  \end{equation*}
  avec $\gamma(t) \in A$ pour tout $t\in [0,1]$.
\end{definition}

%+++++++++++++++++++++++++++++++++++++++++++++++++++++++++++++++++++++++++++++++++++++++++++++++++++++++++++++++++++++++++++
					\section{Topologie des espaces métriques}
%+++++++++++++++++++++++++++++++++++++++++++++++++++++++++++++++++++++++++++++++++++++++++++++++++++++++++++++++++++++++++++

Si $E$ est un ensemble, une \defe{distance}{distance} sur $E$ est une application $d\colon E\times E\to \eR$ telle que pour tout $x,y\in E$,
\begin{enumerate}

\item
$d(x,y)\geq 0$

\item
$d(x,y)=0$ si et seulement si $x=y$,

\item
$d(x,y)=d(y,x)$

\item
$d(x,y)\leq d(x,z)+d(z,y)$.

\end{enumerate}
La dernière condition est l'\defe{inégalité triangulaire}{Inégalité!triangulaire}. Le couple $(E,d)$ d'un ensemble et d'une métrique est un \defe{espace métrique}{espace!métrique}.

Dès que l'ensemble $E$ est muni d'une distance, nous définissons une topologie en disant que les boules
\begin{equation}
	B(x,r)=\{ y\in E\tq d(x,y)<r \}
\end{equation}
sont ouvertes.

\begin{proposition} \label{PropvvSKiE}
    Soit \( E\), un espace métrique et \( K\subset E\). L'ensemble \( K\) est compact si et seulement si toute suite dans \( K\) contient une sous suite convergente dans \( K\).
\end{proposition}

\begin{lemma}   \label{LemnAeACf}
    Si \( K\) est compact et si \( F\) est fermé dans \( K\), alors \( F\) est compact.
\end{lemma}

\begin{proof}
    Nous allons utiliser la caractérisation \ref{PropvvSKiE}. Soit \( (x_n)\) une suite dans \( F\); par la compacité de \( K\) nous pouvons considérer une sous suite \( (y_n)\) qui converge dans \( K\) (proposition \ref{PropvvSKiE}). Étant donné que \( (y_n)\) est une suite convergente contenue dans \( F\) et étant donné que \( F\) est fermé, la limite est dans \( F\), ce qui prouve que \( (x_n)\) possède une sous suite convergente dans $F$ et par conséquent que \( F\) est compact.
\end{proof}

\begin{lemma}       \label{LemooynkH}
    Soit \( A_n\) une suite décroissante de fermés dans un compact \( K\). Alors
    \begin{equation}
        C=\bigcap_{n\in \eN}A_n
    \end{equation}
    est non vide.
\end{lemma}

\begin{proof}
    Soit \( x_n\) une suite dans \( K\) telle que \( x_n\in A_n\). La suite étant contenue dans \( A_1\), elle possède une sous suite \( (y_n=x_{\sigma_1(n)})\) convergente dont la limite est dans \( A_1\). Une queue de la suite \( y_n\) est dans \( A_2\) et nous considérons donc une sous suite convergente dans \( A_2\) donnée par
    \begin{equation}
        z_n=y_{\sigma_2(n)}=x_{\sigma_1\sigma_2(n)}.
    \end{equation}
    En continuant ainsi nous construisons une suite convergente dans \( A_k\). Nous considérons enfin la suite
    \begin{equation}
        y_n=x_{\sigma_1\ldots \sigma_n(n)}.
    \end{equation}
    Pour tout \( k\), une queue de cette suite est une sous suite de \( x_{\sigma_1\ldots \sigma_k(n)}\) et par conséquent cette suite converge dans \( A_k\). La limite de cette suite est donc dans l'intersection demandée.
\end{proof}

\begin{remark}
    Cette propriété est fausse pour les ouverts. Par exemple
    \begin{equation}
        \bigcap_{n>1}\mathopen] 0 , \frac{1}{ n } \mathclose[=\emptyset.
    \end{equation}
\end{remark}

%+++++++++++++++++++++++++++++++++++++++++++++++++++++++++++++++++++++++++++++++++++++++++++++++++++++++++++++++++++++++++++
					\section{Uniforme continuité}
%+++++++++++++++++++++++++++++++++++++++++++++++++++++++++++++++++++++++++++++++++++++++++++++++++++++++++++++++++++++++++++

\begin{proposition}	\label{PropoInvCompCont}
Soit $f\colon A\subset\eR^n\to B\subset\eR^m$ une bijection continue. Si $A$ est compact, alors $f^{-1}\colon B\to A$ est continue.
\end{proposition}

\begin{proposition}		\label{PropIntContMOnIvCont}
Soient $I$ un intervalle dans $\eR$ et $f\colon I\to \eR$ une fonction continue strictement monotone. Alors la fonction réciproque $f^{-1}\colon f(I)\to \eR$ est continue sur l'intervalle $f(I)$.
\end{proposition}

%---------------------------------------------------------------------------------------------------------------------------
\subsection{Espaces d'opérateurs}
%---------------------------------------------------------------------------------------------------------------------------

Soit \( E\), un espace vectoriel. La \defe{topologie \( *\)-faible}{topologie!$*$-faible} sur l'ensemble des opérateurs \( E\to E\) est la topologie de la convergence \( T_n\to T\) si et seulement si \( T_nv\to Tv\) pour tout \( v\in E\).

