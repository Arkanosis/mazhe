% This is part of Mes notes de mathématique
% Copyright (c) 2012
%   Laurent Claessens, Carlotta Donadello
% See the file fdl-1.3.txt for copying conditions.

%+++++++++++++++++++++++++++++++++++++++++++++++++++++++++++++++++++++++++++++++++++++++++++++++++++++++++++++++++++++++++++
					\section{Topologie en général}
%+++++++++++++++++++++++++++++++++++++++++++++++++++++++++++++++++++++++++++++++++++++++++++++++++++++++++++++++++++++++++++

\begin{definition}		\label{DefTopologieGene}
Soit $E$, un ensemble et $\mT$, une partie de l'ensemble de ses parties qui vérifie les propriétés suivantes
\begin{enumerate}

\item
les ensembles $\emptyset$ et $E$ sont dans $\mT$,

\item
Si $I$ est n'importe quel ensemble et si pour tout $i\in I$, nous avons un élément $\mO_i\in\mT$, alors $\cup_{i\in I}\mO_i\in\mT$,

\item
Si $J$ est un ensemble fini et si pour tout $j\in J$, nous avons un élément $\mO_j\in\mT$, alors $\cap_{j\in J}\mO_j\in\mT$.

\end{enumerate}
Les deux dernières propriétés s'énoncent en disant que toute réunions d'éléments de $\mT$ est un élément de $\mT$ et que toute intersection \emph{finie} d'éléments de $\mT$ est un élément de $\mT$.

Un tel choix $\mT$ de sous-ensembles de $E$ est une  \defe{\href{http://fr.wikipedia.org/wiki/Espace_topologique}{topologie}}{topologie} sur $E$, et les éléments de $\mT$ sont appelés des \defe{ouverts}{ouvert}. Nous disons que un sous ensemble $A$ de $E$ est \defe{fermé}{fermé} si son complémentaire, $A^c$ est ouvert.
\end{definition}

Dès que nous avons une topologie, nous avons une notion de convergence de suite : nous disons qu'une suite $x_n$ d'éléments de $E$ \defe{converge}{convergence!en topologie} vers l'élément $x$ de $E$ si pour tout ouvert $\mO$ contenant $x$, il existe un $K$ tel que $k>K$ implique $x_k\in\mO$. Cette définition est exactement celle donnée pour la convergence de suites dans $\eR^n$, à part que nous avons remplacé le mot \og boule\fg{} par \og ouvert\fg.

Dans un espace topologique, nous avons une caractérisation très importante des ouverts.
\begin{theorem}		\label{ThoPartieOUvpartouv}
Une partie $A$ de $E$ est ouverte si et seulement si pour tout $x\in A$, il existe un ouvert autour de $x$ contenu dans $A$.
\end{theorem}

\begin{proof}
Le sens direct est évident : $A$ lui-même est un ouvert autour de $x\in A$, qui est inclus à $A$.

Pour le sens inverse, pour chaque $x\in A$, nous considérons l'ensemble $\mO_x\subset A$, un ouvert autour de $x$. Nous avons que
\begin{equation}	\label{EqAUniondesOx}
	A=\bigcup_{x\in A}\mO_x.
\end{equation}
En effet $A\subset\cup_{x\in A}\mO_x$ parce que tous les éléments de $A$ sont dans un des $\mO_x$, par construction. D'autre part, $\cup_{x\in A}\mO_x\subset A$ parce que chacun des $\mO_x$ est compris dans $A$.

L'union du membre de droite de \eqref{EqAUniondesOx} est une union d'ouverts et est donc un ouvert. Cela prouve que $A$ est un ouvert.

\end{proof}

Une utilisation typique de ce théorème est faite à l'exercice \ref{exo0083}.

%%%%%%%%%%%%%%%%%%%%%%%%%%
%
   \section{Topologie dans \texorpdfstring{$\eR^n$}{Rn}}
%
%%%%%%%%%%%%%%%%%%%%%%%%

Dans cette section, nous travaillons dans l'espace $\eR^n$ pour un certain naturel $n$. Nous y définissons la notion d'ouvert et de fermé, qui sont la base de la topologie générale. Notons que ces définitions n'ont de sens que relativement à l'espace ambiant, aussi un ouvert de $\eR$ ne sera en général pas un ouvert de $\eR^2$~: d'une part, il n'y a pas d'inclusion canonique de $\eR$ dans $\eR^2$ (les ouverts du second ne sont même pas des sous-ensembles du premier) et, d'autre part, les définitions se basent sur la notion de boule de $\eR^n$ qui dépend évidemment de la valeur de $n$ (une boule dans $\eR$ est un intervalle, dans $\eR^2$ c'est un disque, etc.)

%---------------------------------------------------------------------------------------------------------------------------
					\subsection{Ouverts et fermés}
%---------------------------------------------------------------------------------------------------------------------------

\begin{definition}
	La \defe{boule ouverte}{Boule!ouverte} de centre $x_0 \in \eR^n$ et de rayon $r \in
	\eR^+$ est définie par
	\begin{equation}
		B(x_0,r) = \{ x \in \eR^n \tq \norme{x - x_0} < r \},
	\end{equation}
	tandis que la \defe{boule fermée}{Boule!fermée} de centre $x_0$ et de rayon $r$ est
	\begin{equation}
		\bar B(x_0,r) = \{ x \in \eR^n \tq \norme{x - x_0} \leq r \};
	\end{equation}
	la différence est que l'inégalité dans la première est stricte.
\end{definition}

%---------------------------------------------------------------------------------------------------------------------------
					\subsection{Intérieur, adhérence et frontière}
%---------------------------------------------------------------------------------------------------------------------------

\begin{definition}
  Soit $A \subset \eR^n$ et $x \in \eR^n$. Le point $x$ est \defe{intérieur}{intérieur} à $A$ si il existe une boule autour de $x$ complètement contenue dans $A$. L'ensemble des points intérieurs à $A$ est noté $\interieur A$ ou $\mathring A$, de sorte qu'on a précisément
  \begin{equation*}
    x \in \interieur A \iffdefn  \exists \epsilon > 0 \tq
    B(x,\epsilon) \subset A.
  \end{equation*}
\end{definition}


\begin{definition}
Le point $x$ est dans l'\defe{adhérence}{adhérence} de $A$ si toute boule autour de $x$ intersecte $A$. L'ensemble de ces points est noté $\adh A$ ou $\bar A$, et on a donc de manière plus précise
\begin{equation}
	x \in \adh A \iffdefn \forall \epsilon > 0, B(x,\epsilon) \cap A \neq \emptyset
\end{equation}
\end{definition}

\begin{proposition}
Pour $A \subset \eR^n$, nous avons
\begin{equation*}
	\interieur A \subseteq A  \subseteq \adh A
\end{equation*}
\end{proposition}

\begin{definition}
  La \defe{frontière}{frontière} ou le \defe{bord}{bord} de $A$ est défini par $\partial A = \adh A \setminus \interieur A$. L'ensemble $A$ est un \defe{ouvert}{ouvert} si $A = \interieur A$, et c'est un \defe{fermé}{fermé} si $A = \adh A$.
\end{definition}

On vérifiera que les notations et les dénominations sont cohérentes en
prouvant la proposition suivante.
\begin{proposition}Pour $\epsilon > 0$,
  \begin{enumerate}
  \item l'adhérence de $B(x,\epsilon)$ est $\bar B(x,\epsilon)$,
  \item l'intérieur de $\bar B(x,\epsilon)$ est $B(x,\epsilon)$,
  \item la boule ouverte $B(x,\epsilon)$ est un ouvert,
  \item la boule fermée $\bar B(x,\epsilon)$ est un fermé.
  \end{enumerate}
\end{proposition}

Nous avons également les liens suivants entre intérieur, adhérence,
ouvert, fermé et passage au complémentaire (noté ${}^c$)~:
\begin{proposition}
Si $A \subset \eR^n$ et $A^c = \eR^n\setminus A$, nous
  avons
  \begin{enumerate}
  \item $(\interieur A)^c = \adh (A^c)$ et $(\adh A)^c = \interieur
    (A^c)$,
  \item $A$ est ouvert si et seulement si $A^c$ est fermé,
  \item $\interieur A$ est le plus grand ouvert contenu dans $A$,
  \item $\adh A$ est le plus petit fermé contenant $A$,
    % \item
  \end{enumerate}
\end{proposition}

%+++++++++++++++++++++++++++++++++++++++++++++++++++++++++++++++++++++++++++++++++++++++++++++++++++++++++++++++++++++++++++
\section{Maximum, majorant, supremum et compagnie}
%+++++++++++++++++++++++++++++++++++++++++++++++++++++++++++++++++++++++++++++++++++++++++++++++++++++++++++++++++++++++++++

Lorsque vous lisez que la charge maximale d'un camion est de \unit{2.5}{\ton}, est-ce que cela veut dire que vous pouvez y mettre \unit{2.5}{\ton}, mais qui si un oiseau se pose dessus, le camion s'effondre ? Ou bien est-ce que cela signifie qu'à \unit{2.5}{\ton} le camion s'écroule, mais que toute charge inférieure est valable ?

C'est à cette rude question que nous allons nous attaquer maintenant.

\begin{definition}
Soit une partie $A$ de $\eR$. Nous disons qu'un nombre $M$ est un \defe{majorant}{majorant} de $A$ si $M$ est plus grand ou égal que tous les éléments de $A$, c'est à dire si
\begin{equation}
	\forall a\in A,\, M\geq a.
\end{equation}
Un \defe{minorant}{minorant} de $A$ est un nombre $m$ tel que 
\begin{equation}
	\forall a\in A,\, m\leq a.
\end{equation}
\end{definition}

\begin{definition}		\label{DefSupeA}
Soit $A$ une partie majorée de $\eR$. Le \defe{supremum}{supremum} de $A$ est le plus petit des majorants, c'est à dire le nombre $M$ tel que
\begin{enumerate}
	\item
		$M\geq x$ pour tout $x\in A$,
	\item
		pour tout $\varepsilon$, le nombre $M-\varepsilon$ n'est pas un majorant de $a$, c'est à dire qu'il existe un élément $x\in A$ tel que $x>M-\varepsilon$.
\end{enumerate}
Nous notons $\sup A$ le supremum de $A$.

De la même façon, \defe{l'infimum}{infimum} de $A$, noté $\inf A$, est le plus grand de ses minorants. 
\end{definition}
Par convention, si la partie n'est pas bornée vers le haut, nous dirons que son supremum n'existe pas, ou bien qu'il est égal à $+\infty$, suivant les contextes. Pour votre culture générale, sachez toutefois que $\infty\notin\eR$.

La définition est justifiée par le lemme \ref{LemInfUnique} et la proposition \ref{PropBorneSupInf}. Le premier montre que si $A$ possède un infimum, alors il est unique, tandis que le second montre que toute partie majorée de $\eR$ accepter un supremum, et que toute partie minorée accepte un infimum.
\begin{lemma}		\label{LemInfUnique}
	Soit $A$ une partie de $\eR$. Supposons que $m_1$ et $m_2$ soient deux nombres qui vérifient les propriétés de l'infimum de $A$. Alors $m_1=m_2$.
\end{lemma}

\begin{proof}
	Si $_1\neq m_2$, nous pouvons supposer $m_2>m_1$. Dans ce cas, étant donné que $m_1$ est un infimum, $m_2$ ne peut pas minorer $A$, et donc ne peut pas être un infimum.
\end{proof}

\begin{proposition}		\label{PropBorneSupInf}
	Tout sous-ensemble de $\eR$ borné vers le bas possède un infimum; tout sous-ensemble de $\eR$ borné vers le haut possède un supremum.
\end{proposition}

La preuve qui suit est proche de celle donnée par Wikipédia  dans l'article \wikipedia{en}{http://en.wikipedia.org/wiki/Least_upper_bound_principle}{Least uppert bound principle}.

\begin{proof}
	Soit $A$, une partie de $\eR$. Nous allons trouver son infimum en suivant une méthode de dichotomie. Pour cela nous allons construire trois suites en même temps de la façon suivante. D'abord nous choisissons un point $x_0$ de $A$ et un point $x_1$ qui minore $A$ (qui existe par hypothèse) :
	\begin{equation}
		\begin{aligned}[]
			x_0&\text{ est un élément de $A$},\\
			x_1&\text{ est un minorant de $A$},\\
			a_0&=x_0\\
			b_0&=x_1\\
			b_1&=x_1.
		\end{aligned}
	\end{equation}
	Ensuite, nous faisons la récurrence suivante :
	\begin{equation}
		\begin{aligned}[]
			x_{n+1}&=\frac{ a_n+b_n }{2},\\
			a_{n+1}&=\begin{cases}
				a_{n}	&	\text{si $x_{n+1}$ minore $A$}\\
				x_{n+1}	&	 \text{sinon},
			\end{cases}\\
			b_{n+1}&=\begin{cases}
				x_{n+1}	&	\text{si $x_{n+1}$ minore $A$}\\
				b_n	&	 \text{sinon}.
			\end{cases}
		\end{aligned}
	\end{equation}
    Nous allons montrer que \( a_n\) et \( (b_n)\) sont des suites convergentes de même limite et que cette limite est l'infimum de \( A\).

	Soit $n\in\eN$; il y a deux possibilités. Soit $a_n=a_{n-1}$ et $b_n=x_n$, soit $a_n=x_n$ et $b_n=b_{n-1}$. Supposons que nous soyons dans le premier cas (le second se traite de façon similaire). Alors nous avons
	\begin{equation}
		\begin{aligned}[]
			| a_n-b_n |&=| a_{n-1}-x_n |\\
			&=\left| a_{n-1}-\frac{ a_{n-1}+b_{n-1} }{2} \right| \\
			&=\frac{ 1 }{2}| a_{n-1}-b_{n-1} |,
		\end{aligned}
	\end{equation}
	ce qui prouve que $| a_n-b_n |\to 0$. Nous montrons maintenant que la suite \( (a_n)\) est de Cauchy. En effet nous avons
    \begin{equation}
        | a_n-a_{n-1} |=\begin{cases}
          0\\
          \left| \frac{ a_n -b_n}{ 2} \right|   
      \end{cases}\leq \frac{1}{ 2n }.
    \end{equation}
    Il en est de même pour la suite \( (b_n)\). Ce sont deux suites de Cauchy (donc convergentes) qui convergent vers la même limite. Soit \( \ell\) cette limite.
    
	Le nombre $\ell$ minore $A$. En effet si $a\in A$ est plus petit que $\ell$, les éléments $b_n$ tels que $| b_n-\ell |<| a-\ell |$ ne peuvent pas minorer $A$. D'autre part, pour tout $\epsilon$, le nombre $\ell+\epsilon$ ne peut pas minorer $A$. En effet, $\ell$ est la limite de la suite décroissante $(a_n)$, donc il existe $a_n$ entre $\ell$ et $\ell+\epsilon$. Mais $a_n$ ne minore pas $A$, donc $\ell+\epsilon$ ne minore pas non plus $A$.

	Nous avons prouvé que toute partie minorée de $\eR$ possède un infimum. La preuve que toute partie majorée possède un supremum se fait de la même façon.
	
\end{proof}


\begin{definition}
	Si le supremum d'un ensemble appartient à l'ensemble, nous l'appelons \defe{maximum}{maximum}. De la même façon si l'infimum d'un ensemble appartient à l'ensemble, nous disons que c'est le \defe{minimum}{minimum}.
\end{definition}

\begin{example}
	Pour les intervalles, ces notions sont simples : les bornes de l'intervalle sont les supremum et infimum, et ce sont des minima et maxima si l'intervalle est fermé. Le nombre $53$ est un majorant.
	\begin{enumerate}
		\item
			$A=\mathopen[ 1 , 2 \mathclose]$. Tous les nombres plus petits ou égaux à $1$ sont minorants, $1$ est infimum et minimum. Le nombre $2$ est un majorant, le maximum et le supremum.
		\item
			$B=\mathopen] 3 , \pi \mathclose[$. Le nombre $\pi$ est le supremum et est un majorant, mais n'est pas le maximum (parce que $\pi\notin B$). L'ensemble $B$ n'a pas de maximum. Bien entendu, $-1000$ est un minorant.
	\end{enumerate}
\end{example}

Il existe évidement de nombreux exemples plus vicieux.
\begin{example}
	Prenons $E=\{ \frac{1}{ n }\tq n\in\eN_0 \}$, dont les premiers points sont indiqués sur la figure \ref{LabelFigSuiteUnSurn}. Cet ensemble est constitué des nombres $1$, $\frac{ 1 }{2}$, $\frac{1}{ 3 }$, \ldots Le plus grand d'entre eux est $1$ parce que tous les nombres de la forme $\frac{1}{ n }$ avec $n\geq 1$ sont plus petits ou égaux à $1$. Le nombre $1$ est donc maximum de $E$.

	L'ensemble $E$ n'a par contre pas de minimum parce que tout élément de $E$ s'écrit $\frac{1}{ n }$ pour un certain $n$ et est plus grand que $\frac{1}{ n+1 }$ qui est également dans $E$.

	Prouvons que zéro est l'infimum de $E$. D'abord, tous les éléments de $E$ sont strictement positifs, donc zéro est certainement un minorant de $E$. Ensuite, nous savons que pour tout $\varepsilon>0$, il existe un $n$ tel que $\frac{1}{ n }$ est plus petit que $\varepsilon$. L'ensemble $E$ possède donc un élément plus petit que $0+\varepsilon$, et zéro est bien l'infimum.
\end{example}

\newcommand{\CaptionFigSuiteUnSurn}{Les premiers points du type $x_n=1/n$.}
\input{Fig_SuiteUnSurn.pstricks}

L'exemple suivant est une source classique d'erreurs en ce qui concerne l'infimum. Il sera à relire après avoir vu la définition de limite (définition \ref{DefLimiteSuiteNum}).
\begin{example}
	Les premiers points de l'ensemble $F=\{ \frac{ (-1)^n }{ n }\tq n\in\eN_0 \}$ sont représentés à la figure \ref{LabelFigSuiteInverseAlterne}. Bien que (comme nous le verrons plus tard) la limite de la suite $x_n=(-1)^n/n$ soit zéro, il n'est pas correct de dire que zéro est l'infimum de l'ensemble $F$. Le dessin, au contraire, montre bien que $-1$ est le minium (aucun point est plus bas que $-1$), tandis que le maximum est $1/2$.

	Nous reviendrons avec cet exemple dans la suite. Pour l'instant, ayez bien en tête que zéro n'est rien de spécial pour l'ensemble $F$ en ce qui concerne les notions de maximum, minimum et compagnie.
\end{example}
\newcommand{\CaptionFigSuiteInverseAlterne}{Les quelque premiers points du type $(-1)^n/n$.}
\input{Fig_SuiteInverseAlterne.pstricks}

%+++++++++++++++++++++++++++++++++++++++++++++++++++++++++++++++++++++++++++++++++++++++++++++++++++++++++++++++++++++++++++
\section{Point d'accumulation, point isolé}
%+++++++++++++++++++++++++++++++++++++++++++++++++++++++++++++++++++++++++++++++++++++++++++++++++++++++++++++++++++++++++++

Soit $D\subset\eR$. Un point $a\in D$ est \defe{isolé}{isolé!élément de $\eR$} dans $D$ (relativement à $\eR$) si il existe $\varepsilon>0$ tel que 
\begin{equation}
	\mathopen[ a-\varepsilon , a+\varepsilon \mathclose]\cap D=\{ a \}.
\end{equation}
Autrement dit, il existe un intervalle autour de $a$ dans lequel $a$ est le seul élément de $D$.

Un point $a\in \eR$ est un \defe{point d'accumulation}{accumulation!dans $\eR$} de $D$ si pour tout $\varepsilon>0$, 
\begin{equation}
	\Big( \mathopen[ a-\varepsilon , a+\varepsilon \mathclose]\setminus\{ a \} \Big)\cap D\neq\emptyset.
\end{equation}
Autrement dit, quel que soit l'intervalle autour de  $a$ que l'on considère, le point $a$ n'est pas tout seul dans $D$.

\begin{example}
	Prenons $D=\mathopen[ 0 , 1 [\cup\mathopen] 2 , 3 \mathclose]$. Cet ensemble n'a pas de points isolés, et l'ensemble de ses points d'accumulation est $\mathopen[ 0 , 1 \mathclose]\cup\mathopen[ 2,3  \mathclose]$.

	Notez que les points $1$ et $2$ sont des points d'accumulation de $D$ qui ne font pas partie de $D$. Il est possible d'être un point d'accumulation de $D$ sans être dans $D$, mais pour être un point isolé dans $D$, il faut être dans $D$.
\end{example}

\begin{example}
	Soit $D=\{ \frac{1}{ n }\}_{n\in\eN}$. Tous les points de cet ensemble sont des points isolés (vérifier !).  Aucun point de $D$ n'est point d'accumulation. Cependant $0$ est un point d'accumulation.
\end{example}

%---------------------------------------------------------------------------------------------------------------------------
					\subsection{Bornés et compacts}
%---------------------------------------------------------------------------------------------------------------------------


\begin{definition}
  Un sous ensemble $A \subset \eR^n$ est \defe{borné}{borné} si il existe une boule de $\eR^n$ contenant $A$.
\end{definition}

\begin{proposition}
  Toute réunion finie d'ensembles bornés est un ensemble borné. Toute
  partie d'un ensemble borné est un ensemble borné.
\end{proposition}

\begin{definition}
  La partie $A \subset \eR^n$ est \defe{compacte}{compact} si et seulement si, pour tout
  recouvrement de $A$ par des ouverts (c'est-à-dire une collection
  d'ouverts dont la réunion contient $A$) on peut tirer un
  recouvrement fini.
\end{definition}

% En particulier, si on recouvre $A$ par l'ensemble des boules
% $B(x,1)$ où $x$ parcourt $A$ (de sorte que tout point de $A$ est
% dans \og sa\fg{} boule, et donc la réunion des boules recouvre bien
% $A$), on doit pouvoir en tirer un recouvrement fini, c'est-à-dire
% des boules $B(x_1,1), B(x_2,1), \ldots, B(x_k,1)$ (avec $k$ un
% naturel) dont la réunion contient $A$.

\begin{proposition}
Une partie de $\eR^n$ est compacte si et seulement si elle est fermée et bornée.
\end{proposition}

%---------------------------------------------------------------------------------------------------------------------------
					\subsection{Connexité}
%---------------------------------------------------------------------------------------------------------------------------

\begin{definition}
  Le sous ensemble $A \subset \eR^n$ est \defe{connexe par arcs}{Connexe!par arc} si pour tout $x, y \in
  A$, il existe un chemin\footnote{Attention : ici quand on dit \emph{chemin}, on demande que l'application soit continue. Dans de nombreux cours de géométrie différentielle, on demande $ C^{\infty}$. Il faut s'adapter au contexte.} contenu dans $A$ les reliant, c'est-à-dire
  une application continue
  \begin{equation*}
    \gamma : [0,1] \to \eR^n \tq \gamma(0) = x~\text{et}~\gamma(1) = y
  \end{equation*}
  avec $\gamma(t) \in A$ pour tout $t\in [0,1]$.
\end{definition}

%+++++++++++++++++++++++++++++++++++++++++++++++++++++++++++++++++++++++++++++++++++++++++++++++++++++++++++++++++++++++++++
					\section{Topologie des espaces métriques}
%+++++++++++++++++++++++++++++++++++++++++++++++++++++++++++++++++++++++++++++++++++++++++++++++++++++++++++++++++++++++++++

Si $E$ est un ensemble, une \defe{distance}{distance} sur $E$ est une application $d\colon E\times E\to \eR$ telle que pour tout $x,y\in E$,
\begin{enumerate}

\item
$d(x,y)\geq 0$

\item
$d(x,y)=0$ si et seulement si $x=y$,

\item
$d(x,y)=d(y,x)$

\item
$d(x,y)\leq d(x,z)+d(z,y)$.

\end{enumerate}
La dernière condition est l'\defe{inégalité triangulaire}{inégalité!triangulaire}. Le couple $(E,d)$ d'un ensemble et d'une métrique est un \defe{espace métrique}{espace!métrique}.

Dès que l'ensemble $E$ est muni d'une distance, nous définissons une topologie en disant que les boules
\begin{equation}
	B(x,r)=\{ y\in E\tq d(x,y)<r \}
\end{equation}
sont ouvertes.

\begin{proposition} \label{PropvvSKiE}
    Soit \( E\), un espace métrique et \( K\subset E\). L'ensemble \( K\) est compact si et seulement si toute suite dans \( K\) contient une sous suite convergente dans \( K\).
\end{proposition}

\begin{lemma}   \label{LemnAeACf}
    Si \( K\) est compact et si \( F\) est fermé dans \( K\), alors \( F\) est compact.
\end{lemma}

\begin{proof}
    Nous allons utiliser la caractérisation \ref{PropvvSKiE}. Soit \( (x_n)\) une suite dans \( F\); par la compacité de \( K\) nous pouvons considérer une sous suite \( (y_n)\) qui converge dans \( K\) (proposition \ref{PropvvSKiE}). Étant donné que \( (y_n)\) est une suite convergente contenue dans \( F\) et étant donné que \( F\) est fermé, la limite est dans \( F\), ce qui prouve que \( (x_n)\) possède une sous suite convergente dans $F$ et par conséquent que \( F\) est compact.
\end{proof}

\begin{lemma}       \label{LemooynkH}
    Soit \( A_n\) une suite décroissante de fermés dans un compact \( K\). Alors
    \begin{equation}
        C=\bigcap_{n\in \eN}A_n
    \end{equation}
    est non vide.
\end{lemma}

\begin{proof}
    Soit \( x_n\) une suite dans \( K\) telle que \( x_n\in A_n\). La suite étant contenue dans \( A_1\), elle possède une sous suite \( (y_n=x_{\sigma_1(n)})\) convergente dont la limite est dans \( A_1\). Une queue de la suite \( y_n\) est dans \( A_2\) et nous considérons donc une sous suite convergente dans \( A_2\) donnée par
    \begin{equation}
        z_n=y_{\sigma_2(n)}=x_{\sigma_1\sigma_2(n)}.
    \end{equation}
    En continuant ainsi nous construisons une suite convergente dans \( A_k\). Nous considérons enfin la suite
    \begin{equation}
        y_n=x_{\sigma_1\ldots \sigma_n(n)}.
    \end{equation}
    Pour tout \( k\), une queue de cette suite est une sous suite de \( x_{\sigma_1\ldots \sigma_k(n)}\) et par conséquent cette suite converge dans \( A_k\). La limite de cette suite est donc dans l'intersection demandée.
\end{proof}

\begin{remark}
    Cette propriété est fausse pour les ouverts. Par exemple
    \begin{equation}
        \bigcap_{n>1}\mathopen] 0 , \frac{1}{ n } \mathclose[=\emptyset.
    \end{equation}
\end{remark}

%+++++++++++++++++++++++++++++++++++++++++++++++++++++++++++++++++++++++++++++++++++++++++++++++++++++++++++++++++++++++++++
\section{Ensembles nulle part denses}
%+++++++++++++++++++++++++++++++++++++++++++++++++++++++++++++++++++++++++++++++++++++++++++++++++++++++++++++++++++++++++++

Nous allons nous limite au cas de \( \eR\), mais je crois que ça se généralise sans trop de peine aux espaces en tout cas métriques.

\begin{definition}
    Un ensemble est dit \defe{nulle part dense}{nulle part dense}\index{dense!nulle part} si il n'est dense dans aucun intervalle.

    Un ensemble dans \( \eR\) est de \defe{première catégorie}{catégorie!ensemble de première} ou \defe{maigre}{maigre (ensemble)} si il est une union dénombrable d'ensembles nulle part dense (c'est à dire d'ensembles denses sur aucun intervalle).
\end{definition}

\begin{theorem}[Baire\cite{BaireZied}]  \index{Baire!théorème}\index{théorème!Baire}    \label{ThoQGalIO}
    Une réunion dénombrable d'ensembles nulle part denses est d'intérieur vide.
\end{theorem}

\begin{proof}
    Soit \( a\in S\) et \( \epsilon>0\). Nous allons trouver un élément dans \( B(a,\epsilon)\) qui n'est pas dans \( S\). Nous commençons par choisir \( x_1\in B(a,\epsilon)\) et \( r_1<\frac{ \epsilon }{2}\) tel que
    \begin{equation}
        B(x_1,r_1)\cap A_1=\emptyset.
    \end{equation}
    Ensuite nous choisissons \( x_2\in B(x_1,r_1)\) et \( r_2<\epsilon/4\) tel que \( B(x_2,r_2)\subset B(x_1,r_1)\) et \( B(x_2,r_2)\cap A_2=\emptyset\). Notons que \( B(x_2,r_2)\cap A_1=\emptyset\) aussi, par construction.

    Par récurrence nous construisons une suite d'éléments \( x_n\) et de rayons \( r_n<\epsilon/2^n\) tels que
    \begin{enumerate}
        \item
            \( B(x_n,r_n)\cap A_j=\emptyset\) pour tout \( j\leq n\),
        \item
            \( \overline{ B(x_n,r_n) }\subset B(x_{n-1},r_{r-1})\).
    \end{enumerate}
    Cette suite étant de Cauchy (parce que contenue dans des intervalles emboités de rayon décroissant vers zéro), elle converge donc vers un point qui en particulier appartient à \( B(a,\epsilon)\). Mais la limite n'est dans aucun des \( A_n\) et donc pas dans \( S\).
\end{proof}

%+++++++++++++++++++++++++++++++++++++++++++++++++++++++++++++++++++++++++++++++++++++++++++++++++++++++++++++++++++++++++++
					\section{Uniforme continuité}
%+++++++++++++++++++++++++++++++++++++++++++++++++++++++++++++++++++++++++++++++++++++++++++++++++++++++++++++++++++++++++++

\begin{proposition}	\label{PropoInvCompCont}
Soit $f\colon A\subset\eR^n\to B\subset\eR^m$ une bijection continue. Si $A$ est compact, alors $f^{-1}\colon B\to A$ est continue.
\end{proposition}

\begin{proposition}		\label{PropIntContMOnIvCont}
Soient $I$ un intervalle dans $\eR$ et $f\colon I\to \eR$ une fonction continue strictement monotone. Alors la fonction réciproque $f^{-1}\colon f(I)\to \eR$ est continue sur l'intervalle $f(I)$.
\end{proposition}


