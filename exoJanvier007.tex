% This is part of the Exercices et corrigés de mathématique générale.
% Copyright (C) 2009
%   Laurent Claessens
% See the file fdl-1.3.txt for copying conditions.
\begin{exercice}\label{exoJanvier007}

Charles-Édouard souhaite repeindre sa voiture. Pour ce faire, il dispose de pots de peinture verte, jaune, bleue, rouge et noire en suffisance. Il souhaite cependant repeindre séparément le coffre, le capot, le pare-choc, le toit et chacune des quatre portières. De combien de façons différentes peut-il le faire sachant qu'il souhaite qu'au moins une des portières soit verte et que le pare-choc ne peut pas être bleu ? Justifier brièvement.

\corrref{Janvier007}
\end{exercice}
