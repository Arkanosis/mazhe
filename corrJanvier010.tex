% This is part of the Exercices et corrigés de mathématique générale.
% Copyright (C) 2009
%   Laurent Claessens
% See the file fdl-1.3.txt for copying conditions.
\begin{corrige}{Janvier010}

Soit $f : [a,b] \to \eR$ une fonction continue, et dérivable sur $\mathopen]a,b\mathclose[$.  Alors il existe $c \in \mathopen]a,b\mathclose[$ tel que
\begin{equation}
  f^\prime(c) = \frac{f(b)-f(a)}{b-a}.
\end{equation}

%TODO : refaire le dessin
%Un dessin illustrant ce théorème est donné à la figure \ref{LabelFigAccrFini}.
%\newcommand{\CaptionFigAccrFini}{Illustration du théorème des acroissements finis.}
%\input{Fig_AccrFini.pstricks}

\end{corrige}
