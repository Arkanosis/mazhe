% This is part of Mes notes de mathématique
% Copyright (c) 2006-2015
%   Laurent Claessens, Carlotta Donadello
% See the file fdl-1.3.txt for copying conditions.

%+++++++++++++++++++++++++++++++++++++++++++++++++++++++++++++++++++++++++++++++++++++++++++++++++++++++++++++++++++++++++++
\section{Compacité}
%+++++++++++++++++++++++++++++++++++++++++++++++++++++++++++++++++++++++++++++++++++++++++++++++++++++++++++++++++++++++++++
%http://fr.wikipedia.org/wiki/Espace_compact
%http://fr.wikipedia.org/wiki/Théorème_de_Heine-Borel
%http://fr.wikipedia.org/wiki/Émile_Borel
%http://fr.wikipedia.org/wiki/Henri_Léon_Lebesgue

Soit $E$, un sous ensemble de $\eR$. Nous pouvons considérer les ouverts suivants : 
\begin{equation}
    \mO_x=B(x,1)
\end{equation}
pour chaque $x\in E$. Évidement,
\begin{equation}
    E\subseteq \bigcup_{x\in E}\mO_x.
\end{equation}
Cette union est très souvent énorme, et même infinie. Elle contient de nombreuses redondances. Si par exemple $E=[-10,10]$, l'élément $3\in E$ est contenu dans $\mO_{3.5}$, $\mO_{2.7}$ et bien d'autres. Pire : même si on enlève par exemple $\mO_2$ de la liste des ouverts, l'union de ce qui reste continue à être tout $E$. La question est : \emph{est-ce qu'on peut en enlever suffisamment pour qu'il n'en reste qu'un nombre fini ?}
\begin{definition}
Soit $E$, un sous ensemble de $\eR$. Une collection d'ouverts $\mO_i$ est un \defe{recouvrement}{recouvrement} de $E$ si $E\subseteq \bigcup_{i}\mO_i$. Un sous ensemble $E$ de $\eR$ tel que de tout recouvrement par des ouverts, on peut extraire un sous-recouvrement fini est dit \defe{\href{http://fr.wikipedia.org/wiki/Espace_compact}{compact}}{compact}.
\end{definition}

\begin{proposition}
Les ensembles compacts sont fermés et bornés.
\end{proposition}

\begin{proof}
Prouvons d'abord qu'un ensemble compact est borné. Pour cela, supposons que $K$ est un compact non borné vers le haut\footnote{Nous laissons à titre d'exercice le cas où $K$ est borné par le haut et pas par le bas.}. Donc il existe une suite infinie de nombres strictement croissante $x_1<x_2<\ldots$ tels que $x_i\in K$. Prenons n'importe quel recouvrement ouvert de la partie de $K$ plus petite ou égale à $x_1$, et complétons ce recouvrement par les ouverts $\mO_i=]x_{i-1},x_i[$. Le tout forme bien un recouvrement de $K$ par des ouverts. 

Il n'y a cependant pas moyen d'en tirer un sous recouvrement fini parce que si on ne prends qu'un nombre fini parmi les $\mO_i$, on en aura fatalement un maximum, disons $\mO_k$. Dans ce cas, les points $x_{k+1}$, $x_{k+1}$,\ldots ne seront pas dans le choix fini d'ouverts.

Cela prouve que $K$ doit être borné.

Pour prouver que $K$ est fermé, nous allons prouver que le complémentaire est ouvert. Et pour cela, nous allons prouver que si le complémentaire n'est pas ouvert, alors nous pouvons construire un recouvrement de $K$ dont on ne peut pas extraire de sous recouvrement fini.

Si $\eR\setminus K$ n'est pas ouvert, il possède un point, disons $x$, tel que tout voisinage de $x$ intersecte $K$. Soit $B(x,\epsilon_1)$, un de ces voisinages, et prenons $k_1\in K\cap B(x,\epsilon_1)$. Ensuite, nous prenons $\epsilon_2$ tel que $k_1$ n'est pas dans $B(x,\epsilon_1)$, et nous choisissons $k_2\in K\cap B(x,\epsilon_2)$. De cette manière, nous construisons une suite de $k_i\in K$ tous différents et de plus en plus proches de $x$. Prenons un recouvrement quelconque par des ouverts de la partie de $K$ qui n'est pas dans $B(x,\epsilon_1)$. Les nombres $k_i$ ne sont pas dans ce recouvrement.

Nous ajoutons à ce recouvrement les ensembles $\mO=]k_i,k_{i+1}[$. Le tout forme un recouvrement (infini) par des ouverts dont il n'y a pas moyen de tirer un sous recouvrement fini, pour exactement la même raison que la première fois.
\end{proof}

Le résultat suivant le théorème de \href{http://fr.wikipedia.org/wiki/Théorème_de_Heine-Borel}{Borel-Lebesgue}, et la démonstration vient de wikipédia.
\begin{theorem}[\href{http://fr.wikipedia.org/wiki/Émile_Borel}{borel}-\href{http://fr.wikipedia.org/wiki/Henri_Léon_Lebesgue}{Lebesgue}]   \label{ThoBOrelLebesgue}
    Les intervalles de la forme $[a,b]$ sont compacts.
\end{theorem}

\begin{proof}
    Soit $\Omega$, un recouvrement du segment $[a,b]$ par des ouverts, c'est à dire que
    \begin{equation}
        [a,b]\subseteq\bigcup_{\mO\in\Omega}\mO.
    \end{equation}
    Nous notons par $M$ le sous-ensemble de $[a,b]$ des points $m$ tels que l'intervalle $[a,m]$ peut être recouvert par un sous-ensemble fini de $\Omega$. C'est à dire que $M$ est le sous ensemble de $[a,b]$ sur lequel le théorème est vrai. Le but est maintenant de prouver que $M=[a,b]$.
    \begin{description}
        \item[$M$ est non vide] En effet, $a\in M$ parce que il existe un ouvert $\mO\in\Omega$ tel que $a\in\mO$. Donc $\mO$ tout seul recouvre l'intervalle $[a,a]$. 
        \item[$M$ est un intervalle] Soient $m_1$, $m_2\in M$. Le but est de montrer que si $m'\in[m_1,m_2]$, alors $m'\in M$. Il y a un sous recouvrement fini de l'intervalle $[a,m_2]$ (par définition de $m_2\in M$). Ce sous recouvrement fini recouvre évidement aussi $[a,m']$ parce que $[a,m']\subseteq [a,m_2]$, donc $m'\in M$.
        \item[$M$ est une ensemble ouvert] Soit $m\in M$. Le but est de prouver qu'il y a un ouvert autour de $m$ qui est contenu dans $M$. Mettons que $\Omega'$ soit un sous recouvrement fini qui contienne l'intervalle $[a,m]$. Dans ce cas, on a un ouvert $\mO\in\Omega'$ tel que $m\in\mO$. Tous les points de $\mO$ sont dans $M$, vu qu'ils sont tous recouverts par $\Omega'$. Donc $\mO$ est un voisinage de $m$ contenu dans $M$.
        \item[$M$ est un ensemble fermé] $M$ est un intervalle qui commence en $a$, en contenant $a$, et qui finit on ne sait pas encore où. Il est donc soit de la forme $[a,m]$, soit de la forme $[a,m[$. Nous allons montrer que $M$ est de la première forme en démontrant que $M$ contient son supremum $s$. Ce supremum est un élément de $[a,b]$, et donc il est contenu dans un des ouverts de $\Omega$. Disons $s\in\mO_s$. Soit $c$, un élément de $\mO_s$ strictement plus petit que $c$; étant donné que $s$ est supremum de $M$, cet élément $c$ est dans $M$, et donc on a un sous recouvrement fini $\Omega'$ qui recouvre $[a,c]$. Maintenant, le sous recouvrement constitué de $\Omega'$ et de $\mO_s$ est fini et recouvre $[a,s]$.
    \end{description}
    Nous pouvons maintenant conclure : le seul intervalle non vide de $[a,b]$ qui soit à la fois ouvert et fermé est $[a,b]$ lui-même, ce qui prouve que $M=[a,b]$, et donc que $[a,b]$ est compact.
\end{proof}

Par le théorème des valeurs intermédiaires, l'image d'un intervalle par une fonction continue est un intervalle, et nous avons l'importante propriété suivante des fonctions continues sur un compact.

Le théorème suivant est un cas particulier du théorème \ref{ThoMKKooAbHaro}.
\begin{theorem}
    Si $f$ est une fonction continue sur l'intervalle compact $[a,b]$. Alors $f$ est bornée sur $[a,b]$ et elle atteint ses bornes.
\end{theorem}

\begin{proof}
    Étant donné que $[a,b]$ est un intervalle compact, son image est également un intervalle compact, et donc est de la forme $[m,M]$. Ceci découle du théorème \ref{ThoImCompCotComp} et le corollaire \ref{CorImInterInter}. Le maximum de $f$ sur $[a,b]$ est la borne $M$ qui est bien dans l'image (parce que $[m,M]$ est fermé). Idem pour le minimum $m$.
\end{proof}

%+++++++++++++++++++++++++++++++++++++++++++++++++++++++++++++++++++++++++++++++++++++++++++++++++++++++++++++++++++++++++++
\section{Dérivation}
%+++++++++++++++++++++++++++++++++++++++++++++++++++++++++++++++++++++++++++++++++++++++++++++++++++++++++++++++++++++++++++

\begin{lemma}           \label{LemDeccCarr}
    Si $f(x)=x^2$, alors $f'(x)=2x$.
\end{lemma}

\begin{proof}
    Utilisons la définition, et remplaçons $f$ par sa valeur :
    \begin{subequations}
        \begin{align}
            f'(x)   &=\lim_{\epsilon\to 0}\frac{ f(x+\epsilon)-f(x) }{ \epsilon }\\
                &=\lim_{\epsilon\to 0}\frac{ (x+\epsilon)^2-x^2 }{ \epsilon }\\
                &=\lim_{\epsilon\to 0}\frac{ x^2+2x\epsilon+\epsilon^2-x^2 }{ \epsilon }\\
                &=\lim_{\epsilon\to 0}\frac{\epsilon(2x+\epsilon)}{ \epsilon }\\
                &=\lim_{\epsilon\to 0}(2x+\epsilon)\\
                &=2x,
        \end{align}
    \end{subequations}
    ce qu'il fallait prouver.
\end{proof}

Une facile, maintenant.
\begin{proposition}
    La dérivé de la fonction $x\mapsto x$ vaut $1$, en notations compactes : $(x)'=1$.
\end{proposition}

\begin{proof}
D'après la définition de la dérivée, si $f(x)=x$, nous avons
\begin{equation}
    f(x)=\lim_{\epsilon\to 0}\frac{ (x+\epsilon) -x }{\epsilon} =\lim_{\epsilon\to 0}\frac{ \epsilon }{\epsilon} =1,
\end{equation}
et c'est déjà fini.
\end{proof}

Pour continuer, nous allons en faire une un peu plus abstraite.
\begin{proposition}     \label{PropDerrLin}
    La dérivation est une opération linéaire, c'est à dire que
    \begin{enumerate}
        \item $(\lambda f)'=\lambda f'$ pour tout réel $\lambda$ où, pour rappel, la fonction $(\lambda f)$ est définie par $(\lambda f)(x)=\lambda\cdot f(x)$,
        \item $(f+g)'=f'+g'$.
    \end{enumerate}
\end{proposition}

\begin{proof}
Ces deux propriétés découlent des propriétés correspondantes de la limite. Nous allons faire la première, et laisser la seconde à titre d'exercice. Écrivons la définition de la dérivée avec $(\lambda f)$ au lieu de $f$, et calculons un petit peu :
\begin{equation}
    \begin{aligned}[]
        (\lambda f)'(x) &=\lim_{\epsilon\to 0}\frac{ (\lambda f)(x+\epsilon)-(\lambda f)(x) }{ \epsilon }\\
                &=\lim_{\epsilon\to 0}\frac{ \lambda \big( f(x+\epsilon) \big)-\lambda f(x) }{ \epsilon }\\
                &=\lim_{\epsilon\to 0}\lambda \frac{ f(x+\epsilon) -f(x) }{ \epsilon }\\
                &=\lambda \lim_{\epsilon\to 0}\frac{ f(x+\epsilon) -f(x) }{ \epsilon }\\
                &=\lambda f'(x).
    \end{aligned}
\end{equation}
\end{proof}


\begin{proposition}
    La dérivée d'un produit obéit à la \defe{règle de Leibnitz}{Règle de Leibnitz}\index{Leibnitz}:
    \begin{equation}
        (fg)'(x)=f'(x)g(x)+f(g)g'(x).
    \end{equation}
    Cette règle est souvent écrite sous la forme compacte $(fg)'=f'g+g'f$.
\end{proposition}

\begin{proof}
La définition de la dérivée dit que
\begin{equation}        \label{Eqfgrimeepsfgx}
    (fg)'(x)=\lim_{\epsilon\to 0}\frac{f(x+\epsilon)g(x+\epsilon)-f(x)g(x)}{\epsilon}.
\end{equation}
La subtilité est d'ajouter au numérateur la quantité $-f(x)g(x+\epsilon)+f(x)g(x+\epsilon)$, ce qui est permit parce que cette quantité est nulle\footnote{Le coup d'ajouter et enlever la même chose a déjà été fait durant la démonstration du théorème \ref{Tholimfgabab}. C'est une technique assez courante en analyse.}. Le numérateur de \eqref{Eqfgrimeepsfgx} devient donc
\begin{equation}
    \begin{aligned}[]
f(x+\epsilon)g(x+\epsilon)&-f(x)g(x+\epsilon)+f(x)g(x+\epsilon)-f(x)g(x) \\
            &= g(x+\epsilon)\big( f(x+\epsilon)-f(x) \big)+f(x)\big( g(x+\epsilon)-g(x) \big),
    \end{aligned}
\end{equation}
où nous avons effectué deux mises en évidence. Étant donné que nous avons deux termes, nous pouvons couper la limite en deux :
\begin{equation}
    \begin{aligned}[]
        (fg)'(x)    &=\lim_{\epsilon\to 0}g(x+\epsilon)\frac{ f(x+\epsilon)-f(x) }{\epsilon}            &+\lim_{\epsilon\to 0}f(x)\frac{ g(x+\epsilon)-g(x) }{\epsilon}\\
                &=\lim_{\epsilon\to 0}g(x+\epsilon)\lim_{\epsilon\to 0}\frac{ f(x+\epsilon)-f(x) }{\epsilon}    &+f(x)\lim_{\epsilon\to 0}\frac{ g(x+\epsilon)-g(x) }{\epsilon},
    \end{aligned}
\end{equation}
où nous avons utilisé le théorème \ref{Tholimfgabab} pour scinder la première limite en deux, ainsi que la propriété \eqref{Eqbutmultlim} pour sortir le $f(x)$ de la limite dans le second terme. Maintenant, dans le premier terme, nous avons évidement\footnote{Pas tout à fait évidemment : selon le théorème \ref{ThoLimCont}, \emph{limite et continuité}, il faut que $g$ soit continue.} $\lim_{\epsilon\to 0}g(x+\epsilon)=g(x)$. Les limites qui restent sont les définitions classiques des dérivées de $f$ et $g$ au point~$x$ :
\begin{equation}
    (fg)'(x)=g(x)f'(x)-f(x)g'(x),
\end{equation}
ce qu'il fallait démontrer.
\end{proof}

%--------------------------------------------------------------------------------------------------------------------------- 
\subsection{Dérivation et croissance}
%---------------------------------------------------------------------------------------------------------------------------

Supposons une fonction dont la dérivée est positive. Étant donné que la courbe est \og collée \fg{} à ses tangentes, tant que les tangentes montent, la fonction monte. Or, une tangente qui monte correspond à une dérivée positive, parce que la dérivée est le coefficient directeur de la tangente.

Ce résultat très intuitif peut être prouvé rigoureusement. C'est la tache à laquelle nous allons nous atteler maintenant.

\begin{proposition} \label{PropGFkZMwD}
    Si $f$ et $f'$ sont des fonctions continues sur l'intervalle $[a,b]$ et si $f'(x)$ est strictement positive sur $[a,b]$, alors $f$ est croissante sur $[a,b]$.

    De la même manière, si $f'(x)$ est strictement négative sur $[a,b]$, alors $f$ est décroissante sur $[a,b]$.
\end{proposition}

\begin{proof}
    Nous n'allons prouver que la première partie. La seconde partie se prouve en considérant $-f$ et en invoquant alors la première\footnote{Méditer cela.}. Prenons $x_1$ et $x_2$ dans $[a,b]$ tels que $x_1<x_2$. Par hypothèse, pour tout $x$ dans $[x_1,x_2]$, nous avons
    \begin{equation}
        f'(x)=\lim_{\epsilon\to 0}\frac{ f(x+\epsilon)-f(x) }{\epsilon} >0.
    \end{equation}
    Maintenant, la proposition \ref{PropoLimPosFPos} dit que quand une limite est positive, alors la fonction dans la limite est positive sur un voisinage. En appliquant cette proposition à la fonction
    \begin{equation}
        r(\epsilon)=\frac{ f(x+\epsilon)-f(x) }{ \epsilon },
    \end{equation}
    dont la limite en zéro est positive, nous trouvons que $r(\epsilon)>0$ pour tout $\epsilon$ pas trop éloigné de zéro. En particulier, il existe un $\delta>0$ tel que $\epsilon<\delta$ implique $r(\epsilon)>0$; pour un tel $\epsilon$, nous avons donc
    \begin{equation}
        r(\epsilon)=\frac{ f(x+\epsilon)-f(x) }{ \epsilon }>0.
    \end{equation}
    Étant donné que $\epsilon>0$, nous avons que $f(x+\epsilon)-f(x)>0$, c'est à dire que $f$ est strictement croissante entre $x$ et $x+\delta$.

    Jusqu'ici, nous avons prouvé que la fonction $f$ était strictement croissante dans un voisinage autour de chaque point de $[a,b]$. Cela n'est cependant pas encore tout à fait suffisant pour conclure. Ce que nous voudrions faire, c'est de dire, c'est prendre un voisinage $]a,m_1[$ autour de $a$ sur lequel $f$ est croissante. Donc, $f(m_1)>f(a)$. Ensuite, on prend un voisinage $]m_1,m_2[$ de $m_1$ sur lequel $f$ est croissante. De ce fait, $f(m_2)>f(m_1)>f(a)$. Et ainsi de suite, nous voulons construire des $m_3$, $m_4$,\ldots jusqu'à arriver en $b$. Hélas, rien ne dit que ce processus va fonctionner. Il faut trouver une subtilité. Le problème est que les voisinages sur lesquels la fonction est croissante sont peut-être de plus en plus petit, de telle sorte à ce qu'il faille une infinité d'étapes avant d'arriver à bon port (en $b$).

    Heureusement, nous pouvons drastiquement réduire le nombre d'étapes en nous souvenant du théorème de Borel-Lebesgue (numéro \ref{ThoBOrelLebesgue}). Nous notons par $\mO_x$, un ouvert autour de $x$ tel que $f$ soit strictement croissante sur $\mO_x$. Un tel voisinage existe. Cela fait une infinité d'ouverts tels que
    \begin{equation}
        [a,b]\subseteq\bigcup_{x\in[a,b]}\mO_x.
    \end{equation}
    Ce que le théorème dit, c'est qu'on peut en choisir un nombre fini qui recouvre encore $[a,b]$. Soient $\{ \mO_{x_1},\ldots,\mO_{x_n} \}$, les heureux élus, que nous supposons prit dans l'ordre : $x_1<x_2<\ldots<x_n$. Nous avons
    \begin{equation}
        [a,b]\subseteq\bigcup_{i=1}^n\mO_i.
    \end{equation}
    Quitte à les rajouter à la collection, nous supposons que $x_1=a$ et que $x_n=b$. Maintenant nous allons choisir encore un sous ensemble de cette collection d'ouverts. On pose $\mA_1=\mO_{x_1}$. Nous savons que $\mA_1$ intersecte au moins un des autres $\mO_{x_i}$. Cette affirmation vient du fait que $[a,b]$ est connexe (proposition \ref{PropInterssiConn}), et que si $\mO_{x_1}$ n'intersectait personne, alors 
    \begin{equation}
        \begin{aligned}[]
            \mO_{x_1}&&\text{et}&&\bigcup_{i=2}^n\mO_{x_i}
        \end{aligned}
    \end{equation}
    forment une partition de $[a,b]$ en deux ouverts disjoints, ce qui n'est pas possible parce que $[a,b]$ est connexe. Nous nommons $\mA_2$, un des ouverts $\mO_{x_i}$ qui intersecte $\mA_1$. Disons que c'est $\mO_k$. Notons que $\mA_1\cup\mA_2$ est un intervalle sur lequel $f$ est strictement croissante. En effet, si $y_{12}$ est dans l'intersection, $f(a)<f(y_{12})$ parce que $f$ est strictement croissante sur $\mA_1$, et pour tout $x>y_{12}$ dans $\mA_2$, $f(x)>f(y_{12})$ parce que $f$ est strictement croissante dans $\mA_2$. 

    Maintenant, nous éliminons de la liste des $\mO_{x_i}$ tous ceux qui sont inclus à $\mA_1\cup\mA_2$. Dans ce qu'il reste, il y en a automatiquement un qui intersecte $\mA_1\cup\mA_2$, pour la même raison de connexité que celle invoquée plus haut. Nous appelons cet ouvert $\mA_3$, et pour la même raison qu'avant, $f$ est strictement croissante sur $\mA_1\cup\mA_2\cup\mA_3$.

    En recommençant suffisamment de fois, nous finissons par devoir prendre un des $\mO_{x_i}$ qui contient $b$, parce qu'au moins un des $\mO_{x_i}$ contient $b$. À ce moment, nous avons finit la démonstration.
\end{proof}

Il est intéressant de noter que ce théorème concerne la croissance d'une fonction sous l'hypothèse que la dérivée est positive. Il nous a fallu très peu de temps, en utilisant la positivité de la dérivée, pour conclure qu'autour de tout point, la fonction était strictement croissante. À partir de là, c'était pour ainsi dire gagné. Mais il a fallu un réel travail de topologie très fine\footnote{et je te rappelle que nous avons utilisé la proposition \ref{PropInterssiConn}, qui elle même était déjà un très gros boulot !} pour conclure. Étonnant qu'une telle quantité de topologie soit nécessaire pour démontrer un résultat essentiellement analytique dont l'hypothèse est qu'une limite est positive, n'est-ce pas ? 

Une petite facile, maintenant.
\begin{proposition}
    Si $f$ est croissante sur un intervalle, alors $f'\geq 0$ à l'intérieur cet intervalle, et si $f$ est décroissante sur l'intervalle, alors $f'\leq 0$ à l'intérieur de l'intervalle.
\end{proposition}

Note qu'ici, nous demandons juste la croissance de $f$, et non sa \emph{stricte} croissance.

\begin{proof}
    Soit $f$, une fonction croissante sur l'intervalle $I$, et $x$ un point intérieur de $I$. La dérivée de $f$ en $x$ vaut
    \begin{equation}
        f'(x)=\lim_{\epsilon\to 0}\frac{ f(x+\epsilon)-f(x) }{\epsilon},
    \end{equation}
    mais, comme $f$ est croissante sur $I$, nous avons toujours que $f(x+\epsilon)-f(x)\geq0$ quand $\epsilon>0$, et $f(x+\epsilon)-f(x)\leq0$ quand $\epsilon<0$, donc cette limite est une limite de nombre positifs ou nuls, qui est donc positive ou nulle. Cela prouve que $f'(x)\geq 0$.
\end{proof}

% http://fr.wikipedia.org/wiki/Théorème_de_Rolle
% http://gconnan.free.fr/les%20pdf/Deriv.pdf
Les deux prochains théorèmes sont très importants.
\begin{theorem}[\href{http://fr.wikipedia.org/wiki/Théorème_de_Rolle}{Théorème de Rolle}]       \label{ThoRolle}
    Soit $f$, une fonction continue sur $[a,b]$ et dérivable sur $]a,b[$. Si $f(a)=f(b)$, alors il existe un point $c\in]a,b[$ tel que $f'(c)=0$.
\end{theorem}

\begin{proof}
    Étant donné que $[a,b]$ est un intervalle compact, l'image de $[a,b]$ par $f$ est un intervalle compact, soit $[m,M]$ (théorème \ref{ThoImCompCotComp}). Si $m=M$, alors le théorème est évident : c'est que la fonction est constante, et la dérivée est par conséquent nulle. Supposons que $M> f(a)$ (il se peut que $M=f(a)$, mais alors si $f$ n'est pas constante, il faut avoir $m<f(a)$ et le reste de la preuve peut être adaptée).

    Comme $M$ est dans l'image de $[a,b]$ par $f$, il existe $c\in ]a,b[$ tel que $f(c)=M$. Considérons maintenant la fonction
    \begin{equation}
        \tau(x) =\frac{ f(c+x)-f(c) }{ x }.
    \end{equation}
    Par définition, $\lim_{x\to 0}\tau(x)=f'(c)$. Par hypothèse, si $u<c$,
    \begin{equation}
        \tau(u-c) = \frac{ f(u)-f(c) }{ u-c }>0
    \end{equation}
    parce que $u-c<0$ et $f(u)-f(c)<0$. Par conséquent, $\lim_{x\to 0}\tau(x)\geq 0$. Nous avons aussi, pour $v>c$,
    \begin{equation}
        \tau(v-c) = \frac{ f(v)-f(c) }{ v-c }<0
    \end{equation}
    parce que $v-c>0$ et $f(v)-f(c)<0$. Par conséquent, $\lim_{x\to 0}\tau(x)\leq 0$. Mettant les deux ensemble, nous avons $f'(c)=\lim_{x\to 0}\tau(x)=0$, et $c$ est le point que nous cherchions.
\end{proof}

Sur wikipédia, deux démonstrations complètement différentes sont proposées, celle qui est présentée ici est adaptée de celle qui est proposée par le célèste mathémator de \href{http://gconnan.free.fr/les\%20pdf/Deriv.pdf}{Téhessin le Rézéen}.

Le corollaire suivant est le théorème des \defe{accroissements finis}{théorème!accroissements finis!dans $\eR$}.

\begin{theorem}[Accroissements finis]       \label{ThoAccFinis}
    Si $f$ est une fonction continue sur $[a,b]$ et dérivable sur $]a,b[$, alors il existe au moins un réel $c\in]a,b[$ tel que $f(b)-f(a)=(b-a)f'(c)$.
\end{theorem}

\begin{proof}
    Considérons la fonction
    \begin{equation}
        \tau(x)=f(x)-\big( \frac{ f(b)-f(a) }{ b-a }x + f(a) - a\frac{ f(b)-f(a) }{ b-a } \big),
    \end{equation}
    c'est à dire la fonction qui donne la distance entre $f$ et le segment de droite qui lie $(a,f(a))$ à $(b,f(b))$. Par construction, $\tau(a)-\tau(b)=0$, donc le théorème de Rolle s'appliqe à $\tau$ pour laquelle il existe donc un $c\in]a,b[$ tel que $\tau'(c)=0$.

    En utilisant les règles de dérivation, nous trouvons que la dérivée de $\tau$ vaut
    \begin{equation}
        \tau'(x)= f'(x)-\frac{ f(b)-f(a) }{ b-a },
    \end{equation}
    donc dire que $\tau'(c)=0$ revient à dire que $f(b)-f(a)=(b-a)f'(c)$, ce qu'il fallait démontrer.
\end{proof}

\begin{corollary}
Soit $f$ une fonction dérivable sur $[a,b]$ telle que $f'(x) = 0$ pour tout $x \in [a,b]$. Alors $f$ est constante sur $[a,b]$.
\end{corollary}

\begin{proof}
    Si $f$ n'était pas constante sur $[a,b]$, il existerait un $x_1\in ]a,b[$ tel que $f(a)\neq f(x_1)$, et dans ce cas, il existerait un $c\in]a,x_1[$ tel que 
    \begin{equation}
        f'(c)=\frac{ f(x_1)-f(a) }{ x_1-a }\neq 0,
    \end{equation}
    ce qui contredirait les hypothèses.
\end{proof}

\begin{corollary}   \label{CorNErEgcQ}
    Soient $f$ et $g$, deux fonctions dérivables sur $[a,b]$ telles que
    \begin{equation}
        f'(x) = g'(x)
    \end{equation}
    pour tout $x \in [a,b]$. Alors existe un réel $C$ tel que $f (x) = g (x) + C$ pour tout $x\in [a,b]$.
\end{corollary}

\begin{proof}
    Considérons la fonction $h(x)=f(x)-g(x)$, dont la dérivée est, par hypothèse, nulle. L'annulation de la dérivée entraine par le corollaire \ref{CorNErEgcQ} que $h$ est  constante. Si $h(x)=C$, alors $f(x)=g(x)+C$, ce qu'il fallait prouver.
\end{proof}

\begin{definition}  \label{DefXVMVooWhsfuI}
    Soit \( I\) un intervalle ouvert de \( \eR\) et une fonction \( f\colon I\to \eR\). La fonction \( F\colon I\to \eR\) est une \defe{primitive}{primitive!fonction} de \( f\) si \( F\) est dérivable sur \( I\) et si \( F'(x)=f(x)\) pour tout \( x\) dans \( I\).
\end{definition}

Exprimé en termes des primitives, le corollaire \ref{CorNErEgcQ} signifie que
\begin{corollary}  \label{CorZeroCst}
    Si $F$ et $G$ sont deux primitives de la même fonction $f$ sur un intervalle, alors il existe une constante $C$ pour laquelle $F(x)=G(x)+C$.
\end{corollary}
Cela signifie qu'il n'y a, en réalité, pas des milliards de primitives différentes à une fonction. Il y en a essentiellement une seule, et puis les autres, ce sont juste les mêmes, mais décalées d'une constante.

\begin{remark}
    L'hypothèse de se limiter à un intervalle est importante parce que si on considère la fonction sur deux intervalles disjoints, nous pouvons choisir la constante indépendamment dans l'un et dans l'autre. Par exemple la fonction
    \begin{equation}
        F(x)=\begin{cases}
            \ln(x)+1    &   \text{si \( x>0\)}\\
            \ln(x)-7    &    \text{si \( x<0\)}
        \end{cases}
    \end{equation}
    est une primitive de \( \frac{1}{ x }\) sur l'ensemble \( \eR\setminus\{ 0 \}\).

    Certains ne s'en privent pas. Le logiciel \href{ http://sagemath.org }{ Sage } par exemple fait ceci :
    \begin{verbatim}
sage: f(x)=1/x
sage: F=f.integrate(x)
sage: A=F(x)-F(-x)
sage: A.full_simplify()
I*pi
    \end{verbatim}
    En réalité lorsque \( x>0\), Sage définit \( \ln(-x)=\ln(x)+i\pi\). Cela a une certaine logique parce que \( \ln(-1)=i\pi\) (du fait que \(  e^{i\pi}=-1\)), mais si on ne le sait pas, ça peut étonner.
\end{remark}

\begin{normaltext}
    Il existe plusieurs primitives à une fonction donnée. En physique, la constante arbitraire est souvent fixée par une condition initiale, comme nous le verrons dans la section \ref{SecMRUAsecondeGGdQoT}.
\end{normaltext}

%+++++++++++++++++++++++++++++++++++++++++++++++++++++++++++++++++++++++++++++++++++++++++++++++++++++++++++++++++++++++++++
\section{Différentiabilité}
%+++++++++++++++++++++++++++++++++++++++++++++++++++++++++++++++++++++++++++++++++++++++++++++++++++++++++++++++++++++++++++

Note : pour savoir des choses sur la différentielle de \( f\colon E\to F\) avec \( E\) et \( F\) de dimension infinie, il faut aller voir la section \ref{SecLStKEmc}. Ici nous ne parlerons que de dimension finie.

%---------------------------------------------------------------------------------------------------------------------------
\subsection{Le pourquoi et le comment de la dérivée}
%---------------------------------------------------------------------------------------------------------------------------

La notion de dérivée est associée à la recherche de la droite tangente à une courbe. Reprenons rapidement le cheminement. La dérivée de $f\colon \eR\to \eR$ au point $a$ est un nombre $f'(a)$, qui définit donc une application linéaire dont le coefficients angulaire est $f'(a)$, et que nous notons $df_a$ :
\begin{equation}
    \begin{aligned}
        df_a\colon \eR&\to \eR \\
        u&\mapsto f'(a)u. 
    \end{aligned}
\end{equation}
La droite donnée par l'équation
\begin{equation}
    y(a+u)=f'(a)u
\end{equation}
est parallèle à la tangente en $a$. Pour trouver la tangente, il suffit de la décaler de la hauteur qu'il faut. L'équation de la droite tangente au graphe de $f$ au point $\big( a,f(a) \big)$ devient
\begin{equation}        \label{EqDiffRapTgDer}
    y(x)=f(a)+f'(a)(x-a)=f(a)+df_a(x-a).
\end{equation}
Nous nous proposons de généraliser cette formule au cas de la recherche du plan tangent à une surface.
 
%---------------------------------------------------------------------------------------------------------------------------
                    \subsection{Dérivée partielle et directionnelles}
%---------------------------------------------------------------------------------------------------------------------------

Soit une fonction $f:A\subset \mathbb{R}^n \rightarrow \mathbb{R}^m$. Si $n\neq 1$, la notion de \emph{dérivée} de la fonction $f$ n'a plus de sens puisqu'on ne peut plus parler de pente de \emph{la} tangente au graphe de $f$ en un point. On introduit alors quelque notions qui feront, en dimension quelconque, le même travail que la dérivée en dimension un : les dérivées directionnelles et la différentielle. Nous allons voir qu'en dimension un, la différentielle coïncide avec la dérivée.


\begin{definition} 
    Soit un point $a \in int\,A$ et un vecteur $u \in \mathbb{R}^n$ avec $\| u \| =1$. La dérivée de $f$ au point $a$ dans la direction $u$ est donnée par la limite suivante, si elle existe 
    \begin{equation}
        \frac{\partial f}{\partial u}(a) = \lim_{t\rightarrow 0}\frac{f(a+tu) - f(a)}{t}
    \end{equation}
\end{definition}

Géométriquement, il s'agit du taux de variation instantané de $f$ en $a$ dans la direction du vecteur $u$, c'est-à-dire de la pente de la tangente dans la direction du vecteur $u$ au graphe de $f$ au point $(a, f(a))$.

\begin{remark}
On peut reformuler la définition en écrivant $x = a + u$, on obtient~:
\begin{equation}
    \limite[u\neq 0]{u}{0} \frac{f(a+u)-f(a)-T(u)}{\norme{u}} = 0.
\end{equation}
\end{remark}

\begin{remark}
Pourquoi avons-nous posé la condition $\| u \|=1$ ? Le but de la dérivée directionnelle dans la direction $u$ est de savoir à quelle vitesse la fonction monte lorsque l'on se déplace en suivant la direction $u$. Cette information n'aura un caractère \og objectif\fg{} que si l'on avance à une vitesse donnée. En effet, si on se déplace deux fois plus vite, la fonction montera deux fois plus vite. Par convention, nous demandons donc d'avancer à vitesse $1$.
\end{remark}

\subsubsection*{Cas particulier où $n=2$:} $a = (a_1, a_2)$, $u =
(u_1,u_2)$ et
$$\frac{\partial f}{\partial u}(a_1, a_2) = \lim_{t\rightarrow
0}\frac{f(a_1+tu_1,a_2+tu_2) - f(a_1, a_2)}{t}$$

Un cas particulier des dérivées directionnelles est la dérivée partielle. Si nous considérons la base canonique $e_i$ de $\eR^n$, nous notons
\begin{equation}
    \frac{ \partial f }{ \partial x_i }=\frac{ \partial f }{ \partial e_i }.
\end{equation}
Dans le cas d'une fonction à deux variables, nous avons donc les deux dérivées partielles
\begin{equation}
    \begin{aligned}[]
        \frac{ \partial f }{ \partial x }(a)&&\text{et}&&\frac{ \partial f }{ \partial y }(a)
    \end{aligned}
\end{equation}
qui correspondent aux dérivées directionnelles dans les directions des axes. Ces deux nombres représentent de combien la fonction $f$ monte lorsqu'on part de $a$ en se déplaçant dans le sens des axes $X$ et $Y$.

%///////////////////////////////////////////////////////////////////////////////////////////////////////////////////////////
                    \subsubsection{Quelque propriétés et notations}
%///////////////////////////////////////////////////////////////////////////////////////////////////////////////////////////

\begin{enumerate}
\item
 $\forall \alpha \in \mathbb{R}$,
si $v = \alpha\,u$, alors $\frac{\partial f}{\partial v}(a) =
\alpha\,\frac{\partial f}{\partial u}(a)$.
\item Si on prend $u=e_j$ le $j$ème vecteur de la base canonique de
$\mathbb{R}^n$, alors
$$\frac{\partial f}{\partial e_j}(a) = \frac{\partial f}{\partial
x_j}(a)$$ c'est-à-dire que la dérivée de $f$ au point $a$ dans la
direction $e_j$ est la dérivée partielle de $f$ par rapport à sa
$j$ème variable.

\item 
Une fonction peut être dérivable dans certaines directions
mais pas dans d'autres (rappelez vous que si la limite à droite est
différente de la limite à gauche, la limite n'existe pas). 

\item
Même si une fonction est dérivable en un point dans toutes les
directions, on n'est pas sûr qu'elle soit continue en ce point. La
dérivabilité directionnelle n'est donc pas une notion suffisante
pour assurer la continuité. C'est pourquoi on introduit le concept
de \emph{différentiabilité}. 
\end{enumerate}

%---------------------------------------------------------------------------------------------------------------------------
                    \subsection{Différentielle}
%---------------------------------------------------------------------------------------------------------------------------

\begin{definition}      \label{DefDifferentiablFnRn}
Soit un point $a \in int\,A$. La fonction $f$ est \defe{différentiable}{différentiable} au point $a$ si il existe une application linéaire $df_a\colon \eR^n\to \eR^m$ telle que 
\begin{equation}        \label{EqDefDiffableT}
    \lim_{x\to a} \frac{f(x) - f(a) - df_a (x-a)}{\|x-a\|}=0.
\end{equation}
\end{definition}

Si $f$ est différentiable en $a$, l'application $df_a$ est appelée la différentielle de $f$ en $a$. Voyons comment cette application linéaire agit sur les vecteurs de $\mathbb{R}^n$.

Le théorème suivant reprend pas principales propriétés d'une fonction différentiable.
\begin{theorem}     \label{ThoRapPropDiffSi}
Si $f$ est différentiable en $a\in\eR^n$, alors
\begin{enumerate}
\item $f$ est continue en $a$.

\item  Toute les dérivées directionnelles $\partial_uf(a)$ existent et nous avons l'égalité
\begin{equation}        \label{EqDiffPartRap}
    \begin{aligned}
        df_a\colon \eR^n&\to \eR^m \\
        u&\mapsto df_a(u)=\frac{ \partial f }{ \partial u }(a)=\sum_i \frac{ \partial f }{ \partial x_i }u^i,
    \end{aligned}
\end{equation}
si les $u^i$ sont les composantes de $u$ dans la base canonique de $\eR^n$.

La différentielle de $f$ en $a$ envoie donc un vecteur $u$ sur la dérivée directionnelle de $f$ au point $a$ dans la direction $u$. 

\item\label{ItemThoDiffSiLin} L'application $df_a$ est une application linéaire.
\end{enumerate}
\end{theorem}
Le point \ref{ItemThoDiffSiLin} est évidement contenu dans la définition de la différentielle, mais c'est bien de la remettre en toute lettres. En regard avec la formule \eqref{EqDiffPartRap}, elle dit que $\partial_uf(a)$ est linéaire par rapport à $u$.

\subsubsection*{Cas particuliers} \begin{description} \item $n=1$:
$f: \mathbb{R}\rightarrow \mathbb{R}$ est dérivable en $a$ si et
seulement si $f$ est différentiable en $a$ et
$$df_a:\mathbb{R}\rightarrow \mathbb{R}: x \mapsto df_a(x) =
f'(a)\,.\,x$$ \item $n=2$: $f$ est différentiable en $a =(a_1, a_2)$
si et seulement si
$$\lim_{(v_1,v_2)\rightarrow (0,0)} \frac{f(a_1+v_1, a_2+v_2) - f(a_1,a_2) - [ \frac{\partial f}{\partial x}(a)\,v_1+
\frac{\partial f}{\partial y}(a)\,v_2]}{\sqrt{v_1^2+v_2^2}} = 0
$$\end{description}\vspace{0.3cm}


Parmi les vecteurs $u \in \mathbb{R}^n$, un vecteur d'origine $(a,
f(a))$ se distingue des autres: le vecteur gradient de $f$ en $a$
donnant la direction de plus grande pente de $f$ en
$a$.\vspace{0.3cm}

\begin{definition}
La courbe de niveau de $f$ associée à a est donnée par
$$ S_a = f^{-1}\,(f(a)) = \{(x_1, \ldots, x_n)\in \mathbb{R}^n : f(x_1, \ldots,
x_n)=f(a) \}$$
\end{definition}

%---------------------------------------------------------------------------------------------------------------------------
\subsection{Règles de calcul}
%---------------------------------------------------------------------------------------------------------------------------

\begin{proposition}[Règles de calculs] Soient $f$ et $g$ des fonctions
  différentiables en $g(a)$ et $a$ respectivement, alors la composée
  $f\circ g$ est différentiable en $a$ et
  \begin{equation*}
    d (f\circ g)_a = d f_{g(a)} \circ d g_a
  \end{equation*}
  et de plus les jacobiennes correspondantes vérifient
  \begin{equation*}
      J_{f\circ g}(a) = J_f\big( g(a) \big)J_g(a)
  \end{equation*}
  où le membre de droite est le produit (non-commutatif !) des deux matrices.
\end{proposition}

\begin{corollary}[Chain rule] Si $f : \eR^p \to \eR$ et $g : \eR \to
  \eR^p$, alors
  \begin{equation*}
    (f\circ g)^\prime(t) = \sum_{i=1}^p \pder f {x_i}(g(t)) g_i^\prime(t).
  \end{equation*}
\end{corollary}

\begin{remark}
  \begin{enumerate}
  \item Si $p = 1$, on retrouve la règle usuelle de dérivation de
    fonctions composées.

  \item 
      Si $g$ est à plusieurs variables, cette règle permet de déterminer les dérivées partielles de $f \circ g$, puisqu'une dérivée partielle peut être vue comme dérivée usuelle par rapport à une seule variable (voir remarque page \pageref{deriveepartielles}).

  \item Si $f$ est à valeurs vectorielles, cette formule permet de
    retrouver la jacobienne de $f \circ g$ puisqu'il suffit de traiter
    chaque composante de $f$ séparément.
  \end{enumerate}
\end{remark}

%---------------------------------------------------------------------------------------------------------------------------
                    \subsection{Gradient et recherche du plan tangent}
%---------------------------------------------------------------------------------------------------------------------------

Nous avons maintenant en main les concepts utiles pour trouver l'équation du plan tangent à une surface.

De la même manière que la tangente à une courbe était la droite de coefficient directeur donné par la dérivée, maintenant, le plan tangent à une surface est le plan dont les vecteurs directeurs sont les dérivées partielles :

La généralisation de l'équation \eqref{EqDiffRapTgDer} est 
\begin{equation}        \label{EqDefPlanTag}
    T_a(x)=f(a)+\sum_i\frac{ \partial f }{ \partial x_i }(a)(x-a)^i
\end{equation}

Nous introduisons aussi souvent l'opérateur différentiel abstrait \defe{nabla}{nabla}, noté $\nabla$ et qui est donné par le vecteur
\begin{equation}
    \nabla=\left( \frac{ \partial  }{ \partial x_1 },\ldots,\frac{ \partial  }{ \partial x_n } \right).
\end{equation}
Les égalités suivantes sont juste des notations, sommes toutes logiques, liées à $\nabla$ :
\begin{equation}
    \nabla f=\left( \frac{ \partial f }{ \partial x_1 },\ldots,\frac{ \partial f }{ \partial x_n } \right),
\end{equation}
et
\begin{equation}        \label{EqDefGradient}
    \nabla f(a) = \left(\frac{\partial f}{\partial x_1}(a), \frac{\partial f}{\partial x_2}(a), \ldots, \frac{\partial f}{\partial x_n}(a)\right).
\end{equation}
Ce dernier est un élément de $\eR^n$ : chaque entrée est un nombre réel.

\begin{definition} 
Le vecteur gradient de $f$ au point $a$ est le vecteur donné par la formule \eqref{EqDefGradient}.
\end{definition}
La notation $\nabla$ permet d'écrire la différentielle sous forme un peu plus compacte. En effet, la formule \eqref{EqDiffPartRap} peut être notée
\begin{equation}
    df_a(u)=\scal{\nabla f(a)}{u}.
\end{equation}

En utilisant ce produit scalaire, l'équation \eqref{EqDefPlanTag} peut se récrire
\begin{equation}
    T_a(x)=f(a)+\sum_i\frac{ \partial f }{ \partial x_i }(a)(x-a)^i=f(a)+\scal{\nabla f(a)}{x-a}.
\end{equation}

Afin d'éviter les confusions, il est parfois souhaitable de bien mettre les parenthèses et noter $(\nabla f)(a)$ au lieu de $\nabla f(a)$.

\begin{proposition}
$\nabla f(a)\,\bot \,S_a$
\end{proposition}


\begin{equation}        \label{EqPlanTgSansNabla}
    z=f(a)+\sum_i\frac{ \partial f }{ \partial f }(a)(x-a)^i.
\end{equation}

\subsubsection*{Cas particulier où $n=2$:} 
Le plan $T_a$ avec $a=(a_1,a_2)$ a pour équation dans $\eR^3$:
\begin{equation}        \label{EqPlanTgEnDimDeux}
    z = f(a_1,a_2) + \frac{\partial f}{\partial x}(a_1,a_2)\,(x-a_1)+ \frac{\partial f}{\partial y}(a_1,a_2)\,(y-a_2).
\end{equation}

\begin{definition}
  Soit $f : \eR^n \to\eR$ une fonction différentiable en un point
  $a$. Le \emph{plan tangent} au graphe de $f$ en $(a,f(a))$ est
  l'ensemble des points
  \begin{equation*}
    \begin{split}
      T_af &= \{ (x,z) \in \eR^n \times \eR \tq z = f(a) + d f_a (x-a)\}\\
      &= \{ (x,z) \in \eR^n \times \eR \tq z = f(a) + \scalprod{\nabla f(a)}{x-a}\}
    \end{split}
  \end{equation*}
\end{definition}

%---------------------------------------------------------------------------------------------------------------------------
                    \subsection{Différentielle comme élément de l'espace dual}
%---------------------------------------------------------------------------------------------------------------------------

Si nous considérons la base canonique $\{ e_i \}_{i=1,\ldots,n}$ de $\eR^n$. À partir d'elle, nous considérons la \defe{base duale}{base!duale}. En termes pratiques, nous définissons $dx_i$ comme la forme sur $\eR^n$ qui à un vecteur $u$ fait correspondre sa composante $i$ :
\begin{equation}
    dx_i\begin{pmatrix}
    u^1 \\ 
    \vdots  \\ 
    u^n 
\end{pmatrix}=u^i.
\end{equation}
En termes savants, $dx_i$ est le dual de $e_i$. Si tu ne l'as pas encore compris, Jean Doyen va te le faire comprendre !


Maintenant, dans la formule \eqref{EqDiffPartRap}, nous pouvons remplacer $u^i$ par $dx_i(u)$, et écrire
\begin{equation}
    df_a(u)=\sum_i\frac{ \partial f }{ \partial x_i }(a)u^i=\sum_i\frac{ \partial f }{ \partial x_i }(a)dx_i(u).
\end{equation}
Ce qui arrive tout à droite est explicitement vu comme une forme sur $\eR$, dont les composantes dans la base duale sont les dérivées partielles de $f$ au point $a$, agissant sur $u$. En faisant un pas en arrière, nous omettons le $u$, et nous écrivons
\begin{equation}
    df_a=\sum_{i=1}^n\frac{ \partial f }{ \partial x_i }(a)dx^i
\end{equation}

Cette notation $dx_i$ pour la forme duale de $e_i$ est en réalité parfaitement logique parce que $dx^i$ est la différentielle de la projection
\begin{equation}
    \begin{aligned}
        x^i\colon \eR^n&\to \eR \\
        (x^1,\ldots,x^n)&\mapsto x^i. 
    \end{aligned}
\end{equation}
Je te laisse un peu méditer sur cette différentielle de la projection. L'important est que tu aies compris cela d'ici la fin de ta deuxième année.


%---------------------------------------------------------------------------------------------------------------------------
                    \subsection{Prouver qu'un fonction n'est pas différentiable}
%---------------------------------------------------------------------------------------------------------------------------

Chacun des point du théorème \ref{ThoRapPropDiffSi} est en soi un critère pour montrer qu'une fonction n'est pas différentiable en un point.

%///////////////////////////////////////////////////////////////////////////////////////////////////////////////////////////
                    \subsubsection{Continuité}
%///////////////////////////////////////////////////////////////////////////////////////////////////////////////////////////


Le premier critère à vérifier est donc la continuité. Si une fonction n'est pas continue en un point, alors elle n'y sera pas différentiable. Pour rappel, la continuité en $a$ se teste en vérifiant si $\lim_{x\to a}f(x)=f(a)$.

%///////////////////////////////////////////////////////////////////////////////////////////////////////////////////////////
                    \subsubsection{Linéarité}
%///////////////////////////////////////////////////////////////////////////////////////////////////////////////////////////

Un second test est la linéarité de la dérivée directionnelle par rapport à la direction : l'application $u\mapsto\frac{ \partial f }{ \partial u }(a)$ doit être linéaire, sinon $df_a$ n'existe pas.

\begin{example}     \label{Exemple0046Diff}
Examinons la fonction
\begin{equation}
    \begin{aligned}
        f\colon \eR^2&\to \eR \\
        (x,y)&\mapsto \begin{cases}
    \frac{ xy^2 }{ x^2+y^4 }    &   \text{si $(x,y)\neq (0,0)$}\\
    0   &    \text{sinon}.
\end{cases}
    \end{aligned}
\end{equation}
Prenons $u=(u_1,u_2)$ et calculons la dérivée de $f$ dans la direction de $u$ au point~$(0,0)$ :
\begin{equation}
    \begin{aligned}[]
        \frac{ \partial f }{ \partial u }(0,0)  
            &=\lim_{t\to 0}\frac{ f(tu_1,tu_2)-f(0,0) }{ t }\\
            &=\lim_{t\to 0}\frac{1}{ t }\left( \frac{ tu_1t^2u_2 }{ t^2u_1^2+t^4u_2^4 } \right)\\
            &=\lim_{t\to 0}\left( \frac{ u_1u_2^2 }{ u_1^2+t^2u_2^4 } \right)\\
            &=\begin{cases}
    \frac{ u_2^2 }{ u_1 }   &   \text{si $u_1\neq 0$}\\
    0   &    \text{si $u_1=0$}.
\end{cases}
    \end{aligned}
\end{equation}
Cette application n'est pas linéaire par rapport à $u$. En effet, notons
\begin{equation}
    \begin{aligned}
        A\colon \eR^n&\to \eR \\
        u&\mapsto \frac{ \partial f }{ \partial u }(0,0), 
    \end{aligned}
\end{equation}
et vérifions que pour tout $u$ et $v$ dans $\eR^n$ et $\lambda\in\eR$, nous ayons $A(\lambda u)=\lambda A(u)$ et $A(u+v)=A(u)+A(v)$. Le premier fonctionne parce que
\begin{equation}
    A(\lambda u)=A(\lambda u_1,\lambda u_2)=\frac{ \lambda^2 u_2^2 }{ \lambda u_1 }=\lambda\frac{ u_2^2 }{ u_1 }=\lambda A(u).
\end{equation}
Mais nous avons par exemple
\begin{equation}
    A\big( (0,1)+(2,3) \big)=A(2,4)=\frac{ 16 }{ 2 }=8,
\end{equation}
tandis que
\begin{equation}
    A(0,1)+A(2,3)=0+\frac{ 9 }{ 2 }\neq 8.
\end{equation}
La fonction $f$ n'est donc pas différentiable en $(0,0)$, parce que la candidate différentielle, $df_{(0,0)}(u)=\frac{ \partial f }{ \partial u }(0,0)$, n'est même pas linéaire.

\end{example}

Voici une autre façon de traiter la fonction de l'exemple \ref{Exemple0046Diff}.

\begin{example} \label{ExeFHmCLII}
    La figure \ref{LabelFigFWJuNhU} représente le domaine d'une fonction $f\colon \eR^2\to \eR$, et sur chacune des parties, elle est définie différemment.
    \newcommand{\CaptionFigFWJuNhU}{La fonction de l'exemple \ref{ExeFHmCLII}.}
\input{Fig_FWJuNhU.pstricks}

L'expression de $f$ est ici
\begin{equation}
  f(x,y) =
  \begin{cases}
    xy & \text{si $x < 0$ et $y > 0$}\\
    x-y & \text{si $x \geq 0$ et $y \geq 0$}\\
    x^2y & \text{si $x > 0$ et $y < 0$}\\
    x+y & \text{sinon.}
  \end{cases}
\end{equation}

On note que les deux axes forment une zone à problèmes. La zone hors
des axes est un ouvert sur lequel $f$ est différentiable car composée
de polynômes. Analysons chacun des points de la forme $(a,b)$ dans la
zone à problèmes (c'est-à-dire si $ab = 0$).

\subparagraph{Si $a = 0$ et $b > 0$} Un tel point $(0,b)$ est sur
l'axe verticale, dans la moitié supérieure. Pour calculer la limite de
$f$ en ce point, on peut restreindre notre étude au demi-plan ouvert
$y > 0$, ce qui revient à comparer la limite
\begin{equation*}
  \limite[y>0\\x\geq 0] {(x,y)} {(0,b)} f(x,y) =   \limite[y>0\\x\geq
  0] {(x,y)} {(0,b)} x-y = 0 - b = -b
\end{equation*}
avec la limite
\begin{equation*}
  \limite[y>0\\x<0] {(x,y)} {(0,b)} f(x,y) =   \limite[y>0\\x<0]
  {(x,y)} {(0,b)} xy = 0 b = 0
\end{equation*}
qui sont différentes puisque $b$ est supposé non-nul.

\conclusion $f$ n'est pas continue en un point du type $(0,b)$ avec $b
> 0$.

\subparagraph{Si $a = 0$ et $b < 0$} Un tel point $(0,b)$ est sur
l'axe verticale, dans la moitié inférieure. Pour calculer la limite de
$f$ en ce point, on peut restreindre notre étude au demi-plan ouvert
$y < 0$, ce qui revient à comparer la limite
\begin{equation*}
  \limite[y<0\\x\geq 0] {(x,y)} {(0,b)} f(x,y) =   \limite[y<0\\x\geq
  0] {(x,y)} {(0,b)} x^2 y = 0^2 b = 0
\end{equation*}
avec la limite
\begin{equation*}
  \limite[y<0\\x<0] {(x,y)} {(0,b)} f(x,y) =   \limite[y<0\\x<0]
  {(x,y)} {(0,b)} x+y = 0 + b = b
\end{equation*}
qui sont différentes puisque $b$ est supposé non-nul.

\conclusion $f$ n'est pas continue en un point du type $(0,b)$ avec $b
< 0$.

\subparagraph{Si $a > 0$ et $b = 0$} Un tel point $(a,0)$ est sur
l'axe horizontal, dans la moitié droite. Pour calculer la limite de
$f$ en ce point, on peut restreindre notre étude au demi-plan ouvert
$x > 0$, ce qui revient à comparer la limite
\begin{equation*}
  \limite[x>0\\y \geq 0] {(x,y)} {(a,0)} f(x,y) =   \limite[x>0\\y \geq
  0] {(x,y)} {(a,0)} x-y = a - 0 = a
\end{equation*}
avec la limite
\begin{equation*}
  \limite[x>0\\y < 0] {(x,y)} {(a,0)} f(x,y) =   \limite[x>0\\y < 0]
  {(x,y)} {(a,0)} x^2y = a^2 0 = 0
\end{equation*}
qui sont différentes puisque $a$ est supposé non-nul.

\conclusion $f$ n'est pas continue en un point du type $(a,0)$ avec $a
> 0$.

\subparagraph{Si $a < 0$ et $b = 0$} Un tel point $(a,0)$ est sur
l'axe horizontal, dans la moitié gauche. Pour calculer la limite de
$f$ en ce point, on peut restreindre notre étude au demi-plan ouvert
$x < 0$, ce qui revient à comparer la limite
\begin{equation*}
  \limite[x<0\\y> 0] {(x,y)} {(a,0)} f(x,y) =   \limite[x<0\\y>
  0] {(x,y)} {(a,0)} x y = a 0 = 0
\end{equation*}
avec la limite
\begin{equation*}
  \limite[x<0\\y\leq 0] {(x,y)} {(a,0)} f(x,y) =   \limite[x<0\\y\leq0]
  {(x,y)} {(a,0)} x+y = a + 0 = a
\end{equation*}
qui sont différentes puisque $a$ est supposé non-nul.

\conclusion $f$ n'est pas continue en un point du type $(a,0)$ avec $a
< 0$.

\subparagraph{Si $a = 0$ et $b = 0$} Le cas du point $(0,0)$ est
particulier, puisque il est adhérent aux quatre composantes du
domaine où la fonction est définie différemment. Pour étudier la
continuité, il faut donc étudier quatre limites. Ces limites ont déjà
été étudiées ci-dessus et valent toutes $0$, ce qui prouve la
continuité de $f$ en $(0,0)$.

En ce qui concerne la différentiabilité, on sait qu'il est nécessaire
que toutes les dérivées directionnelles existent. Calculons la dérivée
dans la direction $(0,1)$ (au point $(0,0)$)~:
\begin{equation*}
  \limite[t\neq0] t 0 \frac{f((0,0) + t(0,1)) - f(0,0)}{t} =%
  \limite[t\neq0] t 0 \frac{f(0,t)}{t} = \ldots
\end{equation*}
qu'on sépare en deux cas, car $f(0,t)$ possède une formule différente
si $t < 0$ ou si $t \geq 0$~:
\begin{equation*}
  \limite[t\neq0] t 0 \frac{f(0,t)}{t} = %
  \begin{arrowcases}
    \limite[t<0] t 0 \frac{f(0,t)}{t} = \limite[t<0] t 0 \frac{0+t}{t} = 1\\
    \limite[t\geq0] t 0 \frac{f(0,t)}{t} = \limite[t\geq0] t 0
    \frac{0-t}{t} = -1
  \end{arrowcases}
\end{equation*}
ce qui prouve que la limite n'existe pas, donc que la dérivée
directionnelle n'existe pas, et finalement que la fonction n'est pas
différentiable.

\conclusion La fonction donnée est continue hors des axes et au point
$(0,0)$, mais discontinue partout ailleurs sur les axes. Elle est
différentiable hors des axes, mais ne l'est pas sur les axes.

\end{example}

%///////////////////////////////////////////////////////////////////////////////////////////////////////////////////////////
                    \subsubsection{Cohérence des dérivées partielles et directionnelle}
%///////////////////////////////////////////////////////////////////////////////////////////////////////////////////////////

Dans la pratique, nous pouvons calculer $\partial_uf(a)$ pour une direction $u$ générale, et puis en déduire $\partial_xf$ et $\partial_yf$ comme cas particuliers en posant $u=(1,0)$ et $u=(0,1)$. Une chose incroyable, mais pourtant possible est qu'il peut arriver que
\begin{equation}
    \frac{ \partial f }{ \partial u }(a)\neq \sum_i\frac{ \partial f }{ \partial x_i }(a)u^i.
\end{equation}
Ceci se produit lorsque $f$ n'est pas différentiable en $a$. En voici un exemple.

%///////////////////////////////////////////////////////////////////////////////////////////////////////////////////////////
                    \subsubsection{Un candidat dans la définition (marche toujours)}
%///////////////////////////////////////////////////////////////////////////////////////////////////////////////////////////

Lorsqu'une fonction est donné, un candidat différentielle au point $(a_1,a_2)$ est souvent assez simple à trouver en un point :
\begin{equation}
    T(u_1,u_2)=\frac{ \partial f }{ \partial x }(a_1,a_2)u_1+\frac{ \partial f }{ \partial y }(a_1,a_2)u_2.
\end{equation}
L'application $T$ est la candidate différentielle en ce sens que si la différentielle existe, alors elle est égale à $T$. Ensuite, il faut vérifier si
\begin{equation}        \label{EqLimDefDiff}
    \lim_{(x,y)\to (a_1,a_2)} \frac{f(x,y) - f(a_1,a_2) - T\big( (x,y)-(a_1,a_2) \big)}{\| (x,y)-(a_1,a_2) \|}=0
\end{equation}
ou non. Si oui, alors la différentielle existe et $df_{(a,b)}(u)=T(u)$, sinon\footnote{y compris si la limite \eqref{EqLimDefDiff} n'existe même pas.}, la différentielle n'existe pas.

Attention : dans la ZAP, les dérivées partielles $\partial_xf$ et $\partial_yf$ ne peuvent en général pas être calculées en utilisant les règles de calcul (c'est bien pour ça que la ZAP est une zone à problèmes). Il faut d'office utiliser la définition
\begin{equation}
    \frac{ \partial f }{ \partial x }(a_1,a_2)=\lim_{t\to 0}\frac{ f(a_1+t,a_2)-f(a_1,a_2) }{ t },
\end{equation}
et la définition correspondante pour $\partial_yf$.


\subsubsection*{Conclusion}
Soient $f:A\subset \eR^n \rightarrow \eR^m$, et $a\in int\,A$. Si $f$ est différentiable en $a$, $$ (df_a (e_j))_i = d(f_i)_a(e_j) =\frac{\partial f_i}{\partial x_j}(a)= [Jac(f)_{|a}]_{ij}$$ et la matrice de l'application linéaire $df_a$ est la matrice jacobienne $m\times n$ de $f$ en $a$ notée $Jac(f)_{|a}$.


%---------------------------------------------------------------------------------------------------------------------------
                    \subsection{Calcul de différentielles}
%---------------------------------------------------------------------------------------------------------------------------


\begin{remark}      \label{deriveepartielles}
  En pratique, ayant une formule pour la fonction $f$, on dérive --grâce aux règles usuelles de dérivation-- par rapport à la variable $x_i$ en considérant que les autres ($x_j$ avec $j \neq i$) sont des constantes.
\end{remark}

\begin{example}Pour $f(x,y) = xy + x^2$, les dérivées partielles
  s'écrivent
  \begin{equation*}
    \frac{\partial f}{\partial x} = y + 2x \quad\text{et}\quad \frac{\partial f}{\partial y} = x
  \end{equation*}
\end{example}


Des \emph{règles de calcul} sont d'application. En particulier, quand
ces opérations existent, les sommes, différences, produits, quotients
et compositions d'applications différentiables sont différentiables.

Toute application linéaire est différentiable, et sa différentielle en
tout point est égale à l'application elle-même. En particulier, les
\Defn{projections canoniques}, c'est-à-dire les applications du type
$(x,y,z) \mapsto y$, sont linéaires donc différentiables.

\begin{example}
Les cas suivants sont faciles :
  \begin{enumerate}
  \item En combinant les projections canoniques avec les règles de
    calculs, on obtient que toute fonction polynômiale à $n$ variables
    est différentiable comme application de $\eR^n$ dans $\eR$.

  \item Toute fonction rationnelle, du type $f(x) \pardef
    \frac{P(x)}{Q(x)}$ où $P$ et $Q$ sont des polynômes, est
    différentiable en tout point $a$ tel que $Q(a) \neq 0$.

  \item Pour une fonction d'une variable $f : D \subset \eR \to
    \eR$, le caractère différentiable et le caractère dérivable
    coïncident. De plus, on a
    \begin{equation*}
      d f_a(u) = f'(a) u.
    \end{equation*}
  \end{enumerate}
\end{example}

%---------------------------------------------------------------------------------------------------------------------------
                    \subsection{Notes idéologiques quant au concept de plan tangent}
%---------------------------------------------------------------------------------------------------------------------------
\label{ssecConceptPlanTag}

Notons $G$, le graphe d'une fonction $f$, c'est à dire
\begin{equation}
    G=\{ (x,y,z)\in\eR^3\tq z=f(x,y) \}.
\end{equation}
Première affirmation : si $\gamma\colon \eR\to G$ est une courbe telle que $\gamma(0)=\big( a,f(a) \big)$, alors $\gamma'(0)\in\eR^n$ est dans le plan tangent à $G$ au point $\big( a,f(a) \big)$.

Plus fort : tous les éléments du plan tangent sont de cette forme.

Le plan tangent à $G$ en un point $x\in G$ est donc constitué des vecteurs vitesse de tous les chemins qui passent par $x$.

Prenons maintenant $S$, une courbe de niveau de $G$, c'est à dire
\begin{equation}
    S=\{ (x,y)\in\eR^2\tq f(x,y)=C \}.
\end{equation}
Si nous prenons un chemin dans $G$ qui est, de plus, contraint à $S$, c'est à dire tel que $\gamma(t)\in S$, alors $\gamma'(0)$ sera tangent à $G$ (ça, on le savait déjà), mais en plus, $\gamma'(0)$ sera tangent à $S$, ce qui est logique.

La morale est que si vous prenez un chemin qui se ballade dans n'importe quoi, alors la dérivée du chemin sera un vecteur tangent à ce n'importe quoi.

En outre, si $\gamma(t)\in S$ et $\gamma(0)=a$, alors
\begin{equation}
    \scal{\nabla f(a)}{\gamma'(0)}=0,
\end{equation}
c'est à dire que le vecteur tangent à la courbe de niveau est perpendiculaire au gradient. Cela est intuitivement logique parce que la tangente à la courbe de niveau correspond à la direction de \emph{moins} grande pente.


%+++++++++++++++++++++++++++++++++++++++++++++++++++++++++++++++++++++++++++++++++++++++++++++++++++++++++++++++++++++++++++
\section{Jacobienne}
%+++++++++++++++++++++++++++++++++++++++++++++++++++++++++++++++++++++++++++++++++++++++++++++++++++++++++++++++++++++++++++

\subsection{Rappels et définitions}

Dans cette section nous considérons des fonctions $f : D \to \eR^m$
où $D \subset \eR^n$, et un point $a \in \Int D$ où $f$ est
différentiable.
\begin{remark}
  La définition de continuité (resp. différentiabilité) pour une
  fonction à valeurs vectorielles est celle introduite précédemment,
  et on remarque que pour avoir la continuité
  (resp. différentiabilité) de $f$ en un point, il faut et il suffit
  de chacune des composantes de $f = (f_1,\ldots, f_m)$, vues
  séparément comme fonctions à $n$ variables et à valeurs réelles,
  soit continue (resp. différentiable) en ce point.
\end{remark}

\begin{definition}
    La \defe{jacobienne}{matrice!jacobienne} de $f$ en $a$ est la matrice de l'application linéaire donnée par la différentielle. Elle a de nombreuses notations
  \begin{equation}
      J_f(a) = \frac{ \partial (f_1,\ldots, f_m) }{ \partial x_1,\ldots, x_m }=
    \begin{pmatrix}
      \pder {f_1} {x_1}(a) & \ldots &\pder {f_1} {x_n}(a)\\
      \vdots& & \vdots\\
      \pder {f_m} {x_1}(a) & \ldots &\pder {f_m} {x_n}(a)
    \end{pmatrix}
  \end{equation}
  Autrement dit, c'est la matrice composée de l'ensemble des dérivées partielles de $f$. Le \defe{jacobien}{jacobien} de \( f\) au point \( a\) est le déterminant de cette matrice.

  Si $m = 1$, cette matrice ne contient qu'une ligne ; c'est donc un vecteur appelé le \defe{gradient}{gradient} de $f$ au point $a$ et noté $\nabla f(a)$.
\end{definition}

\begin{remark}
  \begin{enumerate}
  \item Si la fonction est supposée différentiable, calculer la
    jacobienne revient à connaître la différentielle. En effet, par
    linéarité de la différentielle et par définition des dérivées
    partielles, nous avons
    \begin{equation*}
      d f_a (u) =%
      \begin{pmatrix}
        \pder {f_1} {x_1}(a) & \ldots &\pder {f_1} {x_n}(a)\\
        \vdots& & \vdots\\
        \pder {f_m} {x_1}(a) & \ldots &\pder {f_m} {x_n}(a)
      \end{pmatrix}
      \begin{pmatrix}u_1\\\vdots\\u_n\end{pmatrix}
    \end{equation*}
    où $u = (u_1, \ldots, u_n)$ et où le membre de droite est un
    produit matriciel

  \item Remarquons que la jacobienne peut exister en un point donné
    sans que la fonction soit différentiable en ce point !
  \end{enumerate}
\end{remark}

%+++++++++++++++++++++++++++++++++++++++++++++++++++++++++++++++++++++++++++++++++++++++++++++++++++++++++++++++++++++++++++
\section{Fonctions à valeurs dans $\eR^n$}
%+++++++++++++++++++++++++++++++++++++++++++++++++++++++++++++++++++++++++++++++++++++++++++++++++++++++++++++++++++++++++++

À peu près toutes les notions que vous connaissez à propos de fonctions de $\eR$ dans $\eR$ se généralises immédiatement au cas de fonctions de $\eR$ dans $\eR^n$.

Nous disons que la fonction $f\colon \eR\to \eR^n$ est \defe{de classe $C^1$}{classe $C^1$} si chacune de ses composantes $f_i$ est de classe $C^1$ en tant que fonctions de $\eR$ dans $\eR$.

La dérivée de $f$ est donnée par la dérivée composante par composante. Pour l'intégrale de $f$, il en va de même : composante par composante. 
\begin{equation}
	\int f(x)dx=\big(  \int f_1(x)dx,\,\int f_2(x)dx,\ldots,\int f_n(x)dx   \big).
\end{equation}

Par exemple si nous considérons le mouvent d'une particule sur une hélice, la position est donnée par
\begin{equation}
	f(t)=\big( R\sin(t),R\cos(t),t \big).
\end{equation}
La vitesse est donnée par
\begin{equation}
	f'(t)=\big( R\cos(t),-R\sin(t),1 \big),
\end{equation}
et l'intégrale sera donnée par
\begin{equation}
	\int f(t)dt=\big( -R\cos(t)+C_1,R\sin(t)+C_2,\frac{ t^2 }{ 2 }+C_3 \big).
\end{equation}

Si nous considérons une pierre lancée horizontalement du sommet d'une falaise avec une vitesse initiale $v_0$, la vitesse de la pierre sera donnée par
\begin{equation}
	v(t)=(v_0,gt).
\end{equation}
Pour trouver la position, nous intégrons la vitesse par rapport au temps :
\begin{equation}
	f(t)=\int v(t)dt=\big( v_0t+C_1,\frac{ gt^2 }{ 2 }+C_2 \big).
\end{equation}
Notez qu'il faut une constante d'intégration différente pour chaque composantes.

\begin{lemma}			\label{LemIneqnormeintintnorm}
	Pour toute fonction $u\colon \mathopen[ a , b \mathclose]\to \eR^n$, nous avons
	\begin{equation}
		\| \int_a^bu(t)dt\|\leq\int_a^b\| u(t) \|dt
	\end{equation}
	pourvu que le membre de gauche ait un sens.
\end{lemma}

\begin{proof}
	Étant donné que $\int_a^bu(t)dt$ est un élément de $\eR^n$, par la proposition \ref{LemSclNormeXi}, il existe un $\xi\in\eR^n$ de norme $1$ tel que
	\begin{equation}
		\| \int_a^bu(t)dt \|=\xi\cdot\int_a^b u(t)dt=\int_a^b u(t)\cdot\xi dt\leq\int_a^b\| u(t) \|   \| \xi \|=\int_a^b\| u(t) \|dt.
	\end{equation}
\end{proof}
% This is part of Géométrie analytique
% Copyright (c) 2010-2011
%   Laurent Claessens
%   Carlotta Donadello
% See the file fdl-1.3.txt for copying conditions.

%+++++++++++++++++++++++++++++++++++++++++++++++++++++++++++++++++++++++++++++++++++++++++++++++++++++++++++++++++++++++++++
\section{Graphes de fonctions de plusieurs variables}		\label{SecGraphesFonc}
%+++++++++++++++++++++++++++++++++++++++++++++++++++++++++++++++++++++++++++++++++++++++++++++++++++++++++++++++++++++++++++

La plus grande partie de ce cours est consacrée à l'étude des fonction de plusieurs variables. Nous allons maintenant donner quelques indication sur comment <<dessiner>> une telle fonction. Vous connaissez déjà la définition de graphe pour une fonction $f$ d'une seule variable à valeurs dans $\eR$ : c'est l'ensemble des point du plan de la forme $(x, f(x))$. Vous voyez que cet ensemble n'est pas vraiment un gros morceau de $\eR^2$ parce que son intérieur est vide : il y a une seule valeur de $f$ qui correspond au point $x$, donc une boule de $\eR^2$ centrée en $(x, f(x))$ de n'importe quel rayon contient toujours des points qui ne font pas partie du graphe de $f$. 

%La première chose qu'on a envie de dire est que un tel graphe est une courbe dans $\eR^2$ mais cela n'est pas toujours vrai. Le graphe de la fonction cosinus est bien une courbe dans dans le plan, mais le graphe de la fonction tangente est une réunion infinie de courbes. Ce qui est vrai est que le graphe d'une fonction d'une variable est \emph{localement} une courbe si la fonction n'est pas trop mal choisie. % exemple? 

Nous voulons donner une définition assez générale pour le graphe d'une fonction
\begin{definition}
  Soit $f$ une fonction de $\eR^m$ dans $\eR^n$. Le \defe{graphe}{graphe!fonction} de $f$ est la partie de $\eR^m\times \eR^n$ de la forme
  \begin{equation}
    \Graph f= \{ (x,y)\in \eR^m\times \eR^n \,|\, y=f(x)\}.
  \end{equation}
\end{definition}
Si $f$ est une fonction de deux variables indépendantes $x$ et $y$ à valeurs dans $\eR$, alors un point dans le graphe de $f$ est un point $(x,y,z)\in\eR^3$ tel que
\begin{equation}
	z=f(x,y),
\end{equation}
ou encore, un point de la forme
\begin{equation}
	\big( x,y,f(x,y) \big).
\end{equation}
%Si $g$ est une fonction d'une variable $x$ à valeurs dans $\eR^2$, alors un point dans le graphe de $g$ prend la forme $(x,g_1(x), g_2(x))$, où $g_1$ et $g_2$ sont les composantes de $g$.  Dans le deux cas le graphe est un sous-ensemble de $\eR^3$. 
Ici nous sommes intéressés par les fonctions de plusieurs variables à valeurs dans $\eR$. Donc, notre définition se spécialise 
\begin{definition}
  Soit $f$ une fonction de $\eR^m$ dans $\eR$. Le graphe de $f$ est la partie de $\eR^m\times \eR$ donné par
  \begin{equation}
    \Graph f= \{ (x,y)\in \eR^m\times \eR \,|\, y=f(x)\}.
  \end{equation}
\end{definition}  
Étant donné que nous ne donneront des exemples que de fonctions de $\eR^2$ dans $\eR$, la définition devient
\begin{equation}
	\Graph f= \{ (x,y,z)\in\eR^2\tq z=f(x,y) \}.
\end{equation}
C'est cette définition qu'il faut garder à l'esprit lorsqu'on travaille sur des dessins en trois dimensions.

%Nous considérons d'abord le cas d'une fonction $f$  de deux variables $x$ et $y$ à valeurs dans $\eR$. L'espace $\eR^3$ a trois dimensions, cela veut dire que il faut fixer trois paramètres indépendants pour désigner un point de manière unique (voir, au cours d'une deuxième lecture de ces notes, la section sur les coordonnées cylindriques et sphériques, \ref{sec_coord}). Le graphe d'une fonction comme $f$ est un sous-ensemble de $\eR^3$ où l'un des trois paramètres est d'office la valeur de $f$, donc il est décrit par seulement deux paramètres $x$ et $y$. Son intérieur est alors vide et, si $f$ est une fonction <<suffisamment gentille>>, $\Graph f$ est localement une surface dans $\eR^3$.    

Nous avons parfois besoin de donner des représentation graphiques d'une fonction. Nous pouvons, par exemple, penser à la fonction que associe à un point de la Terre son altitude. Lorsqu'on part pour une promenade en montagne on a envie de connaitre le graphe de cette fonction qui correspond en fait à la surface de la montagne. Bien sur nous ne voulons pas amener avec nous un modèle en 3D de la montagne donc il nous faut une méthode efficace pour projeter le graphe de $f$ sur le plan $x$-$y$ tout en gardant les informations fondamentales. Pour cela nous avons besoin de deux définitions (à ne pas confondre !)
\begin{definition}
	Soit $f$ une fonction de $\eR^2$ dans $\eR$ et soit $c$ dans $\eR$.  La \defe{$z$-section}{section!de graphe} de $\Graph f$ à la hauteur $c$ est donné par
\[
S^z_c=\{ (x,y,c)\in \eR^3\,|\, f(x,y)=c\}.
\]  
\end{definition}
\begin{definition}\label{def_niveau}
	Soit $f$ une fonction de $\eR^2$ dans $\eR$ et soit $c$ dans $\eR$. La \defe{courbe de niveau}{courbe de niveau} de $f$ à la hauteur $c$ est l'ensemble
\[
N_c=\{ (x,y)\in \eR^2\,|\, f(x,y)=c\}.
\]  
\end{definition}
On peut représenter la fonction $f$ d'une façon très précise en traçant quelques unes de ses courbes de niveau.  Dans la suite on pourra considérer aussi les $x$-sections et les $y$-sections du graphe d'une fonction de deux variables. La $x$-section de $\Graph f$ à la hauteur $a$ est     
\[
S^x_a=\{(a,y,z)\in\eR^3\,|\, f(a,y)=z\}.
\]
Comme vous avez peut être déjà compris, $S^x_a$ est le graphe de la fonction de $y$ qu'on obtient de $f$ en fixant $x=a$. Cette fonction est appelée $x$-section de $f$ pour $x=a$.

Certaines surfaces dans $\eR^3$ sont le graphe d'une fonction. 

\begin{example}
	Quelque graphes importants.
  \begin{description}
    \item[Un plan non vertical] Tout plan dans $\eR^3$ peut être décrit par une équation de la forme 
\[
a(x-x_0)+ b(y-y_0) + c(z-z_0) = r,
\] 
où, $(x_0, y_0, z_0)$ est vecteur dans $\eR^3$, et $a$, $b$, $c$ et $r$ sont des nombres réels. Si $c\neq 0$ alors le plan n'est pas vertical et on peut dire que il est le graphe de la fonction 
\[
P(x,y)= \frac{r+cz_0 -a(x-x_0)-b(y-y_0)}{c},
\]
quitte à choisir des nouvelles constantes $s$, $t$, $q$,
\[
P(x,y)=sx +ty +q.
\]
    \item[Un paraboloïde elliptique] Pour tous $\alpha$ et $\beta$ dans $\eR$ les  graphes des fonctions 
\[
PE_1(x,y)=\frac{x^2}{\alpha^2}+\frac{y^2}{\beta^2}
\]
ou de la fonction 
\[
PE_2(x,y)=-\frac{x^2}{\alpha^2}-\frac{y^2}{\beta^2}
\]
sont des paraboloïdes elliptiques. Le premier est contenu dans le demi-espace $z\geq 0$, l'autre dans $z\leq 0$. Le nom de cette surface vient de la forme de ses sections. En fait toutes  sections $S^z_c$ sont des ellipses, alors que les section $S^x_a$ et $S^y_b$ sont des paraboles.   
    \item[Un paraboloïde hyperbolique (selle)]  Pour tous $\alpha$ et $\beta$ dans $\eR$ les  graphes des fonctions 
\[
PH_1(x,y)=\frac{x^2}{\alpha^2}-\frac{y^2}{\beta^2}
\]
ou de la fonction 
\[
PH_2(x,y)=-\frac{x^2}{\alpha^2}+\frac{y^2}{\beta^2}
\]
sont des paraboloïdes hyperboliques. Remarquez que les  sections $S^z_c$ de ce graphe sont des hyperboles, alors que les section $S^x_a$ et $S^y_b$ sont des paraboles.   
    \item[Une demi-sphère] La fonction $S^+=\sqrt{R^2-x^2-y^2}$ a pour graphe la demi-sphère supérieure centrée en l'origine et de rayon $R$.  
Le dernier de ces exemples nous signale une chose très importante : une sphère entière n'est pas le graphe d'une fonction de $x$ et $y$. Par contre, une demi-sphère est bien le graphe de la fonction $f(x,y)=\sqrt{1-x^2-y^2}$.

L'équation que nous utilisons  pour d'écrire une sphère de rayon $R$ centrée en l'origine est 
\[
x^2+y^2+z^2=R^2
\] 
Donc, à  chaque point  $(x,y)$ dans le disque $x^2+y^2\leq R^2$ (notez que ce disque est contenu dans la section $S^z_0$), on peut associer deux valeurs de $z$ : $z_1=\sqrt{R^2-x^2-y^2}$ et  $z_2=-\sqrt{R^2-x^2-y^2}$. Par définition, une fonction n'associe qu'un seul valeur à chaque point de son domaine, d'où l'impossibilité de décrire cette sphère comme le graphe d'une fonction de $x$ et $y$.

  \end{description}
\end{example}

Considérons la fonction $Sp: \eR^3\to \eR$ qui associe à $(x,y,z)$ la valeur $x^2+y^2+z^2$. La sphère de rayon $R$ centrée en l'origine est l'ensemble de niveau $N_{R^2}$ de $Sp$. L'ensemble de niveau $N_{0}$ de $Sp$ est l'origine, et tous les ensemble de niveau de hauteur négative sont vides. La même chose est vraie pour les ellipsoïdes centrées en l'origine avec les axes $x$, $y$ et $z$ comme axes principaux et comme longueurs de demi-axes $a$, $b$ et $c$. Voici la fonction dont il sont les ensemble de niveau 
\[
El(x,y,z)= \frac{x^2}{a^2}+\frac{y^2}{b^2}+\frac{z^2}{c^2}.
\] 
\begin{example}
	Des ensembles de niveau importants.
  \begin{description}
    \item[Tout graphe] 
	    Le graphe de toute fonction $f$  de $\eR^2$ dans $\eR$ peut être considéré comme l'ensemble de niveau zéro de la fonction $F(x,y,z)=z-f(x,y)$.

    \item[Hyperboloïdes]
	    Les hyperboloïdes, comme les ellipsoïdes, sont une famille d'ensemble de niveau. En particulier, nous considérons des hyperboloïdes dont l'axe de symétrie est l'axe des $z$ et qui sont symétriques par rapport un plan $x$-$y$.  Une fois que les paramètres  $a$, $b$ et $c$ sont fixés la fonction que nous intéresse est 
\[
Hyp(x,y,z)= \frac{x^2}{a^2}+\frac{y^2}{b^2}-\frac{z^2}{c^2}.
\]
Les ensembles de niveau $N_d$ pour $d>0$ sont connexes, on les appelle \emph{hyperboloïdes à une feuille}. L'ensemble de niveau $N_0$ est \emph{cône (elliptique)}, le deux moitiés du cône se touchent en l'origine. Enfin, les ensembles de niveau $N_d$ pour $d<0$ ne sont  pas connexes et pour cette raison on les appelle \emph{hyperboloïdes à deux feuilles}.
  \end{description}
\end{example}


%++++++++++++++++++++++++++++++++++++++++++++++++++++++++++++++++++++++++++++++++++++++++
\section{Dérivée suivant un vecteur}		\label{SecDerDirect}
%++++++++++++++++++++++++++++++++++++++++++++++++++++++++++++++++++++++++++++++++++++++++
\begin{definition}
Soit $f$ une application de $U\subset\eR^m$ dans $\eR$, $a$ un point dans $U$ et $v$ un vecteur de $\eR^m$. On dit que $f$ admet une \defe{dérivée suivant le vecteur $v$ au point $a$}{dérivée!directionnelle} si la fonction $t\mapsto f(a+tv)$ admet une dérivée en $t=0$. La  dérivée de $f$ suivant le vecteur $v$ au point $a$ est alors cette dérivée, et $f$ est dite dérivable suivant $v$ en $a$,
\[
\partial_v f(a)=\lim_{
  \begin{subarray}{l}
    t\to 0\\ t\neq 0 
  \end{subarray}
 }\frac{f(a+tv)-f(a)}{t}.
\] 
\end{definition}

\begin{definition}
  La fonction $f:U\subset\eR^m\to \eR^n$ de composantes $(f_1,\ldots, f_n)$, est dite \defe{dérivable suivant $v$ au point $a$}{} si toute ses composante $f_i$, $i=1,\ldots, n$ sont dérivables suivant $v$ au point $a$. Dans ce cas, nous écrivons
  \begin{equation}
	\partial_v f(a)=\left(\partial_v f_1(a), \ldots, \partial_v f_n(a)\right)^T.
  \end{equation}
\end{definition}
On parle aussi souvent de dérivé \defe{dans la direction}{} du vecteur $v$. Une \defe{direction}{direction} dans $\eR^m$ est un vecteur de norme $1$. Tant que $u$ est un élément non nul de $\eR^m$, nous pouvons parler de la direction de $u$.

\begin{proposition}
Soit $u$ un vecteur de norme $1$ dans $\eR^m$ et soit $v=\lambda u$, avec $\lambda$ dans $\eR$. La fonction $f$ est dérivable suivant $v$ au point $a$ si et seulement si $f$ est dérivable suivant $u$ au point $a$, en outre  
\[
\partial_v f(a)=\lambda\partial_u f(a).
\]
\end{proposition}
\begin{proof}
  \begin{equation}
    \begin{aligned}
  \partial_v f(a)=&\lim_{\begin{subarray}{l}
     t\to 0\\ t\neq 0 
    \end{subarray}}\frac{f(a+tv)-f(a)}{t}=\lim_{\begin{subarray}{l}
     t\to 0\\ t\neq 0 
    \end{subarray}}\frac{f(a+t\lambda u)-f(a)}{t}=\\
&=\lambda \lim_{\begin{subarray}{l}
    t\to 0\\ t\neq 0 
  \end{subarray}}\frac{f(a+t\lambda u)-f(a)}{\lambda t}=\lambda \partial_u f(a).    
    \end{aligned}
  \end{equation}
\end{proof}
\begin{definition}
Soit $f$ une application de $U\subset\eR^m$ dans $\eR$. On appelle \defe{dérivées partielles de $f$ au point $a$}{dérivée!partielle} les dérivées de $f$ suivant les vecteurs de base $e_1,\ldots,e_m $ au point $a$, si elles existent.
\end{definition}
Si $m=2,3$ on peut utiliser la notation $f_x$, $\partial_x$  ou $\partial_1$ pour la dérivée partielle suivant $e_1$, $f_y$, $\partial_y$  ou $\partial_2$  pour la dérivée partielle suivant $e_2$ et $f_z$,  $\partial_z$  ou $\partial_3$  pour la dérivée partielle suivant $e_3$. En général, nous écrivons $\partial_i$ pour noter la la dérivée partielle suivant $e_i$.  

\begin{example}
Les dérivées partielles de la fonction $f(x,y)=xy^3+\sin y$ au point $(0,\pi)$ sont 
\[
\partial_xf(0,\pi)=\frac{ \partial f }{ \partial x }(0,\pi)=f_x(0,\pi)=\lim_{\begin{subarray}{l}
    t\to 0\\ t\neq 0 
  \end{subarray}} \frac{(t\pi^3+\sin \pi)-(\sin \pi)}{t}= \pi^3,
\] 
\[
\partial_yf(0,\pi)=\frac{ \partial f }{ \partial y }(0,\pi)=f_y(0,\pi)=\lim_{\begin{subarray}{l}
    t\to 0\\ t\neq 0 
  \end{subarray}} \frac{0(\pi+t)^3+\sin (t+\pi)-0\cdot \pi^3}{t}= \cos \pi=-1,
\]   
\end{example}
La fonction d'une seule variable qu'on obtient à partir de $f$ en fixant les $p-1$ variables  $x_1,\ldots, x_{i-1}, x_{i+1}, \ldots, x_p$ et qui associe à $x_i$ la valeur $f(x_1,\ldots, x_{i-1}, x_i, x_{i+1}, \ldots, x_p)$, est appelée $x_i$-ème \defe{section}{section} de $f$ en $x_1,\ldots, x_{i-1}, x_{i+1}, \ldots, x_p$. L'$i$-ème dérivée partielle de $f$ au point $a=(x_1,\ldots,x_m)$ est la dérivée de l'$i$-ème section de $f$ au point $x_i$. En pratique, pour calculer les dérivées partielles d'une fonction on fait une dérivation par rapport à la variable choisie en considérant les  autres variables comme des constantes.

\begin{example}
	Considérons la fonction $f(x,y)=2xy^2$. Lorsque nous calculons $\partial_xf(x,y)$, nous faisons comme si $y$ était constant. Nous avons donc $\partial_xf(x,y)=2y^2$. Par contre lors du calcul de $\partial_yf(x,y)$, nous prenons $x$ comme une constante. La dérivée de $y^2$ par rapport à $y$ est évidement $2y$, et par conséquent, $\partial_yf(x,y)=4xy$.
\end{example}

\begin{example}
  La fonction $f(x,y)=x^y$ est dérivable au point $(1,2)$ et on a
\[
\partial_x f(1,2)=(yx^{y-1})_{(x,y)=(1,2)}=2,
\]
\[
\partial_y f(1,2)=\partial_y\left(e^{y\ln x}\right)_{(x,y)=(1,2)}=\left(\ln x e^{y\ln x}\right)_{(x,y)=(1,2)}=\ln\big( 1- e^{2\ln(1)} \big)=0.
\]
\end{example}
\begin{definition}
  Soit $f$ une application de $U\subset\eR^m$ dans $\eR$ et $u$ un vecteur de $\eR^m$. La fonction $f$ est \defe{dérivable sur $U$ suivant le vecteur $u$}{}, si $f$ est dérivable  suivant le vecteur $u$ en tout point de $U$.
\end{definition} 

Pour les fonctions d'une seule variable la dérivabilité en un point $a$ implique la continuité en $a$. Cela n'est pas vrai pour les fonctions de plusieurs variables : il existe des fonction $f$  qui sont dérivables suivant tout vecteur au point $a$ sans pour autant être continue en $a$. 

  \begin{example}
    Considérons la fonction $f:\eR^2\to \eR$ 
    \begin{equation}
      f(x,y)=\left\{
      \begin{array}{ll}
        \frac{x^2y}{x^4+y^2} \qquad&\textrm{si } (x,y)\neq (0,0),\\
        0     & \textrm{sinon}.
      \end{array}
      \right.
    \end{equation}
Pour voir que $f$ n'est pas continue en $(0,0)$ il suffit de calculer la limite de $f$ restreinte à la parabole $y=x^2$
\[
\lim_{x\to 0} f(x,x^2)=\frac{1}{2} \neq 0.
\] 
Pourtant la fonction $f$ est dérivable en $(0,0)$ dans toutes les directions. En effet, soit $v=(v_1,v_2)$. Si $v_2\neq 0$, alors
\[
\partial_v f(a)=\lim_{\begin{subarray}{l}
			t\to 0\\ t\neq 0 
  		\end{subarray}}
  		\frac{t^3v_1^2v^2}{t^5 v_1^4+ t^3v_2^2}=\frac{v_1^2}{v_2},
\] 
tandis que si $v_2=0$, alors la valeur de $f(tv_1, 0)$  est $0$ pour tout $t$ et $v_1$, donc la dérivée partielle de $f$ par rapport à $x$ en l'origine existe et est nulle. 
\end{example}

\begin{example}
    Pour une fonction réelle à variable réelle, la dérivabilité entraine la continuité. Il n'en va pas de même pour les fonctions à plusieurs variables, comme le montre l'exemple suivant :
    \begin{equation}
        f(x,y)=\begin{cases}
            0    &   \text{si \( x=0\)}\\
            \frac{ y }{ x }\sqrt{x^2+y^2}    &    \text{sinon.}
        \end{cases}
    \end{equation}
    Nous avons tout de suite
    \begin{equation}
        \frac{ \partial f }{ \partial y }(0,0)=0.
    \end{equation}
    De plus si \( u_x\neq 0\) nous avons
    \begin{equation}
            \frac{ \partial f }{ \partial u }(0,0)=\frac{ u_y }{ u_x }\| u \|.
    \end{equation}
    Donc toutes les dérivées directionnelles de \( f\) en \( (0,0)\) existent alors que la fonction n'y est manifestement pas continue. En effet sous forme polaire,
    \begin{equation}
        f(r,\theta)=\frac{ r\sin(\theta) }{ \cos(\theta) },
    \end{equation}
    et quelle que soit la valeur de \( r\), en prenant \( \theta\) suffisamment proche de \( \pi/2\), la fraction peut être arbitrairement grande.

    Nous verrons par la proposition \ref{diff1} que la différentiabilité d'une fonction implique sa continuité.
\end{example}

\begin{definition}
	Étant donnés deux points $a$ et $b$ dans $\eR^p$ on appelle \defe{segment}{segment!dans $\eR^p$} d'extrémités $a$ et $b$, et on note $[a,b]$, l'image de $[0,1]$ par l'application $s: [0,1]\to \eR^p$, $s(t)= (1-t)a+tb$.  On pose $]a,b[=s\left(]0,1[\right)$, et  $]a,b]=s\left(]0,1]\right)$. 
\end{definition}
Il faut observer que le segment $[a,b]$ est une courbe orientée : certes en tant que ensembles, $[a,b]=[b,a]$, mais si nous regardons la fonction de $t$ correspondante à $[b,a]$, nous voyons qu'elle va dans le sens inverse de celle qui correspond à $[a,b]$. Nous approfondirons ces questions lorsque nous parlerons d'arcs paramétrés autour de la section \ref{SecArcGeometrique}.

Le segment $[b,a]$ est l'image de l'application $r\colon [0,1]\to \eR^p$ donnée par $r(t)=(1-t)b+ta$.

\begin{theorem}[Accroissement finis pour les dérivées suivant un vecteur]\label{val_medio_1}		\index{théorème!accroissements finis!dérivée directionnelle}
	Soit $U$ un ouvert dans $\eR^m$ et soit $f:U\to\eR^n$ une fonction. Soient $a$ et $b$ deux points distincts dans $U$, tels que le segment $[a,b]$ soit contenu dans $U$. Soit $u$ le vecteur 
	\[
		u=\frac{b-a}{\|b-a\|_m}.
	\] 
	Si $\partial_u f(x)$ existe pour tout $x$ dans $[a,b]$ on a
	\[
		\|f(b)-f(a)\|_n\leq \sup_{x\in[a,b]}\|\partial_uf(x)\|_n\|b-a\|_m.
	\]
\end{theorem}

\begin{proof}
	Nous considérons la fonction $g(t)=f\big( (1-t)a-tb \big)$. Elle décrit la droite entre $a$ et $b$ parce que $g(0)=a$ et $g(1)=b$. En ce qui concerne la dérivée,
	\begin{equation}
		\begin{aligned}[]
			g'(t)&=\lim_{h\to 0} \frac{ g(t+h)-g(t) }{ h }\\
			&=\lim_{h\to 0} \frac{ f\big( (1-t-h)a-(t+h)b \big) }{ h }\\
			&=\lim_{h\to 0} \frac{ f\big( a+(t+h)(b-a) \big)-f\big( a+t(b-a) \big) }{ h }\\
			&=\frac{ \partial f }{ \partial u }\big( a+t(b-a) \big)\| b-a \|.
		\end{aligned}
	\end{equation}
	Le dernier facteur $\| b-a \|$ apparaît pour la normalisation du vecteur $u$. En effet dans la limite, il apparaît $h(b-a)$, ce qui donnerait la dérivée le long de $b-a$, tandis que $u$ vaut $(b-a)/\| b-a \|$.

	Par le théorème des accroissements finis pour $g$, il existe $t_0\in\mathopen] 0 , 1 \mathclose[$ tel que
	\begin{equation}
		g(1)=g(0)+g'(t_0)(1-0).
	\end{equation}
	Donc
	\begin{equation}
		\| g(1)-g(0) \|\leq\sup_{t_0}\| g'(t_0) \|=\sum_{t_0\in\mathopen] 0 , 1 \mathclose[}\left\| \frac{ \partial f }{ \partial u }(a+t_0(b-a)) \right\|\| b-a \|.
	\end{equation}
	Mais lorsque $t_0$ parcours $\mathopen] 0 , 1 \mathclose[$, le point $a+t_0(b-a)$ parcours le segment $\mathopen] a , b \mathclose[$, d'où le résultat.
\end{proof}

\begin{corollary}
	Dans les mêmes hypothèses, si $n=1$, alors il existe $\bar x $ dans $]a,b[$ tel que
	\[
		f(b)-f(a)=\partial_uf(\bar x)\|b-a\|_m.
	\]    
\end{corollary}


%++++++++++++++++++++++++++++++++++++++++++++++++++++++++++++++++++++++++++++++++++++++++
\section{Différentielles}		\label{SecDifferentielle}
%+++++++++++++++++++++++++++++++++++++++++++++++++++++++++++++++++++++++++++++++++++++++++++++++++++++++++++++++++++++++++++
La notion de dérivée partielle (ou de dérivée suivant un vecteur) pour une fonction de plusieurs variables n'est pas une  généralisation de la notion de dérivée en une variable d'espace. En fait, du point de vue géométrique, la dérivée de la fonction $g:\eR\to\eR$ au point $a$ est la pente de la ligne droite tangente au graphe de $g$ au point $(a, g(a))$. Cette ligne, d'équation $r(x)=g'(a)x+g(a)$, est la meilleure approximation affine du graphe de $g$ au point $a$, comme à la figure \ref{LabelFigTangentSegment}.
\newcommand{\CaptionFigTangentSegment}{Tangentes au graphe d'une fonction d'une variable}
\input{Fig_TangentSegment.pstricks}

Le graphe d'une fonction $f$ de $\eR^2$ dans $\eR$ est une surface de deux paramètres dans $\eR^3$. Si l'approximation affine d'une telle surface au point $(x,y,f(x,y))$ existe, alors elle est un plan tangent. En dimension plus haute, le graphe de la fonction $f:\eR^m\to\eR$ est une surface de $m$ paramètres dans $\eR^{m+1}$ et son approximation affine (si elle existe) est un hyperplan de $\eR^m$. 

Nous allons voir que si $f$ prend ses valeurs dans $\eR^n$ l'approximation affine de $f$ au point $a$ est l'élément de $ f(a)+\mathcal{L}(\eR^m,\eR^n)$ qui ressemble le plus à $f$ au voisinage de $a$. Plus précisément, on utilise les définitions suivantes.         
\begin{definition}
  Soient $f$ et $g$ deux applications d'un ouvert $U$ de $\eR^m$ dans $\eR^n$. On dit que $g$ est \defe{tangente}{application!tangente} à $f$ au point $a\in U$ si $f(a)=g(a)$ et 
\[
\lim_{\begin{subarray}{l}
    x\to a\\ x\neq a
  \end{subarray}}\frac{\|f(x)-g(x)\|_n}{\|x-a\|_m}=0.
\]
\end{definition}
La relation de tangence est une relation d'équivalence. Nous sommes particulièrement intéressés par le cas où $f$ admet une application  affine tangente au point $a$. 
\begin{definition}      \label{DefDifferentiellePta}
  Soit $U$ un ouvert dans $\eR^m$ et $a$ un point dans $U$. Soit $f$ une application de $U$ dans $\eR^n$. On dit que $f$ est \defe{différentiable au point $a$}{application!différentiable} s'il existe une application linéaire $T$ de $\eR^m$ dans $\eR^n$ qui satisfait
  \begin{equation}	\label{EqCritereDefDiff}
\lim_{h\to 0_m}\frac{\|f(a+h)-f(a)-T(h)\|_n}{\|h\|_m}=0.    
  \end{equation}
  Si une telle $T$ existe on l'appelle \defe{différentielle}{différentielle} de $f$ au point $a$, et on la note $df(a)$. 
\end{definition}

Note : $df_a$ est \emph{en soi} une application $df(a)\colon \eR^m\to \eR^n$. Nous notons $df_a(u)$\nomenclature{$df_a(u)$}{Application de la différentielle de $f$ sur le vecteur $u$} la valeur de $df_a$ sur le vecteur $u\in\eR^m$.


\newcommand{\CaptionFigDifferentielle}{Interprétation géométrique de la différentielle.}
\input{Fig_Differentielle.pstricks}
En ce qui concerne l'interprétation géométrique, si nous regardons la figure \ref{LabelFigDifferentielle}, et d'ailleurs aussi en voyant la définition \ref{EqCritereDefDiff}, la fonction est différentiable et la différentielle est \( T\) si il existe une fonction \( \alpha\) telle que
\begin{equation}
    f(a+u)-f(a)-T(u)=\alpha(u)
\end{equation}
où la fonction \( \alpha\) satisfait
\begin{equation}		\label{EqPresqueTa}
	\lim_{u\to 0} \frac{ \| \alpha(u)\| }{\| u \|}=0
\end{equation}
C'est cela qui fait écrire \( f(a+u)-f(a)-df_a(u)=o(\| u \|)\) à ceux qui n'ont pas peur de la notation \( o\).

La différentielle $df_a$ est donc la partie linéaire de l'application affine qui approxime au mieux la fonction $f$ autour du point $a$. La notion de différentielle est la vraie généralisation du concept de dérivée pour fonctions de plusieurs variables, en outre elle nous permet d'expliciter la relation qui associe au vecteur $u$ la dérivée $\partial_u f(a)$, pour $f$ et $a$ fixés.  

\begin{remark}
	Si on remplace les normes $\|\cdot\|_m$  et $\|\cdot\|_n$ par d'autres normes, l'existence et la valeur de la différentielle de $f$ au point $a$ ne sont pas remises en cause. En effet, soient  $\|\cdot\|_M$  une norme sur $\eR^m$ et $\|\cdot\|_N$ une norme sur $\eR^n$. Par le théorème \ref{ThoNormesEquiv}, ces normes sont équivalentes à $\| . \|_m$ et $\| . \|_m$ respectivement; il existe donc des constantes $k,\, K,\, l,\,L >0$ telles que  pour tout vecteur $u$ de $\eR^m$ et tout vecteur $v$ de $\eR^n$   
\[
k\|u\|_M\leq \|u\|_m\leq K\|u\|_M,
\]
\[
l\|v\|_N\leq \|v\|_n\leq L\|v\|_N.
\]
Les éléments de $\mathcal{L}(\eR^m, \eR^n)$ sont les mêmes et on a 
\begin{equation}
  \begin{aligned}
 & \frac{l}{K}  \frac{\|f(a+h)-f(a)-T(h)\|_N}{\|h\|_M}\leq \frac{\|f(a+h)-f(a)-T(h)\|_n}{\|h\|_m}\leq\\
&\leq\frac{L}{k} \frac{\|f(a+h)-f(a)-T(h)\|_N}{\|h\|_M}.
  \end{aligned}
\end{equation}
Il est donc possible, pour démontrer la différentiabilité ou pour calculer la différentielle, d'utiliser le critère \eqref{EqCritereDefDiff} avec une norme au choix. Parfois c'est utile.
\end{remark}

\begin{proposition}\label{diff1}
    Si $f$ est différentiable au point $a$ alors
    \begin{enumerate}
        \item
            elle est continue en \( a\),
        \item
            elle admet une dérivée dans toutes les directions de \( \eR^m\),
        \item
            si $T\in\aL(\eR^m,\eR^n)$ est la différentielle de $f$ au point $a$, alors
            \begin{equation}
                T(u)=df_a(u)=\partial_u f(a). 
            \end{equation}
    \end{enumerate}
\end{proposition}
\index{application!différentiable}

La dernière égalité sera de temps en temps utilisée sous la forme
\begin{equation}    \label{EqOWQSoMA}
    df_a(u)=\Dsdd{ f(a+tu) }{t}{0}.
\end{equation}

\begin{proof}
  La limite
\[
\lim_{h\to 0_m}\frac{\|f(a+h)-f(a)-T(h)\|_n}{\|h\|_m}=0,
\]
implique que
 \[
\lim_{h\to 0_m}\|f(a+h)-f(a)-T(h)\|_n=0.
\]
Comme $T$ est dans $\mathcal{L}(\eR^m,\eR^n)$, on a $\lim_{h\to 0}T(h)=0$, d'où la continuité de $f$ au point $a$.

Si $u$ est un vecteur non nul, la différentiabilité de $f$ au point $a$ implique
\[
\lim_{t\to 0}\frac{\|f(a+tu)-f(a)-T(tu)\|_n}{\|tu\|_m}=0,
\]
par la linéarité de $T$ et par l'égalité $\|tu\|_m=|t|\|u\|_m$ on obtient
\[
\lim_{t\to 0}\frac{f(a+tu)-f(a)}{|t|}= T(u).
\]
Donc $f$ est dérivable suivant le vecteur $u$ et $\partial_uf(a)=T(u)=df_a(u)$.
\end{proof}

Cette proposition est à ne pas confondre avec la proposition \ref{Diff_totale} qui dira que si les dérivées partielles \emph{sont continues} sur un voisinage de $a$, alors $f$ est différentiables en $a$.

\begin{corollary}
	Soit $f$ une application de $U$ dans $\eR^n$ différentiable au point $a$ dans $U$. Alors l'application $df(a)$, différentielle de $f$ au point $a$, est unique, c'est à dire que si $T_1$ et $T_2$ sont deux applications vérifiant la condition \eqref{EqCritereDefDiff}, alors $T_1=T_2$.
\end{corollary}

\begin{proof}
	Pour tout vecteur $u$, la proposition précédente implique que $T_1(u)=T_2(u)=\partial_uf(a)$.
\end{proof}

\begin{corollary}
Soit  $f:\eR^m\to \eR^n$ une fonction.
  La dérivabilité de $f$ au point  $a$ suivant tout vecteur de $\eR^m$ est une condition nécessaire pour la différentiabilité de $f$ en $a$.
\end{corollary}

\begin{definition}
	Une fonction $f:\eR^m\to \eR^n$ est dite \defe{différentiable sur l'ouvert $U\subset\eR^m$}{différentiable!sur un ouvert}, si $f$ est différentiable en tout point de $U$. Dans ce cas, la différentielle de $f$ est l'application
	\begin{equation}
		\begin{aligned}
			df\colon U\subset\eR^m&\to \aL(\eR^m,\eR^n) \\
			x&\mapsto df(x). 
		\end{aligned}
	\end{equation}
\end{definition}

\begin{remark}\label{rk_lin}
  Tout élément $T$ de $\mathcal{L}(\eR^m,\eR^n)$ est différentiable en tout point de $\eR^m$ et coïncide avec sa différentielle. En effet, pour tout $a$ et $h$ dans $\eR^m$  on a 
\[
\frac{\|T(a+h)-T(a)-T(h)\|_n}{\|h\|_m}=0.
\]
\end{remark}
La proposition \ref{diff1} nous donne une recette très pratique pour calculer la différentielle d'une fonction de $\eR^m$ dans $\eR^n$.

 \begin{definition}
	 Soit $f$ une fonction différentiable de $\eR^m$ dans $\eR$. On appelle \defe{gradient}{gradient} de $f$ la fonction $\nabla f : \eR^m\to \eR^m$\nomenclature{$\nabla f$}{gradient de la fonction $f$} de composantes
\[
\partial_{1}f,\ldots,\partial_{m}f. 
\] 
Soit $f$ une fonction de $\eR^m$ dans $\eR^n$, $f(a)=(f_1(a),\ldots,f_n(a))^T$. On appelle \defe{matrice jacobienne}{matrice!jacobienne} de $f$ la fonction $J(f) : \eR^m\to \eR^m\times\eR^n$ définie par
\begin{equation}
a\mapsto  \begin{pmatrix}
    \partial_{1}f_1(a) &\ldots&\partial_{m}f_1(a)\\
\vdots&\ddots&\vdots\\
\partial_{1}f_n (a)&\ldots&\partial_{m}f_n(a)\\
  \end{pmatrix}
\end{equation}
\end{definition}

Le lemme suivant regroupe quelque égalités avec lesquelles nous allons souvent travailler. Il explique comment sont liés les dérivées directionnelles, les dérivées partielles et la différentielle.
\begin{lemma}		\label{LemdfaSurLesPartielles}
	Si $f\colon \eR^m\to \eR^n$ est une fonction différentiable, alors
	\begin{equation}
        df_a(u)=\frac{ \partial f }{ \partial u }(a)=\Dsdd{ f(a+tu) }{t}{0}=\sum_{i=1}^mu_i\frac{ \partial f }{ \partial x_i }(a)=\nabla f(a)\cdot u
	\end{equation}
	pour tout vecteur $u\in\eR^m$
\end{lemma}

\begin{proof}
La première égalité est la proposition \ref{diff1}, et la seconde est seulement la définition de la dérivée directionnelle avec des notations un peu plus snob. En particulier nous avons
\begin{equation}
    df_a(e_i)=\frac{ \partial f }{ \partial x_i }(a).
\end{equation}
Pour le reste c'est la linéarité de la différentielle qui joue : le vecteur $u$ peut être écrit de façon unique comme combinaison linéaire des vecteurs de base 
\[
u=\sum_{i=1}^{m}u_i e_i, \qquad  u_i\in\eR,\, \forall i\in\{1,\ldots, m\}.
\]
Alors, la linéarité de $df_a$ nous donne
\begin{equation}
     df_a(u)= df_a\left(\sum_{i=1}^{m}u_i e_i\right)
=\sum_{i=1}^{m}u_i \left(df_ae_i\right)
=\sum_{i=1}^{m}u_i \frac{ \partial f }{ \partial x_i }(a).
 \end{equation}
Le lien avec le gradient est la définition du produit scalaire \eqref{DefYNWUFc}.
\end{proof}

%++++++++++++++++++++++++++++++++++++++++++++++++++++++++++++++++++++++++++++++++++++++++
\section{Propriétés des différentielles}		\label{SecPropDiffs}
%++++++++++++++++++++++++++++++++++++++++++++++++++++++++++++++++++++++++++++++++++++++++++++++++++++++++++++++++++++++++++++++

%---------------------------------------------------------------------------------------------------------------------------
\subsection{Linéarité}
%---------------------------------------------------------------------------------------------------------------------------

La proposition suivante signifie que différentiation est une opération linéaire sur l'ensemble des fonctions différentiables. 
\begin{proposition}		\label{PropDiffLineaire}
  Soient $f$ et $g$ deux fonction de $U\subset\eR^m$ dans $\eR^n$ différentiables au point $a\in U$, et soit $\lambda$ dans $\eR$. Alors les fonctions $f+g$ et $\lambda f$ sont différentiables au point $a$ et on a 
  \begin{equation}
    \begin{aligned}
 &     d(f+g)(a)=df(a)+dg(a), \\
& d(\lambda f)(a)=\lambda df(a),
    \end{aligned}
\end{equation}
\end{proposition}
\begin{proof}
  \begin{equation}
    \begin{aligned}
     & \lim_{h\to 0_m}\frac{\left\|\left(f(a+h)+g(a+h)\right)-\left(f(a)+g(a)\right)-df(a).h-dg(a).h\right\|_n}{\|h\|_m}\leq\\
&\lim_{h\to 0_m}\frac{\|f(a+h)-f(a)-df(a).h\|_n}{\|h\|_m}+\lim_{h\to 0_m}\frac{\|g(a+h)-g(a)-dg(a).h\|_n}{\|h\|_m}=0.
    \end{aligned}
  \end{equation}
  De même on démontre la  propriété $d(\lambda f)(a)=\lambda df(a)$.
\end{proof}

%---------------------------------------------------------------------------------------------------------------------------
\subsection{Produit}
%---------------------------------------------------------------------------------------------------------------------------

Soient $f$ et $g$ deux fonctions de $\eR^m$ dans $\eR^n$. Nous notons $f\cdot g$ la fonction de $\eR^n$ dans $\eR$ donnée par le produit scalaire point par point, c'est à dire
\begin{equation}
	(f\cdot g)(x)=f(x)\cdot g(x)
\end{equation}
pour tout $x\in\eR^m$. Le point dans le membre de droite est le produit scalaire dans $\eR^n$. Le cas particulier $n=1$ revient au produit usuel de fonctions :
\begin{equation}
	(fg)(x)=f(x)g(x).
\end{equation}

\begin{lemma}		\label{LemDiffProsuid}
	Si $f$ et $g$ sont des fonctions différentiables sur $\eR^m$ à valeurs dans $\eR$, alors la fonction produit $fg$ est également différentiable et
	\begin{equation}		\label{EqDifffgProd}
		d(fg)(a)=df(a)g(a)+f(a)dg(a)
	\end{equation}
	au sens où pour chaque $u$ dans $\eR^m$,
	\begin{equation}
		d(fg)(a).u=g(a)df(a).u+f(a)dg(a).u.
	\end{equation}
\end{lemma}
Remarquons qu'ici, $f(a)$ et $g(a)$ sont des réels, donc nous pouvons écrire $f(a)dg(a)$ aussi bien que $dg(a)f(a)$ sans ambigüités. 

\begin{proof}
	Ce que nous devons faire pour vérifier la formule \ref{EqDifffgProd}, c'est de vérifier le critère \eqref{EqCritereDefDiff} en remplaçant $f$ par $fg$ et $T(h)$ par $g(a)df(a).h+f(a)dg(a).h$.

	Ce que nous avons au numérateur est
	\begin{equation}
		\begin{aligned}[]
			\clubsuit&=(fg)(a+h)-(fg)(a)-g(a)df(a).h-f(a)dg(a).h\\
				&=f(a+h)g(a+h)-f(a)g(a)-g(a)df(a).h-f(a)dg(a).h.
		\end{aligned}
	\end{equation}
	Maintenant, nous allons faire apparaître $\big( f(a+h)-f(a)-df(a) \big)g(a+h)$ en ajoutant et soustrayant ce qu'il faut pour conserver $\clubsuit$ :
	\begin{equation}
		\begin{aligned}[]
			\clubsuit&=\big( f(a+h)-f(a)-df(a).h \big)g(a+h)\\
					&\quad +f(a)g(a+h)+g(a+h)df(a).h\\
					&\quad -f(a)g(a)-g(a)df(a).h-f(a)dg(a).h.
		\end{aligned}
	\end{equation}
	Nous mettons maintenant $f(a)$ et $fd(a).h$ en évidence là où c'est possible :
	\begin{equation}
		\begin{aligned}[]
			\clubsuit&=\big( f(a+h)-f(a)-df(a).h \big)g(a+h)\\
				&\quad+f(a)\big( g(a+h)-g(a)-dg(a).h \big)\\
				&\quad+\big( g(a+h)-g(a) \big)df(a).h.
		\end{aligned}
	\end{equation}
    Nous devons maintenant considérer la limite
	\begin{equation}
		\lim_{h\to 0}\frac{ \| \clubsuit \| }{ \| h \| }.
	\end{equation}
    Étant donné que $f$ et $g$ sont différentiables, les deux premiers termes sont nuls :
    \begin{equation}
        \begin{aligned}[]
            \lim_{h\to 0}\frac{ \big( f(a+h)-f(a)-df(a).h \big)}{\| h \|}g(a+h)=0\\
            \lim_{h\to 0} f(a)\frac{ \big( g(a+h)-g(a)-dg(a).h \big)}{\| h \|}=0.
        \end{aligned}
    \end{equation}
    En ce qui concerne le troisième terme, en utilisant la norme d'une application linéaire, nous avons
	\begin{equation}
		\lim_{h\to 0} \frac{ \| df(a).h \| }{ \| h \| }\leq\sup_{h\in\eR^m}\frac{ \| df(a).h \| }{ \| h \| }=\| df(a) \|,
	\end{equation}
    et par conséquent
    \begin{equation}
        \begin{aligned}[]
            0&\leq\lim_{h\to 0} \| g(a+h)-g(a) \|\frac{ \| df(a).h \|\| h \| }{ \| h \| }\\
            &\leq \lim_{h\to 0} \| g(a+h)-g(a) \|\| df(a) \|=0
        \end{aligned}
    \end{equation}
    parce que $g$ est continue (la limite du premier facteur est nulle tandis que la norme de $df(a)$ est un nombre constant). Nous avons donc bien prouvé que la formule \eqref{EqDifffgProd} est la différentielle de $fg$ au point $a$.
\end{proof}
Ce résultat se généralise pour des fonctions $f$ et $g$ de $\eR^m$ dans $\eR^n$.

\begin{proposition}
	Soient $f$ et $g$ deux fonction de $U\subset\eR^m$ dans $\eR^n$ différentiables au point $a\in U$. Alors la fonction $f\cdot g$ est différentiable  au point $a$ et on a 
	\begin{equation}
		g(f\cdot g)(a)=g(a)\cdot df(a)+f(a)\cdot dg(a)
	\end{equation}
	au sens où
	\begin{equation}		\label{Eqdfcdotgexpl}
		d(f\cdot g)_a(u)=g(a)\cdot\big( df_a(u) \big)+f(a)\cdot\big( dg_a(u) \big)
	\end{equation}
	pour tout $u\in\eR^m$.
\end{proposition}
Note : il faut être bien attentif en lisant la formule \eqref{Eqdfcdotgexpl}. Les points à l'intérieur des grandes parenthèses marquent l'application des différentielles sur $u$. Le contenu de ces parenthèses sont donc des éléments de $\eR^n$. Les points devant les parenthèses dénotent le produit scalaire dans $\eR^n$ ($f(a)$ et $dg_a(u)$ sont des éléments de $\eR^n$).

\begin{proof}
	La preuve du cas $n=1$ est déjà faite; c'est la formule \eqref{EqDifffgProd}. Pour le cas général $n\geq 2$, nous passons au composantes en nous rappelant que
	\begin{equation}
		(f\cdot g)(a)=\sum_{i=1}^nf_i(a)g_i(a)=\sum_{i=1}^n(f_ig_i)(a).
	\end{equation}
	En utilisant la linéarité de la différentiation, nous nous réduisons donc au cas des produits $f_ig_i$ qui sont des fonctions de $\eR^m$ dans $\eR$ :
	\begin{equation}
		\begin{aligned}[]
			d(f\cdot g)(a)&=d\left( \sum_{i=1}^n f_ig_i \right)(a)\\
			&=\sum_{i=1}^n\big( df_i(a)g_i(a)+f_i(a)dg_i(a) \big)\\
			&=g(a)\cdot df(a)+f(a)\cdot dg(a).
		\end{aligned}
	\end{equation}
	Ceci termine la preuve.
\end{proof}

%---------------------------------------------------------------------------------------------------------------------------
\subsection{Différentielle de fonction composée}
%---------------------------------------------------------------------------------------------------------------------------

La plus importante entre les règles de différentiation est la règle de différentiation d'une fonction composée (\emph{chain rule} dans les livres anglais et américains). Cette règle généralise la règle de dérivation pour fonctions de $\eR$ dans $\eR$. Il est utile d'introduire d'abord une formulation équivalente de la définition de différentielle
\begin{lemma}\label{Def_diff2}
  Soit $U$ un ouvert de $\eR^m$. La fonction $f: U\to\eR^n$ est différentiable au point $a$ dans $U$, si et seulement s'il existe une fonction $\sigma_f: U\times U\to \eR^n$ telle que
  \begin{subequations}		\label{SubEqsDiff2}
	  \begin{align}
  		\sigma_f(a,a)&=\lim_{x\to a} \sigma_f(a,x)=0\\
		 f(x)&=f(a)+T(x-a)+\sigma_f(a,x)\|x-a\|_m,   \label{def_diff2}
	  \end{align}
  \end{subequations}
pour une certaine application linéaire $T\in\mathcal{L}(\eR^m,\eR^n)$.
\end{lemma}
\begin{proof}
	Si les conditions \eqref{SubEqsDiff2} sont satisfaites alors $T$ est la différentielle de $f$ en $a$. En effet, dans ce cas nous avons
	\begin{equation}
		f(a+h)=f(a)+T(h)+\sigma_f(a,a+h)\| h \|,
	\end{equation}
	et la condition \eqref{EqCritereDefDiff} devient
	\begin{equation}
		\lim_{h\to 0} \frac{ \| \sigma_f(a,a+h) \|\| h \| }{ \| h \| }=\lim_{h\to 0} \| \sigma_f(a,a+h)\| =0
	\end{equation}
	
 
Si $f$ est différentiable au point $a$ il suffit de prendre $T=df(a)$ et 
\[
\sigma_f(a,x)=\frac{f(x)-f(a)-df(a).(x-a)}{\|x-a\|_m}.
\]
\end{proof}

\begin{remark}
	La fonction $\sigma_f(a,x)\| x-a \|_m$ est ce qui avait été appelle $\epsilon(h)$ sur la figure \ref{LabelFigDifferentielle}.
\end{remark}

\begin{proposition}		\label{PropDiffCompose}
Soient $U$ un ouvert de $\eR^m$ et $V$ un ouvert de $\eR^n$. Soient $f: U\to V$  et $g: V \to \eR^p$ deux fonctions différentiables respectivement au point $a$ dans $U$ et $b=f(a)$ dans $V$. Alors la fonction composée $g\circ f: U\to \eR^p $ est différentiable au point $a$ et
\begin{equation}	\label{EqDiffCompose}
    d(g\circ f)_a=dg_{f(a)}\circ df_a.
\end{equation}
\end{proposition}

Note : la formule \eqref{EqDiffCompose} est à comprendre de la façon suivante. Si $u\in\eR^m$, alors
\begin{equation}
    d(g\circ f)_a(u)=\underbrace{dg_{f(a)}}_{\in\aL(\eR^n,\eR^p)}\Big( \underbrace{df_a(u)}_{\in\eR^n} \Big)\in\eR^p.
\end{equation}

\begin{proof}
 En tenant compte du lemme \ref{Def_diff2} on peut écrire 
 \begin{subequations}
	 \begin{align}
		f(a+h)-f(a)&=df_a(h)+\sigma_f(a,a+h)\|h\|_m,	&&\forall h\in U-a,\\
		g(b+k)-g(b)&=dg_b(k)+\sigma_g(b,b+k)\|k\|_n,	&&\forall k\in V-b.
	 \end{align}
 \end{subequations}
On sait que $f(a)=b$ et que $f(a+h)$ est  un élément de $V$ et $f(a+h)=f(a)+k$ pour $k=df(a).h+\sigma_f(a,a+h)\|h\|_m$.  Par substitution dans la deuxième équation on obtient 
\begin{equation}
	\begin{aligned}
		g\big(f(a+h)\big)& - g\big(f(a)\big)\\ 
        &=dg_{f(a)}\Big(df_a(h)+\sigma_f(a,a+h)\|h\|_m\Big)\\
		&\quad+\sigma_g\left(f(a), f(a+h)\right)\left\| df_a(h)+\sigma_f(a,a+h)\|h\|_m\right \|_n\\
		&=g\circ f (a+h) - g\circ f (a)\\
        &= dg_{f(a)}\circ df_a(h) \\
        &\quad +\|h\|_m\Big[ dg_{f(a)}\sigma_f(a,a+h)\\
		&\quad+\sigma_g\left(f(a), f(a+h)\right)\big\| df_a\frac{h}{\|h\|_m}+\sigma_f(a,a+h)\big \|_n\Big],
	\end{aligned}
\end{equation}
donc
\begin{equation}
	(g\circ f) (a+h) - (g\circ f) (a) = dg_f(a)\circ df_a(h) + S(a,a+h) \|h\|_m
\end{equation}
où $S$ représente le contenu du dernier grand crochet. Il ne reste plus qu'à prouver que $S(a,a+h)$ est $o(\|h\|_m)$. En tenant compte du fait que $\sigma_f(a,a+h)$ et $\sigma_g\left(f(a), f(a+h)\right)$ sont $o (\|h\|_m)$,
\begin{equation}
  \begin{aligned}
      & \lim_{h\to 0_m} \frac{S(a,a+h)}{\|h\|_m}= \lim_{h\to 0_m}\frac{dg_{f(a)}\sigma_f(a,a+h)}{\|h\|_m}+ \\
& + \lim_{h\to 0_m}\frac{\sigma_g\left(f(a), f(a+h)\right)\left\| df_a\frac{h}{\|h\|_m}+\sigma_f(a,a+h)\right \|_n}{\|h\|_m} = 0.
  \end{aligned}
\end{equation}
\end{proof}

En appliquant la proposition précédente point par point, nous obtenons le résultat suivant.
\begin{proposition}
Soient $U$ un ouvert de $\eR^m$ et $V$ un ouvert de $\eR^n$. Soient $f: U\to V$  et $g: V \to \eR^p$ deux fonctions différentiables respectivement sur $U $ et sur $V$. Alors la fonction composée $g\circ f: U\to \eR^p $ est différentiable sur $U$.
\end{proposition}
La matrice jacobienne de $g\circ f$ au point $a$ est le produit matriciel des matrices jacobiennes de $f$ et de $f$. Plus précisément, nous avons
\begin{equation}
	J_{g\circ f}(a)=J_g\big( f(a) \big)J_f(a).
\end{equation}
Remarquez que nous considérons la matrice jacobienne de $g$ au point $f(a)$.

Dans la cas particulier où $m=1$ et $f$ est une fonction d'un intervalle $I$ dans $\eR^n$, dérivable au point $a$, on a que la fonction composée $g\circ f$ est dérivable au point $a$ si $g$ est différentiable et alors
\[
(g\circ f)'(a)= dg\left(f(a)\right).f'(a).
\]
En fait, pour les fonction d'une seule variable la dérivabilité coïncide avec la différentiabilité.

Nous avons aussi une formule importante pour la différentielle des formes bilinéaires.
  \begin{lemma}\label{bilin_diff}
    Toute application bilinéaire 
    \begin{equation}
	    \begin{aligned}
		    B\colon \eR^m\times\eR^n&\to \eR^p \\
		    B(a_1,a_2)&=a_1 \star a_2
	    \end{aligned}
    \end{equation}
    est différentiable en tout point $(a_1,a_2)$ de $\eR^m\times\eR^n$, et on a
\[
dB(a_1,a_2).(h_1,h_2)=h_1\star a_2 + a_1\star h_2.
\] 
  \end{lemma}
  \begin{proof}
    \begin{equation}
      \begin{aligned}
  & \frac{\|B(a_1+h_1,a_2+h_2)-B(a_1,a_2)-(h_1\star a_2 + a_1\star h_2)\|_p}{\|(h_1,h_2)\|_{\eR^m\times\eR^n}} = \\ 
&= \frac{\|(a_1+h_1)\star(a_2+h_2)-a_1\star a_2-(h_1\star a_2 + a_1\star h_2)\|_p}{\|(h_1,h_2)\|_{\eR^m\times\eR^n}}=\spadesuit
 \end{aligned}
    \end{equation}
on rajoute et on enlève la quantité $(a_1+h_1)\star a_2$ dans le numérateur, et on obtient  
   \begin{equation}
      \begin{aligned}
%&= \frac{\|(a_1+h_1)\star(a_2+h_2)-(a_1+h_1)\star a_2 +(a_1+h_1)\star a_2- a_1\star a_2-}{\|(h_1,h_2)\|_{\eR^m\times\eR^n}}\\
%&\hspace{7cm}\frac{-(h_1\star a_2 + a_1\star h_2)\|_p}{\quad}=\\
&\spadesuit= \frac{\|(a_1+h_1)\star h_2+h_1\star a_2-(h_1\star a_2 + a_1\star h_2)\|_p}{\|(h_1,h_2)\|_{\eR^m\times\eR^n}}=\\
&= \frac{\|h_1\star h_2\|_p}{\|(h_1,h_2)\|_{\eR^m\times\eR^n}}\leq C\frac{\|h_1\|_m\|h_2\|_n}{\|(h_1,h_2)\|_{\eR^m\times\eR^n}}\leq\\
&\leq C\frac{\|(h_1,h_2)\|^2_{\eR^m\times\eR^n}}{\|(h_1,h_2)\|_{\eR^m\times\eR^n}}= C\|(h_1,h_2)\|_{\eR^m\times\eR^n}.
      \end{aligned}
    \end{equation}
Si on prend la limite de cette expression pour $(h_1,h_2)\to (0_m,0_n)$ on obtient $0$, donc la preuve est complète. À noter, que dans l'avant-dernier passage on a utilisé la continuité des applications linéaires $\pr_m:\eR^m\times\eR^n\to \eR^m$ et $\pr_n: \eR^m\times\eR^n\to \eR^n$ qui à chaque point $(a_1,a_2)$ de $\eR^m\times\eR^n$ associent $a_1$ et $a_2$ respectivement.  
  \end{proof}

%--------------------------------------------------------------------------------------------------------------------------- 
  \subsection{Différentielle et dérivées partielles}
%---------------------------------------------------------------------------------------------------------------------------

\begin{proposition}		\label{Diff_totale}
 Soit $U$ un ouvert dans $\eR^m$ et $a$ un point dans $U$. Soit $f$ une application de $U$ dans $\eR^n$. Si toute les dérivée partielles de $f$ existent sur \( U\) et sont continues au point $a$ alors $f$ est différentiable au point $a$.
\end{proposition}
\begin{proof} 
 On se limite au cas $m=2$.  Pour rendre les calculs plus simples on utilise ici la norme $\|\cdot\|_\infty$ dans l'espace $\eR^2$, mais comme on a vu plus en haut, cela ne peut pas avoir des conséquences sur la différentiabilité de $f$. Si la différentielle de $f$ au point $a$ existe alors elle est définie par la formule
\[
    df_a(v)=\frac{ \partial f }{ \partial x }(a)v_1+\frac{ \partial f }{ \partial y }(a)v_2
\] 
pour tout $v$ dans $\eR^m$. 

On commence par prouver le résultat en supposant que les dérivées partielles de $f$ au point $a$ sont nulles. La différentiabilité de $f$ signifie que pour toute constante  $\varepsilon> 0$ il y a une constante $\delta>0$ telle que si $\|v\|_\infty\leq \delta $ alors 
\[
\frac{\|f(a_1+v_1, a_2+v_2)-f(a_1, a_2)\|_n}{\|v\|_\infty}\leq \varepsilon. 
\]   
On écrit alors 
\begin{equation}
  \begin{aligned}
   & \|f(a_1+v_1, a_2+v_2)-f(a_1, a_2)\|_n=\\
&=\|f(a_1+v_1, a_2+v_2)-f(a_1+v_1, a_2)+f(a_1+v_1, a_2)-f(a_1, a_2)\|_n\leq\\
&\leq \|f(a_1+v_1, a_2+v_2)-f(a_1+v_1, a_2)\|_n+\|f(a_1+v_1, a_2)-f(a_1, a_2)\|_n.
  \end{aligned}
\end{equation}
Comme la dérivée partielle $\partial_x f$ est  nulle au point $a$  on sait que  pour toute constante  $\varepsilon> 0$ il y a une constante $\delta_1>0$ telle que si $|v_1|\leq \delta_1 $ alors
\[
\|f(a_1+v_1, a_2)-f(a_1, a_2)\|_n\leq \varepsilon |v_1|.
\] 
Pour l'autre terme on a, par la proposition \ref{val_medio_1},
\begin{equation}
  \begin{aligned}
   & \|f(a_1+v_1, a_2+v_2)-f(a_1+v_1, a_2)\|_n\leq \\
&\leq \sup\{\|\partial_yf(x)\|_n\,\vert\, x\in S\}|v_2|.
  \end{aligned}
\end{equation}
où $S$ est le segment d'extrémités  $(a_1+v_1, a_2)$ et $ (a_1+v_1, a_2+v_2)$. Comme la  dérivée partielle $\partial_y f$ est continue et nulle au point $a$ on sait que  pour toute constante  $\varepsilon> 0$ il existe une constante $\delta_2>0$ telle que si $\|(u_1,u_2)\|_\infty\leq \delta_2 $ alors $\|\partial_yf(a_1+u_1,a_2+u_2)\|_n\leq \varepsilon$. Si on choisit $\delta = \min\{\delta_1,\,\delta_2\}$ le segment $S$ est contenu dans la boule de rayon $\delta$ centrée au point $a$ et on obtient
\[
 \|f(a_1+v_1, a_2+v_2)-f(a_1, a_2)\|_n\leq \varepsilon |v_1|+\varepsilon |v_2|\leq 2\varepsilon \|v\|_\infty.
\]
Cela prouve que \( f\) est différentiable en \( (a_1,a_2)\) et que la différentielle est nulle :
\begin{equation}
    df_{(a_1,a_2)}=0.
\end{equation}

Dans le cas général, où les dérivées partielles de $f$ au point $a$ ne sont pas spécialement nulles, on peut considérer la fonction\footnote{Vous verrez dans la discussion à propos de la fonction \eqref{EqCJVooJOuXdN} pourquoi cette fonction ne fonctionne pas dans le cas de la dimension infinie.}
\begin{equation}    \label{EqXHVooJeQKrB}    
    g(x,y)=f(x,y)-\partial_1 f(a)x-\partial_2 f(a)y,
\end{equation}
qui a dérivées partielles nulles au point $a$. La fonction $g$ est donc différentiables. La fonction $f$ est maintenant la somme de $g$ et de la fonction linéaire et continue $(x,y)\mapsto \partial_1 f(a)x-\partial_2 f(a)y$. On verra dans la prochaine section que la somme de deux fonctions différentiables est une fonction différentiable. Par conséquent, la fonction $f$ est différentiable.
\end{proof}

\begin{remark}
    En dimension infinie, il n'est pas vrai que l'existence et la continuité de toutes les dérivées partielles en un point implique la différentiabilité en ce point. Pour donner un exemple, nous allons continuer l'exemple \ref{ExHKsIelG}
    avec la fonction \ref{EqCJVooJOuXdN} sur un espace de Hilbert.

    En dimension infinie nous aurons le théorème \ref{ThoOYwdeVt} qui donnera quelque chose de moins fort.
\end{remark}

Étant donné que pour tout vecteur $u$ dans $\eR^m$ on a $\partial_uf(a)=\nabla f(a)\cdot u$, le gradient de $f$ nous donne la direction dans laquelle la croissance de $f$ est maximale. Soit $C$ une colline et soit $f$ la fonction que a chaque point $(x,y)$ de la Terre associe son altitude. Si nous voulons monter la colline le plus vite possible nous n'avons qu'a suivre la direction $\nabla f$ à chaque point. Elle est la projection sur le plan $x$-$y$ de la direction de pente maximale. Au contraire, la direction $-\nabla f$ est la direction de croissance minimale.
   
La matrice jacobienne calculé au point $a$ est la matrice associée canoniquement à l'application linéaire $df_a:\eR^m\to\eR^n$.

%+++++++++++++++++++++++++++++++++++++++++++++++++++++++++++++++++++++++++++++++++++++++++++++++++++++++++++++++++++++++++++
\section{Plan tangent}		\label{SecPlanTangent}
%+++++++++++++++++++++++++++++++++++++++++++++++++++++++++++++++++++++++++++++++++++++++++++++++++++++++++++++++++++++++++++

On a dit au début de cette section que si $f$ est une fonction de $\eR^2$ dans $\eR$ alors le graphe de $f$ est une surface à deux paramètres et que l'application affine tangente au graphe de $f$ au point $(a, f(a))$ est un plan. Maintenant on sait que ce plan est celui d'équation 
\begin{equation}
	T_a(x,y)=f(a_1,a_2)+\frac{ \partial f }{ \partial x }(a_1,a_2)(x-a_1)+\frac{ \partial f }{ \partial y }(a_1,a_2)(y-a_2).
\end{equation}
Le plan tangent au graphe de $f$ au point $a$ est le graphe de cette fonction $T_a$.

\begin{remark}
	Il existe cependant des fonctions différentiables dont les dérivées partielles ne sont pas continues. La construction d'un tel exemple est cependant délicate, et nous le ferons pas ici. Retenez cependant que si dans un exercice vous obtenez que les dérivées partielles ne sont pas continues, vous ne pouvez pas immédiatement en conclure que la fonction ne sera pas différentiable.	 
\end{remark}

%+++++++++++++++++++++++++++++++++++++++++++++++++++++++++++++++++++++++++++++++++++++++++++++++++++++++++++++++++++++++++++
%\section{Calcul de limites}
%+++++++++++++++++++++++++++++++++++++++++++++++++++++++++++++++++++++++++++++++++++++++++++++++++++++++++++++++++++++++++++

%Incidemment, le lemme \ref{Def_diff2} nous donne une nouvelle technique pour calculer des limites à plusieurs variables, similaire à celle du développement asymptotique expliquée dans la section \ref{SecTaylorR}.

%En effet, la formule \eqref{def_diff2} nous permet d'écrire $f(x)$ sous la forme
%\begin{equation}
%	f(x)=f(a)+df(a).(x-a)+\sigma_f(a,x)\| x-a \|
%\end{equation}
%où la fonction $\sigma_f$ satisfait $\lim_{x\to a}\sigma_f(a,x)=0$. Ici, $x$ et $a$ sont des éléments de $\eR^m$.

%++++++++++++++++++++++++++++++++++++++++++++++++++++++++++++++++++++++++++++++++++++++++
\section{Fonctions de classe $\mathcal{C}^1$}
%++++++++++++++++++++++++++++++++++++++++++++++++++++++++++++++++++++++++++++++++++++++++++++++++++++++++++++++++++++++++++++++

Soit $f$ une fonction différentiable de $U$, ouvert de $\eR^m$, dans $\eR^n$. L'application différentielle de $f$ est une application  de $\eR^m$ dans $\mathcal{L}(\eR^m, \eR^n)$ 
\begin{equation}
  \begin{array}{rccc}
    df : & \eR^m & \to & \mathcal{L}(\eR^m, \eR^n)\\
& a& \mapsto & df_a.
  \end{array}
\end{equation}
Nous savons que $\mathcal{L}(\eR^m, \eR^n)$ est un espace vectoriel normé avec la définition \ref{DefDQRooVGbzSm}. Si $T$ est un élément dans $\mathcal{L}(\eR^m, \eR^n)$ alors la norme de $T$ est définie par 
\[
\|T\|_{\mathcal{L}(\eR^m, \eR^n)}=\sup_{x\in\eR^m} \frac{\|T(x)\|_n}{\|x\|_m}=\sup_{\begin{subarray}{l}
    x\in\eR^m\\
\|x\|_m\leq 1
  \end{subarray}} \|T(x)\|_n.
\]

Lorsqu'il existe un $M>0$ tel que $\| df(a) \|_{\aL(\eR^m,\eR^n)}<M$ pour tout $a$ dans $U$, nous disons que la différentielle de $f$ est \defe{bornée}{bornée!différentielle} sur $U$.

\begin{definition}
	La fonction $f$ est dite \defe{de classe $\mathcal{C}^1$}{fonction!de classe  $\mathcal{C}^1$} de $U\subset\eR^m$  dans $\eR^n$ si son application différentielle $df$ est continue de $\eR^m$ dans $\mathcal{L}(\eR^m, \eR^n)$. Nous écrivons $f\in\mathcal{C}^1(U,\eR^n)$\nomenclature{$\aC^1(U,\eR^n)$}{Les applications une fois continument dérivables}.
\end{definition}

\begin{proposition}		\label{PropDerContCun}
	Une fonction \( f\colon U\to \eR^n\) où \( U\) est ouvert dans \( \eR^m\) est de classe \( C^1\) si et seulement si les dérivées partielles de $f$ existent et sont continues.
\end{proposition}

\begin{proof}
	Supposons que les dérivées partielles de $f$ existent et sont continues. Nous savons alors déjà par la proposition \ref{Diff_totale} que la fonction $f$ est différentiable et qu'elle s'exprime sous la forme
	\[
		df_a(h)=\sum_{i=1}^{m}\partial_if (a)h_i, \qquad \forall a \in U,\,\forall h\in\eR^m.
	\]
	Pour montrer que $df$ est continue, nous devons montrer que la quantité $\| df(x)-df(a) \|_{\aL(\eR^m,\eR^n)}$ peut être rendue arbitrairement petite si $\| x-a \|_m$ est rendu petit. Nous avons
	\begin{equation}
		\begin{aligned}
			\| df_x-df_a \|_{\aL}&=\sup_{\| h \|=1}\| df_x(h)-df_a(h) \|\\
			&=\sup_{\| h \|_m=1}\left\|\sum_{i=1}^{m}\left(\partial_if (x)-\partial_if (a)\right)h_i\right\|_n\leq\\
			&\leq\sup_{\| h \|_m=1}\sum_{i=1}^{m} \left\|\left(\partial_if (x)-\partial_if (a)\right)\right\|_n|h_i|\leq\\
			&\leq\sup_{\| h \|_m=1} \|h\|_\infty\sum_{i=1}^{m} \left\|\left(\partial_if (x)-\partial_if (a)\right)\right\|_n\\
			&\leq \sum_{i=1}^m\| \partial_if(x)-\partial_if(a) \|.
		\end{aligned}
	\end{equation}
	Dans ce calcul, nous avons utilisé le fait que si $\| h \|_m\leq 1$, alors $\| h \|_{\infty}\leq 1$. Étant donné la continuité de $\partial_if$, la dernière ligne peut être rendue arbitrairement petite lorsque $x$ est proche e $a$.

Supposons maintenant que $f$ soit dans $\mathcal{C}^1(U,\eR^n)$. Alors 
\[
\left\|\partial_if (x)-\partial_if (a)\right\|_n= \left\|df(x).e_i-df(a).e_i\right\|_n \leq  \left\|df(x)-df(a)\right\|_{\mathcal{L}(\eR^m,\eR^n)},
\]  
la continuité de $df$ implique donc celle de $\partial_i f$ pour tout $i$ dans $\{1,\ldots,m\}$.
\end{proof}
\begin{proposition}
  Soient $U$ un ouvert de $\eR^m$ et $V$ un ouvert de $\eR^n$. Soient $f: U\to V$  dans $\mathcal{C}^1(U,V)$ et $g: V \to \eR^p$ dans $\mathcal{C}^1(V,\eR^n)$.  Alors la fonction composée $g\circ f: U\to \eR^p $ est dans $\mathcal{C}^1(U,\eR^p)$.
\end{proposition}
\begin{proof} On fixe $a$ dans $U$ 
  \begin{equation}
    \begin{aligned}
     \big\|d(g\circ f)(x)&-d(g\circ f)(a)\big\|_{\mathcal{L}(\eR^m,\eR^p)}\\
     &=\left\|dg(f(x))\circ df(x)-dg(f(a))\circ df(a)\right\|_{\mathcal{L}(\eR^m,\eR^p)}\leq\\
&\leq \left\|\left(dg(f(x))-dg(f(a))\right)\circ df(x)\right\|_{\mathcal{L}(\eR^m,\eR^p)}+\\
&\quad+ \left\|dg(f(a))\circ \left(df(x)-df(a)\right)\right\|_{\mathcal{L}(\eR^m,\eR^p)}\leq\\
&\leq \left\|dg(f(x))-dg(f(a))\right\|_{\mathcal{L}(\eR^n,\eR^p)}\left\| df(x)\right\|_{\mathcal{L}(\eR^m,\eR^n)}+\\
&\quad+ \left\|dg(f(a))\right\|_{\mathcal{L}(\eR^n,\eR^p)}\left\| df(x)-df(a)\right\|_{\mathcal{L}(\eR^n,\eR^p)}.\\
    \end{aligned}
  \end{equation}
On peut conclure en passant à la limite $x\to a$ parce que les fonctions $f$, $g$, $df$ et $dg$ sont continues, de telle sorte que
\begin{equation}
	\begin{aligned}[]
		\lim_{x\to a} dg\big( f(x) \big)=dg\big( f(a) \big)\\
		\lim_{x\to a} df(x)=df(a).
	\end{aligned}
\end{equation}
\end{proof}

\begin{remark}
  On peut prouver le même résultat en utilisant la continuité de l'application bilinéaire 
\begin{equation}
  \begin{array}{rccc}
    \circ : & \mathcal{C}^1(U,V)\times\mathcal{C}^1(V,\eR^p)  & \to & \mathcal{L}(U, \eR^p)\\
& (T,S)& \mapsto & T\circ S.
  \end{array}
\end{equation}
\end{remark}

On fixe maintenant une définition largement utilisée dans la suite. 
\begin{definition}
	 Soient $U$ et $V$, deux ouverts de $\eR^m$. Une application $f$ de $U$ dans $V$ est un \defe{difféomorphisme}{difféomorphisme} si elle est bijective, différentiable et dont l'inverse $f^{-1}:V\to U $ est aussi différentiable. 
\end{definition}

\begin{remark}
	Il n'est pas possible d'avoir une application inversible d'un ouvert de $\eR^m$ vers un ouvert de $\eR^n$ si $m\neq n$. Il n'y a donc pas de notion de difféomorphismes entre ouverts de dimensions différentes.
\end{remark}
