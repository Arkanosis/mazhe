% This is part of Mes notes de mathématique
% Copyright (c) 2006-2014
%   Laurent Claessens, Carlotta Donadello
% See the file fdl-1.3.txt for copying conditions.

%+++++++++++++++++++++++++++++++++++++++++++++++++++++++++++++++++++++++++++++++++++++++++++++++++++++++++++++++++++++++++++
\section{Compacité}
%+++++++++++++++++++++++++++++++++++++++++++++++++++++++++++++++++++++++++++++++++++++++++++++++++++++++++++++++++++++++++++
%http://fr.wikipedia.org/wiki/Espace_compact
%http://fr.wikipedia.org/wiki/Théorème_de_Heine-Borel
%http://fr.wikipedia.org/wiki/Émile_Borel
%http://fr.wikipedia.org/wiki/Henri_Léon_Lebesgue

Soit $E$, un sous ensemble de $\eR$. Nous pouvons considérer les ouverts suivants : 
\begin{equation}
    \mO_x=B(x,1)
\end{equation}
pour chaque $x\in E$. Évidement,
\begin{equation}
    E\subseteq \bigcup_{x\in E}\mO_x.
\end{equation}
Cette union est très souvent énorme, et même infinie. Elle contient de nombreuses redondances. Si par exemple $E=[-10,10]$, l'élément $3\in E$ est contenu dans $\mO_{3.5}$, $\mO_{2.7}$ et bien d'autres. Pire : même si on enlève par exemple $\mO_2$ de la liste des ouverts, l'union de ce qui reste continue à être tout $E$. La question est : \emph{est-ce qu'on peut en enlever suffisamment pour qu'il n'en reste qu'un nombre fini ?}
\begin{definition}
Soit $E$, un sous ensemble de $\eR$. Une collection d'ouverts $\mO_i$ est un \defe{recouvrement}{recouvrement} de $E$ si $E\subseteq \bigcup_{i}\mO_i$. Un sous ensemble $E$ de $\eR$ tel que de tout recouvrement par des ouverts, on peut extraire un sous-recouvrement fini est dit \defe{\href{http://fr.wikipedia.org/wiki/Espace_compact}{compact}}{compact}.
\end{definition}

\begin{proposition}
Les ensembles compacts sont fermés et bornés.
\end{proposition}

\begin{proof}
Prouvons d'abord qu'un ensemble compact est borné. Pour cela, supposons que $K$ est un compact non borné vers le haut\footnote{Nous laissons à titre d'exercice le cas où $K$ est borné par le haut et pas par le bas.}. Donc il existe une suite infinie de nombres strictement croissante $x_1<x_2<\ldots$ tels que $x_i\in K$. Prenons n'importe quel recouvrement ouvert de la partie de $K$ plus petite ou égale à $x_1$, et complétons ce recouvrement par les ouverts $\mO_i=]x_{i-1},x_i[$. Le tout forme bien un recouvrement de $K$ par des ouverts. 

Il n'y a cependant pas moyen d'en tirer un sous recouvrement fini parce que si on ne prends qu'un nombre fini parmi les $\mO_i$, on en aura fatalement un maximum, disons $\mO_k$. Dans ce cas, les points $x_{k+1}$, $x_{k+1}$,\ldots ne seront pas dans le choix fini d'ouverts.

Cela prouve que $K$ doit être borné.

Pour prouver que $K$ est fermé, nous allons prouver que le complémentaire est ouvert. Et pour cela, nous allons prouver que si le complémentaire n'est pas ouvert, alors nous pouvons construire un recouvrement de $K$ dont on ne peut pas extraire de sous recouvrement fini.

Si $\eR\setminus K$ n'est pas ouvert, il possède un point, disons $x$, tel que tout voisinage de $x$ intersecte $K$. Soit $B(x,\epsilon_1)$, un de ces voisinages, et prenons $k_1\in K\cap B(x,\epsilon_1)$. Ensuite, nous prenons $\epsilon_2$ tel que $k_1$ n'est pas dans $B(x,\epsilon_1)$, et nous choisissons $k_2\in K\cap B(x,\epsilon_2)$. De cette manière, nous construisons une suite de $k_i\in K$ tous différents et de plus en plus proches de $x$. Prenons un recouvrement quelconque par des ouverts de la partie de $K$ qui n'est pas dans $B(x,\epsilon_1)$. Les nombres $k_i$ ne sont pas dans ce recouvrement.

Nous ajoutons à ce recouvrement les ensembles $\mO=]k_i,k_{i+1}[$. Le tout forme un recouvrement (infini) par des ouverts dont il n'y a pas moyen de tirer un sous recouvrement fini, pour exactement la même raison que la première fois.
\end{proof}

Le résultat suivant le théorème de \href{http://fr.wikipedia.org/wiki/Théorème_de_Heine-Borel}{Borel-Lebesgue}, et la démonstration vient de wikipédia.
\begin{theorem}[\href{http://fr.wikipedia.org/wiki/Émile_Borel}{borel}-\href{http://fr.wikipedia.org/wiki/Henri_Léon_Lebesgue}{Lebesgue}]   \label{ThoBOrelLebesgue}
    Les intervalles de la forme $[a,b]$ sont compacts.
\end{theorem}

\begin{proof}
    Soit $\Omega$, un recouvrement du segment $[a,b]$ par des ouverts, c'est à dire que
    \begin{equation}
        [a,b]\subseteq\bigcup_{\mO\in\Omega}\mO.
    \end{equation}
    Nous notons par $M$ le sous-ensemble de $[a,b]$ des points $m$ tels que l'intervalle $[a,m]$ peut être recouvert par un sous-ensemble fini de $\Omega$. C'est à dire que $M$ est le sous ensemble de $[a,b]$ sur lequel le théorème est vrai. Le but est maintenant de prouver que $M=[a,b]$.
    \begin{description}
        \item[$M$ est non vide] En effet, $a\in M$ parce que il existe un ouvert $\mO\in\Omega$ tel que $a\in\mO$. Donc $\mO$ tout seul recouvre l'intervalle $[a,a]$. 
        \item[$M$ est un intervalle] Soient $m_1$, $m_2\in M$. Le but est de montrer que si $m'\in[m_1,m_2]$, alors $m'\in M$. Il y a un sous recouvrement fini de l'intervalle $[a,m_2]$ (par définition de $m_2\in M$). Ce sous recouvrement fini recouvre évidement aussi $[a,m']$ parce que $[a,m']\subseteq [a,m_2]$, donc $m'\in M$.
        \item[$M$ est une ensemble ouvert] Soit $m\in M$. Le but est de prouver qu'il y a un ouvert autour de $m$ qui est contenu dans $M$. Mettons que $\Omega'$ soit un sous recouvrement fini qui contienne l'intervalle $[a,m]$. Dans ce cas, on a un ouvert $\mO\in\Omega'$ tel que $m\in\mO$. Tous les points de $\mO$ sont dans $M$, vu qu'ils sont tous recouverts par $\Omega'$. Donc $\mO$ est un voisinage de $m$ contenu dans $M$.
        \item[$M$ est un ensemble fermé] $M$ est un intervalle qui commence en $a$, en contenant $a$, et qui finit on ne sait pas encore où. Il est donc soit de la forme $[a,m]$, soit de la forme $[a,m[$. Nous allons montrer que $M$ est de la première forme en démontrant que $M$ contient son supremum $s$. Ce supremum est un élément de $[a,b]$, et donc il est contenu dans un des ouverts de $\Omega$. Disons $s\in\mO_s$. Soit $c$, un élément de $\mO_s$ strictement plus petit que $c$; étant donné que $s$ est supremum de $M$, cet élément $c$ est dans $M$, et donc on a un sous recouvrement fini $\Omega'$ qui recouvre $[a,c]$. Maintenant, le sous recouvrement constitué de $\Omega'$ et de $\mO_s$ est fini et recouvre $[a,s]$.
    \end{description}
    Nous pouvons maintenant conclure : le seul intervalle non vide de $[a,b]$ qui soit à la fois ouvert et fermé est $[a,b]$ lui-même, ce qui prouve que $M=[a,b]$, et donc que $[a,b]$ est compact.
\end{proof}

Par le théorème des valeurs intermédiaires, l'image d'un intervalle par une fonction continue est un intervalle, et nous avons l'importante propriété suivante des fonctions continues sur un compact.

Le théorème suivant est un cas particulier du théorème \ref{ThoMKKooAbHaro}.
\begin{theorem}
    Si $f$ est une fonction continue sur l'intervalle compact $[a,b]$. Alors $f$ est bornée sur $[a,b]$ et elle atteint ses bornes.
\end{theorem}

\begin{proof}
    Étant donné que $[a,b]$ est un intervalle compact, son image est également un intervalle compact, et donc est de la forme $[m,M]$. Ceci découle du théorème \ref{ThoImCompCotComp} et le corollaire \ref{CorImInterInter}. Le maximum de $f$ sur $[a,b]$ est la borne $M$ qui est bien dans l'image (parce que $[m,M]$ est fermé). Idem pour le minimum $m$.
\end{proof}

%+++++++++++++++++++++++++++++++++++++++++++++++++++++++++++++++++++++++++++++++++++++++++++++++++++++++++++++++++++++++++++
\section{Dérivation}
%+++++++++++++++++++++++++++++++++++++++++++++++++++++++++++++++++++++++++++++++++++++++++++++++++++++++++++++++++++++++++++

\begin{lemma}           \label{LemDeccCarr}
    Si $f(x)=x^2$, alors $f'(x)=2x$.
\end{lemma}

\begin{proof}
    Utilisons la définition, et remplaçons $f$ par sa valeur :
    \begin{subequations}
        \begin{align}
            f'(x)   &=\lim_{\epsilon\to 0}\frac{ f(x+\epsilon)-f(x) }{ \epsilon }\\
                &=\lim_{\epsilon\to 0}\frac{ (x+\epsilon)^2-x^2 }{ \epsilon }\\
                &=\lim_{\epsilon\to 0}\frac{ x^2+2x\epsilon+\epsilon^2-x^2 }{ \epsilon }\\
                &=\lim_{\epsilon\to 0}\frac{\epsilon(2x+\epsilon)}{ \epsilon }\\
                &=\lim_{\epsilon\to 0}(2x+\epsilon)\\
                &=2x,
        \end{align}
    \end{subequations}
    ce qu'il fallait prouver.
\end{proof}

Une facile, maintenant.
\begin{proposition}
    La dérivé de la fonction $x\mapsto x$ vaut $1$, en notations compactes : $(x)'=1$.
\end{proposition}

\begin{proof}
D'après la définition de la dérivée, si $f(x)=x$, nous avons
\begin{equation}
    f(x)=\lim_{\epsilon\to 0}\frac{ (x+\epsilon) -x }{\epsilon} =\lim_{\epsilon\to 0}\frac{ \epsilon }{\epsilon} =1,
\end{equation}
et c'est déjà fini.
\end{proof}

Pour continuer, nous allons en faire une un peu plus abstraite.
\begin{proposition}     \label{PropDerrLin}
    La dérivation est une opération linéaire, c'est à dire que
    \begin{enumerate}
        \item $(\lambda f)'=\lambda f'$ pour tout réel $\lambda$ où, pour rappel, la fonction $(\lambda f)$ est définie par $(\lambda f)(x)=\lambda\cdot f(x)$,
        \item $(f+g)'=f'+g'$.
    \end{enumerate}
\end{proposition}

\begin{proof}
Ces deux propriétés découlent des propriétés correspondantes de la limite. Nous allons faire la première, et laisser la seconde à titre d'exercice. Écrivons la définition de la dérivée avec $(\lambda f)$ au lieu de $f$, et calculons un petit peu :
\begin{equation}
    \begin{aligned}[]
        (\lambda f)'(x) &=\lim_{\epsilon\to 0}\frac{ (\lambda f)(x+\epsilon)-(\lambda f)(x) }{ \epsilon }\\
                &=\lim_{\epsilon\to 0}\frac{ \lambda \big( f(x+\epsilon) \big)-\lambda f(x) }{ \epsilon }\\
                &=\lim_{\epsilon\to 0}\lambda \frac{ f(x+\epsilon) -f(x) }{ \epsilon }\\
                &=\lambda \lim_{\epsilon\to 0}\frac{ f(x+\epsilon) -f(x) }{ \epsilon }\\
                &=\lambda f'(x).
    \end{aligned}
\end{equation}
\end{proof}


\begin{proposition}
    La dérivée d'un produit obéit à la \defe{règle de Leibnitz}{Règle de Leibnitz}\index{Leibnitz}:
    \begin{equation}
        (fg)'(x)=f'(x)g(x)+f(g)g'(x).
    \end{equation}
    Cette règle est souvent écrite sous la forme compacte $(fg)'=f'g+g'f$.
\end{proposition}

\begin{proof}
La définition de la dérivée dit que
\begin{equation}        \label{Eqfgrimeepsfgx}
    (fg)'(x)=\lim_{\epsilon\to 0}\frac{f(x+\epsilon)g(x+\epsilon)-f(x)g(x)}{\epsilon}.
\end{equation}
La subtilité est d'ajouter au numérateur la quantité $-f(x)g(x+\epsilon)+f(x)g(x+\epsilon)$, ce qui est permit parce que cette quantité est nulle\footnote{Le coup d'ajouter et enlever la même chose a déjà été fait durant la démonstration du théorème \ref{Tholimfgabab}. C'est une technique assez courante en analyse.}. Le numérateur de \eqref{Eqfgrimeepsfgx} devient donc
\begin{equation}
    \begin{aligned}[]
f(x+\epsilon)g(x+\epsilon)&-f(x)g(x+\epsilon)+f(x)g(x+\epsilon)-f(x)g(x) \\
            &= g(x+\epsilon)\big( f(x+\epsilon)-f(x) \big)+f(x)\big( g(x+\epsilon)-g(x) \big),
    \end{aligned}
\end{equation}
où nous avons effectué deux mises en évidence. Étant donné que nous avons deux termes, nous pouvons couper la limite en deux :
\begin{equation}
    \begin{aligned}[]
        (fg)'(x)    &=\lim_{\epsilon\to 0}g(x+\epsilon)\frac{ f(x+\epsilon)-f(x) }{\epsilon}            &+\lim_{\epsilon\to 0}f(x)\frac{ g(x+\epsilon)-g(x) }{\epsilon}\\
                &=\lim_{\epsilon\to 0}g(x+\epsilon)\lim_{\epsilon\to 0}\frac{ f(x+\epsilon)-f(x) }{\epsilon}    &+f(x)\lim_{\epsilon\to 0}\frac{ g(x+\epsilon)-g(x) }{\epsilon},
    \end{aligned}
\end{equation}
où nous avons utilisé le théorème \ref{Tholimfgabab} pour scinder la première limite en deux, ainsi que la propriété \eqref{Eqbutmultlim} pour sortir le $f(x)$ de la limite dans le second terme. Maintenant, dans le premier terme, nous avons évidement\footnote{Pas tout à fait évidemment : selon le théorème \ref{ThoLimCont}, \emph{limite et continuité}, il faut que $g$ soit continue.} $\lim_{\epsilon\to 0}g(x+\epsilon)=g(x)$. Les limites qui restent sont les définitions classiques des dérivées de $f$ et $g$ au point~$x$ :
\begin{equation}
    (fg)'(x)=g(x)f'(x)-f(x)g'(x),
\end{equation}
ce qu'il fallait démontrer.
\end{proof}

%--------------------------------------------------------------------------------------------------------------------------- 
\subsection{Dérivation et croissance}
%---------------------------------------------------------------------------------------------------------------------------

Supposons une fonction dont la dérivée est positive. Étant donné que la courbe est \og collée \fg{} à ses tangentes, tant que les tangentes montent, la fonction monte. Or, une tangente qui monte correspond à une dérivée positive, parce que la dérivée est le coefficient directeur de la tangente.

Ce résultat très intuitif peut être prouvé rigoureusement. C'est la tache à laquelle nous allons nous atteler maintenant.

\begin{proposition} \label{PropGFkZMwD}
    Si $f$ et $f'$ sont des fonctions continues sur l'intervalle $[a,b]$ et si $f'(x)$ est strictement positive sur $[a,b]$, alors $f$ est croissante sur $[a,b]$.

    De la même manière, si $f'(x)$ est strictement négative sur $[a,b]$, alors $f$ est décroissante sur $[a,b]$.
\end{proposition}

\begin{proof}
    Nous n'allons prouver que la première partie. La seconde partie se prouve en considérant $-f$ et en invoquant alors la première\footnote{Méditer cela.}. Prenons $x_1$ et $x_2$ dans $[a,b]$ tels que $x_1<x_2$. Par hypothèse, pour tout $x$ dans $[x_1,x_2]$, nous avons
    \begin{equation}
        f'(x)=\lim_{\epsilon\to 0}\frac{ f(x+\epsilon)-f(x) }{\epsilon} >0.
    \end{equation}
    Maintenant, la proposition \ref{PropoLimPosFPos} dit que quand une limite est positive, alors la fonction dans la limite est positive sur un voisinage. En appliquant cette proposition à la fonction
    \begin{equation}
        r(\epsilon)=\frac{ f(x+\epsilon)-f(x) }{ \epsilon },
    \end{equation}
    dont la limite en zéro est positive, nous trouvons que $r(\epsilon)>0$ pour tout $\epsilon$ pas trop éloigné de zéro. En particulier, il existe un $\delta>0$ tel que $\epsilon<\delta$ implique $r(\epsilon)>0$; pour un tel $\epsilon$, nous avons donc
    \begin{equation}
        r(\epsilon)=\frac{ f(x+\epsilon)-f(x) }{ \epsilon }>0.
    \end{equation}
    Étant donné que $\epsilon>0$, nous avons que $f(x+\epsilon)-f(x)>0$, c'est à dire que $f$ est strictement croissante entre $x$ et $x+\delta$.

    Jusqu'ici, nous avons prouvé que la fonction $f$ était strictement croissante dans un voisinage autour de chaque point de $[a,b]$. Cela n'est cependant pas encore tout à fait suffisant pour conclure. Ce que nous voudrions faire, c'est de dire, c'est prendre un voisinage $]a,m_1[$ autour de $a$ sur lequel $f$ est croissante. Donc, $f(m_1)>f(a)$. Ensuite, on prend un voisinage $]m_1,m_2[$ de $m_1$ sur lequel $f$ est croissante. De ce fait, $f(m_2)>f(m_1)>f(a)$. Et ainsi de suite, nous voulons construire des $m_3$, $m_4$,\ldots jusqu'à arriver en $b$. Hélas, rien ne dit que ce processus va fonctionner. Il faut trouver une subtilité. Le problème est que les voisinages sur lesquels la fonction est croissante sont peut-être de plus en plus petit, de telle sorte à ce qu'il faille une infinité d'étapes avant d'arriver à bon port (en $b$).

    Heureusement, nous pouvons drastiquement réduire le nombre d'étapes en nous souvenant du théorème de Borel-Lebesgue (numéro \ref{ThoBOrelLebesgue}). Nous notons par $\mO_x$, un ouvert autour de $x$ tel que $f$ soit strictement croissante sur $\mO_x$. Un tel voisinage existe. Cela fait une infinité d'ouverts tels que
    \begin{equation}
        [a,b]\subseteq\bigcup_{x\in[a,b]}\mO_x.
    \end{equation}
    Ce que le théorème dit, c'est qu'on peut en choisir un nombre fini qui recouvre encore $[a,b]$. Soient $\{ \mO_{x_1},\ldots,\mO_{x_n} \}$, les heureux élus, que nous supposons prit dans l'ordre : $x_1<x_2<\ldots<x_n$. Nous avons
    \begin{equation}
        [a,b]\subseteq\bigcup_{i=1}^n\mO_i.
    \end{equation}
    Quitte à les rajouter à la collection, nous supposons que $x_1=a$ et que $x_n=b$. Maintenant nous allons choisir encore un sous ensemble de cette collection d'ouverts. On pose $\mA_1=\mO_{x_1}$. Nous savons que $\mA_1$ intersecte au moins un des autres $\mO_{x_i}$. Cette affirmation vient du fait que $[a,b]$ est connexe (proposition \ref{PropInterssiConn}), et que si $\mO_{x_1}$ n'intersectait personne, alors 
    \begin{equation}
        \begin{aligned}[]
            \mO_{x_1}&&\text{et}&&\bigcup_{i=2}^n\mO_{x_i}
        \end{aligned}
    \end{equation}
    forment une partition de $[a,b]$ en deux ouverts disjoints, ce qui n'est pas possible parce que $[a,b]$ est connexe. Nous nommons $\mA_2$, un des ouverts $\mO_{x_i}$ qui intersecte $\mA_1$. Disons que c'est $\mO_k$. Notons que $\mA_1\cup\mA_2$ est un intervalle sur lequel $f$ est strictement croissante. En effet, si $y_{12}$ est dans l'intersection, $f(a)<f(y_{12})$ parce que $f$ est strictement croissante sur $\mA_1$, et pour tout $x>y_{12}$ dans $\mA_2$, $f(x)>f(y_{12})$ parce que $f$ est strictement croissante dans $\mA_2$. 

    Maintenant, nous éliminons de la liste des $\mO_{x_i}$ tous ceux qui sont inclus à $\mA_1\cup\mA_2$. Dans ce qu'il reste, il y en a automatiquement un qui intersecte $\mA_1\cup\mA_2$, pour la même raison de connexité que celle invoquée plus haut. Nous appelons cet ouvert $\mA_3$, et pour la même raison qu'avant, $f$ est strictement croissante sur $\mA_1\cup\mA_2\cup\mA_3$.

    En recommençant suffisamment de fois, nous finissons par devoir prendre un des $\mO_{x_i}$ qui contient $b$, parce qu'au moins un des $\mO_{x_i}$ contient $b$. À ce moment, nous avons finit la démonstration.
\end{proof}

Il est intéressant de noter que ce théorème concerne la croissance d'une fonction sous l'hypothèse que la dérivée est positive. Il nous a fallu très peu de temps, en utilisant la positivité de la dérivée, pour conclure qu'autour de tout point, la fonction était strictement croissante. À partir de là, c'était pour ainsi dire gagné. Mais il a fallu un réel travail de topologie très fine\footnote{et je te rappelle que nous avons utilisé la proposition \ref{PropInterssiConn}, qui elle même était déjà un très gros boulot !} pour conclure. Étonnant qu'une telle quantité de topologie soit nécessaire pour démontrer un résultat essentiellement analytique dont l'hypothèse est qu'une limite est positive, n'est-ce pas ? 

Une petite facile, maintenant.
\begin{proposition}
    Si $f$ est croissante sur un intervalle, alors $f'\geq 0$ à l'intérieur cet intervalle, et si $f$ est décroissante sur l'intervalle, alors $f'\leq 0$ à l'intérieur de l'intervalle.
\end{proposition}

Note qu'ici, nous demandons juste la croissance de $f$, et non sa \emph{stricte} croissance.

\begin{proof}
    Soit $f$, une fonction croissante sur l'intervalle $I$, et $x$ un point intérieur de $I$. La dérivée de $f$ en $x$ vaut
    \begin{equation}
        f'(x)=\lim_{\epsilon\to 0}\frac{ f(x+\epsilon)-f(x) }{\epsilon},
    \end{equation}
    mais, comme $f$ est croissante sur $I$, nous avons toujours que $f(x+\epsilon)-f(x)\geq0$ quand $\epsilon>0$, et $f(x+\epsilon)-f(x)\leq0$ quand $\epsilon<0$, donc cette limite est une limite de nombre positifs ou nuls, qui est donc positive ou nulle. Cela prouve que $f'(x)\geq 0$.
\end{proof}

% http://fr.wikipedia.org/wiki/Théorème_de_Rolle
% http://gconnan.free.fr/les%20pdf/Deriv.pdf
Les deux prochains théorèmes sont très importants.
\begin{theorem}[\href{http://fr.wikipedia.org/wiki/Théorème_de_Rolle}{Théorème de Rolle}]       \label{ThoRolle}
    Soit $f$, une fonction continue sur $[a,b]$ et dérivable sur $]a,b[$. Si $f(a)=f(b)$, alors il existe un point $c\in]a,b[$ tel que $f'(c)=0$.
\end{theorem}

\begin{proof}
    Étant donné que $[a,b]$ est un intervalle compact, l'image de $[a,b]$ par $f$ est un intervalle compact, soit $[m,M]$ (théorème \ref{ThoImCompCotComp}). Si $m=M$, alors le théorème est évident : c'est que la fonction est constante, et la dérivée est par conséquent nulle. Supposons que $M> f(a)$ (il se peut que $M=f(a)$, mais alors si $f$ n'est pas constante, il faut avoir $m<f(a)$ et le reste de la preuve peut être adaptée).

    Comme $M$ est dans l'image de $[a,b]$ par $f$, il existe $c\in ]a,b[$ tel que $f(c)=M$. Considérons maintenant la fonction
    \begin{equation}
        \tau(x) =\frac{ f(c+x)-f(c) }{ x }.
    \end{equation}
    Par définition, $\lim_{x\to 0}\tau(x)=f'(c)$. Par hypothèse, si $u<c$,
    \begin{equation}
        \tau(u-c) = \frac{ f(u)-f(c) }{ u-c }>0
    \end{equation}
    parce que $u-c<0$ et $f(u)-f(c)<0$. Par conséquent, $\lim_{x\to 0}\tau(x)\geq 0$. Nous avons aussi, pour $v>c$,
    \begin{equation}
        \tau(v-c) = \frac{ f(v)-f(c) }{ v-c }<0
    \end{equation}
    parce que $v-c>0$ et $f(v)-f(c)<0$. Par conséquent, $\lim_{x\to 0}\tau(x)\leq 0$. Mettant les deux ensemble, nous avons $f'(c)=\lim_{x\to 0}\tau(x)=0$, et $c$ est le point que nous cherchions.
\end{proof}

Sur wikipédia, deux démonstrations complètement différentes sont proposées, celle qui est présentée ici est adaptée de celle qui est proposée par le célèste mathémator de \href{http://gconnan.free.fr/les\%20pdf/Deriv.pdf}{Téhessin le Rézéen}.

Le corollaire suivant est le théorème des \defe{accroissements finis}{théorème!accroissements finis!dans $\eR$}.

\begin{theorem}[Accroissements finis]       \label{ThoAccFinis}
    Si $f$ est une fonction continue sur $[a,b]$ et dérivable sur $]a,b[$, alors il existe au moins un réel $c\in]a,b[$ tel que $f(b)-f(a)=(b-a)f'(c)$.
\end{theorem}

\begin{proof}
    Considérons la fonction
    \begin{equation}
        \tau(x)=f(x)-\big( \frac{ f(b)-f(a) }{ b-a }x + f(a) - a\frac{ f(b)-f(a) }{ b-a } \big),
    \end{equation}
    c'est à dire la fonction qui donne la distance entre $f$ et le segment de droite qui lie $(a,f(a))$ à $(b,f(b))$. Par construction, $\tau(a)-\tau(b)=0$, donc le théorème de Rolle s'appliqe à $\tau$ pour laquelle il existe donc un $c\in]a,b[$ tel que $\tau'(c)=0$.

    En utilisant les règles de dérivation, nous trouvons que la dérivée de $\tau$ vaut
    \begin{equation}
        \tau'(x)= f'(x)-\frac{ f(b)-f(a) }{ b-a },
    \end{equation}
    donc dire que $\tau'(c)=0$ revient à dire que $f(b)-f(a)=(b-a)f'(c)$, ce qu'il fallait démontrer.
\end{proof}

\begin{corollary}
Soit $f$ une fonction dérivable sur $[a,b]$ telle que $f'(x) = 0$ pour tout $x \in [a,b]$. Alors $f$ est constante sur $[a,b]$.
\end{corollary}

\begin{proof}
    Si $f$ n'était pas constante sur $[a,b]$, il existerait un $x_1\in ]a,b[$ tel que $f(a)\neq f(x_1)$, et dans ce cas, il existerait un $c\in]a,x_1[$ tel que 
    \begin{equation}
        f'(c)=\frac{ f(x_1)-f(a) }{ x_1-a }\neq 0,
    \end{equation}
    ce qui contredirait les hypothèses.
\end{proof}

\begin{corollary}   \label{CorNErEgcQ}
    Soient $f$ et $g$, deux fonctions dérivables sur $[a,b]$ telles que
    \begin{equation}
        f'(x) = g'(x)
    \end{equation}
    pour tout $x \in [a,b]$. Alors existe un réel $C$ tel que $f (x) = g (x) + C$ pour tout $x\in [a,b]$.
\end{corollary}

\begin{proof}
    Considérons la fonction $h(x)=f(x)-g(x)$, dont la dérivée est, par hypothèse, nulle. L'annulation de la dérivée entraine par le corollaire \ref{CorNErEgcQ} que $h$ est  constante. Si $h(x)=C$, alors $f(x)=g(x)+C$, ce qu'il fallait prouver.
\end{proof}

Exprimé en termes des primitives, ce corollaire signifie que
\begin{corollary}  \label{CorZeroCst}
    Si $F$ et $G$ sont deux primitives de la même fonction $f$ sur un intervalle, alors il existe une constante $C$ pour laquelle $F(x)=G(x)+C$.
\end{corollary}
Cela signifie qu'il n'y a, en réalité, pas des milliards de primitives différentes à une fonction. Il y en a essentiellement une seule, et puis les autres, ce sont juste les mêmes, mais décalées d'une constante.

\begin{remark}
    L'hypothèse de se limiter à un intervalle est importante parce que si on considère la fonction sur deux intervalles disjoints, nous pouvons choisir la constante indépendamment dans l'un et dans l'autre. Par exemple la fonction
    \begin{equation}
        F(x)=\begin{cases}
            \ln(x)+1    &   \text{si \( x>0\)}\\
            \ln(x)-7    &    \text{si \( x<0\)}
        \end{cases}
    \end{equation}
    est une primitive de \( \frac{1}{ x }\) sur l'ensemble \( \eR\setminus\{ 0 \}\).

    Certains ne s'en privent pas. Le logiciel \href{ http://sagemath.org }{ Sage } par exemple fait ceci :
    \begin{verbatim}
sage: f(x)=1/x
sage: F=f.integrate(x)
sage: A=F(x)-F(-x)
sage: A.full_simplify()
I*pi
    \end{verbatim}
    En réalité lorsque \( x>0\), Sage définit \( \ln(-x)=\ln(x)+i\pi\). Cela a une certaine logique parce que \( \ln(-1)=i\pi\) (du fait que \(  e^{i\pi}=-1\)), mais si on ne le sait pas, ça peut étonner.
\end{remark}

\begin{normaltext}
    Il existe plusieurs primitives à une fonction donnée. En physique, la constante arbitraire est souvent fixée par une condition initiale, comme nous le verrons dans la section \ref{SecMRUAsecondeGGdQoT}.
\end{normaltext}

%+++++++++++++++++++++++++++++++++++++++++++++++++++++++++++++++++++++++++++++++++++++++++++++++++++++++++++++++++++++++++++
\section{Différentiabilité}
%+++++++++++++++++++++++++++++++++++++++++++++++++++++++++++++++++++++++++++++++++++++++++++++++++++++++++++++++++++++++++++

Note : pour savoir des choses sur la différentielle de \( f\colon E\to F\) avec \( E\) et \( F\) de dimension infinie, il faut aller voir la section \ref{SecLStKEmc}. Ici nous ne parlerons que de dimension finie.

%---------------------------------------------------------------------------------------------------------------------------
\subsection{Le pourquoi et le comment de la dérivée}
%---------------------------------------------------------------------------------------------------------------------------

La notion de dérivée est associée à la recherche de la droite tangente à une courbe. Reprenons rapidement le cheminement. La dérivée de $f\colon \eR\to \eR$ au point $a$ est un nombre $f'(a)$, qui définit donc une application linéaire dont le coefficients angulaire est $f'(a)$, et que nous notons $df_a$ :
\begin{equation}
    \begin{aligned}
        df_a\colon \eR&\to \eR \\
        u&\mapsto f'(a)u. 
    \end{aligned}
\end{equation}
La droite donnée par l'équation
\begin{equation}
    y(a+u)=f'(a)u
\end{equation}
est parallèle à la tangente en $a$. Pour trouver la tangente, il suffit de la décaler de la hauteur qu'il faut. L'équation de la droite tangente au graphe de $f$ au point $\big( a,f(a) \big)$ devient
\begin{equation}        \label{EqDiffRapTgDer}
    y(x)=f(a)+f'(a)(x-a)=f(a)+df_a(x-a).
\end{equation}
Nous nous proposons de généraliser cette formule au cas de la recherche du plan tangent à une surface.
 
%---------------------------------------------------------------------------------------------------------------------------
                    \subsection{Dérivée partielle et directionnelles}
%---------------------------------------------------------------------------------------------------------------------------

Soit une fonction $f:A\subset \mathbb{R}^n \rightarrow \mathbb{R}^m$. Si $n\neq 1$, la notion de \emph{dérivée} de la fonction $f$ n'a plus de sens puisqu'on ne peut plus parler de pente de \emph{la} tangente au graphe de $f$ en un point. On introduit alors quelque notions qui feront, en dimension quelconque, le même travail que la dérivée en dimension un : les dérivées directionnelles et la différentielle. Nous allons voir qu'en dimension un, la différentielle coïncide avec la dérivée.


\begin{definition} 
    Soit un point $a \in int\,A$ et un vecteur $u \in \mathbb{R}^n$ avec $\| u \| =1$. La dérivée de $f$ au point $a$ dans la direction $u$ est donnée par la limite suivante, si elle existe 
    \begin{equation}
        \frac{\partial f}{\partial u}(a) = \lim_{t\rightarrow 0}\frac{f(a+tu) - f(a)}{t}
    \end{equation}
\end{definition}

Géométriquement, il s'agit du taux de variation instantané de $f$ en $a$ dans la direction du vecteur $u$, c'est-à-dire de la pente de la tangente dans la direction du vecteur $u$ au graphe de $f$ au point $(a, f(a))$.

\begin{remark}
On peut reformuler la définition en écrivant $x = a + u$, on obtient~:
\begin{equation}
    \limite[u\neq 0]{u}{0} \frac{f(a+u)-f(a)-T(u)}{\norme{u}} = 0.
\end{equation}
\end{remark}

\begin{remark}
Pourquoi avons-nous posé la condition $\| u \|=1$ ? Le but de la dérivée directionnelle dans la direction $u$ est de savoir à quelle vitesse la fonction monte lorsque l'on se déplace en suivant la direction $u$. Cette information n'aura un caractère \og objectif\fg{} que si l'on avance à une vitesse donnée. En effet, si on se déplace deux fois plus vite, la fonction montera deux fois plus vite. Par convention, nous demandons donc d'avancer à vitesse $1$.
\end{remark}

\subsubsection*{Cas particulier où $n=2$:} $a = (a_1, a_2)$, $u =
(u_1,u_2)$ et
$$\frac{\partial f}{\partial u}(a_1, a_2) = \lim_{t\rightarrow
0}\frac{f(a_1+tu_1,a_2+tu_2) - f(a_1, a_2)}{t}$$

Un cas particulier des dérivées directionnelles est la dérivée partielle. Si nous considérons la base canonique $e_i$ de $\eR^n$, nous notons
\begin{equation}
    \frac{ \partial f }{ \partial x_i }=\frac{ \partial f }{ \partial e_i }.
\end{equation}
Dans le cas d'une fonction à deux variables, nous avons donc les deux dérivées partielles
\begin{equation}
    \begin{aligned}[]
        \frac{ \partial f }{ \partial x }(a)&&\text{et}&&\frac{ \partial f }{ \partial y }(a)
    \end{aligned}
\end{equation}
qui correspondent aux dérivées directionnelles dans les directions des axes. Ces deux nombres représentent de combien la fonction $f$ monte lorsqu'on part de $a$ en se déplaçant dans le sens des axes $X$ et $Y$.

%///////////////////////////////////////////////////////////////////////////////////////////////////////////////////////////
                    \subsubsection{Quelque propriétés et notations}
%///////////////////////////////////////////////////////////////////////////////////////////////////////////////////////////

\begin{enumerate}
\item
 $\forall \alpha \in \mathbb{R}$,
si $v = \alpha\,u$, alors $\frac{\partial f}{\partial v}(a) =
\alpha\,\frac{\partial f}{\partial u}(a)$.
\item Si on prend $u=e_j$ le $j$ème vecteur de la base canonique de
$\mathbb{R}^n$, alors
$$\frac{\partial f}{\partial e_j}(a) = \frac{\partial f}{\partial
x_j}(a)$$ c'est-à-dire que la dérivée de $f$ au point $a$ dans la
direction $e_j$ est la dérivée partielle de $f$ par rapport à sa
$j$ème variable.

\item 
Une fonction peut être dérivable dans certaines directions
mais pas dans d'autres (rappelez vous que si la limite à droite est
différente de la limite à gauche, la limite n'existe pas). 

\item
Même si une fonction est dérivable en un point dans toutes les
directions, on n'est pas sûr qu'elle soit continue en ce point. La
dérivabilité directionnelle n'est donc pas une notion suffisante
pour assurer la continuité. C'est pourquoi on introduit le concept
de \emph{différentiabilité}. 
\end{enumerate}

%---------------------------------------------------------------------------------------------------------------------------
                    \subsection{Différentielle}
%---------------------------------------------------------------------------------------------------------------------------

\begin{definition}      \label{DefDifferentiablFnRn}
Soit un point $a \in int\,A$. La fonction $f$ est \defe{différentiable}{différentiable} au point $a$ si il existe une application linéaire $df_a\colon \eR^n\to \eR^m$ telle que 
\begin{equation}        \label{EqDefDiffableT}
    \lim_{x\to a} \frac{f(x) - f(a) - df_a (x-a)}{\|x-a\|}=0.
\end{equation}
\end{definition}

Si $f$ est différentiable en $a$, l'application $df_a$ est appelée la différentielle de $f$ en $a$. Voyons comment cette application linéaire agit sur les vecteurs de $\mathbb{R}^n$.

Le théorème suivant reprend pas principales propriétés d'une fonction différentiable.
\begin{theorem}     \label{ThoRapPropDiffSi}
Si $f$ est différentiable en $a\in\eR^n$, alors
\begin{enumerate}
\item $f$ est continue en $a$.

\item  Toute les dérivées directionnelles $\partial_uf(a)$ existent et nous avons l'égalité
\begin{equation}        \label{EqDiffPartRap}
    \begin{aligned}
        df_a\colon \eR^n&\to \eR^m \\
        u&\mapsto df_a(u)=\frac{ \partial f }{ \partial u }(a)=\sum_i \frac{ \partial f }{ \partial x_i }u^i,
    \end{aligned}
\end{equation}
si les $u^i$ sont les composantes de $u$ dans la base canonique de $\eR^n$.

La différentielle de $f$ en $a$ envoie donc un vecteur $u$ sur la dérivée directionnelle de $f$ au point $a$ dans la direction $u$. 

\item\label{ItemThoDiffSiLin} L'application $df_a$ est une application linéaire.
\end{enumerate}
\end{theorem}
Le point \ref{ItemThoDiffSiLin} est évidement contenu dans la définition de la différentielle, mais c'est bien de la remettre en toute lettres. En regard avec la formule \eqref{EqDiffPartRap}, elle dit que $\partial_uf(a)$ est linéaire par rapport à $u$.

\subsubsection*{Cas particuliers} \begin{description} \item $n=1$:
$f: \mathbb{R}\rightarrow \mathbb{R}$ est dérivable en $a$ si et
seulement si $f$ est différentiable en $a$ et
$$df_a:\mathbb{R}\rightarrow \mathbb{R}: x \mapsto df_a(x) =
f'(a)\,.\,x$$ \item $n=2$: $f$ est différentiable en $a =(a_1, a_2)$
si et seulement si
$$\lim_{(v_1,v_2)\rightarrow (0,0)} \frac{f(a_1+v_1, a_2+v_2) - f(a_1,a_2) - [ \frac{\partial f}{\partial x}(a)\,v_1+
\frac{\partial f}{\partial y}(a)\,v_2]}{\sqrt{v_1^2+v_2^2}} = 0
$$\end{description}\vspace{0.3cm}


Parmi les vecteurs $u \in \mathbb{R}^n$, un vecteur d'origine $(a,
f(a))$ se distingue des autres: le vecteur gradient de $f$ en $a$
donnant la direction de plus grande pente de $f$ en
$a$.\vspace{0.3cm}

\begin{definition}
La courbe de niveau de $f$ associée à a est donnée par
$$ S_a = f^{-1}\,(f(a)) = \{(x_1, \ldots, x_n)\in \mathbb{R}^n : f(x_1, \ldots,
x_n)=f(a) \}$$
\end{definition}

%---------------------------------------------------------------------------------------------------------------------------
\subsection{Règles de calcul}
%---------------------------------------------------------------------------------------------------------------------------

\begin{proposition}[Règles de calculs] Soient $f$ et $g$ des fonctions
  différentiables en $g(a)$ et $a$ respectivement, alors la composée
  $f\circ g$ est différentiable en $a$ et
  \begin{equation*}
    d (f\circ g)_a = d f_{g(a)} \circ d g_a
  \end{equation*}
  et de plus les jacobiennes correspondantes vérifient
  \begin{equation*}
      J_{f\circ g}(a) = J_f\big( g(a) \big)J_g(a)
  \end{equation*}
  où le membre de droite est le produit (non-commutatif !) des deux matrices.
\end{proposition}

\begin{corollary}[Chain rule] Si $f : \eR^p \to \eR$ et $g : \eR \to
  \eR^p$, alors
  \begin{equation*}
    (f\circ g)^\prime(t) = \sum_{i=1}^p \pder f {x_i}(g(t)) g_i^\prime(t).
  \end{equation*}
\end{corollary}

\begin{remark}
  \begin{enumerate}
  \item Si $p = 1$, on retrouve la règle usuelle de dérivation de
    fonctions composées.

  \item 
      Si $g$ est à plusieurs variables, cette règle permet de déterminer les dérivées partielles de $f \circ g$, puisqu'une dérivée partielle peut être vue comme dérivée usuelle par rapport à une seule variable (voir remarque page \pageref{deriveepartielles}).

  \item Si $f$ est à valeurs vectorielles, cette formule permet de
    retrouver la jacobienne de $f \circ g$ puisqu'il suffit de traiter
    chaque composante de $f$ séparément.
  \end{enumerate}
\end{remark}

%---------------------------------------------------------------------------------------------------------------------------
                    \subsection{Gradient et recherche du plan tangent}
%---------------------------------------------------------------------------------------------------------------------------

Nous avons maintenant en main les concepts utiles pour trouver l'équation du plan tangent à une surface.

De la même manière que la tangente à une courbe était la droite de coefficient directeur donné par la dérivée, maintenant, le plan tangent à une surface est le plan dont les vecteurs directeurs sont les dérivées partielles :

La généralisation de l'équation \eqref{EqDiffRapTgDer} est 
\begin{equation}        \label{EqDefPlanTag}
    T_a(x)=f(a)+\sum_i\frac{ \partial f }{ \partial x_i }(a)(x-a)^i
\end{equation}

Nous introduisons aussi souvent l'opérateur différentiel abstrait \defe{nabla}{nabla}, noté $\nabla$ et qui est donné par le vecteur
\begin{equation}
    \nabla=\left( \frac{ \partial  }{ \partial x_1 },\ldots,\frac{ \partial  }{ \partial x_n } \right).
\end{equation}
Les égalités suivantes sont juste des notations, sommes toutes logiques, liées à $\nabla$ :
\begin{equation}
    \nabla f=\left( \frac{ \partial f }{ \partial x_1 },\ldots,\frac{ \partial f }{ \partial x_n } \right),
\end{equation}
et
\begin{equation}        \label{EqDefGradient}
    \nabla f(a) = \left(\frac{\partial f}{\partial x_1}(a), \frac{\partial f}{\partial x_2}(a), \ldots, \frac{\partial f}{\partial x_n}(a)\right).
\end{equation}
Ce dernier est un élément de $\eR^n$ : chaque entrée est un nombre réel.

\begin{definition} 
Le vecteur gradient de $f$ au point $a$ est le vecteur donné par la formule \eqref{EqDefGradient}.
\end{definition}
La notation $\nabla$ permet d'écrire la différentielle sous forme un peu plus compacte. En effet, la formule \eqref{EqDiffPartRap} peut être notée
\begin{equation}
    df_a(u)=\scal{\nabla f(a)}{u}.
\end{equation}

En utilisant ce produit scalaire, l'équation \eqref{EqDefPlanTag} peut se récrire
\begin{equation}
    T_a(x)=f(a)+\sum_i\frac{ \partial f }{ \partial x_i }(a)(x-a)^i=f(a)+\scal{\nabla f(a)}{x-a}.
\end{equation}

Afin d'éviter les confusions, il est parfois souhaitable de bien mettre les parenthèses et noter $(\nabla f)(a)$ au lieu de $\nabla f(a)$.

\begin{proposition}
$\nabla f(a)\,\bot \,S_a$
\end{proposition}


\begin{equation}        \label{EqPlanTgSansNabla}
    z=f(a)+\sum_i\frac{ \partial f }{ \partial f }(a)(x-a)^i.
\end{equation}

\subsubsection*{Cas particulier où $n=2$:} 
Le plan $T_a$ avec $a=(a_1,a_2)$ a pour équation dans $\eR^3$:
\begin{equation}        \label{EqPlanTgEnDimDeux}
    z = f(a_1,a_2) + \frac{\partial f}{\partial x}(a_1,a_2)\,(x-a_1)+ \frac{\partial f}{\partial y}(a_1,a_2)\,(y-a_2).
\end{equation}

\begin{definition}
  Soit $f : \eR^n \to\eR$ une fonction différentiable en un point
  $a$. Le \emph{plan tangent} au graphe de $f$ en $(a,f(a))$ est
  l'ensemble des points
  \begin{equation*}
    \begin{split}
      T_af &= \{ (x,z) \in \eR^n \times \eR \tq z = f(a) + d f_a (x-a)\}\\
      &= \{ (x,z) \in \eR^n \times \eR \tq z = f(a) + \scalprod{\nabla f(a)}{x-a}\}
    \end{split}
  \end{equation*}
\end{definition}

%---------------------------------------------------------------------------------------------------------------------------
                    \subsection{Différentielle comme élément de l'espace dual}
%---------------------------------------------------------------------------------------------------------------------------

Si nous considérons la base canonique $\{ e_i \}_{i=1,\ldots,n}$ de $\eR^n$. À partir d'elle, nous considérons la \defe{base duale}{base!duale}. En termes pratiques, nous définissons $dx_i$ comme la forme sur $\eR^n$ qui à un vecteur $u$ fait correspondre sa composante $i$ :
\begin{equation}
    dx_i\begin{pmatrix}
    u^1 \\ 
    \vdots  \\ 
    u^n 
\end{pmatrix}=u^i.
\end{equation}
En termes savants, $dx_i$ est le dual de $e_i$. Si tu ne l'as pas encore compris, Jean Doyen va te le faire comprendre !


Maintenant, dans la formule \eqref{EqDiffPartRap}, nous pouvons remplacer $u^i$ par $dx_i(u)$, et écrire
\begin{equation}
    df_a(u)=\sum_i\frac{ \partial f }{ \partial x_i }(a)u^i=\sum_i\frac{ \partial f }{ \partial x_i }(a)dx_i(u).
\end{equation}
Ce qui arrive tout à droite est explicitement vu comme une forme sur $\eR$, dont les composantes dans la base duale sont les dérivées partielles de $f$ au point $a$, agissant sur $u$. En faisant un pas en arrière, nous omettons le $u$, et nous écrivons
\begin{equation}
    df_a=\sum_{i=1}^n\frac{ \partial f }{ \partial x_i }(a)dx^i
\end{equation}

Cette notation $dx_i$ pour la forme duale de $e_i$ est en réalité parfaitement logique parce que $dx^i$ est la différentielle de la projection
\begin{equation}
    \begin{aligned}
        x^i\colon \eR^n&\to \eR \\
        (x^1,\ldots,x^n)&\mapsto x^i. 
    \end{aligned}
\end{equation}
Je te laisse un peu méditer sur cette différentielle de la projection. L'important est que tu aies compris cela d'ici la fin de ta deuxième année.


%---------------------------------------------------------------------------------------------------------------------------
                    \subsection{Prouver qu'un fonction n'est pas différentiable}
%---------------------------------------------------------------------------------------------------------------------------

Chacun des point du théorème \ref{ThoRapPropDiffSi} est en soi un critère pour montrer qu'une fonction n'est pas différentiable en un point.

%///////////////////////////////////////////////////////////////////////////////////////////////////////////////////////////
                    \subsubsection{Continuité}
%///////////////////////////////////////////////////////////////////////////////////////////////////////////////////////////


Le premier critère à vérifier est donc la continuité. Si une fonction n'est pas continue en un point, alors elle n'y sera pas différentiable. Pour rappel, la continuité en $a$ se teste en vérifiant si $\lim_{x\to a}f(x)=f(a)$.

%///////////////////////////////////////////////////////////////////////////////////////////////////////////////////////////
                    \subsubsection{Linéarité}
%///////////////////////////////////////////////////////////////////////////////////////////////////////////////////////////

Un second test est la linéarité de la dérivée directionnelle par rapport à la direction : l'application $u\mapsto\frac{ \partial f }{ \partial u }(a)$ doit être linéaire, sinon $df_a$ n'existe pas.

\begin{example}     \label{Exemple0046Diff}
Examinons la fonction
\begin{equation}
    \begin{aligned}
        f\colon \eR^2&\to \eR \\
        (x,y)&\mapsto \begin{cases}
    \frac{ xy^2 }{ x^2+y^4 }    &   \text{si $(x,y)\neq (0,0)$}\\
    0   &    \text{sinon}.
\end{cases}
    \end{aligned}
\end{equation}
Prenons $u=(u_1,u_2)$ et calculons la dérivée de $f$ dans la direction de $u$ au point~$(0,0)$ :
\begin{equation}
    \begin{aligned}[]
        \frac{ \partial f }{ \partial u }(0,0)  
            &=\lim_{t\to 0}\frac{ f(tu_1,tu_2)-f(0,0) }{ t }\\
            &=\lim_{t\to 0}\frac{1}{ t }\left( \frac{ tu_1t^2u_2 }{ t^2u_1^2+t^4u_2^4 } \right)\\
            &=\lim_{t\to 0}\left( \frac{ u_1u_2^2 }{ u_1^2+t^2u_2^4 } \right)\\
            &=\begin{cases}
    \frac{ u_2^2 }{ u_1 }   &   \text{si $u_1\neq 0$}\\
    0   &    \text{si $u_1=0$}.
\end{cases}
    \end{aligned}
\end{equation}
Cette application n'est pas linéaire par rapport à $u$. En effet, notons
\begin{equation}
    \begin{aligned}
        A\colon \eR^n&\to \eR \\
        u&\mapsto \frac{ \partial f }{ \partial u }(0,0), 
    \end{aligned}
\end{equation}
et vérifions que pour tout $u$ et $v$ dans $\eR^n$ et $\lambda\in\eR$, nous ayons $A(\lambda u)=\lambda A(u)$ et $A(u+v)=A(u)+A(v)$. Le premier fonctionne parce que
\begin{equation}
    A(\lambda u)=A(\lambda u_1,\lambda u_2)=\frac{ \lambda^2 u_2^2 }{ \lambda u_1 }=\lambda\frac{ u_2^2 }{ u_1 }=\lambda A(u).
\end{equation}
Mais nous avons par exemple
\begin{equation}
    A\big( (0,1)+(2,3) \big)=A(2,4)=\frac{ 16 }{ 2 }=8,
\end{equation}
tandis que
\begin{equation}
    A(0,1)+A(2,3)=0+\frac{ 9 }{ 2 }\neq 8.
\end{equation}
La fonction $f$ n'est donc pas différentiable en $(0,0)$, parce que la candidate différentielle, $df_{(0,0)}(u)=\frac{ \partial f }{ \partial u }(0,0)$, n'est même pas linéaire.

\end{example}

Voici une autre façon de traiter la fonction de l'exemple \ref{Exemple0046Diff}.

\begin{example} \label{ExeFHmCLII}
    La figure \ref{LabelFigFWJuNhU} représente le domaine d'une fonction $f\colon \eR^2\to \eR$, et sur chacune des parties, elle est définie différemment.
    \newcommand{\CaptionFigFWJuNhU}{La fonction de l'exemple \ref{ExeFHmCLII}.}
\input{Fig_FWJuNhU.pstricks}

L'expression de $f$ est ici
\begin{equation}
  f(x,y) =
  \begin{cases}
    xy & \text{si $x < 0$ et $y > 0$}\\
    x-y & \text{si $x \geq 0$ et $y \geq 0$}\\
    x^2y & \text{si $x > 0$ et $y < 0$}\\
    x+y & \text{sinon.}
  \end{cases}
\end{equation}

On note que les deux axes forment une zone à problèmes. La zone hors
des axes est un ouvert sur lequel $f$ est différentiable car composée
de polynômes. Analysons chacun des points de la forme $(a,b)$ dans la
zone à problèmes (c'est-à-dire si $ab = 0$).

\subparagraph{Si $a = 0$ et $b > 0$} Un tel point $(0,b)$ est sur
l'axe verticale, dans la moitié supérieure. Pour calculer la limite de
$f$ en ce point, on peut restreindre notre étude au demi-plan ouvert
$y > 0$, ce qui revient à comparer la limite
\begin{equation*}
  \limite[y>0\\x\geq 0] {(x,y)} {(0,b)} f(x,y) =   \limite[y>0\\x\geq
  0] {(x,y)} {(0,b)} x-y = 0 - b = -b
\end{equation*}
avec la limite
\begin{equation*}
  \limite[y>0\\x<0] {(x,y)} {(0,b)} f(x,y) =   \limite[y>0\\x<0]
  {(x,y)} {(0,b)} xy = 0 b = 0
\end{equation*}
qui sont différentes puisque $b$ est supposé non-nul.

\conclusion $f$ n'est pas continue en un point du type $(0,b)$ avec $b
> 0$.

\subparagraph{Si $a = 0$ et $b < 0$} Un tel point $(0,b)$ est sur
l'axe verticale, dans la moitié inférieure. Pour calculer la limite de
$f$ en ce point, on peut restreindre notre étude au demi-plan ouvert
$y < 0$, ce qui revient à comparer la limite
\begin{equation*}
  \limite[y<0\\x\geq 0] {(x,y)} {(0,b)} f(x,y) =   \limite[y<0\\x\geq
  0] {(x,y)} {(0,b)} x^2 y = 0^2 b = 0
\end{equation*}
avec la limite
\begin{equation*}
  \limite[y<0\\x<0] {(x,y)} {(0,b)} f(x,y) =   \limite[y<0\\x<0]
  {(x,y)} {(0,b)} x+y = 0 + b = b
\end{equation*}
qui sont différentes puisque $b$ est supposé non-nul.

\conclusion $f$ n'est pas continue en un point du type $(0,b)$ avec $b
< 0$.

\subparagraph{Si $a > 0$ et $b = 0$} Un tel point $(a,0)$ est sur
l'axe horizontal, dans la moitié droite. Pour calculer la limite de
$f$ en ce point, on peut restreindre notre étude au demi-plan ouvert
$x > 0$, ce qui revient à comparer la limite
\begin{equation*}
  \limite[x>0\\y \geq 0] {(x,y)} {(a,0)} f(x,y) =   \limite[x>0\\y \geq
  0] {(x,y)} {(a,0)} x-y = a - 0 = a
\end{equation*}
avec la limite
\begin{equation*}
  \limite[x>0\\y < 0] {(x,y)} {(a,0)} f(x,y) =   \limite[x>0\\y < 0]
  {(x,y)} {(a,0)} x^2y = a^2 0 = 0
\end{equation*}
qui sont différentes puisque $a$ est supposé non-nul.

\conclusion $f$ n'est pas continue en un point du type $(a,0)$ avec $a
> 0$.

\subparagraph{Si $a < 0$ et $b = 0$} Un tel point $(a,0)$ est sur
l'axe horizontal, dans la moitié gauche. Pour calculer la limite de
$f$ en ce point, on peut restreindre notre étude au demi-plan ouvert
$x < 0$, ce qui revient à comparer la limite
\begin{equation*}
  \limite[x<0\\y> 0] {(x,y)} {(a,0)} f(x,y) =   \limite[x<0\\y>
  0] {(x,y)} {(a,0)} x y = a 0 = 0
\end{equation*}
avec la limite
\begin{equation*}
  \limite[x<0\\y\leq 0] {(x,y)} {(a,0)} f(x,y) =   \limite[x<0\\y\leq0]
  {(x,y)} {(a,0)} x+y = a + 0 = a
\end{equation*}
qui sont différentes puisque $a$ est supposé non-nul.

\conclusion $f$ n'est pas continue en un point du type $(a,0)$ avec $a
< 0$.

\subparagraph{Si $a = 0$ et $b = 0$} Le cas du point $(0,0)$ est
particulier, puisque il est adhérent aux quatre composantes du
domaine où la fonction est définie différemment. Pour étudier la
continuité, il faut donc étudier quatre limites. Ces limites ont déjà
été étudiées ci-dessus et valent toutes $0$, ce qui prouve la
continuité de $f$ en $(0,0)$.

En ce qui concerne la différentiabilité, on sait qu'il est nécessaire
que toutes les dérivées directionnelles existent. Calculons la dérivée
dans la direction $(0,1)$ (au point $(0,0)$)~:
\begin{equation*}
  \limite[t\neq0] t 0 \frac{f((0,0) + t(0,1)) - f(0,0)}{t} =%
  \limite[t\neq0] t 0 \frac{f(0,t)}{t} = \ldots
\end{equation*}
qu'on sépare en deux cas, car $f(0,t)$ possède une formule différente
si $t < 0$ ou si $t \geq 0$~:
\begin{equation*}
  \limite[t\neq0] t 0 \frac{f(0,t)}{t} = %
  \begin{arrowcases}
    \limite[t<0] t 0 \frac{f(0,t)}{t} = \limite[t<0] t 0 \frac{0+t}{t} = 1\\
    \limite[t\geq0] t 0 \frac{f(0,t)}{t} = \limite[t\geq0] t 0
    \frac{0-t}{t} = -1
  \end{arrowcases}
\end{equation*}
ce qui prouve que la limite n'existe pas, donc que la dérivée
directionnelle n'existe pas, et finalement que la fonction n'est pas
différentiable.

\conclusion La fonction donnée est continue hors des axes et au point
$(0,0)$, mais discontinue partout ailleurs sur les axes. Elle est
différentiable hors des axes, mais ne l'est pas sur les axes.

\end{example}

%///////////////////////////////////////////////////////////////////////////////////////////////////////////////////////////
                    \subsubsection{Cohérence des dérivées partielles et directionnelle}
%///////////////////////////////////////////////////////////////////////////////////////////////////////////////////////////

Dans la pratique, nous pouvons calculer $\partial_uf(a)$ pour une direction $u$ générale, et puis en déduire $\partial_xf$ et $\partial_yf$ comme cas particuliers en posant $u=(1,0)$ et $u=(0,1)$. Une chose incroyable, mais pourtant possible est qu'il peut arriver que
\begin{equation}
    \frac{ \partial f }{ \partial u }(a)\neq \sum_i\frac{ \partial f }{ \partial x_i }(a)u^i.
\end{equation}
Ceci se produit lorsque $f$ n'est pas différentiable en $a$. En voici un exemple.

%///////////////////////////////////////////////////////////////////////////////////////////////////////////////////////////
                    \subsubsection{Un candidat dans la définition (marche toujours)}
%///////////////////////////////////////////////////////////////////////////////////////////////////////////////////////////

Lorsqu'une fonction est donné, un candidat différentielle au point $(a_1,a_2)$ est souvent assez simple à trouver en un point :
\begin{equation}
    T(u_1,u_2)=\frac{ \partial f }{ \partial x }(a_1,a_2)u_1+\frac{ \partial f }{ \partial y }(a_1,a_2)u_2.
\end{equation}
L'application $T$ est la candidate différentielle en ce sens que si la différentielle existe, alors elle est égale à $T$. Ensuite, il faut vérifier si
\begin{equation}        \label{EqLimDefDiff}
    \lim_{(x,y)\to (a_1,a_2)} \frac{f(x,y) - f(a_1,a_2) - T\big( (x,y)-(a_1,a_2) \big)}{\| (x,y)-(a_1,a_2) \|}=0
\end{equation}
ou non. Si oui, alors la différentielle existe et $df_{(a,b)}(u)=T(u)$, sinon\footnote{y compris si la limite \eqref{EqLimDefDiff} n'existe même pas.}, la différentielle n'existe pas.

Attention : dans la ZAP, les dérivées partielles $\partial_xf$ et $\partial_yf$ ne peuvent en général pas être calculées en utilisant les règles de calcul (c'est bien pour ça que la ZAP est une zone à problèmes). Il faut d'office utiliser la définition
\begin{equation}
    \frac{ \partial f }{ \partial x }(a_1,a_2)=\lim_{t\to 0}\frac{ f(a_1+t,a_2)-f(a_1,a_2) }{ t },
\end{equation}
et la définition correspondante pour $\partial_yf$.


\subsubsection*{Conclusion}
Soient $f:A\subset \eR^n \rightarrow \eR^m$, et $a\in int\,A$. Si $f$ est différentiable en $a$, $$ (df_a (e_j))_i = d(f_i)_a(e_j) =\frac{\partial f_i}{\partial x_j}(a)= [Jac(f)_{|a}]_{ij}$$ et la matrice de l'application linéaire $df_a$ est la matrice jacobienne $m\times n$ de $f$ en $a$ notée $Jac(f)_{|a}$.


%---------------------------------------------------------------------------------------------------------------------------
                    \subsection{Calcul de différentielles}
%---------------------------------------------------------------------------------------------------------------------------


\begin{remark}      \label{deriveepartielles}
  En pratique, ayant une formule pour la fonction $f$, on dérive --grâce aux règles usuelles de dérivation-- par rapport à la variable $x_i$ en considérant que les autres ($x_j$ avec $j \neq i$) sont des constantes.
\end{remark}

\begin{example}Pour $f(x,y) = xy + x^2$, les dérivées partielles
  s'écrivent
  \begin{equation*}
    \frac{\partial f}{\partial x} = y + 2x \quad\text{et}\quad \frac{\partial f}{\partial y} = x
  \end{equation*}
\end{example}


Des \emph{règles de calcul} sont d'application. En particulier, quand
ces opérations existent, les sommes, différences, produits, quotients
et compositions d'applications différentiables sont différentiables.

Toute application linéaire est différentiable, et sa différentielle en
tout point est égale à l'application elle-même. En particulier, les
\Defn{projections canoniques}, c'est-à-dire les applications du type
$(x,y,z) \mapsto y$, sont linéaires donc différentiables.

\begin{example}
Les cas suivants sont faciles :
  \begin{enumerate}
  \item En combinant les projections canoniques avec les règles de
    calculs, on obtient que toute fonction polynômiale à $n$ variables
    est différentiable comme application de $\eR^n$ dans $\eR$.

  \item Toute fonction rationnelle, du type $f(x) \pardef
    \frac{P(x)}{Q(x)}$ où $P$ et $Q$ sont des polynômes, est
    différentiable en tout point $a$ tel que $Q(a) \neq 0$.

  \item Pour une fonction d'une variable $f : D \subset \eR \to
    \eR$, le caractère différentiable et le caractère dérivable
    coïncident. De plus, on a
    \begin{equation*}
      d f_a(u) = f'(a) u.
    \end{equation*}
  \end{enumerate}
\end{example}

%---------------------------------------------------------------------------------------------------------------------------
                    \subsection{Notes idéologiques quant au concept de plan tangent}
%---------------------------------------------------------------------------------------------------------------------------
\label{ssecConceptPlanTag}

Notons $G$, le graphe d'une fonction $f$, c'est à dire
\begin{equation}
    G=\{ (x,y,z)\in\eR^3\tq z=f(x,y) \}.
\end{equation}
Première affirmation : si $\gamma\colon \eR\to G$ est une courbe telle que $\gamma(0)=\big( a,f(a) \big)$, alors $\gamma'(0)\in\eR^n$ est dans le plan tangent à $G$ au point $\big( a,f(a) \big)$.

Plus fort : tous les éléments du plan tangent sont de cette forme.

Le plan tangent à $G$ en un point $x\in G$ est donc constitué des vecteurs vitesse de tous les chemins qui passent par $x$.

Prenons maintenant $S$, une courbe de niveau de $G$, c'est à dire
\begin{equation}
    S=\{ (x,y)\in\eR^2\tq f(x,y)=C \}.
\end{equation}
Si nous prenons un chemin dans $G$ qui est, de plus, contraint à $S$, c'est à dire tel que $\gamma(t)\in S$, alors $\gamma'(0)$ sera tangent à $G$ (ça, on le savait déjà), mais en plus, $\gamma'(0)$ sera tangent à $S$, ce qui est logique.

La morale est que si vous prenez un chemin qui se ballade dans n'importe quoi, alors la dérivée du chemin sera un vecteur tangent à ce n'importe quoi.

En outre, si $\gamma(t)\in S$ et $\gamma(0)=a$, alors
\begin{equation}
    \scal{\nabla f(a)}{\gamma'(0)}=0,
\end{equation}
c'est à dire que le vecteur tangent à la courbe de niveau est perpendiculaire au gradient. Cela est intuitivement logique parce que la tangente à la courbe de niveau correspond à la direction de \emph{moins} grande pente.


%+++++++++++++++++++++++++++++++++++++++++++++++++++++++++++++++++++++++++++++++++++++++++++++++++++++++++++++++++++++++++++
\section{Jacobienne}
%+++++++++++++++++++++++++++++++++++++++++++++++++++++++++++++++++++++++++++++++++++++++++++++++++++++++++++++++++++++++++++

\subsection{Rappels et définitions}

Dans cette section nous considérons des fonctions $f : D \to \eR^m$
où $D \subset \eR^n$, et un point $a \in \interieur D$ où $f$ est
différentiable.
\begin{remark}
  La définition de continuité (resp. différentiabilité) pour une
  fonction à valeurs vectorielles est celle introduite précédemment,
  et on remarque que pour avoir la continuité
  (resp. différentiabilité) de $f$ en un point, il faut et il suffit
  de chacune des composantes de $f = (f_1,\ldots, f_m)$, vues
  séparément comme fonctions à $n$ variables et à valeurs réelles,
  soit continue (resp. différentiable) en ce point.
\end{remark}

\begin{definition}
    La \defe{jacobienne}{matrice!jacobienne} de $f$ en $a$ est la matrice de l'application linéaire donnée par la différentielle. Elle a de nombreuses notations
  \begin{equation}
      J_f(a) = \frac{ \partial (f_1,\ldots, f_m) }{ \partial x_1,\ldots, x_m }=
    \begin{pmatrix}
      \pder {f_1} {x_1}(a) & \ldots &\pder {f_1} {x_n}(a)\\
      \vdots& & \vdots\\
      \pder {f_m} {x_1}(a) & \ldots &\pder {f_m} {x_n}(a)
    \end{pmatrix}
  \end{equation}
  Autrement dit, c'est la matrice composée de l'ensemble des dérivées partielles de $f$. Le \defe{jacobien}{jacobien} de \( f\) au point \( a\) est le déterminant de cette matrice.

  Si $m = 1$, cette matrice ne contient qu'une ligne ; c'est donc un vecteur appelé le \defe{gradient}{gradient} de $f$ au point $a$ et noté $\nabla f(a)$.
\end{definition}

\begin{remark}
  \begin{enumerate}
  \item Si la fonction est supposée différentiable, calculer la
    jacobienne revient à connaître la différentielle. En effet, par
    linéarité de la différentielle et par définition des dérivées
    partielles, nous avons
    \begin{equation*}
      d f_a (u) =%
      \begin{pmatrix}
        \pder {f_1} {x_1}(a) & \ldots &\pder {f_1} {x_n}(a)\\
        \vdots& & \vdots\\
        \pder {f_m} {x_1}(a) & \ldots &\pder {f_m} {x_n}(a)
      \end{pmatrix}
      \begin{pmatrix}u_1\\\vdots\\u_n\end{pmatrix}
    \end{equation*}
    où $u = (u_1, \ldots, u_n)$ et où le membre de droite est un
    produit matriciel

  \item Remarquons que la jacobienne peut exister en un point donné
    sans que la fonction soit différentiable en ce point !
  \end{enumerate}
\end{remark}
