Les champs de vecteurs et tout ce qui s'y rapportent jouent un rôle crucial en électromagnétisme. Voir par exemple \cite{Schomblond_em}.

%+++++++++++++++++++++++++++++++++++++++++++++++++++++++++++++++++++++++++++++++++++++++++++++++++++++++++++++++++++++++++++
\section{Les fonctions à valeurs vectorielles}
%+++++++++++++++++++++++++++++++++++++++++++++++++++++++++++++++++++++++++++++++++++++++++++++++++++++++++++++++++++++++++++

Jusqu'à présent nous avons vu des fonctions de plusieurs variables qui prenaient leurs valeurs dans $\eR$. Nous allons maintenant voir ce qu'il se passe lorsque les fonctions prennent leurs valeurs dans $\eR^3$.

Une fonction d'une variable est dite \defe{à valeurs vectorielles}{fonction!valeurs vectorielles} lorsque
\begin{equation}
    \begin{aligned}
        f\colon I\subset \eR&\to \eR^3 \\
        f(x)&=\begin{pmatrix}
            f_1(x)    \\ 
            f_2(x)    \\ 
            f_3(x)    
        \end{pmatrix}.
    \end{aligned}
\end{equation}
Les fonctions $f_i\colon \eR\to \eR$ sont les \defe{composantes}{composante} de $f$. Ce que nous avons raconté à propos des dérivées passe facilement :
\begin{equation}
    \frac{ f(a+\epsilon)-f(a) }{ \epsilon }=
    \begin{pmatrix}
        \frac{ f_1(a+\epsilon)-f_1(a) }{ \epsilon }    \\ 
        \frac{ f_2(a+\epsilon)-f_2(a) }{ \epsilon }    \\ 
        \frac{ f_3(a+\epsilon)-f_3(a) }{ \epsilon }    
    \end{pmatrix}.
\end{equation}
En particulier dès que les fonctions $f_i$ sont dérivables, nous avons
\begin{equation}
    f'(a)=\begin{pmatrix}
        f_1'(a)    \\ 
        f_2'(a)    \\ 
        f_3'(a)    
    \end{pmatrix}
\end{equation}
comme dérivée de la fonction. Cette dérivée est un vecteur.

\begin{example}
    Si
    \begin{equation}
        f\colon x\in\eR\mapsto \begin{pmatrix}
            x^2 e^{x}    \\ 
            \cos(x^2)    \\ 
            x^3+x    
        \end{pmatrix},
    \end{equation}
    alors
    \begin{equation}
        f'(x)=\begin{pmatrix}
            2xe^x+x^2e^x    \\ 
            -2x\sin(x^2)    \\ 
            3x^2+1    
        \end{pmatrix}.
    \end{equation}
\end{example}

%+++++++++++++++++++++++++++++++++++++++++++++++++++++++++++++++++++++++++++++++++++++++++++++++++++++++++++++++++++++++++++
\section{Fonctions vectorielles de plusieurs variables}
%+++++++++++++++++++++++++++++++++++++++++++++++++++++++++++++++++++++++++++++++++++++++++++++++++++++++++++++++++++++++++++

Ce sont les fonctions de la forme
\begin{equation}
    \begin{aligned}
        f\colon \eR^3&\to \eR^3 \\
        \begin{pmatrix}
            x    \\ 
            y    \\ 
            z    
        \end{pmatrix}&\mapsto \begin{pmatrix}
            f_1(x,y,z)\\
            f_2(x,y,z)\\
            f_3(x,y,z)
        \end{pmatrix}.
    \end{aligned}
\end{equation}

En ce qui concerne les dérivées, tout se passe comme avant. Si les dérivées partielles des composantes $f_i$ existent au point $a\in\eR^3$, alors
\begin{equation}
    \begin{aligned}[]
        \frac{ \partial f }{ \partial x }(a)&=\begin{pmatrix}
            \partial_xf_1(a)    \\ 
            \partial_xf_2(a)    \\ 
            \partial_xf_3(a)    \\ 
        \end{pmatrix},&
        \frac{ \partial f }{ \partial y }(a)&=\begin{pmatrix}
            \partial_yf_1(a)    \\ 
            \partial_yf_2(a)    \\ 
            \partial_yf_3(a)    \\ 
        \end{pmatrix},&
        \frac{ \partial f }{ \partial z }(a)&=\begin{pmatrix}
            \partial_zf_1(a)    \\ 
            \partial_zf_2(a)    \\ 
            \partial_zf_3(a)    \\ 
        \end{pmatrix}.
    \end{aligned}
\end{equation}

%+++++++++++++++++++++++++++++++++++++++++++++++++++++++++++++++++++++++++++++++++++++++++++++++++++++++++++++++++++++++++++
\section{Champs de vecteurs}
%+++++++++++++++++++++++++++++++++++++++++++++++++++++++++++++++++++++++++++++++++++++++++++++++++++++++++++++++++++++++++++

Un champ de vecteur est une fonction $f\colon \eR^3\to \eR^3$. Géométriquement, il s'agit simplement de mettre un vecteur en chaque point de l'espace. Cela arrive très souvent en physique.

\begin{example}
    Si un fluide (eau, gaz) coule dans un tube, en tout point le point a une vitesse, qui sera un vecteur généralement dirigé le long du tube.
\end{example}

\begin{example}
    La force d'attraction de la Terre sur une masse $m$ située au point $r=(x,y,z)$ est donnée par
    \begin{equation}
        F(r)=-G\frac{ Mmr }{ \| r \|^3 }.
    \end{equation}
    Dans cette expression, tant $r$ que $F(r)$ sont des vecteurs. Nous l'avons représenté sur la figure \ref{LabelFigChampGraviation}.
    \newcommand{\CaptionFigChampGraviation}{Le champ de gravitation de la Terre.}
    \input{Fig_ChampGraviation.pstricks}

    L'application
    \begin{equation}
        \begin{aligned}
            F\colon \eR^3&\to \eR^3 \\
            r&\mapsto F(r) 
        \end{aligned}
    \end{equation}
    est le champ gravitationnel de la Terre.

\end{example}

%---------------------------------------------------------------------------------------------------------------------------
\subsection{Matrice jacobienne}
%---------------------------------------------------------------------------------------------------------------------------

La \defe{matrice jacobienne}{jacobien} de la fonction $f\colon \eR^3\to \eR^3$ au point $a\in\eR^3$ est la matrice dont les colonnes sont les vecteurs $\frac{ \partial f }{ \partial x }(a)$, $\frac{ \partial f }{ \partial y }(a)$ et $\frac{ \partial f }{ \partial z }(a)$, c'est à dire
\begin{equation}
    J_f(a)=\begin{pmatrix}
        \frac{ \partial f_1 }{ \partial x }(a)   &   \frac{ \partial f_1 }{ \partial y }(a)    &   \frac{ \partial f_1 }{ \partial z }(a)    \\
        \frac{ \partial f_2 }{ \partial x }(a)   &   \frac{ \partial f_2 }{ \partial y }(a)    &   \frac{ \partial f_2 }{ \partial z }(a)    \\
        \frac{ \partial f_3 }{ \partial x }(a)   &   \frac{ \partial f_3 }{ \partial y }(a)    &   \frac{ \partial f_3 }{ \partial z }(a)    
    \end{pmatrix}.
\end{equation}

\begin{example}
    Si 
    \begin{equation}
        f(x,y,z)=\begin{pmatrix}
            xy e^{z}    \\ 
            x^2+\cos(yz)    \\ 
            xyz    
        \end{pmatrix},
    \end{equation}
    alors
    \begin{equation}
        J_f(x,y,z)=\begin{pmatrix}
            ye^z    &   xe^z    &   xye^z    \\
            2x    &   -z\sin(yz)    &   -y\sin(yz)    \\
            yz    &   xz    &   xy
        \end{pmatrix}.
    \end{equation}
\end{example}

%+++++++++++++++++++++++++++++++++++++++++++++++++++++++++++++++++++++++++++++++++++++++++++++++++++++++++++++++++++++++++++
\section{Courbes paramétrés}
%+++++++++++++++++++++++++++++++++++++++++++++++++++++++++++++++++++++++++++++++++++++++++++++++++++++++++++++++++++++++++++

%---------------------------------------------------------------------------------------------------------------------------
\subsection{Définitions et exemples}
%---------------------------------------------------------------------------------------------------------------------------

\begin{definition}
    Un \defe{chemin}{chemin} dans $\eR$ est une application continue
    \begin{equation}
        \begin{aligned}
            \sigma\colon [a,b]&\to \eR^3 \\
            t&\mapsto \sigma(t). 
        \end{aligned}
    \end{equation}
\end{definition}

La fonction $\sigma'(t)$ est la \defe{vitesse}{vitesse d'un chemin} du chemin $\sigma$. Si la fonction $t\mapsto\sigma(t)$ est dérivable, on dit que $\sigma''(t)$ est l'\defe{accélération}{accélération d'un chemin}. Les points $\sigma(a)$ et $\sigma(b)$ sont les extrémités du chemin. L'ensemble
\begin{equation}
    \{ \sigma(t)\tq t\in\mathopen[ a , b \mathclose] \}
\end{equation}
est la \defe{courbe}{courbe} $\sigma$.

\begin{example}
    Soit $v\in\eR^3$ et $x_0\in\eR^3$. Le chemin
    \begin{equation}
        \sigma(t)=x_0+tv
    \end{equation}
    est une droite. Sa vitesse est $\sigma'(t)=v$.    
\end{example}

\begin{example}
    La courbe
    \begin{equation}
        \sigma(t)=\begin{pmatrix}
            \cos(t)    \\ 
            \sin(t)    
        \end{pmatrix}\in\eR^2
    \end{equation}
    avec $t\in\mathopen[ 0 , 2\pi [$ est le cercle unité parcouru une fois dans le sens trigonométrique.

    Notez que si on prend $t\in\mathopen[ 0 , 4\pi [$, nous avons un \emph{autre} chemin; c'est le même cercle unité, mais parcouru \emph{deux} fois. Même si le «dessin» (le graphe) des deux est le même, le chemin n'est pas le même.

    Le chemin
    \begin{equation}
        \gamma(t)=\begin{pmatrix}
            \cos(2\pi-t)    \\ 
            \sin(2\pi-t)    
        \end{pmatrix}
    \end{equation}
    est le cercle unité parcouru une fois dans le sens inverse. Encore une fois le «dessin» est le même, mais le chemin n'est pas le même.
\end{example}

\begin{example}
    Le chemin
    \begin{equation}
        \sigma(t)=\begin{pmatrix}
            t    \\ 
            t^2    
        \end{pmatrix}
    \end{equation}
    est un chemin dont l'image est la parabole d'équation $y=x^2$.
\end{example}

L'importance de la dérivée du chemin réside en le fait qu'elle donne la tangente. En effet le vecteur $\sigma'(t)$ est tangent au graphe de $\sigma$ au point $\sigma(t)$.

\begin{example}
    Pour le cercle,
    \begin{equation}
        \sigma(t)=\begin{pmatrix}
            \cos(t)    \\ 
            \sin(t)    
        \end{pmatrix},
    \end{equation}
    la dérivée est donnée par
    \begin{equation}
        \sigma'(t)=\begin{pmatrix}
            -\sin(t)    \\ 
            \cos(t).    
        \end{pmatrix}
    \end{equation}
    Le produit scalaire $\sigma(t)\cdot \sigma'(t)$ est nul. Le vecteur $\sigma'(t)$ est donc bien tangent (voir exercice \ref{exoDerive-0003}).
\end{example}

\begin{example}
    Le courbe donnée par le chemin
    \begin{equation}
        \sigma(t)=\begin{pmatrix}
            \cos(t)    \\ 
            \sin(t)    \\ 
            t    
        \end{pmatrix}
    \end{equation}
    est une hélice. Sa vitesse est
    \begin{equation}
        \sigma'(t)=\begin{pmatrix}
            -\sin(t)    \\ 
            \cos(t)    \\ 
            1    
        \end{pmatrix}.
    \end{equation}
    Notez que pour tout $t\in\eR$, nous avons $\| \sigma'(t) \|=\sqrt{2}$.
\end{example}

\begin{remark}
    Lorsqu'on parle d'une courbe dans l'espace, l'intervalle sur lequel on considère la variation du paramètre est une donné fondamentale. Elle fait partie intégrante de la définition de la courbe.
\end{remark}

%---------------------------------------------------------------------------------------------------------------------------
\subsection{Longueur d'une courbe paramétrée}
%---------------------------------------------------------------------------------------------------------------------------

Nous pouvons voir un chemin $\sigma$ comme étant la trajectoire d'une particule en fonction du temps. Sa vitesse à l'instant $t$ est le vecteur $\sigma'(t)$, tandis que sa vitesse \emph{scalaire} est le nombre $\| \sigma'(t) \|$. Une question naturelle est de savoir quelle est la longueur de la trajectoire parcourue entre $t=a$ et $t=b$.

Si nous prenons un petit intervalle de temps $dt$, nous pouvons supposer que le mobile avance à la vitesse constante $\| \sigma'(t) \|$. Cela ferait un trajet parcouru de longueur $\| \sigma'(t) \|dt$. Nous prenons donc la définition suivante pour la longueur.

\begin{definition}
    Soit $\sigma\colon \mathopen[ a , b \mathclose]\to \eR^3$ un chemin. La \defe{longueur}{longueur!d'un chemin} du chemin $\sigma$ est le nombre
    \begin{equation}        \label{EqDefLongueurChemin}
        l(\sigma)=\int_a^b\| \sigma'(t) \|dt.
    \end{equation}
    Plus explicitement, si $\sigma(t)=\big( x(t),y(t),z(t) \big)$, alors nous avons la formule
    \begin{equation}
        l(\sigma)=\int_a^b\sqrt{x'(t)^2+y'(t)^2+z'(t)^2}dt.
    \end{equation}
\end{definition}

\begin{example}
    Considérons l'arc de cercle de rayon $R$ interceptée par l'angle $\theta$ présenté sur la figure \ref{LabelFigArcCercleAngle}.
    \newcommand{\CaptionFigArcCercleAngle}{Quelle est la longueur de la partie bleue de ce cercle de rayon $R$ ?}
    \input{Fig_ArcCercleAngle.pstricks}

    Par définition, cette longueur sera
    \begin{equation}
        \int_{\theta_0}^{\theta_1}\sqrt{R^2\sin^2(t)+R^2\cos^2(t)}dt=R(\theta_1-\theta_0).
    \end{equation}
    Le radian comme unité de mesure d'angle est donc l'unité parfaite : elle est la longueur d'arc interceptée (si le rayon est $R=1$).

\end{example}

\begin{example}
    La longueur de l'hélice
    \begin{equation}
        \sigma(t)=\begin{pmatrix}
            \cos(2t)    \\ 
            \sin(2t)    \\ 
            \sqrt{5}t    
        \end{pmatrix}
    \end{equation}
    pour $t\in\mathopen[ 0 , 2\pi \mathclose]$ est donnée par
    \begin{equation}
        l(\sigma)=\int_0^{4\pi}\sqrt{4\sin^2(2t)+4\cos^2(2t)+5}dt=\int_0^{4\pi}\sqrt{9}=12\pi.
    \end{equation}
\end{example}

\begin{definition}
    Soit $\sigma_1\colon \mathopen[ a , b \mathclose]\to \eR^3$, un chemin et $\sigma_2\colon \mathopen[ c , d \mathclose]\to \eR^3$, un autre chemin. On dit que ces chemins sont \defe{équivalents}{équivalents!chemin} si il existe une fonction $\varphi\colon \mathopen[ a , b \mathclose]\to \mathopen[ c , d \mathclose]$ strictement croissante telle que $\sigma_1(t)=\sigma_2\big( \varphi(t) \big)$.
\end{definition}

Deux chemins équivalents parcourent la même courbe dans le même sens. Ils ne le parcourent toutefois pas à la même vitesse. On dit que les chemins sont \defe{opposée}{opposés!chemins} si la fonction $\varphi$ de la définition est strictement décroissante. Dans ce cas, ils ont la même image, mais parcourue dans le sens opposés. Nous disons que deux chemins équivalents sont un \defe{changement de paramétrisation}{paramétrisation} pour la même courbe.

 Dans le cas d'une paramétrisation équivalente, nous avons $\varphi(a)=c$ et $\varphi(b)=d$. Les points de départ et d'arrivée des deux paramètres coïncident. Dans le cas d'un paramètre qui va dans le sens opposé par contre nous avons automatiquement $\varphi(a)=d$ et $\varphi(b)=c$.

\begin{proposition}
    La longueur d'une courbe ne dépend pas du paramètre (équivalent ou opposé) choisi.
\end{proposition}

\begin{proof}
    Soient $\sigma_1\colon \mathopen[ a , b \mathclose]\to \eR^3$ et $\sigma_2\colon \mathopen[ c , d \mathclose]\to \eR^3$ tels que
    \begin{equation}     \label{EqChmsigmaundeuxvp}
        \sigma_1(t)=\sigma_2\big( \varphi(t) \big)
    \end{equation}
    où $\varphi\colon \mathopen[ a , b \mathclose]\to \mathopen[ a , d \mathclose]$ est une bijection strictement monotone. Par définition on a
    \begin{equation}
        l(\sigma_1)=\int_a^b\| \sigma_1'(t) \|dt.
    \end{equation}
    Nous pouvons exprimer la dérivée de $\sigma_1$ en termes de celle de $\sigma_2$ en dérivant la relation \eqref{EqChmsigmaundeuxvp} :
    \begin{equation}
        \sigma_1'(t)=\varphi'(t)\sigma_2'\big( \varphi(t) \big).
    \end{equation}
    En ce qui concerne la norme,
    \begin{equation}
        \| \sigma_1'(t) \|=| \varphi'(t) |\| \sigma_2'(t) \|.
    \end{equation}
    Notez dans cette relation que $\varphi'(t)$ est un nombre (et non un vecteur). Étant donné que nous avons supposé que $\varphi$ était monotone, soit elle est monotone croissante et $\| \varphi'(t) \|=\varphi'(t)$ pour tout $t$, soit elle est monotone décroissante et $\| \varphi'(t) \|='\varphi(t)$ pour tout $t$.

    Considérons d'abord le premier cas, c'est à dire $\| \varphi'(t) \|=\varphi'(t)$. Nous posons $s=\varphi(t)$, $ds=\varphi'(t)dt$. En remplaçant cela dans la formule de la longueur est
    \begin{equation}
        \begin{aligned}[]
            l(\sigma_1)&=\int_a^b\varphi'(t)\| \sigma_2\big( \varphi(t) \big) \|dt\\
            &=\int_{\varphi(a)}^{\varphi(b)}\| \sigma_2'(s) \|ds\\
            &=\int_c^d\| \sigma_2'(s) \|ds\\
            &=l(\sigma_2).
        \end{aligned}
    \end{equation}
    
    Si nous considérons maintenant une paramétrisation strictement décroissante. Dans ce cas, $\varphi'(t)\leq 0$ et $\| \varphi'(t) \|=-\varphi'(t)$. Nous posons encore une fois $s=\varphi(t)$, $ds=\varphi'(t)ds$. Ici il ne faut pas oublier que $\varphi(a)=d$ et $\varphi(b)=c$. Le calcul est à part cela le même en faisant attention au singe :
    \begin{equation}
        \begin{aligned}[]
            l(\sigma_1)&=\int_a^b\varphi'(t)\| \sigma_2\big( \varphi(t) \big) \|dt\\
            &=-\int_{\varphi(a)}^{\varphi(b)}\| \sigma_2'(s) \|ds\\
            &=-\int_d^c\| \sigma_2'(s) \|ds\\
            &=\int_c^d\| \sigma_2'(s) \|ds\\
            &=l(\sigma_2).
        \end{aligned}
    \end{equation}
    Nous avons changé le signe en changeant l'ordre des bornes.
\end{proof}

%+++++++++++++++++++++++++++++++++++++++++++++++++++++++++++++++++++++++++++++++++++++++++++++++++++++++++++++++++++++++++++
\section{Intégrales le long de chemins}
%+++++++++++++++++++++++++++++++++++++++++++++++++++++++++++++++++++++++++++++++++++++++++++++++++++++++++++++++++++++++++++

%---------------------------------------------------------------------------------------------------------------------------
\subsection{Circulation d'un champ de vecteur}
%---------------------------------------------------------------------------------------------------------------------------

\begin{definition}
    Soit $F\colon \eR^3\to \eR^3$ un champ de vecteurs et un chemin $\sigma\colon \mathopen[ a , b \mathclose]\to \eR^3$. On appelle \defe{circulation}{circulation} de $F$ le long du chemin $\sigma$ le scalaire
    \begin{equation}
        \int_a^b F\big( \sigma(t) \big)\sigma'(t)dt.
    \end{equation}
    Il existe de nombreuses notations pour cela; entre autres :
    \begin{equation}
        \int_{\sigma}F=\int_{\sigma} F\cdot ds.
    \end{equation}
\end{definition}

En physique, la circulation de la force le long d'un chemin est la travail de la force.

\begin{example}
    À la surface de la Terre, le champ de gravitation est donné par
    \begin{equation}
        G(x,y,z)=-mg\begin{pmatrix}
            0    \\ 
            0    \\ 
            1    
        \end{pmatrix}.
    \end{equation}
    Si nous considérons un mobile qui monte à vitesse constante jusqu'à la hauteur $h$, c'est à dire le chemin
    \begin{equation}
        \sigma(t)=\begin{pmatrix}
            0    \\ 
            0    \\ 
            t    
        \end{pmatrix}
    \end{equation}
    avec $t\in\mathopen[ 0 , h \mathclose]$. Le travail de la gravitation est alors donné par
    \begin{equation}
        W=\int_0^hG\big( \sigma(t) \big)\cdot\begin{pmatrix}
            0    \\ 
            0    \\ 
            1    
        \end{pmatrix}=
        -mg\int_0^h\begin{pmatrix}
            0    \\ 
            0    \\ 
            1    
        \end{pmatrix}\cdot\begin{pmatrix}
            0    \\ 
            0    \\ 
            1    
        \end{pmatrix}=-mgh.
    \end{equation}
    Cela est bien le résultat usuel de l'énergie potentielle. Nous allons voir bientôt que nous nommons la fonction $mgh$ énergie \emph{potentielle} précisément parce que la force dérive de ce potentiel.
\end{example}

\begin{example}
    Soit le chemin
    \begin{equation}
        \begin{aligned}
            \sigma\colon \mathopen[ 0 , 2\pi \mathclose]&\to \eR^3 \\
            t&\mapsto \begin{pmatrix}
                \sin(t)    \\ 
                \cos(t)    \\ 
                t    
            \end{pmatrix}.
        \end{aligned}
    \end{equation}
    et le champ de vecteurs
    \begin{equation}
        F\begin{pmatrix}
            x    \\ 
            y    \\ 
            z    
        \end{pmatrix}=\begin{pmatrix}
            x    \\ 
            y    \\ 
            z    
        \end{pmatrix}.
    \end{equation}
    La circulation de ce champ de vecteur le long de l'hélice $\sigma$ est
    \begin{equation}
        \begin{aligned}[]
            \int_{\sigma}F\cdot ds&=\int_0^{2\pi}(F\circ \sigma)(t)\cdot \sigma'(t)dt\\
            &=\int_0^{2\pi}\begin{pmatrix}
                \sin(t)    \\ 
                \cos(t)    \\ 
                t    
            \end{pmatrix}\cdot
            \begin{pmatrix}
                \cos(t)    \\ 
                \sin(t)    \\ 
                1    
            \end{pmatrix}dt\\
            &=\int_0^{2\pi}tdt\\
            &=\left[ \frac{ t^2 }{2} \right]_0^{2\pi}\\
            &=2\pi^2.
        \end{aligned}
    \end{equation}
    
\end{example}

\begin{proposition}
    La circulation d'un champ de vecteurs le long d'un chemin ne dépend pas de la paramétrisation. En d'autres termes, si $\sigma_1$ et $\sigma_2$ sont deux chemins équivalents, alors
    \begin{equation}
        \int_{\sigma_1}F=\int_{\sigma_2}F.
    \end{equation}
\end{proposition}

\begin{proof}
    Soient deux chemins $\sigma_1\colon \mathopen[ a , b \mathclose]\to \eR^3$ et $\sigma_2\colon \mathopen[ c , d \mathclose]\to \eR^3$ équivalents, c'est à dire tels que
    \begin{equation}
        \sigma_1(t)=\sigma_2\big( \varphi(t) \big)
    \end{equation}
    où $\varphi\colon \mathopen[ a , b \mathclose]\to \mathopen[ c , d \mathclose]$ strictement croissante. En utilisant le fait que $\sigma_1(t)=\varphi'(t)\sigma_2'\big( \varphi(t) \big)$, nous avons
    \begin{equation}
        \begin{aligned}[]
            \int_{\sigma_1}F\cdot ds&=\int_a^bF\big( \sigma_1(t) \big)\cdot\sigma_1'(t)dt\\
            &=\int_a^bF\Big( \sigma_2\big( \varphi(t) \big) \Big)\cdot\sigma_2'\big( \varphi(t) \big)\varphi'(t)dt\\
            &=\int_{\varphi(a)}^{\varphi(b)}F\big( \sigma_2(s) \big)\cdot\sigma_2(s)ds\\
            &=\int_c^dF\big( \sigma_2(s) \big)\cdot \sigma_2'(s)ds\\
            &=\int_{\sigma_2}F\cdot ds.
        \end{aligned}
    \end{equation}
    où nous avons effectué le changement de variables $s=\varphi(t)$, $ds=\varphi'(t)dt$.
\end{proof}

\begin{remark}
    Si $\sigma_2$ est le chemin opposé de $\sigma$, alors
    \begin{equation}
        \int_{\sigma_2}F=-\int_{\sigma_1}F.
    \end{equation}
\end{remark}

%+++++++++++++++++++++++++++++++++++++++++++++++++++++++++++++++++++++++++++++++++++++++++++++++++++++++++++++++++++++++++++
\section{Circulation d'un champ conservatif}
%+++++++++++++++++++++++++++++++++++++++++++++++++++++++++++++++++++++++++++++++++++++++++++++++++++++++++++++++++++++++++++

Si nous avons une fonction scalaire $V\colon \eR^3\to \eR$, nous pouvons construire un champ de vecteur en prenant le gradient :
\begin{equation}
    F(x)=\nabla V(x).
\end{equation}
On dit que le champ de vecteur $F$ \defe{dérive}{champ dérivant d'un potentiel} de $V$, et on dit que $V$ est le \defe{potentiel}{potentiel} de $F$. Nous posons la définition suivante :
\begin{definition}
    Un champ de vecteurs $F\colon \eR^3\to \eR^3$ est un champ \defe{conservatif}{champ!conservatif} si il existe une fonction $V\colon \eR^3\to \eR$ telle que
    \begin{equation}
        F(x)=\nabla V(x).
    \end{equation}
    Nous disons aussi parfois que le champ $V$ \emph{dérive d'un potentiel} ou bien qu'il s'agit d'un \emph{champ de gradient}.
\end{definition}

Les champs de vecteurs conservatifs sont particulièrement importants parce que presque toutes les forces connues en physiques dérivent d'un potentiel. Nous verrons que la terminologie «conservatif» provient du fait que les forces de ce type conservent l'énergie associée.


\begin{proposition}
    Considérons une fonction $V\colon \eR^3\to \eR$ (que nous appellerons \emph{potentiel}) et le champ de vecteur qui en dérive :
    \begin{equation}
        F=\nabla V.
    \end{equation}
    Alors 
    \begin{equation}
        \int_{\sigma}F\cdot ds=V\big( \sigma(b) \big)-V\big( \sigma(a) \big).
    \end{equation}
    Autrement dit, le travail nécessaires pour déplacer un objet d'un point à un autre dans un champ de force conservatif vaut la différence de potentiel entre le point de départ et le point d'arrivée.
\end{proposition}

\begin{proof}
    Par définition,
    \begin{equation}        \label{Eqintparddeftrav}
        \int_{\sigma} F\cdot ds=\int_a^b F\big( \sigma(t) \big)\cdot \sigma'(t)dt.
    \end{equation}
    Nous pouvons transformer l'intégrante de la façon suivante :
    \begin{equation}
        \begin{aligned}[]
            F\big( \sigma(t) \big)\cdot\sigma'(t)&=\nabla V\big( \sigma(t) \big)\cdot\sigma'(t)\\
            &=\frac{ \partial V }{ \partial x }\big( \sigma(t) \big)\sigma_x'(t) +\frac{ \partial V }{ \partial y }\big( \sigma(t) \big)\sigma_y'(t) +\frac{ \partial V }{ \partial z }\big( \sigma(t) \big)\sigma_z'(t)\\
            &=\frac{ d }{ dt }\Big[ V\big( \sigma(t) \big) \Big]
        \end{aligned}
    \end{equation}
    où nous avons posé
    \begin{equation}
        \sigma(t)=\begin{pmatrix}
            \sigma_x(t)    \\ 
            \sigma_y(t)    \\ 
            \sigma_z(t)    
        \end{pmatrix}
    \end{equation}
    et utilisé à l'envers la formule de dérivation de fonction composée pour
    \begin{equation}
             \frac{ d }{ dt }\Big[ V\big( \sigma(t) \big) \Big]=\Big( (V\circ\sigma)(t) \Big)'.
    \end{equation}
    En remettant ces expressions dans l'intégrale \eqref{Eqintparddeftrav},
    \begin{equation}
        \int_{\sigma}F\cdot ds=\int_a^b\frac{ d }{ dt }\Big[ V\big( \sigma(t) \big) \Big]dt=V\big( \sigma(b) \big)-V\big( \sigma(a) \big).
    \end{equation}
\end{proof}

\begin{example}
    Nous savons que le champ de gravitation dérive d'un potentiel. À la surface de la Terre, le potentiel de gravitation vu par une masse $m$ est donné par la fonction $V(x,y,z)=mgz$. Si nous voulons soulever cette masse d'une hauteur $h$, cela demandera toujours une énergie $mgh$, quel que soit le chemin suivit : en ligne droite vertical, en diagonal, en hélice, \ldots
\end{example}

\begin{example}
    À plus grande échelle, le champ de gravitation est encore un champ qui dérive d'un potentiel. En coordonnées sphériques,
    \begin{equation}
        V(\rho,\theta,\varphi)=k\frac{ m }{ \rho }
    \end{equation}
    Lorsqu'un satellite a une orbite de rayon $R$ autour la Terre, il reste sur la sphère $\rho=R$. Donc il reste sur une surface sur laquelle $V$ est constante. Il n'y a donc pas de travail de la force de gravitation ! C'est pour cela qu'un satellite peut tourner pendant des siècles sans apport énergétique.
\end{example}

\begin{example}
    Soit le champ de vecteurs
    \begin{equation}
        F\begin{pmatrix}
            x    \\ 
            y    
        \end{pmatrix}=\begin{pmatrix}
            y    \\ 
            x    
        \end{pmatrix}
    \end{equation}
    et le chemin
    \begin{equation}
        \sigma(t)=\begin{pmatrix}
            t^4/4    \\ 
            \sin^3(t\frac{ \pi }{2}).    
        \end{pmatrix}
    \end{equation}
    Nous voulons calculer la circulation de $F$ le long du chemin $\sigma$ entre $t=0$ et $t=1$.

    La première chose à voir est que $F=\nabla V$ avec $V(x,y)=xy$. Donc la circulation sera donnée par
    \begin{equation}
        \int_{\sigma}F\cdot ds=V\big( \sigma(1) \big)-V\big( \sigma(0) \big)=V\big( \frac{1}{ 4 },1 \big)-V(0,0)=\frac{1}{ 4 }-0=\frac{1}{ 4 }.
    \end{equation}
    Nous n'avons pas réellement calculé l'intégrale.
\end{example}

%+++++++++++++++++++++++++++++++++++++++++++++++++++++++++++++++++++++++++++++++++++++++++++++++++++++++++++++++++++++++++++
\section{Divergence, rotationnel et l'opérateur nabla}
%+++++++++++++++++++++++++++++++++++++++++++++++++++++++++++++++++++++++++++++++++++++++++++++++++++++++++++++++++++++++++++

Nous avons déjà vu le gradient d'une fonction $f\colon \eR^3\to \eR$
\begin{equation}        \label{EqDefNablaf}
    \nabla f(x,y,z)=\begin{pmatrix}
        \partial_xf(x,y,z)    \\ 
        \partial_yf(x,y,z)    \\ 
        \partial_zf(x,y,z)    
    \end{pmatrix}
\end{equation}
Afin de définir la divergence et le rotationnel, nous introduisons $\nabla$ sous une forme un peu plus abstraite comme le «vecteur»
\begin{equation}
    \nabla=\begin{pmatrix}
        \partial_x    \\ 
        \partial_y    \\ 
        \partial_z
    \end{pmatrix}.
\end{equation}
Vue comme ça, la formule \eqref{EqDefNablaf} est claire.

Si $F$ est un champ de vecteurs, nous introduisons la \defe{divergence}{divergence} de $F$ par
\begin{equation}
    \nabla\cdot F=\frac{ \partial F_x }{ \partial x }+\frac{ \partial F_y }{ \partial y }+\frac{ \partial F_z }{ \partial z }.
\end{equation}
Cela est une fonction. Et nous introduisons le rotationnel du champ de vecteur $F$ par
\begin{equation}
    \begin{aligned}[]
        \nabla\times F&=\begin{vmatrix}
              1_x  &   1_y    &   1_z    \\
            \partial_x    &   \partial_y    &   \partial_z    \\
            F_x    &   F_y    &   F_z
        \end{vmatrix}\\
        &=
        \left( \frac{ \partial F_z }{ \partial y }-\frac{ \partial F_y }{ \partial z } \right)1_x
        -\left( \frac{ \partial F_z }{ \partial x }-\frac{ \partial F_x }{ \partial z } \right)1_y
        +\left( \frac{ \partial F_y }{ \partial x }-\frac{ \partial F_x }{ \partial y } \right)1_z.
    \end{aligned}
\end{equation}
Cela est un champ de vecteur. Conformément à la formule \eqref{EqProdVectEspilonijk}, le rotationnel d'un champ de vecteur peut s'écrire
\begin{equation}
    \nabla\times F=\sum_{ijk}\partial_i F_j 1_k.
\end{equation}

Le gradient, la divergence et le rotationnel consistent à appliquer simplement à $\nabla$ est trois produits qu'on peut effectuer sur un vecteur:
\begin{enumerate}
    \item
        Le produit d'un vecteur par un scalaire multiplie chacune des composantes :
        \begin{equation}
            \begin{pmatrix}
                \partial_x    \\ 
                \partial_y    \\ 
                \partial_z    
            \end{pmatrix}f
            =\begin{pmatrix}
                \partial_xf    \\ 
                \partial_yf    \\ 
                \partial_zf    
            \end{pmatrix}.
        \end{equation}
    \item
        Le produit scalaire d'un vecteur avec un autre vecteur donne lieu à la divergence :
        \begin{equation}
            \begin{pmatrix}
                \partial_x    \\ 
                \partial_y    \\ 
                \partial_z    
            \end{pmatrix}\cdot
            \begin{pmatrix}
                F_x    \\ 
                F_y    \\ 
                F_z    
            \end{pmatrix}=
            \frac{ \partial F_x }{ \partial x }+\frac{ \partial F_y }{ \partial y }+\frac{ \partial F_z }{ \partial z }.
        \end{equation}
    \item
        Le produit vectoriel de deux vecteurs :
        \begin{equation}
            \begin{pmatrix}
                \partial_x    \\ 
                \partial_y    \\ 
                \partial_z    
            \end{pmatrix}\times\begin{pmatrix}
                F_x    \\ 
                F_y    \\ 
                F_z    
            \end{pmatrix}=
            \begin{vmatrix}
                1_x    &   1_y    &   1_z    \\
                \partial_x    &   \partial_y    &   \partial_z    \\
                F_x    &   F_y    &   F_z
            \end{vmatrix}.
        \end{equation}
\end{enumerate}
Ces trois opérations joueront un rôle central en électromagnétisme dans les équations de Maxwell.

\begin{example}
    Soit $F(x,y,z)=x 1_x+xy 1_y+1_z$, c'est à dire
    \begin{equation}
        F(x,y,z)=\begin{pmatrix}
            x    \\ 
            xy    \\ 
            1    
        \end{pmatrix}.
    \end{equation}
    Son rotationnel est donné par
    \begin{equation}
        \nabla\times F=\begin{vmatrix}
            1_x    &   1_y    &   1_z    \\
            \frac{ \partial  }{ \partial x }    &   \frac{ \partial  }{ \partial y }    &   \frac{ \partial  }{ \partial y }    \\
            x    &   xy    &   1
        \end{vmatrix}=
        (0-0)1_x-(0-0)1_y+(y-0)1_z=y1_z=\begin{pmatrix}
            0    \\ 
            0    \\ 
            y    
        \end{pmatrix}.
    \end{equation}
\end{example}

Afin d'étudier comment se comporte la composition de ces opérateurs, nous aurons besoin de ce lemme que nous n'énoncerons pas précisément.
\begin{lemma}       \label{LemPermDerrxyz}
    Si $f\colon \eR^3\to \eR$ est une fonction de classe $C^2$, alors on peut permuter l'ordre des dérivées:
    \begin{equation}
        \begin{aligned}[]
            \frac{ \partial  }{ \partial x }\left( \frac{ \partial f }{ \partial y } \right)&=\frac{ \partial  }{ \partial y }\left( \frac{ \partial f }{ \partial x } \right)\\
            \frac{ \partial  }{ \partial x }\left( \frac{ \partial f }{ \partial z } \right)&=\frac{ \partial  }{ \partial z }\left( \frac{ \partial f }{ \partial x } \right)\\
            \frac{ \partial  }{ \partial z }\left( \frac{ \partial f }{ \partial y } \right)&=\frac{ \partial  }{ \partial y }\left( \frac{ \partial f }{ \partial z } \right)
        \end{aligned}
    \end{equation}
\end{lemma}
La fonction
\begin{equation}
    (x,y,z)\mapsto\frac{ \partial  }{ \partial x }\left( \frac{ \partial f }{ \partial y } \right)(x,y,z)
\end{equation}
sera notée
\begin{equation}
    \frac{ \partial^2f }{ \partial x\partial y }.
\end{equation}

Il y a deux propriétés importantes :
\begin{theorem}
    Soit $f\colon \eR^3\to \eR$ une fonction de classe $C^2$. Alors
    \begin{equation}
        \nabla\times(\nabla f)=0.
    \end{equation}
    Si $F\colon \eR^3\to \eR^3$ est un champ de vecteurs de classe $C^2$, alors
    \begin{equation}
        \nabla\cdot(\nabla\times F)=0.
    \end{equation}
\end{theorem}

\begin{proof}
    Ce sont seulement deux calculs qui manipulent les définitions. Pour le premier, la divergence de $f$ est le champ de vecteurs
    \begin{equation}
        \nabla f=\frac{ \partial f }{ \partial x }1_x+\frac{ \partial f }{ \partial y }1_y+\frac{ \partial f }{ \partial z }1_z.
    \end{equation}
    En mettant ce champ dans la définition du rotationnel,
    \begin{equation}
        \begin{aligned}[]
            \nabla\times(\nabla f)=\begin{vmatrix}
                 1_x   &   1_y    &   1_z    \\
                 \frac{ \partial  }{ \partial x }    &   \frac{ \partial  }{ \partial y }    &   \frac{ \partial  }{ \partial z }    \\
                 \frac{ \partial f }{ \partial x }    &   \frac{ \partial f }{ \partial y }    &   \frac{ \partial f }{ \partial z }
            \end{vmatrix}
            &=\left[ \frac{ \partial  }{ \partial y }\left( \frac{ \partial f }{ \partial z } \right)-\frac{ \partial  }{ \partial z }\left( \frac{ \partial f }{ \partial y } \right) \right]1_x\\
            &\quad-\left[ \frac{ \partial  }{ \partial x }\left( \frac{ \partial f }{ \partial z } \right)-\frac{ \partial  }{ \partial z }\left( \frac{ \partial f }{ \partial x } \right) \right]1_y\\
            &\quad+\left[ \frac{ \partial  }{ \partial x }\left( \frac{ \partial f }{ \partial y } \right)-\frac{ \partial  }{ \partial y }\left( \frac{ \partial f }{ \partial x } \right) \right]1_z.
        \end{aligned}
    \end{equation}
    En utilisant le lemme \ref{LemPermDerrxyz}, chacun des termes fait zéro.

    La seconde propriété se démontre en utilisant le même type de calcul.
\end{proof}

\begin{remark}
    Il n'y a pas de propriétés du même style pour la combinaison $\nabla\times(\nabla\cdot F)$ pour le rotationnel de la divergence. En effet la divergence d'un champ de vecteur est une fonction, et il n'y a pas de rotationnel pour une fonction.
\end{remark}

%+++++++++++++++++++++++++++++++++++++++++++++++++++++++++++++++++++++++++++++++++++++++++++++++++++++++++++++++++++++++++++
\section[Interprétation de la divergence]{Interprétation géométrique et physique de la divergence}
%+++++++++++++++++++++++++++++++++++++++++++++++++++++++++++++++++++++++++++++++++++++++++++++++++++++++++++++++++++++++++++


En physique, on dit qu'un champ de vecteurs à divergence nulle est \defe{incompressible}{incompressible!champ de vecteur}. Nous allons essayer de comprendre pourquoi. Lorsqu'un fluide incompressible se déplace, il faut qu'en chaque point il y autant de fluide qui rentre que de fluide qui sort. Nous allons voir sur quelque exemples que la divergence d'un champ de vecteurs est le «bilan de masse» d'un fluide qui se déplace selon le champ de vecteurs.

Si en un point la divergence est positive, cela signifie qu'il y a une perte de masse et si la divergence est négative, cela signifie qu'il y a une accumulation de masse.

Prenons par exemple un fluide qui se déplace selon le champ de vitesse montré à figure \ref{LabelFigDivergenceUn}.
\newcommand{\CaptionFigDivergenceUn}{Le champ de vecteurs $F(x,y)=\frac{1}{ x }(1,0)$.}
\input{Fig_DivergenceUn.pstricks}

Étant donné que la vitesse diminue lorsque $x$ avance, il y a une accumulation de fluide. Regardez en effet la quantité de fluide qui rentre dans le rectangle par rapport à la quantité de fluide qui en sort. Ce champ de vecteurs a pour équation :
\begin{equation}
    F(x,y)=\frac{1}{ x }\begin{pmatrix}
        1    \\ 
        0    
    \end{pmatrix}=\begin{pmatrix}
        1/x    \\ 
        0    
    \end{pmatrix}.
\end{equation}
Sa divergence vaut donc
\begin{equation}
    (\nabla\cdot F)(x,y)=\frac{ \partial F_x }{ \partial x }(x,y)+\underbrace{\frac{ \partial F_y }{ \partial y }(x,y)}_{=0}=-\frac{1}{ x^2 }.
\end{equation}
Cette divergence étant négative, il y a bien accumulation de fluide en tout point, et d'autant plus que $x$ est petit.

\begin{example}     \label{ExamDivFrot}

    Prenons le champ de vecteurs tournant
    \begin{equation}
        F(x,y)=\frac{1}{ \sqrt{x^2+y^2} }\begin{pmatrix}
            y    \\ 
            -x    
        \end{pmatrix}
    \end{equation}
    représenté à la figure \ref{LabelFigDivergenceDeux}. Cela est un vecteur qui est constamment perpendiculaire au rayon.

    \newcommand{\CaptionFigDivergenceDeux}{Le champ de vecteurs $F(x,y)=(y,-x)$.}
    \input{Fig_DivergenceDeux.pstricks}

    Un fluide dont la vitesse serait donné par ce champ de vecteur se contente de tourner. Intuitivement il ne devrait pas y avoir de divergence parce qu'il n'y a aucune accumulation de fluide. En effet,
    \begin{equation}
        \nabla\cdot F(x,y)=\frac{ -2xy }{ (x^2+y^2)^2 }+\frac{ 2xy }{ (x^2+y^2)^2 }=0.
    \end{equation}
\end{example}

\begin{example}
    Prenons le cas du champ de force de gravitation :
    \begin{equation}
        F(x,y,z)=\frac{1}{ (x^2+y^2+z^2)^{3/2} }\begin{pmatrix}
            x    \\ 
            y   \\
            z
        \end{pmatrix}.
    \end{equation}
    Nous pouvons rapidement remarquer que $\nabla\cdot F=0$. Est-ce que cela peut se comprendre sur le dessin de la figure \ref{LabelFigDivergenceTrois} ?
    \newcommand{\CaptionFigDivergenceTrois}{Le champ de vecteur de la gravité. Nous avons tracé, sur les deux cercles la même densité de vecteurs, c'est à dire le même nombre de vecteurs par unité de surface.}
    \input{Fig_DivergenceTrois.pstricks}

    Essayons de voir combien de fluide entre dans la zone bleue et combien en sort. D'abord, il est certain que les vecteurs qui sortent sont plus courts que ceux qui rentrent, ce qui voudrait dire qu'il y a plus de fluide qui rentre. Mais on voit également que le \emph{nombre} de vecteurs qui sortent est plus grand parce que la seconde sphère est plus grande et qu'il y a un vecteur en chaque point de la sphère.

    Intuitivement nous pouvons dire que la quantité qui rentre dans la sphère de rayon $r_1$ donnée par la taille des vecteurs entrants multiplié par la surface de la sphère, c'est à dire
    \begin{equation}        \label{EqQpinormeVecto}
        4\pi r_1^2\| F(x,y,z) \|,
    \end{equation}
    mais $\| F(x,y,z) \|=\frac{1}{ r_1^2 }$, donc la quantité de fluide entrant est $4\pi$. La quantité de fluide sortant sera la même.

    Cela explique deux choses
    \begin{enumerate}
        \item
            Pourquoi les forces de gravitation et électromagnétiques sont en $1/r^2$; c'est parce que nous vivons dans un monde avec trois dimensions d'espace.
        \item
            Pourquoi il y a un $4\pi$ comme coefficient dans beaucoup d'équations en électromagnétisme; en particulier dans certaines anciennes unités de flux.
    \end{enumerate}
    
\end{example}

\begin{remark}
    Nous allons voir plus loin comment s'assurer que l'équation \eqref{EqQpinormeVecto} représente bien la «quantité de fluide» qui rentre dans la zone délimitée
\end{remark}


%+++++++++++++++++++++++++++++++++++++++++++++++++++++++++++++++++++++++++++++++++++++++++++++++++++++++++++++++++++++++++++
\section{Quelque formules de Leibnitz}
%+++++++++++++++++++++++++++++++++++++++++++++++++++++++++++++++++++++++++++++++++++++++++++++++++++++++++++++++++++++++++++

La divergence étant une combinaison de dérivées, il n'est pas tellement étonnant que la divergence de produits donne lieux à des formules en deux termes. Si $f$ est une fonction et si $F$ et $G$ sont des champs de vecteurs, nous avons (sans démonstrations) :
\begin{equation}        \label{EqLeinDivNablRot}
    \begin{aligned}[]
        \nabla\cdot(fF)&=f\nabla\cdot F+F\cdot\nabla f\\
        \nabla\cdot(F\times G)&=G\cdot\nabla\times F-F\cdot\nabla\times G.
    \end{aligned}
\end{equation}
Nous avons aussi, pour le rotationnel,
\begin{equation}        \label{EqLeinRotfFF}
    \nabla\times(fF)=f\nabla\times F+\nabla f\times F.
\end{equation}
