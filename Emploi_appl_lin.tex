Ce chapitre traitera principalement des applications linéaires de $\eR^m$ dans $\eR^n$, et nous allons voir que ces applications sont celles qui peuvent être écrites sous forme de matrices.

Un des concepts clefs et abstraits de ce chapitre sera la définition d'une norme pour les application linéaires elles-mêmes (définition \ref{DefNormeAppLineaire}). Nous pourrons donc parler de $\| T \|$ si $T$ est une application. Cela nous servira à parler de l'espace des applications linéaires comme un «vulgaire» espace vectoriel normé. Toute la théorie développée durant le chapitre \ref{ChapEspVectNorm} sera donc immédiatement disponible.

Nous n'allons cependant pas développer la théorie des limites de suites d'applications linéaires. Nous nous bornerons à donner quelque exemples d'applications linéaires qui ne sont pas des applications entre $\eR^m$ et $\eR^m$. Les polynômes seront notre exemple le plus détaillé (section \ref{SecEspacePolynomes}).

Pour ce chapitre, vous devez vous concentrer sur les définitions et les exemples. Les démonstrations ne font pas partie de la matière.

