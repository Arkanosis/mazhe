% This is part of the Exercices et corrigés de mathématique générale.
% Copyright (C) 2009-2010
%   Laurent Claessens
% See the file fdl-1.3.txt for copying conditions.


\begin{corrige}{INGE1121La0014}

	La matrice correspondante est 
	\begin{equation}
		A=
		\begin{pmatrix}
			 0	&	1	&	-1/2	&	1/2	\\
			 1	&	0	&	-1/2	&	1/2	\\
			 -1/2	&	-1/2	&	0	&	-1	\\ 
			 1/2	&	1/2	&	-1	&	0	 
		 \end{pmatrix}.
	\end{equation}
	Afin de trouver le rang de cette matrice, nous échelonnons un petit peu. Il y a plusieurs façons de procéder. On peut par exemple faire $L_2\to L_2-2L_4$ et $L_3\to L_3+L_4$ :
	\begin{equation}
		A=
		\begin{pmatrix}
			 0	&	1	&	-1/2	&	1/2	\\
			 0	&	-1	&	3/2	&	1/2	\\
			 0	&	0	&	-1	&	-1	\\ 
			 1/2	&	1/2	&	-1	&	0	 
		 \end{pmatrix},
	\end{equation}
	et ensuite $L_2\to L_2+L_1$:
	\begin{equation}
		A=
		\begin{pmatrix}
			 0	&	1	&	-1/2	&	1/2	\\
			 0	&	0	&	1	&	1	\\
			 0	&	0	&	-1	&	-1	\\ 
			 1/2	&	1/2	&	-1	&	0	 
		 \end{pmatrix}.
	\end{equation}
	À partir de là, nous voyons que la deuxième et la troisième ligne sont égales au signe près, donc le déterminant de la matrice est nul et le rang ne peut pas être $4$. Pour prouver que le rang est $3$, il suffit de prendre à peu près n'importe quelle sous matrice de taille $3$ et de calculer le déterminant. Par exemple la matrice obtenue en enlevant la troisième ligne et la troisième colonne :
	\begin{equation}
		\det\begin{pmatrix}
			0	&	1	&	1/2	\\
			0	&	0	&	1	\\
			1/2	&	1/2	&	0
		\end{pmatrix}=1/2\neq 0.
	\end{equation}
	Le rang de la matrice est donc $3$. 

	Le fait que le déterminant soit nul montre qu'il y aura une valeur propre nulle. Mais le fait que le rang soit $3$ dit également que la valeur propre nulle sera seulement de multiplicité un. Il y aura donc $3$ valeurs propres non nulles.

	Pour les valeurs propres, il faut calculer et résoudre le polynôme caractéristique
	\begin{equation}
		P_A(\lambda)=\det
			\begin{pmatrix}
				 -\lambda	&	1	&	-1/2	&	1/2	\\
				 1	&	-\lambda	&	-1/2	&	1/2	\\
				 -1/2	&	-1/2	&	-\lambda	&	-1	\\ 
				 1/2	&	1/2	&	-1	&	-\lambda	 
			 \end{pmatrix}.
	\end{equation}
	Les solutions avec les multiplicités sont $\{ -1,-1,0,2 \}$, et les vecteurs propres correspondants sont
	\begin{equation}
		\begin{aligned}[]
			2 &\to (1, 1, -1, 1)\\
			0 &\to (1, 1, 1, -1)\\
			-1 &\to (1, -1, 0, 0)\\
			-1 &\to (0, 0, 1, 1)
		\end{aligned}
	\end{equation}
	La forme quadratique n'a pas de genre particulier. Notre ami Gram-Schmidt, suivit d'une renormalisation,nous fournit la base orthonormale suivante :
	\begin{equation}
		\begin{aligned}[]
			(1/2, 1/2, -1/2, 1/2)\\
			(1/2, 1/2, 1/2, -1/2)\\
			(\frac{ \sqrt{2}}{2}, -\frac{ \sqrt{2} }{2}, 0, 0)\\
			(0, 0, \frac{ \sqrt{2} }{2}, \frac{ \sqrt{2} }{2})
		\end{aligned}
	\end{equation}

	En mettant ces vecteurs en colonne, on obtient la matrice de changement de variables
	\begin{equation}
		\begin{aligned}[]
			x_1 &= \frac{ \sqrt{2} }{2}y_3 + \frac{ 1 }{2}y_1 + \frac{ 1 }{2}y_2\\
			x_2 &= -\frac{ \sqrt{2} }{2}y_3 + \frac{ 1 }{2}y_1 + \frac{ 1 }{2}y_2\\
			x_3 &= \frac{1}{ \sqrt{2}}y_4 - \frac{ 1 }{2}y_1 + \frac{ 1 }{2}y_2\\
			x_4 &= \frac{ \sqrt{2} }{2}y_4 + \frac{ 1 }{2}y_1 - \frac{ 1 }{2}y_2
		\end{aligned}
	\end{equation}
	En mettant ces valeurs dans $p(X)$, nous trouvons bien
	\begin{equation}
		p\big( X(Y) \big)=2y_1^2 - y_3^2 - y_4^2,
	\end{equation}
	qui est bien la forme dont les valeurs propres de $A$ sont les coefficients.

	Cette forme n'est pas d'un genre particulier\footnote{À part d'être le genre qui peut arriver à l'examen.}.

\end{corrige}
