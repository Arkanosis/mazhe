% This is part of Outils mathématiques
% Copyright (c) 2011
%   Laurent Claessens
% See the file fdl-1.3.txt for copying conditions.

\begin{exercice}\label{exoDerive-0007}

    Soit $f\colon \eR\to \eR$ une fonction continue. Soient $F$ et $G$ deux primitives de $f$. Montrer que le nombre $F(b)-F(a)=G(b)-G(a)$. En d'autres termes, le nombre $F(b)-F(a)$ ne dépend pas du choix de la primitive de $f$.

\corrref{Derive-0007}
\end{exercice}
