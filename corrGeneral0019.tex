% This is part of the Exercices et corrigés de mathématique générale.
% Copyright (C) 2009
%   Laurent Claessens
% See the file fdl-1.3.txt for copying conditions.
\begin{corrige}{0019}

\begin{enumerate}

\item
Ceci est une intégrale par parie tout ce qu'il y a de plus traditionnel. Nous posons
\begin{equation}
	\begin{aligned}[]
		u&=x	&	dv&= e^{x}dx\\
		du&=dx	&	v&=e^x,
	\end{aligned}
\end{equation}
nous trouvons alors
\begin{equation}
	I=xe^x-\int e^{x}dx=(x-1)e^x.
\end{equation}

\item
Étant donné qu'intégrer $\ln(x)$ ne nous ragoûte pas trop, mais que le dériver serait du meilleur effet en société, nous essayons par partie avec les choix suivants :
\begin{equation}
	\begin{aligned}[]
		u&=\ln(x)		&	dv&=x^2\\
		du&=\frac{1}{ x }	&	v&=\frac{ x^3 }{ 3 }.
	\end{aligned}
\end{equation}
Nous obtenons
\begin{equation}
	I=\frac{ x^3\ln(x) }{ 3 }-\int\frac{ x^3 }{ 3 }=\frac{ x^3\ln(x) }{ 3 }-\frac{ x^3 }{ 9 }.
\end{equation}

\item
Pour celui-ci, il se passe que l'on sait dériver $\arcsin(x)$, donc nous allons faire $\int\arcsin(x)dx=\int 1\cdot\arcsin(x)dx$, et poser
\begin{equation}
	\begin{aligned}[]
		u&=\arcsin(x)		&	dv&=dx\\
		du&=\frac{1}{ \sqrt{1-x^2} }	&	v&=x,
	\end{aligned}
\end{equation}
ce qui amène à
\begin{equation}
	I=x\arcsin(x)-\int\frac{ xdx }{ \sqrt{1-x^2} }=x\arcsin(x)-\sqrt{1-x^2}.
\end{equation}

\item
Celle-ci est un exemple typique de ce qu'il se passe quand on a deux fonctions \og cycliques\fg, c'est à dire deux fonctions dont les dérivées bouclent. Posons
\begin{equation}
	\begin{aligned}[]
		u&=e^x		&	dv&=\sin(x)dx\\
		du&=e^xdx	&	v&=-\cos(x),
	\end{aligned}
\end{equation}
ce qui amène
\begin{equation}
	I=-e^x\cos(x)=\int\cos(x)e^x,
\end{equation}
dont l'intégrale n'a pas l'air plus sympathique que celle de départ. Si nous refaisons par partie,
\begin{equation}
	\begin{aligned}[]
		u&=e^x		&	dv&=\cos(x)dx\\
		du&=e^xdx	&	v&=\sin(x),
	\end{aligned}
\end{equation}
nous tombons sur
\begin{equation}
	I=e^x\big( \sin(x)-\cos(x) \big)-\int\sin(x)e^x.
\end{equation}
L'intégrale qui reste est exactement celle que nous devons calculer. Donc
\begin{equation}
	I=e^x\big( \sin(x)-\cos(x) \big)-I,
\end{equation}
ou encore
\begin{equation}
	I=\frac{1}{ 2 }e^x\big( \sin(x)-\cos(x) \big).
\end{equation}

\end{enumerate}


\end{corrige}
