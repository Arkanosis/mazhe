% This is part of Outils mathématiques
% Copyright (c) 2012
%   Laurent Claessens
% See the file fdl-1.3.txt for copying conditions.

\begin{corrige}{OutilsMath-0145}

Le domaine est la couronne de rayon interne \( 1\) et de rayon externe \( 2\) prise dans le quart de plan. Les bornes polaires sont donc
\begin{subequations}
    \begin{numcases}{}
        r\colon 1\to 2\\
        \theta\colon 0\to \frac{ \pi }{2},
    \end{numcases}
\end{subequations}
et en n'oubliant pas le jacobien \( r\), l'intégrale à calculer est
\begin{equation}
    \int_0^{\pi/2}d\theta\int_1^2 r\cos(r^2)dr.
\end{equation}
L'intégrale sur \( \theta\) revient à multiplier par \( \frac{ \pi }{ 2 }\). En ce qui concerne celle sur \( r\), un bon changement de variable est \( u=r^2\) avec quoi il reste
\begin{equation}
    \frac{ \pi }{2}\int_1^4\cos(u)\frac{ du }{ 2 }=\frac{ \pi }{ 4 }\big( \sin(4)-\sin(1) \big).
\end{equation}

\end{corrige}
