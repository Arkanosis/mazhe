\section{Introduction to Moyal star product}\label{app:Moyal}
%----------------------------------------------

\subsection{General definition of a star product}
%-------------------------------------------

The symbol $A\dcr{\nu}$ denotes the set of formal series of power of $\nu$ with coefficients in $A$, \textit{i.e.}
\[
A\dcr{\nu}=\{\sum_{k=0}^{\infty}a_k\nu^k\text{ with } a_i\in A\}.
\]
  For such formal series, the equality $\sum_{k=0}^{\infty}a_k\nu^k=\sum_{k=0}^{\infty}b_k\nu^k$ means $a_k=b_k$ for all $k$.

Let $(M,P)$ be a Poisson manifold and $\Cinf(M)$, the algebra of differentiable functions on $M$. We know that it is associative and that it is a Lie algebra for the Poisson bracket.


\begin{definition}
A \defe{formal star product}{star product!formal} on $M$ is a bilinear map $\dpt{\ast}{\Cinf(M)\times\Cinf(M)}{\Cinf(M)\dcr{\nu}}$ which can be written as
\begin{equation}
       u\ast v=\sum_{k=0}^{\infty}\nu^kC_k(u,v)
\end{equation}
where the $\dpt{C_k}{\Cinf(M)}{\Cinf(M)}$ are bilinear maps such that
\renewcommand{\labelenumi}{(\roman{enumi})}
\begin{enumerate}
\item the $\eR\dcr{\nu}$-linear extension $\ast$ to $\Cinf(M)\dcr{\nu}\times\Cinf(M)\dcr{\nu}$ is associative:
\begin{equation}
                               u\ast(v\ast w)=(u\ast v)\ast w,
\end{equation}


\item the maps $C_0$ and $C_1$ are subject to
\begin{equation}
\begin{aligned}
  C_0(u,v)         &=uv       & \text{usual product}\\
  C_1(u,v)-C_1(v,u)&=\PB{u}{v}&\text{Poisson bracket}.
\end{aligned}
\end{equation}

\item $1\ast u=u\ast 1=u$.
\end{enumerate}
If moreover the $C_r$ for $r\ge 1$ are bidifferential operators on $M$, we say that $\ast$ is a \defe{differential star product}{star product!differential}.

\end{definition}

\subsection{Definition of the Moyal star product}
%//////////////////////////////////////////////////

\subsubsection{On \texorpdfstring{$\eR^n$}{Rn} with constant \texorpdfstring{$P$}{P}}
%***************************************

We consider the Poisson manifold $(\eR^{2m},P)$  with $P=\frac{1}{2}\sum_{ij}P^{ij}\partial_i\wedge\partial_j$ where $P^{ij}$ are constant. By definition, $\PB{u}{v}=P(u,v)$.

We want to define a star product $\ast$ with $\exp(P^{ij}\partial_i\wedge\partial_j)$, but $\exp P\circ\exp P$ makes no sense because $\dpt{\exp P}{\Cinf(M)\times\Cinf(M)}{\Cinf(M)}$. So we write\index{Moyal star product!on $\protect\eR^{n}$}
\begin{equation}\label{eq:Moyal}
\begin{split}
(u\ast_M v)(x)&=\exp\frac{\nu}{2}(P^{ij}\partial_{y^i}\wedge\partial_{z^j})(u(y)v(z))|_{y=z=x}\\
              &=u(x)v(x)+\frac{\nu}{2}\PB{u}{v}(x)+\ldots
\end{split}
\end{equation}
The Moyal product reads
\begin{equation}   \label{EqDevStatrMoyal}
u\ast_{M}v=uv+\nu\{ u,v \}+\sum_{k=2}^{\infty}\frac{ \nu^{k} }{ k! }
\sum_{
\begin{subarray}{l}
i_{1}\ldots i_{k}\\ j_{1}\ldots j_{k}
\end{subarray}}
\Omega^{i_{1}j_{1}}\ldots\Omega^{i_{k}j_{k}}\partial_{i_{1}}\ldots\partial_{i_{k}}u\,\partial_{j_{1}}\ldots\partial_{j_{k}}v.
\end{equation} 

Let us consider the canonical form of $P$ and a Darboux system of chart:
\[
P=\begin{pmatrix}
0 & \mtu \\
-\mtu & 0
\end{pmatrix},\qquad x=\begin{pmatrix}
x_p \\
x_q
\end{pmatrix}.
\]
We define a \defe{twisted Fourier transform}{Fourier transform!twisted} on $\Cinf(M)$ as follows. For $u\in\Cinf(M)$, $\hu=F(u)\in\Cinf(M)$ is defined by:
\begin{equation}
               \hu(x)=F(u)(y)=\int e^{-iP(x,y)}u(x)dx.
\end{equation}
This satisfies some properties which looks like the true Fourier transform; \emph{inter alia}, this posses an associated ``convolution''\ product:
\begin{equation}\label{1906r1}
       (\hu\times\hv)(y)=\int e^{-iP(\xi,c)}u(c-b)v(b)e^{iP(y,b)}d\xi\,dc\,db
                        =\int e^{iP(y,b)}u(-b)v(b)\,db
                       =F(\underline{u}v)
\end{equation}
where $\underline{u}(x)=u(-x)$.

\begin{lemma}
The twisted Fourier transform has the following property with respect to the derivative:
\[
        \widehat{\partial_iv}=iP^{il}y^l\hv.
\]
\end{lemma}
\begin{proof}
First notice that the integration by part usually gives a ``boundary''\ term, but in a dense subset of $\Cinf(M)$, the following computation is true:
\begin{equation}
\begin{split}
\int(\partial_iv)(x)e^{-iP(x,y)}dx&=-\int v(x)\partial_i\big(e^{-iP^{kl}x^ky^l}\big)dx\\
                            &=iP^{il}y^l\int v(x)e^{-P(x,y)}dx\\
                            &=iP^{il}y^l\hv(y).
\end{split}
\end{equation}
\end{proof}

We state the following lemma without proof.
\begin{lemma}
\[
            \underline{\partial_iu}=-\partial_i\underline{u}
\]
\end{lemma}

\begin{theorem}
The Moyal star product is associative.
\end{theorem}
\begin{proof}
 Equation \eqref{1906r1} allows us to write the first term of $F(u\ast v)$; the two lemmas will help us to deal with the others terms.

With an obvious multi-index notation, we can write:
\begin{equation}
\begin{split}
  F(P^{i_1j_1}\ldots &P^{i_kj_k}\partial_{x^{i_1}\ldots x^{i_k}}u\partial_{x^{j_1}\ldots x^{j_k}}v)\\
  &=P^{IJ}(\widehat{\underline{\partial_Iu}})\times\widehat{\partial_Jv}\\
  &= P^{IJ}(-1)^k\widehat{\partial_I\underline{u}}\times\widehat{\partial_Jv}\\
  &=P^{IJ}(-1)^k(i)^kP^{IL}[\underline{\hu}\cdot y^L]\times P^{JM}(i)^k[\hv\cdot y^M]\\
  &=(-1)^kP^{ML}(\underline{\hu}y^L)\times(\hv y^M)\\
  &=(-1)^k\int (z-y)^LP^{ML}z^M\underline{\hu}(z-y)\hv(z)dz\\
  &=(-1)^k\int P(z-y,z)^k\underline{\hu}(z-y)\hv(z)dz.
\end{split}  
\end{equation}
The last equality comes from the fact that $a^IP^{IJ}b^J=P(a,b)^k$.
It is the way to compute the twisted Fourier transform of $u\ast v$. If we put $\nu=i\lambda$ The whole series gives:
\begin{equation}
\begin{split}
   F(u\ast v)(y)&=\int\exp-\frac{\nu}{2}(P(z-y,z))\underline{\hu}(z-y)\hv(z)dz\\
                &=\int e^{\displaystyle i(\lambda/2)P(z,y)}\hu(y-z)\hv(z)dz\\
		&=(\hu\times_{\lambda}\hv)(y).
\end{split}
\end{equation}		
The operation $\times_{\lambda}$ is called the \defe{twisted convolution}{convolution!twisted}\index{twisted!convolution}. We can end the proof:
\begin{equation}
\begin{split}
   (\hu\times_{\lambda}(\hv\times_{\lambda}\hw))(y)
    &=\int\exp i\frac{\lambda}{2}P(z,y)\hu(u-z)\exp i\frac{\lambda}{2}P(z',z)\hv(z-z')\hw(z')\\
    &=\exp i\frac{\lambda}{2}P(z,y-z')\hu(y-z)\hv(z-z')\hw(z').
\end{split}
\end{equation}
Computing $(\hu\times_{\lambda}\hv)\times_{\lambda}\hw$, we find the same.

\end{proof}

\subsubsection{General Moyal product \texorpdfstring{$M$}{M}}
%********************************************************

The true Moyal star product is nothing but to take a Darboux system of charts on $M$ and to put the $\eR^n$ Moyal product on each.\index{Moyal star product!on general symplectic manifold}

\begin{remark}
The Moyal product of polynomials leads to \emph{finite} sums because the multi-derivation always ends with a zero.
\end{remark}

%+++++++++++++++++++++++++++++++++++++++++++++++++++++++++++++++++++++++++++++++++++++++++++++++++++++++++++++++++++++++++++
\section{Rieffel's deformation by action of \texorpdfstring{$\eR^d$}{Rd}}
%+++++++++++++++++++++++++++++++++++++++++++++++++++++++++++++++++++++++++++++++++++++++++++++++++++++++++++++++++++++++++++
\index{Rieffel}

Let $A$ be a Fréchet $*$-algebra and a strongly continuous action by $*$-homomorphisms $\alpha$ of $\eR^d$ on $A$. We suppose that $A$ has a system of seminorms $\| . \|_k$ defining the topology of $A$ and such that $\alpha_v$ is an isometry for each $\| . \|_k$. The space $A^{\infty}\subset A$ of smooth vectors of the action is a Fréchet subalgebra of $A$ with the seminorms
\begin{equation}
	\| a \|_{k,\mu}=\sup_{| \beta |\leq\mu}\| \partial_{\beta}a \|_k
\end{equation}
where $\partial_{\beta}a$ is defined by
\begin{equation}
	\partial_{\beta}a=\Dsdd{ \alpha_{te_{\beta}}(a) }{t}{0}
\end{equation}
when $| \beta |=1$. Here $e_{\beta}$ is the vector of the canonical basis.

In this setting, Rieffel showed that if $\theta$ is an antisymmetric bilinear non degenerate on $\eR^d$ and if $\hbar>0$, then the formula
\begin{equation}
	a\star_{\hbar} b=\frac{1}{ (\pi \hbar)^{2n} }\int_{V\times V}\alpha_u(a)\alpha_v(b) e^{\frac{ 2i }{ \hbar }\theta(u,v)}dudv
\end{equation}
yields a well defined and continuous product on $A^{\infty}$.


%%%%%%%%%%%%%%%%%%%%%%%%%%
%
   \section{Weyl product}
%
%%%%%%%%%%%%%%%%%%%%%%%%

Let $(V,\Omega)$ be a symplectic vector space of dimension $2n$. For $u$, $v\in C^{\infty}_c(V)$, we define the \defe{Weyl product}{Weyl!product} $u\stWh v$ by the formula
\begin{equation}   \label{eq_def_Weyl}
  (u\stWh v)(x)=\hbar^{-2n}\int_{V\times V}e^{\frac{ 2i }{ \hbar }S^0(x,y,z)}u(y)v(z)\,dy\,dz
\end{equation}
where $S^0(x,y,z)=\Omega(x,y)+\Omega(y,z)+\Omega(z,x)$.

In order to prove stability of Schwartz space $\swS(V)$ under the Weyl product, we follow \cite{Garcia_Bondia}. We begin by defining the following on $\swS(V)$:
\begin{equation}
\begin{aligned}
 (\mu_jf)(u)&=u_jf(u)&\partial_if&=\frac{ \partial f }{ \partial u_i }\\
 (\tau_sf)(u)&=f(u-s)&(\epsilon_sf)(u)&=e^{i\Omega(s,u)}f(u).
\end{aligned}
\end{equation}
We also define 
\begin{subequations}
\begin{equation}
\hat\partial_jf=
\begin{cases}
\partial_{j+N}f&\text{if }1\leq j\leq N\\
-\partial_{j-N}f&\text{if }N<j\leq 2N.
\end{cases}
\end{equation}
In other words, $\hat\partial_i$ is a derivative with respect to the momentum associated with the position $i$. If we denote by $\bar\imath$ the conhjugated variable of $i$ and $\epsilon_i$ the corresponding sign, we can rewrite this definition under the form 
\begin{equation}
\hat\partial_i=\epsilon_i\partial_{\bar\imath}.
\end{equation}

\end{subequations}

The change of variable $s=u-v$ and $t=w-u$ gives
\[ 
  (f\stWh g)(u)=\iint f(u-s)g(t+u)e^{-i\Omega(s,t)}\,ds\,dt.
\]
We define the \defe{twisted convolution}{twisted!convolution}\index{convolution!twisted}
\begin{equation}
 (f\diamond g)(u)=\int f(u-t)g(t)e^{-i\Omega(u,t)}\,dt
\end{equation}


\begin{proposition}
We have the formula
\[ 
  f\stWh g= F^{-1}(Ff\diamond Fg)
\]
where $F$ denotes the Fourier transform. 

\end{proposition}

\begin{proof}
The computation is as follows:
\begin{equation}
\begin{split}
  F^{-1}(Ff\diamond Fg)(u)&=\int e^{i\alpha u}(Ff\diamond Fg)(\alpha)\,d\alpha\\
	&=\int e^{i\alpha u}e^{-i\Omega(\alpha,t)}e^{-\beta(\alpha-t)}e^{-\sigma t}f(\beta)g(\sigma)\,d\beta\,d\sigma\,dt\,d\alpha.
\intertext{If $J$ is the symplectic matrix, $\Omega(\alpha,t)=\alpha\cdot Jt$; we use this fact to make appears Dirac delta's with the change of variable $t'=Jt$}
		&=\int \delta(u-\beta-t')e^{-iJ(\beta-\sigma)\cdot t'}f(\beta)g(\sigma)\,d\sigma\,d\beta\,dt'\\
		&=\int e^{-iJ(\beta-\sigma)\cdot (u-\beta)}f(\beta)g(\sigma)\,d\sigma\,d\beta\\
		&=(f\stWh g)(u).
\end{split}
\end{equation}

\end{proof}

\begin{theorem}
  The Schwartz space is stable for the Weyl product: if $f$, $g\in\swS$, then $f\stWh g\in\swS$. Moreover the map $\swS\times\swS\to\swS$, $(f,g)\mapsto f\stWh g$ is a bilinear continuous map. The following equalities hold:
\begin{subequations}
\begin{equation} \label{eq_pjstW_a}
\partial_j(f\stWh g)=\partial_ij\stWh g+f\stWh\partial_ig
\end{equation}
\begin{equation} 
\begin{split}\label{eq_pjstW_b}
\mu_i(f\stWh g)&=f\stWh\mu_ig+i\hat\partial_i(f\stWh g)\\
	&=\mu_i(f\stWh g)-if\stWh\hat\partial_ig.
\end{split}
\end{equation}

\end{subequations}

\end{theorem}


\begin{proof}
In order to prove equation \eqref{eq_pjstW_a}, we remark that the integral in the right hand side of \eqref{eq_def_Weyl} converges uniformly with respect to $u$. So one can permute the derivative and the integral:
\[ 
 \begin{split} 
  \frac{ \partial }{ \partial u_i }\big( (f\stWh g)(u) \big)&=\int f(v)g(w)\frac{ \partial }{ \partial u_i }\big( e^{i[\Omega(u,v)+\Omega(v,w)+\omega(w,u)]} \big)\,dv\,dw\\
		&=\int f(v)g(w)e^{i[\Omega(u,v)+\Omega(v,w)+\omega(w,u)]}(\epsilon_i v_{\bar\imath}-\epsilon_i w_{\bar\imath})
\end{split}
\]
where $\epsilon_i$ is a sign and $\bar\imath$ is the conjugated variable of $i$. An integration by part leads to \eqref{eq_pjstW_a}.  

Definitions give
\begin{subequations}
\begin{align}
  (f\stWh \mu_ig)(u)&=\int f(u-s)(t_j+u_j)g(t+u)e^{-i\Omega(s,t)}\,ds\,dt\\
 (\hat\partial_jf\stWh g)(u)&=\int \epsilon_j(\partial_{\bar\imath f})(u-s)g(t+u)e^{-i\Omega(s,t)}\,ds\,dt
\end{align}
\end{subequations}
we integrate the second integral by part using formula
\[ 
  \frac{ \partial }{ \partial s_{\bar\jmath} }\big( e^{-i\Omega(s,t)} \big)=i\epsilon_jt_je^{-i\Omega(s,t)}.
\]
Then, combining with the first, we find equation \eqref{eq_pjstW_b}.

An induction argument using uniform convergence of the integral and Leibnitz shows that $f\stWh g$ is smooth. The limit $\lim_{u\to\infty}(f\stWh g)(u)$ is zero because\quext{J'espère que c'est vrai !}
\[ 
  \lim_{u\to\infty}\int f(u-u)e^{it}\,dt=0.
\]
An induction using formula  \eqref{eq_pjstW_b} shows that $f\stWh g\in\swS$.
 
In order to prove continuity, we consider the upper bound
\begin{equation}
\begin{split}
  \| f\stWh g \|_{\infty}&=\sup_u| (f\stWh g)(u) |\\	
		&\leq \int | f(u-s) |\,ds\,\int | g(t+u) |\,dt\\
		&=\| f \|_1\| g \|_1
\end{split}
\end{equation}
The topologies given by seminorms\quext{Il faut vérifier ce point}
\[ 
  q_{\alpha\gamma}(f)=\| \mu^{\alpha}\partial^{\gamma}f \|_1
\]
is equivalent to the one defined by the seminorms
\[ 
  p_{\alpha\gamma}(f)=\| \mu^{\alpha}\partial^{\gamma}f \|_{\infty}.
\]
So the equality
\[ 
  \mu^{\alpha}\partial^{\gamma}(f\stWh g)=\sum_{\beta<\alpha}\sum_{\epsilon<\gamma}(-i)^{| \beta |}\binom{\alpha}{\beta}\binom{\gamma}{\epsilon}\mu^{\alpha-\beta}\partial^{\gamma-\epsilon}f\stWh \hat\partial^{\beta}\partial^{\epsilon}g.
\]
imply
\[ 
  p_{\alpha\gamma}(f\stWh g)\leq\sum_{\beta}\sum_{\epsilon} \binom{\alpha}{\beta}\binom{\gamma}{\epsilon}q_{\alpha-\beta,\gamma-\epsilon}(f)q_{0,\eta+\epsilon}(g).
\]
because $\| f\stWh g \|_{\infty}\leq\| f \|_1\| g \|_1$. This proves that $(f,g)\mapsto f\stWh g$ is continuous.

\end{proof}


%+++++++++++++++++++++++++++++++++++++++++++++++++++++++++++++++++++++++++++++++++++++++++++++++++++++++++++++++++++++++++++
\section{The twisting map}
%+++++++++++++++++++++++++++++++++++++++++++++++++++++++++++++++++++++++++++++++++++++++++++++++++++++++++++++++++++++++++++

We consider the elementary normal symplectic symmetric space $\eS$ defined in section \ref{SecElemNormSymplSpace} and its description of proposition \ref{Prop2807DescSMdarboux}. We follow the paper \cite{QuantifKhalerian}.

Let $\tilde\eS=\{ (a,v,\xi \}=\eR\times\eR^{2n}\times\eR$. The \defe{twisting map}{twisting map} is the smooth family (with respect to $\theta$) of diffeomorphisms
\begin{equation}
	\begin{aligned}
		\phi_{\theta}\colon \tilde\eS&\to \tilde\eS \\
		(a,v,\xi)&\mapsto \Big( a,\cosh\left( \frac{ \theta }{ 4 }\xi \right)^{-1}v,\frac{ 2 }{ \theta }\sinh\left( \frac{ \theta }{ 2 }\xi \right) \Big) 
	\end{aligned}
\end{equation}
with $\theta\in\eR$. We denote by $\varphi_{\theta}\in\Diff(\eR)$ the partial map
\begin{equation}
	\varphi_{\theta}(\xi)=\frac{ 2 }{ \theta }\sinh(\frac{ \theta }{ 2 }\xi).
\end{equation}

We denote by $\mF$ the partial Fourier transform
\begin{equation}
	(\mF u)(a,v,\xi)=\hat u(a,v,\xi)=\int e^{-i\xi t}u(a,v,t)dt.
\end{equation}
The space Schwartz functions in the variables $(a,v,t)$ is denoted by $\swS$ and $\tilde\swS$ in the variables $(a,v,\xi)$. By general theory, the Fourier transform is an isomorphism
\begin{equation}
	\mF\colon \swS\to \tilde\swS
\end{equation}

Let us now consider the space of function
\begin{equation}
	\mO_C=\{ f\in C^{\infty}(\eR^m)\tq \exists r>0\tq\forall\alpha\in\eN^m,| \partial^{\alpha}f(x) |<C_{\alpha}(1+| x |)^r \}.
\end{equation}

\begin{definition}
	The space $\Theta$ is the space of functions $\theta\colon \eR\to  C^{\infty}(\eR)$ denoted by $\tau\colon \theta\to \tau_{\theta}$ such that
	\begin{enumerate}
		\item
			For every $\theta$, the functions $x\mapsto e^{\pm\tau_{\theta}(x)}$ belong to $\mO_C(\eR)$.
		\item
			We have $(\varphi_{\theta}^*\tau_{\theta})|_{\theta=0}\equiv 0$. More explicitly, for every $\xi$ we have
			\begin{equation}
				\lim_{\theta\to 0} (\varphi_{\theta}\circ\tau_{\theta})(\xi)=0
			\end{equation}
	\end{enumerate}
\end{definition}

%+++++++++++++++++++++++++++++++++++++++++++++++++++++++++++++++++++++++++++++++++++++++++++++++++++++++++++++++++++++++++++
\section{Wigner function}
%+++++++++++++++++++++++++++++++++++++++++++++++++++++++++++++++++++++++++++++++++++++++++++++++++++++++++++++++++++++++++++

The Weyl quantization formula is\nomenclature{$\Op$}{Weyl quantization}
\begin{equation}
 \big( \Op(a)u \big)(x)=\iint_{\eR^n\times\eR^n} a\left( \frac{ x+y }{ 2 },\eta \right) e^{2\pi i\scal{ x-y }{ \eta }}u(y)\,dy\,d\eta
\end{equation}
when $a\in\swS'(\eR^n\times\eR^n)$. It defines a continuous map $\dpt{ \Op(a)  }{ \swS(\eR^n) }{ \swS'(\eR^n) }$. Let $\varphi$, the usual gaussian function on $\eR^n$:
\[ 
  \varphi(t)=2^{n/4} e^{-\pi| t |^2},
\]
 and for all $X=(x,\xi)\in\eR^n\times\eR^n$, 
\[ 
  \varphi_X(t)=\varphi(t-x) e^{2\pi i\scald{ t-\frac{ x }{2} }{ \xi }}.
\]
When $u$, $v\in L^2(\eR^n)$, the scalar product is defined by
\begin{equation}
(u,v)=\int u(t)\overline{ v }(t)\,dt
\end{equation}
and Parseval equality leads to 
\begin{equation}  \label{eq_intuvpXb}
  (u,v)=\int_{\eR^n}(u,\varphi_X)(\varphi_X,v)\,dX
\end{equation}
because the integral of $ e^{2\pi i\scald{ z-y }{ \xi }}$ over $\xi$ gives rise to a Dirac delta $\delta(z-y)$. Equation \eqref{eq_intuvpXb} still holds when $u\in\swS(\eR^n)$ and $v\in\swS'(\eR^n)$.

Without proof:
\begin{probleme}
Il manque le reste de l'égalité. Énoncé imcomplet donc. À compléter.
\end{probleme}

\begin{proposition}
If $u\in\swS'(\eR^n)$ and for all $k\in\eN$, the equality
\[ 
  \int_{\eR^{2n}} (1+| X |)^2 | (u,\varphi_X) |\,dX
\]
holds, then $u\in\swS(\eR^n)$.
\end{proposition}

The \defe{Wigner function}{Wigner function} $W_{XX'}$ is defined by
\begin{equation}
  \big( \Op(a)\varphi_X,\varphi_{X'} \big)=\int a(Y)W_{XX'}(Y)\,dY.
\end{equation}
It can be developed under the form\quext{Que je ne suis pas très bien parvenu à démonter.}
\begin{equation}
  W_{XX'}(Y)=2^n e^{-i\pi\Omega(X,X')} e^{2\pi i\Omega(Y,X'-X)} e^{-2\pi\left|  Y-\frac{ X+X' }{ 2 }  \right|^2}
\end{equation}
where $\Omega$ is the symplectic form on $\eR^n\times\eR^n$ defined by
\[ 
  \Omega\big( (x,\xi),(x',\xi') \big)=-\scald{ x }{ \xi' }+\scald{ x' }{ \xi }.
\]

A \defe{weight function}{weight!function} on $\eR^{2n}$ is a positive function $m$ for which there exists $c_1$ and $n_1$ such that
\[ 
  m(X)\leq c_1m(X')\big[ 1+| X-X' |^2 \big]^{n_1}
\]
for all $X,X'\in\eR^{2n}$. A function $a\in C^{\infty}(\eR^{2n})$ is a \defe{symbol of weight}{symbol!of weight $m$} $m$ if  for all multi-index $\alpha\in\eN^{2n}$, the function $m^{-1}D^{\alpha}a$ is bounded, i.e. if
\[ 
  \sup_{(x,\xi)}\Big|  \frac{ D^{\alpha}a(x,\xi) }{ m(x,\xi) } \Big|\leq\infty.
\]


\begin{theorem}
   Let $a\in\swS'(\eR^{2n})$. This is a symbol of weight $m$ if and only if for all $k\in\eN$, there exists a $c>0$ such that 
\[ 
  \big|  (\Op(a)\varphi_X,\varphi_{X'})  \big|\leq c\big( 1+| X-X' |^2 \big)^{-k}m\Big( \frac{ X+X' }{2} \Big)
\]
for all $X,X'\in\eR^{2n}$.
\end{theorem}





