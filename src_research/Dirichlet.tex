% This is part of Mes notes de mathématique
% Copyright (c) 2014
%   Laurent Claessens
% See the file fdl-1.3.txt for copying conditions.

%+++++++++++++++++++++++++++++++++++++++++++++++++++++++++++++++++++++++++++++++++++++++++++++++++++++++++++++++++++++++++++
\section{KMS states and CAR algebras}
%+++++++++++++++++++++++++++++++++++++++++++++++++++++++++++++++++++++++++++++++++++++++++++++++++++++++++++++++++++++++++++

%---------------------------------------------------------------------------------------------------------------------------
\subsection{KMS states}
%---------------------------------------------------------------------------------------------------------------------------

We refer to \cite{CirpianiDirichlet} for missing proofs and more details. The document \cite{VaesLocCompQG} seems interesting too.

\begin{definition}
	Let $\{ \alpha_t\}_{t\in\eR}$ be a strongly continuous one parameter semigroup of automorphisms of a $C^*$-algebra $A$ and $\beta\in\eR$. A state $\omega$ is a $(\alpha,\beta)-KMS$-state if is fulfils the \defe{KMS condition}{KMS!condition}:
	\begin{equation}		\label{EqKMScondPourOmega}
		\omega\big( a\alpha_{i\beta}(b) \big)=\omega(ba)
	\end{equation}
	for every $a$ and $b$ in a norm dense and $\alpha$-invariant $*$-subalgebra.
\end{definition}
Note that for $\beta=0$, the state $\omega$ is a trace.

A map $\Phi\colon A\to A$ is \defe{positive}{positive!map between $C^*$-algebra} when $\Phi(A_+)\subset A_+$ where $A_+$ is the set of positive elements in $A$ (definition \ref{DefApplPositive}). It is \defe{completely positive}{positive!completely!map in a $C^*$-algebra} if for every $n\in\eN_0$, the extension
\begin{equation}
	\begin{aligned}
		\Phi\otimes I_n\colon \eM_(A)&\to \eM_n(A) \\
		(\Phi\otimes I_n)[a_{ij}]&=\big[ \Phi(a_{ij}) \big] 
	\end{aligned}
\end{equation}
is positive on the $C^*$-algebra $\eM_n(A)=A\otimes\eM_n(\eC)$, see definition \ref{DefComplPositive}. The map $\Phi$ is \defe{Markovian}{Markovian} is it is positive and if $\Phi(a)\leq 1_A$ for every $a\in A$ such that $a=a*$ and $a\leq 1_A$. The map is \defe{completely Markovian}{Markovian!completely!map on a $C^*$-algebra} if the map $\Phi\otimes I_n$ is Markovian on $\eM_n(A)$.

A one parameter semigroup $\{ \Phi_t\}_{t\geq 0}$ of maps $\Phi_t\colon A\to A$ has one of these properties when each of $\Phi_t$ has.

%---------------------------------------------------------------------------------------------------------------------------
\subsection{CAR algebras}
%---------------------------------------------------------------------------------------------------------------------------

Let $V$ be a vector space with a bilinear non degenerated symmetric form. The $*$-algebra generated by the elements of $V$ and subject to the relations
\begin{subequations}
	\begin{numcases}{}
		vw+wv=\langle v, w\rangle \\
		v^*=v
	\end{numcases}
\end{subequations}
is the \defe{CAR algebra}{CAR algebra} over $V$. Formally it is the quotient of the tensor algebra over $V$ by the relations $v\otimes w+w\otimes v=\langle v, w\rangle $.

Let $\hH_0$ be a pre-Hilbert space and $\hH$ be its completion. We have a CAR algebra $U(\hH)$ given by the antilinear map $a\colon \hH_0\to \oB(\hH)$ and the relations 
\begin{subequations}
	\begin{numcases}{}
		a(f)a(g)+a(g)a(f)=0\\
		a(f)a(g)^*+a(g)^*a(f)=\langle f, g\rangle 	\label{subEqafagssproddCAR}
	\end{numcases}
\end{subequations}
for every $f$ and $g$ in $\hH_0$\cite{ArvesonCAR}.

If $H$ is a selfadjoint on $\hH$, we can define the automorphism group $\alpha\colon \eR\to \Aut\big( U(\hH) \big)$ by
\begin{enumerate}
	\item
		$\alpha_t\big( a(f) \big)=a\big(  e^{itH}f \big)$,
	\item
		$\alpha_t\big( a^*(f) \big)=a^*\big(  e^{itH}f \big)$
\end{enumerate}
for every $f\in\hH_0$. This is a strongly continuous group of automorphisms of the CAR algebra called the \defe{Bogoliubov transform}{Bogoliubov transform}. If $\beta\in\eR$, the condition for a state $\omega$ to be $(\alpha,\beta)$-KMS requires among others
\begin{equation}		\label{EqomCARagmm}
	\omega\big( a(g)a^*( e^{-\beta H}f) \big)=\omega\Big( a(g)\alpha_{i\beta}\big( a^*(f) \big) \Big)=\omega\big( a^*(f)a(g) \big).
\end{equation}
The CAR relation \eqref{subEqafagssproddCAR} provides
\begin{equation}
	a(g)a^*\big(  e^{-\beta H}f \big)=\langle g,  e^{-\beta H}f\rangle \id-a^*( e^{-\beta  H}f)a(g).
\end{equation}
Writing the relation \eqref{EqomCARagmm} with $(\id+ e^{-\beta H})^{-1}f$ instead of $f$, we get
\begin{equation}
	\omega\Big( a(g)a^*\big(  e^{-\beta H}(\id+ e^{-\beta H})^{-1}f \big) \Big)=\omega\Big( a^*\big( (\id+ e^{-\beta H})^{-1}fa(g) \big) \Big),
\end{equation}
but using the (anti)linearity of $a^*$ and $\omega$ we have
\begin{equation}
	\begin{aligned}[]
		\omega\Big( a^*\big( (\id+ e^{-\beta H})f \big)a(g) \Big)&=\omega\Big( \big( a^*(f)+a^*( e^{-\beta H}f) \big)a(g) \Big)\\
		&=\omega\big( a^*(f)a(g) \big)+\omega\big( a^*( e^{\beta H}f)a(g) \big)\\
		&=\omega\big( a^*(f)a(g) \big)+\langle g,  e^{-\beta H}f\rangle -\omega\big( a(g)a^*( e^{-\beta H}f) \big)
	\end{aligned}
\end{equation}
where we used the fact that $a^*( e^{-\beta H}f)a(g)=\langle g,  e^{-\beta H}f\rangle \id-a(g)a^*( e^{-\beta H}f)$ and the fact that $\omega(\id)=1$ since $\omega$ is a state. In the last line, the first and last terms sum to zero because of the KMS condition. What we obtained is
\begin{equation}
	\omega\Big( a^*\big( [\id+ e^{-\beta H}]f\big)a(g) \Big)=\langle g,  e^{-\beta H}f\rangle .
\end{equation}
If we write this with $[\id+ e^{-\beta H}]^{-1}f$ instead of $f$, we get
\begin{equation}
	\omega\big( a^*(f)a(g) \big)=\langle g,   e^{-\beta H}[\id+ e^{-\beta H}]^{-1}f \rangle .
\end{equation}
It turns out that, using the CAR relations, that formula uniquely defines $\omega$. This is then the unique $(\alpha,\beta)$-KMS state on $U(\hH)$ and is denoted by $\omega_{\beta}$.

%---------------------------------------------------------------------------------------------------------------------------
\subsection{KMS symmetric maps}
%---------------------------------------------------------------------------------------------------------------------------

\begin{definition}
	Let $\{ \alpha_t \}_{t\in\eR}$ be a strongly continuous group of automorphisms of the $C^*$-algebra $A$ and $\omega$ a $(\alpha,\beta)$-KMS-state for some $\beta\in\eR$. A bounded map $\Phi\colon A\to A$ is said to be $(\alpha,\beta)$-\defe{KMS symmetric}{KMS!symmetric map} with respect to $\omega$ if
	\begin{equation}		\label{EqKMSCondPhi}
		\omega\big( b\Phi(a) \big)=\omega\Big( \alpha_{-\frac{ i\beta }{2}}(a)\Phi\big( \alpha_{\frac{ i\beta }{2}}(b) \big) \Big)
	\end{equation}
	for every $a$ and $b$ in a norm dense and $\alpha$-invariant $*$-algebra $B$ of $A_{\alpha}$. 
\end{definition}
We already mentioned that, when $\beta=0$, the state $\omega$ is a trace; what happens with the KMS symmetry condition on $\Phi$ in the case $\beta=0$ is
\begin{equation}
	\omega\big( b\Phi(a) \big)=\omega\big( a\Phi(b) \big).
\end{equation}

Let us suppose that, for a generic $\beta$, the map $\Phi$ commutes with the action. The symmetry condition becomes
\begin{equation}
	\begin{aligned}[]
		\omega(b\Phi(a))&=\omega\Big( \alpha_{-\frac{ i\beta }{2}}(a)\alpha_{\frac{ i\beta }{2}}\big( \Phi(b) \big) \Big)\\
		&=\omega\big( \Phi(b)\alpha_{-\frac{ i\beta }{2}}(a) \big)\\
		&=\omega\big( a\Phi(b) \big)
	\end{aligned}
\end{equation}
where we used twice the $KMS$ condition for $\omega$.

\begin{lemma}
	If $\omega$ is a $(\alpha,\beta)$-KMS state, the map $\Phi$ is KMS-symmetric with respect to $\omega$ if and only if
	\begin{equation}
		\omega\big( \Phi(a)\alpha_{\frac{ i\beta }{2}}(b) \big)=\omega\big( \alpha_{\frac{-i\beta }{2}}(a)\Phi(b) \big).
	\end{equation}
\end{lemma}

\begin{proof}
	We write the usual KMS condition \eqref{EqKMScondPourOmega} for $\omega$ with the replacements $b\to \alpha_{\frac{ -i\beta }{2}}(b)$ and $a\to \Phi(a)$:
	\begin{equation}
		\omega\big( \Phi(a)\alpha_{\frac{ i\beta }{2}}(b) \big)=\omega\big( \alpha_{-\frac{ i\beta }{2}}(b)\Phi(a) \big),
	\end{equation}
	then we use the KMS condition \eqref{EqKMSCondPhi} for $\Phi$ with the replacements $a\to b$ and $b\to\alpha_{\frac{ -i\beta }{2}}(a)$ in order to get $\omega\big( \alpha_{-\frac{ i\beta }{2}}(a)\Phi(b) \big)$.
\end{proof}

We recall that an \defe{entire function}{entire function} is a function holomorphic over $\eC$. We define the set $D_{\beta}$ as
\begin{equation}
	\begin{aligned}[]
		D_{\beta}&=\{ z\in\eC\tq 0<\imag z<\beta \}	&\text{if $\beta\geq 0$}\\
		D_{\beta}&=\{ z\in\eC\tq \beta<\imag z<0 \}	&\text{if $\beta\leq 0$}
	\end{aligned}
\end{equation}

\begin{proposition}
	Let $\alpha=\{ \alpha_t \}_{t\in\eR}$ be a strongly continuous group of automorphisms of the $C^*$-algebra $A$, $\beta\in\eR$ and $\omega$ be a state on $A$. The following two conditions are equivalent:
	\begin{enumerate}
		\item
			$\omega$ is a $(\alpha,\beta)$-KMS state;
		\item
			for every pair $a$, $b$ in $A$, there exists an analytic map $F_{a,b}\colon \overline{ D_{\beta} }\to A$ such that
			\begin{subequations}		\label{EqDEfFabi}
				\begin{align}
					F_{a,b}(t)&=\omega(a\alpha_t(b))\\
					F_{a,b}(t+i\beta)&=\omega(\alpha_t(b)a)
				\end{align}
			\end{subequations}
			for $t\in\eR$.
	\end{enumerate}
	If these conditions are satisfied, we have $| F_{a,b}(z) |\leq\| a \|\cdot\| b \|$ for every $z\in\overline{ D }_{\beta}$.

	Moreover, if $a\in A$ and $b\in A_{\alpha}$, the function $F_{a,b}$ is the restriction to $\bar D_{\beta}$ of the entire function 
	\begin{equation}
		z\mapsto\omega\big( a\alpha_z(b) \big).
	\end{equation}
\end{proposition}
The point in the proposition is that the relations \eqref{EqDEfFabi} define only $F_{a,b}$ on the boundary of $D_{\beta}$.

We have the same kind of proposition for a KMS symmetry.

\begin{proposition}		\label{PropFabcomegaKSM}
	Let $\alpha=\{ \alpha_t \}_{t\in\eR}$ be a strongly continuous automorphism group of the $C^*$-algebra $A$ and $\omega$, a $(\alpha,\beta)$-KMS state for some $\beta\in\eR$. Then for a bounded map $\Phi\colon A\to A$, the following conditions are equivalent:
	\begin{enumerate}
		\item
			$\Phi$ is $(\alpha,\beta)$-KMS symmetric with respect to $\omega$;
		\item
			pour every pair $a$, $b$ in $A$, there exists a continuous bounded analytic map $F_{a,b}\colon \bar D_{\beta}\to A$ such that
			\begin{subequations}
				\begin{align}
					F_{a,b}(t)&=\omega\Big( \alpha_{-\frac{ t }{2}}(a)\Phi\big( \alpha_{\frac{ t }{ 2 }}(b) \big) \Big)\\
					F_{a,b}(t+i\beta)&=\omega\Big( \alpha_{\frac{ t }{2}}(b)\Phi\big( \alpha_{\frac{ t }{ 2 }}(a) \big) \Big)	\label{subEqFabitbhi}
				\end{align}
			\end{subequations}
			for every $t\in\eR$.
	\end{enumerate}
	If these conditions are satisfied, the function $F_{a,b}$ is bounded on $\bar D_{\beta}$ and
	\begin{equation}
		| F_{a,b}(z) |\leq\| \Phi \|\cdot\| a \|\cdot\| b \|
	\end{equation}
	for every $z\in\bar D_{\beta}$. If $a\in A$ and $b\in A_{\alpha}$, the function $F_{a,b}$ is the restriction to $\bar D_{\beta}$ of the entire function
	\begin{equation}
		G_{a,b}(z)=\omega\Big( \alpha_{-\frac{ z }{2}}(a)\Phi\big( \alpha_{\frac{ z }{2}}(b) \big) \Big)
	\end{equation}
	on $\eC$.
\end{proposition}

Let us briefly recall the GNS construction\index{GNS representation} of a $C^*$-algebra $A$ (see theorem \ref{ThoGNScontruction}). Let $\omega$ be a state on $A$. First we consider the space
\begin{equation}
	\mN_{\omega}=\{ a\in A\tq\omega(a^*a)=0 \}=\{ a\in A\tq \omega(b^*a)=0\,\forall b\in A \},
\end{equation}
and we define the Hilbert space $\hH_{\omega}$ as the completion of the quotient $A/\mN_{\omega}$. The representation $\pi_{\omega}$ on $\hH_{\omega}$ is then defined as
\begin{equation}
	\pi_{\omega}(a)[b]=[ab]
\end{equation}
where the bracket $[a]$ denotes the class of $a\in A$ for the quotient by $\mN_{\omega}$.

The kernel of the representation is given by the elements $a$ such that
\begin{equation}
	[ab]=0
\end{equation}
for every $b\in A$. This means that $a\in\mN_{\omega}$. Thus the kernel of the GNS representation is given by the set of elements $a\in A$ such that $\omega(a^*a)=0$.

Suppose that we have an semigroup $\alpha_t$ of automorphisms of $A$ and a $\alpha$-invariant state $\omega$. Then if $a\in\ker(\pi_{\omega})$ we have
\begin{equation}
	\begin{aligned}[]
		\omega\big( \alpha_t(a^*)\alpha_t(a) \big)&=\omega\big( \alpha_t(a^*a) \big)\\
		&=\omega(a^*a)\\
		&=0,
	\end{aligned}
\end{equation}
so that $\alpha_t(a)$ belongs to $\ker(\pi_{\omega})$ too.

\begin{proposition}
	Let $\{ \alpha_t \}_{t\in\eR}$ be a strongly continuous group of automorphisms of the $C^*$-algebra $A$ and $\omega$ be a fixed $(\alpha,\beta)$-KMS state for some $\beta\in\eR$. Then a map $\Phi\colon A\to A$ which is $(\alpha,\beta)$-KMS symmetric with respect to $\omega$ leaves globally invariant the kernel $\ker(\pi_{\omega})$ of the GNS representation of $\omega$.
\end{proposition}

\begin{proof}
	Let $a\in\ker(\pi_{\omega})$. Since $\alpha_t$ is an automorphism of $A$, we also have $\alpha_t(a)\in\ker(\pi_{\omega})$. Let $F_{a,b}$ is the map guaranteed by proposition \ref{PropFabcomegaKSM}; since $\omega$ is positive (it is a state, definition \ref{DefStateCSA}), it fulfils the condition \eqref{eq:omABleq}, so that
	\begin{equation}
		| F_{a,b}(t) |^2=|\omega\Big( \alpha_{-\frac{ t }{2}}(a)\Phi\big( \alpha_{\frac{ t }{ 2 }}(b) \big) \Big)|^2\leq \omega\Big( \alpha_{-\frac{ t }{2}}(a)\big( \alpha_{-\frac{ t }{2}}(a) \big)^* \Big)\omega\Big(  \big(\Phi \alpha_{\frac{ t }{2}}(b) \big)^*\Phi\big( \alpha_{\frac{ t }{2}} \big)(b) \Big).
	\end{equation}
	Since $\alpha_t$ is an automorphism and since $\omega$ is $\alpha_t$-invariant, the first factor becomes
	\begin{equation}
		\omega(\alpha_{-\frac{ t }{2}}(aa^*))=\omega(aa^*)=0.
	\end{equation}
	What we proved is that $| F_{a,b}(t) |=0$ when $t$ is real. Being analytic, the function $F_{a,b}$ vanishes everywhere. By property \eqref{subEqFabitbhi},
	\begin{equation}
		0=F_{a,b}(i\beta)=\omega\big( b\Phi(a) \big).
	\end{equation}
	Taking $b=\Phi(a)^*$, we get $\Phi(a)\in\ker(\pi_{\omega})$.
\end{proof}

%+++++++++++++++++++++++++++++++++++++++++++++++++++++++++++++++++++++++++++++++++++++++++++++++++++++++++++++++++++++++++++
\section{Dirichlet and traces}
%+++++++++++++++++++++++++++++++++++++++++++++++++++++++++++++++++++++++++++++++++++++++++++++++++++++++++++++++++++++++++++

Other source : \cite{TrioloSemifinite,CiprianiStandardForms,CiprianiSauvageotSquareRoots}.

Let $A$ be a $C^*$-algebra and $\tau\colon A^+\to \mathopen[ 0 , \infty \mathclose]$ be a faithful, densely defined, semifinite and lower semicontinuous trace on $A^+$ (see section \ref{SecTraceCstar}). We consider \(\mL\), the set of \defe{square integrable}{square integrable} elements in \(A\), that is:
\begin{equation}
	\mL=\{ a\in A\tq \tau(aa^*)<\infty \}.
\end{equation}
This set is an ideal in $A$ and $\tau$ can be uniquely extended to a linear functional on $\mN=\langle \mL^*\mL\rangle$ (lemma \ref{LemTraceAplusextmlmn}). We still denote by $\tau$ the extension. In particular the formula
\begin{equation}
	(a,b)_{\tau}=\tau(a^*b)
\end{equation}
defines a sesquilinear form on $\mL$. We denote by $L^2(A,\tau)$\nomenclature[C]{$L^2(A,\tau)$}{Space associated to the trace $\tau$ in a $C^*$-algebra} the Hilbert space obtained by completion of $\mL$ with respect to the sesquilinear form $(.,.)_{\tau}$. Such a construction is similar to the one of $L^2(M)$ on page \pageref{PgLdM}. Since the trace is faithful, we have \(\omega(a^*a)=0\) only when \(a=0\). The space \(L^2(A,\tau)\) is thus also the GNS construction described in theorem \ref{ThoGNScontruction}.

We denote by $\eta_{\tau}\colon \mL\to L^2(A,\tau)$ the natural injection whose image is, by construction, dense. The space $\eta_{\tau}(\mL)$ has the product
\begin{equation}
	\langle \eta_{\tau}(a), \eta_{\tau}(b)\rangle =(a,b)_{\tau}=\tau(a^*b).
\end{equation}
If, for $a\in A$ and $b\in\mL$, we define
\begin{equation}
	\pi_{\tau}\eta_{\tau}(a)=\eta_{\tau}(ab),
\end{equation}
we have
\begin{equation}
	\langle \eta_{\tau}(c), \pi_{\tau}(a)\eta_{\tau}(b)\rangle =\tau(c^*ab).
\end{equation}
The map $\pi_{\tau}\colon A\to \oB\big(L^2(A,\tau)\big)$ is a representation of $A$ on $L^2(A,\tau)$. We consider the von~Neumann algebra $L^{\infty}(A,\tau)\vnM=\pi_{\tau}(A)''$.\nomenclature[C]{$L^{\infty}(A,\tau)$}{von~Neumann algebra associated to the trace $\tau$ on the $C^*$-algebra $A$} Since $\tau$ is defined on $\pi_{\tau}(A)$ and that $\vnM$ is a strong closure of $\pi_{\tau}(A)$ (theorem \ref{ThoDoubleCommutant}), the form $\tau$ can be extended to a normal functional on $\vnM$. From now we write $a$ the element $\eta_{\tau}(a)$ in $L^2(A,\tau)$.

Let
\begin{equation}
	C=\overline{ \{ a\in\mL\tq a=a^*\leq 1_{\vnM} \} }\subset L^2(A,\tau).
\end{equation}
This set is convex. Indeed, let $a,b\in C$ and consider $(1-t)a+tb$. We have
\begin{equation}
	(1-t)a+tb\leq(1-t)1_{\vnM}+t 1_{\vnM}=1_{\vnM}.
\end{equation}
We write $a\wedge 1$ the projection of $a$ onto $C$.

The notation is inspired from the fact that if we consider the real function $f(t)=t\wedge 1=\min(t,1)$, we have $a\wedge 1=f(a)$ in the case $a=a^*\in\mL$ by the continuous functional calculus, theorem \ref{ThoContFuncCalculus}. If $b\in C$, we have $\Spec(b)\subset\mathopen] -\infty , 1 \mathclose]$, so that $f(t)=t$ and $f(b)=b$. If $a$ is outside $C$, we have
\begin{equation}
	\Spec(a\wedge 1)=f\big( \Spec(a) \big)\subset\mathopen] -\infty , 1 \mathclose].
\end{equation}
Then we have $\Spec(1-a\wedge 1)\subset\mathopen[ 0 , \infty \mathclose]$, which proves that $1-a\wedge 1\geq 0$.

A quadratic functional
\begin{equation}
	\dirE\colon L^2(A,\tau)\to \mathopen] -\infty , \infty \mathclose]
\end{equation}
can be extended to $\eM_n(A)=A\otimes\eM_n(\eC)$ in the following way. First we consider the trace $\tau_n=\tau\otimes\tr_n$ on $\eM_n(A)$ then we define $\dirE_n\colon L^2\big( \eM_n(A),\tau_n \big)\to \eC$ by
\begin{equation}
	\dirE_n[a]=\sum_{ij}\dirE[a_{ij}].
\end{equation}
The domain of $\dirE$ is the set
\begin{equation}
	\mF=\{ \xi\in L^2(A,\tau)\tq \dirE[\xi]<\infty \}.
\end{equation}
The functional $\dirE$ is said to be
\begin{enumerate}
	\item
		\defe{$J$-real}{real!$J$-real quadratic functional} if $\dirE[J\xi]=\dirE[\xi]$ for every $\xi\in L^2(A,\tau)$;
	\item
		\defe{Markovian}{Markovian!quadratic functional} if $\dirE[\xi\wedge 1]\leq\dirE[\xi]$ for every $\xi\in L^2(A,\tau)$ such that $\xi=J\xi$;
	\item
		\defe{Dirichlet form}{Dirichlet!form} if it is Markovian and lower semicontinuous;
	\item
		\defe{completely Dirichlet}{Dirichlet!completely} if the extensions $\dirE_n$ are Dirichlet for all $n\geq 1$;
	\item
		\defe{regular}{regular!quadratic form} if the subspace $\mB=A\cap\mF$ is dense-norm in $A$ and a form core for $(\dirE,\mF)$ (definition \ref{DefFormCoreDomq});
	\item
		\defe{$C^*$-Dirichlet form}{Dirichlet!form!$C^*$} if it is regular and completely Dirichlet.
\end{enumerate}

\begin{lemma}       \label{LemELfermEinter}
    Let \(E=\{ a\in\mL\tq a=a^*\leq 1_{\vnM} \}\) and \(C=\overline{ E }\). We recall that \(L_+^2(A,\tau)=\overline{ \mL_+ }\). Then
    \begin{equation}
        C\cap L_+^2(A,\tau)=\overline{ \{ a\in\mL\tq 0\leq a\leq 1_{\vnM} \} }.
    \end{equation}
\end{lemma}

\begin{proof}
    First, it is obvious that \(\{ a\in\mL\tq 0\leq a\leq 1_\vnM \}=E\cap\mL_+\). Thus we have to show that \(\overline{ E }\cap\overline{ \mL_+ }=\overline{ E\cap\mL_+ }\). The inclusion 
    \begin{equation}
        \overline{ E\cap\mL_+ }\subset\overline{ E }\cap\overline{ \mL_+ }
    \end{equation}
    is only topology.

    The delicate part is the reverse inclusion. Let \(x\in\overline{ E }\cap\overline{ \mL_+ }\). In a first time we suppose that \(\| x \|=1\). By hypothesis we can find a sequence \(x_i\to x\) with \(x_i\in\mL_+\). Since \(\| x_i \|\to 1\), we consider the sequence
    \begin{equation}
        y_i=\frac{ x_i }{ \| x_i \| }\to x.
    \end{equation}
    The element \(y_i\) still belongs to \(\mL_+\) from proposition \ref{PropAplusConvexCone}. Moreover \(y_i\leq \cun\) because proposition \ref{PropAAsmAuAAu} shows that \(x_i\leq\| x_i \|\cun\). So we have \(y_i\in E\cap \mL_+\) and \(x\in\overline{ E\cap\mL_+ }\).

    Let us now consider any \(x\in\overline{ E }\cap\overline{ \mL_+ }\). By the preceding point we build a sequence \(y_i\in E\cap\mL_+\) such that
    \begin{subequations}
        \begin{numcases}{}
            y_i\to\frac{ x }{ \| x \| }\\
            \| y_i \|=1.
        \end{numcases}
    \end{subequations}
    Then we consider the sequence \(z_i=\| x \|y_i\). This converges to \(x\) and \(z_i\in\mL_+\). We still have to show that \(z_i\in E\).

    First we prove that \(\| x \|\leq \cun\). For that we consider \(a_i\to x\) with \(a_i\in E\). Since \(x\in\overline{ \mL_+ }\) for every \(\epsilon\) we have
    \begin{equation}
        \Spec(a_i)\subset\mathopen[ -\epsilon , \| a_i \| \mathclose].
    \end{equation}
    We conclude that \(a_i\geq 1\). Indeed the spectrum of \(a_i+1\) is made of \(\lambda\) such that \(a_i-(\lambda-1)\cun\) is not invertible. Thus \(\lambda\in\Spec(a_i+1)\) if and only if \(\lambda-1\in\Spec(a_i)\) and \(\lambda-1\geq \epsilon\), so that \(\lambda\geq 1-\epsilon>0\). The spectrum of \(a_i+\cun\) being positive we have \(a_i\geq -\cun\). Now we have \(-\cun\leq a_i\leq \cun\) and proposition \ref{prop:mBABineq} concludes that \(\| a_i \|\leq 1\). Since \(a_i\to x\) we conclude that \(\| x \|\leq 1\).

    \begin{probleme}
        Here I'm using the fact that the norm is continuous. This is due to the fact that we are precisely working on the completion with respect to the norm.
    \end{probleme}
    
    By the definition \(z_i=\| x \|y_i\) and \(\| y_i \|=1\), we have \(\| z_i \|\leq 1\). Once again by proposition \ref{PropAAsmAuAAu} we have \(z_i\leq\| z_i \|\cun\leq \cun\).

\end{proof}

\begin{definition}
    Let $\{ T_t \}_{t\geq 0}$ be a $J$-real, symmetric, strongly continuous semigroup acting on $L^2(A,\tau)$. That semigroup is said to be
    \begin{enumerate}
        \item
            \defe{Markovian}{Markovian!semigroup on $L^2(A,\tau)$} if it leaves globally invariant the set (see lemma \ref{LemELfermEinter})
            \begin{equation}
                C\cap L^2(A,\tau)=\overline{ \{ a\in\mL\tq 0\leq a\leq 1_{\vnM} \} };
            \end{equation}
        \item
            \defe{completely positive}{positive!completely!semigroup on $L^2(A,\tau)$} if its extensions to $L^2\big( \eM_n(A),\tau_n \big)$ are positive;
        \item
            \defe{completely Markovian}{Markovian!completely!semigroup on $L^2(A,\tau)$} if its extensions to $L^2\big( \eM_n(A),\tau_n \big)$ are Markovian.
    \end{enumerate}
\end{definition}

%+++++++++++++++++++++++++++++++++++++++++++++++++++++++++++++++++++++++++++++++++++++++++++++++++++++++++++++++++++++++++++
\section{Modules over $C^*$-algebra}
%+++++++++++++++++++++++++++++++++++++++++++++++++++++++++++++++++++++++++++++++++++++++++++++++++++++++++++++++++++++++++++

\begin{definition}
	Let $A$ be a $C^*$-algebra. A $A$\defe{-bimodule}{bimodule!over a $C^*$-algebra}\index{module!over a $C^*$-algebra} is an Hilbert space $\hH$ with commuting left and right representations. The bimodule is said to be \defe{symmetric}{symmetric!bimodule} if there is an isometric antilinear involution $J\colon \hH\to \hH$ exchanging the left and right actions:
	\begin{equation}
		J(a\xi b)=b^*J(\xi)a^*
	\end{equation}
	for every $a,b\in A$ and $\xi\in\hH$.
\end{definition}

Let \(a\in A\) and \(\xi\in \hH\). If \(\pi\) is a representation of \(A\) on \(\hH\) we have \(\| \pi(a)\xi \|\leq\| \pi(a) \|\| \xi \|\), and lemma \ref{Lemrepresnormpresou} shows that
\begin{equation}        \label{Eqpiaxileqanormxi}
    \| \pi(a)\xi \|\leq \| a \|\| \xi \|.
\end{equation}
In the case of a \(A\)-module we do not explicitly write the representation \(\pi\) and we write \(\| a \xi b\|\leq\| a \|\| b \|\| \xi \|\).

When $\hH$ is a bimodule over $A$, a \defe{derivation}{derivation!over a $C^*$-algebra} is a linear map $\partial\colon D(\partial)\subset A\to \hH$ such that
\begin{equation}
	\partial(ab)=\partial(a)b+a\partial(b)
\end{equation}
for every $a,b\in D(\partial)$. Notice that one always has \(\partial(1)=0\). A derivation is \defe{symmetric}{symmetric!derivation on a $C^*$-algebra} if its domain $D(\partial)$ is a selfadjoint subalgebra and if the $A$-bimodule $(\hH,J)$ is symmetric with $\partial(a^*)=J(\partial a)$.

%---------------------------------------------------------------------------------------------------------------------------
\subsection{Example: gradient of a function}
%---------------------------------------------------------------------------------------------------------------------------

Let $M$ be a Riemannian manifold and consider $A=C_b(M)$, the $C^*$-algebra of continuous bounded function on $M$; $\hH=L^2(TM)$ the space of square summable vector fields on $M$. The gradient
\begin{equation}
	\nabla\colon C^{\infty}_b(M)\to L^2(TM)
\end{equation}
is a derivation and $L^2(TM)$ is a $C_b(M)$-bimodule with the pointwise multiplication.

%---------------------------------------------------------------------------------------------------------------------------
\subsection{Left and right representation}
%---------------------------------------------------------------------------------------------------------------------------

If $a\in A$ is a selfadjoint element, we can represent the $C^*$-algebra $C\big( \Spec(a) \big)$ on $\hH$ by
\begin{equation}
	L_a(f)\xi=\begin{cases}
		f(a)\xi	&	\text{if $f(0)=0$}\\
		\xi	&	 \text{if $f\equiv 1$}
	\end{cases}
\end{equation}
where $f\in C\big( \Spec(a) \big)$ and $\xi\in\hH$. This is a good definition when $f$ is a polynomial. The action with non polynomial continuous functions is defined by density. The right action is defined by
\begin{equation}
	R_a(f)\xi=\begin{cases}
		\xi f(a)	&	\text{if $f(0)=0$}\\
		\xi	&	 \text{if $f\equiv 1$}.
	\end{cases}
\end{equation}

Let \(I\) be an interval in \(\eR\) and \(f\in C^1(I)\). The \defe{quantum derivative}{quantum!derivative} of \(f\) is the function \(\tilde f\in C(I\times I)\) defined by
\begin{equation}
    \tilde f(s,t)=\begin{cases}
        \frac{ f(s)-f(t) }{ s-t }    &   \text{if \(s\neq t\)}\\
        f'(s)    &    \text{if \(s=t\)}.
    \end{cases}
\end{equation}

\begin{lemma}
    Let \(\hH,J\) a symmetric Hilbert \(A\)-bimodule and \( (\partial,D(\partial))\) be a symmetric derivation defined on an involutive subalgebra of \(A\). Let \(a=a^*\in D(\partial)\). Then
    \begin{enumerate}
        \item
            For every polynomials \(f\), we have \(f(a)\in D(\partial)\) and the \emph{chain rule}
            \begin{equation}        \label{EqLeibnitzDerrAbiomolule}
                \partial\big( f(a) \big)=(L_a\otimes R_a)(\tilde f)\partial(a).
            \end{equation}
            holds. Moreover we have the bound
            \begin{equation}        \label{EqBoundpartialfanfpnpa}
                \| \partial\big( f(a) \big) \|\leq \| f' \|_{C(\Spec(a))}\| \partial(a) \|.
            \end{equation}
        \item
            If \( (\partial,D(\partial))\) is closable as operator \(A\to\hH\), then the closure is a derivation.
        \item
            If \( (\partial,D(\partial))\) is a closed derivative \(A\to\hH\), then the formula \eqref{EqLeibnitzDerrAbiomolule} holds for every function \(f\in C^1\big( \Spec(a) \big)\) such that \(f(0)=0\). In particular the domain of a closed derivative is closed for the \(C^1\) functional calculus.
    \end{enumerate}
    
\end{lemma}

\begin{remark}
    The equation \eqref{EqLeibnitzDerrAbiomolule} is a short notation for the following. First we introduce the following representation of \(A\otimes A^{op}\) on \(\hH\):
    \begin{equation}
        (a\otimes b)\sharp \xi=a\xi b.
    \end{equation}
    Then \(  (L_a\otimes R_a)(\tilde f)\xi \) with \(\tilde f(0)=0\) stands for
    \begin{equation}        \label{EqLaRattfsurxi}
        (L_a\otimes R_a)(\tilde f_1\otimes \tilde f_2 )\sharp\xi=\Big( L_a(\tilde f_1)\otimes R_a( \tilde f_2) \Big)\sharp\xi=\tilde f_1(a)\xi\tilde f_2(a).
    \end{equation}
    Here the functions \(\tilde f_i\) are associated with \(\tilde f\) by the map \eqref{EqIsoCABCACBCstar}.
\end{remark}

\begin{proof}
    \begin{enumerate}
        \item
            If \(a=a^*\in D(\partial)\), then \(f(a)\in D(\partial)\) because the domain of \(\partial\) is an algebra.

            Let us begin with the constant function \(f=1\). In this case \(\tilde f(s,t)=0\) and all terms are vanishing in the decomposition \eqref{EqDecompffklCABCACB}, so \(\tilde f=0\otimes 0\). Formula \eqref{EqLaRattfsurxi} then produces \(\partial(1)=0\).

            If \(f(t)=t\), then \(\tilde f(s,t)=1\) and we get \(\partial(a)=\partial(1)\). 

            The general statement comes from the formula
            \begin{equation}
                \frac{ x^n-y^n }{ x-y }=\sum_{k=0}^{n-1} x^ky^{n-1-k}.
            \end{equation}
            Then, for \(f(t)=\sum_n\alpha_n t^n\), one defines
            \begin{equation}
                Df_a=\sum_n \alpha_n\sum_{k=0}^{n-1}  a^k\otimes a^{n-1-k}
            \end{equation}
            and the formula to be checked becomes \(\partial\big( f(a) \big)=Df_a\sharp\partial(a)\) that can be checked by induction.

            Let us now prove the bound \eqref{EqBoundpartialfanfpnpa}. If \(f(t)=t^n\), then
            \begin{equation}
                Df_a\sharp\xi=\sum_{k=0}^{n-1}a^k\xi a^{n-1-k}.
            \end{equation}
            Using the inequality \eqref{Eqpiaxileqanormxi} we found
            \begin{equation}
                \begin{aligned}[]
                    \| Df_a\sharp\xi \| &\leq\sum_k\| a^k\xi a^{n-1-k} \|\\
                    &\leq\sum_k \| a^k \|\| a^{n-1-k} \|\| \xi \|\\
                    &\leq\sum_{k=0}^{n-1}\| a \|^{n-1}\| \xi \|\\
                    &=n\| a \|^{n-1}\| \xi \|.
                \end{aligned}
            \end{equation}
            In this computation we also used the fact that \(  \| ab \|\leq \| a \|\| b \|\). Since \(\| a \|\leq\sup_{t\in\Spec(a)}| t |\) by proposition \ref{ThoSpecBanach}\ref{ItemThoSpecBanachi}, we have
            \begin{equation}
                n\| a \|^{n-1}\leq\sup_{t\in\Spec(a)}| f'(t) |=\| f' \|_{C( \Spec(a) )}.
            \end{equation}
            
        \item
            Let \( a_n\to a\) and \( b_n\to b\) and consider the Leibnitz formula
            \begin{equation}        \label{eqPartialanbnLeibnitz}
                \partial(a_nb_n)=a_n(\partial b_n)+(\partial a_n)b_n
            \end{equation}
            Since the operator \( \partial\) is supposed to be closable (see proposition \ref{PropoOpFermableLim}), we can define \( \partial(a)=\lim\partial(a_n)\). By continuity of the left and right actions, we also have
            \begin{equation}
                \lim a_n\partial b_n=a\partial b.
            \end{equation}
            In the same spirit, \( \lim(a_nb_n)=ab\) and we have \( \lim\partial(a_nb_n)=\partial(ab)\). Thus taking the limit in both side of equation \eqref{eqPartialanbnLeibnitz}, we found \( \partial(ab)=a\partial b+(\partial a)b\).
        \item
            Now we suppose that \( (\partial,D(\partial))\) is closed and we consider \( f\in C^1\big( \Spec(a) \big)\). In particular \( f'\) is continuous and we can consider polynomials \( f_n\) such that \( f'_n\to f'\) and \( f_n\to f\) uniformly on \( \Spec(a)\). Indeed on each connected component \( K\) of \( \Spec(a)\) we have
            \begin{equation}
                \| f_n(t)-f(t) \|\leq \sup_{t\in K}| f'_n(t)-f'(t) |\ell=\ell\| f'_n-f' \|\leq 2\| a \|\| f'_n-f' \|
            \end{equation}
            where \( \ell\) is the length of \( K\). We also used the fact that the length of \( \Spec(a)\) is smaller or equal to \( 2\| a \|\) by  Since \( f'_n\to f_n\) uniformly, the right-hand side can be made smaller than \( \epsilon\) by a suitable choice of \( n\). We suppose that \( f_n(0)=0\) by choosing the suitable primitive of \( f'_n\).

            For each \( n\) we have 
            \begin{equation}
                \| \partial\big( f_n(a) \big) \|\leq \| f'_n \|\| \partial(a) \|.
            \end{equation}
            We prove now that \( \partial\big( f_k(a) \big)\) is a Cauchy sequence. By linearity, \( \partial\big( f_k(a) \big)-\partial\big( f_l(a) \big)=\partial\big( (f_k-f_l)(a) \big)\). By the bound \eqref{EqBoundpartialfanfpnpa} we have
            \begin{equation}        \label{EqpartfkfllmesqCauchy}
                \| \partial\big( (f_k-f_l)(a) \big) \|\leq \| f'_k-f'_l \|\| \partial(a) \|.
            \end{equation}
            Since \( f'_k\to f'\) uniformly, the right hand side of \eqref{EqpartfkfllmesqCauchy} can be set smaller than \( \epsilon\) by choosing suitably large \( k\) and \( l\). This shows that \( f(a)\in D(\partial)\) and, by definition, \( \partial\big( f(a) \big)=\lim\partial\big( f_n(a) \big)\).

            As far as \( \tilde f\) is concerned, we have the uniform limit \( \tilde f_{n,i}\to\tilde f_i\) and then
            \begin{equation}
                \partial\big( f_n(a) \big)=\tilde f_{n,1}(a)\partial(a)\tilde f_{n,2}(a).
            \end{equation}
            By definition the limit of the left hand side is \( \partial\big( f(a) \big)\) while the limit of the right hand side is, by continuity of the actions, \( \tilde f_1(a)\partial(a)\tilde f_2(a)\).
            
    \end{enumerate}
    
\end{proof}

\begin{theorem}
	Let $\big( \partial,D(\partial) \big)$ be a symmetric derivation on $A$ defined on a dense subset with values in a symmetric Hilbert $A$-bimodule $(\hH,J)$. We suppose that $D(\partial)$ is dense in $L^2(A,\tau)$ and that $\big( \partial,D(\partial) \big)$ is a closable form from $L^2(A,\tau)$ to $\hH$. The form $(\dirE,\mF)$ defined by
	\begin{equation}
		\begin{aligned}
			\dirE\colon L^2(A,\tau)&\to \mathopen[ 0 , \infty [ \\
			a&\mapsto \| \partial a \|^2_{\hH} 
		\end{aligned}
	\end{equation}
	is  a $C^*$-Dirichlet form with $\mF=D(\partial)$.

	Moreover, if $\Delta$ is the generator of the completely Markovian semigroup associated to $\dirE$, we have $\Delta=\partial^*\circ \partial$.
\end{theorem}

\begin{proposition}
	Let $\big( \dirE,D(\dirE) \big)$ be a completely Dirichlet form on $L^2(A,\tau)$. Then $\mB=A\cap D(\dirE)$ is an involutive subalgebra of $\vnM$.

	If $\big( \dirE,D(\dirE) \big)$ is a $C^*$-Dirichlet form, the subalgebra $\mB$ is an involutive subalgebra of $A$ and a core form for $\big( \dirE,D(\dirE) \big)$.
\end{proposition}

The subalgebra $\mB=A\cap D(\dirE)$ is the \defe{Dirichlet algebra}{Dirichlet!algebra} of the form $\big( \dirE,D(\dirE) \big)$. We also consider the hermitian sesquilinear form obtained from \(\dirE\) by polarization:
\begin{equation}
    \begin{aligned}
        \dirE\colon D(\dirE)\times D(\dirE)&\to \eC \\
        (\xi,\eta)&\mapsto \frac{1}{ 4 }\Big( \dirE[\xi+\eta]-\dirE[\xi-\eta]+i\dirE[\xi+i\eta]-i\dirE[\xi-i\eta]  \Big).
    \end{aligned}
\end{equation}
It allows to recover \(\dirE\).

\begin{proposition}		\label{PropSesquidEmbmb}
	The sesquilinear form on the algebraic tensor product $\mB\otimes\mB$ given by
	\begin{equation}
		(c\otimes d,a\otimes b)\mapsto\frac{ 1 }{2}\big( \dirE(c,abd^*)+\dirE(cdb^*,a)-\dirE(db^*,c^*a) \big)
	\end{equation}
	is definite positive.
\end{proposition}

We denote by $\hH_0$ the Hilbert space obtained after separation\cite{AlgOpGirard} and completion of $\mB\otimes\mB$ with respect to the sesquilinear form given by proposition \ref{PropSesquidEmbmb}. The inner product will be denoted by $(.|.)_{\hH_0}$ and if $a\otimes b\in\mB\otimes\mB$, we denote by $a\otimes_{\dirE}b$ the canonical image in $\hH_0$.

\begin{theorem}
	Si $\big( \dirE,D(\dirE) \big)$ is a $C^*$-Dirichlet form on $L^2(A,\tau)$, there exists a $A$-bimodule structure on $\hH_0$ characterised by
	\begin{enumerate}
		\item
			$a(b\otimes_{\dirE}c)=ab\otimes_{\dirE}c-a\otimes_{\dirE}bc$
		\item
			$(b\otimes_{\dirE}c)a=b\otimes_{\dirE}ca$
	\end{enumerate}
	for every $a,b,c$ in the Dirichlet algebra $\mB$.
\end{theorem}
