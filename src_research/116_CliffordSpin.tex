% This is part of (almost) Everything I know in mathematics
% Copyright (c) 2013-2014,2016
%   Laurent Claessens
% See the file fdl-1.3.txt for copying conditions.

\subsection{Redefinition of \texorpdfstring{$\Spin(V)$}{Spin(V)}}
%----------------------------------------------------------------

As it, this new definition only holds when $g$ is positive defined.

\begin{probleme}
	When we work with a signature $(p,q)$, maybe we only get the connected part. To be checked.
\end{probleme}

Let us take $v$, $x\in V$ with $g(v,v)=1$. We have 
\[ 
  -vxv^{-1}=-vxv=-2g(x,v)v+xv^2
		=x-2g(x,v)v\in V.
\]
The effect was to reverse the $v$ component of $x$; the map $x\mapsto -vxv^{-1}$ is $\sigma^v$. Now, when $\lambda\in U(1)$ and $w=\lambda v$, we also have that $x\mapsto -wxw^{-1}$ is $\sigma^v$. Now we look at $\chi(a)\colon x\mapsto \alpha(a)xa^{-1}$ with $a=w_{1}\ldots w_{r}$, a product of unitary vectors in $V^{\eC}$. Explicitly,
\[ 
  \chi(a)x=(-1)^{r} w_{1}\ldots w_{r}xw_{r}^{-1}\ldots w_{1}^{-1},
\]
a composition of reflexions in $V$. When $r$ is even, it is a rotation. We conclude that when $a$ is an even product of unitary vectors in $V^{\eC}$, then $\chi(a)\in \SO(V)$. Theorem \ref{CartanDieu} states that any rotation of $V$ is a composition of reflexions. So we define\nomenclature[G]{$\Spin^{c}(V)$}{A group related to $\Spin$}
\begin{equation}
\Spin^{c}(V)=\{ w_{1}\ldots w_{2k}\tq w_{j}\in V^{\eC},\,w_{j}^*w_{j}=1 \}\subset \CCliff^{0}(V),
\end{equation}
and $\chi\colon \Spin^{c}(V)\to \SO(V)$ is a surjective group homomorphism. The inverse in $\Spin^{c}(V)$ is given by
\[ 
  (w_{1}\ldots w_{2k})^{-1}=w_{2k}^*\ldots w_{1}^*=\overline{ w_{2k} }\ldots\overline{ w_{1} }.
\]
In the real case, proposition \ref{prop1001p1} says that $\ker\chi=\eR\invtible$. In the complex case we get  $\ker\chi=\eC\invtible$ and, when we look at $\ker\chi|_{\Spin^{c}(V)}$, we find
\begin{equation}
\ker\chi=U(1).
\end{equation}
Then we find the short exact sequence 
\begin{equation}
\xymatrix{%
   1 \ar[r]^-{\id}&U(1) \ar[r]^-{\id}&\Spin^{c}(V) \ar[r]^-{\chi}&\SO(V)\ar[r]^-{\id}&1.
}
\end{equation}
Let $u=w_{1}\ldots w_{2k}\in\Spin^{c}(V)$ with $w_{j}=\lambda_{j}v_{j}$ and $\lambda_{j}\in V$, so $\tau(u)=w_{2k}\ldots w_{1}$ and
\[ 
  \tau(u)u=w_{2k}\ldots w_{1}w_{1}\ldots w_{2k}
		=\lambda_{1}^{2}\ldots \lambda_{2k}^{2}\in U(1).
\]
This proves that $\tau(u)u$ is central in $\Spin^{c}(V)$. We define the homomorphism
\begin{equation}
\begin{aligned}
\nu \colon \Spin^{c}(V)&\to U(1) \\ 
u&\mapsto \tau(u)u. 
\end{aligned}
\end{equation}
This is a homomorphism because
\[ 
\begin{split}
  \nu(u_{1}u_{2})&=\tau(u_{1}u_{2})u_{1}u_{2}
		=\tau(u_{2})\underbrace{\tau(u_{1})u_{1}}_{\text{central}}u_{2}
		=\tau(u_{2})u_{2}\tau(u_{1})u_{1}\\
		&=\nu(u_{2})\nu(u_{1})
		=\nu(u_{1})\nu(u_{2}).
\end{split}  
\]
The map $\nu$ naturally restricts to $U(1)$ as
\[ 
  \nu(\lambda)=\lambda^{2}.
\]
The combined map $(\chi,\nu)\colon \Spin^{c}(V)\to \SO(V)\times U(1)$ has kernel $\{ \pm 1 \}$. We define\nomenclature[G]{$\Spin(V)$}{The spin group}
\begin{equation}  \label{eq_defSpindeux}
\Spin(V)=\ker\nu|_{\Spin^{c}(V)}.
\end{equation}

\begin{lemma}
This group is the same as the one defined in equation \eqref{defSpinun}. 
\end{lemma}

\begin{proof}
Let $u\in\Spin(V)$ (in the sense of equation \eqref{eq_defSpindeux}). The fact for $u$ to belongs to $\Spin(V)$ implies the two following:
\begin{enumerate}
\item $u\in\Spin^{c}(V)\Rightarrow u^*u=1$,
\item $u\in\ker\nu\Rightarrow \tau(u)u=1$.
\end{enumerate}
The second point says that $u^{-1}=\tau(u)$, which is a first good point to fit the first definition of $\Spin(V)$. Now we have to prove that $u\in\Gamma^{+}(V)$: $u$ must be invertible and $\chi(u)x$ must belongs to $V$ for all $x\in V$. These two points are contained in the definition of $\Spin^{c}(V)$.
\end{proof}
Let us see in the new definition how is $\chi\colon \Spin(V)\to \SO(V)$. On $\Spin^{c}(V)$, we have $\ker\chi=U(1)$, but on $\Spin(V)$ we require moreover $\tau(u)u=1$, thus an element of $\ker\chi$ in $\Spin(V)$ fulfils $\tau(\lambda)\lambda=1$, so that $\lambda=\{ \pm1 \}$. We conclude that $\ker\chi|_{\Spin(V)}=\{ \pm 1 \}$, and then that $\Spin(V)$ is a double covering of $\SO(V)$.\index{double covering!of $\SO(V)$}


\subsection{A few about Lie algebra}
%----------------------------------

\nomenclature[G]{$\spin(p,q)$}{Lie algebra of the group $\Spin(p,q)$}
\begin{proposition}
We have an isomorphism
\[ 
                    \spin(p,q)\simeq\so(p,q)
\]
between the Lie algebras of $\Spin(p,q)$ and $\SO(p,q)$.
\label{prop:spin_so}   
\end{proposition}

\begin{proof}
Using the second part of lemma \ref{Helgason5.1}, with the map $\dpt{\chi}{\Spin(p,q)}{\SO(p,q)}$, we find that $d\chi_e(\spin(p,q))=\so(p,q)$. Then we know (lemma \ref{1203r1}) that 
\[
	\so(p,q)=\spin(p,q)/\ker\,d\chi_e.
\]
On the other hand, the first part of the same lemma gives us that $\chi^{-1}(e)$ is a Lie subgroup of $\Spin(p,q)$ whose Lie algebra is $\ker\,d\chi_e$. But $\chi^{-1}(e)=\eZ_2$, so $\ker\,d\chi_e=\{0\}$.
\end{proof}

Let us now shortly speak about the Lie algebra of $\Gamma(p,q)^+$. A basis of $\Cliff(p,q)^+$ is \[\{1,\gamma_0\cdot\gamma_1,\gamma_0\cdot\gamma_1 ,\gamma_0\cdot\gamma_3
,\gamma_0\cdot\gamma_1\cdot\gamma_2\cdot\gamma_3  \}.\] Thanks to the anticommutation relations, we don't need $\gamma_1\cdot\gamma_2$ in the basis.

Remember that $\Gamma(p,q)^+$ is the set of the $x\in\Cliff^+(p,q)$ such that $x\cdot v\cdot\alpha(x^{-1})$ lies in $V$ for all $v\in V$. Let $x(t)$ be a path in $\Gamma(p,q)^+$ such that $x(0)=e$ and $\dot{x}(0)=X$. Differentiating the definition relation, we find
 \[
 \dot{x}\cdot v\cdot\alpha(x^{-1})|_0+x\cdot v\cdot(-)\alpha(\dot{x})|_0=X\cdot v-v\cdot X,
 \]
 therefore\nomenclature[G]{$\Lie{\Gamma(p,q)^+}$}{Algèbre de $\Gamma(p,q)^+$}
\[
  \Lie{\Gamma(p,q)^+}=\left\{X\in\Cliff^+(p,q)\textrm{ such that } X\cdot v-v\cdot X\in V,\,\forall v\in V\right\}.
\]

It is clear that $\eC$ is a subset of $\Lie{\Gamma(p,q)^+}$, and that $V$ is not. The following computation shows that $V\cdot V$ is a subset $\Lie{\Gamma(p,q)^+}$:
\[
         a\cdot b\cdot v-v\cdot a\cdot b=2\eta(v,a)b-2\eta(v,b)a.
\]
 We can also check that $V\cdot V\cdot V\cdot V\cap\Lie{\Gamma(p,q)^+}=\emptyset$. A basis of $\Lie{\Gamma(p,q)^+}$ is
\[
	\{ 1,e_{\alpha}\cdot e_{\beta}\tq \alpha<\beta \}
\]

 We know that $\ker[\dpt{\chi}{\Gamma(p,q)^+}{\SO(p,q)}]=\eR_0$. So the kernel of the restriction of $d\chi_e$ to $\Lie{\Gamma(p,q)^+}$ is the Lie algebra of $\eR_0$ (see lemma \ref{Helgason5.1}), which is $\eR$. Therefore, a basis of $\spin(p,q)$ is 
\[
	\{e_{\alpha}\cdot e_{\beta}\tq \alpha<\beta\}.
\]

\subsection{Grading \texorpdfstring{$\Lambda W$}{LW}}
%-------------------------------

We already know that $\Lambda W=\eC\oplus W\oplus\Lambda^2W$. This space can be written as \[\Lambda W =\Lambda W^+\oplus\Lambda W^-,\] with $\Lambda W^+=W$ and $\Lambda W^-=\eC\oplus\Lambda^2W$. The interest of such a decomposition lies in the definition of an action of $\Cliff^+(p,q)$ on $\Lambda W $. This action will be defined by $\dpt{\bullet}{\Cliff^+(p,q)\times\Lambda W }{\Lambda W }$, 
  \[
 x\bullet\alpha=\tilde\rho(x)\alpha
 \]
for any $x$ in $\Cliff^+(p,q)$ and any $\alpha$ in $\Lambda W $ (see definition \ref{defrt}).

\begin{proposition}
This action preserves the grading of $\Lambda W $:
\begin{equation}
\begin{split}
 \Cliff^+(p,q)\bullet\Lambda W^+&=\Lambda W^+\\
 \Cliff^+(p,q)\bullet\Lambda W^-&=\Lambda W^-.
\end{split}
\end{equation}

\end{proposition}
\begin{proof}
For $x\in\eC$, theses equalities are obvious. We have to check it for $x=e_i\cdot e_j$. Here, we will just check that $(e_1\cdot e_0)\bullet(v\wedge w)\in\Lambda W^+$. This follows from a simple computation:
\begin{equation}
\begin{split}
\tilde\rho(e_1)\tilde\rho(f_0+g_0)(v\wedge w)&=
                         \tilde\rho(f_1+g_1)\left[-\eta(g_0,v)w+\eta(g_0,w)v\right]\\
                    &=-\eta(g_0,v)f_1\wedge w+\eta(g_0,w)f_1\wedge v\\
                    &\quad+\eta(g_0,v)\eta(g_1,w)-\eta(g_0,w)\eta(g_1,v).
\end{split}
\end{equation}
\end{proof}

Since $\Spin(p,q)$ is a subgroup of $\Cliff^+(p,q)$, we can construct two new representation of $\Spin(p,q)$. These are $\dpt{\rho^{\pm}}{\Spin(p,q)\times\Lambda W ^{\pm}}{\Lambda W ^{\pm}}$,
\begin{equation}
\begin{split}
 \rho^-(s)w^-&=\tilde\rho(s)w^-,\\
 \rho^+(s)w^+&=\tilde\rho(s)w^+,
\end{split}
\end{equation}
for $w^{\pm}$ in $\Lambda W ^{\pm}$. This is no more than the fact that $\tilde\rho$ is reducible and that two invariant subspaces are $\Lambda W^+$ and $\Lambda W^-$.
\subsection{Clifford algebra for \texorpdfstring{$V=\eR^2$}{V=R2}}\label{cliffR2}
%----------------------------------------------

\subsubsection{General definitions}
%/////////////////////////////////

The whole construction can also be applied to $V=\eR^2$ with the Euclidean metric. This is our business now. We take the complex vector space $V^{\eC}$ and an orthonormal basis $\{e_1,e_2\}$. As before, we define
 \[
f_1=\frac{1}{2}(e_1+ie_2),\qquad g_1=\frac{1}{2}(e_1-ie_2).
\]
There are no difficulties to see that $Span(f_1)$ is a completely isotropic subspace\index{isotropic!subspace!in $\eR^{2}$} of $V^{\eC}$. Thus we define $W=\eC f_1$, $\Lambda W =\eC\oplus W$, $\Lambda W^+=\eC$, and $\Lambda W^-=W$\nomenclature{$\Lambda W^{\pm}$}{Spinor space}. The homomorphism $\dpt{\tilde\rho}{V^{\eC}}{\End(\Lambda W )}$\nomenclature{$\dpt{\tilde\rho}{(\eR^2)^{\eC}}{\End(\Lambda W )}$}{Spinor representation} in $\Lambda W $ is defined by
\begin{equation}
\begin{split}
 \tilde\rho(f_1)\alpha&=f_1\wedge\alpha,\\
 \tilde\rho(g_1)\alpha&=-i(g_1)\alpha,
\end{split}
\end{equation}
where $\alpha$ is any element of $\Lambda W $. In the basis $1=\begin{pmatrix}
1 \\
0
\end{pmatrix} $ and $f_1=\begin{pmatrix}
0 \\
1
\end{pmatrix} $, we easily find that
\[
 \tilde\rho(e_1)=\begin{pmatrix}
 0 & -\frac{1}{2} \\
 1 & 0
 \end{pmatrix},\quad\tilde\rho(e_2)=\begin{pmatrix}
 0 & -\frac{i}{2} \\
 -i & 0
 \end{pmatrix}.\]
For $c\in\eR$ we	 also have $\tilde\rho(c)f_1=cf_1$ and $\tilde\rho(c)1=c$, thus we assign the matrix $\begin{pmatrix}
c & 0 \\
0 & c
\end{pmatrix}$ to the number $c$.

As before, we define $\gamma_i=\sqrt{2}\tilde\rho(e_i)$. We immediately have $\gamma_1\gamma_2+\gamma_2\gamma_1=0$ and $\gamma_i\gamma_i=-2\mtu$, so that the $\gamma$'s satisfy the Clifford algebra for the euclidian metric.

For notational conveniences, it proves useful to make a change of basis so that we get
\begin{equation}\label{gammaR2}
\gamma_1=\begin{pmatrix}
0 & -1 \\
1 & 0
\end{pmatrix},\quad\gamma_2=-\begin{pmatrix}
0 & i \\
i & 0
\end{pmatrix}.
\end{equation}

The algebra $\Cliff(2)$\nomenclature[G]{$\Cliff(2)$}{Clifford algebra of $\eR^2$} is isomorphic to the algebra which is generated by direct sum $\Cliff(2)\simeq\eR\oplus\gamma_1\oplus\gamma_2\oplus\eR\gamma_1\gamma_2$. A general element of $\Cliff(2)$ can be written as $x\gamma_1+y\gamma_2+x'\eR+y'\gamma_1\gamma_2$. In the representation of $\tilde\rho$, a general element of $\Cliff(2)$ is therefore
\[\begin{pmatrix}
x'+iy' & x+iy \\
-x+iy & x'-iy'
\end{pmatrix},\] so that we can write the Clifford algebra of $\eR^2$ as\index{algebra!Clifford}
\[
\Cliff(2)=\left\{\begin{pmatrix}
 \alpha & \beta \\
 -\obeta & \oalpha
 \end{pmatrix}\,:\,\alpha,\beta\in\eC\right\}.
\]
The following four matrices provide a basis:
\begin{align}\label{pauli} 
1&=\begin{pmatrix}
1 & 0 \\
0 & 1
\end{pmatrix}, &i&=\begin{pmatrix}
-i & 0 \\
0 & i
\end{pmatrix},&j&=\begin{pmatrix}
0 & i \\
i & 0
\end{pmatrix},&k&=\begin{pmatrix}
0 & 1 \\
-1 & 0
\end{pmatrix}.
\end{align}
We can check that these matrices satisfies the quaternionic algebra\index{quaternion!algebra}\index{algebra!quaternion} :
\begin{equation}
\begin{split}
i^2&=j^2=k^2=-1\\
ij &=-ji=k,\\
jk &=-kj=i,\\
ki &=-ik=j.
\end{split}
\end{equation}
The algebra $\Cliff(2)=\eH$\nomenclature{$\eH$}{quaternionic algebra} is represented by $\tilde\rho$ on $\eC^2$ by the \defe{Pauli matrices}{pauli matrices} $1,i,j,k$ which are given by \eqref{pauli}.

\subsubsection{The maps \texorpdfstring{$\alpha$}{a} and \texorpdfstring{$\tau$}{t}}
%///////////////////////////////////////////////

What are the matrices which represent $V$ ? These are $\tilde\rho(e_1)$ and $\tilde\rho(e_2)$. Thus we can write $V=\Span_{\eR}\{\gamma_1,\gamma_2\}=\Span_{\eR}\{j,k\}$, or
\[
 V=\left\{\begin{pmatrix}
 0 & \xi \\
 -\oxi & 0
 \end{pmatrix}\,:\,\xi\in\eC\right\}.
\]

As before, $\alpha$ is the unique homomorphic extension to $\Cliff(2)$ of $-\id$ on $V$. From the definitions, we get $\alpha(j)=-j$, $\alpha(k)=-k$.
The extension present no difficult. For example: $\alpha(i)=\alpha(jk)=\alpha(j)\alpha(k)=jk=i$, but $\alpha(jk)=\alpha(i)$; then $\alpha(i)=i$. The same gives $\alpha(1)=1$.

The case of $\tau$ is treated in similar way. We find: $\tau(j)=j$, $\tau(k)=k$, $\tau(i)=-i$, $\tau(1)=1$.

Now, we can find the group $\gud$. The condition for $x\in\Cliff(2)$ to be in $\gud$ is $\alpha(x)yx^{-1}$ to belongs to $V$ for all $y\in V$. We put
\[ x=\begin{pmatrix}
\alpha & \beta \\
-\obeta & \oalpha
\end{pmatrix},\qquad\alpha(x)=\begin{pmatrix}
\alpha & -\beta \\
\obeta & \oalpha
\end{pmatrix}.\]
A typical $y$ in $V$ is
\[
 y=\begin{pmatrix}
 0 & \eta \\
 -\oeta & 0
 \end{pmatrix}.
\]
A few computation gives:
\[
 \alpha(x)yx^{-1}=\us{|\alpha|^2+|\beta|^2}\begin{pmatrix}
 \alpha\eta\obeta+\beta\oeta\oalpha & \alpha\alpha\eta-\beta\beta\oeta \\
 \obeta\obeta\eta-\oalpha\oalpha\eta & \eta\alpha\obeta+\oalpha\oeta\beta
 \end{pmatrix}.
\]
If we impose it to be of the form $\begin{pmatrix}
0 & \xi \\
-\oxi & 0
\end{pmatrix} $ for all $\eta\in\eC$, we get, for all $\eta\in\eC$, 
 $\real(\oalpha\beta\oeta)=0$, which implies $\oalpha\beta=0$. So we conclude:
\[
 \gud=\left\{\begin{pmatrix}
 \alpha & 0 \\
 0 & \oalpha
 \end{pmatrix}, \begin{pmatrix}
 0 & \beta \\
 -\obeta & 0
 \end{pmatrix}\,:\,\alpha,\beta\in\eC\textrm{ not both equals zero}\right\}.
\]
Be careful on a point: $\gud$ is the \emph{multiplicative} group generated by these two matrices, not the additive one.

\subsubsection{The spin group}
%////////////////////////////

It present no difficult to find that
\begin{equation}
 \gud^+=\left\{\begin{pmatrix}
 \alpha & 0 \\
 0 & \oalpha
 \end{pmatrix}\,:\,\alpha\neq 0\right\}.
\end{equation}
The \defe{spin group}{spin!group!on $\protect\eR^2$} is made of elements of $\gud^+$ which satisfy $\tau(x)=x^{-1}$. We know that
$\tau\begin{pmatrix}
\alpha & 0 \\
0 & \oalpha
\end{pmatrix} =\begin{pmatrix}
\oalpha & 0 \\
0 & \alpha
\end{pmatrix}$ and that $\begin{pmatrix}
\alpha & 0 \\
0 & \oalpha
\end{pmatrix}^{-1} =\us{\displaystyle\alpha\oalpha}\begin{pmatrix}
\oalpha & 0 \\
0 & \alpha
\end{pmatrix}$. Thus the condition \hbox{$\tau(x)=x^{-1}$} becomes $|\alpha|^2=1$. The first conclusion is that
\begin{equation}
                    \Spin(2)=U(1).
\end{equation}
A typical $s$ in $\Spin(2)$ is
\[s=e^{i\theta}=\begin{pmatrix}
e^{i\theta} & 0 \\
0 & e^{-i\theta}
\end{pmatrix}.\]

The next point is to see the action of $\Spin(2)$ on $V$.\index{action!of $\Spin(2)$ on $\eR^2$} The action of $s\in\Spin(2)$ on a vector $v\in V$ is still defined by $s\bullet v=\chi(s)v=\alpha(s)\cdot v\cdot s^{-1}$. More explicitly:
\begin{equation}
 \chi(s)v=\begin{pmatrix}
 e^{i\theta} & 0 \\
 0 & e^{-i\theta}
 \end{pmatrix} \begin{pmatrix}
 0 & z \\
 -\overline{z} & 0
 \end{pmatrix} \begin{pmatrix}
 e^{-i\theta} & 0 \\
 0 & e^{i\theta}
 \end{pmatrix}=\begin{pmatrix}
 0 & e^{2i\theta}z  \\
 -e^{-2i\theta}\overline{z} & 0
 \end{pmatrix},
\end{equation}
where the  matrix $\begin{pmatrix}
0 & z \\
\overline{z} & 0
\end{pmatrix} $ denotes the representation of the vector $v$ of $V$. This equality can be written $e^{i\theta}\cdot v=e^{2i\theta}v$. If we note $v=v_1+iv_2=\begin{pmatrix}
v_1 \\
v_2
\end{pmatrix} $, we get
\[ e^{2i\theta}\bullet v=\begin{pmatrix}
\cos 2\theta & -\sin 2\theta \\
\sin 2\theta & \cos 2\theta
\end{pmatrix}\begin{pmatrix}
v_1 \\
v_2
\end{pmatrix}. \]
Therefore, we can write
\[\chi(e^{i\theta})=\begin{pmatrix}
\cos 2\theta & -\sin 2\theta \\
\sin 2\theta & \cos 2\theta
\end{pmatrix}.\]

So $\chi$ projects $U(1)$ into $\SO(2)$ with a kernel $\eZ_2$, for this reason, we say that $U(1)$ is a \defe{double covering}{double covering!of $\SO(2)$} of $\SO(2)$. We note it
\begin{equation}
            \eZ_2\rightarrow U(1)\stackrel{\chi}{\rightarrow}\SO(2).
\end{equation}

\section{Clifford modules}  \label{susec_Cliffmodule}\index{Clifford!module}
%---------------------------

References: \cite{ResEtaDiracType,mellor}.

Let $M$ be a manifold. We denote by $\CCliff(M)$ the bundle whose fibre at $x\in M$ is the complex Clifford algebra of the metric $g_x$ : $\CCliff(M)_x=\CCliff(g_x)$. We define the important map
\begin{equation}
\begin{aligned}
 \gamma\colon \Gamma(M,\CCliff(M))&\to \oB(\hH) \\ 
\gamma(dx^{\mu})&\mapsto \gamma^{\mu}(x)  
\end{aligned}
\end{equation}
which can be extended to the whole Clifford algebra.

Let $V$ be a vector space endowed with a bilinear symmetric form. We consider $\Cliff(V)$, the corresponding Clifford algebra. A \defe{Clifford module}{Clifford!module} is a real vector space $E$ with a $\eZ_2$-graduation and a morphism 
\[ 
  \rho_E\colon \Cliff(V)\to \End(E)
\]
of $\eZ_2$-graded vector spaces. It is defined by a linear map $\rho_E\colon V\to \End(V)$ such that
\begin{equation}
\rho_E(v)\rho_E(w)+\rho_E(w)\rho_E(v)=B(v,w)\id
\end{equation}
for every $v$, $w\in E$. The element $\rho_E(x)v$ will often be denoted by $x\cdot v$ and the operation $\rho_E$ is the \defe{Clifford multiplication}{Clifford!multiplication}. The \defe{dual module}{dual module} $E^*$ is defined by $\rho_{E^*}(x)=\rho_E(x^t)^*$, i.e.
\begin{equation}
\langle \rho_{E^*}(x)\psi,v \rangle =(-1)^{| \psi | |x |}\langle \psi, \rho_E\big( \tau(x) \big)v\rangle 
\end{equation}
for every $\psi\in E^*$ and $v\in E$. Here

Let $\cA$ be a $\eZ_2$-graded subalgebra of $\Cliff(V)$ and $E_1$, a $\cA$-module. Then the space\nomenclature{$\Ind_{\cA}^{\Cliff(V)}(E_1)$}{Induced Clifford module}
\[ 
  E=\Ind_{\cA}^{\Cliff(V)}(E_1)=\Cliff(V)\otimes_{\cA}E_1
\]
has a structure of Clifford module, the \defe{induced module}{induced!Clifford module}. The tensor product $\otimes_{\cA}$ is the usual one modulo the subspace spanned by elements of the form 
\[ 
  x\otimes a\cdot y-xa\otimes y
\]
for every $x$, $a\in\Cliff(V)$ and $y\in E_1$. In a similar way, if $E$ is a complex vector space we have a notion of $\CCliff(V)$-module. 

Let $x\in\Cliff(V)$ be such that $x^2=1$. In that case the Clifford multiplication $\rho_E(x)$ decomposes $E$ in eigenspaces
\[ 
  E^{\pm}=\frac{ 1 }{2}\big( 1\pm\rho_E(x) \big)E.
\]

If $V$ is a $n$-dimensional vector space with an oriented orthonormal basis $\{ e_1,\ldots, e_n \}$, the algebra $\Cliff(V)$ has a \defe{volume element}{volume!element} $\omega=e_1e_2\ldots e_n$ which does not depend on the choice of the basis. The volume element squares to
\begin{equation}
\omega^2=(-1)^{n(n+1)/2}.
\end{equation}
In the complex case, we consider the complex vector space $V^{\eC}$ and the complex Clifford algebra $\CCliff(V)=\Cliff(V)\otimes_{\eR}\eC$, and the volume element is defined as
\begin{equation}
\omega_{\eC}=i^{[ (n+1)/2 ]}\omega.
\end{equation}
where $[x]$ is denotes the integer part of $x$. Performing a separate computation for $n$ even or odd, it is easy to see that in both case,
\begin{equation}
\omega_{\eC}^2=1.
\end{equation}
So in the complex case we always have an element in $\Cliff(V)$ which squares to $1$, and a $\CCliff(V)$-module $W$ always accepts a decomposition as $W^{\pm}=\frac{ 1 }{2}(1+\omega_{\eC})W$.

One says that a representation\index{representation!of Clifford algebra} $\rho$ of $\Cliff(V)$ on $W$ is \defe{reducible}{reducible!representation of Clifford} if there exists a splitting $W=W_1\oplus W_2$ such that $\rho(\Cliff(V))W_i\subset W_i$. If the representation is not reducible, it is said to be irreducible. Two representations $\rho_j\colon \Cliff(V)\to \End(W_j)$ are \defe{equivalent}{equivalence!representation of Clifford} if there exists a linear isomorphism $F\colon W_1\to W_2$ such that $F\circ\rho_1(x)\circ F^{-1}=\rho_2(x)$ for every $x\in\Cliff(V)$.

\begin{proposition}
The real Clifford algebra has
\[ 
 \begin{cases}
2&\text{if }n+1=0\mod 4\\
1&\text{otherwise}
\end{cases} 
\]
inequivalent irreducible representations. The complex Clifford algebra $\CCliff(V)$ has
\[ 
 \begin{cases}
     2&\text{if }n \text{ is odd}\\
     1&\text{if }n \text{is even}
\end{cases} 
\]
inequivalent irreducible representations. 
\end{proposition}
\begin{proof}
No proof.
\end{proof}


If $M$ is a manifold, we denote by $\Cliff(M)=\Cliff(TM)$ the bundle whose fiber at $x$ is the Clifford algebras of $T_xM$. We consider an orthonormal basis $\{ e_i \}$ and if $\Sigma$ is a multi-index $\{ 1\leq\sigma_1,\ldots,\leq\sigma_t\leq m \}$, we pose $e_{\Sigma}=e_{\sigma_1}\ldots e_{\sigma_t}\in\Cliff(M)$. By convention, $e_{\emptyset}=1$. Since the elements $e_i$ are ordered, they provide an orientation:
\begin{equation}
d\vol=e_1\wedge\ldots\wedge e_m\in\Wedge^m(M).
\end{equation}
Since the map $e_{\sigma_1}\wedge\ldots\wedge e_{\sigma_t}\mapsto e_{\sigma_1\ldots e_{\sigma_t}}$ is an isomorphism between $\Cliff(M)$ and $\Wedge(M)$, we say that $d\vol\in\Cliff(M)$. Now we define 
\[ 
  \kappa=i^{-[(m+1)/2]}d\vol,
\]
which is nothing else that the volume form normalised in such a way that $\kappa^2=1$. If $m$ is even, it anti-commutes with $TM$, and if $m$ is odd, it commutes with $TM$.

Let $V$ be a $m$-dimensional real vector space, and $\CCliff(V)$, the corresponding complex Clifford algebra.
\begin{lemma}
Every $\CCliff(V)$-module accepts an unique decomposition as sum of irreducible representations as follows
\begin{enumerate}
\item if $m=2n$, there exists one and only one irreducible $\CCliff(V)$-module $\Delta$ and $\dim(\Delta)=2n$,
\item if $m=2n+1$, we have two inequivalent irreducible modules $\Delta_{\pm}$ with $\gamma(\kappa)=\pm 1$ on $\Delta_{\pm}$ and $\dim(\Delta_{\pm})=2^n$.
\end{enumerate}
\end{lemma}
\begin{proof}
No proof.
\end{proof}

Let $V$ be a vector bundle over $M$. A structure of $\Cliff(M)$-module over $V$ is a morphism of unital algebra $\gamma\colon \Cliff(M)\to \End(V)$. When one has a basis $\{ e_i \}$ of $V$, we pose $\gamma_i=\gamma(e_i)$. The following lemma is the lemma 1.2 of \cite{ResEtaDiracType}. 
\begin{lemma}			\label{LemGammaBaseConstant}
Let $V$ be a $\Cliff(V)$-module and $\{ e_i \}$, an orthonormal basis for $TM$ on a contractible open set $V$. Then there exists a local frame for $V$ such that the matrices $\gamma(e_i)$ are constant.
\end{lemma}
We also define $\gamma^i=\gamma(dx^i)=g^{ij}\gamma_j$. One easily proves that
\begin{equation}
\gamma^i\gamma^j+\gamma^j\gamma^i=-2g^{ij}
\end{equation}
where $(g^{ij})$ is the inverse matrix of $(g_{ij})$. If the endomorphisms $\gamma_i$ are constant in the basis $\{ e_i \}$, then the endomorphisms $\gamma^i$ are constant in the basis $\{ f_i=g_{ki}e_k \}$.



\section{Spin structure}	\label{sec:spin_str}
%++++++++++++++++++++++++

We consider a (pseudo-)Riemannian manifold $(M,g)$ with metric signature $(p,q)$, and $\SO(M)$, its frame bundle; it admits a $\SO(p,q)$-principal fiber bundle structure which is well defined by the metric $g$ (see \ref{subsubsecframebundle}).

\begin{definition}
We say that $(M,g)$ is a  \defe{spin manifold}{spin!manifold} if there exists a $\Sppq$-principal bundle $P$ over $M$ and a principal bundle homomorphism $\dpt{\varphi}{P}{\SO(M)}$ which induced covering for the structure groups is $\chi$, i.e.
$\varphi(\xi\cdot s)=\varphi(\xi)\cdot\chi(s)$. A choice of $P$ and $\varphi$ is a \defe{spin structure}{spin!structure} on $M$.
\label{defvarspin}
\end{definition}
\[
\xymatrix{ \Sppq \ar@{~>}[r]	& P \ar[rr]^-{\displaystyle\varphi} \ar[rd]_{\displaystyle\pi} && \SO(M) \ar[ld]^{\displaystyle p}&\SO(p,q) \ar@{~>}[l]  \\& & M }
\]
The wavy arrows mean ``structural group of''.

\begin{remark}
When we will use the concept of spin structure in the physical oriented chapters, we will naturally use $\SLdc$ as group instead of $\Sppq$. The isomorphism $\SLdc\simeq\Sput$ is proved in \cite{Michelson}. A physical motivation of such a structure is given at page \pageref{pg_spinenphyz}.
\end{remark}

\subsection{Example: spin structure on the sphere \texorpdfstring{$S^2$}{S2}}
%----------------------------------------------------------------

It is no difficult to see that $\SO(S^2)\simeq \SO(3)$. Indeed, each element of $\SO(S^2)$ is described by three orthonormal vectors: one which point to an element $x$ of $S^2$ and two which gives a basis of $T_xS^2$. The action $\SO(3)\times S^2\to S^2$ is transitive, and the stabilizer of any element is $\SO(2)$.

We define $\dpt{\alpha}{\SO(3)/\SO(2)}{S^2}$ by $\alpha(g\SO(2))=g$. Proposition \ref{propHelgason4.3} shows that $\alpha$ is a diffeomorphism. Then
\[
                S^2=\frac{\SO(3)}{\SO(2)}.
\]

On the other hand,  we know that
\begin{eqnarray}\label{explsu2} T_eSU(2)=su(2)=\left\{\begin{pmatrix}
ix & \xi \\
-\oxi & -ix
\end{pmatrix}\,:\,\xi\in\eC,x\in\eR\right\}.
\end{eqnarray}
 It is a classical result that $\mathfrak{su}(2)\simeq\eR^3$ not only as set but also as metric space with the identification
\[
\langle X,Y\rangle=-\frac{1}{2}\tr(XY),
\]
 for all $X$, $Y\in su(2)$. As we are in matrix groups, we know (see \cite{Lie} to get more details) that $Ad_xY=xYx^{-1}$. In our case, this gives the formula
\[
                \langle Ad(g)X,Ad(g)Y\rangle=\langle X,Y\rangle.
\]
We can now state the result for $S^2$.

 \begin{proposition}
The manifold $S^2$ with the usual metric induced from $\eR^3$ admits the following spin structure:
\begin{eqnarray}\label{spins2}
\xymatrix{ \Spin(2)\ar@{~>}[r] &SU(2) \ar[rr]^{\displaystyle \varphi=Ad} \ar[rd]_{\displaystyle U(1)}^{\displaystyle\pi} && \SO(3) \ar[ld]^{\displaystyle \SO(2)}_{\displaystyle p} \\& & S^2 },
\end{eqnarray} where the arrow
$\xymatrix{X \ar[r]^{f}_G & Y }$ means that $G$ is the kernel of the map $\dpt{f}{X}{Y}$.
\end{proposition}
\begin{proof}
 First, let us precise the concept of frame bundle for $S^2$, and how it is well described by $\SO(3)$. Let $\{e_1,e_2,e_3\}$ be the canonical basis of $\eR^3$. To $A\in \SO(3)$, we make correspond the basis $\{Ae_2,Ae_3\}$ at the point $Ae_1$ of $S^2$. The projection $\dpt{p}{\SO(3)}{S^2}$ is then defined by $p(A)=Ae_1$. It is clear that we will  define the map $\dpt{\pi}{SU(2)}{S^2}$ in the same way: $\pi(U)=p(Ad(U))$.

For the rest of the demonstration, we will use the ``$su(2)$ description''\ of $\eR^3$ given by \eqref{explsu2} with $\xi=y+iz$.

Now, let us show that $\dpt{\pi}{SU(2)}{S^2}$ is a $\Spin(2)$-principal bundle. Since we had already shown that $\Spin(2)\simeq U(1)$, we define the right action of $\Spin(2)$ on $SU(2)$ by right multiplication: $U\cdot s=Us$ with $s=\begin{pmatrix}
e^{i\theta} & 0 \\
0 & e^{-i\theta}
\end{pmatrix}$. It is clear that $\pi(Us)=\pi(U)$:
\begin{equation}
 Ad(Us)e_1=(Us)\begin{pmatrix}
 1 \\
 0 \\
 0
 \end{pmatrix}s^{-1} U^{-1}=Us
 \begin{pmatrix}
 i&0\\
 0&-i
 \end{pmatrix}s^{-1} U^{-1},
\end{equation}
because $\begin{pmatrix}
i&0\\
0&-i
\end{pmatrix}$ is the vector $e_1$ in the ``$su(2)$ description''\ of $\eR^3$.

In order for $\dpt{\pi}{SU(2)}{S^2}$ to be a $\Spin(2)$-principal bundle, we still need to show that for all $x\in S^2$,
\[
   \pi^{-1}(x)=\left\{\xi\cdot g\tq g\in\Spin(2)\,\forall\xi\in\pi^{-1}(x)\right\}.
\]
Take $A$, $B\in\pi^{-1}(x)$, i.e. $Ae_1=Be_1=x$. We need to find a $s\in\Spin(2)$ such that
\begin{eqnarray}
 \label{1603r3} A=B\cdot s.
\end{eqnarray}
The matrices $A$ and $B$ are such that
\begin{eqnarray}\label{1603r1}
 B^{-1} A\begin{pmatrix}
 i&0\\
 0&-i
         \end{pmatrix}A^{-1} B=\begin{pmatrix}
 i&0\\
 0&-i
         \end{pmatrix}.
\end{eqnarray}
This implies that $B^{-1} A\in\Spin(2)$. As $Ad$ is surjective from $SU(2)$ into $\SO(3)$, a general $C$ in $\SO(3)$ which acts on $e_1$ can be written $Ue_1U^{-1}$ for $U\in SU(2)$ such that $Ad(U)=C$. The condition \eqref{1603r1} becomes
\[
\begin{pmatrix}
\alpha&\beta\\
-\obeta&\oalpha
\end{pmatrix}
\begin{pmatrix}
i&0\\
0&-i
\end{pmatrix}
\begin{pmatrix}
\oalpha&-\beta\\
\obeta&\alpha
\end{pmatrix}=
\begin{pmatrix}
i&0\\
0&-i
\end{pmatrix},
\]
which implies $\alpha=e^{i\theta}$, $\beta=0$. Then $B^{-1} A$ belongs to $\Spin(2)$, and $s=B^{-1} A$ fulfills the condition \eqref{1603r3}.

What about the induced covering for the structural groups ? The structural group of $\dpt{\pi}{SU(2)}{S^2}$ is $\Spin(2)$, while the one of $\dpt{p}{\SO(3)}{S^2}$ is $\SO(2)$. Indeed, for each $x\in S^2$, $\SO(2)$ acts on $T_xS^2$, leaving $x$ unchanged. We have the following associations:
\[
         U\in SU(2)\stackrel{\varphi}{\longrightarrow}A\in \SO(3),
\]
the matrix $A$ being defined by $A\cdot X=UXU^{-1}$. For $s\in\Spin(2)$ we of course also have
\[
         Us\in SU(2)\stackrel{\varphi}{\longrightarrow}As\in \SO(3),
\]
with $As\cdot X=UsXs^{-1} U^{-1}$. As we act by $\Spin(2)$ on $SU(2)$, in the fibres of $\SO(3)$, the action of $\Spin(2)$ is --via $\varphi$-- the composition with $X\to sXs^{-1}$. But this is exactly $\chi(s)X$ because $\alpha(s)=s$, since $s\in\Spin(2)$.
\end{proof}

\subsection{Spinor bundle}
%--------------------------

Let us take once again the spin structure on the (pseudo-)Riemannian manifold $(M,g)$:
\[
  \xymatrix{ \Sppq \ar@{~>}[r]& P \ar[rr]^-{\displaystyle\varphi}
   \ar[rd]_{\displaystyle\pi} && \SO(M) \ar[ld]^{\displaystyle p}&\SO(p,q) \ar@{~>}[l]
   \\& &   M }
\]
with $\varphi(\xi\cdot g)=\varphi(\xi)\cdot\chi(g)$.

Let us define $S=\Lambda W $, and $\mS=P\times_{\rho}S$. Take $\dpt{\rho}{\Sppq\times\mS}{\mS}$, $\rho(g,s)=\tilde\rho(g)s$, where $\tilde\rho$ is the spinor representation of $\Sppq$ on $S$. We also have
$\dpt{\chi}{\Sppq}{\SO_0(p,q)}$, $\chi(g)v=\alpha(g)\cdot v\cdot g^{-1}$, with $\alpha(g)=g$ for $g\in\Sppq$.

The \defe{spinor bundle}{spinor!bundle} is the associated bundle
\begin{equation}
                   \mS=P\times_{\rho}S\to M
\end{equation}
A \defe{spinor field}{spinor!field} is an element of $\Gamma(\mS)$, the space of section of the spinor bundle.

On $\SO(M)$, we look at a connection $1$-form $\alpha\in\Omega^1(\SO(M),so(\eR^m))$,
and, if $T(M)$ is the tensor bundle over $M$, we define a covariant derivative $\dpt{\nabla^{\alpha}}{\cvec(M)\times T(M)}{T(M)}$ by
 \[
             \widehat{\nabla^{\alpha}_X s}=\overline{X}\hat{s},
\]
 for any $s\in T(M)$. See theorem \ref{tho_nablaE}, and the fact that $T(M)$ can be see as an associated bundle; it is explicitly done for $\cvec(M)$ at page \pageref{equivvec}.

As seen in \ref{subsection_levi}, an automatic property of this connection is $\nabla^{\alpha} g=0$ if $g$ is the metric of $M$. The \defe{Levi-Civita connection}{connection!Levi-Civita}\index{connection!Levi-Civita} is the unique\footnote{We will not prove unicity.} such connection which is torsion-free: $T^{\nabla^{\alpha}}=0$.


\begin{proposition}
The $1$-form $\talpha=\varphi^*\alpha\in\Omega^1(P,so(\eR^{m}))$ defines a connection on $P$. See definition \ref{defconnform} and theorem \ref{tho_nablaE}.
\end{proposition}

\begin{proof}
Let us denote by $R_g$ the right action of $g\in\Sppq$ on $P$ (\emph{id est} $R_g\xi=\xi\cdot g$), and by $R_u^{\SO(M)}$ the right action of $u\in\Sopq$ on $\SO(M)$.
We  have to check the usual two conditions of a connection.

\subdem{First condition}
The first one is:
\[
            (R_g^*\talpha)_{\xi}(\Sigma)=Ad(g^{-1})(\talpha_{\xi}(\Sigma)),
\]
for all $\xi\in P$, and $\Sigma\in T\bxi P$. In order to check this, we first remark that $\varphi\circ R_g=R_{\chi(g)}^{\SO(M)}\circ\varphi$. Indeed, for all $\xi\in P$, definition \ref{defvarspin} gives us $\varphi(R_g\xi)=\varphi(\xi\cdot g)=\varphi(\xi)\cdot\chi(g)$.  With this, we can make the following computation:
\begin{equation}\label{1603r4}
\begin{aligned}
R_g^*\talpha&=R_g^*\varphi^*\alpha=(\varphi\circ   R_g)^*\alpha	=(R_{\chi(g)}^{\SO(M)}\circ\varphi)^*\alpha\\
            &=\varphi^*R_{\chi(g)}^{\SO(M)*}\alpha=\varphi^*(Ad(\chi(g)^{-1})\circ\alpha).
\end{aligned}
\end{equation}
The last equality comes from the fact that $\alpha$ is a connection $1$-form. As we are in matrix groups, we have $Ad(g)x=gxg^{-1}$, so
\begin{equation}
   [Ad(\chi(g))x]v=[\chi(g) x \chi(g)^{-1}]v
                  =\chi(g)[xg^{-1} vg]
                  =gxg^{-1}.
\end{equation}
In the first line, the product is the usual matrix product which can be seen as operator composition.

But $(Ad(g)x)v=gxg^{-1} v$. Then $Ad(g)=Ad(\chi(g))$, if we identify $\sppq\simeq\sopq$ by proposition \ref{prop:spin_so}. Moreover, the action of $Ad$ is linear, so it commutes with $\varphi^*$. With these remarks, we can continue the computation \eqref{1603r4}:
\begin{equation}
 \varphi^*(Ad(\chi(g)^{-1})\circ\alpha)=\varphi^*(Ad(g^{-1})\circ\alpha)
                                  =Ad(g^{-1})\circ\varphi^*\alpha
                                  =Ad(g^{-1})\circ\talpha.
\end{equation}
This proves the first condition.

\subdem{Second condition}
The second one is $\talpha(A^*\bxi)=-A$ with the definition \eqref{defastar}. This is also a computation. First remark
\[
 \talpha\bxi(A\bxi^*)=(\varphi^*\alpha)\bxi(A^*\bxi)=\alpha_{\varphi(\xi)}(\varphi_{*\xi}A^*\bxi).
\]
 We compute $\varphi_{*\xi}A^*$ with lemma \ref{lemsur5d}:
\begin{equation}
\begin{split}
 \varphi_{*\xi}A^*&=\dsdd{\varphi(\xi\cdot\exp -tA)}{t}{0} =\dsdd{(R_{\chi(\exp -tA)}^{\SO(M)}\circ\varphi)(\xi)}{t}{0}\\
              &=\dsdd{\varphi(\xi)\cdot\chi(\exp -tA)}{t}{0}=\dsdd{\varphi(\xi)\cdot\exp(-td\chi_eA)}{t}{0}=(d\chi_eA)^*_{\varphi(\xi)}.
\end{split}
\end{equation}
But $d\chi_e=\id_{so(p,q)}$, thus $\varphi_{*\xi}A^*=A^*_{\varphi(\xi)}$. The whole makes that:
\[
\talpha\bxi(A^*\bxi)=\alpha_{\varphi(\xi)}(\varphi_{*\xi}A^*\bxi)=\alpha_{\varphi(\xi)}(A^*_{\varphi(\xi)})=-A.
\] 
This completes the proof.
\end{proof}

\begin{definition}
This connection $1$-form on $P$ is called the \defe{spinor connection}{spinor!connection}. It gives us a covariant derivative on any associated bundle and in particular on the spinor bundle, $\dpt{\tnab}{\cvec(M)\times\Gamma(\mS)}{\Gamma(\mS)}$.
 \label{spinconn}
\end{definition}
\nomenclature[D]{$\dpt{\tnab}{\cvec(M)\times\Gamma(\mS)}{\Gamma(\mS)}$}{Covariant derivative for the spinor connection}

\begin{proposition}
If $X$, $Y\in\cvec(M)$ are such that $X_x=Y_x$, then for all $s\in\Gamma(\mS)$,
\[
              (\tnab_Xs)(x)=(\tnab_Ys)(x).
\]
 \label{2303p1}
\end{proposition}
\begin{proof}
We just have to show that for all vector field $Z$ such that $Z_x=0$, $(\tnab_Zs)(x)=0$. Such a $Z$ can be written as $Z=fZ'$ for a function $f$ on $M$ which satisfies $f(x)=0$. We have:
\[
\tnab_Zs=\tnab_{fZ'}s=f\tnab_{Z'}s,
\]
which is obviously zero at $x$.
\end{proof}

Let $x\in M$ and $\{e_{\alpha\,x}\}$ be an orthonormal basis of $T_xM$. We can extend it to $\{e_{\alpha}\}$, a local basis field around $x$ such that $e_{\alpha}$ is a section of the frame bundle (in other words, we ask the extension to be smooth). The claim of proposition \ref{2303p1} is that $\tnab_{e_{\alpha}}(x)$ is an element of $\mS_x$ which doesn't depend on the extension.

\section[Dirac operator]{Dirac operator\protect\quad{\Huge\Smiley}}		\label{applgamma}
%---------------------------------------------------------------------

\subsection{Preliminary definition}
%----------------------------------

Let $M$ be a $m$-dimensional (pseudo)Riemannian manifold with its spin structure 
\[
\xymatrix{ \Sppq \ar@{~>}[r]& P \ar[rr]^{\displaystyle\varphi} \ar[rd]_{\displaystyle\pi} && \SO(M) \ar[ld]^{\displaystyle p}&\SO(p,q) \ar@{~>}[l]  \\& & M }
\]
where $\varphi$  satisfies $\varphi(\xi\cdot g)=\varphi(\xi)\cdot\chi(g)$.

Recall that for any vector space, one can see $\End{V}=V^*\otimes V$ with the definition $(v^*\otimes v)w=(v^*w)v$. This allows us to define an action of $\Sppq$ on $\End{S}$ by defining an action of $\Sppq$ on $S$ and $S^*$ separately. We know the action 
\begin{equation}
\begin{aligned}
 \Spin(p,q)\times S&\to S \\ 
(g,v)&\mapsto \tilde\rho(g)v,
\end{aligned}
\end{equation}
and as action on $S^*$, we take the dual one
\begin{equation}
\begin{aligned}
 \Spin(p,q)\times S^*&\to S^* \\ 
 g\cdot\alpha&= \alpha\circ\tilde\rho(g^{-1}) 
\end{aligned}
\end{equation}
for all $g\in\Spin(p,q)$ and $\alpha\in S^*$.

Now we can make the following computation with $g\in\Sppq$, $\alpha\in S^*$ and $v\in S$, using the fact that $\tilde\rho$ is linear:
\begin{equation}
\begin{split}
[g\cdot(\alpha\otimes v)]w&=[(\alpha\circ\tilde\rho(g^{-1}))w]\tilde\rho(g)v\\
                          &=\tilde\rho\left([(\alpha\circ\tilde\rho(g^{-1}))w]g\right)v\\
                          &=\big[\tilde\rho(g)\circ(\alpha\otimes v)\circ\tilde\rho(g^{-1})\big]w.
\end{split}
\end{equation}
Then we write the action of $\Sppq$ on $\End{S}$\index{action!of $\Sppq$ on $\End{S}$} by ($A\in\End S$)
\begin{equation}
     g\cdot A=\tilde\rho(g)\circ A\circ\tilde\rho(g^{-1}).                          \label{actspin}
\end{equation}
Notice that this definition is the one required in condition \eqref{equivA}.

The tangent bundle $T_xM$ is given with a metric $g_x$. As usual, we build $S_x=\Lambda W _x$, a completely isotropic subspace of $T_xM$ with respect to the metric $g_x$, and a representation
\[ 
\tilde \rho_x\colon T_xM\to \End(\Lambda W_x)
\]
The first step in the definition of $\gamma(X)$ is to build $\dpt{\ha_X}{P}{\End(\Lambda W )}$ setting\footnote{See subsection \ref{equivvec} for the definition of $\hX$.} $\ha_X(p)=\tilde\rho(\hX_{\varphi(p)})$.

\begin{lemma}
The function $\hat a$ is equivariant, i.e. it satisfies
\begin{equation}
     \ha_X(p\cdot g)=g^{-1}\cdot\ha_X(p)                             \label{equivaX}
\end{equation}
for all $g\in\Sppq$.
\end{lemma}

\begin{proof}
It is no more than a simple computation using the equivariance of $\hX$. Indeed:
\begin{equation}
\begin{split}
 \ha_X(p\cdot g)&=\tilde\rho(\hX_{\varphi(p\cdot g)})=\tilde\rho(\hX_{\varphi(p)\chi(g)})=\tilde\rho(\chi(g^{-1})\cdot\hX_{\varphi(p)})\\
		&=\tilde\rho(g^{-1}\cdot\hX_{\varphi(p)}\cdot g)=\tilde\rho(g^{-1})\circ\tilde\rho(\hX_{\varphi(p)})\circ\tilde\rho(g)\\
                &=g^{-1}\cdot\ha_X(p).
\end{split}
\end{equation}
In the fourth line, the dots mean the Clifford product, and the last equality comes from the definition of the action \eqref{actspin} of $\Sppq$ on $\End{S}$.
\end{proof}

From the discussion of section \ref{sec_fnequiv}, the function $\dpt{\ha_X}{P}{\End{S}}$ defines a section $\dpt{a_X}{M}{\End{\mS}}$. We define $\dpt{\gamma}{\cvec(M)}{\End{\Gamma(\mS)}}$ by
\nomenclature[D]{$\dpt{\gamma}{\cvec(M)}{\End{\Gamma(\mS)}}$}{A key ingredient for Dirac operator}
\begin{equation}		\label{EqDefgammax}
                       \gamma(X)=a_X.
\end{equation}
We immediately have
\[
                     \widehat{\gamma(X)}(p)=\tilde\rho(\hX_{\varphi(p)})
\]
for any $p\in P$. If we define
\begin{equation}\label{3103r1}
  \widehat{\gamma\cdot a_X}(p)=\widehat{\gamma(X)}(p),
\end{equation}
the map $\gamma$ can be seen as an action on the section of $\mS$. Indeed, $\widehat{\gamma\cdot s}_X$ is an equivariant function:
\begin{equation}
\begin{split}
 \hat{\gamma}(p\cdot g)(\ha_X(p\cdot g))
                  &=\rho(g)^{-1}\hat{\gamma}(p)\rho(g)\rho(g^{-1})\ha_X(p)\\
                  &=\rho(g)^{-1}\hat{\gamma}(p)\ha_X(p)\\
                  &=\rho(g^{-1})\widehat{\gamma\cdot a_X}(p),
\end{split}
\end{equation}
 so that
\[
    \widehat{\gamma\cdot a_X}(p)=\rho(g^{-1})\widehat{\gamma\cdot a_X}(p).
\]

The map $\dpt{\widehat{\gamma\cdot a_X}}{P}{\End{\Lambda W }}$ defined by \eqref{3103r1} is equivariant, and thus defines a section
$\gamma\cdot a_X\in\Gamma(\mS)$, as seen in the section \ref{sec_fnequiv}.

\subsection{Definition of Dirac}
%-------------------------------

If we consider a basis $\{e_{\alpha}\}$ of $TM$, \emph{i.e.} $m$ sections $\dpt{e_{\alpha}}{M}{TM}$ such that for all $x$ in $M$, the set $\{e_{\alpha x}\}$ is a basis of $T_xM$, we note $\gamma^{\alpha}:=\gamma(e_{\alpha})\in\End(\mS)$.

\begin{remark}  
This is not always globally possible. The example of the sphere is given in subsection \ref{subsec_DimofModule}. 
\label{rem_secnoglobal}
\end{remark}

For any $s\in\Gamma(\mS)$, we consider the local\footnote{Extensions of $e_{\alpha}$ do not always globally exist, see remark \ref{rem_secnoglobal}.} section $\psi$ of $\mS$ given by
\[
    \psi(x)=
   \sum_{\alpha\beta}g_x(e_{\alpha},e_{\beta})\gamma_x\hbeta(\tnab_{e_{\alpha}}s)(x).
\]

For each $x\in M$, take a $A_x$ in\footnote{By $\SO(g_x)$, we mean the set of all the matrix $A$ such that $A^tg_xA=g$; $A_x$ is an isometry of $(T_xM,g_x)$. In other words, we consider $A$ as a section of what we could call the ``isometry bundle''.} $\SO(g_x)$, and consider the new basis $e'_{\alpha}=A_{\alpha}^{\phantom{\alpha}\beta}e_{\beta}$. As $A$ is an isometry, $g_x(e'_{\alpha},e'_{\beta})=g_x(e_{\alpha},e_{\beta})$; and since $\tilde\rho$ is linear, $\gamma_x'^{\alpha}=\tilde\rho_x(e'_{\alpha x})=A_{\alpha}^{\phantom{\alpha}\beta}\tilde\rho(e_{\beta x})=A_{\alpha}^{\phantom{\alpha}\beta}\gamma_x\hbeta$. In the new basis, the section reads:
\begin{equation}
\begin{split}
   \psi(x)&=\sum_{\alpha\beta\eta\sigma}g_x(e_{\alpha},e_{\beta})
                A_{\beta}^{\phantom{\beta}\sigma}\gamma_x^{\sigma}
                (\tnab_{A_{\alpha}^{\phantom{\alpha}\eta}e_{\eta}}s)(x)\\
          &=\sum_{\alpha\beta\eta\sigma}(A^t)\heta_{\phantom{\eta}\alpha}
                  g_{\alpha\beta}(x)A_{\beta}^{\phantom{\beta}\sigma}
                  \gamma_x^{\sigma}(\tnab_{e_{\eta}}s)(x)\\
          &=\sum_{\eta\sigma}g_x(e_{\eta},e_{\sigma})\gamma_x^{\sigma}(\tnab_{e_{\eta}}s)(x),
\end{split}
\end{equation}
where we used the fact that $A^tgA=g$ and that all the $A_{\alpha}^{\phantom{\alpha}\beta}$ are $C^{\infty}$ functions on $M$, so that
$\tnab_{A_{\alpha}^{\phantom{\alpha}\beta}X}=A_{\alpha}^{\phantom{\alpha}\beta}\tnab_X$.  This shows that $\psi(x)$ doesn't depend on the choice of the basis, so it defines a section from the data of $s$ alone.


The \defe{Dirac operator}{dirac!operator!on $(M,g)$, a spin manifold}\nomenclature[D]{$\Dir$}{Dirac operator} $\Dir\colon \Gamma(\mS)\to \Gamma(\mS)$ acting on a spinor field is defined by 
\begin{equation}\label{dirac}
(\Dir s)(x)=g_x(e_{\alpha},e_{\beta})\gamma_x\hbeta(\tnab_{e_{\alpha}}s)(x).
\end{equation}

\begin{proposition}
If the field of basis $e_{\alpha}\in\cvec(M)$ is everywhere an orthonormal basis, the Dirac operator reads
\begin{equation}
(\Dir s)(x)=g_{\alpha\beta}\gamma^{\alpha}(\tilde\nabla_{e_{\beta}}s)(x)
\end{equation}
where $\gamma^{\alpha}$ is a constant numeric matrix acting on $\Lambda W$.
\end{proposition}

\begin{proof}
The building of the Dirac operator begins by considering the vector space $T_xM$ endowed with the metric $g_x$; then the spinor representation $\tilde\rho_x\colon T_xM\to \End(\Lambda W_x)$ where $\Lambda W_x$ is build from isotropic vectors of $T_xM$ is defined. If the vector fields $e_{\alpha}\in\cvec(M)$ are everywhere orthonormal for the metric $g$, then we have the matricial equality
\begin{equation}
	\tilde\rho_x\big( (e_{\alpha})_x \big)_{ij}=\tilde\rho(v_{\alpha})_{ij}
\end{equation}
where the left hand side describe the matrix component of a linear operator acting on $\Lambda W_x$ while in the right hand side we have the matrix component of a linear operator acting on $\Lambda W$ and $v_{\alpha}$ is a basis on $\eR^n$ with respect to which the metric is the same as the metric $g_x$ in the basis $(e_{\alpha})_x$. Let $\hat{\psi}\colon P\to \Lambda W$ be an equivariant function; from definition \eqref{EqDefgammax} of $\gamma$ we have
\[ 
  \big( \gamma(e_{\alpha}\hat{\psi}) \big)(\xi)=(a_{\alpha}\hat{\psi})(\xi)
\]
where $a_{\alpha}(\xi)=\tilde\rho\Big( \tilde{e}_{\alpha}\big( \phi(\xi) \big) \Big)$. In this expression, $\tilde{e}_{\alpha}$ is the equivariant function associated with the vector field $e_{\alpha}\in\cvec(M)$. It is defined in subsection \ref{equivvec} as
\begin{equation}
\begin{aligned}
 \tilde{e}_{\alpha}\colon \SO(M)&\to \eR^m \\ 
b&\mapsto b^{-1}\big( (e_{\alpha})_{\pi(b)} \big). 
\end{aligned}
\end{equation}
So we have $\hat a_{\alpha}\colon P\to \End(\Lambda W)$ defined by
\[ 
  \hat a_{\alpha}(\xi)=\tilde\rho\big( \varphi(\xi)^{-1}e_{\alpha}(x) \big)
\]
with $x=\pi(\xi)$. Now if $\xi$ is any element of $\pi^{-1}(x)$, we have
\begin{align*}
\big( \gamma(e_{\alpha})\psi \big)(x)&=(a_{\alpha}\psi)(X)=\big[ \xi,\hat a_{\alpha}(\xi)\hat\psi(\xi) \big]
		=\big[ \xi,\tilde\rho\big( \varphi(\xi)^{-1}e_{\alpha}(x) \big)\hat{\psi}(\xi) \big].
\end{align*}
There exists a $g\in\Spin(p,q)$ such that $\varphi(\xi\cdot g)=\mtu$; taking this element and using equivariance of the latter expression,
\begin{align}
  \big( \gamma(e_{\alpha})\psi \big)(x)=\big[ \xi\cdot g,\tilde\rho\big( e_{\alpha}(x) \big)\hat{\psi}(\xi\cdot g) \big]
		=\big[ \xi\cdot g,\gamma^{\alpha}\hat{\psi}(\xi) \big]
		=[\xi,\gamma^{\alpha}\hat{\psi}(\xi)].
\end{align}
What we proved is that $\big( \gamma e_{\alpha}\psi \big)(x)=\gamma^{\alpha}\psi(x)$ is the sense that
\begin{equation}
	\widehat{\gamma(e_{\alpha})\psi}=\gamma^{\alpha}\hat{\psi}.
\end{equation}
Hence the Dirac operator reads
\[ 
  (\Dir s)(x)=g_{\alpha\beta}\gamma^{\alpha}\big( \tilde\nabla_{e_{\beta}}s \big)(x)
\]
in the sense that
\begin{equation}
\widehat{\Dir s}=g_{\alpha\beta}\gamma^{\alpha}\widehat{  \tilde\nabla_{e_{\beta}}s }.
\end{equation}

\end{proof}


An often more convenient way to write the Dirac operator is to consider an orthonormal basis (so that the metric $g$ and the matrices $\gamma$ are constant) and to consider the equivariant functions:
\[ 
  \widehat{\Dir\psi}=g_{\alpha\beta}\gamma^{\alpha}\widehat{\nabla_{e_{\alpha}}\psi}.
\]
This formulation is typically used when one search for Dirac operator on Lie groups. In this case, we choose left invariant vector fields generated by an orthonormal basis of the Lie algebra. The resulting field of basis is everywhere Killing-orthonormal.

Acting on a function $\dpt{f}{M}{\eR}$, it is defined by $\dpt{\Dir}{C^{\infty}(M)}{C^{\infty}(M)}$\index{dirac!operator!on functions},
\begin{equation} \label{eq_defDirac_f}
(\Dir f)(x)=g_x(e_{\alpha},e_{\beta})\gamma\hbeta_x(e_{\alpha x}\cdot f).
\end{equation}
With these definitions, one has
\[(\Dir(fs))(x)=(f\Dir s)(x)+(\Dir f)(x).\] Indeed,
\begin{equation}
\begin{split}
   (\Dir(fs))(x)&=g_{\alpha\beta}\gamma_x\hbeta(\tnab_{e_{\alpha}}fs)(s)\\
                &=g_{\alpha\beta}\Big((e_{\alpha}\cdot f)s(x)+f(x)(\tnab_{e_{\alpha}}s)(x)\Big)\\
                &=f(x)(\Dir s)(x)+g_{\alpha\beta}\gamma_x\hbeta(e_{\alpha x}\cdot f)\\
                &=(f\Dir s)(x)+(\Dir f)(x).
\end{split}
\end{equation}
With that definition, the Dirac operator becomes a derivation of the spinor bundle.


\section[Dirac operator on  \texorpdfstring{$\eR^2$}{R2}]{Example: Dirac operator on \texorpdfstring{$\eR^2$}{R2} with the euclidian metric}\label{Pg_exempleRdeux}\index{dirac!operator!on $\eR^2$}
%---------------------------------------------------

%\subsubsection{Example: tangent bundle}
%--------------------------

Since the frame bundle $B(M)$ is a principal bundle (see subsection \ref{subsec_frbundle}), one can consider some associated bundles on it. We are now going to see that the one given by the definition representation $\dpt{\rho}{GL(n,\eR)}{GL(n,\eR)}$ on $\eR^n$ is the tangent bundle. So we study $B(M)\times_{\rho} \eR^n$. By choosing a basis on each point of $M$, we identify each $T_xM$ to $\eR^n$. An element of $B(M)\times \eR^n$ is a pair $(b,v)$ with $b=(\overline{b}_1,\ldots,\overline{b}_n)$ and $v=(v^1,\ldots,v^n)$. We can identify $v$ to the element of $T_xM$ given by $v=v^i\overline{b}_i$.

In order to build the associated bundle, we make the identifications
\[
  (b,v)\cdot g\sim(b\cdot g,g^{-1} v).
\]
Here, by $gv$ we mean the vector whose components are given by $(gv)^i=v^j\bghd{g}{j}{i}$. The tangent vector given by $(b\cdot g,g^{-1} v)$ is $(g^{-1} v)^i(b\cdot g)_i=v^j\bghd{(g^{-1})}{j}{i}\bghd{g}{i}{k}\overline{b}_k=v^k\overline{b}_k$ So the identification map $\dpt{\psi}{B(M)\times_{\rho}\eR^n}{TM}$ given by
\[
  \psi([b,v])=v^i\overline{b}_i
\]
is well defined.

\index{spin!structure!on $\eR^2$}
The following step is to consider the following spin structure:
 \[\xymatrix{
    \Spin(2)  \ar@{~>}[r]&  \eR^2\times \SO(2) \ar[r]^-{\displaystyle\varphi} & \SO(\eR^2)  & \SO(2) \ar@{~>}[l].
  }\]

We have to define the two actions and $\varphi$. One of the main result of section~\ref{cliffR2} is that $\dpt{\chi}{\Spin(2)=U(1)}{\SO(2)}$ is surjective. So, we can define the action of $\Spin(2)$ on $P$ by
\[(x,b)\cdot s=(x,\chi(s)^{-1} b).\]

On the other hand, an element $A$ in $\SO(\eR^2)$ can be written as $A=\baz{a}{x}$ where $e_i$ is the canonical basis of $T_x\eR^2$, and $a$ is a matrix of $\SO(2)$. See subsection \ref{subsec_frbundle}. For $g\in \SO(2)$, we define
\begin{eqnarray}
 \label{r1504d2}A\cdot g=\{g^{-1} ae_i\}_x.
\end{eqnarray}
and  $\dpt{\varphi}{\eR^2\times \SO(2)}{\SO(\eR^2)}$ by
\[
\varphi(x,b)=\{be_i\}_x.
\]
The following shows that these definitions give a spin structure:
\begin{equation}
   \varphi((x,b)\cdot s)=\varphi(x,\chi(s)^{-1} b)
                    =\{\chi(s)^{-1} be_i\}_x
                    =\{be_i\}_x\cdot\chi(s)
                    =\varphi(x,b)\cdot\chi(s).
\end{equation}


\subsection{Connection on \texorpdfstring{$\SO(\eR^2)$}{SO(R2)}}\index{connection!on $\SO(\eR^2)$}
%///////////////////////////////////////

We are searching for a torsion-free connection on the simplest metric space: the euclidian $\eR^2$. Thus we will try the simplest choice of horizontal space: we want an horizontal vector to be tangent to a curve of the form $X(t)=\baz{b}{x(t)}$. For this reason, we want to define the connection $1$-form by $\omega(X)=b'(0)$. For technical reasons which will soon be apparent, we will not exactly proceed in this manner. For $X(t)=\baz{b}{x(t)}$, we define
\begin{equation}
                       \omega(X)=-(b(t)b(0)^{-1})'(0).
\end{equation}
We of course have $\omega(X)=0$ if and only if $b'(0)=0$: this choice of $\omega$ follows our first idea. In order for $\omega$ to be a connection form, we have to verify the two conditions of definition \ref{defconnform}.

\begin{proposition}
The $1$-form defined by
\[
              \omega(X)=-(b(t)b(0)^{-1})'(0)
\]
for $X=\displaystyle\dsdd{\baz{b(t)}{x(t)}}{t}{0}$ is a connection $1$-form.
\end{proposition}

\begin{proof}
Let $A\in \SO(2)$. If $u=\baz{b}{x}$, equation \eqref{r1504d2} gives:
\[
   A^*_u=\dsdd{\baz{e^{-tA}b}{x}}{t}{0},
\]
 so that $\omega(A^*_u)=-(e^{-tA}bb^{-1})'(0)=A$. This checks the first condition. For the second, one remarks that the path in $\SO(\eR^2)$ which defines the vector $R_{g*}X$ is $(R_{g*}X)(t)=\baz{g^{-1} b(t)}{x}$. It follows that
\begin{equation}
\begin{split}
\omega(R_{g*}X)&=-(g^{-1} b(t)b(0)^{-1} g)'(0)\\
               &=-\left(\AD_{g^{-1}}(b(t)b(0)^{-1})\right)'(0)\\
               &=-Ad_{g^{-1}}(b(t)b(0)^{-1})'(0)\\
               &=Ad_{g^{-1}}\omega(X).
\end{split}
\end{equation}

\end{proof}

\begin{proposition}
The covariant derivative induced on $M$ by this connection is
\begin{equation}\label{derrcovexplicite}
                 \nabla_XY=X(Y).
\end{equation}
\end{proposition}

\begin{proof}
In this demonstration, we will use the equivariant functions defined in \ref{equivvec}. In order to compute $(\nabla_XY)_x$, we have to use the definition of theorem \ref{tho_nablaE}. We first have to compute the horizontal lift of $X$. It is no difficult to see that $\oX_{\baz{b}{x}}$ is given by the path
\[\oX(t)=\baz{b}{X(t)}\]
if the vector field $X$ is given by the path $X(t)$ in $M$. Indeed, it is trivial that $\omega(\oX)=0$, and
\[d\pi_*\oX=\dsdd{\pi\baz{b}{X(t)}}{t}{0}=\dsdd{X(t)}{t}{0}=X.\]

Now, we compute $(\oX\hs)(b)$ for $b=\{Se_i\}_x$. We begin using the basic definitions and notations:
\[
(\oX\hs)(b)=\oX_b\hs=\dsdd{\hs(\oX_b(t))}{t}{0}=\dsdd{\hs(\baz{S}{X(t)})}{t}{0}.
\]
We can rewrite it with $\hY$ instead of $\hs$. By construction (see \eqref{r1404e1}), if $b=\baz{S}{x}$, $\hY(b)=S^{-1}(Y_x)$. Thus
\[
(\oX\hY)(b)=\dsdd{S^{-1}(Y_{X(t)})}{t}{0},
\]
where, if $\{\oui\}$ is a basis of $\eR^m$, then $S$ is 
\begin{equation}
\begin{aligned}
 S\colon\eR^m&\to T_{X(t)}M \\ 
 v^i\oui &\mapsto S^i_jv^j(\partial_j)_{X(t)} 
\end{aligned}
\end{equation}
So if we write $Y_x=Y^i(x)\partial_i$, we have
\[
S^{-1}(Y_{X(t)})=(S^{-1})^i_jY^j(X(t))\oui
\]
and
\[
\dsdd{S^{-1}(Y_{X(t)})}{t}{0}=(S^{-1})^i_j\dsdd{Y^j(X(t))}{t}{0}\oui=(S^{-1})^i_jX(Y^j)\oui.
\]
Since $b$ is an isomorphism, we can apply $b$ on both side of $\hX(b)=b^{-1}(X_x)$, and take $\nabla_XY$ instead of $X$:
\begin{equation}
 (\nabla_XY)(x)=b\big((S^{-1})^i_jX(Y^j)\oui\big)
               =S^k_i(S^{-1})^i_jX(Y^j)(\partial_k)_x
               =X(Y^j)(\partial_j)_x
               =X(Y)_x.
\end{equation}
\end{proof}

From this and definition \ref{deftorsion}, we immediately conclude that our connection is torsion-free. In a certain manner, one can say that our covariant derivative is the usual one.

\subsection{Construction of \texorpdfstring{$\gamma$}{g}}
%//////////////////////////////////////

Now, we construct the map $\gamma$ of subsection \ref{applgamma}. The first step is to define $\dpt{\ha_X}{P}{\End{(\Lambda W )}}$ by
\[
\ha_X(p)=\tilde\rho(\hX_{\varphi(p)}).
\]
Here, $\Lambda W $ is the completely isotropic subspace of $(\eR^2)^{\eC}$ with euclidian metric; thus we can use the result of section \ref{cliffR2}. In particular, we know the representation $\tilde\rho$.

To see it more explicitly, we need the expression of $\hX$. It is given in subsection \ref{equivvec}: if $b$ is the basis $\baz{b}{x}$, $\hY(b)=b^{-1}(Y_x)$. As $\varphi(b,x)=\baz{b}{x}$, we have
\[
\ha_X(b,x)=\tilde\rho(b^{-1}(X_x)).
\]

The subsection \ref{equivendo} explains how to explicitly get $\gamma(X)$ with the definition $\gamma(X)=a_X$. If $\psi$ is a section of $\mS$ and $\psi(x)=[\xi,v]$, the general definition gives us $(a_X\psi)(x)=[\xi,\ha_X(\xi)v]$ and in our particular case, if $\xi=(b,x)$, we get:
\begin{eqnarray}
 \label{gammaX}(\gamma(X)\psi)(x)=[\xi,\tilde\rho(b^{-1}(X_x))v].
\end{eqnarray}

\subsection{Covariant derivative on \texorpdfstring{$\protect\Gamma(\mS)$}{S}}
%///////////////////////////////////////////

Remember the spin structure of $\SO(\eR^2)$: $\varphi(x,S)=\{Se_i\}_x$. We now construct the connection on $P=\eR^2\times \SO(2)$. It is defined by the $1$-form $\tomega=\varphi^*\omega$. If $v$ is a vector of $P$, it is described by a path $v(t)=(x(t),b(t))$, then the path of $d\varphi(v)$ is $\{b(t)e_i\}_{x(t)}$ and $\tomega(v)=\omega(d\varphi(v))=-(b(t)b(0)^{-1})'(0)$.

The next step defining the Dirac operator is to find out an explicit form for the map $\dpt{\tnab}{\cvec(M)\times\Gamma(\mS)}{\Gamma(\mS)}$. A section $s\in\Gamma(\mS)$ is a map $\dpt{s}{M}{\mS=(\eR^2\times \SO(2))\times_{\rho}\Lambda W }$; it is defined by an equivariant function $\dpt{\hs}{P}{\Lambda W }$. In order to find the value of $(\tnab_Xs)(x)$ for $X\in\cvec(M)$, we use the definition
\[
 \widehat{\tnab_Xs}(\xi)=\oX\bxi(\hs)
\]
where $\oX$ is the horizontal lift in the sense of $\tomega$. For the same reason as in the proof of proposition \ref{derrcovexplicite}, $\oX_{(b,x)}$ is given by the path $\oX(t)=(b,X(t))$ where $X(t)$ is the path which defines $X$. So we have
\[
 \widehat{\tnab_Xs}(\xi)=\oX_{(b,x)}(\hs)=\dsdd{\hs(b,X(t))}{t}{0}.
\]
Remark that $\Lambda W $ is a vector space; so for every $\alpha\in\Lambda W $, the identification $T_{\alpha}\Lambda W =\Lambda W $ is correct.

Our first form of $\tnab$ is
\[
(\tnab_Xs)(x)=\Big[\xi,\dsdd{\hs(b,X(t))}{t}{0}\Big],
\]
but we can modify this in order to get simpler expressions. Remark that we have an equivalence class, so that we can always choose the element of the class such that $\xi=(\mtu,x)$. We define $\dpt{\os}{\eR^2}{\Lambda W }$, $\os(v)=\hs(\mtu,v)$. Our second and final form for $\tnab$ is:
\begin{subequations}
 \begin{align}
 (\tnab_Xs)(x)&=\Big[(\mtu,x),\dsdd{\os(X(t))}{t}{0}\Big]\\\label{nabs}
              &=[(\mtu,x),X(\os)],
\end{align}
\end{subequations}
where $X(\os)$ is well defined because $\os$ is a map from $\eR^2$ into a vector space (namely: $\Lambda W $).

\subsection{Dirac operator on the euclidian \texorpdfstring{$\eR^2$}{R2}}
%///////////////////////////////////////////////////
\index{dirac!operator!on $\eR^2$}

We continue to write explicitly the definition \eqref{dirac}. Putting together \eqref{gammaX} and \eqref{nabs}, one finds
\begin{equation}
 \gamma^{\alpha}_x(\tnab_{e_{\beta}}s)(x)	=\gamma(e_{\alpha x})[\xi,e_{\beta}(\os)]
                                     		=[\xi,\tilde\rho(b^{-1}(e_{\alpha x}))e_{\beta}(\os)].
\end{equation}
Here, $e_{\beta}=\partial_{\beta}$ and $b=\mtu$, then
\[
 \gamma^{\alpha}_x(\tnab_{e_{\beta}}s)(x)=[(\mtu,x),\tilde\rho(e_{\alpha})\partial_{\beta}\os].
\]
Now, the Dirac operator reads
\[
 (\Dir s)(x)=[(\mtu,x),\gamma^{\alpha}\partial_{\alpha}\os].
\]

We can obtain a more compact expression by defining ``$Ys$''\ and ``$As$'' when $s\in\Gamma(\mS)$, $Y\in\cvec(\eR^2)$ and $A\in\End{\Lambda W }$. The definitions are
\begin{align*}
(Ys)(x)&=[(\mtu,x),(Y\os)(x)],\\
(As)(x)&=[(\mtu,x),A\os(x)].
\end{align*}
With these conventions, one writes:
\[
(\Dir s)(x)=\gamma^{\alpha}(\partial_{\alpha} s)(x).
\]
This justifies the expression \eqref{dirflat}: $\Dir=\gamma^{\alpha}\partial_{\alpha}$ on flat spaces. With a good choice of basis of $\Lambda W $, the matrices $\gamma^{\alpha}$ are given by \eqref{gammaR2}, and
\[
\gamma^{\alpha}\partial_{\alpha}=
\begin{pmatrix}
0 & -1 \\
1 & 0
\end{pmatrix}\partial_x-
\begin{pmatrix}
0 & i \\
i & 0
\end{pmatrix}\partial_y.
\] 
If we identify $\eR^2$ with $\eC$ we have the following definitions:
\[
\partial_z=\frac{1}{2}(\partial_x-i\partial_y),\qquad\partial_{\overline{z}}=\frac{1}{2}(\partial_x+i\partial_y),\]
so that
\[\Dir=\begin{pmatrix}
0 & -\partial_{\overline{z}} \\
\partial_z & 0
\end{pmatrix}.
\]

%---------------------------------------------------------------------------------------------------------------------------
\subsection{Dirac operator as elliptic pseudo-differential operator}
%---------------------------------------------------------------------------------------------------------------------------
\label{subSecREctBOh}

Let $(M,g)$ be a spin manifold and $D$, its Dirac operator which is locally written under the form $D=\gamma(dx^{\mu})\partial_{\mu}$. So $A_{\mu}(x)=\gamma(dx^{\mu})$, and the principal symbol is
\[ 
  \xi^{\mu}A_{\mu}(x)=\gamma(\xi).
\]
Let us point out that $\gamma(\xi)$ is not a real number, but an endomorphism of the spinor bundle. Using relation \eqref{3101r3}, we find that
\[ 
  \gamma(\xi)^2=-\| \xi \|^2\id,
\]
which is invertible when $\xi\neq 0$. We conclude that Dirac is an elliptic operator\footnote{Definition \ref{DefGLpDEHy}.}.


\section{Clifford algebras and Morita equivalence}
%++++++++++++++++++++++++++++++++++++++++++++++++

Let $\cA$ be an algebra. An algebra $\cB$ is said to be \defe{Morita equivalent}{Morita equivalence}\label{PgMoritaEq} to $\cA$ if $\cB=\End_{\cA}(\modE)$ for some finite projective module $\modE$ over $\cA$. The algebra $\cA$ is Morita equivalent to itself taking the trivial module $\modE=\cA$.

We consider a manifold $M$ of dimension $n=2m$.

\begin{probleme}
	The two following statements are imprecise.
\end{probleme}

\begin{proposition}
A module which implement a Morita equivalence between two $C^*$-algebras is finite projective.
\end{proposition}

\begin{theorem}[Serre-Swan]
If one of the two Morita equivalent is the continuous function space over a manifold $\cA=C(M)$, then the module which gives the Morita equivalence is the section of continuous sections of a vector bundle over $M$, $\modE=\Gamma(E)$.
\end{theorem}
Furthermore, if $\cA=C(M)$ and $\cB=\Gamma(\Cl(M))$, we have $\End E\simeq \Cl(M)$ as isomorphism of vector bundle. Since $\Cl M$ is of rank $2^n$, $\End E$ has same rank and $E_x$ has dimension $\sqrt{2^n}=2^m$. So it is possible to choose the Clifford action in such a way that $\Gamma(E)$ is an irreducible Clifford module.

We often look at an anti-linear map $J\colon \Gamma(E)\to \Gamma(E)$ such that for all $\psi\in\Gamma(E)$
\begin{enumerate}
\item $J(\psi f)=(J\psi)\overline{ f }$ for all $f\in C(M)$,
\item $J(a\psi)=\epsilon(a)a J\psi$ for all $a\in\Gamma^{\infty}(\Cl M)$.
\end{enumerate}
How to define $a\psi$ ? We consider $\cA=C(M)$, $\cB=\Gamma(\Cl M)$ and we define $\Gamma(E)$ is such a way that it implements a Morita equivalence between $\cA$ and $\cB$; hence $\Gamma(E)$ is a $C(M)$-module. From dimensional considerations, we can define on $\Gamma(E)$ a Clifford module structure, i.e. a $C(M)$-linear
\begin{equation}
  c\colon \Gamma(\Cl M)\to \End(\Gamma E),
\end{equation}
hence $a\psi$ makes sense for any $a\in\Gamma^{\infty}(\Cl M)$ and $\psi\in\Gamma(E)$ with definition
\begin{equation}
 (a\psi)(x)=\big( c(a)\psi \big)(x)
		=c(a(x))\psi(x)
\end{equation}

\begin{theorem}
Let $(M,S,J)$ be a spin manifold of dimension $n$. There exists an unique connection 
\[ 
  \nabla^S\colon \Gamma^{\infty}(S)\to \Gamma^{\infty}(S)\otimes\Omega^1(S)
\]
such that
\begin{enumerate}
\item $\scalp{ \nabla^S\psi }{ \phi }+\scalp{ \psi }{ \nabla^S\phi }=d\scalp{ \psi }{ \phi }$,
\item $[\nabla^S,J]=0$,
\item $\nabla^S\big( c(a)\psi \big)=c\big( \nabla a \big)\psi+c(a)\nabla^S\psi$ for all $a\in\Cl(M)$ and $\psi\in\Gamma^{\infty}(S)$.
\end{enumerate}
In the latter, the action of $\Gamma^{\infty}(\Cl M)$ on $\Gamma^{\infty}(S)$ is induced from the action $c\colon \Cl(T^*_xM)\to \End S$. The $\nabla$ which acts on $a$ is the connection extended to $\Gamma^{\infty}(\Cl M)$ by virtue of Leibnitz rule $\nabla(uv)=\nabla(u)v+u\nabla(c)$.

\end{theorem}

\begin{proof}
No proof
\end{proof}


In this setting, we define 
\begin{equation}
\begin{aligned}
 \hat c\colon\Gamma^{\infty}(S)\otimes\Gamma^{\infty}(\Cl M)&\to \Gamma^{\infty}(S) \\ 
 \psi\otimes a&\mapsto c(a)\psi. 
\end{aligned}
\end{equation}
Then we define the \defe{Dirac operator}{dirac!operator} $\Dir\colon \Gamma^{\infty}(S)\to \Gamma^{\infty}(S)$,
\begin{equation}
  \Dir=-i(\hat c\circ\nabla^S).
\end{equation}

\subsection{Example: quantum field theory}
%-----------------------------------------

Let us show how does this operator gives back the usual Dirac operator of quantum field theory. Let $M$ be a manifold and with two local basis $\{ \partial_u \}$ and $\{ \partial_{\alpha} \}$ of $T_xM$. The first one is the ``natural'' basis: $g(\partial_u,\partial_v)=g_{uv}$ has no particular properties while the second one is orthonormal $g(\partial_{\alpha},\partial_{\beta})=\delta_{\alpha\beta}$. The first dual basis is defined by $dx^{\alpha}\partial_{\beta}=\delta^{\alpha}_{\beta}$. 

We write $\partial_{\alpha}=e_{\alpha}^u\partial_u$ and for the dual basis, $dx^{\alpha}=e^{\alpha}_u\,dx^u$. In order these definition to be coherent, we impose $dx^{\alpha}\partial_{\beta}=\delta^{\alpha}_{\beta}$ :
\begin{equation}
  dx^{\alpha}\partial_{\beta}=e_u^{\alpha}dx^u\big( e^v_{\beta}\partial_v \big)
		=e^{\alpha}_ue^v_{\beta}\delta^u_v
		=e^{\alpha}_ue^u_{\beta}.
\end{equation}
We conclude that the \defe{vielbein}{vielbein} $(e^{\alpha}_u)$ is the inverse of $(e^v_{\beta})$: $e^{\alpha}_ue^u_{\beta}=\delta^{\alpha}_{\beta}$. The vielbein are eventually complexes.



\subsection{An other definition of the Dirac operator}
%-----------------------------------------------------

Let us consider an orthonormal basis $\{ e_a \}$ of $M$, i.e. on each $x\in M$, 
\[ 
  g_x(e_a(x),e_b(x))=\eta_{ab}.
\]
This basis is related to a ``natural'' basis $\{ \partial_{\mu} \}$ by
\begin{equation} 
  e_a=e_a^{\mu}\partial_{\mu}
\end{equation}
where  $e_a^{\mu}$ is called \defe{vielbein}{vielbein} (here, they are more precisely $n$-beins). As far as metric is concerned we have
\begin{subequations}
\begin{align}
	g^{\mu\nu}&=e_a^{\mu}e_b^{\nu}\eta_{ab}\\
	\eta_{ab}&=e_a^{\mu}e_b^{\nu}g_{\mu\nu}.
\end{align}
\end{subequations}
If $\nabla$ is the covariant derivative associated with $g$, we define the coefficients $\omega_{\mu a}^b$ by
\begin{equation}
\nabla_{\mu}e_a=\omega_{\mu a}^be_b.
\end{equation}
On the other hand, $\nabla$ is related to the Christoffel symbols by
\begin{equation}
\nabla_{\mu}\partial_{\nu}=\Gamma_{\mu\nu}^{\sigma}\partial_{\sigma}.
\end{equation}
Let $\Cl(M)$ be the Clifford module whose fibre is the Clifford complex algebra $\Cl(T^*_xM)^{\eC}$. We consider $\Gamma(\Cl(M))$, the module of corresponding sections. It gives an algebra morphism
\begin{equation}
\begin{aligned}
 \gamma\colon\Gamma(\Cl(M))&\to \opB(\hH) \\ 
dx^{\mu}&\mapsto \gamma^{\mu}(x)=\gamma^ae_a^{\mu} 
\end{aligned}
\end{equation}
which can be extended to the whole Clifford algebra. One can choose matrices $\gamma^{\mu}(x)$ and $\gamma^a$ to be hermitian; they satisfy
\begin{subequations}
\begin{align}
\gamma^{\mu}(x)\gamma^{\nu}(x)+\gamma^{\nu}(x)\gamma^{\mu}(x)&=-2g(dx^{\mu},dx^{\nu})=-2g^{\mu\nu}\\
\gamma^a\gamma^b+\gamma^b\gamma^a&=-2\eta^{ab}.
\end{align}
\end{subequations}
All this allow us to lift the Levi-Civita connection from the tangent bundle to the spinor bundle by defining
\begin{equation}
\nabla_{\mu}^S=\partial_{\mu}+\omega^S_{\mu}=\partial_{\mu}+\frac{ 1 }{2}\omega_{\mu ab}\gamma^a\gamma^b.
\end{equation}
The \defe{Dirac operator}{operator!Dirac} is then given by
\[ 
  \Dir=\gamma\circ\nabla
\]
and can locally be written under the form
\begin{equation}  \label{eq_Dirac_deux}
\Dir=\gamma^{\mu}(x)(\partial_{\mu}+\omega_{\mu}^S)
	=\gamma^ae_a^{\mu}(\partial_{\mu}+\omega^S_{\mu}).
\end{equation}
