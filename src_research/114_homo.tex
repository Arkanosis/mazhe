% This is part of (almost) Everything I know in mathematics
% Copyright (c) 2013-2014,2016
%   Laurent Claessens
% See the file fdl-1.3.txt for copying conditions.

%+++++++++++++++++++++++++++++++++++++++++++++++++++++++++++++++++++++++++++++++++++++++++++++++++++++++++++++++++++++++++++
\section{Symplectic symmetric spaces}
%+++++++++++++++++++++++++++++++++++++++++++++++++++++++++++++++++++++++++++++++++++++++++++++++++++++++++++++++++++++++++++

\begin{definition}
	A \defe{symplectic symmetric}{symplectic!symmetric space} space is a triple $(M,s,\omega)$ where $(M,s)$ is a symmetric space, $(M,\omega)$ is a symplectic space such that $s_x^*\omega=\omega$ for every $x\in M$.
\end{definition}

\begin{remark}
	We can weaken the symplectic condition in the definition and only ask for $\omega$ to be non degenerate because the condition $s_x^*\omega=\omega$ implies $d\omega=0$.
\end{remark}

%---------------------------------------------------------------------------------------------------------------------------
\subsection{Example}
%---------------------------------------------------------------------------------------------------------------------------

Let $G=\SL(2,\eR)$ and look at the coadjoint action $\Ad^*\colon G\to \GL(\lG^*)$. We consider the element $Z=E-F$ and the orbit
\begin{equation}
	\mO=\Ad^*(G)(Z^{\flat}).
\end{equation}
The space $\lG^*$ has the metric
\begin{equation}
	\langle X^{\flat}, Y^{\flat}\rangle =\beta(X,Y)
\end{equation}
Let us consider $\mfo=Z^{\flat}\in\lG^*$ and consider the stabilizer:
\begin{equation}
	\Stab_{\mfo}(\mO)=\{ g\in G\tq \Ad^*(g)Z^{\flat}=Z^{\flat} \}.
\end{equation}
The Lie algebra is given by
\begin{equation}
	\stab_{\mfo}(\mO)=\{ X\in\lG\tq Z^{\flat}\circ\ad(X)=0 \}.
\end{equation}
The condition of the Lie algebra reads
\begin{equation}
		0=\langle Z^{\flat}, [X,Y]\rangle 
		=\beta(Z,[X,Y])
		=-\beta\big( [X,Z],Y \big)
\end{equation}
for every $Y$, which implies $[X,Z]=0$ because $\beta$ is nondegenerate. Now, in $\gsl(2,\eR)$, the only possibility is that $X$ is proportional to $Z$. Thus the Lie algebra reduces to $\eR Z$ in fact.

%---------------------------------------------------------------------------------------------------------------------------
\subsection{Algebraic setting}
%---------------------------------------------------------------------------------------------------------------------------

We want now to encode the symplectic space structure in an algebraic data. What we are going to discover is the notion of symplectic triple that will be developed in section \ref{SubSecTripleSylple}.

Let $(G,\sigma)$ be an involutive Lie group and $H$ a closed subgroup of $G$ such that
\begin{equation}
	G_0^{\sigma}\subset H\subset G^{\sigma}.
\end{equation}
Let $\pi$ be the projection $\pi\colon G\to M=G/H$. 

The symmetry on the quotient $G/H$ is given by the theorem \ref{ThoStructSymGH}:
\begin{equation}		\label{EaSymGH}
	s_{[g]}[g']=\big[ \sigma(g^{-1}g') \big]
\end{equation}

Now, if we denote by $\sigma$ the differential $d\sigma_e$, we can decompose the Lie algebra $\mG$ into $\mG=\mH\oplus\mP$ and we have the isomorphism (see lemma \ref{LemdpiisomMTM})
\begin{equation}
	d\pi_e|_{\mP}\colon \mP\to T_{\mfo}(M)
\end{equation}
where $\mfo=[e]$. Thus we can see the form $\omega_{\mfo}$ on $\mP$ by
\begin{equation}
	\Omega=\big( d\pi_e|_{\mP} \big)^*\omega_{\mfo}
\end{equation}
and the space $(\mP,\Omega)$ becomes a symplectic vector space.

\begin{lemma}
	We have
	\begin{enumerate}

		\item
			the space $\mK=[\mP,\mP]$ is a Lie subalgebra of $\mH$,

		\item
			the adjoint action of $\mK$ over $\mP$ preserves the symplectic form, i.e.
			\begin{equation}
				\Omega\big( [Z,X],Y \big)+\Omega\big( X,[Z,Y] \big)=0
			\end{equation}

	\end{enumerate}
	
\end{lemma}

\begin{proof}
	Sketch of the proof.

	Let $x_j,y_j\in\mP$. Using the Jacobi identity on the nested commutator $\big[ [x_1,y_1],[x_2,y_2] \big]$ and the facts that $[\mP,\mP]\subset\mH$ and $[\mH,\mP]\subset \mP$, we find the commutator of two elements of $[\mP,\mP]$ belongs to $[\mP,\mP]$.

	First we consider $\mG^{(M)}=\mK\oplus\mP$ and $G(M)$, the associated Lie group. Then we have
	\begin{equation}
		M\simeq G(M)/K.
	\end{equation}
	
	Now one can see that the group $G(M)$ is generated by the products $\{ s_{\mfo}s_x\}$ with $x\in M$. Indeed let $X\in\mP$ and look at $\exp(\mP)$ as map on $M$. Using the symmetry \eqref{EaSymGH} and the fact that $\sigma e^{X/2}= e^{-X/2}$, we have
	\begin{equation}
		s_{\exp(X/2)\cdot \mfo}\mfo= e^{X/2}\big[ \sigma[ e^{-X/2}] \big]=[ e^{X}]= e^{X}\cdot \mfo.
	\end{equation}
	If we act on an other point than $\mfo$, we have
	\begin{equation}
		\begin{aligned}[]
			s_{[ e^{X/2}]}[g]&= e^{X/2}\big[ \sigma( e^{-X/2}g) \big]\\
			&= e^{X/2}\big[  e^{X/2}\sigma(g) \big]\\
			&= e^{X}s_{\mfo}[g]
		\end{aligned}
	\end{equation}
	because $\big[ \sigma(g) \big]=s_{\mfo}[g]$.

	Now, using the lemma \ref{LemAlgEtGroupesGenere}, the fact that the elements $ e^{X}$ with $X\in\mP$ generate $ e^{\mP}$ in $G$ implies that it also generate the elements of the form $ e^{[\mP,\mP]}$ and then the whole $G(M)$. Since the elements $ e^{\mP}$ are of the form $s_{x}s_{\mfo}$, we conclude that $G(M)$ is generated by the products $s_{x}s_{\mfo}$.
	
	Thus we have $g^*\omega=\omega$ for every $g\in G(M)$ because $\omega$ is preserved by all the symmetries.
\end{proof}

%---------------------------------------------------------------------------------------------------------------------------
\subsection{Symplectic triple}
%---------------------------------------------------------------------------------------------------------------------------
\label{SubSecTripleSylple}

A \defe{symplectic triple}{symplectic!triple} is the data of the triple $(\mG,\sigma,\Omega)$ where $(\mG,\sigma)$ is an involutive Lie algebra and $\Omega$ is a $\mK$-invariant nondegenerate $2$-form $\Omega\in\Lambda^2(\mP^*)$. The $\mK$ invariance means that for every $Z\in\mK$ and $X,Y\in\mP$,
\begin{equation}
	\Omega\big( [Z,X],Y \big)+\Omega\big( X,[Z,Y] \big)=0.
\end{equation}

A symplectic triple is the infinitesimal version of a symplectic symmetric space. The following more abstract version of the definition comes from \cite{StrictSolvableSym}:
\begin{definition}
	The triple $(\mG,\sigma,\Omega)$ is a \defe{symplectic triple}{symplectic!triple} when $\Omega\in\Lambda^2\mG$ and
	\begin{enumerate}
		\item  If  $\mG=\mK\oplus\mP$ is the decomposition of $\mG$ into eigenspaces of $\sigma$,  then $[\mP,\mP]=\mK$ and the adjoint representation of $\mK$ on $\mP$ is faithful. ($\mK$ is the eigenspaces with eigenvalue $+1$ of $\sigma$ while $\mP$ is the one of $-1$)
   
		\item The $2$-form $\Omega$ is a Chevalley $2$-cocycle for the trivial representation of $\mG$ on $\eR$.
	  
		\item $i(\mK)\Omega=0$ and $\Omega|_{\mP\times\mP}$ is nondegenerate.
	\end{enumerate}
\end{definition}

Notice that $[\mP,\mP]\subset\mK$ is automatic from the definition of $\mP$ and $\mK$ as eigenspaces of $\sigma$; the hypothesis is the equality.

Let us now see how one build a symplectic symmetric space from the data of the symplectic triple $(\mG,\sigma,\Omega)$. First we consider $G$, the group associated with $\mG$ and $M=G/K$ with the left invariant form $\omega$ build on $\Omega$.

%---------------------------------------------------------------------------------------------------------------------------
\subsection{Example on the Heisenberg algebra}
%---------------------------------------------------------------------------------------------------------------------------

Let $\pH=V\oplus\eR E$  be the Heisenberg algebra of $(V,\Omega^0)$, and consider the derivation
\begin{equation}
	D=\id|_V\oplus(2\id)|_{\eR E}.
\end{equation}
If we consider the algebra $\mA=\eR H$, we build the semi direct product
\begin{equation}
	\mS=\mA\ltimes_D\pH
\end{equation}
with the definition $[H,x]=D(x)$ when $x\in\pH$.

An other split extension that can be done is
\begin{equation}
	\mG_0=\mA\rtimes_{\rho}(\pH\oplus\pH)
\end{equation}
with $\rho=D\oplus(-D)$. The algebra $\mG_0$ is to be endowed with a symplectic triple structure. We define $\sigma_0\colon \mG_0\to \mG_0$
\begin{equation}
	\begin{aligned}[]
		\sigma_0(x,y)&=(y,x)&\in\pH\oplus\pH\\
		\sigma_0(H)&=-H
	\end{aligned}
\end{equation}
and $(\mG_0,\sigma_0)$ is an involutive automorphism. Indeed, we have
\begin{equation}
	\sigma[H,x\oplus y]=\sigma\big( Dx\oplus(-Dy) \big)=-Dy\oplus Dx,
\end{equation}
while
\begin{equation}
	\big[ \sigma H,\sigma(x\oplus y) \big]=[-H,y\oplus x]=-Dy\oplus Dx.
\end{equation}

Let us take the notation\nomenclature[G]{$W_{\pm}$}{The set of elements of the form $(w,\pm w)$ in $W\oplus W$}
\begin{equation}		\label{EqDefNitWpm}
	W_{\pm}=\{ (w,\pm w) \}_{w\in W}.
\end{equation}

If we decompose $\mG_0=\mK_0\oplus\mP_0$, we have
\begin{equation}
	\begin{aligned}[]
		\mK_0&=\pH_+\\
		\mP_0&=\mA\oplus\pH_-.
	\end{aligned}
\end{equation}
We have $H\in\mP$ $x\oplus(-x)\in\mP$ and $x\oplus x\in\mK$.

In fact we have an identification between $\mS$ and $\mP_0$ by
\begin{equation}
	\begin{aligned}
		\mS&\to \mP_0 \\
		a+x&\mapsto a+x_- 
	\end{aligned}
\end{equation}
where $x_{\pm}=\frac{ 1 }{2}(x,\pm x)$. Using the notation \eqref{EqDefNitWpm}, we write
\begin{equation}
	\begin{aligned}[]
		\mK&=\pH_{+}\\
		\mP&=\mA\oplus\pH_{-}.
	\end{aligned}
\end{equation}

Under that identification we have le following lemma.
\begin{lemma}
	We have
	\begin{equation}
		\Lambda^2(\mP_0^*)\simeq\Lambda^2(\mS^*)
	\end{equation}
	and if we define
	\begin{equation}
		\Omega^{\mS}(a+x,a'+x')=\Omega(a+x_-,a'+x'_-),
	\end{equation}
	we have $\Omega\in\Lambda^2(\mP_0^*)$ and it is $\mK_0$ invariant if and only if the two conditions
	\begin{equation}
		\begin{aligned}[]
			\Omega^{\mS}(E,\pH)&=0\\
			\Omega^{\mS}|_{V\times V}&=\frac{ 1 }{2}\Omega^{\mS}(H,E)\Omega^0.
		\end{aligned}
	\end{equation}
	hold.
\end{lemma}

\begin{proof}
	Let us write down the condition of $\mK$-invariance of the symplectic form
	\begin{equation}
		\Omega\big( [x_+,a+y_-],a'+y'_- \big)+\Omega\big( a+y_-,[x_+,a'+y'_-] \big)=0.
	\end{equation}
	If we develop $x_+$ and $y_-$, the commutator in the first term becomes
	\begin{equation}
		\big[ \frac{ 1 }{2}(x,x),a+\frac{ 1 }{2}(y,-y) \big]=-\frac{ a }{2}(Dx,-Dx)+\frac{1}{ 4 }\big( [x,y],-[x,y] \big).
	\end{equation}
	The first term is rewritten as $[x,a]_-$, while the second term is $\frac{ 1 }{2}[x,y]_-$. The sum is then $[x,a+\frac{ 1 }{2}y]_-$. Looking at the definition of $\Omega^{\mS}$, the invariance condition reads
	\begin{equation}
		\begin{aligned}[]
			0&=\Omega\big( [x,a+\frac{ 1 }{2}y],a'+y' \big)+\Omega^{\mS}\big( a+y,[x,a'+\frac{ 1 }{2}y'] \big)\\
			&=\Omega^{\mS}\big( -a Dx+\frac{ 1 }{2}\Omega_0(x,y)E,a'+y' \big)+\Omega^{\mS}\big( a+y,-a'Dx+\frac{ 1 }{2}\Omega_0(x_V,y_V')E \big).
		\end{aligned}
	\end{equation}
	If we look at that condition with $a'=0$, $a=1$, $x_V=0$ and $y=0$ and taking into account $Dx=x_V+2x_EE$ we find
	\begin{equation}
		\Omega^{\mS}(2x_EE,y')=0,
	\end{equation}
	so that $\Omega^{\mS}(E,y')$. We conclude that a necessary condition for the invariance is
	\begin{equation}
		\Omega^{\mZ}(E,\pH)=0.
	\end{equation}
	Now if we consider $y'\in V$, $x_E=0$ and $x_V\neq 0$, we find
	\begin{equation}
		\Omega^{\mS}|_{V\times V}=\frac{ 1 }{2}\Omega^{\mS}(H,E)\Omega_0.
	\end{equation}
	
	One can check that these two conditions insure the $\mK$-invariance of $\Omega^{\mS}$.
\end{proof}

To each non degenerate form satisfying these two conditions corresponds a symplectic triple $(\mG_0,\sigma_0,\Omega)$.

%---------------------------------------------------------------------------------------------------------------------------
\subsection{Realization as coadjoint orbit}
%---------------------------------------------------------------------------------------------------------------------------

Let $(M=G/H,s,\omega)$ be a symmetric symplectic space. We are going to study under which conditions we can realise $M$ as a coadjoint orbit, i.e. we want the two conditions
\begin{enumerate}

	\item
		there exists a $\xi_0\in\mG^*$ such that $\Stab_G(\xi_0)=H$ where $\Stab$ stands for the stabilizer for the coadjoint action of $G$. Let 
		\begin{equation}
			\begin{aligned}
				\Phi\colon M&\to \Ad^*(G)\xi_0=\mO \\
				[g]&\mapsto \Ad^*(g)\xi_0 
			\end{aligned}
		\end{equation}
		be the identification between $M$ and the coadjoint orbit $\mO$.
	\item
		The identification $\Phi$ fits the symplectic structures:
		\begin{equation}
			\Phi^*\omega^{\mO}=\omega
		\end{equation}
		where $\omega^{\mO}$ is the canonical symplectic structure on the coadjoint orbit given by \eqref{eq_omega_Gs}.
\end{enumerate}

\begin{definition}
	A \defe{good polarization}{polarization!good} associated to $\xi_0$ is a Lie subalgebra $\mB$ of $\mG$ which is maximal for the property $\delta\xi_0|_{\mB\times \mB}\equiv 0$ where the alternate bilinear $2$-form $\delta\xi_0$ on $\mG$ is defined by
	\begin{equation}
		\delta\xi_0=\langle \xi_0, [.,.]\rangle.
	\end{equation}
\end{definition}

If $\mB$ is a good polarization, we consider $B=\exp(\mB)$ and we have a representation $\chi\colon \mB\to \gU(1)$ given by
\begin{equation}
	\begin{aligned}
		\chi\colon \mB&\to \gU(1) \\
		\exp(y)&\mapsto  e^{i\langle \xi_0, y\rangle }. 
	\end{aligned}
\end{equation}
It turns out that $\chi$ is a representation even when $\mB$ is non abelian. Indeed, if $x,y\in\mB$, we have
\begin{equation}
	\chi( e^{x} e^{y})=\chi( e^{x+y+W})= e^{i\langle \xi_0, x+y+W\rangle }= e^{i\langle \xi_0, x+y\rangle }= \chi( e^{x})\chi( e^{y})
\end{equation}
where $W$ is a combination if commutators of $x$ and $y$ (Campbell-Baker-Hausdorff) so that by definition of $\mB$, $\langle \xi_0, W\rangle =0$.

Since we are in the hypothesis (see subsection \ref{SubSecUnitInducedPrep}), we can define the induced unitary representation
\begin{equation}
	U\colon G\to \gU(\hH_{\chi})
\end{equation}
where $\hH_{\chi}=L^2(Q,dq)$. Let $dg$ be the left invariant Haar measure on $G$.  To each $u\in L^1(G,dg)$, we make correspond an operator $U(u)$ on $\hH_{\chi}$ given by
\begin{equation}	\label{EqDefUudansHh}
	\langle U(u)\varphi, \psi\rangle =\int_Gu(g)\langle U(g)\varphi, \psi\rangle dg
\end{equation}
for every $\varphi,\psi\in\hH_{\chi}$. Let us prove that this integral exists. We have
\begin{equation}	\label{EqIntdefUgGdgvppsi}
		| \langle U(g)\varphi, \psi\rangle  |\leq\int_G| u(g) | |\langle U(g)\varphi, \psi\rangle  |dg,
\end{equation}
but the Cauchy-Schwartz inequality shows that $| \langle U(g)\varphi, \psi\rangle  |\leq| U(g)\varphi | |\psi |=| \varphi | |\psi |$, so that the integral in \eqref{EqIntdefUgGdgvppsi} is smaller than
\begin{equation}
	| \varphi | | \psi |  \int_G| u(g) |dg
\end{equation}
which exists because we supposed $u\in L^1(G,dg)$.

\begin{probleme}
	The following paragraph can be more precise.
\end{probleme}

We can rewrite the definition \eqref{EqDefUudansHh} using the measure theory given around section \ref{sec_distrib_mesure}. Indeed the space $\opB(\hH)$ of bounded operators\footnote{For linear operators on Hilbert spaces, the fact to be bounded is equivalent to continuity.} on the Hilbert space $\hH$ is endowed with the operator norm for which $\opB(\hH)$ becomes a normed algebra ($\| AB \|_{op}\leq\| A \|_{op}\| B \|_{op}$). The unitary group $\gU(\hH)$ is a subalgebra (because it is closed for the composition), so that one can consider, for each function $u$, the function
\begin{equation}
	\begin{aligned}
		G&\to \opB(\hH) \\
		g&\mapsto u(g)U(g) 
	\end{aligned}
\end{equation}
and its integral
\begin{equation}
	\int_G u(g)U(g)dg
\end{equation}
which is an element in $\opB(\hH)$. This integral is well defined in $\opB(\hH_{\chi})$ because
\begin{equation}
	\| \int_G u(g)U(g)dg \|_{op}\leq\int_G | u(g) |\cdot \| U(g) \|_{op}dg=\| u \|_{L^1}.
\end{equation}

Using the measure theory, one can prove that
\begin{equation}
	\langle U(u)\varphi, \psi\rangle =\langle  \big( \int_G u(g)U(g)dg \big)\varphi , \psi\rangle.
\end{equation}

What we build up to here is a map
\begin{equation}
	\begin{aligned}
		U\colon L^1(G,dg)&\to \opB(\hH_{\chi}) \\
		u&\mapsto U(u) 
	\end{aligned}
\end{equation}
given by
\begin{equation}
	U(u)=\int_G u(g)U(g)dg.
\end{equation}
This map is linear and continuous because $\| U(u) \|_{op}\leq\| u \|_{L^1}$.

We are now going to use the symmetry on $M$ in order to descend $U$ from $L^1(G,dg)$ to $L^1(M)$. Let us take a look at the two projections from $G$:
\begin{equation}
	\xymatrix{%
	G \ar[r]^{\pi^M}\ar[d]_{\pi^Q}		&	G/H	\\
	   G/B
	   }
\end{equation}
If $X,Y\in\mH$, we recall that the definition of $\ad(X)^*$ is
\begin{equation}
	\langle \xi_0, [X,Y]\rangle =-\langle \ad(X)^*\xi_0, Y\rangle,
\end{equation}
but, since $ e^{tX}\in\Stab(\xi_0)$, we have $\Dsdd{ \Ad(\exp(tX))^*\xi_0 }{t}{0}=0$, thus
\begin{equation}
	\langle \ad(X)^*\xi_0, Y\rangle =0
\end{equation}
and we can suppose that the good polarization $\mB$ contains $\mH$. In that case we have the well defined map
\begin{equation}
	\begin{aligned}
		\tilde\pi\colon G/H&\to G/B \\
		gH&\mapsto gB 
	\end{aligned}
\end{equation}
This is well defined because, since $H\subset B$, we have
\begin{equation}
	\tilde\pi(ghH)=ghB=gB
\end{equation}
for every $h\in H$.

\begin{lemma}
	The map $\tilde\pi$ is a submersion.
\end{lemma}
\begin{proof}
	No proof.
\end{proof}
The following diagram commutes:
\begin{equation}
	\xymatrix{%
	G \ar[r]^{\pi^M}\ar[d]_{\pi^Q}		&	G/H\ar[dl]^{\tilde\pi}\\
	   G/B
	   }
\end{equation}

Still two assumptions about $\sigma$:
\begin{enumerate}
	\item
		we suppose that $B$ is stable under $\sigma$,
	\item
		we suppose that $\xi_0$ is $\sigma$-invariant, that is $\xi_0(\sigma X)=\xi_0(X)$.
\end{enumerate}
The second assumption is easy to fulfill. If $\xi_0$ is not $\sigma$-invariant, we consider
\begin{equation}
	\xi_0'=\frac{ 1 }{2}(\xi_0+\sigma^*(\xi_0))
\end{equation}
instead.

Now, the symmetry
\begin{equation}
	\begin{aligned}
		\sigma_H\colon M&\to M \\
		gH&\mapsto \sigma(g)H 
	\end{aligned}
\end{equation}
descends to $Q$ as
\begin{equation}
	\begin{aligned}
		\underline\sigma\colon Q&\to Q \\
		gB&\mapsto\tilde\pi\big( \sigma_H(gH) \big)=\sigma(g)B.
	\end{aligned}
\end{equation}
The so defined map $\underline\sigma$ is well defined because
\begin{equation}
	\underline\sigma(gbB)=\sigma(gb)B=\sigma(g)\sigma(b)B=\sigma(g)B=\underline\sigma(gB).
\end{equation}

\begin{lemma}
	Using the hypothesis of $\sigma$-invariance of $\xi_0$, we have that
	\begin{equation}
		\sigma^*\colon  C^{\infty}(G,\eC)^B\to  C^{\infty}(G,\eC)^B,
	\end{equation}
	the image of a $B$-equivariant function on $G$ by $\sigma^*$ is still $B$-equivariant.
\end{lemma}

\begin{proof}
	Let $\hat\varphi\in C^{\infty}(G,\eC)^B$, then we have
	\begin{equation}
		\begin{aligned}[]
			(\sigma^*\hat\varphi)(gb)&=\hat\varphi\big( \sigma(g)\sigma(b) \big)\\
			&=\chi\big( \sigma(b)^{-1} \big)\hat\varphi\big( \sigma(g) \big)\\
			&= e^{-i\langle \xi_0, \sigma\log(b)\rangle }\hat\varphi(\sigma g)\\
			&= e^{-i\langle \xi_0, \log(b)\rangle }(\sigma^*\hat\varphi)(g)	&\text{because $\xi_0(\sigma X)=\xi_0(X)$}\\
			&=\chi(b^{-1})(\sigma^*\hat\varphi)(g).
		\end{aligned}
	\end{equation}
\end{proof}

Since the measure $dq$ is $\sigma^*$-invariant by hypothesis, we have
\begin{equation}
	\int_Q\overline{ (\underline\sigma^*u) }(q)(\underline\sigma^*v)(q)dq=\int_Q\overline{ u(q') }v(q)\underline\sigma^*dq=\int_Q\overline{ u(q') }v(q)dq
\end{equation}
where we used the change of variable $q'=\sigma q$. A consequence is that $\underline\sigma^*$ is an involution
\begin{equation}
	\underline\sigma^*\colon L^2(Q,dq)\to L^2(Q,dq).
\end{equation}
Since, in an abstract way, we denoted $L^2(Q,dq)$ by $\hH_{\chi}$, we denote by $\Sigma$ the involution $\underline\sigma^*$ on $\hH_{\chi}$. Now we consider the function
\begin{equation}
	\begin{aligned}
		\Omega\colon G&\to \gU(\hH_{\chi}) \\
		g&\mapsto U(g)\Sigma U(g^{-1}) 
	\end{aligned}
\end{equation}
which is a composition of unitary maps. This is not a representation of the group $G$, but we have
\begin{equation}
	\Omega(gh)=\Omega(g)
\end{equation}
for every $h\in H$ and $g\in G$. Indeed let us compute $\widehat{\Omega(gh)\varphi}$ for $\varphi\in\cdD(Q)$. We have
\begin{equation}		\label{EqwOshvhvkl}
		\widehat{\Omega(gh)\varphi}=\hat U(gh)\sigma^*\hat U(h^{-1}g^{-1})\hat\varphi
		=\hat U(h)\hat U(h)\sigma^*\hat U(h^{-1})\hat U(g^{-1})\hat\varphi.
\end{equation}
The element $h$ only appears in the combination $\hat U(h)\sigma^*\hat U(h^{-1})$, so let us see how it acts on an equivariant function $\hat \varphi$. If we evaluate it on $g_0$ we find
\begin{equation}		\label{EqbhUsigmastargzi}
		\big( \hat U(h)\sigma^*\hat U(h^{-1})\hat\varphi \big)(g_0)=\big( \sigma^*\hat U(h^{-1})\hat\varphi \big)(h^{-1}g_0)
		=\hat\varphi\big( h\sigma(h^{-1}g_0) \big).
\end{equation}
Let us recall that we are in the context\footnote{With many notational incoherences.} of subsection \ref{SubSecInducrepresBBGC}: the representation $U$ on $\cdD(Q)$ comes from the regular left representation $\hat U$ on $ C^{\infty}(G,\eC)^B$. Thus we have $\big( \hat U(g)\hat\varphi \big)(g_0)=\hat\varphi(g^{-1}g_0)$. Equation \eqref{EqbhUsigmastargzi} is thus equal to
\begin{equation}
	\hat\varphi\big( h\sigma(h^{-1}g_0) \big)=\hat\varphi\big( h\sigma(h^{-1})\sigma(g_0) \big)=\hat\varphi\big( \sigma(g_0) \big)=(\sigma^*\hat\varphi)(g_0),
\end{equation}
so that equation \eqref{EqwOshvhvkl} does not depend on $h$, which proves that
\begin{equation}
	\Omega(gh)=U(g)\Sigma U(g^{-1})=\Omega(g).
\end{equation}
One consequence of this circumstance is that $\Omega$ is a function which pass to the quotient $G\to G/H$. Thus we consider the map
\begin{equation}
	\Omega\colon M=G/H\to \gU(\hH_{\chi})\subset\opB(\hH).
\end{equation}

Since $M$ is a symplectic manifold, we have a natural volume form
\begin{equation}
	dx=\frac{1}{ n! }\omega^n
\end{equation}
where $n=\frac{ 1 }{2}\dim M$. This measure allows us to consider the map
\begin{equation}
	\begin{aligned}
		\Omega\colon L^1(M,dx)&\to \opB(\hH) \\
		u&\mapsto \Omega(u) 
	\end{aligned}
\end{equation}
defined by
\begin{equation}
	\Omega(u)=\int_M u(x)\Omega(x)dx
\end{equation}
which is a continuous linear map. This is not a representation (even on $G$ the initial $\Omega$ was not a representation and $M$ is not a group), but it is an unitary representation of $M$ in the following sense.

\begin{definition}
	An \defe{unitary representation}{representation!of a symmetric space} is a map $\Omega\colon M\to \gU(\hH)$ of the symmetric space $M$ when it satisfies to the properties
	\begin{enumerate}	
		\item
			$\Omega(x)\Omega(y)\Omega(x)=\Omega(s_xy)$
		\item
			$\Omega(x)^2=\id|_{\hH}$.
	
	\end{enumerate}
	for every $x,y\in M$.
\end{definition}
\section{Hermitian and symplectic spaces}
%+++++++++++++++++++++++++++++++++++++++
\label{SecHermEtSymplecticSpaces}

If $ANK$ is the Iwasawa decomposition of a Lie group\quext{c'est pas mal de dire quel genre de groupes : simple ? semi ? compact ?}, one can consider the manifold $M=G/K$. There is a natural identification
\[
   \mP=T_KM
\]
where $\mP$ comes from the Cartan decomposition $\mG=\mP\oplus\mK$. Indeed, the Iwasawa theorem says that $M\simeq AN$ so that a path in it reads $g(t)=a(t)n(t)$ with $g(0)=e$. But one has a diffeomorphism $A\times N\times K\to G$, so that $g(0)=e$ implies $a(0)=n(0)=e$. Thus Leibnitz makes $g'(0)=a'(0)+n'(0)$ and $g'(0)\in\mA\oplus\mN=\mP$.

Let us recall that when $G$ is a Lie group, and $H$ a closed connected subgroup of $G$, $G/H$ is a manifold on which $G$ acts. This structure is an \defe{homogeneous space}{homogeneous!space}. If moreover $H$ is the set of the fixed points of an involution on $G$, $G/H$ is says to be a \defe{symmetric space}{symmetric!space}.

More precisely, the involution is a $\dpt{\theta}{\mG}{\mG}$ which let fixed $\mH\subset\mG$; then $H$ is the connected Lie group whose Lie algebra is $\mH$. All this makes that the $G/K$ from Iwasawa is a symmetric space.

\begin{definition}  \label{DefCLtjFtD}
    Here, $M$ denotes a connected smooth manifold. An \defe{almost complex structure}{almost!complex structure} on $M$ is a $(1,1)$ tensor field $J$ such that $\forall X\in\cvec(M)$,
    \begin{equation}
       (J\circ J)X=-X.
    \end{equation}
    The tensor field $J$ is a \defe{complex structure}{complex structure} when moreover it satisfies the \emph{integrability condition}: $\forall X,Y\in\cvec(M)$,
    \begin{equation}  \label{DefComplStruct}
       N(X,Y):=[X,Y]+J[JX,Y]+J[X,JY]-[JX,JY]=0.
    \end{equation}
\end{definition}
We already spoke about complex structure in order to define the signed curvature of planar curve around the definition \ref{DEFooTSJXooTIyRXf}.
%TODO : Make clearer the definitions \ref{DefCLtjFtD}, \ref{DefSymHermMGKalg} and \ref{DefKONtphK} that are more or less the same.

\begin{definition}		\label{DefSymHermMGKalg}
  The symmetric space $M=G/K$ is \defe{hermitian}{hermitian!symmetric space} if there exists an endomorphism $J\in\End{\mP}$, $\dpt{J}{\mP}{\mP}$ such that
\begin{subequations}  
\begin{align}  
  J^2&=-id_{\mP},                                           \label{eq:herm_1} \\
  B(JX,JY)&=B(X,Y)            && \forall\,X,Y\in\mP,    \label{eq:herm_2}\\
  \ad (k)\circ J&=J\circ\ad(k)&& \forall\,k\in\mK.      \label{eq:herm_3}
\end{align}    
\end{subequations}
\label{def:hermitien}
\end{definition}

Since one has the identification $\mP=T_KM$, $J$ is only defined on $T_{[e]}M$. The following proposition extends the definition.

\begin{proposition} \label{prop:ext_J}
    The hermitian structure $J$ can be extended to a complex structure $\oJ$ on the whole $TM$.
\end{proposition}

\begin{proof} 
 For $X\in T_{[g]}M$, we define
\begin{equation} 
  \oJ(X):=dL_g\circ J\circ dL_{g^{-1}}X.
\end{equation}
where $dL$ is the differential of $\dpt{L_g}{G/K}{G/K}$, $L_g[h]=[gh]$.
From this, $\oJ^2(X)=-X$ because
\begin{equation}
  (\oJ\circ\oJ)X=( dL_g J dL_{g^{-1}} )\circ( dL_g J dL_{g^{-1}} )X
           =dL_g J^2dL_{g^{-1}}X
	   =-X.
\end{equation}
On the other hand, $J$ satisfies $\ad(k)\circ J=J\circ \ad(k)$ and we want the same for $\oJ$: 
\[
  \ad(X)\circ\oJ=\oJ\circ\ad(X) 
\]
for $X\in T_{[g]}M$. Note that it is true for $[g]=[e]$ because $T_{[e]}M=\mK$. Let us consider $X\in T_{[g]}M$, and let us see what is $ \big( (\ad X)\circ J \big)Y $ for a $Y\in T_{[g]}M$. Consider $x$, $y\in T_{[e]}M$ such that $X=dL_g x$ and $Y=dL_g y$. Suppose one has 
\begin{equation}\label{eq:suppose}
   \ad(dL_g x)Y=dL_g\circ\ad(x)(dL_{g^{-1}}Y);
\end{equation}
then one can compute
\begin{subequations}
    \begin{align}
\big(  \ad(X)\circ\oJ \big)Y&=\ad(dL_g x)\circ dL_g\circ J\circ dL_{g^{-1}}Y\\
&=dL_g\ad(x) \circ J\circ dL_{g^{-1}}Y\label{subEqEMyROwA}\\
	                    &=dL_g\circ J\circ\ad(x)\circ dL_{g^{-1}}Y\\
			    &=(dL_g\circ J\circ dL_{g^{-1}})\circ (dL_g\circ\ad(x)\circ dL_{g^{-1}})\\
			    &=\oJ\circ\ad(X)Y.
    \end{align}
\end{subequations}
The line \eqref{subEqEMyROwA} comes from $\ad(x)\circ J=J\circ\ad(x)$ because $x\in T_{[e]}M$.
 
Now, we prove equation \eqref{eq:suppose} which is rewritten in a more convenient way as $[dL_g x,Y]=dL_g[x,dL_{g^{-1}}Y]$. Thus one has to see if for any $x$, $y\in T_{[e]}M$, 
\[
   dL_g[x,y]=[dL_g x,dL_g y].
\]
This is true because of \cite{Helgason}, proposition 3.3, page 34.  Now we know that $\forall X\in T_{[g]}M$ we have $\oJ\circ\ad(X)=\ad(X)\circ\oJ$ and ${\oJ}\,^2X=-X$.  In order to have a complex structure, one also need to check condition \eqref{DefComplStruct}, which is true because
\begin{equation}
\begin{split}
J[JX,Y]&=-\ad Y\circ JJX=\ad(Y)X=[X,Y],\\
J[X,JY]&=J\circ\ad(X)JY=-[X,Y],\\
-[JX,JY]&=-(\ad JX\circ J)Y
        =-J(\ad JX)Y
	=-[Y,X].
\end{split}
\end{equation}

\end{proof}

\noindent If $X\in T_{[g]}M$,
\begin{equation}\nonumber
\begin{split}
  (\oJ\circ dL_h)X&=dL_{hg}\circ J\circ dL_{(hg)^{-1}}dL_h X\\
           &=dL_{hg}\circ J\circ dL_{g^{-1}} X\\
	   &=dL_h\circ dL_g\circ J\circ dL_{g^{-1}} X\\
	   &=(dL_h\circ \oJ)X.
\end{split}
\end{equation}
so we have an important property:
\begin{equation}\label{eq:J_dL}
   \oJ\circ dL_h=dL_h\circ\oJ.
\end{equation}
From now, it is clear that we will often forget the bar on $\oJ$.  In the same way that $J$ extends to $M$, 

\begin{proposition}
For $X$, $Y\in T_{[g]}M$, the formula
\begin{equation}\label{eq:BdL}
  \overline{ B }(X,Y):=B(dL_{g^{-1}}X,dL_{g^{-1}}Y)
\end{equation}
defines a Riemannian metric on $M$.
\end{proposition}

\begin{proof}
One has to see that it is nondegenerate. Say that $Z\in T_{[g]}M$ is such that for any $X$, 
$\overline{ B }(Z,X)=0$. Then $B(dL_{g^{-1}}Z,dL_{g^{-1}}X)=0$. But $dL_g$ is a vector space isomorphism because 
$dL_g(o)=\Dsdd{L_g(X_t)}{t}{0}$ with $X_t$, a constant path at $[e]\in M$.
       
But since $B$ is nondegenerate, the definition \eqref{eq:BdL} says us $dL_{g^{-1}}Z=0$, and then $Z=0$.
\end{proof}

Now, one knows\quext{Cf cours de géométrie symplectique} that 
\begin{equation}
  \omega^M_x(X,Y)=g_x(JX,Y)
\end{equation}
defines a $G$-invariant symplectic structure on $M$.

In order to see it, one has to show that $(M,g,J)$ is  a Kähler structure. The $G$-invariance comes from the extension of Killing form that we had chosen: $\forall X,Y\in T_{[g]}M$, $B_{[g]}(X,Y)=B(dL_{g^{-1}}X,dL_{g^{-1}}Y)$.
It is clear that 
\begin{equation}
B_{[hg]}(dL_hX,dL_hY)=B_{[g]}(X,Y).
\end{equation}
From this and equation \eqref{eq:J_dL}, one can see the $G$-invariance of $\omega^M$:
\begin{equation}
\begin{split}
   \omega^M_{[hg]}\big((dL_h)_{[g]}X, (dL_h)_{[g]}Y\big)&=B_{[hg]}(JdL_hX,dL_hY)
                                                =B_{[hg]}(dL_h J X,dL_hY)\\
						&=B_{[g]}(JX,Y)
						=\omega^M_{[g]}(X,Y).   
\end{split}
\end{equation}
The formulation of the $G$-invariance is
\begin{equation}
   \omega^M_{[hg]}\Big( (dL_h)_{[g]}X, (dL_h)_{[g]}Y \Big)=\omega^M_{[g]}(X,Y).
\end{equation}

\subsection{The Chevalley cohomology}
%------------------------------------

Let $\mG$ be a Lie algebra (maybe infinite dimensional) and $(V,\rho)$ a representation of $\mG$ on the vector space $V$. The \defe{Chevalley cohomology}{chevalley!cohomology} of $\mG$ associated with the representation $\rho$ is given by the following definitions:

A $p$-cochain is a map $\dpt{C}{\underbrace{\mG\times\ldots\times\mG}_{\text{$p$ times}}}{V}$ which is multi-linear and skew-symmetric. In particular, a $1$-cochain is a linear map $\dpt{\xi}{\mG}{V}$. In the case of the trivial representation on $\eR$, a $1$-cochain is an element of $\mG^*$. The coboundary of a $p$ cochain is the $p+1$-cochain given by
\begin{equation}
\begin{split}
(\delta C)(X_0,\ldots,X_p)=&\sum_{i=0}^{p}(-1)^i\rho(X_i)C(X_0\ldots,\hX_i,\ldots,X_p)\\
                           &+\sum_{i<j}(-1)^{i+j}C\big(  [X_i,X_j],\ldots,\hX_i,\ldots,\hX_j,\ldots,X_p \big).
\end{split}
\end{equation}
The main property is $\delta^2=0$. The others definitions are as usual: a $p$-cocycle is a $p$-cochain $C$ such that $\delta C=0$, a $p$-coboundary is a $p$-cochain which can be written as $\delta B$ for some  $(p-1)$-cochain $B$. Finally, the cohomology classes are:
\begin{equation}
H^{p}_{(V,\rho)}=\frac{\text{$p$-cocycles}}{\text{$p$-cochain}}=H^p_{\rho}(\mG,V).
\end{equation}

When one consider the trivial representation, i.e. $\rho(X)=0$, a $1$-cochain is $\xi\in\mG^*$ and
\begin{equation}  \label{EqDefcochaintrivC}
(\delta\xi)(X,Y)=-\xi([X,Y]).
\end{equation}

\begin{probleme}
Au cas où ça t'intéresserait, je te dis que le signe moins, tu ne l'as ajouté qu'en février 2007. T'étonnes pas si y'a des signes qui foirent plus bas.
\end{probleme}

Now, on the symmetric hermitian space $M=G/K$, one defines a $\Omega\in\Lambda^2(\mG^*)$ by
\begin{subequations}
\begin{align}
   \Omega(X,Y)&=B(JX,Y)&\text{for $X,Y\in\mP$}  \label{eq:def_Omega_1}    \\
   \Omega(\mK,\mG)&=0.                          \label{eq:def_Omega_2} 
\end{align}
\end{subequations}   

A great property of this definition is that $\Omega$ is a $2$-cocycle for the trivial representation of $\mG$ on $\eR$:
\[
\Omega([X,Y],Z)+\Omega([Y,Z],X)+\Omega([Z,X],Y)=0.
\]

Indeed, if $X$ ,$Y$, $Z\in\mP$, the commutators are in $\mK$, so that \eqref{eq:def_Omega_2} makes the whole null. The second case is $X$, $Y\in\mP$ and $Z\in\mK$; for this, we are led to consider the quantity $-B( [Y,Z],JX )-B([Z,X],JY)$. The first term can be transformed as:
\[
\begin{aligned}
  B([Y,Z],JX)&=-B(J[Y,Z],X)&&\text{by def. \eqref{eq:herm_2} } \\
             &=B([Z,JY],X)&&\text{by def.  \eqref{eq:herm_3}}\\
	     &=-B(JY,[Z,X])&&\text{$\ad$-invariance of $B$}\\
	     &=-B([Z,X],JY).
\end{aligned}
\]
So it is zero.

\begin{lemma}[Whitehead's lemma]
If $\mG$ is  a finite dimensional semisimple  Lie algebra and $\rho$ a non trivial\quext{Ce qui n'est pas le cas ici} representation of $\mG$ on $V$, then $\forall\,q\geq 0$,
\[
      H^q(\mG,V)=0.
\]
\end{lemma}

This gives us the existence of a $\xi_0\in\mG^*$ (an Chevalley $1$-cochain) such that 
\[
   \delta\xi_0=\Omega.
\]
\begin{equation}\label{eq:Z_0}
   \xi_0=B(Z_0,.).
\end{equation}

\begin{definition}
	The \defe{center}{center!of a Lie algebra}\nomenclature{$\mZ(\mG)$}{Center of a Lie algebra} of the Lie algebra $\mG$ is the set $\mZ(\mG)\subset\mG$ of elements $Z$ such that $[Z,X]=0$ for every $X\in\mG$. See also the definition of a centralizer on page \pageref{PgDefCentralisateur}.
\end{definition}

\begin{proposition}
The $Z_0$ defined in \eqref{eq:Z_0} and the $J$ of proposition \ref{prop:ext_J} satisfy
\begin{subequations}
\begin{align}
   Z_0&\in\mZ(\mK),\\
   J&=\pm\ad(Z_0)|_{\mP}.
\end{align}   
\end{subequations}


\end{proposition}

\begin{proof}
First, we see that for any $K\in\mK$, $[Z_0,K]=0$. We know from \eqref{eq:def_Omega_2} that $\forall\,G\in\mG$, $K\in\mK$, $\Omega(K,G)=0$, or
\begin{equation}
\begin{split}
  0=\delta B(Z_0,.)(K,G)=-B(Z_0,.)([K,G])
                       =B([K,Z_0],G),
\end{split}
\end{equation}
thus $[K,Z_0]=0$ because $B$ is nondegenerate. We will see below that $Z_0\in\mP$ is not possible.  On the other hand, the condition \eqref{eq:def_Omega_1} gives us
\[
  B( [X,Z_0],Y )=B(JX,Y)
\]
for any $Y\in\mP$. Thus $[X,Z_0]=JX$ and the second claim follows. Let us now see that $Z_0\in\mP$ is not possible (and so we finish the proof of the first claim). We know that $J^2X=[Z_0,[Z_0,X]]$, but for $Z_0$, $X\in\mP$, $[Z_0,X]\in\mK$ and so $J^2X=0$ which is not possible.

\end{proof}


\begin{lemma}
A symmetric space $G/K$ is hermitian if and only if $\mZ(\sK)\neq 0$.
\end{lemma}

\begin{proof}
If the space is hermitian, we just said that the $J$ can be written under the form $J=-\ad(Z_0)|_{\sP}$ for a $Z_0\in\mZ(\sK)$. For the sufficient condition, we define $J=-\ad(Z_0)$ for a certain $Z_0\in\mZ(\sK)$. As a first point for all $k\in\sK$ and $p\in\sP$,
\begin{equation}
\begin{split}
(J\circ\ad k)p=[ [k,p],Z_0]
              =-[ [p,Z_0],k]-[ [Z_0,k],p]
              =[k,[p,Z_0]]
              =(\ad k\circ J)p
\end{split}
\end{equation}
The two other points are
\begin{subequations}
\begin{align}
  J^2=(-\ad Z_0)[X,Z_0]=-[ [Z_0,X],Z_0]
\intertext{and}
  B([Z_0,X],[Z_0,Y])=-B(X,[ Z_0,[Z_0,Y]])=B(X,Y)
\end{align}
\end{subequations}
These are true if $[ [Z_0,X],Z_0]=X$\quext{Mais je ne vois pas comment obtenir \c ca. Si $\ad Z_0$ est un automorphisme de $\sP$, alors je suis d'accord.}

Let us prove that $\mM:=\ker(\ad Z_0)=0$. For remark that $(\sG,B)$ is a Riemannian space and let $W$ be the orthogonal complement of $\mM$ in $\sP$. We begin to prove that $\mM$ is $(\ad\sK)$-invariant.

If $x\in\ket Z_0$ and $k\in \sK$, then
\[ 
  [Z_0,[k,x]]=-[k,[x,Z_0]]-[x,[Z_0,k]]=0
\]
because $[x,Z_0]=0=[Z_0,k]$. Now if $\mM$ is $(\ad\sK)$-invariant, then $W=\mM^{\perp}$ is too because
\[ 
  B([k,x],m)=-B(x,[k,m])=0
\]
since $[k,x]\in\mM$. So we have the orthogonal direct sum $\sP=\mM\oplus W$. We are now going to see that $[\mM,W]=0$. Let $X,X'\in\sP$;
\begin{equation}
  B\big( [ [m,w],X],X' \big)=B\big( [m,w],[X,X'] \big)
		=B\big( w,[ [X,X'],m] \big)
		=0
\end{equation}
since $[X,X']\in\sK$ and $[ [X,X'],m]\in\mM$ from the $\sK$invariance of $\mM$. If we define $A=[m,w]$, the endomorphism $\ad(A)|_{\sP}$ is zero.

We know\quext{Il faut encore voir d'où sort ce truc.} that $[\sK,\sK]=\sP$, and then that
\[ 
  [A,k]=[A,[p,p']]=-[p,[p',A]]-[p',[A,p]]=0.
\]
So $[A,\sG]=0$ and $A=0$ because $\sG$ is semisimple. If we write $\sG=[\sP,\sP]\oplus\sP$, we find
\[ 
  \sG=\big( [\mM,\mM]\oplus\mM \big)\oplus\big( [W,W]\oplus W \big),
\]
where the two brackets commute. It furnish a decomposition of $\sG$ into ideals which impossible from the semi-simplicity assumption. We conclude that $\mM=$ and that $\ad Z_0$ is bijective on $\sP$.

\end{proof}


\begin{lemma}
Let $\sG$ a simple Lie algebra with Iwasawa decomposition $\sG=\sK\oplus\sA\oplus\sN$. We suppose that $\mZ(\sK)\neq 0$. Then $\dim\sA\geq\dim\mZ(\sN)$.
\end{lemma}

\begin{proof}
Let $\dpt{i}{\sR}{\sG}$ be the canonical projection and $\xi_0\in\sG^*$ such that $\delta\xi_0=\Omega$. Since $\sK\cap \sR=\{ 0 \}$, the radical of $\delta(i^*\xi_0)$ is trivial. Indeed, when $X\in\sR$, we have $(i^*\xi_0)X=\xi_0 X$ and equation $\delta(i^*\xi_0)(X,Y)=0$ for all $X$, $Y\in \sR$ gives $B(JX,Y)=0$ because $\sK\cap\sR=\{ 0 \}$. Then $JX=0$ and $X=0$.\quext{\c Ca demande que $B$ soit non d\'eg\'en\'er\'ee sur $\sR$, et je ne vois pas trop pourquoi ce serait vrai.}.

Let $V$ be the radical of $\Omega$ in $\sN$; if $z\in\mZ(\sN)$, then $\Omega(\sN,z)=(\delta\xi_0)(z,\sN)=\xi_0[z,\sN]=0$. Then $\mZ(\sN)\subset V$. Now let us consider the map $\dpt{\psi}{V}{\sA^*}$,
\[ 
  \psi(v)=\Omega(v,.)|_{\sA}.
\]
Let us prove that $\psi$ is injective. For, we consider a $v\in V$ such that $\Omega(v,\sA)=0$. Since $v\in V$, we have $[v,\sN]=0$ and then $\Omega(v,\sN)=0$. So,
\[ 
  0=\Omega(v,\sA\oplus\sN)=\delta(i^*\xi_0)(v,\sR),
\]
 and then $v=0$ because the radical of $i^*\xi_0$ in $\sR$ is only zero. Consequently,
\[ 
  \dim\sA=\dim\sA^*\geq \dim V\geq\dim\mZ(\sN)
\]
because there exists an injection from $V$ into $\sA^*$ and $\mZ(\sN)\subset V$.

\end{proof}


\begin{lemma}
Let us suppose that $\dim\sA=1$ and $\dim\sG\geq 3$. Then

\begin{enumerate}
\item The root system is $\Phi=\{ \pm\alpha,\pm 2\alpha \}$,
\item $\sN=\sG_{\alpha}\oplus\sG_{2\alpha}$ and $\sG_{2\alpha}=\mZ(\sN)$
\item $\dim\mZ(\sN)=\dim\sA=1$
\item There exists a $E\in\mZ(\sN)$ such that $[x,y]=\Omega(x,y)E$ for all $x$, $y\in\sN$. The subspaces $\sA\oplus\sN$ and $\sG_{\alpha}$ are symplectic and orthogonal in $(\sR,\Omega)$. In particular, $\sN$ is an Heisenberg algebra.
\end{enumerate}

\end{lemma}

\begin{proof}
No proof.
\end{proof}

This lemma allows us to parametrize $\sR$ as
\[ 
  r=aA+x+zE
\]
with $x\in\sG_{\alpha}$ and $a\in \sA$ because $\mZ(\sN)$ is spanned by the unique element $E$. Now if we consider a function $u\in C^{\infty}(\sR)$, we can define a partial Fourier transform
\[ 
  F(u)(a,x,\xi)=\hat u(a,x,\xi)=\int_{\mZ(\sN)}e^{-i\xi z}u(aA+x+zE)dz.
\]

\begin{theorem}
Let consider the diffeomorphism $\dpt{\phi_{\hbar}}{\sR}{\sR}$ given by
\[ 
   \phi_{\hbar}(a,x,\xi)=\left( a,\frac{1}{\cosh(\frac{\hbar\xi}{2})}x,\frac{\sinh(\hbar\xi)}{\hbar} \right).
\] 
Then 

\begin{enumerate}
\item $\phi_{\hbar}^*\swS(\sR)\subset\swS(\sR)$,
\item $(\phi^{-1}_{\hbar})^*\swS(\sR)\subset\swS'(\sR)$.
\end{enumerate}
where $\swS$ and its dual $\swS'$ are defined in section \ref{sec:Distrib}.

\end{theorem}

We recall the notation for functions: $\varphi^*f=f\circ\varphi$.

\begin{proof}
For sake of simplicity, we forget about variable $a$, we pose $y=\hbar \xi$ and we look at the function $\dpt{\phi}{\eR^2}{\eR^2}$ given by
\[ 
  \phi(x,y)=(\sech(\frac{y}{2})x,\sinh(y)).
\]
Formula
\[ 
  \frac{\sqrt{2}}{2}(1+\sqrt{1+y^2})^{1/2}=\cosh\big( \frac{\arcsinh(y)}{2} \big),
\]
allows us to write
\begin{equation}
\phi^{-1}(x,y)=\left( \cosh\Big( \frac{\arcsinh(y)}{2} \Big)x,\arcsinh(y) \right).
\end{equation}
We pose $p_{nm}(x,y)=x^ny^m$ and we are going to study 
\[ 
  (p_{nm}\circ\phi^{-1})(x,y).
\]
It has a polynomial grown because, for large $y$, $\sinh(\ln y)=\frac{1}{2} y$. Hence $\arcsinh(y)\simeq \ln(2 y)$. The matrix of $d\phi_{(x,y)}$ is given by
\begin{equation}
d\phi_{(x,y)}=
\begin{pmatrix}
\sech(\frac{y}{2}) & -\frac{x}{2}\tanh(\frac{y}{2})\sech(\frac{y}{2})\\
0                  &   \cosh(y)
\end{pmatrix}.
\end{equation}
Since $\phi$ is a diffeomorphism, and then is bijective,
\begin{equation}
\begin{split}
  \sup_{a\in\eR^2}| p_{nm}(a)(u\circ \phi)(a) |&=\sup_{a\in\eR^2}| p_{nm}(\phi^{-1}(a))u(a) |\\
                                               &\leq \sup_{a\in\eR^2}| P_{MN}(a)u(a) |
\end{split}
\end{equation}
for a choice of $N$, $M\in\eN$. In order to check the derivatives, we need the asymptotic behaviour of $(u\circ\phi)'(x,y)$ given components of\quext{Pour moi, la composante $a=i=2$ ne fonctionne pas parce que c'est $(\partial_2)_{\phi(x,y)}\cosh(y)$.}
\[ 
  \partial_a(u\circ\phi)(x,y)=\sum_i(\partial_iu)_{\phi(x,y)}(\partial_a\phi_i)(x,y).
\]
The derivatives of $\phi^*u$ are the quantities
\[ 
  \partial_a(u\circ\phi^{-1})(x,y)=(\partial_iu)(\phi^{-1}(x,y))\partial_a(\phi_i^{-1})(x,y).
\]
This has a polynomial behaviour. One can see recursively that the same is true for second derivatives $\partial^2_{ab}(u\circ\phi^{-1})$ and higher. This proves that $\phi^*\swS(\eR^2)\subset\swS(\eR^2)$.

In order to see that $(\phi^{-1})^*u\in\swS'$ when $u\in\swS$, we have to prove that
\begin{equation} \label{eq:r1181205}
  \int_{\mU}| x^{-N}y^{-M}(\phi^{-1})^*u(x,y)dxdy |<\infty
\end{equation}
for a choice of $N,M$. Here, $\mU$ is the complement in $\eR^2$ of a compact neighbourhood of the origin. Indeed, the fact for $f$ to belongs to $\swS'$ is the \emph{distribution} $T_f$ to belongs to $\swS'$. In other words, the condition $f\in\swS'$ is the continuity of $\varphi\to\int_X f\varphi$ when $\varphi\in\swS$. The essentially resides in the existence of the integral.

In general -- here, $f$ take the role of $\phi^*u$ -- we have
\[ 
  | \int_X f\varphi |\int | f\varphi |\leq \int | f p_{-N,-M} |
\]
for all $N,M\geq 0$ because $\varphi$ decrease more rapidly than any polynomial. If we find $M$ and $N$ such that $\int | fp_{-N,-M} |<\infty$, then we prove that the distribution belongs to $\swS'$. In our case more precisely, we know that $\varphi$ is smooth. Then it can be majored in any compact set. This is the reason why we write an integral over $\mU$ instead of the whole $\eR^2$.

In equation \eqref{eq:r1181205}, we perform the change of variable $a'=\phi^{-1}(a)$:
\begin{equation}
\begin{split}
\int_{\mU}| x^{-N}y^{-M}(u\circ\phi^{-1})&(a)|\,da=\int_{\mU'}| \frac{1}{p_{MN}(\phi(a'))}u(a') | |J_{\phi}(a') |\,da'\\
                        &=\int_{\mU'}\frac{| u(a) |}{\left|  \big( \frac{x}{\cosh(\frac{y}{2})} \big)^N\sinh(y)^M  \right|}
                                     \left|  \frac{\cosh(y)\cosh(\frac{y}{2})}{}  \right|da\\
                        &=\int_{\mU'}\left| \frac{1}{x^N}2^{1-M}\cosh(\frac{y}{2})^{N-M}\sinh(\frac{y}{2})^{1-M}\right|\, |u(a) |\,da.
\end{split}
\end{equation}
The latter integral is finite when $M\geq 1$ and $M>N$.\quext{Pierre trouve d'autres choses, mais sa conlusion est la même; comme s'il utilisait un autre formulaire de trigono hyperbolique que moi.}

The same kind of upper bound\quext{Traduction de «majoration»} holds for the derivatives of $u\circ\phi^{-1}$ for which we have to study the behaviour of the inverse matrix $(d\phi)^{-1}$. All this proves that $(\phi^{-1})^*\swS(\eR^2)\subset \swS(\eR^2)$.

\end{proof}

\subsection{Involutive symmetric Lie algebras}
%----------------------------------------------

\begin{definition}
An \defe{involutive Lie algebra}{involutive!Lie algebra} is a doublet $(\mG,\sigma)$ where $\mG$ is a real finite dimensional Lie algebra and $\dpt{\sigma}{\mG}{\mG}$ is an involutive automorphism of $\mG$.
\end{definition}

There are three types of triples $(\mG,\sigma,\Omega)$: 
\begin{enumerate}

	\item
		the symplectic triple,\index{triple!symplectic}
	\item
		the exact triple, \index{triple!exact}
	\item
		the elementary solvable exact triple (ESET). \index{triple!elementary solvable}
\end{enumerate}
In these three types, $(\mG,\sigma)$ is an involutive Lie algebra. The following definitions can be found in \cite{StrictSolvableSym}.

Symplectic triples were already defined in section \ref{SubSecTripleSylple}.
\begin{definition}
An \defe{exact triple}{exact triple} is a triple $(\mG,\sigma,\Omega)$ such that
\begin{enumerate}
\item $\mG\stackrel{\sigma}{=}\mK\oplus\mP$ and $[\mP,\mP]=\mK$,
\item $\Omega$ is a Chevalley $2$-coboundary such that $i(\mK)\Omega=0$ and $\Omega|_{\mP\times\mP}$ is a symplectic structure on $\mP$.
\end{enumerate}
\end{definition}
The exact triple has the following differences compared to the symplectic one:
\begin{itemize}
\item $\Omega$ is a coboundary instead as a cocycle,
\item $\Omega|_{\mP\times\mP}$ is not only nondegenerate, but also symplectic.
\end{itemize}
From definition of a coboundary, in an exact triple, there exists a $\xi\in\mG^*$ such that $\Omega=\delta\xi$.

\begin{definition}
An \defe{elementary solvable exact triple}{elementary!solvable exact triple}\index{ESET} (ESET) is an exact triple $(\mG,\sigma,\Omega)$ such that
\begin{enumerate}
\item The Lie algebra $\mG$ is a split extension of abelian algebras:
\begin{equation}   \label{EqSplitmGABab}
  \mG=\mA\oplus_{\rho}\mB.,
\end{equation}
\item the automorphism $\sigma$ preserves the vector space decomposition $\mG=\mA\oplus\mB$.
\end{enumerate}

\end{definition}

\begin{remark}
When one writes $\mG=\mA\oplus_{\pi}\mB$, one has $\pi\colon \mA\to \Der(\mB)$. This is the inverse convention of the one chosen in the article \cite{StrictSolvableSym}.
\end{remark}

 In the case of an ESET, we have $\mA\cap\mK\subset\mA\cap[\mG,\mG]$ because $\mK$ is equal to $[\mP,\mP]$ and is thus included in $[\mG,\mG]$. But $[\mG,\mG]$ can be $[a,b]$, $[a,b]$ or $[b,b]$. The two latter are zero (because $\mA$ and $\mB$ is abelian) and, by definition of the split extension, $[a,b]=\rho(a)b\in\mB$. So $\mA\cap[\mG,\mG]=0$. Therefore,
\[ 
  \mA\cap\mK=0;
\]
we deduce that $\mA\subset\mP$ and $\mK\subset \mB$. Since $\mK\subset\mB$, we define $\mL$ as the complement:
\[ 
  \mB=\mK\oplus\mL.
\]
In particular, $\mK$ and $\mL$ are abelian.

The dimension\index{dimension!of a symplectic triple} of a triple is the dimension of $\mP$ and two triples $(\mG_i,\sigma_i,\Omega_i)$ are \defe{isomorphic}{isomorphism!of symplectic!triple} if there exists a Lie algebra isomorphism $\dpt{\psi}{\mG_1}{\mG_2}$ such that $\psi\circ\sigma_1=\sigma_2\circ \psi$ and $\psi^*\Omega_2=\Omega_1$.

\subsection{Symplectic symmetric spaces and involutive Lie algebra}
%--------------------------------------------------------------------

Let $(M,\omega,s)$ be a symplectic symmetric space; we associate an involutive Lie algebra $(\lG,\sigma)$ in the following way (we omit some non trivial proofs). Let $o\in M$, $G$ the transvection group and $H$, the stabiliser of $o$ in $\Aut(M,\omega,s)$ and $K=G\cap H$. We consider the map
\begin{equation}
	\begin{aligned}
		\tilde\sigma\colon \Aut(M,\omega,s)&\to \Aut(M,\omega,s) \\
		\tilde\sigma(g)&=s_o\circ g\circ s_o. 
	\end{aligned}
\end{equation}

Let $\mG$ be the Lie algebra of the group $G$ and $\dpt{\sigma}{\mG}{\mG}$ the induced involutive automorphism from $\tilde\sigma$. Now, $(\mG,\sigma)$ is an involutive Lie algebra. We have a natural projection $\dpt{\pi}{G}{M}$ because $H$ stabilises $o$, so that $K=G\cap H$ is the stabiliser of $o$ in $\Aut(M,\omega,s)$ which is transitive on $M$. Then $M=G/K$ as homogeneous spaces. One can see that $(\mG,\sigma,\pi^*(\omega_o))$ is a symplectic triple.

The precise proposition is the following.

\begin{proposition}
There exists a bijection between symplectic simply connected symmetric spaces and symplectic triples. This bijection is given up to isomorphism.
\end{proposition}

\subsection{Symmetric spaces and coadjoint orbits}
%-------------------------------------------------

We are now going to describe $(M,\omega,s)$ as coadjoint orbit on $\mG^*$. When a Lie group $G$ of symplectomorphism acts on a symplectic manifold $(M,\omega)$, we say that the action is weakly Hamiltonian\index{Hamiltonian!action} if there exists $\dpt{\mu_X}{M}{\eC}$ such that $i(X^*)\omega=d\mu_X$. If $\dpt{\mu}{\mG}{ C^{\infty}(M)}$ is a Lie algebra homomorphism, we say that the action is Hamiltonian and we usually write $\lambda$ instead of $\mu$.

\begin{proposition}
Let $(\mG,\sigma,\Omega)$ be a simple triple, $(M,\omega,s)$ the associated symmetric simply connected symplectic space and $G$, the transvection group. Then

\begin{enumerate}
\item The action of the transvection group on $M$ is Hamiltonian if and only if there exists a $\xi\in\mG^*$ with $\Omega=\delta\xi$ for the Chevalley cohomology\index{Chevalley!cohomology}.
\item In this case, $(M,\omega,s)$ is a $G$-equivariant symplectic covering of the coadjoint orbit of $\xi$ in $\mG^*$.
\end{enumerate}

\end{proposition}

\begin{definition}
A \defe{symmetric symplectic space}{symmetric!symplectic space} is a triple $(M,\omega,s)$ where
\begin{itemize}
\item $M$ is a connected smooth ($\Cinf$) manifold,
\item $\omega$ is a symplectic form on $M$,
\item $\dpt{s}{M\times M}{M}$ is a smooth map which we write with the notation $s_x(y):=s(x,y)$.
\end{itemize}
These elements must satisfy the following conditions:
 \begin{enumerate}
 \item $\forall x\in M$, $s_x$ is an involutive symplectic diffeomorphism of $(M,\omega)$ which is called the \defe{symmetry}{symmetry} at $x$,
 \item $\forall x\in M$, $x$ is an isolated fixed point of $s_x$,
 \item $\forall x,y\in M$, $s_xs_ys_x=s_{s_x(y)}$.
\end{enumerate}

\end{definition}


\begin{definition}
Two symplectic symmetric spaces $(M,\omega,s)$ and $(M',\omega',s')$  are \defe{isomorphic}{isomorphism!of symplectic!symmetric spaces} if there exists a symplectic diffeomorphism $\dpt{\varphi}{(M,\omega)}{(M',\omega')}$ such that 
\begin{equation}
  \varphi\circ s_x=s'_{\varphi(x)}\circ\varphi.
\end{equation}

\end{definition}


\begin{definition}
An \defe{exact triple}{exact triple} is a triple $(\mG,\sigma,\Omega)$ such that

\begin{enumerate}
\item $(\mG,\sigma)$ is an involutive Lie algebra with $[\mP,\mP]=\mK$ if $\mG=\mK\oplus\mP$ is the decomposition of $\mG$ with respect to $\sigma$.

\item $\Omega$ is a Chevalley $2$-coboundary such that $i(\mK)\Omega=0$ and $\Omega_{\mP\times\mP}$ is symplectic.

\end{enumerate}

\end{definition}

From definition, there exists a $\xi\in\mG^*$ for which $\Omega=\delta\xi$. We can choose it in such a way that $\xi(\mP)=0$; in this case we say that $\xi\in\mK^*$ by abuse of notation. Indeed, put $\Omega=\delta\xi$ with $\xi=\xi'+\eta'$ where $\xi'\in\mK^*$ and $\eta'\in\mP^*$. If we consider $k\in\mK$ and $B=B_k+B_p\in\mG$, using $i(\mK)\Omega=0$, we find
\[ 
  0=\Omega(k,B)=-\xi'[k,B_k]-\eta'[k,B_p].
\]
Taking $B_k=0$, we find $\eta'[\mK,\mP]=0$ while with $B_p=0$, we find $\eta'[\mK,\mK]=0$. Moreover $\eta'[\mP,\mP]=\eta'\mK=0$. Then an acceptable $\eta'$ must satisfy $\eta'[\mG,\mG]=0$, so that
\[ 
  \Omega[A,B]=-\xi'[A,B]
\]
which proves that the $\eta'$ part of $\xi$ has no importance; we can choose it as zero.


Let $(\mG,\sigma)$ be an involutive Lie algebra associated with a triple $(M,\omega,s)$ with transvection group $G$. If $(\mG,\sigma,\Omega)$ is exact, $\mZ(\mG)\subset \mK$ because $[Z,p]=0$ for all $p\in\mP$ whenever $Z\in\mP\cap\mZ(\mG)$. Then $\Omega(Z,p)=0$ for all $p\in \mP$ which is not possible from non degeneracy of $\Omega$.

\subsection{Elementary solvable symmetric spaces}
%-----------------------------------------------

Let $(M,\omega,s)$ a symmetric space with associated triple $(\mG,\sigma,\Omega)$. The space $M$ is \defe{elementary solvable}{elementary!solvable!exact triple} if

\begin{enumerate}
\item $\mG$ is a split extension  (see subsection \ref{subsec:semi_Lie}) of two abelian algebras $\mA$ and $\mB$,
\item the automorphism $\sigma$ preserves the decomposition $\mG=\mB\oplus\mA$.
\end{enumerate}
Since $\mK=[\mP,\mP]$, we have
\[ 
  \mA\cap\mK\subset\mA\cap[\mG,\mG]=0.
\]
Indeed, let $\dpt{\rho}{\mA}{\Der\mB}$ be the split homomorphism; the commutator on the split extension is defined by
\[ 
  [A,B]=\rho(A)B\in\mB.
\]
Then $[\mG,\mG]\subset\mB$. All this shows that $\mA\subset\mP$. So there exists a $\mL\subset\mP$ such that $\mB=\mK\oplus\mL$. Let us show that $\mL$ is abelian.
\[ 
  0=[\mB,\mB]=[\mK,\mK]+[\mK,\mL]+[\mL,\mK]+[\mL,\mL].
\]
The three first terms are in $\mP$ while the last one is included in $\mK$. The identical annihilation of the sum imposes $[\mL,\mL]=0$.

\subsection{Mid-point map}
%-------------------------

Let us now take an ESET $(\mG,\sigma,\Omega)$ and its corresponding ESSS $(M,\omega,s)$. There exists a $\xi\in\mG^*$ such that $\Omega=\delta\xi$ and we define
\begin{equation}
\begin{aligned}
 \zeta\colon \mA\times\mL&\to \eR \\ 
(a,l)&\mapsto \xi(\sinh(a)l) 
\end{aligned}
\end{equation}
where $\sinh(a)l$ has to be understood as $\frac{ 1 }{2}( e^{\rho(a)}- e^{-\rho(a)})l$ with $\rho(a)\in\End(\mB)$ being the splitting homomorphism of \eqref{EqSplitmGABab}. 
\begin{proposition}
Let $(M,\omega,s)$ be a ESSS and $\omega=\Omega=\delta\xi$ the symplectic form of the corresponding ESET. The \defe{mid-point map}{mid-point map} $M\to M$, $x\mapsto x/2$ defined by 
\[ 
  s_{x/2}o=x
\]
is globally defined if and only if $\phi$ is a diffeomorphism.
\end{proposition}
Notice that the affirmation $\omega=\Omega=\delta\xi$ means that one has a symplectic form $\omega\colon \mA\times\mL\to \eR$,
\[ 
  \omega(a,l)=\Omega(a,l)=-\xi\big( [a,l] \big).
\]

\subsection{Kähler structures}
%-------------------------------

\begin{definition}  \label{DefKONtphK}
    If $M$ posses an almost complex structure\footnote{Definition \ref{DefCLtjFtD}.} $J$ and a Riemannian metric $g$, we say that the metric is \defe{hermitian}{hermitian!metric} when 
    \begin{equation}
       g(JX,JY)=g(X,Y).
    \end{equation}
\end{definition}

Notice that a symmetric space must be hermitian (definition \ref{def:hermitien}), hence equation \eqref{eq:herm_3}, implies the integrability condition. 

\begin{definition}
If one has an almost complex structure with an hermitian metric such that $\nabla J=0$, then $(M,J,g)$ is a \defe{Kähler manifold}{kähler!manifold}.
\end{definition}

\begin{remark}
$\nabla J=0$ reads $\forall X,Y\in M$,
\[
    (\nabla_XJ)(Y)=\nabla_X(JY)-J(\nabla_XY)=0.
\]
\end{remark}

\begin{remark}
By ``$\nabla$''\ we mean the Levi-Civita connection for $g$. In particular it is torsion free:
\[
   \nabla_XY-\nabla_YX=[X,Y].
\]
\end{remark}

\begin{lemma}
If $(M,g,J)$ is a Kähler manifold, then $J$ is integrable.
\end{lemma}

\begin{proof}
From the formula $\nabla_Z(JY)=(\nabla_ZJ)Y+J\nabla_ZY$ and the fact that $\nabla J=0$, we know that 
\begin{equation}\label{eq:inter_1}
  \nabla_Z(JY)=J(\nabla_ZY),
\end{equation}
while the torsion-free condition for $\nabla$ gives
\begin{equation}\label{eq:inter_2}
\nabla_XY-\nabla_YX=[X,Y].
\end{equation}
With these two, we find $\nabla_Z(JY)-\nabla_Y(JZ)=J\nabla_ZY-J\nabla_YZ=J[Z,Y]$.  Writing it with $JZ$ instead of $Z$,
\[
   \nabla_{JZ}(JY)+\nabla_Y(Z)=J[JZ,Y].
\]
The anti-symmetric part of this equation gives
\[
   \nabla_{JZ}(JY)+\nabla_YZ-J[JZ,Y]-\nabla_{JY}(JZ)-\nabla_ZY+J[JY,Z].
\]
Using \eqref{eq:inter_1} and \eqref{eq:inter_2}, one finds the thesis.

\end{proof}

When $(M,g,J)$ is a Kähler manifold, one defines the \defe{Kähler $2$-form}{kähler!$2$-form} by
\begin{equation}
\omega(X,Y):=g(X,JY).
\end{equation}

\begin{proposition}
The Kähler $2$-form is a symplectic structure on $M$.
\end{proposition}

\begin{proof}
Since $g$ is nondegenerate and $JX=0$ implies $X=0$, it is clear that $\omega$ is nondegenerate. The antisymmetry of $\omega$ is because the metric is hermitian. The only point is to see that $d\omega=0$.

From \eqref{eq:d_omega_nabla} which gives $d\omega$ in terms of $\nabla\omega$, we see that we just have to prove that $\nabla\omega=0$. By definition,
\begin{align*}
(\nabla_Z\omega)(X,Y)&=Z(\omega(X,Y))-\omega(\nabla_ZX,Y)-\omega(X,\nabla_ZY)\\
                     &=Zg(X,JY)-g(\nabla_ZX,JY)-g(X,J\nabla_ZY).\\
                     &=(\nabla_Zg)(X,JY)=0
\end{align*}		     
because the vanishing of $\nabla J$ implies that $J(\nabla_ZY)=\nabla_Z(JY)$.
\end{proof}

\subsection{Symplectic structure on the Iwasawa component}\index{symplectic!on $R$}
%-------------------------------------

The Iwasawa theorem gives us a global diffeomorphism between $R=AN$ and $M=G/K$ by $\dpt{\varphi}{R}{G/K}$, $\varphi(an)=[an]$. But one has a symplectic form on $M$ : $\omega^M_x(X,Y)=g_x(JX,Y)$. So, $R$ has also a symplectic form defined by, $\forall\,X,Y\in T_{an}R$,
\begin{equation}
\omega^R=\varphi^*\omega^M,
\end{equation}
or more explicitly:
\[
  \omega^R_{an}(X,Y)=\omega^M_{\varphi(an)}(d\varphi_{an}X,d\varphi_{an}Y).
\]

\begin{proposition}
This symplectic form is $R$-invariant under the left action; in other words,  $\forall r\in R$,
\begin{equation}
\omega_{ran}^R\Big(  (dL_r)_{an}X,(dL_r)_{an}Y  \Big)=\omega^R_{an}(X,Y).
\end{equation}

\end{proposition}

\begin{proof}
For a $r\in R$, we want to looks at
\begin{equation}
\omega^R_{ran}(dL_rX,dL_rY)=\omega^M_{[ran]}(d\varphi_{ran}dL_rX,d\varphi_{ran}dL_rY)\\                          
\end{equation}

But we know the invariance of $\omega^M$:
\[
  \omega^M_{[hg]}(dL_hX,dL_hY)=\omega^M_{[g]}(X,Y),
\]
Now, let us show that $d\varphi_{ran}dL_rX=dL_rd\varphi_{an}X$. For this, we consider a path which gives $X\in T_{an}R$: $X(t)\in R$, $X(0)=an$. So,
\begin{equation}
  d\varphi_{ran}(dL_r)_{an}X=\Dsdd{[rX(t)]}{t}{0}
                        =\Dsdd{L_r[X(t)]}{t}{0}
			=(dL_r)_{[an]}d\varphi_{an}X.
\end{equation}
Finally,
\begin{equation}
\begin{split}
\omega_{ran}^R(dL_rX,dL_rY)&=\omega^M_{[ran]}(d\varphi_{ran}(dL_r)_{an}X,d\varphi_{ran}(dL_r)_{an}Y)\\
                           &=\omega^M_{[ran]}( (dL_r)_{[an]}d\varphi_{an}X,(dL_r)_{[an]}d\varphi_{an}Y )\\
			   &=\omega^M_{[an]}(d\varphi_{an}X,d\varphi_{an}Y)\\
			   &=\omega^R_{an}(X,Y).
\end{split}
\end{equation}

\end{proof}


\subsection{Iwasawa coordinates}
%-------------------------------

We consider $M=G/K$, an hermitian symmetric space (irreducible of non-compact type\quext{je ne sais pas ce que \c{c}a veut dire, mais je ne sais pas non plus o\`u on l'utilise. (p. 301 d'Helgason) }). Let us consider a $Z\in\mZ(\mK)$ as before: $\delta B(Z,.)|_{\mP\times\mP}$ is a $\mK$-invariant $2$-form on $\mP$. 
There are some remarkable spaces: $\mR=\mA\oplus\mN$, the Lie algebra of $R=AN$; $\mO=\Ad(G)Z\subset\mG$. We consider the following diffeomorphism:
\begin{subequations}
\begin{align}
&\dpt{\mI}{\mA\oplus\mN}{R},  &\mI(a,n)&=e^ae^n,\\
&\dpt{\varphi}{R}{M},             &\varphi(an)&=[an],\\
&\dpt{\phi}{\mR}{\mO},        &\phi(r)&=\Ad(\mI(r))Z,\\
&\dpt{\lambda}{\mO}{M},       &\lambda(\Ad(g)Z)&=[g].
\end{align}
\end{subequations}
Note that $\mR=T_eR=\mA\oplus\mN=T_r\mR$ where $\mR=T_r\mR$ is a standard identification of vector spaces. The symplectic forms on these spaces are naturally defined by
\begin{subequations}
\begin{align}
   \omega^R&={\varphi^*}^{-1}\omega^M\\
   \omega^{\mR}&=\mI^*\omega^R\\
   \omega^{\mO}&=\lambda^*\omega^R
\end{align}
\end{subequations}
%
By the way, the diffeomorphism $\mI$ is called the \defe{Iwasawa coordinates}{Iwasawa!coordinates}.

\begin{proposition}
The map $\phi$ is bijective.
\end{proposition}

\begin{proof}
For the surjective condition, we have to obtain $\Ad(ank)Z$ under the form $\Ad(\mI(A,N))Z$. For this, remark that one can find $K\in\mK$, $A\in\mA$, $N\in\mN$  such that $k=e^K$, $a=e^A$,$n=e^N$, then
\[
  \Ad(e^Ae^N)Z=\Ad(e^Ae^Ne^K)Z=\Ad(ank)Z.
\]

In order to see the injective condition, let us consider $r,r'\in\mA\oplus\mN$ such that
\[
  \Ad(\mI(r))Z=\Ad(\mI(r'))Z.
\]
Then, $\Ad(\mI(r'))^{-1}\circ\Ad(\mI(r))=\id$. This makes $\Ad(e^{-N'}e^{-A'}e^Ae^N)=id$, so that
\[
   e^{-N'}e^{-A'}=\left(e^Ae^N\right)^{-1},
\]
but $\exp$ is a diffeomorphism, then $(A,N)=(A',N')$.

\end{proof}

\begin{lemma}\label{lem:om_O_om_R}
The symplectic forms $\omega^{\mR}$ and $\omega^{\mO}$ are related by
\begin{equation}
\omega^{\mO}=(\phi^{-1})^*\omega^{\mR}.
\end{equation}
\end{lemma}

\begin{proof}
The definitions make that
\begin{equation}
  (\phi^{-1})^*\omega^{\mR}=(\phi^{-1})^*\mI^*\omega^R
                          =(\phi^{-1})^*\mI^*\varphi^*\omega^M,
\end{equation}
so that we just need to see that $\varphi\circ\mI\circ\phi^{-1}=\lambda$. This is true because for any $g\in K$,
\[
   \varphi\circ\mI\circ\phi^{-1}(\Ad(g)Z)=\varphi\circ\mI(\mI^{-1}(g))=\varphi(g)=[g].
\]

\end{proof}

Now, consider $u\in T_r\mR$, with $r=a+n\in\mA\oplus\mN$, and (just for fun) let us compute $d\phi_r(u)$. In the following computation, $u_A$ and $u_N$ denotes the unit vectors in the direction of $\mA$ and $\mN$.
\begin{equation}
\begin{split}
   d\phi_r(u)&=\Dsdd{  \Ad(e^{a+tu_A}e^{n+tu_N})Z  }{t}{0}\\ 
             &=\Dsdd{ \Ad(e^{tu_A})\Ad(e^{an})Z  }{t}{0}\\
             &\quad+\Ad(e^a)\Dsdd{  \Ad(e^n)\Ad(e^{-n})\Ad(e^{n+tu_N})Z  }{t}{0}\\
	     &=-(u_A^*)_{\phi(r)}\\
	     &\quad+\Ad(e^ae^n)\Dsdd{ \Ad(e^{-n})\Ad( e^{n+tu_N} )Z  }{t}{0}\\
	     &=-(u_A^*)_{\phi(r)}+\Ad(\mI(r))\Dsdd{  \Ad(e^{ CBH(-n,n+tu_N) })Z  }{t}{0},
\end{split}
\end{equation}
where $CBH$ denote the \href{http://en.wikipedia.org/wiki/Baker-Campbell-Hausdorff_formula}{Campbell-Baker-Hausdorff}\index{Campbell-Baker-Hausdorff formula} function defined by
\[
   e^xe^y=e^{CBH(x,y)}.
\]
One maybe knows the formula
\begin{equation}
\Dsdd{  CBH(-n,n+tu_N)  }{t}{0}=F(\ad(n))u_N,
\end{equation}
where $F(\ad(n))$ is defined by the expansion of 
\[
F(z)=\frac{1-e^{-z}}{z}
\]
for $z\in\eC$\quext{Il faut encore aller voir dans Duitsermaat les tenants et aboutissants de ce truc.}. Finally, 
\begin{equation}
d\phi_r(u)=-(u_A^*)_{\phi(r)}-\left(  \Ad(\mI(r))F(\ad(n))u_N  \right)^*_{\phi(r)}.
\end{equation}

Now, remark that $\Ad(e^a)|_{\mA}=id|_{\mA}$ because $\ad a|_{\mA}=0$ ($\mA$ is abelian) and 
the famous lemma \ref{Ad_e}.

The \underline{proposition 1.1 page 5 de BM}\quext{\`A faire\ldots} makes $\omega_x^{\mO}(X^*,Y^*)=-B(x,[X,Y])$ for $x\in\mO$, $X$, $Y\in\mG$

The lemma \ref{lem:om_O_om_R} gives us immediately 
\[
   (\mI^*\omega^R)_r(u,v)=(\phi^*\omega^{\mO})_r(u,v).
\]
\subsection{Summary of the construction}
%---------------------------------------

We pick\quext{L'existence de ce $\mK$ contre-dit ce que je dis quand une autre question à propos du type non-compact} $Z\in\mZ(\mK)$. Then one defines
\[ 
  J=\ad(Z)|_{\mP}
\]
and 
\[ 
  \omega^M(X,y)=
\begin{cases}
 B(JX,Y)&\text{If $X,Y\in\mP$}\\
	0&\text{if $X$ or $Y$ belongs to $\mK$}.
\end{cases}
\]
The maps $\mI$, $\varphi$, $\phi$ and $\lambda$ between spaces $R$, $\mA\oplus\mN$, $M$, $\mR$ and $\mO$ are designed to propagate the symplectic form from $\omega^M$ to $\omega^{\mR}$, $\omega^R$, $\omega^{\mO}$. The group $R$ acts on each of these spaces and in particular on $\mO$ by the adjoint action. One can prove that $\omega^{\mO}:=\lambda^*(\varphi^{-1})^*\omega^M$ is
\[ 
  \omega^{\mO}_x(X^*,Y^*)=-B(x,[X,Y])
\]
for all $X$, $Y\in\mR$. In the whole construction, $\sigma$ is the Cartan involution which gives the decomposition
\[ 
  \sigma=\id|_{\mK}\oplus(-\id)|_{\mP}.
\]
Therefore $\sigma E=-E$ because $E\in\mN\subset\mP$.

The Lie algebra $\mG$ possesses two roots: $\alpha$ and $2\alpha$, so we decompose it as\quext{Le fait d'être de type non compact est peut-être l'absence de composante $K$ pour l'Iwasawa, qu'en penses-tu ?} 
\[ 
  \mG=\mA\oplus\mG_{\alpha}\oplus\mG_{2\alpha}.
\]
We pick $A\in\mA$, $y\in\mG_{\alpha}$ and $E\in\mG_{2\alpha}$. For example, if $B\in \mA$, $[B,y]=\alpha(A)y$.

\subsection{Continuation}
%------------------------

\begin{proposition}
Let $M=G/K$ be an hermitian irreducible symmetric space of non compact type. We suppose that $\dim\mP\geq 4$. We consider the action $\dpt{ \tau }{  \mR\times\mR  }{ \mR }$,
\[ 
  \tau_g(X)=\mI^{-1}(g\mI(X)).
\]
This action is Hamiltonian for the constant symplectic structure $\Omega$ on $\mR$ and the dual momentum maps are given by
\begin{subequations}
\begin{align}
\lambda_A(X)&=2\alpha(A)B(\sigma A,E)n_E&&\text{($A\in \mA$)}\\
\lambda_y(X)&= e^{-\alpha(a)}\Omega(n,y)&&\text{($y\in\mG_{\alpha}$)}\\
\lambda_E(X)&= e^{-Z\alpha(a)}B(\sigma E,E)
\end{align}
\end{subequations}
where $X=(a,n)$ and $n=n_{\alpha}+n_EE$ for the decomposition $\mN=\mG_{\alpha}\oplus\eR E$.

As a consequence, the Moyal star product is $R$-covariant.

\end{proposition}

\begin{proof}
From equation \eqref{eq_XlambdaYs}, we have to prove the identities 
\[ 
  \{ \lambda_X,\lambda_Y \}=X^*(\lambda_Y).
\]
We begin by proving the identity
\[ 
  \{ \lambda_A,\lambda_y \}(L)=A^*_L(\lambda_y)
\]
where $L=(a',n')\in\mR$. In these coordinate, we suppose without loss of generality that $A=(1,0)$. As usual, we will use some abuse of notation as $\mI(L)= e^{a'} e^{n'}= e^{a'A} e^{n'}$;
\begin{equation}
\begin{split}
  A^*_L(\lambda_y)&=\Dsdd{ \lambda_y(\tau_{ e^{-tA}}L) }{t}{0}\\
		&=\Dsdd{ (\lambda_y\circ\mI ^{-1}) e^{(a'-t)A} e^{n'} }{t}{0}\\
		&=\Dsdd{ \lambda_y\big( (a'-t),n' \big) }{t}{0}\\
		&=\Dsdd{  e^{-\alpha(a'-t)}\Omega(n',y) }{t}{0}\\
		&=\Dsdd{  e^{(t-a')\alpha(A)} }{t}{0}\Omega(n',y)\\
		&=\alpha(A) e^{-\alpha(a')\Omega(n',y)}.
\end{split}
\end{equation}
In this computation, we used the fact that $\alpha(a'-t)=(a'-t)\alpha(A)$.
On the other hand, 
\[ 
  \lambda_{[A,y]}(L)=\alpha(A)\lambda_y(L)=\alpha(A) e^{-\alpha(a')\Omega(n',y)}.
\]
This concludes the first check. The check that $\{ \lambda_A,\lambda_E \}=\lambda_{[A,E]}$ is the same, using the fact that $E\in\mG_{2\alpha}$ and thus that $[A,E]=2\alpha(A)E$. For the third, $[y,E]=0$ therefore, we have to prove that $\| \lambda_y,\lambda_E \|=0$. We have
\[ 
  \| \lambda_y,\lambda_E \|(L)=y^*_L(\lambda_E)=\Dsdd{ (\lambda_E\circ\mI^{-1})\big(  e^{-ty} e^{a'} e^{n'} \big) }{t}{0}.
\]
The problem is to commute $ e^{-ty}$ with $ e^{a'}$. Since the $t$ will always stands in front of $y$ and $\lambda_E$ doesn't depends on $y$, the derivative is zero\quext{Je ne crois pas que cette justification soit juste.}.
cs
\[ 
\begin{split}
  A^*_o&=\Dsdd{  e^{-tA}\cdot(0,0) }{t}{0}\\
	&=-A.
\end{split}  
\]
Since $d\mI=\id$, 
\[ 
\begin{split}
\omega^{\mR}(A^*,X)=&\omega^{\mR}(A,x_yy+x_EE)\\
		&=-\omega^{R}(A,x_yy+x_EE),
\end{split}  
\]
but for any element in $\mA\oplus\mN$, via the identification $\mR=[\mA\oplus\mN]$ (the additive class),
\[ 
\begin{split}
d\lambda^{-1}A&=\Dsdd{ \lambda^{-1}[ e^{tA}] }{t}{0}\\
		&=\Dsdd{ \Ad( e^{tA})Z }{t}{0}\\
		&=-A^*.
\end{split}  
\]
Thus 
\begin{equation} \label{eq_omeOmO}
\begin{split}
\omega^{\mR}(A^*,X)&=-\omega^{\mO}(d\lambda^{-1}A,d\lambda^{-1}(x_yy+x_EE))\\
		&=-\omega^{\mO}(A^*,x_yy^*+x_EE^*)
\end{split}  
\end{equation}
where $A^*$, $y^*$ and $E^*$ are taken in the sense of the adjoint action of $R$ on $\mO$.

Now we prove that $\lambda_A$ is well a dual momentum map. For this, we choose $X=x_AA+x_yy+x_EE\in\mR$ and we check the identity $d\lambda_AX=\omega^{\mR}(A^*,X)$ where $A^*$ stands for the given action of $R$ on $\mR$.

A question arise: at which point is taken $\omega^{\mO}$ in equation \eqref{eq_omeOmO} ? Since we compute $\omega^{\mR}$ at identity, we compute $\omega^{\mO}$ at $Z$. So
 \[ 
\begin{split}
  \omega^{\mR}(A^*,X)&=-\omega^{\mO}_Z(A^*,x_yy^*+x_EE^*)\\
		&=-B(Z,[A,y+E])\\
		&=-\alpha(A)B(Z,y)-2\alpha(A)B(Z,E).
\end{split}  
\]
Here, we have to remark that it is not zero because $\mN$ is not included in $\mP$, but is transverse.

\end{proof}
%+++++++++++++++++++++++++++++++++++++++++++++++++++++++++++++++++++++++++++++++++++++++++++++++++++++++++++++++++++++++++++
\section{Elementary normal symplectic spaces}
%+++++++++++++++++++++++++++++++++++++++++++++++++++++++++++++++++++++++++++++++++++++++++++++++++++++++++++++++++++++++++++
\label{SecElemNormSymplSpace}

This section is closely related to the Pyatetskii-Shapiro theory treated in section \ref{SecPyateskiiShapiro}. See \cite{QuantifKhalerian} as reference.

Let $(V,\Omega)$ be a symplectic real vector space of dimension $2n$. We build the \defe{Heisenberg algebra}{heisenberg!algebra} by
\begin{equation}
	\pH=V\oplus \eR E
\end{equation}
with the relation $[v,v']=\Omega(v,v')E$. Now we consider a new element $H$ and $\lA=\eR H$ and the split extension
\begin{equation}
	0\to\pH\to\lS\to\lA\to 0
\end{equation}
where $\lS=\lA\oplus_{\rho}\pH$ and $\rho\colon \lA\to \Der(\pH)$ is given by
\begin{equation}
	\rho(H)(v\oplus tE)=[H,v\oplus tE]=v\oplus 2tE.
\end{equation}
We denote by $(a,v,t)$ an element of $\lS$, that is
\begin{equation}
	(a,v,t)=aH+ v+tE
\end{equation}
with $a,t\in\eR$ and $v\in V$. We consider the $2$-form
\begin{equation}
	\omega^{\sS}=2da\wedge dt+\Omega,
\end{equation}
the pair $(\lS,\omega^{\lS})$ is said to be a \defe{normal elementary symplectic algebra}{elementary!normal symplectic algebra}. We denote by $(\eS,\omega)$ the associated connected simply connected Lie group.

\begin{proposition}		\label{Prop2807DescSMdarboux}
	With the previous notations we have
	\begin{enumerate}

		\item
			The map
			\begin{equation}
				\begin{aligned}
					(\lS,\omega^{\lS})&\to (\eS,\omega) \\
					(a,v,t)&\mapsto  e^{aH} e^{v+tE}
				\end{aligned}
			\end{equation}
			is a global Darboux chart (in particular it is a global diffeomorphism).
			
			By this diffeomorphism we identify $\lS$ and $\eS$, i.e. we will denote by $(a,v,t)$ the element $ e^{aH}e^{v+tE}\in\eS$ as well as the element $aH+v+tE\in\lS$.
		\item
			Within the coordinates $(a,v,t)$ the group law is given by
			\begin{equation}
				(a,v,t)\cdot (a',v',t')=
				(a+a', e^{-a'}v+v', e^{-2a,}t+t'+\frac{ 1 }{2} e^{-a'}\Omega(v,v')).
			\end{equation}
		\item
			If we define
			\begin{equation}		\label{Eq1807StuctSymM}
				s_{(a,v,t)}(a',v',t')=
				\big(2a-a',2\cosh(a-a')v-v',2\cosh(2(a-a'))t+\Omega(v,v')\sinh(a-a')-t'\big),
			\end{equation}
			the space $\eM=(\eS,\omega,\lS)$ becomes a symplectic symmetric space.

		\item
			The structure of symplectic symmetric space  is preserved by the left translations. In other words, for every $x\in\eS$ we have $L_x\in\Aut(\eM)$. And the subgroup $\{ L_x\tq x\in\eS \}$ acts simply transitively on $\eM$.

		\item
			We have
			\begin{equation}
				\SP(V,\Omega)\subset\Aut(\eM)
			\end{equation}
			if we define $g\cdot(a,v,t)=(a,g\cdot v,t)$ for every $g\in\SP(V,\Omega)$.
	\end{enumerate}
\end{proposition}


%+++++++++++++++++++++++++++++++++++++++++++++++++++++++++++++++++++++++++++++++++++++++++++++++++++++++++++++++++++++++++++
\section{Pyatetskii-Shapiro structure theorem}
%+++++++++++++++++++++++++++++++++++++++++++++++++++++++++++++++++++++++++++++++++++++++++++++++++++++++++++++++++++++++++++
\label{SecPyateskiiShapiro}

\begin{definition}
    A \defe{normal $j$-algebra}{normal $j$-algebra} is a triple $(\lS,\alpha,j)$ where
    \begin{enumerate}

        \item
            the Lie algebra $\lS$ is solvable and such that $\ad(X)$ has only real eigenvalues for every $X\in\lS$,
        \item
            the map $j\colon \lS\to \lS$ is an endomorphism of $\lS$ such that $j^2=-1$ and
            \begin{equation}
                [X,Y]+j[jX,Y]+j[X,jY]-[jX,jY]=0
            \end{equation}
            for every $X,Y\in\lS$,
        \item
            $\alpha$ is is  a linear form on $\lS$ such that
            \begin{enumerate}
                \item
                    $\alpha\big( [jX,X] \big)>0$ if $X\neq 0$,
                \item
                    $\alpha\big( [jX,jY] \big)=\alpha\big( [X,Y] \big)$.
            \end{enumerate}

    \end{enumerate} 
\end{definition}

If $\lS'$ is a subalgebra of $\lS$ which is invariant under $j$, then the triple $(\lS',\alpha|_{\lS'},j|_{\lS'})$ is a also normal $j$-algebra and is said to be a \defe{normal $j$-subalgebra}{normal!$j$-subalgebra} of $\lS$.

A normal $j$-algebra has a real inner product defined by the formula
\begin{equation}
    g(X,Y)=\alpha\big( [jX,Y] \big).
\end{equation}

If $\lG$ is an Hermitian Lie algebra\footnote{i.e. the center of its maximal compact is one dimensional.}, we can build a normal $j$-algebra out of $\lG$ in the following way. First, we choose an Iwasawa decomposition
\begin{equation}            \label{EqDecIwalGj}
    \lG=\lA\oplus\lN\oplus\lK,
\end{equation}
and we pick $\lS=\lA\oplus\lN$. Let $G=ANK$ be the group associated with the Iwasawa decomposition \eqref{EqDecIwalGj}. The manifold $M=G/K$ is an Hermitian symmetric space, and we have a global diffeomorphism
\begin{equation}
    \begin{aligned}
        R=AN&\to G/K \\
        g&\mapsto gK 
    \end{aligned}
\end{equation}
which endows the group $R$ with an exact left invariant symplectic structure and a compatible complex structure, see section \ref{SecHermEtSymplecticSpaces}. We define $\alpha$ by $\Omega_e=d\alpha$ ($\Omega$ is exact) and $j$ is the complex structure evaluated at identity.

A normal $j$-algebra build from an Hermitian symmetric space of rank $1$ (i.e. $\dim\lA=1$.) is \defe{elementary}{normal!elementary $j$-algebra}. Elementary normal $j$-algebra are well understood by the following proposition.

\begin{proposition}     \label{PropStructNormalElementaireJalg}
    An elementary normal $j$-algebra is a split extension
    \begin{equation}        \label{EqDecoEleJal}
        \lS_{el}=\lA_{1}\oplus_{\ad}\lN_1=\lA_1\oplus_{\ad}\big( V\oplus\lZ_1 \big)
    \end{equation}
    where $\lN_1$ is an Heisenberg algebra $\lN_1=V\oplus\lZ_1$ and $\lA_1$ is one dimensional. Moreover, $V$ is a symplectic vector space and one can choose $H\in \lA_1$ and $E\in\lZ_1$ in such a way that
    \begin{equation}            \label{EqRelColNormaljAlg}
        \begin{aligned}[]
            [H,v]&=v,\\
            [v,v']&=\Omega(v,v')E,\\
            [H,E]&=2E.
        \end{aligned}
    \end{equation}
\end{proposition}

Any normal $j$-algebra is build from elementary normal $j$-algebras by mean of the following lemma.
\begin{proposition}         \label{PropStructNormalJalg}
    Let $(\lS,\alpha,j)$, a normal $j$-algebra and $\lZ_1$, a one dimensional ideal of $\lS$.
    \begin{enumerate}

        \item
            There exists a vector space $V$ such that
            \begin{equation}
                \lS_1=j\lZ_1+V+\lZ_1
            \end{equation}
            is an elementary normal $j$-algebra, and such that $\lS$ is a split extension
            \begin{equation}        \label{EqDecNormale}
                \lS=\lS'\oplus_{\ad}\lS_1
            \end{equation}
            where $\lS'$ is, itself, a normal $j$-algebra.

        \item
            If $\lS_1=\lA_1\oplus_{\ad}(V\oplus\lZ_1)$, then
            \begin{equation}
                j\lZ_1+\lZ_1=\lA_1\oplus\lZ_1
            \end{equation}
            and
            \begin{equation}
                \begin{aligned}[]
                    [\lS',\lA_1\oplus\lZ_1]&=0,\\
                    [\lS',V]&\subset V.
                \end{aligned}
            \end{equation}
        \item
            Such an ideal $\lZ_1$ exists in every normal $j$-algebra.
    \end{enumerate}
\end{proposition}

Let us see what are the possibilities for $j$. If $jE=aH+b_iv_i+cE$, then
\begin{equation}
    [jE,E]=2aE.
\end{equation}
We can prove that $a\neq 0$. Indeed, if $a=0$, then $jE=cE$ and $-E=j^2E=cjE=c^2E$.

Now, we use the following Jacobi identity on $[H,[jE,v]]$ and the commutation relations, we find $b_i=0$. Now, suppose that $jH=a'H+b'_i+c'E$. In that case,
\begin{equation}
    -E=j^2E=j(aH+cE)=aa'H+ab'_iv_i+ac'E+caH+c^2E.
\end{equation}
Since $a\neq 0$, we have $b'_i=0$. So we have
\begin{equation}
    \begin{aligned}[]
        jE&=aH+cE\\
        jH&=a'H+c'E.
    \end{aligned}
\end{equation}
Expressing that $j^2E=-E$ and $j^2H=-H$, we find the following constrains on the coefficients:
\begin{equation}
    \begin{aligned}[]
        aa'+ca&=0\\
        ac'+c^2&=-1\\
        c'^2+c'a&=-1\\
        c'c+c'c&=0.
    \end{aligned}
\end{equation}
We check that $a\neq 0$, $c'\neq 0$ and $a'=-c$. The remaining relation is $c^2+c'a=-1$. Thus in the basis $\{ H,E \}$, the endomorphism $j$ reads
\begin{equation}
    j=\begin{pmatrix}
        -c  &   a   \\ 
        c'  &   c   
    \end{pmatrix}
\end{equation}
with $\det j=1$.

\begin{lemma}
    An elementary normal $j$-algebra has no proper $j$-ideal.
\end{lemma}

\begin{proof}
    Let $\lI$ be  a $j$-ideal of the elementary normal $j$-algebra $\lS_{el}$. Let $\lS_{el}=\lA\oplus_{\ad}(V\oplus\lZ)$. We denote by $H$ and $E$ the elements of $\lA$ and $\lZ$ (which are one dimensional) who fulfill the standard relations \eqref{EqRelColNormaljAlg}. If $X=aH+b_iv_i+cE\in\lI$, then $\big[ [X,v],v \big]\in\lI$. Using the relations, we conclude that $\lZ\subset\lI$. By $j$-invariance of $\lI$, we have $j\lZ\subset\lI$. Now, the fact that $[jE,v]=av$ implies that $\lI=\lS_{el}$.
\end{proof}

The structure of a normal $j$-algebra $\lS$ is thus as follows. We have the decomposition
\begin{equation}
    \lS=\lS'\oplus_{\ad}\Big( \lA_1\oplus_{\ad}(V_1\oplus\lZ_1) \Big)
\end{equation}
where $\lS'$ is again a normal $j$-algebra. Furthermore, $\dim\lA_1=\dim\lZ_1=1$ and we can choose a basis $H\in\lA_1$, $E\in\lZ_1$ such that
\begin{equation}
    \begin{aligned}[]
        [H,v]&=v\\
        [H,E]&=2E\\
        [v,v']&=\Omega(v,v')E\\
        [\lS',V]&\subset V\\
        [\lS',\lA_1\oplus\lZ_1]&=0.
    \end{aligned}
\end{equation}
for all $v,v'\in V_1$. The algebra $V_1\oplus\lZ_1$ is an Heisenberg algebra. 

The algebra $\lS'$ can be decomposed in the same way again and again up to end up with a sequence of elementary normal $j$-algebra.
