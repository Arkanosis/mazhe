\section{Interpolation theory}
%------------------------------

\begin{probleme}
Il faut citer encore V\'a rrily et le second bouquin de Connes et le truc de Landi comme sources
\end{probleme}

Here is a short review of what is given in the book \cite{ConnesNCG}. Let $B_0$ and $B_1$ be two Banach algebras continuously embedded in a topological vector space. We begin to define
\[ 
  K(\lambda,x)=\inf\{ \| x_{0} \|_{B_0}+\lambda^{-1}\| x_1 \|_{B_1}\tq x_0+x_1=x,\, x_0\in B_0,\,x_1\in B_1 \}
\]
for all $x\in B_0+B_1$ and $\lambda\in]0,\infty[$. For a fixed $x$, we consider the function 
\[ 
  f(\lambda)=\lambda^{\alpha} K(\lambda,x),
\]
and we define the norm of $x\in_{(\alpha,\beta)}$ by
\begin{equation}
	\| x \|_{\alpha,\beta}=\left( \int_{\eR^+_0}f(\lambda)^q \frac{ d\lambda }{ \lambda } \right)^{1/q}
\end{equation}
where $q=1/\beta$.

Now we consider the special case in with $B_0$ is the ideal of compact operators on the Hilbert space $\hH$ and $B_1=\oL^1(\hH)$, both seen as subspaces of $\oL^1(\hH)$.

\begin{proposition}
If $\alpha=1/p$ and $\beta=1/q$, and $q<0$, a compact operator $T$ belongs to $\oL^{(p,q)}(\hH)$ if and only if 
\[ 
  \sum_{N=1}^{\infty}N^{\alpha-1}q^{-1}\sigma_N(T)^q <\infty
\]
where $\sigma_N(T)=\sup\{ \| T|_E \|_1\tq \dim E=N \}$. When $q=\infty$, we have $T\in\oL^{(p,\infty)}$ if and only if $N^{\alpha-1}\sigma_N(T)$ is a bounded sequence.
\end{proposition}



\begin{proposition}
Each space $\oL^{(p,q)}$ with $1<p<\infty$ and $1\leq q\leq\infty$ is a two-sided ideal in $\oL(\hH)$ and when $p_1<p_2$, we have 
\[ 
\oL^{(p_1,q_1)}\subset\oL^{(p_1,q_1)}. 
\]
When $p_1=p_2$, this inclusion holds if $q_1\leq q_2$.
\end{proposition}

Notice that the fact to say that the sequence $N^{(\alpha-1)}\sigma_N(t)$ is bounded is the same as to say that $\sigma_N(T)=O(N^{1-\alpha})$. This in turns is noting else than $\mu_n(T)=O(n^{-\alpha})$ and the norm that we put on $\oL^{(p,\infty)}$ is 
\[ 
  \| T \|_{p,\infty}=\sup_{N\geq 1}\frac{1}{ N^{1-\alpha} }\sigma_N(T).
\]

\section{Trace class operators}
%++++++++++++++++++++++++++++++

\subsection{Trace}
%-----------------

We say that a bounded linear operator $A$ on the Hilbert space $\hH$ is \defe{trace class}{trace!class operator} if for a certain basis $\{ e_k \}$, the sum
\begin{equation}	\label{EqDeftRCLAss}
\sum_k\langle A^*Ae_k, e_k\rangle^{1/2}
\end{equation}
converges. In that case the sum $\sum_k\langle Ae_k, e_k\rangle $ converges absolutely (because each term in the sum \eqref{EqDeftRCLAss} is positive) and is thus independent with respect of the choice of the basis. That number is called the \defe{trace}{trace!of an operator}:
\begin{equation}\nomenclature{$\tr(A)$}{Trace of the operator $A$}
\tr(A)=\sum_k\langle Ae_k, e_k\rangle.
\end{equation}

In the space $\oL^1$ of trace class operators, we have that\nomenclature[F]{$\| T \|_1$}{Another norm for operators in $\oL^1$.}
\begin{equation}
\| T \|_1=\tr| T |
\end{equation}
is a norm which is not equal to the operator norm $\| T \|=\mu_0(T)$. More generally we have the following lemma.
\begin{lemma}
We have 
\begin{equation}
    \sigma_n(T)=\sup\{ \| TP_n \|\text{ such that }P_n \text{ is a projector of rank } n \},
\end{equation}
and each $\sigma_n$ is a norm on the space $\oK(\hH)$\nomenclature[F]{$\oK(\hH)$}{The space of compact operators over the Hilbert space $\hH$} of compact operators over the Hilbert space $\hH$. 
\end{lemma}

From formula \eqref{Defmuncaharacinfn}, the sequence of $\mu_k(T)$ is decreasing, so that we get the inequalities 
\[ 
  \sigma_n(T)\leq n\mu_0(T)=n\| T \|.
\]

\begin{lemma}
We have the formula
\begin{equation}	\label{EqsigmainfRST}
\sigma_n(T)=\inf\{ \| R \|_1+n\| S \|\text{ such that }R\in \oK,S\in\oK,R+S=T \}
\end{equation}
for every $T\in\oK(\hH)$.
\end{lemma}

\begin{proof}
It $T=R+S$, we have $\sigma_n(T)\leq\sigma_n(R)+\sigma_n(S)\leq \| R \|_1+n\| S \|$. Now we have to prove that $\sigma_n(T)$ actually reaches the infimum.  We can suppose that $T$ is positive; if not, we can change every signs in \eqref{EqsigmainfRST}, and nothing is changed. So lemma \ref{Lemmulamequ} is applicable. Let $P_n$ be the rank $n$ projector over the space spanned by the eigenvectors corresponding to the eigenvalues $\mu_0,\cdots,\mu_{n-1}$ of $T$. Then we consider $R=(T-\mu_n)P_n$ and $S=\mu_nP_n+T(1-P_n)$. Then we have $\| R \|_1=\sum_{k<n}(\mu_k-\mu_n)=\sigma_n(T)-n\mu_n$ and $\| S \|=\mu_n$. We conclude that
\[ 
  \sigma_n(T)=\| R \|_1+n\mu_n=\| R \|_1+n\| S \|,
\]
which concludes the proof.
\end{proof}
That formula only holds for $n\in\eN$, but we can \emph{define}
\begin{equation}
\sigma_{\lambda}=\inf\{ \| R \|_1+\lambda\| S \|\text{ such that }R \}
\end{equation}
for every $\lambda\in\eR^{+}$.

\begin{proposition}
If $0\leq\lambda\leq 1$, we have
\[ 
  \sigma_{\lambda}=\lambda\| T \|,
\]
and more generally if $\lambda=n+t$ with $n\in\eN$, $0\leq t\leq 1$ we have
\begin{equation}	\label{Eqsigunmoinstn}
\sigma_{\lambda}(T)=(1-T)\sigma_n(T)+t\sigma_{n+1}(T).
\end{equation}
Moreover we have that the function $\lambda\mapsto\sigma_{\lambda}(T)$ is an increasing piecewise linear concave function.
\end{proposition}
\begin{proof}
No proof.
\end{proof}
The inequality \eqref{Eqsigunmoinstn} makes that $\sigma_{\lambda}$ is a norm that satisfies in particular the triangular inequality.

\begin{proposition}
The inequality \eqref{Eqsigunmoinstn} can be reinforced into
\begin{equation}
\sigma_{\lambda}(A)+\sigma_{\mu}(B)\leq \sigma_{\lambda+\mu}(A+B)
\end{equation}
when $A$ and $B$ are positive operators. Combining with the triangular inequality, we find
\begin{equation}
\sigma_{\lambda}(A+B)\leq\sigma_{\lambda}(A)+\sigma_{\lambda}(B)\leq_\sigma{2\lambda}(A+B)
\end{equation}
under the same assumptions.
\end{proposition}

%%%%%%%%%%%%%%%%%%%%%%%%%%
%
   \section{Dixmier traces}
%
%%%%%%%%%%%%%%%%%%%%%%%%

We follow the approach given in \cite{Landi,itoNCG_Varilly}.

\subsection{Banach limit}
%------------------------

Let $l_{\infty}$\nomenclature[F]{$l_{\infty}$}{Space of complex-valued bounded sequences} be the Banach space of complex-valued bounded sequences. A \defe{Banach limit}{Banach!limit} is a linear functional $\phi\colon l_{\infty}\to \eR$ such that for every real sequences $x$ and $y$, we have
\begin{itemize}
\item $\phi(\lambda x+\mu y)=\lambda\phi(x)+\mu\lambda(y)$
\item if $x\geq 0$, then $\phi(x)\geq 0$,
\item if $S$ is the \defe{shift operator}{shift operator} $(Sx)_i=x_{i+1}$, then $\phi(x)=\phi(Sx)$,
\item if $x$ converges, then $\phi(x)=\lim x$.
\end{itemize}
Such a functional is not unique: if $\phi$ and $\varphi$ are two such Banach limits, one can find a sequence $x$ such that $\phi(x)\neq\varphi(x)$. Such an example has to be non-convergent.

\subsection{Infinitesimal operator}
%----------------------------------

Proposition \ref{prop_comp_ini} leads us to think to compact operators as infinitesimals because it is almost zero on th major part of the space. Here is the precise definition.

\begin{definition}
Let $\alpha\in\eR^+$. An \defe{infinitesimal}{infinitesimal operator} of order $\alpha$ is a compact operator $T$ such that there exists a $C\leq\infty$ for which
\[ 
  \mu_n(T)\leq Cn^{-\alpha}.
\]
for all $n\geq 1$.
\end{definition}

When $T_1$ and $T_2$ are two compact operators, one can prove that 
\[ 
  \mu_{n+m}(T_1T_2)\leq\mu_n(T_1)\mu_m(T_2)n,
\]
hence if $T_j$ is of order $\alpha_j$, $T_1T_2$ is of order $\leq\alpha_1+\alpha_2$. Moreover the infinitesimals of order $\alpha$ form a two-sided ideal (non closed) in $\opB(\hH)$ because for all $T\in\opK(\hH)$ and $B\in\opB(\hH)$, we have
\begin{equation}
\begin{split}
\mu_n(TB)&\leq\| B \|\mu_n(T)\\
\mu_n(BT)&\leq\| B \|\mu_n(T).
\end{split}
\end{equation}
For a proof, see bibliography of \cite{Landi}.

Remark that for an infinitesimal of order $1$, the characteristic values are bounded by $\mu_n(T)\leq 1/n$, so that there are no reason for such a $T$ be belongs to $\oL^1$, but the divergence of $\tr(T)$ is at most logarithmic:
\[ 
  \sum_{n=1}^{N-1}\mu_n(T)\leq C\ln N.
\]

\subsection{Dixmier trace}
%-------------------------

We want to build a trace which is non zero on infinitesimals of order $1$, but which vanishes on infinitesimals of lager order. The usual trace is defined, for $T\in\mathscr{L}^{1}$,  by\index{trace}
\[ 
  \tr T:=\sum_n\langle T\xi_n|\xi_n\rangle
\]
and is independent of the chosen orthonormal basis $\{ \xi_n \}$ of $\hH$. When $T$ is positive and compact, we define the trace by
\[ 
  \tr T=\sum_{n=1}^{\infty}\mu_n(T).
\]
The problem is that infinitesimals of order $1$ are not in general in $\mathscr{L}^{1}$ because on these operators, we do not have a better control that $\mu_n(T)\leq \frac{ C }{ n }$. Hence the sum can diverge. Worse: the space $\mathscr{L}^{1}$ contains infinitesimals of order lager than $1$. However we know that in the case of infinitesimals positive operators of order $1$ is at most logarithmic :
\[ 
  \sum_{n=0}^{N-1}\mu_n(T)\leq C\ln N.
\]
We are going to find a way to extract the coefficient of the logarithmic divergence. We denote by $\mathscr{L}^{(1,\infty)}$ the ideal of compact operators which are infinitesimals of order $1$. If $T\in\mathscr{L}^{(1,\infty)}$, we want to define the trace by
\[ 
  \lim_{N\to\infty}\frac{1}{ \ln N }\sum_{n=0}^{N-1}\mu_n(T).
\]
This definition has two main problems: it is not specially linear in $T$ and does not converge in general. Let the sums
\[ 
  \sigma_N(T)=\sum_{n=1}^{N-1}\mu_n(T)\quad{ and }\quad\gamma_N(T)=\frac{ \sigma_N(T) }{ \ln N }.
\]
One can prove that 
\begin{subequations}
\begin{align}
\sigma_N(T_1+T_2)&\leq\sigma_N(T_1)+\sigma_N(T_2)\\
\sigma_{2N}(T_1+T_2)&\geq\sigma_N(T_1)+\sigma_N(T_2);
\end{align}
\end{subequations}
the second relation only holds with $T_1,T_2\geq0$. Therefore, when $T_1,T_2>0$, we have
\begin{equation} \label{eq_gammaNleq}
\begin{split}
  \gamma_N(T_1+T_2)&\leq \gamma_N(T_1)+\gamma_N(T_2)\\
		&\leq \frac{ \sigma_{2N}(T_1+T_2) }{ \ln N }\\
		&=\frac{ \gamma_{2N}(T_1+T_2) }{ \ln N }\ln 2N\\
		&\leq \gamma_{2N}(T_1+T_2)\big( 1+\frac{ \ln 2 }{ \ln N } \big)
\end{split}
\end{equation}
because for suitably large $N$,
\[ 
  1+\frac{ \ln 2 }{ \ln N }=\frac{ \ln N+\ln 2 }{ \ln N }\leq\frac{ \ln 2N }{ \ln N }=\ln N.
\]
If the sequence $\gamma_N$ converges, then it is linear because when $N\to\infty$, we have $1+\ln 2/\ln N\to1$; hence equalities
\[ 
  \gamma_N(T_1+T_2)\leq\gamma_N(T_2)+\gamma_N(T_2)\leq\gamma_{2N}(T_2+T_2)\big( 1+\frac{ \ln 2 }{ \ln N } \big).
\]
fix the limit of $\gamma_N(T_1)+\gamma_N(T_2)$ on the one of $\gamma_{2N}(T_1+T_2)$. This however does not resolve the problem of convergence of $\gamma_n$, even when it is bounded.

The trick is to not take the usual limit, but to define a linear form $\lim_{\omega}$ on the space $l^{\infty}(\eN)$ of bounded sequences and to impose to $\lim_{\omega}$ to fulfil certain conditions.

From remark on page \pageref{pg_char_inv_U}, we know that the values $\mu_n(T)$ are unitary invariant, hence the sequence $(\gamma_N)$ is also unitary invariant. This leads us to search for an unitary invariant form $\lim_{\omega}$. The following proposition allows us to only define $\lim_{\omega}$ in the positive part of $\mathscr{L}^{(1,\infty)}$.

\begin{proposition}
The space $\mathscr{L}^{(1,\infty)}$ is generated by its positive part.
\end{proposition}
\begin{proof}
No proof.
\end{proof}
Here are the conditions we impose to $\lim_{\omega}\colon l^{\infty}(\eN)\to \eN$ :
\begin{enumerate}
\item it is a linear form,
\item $\lim_{\omega}(\gamma_N)\geq 0$ is $\gamma_N\geq0$,
\item $\lim_{\omega}(\gamma_N)=\lim\gamma_N$ if the usual limit exists,
\item\label{limomiii}$\lim_{\omega}(\gamma_1,\gamma_1,\gamma_2,\gamma_2,\gamma_3,\gamma_3)=\lim_{\omega}(\gamma_N)$,
\item\label{limomiv} $\lim_{\omega}(\gamma_{2N})=\lim_{\omega}(\gamma_N)$.
\end{enumerate}
The condition \ref{limomiv} is the \emph{scale invariance}\index{scale invariance}; this property is equivalent to the property \ref{limomiv}. Dixmier has found a lot of such form. For each of them, one has a trace
\begin{equation}
   \tr_{\omega}(T)=\lim_{\omega}\frac{1}{ \ln N }\sum_{n=0}^{N-1}\mu_n(T) 
\end{equation}
for positive $T\in\mathscr{L}^{(1,\infty)}$. When $T_1$ and $T_2$ are positive, we have linearity :
\[ 
  \tr_{\omega}(T_1+T_2)=\tr_{\omega}(T_1)+\tr_{\omega}(T_2).
\]
Since $\mathscr{L}^{(1,\infty)}$ is generated by its positive part, the form $\tr_{\omega}$ ---which is initially only defined for positive operators $T$--- extends to the whole $\mathscr{L}^{(1,\infty)}$ with properties
\begin{enumerate}
\item $\tr_{\omega}(T)\geq0$ if $T\leq0$,
\item $\tr_{\omega}(\lambda_1T_2+\lambda_2 T_2)=\lambda_1\tr_{\omega}(T_1)+\lambda_2\tr_{\omega}(T_2)$,
\item $\tr_{\omega}(BT)=\tr_{\omega}(TB)$ for all $B\in\opB(\hH)$,
\item\label{item_tromTiv} $\tr_{\omega}(T)=0$ if $T$ is an infinitesimal of order larger than $1$.
\end{enumerate}
For a proof, see \cite{Landi}. For \ref{item_tromTiv}, remark that the space of infinitesimals of order larger than $1$ form a two-sided ideal whose elements fulfil $n\mu_n(T)\to 0$. Then the sequence $(\gamma_B)$ converges to zero too and the Dixmier trace vanishes.


\subsection{Dixmier: second}
%---------------------------

The set of \defe{infinitesimals of order $1$}{infinitesimal!of order $1$} is the normed ideal
\begin{equation}
\oL^{1+}=\{ T\in\oK\tq \| T \|_{1+}<\infty \}
\end{equation}
where the norm $\| T \|_{1+}$\nomenclature[F]{$\| T \|_{1+}$}{Norm for the order one infinitesimals} is defined by
\begin{equation}
\| T \|_{1+}=\sup_{\lambda\geq a}\frac{ \sigma_{\lambda}(T) }{ \ln\lambda }.
\end{equation}
That idea include the trace class operators. We define\nomenclature[F]{$\oL^p$}{A functional space around the Dixmier trace}
\begin{equation}
\oL^p=\{ T\in\oK\tq \tr| T |^p<\infty \}.
\end{equation}

\begin{proposition}
On the space $\oL^p$, we have
\[ 
  \sigma_{\lambda}=O(\lambda^{1-1/p})
\]
and $\oL^{1+}\subset\oL^p$ when $p>1$.
\end{proposition}
Notice that, when $T\in\oL^{1+}$, the function $\lambda\mapsto\sigma_{\lambda}(T)/\ln\lambda$ is bounded and continuous on the interval $[e,\infty[$. It belongs thus to the $C^*$-algebra $C_b\big( [e,\infty[\big)$. So we can use the \defe{Ces\` aro means}{ces\` aro mean}:
\begin{equation}	\label{EqCearomaen}
\tau_{\lambda}(T)=\frac{1}{ \ln\lambda }\int_e^{\lambda}\frac{ \sigma_u(T) }{ \ln u }\frac{ du }{ u },
\end{equation}
and the function $\lambda\mapsto\tau_{\lambda}(T)$ still belongs to $C_B\big( [e,\infty[ \big)$ with $\| T \|_{1+}$ as upper bound.

\begin{proposition}
The double inequality
\[ 
  0\leq \tau_{\lambda}(A)+\tau_{\lambda}(B)-\tau_{\lambda}(A+B)\leq\big( \| A \|_{1+}+\| B \|_{1+} \big)\ln 2\frac{ \ln\ln\lambda }{ \ln\lambda }.
\]
holds for $A$, $B\in\oL^{1+}$.
\end{proposition}
\begin{proof}
No proof.
\end{proof}
That proves that $\tau\lambda$ becomes additive when $\lambda$ goes to infinity. We can work on that in order to make it additive. First we consider 
\[ 
  \mB=C_b\big( [e,\infty[ \big)/C_0\big( [e,\infty[ \big),
\]
and we consider $[\tau(A)]$, the class of $\tau(A)$ (i.e. the function $\lambda\mapsto\tau_{\lambda}(A)$) with respect to that quotient.

\begin{proposition}
The map $[\tau]$ is additive, positive and homogeneous from the positive cone in $\oL^{1+}$ to $\mB$. Moreover
\[ 
  [\tau(UAU^{-1})]=[\tau(A)]
\]
for every unitary $U$.
\end{proposition}
That makes that $[\tau]$ extends to a linear map $[\tau]\colon \oL^{1+}\to \mB$ such that $[\tau](ST)=[\tau](TS)$ for all $T\in\oL^{1+}$ and $S$.

Now if $\omega\colon \mB\to \eC$ is any state, we define the \defe{Dixmier trace}{Dixmier trace} as
\begin{equation}
\tr_{\omega}(T)=\omega\big( [\tau](T) \big).
\end{equation}
That definition has a problem : the $C^*$-algebra $\mB$ being non separable, one cannot exhibit a state, so that the formula is in practice unusable.

\subsection{Noncommutative integral}
%-----------------------------------

Let us consider $f\in C_b\big( [e,\infty \big)$; the limit $\lim_{\lambda\to\infty}f(\lambda)$ exists if and only if $\omega(f)$ does not depend on $\omega$ because of the quotient by $C_0\big( [e,\infty[ \big)$ which makes that $\omega$ can only depend on the behaviour near infinity. We say that the operator $T\in\oL^{1+}$ is \defe{measurable}{measurable operator} if the function $\lambda\mapsto\tau_{\lambda}(T)$ converges when $\lambda\to\infty$. In that case, $\tr_{\omega}(T)$ equals that limit, and we denote by $\dashint T$ the common value of the Dixmier traces:\nomenclature[F]{$\dashint T$}{The noncommutative integral}
\begin{equation}
\dashint T=\lim_{\lambda\to\infty}\tau_{\lambda}(T)
\end{equation}
if the limit exists. That is the \defe{noncommutative integral}{noncommutative!integral} of $T$.

It $T$ is a compact operator and if $\sigma_n(T)/\ln n$ converges when $n\to\infty$, then the limit $\lim_{\lambda\to\infty}\tau_{\lambda}(T)$ exists and $T$ is a measurable operator in $\oL^{1+}$. Indeed in that case the quantity $\sigma_u(T)/\ln u$ becomes constant when $u$ is large, so that we are left in the definition \eqref{EqCearomaen} with
\[ 
  \lim_{\lambda\to\infty}\frac{C}{ \ln\lambda }\int_e^{\lambda}\frac{ du }{ u }=C\lim_{\lambda\to\infty}\frac{ \ln\lambda-1 }{ \ln\lambda }.
\]
which exists.

\subsection{Residues}
%-------------------

Let $M$ be a compact Riemannian spin manifold of dimension $n$. Let $T$ be a pseudo-differential operator of order $-n$ acting on the sections of a complex vector bundle $E\to M$. Its \defe{residue}{residue} is defined by
\begin{equation}
\ResW T=\frac{1}{ n(2\pi)^n }\int_{S^*M}\tr_E\sigma_{-n}(T)\,d\mu
\end{equation}
where $\sigma_{-n}\colon T^*M\to \End E$ is the principal symbol of $T$ (is as a homogeneous function of degree $-n$) and the integral is taken on the cosphere\index{cosphere} 
\[ 
  S^*M=\{ (x,\xi)\in T^*M\tq \| \xi \|=1 \}
\]
with the measure $d\mu=dx\,d\xi$.

