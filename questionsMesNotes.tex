% This is part of Mes notes de mathématique
% Copyright (c) 2011-2015
%   Laurent Claessens
% See the file fdl-1.3.txt for copying conditions.

%+++++++++++++++++++++++++++++++++++++++++++++++++++++++++++++++++++++++++++++++++++++++++++++++++++++++++++++++++++++++++++
\section{Les questions pour lesquelles je n'ai pas (encore) de réponse}
%+++++++++++++++++++++++++++++++++++++++++++++++++++++++++++++++++++++++++++++++++++++++++++++++++++++++++++++++++++++++++++

Ces notes ne sont pas relues de façon systématique. Aucune garantie. Merci de me signaler toute faute ou remarque : le relecteur c'est toi.

Voici une petite liste de questions que je me pose ou de choses écrites dont je ne suis pas certain. Si vous avez un avis ou une réponse à un des points, merci de vous faire connaître.

%---------------------------------------------------------------------------------------------------------------------------
\subsection{Mes questions d'analyse.}
%---------------------------------------------------------------------------------------------------------------------------

\begin{enumerate}
    \item
        Que penser de la remarque \ref{RemfdJcQF} qui dit qu'on doit avoir un théorème de complétion de partie orthonormale en une base orthonormale pour un espace de Hilbert ? C'est vrai ?
    \item
        À propos de formule sommatoire de Poisson, est-ce que l'exemple \ref{ExDLjesf} est bien fait ? En particulier la formule \eqref{EqjrNxLr} est-elle correcte et bien justifiée ?
    \item 
        L'exemple \ref{ExfYXeQg} parle d'inverser une intégrale et une dérivée au sens des distributions pour prouver que la dérivée de \( \int_0^xg(t)dt\) par rapport à \( x\) est \( g\). Rendre cela rigoureux.
    \item
        À propos du théorème de récurrence de Poincaré \ref{ThoYnLNEL}, l'application \( \phi\) doit être mesurable ? Répondre à la question posée sur la page de discussion de \wikipedia{fr}{Théorème_de_récurrence_de_Poincaré}{l'article sur wikipédia}.
    \item
        Toujours à propos du théorème de récurrence de Poincaré, il me semble qu'il y a un énoncé qui insiste sur la compacité de l'espace des phase et une démonstration utilisant la propriété de sous-recouvrement fini. Je serais content de retrouver cela. (ce serait sans doute mettable dans la leçon sur l'utilisation de la compacité)
    \item
        La démonstration de la proposition \ref{PropDerrSSIntegraleDSD} qui donne la dérivée sous l'intégrale est de moi. Vérifier si c'est correct.
    \item
        Pour l'équation différentielle de Hill : \( y''+qy=0\) avec \( q\) de classe \( C^1\), moi j'argumente dans la sous-section \ref{SubSecDWwVVPa} que les solutions sont \( C^3\). Dans le document \cite{KXjFWKA}, il ne dit que \( C^2\). Il n'y a pas contradiction, mais \ldots
    \item 
        Est-ce qu'on peut permuter la limite et l'intégrale dans \( L^2(I)\) avec \( I=\mathopen] 0 , 1 \mathclose[\) ? Si \( (u_n)\) est une suite convergente dans \( L^2\) vers \( u\), est-ce qu'on a \( \lim_{n\to \infty} \int_Iu_n=\int_Iu\) ?

            Voir le problème \ref{ProbTOElufz}.

    \item
        Est-ce que \( L^p\big( S^1 \big)=L^p\big( \mathopen[ 0 , 2\pi \mathclose] \big)=L^p\big( \mathopen] 0 , 2\pi \mathclose[ \big)\) ? J'imagine que oui parce qu'un point de plus ou de moins ne change rien dans \( L^p\). En tout cas je vois bien une bijection :
            \begin{equation}
                \begin{aligned}
                    \psi\colon L^p\big( \mathopen[ 0 , 2\pi \mathclose] \big)&\to L^p\big( \mathopen] 0 , 2\pi \mathclose[ \big) \\
                        [f]&\mapsto \text{la classe de} f(x)=\begin{cases}
                            f(x)    &   \text{si \( x\in\mathopen] 0 , 2\pi \mathclose[\)}\\
                            0    &    \text{si \( x=0\) ou \( x=2\pi\)}.
                        \end{cases}
                \end{aligned}
            \end{equation}
            Si ces espaces ne sont pas égaux, alors il faut un peu faire attention aux notations que j'utilise dans toute la partie sur les approximations de l'unité, et dans le théorème \ref{ThoQGPSSJq} qui donne la densité des polynômes trigonométriques dans \( L^p(S^1)\) -- qui est le point de départ de la théorie de Fourier sur \( L^2\). 

        \item
            Le lemme \ref{LemYFoWqmS} dit que toute fonction mesurable est limite croissantes de fonctions étagées mesurables. Est-ce que la preuve est correcte et lisible ?

        \item 
            Dans \cite{OEVAuEz}, on parle de la proposition \ref{PropZMKYMKI} à sa page \( 10\). Comment est-ce qu'on justifie le passage
            \begin{equation}
                \int_{\eR^d}T\big( y\mapsto \varphi(x)\psi(x-y) \big)dx=T\Big( y\mapsto\int_{\eR^d}\varphi(x)\psi(x-y)dx \Big).
            \end{equation}
            Sylvie Benzoni précise que «ceci demanderai quelque justification». Où trouver lesdites justifications ? Il s'agit de permuter une distribution et une intégrale.

        \item
            En ce qui concerne les questions «fines» de topologie concernant l'espace \( \swD(K)\) des fonctions \(  C^{\infty}\) à support dans le compact \( K\), il faut voir si les énoncés, les preuves et l'enchaînement des théorèmes et propositions
            \begin{enumerate}
                \item Proposition \ref{PropQAEVcTi} pour dire que \( \swD(K)\) est métrique et complet (et donc de Baire).
                \item La proposition \ref{PropSYMEZGU} qui donne la complétude de \( \big( C(X,Y),\| . \|_{\infty} \big)\) lorsque \( X\) est compact et \( Y\) métrique complet.
                \item Le théorème \ref{ThoNBrmGIg} de Banach-Steinhaus avec les semi-normes. Le fait qu'il s'applique bien à \( \swD(K)\).
                \item L'utilisation de cette version du théorème de Banach-Steinhaus dans la proposition \ref{PropLKtBsVi}. En particulier l'utilisation de la famille de fonctionnelles \eqref{EqBEKoqMb} et la preuve qu'elles sont continues.
                \item
                    L'équivalence entre les deux topologies de la proposition \ref{PropLOwUvCO} et le fait que cela s'applique bien à \( \swD(K)\).
                \item
                    Et enfin l'utilisation de tout ça pour donner l'unique solution de l'équation de Schrödinger du théorème \ref{ThoLDmNnBR}. Est-ce que l'énoncé de ce théorème est déjà correct ?
                \item
                    En particulier la fonction \eqref{EqEVtJcnz} me semble être un petit bricolage. 
            \end{enumerate}
    \item
        Dans la démonstration du théorème de Baire (\ref{ThoQGalIO}), il manque peut-être quelque fermetures sur les boules.
    \item
        Préciser l'énoncé et donner une démonstration de la proposition \ref{PropMpBStL} qui traite de sommes dénombrables.
    \item
        La justification que \( f_n\) est borélienne dans la proposition \ref{PropfqvLOl} mérite plus de détails.
    \item
        Où trouver une preuve de la proposition \ref{PropKZDqTR} sur le supplémentaire topologique ?
    \item
        La partie «unicité» du théorème de Cauchy-Lipschitz \ref{ThokUUlgU}.
    \item
        L'inversion entre la somme et l'intégrale de l'équation \eqref{EqXSgZGw}.
    \item   \label{ItemLPrIWZhPg}
        À propos de différentiabilité pour une application entre deux espaces de Banach. Le théorème \ref{ThoOYwdeVt} dit que si
    \begin{equation}
        \Dsdd{ f(x+tu) }{t}{0}
    \end{equation}
    existe pour tout \( x\in B(a,r)\) et est continue (par rapport à \( x\)) en \( x=a\), et que de plus \( \frac{ \partial f }{ \partial u }(a)=0\) pour tout \( u\), alors \( f\) est différentiable en \( a\).

    Est-ce que cet énoncé est encore vrai lorsque \( \frac{ \partial f }{ \partial u }(a)\neq 0\) ? Est-ce qu'il existe un théorème comparable à la proposition \ref{Diff_totale} pour la dimension infinie ? Il me semble que non, mais je n'ai pas de contre-exemples en tête. Par contre, je vois bien où la preuve bloque : c'est le lien entre la linéarité et la différentiabilité.

    % Lorsque cette question aura une réponse, on pourra peut-être décommenter les choses à la position 229262367.

    \item

        À propos du théorème de la fonction implicite \ref{ThoAcaWho}. Est-ce que la partie unicité est correctement énoncée et démontrée ? En particulier il me semble qu'il manque la mention d'un voisinage de \( y_0\) dans \cite{SNPdukn}.

    \item

        Il me faudrait un exemple de partie de \( \eR\) qui soit non borélienne mais mesurable au sens de Lebesgue. Comme je suis exigeant, je le veux le plus constructif possible : pas d'axiome du choix. Pour l'instant le mieux que j'aie trouvé est une feuille de TD\cite{XSHoosgoQa} qui utiliser des ordinaux.

    \item

        À propos de l'écriture décimale des nombres, la proposition \ref{PropSAOoofRlQR} et le théorème \ref{ThoRXBootpUpd} sont à relire attentivement parce que les démonstrations sont presque complètement des inventions personnelles.

    \item

        À propos de l'ensemble de Cantor, le lemme \ref{LemAZGoosKzEm} et la proposition \ref{PropTPPooDySbm} sont à relire attentivement : les démonstrations sont des inventions personnelles.

    \item

        La démonstration du théorème de Lyapunov \ref{ThoBSEJooIcdHYp} est en grande partie de l'invention personnelle extrapolée de divers morceaux pris par-ci par-là. En particulier tout ce qu'il y a autour de \eqref{subeqsFNPJooERJkxO} et l'utilisation par le théorème d'explosion en temps fini pour déduire l'existence sur \( \eR\) de la solution mérite d'être relu, et plutôt deux fois qu'une.

    \item

        Dans la preuve de la dualité de \( L^p\), théorème \ref{ThoLPQPooPWBXuv}, il y a une partie dans laquelle on diffère le cas \( p= 1\) des autres. À quelle endroit la partie \( p=1\) ne fonctionne pas ? Est-ce seulement le fait que
        \begin{equation}
            \| \mtu_E \|_p=\mu(E)^{1/p} 
        \end{equation}
        et non \( \mu(E)\) ?

\end{enumerate}

%---------------------------------------------------------------------------------------------------------------------------
\subsection{Mes questions d'algèbre, géométrie.}
%---------------------------------------------------------------------------------------------------------------------------

\begin{enumerate}
    \item
        La «décomposition en facteurs premiers» dans \( \eZ[i\sqrt{2}]\) que je donne dans l'exemple \ref{ExluqIkE} est-elle correcte ? En particulier le lemme \ref{LemTScCIv} ?
    \item
        Est-ce que la fin de la démonstration \ref{ThojCJpFW} avec cette histoire d'ensemble \( \{ \xi_k^q\tq q\in \eN \}\) fini est compréhensible ?
    \item
        Les représentations \emph{irréductibles} sont les modules \emph{indécomposables}. Quid des modules irréductibles ? C'est pas un peu dingue de ne pas utiliser le mot «irréductible» pour désigner les mêmes choses dans le cas des modules et celui des représentations ?
    \item
        Rendre rigoureuse la remarque \eqref{RemmQjZOA} qui dit que les matrices ont le polynôme minimal est égal au polynôme caractéristique sont denses dans les matrices.
    \item
        Soit \( \eL\) une extension algébrique de corps de \( \eK\) et \( a\in \eL\). Est-ce que le polynôme minimal de \( a\) dans \( \eK[X]\) est l'unique polynôme irréductible unitaire \( P\in \eK[X]\) tel que \( P(a)=0\) ? D'ailleurs, est-ce qu'un tel polynôme existe ? est unique ? J'utilise cela dans la proposition \ref{PropUmxJVw}.
    \item
        La partie initiation de récurrence (\( r=2\)) de la preuve de la proposition \ref{PropSVvAQzi} à propos de convexe et de barycentre est-elle correcte ? Ce passage de l'espace affine à l'espace vectoriel sous-jacent me paraît un peu facile.
    \item
        Le lemme \ref{LemUELTuwK} à propos de PGCD de polynômes est-il correct ? J'imagine que le polynôme \( \pgcd(P,PU+R)\) ne dépend pas de \( U\). Est-ce exact ?
    \item
        Est-ce que parler de sous-groupe «normal» au lieu de «distingué» est un anglicisme ?
    \item
        Est-ce que l'énoncé et la démonstration de la proposition \ref{PropyMTEbH} sont corrects ? Si \( a\) et \( b\) sont des racines de \( P\), alors \( \mu_a\mu_b\) divise \( P\) (si \( \mu_a\neq \mu_b\)). Cette proposition est utilisée dans la démonstration de l'irréductibilité des polynômes cyclotomiques (proposition \ref{PropoIeOVh}).
    \item
        À quoi sert l'hypothèse «autre que \( \eF_2\)» dans le lemme \ref{LemcDOTzM} ? Peut-être dans la notion de déterminant parce qu'en caractéristique \( 2\), l'antisymétrie d'une forme linéaire n'implique le fait qu'elle soit alternée.
    \item
        L'inversibilité de la somme de Gauss (proposition \ref{PropciRUov}) est-elle bien démontrée ?
    \item
        Des commentaires sur l'exemple \ref{ExfUqQXQ} qui montre que \( X^p-X+1\) est irréductible sur \( \eF_p\).
    \item
        Les idéaux de \( A/I\) sont en bijection avec les idéaux de \( A\) contenant \( I\). Justification de l'équation \eqref{EqKbrizu}.
    \item
        L'énoncé et la démonstration d'une des multiples version du théorème de l'élément primitif, proposition \ref{PropNsLqWb}.
    \item
        Pourquoi la pseudo-réduction simultanée (corollaire \ref{CorNHKnLVA}) est-elle \emph{pseudo} ? Pourtant les matrices sont bel et bien simultanément diagonalisées.
    \item
        À propos d'extensions algébriques, est-ce que la proposition \ref{PropURZooVtwNXE} est correcte ? Est-ce qu'implicitement, il n'y a pas un sur-corps de \( \eK\) dans lequel il faut travailler ?
    \item
        Est-ce que la proposition \ref{PropRARooKavaIT} disant qu'un polynôme minimal est irréductible et premier avec tout polynôme non annulateur est correcte ?
    \item
        À propos de construction à la règle et au compas. Pour l'addition d'angles, l'exemple \ref{ExOVDooXnWPDl} explique comment on construit la somme de deux angles. Le problème est que cette construction se fait par intersection de deux cercles. Une des deux intersections donne \( \alpha+\beta\) et l'autre donne \( \alpha-\beta\). Comment par construction peut-on choisir le bon point ?
    \item
        À propos de chiffrement RSA, quelle est la probabilité que le message \( M\) ne soit pas premier avec \( p\) ? Est-ce que Alice (qui est celle qui chiffre avec la clef de Bob) peut le vérifier ? Que penser des points que j'énumère à la page \pageref{PageAKTBooMDeQxY} au dessus du problème \ref{ProbGAYFooZATuYy} ?
\end{enumerate}

%---------------------------------------------------------------------------------------------------------------------------
\subsection{Mes questions de probabilité et statistiques.}
%---------------------------------------------------------------------------------------------------------------------------

\begin{enumerate}
    \item
        Le paradoxe des familles de \ref{subSecGXVYooTDdZaB} est-il correctement énoncé et résolu ? 
    \item
        Si \( X\) est une variable aléatoire et \( A\) un événement, est-ce qu'il est vrai que \( E\big( P(A|X) \big)=P(A)\) ? Démonstration ?
    \item
        À propos de convergence en loi et en probabilité vers une constante, dans \cite{CourgGudRennes}, l'équation \eqref{EqPXngeqetaapumP} arrive avec une inégalité. Pourquoi ?
        % Lorsque tu réponds à cette question, tu peux enlever la question TpijUi
    \item
        Est-ce qu'une partie compacte d'un espace mesuré est toujours mesurable et de mesure finie ? Est-ce que la question a un sens ? Quelle est la topologie associée à une mesure ? Les ouverts seraient une base des mesurables en un certain sens ?  (par analogie aux ouverts qui sont la base des boréliens dans \( \eR^n\))

        Dans cette optique, on pourrait sans doute affaiblir l'hypothèse «de mesure finie» de l'énoncé du corollaire \ref{CorRSczQD} du théorème de Stone-Weierstrass.
    \item
        Est-ce que le théorème d'arrêt de Doob \ref{ThoZTrdjtZ} est correctement énoncé ? En particulier la seconde condition. 
    \item
        À propos du problème de la ruine du joueur. Dans \cite{KXjFWKA}, l'équation \eqref{EqABPXmgr} vient avec une \( \limsup\) et non une limite normale. Je ne comprends pas pourquoi.
    \item
        Est-ce que si \( X\) et \( Y\) sont des variables aléatoires indépendantes on a
        \begin{equation}
            E(XY|\tribA)=E(X|\tribA)E(Y|\tribA).
        \end{equation}
        Cela permettrait de ne pas utiliser la proposition \ref{PropRNBtfql} pour prouver l'équation \eqref{EqWTkXcEK}.
    \item   \label{ItemIQDVooAcFUuH}
        À propos du théorème central limite \ref{ThoOWodAi}, est-ce que la remarque \ref{RemRHFDooGbaPYu} est correcte ? Si pour une certaine variable aléatoire \( X\) on a \( E(X)=m\), alors \( P(X=m+a)=P(X=m-a)\). Cela me semble un peu facile, et même faux.

        Par contre si \( n\) est assez grand on a
        \begin{equation}
            P(\bar X_n=m+a)=P(\bar X_n=m-a).
        \end{equation}
        Cela est correct ?
\end{enumerate}

%--------------------------------------------------------------------------------------------------------------------------- 
\subsection{Mes questions de modélisation}
%---------------------------------------------------------------------------------------------------------------------------

\begin{enumerate}
    \item
        Pour un état d'une chaîne de Markov, est-ce que le mot «transient» est un anglicisme pour «transitoire» ?
\end{enumerate}

%TODO : dans l'index, le «é» doit aller avec le «e» et non au début de l'alphabet.

%+++++++++++++++++++++++++++++++++++++++++++++++++++++++++++++++++++++++++++++++++++++++++++++++++++++++++++++++++++++++++++ 
\section{Comment m'aider à rendre ces notes plus utiles ?}
%+++++++++++++++++++++++++++++++++++++++++++++++++++++++++++++++++++++++++++++++++++++++++++++++++++++++++++++++++++++++++++

Voici quelque pistes.

\begin{enumerate}
    \item
        M'écrire pour me signaler toutes les fautes.
    \item
        M'envoyer une liste de théorèmes et de résultats que tout candidat à l'agrégation devrait connaître.
    \item
        Répondre aux questions ci-dessus.
    \item
        Mettre vos propres notes sous licence FDL pour me permettre de les copier ou de les inclure. 
    \item
        Mettre une copie de (ou un lien vers) ces notes sur votre site.
    \item
        Écrire au jury d'agrégation pour dire que ces notes vous plaisent et que vous voudriez les avoir pour l'oral.
\end{enumerate}

%Un peu partout des \verb+%TODO+ sont placés dans les sources \LaTeX. Ils décrivent des choses qu'il serait bon de faire. Si vous avez un avis sur l'un d'eux, n'hésitez pas à me le faire savoir.
%\begin{quote}
%    \texttt{grep TODO *.tex}
%\end{quote}
