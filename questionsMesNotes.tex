% This is part of Mes notes de mathématique
% Copyright (c) 2011-2012
%   Laurent Claessens
% See the file fdl-1.3.txt for copying conditions.

%+++++++++++++++++++++++++++++++++++++++++++++++++++++++++++++++++++++++++++++++++++++++++++++++++++++++++++++++++++++++++++
\section{Les points faibles de ce document}
%+++++++++++++++++++++++++++++++++++++++++++++++++++++++++++++++++++++++++++++++++++++++++++++++++++++++++++++++++++++++++++

Ceci sont des notes «prises au vol». Aucune garantie. Merci de me signaler toute faute ou remarque. 

Voici une petite liste de questions que je me pose ou de choses écrites dont je ne suis pas certain. Si vous avez un avis ou une réponse à un des points, merci de vous faire connaître.

%---------------------------------------------------------------------------------------------------------------------------
\subsection{Mes questions d'analyse.}
%---------------------------------------------------------------------------------------------------------------------------

\begin{enumerate}
    \item
        Que penser de la remarque \ref{RemfdJcQF} qui dit qu'on doit avoir un théorème de complétion de partie orthonormale en une base orthonormale pour un espace de Hilbert ? C'est vrai ?
    \item
        À propos de formule sommatoire de Poisson, est-ce que l'exemple \ref{ExDLjesf} est bien fait ? En particulier la formule \eqref{EqjrNxLr} est-elle correcte et bien justifiée ?
    \item 
        L'exemple \ref{ExfYXeQg} parle d'inverser une intégrale et une dérivée au sens des distributions pour prouver que la dérivée de \( \int_0^xg(t)dt\) par rapport à \( x\) est \( g\). Rendre cela rigoureux.
\end{enumerate}

%---------------------------------------------------------------------------------------------------------------------------
\subsection{Mes questions d'algèbre,géométrie.}
%---------------------------------------------------------------------------------------------------------------------------

\begin{enumerate}
    \item
        La «décomposition en facteurs premiers» dans \( \eZ[i\sqrt{2}]\) que je donne dans l'exemple \ref{ExluqIkE} est-elle correcte ? En particulier le lemme \ref{LemTScCIv} ?
    \item
        Est-ce que la fin de la démonstration \ref{ThojCJpFW} avec cette histoire d'ensemble \( \{ \xi_k^q\tq q\in \eN \}\) fini est compréhensible ?
\end{enumerate}

%---------------------------------------------------------------------------------------------------------------------------
\subsection{Mes questions de probabilité et statistiques.}
%---------------------------------------------------------------------------------------------------------------------------

\begin{enumerate}
    \item
        Si \( X\) est une variable aléatoire et \( A\) un événement, est-ce qu'il est vrai que \( E\big( P(A|X) \big)=P(A)\) ? Démonstration ?
    \item
        À propos de convergence en loi et en probabilité vers une constante, dans \cite{CourgGudRennes}, l'équation \eqref{EqPXngeqetaapumP} arrive avec une inégalité. Pourquoi ?
        % Lorsque tu réponds à cette question, tu peux enlever la question TpijUi
    \item
        Est-ce qu'une partie compacte d'un espace mesuré est toujours mesurable et de mesure finie ? Est-ce que la question a un sens ? Quelle est la topologie associée à une mesure ? Les ouverts seraient une base des mesurables en un certain sens ?  (par analogie aux ouverts qui sont la base des boréliens dans \( \eR^n\))

        Dans cette optique, on pourrait sans doute affaiblir l'hypothèse «de mesure finie» de l'énoncé du corollaire \ref{CorRSczQD} du théorème de Stone-Weierstrass.
\end{enumerate}

%---------------------------------------------------------------------------------------------------------------------------
\subsection{Mes questions pas encore triées}
%---------------------------------------------------------------------------------------------------------------------------

\begin{enumerate}
    \item
        Rendre rigoureuse la remarque \eqref{RemmQjZOA} qui dit que les matrices ont le polynôme minimal est égal au polynôme caractéristique sont denses dans les matrices.
    \item
        Soit \( \eL\) une extension algébrique de corps de \( \eK\) et \( a\in \eL\). Est-ce que le polynôme minimal de \( a\) dans \( \eK[X]\) est l'unique polynôme irréductible unitaire \( P\in \eK[X]\) tel que \( P(a)=0\) ? D'ailleurs, est-ce qu'un tel polynôme existe ? est unique ? J'utilise cela dans la proposition \ref{PropUmxJVw}.
    \item
        Pour un état d'une chaîne de Markov, est-ce que le mot «transient» est un anglicisme pour «transitoire» ?
    \item
        Est-ce que parler de sous-groupe «normal» au lieu de «distingué» est un anglicisme ?
    \item
        La preuve de la proposition \ref{PropoIeOVh} démontrant l'irréductibilité des polynômes cyclotomiques sur \( \eQ\) me semble être un bricolage.
    \item
        Est-ce que l'énoncé et la démonstration de la proposition \ref{PropyMTEbH} sont corrects ? Si \( a\) et \( b\) sont des racines de \( P\), alors \( \mu_a\mu_b\) divise \( P\) (si \( \mu_a\neq \mu_b\)). Cette proposition est utilisée dans la démonstration de l'irréductibilité des polynômes cyclotomiques (proposition \ref{PropoIeOVh}).
    \item
        Dans la démonstration du théorème de Baire (\ref{ThoQGalIO}), il manque peut-être quelque fermetures sur les boules.
    \item
        Préciser l'énoncé et donner une démonstration de la proposition \ref{PropMpBStL} qui traite de sommes dénombrables.
    \item
        À quoi sert l'hypothèse «autre que \( \eF_2\)» dans le lemme \ref{LemcDOTzM} ? Peut-être dans la notion de déterminant parce qu'en caractéristique \( 2\), l'antisymétrie d'une forme linéaire n'implique le fait qu'elle soit alternée.
    \item
        L'inversibilité de la somme de Gauss (proposition \ref{PropciRUov}) est-elle bien démontrée ?
    \item
        Trouver une preuve que l'anneau des polynômes est factoriel (proposition \ref{PropqGZXvr})
    \item
        Des commentaires sur l'exemple \ref{ExfUqQXQ} qui montre que \( X^p-X+1\) est irréductible sur \( \eF_p\).
    \item
        La justification que \( f_n\) est borélienne dans la proposition \ref{PropfqvLOl} mérite plus de détails.
    \item
        Une preuve du théorème \ref{ThoRWEoqY} qui donne la densité de \( C^{\infty}_c\) dans \( L^1\).
    \item
        L'énoncé et la démonstration de la proposition \ref{PropNsLqWb}.
    \item
        Où trouver une preuve de la proposition \ref{PropKZDqTR} sur le supplémentaire topologique ?
    \item
        La preuve du théorème \ref{ThoJsBKir}.
    \item
        La preuve du lemme \ref{LemjXywjH}.
    \item
        La preuve du théorème de Fredholm \ref{ThoagJPZJ}.
    \item
        La preuve du lemme \ref{LemooynkH}.
    \item
        La partie «unicité» du théorème \ref{ThokUUlgU}.
    \item
        La preuve de la proposition \ref{PropRZCKeO}.
    \item
        L'inversion entre la somme et l'intégrale de l'équation \eqref{EqXSgZGw}.
    \item
        Les idéaux de \( A/I\) sont en bijection avec les idéaux de \( A\) contenant \( I\). Justification de l'équation \eqref{EqKbrizu}.
\end{enumerate}

Un peu partout des \verb+%TODO+ sont placés dans les sources \LaTeX. Ils décrivent des choses qu'il serait bon de faire. Si vous avez un avis sur l'un d'eux, n'hésitez pas à me le faire savoir.
\begin{quote}
    \texttt{grep TODO *.tex}
\end{quote}

