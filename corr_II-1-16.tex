% This is part of the Exercices et corrigés de CdI-2.
% Copyright (C) 2008, 2009
%   Laurent Claessens
% See the file fdl-1.3.txt for copying conditions.


\begin{corrige}{_II-1-16}

\begin{enumerate}

\item
$y''=y'^2+y^2$.
Étant donné que la variable $t$ n'intervient pas dans l'équation, poser
\begin{equation}
	z\big( y(t) \big)=y'(t)
\end{equation}
permet de diminuer l'ordre de l'équation. Le calcul de $y'$ et $y''$ en termes de $z$ est usuel, et il faut encore utiliser l'astuce
\begin{equation}
	y''(t)=\frac{ 1 }{2}\frac{ d }{ dy }\Big( z^2(y) \Big)_{y=y(t)}.
\end{equation}
L'équation à résoudre devient
\begin{equation}
	\frac{ 1 }{2}\frac{ d }{ dy }\Big( z^2(y) \Big)_{y=y(t)}=z\big( y(t) \big)^2+y(t)^2,
\end{equation}
où maintenant nous voyons $y(t)$ comme variable indépendante, ce qui fait que la fonction $z(u)$ vérifie l'équation
\begin{equation}
	\frac{ 1 }{2}\frac{ d }{ du }(z^2)=z^2+u^2.
\end{equation}
Pour résoudre cela, nous posons $v(z)=z^2$, et alors $v$ doit vérifier $\frac{ v' }{ 2 }=v+u^2$, dont la solution est
\begin{equation}
	v(u)=C e^{2u}-(u^2+u+\frac{ 1 }{2}).
\end{equation}
Nous trouvons donc l'équation suivante pour $y'$ :
\begin{equation}
	y'(t)=\big( C e^{2y}-y^2-y-\frac{ 1 }{2} \big),
\end{equation}
que nous pouvons résoudre à quadrature près :
\begin{equation}
	t-t_0=\int_{y_0}^y\frac{ dy }{ \big( C e^{2y}-y^2-y-\frac{ 1 }{2} \big) }.
\end{equation}


\item
$ayy''+by'^2+cy^2$.
Dans ce cas-ci, poser $z\big( y(t) \big)=y'(t)$ n'est pas suffisant; nous posons plutôt
\begin{equation}
	z\big( y(t) \big)=y'(t)^2.
\end{equation}
La dérivation par rapport à $t$ de cette équation donne
\begin{equation}		\label{EqDerrtChII116}
	z'\big( y(t) \big)y'(t)=2y'(t)y''(t)
\end{equation}
où $z'$ désigne la dérivée de $z$ par rapport à sa variable (c'est à dire $y$, et non $t$). Simplifions\footnote{Je ne sais pas très bien ce qu'il se passe si $y'$ vaut zéro en un point.} par $y'(t)$, et remplaçons dans l'équation de départ :
\begin{equation}
	ay(t)\frac{1 }{2}z'\big( y(t) \big)+bz\big( y(t) \big)+cy(t)^2=0.
\end{equation}
Si nous voyons maintenant $y(t)$ comme variable de la fonction $z$, la fonction $z(y)$ vérifie l'équation différentielle
\begin{equation}
	auz'+2bz+2cu^2=0,
\end{equation}
qui est une équation linéaire. L'équation homogène a pour solution
\begin{equation}
	z_H(u)=Au^{-2b/a}.
\end{equation}
La méthode de variation des constantes va fournir la solution générale à l'équation non homogène. L'équation différentielle que nous trouvons pour $A(u)$ est
\begin{equation}		\label{EqPourAVarCtII116}
	A'(u)=-\frac{ 2c }{ a }u^{1+\frac{ 2b }{ a }}.
\end{equation}
Avant d'écrire $A(u)$, il faut distinguer deux cas. En effet, $u^{\alpha}$ s'intègre différemment suivant que $\alpha=-1$ ou non. Les deux cas à distinguer sont
\begin{enumerate}
\item $1+2b/a=-1$, c'est à dire $a=-b$,
\item $1+2b/a\neq -1$, c'est à dire $a\neq -b$.
\end{enumerate}
Dans le premier cas, le plus simple est en fait de remonter à l'équation de départ et d'y mettre $a=-b$. Ce que nous trouvons est
\begin{equation}
	\frac{ yy''-y'^2 }{ y^2 }=-\frac{ c }{ a }.
\end{equation}
Si on remarque que le membre de gauche est $\big( \ln| y | \big)''$, alors on trouve
\begin{equation}
	\ln| y |=-\frac{ c }{ 2a }t^2+At+B,
\end{equation}
ou encore
\begin{equation}
	y(t)=\alpha  \exp\left( -\frac{ c }{ 2a }t^2+At \right)
\end{equation}
où $\alpha$ est une constante positive parce qu'elle provient de $ e^{B}$.

Pour le second cas ($a\neq -b$), nous continuons en résolvant \eqref{EqPourAVarCtII116} :
\begin{equation}		\label{EqExpAvarII116}
	A(u)=-\frac{ c }{ a+b }u^{2(1+b/a)}+L,
\end{equation}
et $z(u)=A(u)u^{-2b/a}$. En substituant l'expression \eqref{EqExpAvarII116} dans $z(u)$,
\begin{equation}
	z(u)=-\frac{ c }{ a+b }u^2+Lu^{-2b/a}.
\end{equation}
Maintenant nous nous souvenons que $z\big( y(t) \big)=y'(t)^2$, donc nous trouvons une nouvelle équation pour $y(t)$ :
\begin{equation}
	y'(t)^2=-\frac{ c }{ a+b }y(t)^2+Ly(y)^{-2b/a}.
\end{equation}
Cette équation est du type 
\begin{equation}
	f(y,y')=g(y)y'^2+h(y),
\end{equation}
avec $g(y)=1$, $f(y,y')=0$ et $h(y)=-\frac{ c }{ a+b }y^2+Ly^{-2b/a}$. Il y a une méthode pour elle à la page II.79, mais je propose d'arrêter ici.

\end{enumerate}
\end{corrige}
