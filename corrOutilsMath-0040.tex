% This is part of Exercices et corrigés de CdI-1
% Copyright (c) 2011
%   Laurent Claessens
% See the file fdl-1.3.txt for copying conditions.

\begin{corrige}{OutilsMath-0040}

    La dérivée de la hauteur en fonction du temps est
    \begin{equation}
        h'(t)=-gt+v_0,
    \end{equation}
    cette dérivée s'annule en $t=\frac{ v_0 }{ g }$ qui correspond à un maximum parce que $v_0$ et $g$ sont positifs (faire un tableau de signe de la dérivée).

    La hauteur maximale atteinte est donnée par
    \begin{equation}
        h\left( \frac{ v_0 }{ g } \right)=h_0+\frac{ v_0^2 }{ 2g }.
    \end{equation}

    \begin{verbatim}
sage: var('h0,g,v0,t')
(h0, g, v0, t)
sage: h(t)=h0-g*t**2/2+v0*t 
sage: solve(h.diff(t)==0,t)
[t == v0/g]
sage: h(v0/g)
1/2*v0^2/g + h0
    \end{verbatim}
    Notez que la fonction \info{solve} retourne une \emph{liste} de solutions. Ici la liste se réduit à un seul élément. Si on ne veut pas recopier à la main la solution, on peut faire comme ceci:
    \begin{verbatim}
sage: h( solve(h.diff(t)==0,t)[0].rhs()  )
1/2*v0^2/g + h0
    \end{verbatim}
    Ici nous avons écrit \info{[0]} pour indiquer que nous voulions la première solution (la numérotation des listes commence à zéro, et non à un!), et ensuite nous avons utilisé \info{.rhs()} pour demander le membre de droite de la solution.

\end{corrige}
