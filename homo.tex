Most of the material of this section can be found in a more general framework in the references \cite{Helgason, Loos, kobayashi, kobayashi2}. 

%+++++++++++++++++++++++++++++++++++++++++++++++++++++++++++++++++++++++++++++++++++++++++++++++++++++++++++++++++++++++++++
\section{Action of groups on sets}
%+++++++++++++++++++++++++++++++++++++++++++++++++++++++++++++++++++++++++++++++++++++++++++++++++++++++++++++++++++++++++++

Recall that the action of a group is \defe{transitive}{transitive} when it has only one orbit (i.e. each point can reach anyone other point). An action is \defe{free}{free!action}  if the fact that $g\cdot x=x$ for all $x\in M$ implies $g=e$. In other words, the action is free when $e$ is the only element to be represented by the identity. The action is \defe{simply transitive}{simply transitive action} when it has only one orbit and the stabilizer of one point is reduced to identity, in other words when $\forall\,(x,y)\in M^{2}$, $\exists!\,g\in G$ such that $x\cdot g=y$.

\begin{lemma}		\label{LemCompactSurFermeFerme}
	Let $G$ be a Lie group acting on a manifold $M$. Consider $K$, a compact subgroup of $G$ and $F$, a closed set in $M$. The set $K\cdot F$ is closed in $M$.
\end{lemma}

\begin{proof}
	We will prove that any sequence in $K\cdot F$ which converges in $M$ converges in $K\cdot F$. Let $\{ k_n\}\in K$ and $\{ \xi_n\}\in F$ and suppose that the sequence $\phi_n=k_n\cdot \xi_n$ converges to $\phi\in M$.

	Since $K$ is compact, the sequence $\{ k_n \}$ has a converging subsequence. Thus, without loss of generality, we can suppose that $k_n\to k\in K$ and $k_n\cdot \xi_n\to\phi\in M$. Since we are considering the action of a group, and since $K$ is a subgroup, we also have $k^{-1}_n\cdot \phi_n=\xi_n$. The action being continuous on $M$, we have
	\begin{equation}
		k_n^{-1}\cdot \phi_n\to k^{-1}\cdot \phi,
	\end{equation}
	so that $\xi_n\to k^{-1}\cdot\phi$. But $\{ \xi_n \}$ is a sequence in the closed space $F$. Thus its limit must belong to $F$: we have $\phi\in F$. Thus $k^{-1}\cdot\phi\in F$ and finally $\phi\in K\cdot F$.
\end{proof}

\subsection{Fundamental and invariant fields}
%--------------------------------------------
\label{Subsec_Funda_conv}

Let $G$ be a Lie group with Lie algebra $\lG$. For each element of $\lG$, there are two distinguished vector fields on $G$, the \defe{left invariant}{left invariant!vector field} and the \defe{right invariant}{right!invariant!vector field} one:
\begin{align}
\tilde X_g&=\Dsdd{  ge^{tX} }{t}{0}	&\utilde X_g&=\Dsdd{ e^{tX}g }{t}{0}\\
dL_h\tilde X_g&=\tilde X_{hg}		&dR_h\utilde X_g&=\utilde X_{gh}.
\end{align}

When $G$ is a Lie group with an action on the manifold $M$ denoted by
\begin{equation}
\begin{aligned}
 \tau\colon G\times M&\to M \\ 
(g,x)&\mapsto \tau_g(x),
\end{aligned}
\end{equation}
we define the \defe{fundamental vector field}{fundamental!vector field} associated with $X\in\lG$ on the point $x\in M$ by
\begin{equation}			\label{EqDefChmpFonfOff}
X^*_x=\Dsdd{ \tau_{ e^{-tX}}(x) }{t}{0}.
\end{equation}
An usual case is the one of a Lie group acting on itself for which we have
\begin{equation}		\label{EqChmpFondGp}
  X^*_g=\Dsdd{ e^{-tX}g }{t}{0}.
\end{equation}

\section{Rough introduction to homogeneous spaces}
%--------------------------------------------------
\label{SubSechoappahomsp}\label{SecRoughomo}

An \defe{homogeneous space}{homogeneous!space}\label{pg:esp_homo} is a differentiable manifold with a transitive diffeomorphism group. 

An important class of homogeneous space is given by the coset spaces. When we have a topological group $G$ and a closed subgroup $H$, the coset space $G/H$ has a structure of homogeneous space. Theorem \ref{tho:homeo_action} shows that almost every homogeneous space is of this class. For this, we use classes on right:
\[
  [g]=\{gh:h\in H\}.
\]
The canonic projection is $\pi\colon G\to M$ and we denote $\mfo=[e]$. The following construction shows that (almost\footnote{Problems are possible with topology choices and differentiability of certain maps.}) every homogeneous space are of this kind.

 Let $M$ be a homogeneous space; $\mfo\in M$, a point; $G$, a group which acts transitively on $M$ (in particular, $G\mfo=M$); and $H$, the stabilizer of $\mfo$ in $G$. Then, the map $[g]\mapsto g\mfo$ is a homogeneous space isomorphism between $M$ and $G/H$. This thesis only deals with this kind of homogeneous spaces. The Lie algebras of $G$ and $H$ are denoted by $\lG$ and $\lH$ respectively.


Let $\dpt{\pi}{G}{M=G/H}$ be the canonical projection. We denote $\mfo:=[e]$. It is clear that $\dpt{d\pi_e}{\lG}{T_{\mfo}M}$ is surjective.
\begin{proposition}
	The kernel of the differential of the projection is given by
	\begin{equation}
 		\ker(d\pi_e)=\lH.
	\end{equation}
\end{proposition}

\begin{proof}
It is easy to see that $\lH\subset \ker(d\pi_e)$ but it turns out to be non trivial to prove the inverse. Our demonstration follows a part of the one of the proposition 4.3 of \cite{Helgason} (cf proposition \ref{propHelgason4.3}).

Let $X$ be in $\ker(d\pi_e)$. Since $d\pi_eX\in T_{\mfo}M$, it can be applied on a function $\dpt{f}{M}{\eR}$. As $d\pi_eX=0$ on any function and $X=\dsdd{\exp tX}{t}{0}$, we have
\begin{equation}
   0=(d\pi_eX)f=\dsdd{f(\pi\circ X)(t)}{t}{0}
               =\dsdd{f([\exp tX])}{t}{0}                      \label{r2904e1}
\end{equation}

But proposition \ref{Helgason4.2} makes $\exp sX\in G$ acting on $M$. As a $f$, we can consider $g(q)=f(\exp sX\cdot q)$. Replacing $f$  by $g$ in \eqref{r2904e1}, we get:
\begin{equation}
   0=(d\pi_eX)g=\dsdd{g([\exp tX])}{t}{0}
               =\dsdd{f([\exp(s+t)X])}{t}{0}
\end{equation}

Then for any function $f$, the number $f([\exp tX])\in\eR$ doesn't depend on $t$, but for $t=0$, $[\exp tX]=\mfo$. Then $\forall t\in\eR$, $\exp tX\in H$, and therefore, $X\in \lH$.
\end{proof}

\begin{lemma}
	We have
	\begin{equation}  \label{Eqdpigdtaudpi}
		d\pi_g\circ dL_g=d\tau_g\circ d\pi_e.
	\end{equation}
\end{lemma}

\begin{proof}
	Let $X\in \lG$ be the tangent vector to the curve $X(t)$ in $G$. We have
	\begin{equation}
		(d\pi_g\circ dL_g)(X)=\Dsdd{ \pi\big( gX(t) \big) }{t}{0}=\Dsdd{ \tau_g\pi\big( X(t) \big) }{t}{0}=(d\tau_g\circ d\pi_e)(X)
	\end{equation}
	where we used the fact that, by definition, the action is given by $\tau_g\pi(g')=\pi(gg')$.
\end{proof}

\begin{definition}
	The homogeneous space $M=G/H$ is said to be \defe{reductive}{reductive!homogeneous space} if there exists a subspace $\lM$ of $\lG$ such that
		\begin{itemize}
		\item $\lM\oplus\lH=\lG$,
		\item $[\lH,\lM]\subset\lM$.
	\end{itemize}
\end{definition}
Because of the second condition, such a $\lM$ is said to be \defe{$H$-invariant}{h@$H$-invariant subspace}.

\begin{lemma}		\label{LemdpiisomMTM}
If $\lM$ is reductive, the restriction $\dpt{d\pi_e}{\lM}{T_{\mfo}M}$ is an isomorphism between $\lM$ and $T_{\mfo}M$.
\end{lemma}
\begin{proof}
As $\dpt{d\pi_e}{\lG}{T_{\mfo}M}$ is surjective, $\lG=\lH\oplus\lM$ and $\lH$ is the kernel, $\dpt{d\pi_e}{\lM}{T_{\mfo}M}$ must be surjective. On the other hand, if we have $d\pi_em=d\pi_en$ for $n$, $m\in\lM$, $(m-n)\in Ker(d\pi_e)=\lH$ which is impossible because $\lG=\lH\oplus\lM$ is a direct sum.
\end{proof}

We can generalize this proposition by considering the space $\lQ_g=dL_g\lQ$. 
\begin{proposition} 		\label{PropDiffPiBijTgGH}\label{Cordpiietwii}
The differential $\dpt{d\pi}{\lM_g}{T_{[g]}M}$ of the canonical projection provides an isomorphism between $\lM_g$ and $T_{[g]}M$.
\end{proposition}

\begin{proof}
In order to prove injectivity, take a $X\in \lM_g$ (i.e. $X=dL_gX'$ for a certain $X'\in\lM$) such that $d\pi X=0$. If $X'=X'_h+X'_m$, we have
\[ 
0=\Dsdd{ \pi\big( g e^{tX'_h+tX'_m} \big) }{t}{0}
		=\Dsdd{ \pi\big( g e^{tX'_m} e^{tX'_h} \big) }{t}{0}
		=\Dsdd{ \pi\big( g e^{tX'_m} \big) }{t}{0}
		=d\tau_g d\pi X'_m
\]
where $d\pi X'_m\neq 0$ by definition of the quotient. Now, $d\tau_g\colon T_{[e]}M\to T_{[g]}M$ is a surjective linear map between two vector spaces of same dimension. Thus $d\tau_g$ is bijective and $d\pi X'_m=0$, which proves that $X'_m=0$ by lemma \ref{LemdpiisomMTM}.
\end{proof}

The homogeneous space $G/H$ is endowed with its \defe{natural topology}{natural topology} which is defined by the requirement that the projection $\pi$ is continuous and open. We refer to \cite{Helgason} for the properties of that topology.

\subsection{Killing induced product}		\label{SubsecKillHomo}
%----------------------------------

The product will be described with more details in point \ref{SubSubSecTheKillingHomo}.

Since the Killing form $B$ is an $\Ad_H$-invariant product on $\lQ$, we can define
\begin{equation}
B_g(X,Y)=B_e(dL_{g^{-1}}X,dL_{g^{-1}}Y)
\end{equation}
which descent (see \cite{Kerin} for properties) to a homogeneous metric on $T_{[g]}M$:
\begin{equation}  \label{EqDefMetrHomo}
B_{[g]}(d\pi X,d\pi Y)=B_g(\pr X,\pr Y)
\end{equation}
where $\dpt{\pr}{T_gG}{dL_g\lQ}$ is the canonical projection. An useful property of that projection is $\pr(dL_gX)=dL_gX_Q$ when $X=X_Q+X_H$. Using that property, we can write the product under the more manageable form
\[ 
  B_{[g]}(d\mu_gX,d\mu_gY)=B_e(\pr X,\pr Y)
\]
for all $X$, $Y\in\lG$ where we wrote $\mu_g=\tau_g\circ \pi$.

Although equation \eqref{EqChmpFondGp} looks like \eqref{EqDefChmpFonfOff}, we find a major difference here: the norm of $q_i^*[g]$ is not a constant. One should expect that it was a constant because \eqref{EqDefChmpFonfOff} expresses a left translation while the Killing form is invariant under left translations. But the metric \eqref{EqDefMetrHomo} is a composition of the Killing form with a projection. Let us study this case in details in computing the product of two vectors of the form
\[ 
  X^*_{[g]}=d\pi\Dsdd{  e^{-tX}g }{t}{0},
\]
with $X\in\lQ$:
\[ 
\begin{split}
  B_{[g]}(X^*,Y^*)&=B_g\big( \pr\Dsdd{  e^{-tX}g }{t}{0},\pr\Dsdd{  e^{-tY}g }{t}{0} \big)\\
		&=B_g\big( dL_g\pr\Ad(g^{-1})X,dL_g\pr\Ad(g^{-1})Y \big)\\
		&=B_e\Big(   \big( \Ad(g^{-1})X \big)_{\lQ},\big( \Ad(g^{-1})Y \big)_{\lQ}  \Big)\\
		&\neq B_e\Big(   \Ad(g^{-1})X_{\lQ},\Ad(g^{-1})Y_{\lQ}  \Big)\\
		&=B_e(X,Y)
\end{split}
\]
where the symbol $\neq$ has to be understood as ``not equal in general'' because equality holds of course for certain particular vectors such as zero.


\subsection{Homogeneous space}
%-----------------------------


\begin{proposition}
Let $M=G/K$ be a homogeneous space where the Lie algebra $\lG$ has the Cartan decomposition $\lG=\lK\oplus\lP$. Then
\begin{enumerate}
\item $T_{[\mtu]}M=\lP$
\item $TM=G\times_{\Ad(K)}\lP$
\end{enumerate}

\end{proposition}

\begin{proof}
The first part is already know. For the second, an element of $G\times_{\Ad(K)}\lP$ is of the form $[g,X]$ where $g\in G$, $X\in\lP$ and the equivalence relation $(g,X)\sim(gk,\Ad(k)X)$ for all $k\in K$.

Let us define $\psi\colon G\times_{\Ad(K)}\lP\to TM$ by
\[ 
  \psi[g,X]=[dL_gX]
\]
where $[Y]=d\pi_gY$ when $Y\in T_gG$. We are going to prove that $\psi$ is injective and surjective. Suppose $\psi[g,X]=\psi[h,Y]$. Since $[dL_gX]=[dL_hY]$, we have $T_{[g]}M=T_{[h]}M$ and there exists a $k\in K$ such that $h=gk$. We have to prove that $\Ad(k)X=Y$. We have
\[ 
  d\pi_g(dL_gX)=d\pi_{gk}(dL_{gk}Y),
\]
but
\[ 
  d\pi_{gk}(dL_{gk}Y)=\Dsdd{ \pi(gkY(t)) }{t}{0}
		=\Dsdd{ \pi(gkY(t)k^{-1}) }{t}{0}
		=d\pi_gdL_g\Ad(k)Y.
\]
This proves injectivity of $\psi$. For surjectivity, take $X_{[g]}\in T_{[g]}M$: there exists a $\tilde X_g\in t_gG$ such that $X_{[g]}=d\pi_g\tilde X_g$. So, for a certain $X\in\lG$, 
\[ 
  X_{[g]}=d\pi_gdL_gX
		=[dL_gX]
\]
It remains to be proved that one can choose $X\in\lP$. Let us decompose $X=X_p+X_k$; it gives
\[ 
  [dL_g(X_p+X_k)]=[dL_gX_p]+[dL_gX_k],
\]
but the latter is
\[ 
  \Dsdd{ \pi(gX_k(t)) }{t}{0}=\Dsdd{ \pi(g) }{t}{0}=0
\]
because $X_k(t)\in K$ by definition.
\end{proof}



Let us consider $G/H$, a homogeneous space. If the Lie algebra $\lH$ is moreover the set of points fixed by an involution $\dpt{\theta}{\lG}{\lG}$, the quotient $G/H$ is said to be a \defe{symmetric space}{symmetric!space}.

Let us point out that the Iwasawa decomposition naturally gives rise to a symmetric space: $AN=G/K$.

\begin{remark}
This is not our final definition of a symmetric space. A more precise definition will be given later, see section \ref{sec:symm}.
\end{remark}

%///////////////////////////////////////////////////////////////////////////////////////////////////////////////////////////
					\subsubsection{Frame bundle over reductive homogeneous spaces}
%///////////////////////////////////////////////////////////////////////////////////////////////////////////////////////////
\label{PgFrameHomo}

Let us consider an homogeneous space of the for $M=G/H$ with $G=\SO_0(p,q)$, and $\mG=\lH\oplus\lQ$. We have $T_{\mfo}(G/H)=\lQ$ and $T_{[g]}(G/H)=dL_g\lQ=\lQ_g$. Let $V=\eR^{p,q}$ on which $G$ acts by definition. Let $B(\mfo)$ the set of orthonormal frames of $\lQ$: the linear isometries $b\colon V\to \lQ$. The we consider
\begin{equation}
	B\big( [G] \big)=\{ [v\mapsto dL_gb(v)]\tq b\in B(\mfo) \}.
\end{equation}
The frame bundle of $G/H$ is
	\begin{equation}
\xymatrix{%
   \SO(p,q) \ar@{~>}[r]		&	B\big( [G] \big)\ar[d]^{\pi}\\
   				&	G/H,
 }
\end{equation}
where the action of $\SO(p,q)$ is given by
\begin{equation}
	(b\cdot g)(v)=b(gv).
\end{equation}

\subsection{Invariant metric on homogeneous space (first)}
%---------------------------------------------------------

Let $G/H$ be a reductive homogeneous space, i.e. $\lG=\lH\oplus\lM$ with $[\lH,\lM]\subseteq\lM$. We denote by $T_A,T_B,\ldots$ the generators of $\lG$ while $T_i,T_j,\ldots$ particularise the generators of $\lH$ and $T_a,T_b,\ldots$ the ones of $\lM$. The reducibility condition reads
\begin{equation}
  [T_i,T_a]=C_{ia}^AT_A=C_{ia}^bT_b.
\end{equation}

\begin{probleme}        \label{ProbAvecCorwell}
Vas voir dans Cornwell (que tu dois ajouter \`a la biblio) comment on fait pour montrer que $C_{AB}^C$ est compl\`etement antisym dans tout les sens. Avec \c ca, je devrais pouvoir dire que $C_{ij}^A$ sont nuls.
\end{probleme}

An element of $G$ can be locally parametrized with $\dim G$ real numbers; for example
\[
   g(y^a,x^i)=e^{y^aT_a}e^{x^iT_i}
\]
in a neighbourhood of identity. The classes $[g]\in G/H$ are given by only $\dim G-\dim H=\dim\lM=m$ real numbers and we can consider a choice of a representative of each class, i.e. a map $\dpt{L}{\eR^m}{G}$ with $L(y)\in[g]$ if $g=e^{y^aT_a}h$. For example, a possible choice is
\[
  L(y)=e^{y^aT_a}.
\]
If we multiply at left $L(y)$ by $g\in G$, we obtain an element of another class whose representative is $L(y')$. Then
\begin{equation}\label{eq:gLyL}
  gL(y)=L(y')h
\end{equation}
where $y'\in\eR^m$ and $h\in H$ both depend on $g$ and $y$ (and the choice of the representative $L$). We consider the $\lG$-valued $1$-form on $\eR^m$ defined at $y\in\eR^m$ by
\begin{equation}
    V(y)=L(y)^{-1} dL_y.
\end{equation}
If $v(t)$ is a path in $\eR^m$ with $v(0)=y$, it defines a vector $v\in T_y\eR^m$ and
\[
  V(y)v=\Dsdd{ L(y)^{-1} L(v(t)) }{t}{0}\in\lG.
\]
So we can develop it with respect to a basis of $\lG$:
\[
   V(y)=T_aV^a(y)+T_i\Omega^i(y).
\]
Now we are going to write $V(y')$ when $y'$ is given by relation \eqref{eq:gLyL}. First, it is clear that $L(y')^{-1}=hL(y)^{-1} g^{-1}$. Now if $y'=f(y)$, we have
\[
   d(l\circ f)_y=dL_{y'}\circ df_y,
\]
then
\begin{equation}\label{eq:VyphL}
V(y')=hL(y)^{-1} g^{-1} d(L\circ f)_y\circ(df^{-1})_{y'}.
\end{equation}
We will forget the $(df^{-1})_{y'}$, keeping in mind that if $V(y')$ is applied to a vector of $T_y\eR^m$, we have to transport the vector with $f$. Now let us explicit the expression \eqref{eq:VyphL}. For this, remark that $L\circ f=gL(\cdot)h^{-1}$ where $h$ is a map from $\eR^m$ to $H$. Then we have tu use the Leibnitz formula; let $v\in T_y\eR^m$,
\begin{equation}
  d(L\circ f)_yv=\Dsdd{ gL(v(t))h^{-1}(v(t)) }{t}{0}
                =g(dLv)h^{-1}(y)+gL(y)(dh^{-1} v).
\end{equation}
So,
\begin{equation}
\begin{split}
   V(y')&=hL(y)^{-1} g^{-1}\big(  g(dL_y) h^{-1}+gL(y)dh^{-1}_y   \big)\\
        &=h V(y)h^{-1}+hdh^{-1}_y.
\end{split}
\end{equation}
Then the transformation rule of $V$ under an action of $G$ is given by
\begin{equation}
    V(y')=hV(y)h^{-1}+hdh^{-1}_y.
\end{equation}
In particular this induces a transformation rule for $V^a$ by 
\begin{equation}\label{eq:trans_V}
V^a(y')=\big(  hV(y)h^{-1}  \big)^a=\big(\Ad(h)V(y)\big)^a=V^A(y)\bghd{D(h^{-1})}{A}{a}
\end{equation}
where $D$ is defined by $\Ad(g^{-1})T_A=\bghd{D(g)}{A}{B}T_B$.

\subsubsection{Infinitesimal expressions}
%////////////////////////////////////////

Now we want to write the equation \eqref{eq:gLyL} in the case where $g$ is close to the identity. We start by considering $g$ under the form $g=e^{\epsilon^AT_A}$ with small $\epsilon$. If we write $h$ under the form $h=e^{R^iT_i}$, $R^i$ is a function of $y$ and $\epsilon$. If we suppose that $\epsilon$ is very small (our intention is to make a derivation with respect to epsilon at $\epsilon=0$), we can suppose that $R$ is linear with respect to epsilon. Then $h=e^{\epsilon^AW_A^i(y)T_i}$. For the same reason, $y'$ is linear with respect to epsilon: ${y'}^{\alpha}=y^{\alpha}+\epsilon^AK_A^{\alpha}(y)$. So we write
\[
   e^{\epsilon^AT_A}L(y)=L(  y^{\alpha}+\epsilon^AK_A^{\alpha}(y)  )e^{\epsilon^AW_A^i(y)T_i}
\]
which an equality in $G$. Let us derive it with respect to $\epsilon^A$ at $\epsilon=0$. Note that by $L(y^{\alpha})$, we mean $L(y^{\alpha} e_{\alpha})$ where $\{e_{\alpha}\}$ is the canonical basis of $\eR^m$. Then we find
\[
   T_AL(y)=dL_y (  K^{\alpha}_A(y)\partial_{\alpha})+L(y)W_A^i(y)T_i.
\]
If one multiply it at left by $L(y)^{-1}$,
\begin{equation}\label{eq:DLAB}
    \bghd{D(L(y))}{A}{B}T_B=V(y)\big(  K_A^{\alpha}(y)\partial_{\alpha}  \big)+W_A^i(y)T_i.
\end{equation}
Remark that $K_A^{\alpha}(y)$ is just a real number, then it can get out the form $V(y)$. From notational convenience, we write $V(y)\partial_{\alpha}=V_{\alpha}(y)$. We write separately the $\lH$ and $\lM$ components in equation \eqref{eq:DLAB}:
\begin{subequations}\label{eq:DlyA}
\begin{align}
 \bghd{D(L(y))}{A}{i}T_i&=K_A^{\alpha}(y)(\Omega^i(y)T_i)(\partial_{\alpha})+W_A^i(y)T_i \\
 \bghd{D(L(y))}{A}{b}T_b&=K_A^{\alpha}(y)(V^b(y)T_b)(\partial_{\alpha}).
\end{align}
\end{subequations}
Be careful on one fact: the expression $V^b(y)T_b(\partial_{\alpha})$ means $V^b(y)(\partial_{\alpha})T_b$ where which is the product of the vector $T_b\in\lG$ by the real $V^b(y)T_b(\partial_{\alpha})$. So we can ``simplify''{} the $T_A$'s in equations \eqref{eq:DlyA} to find
\begin{subequations}
\begin{align}
  W_A^i(y)            &= \bghd{D(l(y)}{A}{i}-K_A^{\alpha}(y)\Omega^i_{\alpha}(y)\\
  \bghd{D(l(y)}{A}{b} &= K_A^{\alpha}(y)V_{\alpha}^b(y)
\end{align}
\end{subequations}
whose are equalities in $\eR$.

Let us find a form for $\bghd{D(h^{-1})}{A}{b}$ when $h$ is given by equation \eqref{eq:gLyL} with ${y'}^{\alpha}=y^{\alpha}+\epsilon^A K_A^{\alpha}(y)$ and for small $\epsilon$. The matrix $D(g)$ is given by
\[
   \bghd{D(g)}{A}{B}T_B=g^{-1} T_A g=\Dsdd{\AD_{g^{-1}} e^{tT_A}}{t}{0}=\Ad(g^{-1})T_A.
\]
So in our case,
\begin{equation}
  \bghd{ D(e^{\epsilon^BW_B^i(y)T_i}) }{A}{C}T_C=\Ad\big(\exp(\epsilon^BW_B^i(y)T_i)\big)T_A.
\end{equation}
If we derive it with respect to $\epsilon^B$ at $\epsilon=0$, we find
\begin{equation}
\begin{split}  
    \Dsddb{  \bghd{D( \exp(\epsilon^BW_B^i(y)T_i)  )}{A}{C}T_C  }{\epsilon^B}{\epsilon}{0}
              &=\ad(W_B^i(y)T_i)T_A\\
              &=W_B^i(y)\bghd{C}{iA}{D}T_D,
\end{split}
\end{equation}
so that we can power expand $\bghd{D(h^{-1})}{A}{a}$ with respect to $\epsilon$ around $\epsilon=0$:
\begin{equation}\label{eq:Dinfin}
  \bghd{D(h^{-1})}{A}{a}=\delta^a_A+\epsilon^BW_B^i(y)\bghd{C}{iA}{a}+\ldots
\end{equation}
Then equation \eqref{eq:trans_V} reads $V^a(y')-V^a(y)=V^a(y)\epsilon^BW_B^i(y)\bghd{C}{iA}{a}$, but remarking that the reducibility makes $C_{ij}^a=0$,
\begin{equation}
 V^a(y+\delta y)-V^a(y)=\epsilon^BW_B^i(y)\bghd{C}{ib}{a}V^b(y).
\end{equation}
The fact that $\bghd{C}{ib}{a}$ can be made skews-symmetric shows that this equation describe the infinitesimal action of $G$ on $V(y)$ by the action of $so(n)$. It allows us to state the following theorem.


\subsubsection{Invariant metric}
%///////////////////////////////


\begin{theorem}
The metric
\begin{equation}\label{eq:metric_GH}
   g\bab=g(\partial_{\alpha},\partial_{\beta})=\delta_{ab}V_{\alpha}^aV_{\beta}^b
\end{equation}
is invariant with respect to the left action of $G$. 
\end{theorem}

\begin{probleme}
Regardes si il faut semisimple pour obtenir l'antisymétrie des constantes de structure.
\end{probleme}

An other way to write this metric is 
\[
  g=\delta_{ab}(V^a\otimes V^b).
\]

\begin{proof}
We have to show that $g_{y'}(\partial_{\alpha},\partial_{\beta})=g_y(\partial_{\alpha},\partial_{\beta})$. For this we will show that the derivative of $g\bab(y)$ with respect to $y$ is zero. So we write ${y'}^{\alpha}+\epsilon^AK_A^{\alpha}(y)$ and 
\[
   g\bab(y')=\delta_{ab}V^a_{\alpha}(y)V^b_{\beta}(y)\bghd{D(h^{-1})}{A}{a}\bghd{D(h^{-1})}{B}{b}.
\]
The computation is performed using \eqref{eq:Dinfin} which gives $\left.\bghd{ D(h^{-1}) }{A}{a}\right|_{\epsilon=0}=\delta^a_A$ and 
\begin{equation}
\begin{split}
  \Dsddb{  g\bab(y')  }{\epsilon^C}{\epsilon}{0}&=\delta_{ab}V_{\alpha}^A(y)V_{\beta}^B
                                       \left\{  
     \Dsddb{  \bghd{D(h^{-1})}{A}{a}  }{\epsilon^C}{\epsilon}{0}\left.\bghd{D(h^{-1})}{B}{b}\right|_{\epsilon=0}.
                                       \right.\\
 &\phantom{ =\delta_{ab}V_{\alpha}^A(y)V_{\beta}^B  }\quad  \left. \left.\bghd{D(h^{-1})}{A}{^a}\right|_{\epsilon=0}\Dsddb{ \bghd{D(h^{-1})}{B}{b} }{\epsilon^C}{\epsilon}{0}\right\}\\
&=\delta_{ab}V^a_{\alpha}(y)V_{\beta}^B(y)\left[  \delta_B^bW^i_C(y)\bghd{C}{iA}{a}+\delta_A^a W^i_C(y)\bghd{C}{iB}{b}  \right]\\
&=\sum_a V^A_{\alpha}(y)V_{\beta}^a(y)W^i_C(y)\bghd{C}{iA}{a}+\sum_b V^B_{\alpha}(y)V_{\beta}^B(y)W^i_C(y)\bghd{C}{iB}{b}
\end{split}
\end{equation}
Taking into account the fact that $\bghd{C}{ij}{a}=0$, one can reduce some summations like $V_{\alpha}^A(y)\bghd{C}{iA}{a}=V_{\alpha}^b(y)\bghd{C}{ib}{a}$. Using the antisymmetry of $\bghd{C}{ib}{a}$ with respect to $a,b$, we find that the sum is zero.

\end{proof}

There is an other invariant metric:
\begin{equation}
   g\bab=B_{ab}V_{\alpha}^aV_{\beta}^b
\end{equation}
where $B$ is the matrix of the Killing form. Following the same proof of the invariance than the previous one, one finds
\[
  \Dsddb{ g\bab(y') }{\epsilon^C}{\epsilon}{0}=V^a_{\alpha}(y)V^b_{\beta}(y)W_C^i( B_{cb}\bghd{C}{ia}{c}+
                               B_{ac}\bghd{C}{ib}{c} ).
\]
This is zero because of the formula $B((\ad X) Y,Z)=-B(Y,(\ad X) Z)$. 

\begin{probleme}
Apparement cette métrique est invariante indépendament d'hypothèse de semi simplicité.
\end{probleme}

\subsubsection{The choice of \texorpdfstring{$L(y)$}{L(y)}}
%///////////////////////////////////

Let us see what happens if we had done the work with $L'(y)=L(y)h(y)$ instead of $L(y)$. In this case, 
\[
   V'(y)=L'(y)^{-1} dL'_y=h(y)^{-1} L(y)^{-1} dL'_y,
\]
but using Leibnitz formula, we find
\begin{equation}
   dL'_yv=\Dsdd{L(v(t))h(v(t))}{t}{0}
         =(dL_yv)h(y)+L(y)dh_yv,
\end{equation}
so that $V'(y)=V(y)$ up to a renaming $h\leftrightarrow h^{-1}$. The conclusion is that $\delta_{ab}V_{\alpha}^a V_{\beta}^b$ and $B_{ab}V_{\alpha}^aV_{\beta}^b$ are independent of the choice of $L$.

\subsection{Homogeneous metric on homogeneous spaces}
%----------------------------------------------------------

\subsubsection{One way to obtain it}
%////////////////////////////////////

Let $M=G/H$ be a homogeneous space. A Riemannian metric $\scald{\cdot}{\cdot}$ on $M$ is \defe{homogeneous}{homogeneous!Riemannian metric} when
\begin{equation}\label{eq:def:homo_metric}
\scald{dL_gv}{dL_gw}_{g[x]}=\scald{v}{w}_{[x]}
\end{equation}
for all $g\in G$, $[x]\in M$. Note that this formula cannot define an inner product on each $T_{[x]}M$ from the data of an inner product on $T_{\mfo}M$ because --unless certain conditions-- it is not well-defined.

From the definition of the homogeneous structure of $G/H$, all element of $H$ fixes $\mfo=[e]$ (by the left action).Then $dL_h$ is an automorphism of $T_{\mfo}M$ and we can define the \defe{isotropic representation}{isotropic!representation}\index{representation!isotropic} $\dpt{\rho}{H}{\Aut(T_{\mfo})M}$ by
\[
   \rho(h)X=dL_hX
\]
with $X\in T_{\mfo}M$.

Now let $\scald{\cdot}{\cdot}$ be an inner product on $T_{\mfo}M$ (for example the Killing form on the $\lM$ part of $\lG=\lM\oplus\lH$ in the reductive case). We can try to export this product at $[g]$ by the formula 
\begin{equation}\label{eq:scal_gdee}
  \scald{v}{w}_{[g]}=\scal{dL_{g^{-1}}v}{dL_{g^{-1}}w}_{\mfo}.
\end{equation}

\begin{proposition}
The product $\scald{\cdot}{\cdot}_{[g]}$ defined by formula \eqref{eq:scal_gdee} is well defined if and only if $\scald{\cdot}{\cdot}_{\mfo}$ is invariant under the isotropic representation.
\end{proposition}

\begin{proof}
Let us proof the necessary condition; the sufficient one is just the same written backward. The assumption makes 
\begin{equation}
\begin{split}
  \scald{v}{w}_{[gh]}&=\scald{dL_{h^{-1} g^{-1}}v,}{dL_{h^{-1} g^{-1}}w}_{\mfo}\\
                     &\stackrel{!}{=}\scald{dL_{g^{-1}}v}{dL_{g^{-1}}w}_{\mfo}
\end{split}
\end{equation}
for every $v$, $w\in T_{[g]}M$, $g\in G$, $h\in H$. In particular,
\[
   \scald{dL_h X}{dL_hY}_{\mfo}=\scald{X}{Y}_{\mfo}
\]
for all $X$, $Y\in T_{\mfo}M$.
\end{proof}

Two remarkable properties of this inner product are the fact that it is a homogeneous Riemannian structure and that \emph{all} the homogeneous metric are such. In order to see the first claim, just remark that if $[x]\in M$,
\begin{equation}
\begin{split}
\scald{dL_gv}{dL_gw}_{g[x]}&=\scald{dL_{x^{-1} g^{-1}}dL_gv}{dL_{x^{-1} g^{-1}}dL_gw}_{\mfo}\\
                           &=\scald{dL_{x^{-1}}v}{dL_{x^{-1}}w}_{\mfo}\\
                           &=\scald{v}{w}_{\mfo}.
\end{split}
\end{equation}
The second claim comes from the choice $g=x^{-1}$ in the definition \eqref{eq:def:homo_metric}.

\subsubsection{One other way to obtain it}
%//////////////////////////////////////////
\label{SubSubSecTheKillingHomo}

Let us consider a metric on $\lG$ and see in which case it can be extended to gives rise to a well defined homogeneous metric on the quotient $M=G/H$. Let $\lG=T_eG$, $\lH=T_eH$ and $\lG=\lM\oplus\lH$. Using $dL$, we can propagate the space $\lM$ to the point $g\in G$ by defining
\[
  \lM_g=dL_g\lM.\label{pg:M_g}
\]
We saw in proposition \ref{PropDiffPiBijTgGH} that $\lM_g$ was isomorphic to $T_{[g]}M$.


Let $\scald{.}{.}$ be a product on $\lG$ which is $\Ad_H$-invariant on $\lM$. We claim that the following construction gives a well defined and homogeneous product on $\lG$. First, the product on $\lG$ extends to a product on $T_gG$ for every $g$ by
\[
  \scald{X}{Y}_g=\scald{dL_{g^{-1}}X}{dL_{g^{-1}}Y};
\]
this induces the following inner product on $T_{[g]}(G/H)$ that will reveal to be well defined under the current assumptions:
\begin{equation}\label{eq:scal_TgM}
	\scald{d\pi_g X}{d\pi_g Y}_{[g]}=\scald{X}{Y}_g
\end{equation}
where $X,Y\in \lM_g=dL_g\lM$. Indeed, the map $\dpt{d\pi}{\lM_g}{T_{[g]}M}$ is an isomorphism, hence for all $v\in T_{[g]}M$, there exists one and only one $X\in\lM_g$ such that $d\pi X=v$. Since $\dpt{d\pi}{\lM_{gh}}{T_{[g]}M}$ is also an isomorphism, the condition for \eqref{eq:scal_TgM} to be a good definition, we must have 
\begin{equation}
\scald{X}{Y}_g=\scald{dL_{g^{-1}} X}{dL_{g^{-1}} Y}_e
              =\scald{X'}{Y'}_{gh}
      \end{equation}
where $X'=dR_hX$. It is easy to remark that this condition is the $\Ad_H$-invariance of the inner product $\scald{\cdot}{\cdot}_e$ defined on $\lG$.

The reader should remark that all the conditions are satisfied by the Killing inner product.

Now, if $X$ is any element of $\lG$, we define successively
\begin{equation}		\label{EqDefProdGsurH}
	\begin{aligned}[]
		\scald{d\pi_gdL_gX}{d\pi_gdL_gY }_{[g]}&=\scald{d\pi_gdL_gX_{\lM}}{d\pi_gdL_gY_{\lM}}_{[g]}\\
		&=\scald{dL_gX_{\lM}}{dL_gY_{\lM}}_g\\
		&=\scald{\pr_{\lM}X}{\pr_{\lM}Y}_e.
	\end{aligned}
\end{equation}
The last line is the usual Killing form on $\lG$, or any other inner product which has the right properties.

Let us prove that the first line is well defined. First, notice that $d\pi_g\colon \lM_g\to T_{[g]}M$ is an isomorphism, thus there exists one and only one $\tilde X\in dL_g\lM$ such that $d\pi_g \tilde X=d\pi_gdL_g X$. Since
\begin{equation}
	d\pi_g\big( dL_g X_{\lM} \big)=\Dsdd{ \pi\big( g e^{tX_{\lM}} \big) }{t}{0}=0,
\end{equation}
we know that
\begin{equation}
	d\pi_gdL_g X_{\lQ}=d\pi_gdL_g X
\end{equation}
for every $X\in \lG$.

\section{Symmetric spaces}\label{sec:symm}
%++++++++++++++++++++++++
This section is mainly taken from \cite{Loos,Dixmier,SSSSS,Dieu2}.

\subsection{Basic facts}
%-----------------------

\begin{definition} 
A \defe{symmetric space}{symmetric!space} is a manifold $M$ and an analytic ``multiplication``\ $\dpt{\mu}{M\times M}{M}$ --written $s_x(y)$ as $\mu(x,y)$-- such that

\begin{enumerate}
\item $\forall x\in M$, $s_x$ is an involutive diffeomorphism of $M$ called ``the symmetry at $x$ ``,

\item $\forall x\in M$, $x$ is an isolated fixed point of $s_x$,
\item $\forall x,y\in M$ , $s_x\circ s_y\circ s_x=s_{s_x(y)}$.
\end{enumerate}
\label{def:esp_sym}
\end{definition}
\begin{definition}
An \defe{homomorphism}{homomorphism!of symmetric space} of symmetric space $(M,s)$ and $(M',S)$ is an analytic map $\dpt{\varphi}{M}{M'}$ which satisfies
\[
    \varphi( s_x(y) )=S_{\varphi(x)}\varphi(y).
\]
\end{definition}

Immediately, for any $z$ in $M$, the symmetry $s_z$ is an automorphism of $M$ (as symmetric space). Indeed,
\begin{equation}
  s_{s_z(x)}(s_z(y))=s_z\circ s_x\circ s_z\circ s_z(y)
                    =(s_z\circ s_x)(y)
\end{equation}
The group generated by all the $s_x\circ s_y$ ($x$, $y\in M$) is the \defe{displacement group}{displacement group} and is denoted by $G(M)$.

\begin{lemma}
The displacement group is a normal subgroup of $\Aut(M)$.
\end{lemma}

\begin{proof}
If $\varphi$ is an automorphism of $M$, we have
\begin{equation}
  \varphi\circ s_x\circ s_y\circ\varphi^{-1}=s_{\varphi(x)}\circ\varphi\circ s_y\circ\varphi^{-1}
                                   =s_{\varphi(x)}\circ s_{\varphi(y)}
\end{equation}
  because $\varphi\circ s_{x}=s_{\varphi(x)}\circ\varphi$.
\end{proof}


We define $\dpt{Q}{M}{G(M)}$ by $Q(x)=s_xs_e$. This is the \defe{quadratic representation}{quadratic representation}\index{representation!quadratic} of $M$. Since $Q(x)Q(y)^{-1}=s_xs_y$, $Q(M)$ generate $G(M)$.

\begin{theorem}		\label{ThoStructSymGH}\label{tho:sym_homo}
	The space $M$ is symmetric for the structure
	\begin{equation}\label{eq:sym_M}
		s_{[x]}[y]=[x\sigma(x)^{-1}\sigma(y)],
	\end{equation}
	while $L_{\sigma}$ is a symmetric space for
	\begin{equation}
		s_x(y)=xy^{-1} x.
	\end{equation}
	The map $\dpt{q}{M}{L}$, $q([x])=x\sigma(x^{-1})$ is a homomorphism from $M$ to $L_{\sigma}$ and $L/L\hsigma$ is isomorphic to $L_{\sigma}$ by $q$.

	Moreover $\dpt{\tau}{L}{\Aut(M)}$ is a homomorphism and the displacement group $G(M)$ is the subgroup of $\tau(L)$ generated by $\tau(L_{\sigma})$.

\end{theorem}

\begin{proof}
\subdem{Symmetric structure on $M$}
  First we prove that $M$ is symmetric. The symmetry \eqref{eq:sym_M} is well defined: if $k,k'\in K$,
  \begin{equation}
  s_{[xk]}[yk']=[xk\sigma(xk)^{-1}\sigma(yk')]
               =[x\sigma(x)^{-1}\sigma(y)].
\end{equation}
 It is clear that $s_{[x]}\circ s_{[x]}=id$ because
\begin{equation}
\begin{split}
  s_{[x]}\circ s_{[x]}[y]&=s_{[x]}([ x\sigma(x)^{-1}\sigma(y) ])
                         =[x\sigma(x)^{-1}\sigma(x)\sigma(\sigma(x)^{-1})(\sigma\circ\sigma)(y)]\\
			 &=[xx^{-1} y]
			 =[y].
\end{split}
\end{equation}

Since $L$ acts transitively by automorphism on $M$ (this is: $\tau(x)$ is an automorphism of $M$ and we can always find a $x\in L$ such that $\tau(x)[y]=[z]$ for given $y$, $z\in M$), we just have to prove the property of isolated fixed point for $[e]\in M=L/K$. So we consider $s_{\mfo}$ ($\mfo=[e]$) on a neighbourhood of $\mfo$ in $M$.
\subdem{Identification $T_{\mfo}M=\lL_-$}

Now we show how to identify (as vector spaces) $T_{\mfo}M$ with
\[
   \lL_-=\{X\in\lL\tq \sigma(X)=-X\}.
\]
where $\lL$ is the Lie algebra of $L$. For this, we will show that $\dpt{\psi}{\lL_-}{T_{\mfo}M}$,
\begin{equation}
  \psi(X)=\Dsdd{[X(t)]}{t}{0}
\end{equation}
if $X(t)$ is a path in $L$ whose derivative is $X$. Any vector in $T_{\mfo}M$ comes from a path $[Y(t)]$ where $Y(t)\in L$ can be written as $Y(t)=c(t)k(t)$ where $k(t)\in K$ has no continuity property, and $c$ is the ``main``\ part of the path. Then
\[
  \psi(c'(0))=\Dsdd{ [Y(t)] }{t}{0}
\]
and $\psi$ is surjective. In order to see the injectivity, remark that in a neighbourhood of $e$,
\[
   \sigma(e^{tX})=e^{td\sigma X}=e^{-tX}
\]
because $X\in\lL_-$. With other words, if $X\in\lL_-$,

 \[
    \sigma(X(t))=X(t)^{-1}
 \]
when $t$ is small. But $e$ is an isolated fixed point of the inversion. Then $\psi(X)=0$ let only one possibility: $[X(t)]=cst$. Thus (for small $t$) $X(t)$ can be written as $X(t)=gk(t)$ with $\sigma(k(t))=k(t)\in L$. Since $X(0)=e$, $k(0)=g^{-1}$ and $\sigma(g)=k(0)^{-1}=g$. Then
 \[
   \sigma(X(t))=X(t).
 \]
But on the other hand, $X\in\lL_-$ implies $\sigma(X(t))=X(t)^{-1}$ and finally $X(t)=X(t)^{-1}$, so that $X(t)=e$. See eventually the error \ref{err:decomp}\label{pg:X_t}.

Now we can see that $0$ is an isolated fixed point of $s_{\mfo}$. We looks at $(ds_{\mfo})_{\mfo}X$ with $X\in\lL_-\equiv T_{\mfo}M$.
\begin{equation}
\begin{split}
  (ds_{\mfo})X&=(ds_{\mfo}\circ\psi)(X)
		=ds_{\mfo}\Dsdd{[X(t)]}{t}{0}
                =\Dsdd{ [\sigma(X(t))] }{t}{0}\\
		&=d\sigma\Dsdd{[X(t)]}{t}{0}
		=\sigma(X)
		=-X.
\end{split}
\end{equation}

With the notation $X^*=\sigma(x)^{-1}$,
\begin{equation}
\begin{split}
  q\left( s_{[x]}[y]  \right)&=q( [xx^*\sigma(y)] )\\
                             &=xx^*\sigma(y)\sigma^2(y)^{-1}\sigma(x^*)^{-1}\sigma(x)^{-1}\\
			     &=(xx^*)(\sigma(y)y^{-1})(xx^*)\\
			     &=q(x)q(y)^{-1} q(y),
\end{split}
\end{equation}
then
\[
   q(s_{[x]}[y])=s_{q(x)}q(y)
\]
and $q$ is a homomorphism between $M$ and $L$ for they respective symmetric spaces structure. It is contained in the definition of $q$ that
\[
   q(M)=L_{\sigma}=\{x\sigma(x)^{-1}\tq x\in L\}
\]

On the other hand, $q([x])=q([y])$ if and only if $xx^*=yy^*$ which is equivalent to  $x^{-1} y\in L\hsigma$. But $x^{-1} y\in L\hsigma$ implies $\ovx=\ovy$ where the bar stands for the classes with respect to $L_{\sigma}$. In definitive, $q([x])=q([y])$ if and only if $\ovx=\ovy$. Hence, $q$ is an isomorphism of symmetric spaces between $L_{\sigma}$ and $L/L\hsigma$.

We recall the definition $\dpt{\tau(x)}{M}{M}$, $\tau(x)[y]=[xy]$, and we use the quadratic representation of $M$:
\begin{equation}
 Q([x])[y]=s_{[x]}s_{\mfo}[y]
          =s_{[x]}([\sigma(y)])
	  =\tau(xx^*)[y].
\end{equation}
Then $G(M)$ is generated by $Q(M)=\tau(q(M))=\tau(L_{\sigma})$.
\end{proof}

\subsection{Choice of a Cartan involution}
%----------------------------------------

Let $\lG=\lK\oplus\lP$ be the Cartan decomposition of $\lG$ and $B$, the Killing form on $\lG$. We know that the linear transformation of $\lG$ defined by
 
\[ 
 \theta(X)=\begin{cases}
             X & \text{if $X\in\lK$}\\
	     -X& \text{if $X\in\lP$}
           \end{cases}
\] 
 is an involutive automorphism of $\lG$; and the bilinear form
\[
  (X,Y)\to\scal{X}{Y}:=-B(X,\theta Y)
\]
is positive definite on $\lG$.

\begin{theorem}
Let $\lG$ be a real semisimple Lie algebra, $\sigma$ an involutive automorphism and $\theta$ a Cartan involution.\index{Cartan!involution}\index{involutive!automorphism}\index{involution!Cartan} Then

\begin{enumerate}
\item there exists a Cartan involution $\theta_1$ such that $[\sigma,\theta_1]=0$,
\item if $\theta_1$ and $\theta_2$ are two such involutions then they are conjugated by an automorphism of $\lG$ of the form $e^{\ad X}$ with $\sigma(X)=X$.
\end{enumerate}
\label{tho:sigma_theta}
\end{theorem}

%L'ancienne version de la première preuve est donnée en commentaire plus bas, si ça t'intéresse.
The first point is contained in theorem \ref{tho:sigma_theta_un} and the second one is exactly the corollary \ref{cor:Cartan_conj_inner}



% \subdem{First item}
% We consider $\nu=\sigma\theta$, this fulfils $\scal{\nu X}{Y}=\scal{X}{\nu Y}$ because $\sigma$ and $\theta$ are automorphism of $\lG$ and the Killing form is invariant under the automorphism (cf. proposition \ref{prop:auto_2}). Then the matrix of $\nu^2$ as linear operator on $\lG$ is positive definite and can be diagonalised with positive eigenvalues. Then there exists an unique symmetric linear transformation $A$ of $\lG$ such that $\nu^2=e^A$.
% 
% Now we prove that for any $t\in\eR$, $e^{tA}$ is an automorphism of $\lG$. Consider $\{X_1,\ldots,X_n\}$ an orthonormal (with respect to $\scal{.}{.}$) basis of $\lG$ in which $\nu$ is diagonal:
% \begin{subequations}
% \begin{align}
%   \nu(X_i)&=\lambda_iX_i\\
%   \nu^2(X_i)&=e^{a_i}X_i,
% \end{align}  
% \end{subequations}
% (no sum at all) where the $a_i$ are the diagonals elements of $A$. The structure constants are as usual defined by
% \begin{equation}
%    [X_i,X_j]=c_{ij}^kX_k.  
% \end{equation}
% Since $\sigma$ and $\theta$ are automorphisms, $\nu^2$ is also one. Then 
% \[
% \nu^2[X_i,X_j]=c_{ij}^k\nu^2(X_k)=c_{ij}^ke^{a_k}X_k
% \]
% can also be computed as
% \[
%    \nu^2[X_i,X_j]=[\nu^2X_i,\nu^2X_j]=e^{a_i}e^{a_j}c_{ij}^kX_k,
% \]
% so that $c_{ij}^ke^{a_k}=c_{ij}^ke^{a_i}e^{a_j}$, and then $\forall t\in\eR$,
% \[
%    c_{ij}^ke^{ta_k}=c_{ij}^ke^{ta_i}e^{ta_j},
% \]
% which prove that $e^{tA}$ is an automorphism of $\lG$. Consequently, $A$ is a derivation of $\lG$ from lemma \ref{lem:autom_derr}.
% 
% By the way remark that $\nu=e^{A/2}$ and $[e^{tA},\nu]=0$ because this can be view as a common matricial commutator. Other thinks to be remarked are $\nu^{-1}=\theta\sigma$, $\theta\nu^{-1}\theta=\theta\sigma\theta\sigma$ and finally, 
% \[
%   \theta e^{tA}=e^{-tA}\theta.
% \]
% 
% Now we pose $\theta_1=e^{tA}\theta e^{-tA}$. With it and $t=1/4$, 
% $\sigma\theta_1= \sigma\theta e^{-A/2}$ and $\theta_1\sigma=\theta\sigma e^{A/2}$. Then 
% \begin{equation}
%   \sigma\theta_1=\theta_1\sigma=id
% \end{equation}
% when $\theta_1=e^{A/4}\theta e^{-A/4}$ with a suitable $A$.
% \quext{\c{C}a me semble quand m\^eme un peu fort que ce soit carrément l'identité.}

% Let us consider now an other Cartan involution $\theta_2$ such that $\sigma\theta_2= \theta_2\sigma$. Remark that $\theta_2$ is a particular case of a $\sigma$ (an involutive automorphism), then we can apply the first case with $\theta_2$ instead of $\sigma$: we build 
% \[
%   \theta_3= e^{B/4}\theta_1 e^{-B/4}
% \]
% with a suitable $B$ such that $\theta_3\theta_2=\theta_2\theta_3$. Here, 
% $e^{B}=\theta_2\theta_1\theta_2\theta_1$ commute with $\sigma$ because $\theta_1$ and $\theta_2$ does.
% 
% Since $B$ is a derivation, by proposition \ref{prop:ss_derr_int} which make $\ad\lG=\partial\lG$ from the fact that $\lG$ is semisimple, it can be written as $B=\ad X$ for a certain $X\in\lG$. We naturally rescale the $X$ to get $\ad X=B/4$. In order for $e^{tB}$ to commute with $\sigma$, we need $\sigma(X)=X$.
% 
% Now we consider the Cartan decomposition of $\lG$ with respect to $\theta_2$ and $\theta_3$:
% \begin{equation}
% \lG=\lK_2\oplus\lP_2,\qquad \lG=\lK_3\oplus\lP_3.
% \end{equation}
% 
% Naturally, $\lK_2=(\lK_2\cap\lK_3)\cup(\lK_2\cap\lP_3)$. But on $\lK_2$, the Killing form is negative definite, then $\lK_2\cap\lP_3=0$. In the same\quext{Ici, j'ai l'impression qu'on ne fait aps moins que démontrer qu'il n'existe qu'une seule forme de Cartan sur $\lG$} way, $\lK_3\cap\lP_2=0$ and then $\theta_2=\theta_3$, so that 
% \[
%    \theta_2=e^{B/4}\theta_1 e^{-B/4}
% \]
% with $B/4=\ad X$ as we wanted.


\subsection{Affine Symmetric spaces}
%-----------------------------------

The matter may be found in chapter XI of \cite{kobayashi2}

Let $M$ be a $n$-dimensional manifold endowed with a connection $\nabla$. The \defe{symmetry}{symmetry!in an affine space} at $x\in M$, denoted by $s_x$, is defined on a normal neighbourhood of $x$ by $\exp_x X\to\exp_x(-X)$. Properties of the exponential and normal neighbourhood make it a well defined diffeomorphism because it doesn't depends on the choice of the normal neighbourhood.

It clearly fulfils $s_x^2=id$ and $x$ is an isolated fixed point of $s_x$.

\begin{probleme}
je ne vois pas comment d\'emontrer que $s_{s_xy}=s_x\circ s_y\circ s_x$.
\end{probleme}

If we consider the normal coordinates in a neighbourhood around $x$, it is clear that 
\[
s_x(x_1,\ldots,x_n)=(-x_1,\ldots,-x_n).
\]
 Then $(ds_x)_x=-I_x$ where $\dpt{I_x}{T_xM}{T_xM}$ is the identity.

If for all $x\in M$, the map $s_x$ is an affine transformation, we say that $M$ is a locally affine symmetric space\footnote{A differentiable map $\dpt{f}{M}{M'}$ is an \defe{affine}{affine!map} if $\dpt{df}{TM}{TM'}$ transforms all horizontal curves to an horizontal curve. An affine transformation automatically fulfils
\[
  f(\exp X)=\exp(df X)
\]
for all $X\in T_xM$.}.


\begin{lemma}
On an affine locally symmetric space, an odd tensor invariant under $s_x$ is zero at $x$.
\end{lemma}

\begin{proof}
From $(ds_x)_x=-I_x$, the transformation $s_x$ transforms a tensor $K$ of degree $p$ into $(-1)^p K$.
\end{proof}


\begin{theorem}
Let $M$ and $M'$ be two manifolds with $\nabla T=\nabla R=\nabla T'=\nabla R'=0$ and a linear endomorphism $\dpt{F}{T_{x_0}M}{T_{y_0}M}$  such that $FT_{x_0}=T'_{y_0}$ and $FR_{x_0}=R'_{y_0}$.

Then there locally exists an isometry\quext{relis pour voir si c'est bien \c ca.} $\dpt{f}{M}{M'}$ such that $f(x_0)=y_0$ and $(df)_{x_0}=F$.
\end{theorem}

\begin{proposition}
A manifold $M$ with an affine connection is affine locally symmetric if and only if
\[
  T=0\text{ and }\nabla R=0.
\]
\end{proposition}


\begin{proof}
Since $s_x$ is affine, it preserves $T$ and $\nabla R$ whose are tensor of degree $3$ and $5$. From lemma, they are zero.

For the converse, $-I_x$ preserves $R_x$ because $R$ is a tensor of degree $4$. In this context, the theorem gives a $\dpt{f}{M}{M}$ such that $f(x)=x$ and $df_x=-I_x$. But $f$ is also an affine transformation, then
\begin{equation}
  f(\exp X)=\exp(df X)
           =\exp(-I_x X)
           =\exp(-X).
\end{equation}
This gives $f=s_x$.
\end{proof}

Two results without proof.

\begin{theorem}
If $M$ is a differentiable manifold with a linear connection such that $\nabla T=0$ and $\nabla R=0$, then the atlas of normal coordinates gives to $M$ a structure of analytic manifold and the connection is analytic.
\end{theorem}
It comes from the page 223 of \cite{kobayashi}.

\begin{proposition}
Let $M$ be a connected, simply connected and complete manifold with a linear connection such that $\nabla T=0=\nabla R$. Let $\dpt{F}{T_xM}{T_yM}$ a linear isomorphism such that $T_x\to T_y$ and $R_x\to R_y$.

Then there exist one and only one affine transformation $f$ of $M$ such that $f(x)=y$ and $df_x=F$.
\end{proposition}
In particular the group $\mA(M)$ of the affine transformations of $M$ is transitive on $M$. This comes from page 265 of \cite{kobayashi}.

A manifold $M$ with an affine connection is an \defe{affine symmetric space}{affine!symmetric space} if for all $x\in M$, the symmetry $s_x$ can be globally extended to an affine transformation of $M$. Thanks to the latter proposition, an affine locally symmetric complete and simply connected space is affine.

 
\begin{proposition}
An affine symmetric space is complete.
\end{proposition}

\begin{proof}
Let $\gamma$ be a geodesic from $x$ to $y$, i.e. $\gamma(0)=x$ and $\gamma(1)=y$. Let us pose $\gamma(1+t)=s_y(\gamma(1-t-)$ for $0\leq t\leq a$. It extends $\gamma$ beyond $y$. Let us prove that the extension still is a geodesic. For a certain $Y\in T_xM$, we have $y=\exp_xY$, so for $t$ between $0$ and $1$, $\gamma(t)=\exp_x(tY)$. Let $Y_t$ be the parallel vector field along $\gamma$ with $Y_0=Y$; for example, $x=\exp_y(-Y_1)$ and $\exp_x(tY)=\exp_y(t-1)Y_1$.
\begin{equation}
   \gamma(1+t)=s_y(\exp_y(-tY_1))
              =\exp_y(tY_1).
\end{equation}
\end{proof}

\begin{proposition}
The group of affine transformations of an affine symmetric space is transitive.
\end{proposition}

\begin{proof}
Let $x$ and $y$ be two points in $M$. There exists a sequence of convex normal neighbourhood $\mU_1,\ldots,\mU_k$ such that $x\in\mU_1$, $y\in\mU_k$, $\mU_i\cap\mU_{i+1}\neq\emptyset$. So one can reach $y$ from $x$ with geodesic segments. This construction is just the fact that, if $\dpt{c}{[a,b]}{M}$, is a path from $x$ to $y$, the set $c([a,b])$ is compact in $M$. On each point of $c([a,b])$ we consider a convex neighbourhood which gives an open covering of a compact set.

It remains to be proved that if $x$ and $y$ are reachable by a geodesic curve, then they can be reached by an affine transformation. If $y=\exp_x Y$ and $z=\exp_x(\frac{1}{2} Y)$, then  $s_z(x)=y$.

\end{proof}

One can prove that the group $\mA(M)$ is a Lie group. We denote by $G$ its identity component. It is clear that is a group acts transitively on a manifold, then its identity component also acts transitively. Then $G$ acts transitively on $M$ and one has a homogeneous space structure which allows us to write $M=G/H$.

\begin{probleme}
	Une remarque de Stéphane. 
	L’espace symetrique $M$ est vu comme $G/H$ ou $G =$ composante connexe à l’identite de $Aff(M) =$ larger connex group of affine transformation (théorème \ref{ThoGplugdSymssgpAff}), ou $Aff(M)=$groupe des transformations affines de $M$.

A-t-on forcement : composante connexe à l’identite de $Aff(M)  =$ larger connex group of $Aff(M)$ ?
(genre, pourrait-on avoir un ss-groupe connexe qui ne passe pas par l’identité, plus grand que tous ceux qui passent par l’identité ?) 

On sait que toutes les symetries $s_x$ sont des transformations affines, mais le contraire n’est pas vrai ; en particulier, $Aff(M)$  peut contenir des elements qui ne sont pas des symetries.  Ceci pour savoir la definition du theoreme  \ref{ThoGplugdSymssgpAff}, etait la meme que celle du theoreme 2.4 du papier de Pierre ci-joint, où il exprime $M=G/H$, avec cette fois $G =$ transvection group de $M$, qui est INCLUS dans $Aff(M)$. 

Dans ton theoreme \ref{ThoGplugdSymssgpAff}, $G$ be the LARGER connex group of affine transformation, tandis que chez Pierre Th. 2.4, (i), le transvection group $G$ is the SMALLEST subgroup of Aff(M) which is transitive etc.

\end{probleme}

More precisely, we have the

\begin{theorem}		\label{ThoGplugdSymssgpAff}
Let $G$ be the largest connected group of affine transformation of an affine symmetric space $M$ and $H$, the isotropy group of a fixed point $o\in M$, so that $M=G/H$.

Let $s_o$ the symmetry of $M$ at $o$ and $\sigma$ the automorphism of $G$ defined by
\[
   \sigma(g)=s_o\circ g\circ s_o^{-1}.
\]
Let $G_{\sigma}$ the closed subgroup of $G$ which fixes $\sigma$. Then $G^o_{\sigma}\subset H\subset G_{\sigma}$.
\end{theorem}

\begin{proof}
Let $h\in H$ and $\sigma(h)=s_o\circ h\circ s_o^{-1}$. We know that $(ds_o)_o=-I_o$, then $(d\sigma(h))_o=dh_o$. But general theory about affine transformations says that if two affine transformations has same differential at one point then they are equals. In our case, it gives $\sigma(h)=h$; therefore $H\subset G_{\sigma}$.

Let now $g_t$ be a one parameter subgroup of $G_{\sigma}$. From the definition of $\sigma$, $s_o\circ g_t=\sigma(g_t)\circ s_o$, then $s_o\circ g_t(o)=g_t\circ s_o(o)=g_t(o)$. Then the orbit $g_t(o)$ is fixed by $s_o$. But $o$ is an isolated fixed point of $s_o$, then $g_t(o)=o$ for all $t$ and $g_y\in H$.

From general theory of Lie groups, a connected Lie group is generated by its one parameter subgroups. Then $G_{\sigma}^o$ is generated by elements which fix $\sigma$. So $G_{\sigma}^o\subset H$.
\end{proof}

\subsection{Symmetric pair}
%--------------------------

Let $G$ be a connected Lie group and $H$, a closed subgroup.
\begin{definition}
We say that $(G,H)$ is a \defe{symmetric pair}{symmetric!pair} if there exists an analytic involutive automorphism\index{involutive!automorphism} $\dpt{\sigma}{\lG}{\lG}$ such that $(H_{\sigma})\subset H\subset H_{\sigma}$
where $H_{\sigma}$ is the set of fixed points by $\sigma$. If the group $\Ad_G(H)$ is compact, the pair is \defe{Riemannian}{Riemannian symmetric pair}.
\end{definition}
Note: by $\Ad_G(H)$ we mean the Lie subgroup of $\Ad_G(G)$ which is the image of $H$ by $\Ad_G$.


\begin{proposition}
Let $(G,K)$ be a Riemannian symmetric pair and $\lK$ the Lie algebra of $K$. We denote by $\mZ$ the center of the Lie algebra $\lG$. If $\lK\cap\mZ=\{0\}$, then there exists one and only one analytic involutive automorphism $\sigma$ of $G$ such that $(K_{\sigma})_0\subset K\subset K_{\sigma}$.
\end{proposition}

\begin{proof}
The point is the unicity: the existence is contained in the definition of a symmetric pair. Let us consider two such automorphism $\sigma_1$ and $\sigma_2$. As far as the Lie algebras are concerned, the identity component only is relevant. Since $(K_{\sigma_1})_0=(K_{\sigma_1})_0$; thus $\lK_1=\lK_2$. We consider the respective decompositions of $\lG$ for $\sigma_1$ and $\sigma_2$:
\begin{subequations}
\begin{align}
  \lG&=\lK\oplus\lP_1\\
  \lG&=\lK\oplus\lP_2.
\end{align}
\end{subequations}
where $\lP_i$ is the eigenspace with eigenvalue $-1$ for the automorphism $d\sigma_i$ of $\lG$. Since the Killing form $B$ of $\lG$ is invariant under $\sigma_i$, $\lK$ is $B$-orthogonal to $\lP_i$. Indeed $B(k,p)=B(d\sigma_i k,d\sigma_ip)=-B(k,p)$; then $B(k,p)=0$. Consider $X_1\in\lP_1$ and $T\in\lK$. We have a $X_2\in\lP_2$ such that $X_1=T+X_2$. Since $\lP_i\perp\lK$,
\[
  0=B(k,X_1)=B(k,T)+B(k,X_2),
\]
then $B(k,T)=0$ and $T\perp\lK$. In particular, $B(T,T)=0$. From proposition \ref{prop:K_Z_Killing}, $B$ is strictly negative definite on $\lK$; then $T=0$ so that $\lP_1=\lP_2$ and $\sigma_1=\sigma_2$.

Now we have to see that the $\lK$ here is actually the $\lK$ of the proposition \ref{prop:K_Z_Killing} in order to see that it is applicable. Since the pair is Riemannian, $\Ad(K)$ --which is the analytic Lie subgroup of $\Int(\lG)$ image of $K$ by $\Ad$-- is compact. The Lie algebra of $\Ad(K)$ is given by thinks of the form
\begin{equation}
  \Dsdd{ \Ad(k(t)) }{t}{0}=d\Ad_e(k'(0))
                          =\ad k'(0) 
\end{equation}
then the Lie algebra of $\Ad(\lK)$ is $\ad(\lK)$. Thus the fact that $\Ad(K)$ is compact is equivalent than the fact that $\lK$ is compactly embedded in $\lG$.

\end{proof}

We can build a symmetric pair from an involutive automorphism\index{involutive!automorphism} $\sigma$ of $G$. Take $H=(G_{\sigma})_0$ and the pair $(G,H)$; it is clear that it is a symmetric pair. However it is not automatically a Riemannian one.

\begin{proposition}
Consider a Lie algebra $\lG$ and a direct decomposition $\lG=\lH\oplus\lM$. Then the map $\sigma=id_{\lH}\oplus(-id)_{\lM}$ is an automorphism of $\lG$ if and only if
\begin{subequations}
\begin{align}
  [\lM,\lM]&\subset\lH\\
  [\lH,\lM]&\subset\lM
\end{align}   
\end{subequations}
\label{prop:invol_ssi_comm}
\end{proposition}

\begin{proof}
We just have to compute $\sigma[h+m,h'+m']$ and $[\sigma(h+m),\sigma(h'+m')]$ and see under which conditions it is equal.
\end{proof}

\subsubsection{Example: Lie group}
%-------------------------------

Let $L$ be a Lie group endowed with the structure\index{symmetric!space!Lie group}
\begin{equation}
  s_xy=xy^{-1} x.
\end{equation}
It is immediate to check that $\forall x,y\in L$,  $(s_x\circ s_x)(y)=y$ and $s_x\circ s_y\circ s_x=s_{s_x(y)}$. In order to see that $x$ is an isolated fixed point of $s$, first remark that 
\[
    x(s_y(z))=(xy)(xz)^{-1}(xy)=s_{xy}(xz),
\]
so that one just needs to check the property on $s_e$ because the left translation is analytic. Since $s_e(y)=y^{-1}$, the property follows from the fact that $e$ is an isolated fixed point for the inversion in a topological group.

\subsubsection{Example: homogeneous spaces}\index{homogeneous!space}
%---------------------------------------

Let $L$ be a connected Lie group with an involutive automorphism\index{involution!automorphism}, and $L\hsigma$ the set of fixed points by $\sigma$. We consider a subgroup $K$ such that $L_0\hsigma\subset K\subset L\hsigma$. The space $L\hsigma$ is closed because it is defined by some equalities. The theorem \ref{tho:H_ferme} assure us that as topological Lie subgroup of $L$, $K$ is also closed.
 
Now we consider $M=L/K$ and for $x\in L$, we define the translations $\dpt{\tau(x)}{M}{M}$, $\tau(x)[y]=[xy]$ where the classes are defined with respect to $K$: $[x]=[xk]$ for any $k\in K$. We also define
\begin{equation}
   L_{\sigma}=\{ x\sigma(x)^{-1}:x\in L \};
\end{equation}
this is the space of the \defe{symmetric elements}{symmetric!elements} of $L$.



\subsection{Symmetric spaces as quotient}
%----------------------------------------

We saw in the previous subsection that an affine symmetric space gives rise to a homogeneous space $G/H$ and an involutive automorphism $\sigma$ of $G$. From now we define a \defe{symmetric space}{symmetric!space} as a triple $(G,H,\sigma)$ where

\begin{itemize}
\item $G$ is a Lie group,
\item $H$ is a closed subgroup of $G$,
\item $\sigma$ is an involutive automorphism of $G$ such that $G_{\sigma}^0\subset H\subset G_{\sigma}$
\end{itemize}
where $G_{\sigma}=\{g\in G\tq \sigma(g)=g\}$. The space is \defe{effective}{effective!symmetric space} if the largest normal subgroup $N$ of $G$ contained in $H$ reduces to the identity. As $N$ is normal in $G$ and contained in $H$, the quotients $G/N$ and $H/N$ admits a canonical group structure. 

Here, we will suppose that $G$ is connected, but it is not an important issue.

\begin{proposition}
If $\sigma'$ is the involutive automorphism on $G/N$ induced from $\sigma$ and $(G,H,\sigma)$ is a symmetric space, then $(G/N,H/N,\sigma')$ is an effective symmetric space.
\end{proposition}

Remark that $\sigma'$ is well defined because, from definition, $\sigma'([g])=[\sigma(g)]$, then for $h\in H$
\begin{equation}
\sigma'[gh]=[\sigma(g)\sigma(h)]
           =[\sigma(g)h]
           =[\sigma(g)]
\end{equation}
because $H\subset G_{\sigma}$ implies $\sigma(h)=h$.

\begin{proof}
Let $S$ be a normal subgroup of $G/N$ contained in $G/H$. From the definitions, $S=\{id\}$. Indeed $S\subset H/N$; let $[a]\in S$ and $[g]\in G/N$. The first point is that $[gag^{-1}]\in S$. On the other hand if $[a]\in S$, then $a\in H$ because $S\subset H/N$ and $N\subset H$. Then for all representative $g$ of $[g]$, $gag^{-1}\in H$. In particular for all $g\in G$, $gag^{-1}\in H$ and $a\in H$. From this, the thesis is immediate.
\end{proof}

Following definition \ref{def:esp_sym} of a symmetric space, we should define good symmetries on $G/H$ from the data of $(G,H,\sigma)$. At $\mfo\in G/H$, we define $s_{\mfo}=\sigma'$. Let $g\cdot\mfo=[g]$ be a fixed point of $s_{\mfo}$ for a certain $g\in G$. Hence $\sigma'([g])=[g]$, but from the definition of $\sigma'$, we also have $\sigma'(g\cdot \mfo)=[\sigma(g)]$. Then $\sigma(g)\in[g]$. Let $h=g^{-1}\sigma(g)\in H$. Since $\sigma(h)=h$, $h^2=h\sigma(h)$, but $\sigma(h)=\sigma(g^{-1}\sigma(g))$, then $h^2=e$.
Since $\sigma$ is an automorphism, if $g$ is near the identity, then $h$ will be too and $h^2=e$. So $g$ is near the identity and invariant under $\sigma$, then $g\in G_{\sigma}^0\subset H$ and $g\cdot\mfo=\mfo$.

Let $x=g\cdot\mfo$. We set
\[
   s_x=g\circ s_{\mfo}\circ g^{-1}.
\]
As a first remark, the choice of $g\in G$ such that $x=g\cdot\mfo$. Indeed consider a $k\in G$ such that $gk\cdot\mfo=x=g\cdot\mfo$; we must show that $s_x=gks_{\mfo}k^{-1} g^{-1}$. Since $k\in H$, it is sufficient to prove that for all $h\in H$, $h\circ s_{\mfo}=s_{\mfo}$. For this, let $[g]\in G/H$.
\begin{equation}
s_{\mfo}[h^{-1} g]=[\sigma(h^{-1})\sigma(g)]
    =[h^{-1} \sigma(g)]
    =h^{-1} s_{\mfo}[g],
\end{equation}
so that $(hs_{\mfo}h^{-1})=s_{\mfo}[g]$.

The \defe{transvection group}{transvection group} is the subgroup of $\Aut(M,\omega,s)$ spanned by 
\[ 
  \{ s_x\circ s_y\tq x,y\in M \}.
\]

The definition of $s_x$ also fulfils $s_x\circ s_y\circ s_x^{-1}=s_{  s_x(y)  }$. In order to see it, let us consider $x=g\cdot\mfo$ and $y=k\cdot\mfo$.
\begin{equation}
s_x\circ s_y\circ s_x^{-1}=g s_{\mfo}g^{-1} ks_{\mfo}k^{-1} gs_{\mfo}^{-1} g^{-1}
                 =s_{ (gs_{\mfo}g^{-1} k)\cdot\mfo }
                 =s_{s_xy}y.
\end{equation}
  

\subsection{Symmetric Lie algebras}
%-----------------------------------

A \defe{symmetric Lie algebra}{symmetric!Lie algebra} is a triple $(\lG,\lH,\sigma)$ with

\begin{itemize}
\item $\lG$: a Lie algebra,
\item $\lH$: a Lie subalgebra of $\lG$,
\item $\sigma$: an involutive automorphism of $\lG$ whose $\lH$ is the set of fixed points.
\end{itemize}

\begin{proposition}
Every symmetric space $(G,H,\sigma)$ gives rise to a symmetric Lie algebra $(\lG,\lH,\sigma')$ with $\lG$ and $\lH$ being the Lie algebras of $G$ and $H$ while $\sigma'=d\sigma_e$.
\end{proposition}

\begin{proof}
If $X\in \lH$, one has $d\sigma_eX=\Dsdd{\sigma e^{tX}}{t}{0}$. Since $\lH$ is the Lie algebra of a Lie subgroup of $G$, equation \eqref{eq:path_alg} makes
\[
  \lH=\{X\in\lG\tq t\to e^{tX}\,\text{is a path in $H$}\}.
\]
Then $e^{tX}\in H$ for all $t$ and $\sigma(e^{tX})=e^{tX}\in H$ because $H\subset G_{\sigma}$. This proves that the elements of $\lH$ are fixed by $\sigma'$.

Let us see the converse. If $X$ is fixed by $\sigma'$
\[
  d\sigma_eX=\Dsdd{\sigma(e^{tX})}{t}{0}\stackrel{!}{=}X=\Dsdd{e^{tX}}{t}{0}.
\]
This equation shows that $e^{tX}$ and $\sigma e^{tX}=e^{t\sigma' X}$ are two exponential path whose start at the same point with the same tangent vector. They are equals on a neighbourhood of $e$. In this case, $\sigma(e^{tX})=e^{tX}$ and $e^{tX}\in H$. This gives $X\in \lH$.

\end{proof}

The association of a symmetric space to a symmetric Lie algebra is less automatic. Let $(\lG,\lH,\sigma')$ be a symmetric Lie algebra. We first have to find a connected, simply connected Lie group $G$ whose Lie algebra is $\lG$. From this we define $\dpt{\sigma}{G}{G}$ by
\[
  \sigma(e^X)=e^{\sigma'X}.
\]
This is a local definition. Under analyticity hypothesis, one can extend $\sigma$ into the whole $G$. Now one can take any subgroup $H$ of $G$ such that $G_{\sigma}^0\subset H\subset G_{\sigma}$ to complete the symmetric space $(G,H,\sigma)$.

\begin{probleme}
Il est dit que $H$ est ferm\'e parce qu'il est inclu \`a $G_{\sigma}$ qui l'est.
\end{probleme}

\subsubsection{Symmetric and reductive Lie algebras}
%////////////////////////////////////////////////////

Let $(\lG,\lH,\sigma)$ be a symmetric Lie algebra. As linear transformation of the vector space, $\sigma$ has eigenvalues $1$ and $-1$ (because it is involutive) and then induces a decomposition
\begin{equation}
\lG=\lH\oplus\lM
\end{equation}
where $\lH$ is the $+1$ eigenspace and $\lM$ the $-1$ eigenspace. This is the \defe{canonical decomposition}{canonic!decomposition!of a symmetric Lie algebra}. It is easy to see that this decomposition fulfils
\begin{equation}  \label{eq:propreduc}
[\lH,\lH]\subset\lH,\quad[\lH,\lM]\subset\lM,\quad[\lM,\lM]\subset\lH.
\end{equation}
On the one hand, if it exists, the homogeneous space $G/H$ is automatically reductive. On the other hand if we have a Lie algebra $\lG$ and a decomposition $\lG=\lH\oplus\lM$ which fulfils \eqref{eq:propreduc}, then definition $\sigma=\id_{\lH}\oplus(-\id)_{\lM}$ gives a symmetric Lie algebra $(\lG,\lH,\sigma)$.

A homogeneous space is symmetric if and only if it is reductive.

\begin{proposition}
Let $(G,H,\sigma)$ a symmetric space, $(\lG,\lH,\sigma')$ the corresponding symmetric Lie algebra and $\lG=\lH\oplus\lM$ its canonical decomposition. Then
\[
   \Ad(H)\lM\subset\lM.
\]
\end{proposition}

\begin{proof}
If $X\in\lM$ and $h=e^Y\in H$, then
\begin{align*}
\sigma'(\Ad(e^Y)X)=\Ad(e^{\sigma'Y})(\sigma'X)
                  =\Ad(\sigma h)(\sigma'X)
                  =\Ad(h)(-X)
                  =-\Ad(h)X
\end{align*}
because $\sigma h=h$ and $\sigma'X=-X$. So $\Ad(h)X\in\lM$ because it has eigenvalue $-1$ for $\sigma$.
\end{proof}

\subsubsection{An affine example}
%////////////////////////////////

Let $M$ be an affine locally symmetric $n$-dimensional space. We consider $x\in M$, $\lM=T_xM$ and the curvature tensor $R_x$. Let $\lH$ be the set of linear endomorphism $\dpt{U}{T_xM}{T_xM}$ which sends $R_x$ on zero. More precisely an endomorphism of $T_xM$ extends to a derivation of the tensor algebra with the definition
\begin{equation} \label{eq:defHa}
(U\cdot R_x)(X,Y)=U\big( R_x(X,Y) \big)-R_x(UX,Y)-R_x(X,UY)-R_x(X,Y)\circ U.
\end{equation}
In order to understand the last term, let us recall ourself that the curvature is given, from the connection $\dpt{\nabla}{\cvec(M)\times\cvec(M)}{\cvec(M)}$ by formula
\[
  R(X,U)Z=\nabla_X\nabla_Y Z-\nabla_Y\nabla_XZ-\nabla_{[X,Y]}Z.
\]
So one can see pointwise $\dpt{R_x}{T_xM\times T_xM}{\End(T_xM)}$. Now we define $\lH$ by the condition $(U\cdot R_x)(X,Y)=0$ for all $X$, $Y\in\lM$. One can prove that $\lH$ is a Lie algebra for the usual bracket.

Remark that for all $X$, $Y\in\lM$, the endomorphism $R_x(X,Y)$ belongs to $\lH$ because $R(X,Y)=[\nabla_X,\nabla_Y]-\nabla_{[X,Y]}$ and $\nabla R=0$. We consider the direct sum $\lG=\lM\oplus\lH$ on which we put a Lie algebra structure by defining
\begin{subequations}
\begin{align}
[X,Y]&=-R(X,Y)&X,Y\in\lM\\
[U,X]&=UX&U\in\lH,X\in\lM\\
[U,V]&=[U,V]&U,V\in\lH.
\end{align}
\end{subequations}
We have to check the Jacobi identities. The first case is $X,Y$, $Z\in\lM$. It gives $[X,Y]=-R(X,Y)\in\lH$, then $[[X,Y],Z]=-R(X,Y)Z$ and the cyclic sum is zero from Bianchi. If $X$, $Y\in\lM$ and $U\in\lH$, then
\begin{subequations}
\begin{align}
[[X,Y],U]&=-[R(X,Y),U]=-R(X,Y)\circ U+U\circ R(X,Y)\\
[[Y,U],X]&=[-UY,X]=R(UY,X)\\
[[U,X]Y]&=[UX,Y]=R(UX,Y).
\end{align}
\end{subequations}
From definition \eqref{eq:defHa}, the sum of these three terms is zero. The last case, $U$, $V\in\lH$, $X\in\lM$, is easy.

With all that, the algebra $\lG=\lM\oplus\lH$ becomes a Lie algebra satisfying \eqref{eq:propreduc}. Then it gives rise to a symmetric Lie algebra with $\sigma=\id_{\lM}\oplus(-\id)_{\lH}$


\subsection{Connection on symmetric spaces} \index{connection!on symmetric spaces}
%------------------------------------------

We use theory from \cite{Loos}. First, we extend the notion of tangent bundle. Consider a smooth curve $\dpt{c}{\eR}{M}$. Its \defe{acceleration}{acceleration} $\ddot c(0)$ at the point $c(0)$ is defined by its action on a function $\dpt{f}{M}{\eR}$:
\begin{equation}
  \ddot c(0)j=\frac{d^2}{dt^2}\big( f(c(t))\big)
\end{equation}
with usual abuse of notation. The set of such accelerations at $x$ is denoted by $T^2_xM$, and we naturally define the bundle $T^2M$ with a suitable manifold structure. The set of sections of $T^2M$ is logically denoted by $\cvec^2(M)$. If $X$, $Y\in\cvec(M)$, we define the \defe{symmetric product}{symmetric!product} by
\begin{equation}
X\ltimes Y=\frac{1}{2}(X\otimes Y+Y\otimes X),
\end{equation}
and the ``composition product''
\begin{equation}
  X\bullet Y=\pr_{T^2M} XY,
\end{equation}
or in local coordinates
\[ 
  (X\bullet Y)_x f= X^i(x)Y^j(x)\left.\frac{\partial^2f}{\partial x_i\partial x_j}\right|_x.
\]
We finally define, for $X$, $Y\in\cvec(M)$
\begin{equation}
\left\{
\begin{aligned}
  P_2(X,Y)&=\frac{1}{2}(X\otimes Y+Y\otimes X)\\
  P_2(X)&=0.
\end{aligned}
\right. 
\end{equation}

\begin{lemma}
For each connection on $TM$, there exists one and only one connection form $\dpt{\Gamma}{\cvec(M)\times\cvec(M)}{T^2M}$ such that
\begin{equation} \label{eq:PdGamlt}
  P_2(\Gamma(X,Y))+X\ltimes Y=0.
\end{equation}
The correspondence is given by
\begin{equation}
  \nabla_XY=XY+\Gamma(X,Y)
\end{equation}


\begin{proof}[Sketch of proof]
Let us just give the link between equation \eqref{eq:PdGamlt} and our general culture about Christoffel symbols\index{Christoffel symbol}. The general form of a $\Gamma(X,Y)\in T^2M$ which is bilinear with respect to $X$ and $T$ is
\[ 
  \Gamma(X,Y)=X^kY^l\Gamma_{kl}^{ij}\partial^2_{ij}+X^kY^l\Gamma_{kl}^i\partial_i.
\]
If we look at the coefficient of $\partial_i\otimes\partial_j$ when we impose the condition
\[ 
  X^kY^l\Gamma_{kl}^{ij}\frac{1}{2}(\partial_i\otimes\partial_j+\partial_j\otimes\partial_i)+\frac{1}{2}(X^iY^j\partial_i\otimes\partial_j+X^iY^j\partial_j\otimes\partial_i)=0,
\]
we find
\[ 
  X^kY^l(\Gamma_{kl}^{ij}+\Gamma_{kl}^{ji})=-X^iY^j.
\]
If we suppose that $\Gamma_{kl}^{ij}$ is symmetric with respect to $ij$, we find $\Gamma_{kl}^{ij}=-\frac{1}{2}\delta_j^i\delta_l^j$, so
\begin{equation}
 \Gamma(X,Y)=-\frac{1}{2}X^iY^j\partial^2_{ij}+X^kY^l\Gamma_{kl}^i\partial_i.
\end{equation}


\end{proof}


\end{lemma}

Let us now consider a manifold $M$ with a product $\mu(x,y)=x\cdot y$ which is $s_xy$ in the case of a symmetric space. It induces a product on $TM$ by the following formula:
\begin{equation}  \label{eq:defcdotXY}
(X\cdot Y)f=(X\otimes Y)(f\circ \mu).
\end{equation}
More explicitly, the function $\dpt{f\circ\mu}{M\times M}{\eR}$ has two entries; the product $X\otimes Y$ apply with $X$ on the first entry and $Y$ on the second one:
\[ 
  (X\cdot Y)_xf=\DDsdd{ (f\circ\mu)(X_x(t),Y_x(s)) }{t}{0}{s}{0}
\]
where $\dpt{X_x,Y_x}{\eR}{M}$ are path defining $X_x$ and $Y_x\in T_xM$. We can extend pointwise this product to a product between vector fields: $(X\cdot Y)_x=X_x\cdot Y_x$ when $X$, $Y\in\cvec(M)$.

An easy adaptation of equation \eqref{eq:defcdotXY} defines $X\cdot x$ when $X\in T_pM$ and $x\in M$:
\begin{equation}
(X\cdot p)f=(u\otimes p)(f\circ\mu)=\Dsdd{ f\big( u(t)\cdot p \big) }{t}{0}
\end{equation}
because the expression $(u\otimes\mu)(f\circ \mu)$ suggests to put $u$ in the first entry of $\mu$ and $p$ in the second one.
It defines a $X\cdot p\in T_pM$. From $v\in T_oM$, we can build $\tilde v\in\cvec(M)$ by
\begin{equation}
  \tilde v_p=\frac{1}{2}v\cdot(o\cdot p),
\end{equation}
explicitly:
\[ 
  \tilde v_pf=\frac{1}{2}\Dsdd{ f\big( v(t)\cdot(o\cdot p) \big) }{t}{0}
\]

\begin{theorem}
  When $X$, $Y\in\cvec(M)$, formula
\begin{equation}
   \Gamma(X,Y)=\frac{1}{2}X\cdot Y
\end{equation}
defines a connection on $M$.

\end{theorem}

\begin{proof}
 Since $XY=X(Y^i\partial_i+X^iY^j\partial_{ij}$, we immediately see that  $P_2(XY)=X\ltimes T$;
it remains to be proved that $-v\tilde u=\frac{1}{2}u\cdot v$ for all $u$, $v\in T_oM$. This is a computation using the definitions:
\begin{equation}
\begin{split}
   (v\tilde u)f=v(\tilde u f)&=\Dsdd{ (\tilde u f)_{v(s)} }{s}{0}\\
		&=\Dsdd{     \frac{1}{2}\Dsdd{ f\big( u(t)\cdot (o\cdot v(s)) \big) }{t}{0}       }{s}{0}\\
		&=\frac{1}{2}\Dsdd{   (df\circ d\mu_{u(t)}\circ \underbrace{d\mu_o}_{=-\mtu})v   }{t}{0}\\
		&=-\frac{1}{2}\Dsdd{ df\circ d\mu_{u(t)}v }{t}{0}\\
		&=-\frac{1}{2} \DDsdd{ f\big( u(t)\cdot v(s) \big) }{t}{0}{s}{0}\\
		&=-\frac{1}{2}(u\cdot v)f.
\end{split}
\end{equation}
\end{proof}

\subsection{Canonical connection and covariant derivative}  \label{subsecCanConCovDer}
%----------------------------------------------------------

Let $G$ be a Lie group, $H$ a closed Lie subgroup and let us consider the principal bundle
\begin{equation}
\xymatrix{%
   H \ar@{~>}[r]		&	G\ar[d]^{\pi}\\
   				&	G/H
 }
\end{equation}
with the action of $H$ on $G$ being defined by $g\cdot h=gh$ and $\pi$ being the canonical projection. We have a canonical identification $T_{[e]}(G/H)=\mG/\mH$. We suppose that $G$ is connected and that $(G,H)$ is a symmetric pair: we have an involutive automorphism $\sigma\colon G\to G$ for which $H$ is the set of fixed points. We suppose moreover that $H$ does not contain non trivial normal subgroups. Let $\mQ$ be the space of vector such that $d\sigma(X)=-X$. By the canonical projection parallel to $\mH$, we have an identification $\mQ=\mG/\mH$.

When $g\in G$, we define
\begin{equation}
\begin{aligned}
 r(g)\colon \mQ&\to T_{[g]}(G/H) \\ 
  X&\mapsto d\pi dL_gX.
\end{aligned}
\end{equation}
We have $r(g)=r(g')$ when there exists a $h\in H$ such that $g'=gh$ and $\rho(h)=\id$ where $\rho$ is defined by
\begin{equation}
\begin{aligned}
 \rho(t)\colon \mQ&\to \mQ \\ 
  X&\mapsto dL_t(X) 
\end{aligned}
\end{equation}
for all $t\in H$. This definition works because of the identification $\mQ=\mG/\mH$.



%+++++++++++++++++++++++++++++++++++++++++++++++++++++++++++++++++++++++++++++++++++++++++++++++++++++++++++++++++++++++++++
\section{Symplectic symmetric spaces}
%+++++++++++++++++++++++++++++++++++++++++++++++++++++++++++++++++++++++++++++++++++++++++++++++++++++++++++++++++++++++++++

\begin{definition}
	A \defe{symplectic symmetric}{symplectic!symmetric space} space is a triple $(M,s,\omega)$ where $(M,s)$ is a symmetric space, $(M,\omega)$ is a symplectic space such that $s_x^*\omega=\omega$ for every $x\in M$.
\end{definition}

\begin{remark}
	We can weaken the symplectic condition in the definition and only ask for $\omega$ to be non degenerate because the condition $s_x^*\omega=\omega$ implies $d\omega=0$.
\end{remark}

%---------------------------------------------------------------------------------------------------------------------------
\subsection{Example}
%---------------------------------------------------------------------------------------------------------------------------

Let $G=\SL(2,\eR)$ and look at the coadjoint action $\Ad^*\colon G\to \GL(\lG^*)$. We consider the element $Z=E-F$ and the orbit
\begin{equation}
	\mO=\Ad^*(G)(Z^{\flat}).
\end{equation}
The space $\lG^*$ has the metric
\begin{equation}
	\langle X^{\flat}, Y^{\flat}\rangle =\beta(X,Y)
\end{equation}
Let us consider $\mfo=Z^{\flat}\in\lG^*$ and consider the stabilizer:
\begin{equation}
	\Stab_{\mfo}(\mO)=\{ g\in G\tq \Ad^*(g)Z^{\flat}=Z^{\flat} \}.
\end{equation}
The Lie algebra is given by
\begin{equation}
	\stab_{\mfo}(\mO)=\{ X\in\lG\tq Z^{\flat}\circ\ad(X)=0 \}.
\end{equation}
The condition of the Lie algebra reads
\begin{equation}
		0=\langle Z^{\flat}, [X,Y]\rangle 
		=\beta(Z,[X,Y])
		=-\beta\big( [X,Z],Y \big)
\end{equation}
for every $Y$, which implies $[X,Z]=0$ because $\beta$ is nondegenerate. Now, in $\gsl(2,\eR)$, the only possibility is that $X$ is proportional to $Z$. Thus the Lie algebra reduces to $\eR Z$ in fact.

%---------------------------------------------------------------------------------------------------------------------------
\subsection{Algebraic setting}
%---------------------------------------------------------------------------------------------------------------------------

We want now to encode the symplectic space structure in an algebraic data. What we are going to discover is the notion of symplectic triple that will be developed in section \ref{SubSecTripleSylple}.

Let $(G,\sigma)$ be an involutive Lie group and $H$ a closed subgroup of $G$ such that
\begin{equation}
	G_0^{\sigma}\subset H\subset G^{\sigma}.
\end{equation}
Let $\pi$ be the projection $\pi\colon G\to M=G/H$. 

The symmetry on the quotient $G/H$ is given by the theorem \ref{ThoStructSymGH}:
\begin{equation}		\label{EaSymGH}
	s_{[g]}[g']=\big[ \sigma(g^{-1}g') \big]
\end{equation}

Now, if we denote by $\sigma$ the differential $d\sigma_e$, we can decompose the Lie algebra $\mG$ into $\mG=\mH\oplus\mP$ and we have the isomorphism (see lemma \ref{LemdpiisomMTM})
\begin{equation}
	d\pi_e|_{\mP}\colon \mP\to T_{\mfo}(M)
\end{equation}
where $\mfo=[e]$. Thus we can see the form $\omega_{\mfo}$ on $\mP$ by
\begin{equation}
	\Omega=\big( d\pi_e|_{\mP} \big)^*\omega_{\mfo}
\end{equation}
and the space $(\mP,\Omega)$ becomes a symplectic vector space.

\begin{lemma}
	We have
	\begin{enumerate}

		\item
			the space $\mK=[\mP,\mP]$ is a Lie subalgebra of $\mH$,

		\item
			the adjoint action of $\mK$ over $\mP$ preserves the symplectic form, i.e.
			\begin{equation}
				\Omega\big( [Z,X],Y \big)+\Omega\big( X,[Z,Y] \big)=0
			\end{equation}

	\end{enumerate}
	
\end{lemma}

\begin{proof}
	Sketch of the proof.

	Let $x_j,y_j\in\mP$. Using the Jacobi identity on the nested commutator $\big[ [x_1,y_1],[x_2,y_2] \big]$ and the facts that $[\mP,\mP]\subset\mH$ and $[\mH,\mP]\subset \mP$, we find the commutator of two elements of $[\mP,\mP]$ belongs to $[\mP,\mP]$.

	First we consider $\mG^{(M)}=\mK\oplus\mP$ and $G(M)$, the associated Lie group. Then we have
	\begin{equation}
		M\simeq G(M)/K.
	\end{equation}
	
	Now one can see that the group $G(M)$ is generated by the products $\{ s_{\mfo}s_x\}$ with $x\in M$. Indeed let $X\in\mP$ and look at $\exp(\mP)$ as map on $M$. Using the symmetry \eqref{EaSymGH} and the fact that $\sigma e^{X/2}= e^{-X/2}$, we have
	\begin{equation}
		s_{\exp(X/2)\cdot \mfo}\mfo= e^{X/2}\big[ \sigma[ e^{-X/2}] \big]=[ e^{X}]= e^{X}\cdot \mfo.
	\end{equation}
	If we act on an other point than $\mfo$, we have
	\begin{equation}
		\begin{aligned}[]
			s_{[ e^{X/2}]}[g]&= e^{X/2}\big[ \sigma( e^{-X/2}g) \big]\\
			&= e^{X/2}\big[  e^{X/2}\sigma(g) \big]\\
			&= e^{X}s_{\mfo}[g]
		\end{aligned}
	\end{equation}
	because $\big[ \sigma(g) \big]=s_{\mfo}[g]$.

	Now, using the lemma \ref{LemAlgEtGroupesGenere}, the fact that the elements $ e^{X}$ with $X\in\mP$ generate $ e^{\mP}$ in $G$ implies that it also generate the elements of the form $ e^{[\mP,\mP]}$ and then the whole $G(M)$. Since the elements $ e^{\mP}$ are of the form $s_{x}s_{\mfo}$, we conclude that $G(M)$ is generated by the products $s_{x}s_{\mfo}$.
	
	Thus we have $g^*\omega=\omega$ for every $g\in G(M)$ because $\omega$ is preserved by all the symmetries.
\end{proof}

%---------------------------------------------------------------------------------------------------------------------------
\subsection{Symplectic triple}
%---------------------------------------------------------------------------------------------------------------------------
\label{SubSecTripleSylple}

A \defe{symplectic triple}{symplectic!triple} is the data of the triple $(\mG,\sigma,\Omega)$ where $(\mG,\sigma)$ is an involutive Lie algebra and $\Omega$ is a $\mK$-invariant nondegenerate $2$-form $\Omega\in\Lambda^2(\mP^*)$. The $\mK$ invariance means that for every $Z\in\mK$ and $X,Y\in\mP$,
\begin{equation}
	\Omega\big( [Z,X],Y \big)+\Omega\big( X,[Z,Y] \big)=0.
\end{equation}

A symplectic triple is the infinitesimal version of a symplectic symmetric space. The following more abstract version of the definition comes from \cite{StrictSolvableSym}:
\begin{definition}
	The triple $(\mG,\sigma,\Omega)$ is a \defe{symplectic triple}{symplectic!triple} when $\Omega\in\Lambda^2\mG$ and
	\begin{enumerate}
		\item  If  $\mG=\mK\oplus\mP$ is the decomposition of $\mG$ into eigenspaces of $\sigma$,  then $[\mP,\mP]=\mK$ and the adjoint representation of $\mK$ on $\mP$ is faithful. ($\mK$ is the eigenspaces with eigenvalue $+1$ of $\sigma$ while $\mP$ is the one of $-1$)
   
		\item The $2$-form $\Omega$ is a Chevalley $2$-cocycle for the trivial representation of $\mG$ on $\eR$.
	  
		\item $i(\mK)\Omega=0$ and $\Omega|_{\mP\times\mP}$ is nondegenerate.
	\end{enumerate}
\end{definition}

Notice that $[\mP,\mP]\subset\mK$ is automatic from the definition of $\mP$ and $\mK$ as eigenspaces of $\sigma$; the hypothesis is the equality.

Let us now see how one build a symplectic symmetric space from the data of the symplectic triple $(\mG,\sigma,\Omega)$. First we consider $G$, the group associated with $\mG$ and $M=G/K$ with the left invariant form $\omega$ build on $\Omega$.

%---------------------------------------------------------------------------------------------------------------------------
\subsection{Example on the Heisenberg algebra}
%---------------------------------------------------------------------------------------------------------------------------

Let $\pH=V\oplus\eR E$  be the Heisenberg algebra of $(V,\Omega^0)$, and consider the derivation
\begin{equation}
	D=\id|_V\oplus(2\id)|_{\eR E}.
\end{equation}
If we consider the algebra $\mA=\eR H$, we build the semi direct product
\begin{equation}
	\mS=\mA\ltimes_D\pH
\end{equation}
with the definition $[H,x]=D(x)$ when $x\in\pH$.

An other split extension that can be done is
\begin{equation}
	\mG_0=\mA\rtimes_{\rho}(\pH\oplus\pH)
\end{equation}
with $\rho=D\oplus(-D)$. The algebra $\mG_0$ is to be endowed with a symplectic triple structure. We define $\sigma_0\colon \mG_0\to \mG_0$
\begin{equation}
	\begin{aligned}[]
		\sigma_0(x,y)&=(y,x)&\in\pH\oplus\pH\\
		\sigma_0(H)&=-H
	\end{aligned}
\end{equation}
and $(\mG_0,\sigma_0)$ is an involutive automorphism. Indeed, we have
\begin{equation}
	\sigma[H,x\oplus y]=\sigma\big( Dx\oplus(-Dy) \big)=-Dy\oplus Dx,
\end{equation}
while
\begin{equation}
	\big[ \sigma H,\sigma(x\oplus y) \big]=[-H,y\oplus x]=-Dy\oplus Dx.
\end{equation}

Let us take the notation\nomenclature[G]{$W_{\pm}$}{The set of elements of the form $(w,\pm w)$ in $W\oplus W$}
\begin{equation}		\label{EqDefNitWpm}
	W_{\pm}=\{ (w,\pm w) \}_{w\in W}.
\end{equation}

If we decompose $\mG_0=\mK_0\oplus\mP_0$, we have
\begin{equation}
	\begin{aligned}[]
		\mK_0&=\pH_+\\
		\mP_0&=\mA\oplus\pH_-.
	\end{aligned}
\end{equation}
We have $H\in\mP$ $x\oplus(-x)\in\mP$ and $x\oplus x\in\mK$.

In fact we have an identification between $\mS$ and $\mP_0$ by
\begin{equation}
	\begin{aligned}
		\mS&\to \mP_0 \\
		a+x&\mapsto a+x_- 
	\end{aligned}
\end{equation}
where $x_{\pm}=\frac{ 1 }{2}(x,\pm x)$. Using the notation \eqref{EqDefNitWpm}, we write
\begin{equation}
	\begin{aligned}[]
		\mK&=\pH_{+}\\
		\mP&=\mA\oplus\pH_{-}.
	\end{aligned}
\end{equation}

Under that identification we have le following lemma.
\begin{lemma}
	We have
	\begin{equation}
		\Lambda^2(\mP_0^*)\simeq\Lambda^2(\mS^*)
	\end{equation}
	and if we define
	\begin{equation}
		\Omega^{\mS}(a+x,a'+x')=\Omega(a+x_-,a'+x'_-),
	\end{equation}
	we have $\Omega\in\Lambda^2(\mP_0^*)$ and it is $\mK_0$ invariant if and only if the two conditions
	\begin{equation}
		\begin{aligned}[]
			\Omega^{\mS}(E,\pH)&=0\\
			\Omega^{\mS}|_{V\times V}&=\frac{ 1 }{2}\Omega^{\mS}(H,E)\Omega^0.
		\end{aligned}
	\end{equation}
	hold.
\end{lemma}

\begin{proof}
	Let us write down the condition of $\mK$-invariance of the symplectic form
	\begin{equation}
		\Omega\big( [x_+,a+y_-],a'+y'_- \big)+\Omega\big( a+y_-,[x_+,a'+y'_-] \big)=0.
	\end{equation}
	If we develop $x_+$ and $y_-$, the commutator in the first term becomes
	\begin{equation}
		\big[ \frac{ 1 }{2}(x,x),a+\frac{ 1 }{2}(y,-y) \big]=-\frac{ a }{2}(Dx,-Dx)+\frac{1}{ 4 }\big( [x,y],-[x,y] \big).
	\end{equation}
	The first term is rewritten as $[x,a]_-$, while the second term is $\frac{ 1 }{2}[x,y]_-$. The sum is then $[x,a+\frac{ 1 }{2}y]_-$. Looking at the definition of $\Omega^{\mS}$, the invariance condition reads
	\begin{equation}
		\begin{aligned}[]
			0&=\Omega\big( [x,a+\frac{ 1 }{2}y],a'+y' \big)+\Omega^{\mS}\big( a+y,[x,a'+\frac{ 1 }{2}y'] \big)\\
			&=\Omega^{\mS}\big( -a Dx+\frac{ 1 }{2}\Omega_0(x,y)E,a'+y' \big)+\Omega^{\mS}\big( a+y,-a'Dx+\frac{ 1 }{2}\Omega_0(x_V,y_V')E \big).
		\end{aligned}
	\end{equation}
	If we look at that condition with $a'=0$, $a=1$, $x_V=0$ and $y=0$ and taking into account $Dx=x_V+2x_EE$ we find
	\begin{equation}
		\Omega^{\mS}(2x_EE,y')=0,
	\end{equation}
	so that $\Omega^{\mS}(E,y')$. We conclude that a necessary condition for the invariance is
	\begin{equation}
		\Omega^{\mZ}(E,\pH)=0.
	\end{equation}
	Now if we consider $y'\in V$, $x_E=0$ and $x_V\neq 0$, we find
	\begin{equation}
		\Omega^{\mS}|_{V\times V}=\frac{ 1 }{2}\Omega^{\mS}(H,E)\Omega_0.
	\end{equation}
	
	One can check that these two conditions insure the $\mK$-invariance of $\Omega^{\mS}$.
\end{proof}

To each non degenerate form satisfying these two conditions corresponds a symplectic triple $(\mG_0,\sigma_0,\Omega)$.

%---------------------------------------------------------------------------------------------------------------------------
\subsection{Realization as coadjoint orbit}
%---------------------------------------------------------------------------------------------------------------------------

Let $(M=G/H,s,\omega)$ be a symmetric symplectic space. We are going to study under which conditions we can realise $M$ as a coadjoint orbit, i.e. we want the two conditions
\begin{enumerate}

	\item
		there exists a $\xi_0\in\mG^*$ such that $\Stab_G(\xi_0)=H$ where $\Stab$ stands for the stabilizer for the coadjoint action of $G$. Let 
		\begin{equation}
			\begin{aligned}
				\Phi\colon M&\to \Ad^*(G)\xi_0=\mO \\
				[g]&\mapsto \Ad^*(g)\xi_0 
			\end{aligned}
		\end{equation}
		be the identification between $M$ and the coadjoint orbit $\mO$.
	\item
		The identification $\Phi$ fits the symplectic structures:
		\begin{equation}
			\Phi^*\omega^{\mO}=\omega
		\end{equation}
		where $\omega^{\mO}$ is the canonical symplectic structure on the coadjoint orbit given by \eqref{eq_omega_Gs}.
\end{enumerate}

\begin{definition}
	A \defe{good polarization}{polarization!good} associated to $\xi_0$ is a Lie subalgebra $\mB$ of $\mG$ which is maximal for the property $\delta\xi_0|_{\mB\times \mB}\equiv 0$ where the alternate bilinear $2$-form $\delta\xi_0$ on $\mG$ is defined by
	\begin{equation}
		\delta\xi_0=\langle \xi_0, [.,.]\rangle.
	\end{equation}
\end{definition}

If $\mB$ is a good polarization, we consider $B=\exp(\mB)$ and we have a representation $\chi\colon \mB\to \gU(1)$ given by
\begin{equation}
	\begin{aligned}
		\chi\colon \mB&\to \gU(1) \\
		\exp(y)&\mapsto  e^{i\langle \xi_0, y\rangle }. 
	\end{aligned}
\end{equation}
It turns out that $\chi$ is a representation even when $\mB$ is non abelian. Indeed, if $x,y\in\mB$, we have
\begin{equation}
	\chi( e^{x} e^{y})=\chi( e^{x+y+W})= e^{i\langle \xi_0, x+y+W\rangle }= e^{i\langle \xi_0, x+y\rangle }= \chi( e^{x})\chi( e^{y})
\end{equation}
where $W$ is a combination if commutators of $x$ and $y$ (Campbell-Baker-Hausdorff) so that by definition of $\mB$, $\langle \xi_0, W\rangle =0$.

Since we are in the hypothesis (see subsection \ref{SubSecUnitInducedPrep}), we can define the induced unitary representation
\begin{equation}
	U\colon G\to \gU(\hH_{\chi})
\end{equation}
where $\hH_{\chi}=L^2(Q,dq)$. Let $dg$ be the left invariant Haar measure on $G$.  To each $u\in L^1(G,dg)$, we make correspond an operator $U(u)$ on $\hH_{\chi}$ given by
\begin{equation}	\label{EqDefUudansHh}
	\langle U(u)\varphi, \psi\rangle =\int_Gu(g)\langle U(g)\varphi, \psi\rangle dg
\end{equation}
for every $\varphi,\psi\in\hH_{\chi}$. Let us prove that this integral exists. We have
\begin{equation}	\label{EqIntdefUgGdgvppsi}
		| \langle U(g)\varphi, \psi\rangle  |\leq\int_G| u(g) | |\langle U(g)\varphi, \psi\rangle  |dg,
\end{equation}
but the Cauchy-Schwartz inequality shows that $| \langle U(g)\varphi, \psi\rangle  |\leq| U(g)\varphi | |\psi |=| \varphi | |\psi |$, so that the integral in \eqref{EqIntdefUgGdgvppsi} is smaller than
\begin{equation}
	| \varphi | | \psi |  \int_G| u(g) |dg
\end{equation}
which exists because we supposed $u\in L^1(G,dg)$.

\begin{probleme}
	The following paragraph can be more precise.
\end{probleme}

We can rewrite the definition \eqref{EqDefUudansHh} using the measure theory given around section \ref{sec_distrib_mesure}. Indeed the space $\opB(\hH)$ of bounded operators\footnote{For linear operators on Hilbert spaces, the fact to be bounded is equivalent to continuity.} on the Hilbert space $\hH$ is endowed with the operator norm for which $\opB(\hH)$ becomes a normed algebra ($\| AB \|_{op}\leq\| A \|_{op}\| B \|_{op}$). The unitary group $\gU(\hH)$ is a subalgebra (because it is closed for the composition), so that one can consider, for each function $u$, the function
\begin{equation}
	\begin{aligned}
		G&\to \opB(\hH) \\
		g&\mapsto u(g)U(g) 
	\end{aligned}
\end{equation}
and its integral
\begin{equation}
	\int_G u(g)U(g)dg
\end{equation}
which is an element in $\opB(\hH)$. This integral is well defined in $\opB(\hH_{\chi})$ because
\begin{equation}
	\| \int_G u(g)U(g)dg \|_{op}\leq\int_G | u(g) |\cdot \| U(g) \|_{op}dg=\| u \|_{L^1}.
\end{equation}

Using the measure theory, one can prove that
\begin{equation}
	\langle U(u)\varphi, \psi\rangle =\langle  \big( \int_G u(g)U(g)dg \big)\varphi , \psi\rangle.
\end{equation}

What we build up to here is a map
\begin{equation}
	\begin{aligned}
		U\colon L^1(G,dg)&\to \opB(\hH_{\chi}) \\
		u&\mapsto U(u) 
	\end{aligned}
\end{equation}
given by
\begin{equation}
	U(u)=\int_G u(g)U(g)dg.
\end{equation}
This map is linear and continuous because $\| U(u) \|_{op}\leq\| u \|_{L^1}$.

We are now going to use the symmetry on $M$ in order to descend $U$ from $L^1(G,dg)$ to $L^1(M)$. Let us take a look at the two projections from $G$:
\begin{equation}
	\xymatrix{%
	G \ar[r]^{\pi^M}\ar[d]_{\pi^Q}		&	G/H	\\
	   G/B
	   }
\end{equation}
If $X,Y\in\mH$, we recall that the definition of $\ad(X)^*$ is
\begin{equation}
	\langle \xi_0, [X,Y]\rangle =-\langle \ad(X)^*\xi_0, Y\rangle,
\end{equation}
but, since $ e^{tX}\in\Stab(\xi_0)$, we have $\Dsdd{ \Ad(\exp(tX))^*\xi_0 }{t}{0}=0$, thus
\begin{equation}
	\langle \ad(X)^*\xi_0, Y\rangle =0
\end{equation}
and we can suppose that the good polarization $\mB$ contains $\mH$. In that case we have the well defined map
\begin{equation}
	\begin{aligned}
		\tilde\pi\colon G/H&\to G/B \\
		gH&\mapsto gB 
	\end{aligned}
\end{equation}
This is well defined because, since $H\subset B$, we have
\begin{equation}
	\tilde\pi(ghH)=ghB=gB
\end{equation}
for every $h\in H$.

\begin{lemma}
	The map $\tilde\pi$ is a submersion.
\end{lemma}
\begin{proof}
	No proof.
\end{proof}
The following diagram commutes:
\begin{equation}
	\xymatrix{%
	G \ar[r]^{\pi^M}\ar[d]_{\pi^Q}		&	G/H\ar[dl]^{\tilde\pi}\\
	   G/B
	   }
\end{equation}

Still two assumptions about $\sigma$:
\begin{enumerate}
	\item
		we suppose that $B$ is stable under $\sigma$,
	\item
		we suppose that $\xi_0$ is $\sigma$-invariant, that is $\xi_0(\sigma X)=\xi_0(X)$.
\end{enumerate}
The second assumption is easy to fulfill. If $\xi_0$ is not $\sigma$-invariant, we consider
\begin{equation}
	\xi_0'=\frac{ 1 }{2}(\xi_0+\sigma^*(\xi_0))
\end{equation}
instead.

Now, the symmetry
\begin{equation}
	\begin{aligned}
		\sigma_H\colon M&\to M \\
		gH&\mapsto \sigma(g)H 
	\end{aligned}
\end{equation}
descends to $Q$ as
\begin{equation}
	\begin{aligned}
		\underline\sigma\colon Q&\to Q \\
		gB&\mapsto\tilde\pi\big( \sigma_H(gH) \big)=\sigma(g)B.
	\end{aligned}
\end{equation}
The so defined map $\underline\sigma$ is well defined because
\begin{equation}
	\underline\sigma(gbB)=\sigma(gb)B=\sigma(g)\sigma(b)B=\sigma(g)B=\underline\sigma(gB).
\end{equation}

\begin{lemma}
	Using the hypothesis of $\sigma$-invariance of $\xi_0$, we have that
	\begin{equation}
		\sigma^*\colon  C^{\infty}(G,\eC)^B\to  C^{\infty}(G,\eC)^B,
	\end{equation}
	the image of a $B$-equivariant function on $G$ by $\sigma^*$ is still $B$-equivariant.
\end{lemma}

\begin{proof}
	Let $\hat\varphi\in C^{\infty}(G,\eC)^B$, then we have
	\begin{equation}
		\begin{aligned}[]
			(\sigma^*\hat\varphi)(gb)&=\hat\varphi\big( \sigma(g)\sigma(b) \big)\\
			&=\chi\big( \sigma(b)^{-1} \big)\hat\varphi\big( \sigma(g) \big)\\
			&= e^{-i\langle \xi_0, \sigma\log(b)\rangle }\hat\varphi(\sigma g)\\
			&= e^{-i\langle \xi_0, \log(b)\rangle }(\sigma^*\hat\varphi)(g)	&\text{because $\xi_0(\sigma X)=\xi_0(X)$}\\
			&=\chi(b^{-1})(\sigma^*\hat\varphi)(g).
		\end{aligned}
	\end{equation}
\end{proof}

Since the measure $dq$ is $\sigma^*$-invariant by hypothesis, we have
\begin{equation}
	\int_Q\overline{ (\underline\sigma^*u) }(q)(\underline\sigma^*v)(q)dq=\int_Q\overline{ u(q') }v(q)\underline\sigma^*dq=\int_Q\overline{ u(q') }v(q)dq
\end{equation}
where we used the change of variable $q'=\sigma q$. A consequence is that $\underline\sigma^*$ is an involution
\begin{equation}
	\underline\sigma^*\colon L^2(Q,dq)\to L^2(Q,dq).
\end{equation}
Since, in an abstract way, we denoted $L^2(Q,dq)$ by $\hH_{\chi}$, we denote by $\Sigma$ the involution $\underline\sigma^*$ on $\hH_{\chi}$. Now we consider the function
\begin{equation}
	\begin{aligned}
		\Omega\colon G&\to \gU(\hH_{\chi}) \\
		g&\mapsto U(g)\Sigma U(g^{-1}) 
	\end{aligned}
\end{equation}
which is a composition of unitary maps. This is not a representation of the group $G$, but we have
\begin{equation}
	\Omega(gh)=\Omega(g)
\end{equation}
for every $h\in H$ and $g\in G$. Indeed let us compute $\widehat{\Omega(gh)\varphi}$ for $\varphi\in\cdD(Q)$. We have
\begin{equation}		\label{EqwOshvhvkl}
		\widehat{\Omega(gh)\varphi}=\hat U(gh)\sigma^*\hat U(h^{-1}g^{-1})\hat\varphi
		=\hat U(h)\hat U(h)\sigma^*\hat U(h^{-1})\hat U(g^{-1})\hat\varphi.
\end{equation}
The element $h$ only appears in the combination $\hat U(h)\sigma^*\hat U(h^{-1})$, so let us see how it acts on an equivariant function $\hat \varphi$. If we evaluate it on $g_0$ we find
\begin{equation}		\label{EqbhUsigmastargzi}
		\big( \hat U(h)\sigma^*\hat U(h^{-1})\hat\varphi \big)(g_0)=\big( \sigma^*\hat U(h^{-1})\hat\varphi \big)(h^{-1}g_0)
		=\hat\varphi\big( h\sigma(h^{-1}g_0) \big).
\end{equation}
Let us recall that we are in the context\footnote{With many notational incoherences.} of subsection \ref{SubSecInducrepresBBGC}: the representation $U$ on $\cdD(Q)$ comes from the regular left representation $\hat U$ on $ C^{\infty}(G,\eC)^B$. Thus we have $\big( \hat U(g)\hat\varphi \big)(g_0)=\hat\varphi(g^{-1}g_0)$. Equation \eqref{EqbhUsigmastargzi} is thus equal to
\begin{equation}
	\hat\varphi\big( h\sigma(h^{-1}g_0) \big)=\hat\varphi\big( h\sigma(h^{-1})\sigma(g_0) \big)=\hat\varphi\big( \sigma(g_0) \big)=(\sigma^*\hat\varphi)(g_0),
\end{equation}
so that equation \eqref{EqwOshvhvkl} does not depend on $h$, which proves that
\begin{equation}
	\Omega(gh)=U(g)\Sigma U(g^{-1})=\Omega(g).
\end{equation}
One consequence of this circumstance is that $\Omega$ is a function which pass to the quotient $G\to G/H$. Thus we consider the map
\begin{equation}
	\Omega\colon M=G/H\to \gU(\hH_{\chi})\subset\opB(\hH).
\end{equation}

Since $M$ is a symplectic manifold, we have a natural volume form
\begin{equation}
	dx=\frac{1}{ n! }\omega^n
\end{equation}
where $n=\frac{ 1 }{2}\dim M$. This measure allows us to consider the map
\begin{equation}
	\begin{aligned}
		\Omega\colon L^1(M,dx)&\to \opB(\hH) \\
		u&\mapsto \Omega(u) 
	\end{aligned}
\end{equation}
defined by
\begin{equation}
	\Omega(u)=\int_M u(x)\Omega(x)dx
\end{equation}
which is a continuous linear map. This is not a representation (even on $G$ the initial $\Omega$ was not a representation and $M$ is not a group), but it is an unitary representation of $M$ in the following sense.

\begin{definition}
	An \defe{unitary representation}{representation!of a symmetric space} is a map $\Omega\colon M\to \gU(\hH)$ of the symmetric space $M$ when it satisfies to the properties
	\begin{enumerate}	
		\item
			$\Omega(x)\Omega(y)\Omega(x)=\Omega(s_xy)$
		\item
			$\Omega(x)^2=\id|_{\hH}$.
	
	\end{enumerate}
	for every $x,y\in M$.
\end{definition}
\section{Hermitian and symplectic spaces}
%+++++++++++++++++++++++++++++++++++++++
\label{SecHermEtSymplecticSpaces}

If $ANK$ is the Iwasawa decomposition of a Lie group\quext{c'est pas mal de dire quel genre de groupes : simple ? semi ? compact ?}, one can consider the manifold $M=G/K$. There is a natural identification
\[
   \mP=T_KM
\]
where $\mP$ comes from the Cartan decomposition $\mG=\mP\oplus\mK$. Indeed, the Iwasawa theorem says that $M\simeq AN$ so that a path in it reads $g(t)=a(t)n(t)$ with $g(0)=e$. But one has a diffeomorphism $A\times N\times K\to G$, so that $g(0)=e$ implies $a(0)=n(0)=e$. Thus Leibnitz makes $g'(0)=a'(0)+n'(0)$ and $g'(0)\in\mA\oplus\mN=\mP$.

Let us recall that when $G$ is a Lie group, and $H$ a closed connected subgroup of $G$, $G/H$ is a manifold on which $G$ acts. This structure is an \defe{homogeneous space}{homogeneous!space}. If moreover $H$ is the set of the fixed points of an involution on $G$, $G/H$ is says to be a \defe{symmetric space}{symmetric!space}.

More precisely, the involution is a $\dpt{\theta}{\mG}{\mG}$ which let fixed $\mH\subset\mG$; then $H$ is the connected Lie group whose Lie algebra is $\mH$. All this makes that the $G/K$ from Iwasawa is a symmetric space.

\begin{definition}		\label{DefSymHermMGKalg}
  The symmetric space $M=G/K$ is \defe{hermitian}{hermitian!symmetric space} is there exists an endomorphism $J\in\End{\mP}$, $\dpt{J}{\mP}{\mP}$ such that
\begin{subequations}  
\begin{align}  
  J^2&=-id_{\mP},                                           \label{eq:herm_1} \\
  B(JX,JY)&=B(X,Y)            && \forall\,X,Y\in\mP,    \label{eq:herm_2}\\
  \ad (k)\circ J&=J\circ\ad(k)&& \forall\,k\in\mK.      \label{eq:herm_3}
\end{align}    
\end{subequations}
\label{def:hermitien}
\end{definition}

Since one has the identification $\mP=T_KM$, $J$ is only defined on $T_{[e]}M$. The following proposition extends the definition.

\begin{proposition}
The hermitian structure $J$ can be extended to a complex structure $\oJ$ on the whole $TM$.
\label{prop:ext_J}
 \end{proposition}

\begin{proof} 
 For $X\in T_{[g]}M$, we define
\begin{equation} 
  \oJ(X):=dL_g\circ J\circ dL_{g^{-1}}X.
\end{equation}
where $dL$ is the differential of $\dpt{L_g}{G/K}{G/K}$, $L_g[h]=[gh]$.
From this, $\oJ^2(X)=-X$ because
\begin{equation}
  (\oJ\circ\oJ)X=( dL_g J dL_{g^{-1}} )\circ( dL_g J dL_{g^{-1}} )X
           =dL_g J^2dL_{g^{-1}}X
	   =-X.
\end{equation}
On the other hand, $J$ satisfies $\ad(k)\circ J=J\circ \ad(k)$ and we want the same for $\oJ$: 
\[
  \ad(X)\circ\oJ=\oJ\circ\ad(X) 
\]
for $X\in T_{[g]}M$. Note that it is true for $[g]=[e]$ because $T_{[e]}M=\mK$. Let us consider $X\in T_{[g]}M$, and let us see what is $ \big( (\ad X)\circ J \big)Y $ for a $Y\in T_{[g]}M$. Consider $x$, $y\in T_{[e]}M$ such that $X=dL_g x$ and $Y=dL_g y$. Suppose one has 
\begin{equation}\label{eq:suppose}
   \ad(dL_g x)Y=dL_g\circ\ad(x)(dL_{g^{-1}}Y);
\end{equation}
then one can compute
\begin{subequations}
    \begin{align}
\big(  \ad(X)\circ\oJ \big)Y&=\ad(dL_g x)\circ dL_g\circ J\circ dL_{g^{-1}}Y\\
&=dL_g\ad(x) \circ J\circ dL_{g^{-1}}Y\label{subEqEMyROwA}\\
	                    &=dL_g\circ J\circ\ad(x)\circ dL_{g^{-1}}Y\\
			    &=(dL_g\circ J\circ dL_{g^{-1}})\circ (dL_g\circ\ad(x)\circ dL_{g^{-1}})\\
			    &=\oJ\circ\ad(X)Y.
    \end{align}
\end{subequations}
The line \eqref{subEqEMyROwA} comes from $\ad(x)\circ J=J\circ\ad(x)$ because $x\in T_{[e]}M$.
 
Now, we prove equation \eqref{eq:suppose} which is rewritten in a more convenient way as $[dL_g x,Y]=dL_g[x,dL_{g^{-1}}Y]$. Thus one has to see if for any $x$, $y\in T_{[e]}M$, 
\[
   dL_g[x,y]=[dL_g x,dL_g y].
\]
This is true because of \cite{Helgason}, proposition 3.3, page 34.  Now we know that $\forall X\in T_{[g]}M$ we have $\oJ\circ\ad(X)=\ad(X)\circ\oJ$ and ${\oJ}\,^2X=-X$.  In order to have a complex structure, one also need to check condition \eqref{DefComplStruct}, which is true because
\begin{equation}
\begin{split}
J[JX,Y]&=-\ad Y\circ JJX=\ad(Y)X=[X,Y],\\
J[X,JY]&=J\circ\ad(X)JY=-[X,Y],\\
-[JX,JY]&=-(\ad JX\circ J)Y
        =-J(\ad JX)Y
	=-[Y,X].
\end{split}
\end{equation}

\end{proof}

\noindent If $X\in T_{[g]}M$,
\begin{equation}\nonumber
\begin{split}
  (\oJ\circ dL_h)X&=dL_{hg}\circ J\circ dL_{(hg)^{-1}}dL_h X\\
           &=dL_{hg}\circ J\circ dL_{g^{-1}} X\\
	   &=dL_h\circ dL_g\circ J\circ dL_{g^{-1}} X\\
	   &=(dL_h\circ \oJ)X.
\end{split}
\end{equation}
so we have an important property:
\begin{equation}\label{eq:J_dL}
   \oJ\circ dL_h=dL_h\circ\oJ.
\end{equation}
From now, it is clear that we will often forget the bar on $\oJ$.  In the same way that $J$ extends to $M$, 

\begin{proposition}
For $X$, $Y\in T_{[g]}M$, the formula
\begin{equation}\label{eq:BdL}
  \overline{ B }(X,Y):=B(dL_{g^{-1}}X,dL_{g^{-1}}Y)
\end{equation}
defines a Riemannian metric on $M$.

\end{proposition}

\begin{proof}
One has to see that it is nondegenerate. Say that $Z\in T_{[g]}M$ is such that for any $X$, 
$\overline{ B }(Z,X)=0$. Then $B(dL_{g^{-1}}Z,dL_{g^{-1}}X)=0$. But $dL_g$ is a vector space isomorphism because 
$dL_g(o)=\Dsdd{L_g(X_t)}{t}{0}$ with $X_t$, a constant path at $[e]\in M$.
       
But since $B$ is nondegenerate, the definition \eqref{eq:BdL} says us $dL_{g^{-1}}Z=0$, and then $Z=0$.
\end{proof}

Now, one knows\quext{Cf cours de géométrie symplectique} that 
\begin{equation}
  \omega^M_x(X,Y)=g_x(JX,Y)
\end{equation}
defines a $G$-invariant symplectic structure on $M$.

In order to see it, one has to show that $(M,g,J)$ is  a Kähler structure. The $G$-invariance comes from the extension of Killing form that we had chosen: $\forall X,Y\in T_{[g]}M$, $B_{[g]}(X,Y)=B(dL_{g^{-1}}X,dL_{g^{-1}}Y)$.
It is clear that 
\begin{equation}
B_{[hg]}(dL_hX,dL_hY)=B_{[g]}(X,Y).
\end{equation}
From this and equation \eqref{eq:J_dL}, one can see the $G$-invariance of $\omega^M$:
\begin{equation}
\begin{split}
   \omega^M_{[hg]}\big((dL_h)_{[g]}X, (dL_h)_{[g]}Y\big)&=B_{[hg]}(JdL_hX,dL_hY)
                                                =B_{[hg]}(dL_h J X,dL_hY)\\
						&=B_{[g]}(JX,Y)
						=\omega^M_{[g]}(X,Y).   
\end{split}
\end{equation}
The formulation of the $G$-invariance is
\begin{equation}
   \omega^M_{[hg]}\Big( (dL_h)_{[g]}X, (dL_h)_{[g]}Y \Big)=\omega^M_{[g]}(X,Y).
\end{equation}

\subsection{The Chevalley cohomology}
%------------------------------------

Let $\mG$ be a Lie algebra (maybe infinite dimensional) and $(V,\rho)$ a representation of $\mG$ on the vector space $V$. The \defe{Chevalley cohomology}{chevalley!cohomology} of $\mG$ associated with the representation $\rho$ is given by the following definitions:

A $p$-cochain is a map $\dpt{C}{\underbrace{\mG\times\ldots\times\mG}_{\text{$p$ times}}}{V}$ which is multi-linear and skew-symmetric. In particular, a $1$-cochain is a linear map $\dpt{\xi}{\mG}{V}$. In the case of the trivial representation on $\eR$, a $1$-cochain is an element of $\mG^*$. The coboundary of a $p$ cochain is the $p+1$-cochain given by
\begin{equation}
\begin{split}
(\delta C)(X_0,\ldots,X_p)=&\sum_{i=0}^{p}(-1)^i\rho(X_i)C(X_0\ldots,\hX_i,\ldots,X_p)\\
                           &+\sum_{i<j}(-1)^{i+j}C\big(  [X_i,X_j],\ldots,\hX_i,\ldots,\hX_j,\ldots,X_p \big).
\end{split}
\end{equation}
The main property is $\delta^2=0$. The others definitions are as usual: a $p$-cocycle is a $p$-cochain $C$ such that $\delta C=0$, a $p$-coboundary is a $p$-cochain which can be written as $\delta B$ for some  $(p-1)$-cochain $B$. Finally, the cohomology classes are:
\begin{equation}
H^{p}_{(V,\rho)}=\frac{\text{$p$-cocycles}}{\text{$p$-cochain}}=H^p_{\rho}(\mG,V).
\end{equation}

When one consider the trivial representation, i.e. $\rho(X)=0$, a $1$-cochain is $\xi\in\mG^*$ and
\begin{equation}  \label{EqDefcochaintrivC}
(\delta\xi)(X,Y)=-\xi([X,Y]).
\end{equation}

\begin{probleme}
Au cas où ça t'intéresserait, je te dis que le signe moins, tu ne l'as ajouté qu'en février 2007. T'étonnes pas si y'a des signes qui foirent plus bas.
\end{probleme}

Now, on the symmetric hermitian space $M=G/K$, one defines a $\Omega\in\Lambda^2(\mG^*)$ by
\begin{subequations}
\begin{align}
   \Omega(X,Y)&=B(JX,Y)&\text{for $X,Y\in\mP$}  \label{eq:def_Omega_1}    \\
   \Omega(\mK,\mG)&=0.                          \label{eq:def_Omega_2} 
\end{align}
\end{subequations}   

A great property of this definition is that $\Omega$ is a $2$-cocycle for the trivial representation of $\mG$ on $\eR$:
\[
\Omega([X,Y],Z)+\Omega([Y,Z],X)+\Omega([Z,X],Y)=0.
\]

Indeed, if $X$ ,$Y$, $Z\in\mP$, the commutators are in $\mK$, so that \eqref{eq:def_Omega_2} makes the whole null. The second case is $X$, $Y\in\mP$ and $Z\in\mK$; for this, we are led to consider the quantity $-B( [Y,Z],JX )-B([Z,X],JY)$. The first term can be transformed as:
\[
\begin{aligned}
  B([Y,Z],JX)&=-B(J[Y,Z],X)&&\text{by def. \eqref{eq:herm_2} } \\
             &=B([Z,JY],X)&&\text{by def.  \eqref{eq:herm_3}}\\
	     &=-B(JY,[Z,X])&&\text{$\ad$-invariance of $B$}\\
	     &=-B([Z,X],JY).
\end{aligned}
\]
So it is zero.

\begin{lemma}[Whitehead's lemma]
If $\mG$ is  a finite dimensional semisimple  Lie algebra and $\rho$ a non trivial\quext{Ce qui n'est pas le cas ici} representation of $\mG$ on $V$, then $\forall\,q\geq 0$,
\[
      H^q(\mG,V)=0.
\]
\end{lemma}

This gives us the existence of a $\xi_0\in\mG^*$ (an Chevalley $1$-cochain) such that 
\[
   \delta\xi_0=\Omega.
\]
\begin{equation}\label{eq:Z_0}
   \xi_0=B(Z_0,.).
\end{equation}

\begin{definition}
	The \defe{center}{center!of a Lie algebra}\nomenclature{$\mZ(\mG)$}{Center of a Lie algebra} of the Lie algebra $\mG$ is the set $\mZ(\mG)\subset\mG$ of elements $Z$ such that $[Z,X]=0$ for every $X\in\mG$. See also the definition of a centralizer on page \pageref{PgDefCentralisateur}.
\end{definition}

\begin{proposition}
The $Z_0$ defined in \eqref{eq:Z_0} and the $J$ of proposition \ref{prop:ext_J} satisfy
\begin{subequations}
\begin{align}
   Z_0&\in\mZ(\mK),\\
   J&=\pm\ad(Z_0)|_{\mP}.
\end{align}   
\end{subequations}


\end{proposition}

\begin{proof}
First, we see that for any $K\in\mK$, $[Z_0,K]=0$. We know from \eqref{eq:def_Omega_2} that $\forall\,G\in\mG$, $K\in\mK$, $\Omega(K,G)=0$, or
\begin{equation}
\begin{split}
  0=\delta B(Z_0,.)(K,G)=-B(Z_0,.)([K,G])
                       =B([K,Z_0],G),
\end{split}
\end{equation}
thus $[K,Z_0]=0$ because $B$ is nondegenerate. We will see below that $Z_0\in\mP$ is not possible.  On the other hand, the condition \eqref{eq:def_Omega_1} gives us
\[
  B( [X,Z_0],Y )=B(JX,Y)
\]
for any $Y\in\mP$. Thus $[X,Z_0]=JX$ and the second claim follows. Let us now see that $Z_0\in\mP$ is not possible (and so we finish the proof of the first claim). We know that $J^2X=[Z_0,[Z_0,X]]$, but for $Z_0$, $X\in\mP$, $[Z_0,X]\in\mK$ and so $J^2X=0$ which is not possible.

\end{proof}


\begin{lemma}
A symmetric space $G/K$ is hermitian if and only if $\mZ(\sK)\neq 0$.
\end{lemma}

\begin{proof}
If the space is hermitian, we just said that the $J$ can be written under the form $J=-\ad(Z_0)|_{\sP}$ for a $Z_0\in\mZ(\sK)$. For the sufficient condition, we define $J=-\ad(Z_0)$ for a certain $Z_0\in\mZ(\sK)$. As a first point for all $k\in\sK$ and $p\in\sP$,
\begin{equation}
\begin{split}
(J\circ\ad k)p=[ [k,p],Z_0]
              =-[ [p,Z_0],k]-[ [Z_0,k],p]
              =[k,[p,Z_0]]
              =(\ad k\circ J)p
\end{split}
\end{equation}
The two other points are
\begin{subequations}
\begin{align}
  J^2=(-\ad Z_0)[X,Z_0]=-[ [Z_0,X],Z_0]
\intertext{and}
  B([Z_0,X],[Z_0,Y])=-B(X,[ Z_0,[Z_0,Y]])=B(X,Y)
\end{align}
\end{subequations}
These are true if $[ [Z_0,X],Z_0]=X$\quext{Mais je ne vois pas comment obtenir \c ca. Si $\ad Z_0$ est un automorphisme de $\sP$, alors je suis d'accord.}

Let us prove that $\mM:=\ker(\ad Z_0)=0$. For remark that $(\sG,B)$ is a Riemannian space and let $W$ be the orthogonal complement of $\mM$ in $\sP$. We begin to prove that $\mM$ is $(\ad\sK)$-invariant.

If $x\in\ket Z_0$ and $k\in \sK$, then
\[ 
  [Z_0,[k,x]]=-[k,[x,Z_0]]-[x,[Z_0,k]]=0
\]
because $[x,Z_0]=0=[Z_0,k]$. Now if $\mM$ is $(\ad\sK)$-invariant, then $W=\mM^{\perp}$ is too because
\[ 
  B([k,x],m)=-B(x,[k,m])=0
\]
since $[k,x]\in\mM$. So we have the orthogonal direct sum $\sP=\mM\oplus W$. We are now going to see that $[\mM,W]=0$. Let $X,X'\in\sP$;
\begin{equation}
  B\big( [ [m,w],X],X' \big)=B\big( [m,w],[X,X'] \big)
		=B\big( w,[ [X,X'],m] \big)
		=0
\end{equation}
since $[X,X']\in\sK$ and $[ [X,X'],m]\in\mM$ from the $\sK$invariance of $\mM$. If we define $A=[m,w]$, the endomorphism $\ad(A)|_{\sP}$ is zero.

We know\quext{Il faut encore voir d'où sort ce truc.} that $[\sK,\sK]=\sP$, and then that
\[ 
  [A,k]=[A,[p,p']]=-[p,[p',A]]-[p',[A,p]]=0.
\]
So $[A,\sG]=0$ and $A=0$ because $\sG$ is semisimple. If we write $\sG=[\sP,\sP]\oplus\sP$, we find
\[ 
  \sG=\big( [\mM,\mM]\oplus\mM \big)\oplus\big( [W,W]\oplus W \big),
\]
where the two brackets commute. It furnish a decomposition of $\sG$ into ideals which impossible from the semi-simplicity assumption. We conclude that $\mM=$ and that $\ad Z_0$ is bijective on $\sP$.

\end{proof}


\begin{lemma}
Let $\sG$ a simple Lie algebra with Iwasawa decomposition $\sG=\sK\oplus\sA\oplus\sN$. We suppose that $\mZ(\sK)\neq 0$. Then $\dim\sA\geq\dim\mZ(\sN)$.
\end{lemma}

\begin{proof}
Let $\dpt{i}{\sR}{\sG}$ be the canonical projection and $\xi_0\in\sG^*$ such that $\delta\xi_0=\Omega$. Since $\sK\cap \sR=\{ 0 \}$, the radical of $\delta(i^*\xi_0)$ is trivial. Indeed, when $X\in\sR$, we have $(i^*\xi_0)X=\xi_0 X$ and equation $\delta(i^*\xi_0)(X,Y)=0$ for all $X$, $Y\in \sR$ gives $B(JX,Y)=0$ because $\sK\cap\sR=\{ 0 \}$. Then $JX=0$ and $X=0$.\quext{\c Ca demande que $B$ soit non d\'eg\'en\'er\'ee sur $\sR$, et je ne vois pas trop pourquoi ce serait vrai.}.

Let $V$ be the radical of $\Omega$ in $\sN$; if $z\in\mZ(\sN)$, then $\Omega(\sN,z)=(\delta\xi_0)(z,\sN)=\xi_0[z,\sN]=0$. Then $\mZ(\sN)\subset V$. Now let us consider the map $\dpt{\psi}{V}{\sA^*}$,
\[ 
  \psi(v)=\Omega(v,.)|_{\sA}.
\]
Let us prove that $\psi$ is injective. For, we consider a $v\in V$ such that $\Omega(v,\sA)=0$. Since $v\in V$, we have $[v,\sN]=0$ and then $\Omega(v,\sN)=0$. So,
\[ 
  0=\Omega(v,\sA\oplus\sN)=\delta(i^*\xi_0)(v,\sR),
\]
 and then $v=0$ because the radical of $i^*\xi_0$ in $\sR$ is only zero. Consequently,
\[ 
  \dim\sA=\dim\sA^*\geq \dim V\geq\dim\mZ(\sN)
\]
because there exists an injection from $V$ into $\sA^*$ and $\mZ(\sN)\subset V$.

\end{proof}


\begin{lemma}
Let us suppose that $\dim\sA=1$ and $\dim\sG\geq 3$. Then

\begin{enumerate}
\item The root system is $\Phi=\{ \pm\alpha,\pm 2\alpha \}$,
\item $\sN=\sG_{\alpha}\oplus\sG_{2\alpha}$ and $\sG_{2\alpha}=\mZ(\sN)$
\item $\dim\mZ(\sN)=\dim\sA=1$
\item There exists a $E\in\mZ(\sN)$ such that $[x,y]=\Omega(x,y)E$ for all $x$, $y\in\sN$. The subspaces $\sA\oplus\sN$ and $\sG_{\alpha}$ are symplectic and orthogonal in $(\sR,\Omega)$. In particular, $\sN$ is an Heisenberg algebra.
\end{enumerate}

\end{lemma}

\begin{proof}
No proof.
\end{proof}

This lemma allows us to parametrize $\sR$ as
\[ 
  r=aA+x+zE
\]
with $x\in\sG_{\alpha}$ and $a\in \sA$ because $\mZ(\sN)$ is spanned by the unique element $E$. Now if we consider a function $u\in C^{\infty}(\sR)$, we can define a partial Fourier transform
\[ 
  F(u)(a,x,\xi)=\hat u(a,x,\xi)=\int_{\mZ(\sN)}e^{-i\xi z}u(aA+x+zE)dz.
\]

\begin{theorem}
Let consider the diffeomorphism $\dpt{\phi_{\hbar}}{\sR}{\sR}$ given by
\[ 
   \phi_{\hbar}(a,x,\xi)=\left( a,\frac{1}{\cosh(\frac{\hbar\xi}{2})}x,\frac{\sinh(\hbar\xi)}{\hbar} \right).
\] 
Then 

\begin{enumerate}
\item $\phi_{\hbar}^*\swS(\sR)\subset\swS(\sR)$,
\item $(\phi^{-1}_{\hbar})^*\swS(\sR)\subset\swS'(\sR)$.
\end{enumerate}
where $\swS$ and its dual $\swS'$ are defined in section \ref{sec:Distrib}.

\end{theorem}

We recall the notation for functions: $\varphi^*f=f\circ\varphi$.

\begin{proof}
For sake of simplicity, we forget about variable $a$, we pose $y=\hbar \xi$ and we look at the function $\dpt{\phi}{\eR^2}{\eR^2}$ given by
\[ 
  \phi(x,y)=(\sech(\frac{y}{2})x,\sinh(y)).
\]
Formula
\[ 
  \frac{\sqrt{2}}{2}(1+\sqrt{1+y^2})^{1/2}=\cosh\big( \frac{\arcsinh(y)}{2} \big),
\]
allows us to write
\begin{equation}
\phi^{-1}(x,y)=\left( \cosh\Big( \frac{\arcsinh(y)}{2} \Big)x,\arcsinh(y) \right).
\end{equation}
We pose $p_{nm}(x,y)=x^ny^m$ and we are going to study 
\[ 
  (p_{nm}\circ\phi^{-1})(x,y).
\]
It has a polynomial grown because, for large $y$, $\sinh(\ln y)=\frac{1}{2} y$. Hence $\arcsinh(y)\simeq \ln(2 y)$. The matrix of $d\phi_{(x,y)}$ is given by
\begin{equation}
d\phi_{(x,y)}=
\begin{pmatrix}
\sech(\frac{y}{2}) & -\frac{x}{2}\tanh(\frac{y}{2})\sech(\frac{y}{2})\\
0                  &   \cosh(y)
\end{pmatrix}.
\end{equation}
Since $\phi$ is a diffeomorphism, and then is bijective,
\begin{equation}
\begin{split}
  \sup_{a\in\eR^2}| p_{nm}(a)(u\circ \phi)(a) |&=\sup_{a\in\eR^2}| p_{nm}(\phi^{-1}(a))u(a) |\\
                                               &\leq \sup_{a\in\eR^2}| P_{MN}(a)u(a) |
\end{split}
\end{equation}
for a choice of $N$, $M\in\eN$. In order to check the derivatives, we need the asymptotic behaviour of $(u\circ\phi)'(x,y)$ given components of\quext{Pour moi, la composante $a=i=2$ ne fonctionne pas parce que c'est $(\partial_2)_{\phi(x,y)}\cosh(y)$.}
\[ 
  \partial_a(u\circ\phi)(x,y)=\sum_i(\partial_iu)_{\phi(x,y)}(\partial_a\phi_i)(x,y).
\]
The derivatives of $\phi^*u$ are the quantities
\[ 
  \partial_a(u\circ\phi^{-1})(x,y)=(\partial_iu)(\phi^{-1}(x,y))\partial_a(\phi_i^{-1})(x,y).
\]
This has a polynomial behaviour. One can see recursively that the same is true for second derivatives $\partial^2_{ab}(u\circ\phi^{-1})$ and higher. This proves that $\phi^*\swS(\eR^2)\subset\swS(\eR^2)$.

In order to see that $(\phi^{-1})^*u\in\swS'$ when $u\in\swS$, we have to prove that
\begin{equation} \label{eq:r1181205}
  \int_{\mU}| x^{-N}y^{-M}(\phi^{-1})^*u(x,y)dxdy |<\infty
\end{equation}
for a choice of $N,M$. Here, $\mU$ is the complement in $\eR^2$ of a compact neighbourhood of the origin. Indeed, the fact for $f$ to belongs to $\swS'$ is the \emph{distribution} $T_f$ to belongs to $\swS'$. In other words, the condition $f\in\swS'$ is the continuity of $\varphi\to\int_X f\varphi$ when $\varphi\in\swS$. The essentially resides in the existence of the integral.

In general -- here, $f$ take the role of $\phi^*u$ -- we have
\[ 
  | \int_X f\varphi |\int | f\varphi |\leq \int | f p_{-N,-M} |
\]
for all $N,M\geq 0$ because $\varphi$ decrease more rapidly than any polynomial. If we find $M$ and $N$ such that $\int | fp_{-N,-M} |<\infty$, then we prove that the distribution belongs to $\swS'$. In our case more precisely, we know that $\varphi$ is smooth. Then it can be majored in any compact set. This is the reason why we write an integral over $\mU$ instead of the whole $\eR^2$.

In equation \eqref{eq:r1181205}, we perform the change of variable $a'=\phi^{-1}(a)$:
\begin{equation}
\begin{split}
\int_{\mU}| x^{-N}y^{-M}(u\circ\phi^{-1})&(a)|\,da=\int_{\mU'}| \frac{1}{p_{MN}(\phi(a'))}u(a') | |J_{\phi}(a') |\,da'\\
                        &=\int_{\mU'}\frac{| u(a) |}{\left|  \big( \frac{x}{\cosh(\frac{y}{2})} \big)^N\sinh(y)^M  \right|}
                                     \left|  \frac{\cosh(y)\cosh(\frac{y}{2})}{}  \right|da\\
                        &=\int_{\mU'}\left| \frac{1}{x^N}2^{1-M}\cosh(\frac{y}{2})^{N-M}\sinh(\frac{y}{2})^{1-M}\right|\, |u(a) |\,da.
\end{split}
\end{equation}
The latter integral is finite when $M\geq 1$ and $M>N$.\quext{Pierre trouve d'autres choses, mais sa conlusion est la même; comme si il utilisait un autre formulaire de trigono hyperbolique que moi.}

The same kind of upper bound\quext{Traduction de «majoration»} holds for the derivatives of $u\circ\phi^{-1}$ for which we have to study the behaviour of the inverse matrix $(d\phi)^{-1}$. All this proves that $(\phi^{-1})^*\swS(\eR^2)\subset \swS(\eR^2)$.

\end{proof}

\subsection{Involutive symmetric Lie algebras}
%----------------------------------------------

\begin{definition}
An \defe{involutive Lie algebra}{involutive!Lie algebra} is a doublet $(\mG,\sigma)$ where $\mG$ is a real finite dimensional Lie algebra and $\dpt{\sigma}{\mG}{\mG}$ is an involutive automorphism of $\mG$.
\end{definition}

There are three types of triples $(\mG,\sigma,\Omega)$: 
\begin{enumerate}

	\item
		the symplectic triple,\index{triple!symplectic}
	\item
		the exact triple, \index{triple!exact}
	\item
		the elementary solvable exact triple (ESET). \index{triple!elementary solvable}
\end{enumerate}
In these three types, $(\mG,\sigma)$ is an involutive Lie algebra. The following definitions can be found in \cite{StrictSolvableSym}.

Symplectic triples were already defined in section \ref{SubSecTripleSylple}.
\begin{definition}
An \defe{exact triple}{exact triple} is a triple $(\mG,\sigma,\Omega)$ such that
\begin{enumerate}
\item $\mG\stackrel{\sigma}{=}\mK\oplus\mP$ and $[\mP,\mP]=\mK$,
\item $\Omega$ is a Chevalley $2$-coboundary such that $i(\mK)\Omega=0$ and $\Omega|_{\mP\times\mP}$ is a symplectic structure on $\mP$.
\end{enumerate}
\end{definition}
The exact triple has the following differences compared to the symplectic one:
\begin{itemize}
\item $\Omega$ is a coboundary instead as a cocycle,
\item $\Omega|_{\mP\times\mP}$ is not only nondegenerate, but also symplectic.
\end{itemize}
From definition of a coboundary, in an exact triple, there exists a $\xi\in\mG^*$ such that $\Omega=\delta\xi$.

\begin{definition}
An \defe{elementary solvable exact triple}{elementary!solvable exact triple}\index{ESET} (ESET) is an exact triple $(\mG,\sigma,\Omega)$ such that
\begin{enumerate}
\item The Lie algebra $\mG$ is a split extension of abelian algebras:
\begin{equation}   \label{EqSplitmGABab}
  \mG=\mA\oplus_{\rho}\mB.,
\end{equation}
\item the automorphism $\sigma$ preserves the vector space decomposition $\mG=\mA\oplus\mB$.
\end{enumerate}

\end{definition}

\begin{remark}
When one writes $\mG=\mA\oplus_{\pi}\mB$, one has $\pi\colon \mA\to \Der(\mB)$. This is the inverse convention of the one chosen in the article \cite{StrictSolvableSym}.
\end{remark}

 In the case of an ESET, we have $\mA\cap\mK\subset\mA\cap[\mG,\mG]$ because $\mK$ is equal to $[\mP,\mP]$ and is thus included in $[\mG,\mG]$. But $[\mG,\mG]$ can be $[a,b]$, $[a,b]$ or $[b,b]$. The two latter are zero (because $\mA$ and $\mB$ is abelian) and, by definition of the split extension, $[a,b]=\rho(a)b\in\mB$. So $\mA\cap[\mG,\mG]=0$. Therefore,
\[ 
  \mA\cap\mK=0;
\]
we deduce that $\mA\subset\mP$ and $\mK\subset \mB$. Since $\mK\subset\mB$, we define $\mL$ as the complement:
\[ 
  \mB=\mK\oplus\mL.
\]
In particular, $\mK$ and $\mL$ are abelian.

The dimension\index{dimension!of a symplectic triple} of a triple is the dimension of $\mP$ and two triples $(\mG_i,\sigma_i,\Omega_i)$ are \defe{isomorphic}{isomorphism!of symplectic!triple} if there exists a Lie algebra isomorphism $\dpt{\psi}{\mG_1}{\mG_2}$ such that $\psi\circ\sigma_1=\sigma_2\circ \psi$ and $\psi^*\Omega_2=\Omega_1$.

\subsection{Symplectic symmetric spaces and involutive Lie algebra}
%--------------------------------------------------------------------

Let $(M,\omega,s)$ be a symplectic symmetric space; we associate an involutive Lie algebra $(\lG,\sigma)$ in the following way (we omit some non trivial proofs). Let $o\in M$, $G$ the transvection group and $H$, the stabiliser of $o$ in $\Aut(M,\omega,s)$ and $K=G\cap H$. We consider the map
\begin{equation}
	\begin{aligned}
		\tilde\sigma\colon \Aut(M,\omega,s)&\to \Aut(M,\omega,s) \\
		\tilde\sigma(g)&=s_o\circ g\circ s_o. 
	\end{aligned}
\end{equation}

Let $\mG$ be the Lie algebra of the group $G$ and $\dpt{\sigma}{\mG}{\mG}$ the induced involutive automorphism from $\tilde\sigma$. Now, $(\mG,\sigma)$ is an involutive Lie algebra. We have a natural projection $\dpt{\pi}{G}{M}$ because $H$ stabilises $o$, so that $K=G\cap H$ is the stabiliser of $o$ in $\Aut(M,\omega,s)$ which is transitive on $M$. Then $M=G/K$ as homogeneous spaces. One can see that $(\mG,\sigma,\pi^*(\omega_o))$ is a symplectic triple.

The precise proposition is the following.

\begin{proposition}
There exists a bijection between symplectic simply connected symmetric spaces and symplectic triples. This bijection is given up to isomorphism.
\end{proposition}

\subsection{Symmetric spaces and coadjoint orbits}
%-------------------------------------------------

We are now going to describe $(M,\omega,s)$ as coadjoint orbit on $\mG^*$. When a Lie group $G$ of symplectomorphism acts on a symplectic manifold $(M,\omega)$, we say that the action is weakly Hamiltonian\index{Hamiltonian!action} if there exists $\dpt{\mu_X}{M}{\eC}$ such that $i(X^*)\omega=d\mu_X$. If $\dpt{\mu}{\mG}{ C^{\infty}(M)}$ is a Lie algebra homomorphism, we say that the action is Hamiltonian and we usually write $\lambda$ instead of $\mu$.

\begin{proposition}
Let $(\mG,\sigma,\Omega)$ be a simple triple, $(M,\omega,s)$ the associated symmetric simply connected symplectic space and $G$, the transvection group. Then

\begin{enumerate}
\item The action of the transvection group on $M$ is Hamiltonian if and only if there exists a $\xi\in\mG^*$ with $\Omega=\delta\xi$ for the Chevalley cohomology\index{Chevalley!cohomology}.
\item In this case, $(M,\omega,s)$ is a $G$-equivariant symplectic covering of the coadjoint orbit of $\xi$ in $\mG^*$.
\end{enumerate}

\end{proposition}

\begin{definition}
A \defe{symmetric symplectic space}{symmetric!symplectic space} is a triple $(M,\omega,s)$ where
\begin{itemize}
\item $M$ is a connected smooth ($\Cinf$) manifold,
\item $\omega$ is a symplectic form on $M$,
\item $\dpt{s}{M\times M}{M}$ is a smooth map which we write with the notation $s_x(y):=s(x,y)$.
\end{itemize}
These elements must satisfy the following conditions:
 \begin{enumerate}
 \item $\forall x\in M$, $s_x$ is an involutive symplectic diffeomorphism of $(M,\omega)$ which is called the \defe{symmetry}{symmetry} at $x$,
 \item $\forall x\in M$, $x$ is an isolated fixed point of $s_x$,
 \item $\forall x,y\in M$, $s_xs_ys_x=s_{s_x(y)}$.
\end{enumerate}

\end{definition}


\begin{definition}
Two symplectic symmetric spaces $(M,\omega,s)$ and $(M',\omega',s')$  are \defe{isomorphic}{isomorphism!of symplectic!symmetric spaces} if there exists a symplectic diffeomorphism $\dpt{\varphi}{(M,\omega)}{(M',\omega')}$ such that 
\begin{equation}
  \varphi\circ s_x=s'_{\varphi(x)}\circ\varphi.
\end{equation}

\end{definition}


\begin{definition}
An \defe{exact triple}{exact triple} is a triple $(\mG,\sigma,\Omega)$ such that

\begin{enumerate}
\item $(\mG,\sigma)$ is an involutive Lie algebra with $[\mP,\mP]=\mK$ if $\mG=\mK\oplus\mP$ is the decomposition of $\mG$ with respect to $\sigma$.

\item $\Omega$ is a Chevalley $2$-coboundary such that $i(\mK)\Omega=0$ and $\Omega_{\mP\times\mP}$ is symplectic.

\end{enumerate}

\end{definition}

From definition, there exists a $\xi\in\mG^*$ for which $\Omega=\delta\xi$. We can choose it in such a way that $\xi(\mP)=0$; in this case we say that $\xi\in\mK^*$ by abuse of notation. Indeed, put $\Omega=\delta\xi$ with $\xi=\xi'+\eta'$ where $\xi'\in\mK^*$ and $\eta'\in\mP^*$. If we consider $k\in\mK$ and $B=B_k+B_p\in\mG$, using $i(\mK)\Omega=0$, we find
\[ 
  0=\Omega(k,B)=-\xi'[k,B_k]-\eta'[k,B_p].
\]
Taking $B_k=0$, we find $\eta'[\mK,\mP]=0$ while with $B_p=0$, we find $\eta'[\mK,\mK]=0$. Moreover $\eta'[\mP,\mP]=\eta'\mK=0$. Then an acceptable $\eta'$ must satisfy $\eta'[\mG,\mG]=0$, so that
\[ 
  \Omega[A,B]=-\xi'[A,B]
\]
which proves that the $\eta'$ part of $\xi$ has no importance; we can choose it as zero.


Let $(\mG,\sigma)$ be an involutive Lie algebra associated with a triple $(M,\omega,s)$ with transvection group $G$. If $(\mG,\sigma,\Omega)$ is exact, $\mZ(\mG)\subset \mK$ because $[Z,p]=0$ for all $p\in\mP$ whenever $Z\in\mP\cap\mZ(\mG)$. Then $\Omega(Z,p)=0$ for all $p\in \mP$ which is not possible from non degeneracy of $\Omega$.

\subsection{Elementary solvable symmetric spaces}
%-----------------------------------------------

Let $(M,\omega,s)$ a symmetric space with associated triple $(\mG,\sigma,\Omega)$. The space $M$ is \defe{elementary solvable}{elementary!solvable!exact triple} if

\begin{enumerate}
\item $\mG$ is a split extension  (see subsection \ref{subsec:semi_Lie}) of two abelian algebras $\mA$ and $\mB$,
\item the automorphism $\sigma$ preserves the decomposition $\mG=\mB\oplus\mA$.
\end{enumerate}
Since $\mK=[\mP,\mP]$, we have
\[ 
  \mA\cap\mK\subset\mA\cap[\mG,\mG]=0.
\]
Indeed, let $\dpt{\rho}{\mA}{\Der\mB}$ be the split homomorphism; the commutator on the split extension is defined by
\[ 
  [A,B]=\rho(A)B\in\mB.
\]
Then $[\mG,\mG]\subset\mB$. All this shows that $\mA\subset\mP$. So there exists a $\mL\subset\mP$ such that $\mB=\mK\oplus\mL$. Let us show that $\mL$ is abelian.
\[ 
  0=[\mB,\mB]=[\mK,\mK]+[\mK,\mL]+[\mL,\mK]+[\mL,\mL].
\]
The three first terms are in $\mP$ while the last one is included in $\mK$. The identical annihilation of the sum imposes $[\mL,\mL]=0$.

\subsection{Mid-point map}
%-------------------------

Let us now take an ESET $(\mG,\sigma,\Omega)$ and its corresponding ESSS $(M,\omega,s)$. There exists a $\xi\in\mG^*$ such that $\Omega=\delta\xi$ and we define
\begin{equation}
\begin{aligned}
 \zeta\colon \mA\times\mL&\to \eR \\ 
(a,l)&\mapsto \xi(\sinh(a)l) 
\end{aligned}
\end{equation}
where $\sinh(a)l$ has to be understood as $\frac{ 1 }{2}( e^{\rho(a)}- e^{-\rho(a)})l$ with $\rho(a)\in\End(\mB)$ being the splitting homomorphism of \eqref{EqSplitmGABab}. 
\begin{proposition}
Let $(M,\omega,s)$ be a ESSS and $\omega=\Omega=\delta\xi$ the symplectic form of the corresponding ESET. The \defe{mid-point map}{mid-point map} $M\to M$, $x\mapsto x/2$ defined by 
\[ 
  s_{x/2}o=x
\]
is globally defined if and only if $\phi$ is a diffeomorphism.
\end{proposition}
Notice that the affirmation $\omega=\Omega=\delta\xi$ means that one has a symplectic form $\omega\colon \mA\times\mL\to \eR$,
\[ 
  \omega(a,l)=\Omega(a,l)=-\xi\big( [a,l] \big).
\]


\subsection{Kähler structures}
%-------------------------------


Here, $M$ denotes a connected smooth manifold. An \defe{almost complex structure}{almost!complex structure} on $M$ is a $(1,1)$ tensor field $J$ such that $\forall X\in\cvec(M)$,
\[
   (J\circ J)X=-X
\]
The tensor field $J$ is a \defe{complex structure}{complex structure} when moreover it satisfies the \emph{integrability condition}: $\forall X,Y\in\cvec(M)$,
\begin{equation}  \label{DefComplStruct}
   N(X,Y):=[X,Y]+J[JX,Y]+J[X,JY]-[JX,JY]=0.
\end{equation}
If $M$ posses an almost complex structure $J$ and a Riemannian metric $g$, we say that the metric is \defe{hermitian}{hermitian!metric} when 
\[
   g(JX,JY)=g(X,Y).
\]

Notice that a symmetric space must be hermitian (definition \ref{def:hermitien}), hence equation \eqref{eq:herm_3}, implies the integrability condition. 

\begin{definition}
If one has an almost complex structure with an hermitian metric such that $\nabla J=0$, then $(M,J,g)$ is a \defe{Kähler manifold}{kähler!manifold}.
\end{definition}

\begin{remark}
$\nabla J=0$ reads $\forall X,Y\in M$,
\[
    (\nabla_XJ)(Y)=\nabla_X(JY)-J(\nabla_XY)=0.
\]
\end{remark}

\begin{remark}
By ``$\nabla$''\ we mean the Levi-Civita connection for $g$. In particular it is torsion free:
\[
   \nabla_XY-\nabla_YX=[X,Y].
\]
\end{remark}

\begin{lemma}
If $(M,g,J)$ is a Kähler manifold, then $J$ is integrable.
\end{lemma}

\begin{proof}
From the formula $\nabla_Z(JY)=(\nabla_ZJ)Y+J\nabla_ZY$ and the fact that $\nabla J=0$, we know that 
\begin{equation}\label{eq:inter_1}
  \nabla_Z(JY)=J(\nabla_ZY),
\end{equation}
while the torsion-free condition for $\nabla$ gives
\begin{equation}\label{eq:inter_2}
\nabla_XY-\nabla_YX=[X,Y].
\end{equation}
With these two, we find $\nabla_Z(JY)-\nabla_Y(JZ)=J\nabla_ZY-J\nabla_YZ=J[Z,Y]$.  Writing it with $JZ$ instead of $Z$,
\[
   \nabla_{JZ}(JY)+\nabla_Y(Z)=J[JZ,Y].
\]
The anti-symmetric part of this equation gives
\[
   \nabla_{JZ}(JY)+\nabla_YZ-J[JZ,Y]-\nabla_{JY}(JZ)-\nabla_ZY+J[JY,Z].
\]
Using \eqref{eq:inter_1} and \eqref{eq:inter_2}, one finds the thesis.

\end{proof}

When $(M,g,J)$ is a Kähler manifold, one defines the \defe{Kähler $2$-form}{kähler!$2$-form} by
\begin{equation}
\omega(X,Y):=g(X,JY).
\end{equation}

\begin{proposition}
The Kähler $2$-form is a symplectic structure on $M$.
\end{proposition}

\begin{proof}
Since $g$ is nondegenerate and $JX=0$ implies $X=0$, it is clear that $\omega$ is nondegenerate. The antisymmetry of $\omega$ is because the metric is hermitian. The only point is to see that $d\omega=0$.

From \eqref{eq:d_omega_nabla} which gives $d\omega$ in terms of $\nabla\omega$, we see that we just have to prove that $\nabla\omega=0$. By definition,
\begin{align*}
(\nabla_Z\omega)(X,Y)&=Z(\omega(X,Y))-\omega(\nabla_ZX,Y)-\omega(X,\nabla_ZY)\\
                     &=Zg(X,JY)-g(\nabla_ZX,JY)-g(X,J\nabla_ZY).\\
                     &=(\nabla_Zg)(X,JY)=0
\end{align*}		     
because the vanishing of $\nabla J$ implies that $J(\nabla_ZY)=\nabla_Z(JY)$.
\end{proof}

\subsection{Symplectic structure on the Iwasawa component}\index{symplectic!on $R$}
%-------------------------------------

The Iwasawa theorem gives us a global diffeomorphism between $R=AN$ and $M=G/K$ by $\dpt{\varphi}{R}{G/K}$, $\varphi(an)=[an]$. But one has a symplectic form on $M$ : $\omega^M_x(X,Y)=g_x(JX,Y)$. So, $R$ has also a symplectic form defined by, $\forall\,X,Y\in T_{an}R$,
\begin{equation}
\omega^R=\varphi^*\omega^M,
\end{equation}
or more explicitly:
\[
  \omega^R_{an}(X,Y)=\omega^M_{\varphi(an)}(d\varphi_{an}X,d\varphi_{an}Y).
\]

\begin{proposition}
This symplectic form is $R$-invariant under the left action; in other words,  $\forall r\in R$,
\begin{equation}
\omega_{ran}^R\Big(  (dL_r)_{an}X,(dL_r)_{an}Y  \Big)=\omega^R_{an}(X,Y).
\end{equation}

\end{proposition}

\begin{proof}
For a $r\in R$, we want to looks at
\begin{equation}
\omega^R_{ran}(dL_rX,dL_rY)=\omega^M_{[ran]}(d\varphi_{ran}dL_rX,d\varphi_{ran}dL_rY)\\                          
\end{equation}

But we know the invariance of $\omega^M$:
\[
  \omega^M_{[hg]}(dL_hX,dL_hY)=\omega^M_{[g]}(X,Y),
\]
Now, let us show that $d\varphi_{ran}dL_rX=dL_rd\varphi_{an}X$. For this, we consider a path which gives $X\in T_{an}R$: $X(t)\in R$, $X(0)=an$. So,
\begin{equation}
  d\varphi_{ran}(dL_r)_{an}X=\Dsdd{[rX(t)]}{t}{0}
                        =\Dsdd{L_r[X(t)]}{t}{0}
			=(dL_r)_{[an]}d\varphi_{an}X.
\end{equation}
Finally,
\begin{equation}
\begin{split}
\omega_{ran}^R(dL_rX,dL_rY)&=\omega^M_{[ran]}(d\varphi_{ran}(dL_r)_{an}X,d\varphi_{ran}(dL_r)_{an}Y)\\
                           &=\omega^M_{[ran]}( (dL_r)_{[an]}d\varphi_{an}X,(dL_r)_{[an]}d\varphi_{an}Y )\\
			   &=\omega^M_{[an]}(d\varphi_{an}X,d\varphi_{an}Y)\\
			   &=\omega^R_{an}(X,Y).
\end{split}
\end{equation}

\end{proof}


\subsection{Iwasawa coordinates}
%-------------------------------

We consider $M=G/K$, an hermitian symmetric space (irreducible of non-compact type\quext{je ne sais pas ce que \c{c}a veut dire, mais je ne sais pas non plus o\`u on l'utilise. (p. 301 d'Helgason) }). Let us consider a $Z\in\mZ(\mK)$ as before: $\delta B(Z,.)|_{\mP\times\mP}$ is a $\mK$-invariant $2$-form on $\mP$. 
There are some remarkable spaces: $\mR=\mA\oplus\mN$, the Lie algebra of $R=AN$; $\mO=\Ad(G)Z\subset\mG$. We consider the following diffeomorphism:
\begin{subequations}
\begin{align}
&\dpt{\mI}{\mA\oplus\mN}{R},  &\mI(a,n)&=e^ae^n,\\
&\dpt{\varphi}{R}{M},             &\varphi(an)&=[an],\\
&\dpt{\phi}{\mR}{\mO},        &\phi(r)&=\Ad(\mI(r))Z,\\
&\dpt{\lambda}{\mO}{M},       &\lambda(\Ad(g)Z)&=[g].
\end{align}
\end{subequations}
Note that $\mR=T_eR=\mA\oplus\mN=T_r\mR$ where $\mR=T_r\mR$ is a standard identification of vector spaces. The symplectic forms on these spaces are naturally defined by
\begin{subequations}
\begin{align}
   \omega^R&={\varphi^*}^{-1}\omega^M\\
   \omega^{\mR}&=\mI^*\omega^R\\
   \omega^{\mO}&=\lambda^*\omega^R
\end{align}
\end{subequations}
%
By the way, the diffeomorphism $\mI$ is called the \defe{Iwasawa coordinates}{Iwasawa!coordinates}.

\begin{proposition}
The map $\phi$ is bijective.
\end{proposition}

\begin{proof}
For the surjective condition, we have to obtain $\Ad(ank)Z$ under the form $\Ad(\mI(A,N))Z$. For this, remark that one can find $K\in\mK$, $A\in\mA$, $N\in\mN$  such that $k=e^K$, $a=e^A$,$n=e^N$, then
\[
  \Ad(e^Ae^N)Z=\Ad(e^Ae^Ne^K)Z=\Ad(ank)Z.
\]

In order to see the injective condition, let us consider $r,r'\in\mA\oplus\mN$ such that
\[
  \Ad(\mI(r))Z=\Ad(\mI(r'))Z.
\]
Then, $\Ad(\mI(r'))^{-1}\circ\Ad(\mI(r))=\id$. This makes $\Ad(e^{-N'}e^{-A'}e^Ae^N)=id$, so that
\[
   e^{-N'}e^{-A'}=\left(e^Ae^N\right)^{-1},
\]
but $\exp$ is a diffeomorphism, then $(A,N)=(A',N')$.

\end{proof}

\begin{lemma}\label{lem:om_O_om_R}
The symplectic forms $\omega^{\mR}$ and $\omega^{\mO}$ are related by
\begin{equation}
\omega^{\mO}=(\phi^{-1})^*\omega^{\mR}.
\end{equation}
\end{lemma}

\begin{proof}
The definitions make that
\begin{equation}
  (\phi^{-1})^*\omega^{\mR}=(\phi^{-1})^*\mI^*\omega^R
                          =(\phi^{-1})^*\mI^*\varphi^*\omega^M,
\end{equation}
so that we just need to see that $\varphi\circ\mI\circ\phi^{-1}=\lambda$. This is true because for any $g\in K$,
\[
   \varphi\circ\mI\circ\phi^{-1}(\Ad(g)Z)=\varphi\circ\mI(\mI^{-1}(g))=\varphi(g)=[g].
\]

\end{proof}

Now, consider $u\in T_r\mR$, with $r=a+n\in\mA\oplus\mN$, and (just for fun) let us compute $d\phi_r(u)$. In the following computation, $u_A$ and $u_N$ denotes the unit vectors in the direction of $\mA$ and $\mN$.
\begin{equation}
\begin{split}
   d\phi_r(u)&=\Dsdd{  \Ad(e^{a+tu_A}e^{n+tu_N})Z  }{t}{0}\\ 
             &=\Dsdd{ \Ad(e^{tu_A})\Ad(e^{an})Z  }{t}{0}\\
             &\quad+\Ad(e^a)\Dsdd{  \Ad(e^n)\Ad(e^{-n})\Ad(e^{n+tu_N})Z  }{t}{0}\\
	     &=-(u_A^*)_{\phi(r)}\\
	     &\quad+\Ad(e^ae^n)\Dsdd{ \Ad(e^{-n})\Ad( e^{n+tu_N} )Z  }{t}{0}\\
	     &=-(u_A^*)_{\phi(r)}+\Ad(\mI(r))\Dsdd{  \Ad(e^{ CBH(-n,n+tu_N) })Z  }{t}{0},
\end{split}
\end{equation}
where $CBH$ denote the \href{http://en.wikipedia.org/wiki/Baker-Campbell-Hausdorff_formula}{Campbell-Baker-Hausdorff}\index{Campbell-Baker-Hausdorff formula} function defined by
\[
   e^xe^y=e^{CBH(x,y)}.
\]
One maybe knows the formula
\begin{equation}
\Dsdd{  CBH(-n,n+tu_N)  }{t}{0}=F(\ad(n))u_N,
\end{equation}
where $F(\ad(n))$ is defined by the expansion of 
\[
F(z)=\frac{1-e^{-z}}{z}
\]
for $z\in\eC$\quext{Il faut encore aller voir dans Duitsermaat les tenants et aboutissants de ce truc.}. Finally, 
\begin{equation}
d\phi_r(u)=-(u_A^*)_{\phi(r)}-\left(  \Ad(\mI(r))F(\ad(n))u_N  \right)^*_{\phi(r)}.
\end{equation}

Now, remark that $\Ad(e^a)|_{\mA}=id|_{\mA}$ because $\ad a|_{\mA}=0$ ($\mA$ is abelian) and 
the famous lemma \ref{Ad_e}.

The \underline{proposition 1.1 page 5 de BM}\quext{\`A faire\ldots} makes $\omega_x^{\mO}(X^*,Y^*)=-B(x,[X,Y])$ for $x\in\mO$, $X$, $Y\in\mG$

The lemma \ref{lem:om_O_om_R} gives us immediately 
\[
   (\mI^*\omega^R)_r(u,v)=(\phi^*\omega^{\mO})_r(u,v).
\]
\subsection{Summary of the construction}
%---------------------------------------

We pick\quext{L'existence de ce $\mK$ contre-dit ce que je dis quand une autre question à propos du type non-compact} $Z\in\mZ(\mK)$. Then one defines
\[ 
  J=\ad(Z)|_{\mP}
\]
and 
\[ 
  \omega^M(X,y)=
\begin{cases}
 B(JX,Y)&\text{If $X,Y\in\mP$}\\
	0&\text{if $X$ or $Y$ belongs to $\mK$}.
\end{cases}
\]
The maps $\mI$, $\varphi$, $\phi$ and $\lambda$ between spaces $R$, $\mA\oplus\mN$, $M$, $\mR$ and $\mO$ are designed to propagate the symplectic form from $\omega^M$ to $\omega^{\mR}$, $\omega^R$, $\omega^{\mO}$. The group $R$ acts on each of these spaces and in particular on $\mO$ by the adjoint action. One can prove that $\omega^{\mO}:=\lambda^*(\varphi^{-1})^*\omega^M$ is
\[ 
  \omega^{\mO}_x(X^*,Y^*)=-B(x,[X,Y])
\]
for all $X$, $Y\in\mR$. In the whole construction, $\sigma$ is the Cartan involution which gives the decomposition
\[ 
  \sigma=\id|_{\mK}\oplus(-\id)|_{\mP}.
\]
Therefore $\sigma E=-E$ because $E\in\mN\subset\mP$.

The Lie algebra $\mG$ possesses two roots: $\alpha$ and $2\alpha$, so we decompose it as\quext{Le fait d'être de type non compact est peut-être l'absence de composante $K$ pour l'Iwasawa, qu'en penses-tu ?} 
\[ 
  \mG=\mA\oplus\mG_{\alpha}\oplus\mG_{2\alpha}.
\]
We pick $A\in\mA$, $y\in\mG_{\alpha}$ and $E\in\mG_{2\alpha}$. For example, if $B\in \mA$, $[B,y]=\alpha(A)y$.

\subsection{Continuation}
%------------------------

\begin{proposition}
Let $M=G/K$ be an hermitian irreducible symmetric space of non compact type. We suppose that $\dim\mP\geq 4$. We consider the action $\dpt{ \tau }{  \mR\times\mR  }{ \mR }$,
\[ 
  \tau_g(X)=\mI^{-1}(g\mI(X)).
\]
This action is Hamiltonian for the constant symplectic structure $\Omega$ on $\mR$ and the dual momentum maps are given by
\begin{subequations}
\begin{align}
\lambda_A(X)&=2\alpha(A)B(\sigma A,E)n_E&&\text{($A\in \mA$)}\\
\lambda_y(X)&= e^{-\alpha(a)}\Omega(n,y)&&\text{($y\in\mG_{\alpha}$)}\\
\lambda_E(X)&= e^{-Z\alpha(a)}B(\sigma E,E)
\end{align}
\end{subequations}
where $X=(a,n)$ and $n=n_{\alpha}+n_EE$ for the decomposition $\mN=\mG_{\alpha}\oplus\eR E$.

As a consequence, the Moyal star product is $R$-covariant.

\end{proposition}

\begin{proof}
From equation \eqref{eq_XlambdaYs}, we have to prove the identities 
\[ 
  \{ \lambda_X,\lambda_Y \}=X^*(\lambda_Y).
\]
We begin by proving the identity
\[ 
  \{ \lambda_A,\lambda_y \}(L)=A^*_L(\lambda_y)
\]
where $L=(a',n')\in\mR$. In these coordinate, we suppose without loss of generality that $A=(1,0)$. As usual, we will use some abuse of notation as $\mI(L)= e^{a'} e^{n'}= e^{a'A} e^{n'}$;
\begin{equation}
\begin{split}
  A^*_L(\lambda_y)&=\Dsdd{ \lambda_y(\tau_{ e^{-tA}}L) }{t}{0}\\
		&=\Dsdd{ (\lambda_y\circ\mI ^{-1}) e^{(a'-t)A} e^{n'} }{t}{0}\\
		&=\Dsdd{ \lambda_y\big( (a'-t),n' \big) }{t}{0}\\
		&=\Dsdd{  e^{-\alpha(a'-t)}\Omega(n',y) }{t}{0}\\
		&=\Dsdd{  e^{(t-a')\alpha(A)} }{t}{0}\Omega(n',y)\\
		&=\alpha(A) e^{-\alpha(a')\Omega(n',y)}.
\end{split}
\end{equation}
In this computation, we used the fact that $\alpha(a'-t)=(a'-t)\alpha(A)$.
On the other hand, 
\[ 
  \lambda_{[A,y]}(L)=\alpha(A)\lambda_y(L)=\alpha(A) e^{-\alpha(a')\Omega(n',y)}.
\]
This concludes the first check. The check that $\{ \lambda_A,\lambda_E \}=\lambda_{[A,E]}$ is the same, using the fact that $E\in\mG_{2\alpha}$ and thus that $[A,E]=2\alpha(A)E$. For the third, $[y,E]=0$ therefore, we have to prove that $\| \lambda_y,\lambda_E \|=0$. We have
\[ 
  \| \lambda_y,\lambda_E \|(L)=y^*_L(\lambda_E)=\Dsdd{ (\lambda_E\circ\mI^{-1})\big(  e^{-ty} e^{a'} e^{n'} \big) }{t}{0}.
\]
The problem is to commute $ e^{-ty}$ with $ e^{a'}$. Since the $t$ will always stands in front of $y$ and $\lambda_E$ doesn't depends on $y$, the derivative is zero\quext{Je ne crois pas que cette justification soit juste.}.
cs
\[ 
\begin{split}
  A^*_o&=\Dsdd{  e^{-tA}\cdot(0,0) }{t}{0}\\
	&=-A.
\end{split}  
\]
Since $d\mI=\id$, 
\[ 
\begin{split}
\omega^{\mR}(A^*,X)=&\omega^{\mR}(A,x_yy+x_EE)\\
		&=-\omega^{R}(A,x_yy+x_EE),
\end{split}  
\]
but for any element in $\mA\oplus\mN$, via the identification $\mR=[\mA\oplus\mN]$ (the additive class),
\[ 
\begin{split}
d\lambda^{-1}A&=\Dsdd{ \lambda^{-1}[ e^{tA}] }{t}{0}\\
		&=\Dsdd{ \Ad( e^{tA})Z }{t}{0}\\
		&=-A^*.
\end{split}  
\]
Thus 
\begin{equation} \label{eq_omeOmO}
\begin{split}
\omega^{\mR}(A^*,X)&=-\omega^{\mO}(d\lambda^{-1}A,d\lambda^{-1}(x_yy+x_EE))\\
		&=-\omega^{\mO}(A^*,x_yy^*+x_EE^*)
\end{split}  
\end{equation}
where $A^*$, $y^*$ and $E^*$ are taken in the sense of the adjoint action of $R$ on $\mO$.

Now we prove that $\lambda_A$ is well a dual momentum map. For this, we choose $X=x_AA+x_yy+x_EE\in\mR$ and we check the identity $d\lambda_AX=\omega^{\mR}(A^*,X)$ where $A^*$ stands for the given action of $R$ on $\mR$.

A question arise: at which point is taken $\omega^{\mO}$ in equation \eqref{eq_omeOmO} ? Since we compute $\omega^{\mR}$ at identity, we compute $\omega^{\mO}$ at $Z$. So
 \[ 
\begin{split}
  \omega^{\mR}(A^*,X)&=-\omega^{\mO}_Z(A^*,x_yy^*+x_EE^*)\\
		&=-B(Z,[A,y+E])\\
		&=-\alpha(A)B(Z,y)-2\alpha(A)B(Z,E).
\end{split}  
\]
Here, we have to remark that it is not zero because $\mN$ is not included in $\mP$, but is transverse.

\end{proof}
%+++++++++++++++++++++++++++++++++++++++++++++++++++++++++++++++++++++++++++++++++++++++++++++++++++++++++++++++++++++++++++
\section{Elementary normal symplectic spaces}
%+++++++++++++++++++++++++++++++++++++++++++++++++++++++++++++++++++++++++++++++++++++++++++++++++++++++++++++++++++++++++++
\label{SecElemNormSymplSpace}

This section is closely related to the Pyatetskii-Shapiro theory treated in section \ref{SecPyateskiiShapiro}. See \cite{QuantifKhalerian} as reference.

Let $(V,\Omega)$ be a symplectic real vector space of dimension $2n$. We build the \defe{Heisenberg algebra}{heisenberg!algebra} by
\begin{equation}
	\pH=V\oplus \eR E
\end{equation}
with the relation $[v,v']=\Omega(v,v')E$. Now we consider a new element $H$ and $\lA=\eR H$ and the split extension
\begin{equation}
	0\to\pH\to\lS\to\lA\to 0
\end{equation}
where $\lS=\lA\oplus_{\rho}\pH$ and $\rho\colon \lA\to \Der(\pH)$ is given by
\begin{equation}
	\rho(H)(v\oplus tE)=[H,v\oplus tE]=v\oplus 2tE.
\end{equation}
We denote by $(a,v,t)$ an element of $\lS$, that is
\begin{equation}
	(a,v,t)=aH+ v+tE
\end{equation}
with $a,t\in\eR$ and $v\in V$. We consider the $2$-form
\begin{equation}
	\omega^{\sS}=2da\wedge dt+\Omega,
\end{equation}
the pair $(\lS,\omega^{\lS})$ is said to be a \defe{normal elementary symplectic algebra}{elementary!normal symplectic algebra}. We denote by $(\eS,\omega)$ the associated connected simply connected Lie group.

\begin{proposition}		\label{Prop2807DescSMdarboux}
	With the previous notations we have
	\begin{enumerate}

		\item
			The map
			\begin{equation}
				\begin{aligned}
					(\lS,\omega^{\lS})&\to (\eS,\omega) \\
					(a,v,t)&\mapsto  e^{aH} e^{v+tE}
				\end{aligned}
			\end{equation}
			is a global Darboux chart (in particular it is a global diffeomorphism).
			
			By this diffeomorphism we identify $\lS$ and $\eS$, i.e. we will denote by $(a,v,t)$ the element $ e^{aH}e^{v+tE}\in\eS$ as well as the element $aH+v+tE\in\lS$.
		\item
			Within the coordinates $(a,v,t)$ the group law is given by
			\begin{equation}
				(a,v,t)\cdot (a',v',t')=
				(a+a', e^{-a'}v+v', e^{-2a,}t+t'+\frac{ 1 }{2} e^{-a'}\Omega(v,v')).
			\end{equation}
		\item
			If we define
			\begin{equation}		\label{Eq1807StuctSymM}
				s_{(a,v,t)}(a',v',t')=
				\big(2a-a',2\cosh(a-a')v-v',2\cosh(2(a-a'))t+\Omega(v,v')\sinh(a-a')-t'\big),
			\end{equation}
			the space $\eM=(\eS,\omega,\lS)$ becomes a symplectic symmetric space.

		\item
			The structure of symplectic symmetric space  is preserved by the left translations. In other words, for every $x\in\eS$ we have $L_x\in\Aut(\eM)$. And the subgroup $\{ L_x\tq x\in\eS \}$ acts simply transitively on $\eM$.

		\item
			We have
			\begin{equation}
				\SP(V,\Omega)\subset\Aut(\eM)
			\end{equation}
			if we define $g\cdot(a,v,t)=(a,g\cdot v,t)$ for every $g\in\SP(V,\Omega)$.
	\end{enumerate}
\end{proposition}
