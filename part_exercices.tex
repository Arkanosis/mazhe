% This is part of Mes notes de mathématique
% Copyright (c) 2011-2012,2014
%   Laurent Claessens
% See the file fdl-1.3.txt for copying conditions.

\chapter{Géométrie analytique (Besançon)}

%+++++++++++++++++++++++++++++++++++++++++++++++++++++++++++++++++++++++++++++++++++++++++++++++++++++++++++++++++++++++++++
\section{Espaces vectoriels normés}
%+++++++++++++++++++++++++++++++++++++++++++++++++++++++++++++++++++++++++++++++++++++++++++++++++++++++++++++++++++++++++++
%---------------------------------------------------------------------------------------------------------------------------
\subsection{Normes}
%---------------------------------------------------------------------------------------------------------------------------

\Exo{EspVectoNorme0001}
\Exo{GeomAnal-0040}
\Exo{GeomAnal-0041}
\Exo{GeomAnal-0042}
\Exo{GeomAnal-0043}
\Exo{GeomAnal-0044}

%---------------------------------------------------------------------------------------------------------------------------
\subsection{Topologie}
%---------------------------------------------------------------------------------------------------------------------------

\Exo{EspVectoNorme0002}
\Exo{EspVectoNorme0008}
\Exo{EspVectoNorme0003}
\Exo{EspVectoNorme0004}
\Exo{EspVectoNorme0005}
\Exo{EspVectoNorme0006}
\Exo{EspVectoNorme0007}
\Exo{EspVectoNorme0009}



% Pour des raisons de compatibilité avec «Mes notes de mathématique», la section «Système de coordonnées» est supprimée
% mars 2012
%+++++++++++++++++++++++++++++++++++++++++++++++++++++++++++++++++++++++++++++++++++++++++++++++++++++++++++++++++++++++++++
%\section{Systèmes de coordonnées}
%+++++++++++++++++++++++++++++++++++++++++++++++++++++++++++++++++++++++++++++++++++++++++++++++++++++++++++++++++++++++++++

%\Exo{OutilsMath-0002}
%\Exo{OutilsMath-0003}  % cet exercice a la figure TriangleRectangle qu'il faut remettre dans la liste 
                        % si on on le remet.
%\Exo{OutilsMath-0005}  % Cet exerice est aussi dans Outils Math
%\Exo{OutilsMath-0006}
%\Exo{GeomAnal-0034}

%+++++++++++++++++++++++++++++++++++++++++++++++++++++++++++++++++++++++++++++++++++++++++++++++++++++++++++++++++++++++++++
\section{Courbes et surfaces}
%+++++++++++++++++++++++++++++++++++++++++++++++++++++++++++++++++++++++++++++++++++++++++++++++++++++++++++++++++++++++++++
\Exo{CourbesSurfaces0001}
\Exo{CourbesSurfaces0002}
\Exo{CourbesSurfaces0003}
%\Exo{CourbesSurfaces0004}
\Exo{CourbesSurfaces0005}
\Exo{CourbesSurfaces0006}
\Exo{CourbesSurfaces0007}
\Exo{CourbesSurfaces0008}
%\Exo{CourbesSurfaces0009}
%\Exo{CourbesSurfaces0010}
%\Exo{CourbesSurfaces0011}
%\Exo{CourbesSurfaces0012}
%\Exo{CourbesSurfaces0013}
%\Exo{CourbesSurfaces0014}
%\Exo{CourbesSurfaces0015}
%\Exo{CourbesSurfaces0016}
\Exo{GeomAnal-0019}


%\Exo{GeomAnal-0038}    % TODO : comprendre pourquoi cet exercice a été éliminé

%+++++++++++++++++++++++++++++++++++++++++++++++++++++++++++++++++++++++++++++++++++++++++++++++++++++++++++++++++++++++++++
\section{Limite et continuité}
%+++++++++++++++++++++++++++++++++++++++++++++++++++++++++++++++++++++++++++++++++++++++++++++++++++++++++++++++++++++++++++

\Exo{LimiteContinue0001}
\Exo{LimiteContinue0002}
\Exo{LimiteContinue0003}
\Exo{LimiteContinue0004}
\Exo{LimiteContinue0005}
\Exo{LimiteContinue0006}
\Exo{LimiteContinue0007}
\Exo{LimiteContinue0008}
\Exo{LimiteContinue0009}
\Exo{LimiteContinue0010}
\Exo{LimiteContinue0011}
\Exo{ExamDec2011-0002}




% \Exo{GeomAnal-0035}   % TODO : voir pourquoi on n'avait pas mis cet exercice.

%+++++++++++++++++++++++++++++++++++++++++++++++++++++++++++++++++++++++++++++++++++++++++++++++++++++++++++++++++++++++++++
\section{Calcul différentiel}
%+++++++++++++++++++++++++++++++++++++++++++++++++++++++++++++++++++++++++++++++++++++++++++++++++++++++++++++++++++++++++++
%---------------------------------------------------------------------------------------------------------------------------
\subsection{Dérivées partielles}
%---------------------------------------------------------------------------------------------------------------------------

\Exo{CalculDifferentiel0001}
\Exo{CalculDifferentiel0002}

%---------------------------------------------------------------------------------------------------------------------------
\subsection{Différentielles}
%---------------------------------------------------------------------------------------------------------------------------

\Exo{CalculDifferentiel0003}
\Exo{CalculDifferentiel0004}
\Exo{CalculDifferentiel0021}

%---------------------------------------------------------------------------------------------------------------------------
\subsection{Dérivée d'applications composées}
%---------------------------------------------------------------------------------------------------------------------------

\Exo{CalculDifferentiel0006}
\Exo{CalculDifferentiel0007}
\Exo{CalculDifferentiel0008}
\Exo{CalculDifferentiel0009}
\Exo{CalculDifferentiel0010}
\Exo{CalculDifferentiel0011}
\Exo{CalculDifferentiel0012}
\Exo{CalculDifferentiel0013}

\Exo{CalculDifferentiel0015}
\Exo{CalculDifferentiel0016}
\Exo{CalculDifferentiel0017}

\Exo{CalculDifferentiel0019}
\Exo{CalculDifferentiel0020}



%+++++++++++++++++++++++++++++++++++++++++++++++++++++++++++++++++++++++++++++++++++++++++++++++++++++++++++++++++++++++++++
\section{Intégrales multiples}
%+++++++++++++++++++++++++++++++++++++++++++++++++++++++++++++++++++++++++++++++++++++++++++++++++++++++++++++++++++++++++++
\Exo{IntegralesMultiples0001}
\Exo{IntegralesMultiples0002}
\Exo{IntegralesMultiples0003}
\Exo{IntegralesMultiples0004}
\Exo{IntegralesMultiples0005}
\Exo{IntegralesMultiples0006}
\Exo{IntegralesMultiples0007}
\Exo{IntegralesMultiples0008}
\Exo{IntegralesMultiples0009}
\Exo{IntegralesMultiples0010}
\Exo{IntegralesMultiples0011}
\Exo{IntegralesMultiples0012}
\Exo{IntegralesMultiples0013}
\Exo{GeomAnal-0050}
\Exo{IntegralesMultiples0014}
\Exo{IntegralesMultiples0015}
\Exo{IntegralesMultiples0016}
\Exo{IntegralesMultiples0017}
\Exo{IntegralesMultiples0018}
\Exo{IntegralesMultiples0019}
\Exo{IntegralesMultiples0020}
\Exo{IntegralesMultiples0021}
\Exo{IntegralesMultiples0022}
\Exo{IntegralesMultiples0023}
\Exo{IntegralesMultiples0024}
\Exo{IntegralesMultiples0025}
\Exo{IntegralesMultiples0026}
\Exo{IntegralesMultiples0027}
\Exo{IntegralesMultiples0028}
\Exo{IntegralesMultiples0029}
\Exo{IntegralesMultiples0030}





%+++++++++++++++++++++++++++++++++++++++++++++++++++++++++++++++++++++++++++++++++++++++++++++++++++++++++++++++++++++++++++
\section{Autres exercices}
%+++++++++++++++++++++++++++++++++++++++++++++++++++++++++++++++++++++++++++++++++++++++++++++++++++++++++++++++++++++++++++

Cette section contient entre autres des exercices donnés à des devoirs, interrogations et DS.

\Exo{devoir1-0001}
\Exo{devoir1-0004}
\Exo{devoir1-0005}
\Exo{devoir1-0006}


\Exo{devoir2-0002}
\Exo{devoir2-0003}
\Exo{devoir2-0004}
\Exo{devoir2-0005}
\Exo{devoir2-0006}
\Exo{devoir2-0007}
\Exo{devoir2-0008}
\Exo{devoir2-0009}


\Exo{CourbesSurfaces0004}
\Exo{devoir3-0002}
\Exo{CourbesSurfaces0009}
\Exo{devoir3-0004}



\Exo{controlecontinu0002}
\Exo{controlecontinu0005}
\Exo{controlecontinu0006}
\Exo{controlecontinu0008}
\Exo{controlecontinu0010}
\Exo{controlecontinu0011}


\Exo{controlecontinu0001}
\Exo{controlecontinu0003}
\Exo{controlecontinu0004}
\Exo{controlecontinu0007}
\Exo{controlecontinu0012}
\Exo{controlecontinu0013}


\Exo{GeomAnal-0015}
\Exo{GeomAnal-0016}
\Exo{GeomAnal-0018}
\Exo{GeomAnal-0021}
\Exo{GeomAnal-0023}
\Exo{GeomAnal-0027}


\Exo{GeomAnal-0017}
\Exo{GeomAnal-0020}
\Exo{GeomAnal-0022}
\Exo{GeomAnal-0024}
\Exo{GeomAnal-0025}
\Exo{GeomAnal-0026}


\Exo{DS2011-0001}
\Exo{DS2011-0002}
\Exo{DS2011-0003}
\Exo{DS2011-0004}


\Exo{GeomAnal-0045}
\Exo{GeomAnal-0046}
\Exo{GeomAnal-0047}
\Exo{GeomAnal-0048}
\Exo{GeomAnal-0049}


%+++++++++++++++++++++++++++++++++++++++++++++++++++++++++++++++++++++++++++++++++++++++++++++++++++++++++++++++++++++++++++
\section{Exercices pour aller plus loin}
%+++++++++++++++++++++++++++++++++++++++++++++++++++++++++++++++++++++++++++++++++++++++++++++++++++++++++++++++++++++++++++


\Exo{GeomAnal-0001}    % position 31124
\Exo{GeomAnal-0002}    % position 23657
\Exo{GeomAnal-0003}    % position 28183
\Exo{GeomAnal-0004}    % position 55702
\Exo{GeomAnal-0005}
\Exo{GeomAnal-0006}
\Exo{GeomAnal-0007}
\Exo{GeomAnal-0008}    % position 25804
\Exo{GeomAnal-0009}    % position 26329
\Exo{GeomAnal-0010}
\Exo{GeomAnal-0011}
\Exo{GeomAnal-0012}
\Exo{GeomAnal-0013}
\Exo{GeomAnal-0014}


\Exo{CalculDifferentiel0005}
\Exo{CalculDifferentiel0014}
\Exo{CalculDifferentiel0018}

\chapter{Exercices d'analyse (Bruxelles)}
% This is part of the Exercices et corrigés de mathématique générale.
% Copyright (C) 2009-2011
%   Laurent Claessens
% See the file fdl-1.3.txt for copying conditions.
%+++++++++++++++++++++++++++++++++++++++++++++++++++++++++++++++++++++++++++++++++++++++++++++++++++++++++++++++++++++++++++
					\section{Limites}
%+++++++++++++++++++++++++++++++++++++++++++++++++++++++++++++++++++++++++++++++++++++++++++++++++++++++++++++++++++++++++++

\Exo{General0010}
\Exo{General0011}
\Exo{0013}
\Exo{0017}
\Exo{0016}
\Exo{0024}

%+++++++++++++++++++++++++++++++++++++++++++++++++++++++++++++++++++++++++++++++++++++++++++++++++++++++++++++++++++++++++++
					\section{Dérivées et optimisation}
%+++++++++++++++++++++++++++++++++++++++++++++++++++++++++++++++++++++++++++++++++++++++++++++++++++++++++++++++++++++++++++

\Exo{General0012}
\Exo{General0013}
\Exo{General0014}
\Exo{General0015}
\Exo{General0016}

%+++++++++++++++++++++++++++++++++++++++++++++++++++++++++++++++++++++++++++++++++++++++++++++++++++++++++++++++++++++++++++
					\section{Primitives et intégration}
%+++++++++++++++++++++++++++++++++++++++++++++++++++++++++++++++++++++++++++++++++++++++++++++++++++++++++++++++++++++++++++

\Exo{General0017}
\Exo{General0018}
\Exo{General0019}
\Exo{General0020}
\Exo{General0021}
\Exo{General0022}
\Exo{General0023}
\Exo{General0024}
\Exo{General0025}
\Exo{General0026}
\Exo{General0027}

%---------------------------------------------------------------------------------------------------------------------------
					\subsection{Longueur d'un arc de courbe}
%---------------------------------------------------------------------------------------------------------------------------

\Exo{Inter0012}
\Exo{Inter0013}

%---------------------------------------------------------------------------------------------------------------------------
					\subsection{Aire d'une surface de révolution}
%---------------------------------------------------------------------------------------------------------------------------

\Exo{Inter0015}
\Exo{Inter0014}
\Exo{Inter0016}


% This is part of the Exercices et corrigés de mathématique générale.
% Copyright (C) 2009
%   Laurent Claessens
% See the file fdl-1.3.txt for copying conditions.
%+++++++++++++++++++++++++++++++++++++++++++++++++++++++++++++++++++++++++++++++++++++++++++++++++++++++++++++++++++++++++++
					\section{Équations différentielles}
%+++++++++++++++++++++++++++++++++++++++++++++++++++++++++++++++++++++++++++++++++++++++++++++++++++++++++++++++++++++++++++

%---------------------------------------------------------------------------------------------------------------------------
					\subsection{Équations à variables séparées}
%---------------------------------------------------------------------------------------------------------------------------

\Exo{EquaDiff0001}

%---------------------------------------------------------------------------------------------------------------------------
					\subsection{Équations homogènes}
%---------------------------------------------------------------------------------------------------------------------------

\Exo{EquaDiff0002}

%---------------------------------------------------------------------------------------------------------------------------
					\subsection{Équations linéaires}
%---------------------------------------------------------------------------------------------------------------------------

\Exo{EquaDiff0003}

%---------------------------------------------------------------------------------------------------------------------------
					\subsection{Problèmes divers}
%---------------------------------------------------------------------------------------------------------------------------

\Exo{EquaDiff0004}
\Exo{EquaDiff0005}
\Exo{EquaDiff0006}
\Exo{EquaDiff0007}
\Exo{EquaDiff0008}
\Exo{EquaDiff0009}

%---------------------------------------------------------------------------------------------------------------------------
					\subsection{Équations différentielles du second ordre}
%---------------------------------------------------------------------------------------------------------------------------

\Exo{EquaDiff0010}
\Exo{EquaDiff0011}
\Exo{EquaDiff0012}


\Exo{EquaDiff0013}
\Exo{EquaDiff0015}
\Exo{EquaDiff0014}
\Exo{EquaDiff0016}



% This is part of the Exercices et corrigés de mathématique générale.
% Copyright (C) 2009-2010
%   Laurent Claessens
% See the file fdl-1.3.txt for copying conditions.
%+++++++++++++++++++++++++++++++++++++++++++++++++++++++++++++++++++++++++++++++++++++++++++++++++++++++++++++++++++++++++++
					\section{Fonctions de deux variables réelles}
%+++++++++++++++++++++++++++++++++++++++++++++++++++++++++++++++++++++++++++++++++++++++++++++++++++++++++++++++++++++++++++

%---------------------------------------------------------------------------------------------------------------------------
					\subsection{Tracer}
%---------------------------------------------------------------------------------------------------------------------------

\Exo{FoncDeuxVar0001}

%---------------------------------------------------------------------------------------------------------------------------
\subsection{Limites à deux variables}
%---------------------------------------------------------------------------------------------------------------------------

\Exo{FoncDeuxVar0010}
\Exo{FoncDeuxVar0011}
\Exo{FoncDeuxVar0012}
\Exo{FoncDeuxVar0013}
\Exo{FoncDeuxVar0014}
\Exo{FoncDeuxVar0015}

\Exo{FoncDeuxVar0016}
\Exo{FoncDeuxVar0018}	

%---------------------------------------------------------------------------------------------------------------------------
\subsection{Dérivées partielles, différentielles totales}
%---------------------------------------------------------------------------------------------------------------------------
\Exo{FoncDeuxVar0002}
\Exo{FoncDeuxVar0003}

%---------------------------------------------------------------------------------------------------------------------------
\subsection{Différentiabilité, accroissements finis}
%---------------------------------------------------------------------------------------------------------------------------

\Exo{FoncDeuxVar0019}
\Exo{Maximisation-0001}
\Exo{FoncDeuxVar0026}
\Exo{FoncDeuxVar0021}
\Exo{FoncDeuxVar0022}
\Exo{FoncDeuxVar0023}
\Exo{DerrivePartielle-0000}
\Exo{DerrivePartielle-0001}
\Exo{FoncDeuxVar0025}

%---------------------------------------------------------------------------------------------------------------------------
\subsection{Plan tangent}
%---------------------------------------------------------------------------------------------------------------------------

\Exo{FoncDeuxVar0027}
\Exo{DerrivePartielle-0002}

%---------------------------------------------------------------------------------------------------------------------------
\subsection{Dérivées de fonctions composées}
%---------------------------------------------------------------------------------------------------------------------------

\Exo{DerrivePartielle-0003}
\Exo{FoncDeuxVar0017}
\Exo{DerrivePartielle-0004}
\Exo{DerrivePartielle-0005}
\Exo{FoncDeuxVar0030}
\Exo{FoncDeuxVar0024}
\Exo{FoncDeuxVar0020}
\Exo{DerrivePartielle-0006}

%---------------------------------------------------------------------------------------------------------------------------
\subsection{Dérivées de fonctions implicites}
%---------------------------------------------------------------------------------------------------------------------------
\Exo{FoncDeuxVar0004}
\Exo{FoncDeuxVar0005}
\Exo{FoncDeuxVar0006}
\Exo{FoncDeuxVar0007}

%---------------------------------------------------------------------------------------------------------------------------
\subsection{Extrema}
%---------------------------------------------------------------------------------------------------------------------------

\Exo{FoncDeuxVar0008}
\Exo{FoncDeuxVar0009}
\Exo{FoncDeuxVar0029}
\Exo{FoncDeuxVar0028}
\Exo{DerrivePartielle-0007}
\Exo{Maximisation-0002}
\Exo{DerrivePartielle-0008}
\Exo{DerrivePartielle-0009}
\Exo{DerrivePartielle-0010}
\Exo{Maximisation-0000}


% This is part of the Exercices et corrigés de mathématique générale.
% Copyright (C) 2009-2011
%   Laurent Claessens
% See the file fdl-1.3.txt for copying conditions.
Lorsque nous demandons d'étudier une fonction, nous demandons les éléments suivants : domaine de définition, croissance, extrema, points d'inflexion, asymptote et dessiner le graphe.

\section{TP-1}
\section{TP-2}
\section{TP-3}

\Exo{III-1}
\Exo{III-2}
\Exo{III-3}


\Exo{III-4}
\Exo{III-5}


\section{TP-4}

\Exo{TP40001}

\Exo{TP40002}
\Exo{TP40003}
\Exo{TP40004}
\Exo{TP40005}

%---------------------------------------------------------------------------------------------------------------------------
\subsection{Quelque fautes usuelles}
%---------------------------------------------------------------------------------------------------------------------------

Pour l'exercice \ref{exoTP40001}, les fautes les plus souvent commises sont
\begin{enumerate}

	\item
		$f'= e^{2x}$ implique $f=\frac{1}{ 2 } e^{x}$. Cela n'est pas vrai. La dérivée de $ e^{2x}$ est $2 e^{2x}$. Le $2$ reste dans l'exponentielle.

	\item
		Lorsqu'on intègre par partie, il faut aussi mettre les bornes pour le morceau qui n'est pas dans la nouvelle intégrale :
		\begin{equation}
			\int_a^b fg'=[fg]_a^b-\int_a^bf'g.
		\end{equation}
\end{enumerate}

Pour l'exercice \ref{exoTP40002}, les fautes les plus souvent commises sont
\begin{enumerate}

	\item
		Lorsqu'on a trouvé la solution générale $y_k(x)$ qui dépend du paramètre $k$ (ou $C$), il faut encore trouver la valeur du paramètre $k$ telle que $y_k(\pi)=0$.

\end{enumerate}


Pour l'exercice \ref{exoTP40003}, les fautes les plus souvent commises sont
\begin{enumerate}

	\item
		Ne pas oublier que $e^0=1$.

\end{enumerate}


\section{TP-5}
\Exo{TP50001}
\Exo{TP50002}
\Exo{TP50003}
\Exo{TP50004}


\chapter{Exercices d'analyse numérique (Louvain-la-Neuve)}

\Exo{SerieUn0002}
\Exo{SerieUn0003}
\Exo{SerieUn0001}
\Exo{SerieUn0004}
\Exo{SerieUn0005}
\Exo{SerieUn0006}
\Exo{SerieUn0007}
\Exo{SerieUn0008}
\Exo{SerieUn0009}

\Exo{SerieTrois0002}
\Exo{SerieDeux0001}
\Exo{SerieDeux0002}
\Exo{SerieDeux0003}
\Exo{SerieDeux0004}
\Exo{SerieDeux0005}
\Exo{SerieDeux0006}

\Exo{SerieTrois0001}
\Exo{SerieTrois0004}
\Exo{SerieTrois0003}	% Volontairement inversé avec le numéro 4.


\Exo{SerieQuatre0001}
\Exo{SerieQuatre0002}
\Exo{SerieQuatre0003}
\Exo{SerieQuatre0004}
\Exo{SerieQuatre0005}
\Exo{SerieQuatre0006}
\Exo{SerieQuatre0007}
\Exo{SerieQuatre0008}
\Exo{SerieCinq0001}
\Exo{SerieCinq0002}
\Exo{SerieCinq0003}
\Exo{SerieCinq0004}
\Exo{SerieCinq0006}
\Exo{SerieCinq0005}

Les exercices qui suivent proviennent d'examens d'années précédentes.

\Exo{examens-0000}
\Exo{examens-0001}


\chapter{Mathématique générale pour des ingénieurs (Louvain-la-Neuve)}
% This is part of the Exercices et corrigés de mathématique générale.
% Copyright (C) 2009-2010
%   Laurent Claessens
% See the file fdl-1.3.txt for copying conditions.
%+++++++++++++++++++++++++++++++++++++++++++++++++++++++++++++++++++++++++++++++++++++++++++++++++++++++++++++++++++++++++++
					\section{Déterminants et systèmes d'équations}
%+++++++++++++++++++++++++++++++++++++++++++++++++++++++++++++++++++++++++++++++++++++++++++++++++++++++++++++++++++++++++++

\Exo{INGE1121La0007}
\Exo{Lineraire0029}
\Exo{Lineraire0030}
\Exo{Lineraire0031}
\Exo{INGE1121La0006}
\Exo{Lineraire0001}
\Exo{INGE1121La0016}
\Exo{Lineraire0002}
\Exo{INGE1121La0010}
\Exo{INGE1121La0009}

\paragraph{Un problème de Bachet (XVIIème siècle)}
\Exo{Lineraire0003}

\paragraph{Un problème stupide (XXième siècle)}
\Exo{Lineraire0004}

%+++++++++++++++++++++++++++++++++++++++++++++++++++++++++++++++++++++++++++++++++++++++++++++++++++++++++++++++++++++++++++
					\section{Opérations sur les matrices}
%+++++++++++++++++++++++++++++++++++++++++++++++++++++++++++++++++++++++++++++++++++++++++++++++++++++++++++++++++++++++++++

\Exo{Lineraire0005}
\Exo{Lineraire0006}
\Exo{Lineraire0007}
\Exo{Lineraire0008}
\Exo{Lineraire0009}
\Exo{Lineraire0010}
\Exo{Lineraire0011}
\Exo{Lineraire0012}
\Exo{INGE1121La0008}

%+++++++++++++++++++++++++++++++++++++++++++++++++++++++++++++++++++++++++++++++++++++++++++++++++++++++++++++++++++++++++++
					\section{Espaces vectoriels}
%+++++++++++++++++++++++++++++++++++++++++++++++++++++++++++++++++++++++++++++++++++++++++++++++++++++++++++++++++++++++++++

\Exo{Lineraire0013}
\Exo{Lineraire0014}
\Exo{Lineraire0018}
\Exo{Lineraire0015}
\Exo{Lineraire0016}
\Exo{Lineraire0017}
\Exo{Lineraire0019}
\Exo{Lineraire0020}
\Exo{Lineraire0021}
\Exo{Lineraire0022}
\Exo{Lineraire0023}
\Exo{Lineraire0024}
\Exo{Lineraire0025}
\Exo{Lineraire0026}
\Exo{Lineraire0027}

%---------------------------------------------------------------------------------------------------------------------------
					\subsection{Orthogonalité}
%---------------------------------------------------------------------------------------------------------------------------

\Exo{Lineraire0028}
\Exo{INGE1121La0001}
\Exo{INGE1121La0002}
\Exo{INGE1121La0003}
\Exo{INGE1121La0004}
\Exo{INGE1121La0005}

%+++++++++++++++++++++++++++++++++++++++++++++++++++++++++++++++++++++++++++++++++++++++++++++++++++++++++++++++++++++++++++
					\section{Valeurs propres et vecteurs propres}
%+++++++++++++++++++++++++++++++++++++++++++++++++++++++++++++++++++++++++++++++++++++++++++++++++++++++++++++++++++++++++++

\Exo{Lineraire0032}
\Exo{Lineraire0033}
\Exo{Lineraire0034}
\Exo{Lineraire0035}
\Exo{Lineraire0036}
\Exo{Lineraire0037}
\Exo{Lineraire0038}
\Exo{Lineraire0039}
\Exo{Lineraire0040}
\Exo{Lineraire0041}
\Exo{Lineraire0042}
\Exo{INGE1121La0020}
\Exo{exoMatrices-0001}

%+++++++++++++++++++++++++++++++++++++++++++++++++++++++++++++++++++++++++++++++++++++++++++++++++++++++++++++++++++++++++++
					\section{Triangularisation}
%+++++++++++++++++++++++++++++++++++++++++++++++++++++++++++++++++++++++++++++++++++++++++++++++++++++++++++++++++++++++++++

\Exo{INGE1121La0021}
\Exo{INGE1121La0011}

%+++++++++++++++++++++++++++++++++++++++++++++++++++++++++++++++++++++++++++++++++++++++++++++++++++++++++++++++++++++++++++
					\section{Formes quadratiques}
%+++++++++++++++++++++++++++++++++++++++++++++++++++++++++++++++++++++++++++++++++++++++++++++++++++++++++++++++++++++++++++

\Exo{INGE1121La0012}
\Exo{INGE1121La0013}
\Exo{INGE1121La0017}
\Exo{INGE1121La0014}
\Exo{INGE1121La0018}
\Exo{INGE1121La0015}
\Exo{INGE1121La0019}


% This is part of the Exercices et corrigés de mathématique générale.
% Copyright (C) 2010,2014
%   Laurent Claessens
% See the file fdl-1.3.txt for copying conditions.

%+++++++++++++++++++++++++++++++++++++++++++++++++++++++++++++++++++++++++++++++++++++++++++++++++++++++++++++++++++++++++++ 
\section{Interrogation de mars 2010}
%+++++++++++++++++++++++++++++++++++++++++++++++++++++++++++++++++++++++++++++++++++++++++++++++++++++++++++++++++++++++++++

\Exo{Mars20100001}
\Exo{Mars20100002}
\Exo{Mars20100003}
\Exo{Mars20100004}


\chapter{Mathématique générale pour les géographes (Bruxelles)}
% This is part of the Exercices et corrigés de mathématique générale.
% Copyright (C) 2009
%   Laurent Claessens
% See the file fdl-1.3.txt for copying conditions.
\Exo{Janvier001}
\Exo{Janvier002}
\Exo{Janvier003}
\Exo{Janvier004}
\Exo{Janvier005}
\Exo{Janvier006}
\Exo{Janvier007}
\Exo{Janvier008}
\Exo{Janvier009}
\Exo{Janvier010}
\Exo{Janvier011}
\Exo{Janvier012}
\Exo{Janvier013}
\Exo{Janvier014}


\chapter{Travaux personnels 2009 (Bruxelles)}

\chapter{Autres exercices}
% This is part of the Exercices et corrigés de mathématique générale.
% Copyright (C) 2009-2010,2015
%   Laurent Claessens
% See the file fdl-1.3.txt for copying conditions.

%+++++++++++++++++++++++++++++++++++++++++++++++++++++++++++++++++++++++++++++++++++++++++++++++++++++++++++++++++++++++++++ 
\section{Pré(?)-requis}
%+++++++++++++++++++++++++++++++++++++++++++++++++++++++++++++++++++++++++++++++++++++++++++++++++++++++++++++++++++++++++++

Cette section contient quelque exercices du type de ce qui est plus ou moins censé être connu à l'entrée de l'université dans diverses sections scientifiques\footnote{Ils proviennent surtout d'un cours pour ingénieur de Louvain-la-Neuve.}.

\Exo{INGE1114-0006}
\Exo{INGE1114-0008}
\Exo{INGE1114-0009}

\Exo{INGE1114-0010}
\Exo{INGE1114-0011}
\Exo{INGE1114-0012}
%\Exo{INGE1114-0013}
%\Exo{INGE1114-0014}
%\Exo{INGE1114-0015}
\Exo{INGE1114-0016}
\Exo{INGE1114-0017}
\Exo{INGE1114-0018}
%\Exo{INGE1114-0019}
%\Exo{INGE1114-0020}


\Exo{INGE11140023}
\Exo{INGE11140024}
\Exo{INGE11140025}
\Exo{INGE11140027}

%+++++++++++++++++++++++++++++++++++++++++++++++++++++++++++++++++++++++++++++++++++++++++++++++++++++++++++++++++++++++++++
\section{Limites et continuité}
%+++++++++++++++++++++++++++++++++++++++++++++++++++++++++++++++++++++++++++++++++++++++++++++++++++++++++++++++++++++++++++

\Exo{INGE11140028}
\Exo{INGE11140029}
\Exo{INGE11140030}
\Exo{INGE11140031}
\Exo{INGE11140032}

%+++++++++++++++++++++++++++++++++++++++++++++++++++++++++++++++++++++++++++++++++++++++++++++++++++++++++++++++++++++++++++
\section{Suites numériques}
%+++++++++++++++++++++++++++++++++++++++++++++++++++++++++++++++++++++++++++++++++++++++++++++++++++++++++++++++++++++++++++

\Exo{INGE11140033}
\Exo{INGE11140034}
\Exo{INGE11140035}
\Exo{INGE11140036}
\Exo{INGE11140037}
% This is part of the Exercices et corrigés de mathématique générale.
% Copyright (C) 2009-2011
%   Laurent Claessens
% See the file fdl-1.3.txt for copying conditions.
%+++++++++++++++++++++++++++++++++++++++++++++++++++++++++++++++++++++++++++++++++++++++++++++++++++++++++++++++++++++++++++
					\section{Limites}
%+++++++++++++++++++++++++++++++++++++++++++++++++++++++++++++++++++++++++++++++++++++++++++++++++++++++++++++++++++++++++++

\Exo{General0010}
\Exo{General0011}
\Exo{0013}
\Exo{0017}
\Exo{0016}
\Exo{0024}

%+++++++++++++++++++++++++++++++++++++++++++++++++++++++++++++++++++++++++++++++++++++++++++++++++++++++++++++++++++++++++++
					\section{Dérivées et optimisation}
%+++++++++++++++++++++++++++++++++++++++++++++++++++++++++++++++++++++++++++++++++++++++++++++++++++++++++++++++++++++++++++

\Exo{General0012}
\Exo{General0013}
\Exo{General0014}
\Exo{General0015}
\Exo{General0016}

%+++++++++++++++++++++++++++++++++++++++++++++++++++++++++++++++++++++++++++++++++++++++++++++++++++++++++++++++++++++++++++
					\section{Primitives et intégration}
%+++++++++++++++++++++++++++++++++++++++++++++++++++++++++++++++++++++++++++++++++++++++++++++++++++++++++++++++++++++++++++

\Exo{General0017}
\Exo{General0018}
\Exo{General0019}
\Exo{General0020}
\Exo{General0021}
\Exo{General0022}
\Exo{General0023}
\Exo{General0024}
\Exo{General0025}
\Exo{General0026}
\Exo{General0027}

%---------------------------------------------------------------------------------------------------------------------------
					\subsection{Longueur d'un arc de courbe}
%---------------------------------------------------------------------------------------------------------------------------

\Exo{Inter0012}
\Exo{Inter0013}

%---------------------------------------------------------------------------------------------------------------------------
					\subsection{Aire d'une surface de révolution}
%---------------------------------------------------------------------------------------------------------------------------

\Exo{Inter0015}
\Exo{Inter0014}
\Exo{Inter0016}

% This is part of the Exercices et corrigés de mathématique générale.
% Copyright (C) 2009
%   Laurent Claessens
% See the file fdl-1.3.txt for copying conditions.
%+++++++++++++++++++++++++++++++++++++++++++++++++++++++++++++++++++++++++++++++++++++++++++++++++++++++++++++++++++++++++++
					\section{Équations différentielles}
%+++++++++++++++++++++++++++++++++++++++++++++++++++++++++++++++++++++++++++++++++++++++++++++++++++++++++++++++++++++++++++

%---------------------------------------------------------------------------------------------------------------------------
					\subsection{Équations à variables séparées}
%---------------------------------------------------------------------------------------------------------------------------

\Exo{EquaDiff0001}

%---------------------------------------------------------------------------------------------------------------------------
					\subsection{Équations homogènes}
%---------------------------------------------------------------------------------------------------------------------------

\Exo{EquaDiff0002}

%---------------------------------------------------------------------------------------------------------------------------
					\subsection{Équations linéaires}
%---------------------------------------------------------------------------------------------------------------------------

\Exo{EquaDiff0003}

%---------------------------------------------------------------------------------------------------------------------------
					\subsection{Problèmes divers}
%---------------------------------------------------------------------------------------------------------------------------

\Exo{EquaDiff0004}
\Exo{EquaDiff0005}
\Exo{EquaDiff0006}
\Exo{EquaDiff0007}
\Exo{EquaDiff0008}
\Exo{EquaDiff0009}

%---------------------------------------------------------------------------------------------------------------------------
					\subsection{Équations différentielles du second ordre}
%---------------------------------------------------------------------------------------------------------------------------

\Exo{EquaDiff0010}
\Exo{EquaDiff0011}
\Exo{EquaDiff0012}


\Exo{EquaDiff0013}
\Exo{EquaDiff0015}
\Exo{EquaDiff0014}
\Exo{EquaDiff0016}


% This is part of the Exercices et corrigés de mathématique générale.
% Copyright (C) 2009-2010
%   Laurent Claessens
% See the file fdl-1.3.txt for copying conditions.
%+++++++++++++++++++++++++++++++++++++++++++++++++++++++++++++++++++++++++++++++++++++++++++++++++++++++++++++++++++++++++++
					\section{Fonctions de deux variables réelles}
%+++++++++++++++++++++++++++++++++++++++++++++++++++++++++++++++++++++++++++++++++++++++++++++++++++++++++++++++++++++++++++

%---------------------------------------------------------------------------------------------------------------------------
					\subsection{Tracer}
%---------------------------------------------------------------------------------------------------------------------------

\Exo{FoncDeuxVar0001}

%---------------------------------------------------------------------------------------------------------------------------
\subsection{Limites à deux variables}
%---------------------------------------------------------------------------------------------------------------------------

\Exo{FoncDeuxVar0010}
\Exo{FoncDeuxVar0011}
\Exo{FoncDeuxVar0012}
\Exo{FoncDeuxVar0013}
\Exo{FoncDeuxVar0014}
\Exo{FoncDeuxVar0015}

\Exo{FoncDeuxVar0016}
\Exo{FoncDeuxVar0018}	

%---------------------------------------------------------------------------------------------------------------------------
\subsection{Dérivées partielles, différentielles totales}
%---------------------------------------------------------------------------------------------------------------------------
\Exo{FoncDeuxVar0002}
\Exo{FoncDeuxVar0003}

%---------------------------------------------------------------------------------------------------------------------------
\subsection{Différentiabilité, accroissements finis}
%---------------------------------------------------------------------------------------------------------------------------

\Exo{FoncDeuxVar0019}
\Exo{Maximisation-0001}
\Exo{FoncDeuxVar0026}
\Exo{FoncDeuxVar0021}
\Exo{FoncDeuxVar0022}
\Exo{FoncDeuxVar0023}
\Exo{DerrivePartielle-0000}
\Exo{DerrivePartielle-0001}
\Exo{FoncDeuxVar0025}

%---------------------------------------------------------------------------------------------------------------------------
\subsection{Plan tangent}
%---------------------------------------------------------------------------------------------------------------------------

\Exo{FoncDeuxVar0027}
\Exo{DerrivePartielle-0002}

%---------------------------------------------------------------------------------------------------------------------------
\subsection{Dérivées de fonctions composées}
%---------------------------------------------------------------------------------------------------------------------------

\Exo{DerrivePartielle-0003}
\Exo{FoncDeuxVar0017}
\Exo{DerrivePartielle-0004}
\Exo{DerrivePartielle-0005}
\Exo{FoncDeuxVar0030}
\Exo{FoncDeuxVar0024}
\Exo{FoncDeuxVar0020}
\Exo{DerrivePartielle-0006}

%---------------------------------------------------------------------------------------------------------------------------
\subsection{Dérivées de fonctions implicites}
%---------------------------------------------------------------------------------------------------------------------------
\Exo{FoncDeuxVar0004}
\Exo{FoncDeuxVar0005}
\Exo{FoncDeuxVar0006}
\Exo{FoncDeuxVar0007}

%---------------------------------------------------------------------------------------------------------------------------
\subsection{Extrema}
%---------------------------------------------------------------------------------------------------------------------------

\Exo{FoncDeuxVar0008}
\Exo{FoncDeuxVar0009}
\Exo{FoncDeuxVar0029}
\Exo{FoncDeuxVar0028}
\Exo{DerrivePartielle-0007}
\Exo{Maximisation-0002}
\Exo{DerrivePartielle-0008}
\Exo{DerrivePartielle-0009}
\Exo{DerrivePartielle-0010}
\Exo{Maximisation-0000}

% This is part of the Exercices et corrigés de mathématique générale.
% Copyright (C) 2009-2011,2014
%   Laurent Claessens
% See the file fdl-1.3.txt for copying conditions.
Lorsque nous demandons d'étudier une fonction, nous demandons les éléments suivants : domaine de définition, croissance, extrema, points d'inflexion, asymptote et dessiner le graphe.


\Exo{III-1}
\Exo{III-2}
\Exo{III-3}
\Exo{III-4}
\Exo{III-5}
\Exo{TP40001}
\Exo{TP40002}
\Exo{TP40003}
\Exo{TP40004}
\Exo{TP40005}
\Exo{TP50001}
\Exo{TP50002}
\Exo{TP50003}
\Exo{TP50004}

%---------------------------------------------------------------------------------------------------------------------------
\subsection{Quelque fautes usuelles}
%---------------------------------------------------------------------------------------------------------------------------

Pour l'exercice \ref{exoTP40001}, les fautes les plus souvent commises sont
\begin{enumerate}

	\item
		$f'= e^{2x}$ implique $f=\frac{1}{ 2 } e^{x}$. Cela n'est pas vrai. La dérivée de $ e^{2x}$ est $2 e^{2x}$. Le $2$ reste dans l'exponentielle.

	\item
		Lorsqu'on intègre par partie, il faut aussi mettre les bornes pour le morceau qui n'est pas dans la nouvelle intégrale :
		\begin{equation}
			\int_a^b fg'=[fg]_a^b-\int_a^bf'g.
		\end{equation}
\end{enumerate}

Pour l'exercice \ref{exoTP40002}, les fautes les plus souvent commises sont
\begin{enumerate}

	\item
		Lorsqu'on a trouvé la solution générale $y_k(x)$ qui dépend du paramètre $k$ (ou $C$), il faut encore trouver la valeur du paramètre $k$ telle que $y_k(\pi)=0$.

\end{enumerate}

Pour l'exercice \ref{exoTP40003}, les fautes les plus souvent commises sont
\begin{enumerate}

	\item
		Ne pas oublier que $e^0=1$.
\end{enumerate}


% This is part of Exercices et corrigés de CdI-1
% Copyright (c) 2011,2014
%   Laurent Claessens
% See the file fdl-1.3.txt for copying conditions.

\section{Intégrales de surface, Stokes et Green}


\setcounter{CountExercice}{0}


\noindent{\bf Exercice 6}\\

{\bf $(a)$ La suite $[k\rightarrow \f{1}{k}]$ est convergente.}\\

\noindent Nous allons montrer que cette suite converge vers $0$. Il faut donc prouver la chose suivante: 
   \begin{equation}\label{eqn1}\forall \epsilon >0 \hspace{0,3cm} \exists K_\epsilon \in \eN \hspace{0,3cm} {\rm tq}  \hspace{0,3cm}  \forall k\geq K_\epsilon, \hspace{0,3cm}  |x_k-x|<\epsilon\end{equation}
{Remarque}: On pourrait également montrer que cette suite est {\it de Cauchy} pour prouver qu'elle est convergente sans devoir déterminer sa limite.\\

\noindent Pour prouver que (\ref{eqn1}) s'applique bien à la suite des $\f{1}{k}$ il nous faut montrer que

   \begin{equation}\label{eqn2}\forall \epsilon >0 \hspace{0,3cm} \exists K_\epsilon \in \eN \hspace{0,3cm} {\rm tq}  \hspace{0,3cm}  \forall k\geq K_\epsilon,  \hspace{0,3cm} \f{1}{k}<\epsilon\end{equation}

   \noindent Ceci est une conséquence immédiate de l'exercice précédent. On peut également le montrer de la manière suivante: à $\epsilon$ positif donné, si nous arrivons à déterminer l'indice $K_\epsilon$ de \eqref{eqn2} tel que $\forall k\geq K_\epsilon,  \hspace{0,3cm} \f{1}{k}<\epsilon$, il est clair que la suite satisfait à la définition. Or, $\f{1}{k} < \epsilon \leftrightarrow \f{1}{\epsilon} <k$. Donc si nous prenons $K_\epsilon := \ulcorner  1/\epsilon \urcorner+1$, on a bien que $\forall k\geq K_\epsilon$, $\f{1}{k}<\epsilon$, ce qui est ce qu'il fallait démontrer.

\vspace{1cm}
{\bf $(b)$ La suite $(1, \f{1}{2}, -\f{1}{3},  \f{1}{4}, -\f{1}{5}, \ldots )$ est convergente.}\\

\noindent On remarque que cette suite tend vers zéro. (Il suffit de voir que le numérateur est borné et que le dénominateur  tend vers l'infini). Si on l'écrit  sous la forme standard, on obtient:                 
              \[x_1 = 1, x_k = \f{(-1)^k}{k} \hspace{0.3cm} \forall k\geq 2\] 
Donc, ce que nous voulons voir est que $x_k \longrightarrow_{k\rightarrow  \infty} 0$, i.e.: 
   \begin{equation}\label{eqn3}\forall \epsilon >0 \hspace{0,3cm} \exists K_\epsilon \in \eN \hspace{0,3cm} {\rm tq}                       
       \hspace{0,3cm}  \forall k\geq K_\epsilon,  \hspace{0,3cm} |\f{(-1)^k}{k}|<\epsilon\end{equation}
       
\noindent Étant donné que $|(-1)^k| = 1 \, \forall k$, il est clair que l'équation (\ref{eqn3}) est la même que l'équation (\ref{eqn2}), et donc que l'on peut affirmer que pour tout $\epsilon > 0$, il suffit de prendre $K\geq \f{1}{\epsilon}$ et la condition est satisfaite.

\noindent{\bf Exercice 7}\\

Ici il est demandé de prouver de nouvelles règles de calcul en repartant de la définition de la convergence vers l'infini:
\begin{equation}
 \label{eqnconvinfGene} x_k \longrightarrow \infty \hspace{0.3cm} {\rm si} \hspace{0.3cm}  \forall M > 0 \hspace{0.3cm} \exists K_M \in \eN \hspace{0.3cm} {\rm tq} \hspace{0.3cm} \forall k \geq K_M, x_k \geq M \end{equation}
{\bf (a) $ \lim(x_k+y_k) = +\infty$.}\\

\noindent On veut voir  la chose suivante:
\begin{equation}\label{eqnconvinfCasA}
   \forall M > 0 \hspace{0.3cm} \exists K_M \in \eN \hspace{0.3cm} {\rm tq} \hspace{0.3cm} \forall k \geq K_M, x_k + y_k \geq M 
  \end{equation}

\noindent Soit $M> 0$. Comme $x_k$ et $y_k$ convergent à l'infini, on sait que 
\[\left\{\begin{array}{c}   
         \exists K^x_M \in \eN \hspace{0.3cm} {\rm tq} \hspace{0.3cm} \forall k \geq K^x_M, x_k \geq \f{M}{2}\\																		 
        \exists K^y_M \in \eN \hspace{0.3cm} {\rm tq} \hspace{0.3cm} \forall k \geq K^y_M, y_k \geq \f{M}{2},																		
\end{array}\right.\]
et donc il suffit de prendre $K_M = \max(K_M^x, K_M^y)$ dans (\ref{eqnconvinfCasA}) pour s'assurer que la définition est satisfaite.


\vspace{0.5cm}
\noindent{\bf (b) $ \lim(x_ky_k) = +\infty$.}\\

\noindent On veut voir la chose suivante:
\begin{equation}
 \label{eqnconvinfprod}  \forall M > 0 \hspace{0.3cm} \exists K_M \in \eN \hspace{0.3cm} {\rm tq} \hspace{0.3cm} \forall k \geq K_M, x_k  y_k \geq M \end{equation}

\noindent Soit $M> 0$. Comme $x_k$ et $y_k$ convergent à l'infini, on sait que 
\[\left\{\begin{array}{c}   
         \exists K^x_M \in \eN \hspace{0.3cm} {\rm tq} \hspace{0.3cm} \forall k \geq K^x_M, x_k \geq \sqrt M\\																		 
        \exists K^y_M \in \eN \hspace{0.3cm} {\rm tq} \hspace{0.3cm} \forall k \geq K^y_M, y_k \geq \sqrt M,																		
\end{array}\right.\]

\noindent et donc il suffit  de prendre  $K_M = \max(K_M^x, K_M^y)$ dans (\ref{eqnconvinfprod}) pour s'assurer que la définition est satisfaite.

\vspace{0.5cm}
\noindent{\bf (d) Soit $z_k$ une suite tendant vers un réel $a$ strictement positif. Prouvez que $\lim(x_k  z_k) = +\infty$.}\\

Le but de l'exercice est toujours le même, c'est à dire de prouver que 
\begin{equation}		\label{eqnconvinfz}
  \forall M > 0 \hspace{0.3cm} \exists K_M \in \eN \hspace{0.3cm} {\rm tq} \hspace{0.3cm} \forall k \geq K_M, \;x_k  z_k \geq M 
\end{equation}

\noindent Soit $M>0$. On sait  que:

\begin{equation}
\label{eqn12}\left\{\begin{array}{l}   
        \forall \tilde{M} >0 \;\exists K^x_{\tilde{M}} \in \eN \hspace{0.3cm} {\rm tq} \hspace{0.3cm} \forall k \geq K^x_{\tilde{M}},\; x_k \geq  \tilde{M} \\																		 
       \forall \epsilon >0\;\exists K^z_\epsilon \in \eN \hspace{0.3cm} {\rm tq} \hspace{0.3cm} \forall k \geq K^z_\epsilon,\; |z_k-a| <\epsilon,																		
\end{array}\right.\end{equation}

\noindent Prenons un $\epsilon$ tel que $a-\epsilon>0$. Par la deuxième partie de (\ref{eqn12}) on voit qu'il existe un indice $ K^z_\epsilon$ tel que $ \forall k \geq K^z_\epsilon,\; z_k > a-\epsilon >0$.

\noindent Prenons un $\tilde{M}$ tel que $M= \tilde{M}(a-\epsilon)$. Par la première partie de (\ref{eqn12}) on voit qu'il existe un indice $ K^x_{\tilde{M}} $ tel que $\forall k \geq K^x_{\tilde{M}},\; x_k \geq  \tilde{M} $.

												
\noindent et donc il suffit  de prendre  $K_M = \max(K_{\tilde{M}}^x, K^z_\epsilon)$ dans (\ref{eqnconvinfz}) pour avoir que 
\[ \forall k \geq K_M, \;x_k  z_k \geq \tilde{M}(a-\epsilon)=M.\]


\noindent{\bf Exercice 8}\\

\noindent Une suite $x_k$ est bornée si $\exists N>0$ tel que $\forall k$, $|x_k| < N$.

\noindent On veut voir que $\f{x_k}{y_k}\longrightarrow 0$, i.e.

\begin{equation} 
\label{eqnconvborne}  \forall  \epsilon > 0 \hspace{0.3cm} \exists K_\epsilon \in \eN \hspace{0.3cm} {\rm tq} \hspace{0.3cm} \forall k \geq K_\epsilon, \; |\f{x_k}{y_k}| < \epsilon \end{equation}

\noindent Soit $\epsilon >0$. Comme la suite $x_k$ est bornée, on a que  $|\f{x_k}{y_k}|<\f{N}{|y_k|}\; \forall k$. On utilise maintenant le fait que $y_k \longrightarrow \infty$. Prenons $M=\f{N}{\epsilon}$. On peut écrire que $\exists K_M$ tel que $\forall k \geq K_M, \; y_k \geq M=\f{N}{\epsilon}$, et donc si dans (\ref{eqnconvborne}) on prend $K_\epsilon= K_M$ on a:\[\forall k \geq K_\epsilon,\; \; |\f{x_k}{y_k}|<\f{N}{|y_k|}<\f{N}{N/\epsilon}=\epsilon.\]



\noindent{\bf Exercice 9}\\

\noindent Pour cet exercice, on peut utiliser les règles de calcul. Il faut faire attention que ces règles ne s'appliquent que si toutes les limites existent!

\vspace{0.5cm}
\noindent{ (a)} $x_k = \f{k+2}{k}\cos(k\pi)$\\

\noindent On voit que cette suite va dans deux directions différentes, $+1$ et $-1$ à cause du facteur $\cos(k\pi)=(-1)^k$. Elle ne converge donc pas. Pour le prouver, on peut prendre deux suites partielles de la suite $x_k$ qui convergent vers des limites différentes. 

\noindent Choisissons \[\left\{ \begin{array}{rcl} y_k &= x_{2k}&= \f{(2k)+2}{2k}(-1)^{2k}\\
 							  z_k &= x_{2k+1} &= \f{(2k+1)+2}{2k+1} (-1)^{2k+1}\end{array}\right.\]

\noindent Comme $x_k =\f{k+1}{k}= 1+\f{1}{k}$	et que $\f{1}{k} \rightarrow  0$, nous pouvons appliquer les règles de calcul et en déduire que $x_k \rightarrow  1$. On fait la même chose pour $y_k$.				  


\vspace{0.5cm}
\noindent{ (c)} $x_k = \f{k^3+k+1}{5k^3+2}$\\

\noindent Nous avons que \[\forall k, \;\;\;\;x_k =\; (\f{k^3}{k^3})\f{1+\f{1}{k} +\f{1}{k^3}}{5+\f{2}{k^3}}=\;\f{1+\f{1}{k} +\f{1}{k^3}}{5+\f{2}{k^3}} \]
Comme \[\forall k \geq 1\;\; \f{1}{k^3} \; \leq \;\f{1}{k^2}\; \leq \; \f{1}{k}\] et comme $\f{1}{k}\rightarrow 0$, nous pouvons appliquer la règle de l'étau pour voir que \[\f{1}{k^3} \rightarrow 0 \; \; \; {\rm et } \;\; \;\f{1}{k^2} \rightarrow 0.\]
En appliquant les règles de calcul à la suite $x_k$ transformée, on voit donc que $x_k \rightarrow  \f{1}{5}$.

\vspace{0.5cm}
\noindent{ (d)} $x_k = \f{k+(-1)^k}{k-(-1)^k}$\\

\noindent On peut le voir par exemple par la règle de l'étau:
\[\forall k \geq 0, \;\;\; \f{k-1}{k+1} \leq \f{k+(-1)^k}{k-(-1)^k} \leq \f{k+1}{k-1}. \]
Or, comme les deux suites qui bornent la suite $x_k$ convergent toutes les deux vers $1$, il est clair que $x_k$ converge aussi vers $1$.


\vspace{0.5cm}
\noindent{ (d)} $x_k = x_{k-1}^2\;+\;1,\, x_1=1$\\

\noindent Suite définie par récurrence. Ses premiers éléments sont \[1, \; 2, \; 5, \;  26, \; 677, \; \ldots\]
Toute  limite admissible réelle finie $l$  de cette suite doit satisfaire à \[l=l^2+1\] ce qui implique qu'elle ne peut avoir de limite réelle finie. En regardant ses premiers éléments, on remarque immédiatement qu'elle semble converger à l'infini. Nous allons le prouver en utilisant la définition.

\noindent Soit $M> 0$. On a que \[x_k \geq k \, \forall k.\] En effet (par récurrence sur $k$): il est clair que $x_1 \geq 1$. Supposons que $x_k \geq k$. Ceci implique t-il que $x_{k+1}\geq k+1$? Par définition des $x_k$, $x_{k+1} = x_k^2+1$. Par l'hypothèse de récurrence, on a donc $x_{k+1}\geq (k)^2 +1\geq k+1$ ce qui prouve le résultat. Comme la suite $y_k=k$ converge à l'infini, il en est de même pour la suite $x_k$.



\section{Continuité de fonctions réelles}


\begin{center}
\LARGE \bf
Travaux Personnels 
\end{center}

\begin{bf}
\begin{center}
BAC2 en sciences mathématiques et physiques
\end{center}
\end{bf}

{\bf Exercice 1.} Calculer les limites suivantes

\b
a) $\displaystyle \lim_{n \to \infty} \left( 1+ \frac{2}{n-4} \right)^n$

\medskip
b) 
$\displaystyle \lim_{n \to \infty} 
         \left( 1+ \frac 1n \right)^{\sqrt{n}}$

\medskip
c) $\displaystyle \lim_{x \to \infty} 
    \left( 1+ \frac \alpha x \right)^x$

\medskip
d) 
$\displaystyle \lim_{x \to 0} \frac{\log \left( 1+ \alpha x \right)}{x}$


\medskip
e) 
$\displaystyle \lim_{x \to \infty} 
\frac{a_0+a_1x + \dots +a_nx^n}{b_0+b_1x + \dots +b_mx^m}$
\quad où\, $a_j, b_j \in \eC$ \,et\, $n,m \ge 0$

\medskip
f) 
$\displaystyle \lim_{x \to 0} \frac{\sqrt{1-\cos x}}{x}$  




{\bf Exercice 2.} Prouver que

\medskip
a)
$\displaystyle \lim_{x \to \infty} x^{\frac 1x} = \lim_{x \to 0^+} x^x = 1$

\medskip
b)
$\displaystyle \lim_{x \to \infty} \frac{x^{\ln x}}{{\mathrm e}^x} =0$
\quad
càd ${\mathrm e}^x$ croit plus vite que $x^{\ln x}$


{\bf Exercice 3.} Prouver que
$$
\cosh 2x \,=\, \cosh^2 x + \sinh^2 x,
\qquad
\sinh 2x \,=\, 2 \sinh x \cosh x
$$


{\bf Exercice 4.} Prouver que

a)
$1 + \cos z + \cos 2z + \dots + \cos nz = \displaystyle \cos \frac{nz}{2} \cdot \frac{\sin (n+1)z/2}{\sin z/2}$

b)
$1 + \sin z + \sin 2z + \dots + \sin nz = \displaystyle \sin \frac{nz}{2} \cdot \frac{\sin (n+1)z/2}{\sin z/2}$

{\it Aide:}\;
$\displaystyle \sum_{k=0}^n 
\euler^{\sii kz} 
= 
\frac{1-\euler^{\sii (n+1)z}}{1-\euler^{\sii z}}
= \euler^{\sii nz/2} \cdot \frac{\euler^{\sii (n+1)z/2} - \euler^{-\sii (n+1)z/2}}{
\euler^{\sii z/2}-\euler^{-\sii z/2}}
$

Rappelons qu'une fonction $f \colon \mathbb{C} \supset D \to \eC$ est {\bf uniformément continue} si pour tout $\epsilon >0$ il existe un \( \delta>0\) tel que 
$$
|x-y| < \delta \,\Longrightarrow\, |f(x)-f(y)| < \epsilon 
\quad \text{ pour tout }\, x,y \in D.
$$
Prouver que la fonction $f \colon \eR \to \eR$, $x \mapsto x^2$ est continue, mais n'est pas uniformément continue.


\section{Intégrales, longueur de courbes, EDO's linéaires}


\exerNico 
Soient $n,m \in \eN \cup \{0\}$.
Calculer
$$
\int_0^1 x^n (1-x)^m \,dx
\quad \text{ et } \quad
\int_{-1}^1 (1+x)^n (1-x)^m \,dx
$$

{\bf Solution:}
Posons $I_{n,m} := \int_0^1 x^n (1-x)^m \,dx$.
Intégration par partie donne
la formule récursive
$$
I_{n,m} \,=\, \frac {m}{n+1} I_{n+1,m-1}.
$$
Avec $I_{n+m,0} = \frac{1}{n+m+1}$ nous obtenons
$$
I_{n,m} \,=\, \frac{n!\,m!}{(n+m+1)!}
$$
La substitution $x := 2t-1$ fournit
$$
\int_{-1}^1 (1+x)^n (1-x)^m \,dx
\,=\, 2^{n+m+1} \int_0^1 t^n (1-t)^m \,dt \,=\,  2^{n+m+1} 
\cdot I_{n,m}. 
$$




\exerNico 
Soient $a,b >0$. 
Calculer
$$
\int_0^{\pi /2} \displaystyle \frac{d \varphi}{a^2 \sin^2 \varphi + b^2 \cos^2 \varphi}
$$

{\bf Solution:}
$$
\,=\, \int_0^{\pi /2} \frac{1 / \cos^2 \varphi}{a^2 \tan^2 \varphi+b^2} d\varphi \,=\, \int_0^\infty \frac {dt}{a^2t^2 + b^2} \,=\, \frac{\pi}{2ab}  
$$


\exerNico  
Calculer la longueur de l'arc de la parabole $y = x^2,\;x \in [0,b]$.

\medskip
{\bf Solution:}
$$
s \,=\, \int_0^b \sqrt{1+4x^2} \,dx \,=\, \frac b 2 \sqrt{1+4b^2}+ \frac 14 \ln \left(2b+ \sqrt{1+4b^2} \right)
$$


\exerNico  
La {\bf parabole de Neil} $\nu$ est la courbe définie par $\nu (t) = (t^2,t^3)$, pour  $t \in \eR$.

\medskip
a)
Esquisser la parabole de Neil.

\medskip
b)
Quelle est la signification du paramètre $t$?

\medskip
{\bf Solution:} $t = \tan \alpha$

\medskip
c)
Calculer la longueur de l'arc 
$\left\{ \nu (t) \mid t \in [0,\tau] \right\}$.


\medskip
{\bf Solution:}
$$
s \,=\, \int_0^\tau \sqrt{4 t^2+9t^4} \,d\tau \,=\, \frac{8}{27} \left( \left(1+ \frac 94 \tau^2\right)^{3/2}-1 \right)
$$



\exerNico  
La {\bf hélice} $\gamma$ de pas $2 \pi h$ est la courbe dans $\eR^3$ définie par
$$
\gamma(t) \,=\, \left( r \cos t , r \sin t , h t \right)  .
$$


\medskip
a)
Esquisser la hélice.

\medskip
b)
Expliquer le mot ``pas''.


\medskip
c)
Calculer la longueur de l'arc sur la hélice si on fait un tour.

\medskip
{\bf Solution:} 
$\int_0^{2\pi} \sqrt{r^2+h^2} \,dt \,=\, 2 \pi \sqrt{r^2+h^2}$


\bigskip
\exerNico 
Calculer un système fondamental réel pour

\medskip
a) $y^{(4)}-y = 0$,

\medskip
b) $y^{(4)} +4y'' +4y = 0$,

\medskip
c) $y^{(4)} -2y^{(3)} +5y'' = 0$.


\bigskip
{\bf Solution:}

\medskip
a) ${\rm e}^x, {\rm e}^{-x}, \cos x, \sin x$

\medskip
b) $\cos \sqrt{2} x, x \cos \sqrt{2}x, \sin \sqrt{2}x, x \sin \sqrt{2}x$

\medskip
c)
$1, x, {\rm e}^x \cos 2x, {\rm e}^x \sin 2x$



\bigskip
\exerNico 
Déterminer une solution particulière de l'équation
$y''+y=q$ pour

\medskip
a) $q = x^3$,

\medskip
b) $q = \sinh x$,

\medskip
c) $q = 1/\sin x$.
 

\bigskip
{\bf Solution:}

\medskip
a) $x^3 - 6 x$

\medskip
b) $\frac 12 \sinh x$

\medskip
c) $\sin x \cdot \ln |\sin x| - x \cos x$


\bigskip
\exerNico  
L'équation différentielle $m \ddot y = mg - k\dot y$ 
décrit la chute d'un corps soumit
à la gravitation si la friction est proportionnelle à la vitesse (``un homme tombant de l'avion'').

\medskip
Calculer la solution avec $y(0) =0, \dot y(0) = 0$.
Calculer la ``vitesse finale'' $v_\infty = \displaystyle \lim_{t \to \infty} \dot y (t)$.



\bigskip
{\bf Solution:}

\medskip
L'équation homogène $\ddot y + k/m \cdot y = 0$
possède les solutions $c_1+c_2 {\rm e}^{-k/m \cdot t}$,
où $c_1, c_2 \in \eR$.
 
L'équation inhomogène $\ddot y + k/m \cdot y = g$
possède comme solution particulière une fonction lineaire, càd 
$y_p = (mg/k)t)$.
En tenant compte des conditions initiales nous obtenons
$$
y(t) \,=\, \frac{mg}{k} \left( t-\frac mk (1-{\rm e}^{-k/m \cdot t})\right).
$$
En particulier, $v_\infty = mg/k$. 

 




\bigskip
\exerNico  
Regardons l'ensemble des solutions de l'équation différentielle $P({\rm D})y =0$.

Montrer l'équivalence entre les propositions suivantes :
\begin{enumerate}

\item
Pour toute solution $y$ on a $\displaystyle \lim_{t \to \infty} y(t) = 0$

\item
Pour toute racine $z$ du polynôme caractéristique on a ${\rm Re}\, z <0$.

\end{enumerate}
Dans ce cas, l'équation différentielle est appelé  \defe{asymptotiquement stable}{asymptotiquement stable}.

\bigskip
{\bf Solution:}
On a 
$\displaystyle \lim_{t \to \infty} y(t) = 0$ pour toute solution $y$ ssi c'est vrai pour tout élément d'un système fondamental.
On a $\displaystyle \lim_{t \to \infty} t^k {\rm e}^{\gl t}=0$ ssi ${\rm Re }\,\gl <0$,
d'où l'affirmation suit.






\section{Calcul de limites}

\exerNico Déterminez si les limites suivantes existent et dans
l'affirmative calculez les en utilisant, s'il y a lieu, la règle de
l'Hospital ou la règle de l'étau.
\begin{enumerate}
\item $  \lim_{x \rightarrow  +\infty} \frac{x+1}{x^2+2} $
\item $  \lim_{x \rightarrow  +\infty} \frac{\sin(x)}{x} $
\item $  \lim_{x \rightarrow  0} \frac{\sin(x)}{x} $
\item $  \lim_{x \rightarrow  +\infty}  \frac{x ^n}{e ^x} $
\item $  \lim_{x \rightarrow  +\infty} (1 + \frac{a}{x})^x $
\item $  \lim_{x \rightarrow  0} (\frac{1}{\sin(x)} - \frac{1}{x} )$
\item $  \lim_{x \rightarrow  +\infty} \cos( 2 \pi x) $
\item $  \lim_{x \rightarrow  +\infty} \frac{1}{\sin(x)+2}(x) +\ln(x)\cos(x) $
\item $  \lim_{x \rightarrow  +\infty} \frac{ \ln(x)(\sin(x) +2)}{x} $
\item $  \lim_{x \rightarrow  +\infty} x ^\frac{1}{x} $
\end{enumerate}

\exerNico Déterminez si les limites suivantes existent et dans
l'affirmative calculez-les.
\begin{enumerate}
\item $  \lim_{x \rightarrow  0} x \sin(\frac{1}{x}) $
\item $  \lim_{x \rightarrow  0} \frac{\sin(\sin(x))}{x} $
\item $  \lim_{x \rightarrow  +\infty} (\ln(x))^\frac{1}{1 - \ln(x)}$
\end{enumerate}

\exerNico Calculez les limites suivantes:
\begin{enumerate}
\item $  \lim_{x \rightarrow  +\infty} \frac{\ln(x)}{x ^a} $
\item $  \lim_{x \rightarrow  +\infty} \frac{\ln(x)^a}{x ^b} $
\item $  \lim_{x \rightarrow  +\infty} a ^x $
\item $  \lim_{x \rightarrow  +\infty} a ^\frac{1}{x} $
\end{enumerate}
où $a$ et $b$ sont des réels positifs.
%

%

\exerNico Déterminez, pour chacune des suites suivantes, si elle converge
et dans l'affirmative calculez sa limite.
\begin{enumerate}
\item $  k \rightarrow  \cos( 2 \pi k) $
\item $  k \rightarrow  \cos(\frac{\pi}{3} k) $
\item $  k \rightarrow  k(a ^\frac{1}{k} -1 ) $
\end{enumerate}
où $a$ est une réel.\\



\exerNico Calculez  les limites suivantes si elles existent.
\begin{enumerate}
\item $  \lim_{x \rightarrow  +\infty} \cos x $
\item $  \lim_{x \rightarrow  \pm \infty }\sqrt{2x^4+3}-x^2 $

\end{enumerate}

\exerNico Déterminez si la limite de chacune des suites suivantes
existe et dans l'affirmative calculez la.
\begin{enumerate}
\item $  \lim_{k \rightarrow  +\infty }(\frac{a k +1}{k})^k $
\item $  \lim_{k \rightarrow  +\infty}\frac{1}{\sin(\frac{\pi}{6}k)+1}(k) + \ln(k)\cos(\frac{\pi}{5}k)$
\item $  \lim_{k \rightarrow  +\infty} \frac{\ln(k)(\sin(\frac{\pi}{3}k) +1)}{k} $
\item $  \lim_{k \rightarrow  +\infty } \sqrt[3k]{k} (1 +
\frac{1}{3k})^{3k} $
\end{enumerate}
où $a$ est un réel. 

\section{Dérivabilité}



\exerNico Déterminez l'ensemble des points où les fonctions suivantes
sont continues et celui où elles sont dérivables. Prouvez soigneusement
vos résultats.
\begin{enumerate}
\item $ x \rightarrow x]$
\item $ x \rightarrow |x| $
\item $ x \rightarrow
	\left\{ \begin{array}{ll}
	\frac{1}{x} & \mbox{si } x \not= 0 \\
	0 & \mbox{sinon}
	\end{array} \right. $
\item $ x \rightarrow x^2  $
\end{enumerate}




\exerNico Étudiez la dérivabilité et la continuité
de la dérivée de chacune des fonctions suivantes:
\begin{enumerate}
\item $ x \rightarrow
\left\{ \begin{array}{ll}
0 & \mbox{si } x \not= 0 \\
1 & \mbox{sinon}
\end{array} \right.$
%
\item $ x \rightarrow
\left\{ \begin{array}{ll}
\sin(\frac{1}{x}) & \mbox{si } x \not= 0 \\
0 & \mbox{sinon}
\end{array} \right.$
%
\item $ x \rightarrow
\left\{ \begin{array}{ll}
x \sin(\frac{1}{x}) & \mbox{si } x \not= 0 \\
0 & \mbox{sinon}
\end{array} \right.$
%
\item $ x \rightarrow
\left\{ \begin{array}{ll}
x^2 \sin(\frac{1}{x}) & \mbox{si } x \not= 0 \\
0 & \mbox{sinon}
\end{array} \right.$
\end{enumerate}

Le but de cet exercice est aussi d'exhiber des exemples illustrant les
différents types de comportements possibles, relativement à la
continuité et la dérivabilité, d'une fonction en un point.

\exerNico Étudiez la dérivabilité et la continuité
de la dérivée de chacune des fonctions suivantes:
\begin{enumerate}
\item $ x \rightarrow
\left\{ \begin{array}{ll}
\frac{2x+a}{1+e^{\frac{1}{x}}} & \mbox{si } x \not= 0 \\
0 & \mbox{sinon}
\end{array} \right.$
%
\item $ x \rightarrow
\left\{ \begin{array}{ll}
\frac{\sin(x)}{x} & \mbox{si } x \not= 0 \\
1 & \mbox{sinon}
\end{array} \right.$
%
\item $ x \rightarrow
\left\{ \begin{array}{ll}
e^{\frac{-1}{x}} & \mbox{si } x > 0 \\
0 & \mbox{sinon}
\end{array} \right.$
%
\item $ [-\frac{1}{2}, \frac{1}{2}] \rightarrow \eR: x \rightarrow
\left\{ \begin{array}{ll}
(\frac{\sin(2x)}{x})^{x+1} & \mbox{si } x \not= 0 \\
1 & \mbox{sinon}
\end{array} \right.$
\end{enumerate}
où $a$ et $b$ sont des réels.


 \exerNico Considérons la fonction
$$f:\mathbb{R}\rightarrow\mathbb{R}:x\mapsto f(x)=\left\{
\begin{array}{ll}
x&\text{si }x\text{ est rationnel}\\
0&\text{si }x\text{ est irrationnel}
\end{array}
\right.$$

Vérifiez que $f$ est continue en $0$ mais n'est ni dérivable à  gauche ni dérivable à droite en
$0$.

\exerNico 
\begin{enumerate}
\item Soit $(X,d)$ un espace métrique et $f \colon (X,d) \to \eR$ une application continue.
Montrer que l'ensemble $$\left\{ x \mid f(x) = 0 \right\}$$ est fermé.

\item Soit $f \colon \eR \to \eR$ une application continue.
Montrer que l'ensemble 
$$
\{ x \in \eR \mid f(x) = x\}
$$
des points fixes de $f$ est fermé.

\end{enumerate}

\exerNico  Soit $A$ un sous ensemble de l'espace métrique $(X,d)$.
Montrer que la fonction
$$
\dist_A \colon X \to \eR,
\quad x \mapsto \inf_{a \in A} d(a,x)
$$ 
est continue.


\exerNico  Soient $(X,d_X)$, $(Y,d_Y)$ deux espaces métriques.
Une application $f \colon X \to Y$ est {\bf Lipschitzienne}
s'il existe une constante $L \ge 0$ telle que
$$
d_Y \bigl( f(x), f(x') \bigr) \,\le\, L \,d_X (x,x') 
\quad \text{ pour tout } x,x' \in X.
$$
Dans ce cas, on dit que $f$ est {\bf $L$-Lipschitzienne}.


\begin{enumerate}
\item
Montrer qu'une application Lipschitzienne est continue.
\item Montrer qu'une application $f \colon \eR \to \eR$, $x \mapsto ax+b$
est Lipschitzienne.
Quelle est la plus petite constante $L$ qui convienne?

\item Montrer que les fonctions $z \mapsto |z|$, 
$z \mapsto \overline z$,
$z \mapsto {\rm Re\,} z$ et $z \mapsto {\rm Im\,} z$ 
de $\eC$ dans $\eR$ sont Lipschitziennes.
Quelle sont les plus petites constantes $L$ qui conviennent?
\item Montrer que la fonction $\dist_A \colon X \to \eR$ de l'Exercice~13 est Lipschitzienne.

\end{enumerate}

% This is part of Exercices et corrigés de CdI-1
% Copyright (c) 2011,2014
%   Laurent Claessens
% See the file fdl-1.3.txt for copying conditions.




\section{Intégration}

\exerNico 
Soient $n,m \in \eN \cup \{0\}$.
Calculer
$$
\int_0^1 x^n (1-x)^m \,dx
\quad \text{ et } \quad
\int_{-1}^1 (1+x)^n (1-x)^m \,dx
$$



\exerNico 
Soient $a,b >0$. 
Calculer
$$
\int_0^{\pi /2} \displaystyle \frac{d \varphi}{a^2 \sin^2 \varphi + b^2 \cos^2 \varphi}
$$


\exerNico  
Calculer la longueur de l'arc de la parabole $y = x^2,\;x \in [0,b]$.

\exerNico  
La {\bf parabole de Neil} $\nu$ est la courbe définie par
$\nu (t) = (t^2,t^3), \, t \in \eR^n$.
\begin{enumerate}
\item Esquisser la parabole de Neil.

\item Quelle est la signification du paramètre $t$?

\item Calculer la longueur de l'arc 
$\left\{ \nu (t) \mid t \in [0,\tau] \right\}$.
\end{enumerate}

\exerNico  
Une {\bf hélice} $\gamma$ de pas $2 \pi h$ est une courbe dans $(\eR^n)^3$ définie par
$$
\gamma (t) \,=\, \left( r \cos t , r \sin t , h t \right)  .
$$

\begin{enumerate}
\item Esquisser $\gamma$ et expliquer le mot ``pas''.

\item Calculer la longueur de l'arc sur la hélice si on fait un tour.
\end{enumerate}

\exerNico Calculez la longueur des arcs de courbe suivants:
\begin{enumerate}
\item $y= \ln(1-x^2)  \hspace{3.5cm} 0\leq x\leq \f{1}{2}$
\item  $y= x^{3/2}  \hspace{4.57cm} 0\leq x\leq 5$
\item $y = 1-\ln(\cos x) \hspace{3cm} 0\leq x \leq \f{\pi}{4}$
\item l'arc de cubique déterminé par $y=x^3+x^2+x+1$ avec $0\leq x \leq 1$.
\end{enumerate}

% This is part of Exercices et corrigés de CdI-1
% Copyright (c) 2011,2013-2014
%   Laurent Claessens
% See the file fdl-1.3.txt for copying conditions.

%+++++++++++++++++++++++++++++++++++++++++++++++++++++++++++++++++++++++++++++++++++++++++++++++++++++++++++++++++++++++++++
                    \section{Quelque corrections}
%+++++++++++++++++++++++++++++++++++++++++++++++++++++++++++++++++++++++++++++++++++++++++++++++++++++++++++++++++++++++++++

Ce qui suit sont des corrections d'exercices donnée sur les feuilles distribuées au début de l'année.

\noindent 31.
\begin{enumerate}
\item $df_{(1,1)}$ et $dg_{(\sqrt2,\frac{\pi}{4})}$\\
    \[\begin{array}{l}\frac{ \partial f }{ \partial x }(x,y) = \frac{1}{y}\ln(\frac{x}{y})e^{\frac{x}{y}}+\frac{1}{x}e^{\frac{x}{y}}\\
            \frac{ \partial f }{ \partial x }(1,1)=e\\
            \frac{ \partial f }{ \partial y }(x,y)=-\frac{x}{y^2}\ln(\frac{x}{y})e^{\frac{x}{y}}-xe^{\frac{x}{y}}\\
        \frac{ \partial f }{ \partial x }(1,1)=e\end{array}\]
            
 \noindent Par les règles de calcul, $f$ est différentiable en $(1,1)$. la différentielle $df_{(1,1)}$ est donc représentée dans les bases canoniques de $\eR^2$ et $\eR$ par la matrice jacobienne (ici gradient):\[df_{(1,1)}=(e \;\; -e)\]
 
 \[\begin{array}{lclllllcl}\frac{ \partial g_1 }{ \partial r }(r,\theta) &=&\cos(\theta)& & & & \frac{ \partial g_1 }{ \partial \theta }(r,\theta)   & =&-r\sin(\theta)\\
         \frac{ \partial g_1 }{ \partial r }(\sqrt2, \frac{\pi}{4})&=&\frac{\sqrt2}{2}& & &&\frac{ \partial g_1 }{ \partial \theta }(\sqrt2, \frac{\pi}{4})& =&-1 \\
         \frac{ \partial g_2 }{ \partial r }(r,\theta) &=&\sin(\theta)&  && &\frac{ \partial g_2 }{ \partial \theta }(r,\theta)  &=&r\cos(\theta) \\
     \frac{ \partial g_2 }{ \partial r }(\sqrt2, \frac{\pi}{4})&=&-\frac{\sqrt2}{2}&& & &\frac{ \partial g_2 }{ \partial \theta }(\sqrt2, \frac{\pi}{4})& = &1\end{array}\]

La fonction $g$ est également différentiable en $(\sqrt2, \frac{\pi}{4})$ et sa matrice Jacobienne est:
\[dg_{(\sqrt2, \frac{\pi}{4})}=\left(\begin{array}{cc} \frac{\sqrt2}{2} & -1\\
    \frac{\sqrt2}{2}&1\end{array}\right)\]  


\item $\tilde{f} \;=\;e^{\cos(\theta)}\ln(\cos(\theta))$.
\item On voit d'abord que $g(\sqrt2, \frac{\pi}{4})\;=\;(1,1)$. Donc
    \[\begin{array}{cccc} d\tilde{f}_{(\sqrt2, \frac{\pi}{4})} & = & df_{g(\sqrt2, \frac{\pi}{4})}\circ dg_{(\sqrt2, \frac{\pi}{4})}\\
        & =& df_{(1,1)}\circ dg_{(\sqrt2, \frac{\pi}{4})} \end{array}\]
                                et  la matrice jacobienne de la différentielle de la composée est donc:\[d\tilde{f}_{(\sqrt2, \frac{\pi}{4})}=(e\;\;-e)\left(\begin{array}{cc} \frac{\sqrt2}{2} & -1\\
                                    \frac{\sqrt2}{2}&1\end{array}\right)=(0\;\;-2e)\]

                                        

\end{enumerate}


\noindent 32.
\begin{enumerate}
    \item $\frac{ \partial g }{ \partial u } = e^v\frac{ \partial f }{ \partial x }(\star,\star)+2uv\frac{ \partial f }{ \partial y }(\star,\star)$
    \item $\frac{ \partial g }{ \partial v } = ue^v\frac{ \partial f }{ \partial x }(\star,\star)+(1+u^2)\frac{ \partial f }{ \partial y }(\star,\star)$
\end{enumerate}
où $(\star,\star) = (ue^v,v(1+u^2))$.

\vspace{1cm}

\noindent 33. \\

\noindent Soit $g:\eR^2\rightarrow \eR:(x,y)\rightarrow  f(x^2-y^2)$. Dérivées partielles de:\[(x,y)\rightarrow  y(\partial_xg)(x,y)+x(\partial_yg)(x,y)?\]
Nommons cette fonction $h$. 
\begin{enumerate}
\item $\partial_xg(x,y) = 2xf'(x^2-y^2)$
\item$\partial_yg(x,y) = -2yf'(x^2-y^2)$
\end{enumerate}
et donc $h(x,y) = 0 \, \forall (x,y)\in \eR^2$.

\vspace{1cm}


\noindent 34. \\

\noindent $h(t)=f(t,g(t^2))$.\\

\begin{enumerate}
    \item $h'(t)=\frac{ \partial f }{ \partial x }(\star,\star)+\frac{ \partial f }{ \partial y }(\star,\star)2tg'(t^2)$
    \item $ \begin{array}{rl} h''(t)=     &   \frac{ \partial^2f }{ \partial x }(\star,\star)+4tg'(t^2)\frac{ \partial^2f }{ \partial x\partial y }(\star,\star)+4t^2(g'(t^2))^2\frac{ \partial^2f }{ \partial y^2 }(\star,\star) \\           
        & +[2g'(t^2)+4t^2g''(t^2)]\frac{ \partial f }{ \partial y }(\star,\star)\end{array}$

\end{enumerate}
où $(\star,\star) = (t,g(t^2))$.

\vspace{1cm}

\noindent 35.
\[h:\eR^2\rightarrow \eR:(u,v)\rightarrow  f(g(ue^v),g(v)(1+u^2))^{g(u+v)}\]

\noindent Comme toujours il vaut mieux faire ce genre d'exercices prudemment. Renommons donc les diverses composantes de cette fonction.\\

\noindent Soit $l(u,v)=f(g(ue^v),g(v)(1+u^2))$. On a alors:
\begin{enumerate}
    \item $\frac{ \partial l }{ \partial u }(u,v) = \frac{ \partial f }{ \partial x }(\star,\star)g'(ue^v)e^v + \frac{ \partial f }{ \partial y }(\star,\star)g(v)2u$
    \item $\frac{ \partial l }{ \partial v }(u,v) = \frac{ \partial f }{ \partial x }(\star,\star)g'(ue^v)ue^v+\frac{ \partial f }{ \partial y }(\star,\star)g'(v)(1+u^2)$
\end{enumerate}
o\`{u} $(\star,\star)=(g(ue^v),g(v)(1+u^2))$.\\

\noindent Alors $h(u,v)=l(u,v)^{g(u+v)} = e^{g(u+v)\ln(l(u,v))}$ qui est facile à dériver:

\begin{enumerate}
    \item $\frac{ \partial h }{ \partial u } = [g'(u+v)\ln(l(u,v))+\frac{g(u+v)}{l(u,v)}\frac{ \partial l }{ \partial u }(u,v)] l(u,v)^{g(u+v)}$
    \item $\frac{ \partial h }{ \partial v } = [g'(u+v)\ln(l(u,v))+\frac{g(u+v)}{l(u,v)}\frac{ \partial l }{ \partial v }(u,v)] l(u,v)^{g(u+v)}$
\end{enumerate}



\noindent 26.
\begin{enumerate}
\item $(x,y)\rightarrow  3x^2+x^3y+x$.\\
Combinaison linéaire de fonctions continues et différentiables sur $\eR^2$ (Exercice: prouver rigoureusement que les polynômes sont bien des fonctions continues et différentiables sur $\eR^2$).


\item \(  (x,y)\rightarrow\begin{cases}
        e    &   \text{si $ xy\neq 0$}\\
        e^{x+y}    &    \text{sinon}
    \end{cases}\)
N.B.: Il est toujours utile de se représenter le domaine de chacune des fonctions.

\noindent La première remarque est que cette fonction est clairement continue et différentiable en tout point hors de $\{xy=0\}$ (fonction constante). Sur $\{xy=0\}$?
\begin{enumerate}
\item Continuité:\\
Prenons un point dans $\{xy=0\}$, par exemple le point $(a,0)$ (Remarquez que le cas $(0,b)$ est réglé par symétrie). Pour voir si la fonction est continue en ce point il faut voir si \[\lim_{(x,y)\rightarrow (0,0)}f(x,y)=f(0,0)=e^a.\] Si on prend deux manières différentes d'aller vers $(a,0)$ ($y=0$ puis $x=a$) on voit que si $a \neq1$ la fonction ne peut pas être continue. Et en $(1,0)$? Si on $(x,y)\rightarrow (1,0)$ avec d'abord $y=0$ puis $y\neq0$ on aura regardé toutes les manières de tendre vers $(1,0)$. Or dans les deux cas les limites valent $e = f(1,0)$, ce qui prouve que la fonction est continue en $(1,0)$ (et $(0,1)$ par symétrie).

\item Différentiabilité:\\
Comme la fonction est discontinue en tout point $(a,0)$ et $(0,b)$ avec $a\neq1$ et $b\neq1$ elle est aussi non différentiable en chacun de ces points. Il reste donc les points $(1,0)$ et $(0,1)$. Comme toujours, nous regardons d'abord les dérivées directionnelles en $(1,0)$:
\[\frac{ \partial f }{ \partial u }(1,0) \;=\;\lim_{t\rightarrow 0}\frac{f(1+tu_1,tu_2)-e}{t}\]
Il y a deux possibilités: $u_2=0$ et donc $u=(\pm1.0)$ ou$u_2\neq0$ (pourquoi ne regarde-t-on que ces deux cas?).

\begin{enumerate}
\item si $u\neq(\pm1,0)$.\\
    $\frac{ \partial f }{ \partial u }(1,0) \;=\;\lim_{t\rightarrow 0}\frac{e-e}{t}\;=\;0$.
\item si $u=(\pm1,0)$, i.e. si $u=(1,0) = e_1$\\
    $\frac{ \partial f }{ \partial u }(1,0) = \frac{ \partial f }{ \partial x }(1,0)=\lim_{t\rightarrow 0}\frac{f(1+t,0)-e}{t}=\lim_{t\rightarrow 0}\frac{e^{1+t}-e}{t} =^H0$.
\end{enumerate}
\end{enumerate}
\underline{Conclusion}:\\
Si $f$ était différentiable en $(1,0)$, on aurait que sa différentielle prendrait la forme suivante:
\[\begin{array}{cc} df_{(1,0)}u& = \frac{ \partial f }{ \partial x }(1,0)u_1+\frac{ \partial f }{ \partial y }(1,0)u_2\\
    & = eu_1\;\;\forall u\in\eR^2 \end{array} \]
Sa différentielle satisferait également à:
\[  df_{(1,0)}u = \frac{ \partial f }{ \partial u }(1,0) = 0 \;\; \forall u \neq (\pm1,0) \in \eR^2\]
Les deux propriétés étant contradictoires, la fonction $f$ ne peut être différentiable en $(1,0)$ (ni en $(0,1)$ par symétrie).               

\item \( \rightarrow  \begin{cases}
        \frac{ x }{ y }    &   \text{si \( y\neq 0\)}\\
        0    &    \text{sinon}
    \end{cases}\)

Continue et différentiable sur $\eR-\{y=0\}$. Sur l'axe $y=0$ elle n'est pas continue.        
\item \( \rightarrow  \begin{cases}
        x+ay    &   \text{si \( x>0\)}\\
        x    &    \text{sinon}
    \end{cases}\)
Si $a=0$ fonction continue et différentiable sur $\eR^2$. Si $a\neq0$, fonction continue et différentiable partout en dehors de l'axe $x=0$. Sur cet axe, elle est discontinue en tout point sauf en $(0,0)$ où elle est continue. Mais elle n'est pas différentiable en $(0,0)$ car toutes ses dérivées directionnelles  n'y sont pas définies.
\item \(  \rightarrow\begin{cases}
        \frac{ xy^5 }{ x^6+y^6 }    &   \text{si \( x\neq y\)}\\
        0    &    \text{sinon}
    \end{cases}\)
Fonction continue et différentiable partout en dehors de la droite $x=y$.  La fonction est discontinue en chacun des points de cette droite.

\end{enumerate}

30.
\begin{enumerate}
\item $(u,v)\rightarrow  u^3+12u^2v-5v^3$\\
\begin{enumerate}
    \item $\frac{ \partial f }{ \partial u } = 3u^2+24uv$
    \item $\frac{ \partial f }{ \partial v } = 12u^2-15v^2$
\end{enumerate}
\item $(u,v)\rightarrow  f(u^2)\ln(v)$\\
\begin{enumerate}
    \item $\frac{ \partial f }{ \partial u } = 2uf'(u^2)\ln(v)$
    \item $\frac{ \partial f }{ \partial v } = \frac{f(u^2)}{v}$
\end{enumerate}
\item $(x,y)\rightarrow \tan(x+y^2)$\\
\begin{enumerate}
    \item $\frac{ \partial f }{ \partial x } =\frac{1}{cos^2(x+y^2)}$
    \item $\frac{ \partial f }{ \partial v } = \frac{2y}{cos^2(x+y^2)}$
\end{enumerate} 
\item $(r,\theta)\rightarrow  r^\theta$
\begin{enumerate}
    \item $\frac{ \partial f }{ \partial r } =\theta r^{\theta-1}$
    \item $\frac{ \partial f }{ \partial \theta }<++> =\ln(r)r^\theta$
\end{enumerate}
\item $(x,y)\rightarrow (x+3)e^x$
\begin{enumerate}
    \item $\frac{ \partial f }{ \partial x } =e^x(x+4)$
    \item $\frac{ \partial f }{ \partial y } =0$
\end{enumerate}
\item $(u,v)\rightarrow  \ln(f(uv)) $\\

\begin{enumerate}
    \item $\frac{ \partial f }{ \partial u } = \frac{vf'(uv)}{f(uv)}$
    \item $\frac{ \partial f }{ \partial v } = \frac{uf'(uv)}{f(uv)}$
\end{enumerate}\pagebreak
\end{enumerate}

\noindent 32.
\begin{enumerate}
    \item $\frac{ \partial g }{ \partial u } = e^v\frac{ \partial f }{ \partial x }(\star,\star)+2uv\frac{ \partial f }{ \partial y }(\star,\star)$
    \item $\frac{ \partial g }{ \partial v } = ue^v\frac{ \partial f }{ \partial x }(\star,\star)+(1+u^2)\frac{ \partial f }{ \partial y }(\star,\star)$
\end{enumerate}
o\`{u} $\frac{ \partial f }{ \partial x }(\star,\star) = \frac{ \partial f }{ \partial x }(ue^v,v(1+u^2))$ et $\frac{ \partial f }{ \partial y }(\star,\star) = \frac{ \partial f }{ \partial y }(ue^v,v(1+u^2))$.

\noindent 34. $h(t)=f(t,g(t^2))$.\\

\begin{enumerate}
    \item $h'(t)=\frac{ \partial f }{ \partial x }(\star,\star)+\frac{ \partial f }{ \partial y }(\star,\star)2tg'(t^2)$
    \item $ \begin{array}{rl} h''(t)=     &   \frac{ \partial^2f }{ \partial x^2 }(\star,\star)+4tg'(t^2)\frac{ \partial^2f }{ \partial x\partial y }(\star,\star)+4t^2(g'(t^2))^2\frac{ \partial^2f }{ \partial y^2 }(\star,\star) \\           
        & +[2g'(t^2)+4t^2g''(t^2)]\frac{ \partial f }{ \partial y }(\star,\star)\end{array}$

\end{enumerate}
où $(\star,\star) = (t,g(t^2))$.




 \section{Intégration}
 \subsection{Série A}
 Exercice 11
 \begin{enumerate}
   \exr $\int \frac{x^3+3x+1}{x} d x = \frac{x^3}3 + 3x + \ln(x)$%
   \exr $\int x^2d x = \frac{x^3}3$%
   \exr $\int 3(x^2+1)^2 d x = \int 3 x^4 + 6 x^2 + 3 d x = \frac 35
   x^5 + 2 x^3 + 3x$%
   \exr $\int (3x^2 - 6x)^3 (x-1) d x = \frac1{12} (3x^2 - 6x)^4$
 \end{enumerate}

 Exercice 12
 \begin{enumerate}
   \exr $\int \sin^2(x^2+1) \cos(x^2+1) x d x = \frac16
   \sin(x^2+1)^3$%
   \exr $\int \tan(x) d x = -\ln\abs{\cos(x)}$%
   \exr $\int \frac{1}{(2+\sqrt{x})\sqrt x} d x= 2 \ln(2+\sqrt{x})$%
   \exr $\int \frac{\ln(x)}{x(1- \ln^2(x)} d x = \frac12
   \ln\abs{1-\ln^2(x)}$%
 \end{enumerate}



   Travaux perso 2 ---------------

   1. Soit deux réels $x$ et $y$ vérifiant $0 < x < y$. On veut montrer
   que pour tout naturel $k \geq 2$, on a
   \[0 < \sqrt[k]{y} - \sqrt[k]{x} < \sqrt[k]{y-x}.\]

   La première inégalité vient de l'inégalité $x < y$ élevée à la
   puissance $\frac1k$.

   On peut ré-écrire la deuxième, sachant que $x > 0$, en divisant par
   $\sqrt[k]{x}$ pour obtenir
   \[\sqrt[k]{\frac yx} - 1 - \sqrt[k]{\frac yx-1} < 0 \quad \text{
     avec $\frac xy > 1$}\] ce qui s'écrit encore $f(t) < 0$ en posant
   $f(t) \pardef \sqrt[k]t - \sqrt[k]{t-1} - 1$. On peut alors étudier
   la fonction $f$. Étant donné que $f(1) = 0$, il suffirait que $f$
   soit strictement décroissante sur $]1;\infty[$ pour qu'on ait
   l'inégalité voulue, à savoir $f(t) < 0$ dès que $t > 1$.

   Pour le montrer, on voit que
   \[f^\prime(t) = \frac 1k \left(t^{\frac{1-k}k} -
     ({t-1})^{\frac{1-k}k}\right)\] d'où on tire les équivalences
   suivantes
   \begin{align}
     & & f^\prime(t) < 0\\
     &\ssi& t^{\frac{1-k}k} < ({t-1})^{\frac{1-k}k}\\
     &\ssi& t^{1-k} < ({t-1})^{1-k}\\
     &\ssi& t > t-1\\
     &\ssi& 0 > -1
   \end{align}
   où la dernière inégalité est manifestement vraie, ce qui prouve la
   première inégalité et achève l'exercice.

   2.


 \paragraph{Exercice 1}
 \begin{enumerate}
 \item Par exemple, $B(x,r)$ avec $x \in \eR^n$ et $r > 0$.

 \item On utilise la densité de $\eQ$ dans $\eR$ pour voir que $B(q,r)$
   ($q \in \eQ^n$ et $r > 0$) est également une base.

   On observe ensuite que seuls les $r$ \og petits\fg{} sont utiles,
   donc on se restreint aux boules de la forme $B(q,1/n)$ ($q \in
   \eQ^n$ et $n \in \eN_0$). Cet ensemble de boules est une base
   dénombrable\marginpar{Pourquoi ?} de la topologie usuelle sur
   $\eR^n$.
 \end{enumerate}

 \paragraph{Exercice 2}
 \emph{Principe.} L'idée est de considérer une propriété topologique
 (invariante par homéomorphisme) et de voir qu'elle est vérifiée par
 les ouverts de $\eR^2$ mais pas ceux du cône.

 \begin{lem}Si $V$ est un voisinage de $0$ sur le cône $C$, alors
   $V\setminus\{0\}$ n'est pas connexe, donc n'est pas connexe par
   arc.\end{lem}
 \begin{proof}Le cône $C$ est la réunion de $C^+ = C \cap
   \left(\eR^2\times \eR_0^+\right)$ et $C^- = C \cap \left(\eR^2\times
     \eR_0^-\right)$ car le seul point à cote nulle est la singularité
   $0$. Dès lors, $V$ s'écrit comme l'union disjointe de $V\cap C^+$
   et $V\cap C^-$, qui sont non-vides. Donc $V$ n'est pas
   connexe.\end{proof}

On procède en deux étapes, en montrant d'abord qu'il
   existe des points en \og dessous\fg{} et au \og dessus\fg{} de
   $0$, puis en essayant de les relier.
     Comme $V$ est un voisinage de $0$, il existe un ouvert $U$ du
     cône centré en $0$ inclut à $V$. Donc par définition de la
     topologie induite, et puisque les boules forment une base de la
     topologie de $\eR^3$, il existe une boule $B$ centrée en $0$ dont
     $U$ est la trace sur $C$, telle que $0 \in (B \cap C) \subset
     V$. On choisit $p = (p_x,p_y,p_z) \in (B \cap C)$, et en
     considérant $p^\prime = (p_x, p_y, -p_z)$ on a ainsi trouvé deux
     points qui vérifient $p_z > 0$ et $p^\prime_z < 0$ (au besoin,
     on les échange).

   \begin{enumerate}
   \item Supposons que $V\setminus\{0\}$ soit connexe par arc. Donc
     il existe un chemin
     \[\gamma : [0;1] \to V\setminus\{0\} : t \mapsto
     (\gamma_x(t),\gamma_y(t),\gamma_z(t))\] qui relie $p$ à
     $p^\prime$ et qui vérifie $\gamma_z(0) = p_z > 0$ et
     $\gamma_z(1) = -p_z < 0$. Or $\gamma_z(t)$ est une fonction
     continue (car $\gamma$ est continu), donc par le théorème des
     valeurs intermédiaires, il existe $\bar t$ qui vérifie
     $\gamma_z(\bar t) = 0$. Or le seul point de $C$ dont la cote
     (coordonnée en $z$) soit nulle est le sommet $0$ qui n'est pas
     dans $V\setminus\{0\}$, d'où la contradiction.
   \end{enumerate}

 \begin{rem}Soient deux espaces topologiques $E$ et $F$, et $f :
   E\to F$ un homéomorphisme. Pour toute partie $A$ de $E$,
   l'espace $E\setminus A$ est homéomorphe au sous-espace $F\setminus
   f(A)$ via la restriction $f_{\vert E\setminus A}$.\end{rem}

 \begin{lem}Soient deux espaces topologiques $E$ et $F$, et $f :
   E\to F$ un homéomorphisme. $E$ est connexe par arc si et
   seulement si $F$ l'est.\end{lem}
 \begin{proof}On montre en réalité que l'image d'un connexe par arc
   par une application continue est un connexe par arc, ce qui
   implique chaque sens de l'équivalence de l'énoncé.

   Soient $p$ et $q$ des points de $F$. Il existe un chemin reliant
   un antécédent de $p$ et un antécédent de $q$ (dans $E$). L'image
   de ce chemin est un chemin reliant $p$ et $q$ (dans $F$) puisque
   composé d'applications continues.
 \end{proof}

 \begin{lem}Une sphère de $\eR^n$ est connexe par arc si $n >
   1$\end{lem}
 \begin{proof}On voit qu'un cercle est connexe par arc car on a une
   paramétrisation en sinus et cosinus. Pour une sphère $S$ de centre
   $a$ en dimension $n > 2$, on se donne $p$ et $q$ sur $S$ et on
   définit $P$ le plan affine passant par $a$, $p$ et $q$. Alors $P
   \cap S$ est un cercle, donc on peut relier $p$ à $q$ par un chemin
   dans cette intersection.

   Pour voir sur une formule que $P \cap S$ est un cercle, on peut
   écrire $x - a = \lambda(a-p) + \mu(a-q)$ l'équation (en $x$) du
   plan $P$, et $\module{x-a}^2 = R^2$ l'équation (en $x$) de la
   sphère. En injectant, on obtient une équation du second degré en
   $\lambda,\mu$ qui se révèle être l'équation d'un cercle à une
   transformation affine près.
 \end{proof}

 \begin{lem}Un ouvert connexe par arc dans $\eR^n$ ($n \geq 2$) reste
   connexe par arc même si on lui enlève un point.\end{lem}
 \begin{proof}
   En effet, soit $U$ un tel ouvert connexe par arc, et $p$ un point
   de $U$. Soient $x$ et $y$ sur $U\setminus\{p\}$. Il existe un
   chemin $\gamma$ de $x$ à $y$. Si le chemin ne passe pas par $p$,
   c'est gagné. Si il passe par $p$, on choisit une boule $B$ fermée
   (de rayon non-nul) centrée en $p$ qui ne contient ni $x$ ni
   $y$. On note
   \[E = \gamma^{-1}(B) \subset [0;1]\] c'est un ensemble compact
   (fermé, par continuité de $\gamma$, et borné) dont on regarde le
   maximum $\bar t$ et le minimum $\underline t$.

   Il reste enfin à définir un chemin entre $p$ et $q$ par morceaux
   \begin{enumerate}
   \item Les points $p$ et $\gamma(\underline t)$ sont reliés par
     $\gamma$,
   \item Par connexité par arc, il existe un chemin sur la sphère qui
     relie $\gamma(\underline t)$ à $\gamma(\bar t)$,
   \item et enfin $\gamma(\bar t)$ et $q$ sont reliés via $\gamma$;
   \end{enumerate}
   ce qui achève la construction d'un chemin continu entre $p$ et
   $q$.
 \end{proof}
 Pour conclure l'exercice, par l'absurde, on prend un voisinage
 connexe et ouvert $V$ de $0$ dans le cône, homéomorphe à un ouvert
 connexe $U$ de $\eR^2$. Or $V\setminus\{0\}$ n'est pas connexe par
 arc, alors que l'ouvert dont on retire un point reste connexe par
 arc. C'est impossible, donc l'homéomorphisme n'existe pas, et le
 cône n'est pas une variété de dimension $2$.


\chapter{Des corrections pour les pharmaciens (Bruxelles)}
% This is part of the Exercices et corrigés de mathématique générale.
% Copyright (C) 2009
%   Laurent Claessens
% See the file fdl-1.3.txt for copying conditions.
\paragraph{page 33, 2e quadri, séance 3}

\subparagraph{ex 23}
D'après la page 16bis, une \emph{fonction sinusoïdale} est une fonction de la forme $f(x) = a \sin(\omega x + \varphi)$. Pour avoir $f(0) = 0$, on peut par exemple choisir $\varphi = 0$. Par ailleurs, le maximum de la fonction sinus est $1$, donc la maximum de $f(x)$ est $a$ ; choisissons donc $a = 2$. Reste à déterminer $\omega$ pour qu'un maximum soit atteint en $x = 1$. Or on sait qu'un maximum de la fonction sinus est atteint en $\frac\pi2$ par exemple ; choisissons donc $\omega = \frac\pi2$, de sorte que si $x = 1$, on obtient $2 \sin(\frac\pi2) = 2$ qui est bien le maximum de $f$.

La fonction choisie est donc $f$ définie par $f(x) = 2 \sin(\frac \pi 2 x)$.

\subparagraph{ex 25}
Notons $f$ la fonction définie par
\begin{equation*}
	f(x) = \frac{x2+px+q2}{x}
\end{equation*}
et calculons
\begin{equation*}
	f^\prime(x) = \frac{x2-q2}{x2}
\end{equation*}
de sorte que $f^\prime(x) = 0$ si et seulement si $x = \pm q$ (avec $q \neq 0$, pour que la dérivée ait un sens). On remarque également, par le signe de la dérivée, que l'extrémum obtenu en $-| q |$ est un maximum, l'autre étant un minimum.

Dès lors nous aurons deux cas possibles : si on impose $f(-q) = 0$ et
$f(q) = 4$, on obtient $q = 1$ et $p = 2$, et si on impose $f(-q) = 4$
et $f(q) = 0$, on obtient $q = -1$ et $p = 2$.

Dans les deux cas, les extrémas sont $(-1,0)$ (maximum) et $(1,4)$
(minimum).

\subparagraph{ex 29}
Si le carton fait $a$ de largeur et $b$ de hauteur, alors la surface
imprimable --tenant compte des marges-- est $(a-3)(b-2)$ et est fixée
à $54$. On veut donc minimiser la surface de carton, donnée par $ab$,
sachant que $a = 3 + \frac{54}{b-2}$.

Définissons $f(b) = 3b + \frac{54b}{b-2}$, et trouvons-en le
minimum. Sa dérivée est
\begin{equation*}
  f^\prime(b) = 3 + \frac{54(b-2) - 54b}{(b-2)2} = \frac{3 (b-2)2 - 108}{(b-2)2}
\end{equation*}
et s'annule pour $b = -4$ (à rejeter, n'a pas de sens pour une
longueur) ou $b = 8$. D'après le signe de la dérivée, $b = 8$ fournit
bien un minimum.

En conclusion, $b = 8$ et $a = 12$.

\subparagraph{ex 30} Soient $a$ et $b$ ces nombres. On sait $a, b \geq
0$ et $a+b = 12$, donc $b = 12 - a$.
\begin{enumerate}
\item On veut minimiser $a2+ b2 = a2 + (12-a)2 = f(a)$. La dérivée
  $f^\prime(a) = 2a - 2 (12 - a)$ s'annule pour $a = 6$. La solution
  est donc $a = b = 6$.

\item On veut maximiser $a b2$ (ou $ba2$, mais il suffit d'échanger
  les nombres pour retomber sur le premier cas). On définit $f(a) = a
  (12-a)2$, et la dérivée
  \begin{equation*}
    f^\prime(a) = (12-a)2 - 2 a (12-a)
  \end{equation*}
  s'annule lorsque $a = 12$ (mais alors $b = 0$, à rejeter, ceci n'est
  pas un maximum) ou lorsque $a = 4$ ; les solutions sont donc $(4,8)$
  et le symétrique $(8,4)$.

\item On veut maximiser $ba3$ (même remarque que ci-dessus), donc on
  définit
  \begin{math}
    f(a) = (12-a) a3
  \end{math}
  dont la dérivée est
  \begin{equation*}
    f^\prime(a) = -a3 + 3 (12-a)a2
  \end{equation*}
  et s'annule pour $a = 0$ (pas un maximum) ou $9 = a$ ; donc les
  solutions sont $(9,3)$ et $(3,9)$.
\end{enumerate}

\subparagraph{ex 32}
On cherche $(x,y)$ tel que $y2 = 4ax$ et minimisant la distance
\begin{equation*}
d = \| (x,y)-(2a,a) \| = \sqrt{(x-2a)2 + (y-a)2} =
\sqrt{\left(\frac{y2}{4a}-2a\right)2 + (y-a)2}
\end{equation*}

Remarquons que minimiser $d$ revient à minimiser $d2$, donc posons
\begin{equation*}
f_a(y) = \left(\frac{y2}{4a}-2a\right)2 + (y-a)2
\end{equation*}
et calculons la dérivée
\begin{equation*}
  f_a^\prime(y)  = 2 \left(\frac{y2}{4a}-2a\right) \frac y a + 2
  (y-a) = \frac{y3}{4a2}-2a
\end{equation*}
qui s'annule lorsque $y = 2a$. Donc la solution est $(x,y) = (a,2a)$

\subparagraph{ex 33}
Soit $x$ la distance ``sur le rivage'' par rapport au premier bâteau
où sera débarqué le passager ($x$ est entre $0$ et $5$). Alors il
s'agit de minimiser
\begin{equation*}
  d(x) = \sqrt{9+x2} + \sqrt{(5-x)2 + 81}
\end{equation*}
donc on calcule la dérivée
\begin{equation*}
  d^\prime(x) = \frac{x}{\sqrt{9+x2}} - \frac{5-x}{\sqrt{(5-x)2 + 81}}
 = \frac{x\sqrt{(5-x)2 + 81} + (x-5)\sqrt{9+x2}}{\sqrt{9+x2}\sqrt{(5-x)2 + 81}}
\end{equation*}
qui s'annule lorsque $3 |x| = |x-5|$ càd lorsque $3x = 5 - x$
(car $x \in [0,5]$), donc $x = \frac{5}{4}$.

Le trajet minimal du bâteau est donc $d(\frac{5}{4}) = 13$.

Une autre de manière de voir le problème est de considérer le principe
de réflexion : le trajet minimal est alors donné par $\sqrt{52 +
  122} = 13$.



%TODO : il faudra réhabiliter ces exercices.
%\chapter{Correction des exercices de rappel}    % Correspond à CdI1
%Ces corrections correspondent à des énoncés que je ne suis pas certain d'encore avoir.
%% This is part of Exercices et corrigés de CdI-1
% Copyright (c) 2011-2012,2014
%   Laurent Claessens
% See the file fdl-1.3.txt for copying conditions.

%\usepackage{graphicx}
%\newcommand{\e}{\'{e}}
%\newcommand{\ee}{\`{e}}
%\newcommand{\ac}{\`{a} }
%\newcommand{\f}{\frac}
%\newcommand{\arcth}{{\rm arctanh}}
%\newcommand{\arcsh}{{\rm arcsinh}}
%\newcommand{\arcch}{{\rm arccosh}}
%\newcommand{\csec}{{\rm cosec}}
%\newcommand{\cotan}{{\rm cotg}}
%\newcommand{\cis}{(\cos+i\sin)( }
%\newcommand{\Rn}{\rm {I\!\!\, R}} 

%+++++++++++++++++++++++++++++++++++++++++++++++++++++++++++++++++++++++++++++++++++++++++++++++++++++++++++++++++++++++++++
					\section{Nombres complexes}
%+++++++++++++++++++++++++++++++++++++++++++++++++++++++++++++++++++++++++++++++++++++++++++++++++++++++++++++++++++++++++++

\[ \begin{tabular}{| c || c || | c | | c | | c | | c | c | }
\hline  no &         ${ 1}                                                         $    & no     &        ${ 2}  $                           	         & no         &      ${ 3}  $            \\    \hline \hline
           a   &      $4+5i                      $  				            & a  &   $4\cis{\frac{5\pi}{6}})$     		                  &a&  $\pm(\frac{\sqrt 2}{2}+ \frac{\sqrt 2}{2}i)$  \\ \hline 
           b  &      $11i                                        $   			   & b  &  $\frac{-1+i\sqrt 3}{4}=\frac{ 1 }{2} e^{\frac{ 2i\pi }{ 3 }}      $          		          &b & $3\cis \frac{\pi}{6}+\frac{2k\pi}{6})$ \\ \hline
           c   &      $\frac{31}{21}-\frac{7}{20}i                      $   		   & c  & $\cis\frac{\pi}{6})	 $   				   &c&  $\cis \frac{\pi}{6}+\frac{2k\pi}{6})$  \\ \hline 
           d   &      $-1+7i                                  $   			   & d &   $\sqrt 2\cis\frac{-3\pi}{4})$ 			        &d&  $ 2\cis -\frac{\pi}{6}+\frac{2k\pi}{6}) $\\ \hline 
           e   &      $58          $   						   & e &  $4   	\cis\frac{\pi}{6})	 $			&e& $\cis \frac{2k\pi}{3})$ \\ \hline
           f   &      $\frac{23}{30}-\frac{2}{15}i   $    			            & f  & $16      $       		 
& &  \\ \hline
           g   &      $\frac{12+5i}{13}   			 $   			   & g  & $8i$                  					 & &  \\ \hline
           h   &    $\frac{5-7i}{74}  			 $ 			    	   & h  & $\frac{-1-\sqrt3 i}{2}= e^{\frac{ 2i\pi }{ 3 }}$                                         & & \\ \hline
           i    &      $      	\frac{6+2\sqrt 7 i}{64}         $  			   & i  & $ \frac{1}{8}$                &          &                           \\  \hline
           j    &      $      	1				         $    		  & j  & $ \sqrt 2 \cis \frac{-\pi}{4} )$                &          &                           \\  \hline
          k   &      $      	1-12i				         $    		  &  &                 &          &                           \\  \hline           
\end{tabular} \]


\noindent{ Exercice $4$}\\
$(a)$ { Montrer que si $(x+iy)$ est une racine carrée de $(a+ib)$ o\`{u} $x, y, a, b \in \eR^n$, alors $x$ et $y$ sont solutions des équations }
\[ \begin{array}{c} x^2-y^2=a\\ 
             			2xy=b \end{array}\]
\\


\noindent Pour le voir il suffit d'écrire $(x+iy)^2 = a+ib$ et d'égaliser partie réelle à partie réelle, partie imaginaire à partie imaginaire.\\
$(b)$\hspace{0,3cm} $\pm(\frac{\sqrt 2 }{2}(3+i))$

\[ \begin{tabular}{| c || c || | c | | c | |}
\hline  no &         ${ 5}                                                         $ 		   & no     &                            	           \\    \hline \hline
           a   &  $    \frac{1\pm i}{2}                    $  				            & f 	 	&   		                    \\ \hline 
           b  &      $ \frac{3i \pm \sqrt23}{4}                                       $   	   & g  &            		          \\ \hline
           c   &      $\frac{2i}{1-i}$,$-2\frac{1+i}{1-i}                      $   		   & h  & $-2+2^{1/5} \cis\frac{\pi}{15}+\frac{2k\pi}{5})	 $   				    \\ \hline 
           d   &      $                             $   			  	   & i &   		    \\ \hline 
         e   &      $\pm\sqrt{1+\sqrt2}\cis \frac{\pi}{4}) $ ,    						   &   &     	 	\\
              &      $\pm\sqrt{\sqrt2-1}\cis \frac{-\pi}{4})   $   						   &   &                \\ \hline
          \end{tabular} \]
          
\noindent{ Exercice $6$. 
Par application de la formule donnant $z^n$, exprimez $\cos2x$, $\cos3x$ et $\cos4x$ en fonction de puissances de $\cos x$.}
          \\
          
          
\noindent Poser $z=\cos x + i\sin x$. On sait alors que \[z^n=\underbrace{(\cos x+i\sin x)^n}_*=\cos nx + i \sin nx.\] Il faut donc développer $*$ par la formule du binôme et égaler les deux membres. Par exemple:\[\begin{array}{cccrcc}
               						z^2= & (\cos x+i\sin x)^2 & = & \cos^2 x-\sin^2 x+2i\cos x\sin x &   & \\
							          & 	\Longrightarrow    &    &          \cos^2 x-\sin^2 x       	&= &	\cos2x \\
							           & 	\Longrightarrow    &      &          \cos2x    	&= &	 2\cos^2x-1\end{array}\]

\noindent On trouvera alors: \[ \begin{array} {rl}
					\cos3x&=4\cos^3x--3\cos x\\
					\cos4x&=8\cos^4x-8\cos^2x+1\end{array}\]


\noindent{ Exercice $7$}\\
$a=2$, $b=-\frac{\sqrt2}{4}$.


\section{Graphes de quelques fonctions qu'il est bon de connaître}

% TODO: refaire les dessins manquants.

Les exponentielles sont à la figure \ref{LabelFigDessinExp}.
\newcommand{\CaptionFigDessinExp}{Des exponentielles}
\input{auto/pictures_tex/Fig_DessinExp.pstricks}

%Les exponentielles sont à la figure \ref{LabelFigDessinExp}, 
%\newcommand{\CaptionFigDessinExp}{Des exponentielles}
%\input{auto/pictures_tex/Fig_DessinExp.pstricks}

%Les logarithmes sont à la figure \ref{LabelFigDessinLog}
%\newcommand{\CaptionFigDessinLog}{Des logarithmes. Aucune des deux ne monte très vite, et plus la base augmente, moins ça monte vite.}
%\input{auto/pictures_tex/Fig_DessinLog.pstricks}

%À la figure \ref{LabelFigDessinAbs} se trouvent mes valeurs absolues
%\newcommand{\CaptionFigDessinAbs}{Des graphiques de valeurs absolues}
%\input{auto/pictures_tex/Fig_DessinAbs.pstricks}

%À la figure \ref{LabelFigDessinHyperbolique} se trouvent mes fonctions hyperboliques
%\newcommand{\CaptionFigDessinHyperbolique}{Des graphiques de fonctions hyperboliques}
%\input{auto/pictures_tex/Fig_DessinHyperbolique.pstricks}

%Le graphe de $\sin(x)/x$ est sur la figure \ref{LabelFigSinxx}
%\newcommand{\CaptionFigSinxx}{La fonction $x\mapsto\frac{ \sin(x) }{ x }$}
%\input{auto/pictures_tex/Fig_Sinxx.pstricks}

%Le graphe de $\cos(x)/x$ est à la figure \ref{LabelFigCosxx}.
%\newcommand{\CaptionFigCosxx}{La fonction $x\mapsto\frac{ \cos(x) }{ x }$}
%\input{auto/pictures_tex/Fig_Cosxx.pstricks}

%D'autres valeurs absolues à la figure \ref{LabelFigAbsx}
%\newcommand{\CaptionFigAbsx}{La fonction valeur absolue et quelque autres.}
%\input{auto/pictures_tex/Fig_Absx.pstricks}

\section{Intégration}

Petite note stratégique pour l'intégration par partie. Une partie de l'art est de choisir correctement que $u$ et quel $dv$ on prend. Il y a une petite règle qui permet de choisir assez souvent : l'ordre de priorité du choix pour $u$ est
\let\OldTheEnumi\theenumi
\renewcommand{\theenumi}{\arabic{enumi}}
\begin{enumerate}
\item $\ln x$
\item $x^n$
\item $ e^{kx}$,
\end{enumerate}
\let\theenumi\OldTheEnumi
c'est à dire que s'il y a un logarithme, il faut le choisir comme $u$; s'il n'y a pas de logarithme, mais une puissance de $x$, alors il faut choisir la puissance de $x$; et s'il n'y a ni logarithme ni puissance, alors on choisit l'exponentielle s'il y en a.

%--------------------------------------------------------------------------------------------------------------------------- 
\subsection{Un exemple d'intégrale pas simple}
%---------------------------------------------------------------------------------------------------------------------------



Dans la méthode de l'intégration de fraction rationnelles, l'apothéose est de devoir intégrer 
\begin{equation}
	K_s=\int \frac{ dt }{ (1+t^2)^s }.
\end{equation}
Afin de prouver la formule de récurrence, nous commençons par écrire le numérateur sous la forme $(1+t^2-t^2)dt$ :
\begin{equation}
	K_s=\int \frac{ 1 }{ (1+t^2)^{s-1} }-\int \frac{ t^2 }{ (1+t^2)^s }.
\end{equation}
Le premier terme vaut $K_{s-1}$, tandis que nous intégrons le second par partie en posant
\begin{equation}
	\begin{aligned}[]
		u&=t		&	dv&=\frac{ t }{ (1+t^2)^s }\\
		du&=dt		&	v=&\frac{1}{ 2(s-1) }\cdot \frac{1}{ (1+t^2)^{s-1} }.
	\end{aligned}
\end{equation}
Nous tombons sur
\begin{equation}
	K_{s}=K_{s-1}-\left( \frac{ t }{ 2(1-s)(1+t^2)^{s-1} } - \frac{1}{ 2(1-s) }\int \frac{ dt }{ (1+t)^{s-1} } \right)
\end{equation}
Le dernier terme donne encore un multiple de $K_{s-1}$.

\[ \begin{tabular}{| c || c || | c | | c | | c | | c | c | }
\hline  no &         ${ 11}                                                         $    & no     &        ${ 12}  $                           	         &          &       \\    \hline \hline
           $1$   &      $\frac{x^3}{3}+3x+\ln(x)                        $  & $1$  &   $\frac{\sin^3(x^2+1)}{6}$     		&$10$&  $\frac{2}{3}(2+t)^{3/2}$  \\ \hline 
           $2$   &      $ \frac{x^3}{3}                                              $   & $2$  &  $(t^2+6)^{3/2}       $          		&$11$&  $\frac{2}{3}(1+e^x)^{3/2}$ \\ \hline
           $3$   &      $3\frac{x^5}{5} + 2x^3+3x                         $   & $3$  & $\sin(4+y^3)	 $      &$12$&  $\frac{1}{9}(1+e^{3x})^3$  \\ \hline
           $4$   &      $\frac{4x^{7/2}}{7}                                       $   &  $4$ & $\ln| x+2|$    					&$13$&  $ 2\ln  | 2+\sqrt x | $\\ \hline 
           $5$   &      $\frac{y^5}{5}+2y-\frac{y^{-3}}{3}                   $   &  $5$ &  $-\ln| \cos x|$       		&$14$		&$-\frac{1}{ 2 }\ln\big( 1-\ln(x)^2 \big)$ \\ \hline
           $6$   &      $\frac{x^2}{2}+\frac{x^3}{3}-\frac{4x^{5/2}}{5}	 $    & $6$  & $\frac{-1}{14}\ln| 2 - 7 x^2|       $        &$15$& $\ln  |x+ \sin x | $ \\ \hline
           $7$   &      $\frac{3x^2-6x)^4}{24}   			 $    & $7$  & $ \ln |1 + \ln x|  $                  &$16$& $\frac{\sin^2x}{2}$  \\ \hline
                     &                        					      & $8$  & $-\frac{1}{\sin( x )}$                         &$17$ & $2e^{\sqrt x} + \frac{2}{3}x^{3/2} $\\ \hline
                     &      $      					         $     & $9$  & $ \frac{1}{16}\ln  | b+4x^4 | $                &          &                           \\  \hline
                 
             
\end{tabular} \]


\[ \begin{tabular}{| c || c || | c | | c |c | | c | }
\hline  no &         ${ 13}                                          $    & no    &  ${ 14}  $                                                              & no & ${ 17}  $       \\    \hline \hline
           $1$   &      $      \frac{5^{2x}}{2\ln 5}                  $        & $1$  &  $ -\frac{1}{3} \arctan\frac{x-1}{3}$                               & $1$ & $2\sin \sqrt x$			  \\ \hline 
           $2$   &      $        \frac{e^{\sin 2t}}{2}                  $       & $2$  &  $\frac{1}{2\sqrt 5 }\ln | \frac{x-\sqrt 5 }{x+\sqrt 5} |  $ & $2$ & $\tan (x) - x$         		\\ \hline
           $3$   &      $2e^{\sqrt x }                              $        & $3$  &  $\frac{1}{5}\ln | \frac{x }{x- 5} |$                                     & $3$ &  $\frac{2\sin(\frac{3x}{2})}{3(\cos(\frac{3x}{2})-\sin(\frac{3x}{2}))}$	\\ \hline  
           $4$   &      $ e^{\tan y}                                   $      &  $4$ &  $\frac{1}{2}\arctan \frac{x}{2}$    	                                & $4$ &	 $\frac{1}{\pi}\sin(\pi x)$					\\ \hline 
           $5$   &      $\frac{e^{4x}-e^{-4x}}{4}-2x               $        &  $5$ &  $\frac{1}{6}\ln|\frac{-1+3x}{1+3x} |$       		    &         &		 \\ \hline
                   &              							   & $6$  &  $\frac{4}{\sqrt 26}\arctan\frac{5x+3}{\sqrt 26} $ &         &       \\ \hline
                   &       			                                      & $7$  &  $\frac{4}{13}\ln|\frac{-2+x}{3+5x} | $                           &          &      \\ \hline
                        
\end{tabular} \]

\[ \begin{tabular}{| c || c || | c | | c | | }
\hline       no &         ${ 15}                                          $                                   &  no    &  ${ 16}  $                                                                                              \\    \hline \hline
             $1$   &      $      \frac{1}{3}\arcsin \left(\frac{x-1}{3}\right)                  $          & $1$  &  $ -\frac{x^4}{4} \arctan x - \frac{x^3}{12}+\frac{x}{4}-\frac{1}{4}\arctan x$    \\ \hline 
            $2$   &      $      \arcch(x+1)              $                                                     & $2$  &  $\frac{x^4\ln| x|}{4}-\frac{x^4}{16} $          						\\ \hline
                      &   and           $   2\arcsh(\sqrt{\frac{x}{2}}) $         			        & $3$  &  $\frac{e^x}{1+x}$       									\\ \hline 
           $4$   &        $  7^{-1/2}\arcsh\left( \frac{-4+7x}{\sqrt 47} \right)               $                               &  $4$ &  $-\frac{1}{4}(-1+2y^2)\cos 2y + \frac{y}{2}$    					\\ \hline 
         $4$   &      $ \arcch(2x+3)     $       							&  $5$ &  $\frac{1}{27}e^{3x}(2-6x+9x^2)$       				 \\ \hline
           $5$&       $\frac{2\sqrt{-2+x} \ln |\sqrt{-2 + x} +\sqrt{x}| }{\sqrt{2-x}}   $               & $6$  &  $\frac{\cos^2 x}{2}-\frac{1}{8}\cos 4x $        \\ \hline
            $6$   &      $-2\sqrt{3}\arcsin\left(\frac{-2-3x}{\sqrt{19}}\right)$                & $7$  &  $\frac{1}{4}(x^2-2\cos x -2x\sin x) $                \\ \hline
                $7$      &       			     $\frac{6}{\sqrt{7}}\arcsh\left(\frac{-4+7x}{\sqrt 45}\right)$                                                                      & $8$  &  Pas exprimable comme fonction élémentaire             \\ \hline
                   &       			                                                                         & $9$  &   $-x\cos x +\sin x$           \\ \hline
                    &       			                                                                         & $10$ &  $(x-1)e^x$            \\ \hline
                    &       			                                                                         & $11$ &  $2x\cos x + (x^2-2)\sin x$            \\ \hline
  \end{tabular} \]

 
\[ \begin{tabular}{| c || c ||  }
\hline       no &         ${ 18}                                          $                                                                                                               \\    \hline \hline
             $1$   & $\frac{1}{12}(4x^3+6\arctan x +3\ln(\frac{x-1}{x+1})      $            \\ \hline 
            $2$   &  $ x+\frac{4}{\sqrt 3}\arctan(\frac{1+2x}{\sqrt 3})+ \frac{2}{3}\ln(x-1)- \frac{1}{3}\ln(1+x+x^2) +\frac{4}{3}\ln(x^3-1)$                 \\ \hline

           $3$   &      $  - \frac{1}{\sqrt 3}\arctan(\frac{1+2x}{\sqrt 3})     +\ln(x)- \frac{1}{2}\ln(1+x+x^2)       $ 						\\ \hline 
           $4$   &      $   \frac{1}{4\sqrt 2}(-2\arctan(1-\sqrt2x)+2\arctan(1+\sqrt2x)+\ln(\frac{1+\sqrt2x+x^2}{-1+\sqrt2x-x^2})             $                                                 				\\ \hline 
              $5$   &      $   -3\arctan x+\frac{1}{2}\ln(1+x^2)$   				 \\ \hline
               $6$  &      $-\frac{2}{x}-\sqrt2\arctan(\frac{x}{\sqrt2})+\frac{1}{2}\ln(2+x^2)$                       \\ \hline
              $7$ &      $\frac{1}{3\sqrt5}\arctan(\frac{-2+3x}{\sqrt5}) -\frac{1}{6}\ln(3-4x+3x^2) $                        \\ \hline
               $8$ &     $-2x+\frac{x^2}{2}+4\ln(3+2x)$  			                                                                                \\ \hline
              $9$     &     $\frac{1}{2}\ln x -\frac{1}{2}\ln(2+x)$  			                                                                       \\ \hline
               $10$  &  A voir...     			                                                               \\ \hline
              $11$  &  $\frac{3}{2}\arctan(2x)+\frac{1}{4}\ln(1+4x^2)$			                                                               \\ \hline   
               $12$ &  $2\sqrt2\arctan(\frac{-4+x}{\sqrt2})+\frac{1}{2}\ln(18-8x+x^2)$			                                                               \\ \hline   
              $13$  &  $-\frac{1}{22}(11+3\sqrt11)\ln(3+\sqrt{11}+x)+-\frac{1}{22}(11-3\sqrt11)\ln(-3+\sqrt{11}-x)$			                                                               \\ \hline   
              $14$  &  $\frac{9}{5\sqrt{14}}\arctan(\frac{-1+5x}{\sqrt{14}})-\frac{1}{10}\ln(3-2x+5x^2)$			                                                               \\ \hline   
              $15$  &  $\frac{x^2}{2}+\arctan x-\frac{1}{2}\ln(1+x^2)$			                                                               \\ \hline   
              $16$  &  $4x-\frac{15}{2}\arctan(\frac{x}{2})$			                                                               \\ \hline   
              $17$  & A voir...		                                                               \\ \hline   
              $18$  &  $-\ln x +3\ln(-1+3x)-\ln(1+3x)$			                                                               \\ \hline   
              $19$  &  $\ln(-1+ x) +3\ln(x)-2\ln(2+x)$			                                                               \\ \hline   
              $20$  & A voir...	                                                               \\ \hline   
              $21$  &  $\frac{1}{4\sqrt2}(-2\arctan(1-\sqrt2 x)+2\arctan(1+\sqrt2x)+\ln(\frac{-1+\sqrt2x-x^2}{1+\sqrt2x+x^2})$			                                                               \\ \hline   
\end{tabular} \]


\[ \begin{tabular}{| c || c ||  }
\hline       no &         ${ 18}                              $       mal classés                                                                                                        \\    \hline \hline
             $1$   & $-\frac{\cos x}{2}-\frac{1}{10}\cos(5x)      $            \\ \hline 
            $2$   &  $ \frac{x}{2}-\frac{1}{4}\sin(2x)$                 \\ \hline

           $3$   &      $  \arctan(\ln(x)) $ 						\\ \hline 
           $4$   &      $   \frac{1}{2}\ln[\frac{-1+\ln x}{1+\ln x}]            $                                                 				\\ \hline 
           $5$	  &      $x$                                                       \\ \hline   
\end{tabular} \]





\[ \begin{tabular}{| c || c ||  }
\hline       no &         ${ 19}                                          $                                                                                                               \\    \hline \hline
             $1$   & $-x+\frac{1}{3}\tan(3x)$            \\ \hline 
            $2$   &  $ \ln(\cos x)+\frac{1}{2}\sec^2(x)$                 \\ \hline
           $3$   &   A voir...						\\ \hline 
           $4$   &      $   \frac{1}{ab}\arctan(\frac{b\tan y}{a})             $     \\ \hline 
              $5$   &      $  \frac{x}{5}+\frac{4}{15}\ln(3\cos x-\sin x) + \frac{1}{3}\ln(\sin x)    $   				 \\ \hline
               $6$  &      $-\frac{2}{\sqrt{-a^2+b^2}}\arcth(\frac{(a-b)\tan(\frac{x}{2})}{\sqrt{-a^2+b^2})}$                \\ \hline
              $7$ &     A voir...                       \\ \hline
               $8$ &     $\ln(-7 + \cos(2 x))$  			                                                                                \\ \hline
              $9$     &     $\frac{\tan[x]}{2}$  			                                                                       \\ \hline
               $10$  &  $x - \frac{1}{3}\tan(\frac{3x}{2})$  			                                                               \\ \hline
              $11$  &  $\frac{3\sin(x)}{4}+\frac{1}{12}\sin(3x)$			                                                               \\ \hline   
               $12$ &  $\frac{1}{2440}[1225\sin x +245\sin 3x+49\sin 5x + 5\sin 7x]$			                                                               \\ \hline   
              $13$  &  $\frac{1}{192}[-60+60x -\sin(6-6x) -9\sin(4-4x)+45\sin(2-2x)]$			                                                               \\ \hline   
              $14$  &  $\frac{1}{384}[24x -3\sin(4x) -3\sin(8x)+\sin(12x)]$			                                                               \\ \hline   
              $15$  &  $\frac{1}{\sqrt{-b^2+ac}}\arctan[\frac{b+c\tan x}{\sqrt{-b^2+ac}}]$ + discussion			                                                               \\ \hline   
              $16$  &  $5\csec^2(\frac{x}{5})-\frac{5}{4}\csec^4( \frac{x}{5})+5\ln(\sin \frac{x}{5})$			                                                               \\ \hline   
              $17$  & $\frac{1}{35}[6+\cos(2x)]\sec^2(x)\tan^5(x)$		                                                               \\ \hline   
              $18$  &  A voir...                                                               \\ \hline   
              $19$  &  $25[\cos(x)+\frac{1}{3}\cos(3x)-\ln(\cos(\frac{x}{2}))+\ln(\sin(\frac{x}{2}))]$			                                                               \\ \hline   
              $20$  & $-\ln[\cos(\frac{x}{2} )]+\ln[\cos(\frac{x}{2})+\sin(\frac{x}{2})]$                                                          \\ \hline   
              $21$  &  $-\ln[\cos(\frac{x}{2} )]+\frac{1}{2} \ln[\sin(\frac{x}{2})]-\frac{1}{4} \sec^2(\frac{x}{2})]$			                                                               \\ \hline   
              $22$  &  $-\frac{2}{3}\arctan[3\cotg(\frac{x}{2})]$			                                                               \\ \hline   
              $23$  &  $\frac{x}{5}-\frac{8}{15}\arcth[\frac{1}{3}\tan(\frac{x}{2})]$			                                                               \\ \hline   
              $24$  &  $-\arctan[\cos(x)]$			                                                               \\ \hline   
              $25$  &  $(2+\sec^2[\frac{x}{3}])\tan(\frac{x}{3})$			                                                               \\ \hline   

\end{tabular} \]


\[ \begin{tabular}{| c || c ||  }
\hline       no &         ${ 20}                              $                                                                                        \\    \hline \hline
             $1$   & $2\arctan(\sqrt x)   $            \\ \hline 
            $2$   &  $ \ln\frac{\sqrt z +1}{\sqrt z -1}$                 \\ \hline

           $3$   &      $\frac{3}{2}\ln(-3+\sqrt x)+\frac{1}{2}\ln(1+\sqrt x)$ 						\\ \hline 
           $4$   &      $  \frac{1}{2}   \ln\frac{\sqrt{1+ x} -2}{\sqrt{1+x} +2}       $                                                 				\\ \hline 
           $5$	  &      $-3\ln(-1+x^{1/3})+\ln x$                                                       \\ \hline   
           $6$	  &      $\frac{2}{3}\sqrt{3+2x}-\frac{2}{3}\sqrt{\frac{5}{3}}\arcth[\sqrt{\frac{3}{5}}\sqrt{3+2x}]$                                                       \\ \hline   

\end{tabular} \]


La suite viendra.


\chapter{Exercices de calcul différentiel et intégral 2}
% This is part of Exercices et corrigés de CdI-1
% Copyright (c) 2011
%   Laurent Claessens
% See the file fdl-1.3.txt for copying conditions.

%+++++++++++++++++++++++++++++++++++++++++++++++++++++
\section{Supremum, maximum}

\Exo{0001}
\Exo{0002}
\Exo{0003}
\Exo{00035}


\Exo{0004}
\Exo{0005}


%++++++++++++++++++++++++++++++++++++++++++++++++++++
\section{Suites}

\Exo{0006}
\Exo{0007}
\Exo{0008}
\Exo{0009}
\Exo{0010}
\Exo{0011}
\Exo{0012}
\Exo{0014}

\Exo{0015}
\Exo{0018}
\Exo{0019}
\Exo{0020}


\subsection{Suites définies par récurrence}

\Exo{0021}
\Exo{0022}


\section{Calcul de limites}
\label{SecCalcLimFHtQNu}

\Exo{0023}
\Exo{0025}
\Exo{0026}
\Exo{0027}

\subsection{Limites à deux variables}

\Exo{0028}
\Exo{0029}
\Exo{0030}

\Exo{LimSup0001}

%+++++++++++++++++++++++++++++++++++++++++++++++++++++++++++++++++++++++++++++++++++++++++++++++++++++++++++++++++++++++++++
					\section{Limite et continuité}
%+++++++++++++++++++++++++++++++++++++++++++++++++++++++++++++++++++++++++++++++++++++++++++++++++++++++++++++++++++++++++++

\Exo{continueSupl1}
\Exo{continueSupl2}


\Exo{0031}
\Exo{0032}
\Exo{0033}
\Exo{0034}
\Exo{0035}


\Exo{0036}
\Exo{reserve0001}
\Exo{0037}
\Exo{0038}
\Exo{0039}
\Exo{0040}

\Exo{continueSup0003}
\Exo{continueSup0004}
\Exo{continueSup0005}


%+++++++++++++++++++++++++++++++++++++++++++++++++++++++++++++++++++++++++++++++++++++++++++++++++++++++++++++++++++++++++++
					\section{Dérivées partielles et différentiabilité}
%+++++++++++++++++++++++++++++++++++++++++++++++++++++++++++++++++++++++++++++++++++++++++++++++++++++++++++++++++++++++++++


\Exo{0041}
\Exo{0042}
\Exo{0043}
\Exo{0044}
\Exo{0045}
\Exo{0046}
\Exo{0047}
\Exo{0048}
\Exo{0049}
\Exo{0050}
\Exo{0051}
\Exo{0052}
\Exo{0053}
\Exo{0054}
\Exo{0055}
\Exo{0056}
\Exo{0057}
\Exo{0058}
\Exo{0059}
\Exo{0060}
\Exo{0061}

%+++++++++++++++++++++++++++++++++++++++++++++++++++++++++++++++++++++++++++++++++++++++++++++++++++++++++++++++++++++++++++
					\section{Séries et séries de puissances}
%+++++++++++++++++++++++++++++++++++++++++++++++++++++++++++++++++++++++++++++++++++++++++++++++++++++++++++++++++++++++++++

\Exo{0062}
\Exo{0063}
\Exo{0064}
\Exo{0065}
\Exo{0066}
\Exo{0067}


%+++++++++++++++++++++++++++++++++++++++++++++++++++++++++++++++++++++++++++++++++++++++++++++++++++++++++++++++++++++++++++
					\section{Exercices de topologie}
%+++++++++++++++++++++++++++++++++++++++++++++++++++++++++++++++++++++++++++++++++++++++++++++++++++++++++++++++++++++++++++

Si $A_n$ est une suite d'ensemble, le symbole
\begin{equation}
	\bigcap_{n=1}^{\infty}A_n
\end{equation}
désigne l'ensemble des éléments qui sont dans $A_n$ pour tout $n\in\eN$. Remarquez que l'infini \emph{n'est pas} un élément de $\eN$ ! L'intersection se fait donc de $n=1$ à l'infini; l'infini non compris.

Prenons comme exemple le cas du point \ref{ItemF0072} de l'exercice \ref{exo0072}. Étant donné que $A_n=\mathopen]-\frac{1}{ n },\frac{1}{ n }\mathclose[$, on pourrait croire que $A_{\infty}=\mathopen]0,0\mathclose[=\emptyset$, et que par conséquent, l'intersection $\cap_{n=1}^{\infty}$ est vide.

%---------------------------------------------------------------------------------------------------------------------------
					\subsection{Exercices ultra basiques}
%---------------------------------------------------------------------------------------------------------------------------


\Exo{0071}
\Exo{0072}
\Exo{0073}
\Exo{0074}
\Exo{0075}
\Exo{0076}
\Exo{0077}
\Exo{0078}
\Exo{0079}
\Exo{0080}

%---------------------------------------------------------------------------------------------------------------------------
					\subsection{Exercices simplement basiques}
%---------------------------------------------------------------------------------------------------------------------------

Les exercices qui suivent ne seront en principe pas vus aux séances (faute de temps, plus que faute d'envie), mais ils sont certainement très intéressants à regarder pour celles et ceux qui désirent en savoir un peu plus sur la topologie.

\Exo{0081}
\Exo{0082}
\Exo{0083}
\Exo{0084}
\Exo{0085}
\Exo{0086}
\Exo{0087}
\Exo{0088}
\Exo{0089}


%+++++++++++++++++++++++++++++++++++++++++++++++++++++++++++++++++++++++++++++++++++++++++++++++++++++++++++++++++++++++++++
					\section{Fonctions d'une variable réelle (suite)}
%+++++++++++++++++++++++++++++++++++++++++++++++++++++++++++++++++++++++++++++++++++++++++++++++++++++++++++++++++++++++++++

\Exo{0090}
\Exo{0091}
\Exo{0092}
\Exo{0093}
\Exo{0094}
\Exo{0095}
\Exo{0096}
\Exo{0097}
\Exo{0098}
\Exo{0099}
\Exo{0100}


%+++++++++++++++++++++++++++++++++++++++++++++++++++++++++++++++++++++++++++++++++++++++++++++++++++++++++++++++++++++++++++
					\section{Développements de Taylor et Maclaurin}
%+++++++++++++++++++++++++++++++++++++++++++++++++++++++++++++++++++++++++++++++++++++++++++++++++++++++++++++++++++++++++++

\Exo{Devel0001}
\Exo{Devel0002}
\Exo{Devel0003}
\Exo{Devel0004}

\Exo{Devel0009}

\Exo{Devel0005}
\Exo{Devel0006}
\Exo{Devel0007}
\Exo{Devel0008}


\Exo{reserve0002}

%+++++++++++++++++++++++++++++++++++++++++++++++++++++++++++++++++++++++++++++++++++++++++++++++++++++++++++++++++++++++++++
					\section{Optimisation sans contraintes}
%+++++++++++++++++++++++++++++++++++++++++++++++++++++++++++++++++++++++++++++++++++++++++++++++++++++++++++++++++++++++++++

\Exo{OptimSS0001}
\Exo{OptimSS0002}
\Exo{OptimSS0003}
\Exo{OptimSS0004}
\Exo{OptimSS0005}
\Exo{OptimSS0006}


%+++++++++++++++++++++++++++++++++++++++++++++++++++++++++++++++++++++++++++++++++++++++++++++++++++++++++++++++++++++++++++
					\section{Équations différentielles}
%+++++++++++++++++++++++++++++++++++++++++++++++++++++++++++++++++++++++++++++++++++++++++++++++++++++++++++++++++++++++++++


%---------------------------------------------------------------------------------------------------------------------------
					\subsection{Équations différentielles du premier ordre}
%---------------------------------------------------------------------------------------------------------------------------

\Exo{EqsDiff0001}
\Exo{EqsDiff0002}
\Exo{EqsDiff0003}
\Exo{EqsDiff0004}
\Exo{EqsDiff0005}

%---------------------------------------------------------------------------------------------------------------------------
					\subsection{Équations différentielles du second ordre}
%---------------------------------------------------------------------------------------------------------------------------

\Exo{EqsDiff0006}
\Exo{EqsDiff0007}
\Exo{EqsDiff0008}
\Exo{EqsDiff0009}

%---------------------------------------------------------------------------------------------------------------------------
					\subsection{Équations différentielles : modélisation}
%---------------------------------------------------------------------------------------------------------------------------

\Exo{EqsDiff0010}
\Exo{EqsDiff0011}
\Exo{EqsDiff0012}

\Exo{EqsDiff0013}
\Exo{EqsDiff0014}
\Exo{EqsDiff0015}
\Exo{EqsDiff0016}



%+++++++++++++++++++++++++++++++++++++++++++++++++++++++++++++++++++++++++++++++++++++++++++++++++++++++++++++++++++++++++++
					\section{Intégrales multiples}
%+++++++++++++++++++++++++++++++++++++++++++++++++++++++++++++++++++++++++++++++++++++++++++++++++++++++++++++++++++++++++++

%%%%%%%%%%%%%%%%%%%%%%%%%
%
% Tous les exercices de cette section ont été repris dans OutilsMath le 3 avril 2011.
%
%%%%%%%%%%%%%%%%%%%%%%%%

\Exo{IntMult0001}
\Exo{IntMult0002}

Calculer le volume ou la surface d'un domaine revient à intégrer la fonction constante $1$ sur le domaine. Si nous effectuons un changement de variables, le jacobien intervient toutefois.

\Exo{IntMult0003}
\Exo{IntMult0004}
\Exo{IntMult0005}
\Exo{IntMult0006}
\Exo{IntMult0007}
\Exo{IntMult0008}
\Exo{IntMult0009}
% Il n'y a plus de IntMult0010 parce qu'il traitait de l'intégrale gausienne et a été 
% intégré aux notes d'agrégation. 4 août 2012.

\Exo{IntMult0011}
\Exo{IntMult0012}
\Exo{IntMult0013}


%+++++++++++++++++++++++++++++++++++++++++++++++++++++++++++++++++++++++++++++++++++++++++++++++++++++++++++++++++++++++++++
					\section{Théorème de la fonction implicite}
%+++++++++++++++++++++++++++++++++++++++++++++++++++++++++++++++++++++++++++++++++++++++++++++++++++++++++++++++++++++++++++


\Exo{Implicite0001}                                                                
\Exo{Implicite0002}                                                                
\Exo{Implicite0003}                                                                
\Exo{Implicite0004}                                                                
\Exo{Implicite0005}                                                                
\Exo{Implicite0006}                                                                
\Exo{Implicite0007}                                                                
\Exo{Implicite0008}                                                                
\Exo{Implicite0009}                                                                


%+++++++++++++++++++++++++++++++++++++++++++++++++++++++++++++++++++++++++++++++++++++++++++++++++++++++++++++++++++++++++++
\section{Variétés et extrema liés}
%+++++++++++++++++++++++++++++++++++++++++++++++++++++++++++++++++++++++++++++++++++++++++++++++++++++++++++++++++++++++++++

\Exo{Variete0001}                                                                
\Exo{Variete0002}                                                                
\Exo{Variete0003}                                                                
\Exo{Variete0004}                                                                
\Exo{Variete0005}                                                                

%+++++++++++++++++++++++++++++++++++++++++++++++++++++++++++++++++++++++++++++++++++++++++++++++++++++++++++++++++++++++++++
\section{Intégrales curvilignes}
%+++++++++++++++++++++++++++++++++++++++++++++++++++++++++++++++++++++++++++++++++++++++++++++++++++++++++++++++++++++++++++


\Exo{Variete0006}
\Exo{Variete0007}                                                                
\Exo{Variete0008}                                                                
\Exo{Variete0009}                                                                


\Exo{Variete0010}       % repris en version très allégée dans OutilsMath. Dans OutilsMath, il y aura un corrigé                                                              
\Exo{Variete0011}                                                                


%+++++++++++++++++++++++++++++++++++++++++++++++++++++++++++++++++++++++++++++++++++++++++++++++++++++++++++++++++++++++++++
\section{Intégrales de surface, Stokes et Green}
%+++++++++++++++++++++++++++++++++++++++++++++++++++++++++++++++++++++++++++++++++++++++++++++++++++++++++++++++++++++++++++

\Exo{Variete0012}                                                                
\Exo{Variete0013}                                                                
\Exo{Variete0014}                                                                
\Exo{Variete0015}                                                                
\Exo{Variete0016}                                                                
\Exo{Variete0017}                                                                



\Exo{Variete0018}                                                                
\Exo{Variete0019}                                                                
\Exo{Variete0020}         

\begin{center}
	Bonnes vacances !
\end{center}

% This is part of the Exercices et corrigés de CdI-2.
% Copyright (C) 2008, 2009
%   Laurent Claessens
% See the file fdl-1.3.txt for copying conditions.

%+++++++++++++++++++++++++++++++++++++++++++++++++++++++++++++++++++++++++++++++++++++++++++++++++++++++++++++++++++++++++++
					\section{Suites de fonctions}
%+++++++++++++++++++++++++++++++++++++++++++++++++++++++++++++++++++++++++++++++++++++++++++++++++++++++++++++++++++++++++++

	\Exo{_I-1-1}
	\Exo{_I-1-2}
	\Exo{_I-1-3}
	\Exo{_I-1-4}
	\Exo{_I-1-5}
	\Exo{_I-1-6}
	\Exo{_I-1-7}
	\Exo{_I-1-8}
	\Exo{_I-1-8b}
	\Exo{_I-1-9}

%+++++++++++++++++++++++++++++++++++++++++++++++++++++++++++++++++++++++++++++++++++++++++++++++++++++++++++++++++++++++++++
					\section{Séries de fonctions}
%+++++++++++++++++++++++++++++++++++++++++++++++++++++++++++++++++++++++++++++++++++++++++++++++++++++++++++++++++++++++++++

	\Exo{_I-1-10}
	\Exo{_I-1-11}
	\Exo{_I-1-12}
	\Exo{_I-1-13}
	\Exo{_I-1-14}
	\Exo{_I-1-15}
	\Exo{_I-1-16}

Après avoir fait le théorème de Borel, essayons de voir ce qu'il dit. Le théorème de la page I.29 nous donne facilement, sous forme de séries de puissances, une fonction $ C^{\infty}$ sur un intervalle dont les dérivées en zéro sont données à l'avance. L'exercice que nous venons de faire nous permet de trouver une fonction $ C^{\infty}$ sur tout $\eR$ dont les dérivées sont données à l'avance.

% TODO : La ligne suivante m'a l'air un peu trop simple pour être vraie ... y réfléchir encore un peu avant de décommenter :
%De plus, la fonction $u$ ainsi définie est à support compact, contenu dans le support de $f$, parce que $t_n\leq 1$. Nous obtenons ainsi des fonctions $ C^{\infty}$ à support compact dont les dérivées en un point sont données à l'avance.

%+++++++++++++++++++++++++++++++++++++++++++++++++++++++++++++++++++++++++++++++++++++++++++++++++++++++++++++++++++++++++++
					\section{Existence d'intégrales}
%+++++++++++++++++++++++++++++++++++++++++++++++++++++++++++++++++++++++++++++++++++++++++++++++++++++++++++++++++++++++++++

En vertu des différents théorèmes, l'étude de la convergence et de l'existence d'une intégrale d'une fonction sur $[a,\infty[$ se fait dans l'ordre suivant :
\begin{itemize}
\item si l'intégrale de $f$ existe, alors elle converge,
\item si la fonction est positive et que son intégrale n'existe pas, alors elle ne converge pas,
\item si le signe de la fonction est variable et que l'intégrale n'existe pas, alors elle peut converger ou non, selon les cas.
\end{itemize}

	\Exo{_I-2-1}
	\Exo{_I-2-2}
	\Exo{_I-2-3}
	\Exo{_I-2-4}
	
%+++++++++++++++++++++++++++++++++++++++++++++++++++++++++++++++++++++++++++++++++++++++++++++++++++++++++++++++++++++++++++
					\section{Fonctions définies par des intégrales}
%+++++++++++++++++++++++++++++++++++++++++++++++++++++++++++++++++++++++++++++++++++++++++++++++++++++++++++++++++++++++++++


	\Exo{_I-3-1}
	\Exo{_I-3-2}
	\Exo{CourbesSurfaces0017}


%+++++++++++++++++++++++++++++++++++++++++++++++++++++++++++++++++++++++++++++++++++++++++++++++++++++++++++++++++++++++++++
					\section{Convergence, continuité et dérivation sous le signe intégral}
%+++++++++++++++++++++++++++++++++++++++++++++++++++++++++++++++++++++++++++++++++++++++++++++++++++++++++++++++++++++++++++

	\Exo{_I-3-4}
	\Exo{_I-3-5}
	\Exo{_I-3-6}
	\Exo{_I-3-7}
	\Exo{_I-3-8}
	\Exo{_I-3-9}
	\Exo{_I-3-10}
	\Exo{_I-3-11}
%+++++++++++++++++++++++++++++++++++++++++++++++++++++++++++++++++++++++++++++++++++++++++++++++++++++++++++++++++++++++++++
					\section{Quelque propriétés des espaces fonctionnels}
%+++++++++++++++++++++++++++++++++++++++++++++++++++++++++++++++++++++++++++++++++++++++++++++++++++++++++++++++++++++++++++

	\Exo{_I-4-1}
	\Exo{_I-4-2}
	\Exo{_I-4-3}


% This is part of the Exercices et corrigés de CdI-2.
% Copyright (C) 2008, 2009
%   Laurent Claessens
% See the file fdl-1.3.txt for copying conditions.


%+++++++++++++++++++++++++++++++++++++++++++++++++++++++++++++++++++++++++++++++++++++++++++++++++++++++++++++++++++++++++++
					\section{Équations différentielles}
%+++++++++++++++++++++++++++++++++++++++++++++++++++++++++++++++++++++++++++++++++++++++++++++++++++++++++++++++++++++++++++

%---------------------------------------------------------------------------------------------------------------------------
					\subsection{Équations différentielles résolubles}
%---------------------------------------------------------------------------------------------------------------------------

\Exo{_II-1-01}
\Exo{_II-1-02}
\Exo{_II-1-03}
\Exo{_II-1-04}
\Exo{_II-1-05}
\Exo{_II-1-06}

%---------------------------------------------------------------------------------------------------------------------------
					\subsection{Équation de Bernoulli}
%---------------------------------------------------------------------------------------------------------------------------

\Exo{reserve0004}
\Exo{_II-1-08}

%TODO : changer tous les noms de fichiers contenant «_».

%---------------------------------------------------------------------------------------------------------------------------
					\subsection{Équations de Ricatti}
%---------------------------------------------------------------------------------------------------------------------------


\Exo{_II-1-09}

%---------------------------------------------------------------------------------------------------------------------------
					\subsection{Équations homogènes}
%---------------------------------------------------------------------------------------------------------------------------


\Exo{_II-1-10}

%---------------------------------------------------------------------------------------------------------------------------
					\subsection{Équations différentielles exactes. Facteurs intégrants}
%---------------------------------------------------------------------------------------------------------------------------


\Exo{_II-1-11}
\Exo{_II-1-12}


%---------------------------------------------------------------------------------------------------------------------------
					\subsection{L'équation \texorpdfstring{$y'=f\left( \frac{ at+by+c }{ a't+b'y+c' } \right)$}{yat}}
%---------------------------------------------------------------------------------------------------------------------------

\Exo{_II-1-13}

%---------------------------------------------------------------------------------------------------------------------------
					\subsection{Équation d'Euler}
%---------------------------------------------------------------------------------------------------------------------------

\Exo{_II-1-14}

%---------------------------------------------------------------------------------------------------------------------------
					\subsection{Équation dont on peut réduire l'ordre}
%---------------------------------------------------------------------------------------------------------------------------

\Exo{_II-1-15}
\Exo{_II-1-16}

%---------------------------------------------------------------------------------------------------------------------------
					\subsection{Quelque exemples d'équations résolubles}
%---------------------------------------------------------------------------------------------------------------------------

\Exo{_II-1-17}

%---------------------------------------------------------------------------------------------------------------------------
					\subsection{L'équation différentielle linéaire du second ordre}
%---------------------------------------------------------------------------------------------------------------------------

\Exo{_II-1-20}
\Exo{_II-1-21}
\Exo{_II-1-22}
\Exo{_II-1-23}
\Exo{_II-1-24}

%---------------------------------------------------------------------------------------------------------------------------
					\subsection{Équation différentielle implicite du premier ordre}
%---------------------------------------------------------------------------------------------------------------------------

\Exo{_II-1-18}
\Exo{_II-1-19}
\Exo{_II-1-28}
\Exo{_II-1-25}
\Exo{_II-1-26}
\Exo{_II-1-27}

%+++++++++++++++++++++++++++++++++++++++++++++++++++++++++++++++++++++++++++++++++++++++++++++++++++++++++++++++++++++++++++
					\section{Systèmes différentiels linéaires à coefficients constants}
%+++++++++++++++++++++++++++++++++++++++++++++++++++++++++++++++++++++++++++++++++++++++++++++++++++++++++++++++++++++++++++


\Exo{_II-2-01}
\Exo{_II-2-02}
\Exo{_II-2-03}
\Exo{_II-2-04}
\Exo{_II-2-05}
\Exo{_II-2-06}
\Exo{_II-2-07}



\chapter{Exercices de calcul différentiel et intégral 1} % Ce sont des exercices de CdI-1
% This is part of Exercices et corrigés de CdI-1
% Copyright (c) 2011
%   Laurent Claessens
% See the file fdl-1.3.txt for copying conditions.




% This is part of Exercices et corrigés de CdI-1
% Copyright (c) 2011
%   Laurent Claessens
% See the file fdl-1.3.txt for copying conditions.

\setcounter{CountExercice}{1}

%+++++++++++page de garde+++++++++++++++++++++++++++++++



\section{Intégrales de surface, Stokes et Green}




\setcounter{CountExercice}{0}


\noindent{\bf Exercice 6}\\

{\bf $(a)$ La suite $[k\rightarrow \f{1}{k}]$ est convergente.}\\

\noindent Nous allons montrer que cette suite converge vers $0$. Il faut donc prouver la chose suivante: 
   \begin{equation}\label{eqn1}\forall \epsilon >0 \hspace{0,3cm} \exists K_\eps \in \N \hspace{0,3cm} {\rm tq}  \hspace{0,3cm}  \forall k\geq K_\eps, \hspace{0,3cm}  |x_k-x|<\eps\end{equation}
{Remarque}: On pourrait également montrer que cette suite est {\it de Cauchy} pour prouver qu'elle est convergente sans devoir déterminer sa limite.\\

\noindent Pour prouver que (\ref{eqn1}) s'applique bien à la suite des $\f{1}{k}$ il nous faut montrer que

   \begin{equation}\label{eqn2}\forall \epsilon >0 \hspace{0,3cm} \exists K_\eps \in \N \hspace{0,3cm} {\rm tq}  \hspace{0,3cm}  \forall k\geq K_\eps,  \hspace{0,3cm} \f{1}{k}<\eps\end{equation}

\noindent Ceci est une conséquence immédiate de l'exercice précédent. On peut également le montrer de la manière suivante: à $\epsilon$ positif donné, si nous arrivons à déterminer l'indice $K_\eps$ de (\ref{eqn2}) tel que $\forall k\geq K_\eps,  \hspace{0,3cm} \f{1}{k}<\eps$, il est clair que la suite satisfait à la définition. Or, $\f{1}{k} < \eps \leftrightarrow \f{1}{\eps} <k$. Donc si nous prenons $K_\eps := \ceil\f{1}{\eps})+1$, on a bien que $\forall k\geq K_\eps$, $\f{1}{k}<\eps$, ce qui est ce qu'il fallait démontrer.


\vspace{1cm}
{\bf $(b)$ La suite $(1, \f{1}{2}, -\f{1}{3},  \f{1}{4}, -\f{1}{5}, \ldots )$ est convergente.}\\

\noindent On remarque que cette suite tend vers zéro. (Il suffit de voir que le numérateur est borné et que le dénominateur  tend vers l'infini). Si on l'écrit  sous la forme standard, on obtient:                 
              \[x_1 = 1, x_k = \f{(-1)^k}{k} \hspace{0.3cm} \forall k\geq 2\] 
Donc, ce que nous voulons voir est que $x_k \lra_{k\rightarrow  \infty} 0$, i.e.: 
   \begin{equation}\label{eqn3}\forall \epsilon >0 \hspace{0,3cm} \exists K_\eps \in \N \hspace{0,3cm} {\rm tq}                       
       \hspace{0,3cm}  \forall k\geq K_\eps,  \hspace{0,3cm} |\f{(-1)^k}{k}|<\eps\end{equation}
       
\noindent Étant donné que $|(-1)^k| = 1 \hs \forall k$, il est clair que l'équation (\ref{eqn3}) est la même que l'équation (\ref{eqn2}), et donc que l'on peut affirmer que pour tout $\epsilon > 0$, il suffit de prendre $K\geq \f{1}{\epsilon}$ et la condition est satisfaite.

\noindent{\bf Exercice 7}\\

Ici il est demandé de prouver de nouvelles règles de calcul en repartant de la définition de la convergence vers l'infini:
\begin{equation}
 \label{eqnconvinfGene} x_k \lra \infty \hspace{0.3cm} {\rm si} \hspace{0.3cm}  \forall M > 0 \hspace{0.3cm} \exists K_M \in \N \hspace{0.3cm} {\rm tq} \hspace{0.3cm} \forall k \geq K_M, x_k \geq M \end{equation}
{\bf (a) $ \lim(x_k+y_k) = +\infty$.}\\

\noindent On veut voir  la chose suivante:
\begin{equation}
 \label{eqnconvinfCasA}  \forall M > 0 \hspace{0.3cm} \exists K_M \in \N \hspace{0.3cm} {\rm tq} \hspace{0.3cm} \forall k \geq K_M, x_k + y_k \geq M \end{equation}

\noindent Soit $M> 0$. Comme $x_k$ et $y_k$ convergent à l'infini, on sait que 
\[\left\{\begin{array}{c}   
         \exists K^x_M \in \N \hspace{0.3cm} {\rm tq} \hspace{0.3cm} \forall k \geq K^x_M, x_k \geq \f{M}{2}\\																		 
        \exists K^y_M \in \N \hspace{0.3cm} {\rm tq} \hspace{0.3cm} \forall k \geq K^y_M, y_k \geq \f{M}{2},																		
\end{array}\right.\]
et donc il suffit de prendre $K_M = \max(K_M^x, K_M^y)$ dans (\ref{eqnconvinfCasA}) pour s'assurer que la définition est satisfaite.


\vspace{0.5cm}
\noindent{\bf (b) $ \lim(x_ky_k) = +\infty$.}\\

\noindent On veut voir la chose suivante:
\begin{equation}
 \label{eqnconvinfprod}  \forall M > 0 \hspace{0.3cm} \exists K_M \in \N \hspace{0.3cm} {\rm tq} \hspace{0.3cm} \forall k \geq K_M, x_k  y_k \geq M \end{equation}

\noindent Soit $M> 0$. Comme $x_k$ et $y_k$ convergent à l'infini, on sait que 
\[\left\{\begin{array}{c}   
         \exists K^x_M \in \N \hspace{0.3cm} {\rm tq} \hspace{0.3cm} \forall k \geq K^x_M, x_k \geq \sqrt M\\																		 
        \exists K^y_M \in \N \hspace{0.3cm} {\rm tq} \hspace{0.3cm} \forall k \geq K^y_M, y_k \geq \sqrt M,																		
\end{array}\right.\]

\noindent et donc il suffit  de prendre  $K_M = \max(K_M^x, K_M^y)$ dans (\ref{eqnconvinfprod}) pour s'assurer que la définition est satisfaite.

\vspace{0.5cm}
\noindent{\bf (d) Soit $z_k$ une suite tendant vers un réel $a$ strictement positif. Prouvez que $\lim(x_k  z_k) = +\infty$.}\\

Le but de l'exercice est toujours le même, c'est à dire de prouver que 
\begin{equation}		\label{eqnconvinfz}
  \forall M > 0 \hspace{0.3cm} \exists K_M \in \N \hspace{0.3cm} {\rm tq} \hspace{0.3cm} \forall k \geq K_M, \;x_k  z_k \geq M 
\end{equation}

\noindent Soit $M>0$. On sait  que:

\begin{equation}
\label{eqn12}\left\{\begin{array}{l}   
        \forall \tilde{M} >0 \;\exists K^x_{\tilde{M}} \in \N \hspace{0.3cm} {\rm tq} \hspace{0.3cm} \forall k \geq K^x_{\tilde{M}},\; x_k \geq  \tilde{M} \\																		 
       \forall \epsilon >0\;\exists K^z_\eps \in \N \hspace{0.3cm} {\rm tq} \hspace{0.3cm} \forall k \geq K^z_\eps,\; |z_k-a| <\epsilon,																		
\end{array}\right.\end{equation}

\noindent Prenons un $\epsilon$ tel que $a-\epsilon>0$. Par la deuxième partie de (\ref{eqn12}) on voit qu'il existe un indice $ K^z_\eps$ tel que $ \forall k \geq K^z_\eps,\; z_k > a-\epsilon >0$.

\noindent Prenons un $\tilde{M}$ tel que $M= \tilde{M}(a-\epsilon)$. Par la première partie de (\ref{eqn12}) on voit qu'il existe un indice $ K^x_{\tilde{M}} $ tel que $\forall k \geq K^x_{\tilde{M}},\; x_k \geq  \tilde{M} $.

												
\noindent et donc il suffit  de prendre  $K_M = \max(K_{\tilde{M}}^x, K^z_\eps)$ dans (\ref{eqnconvinfz}) pour avoir que 
\[ \forall k \geq K_M, \;x_k  z_k \geq \tilde{M}(a-\epsilon)=M.\]


\noindent{\bf Exercice 8}\\

\noindent Une suite $x_k$ est bornée si $\exists N>0$ tel que $\forall k$, $|x_k| < N$.

\noindent On veut voir que $\f{x_k}{y_k}\lra 0$, i.e.

\begin{equation} 
\label{eqnconvborne}  \forall  \epsilon > 0 \hspace{0.3cm} \exists K_\epsilon \in \N \hspace{0.3cm} {\rm tq} \hspace{0.3cm} \forall k \geq K_\epsilon, \; |\f{x_k}{y_k}| < \epsilon \end{equation}

\noindent Soit $\epsilon >0$. Comme la suite $x_k$ est bornée, on a que  $|\f{x_k}{y_k}|<\f{N}{|y_k|}\; \forall k$. On utilise maintenant le fait que $y_k \lra \infty$. Prenons $M=\f{N}{\epsilon}$. On peut écrire que $\exists K_M$ tel que $\forall k \geq K_M, \; y_k \geq M=\f{N}{\epsilon}$, et donc si dans (\ref{eqnconvborne}) on prend $K_\epsilon= K_M$ on a:\[\forall k \geq K_\eps,\; \; |\f{x_k}{y_k}|<\f{N}{|y_k|}<\f{N}{N/\epsilon}=\epsilon.\]



\noindent{\bf Exercice 9}\\

\noindent Pour cet exercice, on peut utiliser les règles de calcul. Il faut faire attention que ces règles ne s'appliquent que si toutes les limites existent!

\vspace{0.5cm}
\noindent{ (a)} $x_k = \f{k+2}{k}\cos(k\pi)$\\

\noindent On voit que cette suite va dans deux directions différentes, $+1$ et $-1$ à cause du facteur $\cos(k\pi)=(-1)^k$. Elle ne converge donc pas. Pour le prouver, on peut prendre deux suites partielles de la suite $x_k$ qui convergent vers des limites différentes. 

\noindent Choisissons \[\left\{ \begin{array}{rcl} y_k &= x_{2k}&= \f{(2k)+2}{2k}(-1)^{2k}\\
 							  z_k &= x_{2k+1} &= \f{(2k+1)+2}{2k+1} (-1)^{2k+1}\end{array}\right.\]

\noindent Comme $x_k =\f{k+1}{k}= 1+\f{1}{k}$	et que $\f{1}{k} \rightarrow  0$, nous pouvons appliquer les règles de calcul et en déduire que $x_k \rightarrow  1$. On fait la même chose pour $y_k$.				  


\vspace{0.5cm}
\noindent{ (c)} $x_k = \f{k^3+k+1}{5k^3+2}$\\

\noindent Nous avons que \[\forall k, \;\;\;\;x_k =\; (\f{k^3}{k^3})\f{1+\f{1}{k} +\f{1}{k^3}}{5+\f{2}{k^3}}=\;\f{1+\f{1}{k} +\f{1}{k^3}}{5+\f{2}{k^3}} \]
Comme \[\forall k \geq 1\;\; \f{1}{k^3} \; \leq \;\f{1}{k^2}\; \leq \; \f{1}{k}\] et comme $\f{1}{k}\rightarrow 0$, nous pouvons appliquer la règle de l'étau pour voir que \[\f{1}{k^3} \rightarrow 0 \; \; \; {\rm et } \;\; \;\f{1}{k^2} \rightarrow 0.\]
En appliquant les règles de calcul à la suite $x_k$ transformée, on voit donc que $x_k \rightarrow  \f{1}{5}$.

\vspace{0.5cm}
\noindent{ (d)} $x_k = \f{k+(-1)^k}{k-(-1)^k}$\\

\noindent On peut le voir par exemple par la règle de l'étau:
\[\forall k \geq 0, \;\;\; \f{k-1}{k+1} \leq \f{k+(-1)^k}{k-(-1)^k} \leq \f{k+1}{k-1}. \]
Or, comme les deux suites qui bornent la suite $x_k$ convergent toutes les deux vers $1$, il est clair que $x_k$ converge aussi vers $1$.


\vspace{0.5cm}
\noindent{ (d)} $x_k = x_{k-1}^2\;+\;1,\hs x_1=1$\\

\noindent Suite définie par récurrence. Ses premiers éléments sont \[1, \; 2, \; 5, \;  26, \; 677, \; \ldots\]
Toute  limite admissible réelle finie $l$  de cette suite doit satisfaire à \[l=l^2+1\] ce qui implique qu'elle ne peut avoir de limite réelle finie. En regardant ses premiers éléments, on remarque immédiatement qu'elle semble converger à l'infini. Nous allons le prouver en utilisant la définition.

\noindent Soit $M> 0$. On a que \[x_k \geq k \hs \forall k.\] En effet (par récurrence sur $k$): il est clair que $x_1 \geq 1$. Supposons que $x_k \geq k$. Ceci implique t-il que $x_{k+1}\geq k+1$? Par définition des $x_k$, $x_{k+1} = x_k^2+1$. Par l'hypothèse de récurrence, on a donc $x_{k+1}\geq (k)^2 +1\geq k+1$ ce qui prouve le résultat. Comme la suite $y_k=k$ converge à l'infini, il en est de même pour la suite $x_k$.



\section{Continuité de fonctions réelles}


\begin{center}
\LARGE \bf
Travaux Personnels 
\end{center}

\begin{bf}
\begin{center}
BAC2 en sciences mathématiques et physiques
\end{center}
\end{bf}

{\bf Exercice 1.} Calculer les limites suivantes

\b
a) $\displaystyle \lim_{n \to \infty} \left( 1+ \frac{2}{n-4} \right)^n$

\medskip
b) 
$\displaystyle \lim_{n \to \infty} 
         \left( 1+ \frac 1n \right)^{\sqrt{n}}$

\medskip
c) $\displaystyle \lim_{x \to \infty} 
    \left( 1+ \frac \ga x \right)^x$

\medskip
d) 
$\displaystyle \lim_{x \to 0} \frac{\log \left( 1+ \ga x \right)}{x}$


\medskip
e) 
$\displaystyle \lim_{x \to \infty} 
\frac{a_0+a_1x + \dots +a_nx^n}{b_0+b_1x + \dots +b_mx^m}$
\quad où\, $a_j, b_j \in \eC$ \,et\, $n,m \ge 0$

\medskip
f) 
$\displaystyle \lim_{x \to 0} \frac{\sqrt{1-\cos x}}{x}$  




{\bf Exercice 2.} Prouver que

\medskip
a)
$\displaystyle \lim_{x \to \infty} x^{\frac 1x} = \lim_{x \to 0^+} x^x = 1$

\medskip
b)
$\displaystyle \lim_{x \to \infty} \frac{x^{\ln x}}{{\mathrm e}^x} =0$
\quad
càd ${\mathrm e}^x$ croit plus vite que $x^{\ln x}$


{\bf Exercice 3.} Prouver que
$$
\cosh 2x \,=\, \cosh^2 x + \sinh^2 x,
\qquad
\sinh 2x \,=\, 2 \sinh x \cosh x
$$


{\bf Exercice 4.} Prouver que

a)
$1 + \cos z + \cos 2z + \dots + \cos nz = \displaystyle \cos \frac{nz}{2} \cdot \frac{\sin (n+1)z/2}{\sin z/2}$

b)
$1 + \sin z + \sin 2z + \dots + \sin nz = \displaystyle \sin \frac{nz}{2} \cdot \frac{\sin (n+1)z/2}{\sin z/2}$

{\it Aide:}\;
$\displaystyle \sum_{k=0}^n 
\euler^{\sii kz} 
= 
\frac{1-\euler^{\sii (n+1)z}}{1-\euler^{\sii z}}
= \euler^{\sii nz/2} \cdot \frac{\euler^{\sii (n+1)z/2} - \euler^{-\sii (n+1)z/2}}{
\euler^{\sii z/2}-\euler^{-\sii z/2}}
$

Rappelons qu'une fonction $f \colon \mathbb{C} \supset D \to \eC$ est {\bf uniformément continue} si pour tout $\eps >0$ il existe un $\gd >0$ tel que 
$$
|x-y| < \gd \,\Longrightarrow\, |f(x)-f(y)| < \eps 
\quad \text{ pour tout }\, x,y \in D.
$$
Prouver que la fonction $f \colon \eR \to \eR$, $x \mapsto x^2$ est continue, mais n'est pas uniformément continue.


\section{Intégrales, longueur de courbes, EDO's linéaires}


\exerNico 
Soient $n,m \in \NN \cup \{0\}$.
Calculer
$$
\int_0^1 x^n (1-x)^m \,dx
\quad \text{ et } \quad
\int_{-1}^1 (1+x)^n (1-x)^m \,dx
$$

{\bf Solution:}
Posons $I_{n,m} := \int_0^1 x^n (1-x)^m \,dx$.
Intégration par partie donne
la formule récursive
$$
I_{n,m} \,=\, \frac {m}{n+1} I_{n+1,m-1}.
$$
Avec $I_{n+m,0} = \frac{1}{n+m+1}$ nous obtenons
$$
I_{n,m} \,=\, \frac{n!\,m!}{(n+m+1)!}
$$
La substitution $x := 2t-1$ fournit
$$
\int_{-1}^1 (1+x)^n (1-x)^m \,dx
\,=\, 2^{n+m+1} \int_0^1 t^n (1-t)^m \,dt \,=\,  2^{n+m+1} 
\cdot I_{n,m}. 
$$




\exerNico 
Soient $a,b >0$. 
Calculer
$$
\int_0^{\pi /2} \displaystyle \frac{d \gf}{a^2 \sin^2 \gf + b^2 \cos^2 \gf}
$$

{\bf Solution:}
$$
\,=\, \int_0^{\pi /2} \frac{1 / \cos^2 \gf}{a^2 \tan^2 \gf+b^2} d\gf \,=\, \int_0^\infty \frac {dt}{a^2t^2 + b^2} \,=\, \frac{\pi}{2ab}  
$$


\exerNico  
Calculer la longueur de l'arc de la parabole $y = x^2,\;x \in [0,b]$.

\medskip
{\bf Solution:}
$$
s \,=\, \int_0^b \sqrt{1+4x^2} \,dx \,=\, \frac b 2 \sqrt{1+4b^2}+ \frac 14 \ln \left(2b+ \sqrt{1+4b^2} \right)
$$


\exerNico  
La {\bf parabole de Neil} $\nu$ est la courbe définie par $\nu (t) = (t^2,t^3)$, pour  $t \in \eR$.

\medskip
a)
Esquisser la parabole de Neil.

\medskip
b)
Quelle est la signification du paramètre $t$?

\medskip
{\bf Solution:} $t = \tan \ga$

\medskip
c)
Calculer la longueur de l'arc 
$\left\{ \nu (t) \mid t \in [0,\tau] \right\}$.


\medskip
{\bf Solution:}
$$
s \,=\, \int_0^\tau \sqrt{4 t^2+9t^4} \,d\tau \,=\, \frac{8}{27} \left( \left(1+ \frac 94 \tau^2\right)^{3/2}-1 \right)
$$



\exerNico  
La {\bf hélice} $\gamma$ de pas $2 \pi h$ est la courbe dans $\RR^3$ définie par
$$
\gamma(t) \,=\, \left( r \cos t , r \sin t , h t \right)  .
$$


\medskip
a)
Esquisser la hélice.

\medskip
b)
Expliquer le mot ``pas''.


\medskip
c)
Calculer la longueur de l'arc sur la hélice si on fait un tour.

\medskip
{\bf Solution:} 
$\int_0^{2\pi} \sqrt{r^2+h^2} \,dt \,=\, 2 \pi \sqrt{r^2+h^2}$


\bigskip
\exerNico 
Calculer un système fondamental réel pour

\medskip
a) $y^{(4)}-y = 0$,

\medskip
b) $y^{(4)} +4y'' +4y = 0$,

\medskip
c) $y^{(4)} -2y^{(3)} +5y'' = 0$.


\bigskip
{\bf Solution:}

\medskip
a) ${\rm e}^x, {\rm e}^{-x}, \cos x, \sin x$

\medskip
b) $\cos \sqrt{2} x, x \cos \sqrt{2}x, \sin \sqrt{2}x, x \sin \sqrt{2}x$

\medskip
c)
$1, x, {\rm e}^x \cos 2x, {\rm e}^x \sin 2x$



\bigskip
\exerNico 
Déterminer une solution particulière de l'équation
$y''+y=q$ pour

\medskip
a) $q = x^3$,

\medskip
b) $q = \sinh x$,

\medskip
c) $q = 1/\sin x$.
 

\bigskip
{\bf Solution:}

\medskip
a) $x^3 - 6 x$

\medskip
b) $\frac 12 \sinh x$

\medskip
c) $\sin x \cdot \ln |\sin x| - x \cos x$


\bigskip
\exerNico  
L'équation différentielle $m \ddot y = mg - k\dot y$ 
décrit la chute d'un corps soumit
à la gravitation si la friction est proportionnelle à la vitesse (``un homme tombant de l'avion'').

\medskip
Calculer la solution avec $y(0) =0, \dot y(0) = 0$.
Calculer la ``vitesse finale'' $v_\infty = \displaystyle \lim_{t \to \infty} \dot y (t)$.



\bigskip
{\bf Solution:}

\medskip
L'équation homogène $\ddot y + k/m \cdot y = 0$
possède les solutions $c_1+c_2 {\rm e}^{-k/m \cdot t}$,
où $c_1, c_2 \in \RR$.
 
L'équation inhomogène $\ddot y + k/m \cdot y = g$
possède comme solution particulière une fonction lineaire, càd 
$y_p = (mg/k)t)$.
En tenant compte des conditions initiales nous obtenons
$$
y(t) \,=\, \frac{mg}{k} \left( t-\frac mk (1-{\rm e}^{-k/m \cdot t})\right).
$$
En particulier, $v_\infty = mg/k$. 

 




\bigskip
\exerNico  
Regardons l'ensemble des solutions de l'équation différentielle $P({\rm D})y =0$.

Montrer l'équivalence entre les propositions suivantes :
\begin{enumerate}

\item
Pour toute solution $y$ on a $\displaystyle \lim_{t \to \infty} y(t) = 0$

\item
Pour toute racine $z$ du polynôme caractéristique on a ${\rm Re}\, z <0$.

\end{enumerate}
Dans ce cas, l'équation différentielle est appelé  \defe{asymptotiquement stable}{Asymptotiquement stable}.

\bigskip
{\bf Solution:}
On a 
$\displaystyle \lim_{t \to \infty} y(t) = 0$ pour toute solution $y$ ssi c'est vrai pour tout élément d'un système fondamental.
On a $\displaystyle \lim_{t \to \infty} t^k {\rm e}^{\gl t}=0$ ssi ${\rm Re }\,\gl <0$,
d'où l'affirmation suit.






\section{Calcul de limites}

\exerNico Déterminez si les limites suivantes existent et dans
l'affirmative calculez les en utilisant, s'il y a lieu, la règle de
l'Hospital ou la règle de l'étau.
\begin{enumerate}
\item $  \lim_{x \rightarrow  +\infty} \frac{x+1}{x^2+2} $
\item $  \lim_{x \rightarrow  +\infty} \frac{\sin(x)}{x} $
\item $  \lim_{x \rightarrow  0} \frac{\sin(x)}{x} $
\item $  \lim_{x \rightarrow  +\infty}  \frac{x ^n}{e ^x} $
\item $  \lim_{x \rightarrow  +\infty} (1 + \frac{a}{x})^x $
\item $  \lim_{x \rightarrow  0} (\frac{1}{\sin(x)} - \frac{1}{x} )$
\item $  \lim_{x \rightarrow  +\infty} \cos( 2 \pi x) $
\item $  \lim_{x \rightarrow  +\infty} \frac{1}{\sin(x)+2}(x) +\ln(x)\cos(x) $
\item $  \lim_{x \rightarrow  +\infty} \frac{ \ln(x)(\sin(x) +2)}{x} $
\item $  \lim_{x \rightarrow  +\infty} x ^\frac{1}{x} $
\end{enumerate}

\exerNico Déterminez si les limites suivantes existent et dans
l'affirmative calculez-les.
\begin{enumerate}
\item $  \lim_{x \rightarrow  0} x \sin(\frac{1}{x}) $
\item $  \lim_{x \rightarrow  0} \frac{\sin(\sin(x))}{x} $
\item $  \lim_{x \rightarrow  +\infty} (\ln(x))^\frac{1}{1 - \ln(x)}$
\end{enumerate}

\exerNico Calculez les limites suivantes:
\begin{enumerate}
\item $  \lim_{x \rightarrow  +\infty} \frac{\ln(x)}{x ^a} $
\item $  \lim_{x \rightarrow  +\infty} \frac{\ln(x)^a}{x ^b} $
\item $  \lim_{x \rightarrow  +\infty} a ^x $
\item $  \lim_{x \rightarrow  +\infty} a ^\frac{1}{x} $
\end{enumerate}
où $a$ et $b$ sont des réels positifs.
%

%

\exerNico Déterminez, pour chacune des suites suivantes, si elle converge
et dans l'affirmative calculez sa limite.
\begin{enumerate}
\item $  k \rightarrow  \cos( 2 \pi k) $
\item $  k \rightarrow  \cos(\frac{\pi}{3} k) $
\item $  k \rightarrow  k(a ^\frac{1}{k} -1 ) $
\end{enumerate}
où $a$ est une réel.\\



\exerNico Calculez  les limites suivantes si elles existent.
\begin{enumerate}
\item $  \lim_{x \rightarrow  +\infty} \cos x $
\item $  \lim_{x \rightarrow  \pm \infty }\sqrt{2x^4+3}-x^2 $

\end{enumerate}

\exerNico Déterminez si la limite de chacune des suites suivantes
existe et dans l'affirmative calculez la.
\begin{enumerate}
\item $  \lim_{k \rightarrow  +\infty }(\frac{a k +1}{k})^k $
\item $  \lim_{k \rightarrow  +\infty}\frac{1}{\sin(\frac{\pi}{6}k)+1}(k) + \ln(k)\cos(\frac{\pi}{5}k)$
\item $  \lim_{k \rightarrow  +\infty} \frac{\ln(k)(\sin(\frac{\pi}{3}k) +1)}{k} $
\item $  \lim_{k \rightarrow  +\infty } \sqrt[3k]{k} (1 +
\frac{1}{3k})^{3k} $
\end{enumerate}
où $a$ est un réel. 

\section{Dérivabilité}



\exerNico Déterminez l'ensemble des points où les fonctions suivantes
sont continues et celui où elles sont dérivables. Prouvez soigneusement
vos résultats.
\begin{enumerate}
\item $ x \rightarrow x]$
\item $ x \rightarrow |x| $
\item $ x \rightarrow
	\left\{ \begin{array}{ll}
	\frac{1}{x} & \mbox{si } x \not= 0 \\
	0 & \mbox{sinon}
	\end{array} \right. $
\item $ x \rightarrow x^2  $
\end{enumerate}




\exerNico Étudiez la dérivabilité et la continuité
de la dérivée de chacune des fonctions suivantes:
\begin{enumerate}
\item $ x \rightarrow
\left\{ \begin{array}{ll}
0 & \mbox{si } x \not= 0 \\
1 & \mbox{sinon}
\end{array} \right.$
%
\item $ x \rightarrow
\left\{ \begin{array}{ll}
\sin(\frac{1}{x}) & \mbox{si } x \not= 0 \\
0 & \mbox{sinon}
\end{array} \right.$
%
\item $ x \rightarrow
\left\{ \begin{array}{ll}
x \sin(\frac{1}{x}) & \mbox{si } x \not= 0 \\
0 & \mbox{sinon}
\end{array} \right.$
%
\item $ x \rightarrow
\left\{ \begin{array}{ll}
x^2 \sin(\frac{1}{x}) & \mbox{si } x \not= 0 \\
0 & \mbox{sinon}
\end{array} \right.$
\end{enumerate}

Le but de cet exercice est aussi d'exhiber des exemples illustrant les
différents types de comportements possibles, relativement à la
continuité et la dérivabilité, d'une fonction en un point.

\exerNico Étudiez la dérivabilité et la continuité
de la dérivée de chacune des fonctions suivantes:
\begin{enumerate}
\item $ x \rightarrow
\left\{ \begin{array}{ll}
\frac{2x+a}{1+e^{\frac{1}{x}}} & \mbox{si } x \not= 0 \\
0 & \mbox{sinon}
\end{array} \right.$
%
\item $ x \rightarrow
\left\{ \begin{array}{ll}
\frac{\sin(x)}{x} & \mbox{si } x \not= 0 \\
1 & \mbox{sinon}
\end{array} \right.$
%
\item $ x \rightarrow
\left\{ \begin{array}{ll}
e^{\frac{-1}{x}} & \mbox{si } x > 0 \\
0 & \mbox{sinon}
\end{array} \right.$
%
\item $ [-\frac{1}{2}, \frac{1}{2}] \rightarrow \R: x \rightarrow
\left\{ \begin{array}{ll}
(\frac{\sin(2x)}{x})^{x+1} & \mbox{si } x \not= 0 \\
1 & \mbox{sinon}
\end{array} \right.$
\end{enumerate}
où $a$ et $b$ sont des réels.


 \exerNico Considérons la fonction
$$f:\mathbb{R}\rightarrow\mathbb{R}:x\mapsto f(x)=\left\{
\begin{array}{ll}
x&\text{si }x\text{ est rationnel}\\
0&\text{si }x\text{ est irrationnel}
\end{array}
\right.$$

Vérifiez que $f$ est continue en $0$ mais n'est ni dérivable à  gauche ni dérivable à droite en
$0$.

\exerNico 
\begin{enumerate}
\item Soit $(X,d)$ un espace métrique et $f \colon (X,d) \to \RR$ une application continue.
Montrer que l'ensemble $$\left\{ x \mid f(x) = 0 \right\}$$ est fermé.

\item Soit $f \colon \RR \to \RR$ une application continue.
Montrer que l'ensemble 
$$
\{ x \in \RR \mid f(x) = x\}
$$
des points fixes de $f$ est fermé.

\end{enumerate}

\exerNico  Soit $A$ un sous ensemble de l'espace métrique $(X,d)$.
Montrer que la fonction
$$
\dist_A \colon X \to \RR,
\quad x \mapsto \inf_{a \in A} d(a,x)
$$ 
est continue.


\exerNico  Soient $(X,d_X)$, $(Y,d_Y)$ deux espaces métriques.
Une application $f \colon X \to Y$ est {\bf Lipschitzienne}
s'il existe une constante $L \ge 0$ telle que
$$
d_Y \bigl( f(x), f(x') \bigr) \,\le\, L \,d_X (x,x') 
\quad \text{ pour tout } x,x' \in X.
$$
Dans ce cas, on dit que $f$ est {\bf $L$-Lipschitzienne}.


\begin{enumerate}
\item
Montrer qu'une application Lipschitzienne est continue.
\item Montrer qu'une application $f \colon \RR \to \RR$, $x \mapsto ax+b$
est Lipschitzienne.
Quelle est la plus petite constante $L$ qui convienne?

\item Montrer que les fonctions $z \mapsto |z|$, 
$z \mapsto \overline z$,
$z \mapsto {\rm Re\,} z$ et $z \mapsto {\rm Im\,} z$ 
de $\eC$ dans $\eR$ sont Lipschitziennes.
Quelle sont les plus petites constantes $L$ qui conviennent?
\item Montrer que la fonction $\dist_A \colon X \to \RR$ de l'Exercice~13 est Lipschitzienne.

\end{enumerate}


% This is part of Exercices et corrigés de CdI-1
% Copyright (c) 2011,2014
%   Laurent Claessens
% See the file fdl-1.3.txt for copying conditions.




\section{Intégration}

\exerNico 
Soient $n,m \in \eN \cup \{0\}$.
Calculer
$$
\int_0^1 x^n (1-x)^m \,dx
\quad \text{ et } \quad
\int_{-1}^1 (1+x)^n (1-x)^m \,dx
$$



\exerNico 
Soient $a,b >0$. 
Calculer
$$
\int_0^{\pi /2} \displaystyle \frac{d \varphi}{a^2 \sin^2 \varphi + b^2 \cos^2 \varphi}
$$


\exerNico  
Calculer la longueur de l'arc de la parabole $y = x^2,\;x \in [0,b]$.

\exerNico  
La {\bf parabole de Neil} $\nu$ est la courbe définie par
$\nu (t) = (t^2,t^3), \, t \in \Rn$.
\begin{enumerate}
\item Esquisser la parabole de Neil.

\item Quelle est la signification du paramètre $t$?

\item Calculer la longueur de l'arc 
$\left\{ \nu (t) \mid t \in [0,\tau] \right\}$.
\end{enumerate}

\exerNico  
Une {\bf hélice} $\gamma$ de pas $2 \pi h$ est une courbe dans $\Rn^3$ définie par
$$
\gamma (t) \,=\, \left( r \cos t , r \sin t , h t \right)  .
$$

\begin{enumerate}
\item Esquisser $\gamma$ et expliquer le mot ``pas''.

\item Calculer la longueur de l'arc sur la hélice si on fait un tour.
\end{enumerate}

\exerNico Calculez la longueur des arcs de courbe suivants:
\begin{enumerate}
\item $y= \ln(1-x^2)  \hspace{3.5cm} 0\leq x\leq \f{1}{2}$
\item  $y= x^{3/2}  \hspace{4.57cm} 0\leq x\leq 5$
\item $y = 1-\ln(\cos x) \hspace{3cm} 0\leq x \leq \f{\pi}{4}$
\item l'arc de cubique déterminé par $y=x^3+x^2+x+1$ avec $0\leq x \leq 1$.
\end{enumerate}


 
\chapter{Corrections en vrac}   % Ce sont des corrections de CdI-1
% This is part of Exercices et corrigés de CdI-1
% Copyright (c) 2011
%   Laurent Claessens
% See the file fdl-1.3.txt for copying conditions.

%+++++++++++++++++++++++++++++++++++++++++++++++++++++++++++++++++++++++++++++++++++++++++++++++++++++++++++++++++++++++++++
					\section{Quelque corrections}
%+++++++++++++++++++++++++++++++++++++++++++++++++++++++++++++++++++++++++++++++++++++++++++++++++++++++++++++++++++++++++++



\noindent 31.
\begin{enumerate}
\item $\dst{df_{(1,1)}}$ et $\dst{dg_{(\sqrt2,\frac{\pi}{4})}}$\\
\[\ba{l}\dst{\dfdx(x,y) = \f{1}{y}\ln(\f{x}{y})e^{\f{x}{y}}+\f{1}{x}e^{\f{x}{y}}}\\
            \dst{\dfdx(1,1)=e}\\
            \dst{\dfdy(x,y)=-\f{x}{y^2}\ln(\f{x}{y})e^{\f{x}{y}}-xe^{\f{x}{y}}}\\
            \dst{\dfdx(1,1)=e}\ea\]
            
 \noindent Par les règles de calcul, $f$ est différentiable en $(1,1)$. la différentielle $df_{(1,1)}$ est donc représentée dans les bases canoniques de $\eR^2$ et $\eR$ par la matrice jacobienne (ici gradient):\[df_{(1,1)}=(e \;\; -e)\]
 
\[\ba{lclllllcl}\dst{\dgudr(r,\theta)} &=&\cos(\theta)& & & & \dst{\dgudth(r,\theta)}   & =&-r\sin(\theta)\\
            \dst{\dgudr(\sqrt2, \f{\pi}{4})}&=&\f{\sqrt2}{2}& & &&\dst{\dgudth(\sqrt2, \f{\pi}{4})}& =&-1 \\
            \dst{\dgddr(r,\theta)} &=&\sin(\theta)&  && &\dst{\dgddth(r,\theta)}  &=&r\cos(\theta) \\
            \dst{\dgddr(\sqrt2, \f{\pi}{4})}&=&-\f{\sqrt2}{2}&& & &\dst{\dgddth(\sqrt2, \f{\pi}{4})}& = &1\ea\]

La fonction $g$ est également différentiable en $(\sqrt2, \f{\pi}{4})$ et sa matrice Jacobienne est:
\[dg_{(\sqrt2, \f{\pi}{4})}=\left(\ba{cc} \f{\sqrt2}{2} & -1\\
							\f{\sqrt2}{2}&1\ea\right)\]	


\item $\dst{\tilde{f} \;=\;e^{\cos(\theta)}\ln(\cos(\theta))}$.
\item On voit d'abord que $g(\sqrt2, \f{\pi}{4})\;=\;(1,1)$. Donc
\[\ba{cccc} d\tilde{f}_{(\sqrt2, \f{\pi}{4})} & = & df_{g(\sqrt2, \f{\pi}{4})}\circ dg_{(\sqrt2, \f{\pi}{4})}\\
							    & =& df_{(1,1)}\circ dg_{(\sqrt2, \f{\pi}{4})} \ea\]
et  la matrice jacobienne de la différentielle de la composée est donc:\[d\tilde{f}_{(\sqrt2, \f{\pi}{4})}=(e\;\;-e)\left(\ba{cc} \f{\sqrt2}{2} & -1\\
							\f{\sqrt2}{2}&1\ea\right)=(0\;\;-2e)\]

							    		

\end{enumerate}


\noindent 32.
\begin{enumerate}
\item $\dst{\dgdu = e^v\dfdx(\star,\star)+2uv\dfdy(\star,\star)}$
\item $\dst{\dgdv = ue^v\dfdx(\star,\star)+(1+u^2)\dfdy(\star,\star)}$
\end{enumerate}
où $\dst{(\star,\star) = (ue^v,v(1+u^2))}$.

\vspace{1cm}

\noindent 33. \\

\noindent Soit $\dst{g:\eR^2\rightarrow \eR:(x,y)\rightarrow  f(x^2-y^2)}$. Dérivées partielles de:\[(x,y)\rightarrow  y(\partial_xg)(x,y)+x(\partial_yg)(x,y)?\]
Nommons cette fonction $h$. 
\begin{enumerate}
\item $\dst{\partial_xg(x,y) = 2xf'(x^2-y^2)}$
\item$\dst{\partial_yg(x,y) = -2yf'(x^2-y^2)}$
\end{enumerate}
et donc $\dst{h(x,y) = 0 \hs \forall (x,y)\in \eR^2}$.

\vspace{1cm}


\noindent 34. \\

\noindent $\dst{h(t)=f(t,g(t^2))}$.\\

\begin{enumerate}
\item $\dst{h'(t)=\dfdx(\star,\star)+\dfdy(\star,\star)2tg'(t^2)}$
\item $ \ba{rl} h''(t)=     &  \dst{ \ddfdx(\star,\star)+4tg'(t^2)\ddfdxy(\star,\star)+4t^2(g'(t^2))^2\ddfdy(\star,\star) }\\     		
				    & \dst{+[2g'(t^2)+4t^2g''(t^2)]\dfdy(\star,\star)}\ea$

\end{enumerate}
où $(\star,\star) = (t,g(t^2))$.

\vspace{1cm}

\noindent 35.
\[h:\eR^2\rightarrow \eR:(u,v)\rightarrow  f(g(ue^v),g(v)(1+u^2))^{g(u+v)}\]

\noindent Comme toujours il vaut mieux faire ce genre d'exercices prudemment. Renommons donc les diverses composantes de cette fonction.\\

\noindent Soit $l(u,v)=f(g(ue^v),g(v)(1+u^2))$. On a alors:
\begin{enumerate}
\item $\dst{\dldu(u,v) = \dfdx(\star,\star)g'(ue^v)e^v + \dfdy(\star,\star)g(v)2u}$
\item $\dst{\dldv(u,v) = \dfdx(\star,\star)g'(ue^v)ue^v+\dfdy(\star,\star)g'(v)(1+u^2)}$
\end{enumerate}
o\`{u} $(\star,\star)=(g(ue^v),g(v)(1+u^2))$.\\

\noindent Alors $\dst{h(u,v)=l(u,v)^{g(u+v)} = e^{g(u+v)\ln(l(u,v))}}$ qui est facile à dériver:

\begin{enumerate}
\item $\dst{\dhdu = [g'(u+v)\ln(l(u,v))+\f{g(u+v)}{l(u,v)}\dldu(u,v)] l(u,v)^{g(u+v)}}$
\item $\dst{\dhdv = [g'(u+v)\ln(l(u,v))+\f{g(u+v)}{l(u,v)}\dldv(u,v)] l(u,v)^{g(u+v)}}$
\end{enumerate}



\noindent 26.
\begin{enumerate}
\item $(x,y)\rightarrow  3x^2+x^3y+x$.\\
Combinaison linéaire de fonctions continues et différentiables sur $\eR^2$ (Exercice: prouver rigoureusement que les polyn\^{o}mes sont bien des fonctions continues et différentiables sur $\eR^2$).

\item  $\dst{x\rightarrow \left\{\ba{c}e \; {\rm si}\;xy\neq0\\
              			        e^{x+y}\;{\rm sinon} \ea\right.}$\\
N.B.: Il est toujours utile de se représenter le domaine de chacune des fonctions. 

\noindent La première remarque est que cette fonction est clairement continue et différentiable en tout point hors de $\{xy=0\}$ (fonction constante). Sur $\{xy=0\}$?
\begin{enumerate}
\item Continuité:\\
Prenons un point dans $\{xy=0\}$, par exemple le point $(a,0)$ (Remarquez que le cas $(0,b)$ est réglé par symétrie). Pour voir si la fonction est continue en ce point il faut voir si \[\dst{\lim_{(x,y)\rightarrow (0,0)}f(x,y)=f(0,0)=e^a}.\] Si on prend deux manières différentes d'aller vers $(a,0)$ ($y=0$ puis $x=a$) on voit que si $a \neq1$ la fonction ne peut pas \^{e}tre continue. Et en $(1,0)$? Si on $(x,y)\rightarrow (1,0)$ avec d'abord $y=0$ puis $y\neq0$ on aura regardétoutes les manières de tendre vers $(1,0)$. Or dans les deux cas les limites valent $e = f(1,0)$, ce qui prouve que la fonction est continue en $(1,0)$ (et $(0,1)$ par symétrie).

\item Différentiabilité:\\
Comme la fonction est discontinue en tout point $(a,0)$ et $(0,b)$ avec $a\neq1$ et $b\neq1$ elle est aussi non différentiable en chacun de ces points. Il reste donc les points $(1,0)$ et $(0,1)$. Comme toujours, nous regardons d'abord les dérivées directionnelles en $(1,0)$:
\[\dfdu(1,0) \;=\;\lim_{t\rightarrow 0}\f{f(1+tu_1,tu_2)-e}{t}\]
Il y a deux possibilités: $u_2=0$ et donc $u=(\pm1.0)$ ou$u_2\neq0$ (pourquoi ne regarde-t-on que ces deux cas?).

\begin{enumerate}
\item si $u\neq(\pm1,0)$.\\
$\dst{\dfdu(1,0) \;=\;\lim_{t\rightarrow 0}\f{e-e}{t}\;=\;0}$.
\item si $u=(\pm1,0)$, i.e. si $u=(1,0) = e_1$\\
$\dst{\dfdu(1,0) = \dfdx(1,0)=\lim_{t\rightarrow 0}\f{f(1+t,0)-e}{t}=\lim_{t\rightarrow 0}\f{e^{1+t}-e}{t} =^H0}$.
\end{enumerate}
\end{enumerate}
\underline{Conclusion}:\\
Si $f$ était différentiable en $(1,0)$, on aurait que sa différentielle prendrait la forme suivante:
\[\ba{cc} df_{(1,0)}u& = \dfdx(1,0)u_1+\dfdy(1,0)u_2\\
			      & = eu_1\;\;\forall u\in\eR^2 \ea \]
Sa différentielle satisferait également \ac:
\[	df_{(1,0)}u = \dfdu(1,0) = 0 \;\; \forall u \neq (\pm1,0) \in \eR^2\]
Les deux propriétés étant contradictoires, la fonction $f$ ne peut \^{e}tre différentiable en $(1,0)$ (ni en $(0,1)$ par symétrie). 		      
\item		  $\dst{x\rightarrow \left\{\ba{c}\f{x}{y} \; {\rm si}\;y\neq0\\
              			       0\;{\rm sinon} \ea\right.}$\\
			       
Continue et différentiable sur $\eR-\{y=0\}$. Sur l'axe $y=0$ elle n'est pas continue.	       
			       
\item		  $\dst{x\rightarrow \left\{\ba{l}x+ay \; {\rm si}\;x>0\\
              			                       x\;{\rm sinon} \ea\right.}$\\
Si $a=0$ fonction continue et différentiable sur $\eR^2$. Si $a\neq0$, fonction continue et différentiable partout en dehors de l'axe $x=0$. Sur cet axe, elle est discontinue en tout point sauf en $(0,0)$ o\`{u} elle est continue. Mais elle n'est pas différentiable en $(0,0)$ car toutes ses dérivées directionnelles  n'y sont pas définies.
	      
\item     $\dst{x\rightarrow \left\{\ba{c}\f{xy^5}{x^6+y^6} \; {\rm si}\;x\neq y\\
              			       0\;{\rm sinon} \ea\right.}$\\
Fonction continue et différentiable partout en dehors de la droite $x=y$.  La fonction est discontinue en chacun des points de cette droite.

\end{enumerate}

30.
\begin{enumerate}
\item $(u,v)\rightarrow  u^3+12u^2v-5v^3$\\
\begin{enumerate}
\item $\dst{\dfdu = 3u^2+24uv}$
\item $\dst{\dfdv = 12u^2-15v^2}$
\end{enumerate}
\item $(u,v)\rightarrow  f(u^2)\ln(v)$\\
\begin{enumerate}
\item $\dst{\dfdu = 2uf'(u^2)\ln(v)}$
\item $\dst{\dfdv = \f{f(u^2)}{v}}$
\end{enumerate}
\item $(x,y)\rightarrow \tan(x+y^2)$\\
\begin{enumerate}
\item $\dst{\dfdx =\f{1}{cos^2(x+y^2)}}$
\item $\dst{\dfdv = \f{2y}{cos^2(x+y^2)}}$
\end{enumerate}
\item $(r,\theta)\rightarrow  r^\theta$
\begin{enumerate}
\item $\dst{\dfdr =\theta r^{\theta-1}}$
\item $\dst{\dfdth =\ln(r)r^\theta}$
\end{enumerate}
\item $(x,y)\rightarrow (x+3)e^x$
\begin{enumerate}
\item $\dst{\dfdx =e^x(x+4)}$
\item $\dst{\dfdy =0}$
\end{enumerate}
\item $(u,v)\rightarrow  \ln(f(uv)) $\\

\begin{enumerate}
\item $\dst{\dfdu = \f{vf'(uv)}{f(uv)}}$
\item $\dst{\dfdv = \f{uf'(uv)}{f(uv)}}$
\end{enumerate}\pagebreak
\end{enumerate}

\noindent 32.
\begin{enumerate}
\item $\dst{\dgdu = e^v\dfdx(\star,\star)+2uv\dfdy(\star,\star)}$
\item $\dst{\dgdv = ue^v\dfdx(\star,\star)+(1+u^2)\dfdy(\star,\star)}$
\end{enumerate}
o\`{u} $\dst{\dfdx(\star,\star) = \dfdx(ue^v,v(1+u^2))}$ et $\dst{\dfdy(\star,\star) = \dfdy(ue^v,v(1+u^2))}$.

\noindent 34. $\dst{h(t)=f(t,g(t^2))}$.\\

\begin{enumerate}
\item $\dst{h'(t)=\dfdx(\star,\star)+\dfdy(\star,\star)2tg'(t^2)}$
\item $ \ba{rl} h''(t)=     &  \dst{ \ddfdx(\star,\star)+4tg'(t^2)\ddfdxy(\star,\star)+4t^2(g'(t^2))^2\ddfdy(\star,\star) }\\     		
				    & \dst{+[2g'(t^2)+4t^2g''(t^2)]\dfdy(\star,\star)}\ea$

\end{enumerate}
où $(\star,\star) = (t,g(t^2))$.




 \section{Intégration}
 \subsection{Série A}
 Exercice 11
 \begin{enumerate}
   \exr $\int \frac{x^3+3x+1}{x} d x = \frac{x^3}3 + 3x + \ln(x)$%
   \exr $\int x^2d x = \frac{x^3}3$%
   \exr $\int 3(x^2+1)^2 d x = \int 3 x^4 + 6 x^2 + 3 d x = \frac 35
   x^5 + 2 x^3 + 3x$%
   \exr $\int (3x^2 - 6x)^3 (x-1) d x = \frac1{12} (3x^2 - 6x)^4$
 \end{enumerate}

 Exercice 12
 \begin{enumerate}
   \exr $\int \sin^2(x^2+1) \cos(x^2+1) x d x = \frac16
   \sin(x^2+1)^3$%
   \exr $\int \tan(x) d x = -\ln\abs{\cos(x)}$%
   \exr $\int \frac{1}{(2+\sqrt{x})\sqrt x} d x= 2 \ln(2+\sqrt{x})$%
   \exr $\int \frac{\ln(x)}{x(1- \ln^2(x)} d x = \frac12
   \ln\abs{1-\ln^2(x)}$%
 \end{enumerate}



   Travaux perso 2 ---------------

   1. Soit deux réels $x$ et $y$ vérifiant $0 < x < y$. On veut montrer
   que pour tout naturel $k \geq 2$, on a
   \[0 < \sqrt[k]{y} - \sqrt[k]{x} < \sqrt[k]{y-x}.\]

   La première inégalité vient de l'inégalité $x < y$ élevée à la
   puissance $\frac1k$.

   On peut ré-écrire la deuxième, sachant que $x > 0$, en divisant par
   $\sqrt[k]{x}$ pour obtenir
   \[\sqrt[k]{\frac yx} - 1 - \sqrt[k]{\frac yx-1} < 0 \quad \text{
     avec $\frac xy > 1$}\] ce qui s'écrit encore $f(t) < 0$ en posant
   $f(t) \pardef \sqrt[k]t - \sqrt[k]{t-1} - 1$. On peut alors étudier
   la fonction $f$. Étant donné que $f(1) = 0$, il suffirait que $f$
   soit strictement décroissante sur $]1;\infty[$ pour qu'on ait
   l'inégalité voulue, à savoir $f(t) < 0$ dès que $t > 1$.

   Pour le montrer, on voit que
   \[f^\prime(t) = \frac 1k \left(t^{\frac{1-k}k} -
     ({t-1})^{\frac{1-k}k}\right)\] d'où on tire les équivalences
   suivantes
   \begin{align}
     & & f^\prime(t) < 0\\
     &\ssi& t^{\frac{1-k}k} < ({t-1})^{\frac{1-k}k}\\
     &\ssi& t^{1-k} < ({t-1})^{1-k}\\
     &\ssi& t > t-1\\
     &\ssi& 0 > -1
   \end{align}
   où la dernière inégalité est manifestement vraie, ce qui prouve la
   première inégalité et achève l'exercice.

   2.


 \paragraph{Exercice 1}
 \begin{enumerate}
 \item Par exemple, $B(x,r)$ avec $x \in \R^n$ et $r > 0$.

 \item On utilise la densité de $\Q$ dans $\R$ pour voir que $B(q,r)$
   ($q \in \Q^n$ et $r > 0$) est également une base.

   On observe ensuite que seuls les $r$ \og petits\fg{} sont utiles,
   donc on se restreint aux boules de la forme $B(q,1/n)$ ($q \in
   \Q^n$ et $n \in \N_0$). Cet ensemble de boules est une base
   dénombrable\marginpar{Pourquoi ?} de la topologie usuelle sur
   $\R^n$.
 \end{enumerate}

 \paragraph{Exercice 2}
 \emph{Principe.} L'idée est de considérer une propriété topologique
 (invariante par homéomorphisme) et de voir qu'elle est vérifiée par
 les ouverts de $\R^2$ mais pas ceux du cône.

 \begin{lem}Si $V$ est un voisinage de $0$ sur le cône $C$, alors
   $V\setminus\{0\}$ n'est pas connexe, donc n'est pas connexe par
   arc.\end{lem}
 \begin{proof}Le cône $C$ est la réunion de $C^+ = C \cap
   \left(\R^2\times \R_0^+\right)$ et $C^- = C \cap \left(\R^2\times
     \R_0^-\right)$ car le seul point à cote nulle est la singularité
   $0$. Dès lors, $V$ s'écrit comme l'union disjointe de $V\cap C^+$
   et $V\cap C^-$, qui sont non-vides. Donc $V$ n'est pas
   connexe.\end{proof}

On procède en deux étapes, en montrant d'abord qu'il
   existe des points en \og dessous\fg{} et au \og dessus\fg{} de
   $0$, puis en essyant de les relier.
     Comme $V$ est un voisinage de $0$, il existe un ouvert $U$ du
     cône centré en $0$ inclu à $V$. Donc par définition de la
     topologie induite, et puisque les boules forment une base de la
     topologie de $\R^3$, il existe une boule $B$ centrée en $0$ dont
     $U$ est la trace sur $C$, telle que $0 \in (B \cap C) \subset
     V$. On choisit $p = (p_x,p_y,p_z) \in (B \cap C)$, et en
     considérant $p^\prime = (p_x, p_y, -p_z)$ on a ainsi trouvé deux
     points qui vérifient $p_z > 0$ et $p^\prime_z < 0$ (au besoin,
     on les échange).

   \begin{enumerate}
   \item Supposons que $V\setminus\{0\}$ soit connexe par arc. Donc
     il existe un chemin
     \[\gamma : [0;1] \to V\setminus\{0\} : t \mapsto
     (\gamma_x(t),\gamma_y(t),\gamma_z(t))\] qui relie $p$ à
     $p^\prime$ et qui vérifie $\gamma_z(0) = p_z > 0$ et
     $\gamma_z(1) = -p_z < 0$. Or $\gamma_z(t)$ est une fonction
     continue (car $\gamma$ est continu), donc par le théorème des
     valeurs intermédiaires, il existe $\bar t$ qui vérifie
     $\gamma_z(\bar t) = 0$. Or le seul point de $C$ dont la cote
     (coordonnée en $z$) soit nulle est le sommet $0$ qui n'est pas
     dans $V\setminus\{0\}$, d'où la contradiction.
   \end{enumerate}

 \begin{rem}Soient deux espaces topologiques $E$ et $F$, et $f :
   E\to F$ un homéomorphisme. Pour toute partie $A$ de $E$,
   l'espace $E\setminus A$ est homéomorphe au sous-espace $F\setminus
   f(A)$ via la restriction $f_{\vert E\setminus A}$.\end{rem}

 \begin{lem}Soient deux espaces topologiques $E$ et $F$, et $f :
   E\to F$ un homéomorphisme. $E$ est connexe par arc si et
   seulement si $F$ l'est.\end{lem}
 \begin{proof}On montre en réalité que l'image d'un connexe par arc
   par une application continue est un connexe par arc, ce qui
   implique chaque sens de l'équivalence de l'énoncé.

   Soient $p$ et $q$ des points de $F$. Il existe un chemin reliant
   un antécédant de $p$ et un antécédant de $q$ (dans $E$). L'image
   de ce chemin est un chemin reliant $p$ et $q$ (dans $F$) puisque
   composé d'applications continues.
 \end{proof}

 \begin{lem}Une sphère de $\R^n$ est connexe par arc si $n >
   1$\end{lem}
 \begin{proof}On voit qu'un cercle est connexe par arc car on a une
   paramétrisation en sinus et cosinus. Pour une sphère $S$ de centre
   $a$ en dimension $n > 2$, on se donne $p$ et $q$ sur $S$ et on
   définit $P$ le plan affin passant par $a$, $p$ et $q$. Alors $P
   \cap S$ est un cercle, donc on peut relier $p$ à $q$ par un chemin
   dans cette intersection.

   Pour voir sur une formule que $P \cap S$ est un cercle, on peut
   écrire $x - a = \lambda(a-p) + \mu(a-q)$ l'équation (en $x$) du
   plan $P$, et $\module{x-a}^2 = R^2$ l'équation (en $x$) de la
   sphère. En injectant, on obtient une équation du second degré en
   $\lambda,\mu$ qui se révèle être l'équation d'un cercle à une
   transformation affine près.
 \end{proof}

 \begin{lem}Un ouvert connexe par arc dans $\R^n$ ($n \geq 2$) reste
   connexe par arc même si on lui enlève un point.\end{lem}
 \begin{proof}
   En effet, soit $U$ un tel ouvert connexe par arc, et $p$ un point
   de $U$. Soient $x$ et $y$ sur $U\setminus\{p\}$. Il existe un
   chemin $\gamma$ de $x$ à $y$. Si le chemin ne passe pas par $p$,
   c'est gagné. Si il passe par $p$, on choisit une boule $B$ fermée
   (de rayon non-nul) centrée en $p$ qui ne contient ni $x$ ni
   $y$. On note
   \[E = \gamma^{-1}(B) \subset [0;1]\] c'est un ensemble compact
   (fermé, par continuité de $\gamma$, et borné) dont on regarde le
   maximum $\bar t$ et le minimum $\underline t$.

   Il reste enfin à définir un chemin entre $p$ et $q$ par morceaux
   \begin{enumerate}
   \item Les points $p$ et $\gamma(\underline t)$ sont reliés par
     $\gamma$,
   \item Par connexité par arc, il existe un chemin sur la sphère qui
     relie $\gamma(\underline t)$ à $\gamma(\bar t)$,
   \item et enfin $\gamma(\bar t)$ et $q$ sont reliés via $\gamma$;
   \end{enumerate}
   ce qui achève la construction d'un chemin continu entre $p$ et
   $q$.
 \end{proof}
 Pour conclure l'exercice, par l'absurde, on prend un voisinage
 connexe et ouvert $V$ de $0$ dans le cône, homéomorphe à un ouvert
 connexe $U$ de $\R^2$. Or $V\setminus\{0\}$ n'est pas connexe par
 arc, alors que l'ouvert dont on retire un point reste connexe par
 arc. C'est impossible, donc l'homéomorphisme n'existe pas, et le
 cône n'est pas une variété de dimension $2$.

% This is part of Exercices et corrigés de CdI-1
% Copyright (c) 2011,2013-2014
%   Laurent Claessens
% See the file fdl-1.3.txt for copying conditions.

\section{Équations différentielles du premier ordre}

\subsection{Exercices}

\paragraph{Exercice 104.}
En remplaçant $y$ dans l'équation par $f(t) = t^4 e^{2t}$,
l'équation devient, après simplifications
\begin{equation*}
\left( \left( b+4\,a+4\right) \,{t}^{4}+\left( 8\,a+16\right)
\,{t}^{3}\right) = 0 
\end{equation*}
qui doit être vraie pour toute valeur de $t$. Un polynôme est nul si
et seulement chacun des coefficients est nul, donc l'équation se
ramène au système
\begin{equation*}
\begin{cases}
b + 4a + 4 = 0\\
8a + 16 = 0
\end{cases}
\end{equation*}
dont l'unique est solution est $(a,b) = (-2,4)$.

\paragraph{Exercice 105.}
\begin{enumerate}
\item $y(t) = \ln(\frac{t^4}{4} + \frac{t^2}{2} + K)$
\item $y(t) = \tan(t + K)$
\item L'intégration directe donne la relation
\begin{equation}\label{eqyplusexpy}
y(t) + e^{y(t)} = \sin(t) + K
\end{equation}
et il faut encore justifier l'éventuelle existence et/ou unicité
d'un tel $y(t)$. Pour ce faire, nous aurons besoin du lemme suivant.

\begin{lemma}
La fonction $f : \eR \to \eR : z \mapsto z + e^z$ est une
bijection.
\end{lemma}
\begin{proof}
La dérivée de $f$ est strictement positive, donc $f$ est strictement
croissante, donc $f$ est injective.

Par ailleurs, étant donné que
\begin{equation*}
\limite z {-\infty} f(z) = -\infty \quad\text{et}\quad \limite z
{+\infty} f(z) = +\infty,
\end{equation*}
le théorème des valeurs intermédiaires (qui affirme que l'image
d'une fonction continue est un intervalle) dit que l'image de $f$
contient n'importe quel intervalle arbitrairement grand, donc
contient $\eR$ entier. Ceci prouve la surjectivité de $f$.
\end{proof}

Ce lemme montre qu'il existe une (unique) fonction $g : \eR \to \eR$
qui est réciproque de $f$. On en déduit, en récrivant l'équation
(\ref{eqyplusexpy}) sous la forme
\begin{equation*}
f(y(t)) = \sin(t) + K
\end{equation*}
et en lui appliquant $g$, que $y(t) = g(\sin(t) + K)$ existe et est
univoquement définie.

\item L'équation $y^\prime = y^2$ pourrait poser un problème pour
trouver des solutions $y$ pour lesquelles il existe $t$ tel que
$y(t) = 0$, car on ne peut alors pas diviser par $y^2$.

On commence par remarquer que $y(t) = 0$ (fonction identiquement
nulle) est une solution de l'équation.

Si $y$ ne s'annule pas sur un certain intervalle fixé, l'équation
s'y écrit
\begin{equation*}
\frac{y^\prime}{y^2} = 1
\end{equation*}
dont les solutions sont de la forme $y(t) = \frac{-1}{t + K}$ (où
$K$ est une constante). Une telle solution ne peut pas tendre vers
$0$, dès lors une solution définie sur un intervalle est soit
identiquement nulle, soit ne s'annule pas du tout.

Si on ne s'intéresse qu'à des fonctions définies sur des
intervalles, il n'y a donc que ces solutions : soit $y(t) = 0$, soit
$y(t)$ est de la forme $\frac{-1}{t + K}$ pour une certaine
constante $K$.

La solution générale sur un domaine quelconque s'obtient en prenant
l'union sur des intervalles disjoints de solutions du type
précédent.

\item L'équation $y^\prime = y^{\frac13}$ pose le même problème que
l'équation précédente, mais la solution est différente.

On remarque à nouveau que la fonction nulle est solution, et on
s'intéresse aux autres solutions.

Si $y$ est une solution qui ne s'annule pas sur un certain
intervalle fixé, elle satisfait
\begin{equation*}
\frac{y^\prime}{y^{\sfrac13}} = 1
\end{equation*}
et donc est de la forme $y(t) = \pm (\frac{2x}{3} +
K)^{\sfrac23}$. Une telle solution est définie pour $x \geq
\sfrac{3K}2$ et tend vers $0$ lorsque $x \to \sfrac{3K}2$, et donc
peut se recoller avec une solution nulle \og sur le bord gauche de
l'intervalle\fg{} (sur le bord droit, la solution tend vers $\pm
\infty$).

La solution générale sur un intervalle est donc une solution de la
forme
\begin{equation*}
y(t) =
\begin{cases}
0& \text{si $x \leq \sfrac{3K}2$}\\
\pm (\frac{2x}{3} + K)^{\sfrac23}& \text{si $x >
\sfrac{3K}2$}\end{cases}
\end{equation*}
pour un certaine constante $K$, ou alors $y(t)$ est identiquement
nulle.

\item L'équation est équivalente à
\begin{equation*}
\frac{yy^\prime}{y^2+1} = - \sin(t)
\end{equation*}
ce qu'on intègre pour obtenir
\begin{equation*}
\frac12\ln(y^2 + 1) = - \cos(t) + K
\end{equation*}
c'est-à-dire $y^2 = -1 + K e^{-2 \cos(t)}$ pour une certaine
constante (positive) $K$. Selon la valeur de $K$, ces solutions sont
définies ou non sur tout $\eR$ :
\begin{enumerate}
\item Si $K > e^{2}$, le membre de droite est strictement positif
pour tout $t$ et on peut en prendre la racine (solution sur $\eR$)
\item Si $K < e^{2}$, le membre de droite est négatif pour
certaines valeurs de $t$ et on ne peut pas en prendre la racine.
\item Si $K = e^{2}$, le choix d'une racine du membre de droite
(positive ou négative) donne une fonction qui n'est pas dérivable
aux points où elle est nulle (car la racine n'est pas dérivable en
ces points). Par contre, si on change de choix de signe pour la
racine à chaque fois que le membre de droite s'annule, la fonction
obtenue est dérivable.
\end{enumerate}

Ces trois cas seraient plus clairs sur une illustration.
% % \psset{unit=1pt,yunit=20pt,plotpoints=300,plotstyle=curve}
% \psaxes{<->}(2,2)(0,0)(4,3)
% \psplot[linecolor=green]{-3.14}{3.14}{x cos}
% % \psplot[linecolor=red]{-360}{360}{-1 9 2.73 2 x cos mul exp mul
% %   add sqrt}%

% $y^2 = -1 + 6*exp(2 cos(x))$ $y^2 = -1 + 9*exp(2 cos(x))$ $y^2 =
% -1 + exp(2 + 2 cos(x))$

% % \begin{mfpic}

% % \end{mfpic}
\end{enumerate}

\paragraph{Exercice 106.}
Il s'agit ici de reprendre les solutions générales de l'exercice
ci-dessus en sélectionnant la ou les solutions qui satisfont au
problème de Cauchy. Dans les cas agréables cela revient simplement à
déterminer la constante. Dans les autres cas, il faut vérifier
l'existence et l'unicité de la solution.


\paragraph{Exercice 116.}
Dans chacun de ces exercices, il s'agit d'intégrer la fonction donnée
sur un domaine précisé.

\begin{enumerate}
\item Le domaine d'intégration est un rectangle, le choix des bornes
est donc simple~:
\begin{equation*}\begin{split}
\int_0^2 \left(\int_0^1 (4 - x^2 - y^2) d x\right) d y%
& = \int_0^2 \left[ 4x - \frac{x^3}3 - y^2x \right]_{x=0}^{x=1}
d y\\%
& = \int_0^2 4 - \frac13 - y^2 d y\\
& =\left[\left(4 - \frac13 \right)y - \frac{y^3}3 \right]_0^2\\
& = 8 - \frac{10}3
\end{split}\end{equation*}
\item
\end{enumerate}

\section{Théorème de la fonction implicite}




\subsection{Exercices}

\paragraph{Exercice 129.}
\begin{enumerate}
\item Une telle fonction $Z$ doit vérifier $F(1,1,Z(1,1)) = 0$, donc
en particulier, en notant $z_0 = Z(1,1)$, il faut $z_0 + \ln(z_0) =
1$. Une solution évidente est $z_0 = 1$.

Montrons que cette solution est unique : la fonction auxiliaire $g :
\eR_0^+ \to \eR : z \mapsto z + \ln(z)$ est strictement croissante
(sa dérivée est strictement positive sur son domaine) et en
particulier injective. L'équation en $z_0$ se réécrit sous la forme
$g(z_0) = 1$, et l'injectivité nous assure l'unicité de la solution.

Au point $(x_0,y_0,z_0) = (1,1,1)$ on peut appliquer le théorème de
la fonction implicite puisque $\pder F z(1,1,1) = 1 + 1 = 2 \neq 0$,
et donc on a un voisinage $U$ de $(1,1)$ et une unique fonction $Z :
U \to \eR$ satisfaisant à la condition énoncée.

\item Notons $Z = Z(x,y)$ pour la simplicité. Pour tout $(x,y)$ dans
$U$, nous avons donc $Z + \ln(Z) - xy = 0$. En particulier on peut
dériver cette identité par rapport à $x$ et à $y$, d'où
\begin{equation*}
\begin{split}
\pder Z x + \frac 1Z \pder Z x - y = 0\\
\pder Z y + \frac 1Z \pder Z y - x = 0
\end{split}
\end{equation*}
pour tout $(x,y) \in U$, et on en tire (on note $\partial_x$ au lieu
de $\pder {} x$)
\begin{equation*}
\partial_x Z = \frac y {1+\frac1Z} = \frac{y Z}{1+Z}
\quad\text{et}\quad   \partial_y Z = \frac x {1+\frac1Z} = \frac {x Z} {1+Z}
\end{equation*}

On peut enfin dériver l'une ou l'autre de ces égalités pour obtenir
les dérivées secondes, en remplaçant ensuite les occurrences de
$\partial_x Z$ et $\partial_y Z$ par leur expression ci-dessus, par
exemple
\begin{equation*}
\begin{split}
\partial^2_{xy} Z &= \frac{(x \partial_x Z + Z)(1+Z) - x
Z \partial_x Z}{(1+Z)^2}\\
&= \frac{(x y Z + Z (1+Z)^2)}{(1+Z)^3}
\end{split}
\end{equation*}
où $\partial^2_{xy}$ désigne la dérivée partielle seconde par
rapport à $y$ puis par rapport à $x$.
\end{enumerate}
\paragraph{Exercice 130.}
Si $x = 1$, l'équation nous donne $1 = y$, donc la fonction doit
vérifier $Y(1) = 1$. Montrons d'abord qu'une telle fonction existe :
on considère
\begin{equation*}
F : \eR_0^+ \to \eR_0^+ : (x,y) \mapsto x^y - y^x
\end{equation*}
et on vérifie que $\partial_y F(1,1) = - 1 \neq 0$. Le théorème de la
fonction implicite s'applique, et fournit effectivement un voisinage
$U$ de $1$ et une unique fonction $Y : U\subset \eR \to \eR$
vérifiant $Y(1) = 1$ et vérifiant l'équation $x^{Y(x)} - {Y(x)}^x$.

Pour calculer la dérivée, on peut ré-écrire l'équation (en notant $Y =
Y(x)$ pour simplifier la notation) sous la forme
\begin{equation*}
Y \ln(x) = x \ln(Y)
\end{equation*}
et en dérivant par rapport à $x$ on obtient
\begin{equation*}
Y^\prime \ln(x) + Y\frac1x = \ln(Y) + x \frac {Y^\prime} Y
\end{equation*}
d'où on tire en réarrangeant les termes
\begin{equation*}
Y^\prime(x) =   \frac Y x \frac{x \ln(Y) - Y}{Y \ln(x) - x} =
\frac{\ln(Y) - \frac Y x}{\ln(x) - \frac x Y}
\end{equation*}

\paragraph{Exercice 131.}
L'équation implicite pour $(x,y) = (0,0)$ devient $z e^z = 0$ dont
l'unique solution est $z = 0$, donc une telle fonction $Z$ doit
vérifier $Z(0,0) = 0$. Pour vérifier l'existence et l'unicité de la
fonction $Z$, on considère la fonction
\begin{equation*}
F : \eR^3 \to \eR : (x,y,z) \mapsto z e^z - x - y.
\end{equation*}
On calcule $\partial_z F(0,0,0) = 1 \neq 0$, de sorte que le théorème
de la fonction implicite s'applique et fournit une unique fonction $Z$
telle que demandée.

Pour écrire le polynôme de Taylor il suffit de calculer les dérivées
de $Z$, ce qu'on fait en utilisant la relation $Z e^Z = x+y$~:
\begin{equation*}
\begin{split}
\partial_x Z (1 + Z) e^Z = 1 \Rightarrow \partial_x Z = \frac{e^{-Z}}{1+Z}\\
\partial_y Z (1 + Z) e^Z = 1 \Rightarrow \partial_y Z = \frac{e^{-Z}}{1+Z}\\
\end{split}
\end{equation*}
où on note comme toujours $Z = Z(x,y)$ pour simplifier l'écriture. On
calcule également les dérivées secondes~:
\begin{equation*}
\partial^2_{xx} Z = \partial^2_{yx} Z = \partial^2_{yy} Z =%
-\frac{Z+2}{(1+Z)^3} e^{-2 Z}
\end{equation*}
et donc le polynôme de Taylor à l'ordre $2$ s'écrit
\begin{equation*}
Z(x,y) = x + y - x^2 - 2 xy - y^2 + o(\norme{(x,y)}^2)
\end{equation*}

\paragraph{Exercice 132.}
Pour $x = \sfrac34$, l'équation $F(\sfrac34,y,z) = 0$ devient le
système
\begin{equation*}
\begin{cases}
\frac9{16} + y^2 + z^2 - 1 = 0\\
\frac9{16} + y^2 - \frac3{4} = 0
\end{cases}
\end{equation*}
qui a exactement les quatre solutions $(y,z) = (\pm
\frac{\sqrt3}4,\pm\frac12)$.

Le jacobien partiel de $F$ par rapport à $(y,z)$ est donné par
\begin{equation}\label{exo132-jacobienpartiel}\tag{*}
\begin{vmatrix}
2 y & 2z\\
2 y & 0
\end{vmatrix} = -4 y z
\end{equation}
et donc est non-nul en chacun des quatre points $(\pm
\frac{\sqrt3}4,\pm\frac12)$, ce qui permet d'appliquer le théorème de
la fonction implicite. Ceci prouve l'existence des quatre fonctions
$\varphi$ demandées, correspondant chacune à un des points ci-dessus.

Pour calculer les dérivées, on sait que ces fonctions vérifient
\begin{equation*}
\begin{cases}
x^2 + Y(x)^2 + Z(x)^2 = 1\\
x^2 + Y(x)^2 = x
\end{cases}
\end{equation*}
pour tout $x$ dans un voisinage de $\sfrac34$. En particulier on peut
dériver ces deux équations pour obtenir (on note $Y = Y(x)$ et $Z =
Z(x)$ pour alléger la notation)~:
\begin{equation*}
\begin{cases}
2 x + 2 Y Y^\prime + 2 Z Z^\prime = 0\\
2 x + 2 Y Y^\prime = 1
\end{cases}
\end{equation*}
d'où on tire
\begin{equation*}
\begin{cases}
Z^\prime = \frac {-1}{2Z}\\
Y^\prime = \frac{1 - 2 x}{2 Y}
\end{cases}
\end{equation*}
où on remarque que la division par $Y$ et par $Z$ est bien définie si
$x$ est assez proche de $\sfrac34$ puisque le jacobien partiel
(\ref{exo132-jacobienpartiel}) est non nul.

\begin{remark}
La formule donnant la dérivée ne dépend pas explicitement du point
autour duquel on fait le calcul, mais dépend bien sûr encore de la
valeur de $Z$ en ce point.
\end{remark}

\paragraph{Exercice 133.}
Étant donnée la relation, on vérifie que la fonction $Y$ définie
implicitement par
\begin{equation*}
e^{yx} - 1 = x^2 + y
\end{equation*}
doit satisfaire à $Y(0) = 0$. Une première tentative montre que la
limite demandée est donc du type indéterminé \og $\frac00$\fg{}. Afin
d'appliquer la \emph{règle de L'hospital} on veut d'abord vérifier que
la fonction $Y$ est bien dérivable autour de $0$.

Pour ce faire, on considère
\begin{equation*}
F : \eR^2 \to \eR : (x,y) \mapsto e^{xy} - 1 - x^2 - y
\end{equation*}
et on a $\partial_y F(0,0) = - 1 \neq 0$, donc le théorème de la
fonction implicite s'applique et assure l'existence d'une fonction $Y$
de classe $C^\infty$ autour de $0$ vérifiant
\begin{equation*}
e^{x Y} - 1 = x^2 + Y
\end{equation*}
où on note $Y = Y(x)$ pour alléger la notation. En particulier sa
dérivée doit satisfaire à l'équation
\begin{equation*}
e^{x Y} (Y + x Y^\prime) = 2 x + Y^\prime
\end{equation*}
et donc, pour $x = 0$ on obtient
\begin{equation*}
Y(0) = Y^\prime(0)
\end{equation*}
ce qui montre que la dérivée s'annule en $0$.

La limite devient donc
\begin{equation*}
\limite x 0 \frac{Y(x)}{\cos(x) - 1} = \limite x 0 \frac{Y^\prime(x)}{-\sin(x)}
\end{equation*}
et est à nouveau du type indéterminé \og $\frac00$\fg{}.

Dérivons à nouveau l'équation satisfaite par $Y^\prime$ pour obtenir
\begin{equation*}
e^{x Y} (Y + x Y^\prime)^2 + e^{x Y} (Y^\prime + Y^\prime + x
Y^{\prime\prime}) = 2 + Y^{\prime\prime}
\end{equation*}
ce qui, pour $x = 0$, donne
\begin{equation*}
Y(0)^2 + 2 Y^\prime(0) = 2 + Y^{\prime\prime}(0)
\end{equation*}
et donc $Y^{\prime\prime}(0) = -2$ puisque $Y(0) = Y^\prime(0) = 0$.

La limite devient donc
\begin{equation*}
\limite x 0 \frac{Y(x)}{\cos(x) - 1} = \limite x 0
\frac{Y^{\prime\prime}(x)}{-\cos(x)} = 2
\end{equation*}
et la réponse attendue est donc $2$.

\paragraph{Exercice 134.}
Au voisinage du point $(0,1)$ la courbe peut s'écrire sous la forme $y
= Y(x)$ par application du théorème de la fonction implicite (le
vérifier !). La tangente à la courbe est alors donnée par l'équation
\begin{equation*}
y - 1 = Y^\prime(0) (x - 0)
\end{equation*}
ce qui implique de calculer $Y^\prime(0)$.

Par définition $Y(x)$ vérifie $Y(x)^2 + \sin(x Y(x))  = 1$ et donc sa
dérivée vérifie
\begin{equation*}
2 Y(x) Y^\prime(x) + \cos(x Y(x)) (Y(x) + x Y^\prime(x)) = 0
\end{equation*}
et donc en $x = 0$, on obtient
\begin{equation*}
2 Y(0) Y^\prime(0) + Y(0) = 0
\end{equation*}
ce qui donne $Y^\prime(0) = -\sfrac{1}{2}$ puisque sachant que $Y(0) =
1$.

\paragraph{Exercice 135.}

Nous savons que pour une surface donnée sous forme implicite $F(x,y,z) =
k$, le vecteur gradient $\nabla F$ en un point de la surface est normal
à cette surface.

Dans le cadre de cet exercice, on est donc ramené à chercher les
points sur la surface dont le gradient est un multiple de $(1,-2,2)$
(qui est le vecteur normal au plan donné).

Le gradient au point $(x,y,z)$ est donné par $(8x,32y,16z)$, et on
veut qu'il existe $\lambda \neq 0$ tel que $(8x,32y,16z) =
(\lambda,-2\lambda,2\lambda)$. Cette condition combinée à la condition
d'appartenance à la surface fournit une équation pour $\lambda$ qu'il
suffit de résoudre.

La fin de l'exercice dépend de la manière de compléter l'énoncé,
puisqu'il faut expliciter l'équation des plans tangents.

\paragraph{Exercice 136.}
\begin{enumerate}
\item Étant donné que $ M = F^{-1}(0,0)$ et que $\{(0,0)\}$ est un
ensemble fermé (ne contient qu'un seul point !), on en déduit que
$ M$ est l'image réciproque par une fonction continue d'un
fermé, donc $ M$ est un ensemble fermé (vérifier !).  Par
ailleurs, $ M$ est complètement contenue dans la sphère de rayon
$1$, donc est bornée. Ces deux propriétés fournissent la compacité.

L'ensemble $ M$ est donné sous forme implicite par l'annulation
de deux fonctions, dont les gradients sont $(1,1,1)$ et $(2x,
2y,2z)$. Ces deux vecteurs linéairement dépendant si et seulement si
$x = y = z$, ce qui n'est pas possible pour un point de $ M$. On
en déduit que les gradients sont indépendants sur $ M$, et donc
que $ M$ est une variété $C^1$ de dimension $3 - 2 = 1$.

En fait, c'est l'intersection d'une sphère et d'un plan, c'est donc
un cercle.

\item Pour $y = 0$, on observe qu'il faut que $X(0) + Z(0) = 0$ et
$X(0)^2 + Z(0)^2 = 1$. On en déduit que $(X(0),Z(0))$ vaut soit
$\frac{\sqrt2}{2},-\frac{\sqrt2}{2})$, soit
$(-\frac{\sqrt2}{2},\frac{\sqrt2}{2})$. Pour chacun de ces points le
théorème de la fonction implicite s'applique et fournit donc deux
paires de fonctions $(X_1,Z_1)$ (avec $X_1(0) = \frac{\sqrt2}{2}$) et
$(X_2,Z_2)$ (avec $X_2(0) = -\frac{\sqrt2}{2}$).

\item La meilleure approximation polynômiale de degré $1$ est donnée
par le polynôme de Mc Laurin d'ordre $1$, donc on calcule la dérivée
première des fonctions $X_1$ et $X_2$.

En dérivant les équations qui définissent $X_1$ et $Z_1$ (et $X_2$
et $Z_2$ également, ce sont les mêmes !), on obtient les relations
\begin{equation*}
\begin{cases}
X_1^\prime + 1 + Z_1^\prime = 0\\
2 X_1 X_1^\prime + 2 y + 2 Z_1 Z_1^\prime = 0
\end{cases}
\end{equation*}
ce qui, pour $y = 0$, fournit $X_1^\prime(0) = Z_1^\prime(0) =
-\sfrac12$. Et en mettant un indice ${}_2$ partout, on obtient la
même chose $X_2^\prime(0) = Z_2^\prime(0) = -\sfrac12$.

Le polynôme de Mc Laurin pour $X_1$ et $X_2$ s'écrit donc
\begin{equation*}
\begin{split}
X_1(y) &= \frac{\sqrt2}{2} - \frac12 y + o(\abs y)\\
X_2(y) &= -\frac{\sqrt2}{2} - \frac12 y + o(\abs y)\\
\end{split}
\end{equation*}
\end{enumerate}

\paragraph{Exercice 137.}
\begin{enumerate}
\item Considérons l'application $F : \eR\times\eR^2 \to \eR^2 :
(t,v) \mapsto v - \varphi_t(v)$.  On sait que $F(0,v_0) = 0$ par
définition de $v_0$. On sait également que pour $t = 0$, la
différentielle de l'application partielle $\eR^2 \to \eR^2 : v
\mapsto F(v)$ vaut $\Id - d\varphi_0$ et est donc inversible par
hypothèse sur le spectre. On en déduit que le théorème de la
fonction implicite s'applique, et qu'il existe un voisinage
$\mathopen\rbrack-\epsilon,\epsilon\mathclose\lbrack$ de $t = 0$, un
voisinage $U$ de $v = v_0$ et une unique application
\begin{equation*}
V : \mathopen\rbrack-\epsilon,\epsilon\mathclose\lbrack \to U
\end{equation*}
telle que $F(t,V(t)) = 0$ pour tout $t \in
\mathopen\rbrack-\epsilon,\epsilon\mathclose\lbrack$. C'est-à-dire
$V(t) = \varphi_t(V(t))$ pour $t \in
\mathopen\rbrack-\epsilon,\epsilon\mathclose\lbrack$, et donc $V(t)$
est un point fixe de $\varphi_t$.

\item Les applications $\varphi_t : (x,y) \mapsto (x+t,y)$ n'ont pas
de point fixe, sauf pour $t = 0$. A titre d'information, on dit que
c'est une action (la variable $t$ \og agit\fg{} puisqu'elle
translate vers la droite) \emph{libre} (sans point fixe autre que $t
= 0$) de la droite $\eR$ sur le plan $\eR^2$.
\end{enumerate}

\section{Intégrales curvilignes}

\subsection{Exercices}
\paragraph{Exercice 144$^\prime$}
\begin{enumerate}
\item Une paramétrisation est donnée, il reste à intégrer
\begin{equation*}
\int_0^{2\pi} \norme{\gamma^\prime(t)} d t
\end{equation*}
où $\gamma^\prime(t) = (a (1 - \cos(t)), a \sin(t))$, c'est-à-dire
\begin{equation*}
\int_0^{2\pi} a \sqrt{2 - 2 \cos(t)} d t = 8a
\end{equation*}
où on a utilisé l'égalité trigonométrique
\begin{equation*}
1 - \cos(t) = 2\sin^2\left(\frac t2\right).
\end{equation*}

\item On sait qu'on peut paramétriser cet astroïde par
\begin{equation*}
\begin{cases}
\sqrt[3]x = \sqrt[3]a \cos(t)\\
\sqrt[3]y = \sqrt[3]a \sin(t)
\end{cases}
\end{equation*}
ce qui donne le chemin
\begin{equation*}
\gamma(t) = (a \cos^3(t), a \sin^3(t)) \donc \gamma^\prime(t) = (-
3 a \cos^2(t) \sin(t), 3 a \sin^2(t) \cos(t))
\end{equation*}
et l'intégrale devient, grace aux relations trigonométriques,
\begin{equation*}
\int_0^{2\pi} 3 a \sqrt{\sin^2(t)\cos^2(t)} d t = \frac{3a}2 \int_0^{2\pi} \abs{\sin(2t)} d t
\end{equation*}
où il faut encore faire attention au signe. Par périodicité et par
symétrie, on se ramène à 
\begin{equation*}
6a \int_0^{\frac\pi2} \sin(2t) d t = 6 a
\end{equation*}

\item Le chemin peut être paramétrisé de la manière suivante
\begin{equation*}
\gamma(t) =%
\begin{cases}
(t+1,0) & -1 \leq t \leq 0\\
(1-t,t) & 0 \leq t \leq 1\\
(0,2-t) & 1 \leq t \leq 2
\end{cases}
\end{equation*}
c'est un chemin $C^1$ par morceaux% , où les morceaux sont
%   $\mathopen\rbrack-1,0\mathopen\lbrack$,
%   $\mathopen\rbrack0,1\mathopen\lbrack$ et
%   $\mathopen\rbrack1,2\mathopen\lbrack$
. On a
\begin{equation*}
\gamma^\prime(t) =%
\begin{cases}
(1,0) & -1 < t < 0\\
(-1,1) & 0 < t < 1\\
(0,-1) & 1 < t < 2
\end{cases}%
\quad\text{et donc}\quad%
\norme{\gamma^\prime(t)} =%
\begin{cases}
1 & -1 < t < 0\\
\sqrt2 & 0 < t < 1\\
1 & 1 < t < 2
\end{cases}%
\end{equation*}
et l'intégrale devient donc
\begin{equation*}
\int_{-1}^0 (t+1) d t + \int_0^1 \sqrt2 d t + \int_1^2 (2-t) \d
t = 1+\sqrt2
\end{equation*}
\end{enumerate}


\paragraph{Exercice 143 = 145$^\prime$.}
\begin{enumerate}
\item Il faut calculer
\begin{equation*}
\int_0^{\frac{1}{2}} \sqrt{1 + \frac{4 x^2}{(1-x^2)^2}} d x
\end{equation*}

\item Calculer
\begin{equation*}
\int_0^{5} \sqrt{1 + \frac94 x} d x
\end{equation*}

\item Calculer
\begin{equation*}
\int_0^{\frac\pi4} \sqrt{1 + \tan^2(x)}\ d x = \int_0^{\frac\pi4} \frac{1}{\cos(x)} d x
\end{equation*}
\end{enumerate}

\paragraph{Exercice 146$^\prime$}
\begin{enumerate}
\item On paramétrise ce cercle de la façon usuelle mais en prenant
attention au sens
\begin{equation*}
\gamma(t) = (\cos(-t),\sin(-t)) = (\cos(t),-\sin(t)) \qquad t \in [0,2\pi]
\end{equation*}
d'où on tire $\gamma^\prime(t) = (-\sin(t),-\cos(t))$ et l'intégrale
devient
\begin{equation*}
\int_0^{2\pi} (-\sin^3(t) - \cos^3(t)) d t = 0.
\end{equation*}
L'intégrale est nulle parce que ces deux fonctions ($\sin^3$ et
$\cos^3$) possèdent un centre de symétrie, et qu'on les intègre sur
une période complète.

\item Le chemin se paramétrise par 
\begin{equation*}
\gamma(t) = (\cos(t),\sin(t)) \donc \gamma^\prime(t) = (-\sin(t),\cos(t))
\end{equation*}
et donc l'intégrale devient
\begin{equation*}
\int_0^{2\pi} \scalprod{G(\gamma(t))}{\gamma^\prime(t)}d t =
\int_0^{2\pi} \left(-\sin^3(t) + \cos^3(t)\right)d t = 0
\end{equation*}
Cette intégrale est nulle pour les même raisons que ci-dessus.

\item
La paramétrisation est donnée, et on a 
\begin{equation*}
	\gamma^\prime(t) = (-a \sin(t), a \cos(t), b)
\end{equation*}
donc l'intégrale devient
\begin{equation}
	\begin{aligned}[]
		\int_0^{2\pi} \Big[ \big(a \sin(t) - b t\big) (-a\sin(t)) &+ \big(bt - a \sin(t)\big) a \cos(t) \\
		&+ ab(\cos(t)-\sin(t)) \Big] dt
	\end{aligned}
\end{equation}
et vaut $- \pi a (a+ 2b)$ après calculs.

\item
Sur le chemin donné, l'intégrale vaut
\begin{equation*}
	\int_\gamma -x d x - y d y - z d z = \int_\gamma - \frac12 d f
\end{equation*}
où $f(x,y,z) = x^2+y^2+z^2$. Cette intégrale est donc nulle puisque le chemin $\gamma$ est fermé (c'est un cercle).

On peut même aller plus loin : sur la sphère $f \equiv 1$ donc $df= 0$.
\end{enumerate}

\paragraph{Exercice 144 = 147$^\prime$.}
Ayant $y = x^3 + x^2 + x + 1$ on a $d y = (3 x^2 + 2 x + 1)d x$ donc
il faut calculer
\begin{equation*}
\int_0^1 (x (3 x^2 + 2 x + 1) + x^3 + x^2 + x + 1)d x
\end{equation*}

\paragraph{Exercice 145 = 148$^\prime$.}
Il faut calculer
\begin{enumerate}
\item
\begin{equation*}
\int_0^2 (x^2 + \frac{x^2}{2}) d x
\end{equation*}
\item
\begin{equation*}
\int_0^2 (\frac{x^3}{2} + \frac{x^3}{2}) d x
\end{equation*}
\item 
\begin{equation*}
\int_0^1  (16 y^4 + 4 y^4)d y
\end{equation*}
\item La ligne brisée est un peu inutile...
\end{enumerate}
Ces quatres résultats sont identiques, ce qui laisse penser qu'en
réalité la $1$-forme $\omega = 2 x y d x + x^2 d y$ est une forme
exacte. En effet, on vérifie aisément que $\omega = d (x^2 y)$.

\paragraph{Exercice 146 = 149$^\prime$.}
On peut paramétriser l'hélice par
les équations
\begin{equation*}
\begin{cases}
x = \cos(2 \pi t)\\
y = \sin(2 \pi t)\\
z = t
\end{cases}\qquad t \in [0;1]
\end{equation*}
et l'intégrale devient
\begin{equation*}
\int_0^1 \left(-2\pi \sin^2(2\pi t) + 2\pi \cos^2(2\pi t) +
t^2\right) d t = \frac13  
\end{equation*}

On pouvait aussi remarquer que
\begin{equation*}
y d x + x d y + z^2 d z = d f \quad\text{où $f(x,y,z) = xy + \frac13 z^3$}
\end{equation*}
et donc l'intégrale vaut bien $f(1,0,1) - f(1,0,0) = \sfrac13$.

\section{Intégrales de surface}





\paragraph{Exercice 147.}
\begin{enumerate}
\item La sphère est paramétrisée en coordonnées sphériques (avec un
rayon $r$ constant) par
\begin{equation*}
\begin{cases}
x = r \cos(u)\sin(v)\\
y = r \sin(u)\sin(v)\\
z = r \cos(v)
\end{cases}\qquad\text{avec}\quad
\begin{cases}
u \in \llbrack{0;2\pi}\\
v \in \llbrack{0;\pi}
\end{cases}
\end{equation*}
c'est-à-dire la paramétrisation est
\begin{equation*}
F : \llbrack{0;2\pi} \times \llbrack{0;\pi} \to \eR^3 : (u,v)
\mapsto (r \cos(u)\sin(v), r\sin(u)\sin(v), r\cos(v))
\end{equation*}
et on calcule que l'élément de surface vaut
\begin{equation*}
\norme{\pder F u \wedge \pder F v} d u d v = r^2 \sin (v) d u d v
\end{equation*}
Dès lors, la surface de la sphère vaut
\begin{equation*}
\int_0^\pi \int_0^{2\pi} r^2 \sin(v) d u d v = 2r^2 \pi \crochets{-\cos(v)}_0^\pi = 4 \pi r^2
\end{equation*}

\item En coordonnées cylindriques $(\rho,\theta,z)$, le cylindre plein
de rayon $r$ tel qu'il est décrit a pour équation
\begin{equation*}
\rho \leq r \cos(\theta)
\end{equation*}
et la sphère s'écrit $\rho^2 + z^2 = r^2$. On peut donc paramétriser
la surface demandée via
\begin{equation*}
\begin{split}
F_1 \equiv x &= \rho \cos \theta\\
F_2 \equiv y &= \rho \sin \theta\\
F_3 \equiv z &= \pm \sqrt{r^2 - \rho^2}
\end{split}
\end{equation*}
où le signe $\pm$ indique qu'il y a deux morceaux de surface.

On peut calculer le produit vectoriel
\begin{equation*}
\pder F \theta \wedge \pder F \rho =
\begin{vmatrix}
\vect{e_x} & \vect{e_y} & \vect{e_z}\\
-\rho\sin\theta & \rho\cos\theta & 0\\
\cos\theta & \sin\theta & \pm \frac{-\rho}{r^2-\rho^2}
\end{vmatrix} = (\pm\frac{-\rho^2 \cos\theta}{r^2-\rho^2}, \pm
\frac{- \rho^2 \sin\theta}{r^2-\rho^2}, -\rho)
\end{equation*}
dont la norme donne l'élément de surface
\begin{equation*}
d \sigma = \frac{r \rho}{\sqrt{r^2-\rho^2}} d \rho d \theta
\end{equation*}
ce qui permet de calculer l'intégrale sur l'un des morceaux :
\begin{equation*}
\begin{split}
\int_{-\frac\pi2}^{\frac\pi2} \int_0^{r\cos\theta}\frac{r
\rho}{\sqrt{r^2-\rho^2}} d \rho d \theta %
&= r \int_{-\frac\pi2}^{\frac\pi2} \crochets{- \sqrt{r^2-\rho^2}}_0^{r\cos\theta} \theta\\
&= r \int_{-\frac\pi2}^{\frac\pi2} (- r \abs{\sin\theta} + r)\\
	&=d\theta = \pi r^2 - 2 r^2 = r^2 (\pi - 2)
\end{split}
\end{equation*}

\item Le morceau de cône se accepte la paramétrisation suivante en coordonnées cylindriques :
\begin{equation*}
\rho = \abs z \quad 0 < z < b
\end{equation*}
et puisque $z > 0$, on a $z = \rho$. On peut donc expliciter le
changement de coordonnées~:
\begin{equation*}
\begin{cases}
F_1 \equiv x = \rho \cos \theta\\
F_2 \equiv y = \rho \cos \theta\\
F_3 \equiv \rho = \rho
\end{cases} \qquad \text{avec $0 < \rho < b$}
\end{equation*}
l'élément de surface peut alors se calculer comme suit
\begin{equation*}
\pder F \rho \wedge \pder F \theta = %
\begin{vmatrix}
\vect{e_x} & \vect{e_y} & \vect{e_z}\\
\cos\theta & \sin\theta & 1\\
-\rho \sin\theta & \rho \cos \theta & 0
\end{vmatrix} = (- \rho \cos\theta, - \rho \sin\theta, \rho),
\end{equation*}
donc $d\sigma = \abs \rho \sqrt2 = \rho \sqrt 2$, et l'intégrale devient
\begin{equation*}
\int_0^b \int_0^{2\pi} \rho^2 \sqrt 2 d \theta \rho = \frac {2\sqrt2
\pi}3 b^3
\end{equation*}

\item La sphère se paramétrise en coordonnées sphériques par
\begin{equation*}
\begin{cases}
x = \cos u \sin v\\
y = \sin u \sin v\\
z = \cos v
\end{cases} \qquad\text{avec }%
\begin{cases}
u \in \llbrack {0, 2\pi}\\
v \in \llbrack {0, \pi}
\end{cases}
\end{equation*}
dont l'élément de volume a déjà été calculé et vaut $\sin(v) d u \d
v$.

L'équation du cône devient
\begin{equation*}
\cos^2 v = \sin^2 v
\end{equation*}
c'est-à-dire $\sin v = \abs{\cos v}$ (car $\sin v \geq 0$), donc les
choix possibles sont
\begin{equation*}
v = \frac\pi4 \qquad\text{ou}\qquad v = \frac{3\pi}4
\end{equation*}
Étant donné qu'on veut être \emph{dans} le cône, il faut
\begin{equation*}
v \in \lrbrack{0, \frac\pi4}\cup \llbrack{\frac{3\pi}4, \pi}
\end{equation*}
et donc l'intégrale devient
\begin{equation*}
\int_0^{2\pi} \left(\int_0^{\frac\pi4} \sin(v)d v +
\int_{\frac{3\pi}4}^\pi \sin(v) d v\right)d u =
\int_{0}^{2\pi} (2 - \sqrt 2) d u = 2 \pi (2 - \sqrt 2).
\end{equation*}

\item En coupant le cylindre le long de la droite
\begin{equation*}
\begin{cases}
x = 0\\ y = 1
\end{cases}
\end{equation*}
on remarque que la surface limitée par l'hélice
\begin{equation*}
\begin{cases}
x = \sin t\\
y = \cos t\\
z = \frac t {2\pi}
\end{cases}
\end{equation*}
est un triangle dont la base est la circonférence du cylindre et la
hauteur est le pas de l'hélice, c'est-à-dire son aire vaut $\frac12
(2\pi \cdot 1) = \pi$.
\end{enumerate}

\paragraph{Exercice 150.}
\begin{enumerate}
\item Si $P(x,y) = 2 (x^2 + y^2)$ et $Q(x,y) = (x+y)^2$, l'intégrale
demandée rentre dans les conditions du théorème de Green, puisque le
domaine est le périmètre $\gamma$ d'un triangle plein $T$, dont le bord
admet clairement en chaque point un vecteur normal extérieur. Dès
lors
\begin{equation*}
\begin{split}
\int_\gamma 2 (x^2 + y^2) d x + (x+y)^2 d y &= \iint_T (2 x -
2 y )d x d y\\
&= \int_1^2\int_1^y 2 (x-y) d x d y\\
&= \int_1^2 \crochets{x^2 - 2 x y}_1^y d y\\
&= \int_1^2 (- y^2 - 1 + 2 y) d y = - \frac13
\end{split}
\end{equation*}

\item Le théorème de Green s'applique, et il s'agit donc d'intégrer
\begin{equation*}
- \iint_S 4 x y d x d y
\end{equation*}
où $S$ est le disque de rayon $R$. Par passage en coordonnées polaires, on trouve
\begin{equation*}
- \int_0^{2\pi} \int_0^R 4\rho^3 \sin\theta \cos\theta d \rho d\theta = - R^4 \frac12 \crochets{\sin^2(\theta)}_0^{2\pi} = 0
\end{equation*}
ce qu'on aurait pu deviner en utilisant les symétries du problème.

\item D'après le théorème de Green, on a
\begin{equation*}
\int_\gamma d x + x d y = \int_0^1 \int_{y^2}^{\sqrt y} d x
d y = \frac13
\end{equation*}

\item Tout le monde sait que l'ellipse $E$ a pour aire $\pi a b$. On
peut aussi le calculer en utilisant le théorème de Green~:
\begin{equation*}
\iint_E d x d y = \int_\gamma (d x + x d y)
\end{equation*}
mais c'est plus compliqué.

\item Par le théorème de Green, cette intégrale vaut
\begin{equation*}
-\iint_D x^2 + y^2 = -\int_0^{2\pi}\int_0^1 \rho^3 d\rho d \theta
= -\frac\pi2
\end{equation*}
où $D$ est le disque unité. Ne pas oublier le signe qui vient du fait que l'on demande de tourner dans le sens horloger, alors que $\int_{\partial D}\omega$ n'est égal à $\int_{\gamma}\omega$ que lorsque $\gamma$ parcours $\partial D$ dans le sens trigonométrique, c'est à dire dans l'autre sens.
\end{enumerate}

\paragraph{Exercice 151.}
\begin{enumerate}

\item Si on applique le théorème de Stokes dans le plan $z = 0$, cela revient en fait à appliquer le théorème de Green, en effet le rotationnel vaut $(0, 0, 2 x - 2 y)$ et le vecteur normal vaut $(0,0,1)$. L'intégrale devient donc une intégrale sur le disque $D$ unité dans le plan $z = 0$
\begin{equation*}
\iint_D \nabla\times G \cdot d S = \iint_D (2x - 2 y) d x d y
\end{equation*}
et elle vaut zéro par les symétries (ou par calcul).

Si on applique le théorème de Stokes à la demi sphère unité, on se rappelle que le rotationnel vaut
\begin{equation*}
\nabla\times G = (0, 0, 2x - 2 y)
\end{equation*}
et que pour calculer le flux de $\nabla\times G$ au travers de la demi sphère $S$, on peut utiliser le théorème de la divergence. En effet, celui-ci établit que
\begin{equation*}
\iint_D \nabla\times G \cdot d S + \iint_S \nabla\times G \cdot d S = \iiint_V
\nabla\cdot \nabla\times G
\end{equation*}
où $V$ est la demi sphère unité pleine. Or $\nabla\cdot\nabla\times G = 0$. On en
déduit que l'intégrale est nulle. D'après le calcul ci-dessous, nous
avons donc
\begin{equation*}
\iint_S \nabla\times \cdot d S = 0 - 0 = 0.
\end{equation*}

\item Soit $\gamma(t) = (\cos(t), \sin(t), 0)$. L'intégrale demandée
est
\begin{equation*}
\int_0^{2\pi} (\cos^3(t) \sin(t) + \sin^3(t) \cos(t)) d t = 0
\end{equation*}
ou, par la formule de stokes~:
\begin{equation*}
\iint_D \scalprod{(0,2,0)}{(0,0,1)} d x d y = 0
\end{equation*}
où $(0,0,1)$ représente le vecteur normal au disque unité $D$ dans
le plan $z = 0$.

\item Utilisons le théorème de Stokes. Le rotationnel de $G =
(y+z,z+x,x+y)$ vaut $\nabla\times G = (0,0,0)$. Voilà qui est réglé.

\item 
Le rotationnel de $G = (y,z,x)$ vaut $(-1,-1,-1)$. Le vecteur normal au disque $x^2 + z^2 = a^2$ dans le plan $y = 0$ (ordre des paramètres~: $(x,z)$)vaut $(1,0,0) \wedge (0,0,1) = (0,-1,0)$. On peut donc calculer l'intégrale de
\begin{equation*}
\iint_D d x d y = \pi a^2
\end{equation*}
puisque $\iint_D d x d y$ est l'aire du disque de rayon $a$. On en conclut que
\begin{equation*}
\int_\gamma y d x + z d y + x d z = \pi a^2
\end{equation*}
où $\gamma$ est le cercle donné, orienté par le vecteur $(0,0,1)$ au point $(1,0,0)$.
\end{enumerate}


