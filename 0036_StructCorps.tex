% This is part of Mes notes de mathématique
% Copyright (c) 2011-2014
%   Laurent Claessens
% See the file fdl-1.3.txt for copying conditions.

%---------------------------------------------------------------------------------------------------------------------------
\subsection{Matrices}
%---------------------------------------------------------------------------------------------------------------------------

\begin{proposition}
    Nous avons
    \begin{equation}
        | \GL(n,\eF_p) |=(p^n-1)(p^n-p)\ldots (p^n-p^{n-1}).
    \end{equation}
\end{proposition}

\begin{proof}
    Par construction il existe une bijection entre \( \GL(n,\eF_p)\) et l'ensemble des bases de \( \eF_p^n\). Nous devons donc seulement compter le nombre de bases. Pour le premier vecteur de base nous avons le choix entre les \( p^n-1\) éléments non nuls de \( \eF_p^n\). Pour le second nous avons le choix entre \( p^n-p\) éléments, et ainsi de suite.
\end{proof}

%+++++++++++++++++++++++++++++++++++++++++++++++++++++++++++++++++++++++++++++++++++++++++++++++++++++++++++++++++++++++++++
\section{Minuscule morceau sur la théorie de Galois}
%+++++++++++++++++++++++++++++++++++++++++++++++++++++++++++++++++++++++++++++++++++++++++++++++++++++++++++++++++++++++++++

Vous trouverez des détails et des preuves à propos de la théorie de Galois dans \cite{GalIEl}.

\begin{definition}
    Soit $\eK$, un corps.
    
    Le \defe{groupe de Galois}{groupe!de Galois} d'une extension \( \eL\) de \( \eK\) est le groupe des automorphismes de \( \eL\) laissant \( \eK\) invariant. 

    Le groupe de Galois d'un polynôme sur \( \eK\) est le groupe de Galois de son corps de décomposition sur \( \eK\).
\end{definition}

\begin{definition}
    Des éléments \( b_1,\ldots, b_n\) d'une extension de \( \eK\) sont \defe{algébriquement indépendants}{algébriquement!indépendant}\index{indépendance!algébrique} si ils ne satisfont à aucune relation du type
    \begin{equation}
        \sum \alpha_{i_1\ldots i_n}b_1^{i_1}\ldots b_n^{i_n}=0
    \end{equation}
    avec \( \alpha_{i_1\ldots i_n}\in \eK\).
\end{definition}

Nous disons que l'équation
\begin{equation}
    x^n+a_{n-1}x^{n-1}+\ldots+a_1x+a_0=0
\end{equation}
est l'\defe{équation générale}{equation@équation!générale de degré $n$} de degré \( n\) si les coefficients \( a_i\) sont algébriquement indépendants sur \( \eK\).

\begin{theorem}
    Le groupe de Galois d'un polynôme de degré \( n\) est isomorphe au groupe symétrique \( S_n\).
\end{theorem}

\begin{corollary}
    L'équation générale de degré \( n\) est résoluble par radicaux si et seulement si \( n\geq 5\).
\end{corollary}
