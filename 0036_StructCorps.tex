% This is part of Mes notes de mathématique
% Copyright (c) 2011-2014
%   Laurent Claessens
% See the file fdl-1.3.txt for copying conditions.

%---------------------------------------------------------------------------------------------------------------------------
\subsection{Matrices}
%---------------------------------------------------------------------------------------------------------------------------

\begin{proposition}
    Nous avons
    \begin{equation}
        | \GL(n,\eF_p) |=(p^n-1)(p^n-p)\ldots (p^n-p^{n-1}).
    \end{equation}
\end{proposition}

\begin{proof}
    Par construction il existe une bijection entre \( \GL(n,\eF_p)\) et l'ensemble des bases de \( \eF_p^n\). Nous devons donc seulement compter le nombre de bases. Pour le premier vecteur de base nous avons le choix entre les \( p^n-1\) éléments non nuls de \( \eF_p^n\). Pour le second nous avons le choix entre \( p^n-p\) éléments, et ainsi de suite.
\end{proof}

%+++++++++++++++++++++++++++++++++++++++++++++++++++++++++++++++++++++++++++++++++++++++++++++++++++++++++++++++++++++++++++
\section{Minuscule morceau sur la théorie de Galois}
%+++++++++++++++++++++++++++++++++++++++++++++++++++++++++++++++++++++++++++++++++++++++++++++++++++++++++++++++++++++++++++

Vous trouverez des détails et des preuves à propos de la théorie de Galois dans \cite{GalIEl}.

\begin{definition}
    Soit $\eK$, un corps.
    
    Le \defe{groupe de Galois}{groupe!de Galois} d'une extension \( \eL\) de \( \eK\) est le groupe des automorphismes de \( \eL\) laissant \( \eK\) invariant. 

    Le groupe de Galois d'un polynôme sur \( \eK\) est le groupe de Galois de son corps de décomposition sur \( \eK\).
\end{definition}

\begin{definition}
    Des éléments \( b_1,\ldots, b_n\) d'une extension de \( \eK\) sont \defe{algébriquement indépendants}{algébriquement!indépendant}\index{indépendance!algébrique} si ils ne satisfont à aucune relation du type
    \begin{equation}
        \sum \alpha_{i_1\ldots i_n}b_1^{i_1}\ldots b_n^{i_n}=0
    \end{equation}
    avec \( \alpha_{i_1\ldots i_n}\in \eK\).
\end{definition}

Nous disons que l'équation
\begin{equation}
    x^n+a_{n-1}x^{n-1}+\ldots+a_1x+a_0=0
\end{equation}
est l'\defe{équation générale}{equation@équation!générale de degré $n$} de degré \( n\) si les coefficients \( a_i\) sont algébriquement indépendants sur \( \eK\).

\begin{theorem}
    Le groupe de Galois d'un polynôme de degré \( n\) est isomorphe au groupe symétrique \( S_n\).
\end{theorem}

\begin{corollary}
    L'équation générale de degré \( n\) est résoluble par radicaux si et seulement si \( n\geq 5\).
\end{corollary}

%+++++++++++++++++++++++++++++++++++++++++++++++++++++++++++++++++++++++++++++++++++++++++++++++++++++++++++++++++++++++++++
\section{Mini introduction aux nombres \texorpdfstring{p}{$p$}-adiques}
%+++++++++++++++++++++++++++++++++++++++++++++++++++++++++++++++++++++++++++++++++++++++++++++++++++++++++++++++++++++++++++


\subsection{La flèche d'Achille}\label{s:un}

C'est un grand classique que je donne ici juste comme introduction pour montrer que des série infinies peuvent donner des nombres finis de manière tout à fait intuitive.

Achille tire une flèche vers un arbre situé à $\unit{10}{\meter}$ de lui. Disons que la flèche avance à une vitesse constante de $\unit{1}{\meter\per\second}$. Il est clair que la flèche mettra $\unit{10}{\second}$ pour toucher l'arbre. En $\unit{5}{\second}$, elle aura parcouru la moitié de son chemin. On le note :
\[
\text{temps}=5s+\ldots
\]
Reste \( \unit{5}{\meter}\) à faire. En $\unit{2.5}{\second}$, elle aura fait la moitié de ce chemin chemin, soit $2.5m=\frac{10}{4}m$. On le note :
\[
\text{temps}=\frac{10}{2}s+\frac{10}{4}s+
\]
Reste $2.5m$ à faire. La moitié de ce trajet, soit $\frac{10}{8}m$, est parcouru en $\frac{10}{8}s$; on le note encore, mais c'est la dernière fois !

\[
\text{temps}=\frac{10}{2}s+\frac{10}{4}s+\frac{10}{8}s+
\]
En continuant ainsi à regarder la flèche qui parcours des demi-trajets puis des demi de demi-trajets et encore des demi de demi de demi-trajets, et en sachant que le temps total est $10s$, on trouve :
\[
10\left( \frac{1}{2}+\frac{1}{4}+\frac{1}{8}+\frac{1}{16}+\ldots  \right)=10.
\]
On doit donc croire que la somme jusqu'à l'infini des inverse des puissances de deux vaut $1$ :
\[
   \sum_{n=1}^{\infty}\frac{1}{2^n}=1.
\]
Cela peut être démontré à la loyale.

\subsection{La tortue et Achille}

Maintenant qu'on est convaincu que des sommes infinies peuvent représenter des nombres tout à fait normaux, passons à un truc plus marrant.

Achille, qui marche peinard à $\unit{10}{\meter\per\hour}$, part avec $1m$ d'avance sur une tortue qui avance à $\unit{1}{\meter\per\hour}$. Le temps que la tortue arrive au point de départ d'Achille, Achille aura parcouru $10m$, et le temps que la tortue mettra pour arriver à ce point, eh bien, Achille ne sera déjà plus là : il sera à $100m$. Si la tortue tient bon pendant un temps infini, et si l'on est confiant en le genre de raisonnements faits à la section \ref{s:un}, elle rattrapera Achille dans 
\[
1m+10m+100m+1000m+\ldots
\]
Autant dire que ça ne risque pas d'arriver. Et pourtant, mettons en équations : 
\begin{subequations}
    \begin{numcases}{}
        x_{\text{Achile}}(t)=1+10t\\
        x_{\text{tortue}}(t)=t.
    \end{numcases}
\end{subequations}
La tortue rejoints Achille au temps \( t\) tel que \( x_{\text{Achille}(t)}=x_{\text{tortue}}(t)\). Un mini calcul donne $t=-1/9$. Physiquement, c'est une situation logique. Peut-on en déduire une égalité mathématique du style de 
\[
1+10+100+1000+\ldots=-\frac{1}{9}\; ???
\]
Là où les choses deviennent jolies, c'est quand on cherche à voir ce que peut bien être la valeur d'un hypothétique $x=1+10+100+1000+\ldots$. En effet, logiquement on devrait avoir
\begin{equation*}
\begin{split}
\frac{x}{10}&=\frac{1}{10}+1+10+100+\ldots\\
            &=\frac{1}{10}+x.
\end{split}
\end{equation*}
Reste à résoudre l'équation du premier degré : $\frac{x}{10}=x+\frac{1}{10}$. Ai-je besoin de donner la solution ?

%---------------------------------------------------------------------------------------------------------------------------
\subsection{Dans les nombres \texorpdfstring{p}{$ p$}-adiques, c'est vrai}
%---------------------------------------------------------------------------------------------------------------------------

Nous nous proposons d'apprendre sur les nombres \( p\)-adiques juste ce qu'il faut pour montrer que l'égalité
\begin{equation}
    \sum_{k=0}^{\infty}10^k=-\frac{1}{ 9 }
\end{equation}
est vraie dans les nombres \( 5\)-adiques. Tout ce qu'il faut est sur \wikipedia{fr}{Nombre_p-adique}{wikipedia}.

Soit \( a\in \eN\) et \( p\), un nombre premier. La \defe{valuation}{valuation!$p$-adique} \( p\)-adique de \( a\) est l'exposant de \( p\) dans la décomposition de \( a\) en nombres premiers. On la note \( v_p(a)\). Pour un rationnel on définit
\begin{equation}
    v_p\left( \frac{ a }{ b } \right)=v_p(a)-v_p(b)
\end{equation}
La \defe{valeur absolue}{valeur absolue!$p$-adique} \( p\)-adique de \( r\in \eQ\) est 
\begin{equation}
    | r |_p=p^{-v_p(r)}.
\end{equation}
Nous posons \( | 0 |_p=0\). De là nous considérons la distance
\begin{equation}
    d_p(x,y)=| x-y |_p.
\end{equation}

\begin{lemma}
    L'espace \( (\eQ,d_p)\) est un espace métrique.
\end{lemma}

Nous considérons maintenant \( p=5\). Étant donné que \( a=5\cdot 2\) nous avons \( v_5(10)=1\) et
\begin{equation}
    v_5\left( \frac{1}{ 9 } \right)=v_5(1)-v_5(9)=0.
\end{equation}
Nous avons
\begin{equation}
    \sum_{k=0}^N10^k+\frac{1}{ 9 }=\frac{ 10^{N+1} }{ 9 }
\end{equation}
mais
\begin{equation}
    v_p\left( \frac{ 10^{N+1} }{ 9 } \right)=v_5(10^{N+1})-v_5(9)=N+1.
\end{equation}
Par conséquent
\begin{equation}
    d_5\big( \sum_{k=0}^N10^k,-\frac{1}{ 9 } \big)=| \frac{ 10^{N+1} }{ 9 } |_p=p^{-(N+1)}.
\end{equation}
En passant à la limite,
\begin{equation}
    \lim_{N\to \infty} d_5\big( \sum_{k=0}^N10^k,-\frac{1}{ 9 } \big)=0,
\end{equation}
ce qui signifie que
\begin{equation}
    \sum_{k=0}^{\infty}10^k=-\frac{1}{ 9 }.
\end{equation}
