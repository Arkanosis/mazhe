% This is part of Mes notes de mathématique
% Copyright (c) 2011-2013
%   Laurent Claessens
% See the file fdl-1.3.txt for copying conditions.

%---------------------------------------------------------------------------------------------------------------------------
\subsection{Matrices}
%---------------------------------------------------------------------------------------------------------------------------

\begin{proposition}
    Nous avons
    \begin{equation}
        | \GL(n,\eF_p) |=(p^n-1)(p^n-p)\ldots (p^n-p^{n-1}).
    \end{equation}
\end{proposition}

\begin{proof}
    Par construction il existe une bijection entre \( \GL(n,\eF_p)\) et l'ensemble des bases de \( \eF_p^n\). Nous devons donc seulement compter le nombre de bases. Pour le premier vecteur de base nous avons le choix entre les \( p^n-1\) éléments non nuls de \( \eF_p^n\). Pour le second nous avons le choix entre \( p^n-p\) éléments, et ainsi de suite.
\end{proof}

%+++++++++++++++++++++++++++++++++++++++++++++++++++++++++++++++++++++++++++++++++++++++++++++++++++++++++++++++++++++++++++
\section{Extensions de corps}
%+++++++++++++++++++++++++++++++++++++++++++++++++++++++++++++++++++++++++++++++++++++++++++++++++++++++++++++++++++++++++++

%---------------------------------------------------------------------------------------------------------------------------
\subsection{Clôture algébrique}
%---------------------------------------------------------------------------------------------------------------------------

\begin{theorem}
    Tout corps \( \eK\) possède une clôture algébrique \( \Omega\). De plus si \( \eL\) est une extension de \( \eK\), alors \( \eL\) est \( \eK\)-isomorphe à un sous corps de \( \Omega\).
\end{theorem}
Les deux parties de ce théorème utilisent l'axiome du choix.

Notons en particulier que si \( \Omega'\) est une autre clôture algébrique de \( \eK\), alors \( \Omega\) et \( \Omega'\) sont des sous corps l'un de l'autre et sont donc \( \eK\)-isomorphes.

\begin{lemma}
    Les polynômes \( P,Q\in \eK[X]\) ne sont pas premiers entre eux si et seulement si ils ont une racine commune dans la clôture algébrique \( \Omega\) de \( \eK\).
\end{lemma}

\begin{proof}
    Soit \( A\) un polynôme non inversible divisant \( P\) et $Q$. Par définition de \( \Omega\), ce polynôme \( A\) a une racine dans \( \Omega\) qui est alors une racine commune à \( P\) et \( Q\) dans \( \Omega\).

    Pour le sens inverse, si \( \alpha\) est une racine commune de \( P\) et \( Q\), alors le polynôme \( X-\alpha\) divise \( P\) et \( Q\) et donc \( P\) et \( Q \) ne sont pas premiers entre eux.
\end{proof}


%---------------------------------------------------------------------------------------------------------------------------
\subsection{Extensions séparables}
%---------------------------------------------------------------------------------------------------------------------------

Source : \cite{vgQYwF}.

Notons que dans ce qui va suivre nous allons parler de \( \eK[X]\), l'ensemble des polynômes sur un corps. Cela ne s'applique donc pas à \( \eZ[X]\) par exemple.

Une des choses intéressantes avec les extensions séparables c'est qu'elles vérifient le théorème de l'élément primitif (\ref{ThoORxgBC}).

\begin{definition}
    Soit \( \eK\) un corps. Un polynôme \emph{irréductible} \( P\in \eK[X]\) est \defe{séparable}{séparable!polynôme irréductible}\index{polynôme!irréductible!séparable} sur $\eK$ si dans un corps de décomposition, ses racines sont distinctes.

    Si \( P\) est un polynôme non constant dont la décomposition en irréductibles est \( P=P_1\ldots P_r\), nous disons qu'il est \defe{séparable}{séparable!polynôme non constant}\index{polynôme!séparable} si tous les \( P_i\) le sont.
\end{definition}

La proposition suivante donne un sens à la définition de polynôme irréductible séparable.
\begin{proposition}
    Soit \( P\) irréductible dans \( \eK[X]\) ayant des racines distinctes dans le corps de décomposition \( \eL\). Si \( \eL'\) est un autre corps de décomposition pour \( P\), alors \( P\) a aussi ses racines distinctes dans \( \eL\).
\end{proposition}

\begin{proof}
    L'ingrédient est la proposition \ref{PropTMkfyM} qui donne l'unicité du corps de décomposition à \( \eK\)-isomorphisme près. Soit donc \( \psi\colon \eL\to \eL'\) un isomorphisme laissant invariant les éléments de \( \eK\). D'une part, étant donné que \( P\) est à coefficients dans \( \eK\), nous avons \( \psi(P)=P\). D'autre part dans \( \eL\) le polynôme \( P\) s'écrit
    \begin{equation}
        P=a(X-\alpha_1)\ldots (X-\alpha_n)
    \end{equation}
    avec \( a\in \eK\) et \( \alpha_i\in \eL\). Nous avons donc
    \begin{equation}
        P=\psi(P)=a(X-\psi(\alpha_1))\ldots (X-\psi(\alpha_n)).
    \end{equation}
    Donc les racines de \( P\) dans \( \eL'\) sont les éléments \( \psi(\alpha_i)\) qui sont distincts.
\end{proof}

\begin{example}
    Un polynôme peut être séparable sur un corps, mais non séparable sur un autre. Soit \( \eL=\eF_p(T)\) et \( \eK=\eF_p(T^p)\). Nous considérons le polynôme
    \begin{equation}
        P=X^p-T^p
    \end{equation}
    dans \( \eK[X]\). Par le morphisme de Frobenius nous avons 
    \begin{equation}
        P=(X-T)^p
    \end{equation}
    dans \( \eL[X]\). Le polynôme \( P\) est irréductible sur \( \eK[X]\) parce que ses diviseurs sont de la forme \( (X-T)^k\) qui contiennent \( T^k\) qui n'est pas dans \( \eK\) (sauf si \( k=n\) ou \( k=0\)).

    Ce polynôme n'est pas séparable sur \( \eK\) parce que dans le corps de décomposition \( \eL\), la racine \( T\) est multiple. Notons bien le raisonnement : \( P\) étant irréductible, pour savoir si il est séparable, on le regarde dans un corps de décomposition.

    Par contre si nous regardons \( P\) dans \( \eL[X]\) alors \( P\) n'est plus irréductible parce que ses facteurs irréductibles sont \( (X-T)\). N'étant pas irréductible, nous regardons les racines de \emph{ses facteurs irréductibles}. Or chacun des facteurs irréductibles étant \( X-T\), les racines sont simples.
\end{example}

\begin{example}
    Le polynôme \( (X-1)^3\) est séparable sur \( \eQ\) parce que ses facteurs irréductibles dans \( \eQ[X]\) sont \( X-1\) qui ont des racines simples.
\end{example}

\begin{example}
    Le polynôme \( (X^2+1)^2\) est séparable dans \( \eQ[X]\). En effet, il a pour facteurs irréductible le polynôme \( X^2+1\) dont les racines sont \( \pm i\) dans l'extension \( \eQ(i)\).
\end{example}

\begin{proposition}[\cite{vgQYwF}]  \label{PropolyeZff}
    Soit \( P\in \eK[X]\) un polynôme non constant. Les trois propriétés suivantes sont équivalentes.
    \begin{enumerate}
        \item\label{ItemdqPFUi}
            \( P\) a une racine multiple dans une extension de \( \eK\). C'est à dire qu'il existe une extension de \( \eK\) dans laquelle \( P\) a une racine multiple.
        \item\label{ItemdqPFUib}
            \( P\) a une racine multiple dans tout corps de décomposition .
        \item\label{ItemdqPFUii}
            \( P\) et \( P'\) ont une racine commune dans une extension de \( \eK\).
        \item\label{ItemdqPFUiii}
            le degré de \( \pgcd(P,P')\) est \( \geq 1\).
    \end{enumerate}
\end{proposition}
\index{corps!extension}

\begin{proof}
    \begin{subproof}
    \item[\ref{ItemdqPFUi}\( \Rightarrow\)\ref{ItemdqPFUib}] Soit \( a\), une racine multiple de \( P\) dans une extension \( \eL\) de \( \eK\), et \( \eE\), un corps de décomposition de \( P\). Alors nous voulons prouver que \( P\) ait une racine multiple dans \( \eE\).

        Nous pouvons voir \( P\in \eL[X]\), et construire une corps de décomposition \( \eE'\) qui est une extension de \( \eL\). Vu que \( \eE\) et \( \eE'\) sont deux corps de décomposition de \( P\) 
        % iDIUoR
        nous avons un isomorphisme \( \psi\colon \eE\to \eE'\). Si \( a\in \eE\) est une racine multiple de \( P\), alors \( \psi(a)\) est une racine multiple de \( P\) dans \( \eE'\) parce que
        \begin{equation}
            P\big( \psi(a) \big)=\psi\big( P(a) \big).
        \end{equation}
    \item[\ref{ItemdqPFUi}\( \Rightarrow\)\ref{ItemdqPFUii}] Soit \( \eL\) un corps de décomposition de \( P\) sur \( \eK\) et \( a\in \eL\), une racine multiple de \( P\). On a alors \( P=(X-a)^2Q\) avec \( Q\in \eL[X]\). En dérivant,
        \begin{equation}
            P'=2(X-a)Q+(X-a)^2Q',
        \end{equation}
        et donc \( a\) est également une racine de \( P'\).
    \item[\ref{ItemdqPFUii}\( \Rightarrow\)\ref{ItemdqPFUiii}] Soit \( D\) un \( \pgcd\) de \( P\) et \( P'\). D'après le théorème de Bézout il existe \( A,B\in \eK[X]\) tels que 
        \begin{equation}
            AP+BP'=D.
        \end{equation}
        Si \( a\) est une racine commune de \( P\) et \( P'\) dans une extension \( \eL\), alors c'est aussi une racine de \( D\) et donc \( \deg(D)\geq 1\).
    \item[\ref{ItemdqPFUiii}\(\Rightarrow\)\ref{ItemdqPFUi}] Si le degré de \( D\) est plus grand ou égal à \( 1\), alors nous considérons une racine \( a\) de \( D\) dans \( \eL\) (une extension de \( \eK\)). Étant donné que \( D\) divise \( P\) et \( P'\), l'élément \( a\) est une racine commune de \( P\) et \( P'\). Nous montrons maintenant que \( a\) est alors une racine multiple de \( P\). Vu que \( P(a)=0\) nous avons
        \begin{equation}
            P=(X-a)Q,
        \end{equation}
        et \( P'=Q+(X-a)Q'\). Mais alors \( P'(a)=Q(a)\) et donc \( Q(a)=0\) et donc \( a\) est une racine double de \( P\). Par conséquent \( a\) est une racine multiple de \( P\) dans \( \eK\).
    \end{subproof}
\end{proof}
Notons que si \( P\) est irréductible, cette proposition donne des conditions pour que \( P\) ne soit pas séparable.

\begin{proposition}
    Soit \( P\in \eK[X]\) irréductible. Le polynôme \( P\) est séparable si et seulement si \( P'\neq 0\).
\end{proposition}

\begin{proof}
    Soit \( D=\pgcd(P,P')\) et nous voudrions prouver que \( \deg(D)\geq 1\) si et seulement si \( P'=0\). Si \( P'=0\), alors \( \pgcd(P,P')=P\) est donc \( \deg'(D)\geq 1\).

    Dans l'autre sens, si \( P\) est irréductible, il est associé à \( D\) parce qu'il n'a pas d'autres diviseurs que lui-même (à part \( 1\)). Nous avons donc \( P=\lambda D\) avec \( \lambda\in \eK\) et donc \( \deg(P)\geq 1\). Cela prouve immédiatement que \( P'\neq 0\).
\end{proof}

\begin{corollary}   \label{CorUjfJSE}
    Si \( \eK\) est de caractéristique nulle, alors tout polynôme de \( \eK[X]\) est séparable.
\end{corollary}

\begin{proof}
    Il suffit de montrer que les irréductibles sont séparables. Soit \( P\) un polynôme irréductible et unitaire de degré \( d\). Le terme de plus haut degré de \( P'\) est alors \( dX^{d-1}\) qui est non nul parce que \( d\neq 0\) en caractéristique nulle. Donc \( P'\neq 0\) et donc \( P\) est séparable par la proposition \ref{PropolyeZff}.
\end{proof}

\begin{definition}
    Soit \( \eL\) une extension algébrique de \( \eK\).
    \begin{enumerate}
        \item
            On dit que l'élément \( a\in \eL\) est \defe{séparable}{séparable!élément d'une extension} sur \( \eK\) si son polynôme minimal dans \( \eK[X]\) est séparable sur \( \eK\).
        \item
            L'extension \( \eL\) est \defe{séparable}{séparable!extension de corps} si tous ses éléments sont séparables.
    \end{enumerate}
\end{definition}

\begin{proposition} \label{PropUmxJVw}
    Soit \( \eK\) un corps. Les conditions suivantes sont équivalentes :
    \begin{enumerate}
        \item
            toutes les extensions algébriques de \( \eK\) sont séparables;
        \item
            tout polynôme irréductible de \( \eK[X]\) est séparable.
    \end{enumerate}
    En particulier les extensions algébriques des corps de caractéristique nulle sont toutes séparables.
\end{proposition}

\begin{proof}
    Soit \( P\) un polynôme irréductible dans \( \eK[X]\). Soient \( a\) et \( b\) des racines de \( P\) dans un corps de décomposition \( \eL\). Par hypothèse, \( \eL\) est séparables, donc les polynômes minimaux \( \mu_a\) et \( \mu_b\) sont séparables dans \( \eK[X]\). Étant donné que \( P\) est irréductible, il est le seul à se diviser lui-même et donc \( \mu_a=\mu_b=P\). Donc \( P\) est séparable sur \( \eK\).

    Dans l'autre sens, soit \( \eL\) une extension algébrique de \( \eK\), soit \( a\in \eL\) et le polynôme minimal \( \mu_a\in \eK[X]\). Par définition il est irréductible et donc séparable (par hypothèse). Donc \( a\) est séparable et \( \eL\) est une extension séparable.

    La dernière phrase est une conséquence du corollaire \ref{CorUjfJSE}.
\end{proof}

\begin{example} \label{ExvQTyBl}
    Une des conséquences les plus intéressantes de la proposition \ref{PropUmxJVw} est que toutes les extensions algébriques de \( \eQ\) sont séparables.
\end{example}

\begin{theorem}[Théorème de l'élément primitif\cite{rqrNyg}]\index{théorème!élément primitif}   \label{ThoORxgBC}
    Toute extension de corps séparable finie admet un élément primitif.

    Plus explicitement, soient \( \alpha_1,\ldots, \alpha_n\) des éléments algébriques séparables sur \( \eK\); alors \( \eL=\eK(\alpha_1,\ldots, \alpha_n)\) admet un élément primitif.
\end{theorem}

\begin{proof}
    Si le corps \( \eK\) est fini, alors \( \eL\) est également fini. Donc \( \eL^*\) est cyclique par le théorème \ref{ThobkwCMm}. Si \( \theta\) est un générateur de \( \eL^*\), alors \( \eL=\eK(\theta)\).

    Passons au cas où \( \eK\) est infini. Il suffit d'examiner le cas \( n=2\); en effet pour \( n=1\) c'est trivial et si \( n>2\), alors
    \begin{equation}
        \eK(\alpha_1,\ldots, \alpha_n)=\eK(\alpha_1,\ldots, \alpha_{n-1})(\alpha_n),
    \end{equation}
    et donc si \( \eK(\alpha_1,\ldots, \alpha_{n-1})=\eK(\theta)\), nous avons
    \begin{equation}
        \eK(\alpha_1,\ldots, \alpha_n)=\eK(\theta,\alpha_n)
    \end{equation}
    et nous sommes réduit au cas \( n=2\) par récurrence. 

    Soit donc \( \eL=\eK(\alpha,\beta)\); soit \( P\) le polynôme minimal de \( \alpha\) sur \( \eK\) et \( Q\) celui de \( \beta\). Nous nommons \( \eE\), un corps de décomposition de \( PQ\). Nous avons \( \eL\subset \eE\). Vu que \( P\) et \( Q\) sont polynômes minimaux d'éléments qui sont par hypothèse séparables, les polynômes \( P\) et \( Q\) sont séparables. Donc dans \( \eE\) les racines de \( P\) sont distinctes parce que \( P\) est irréductible (et idem pour \( Q\)). Soient les racines
    \begin{equation}
        \alpha_1=\alpha,\alpha_2,\ldots, \alpha_r
    \end{equation}
    de \( P\) dans \( \eE\) et les racines
    \begin{equation}
        \beta_1=\beta,\beta_2,\ldots, \beta_s
    \end{equation}
    de \( Q\) dans \( \eE\). Ici \( r\) et \( s\) sont les degrés de \( P\) et \( Q\).

    Si \( s=1\) alors \( Q=X-\beta\) et donc \( \beta\in \eK\) (parce que \( Q\in \eK[X]\)). Du coup nous avons \( \eL=\eK(\alpha)\) et le théorème est démontré. Nous supposons donc maintenant que \( s\geq 2\).

    Pour chaque \( (i,j)\in\llbracket 1,r\rrbracket\times \llbracket 2,s\rrbracket\), l'équation \( \alpha_i+x\beta_k=\alpha_1+x\beta_1\) pour \( x\in \eK\) a au plus\footnote{La solution \eqref{EqWzUFHe} peut être dans \( \eL\) et non dans \( \eK\). L'équation peut donc très bien ne pas avoir de solutions \( x\in \eK\).} une solution donnée le cas échéant par
    \begin{equation}    \label{EqWzUFHe}
        x=(\alpha_i-\alpha_1)(\beta_1-\beta_k)^{-1}
    \end{equation}
    Notons que cela est de toutes façons dans \( \eL\) et qu'étant donné que \( \beta_1\neq \beta_k\), cette solution a un sens (ici on utilise l'hypothèse de séparabilité). Étant donné que \( \eK\) est infini nous pouvons donc trouver un \( c\in \eK\) qui ne résout aucune des équations \eqref{EqWzUFHe} :
    \begin{equation}
        \alpha_i+c\beta_k\neq \alpha_1+c\beta_1.
    \end{equation}
    Nous posons \( \theta=\alpha+c\beta\) et nous prétendons que \( \eL=\eK(\theta)\). Montrons que \( \beta=\eK(\theta)\). Soit dans \( \eK(\theta)[T]\) les polynômes \( Q(T)\) et \( S(T)=P(\theta-cT)\). Nous nommons \( R\) le PGCD de ces deux polynômes.
    
    D'une part, une racine de \( R\) doit être une racine de \( Q\), et donc être un des \( \beta_i\). D'autre part, le choix de \( c\) fait que \( \beta\) est une racine de \( R\) parce que
    \begin{equation}
        S(\beta)=P(\alpha+c\beta-c\beta)=P(\alpha)=0.
    \end{equation}
    Enfin si \( k\geq 2\), alors
    \begin{equation}
        S(\beta_k)=P\big(\alpha_1+c(\beta-\beta_k)\big)\neq 0
    \end{equation}
    parce que \( \alpha_1+c(\beta+\beta_k)\) n'est aucun des \( \alpha_i\). Nous concluons que \( \beta\) est l'unique racine de \( R\) et donc que 
    \begin{equation}
        R=X-\beta\in \eK(\theta)[T],
    \end{equation}
    et donc \( \beta\in \eK(\theta)\).

    De plus \( \alpha=\theta-c\beta\) est alors immédiatement dans \( \eK(\theta)\). À partir du moment où \( \alpha\) et \( \beta\) sont dans \( \eK(\theta)\), nous avons obtenu \( \eL=\eK(\alpha,\beta)=\eK(\theta)\).

\end{proof}

\begin{example}
    Le théorème de l'élément primitif \ref{ThoORxgBC} ne tient pas pour les corps non commutatifs. Par exemple si nous considérons pour \( \eK\) le corps des quaternions\index{quaternion} et comme \( G\) le groupe à \( 8\) éléments \( \{ \pm 1,\pm i,\pm j,\pm k \}\). Ce dernier groupe n'est pas cyclique alors qu'il est un groupe fini dans \( \eK^*\).
\end{example}

\begin{example}
    Il est aussi possible pour un groupe fini d'avoir \( \omega(G)=| G |\) sans pour autant que \( G\) soit cyclique. Par exemple pour \( G=S_3\), nous avons \( | S_3 |=6\) alors que les éléments de \( S_3\) sont soit d'ordre \( 2\) soit d'ordre \( 3\) et \( \omega(G)=\ppcm(2,3)=6\). Pourtant \( S_3\) n'est pas cyclique.
\end{example}


\begin{remark}
    Il est prouvé dans \cite{rqrNyg} que \( \eC\) est algébriquement clos à partir du théorème de l'élément primitif \ref{ThoORxgBC}, mais c'est encore un petit peu de travail.
\end{remark}

%---------------------------------------------------------------------------------------------------------------------------
\subsection{Idéal maximum}
%---------------------------------------------------------------------------------------------------------------------------

\begin{definition}
    Un nombre (dans \( \eC\)) est \defe{transcendant}{transcendant} si il n'est racine d'aucun polynôme non nul à coefficients entiers. Plus généralement si \( \eL\) est un extension du corps \( \eK\) alors si \( t\in \eL\) est une racine d'un polynôme dans \( \eK[X]\) nous disons que \( t\) est \defe{algébrique}{algébrique!par rapport à une extension de corps} sur \( \eK\); sinon nous disons que \( t\) est \defe{transcendant}{transcendant!par rapport à une extension de corps} sur \( \eK\).
\end{definition}

\begin{definition}
    Une \( \eK\)-algèbre est de \defe{type fini}{type!fini!en algèbre} si elle est le quotient de \( \eK[X_1,\ldots, X_n]\) par un idéal (pour un certain \( n\)).
\end{definition}

\begin{theorem}[\wikipedia{fr}{Idéal_maximal}{wikipédia}]\index{idéal!maximum}       \label{ThorqTTiJ}
    un idéal \( I\) d'un anneau commutatif \( \eA\) est maximal si et seulement si le quotient \( \eA/I\) est un corps.
\end{theorem}
%TODO : faire la démonstration

\begin{theorem}[\cite{OorXst}]      \label{ThonoZyKa}
    Soit \( \eK\) un corps et \( B\), une \( \eK\)-algèbre de type fini. Si \( B\) est un corps, alors c'est une extension algébrique finie de \( \eK\).
\end{theorem}
%TODO : faire la démonstration

\begin{theorem}[\cite{OorXst}]  \label{ThowgZYqx}
    Si \( \eK\) est un corps algébriquement clos, les idéaux maximaux de \( \eK[X_1,\ldots, X_n]\) sont de la forme
    \begin{equation}
        (X_1-a_1,\ldots, X_n-a_n)
    \end{equation}
    où les \( a_i\) sont des éléments de \( \eK\).
\end{theorem}

\begin{proof}
    Nous commençons par montrer que
    \begin{equation}
        J=(X_1-a_1,\ldots, X_n-a_n)
    \end{equation}
    est un idéal maximum. Pour cela nous considérons le morphisme surjectif d'anneaux
    \begin{equation}
        \begin{aligned}
            \phi\colon \eK[X_1,\ldots, X_n]&\to \eK \\
            P&\mapsto P(a_1,\ldots, a_n). 
        \end{aligned}
    \end{equation}
    Soit \( P\in\ker(\phi)\); nous écrivons la division euclidienne de \( P\) par \( X-a_1\) puis celle du reste par \( X-a_2\) et ainsi de suite :
    \begin{equation}    \label{EqDAkijH}
        P=(X-a_1)Q_1+\ldots +(X_n-a_n)Q_n+R
    \end{equation}
    où \( R\) doit être une constante parce que le premier reste est de degré zéro en \( X_1\), le second est de degré zéro en \( X_1\) et \( X_2\), etc. Afin d'identifier cette constante, nous appliquons l'égalité \eqref{EqDAkijH} à \( (a_1,\ldots, a_n)\) et en nous rappelant que \( P\in \ker(\phi)\) nous obtenons
    \begin{equation}
        0=P(a_1,\ldots, a_n)=R,
    \end{equation}
    donc \( R=0\) et \( P=(X_1-a_1)Q_1+\ldots +(X_n-a_n)Q_n\), c'est à dire \( P\in J\). Nous avons donc \( \ker(\phi)\subset J\). Par ailleurs \( J\subset \ker(\phi)\) est évident, donc \( J=\ker(\phi)\).

    Vu que \( J\) est le noyau de l'application \( \eK[X_1,\ldots, X_n]\to \eK\), nous avons 
    \begin{equation}
        \frac{ \eK[X_1,\ldots, X_n] }{ J }=\eK.
    \end{equation}
    Donc \( J\) est un idéal maximal parce que tout polynôme n'étant pas dans \( J\) doit avoir un terme indépendant non nul et donc être dans \( \eK\) vis à vis du quotient \( \eK[X_1,\ldots, X_n]/J\).

    Nous montrons maintenant l'implication inverse. Nous supposons que \( I\) est un idéal maximum et nous montrons qu'il doit être égal à \( J\) (pour un certain choix de \( a_1,\ldots, a_n\)).

    Le quotient
    \begin{equation}
        \frac{ \eK[X_1,\ldots, X_n] }{ I }
    \end{equation}
    est une \( \eK\)-algèbre de type fini (définition). De plus c'est un corps par le théorème \ref{ThorqTTiJ}. C'est donc une extension algébrique finie de \( \eK\) par le théorème \ref{ThonoZyKa}. Mais \( \eK\) étant algébriquement clos, il est sa propre et unique extension algébrique; nous en déduisons que
    \begin{equation}
        \frac{ \eK[X_1,\ldots, X_n] }{ I }=\eK.
    \end{equation}
    Donc pour tout \( 1\leq i\leq n\), il existe \( a_i\in \eK\) tel que \( X_i-a_i\in I\), sinon le monôme \( X_i\) ne se projetterait pas sur un élément dans \( \eK\) dans le quotient. Cela prouve que \( J\) est contenu dans \( I\); par maximalité nous avons donc \( I=J\).
\end{proof}

\begin{corollary}
    Soit \( \eK\) un corps algébriquement clos et \( I\), un idéal de \( \eK[X_1,\ldots, X_n]\). Si nous notons
    \begin{equation}
        V(I)=\{ x\in \eK^n\tq P(x_1,\ldots, x_n)=0 \}
    \end{equation}
    l'ensemble des racines communes à tous les éléments de \( I\), on a \( V(I)=\emptyset\) si et seulement si \( I=\eK[X_1,\ldots, X_n]\).
\end{corollary}

\begin{proof}
    Si \( I=\eK[X_1,\ldots, X_n]\) en particulier \( 1\in I\) et nous avons évidemment \( V(I)=\emptyset\). Le sens difficile est l'autre sens.

    Supposons que \( I\neq \eK[X_1,\ldots, X_n]\) et que \( K\) est un idéal maximum contenu dans \( I\). Nous savons déjà par le théorème \ref{ThowgZYqx} que \( K\) est de la forme \( K=(X_1-a_1,\ldots, X_n-a_n)\). Un élément de \( I\) est dans \( K\), donc si \( P\in I\) nous avons
    \begin{equation}
        P(a_1,\ldots, a_n)=0,
    \end{equation}
    c'est à dire que \( (a_1,\ldots, a_n)\in V(I)\) et donc que \( V(I)\neq 0\).
\end{proof}

%+++++++++++++++++++++++++++++++++++++++++++++++++++++++++++++++++++++++++++++++++++++++++++++++++++++++++++++++++++++++++++ 
\section{Intégration de fractions rationnelles}
%+++++++++++++++++++++++++++++++++++++++++++++++++++++++++++++++++++++++++++++++++++++++++++++++++++++++++++++++++++++++++++

Mes sources pour parler d'intégration de fractions rationnelles : \cite{MKucxNb,CPheFRq,LTjwacY,KXjFWKA}.

\begin{theorem}[Rothstein-Trager]
    Soient \( P,Q\in \eQ[X]\) premiers entre eux avec \( \pgcd(P,Q)=1\) et \( \deg(P)<\deg(Q)\). Nous supposons que \( Q\) est unitaire et sans facteurs carrés. Supposons que nous puissions écrire, dans un extension \( \eK\) de \( \eQ\) la primitive de \( P/Q\) de la façon suivante :
    \begin{equation}        \label{EqCHVaDay}
        \int\frac{ P }{ Q }=\sum_{i=1}^n c_i\ln(P_i)
    \end{equation}
    où les \( c_i\) sont des constantes non nulles et deux à deux distinctes et où les \( P_i\) sont des polynômes unitaires non constants sans facteurs carrés et premiers deux à deux entre eux dans \( \eK[X]\).

    Alors les \( c_i\) sont les racines distinctes du polynôme
    \begin{equation}
        R(Y)=\res_X(P-YQ',Q)\in \eK[Y]
    \end{equation}
    et
    \begin{equation}
        P_i=\pgcd(P-c_iQ',Q).
    \end{equation}
\end{theorem}

\begin{proof}
    Nous posons 
    \begin{equation}
        U_i=\prod_{j\neq i}P_j.
    \end{equation}
    \begin{subproof}
    \item[Question de division]
    Ensuite nous dérivons formellement l'équation \eqref{EqCHVaDay} et nous multiplions les deux côtés du résultat par \( \prod_{j=1}^nP_j\) :
    \begin{equation}        \label{EqGSJKyDw}
        P\prod_{j=1}^nP_j=Q\sum_{i=1}^nc_i\frac{ P'i }{ P_i }\prod_{j=1}^nP_j=Q\sum_{i=1}^nc_iP'_iU_i.
    \end{equation}
    Une première chose que nous en tirons est que \( Q\) divise le produit \( P\prod_{j=1}^nP_j\); mais \( P\) et \( Q\) étant premiers entre eux, 
    \begin{equation}
        Q\divides \prod_{j=1}^nP_j
    \end{equation}
    par le théorème de Gauss \ref{ThoLLgIsig}.

    Une seconde chose que nous tirons de \eqref{EqGSJKyDw} est que \( P_j\) divise \( Q\sum_{i=1}^nc_iP'_iU_i\). De cette somme, à cause du \( U_i\) qui est divisé par \( P_j\) pour tout \( i\) sauf \( i=j\), le polynôme \( P_j\) divise tous les termes sauf peut-être un. Donc il les divise tous et en particulier
    \begin{equation}
        P_j\divides Qc_jP'_JU_j
    \end{equation}
    En nous souvenant que les \( P_k\) sont premiers entre eux, \( P_j\) ne divise pas \( U_j\). De plus \( P_j\) étant sans facteurs carrés, les polynômes \( P_j\) et \( P'_j\) sont premiers entre eux. Il ne reste que \( Q\). Nous en déduisons que
    \begin{equation}
        P_j\divides Q
    \end{equation}
    pour tout \( 1\leq j\leq n\). Et vu que les \( P_i\) sont premiers entre eux, le fait que chacun divise \( Q\) implique que leur produit divise \( Q\), c'est à dire
    \begin{equation}
        \prod_{j=1}^nP_j\divides Q.
    \end{equation}
    Or nous avions déjà prouvé la division contraire. Du fait que les deux polynômes sont unitaires nous en déduisons qu'ils sont en réalité égaux :
    \begin{equation}        \label{EqJImORVe}
        Q=\prod_{j=1}^nP_j.
    \end{equation}
    Nous pouvons simplifier les deux membres de \eqref{EqGSJKyDw} par cela :
    \begin{equation}        \label{EqJMtGhGR}
        P=\sum_{i=1}^nc_iP'_iU_i.
    \end{equation}
    
\item[\( P_i\) divise \( P-c_iQ'\)]

    En dérivant \eqref{EqJImORVe} nous trouvons
    \begin{equation}
        Q'=\sum_{j=1}^nP'_jU_j,
    \end{equation}
    et en écrivant \( P\) sous sa forme \eqref{EqJMtGhGR},
    \begin{equation}
        P-c_iQ'=\sum_{j=1}^nc_jP'_jU_j-\sum_{j=1}^nc_iP'_jU_j=\sum_{j=1}^n(c_j-c_i)P'_jU_j.
    \end{equation}
    Le terme \( i=j\) de la somme est nul; en ce qui concerne les autres termes, ils sont divisés par \( P_i\) parce que \( P_i\divides U_j\). Donc \( P_i\) divise tous les termes de la somme et nous avons
    \begin{equation}
        P_i\divides P-c_iQ'.
    \end{equation}
    <++>

        
    \end{subproof}
    <++>
\end{proof}
<++>

%+++++++++++++++++++++++++++++++++++++++++++++++++++++++++++++++++++++++++++++++++++++++++++++++++++++++++++++++++++++++++++
\section{Minuscule morceau sur la théorie de Galois}
%+++++++++++++++++++++++++++++++++++++++++++++++++++++++++++++++++++++++++++++++++++++++++++++++++++++++++++++++++++++++++++

Vous trouverez des détails et des preuves dans \cite{GalIEl}.

\begin{definition}
    Soit $\eK$, un corps.
    
    Le \defe{groupe de Galois}{groupe!de Galois} d'une extension \( \eL\) de \( \eK\) est le groupe des automorphismes de \( \eL\) laissant \( \eK\) invariant. 

    Le groupe de Galois d'un polynôme sur \( \eK\) est le groupe de Galois de son corps de décomposition sur \( \eK\).
\end{definition}

\begin{definition}
    Des éléments \( b_1,\ldots, b_n\) d'une extension de \( \eK\) sont \defe{algébriquement indépendants}{algébriquement!indépendant}\index{indépendance!algébrique} si ils ne satisfont à aucune relation du type
    \begin{equation}
        \sum \alpha_{i_1\ldots i_n}b_1^{i_1}\ldots b_n^{i_n}=0
    \end{equation}
    avec \( \alpha_{i_1\ldots i_n}\in \eK\).
\end{definition}

Nous disons que l'équation
\begin{equation}
    x^n+a_{n-1}x^{n-1}+\ldots+a_1x+a_0=0
\end{equation}
est l'\defe{équation générale}{équation!générale de degré $n$} de degré \( n\) si les coefficients \( a_i\) sont algébriquement indépendants sur \( \eK\).

\begin{theorem}
    Le groupe de Galois d'un polynôme de degré \( n\) est isomorphe au groupe symétrique \( S_n\).
\end{theorem}

\begin{corollary}
    L'équation générale de degré \( n\) est résoluble par radicaux si et seulement si \( n\geq 5\).
\end{corollary}

%+++++++++++++++++++++++++++++++++++++++++++++++++++++++++++++++++++++++++++++++++++++++++++++++++++++++++++++++++++++++++++
\section{Mini introduction aux nombres \texorpdfstring{p}{$p$}-adiques}
%+++++++++++++++++++++++++++++++++++++++++++++++++++++++++++++++++++++++++++++++++++++++++++++++++++++++++++++++++++++++++++


\subsection{La flèche d'Achille}\label{s:un}

C'est un grand classique que je donne ici juste comme introduction pour montrer que des série infinies peuvent donner des nombres finis de manière tout à fait intuitive.

Achille tire une flèche vers un arbre situé à $\unit{10}{\meter}$ de lui. Disons que la flèche avance à une vitesse constante de $\unit{1}{\meter\per\second}$. Il est clair que la flèche mettra $\unit{10}{\second}$ pour toucher l'arbre. En $\unit{5}{\second}$, elle aura parcouru la moitié de son chemin. On le note :
\[
\text{temps}=5s+\ldots
\]
Reste \( \unit{5}{\meter}\) à faire. En $\unit{2.5}{\second}$, elle aura fait la moitié de ce chemin chemin, soit $2.5m=\frac{10}{4}m$. On le note :
\[
\text{temps}=\frac{10}{2}s+\frac{10}{4}s+
\]
Reste $2.5m$ à faire. La moitié de ce trajet, soit $\frac{10}{8}m$, est parcouru en $\frac{10}{8}s$; on le note encore, mais c'est la dernière fois !

\[
\text{temps}=\frac{10}{2}s+\frac{10}{4}s+\frac{10}{8}s+
\]
En continuant ainsi à regarder la flèche qui parcours des demi-trajets puis des demi de demi-trajets et encore des demi de demi de demi-trajets, et en sachant que le temps total est $10s$, on trouve :
\[
10\left( \frac{1}{2}+\frac{1}{4}+\frac{1}{8}+\frac{1}{16}+\ldots  \right)=10.
\]
On doit donc croire que la somme jusqu'à l'infini des inverse des puissances de deux vaut $1$ :
\[
   \sum_{n=1}^{\infty}\frac{1}{2^n}=1.
\]
Cela peut être démontré à la loyale.

\subsection{La tortue et Achille}

Maintenant qu'on est convaincu que des sommes infinies peuvent représenter des nombres tout à fait normaux, passons à un truc plus marrant.

Achille, qui marche peinard à $\unit{10}{\meter\per\hour}$, part avec $1m$ d'avance sur une tortue qui avance à $\unit{1}{\meter\per\hour}$. Le temps que la tortue arrive au point de départ d'Achille, Achille aura parcouru $10m$, et le temps que la tortue mettra pour arriver à ce point, eh bien, Achille ne sera déjà plus là : il sera à $100m$. Si la tortue tient bon pendant un temps infini, et si l'on est confiant en le genre de raisonnements faits à la section \ref{s:un}, elle rattrapera Achille dans 
\[
1m+10m+100m+1000m+\ldots
\]
Autant dire que ça ne risque pas d'arriver. Et pourtant, mettons en équations : 
\begin{subequations}
    \begin{numcases}{}
        x_{\text{Achile}}(t)=1+10t\\
        x_{\text{tortue}}(t)=t.
    \end{numcases}
\end{subequations}
La tortue rejoints Achille au temps \( t\) tel que \( x_{\text{Achille}(t)}=x_{\text{tortue}}(t)\). Un mini calcul donne $t=-1/9$. Physiquement, c'est une situation logique. Peut-on en déduire une égalité mathématique du style de 
\[
1+10+100+1000+\ldots=-\frac{1}{9}\; ???
\]
Là où les choses deviennent jolies, c'est quand on cherche à voir ce que peut bien être la valeur d'un hypothétique $x=1+10+100+1000+\ldots$. En effet, logiquement on devrait avoir
\begin{equation*}
\begin{split}
\frac{x}{10}&=\frac{1}{10}+1+10+100+\ldots\\
            &=\frac{1}{10}+x.
\end{split}
\end{equation*}
Reste à résoudre l'équation du premier degré : $\frac{x}{10}=x+\frac{1}{10}$. Ai-je besoin de donner la solution ?

%---------------------------------------------------------------------------------------------------------------------------
\subsection{Dans les nombres \texorpdfstring{p}{$ p$}-adiques, c'est vrai}
%---------------------------------------------------------------------------------------------------------------------------

Nous nous proposons d'apprendre sur les nombres \( p\)-adiques juste ce qu'il faut pour montrer que l'égalité
\begin{equation}
    \sum_{k=0}^{\infty}10^k=-\frac{1}{ 9 }
\end{equation}
est vraie dans les nombres \( 5\)-adiques. Tout ce qu'il faut est sur \wikipedia{fr}{Nombre_p-adique}{wikipedia}.

Soit \( a\in \eN\) et \( p\), un nombre premier. La \defe{valuation}{valuation!$p$-adique} \( p\)-adique de \( a\) est l'exposant de \( p\) dans la décomposition de \( a\) en nombres premiers. On la note \( v_p(a)\). Pour un rationnel on définit
\begin{equation}
    v_p\left( \frac{ a }{ b } \right)=v_p(a)-v_p(b)
\end{equation}
La \defe{valeur absolue}{valeur absolue!$p$-adique} \( p\)-adique de \( r\in \eQ\) est 
\begin{equation}
    | r |_p=p^{-v_p(r)}.
\end{equation}
Nous posons \( | 0 |_p=0\). De là nous considérons la distance
\begin{equation}
    d_p(x,y)=| x-y |_p.
\end{equation}

\begin{lemma}
    L'espace \( (\eQ,d_p)\) est un espace métrique.
\end{lemma}

Nous considérons maintenant \( p=5\). Étant donné que \( a=5\cdot 2\) nous avons \( v_5(10)=1\) et
\begin{equation}
    v_5\left( \frac{1}{ 9 } \right)=v_5(1)-v_5(9)=0.
\end{equation}
Nous avons
\begin{equation}
    \sum_{k=0}^N10^k+\frac{1}{ 9 }=\frac{ 10^{N-1} }{ 9 }
\end{equation}
mais
\begin{equation}
    v_p\left( \frac{ 10^{N-1} }{ 9 } \right)=v_5(10^{N-1})-v_5(9)=N-1.
\end{equation}
Par conséquent
\begin{equation}
    d_5\big( \sum_{k=0}^N,-\frac{1}{ 9 } \big)=| \frac{ 10^{N-1} }{ 9 } |_p=p^{-(N-1)}.
\end{equation}
En passant à la limite,
\begin{equation}
    \lim_{N\to \infty} d_5\big( \sum_{k=0}^N,-\frac{1}{ 9 } \big)=0,
\end{equation}
ce qui signifie que
\begin{equation}
    \sum_{k=0}^{\infty}10^k=-\frac{1}{ 9 }.
\end{equation}
