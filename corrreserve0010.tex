% This is part of Mes notes de mathématique
% Copyright (c) 2012
%   Laurent Claessens
% See the file fdl-1.3.txt for copying conditions.


\begin{corrige}{reserve0010}

Un simple calcul montre que
\begin{equation}
    \nabla\times\begin{pmatrix}
        y\cos(xy)+1    \\ 
        x\cos(xy)    \\ 
        0    
    \end{pmatrix}
\end{equation}
est nul. Cela montre que le champ de vecteur donné est bien conservatif (c'est à dire que c'est un champ de gradient).

Afin de trouver la fonction, nous devons résoudre le système
\begin{subequations}
    \begin{numcases}{}
        \frac{ \partial f }{ \partial x }=y\cos(xy)+1\\
        \frac{ \partial f }{ \partial y }=x\cos(xy).
    \end{numcases}
\end{subequations}
La première équation suggère d'écrire
\begin{equation}
    f(x,y)=\sin(xy)+x+c(y)
\end{equation}
où \( c(y)\) est une constante par rapport à \( x\), c'est à dire une fonction de \( y\). En remplaçant dans la seconde équation nous voyons que la fonction
\begin{equation}
    f(x,y)=\sin(xy)+x
\end{equation}
fait l'affaire.

\end{corrige}
