\begin{exercice}\label{exoMatlab0031}

Un échantillon contenant trois substances radioactives voit son activité décroître suivant une loi du type
\[ f(t) = ae^{-\lambda_1t} + be^{-\lambda_2t} + ce^{-\lambda_3t} . \]
On suppose que $\lambda_1 = 1.23$, $\lambda_2 = 0.26$, $\lambda_3 = 0.1$, le temps $t$ étant exprimé en jours.
\begin{enumerate}
\item En résolvant un système linéaire de trois équations à trois inconnues, calculez $a$, $b$ et $c$ sachant que
\[ f(0) = 100, f(2) = 62 , f(6) = 32 . \]
\item Avec les valeurs de $a$, $b$, $c$ obtenues ci-dessus, définissez la fonction $f$ dans un fichier et utilisez cette définition pour représenter le graphe de $f$ sur l'intervalle $[0,8]$.
\item Trouvez $\tau\in[0,6]$ tel que $f(\tau) = 50$.
\end{enumerate}

\corrref{Matlab0031}
\end{exercice}
