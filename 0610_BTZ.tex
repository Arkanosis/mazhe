% This is part of (almost) Everything I know in mathematics and physics
% Copyright (c) 2013-2014
%   Laurent Claessens
% See the file fdl-1.3.txt for copying conditions.

\begin{abstract}
This chapter deals with black holes in anti de Sitter spaces. The latter are the simplest non flat solutions to Einstein's equations with constant negative cosmological constant; they are in particular pseudo-Riemannian manifolds that carry a causal structure, physically due to the finiteness of speed of light. That physical restriction is mathematically encoded by the existence of three types of geodesics: the space-, time- and light-like ones, existence which is in turn implied by the non positivity of the metric. A causal structure is introduced by defining two points as \emph{causally connected} when there exists a time- or light-like path connecting them.

 The  originality of our approach is that the $l$-dimensional space $AdS_l$ is seen as a quotient of groups $\SO(2,l-1)/\SO(1,l-1)=G/H$, and that the special causal black hole structure is described in terms of orbits of the action of a subgroup of the isometry group of the manifold.

Using symmetric spaces techniques, we show that closed orbits of the Iwasawa subgroup of $\SO(2,l-1)$ naturally define a causal black hole singularity in anti de Sitter spaces in $l \geq 3$ dimensions. In particular, we recover for $l=3$ the non-rotating massive BTZ black hole. The method presented here is very simple and in principle generalizable to any semisimple symmetric space.

The main references for this part are \cite{lcTNAdS,articleBVCS,These}.

\end{abstract}

%%%%%%%%%%%%%%%%%%%%%%%%%%
 %
    \section{Introduction}
%
%%%%%%%%%%%%%%%%%%%%%%%%

\subsection{Physics and mathematics of black holes}	\label{SubSecGeneBH}
%--------------------------------------------------

\subsubsection{Notion of Causality}
%\\\///////////////////////////////

This subsection is devoted to introduce the mathematical definition of a black hole from the intuitive physical notions of causality and maximality of the speed of light. Let us pose the origin of time and space respectively now and here. So we are at $(0,0)$. If we denote by $c$ the seed of light, we cannot reach the moon before time $\unit{340000}{\kilo\meter}/c$. More generally we cannot reach a point at spacial distance $d$ within a time inferior to $d/c$. Then the space is thus divided into three very different regions with respect to causality: the points that we can reach traveling slower than light, the points that only light can reach and points that we cannot reach at all.
%see figure \ref{FigMink}.
%\begin{figure}[ht]
%\begin{center}
%\begin{pspicture}(-3.3,-1.2)(3.3,3.2)
%   \psset{PointSymbol=none, PointName=none}
%   \pstGeonode(0,0){O}(1,0){X}(0,1){Y}(-2.9,-0.9){Bg}(2.9,2.9){Hd}(2.0,0.8){P}
%   \pstProjectionOrth{O}{X}{Hd}{Ax}
%   \pstProjectionOrth{O}{Y}{Hd}{At}
%   \pstProjectionOrth{O}{X}{Bg}{Bx}
%   \pstProjectionOrth{O}{Y}{Bg}{Bt}
%   \pstHomO[HomCoef=0.9]{Ax}{Hd}[Bld]
%   \pstOrtSym{O}{Y}{Bld}[Blg]
%	\pspolygon[fillstyle=vlines,hatchcolor=red,linecolor=white](O)(Ax)(Bld)
%	\pspolygon[fillstyle=vlines,hatchcolor=red,linecolor=white](O)(Bx)(Blg)
%	\psline{->}(Bt)(At)
%	\psline{->}(Bx)(Ax)
%	\psline[linecolor=yellow]{->}(O)(Bld)
%	\psline[linecolor=yellow]{->}(O)(Blg)
%	\pstMarquePoint{At}{0.3;0}{$t$}
%	\pstMarquePoint{Ax}{0.3;0}{$x$}
%	\pstMarquePoint{Bld}{0.3;90}{$s=0$}
%	\psellipse[fillstyle=solid,fillcolor=white,linecolor=white](P)(0.7,0.4)
%	\pstMarquePoint{P}{0.1;35}{$s<0$}
%	\psellipse[fillstyle=solid,fillcolor=white,linecolor=white](Y)(0.7,0.4)
%	\pstMarquePoint{Y}{0.1;35}{$s>0$}
%\end{pspicture}
%\end{center}
%\caption{Yellow lines correspond to the travel of a light ray, the red zone is unreachable by an observer located at $(0,0)$.}\label{FigMink}
%\end{figure}

%TODO : refaire cette figure.

It is convenient to introduce the function $s(t,x)=c^2t^2-x^2$ which basically says you which points are accepted and which points are unaccepted. The mathematical way to implement these ideas is to consider a pseudo-Riemannian manifold $(M,g)$. The \defe{norm}{pseudo-Riemannian!norm} of a vector $X\in T_xM$ is defined as $\| X \|^2=g_x(X,X)$. There are three possibilities:
\begin{itemize}\label{PgDefsGenre}
	\item if $\| X \|^2>0$, we say that $X$ is \defe{time-like}{time-like},
	\item if $\| X \|^2<0$, we say that $X$ is \defe{space-like}{space-like},
	\item if $\| X \|^2=0$, we say that $X$ is \defe{light-like}{light-like}.
\end{itemize}
A path $c\colon \eR\to M$ is time, space or light-like when its tangent vector is everywhere time, space or light-like. The manifold $M$ is \defe{time orientable}{time orientation} if it accepts an everywhere time-like vector field. A \emph{time orientation} is the choice of such a vector field. If $T$ is a time orientation on $M$, we say that a vector $X_x\in T_xM$ is \defe{future directed}{future!directed vector} if $g_x(T_x,X)>0$. From now we suppose that a choice of time orientation is possible and done.

The concept of causality is now easy to determine. If $x$ and $y$ belong to $M$, the point $x$ has a \defe{causal influence}{causal!influence} on $y$ if there exists a future directed path $c\colon [0,1]\to M$ such that $c(0)=x$ and $c(1)=y$. One has to notice that the relation \emph{has a causal influence on} is not symmetric in general, but there exist some examples in which it is symmetric.

%As example, consider the space $\eR^2$ represented on the figure \ref{FigMink}. 
As example consider the space $M=\eR^2$ endowed with the constant pseudo-Riemannian structure $g=\begin{pmatrix}c^2&0\\0&1 \end{pmatrix}$. That space is the \defe{Minkowski space}{Minkowski!space}. The relation of causality is given by the previously mentioned function $s$; this relation is \emph{never} symmetric and there exist pairs of point who have no causal effect on each other. If one takes the quotient by the relation $t\sim t+1$, we get a space in which the causality is everywhere symmetric.

\subsubsection{Notion of singularity and black hole}
%////////////////////////////////////////////////

Up to the choice of a time orientation, a pseudo-Riemannian manifold is comes with a canonical notion of causality. In order to have a black hole in our causal space we need an extra structure:~the singularity. We take here a very conservative point of view and we say that a \emph{singularity} in $M$ is any strict subset of $M$. In the literature one often add conditions on the singularity such like to be a submanifold, time-like, connected,\ldots of course most of ``real live'' singularities fulfil that kind of conditions.

The singularity defines two types of points in the space: the ones from which every time-like path intersect the singularity (from a physical point of view, these points correspond to observers who will fall in the singularity without doubt) and the points from which at least one time-like path does not intersect the singularity. We define the black hole associated with the singularity $\hS$ as
\begin{align} 
  BH=\big\{ x\in M\tq \forall &\text{ future directed time-like path } c \text{ with } c(0)=x,\\
			&\exists t\geq0  \text{ such that } c(t)\in \hS \big\}.
\end{align}
The easiest example is given by defining a small line as singular in the Minkowski space as shown in figure \ref{LabelFigEJRsWXw}. 
\newcommand{\CaptionFigEJRsWXw}{The red line is the singularity and the green zone is the black hole associated with.}
\input{Fig_EJRsWXw.pstricks}
%\begin{figure}[ht]
%\begin{center}
%\begin{pspicture}(-3.3,-1.2)(3.3,3.2)
%   \psset{PointSymbol=none, PointName=none}
%   \pstGeonode(-3.3,-1.2){cg}(3.3,3.2){cd}		% Ceci sont les points qui définissent le cadre. Il faut les laisser synchronisés avec la bounding box donnée pour la pspicture.
%   \pstGeonode(0,0){O}(1,0){X}(0,1){Y}(1,1){lcd}	% lcd est le point qui définit la pente du cône de lumière
%   \pstOrtSym{O}{Y}{lcd}[lcg]
%   \pstGeonode(-3,1.8){sg}(-0.8,1.5){sd}			% Placer la singularité 

%  \pstTranslation{O}{lcd}{sd}[lcds]
%  \pstTranslation{O}{lcg}{sg}[lcgs]
%  \pstInterLL{sd}{lcds}{sg}{lcgs}{t}			% Trois lignes pour trouver le trou noir


 %  \pstHomO[HomCoef=0.9]{O}{cd}[Hd]
 %  \pstHomO[HomCoef=0.9]{O}{cg}[Bg]			% Ici je place les bords des axes et du dessin proprement dit.

%  \pstProjectionOrth{O}{X}{Hd}{Ax}
%  \pstProjectionOrth{O}{Y}{Hd}{At}
%  \pstProjectionOrth{O}{X}{Bg}{Bx}
%   \pstProjectionOrth{O}{Y}{Bg}{Bt}
%   \pstHomO[HomCoef=0.9]{Ax}{Hd}[Bld]
%   \pstOrtSym{O}{Y}{Bld}[Blg]
	%\pspolygon[fillstyle=vlines,hatchcolor=red,linecolor=white](O)(Ax)(Bld)
	%\pspolygon[fillstyle=vlines,hatchcolor=red,linecolor=white](O)(Bx)(Blg)
%	\psline{->}(Bt)(At)
%	\psline{->}(Bx)(Ax)
	%\psline[linecolor=yellow]{->}(O)(Bld)
	%\psline[linecolor=yellow]{->}(O)(Blg)
%	\pstMarquePoint{At}{0.3;0}{$t$}
%	\pstMarquePoint{Ax}{0.3;0}{$x$}

	%\pstMarquePoint{Bld}{0.3;90}{$s=0$}
	%\psellipse[fillstyle=solid,fillcolor=white,linecolor=white](P)(0.7,0.4)
	%\pstMarquePoint{P}{0.1;35}{$s<0$}
	%\psellipse[fillstyle=solid,fillcolor=white,linecolor=white](Y)(0.7,0.4)
	%\pstMarquePoint{Y}{0.1;35}{$s>0$}
%	\pspolygon[fillstyle=vlines,hatchcolor=green,linecolor=green](sg)(sd)(t)
%	\psline[linecolor=red](sg)(sd)
%\end{pspicture}
%\end{center}
%\caption{The red line is the singularity and the green zone is the black hole associated with.}\label{FigBHMink}
%\end{figure}


In order the construction to be non trivial, we ask the black hole to be bigger than the singularity (that is of course part of the black hole), but smaller that the full space.

%The basic notions needed in order to define a causal structure on a time orientable pseudo-Riemannian manifold are that of time-, light- and space-like tangent vector. A tangent vector is said to be respectively \emph{time-}, \emph{space-} or \emph{light-like} when its norm is positive, negative or null; physically, only time-like vectors are allowed to be the velocity of an observer (this is the fact that light speed cannot be attained by a massive particle), and it is only possible for massless particle (such as photons) to follow trajectories with light-like tangent vectors.

From a geometric point of view, a black hole is the data of a causal manifold $M$ together with a subset $\hS \subset M$ called \emph{singularity} such that the whole manifold is divided into two parts: the \emph{interior} and the \emph{exterior} of the black hole. A point is said to be \emph{interior} if all future light-like geodesics through the point have a non empty intersection with the singularity. A point is \emph{exterior} if it is not interior. An important subset of the space is the \emph{event horizon}: the boundary between these two subsets.

\subsection{BTZ black hole}		\index{BTZ black hole}
%------------------------------

The BTZ black hole introduced and developed by Bañados, Teitelbaum, Zannelli and Henneaux in \cite{BTZ_un,BTZ_deux} is an example of a black hole whose singularity is not motivated by metric divergences\footnote{It turns out that general relativity accepts a lot of solutions presenting metric divergences; or more precisely, there are a lot of \emph{physical situations} from which Einstein's equations lead to divergences of some metric invariant such as the curvature.}. The construction is roughly as follows. We consider the anti de Sitter space in which we pick up a Killing vector field whose sign of norm is not constant. Then we perform a \emph{discrete} quotient along the integral curves of this vector field. Of course we obtain a lot of closed geodesics. The point is that, in the region where the Killing vector field is space-like, these closed curves are space-like. That violates the physical principle of causality. For that reason, we decree that this region is singular or, equivalently, that the boundary of this region is singular. The BTZ singularity is then the loci where the chosen Killing vector field has a vanishing norm. Since discrete quotients do not affect local structures, the resulting space remains a solution of the $(2+1)$-dimensional general relativity with negative cosmological constant\footnote{For honesty, we have to warn the reader that the real world's cosmological constant has been measured very small but positive. We also have to point out that the four dimensional anti de Sitter space is a solution of general relativity \emph{without masses}. From a physical point of view, this thesis has to be seen as a toy model.}. In this context one can define pertinent notions of  \emph{mass} and \emph{angular momentum} which depend on the chosen Killing vector field.

\begin{probleme}
Il faut trouver une référence pour dire que la constante cosmologique est positive.
\end{probleme}

In the case of the \emph{non-rotating massive} BTZ black hole, the structure of the singularity and the horizon are closely related to the action of a minimal parabolic (Iwasawa) subgroup of the isometry group of anti de Sitter, see \cite{BTZB_deux,Keio}. The whole work on the BTZ black hole and the fact that it belongs to the class of causal symmetric spaces (for definitions and some examples, see \cite{HilgertOlaf}) motivate the following definition:

\begin{definition}
A \defe{causal solvable symmetric black hole}{causal!solvable symmetric black hole} is a causal symmetric space where the closed orbits of minimal parabolic subgroups of its isometry group define a black hole singularity. See section \ref{SecCausal} for definitions of causality and singularity in the $AdS$ case.
\label{Def1}
\end{definition}

\subsection{Generalization and group setting}
%--------------------------------------------
\label{SubSecGEneBHGrop}


The original BTZ black hole was constructed in dimension three, but we will see in this chapter that, exploiting their group theoretical description, they can easily be generalized to any dimension, as pointed out in \cite{BDRS,lcTNAdS}.  Notice that higher-dimensional generalizations of the BTZ construction have been studied in the physics literature, by classifying the one-parameter isometry subgroups of $\Iso(AdS_l)=\SO(2,l-1)$, see \cite{Figueroa,AdSBH,Madden,BanadosIQxXuEh,Aminneborg,HolstPeldan}, but these approaches do not exploit the symmetric space structure of anti de Sitter.

The structure that will be described with full details in next pages may be summarized as follows. Take $G=\SO(2,l-1)$, fix a Cartan involution $\theta$ and a $\theta$-commuting involutive automorphism $\sigma$ of $G$ such that the subgroup $H$ of $G$ of the elements fixed by $\sigma$ is locally isomorphic to $\SO(1,l-1)$. The quotient space $M=G/H$ is a $l$-dimensional Lorentzian symmetric space, the {\sl anti de Sitter space-time}.  We denote by $\sG$ and $\sH$ the Lie algebras of $G$ and $H$. We have the decomposition $\sG=\sH\oplus\sQ$ into the $\pm 1$-eigenspace  of the differential at $e$ of $\sigma$ that we denote again by $\sigma$.  We also consider $\sG=\sK\oplus\sP$, the Cartan decomposition induced by $\theta$; and $\sA$, a $\sigma$-stable maximally abelian subalgebra of $\sP$. A positive system of roots is chosen  and let $\sN$ be the corresponding nilpotent subalgebra (see Iwasawa decomposition, theorem \ref{ThoIwasawaVrai}).  Set  $\overline{\sN}=\theta(\sN)$, $\sR=\sA\oplus\sN$ and $\overline{\sR}=\sA\oplus\overline{\sN}$. Finally denote by $R=AN$ and $\overline{R}=A\overline{N}$ the corresponding analytic subgroups of $G$.  One then has

\begin{theorem}
The $l$-dimensional anti de Sitter space with $l\geq 3$, seen as the symmetric space $\SO(2,l-1)/\SO(1,l-1)$, becomes a causal solvable symmetric black hole, as defined above, when the closed orbits of the Iwasawa subgroup $R$ of $\SO(2,l-1)$ and its Cartan conjugated $\overline{ R }$ are said to be singular. There exists in particular a non empty event horizon. The group $R$ has exactly two such closed orbits. 
\label{ThoLeBut}
 \end{theorem}

This chapter intends to prove this theorem, and for the sake of completeness, we also analyze in some detail in section \ref{sec_AdSdeux} the two-dimensional case, for which the construction does not yield a black hole structure.

The black hole causal structure is thus completely determined by the action of a solvable group.  This observation gives prominence to potential embeddings of these spaces in the framework of noncommutative geometry, in defining noncommutative causal black holes (see also \cite{BDRS}) through the existence of universal deformation formulae for solvable groups actions which have been obtained in the context of WKB-quantization of symplectic symmetric spaces \cite{StrictSolvableSym,Biel-Massar-2}. These issues are investigated in chapter \ref{ChDefoBH} and in \cite{articleBVCS}.


%---------------------------------------------------------------------------------------------------------------------------
					\subsection{Some notations}
%---------------------------------------------------------------------------------------------------------------------------

We are going to use the following notations. We denotes the \defe{free part}{free!part of a black hole} of the space $AdS_l$ by $\hF_l$; this is the subset of $AdS_l$ for which there exists a light-like direction which escapes the singularity. We denote by $BH_l$ the black hole in $AdS_l$; this is the set of points from which all the light-like geodesics intersect the singularity.



%+++++++++++++++++++++++++++++++++++++++++++++++++++++++++++++++++++++++++++++++++++++++++++++++++++++++++++++++++++++++++++
\section{Visite guidée}
%+++++++++++++++++++++++++++++++++++++++++++++++++++++++++++++++++++++++++++++++++++++++++++++++++++++++++++++++++++++++++++

%---------------------------------------------------------------------------------------------------------------------------
\subsection{En termes de BTZ}
%---------------------------------------------------------------------------------------------------------------------------

Nous travaillons dans $AdS_l=\SO(2,l-1)/\SO(1,l-1)=G/H$. Nous définissons les orbites fermées de $AN$ et $A\bar N$ (le groupe d'Iwasawa de $G$ et son conjugué par une involution de Cartan) comme \emph{singulières}.

Il a été prouvé il y a déjà bien longtemps que cette définition donne lieu à une structure de trou noir. Cette structure est par ailleurs la même, en dimension 3, que celle du trou noir BTZ connu de la physique. J'ai récemment poussé un peu plus loin et donné la structure de l'horizon en dimension $4$ en termes de celle en dimension $3$. Il se fait que (théorème \ref{ThoEqHorQCoore})

\begin{theorem}
L'horizon de $AdS_4$ est donné par
\begin{equation}		
	\hH_4=G_V\cdot \iota(\hH_3)\cup G_X\cdot\iota(\hH_3),
\end{equation}
où $\iota\colon \eR^4\to \eR^5$ est l'inclusion de $AdS_3$ dans $AdS_4$ et où les groupes $G_V$ et $G_X$ sont donnés par
\begin{equation}
	G_V=\{  e^{\alpha V}\tq\alpha\in\eR \},
\end{equation}
le vecteur $V$ étant l'élément de base de l'espace de racine $\sG_{(0,1)}$ de $\SO(2,3)$, et $X$ est l'élément de base de $\sG_{(0,-1)}$. Ces espaces de racines sont vides dans le cas de $AdS_3$.
\end{theorem}
L'inclusion $\iota$ peut également être vue comme l'inclusion du groupe $\SO(2,3)$ dans $\SO(2,4)$. La preuve est faite avec du calcul matriciel explicite très peu généralisable à d'autres espaces symétriques.

L'énoncé de ce théorème est la seule chose élégante de la section \ref{SecHOrOrbEquation}. Le reste est du calcul matriciel. Ce théorème donne, cependant, une bonne idée de ce vers quoi on va : il semble possible que les horizons en dimension supérieure s'obtiennent par récurrence. Le groupe qui générerait la singularité en dimension $l$ serait le groupe généré par les espaces de racines $(1,0)$ et $(0,1)$ de $\SO(2,l-1)$.

Affin de trouver des preuves plus intrinsèques, on commence par bien définir les différents éléments de l'algèbre, et en particulier la base de $\sQ$ en termes des espaces de racines. Cela se passe à la section \ref{SecRebuildStructRoot}. Je définit par exemple
\begin{equation}
	\begin{aligned}[]
		q_0&=(X_{++})_{\sQ\sK}\\
		q_2&=(X_{++})_{\sQ\sP},
	\end{aligned}
\end{equation}
et je montre que le premier est de norme (de Killing) positive et le second de norme négative, mais qu'en valeur absolue, ils ont la même norme. Je choisit donc $X_{++}$ de telle façon à ce que $q_0$ et $q_1$ soient normés à $1$. Les vecteurs $q_0\pm q_1$ sont donc de genre lumière.

Toute une série de propriétés sont ensuite prouvées. Le but est évidement de construire, de façon intrinsèque, une base de $\sA$, $\sN$, $\sK$ et de $\sQ$ de telle façon à avoir toutes les propriétés agréables que les matrices explicites avaient.

Cette partie est destinée à être remplacée par une application du théorème de structure de Pyatetskii-Shapiro. Un petit changement de base sera toutefois indispensable parce que l'élément dont l'annulation de la norme du champ de vecteur fondamental donne la singularité n'est pas dans la base donnée par Pyatetskii-Shapiro.

Tant que l'on travaillait avec des matrices et qu'on utilisait explicitement le fait que $AdS$ était un sous-ensemble de $\eR^n$, nous utilisions la caractérisation suivante de la singularité :
\begin{equation}
	\hS\equiv t^2-y^2=0.
\end{equation}
Maintenant, il est bon d'utiliser une caractérisation de la singularité qui ne fait pas appel aux coordonnées. Une telle caractérisation existe : si $J_1$ est un élément de $\sA\cap\sH$ (qui est de dimension $1$), alors la singularité est donnée par
\begin{equation}
	\hS\equiv \| J_1^* \|=0
\end{equation}
où $J_1^*$ est le champ de vecteur fondamental de l'action de $G$ sur $G/H$ associé au vecteur $J_1$. Cette caractérisation fait qu'un point $[g]\in G/H$ est dans la singularité si et seulement si le vecteur
\begin{equation}		\label{VisiteEqprQcaract}
	\pr_{\sQ}\left( \Ad(g^{-1})J_1 \right)
\end{equation}
a une norme nulle. Ici, $\pr_{\sQ}$ est la projection sur $\sQ$.

À part une foule de petits détail encore à vérifier, il est maintenant prouvé, en utilisant la caractérisation \eqref{VisiteEqprQcaract}, qu'un point $[kan]$ est dans la singularité si et seulement si il appartient à $[AN]$, $[A\bar N]$, $[-\mtu_{\SO(2)}AN]$ ou $[-\mtu_{\SO(2)}A\bar N]$, c'est à dire à une des orbites fermées de $AN$ ou de $A\bar N$.


Tout cela est dans le chapitre \ref{ChapBHinAdS}.

%---------------------------------------------------------------------------------------------------------------------------
\subsection{En termes de généralisations}
%---------------------------------------------------------------------------------------------------------------------------

Afin de se mettre dans une perspective de généralisation, l'idée suivante est proposée.

\begin{enumerate}

	\item
		


On considère un espace homogène symétrique $G/H$ où $G$ a 1000 décompositions d'Iwasawa possibles.

\item
 On sait par des arguments d'hermiticité et de $\mZ(\sK)$ non nul que la composante d'Iwasawa de $G$ est une $j$-algèbre.

\item
Il y a sûrement un argument pour dire qu'il existe des choix d'Iwasawa qui font que les racines positives et les éléments correspondants de $\sA\oplus\sN$ tombent exactement dans les $A$ ,$Z$ et $V $ de la décomposition en $j$-algèbres élémentaires.

\item
On choisit cette décomposition particulière d'Iwasawa comme décomposition "de référence".

\item
 On définit la singularité sur $G/H$ par $\| H1+H2\|=0$. Ici, c'est la première fois que le $H$ apparaît dans la construction.

 \item
 On considère l'Iwasawa qui change de base dans $\sA$ pour choisir $J1=H1+H2$ et $J2=H1-H2$. Cela devrait être fait sans changer de décomposition $\sG=\sK\oplus\sP$.

\item		\label{ItemVGDern}
 On prouve que la singularité est les orbites fermées de $AN$ et $A\theta(N)$ pour cette nouvelle Iwasawa.
\end{enumerate}


Le point \ref{ItemVGDern} est là uniquement pour montrer que l'ensemble de la construction redonne le BTZ déjà connu.

\section{Connectedness of groups and anti de Sitter spaces}
%++++++++++++++++++++++++++++++++++++++++++++++++++++++++++++++++

\label{PgDisGeoConnSO}Let us give some detail on the geometric nature of the two connected components of $\SO(p,q)$\footnote{See lemma \ref{LemOHjzfsL}.}; a physical discussion in the case of $\SO(1,3)$ can be found in the reference \cite{Schomblond_em}. What is proved in \cite{HelgasonSym} is that $\SO(p,q)$ is homeomorphic to the topological product
\[ 
  \SO(p,q)=\SO(p,q)\cap\SU(p+q)\times \eR^{d}=\SO(p,q)\cap\SO(p+q)\times \eR^{d}
\]
for some $d\in\eN$. Hence an element of $\SO(p,q)$ reads
\[ 
  \begin{pmatrix}
A&0\\
0&B
\end{pmatrix}\times v
\]
where $v\in\eR^{d}$, $A\in \gO(p)$, $B\in\gO(q)$ are such that $\det A\det B=1$. The $v$ part corresponds to boost while $A$ and $B$ correspond to pure temporal and pure spatial rotations. An element of $\gO(n)$ has always determinant equals to $\pm 1$. Therefore one can decompose the rotation part as $(\det A=\det B=1)\otimes (\det A=\det B=-1)$. Both parts are connected.

Hence the first connected component contains $\mtu$ while the second one contains the element that simultaneously changes the sign of one spacial and one time direction.

\subsection{The quotient for anti de Sitter}
%--------------------------------------------

Homogeneous space considerations (see section \ref{SecSymeStructAdS}) will naturally lead us to define the anti de Sitter space as the quotient $G/H=\SO(2,l-1)/\SO(1,l-1)$ while the black hole definition (section \ref{SecCausal}) needs to consider Iwasawa decompositions of $G$. So we face the problem that the Iwasawa theorem \ref{ThoIwasawaVrai} only works with connected groups. In order to prevent any problems of this type, we prove now that, if $G_0$ and $H_0$ denote the identity component of $\SO(2,l-1)$ and $\SO(1,l-1)$ respectively, then $G/H=G_0/H_0$.

The groups that are considered here have only two connected components $G_0$ and $G_1$. We can chose $i_1\in G_1\cap H$ such that $i_1^2=\mtu$. Using lemma \ref{LemConnSpecMo}, it easy to prove that 
\begin{itemize}
\item $G_0G_0=G_0$,
\item $G_0G_1=G_1$,
\item $G_1G_1=G_0$.
\end{itemize}
For the last one, take $g$ and $g'$ in $G_1$. Then consider $g_0$ and $g'_0$ in $G_0$ such that $g=g_0i_1$ and $g'=g_0'i_1$. If $g_0(t)$ and $g'_0(t)$ are path from $\mtu$ to $g_0$ and $g_0'$, then the path $g_0(t)i_1g'_0(t)i_1$ is a path from $\mtu$ to $gg'$.

\begin{proposition} \label{PropGHconn}
The map
\begin{equation}
\begin{aligned}
 \psi\colon G/H&\to G_0/H_0 \\ 
[g]&\mapsto \overline{ g_0 } 
\end{aligned}
\end{equation}
where we define $g_0=g$ when $g\in G_0$ or $g_0=gi_1$ when $g\in G_1$ is a diffeomorphism. The classes are $[g]=\{ gh\tq h\in H \}$ and $\overline{ g }=\{ gh_0\tq h_0\in H_0 \}$.
\end{proposition}

\begin{proof}
First we prove that $\psi$ is well defined. For that we suppose that $[g]=[g']$. There are three cases:
\begin{enumerate}
\item The elements $g$ and $g'$ both belong to $G_0$. In this case, $g'=gh_0$ with $h_0\in H_0$ and $\overline{ gh }=\overline{ g }$.
\item The element $g$ belongs to $G_0$ while $g'$ belongs to $G_1$. In this case, $g'=gh$ with $h=h_0i_1$ and $h_0\in H_0$. Then $\psi[g]=\overline{ g }$ and $\psi[g']= \overline{ (gh_0i_1)_0 }=\overline{ gh_0i_1i_1 }=\overline{ gh_0 }=\overline{ g } $.
\item The case with $g$ and $g'$ in $G_1$ is similar.
\end{enumerate}

The fact that the map $\psi$ is surjective is clear. For injectivity, let $\psi[g]=\psi[g']$, i.e. there exists a $h_0$ in $H_0$ such that $g'_0=g_0h_0$. Thus we have $g'i_1^k=gi_1^lh_0$ with $k,l=0,1$ following the cases. Then $g'=gi_1^lh_0i_1^k$ in which $i_1^lh_0i_1^k$ belongs to $H$, so that $[g']=[g]$.

\end{proof}


\section{Symmetric space structure on anti de Sitter}\label{SecSymeStructAdS}
%------------------------------------------

The $l$-dimensional anti de Sitter space $AdS_l$ can be described as set of points $(u,t,x_1,\ldots,x_{l-1})\in \eR^{2,l-1}$  such that $u^2+t^2-x_1^2-\ldots-x_{l-1}^2=1$. The next few pages are devoted to describe the homogeneous and symmetric space structures on $AdS_l$ induced by the transitive an isometric action of $\SO(2,l-1)$. We suppose that the groups $\SO(2,l-1)$ and $\SO(1,l-1)$ are parametrized in such a way that the second, seen as subgroup of the first one, leaves unchanged the vector $(1,0,\ldots,0)$. In this case, proposition 4.3 of chapter II in \cite{Helgason} provides the homogeneous space isomorphism
\begin{equation}
\begin{aligned}
  \SO(2,l-1)/\SO(1,l-1)&\to AdS_l \\ 
[g]&\mapsto  
 g\cdot
\begin{pmatrix}
1\\0\\\vdots
\end{pmatrix}
\end{aligned}
\end{equation}
where the dot denotes the usual ``matrix times vector'' action of the representative $g\in [g]$ in the defining representation of $\SO(2,l-1)$ on $\eR^{2,l-1}$. As far as notations are concerned, the classes are taken from the right:  $[g]=\{gh\tq h\in H\}$; in particular the class of the identity $e$ is denoted by $\mfo$; the groups $\SO(2,l-1)$ and $\SO(1,l-1)$ are denoted by $G$ and $H$ respectively and their Lie algebras by $\sG$ and $\sH$. Following proposition \ref{PropGHconn}, we can in fact only consider the identity components of $G$ and $H$. We denote by $\tau$ the natural action of $G$ on $G/H$:
\begin{equation}
\begin{aligned}
 \tau\colon G\times AdS_l&\to AdS_l \\ 
   \tau_r[g]&= [rg] 
\end{aligned}
\end{equation}

As far as dimensions are concerned, a candidate $R\subset G$ such that $R\cdot\mfo$ is open must satisfy
\begin{equation}\label{cond_dim}
                  \dim\mR\geq\dim M.
\end{equation}

The case that interest us is $G=\SOdn$ and $H=\SOun$:\nomenclature{$AdS_n$}{Anti de Sitter space}
\[
M=AdS_{n+1}=\dfrac{\SOdn}{\SOun},
\]
 so that we have to consider the action of $\SO(2,n)$ on $AdS_n$.  If $ANK$ is the Iwasawa decomposition of $\SO(2,n)$, we can consider more particularly the action of $R=AN$, and ask us if the orbit $R\cdot\mfo$ is open or not. It is easy to see that the condition \eqref{cond_dim} is satisfied. Indeed,
\[
 \dim\lG=\frac{n(n-1)}{2}+2n+1,\qquad\dim\lK=\frac{n(n-1)}{2}+1,
\]
so that $\dim(\mA\oplus\mN)=2n$, but $\dim AdS_n=n$. The Iwasawa subgroup\index{Iwasawa!group} $AN$ is a candidate for $AN\cdot\mfo$ to be open in $AdS_n$.

\begin{lemma}		\label{lem:Killing_ss_descent}
If $G$ is a semisimple Lie group and $H$ a semisimple subgroup of $G$, the restrictions on $H$ of the Killing form of $G$ is nondegenerate.
\end{lemma}
\begin{probleme}
Il faut une citation pour ce lemme.
\label{ProbCitLemDesc}
\end{probleme}

\begin{proposition}
The homogeneous space $AdS_l$ is reductive\index{reductive!$AdS_n$}.
\label{PropAdSreduct}
\end{proposition}

\begin{proof}
The proof relies on lemma \ref{lem:Killing_ss_descent} and the fact that $\SO(2,n)$ is semisimple. From the Killing form of $G$, one defines
\[
   \sQ=\sH^{\perp}=\{X\in\sG:B(X,H)=0\,\forall H\in\sH\}.
\]
Let $H$, $H'\in\sH$ and $Y\in\sQ$. From $\ad$-invariance of the Killing form, we have $B([H,Y],H')=0$. Hence $(\ad(\sH)\sQ)\subset \sQ$ and the claim is proved.

\end{proof}

Matrices of $\SO(2,n)$ are $(2+n)\times(2+n)$ matrices while the $n$-dimensional anti de Sitter space is a quotient of $\SO(2,n-1)$. In order to avoid confusions, we will reserve the letter $n$ to the study of the group $\SO(2,n)$ and the letter $l$ will denote the dimension of the anti de Sitter space which will thus be $AdS_l$.

We define the involutive automorphism $\sigma=\id|_{\sH}\oplus(-\id)|_{\sQ}$.  The vector space $\sQ$ can be identified with the tangent space $T_{[e]}AdS_l$, and that identification can be extended by defining $\sQ_g=dL_g\sQ$. In this case $\dpt{d\pi}{\sQ_g}{T_{[g]}AdS_l}$ is a vector space isomorphism.\label{PgdpibaseQTgM} An homogeneous metric on $T_{[g]}AdS_l$ is defined as in subsection \ref{SubsecKillHomo}.

Cartan decomposition of $\SO(2,l-1)$ are of crucial importance in chapter \ref{ChapBHinAdS}, so that we want to use a Cartan involution $\theta$ such that $[\sigma,\theta]=0$ (see \cite{Loos} page 153, theorem 2.1). One can show that $X\mapsto -X^t$ has that property. The corresponding Cartan decomposition is described in subsection \ref{SubSecCartandeuxN}.

As a consequence of relations \eqref{EqDefRedHQ}, 
\begin{equation}  \label{EqdpiAdpi}
d\pi\Ad(h)=\Ad(h) d\pi
\end{equation}
because, if $X\in\sQ$, $d\pi^{-1}(X)=\{ X+Y\tq Y\in\sH \}$, so $\Ad(h)Y\in\sH$ and $\Ad(h)X\in \sQ$.

\subsection{Anti de Sitter as symmetric space}\index{symmetric!space}
%--------------------------------------
\label{pg:AdS_n_syme}

We know the decomposition $\sodn=\sQ\oplus\sH$. From equation \eqref{EqDefRedHQ} one can find an involutive automorphism $\sigma$ of $\sG$ which leaves $\sH$ invariant. 

There exists a neighbourhood $U$ of $0$ in $\sodn$ on which $\exp$ is diffeomorphic to a neighbourhood $V$ of $e$ in $\SOdn$. We define $\dpt{\sigma_G}{V}{V}$ by $\sigma_G(e^X)=e^{\sigma X}$. Now, this $\sigma_G$ can be extended to the whole $G$. From now we will denote by $\sigma$ this map or its differential (i.e. an abuse of notation between $\sigma$ and $d\sigma_e$).

  
All this make $(\SOdn,\SOun)$ a symmetric pair. Since $H=\SOun$ is connected and fixed by $\sigma$, $H=H_{\sigma}=(H_{\sigma})_0$. Thus theorem  \ref{tho:sigma_theta} gives us a Cartan involution $\theta$ on $\sG$ such that $[\sigma,\theta]=0$ and theorem \ref{tho:sym_homo} gives a symmetric structure to $M=G/H$. Now we understand the computations of page \pageref{pg:calcul_sigma_theta}.

\section{Causality, light cone and related topics on anti de Sitter} \label{SecCausal}
%++++++++++++++++++++++++++++++++++++++++++++++++++++++++++++++++++

We particularize the general definitions of subsection \ref{SubSecGeneBH} to the case of the anti de Sitter space. We consider the $l$-dimensional\footnote{The symbol $n$ denotes the number of space-like directions of the underlying space of the matricial group $\SO(2,n)$; this space has dimension $n+2$ while $AdS$ is a quotient by (something like) one time-like direction. In order to avoid confusions, the symbol $l$ denotes the dimension of the $AdS$ space. This is the reason for which we write $\SOdn$ and $AdS_l$. %
    So equation \eqref{eq:defAdS} is best written as \[AdS_l=\frac{\SOdn}{\SO(1,n)}.\]} anti de Sitter space
\begin{equation}    \label{eq:defAdS}
  AdS_l=\frac{ \SO(2,l-1) }{ \SO(1,l-1) }(\equiv u^2+t^2-x_1^2-\cdots-x_{l-1}^2=1).
 \end{equation}
According to proposition \ref{PropGHconn}, we can only consider the identity component of $\SO(2,l-1)$ and $\SO(1,l-1)$ instead of full groups\footnote{Since we are about to consider Iwasawa decompositions of these groups, actually we \emph{have to} use the identity components.}. The metric that we put on $AdS_l$ is the one induced from the Killing form of $\SO(2,l-1)$ by formula \eqref{EqDefMetrHomo}. This metric has a Minkowskian signature, so that we have  natural notions of time-, space- and light-like vectors. From now we denote by $G$ and $H$ the groups $\SO(2,l-1)$ and $\SO(1,l-1)$. 

An other beautiful way to see that the metric on $AdS$ as one and only one time-like direction is the following. The tangent space of $AdS$ at the point $(u,t,x_1,\cdots,x_{l-1})$ is the orthogonal complement (in $\eR^{2,l-1}$) of that vector. From the very definition of $AdS$, the given vector is time-like (its norm is $1$), so that it remains one and only one time-like vector in the tangent space.

The connected group $\SO_0(2,l-1)$ admits an Iwasawa decomposition $ANK$ (see theorem \ref{ThoIwasawaVrai}). Let $A\bar N$ be the $\theta$-conjugate\footnote{Roughly speaking, it corresponds to different choices in the Iwasawa decomposition of $\SO(2,l-1)$.}group of $AN$ where $\theta$ is the Cartan involution of subsection \ref{SubSecCartandeuxN}. We will see that the actions of $AN$ and $A\bar N$ have closed and open orbits. The closed ones are denoted by $\hS_{AN}$ and $\hS_{A\bar N}$. The following definition is motivated all previously existing work about BTZ black hole.

\begin{remark}
Here, we consider $\SOun$ as a subgroup of $\SOdn$. Thus the matrices of $\SOun$ are $(n+2)\times (n+2)$ of the form
\[
\begin{pmatrix}
   1 & 0\\
   0 & \fbox{M}
\end{pmatrix}
\]
where $M$ is a $(n+1)\times (n+1)$ matrix of the ``true''\ $\SOun$. From this, one can believe the closeness of $\SOun$ in $\SOdn$.
\end{remark}

In order to get a full definition of the black hole and its structure, we need to define and characterise the notions of light ray and light cone. These notions are of course directly issued from physics of relativity.  
\begin{definition}
A \defe{light ray}{light!ray} is a geodesic whose tangent vector is everywhere light-like.
\label{lightraycone}
 \end{definition}

The \defe{causal structure}{causal!structure} of a general pseudo-Riemannian manifold $M$ is the fact that two points are said to be \emph{causally connected} when there exists a light ray which passes by both points. More precisely, we say that $x$ has a \defe{causal effect}{causal!effect} on $y$ if there exists a future oriented time-like path $c\colon [0,1]\to M$ such that $c(0)=x$ and $c(1)=y$.

A light ray trough $\mfo$ is given by a vector of $\sQ$ with vanishing norm. So let us study these vectors. Let $E_1=q_0+q_1$ and $k$, a general element of  $\SO(n)$ which reads $k= e^{K}$ with $K=a^{ij}(E_{ij}-E_{ji})$, $i,j\geq 3$ and $a^{ij}=-a^{ji}$.  If we pose $A_j=E_{1j}+E_{j1}$, we have $[K,E_1]=(2a)^{j3}A_j$ and $[K,A_k]=a^{jk}A_j$. Hence,
\[
\ad(K)^nE_1=\big((2a)^n\big)^{k3}A_k,
\]
and
\begin{equation} \label{eq:Adkeu} 
\begin{split}
\Ad(k)E_1=e^{\ad K}E_1&=E_1+\sum_{n\geq 1}\big( (2a)^n\big)^{k3}A_k\\
	      &=E_1+\sum_{n=0}^{\infty}\big(  (2a)^n \big)^{k3}A_k-\delta^{j3}A_j\\
		&=E_1-E_{31}-E_{13}+\big( e^{2a}\big)^{j3}A_j\\
	      &=q_0+\sum_{j=1}^{l-1}w_jq_j
\end{split}
\end{equation}
where $w_i=\big(  e^{2a} \big)^{i3}$. Under an explicit form, we have 
 \begin{equation} \label{eq:AdkE} 
   \Ad(k)E_1=
\begin{pmatrix}
0&1&w_1&w_2&\ldots\\
-1\\
w_1\\
w_2\\
\vdots
\end{pmatrix}
\end{equation}
The exponential $ e^{2a}$ being an element of $\SO(n)$, the parameters $w_i$ are restricted by the condition $\sum_{k}w_k^2=1$.  Remark moreover that \emph{every} matrix of $\SO(2)$ can be written under the form $e^{2a}$ for a good choice of $a\in\sod$. The light cone is therefore given by the set of vectors of the form $(1,w_i)$ with $\|w\|^2=1$. If we consider the metric $diag(+--\cdots)$ on $\sQ$ with respect to the basis $\{q_i\}$, we have
\[
  \|\Ad(k)E_1\|^2=0.
\]
This is coherent with the intuitive notion of light cone. On the one hand \emph{every} light-like vector of $\sQ$ reads $\Ad(k)E_1$ for some $k\in\SO(n)$. On the other hand every nilpotent element of $\sQ$ is light-like because trace of nilpotent matrix is zero (using \wikipedia{en}{Engel_theorem}{Engel's theorem}). In definitive, we proved the following:

\begin{proposition}		\label{PropNormZeroEQnil}
When $E$ is any nilpotent element of $\sQ$, the set of light-like vectors of $\sQ$ is parametrized by $\lambda\Ad(k)E$ with $k\in\SO(n)$ and $\lambda\in\eR$.
\label{PropToutVectLumQ}
\end{proposition}

\begin{corollary}		\label{CorNormZeroEQnil}
An element of $\sQ$ has a vanishing norm if and only if it is nilpotent.
\end{corollary}

\begin{proof}
We know that, when $E$, is any nilpotent in $\sQ$, the set of vanishing norm vectors in $\sQ$ are given by $\{ \lambda\Ad(k)E \}$, but all these vectors are nilpotent.
\end{proof}

Let us point out the fact that only the first column of the ``direction''{} $k\in \SO(n)$ has an importance in causality issues. So the word ``directions''{} will often be used to refer to the vector $w$. It is not a particular feature of our particular matrix representation choice. Indeed the element $k$ only appears in the combination $\Ad(k)E$ which is a light-like vector in $\sQ$, i.e. $\Ad(k)E=tq'_0+\sum_i x_iq'_i$ with $t^2-\sum_i x_i^2=0$ for any orthonormal basis $\{q'_i\}$ of $\sQ$. As far as causality is concerned, a rescaling $\Ad(k)E$ to $\lambda\Ad(k)E$ has no importance, so one can choice $t=1$ and find back $\sum_i x_i^2=1$. We see that it is a natural feature that the light-like rays are parametrized by  unital vectors of $\eR^n$.

\begin{lemma}		\label{LemGeodGenreLumiere}
Let $E$ be a nilpotent element in $\sQ$, and $\pi: G \rightarrow G/H$, the canonical projection. A light ray through $[g]\in AdS_l$ has the form
\begin{equation}
   l^k_{[g]}(s)=\pi\big( ge^{-s\Ad(k)E} \big)
\end{equation}
for a certain $k\in K_H=K\cap H=\SO(n)$.
 \label{lem:AdkEcone}
\end{lemma}

\begin{proof}
General theory of symmetric spaces (see \cite{kobayashi2}, pages 230--233, particularly theorem 3.2) proves that a light ray through $\mfo=[e]$ has the form
\[
  l(s)=\pi\big( e^{sX} \big).
\]

\begin{probleme}
	Il me semble que ce qui est de cette forme, ce sont les géodésiques, et non les rayons de lumière. Relire Kobayashi-Nomizu.
\end{probleme}


In our context, we have the additional request for the tangent vector to be light-like. Proposition \ref{PropToutVectLumQ} thus imposes $X$ to be of the form $\Ad(k)E$. That proves the claim for geodesics trough $\mfo$.

The fact that $d\tau_g$ is an nondegenerate isometry then extends the result to all points.

\end{proof}

\begin{corollary}		\label{CorNilLightQ}
If $E$ is nilpotent in $\sQ$, then $\{\Ad(k)E\}_{k\in K_H}$ is the set of light-like vectors in $T_{[\mfo]}AdS_l\simeq\sQ$. Therefore
\begin{equation}
  \exp_{\mfo}( t\Ad(k)E )=\exp(t\Ad(k)E)\cdot\mfo.
\end{equation}
is the light cone of $\mfo$ in $AdS_l$.
    Note that in this equation, the first $\exp$ is the
one defined from the $AdS_l$-connection while the second is the exponential from a Lie algebra to the Lie group. It comes from the fact that in a symmetric space, $\exp_o v=e^z\cdot\mfo$.
\end{corollary}

In order to fix ideas, we will always use the element $E_1$ as choice of nilpotent element in $\sQ$ in order to parametrize light-cone.  Since $\SO(2,l-1)$ acts on $AdS_l$ by isometries, the \defe{light cone}{light!cone} at $\pi(g)$ is given by a translation of the one at $\mfo$:
\begin{equation}	\label{eq_defcone}
  C^+_{\pi(g)}=g\cdot C_{\mfo}=\{  \pi\big( g e^{t\Ad(k)E_1}  \big)  \}_{\substack{t\in\eR^+\\ k\in K_H}}.
\end{equation} 
The product being taken at left while the quotient being taken at right, one can fear a problem of well definiteness in this expression. The following proposition shows that all is right.

\begin{proposition}		\label{PropDefConeIndepRepre}
Definition \eqref{eq_defcone} is independent of the representative $g$ in the class $\pi(g)$. In other words,
\begin{equation}  \label{eq_statdefcone}
  \{ \Ad(hk)E_1 \}_{k\in K_H}=\{ \Ad(k)E_1 \}_{k\in K_H}
\end{equation} 
for all $h\in H$. 
\end{proposition}

\begin{proof}
The metric on $\sQ$ is the restriction of the Killing form of $\sG$ (notice that $\sQ$ has no own Killing form for the simple reason that it is not a Lie algebra). From $\Ad$-invariance, we have in particular
\[
  B\big(\Ad(h)X,\Ad(h)Y \big)=B(X,Y)
\]
for all $h\in \SO(1,l-1)$. The point is that reducibility makes $\Ad(h)X\in\sQ$ when $X\in\sQ$. The element $\Ad(hk)E_1$ in the left hand side of equation \eqref{eq_statdefcone} being zero-normed in $\sQ$, it reads $\Ad(k')E_1$ for some $k'\in K_H$. That proves the inclusion in one sense. For the second inclusion, we have to find a $k'\in K_H$ such that $\Ad(hk')E_1=\Ad(k)E_1$. Existence of such a $k'$ follows from the fact that $\Ad(h^{-1}k)E_1$ is a light-like vector of $\sQ$.
\end{proof}

\begin{remark}		\label{RemGedNonInvarChoix}
Although the \emph{set} of geodesics $\{ \pi(g e^{s\Ad(k)E_2}) \}$ is equal to the \emph{set} of geodesics $\pi(gh e^{s\Ad(k)E_1})$, each geodesic are not independent in the choice of the representative $g$: $\pi(g e^{\Ad(k)E_1})\neq\pi(gh e^{\Ad(k)E_1})$ in general.

In particular, in the setting of the anti de Sitter black hole, the property ``intersect the singularity'' for the geodesic $\pi(g e^{s\Ad(k)E_1})$ is not invariant under the choice of the representative $g$ in the class $[g]$.
\end{remark}

It is also possible to prove result of independence \ref{PropDefConeIndepRepre} with a lot of matricial computations: let us decompose $h=a_hn_hk_h$; the part $k_h$ is just a redefinition of $k$ in equation \eqref{eq_statdefcone}, so we forget it. We begin by proving that \eqref{eq_statdefcone} holds whenever $\Ad(h)\in \SO(\sQ)$. Consider $\Ad(k')E_1=X\in\sQ$. If $\Ad(h)\in \SO(\sQ)$, then $\Ad(h^{-1})\in \SO(\sQ)$ too and we consider $Y=\Ad(h^{-1})X$ which is a vector of norm zero in $\sQ$. There exists $\bar k\in K_H$ such that $\Ad(\bar k)E_1X=Y$. Now,
\begin{equation}
\Ad(h\bar kk')E_1=\Ad(h\bar k)X
		=\Ad(h)Y
		=X.
\end{equation}
In order to prove that $\Ad(a_h)\in \SO(\sQ)$, we compute
\[
  \ad(J_1)\begin{pmatrix}
0	& z	& w_1	& w_2	& w3\\
-z\\
w_1\\
w_2\\
w_3
\end{pmatrix}
=\ad(J_1)(zq_0+w_iq_i).
\]
In the basis $\{ q_0,q_i \}$, we see that
\[
  \ad(J_1)=\begin{pmatrix}
0&0&-1&0\\
0\\-1\\0
\end{pmatrix}\in\mathfrak{so}(1,3),
\]
so $\Ad(J_1)\in \SO(\sQ)$. On the other hand, a general element of $\sN_{\sH}$ is
\[
  A=\begin{pmatrix}
\cdot\\
&\cdot& a&\cdot& v\\
&a&\cdot &-a&\cdot\\
&  \cdot& a&\cdot& v\\
&v&\cdot&-v&\cdot\\
\end{pmatrix},
\]
and simple computations shows that on $\sQ$,
\[
  \ad(A)=\begin{pmatrix}
\cdot &-a&\cdot&-v\\
-a&\cdot&-a&\cdot\\
\cdot&a&\cdot &v\\
-v&\cdot&v&\cdot
\end{pmatrix}\in\mathfrak{so}(1,3).
\]

\subsection{Time orientation}
%////////////////////////////

A \defe{time orientation}{time!orientation} on $\sQ$ is the choice of a vector $T$ such that $\scal{T}{T}>0$. When such a choice is made, a vector $v$ is \defe{future directed}{future!directed vector} when $\scal{v}{T}>0$. In our case, the choice is the intuitive one: the vector $q_0$ defines the time orientation on $\sQ$ and $v=(v^0,v^1,v^2,v^3)$ is future directed if and only if $v^0>0$. So a light-like future directed vector is always --up to a positive multiple-- of the form $(1,\overline{v})$ with $\|\overline{v}\|=1$. For this reason, the set
\begin{equation}	\label{EqTousVecLumTy}
  \{t\Ad(k)E_1\}_{%
\begin{subarray}{l}
t>0\\k\in \SO(3)
\end{subarray}
}
\end{equation}
is exactly the set of light-like future-directed vectors of $\sQ$.

We are now able to define causality as follows.  A point $[g]\in AdS_l$ belongs to the \defe{interior region}{interior!region} if for every direction $k\in K_H$, the future light ray $l^k_{[g]}$ intersects the singularity within a \emph{finite} time.  In other words, it is interior when the whole light cone ends up in the singularity.  A point which is not interior is said to be \defe{exterior}{exterior!point}. A particularly important set is the \defe{event horizon}{event horizon}, or simply \emph{horizon}, defined as the boundary of the interior. When a space contains a non trivial causal structure (i.e. when there exists a non empty horizon), we say that the definition of singularity gives rise to a \defe{black hole}{black hole}.  By extension, the term ``black hole'' often refers to the set of interior points.

%///////////////////////////////////////////////////////////////////////////////////////////////////////////////////////////
\subsubsection{Singularity}
%///////////////////////////////////////////////////////////////////////////////////////////////////////////////////////////

\begin{definition}		\label{Singular}
    The \defe{singularity}{singularity} in $AdS_l$ is the set
    \[
      \hS=\text{singularity}=\hS_{AN}\cup\hS_{A\bar N},
    \]
    so that a point is \defe{singular}{singular!point in a black hole} when it belongs to a closed orbit of $AN$ or $A\bar N$. The \defe{black hole}{black hole} is defined as
    \[
      BH=\{ x\in AdS_{l} \text{ st } \forall \text{ time-like vector } k\in T_xAdS_l,\,  l^k_x\cap\mathcal{S}\neq\emptyset \}
    \]
    where $l^k_x$ is the (future directed) geodesic in the direction $k$ starting at $x$ (see equation \eqref{EqTousVecLumTy} and the discussion above).
\end{definition}

The aim of this chapter is to prove that the so-defined black hole is non trivial in the sense that the following inclusions are strict:
\begin{equation}		\label{EqhSssubBH}
 \hS\subset BH\subset AdS_l.
 \end{equation}

\subsection{Action of \texorpdfstring{$H$}{H} and \texorpdfstring{$\Ad(\sQ)$}{AdQ}}
%///////////////////////////////

Remember that we decree closed orbits to be \emph{singular}. Now the fact for a point $\pi(g)\in AdS_l$ to be \emph{exterior} is that there exists an non empty set $\mO$ of $K_H$ such that $\forall k\in\mO$,
\[
  \pi\big( g e^{t\Ad(k)E_1}  \big)\cap\mS=\emptyset.
\]

The restriction of the Killing form to $\sQ$ reads
\begin{subequations}
\begin{align}
	B(q_0,q_0)&=\tr(q_0q_0)=-2,\\
	B(q_{i},q_{i})&=\tr(q_{i},q_{i})=2&\textrm{for $i\geq 1$}.
\end{align}
\end{subequations}
So the norm on $\sQ$ is $\| X \|=-\frac{ 1 }{2}B(X,X)$. The bi-invariance of the Killing form and the fact that the decomposition $\sG=\sQ\oplus\sH$ is reductive  imply $\| \Ad(h)X \|=\| X \|$, hence
\begin{equation}  \label{EqInclAdHSOq}
  \Ad(H)|_{\sQ}\subset\SO(\sQ).
\end{equation} 
A question is to know the kernel of this inclusion: which $h\in H$ fulfill $\Ad(h)q_i=q_i$ for all $i$ ? The equation $Aq_iA^{-1}=q_i$ can be simplified (from a computational point of view) using the relation $A^{-1}=\eta A^t\eta$ which defines $\SO(1,n)$. It is a somewhat long but easy computation to prove that $A=\pm\mtu$ are the only two solutions in $SO(1,n)$ to the system $A(q_i\eta)A^t=q_i\eta$.

One can go further than inclusion \eqref{EqInclAdHSOq} and prove the following
\begin{proposition}		
 Let $h\in H_0$ seen as a matrix acting on $\eR^{1,l-1}$ and let see $\Ad(h)$ as a matrix acting on $\sQ$. In this case we have $\Ad(h)_{ij}=h_{ij}$. In particular
\begin{equation}
   \Ad(H_0)=\SO_0(\sQ)
\end{equation} 
where the index zero denotes the identity component.
\label{PropSOADHequal}
\end{proposition}

\begin{proof}
We will prove that for each unital vector $X\in\sQ$, the element $\Ad(h)X$ is a general element of norm $1$ in $\sQ$ when $h$ runs over $H_0$. Explicit matrix computation will show by the way the equality  $\Ad(h)_{ij}=h_{ij}$. The general product to be computed is
\[ 
\Ad(h)X=
  \begin{pmatrix}
1	&	0\\
0	&
\begin{pmatrix}
&&\\
&h^{-1}\\
&&
\end{pmatrix}
\end{pmatrix}
\begin{pmatrix}
0&-w_0&w_1&\cdots\\
w_0\\
w_1\\
\vdots
\end{pmatrix}
  \begin{pmatrix}
1	&	0\\
0	&
\begin{pmatrix}
&&\\
&h\\
&&
\end{pmatrix}
\end{pmatrix}.
\]
But we know that the result is a matrix of $\sQ$, so it is sufficient to compute the first line. If we denote by $c_i$ the columns of $h$, we find
\[ 
  \Ad(h)X=\sum_{i=0}^{l-1}(w\cdot c_i)q_i
\]
where the dot denotes the inner product of $\eR^{1,l-1}$. Since $\{ c_i \}$ is a general orthonormal basis of $\eR^{1,l-1}$, the latter expression is a general vector of norm $1$ in $\sQ$.
\end{proof}



\section{Open and closed orbits}
%+++++++++++++++++++++++++++++++

\subsection{Openness of orbits in homogeneous spaces} \label{subsec:question}
%---------------------------------------------------

\begin{proposition}
The orbits of $AN$ are submanifolds of $G/H$.
\label{pg:orbit_ssvar}
\end{proposition}

\begin{proof}
 Indeed proposition \ref{prop:orbit_N_ss_var} makes $R/(R\cap H)$ the orbit of $\pi(e)$ by $R$ and assure us that it is a submanifold of $G/H$. That proves the proposition for the orbit of $e$. 

For the other orbits, we consider the group $R_z=\AD(z^{-1})R$ which is also a Lie  subgroup of $G$. The space $R_z/(R_z\cap H)$ is isomorphic to the orbit of $\pi(e)$ under the action of $R_z$. Therefore $zR[z^{-1}]$ is a submanifold of $G/H$ and the very definition of a Lie group makes that  $R[z^{-1}]$ is a submanifold too.

\end{proof}

Let us start by computing the closed orbits of the actions of $AN$ and $A\bar{N}$ on $AdS_l$. In order to see if $[g]\in AdS_l$ belongs to a closed orbit of $AN$, we ``compare'' the space spanned by the basis $\{d\pi dL_g q_i\}$ of $T_{[g]}AdS_l$ and the space spanned by the fundamental vectors of the action. If these two spaces are equal, then $[g]$ belongs to an open orbit (because a submanifold is open if and only if it has same dimension as the main manifold). That idea is precisely contained in the following theorem which holds for any homogeneous space $M=G/H$.

\begin{probleme}
Il faut trouver une référence pour ce théorème.
\end{probleme}


\begin{theorem}
If $R$ is a subgroup of $G$ with Lie algebra $\sR$, then the orbit $R\cdot \mfo$ is open in $G/H$ if and only if the projection $\dpt{\pr}{\sR}{\sQ}$ parallel to $\sH$ is surjective.
\label{tho:pr_ouvert}
\end{theorem}

The projection is defined by $\pr(X)=X_{\sQ}$ if $X=X_{\sQ}+X_{\sH}$ is the decomposition of $X\in\sG$ with respect to the decomposition $\sG=\sH\oplus\sQ$. We need two lemmas before to prove the theorem.

\begin{lemma}
The orbit $R\cdot\mfo$ is open if and only if
\[
    \Span\{X^*_{\mfo}|X\in\mR\}=T_{\mfo}M
\]
where $X^*$ is the fundamental field defined by equation \eqref{EqDefChmpFonfOff}.
\label{lem:equiv_1}
\end{lemma}

\begin{proof}
From general theory of fundamental fields (lemma \ref{LemFundSpansTan}) we know that
\[
\Span\{X^*_{\mfo}|X\in\sG\}=T_{\mfo}M.
\]
The game is now to prove that one can replace $\sG$ by $\sR$ if and only if $R\cdot \mfo$ is open.

\subdem{Necessary condition}
If $R\cdot\mfo$ is open, we have a neighbourhood of $\mfo$ which is contained in $R\cdot\mfo$. Then for any $X\in\sG$, and for a small enough $t$, the element $e^{-tX}\cdot\mfo$ belongs to $R\cdot\mfo$. Hence we have a path $r_X(t)$ in $R$ such that $e^{-tX}\cdot\mfo=r_X(t)\cdot\mfo$:
\[
      \Dsdd{e^{-tX}\cdot\mfo}{t}{0}=\Dsdd{r_X(t)\cdot\mfo}{t}{0}.
\]
Since $r_X(t)$ is a path in $R$, we can replace it by a $e^{-tY}$ with a $Y\in\mR$ in the derivative. For this $Y$, we have $X^*_{\mfo}=Y^*_{\mfo}$.

\subdem{Sufficient condition} We have $\dim(R\cdot\mfo)=\dim\Span\{ X^*_{\mfo}\tq X\in\sR \}=\dim T_{\mfo}M$,
so $R\cdot\mfo$ has the same dimension as $M$. The conclusion follows from the fact that a submanifold is open if and only if it has maximal dimension.

\end{proof}

\begin{lemma}
The canonical projection is surjective from $\sR$ to the tangent space to identity:
\begin{equation}\label{eq:equiv_2}
    \Span\{X^*_{\mfo}|X\in\mR\}=d\pi_e(\mR).
\end{equation}

\label{XsdpiR}

\end{lemma}

\begin{proof}
 Consider the following computation when $X\in\mR=T_eR$ is given by the path $X(t)=e^{tX}$:
\begin{equation}
  d\pi_e X=\Dsdd{[X(t)]}{t}{0}
	=\Dsdd{e^{tX}\mfo}{t}{0}
	=Y^*_{\mfo}
\end{equation}
with $Y=-X$. Reading these lines from left to right shows that $d\pi_e(\mR)\subseteq\{X^*_{\mfo}:X\in\mR\}$ while reading it from right to left shows the inverse inclusion.
\end{proof}

%\begin{proposition}
%The orbit $R\cdot\mfo$ is open in $G/H$ if and only if $\dpt{\pr}{\mR}{\sQ}$ is surjective.
%\label{prop:ouvert_ssi}
%\end{proposition}

We are now able to prove the theorem.

\begin{proof}[Proof of theorem \ref{tho:pr_ouvert}]
From lemma \ref{lem:equiv_1} and lemma \ref{XsdpiR}, the orbit $R\cdot\mfo$ is open if and only if $\dpt{d\pi_e}{\mR}{T_{\mfo}M}$ is surjective. On the one hand any $X\in\mR$ can uniquely be written as $X=X_{\sH}+X_{\sQ}$ with $X_{\sH}\in\sH$ and $X_{\sQ}\in\sQ$. On the other hand it is clear that $d\pi_e X_{\sH}=0$, thus $R\cdot\mfo$ is open if and only if $\dpt{d\pi_e}{\RM}{T_{\mfo}M}$ is surjective.

Now, recall that $d\pi_e$ is surjective from $\sG$, hence it is surjective from $\sQ$. The first conclusion is that if $\dpt{\pr}{\mR}{\sQ}$ is surjective, then $R\cdot\mfo$ is open. The inverse implication remains to be proved.

We know that openness $R\cdot\mfo$ implies that $\dpt{d\pi_e}{\RM}{T_{\mfo}M}$ is bijective (surjective because $R\cdot\mfo$ is open and injective because $\dpt{d\pi_e}{\sQ}{T_{\mfo}M}$ is injective by lemma \ref{LemdpiisomMTM}). From all that, one concludes that $\RM=\sQ$. Indeed,  suppose that $X_{\sQ}\in\sQ$ and $X_{\sQ}\notin\RM$. Since $\dpt{d\pi_e}{\RM}{T_{\mfo}M}$ is surjective, there exists a $X_{\sQ}'\in\RM$ such that $d\pi_eX_{\sQ}'=d\pi_eX_{\sQ}'$. This is impossible because $d\pi_e$ is injective from the whole $\sQ$.

\end{proof}

\subsection{Open orbits in anti de Sitter spaces}
%----------------------------------------------

Now the strategy is to to check openness of the $R$-orbit of $[g]$ by checking openness of the $\AD(g^{-1})R$-orbit of $\mfo$ using the theorem \ref{tho:pr_ouvert}.

The problem is simplified by the following remark.  We know that matrices of $K$ and $H$ are given by
\begin{equation}	\label{eq:K_H_SO}
  K\leadsto \begin{pmatrix}
                \SO(2)&   \\
		      & \SO(n)
            \end{pmatrix},\quad
  H\leadsto \begin{pmatrix}
                    1 & \\
		     & \SOun
            \end{pmatrix},
\end{equation}
so we obviously have
\[
\bigcup_{s\in \SO(2)} \tau_{AN}([s]) =\bigcup_{\substack { s\in \SO(2)\\ h\in \SO(n)}}[ANsh] =\bigcup_{k\in K} [ANk] =[G].
\]
This is nothing else than the fact that the $AN$-orbits are $AN$-invariant.
So the $K$ part of $[g]=ank$ alone fixes the orbit which contains $[g]$ and we have at most one orbit for each element in $\SO(2)$. Computations using theorem \ref{tho:pr_ouvert} show that the $R$-orbits of $[\mu]$ with
\[
\mu=
\begin{pmatrix}
\cos\mu &\sin\mu\\
-\sin\mu&\cos\mu\\
&&\mtu
\end{pmatrix}
\]
is not open if and only if $\sin \mu=0$. We will see later that they are actually closed (page \pageref{PgTopoOrb}), so that the singularity is described as
\begin{equation}\label{Sing2}
\hS=[AN(\pm\mtu_{\SO(2)})]\bigcup[A \bar{N}(\pm\mtu_{\SO(2)})].
\end{equation}
 Because of $AN$-invariance of the $AN$-orbits, the equation of the $AN$-closed orbits can be expressed as
\begin{equation}
\sin \mu=0.
\end{equation}

Let us recall that $-\mtu_{\SO(2)}=k_{\theta}= e^{\pi q_0}$. With these notations, we have that the closed orbits of $AN$ are
\begin{equation}
	\begin{aligned}[]
		[AN]&&\text{and}&&[ANk_{\theta}]=[k_{\theta}A\bar N],
	\end{aligned}
\end{equation}
while the closed orbits of $A\bar N$ are given by
\begin{equation}
	\begin{aligned}[]
		[A\bar N]&&\text{and}&&[A\bar Nk_{\theta}]=[k_{\theta}AN].
	\end{aligned}
\end{equation}

Notice that there are some differences between the two choices of Iwasawa decompositions of equations \eqref{TabelPrem} and \eqref{TableSeconde} in the determination of open and closed orbits. In the $AN$ Iwasawa decomposition, up to matrices of $\sH$ (given in equation \eqref{eq:gene_H}), a general matrix of $\sR$ is $jJ_1+mM+lL+kJ_2$. If we note $x=m+l$,
\begin{equation} \label{eq:geneR}
\sR\leadsto
\begin{pmatrix}
0&x&k&-x\\
-x\\
k\\
-x
\end{pmatrix}
\end{equation}
and it is obvious that the matrix $q_0$ can't be obtained by combinations of such matrices. So the $R$-orbit of $\mfo$ is not open.

We can do the same computation with the Iwasawa group $\bar\sR=\sA\oplus\bar\sN$. A general element of this is of the form $jJ_1+kJ_2+nN+fF$. If we write $a=n+f$ and $b=n-f$, we get
\begin{equation}
	\begin{pmatrix}
 0	&	a	&	k	&	a	&	0\\ 
 -a	&	0	&	b	&	j	&	0\\ 
 k	&	b	&	0	&	b	&	0\\ 
 a	&	j	&	-b	&	0	&	0\\ 
0	&	0	&	0	&	0	&	0
 \end{pmatrix}.
\end{equation}
Looking at the positions of the $a$, we see that it is impossible to put the element $q_0$ under that form. We deduce that the $\bar R$-orbits of $\mfo$ are not open neither.

That situation is, however, not generic. If we use for example the other Iwasawa decomposition, the one of subsection \ref{SubSecANbarIwa}, the result is completely different. We have
\begin{align}
  q_{0}&=\pr\left( \frac{ N+M }{ 2 } \right),
&q_{1}&=\pr H_{2},
&q_{2}&=\pr\left( N-\frac{ N+M }{ 2 } \right),
\end{align}
and other elements of $\sQ$ are projections of the matrices $V_{i}$'s.  So we see that the map $\dpt{\pr}{\iR}{\sQ}$ is surjective and \label{pg:mfo_ouvert} the orbit $R\cdot\mfo$ is open.

Here is some explicit matricial computation.
\[
   M+N=2
 \underbrace{
\begin{pmatrix}
  0 &1&0&0\\
  -1&0&0&0\\
  0 &0&0&0\\
  0 &0&0&0
\end{pmatrix}}_{\displaystyle\in\sQ}
+
\underbrace{
\begin{pmatrix}
  0&0&0&0\\
  0&0&2&0\\
  0&2&0&0\\
  0&0&0&0
\end{pmatrix}}_{\displaystyle\in\sH},
\]
thus $\pr(\frac{M+N}{2})$ is yet a part of $\sQ$. An other:
\[
 \underbrace{
\begin{pmatrix}
  0 &0&1&0\\
  0&0&0&0\\
  1 &0&0&0\\
  0 &0&0&0
\end{pmatrix}}_{\displaystyle =m_2}
= \underbrace{
\begin{pmatrix}
  0 &0  &1 &0\\
  0 &0  &0 &-1\\
  1 & 0 &0 &0\\
  0 &-1 &0 &0
\end{pmatrix}}_{\displaystyle =H_1}
+
\underbrace{
\begin{pmatrix}
  0&0&0&0\\
  0&0&0&1\\
  0&0&0&0\\
  0&1&0&0
\end{pmatrix}}_{\displaystyle =h\in\sH},
\]
so that $\pr H_1=\pr(m_2-h)=m_2$. Third,
\[
\underbrace{
\begin{pmatrix}
 0&0&0&1\\
 0\\
 0\\
 1
\end{pmatrix}}_{\displaystyle=m_3}
=
\underbrace{N-\frac{M+N}{2}}_{\displaystyle\in\iR}
-
\underbrace{%
\begin{pmatrix}
  0&0&0 &0\\
  0&0&0 &0\\
  0&0&0 &1\\
  0&0&-1&0\\
\end{pmatrix}}_{\displaystyle\in\sH},
\]
thus $\pr(N-\frac{M+N}{2})=m_3$. The last possibility in $\sQ$ is $m_i=E_{1i}+E_{i1}$ ($i\geq 5$), but
\[
  \underbrace{V_i}_{\displaystyle\in\iR}
    =\underbrace{E_{1i}+E_{i1}}_{\displaystyle =m_i}+\underbrace{E_{3i}-E_{i3}}_{\displaystyle\in\sH}.
\]

\subsection{Two other characterizations of the singularity}		\label{SubSecTwoCharSing}
%++++++++++++++++++++++++++++++++++++++++++++++++++++++++
 
In this short section, we first give a coordinatewise characterization of the singularity (which allows some brute force computations), and then we point out that the vector field $J_1^*$ has vanishing norm on the singularity (see also proposition \ref{PropAdSDeuxJannule}). That should make the connection with the quotient construction of the original BTZ black hole.  Notice that we do not classify all vectors from which vanishing of the norm define a singularity. The point is that one can make our black hole ``causally inextensible'' by making a discrete quotient of $AdS_l$ along the integral curves of $J^*_1$.

\begin{proposition}		\label{Proptcarrycarr}
In term of the embedding of $AdS_l$ in $\eR^{2,l-1}$, the closed orbits of $AN \subset \SO(2,l-1)$ are located at $y-t = 0$.  Similarly, the closed orbits of $A \bar{N}$ correspond to $y+t=0$. In other words, the equation
\begin{equation} \label{tcarrycarr}
t^2-y^2=0
 \end{equation}
describes the singularity $\hS=\hS_{AN}\cup\hS_{A\bar{N}}$.

More precisely, a point belongs to a closed orbit of $AN$ if and only if $t-y=0$ and to a closed orbit of $A\bar N$ if and only if $t+y=0$.
\end{proposition}

\begin{proof}
The different fundamental vector fields of the $AN$ action can be computed with the matricial relation $X^*_{[g]}=-Xg\cdot\mfo$. For example, in $AdS_3$,
\[
\begin{split}
   M^*_{[g]}&=
\begin{pmatrix}
0&-1&0&1\\
1&0&-1&0\\
0&-1&0&1\\
1&0&-1&0
\end{pmatrix}
\begin{pmatrix}
u\\t\\x\\y
\end{pmatrix}
=
\begin{pmatrix}
-t+y\\u-x\\-t+y\\u-x
\end{pmatrix}\\
&=(y-t)\partial_u+(u-x)\partial_t+(y-t)\partial_x+(u-x)\partial_y.
\end{split}
\]
Full results are
\begin{subequations}\label{Gen}
\begin{align}
J_1^*&=-y\partial_t-t\partial_y							\label{EqNormeJun}\\
J_2^*&=-x\partial_u-u\partial_x                                                      \label{eq:Jds}\\
M^*  &=(y-t)\partial_u+(u-x)\partial_t+(y-t)\partial_x+(u-x)\partial_y\\
L^*  &=(y-t)\partial_u+(u+x)\partial_t+(t-y)\partial_x+(u+x)\partial_y\\
W_i^*&=-x_i\partial_t-x_i\partial_y+(y-t)\partial_i\\
V_j^*&=-x_j\partial_u-x_j\partial_x+(x-u)\partial_j,
\label{eq:Vjs}
\end{align}
\end{subequations}
with $i,j=3,\ldots,l-1$.
First consider points satisfying $t-y=0$. It is clear that, at these points, the $l$ vectors $J_1^*$, $M^*$, $L^*$ and $W_i^*$ only span the direction $\partial_t+\partial_y$. Thus, there are at most $l-1$ linearly independent vectors amongst the $2(l-1)$ vectors \eqref{Gen}. We conclude that a point satisfying $t-y=0$ belongs to a closed orbit of $AN$.

Now we show that a point with $t-y\neq 0$ belongs to an open orbit of $AN$. It is easy to see that $J_1^*$, $M^*$ and $L^*$ are three linearly independent vectors. The vectors $V_i^*$ gives us $l-3$ more. Then they span a $l$-dimensional space.

The same can be done with the closed orbits of $A\bar{N}$. We have
\begin{subequations}
\begin{align}
	N^*	&=	-(y+t)\partial_u+(u-x)\partial_t-(y+t)\partial_x+(x-u)\partial_y\\
	F^*	&=	-(y+t)\partial_u+(x+y)\partial_t+(y+t)\partial_x-(x+u)\partial_y\\
	X^*_i	&=	-x_i\partial_u+x_i\partial_x-(x+u)\partial_i\\
	Y_j^*	&=	z\partial_t-z\partial_y+(y+t)\partial_i.
\end{align}
\end{subequations}
When $t+y=0$, the vectors $J_1^*$, $N^*$, $F^*$ and $Y_j^*$ only span the direction $\partial_t-\partial_y$. On the other hand, if $t+y\neq 0$, we look at the vectors $J_1^*$, $N^*$ and $F^*$. The vector $J_1^*$ is linearly independent of $N^*$ and $F^*$ because is does not contain a $\partial_u$ component. Now, the vector $N^*$ contains a component $\partial_u+\partial_x$ while $F^*$ contains $\partial_u-\partial_x$. We conclude that the vectors $J_1^*$, $N^*$ and $F^*$ span three linearly independent vectors. Thus a point with $t+y\neq 0$ belongs to an open orbit of $A\bar N$.

The result is that a point belongs to a closed orbit of $A\bar{N}$ if and only if $t+y=0$.
\end{proof}
This shows that in the three dimensional case, our black hole reduces to the previously existing one. 

The following corollary shows that a discrete quotient of $AdS_l$ along the orbits of $J_1^*$ gives a direct higher-dimensional generalization of the non-rotating BTZ black hole.
\begin{corollary}
The singularity coincides with the set of points in $AdS_l$ where $\| J_1^* \|^2 = 0$ for the metric induced from the ambient space $\eR^{2,l-1}$.
\label{CorJannsingul}
\end{corollary}

\begin{proof}
The expression \eqref{EqNormeJun} shows that the norm of $J_1^* $ is $y^2-t^2$ which vanishes on the singularity.
\end{proof}

In the three-dimensional case, it was shown in \cite{BTZ_deux,BTZB_un} that the non-rotating BTZ black hole singularity is precisely given by equation \eqref{tcarrycarr}. Hence, the following is a particular case of theorem \ref{ThoLeBut}:

\begin{corollary}
 The non-rotating BTZ black hole is a causal symmetric solvable black hole.
\end{corollary}
\subsection{A criterion with the tangent spaces}\label{subsec:R_z}
%-----------------------------------------------

Since $G$ acts transitively on $G/H$, the tangent spaces of $G/H$ at different points are not really independents: it is possible to guess global structure from consideration about tangent spaces. If $\mO$ denotes the orbit of $[z]$ under $R$, we have
\[
   T_{[z]}\mO=\Span\{X^*_{[z]}\tq X\in\sR\}.
\]
\begin{probleme}
C'est le lemme \ref{lem:equiv_1} pris en un autre point. Il faut trouver un argument pour voir que c'est correct.
\end{probleme}

We can work out the structure of the fundamentals vector fields\index{fundamental!vector field}:
 \begin{equation}
  X^*_{[z]}=\Dsdd{ \pi(e^{-tX}z) }{t}{0}
	   =(d\pi)_z(dR_z)_e\Dsdd{e^{-tX}}{t}{0}
	   =-(d\pi)_z\utX_z
\end{equation}
where $\utX$ denotes the right invariant vector field of $X\in\sG$. 

\begin{probleme}
Il y a presque certainement une faute de notation entre le tilde au-dessus et celui en-dessous.
\end{probleme}

If $\tau$ is the action of $G$ on $G/H$, we can try to bring the expression of $X^*_{[z]}$ in $T_{[e]}\mO$ in the following sense:
\begin{equation}
\begin{split}
(d\tau_{z^{-1}})_{[z]}X^*_{[z]}&=(d\tau_{z^{-1}})_{\pi(z)}(d\pi)_z\utX_z
                              =d(\tau_{z^{-1}}\circ \pi)_z\utX_z\\
			      &=\Dsdd{ \pi( z^{-1} e^{-tX}z ) }{t}{0}
			      =(d\pi)_e\Ad(z^{-1})X.
\end{split}
\end{equation}
Now we define the space\nomenclature{$\sR_z$}{Trick to compute open orbits}
\begin{equation}
\sR_z=(d\pi)_e\Ad(z^{-1})\sR,
\end{equation}
and we can state a necessary condition for two points to belongs to the same orbit.

\begin{proposition}
If the elements $z$ and $z'$ of $G$ are related by $r\in R$ (i.e. $z'=rz$), then $\sR_{z'}=\sR_z$.
\end{proposition}

\begin{proof}
If is just a computation. Let $z'=rz$; we have
\begin{equation}
\begin{split}
\sR_{z'}=(d\pi)_e\Ad(z'\,\!^{-1})\sR
        =(d\pi)_e\Ad(z^{-1} r^{-1})\sR
	=(d\pi)_e\Ad(z^{-1})\Ad(r^{-1})\sR
	=\sR_z
\end{split}
\end{equation}
because $\Ad(r^{-1})\sR=\sR$.
\end{proof}

Taking the general form \eqref{eq:geneR} of an element in $\sR$, we compute
\begin{equation}
\begin{split}
\Ad(z)\sR&=
\begin{pmatrix}
\cos\mu & \sin\mu\\
-\sin\mu & \cos\mu\\
&&\mtu
\end{pmatrix}
\begin{pmatrix}
0&x&k&-x\\
-x&0&0&j\\
k&0&0&0\\
-x&j&0&0
\end{pmatrix}
\begin{pmatrix}
\cos\mu & -\sin\mu\\
\sin\mu & \cos\mu\\
&&\mtu
\end{pmatrix}
\\
&\simeq
\begin{pmatrix}
0& x&k\cos\mu &-x\cos\mu+j\sin\mu\\
-x\\
k\cos\mu\\
-c\cos\mu+j\sin\mu
\end{pmatrix}
\end{split}
\end{equation}
where $\simeq$ stand for ``equals up to a matrix of $\sH$''. If $\cos\mu=0$, then $\sR_z\neq\sR$. This shows that 
\begin{equation}
\begin{pmatrix}
0&1\\
-1&0\\
&&1\\
&&&1
\end{pmatrix}
\text{ and }
\begin{pmatrix}
0&-1\\
1&0\\
&&1\\
&&&1
\end{pmatrix}
\end{equation}
does not belong to the orbit of $\mfo$. We also see that $\cos \mu=-1$ is either not in the orbit of $\mfo$.

\subsubsection{Search for open orbits}
%/////////////////////////////////////

We consider the group $R_z=\AD(z)R=zRz^{-1}$ for some $z\in \SO(2)$. The openness of the $R_z$-orbit of $\mfo$ is the same as the one of the $R$-orbit of $[z^{-1}]$. Matrices of $\sR_z=z\sR z^{-1}$ are easy to find. Here are the projections on $\sQ$: 
\begin{subequations}
\begin{align}
\pr_{\sQ} M_z&=
\begin{pmatrix}
0&1&\sin\mu&-\cos\mu\\
-1\\
\sin\mu\\
-\cos\mu
\end{pmatrix}
&\pr_{\sQ} L_z&=
\begin{pmatrix}
0&1&-\sin\mu&-\cos\mu\\
-1\\-\sin\mu\\
-\cos\mu
\end{pmatrix}\\
 \pr_{\sQ} {J_1}_z&=
\begin{pmatrix}
0&0&0&\sin\mu&\\
0\\
0\\
\sin\mu
\end{pmatrix}
&\pr_{\sQ} {J_2}_z&=
\begin{pmatrix}
0&0&\cos\mu&0\\
0\\
\cos\mu\\
0
\end{pmatrix}\\
 \pr_{\sQ} {W_i}_z&=\sin\mu(E_{i1}+E_{1i})&\pr_{\sQ} {V_i}_z&=\cos\mu(E_{i1}+E_{1i})
\end{align}
\end{subequations}
When $\sin\mu$ and $\cos\mu$ are non-zero, we have
\begin{subequations}
\begin{align}
q_0&=\frac{1}{2}\big(M_z+L_z+\frac{\cos\mu}{\sin\mu}{J_1}_z\big)&q_1&=\us{\cos\mu}{J_2}_z\\
q_2&=\us{\sin\mu}{J_1}_z&q_i&=\us{\sin\mu}{W_i}_z=\us{\cos\mu}{V_i}_z
\end{align}
\end{subequations}
 So when $\sin\mu=0$, the element $q_{0}$ does not belong to $\pr_{\sQ}\sR_{z}$. Hence the $R_z$-orbit of $\mfo$ is non open if and only if $\sin \mu=0$.

\subsection{Orbits  and topology}
%--------------------------------
\label{PgTopoOrb}

Let  $D^{\pm}=AN\SO(n)\SO(2)^{\pm}$ where $\SO(2)^{\pm}$ are the subgroups of $\SO(2)\subset \SO(2,n)$ with strictly positive (negative) cosine. We see $\SO(2)$ and $\SO(n)$ as subgroups of $\SO(2,n)$ in the way indicated by equation \eqref{eq:K_H_SO}. Notice that the parts $\SO(2)$ and $\SO(n)$ are commuting and that $\SO(n)\subset H$. The notation $-\mtu_{\SO(2)}$ refers to the element of $\SO(2,n)$ which the identity as $AN$-component and $-\mtu$ as $\SO(2)$-component.

A continuous path from $[D^+]$ to $[D^-]$ must pass trough an element of the form $[AN\mtu_{\SO(2)}]$. We saw that the $AN$-orbit of such an element is not open while the $AN$-orbit of an element of $[D^+]$ is open. So we deduce that an orbit passing trough $[D^+]$ does not intersect $[D^-]$.

The set $[D^+]$ is connected in $G/H$ and $D^+$  being open in $G$, the set $[D^+]=\pi(D^+)$ is also open in $G/H$ from the definition of the topology (see theorem \ref{tho:struc_anal}). Now, the orbits of $AN$ in $[D^+]$ (who are all open) furnish an open partition of $[D^+]$. Such a partition is impossible for an open connected set. We deduce that $[D^+]$ is only one orbit of $AN$ in $G/H$. The same can be done with $[D^-]$.

We are left with the sets $[AN]$ and $[AN(-\mtu_{\SO(2)})]$ whose union is closed because we just saw that the complement is open. Now we prove that these two sets are disjoint, in such a way that they have to be separately closed. Existence of an intersection point between $[AN]$ and $[AN(-\mtu_{\SO(2)})]$ would lead to the existence of a $h\in H$ such that $an\mtu_{\SO(2)}=(-\mtu_{\SO(2)})h$, or
\[ 
  h=(-\mtu_{\SO(2)})an,
\]
that is a non trivial $K$-component to $h$ in the decomposition $KAN$, but the only $K$-component in $H$ is $\SO(n)$. Hence such a $h$ does not exist and $R[\mtu]\cap R[-\mtu_{\SO(2)}]=\emptyset$.

The conclusion is that the Iwasawa group $AN$ has only four orbits :
\begin{align}
[D^+],&&[D^-],&&[AN\mtu_{\SO(2)}],&&[AN(-\mtu_{\SO(2)})].
\end{align}
The two first are open and the other two are closed. Remark\label{PgNoticeKpassung} that an element of $[K]$ does not belong to a closed orbit of $AN$ or $A\bar N$.


\subsection{The volume form method}    \label{subsecVolumeForm}
%-----------------------------------

Let us give an alternative to proposition \ref{tho:pr_ouvert} to study the openness of an $AN$-orbit. We explain the method for $\hS_{AN}$, but the same with trivial adaptations is true for $\hS_{A\bar{N}}$.

If $x\in M$ belongs to $\hS_{AN}$, the tangent space of its $AN$-orbit has lower dimension than the tangent space of $M$.  In this case the volume spanned by the fundamental vectors at $x$ is zero.  The idea is to build the volume form $\nu_x$ of $T_xM$ and then apply it on a basis of the fundamental fields.  If the result is zero, then $x$ belongs to the $\hS_{AN}$.  More precisely, the action is given by
		\begin{equation}
		\begin{aligned}
			\tau \colon AN\times M &\to M\
			(an,[g])&\mapsto [ang].
		\end{aligned}
	\end{equation}	
If $X\in\sA\oplus\sN$ and $[g]\in M$, then
\begin{equation}
  X^*_{[g]}=-d(\pi\circ R_g)X.
\end{equation}
As mentioned in corollary \ref{Cordpiietwii}, if $\{q_i\}$ is a basis of $\sQ$ then a basis of $T_{[g]}M$ is given by $\{d\pi dL_gq_i\}$. We define
\[
\nu=q_0^{\flat}\wedge q_1^{\flat}\wedge \ldots \wedge
q_{l-1}^{\flat}
\]
where $q_{i[g]}^{\flat}=B_{[g]}(d\pi dL_g q_i,\cdot)$. The condition for $[g]$ to belongs to $\hS_{AN}$ reads
\begin{equation}\label{eq:nusurN}
\nu_{[g]}(N_1^*{}_{[g]},N_2^*{}_{[g]},\ldots,N_l^*{}_{[g]})=0
\end{equation}
for every choices of $N_j$ in a basis of $\sA\oplus\sN$. It corresponds to the vanishing of $l \times l$ determinants. Our purpose is now to compute the products
\[
\begin{split}
  B_{[g]}(d\pi dL_g q_i,N^*_j{}_{[h]})	&=-B_g(\pr dL_g q_i,\pr dR_g N_j)\\
					&=-B_g(dL_g q_i,dR_g N_j)\\
					&=-B_e(q_i,\Ad(g^{-1})N_j).
\end{split}
\]
where $\pr\colon T_{g}M\to dL_{g}\sQ$ is the projection. The step from the first to the second line is as follows. First, $\pr dL_gq_i=dL_gq_i$ by definition. For the second, let us write $dR_g X=dL_g X_h+dL_g X_q$ with $X_h\in\sH$ and $X_q\in\sQ$. From equations \eqref{EqDefRedHQ}, we see that $B(\sQ,\sH)=0$, so $B(dL_g q_i,dL_g X_h+dL_g X_q)=B(dL_g q_i,dL_g X_q)$. Remark that one cannot do it computing $\|J_i^*\|$.

We consider the quantity 
\[
\Delta_{ij}([g])=B(q_i,\Ad(g^{-1})N_j)
\]
where $N_j$ runs over a basis of $\sA\oplus\sN$ and $q_i$ a one of $\sQ$. Our problem of light cone (see explanations in section \ref{SecCausal}) leads us to compute
 \begin{equation} \label{eq:elemtr}
\Delta_{ij}(\pi(ge^{-tk\cdot E}))=B(\Ad(e^{-tk\cdot
E})q_i,\Ad(g^{-1})N_j)
\end{equation}
where $k\cdot E$ is a notation for $\Ad(k)E$.

A way to proceed is, following proposition \ref{prop:enuc},  to express all our elements of $\so(2,n)$ in the root space decomposition
\[
\sG=\sG_{(0,0)}\bigoplus_{\lambda\in\Sigma}\sG_{\lambda}.
\]
The purpose of that resides in the fact that the Killing form $B(X,Y)$ is easier to compute when $X$ and $Y$ belongs to some root spaces.

An important computational remark is the fact that $E$ is nilpotent, so $\Ad(k)E$ also is and $\Ad(e^{-t\Ad(k)E})X=e^{-t\ad(k)E}X$ only gives second order expressions with respect to $t$. These computations are nevertheless heavy, but can fortunately be circumvented by a simple counting of dimensions, as we describe in proposition \ref{Proptcarrycarr}.

Let us make some computations now. In a first time, we restrict ourself to elements in $K$: we put $g=e^{uR}$ with
\[
R=
\begin{pmatrix}
0&1\\
-1&0\\
&&0\\
&&&0
\end{pmatrix}\in\sK.
\]
On the other hand, an useful way to express $k\cdot E_1$ is the following (cf. equations \eqref{eq:Adkeu} ):
\[
\Ad(k)E_1=
\begin{pmatrix}
0&1&w_1&w_2&w_3\\
-1\\
w_1\\
w_2\\
w_3
\end{pmatrix}
=q_0+w_1q_1+w_2q_2+w_3q_3\in \sQ.
\]
It should be noted that by choosing $k$, all the vectors $\begin{pmatrix}w_1&w_2&w_3\end{pmatrix}$ with $\|w\|^2=1$ are possible.

Let us begin by systematically computing the elements $[k\cdot E_1,q_i]$ and $[k\cdot E_1,[k\cdot E_1,q_i]]$; the others $\ad(k\cdot E_1)^nq_i$ are zero because one can see that $\ad(E_1)^3X_{\alpha}=0$ for all $X_{\alpha}$ in the root spaces. All computations can be performed by decomposing the $q_i$'s in the root space basis and using the known commutations relations between root spaces. The way we choose here is to directly use the huge formula
\begin{equation}
\begin{split}
[k\cdot E_1,[k\cdot E_1]]&=w_1^2[q_1,[q_1,q_i]]\\
                         &\quad +w_1w_2\big(  [q_1,[q_2,q_i]]+[q_2,[q_1,q_i]]  \big)\\
                         &\quad +w_1w_3\big(  [q_1,[q_3,q_i]]+[q_3,[q_1,q_i]]  \big)\\
                         &\quad +w_2^2[q_2,[q_2,q_i]]\\
                         &\quad +w_2w_3\big(  [q_2,[q_3,q_i]]+[q_3,[q_2,q_i]]  \big)\\
                         &\quad +w_3^2[q_3,[q_3,q_i]]\\
\end{split}
\end{equation}
and use the commutations relations between the $q_i$'s. The results are
\begin{equation}
\begin{split}
\ad(k\cdot E_1)q_0 &=\frac{w_1}{4}(M+N-L-F)+w_2J_1+\frac{w_3}{2}(W-Y)\\
\ad(k\cdot E_1)q_1 &=\frac{1}{4}(L+F-M-N)+\frac{w_2}{4}(F+M-L-M)-\frac{w_3}{2}(V+X)\\
\ad(k\cdot E_1)q_2 &=J_1+\frac{w_1}{4}(L+N-F-M)-\frac{w_3}{2}(W+Y)\\
\ad(k\cdot E_1)q_3 &=\frac{1}{2}(Y-W)+\frac{w_1}{2}(V+X)+\frac{w_2}{2}(W+Y)\\
\end{split}
\end{equation}
and
\begin{equation}
\begin{split}
\ad(k\cdot E_1)^2q_0 &=k\cdot E_1\\
\ad(k\cdot E_1)^2q_1 &=-w_1q_0+(w_2^2+w_3^2-1)q_1-w_1w_2q_2-w_1w_3q_3\\
\ad(k\cdot E_1)^2q_2 &=-w_2q_0-w_1w_2q_1+(w_1^2+w_3^2-1)q_2-w_2w_3q_3\\
\ad(k\cdot E_1)^2q_3 &=-w_3q_0-w_1w_3q_1-w_2w_3q_2+(w_1^2+w_2^2-1)q_3\\
\end{split}
\end{equation}
It is rather easy to check that $\ad(k\cdot E_1)^3q_i=0$ by virtue of $\|w\|^2=0$. All these expressions have to be extended in the basis of the root spaces.
\begin{equation}
\begin{split}
\ad(k\cdot E_1)^2q_0 &=\frac{1}{4}(M+N+L+F)+\frac{w_2}{4}(N+F-M-L)\\
                     &\quad +\frac{w_3}{2}(V-X)+w_1q_1\\
\ad(k\cdot E_1)^2q_1 &=-\frac{w_1}{4}(M+N+L+F)+ \frac{w_1w_2}{4}(M+L-N-F)\\
                     &\quad +\frac{w_1w_3}{2}(X-V) +(w_2^2+w_3^2-1)q_1\\
\ad(k\cdot E_1)^2q_2 &=-\frac{w_2}{4}(M+N+L+F)+\frac{w_1^2+w_3^2-1}{4}(N+F-M-L)\\
                     &\quad +\frac{w_2w_3}{2} (X-V)-w_1w_2q_1\\
\ad(k\cdot E_1)^2q_3 &= -\frac{w_3}{4}(M+N+L+F) +\frac{w_2w_3}{4}(M+L-N-F)\\
                     &\quad +\frac{w_1^2+w_2^2-1}{2}(V-X)-w_1w_3q_1
\end{split}
\end{equation}


\subsubsection{The column of \texorpdfstring{$V$}{V}}
%///////////////////////////////

An explicit computation shows that
\begin{equation}
\begin{split}
\Ad(e^{uR})V&=
\begin{pmatrix}
&&&&\cos u\\
&&&&-\sin u\\
&&&&1\\
&&&&0\\
\cos u&-\sin u&-1&0&0
\end{pmatrix}\\
  &=\frac{1}{2}(1-\cos u)X+\frac{1}{2}(\sin u) Y\\
  &\quad+\frac{1}{2}(1+\cos u)V-\frac{1}{2}(\sin u) W.
\end{split}
\end{equation}

\begin{remark}
Because of the invert in \eqref{eq:elemtr}, we are looking at the destiny of the point $[e^{-uR}]$, not the one of $[e^{uR}]$.
\end{remark}

Thanks to the properties of the root space decomposition, we know that the only non zero Killing form containing $X,Y,V,W$ are $B(W,Y)$ and $B(V,X)$. So in the expression
\[
\Ad(k\cdot E_1)q_0=q_0+\frac{tw_1}{4}(N+M+L+F)+tw_2J_1+\frac{tw_3}{2}(W-Y)+\frac{t^2w_3}{2}(V-X),
\]
we can forget the three first terms when we compute $\Delta_{q_0,V}$. The result is
\begin{equation}
\boxed{\Delta_{q_0,V}=B(W,Y)\frac{tw_3}{2}\sin u-B(V,X)\frac{t^2w_3}{4}\cos u}
\end{equation}
In the same way,
\begin{equation}
\boxed{\Delta_{q_1,V}=-B(V,X)\left( \frac{tw_3}{2}+\frac{t^2w_2w_3}{4} \right)},
\end{equation}
\begin{equation}
 \boxed{ \Delta_{q_2,V}=B(X,V)\frac{t^2w_2w_3}{4}\cos u },
\end{equation}
\begin{equation}
\boxed{\Delta_{q_3,V}=-B(V,X)\frac{1}{2}\big(  \cos u-tw_1+\frac{t^2}{2}(w_1^2+w_2^2-1)\cos u  \big)-B(W,Y)\frac{t}{2}\sin u}
\end{equation}
Remark that the only term in this column which doesn't vanishes when $t=0$ contains $\cos u$.

\subsubsection{The column of \texorpdfstring{$J_1$}{J1}}
%//////////////////////////////////

\begin{probleme}
C'est justement un de ceux que tu soup\c connes de ne servir \`a rien.
\end{probleme}

A direct computation shows that 
\begin{equation}
\begin{split}
\Ad(e^{uR})J_1&=\sin(u) q_2+\cos(u) J_1\\
              &=\us{4}\sin(u)(N+F-M-L)+\cos(u) J_1.
\end{split}
\end{equation}
We only have to consider the non zero Killing form $B(J_1,J_1)$, $B(W,Y)$, $B(V,X)$, $B(N,L)$, $B(M,F)$.
\begin{equation}
\boxed{\Delta_{q_0,J_1}=6t^2w_2\sin u+6tw_2\cos u}
\end{equation}

\subsubsection{The column of \texorpdfstring{$J_2$}{J2}}
%//////////////////////////////////

For the computation of $\Ad(e^{uR})J_2$, we recall that $R=q_0$ and $J_2=q_1$. It is easy to see that $[q_0,q_1]=\us{4}(L+F-M-N)$ and $[q_0,[q_0,q_1]]=-q_1$, so that the exponential series looks good and gives
\[
  \Ad(e^{uR})q_1= \cos(u)q_1+\frac{\sin u}{4}(L+F-M-N).
\]
A lot of computation gives
\begin{equation}
\boxed{\Delta_{q_0,J_2}=3t^2w_1\cos u-6tw_1\sin u}
\end{equation}



\begin{equation}
\boxed{\Delta_{q_1,J_2}=-3t^2w_1\cos u+6t\sin u+6\cos u}
\end{equation}



\begin{equation}
\boxed{\Delta_{q_2,J_2}=-3t^2w_1w_2\cos u}
\end{equation}


\begin{equation}
\boxed{\Delta_{q_3,J_2}=-3t^2w_1w_3\cos u}
\end{equation}


\subsubsection{The column of \texorpdfstring{$M$}{M}}
%///////////////////////////////

The first computation is 
\[
  \Ad(e^{uR})M=\frac{1-\cos u}{2}(F-M)+\sin(u)(q_1+J_1)+M.
\]

\begin{equation}
\boxed{\Delta_{q_0,M}=6(1+w_1\sin u)+6t(w_1(1-\cos u)+w_2\sin u)+3t^2(1+w_2\cos u).
}
\end{equation}

\begin{align}
\Delta_{q_1,M}=B
\Big(&
  q_1+\frac{t}{4}(F-M)+\frac{tw_2}{4}(F-M)\\
          &+\frac{t^2}{2}
      \big[
            -\frac{w_1}{4}(F+M)+\frac{w_1w_2}{4}(M-F)-w_1^2q_1
      \big],\\
  &\frac{1}{2}(1-\cos u)(F-M)+\sin u(q_1+J_1)+M
\Big).
\end{align}
Collecting the terms and using the following relations,
\begin{subequations}
\begin{align}
B(M+F,F-M)&=0&B(M-F,F-M)&=3\cdot 16\\
B(F-M,M)&=3\cdot 8&B(F+M,M)&=3\cdot 8.
\end{align}
\end{subequations}
we find
\begin{equation}
\boxed{\Delta_{q_1,M}=6\sin u-6t(2-\cos u)(1+w_2)+3t^2(w_1+w_1w_2\cos u-w_2^2\sin u)}
\end{equation}

\begin{equation}
\boxed{%
\begin{aligned}
\Delta_{q_2,M}=-6(2-\cos u)&+6t(\sin u+w_1)\\&+3t^2\big(
-w_2+\frac{w_2^2}{2}(1-\cos u)-w_1w_2\sin u
\big)
\end{aligned}
}
\end{equation}

\begin{equation}
\boxed
{
  \Delta_{q_3,M}=3t^2w_3(1+w_2(2-\cos u)-w_1\sin u).
}
\end{equation}

\subsubsection{Existence for \texorpdfstring{$AdS_3$}{AdS3}}
%////////////////////////////////////

From computer computations, the (non identically zero) volume determinants are given by
\begin{subequations}
\begin{align}
  &-32(tw_2+t\cos u-\sin u)^2\big(\cos u+t(\sin u-w_1)\big)\\
  &-32(tw_2+t\cos u-\sin u)^9\\
\begin{split}
16t^2w_3\Big(&w_2(w_1-\sin u)+\cos u(w_1-\sin u)-w_2\cos u\\
	&-\cos^2u+\sin u(-w_1+\sin u)
\Big)+16tw_3\cos u\sin u
\end{split}\\
&16tw_3
\big(
 -\cos u+t(w_1-\sin u)
\big)
(tw_2+t\cos u-\sin u)
\end{align}
\end{subequations}


One can deduce the existence of an horizon. Indeed the vanishing of all the determinants for a point in $[\SO(2)]$ with respect to the $AN$ singularity only requires 
\begin{subequations} \label{eq:annul_trois}
\begin{align}
t_{AN}=\frac{\sin u}{\cos u-\sin k}
\intertext{while the same for $A \overline{N}$ requires}
   t_{A\overline{N}}=\frac{\sin u}{\sin k+\cos u}
\end{align}
\end{subequations}
The (class of the) point $u$ belongs to the black hole if for all $k\in \SO(2)$, $t_{AN}>0$ or $t_{A\overline{N}}>0$. In this case, all light-like geodesic from the point $u$ fall into the hole after a positive time. There are two possibilities  :
\begin{subequations}
\begin{align}
\begin{split}
\sin u<0\\
\cos u<0
\end{split}\\
\intertext{or}
\begin{split}
\sin u>0\\
\cos u>0
\end{split}
\end{align}
\end{subequations}
Let us insist to the fact that the points $u=0$ and $u=\pi$ are not in the horizon although they separate black points and free points. These two points belongs to the singularity. In fact the spaces $\sin u\geq0$ and $\sin u \leq0$ are two completely separated spaces.

So in the space $\sin u\geq 0$, the point $u=\pi/2$ is part of the horizon. This proves the existence of an horizon and gives one point of it. The determination of the horizon is not likely easy.

\subsubsection{Existence for \texorpdfstring{$AdS_4$}{AdS4}}
%///////////////////////////////////

One can parametrize $\Ad(k)E_1$ as
\begin{equation}
\Ad(k)E_1=
\begin{pmatrix}
0&1&w_1&w_2&w_3\\
-1\\
w_1\\
w_2\\
w_3
\end{pmatrix}.
\end{equation}
The volume forms for the $AN$ and $A \overline{N}$ orbits are respectively annihilated by
\begin{equation}
t_{AN}=\frac{\sin u}{\cos u+w_2}, \text{ and } t_{A \overline{N}}=\frac{\sin u}{\cos u-w_2}.
\end{equation}
These are the same as \eqref{eq:annul_trois}. Once again the doomed part of the space is given by
\begin{subequations}
\begin{align}
\begin{split}
\sin u<0\\
\cos u<0
\end{split}\\
\intertext{or}
\begin{split}  \label{eq:possdeux}
\sin u>0\\
\cos u>0
\end{split}
\end{align}
\end{subequations}
For example in the case \eqref{eq:possdeux}, the directions with $\cos u<w_2<-\cos u$ escape the singularity.
\section{Existence of a non trivial horizon}		\label{SecExistenceHor}
%++++++++++++++++++++++++++++++++++++++++++++

We are now able to prove that definition \ref{Singular} provides a non empty horizon satisfying condition \eqref{EqhSssubBH}.  First we  consider points of the form $\SO(2)\cdot\mfo$, which are parametrized by an angle $\mu$. By lemma \ref{LemGeodGenreLumiere}, up to the choice of this parametrization, a light-like geodesic trough $\mu$ is given by
 \begin{equation}
   K\cdot \mbox{e}^{-s\Ad(k)E_1}\cdot\mfo
\end{equation}
with $k\in \SO(l-1)$ and  $s\in\eR$. Using the isomorphism $[g]\mapsto g\cdot \mfo$ between $G/H$ and $AdS_l$, we find
\begin{equation}		\label{EqhohnCondHOrExpl}
  l^k_{[u]}(s)= \pi\big( u e^{s\Ad(k)E_1} \big)=
\begin{pmatrix}
\cos\mu&\sin\mu\\
-\sin\mu&\cos\mu\\
&&1\\
&&&1\\
&&&&1\\
&&&&&\ddots
\end{pmatrix}
 e^{s\Ad(k)E_1}
\begin{pmatrix}
1\\0\\0\\0\\0\\\vdots
\end{pmatrix}
=
\begin{pmatrix}
u_{k}(s)\\t_{k}(s)\\x_{k}(s)\\y_{k}(s)\\z_{k}(s)\\\vdots
\end{pmatrix}
\end{equation}
According to proposition \ref{Proptcarrycarr}, this geodesic reaches the singularity in the future if $t_{k}(s)^{2}-y_{k}(s)^{2}=0$ for a certain positive $s$. Since $\Ad(k)E_1$ is nilpotent, the computation of $ e^{s\Ad(k)E_1}$ is simple and we only need the first column because it only acts on the first basis vector. A short computation shows that
\begin{equation}  \label{EqGedCompo}
  l_{[\mu]}^{k}(s)=
\begin{pmatrix}
\cos\mu-s\sin\mu\\
-\sin\mu-s\cos\mu\\
sw_{1}\\
sw_{2}\\
\vdots
\end{pmatrix}.
\end{equation}

We used the computation
\[
  e^{s\Ad(k)E_1}=\mtu+s
\begin{pmatrix}
0&1&w_1&w_2&w_3&\cdots\\
-1\\w_1\\w_2\\w_3\\\vdots
\end{pmatrix}
+\frac{s^2}{2}
\begin{pmatrix}
0&0&0&0&0&\cdots\\
0&-1&-w_1&-w_2&-w_3&\cdots\\
0&w_1&w_1w_1&w_1w_2&w_1w_3&\cdots\\
0&w_2&w_2w_1&w_2w_2&w_2w_3&\cdots\\
\vdots&\vdots&\vdots&\vdots&\vdots
\end{pmatrix}
+\cdots
\]
Notice that the sum if finite because $E_1$ is nilpotent. However, the first power of $E_1$ which vanishes depends on the dimension.

We conclude that the geodesic reaches $\hS_{AN}$ and $\hS_{A\bar{N}}$ for values $s_{AN}$ and $s_{A\bar{N}}$ of the affine parameter, given by
\begin{align}   \label{eq:tempssingul}
 s_{AN}&= \frac{\sin\mu}{\cos\mu - w_2}&s_{A\bar{N}}&= \frac{\sin\mu}{\cos\mu + w_2}
\end{align}
where $w_{2}$ is the second component of the first column of $k$, see equation \eqref{eq:AdkE}; in particular $-1\leq w_2 \leq 1$.

Since the part $\sin \mu =0$ is precisely  $\hS_{AN}$, we may restrict ourselves to the open connected domain of $AdS_l$ given by $\sin \mu > 0$. More precisely, $\sin\mu=0$ is the equation of $\hS_{AN}$ in the $ANK$ decomposition. In the same way, $\hS_{A\bar{N}}$ is given by $\sin\mu'=0$ in the $A\bar{N}K$ decomposition.  In order to escape the singularity, the point $[\mu]$ needs both $s_{AN}$ and $s_{A\bar{N}}$ to be strictly positive.  It is only possible to find directions (i.e. a parameter $w_2$) which respects this condition when $\cos \mu>0$.  So the point
\begin{equation}  \label{EqUnPtHoriz}
u\equiv \cos\mu=0
\end{equation}
is one point of the horizon. Theorem \ref{ThoLeBut} is now proved. Remark that the two-dimensional case here appears as degenerate. Therefore, it is treated later in section \ref{SecAdS2}, where we show that \emph{no black hole arises from this construction in $AdS_2$}.

The following proposition contains some physical intuition about the nature of the horizon.

\begin{proposition}
A light-like geodesic which escapes the singularity (i.e. which does not intersect $\hS$) and which passes trough a point of the horizon is contained in the horizon.
\end{proposition}

\begin{proof}
Let $x=[g]$ be a point of the horizon and $\pi(ge^{tAd(k)E_1})$, a light-like geodesic escaping the singularity. Near from $x$, there exists a point $y=[g']$ in the black hole. From definition of a black hole, for all $k\in \SO(3)$ and $t_{0}\in\eR^{+}$, points of the form  $\pi(g'e^{t_0Ad(k)E_1})$ also belong to the black hole. From continuity, in each neighbourhood of $\pi(ge^{t_0Ad(k)E_1})$, there is such a $\pi(g'e^{t_0Ad(k)E_1})$. This proves that $\pi(ge^{t_0Ad(k)E_1})$ belongs to the closure of the black hole. But it is not in the interior of the black hole because (by assumption) the given geodesic escapes the singularity, so every point of the form $\pi\big( g e^{t_0\Ad(k)E_1} \big)$ belongs to the horizon.
\end{proof}

\begin{proposition}		\label{PropTNFerme}
The set $BH_l\setminus\hS_l$ is open.
\end{proposition}

\begin{proof}
A point $v\in AdS_l$ belongs to $BH_l\setminus\hS_l$ if and only if all  the solutions in $s$ of the equation
\begin{equation}
	(T\pm Y)\big( v e^{s\Ad(k)E_1} \big)\in\hS_l
\end{equation}
are strictly positive (and non infinite). The \emph{strict} is due to the fact that we excluded $\hS_l$ itself. Let $s_{\pm(v,k)}$ be these solutions for the point $v\in AdS_l$ and the direction $k\in S^l$. Let now consider $v_0\in BH_l\setminus\hS_l$. The function $s_{\pm}(v_0,.)\colon S^l\to \eR$ is a continuous function on the compact set $S^l$, thus its image is a compact subset of $\eR_0^+$, because the function reach its extrema.

The function $v\mapsto s_{\pm}(v,k)$ is also continuous, so that, if $\epsilon$ is small enough, and if $v\in B(v_0,\epsilon)$, the image of $s_{\pm}(v,.)$ is still a compact subset of $\eR_0^+$. That means that, from the point $v$, every light-like geodesic intersect the singularity within a finite strictly positive time, this is the fact that $v\in BH_l\setminus\hS_l$.
\end{proof}

\begin{corollary}		\label{CorTNFermeHorEchape}
The set of free points in $AdS_l$ is closed and the points on the horizon do have at least one direction which escape the singularity.
\end{corollary}

Let us consider the point of the horizon that we know (the one given by \eqref{EqUnPtHoriz}), and see how can that point hope to escape the singularity.  Equations \eqref{eq:tempssingul} which give the time needed to fall into the singularity become
\begin{align}
  t_{AN}&=\frac{1}{w_{2}}&t_{A \bar{N}}&=-\frac{1}{w_{2}}.
\end{align}
So for every $w_{2}\neq 0$, this point reaches the singularity within a finite time. Taking the direction $w_{2}=0$ the point is able to reject his fall to infinity. This agrees to physical intuition which is that the horizon corresponds to points that fall into the singularity within an infinite time.

Up to a reparametrization of $\SO(n)$, the safe directions are given by (equation \eqref{eq:AdkE} with $w_2=0$)
\[
   \Ad(k)E_1=
\begin{pmatrix}
0&1&\cos a&0&\sin a\\
-1\\
\cos a\\
0\\
\sin a
\end{pmatrix}.
\]
A direct  computation of equation \eqref{EqGedCompo}  shows that the points of the horizon that are joined by this way are given by
$
\begin{pmatrix}
-1\\
0\\
\cos a\\
0\\
\sin a
\end{pmatrix}.
$
