% This is part of the Exercices et corrigés de mathématique générale.
% Copyright (C) 2009-2011
%   Laurent Claessens
% See the file fdl-1.3.txt for copying conditions.
\begin{corrige}{General0010}

\begin{enumerate}

\item
Le numérateur et le dénominateur s'annulent en $x=2$, donc nous pouvons certainement simplifier la fraction par $(x-2)$. En effet, une simple factorisation du second degré nous montre que $x^2+x-6=(x-2)(x+3)$, tandis que le dénominateur est un produit remarquable : $x^2-4=(x+2)(x-2)$. Il reste donc
\begin{equation}
	\lim_{x\to 2} \frac{ (x-2)(x+3) }{ (x-2)(x+2) }=\lim_{x\to 2} \frac{ x+3 }{ x+2 }=\frac{ 5 }{ 4 }.
\end{equation}

\item
En utilisant la règle de l'Hospital, nous trouvons
\begin{equation}
	\lim_{x\to \frac{ \pi }{ 4 }} \frac{ 1-\tan(x) }{ \cos(2x) }=\lim_{x\to \frac{ \pi }{ 4 }} \frac{ -1/\cos^2(x) }{ -2\sin(2x) }\lim_{x\to \frac{ \pi }{ 4 }} \frac{ 1 }{ 2\sin(2x)\cos^2(x) }=1.
\end{equation}

\item
\item
\item
Lorsqu'on voit la fonction $x/\sqrt{1+x^2}$, il y a moyen de facilement deviner la solution. En effet, lorsque $x$ est très grand, le $1$ à côté du $x^2$ ne vaut pas grand chose, et on peut remplacer $1+x^2$ par $x^2$. Le dénominateur devient donc $\sqrt{x^2}=| x |$, et la limite est $\pm 1$.

Affin de calculer plus précisément, utilisons la règle de l'Hospital :
\begin{equation}
	l=\lim_{x\to \pm\infty} \frac{ x }{ \sqrt{1+x^2} }=\lim_{x\to \pm\infty} \frac{1}{ x\frac{ 1 }{ \sqrt{1+x^2} } }=\lim_{x\to \pm\infty} \frac{ \sqrt{1+x^2} }{ x }.
\end{equation}
À ce moment, nous avons l'impression de tourner en rond : nous sommes tombés exactement sur l'inverse de la fonction de départ. En réalité, ce que nous avons prouvé, c'est que $l=1/l$, ce qui n'est possible qu'avec $l=1$ ou $l=-1$.

Il faut encore déterminer si la limite est $1$ ou $-1$ (en l'infini et en moins l'infini). Rien qu'en regardant la fonction, nous voyons qu'elle est positive pour les $x>0$, donc sa limite doit être $+1$ en $+\infty$. De la même manière, la fonction est négative pour $x<0$, donc sa limite en $-\infty$ doit être $-\infty$.

\item
Pour rappel : $(a^x)'=a^x\ln(x)$. En sachant cela, la règle de l'Hospital montre que
\begin{equation}
	\lim_{x\to 0} \frac{ a^x-b^x }{ \sin(x) }=\lim_{x\to 0} \frac{ a^x\ln(a)-b^x\ln(b) }{ \cos(x) }=\ln(a)-\ln(b).
\end{equation}

\item
\item
\item
\item
\item
Règle de l'Hospital immédiate :
\begin{equation}
	\lim_{x\to 0} \frac{ 1-\cos(x) }{ \sin(x) }=\lim_{x\to 0} \frac{ \sin(x) }{ \cos(x) }=0.
\end{equation}

\item
\item
\item

On a un produit de type $0\times\infty$. Affin d'utiliser la règle de l'Hospital, nous la écrivons $x^m\ln(x)=\ln(x)/(x^{-m})$, et nous trouvons
\begin{equation}
	\lim_{x\to 0} \frac{ \ln(x) }{ x^{-m} }=\lim_{x\to 0} \frac{ 1/x }{ -mx^{-m-1} }=\lim_{x\to 0} \frac{1}{ -mx^{-m} }=0.
\end{equation}


\item
\item
\item
\item
\item

\end{enumerate}

\end{corrige}
