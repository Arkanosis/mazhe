% This is part of Agregation : modélisation
% Copyright (c) 2011
%   Laurent Claessens
% See the file fdl-1.3.txt for copying conditions.

\begin{corrige}{Model-0006}

    Nous savons que la moyenne empirique est un estimateur de la moyenne :
    \begin{equation}
        \bar X_n=\frac{1}{ n }\sum_{i=1}^nX_i.
    \end{equation}
    Nous cherchons un intervalle du type \( I=\mathopen[ \bar X_n-\epsilon , \bar X_n+\epsilon \mathclose]\) pour lequel \( P(m\in I)=1-\alpha\). Nous savons que la variable aléatoire
    \begin{equation}
        \frac{ \bar X_n-m }{ \alpha/\sqrt{n} }
    \end{equation}
    suit une loi \( \dN(0,1)\), mais la variance est inconnue. La subtilité à savoir est que la variable aléatoire
    \begin{equation}
        Z=\frac{ \bar X_n-m }{ S_n/\sqrt{\sigma} }
    \end{equation}
    où \( S_n^2=\sum_i(X_i-\bar X_n)^2/(n-1)\) suit une loi de Student\index{Student} à \( n\) degrés de liberté \( \dT(n-1)\) en vertu du théorème de Cochran \ref{ThoCochraneChiStudent}. Comme il est usuel de le faire, nous inversons l'intervalle :
    \begin{subequations}
        \begin{align}
            1-\alpha&=P\left( -\epsilon\leq \bar X_n-m\leq \epsilon \right)\\
            &=P\left( -\frac{ \epsilon\sqrt{n} }{ S_n }\leq Z\leq \frac{ \epsilon\sqrt{n} }{ S_n } \right).
        \end{align}
    \end{subequations}
    Les valeurs se trouvent dans des tables; par exemple pour \( n=10\) et \( \alpha=5\%\) nous trouvons
    \begin{equation}
        \frac{ \epsilon\sqrt{n} }{ S_n }=2.262.
    \end{equation}
    Plus généralement nous notons \( t_{n-1,1-\frac{ \alpha }{2}}\) le quantile d'ordre \( 1-\frac{ \alpha }{2}\) de la loi \( \dT(n-1)\), c'est à dire le nombre tel que
    \begin{equation}
        P(Z\leq t_{n-1,1-\frac{ \alpha }{2}})=1-\frac{ \alpha }{2}
    \end{equation}
    si \( Z\sim\dT(n-1)\). L'intervalle de confiance est alors donné par
    \begin{equation}
        I=\left[ \bar X_n-\frac{ t_{n-1,1-\frac{ \alpha }{2}}S_n }{ \sqrt{n} },\bar X_n+\frac{ t_{n-1,1-\frac{ \alpha }{2}}S_n }{ \sqrt{n} } \right].
    \end{equation}
    Cela est un intervalle exact pour \( m\) au niveau de confiance \( 1-\alpha\).

    Nous pouvons aussi trouver un intervalle asymptotique en utilisant le théorème central limite :
    \begin{equation}
        \frac{ \bar X_n-m }{ S_n/\sqrt{n} }\stackrel{\hL}{\longrightarrow} T
    \end{equation}
    avec \( T\sim\dN(0,1)\). 

    Rappel : dire que \( I_n\) est un \defe{intervalle de confiance asymptotique}{intervalle de confiance!asymptotique} signifie que
    \begin{equation}
        \lim_{n\to \infty} P(m\in I_n)= 1-\alpha.
    \end{equation}

    En ce qui concerne la variance \( \sigma^2\), l'intervalle de confiance se construit en utilisant la partie \ref{ItemThoCochraneChiStudentii} du théorème de Cochran \ref{ThoCochraneChiStudent}. Nous introduisons la variable aléatoire pivot
    \begin{equation}
        Z=(n-1)\frac{ S_n^2 }{ \sigma^2 }
    \end{equation}
    qui suit une loi \( \chi^2(n-1)\). Cette loi n'étant pas symétrique (voir figure \ref{LabelFigChiSquared}), nous n'allons pas chercher un intervalle de confiance symétrique. Nous cherchons \( c_1\) et \( c_2\) tels que
    \begin{subequations}
        \begin{numcases}{}
            P\big( \sigma^2\in\mathopen[ c_1 , c_2 \mathclose] \big)=1-\alpha\\
            P\big( \sigma^2\in\mathopen[ 0 , c_1 \mathclose] \big)=\frac{ \alpha }{2}\\
            P\big( \sigma^2\in\mathopen[ c_2 , \infty \mathclose[ \big)=\frac{ \alpha }{2}
        \end{numcases}
    \end{subequations}
    
    La situation est représentée à la figure \ref{LabelFigChiSquaresQuantile}.
    \newcommand{\CaptionFigChiSquaresQuantile}{L'intervalle de confiance pour une $\chi^2$.}
    \input{Fig_ChiSquaresQuantile.pstricks}

    Le construction des nombres \( c_1\) et \( c_2\) passe par la relation


\end{corrige}
