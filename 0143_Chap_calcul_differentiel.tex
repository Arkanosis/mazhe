% This is part of Mes notes de mathématique
% Copyright (c) 2006-2014
%   Laurent Claessens, Carlotta Donadello
% See the file fdl-1.3.txt for copying conditions.
 
%+++++++++++++++++++++++++++++++++++++++++++++++++++++++++++++++++++++++++++++++++++++++++++++++++++++++++++++++++++++++++++
\section{Dérivée directionnelle de fonctions composées}		\label{SecDerDirFnComp}
%+++++++++++++++++++++++++++++++++++++++++++++++++++++++++++++++++++++++++++++++++++++++++++++++++++++++++++++++++++++++++++

Étant donné que nous allons voir en détail la différentielle de fonctions composées à la proposition \ref{PropDiffCompose}, nous n'allons pas rentrer dans tous les détail ici.

Nous savons déjà comment dériver les fonctions composées de $\eR$ dans $\eR$. Si nous avons deux fonctions $f\colon \eR\to \eR$ et $u\colon \eR\to \eR$, nous formons la composée $\varphi=f\circ u\colon \eR\to \eR$ dont la dérivée vaut
\begin{equation}
	\varphi'(a)=f'\big( u(a) \big)u'(a).
\end{equation}


Considérons maintenant le cas un peu plus compliqué des fonctions $f\colon \eR\to \eR$ et $u\colon \eR^2\to \eR$, et de la composée
\begin{equation}
	\begin{aligned}
		\varphi\colon \eR^2&\to \eR \\
		\varphi(x,y)&= f\big( u(x,y) \big). 
	\end{aligned}
\end{equation}
Afin de calculer la dérivée partielle de $\varphi$ par rapport à $x$, nous admettons que pour tout $a$, $b$ et $t$, il existe $c\in\mathopen[ a , a+t \mathclose]$ tel que
\begin{equation}
	u(a+t,b)=u(a,b)+t\frac{ \partial u }{ \partial x }(c,b).
\end{equation}
Cela est une généralisation immédiate du théorème \ref{ThoAccFinis}. Nous devons calculer
\begin{equation}		\label{EqPremPasDiffxvp}
	\frac{ \partial \varphi }{ \partial x }(a,b)=\lim_{t\to 0} \frac{ \varphi(a+t,b)-\varphi(a,b) }{ t }=\lim_{t\to 0} \frac{ f\big( u(a+t,b) \big)-g\big( u(a,b) \big) }{ t }.
\end{equation}
Étant donné l'hypothèse que nous avons faite sur $u$, nous avons
\begin{equation}
	f\big( u(a+t,b) \big)=f\big( u(a,b)+t\frac{ \partial u }{ \partial x }(c,b) \big).
\end{equation}
En utilisant le théorème des accroissements finis pour $f$, nous avons un point $d$ entre $u(a,b)$ et $u(a,b)+t\frac{ \partial u }{ \partial x }(c,b)$ tel que
\begin{equation}
	f\big( u(a,b)+t\frac{ \partial u }{ \partial x }(c,b) \big)=f\big( u(a,b) \big)+t\frac{ \partial u }{ \partial x }(c,b)f'(d).
\end{equation}
Le numérateur de \eqref{EqPremPasDiffxvp} devient donc
\begin{equation}
	t\frac{ \partial u }{ \partial x }(c,b)f'(d).
\end{equation}
Certes les points $c$ et $d$ sont inconnus, mais nous savons que $c$ est entre $a$ et $a+t$ ainsi que $d$ se situe entre $u(a,b)$ et $u(a,b)+t\frac{ \partial u }{ \partial x }(c,b)$. Lorsque nous prenons la limite $t\to 0$, nous avons donc $\lim_{t\to 0} c=a$ et $\lim_{t\to 0} d=u(a,b)$. Nous avons alors
\begin{equation}
	\lim_{t\to 0} \frac{ t\frac{ \partial u }{ \partial x }(c,b)f'(d) }{ t }=\frac{ \partial u }{ \partial x }(a,b)f'\big( u(a,b) \big).
\end{equation}
La formule que nous avons obtenue (de façon pas très rigoureuse) est
\begin{equation}
	\frac{ \partial  }{ \partial x }f\big( u(x,y) \big)=\frac{ \partial u }{ \partial x }(x,y)f'\big( u(x,y) \big).
\end{equation}

Prenons maintenant un cas un peu plus compliqué où nous voudrions savoir les dérivées partielles de la fonction $\varphi$ donnée par
\begin{equation}
	\varphi(x,y,z)=f\big( u(x,y),v(x,y,z) \big)
\end{equation}
où $f\colon \eR^2\to \eR$, $u\colon \eR^2\to \eR$ et $v\colon \eR^3\to \eR$. 

Commençons par la dérivée partielles par rapport à $z$. Étant donné que $\varphi$ ne dépend de $z$ que via la seconde entrée de $f$, il est normal que seule la dérivée partielle de $f$ par rapport à sa seconde entrée arrive dans la formule :
\begin{equation}
	\frac{ \partial \varphi }{ \partial z }(x,y,z)=\frac{ \partial f }{ \partial v }\big( u(x,y),v(x,y,z) \big)\frac{ \partial v }{ \partial z }(x,y,z).
\end{equation}
La dérivée partielle par rapport à $y$ demande de tenir compte en même temps de la façon dont $f$ varie avec sa première entrée et la façon dont elle varie avec sa seconde entrée; cela nous fait deux termes :
\begin{equation}
	\frac{ \partial \varphi }{ \partial y }(x,y,z)=\frac{ \partial f }{ \partial u }\big( u(x,y),v(x,y,z) \big)\frac{ \partial u }{ \partial y }(x,y)+\frac{ \partial f }{ \partial v }\big( u(x,y),v(x,y,z) \big)\frac{ \partial v }{ \partial y }(x,y,z).
\end{equation}


Cette formule a une interprétation simple. Lançons un caillou du sommet d'une falaise. Son mouvement est une chute libre avec une vitesse initiale horizontale :
\begin{subequations}
	\begin{numcases}{}
		x(t)=v_0t\\
		y(t)=h_0-\frac{ gt^2 }{ 2 }
	\end{numcases}
\end{subequations}
où $v_0$ est la vitesse initiale horizontale et $h_0$ est la hauteur de la falaise. Si nous sommes intéressés à la distance entre le caillou et le bas de la falaise (point $(0,0)$), le théorème de Pythagore nous dit que
\begin{equation}
	d(t)=\sqrt{x^2(t),y^2(t)}.
\end{equation}
Pour trouver la variation de la distance par rapport au temps il faut savoir de combien la distance varie lorsque $x$ varie et multiplier par la variation de $x$ par rapport à $t$, et puis faire la même chose avec $y$.

\begin{theorem}		\label{ThoDerDirFnComp}
	Soit $g\colon \eR^m\to \eR^n$ une fonction différentiable en $a$, et $f\colon \eR^n\to \eR^p$ une fonction différentiable en $g(a)$. Si nous définissons $\varphi(x)=(f\circ g)(x)$, alors pour tout $i=1,\ldots,m$, nous avons
	\begin{equation}
		\frac{ \partial \varphi }{ \partial x_i }(a)=\sum_{k=1}^n\frac{ \partial f }{ \partial y_k }\big( g(a) \big)\frac{ \partial g }{ \partial x_i }
	\end{equation}
	où $\frac{ \partial f }{ \partial y_k }$ dénote la dérivée partielle de $f$ par rapport à sa $k$-ième variable.
\end{theorem}

Donnons un exemple d'utilisation de cette formule. Si
\begin{equation}
	\begin{aligned}[]
		g\colon \eR^2\to \eR^3\\
		f\colon \eR^3\to \eR,
	\end{aligned}
\end{equation}
nous avons $\varphi\colon \eR^2\to \eR$. Les dérivées partielles de $\varphi$ sont données par les formules
\begin{equation}
	\frac{ \partial \varphi }{ \partial x }(x,y)=\frac{ \partial f }{ \partial x_1 }\big( g(x,y) \big)\frac{ \partial g_1 }{ \partial x }(x,y)+\frac{ \partial f }{ \partial x_2 }\big( g(x,y) \big)\frac{ \partial g_2 }{ \partial y }(x,y)+\frac{ \partial f }{ \partial x_3 }\big( g(x,y) \big)\frac{ \partial g_3 }{ \partial x }(x,y)
\end{equation}
et
\begin{equation}
	\frac{ \partial \varphi }{ \partial y }(x,y)=\frac{ \partial f }{ \partial x_1 }\big( g(x,y) \big)\frac{ \partial g_1 }{ \partial y }(x,y)+\frac{ \partial f }{ \partial x_2 }\big( g(x,y) \big)\frac{ \partial g_2 }{ \partial y }(x,y)+\frac{ \partial f }{ \partial x_3 }\big( g(x,y) \big)\frac{ \partial g_3 }{ \partial y }(x,y)
\end{equation}
Notez que les dérivées de $\varphi$ et des composantes de $g$ sont calculées en $(x,y)$, tandis que celles de $f$ sont calculées en $g(x,y)$.

%++++++++++++++++++++++++++++++++++++++++++++++++++++++++++++++++++++++++++++++++++++++++++++++++++++++++++++++++++++++
\section{Théorèmes des accroissements finis}		\label{SecThoAccrsFinis}
%++++++++++++++++++++++++++++++++++++++++++++++++++++++++++++++++++++++++++++++++++++++++++++++++++++++++++++++++++++++

Nous avons déjà démontré (lemme \ref{LemdfaSurLesPartielles}) que si $f$ est différentiable au point $x$ alors  $df_x(u)=\partial_uf(x)$. Une importante conséquence est le théorème des accroissements finis
\begin{theorem}[Accroissements finis, inégalité de la moyenne]\label{val_medio_2}
   Soit $U$ un ouvert dans $\eR^m$ et soit $f:U\to\eR^n$ une fonction différentiable. Soient $a$ et $b$ deux point dans $U$, $a\neq b$, tels que le segment $[a,b]$ soit contenu dans $U$. Alors
\[
\|f(b)-f(a)\|_n\leq \sup_{x\in[a,b]}\|df(x)\|_{\mathcal{L}(\eR^m,\eR^n)}\|b-a\|_m.
\]
\end{theorem}
\index{application!différentiable}
\index{inégalité!de la moyenne}
\index{théorème!accroissements finis!forme générale}.
Dans le cas de dimension infinie, il y a le théorème \ref{ThoNAKKght}.

\begin{proof}
 On utilise le théorème \ref{val_medio_1} et le fait que 
\[
\|\partial_u f(x)\|_n\leq \|df(x)\|_{\mathcal{L}(\eR^m,\eR^n)}\|u\|_m,
\]
pour tout $u$ dans $\eR^m$.
\end{proof}

La proposition suivante est une application fondamentale du théorème des accroissements finis \ref{val_medio_2}.
\begin{proposition}		\label{PropAnnulationEtConstance}
	Soit $U$ un ouvert connexe par arcs de $\eR^m$ et une fonction $f\colon U\to \eR^n$. Les conditions suivantes sont équivalentes :
	\begin{enumerate}
		\item\label{ItemPropCstDiffZeroi}
			$f$ est constante;
		\item\label{ItemPropCstDiffZeroii}
			$f$ est différentiable et $df(a)=0$ pour tout $a\in U$;
		\item\label{ItemPropCstDiffZeroiii}
			les dérivées partielles $\partial_1f,\ldots,\partial_mf$ existent et sont nulles sur $U$.
	\end{enumerate}
\end{proposition}
\index{connexité!par arc!fonction différentiable}
\index{différentiabilité}

\begin{proof}
	Nous allons démonter les équivalences en plusieurs étapes. D'abord \ref{ItemPropCstDiffZeroi} $\Rightarrow$ \ref{ItemPropCstDiffZeroii}, puis \ref{ItemPropCstDiffZeroii} $\Rightarrow$ \ref{ItemPropCstDiffZeroiii}, ensuite \ref{ItemPropCstDiffZeroiii} $\Rightarrow$ \ref{ItemPropCstDiffZeroii} et enfin \ref{ItemPropCstDiffZeroii} $\Rightarrow$ \ref{ItemPropCstDiffZeroi}.

	Commençons par montrer que la condition \ref{ItemPropCstDiffZeroi} implique la condition \ref{ItemPropCstDiffZeroii}. Si $f(x)$ est constante, alors la condition \eqref{EqCritereDefDiff} est vite vérifiée en posant $T(h)=0$.

	Afin de voir que la condition \ref{ItemPropCstDiffZeroii} implique la condition \ref{ItemPropCstDiffZeroiii}, remarquons d'abord que la différentiabilité de $f$ implique que les dérivées partielles existent (proposition \ref{diff1}) et que nous avons l'égalité $df(a).u=\sum_iu_i\partial_if(a)$ pour tout $u\in\eR^m$ (lemme \ref{LemdfaSurLesPartielles}). L'annulation de $\sum_iu_i\partial_if(a)$ pour tout $u$ implique l'annulation des $\partial_if(a)$ pour tout $i$.

	Prouvons maintenant que la propriété \ref{ItemPropCstDiffZeroiii} implique la propriété \ref{ItemPropCstDiffZeroii}. D'abord, par la proposition \ref{Diff_totale}, l'existence et la continuité des dérivées partielles $\partial_if(a)$ implique la différentiabilité de $f$. Ensuite, la formule $df(a).u=\sum_i u_i\partial_if(a)$ implique que $df(a)=0$. 
	
	
	Il reste à montrer que \ref{ItemPropCstDiffZeroii} implique la condition \ref{ItemPropCstDiffZeroi}, c'est à dire que l'annulation de la différentielle implique la constance de la fonction. C'est ici que nous allons utiliser le théorème des accroissements finis. En effet, si $a$ et $b$ sont des points de $U$, le théorème \ref{val_medio_2} nous dit que
	\begin{equation}
		\|f(b)-f(a)\|_n\leq \sup_{x\in[a,b]}\|df(x)\|_{\mathcal{L}(\eR^m,\eR^n)}\|b-a\|_m.
	\end{equation}
	Mais $\| df(x) \|=0$ pour tout $x\in U$, donc ce supremum est nul et $f(b)=f(a)$, ce qui signifie la constance de la fonction.
\end{proof}

%\begin{proof}
%  \begin{itemize}
%  \item Le théorème \ref{val_medio_2} nous dit que si la différentielle de $f$ est nulle alors $f$ est constante sur chaque segment contenu dans $U$. Cela nous dit que $f$ est constante sur chaque boule contenue dans $U$, donc $f $ est localement constante. Il est possible de démontrer que toute fonction localement constante sur un connexe est constante.  
%\item Si toutes les dérivées partielles $\partial_1 f, \ldots, \partial_m f $ existents et sont identiquement nulles sur $U$ alors $f$ est différentiable et sa différentielle est identiquement nulle. On utilise la première partie de la preuve pour conclure. 
%  \end{itemize}
%\end{proof}

%+++++++++++++++++++++++++++++++++++++++++++++++++++++++++++++++++++++++++++++++++++++++++++++++++++++++++++++++++++++++++++
\section{Fonctions Lipschitziennes}
%+++++++++++++++++++++++++++++++++++++++++++++++++++++++++++++++++++++++++++++++++++++++++++++++++++++++++++++++++++++++++++

\begin{definition}
    Soient \( (E,d_E)\) et \( (F,d_F)\) deux espaces métriques\footnote{Pour rappel, les espaces métriques sont définis par la définition \ref{DefMVNVFsX} et le théorème \ref{ThoORdLYUu}; je précise que nous ne supposons pas que \( E\) soit vectoriel; en particulier il peut être un ouvert de \( \eR^n\).}, \( f\colon E\to F\) une application et un réel \( k\) strictement positif. Nous disons que \( f\) est \defe{Lipschitzienne}{Lipschitzienne} de constante $k$ sur \( E\) si pour tout \( x,y\in E\),
    \begin{equation}
        d_F\big( f(x)-f(y) \big)\leq kd_E(x,y).
    \end{equation}
\end{definition}
%TODO : faire la chasse aux endroits où cette définition devrait être référencée.
Soit \( f\) une fonction \( k\)-Lipschitzienne. Si \( y\in \overline{ B(x,\delta)}\) alors \( \| x-y \|\leq\delta\) et donc \( \big\| f(x)-f(y) \big\|\leq k\delta\). Cela signifie que la condition Lipschitz pour s'énoncer en termes de boules fermées par
\begin{equation}    \label{EqDZvtUbn}
    f\big( \overline{ B(x,\delta) } \big)\subset \overline{  B\big( f(x),k\delta \big) }
\end{equation}
tant que \( \overline{ B(x,\delta) } \) est contenue dans le domaine sur lequel \( f\) est Lipschitz.

\begin{proposition}
  Soit  $U$ un ouvert convexe  de $\eR^m$, et soit $f:U\to \eR^n$ une fonction différentiable. La fonction $f$ est Lipschitzienne sur $U$ si et seulement si $df$ est bornée sur $U$.  
\end{proposition}
\begin{proof}
	Le fait que l'application différentielle $df$ soit bornée signifie qu'il existe un $M>0$ dans $\eR$ tel que $\|df_a\|_{\mathcal{L}(\eR^m,\eR^n)}\leq M$, pour tout $a$ dans $U$. Si cela est le cas, alors le théorème \ref{val_medio_2} et la convexité\footnote{La convexité de $U$ sert à assurer que la droite reliant $a$ à $b$ est contenue dans $U$; c'est ce que nous utilisons dans la démonstration du théorème \ref{val_medio_2}.} de $U$ impliquent évidemment que $f$ est de Lipschitz de constante plus petite ou égale à $M$.
	
	Inversement, si $f$ est de Lipschitz de constante $k$, alors pour tout $a$ dans $U$ et $u$ dans $\eR^m$ on a 
	\[
		\left\|\frac{f(a+tu)-f(a)}{t}\right\|_n\leq k \|u\|_m,
	\]   
	En passant à la limite pour $t\to 0$ on a 
	\[
		\|\partial_u f(a)\|_n=\|df_a(u)\|_n\leq k \|u\|_m,
	\]
	donc la norme de $df_a$ est majorée par $k$ pour tout $a$ dans $U$.   
\end{proof}

Notez cependant qu'une fonction peut être Lipschitzienne sans être différentiable.

\begin{proposition} \label{PropFZgFTEW}
    Une fonction Lipschitzienne \( f\colon \eR\to \eR\) est continue.
\end{proposition}

\begin{proof}
    Prouvons la continuité en \( a\in \eR\). Pour tout \( x\) nous avons
    \begin{equation}
        \big| f(x)-f(a) \big|\leq k| x-a |.
    \end{equation}
    Si \( \epsilon>0\) est donné, il suffit de prendre \( \delta<\frac{ \epsilon }{ k }\) pour avoir
    \begin{equation}
        \big| f(x)-f(a) \big|\leq k\frac{ \epsilon }{ k }=\epsilon.
    \end{equation}
    Donc \( f\) est continue en \( a\).
\end{proof}
Implicitement nous avons utilisé le théorème \ref{ThoESCaraB}.

%++++++++++++++++++++++++++++++++++++++++++++++++++++++++++++++++++++++++++++++++++++++++
\section{Différentielles d'ordre supérieur}		\label{SecDiffOrdSup}
%++++++++++++++++++++++++++++++++++++++++++++++++++++++++++++++++++++++++++++++++++++++++++++++++++++++++++++++++++++++++++++++
\begin{definition}
	Soit $U$ un ouvert de $\eR^m$ et  $f:U\subset\eR^m\to \eR^n$ une fonction. La fonction $f$ est dite \defe{deux fois différentiable}{différentiable!deux fois} au point $a$ dans $U$,  si $f$ est différentiable dans un voisinage de $a$, et sa différentielle $df$ est différentiable au point $a$ en tant que application de $U$ dans $\mathcal{L}(\eR^m, \eR^n)$.  

%--------------------------------------------------------------------------------------------------------------------------- 
\subsection{Identification des espaces d'applications multilinéaires}
%---------------------------------------------------------------------------------------------------------------------------

La fonction $f$ sera dite deux fois différentiable sur l'ensemble $U$ si elle est deux fois différentiable en chaque point de $U$.
\end{definition}
La différentielle de la différentielle de $f$ est notée 
\[
d(df)(a)=d^2f(a),
\]
et est une application de $U$ dans $\mathcal{L}(\eR^m,\mathcal{L}(\eR^m, \eR^n) )$. Comme on a vu dans la proposition \ref{isom_isom}, l'espace $\mathcal{L}(\eR^m,\mathcal{L}(\eR^m, \eR^n) )$ est isométriquement isomorphe à l'espace $\mathcal{L}(\eR^m\times\eR^m, \eR^n )$. On verra comment cette propriété  est utilisé dans l'exemple \ref{bilin_2diff}.


Soient \( V\) et \( W\) deux espace vectoriel normés de dimension finie et \( \mO\) un ouvert autour de \( x\in V\). D'une part l'espace des applications linéaires \( \aL(V,W)\) est lui-même un espace vectoriel normé de dimension finie, et on peut identifier \(  \aL\big( V,\aL^{(k)}(V,W) \big)\)\nomenclature[Y]{\( \aL^{(n)}(V,W)\)}{L'espace des applications \( n\)-linéaires \( V^n\to W\)} avec \( \aL^{(k+1)}(V,W)\), ce qui nous permet de dire que la \( k\)\ieme\ différentielle est une application
\begin{equation}
    d^kf\colon \mO\to \aL^{(k)}(V,W).
\end{equation}
Plus précisément, l'identification se fait de la façon suivante : si \( \omega\in \aL\big( V,\aL^{(k)}(V,W) \big)\), alors \( \omega\) vu dans \( \aL^{(k+1)}(V,W)\) est définie par
\begin{equation}
    \omega(u_1,\ldots, u_{k+1})=\omega(u_1)(u_2,\ldots, u_{k+1}).    
\end{equation}


Cela étant posé nous pouvons donner les définition.

%--------------------------------------------------------------------------------------------------------------------------- 
\subsection{Fonctions différentiables plusieurs fois}
%---------------------------------------------------------------------------------------------------------------------------

\begin{definition}[\cite{ZCKMFRg}]  \label{DefPNjMGqy}
    La fonction \( f\colon \mO\subset V\to W\) est
    \begin{enumerate}
        \item
            de classe \( C^0\) si elle est continue,
        \item
            de classe \( C^1\) si \( df\colon \mO\to \aL(V,W)\) est continue,
        \item
            de classe \( C^k\) si \( d^kf\colon \mO\to \aL^{(k)}(V,W)\) est continue,
        \item
            de classe \(  C^{\infty}\) si \( f\) est dans \( \bigcap_{k=0}^{\infty}C^k(V,W)\).
    \end{enumerate}
\end{definition}
\index{application!différentiable}
\index{application!de classe \( C^k\)}

\begin{definition}
    Un \( C^k\)-difféomorphisme\index{difféomorphisme!de classe \( C^k\)} est une application inversible de classe \( C^k\) dont l'inverse est également de classe \( C^k\).
\end{definition}

\begin{example} \label{ExZHZYcNH}
    Voyons commet la différentielle seconde fonctionne. Soit \( f\in C^2(V,W)\); nous notons \( \varphi=df\colon V\to \aL(V,W)\) et donc nous voulons étudier la fonction
    \begin{equation}
        d\varphi\colon V\to \aL\big( V,\aL(V,W) \big).
    \end{equation}
    Si \( a,u\in V\)  nous avons
    \begin{equation}
        d\varphi_a(u)=\Dsdd{ \varphi(a+tu) }{t}{0}
    \end{equation}
    qui est une dérivée dans \( \aL(V,W)\) -- pas de problèmes : c'est un espace vectoriel normé de dimension finie. Par linéarité, nous pouvons faire entrer l'argument de \( d\varphi_a(u)\) dans la dérivée :
    \begin{subequations}
        \begin{align}
            d\varphi_a(u)v&=\Dsdd{ \varphi(a+tu)v }{t}{0}\\
            &=\Dsdd{ df_{a+tu}(v) }{t}{0}\\
            &=\Dsdd{  \Dsdd{ f(a+tu+sv) }{s}{0}  }{t}{0}\\
            &=\Dsdd{ \frac{ \partial f }{ \partial v }(a+tu) }{t}{0}\\
            &=\frac{ \partial^2f  }{ \partial u\partial v }(a).
        \end{align}
    \end{subequations}
    Par conséquent nous voyons
    \begin{equation}\label{EqQHINNtD}
        \begin{aligned}
            d^2f\colon V&\to \aL^{(2)}(V,W) \\
            d^2f_a(u,v)&=\frac{ \partial^2f  }{ \partial u\partial v }(a). 
        \end{aligned}
    \end{equation}
    
    Dans le cas d'une fonction \( f\colon \eR\to \eR\), nous avons une seule direction et par linéarité de \eqref{EqQHINNtD} par rapport à \( u\) et \( v\), nous avons
    \begin{equation}
        d^2f_a(u,v)=f''(a)uv
    \end{equation}
    où les produits sont des produits usuels dans \( \eR\) et \( f''\) est la dérivée seconde usuelle.
\end{example}


\begin{example}\label{bilin_2diff}
	Soit $B:\eR^m\times \eR^m\to\eR^n$ une application bilinéaire. On définit $f:\eR^m\to\eR^n$ par $f(x)=B(x,x)$. Le lemme \ref{bilin_diff} nous dit que $B$ est différentiable. Cela implique la différentiabilité de $f$. Pour trouver la différentielle de la fonction $f$, nous écrivons $f=B\circ s$ où $s\colon \eR^m\to \eR^m\times\eR^m$ est l'application $s(x)=(x,x)$. En utilisant la règle de différentiation de fonctions composées,
	\begin{equation}
		df(a)=dB\big( s(a) \big)\circ ds(a).
	\end{equation}
	Mais $ds(a).u=(u,u)$ parce que $s(a+h)-s(a)-(h,h)=0$. Par conséquent,
	\begin{equation}		\label{EqdBsaExp}
		df(a).u=dB\big( s(a) \big)(u,u)=B(u,a)+B(a,u)
	\end{equation}
	où nous avons utilisé la formule du lemme \ref{bilin_diff}. La formule \eqref{EqdBsaExp} peut être écrite sous la forme compacte
	\begin{equation}
		df(a)=B(\cdot,a)+B(a,\cdot)
	\end{equation}
	La fonction $df(a)$ ainsi écrite est linéaire par rapport à $a$, donc différentiable. En outre elle coïncide avec sa différentielle, comme on a vu dans le remarque \ref{rk_lin}, au sens que la différentielle de $df$ au point $a$ sera l'application que à chaque $x$ dans $\eR^m$ associe l'application linéaire $B(x,\cdot)+B(\cdot, x)$. On voit bien que $d^2f$ au point $a$ est une application de $\eR^m$ vers l'espace des applications linéaires $\mathcal{L}(\eR^m, \eR^n)$. On peut utiliser d'autre part l'isomorphisme des espaces $\mathcal{L}(\eR^m,\mathcal{L}(\eR^m, \eR^n) )$ et $\mathcal{L}(\eR^m\times\eR^m, \eR^n )$ et dire que, une fois que $a$ est fixé, l'application $d^2f(a)$ est une application bilinéaire sur $\eR^m\times\eR^m$. On écrit alors $d^2f(a)(x,y)=B(x,y)+B(y,x)$.   
\end{example}

Une condition nécessaire et suffisante pour l'existence de la différentielle seconde est la suivante
\begin{proposition}
   Soit $U$ un ouvert de $\eR^m$ et  $f:U\subset\eR^m\to \eR^n$ une fonction. La fonction $f$ est deux fois différentiable au point $a$ si et seulement si les dérivées partielles $\partial_1 f, \ldots, \partial_m f $ sont différentiables en $a$. 
\end{proposition}
Cela veut dire, en particulier, que $f$ est deux fois différentiable si et seulement si ses dérivées partielles secondes, $\partial_i\partial_j f$, pour toute couple d'indices $i,j$  dans $\{1,\ldots, m\}$, existent et sont continues. Pour les différentielles d'ordre supérieur on a la définition suivante.
\begin{definition}
  Soit $U$ un ouvert de $\eR^m$ et  $f:U\subset\eR^m\to \eR^n$ une fonction. On dit que $f$ est de classe $\mathcal{C}^k$, c'est à dire que $f$ est $k$ fois différentiable,  pour $k$ dans $\eN$, $k\geq 1$, si les dérivées partielles $\partial_1 f, \ldots, \partial_m f $ existent et sont de classe $\mathcal{C}^{k-1}$. 
\end{definition}
%TODO : cette définition n'est pas la définition. La vraie définition est la définition \ref{DefPNjMGqy}. 
% En particulier, il ne faut pas mettre de label sur ceci et réfléchir à en faire une proposition.
La différentielle seconde dans l'exemple  \ref{bilin_2diff} est symétrique, c'est à dire que $d^2f(a)(x_1,x_2)=d^2f(a)(x_2,x_1)$. En fait toute différentielle seconde est symétrique.  


\begin{theorem}[Schwarz]\label{Schwarz}
 Soit $U$ un ouvert de $\eR^m$ et  $f:U\subset\eR^m\to \eR^n$ une fonction de classe $\mathcal{C}^2$. Alors, pour toute couple $i,j$ d'indices dans $\{1,\ldots, m\}$ et pour tout point $a$ dans $U$, on a 
\[
\frac{\partial^2 f}{\partial  x_i\partial x_j}(a)=\frac{\partial^2 f}{\partial  x_j\partial x_i}(a).
\]
\end{theorem}
\begin{proof}
  Pour simplifier l'exposition nous nous limitons ici au cas $m=2$. Soit $(h,g)$ un vecteur fixé dans $\eR^2$. Pour tout  $v=(x,y)$ dans $\eR^2$ on note
  \begin{equation}
    \begin{array}{c}
      \Delta_h f(v)=f(v+he_1) -f(v) = f(x+h,y)-f(x,y),\\ 
      \Delta_g f(v)=f(v+ge_2) -f(v) = f(x,y+g)-f(x,y),\\ 
    \end{array}
  \end{equation}
Nous avons
\begin{equation}
  \begin{array}{c}
   \Delta_g   \Delta_h f(v)=\left(f(x+h,y+g)-f(x,y+g)\right)-\left(f(x+h,y)-f(x,y)\right),\\
   \Delta_h   \Delta_g f(v)=\left(f(x+h,y+g)-f(x+h,y)\right)-\left(f(x,y+g)-f(x,y)\right),
  \end{array}
\end{equation}
donc, 
\begin{equation}
  \frac{1}{g} \Delta_g  \left(\frac{1}{h} \Delta_h f(v)\right) = \frac{1}{h} \Delta_h \left(\frac{1}{g} \Delta_g f(v)\right).
\end{equation}
On utilise alors le théorème des accroissements finis
\[
\frac{1}{h} \Delta_h f(v)=\frac{1}{h}\big(f(x+h,y)-f(x,y)\big)=\frac{1}{h}\partial_1f(x+t_1h,y )h=\partial_1f(x+t_1h, y),
\]
pour un certain $t_1$ dans $]0,1[$. De même on obtient 
\[
\frac{1}{g} \Delta_g f(v)= \partial_2 f(x, y+t_2g),
\]
pour un certain $t_2$ dans $]0,1[$. Alors
 \begin{equation}
  \frac{1}{g} \Delta_g  \big(\partial_1f(x+t_1h, y)\big) = \frac{1}{h} \Delta_h \big(\partial_2 f(x, y+t_2g)\big).
\end{equation}
En appliquant encore une fois le théorème des accroissements finis on a
 \begin{equation}
  \partial_2\partial_1f(x+t_1h, y+s_1g) = \partial_1\partial_2 f(x+s_2h, y+t_2g).
\end{equation} 
Il suffit maintenant de passer à la limite pour $(h,g) \to (0,0)$ et de se souvenir du fait que $f$ est $\mathcal{C}^2$ seulement si ses dérivées partielles secondes sont continues pour avoir $\partial_2\partial_1f(v)=\partial_1\partial_2 f(v)$.
\end{proof}
Si $f$ est deux fois différentiable $d^2f(a)$ est l'application bilinéaire associée avec la matrice symétrique
\begin{equation}
 H_f(a)= \begin{pmatrix}
    \partial^2_1f(a)& \ldots& \partial_1\partial_m f(a)\\
    \vdots& \ddots& \vdots\\
    \partial_1\partial_m f(a)&\ldots&\partial^2_1f(a),
  \end{pmatrix}
\end{equation}
Cette matrice est dite la matrice \defe{hessienne}{hessienne} de $f$. 

\begin{example}
  Montrons qu'il n'existe pas de fonctions $f$ de classe $\mathcal{C}^2$ telles que $\partial_xf(x,y)= 5\sin x$ et $\partial_y(x,y)=6x+y$.  Ceci est vite fait en appliquant le théorème de Schwarz, \ref{Schwarz}; ce que nous trouvons est
\[
\partial_y (\partial_xf)= 0\neq \partial_x(\partial_yf)= 6.
\]
Donc, l'existence d'une fonction $f$ de classe $\mathcal{C}^2$ telle que $\partial_x(x,y)= 5\sin x$ et $\partial_yf(x,y)=6x+y$ serait en contradiction avec le théorème.  
\end{example}

%+++++++++++++++++++++++++++++++++++++++++++++++++++++++++++++++++++++++++++++++++++++++++++++++++++++++++++++++++++++++++++
\section{Développement asymptotique, théorème de Taylor}
%+++++++++++++++++++++++++++++++++++++++++++++++++++++++++++++++++++++++++++++++++++++++++++++++++++++++++++++++++++++++++++
\label{AppSecTaylorR}

Le théorème suivant généralise à l'utilisation de toutes les dérivées disponibles le résultat de développement limité donné par la proposition \ref{PropUTenzfQ}.
\begin{theorem}[Théorème de Taylor]		\label{ThoTaylor}
Soit $I\subset$ un intervalle non vide et non réduit à un point de $\eR$ ainsi que $a\in I$. Soit une fonction $f\colon I\to \eR$ telle que $f^{(n)}(a)$ existe. Alors il existe une fonction $\epsilon$ définie sur $I$ et à valeurs dans $\eR$ vérifiant les deux conditions suivantes :
\begin{subequations}		\label{SubEqsDevTauil}
	\begin{align}
		\lim_{x\to a}\epsilon(x)&=0,\\
		f(x)&=T^a_{f,n}(x)+\epsilon(x)(x-a)^{n}	&&\forall x\in I		\label{subeqfTepseqb}
	\end{align}
\end{subequations}
où $T^a_{f,n}(x)=\sum_{k=0}^n\frac{ f^{(k)}(a) }{ k! }(x-a)^k$ et $f^{(k)}$ dénote la $k$-ième dérivée de $f$ (en particulier, $f^{(0)}=f$, $f^{(1)}=f'$).\nomenclature{$f^{(n)}$}{La $n$-ième dérivée de la fonction $f$}
\end{theorem}

Nous insistons sur le fait que la formule \eqref{subeqfTepseqb} est une égalité, et non une approximation. Ce qui serait une approximation serait de récrire la formule dans le terme contenant $\epsilon$.

Le polynôme $T^a_{f,n}$ est le \defe{polynôme de Taylor}{Taylor} de $f$ au point $a$ à l'ordre $n$.  Une preuve du théorème peut être trouvée dans \cite{TrenchRealAnalisys}, théorème 2.5.1 à la page 99. La version donnée ici est inspirée de l'article sur \wikipedia{fr}{Développement_de_Taylor}{Wikipédia}\footnote{http://fr.wikipedia.org/wiki/Développement\_de\_Taylor}, qui donne également une preuve du résultat.

En termes de notations, nous définissons l'ensemble $o(x)$\nomenclature{$o(x)$}{fonction tendant rapidement vers zéro} l'ensemble des fonctions $f$ telles que
\begin{equation}
	\lim_{x\to 0} \frac{ f(x) }{ x }=0.
\end{equation}
Plus généralement si $g$ est une fonction telle que $\lim_{x\to 0} g(x)=0$, nous disons $f\in o(g)$ si
\begin{equation}
	\lim_{x\to 0} \frac{ f(x) }{ g(x) }=0.
\end{equation}
De façon intuitive, l'ensemble $o(g)$ est l'ensemble des fonctions qui tendent vers zéro «plus vite» que $g$.


Nous pouvons donner un énoncé alternatif au théorème \ref{ThoTaylor} en définissant $h(x)=\epsilon(x+a)x^n$. Cette fonction est définie exprès pour avoir
\begin{equation}
	h(x-a)=\epsilon(x)(x-a)^n,
\end{equation}
et donc
\begin{equation}
	\lim_{x\to 0} \frac{ h(x) }{ x^n }=\lim_{x\to 0} \epsilon(x-a)=\lim_{x\to a}\epsilon(x)=0. 
\end{equation}
Donc $h\in o(x^n)$.

Le théorème dit donc qu'il existe une fonction $\alpha\in o(x^n)$ telle que
\begin{equation}
	f(x)=T^a_{f,n}(x)+\alpha(x-a).
\end{equation}
pour tout $x\in I$. 

\begin{example}
	Le développement du cosinus est donné par
	\begin{equation}
		\cos(x)=1-\frac{ x^2 }{ 2 }+\frac{ x^4 }{ 4! }-\frac{ x^6 }{ 6! }\cdots
	\end{equation}
	Nous avons donc l'existence d'une fonction $h_1\in o(x^2)$ telle que $\cos(x)=1-\frac{ x^2 }{ 2 }+h_1(x)$. Il existe aussi une autre fonction $h_2\in o(x^4)$ telle que $\cos(x)=1-\frac{ x^2 }{ 2 }+\frac{ x^4 }{ 4! }+h_2(x)$.
\end{example}

\begin{example}		\label{ExempleUtlDev}
	Une des façons les plus courantes d'utiliser les formules \eqref{SubEqsDevTauil} est de développer $f(a+t)$ pour des petits $t$ en posant $x=a+t$ dans la formule :
	\begin{equation}	\label{EqDevfautouraeps}
		f(a+t)=f(a)+f'(a)t+f''(a)\frac{ t^2 }{ 2 }+\epsilon(a+t)t^2
	\end{equation}
	avec $\lim_{t\to 0} \epsilon(a+t)=0$. Ici, la fonction $T$ dont on parle dans le théorème est $T_{f,2}^a(a+t)=f(a)+f'(a)t+f''(a)\frac{ t^2 }{2}$.

	Lorsque $x$ et $y$ sont deux nombres «proches\footnote{par exemple dans une limite $(x,y)\to(h,h)$.}», nous pouvons développer $f(y)$ autour de $f(x)$ :
	\begin{equation}		\label{Eqfydevfx}
		f(y)=f(x)+f'(x)(y-x)+f''(x)\frac{ (y-x)^2 }{ 2 }+\epsilon(y-x)(y-x)^2,
	\end{equation}
	et donc écrire
	\begin{equation}
		f(x)-f(y)=-f'(x)(y-x)-f''(x)\frac{ (y-x)^2 }{ 2 }-\epsilon(y-x)(y-x)^2.
	\end{equation}
	De cette manière nous obtenons une formule qui ne contient plus que $y$ dans la différence $y-x$.
\end{example}

%---------------------------------------------------------------------------------------------------------------------------
\subsection{Fonctions «petit o» }
%---------------------------------------------------------------------------------------------------------------------------

Nous voulons formaliser l'idée d'une fonction qui tend vers zéro \og plus vite\fg{} qu'une autre. Nous disons que $f\in o\big(\varphi(x)\big)$ si
\begin{equation}
    \lim_{x\to 0} \frac{ f(x) }{ \varphi(x) }=0.
\end{equation}
En particulier, nous disons que $f\in o(x)$ lorsque $\lim_{x\to 0} f(x)/x=0$.

\begin{remark}
    À titre personnel, l'auteur de ces lignes déconseille d'utiliser cette notation qui est un peu casse-figure pour qui ne la maîtrise pas bien.
\end{remark}

%---------------------------------------------------------------------------------------------------------------------------
\subsection{Formule et reste}
%---------------------------------------------------------------------------------------------------------------------------

\begin{proposition}     \label{PropDevTaylorPol}
    Soient $f\colon I\subset\eR\to \eR$ et $a\in\Int(I)$. Soit un entier $k\geq 1$. Si $f$ est $k$ fois dérivable en $a$, alors il existe un et un seul polynôme $P$ de degré $\leq k$ tel que
    \begin{equation}
        f(x)-P(x-a)\in o\big( | x-a |^k \big)
    \end{equation}
    lorsque $x\to a$, $x\neq a$. Ce polynôme  est donné par
    \begin{equation}
        P(h)=f(a)+f'(a)h+\frac{ f''(a) }{ 2! }h^2+\ldots+\frac{ f^{(k)}(a) }{ k! }h^k.
    \end{equation}
    Notons encore deux façons alternatives d'écrire le résultat. Si \( f\in C^k\) il existe une fonction \( \alpha\) telle que \( \lim_{t\to 0} \alpha(t)=0\) et
    \begin{equation}
        f(x)=\sum_{n=0}^k\frac{ f^{(n)}(a) }{ n! }(x-a)^n+(x-a)^n\alpha(x-a).
    \end{equation}
    Si \( f\in C^{k+1}\) alors
    \begin{equation}        \label{EquQtpoN}
        f(x)=\sum_{n=0}^k\frac{ f^{(n)}(a) }{ n! }(x-a)^n+(x-a)^{n+1}\xi(x-a)
    \end{equation}
    où \( \xi\) est une fonction telle que \( \xi(t)\) tend vers une constante lorsque \( t\to 0\).
\end{proposition}

La proposition suivant donne une intéressante façon de trouver le reste d'un développement de Taylor.
\begin{proposition}     \label{PropResteTaylorc}
Soient $I$, un intervalle dans $\eR$ et $f\colon I\to \eR$ une fonction de classe $C^k$ sur $I$ telle que $f^{(k+1)}$ existe sur $I$. Soient $a\in\Int(I)$ et $x\in I$. Alors il existe $c$ strictement compris entre $x$ et $a$ tel que 
\begin{equation}
    R_{f,a,k}(x)=\frac{ f^{(k+1)}(c) }{ (k+1)! }(x-a)^{k+1}.
\end{equation}
\end{proposition}

%--------------------------------------------------------------------------------------------------------------------------- 
\subsection{Reste intégral}
%---------------------------------------------------------------------------------------------------------------------------

\begin{proposition}[Formule de Taylor avec reste intégral\cite{VBYOJrU}]\label{PropAXaSClx}
    Soient \( X\) et \( Y\) des espaces normés et un ouvert \( \mO\subset X\). Si \( f\in C^m(\mO,Y)\) et si \( [p,x]\subset \mO\) alors
    \begin{equation}
        \begin{aligned}[]
            f(x)=f(p)&+\sum_{k=1}^{m-1}\frac{1}{ k! }(d^kf)_p (x-p)^k \\
            &+\frac{1}{ (m-1)! }\int_0^1(1-t)^{m-1}(d^mf)_{ p+t(x-p) }(x-p)^m \
        \end{aligned}
    \end{equation}
    où \( \omega_pu^k\) signifie \( \omega_p(u,\ldots, u)\) lorsque \( \omega\in \Omega^k\).
\end{proposition}
\index{formule!Taylor!reste intégral}
Comme expliqué dans l'exemple \ref{ExZHZYcNH}, toute ces applications de différentielles se réduisent à des termes de la forme
\begin{equation}
    f^{(k)}(p)(x-p)^k
\end{equation}
dans le cas d'une fonction \( \eR\to\eR\).

%---------------------------------------------------------------------------------------------------------------------------
\subsection{Exemple : un calcul heuristique de limite}
%---------------------------------------------------------------------------------------------------------------------------
\label{SubSecCalcLimHeuris}

Soit à calculer la limite suivante :
\begin{equation}
    \lim_{x\to 0} \frac{  e^{-2\cos(x)+2}\sin(x) }{ \sqrt{ e^{2\cos(x)+2}}-1 }.
\end{equation}
La stratégie que nous allons suivre pour calculer cette limite est de développer certaines parties de l'expression en série de Taylor, afin de simplifier l'expression. La première chose à faire est de remplacer $ e^{y(x)}$ par $1+y(x)$ lorsque $y(x)\to 0$. La limite devient
\begin{equation}
    \lim_{x\to 0} \frac{ \big( -2\cos(x)+3 \big)\sin(x) }{ \sqrt{-2\cos(x)+2} }.
\end{equation}
Nous allons maintenant remplacer $\cos(x)$ par $1$ au numérateur et par $1-x^2/2$ au dénominateur. Pourquoi ? Parce que le cosinus du dénominateur est dans une racine, donc nous nous attendons à ce que le terme de degré deux du cosinus donne un degré un en dehors de la racine, alors que du degré un est exactement ce que nous avons au numérateur : le développement du sinus commence par $x$.

Nous calculons donc
\begin{equation}
    \begin{aligned}[]
        \lim_{x\to 0} \frac{ \sin(x) }{ \sqrt{-2\left( 1-\frac{ x^2 }{ 2 } \right)+2} }=\lim_{x\to 0} \frac{ \sin(x) }{ x }=1.
    \end{aligned}
\end{equation}
Tout ceci n'est évidement pas très rigoureux, mais en principe vous avez tous les éléments en main pour justifier les étapes.

%+++++++++++++++++++++++++++++++++++++++++++++++++++++++++++++++++++++++++++++++++++++++++++++++++++++++++++++++++++++++++++ 
\section{Fonctions convexes}
%+++++++++++++++++++++++++++++++++++++++++++++++++++++++++++++++++++++++++++++++++++++++++++++++++++++++++++++++++++++++++++

\begin{definition}[\cite{JFihMcQ}]  \label{DefVQXRJQz}
    Une fonction $f$ d’un intervalle $I$ de \( \eR\) vers \( \eR\) est dite \defe{convexe}{convexe!fonction}\index{fonction!convexe} lorsque, pour tous \( x_1\) et \( x_2\) de $I$ et tout $\lambda$ dans $[0, 1]$ nous avons
    \begin{equation}
        f\big(\lambda\, x_1+(1-\lambda)\, x_2\big) \leq \lambda\, f(x_1)+(1-\lambda)\, f(x_2)
    \end{equation}
    Si l'inégalité est stricte, alors nous disons que la fonction \( f\) est \defe{strictement convexe}{convexe!strictement}.
\end{definition}

Dans l'étude des fonctions convexes nous allons souvent utiliser la fonction \defe{taux d'accroissement}{taux d'accroissement} qui est, pour \( \alpha\) dans le domaine de convexité de \( f\) définie par
\begin{equation}    \label{EqRYBazWd}
    \begin{aligned}
        \tau_{\alpha}\colon I\setminus\{ \alpha \}&\to \eR \\
        x&\mapsto \frac{ f(x)-f(\alpha) }{ x-\alpha }. 
    \end{aligned}
\end{equation}

\begin{proposition}[Inégalité des pentes\cite{OJIMBtv}] \label{PropMDMGjGO}
    Soit \( f\) une fonction convexe sur un intervalle \( I\subset \eR\). Alors pour tout \( a<b<c\) dans \( I\) nous avons\footnote{Les inégalités sont strictes si la fonction \( f\) est strictement convexe.}
    \begin{equation}
        \frac{ f(b)-f(a)  }{ b-a }\leq\frac{ f(c)-f(a) }{ c-a }\leq \frac{ f(c)-f(b) }{ c-b }.
    \end{equation}
    En d'autres termes,
    \begin{equation}
        \tau_a(b)\leq\tau_a(c)\leq \tau_b(c),
    \end{equation}
    c'est à dire que \( \tau\) est croissante en ses deux arguments.
\end{proposition}
\index{Inégalité!des pentes}

\begin{proof}
    D'abord les inégalités \( a<b<c\) impliquent \( 0<b-a<c-a\) et donc
    \begin{equation}
        \lambda=\frac{ b-a }{ c-a }<1.
    \end{equation}
    L'astuce est de remarquer que \( (1-\lambda)a+\lambda c=b\). Donc \( \lambda\) a toutes les bonnes propriétés pour être utilisé dans la définition de la convexité :
    \begin{equation}
        f\big( (1-\lambda)a+\lambda c \big)\leq \lambda f(c)+(1-\lambda)f(a),
    \end{equation}
    c'est à dire
    \begin{equation}
        f(b)-f(a)\leq \lambda\big( f(c)-f(a) \big)
    \end{equation}
    ou encore, en remplaçant \( \lambda\) par sa valeur :
    \begin{equation}
        \frac{ f(b)-f(a) }{ b-a }\leq \frac{ f(c)-f(a) }{ c-a }.
    \end{equation}
    Cela fait déjà une des inégalités à savoir.
    
    D'autre part en partant de \( -a<-b<-c\) nous posons
    \begin{equation}
        0<\lambda=\frac{ c-b }{ c-a }.
    \end{equation}
    Nous avons à nouveau \( b=(1-\lambda)c+\lambda a\) et nous pouvons obtenir la seconde inégalité
    \begin{equation}
        \frac{ f(c)-f(a) }{ c-a }\leq \frac{ f(c)-f(b) }{ c-b }.
    \end{equation}
\end{proof}

%--------------------------------------------------------------------------------------------------------------------------- 
\subsection{Convexité et régularité}
%---------------------------------------------------------------------------------------------------------------------------

\begin{proposition}[\cite{MonCerveau}] \label{PropYKwTDPX}
    Une fonction dérivable sur un intervalle \( I\) de \( \eR\) est (strictement) convexe si et seulement si sa dérivée est (strictement) croissante sur \( I\).
\end{proposition}

\begin{proof}
    \begin{subproof}
        \item[Sens réciproque]
    Nous commençons par supposer que \( f'\) est strictement croissante. Soient \( a,b\in I\) et \( y(x)\) la droite passant par les points \( \big( a,f(a) \big)\) et \( b,f(b)\); notre but est de prouver que \( y(x)>f(x)\) pour tout \( x\in\mathopen] a , b \mathclose[\). Nous allons commencer par montrer que l'équation \( f(x)=y(x)\) n'admet pas de solutions à l'extérieur de \( \mathopen[ a , b \mathclose]\). Nous notons \( \alpha\) le coefficient directeur de la droite \( y\); par le théorème de Rolle \ref{ThoRolle}, il existe \( c\in\mathopen] a , b \mathclose[\) tel que \( f'(c)=\alpha\). Vu que \( f'\) est strictement croissante,
    \begin{subequations}
        \begin{numcases}{}
            f'(a)<\alpha\\
            f'(b)>\alpha.
        \end{numcases}
    \end{subequations}
    Si nous posons \( g(x)=f(x)-y(x)=f'(x)-\alpha\), nous avons \( g'(x)<0\) pour \( x<a\), de telle sorte que \( g\) soit strictement décroissante\footnote{Proposition \ref{PropGFkZMwD}.} pour \( x<a\); étant donné que \( g(a)=0\), cela signifie que \( g\) ne s'annule pas avant \( a\). Idem pour \( x>b\). Il n'y a donc pas de solutions à \( f(x)=y(x)\) à l'extérieur de \( \mathopen[ a , b \mathclose]\). Il n'y en a donc pas non plus à l'intérieur de \( \mathopen[ a , b \mathclose]\) parce que si \( a<r<b\) étaient des solutions, ce que nous venons de faire, appliqué aux solutions \( r\) et \( b\), exclu que \( a\) soit une solution.

    Étant donné que \( f\) et \( y\) sont des fonctions continues (parce que \( f\) est dérivable, proposition \ref{PropSFyxOWF}), il suffit de montrer que \( f(x)<y(x)\) pour un \( x\) entre \( a\) et \( b\) pour obtenir l'inégalité pour tout \( x\) entre \( a\) et \( b\). La fonction \( f\) étant dérivable, nous avons
    \begin{equation}
        f'(a)=\lim_{h\to 0} \frac{ f(a+h)-f(a) }{ h }<\alpha.
    \end{equation}
    Nous allons nous intéresser à cette limite pour \( h>0\) et en particulier il existe un \( \epsilon>0\) tel que pour tout \( 0<h<\epsilon\) nous ayons
    \begin{equation}
        f(a+h)<\alpha h+f(a)=y(a+h).
    \end{equation}
    Donc oui, sur un intervalle après \( a\) nous avons bien que la fonction est plus basse que la droite.
    
\item[Sens direct]

    Nous montrons maintenant que si \( f\) est strictement convexe et dérivable, alors sa dérivée est strictement croissante. Nous prenons donc \( a<b\) et \( x\in\mathopen] a , b \mathclose[\). L'inégalité des pentes (proposition \ref{PropMDMGjGO}) donne
    \begin{equation}
        \frac{ f(x)-f(a) }{ x-a }<\frac{ f(b)-f(a) }{ b-a }<\frac{ f(b)-f(x) }{ b-x }.
    \end{equation}
    Étant donné que \( f\) est dérivable en \( a\) nous avons parfaitement le droite de prendre la limite \( x\to a\) dans la première inégalité et \( x\to b\) dans la seconde pour obtenir 
    \begin{equation}
        f'(a)<\frac{ f(b)-f(a) }{ b-a }<f'(b).
    \end{equation}
    Cela prouve que \( f'\) est croissante.

    \end{subproof}
\end{proof}

\begin{theorem}[\cite{RIKpeIH}] \label{ThoGXjKeYb}
    Une fonction \( f\) de classe \( C^2\) est convexe si et seulement si \( f''\) est positive. (et ajouter les «strictement» là où ça va)
\end{theorem}

\begin{proof}
    La fonction est \( C^2\), donc \( f''\) est positive si et seulement si \( f'\) est croissante (proposition \ref{PropGFkZMwD}) alors que la proposition \ref{PropYKwTDPX} nous jure que \( f\) sera convexe si et seulement si \( f'\) est croissante.
\end{proof}

\begin{definition}
    Une fonction est \defe{concave}{concave} si son opposée est convexe.
\end{definition}

\begin{example} \label{ExPDRooZCtkOz}
    Quelque exemples utilisant le théorème \ref{ThoGXjKeYb}
    \begin{enumerate}
        \item
    La fonction \( x\mapsto x^2\) est convexe parce que sa dérivée seconde est la constante (positive) \( 2\).
\item La fonction \( x\mapsto\frac{1}{ x }\) est convexe sur \( \eR^+\setminus\{ 0 \}\) (sa dérivée seconde est \( 2x^{-3}\)).
\item
    La fonction exponentielle est également convexe.
\item
    La fonction \( \ln\) est concave parce que la dérivée seconde de \( -\ln\) est \( \frac{1}{ x^2 }\) qui est strictement positif.
    \end{enumerate}
\end{example}

\begin{lemma}[\cite{GYfviRu}]   \label{LemKLTsHIQ}
    Une fonction convexe sur un ouvert
    \begin{enumerate}
        \item
            y admet des dérivées à gauche et à droite en chaque point,
        \item
            y est continue.
    \end{enumerate}
\end{lemma}

\begin{proof}
    Soit \( I=\mathopen] a , b \mathclose[\) un intervalle sur lequel \( f\) est convexe et \( \alpha\in I\). Nous allons prouver que \( f\) est continue en \( \alpha\). Nous considérons \( \tau_{\alpha}\) le taux d'accroissement définit par \eqref{EqRYBazWd}; c'est une fonction croissante comme précisé dans l'inégalité des trois pentes \ref{PropMDMGjGO} et de plus \( \tau_{\alpha}(x)\) est bornée supérieurement par \( \tau_{\alpha}(b)\) pour \( x<\alpha\) et inférieurement par \( \tau_{\alpha}(a)\) pour \( x>\alpha\). Les limites existent donc et sont finies par la proposition \ref{PropMTmBYeU}. Autrement dit les limites
        \begin{subequations}
            \begin{align}
                \lim_{x\to \alpha+} \frac{ f(x)-f(\alpha) }{ x-\alpha }\\
                \lim_{x\to \alpha^-} \frac{ f(x)-f(\alpha) }{ x-\alpha }
            \end{align}
        \end{subequations}
        existent et sont finies, c'est à dire que la fonction \( f\) admet une dérivée à gauche et à droite.

        Pour tout \( x\) nous avons les inégalités
        \begin{equation}
            \tau_{\alpha}(a)\leq \frac{ f(x)-f(\alpha) }{ x-\alpha }\leq \tau_{\alpha}(b).
        \end{equation}
        En posant \( k=\max\{ \tau_{\alpha}(a),\tau_{\alpha}(b) \}\) nous avons
        \begin{equation}
            \big| f(x)-f(\alpha) \big|\leq k| x-\alpha |.
        \end{equation}
        La fonction est donc Lipschitzienne et par conséquent continue par la proposition \ref{PropFZgFTEW}.
\end{proof}

\begin{remark}
    Les dérivées à gauche et à droite ne sont a priori pas égales. Penser par exemple à une fonction affine par morceaux dont les pentes augmentent à chaque morceau.
\end{remark}

\begin{lemma}[\cite{KXjFWKA}]   \label{LemXOUooQsigHs}
    L'application
    \begin{equation}
        \begin{aligned}
            \phi\colon S^{++}(n,\eR)&\to \eR \\
            A&\mapsto \det(A) 
        \end{aligned}
    \end{equation}
    est \defe{log-convave}{concave!log-concave}\index{log-concave}, c'est à dire que l'application \( \ln\circ\phi\) est concave. De façon équivalente, si \( A,B\in S^{++}\) et si \( \alpha+b=1\), alors
    \begin{equation}    \label{EqSPKooHFZvmB}
        \det(\alpha A+\beta B)\geq \det(A)^{\alpha}\det(B)^{\beta}.
    \end{equation}
\end{lemma}
Ici \( S^{++}\) est l'ensemble des matrices symétriques strictement définies positives, définition \ref{DefAWAooCMPuVM}.

\begin{proof}
    Nous commençons par prouver que l'équation \eqref{EqSPKooHFZvmB} est équivalente à la log-concavité du déterminant. Pour cela il suffit de remarquer que les propriétés de croissance et d'additivité du logarithme donnent l'équivalence entre
    \begin{equation}
        \ln\Big( \det(\alpha A+\beta B) \Big)\geq \ln\Big( \det(\alpha A) \Big)+\ln\Big( \det(\beta B) \Big),
    \end{equation}
    et
    \begin{equation}    \label{EqTJYooBWiRrn}
        \det(\alpha A+\beta B)\geq \det(A)^{\alpha}\det(B)^{\beta}.
    \end{equation}

    Le théorème de pseudo-réduction simultanée, corollaire \ref{CorNHKnLVA}, appliqué aux matrices \( A\) et \( B\) nous donne une matrice inversible \( Q\) telle que
    \begin{subequations}
        \begin{numcases}{}
            B=Q^tDQ\\
            A=Q^tQ
        \end{numcases}
    \end{subequations}
    avec 
    \begin{equation}
        D=\begin{pmatrix}
            \lambda_1    &       &       \\
                &   \ddots    &       \\
                &       &   \lambda_n
        \end{pmatrix},
    \end{equation}
    \( \lambda_i>0\). Nous avons alors
    \begin{equation}
        \det(A)^{\alpha}\det(B)^{\beta}=\det(Q)^{2\alpha}\det(Q)^{2\beta}\det(D)^{\beta}=\det(Q)^2\det(D)^{\beta}
    \end{equation}
    (parce que \( \alpha+\beta=1\)) et
    \begin{equation}
        \det(\alpha A+\beta B)=\det(\alpha Q^tQ+\beta Q^tDQ)=\det\big( Q^t(\alpha\mtu+\beta D)Q \big)=\det(Q)^2\det(\alpha\mtu+\beta D).
    \end{equation}
    L'inégalité \eqref{EqTJYooBWiRrn} qu'il nous faut prouver se réduit donc  à
    \begin{equation}
        \det(\alpha \mtu+\beta D)\geq \det(D)^{\beta}.
    \end{equation}
    Vue la forme de \( D\) nous avons
    \begin{equation}
        \det(\alpha\mtu+\beta D)=\prod_{i=1}^n(\alpha+\beta\lambda_i)
    \end{equation}
    et 
    \begin{equation}
        \det(D)^{\beta}=\big( \prod_{i=1}^{n}\lambda_i \big)^{\beta}.
    \end{equation}
    Il faut donc prouver que
    \begin{equation}\label{EqGFLooOElciS}    
        \prod_{i=1}^n(\alpha+\beta\lambda_i)\geq \big( \prod_{i=1}^n\lambda_i \big)^{\beta}.
    \end{equation}
    Cette dernière égalité de produit sera prouvée en passant au logarithme. Vu que le logarithme est concave par l'exemple \ref{ExPDRooZCtkOz}, nous avons pour chaque \( i\) que
    \begin{equation}
        \ln(\alpha+\beta\lambda_i)\geq \alpha\ln(1)+\beta\ln(\lambda_i)=\beta\ln(\lambda_i).
    \end{equation}
    En sommant cela sur \( i\) et en utilisant les propriétés de croissance et de multiplicativité du logarithme nous obtenons successivement
    \begin{subequations}
        \begin{align}
            \sum_{i=1}^n\ln(\alpha+\beta\lambda_i)\geq \beta\sum_i\ln(\lambda_i)\\
            \ln\big( \prod_i(\alpha+\beta\lambda_i) \big)\geq\ln\Big( \big( \prod_i\lambda_i \big)^{\beta} \Big)\\
            \prod_i(\alpha+\beta\lambda_i)\geq\big( \prod_i\lambda_i \big)^{\beta},
        \end{align}
    \end{subequations}
    ce qui est bien \eqref{EqGFLooOElciS}.
\end{proof}

\begin{proposition}[\cite{CLTooTlwZoz}] \label{PropHRLooTqIJPS}
    Si \( f\colon \eR^n\to \eR\) est de classe \( C^2\), elle est convexe si et seulement si sa matrice hessienne est définie positive pour tout \( x\).
\end{proposition}

\begin{proposition}[\cite{MonCerveau}] \label{PropPEJCgCH}
    Si \( g\) est une fonction convexe, il existe deux suites réelles \( (a_n)\) et \( (b_n)\) telles que
    \begin{equation}
        g(x)=\sup_{n\in \eN}(a_nx+b_n).
    \end{equation}
\end{proposition}
\index{fonction!convexe}
\index{densité!de \( \eQ\) dans \( \eR\)!utilisation}

\begin{proof}
    Pour \( u\in \eR\) nous considérons \( a(u)\) et \( b(u)\) tels que la droite \( y(x)=a(u)x+b(u)\) vérifie \( y(u)=g(u)\) et \( y(x)\leq g(x)\) pour tout \( x\). Il s'agit d'une droite coupant le graphe de \( g\) en \( x=u\) et restant en dessous. Nous considérons alors \( (u_n)\) une suite quelconque dense dans \( \eR\) (disons les rationnels pour fixer les idées) et nous posons
    \begin{subequations}
        \begin{numcases}{}
            a_n=a(u_n)\\
            b_n=b(u_n).
        \end{numcases}
    \end{subequations}
    Si \( q\in \eQ\) alors \( a_nx+b_n\leq g(x)\) pour tout \( n\) et \( g(q)\) est le supremum qui est atteint pour le \( n\) tel que \( u_n=q\). Si maintenant \( x\) n'est pas dans \( \eQ\) il faut travailler plus.

    Nous prenons \( (\tilde q_n)\), une sous-suite de \( (q_n)\) convergeant vers \( x\) et \( N\) suffisamment grand pour que pour tout \( n\geq N\) on ait \( | \tilde q_n-x |\leq \epsilon\) et \( | g(\tilde q_n)-g(x) |\leq \epsilon\); cela est possible grâce à la continuité de \( g\) (lemme \ref{LemKLTsHIQ}). Ensuite les sous-suites \( (\tilde a_n)\) et \( (\tilde b_n)\) sont celles qui correspondent :
    \begin{equation}
        \tilde a_n\tilde q_n+\tilde b_n=g(\tilde q_n).
    \end{equation}
    Nous considérons la majoration
    \begin{subequations}
        \begin{align}
            | \tilde a_nx+\tilde b_n-g(x) |&\leq| \tilde a_nx+\tilde b_n-(\tilde a_n\tilde q_n+\tilde b_n) |+\underbrace{| \tilde a_n\tilde q_n+\tilde b_n-g(\tilde q_n) |}_{=0}+\underbrace{| g(\tilde q_n)-g(x) |}_{\leq \epsilon}\\
            &\leq | \tilde a_n | |x-\tilde q_n |+\epsilon\\
            &=\epsilon\big( | \tilde a_n |+1 \big).
        \end{align}
    \end{subequations}
    Il nous reste à montrer que \( | \tilde a_n |\) est borné par un nombre ne dépendant pas de \( n\) (pour les \( n>N\)).

    Vu que la droite de coefficient directeur \( \tilde a_n\) et passant par le point \( \big( \tilde q_n,g(\tilde q_n) \big)\) reste en dessous du graphe de \( g\), nous avons pour tout \( n\) et tout \( y\in \eR\) l'inégalité
    \begin{equation}
        g(y)\geq \tilde a_n(y-\tilde q_n)+g(\tilde q_n)\in \tilde a_nB(y-x,\epsilon)+B\big( g(x),\epsilon \big).
    \end{equation}
    Si \( \tilde a_n\) n'est pas borné vers le haut, nous prenons \( y\) tel que \( B(y-x,\epsilon)\) soit minoré par un nombre \( k\) strictement positif et nous obtenons
    \begin{equation}
        g(y)\geq k\tilde a_n+l
    \end{equation}
    avec \( k\) et \( l\) indépendants de \( n\). Cela donne \( g(y)=\infty\). Si au contraire \( \tilde a_n\) n'est pas borné vers le bas, nous prenons $y$ tel que \( B(y-x,\epsilon)\) est majoré par un nombre \( k\) strictement négatif. Nous obtenons encore \( g(y)=\infty\).

    Nous concluons que \( | \tilde a_n |\) est bornée.
\end{proof}

%--------------------------------------------------------------------------------------------------------------------------- 
\subsection{Quelque inégalités}
%---------------------------------------------------------------------------------------------------------------------------

%///////////////////////////////////////////////////////////////////////////////////////////////////////////////////////////
\subsubsection{Inégalité de Jensen}
%///////////////////////////////////////////////////////////////////////////////////////////////////////////////////////////

\begin{proposition}[Inégalité de Jensen]    \label{PropXIBooLxTkhU}
    Soit \( f\colon \eR\to \eR\) une fonction convexe et des réels \( x_1\),\ldots,  \( x_n\). Soient des nombres positifs \( \lambda_1\),\ldots,  \( \lambda_n\) formant une combinaison convexe\footnote{Définition \ref{DefIMZooLFdIUB}.}. Alors
    \begin{equation}
        f\big( \sum_i\lambda_ix_i \big)\leq \sum_i\lambda_if(x_i).
    \end{equation}
\end{proposition}
\index{inégalité!Jensen}

\begin{proof}
    Nous procédons par récurrence sur \( n\), en sachant que \( n=2\) est la définition de la convexité de \( f\). Vu que
    \begin{equation}
        \sum_{k=1}^n\lambda_kx_k=\lambda_nx_n+(1-\lambda_n)\sum_{k=1}^{n-1}\frac{ \lambda_kx_k }{ 1-\lambda_n },
    \end{equation}
    nous avons
    \begin{equation}
        f\big( \sum_{k=1}^n\lambda_kx_k \big)\leq \lambda_nf(x_n)+(1-\lambda_n)f\big( \sum_{k=1}^{n-1}\frac{ \lambda_kx_k }{ 1-\lambda_n } \big).
    \end{equation}
    La chose à remarquer est que les nombres \( \frac{ \lambda_k }{ 1-\lambda_n }\) avec \( k\) allant de \( 1\) à \( n-1\) forment eux-mêmes une combinaison convexe. L'hypothèse de récurrence peut donc s'appliquer au second terme du membre de droite :
    \begin{equation}
        f\big( \sum_{k=1}^n\lambda_kx_k \big)\leq \lambda_nf(x_n)+(1-\lambda_n)\sum_{k=1}^{n-1}\frac{ \lambda_k }{ 1-\lambda_n }f(x_k)=\lambda_nf(x_n)+\sum_{k=1}^{n-1}\lambda_kf(x_k).
    \end{equation}
\end{proof}

%///////////////////////////////////////////////////////////////////////////////////////////////////////////////////////////
\subsubsection{Inégalité arithmético-géométrique}
%///////////////////////////////////////////////////////////////////////////////////////////////////////////////////////////

La proposition suivante dit que la moyenne arithmétique de nombres strictement positifs est supérieure ou égale à la moyenne géométrique.
\begin{proposition}[Inégalité arithmético-géométrique\cite{CENooZKvihz}]    \label{PropWDPooBtHIAR}
    Soient \( x_1\),\ldots, \( x_n\) des nombres strictement positifs. Nous posons
    \begin{equation}
        m_a=\frac{1}{ n }(x_1+\ldots +x_n)
    \end{equation}
    et
    \begin{equation}
        m_g=\sqrt[n]{x_1\ldots x_n}
    \end{equation}
    Alors \( m_g\leq m_a\) et \( m_g=m_a\) si et seulement si \( x_i=x_j\) pour tout \( i,j\).
\end{proposition}
\index{inégalité!arithmético-géométrique}

\begin{proof}
    Par hypothèse les nombres \( m_a\) et \( m_g\) sont tout deux strictement positifs, de telle sorte qu'il est équivalent de prouver \( \ln(m_g)\leq \ln(m_a)\) ou encore
    \begin{equation}
        \frac{1}{ n }\big( \ln(x_1)+\ldots +\ln(x_n) \big)\leq \ln\left( \frac{ x_1+\ldots +x_n }{ n } \right).
    \end{equation}
    Cela n'est rien d'autre que l'inégalité de Jensen de la proposition \ref{PropXIBooLxTkhU} appliquée à la fonction \( \ln\) et aux coefficients \( \lambda_i=\frac{1}{ n }\).
\end{proof}

%///////////////////////////////////////////////////////////////////////////////////////////////////////////////////////////
\subsubsection{Inégalité de Kantorovitch}
%///////////////////////////////////////////////////////////////////////////////////////////////////////////////////////////

\begin{proposition}[Inégalité de Kantorovitch\cite{EYGooOoQDnt}]    \label{PropMNUooFbYkug}
    Soit \( A\) une matrice symétrique strictement définie positive dont les plus grandes et plus petites valeurs propres sont \( \lambda_{min}\) et \( \lambda_{max}\). Alors pour tout \( x\in \eR^n\) nous avons
    \begin{equation}
        \langle Ax, x\rangle \langle A^{-1}x, x\rangle \leq \frac{1}{ 4 }\left( \frac{ \lambda_{min} }{ \lambda_{max} }+\frac{ \lambda_{max} }{ \lambda_{min} } \right)^2\| x^4 \|.
    \end{equation}
\end{proposition}
\index{inégalité!Kantorovitch}

\begin{proof}
    Sans perte de généralité nous pouvons supposer que \( \| x \|=1\). Nous diagonalisons\footnote{Théorème spectral \ref{ThoeTMXla}.} la matrice \( A\) par la matrice orthogonale  \( P\in\gO(n,\eR)\) : \( A=PDP^{-1}\) et \( A^{-1}=PD^{-1}P^{-1}\) où \( D\) est  une matrice diagonale formée des valeurs propres de \( A\).

    Nous posons \( \alpha=\sqrt{\lambda_{min}\lambda_{max}}\) et nous regardons la matrice
    \begin{equation}
        \frac{1}{ \alpha }A+tA^{-1}
    \end{equation}
    dont les valeurs propres sont
    \begin{equation}
        \frac{ \lambda_i }{ \alpha }+\frac{ \alpha }{ \lambda_i }
    \end{equation}
    parce que les vecteurs propres de \( A\) et de \( A^{-1}\) sont les mêmes (ce sont les valeurs de la diagonale de \( D\)). Nous allons quelque peu étudier la fonction
    \begin{equation}
        \theta(x)=\frac{ x }{ \alpha }+\frac{ \alpha }{ x }.
    \end{equation}
    Elle est convexe en tant que somme de deux fonctions convexes. Elle a son minimum en \( x=\alpha\) et ce minimum vaut \( \theta(\alpha)=2\). De plus
    \begin{equation}
        \theta(\lambda_{max})=\theta(\lambda_{min})=\sqrt{\frac{ \lambda_{min} }{ \lambda_{max} }}+\sqrt{\frac{ \lambda_{max} }{ \lambda_{min} }}.
    \end{equation}
    Une fonction convexe passant deux fois par la même valeur doit forcément être plus petite que cette valeur entre les deux\footnote{Je ne suis pas certain que cette phrase soit claire, non ?} : pour tout \( x\in\mathopen[ \lambda_{min} , \lambda_{max} \mathclose]\),
    \begin{equation}
        \theta(x)\leq  \sqrt{\frac{ \lambda_{min} }{ \lambda_{max} }}+\sqrt{\frac{ \lambda_{max} }{ \lambda_{min} }}.
    \end{equation}
    
    Nous sommes maintenant en mesure de nous lancer dans l'inégalité de Kantorovitch.
    \begin{subequations}
        \begin{align}
            \sqrt{\langle Ax, x\rangle \langle A^{-1}x, x\rangle }&\leq\frac{ 1 }{2}\left( \frac{ \langle Ax, x\rangle  }{ \alpha }+\alpha\langle A^{-1}x, x\rangle  \right)\label{subEqUKIooCWFSkwi}\\
            &=\frac{ 1 }{2}\langle   \big( \frac{ A }{ \alpha }+\alpha A^{-1} \big)x , x\rangle \\
            &\leq\frac{ 1 }{2}\Big\| \big( \frac{ A }{ \alpha }+\alpha A^{-1} \big)x \|\| x \| \label{subEqUKIooCWFSkwiii}\\
            &\leq \frac{ 1 }{2}\| \frac{ A }{ \alpha }+\alpha A^{-1} \| \label{subEqUKIooCWFSkwiv}
        \end{align}
    \end{subequations}
    Justifications :
    \begin{itemize}
        \item \ref{subEqUKIooCWFSkwi} par l'inégalité arithmético-géométrique, proposition \ref{PropWDPooBtHIAR}. Nous avons aussi inséré \( \alpha\frac{1}{ \alpha }\) dans le produit sous la racine.
        \item \ref{subEqUKIooCWFSkwiii} par l'inégalité de Cauchy-Schwarz, théorème \ref{ThoAYfEHG}.
        \item \ref{subEqUKIooCWFSkwiv} par la définition de la norme opérateur de la proposition \ref{PropNormeAppLineaire}
    \end{itemize}
    La norme opérateur est la plus grande des valeurs propres. Mais les valeurs propres de \( A/\alpha+\alpha A^{-1}\) sont de la forme \( \theta(\lambda_i)\), et tous les \( \lambda_i\) sont entre \( \lambda_{min} \) et \( \lambda_{max}\). Donc la plus grande valeur propre de \( A/\alpha+\alpha A^{-1}\) est \( \theta(x)\) pour un certain \( x\in\mathopen[ \lambda_{min} , \lambda_{max} \mathclose]\). Par conséquent
    \begin{equation}
            \sqrt{\langle Ax, x\rangle \langle A^{-1}x, x\rangle }\leq \frac{ 1 }{2}\| \frac{ A }{ \alpha }+\alpha A^{-1} \| \leq \sqrt{\frac{ \lambda_{min} }{ \lambda_{max} }}+\sqrt{\frac{ \lambda_{max} }{ \lambda_{min} }}.
    \end{equation}
\end{proof}

%+++++++++++++++++++++++++++++++++++++++++++++++++++++++++++++++++++++++++++++++++++++++++++++++++++++++++++++++++++++++++++
\section{Suites de fonctions}
%+++++++++++++++++++++++++++++++++++++++++++++++++++++++++++++++++++++++++++++++++++++++++++++++++++++++++++++++++++++++++++

%---------------------------------------------------------------------------------------------------------------------------
\subsection{Convergence de suites de fonctions}
%---------------------------------------------------------------------------------------------------------------------------

Nous considérons un espace normé \( (\Omega,\| . \|)\). Nous disons qu'une suite de fonctions \( f_n\) \defe{converge}{convergence!en norme} vers \( f\) pour la norme \( \| . \|\) si \( \forall \epsilon>0\), \( \exists N\) tel que \( n\geq N\) implique \( \| f_n-f \|<\epsilon\).

Dans le cas particulier de la norme 
\begin{equation}
    \| f \|_{\infty}=\sup_{x\in\Omega}| f(x) |,
\end{equation}
nous parlons que \defe{convergence uniforme}{convergence!uniforme!suite de fonctions}.

\begin{theorem}[Critère de Cauchy]  \label{ThoCauchyZelUF}
    Une suite de fonctions  \( (f_n)_{n\in\eN}\) sur \( \Omega\) converge en norme sur \( \Omega\) si et seulement si \( \forall\epsilon>0\), \( \exists N\) tel que
    \begin{equation}
        \| f_n-f_m \|<\epsilon
    \end{equation}
    pour \( n,m>N\).
\end{theorem}

\begin{corollary}       \label{CorCauchyCkXnvY}
    La série \( \sum f_n\) converge en norme sur \( \Omega\) si et seulement si \( \exists N\) tel que
    \begin{equation}
        \| f_n+\ldots+f_m \|\leq \epsilon
    \end{equation}
    pour tout \( n,m>N\).
\end{corollary}

\begin{proof}
    L'hypothèse montre que la suite des sommes partielles de la série \( \sum f_n\) vérifie le critère de Cauchy du théorème \ref{ThoCauchyZelUF}.
\end{proof}

%--------------------------------------------------------------------------------------------------------------------------- 
\subsection{Convergence uniforme}
%---------------------------------------------------------------------------------------------------------------------------

\begin{definition}[\cite{TrenchRealAnalisys}]
    Nous disons qu'une suite de fonctions \( (f_n)\) définies sur un ensemble \( A\) \defe{converge uniformément}{convergence!uniforme} vers une fonction \( f\) si
    \begin{equation}
        \lim_{n\to \infty} \| f_n-f \|_A=0
    \end{equation}
    où \( \| g \|_A=\sup_{x\in A}\| g(x) \|\).
\end{definition}

\begin{proposition}[Critère de Cauchy uniforme\cite{LCbyNWQ}]   \label{PropNTEynwq}
    Soit \( X\) un espace topologique et \( (Y,d)\) un espace topologique complet. La suite de fonction \( f_n\colon X\to Y\) converge uniformément sur \( A\) si et seulement si pour tout \( \epsilon>0\) il existe \( N\in \eN\) tel que si \( k,l>N\) alors
    \begin{equation}
        d\big( f_k(x),f_l(x) \big)\leq \epsilon
    \end{equation}
    pour tout \( x\in X\).
\end{proposition}
\index{Cauchy!critère!uniforme}
\index{critère!Cauchy!uniforme}
Grosso modo, cela dit que si qu'une suite de Cauchy pour la norme uniforme est une suite uniformément convergente. Le fait que la suite converge fait partie du résultat et n'est pas une hypothèse. Ce critère sera utilisé pour montrer que \( \big( C(K),\| . \|_{\infty} \big)\) est complet, proposition \ref{PropSYMEZGU}. 

\begin{proof}
    Si \( f_n\stackrel{unif}{\longrightarrow}f\) alors le critère est satisfait; c'est dans l'autre sens que la preuve est intéressante.

    Soit donc une suite de fonctions satisfaisant au critère et montrons qu'elle converge uniformément. Pour tout \( x\in X\) la suite \( n\mapsto f_n(x)\) est de Cauchy dans l'espace complet \( Y\); nous avons donc convergence ponctuelle \( f_n\to f\). Nous devons prouver que cette convergence est uniforme. Soit \( \epsilon>0\) et \( N\in \eN\) tel que si \( k,l>N\) alors
    \begin{equation}
        d\big( f_k(x),f_l(x) \big)\leq \epsilon
    \end{equation}
    pour tout \( x\in X\). Si nous nous fixons un tel \( k\) et un \( x\in A\) nous considérons l'inégalité
    \begin{equation}
        d\big( f_k(x),f_l(x) \big)\leq \epsilon
    \end{equation}
    qui est vraie pour tout \( l\). En passant à la limite \( l\to\infty\) (limite qui commute avec la fonction distance par définition de la topologie) nous avons
    \begin{equation}
        d\big( f_k(x),f(x) \big)\leq \epsilon.
    \end{equation}
    Cette inégalité étant valable pour tout \( x\in X\), cela signifie que \( f_n\stackrel{unif}{\longrightarrow}f\).
\end{proof}

\begin{theorem}[Limite uniforme de fonctions continues]			\label{ThoUnigCvCont}
    Soit \( A\), un ensemble mesuré et \( f_n\colon A\to \eR^n\), une suite de fonctions continues convergeant uniformément vers \( f\). Si les fonctions \( f_n\) sont toutes continues en \( x_0\in A\), alors \( f\) est continue en \( x_0\).
\end{theorem}

\begin{proof}
    Soit \( \epsilon>0\). Si \( x\in A\) nous avons, pour tout \( n\), la majoration
    \begin{subequations}
        \begin{align}
            \| f(x)-f(x_0) \|&\leq \| f(x)-f_n(x) \|+\| f_n(x)-f_n(x_0) \|+\| f_n(x_0)-f(x_0) \|\\
            &\leq\| f_n(x)-f_n(x_0) \|+2\| f_n-f \|_{\infty}.
        \end{align}
    \end{subequations}
    Grâce à l'uniforme convergence, nous considérons \(N\in \eN\) tel que \( \| f_n-f \|\leq \epsilon\) pour tout \( n\geq N\). Pour de tels \( n\), nous avons
    \begin{equation}
        \| f(x)-f(x_0) \|\leq 2\epsilon\| f_n-f \|+\| f_n(x)-f_n(x_0) \|.
    \end{equation}
    La continuité de \( f_n\) nous fournit un \( \delta>0\) tel que \( \| f_n(x_0)-f_n(x) \|<\epsilon\) dès que \( \| x-x_0 \|<\delta\). Pour ce \( \delta\), nous avons alors \( \| f(x)-f(x_0) \|<\epsilon\).
\end{proof}

\begin{theorem}[Théorème de Dini\cite{JIFGuct}] \label{ThoUFPLEZh}
    Soit \( D\) un espace métrique compact et une suite de fonctions \( f_n\in C(D,\eR)\) telle que
    \begin{enumerate}
        \item
            \( f_n\to g\) ponctuellement,
        \item
            \( g\in C(D,\eR)\),
        \item
            la suite \( (f_n)\) est croissante, c'est à dire que pour tout \( x\in D\) et pour tout \( n\geq 0\) nous avons \( f_{n+1}(x)\geq f_n(x)\).
    \end{enumerate}
    Alors la convergence est uniforme.
\end{theorem}
\index{convergence!uniforme!théorème de Dini}
\index{compacité!théorème de Dini}
\index{théorème!Dini}

\begin{proof}
    Soit \( x\in D\) et \( \epsilon>0\). Il existe \( N(x)\in \eN\) tel que
    \begin{equation}
        g(x)-\epsilon\leq f_{N(x)}\leq g(x).
    \end{equation}
    De plus \( g\) et \( f_{N(x)}\) sont des fonctions continues, donc il existe \( \eta(x)\) tel que si \( y\in B\big( x,\eta(x) \big)\) alors
    \begin{subequations}
        \begin{align}
            g(y)&\in B\big( g(x),\epsilon \big) \label{subEqXKjgKgv}\\
            f_{N(x)}(y)&\in B\big( f_{N(x)}(x),\epsilon \big)   \label{subEqHTiYZLd}.
        \end{align}
    \end{subequations}
    Si \( n\geq N(x)\) et si \( y\in B(x,\eta(x))\) alors nous avons les majorations
    \begin{equation}
            g(y)\geq f_n(y)
            \geq f_{N(x)}(y)
            \geq f_{N(x)}(x)-\epsilon
            \geq g(x)-2\epsilon
            \geq g(y)-3\epsilon.
    \end{equation}
    Justifications :
    \begin{multicols}{2}
        \begin{enumerate}
            \item
                Les deux première inégalités sont la croissance de la suite.
            \item
                La suivante est \eqref{subEqHTiYZLd}.
            \item
                Ensuite il y a le choix de \( N(x)\).
            \item
                Et enfin il y a \eqref{subEqXKjgKgv}.
        \end{enumerate}
    \end{multicols}
    Nous retenons que si \( x\in D\) et si \( n\geq N(x)\) alors
    \begin{equation}    \label{EqJCMktdj}
        g(y)\geq f_n(y)\geq g(y)-3\epsilon
    \end{equation}
    pour tout \( y\in B(x,\eta(x))\).

    Nous utilisons maintenant la compacité de \( D\). Pour chaque \( x\in D\) nous pouvons considérer la boule ouverte \( B\big( x,\eta(x) \big)\); ces boules recouvrent \( D\). Nous en extrayons un sous-recouvrement fini, c'est à dire un ensemble fini d'éléments \( x_1\),\ldots, \( x_K\) tels que
    \begin{equation}
        D=\bigcup_{k=1}^K B\big(x_k,\eta(x_k)\big).
    \end{equation}
    Si à ce moment vous ne comprenez pas pourquoi c'est une égalité au lieu d'une inclusion, il faut lire l'exemple \ref{ExKYZwYxn}. Considérons 
    \begin{equation}
        n\geq N=\max\{ N(x_1),\ldots, N(x_K) \}.
    \end{equation}
    Pour tout \( y\in D\) il existe \( k\in\{ 1,\ldots, K \}\) tel que \( y\in B\big( x_k,\eta(x_k) \big)\), et vu que \( n\geq N(x_k)\) nous reprenons la majoration \eqref{EqJCMktdj} :
    \begin{equation}
        g(y)\geq f_n(y)\geq g(y)-3\epsilon.
    \end{equation}
    Pour le \( n\) choisi nous avons ces inégalités pour tout \( y\in D\), c'est à dire que nous avons \( \| f_n-g \|\leq 3\epsilon\) et donc la convergence uniforme.
\end{proof}

%--------------------------------------------------------------------------------------------------------------------------- 
\subsection{Permuter avec les dérivées partielles}
%---------------------------------------------------------------------------------------------------------------------------

\begin{theorem}		\label{ThoSerUnifDerr}
	Soit $U\subset\eR^n$ ouvert, $f_k\colon U\to \eR$ et $f_k$ de classe $C^1$. Supposons que $f_k$ converge simplement vers $f$ et que $\partial_if_k$ converge uniformément sur tout compact  vers une fonction $g_i$ pour $i=1,\ldots,n$. Alors $f$ est de classe $C^1$ et $\partial_if=g_i$. De plus, $f_k$ converge vers $f$ uniformément.
\end{theorem}
\index{permutation!dérivée et limite}
%TODO : une preuve.

