% This is part of Mes notes de mathématique
% Copyright (c) 2006-2014
%   Laurent Claessens, Carlotta Donadello
% See the file fdl-1.3.txt for copying conditions.

%+++++++++++++++++++++++++++++++++++++++++++++++++++++++++++++++++++++++++++++++++++++++++++++++++++++++++++++++++++++++++++
\section{Fonctions à valeurs dans $\eR^n$}
%+++++++++++++++++++++++++++++++++++++++++++++++++++++++++++++++++++++++++++++++++++++++++++++++++++++++++++++++++++++++++++

À peu près toutes les notions que vous connaissez à propos de fonctions de $\eR$ dans $\eR$ se généralises immédiatement au cas de fonctions de $\eR$ dans $\eR^n$.

Nous disons que la fonction $f\colon \eR\to \eR^n$ est \defe{de classe $C^1$}{classe $C^1$} si chacune de ses composantes $f_i$ est de classe $C^1$ en tant que fonctions de $\eR$ dans $\eR$.

La dérivée de $f$ est donnée par la dérivée composante par composante. Pour l'intégrale de $f$, il en va de même : composante par composante. 
\begin{equation}
	\int f(x)dx=\big(  \int f_1(x)dx,\,\int f_2(x)dx,\ldots,\int f_n(x)dx   \big).
\end{equation}

Par exemple si nous considérons le mouvent d'une particule sur une hélice, la position est donnée par
\begin{equation}
	f(t)=\big( R\sin(t),R\cos(t),t \big).
\end{equation}
La vitesse est donnée par
\begin{equation}
	f'(t)=\big( R\cos(t),-R\sin(t),1 \big),
\end{equation}
et l'intégrale sera donnée par
\begin{equation}
	\int f(t)dt=\big( -R\cos(t)+C_1,R\sin(t)+C_2,\frac{ t^2 }{ 2 }+C_3 \big).
\end{equation}

Si nous considérons une pierre lancée horizontalement du sommet d'une falaise avec une vitesse initiale $v_0$, la vitesse de la pierre sera donnée par
\begin{equation}
	v(t)=(v_0,gt).
\end{equation}
Pour trouver la position, nous intégrons la vitesse par rapport au temps :
\begin{equation}
	f(t)=\int v(t)dt=\big( v_0t+C_1,\frac{ gt^2 }{ 2 }+C_2 \big).
\end{equation}
Notez qu'il faut une constante d'intégration différente pour chaque composantes.

\begin{lemma}			\label{LemIneqnormeintintnorm}
	Pour toute fonction $u\colon \mathopen[ a , b \mathclose]\to \eR^n$, nous avons
	\begin{equation}
		\| \int_a^bu(t)dt\|\leq\int_a^b\| u(t) \|dt
	\end{equation}
	pourvu que le membre de gauche ait un sens.
\end{lemma}

\begin{proof}
	Étant donné que $\int_a^bu(t)dt$ est un élément de $\eR^n$, par la proposition \ref{LemSclNormeXi}, il existe un $\xi\in\eR^n$ de norme $1$ tel que
	\begin{equation}
		\| \int_a^bu(t)dt \|=\xi\cdot\int_a^b u(t)dt=\int_a^b u(t)\cdot\xi dt\leq\int_a^b\| u(t) \|   \| \xi \|=\int_a^b\| u(t) \|dt.
	\end{equation}
\end{proof}
% This is part of Géométrie analytique
% Copyright (c) 2010-2011
%   Laurent Claessens
%   Carlotta Donadello
% See the file fdl-1.3.txt for copying conditions.

%+++++++++++++++++++++++++++++++++++++++++++++++++++++++++++++++++++++++++++++++++++++++++++++++++++++++++++++++++++++++++++
\section{Graphes de fonctions de plusieurs variables}		\label{SecGraphesFonc}
%+++++++++++++++++++++++++++++++++++++++++++++++++++++++++++++++++++++++++++++++++++++++++++++++++++++++++++++++++++++++++++

La plus grande partie de ce cours est consacrée à l'étude des fonction de plusieurs variables. Nous allons maintenant donner quelques indication sur comment <<dessiner>> une telle fonction. Vous connaissez déjà la définition de graphe pour une fonction $f$ d'une seule variable à valeurs dans $\eR$ : c'est l'ensemble des point du plan de la forme $(x, f(x))$. Vous voyez que cet ensemble n'est pas vraiment un gros morceau de $\eR^2$ parce que son intérieur est vide : il y a une seule valeur de $f$ qui correspond au point $x$, donc une boule de $\eR^2$ centrée en $(x, f(x))$ de n'importe quel rayon contient toujours des points qui ne font pas partie du graphe de $f$. 

%La première chose qu'on a envie de dire est que un tel graphe est une courbe dans $\eR^2$ mais cela n'est pas toujours vrai. Le graphe de la fonction cosinus est bien une courbe dans dans le plan, mais le graphe de la fonction tangente est une réunion infinie de courbes. Ce qui est vrai est que le graphe d'une fonction d'une variable est \emph{localement} une courbe si la fonction n'est pas trop mal choisie. % exemple? 

Nous voulons donner une définition assez générale pour le graphe d'une fonction
\begin{definition}
  Soit $f$ une fonction de $\eR^m$ dans $\eR^n$. Le \defe{graphe}{graphe!fonction} de $f$ est la partie de $\eR^m\times \eR^n$ de la forme
  \begin{equation}
    \Graph f= \{ (x,y)\in \eR^m\times \eR^n \,|\, y=f(x)\}.
  \end{equation}
\end{definition}
Si $f$ est une fonction de deux variables indépendantes $x$ et $y$ à valeurs dans $\eR$, alors un point dans le graphe de $f$ est un point $(x,y,z)\in\eR^3$ tel que
\begin{equation}
	z=f(x,y),
\end{equation}
ou encore, un point de la forme
\begin{equation}
	\big( x,y,f(x,y) \big).
\end{equation}
%Si $g$ est une fonction d'une variable $x$ à valeurs dans $\eR^2$, alors un point dans le graphe de $g$ prend la forme $(x,g_1(x), g_2(x))$, où $g_1$ et $g_2$ sont les composantes de $g$.  Dans le deux cas le graphe est un sous-ensemble de $\eR^3$. 
Ici nous sommes intéressés par les fonctions de plusieurs variables à valeurs dans $\eR$. Donc, notre définition se spécialise 
\begin{definition}
  Soit $f$ une fonction de $\eR^m$ dans $\eR$. Le graphe de $f$ est la partie de $\eR^m\times \eR$ donné par
  \begin{equation}
    \Graph f= \{ (x,y)\in \eR^m\times \eR \,|\, y=f(x)\}.
  \end{equation}
\end{definition}  
Étant donné que nous ne donneront des exemples que de fonctions de $\eR^2$ dans $\eR$, la définition devient
\begin{equation}
	\Graph f= \{ (x,y,z)\in\eR^2\tq z=f(x,y) \}.
\end{equation}
C'est cette définition qu'il faut garder à l'esprit lorsqu'on travaille sur des dessins en trois dimensions.

%Nous considérons d'abord le cas d'une fonction $f$  de deux variables $x$ et $y$ à valeurs dans $\eR$. L'espace $\eR^3$ a trois dimensions, cela veut dire que il faut fixer trois paramètres indépendants pour désigner un point de manière unique (voir, au cours d'une deuxième lecture de ces notes, la section sur les coordonnées cylindriques et sphériques, \ref{sec_coord}). Le graphe d'une fonction comme $f$ est un sous-ensemble de $\eR^3$ où l'un des trois paramètres est d'office la valeur de $f$, donc il est décrit par seulement deux paramètres $x$ et $y$. Son intérieur est alors vide et, si $f$ est une fonction <<suffisamment gentille>>, $\Graph f$ est localement une surface dans $\eR^3$.    

Nous avons parfois besoin de donner des représentation graphiques d'une fonction. Nous pouvons, par exemple, penser à la fonction que associe à un point de la Terre son altitude. Lorsqu'on part pour une promenade en montagne on a envie de connaitre le graphe de cette fonction qui correspond en fait à la surface de la montagne. Bien sur nous ne voulons pas amener avec nous un modèle en 3D de la montagne donc il nous faut une méthode efficace pour projeter le graphe de $f$ sur le plan $x$-$y$ tout en gardant les informations fondamentales. Pour cela nous avons besoin de deux définitions (à ne pas confondre !)
\begin{definition}
	Soit $f$ une fonction de $\eR^2$ dans $\eR$ et soit $c$ dans $\eR$.  La \defe{$z$-section}{section!de graphe} de $\Graph f$ à la hauteur $c$ est donné par
\[
S^z_c=\{ (x,y,c)\in \eR^3\,|\, f(x,y)=c\}.
\]  
\end{definition}
\begin{definition}\label{def_niveau}
	Soit $f$ une fonction de $\eR^2$ dans $\eR$ et soit $c$ dans $\eR$. La \defe{courbe de niveau}{courbe de niveau} de $f$ à la hauteur $c$ est l'ensemble
\[
N_c=\{ (x,y)\in \eR^2\,|\, f(x,y)=c\}.
\]  
\end{definition}
On peut représenter la fonction $f$ d'une façon très précise en traçant quelques unes de ses courbes de niveau.  Dans la suite on pourra considérer aussi les $x$-sections et les $y$-sections du graphe d'une fonction de deux variables. La $x$-section de $\Graph f$ à la hauteur $a$ est     
\[
S^x_a=\{(a,y,z)\in\eR^3\,|\, f(a,y)=z\}.
\]
Comme vous avez peut être déjà compris, $S^x_a$ est le graphe de la fonction de $y$ qu'on obtient de $f$ en fixant $x=a$. Cette fonction est appelée $x$-section de $f$ pour $x=a$.

Certaines surfaces dans $\eR^3$ sont le graphe d'une fonction. 

\begin{example}
	Quelque graphes importants.
  \begin{description}
    \item[Un plan non vertical] Tout plan dans $\eR^3$ peut être décrit par une équation de la forme 
\[
a(x-x_0)+ b(y-y_0) + c(z-z_0) = r,
\] 
où, $(x_0, y_0, z_0)$ est vecteur dans $\eR^3$, et $a$, $b$, $c$ et $r$ sont des nombres réels. Si $c\neq 0$ alors le plan n'est pas vertical et on peut dire que il est le graphe de la fonction 
\[
P(x,y)= \frac{r+cz_0 -a(x-x_0)-b(y-y_0)}{c},
\]
quitte à choisir des nouvelles constantes $s$, $t$, $q$,
\[
P(x,y)=sx +ty +q.
\]
    \item[Un paraboloïde elliptique] Pour tous $\alpha$ et $\beta$ dans $\eR$ les  graphes des fonctions 
\[
PE_1(x,y)=\frac{x^2}{\alpha^2}+\frac{y^2}{\beta^2}
\]
ou de la fonction 
\[
PE_2(x,y)=-\frac{x^2}{\alpha^2}-\frac{y^2}{\beta^2}
\]
sont des paraboloïdes elliptiques. Le premier est contenu dans le demi-espace $z\geq 0$, l'autre dans $z\leq 0$. Le nom de cette surface vient de la forme de ses sections. En fait toutes  sections $S^z_c$ sont des ellipses, alors que les section $S^x_a$ et $S^y_b$ sont des paraboles.   
    \item[Un paraboloïde hyperbolique (selle)]  Pour tous $\alpha$ et $\beta$ dans $\eR$ les  graphes des fonctions 
\[
PH_1(x,y)=\frac{x^2}{\alpha^2}-\frac{y^2}{\beta^2}
\]
ou de la fonction 
\[
PH_2(x,y)=-\frac{x^2}{\alpha^2}+\frac{y^2}{\beta^2}
\]
sont des paraboloïdes hyperboliques. Remarquez que les  sections $S^z_c$ de ce graphe sont des hyperboles, alors que les section $S^x_a$ et $S^y_b$ sont des paraboles.   
    \item[Une demi-sphère] La fonction $S^+=\sqrt{R^2-x^2-y^2}$ a pour graphe la demi-sphère supérieure centrée en l'origine et de rayon $R$.  
Le dernier de ces exemples nous signale une chose très importante : une sphère entière n'est pas le graphe d'une fonction de $x$ et $y$. Par contre, une demi-sphère est bien le graphe de la fonction $f(x,y)=\sqrt{1-x^2-y^2}$.

L'équation que nous utilisons  pour d'écrire une sphère de rayon $R$ centrée en l'origine est 
\[
x^2+y^2+z^2=R^2
\] 
Donc, à  chaque point  $(x,y)$ dans le disque $x^2+y^2\leq R^2$ (notez que ce disque est contenu dans la section $S^z_0$), on peut associer deux valeurs de $z$ : $z_1=\sqrt{R^2-x^2-y^2}$ et  $z_2=-\sqrt{R^2-x^2-y^2}$. Par définition, une fonction n'associe qu'un seul valeur à chaque point de son domaine, d'où l'impossibilité de décrire cette sphère comme le graphe d'une fonction de $x$ et $y$.

  \end{description}
\end{example}

Considérons la fonction $Sp: \eR^3\to \eR$ qui associe à $(x,y,z)$ la valeur $x^2+y^2+z^2$. La sphère de rayon $R$ centrée en l'origine est l'ensemble de niveau $N_{R^2}$ de $Sp$. L'ensemble de niveau $N_{0}$ de $Sp$ est l'origine, et tous les ensemble de niveau de hauteur négative sont vides. La même chose est vraie pour les ellipsoïdes centrées en l'origine avec les axes $x$, $y$ et $z$ comme axes principaux et comme longueurs de demi-axes $a$, $b$ et $c$. Voici la fonction dont il sont les ensemble de niveau 
\[
El(x,y,z)= \frac{x^2}{a^2}+\frac{y^2}{b^2}+\frac{z^2}{c^2}.
\] 
\begin{example}
	Des ensembles de niveau importants.
  \begin{description}
    \item[Tout graphe] 
	    Le graphe de toute fonction $f$  de $\eR^2$ dans $\eR$ peut être considéré comme l'ensemble de niveau zéro de la fonction $F(x,y,z)=z-f(x,y)$.

    \item[Hyperboloïdes]
	    Les hyperboloïdes, comme les ellipsoïdes, sont une famille d'ensemble de niveau. En particulier, nous considérons des hyperboloïdes dont l'axe de symétrie est l'axe des $z$ et qui sont symétriques par rapport un plan $x$-$y$.  Une fois que les paramètres  $a$, $b$ et $c$ sont fixés la fonction que nous intéresse est 
\[
Hyp(x,y,z)= \frac{x^2}{a^2}+\frac{y^2}{b^2}-\frac{z^2}{c^2}.
\]
Les ensembles de niveau $N_d$ pour $d>0$ sont connexes, on les appelle \emph{hyperboloïdes à une feuille}. L'ensemble de niveau $N_0$ est \emph{cône (elliptique)}, le deux moitiés du cône se touchent en l'origine. Enfin, les ensembles de niveau $N_d$ pour $d<0$ ne sont  pas connexes et pour cette raison on les appelle \emph{hyperboloïdes à deux feuilles}.
  \end{description}
\end{example}


%++++++++++++++++++++++++++++++++++++++++++++++++++++++++++++++++++++++++++++++++++++++++
\section{Dérivée suivant un vecteur}		\label{SecDerDirect}
%++++++++++++++++++++++++++++++++++++++++++++++++++++++++++++++++++++++++++++++++++++++++
\begin{definition}
Soit $f$ une application de $U\subset\eR^m$ dans $\eR$, $a$ un point dans $U$ et $v$ un vecteur de $\eR^m$. On dit que $f$ admet une \defe{dérivée suivant le vecteur $v$ au point $a$}{dérivée!directionnelle} si la fonction $t\mapsto f(a+tv)$ admet une dérivée en $t=0$. La  dérivée de $f$ suivant le vecteur $v$ au point $a$ est alors cette dérivée, et $f$ est dite dérivable suivant $v$ en $a$,
\[
\partial_v f(a)=\lim_{
  \begin{subarray}{l}
    t\to 0\\ t\neq 0 
  \end{subarray}
 }\frac{f(a+tv)-f(a)}{t}.
\] 
\end{definition}

\begin{definition}
  La fonction $f:U\subset\eR^m\to \eR^n$ de composantes $(f_1,\ldots, f_n)$, est dite \defe{dérivable suivant $v$ au point $a$}{} si toute ses composante $f_i$, $i=1,\ldots, n$ sont dérivables suivant $v$ au point $a$. Dans ce cas, nous écrivons
  \begin{equation}
	\partial_v f(a)=\left(\partial_v f_1(a), \ldots, \partial_v f_n(a)\right)^T.
  \end{equation}
\end{definition}
On parle aussi souvent de dérivé \defe{dans la direction}{} du vecteur $v$. Une \defe{direction}{direction} dans $\eR^m$ est un vecteur de norme $1$. Tant que $u$ est un élément non nul de $\eR^m$, nous pouvons parler de la direction de $u$.

\begin{proposition}
Soit $u$ un vecteur de norme $1$ dans $\eR^m$ et soit $v=\lambda u$, avec $\lambda$ dans $\eR$. La fonction $f$ est dérivable suivant $v$ au point $a$ si et seulement si $f$ est dérivable suivant $u$ au point $a$, en outre  
\[
\partial_v f(a)=\lambda\partial_u f(a).
\]
\end{proposition}
\begin{proof}
  \begin{equation}
    \begin{aligned}
  \partial_v f(a)=&\lim_{\begin{subarray}{l}
     t\to 0\\ t\neq 0 
    \end{subarray}}\frac{f(a+tv)-f(a)}{t}=\lim_{\begin{subarray}{l}
     t\to 0\\ t\neq 0 
    \end{subarray}}\frac{f(a+t\lambda u)-f(a)}{t}=\\
&=\lambda \lim_{\begin{subarray}{l}
    t\to 0\\ t\neq 0 
  \end{subarray}}\frac{f(a+t\lambda u)-f(a)}{\lambda t}=\lambda \partial_u f(a).    
    \end{aligned}
  \end{equation}
\end{proof}
\begin{definition}
Soit $f$ une application de $U\subset\eR^m$ dans $\eR$. On appelle \defe{dérivées partielles de $f$ au point $a$}{dérivée!partielle} les dérivées de $f$ suivant les vecteurs de base $e_1,\ldots,e_m $ au point $a$, si elles existent.
\end{definition}
Si $m=2,3$ on peut utiliser la notation $f_x$, $\partial_x$  ou $\partial_1$ pour la dérivée partielle suivant $e_1$, $f_y$, $\partial_y$  ou $\partial_2$  pour la dérivée partielle suivant $e_2$ et $f_z$,  $\partial_z$  ou $\partial_3$  pour la dérivée partielle suivant $e_3$. En général, nous écrivons $\partial_i$ pour noter la la dérivée partielle suivant $e_i$.  

\begin{example}
Les dérivées partielles de la fonction $f(x,y)=xy^3+\sin y$ au point $(0,\pi)$ sont 
\[
\partial_xf(0,\pi)=\frac{ \partial f }{ \partial x }(0,\pi)=f_x(0,\pi)=\lim_{\begin{subarray}{l}
    t\to 0\\ t\neq 0 
  \end{subarray}} \frac{(t\pi^3+\sin \pi)-(\sin \pi)}{t}= \pi^3,
\] 
\[
\partial_yf(0,\pi)=\frac{ \partial f }{ \partial y }(0,\pi)=f_y(0,\pi)=\lim_{\begin{subarray}{l}
    t\to 0\\ t\neq 0 
  \end{subarray}} \frac{0(\pi+t)^3+\sin (t+\pi)-0\cdot \pi^3}{t}= \cos \pi=-1,
\]   
\end{example}
La fonction d'une seule variable qu'on obtient à partir de $f$ en fixant les $p-1$ variables  $x_1,\ldots, x_{i-1}, x_{i+1}, \ldots, x_p$ et qui associe à $x_i$ la valeur $f(x_1,\ldots, x_{i-1}, x_i, x_{i+1}, \ldots, x_p)$, est appelée $x_i$-ème \defe{section}{section} de $f$ en $x_1,\ldots, x_{i-1}, x_{i+1}, \ldots, x_p$. L'$i$-ème dérivée partielle de $f$ au point $a=(x_1,\ldots,x_m)$ est la dérivée de l'$i$-ème section de $f$ au point $x_i$. En pratique, pour calculer les dérivées partielles d'une fonction on fait une dérivation par rapport à la variable choisie en considérant les  autres variables comme des constantes.

\begin{example}
	Considérons la fonction $f(x,y)=2xy^2$. Lorsque nous calculons $\partial_xf(x,y)$, nous faisons comme si $y$ était constant. Nous avons donc $\partial_xf(x,y)=2y^2$. Par contre lors du calcul de $\partial_yf(x,y)$, nous prenons $x$ comme une constante. La dérivée de $y^2$ par rapport à $y$ est évidement $2y$, et par conséquent, $\partial_yf(x,y)=4xy$.
\end{example}

\begin{example}
  La fonction $f(x,y)=x^y$ est dérivable au point $(1,2)$ et on a
\[
\partial_x f(1,2)=(yx^{y-1})_{(x,y)=(1,2)}=2,
\]
\[
\partial_y f(1,2)=\partial_y\left(e^{y\ln x}\right)_{(x,y)=(1,2)}=\left(\ln x e^{y\ln x}\right)_{(x,y)=(1,2)}=\ln\big( 1- e^{2\ln(1)} \big)=0.
\]
\end{example}
\begin{definition}
  Soit $f$ une application de $U\subset\eR^m$ dans $\eR$ et $u$ un vecteur de $\eR^m$. La fonction $f$ est \defe{dérivable sur $U$ suivant le vecteur $u$}{}, si $f$ est dérivable  suivant le vecteur $u$ en tout point de $U$.
\end{definition} 

Pour les fonctions d'une seule variable la dérivabilité en un point $a$ implique la continuité en $a$. Cela n'est pas vrai pour les fonctions de plusieurs variables : il existe des fonction $f$  qui sont dérivables suivant tout vecteur au point $a$ sans pour autant être continue en $a$. 

  \begin{example}
    Considérons la fonction $f:\eR^2\to \eR$ 
    \begin{equation}
      f(x,y)=\left\{
      \begin{array}{ll}
        \frac{x^2y}{x^4+y^2} \qquad&\textrm{si } (x,y)\neq (0,0),\\
        0     & \textrm{sinon}.
      \end{array}
      \right.
    \end{equation}
Pour voir que $f$ n'est pas continue en $(0,0)$ il suffit de calculer la limite de $f$ restreinte à la parabole $y=x^2$
\[
\lim_{x\to 0} f(x,x^2)=\frac{1}{2} \neq 0.
\] 
Pourtant la fonction $f$ est dérivable en $(0,0)$ dans toutes les directions. En effet, soit $v=(v_1,v_2)$. Si $v_2\neq 0$, alors
\[
\partial_v f(a)=\lim_{\begin{subarray}{l}
			t\to 0\\ t\neq 0 
  		\end{subarray}}
  		\frac{t^3v_1^2v^2}{t^5 v_1^4+ t^3v_2^2}=\frac{v_1^2}{v_2},
\] 
tandis que si $v_2=0$, alors la valeur de $f(tv_1, 0)$  est $0$ pour tout $t$ et $v_1$, donc la dérivée partielle de $f$ par rapport à $x$ en l'origine existe et est nulle. 
\end{example}

\begin{example}
    Pour une fonction réelle à variable réelle, la dérivabilité entraine la continuité. Il n'en va pas de même pour les fonctions à plusieurs variables, comme le montre l'exemple suivant :
    \begin{equation}
        f(x,y)=\begin{cases}
            0    &   \text{si \( x=0\)}\\
            \frac{ y }{ x }\sqrt{x^2+y^2}    &    \text{sinon.}
        \end{cases}
    \end{equation}
    Nous avons tout de suite
    \begin{equation}
        \frac{ \partial f }{ \partial y }(0,0)=0.
    \end{equation}
    De plus si \( u_x\neq 0\) nous avons
    \begin{equation}
            \frac{ \partial f }{ \partial u }(0,0)=\frac{ u_y }{ u_x }\| u \|.
    \end{equation}
    Donc toutes les dérivées directionnelles de \( f\) en \( (0,0)\) existent alors que la fonction n'y est manifestement pas continue. En effet sous forme polaire,
    \begin{equation}
        f(r,\theta)=\frac{ r\sin(\theta) }{ \cos(\theta) },
    \end{equation}
    et quelle que soit la valeur de \( r\), en prenant \( \theta\) suffisamment proche de \( \pi/2\), la fraction peut être arbitrairement grande.

    Nous verrons par la proposition \ref{diff1} que la différentiabilité d'une fonction implique sa continuité.
\end{example}

\begin{definition}
	Étant donnés deux points $a$ et $b$ dans $\eR^p$ on appelle \defe{segment}{segment!dans $\eR^p$} d'extrémités $a$ et $b$, et on note $[a,b]$, l'image de $[0,1]$ par l'application $s: [0,1]\to \eR^p$, $s(t)= (1-t)a+tb$.  On pose $]a,b[=s\left(]0,1[\right)$, et  $]a,b]=s\left(]0,1]\right)$. 
\end{definition}
Il faut observer que le segment $[a,b]$ est une courbe orientée : certes en tant que ensembles, $[a,b]=[b,a]$, mais si nous regardons la fonction de $t$ correspondante à $[b,a]$, nous voyons qu'elle va dans le sens inverse de celle qui correspond à $[a,b]$. Nous approfondirons ces questions lorsque nous parlerons d'arcs paramétrés autour de la section \ref{SecArcGeometrique}.

Le segment $[b,a]$ est l'image de l'application $r\colon [0,1]\to \eR^p$ donnée par $r(t)=(1-t)b+ta$.

\begin{theorem}[Accroissement finis pour les dérivées suivant un vecteur]\label{val_medio_1}		\index{théorème!accroissements finis!dérivée directionnelle}
	Soit $U$ un ouvert dans $\eR^m$ et soit $f:U\to\eR^n$ une fonction. Soient $a$ et $b$ deux points distincts dans $U$, tels que le segment $[a,b]$ soit contenu dans $U$. Soit $u$ le vecteur 
	\[
		u=\frac{b-a}{\|b-a\|_m}.
	\] 
	Si $\partial_u f(x)$ existe pour tout $x$ dans $[a,b]$ on a
	\[
		\|f(b)-f(a)\|_n\leq \sup_{x\in[a,b]}\|\partial_uf(x)\|_n\|b-a\|_m.
	\]
\end{theorem}

\begin{proof}
	Nous considérons la fonction $g(t)=f\big( (1-t)a-tb \big)$. Elle décrit la droite entre $a$ et $b$ parce que $g(0)=a$ et $g(1)=b$. En ce qui concerne la dérivée,
	\begin{equation}
		\begin{aligned}[]
			g'(t)&=\lim_{h\to 0} \frac{ g(t+h)-g(t) }{ h }\\
			&=\lim_{h\to 0} \frac{ f\big( (1-t-h)a-(t+h)b \big) }{ h }\\
			&=\lim_{h\to 0} \frac{ f\big( a+(t+h)(b-a) \big)-f\big( a+t(b-a) \big) }{ h }\\
			&=\frac{ \partial f }{ \partial u }\big( a+t(b-a) \big)\| b-a \|.
		\end{aligned}
	\end{equation}
	Le dernier facteur $\| b-a \|$ apparaît pour la normalisation du vecteur $u$. En effet dans la limite, il apparaît $h(b-a)$, ce qui donnerait la dérivée le long de $b-a$, tandis que $u$ vaut $(b-a)/\| b-a \|$.

	Par le théorème des accroissements finis pour $g$, il existe $t_0\in\mathopen] 0 , 1 \mathclose[$ tel que
	\begin{equation}
		g(1)=g(0)+g'(t_0)(1-0).
	\end{equation}
	Donc
	\begin{equation}
		\| g(1)-g(0) \|\leq\sup_{t_0}\| g'(t_0) \|=\sum_{t_0\in\mathopen] 0 , 1 \mathclose[}\left\| \frac{ \partial f }{ \partial u }(a+t_0(b-a)) \right\|\| b-a \|.
	\end{equation}
	Mais lorsque $t_0$ parcours $\mathopen] 0 , 1 \mathclose[$, le point $a+t_0(b-a)$ parcours le segment $\mathopen] a , b \mathclose[$, d'où le résultat.
\end{proof}

\begin{corollary}
	Dans les mêmes hypothèses, si $n=1$, alors il existe $\bar x $ dans $]a,b[$ tel que
	\[
		f(b)-f(a)=\partial_uf(\bar x)\|b-a\|_m.
	\]    
\end{corollary}


%++++++++++++++++++++++++++++++++++++++++++++++++++++++++++++++++++++++++++++++++++++++++
\section{Différentielles}		\label{SecDifferentielle}
%+++++++++++++++++++++++++++++++++++++++++++++++++++++++++++++++++++++++++++++++++++++++++++++++++++++++++++++++++++++++++++
La notion de dérivée partielle (ou de dérivée suivant un vecteur) pour une fonction de plusieurs variables n'est pas une  généralisation de la notion de dérivée en une variable d'espace. En fait, du point de vue géométrique, la dérivée de la fonction $g:\eR\to\eR$ au point $a$ est la pente de la ligne droite tangente au graphe de $g$ au point $(a, g(a))$. Cette ligne, d'équation $r(x)=g'(a)x+g(a)$, est la meilleure approximation affine du graphe de $g$ au point $a$, comme à la figure \ref{LabelFigTangentSegment}.
\newcommand{\CaptionFigTangentSegment}{Tangentes au graphe d'une fonction d'une variable}
\input{Fig_TangentSegment.pstricks}

Le graphe d'une fonction $f$ de $\eR^2$ dans $\eR$ est une surface de deux paramètres dans $\eR^3$. Si l'approximation affine d'une telle surface au point $(x,y,f(x,y))$ existe, alors elle est un plan tangent. En dimension plus haute, le graphe de la fonction $f:\eR^m\to\eR$ est une surface de $m$ paramètres dans $\eR^{m+1}$ et son approximation affine (si elle existe) est un hyperplan de $\eR^m$. 

Nous allons voir que si $f$ prend ses valeurs dans $\eR^n$ l'approximation affine de $f$ au point $a$ est l'élément de $ f(a)+\mathcal{L}(\eR^m,\eR^n)$ qui ressemble le plus à $f$ au voisinage de $a$. Plus précisément, on utilise les définitions suivantes.         
\begin{definition}
  Soient $f$ et $g$ deux applications d'un ouvert $U$ de $\eR^m$ dans $\eR^n$. On dit que $g$ est \defe{tangente}{application!tangente} à $f$ au point $a\in U$ si $f(a)=g(a)$ et 
\[
\lim_{\begin{subarray}{l}
    x\to a\\ x\neq a
  \end{subarray}}\frac{\|f(x)-g(x)\|_n}{\|x-a\|_m}=0.
\]
\end{definition}
La relation de tangence est une relation d'équivalence. Nous sommes particulièrement intéressés par le cas où $f$ admet une application  affine tangente au point $a$. 
\begin{definition}      \label{DefDifferentiellePta}
  Soit $U$ un ouvert dans $\eR^m$ et $a$ un point dans $U$. Soit $f$ une application de $U$ dans $\eR^n$. On dit que $f$ est \defe{différentiable au point $a$}{application!différentiable} s'il existe une application linéaire $T$ de $\eR^m$ dans $\eR^n$ qui satisfait
  \begin{equation}	\label{EqCritereDefDiff}
\lim_{h\to 0_m}\frac{\|f(a+h)-f(a)-T(h)\|_n}{\|h\|_m}=0.    
  \end{equation}
  Si une telle $T$ existe on l'appelle \defe{différentielle}{différentielle} de $f$ au point $a$, et on la note $df(a)$. 
\end{definition}

Note : $df_a$ est \emph{en soi} une application $df(a)\colon \eR^m\to \eR^n$. Nous notons $df_a(u)$\nomenclature{$df_a(u)$}{Application de la différentielle de $f$ sur le vecteur $u$} la valeur de $df_a$ sur le vecteur $u\in\eR^m$.


\newcommand{\CaptionFigDifferentielle}{Interprétation géométrique de la différentielle.}
\input{Fig_Differentielle.pstricks}
En ce qui concerne l'interprétation géométrique, si nous regardons la figure \ref{LabelFigDifferentielle}, et d'ailleurs aussi en voyant la définition \ref{EqCritereDefDiff}, la fonction est différentiable et la différentielle est \( T\) si il existe une fonction \( \alpha\) telle que
\begin{equation}
    f(a+u)-f(a)-T(u)=\alpha(u)
\end{equation}
où la fonction \( \alpha\) satisfait
\begin{equation}		\label{EqPresqueTa}
	\lim_{u\to 0} \frac{ \| \alpha(u)\| }{\| u \|}=0
\end{equation}
C'est cela qui fait écrire \( f(a+u)-f(a)-df_a(u)=o(\| u \|)\) à ceux qui n'ont pas peur de la notation \( o\).

La différentielle $df_a$ est donc la partie linéaire de l'application affine qui approxime au mieux la fonction $f$ autour du point $a$. La notion de différentielle est la vraie généralisation du concept de dérivée pour fonctions de plusieurs variables, en outre elle nous permet d'expliciter la relation qui associe au vecteur $u$ la dérivée $\partial_u f(a)$, pour $f$ et $a$ fixés.  

\begin{remark}
	Si on remplace les normes $\|\cdot\|_m$  et $\|\cdot\|_n$ par d'autres normes, l'existence et la valeur de la différentielle de $f$ au point $a$ ne sont pas remises en cause. En effet, soient  $\|\cdot\|_M$  une norme sur $\eR^m$ et $\|\cdot\|_N$ une norme sur $\eR^n$. Par le théorème \ref{ThoNormesEquiv}, ces normes sont équivalentes à $\| . \|_m$ et $\| . \|_m$ respectivement; il existe donc des constantes $k,\, K,\, l,\,L >0$ telles que  pour tout vecteur $u$ de $\eR^m$ et tout vecteur $v$ de $\eR^n$   
\[
k\|u\|_M\leq \|u\|_m\leq K\|u\|_M,
\]
\[
l\|v\|_N\leq \|v\|_n\leq L\|v\|_N.
\]
Les éléments de $\mathcal{L}(\eR^m, \eR^n)$ sont les mêmes et on a 
\begin{equation}
  \begin{aligned}
 & \frac{l}{K}  \frac{\|f(a+h)-f(a)-T(h)\|_N}{\|h\|_M}\leq \frac{\|f(a+h)-f(a)-T(h)\|_n}{\|h\|_m}\leq\\
&\leq\frac{L}{k} \frac{\|f(a+h)-f(a)-T(h)\|_N}{\|h\|_M}.
  \end{aligned}
\end{equation}
Il est donc possible, pour démontrer la différentiabilité ou pour calculer la différentielle, d'utiliser le critère \eqref{EqCritereDefDiff} avec une norme au choix. Parfois c'est utile.
\end{remark}

\begin{proposition}\label{diff1}
    Si $f$ est différentiable au point $a$ alors
    \begin{enumerate}
        \item
            elle est continue en \( a\),
        \item
            elle admet une dérivée dans toutes les directions de \( \eR^m\),
        \item
            si $T\in\aL(\eR^m,\eR^n)$ est la différentielle de $f$ au point $a$, alors
            \begin{equation}
                T(u)=df_a(u)=\partial_u f(a). 
            \end{equation}
    \end{enumerate}
\end{proposition}
\index{application!différentiable}

La dernière égalité sera de temps en temps utilisée sous la forme
\begin{equation}    \label{EqOWQSoMA}
    df_a(u)=\Dsdd{ f(a+tu) }{t}{0}.
\end{equation}

\begin{proof}
  La limite
\[
\lim_{h\to 0_m}\frac{\|f(a+h)-f(a)-T(h)\|_n}{\|h\|_m}=0,
\]
implique que
 \[
\lim_{h\to 0_m}\|f(a+h)-f(a)-T(h)\|_n=0.
\]
Comme $T$ est dans $\mathcal{L}(\eR^m,\eR^n)$, on a $\lim_{h\to 0}T(h)=0$, d'où la continuité de $f$ au point $a$.

Si $u$ est un vecteur non nul, la différentiabilité de $f$ au point $a$ implique
\[
\lim_{t\to 0}\frac{\|f(a+tu)-f(a)-T(tu)\|_n}{\|tu\|_m}=0,
\]
par la linéarité de $T$ et par l'égalité $\|tu\|_m=|t|\|u\|_m$ on obtient
\[
\lim_{t\to 0}\frac{f(a+tu)-f(a)}{|t|}= T(u).
\]
Donc $f$ est dérivable suivant le vecteur $u$ et $\partial_uf(a)=T(u)=df_a(u)$.
\end{proof}

Cette proposition est à ne pas confondre avec la proposition \ref{Diff_totale} qui dira que si les dérivées partielles \emph{sont continues} sur un voisinage de $a$, alors $f$ est différentiables en $a$.

\begin{corollary}
	Soit $f$ une application de $U$ dans $\eR^n$ différentiable au point $a$ dans $U$. Alors l'application $df(a)$, différentielle de $f$ au point $a$, est unique, c'est à dire que si $T_1$ et $T_2$ sont deux applications vérifiant la condition \eqref{EqCritereDefDiff}, alors $T_1=T_2$.
\end{corollary}

\begin{proof}
	Pour tout vecteur $u$, la proposition précédente implique que $T_1(u)=T_2(u)=\partial_uf(a)$.
\end{proof}

\begin{corollary}
Soit  $f:\eR^m\to \eR^n$ une fonction.
  La dérivabilité de $f$ au point  $a$ suivant tout vecteur de $\eR^m$ est une condition nécessaire pour la différentiabilité de $f$ en $a$.
\end{corollary}

\begin{definition}
	Une fonction $f:\eR^m\to \eR^n$ est dite \defe{différentiable sur l'ouvert $U\subset\eR^m$}{différentiable!sur un ouvert}, si $f$ est différentiable en tout point de $U$. Dans ce cas, la différentielle de $f$ est l'application
	\begin{equation}
		\begin{aligned}
			df\colon U\subset\eR^m&\to \aL(\eR^m,\eR^n) \\
			x&\mapsto df(x). 
		\end{aligned}
	\end{equation}
\end{definition}

\begin{remark}\label{rk_lin}
  Tout élément $T$ de $\mathcal{L}(\eR^m,\eR^n)$ est différentiable en tout point de $\eR^m$ et coïncide avec sa différentielle. En effet, pour tout $a$ et $h$ dans $\eR^m$  on a 
\[
\frac{\|T(a+h)-T(a)-T(h)\|_n}{\|h\|_m}=0.
\]
\end{remark}
La proposition \ref{diff1} nous donne une recette très pratique pour calculer la différentielle d'une fonction de $\eR^m$ dans $\eR^n$.

 \begin{definition}
	 Soit $f$ une fonction différentiable de $\eR^m$ dans $\eR$. On appelle \defe{gradient}{gradient} de $f$ la fonction $\nabla f : \eR^m\to \eR^m$\nomenclature{$\nabla f$}{gradient de la fonction $f$} de composantes
\[
\partial_{1}f,\ldots,\partial_{m}f. 
\] 
Soit $f$ une fonction de $\eR^m$ dans $\eR^n$, $f(a)=(f_1(a),\ldots,f_n(a))^T$. On appelle \defe{matrice jacobienne}{matrice!jacobienne} de $f$ la fonction $J(f) : \eR^m\to \eR^m\times\eR^n$ définie par
\begin{equation}
a\mapsto  \begin{pmatrix}
    \partial_{1}f_1(a) &\ldots&\partial_{m}f_1(a)\\
\vdots&\ddots&\vdots\\
\partial_{1}f_n (a)&\ldots&\partial_{m}f_n(a)\\
  \end{pmatrix}
\end{equation}
\end{definition}

Le lemme suivant regroupe quelque égalités avec lesquelles nous allons souvent travailler. Il explique comment sont liés les dérivées directionnelles, les dérivées partielles et la différentielle.

\begin{lemma}		\label{LemdfaSurLesPartielles}
	Si $f\colon \eR^m\to \eR^n$ est une fonction différentiable, alors
	\begin{equation}
		\partial_uf(a)=df_a(u)=\sum_{i=1}^mu_i\partial_if(a)=\nabla f(a)\cdot u
	\end{equation}
	pour tout vecteur $u\in\eR^m$
\end{lemma}

\begin{proof}
 La première égalité est la proposition \ref{diff1}.

On sait que le vecteur $u$ peut être écrit de façon unique comme combinaison linéaire des vecteurs de base 
\[
u=\sum_{i=1}^{m}u_i e_i, \qquad  u_i\in\eR,\, \forall i\in\{1,\ldots, m\}.
\]
Alors, la linéarité de $df(a)$ nous donne
\begin{equation}
     df(a).u= df(a).\sum_{i=1}^{m}u_i e_i
=\sum_{i=1}^{m}u_i \left(df(a).e_i\right)
=\sum_{i=1}^{m}u_i \partial_{i}f(a).
 \end{equation}
 Ceci fournit la seconde égalité. La troisième est la définition du produit scalaire \eqref{DefYNWUFc}.
\end{proof}

%++++++++++++++++++++++++++++++++++++++++++++++++++++++++++++++++++++++++++++++++++++++++
\section{Propriétés des différentielles}		\label{SecPropDiffs}
%++++++++++++++++++++++++++++++++++++++++++++++++++++++++++++++++++++++++++++++++++++++++++++++++++++++++++++++++++++++++++++++

%---------------------------------------------------------------------------------------------------------------------------
\subsection{Linéarité}
%---------------------------------------------------------------------------------------------------------------------------

La proposition suivante signifie que différentiation est une opération linéaire sur l'ensemble des fonctions différentiables. 
\begin{proposition}		\label{PropDiffLineaire}
  Soient $f$ et $g$ deux fonction de $U\subset\eR^m$ dans $\eR^n$ différentiables au point $a\in U$, et soit $\lambda$ dans $\eR$. Alors les fonctions $f+g$ et $\lambda f$ sont différentiables au point $a$ et on a 
  \begin{equation}
    \begin{aligned}
 &     d(f+g)(a)=df(a)+dg(a), \\
& d(\lambda f)(a)=\lambda df(a),
    \end{aligned}
\end{equation}
\end{proposition}
\begin{proof}
  \begin{equation}
    \begin{aligned}
     & \lim_{h\to 0_m}\frac{\left\|\left(f(a+h)+g(a+h)\right)-\left(f(a)+g(a)\right)-df(a).h-dg(a).h\right\|_n}{\|h\|_m}\leq\\
&\lim_{h\to 0_m}\frac{\|f(a+h)-f(a)-df(a).h\|_n}{\|h\|_m}+\lim_{h\to 0_m}\frac{\|g(a+h)-g(a)-dg(a).h\|_n}{\|h\|_m}=0.
    \end{aligned}
  \end{equation}
  De même on démontre la  propriété $d(\lambda f)(a)=\lambda df(a)$.
\end{proof}

%---------------------------------------------------------------------------------------------------------------------------
\subsection{Produit}
%---------------------------------------------------------------------------------------------------------------------------

Soient $f$ et $g$ deux fonctions de $\eR^m$ dans $\eR^n$. Nous notons $f\cdot g$ la fonction de $\eR^n$ dans $\eR$ donnée par le produit scalaire point par point, c'est à dire
\begin{equation}
	(f\cdot g)(x)=f(x)\cdot g(x)
\end{equation}
pour tout $x\in\eR^m$. Le point dans le membre de droite est le produit scalaire dans $\eR^n$. Le cas particulier $n=1$ revient au produit usuel de fonctions :
\begin{equation}
	(fg)(x)=f(x)g(x).
\end{equation}

\begin{lemma}		\label{LemDiffProsuid}
	Si $f$ et $g$ sont des fonctions différentiables sur $\eR^m$ à valeurs dans $\eR$, alors la fonction produit $fg$ est également différentiable et
	\begin{equation}		\label{EqDifffgProd}
		d(fg)(a)=df(a)g(a)+f(a)dg(a)
	\end{equation}
	au sens où pour chaque $u$ dans $\eR^m$,
	\begin{equation}
		d(fg)(a).u=g(a)df(a).u+f(a)dg(a).u.
	\end{equation}
\end{lemma}
Remarquons qu'ici, $f(a)$ et $g(a)$ sont des réels, donc nous pouvons écrire $f(a)dg(a)$ aussi bien que $dg(a)f(a)$ sans ambigüités. 

\begin{proof}
	Ce que nous devons faire pour vérifier la formule \ref{EqDifffgProd}, c'est de vérifier le critère \eqref{EqCritereDefDiff} en remplaçant $f$ par $fg$ et $T(h)$ par $g(a)df(a).h+f(a)dg(a).h$.

	Ce que nous avons au numérateur est
	\begin{equation}
		\begin{aligned}[]
			\clubsuit&=(fg)(a+h)-(fg)(a)-g(a)df(a).h-f(a)dg(a).h\\
				&=f(a+h)g(a+h)-f(a)g(a)-g(a)df(a).h-f(a)dg(a).h.
		\end{aligned}
	\end{equation}
	Maintenant, nous allons faire apparaître $\big( f(a+h)-f(a)-df(a) \big)g(a+h)$ en ajoutant et soustrayant ce qu'il faut pour conserver $\clubsuit$ :
	\begin{equation}
		\begin{aligned}[]
			\clubsuit&=\big( f(a+h)-f(a)-df(a).h \big)g(a+h)\\
					&\quad +f(a)g(a+h)+g(a+h)df(a).h\\
					&\quad -f(a)g(a)-g(a)df(a).h-f(a)dg(a).h.
		\end{aligned}
	\end{equation}
	Nous mettons maintenant $f(a)$ et $fd(a).h$ en évidence là où c'est possible :
	\begin{equation}
		\begin{aligned}[]
			\clubsuit&=\big( f(a+h)-f(a)-df(a).h \big)g(a+h)\\
				&\quad+f(a)\big( g(a+h)-g(a)-dg(a).h \big)\\
				&\quad+\big( g(a+h)-g(a) \big)df(a).h.
		\end{aligned}
	\end{equation}
    Nous devons maintenant considérer la limite
	\begin{equation}
		\lim_{h\to 0}\frac{ \| \clubsuit \| }{ \| h \| }.
	\end{equation}
    Étant donné que $f$ et $g$ sont différentiables, les deux premiers termes sont nuls :
    \begin{equation}
        \begin{aligned}[]
            \lim_{h\to 0}\frac{ \big( f(a+h)-f(a)-df(a).h \big)}{\| h \|}g(a+h)=0\\
            \lim_{h\to 0} f(a)\frac{ \big( g(a+h)-g(a)-dg(a).h \big)}{\| h \|}=0.
        \end{aligned}
    \end{equation}
    En ce qui concerne le troisième terme, en utilisant la norme d'une application linéaire, nous avons
	\begin{equation}
		\lim_{h\to 0} \frac{ \| df(a).h \| }{ \| h \| }\leq\sup_{h\in\eR^m}\frac{ \| df(a).h \| }{ \| h \| }=\| df(a) \|,
	\end{equation}
    et par conséquent
    \begin{equation}
        \begin{aligned}[]
            0&\leq\lim_{h\to 0} \| g(a+h)-g(a) \|\frac{ \| df(a).h \|\| h \| }{ \| h \| }\\
            &\leq \lim_{h\to 0} \| g(a+h)-g(a) \|\| df(a) \|=0
        \end{aligned}
    \end{equation}
    parce que $g$ est continue (la limite du premier facteur est nulle tandis que la norme de $df(a)$ est un nombre constant). Nous avons donc bien prouvé que la formule \eqref{EqDifffgProd} est la différentielle de $fg$ au point $a$.
\end{proof}
Ce résultat se généralise pour des fonctions $f$ et $g$ de $\eR^m$ dans $\eR^n$.

\begin{proposition}
	Soient $f$ et $g$ deux fonction de $U\subset\eR^m$ dans $\eR^n$ différentiables au point $a\in U$. Alors la fonction $f\cdot g$ est différentiable  au point $a$ et on a 
	\begin{equation}
		g(f\cdot g)(a)=g(a)\cdot df(a)+f(a)\cdot dg(a)
	\end{equation}
	au sens où
	\begin{equation}		\label{Eqdfcdotgexpl}
		d(f\cdot g)(a).u=g(a)\cdot\big( df(a).u \big)+f(a)\cdot\big( dg(a).u \big)
	\end{equation}
	pour tout $u\in\eR^m$.
\end{proposition}
Note : il faut être bien attentif en lisant la formule \eqref{Eqdfcdotgexpl}. Les points à l'intérieur des grandes parenthèses marquent l'application des différentielles sur $u$. Le contenu de ces parenthèses sont donc des éléments de $\eR^n$. Les points devant les parenthèses dénotent le produit scalaire dans $\eR^n$ ($f(a)$ et $dg(a).u$ sont des éléments de $\eR^n$).

\begin{proof}
	La preuve du cas $n=1$ est déjà faite; c'est la formule \eqref{EqDifffgProd}. Pour le cas général $n\geq 2$, nous passons au composantes en nous rappelant que
	\begin{equation}
		(f\cdot g)(a)=\sum_{i=1}^nf_i(a)g_i(a)=\sum_{i=1}^n(f_ig_i)(a).
	\end{equation}
	En utilisant la linéarité de la différentiation, nous nous réduisons donc au cas des produits $f_ig_i$ qui sont des fonctions de $\eR^m$ dans $\eR$ :
	\begin{equation}
		\begin{aligned}[]
			d(f\cdot g)(a)&=d\left( \sum_{i=1}^n f_ig_i \right)(a)\\
			&=\sum_{i=1}^n\big( df_i(a)g_i(a)+f_i(a)dg_i(a) \big)\\
			&=g(a)\cdot df(a)+f(a)\cdot dg(a).
		\end{aligned}
	\end{equation}
	Ceci termine la preuve.
\end{proof}

%---------------------------------------------------------------------------------------------------------------------------
\subsection{Composition}
%---------------------------------------------------------------------------------------------------------------------------

La plus importante entre les règles de différentiation est la règle de différentiation d'une fonction composée, dite règle de la chaine (\emph{chain rule} dans les livres anglais et américains). Cette règle généralise la règle de dérivation pour fonctions de $\eR$ dans $\eR$. Il est utile d'introduire d'abord une formulation équivalente de la définition de différentielle
\begin{lemma}\label{Def_diff2}
  Soit $U$ un ouvert de $\eR^m$. La fonction $f: U\to\eR^n$ est différentiable au point $a$ dans $U$, si et seulement s'il existe une fonction $\sigma_f: U\times U\to \eR^n$ telle que
  \begin{subequations}		\label{SubEqsDiff2}
	  \begin{align}
  		\sigma_f(a,a)&=\lim_{x\to a} \sigma_f(a,x)=0\\
		 f(x)&=f(a)+T(x-a)+\sigma_f(a,x)\|x-a\|_m,   \label{def_diff2}
	  \end{align}
  \end{subequations}
pour une certaine application linéaire $T\in\mathcal{L}(\eR^m,\eR^n)$.
\end{lemma}
\begin{proof}
	Si les conditions \eqref{SubEqsDiff2} sont satisfaites alors $T$ est la différentielle de $f$ en $a$. En effet, dans ce cas nous avons
	\begin{equation}
		f(a+h)=f(a)+T(h)+\sigma_f(a,a+h)\| h \|,
	\end{equation}
	et la condition \eqref{EqCritereDefDiff} devient
	\begin{equation}
		\lim_{h\to 0} \frac{ \| \sigma_f(a,a+h) \|\| h \| }{ \| h \| }=\lim_{h\to 0} \| \sigma_f(a,a+h)\| =0
	\end{equation}
	
 
Si $f$ est différentiable au point $a$ il suffit de prendre $T=df(a)$ et 
\[
\sigma_f(a,x)=\frac{f(x)-f(a)-df(a).(x-a)}{\|x-a\|_m}.
\]
\end{proof}

\begin{remark}
	La fonction $\sigma_f(a,x)\| x-a \|_m$ est ce qui avait été appelle $\epsilon(h)$ sur la figure \ref{LabelFigDifferentielle}.
\end{remark}

\begin{proposition}		\label{PropDiffCompose}
Soient $U$ un ouvert de $\eR^m$ et $V$ un ouvert de $\eR^n$. Soient $f: U\to V$  et $g: V \to \eR^p$ deux fonctions différentiables respectivement au point $a$ dans $U$ et $b=f(a)$ dans $V$. Alors la fonction composée $g\circ f: U\to \eR^p $ est différentiable au point $a$ et on a 
\begin{equation}	\label{EqDiffCompose}
	d(g\circ f)(a)=dg\big(f(a)\big)\circ df(a).
\end{equation}
\end{proposition}

Note : la formule \eqref{EqDiffCompose} est à comprendre de la façon suivante. Si $u\in\eR^m$, alors
\begin{equation}
	d(g\circ f)(a).u=\underbrace{dg\big( f(a) \big)}_{\in\aL(\eR^n,\eR^p)}\Big( \underbrace{df(a).u}_{\in\eR^n} \Big)\in\eR^p.
\end{equation}

\begin{proof}
 En tenant compte du lemme \ref{Def_diff2} on peut écrire 
 \begin{subequations}
	 \begin{align}
		f(a+h)-f(a)&=df(a).h+\sigma_f(a,a+h)\|h\|_m,	&&\forall h\in U-a,\\
		g(b+k)-g(b)&=dg(b).k+\sigma_g(b,b+k)\|k\|_n,	&&\forall k\in V-b.
	 \end{align}
 \end{subequations}
On sait que $f(a)=b$ et que $f(a+h)$ est  un élément de $V$ et $f(a+h)=f(a)+k$ pour $k=df(a).h+\sigma_f(a,a+h)\|h\|_m$.  Par substitution dans la deuxième équation on obtient 
\begin{equation}
	\begin{aligned}
		g\big(f(a+h)\big)& - g\big(f(a)\big)\\ 
		&=dg\left(f(a)\right).\Big(df(a).h+\sigma_f(a,a+h)\|h\|_m\Big)\\
		&\quad+\sigma_g\left(f(a), f(a+h)\right)\left\| df(a).h+\sigma_f(a,a+h)\|h\|_m\right \|_n\\
		&=g\circ f (a+h) - g\circ f (a)\\
		&= dg\left(f(a)\right)\circ df(a).h \\
		&\quad +\|h\|_m\Big[ dg\left(f(a)\right).\sigma_f(a,a+h)\\
		&\qquad+\sigma_g\left(f(a), f(a+h)\right)\big\| df(a).\frac{h}{\|h\|_m}+\sigma_f(a,a+h)\big \|_n\Big],
	\end{aligned}
\end{equation}
donc
\begin{equation}
	(g\circ f) (a+h) - (g\circ f) (a) = dg\left(f(a)\right)\circ df(a).h + S(a,a+h) \|h\|_m
\end{equation}
où $S$ représente le contenu du dernier grand crochet. Il ne reste plus qu'à prouver que $S(a,a+h)$ est $o(\|h\|_m)$. En tenant compte du fait que $\sigma_f(a,a+h)$ et $\sigma_g\left(f(a), f(a+h)\right)$ sont $o (\|h\|_m)$,
\begin{equation}
  \begin{aligned}
   & \lim_{h\to 0_m} \frac{S(a,a+h)}{\|h\|_m}= \lim_{h\to 0_m}\frac{dg\left(f(a)\right).\sigma_f(a,a+h)}{\|h\|_m}+ \\
& + \lim_{h\to 0_m}\frac{\sigma_g\left(f(a), f(a+h)\right)\left\| df(a).\frac{h}{\|h\|_m}+\sigma_f(a,a+h)\right \|_n}{\|h\|_m} = 0.
  \end{aligned}
\end{equation}
\end{proof}

En appliquant la proposition précédente point par point, nous obtenons le résultat suivant.
\begin{proposition}
Soient $U$ un ouvert de $\eR^m$ et $V$ un ouvert de $\eR^n$. Soient $f: U\to V$  et $g: V \to \eR^p$ deux fonctions différentiables respectivement sur $U $ et sur $V$. Alors la fonction composée $g\circ f: U\to \eR^p $ est différentiable sur $U$.
\end{proposition}
La matrice jacobienne de $g\circ f$ au point $a$ est le produit matriciel des matrices jacobiennes de $f$ et de $f$. Plus précisément, nous avons
\begin{equation}
	J_{g\circ f}(a)=J_g\big( f(a) \big)J_f(a).
\end{equation}
Remarquez que nous considérons la matrice jacobienne de $g$ au point $f(a)$.

Dans la cas particulier où $m=1$ et $f$ est une fonction d'un intervalle $I$ dans $\eR^n$, dérivable au point $a$, on a que la fonction composée $g\circ f$ est dérivable au point $a$ si $g$ est différentiable et alors
\[
(g\circ f)'(a)= dg\left(f(a)\right).f'(a).
\]
En fait, pour les fonction d'une seule variable la dérivabilité coïncide avec la différentiabilité.

Nous avons aussi une formule importante pour la différentielle des formes bilinéaires.
  \begin{lemma}\label{bilin_diff}
    Toute application bilinéaire 
    \begin{equation}
	    \begin{aligned}
		    B\colon \eR^m\times\eR^n&\to \eR^p \\
		    B(a_1,a_2)&=a_1 \star a_2
	    \end{aligned}
    \end{equation}
    est différentiable en tout point $(a_1,a_2)$ de $\eR^m\times\eR^n$, et on a
\[
dB(a_1,a_2).(h_1,h_2)=h_1\star a_2 + a_1\star h_2.
\] 
  \end{lemma}
  \begin{proof}
    \begin{equation}
      \begin{aligned}
  & \frac{\|B(a_1+h_1,a_2+h_2)-B(a_1,a_2)-(h_1\star a_2 + a_1\star h_2)\|_p}{\|(h_1,h_2)\|_{\eR^m\times\eR^n}} = \\ 
&= \frac{\|(a_1+h_1)\star(a_2+h_2)-a_1\star a_2-(h_1\star a_2 + a_1\star h_2)\|_p}{\|(h_1,h_2)\|_{\eR^m\times\eR^n}}=\spadesuit
 \end{aligned}
    \end{equation}
on rajoute et on enlève la quantité $(a_1+h_1)\star a_2$ dans le numérateur, et on obtient  
   \begin{equation}
      \begin{aligned}
%&= \frac{\|(a_1+h_1)\star(a_2+h_2)-(a_1+h_1)\star a_2 +(a_1+h_1)\star a_2- a_1\star a_2-}{\|(h_1,h_2)\|_{\eR^m\times\eR^n}}\\
%&\hspace{7cm}\frac{-(h_1\star a_2 + a_1\star h_2)\|_p}{\quad}=\\
&\spadesuit= \frac{\|(a_1+h_1)\star h_2+h_1\star a_2-(h_1\star a_2 + a_1\star h_2)\|_p}{\|(h_1,h_2)\|_{\eR^m\times\eR^n}}=\\
&= \frac{\|h_1\star h_2\|_p}{\|(h_1,h_2)\|_{\eR^m\times\eR^n}}\leq C\frac{\|h_1\|_m\|h_2\|_n}{\|(h_1,h_2)\|_{\eR^m\times\eR^n}}\leq\\
&\leq C\frac{\|(h_1,h_2)\|^2_{\eR^m\times\eR^n}}{\|(h_1,h_2)\|_{\eR^m\times\eR^n}}= C\|(h_1,h_2)\|_{\eR^m\times\eR^n}.
      \end{aligned}
    \end{equation}
Si on prend la limite de cette expression pour $(h_1,h_2)\to (0_m,0_n)$ on obtient $0$, donc la preuve est complète. À noter, que dans l'avant-dernier passage on a utilisé la continuité des applications linéaires $\pr_m:\eR^m\times\eR^n\to \eR^m$ et $\pr_n: \eR^m\times\eR^n\to \eR^n$ qui à chaque point $(a_1,a_2)$ de $\eR^m\times\eR^n$ associent $a_1$ et $a_2$ respectivement.  
  \end{proof}

%--------------------------------------------------------------------------------------------------------------------------- 
  \subsection{Différentielle et dérivées partielles}
%---------------------------------------------------------------------------------------------------------------------------

\begin{proposition}		\label{Diff_totale}
 Soit $U$ un ouvert dans $\eR^m$ et $a$ un point dans $U$. Soit $f$ une application de $U$ dans $\eR^n$. Si toute les dérivée partielles de $f$ existent dans un voisinage de $a$ et sont continues au point $a$ alors $f$ est différentiable au point $a$.
\end{proposition}
\begin{proof} 
 On se limite au cas $m=2$.  Pour rendre les calculs plus simples on utilise ici la norme $\|\cdot\|_\infty$ dans l'espace $\eR^2$, mais comme on a vu plus en haut, cela ne peut pas avoir des conséquences sur la différentiabilité de $f$. Si la différentielle de $f$ au point $a$ existe alors elle est définie par la formule
\[
df(a).v=\partial_{x}f(a)v_1+\partial_{y}f(a)v_2, 
\] 
pour tout $v$ dans $\eR^m$. 

On commence par prouver le résultat en supposant que les dérivées partielles de $f$ au point $a$ sont nulles. La différentiabilité de $f$ signifie que pour toute constante  $\varepsilon> 0$ il y a une constante $\delta>0$ telle que si $\|v\|_\infty\leq \delta $ alors 
\[
\frac{\|f(a_1+v_1, a_2+v_2)-f(a_1, a_2)\|_n}{\|v\|_\infty}\leq \varepsilon. 
\]   
On écrit alors 
\begin{equation}
  \begin{aligned}
   & \|f(a_1+v_1, a_2+v_2)-f(a_1, a_2)\|_n=\\
&=\|f(a_1+v_1, a_2+v_2)-f(a_1+v_1, a_2)+f(a_1+v_1, a_2)-f(a_1, a_2)\|_n\leq\\
&\leq \|f(a_1+v_1, a_2+v_2)-f(a_1+v_1, a_2)\|_n+\|f(a_1+v_1, a_2)-f(a_1, a_2)\|_n.
  \end{aligned}
\end{equation}
Comme la dérivée partielle $\partial_x f$ est  nulle au point $a$  on sait que  pour toute constante  $\varepsilon> 0$ il y a une constante $\delta_1>0$ telle que si $|v_1|\leq \delta_1 $ alors
\[
\|f(a_1+v_1, a_2)-f(a_1, a_2)\|_n\leq \varepsilon |v_1|.
\] 
Pour l'autre terme on a, par la proposition \ref{val_medio_1},
\begin{equation}
  \begin{aligned}
   & \|f(a_1+v_1, a_2+v_2)-f(a_1+v_1, a_2)\|_n\leq \\
&\leq \sup\{\|\partial_yf(x)\|_n\,\vert\, x\in S\}|v_2|.
  \end{aligned}
\end{equation}
où $S$ est le segment d'extrémités  $(a_1+v_1, a_2)$ et $ (a_1+v_1, a_2+v_2)$. Comme la  dérivée partielle $\partial_y f$ est continue et nulle au point $a$ on sait que  pour toute constante  $\varepsilon> 0$ il existe une constante $\delta_2>0$ telle que si $\|(u_1,u_2)\|_\infty\leq \delta_2 $ alors $\|\partial_yf(a_1+u_1,a_2+u_2)\|_n\leq \varepsilon$. Si on choisit $\delta = \min\{\delta_1,\,\delta_2\}$ le segment $S$ est contenu dans la boule de rayon $\delta$ centrée au point $a$ et on obtient
\[
 \|f(a_1+v_1, a_2+v_2)-f(a_1, a_2)\|_n\leq \varepsilon |v_1|+\varepsilon |v_2|\leq 2\varepsilon \|v\|_\infty.
\]

Dans le cas général, où les dérivées partielles de $f$ au point $a$ ne sont pas spécialement nulles, on peut considérer la fonction $g(x,y)=f(x,y)-\partial_1 f(a)x-\partial_2 f(a)y$, qui a dérivées partielles nulles au point $a$. La fonction $g$ est donc différentiables. La fonction $f$ est maintenant la somme de $g$ et de la fonction linéaire et continue $(x,y)\mapsto \partial_1 f(a)x-\partial_2 f(a)y$. On verra dans la prochaine section que la somme de deux fonctions différentiables est une fonction différentiable. Par conséquent, la fonction $f$ est différentiable.

 \end{proof}

Étant donné que pour tout vecteur $u$ dans $\eR^m$ on a $\partial_uf(a)=\nabla f(a)\cdot u$, le gradient de $f$ nous donne la direction dans laquelle la croissance de $f$ est maximale. Soit $C$ une colline et soit $f$ la fonction que a chaque point $(x,y)$ de la Terre associe son altitude. Si nous voulons monter la colline le plus vite possible nous n'avons qu'a suivre la direction $\nabla f$ à chaque point. Elle est la projection sur le plan $x$-$y$ de la direction de pente maximale. Au contraire, la direction $-\nabla f$ est la direction de croissance minimale.
   
La matrice jacobienne calculé au point $a$ est la matrice associée canoniquement à l'application linéaire $df(a):\eR^m\to\eR^n$.

%+++++++++++++++++++++++++++++++++++++++++++++++++++++++++++++++++++++++++++++++++++++++++++++++++++++++++++++++++++++++++++
\section{Plan tangent}		\label{SecPlanTangent}
%+++++++++++++++++++++++++++++++++++++++++++++++++++++++++++++++++++++++++++++++++++++++++++++++++++++++++++++++++++++++++++

On a dit au début de cette section que si $f$ est une fonction de $\eR^2$ dans $\eR$ alors le graphe de $f$ est une surface à deux paramètres et que l'application affine tangente au graphe de $f$ au point $(a, f(a))$ est un plan. Maintenant on sait que ce plan est celui d'équation 
\begin{equation}
	T_a(x,y)=f(a_1,a_2)+\frac{ \partial f }{ \partial x }(a_1,a_2)(x-a_1)+\frac{ \partial f }{ \partial y }(a_1,a_2)(y-a_2).
\end{equation}
Le plan tangent au graphe de $f$ au point $a$ est le graphe de cette fonction $T_a$.


\begin{remark}
	Il existe cependant des fonctions différentiables dont les dérivées partielles ne sont pas continues. La construction d'un tel exemple est cependant délicate, et nous le ferons pas ici. Retenez cependant que si dans un exercice vous obtenez que les dérivées partielles ne sont pas continues, vous ne pouvez pas immédiatement en conclure que la fonction ne sera pas différentiable.	 
\end{remark}

%+++++++++++++++++++++++++++++++++++++++++++++++++++++++++++++++++++++++++++++++++++++++++++++++++++++++++++++++++++++++++++
%\section{Calcul de limites}
%+++++++++++++++++++++++++++++++++++++++++++++++++++++++++++++++++++++++++++++++++++++++++++++++++++++++++++++++++++++++++++

%Incidemment, le lemme \ref{Def_diff2} nous donne une nouvelle technique pour calculer des limites à plusieurs variables, similaire à celle du développement asymptotique expliquée dans la section \ref{SecTaylorR}.

%En effet, la formule \eqref{def_diff2} nous permet d'écrire $f(x)$ sous la forme
%\begin{equation}
%	f(x)=f(a)+df(a).(x-a)+\sigma_f(a,x)\| x-a \|
%\end{equation}
%où la fonction $\sigma_f$ satisfait $\lim_{x\to a}\sigma_f(a,x)=0$. Ici, $x$ et $a$ sont des éléments de $\eR^m$.

%++++++++++++++++++++++++++++++++++++++++++++++++++++++++++++++++++++++++++++++++++++++++
\section{Fonctions de classe $\mathcal{C}^1$}
%++++++++++++++++++++++++++++++++++++++++++++++++++++++++++++++++++++++++++++++++++++++++++++++++++++++++++++++++++++++++++++++

Soit $f$ une fonction différentiable de $U$, ouvert de $\eR^m$, dans $\eR^n$. L'application différentielle de $f$ est une application  de $\eR^m$ dans $\mathcal{L}(\eR^m, \eR^n)$ 
\begin{equation}
  \begin{array}{rccc}
    df : & \eR^m & \to & \mathcal{L}(\eR^m, \eR^n)\\
& a& \mapsto & df(a).
  \end{array}
\end{equation}
Nous savons depuis \ref{subsecNomrApplLin} que $\mathcal{L}(\eR^m, \eR^n)$ était un espace vectoriel normé. Si $T$ est un élément dans $\mathcal{L}(\eR^m, \eR^n)$ alors la norme de $T$ est définie par 
\[
\|T\|_{\mathcal{L}(\eR^m, \eR^n)}=\sup_{x\in\eR^m} \frac{\|T(x)\|_n}{\|x\|_m}=\sup_{\begin{subarray}{l}
    x\in\eR^m\\
\|x\|_m\leq 1
  \end{subarray}} \|T(x)\|_n.
\]

Lorsqu'il existe un $M>0$ tel que $\| df(a) \|_{\aL(\eR^m,\eR^n)}<M$ pour tout $a$ dans $U$, nous disons que la différentielle de $f$ est \defe{bornée}{bornée!différentielle} sur $U$.

\begin{definition}
	La fonction $f$ est dite \defe{de classe $\mathcal{C}^1$}{fonction!de classe  $\mathcal{C}^1$} de $U\subset\eR^m$  dans $\eR^n$ si son application différentielle $df$ est continue de $\eR^m$ dans $\mathcal{L}(\eR^m, \eR^n)$. Nous écrivons $f\in\mathcal{C}^1(U,\eR^n)$\nomenclature{$\aC^1(U,\eR^n)$}{Les applications une fois continument dérivables}.
\end{definition}

\begin{proposition}		\label{PropDerContCun}
	Une fonction \( f\colon U\to \eR^n\) où \( U\) est ouvert dans \( \eR^m\) est de classe \( C^1\) si et seulement si les dérivées partielles de $f$ existent et sont continues.
\end{proposition}

\begin{proof}
	Supposons que les dérivées partielles de $f$ existent et sont continues. Nous savons alors déjà par la proposition \ref{Diff_totale} que la fonction $f$ est différentiable et qu'elle s'exprime sous la forme
	\[
		df_a(h)=\sum_{i=1}^{m}\partial_if (a)h_i, \qquad \forall a \in U,\,\forall h\in\eR^m.
	\]
	Pour montrer que $df$ est continue, nous devons montrer que la quantité $\| df(x)-df(a) \|_{\aL(\eR^m,\eR^n)}$ peut être rendue arbitrairement petite si $\| x-a \|_m$ est rendu petit. Nous avons
	\begin{equation}
		\begin{aligned}
			\| df_x-df_a \|_{\aL}&=\sup_{\| h \|=1}\| df_x(h)-df_a(h) \|\\
			&=\sup_{\| h \|_m=1}\left\|\sum_{i=1}^{m}\left(\partial_if (x)-\partial_if (a)\right)h_i\right\|_n\leq\\
			&\leq\sup_{\| h \|_m=1}\sum_{i=1}^{m} \left\|\left(\partial_if (x)-\partial_if (a)\right)\right\|_n|h_i|\leq\\
			&\leq\sup_{\| h \|_m=1} \|h\|_\infty\sum_{i=1}^{m} \left\|\left(\partial_if (x)-\partial_if (a)\right)\right\|_n\\
			&\leq \sum_{i=1}^m\| \partial_if(x)-\partial_if(a) \|.
		\end{aligned}
	\end{equation}
	Dans ce calcul, nous avons utilisé le fait que si $\| h \|_m\leq 1$, alors $\| h \|_{\infty}\leq 1$. Étant donné la continuité de $\partial_if$, la dernière ligne peut être rendue arbitrairement petite lorsque $x$ est proche e $a$.

Supposons maintenant que $f$ soit dans $\mathcal{C}^1(U,\eR^n)$. Alors 
\[
\left\|\partial_if (x)-\partial_if (a)\right\|_n= \left\|df(x).e_i-df(a).e_i\right\|_n \leq  \left\|df(x)-df(a)\right\|_{\mathcal{L}(\eR^m,\eR^n)},
\]  
la continuité de $df$ implique donc celle de $\partial_i f$ pour tout $i$ dans $\{1,\ldots,m\}$.
\end{proof}
\begin{proposition}
  Soient $U$ un ouvert de $\eR^m$ et $V$ un ouvert de $\eR^n$. Soient $f: U\to V$  dans $\mathcal{C}^1(U,V)$ et $g: V \to \eR^p$ dans $\mathcal{C}^1(V,\eR^n)$.  Alors la fonction composée $g\circ f: U\to \eR^p $ est dans $\mathcal{C}^1(U,\eR^p)$.
\end{proposition}
\begin{proof} On fixe $a$ dans $U$ 
  \begin{equation}
    \begin{aligned}
     \big\|d(g\circ f)(x)&-d(g\circ f)(a)\big\|_{\mathcal{L}(\eR^m,\eR^p)}\\
     &=\left\|dg(f(x))\circ df(x)-dg(f(a))\circ df(a)\right\|_{\mathcal{L}(\eR^m,\eR^p)}\leq\\
&\leq \left\|\left(dg(f(x))-dg(f(a))\right)\circ df(x)\right\|_{\mathcal{L}(\eR^m,\eR^p)}+\\
&\quad+ \left\|dg(f(a))\circ \left(df(x)-df(a)\right)\right\|_{\mathcal{L}(\eR^m,\eR^p)}\leq\\
&\leq \left\|dg(f(x))-dg(f(a))\right\|_{\mathcal{L}(\eR^n,\eR^p)}\left\| df(x)\right\|_{\mathcal{L}(\eR^m,\eR^n)}+\\
&\quad+ \left\|dg(f(a))\right\|_{\mathcal{L}(\eR^n,\eR^p)}\left\| df(x)-df(a)\right\|_{\mathcal{L}(\eR^n,\eR^p)}.\\
    \end{aligned}
  \end{equation}
On peut conclure en passant à la limite $x\to a$ parce que les fonctions $f$, $g$, $df$ et $dg$ sont continues, de telle sorte que
\begin{equation}
	\begin{aligned}[]
		\lim_{x\to a} dg\big( f(x) \big)=dg\big( f(a) \big)\\
		\lim_{x\to a} df(x)=df(a).
	\end{aligned}
\end{equation}
\end{proof}

\begin{remark}
  On peut prouver le même résultat en utilisant la continuité de l'application bilinéaire 
\begin{equation}
  \begin{array}{rccc}
    \circ : & \mathcal{C}^1(U,V)\times\mathcal{C}^1(V,\eR^p)  & \to & \mathcal{L}(U, \eR^p)\\
& (T,S)& \mapsto & T\circ S.
  \end{array}
\end{equation}
\end{remark}

On fixe maintenant une définition largement utilisée dans la suite. 
\begin{definition}
	 Soient $U$ et $V$, deux ouverts de $\eR^m$. Une application $f$ de $U$ dans $V$ est un \defe{difféomorphisme}{difféomorphisme} si elle est bijective, différentiable et dont l'inverse $f^{-1}:V\to U $ est aussi différentiable. 
\end{definition}

\begin{remark}
	Il n'est pas possible d'avoir une application inversible d'un ouvert de $\eR^m$ vers un ouvert de $\eR^n$ si $m\neq n$. Il n'y a donc pas de notion de difféomorphismes entre ouverts de dimensions différentes.
\end{remark}
 
%+++++++++++++++++++++++++++++++++++++++++++++++++++++++++++++++++++++++++++++++++++++++++++++++++++++++++++++++++++++++++++
\section{Dérivée directionnelle de fonctions composées}		\label{SecDerDirFnComp}
%+++++++++++++++++++++++++++++++++++++++++++++++++++++++++++++++++++++++++++++++++++++++++++++++++++++++++++++++++++++++++++

Étant donné que nous allons voir en détail la différentielle de fonctions composées à la proposition \ref{PropDiffCompose}, nous n'allons pas rentrer dans tous les détail ici.

Nous savons déjà comment dériver les fonctions composées de $\eR$ dans $\eR$. Si nous avons deux fonctions $f\colon \eR\to \eR$ et $u\colon \eR\to \eR$, nous formons la composée $\varphi=f\circ u\colon \eR\to \eR$ dont la dérivée vaut
\begin{equation}
	\varphi'(a)=f'\big( u(a) \big)u'(a).
\end{equation}


Considérons maintenant le cas un peu plus compliqué des fonctions $f\colon \eR\to \eR$ et $u\colon \eR^2\to \eR$, et de la composée
\begin{equation}
	\begin{aligned}
		\varphi\colon \eR^2&\to \eR \\
		\varphi(x,y)&= f\big( u(x,y) \big). 
	\end{aligned}
\end{equation}
Afin de calculer la dérivée partielle de $\varphi$ par rapport à $x$, nous admettons que pour tout $a$, $b$ et $t$, il existe $c\in\mathopen[ a , a+t \mathclose]$ tel que
\begin{equation}
	u(a+t,b)=u(a,b)+t\frac{ \partial u }{ \partial x }(c,b).
\end{equation}
Cela est une généralisation immédiate du théorème \ref{ThoAccFinis}. Nous devons calculer
\begin{equation}		\label{EqPremPasDiffxvp}
	\frac{ \partial \varphi }{ \partial x }(a,b)=\lim_{t\to 0} \frac{ \varphi(a+t,b)-\varphi(a,b) }{ t }=\lim_{t\to 0} \frac{ f\big( u(a+t,b) \big)-g\big( u(a,b) \big) }{ t }.
\end{equation}
Étant donné l'hypothèse que nous avons faite sur $u$, nous avons
\begin{equation}
	f\big( u(a+t,b) \big)=f\big( u(a,b)+t\frac{ \partial u }{ \partial x }(c,b) \big).
\end{equation}
En utilisant le théorème des accroissements finis pour $f$, nous avons un point $d$ entre $u(a,b)$ et $u(a,b)+t\frac{ \partial u }{ \partial x }(c,b)$ tel que
\begin{equation}
	f\big( u(a,b)+t\frac{ \partial u }{ \partial x }(c,b) \big)=f\big( u(a,b) \big)+t\frac{ \partial u }{ \partial x }(c,b)f'(d).
\end{equation}
Le numérateur de \eqref{EqPremPasDiffxvp} devient donc
\begin{equation}
	t\frac{ \partial u }{ \partial x }(c,b)f'(d).
\end{equation}
Certes les points $c$ et $d$ sont inconnus, mais nous savons que $c$ est entre $a$ et $a+t$ ainsi que $d$ se situe entre $u(a,b)$ et $u(a,b)+t\frac{ \partial u }{ \partial x }(c,b)$. Lorsque nous prenons la limite $t\to 0$, nous avons donc $\lim_{t\to 0} c=a$ et $\lim_{t\to 0} d=u(a,b)$. Nous avons alors
\begin{equation}
	\lim_{t\to 0} \frac{ t\frac{ \partial u }{ \partial x }(c,b)f'(d) }{ t }=\frac{ \partial u }{ \partial x }(a,b)f'\big( u(a,b) \big).
\end{equation}
La formule que nous avons obtenue (de façon pas très rigoureuse) est
\begin{equation}
	\frac{ \partial  }{ \partial x }f\big( u(x,y) \big)=\frac{ \partial u }{ \partial x }(x,y)f'\big( u(x,y) \big).
\end{equation}

Prenons maintenant un cas un peu plus compliqué où nous voudrions savoir les dérivées partielles de la fonction $\varphi$ donnée par
\begin{equation}
	\varphi(x,y,z)=f\big( u(x,y),v(x,y,z) \big)
\end{equation}
où $f\colon \eR^2\to \eR$, $u\colon \eR^2\to \eR$ et $v\colon \eR^3\to \eR$. 

Commençons par la dérivée partielles par rapport à $z$. Étant donné que $\varphi$ ne dépend de $z$ que via la seconde entrée de $f$, il est normal que seule la dérivée partielle de $f$ par rapport à sa seconde entrée arrive dans la formule :
\begin{equation}
	\frac{ \partial \varphi }{ \partial z }(x,y,z)=\frac{ \partial f }{ \partial v }\big( u(x,y),v(x,y,z) \big)\frac{ \partial v }{ \partial z }(x,y,z).
\end{equation}
La dérivée partielle par rapport à $y$ demande de tenir compte en même temps de la façon dont $f$ varie avec sa première entrée et la façon dont elle varie avec sa seconde entrée; cela nous fait deux termes :
\begin{equation}
	\frac{ \partial \varphi }{ \partial y }(x,y,z)=\frac{ \partial f }{ \partial u }\big( u(x,y),v(x,y,z) \big)\frac{ \partial u }{ \partial y }(x,y)+\frac{ \partial f }{ \partial v }\big( u(x,y),v(x,y,z) \big)\frac{ \partial v }{ \partial y }(x,y,z).
\end{equation}


Cette formule a une interprétation simple. Lançons un caillou du sommet d'une falaise. Son mouvement est une chute libre avec une vitesse initiale horizontale :
\begin{subequations}
	\begin{numcases}{}
		x(t)=v_0t\\
		y(t)=h_0-\frac{ gt^2 }{ 2 }
	\end{numcases}
\end{subequations}
où $v_0$ est la vitesse initiale horizontale et $h_0$ est la hauteur de la falaise. Si nous sommes intéressés à la distance entre le caillou et le bas de la falaise (point $(0,0)$), le théorème de Pythagore nous dit que
\begin{equation}
	d(t)=\sqrt{x^2(t),y^2(t)}.
\end{equation}
Pour trouver la variation de la distance par rapport au temps il faut savoir de combien la distance varie lorsque $x$ varie et multiplier par la variation de $x$ par rapport à $t$, et puis faire la même chose avec $y$.

\begin{theorem}		\label{ThoDerDirFnComp}
	Soit $g\colon \eR^m\to \eR^n$ une fonction différentiable en $a$, et $f\colon \eR^n\to \eR^p$ une fonction différentiable en $g(a)$. Si nous définissons $\varphi(x)=(f\circ g)(x)$, alors pour tout $i=1,\ldots,m$, nous avons
	\begin{equation}
		\frac{ \partial \varphi }{ \partial x_i }(a)=\sum_{k=1}^n\frac{ \partial f }{ \partial y_k }\big( g(a) \big)\frac{ \partial g }{ \partial x_i }
	\end{equation}
	où $\frac{ \partial f }{ \partial y_k }$ dénote la dérivée partielle de $f$ par rapport à sa $k$-ième variable.
\end{theorem}

Donnons un exemple d'utilisation de cette formule. Si
\begin{equation}
	\begin{aligned}[]
		g\colon \eR^2\to \eR^3\\
		f\colon \eR^3\to \eR,
	\end{aligned}
\end{equation}
nous avons $\varphi\colon \eR^2\to \eR$. Les dérivées partielles de $\varphi$ sont données par les formules
\begin{equation}
	\frac{ \partial \varphi }{ \partial x }(x,y)=\frac{ \partial f }{ \partial x_1 }\big( g(x,y) \big)\frac{ \partial g_1 }{ \partial x }(x,y)+\frac{ \partial f }{ \partial x_2 }\big( g(x,y) \big)\frac{ \partial g_2 }{ \partial y }(x,y)+\frac{ \partial f }{ \partial x_3 }\big( g(x,y) \big)\frac{ \partial g_3 }{ \partial x }(x,y)
\end{equation}
et
\begin{equation}
	\frac{ \partial \varphi }{ \partial y }(x,y)=\frac{ \partial f }{ \partial x_1 }\big( g(x,y) \big)\frac{ \partial g_1 }{ \partial y }(x,y)+\frac{ \partial f }{ \partial x_2 }\big( g(x,y) \big)\frac{ \partial g_2 }{ \partial y }(x,y)+\frac{ \partial f }{ \partial x_3 }\big( g(x,y) \big)\frac{ \partial g_3 }{ \partial y }(x,y)
\end{equation}
Notez que les dérivées de $\varphi$ et des composantes de $g$ sont calculées en $(x,y)$, tandis que celles de $f$ sont calculées en $g(x,y)$.

%++++++++++++++++++++++++++++++++++++++++++++++++++++++++++++++++++++++++++++++++++++++++++++++++++++++++++++++++++++++
\section{Théorèmes des accroissements finis}		\label{SecThoAccrsFinis}
%++++++++++++++++++++++++++++++++++++++++++++++++++++++++++++++++++++++++++++++++++++++++++++++++++++++++++++++++++++++

Nous avons déjà démontré (lemme \ref{LemdfaSurLesPartielles}) que si $f$ est différentiable au point $x$ alors  $df_x(u)=\partial_uf(x)$. Une importante conséquence est le théorème des accroissements finis
\begin{theorem}[Accroissements finis, inégalité de la moyenne]\label{val_medio_2}
   Soit $U$ un ouvert dans $\eR^m$ et soit $f:U\to\eR^n$ une fonction différentiable. Soient $a$ et $b$ deux point dans $U$, $a\neq b$, tels que le segment $[a,b]$ soit contenu dans $U$. Alors
\[
\|f(b)-f(a)\|_n\leq \sup_{x\in[a,b]}\|df(x)\|_{\mathcal{L}(\eR^m,\eR^n)}\|b-a\|_m.
\]
\end{theorem}
\index{application!différentiable}
\index{inégalité!de la moyenne}
\index{théorème!accroissements finis!forme générale}.

\begin{proof}
 On utilise le théorème \ref{val_medio_1} et le fait que 
\[
\|\partial_u f(x)\|_n\leq \|df(x)\|_{\mathcal{L}(\eR^m,\eR^n)}\|u\|_m,
\]
pour tout $u$ dans $\eR^m$.
\end{proof}

La proposition suivante est une application fondamentale du théorème des accroissements finis \ref{val_medio_2}.
\begin{proposition}		\label{PropAnnulationEtConstance}
	Soit $U$ un ouvert connexe par arcs de $\eR^m$ et une fonction $f\colon U\to \eR^n$. Les conditions suivantes sont équivalentes :
	\begin{enumerate}
		\item\label{ItemPropCstDiffZeroi}
			$f$ est constante;
		\item\label{ItemPropCstDiffZeroii}
			$f$ est différentiable et $df(a)=0$ pour tout $a\in U$;
		\item\label{ItemPropCstDiffZeroiii}
			les dérivées partielles $\partial_1f,\ldots,\partial_mf$ existent et sont nulles sur $U$.
	\end{enumerate}
\end{proposition}
\index{connexité!par arc!fonction différentiable}
\index{différentiabilité}

\begin{proof}
	Nous allons démonter les équivalences en plusieurs étapes. D'abord \ref{ItemPropCstDiffZeroi} $\Rightarrow$ \ref{ItemPropCstDiffZeroii}, puis \ref{ItemPropCstDiffZeroii} $\Rightarrow$ \ref{ItemPropCstDiffZeroiii}, ensuite \ref{ItemPropCstDiffZeroiii} $\Rightarrow$ \ref{ItemPropCstDiffZeroii} et enfin \ref{ItemPropCstDiffZeroii} $\Rightarrow$ \ref{ItemPropCstDiffZeroi}.

	Commençons par montrer que la condition \ref{ItemPropCstDiffZeroi} implique la condition \ref{ItemPropCstDiffZeroii}. Si $f(x)$ est constante, alors la condition \eqref{EqCritereDefDiff} est vite vérifiée en posant $T(h)=0$.

	Afin de voir que la condition \ref{ItemPropCstDiffZeroii} implique la condition \ref{ItemPropCstDiffZeroiii}, remarquons d'abord que la différentiabilité de $f$ implique que les dérivées partielles existent (proposition \ref{diff1}) et que nous avons l'égalité $df(a).u=\sum_iu_i\partial_if(a)$ pour tout $u\in\eR^m$ (lemme \ref{LemdfaSurLesPartielles}). L'annulation de $\sum_iu_i\partial_if(a)$ pour tout $u$ implique l'annulation des $\partial_if(a)$ pour tout $i$.

	Prouvons maintenant que la propriété \ref{ItemPropCstDiffZeroiii} implique la propriété \ref{ItemPropCstDiffZeroii}. D'abord, par la proposition \ref{Diff_totale}, l'existence et la continuité des dérivées partielles $\partial_if(a)$ implique la différentiabilité de $f$. Ensuite, la formule $df(a).u=\sum_i u_i\partial_if(a)$ implique que $df(a)=0$. 
	
	
	Il reste à montrer que \ref{ItemPropCstDiffZeroii} implique la condition \ref{ItemPropCstDiffZeroi}, c'est à dire que l'annulation de la différentielle implique la constance de la fonction. C'est ici que nous allons utiliser le théorème des accroissements finis. En effet, si $a$ et $b$ sont des points de $U$, le théorème \ref{val_medio_2} nous dit que
	\begin{equation}
		\|f(b)-f(a)\|_n\leq \sup_{x\in[a,b]}\|df(x)\|_{\mathcal{L}(\eR^m,\eR^n)}\|b-a\|_m.
	\end{equation}
	Mais $\| df(x) \|=0$ pour tout $x\in U$, donc ce supremum est nul et $f(b)=f(a)$, ce qui signifie la constance de la fonction.
\end{proof}

%\begin{proof}
%  \begin{itemize}
%  \item Le théorème \ref{val_medio_2} nous dit que si la différentielle de $f$ est nulle alors $f$ est constante sur chaque segment contenu dans $U$. Cela nous dit que $f$ est constante sur chaque boule contenue dans $U$, donc $f $ est localement constante. Il est possible de démontrer que toute fonction localement constante sur un connexe est constante.  
%\item Si toutes les dérivées partielles $\partial_1 f, \ldots, \partial_m f $ existents et sont identiquement nulles sur $U$ alors $f$ est différentiable et sa différentielle est identiquement nulle. On utilise la première partie de la preuve pour conclure. 
%  \end{itemize}
%\end{proof}

%+++++++++++++++++++++++++++++++++++++++++++++++++++++++++++++++++++++++++++++++++++++++++++++++++++++++++++++++++++++++++++
\section{Fonctions Lipschitziennes}
%+++++++++++++++++++++++++++++++++++++++++++++++++++++++++++++++++++++++++++++++++++++++++++++++++++++++++++++++++++++++++++

\begin{definition}
	Soient $A$ un intervalle de $\eR^m$ (non vide et non réduit à un point), $f\colon A\to \eR^m$ une fonction et $k$ un réel strictement positif. On dit que $f$ est \defe{Lipschitzienne}{Lipschitzienne} de constante $k$ sur $A$ si pour tout $x$ et $y$ dans $A$,
	\begin{equation}
		\|  f(x)-f(y) \|_n\leq k\| x-y \|_m.
	\end{equation}
\end{definition}

\begin{proposition}
  Soit  $U$ un ouvert convexe  de $\eR^m$, et soit $f:U\to \eR^n$ une fonction différentiable. La fonction $f$ est Lipschitzienne sur $U$ si et seulement si $df$ est bornée sur $U$.  
\end{proposition}
\begin{proof}
	Le fait que l'application différentielle $df$ soit bornée signifie qu'il existe un $M>0$ dans $\eR$ tel que $\|df(a)\|_{\mathcal{L}(\eR^m,\eR^n)}\leq M$, pour tout $a$ dans $U$. Si cela est le cas, alors le théorème \ref{val_medio_2} et la convexité\footnote{La convexité de $U$ sert à assurer que la droite reliant $a$ à $b$ est contenue dans $U$; c'est ce que nous utilisons dans la démonstration du théorème \ref{val_medio_2}.} de $U$ impliquent évidemment que $f$ est de Lipschitz de constante plus petite ou égale à $M$.
	
	Inversement, si $f$ est de Lipschitz de constante $k$, alors pour tout $a$ dans $U$ et $u$ dans $\eR^m$ on a 
	\[
		\left\|\frac{f(a+tu)-f(a)}{t}\right\|_n\leq k \|u\|_m,
	\]   
	En passant à la limite pour $t\to 0$ on a 
	\[
		\|\partial_u f(a)\|_n=\|df(a).u\|_n\leq k \|u\|_m,
	\]
	donc la norme de $df(a)$ est majorée par $k$ pour tout $a$ dans $U$.   
\end{proof}

Notez cependant qu'une fonction peut être Lipschitzienne sans être différentiable.

\begin{proposition} \label{PropFZgFTEW}
    Une fonction Lipschitzienne \( f\colon \eR\to \eR\) est continue.
\end{proposition}

\begin{proof}
    Prouvons la continuité en \( a\in \eR\). Pour tout \( x\) nous avons
    \begin{equation}
        \big| f(x)-f(a) \big|\leq k| x-a |.
    \end{equation}
    Si \( \epsilon>0\) est donné, il suffit de prendre \( \delta<\frac{ \epsilon }{ k }\) pour avoir
    \begin{equation}
        \big| f(x)-f(a) \big|\leq k\frac{ \epsilon }{ k }=\epsilon.
    \end{equation}
    Donc \( f\) est continue en \( a\).
\end{proof}
Implicitement nous avons utilisé le théorème \ref{ThoESCaraB}.

%++++++++++++++++++++++++++++++++++++++++++++++++++++++++++++++++++++++++++++++++++++++++
\section{Différentielles d'ordre supérieur}		\label{SecDiffOrdSup}
%++++++++++++++++++++++++++++++++++++++++++++++++++++++++++++++++++++++++++++++++++++++++++++++++++++++++++++++++++++++++++++++
\begin{definition}
	Soit $U$ un ouvert de $\eR^m$ et  $f:U\subset\eR^m\to \eR^n$ une fonction. La fonction $f$ est dite \defe{deux fois différentiable}{différentiable!deux fois} au point $a$ dans $U$,  si $f$ est différentiable dans un voisinage de $a$, et sa différentielle $df$ est différentiable au point $a$ en tant que application de $U$ dans $\mathcal{L}(\eR^m, \eR^n)$.  

%--------------------------------------------------------------------------------------------------------------------------- 
\subsection{Identification des espaces d'applications multilinéaires}
%---------------------------------------------------------------------------------------------------------------------------

La fonction $f$ sera dite deux fois différentiable sur l'ensemble $U$ si elle est deux fois différentiable en chaque point de $U$.
\end{definition}
La différentielle de la différentielle de $f$ est notée 
\[
d(df)(a)=d^2f(a),
\]
et est une application de $U$ dans $\mathcal{L}(\eR^m,\mathcal{L}(\eR^m, \eR^n) )$. Comme on a vu dans la proposition \ref{isom_isom}, l'espace $\mathcal{L}(\eR^m,\mathcal{L}(\eR^m, \eR^n) )$ est isométriquement isomorphe à l'espace $\mathcal{L}(\eR^m\times\eR^m, \eR^n )$. On verra comment cette propriété  est utilisé dans l'exemple \ref{bilin_2diff}.


Soient \( V\) et \( W\) deux espace vectoriel normés de dimension finie et \( \mO\) un ouvert autour de \( x\in V\). D'une part l'espace des applications linéaires \( \aL(V,W)\) est lui-même un espace vectoriel normé de dimension finie, et on peut identifier \(  \aL\big( V,\aL^{(k)}(V,W) \big)\)\nomenclature[Y]{\( \aL^{(n)}(V,W)\)}{L'espace des applications \( n\)-linéaires \( V^n\to W\)} avec \( \aL^{(k+1)}(V,W)\), ce qui nous permet de dire que la \( k\)\ieme\ différentielle est une application
\begin{equation}
    d^kf\colon \mO\to \aL^{(k)}(V,W).
\end{equation}
Plus précisément, l'identification se fait de la façon suivante : si \( \omega\in \aL\big( V,\aL^{(k)}(V,W) \big)\), alors \( \omega\) vu dans \( \aL^{(k+1)}(V,W)\) est définie par
\begin{equation}
    \omega(u_1,\ldots, u_{k+1})=\omega(u_1)(u_2,\ldots, u_{k+1}).    
\end{equation}


Cela étant posé nous pouvons donner les définition.

%--------------------------------------------------------------------------------------------------------------------------- 
\subsection{Fonctions différentiables plusieurs fois}
%---------------------------------------------------------------------------------------------------------------------------

\begin{definition}[\cite{ZCKMFRg}]
    La fonction \( f\colon \mO\subset V\to W\) est
    \begin{enumerate}
        \item
            de classe \( C^0\) si elle est continue,
        \item
            de classe \( C^1\) si \( df\colon \mO\to \aL(V,W)\) est continue,
        \item
            de classe \( C^k\) si \( d^kf\colon \mO\to \aL^{(k)}(V,W)\) est continue,
        \item
            de classe \(  C^{\infty}\) si \( f\) est dans \( \bigcap_{k=0}^{\infty}C^k(V,W)\).
    \end{enumerate}
\end{definition}
\index{application!différentiable}
\index{application!de classe \( C^k\)}

\begin{definition}
    Un \( C^k\)-difféomorphisme\index{difféomorphisme!de classe \( C^k\)} est une application inversible de classe \( C^k\) dont l'inverse est également de classe \( C^k\).
\end{definition}

\begin{example} \label{ExZHZYcNH}
    Voyons commet la différentielle seconde fonctionne. Soit \( f\in C^2(V,W)\); nous notons \( \varphi=df\colon V\to \aL(V,W)\) et donc nous voulons étudier la fonction
    \begin{equation}
        d\varphi\colon V\to \aL\big( V,\aL(V,W) \big).
    \end{equation}
    Si \( a,u\in V\)  nous avons
    \begin{equation}
        d\varphi_a(u)=\Dsdd{ \varphi(a+tu) }{t}{0}
    \end{equation}
    qui est une dérivée dans \( \aL(V,W)\) -- pas de problèmes : c'est un espace vectoriel normé de dimension finie. Par linéarité, nous pouvons faire entrer l'argument de \( d\varphi_a(u)\) dans la dérivée :
    \begin{subequations}
        \begin{align}
            d\varphi_a(u)v&=\Dsdd{ \varphi(a+tu)v }{t}{0}\\
            &=\Dsdd{ df_{a+tu}(v) }{t}{0}\\
            &=\Dsdd{  \Dsdd{ f(a+tu+sv) }{s}{0}  }{t}{0}\\
            &=\Dsdd{ \frac{ \partial f }{ \partial v }(a+tu) }{t}{0}\\
            &=\frac{ \partial^2f  }{ \partial u\partial v }(a).
        \end{align}
    \end{subequations}
    Par conséquent nous voyons
    \begin{equation}\label{EqQHINNtD}
        \begin{aligned}
            d^2f\colon V&\to \aL^{(2)}(V,W) \\
            d^2f_a(u,v)&=\frac{ \partial^2f  }{ \partial u\partial v }(a). 
        \end{aligned}
    \end{equation}
    
    Dans le cas d'une fonction \( f\colon \eR\to \eR\), nous avons une seule direction et par linéarité de \eqref{EqQHINNtD} par rapport à \( u\) et \( v\), nous avons
    \begin{equation}
        d^2f_a(u,v)=f''(a)uv
    \end{equation}
    où les produits sont des produits usuels dans \( \eR\) et \( f''\) est la dérivée seconde usuelle.
\end{example}


\begin{example}\label{bilin_2diff}
	Soit $B:\eR^m\times \eR^m\to\eR^n$ une application bilinéaire. On définit $f:\eR^m\to\eR^n$ par $f(x)=B(x,x)$. Le lemme \ref{bilin_diff} nous dit que $B$ est différentiable. Cela implique la différentiabilité de $f$. Pour trouver la différentielle de la fonction $f$, nous écrivons $f=B\circ s$ où $s\colon \eR^m\to \eR^m\times\eR^m$ est l'application $s(x)=(x,x)$. En utilisant la règle de différentiation de fonctions composées,
	\begin{equation}
		df(a)=dB\big( s(a) \big)\circ ds(a).
	\end{equation}
	Mais $ds(a).u=(u,u)$ parce que $s(a+h)-s(a)-(h,h)=0$. Par conséquent,
	\begin{equation}		\label{EqdBsaExp}
		df(a).u=dB\big( s(a) \big)(u,u)=B(u,a)+B(a,u)
	\end{equation}
	où nous avons utilisé la formule du lemme \ref{bilin_diff}. La formule \eqref{EqdBsaExp} peut être écrite sous la forme compacte
	\begin{equation}
		df(a)=B(\cdot,a)+B(a,\cdot)
	\end{equation}
	La fonction $df(a)$ ainsi écrite est linéaire par rapport à $a$, donc différentiable. En outre elle coïncide avec sa différentielle, comme on a vu dans le remarque \ref{rk_lin}, au sens que la différentielle de $df$ au point $a$ sera l'application que à chaque $x$ dans $\eR^m$ associe l'application linéaire $B(x,\cdot)+B(\cdot, x)$. On voit bien que $d^2f$ au point $a$ est une application de $\eR^m$ vers l'espace des applications linéaires $\mathcal{L}(\eR^m, \eR^n)$. On peut utiliser d'autre part l'isomorphisme des espaces $\mathcal{L}(\eR^m,\mathcal{L}(\eR^m, \eR^n) )$ et $\mathcal{L}(\eR^m\times\eR^m, \eR^n )$ et dire que, une fois que $a$ est fixé, l'application $d^2f(a)$ est une application bilinéaire sur $\eR^m\times\eR^m$. On écrit alors $d^2f(a)(x,y)=B(x,y)+B(y,x)$.   
\end{example}

Une condition nécessaire et suffisante pour l'existence de la différentielle seconde est la suivante
\begin{proposition}
   Soit $U$ un ouvert de $\eR^m$ et  $f:U\subset\eR^m\to \eR^n$ une fonction. La fonction $f$ est deux fois différentiable au point $a$ si et seulement si les dérivées partielles $\partial_1 f, \ldots, \partial_m f $ sont différentiables en $a$. 
\end{proposition}
Cela veut dire, en particulier, que $f$ est deux fois différentiable si et seulement si ses dérivées partielles secondes, $\partial_i\partial_j f$, pour toute couple d'indices $i,j$  dans $\{1,\ldots, m\}$, existent et sont continues. Pour les différentielles d'ordre supérieur on a la définition suivante.
\begin{definition}
  Soit $U$ un ouvert de $\eR^m$ et  $f:U\subset\eR^m\to \eR^n$ une fonction. On dit que $f$ est de classe $\mathcal{C}^k$, c'est à dire que $f$ est $k$ fois différentiable,  pour $k$ dans $\eN$, $k\geq 1$, si les dérivées partielles $\partial_1 f, \ldots, \partial_m f $ existent et sont de classe $\mathcal{C}^{k-1}$. 
\end{definition}
La différentielle seconde dans l'exemple  \ref{bilin_2diff} est symétrique, c'est à dire que $d^2f(a)(x_1,x_2)=d^2f(a)(x_2,x_1)$. En fait toute différentielle seconde est symétrique.  


\begin{theorem}[Schwarz]\label{Schwarz}
 Soit $U$ un ouvert de $\eR^m$ et  $f:U\subset\eR^m\to \eR^n$ une fonction de classe $\mathcal{C}^2$. Alors, pour toute couple $i,j$ d'indices dans $\{1,\ldots, m\}$ et pour tout point $a$ dans $U$, on a 
\[
\frac{\partial^2 f}{\partial  x_i\partial x_j}(a)=\frac{\partial^2 f}{\partial  x_j\partial x_i}(a).
\]
\end{theorem}
\begin{proof}
  Pour simplifier l'exposition nous nous limitons ici au cas $m=2$. Soit $(h,g)$ un vecteur fixé dans $\eR^2$. Pour tout  $v=(x,y)$ dans $\eR^2$ on note
  \begin{equation}
    \begin{array}{c}
      \Delta_h f(v)=f(v+he_1) -f(v) = f(x+h,y)-f(x,y),\\ 
      \Delta_g f(v)=f(v+ge_2) -f(v) = f(x,y+g)-f(x,y),\\ 
    \end{array}
  \end{equation}
Nous avons
\begin{equation}
  \begin{array}{c}
   \Delta_g   \Delta_h f(v)=\left(f(x+h,y+g)-f(x,y+g)\right)-\left(f(x+h,y)-f(x,y)\right),\\
   \Delta_h   \Delta_g f(v)=\left(f(x+h,y+g)-f(x+h,y)\right)-\left(f(x,y+g)-f(x,y)\right),
  \end{array}
\end{equation}
donc, 
\begin{equation}
  \frac{1}{g} \Delta_g  \left(\frac{1}{h} \Delta_h f(v)\right) = \frac{1}{h} \Delta_h \left(\frac{1}{g} \Delta_g f(v)\right).
\end{equation}
On utilise alors le théorème des accroissements finis
\[
\frac{1}{h} \Delta_h f(v)=\frac{1}{h}\big(f(x+h,y)-f(x,y)\big)=\frac{1}{h}\partial_1f(x+t_1h,y )h=\partial_1f(x+t_1h, y),
\]
pour un certain $t_1$ dans $]0,1[$. De même on obtient 
\[
\frac{1}{g} \Delta_g f(v)= \partial_2 f(x, y+t_2g),
\]
pour un certain $t_2$ dans $]0,1[$. Alors
 \begin{equation}
  \frac{1}{g} \Delta_g  \big(\partial_1f(x+t_1h, y)\big) = \frac{1}{h} \Delta_h \big(\partial_2 f(x, y+t_2g)\big).
\end{equation}
En appliquant encore une fois le théorème des accroissements finis on a
 \begin{equation}
  \partial_2\partial_1f(x+t_1h, y+s_1g) = \partial_1\partial_2 f(x+s_2h, y+t_2g).
\end{equation} 
Il suffit maintenant de passer à la limite pour $(h,g) \to (0,0)$ et de se souvenir du fait que $f$ est $\mathcal{C}^2$ seulement si ses dérivées partielles secondes sont continues pour avoir $\partial_2\partial_1f(v)=\partial_1\partial_2 f(v)$.
\end{proof}
Si $f$ est deux fois différentiable $d^2f(a)$ est l'application bilinéaire associée avec la matrice symétrique
\begin{equation}
 H_f(a)= \begin{pmatrix}
    \partial^2_1f(a)& \ldots& \partial_1\partial_m f(a)\\
    \vdots& \ddots& \vdots\\
    \partial_1\partial_m f(a)&\ldots&\partial^2_1f(a),
  \end{pmatrix}
\end{equation}
Cette matrice est dite la matrice \defe{hessienne}{hessienne} de $f$. 

\begin{example}
  Montrons qu'il n'existe pas de fonctions $f$ de classe $\mathcal{C}^2$ telles que $\partial_xf(x,y)= 5\sin x$ et $\partial_y(x,y)=6x+y$.  Ceci est vite fait en appliquant le théorème de Schwarz, \ref{Schwarz}; ce que nous trouvons est
\[
\partial_y (\partial_xf)= 0\neq \partial_x(\partial_yf)= 6.
\]
Donc, l'existence d'une fonction $f$ de classe $\mathcal{C}^2$ telle que $\partial_x(x,y)= 5\sin x$ et $\partial_yf(x,y)=6x+y$ serait en contradiction avec le théorème.  
\end{example}

%+++++++++++++++++++++++++++++++++++++++++++++++++++++++++++++++++++++++++++++++++++++++++++++++++++++++++++++++++++++++++++
\section{Développement asymptotique, théorème de Taylor}
%+++++++++++++++++++++++++++++++++++++++++++++++++++++++++++++++++++++++++++++++++++++++++++++++++++++++++++++++++++++++++++
\label{AppSecTaylorR}

Le théorème suivant généralise à l'utilisation de toutes les dérivées disponibles le résultat de développement limité donné par la proposition \ref{PropUTenzfQ}.
\begin{theorem}[Théorème de Taylor]		\label{ThoTaylor}
Soit $I\subset$ un intervalle non vide et non réduit à un point de $\eR$ ainsi que $a\in I$. Soit une fonction $f\colon I\to \eR$ telle que $f^{(n)}(a)$ existe. Alors il existe une fonction $\epsilon$ définie sur $I$ et à valeurs dans $\eR$ vérifiant les deux conditions suivantes :
\begin{subequations}		\label{SubEqsDevTauil}
	\begin{align}
		\lim_{x\to a}\epsilon(x)&=0,\\
		f(x)&=T^a_{f,n}(x)+\epsilon(x)(x-a)^{n}	&&\forall x\in I		\label{subeqfTepseqb}
	\end{align}
\end{subequations}
où $T^a_{f,n}(x)=\sum_{k=0}^n\frac{ f^{(k)}(a) }{ k! }(x-a)^k$ et $f^{(k)}$ dénote la $k$-ième dérivée de $f$ (en particulier, $f^{(0)}=f$, $f^{(1)}=f'$).\nomenclature{$f^{(n)}$}{La $n$-ième dérivée de la fonction $f$}
\end{theorem}

Nous insistons sur le fait que la formule \eqref{subeqfTepseqb} est une égalité, et non une approximation. Ce qui serait une approximation serait de récrire la formule dans le terme contenant $\epsilon$.

Le polynôme $T^a_{f,n}$ est le \defe{polynôme de Taylor}{Taylor} de $f$ au point $a$ à l'ordre $n$.  Une preuve du théorème peut être trouvée dans \cite{TrenchRealAnalisys}, théorème 2.5.1 à la page 99. La version donnée ici est inspirée de l'article sur \wikipedia{fr}{Développement_de_Taylor}{Wikipédia}\footnote{http://fr.wikipedia.org/wiki/Développement\_de\_Taylor}, qui donne également une preuve du résultat.

En termes de notations, nous définissons l'ensemble $o(x)$\nomenclature{$o(x)$}{fonction tendant rapidement vers zéro} l'ensemble des fonctions $f$ telles que
\begin{equation}
	\lim_{x\to 0} \frac{ f(x) }{ x }=0.
\end{equation}
Plus généralement si $g$ est une fonction telle que $\lim_{x\to 0} g(x)=0$, nous disons $f\in o(g)$ si
\begin{equation}
	\lim_{x\to 0} \frac{ f(x) }{ g(x) }=0.
\end{equation}
De façon intuitive, l'ensemble $o(g)$ est l'ensemble des fonctions qui tendent vers zéro «plus vite» que $g$.


Nous pouvons donner un énoncé alternatif au théorème \ref{ThoTaylor} en définissant $h(x)=\epsilon(x+a)x^n$. Cette fonction est définie exprès pour avoir
\begin{equation}
	h(x-a)=\epsilon(x)(x-a)^n,
\end{equation}
et donc
\begin{equation}
	\lim_{x\to 0} \frac{ h(x) }{ x^n }=\lim_{x\to 0} \epsilon(x-a)=\lim_{x\to a}\epsilon(x)=0. 
\end{equation}
Donc $h\in o(x^n)$.

Le théorème dit donc qu'il existe une fonction $\alpha\in o(x^n)$ telle que
\begin{equation}
	f(x)=T^a_{f,n}(x)+\alpha(x-a).
\end{equation}
pour tout $x\in I$. 

\begin{example}
	Le développement du cosinus est donné par
	\begin{equation}
		\cos(x)=1-\frac{ x^2 }{ 2 }+\frac{ x^4 }{ 4! }-\frac{ x^6 }{ 6! }\cdots
	\end{equation}
	Nous avons donc l'existence d'une fonction $h_1\in o(x^2)$ telle que $\cos(x)=1-\frac{ x^2 }{ 2 }+h_1(x)$. Il existe aussi une autre fonction $h_2\in o(x^4)$ telle que $\cos(x)=1-\frac{ x^2 }{ 2 }+\frac{ x^4 }{ 4! }+h_2(x)$.
\end{example}

\begin{example}		\label{ExempleUtlDev}
	Une des façons les plus courantes d'utiliser les formules \eqref{SubEqsDevTauil} est de développer $f(a+t)$ pour des petits $t$ en posant $x=a+t$ dans la formule :
	\begin{equation}	\label{EqDevfautouraeps}
		f(a+t)=f(a)+f'(a)t+f''(a)\frac{ t^2 }{ 2 }+\epsilon(a+t)t^2
	\end{equation}
	avec $\lim_{t\to 0} \epsilon(a+t)=0$. Ici, la fonction $T$ dont on parle dans le théorème est $T_{f,2}^a(a+t)=f(a)+f'(a)t+f''(a)\frac{ t^2 }{2}$.

	Lorsque $x$ et $y$ sont deux nombres «proches\footnote{par exemple dans une limite $(x,y)\to(h,h)$.}», nous pouvons développer $f(y)$ autour de $f(x)$ :
	\begin{equation}		\label{Eqfydevfx}
		f(y)=f(x)+f'(x)(y-x)+f''(x)\frac{ (y-x)^2 }{ 2 }+\epsilon(y-x)(y-x)^2,
	\end{equation}
	et donc écrire
	\begin{equation}
		f(x)-f(y)=-f'(x)(y-x)-f''(x)\frac{ (y-x)^2 }{ 2 }-\epsilon(y-x)(y-x)^2.
	\end{equation}
	De cette manière nous obtenons une formule qui ne contient plus que $y$ dans la différence $y-x$.
\end{example}

%---------------------------------------------------------------------------------------------------------------------------
\subsection{Fonctions «petit o» }
%---------------------------------------------------------------------------------------------------------------------------

Nous voulons formaliser l'idée d'une fonction qui tend vers zéro \og plus vite\fg{} qu'une autre. Nous disons que $f\in o\big(\varphi(x)\big)$ si
\begin{equation}
    \lim_{x\to 0} \frac{ f(x) }{ \varphi(x) }=0.
\end{equation}
En particulier, nous disons que $f\in o(x)$ lorsque $\lim_{x\to 0} f(x)/x=0$.

\begin{remark}
    À titre personnel, l'auteur de ces lignes déconseille d'utiliser cette notation qui est un peu casse-figure pour qui ne la maîtrise pas bien.
\end{remark}

%---------------------------------------------------------------------------------------------------------------------------
\subsection{Formule et reste}
%---------------------------------------------------------------------------------------------------------------------------

\begin{proposition}     \label{PropDevTaylorPol}
    Soient $f\colon I\subset\eR\to \eR$ et $a\in\Int(I)$. Soit un entier $k\geq 1$. Si $f$ est $k$ fois dérivable en $a$, alors il existe un et un seul polynôme $P$ de degré $\leq k$ tel que
    \begin{equation}
        f(x)-P(x-a)\in o\big( | x-a |^k \big)
    \end{equation}
    lorsque $x\to a$, $x\neq a$. Ce polynôme  est donné par
    \begin{equation}
        P(h)=f(a)+f'(a)h+\frac{ f''(a) }{ 2! }h^2+\ldots+\frac{ f^{(k)}(a) }{ k! }h^k.
    \end{equation}
    Notons encore deux façons alternatives d'écrire le résultat. Si \( f\in C^k\) il existe une fonction \( \alpha\) telle que \( \lim_{t\to 0} \alpha(t)=0\) et
    \begin{equation}
        f(x)=\sum_{n=0}^k\frac{ f^{(n)}(a) }{ n! }(x-a)^n+(x-a)^n\alpha(x-a).
    \end{equation}
    Si \( f\in C^{k+1}\) alors
    \begin{equation}        \label{EquQtpoN}
        f(x)=\sum_{n=0}^k\frac{ f^{(n)}(a) }{ n! }(x-a)^n+(x-a)^{n+1}\xi(x-a)
    \end{equation}
    où \( \xi\) est une fonction telle que \( \xi(t)\) tend vers une constante lorsque \( t\to 0\).
\end{proposition}

La proposition suivant donne une intéressante façon de trouver le reste d'un développement de Taylor.
\begin{proposition}     \label{PropResteTaylorc}
Soient $I$, un intervalle dans $\eR$ et $f\colon I\to \eR$ une fonction de classe $C^k$ sur $I$ telle que $f^{(k+1)}$ existe sur $I$. Soient $a\in\Int(I)$ et $x\in I$. Alors il existe $c$ strictement compris entre $x$ et $a$ tel que 
\begin{equation}
    R_{f,a,k}(x)=\frac{ f^{(k+1)}(c) }{ (k+1)! }(x-a)^{k+1}.
\end{equation}
\end{proposition}

%--------------------------------------------------------------------------------------------------------------------------- 
\subsection{Reste intégral}
%---------------------------------------------------------------------------------------------------------------------------

\begin{proposition}[Formule de Taylor avec reste intégral\cite{VBYOJrU}]\label{PropAXaSClx}
    Soient \( X\) et \( Y\) des espaces normés et un ouvert \( \mO\subset X\). Si \( f\in C^m(\mO,Y)\) et si \( [p,x]\subset \mO\) alors
    \begin{equation}
        \begin{aligned}[]
            f(x)=f(p)&+\sum_{k=1}^{m-1}\frac{1}{ k! }(d^kf)_p (x-p)^k \\
            &+\frac{1}{ (m-1)! }\int_0^1(1-t)^{m-1}(d^mf)_{ p+t(x-p) }(x-p)^m \
        \end{aligned}
    \end{equation}
    où \( \omega_pu^k\) signifie \( \omega_p(u,\ldots, u)\) lorsque \( \omega\in \Omega^k\).
\end{proposition}
\index{formule!Taylor!reste intégral}
Comme expliqué dans l'exemple \ref{ExZHZYcNH}, toute ces applications de différentielles se réduisent à des termes de la forme
\begin{equation}
    f^{(k)}(p)(x-p)^k
\end{equation}
dans le cas d'une fonction \( \eR\to\eR\).

%---------------------------------------------------------------------------------------------------------------------------
\subsection{Exemple : un calcul heuristique de limite}
%---------------------------------------------------------------------------------------------------------------------------
\label{SubSecCalcLimHeuris}

Soit à calculer la limite suivante :
\begin{equation}
    \lim_{x\to 0} \frac{  e^{-2\cos(x)+2}\sin(x) }{ \sqrt{ e^{2\cos(x)+2}}-1 }.
\end{equation}
La stratégie que nous allons suivre pour calculer cette limite est de développer certaines parties de l'expression en série de Taylor, afin de simplifier l'expression. La première chose à faire est de remplacer $ e^{y(x)}$ par $1+y(x)$ lorsque $y(x)\to 0$. La limite devient
\begin{equation}
    \lim_{x\to 0} \frac{ \big( -2\cos(x)+3 \big)\sin(x) }{ \sqrt{-2\cos(x)+2} }.
\end{equation}
Nous allons maintenant remplacer $\cos(x)$ par $1$ au numérateur et par $1-x^2/2$ au dénominateur. Pourquoi ? Parce que le cosinus du dénominateur est dans une racine, donc nous nous attendons à ce que le terme de degré deux du cosinus donne un degré un en dehors de la racine, alors que du degré un est exactement ce que nous avons au numérateur : le développement du sinus commence par $x$.

Nous calculons donc
\begin{equation}
    \begin{aligned}[]
        \lim_{x\to 0} \frac{ \sin(x) }{ \sqrt{-2\left( 1-\frac{ x^2 }{ 2 } \right)+2} }=\lim_{x\to 0} \frac{ \sin(x) }{ x }=1.
    \end{aligned}
\end{equation}
Tout ceci n'est évidement pas très rigoureux, mais en principe vous avez tous les éléments en main pour justifier les étapes.
