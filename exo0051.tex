% This is part of Exercices et corrigés de CdI-1
% Copyright (c) 2011
%   Laurent Claessens
% See the file fdl-1.3.txt for copying conditions.

\begin{exercice}\label{exo0051}

Calculez les différentielles en  $(0,0)$ des fonctions suivantes:

\begin{enumerate}
\item $f_1(x,y)=\left\{\begin{array}{cl}
                        \dfrac{\sin(\pi(15x^3+x^2+(y-2)^2))}{\pi(x^2+(y-2)^2)} & {\rm quand \ le \ d\acute{e}nominateur \ est \ non \ nul}  \\
                        1             						    & {\rm ailleurs}
                        
                        \end{array}\right.$
\item $f_2(x,y)= (x+1)^{x+y}$
\item $g_1:\Rn^2\lra\Rn^2:(u,v)\lra(\ln(\sin^2(u)+1),uv)$
\item $g_2:\Rn^2\lra\Rn^3:(u,v)\lra(\ln(\tan^2(u)+1),uv-\pi, \cos(\cos(uv)))$
\item $h=f_1\circ g_1$
\end{enumerate}

\corrref{0051}
\end{exercice}
