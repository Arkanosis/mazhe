% This is part of Mes notes de mathématique
% Copyright (c) 2011-2013
%   Laurent Claessens
% See the file fdl-1.3.txt for copying conditions.

\thispagestyle{empty}
\begin{center}
  \begin{minipage}{15cm}
    \hrule\par
    \vspace{2mm}
    \begin{center}
    \Huge \bfseries Mes notes de mathématique \par
    \end{center}
    \hrule\par
  \end{minipage}
\end{center}

\vspace{2cm}

\begin{center}
    Laurent \textsc{Claessens}\\
    %Université libre de Bruxelles (2008-2009)\\
    %Université catholique de Louvain (2009-2010)\\
    %Université de Franche-Comté (2010-2012)\\
    \today\\
    \url{http://student.ulb.ac.be/~lclaesse/mes_notes-2014.pdf}

    \vspace{1cm}

    Une version {\bf disponible dans la bibliothèque de l'agrégation à Paris}, est à l'adresse\\
    \url{http://student.ulb.ac.be/~lclaesse/mes_notes-2012.pdf}

\end{center}

\vfill

\LogoEtLicence

% Supprimer la lise à jour automatique dans le scipt de mise en ligne.
% Supprimer le «Une version de ces notes est disponible dans la bibliothèque de l'agrégation» 
% Ajouter ici l'ISBN. Pour les révisions, mettre un nouvel ISBN et indiquer que c'est une révision.
% Pour l'ISBN:
% Copier tout dans un nouveau répertoire
% Créer une nouvelle branche git
% Coder en dur la date (càd enlever \today)
% Comme c'est pour imprimer, regarder si c'est pas mieux d'enlever l'option ``oneside'' de la classe book.
% Enlever les couleurs, en particulier les URL et les liens internes.

% http://www.bnf.fr/fr/professionnels/s_informer_obtenir_isbn/s.qu_est_ce_que_isbn.html

% Il faut écrire l'ISBN au verso de la page de titre, au bas de la dernière page de couverture et au bas de la dernière page de la jaquette des livres ;



% De temps en temps, il faut renvoyer une nouvelle version à
% http://megamaths.perso.neuf.fr/

%\clearpage

% Ceci est pour avoir l'ISBN au dos de la dernière page de couverture.
\clearpage

\thispagestyle{empty}

\hphantom{a}
\vfill

\LogoEtLicence
\clearpage
