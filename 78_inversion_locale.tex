% This is part of Mes notes de mathématique
% Copyright (c) 2011-2016
%   Laurent Claessens
% See the file fdl-1.3.txt for copying conditions.

%+++++++++++++++++++++++++++++++++++++++++++++++++++++++++++++++++++++++++++++++++++++++++++++++++++++++++++++++++++++++++++ 
\section{Complétude avec la norme uniforme}
%+++++++++++++++++++++++++++++++++++++++++++++++++++++++++++++++++++++++++++++++++++++++++++++++++++++++++++++++++++++++++++

\begin{proposition}[Limite uniforme de fonctions continues]\label{PropCZslHBx}
    Soit \( X\) un espace topologique et \( (Y,d)\) un espace métrique. Si une suite de fonctions \( f_n\colon X\to Y\) continues converge uniformément, alors la limite est séquentiellement continue\footnote{Si \( X\) est métrique, alors c'est la continuité usuelle par la proposition \ref{PropFnContParSuite}.}.
\end{proposition}

\begin{proof}
    Soit \( a\in X\) et prouvons que \( f\) est séquentiellement continue en \( a\). Pour cela nous considérons une suite \( x_n\to a\) dans \( X\). Nous savons que \( f(x_n)\stackrel{Y}{\longrightarrow}f(x)\). Pour tout \(k\in \eN\), tout \( n\in \eN\) et tout \( x\in X\) nous avons la majoration
    \begin{equation}
        \big\| f(x_n)-f(x) \big\|\leq \big\| f(x_n)-f_k(x_n) \big\|+\big\| f_k(x_n)-f_k(x) \big\|+\big\| f_k(x)-f(x) \big\|\leq 2\| f-f_k \|_{\infty}+\big\| f_k(x_n)-f_k(x) \big\|.    
    \end{equation}
    Soit \( \epsilon>0\). Si nous choisissons \( k\) suffisamment grand la premier terme est plus petit que \( \epsilon\). Et par continuité de \( f_k\), en prenant \( n\) assez grand, le dernier terme est également plus petit que \( \epsilon\).
\end{proof}

\begin{proposition} \label{PropSYMEZGU}
    Soit \( X\) un espace topologique métrique \( (Y,d)\) un espace espace métrique complet. Alors l'espace des fonctions continues et bornées \( X\to Y\) muni de la norme uniforme \( \big( C^0_b(X,Y),\| . \|_{\infty} \big)\) est complet.
\end{proposition}
\index{espace!complet!\( C^0_b(X,Y)\),norme uniforme}

\begin{proof}
    Soit \( (f_n)\) une suite de Cauchy dans \( C(X,Y)\), c'est à dire que pour tout \( \epsilon>0\) il existe \( N\in \eN\) tel que si \( k,l>N\) nous avons \( \| f_k-f_l \|_{\infty}\leq \epsilon\). Cette suite vérifie le critère de Cauchy uniforme \ref{PropNTEynwq} et donc converge uniformément vers une fonction \( f\colon X\to Y\). La continuité de la fonction \( f\) découle de la convergence uniforme et de la proposition \ref{PropCZslHBx} (c'est pour avoir l'équivalence entre la continuité séquentielle et la continuité normale que nous avons pris l'hypothèse d'espace métrique).
\end{proof}
    Notons que si \( X\) est compact, les fonctions continues sont bornées par le théorème \ref{ThoImCompCotComp} et nous pouvons simplement dire que \( C^0(X,Y)\) est complet, sans préciser que nous parlons des fonctions bornées.


\begin{lemma}       \label{LemdLKKnd}
    Soient \( A\) et \( B\) deux espaces compact. L'ensemble des fonctions continues de \( A\) vers \( B\) muni de la norme uniforme est complet.
\end{lemma}
\index{espace!complet!\(\big( C(A,B),\| . \|_{\infty} \big)\)}
% TODO : revoir cette preuve à la lumière du critère de Cauchy uniforme \ref{PropNTEynwq}.

\begin{proof}
    Soit \( (f_k)\) une suite de Cauchy de fonctions dans \( C(A,B)\). Pour chaque \( x\in A \) nous avons
    \begin{equation}
        \| f_k(x)-f_l(x) \|_B\leq \| f_k-f_l \|_{\infty},
    \end{equation}
    de telle sorte que la suite \( (f_k(x))\) est de Cauchy dans \( B\) et converge donc vers un élément de \( B\). La suite de Cauchy \( (f_k)\) converge donc ponctuellement vers une fonction \( f\colon A\to B\). Nous devons encore voir que cette fonction est continue; ce sera l'uniformité de la norme qui donnera la continuité. En effet soit \( x_n\to x\) une suite dans \( A\) convergent vers \( x\in A\). Pour chaque \( k\in \eN\) nous avons
    \begin{equation}
        \| f(x_n)-f(x) \|\leq \| f(x_n)-f_k(x_n) \|  +\| f_k(x_n)-f_k(x) \|+\| f_k(x)-f(x) \|.
    \end{equation}
    En prenant \( k\) et \( n\) assez grands, cette expression peut être rendue aussi petite que l'on veut; le premier et le troisième terme par convergence ponctuelle \( f_k\to f\), le second terme par continuité de \( f_k\). La suite \( f(x_n)\) est donc convergente vers \( f(x)\) et la fonction \( f\) est continue.
\end{proof}

%+++++++++++++++++++++++++++++++++++++++++++++++++++++++++++++++++++++++++++++++++++++++++++++++++++++++++++++++++++++++++++ 
\section{Théorèmes de point fixe}
%+++++++++++++++++++++++++++++++++++++++++++++++++++++++++++++++++++++++++++++++++++++++++++++++++++++++++++++++++++++++++++

\begin{InternalLinks}
    Nous allons voir les résultats suivants.
    \begin{description}
        \item[Théorème de Picard] \ref{ThoEPVkCL} donne un point fixe comme limite d'itéré d'une fonction Lipschitz. Il aura pour conséquence le théorème de Cauchy-Lipschitz \ref{ThokUUlgU}, l'équation de Fredholm, théorème \ref{ThoagJPZJ} et le théorème d'inversion locale dans le cas des espaces de Banach \ref{ThoXWpzqCn}.
    \item[Théorème de Brouwer] qui donne un point fixe pour une application d'une boule vers elle-même. Nous allons donner plusieurs versions et preuves.
            \begin{enumerate}
                \item
                    Dans \( \eR^n\) en version \( C^{\infty}\) via le théorème de Stokes, proposition \ref{PropDRpYwv}.
                \item
                    Dans \( \eR^n\) en version continue, en s'appuyant sur le cas \( C^{\infty}\) et en faisant un passage à la limite, théorème \ref{ThoRGjGdO}.
                \item
                    Dans \( \eR^2\) via l'homotopie, théorème \ref{ThoLVViheK}. Oui, c'est très loin. Et c'est normal parce que ça va utiliser la formule de l'indice qui est de l'analyse complexe\footnote{On aime bien parce que ça ne demande pas Stokes, mais quand même hein, c'est pas gratos non plus.}.
            \end{enumerate}
        \item[Théorème de Markov-Kakutani]\ref{ThoeJCdMP} qui donne un point fixe à une application continue d'un convexe fermé borné dans lui-même. Ce théorème donnera la mesure de Haar \ref{ThoBZBooOTxqcI} sur les groupes compacts.
        \item[Théorème de Schauder] \ref{ThovHJXIU} qui est une version valable en dimension infinie du théorème de Brouwer. Il a pour conséquence le théorème de Cauchy-Arzela \ref{ThoHNBooUipgPX} pour les équations différentielles.
    \end{description}

    Le théorème de Schauder \ref{ThovHJXIU} permet de démontrer une version du théorème de Cauchy-Lipschitz (théorème \ref{ThokUUlgU}) sans la condition Lipschitz, mais alors sans unicité de la solution. Notons que de ce point de vue nous sommes dans la même situation que la différence entre le théorème de Brouwer et celui de Picard : hors hypothèse de type «contraction», point d'unicité.
\end{InternalLinks}

%--------------------------------------------------------------------------------------------------------------------------- 
\subsection{Picard}
%---------------------------------------------------------------------------------------------------------------------------

\begin{definition}      \label{DEFooRSLCooAsWisu}
    Une application \( f\colon (X,\| . \|_X)\to (Y,\| . \|_Y)\) entre deux espaces métriques est une \defe{contraction}{contraction} si elle est \( k\)-\defe{Lipschitz}{Lipschitz} pour un certain \( 0\leq k<1\), c'est à dire si pour tout \( x,y\in X\) nous avons
    \begin{equation}
        \| f(x)-f(y) \|_Y\leq k\| x-y \|_{X}.
    \end{equation}
\end{definition}

\begin{theorem}[Picard \cite{ClemKetl,NourdinAnal}\footnote{Il me semble qu'à la page 100 de \cite{NourdinAnal}, l'hypothèse H1 qui est prouvée ne prouve pas Hn dans le cas \( n=1\). Merci de m'écrire si vous pouvez confirmer ou infirmer. La preuve donnée ici ne contient pas cette «erreur».}.]     \label{ThoEPVkCL}
    Soit \( X\) un espace métrique complet et \( f\colon X\to X\) une application contractante, de constante de Lipschitz \( k\). Alors \( f\) admet un unique point fixe, nommé \( \xi\). Ce dernier est donné par la limite de la suite définie par récurrence 
    \begin{subequations}
        \begin{numcases}{}
            x_0\in X\\
            x_{n+1}=f(x_n).
        \end{numcases}
    \end{subequations}
    De plus nous pouvons majorer l'erreur par
    \begin{equation}    \label{EqKErdim}
        \| x_n-x \|\leq \frac{ k^n }{ 1-k }\| x_n-x_{n-1} \|\leq \frac{ k^n }{ 1-k }\| x_1-x_0 \|.
    \end{equation}

    Soit \( r>0\), \( a\in X\) tels que la fonction \( f\) laisse la boule \( K=\overline{ B(a,r) }\) invariante (c'est à dire que \( f\) se restreint à \( f\colon K\to K\)). Nous considérons les suites \( (u_n)\) et \( (v_n)\) définies par
    \begin{subequations}
        \begin{numcases}{}
            u_0=v_0\in K\\
            u_{n+1}=f(v_n), v_{n+1}\in B(u_n,\epsilon).
        \end{numcases}
    \end{subequations}
    Alors le point fixe \( \xi\) de \( f\) est dans \( K\) et la suite \( (v_n)\) satisfait l'estimation
    \begin{equation}
        \| v_n-\xi \|\leq \frac{ k^n }{ 1-k }\| u_1-u_0 \|+\frac{ \epsilon }{ 1-k }.
    \end{equation}
\end{theorem}
\index{théorème!Picard}
\index{point fixe!Picard}

La première inégalité \eqref{EqKErdim} donne une estimation de l'erreur calculable en cours de processus; la seconde donne une estimation de l'erreur calculable avant de commencer.

\begin{proof}
    
    Nous commençons par l'unicité du point fixe. Si \( a\) et \( b\) sont des points fixes, alors \( f(a)=a\) et \( f(b)=b\). Par conséquent
    \begin{equation}
        \| f(a)-f(b) \|=\| a-b \|,
    \end{equation}
    ce qui contredit le fait que \( f\) soit une contraction.

    En ce qui concerne l'existence, notons que si la suite des \( x_n\) converge dans \( X\), alors la limite est un point fixe. En effet en prenant la limite des deux côtés de l'équation \( x_{n+1}=f(x_n)\), nous obtenons \( \xi=f(\xi)\), c'est à dire que \( \xi\) est un point fixe de \( f\). Notons que nous avons utilisé ici la continuité de \( f\), laquelle est une conséquence du fait qu'elle soit Lipschitz. Nous allons donc porter nos efforts à prouver que la suite est de Cauchy (et donc convergente parce que \( X\) est complet). Nous commençons par prouver que \( \| x_{n+1}-x_n \|\leq k^n\| x_0-x_1 \|\). En effet pour tout \( n\) nous avons
    \begin{equation}
        \| x_{n+1}-x_n \|=\| f(x_n)-f(x_{n-1}) \|\leq k\| x_n-x_{n-1} \|.
    \end{equation}
    La relation cherchée s'obtient alors par récurrence. Soient \( q>p\). En utilisant une somme télescopique,
    \begin{subequations}
        \begin{align}
            \| x_q-x_p \|&\leq \sum_{l=p}^{q-1}\| x_{l+1}-x_l \|\\
            &\leq\left( \sum_{l=p}^{q-1}k^l \right)\| x_1-x_0 \|\\
            &\leq\left(\sum_{l=p}^{\infty}k^l\right)\| x_1-x_0 \|.
        \end{align}
    \end{subequations}
    Étant donné que \( k<1\), la parenthèse est la queue d'une série qui converge, et donc tend vers zéro lorsque \( p\) tend vers l'infini.

    En ce qui concerne les inégalités \eqref{EqKErdim}, nous refaisons une somme télescopique :
    \begin{subequations}
        \begin{align}
            \| x_{n+p}-x_n \|&\leq \| x_{n+p}-x_{n+p-1} \|+\ldots +\| x_{n+1}-x_n \|\\
            &\leq k^p\| x_n-x_{n-1} \|+k^{p-1}\| x_n-x_{n-1} \|+\ldots +k\| x_n-x_{n-1} \|\\
            &=k(1+\ldots +k^{p-1})\| x_n-x_{n-1}\|  \\
            &\leq \frac{ k }{ 1-k }\| x_n-x_{n-1} \|.
        \end{align}
    \end{subequations}
    En prenant la limite \( p\to \infty\) nous trouvons
    \begin{equation}        \label{EqlUMVGW}
        \| \xi-x_n \|\leq \frac{ k }{ 1-k }\| x_n-x_{n-1} \|\leq \frac{ k }{ 1-k }\| x_1-x_0 \|.
    \end{equation}

    Nous passons maintenant à la seconde partie du théorème en supposant que \( f\) se restreigne en une fonction \( f\colon K\to K\). D'abord \( K\) est encore un espace métrique complet, donc la première partie du théorème s'y applique et \( f\) y a un unique point fixe.
    
    Nous allons montrer la relation par récurrence. Tout d'abord pour \( n=1\) nous avons
    \begin{equation}
        \| v_1-\xi \|\leq\| v_1-u_1 \|+\| u_1-\xi \|\leq \epsilon+\frac{ k }{ 1-k }\| u_1-u_0 \|
    \end{equation}
    où nous avons utilisé l'estimation \eqref{EqlUMVGW}, qui reste valable en remplaçant \( x_1\) par \( u_1\)\footnote{Elle n'est cependant pas spécialement valable si on remplace \( x_n\) par \( u_n\).}. Nous pouvons maintenant faire la récurrence :
    \begin{subequations}
        \begin{align}
            \| v_{n+1}-\xi \|&\leq \| v_{n+1}-u_{n+1} \|+\| u_{n+1}-\xi \|\\
            &\leq \epsilon+k\| v_n-\xi \|\\
            &\leq \epsilon+k\left( \frac{ k^n }{ 1-k }\| u_1-u_0 \|+\frac{ \epsilon }{ 1-k } \right)\\
            &=\frac{ \epsilon }{ 1-k }+\frac{ k^{n+1} }{ 1-k }\| u_1-u_0 \|.
        \end{align}
    \end{subequations}
\end{proof}

\begin{remark}
    Ce théorème comporte deux parties d'intérêts différents. La première partie est un théorème de point fixe usuel, qui sera utilisé pour prouver l'existence de certaines équations différentielles.

    La seconde partie est intéressante d'un point de vie numérique. En effet, ce qu'elle nous enseigne est que si à chaque pas de calcul de la récurrence \( x_{n+1}=f(x_n)\) nous commettons une erreur d'ordre de grandeur \( \epsilon\), alors le procédé (la suite \( (v_n)\)) ne converge plus spécialement vers le point fixe, mais tend vers le point fixe avec une erreur majorée par \( \epsilon/(k-1)\).
\end{remark}

\begin{remark}
Au final l'erreur minimale qu'on peut atteindre est de l'ordre de \( \epsilon\). Évidemment si on commet une faute de calcul de l'ordre de \( \epsilon\) à chaque pas, on ne peut pas espérer mieux.
\end{remark}

\begin{remark}  \label{remIOHUJm}
    Si \( f\) elle-même n'est pas contractante, mais si \( f^p\) est contractante pour un certain \( p\in \eN\) alors la conclusion du théorème de Picard reste valide et \( f\) a le même unique point fixe que \( f^p\). En effet nommons \( x\) le point fixe de \( f\) : \( f^p(x)=x\). Nous avons alors
    \begin{equation}
        f^p\big( f(x) \big)=f\big( f^p(x) \big)=f(x),
    \end{equation}
    ce qui prouve que \( f(x)\) est un point fixe de \( f^p\). Par unicité nous avons alors \( f(x)=x\), c'est à dire que \( x\) est également un point fixe de \( f\).
\end{remark}

Si la fonction n'est pas Lipschitz mais presque, nous avons une variante.
\begin{proposition}
    Soit \( E\) un ensemble compact\footnote{Notez cette hypothèse plus forte} et si \( f\colon E\to E\) est une fonction telle que
    \begin{equation}        \label{EqLJRVvN}
        \| f(x)-f(y) \|< \| x-y \|
    \end{equation}
    pour tout \( x\neq y\) dans \( E\) alors \( f\) possède un unique point fixe.
\end{proposition}

\begin{proof}
    La suite \( x_{n+1}=f(x_n)\) possède une sous suite convergente. La limite de cette sous suite est un point fixe de \( f\) parce que \( f\) est continue. L'unicité est due à l'aspect strict de l'inégalité \eqref{EqLJRVvN}.
\end{proof}

\begin{theorem}[Équation de Fredholm]\index{Fredholm!équation}\index{équation!Fredholm}     \label{ThoagJPZJ}
    Soit \( K\colon \mathopen[ a , b \mathclose]\times \mathopen[ a , b \mathclose]\to \eR\) et \( \varphi\colon \mathopen[ a , b \mathclose]\to \eR\), deux fonctions continues. Alors si \( \lambda\) est suffisamment petit, l'équation
    \begin{equation}
        f(x)=\lambda\int_a^bK(x,y)f(y)dy+\varphi(x)
    \end{equation}
    admet une unique solution qui sera de plus continue sur \( \mathopen[ a , b \mathclose]\).
\end{theorem}

\begin{proof}
    Nous considérons l'ensemble \( \mF\) des fonctions continues \( \mathopen[ a , b \mathclose]\to\mathopen[ a , b \mathclose]\) muni de la norme uniforme. Le lemme \ref{LemdLKKnd} implique que \( \mF\) est complet. Nous considérons l'application \( \Phi\colon \mF\to \mF\) donnée par
    \begin{equation}
        \Phi(f)(x)=\lambda\int_a^bK(x,y)f(y)dy+\varphi(x). 
    \end{equation}
    Nous montrons que \( \Phi^p\) est une application contractante pour un certain \( p\). Pour tout \( x\in \mathopen[ a , b \mathclose]\) nous avons
    \begin{subequations}
        \begin{align}
            \| \Phi(f)-\Phi(g) \|_{\infty}&\leq \| \Phi(f)(x)-\Phi(g)(x) \|\\
            &=| \lambda |\Big\| \int_a^bK(x,y)\big( f(y)-g(y) \big)dy  \Big\|\\
            &\leq | \lambda |\| K \|_{\infty}| b-a |\| f-g \|_{\infty}
        \end{align}
    \end{subequations}
    Si \( \lambda\) est assez petit, et si \( p\) est assez grand, l'application \( \Phi^p\) est donc une contraction. Elle possède donc un unique point fixe par le théorème de Picard \ref{ThoEPVkCL}.
\end{proof}

%--------------------------------------------------------------------------------------------------------------------------- 
\subsection{Brouwer}
%---------------------------------------------------------------------------------------------------------------------------
\label{subSecZCCmMnQ}

\begin{proposition}
    Soit \( f\colon \mathopen[ a , b \mathclose]\to \mathopen[ a , b \mathclose]\) une fonction continue. Alors \( f\) accepte un point fixe.
\end{proposition}

\begin{proof}
    En effet si nous considérons \( g(x)=f(x)-x\) alors nous avons \( g(a)=f(a)-a\geq 0\) et \( g(b)=f(b)-b\leq 0\). Si \( g(a)\) ou \( g(b)\) est nul, la proposition est démontrée; nous supposons donc que \( g(a)>0\) et \( g(b)<0\). La proposition découle à présent du théorème des valeurs intermédiaires \ref{ThoValInter}.
\end{proof}

\begin{example}
    La fonction \( x\mapsto\cos(x)\) est continue entre \( \mathopen[ -1 , 1 \mathclose]\) et \( \mathopen[ -1 , 1 \mathclose]\). Elle admet donc un point fixe. Par conséquent il existe (au moins) une solution à l'équation \( \cos(x)=x\).
\end{example}

\begin{proposition}[Brouwer dans \( \eR^n\) version \(  C^{\infty}\) via Stokes]     \label{PropDRpYwv}
    Soit \( B\) la boule fermée de centre \( 0\) et de rayon \( 1\) de \( \eR^n\) et \( f\colon B\to B\) une fonction \(  C^{\infty}\). Alors \( f\) admet un point fixe.
\end{proposition}
\index{point fixe!Brouwer}

\begin{proof}
    Supposons que \( f\) ne possède pas de points fixes. Alors pour tout \( x\in B\) nous considérons la ligne droite partant de \( x\) dans la direction de \( f(x)\) (cette droite existe parce que \( x\) et \( f(x)\) sont supposés distincts). Cette ligne intersecte \( \partial B\) en un point que nous appelons \( g(x)\). Prouvons que cette fonction est \( C^k\) dès que \( f\) est \( C^k\) (y compris avec \( k=\infty\)).

   Le point \( g(x) \) est la solution du système
    \begin{subequations}
        \begin{numcases}{}
        g(x)-f(x)=\lambda\big( x-f(x) \big)\\
        \| g(x) \|^2=1\\
        \lambda\geq 0.
        \end{numcases}
    \end{subequations}
    En substituant nous obtenons l'équation
    \begin{equation}
        P_x(\lambda)=\| \lambda\big( x-f(x) \big)+f(x) \|^2-1=0,
    \end{equation}
    ou encore
    \begin{equation}
        \lambda^2\| x-f(x) \|^2+2\lambda\big( x-f(x) \big)\cdot f(x)+\| f(x) \|^2-1=0.
    \end{equation}
    En tenant compte du fait que \( \| f(x)<1 \|\) (pare que les images de \( f\) sont dans \( \mB\)), nous trouvons que \( P_x(0)\leq 0\) et \( P_x(1)\leq 0\). De même \( \lim_{\lambda\to\infty} P_x(\lambda)=+\infty\). Par conséquent le polynôme de second degré \( P_x\) a exactement deux racines distinctes \( \lambda_1\leq 0\) et \( \lambda_2\geq 1\). La racine que nous cherchons est la seconde. Le discriminant est strictement positif, donc pas besoin d'avoir peur de la racine dans
    \begin{equation}
        \lambda(x)=\frac{ -\big( x-f(x) \big)\cdot f(x)+\sqrt{   \Delta_x  } }{ \| x-f(x) \|^2 }
    \end{equation}
    où 
    \begin{equation}
        \Delta_x=4\Big( \big( x-f(x) \big)\cdot f(x) \Big)^2-4\| x-f(x) \|^2\big( \| f(x) \|^2-1 \big).
    \end{equation}
    Notons que la fonction \( \lambda(x)\) est \( C^k\) dès que \( f\) est \( C^k\); et en particulier elle est \( C^{\infty}\) si \( f\) l'est.

    En résumé la fonction \( g\) ainsi définie vérifie deux propriétés :
    \begin{enumerate}
        \item
            elle est \(  C^{\infty}\);
        \item
            elle est l'identité sur \( \partial B\).
    \end{enumerate}
    La suite de la preuve consiste à montrer qu'une telle rétraction sur \( B\) ne peut pas exister\footnote{Notons qu'il n'existe pas non plus de rétractions continues sur \( B\), mais pour le montrer il faut utiliser d'autres méthodes que Stokes, ou alors présenter les choses dans un autre ordre.}.

    Nous considérons une forme de volume \( \omega\) sur \( \partial B\) : l'intégrale de \( \omega\) sur \( \partial B\) est la surface de \( \partial B\) qui est non nulle. Nous avons alors
    \begin{equation}
        0<\int_{\partial B}\omega
        =\int_{\partial B}g^*\omega
        =\int_Bd(g^*\omega)
        =\int_Bg^*(d\omega)
        =0
    \end{equation}
    Justifications :
    \begin{itemize}
        \item 
            L'intégrale \( \int_{\partial B}\omega\) est la surface de \( \partial B\) et est donc non nulle.
        \item
            La fonction \( g\) est l'identité sur \( \partial B\). Nous avons donc \( \omega=g^*\omega\).
        \item
            Le lemme \ref{LemdwLGFG}.
        \item
            La forme \( \omega\) est de volume, par conséquent de degré maximum et \( d\omega=0\).
    \end{itemize}
\end{proof}

Un des points délicats est de se ramener au cas de fonctions \( C^{\infty}\). Pour la régularisation par convolution, voir \cite{AllardBrouwer}; pour celle utilisant le théorème de Weierstrass, voir \cite{KuttlerTopInAl}.
\begin{theorem}[Brouwer dans \( \eR^n\) version continue]\label{ThoRGjGdO}
    Soit \( B\) la boule fermée de centre \( 0\) et de rayon \( 1\) de \( \eR^n\) et \( f\colon B\to B\) une fonction continue. Alors \( f\) admet un point fixe.
\end{theorem}
\index{théorème!Brouwer}

\begin{proof}
    Nous commençons par définir une suite de fonctions
    \begin{equation}
        f_k(x)=\frac{ f(x) }{ 1+\frac{1}{ k } }.
    \end{equation}
    Nous avons \( \| f_k-f \|_{\infty}\leq \frac{1}{ 1+k }\) où la norme est la norme uniforme sur \( B\). Par le théorème de Weierstrass \ref{ThoWmAzSMF} il existe une suite de fonctions \(  C^{\infty}\) \( g_k\) telles que
    \begin{equation}
        \|  g_k-f_k\|_{\infty}\leq\frac{1}{ 1+k }.
    \end{equation}
    Vérifions que cette fonction \( g_k\) soit bien une fonction qui prend ses valeurs dans \( B\) :
    \begin{subequations}
        \begin{align}
            \| g_k(x) \|&\leq \| g_k(x)-f_k(x) \|+\| f_k(x) \|\\
            &\leq \frac{1}{ 1+k }+\frac{ \| f(x) \| }{ 1+\frac{1}{ k } }\\
            &\leq \frac{1}{ 1+k}+\frac{1}{ 1+\frac{1}{ k } }\\
            &=1.
        \end{align}
    \end{subequations}
    Par la version \(  C^{\infty}\) du théorème (proposition \ref{PropDRpYwv}), \( g_k\) admet un point fixe que l'on nomme \( x_k\).

    Étant donné que \( x_k\) est dans le compact \( B\), quitte à prendre une sous suite nous supposons que la suite \( (x_k)\) converge vers un élément \( x\in B\). Nous montrons maintenant que \( x\) est un point fixe de \( f\) :
    \begin{subequations}
        \begin{align}
            \| f(x)-x \|&=\| f(x)-g_k(x)+g_k(x)-x_k+x_k-x \|\\
            &\leq \| f(x)-g_k(x) \| +\underbrace{\| g_k(x)-x_k \|}_{=0}+\| x_k-x \|\\
            &\leq \frac{1}{ 1+k }+\| x_k-x \|.
        \end{align}
    \end{subequations}
    En prenant le limite \( k\to\infty\) le membre de droite tend vers zéro et nous obtenons \( f(x)=x\).
\end{proof}

%---------------------------------------------------------------------------------------------------------------------------
\subsection{Théorème de Schauder}
%---------------------------------------------------------------------------------------------------------------------------

Une conséquence du théorème de Brouwer est le théorème de Schauder qui est valide en dimension infinie.

\begin{theorem}[Théorème de Schauder\cite{LeDretSc}]\index{théorème!Schauder}       \label{ThovHJXIU}
    Soit \( E\), un espace vectoriel normé, \( K\) un convexe compact de \( E\) et \( f\colon K\to K\) une fonction continue. Alors \( f\) admet un point fixe.
\end{theorem}
\index{théorème!Schauder}
\index{point fixe!Schauder}

\begin{proof}
    Étant donné que \( f\colon K\to K\) est continue, elle y est uniformément continue. Si nous choisissons \( \epsilon\) alors il existe \( \delta>0\) tel que 
    \begin{equation}
        \| f(x)-f(y) \|\leq \epsilon
    \end{equation}
    dès que \( \| x-y \|\leq \delta\). La compacité de \( K\) permet de choisir un recouvrement fini par des ouverts de la forme
    \begin{equation}    \label{EqKNPUVR}
        K\subset \bigcup_{1\leq i\leq p}B(x_j,\delta)
    \end{equation}
    où \( \{ x_1,\ldots, x_p \}\subset K\). Nous considérons maintenant \( L=\Span\{ f(x_j)\tq 1\leq j\leq p \}\) et
    \begin{equation}
        K^*=K\cap L.
    \end{equation}
    Le fait que \( K\) et \( L\) soient convexes implique que \( K^*\) est convexe. L'ensemble \( K^*\) est également compact parce qu'il s'agit d'une partie fermée de \( K\) qui est compact (lemme \ref{LemnAeACf}). Notons en particulier que \( K^*\) est contenu dans un espace vectoriel de dimension finie, ce qui n'est pas le cas de \( K\).

    Nous allons à présent construire une sorte de partition de l'unité subordonnée au recouvrement \eqref{EqKNPUVR} sur \( K\) (voir le lemme \ref{LemGPmRGZ}). Nous commençons par définir
    \begin{equation}
        \psi_j(x)=\begin{cases}
            0    &   \text{si \( \| x-x_j \|\geq \delta\)}\\
            1-\frac{ \| x-x_j \| }{ \delta }    &    \text{sinon}.
        \end{cases}
    \end{equation}
    pour chaque \( 1\leq j\leq p\). Notons que \( \psi_j\) est une fonction positive, nulle en-dehors de \( B(x_j,\delta)\). En particulier la fonction suivante est bien définie :
    \begin{equation}
        \varphi_j(x)=\frac{ \psi_j(x) }{ \sum_{k=1}^p\psi_k(x) }
    \end{equation}
    et nous avons \( \sum_{j=1}^p\varphi_j(x)=1\). Les fonctions \( \varphi_j\) sont continues sur \( K\) et nous définissons finalement
    \begin{equation}
        g(x)=\sum_{j=1}^p\varphi_j(x)f(x_j).
    \end{equation}
    Pour chaque \( x\in K\), l'élément \( g(x)\) est une combinaison des éléments \( f(x_j)\in K^*\). Étant donné que \( K^*\) est convexe et que la somme des coefficients \( \varphi_j(x)\) vaut un, nous avons que \( g\) prend ses valeurs dans \( K^*\) par la proposition \ref{PropPoNpPz}.

    Nous considérons seulement la restriction \( g\colon K^*\to K^*\) qui est continue sur un compact contenu dans un espace vectoriel de dimension finie. Le théorème de Brouwer nous enseigne alors que \( g\) a un point fixe (proposition \ref{ThoRGjGdO}). Nous nommons \( y\) ce point fixe. Notons que \( y\) est fonction du \( \epsilon\) choisit au début de la construction, via le \( \delta\) qui avait conditionné la partition de l'unité.

    Nous avons
    \begin{subequations}        \label{EqoXuTzE}
        \begin{align}
            f(y)-y&=f(y)-g(y)\\
            &=\sum_{j=1}^p\varphi_j(y)f(y)-\sum_{j=1}^p\varphi_j(y)f(x_j)\\
            &=\sum_{j=1}^p\varphi(j)(y)\big( f(y)-f(x_j) \big).
        \end{align}
    \end{subequations}
    Par construction, \( \varphi_j(y)\neq 0\) seulement si \( \| y-x_j \|\leq \delta\) et par conséquent seulement si \( \| f(y)-f(x_j) \|\leq \epsilon\). D'autre par nous avons \( \varphi_j(y)\geq 0\); en prenant la norme de \eqref{EqoXuTzE} nous trouvons
    \begin{equation}
        \| f(y)-y \|\leq \sum_{j=1}^p\| \varphi_j(y)\big( f(y)-f(x_j) \big) \|\leq \sum_{j=1}^p\varphi_j(y)\epsilon=\epsilon.
    \end{equation}
    Nous nous souvenons maintenant que \( y\) était fonction de \( \epsilon\). Soit \( y_m\) le \( y\) qui correspond à \( \epsilon=2^{-m}\). Nous avons alors
    \begin{equation}
        \| f(y_m)-y_m \|\leq 2^{-m}.
    \end{equation}
    L'élément \( y_m\) est dans \( K^*\) qui est compact, donc quitte à choisir une sous suite nous pouvons supposer que \( y_m\) est une suite qui converge vers \( y^*\in K\)\footnote{Notons que même dans la sous suite nous avons \( \| f(y_m)-y_m \|\leq 2^{-m}\), avec le même «\( m\)» des deux côtés de l'inégalité.}. Nous avons les majorations
    \begin{equation}
        \| f(y^*)-y^* \|\leq \| f(y^*)-f(y_m) \|+\| f(y_m)-y_m \|+\| y_m-y^* \|.
    \end{equation}
    Si \( m\) est assez grand, les trois termes du membre de droite peuvent être rendus arbitrairement petits, d'où nous concluons que
    \begin{equation}
        f(y^*)=y^*
    \end{equation}
    et donc que \( f\) possède un point fixe.
\end{proof}


%--------------------------------------------------------------------------------------------------------------------------- 
\subsection{Théorème de Markov-Kakutani et mesure de Haar}
%---------------------------------------------------------------------------------------------------------------------------

\begin{definition}
    Soit \( G\) un groupe topologique. Une \defe{mesure de Haar}{mesure!de Haar} sur \( G\) est une mesure \( \mu\) telle que 
    \begin{enumerate}
        \item
            \( \mu(gA)=\mu(A)\) pour tout mesurable \( A\) et tout \( g\in G\),
        \item
            \( \mu(K)<\infty\) pour tout compact \( K\subset G\).
    \end{enumerate}
    Si de plus le groupe \( G\) lui-même est compact nous demandons que la mesure soit normalisée : \( \mu(G)=1\).
\end{definition}

Le théorème suivant nous donne l'existence d'une mesure de Haar sur un groupe compact.
\begin{theorem}[Markov-Katutani\cite{BeaakPtFix}]\index{théorème!Markov-Takutani}   \label{ThoeJCdMP}
    Soit \( E\) un espace vectoriel normé et \( L\), une partie non vide, convexe, fermée et bornée de \( E'\). Soit \( T\colon L\to L\) une application continue. Alors \( T\) a un point fixe.
\end{theorem}

\begin{proof}
    Nous considérons un point \( x_0\in L\) et la suite
    \begin{equation}
        x_n=\frac{1}{ n+1 }\sum_{i=0}^n T^ix_0.
    \end{equation}
    La somme des coefficients devant les \( T^i(x_0)\) étant \( 1\), la convexité de \( L\) montre que \( x_n\in L\). Nous considérons l'ensemble
    \begin{equation}
        C=\bigcap_{n\in \eN}\overline{ \{ x_m\tq m\geq n \} }.
    \end{equation}
    Le lemme \ref{LemooynkH} indique que \( C\) n'est pas vide, et de plus il existe une sous suite de \( (x_n)\) qui converge vers un élément \( x\in C\). Nous avons
    \begin{equation}
        \lim_{n\to \infty} x_{\sigma(n)}(v)=x(v)
    \end{equation}
    pour tout \( v\in E\). Montrons que \( x\) est un point fixe de \( T\). Nous avons
    \begin{subequations}
        \begin{align}
            \| (Tx_{\sigma(k)}-x_{\sigma(k)})v \|&=\Big\| T\frac{1}{ 1+\sigma(k) }\sum_{i=0}^{\sigma(k)}T^ix_0(v)-\frac{1}{ 1+\sigma(k) }\sum_{i=0}^{\sigma(k)}T^ix_0(v) \Big\|\\
            &=\Big\| \frac{1}{ 1+\sigma(k) }\sum_{i=0}^{\sigma(k)}T^{i+1}x_0(v)-T^ix_0(v) \Big\|\\
            &=\frac{1}{ 1+\sigma(k) }\big\| T^{\sigma(k)+1}x_0(v)-x_0(v) \big\|\\
            &\leq\frac{ 2M }{ \sigma(k)+1 }
        \end{align}
    \end{subequations}
    où \( M=\sum_{y\in L}\| y(v) \|<\infty\) parce que \( L\) est borné. En prenant \( k\to\infty\) nous trouvons
    \begin{equation}
        \lim_{k\to \infty} \big( Tx_{\sigma(k)}-x_{\sigma(k)} \big)v=0,
    \end{equation}
    ce qui signifie que \( Tx=x\) parce que \( T\) est continue.
\end{proof}

Le théorème suivant est une conséquence du théorème de Markov-Katutani.
\begin{theorem} \label{ThoBZBooOTxqcI}
    Si \( G\) est un groupe topologique compact possédant une base dénombrable de topologie alors \( G\) accepte une unique mesure de Haar normalisée. De plus elle est unimodulaire :
    \begin{equation}
        \mu(Ag)=\mu(gA)=\mu(A)
    \end{equation}
    pour tout mesurables \( A\subset G\) et tout élément \( g\in G\).
\end{theorem}
\index{mesure!de Haar}

%+++++++++++++++++++++++++++++++++++++++++++++++++++++++++++++++++++++++++++++++++++++++++++++++++++++++++++++++++++++++++++
                    \section{Théorèmes d'inversion locale et de la fonction implicite}
%+++++++++++++++++++++++++++++++++++++++++++++++++++++++++++++++++++++++++++++++++++++++++++++++++++++++++++++++++++++++++++

%---------------------------------------------------------------------------------------------------------------------------
\subsection{Mise en situation}
%---------------------------------------------------------------------------------------------------------------------------

Dans un certain nombre de situation, il n'est pas possible de trouver des solutions explicites aux équations qui apparaissent. Néanmoins, l'existence «théorique» d'une telle solution est souvent déjà suffisante. C'est l'objet du théorème de la fonction implicite.

Prenons par exemple la fonction sur $\eR^2$ donnée par 
\begin{equation}
    F(x,y)=x^2+y^2-1.
\end{equation}
Nous pouvons bien entendu regarder l'ensemble des points donnés par $F(x,y)=0$. C'est le cercle dessiné à la figure \ref{LabelFigCercleImplicite}.
\newcommand{\CaptionFigCercleImplicite}{Un cercle pour montrer l'intérêt de la fonction implicite. Si on donne \( x\), nous ne pouvons pas savoir si nous parlons de \( P\) ou de \( P'\).}
\input{Fig_CercleImplicite.pstricks}

%\ref{LabelFigCercleImplicite}.
%\newcommand{\CaptionFigCercleImplicite}{Un cercle pour montrer l'intérêt de la fonction implicite.}
%\input{Fig_CercleImplicite.pstricks}

Nous ne pouvons pas donner le cercle sous la forme $y=y(x)$ à cause du $\pm$ qui arrive quand on prend la racine carrée. Mais si on se donne le point $P$, nous pouvons dire que \emph{autour de $P$}, le cercle est la fonction
\begin{equation}
    y(x)=\sqrt{1-x^2}.
\end{equation}
Tandis que autour du point $P'$, le cercle est la fonction
\begin{equation}
    y(x)=-\sqrt{1-x^2}.
\end{equation}
Autour de ces deux point, donc, le cercle est donné par une fonction. Il n'est par contre pas possible de donner le cercle autour du point $Q$ sous la forme d'une fonction.

Ce que nous voulons faire, en général, est de voir si l'ensemble des points tels que
\begin{equation}
    F(x_1,\ldots,x_n,y)=0
\end{equation}
peut être donné par une fonction $y=y(x_1,\ldots,x_n)$. En d'autre termes, est-ce qu'il existe une fonction $y(x_1,\ldots,x_n)$ telle que
\begin{equation}
    F\big( x_1,\ldots,x_n,y(x_1,\ldots,x_n)\big)=0.
\end{equation}

Plus généralement, soit une fonction
\begin{equation}
    \begin{aligned}
        F\colon D\subset \eR^n\times \eR^m&\to \eR^m \\
        (x,y)&\mapsto \big( F_1(x,y),\ldots, F_m(x,y) \big) 
    \end{aligned}
\end{equation}
avec $x = (x_1,\ldots, x_n)$ et $y = (y_1,\ldots,y_m)$. Pour chaque $x$ fixé, on s'intéresse aux solutions du système de $m$ équations $F(x,y) = 0$ pour les inconnues $y$ ; en particulier, on voudrait pouvoir écrire $y = \varphi(x)$ vérifiant $F(x,\varphi(x)) = 0$.

%--------------------------------------------------------------------------------------------------------------------------- 
\subsection{Théorème d'inversion locale}
%---------------------------------------------------------------------------------------------------------------------------

\begin{lemma}[\cite{ZCKMFRg}] \label{LemGZoqknC}
    Soit \( E\) un espace de Banach (métrique complet) et \( \mO\) un ouvert de \( E\). Nous considérons une \( \lambda\)-contraction \( \varphi\colon \mO\to E\). Alors l'application
    \begin{equation}
    f\colon x\mapsto x+\varphi(x)
    \end{equation}
    est un homéomorphisme entre \( \mO\) et un ouvert de \( E\). De plus \( f^{-1}\) est Lipschitz de constante plus petite ou égale à \( (1-\lambda)^{-1}\).
\end{lemma}
Cette proposition utilise le théorème de point fixe de Picard \ref{ThoEPVkCL},
et sera utilisée pour démontrer le théorème d'inversion locale \ref{ThoXWpzqCn}.
% note que garder deux lignes ici est important pour vérifier les références vers le futur : la seconde ligne peut être ignorée, pas la seconde.

\begin{proof}
        Soient \( x_1,x_2\in\mO\). Nous posons \( y_1=f(x_1)\) et \( y_2=f(x_2)\). En vertu de l'inégalité de la proposition \ref{PropNmNNm} nous avons
        \begin{subequations}    \label{subEqEBJsBfz}
            \begin{align}
            \big\| f(x_2)-f(x_1) \big\|&=\big\| x_2+\varphi(x_2)-x_1-\varphi(x_1) \big\|\\
        &\geq \Big|        \| x_2-x_1 \|-\big\| \varphi(x_2)-\varphi(x_1) \big\|  \Big|\\
    &\geq   (1-\lambda)\| x_2-x_1 \|.
            \end{align}
        \end{subequations}
        À la dernière ligne les valeurs absolues sont enlevées parce que nous savons que ce qui est à l'intérieur est positif. Cela nous dit d'abord que \( f\) est injective parce que \( f(x_2)=f(x_1)\) implique \( x_2=x_1\). Donc \( f\) est inversible sur son image. Nous posons \( A=f(\mO)\) et nous devons prouver que que \( f^{-1}\colon A\to \mO\) est continue, Lipschitz de constante majorée par \( (1-\lambda)^{-1}\) et que \( A\) est ouvert.

    Les inéquations \eqref{subEqEBJsBfz} nous disent que
    \begin{equation}
    \big\| f^{-1}(y_1)-f^{-1}(y_2) \big\|\leq \frac{ \| y_1-y_2 \| }{ 1-\lambda },
    \end{equation}
    c'est à dire que
    \begin{equation}
        f^{-1}\big( B(y,r) \big)\subset B\big( f^{-1}(y),\frac{ r }{ 1-\lambda } \big),
    \end{equation}
    ce qui signifie que \( f^{-1}\) est Lipschitz de constante souhaitée et donc continue.

    Il reste à prouver que \( f(\mO)\) est ouvert. Pour cela nous prenons \( y_0=f(x_0)\) dans \( f(\mO)\) est nous prouvons qu'il existe \( \epsilon\) tel que \( B(y_0,\epsilon)\) soit dans \( f(\mO)\). Il faut donc que pour tout \( y\in B(y_0,\epsilon)\), l'équation \( f(x)=y\) ait une solution. Nous considérons l'application
    \begin{equation}
        L_y\colon x\mapsto y-\varphi(x).
    \end{equation}
    Ce que nous cherchons est un point fixe de \( L_y\) parce que si \( L_y(x)=x\) alors \( y=x+\varphi(x)=f(x)\). Vu que
    \begin{equation}
        \big\| L_y(x)-L_y(x') \big\|=\big\| \varphi(x)-\varphi(x') \big\|\leq\lambda\| x-x' \|,
    \end{equation}
    l'application \( L_y\) est une contraction de constante \( \lambda\). Par ailleurs \( x_0\) est un point fixe de \( L_{y_0}\), donc en vertu de la caractérisation \eqref{EqDZvtUbn} des fonctions Lipschitziennes, 
    \begin{equation}
        L_{y_0}\big( \overline{ B(x_0,\delta) } \big)\subset \overline{ B\big( L_{y_0}(x_0),\lambda\delta \big) }=\overline{ B(x_0,\lambda\delta) }.
    \end{equation}
    Vu que pour tout \( y\) et \( x\) nous avons \( L_y(x)=L_{y_0}(x)+y-y_0\),
    \begin{equation}
    L_y\big( \overline{ B(x_0,\delta) } \big)=L_{y_0}\big( \overline{ B(x_0,\delta) } \big)+(y-y_0)\subset \overline{ B(x_0,\lambda\delta) }+(y-y_0)\subset \overline{ B(x_0),\lambda\delta+\| y-y_0 \| }.
    \end{equation}
    Si \( \epsilon<(1-\lambda)\delta\) alors \( \lambda\delta+\| y-y_0 \|<\delta\). Un tel choix de \( \epsilon>0\) est possible parce que \( \lambda<1\). Pour une telle valeur de \( \epsilon\) nous avons
    \begin{equation}
        L_y\big( \overline{ B(x_0,\delta) } \big)\subset \overline{ B(x_0,\delta) }.
    \end{equation}
    Par conséquent \( L_y\) est une contraction sur l'espace métrique complet \( \overline{ B(x_0,\delta) }\), ce qui signifie que \( L_y\) y possède un point fixe par le théorème de Picard \ref{ThoEPVkCL}.
\end{proof}

Le théorème d'inversion locale s'énonce de la façon suivante dans \( \eR^n\) :
\begin{theorem}[Inversion locale dans \( \eR^n\)] % Ne pas mettre de label ici parce qu'il faut référencer l'autre, celui dans Banach.
    Soit \( f\in C^k(\eR^n,\eR^n)\) et \( x_0\in \eR^n\). Si \( df_{x_0}\) est inversible, alors il existe un voisinage ouvert \( U\) de \( x_0\) et \( V\) de \( f(x_0)\) tels que \( f\colon U\to V\) soit un \( C^k\)-difféomorphisme. (c'est à dire que \( f^{-1}\) est également de classe \( C^k\))
\end{theorem}

Nous allons le démontrer dans le cas un peu plus général (mais pas plus cher\footnote{Sauf la justification de la régularité de l'application \( A\mapsto A^{-1}\)}) des espaces de Banach en tant que conséquence du théorème de point fixe de Picard \ref{ThoEPVkCL}.

\begin{theorem}[Inversion locale dans un espace de Banach\cite{ZCKMFRg,OWTzoEK}] \label{ThoXWpzqCn}
    Soit une fonction \( f\in C^p(E,F)\) avec \( p\geq 1\) entre deux espaces de Banach. Soit \( x_0\in E\) tel que \( df_{x_0}\) soit une bijection bicontinue\footnote{En dimension finie, une application linéaire est toujours continue et d'inverse continu.}. Alors il existe un voisinage ouvert \( V\) de \( x_0\) et \( W\) de \( f(x_0)\) tels que
    \begin{enumerate}
        \item
        \( f\colon V\to W\) soit une bijection,
    \item
        \( f^{-1}\colon W\to V\) soit de classe \( C^p\).
    \end{enumerate}
\end{theorem}
\index{application!différentiable}
\index{théorème!inversion locale}

\begin{proof}
    Nous commençons par simplifier un peu le problème. Pour cela, nous considérons la translation \( T\colon x\mapsto x+x_0 \) et l'application linéaire
    \begin{equation}
        \begin{aligned}
            L\colon \eR^n&\to \eR^n \\
            x&\mapsto (df_{x_0})^{-1}x
        \end{aligned}
    \end{equation}
    qui sont tout deux des difféomorphismes (\( L\) en est un par hypothèse d'inversibilité). Quitte à travailler avec la fonction \( k=L\circ f\circ T\), nous pouvons supposer que \( x_0=0\) et que \( df_{x_0}=\mtu\). Pour comprendre cela il faut utiliser deux fois la formule de différentielle de fonction composée de la proposition \ref{EqDiffCompose} :
    \begin{equation}
        dk_0(u)=dL_{(f\circ T)(0)}\Big( df_{T(0)}dT_0(u) \Big).
    \end{equation}
    Vu que \( L\) est linéaire, sa différentielle est elle-même, c'est à dire \( dL_{(f\circ T)(0)}=(df_{x_0})^{-1}\), et par ailleurs \( dT_0=\mtu\), donc
    \begin{equation}
        dk_0(u)=(df_{x_0})^{-1}\Big( df_{x_0}(u) \Big)=u,
    \end{equation}
    ce qui signifie bien que \( dk_0=\mtu\). Pour tout cela nous avons utilisé en plein le fait que \( df_{x_0}\) était inversible.

Nous posons \( g=f-\mtu\), c'est à dire \( g(x)=f(x)-x\), qui a la propriété \( dg_0=0\). Étant donné que \( g\) est de classe \( C^1\), l'application\footnote{Ici \( \GL(F)\) est l'ensemble des applications linéaires, inversibles et continues de \( F\) dans lui-même. Ce ne sont pas spécialement des matrices parce que nous n'avons pas d'hypothèses sur la dimension de \( F\), finie ou non.}
    \begin{equation}
        \begin{aligned}
            dg\colon E&\to \GL(F) \\
            x&\mapsto dg_x 
        \end{aligned}
    \end{equation}
    est continue. En conséquence de quoi nous avons un voisinage \( U'\) de \( 0 \) pour lequel
    \begin{equation}    \label{EqSGTOfvx}
        \sup_{x\in U'}\| dg_x \|<\frac{ 1 }{2}.
    \end{equation}
    Maintenant le théorème des accroissements finis \ref{ThoNAKKght} (\ref{val_medio_2} pour la dimension finie) nous indique que pour tout \( x,x'\in U'\) nous avons\footnote{Ici nous supposons avoir choisi \( U'\) convexe afin que tous les \( a\in \mathopen[ x , x' \mathclose]\) soient bien dans \( U'\) et donc soumis à l'inéquation \eqref{EqSGTOfvx}, ce qui est toujours possible, il suffit de prendre une boule.}
    \begin{equation}
        \| g(x')-g(x) \|\leq \sup_{a\in\mathopen[ x , x' \mathclose]}\| dg_a \| \cdot \| x-x' \|\leq \frac{ 1 }{2}\| x-x' \|,
    \end{equation}
    ce qui prouve que \( g\) est une contraction au moins sur l'ouvert \( U'\). Nous allons aussi donner une idée de la façon dont \( f\) fonctionne : si \( x_1,x_2\in U'\) alors
    \begin{subequations}
        \begin{align}
            \| x_1-x_2 \|&=\| g(x_1)-f(x_1)-g(x_2)+f(x_2) \| \\
            &\leq \| g(x_1)-g(x_2) \|+\| f(x_1)-f(x_2) \|\\
            &\leq \frac{ 1 }{2}\| x_1-x_2 \|+\| f(x_1)-f(x_2) \|,
        \end{align}
    \end{subequations}
    ce qui montre que
    \begin{equation}
        \| x_1-x_2 \|\leq 2\| f(x_1)-f(x_2) \|.
    \end{equation}
    Maintenant que nous savons que \( g\) est contractante de constante \( \frac{ 1 }{2}\) et que \( f=g+\mtu\) nous pouvons utiliser la proposition \ref{LemGZoqknC} pour conclure que \( f\) est un homéomorphisme sur un ouvert \( U\) (partie de \( U'\)) de \( E\) et \( f^{-1}\) a une constante de Lipschitz plus petite ou égale à \( (1-\frac{ 1 }{2})^{-1}=2\).

    Nous allons maintenant prouver que \( f^{-1}\) est différentiable et que sa différentielle est donnée par \( (df^{-1})_{f(x)}=(df_x)^{-1}\).

    Soient \( a,b\in U\) et \( u=b-a\). Étant donné que \( f\) est différentiable en \( a\), il existe une fonction \( \alpha\in o(\| u \|)\) telle que
    \begin{equation}
        f(b)-f(a)-df_a(u)=\alpha(u).
    \end{equation}
    En notant \( y_a=f(a)\) et \( y_b=f(b)\) et en appliquant \( (df_a)^{-1}\) à cette dernière équation,
    \begin{equation}
        (df_a)^{-1}(y_b-y_a)-u=(df_a)^{-1} \big( \alpha(u) \big).
    \end{equation}
    Vu que \( df_a\) est bornée (et son inverse aussi), le membre de droite est encore une fonction \( \beta\) ayant la propriété \( \lim_{u\to 0}\beta(u)/\| u \|=0\); en réordonnant les termes,
    \begin{equation}
        b-a=(df_a)^{-1}(y_b-y_a)+\beta(u)
    \end{equation}
    et donc
    \begin{equation}
        f^{-1}(y_b)-f^{-1}(y_a)-(df_a)^{-1}(y_b-y_a)=\beta(u),
    \end{equation}
    ce qui prouve que \( f^{-1}\) est différentiable et que \( (df^{-1})_{y_a}=(df_a)^{-1}\).

    La différentielle \( df^{-1}\) est donc obtenue par la chaine
    \begin{equation}
    \xymatrix{%
        df^{-1}\colon f(U) \ar[r]^-{f^{-1}}     &   U'\ar[r]^-{df}&\GL(F)\ar[r]^-{\Inv}&\GL(F)
       }
    \end{equation}
    où l'application \( \Inv\colon \GL(F)\to \GL(F)\) est l'application \( X\mapsto X^{-1}\) qui est de classe \(  C^{\infty}\) par le théorème \ref{ThoCINVBTJ}. D'autre part, par hypothèse \( df\) est une application de classe \( C^{k-1}\)\quext{Ici il me semble que dans \cite{ZCKMFRg} il est fautivement noté \( C^k\).} et donc au minimum \( C^0\) parce que \( k\geq 1\). Enfin, l'application \( f^{-1}\colon f(U)\to U\) est continue (parce que la proposition \ref{LemGZoqknC} précise que \( f\) est un homéomorphisme). Donc toute la chaine est continue et \( df^{-1}\) est continue. Cela entraine immédiatement que \( f^{-1}\) est \( C^1\) et donc que toute la chaine est \( C^1\).

    Par récurrence nous obtenons la chaine
    \begin{equation}
    \xymatrix{%
        df^{-1}\colon f(U) \ar[r]^-{f^{-1}}_-{C^{k-1}}     &   U'\ar[r]^-{df}_-{C^{k-1}}&\GL(F)\ar[r]^-{\Inv}_-{ C^{\infty}}&\GL(F)
       }
    \end{equation}
    qui prouve que \( df^{-1}\) est \( C^{k-1} \) et donc que \( f^{-1}\) est \( C^k\). La récurrence s'arrête ici parce que \( df\) n'est pas mieux que \( C^{k-1}\).
\end{proof}

\begin{normaltext}      \label{NomDJMUooTRUVkS}
    Nous allons montrer que l'application
    \begin{equation}
        \begin{aligned}
            f\colon S^{++}(n,\eR)&\to S^{++}(n,\eR) \\
            A&\mapsto \sqrt{A} 
        \end{aligned}
    \end{equation}
    est une difféomorphisme.

    Cependant \( S^{++}(n,\eR)\) n'est pas un ouvert de \( \eM(n,\eR)\) et nous ne savons pas ce qu'est la différentielle d'une application non définie sur un ouvert. Nous allons donc en réalité montrer que l'application racine carré existe sur un voisinage de chacun des points de \( S^{++}(n,\eR)\). Et comme une union quelconque d'ouverts est un ouvert, la fonction \( f\) sera bien définie sur un ouvert de \( \eM(n,\eR)\).
\end{normaltext}

\begin{lemma}       \label{LemLBFOooDdNcgy}
    L'application 
    \begin{equation}
        \begin{aligned}
            f\colon S^{++}(n,\eR)&\to S^{++}(n,\eR) \\
            A&\mapsto A^2 
        \end{aligned}
    \end{equation}
    est un \(  C^{\infty}\)-difféomorphisme.
\end{lemma}

\begin{proof}
    Prouvons d'abord que \( f\) prend ses valeurs dans \( S^{++}(n,\eR)\). Si \( A\in S^{++}(n,\eR)\) alors par la diagonalisation \ref{ThoeTMXla} elle s'écrit \( A=QDQ^{-1}\) où \( D\) est diagonale avec des nombres strictement positifs sur la diagonale. Avec cela, \( A^2=QD^2Q^{-1}\) où \( D^2\) contient encore des nombres strictement positifs sur la diagonale.

    L'application \( f\) étant essentiellement des polynôme en les entrées de \( A\), elle est de classe \( C^{\infty}\).

    Passons à l'étude de la différentielle. Comme mentionné en \ref{NomDJMUooTRUVkS} nous allons en réalité voir \( f\) sur un ouvert de \( \eM(n,\eR)\) autour de \( A\in S^{++}(n,\eR)\). Par conséquent si \( A\in S^{++}(n,\eR)\),
    \begin{subequations}
        \begin{align}
            df\colon S^{++}(n,\eR)&\to \aL\big( \eM(n,\eR),\eM(n,\eR) \big)\\
            df_A\colon \eM(n,\eR)&\to \eM(n,\eR).
        \end{align}
    \end{subequations}
    Le calcul de \( df_A\) est facile. Soit \( u\in \eM(n,\eR)\) et faisons le calcul en utilisant la formule du lemme \eqref{LemdfaSurLesPartielles} :
    \begin{subequations}
        \begin{align}
            df_A(u)&=\Dsdd{ f(A+tu) }{t}{0}\\
            &=\Dsdd{ A^2+tAu+tuA+t^2u^2 }{t}{0}\\
            &=Au+uA.
        \end{align}
    \end{subequations}
    Nous allons utiliser le théorème d'inversion locale \ref{ThoXWpzqCn} à la fonction \( f\). Dans la suite, \( A\) est une matrice de \( S^{++}(n,\eR)\).

    \begin{subproof}
        \item[\( df_A\) est injective]
            Soit \( M\in \eM(n,\eR)\) dans le noyau de \( df_A\). En posant \( M'=A^{-1}MQ\) nous avons \( M=QM'Q^{-1}\) et on applique \( df_A\) à \( QM'Q^{-1}\) :
            \begin{equation}
                df_A(QM'Q^{-1})=Q\big( DM+MD \big)Q^{-1}.
            \end{equation}
            où \( D=\begin{pmatrix}
                \lambda_1    &       &       \\
                    &   \ddots    &       \\
                    &       &   \lambda_n
                \end{pmatrix}\) avec \( \lambda_i>0\). La matrice \( D\) est inversible. Nous avons \( M'=-DM'D^{-1}\), et en coordonnées,
                \begin{subequations}
                    \begin{align}
                        M'_{ij}&=-\sum_{kl}D_{ikM'_{kl}}D^{-1}_{lj}\\
                        &=-\sum_{kl}\lambda_i\delta_{ik}M'_{kl}\frac{1}{ \lambda_j }\delta_{lj}\\
                        &=-\frac{ \lambda_i }{ \lambda_i }M'_{ij}.
                    \end{align}
                \end{subequations}
                C'est à dire que \( M'_{ij}=-\frac{ \lambda_i }{ \lambda_j }M'_{ij}\) avec \( -\frac{ \lambda_i }{ \lambda_j }<0\). Cela implique \( M'=0\) et par conséquent \( M=0\).
            \item[\( df_A\) est surjective]
                Soit \( N\in \eM(n,\eR)\); nous cherchons \( M\in \eM(n,\eR)\) tel que \( df_A(M)=N\). Nous posons \( N'=Q^{-1} NQ\) et \( M=QM'Q^{-1}\), ce qui nous donne à résoudre \( df_D(M')=N'\). Passons en coordonnées :
                \begin{subequations}
                    \begin{align}
                        (DM'+M'D)_{ij}&=\sum_k(\delta_{ik}\lambda_iM'_{kj}+M'_{ik}\delta_{kj}\lambda_j)\\&
                        &=M'_{ij}(\lambda_i+\lambda_j)
                    \end{align}
                \end{subequations}
                où \( \lambda_i+\lambda_j\neq 0\). Il suffit donc de prendre la matrice \( M'\) donnée par
                \begin{equation}
                    M'_{ij}=\frac{1}{ \lambda_i+\lambda_j }N'_{ij}
                \end{equation}
                pour que \( df_A(M')=N'\).
    \end{subproof}
    
    Le théorème d'inversion locale donne un voisinage \( V\) de $A$ dans \( \eM(n,\eR)\) et un voisinage \( W\) de \( A^2\) dans \( \eM(n,\eR)\) tels que \( f\colon V\to W\) soit une bijection  et \( f^{-1}\colon W\to V\) soit de même régularité, en l'occurrence \( C^{\infty}\).
\end{proof}

\begin{remark}
    Oui, il y a des matrices non symétriques qui ont une unique racine carré.
\end{remark}

La proposition suivante, qui dépend du le théorème d'inversion locale par le lemme \ref{LemLBFOooDdNcgy}, donne plus de régularité à la décomposition polaire donnée dans le théorème \ref{ThoLHebUAU}.
\begin{proposition}[Décomposition polaire : cas réel (suite)]       \label{PropWCXAooDuFMjn}
    L'application
    \begin{equation}
        \begin{aligned}
            f\colon \gO(n,\eR)\times S^{++}(n,\eR)&\to \GL(n,\eR) \\
            (Q,S)&\mapsto SQ 
        \end{aligned}
    \end{equation}
    est un difféomorphisme de classe \( C^{\infty}\).
\end{proposition}

\begin{proof}
    Si \( M\) est donnée dans \( \GL(n,\eR)\) alors la décomposition polaire \( M=QS\) est donnée par \( S=\sqrt{MM^t}\) et \( Q=MS^{-1}\). Autrement dit, si nous considérons la fonction de décomposition polaire
    \begin{equation}
        f\colon \gO(n,\eR)\times S^{++}(n,\eR)\to \GL(n,\eR)
    \end{equation}
    alors
    \begin{equation}
        f^{-1}(M)=\big(  M(\sqrt{MM^t})^{-1},\sqrt{MM^t}  \big).
    \end{equation}
    Nous avons vu dans le lemme \ref{LemLBFOooDdNcgy} que la racine carré était un \( C^{\infty}\)-difféomorphisme. Le reste n'étant que des produits de matrice, la régularité est de mise.
\end{proof}

%--------------------------------------------------------------------------------------------------------------------------- 
\subsection{Théorème de la fonction implicite}
%---------------------------------------------------------------------------------------------------------------------------

Nous énonçons et le démontrons le théorème de la fonction implicite dans le cas d'espaces de Banach.
\begin{theorem}[Théorème de la fonction implicite dans Banach\cite{SNPdukn}] \label{ThoAcaWho}
    Soient \( E\), \( F\) et \( G\) des espaces de Banach et des ouverts \( U\subset E\), \( V\subset F\). Nous considérons une fonction \( f\colon U\times V\to G\) de classe \( C^r\) telle que\footnote{La notation \( d_y\) est la différentielle partielle de la définition \ref{VJM_CtSKT}.}
    \begin{equation}
        d_yf_{(x_0,y_0)}\colon F\to G
    \end{equation}
    soit un isomorphisme pour un certain \( (x_0,y_0)\in U\times V\).

    Alors nous avons des voisinages \( U_0\) de \( x_0\) dans \( E\) et \( W_0\) de \( f(x_0,y_0)\) dans \( G\) et une fonction de classe \( C^r\) 
    \begin{equation}
        g\colon U_0\times W_0\to V
    \end{equation}
    telle que 
    \begin{equation}
        f\big( x,g(x,w) \big)=w
    \end{equation}
    pour tout \( (x,w)\in U_0\times W_0\).
    
    Cette fonction \( g\) est unique au sens suivant : il existe un voisinage \( V_0 \) de \( y_0\) tel que si \( (x,y)\in U_0\times V_0\) et \( w\in W_0\) satisfont à \( f(x,y)=w\) alors \( y=g(x,w)\). Autrement dit, la fonction \( g\colon U_0\times W_0\to V_0\) est unique.
\end{theorem}
\index{théorème!fonction implicite dans Banach}

\begin{proof}
    Nous commençons par considérer la fonction
    \begin{equation}
        \begin{aligned}
            \Phi\colon U\times V&\to E\times G \\
            (x,y)&\mapsto \big( x,f(x,y) \big)
        \end{aligned}
    \end{equation}
    et sa différentielle 
    \begin{subequations}
        \begin{align}
            d\Phi_{(x_0,y_0)}(u,v)&=\Dsdd{ \big( x_0+tu,f(x_0+tu,y_0+tv) \big) }{t}{0}\\
            &=\left( \Dsdd{ x_0+tu }{t}{0},\Dsdd{ f(x_0+tu,y_0+tv) }{t}{0} \right)\\
            &=\left( u,df_{(x_0,y_0)}(u,v) \right).
        \end{align}
    \end{subequations}
    Nous utilisons alors la proposition \ref{PropLDN_nHWDF} pour conclure que
    \begin{equation}
        d\Phi_{(x_0,y_0)}(u,v)=\big( u,(d_1f)_{(x_0,y_0)}(u)+(d_2f)_{(x_0,y_0)}(v) \big),
    \end{equation}
    mais comme par hypothèse \( (d_2f)_{(x_0,y_0)}\colon F\to G\) est un isomorphisme, l'application \( d\Phi_{(x_0,y_0)}\colon E\times F\to E\times G\) est également un isomorphisme. Par conséquent le théorème d'inversion locale \ref{ThoXWpzqCn} nous indique qu'il existe un voisinage \( \mO\) de \( (x_0,y_0)\) et \( \mP\) de \( \Phi(x_0,y_0)\) tels que \( \Phi\colon \mO\to \mP\) soit une bijection et \( \Phi^{-1}\colon \mP\to \mO\) soit de classe \( C^r\). Vu que \( \mP\) est un voisinage de
    \begin{equation}
        \Phi(x_0,y_0)=\big( x_0,f(x_0,y_0) \big),
    \end{equation}
    nous pouvons par \ref{PropDXR_KbaLC} le choisir un peu plus petit de telle sorte à avoir \( \mP=U_0\times W_0\) où \( U_0\) est un voisinage de \( x_0\) et \( W_0\) un voisinage de \( f(x_0,y_0)\). Dans ce cas nous devons obligatoirement aussi restreindre \( \mO\) à \( U_0\times V_0\) pour un certain voisinage \( V_0\) de \( y_0\). L'application \( \Phi^{-1}\) a obligatoirement la forme
    \begin{equation}    \label{EqMHT_QrHRn}
        \begin{aligned}
            \Phi^{-1}\colon U_0\times W_0&\to U_0\times V_0 \\
            (x,w)&\mapsto \big( x,g(x,w) \big) 
        \end{aligned}
    \end{equation}
    pour une certaine fonction \( g\colon U_0\times W_0\to V\). Cette fonction \( g\) est la fonction cherchée parce qu'en appliquant \( \Phi\) à \eqref{EqMHT_QrHRn}, 
    \begin{equation}
        (x,w)=\Phi\big( x,g(x,w) \big)=\Big( x,f\big( x,g(x,w) \big) \Big),
    \end{equation}
    qui nous dit que pour tout \( x\in U_0\) et tout \( w\in W_0\) nous avons
    \begin{equation}
        f\big( x,g(x,w) \big)=w.
    \end{equation}

    Si vous avez bien suivi le sens de l'équation \eqref{EqMHT_QrHRn} alors vous avez compris l'unicité. Sinon, considérez \( (x,y)\in U_0\times V_0\) et \( w\in W_0\) tels que \( f(x,y)=w\). Alors \( \big( x,f(x,y) \big)=(x,w)\) et 
    \begin{equation}
        \Phi(x,y)=(x,w).
    \end{equation}
    Mais vu que \( \Phi\colon U_0\times V_0\to U_0\times W_0\) est une bijection, cette relation définit de façon univoque l'élément \( (x,y)\) de \( U_0\times V_0\), qui ne sera autre que \( g(x,w)\).
\end{proof}

Le théorème de la fonction implicite s'énonce de la façon suivante pour des espaces de dimension finie.
% Attention : avant de citer ce théorème, voir si il est suffisant. Ici \varphi a une variable; dans l'autre énoncé il en a deux.
\begin{theorem}[Théorème de la fonction implicite en dimension finie]   \label{ThoRYN_jvZrZ}
    Soit une fonction \( F\colon \eR^n\times \eR^m\to \eR^m\) de classe \( C^k\) et \( (\alpha,\beta)\in \eR^n\times \eR^m\) tels que
    \begin{enumerate}
        \item
            \( F(\alpha,\beta)=0\),
        \item
            \( \frac{ \partial (F_1,\ldots, F_m) }{ \partial (y_1,\ldots, y_m) }\neq 0\), c'est à dire que \( (d_yF)_{(\alpha,\beta)} \) est inversible.
    \end{enumerate}
    Alors il existe un voisinage ouvert \( V\) de \( \alpha\) dans \( \eR^n\), un voisinage ouvert \( W\) de \( \beta\) dans \( \eR^m\) et une application \( \varphi\colon V\to W\) de classe \( C^k\)  telle que pour tout \( x\in V\) on ait
    \begin{equation}
        F\big( x,\varphi(x) \big)=0.
    \end{equation}
    De plus si \( (x,y)\in V\times W\) satisfait à \( F(x,y)=0\), alors \( y=\varphi(x)\).
\end{theorem}
\index{théorème!fonction implicite dans \( \eR^n\)}

\begin{remark}\label{RemPYA_pkTEx}
    Notons que cet énoncé est tourné un peu différemment en ce qui concerne le nombre de variables dont dépend la fonction implicite : comparez
    \begin{subequations}
        \begin{align}
            f\big( x,g(x,w) \big)=w\\
            F\big( x,\varphi(x) \big)=0.
        \end{align}
    \end{subequations}
    Le deuxième est un cas particulier du premier en posant 
    \begin{equation}
        F(x,y)=f(x,y)-f(x_0,y_0)
    \end{equation}
    et donc en considérant \( w\) comme valant la constante \( f(x_0,y_0)\); dans ce cas la fonction \( g\) ne dépend plus que de la variable \( x\).

\end{remark}

\begin{example}
    La remarque \ref{RemPYA_pkTEx} signifie entre autres que le théorème \ref{ThoAcaWho} est plus fort que \ref{ThoRYN_jvZrZ} parce que le premier permet de choisir la valeur d'arrivée. Parlons de l'exemple classique du cercle et de la fonction \( f(x,y)=x^2+y^2\). Nous savons que
    \begin{equation}
        f(\alpha,\beta)=1.
    \end{equation}
    Alors le théorème \ref{ThoAcaWho} nous donne une fonction \( g\) telle que
    \begin{equation}
        f(x,g(x,r))=r
    \end{equation}
    tant que \( x\) est proche de \( \alpha\), que \( r\) est proche de \( 1\) et que \( g\) donne des valeurs proches de \( \beta\).

    L'énoncé \ref{ThoRYN_jvZrZ} nous oblige à travailler avec la fonction \( F(x,y)=x^2+y^2-1\), de telle sorte que
    \begin{equation}
        F(\alpha,\beta)=0,
    \end{equation}
    et que nous ayons une fonction \( \varphi\) telle que
    \begin{equation}
        F(x,\varphi(x))=0.
    \end{equation}
    La fonction \( \varphi\) ne permet donc que de trouver des points sur le cercle de rayon \( 1\).
\end{example}

%---------------------------------------------------------------------------------------------------------------------------
\subsection{Exemple}
%---------------------------------------------------------------------------------------------------------------------------
    
Le théorème de la fonction implicite a pour objet de donner l'existence de la fonction $\varphi$. Maintenant nous pouvons dire beaucoup de choses sur les dérivées de $\varphi$ en considérant la fonction
\begin{equation}
    x\mapsto F\big( x,\varphi(x) \big).
\end{equation}
Par définition de $\varphi$, cette fonction est toujours nulle. En particulier, nous pouvons dériver l'équation
\begin{equation}
    F\big( x,\varphi(x) \big)=0,
\end{equation}
et nous trouvons plein de choses.


Prenons par exemple la fonction
\begin{equation}
    F\big( (x,y),z \big)=ze^z-x-y,
\end{equation}
et demandons nous ce que nous pouvons dire sur la fonction $z(x,y)$ telle que
\begin{equation}
    F\big( x,y,z(x,y) \big)=0,
\end{equation}
c'est à dire telle que
\begin{equation}        \label{EqDefZImplExemple}
    z(x,y) e^{z(x,y)}-x-y=0.
\end{equation}
pour tout $x$ et $y\in\eR$. Nous pouvons facilement trouver $z(0,0)$ parce que
\begin{equation}
    z(0,0) e^{z(0,0)}=0,
\end{equation}
donc $z(0,0)=0$.

Nous pouvons dire des choses sur les dérivées de $z(x,y)$. Voyons par exemple $(\partial_xz)(x,y)$. Pour trouver cette dérivée, nous dérivons la relation \eqref{EqDefZImplExemple} par rapport à $x$. Ce que nous trouvons est
\begin{equation}
    (\partial_xz)e^z+ze^z(\partial_xz)-1=0.
\end{equation}
Cette équation peut être résolue par rapport à $\partial_xz$~:
\begin{equation}
    \frac{ \partial z }{ \partial x }(x,y)=\frac{1}{ e^z(1+z) }.
\end{equation}
Remarquez que cette équation ne donne pas tout à fait la dérivée de $z$ en fonction de $x$ et $y$, parce que $z$ apparaît dans l'expression, alors que $z$ est justement la fonction inconnue. En général, c'est la vie, nous ne pouvons pas faire mieux.

Dans certains cas, on peut aller plus loin. Par exemple, nous pouvons calculer cette dérivée au point $(x,y)=(0,0)$ parce que $z(0,0)$ est connu :
\begin{equation}
    \frac{ \partial z }{ \partial x }(0,0)=1.
\end{equation}
Cela est pratique pour calculer, par exemple, le développement en Taylor de $z$ autour de $(0,0)$.

\begin{example}
    Est-ce que l'équation \( e^{y}+xy=0\) définit au moins localement une fonction \( y(x)\) ? Nous considérons la fonction
    \begin{equation}
        f(x,y)=\begin{pmatrix}
            x    \\ 
            e^{y}+xy    
        \end{pmatrix}
    \end{equation}
    La différentielle de cette application est
    \begin{equation}
            df_{(0,0)}(u)=\frac{ d }{ dt }\Big[ f(tu_1,tu_2) \Big]_{t=0}
            =\frac{ d }{ dt }\begin{pmatrix}
                tu_1    \\ 
                e^{tu_2}+t^2u_1u_2    
            \end{pmatrix}_{t=0}
            =\begin{pmatrix}
                u_1    \\ 
                u_2    
            \end{pmatrix}.
    \end{equation}
    L'application \( f\) définit donc un difféomorphisme local autour des points \( (x_0,y_0)\) et \( f(x_0,y_0)\). Soit \( (u,0)\) un point dans le voisinage de \( f(x_0,y_0)\). Alors il existe un unique \( (x,y)\) tel que
    \begin{equation}
        f(x,y)=\begin{pmatrix}
               x \\ 
            e^y+xy    
        \end{pmatrix}=
        \begin{pmatrix}
            u    \\ 
                0
        \end{pmatrix}.
    \end{equation}
    Nous avons automatiquement \( x=u\) et \( e^y+xy=0\). Notons toutefois que pour que ce procédé donne effectivement une fonction implicite \( y(x)\) nous devons avoir des points de la forme \( (u,0)\) dans le voisinage de \( f(x_0,y_0)\).
\end{example}

%--------------------------------------------------------------------------------------------------------------------------- 
\subsection{Théorème de Von Neumann}
%---------------------------------------------------------------------------------------------------------------------------

\begin{lemma}[\cite{KXjFWKA}]
    Soit \( G\), un sous groupe fermé de \( \GL(n,\eR)\) et 
    \begin{equation}
        \mL_G=\{ m\in \eM(n,\eR)\tq  e^{tm}\in G\,\forall t\in\eR \}.
    \end{equation}
    Alors \( \mL_G\) est un sous-espace vectoriel de \( \eM(n,\eR)\).
\end{lemma}

\begin{proof}
    Si \( m\in\mL_G\), alors \( \lambda m\in\mL_G\) par construction. Le point délicat à prouver est le fait que si \( a,b\in \mL_G\), alors \( a+b\in\mL_G\). Soit \( a\in \eM(n,\eR)\); nous savons qu'il existe une fonction \( \alpha_a\colon \eR\to \eM\) telle que
    \begin{equation}
        e^{ta}=\mtu+ta+\alpha_a(t)
    \end{equation}
    et 
    \begin{equation}
        \lim_{t\to 0} \frac{ \alpha(t) }{ t }=0.
    \end{equation}
    Si \( a\) et \( b\) sont dans \( \mL_G\), alors \(  e^{ta} e^{tb}\in G\), mais il n'est pas vrai en général que cela soit égal à \(  e^{t(a+b)}\). Pour tout \( k\in \eN\) nous avons
    \begin{equation}
        e^{a/k} e^{b/k}=\left( \mtu+\frac{ a }{ k }+\alpha_a(\frac{1}{ k }) \right)\left( \mtu+\frac{ b }{ k }+\alpha_b(\frac{1}{ k }) \right)=\mtu+\frac{ a+b }{2}+\beta\left( \frac{1}{ k } \right)
    \end{equation}
   où \( \beta\colon \eR\to \eM\) est encore une fonction vérifiant \( \beta(t)/t\to 0\). Si \( k\) est assez grand, nous avons
   \begin{equation}
       \left\| \frac{ a+b }{ k }+\beta(\frac{1}{ k })  \right\|<1,
   \end{equation}
   et nous pouvons profiter du lemme \ref{LemQZIQxaB} pour écrire alors
   \begin{equation}
       \left(  e^{a/k} e^{b/k} \right)^k= e^{k\ln\big(\mtu+\frac{ a+b }{ k }+\beta(\frac{1}{ k })\big)}.
   \end{equation}
   Ce qui se trouve dans l'exponentielle est
   \begin{equation}
       k\left[ \frac{ a+b }{ k }+\alpha( \frac{1}{ k })+\sigma\left( \frac{ a+b }{ k }+\alpha(\frac{1}{ k }) \right) \right].
   \end{equation}
   Les diverses propriétés vues montrent que le tout tend vers \( a+b\) lorsque \( k\to \infty\). Par conséquent
   \begin{equation}
       \lim_{k\to \infty} \left(  e^{a/k} e^{b/k} \right)^k= e^{a+b}.
   \end{equation}
   Ce que nous avons prouvé est que pour tout \( t\), \(  e^{t(a+b)}\) est une limite d'éléments dans \( G\) et est donc dans \( G\) parce que ce dernier est fermé.
\end{proof}

Vu que \( \mL_G\) est un sous-espace vectoriel de \( \eM(n,\eR)\), nous pouvons considérer un supplémentaire \( M\).

\begin{lemma}   \label{LemHOsbREC}
    Il n'existe pas se suites \( (m_k)\) dans \( M\setminus\{ 0 \}\) convergeant vers zéro et telle que \(  e^{m_k}\in G\) pour tout \( k\).
\end{lemma}

\begin{proof}
    Supposons que nous ayons \( m_k\to 0\) dans \( M\setminus\{ 0 \}\) avec \(  e^{m_k}\in G\). Nous considérons les éléments \( \epsilon_k=\frac{ m_k }{ \| m_k \| }\) qui sont sur la sphère unité de \(\GL(n,\eR)\). Quitte à prendre une sous-suite, nous pouvons supposer que cette suite converge, et vu que \( M\) est fermé, ce sera vers \( \epsilon\in M\) avec \( \| \epsilon \|=1\). Pour tout \( t\in \eR\) nous avons
    \begin{equation}
        e^{t\epsilon}=\lim_{k\to \infty}  e^{t\epsilon_k}.
    \end{equation}
    En vertu de la décomposition d'un réel en partie entière et décimale, pour tout \( k\) nous avons \( \lambda_k\in \eZ\) et \( | \mu_k |\leq \frac{ 1 }{2}\) tel que \( t/\| m_k \|=\lambda_k+\mu_k\). Avec ça,
    \begin{equation}
        e^{t\epsilon}=\lim_{k\to \infty}\exp\Big( \frac{ t }{ m_k }m_k \Big)=\lim_{k\to \infty}  e^{\lambda_km_k} e^{\mu_km_k}.
    \end{equation}
    Pour tout \( k\) nous avons \(  e^{\lambda_km_k}\in G\). De plus \( | \mu_k |\) étant borné et \( m_k\) tendant vers zéro nous avons \(  e^{\mu_km_k}\to 1\). Au final
    \begin{equation}
        e^{t\epsilon}=\lim_{k\to \infty}  e^{t\epsilon_k}\in G
    \end{equation}
    Cela signifie que \( \epsilon\in\mL_G\), ce qui est impossible parce que nous avions déjà dit que \( \epsilon\in M\setminus\{ 0 \}\).
\end{proof}

\begin{lemma}   \label{LemGGTtxdF}
    L'application
    \begin{equation}
        \begin{aligned}
            f\colon \mL_G\times M&\to \GL(n,\eR) \\
            l,m&\mapsto  e^{l} e^{m} 
        \end{aligned}
    \end{equation}
    est un difféomorphisme local entre un voisinage de \( (0,0)\) dans \( \eM(n,\eR)\) et un voisinage de \( \mtu\) dans \( \exp\big( \eM(n,\eR) \big)\).
\end{lemma}
Notons que nous ne disons rien de \(  e^{\eM(n,\eR)}\). Nous n'allons pas nous embarquer à discuter si ce serait tout \( \GL(n,\eR)\)\footnote{Vu les dimensions y'a tout de même peu de chance.} ou bien si ça contiendrait ne fut-ce que \( G\).

\begin{proof}
    Le fait que \( f\) prenne ses valeurs dans \( \GL(n,\eR)\) est simplement dû au fait que les exponentielles sont toujours inversibles. Nous considérons ensuite la différentielle : si \( u\in \mL_G\) et \( v\in M\) nous avons
    \begin{equation}
        df_{(0,0)}(u,v)=\Dsdd{ f\big( t(u,v) \big) }{t}{0}=\Dsdd{  e^{tu} e^{tv} }{t}{0}=u+v.
    \end{equation}
    L'application \( df_0\) est donc une bijection entre \( \mL_G\times M\) et \( \eM(n,\eR)\). Le théorème d'inversion locale \ref{ThoXWpzqCn} nous assure alors que \( f\) est une bijection entre un voisinage de \( (0,0)\) dans \( \mL_G\times M\) et son image. Mais vu que \( df_0\) est une bijection avec \( \eM(n,\eR)\), l'image en question contient un ouvert autour de \( \mtu\) dans \( \exp\big( \eM(n,\eR) \big)\).
\end{proof}

\begin{theorem}[Von Neumann\cite{KXjFWKA,ISpsBzT,GpAlgLie_Faraut,Lie_groups}]       \label{ThoOBriEoe}
    Tout sous-groupe fermé de \( \GL(n,\eR)\) est une sous-variété de \( \GL(n,\eR)\).
\end{theorem}
\index{théorème!Von Neumann}
\index{exponentielle!de matrice!utilisation}

\begin{proof}
    Soit \( G\) un tel groupe; nous devons prouver que c'est localement difféomorphe à un ouvert de \( \eR^n\). Et si on est pervers, on ne va pas faire localement difféomorphe à un ouvert de \( \eR^n\), mais à un ouvert d'un espace vectoriel de dimension finie. Nous allons être pervers.

    Étant donné que pour tout \( g\in G\), l'application 
    \begin{equation}
        \begin{aligned}
            L_g\colon G&\to G \\
            h&\mapsto gh 
        \end{aligned}
    \end{equation}
    est de classe \(  C^{\infty}\) et d'inverse \(  C^{\infty}\), il suffit de prouver le résultat pour un voisinage de \( \mtu\).

    Supposons d'abord que \( \mL_G=\{ 0 \}\). Alors \( 0\) est un point isolé de \( \ln(G)\); en effet si ce n'était pas le cas nous aurions un élément \( m_k\) de \( \ln(G)\) dans chaque boule \( B(0,r_k)\). Nous aurions alors \( m_k=\ln(a_k)\) avec \( a_k\in G\) et donc
    \begin{equation}
        e^{m_k}=a_k\in G.
    \end{equation}
    De plus \( m_k\) appartient forcément à \( M\) parce que \( \mL_G\) est réduit à zéro. Cela nous donnerait une suite \( m_k\to 0\) dans \( M\) dont l'exponentielle reste dans \( G\). Or cela est interdit par le lemme \ref{LemHOsbREC}. Donc \( 0\) est un point isolé de \( \ln(G)\). L'application \(\ln\) étant continue\footnote{Par le lemme \ref{LemQZIQxaB}.}, nous en déduisons que \( \mtu\) est isolé dans \( G\). Par le difféomorphisme \( L_g\), tous les points de \( G\) sont isolés; ce groupe est donc discret et par voie de conséquence une variété.

    Nous supposons maintenant que \( \mL_G\neq\{ 0 \}\). Nous savons par la proposition \ref{PropXFfOiOb} que 
    \begin{equation}
        \exp\colon \eM(n,\eR)\to \eM(n,\eR)
    \end{equation}
    est une application \(  C^{\infty}\) vérifiant \( d\exp_0=\id\). Nous pouvons donc utiliser le théorème d'inversion locale \ref{ThoXWpzqCn} qui nous offre donc l'existence d'un voisinage \( U\) de \( 0\) dans \( \eM(n,\eR)\) tel que \( W=\exp(U)\) soit un ouvert de \( \GL(n,\eR)\) et que \( \exp\colon U\to W\) soit un difféomorphisme de classe \(  C^{\infty}\).

    Montrons que quitte à restreindre \( U\) (et donc \( W\) qui reste par définition l'image de \( U\) par \( \exp\)), nous pouvons avoir \( \exp\big( U\cap\mL_G \big)=W\cap G\). D'abord \( \exp(\mL_G)\subset G\) par construction. Nous avons donc \( \exp\big( U\cap\mL_G \big)\subset W\cap G\). Pour trouver une restriction de \( U\) pour laquelle nous avons l'égalité, nous supposons que pour tout ouvert \( \mO\) dans \( U\), 
    \begin{equation}
        \exp\colon \mO\cap\mL_G\to \exp(\mO)\cap G
    \end{equation}
    ne soit pas surjective. Cela donnerait un élément de \( \mO\cap\complement\mL_G\) dont l'image par \( \exp\) n'est pas dans \( G\). Nous construisons ainsi une suite en considérant une boule \( B(0,\frac{1}{ k })\) inclue à \( U\) et \( x_k\in B(0,\frac{1}{ k })\cap\complement\mL_G\) vérifiant \(  e^{x_k}\in G\). Vu le choix des boules nous avons évidemment \( x_k\to 0\).

    L'élément \(  e^{x_k}\) est dans \(  e^{\eM(n,\eR)}\) et le difféomorphisme du lemme \ref{LemGGTtxdF}\quext{Il me semble que l'utilisation de ce lemme manque à l'avant-dernière ligne de la preuve chez \cite{KXjFWKA}.} nous donne \( (l_k,m_k)\in \mL_G\times M\) tel que \(  e^{l_k} e^{m_k}= e^{x_k}\). À ce point nous considérons \( k\) suffisamment grand pour que \(  e^{x_k}\) soit dans la partie de l'image de \( f\) sur lequel nous avons le difféomorphisme. Plus prosaïquement, nous posons
    \begin{equation}
        (l_k,m_k)=f^{-1}( e^{x_k})
    \end{equation}
    et nous profitons de la continuité pour permuter la limite avec \( f^{-1}\) :
    \begin{equation}
        \lim_{k\to \infty} (l_k,m_k)=f^{-1}\big( \lim_{k\to \infty}  e^{x_k} \big)=f^{-1}(\mtu)=(0,0).
    \end{equation}
    En particulier \( m_k\to 0\) alors que \(  e^{m_k}= e^{x_k} e^{-l_k}\in G\). La suite \( m_k\) viole le lemme \ref{LemHOsbREC}. Nous pouvons donc restreindre \( U\) de telle façon à avoir
    \begin{equation}
        \exp\big( U\cap\mL_G \big)=W\cap G.
    \end{equation}
    Nous avons donc un ouvert de \( \mL_G\) (l'ouvert \( U\cap\mL_G\)) qui est difféomorphe avec l'ouvert \( W\cap G\) de \( G\). Donc \( G\) est une variété et accepte \( \mL_G\) comme carte locale.

\end{proof}

\begin{remark}
    En termes savants, nous avons surtout montré que si \( G\) est un groupe de Lie d'algèbre de Lie \( \lG\), alors l'exponentielle donne un difféomorphisme local entre \( \lG\) et \( G\).
\end{remark}
