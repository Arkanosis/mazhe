% This is part of the Exercices et corrigés de CdI-2.
% Copyright (C) 2008, 2009
%   Laurent Claessens
% See the file fdl-1.3.txt for copying conditions.


\begin{corrige}{113}

Un petit graphe de la fonction est donné à la figure \ref{LabelFigexouut}
\newcommand{\CaptionFigexouut}{Le graphe de la fonction $f_n$.}
\input{Fig_exouut.pstricks}

\begin{enumerate}
\item La suite converge vers $f(x)=0$ sur $\eR$.
\item Pas de convergence uniforme sur $\eR$ parce que pour tout $n$, il existe un $x\in\eR$ tel que $f_n(x)=1$, par exemple $x=n+\frac{ 1 }{2}$.
\item Soit $K$, un compact de $\eR$ et $M$, un majorant entier de $K$. $\forall n>M$, on a $f_n(x)=0$, donc la convergence uniforme sur tout compact est triviale (pour chaque compact, il existe un moment où la suite ne bouge même plus).
\end{enumerate}

\end{corrige}
