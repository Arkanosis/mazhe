\begin{abstract}
Deformation is a main theme of research in the present work. We begin here to describe WKB quantization and a general method to guess deformations of function algebras. The role of Darboux charts and momentum maps appears clearly. A careful example is given by the deformation of $\SL(2,\eR)$.

We prove a useful result (from \cite{articleBVCS}), the extension lemma, which allows to deform a split extension when one knows a deformation of the two components of the extension. The kernel is simply the product of the two kernels.

Then we see the principle of deformation by action of group: when a Lie group is deformable, one can find a deformation of any manifold on which the group acts. Universal formulas exist in some cases. This is why deformations of groups are studied. An application of that extension lemma to the Iwasawa subgroup of $\SO(2,n)$ is given in chapter \ref{ChapNoteDev}.

\end{abstract}

\section{WKB quantization}\label{subsec:WKB}
%--------------------------------

More details can be found in the article \cite{lcBBM}. A manifold $M$ is given with its usual commutative and associative algebra $(C^{\infty}(M),\cdot)$ of smooth functions. A \defe{deformation}{deformation}, or a \emph{quantization}\index{quantization}\footnote{In fact, we make a difference between these two words. A \emph{deformation} is only the fact to find a new product from an old one; the new product depends on a parameter and has to reduce to the old one when the parameter goes to zero. A \emph{quantization} is a deformation in which the first order term (whatever it means) of the new product contains the symplectic structure as in condition \eqref{EqExigSymplePremOrd} below.}, of $M$ is the data of a new product $\star^M_{\hbar}$ on a functional space over $M$. 

Let $G$ be a Lie group acting on a manifold $M$. We consider $\Fun(M,\eC)$\nomenclature{$\Fun(M)$}{Functions on the manifold $M$}, the space of all the maps from $M$ to $\eC$, without any regularity conditions. The \defe{regular left representation}{regular!representation}\index{representation!regular left} of $G$ on $M$ is the representation of $G$ on $\Fun(M)$ given by
\begin{equation}
   [ L^*_g(a) ](h)=a(gh)
\end{equation}
for all $a\in\Fun(M)$, $g$, $h\in G$. 

\ifthenelse{\value{siTHZ}=1}{}{
A $G$-invariant WKB quantization of $M$ is a product on a space of functions $A^{M}$ on $M$ of the form
\[ 
  (u\star^M_{\hbar}v)(x)=\int_{M\times M} a_{\hbar}(x_1,x_2,x) e^{\frac{ i }{ \hbar }S(x_1,x_2,x)} u(x_1)v(x_2)\,dx_1\,dx_2
\]
for which we require, among other conditions, (see complete definition \ref{DefWKBCompl}) 
\begin{itemize}
\item $A^M\subset \Fun(M)$ is invariant under the regular left representation of $G$ and contains at least the smooth compactly supported functions,
\item the pair $(A^M,\star^M_{\hbar})$ is an associative algebra,
\item the functions $a_{\hbar}$ and $S$ are invariant under the left regular action of $G$,
\item $\forall\, x\in M$ and $\forall\,u,v\in A^M$ the product accepts an asymptotic expansion compatible with the symplectic structure in the following sense:
\[ 
  (u\ast_{\hbar} v)(x)\sim u(x)v(x)+\frac{ \hbar }{ i }c_{1}(u,v)(x)+o(\hbar^{2})
\]
where $c_{1}$ satisfies $c_{1}(u,v)-c_{1}(v,u)=2\{ u,v \}$.
\end{itemize}
The main property of this product is its $G$-invariance:
\[ 
  L_g(u\star^M_{\hbar}v)=(L_gu)\star^M_{\hbar}(L_gv).
\]
}		% Fin d'un siTHZ sur une définition raccourcie de WKB.

\subsection{Definitions and general setting}
%------------------------------------------

Let $(M,\omega,\nabla)$ be an affine symplectic manifold, i.e. a $2n$-dimensional symplectic manifold $(M,\omega)$ endowed with a torsion-free connection $\nabla$ such that $\nabla\omega=0$. The \defe{automorphism}{automorphism!of affine symplectic manifold} group $\Aut(M,\omega,\nabla)$\nomenclature[G]{$\Aut(M,\omega,\nabla)$}{Automorphism group of an affine symplectic manifold} is defined as
\[ 
  \Aut(M,\omega,\nabla)=\gpAff(\nabla)\cap\gpSymp(\omega)
\]
where $\gpAff(\nabla)$ is the group of affine transformations of the affine manifold $(M,\nabla)$ and $\gpSymp(\omega)$ is the group of symplectomorphisms of $(M,\omega)$.


\begin{probleme}
Non mais; où intervient $\nabla$ dans cette définition ? D'après Pierre, il est contenu dans les troisième ordre, mais il faudrait une référence.
\label{ProbNablades}
\end{probleme}

Let $R$ be a subgroup of $\Aut(M,\omega,\nabla)$. The following definition of a $R$-invariant WKB quantization can be found in \cite{StrictSolvableSym}.
\begin{definition}
A $R$-invariant \defe{WKB quantization}{WKB quantization} of $(M,\omega,\nabla)$ is the data of a product 
\begin{equation}	\label{EqFormeWKBdel}
(u\star_{\theta}v)(x)=\frac{1}{ \theta^{2n} }\int_{M\times M} a_{\theta}(x,y,z) e^{\frac{ i }{ \theta }S(x,y,z)}u(y)v(z)\,dy\,dz
\end{equation}
(where $dy\,dz$ is the Liouville measure $\omega^{n}/n!$) with the following constrains:
\begin{enumerate}
\item For each $\theta$, we have a space $A_{\theta}$ containing the space $ C^{\infty}_{c}(M)$ of compactly supported smooth functions. The product $\star_{\theta}$ extends to $A_{\theta}$ in such a way that $(A_{\theta},\star_{\theta})$ becomes a one-parameter family of associative $*$-algebras.
\item The product $\ast_{0}$ on $A_{0}$ is the usual pointwise product  and $(A_{0},\ast_{0})$ is a Poisson subalgebra of $ C^{\infty}(M)$ for the induced Poisson structure from the symplectic form $\omega$.
\item $\forall \theta\geq 0$, the space $A_{\theta}$ is a $*$-vector subspace of $ C^{\infty}(M)$ such that \[ 
   C^{\infty}_{c}(M)\subset A_{0}\subset A_{\theta}
\]
where the involution $*$ on $  C^{\infty}(M)$ is the usual complex conjugation.
\item $S$ is a real valued smooth function $S\colon M\times M\times M\to \eR$ such that for all $x_{0}\in M$, the function $S(x_0,.,.)\in C^{\infty}(M\times M)$ has a nondegenerate critical point at $(x_0,x_0)$.
\item The functions $a_{\theta}$ are positive real-valued: 
\[ 
  a_{\theta}\colon M\times M\times M\to \eR^{+}.
\]
\item The functions $S$ and $a_{\theta}$ are invariant under the diagonal action of $R$ on $M\times M\times M$.
\item $\forall\, x\in M$ and $\forall\,u,v\in C^{\infty}_{c}(M)$ with support in a suitably small neighbourhood of $x$, a stationary phase method yields the extension
\begin{equation}  
  (u\star_{\theta} v)(x)\sim u(x)v(x)+\frac{ \theta }{ i }c_{1}(u,v)(x)+o(\theta^{2})
\end{equation}
where $c_{1}$ satisfies
\begin{equation}		\label{EqExigSymplePremOrd}
  c_{1}(u,v)-c_{1}(v,u)=2\{ u,v \}.
\end{equation}

\end{enumerate}
\label{DefWKBCompl}
\end{definition}

We emphasize the fact that the functional space $A^M$ is stable under $\star_{\theta}$: this is a \emph{strict} quantization in contrast to a \emph{formal} star product which only stabilises the space of formal power series of $\theta$.

An example of WKB quantization is the Weyl product which is nothing but an integral reformulation of the Moyal star product:
\[ 
  (f\star^W_{\hbar}g)(x)=\frac{1}{ \hbar^{2n} }\int_{\mathbb{R}^{2n}\times\mathbb{R}^{2n}}  e^{\frac{ 2i }{ \hbar }S^0(x,y,z)}f(y)g(z)\,dy\,dz
\]
where $S^0(x,y,z)=\Omega(x,y)+\Omega(y,z)+\Omega(z,x)$, and $\Omega$ denotes the usual symplectic form on $\mathbb{R}^{2n}$.

\begin{probleme}
Ce serait bien d'avoir une référence pour cette affirmation. Et aussi de savoir si il faut un $1/\hbar^{2n}$ devant l'intégrale.
\label{ProbWeylMoy}
\end{probleme}

The function $K=a_{\theta} e^{\frac{ i }{ \theta }S}$ is the \defe{kernel}{kernel!for a WKB quantization} of the product $\star_{\theta}$. The \defe{associativity}{associativity!of a WKB quantization} of $\star_{\theta}$ on the functional space $A_{\theta}$ is the fact that the equality
\[ 
  \big( (u\star_{\theta}v)\star_{\theta}r \big)(x)=\big( u\star_{\theta}(v\star_{\theta}r) \big)(x)
\]
holds for every $u$, $v$, $r\in A_{\theta}$ and $x\in M$.  That condition translates under an integral form to the following relation
\begin{equation}\label{EqCondAssoc}
\begin{split}
&\int_{M\times M}K(x,y,z)\left[ \int_{M\times M}K(y,t,s)u(t)v(s)\mu_M(t,s) \right] r(z)\mu_M(y,z)\\
&=\int_{M\times M}K(x,y,z)u(y)\left[ \int_{M\times M}K(z,t,s)v(t)r(s)\mu_M(t,s) \right]\mu_M(y,z)
\end{split}
\end{equation}
where $\mu_M(y,z)=\mu_M(y)\mu_M(z)$ is the Liouville measure on $M$. Performing formal manipulations (such as a Fubini theorem), one can express this condition as
\begin{equation}		\label{EqAssosssens}
\int_{M}K(x,y,t)K(t,p,q)\mu(t)=\int_{M}K(x,\tau,q)K(\tau,y,p)\,\mu(\tau).
\end{equation}
That form is easier to handle and to check, but it is meaningless in general.

The fact to have a \defe{left invariant kernel}{left!invariant!kernel} on a group $G$ means that the kernel $K\colon G\times G\times G\to \eC$ has the property $L_g^*K=K$, or
 \begin{equation}
K(gh_{1},gh_2,gh_{3})=K(h_1,h_2,h_{3})
\end{equation}
for every $g\in G$.  The following lemma allows us to use group isomorphisms to push forward a kernel from a group to another.
\begin{lemma}   
Let $G_{1}$ and $G_{2}$ be two symplectic Lie groups and $K_1$, a left invariant kernel on $G_{1}$ which provides an associative product on the functional space $A_1$. Let $\phi\colon G_{2}\to G_{1}$ be a symplectic Lie group isomorphism. Then the kernel $K_2=\phi^*K_1$ is invariant and gives rise to an associative product on $A_2=\phi^*A_1$.
\label{LemKerINvarIsom}
\end{lemma}

\begin{proof}
By definition,
\[ 
(\phi^*K_1)(h_1,h_2,h_3)=K_1\big( \phi(h_1),\phi(h_2),\phi(h_3)\big).
\]
Therefore, using the left invariance of $K_1$, we have
\[ 
\begin{split}
	L_{g}^*\phi^*K_2=(\phi\circ L_{g})^*K_2=(L_{g}\circ\phi)^*K_2=\phi^*L_{\phi(g)}^*K_1=\phi^*K_1.
\end{split}  
\]
That proves left invariance of $\phi^*K_1$ on $G_{2}$.  Now we prove the associativity of $K_2$, this is to check condition  \eqref{EqCondAssoc}. We have
\[
\begin{split}
%A&=\\
 &\int_{G_2\times G_2}K_2(x,y,z)\Bigg[ \int_{G_2\times G_2}K_2(y,t,s)(\phi^*u)(t)(\phi^*v)(s)\mu_2(t,s) \Bigg]\\
&\qquad(\phi^*r)(z)\mu_2(y,z)\\
=&\int_{G_2\times G_2}K_1(\phi x,\phi y,\phi z)\Bigg[  \int_{G_2\times G_2}K_1(\phi y,\phi t,\phi s)u(\phi t)v(\phi s)\mu_2(t,s)  \Bigg]\\
&\qquad r(\phi z)\mu_2(y,z).
\end{split}
\]
We perform in this integral the change of variables $\tau_y=\phi y$, $\tau_t=\phi t$, $\tau_z=\phi z$ and $\tau_s=\phi s$. This does not affect the measure because  $\phi$ is a symplectomorphism and $\mu_i$ are the Liouville measures on $G_i$, so that for example,  $\mu_2(t)=\mu_2(\phi^{-1}\tau_t)=\mu_1(\tau_t)$. The previous integral becomes
\[ 
\begin{split}
  &\int_{G_1\times G_1}K_1(\phi x,\tau_y,\tau_z)\Bigg[  \int_{G_1\times G_1}K_1(\tau_y,\tau_t,\tau_s)u(\tau_t)v(\tau_s)\mu_1(\tau_t,\tau_s)  \Bigg]\\
&\qquad r(\tau_z)\mu_1(\tau_y,\tau_z).
\end{split}
\]
Using now the associativity of $K_1$ on $G_1$ and performing the inverse change of variables, we find
\[ 
\begin{split}
\int_{G_2\times G_2}K_2(x,y,z)(\phi^*u)(y)\Bigg[ \int_{G_2\times G_2}K_2(z,t,s)(\phi^*v)(t)(\phi^*r)(s)&\mu_2(t,s)   \Bigg]\\
					&\mu_2(y,z),
\end{split}  
\]
which proves the associativity of $K_2$ on $\phi^*A_1$.

Notice that condition \eqref{EqAssosssens} can be checked in much the same way.

\end{proof}

\ifthenelse{\value{siTHZ}=1}{}{The proposition \ref{ProperrProdInvarDiffeo} gives an improved form of this lemma. Unfortunately, this generalization revealed to be a mistake.} It is worth noticing that lemma \ref{LemKerINvarIsom} needs a group isomorphism while one often only has a Lie algebra isomorphism. Due to Campbell-Backer-Hausdorff formula, it may be very difficult to find a group isomorphism from an algebra one. \ifthenelse{\value{siTHZ}=1}{}{An example of this difficulty is in subsection  \ref{SubSecRemetreMu}.}

\begin{remark}
Most of the time, the symplectic condition \eqref{EqExigSymplePremOrd} does not have to be checked because we just define the symplectic form $\omega_2$ on $G_2$ as $\omega_2=\phi^*\omega_1$ where $\omega_1$ is the symplectic form on $G_1$.
\end{remark}

\begin{definition}
When $\alpha\colon G\times A\to A$ is an action of a Lie group $G$ on a vector space $A$, one says that the element $a\in A$ is a \defe{differentiable vector}{differentiable!vector} of $\alpha$ if the map $g\mapsto\alpha_g(a)$ is a differentiable map from $G$ into $A$.
\end{definition}

We are now interested in the regular left representation $L\colon R\times A_{\theta}\to  A_{\theta}$ defined as usual by $\big( L_r(u) \big)(x)=u(r\cdot x)$. A function $u\in A_{\theta}$ is a differentiable vector of $L$ when the map 
\begin{equation}
\begin{aligned}
 \alpha_u\colon R&\to  A_{\theta} \\ 
r&\mapsto L_r(u) 
\end{aligned}
\end{equation}
is differentiable. The differential of $\alpha_u$ is what we will denote by $dL$ in the next few pages: $dL(X)u=(d\alpha_u)_eX$. By definition, 
\[ 
  (d\alpha_u)_eX=\Dsdd{ \alpha_u( e^{tX}) }{t}{0}=\Dsdd{ L_{ e^{tX}}(u) }{t}{0},
\]
and the element $(d\alpha_u)_eX\in A_{\theta}$ applied to $x\in M$ is
\begin{equation}
 \big( dL(X)u \big)(x) =\Big( (d\alpha_u)X \Big)(x)=\Dsdd{ L_{ e^{tX}}(u)x }{t}{0}=\Dsdd{ u( e^{tX}\cdot x) }{t}{0}.
\end{equation}
We denote by $ A_{\theta}^{\infty}$ the space of differentiable vectors of the representation $L$.

If one particularises to  $ A_{\theta}\subset  C^{\infty}(R)$ (the manifold $M$ being $R$ itself), the vector fields of $R$ naturally act on $ A_{\theta}$. In particular, if $u\colon R\to \eC$ and $X\in \mR$ we have
\[ 
  \big( X^*(u) \big)(r)=X^*_r(u)=\Dsdd{ u\big(  e^{-tX}r \big) }{t}{0}=\big( dL(-X)u \big)(r), 
\]
so that
\begin{equation}
   dL(X)=-X^*
\end{equation}
holds on the space of differentiable vectors $ A_{\theta}^{\infty}$.

\begin{definition}
A formal star product $\dpt{\ast_G}{\Cinf(M)\dcr{\nu}\times \Cinf(M)\dcr{\nu}}{\Cinf(M)\dcr{\nu}}$ is said to be \defe{$\mG$-covariant}{covariant!star product} if for all $X$, $Y\in\mG$,
\begin{equation}
[\lambda_X,\lambda_Y]_{\ast_G}=2\nu\{\lambda_X,\lambda_Y\}
\end{equation}
where $[\lambda_{X},\lambda_{Y}]_{\ast_{G}}:=\lambda_X\ast_G\lambda_Y-\lambda_Y\ast_G\lambda_X$. In other words the start product is $\mG$-covariant when the expected terms of higher order in the right hand side are zero. 
\end{definition}

A crucial use of $\mG$-covariance will be done in proposition \ref{Proprhonureprez} in order to build a map $\rho_{\nu}$ that fulfils the following proposition (instead of $dL$ itself). 
\begin{proposition}
In the setting of definition \ref{DefWKBCompl}, the map $dL$ is a representation  by derivation of $\mR$  on~$ A_{\theta}^{\infty}$.
\label{prop:dL_reprez}
\end{proposition}

\begin{proof}
We will not pay attention on the domain $A_{\theta}$. Its definition will come later.  First, we prove that $\dpt{dL}{\mR}{\End{ A_{\theta}^{\infty}}}$ is a representation. Indeed,
\begin{equation}
\begin{split}
  dL([X,Y])u=\Dsddp{L^*_{\exp(-t[X,Y])}u}{t}{0}
            &=\Dsddp{ [ L^*_{\exp(-tX)},L^*_{\exp(-tY)}  ] u}{t}{0}\\
	    &=\big[dL(X),dL(Y)\big]u.
\end{split}
\end{equation}
Next, $L_R$-invariance of $\ast_{\theta}$ yields 
\[ 
  \big( L^*_{\exp -tX}u \big)\ast_{\theta}\big( L^*_{\exp -tX}v \big)=L^*_{\exp -tX}(u\ast_{\theta} v).
\]
If we derive this equality with respect to $t$ at $t=0$, we find
\[
   dL(X)u\ast_{\theta} v+u\ast_{\theta} dL(X)v=dL(X)(u\ast_{\theta} v).
\]
\end{proof}


\subsection{Deformation of Iwasawa subgroups}	%\label{Sec:mG-covariance}
%-------------------------------------------

The motivation in deforming (or quantizing) groups resides in the method of deformation by group action (appendix \ref{SecDefAction}) which states that if one can deform a group, one can write a formula for a deformed product on any manifold on which the group acts.

Let first describe the next few steps in the construction of WKB quantizations of groups. Let $G$ be a semisimple Lie group with its Iwasawa decomposition $G=ANK$. The group $R=AN$ is solvable and can be seen as the homogeneous space $R=G/K$. We consider the canonical multiplicative action $\dpt{\tau}{G\times R}{R}$ which we restrict to $\dpt{\tau}{R\times R}{R}$. We are interested in a $R$-invariant quantization of $R$. Here is a summary of the notations that will be used.

\begin{itemize}
\item $\stM$ is the Moyal star product on $\eR^n$ endowed with its canonical symplectic form,
\item $\star^R_{\theta}$ is the product we are searching for. It has to be defined at least on $ C_c^{\infty}(R)$ and should be extended to $ C^{\infty}(R)$,
\item $A^R\subset\Fun(R,\eC)$ must contain $ C^{\infty}_c(R)$. The purpose is $(A^R,\star^R_{\theta})$ to be an associative algebra and $A^R$ to be invariant under the left regular representation of $R$,
\item $\eA_{\nu}= C^{\infty}(R)[ [\nu]]$ is an intermediary space which serves to guess $\star^R_{\theta}$ and perform formal manipulations with $\rho_{\nu}$ and $dL$,
\item $\ast_M^R$ is the pull-back of Moyal to $\eA_{\nu}$. It serves to formal manipulations in order to guess the twist that defines $\ast^R_{\nu}$,
\item $\ast^R_{\nu}$ is the product on $\eA_{\nu}$. The problem of determining that product is formal. When this problem is solved, we have to prove that in a well chosen $A^R$, taking $\ast_{\nu}^R\to\star^R_{\theta}$ yields a solution to the problem. As previously noticed, in order to make sense, one has to apply $dL$ on the subspace $\eA_{\nu}^{\infty}$ of differentiable vector of the regular left representation. We will however not take care of this issue in the formal manipulations.
\end{itemize}

The main steps are the following:
{\renewcommand{\theenumi}{\arabic{enumi}.}
\begin{enumerate}
\item In the case of a WKB product we saw in proposition \ref{prop:dL_reprez} that $dL$ is a representation of $\mR$ on $\eA_{\nu}^{\infty}$. Hence we will try to build a formal product for which $dL$ is a representation by derivation. From this point of view, the manipulation with $\rho_{\nu}$ is only a trick designed to guess a product formula.

\item We suppose that the group $R$ ---the one that we are trying to quantize--- has a symplectic structure $\omega$ and we consider $\phi\colon \eR^{2n}\to R$, a Darboux chart; i.e. $\omega=\phi^*\Omega$ where $\Omega$ is the canonic symplectic form on $\eR^{2n}$.

\item We suppose that the left action of $R$ on itself is strongly hamiltonian and we denote by $\lambda_X$ the momentum maps. We suppose that the Moyal product is $\mG$-covariant\footnote{In fact, we only need the $\mR$-covariance.}.

\item We pose $\rho_{\nu}(X)=\frac{1}{ 2\nu }\ad_{\ast_M^R}(\lambda_X)$. The $\mR$-covariance of $\ast_M^R$ is used in order to prove that $\rho_{\nu}$ is a  representation by derivations of $\mR$ on $(\eA_{\nu},\ast_M^R)$. 
\item If one can find an intertwining operator between $dL$ and $\rho_{\nu}$ (i.e. if they are equivalent representations), we define $\ast_{\nu}^R$ as the pull-back of $\ast_M^R$ by this intertwining operator. In this case, we prove that $dL$ is a representation by derivations of the product $\ast_{\nu}^R$.
\end{enumerate}

}		% Fin du groupe qui fait que localement les enumerate soient en chiffres arabes au lieu de romains.
		% C'est quand même un petit hack latex que je ne sais pas si il est très propre.
\ifthenelse{\value{siTHZ}=1}{}{It is time to read appendix \ref{app:Moyal} about the Moyal star-product.}  

We try now to find a formal product $\ast_{\nu}^R$ on $\eA_{\nu}^{\infty}$ such that $dL$ is a representation by derivations. For this purpose we suppose $R$ to accept a symplectic structure $\omega$ and $\phi\colon \eR^{2n}\to R$ to be a Darboux chart, i.e. $\omega=\phi^*\Omega$ where $\Omega$ denotes the canonical symplectic form on $\eR^{2n}$. Then we bring the Moyal product of $\eR^{2n}$ to $R$ by the usual formula
\begin{equation}   
	(u\ast_M^R v)=(u\circ\phi\ast_M v\circ\phi)\circ\phi^{-1}.
\end{equation}
We suppose that product to be $\mG$-covariant\footnote{Only the $\mR$-covariance will be actually used.}:
\begin{equation}
  [\lambda_X,\lambda_Y]_{\ast_M^R}=2\nu \{ \lambda_X,\lambda_Y \}_R.
\end{equation}
 Now we consider the left action of $R$ on itself and we suppose that this is an Hamiltonian action for the symplectic structure $\omega=\phi^*\Omega$ with dual momentum maps $\dpt{\lambda_X}{R}{\eC}$. We define, for each $X\in\mR$, a linear map, $\dpt{\rho_{\nu}(X)}{\eA_{\nu}}{\eA_{\nu}}$ by
\begin{equation}
\begin{aligned}
 \rho_{\nu}\colon \mR&\to \End{\eA_{\nu}} \\ 
X&\mapsto\us{2\nu}\ad_{\ast_M^R}(\lambda_X)
\end{aligned}
\end{equation}
Notice that the formal series of $[\lambda_X,u]_{\ast_M^R}$ begins with order one, so the division by $\nu$ make sense in the space of formal series.  The main interest of $\rho_{\nu}$ is to be as we want $dL$ to be. So it will be used to guess how to twist the product in order to make $dL$ work as $\rho_{\nu}$.

\begin{proposition}
The map $\rho_{\nu}$ is a representation of $\mR$ on $\eA_{\nu}$, and $\rho_{\nu}(X)$ is a derivation of $(\eA_{\nu},\ast_M^R)$ for each $X\in\mR$.
\label{Proprhonureprez}
\end{proposition}

\begin{proof}
The proof  that $\rho_{\nu}$ is a representation is only to check that the relation $[\rho_{\nu}(X),\rho_{\nu}(Y)]f=\rho_{\nu}([X,Y])f$ holds for any $X$, $Y\in\mR$ and $f\in\eA_{\nu}$. Using the $\mG$-covariance and the Jacobi identity,
\begin{equation}
\begin{split}
  \rho_{\nu}([X,Y])f&=\us{4\nu^2}\ad_{\ast_M^R}(2\nu\lambda_{[X,Y]})f
     		=\us{4\nu^2}\ad_{\ast_M^R}([\lambda_X,\lambda_Y]_{\ast_M^R})f\\
		&=\us{4\nu^2}[[\lambda_X,\lambda_Y]_{\ast_M^R},f]_{\ast_M^R}\\
	     	&=\frac{1}{ 4\nu^2 }(\ad_{\ast_M^R}\lambda_X\circ\ad_{\ast_M^R}\lambda_Y-\ad_{\ast_M^R}\lambda_Y\circ\ad_{\ast_M^R}\lambda_X)f\\
		&=[\rho_{\nu}(X),\rho_{\nu}(Y)]f.
\end{split}
\end{equation}
It remains to check that $\rho_{\nu}(X)(u\ast_M^R v)=\rho_{\nu}(X)u\ast_M^R v+u\ast_M^R\rho_{\nu}(X)v$ for every $X\in\mR$. This is once again just a computation.
\begin{equation}
\begin{split}
   \rho_{\nu}(X)u\ast_M^R v+u\ast_M^R\rho_{\nu}(X)v&=\us{2\nu}(\lambda_X\ast_M^R u-u\ast_M^R\lambda_X)\ast_M^R v\\
                           &\quad+\us{2\nu}u\ast_M^R(\lambda_X\ast_M^R v-v\ast_M^R\lambda_X)\\
                           &=\us{2\nu}\ad_{\ast_M^R}\lambda_X(u\ast_M^R v).
\end{split}
\end{equation} 
\end{proof}
Notice that the $\mG$-covariance of $\ast_M^R$ was used to prove that $\rho_{\nu}$ is a representation.  Now, if we could show that $\rho_{\nu}=dL$, then the answer to our deformation problem would be $A_{\theta}=\eA^{\infty}_{\nu}$ and $\stt=\ast_M^R$. But instead of that we have $\rho_{\nu}=dL+o(\nu)$ because
\begin{equation}
\begin{split}
  \rho_{\nu}(X)u&=\us{2\nu}[\lambda_X,u]_{\ast_M^R}
         =\us{2\nu} 2\nu\{\lambda_X,u\}+o(\nu)
	 =X^*(u)+o(\nu)\\
	&=-dL(X)u+o(\nu)
\end{split}
\end{equation}
where the notion of fundamental field $X^*$ is taken for the regular left representation (which is Hamiltonian). That shows that $\rho_{\nu}$ is something like a deformation of $dL$. As a consequence, one has $dL(X)=X_{\lambda_X}$, or
 \begin{equation}\label{eq:dL_et_Poisson}
 dL(x)u=X_{\lambda_X}(u)=\{\lambda_X,u\}
 \end{equation}
(see subsection \ref{app:ham_act}).
 
Since $\rho_{\nu}$ is not $dL$, the hope is to see if $\rho_{\nu}$ and $dL$ should be \emph{equivalent} representations. As next proposition shows, the fact to find an equivalence between $\rho_{\nu}$ and $dL$ actually solves the problem to find a product for which $dL$ is a representation by derivation. 
\begin{proposition}
Let $\dpt{\mT}{\eA_{\nu}}{\eA_{\nu}}$ be an intertwining operator between $dL$ and $\rho_{\nu}$:
\begin{equation}\label{eq:TrnT} 
   \mT\rho_{\nu}(X) \mT^{-1}=dL(X).
\end{equation} 
If we define the star product $\ast_{\nu}^R$ by
\begin{equation}	\label{Eq_candprodANSL}
   u\ast_{\nu}^R v=\mT_{\nu}(\mT_{\nu}^{-1} u\ast^{R}_M \mT_{\nu}^{-1} v),
\end{equation}
$dL$ becomes a derivation of $\ast_{\nu}^R$.
\label{prop:def_stn}
\end{proposition}

\begin{proof}
If we develop the expression of $dL(X)(u\ast^R_M v)$, we find $\mT\rho_{\nu}(X)(\mT^{-1} u\ast^R_M \mT^{-1} v)$, using the fact that $\rho_{\nu}$ is a derivation of $\ast^R_M$, one easily finds $dL(X)u\ast^R_M v+u\ast_{\nu}^R dL(X)v$.
\end{proof}
\section{Deformation of \texorpdfstring{$\SL(2,\eR)$}{SL2R}}		\label{sec:unifsl}
%++++++++++++++++++++++++++++++++++++++++++++++++++++++++++

\begin{abstract}
This section shows in some detail an instructive example of deformation of an Iwasawa subgroup: the Iwasawa subgroup of $\SL(2,\eR)$.
In this section we will use the parametrization \eqref{EqParmalSL} of $\SL(2,\eR)$, as well as the notations $G=\SL(2,\eR)$ and $\mG=\sldr$. Here are the main steps that will be performed:
{
\renewcommand{\theenumi}{\arabic{enumi}.}

\begin{enumerate}
\item The Iwasawa component $R=AN=G/K$ provides a double covering onto $\mO=\Ad(G)Z$ where $Z$ is any element of $\mK$ (which is one dimensional). The adjoint orbit $\mO$ being endowed with a canonical symplectic form described in subsection \ref{sub:coadjoint}, we consider on $R$ the corresponding symplectic structure.

\item The map $(a,l)\mapsto \Ad( e^{aH} e^{lE})Z$ turns out to be a global Darboux chart and induces the diffeomorphism
\[ 
  R\simeq \mO\simeq \eR^2.
\]
Under these identifications, the adjoint action of $R$ on $\mO$ becomes the simple multiplication of $R$ in itself, which is strongly hamiltonian.
\item  The Moyal product is $\gsl(2,\eR)$-covariant for the action of $\SL(2,\eR)$ on $\eR^2$.

\item We explicitly build the intertwining operator between $\rho_{\nu}$ and $dL$ and we write down a product (see proposition \ref{prop:def_stn}).

\item A theorem is stated in which we list the properties of the so constructed product.

\end{enumerate}

}				% Fin d'un groupe pour faire localement numéroter en chiffres arabes.
\end{abstract}

\subsection{Actions and Symplectic structure}
%--------------------------------------------

For our purpose, we consider the $\Ad^*$-invariant $2$-form $\xi_0\in\mG^*$ and $\widetilde{\mO}=\Ad^*(G)\xi_0$. As seen in \ref{sub:coadjoint}, the orbit $\widetilde{\mO}$ is a symplectic manifold with 
 \begin{equation}  \label{EqAStrucSympCoAdj}
  \widetilde{\omega}_{\xi}(X^*,Y^*)=\langle \xi,[X,Y]\rangle
\end{equation}
 and the dual momentum maps are $\lambda_X(\xi)=\langle\xi,X\rangle$.

\begin{probleme} 
	Check if these are actually the momentum maps.
\label{ProbMom}
\end{probleme}

In the present framework, we can work with adjoint orbits instead of the coadjoint ones because the group $G=\SL(2,R)$ is semisimple. Indeed, in this case, the Killing form\index{killing!form} $\dpt{B}{\mG\times\mG}{\eR}$ gives a $\Ad(G)$-equivariant isomorphism between $\mG$ and $\mG^*$. In order to see that, recall that a basic property of the Killing form is
\begin{equation}\label{eq:B_ad_invar}
   B\big( (\ad X)Y,Z \big)=-B\big(Y, (\ad X)Z \big),
\end{equation}
and when the group is semisimple, $B$ is nondegenerate. The isomorphism is given by
$\dpt{B'}{\mG}{\mG^*}$, $B'(X)Y=B(X,Y)$. The fact that $B$ is nondegenerate makes $B'$ an isomorphism, and the property \eqref{eq:B_ad_invar} gives the $\ad(G)$-equivariance of $B'$:
\[
   B'\big( (\ad X)Y \big)Z=-B'(Y)\big( (\ad X)Z\big).
\]

Here, in contrast with the case studied in \ref{sub:coadjoint}, we are working with adjoint orbits (and not the \underline{co}adjoint orbits), so the subalgebra to be studied is no more $\widetilde{\mO}$ but
\[
    \mO=\Ad(G)Z,
\]
where $Z$ is the generator of $\mK$ and the symplectic form is not exactly \eqref{eq_omega_Gs}, but
\begin{equation}\label{eq:omega_G}
  \omega_X(A^*,B^*)=B(X,[A,B]).
\end{equation}
The action of $G$ on $\mO$ is $g\cdot X=\Ad(g)X$. The corresponding notion of fundamental field is given by
\[
   X^*_{\phi(a,l)}=\Dsdd{ \Ad(e^{-tX})\phi(a,l) }{t}{0}.
\]
The Iwasawa theorem \ref{ThoIwasawaVrai} claims that $G/K=AN$ and that we have global diffeomorphism $\mA\oplus\mN\to AN$, $(a,n)\to e^ae^n$; $\mA\to A$, $a\to e^a$; $\mN\to N$, $n\to e^n$. We define $\mR=\mA\oplus\mN$ and the global diffeomorphism 
\begin{equation}	\label{EqDefphiaHlEZ}
\begin{aligned}
 \phi\colon \mA\oplus\mN&\to \mO \\ 
 aH+lE&\mapsto \Ad( e^{aH} e^{lE})Z.
\end{aligned}
\end{equation}
 That map can also be seen as 
\begin{equation}
\begin{aligned}
 \phi\colon\eR^2&\to \mO \\ 
(a,l)&\mapsto\Ad(e^{aH}e^{lE})Z.
\end{aligned}
\end{equation}
 In this way, we identify $\mA\oplus\mN$ and $\eR^2$ as two dimensional space.
\begin{proposition}
As homogeneous space, there is a double covering
\begin{equation}
\begin{aligned}
 \psi\colon G/K&\to \mO \\ 
[g]&\mapsto \Ad(g)Z. 
\end{aligned}
\end{equation}

\end{proposition}
\begin{proof}
The map $\psi$ is well defined and injective (up to the double covering) because the stabilizer of $\mK$ is $K$\ifthenelse{\value{siTHZ}=1}{}{ from theorem \ref{tho:Stab_K}}. The surjective condition is clear. The \emph{double} covering is expressed by the fact that $\psi([g])=\psi([g'])$ if and only if $g=\pm g'$.
\end{proof}

The symplectic $2$-form $\omega$ on $\mO$ induces a symplectic form  
\[
  \Omega=\phi^*\omega
\]
 on $\mA\oplus\mN\simeq\eR^2$.

\begin{proposition}
   The $2$-form $\phi^*\omega$ is constant and its value is
\[
              \Omega:=\phi^*\omega=-2B(F,E)da\wedge dl=\beta da\wedge dl;
\]
in other words, $\phi$ is a $\emph{global}$ Darboux chart for $\mO$.
\label{prop:Omega}
\end{proposition}

\begin{proof}
We have to compute
\[
   \Omega_{(a,l)}(\partial_a,\partial_l)=\omega_{\phi(a,l)}\big( (d\phi)_{(a,l)}\partial_a,(d\phi)_{(a,l)}\partial_l\big).
\]
First, we show that $d\phi(\partial_a)=-H^*_{\phi}$: 
\[
\begin{split}
   d\phi_{(a,l)}\partial_a&=\Dsdd{\phi(a+t,l)}{t}{0}
                    	=\Dsdd{\Ad(e^{(a+t)H}e^{lE})Z}{t}{0}\\
                    	&=\Dsdd{ \Ad(e^{tH}e^{aH}e^{lE})Z  }{t}{0}
		    	=\Dsdd{ \Ad(e^{tH})\phi(a,l) }{t}{0}\\
		    	&=-H^*_{\phi(a,l)}.
\end{split}
\]
In the same way, we find $d\phi(\partial_l)=\big( \Ad(e^{aH})E\big)^*_{\phi}$:
\[
 \begin{aligned}
   d\phi_{(a,l)}\partial_l&=\Dsdd{ \Ad(e^{aH}e^{l+tE})Z }{t}{0}
                    =\Dsdd{ \Ad(e^{aH}e^{tE} e^{-aH}e^{aH}e^{lE} )Z }{t}{0}\\
		    &=\Dsdd{ \Ad(e^{aH}e^{tE}e^{-aH})\phi(a,l) }{t}{0}
		    =\Dsdd{ \Ad( e^{t\Ad(e^{aH})E} )\phi(a,l) }{t}{0}\\
		    &=-\big( \Ad(e^{aH})E\big)^*_{\phi(a,l)}.
\end{aligned}
\]
Using formula \eqref{eq:omega_G} for the symplectic form, 
\begin{equation}
\begin{split}
  \Omega_{(a,l)}(\partial_a,\partial_l)&=B\big( \phi(a,l),[-H,-\Ad(e^{aH})E]\big) \\
                           &=B\big( \Ad(e^{aH})\Ad(e^{lE})Z,\Ad(e^{aH})[H,E]\big)\\
			   &=2B\big( Z,\Ad(e^{-lE})E \big)\\
			   &=2B(Z,E).
\end{split}
\end{equation}
Defining $\beta=-2B(E,F)$ we write it as
\begin{equation} 
  \Omega=\phi^*\omega=-2B(F,E)da\wedge dl=\beta da\wedge dl.
\end{equation}
\end{proof}
So, as symplectic manifold, $(\mO,\omega)$ is nothing but $(\eR^2,da\wedge dl)$, the diffeomorphism being $\phi$. The symplectic structure $\Omega$ induces a Poisson structure $P$ given by equation \eqref{eq:def_Poisson}. In the present case, it reads
\begin{align}
(\Omega_{ij})&=\beta\begin{pmatrix}
0 & 1 \\
-1 & 0
\end{pmatrix}
&(P)&=\beta^{-1}\begin{pmatrix}
0 & -1 \\
1 & 0
\end{pmatrix}
\end{align}
 and
\begin{equation}\label{eq:Poisson}
  \{f,g\}=\beta^{-1}(\partial_lf\partial_ag-\partial_af\partial_lg).
\end{equation}

The action of $G$ on $\mO$ can be turned into an action on $\eR^2$ using the chart $\phi$. It is done by defining $\dpt{\tau}{G\times\eR^2}{\eR^2}$,
\begin{equation}
   \tau=\phi^{-1}\circ \Ad\circ\phi,
\end{equation}
or $\tau_g(a,l)=\phi^{-1}\big( \Ad(g)\phi(a,l)\big)$.  The notion of fundamental field\index{fundamental!vector field!on $\eR^2$} at $x=(a,l)\in\eR^2$ is thus given by
\begin{equation}
  X^*_x=\Dsdd{e^{-tX}\cdot x}{t}{0}
       =\Dsdd{ \phi^{-1}\big( \Ad(e^{-tX})\phi(a,l)  \big) }{t}{0},
\end{equation}
for which we will often use the path representation
\[
   X^*_x(t)=\phi^{-1}\big( \Ad(e^{-tX})\phi(a,l)  \big).
\]
From $\Ad$-invariance of $\omega$,
\[
   \tau^*\Omega=\tau^*\phi^*\omega
               =(\phi\circ\phi^{-1}\circ \Ad\circ\phi)^*\omega
	       =\phi^*(\Ad)^*\omega
	       =\phi^*\omega
	       =\Omega.
\]
Thus the symplectic form is $G$-invariant:
\begin{equation}     \label{eq:tau_s_Omega}
  \tau^*\Omega=\Omega,
\end{equation}
That implies in particular that $\tau$ satisfies theorem \ref{tho:equiv_Poisson}.

\begin{proposition}
The action $\tau$ of $G$ on the symplectic space $(\eR^2,\Omega)$ is Hamiltonian and the dual momentum maps $\dpt{\lambda'_X}{\eR^2}{\eR}$ are given by (cf .\ref{def:app_mom_mom_duale})
\begin{equation}
  \lambda'_X(a,l)=-B\big(X,\phi(a,l)\big)
\end{equation}
for each $X\in\mG$.
\label{prop:lambda_X}
\end{proposition}

\begin{proof}
We have first to check the identity $i(X^*)\Omega=i(X^*)(\phi^*\omega)=d\lambda'_X$. Let us apply both sides on the vector\ifthenelse{\value{siTHZ}=1}{}{\footnote{Existence comes from lemma \ref{LemFundSpansTan}.}} $A^*_x$, with $A\in\mG$ and $x=(a,l)\in\eR^2$. On the one hand
\[
  i(X^*_x)\Omega_x(A^*_x)=\omega_{\phi(x)}\big(   d\phi_xX^*_x,d\phi_xA^*_x   \big),
\]
but  
\begin{equation}
  d\phi_xX^*_x=\Dsdd{\phi(X^*_x(t))}{t}{0}
              =\Dsdd{ \Ad(e^{-tX})\phi(aH,lE) }{t}{0}
	      =-X^*_{\phi(a,l)}.
\end{equation}
The same being true for $A$,
\[
  i(X^*_x)\Omega_x(A^*_x)=\omega_{\phi(x)}(X^*_{\phi(x)},A^*_{\phi(x)})=B(\phi(x),[X,A]).
\]
On the other hand,
\begin{equation}
\begin{aligned}
    (d\lambda'_X)_x(A^*_x)&=\Dsdd{ (\lambda'_X\circ\phi^{-1})\Big(   \Ad(e^{tA})\phi(a,l)   \Big) }{t}{0}\\
                         &=\Dsdd{  B\Big(X,\Ad(e^{tA})\phi(a,l) \Big)  }{t}{0} \\
			 &=B\Big(  \Dsdd{\Ad(e^{tA})\phi(x)}{t}{0},X   \Big)    &\text{$B$ is linear}\\
			 &=B\Big(  (\ad A)\phi(x),X   \Big)\\
			 &=-B\big(\phi(x),(\ad A)X\big) &\text{$B$ is $\Ad$-invariant}\\
			 &=B(\phi(x),[X,A]).
\end{aligned}
\end{equation}
That proves that $i(X^*)\Omega=d\lambda'_X$.  The second part of the proof is to see that condition \eqref{eq:hamil} holds.  Using the fact that $X_{\lambda'_Y}=Y^*$, we find
\[ 
\begin{split}
  \{ \lambda'_X,\lambda'_Y \}(a,l)&=-\Omega(X_{\lambda'_X},X_{\lambda'_Y})
		=-\Omega_{(a,l)}(X^*,Y^*)\\
		&=-\omega_{\phi(a,l)}(X^*,Y^*)
		=-B([X,Y],\phi(a,l))\\
		&=\lambda'_{[X,Y]}(a,l)
\end{split}
\]
where the star refers to the action on $\mO$. Explicit computations of Poisson bracket between $\lambda'_X$'s at page \pageref{pg:explic_com_lamb} will confirm that result.

\end{proof}

We are now able to furnish explicit formulas for $\lambda'_H$, $\lambda'_E$ and $\lambda'_F$ by virtue of the latter proposition.  The first computation is:
\begin{equation}
\begin{aligned}
  \lambda'_H(a,l)&=-B(H,\Ad(e^{lE})Z)
                =-B( \Ad(e^{-lE})H,Z )\\
		&=-B(H+[-lE,H]+\ldots,Z)
		=-B(H,Z)+B([-lE,H],Z)\\
		&=-2lB(E,F),
\end{aligned}
\end{equation}
so
\begin{equation}   \label{EqlamHal}
  \lambda'_H(a,l)=-\beta l.
\end{equation}
Second,
\begin{equation}
  \lambda'_E(a,l)=-B(\Ad(e^{-aH})E,\Ad(e^{lE}))
		=-e^{-2a}B(\Ad(e^{-lE})E,Z)
		=-\frac{\beta}{2}e^{-2a}.
\end{equation}
Then,
\begin{equation}  \label{EqlamEal}
\lambda'_E(a,l)=-\frac{\beta}{2}e^{-2a}.
\end{equation}
The last one is
\begin{equation}
\begin{aligned}
\lambda'_F(a,l)&=-B\big(  \Ad(e^{lE})Z,e^{-aH}F  \big)
              =-e^{2a}B\big(Z,  \Ad(e^{-lE})F   \big)\\
	      &=-e^{2a}B\big(Z, F-l[E,F] +\frac{l^2}{2} [E,[E,F]]+\ldots  \big)\\
	      &=-e^{2a}\left[     B(Z,F)-lB(Z,H)-\frac{l^2}{2}B(Z,2E)        \right]\\
	      &=-e^{2a}\big(  B(Z,F)+l^2B(F,E)   \big)\\
	      &=-e^{2a}\big(  -\frac{\beta}{2}-l^2\frac{\beta}{2}   \big)
	      =e^{2a}\frac{\beta}{2}(l^2+1).
\end{aligned}
\end{equation}
Finally,
\begin{equation}  \label{EqlamFal}
\lambda'_F(a,l)=\frac{\beta}{2}e^{2a}(l^2+1).
\end{equation}
Using formula \eqref{EqPoisson} for the Poisson bracket, one can check that the required relations \eqref{eq:hamil} are satisfied:
\begin{subequations}  \label{pg:explic_com_lamb}
\begin{align}
  \{\lambda'_H,\lambda'_E\}&=2\lambda'_E\\
  \{\lambda'_H,\lambda'_F\}&=-2\lambda'_F\\
  \{\lambda'_E,\lambda'_F\}&=\lambda'_H.
\end{align}
\end{subequations}
This confirms the fact that our action of $\SL(2,\eR)$ on $AN$ is Hamiltonian.

Using the global diffeomorphism \eqref{EqDefphiaHlEZ}, and the map 
\begin{equation}
\begin{aligned}
 j\colon AN&\to \mO \\ 
r&\mapsto \Ad(r)Z 
\end{aligned}
\end{equation}
we identify
\[
   R\simeq\mO\simeq\eR^2.
\]
The action of $R$ on itself induced from the adjoint action of $R$ on $\mO$ is 
\[ 
  r\cdot s=j^{-1}\big( r\cdot j(s) \big)=j^{-1}\big( \Ad(rs)Z \big)=rs.
\]
It is the left multiplicative action required in definition \ref{DefWKBCompl}. The Lie group $R$ is endowed with the symplectic form 
\[ 
\omega^R=j^*{\phi^{-1}}^*\Omega.
\]
The notion of fundamental vector for the action of $R$ on itself is given by
\begin{equation}
  X^*_r=\Dsdd{e^{-tX}\cdot r}{t}{0} 
		=\Dsdd{ j^{-1}\big( e^{-tX}\cdot j(r) \big) }{t}{0}
		=dj^{-1} X^*_{j(r)},
\end{equation}
but we know that 
\[ 
  e^{-tX}\cdot j(r)=\Ad(e^{-tX r})Z=[\phi\circ \tau(e^{-tX}r)\circ \phi^{-1}]Z,
\]
 then
\[
X^*_r=dj^{-1}\circ d\phi X^*_{r\cdot \phi^{-1}(Z)}.
\]
If $r=e^{aH}e^{lE}$, then $r\cdot \phi^{-1}(Z)=(a,l)$ and
\begin{equation}
   X^*_r=(dj^{-1}\circ d\phi) X^*_{(a,l)}
\end{equation}
where the fundamental field of the right hand side is taken in the sense of the action of $R$ on $\eR^2$. 

The following proposition shows that the explicit form of $\lambda$ and $\lambda'$ are the same up to natural identifications.

\begin{proposition}  
The left multiplicative action of $R$ on itself is Hamiltonian and the dual momentum maps are given by  $\lambda_X\colon R\to \eC$,
\begin{equation}
\lambda_X= \lambda'_X\circ\phi^{-1}\circ j.
\end{equation}
for each $X\in\mR$.
\label{PropMomslR}
\end{proposition}

\begin{proof}
Once again, the proof is just a verification of the two properties of a momentum map. The first one is
\begin{equation}
\begin{split}
  i(X^*_r)\omega^R Y&=\omega^R_r(dj^{-1} d\phi X^*_{(a,l)},Y)
		=\Omega_{(\phi^{-1}\circ j)(r)}\big( X^*_{(a,l)},d\phi^{-1} dj_r Y \big)\\
		&=(\lambda'_X\circ d\phi^{-1}\circ dj)Y
		=d\lambda_X Y.
\end{split}
\end{equation}
For the second condition, we consider $r=e^{aH}e^{lE}$ and
\begin{equation}
\begin{split}
  \{ \lambda_X,\lambda_Y \}(r)&=X^*_r(\lambda_Y)
		=(dj^{-1} d\phi X^*_{(a,l)})(\lambda'_Y\circ \phi^{-1}\circ j)\\
		&=X^*_{(a,l)}(\lambda'_Y)
		=\lambda'_{[X,Y]}(a,l)\in\eC
\end{split}
\end{equation}
while 
\[ 
  \lambda_{[X,Y]}(r)=\lambda'_{[X,Y]}\circ\phi^{-1}\circ j(r)=\lambda'_{[X,Y]}(a,l).
\]
\end{proof}

\subsection{Guessing the star product}
%--------------------------------------

The Moyal star product is invariant under the action of $\eR^2$ on itself $L_xy=x+y$ in the sense that if we pose $(L_y^*f)(x)=f(x+y)$ it is clear that
\begin{equation}
\begin{split}
    (L_s^*f\ast_M L_s^*g)(x)&=
    \exp\left[{\displaystyle\frac{\nu}{2}P^{ij}(\partial_{y^i}\wedge\partial_{z^j})}\right]f(y+s)g(z+s)|_{y=z=x}\\
	                    &=L^*_s(f\ast_M g)(x).
\end{split}
\end{equation}
We are however not interested by that action on $\eR^2$. The action which we look at is the one of $\SL(2,R)$.

\begin{proposition}
The product $\ast_M$ is $\gsl(2,\eR)$-invariant at order $0$ and $1$.
\end{proposition}

\begin{proof}
The invariance at order zero is given with some concise notations by
\[
 (gu)(gv)(x)=u(gx)v(gx)=(uv)(gx),
\]
The action $\tau_g$ of an element $g\in G$ satisfies $\tau_g^*\Omega=\Omega$ (equation \eqref{eq:tau_s_Omega}), so  theorem \ref{tho:equiv_Poisson} gives $\{u\circ\tau_g,v\circ\tau_g\}=\{u,v\}\circ\tau_g$.  Since Poisson bracket\index{Poisson structure!and Moyal product} is the first term of the Moyal product, at first order
\[
  \tau_g^*(u\ast_M v)=\tau_g^*u\ast_M\tau_g^*v.
\]


\end{proof}

\begin{proposition}
   The product $\ast_M$ is $\sldr$-covariant for the homomorphism given by proposition \ref{prop:lambda_X} or equivalently by equations \eqref{EqlamHal}, \eqref{EqlamEal}, and \eqref{EqlamFal}.
\end{proposition}

\begin{proof}
The Moyal star product can be written as
\[
   u\ast_M v=\sum \frac{\nu^k}{k!}P_k(u,v)
\]
with $P_k(u,v)=\Omega^{IJ}\partial_Iu\partial_Jv$ where $I$ and $J$ are summed over $k$-uple of $0$ and $1$, including a sum over $k$ itself ($x^0=a$, $x^1=l$). For a given $I$, there is only one $J$ such that $\Omega^{IJ}\neq 0$. There are $\binom{k}{m}$ multi-indices $I$ providing the term $\partial_I=\partial_0^m\partial_1^n$ with $n+m=k$. For each of them, $\Omega^{IJ}=\me{n}$. Therefore
\begin{equation}\label{eq:P_k}
  P_k(u,v)=\sum_{m=0}^k\me{k-m}\binom{k}{m}\partial_0^m\partial_1^nu\,\partial_0^n\partial_1^mv.
\end{equation}
For example,
\[
  P_1(u,v)=-\partial_1u\partial_0v+\partial_0u\partial_1v=\{u,v\}.
\]
If $k$ is even, the expression \eqref{eq:P_k} is symmetric with respect of $u$ and $v$, so that these terms will not contribute in the computation of the commutators $[u,v]_{\ast_M}$. We are left with
\begin{equation}\label{eq:comm_lambda_X}
   [\lambda'_X,u]_{\ast_M}
         =2\sum_{k=0}^{\infty}\frac{\nu^{2k+1}}{(2k+1)!}P_{2k+1}(\lambda'_X,u).
\end{equation}

First we compute $[\lambda'_H,u]_{\ast_M}$:
\begin{equation}
   P_{2k+1}(\lambda'_H,u)=\delta_{k0}(-\partial_1\lambda'_H\partial_0 u+\partial_0\lambda'_H\partial_1 u)=\delta_{k0}\{\lambda',u\},
\end{equation}
thus
\begin{equation}
\begin{split}
  [\lambda'_H,u]_{\ast_M}=2\nu P_1(\lambda'_H,u)
                        =2\nu\{\lambda'_H,u\}
			=2\nu\beta\partial_au.
\end{split}
\end{equation}
By the way, we point out the relation
\[
\ad_{\ast_M}\lambda'_H=2\nu\beta\partial_a.
\]

Now, we turn our attention to the commutator $[\lambda'_E,u]_{\ast_M}$:
\begin{equation}
\begin{split}
  P_{2k+1}(\lambda'_E,u)
      &=-\sum_{n=0}^{k}\me{m}\binom{2k+1}{m}\binom{2k+1}{m}(\partial_0^m\partial_1^n\lambda'_E)\,(\partial_0^n\partial_1^mu)\\
      &=\partial_a^{2k+1}\big( -\frac{\beta}{2}e^{-2a} \big)\partial_l^{2k+1}u 
	=\beta 2^{2k}e^{-2a}\partial_l^{2k+1}u,
\end{split}
\end {equation}
thus
\begin{equation}  \label{eq:comm_lambda_E}
\begin{split}
  [\lambda'_E,u]_{\stM}&=2\sum_{k=0}^{\infty}\frac{\nu^{2k+1}}{(2k+1)!}\beta
                            2^{2k}e^{-2a}\partial_l^{2k+1}u
	             =\beta e^{-2a}\sinh(2\nu\partial_l)u,
\end{split}
\end{equation}
so that
\[
     \ad_{\stM}\lambda'_E=\beta e^{-2a}\sinh(2\nu\partial_l).
\]
Last we check $[\lambda'_E,\lambda'_F]_{\stM}=2\nu\{\lambda'_E,\lambda'_F\}$. When $u=0$, the only non vanishing term in the sum \eqref{eq:comm_lambda_E} is $k=0$. Since $\partial^{3}_{l}\lambda'_{F}=0$,
\[
   [\lambda'_E,\lambda'_F]_{\stM}=2\nu\beta e^{-2a}\partial_l\lambda'_F,
\]
but
\begin{equation}
 2\nu\{\lambda'_E,\lambda'_F\}
             =2\nu(\partial_a\lambda'_E\partial_l\lambda'_F-\partial_l\lambda'_E\partial_a\lambda'_F)
	     =2\nu\beta e^{-2a}\partial_l\lambda'_F.
\end{equation}
\end{proof}

Before to go on, let us compute the operator $\ad_{\stM}\lambda'_F$ in order to complete our collection. We take once again the formula \eqref{eq:P_k}, with $\lambda'_F$ and $u$:
\begin{equation}
   P_{2k+1}(\lambda'_F,u)=-\sum_{m=0}^{2k+1}\me{m}\binom{2k+1}{m}\partial_a^m\partial_l^n\lambda'_F
                                                                \partial_a^n\partial_l^m u.
\end{equation}
It is clear that $\lambda'_F$ can be derived  only two times with respect of $l$ and as much as we want with respect of $a$. Then possible $n$ are $n=0,1,2$, whose corresponding $m$ are $2k-1$, $2k$, and $2k+1$. Some computations lead to
\begin{equation}
\begin{split}
    P_{2k+1}(\lambda'_F,u)&=-k(2k+1)\beta 2^{2k-1}e^{2a}\partial_a^2\partial_l^{2k-1}u\\
                    &\quad +(2k+1)\beta 2^{2k}l\partial_a\partial_l^{2k}u\\
                    &\quad -\beta 2^{2k}(1+l^2)e^{2a}\partial_l^{2k+1}u.
\end{split}
\end{equation}
Replacing into the series \eqref{eq:comm_lambda_X}, we find
\begin{align*}
\begin{split}
[\lambda'_F,u]_{\stM}&=e^{2a}\Big\{
   \sum\frac{\nu^{2k+1}}{(2k+1)!}\beta (-k)(2k+1)\frac{2^{2k+1}}{2}
                     \partial_a^2\partial_l^{2k-1}u\\
 &\qquad+\sum\frac{\nu^{2k+1}}{(2k+1)!}(2k+1)2^{2k+1}l\partial_a\partial_l^{2k}u\\
 &\qquad\beta\sum\frac{\nu^{2k+1}}{(2k+1)!}2^{2k+1}(1+l^2)\partial_l^{2k+1}u\Big\}
\end{split}\\
\begin{split}
 &=-\beta e^{2a}\partial_a^2\sum_{k=1}^{\infty}\frac{(2\nu)^{2k}}{(2k)!}k\nu\partial_l^{2k-1}\\
 &\quad+2\beta\nu e^{2a}\partial_a\circ\cosh(2\nu\partial_l)\\
 &\quad-\beta e^{2a}(1+l^2)\sinh(2\nu\partial_l).
\end{split}
\end{align*}
Finally,
\begin{equation}
\begin{split}
   \ad_{\stM}\lambda'_F&=-\nu^2\beta e^{2a}\partial_a^2\circ\sinh(2\nu\partial_l)\\
                     &\quad+2\nu\beta e^{2a}l\partial_a\circ\cosh(2\nu\partial_l)\\
		     &\quad-e^{2a}(1+l^2)\sinh(2\nu\partial_l).
\end{split}
\end{equation}

\begin{corollary}
The star product $\ast_M^R$ on $R$ defined for $u,v\in C^{\infty}(R)$ by
\begin{equation}
  (u\ast_M^R v)(r)=( u\circ T^{-1}\stM v\circ T^{-1} )T(r)
\end{equation}
where $T=\phi^{-1}\circ j$ is covariant for the functions $\lambda$ of proposition \ref{PropMomslR}.

\end{corollary}
Remark that from general theory of star products, the so-defined $\ast_R$ is a formal star product on $R$.

\begin{proof}
From definition of $\ast_M^R$, on the one hand
\[ 
  (\lambda_X\ast_M^R \lambda_Y)(r)-X\leftrightarrow Y=(\lambda'_X\stM\lambda'_Y)T(r)-X\leftrightarrow Y
=2\nu\{  \lambda'_X,\lambda'_Y \}_{\eR^2}T(r),
\]
while on the other hand, $\omega^R=T^*\Omega$, so that \ifthenelse{\value{siTHZ}=1}{}{point \ref{ite_equivii} of theorem \ref{tho:equiv_Poisson}  gives}
\[ 
  \{ \lambda'_X,\lambda'_Y \}_{\eR^2}\circ T=\{ \lambda'_X\circ T,\lambda'_Y\circ T \}_R=\{ \lambda_X,\lambda_Y \}_R.
\]

\end{proof}

All that makes the theory developed earlier (in particular proposition \ref{prop:def_stn}) valid here. So we pose
\begin{equation}
\begin{aligned}
 {\rho_{\nu}}\colon\sR &\to {\End\big(  C^{\infty}(R)[ [\nu]] \big)} \\ 
  X &\mapsto {\frac{1}{2\nu}\ad_{\ast_M^R}(\lambda_X);} 
\end{aligned}
\end{equation}
using the explicit expressions of $ad_{\stM}(\lambda'_X)$, we find
\begin{align}
  \rho_{\nu}(H)&=\beta\partial_a,	&\rho_{\nu}(E)&=\frac{\beta}{2\nu}e^{-2a}\sinh(2\nu\partial_l).
\end{align}
Using \eqref{eq:dL_et_Poisson} with $\lambda_H=-\beta l$, it is clear that $dL(H)=-\beta\{l,u\}=\beta\partial_a u$. Therefore
\begin{equation}
   \rho_{\nu}(H)=dL(H),
\end{equation}
but the requested identity $\rho_{\nu}(E)=dL(E)$ will not hold. The problem is that $dL(X)=X_{\lambda_X}$ is a vector field, while $\rho_{\nu}(E)$ comes with (infinitely) multiple derivatives, hence this is not a vector field. Conclusion: the operator $\mT$ of equation \eqref{eq:TrnT} must not act on the variable~$a$.

First we consider a partial Fourier transform $\mF$\nomenclature[F]{$\mF$}{Fourier transform}\index{partial!Fourier transform}:
\begin{equation}
 (\mF u)(a,\alpha)=\hu(a,\alpha):=\us{\sqrt{2\pi}}\int e^{-i\alpha l}u(a,l)dl,
\end{equation}
the inverse being given by
\begin{equation}
 (\mF^{-1}\hu)(a,l)=\us{\sqrt{2\pi}}\int e^{il\alpha}\hu(a,\alpha)d\alpha.
\end{equation}
It is clear that $\mF\rho_{\nu}(H)\mF^{-1}=\rho_{\nu}(H)$, but $\mF\rho_{\nu}(E)\mF^{-1}=\frac{\beta}{2\nu}e^{-2a}\sinh(2i\nu\alpha)$. Indeed, if we define $\hv(a,\alpha)=\sinh(2i\nu\alpha)\hu(a,\alpha)$, 
\begin{equation}
\begin{split}
 (\rho_{\nu}(E)\mF^{-1}\hu)(a,l)&=\frac{\beta}{2\nu}e^{-2a}\us{\sqrt{2\pi}}\sinh(2\nu\partial_l)\int e^{il\alpha}\hu(a,\alpha)d\alpha\\
                        &=\frac{\beta}{2\nu}e^{-2a}\us{\sqrt{2\pi}}\int e^{il\alpha} \sinh(2i\alpha\nu)\hu(a,\alpha)d\alpha\\
			&=\frac{\beta}{2\nu}e^{-2a}(\mF^{-1}\hv)(a,l).
\end{split}
\end{equation}
This is nothing but the fact that the Fourier transform turns a derivation into a multiplication.

As can be seen on an asymptotic expansion, the deformation $\nu$ parameter is necessarily purely imaginary, then we can here pose $\nu=i\theta$ with $\theta\in\eR$, so that
\begin{equation}\label{eq:FrnEF}
   \mF\rho_{\nu}(E)\mF^{-1}=\frac{\beta i}{2\theta}e^{-2a}\sinh(2\alpha\theta).
\end{equation}
Using \eqref{eq:dL_et_Poisson}, we find 
\begin{equation}\label{eq:dLE}
   dL(E)=\beta e^{-2a}\partial_l.
\end{equation}
 Comparing it with the expression of $\mF\rho_{\nu}(E)\mF^{-1}$, we see that (up to constant factor) we have to act in such a way that $\sinh(2\alpha\theta)$ is converted into a derivation. This is done by a Fourier transform. We pose $\xi=\sinh(2\theta\alpha)$ and
\[
  \tilde f (a,\xi)=\us{\sqrt{2\pi}}\int e^{i\xi p}f(a,p)dp.
\]
As usual, 
\[
   \widetilde{\partial_{\alpha}f}=-i\xi\tilde f .
\]
This suggests us to consider the change of variable 
\[
  \phi_{\theta}(a,\alpha)=(a,\us{2\theta}\sinh(2\theta\alpha)),
\]
and finally,
\begin{equation}
   \mT_{\theta}:=\mF^{-1}\circ\phi_{\theta}^*\circ\mF,
\end{equation}
where $\phi_{\theta}^*$ is defined by $(\phi_{\theta}^*u)(a,\alpha)=u(a,\us{2\theta}\sinh(2\theta\alpha))$. The result of our construction is the following which proves that we are in the situation of proposition~\ref{prop:def_stn}.
\begin{theorem}
\[
   \mT_{\theta}\circ\rho_{\nu}(E)\circ \mT_{\theta}^{-1}=\beta e^{-2a}\partial_l=dL(E).
\]
\end{theorem}
\begin{proof}
Notice that $(\phi_{\theta}^*\mF u)(a,\alpha)=\hu(a,\sinh(2\theta\alpha))$, and then define $\hv(a,\alpha)=\hu(a,\sinh(2\theta\alpha))$; equation \eqref{eq:FrnEF} is
\[
   (\mF\rho_{\nu}(E)\mF^{-1}\hv)(a,\alpha)=\frac{\beta i}{2\theta}e^{-2a}\sinh(2\theta\alpha)\hv(a,\alpha).
\]
Applying $(\phi^*)^{-1}$ on the right hand side, we find $\frac{\beta i}{2\theta}e^{-2a}2\theta\alpha\hu(a,\alpha)$.  This allows us to compute
\[ 
\begin{split}
  (\mT_{\theta}\rho_{\nu}(E)\mT_{\theta}^{-1})u(a,l)&=\beta ie^{-2a}\mF^{-1}(\alpha\hu)(a,l)\\
                                        &=\beta ie^{-2a}\us{\sqrt{2\pi}}\int \hu(a,\alpha)(-i)\partial_l e^{il\alpha}d\alpha\\
					&=\beta e^{-2a}(\partial_lu)(a,l).
\end{split}
\]
\end{proof}

\subsection{Formula for the product}
%-----------------------------------

The fact the $\mT_{\theta}$ intertwines $\rho_{\nu}$ and $dL$ makes that the candidate to be a product on the $AN$ of $\SL(2,\eR)$ can be computed using formula \eqref{Eq_candprodANSL}. Computations are rather long and done in the articles \cite{StrictSolvableSym} and \cite{Biel-Massar} (see particularly point 4), so we will not give them here. We will also not precise the functional space of convergence for the resulting product.

\begin{probleme}
Il faut voir si ça entrelace effectivement ces deux cocos, ou bien juste $\rho_{\nu}(E)$ et $dL(E)$.
\label{Probintertw}
\end{probleme}

We are searching for a function $\dpt{\rho}{R}{\eR}$ such that
\begin{equation}\label{eq:1s15}
  \int_Rf(r)\rho(r)dr=\int_Rf(br)\rho(r)dr,
\end{equation}
where $dr=da\wedge dl$. We consider the multiplication map $\dpt{\phi}{R\times R}{R}$, $\phi(r,r')=rr'$, and its partial derivatives $A^i_j=\dsd{\phi^i}{r^j}$.  We will perform the change of variable $ s =br$ in \eqref{eq:1s15}. So, $ s^i=\phi^i(b,r)$, and if we pose , we have $dr^k=(A^{-1})^k_id s^i$. We find
\[
  \int_Rf(r)\rho(r)dr=\int_Rf( s)\rho(b^{-1} s)\det(A^{-1})d s.
\]
In the right-hand side, we rename $ s$ to $r$ and we impose the equality to be correct for all functions $\dpt{f}{R}{\eC}$. Then for all $r$, $k\in R$,
\[
   \rho(r)=\rho(k^{-1} r)\det(A^{-1}).
\]
 In order to compute the value of $\det(A)$, we have to write $\big(dL_{(a,b)}\big)_{(a',l')}$ in a matrix form. If we consider $v=(a'(t),l'(t))$, we have
\begin{equation}
\big(dL_{(a,b)}\big)_{(a',l')}v=\frac{d}{dt}\begin{pmatrix}
e^{a+a'(t)} & e^{a+a'(t)}(l'(t)+e^{-2a'(t)l}) \\
0 & e^{-a-a'(t)}
\end{pmatrix}_{t=0},
\end{equation}
but we know that $v$ can be written as
\[
  v=\begin{pmatrix}
\frac{da'}{dt}e^{a'} & \frac{da'}{dt}e^{a'}l'+e^{a'}\frac{dl'}{dt} \\
0 & -\frac{da'}{dt}e^{a'}
\end{pmatrix}
\]
Now, if we want to write $\big(dL_{(a,b)}\big)_{(a',l')}$ as $\begin{pmatrix}A&B\\C&D\end{pmatrix}$, we obtain: $A=e^a$, $C=0$, $D=e^{-a}$, and thus $\det(A)=1$.

So we can use $\rho(r)=1$ for all $r\in R$, and the integral of $\dpt{f}{R}{\eC}$ over $R$ can be written as
\[
\int_Rf(r)dr
\]
with $dr=da\wedge dl$.

In the parametrization 
\[ 
  (a,l)=\begin{pmatrix}
 e^{a}		&  e^{a}l\\
0		&	 e^{-a}
\end{pmatrix},
\]
of $R=AN$ the form $da\wedge dl$ is a left invariant measure, so the integral of the function $f\colon R\to \eR$ on $R$ is given by
\[ 
  \int_R f=\int_{\eR^2} f(a,l)da\,dl.
\]
Remark that $da\,dl$ is the Liouville measure\index{Liouville measure!for $\SL(2,\eR)$} by proposition \ref{prop:Omega}. It is important for definition \ref{DefWKBCompl}.

We consider a subset $\eA\subset\Fun(R)$, and we define the product $\star^{R}_{\theta}$ on $\eA$ by
\begin{equation}\label{eq:star_R}
\begin{split}
(a\star^{R}t b)(a_0,l_0)
=\int_{R\times R}K^R_{\theta}\big((a_0,l_0)&,(a_1,l_1),(a_2,l_2)\big)\\
                                           &a(a_1,l_1)b(a_2,l_2)da_1dl_1da_2dl_2.
\end{split}
\end{equation}
where
\[
K_{\theta}^A(g_0,g_1,g_2)=\us{\theta^2}\mA^R(g_0,g_1,g_2)e^{i\theta \mS^R(g_0,g_1,g_2)}
\]
 with
\begin{subequations}
\begin{align}
  \mA^R(g_0,g_1,g_2)&=\bigoplus_{0,1,2}\cosh(a_1-a_2)\\
  \mS^R(g_0,g_1,g_2)&=\bigoplus_{0,1,2}\sinh(2(a_0-a_1))l_2.
\end{align}
\end{subequations}
Here, the symbol $\bigoplus_{0,1,2}$ stands for a cyclic sum over the indices $0,1,2$.
These functions are invariant\index{invariant!function on a group} under the $\SL(2,R)$ left action: when $x$, $y$, $z$, $g\in\SL(2,R)$,
\begin{equation}
 \mA^R(gx,gy,gz)=\mA^R(x,y,z)\quad\text{and}\quad \mS^R(gx,gy,gz)=\mS^R(x,y,z).
\end{equation}
 The first equality is clear; let us show the second. If $x=(a_x,l_x)$ (the same for $y$ and $z$) and $g=(a,l)$, the computation of $\mS^R(gx,gy,gz)$ is the one of $\mS^R(x,y,z)$ with the replacements
\begin{subequations}
\begin{align}
a_x&\rightarrow a_x+l\\
l_x&\rightarrow l_x+e^{-2a_x}l,
\end{align}
\end{subequations}
and the same for $y$ and $z$. Using the formula $\sinh t=\frac{1}{2}(e^{t}-e^{-t})$, one finds the right cancellations.

\begin{probleme}
Il faut trouver l'article où ce résultat se trouve, et citer le théorème exact.
\label{ProbEnonSLdef}
\end{probleme}

\subsubsection{Remark on (formal) star product}\label{subsec:rem_on_sp}
%/////////////////////////////////////////////

One can  find a definition of an asymptotic expansion for oscillating integrals in \cite{Dieu7} under the form
\[
   I_{\lambda}=\int e^{(i/\lambda)S(x)}\phi(x)\sim\sum_n\lambda^nc_n.
\]
It can be shown that such a  expansion used on \eqref{eq:star_R} gives rise of a formal star product:
\begin{equation}	\label{EqDevFedFor}
(a\star^{R} b)(g)\sim a(g)b(g)+\frac{\theta}{2i}\{a,b\}(g)+o(\theta^2).
\end{equation}

\section{Extension lemma}		\label{SecExtLem}
%-------------------------

Let $(\mfs_i,\Omega_i)_{i=1,2}$ be  symplectic Lie algebras and $(S_i,\omega_i)$ the respective Lie groups with left invariant symplectic forms: $(\omega_i)_g=(L_g)^*\Omega_i$. 
We suppose to know a homomorphism  $\dpt{\rho}{\mfs_1}{\Der(\mfs_2)\cap\mfsp(\Omega_2)}$ and a Darboux chart $\phi_i\colon \mathfrak{s}_i\to S_i$ for each of the two symplectic Lie groups. Our first purpose is to build a Darboux chart on the split extension
\[
   \mfs:=\mfs_1\oplus_{\rho}\mfs_2.
\]

\begin{remark}
Most of the time we are in the inverse situation: we have an algebra $\mathfrak{s}$ which turn out to be a split extension $\mathfrak{s}_{1}\oplus_{\ad}\mathfrak{s}_{2}$ for which we have to check that $\ad(\mathfrak{s}_{1})$ is a symplectic action of $\mathfrak{s}_{1}$ on $(\mathfrak{s}_{2},\Omega_{2})$. See the example of section \ref{SecUnifSOdn}.
\end{remark}
 
\begin{proposition}
In this setting, the map $\dpt{\phi}{\mfs}{S}$,
\begin{equation}
  \phi(X_1,X_2)=\phi_2(X_2)\phi_1(X_1)
\end{equation}
is a Darboux chart.
\label{prop:Darboux}
\end{proposition}

\begin{proof}
An element $X\in T_{\phi^{-1}(g)}(\mfs_1\oplus\mfs_2)=\mathfrak{s}_1\oplus\mathfrak{s}_2$ is denoted by $X=(X_1,X_2)$ with $X_i\in\mathfrak{s}_i$, and the symplectic form on $\mfs_1\oplus\mfs_2$ is given by
\begin{equation}		\label{eq:Omega}
   \Omega\big( (A_1,A_2),(B_1,B_2) \big)=\Omega_1(A_1,B_1)+\Omega_2(A_2,B_2)
\end{equation}
where we identify  $\mfs_i$ and $T_e\mfs_i$.  Let $A$ and $B$ belongs to $T_{\phi^{-1}(g)}(\mathfrak{s}_1\oplus\mathfrak{s}_2)$. We have to show that the quantity
\begin{equation}\label{eq:omega_g_et_omega_e}
\begin{split}
\omega_g\Big(   (d\phi)_{\phi^{-1}(g)}A&,(d\phi)_{\phi^{-1}(g)}B      \Big)\\
                                &=\omega_e\Big(  (dL_{g^{-1}})_g (d\phi)_{\phi^{-1}(g)}A,(dL_{g^{-1}})_g(d\phi)_{\phi^{-1}(g)}B      \Big)
\end{split}  
\end{equation}
does not depend on $g$.

The vector $A$ is represented by a path $A(t)=( A_1(t),A_2(t) )$ with $A_i(t)\in\mfs_i$. In order to characterise that path, we want first to know precisely what is $A_i(0)$. Since $A\in T_{\phi^{-1}(g)}\mfs$, the path must fulfil $\phi( A_1(0),A_2(0) )=g$, or
\begin{equation}
  \phi_2(A_2(0))\phi_1(A_1(0))=g.
\end{equation}
We denote $A_i(0)=G_i\in\mfs_i$ and $\phi_i(G_i)=g_i$. The relation between $g_1$ and $g_2$ is $g_2g_1=g$.  In particular, it is wrong to say ``$A_1(0)=\phi^{-1}(g)$, thus $\phi_1(A_1(0))=g$''. This point being clear, 
\begin{equation}
  (d\phi)_{\phi^{-1}(g)}A=\Dsdd{\phi(A(t))}{t}{0}=\Dsdd{ \phi_2(A_2(t))\phi_1(A_1(t)) }{t}{0}.
\end{equation}
If one particularises to the case $A\in\mfs_2$, that is $A_1(t)=cst=G_1$,
\begin{equation}
  (d\phi)_{\phi^{-1}(g)}A=(dR_{g_1})_{g_2}(d\phi_2)_{G_2}A_2.
\end{equation}

Since $g=g_1g_2$, we have $L_{g^{-1}}=L_{g_1^{-1}}\circ L_{g_2^{-1}}$, and the first argument of $\omega_e$ in equation \eqref{eq:omega_g_et_omega_e} is
\[
   (dL_{g_1^{-1}})_{g_1}(dL_{g_2^{-1}})_{g_2g_1}(dR_{g_1})_{g_2}(d\phi_2)_{G_2}A_2.
\]
If we write that in terms of the derivative of the path $A_2(t)$, what we get in the derivative is
\begin{equation}
g_1^{-1} g_2^{-1} \phi_2(A_2(t))g_1=\AD_{g_1^{-1}}\Big( g_2^{-1}\phi_2(A_2(t)) \Big).
\end{equation}
Since $g^{-1}_2\phi\big( A_2(0) \big)=g^{-1}_2\phi_2(G_2)=e$, the derivative of that term is
\begin{equation}
  (dL_{g^{-1}})_g(d\phi)_{\phi^{-1}(g)}A=\Ad_{g_1^{-1}}\Big(  (dL_{g_2^{-1}})_{g_2}(d\phi_2)_{G_2}A  \Big)
\end{equation}
with some abuse between $A\in\mfs$ and $A_2\in\mfs_2$.  Doing the same computation with $B\in\mfs_1$ (so that $B_2(t)=cst=G_2$), we find
\begin{equation}
   (d\phi)_{\phi^{-1}(g)}B=\Dsdd{ g_2\phi_1(B_1(t)) }{t}{0}
                         =(dL_{g_2})_{g_1}(d\phi_1)_{G_1}B_1,
\end{equation}
 and what appears in $\omega_e$ reads 
\begin{equation}
  (dL_{g^{-1}})_g(dL_{g_2})_{g_1}(d\phi_1)_{G_1}B=(dL_{g_1^{-1}})_{g_1}(d\phi_1)_{G_1}B.
\end{equation}
Finally, for $A\in\mfs_2$ and $B\in\mfs_1$,
\begin{equation}\label{eq:gros_omega_e}
\begin{split}
\omega_g\big( (d\phi)_{\phi^{-1}(g)}A&,(d\phi)_{\phi^{-1}(g)}B \big)\\
                                    &= \omega_e\Big(
       \Ad_{g_1^{-1}}\big[  (dL_{g_2^{-1}})_{g_2}(d\phi_2)_{G_2}A  \big],
       (dL_{g_1^{-1}})_{g_1}(d\phi_1)_{G_1}B
         \Big). 
\end{split} 
\end{equation}
The first argument belongs to $T\mfs_2$ (because $g_2\in\mfs_2$) while the second belongs to $T\mfs_1$. Hence definition \eqref{eq:Omega} makes the right hand side vanishing.

If we want to compute equation \eqref{eq:gros_omega_e} with $A$, $B\in\mfs_2$,
\begin{equation}
\begin{split}
\omega_e\Big(  
           \Ad_{g_1^{-1}}\big[&  (dL_{g_2^{-1}})_{g_2}(d\phi_2)_{G_2}A  \big],
	   \Ad_{g_1^{-1}}\big[  (dL_{g_2^{-1}})_{g_2}(d\phi_2)_{G_2}B  \big]   
        \Big)\\
  &=\Omega(\ldots)=\underbrace{\Omega_1(\ldots)}_{=0}+\Omega_2(\ldots)\\
  &=\Big( \Ad_{g_1^{-1}}^*\Omega_2  \Big)
     \Big(
        (dL_{g_2^{-1}})_{g_2}(d\phi_2)_{G_2}A,\ldots B
     \Big)
\end{split}
\end{equation}
At this point, notice that $\Ad^*_{g_1}\Omega_2=\Omega_2$. Indeed the exponential $\exp\colon \mathfrak{s}_1\to S_1$ being surjective, there exists a $X_1\in\mathfrak{s}_1$ such that $\Ad(g_1)= e^{\ad(X_1)}$. Now, $\ad(X_1)\in\gsp(\Omega_2)$ by assumption, so that $\Ad(g_1)\in\SP(\Omega_2)$. The previous expression becomes
\begin{equation}
\begin{split}
  \Omega_2( (dL_{g_2^{-1}})_{g_2}(d\phi_2)_{G_2}A,\ldots B )
  &=(\omega_2)_{g_2}( (d\phi_2)_{g_2}A,\ldots B )\\
  &=(\phi_2^*\omega_2)_{G_2}(A,B)\\
  &=\Omega_2(A,B).
\end{split}  
\end{equation}
The last line is the fact that $\phi_2$ is a Darboux chart: $ \phi_2^*\omega_2=\Omega_2$. The case with $A$, $B\in\mfs_1$ yields to compute
\[
   \omega_e\Big(
                   (dL_{g_1^{-1}})_{g_1}(d\phi_1)_{G_1}A,  (dL_{g_1^{-1}})_{g_1}(d\phi_1)_{G_1}B
           \Big).
\]
It is done by the same way as the previous cases.
\end{proof}

A direct computation shows the following \defe{extension lemma}{extension!lemma}.
\begin{lemma}[Extension lemma]
Let $K_i\in \Fun(S_i^3) $ be a left invariant three point kernel on $S_i$ ($i=1,2$).  Assume that $K_2\otimes1\in \Fun(S^3)$ is invariant under conjugation by elements of $S_1$.  Then $K:=K_1\otimes K_2\in\Fun(S^3)$ is left invariant (under $S$).
\label{EXT}
\end{lemma}
 
\begin{proof}
An element of $S$ has the form $g_1g_2$ with $g_i\in S_i$, the multiplication being given by $(g_1g_2)(a_1a_2)=(g_1a_1)(g_2a_2)$. Using this rule, the definition of the tensor product, and the left invariance of both $K_i$,
\[ 
\begin{split}
\big( L_{g_1g_2}(K_1\otimes K_2) \big)&(a_1a_2,b_1b_2,c_1c_2)\\
			&=(K_1\otimes K_2)\big( (g_1g_2)(a_1a_2),(g_1g_2)(b_1b_2),(g_1,g_2)(c_1c_2) \big)\\
			&=K_1(g_1a_1,g_1b_1,g_1c_1)K_2(g_2a_2,g_2b_2,g_2c_2)\\
			&=K_1(a_1,b_1,c_1)K_2(a_2,b_2,c_2)\\
			&=(K_1\otimes K_2)(a_1a_2,b_1b_2,c_1c_2).
\end{split}  
\]

\end{proof}

This lemma shows that if one has kernels on $S_1$ and $S_2$ satisfying the above hypotheses, their tensor product provides a kernel for an associative left invariant kernel on $S=S_1\otimes_{\rho} S_2$.  Proposition \ref{prop:Darboux} allows us to hope that the product on $S$ will satisfy the same kind of symplectic compatibility as the products on $S_i$; in particular when the latter were constructed using Darboux chart in the same way as the product described in section  \ref{sec:unifsl}.

\section{Deformation by action of group}  	\label{SecDefAction}		%\label{sec:framedef}
%----------------------------------------

The procedure of deformation by group action is described in \cite{TrsStProd}. Let $G$ be a Lie group. We suppose to know a subset $A^G$ of $\Fun(G,\eC)$  such that
\begin{enumerate}
\item $A^G$ is invariant under the left regular representation of $G$ on itself,
\item $A^G$ is provided with a $G$-invariant product $\stG$\nomenclature{$\stG$}{Star product on $G$} such that $(A^G,\stG)$ is an associative algebra. The $G$-invariance means that $\forall a,b\in A^G$,
\[
    (L^*_ga)\stG(L^*_gb)=L^*_g(a\stG b).
\]
\end{enumerate}

Notice that we do not impose any regularity condition on this product. The reason is that the deformation by group action is a formal procedure which allows to guess a product on a manifold. The ``true'' work to prove convergences and invariances has to be done on the level of the deformed manifold.

Now, let $X$ be a manifold endowed with a right action  $\dpt{\tau}{G\times X}{X}$ of $G$.  For $u\in\Fun(X)$, $x\in X$ and $g\in G$, we consider $\alpha^x(u)\in\Fun(G)$ and $\alpha_g(u)\in\Fun(X)$ defined by
\begin{equation}		\label{EqDefalphaxu}
   \alpha^x(u)(g)=\alpha_g(u)(x)=u(\tau_{g^{-1}}(x)),
\end{equation}
and the following functional space on $X$:
\[
   A^X=\{u\in\Fun(X)|\alpha^x(u)\in A^G\,\forall x\in X\}.
\]
For example, the $A^X$ corresponding to $A^G=\Fun(G)$ is the whole $\Fun(X)$.  For $u$, $v\in A^X$, we define $\dpt{\stX}{A^X\times A^X}{\Fun(X)}$\nomenclature{$\stX$}{Star product on $X$}
\begin{equation}\label{eq:def_stG}
   (u\stX v)(x)=\big(\alpha^x(u)\stG \alpha^x(v)\big)(e)
\end{equation}
where $e$ is the identity of $G$.

%\label{lem:dist_alpha}

\begin{theorem}
The product \eqref{eq:def_stG} obtained by action of the group $G$ on the manifold $X$ fulfils the following properties:
\begin{enumerate}
\item\label{itemthostG} The operation $\alpha^x$ intertwines the products $\star^X$ and $\star^G$:
\[
   \alpha^x(u\stX v)=(\alpha^xu)\stG(\alpha^xv).
\]

\item $A^X$ is stable under $\stX$,
\item $(A^X,\stX)$ is an associative algebra.
\end{enumerate}
\end{theorem}

% Attention : dans le texte, je dit des ``first'', ``second'' et ``third'' point. Donc en cas de changement de numérotation, je dois adapter le texte.

\begin{proof}
First remark that $\alpha^{\tau_{g^{-1}}(x)}u=L^*_g\alpha^xu$ because
\[ 
(\alpha^{\tau_{g^{-1}}(x)}u)(h)=u\big(\tau_{(gh)^{-1}}(x)\big)=(\alpha^xu)(gh)
=\big(L^*_g\alpha^xu\big)(h),
\]
It follows that
\begin{equation}
\begin{split}
\alpha^x(u\stX v)(g)&=(u\stX v)(\tau_{g^{-1}}(x))
                    =(\alpha^{\tau_{g^{-1}}(x)}u\stG\alpha^{\tau_{g^{-1}}(x)}v) (e)\\
		    &=\left[ L^*_g(\alpha^xu\stG\alpha^xv) \right](e)
		    =(\alpha^xu\stG\alpha^xv)(g).
\end{split}
\end{equation}
The first point is proved.

Using the first point, we see that $\alpha^{\tau_{g^{-1}}(x)}u$ belongs to $A^G$ because $\alpha^xu\in G^G$ and $A^G$ is stable under $L_g$. So we have the second point.  For the third one,
\[ 
\begin{split}
[(u\stX v)\stX w](x)&=\big(\alpha^x(u\stX v)\stG\alpha^x(w)\big)(e)\\
                    &=\big((\alpha^xu\stG\alpha^xv)\stG\alpha^x(w)\big)(e).
\end{split}
\]
The conclusion follows from associativity of $\star^G$.
\end{proof}

Let us summarize what was done up to now. When $G$ acts on $X$, and when we have a ``good'' product on $A^G\subset\Fun(G)$, we are able to build an associative product on $A^X\subset\Fun(X)$. The space $A^X$ is defined by $A$ and the action. So a deformation of a group gives rise to a deformation of any manifold on which the group acts. This is why we call it an ``universal``\ deformation. That universal construction is the motivation to deform groups.

\begin{lemma}   
A function $u$ belongs to $A^{X}$ if and only if there exists one $y$ such that $\alpha^y(u)\in A^G$ in each $g$-orbit in $X$.
		\label{LemUnPtParOrbite}
\end{lemma}

\begin{proof}
The necessary condition is direct because, when $u\in A^X$, the function $\alpha^x(u)$ belongs to $A^G$ for every $x$. For the sufficient condition, suppose $\alpha^y(u)\in A^{G}$, then $\alpha^{g\cdot y}(u)=L_{g^{-1}}^*(\alpha^yu)\in A^{G}$ for all $g$ because $A^{G}$ is left invariant. If it holds for a $y$ in each $G$-orbit, then $\alpha^xu\in A^{G}$ for all $x\in X$.
\end{proof}

The content of this lemma is that if one wants to check if a given function $u$ belongs to $A^{X}$, one only has to check is $\alpha^yu\in A^{G}$ for one $y$ in each $G$-orbit.


The functions $\alpha^x(u)$ are not ``gentle'' functions, even when $u$ is. Let us give two examples of pathology that can occur in $\alpha^x(u)$ without to be present in $u$. Firstly,  if the action is the identity, the support of $\alpha^x(u)$ is the whole $G$ which can be non compact. So, even when $u$ is compactly supported, there are no guarantee with respect to the support of $\alpha^x(u)$. 

Secondly, the function $\alpha^x(u)$ is of course bounded; but the derivatives are not specially such. Indeed, in order to fix ideas, suppose that the group $G$ is a two parameter group and that the manifold $X$ is a two dimensional manifold. In this case, one can write
\begin{equation}		\label{EqDefziDefA}
  f(a,l)=\alpha^x(u)(a,l)=u\big( z_1(a,l),z_2(a,l) \big)
\end{equation}
where $x$ is a parameter in the functions $z_i$. Depending on the action, the function $z$ can be very odd. In particular, the derivatives
\[ 
  (\partial_af)(a,l)=(\partial_1u)(z_1,z_2)(\partial_az_1)(a,l)+(\partial_2u)(z_1,z_2)(\partial_az_2)(a,l)
\]
in which $\partial_az_i$ can be divergent. Even worse, the degree of the divergence can increase with the degree of the derivation. Two examples of such a hill behaviour are given in section \ref{SecEplolUnter}. 

\section{Deformation of \texorpdfstring{$\SU(1,n)$}{SU1n}}   \label{SecDefSURme}
%+++++++++++++++++++++++++++++++++++++++++++++++++++++++++

Before going on with the construction of a deformation of one dimensional split extensions of Heisenberg algebras, we have to recall a result on deformation in $\SU(1,n)$. The product on the extension of Heisenberg algebra will be nothing else than a transport of this one.

The article \cite{Biel-Massar} provides a formal universal deformation formula for the actions of the Iwasawa component $\SUR_0:=\SUA_0\SUN_0$ of $\SU(1,n)$ under an oscillatory integral form.  It turns out (see \cite{lcBBM}) that this deformation formula is in fact non-formal for proper actions on topological spaces. 

Here is the precise result. The Iwasawa decomposition of $\SU(1,n)$ induces the identification $\SUR_0=\SU(1,n)/U(n)$. The group $\SUR_0$ is endowed with a (family of) left invariant symplectic structure(s)\footnote{This is done using the hermitian symmetric structure, cf proposition 1.1 in \cite{Biel-Massar}.} $\omega$.  If we denote by $\sR_0=\sA_0\oplus\sN_0$ the Lie algebra of $\SUR_0$, the map
\begin{equation}  \label{DARBOUX}
\begin{aligned}
 \phi_0\colon \sR_0&\to \SUR_0 \\ 
(a,n)&\mapsto \exp(a)\exp(n) 
\end{aligned}
\end{equation}
reveals to be a global Darboux chart for $(\SUR_0,\omega)$.  The nilpotent component appears to accept a decomposition $\sN_0=V\times\eR Z$ in which the Lie bracket reads
\[ 
[(x,z)\,,\,(x',z')]=\Omega_V(x,x')\,Z; 
\]
the full Iwasawa component is now parametrized by $\sR_0=\{(a,v,z)\,|\,,a,z\in\eR;x\in V\}$. The interest of this situation resides in the fact that the algebra $\sR_0$ turns out to be a one dimensional split extension of an Heisenberg algebra; namely, 
\[ 
\sR_0=\mF(\mtu,0,2).
\]
The deformation result is the following.

\begin{theorem}
For every non-zero $\theta\in\eR$, there exists a Fréchet function space $\swE_\theta$ satisfying the inclusions $C^\infty_c(\SUR_0)\subset\swE_\theta\subset C^\infty(\SUR_0)$, such that, defining for all $u,v\in C^\infty_c(\SUR_0)$  
\begin{equation}  \label{PRODUCT}
\begin{split}
(u\star_\theta v)(a_0,x_0,z_0)
		:=\frac{1}{\theta^{\dim \SUR_0}} \int_{ \SUR_0\times \SUR_0}& \cosh(2(a_1-a_2))\\
		&[\cosh(a_2- a_0)\cosh(a_0-a_1)\,]^{\dim \SUR_0-2}\\
&\times \exp \Big( \frac{2i}{\theta}\varphi(r_0,r_1,r_2)\Big)\\
		&\times u(a_1,x_1,z_1)\,v(a_2,x_2,z_2)\, da\,dx\,dz;
\end{split}
\end{equation}
where 
\[ 
\begin{split}
  \varphi(r_0,r_1,r_2)=&S_V\big(\cosh(a_1-a_2)x_0, \cosh(a_2-a_0)x_1, \cosh(a_0-a_1)x_2\big)\\
			&-\bigoplus_{0,1,2}\sinh(2(a_0-a_1))z_2 
\end{split}
\]
with $S_V(x_0,x_1,x_2):=\Omega_V(x_0,x_1)+\Omega_V(x_1,x_2)+\Omega_V(x_2,x_0)$ is the phase for the Weyl product on $C^\infty_c(V)$ and $\bigoplus_{0,1,2}$ stands for cyclic summation, one has: 

\begin{enumerate}

\item\label{tBMi} 
	 $u\star_\theta v$ is smooth and the map $ C^\infty_c(\SUR_0)\times C^\infty_c(\SUR_0) \to C^\infty(\SUR_0)$ extends to an associative product on $\swE_\theta$. The pair $(\swE_\theta,\star_\theta)$ is a (pre-$C^\star$) Fréchet algebra.

\item\label{tBMii}
	 In coordinates $(a,x,z)$ the group multiplication law reads
\[ 
	L_{(a,x,z)}(a',x',z')=\left(a+a',e^{-a'}x+x',e^{-2a'}z+z'+\frac{1}{2}\Omega_V(x,x')e^{-a'}
\right).
\]
The phase and amplitude occurring in formula \eqref{PRODUCT} are both invariant under the left action $L:\SUR_0\times \SUR_0\to \SUR_0$.

\item\label{tBMiii}
	 Formula \eqref{PRODUCT} admits a formal asymptotic expansion of the form:
 \begin{equation*}
	u\star_\theta v\sim \,uv\,+\,\frac{\theta}{2i}\{u,v\}\,+O(\theta^2); 
\end{equation*}
where $\{\,,\,\}$ denotes the symplectic Poisson bracket on $C^\infty(\SUR_0)$ associated with $\omega$.  The full series yields an associative formal star product on $(\SUR_0,\omega)$ denoted by $\tilde{\star}_\theta$. 
 \end{enumerate}
\label{ThoDefHeizsansB}
\end{theorem}

The setting and \ref{tBMi} and \ref{tBMii} may be found in \cite{Biel-Massar}, while \ref{tBMiii} is a straightforward adaptation  to $\SUR_0$ of \cite{lcBBM}.

This theorem together with the isomorphisms given in \ref{SubsecIsomsdX} only provide a product on extensions of type $(d\mtu,0,2d)$. But we saw that the extensions $(\BX ,0,2d)$ with $\BX \neq\mtu$ are different. Hence the generalization of this result to other extensions is not straightforward. We address now this question.

%%%%%%%%%%%%%%%%%%%%%%%%%%
%
   \section[Split extensions of Heisenberg algebras]{One dimensional split extensions of Heisenberg algebras} \label{SecExtHeiz}
%
%%%%%%%%%%%%%%%%%%%%%%%%

\subsection{Introduction}
%------------------------

The one dimensional extensions of Heisenberg algebras are classified by triples $(\BX,\mu,d)$. The quantization in the case $(\id,0,\mu)$  reveals to be a particular case of the one studied in \cite{Biel-Massar} (see also section \ref{SecDefSURme}), while quantization of other extensions can be found using symmetries of the kernel. Here we are reporting results of \cite{articleBVCS} and most of proofs (in particular the trick of subsection \ref{subsecTrick} which allows to extend the known product to every one dimensional split extensions) are due to Y. Voglaire. It is to be published in his future PhD thesis.

The kernel of the quantization of \cite{Biel-Massar} will be denoted by $K$.  Then we will give a way to twist $K$ in order to obtain a kernel $K'$ on any extension of the form $(\BX,0,2)$. Quantizations of other extensions can be obtained by composing with Lie group isomorphisms. The kernel for an arbitrary extension is denoted by $K_{0}(\BX,\mu,d)$, or simply $K_{0}$ when there are no possible ambiguity.

When we will deal with the anti de Sitter situation, our starting point will be this $K_{0}$ that we will have to adapt to another symplectic form that $\delta E^*$ invoking lemma \ref{LemJumpCoadOrb}.


\subsection{General definitions}
%--------------------------------

Let $\pH_{n}=V\oplus\eR E$ be the \hyperlink{HyperHeisenberg}{Heisenberg algebra} of dimension $2n+1$, with a natural symplectic structure defined from the Heisenberg algebra structure:
 \[
[v,w]=\Omega(v,w)E
\]
for all $v$, $w\in V$. Now we consider a one dimensional algebra $\mA=\eR A$ generated by an element $A$, and we build the split extension of $\pH_{n}$ by $\mA$:
\begin{equation}
\mF(\rho)=\mA\oplus_{\rho}\pH_{n}
\end{equation}
where the split homomorphism is an action by derivation $\rho\colon \mA\to \Der(\pH_{n})$. The so obtained algebra is what we call a \defe{one dimensional extension of Heisenberg algebra}{extension!of Heisenberg algebra}. Let us study the possibilities for $\rho(A)$. From linearity, its general form is 
\[ 
 \rho(A)(v,z)=\rho(A)(v,0)+\rho(A)(0,z)
	=(\BX v,\mu(v))+(zv_{0},2dz)
\]
with $\BX \in\End(V)$, $\mu\in V^*$, $v_{0}\in V$ and $d\in\eR$. Since $\eR E=[\pH_{n},\pH_{n}]$, the fact that $\rho(A)$ is a derivation of $\pH_{n}$, implies that $v_{0}=0$ because
\begin{equation}
   \rho(A)\eR E=\rho(A)[\pH_{n},\pH_{n}]
		=[\rho(A)\pH_{n},\pH_{n}]+[\pH_{n},\rho(A)\pH_{n}]\subset \eR E.
\end{equation}
Thus we have
\begin{equation}    \label{EqrhoBmudz}
\rho(A)(v,z)=(\BX v,\mu(v)+2dz).
\end{equation}
\ifthenelse{\value{siART}=1}{}{
All that is summarized on the short exact sequence
\[
	\xymatrix{ 0\ar[r]&\pH_{n}\ar[r]^{i} & \mR \ar@<2pt>[r]^{r} & \mA \ar@<2pt>[l]^{i}\ar[r]&0 }
\]
where the two $i$ are injections and $r$ is the projection. Of course $r\circ i=\id$.}
From commutation relations in $\pH_{n}$, we easily find 
\[ 
  [(v,z),(v',z')]=[v,v']=\Omega(v,v')E.
\]
Applying $\rho(A)$ to this equality, and using the fact that this is a derivation, we find
\[ 
  \Omega(\BX ,v')E+\Omega(v,\BX ')E=\rho(A)\Omega(v,v')E=2d\Omega(v,v')E
\]
which can be rewritten as
\begin{equation}
\Omega\big( (\BX -d\,\mtu)v,v' \big)+\Omega\big( v,(\BX -d\,\mtu)v' \big)=0.
\end{equation}
In conclusion, the endomorphism $\rho(A)$ is given by a triple $(\BX ,\mu,2d)$ with $(\BX -d\,\mtu)\in\gsp(V,\Omega)$, $\mu\in V^*$ and $d\in\eR$. Using this result, we write the general commutator on $\mR=\mA\oplus_{\rho}\pH_{n}$ under the form
\begin{equation}  \label{EqGeneExtHeizCom}
\big[ (a,v,z),(a',v',z') \big]=\big( 0,\BX (av'-a'v),\mu(av'-a'v)+2d(az'-a'z)+\Omega(v,v') \big)
\end{equation}
where we adopted the notation
\begin{equation} 
(a,v,z)=aA+v+zE.
\end{equation}

\subsection{Symplectic structure}
%---------------------------------

The following proposition gives a symplectic structure on $\mF$.

\begin{proposition}   
The algebra $(\BX,\mu,d)$ endowed with
\begin{equation}
 \Omega^{\mF}=-\delta E^*=E^*([.,.])
 \end{equation}
where the star denotes the Chevalley cocycle defined by \eqref{EqDefChevCoycl} is symplectic if and only if $d\neq 0$.
\label{PropSymplestarEG}
\end{proposition}

\begin{proof}
It is evident that $\Omega^{\mF}$ is closed because it is exact. For non-degeneracy, we compute
\[ 
\begin{split}
\Omega^{\mF}&=E^*[.,.]=a\mu(v')-a'\mu(v)+2d(az'-a'z)+\Omega(v,v')\\
	&=\begin{pmatrix}
0	&	\mu	&	2d\\
-\mu^{t}&	\Omega	&	0\\
-2d	&	0	&	0	
\end{pmatrix}
\end{split}  
\]
whose determinant is $\det\Omega^{\mF}=-4d^{2}\det\Omega$ which is non vanishing if and only if $d\neq 0$.

\end{proof}

This symplectic algebra  is denoted by $\mF_{\Omega}(\BX ,\mu,d)$, or simply $\mF$ when there are no possible confusions.

Since we are only interested in symplectic algebras, we suppose $d\neq 0$ and we look at extensions of type $(d\BX ,d\mu,2d)$ with $\BX -d\mtu\in\gsp(V,\Omega)$. The bracket is given by
\begin{equation}   \label{EqCommGeneF}
\big[ (a,v,z),(a',v',z') \big]=\big( 0,d\BX (a'v-a'v),d\mu(av'-a'v)+2d(az'-a'z)+\Omega(v,v') \big).
\end{equation}

\ifthenelse{\value{siTHZ}=1}{}{
Then we consider $F$, the corresponding group and the left invariant symplectic form
\[ 
  \omega_{g}(X_{g},Y_{g})=\Omega(dL_{g^{-1}}X_{g},dL_{g^{-1}}Y_{g}).
\]
By construction, $L_{g}^*\omega=\omega$. One can prove that the Iwasawa coordinates
\begin{equation}
\begin{aligned}
 I\colon \mF&\to F \\ 
(a,n)&\mapsto  e^{aA} e^{n} 
\end{aligned}
\end{equation}
is not a Darboux chart, but one can find a twist.

\begin{proposition}
The chart \begin{equation}
\begin{aligned}
 \tilde I\colon \mF&\to F \\ 
\tilde I(a,v,z)&=  e^{aA} e^{v} e^{\big(z+\frac{1}{ 4 }\Omega(v\BX )\big)E}
\end{aligned}
\end{equation}
is Darboux for the symplectic group $(F,\omega)$. In other words,
\[ 
  \tilde I^*\omega=\Omega.
\]
\end{proposition}
\begin{proof}
No proof.
\begin{problemeT}
La preuve de ceci n'est pas donnée dans l'article BVCS. Tu devras sans doute la recopier du mémoire de Yan.
\label{ProbMemYan}
\end{problemeT}

\end{proof}

This chart provides an action of $F$ on $\mF$ by
\[ 
  \tau_{g}(X)=(\tilde I^{-1}\circ L_{g}\circ \tilde I)(X)
\]
for $X\in\mF$, $g\in F$.\begin{proposition}
The action $\tau$ is symplectic and strongly hamiltonian. The momentum maps are
\begin{subequations}\label{EqAppMomHamActSPHm}
\begin{align}  
  \lambda_{A}(X)&=d\mu(v)+2dz,\\
\lambda_{w}(X)&=\mu\left( \frac{  e^{-2da}- e^{da\BX } }{ d(2-\BX ) }w \right)-\Omega(v, e^{-da\BX }w),\\
\lambda_{E}(X)&= e^{-2da}.
\end{align}
\end{subequations}
\end{proposition}

\begin{proof}
\ifthenelse{\value{siTHZ}=1}{No proof.}{ The action $\tau$ is symplectic because $I^*\omega=\Omega$ and $L_{g}^*\omega=\omega$.}
\end{proof}

\begin{proposition}
The Moyal product on $(\mF,\Omega^{\mF})$ is $\mF$-covariant.
\end{proposition}

\begin{proof}
\ifthenelse{\(\value{siART}=1\)\OR \(\value{siTHZ}=1\)}{No proof.}{%
Using the expansion \eqref{EqDevStatrMoyal} of the Moyal product and the explicit form \eqref{EqAppMomHamActSPHm} of the momentum maps, we remark that the higher order terms in $\lambda_{X}\ast_{M}\lambda_{Y}$ are vanishing.
}
\end{proof}
}	% Fin du siTHZ sur des trucs inutiles qui servent à quantifier les extensions avec (1,0,2). Elles sont en fait toute cuites par le théorème de SU. 

\subsection{Isomorphisms}  \label{SubsecIsomsdX}
%------------------------

The extension obtained by the derivation $D=(\BX ,\mu,d)$ is \emph{a priori} not the same as the one obtained by $D'=(\BX',\mu',d')$. Two extensions are isomorphic when there exists a linear bijection $dL\colon \mF_{D}\to \mF_{D'}$ such that\footnote{the reason why we write $dL$ instead of $L$ comes from the fact that we will be interested in the corresponding group isomorphism later.}
\begin{subequations}
\begin{align}
dL\big( [X,Y]_{D'} \big)&=\big[ dL(X),dL(Y) \big]_{D'}\\
(dL)^*\Omega^{D'}&=\Omega^{D}.
\end{align} 
\end{subequations}
We find the following isomorphisms:
\begin{subequations}   \label{SubEqsIsommud}
\begin{itemize}
\item $\mF(d\BX ,d\mu,2d)\simeq \mF(\BX ,\mu,d)$ by
\begin{equation}
   dL(a,v,z)=(da,v,z),
\end{equation}
\item $\mF(\BX ,\mu,2)\simeq\mF(\BX ,0,2) $ by
\begin{equation}
   dL(a,v,z)=(a,v+au,z),
\end{equation}
where $u$ is the vector of $V$ satisfying $i(u)\Omega=\mu$,
\item $\mF( \BX ,0,2)\simeq\mF(\BX ',0,2)$ by
\begin{equation}
   dL(a,v,z)=(a,M(v),z)
\end{equation}
where $M\in\SP(V,\Omega)$ fulfills $M\BX M^{-1}=\BX'$ or, equivalently,
\[
M(\BX-\mtu)M^{-1}=\BX'-\mtu.
\]
\end{itemize}
\end{subequations}
The third isomorphism only gives the equivalence between $\BX-\mtu$ and $\BX'-\mtu$ when they belongs to the same orbit of the adjoint action of $\SP(V,\Omega)$. In particular, there are no isomorphisms between the identity and anything else.

 Let us prove the second isomorphism. If $D=(\BX ,\mu,2)$, $D'=(\BX ,0,2)$, $Y=(a,v,z)\in\mF$ and $Y'=(a',v',z')\in \mF$, we have
\begin{equation}   \label{EqPrIsoDDtilde}
\begin{split}
   dL\big(  [Y,Y']_{D}  \big)&=dL\big(   0,\BX (av'-a'v),\mu(av'-a'v)+2(az'-a'z)+\Omega(v,v')     \big)\\
		&=\big(   0,\BX (av'-a'v),\mu(av'-a'v) +2(az'-a'z)+\Omega(v,v')  \big),
\end{split}
\end{equation}
while
\[
  \big[ dL(Y),dL(Y')   \big]_{D'}=\Big(   0,\BX \big(  a(v'+a'u)-a'(v+au)   \big),2(az'+a'z)+\Omega(v+au,v'+a'u)    \Big),
\]
but
\[
   \Omega(v+au,v'+a'u)=\Omega(v,v')+\mu(av'-a'v),
\]
thus we find the same as in equations \eqref{EqPrIsoDDtilde}.

\subsection{Extensions with non trivial \texorpdfstring{$\protect\BX $}{X}}	\label{subsecTrick}
%-------------------------------------------------------------------------

The group $F(\mtu,0,2)$ is provided with a kernel $K\colon F\times F\times F\to \eC$ by theorem \ref{ThoDefHeizsansB}.  The symplectic group $\SP(V,\Omega)$ acts on $F$ by
\begin{equation}
\begin{aligned}
 \Phi\colon \SP(V,\Omega)\times F&\to F \\ 
(M,I(a,v,z))&\mapsto \Phi_{M}(I(a,v,z)):=I(a,M(v),z)
\end{aligned}
\end{equation}
where 
\begin{equation}
\begin{aligned}
 I\colon \mF&\to F \\ 
(a,n)&\mapsto  e^{aA} e^{n} 
\end{aligned}
\end{equation}
is the Iwasawa coordinate on $F$.

\begin{proposition}  
The kernel $K$ is invariant under this action: $\Phi^*_{M}K=K$.
 \label{PropkernelinvarSp}
\end{proposition}

\begin{proof}
We are looking on the kernel in expression \eqref{PRODUCT}. The amplitude of $K$, i.e. all what lies outside the exponential, and the cyclic sum in the phase only depend on the $a_{i}$'s. So $\Phi_{M}$ does not act on them. As far as $S_{0}$ is concerned, up to coefficients which only depend on the $a_{i}$'s, it is a sum of elements of the form $\Omega(Mv_{i},Mv_{j})=\Omega(v_{i},v_{j})$.
\end{proof}

Let $\BX $ be a matrix such that $\bar{\BX }=\BX -\mtu\in\gsp(V,\Omega)$ and $\mF'=\mF'(\BX,0,2)$. We consider $\mS$, the one dimensional subalgebra of $\gsp(V,\Omega)$ generated by $\bar{\BX}$ and we define
\begin{equation}
  \mG=\mS\oplus_{\rho}\mF
\end{equation} 
with 
\[ 
  \rho(\bar{\BX })(a,v,z)=(0,\bar{\BX }v,0).
\]
We denote by $G$ and $S$ the corresponding groups. We have in particular $F\simeq G/S$. An element of $\mG$ has the form
\begin{equation}	\label{EqDefkavzG}
  (k\bar\BX,a,v,z)=k\bar\BX+aA+v+zE.
\end{equation}


\begin{proposition}
The group $F'$ is a subgroup of $G$.
\end{proposition}
\begin{proof}
We will prove that $\mF'$ is isomorphic to a subalgebra of $\mG$, namely, the subalgebra $\mL\subset \mS\oplus_{\rho}\mF$,
\[ 
  \mL=\eR(A+\bar\BX)\oplus_{\sigma}(V+\eR E)
\]
where $\sigma$ is the splitting homomorphism \eqref{EqrhoBmudz} of $\mF$, which in the present case reads $\sigma(A+\bar\BX)(0,v,z)=(0,Xv,2z)$. In other words, the algebra $\mL$ is made of elements of the form \eqref{EqDefkavzG} with $k=a$.  The isomorphism is 
\begin{equation}
\begin{aligned}
\phi  \colon \mL&\to \mF'(\BX,0,2) \\ 
a(A+\bar\BX)+v+zE&\mapsto aA+v+zE. 
\end{aligned}
\end{equation}
Indeed, using formula \eqref{EqCommGeneF} with $d=1$ and $\mu=0$, we find
\[ 
\begin{split}
\Big[ \phi\big( a(A+\bar{\BX })+v+zE \big)&,\phi\big( a'(A+\bar{\BX })+v'+z'E \big) \Big]_{(\BX,0,2)}\\
		&=\big[ aA+v+zE,a'A,v'+z'E \big]\\
	 	&=\BX (av'-a'v)+\big(2(az'-a'z)+\Omega(v,v')\big)E\\
	&=\phi\big( X(av'-a'v),2(az'-a'z+\Omega(v,v'))  \big)\\
&=\phi\big[ a(A+\bar{\BX })+v+zE,a'(A+\bar{\BX })+v'+z'E \big].
\end{split}  
\]
\end{proof}

From now on, we identify $\mF'$ with $\mL$ by the isomorphism $\phi$ which will no longer be explicitly written.  Image of $F'$ in $G$ by the isomorphism are elements of the form
\[ 
  g'= e^{a(A+\bar{\BX })} e^{v+zE}.
\]
Since the elements $ e^{A}$ and $ e^{\BX }$ commute in $G$, we can decompose an element $\phi^{-1}(g')$ as
\[ 
  \underbrace{e^{a\bar{\BX }}}_{\in S}\underbrace{e^{aA} e^{v+zE}}_{\in F}.
\]
The element $a(A+\bar{\BX })+v+zE$ seen in $\mS\oplus_{\rho}\mF$ will be denoted by $(a,v,z)$ as well
\[ 
  (a,v,z)=\phi^{-1}(aA+v+zE).
\]
We consider the following coordinate on $F'$:
\begin{equation} 
\begin{aligned}
 J\colon \mF'&\to F' \\ 
(a,v,z)&\mapsto  e^{a(A+\bar{\BX })} e^{v+zE}.
\end{aligned}
\end{equation}

\begin{proposition}
The group $F'$ is diffeomorphic to the homogeneous space $F\simeq G/S$.
\end{proposition}

\begin{proof}
We will prove that $F'$ acts simply transitively on $G/S$. Let us look at 
\begin{equation}		\label{EqgpJSF}
  g'=J(a,v,z)= \underbrace{e^{a(A+\bar{\BX })}}_{g'_{S}} \underbrace{e^{v+zE}}_{g'_{F}}.
\end{equation}
Noticing that $ e^{a\bar{\BX }}[e]=[ e^{a\bar{\BX }}]=[e]$ we find
\[ 
\begin{split}
g'[e]= e^{a\bar{\BX }} e^{aA} e^{v+zE} e^{-a\bar{\BX }} e^{a\bar{\BX }}[e]
		=\AD( e^{a\bar{\BX }})\big(  e^{aA} e^{v+zE} \big)[e].
\end{split}  
\]
In $\mG=\mS\oplus_{\rho}\mF$, by definition of $\rho$, we 
\ifthenelse{\value{siTHZ}=1}{have 
$\AD( e^{a\bar{\BX }})\big(  e^{aA} e^{v+zE} \big)= e^{aA} e^{ e^{a\bar{\BX }}v+zE},$}{
successively have
\begin{subequations}
\begin{align}
\ad(a\bar{\BX })(a,v,z)&=(0,a\bar{\BX }v,0)\\
\Ad( e^{a\bar{\BX }})(a,v,z)&=(a, e^{e\bar{\BX }}v,z)\\
\AD( e^{a\bar{\BX }})\big(  e^{aA} e^{v+zE} \big)&= e^{aA} e^{ e^{a\bar{\BX }}v+zE},
\end{align}
\end{subequations}
}
thus $g'[e]= e^{aA} e^{ e^{a\bar{\BX }}v+zE}[e]=[I(a, e^{a\bar{\BX }}v,z)]$. So, in order to get the element $[I(a,v,z)]\in G/S$, we have to act on $[e]$ with the element $g'=J(a, e^{-a\bar{\BX }}v,z)$. All that proves that the map
\begin{equation}
\begin{aligned}
 H\colon F'&\to G/S \\ 
(a,v,z)&\mapsto \big[ I(a, e^{a\bar{\BX }}v,z) \big]  
\end{aligned}
\end{equation}
is a diffeomorphism.
\end{proof}

The work done up to now provides a diffeomorphism
\begin{equation}
\begin{aligned}
 \varphi\colon F'&\to F \\ 
\varphi\big( J(a,v,z) \big)&=I(a, e^{a\bar{\BX }}v,z)
\end{aligned}
\end{equation}
which has suitable properties listed in the proposition below.

\begin{proposition}
This map $\varphi\colon F'\to F$ has the following properties:
\begin{enumerate}
\item if $g'=J(a,v,z)=g'_{S}g'_{F}$ in the sense of decomposition \eqref{EqgpJSF},
\begin{equation}
\varphi\circ L_{g'}=\AD(g'_{S})\circ L_{g'_{F}}\circ\varphi=\Phi_{ e^{a\bar{\BX }}}\circ L_{g'_{F}}\circ\varphi=\Phi_{g'_{S}}\circ L_{g'_{F}}\circ\varphi,
\end{equation}
\item the differential fulfils
\begin{equation}
d(\varphi\circ J)_{(0,0,0)}=dI_{(0,0,0)},
\end{equation}
\item if $\omega$ is the left invariant symplectic form on $F$ and $\omega'$ the one on $F'$, we have
\[ 
  \varphi^*\omega=\omega',
\]
in other words, $\varphi$ is a symplectomorphism.

\end{enumerate}

\end{proposition}
\begin{proof}

The first point is a computation:
\[ 
\begin{split}
\varphi\big( L_{g'_Sg'_F}(g_sg_E) \big)&=\varphi\big( g'_Sg_S\AD(g_S^{-1})(g'_F)g_F \big)\\
		&=\AD(g'_Sg_S)\big( \AD(g_S^{-1})(g'_F)g_F \big)\\
		&=\AD(g'_S)\big( g'_F\AD(g_S)(g_F) \big)\\
		&=\big( \AD(g'_S)\circ L_{g'_F} \big)\big( \varphi(g_Sg_F) \big).
\end{split}  
\]
When $g'=g'_Sg'_F=J(a,v,z)$, we have $g'_S=\exp(a\BX)$ and $g'_F=I(a,v,z)$, so the result is given by
\[ 
  \AD(g'_S)(g'_F)= e^{ \Ad(a\BX) }I(a,v,z)=I(a, e^{a\BX}v,z)=\Phi_{ e^{a\BX}}I(a,v,z).
\]
That concludes the proof of the first point.  For the second statement, we have $(\varphi\circ J)(a,v,z)=\Phi_{ e^{a\BX}}I(a,v,z)$, so
\begin{equation}
\begin{split}
	d(\varphi \circ J)_{(0,0,0)}(Y_a,Y_v,Y_z)&=\Dsdd{  \Phi_{ e^{tY_a}}I(tY_a,tY_v,tY_z)   }{t}{0}\\
		&=\Dsdd{ I\big( tY_a, e^{tY_a\BX}tY_v,tY_z \big) }{t}{0}\\
		&=dI_{(0,0,0)}(Y_a,Y_v,Y_z).
\end{split}
\end{equation}
For the third point, we denote by $e$ and $e'$ the neutral of $F$ and $F'$. On the one hand,
\[ 
  (\varphi^*\omega)_{g'}=\omega_{\varphi(g')}\circ d\varphi_{g'}=\omega_e\circ d\big( L_{\varphi(g')^{-1}}\circ\varphi \big)_{g'};
\]
on the other hand, $\omega'_{g'}=\omega'_{e'}\circ d\big(L_{(g')^{-1}})_{g'}$. Hence, in order to have $\varphi^*\omega=\omega'$, it is necessary that
\[ 
  \omega'_{e'}\circ dJ_{(0,0,0)}=\omega_e\circ d\big(  L_{\varphi(g')^{-1}}\circ\varphi\circ L_{g'}  \big)_{e'}\circ dJ_{(0,0,0)}.
\]
But, for $g'=g'_Sg'_F$, we have
\[ 
\begin{split}
L_{\varphi(g')^{-1}}\circ\varphi\circ L_{g'}(g)&=\varphi(g')^{-1}\varphi(g'g)\\
		&=\varphi(g')^{-1}\AD(g'_S)\big( g'_F\varphi(g) \big)\\
		&=\AD\big( (g'_F)^{-1}\big) \AD(g'_S)\big( g'_F\varphi(g) \big)\\
		&=\big( \AD(g'_S)\circ\varphi \big)(g).
\end{split}  
\]
The first property yields
\[ 
  d\big( L_{\varphi(g')^{-1}}\circ\varphi\circ J \big)_{(0,0,0)}=\Ad(g'_S)\circ dI_{(0,0,0)}=d(\Phi_{ e^{a\BX}})_{e}\circ dI_{(0,0,0)}.
\]
Since $\omega_e$ is invariant under $\Phi_{ e^{a\BX}}$, it remains to be proved that $\omega'_{e'}\circ dJ_{(0,0,0)}=\omega_e\circ dI_{(0,0,0)}$. This is true because, in these coordinates, both sides applied on vectors $(Y_a,Y_v,Y_z)$ and $(Z_a,Z_v,Z_z)$ give
\[ 
  2(Y_aZ_z-Z_aY_z)+\Omega(Y_v,Z_v),
\]
so $\varphi$ is a symplectomorphism.

\end{proof}

Now, if $K$ is the kernel on $F$, we define the kernel on $F'$ by
\begin{equation}\label{EqKerRprime}
\begin{aligned}
 K'\colon F'\times F'\times F'&\to \eC\\
K'&=\varphi^*K. 
\end{aligned}
\end{equation}


\begin{theorem} 
The kernel $K'$ is\index{kernel!for extension $(\BX ,0,2)$ of Heisenberg}
\begin{itemize}
\item left invariant under $F'$,
\item associative on $F'$.
\end{itemize}
 \label{ThoDefoHeizAvecB}
\end{theorem}
\begin{proof}
For left invariance, let $g'=J(a,v,z)$. We have
\[ 
  L_{g'}^*K'=\big( \varphi\circ L_{J(a,v,z)} \big)^*K=\big( \Phi_{ e^{a\BX}}\circ L_{g'_F}\circ\varphi \big)^*K=\varphi^*L_{g'_F}^*\Phi_{ e^{a\BX}}^*K=K',
\]
because of left invariance of $K$ under $F$ and its invariance under $\Phi$. Associativity can be checked in much the same way as in lemma \ref{LemKerINvarIsom}.
\end{proof}

\ifthenelse{\value{siTHZ}=1}{}{
\subsection{Get the \texorpdfstring{$\mu$}{u} back}  \label{SubSecRemetreMu}
%--------------------------------------------------

Equation \eqref{EqKerRprime} provides a kernel on $F'$, the group generated by the algebra
\[ 
  \mF'=\eR A\oplus_{(\BX ,0,2)}\pH_{n}.
\]
Now we try to find a kernel for the group corresponding to the algebra 
\[ 
  \mF_{1}=\eR A\oplus_{(d\BX ,d\mu,2d)}\pH_{n}.
\]
The algebra isomorphism is know from equations \eqref{SubEqsIsommud}, it is
\begin{equation}
\begin{aligned}
 dL\colon \eR A\oplus_{(d\BX ,d\mu,2d)}\pH_{n}&\to \eR A\oplus_{(\BX ,0,2)}\pH_{n} \\ 
dL(aA+v+zE)&=daA+(v+au)+zE 
\end{aligned}
\end{equation}
where $u$ is characterised by
\[ 
  i(u)\Omega=\mu.
\]
The problem is to find the group isomorphism $L$. Since $L$ must be an isomorphism, we have
\[ 
L( e^{aA} e^{v+zE})=L( e^{aA})L( e^{v+zE})
		= e^{daA+au} e^{v+zE}
\]
that we want to write under the form $ e^{a'A} e^{v'+z'E}$. So we write
\[ 
   e^{daA+au}= e^{daA}\underbrace{e^{-daA} e^{daA+au}}_{\text{to be traeated}}.
\]
}

Let $F=F(\mtu,0,2)$ and $F'=F'(\BX,0,2)$. By proposition \ref{PropkernelinvarSp},  the kernel $K$ on $F$ is invariant under $\SP(V,\Omega)$, i.e. $\Phi^*_{M}K=K$ for all $M\in\SP(V,\Omega)$. The action of $\SP(V,\Omega)$ on $F$ is given by
\[ 
  \Phi_{M}\big( I(a,v,z) \big)=I(a,Mv,z).
\]
Define the map $\Phi'_M\colon F'\to F'$,
\begin{equation}   \label{EqDefPhiprimeM}
\Phi_{M}'\big( J(a,v,z) \big)=J\big( a, e^{-a\bar{\BX }}M e^{a\bar{\BX }}v,z \big)
\end{equation}
which fulfils 
\[ 
  \Phi_{M}\circ\varphi=\varphi\circ\Phi_{M}'.
\]
Thus, using the $\SP(V,\Omega)$-invariance of $K$, we have
\[ 
\phi_{M}'{}^*K'=(\varphi\circ\phi_{M}')^*K=(\phi_{M}\circ\phi)^*K=\varphi^*K=K'.
\]
This proves that $K'$ is also invariant under $\SP(V,\Omega)$ too.

\subsection{Jump from one kernel to another}
%-------------------------------------------

We know the deformation of $\SU(1,n)$ described in section \ref{SecDefSURme}.

We have a kernel for the extensions $F_{\delta E^*}(d\mtu,0,2d)$ and $F_{\delta E^*}(\BX ,0,2)$. We can consider the isomorphism $L\colon F(\BX ,0,2)\to F(d\BX ,d\mu,2d)$ which is the lift of
\begin{equation}
\begin{aligned}
 dL\colon \mF(\BX ,0,2)&\to \mF(d\BX ,d\mu,2d) \\ 
  (a,v,z)&\mapsto (da,v+au,z).
\end{aligned}
\end{equation}
If $K'$ is a kernel on $F_{\delta E^*}(d\BX ,d\mu,2d)$, then
\[ 
  K_{0}=L^*K'
\]
is a kernel on $F_{\delta E^*}(\BX ,0,2)$.

An action $\Phi_{0}(M)\colon F(d\BX ,d\mu,2d)\to F(d\BX ,d\mu,2d)$ is given by
\begin{equation}
  \Phi_{0}(M)=L^{-1}\circ\Phi'(M)\circ L
\end{equation}
where $\Phi'(M)\colon F(\BX ,0,2)\to F(\BX ,0,2)$ is given by equation \eqref{EqDefPhiprimeM}. By lemma \ref{LemKerINvarIsom}, the kernel $K_{0}$ is left invariant under the action of $F$ and invariant under the following action of $\SP(V,\Omega)$:
\[ 
  \Phi_{0}(M)^*K_{0}=K_{0}.
\]
\begin{lemma}

Let $\delta\eta^*$ and $\delta\xi^*$ be two exact forms on $\mF$ such that $\xi^*$ and $\eta^*$ belong to the same coadjoint orbit\index{coadjoint!orbit}: there exists a $g\in F$ such that
\begin{equation}
\xi^*\circ\Ad(g)=\eta^*.
\end{equation}
A solution of the problem to find an automorphism $\sigma\colon F\to F$ such that
\begin{equation} \label{EqHypLemSigmaAD}
\delta\eta^*_{\sigma(h)}\big( d\sigma_{h}X_{h},d\sigma_{h}Y_{h} \big)=\delta\xi^*_{h}(X_{h},Y_{h})
\end{equation}
for all $h\in F$ and $X_{h}$, $Y_{h}\in T_{h}F$ is given by $\sigma=\AD(g^{-1})$.
\label{LemJumpCoadOrb} 
\end{lemma}

\begin{proof}
Transported to the identity, the condition \eqref{EqHypLemSigmaAD} becomes:
\[ 
\begin{split}
\delta\eta^*\big( dL_{\sigma(h)^{-1}}d\sigma_{h}X_{h},&dL_{\sigma(h)^{-1}}d\sigma_{h}Y_{h}  \big)\stackrel{!}{=}\delta\xi^*\big( dL_{h^{-1}}X_{h},dL_{h^{-1}}Y_{h} \big)\\
		&=\delta\eta^*\big( \Ad(g^{-1})dL_{h^{-1}}X_{h},\Ad(g^{-1})dL_{h^{-1}}Y_{h}  \big).
  \end{split}
\]
 If $X_{h}=\dsdd{ X_{h}(t)}{t}{0}$, we are searching for a $\sigma$ such that
\[ 
  \Dsdd{ \sigma(h)^{-1}\sigma\big( X_{h}(t) \big) }{t}{0}=\Dsdd{ \AD(g^{-1})\big( h^{-1}X_{h}(t) \big) }{t}{0}.
\]
Since $\sigma$ is a group isomorphism, $\sigma(h)^{-1}=\sigma(h^{-1})$ and the constraint on $\sigma$ becomes
\[ 
  \sigma\big(h^{-1}X_{h}(t)\big)=g^{-1}\big( h^{-1}X_{h}(t) \big)g.
\]
A solution is therefore
\begin{equation}
\sigma=\AD(g^{-1}).
\end{equation}

\end{proof}
