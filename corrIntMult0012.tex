% This is part of Exercices et corrigés de CdI-1
% Copyright (c) 2011
%   Laurent Claessens
% See the file fdl-1.3.txt for copying conditions.

\begin{corrige}{IntMult0012}

	Ce qui est déjà certain, c'est que l'intégrale sur $z$ va de $0$ à $2$. Les solutions de l'équation $1-x^2=1-x$, étant $x=0$ et $x=1$, l'intégrale sur $x$ va de $0$ à $1$. Pour chaque $x$, la variable $t$ prend les valeurs de $1-x$ à $1-x^2$. L'intégrale à calculer est donc
	\begin{equation}
		I=\int_0^2dz\int_0^1dx\int_{1-x}^{1-x^2}(x+y)dy=\int_0^2=2\int_0^1\left[ xy+\frac{ y^2 }{2} \right]_{1-x}^{1-x^2}dx=\frac{ 11 }{ 30 }.
	\end{equation}
	
\end{corrige}
