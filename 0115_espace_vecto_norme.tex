% This is part of Mes notes de mathématique
% Copyright (c) 2008-2014
%   Laurent Claessens
% See the file fdl-1.3.txt for copying conditions.

%+++++++++++++++++++++++++++++++++++++++++++++++++++++++++++++++++++++++++++++++++++++++++++++++++++++++++++++++++++++++++++
\section{Fonctions}		\label{Sect_fonctions}
%+++++++++++++++++++++++++++++++++++++++++++++++++++++++++++++++++++++++++++++++++++++++++++++++++++++++++++++++++++++++++++

Soient $(V,\| . \|_V)$ et $(W,\| . \|_W)$ deux espaces vectoriels normés, et une fonction $f$ de $V$ dans $W$. Il est maintenant facile de définir les notions de limites et de continuité pour de telles fonctions en copiant les définitions données pour les fonctions de $\eR$ dans $\eR$ en changeant simplement les valeurs absolues par les normes sur $V$ et $W$.

En nous inspirant de la définition \ref{DefLimiteFonction}, nous écrivons
\begin{definition}		\label{LimiteDansEVN}
	Soit $f\colon V\to W$ une fonction de domaine \( \Domaine(f)\subset V\) et soit $a$ un point d'accumulation de $\Domaine(f)$. Nous disons que $f$ \defe{admet une limite}{limite!espace vectoriel normé} en $a$ si il existe un élément $\ell\in W$ tel que pour tout $\varepsilon>0$, il existe un $\delta>0$ tel que pour tout $x\in \Domaine(f)$,
    \begin{equation}        \label{EqDefLimzxmasubV}
		0<\| x-a \|_V<\delta\,\Rightarrow\,\| f(x)-\ell \|_W<\varepsilon.
	\end{equation}
	Dans ce cas, nous écrivons $\lim_{x\to a} f(x)=\ell$ et nous disons que $\ell$ est la \defe{limite}{limite} de $f$ lorsque $x$ tend vers $a$.
\end{definition}

\begin{remark}
    Le fait que nous limitions la formule \eqref{EqDefLimzxmasubV} aux \( x\) dans le domaine de \( f\) n'est pas anodin. Considérons la fonction \( f(x)=\sqrt{x^2-4}\), de domaine \( | x |\geq 2\). Nous avons
    \begin{equation}
        \lim_{x\to 2} \sqrt{x^2-4}=0.
    \end{equation}
    Nous ne pouvons pas dire que cette limite n'existe pas en justifiant que la limite à gauche n'existe pas. Les points \( x<2\) sont hors du domaine de \( f\) et ne comptent dons pas dans l'appréciation de l'existence de la limite.

    Vous verrez plus tard que ceci provient de la \wikipedia{fr}{Topologie_induite}{topologie induite} de \( \eR\) sur l'ensemble \( \mathopen[ 2 , \infty [\).
\end{remark}

\begin{definition}\label{DefContDansEVN}
	Une fonction $f\colon D\subset V\to W$ entre deux espaces vectoriels normés $V$ et $W$ est dite \defe{continue}{continue!fonction sur espace vectoriel normé} au point $a\in\bar D$ si $f(x)$ admet une limite pour $x$ tendant vers $a$ et si $\lim_{x\to a} f(x)=f(a)$.
\end{definition}

Une caractérisation très importante des fonctions continues est que l'image inverse d'un ouvert par une fonction continue est ouverte.

\begin{theorem}		\label{ThoContiueImageInvOUvert}
	Soient $V$ et $W$ deux espaces vectoriels normés. Une fonction $f$ de $V$ vers $W$ est continue si et seulement si pour tout ouvert $\mO$ dans $W$, l'ensemble $f^{-1}(\mO)$ est ouvert dans $V$.
\end{theorem}

\begin{proof}
	Supposons d'abord que $f$ est continue. Soit $\mO$ un ouvert de $W$, et prouvons que $f^{-1}(\mO)$ est ouvert. Pour cela, nous allons prouver qu'autour de chaque point $x$ de $f^{-1}(\mO)$, il existe une boule contenue dans $f^{-1}(\mO)$. Nous notons $y=f(x)\in\mO$. Étant donné que $\mO$ est ouvert dans $W$, il existe un rayon $r$ tel que
	\begin{equation}
		B_W\big( f(x),r \big)\subset\mO.
	\end{equation}
	Nous avons ajouté l'indice $W$ pour nous rappeler que c'est une boule dans $W$. Mais la continuité de $f$ implique qu'il existe un rayon $\delta$ tel que $\| x-a \|_V<\delta$ implique $\big\| f(x)-f(a) \big\|_W<r$. Ayant choisit un tel $\delta$, nous savons que si $a\in B_V(x,\delta)$, alors $f(a)\in B_W\big( f(x),r \big)\subset \mO$. Dans ce cas, $a\in f^{-1}(\mO)$. Nous avons donc montré que $B_V(x,\delta)\subset f^{-1}(\mO)$, ce qui prouve que $f^{-1}(\mO)$ est ouvert.

	Supposons maintenant que pour tout ouvert $\mO$ de $W$, l'ensemble $f^{-1}(\mO)$ est ouvert. Nous allons montrer qu'alors $f$ est continue. Soit $x\in V$ et $\varepsilon>0$. Nous devons trouver $\delta$ tel que $0<\| x-a \|_V<\delta$ implique $\| f(a)-f(x) \|_W<\varepsilon$.

	Considérons la boule ouverte $\mO=B_W\big( f(x),\varepsilon \big)$, et son image inverse $f^{-1}(\mO)$ qui est également ouverte par hypothèse. Étant donné que $f(x)\in\mO$, nous avons évidemment $x\in f^{-1}(\mO)$ et donc il existe une boule centrée en $x$ et contenue dans $f^{-1}(\mO)$. Soit $\delta$ le rayon de cette boule :
	\begin{equation}
		B_V\big( x,\delta \big)\subset f^{-1}(\mO).
	\end{equation}
	Par définition de l'image inverse, nous avons aussi $g\big( B_V(x,\delta) \big)\subset\mO$. En récapitulant,
	\begin{equation}
		\| x-a \|_V<\delta\Rightarrow a\in B_V(x,\delta)\Rightarrow f(a)\in\mO=B_W\big( f(x),\varepsilon \big)\Rightarrow\| f(a)-f(x) \|_W<\varepsilon.
	\end{equation}
	Ceci conclut la preuve.
\end{proof}

\begin{remark}
	Cette propriété des fonctions continues est tellement importante qu'elle est souvent prise comme définition de la continuité.
\end{remark}

Un résultat important dans la théorie des fonctions sur les espaces vectoriels normés est qu'une fonction continue sur un compact est bornée et atteint ses bornes. Ce résultat sera (dans d'autres cours) énormément utilisé pour trouver des maxima et minima de fonctions. Le théorème exact est le suivant.

\begin{theorem}		\label{WeierstrassEVN}
	Soit $K\subset V$ une partie compacte (fermée et bornée) d'un espace vectoriel normé $v$. Si $f\colon K\subset V\to \eR$ est une fonction continue, alors $f$ est bornée, et atteint ses bornes. 
	
	C'est à dire qu'il existe $x_0\in K$ tel que $f(x_0)=\inf\{ f(x)\tq x\in K \}$ ainsi que $x_1$ tel que $f(x_1)=\sup\{ f(x)\tq x\in K \}$.
\end{theorem}

Ce résultat sera prouvé dans le théorème \ref{ThoWeirstrassRn} dans le cas particulier de $V=\eR^n$ et dans le théorème \ref{ThoMKKooAbHaro} pour le cas général. Nous n'allons donc pas donner de démonstration de ce théorème ici. Nous allons par contre donner la preuve d'un résultat un peu plus général.

\begin{proposition}		\label{PropContinueCompactBorne}
	Soient $V$ et $W$ deux espaces vectoriels normés. Soit $K$, une partie compacte de $V$, et $f\colon K\to W$, une fonction continue. Alors l'image $f(K)$ est compacte dans $W$.
\end{proposition}
Ce résultat est démontré dans un cadre plus général par le théorème \ref{ThoImCompCotComp}.

\begin{proof}
	Nous allons prouver que $f(K)$ est fermée et bornée.
	\begin{description}
		\item[$f(K)$ est fermé] Nous allons prouver que si $(y_n)$ est une suite convergente contenue dans $f(K)$, alors la limite est également contenue dans $f(K)$. Dans ce cas, nous aurons que l'adhérence de $f(K)$ est contenue dans $f(K)$ et donc que $f(K)$ est fermé. Pour chaque $n\in\eN$, le vecteur $y_n$ appartient à $f(K)$ et donc il existe un $x_n\in K$ tel que $f(x_n)=y_n$. La suite $(x_n)$ ainsi construite est une suite dans le fermé $K$ et possède donc une sous-suite convergente (proposition \ref{ThoBolzanoWeierstrassRn}). Notons $(x'_n)$ cette sous-suite convergente, et $a$ sa limite : $\lim(x'_n)=a\in K$. Le fait que la limite soit dans $K$ provient du fait que $K$ est fermé.

			Nous pouvons considérer la suite $f(x'_n)$ dans $W$. Cela est une sous-suite de la suite $(y_n)$, et nous avons $\lim f(x'_n)=a$ parce que $f$ est continue. Par conséquent nous avons
			\begin{equation}
				f(a)=\lim f(x'_n)=\lim y_n.
			\end{equation}
			Cela prouve que la limite de $(y_n)$ est dans $f(K)$ et par conséquent que $f(K)$ est fermé.

		\item[$f(K)$ est borné]
			Si $f(K)$ n'est pas borné, nous pouvons trouver une suite $(x_n)$ dans $K$ telle que
			\begin{equation}		\label{EqfxnWgeqn}
				\| f(x_n) \|_W>n
			\end{equation}
			Mais par ailleurs, l'ensemble $K$ étant compact (et donc fermé), nous avons une sous-suite $(x'_n)$ qui converge dans $K$. Disons $\lim(x'_n)=a\in K$. 
			
			Par la continuité de $f$ nous avons alors $f(a)=\lim f(x'_n)$, et donc
			\begin{equation}
				| f(a) |=\lim | f(x'_n) |.
			\end{equation}
			La suite $f(x'_n)$ est alors une suite bornée, ce qui n'est pas possible au vu de la condition \eqref{EqfxnWgeqn} imposée à la suite de départ $(x_n)$.
	\end{description}
\end{proof}

\begin{corollary}	\label{CorFnContinueCompactBorne}
	Une fonction $f\colon K\to \eR$ où $K$ est une partie compacte d'un espace vectoriel normé est toujours bornée.
\end{corollary}

\begin{proof}
	En effet, la proposition \ref{PropContinueCompactBorne} montre que $f(K)$ est compact et donc borné.
\end{proof}


%+++++++++++++++++++++++++++++++++++++++++++++++++++++++++++++++++++++++++++++++++++++++++++++++++++++++++++++++++++++++++++
\section{Produit fini d'espaces vectoriels normés}\label{sec_prod}
%+++++++++++++++++++++++++++++++++++++++++++++++++++++++++++++++++++++++++++++++++++++++++++++++++++++++++++++++++++++++++++

Dans cette sections nous parlons de produits finis d'espaces. Cela ne signifie pas que chacun des espaces soient séparément de dimension finie.

%---------------------------------------------------------------------------------------------------------------------------
\subsection{Norme}
%---------------------------------------------------------------------------------------------------------------------------

La définition de la norme sur un produit d'espaces vectoriels normés découle immédiatement de la définition de la distance \ref{DefZTHxrHA} :
\begin{definition}  \label{DefFAJgTCE}
    Soient $V$ et $W$ deux espaces vectoriels normés. On appelle \defe{espace produit}{produit!d'espaces vectoriels normés} de $V$ et $W$ le produit cartésien $V\times W$ 
    \begin{equation}
    V\times W=\{(v,w)\,|\, v\in V,\, w\in W\},
    \end{equation}
    muni de la norme $\|\cdot \|_{V\times W}$
    \begin{equation}	\label{EqNormeVxWmax}
        \|(v,w) \|_{V\times W}=\max\{\|v\|_{V},\|w\|_W\}.
    \end{equation}
\end{definition}
Il est presque immédiat de vérifier que le produit cartésien $V\times W$ est un espace vectoriel pour les opération de somme et multiplication par les scalaires définies composante par composante. C'est à dire,  si $(v_1,w_1)$, $(v_2,w_2)$ sont dans $V\times W$ et $a$, $b$ sont des scalaires, alors  
\begin{equation}
 a (v_1,w_1)+ b(v_2,w_2)=(av_1,aw_1)+ (bv_2,bw_2)=(av_1+bv_2,aw_1+bw_2).
\end{equation}

\begin{lemma}
	L'opération $\|\cdot \|_{V\times W}\colon V\times W\to \eR$ est une norme.
\end{lemma}

\begin{proof}
	On doit vérifier les trois conditions de la définition \ref{DefNorme}.
	\begin{itemize}
		\item Soit $(v,w)$ dans $V\times W$ tel que $\|(v,w)\|_{V\times W}=\max\{\|v\|_{V},\|w\|_W\}=0$. Alors $\|v\|_V=0$ et $\|w\|_W=0$, donc $v=0_V$ et $w=0_W$. Cela implique $(v,w)=(0_v,0_w)=0_{V\times W}$. 
		\item Pour tout $a$ dans $\eR$ et $(v,w)$ dans $V\times W$,  la norme $\|a (v,w)\|_{V\times W}$ est donnée par  $\max\{\|av\|_{V},\|aw\|_W\}$. On peut factoriser $\|av\|_{V}=|a|\|v\|_{V}$ et $\|aw\|_W=|a|\|w\|_W$ et donc $\|a (v,w)\|_{V\times W}=|a|\max\{\|v\|_{V},\|w\|_W\}=|a|\|(v,w)\|_{V\times W}$.
		\item Soient $(v_1,w_1)$ et $(v_2,w_2)$ dans $V\times W$. 
		\begin{equation}
			\begin{aligned}
				\|(v_1,w_1)+(v_2,w_2)\|_{V\times W}&=\max\{\|v_1+v_2\|_{V},\|w_1+w_2\|_W\}\\
				&\leq \max\{\|v_1\|_V+\|v_2\|_{V},\|w_1\|_W+\|w_2\|_W\}\\
				&\leq\max\{\|v_1\|_V,\|w_1\|_W\}+ \max\{\|v_2\|_{V},\|w_2\|_W\}\\
				&=\|(v_1,w_1)\|_{V\times W}+\|(v_2,w_2)\|_{V\times W}.
			\end{aligned}
		\end{equation}
	\end{itemize} 
\end{proof}
On remarque tout de suite que la norme $\|\cdot\|_\infty$ sur $\eR^2$ est la norme de l'espace produit $\eR\times\eR$. En outre cette définition nous permet de trouver plusieurs nouvelles normes dans les espaces $\eR^p$. Par exemple, si nous écrivons $\eR^4$ comme $\eR^2\times \eR^2$ on peut munir $\eR^4$ de la norme produit
\[
\|(x_1,x_2,x_3,x_4)\|_{\infty, 2}=\max\{\|(x_1,x_2)\|_\infty, \|(x_3,x_4)\|_2\}. 
\]    
Les applications de projection de l'espace produit $V\times W$ vers les espaces <<facteurs>>, $V$ $W$ sont notées $\pr_V$ et $\pr_W$ et sont définies par
\begin{equation}
	\begin{aligned}
		\pr_V\colon V\times W&\to V \\
		(v,w)&\mapsto v 
	\end{aligned}
\end{equation}
et
\begin{equation}
	\begin{aligned}
		\pr_W\colon V\times W &\to W \\
		(v,w)&\mapsto w. 
	\end{aligned}
\end{equation}
Les inégalités suivantes sont évidentes
\begin{equation}
	\begin{aligned}[]
		\|\pr_V(v,w)\|_V&\leq \|(v,w)\|_{V\times W} \\
		\|\pr_W(v,w)\|_W&\leq \|(v,w)\|_{V\times W}.
	\end{aligned}
\end{equation}
La topologie de l'espace produit est induite par les topologies des espaces <<facteurs>>. La construction est faite en deux passages : d'abord nous disons que une partie $A\times B$ de $V\times W$ est ouverte si $A$ et $B$ sont des parties ouvertes de $V$ et de $W$ respectivement.  Ensuite nous définissons que une partie quelconque de $V\times W$ est ouverte si elle est une intersection finie ou une réunion de parties ouvertes de $V\times W$ de la forme $A\times B$. 

Ce choix de topologie donne deux propriétés utiles de l'espace produit 
\begin{enumerate}
	\item
		Les projections sont des \defe{applications ouvertes}{application!ouverte}. Cela veut dire que l'image par $\pr_V$ (respectivement $\pr_W$) de toute partie ouverte de $V\times W$ est une partie ouverte de $V$ (respectivement $W$). 
	\item 
		Pour toute partir $A$ de $V$ et $B$ de $W$, nous avons $\Int (A\times B)=\Int A\times \Int B$.\label{PgovlABeqbAbB}
\end{enumerate}
Une propriété moins facile a prouver est que pour toute partie $A$ de $V$ et $B$ de $W$ nous avons  $\overline{A\times B}=\bar{A}\times \bar{B}$. Voir le lemme \ref{LemCvVxWcvVW}.
% position 26329
%et l'exercice \ref{exoGeomAnal-0009}.
  
Ce que nous avons dit jusqu'ici est valable pour tout produit d'un nombre fini d'espaces vectoriels normés. En particulier, pour tout $m>0$  l'espace  $\eR^m$ peut être considéré comme le produit de $m$ copies de $\eR$. 

\begin{example}
	Si $V$ et $W$ sont deux espaces vectoriels, nous pouvons considérer le produit $E=V\times W$. Les projections $\pr_V$ et $\pr_W$\nomenclature{$\pr_V$}{projection de $V\times W$ sur $V$}, définies dans la section \ref{sec_prod}, sont des applications linéaires. 

	En effet, la projection $\pr_V\colon V\times W\to V$ est donnée par $\pr_V(v,w)=v$. Alors,
	\begin{equation}
		\begin{aligned}[]
			\pr_V\big( (v,w)+(v',w') \big)&=\pr_V\big( (v+v'),(w+w') \big)\\
			&=v+v'\\
			&=\pr_V(v,w)+\pr_V(v',w'),
		\end{aligned}
	\end{equation}
	et
	\begin{equation}
		\pr_V\big( \lambda(v,w) \big)=\pr_V\big( (\lambda v,\lambda w) \big)=\lambda v=\lambda\pr_V(v,w).
	\end{equation}
	Nous laissons en exercice le soin d'adapter ces calculs pour montrer que $\pr_W$ est également une projection.
\end{example}

\begin{proposition} \label{PropDXR_KbaLC}
    Si \( \mO\) est un voisinage de \( (a,b)\) dans \( V\times W\) alors \( \mO\) contient un ouvert de la forme \( B(a,r)\times B(b,r)\).
\end{proposition}

\begin{proof}
    Vu que \( \mO\) est un voisinage, il contient un ouvert et donc une boule
    \begin{equation}
        B\big( (a,b),r \big)=\{ (v,w)\in V\times W\tq \max\{ \| v-a \|,\| w-b \| \}< r \}.
    \end{equation}
    Évidemment l'ensemble \( B(a,r)\times B(b,r)\) est dedans.
\end{proof}

%---------------------------------------------------------------------------------------------------------------------------
\subsection{Suites}
%---------------------------------------------------------------------------------------------------------------------------

Nous allons maintenant parler de suites dans $V\times W$. Nous noterons $(v_n,w_n)$ la suite dans $V\times W$ dont l'élément numéro $n$ est le couple $(v_n,w_n)$ avec $v_n\in V$ et $w_n\in W$. La notions de convergence de suite découle de la définition de la norme via la définition usuelle \ref{DefCvSuiteEGVN}. Il se fait que dans le cas des produits d'espaces, la convergence d'une suite est équivalente à la convergence des composantes. Plus précisément, nous avons le lemme suivant.
\begin{lemma}		\label{LemCvVxWcvVW}
	La suite $(v_n,w_n)$ converge vers $(v,w)$ dans $V\times W$ si et seulement les suites $(v_n)$ et $(w_n)$ convergent séparément vers $v$ et $w$ respectivement dans $V$ et $W$. 
\end{lemma}

\begin{proof}
	Pour le sens direct, nous devons étudier le comportement de la norme de $(v_n,w_n)-(v,w)$ lorsque $n$ devient grand. En vertu de la définition de la norme dans $V\times W$ nous avons
	\begin{equation}
		\Big\| (v_n,w_n)-(v,w) \Big\|_{V\times W}=\max\big\{ \| v_n-v \|_V,\| w_n-w \|_W \big\}.
	\end{equation}
	Soit $\varepsilon>0$. Par définition de la convergence de la suite $(v_n,w_n)$, il existe un $N\in\eN$ tel que $n>N$ implique
	\begin{equation}
		\max\big\{ \| v_n-v \|_V,\| w_n-w \|_W \big\}<\varepsilon,
	\end{equation}
	et donc en particulier les deux inéquations
	\begin{subequations}
		\begin{align}
			\| v_n-v \|&<\varepsilon\\
			\| w_n-w \|&<\varepsilon.
		\end{align}
	\end{subequations}
	De la première, il ressort que $(v_n)\to v$, et de la seconde que $(w_n)\to w$.

	Pour le sens inverse, nous avons pour tout $\varepsilon$ un $N_1$ tel que $\| v_n-v \|_V\leq\varepsilon$ pour tout $n>N_1$ et un $N_2$ tel que $\| w_n-w \|_W\leq\varepsilon$ pour tout $n>N_2$. Si nous posons $N=\max\{ N_1,N_2 \}$ nous avons les deux inégalités simultanément, et donc
	\begin{equation}
		\max\big\{ \| v_n-v \|_V,\| w_n-w \|_W \big\}<\varepsilon,
	\end{equation}
	ce qui signifie que la suite $(v_n,w_n)$ converge vers $(v,w)$ dans $V\times W$.
\end{proof}

\begin{remark}		\label{RemTopoProdPasRm}
	Il faut remarquer que la norme \eqref{EqNormeVxWmax} est une norme \emph{par défaut}. C'est la norme qu'on met quand on ne sait pas quoi mettre. Or il y a au moins un cas d'espace produit dans lequel on sait très bien quelle norme prendre : les espaces $\eR^m$. La norme qu'on met sur $\eR^2$ est
	\begin{equation}
		\| (x,y) \|=\sqrt{x^2+y^2},
	\end{equation}
	et non la norme «par défaut» de $\eR^2=\eR\times\eR$ qui serait
	\begin{equation}
		\| (x,y) \|=\max\{ | x |,| y | \}.
	\end{equation}
	Les théorèmes que nous avons donc démontré à propos de $V\times W$ ne sont donc pas immédiatement applicables au cas de $\eR^2$.

	Cette remarque est valables pour tous les espaces $\eR^m$. À moins de mention contraire explicite, nous ne considérons jamais la norme par défaut \eqref{EqNormeVxWmax} sur un espace $\eR^m$.
\end{remark}

Étant donné la remarque \ref{RemTopoProdPasRm}, nous ne savons pas comment calculer par exemple la fermeture du produit d'intervalle $\mathopen] 0,1 ,  \mathclose[\times\mathopen[ 4 , 5 [$. Il se fait que, dans $\eR^m$, les fermetures de produits sont quand même les produits de fermetures.

\begin{proposition}		\label{PropovlAxBbarAbraB}
	Soit $A\subset\eR^m$ et $B\subset\eR^m$. Alors dans $\eR^{m+n}$ nous avons $\overline{ A\times B }=\bar A\times \bar B$.
\end{proposition}

La démonstration risque d'être longue; nous ne la faisons pas ici.

%---------------------------------------------------------------------------------------------------------------------------
\section{Norme opérateur}
%+++++++++++++++++++++++++++++++++++++++++++++++++++++++++++++++++++++++++++++++++++++++++++++++++++++++++++++++++++++++++++
\label{SeckwyQjK}

\begin{definition}[\cite{BrunelleMatricielle}]  \label{DefOYPooZIoWnI}
    Soit \( E\) un espace vectoriel (pas spécialement de dimension finie). Une  \defe{norme}{norme} sur $E$ est une application $\| . \|\colon E\to \eR$ telle que
    \begin{enumerate}
        \item
            $\| v \|=0$ seulement si $A=0$,
        \item
            $\| \lambda v \|=| \lambda |\cdot\| v \|$,
        \item
            $\| v+w \|\leq\| v \|+\| w \|$

    \end{enumerate}
    pour tout $v,w\in E$ et pour tout $\lambda\in\eR$.
\end{definition}

\begin{definition}  \label{DefDQRooVGbzSm}
    Si \( V\) et \( W\) sont des espaces vectoriels nous munissons \( \aL(V,W)\) d'une structure d'espace vectoriel en définissant la somme et le produit par un scalaire de la façon suivante. Si $T$ et $U$ sont des élément de $\aL(V,W)$ et si $\lambda$ est un réel, nous définissons les éléments $T+U$ et $\lambda T$ par
    \begin{enumerate}
        \item
            $(T+U)(x)=T(x)+U(x)$;
        \item
            $(\lambda T)(x)=\lambda T(x)$
    \end{enumerate}
    pour tout \( x\in V\).
\end{definition}

La proposition suivante donne une norme (au sens de la définition \ref{DefNorme}) sur $\aL(V,W)$ afin d'obtenir un espace vectoriel normé.
\begin{proposition}		\label{DefNormeAppLineaire}
    Soit le nombre
	\begin{equation}
        \|T\|_{\aL}=\sup_{x\in V}\frac{\|T(x)\|_{W}}{\|x\|_{V}}.
	\end{equation}
    \begin{enumerate}
        \item
            Si \( V\) est de dimension finie, alors \( \| T \|_{\aL}<\infty\).
        \item
            L'application \( T\mapsto\| T \|_{\aL}\) est une norme sur l'espace vectoriel des applications linéaires \( V\to W\).
        \item
            Nous avons la formule
            \begin{equation}    \label{EqFZPooIoecGH}
                \| T \|_{\aL}=\sup_{x\in V}\frac{\|T(x)\|_{W}}{\|x\|_{V}} =\sup_{\|x\|_{V}\leq 1}\|T(x)\|_{W}
            \end{equation}
    \end{enumerate}
\end{proposition}
\index{norme!d'une application linéaire}
Le nombre \( \| T \|_{\aL}\) est la \defe{norme}{norme!d'application linéaire} de $T$. La norme opérateur est liée à la continuité par la proposition \ref{PropmEJjLE}.

\begin{proof}

    Si \( V\) est de dimension finie alors l'ensemble $\{ \| x \|\leq 1 \}$ est compact par le théorème de Borel-Lebesgue \ref{ThoXTEooxFmdI}. Alors la fonction 
    \begin{equation}
        x\mapsto \frac{ \| T(x) \| }{ \| x \| }
    \end{equation}
    est une fonction continue sur un compact. Le corollaire \ref{CorFnContinueCompactBorne} nous dit alors qu'elle est bornée. Le supremum est donc un nombre réel fini.

    Nous vérifions que l'application $\| . \|$ de $\aL(V,W)$ dans $\eR$ ainsi définie est effectivement une norme.
    \begin{enumerate}
        \item
            $\|T\|_{\aL}=0$ signifie que $\|T(x)\|=0$ pour tout $x$ dans $V$. Comme  $\|\cdot\|_W$ est une norme nous concluons que $T(x)=0_{n}$ pour tout $x$ dans $V$, donc $T$ est l'application nulle. 
    \item
        Pour tout $a$ dans $\eR$ et tout  $T$ dans $\aL(V,W)$ nous avons
        \begin{equation}
            \|aT\|_{\mathcal{L}}=\sup_{\|x\|_{V}\leq 1}\|aT(x)\|_{W}=|a|\sup_{\|x\|_{V}\leq 1}\|T(x)\|_{W}=|a|\|T\|_{\mathcal{L}}.
        \end{equation}
    \item 
        Pour tous $T_1$ et $T_2$ dans $\aL(V,W)$ nous avons
      \begin{equation}\nonumber
        \begin{aligned}
           \|T_1+ T_2\|_{\mathcal{L}}&=\sup_{\|x\|\leq 1}\|T_1(x)+T_2(x)\|\leq\\
     &\leq\sup_{\|x\|\leq 1}\|T_1(x)\| +\sup_{\|x\|\leq 1}\|T_2(x)\|\\
     &=\|T_1\|\|T_2\|.
        \end{aligned}
      \end{equation}
    \end{enumerate}


    Enfin nous prouvons la formule alternative \eqref{EqFZPooIoecGH}. Nous allons montrer que les ensembles sur lesquels ont prend le supremum sont en réalité les mêmes :
    \begin{equation}
        \underbrace{\left\{ \frac{ \| Ax \| }{ \| x \| }\right\}_{x\neq 0}}_{A}=\underbrace{\left\{ \| Ax \|\tq \| x \|=1 \right\}}_{B}.
    \end{equation}
    Attention : ce sont des sous-ensembles de réels; pas de sous-ensembles de \( \eM(\eR)\) ou des sous-ensembles de \( \eR^n\).

    Pour la première inclusion, prenons un élément de \( A\), et prouvons qu'il est dans \( B\). C'est à dire que nous prenons \( x\in V\) et nous considérons le nombre \( \| Ax \|/\| x \|\). Le vecteur \( y=x/\| x \|\) est un vecteur de norme $1$, donc la norme de \( Ay\) est un élément de \( B\), mais
    \begin{equation}
        \| Ay \|=\frac{ \| Ax \| }{ \| x \| }.
    \end{equation}
    Nous avons donc \( A\subset B\).

    L'inclusion \( B\subset A\) est immédiate.
\end{proof}

\begin{proposition}
    Pour tout norme algébrique, le rayon spectral d'une matrice sur \( \eC\) est toujours plus petit que sa norme. C'est à dire que nous avons toujours \( \rho(A)\leq \| A \|\) pour toute norme algébrique \( \| . \|\).
\end{proposition}

\begin{definition}
    Le \defe{\wikipedia{en}{Spectral_radius}{rayon spectral}}{rayon spectral} d'une matrice carrée $A$, noté $\rho(A)$, est défini de la manière suivante :
    \begin{equation}
        \rho(A)=\max_i|\lambda_i|
    \end{equation}
    où les $\lambda_i$ sont les valeurs propres de $A$.
\end{definition}

\begin{example}     \label{ExemdefnormpMrt}
    Pour chaque norme sur \( \eR^n\), nous pouvons définir une norme correspondante sur \( \eM_n(\eR)\), appelée \defe{\wikipedia{fr}{Norme_d'opérateur}{norme opérateur}}{norme!opérateur}. Si \( \| . \|\) est une norme sur \( \eR^n\), nous définissons \( \| A \|\) par
    \begin{equation}\label{EqThUCEJ}
        \|A\|=\sup_{\|x\|\neq 0}\frac{\|Ax\|}{\|x\|}
    \end{equation}
    En particulier, cela donne lieu à toutes les normes \( \| A \|_p\) qui correspondent aux normes \( \| . \|_p\) sur \( \eR^n\). Cette norme est la norme \defe{subordonnée}{norme!subordonnée} à celle sur \( \eR^n\).
\end{example}
La norme opérateur est souvent écrite \( \| A \|_{\infty}\) parce que cette norme donne lieu à la \defe{topologie forte}{topologie!forte} sur l'espace des opérateurs. La topologie forte n'est pas la seule possible. Il existe aussi par exemple la \defe{topologie faible}{topologie!faible} donnée par la notion de convergence \( A_i\to A\) si et seulement si \( A_ix\to Ax\) pour tout \( x\in E\).

Il faut noter que la topologie faible n'est pas une topologie métrique. Cela même si la condition \( A_ix\to Ax\), elle, est métrique vu qu'elle est écrite dans \( E\).
%TODO : il faut mettre au clair quelle est vraiment la topologie faible à partir des ouverts.
et que dans le cas où \( E\) est de dimension infinie, elle est réellement différente de la topologie forte. Nous verrons à la sous-section \ref{subsecaeSywF} que dans le cas des projections sur un espaces de Hilbert, l'égalité
\begin{equation}
    \sum_{i=1}^{\infty}\pr_{u_i}=\id
\end{equation}
est vraie pour la topologie faible, mais pas pour la topologie forte.

\begin{definition}[Norme d'algèbre]  \label{DefJWRWQue}
    Si \( A\) est une algèbre\footnote{Définition \ref{DefAEbnJqI}.}, une \defe{norme d'algèbre}{norme!d'algèbre} sur \( A\) est une norme telle que pour toute \( u,v\in A\),
    \begin{equation}
        \| uv \|\leq \| u \|\| v \|.
    \end{equation}
\end{definition}
L'intérêt d'une norme d'algèbre est entre autres de mieux se comporter pour les séries, voir par exemple \ref{subsecEVnZXgf}.

\begin{definition}
    Si \( E\) et \( F\) sont deux espaces vectoriels normés nous notons \( \GL(E,F)\)\nomenclature[Y]{\( \GL(E,F)\)}{les bijections linéaires et continues} le sous-espace de \( \aL(E,F)\) des isométries linéaires de \( E\) vers \( F\). Un élément de \( \GL(E,F)\) est donc linéaire, continue et bijective.
\end{definition}

\begin{proposition} \label{PropEDvSQsA}
    Si \( E\) et \( F\) sont des espaces vectoriels normés alors la norme opérateur est une norme d'algèbre\footnote{Définition \ref{DefJWRWQue}.} sur \( \GL(E,F)\) :
    \begin{equation}
        \| AB \|\leq \| A \|\| B \|
    \end{equation}
    pour tout \( A,B\in\GL(E)\). De plus pour tout \( A\in \aL(E,F)\), et pour tout \( u\in E\) nous avons la majoration
    \begin{equation}
        \| Au \|\leq \| A \|\| u \|.
    \end{equation}
\end{proposition}

\begin{proof}
    Nous avons
    \begin{equation}
    \| AB \|=\sup_{x\in E}\frac{ \| ABx \| }{ \| x \| }\frac{ \| Bx \| }{ \| Bx \| }=\sup_{x\in E}\frac{ \| A(Bx) \| }{ \| Bx \| }\frac{ \| Bx \| }{ \| x \| }\leq \sup_{x\in E}\frac{ \| A(Bx) \| }{ \| Bx \| }\sup_{x\in E}\frac{ \| Bx \| }{ \| x \| }.
    \end{equation}
    Le premier facteur est égal à \( \| A \|\) parce que \( B\) est surjective. Le second est \( \| B \|\) par définition.

    Si \( u\in E\) alors
    \begin{equation}
        \| A \|=\sup_{x\in E}\frac{ \| Ax \| }{ \| x \| }\geq \frac{ \| Au \| }{ \| u \| },
    \end{equation}
    donc le résultat annoncé : \( \| Au \|\leq \| A \|\| u \|\).
\end{proof}
Notons qu'en réalité nous n'avons utilisé seulement le fait que \( B\) était surjective

\begin{lemma}   \label{LemWWXVSae}
Soit \( F\) un espace de Banach et deux suites \( A_k\to A\) et \( B_k\to B\) dans \( \aL(F,F)\). Alors \( A_k\circ B_k\to A\circ B\) dans \( \aL(F,F)\), c'est à dire
\begin{equation}
    \lim_{n\to \infty} (A_kB_k)=\left( \lim_{n\to \infty} A_k \right)\left( \lim_{n\to \infty} B_k \right).
\end{equation}
\end{lemma}

\begin{proof}
    Il suffit d'écrire
    \begin{equation}
        \| A_kB_k-AB \|\leq \| A_kB_k-A_kB \|+\| A_kB-AB \|.
    \end{equation}
    Le premier terme tend vers zéro pour \( k\to\infty\) parce que 
    \begin{subequations}
        \begin{align}
            \| A_kB_k-A_kB \|=\| A_k(B_k-B) \|\leq \| A_k \|\| B_k-B \|\to \| A \|\cdot 0=0
        \end{align}
    \end{subequations}
    où nous avons utilisé la propriété fondamentale de la norme opérateur : la proposition \ref{PropEDvSQsA}. Le second terme tend également vers zéro pour la même raison.
\end{proof}

\begin{proposition}[Distributivité de la somme infinie] \label{PropQXqEPuG}
    Soient \( E\) un espace normé, une suite \( (u_k)\) dans \( \GL(E)\) ainsi que \( a\in\GL(E)\). Pourvu que la série \( \sum_{n=0}^{\infty}u_k\) converge nous avons
    \begin{equation}
        \left( \sum_{k=0}^{\infty}u_k \right)a=\sum_{k=0}^{\infty}(u_ka).
    \end{equation}
\end{proposition}

\begin{proof}
    Par définition de la somme infinie,
    \begin{equation}
        \spadesuit=\left( \sum_{k=0}^{\infty}u_k \right)a=\left( \lim_{n\to \infty} \sum_{k=0}^nu_k \right)a.
    \end{equation}
    Le lemme \ref{LemWWXVSae} appliqué à \( n\mapsto\sum_{k=0}^nu_k\) et à la suite constante \( a\) nous donne
    \begin{equation}    \label{EqOAoopjz}
        \spadesuit=\lim_{n\to \infty} \left( \sum_{k=0}u_ka \right),
    \end{equation}
    ce que nous voulions par distributivité de la somme finie : dans \eqref{EqOAoopjz}, le \( a\) est dans ou hors de la somme, au choix. L'important est qu'il soit dans la limite.
\end{proof}


\begin{lemma}
    Si \( E\) et \( F\) sont des espaces de Banach\quext{Je crois qu'il ne faut pas que \( E\) soit complet.}, l'espace \( \aL(E,F)\) est également de Banach.
\end{lemma}

\begin{proof}
    Soit \( (u_n)\) une suite de Cauchy dans \( \aL(E,F)\); si \( x\in E\) il existe \( N\) tel que si \( l,m>N\) alors \( \| a_l-a_m \|<\epsilon\), c'est à dire que pour tout \( \| x \|=1\) on a \( \| u_l(x)-u_n(x) \|<\epsilon\). Cela signifie que \( u_n(x)\) est une suite de Cauchy dans l'espace complet \( F\). Cette suite converge et nous pouvons définir l'application \( u\colon E\to F\) par
    \begin{equation}
        u(x)=\lim_{n\to \infty} u_n(x).
    \end{equation}
    Il suffit maintenant de prouver que \( u\) est linéaire, ce qui est une conséquence directe de la linéarité de la limite :
    \begin{equation}
        u(\alpha x+\beta y)=\lim_{n\to \infty} \big( \alpha u_n(x)+\beta u_n(y) \big).
    \end{equation}
\end{proof}

\begin{proposition}     \label{PropQAjqUNp}
    Soit \( E\) un espace de Banach (espace vectoriel normé complet). Si \( A\in\aL(E,E)\) avec \( \| A \|<1\) pour la norme opérateur, alors \( (\mtu-A)\) est inversible et son inverse est donné par
    \begin{equation}
        (\mtu-A)^{-1}=\sum_{k=0}^{\infty}A^k.
    \end{equation}
    Le résultat tient aussi si \( A\) est nilpotente, même si sa norme n'est pas majorée par \( 1\).
\end{proposition}
\index{série!donnant \( (1-A)^{-1}\)}

\begin{proof}
    Étant donné que la norme opérateur est une norme algébrique (proposition \ref{PropEDvSQsA}), nous avons \( \| A^k \|\leq \| A \|^k\). Par conséquent la série \( \| A^k \|\) est majorée par la série géométrique qui converge. Par conséquent \( \sum_{k}A^k\) est une série absolument convergence et donc convergente par la proposition \ref{PropAKCusNM} et le fait que \( \aL(E)\) est complet (proposition \ref{PropQAjqUNp}).
    
    Montrons à présent que la somme est l'inverse de \( \mtu-A\) en utilisant le produit terme à terme autorisé par la proposition \ref{PropQXqEPuG} :
    \begin{equation}
        \sum_{k=0}^nA^k(\mtu-A)=\sum_{k=0}^n(A^k-A^{k+1})=\mtu-A^{n+1}.
    \end{equation}
    Par conséquent 
    \begin{equation}
        \| \mtu-\sum_{k=0}^nA^k(\mtu-A) \|=\| A^{n+1} \|\leq \| A \|^{n+1}\to 0.
    \end{equation}

    Si \( A\) est nilpotente, la convergence de \( \sum_{k=0}^{\infty}A^k\) ne pose pas de problèmes parce que la somme est finie. Le fait que cette somme soit \( (\mtu-A)^{-1}\) s'obtient de la même façon, mais il ne faut pas faire la dernière majoration : si \( A\) est nilpotente, il tombe sous le sens que \( \| A^{n+1} \|\to 0\), mais il n'est cependant pas vrai que \( \| A \|^{n+1}\) tende vers zéro.
\end{proof}

%+++++++++++++++++++++++++++++++++++++++++++++++++++++++++++++++++++++++++++++++++++++++++++++++++++++++++++++++++++++++++++ 
\section{Applications multilinéaires}
%+++++++++++++++++++++++++++++++++++++++++++++++++++++++++++++++++++++++++++++++++++++++++++++++++++++++++++++++++++++++++++

\begin{definition}[Application multilinéaire]       \label{DefFRHooKnPCT}
    Une application $T: \eR^{m_1}\times \ldots \times\eR^{m_k}\to\eR^p $ est dite \defe{\( k\)-linéaire}{application!multilinéaire} si pour tout $X=(x_1, \ldots,x_k)$ dans $ \eR^{m_1}\times \ldots \times\eR^{m_k}$ les applications $x_i\mapsto T(x_1, \ldots, x_i,\ldots,x_k)$ sont linéaires pour tout $i$ dans $\{1,\ldots,k\}$, c'est à dire
	\begin{equation}
		\begin{aligned}[]
			T(\cdot,x_2, \ldots, x_i,\ldots,x_k)&\in \mathcal{L}(\eR^{m_1}, \eR^p),\\
			T(x_1,\cdot, \ldots, x_i,\ldots,x_k)&\in \mathcal{L}(\eR^{m_2}, \eR^p),\\
						& \vdots\\
			T(x_1, \ldots, x_i,\ldots,x_{k-1},\cdot)&\in \mathcal{L}(\eR^{m_k}, \eR^p).\\
		\end{aligned}
	\end{equation}
	En particulier lorsque $k=2$, nous parlons d'applications \defe{bilinéaires}{bilinéaire}. Vous pouvez deviner ce que sont les applications \emph{tri}linéaire ou \emph{quadri}linéaire.
\end{definition}

L'ensemble des applications $k$-linéaires de $ \eR^{m_1}\times \ldots \times\eR^{m_k}$ dans $\eR^p$ est noté $\mathcal{L}(\eR^{m_1}\times \ldots \times\eR^{m_k}, \eR^p)$ ou $\mathcal{L}(\eR^{m_1}, \ldots,\eR^{m_k}; \eR^p)$.

\begin{example}
  Soit $A$ une matrice avec $m$ lignes et $n$ colonnes. L'application bilinéaire de $\eR^m\times \eR^n$ dans $\eR$ associée à $A$ est définie par
\[
T_A(x,y)= x^TAy=\sum_{i,j}a_{i,j}x_i y_j, \qquad \forall x\in \eR^m, \, y \in \eR^n.
\]
\end{example}

Nous énonçons la proposition suivante dans le cas d'espaces vectoriels normés\footnote{Sans hypothèses sur la dimension.} parce que nous allons l'utiliser dans ce cas, mais le cas particulier \( E_i=\eR^{m_i}\) et \( F=\eR^p\) est important.
\begin{proposition} \label{PropUADlSMg}
    Soient des espaces vectoriels normés \( E_i\) et \( F\). Une application \( n\)-linéaire
    \begin{equation}
        T\colon E_1\times\ldots\times E_n\to F
    \end{equation}
    est est continue si et seulement s'il existe un réel $L\geq 0$ tel que
  \begin{equation}\label{limitatezza}
     \|T(x_1, \ldots,x_n)\|_F\leq L \|x_1\|_{F_1}\cdots\|x_n\|_{F_n}, \qquad \forall x_i\in E_i.
  \end{equation}
\end{proposition}

\begin{proof}
    Pour simplifier l'exposition nous nous limitons au cas $n=2$ et nous notons $T(x,y)=x*y$

    Supposons que l'inégalité \eqref{limitatezza} soit satisfaite. 
    \begin{equation}\label{LimImplCont}
      \begin{aligned}
        \|x*y-x_0*y_0\|&=\|(x-x_0)*y-x_0*(y-y_0)\|\\
    &\leq \|(x-x_0)*y\|+\|x_0*(y-y_0)\|\\
    &\leq L\|x-x_0\|\|y\| + L\|x_0\|\|y-y_0\|.
      \end{aligned}
    \end{equation}
    Si $x\to x_0$ et $y\to y_0$  on voit que $T$ est continue en passant à la limite aux deux côtes de l'inégalité \eqref{LimImplCont}.

    Soit $T$ continue en $(0,0)$. Évidemment\footnote{Dans la formule suivante, les trois zéros sont les zéros de trois espaces différents.} $0*0=0$, donc il existe $\delta>0$ tel que si $x\in B_{E_1}(0,\delta)$ et $y\in B_{E_2}(0,\delta)$ alors $\|x*y\|\leq 1$. En particulier si \( (x,y)\in B_{E_1\times E_2}(0,\delta)\) nous sommes dans ce cas. Soient maintenant  $x\in E_1\setminus\{ 0 \}$  et $y\in E_2\setminus\{ 0\}$
    \begin{equation}
        x*y=\left(\frac{\|x\|}{\delta}\frac{\delta x}{\|x\|}\right)*\left(\frac{\|y\|}{\delta}\frac{\delta y}{\|y\|}\right)
    =\frac{\|x\|\|y\|}{\delta^2} \left(\frac{\delta x}{\|x\|}\right)*\left(\frac{\delta y}{\|y\|}\right).
     \end{equation}
    On remarque que $\delta x/\|x\|_m$ est dans la boule de rayon $\delta$ centrée en $0_m$ et que $\delta y/\|y\|_n$ est dans la boule de rayon $\delta$ centrée en $0_n$. On conclut 
    \[
     x*y\leq \frac{\|x\|_m\|y\|_n}{\delta^2}.
    \]
    Il faut prendre $L=1/\delta^2$.
\end{proof}

La norme de \( T\) est alors définie comme la plus petite constante \( L\) qui fait fonctionner la proposition \ref{PropUADlSMg}.
\begin{definition}  \label{DefKPBYeyG}
	La norme sur l'espace $\aL(E_1\times \cdots\times E_n, F)$ des applications $k$-linéaires et continues est 
	\begin{equation}
        \|T\|_{E_1\times \ldots\times E_n}=\sup\{ \|T(u_1, \ldots,u_k)\|_{F}\,\vert\,\|u_i\|_{E_i}\leq 1, i=1,\ldots, k \}.
	\end{equation}
\end{definition}
Nous avons donc automatiquement
\begin{equation}    \label{EqYLnbRbC}
    \| T(u,v) \|\leq \| T \|\| u \|\| v \|.
\end{equation}
Et nous notons que cette norme est uniquement définie pour les applications linéaires continues. Ce n'est pas très grave parce qu'alors nous définissons \( \| T \|=\infty\) si \( T\) n'est pas continue. Cela pour retrouver le principe selon lequel on est continue si et seulement si on est borné.

\begin{proposition}\label{isom_isom}
  On définit les fonctions
  \begin{equation}
    \begin{array}{rccc}
      \omega_g: & \mathcal{L}(\eR^{m}\times\eR^{n}, \eR^p)&\to &\mathcal{L}(\eR^{m}, \mathcal{L}(\eR^{n}, \eR^p)),\\
      \omega_d: & \mathcal{L}(\eR^{m}\times\eR^{n}, \eR^p)&\to &\mathcal{L}(\eR^{n}, \mathcal{L}(\eR^{m}, \eR^p)),
    \end{array}
  \end{equation}
par 
\[
\omega_g(T)(x)=T(x,\cdot), \qquad \forall x\in\eR^m,
\]
et
\[
\omega_d(T)(y)=T(\cdot, y), \qquad \forall y\in\eR^n.
\]
Les fonctions $\omega_g$ et $\omega_d$ sont des isomorphismes qui préservent les normes.    
\end{proposition}

%+++++++++++++++++++++++++++++++++++++++++++++++++++++++++++++++++++++++++++++++++++++++++++++++++++++++++++++++++++++++++++ 
\section{Calcul différentiel dans un espace vectoriel normé}
%+++++++++++++++++++++++++++++++++++++++++++++++++++++++++++++++++++++++++++++++++++++++++++++++++++++++++++++++++++++++++++
\label{SecLStKEmc}

Nous développons dans cette section le concept de différentielle de fonction de et vers des espaces vectoriels normés au lieu de \( \eR^n\).

%--------------------------------------------------------------------------------------------------------------------------- 
\subsection{Différentielle}
%---------------------------------------------------------------------------------------------------------------------------

\begin{definition}  \label{DefKZXtcIT}
    Soit une application \( f\colon E\to F\) entre deux espaces de Banach. Nous disons que \( f\) est \defe{différentiable}{différentiable!dans un Banach} en \( a\in E\) si il existe une application linéaire continue\footnote{Nous demandons bien que le candidat différentielle soit continue; en dimension infinie ce n'est pas le cas de toutes les fonctions linéaires, comme le montre l'exemple \ref{ExHKsIelG}.} \( T\colon E\to F\) telle que
    \begin{equation}\label{EqIQuRGmO}
        \lim_{h\to 0} \frac{ f(a+h)-f(a)-T(h) }{ \| h \| }=0.
    \end{equation}
\end{definition}

%--------------------------------------------------------------------------------------------------------------------------- 
\subsection{(non ?) Différentiabilité des applications linéaires}
%---------------------------------------------------------------------------------------------------------------------------

Si \( E\) et \( F\) sont deux espaces vectoriels nous notons \( \aL(E,F)\)\nomenclature[Y]{\( \aL(E,F)\)}{Les applications linéaires de \( E\) vers \( F\)} l'ensemble des applications linéaires de \( E\) vers \( F\) et \( \cL(E,F)\)\nomenclature[Y]{\( \cL\)}{Les applications linéaires continues de \( E\) vers \( F\)} l'ensemble des applications linéaires continues de \( E\) vers \( F\). Ces espaces seront bien entendu, sauf mention du contraire, toujours munis de la norme opérateur de l'exemple \ref{ExemdefnormpMrt}. 

\begin{example}[Une application linéaire non continue]  \label{ExHKsIelG}
    Soit \( V\) l'espace vectoriel normé des suites \emph{finies} de réels muni de la norme usuelle $\| c \|=\sqrt{\sum_{i=0}^{\infty}| c_i |^2}$ où la somme est finie. Nous nommons \( \{ e_k \}_{k\in \eN}\) la base usuelle de cet espace, et nous considérons l'opérateur \( f\colon V\to V\) donnée par \( f(e_k)=ke_k\). C'est évidemment linéaire, mais ce n'est pas continu en zéro. En effet la suite \( u_k=e_k/k\) converge vers \( 0\) alors que \( f(u_k)=e_k\) ne converge pas.
\end{example}

Cet exemple aurait pu également être donnée dans un espace de Hilbert, mais il aurait fallu parler de domaine.
%TODO : le faire, et regarder si Hilbet n'est pas la complétion de cet espace. Référencer à l'endroit qui définit l'espace vectoriel librement engendré. Ici ce serait par N.

%TODO : dire qu'une application bilinéaire sur RxR n'est pas une application linéaire sur R^2

\begin{example}[Une autre application linéaire non continue\cite{GTkeGni}]
    En dimension infinie, une application linéaire n'est pas toujours continue. Soit \( E\) l'espace des polynômes à coefficients réels sur \( \mathopen[ 0 , 1 \mathclose]\) muni de la norme uniforme. L'application de dérivation \( \varphi\colon E\to E\), \( \varphi(P)=P'\) n'est pas continue.

    Pour la voir nous considérons la suite \( P_n=\frac{1}{ n }X^n\). D'une part nous avons \( P_n\to 0\) dans \( E\) parce que \( P_n(x)=\frac{ x^n }{ n }\) avec \( x\in \mathopen[ 0 , 1 \mathclose]\). Mais en même temps nous avons \( \varphi(P_n)=X^{n-1}\) et donc \( \| \varphi(P_n) \|=1\).

    Nous n'avons donc pas \( \lim_{n\to \infty} \varphi(P_n)=\varphi(\lim_{n\to \infty} P_n)\) et l'application \( \varphi\) n'est pas continue en \( 0\). Elle n'est donc continue nulle part par linéarité.

    Nous avons utilisé le critère séquentiel de la continuité, voir la définition \ref{DefENioICV} et la proposition \ref{PropFnContParSuite}.
\end{example}

Nous avons cependant le résultat suivant.
\begin{proposition}[\cite{GKPYTMb}] \label{PropmEJjLE}
    Soient \( E\) et \( F\) des espaces vectoriels normés, et \( u\colon E\to F\) une application linéaire. Alors \( u\) est bornée\footnote{Au sens où \( \| u \|<\infty\) pour la norme opérateur.} si et seulement si elle est continue.
\end{proposition}
\index{opérateur!linéaire!borné}
\index{application!linéaire!bornée}

\begin{proof}
    Nous commençons par supposer que \( u\) est bornée. Pour tout \( x,y\in E\) nous avons
    \begin{equation}
        \| u(x)-u(y) \|=\| u(x-y) \|\leq \| u \|\| x-y \|.
    \end{equation}
    En particulier si \( x_n\stackrel{E}{\longrightarrow}x\) alors
    \begin{equation}
        0\leq \| u(x_n)-u(x) \|\leq \| u \|\| x-x_n \|\to 0
    \end{equation}
    et \( u\) est continue en vertu de la caractérisation séquentielle de la continuité, proposition \ref{PropFnContParSuite}.

Supposons maintenant que \( \| u \|\) ne soit pas borné, c'est à dire que l'ensemble \( \{ \| u(x) \|\tq \| x \|=1 \}\) ne soit pas borné. Alors pour tout \( k\geq 1\) il existe \( x_k\in B(0,1)\) tel que \( \| u(x_k) \|>k\). La suite \( x_k/k\) tend vers zéro parce que \( \| x_k \|=1\), mais \( \| u(x_k) \|\geq 1\) pour tout \( k\). Cela montre que \( u\) n'est pas continue.
\end{proof}
Cette proposition permet de retrouver l'exemple \ref{ExHKsIelG} plus simplement. Si \( \{ e_k \}_{k\in \eN}\) est une base d'un espace vectoriel normé formée de vecteurs de norme \( 1\), alors l'opérateur linéaire donné par \( u(e_k)=ke_k\) n'est pas borné et donc pas continu.

C'est également ce résultat qui montre que le produit scalaire est continu sur un espace de Hilbert par exemple.

\begin{lemma}
    Si \( f\) est linéaire et différentiable alors \( df_a(u)=f(u)\).
\end{lemma}

\begin{proof}
    En effet la linéarité de \( f\) donne
    \begin{equation}
        f(a+h)-f(a)-f(h)=0
    \end{equation}
    pour tout \( h\). Donc la limite \eqref{EqIQuRGmO} est nulle. Les applications linéaires non continues ne sont donc pas différentiables.
\end{proof}

\begin{lemma}   \label{LemLLvgPQW}
    Une application linéaire continue est de classe \(  C^{\infty}\).
\end{lemma}

\begin{proof}
    Soit \( a\in E\). Étant donné que \( f\) est linéaire et continue, elle est différentiable et
    \begin{equation}
        \begin{aligned}
            df\colon E&\to \cL(E,F) \\
            a&\mapsto f 
        \end{aligned}
    \end{equation}
    est une fonction constante et en particulier continue; nous avons donc \( f\in C^1\). Pour la différentielle seconde nous avons \( d(df)_a=0\) parce que \( df(a+h)-df(a)=f-f=0\). Toutes les différentielles suivantes sont nulles.
\end{proof}

%--------------------------------------------------------------------------------------------------------------------------- 
\subsection{Dérivation en chaine et formule de Leibnitz}
%---------------------------------------------------------------------------------------------------------------------------

\begin{proposition} \label{PropOYtgIua}
    Soient \( f_i\colon U\to F_i\), des fonctions de classe \( C^r\) où \( U\) est ouvert dans l'espace vectoriel normé \( E\) et les \( F_i\) sont des espaces vectoriels normés. Alors l'application
    \begin{equation}
        \begin{aligned}
        f=f_1\times \cdots\times f_n\colon U&\to F_1\times \cdots\times F_n \\
    x&\mapsto \big( f_1(x),\ldots, f_n(x) \big) 
        \end{aligned}
    \end{equation}
    est de classe \( C^r\) et
    \begin{equation}
    d^rf=d^rf_1\times\ldots d^rf_n.
    \end{equation}
\end{proposition}

\begin{proof}
    Soit \( x\in U\) et \( h\in E\). La différentiabilité des fonctions \( f_i\) donne
    \begin{equation}
        f_i(x+h)=f_i(x)+(df_i)_x(h)+\alpha_i(h)
    \end{equation}
    avec \( \lim_{h\to 0} \alpha_i(h)/\| h \|=0\). Par conséquent
    \begin{equation}
        f(x+h)=\big( \ldots, f_i(x)+(df_i)_x(h)+\alpha_i(h),\ldots \big)= \big( \ldots,f_i(x),\ldots \big)+ \big( \ldots,(df_i)_x(h),\ldots \big)+ \big( \ldots,\alpha_i(h),\ldots \big).
    \end{equation}
    Mais la définition \ref{DefFAJgTCE} de la norme dans un espace produit donne
    \begin{equation}
        \lim_{h\to 0} \frac{ \| \big( \alpha_1(h),\ldots, \alpha_n(h) \big) \| }{ \| h \| }=0,
    \end{equation}
    ce qui nous permet de noter \( \alpha(h)=\big( \alpha_1(h),\ldots, \alpha_n(h) \big)\) et avoir \( \lim_{h\to 0} \alpha(h)/\| h \|=0\). Avec tout ça nous avons bien
    \begin{equation}
        f(x+h)=f(x)+\big( (df_1)_x(h)+\ldots +(df_n)_x(h) \big)+\alpha(h),
    \end{equation}
    ce qui signifie que \( f\) est différentiable et
    \begin{equation}
        df_x=\big( df_1,\ldots, df_n \big).
    \end{equation}
\end{proof}

\begin{theorem}[Différentielle de fonctions composées\cite{SNPdukn}]    \label{ThoAGXGuEt}
    Soient \( E\), \( F\) et \( G\) des espaces vectoriels normés, \( U\) ouvert dans \( E\) et \( V\) ouvert dans \( F\). Soient des applications de classe \( C^r\) (\( r\geq 1\))
    \begin{subequations}
        \begin{align}
            f\colon U\to V\\
            g\colon V\to G.
        \end{align}
    \end{subequations}
    Alors l'application \( g\circ f\colon V\to G\) est de classe \( C^r\) et
    \begin{equation}\label{EqHFmezmr}
        d(g\circ f)_x=dg_{f(x)}\circ df_x.
    \end{equation}
\end{theorem}

\begin{proof}
    Nous nous fixons \( x\in U\). La fonction \( f\) est différentiable en \( x\in U\) et \( g\) en \( f(x)\), donc nous pouvons écrire
    \begin{equation}
        f(x+h)=f(x)+df_x(h)+\alpha(h)
    \end{equation}
    et
    \begin{equation}
        g\big( f(x)+u \big)=g\big( f(x) \big)+dg_{f(x)}(u)+\beta(u)
    \end{equation}
    où la fonction \( \alpha\) a la propriété que
    \begin{equation}
        \lim_{h\to 0} \frac{ \| \alpha(h) \| }{ \| h \| }=0;
    \end{equation}
    et la même chose pour \( \beta\). La fonction composée en \( x+h\) s'écrit donc
    \begin{equation}    \label{EqCXcfhfH}
        (g\circ f)(x+h)=g\big( f(x)+df_x(h)+\alpha(h) \big)=g\big( f(x) \big)+dg_{f(x)}\big( df_x(h)+\alpha(h) \big)+\beta\big( df_x(h)+\alpha(h) \big).
    \end{equation}
    Nous montrons que tous les «petits» termes de cette formule peuvent être groupés. D'abord si \( h\) est proche de \( 0\), nous avons
    \begin{equation}
        \frac{ \| df_x(h)+\alpha(h) \| }{ \| h \| }\leq\frac{ \| df_x \|\| h \| }{ \| h \| }+\frac{ \| \alpha(h) \| }{ \| h \| }.
    \end{equation}
    Si \( h\) est petit, le second terme est arbitrairement petit, donc en prenant n'importe que \( M>\| df_x \|\) nous avons
    \begin{equation}
        \frac{ \| df_x(h)+\alpha(h) \| }{ \| h \| }\leq M.
    \end{equation}
    Par ailleurs, nous avons
    \begin{equation}
        \frac{ \| \beta\big( df_x(h)+\alpha(h) \big) \| }{ \| h \| }=\frac{  \| \beta\big( df_x(h)+\alpha(h) \big) \|  }{ \| df_x(h)+\alpha(h) \| }\frac{  \| df_x(h)+\alpha(h) \|  }{ \| h \| }\leq M\frac{  \| \beta\big( df_x(h)+\alpha(h) \big) \|  }{   \| df_x(h)+\alpha(h) \| }.
    \end{equation}
    Vu que la fraction est du type \( \frac{ \beta( f(h)) }{ f(h) }\) avec \( \lim_{h\to 0} f(h)=0\), la fraction tend vers zéro lorsque \( h\to 0\). En posant
    \begin{equation}
        \gamma_1(h)=\beta\big( df_x(h)+\alpha(h) \big)
    \end{equation}
    nous avons \( \lim_{h\to 0} \gamma_1(h)/\| h \|=0\).

    L'autre candidat à être un petit terme dans \eqref{EqCXcfhfH} est traité en utilisant la proposition \ref{PropEDvSQsA} :
    \begin{equation}
        \| dg_{f(x)}\big( \alpha(h) \big) \|\leq \| dg_{f(x)} \|\| \alpha(h) \|.
    \end{equation}
    Donc
    \begin{equation}
        \frac{ \| dg_{f(x)}\big( \alpha(h) \big) \| }{ \| h \| }\leq \| dg_{f(x)} \|\frac{ \| \alpha(h) \| }{ \| h \| },
    \end{equation}
    ce qui nous permet de poser
    \begin{equation}
        \gamma_2(h)=dg_{f(x)}\big( \alpha(h) \big)
    \end{equation}
    avec \( \gamma_2\) qui a la même propriété que \( \gamma_1\). Avec tout cela, en posant \( \gamma=\gamma_1+\gamma_2\) nous récrivons
    \begin{equation}
        (g\circ f)(x+h)=g\big( f(x) \big)+dg_{f(x)}\big( df_x(h) \big)+\gamma(h)
    \end{equation}
    avec \( \lim_{h\to 0} \frac{ \gamma(h) }{ \| h \| }=0\). Tout cela pour dire que
    \begin{equation}
        \lim_{h\to 0} \frac{ (g\circ f)(x+h)-(g\circ f)(x)-\big( dg_{f(x)}\circ df_x \big)(h) }{ \| h \| }=0,
    \end{equation}
    ce qui signifie que 
    \begin{equation}
        d(g\circ f)_x=dg_{f(x)}\circ df_x.
    \end{equation}
    Nous avons donc montré que si \( f\) et \( g\) sont différentiables, alors \( g\circ f\) est différentiable avec différentielle donnée par \eqref{EqHFmezmr}.

    Nous passons à la régularité. Nous supposons maintenant que \( f\) et \( g\) sont de classe \( C^r\) et nous considérons l'application
    \begin{equation}
        \begin{aligned}
            \varphi\colon L(F,G)\times L(E,F)&\to L(E,G) \\
            (A,B)&\mapsto A\circ B. 
        \end{aligned}
    \end{equation}
    Montrons que l'application \( \varphi\) est continue en montrant qu'elle est bornée\footnote{Proposition \ref{PropmEJjLE}.}. Pour cela nous écrivons la norme opérateur
    \begin{equation}
        \| \varphi \|=\sup_{\| (A,B) \|=1}\| \varphi(A,B) \|=\sup_{\| (A,B) \|=1}\| A\circ B \|\leq\sup_{\| (A,B) \|=1}\| A \|\| B \|\leq 1.
    \end{equation}
    Pour ce calcul nous avons utilisé le fait que la norme opérateur soit une norme algébrique (proposition \ref{PropEDvSQsA}) ainsi que la définition \ref{DefFAJgTCE} de la norme sur un espace produit pour la dernière majoration. L'application \( \varphi\) est donc continue et donc \(  C^{\infty}\) par le lemme \ref{LemLLvgPQW}. Nous considérons également l'application
    \begin{equation}
        \begin{aligned}
        \psi\colon U&\to L(F,G)\times L(E,F) \\
        x&\mapsto \big( dg_{f(x)},df_x \big). 
        \end{aligned}
    \end{equation}
    Vu que \( f\) et \( g\) sont \( C^1\), l'application \( \psi\) est continue. Ces deux applications \( \varphi\) et \( \psi\) sont choisies pour avoir
    \begin{equation}
        (\varphi\circ\psi)(x)=\varphi\big( dg_{f(x)},df_x \big)=dg_{f(x)}\circ df_x,
    \end{equation}
    c'est à dire \( \varphi\circ\psi=d(g\circ f)\). Les applications \( \varphi\) et \( \psi\) étant continues, l'application \( d(g\circ f)\) est continue, ce qui prouve que \( g\circ f\) est \( C^1\).

    Si \( f\) et \( g\) sont \( C^r\) alors \( dg\in C^{r-1}\) et \( dg\circ f\in C^{r-1}\) où il ne faut pas se tromper : \( dg\colon F\to L(F,G)\) et \( f\colon U\to F\); la composée est \( dg\circ f\colon x\mapsto dg_{f(x)}\in L(F,G)\). 
    
    Pour la récurrence nous supposons que \( f,g\in C^{r-1}\) implique \( g\circ f\in C^{r-1}\) pour un certain \( r\geq 2\) (parce que nous venons de prouver cela avec \( r=1\) et \( r=2\)). Soient \( f,g\in C^r\) et montrons que \( g\circ f\in C^r\). Par la proposition \ref{PropOYtgIua} nous avons
    \begin{equation}
        \psi=dg\circ f\times df\in C^{r-1},
    \end{equation}
    et donc \( d(g\circ f)=\varphi\circ\psi\in C^{r-1}\), ce qui signifie que \( g\circ f\in C^r\).
\end{proof}

\begin{lemma}[Leibnitz pour les formes bilinéaires\cite{SNPdukn}]\label{LemFRdNDCd}
    Si \( B\colon E\times F\to G\) est bilinéaire et continue, elle est \(  C^{\infty}\) et
    \begin{equation}    \label{EqXYJgDBt}
        dB_{(x,y)}(u,v)=B(x,v)+B(u,y).
    \end{equation}
\end{lemma}

\begin{proof}
    D'abord le membre de droite de \eqref{EqXYJgDBt} est une application linéaire et continue, donc c'est un bon candidat à être différentielle. Nous allons prouver que ça l'est, ce qui prouvera la différentiabilité de \( B\). Avec ce candidat, le numérateur de la définition \eqref{EqIQuRGmO} s'écrit dans notre cas
    \begin{equation}
        B\big( (x,y)+(u,v) \big)-B(x,y)-B(x,v)-B(u,y)=B(u,v).
    \end{equation}
    Il reste à voir que 
    \begin{equation}
        \lim_{ (u,v)\to (0,0) } \frac{ B(u,v) }{ \| (u,v) \| }=0
    \end{equation}
    Par l'équation \eqref{EqYLnbRbC} nous avons
    \begin{equation}
        \frac{ \| B(u,v) \| }{ \| (u,v) \| }\leq \frac{ \| B \|\| u \|\| v \| }{ \| u \| }=\| B \|\| v \|
    \end{equation}
    parce que \( \| (u,v) \|\geq \| u \|\). À partir de là il est maintenant clair que
    \begin{equation}
        \lim_{(u,v)\to (0,0)}\frac{ \| B(u,v) \| }{ \| (u,v) \| }=0,
    \end{equation}
    ce qu'il fallait.
\end{proof}

\begin{proposition}[Règle de Leibnitz\cite{SNPdukn}]
    Soient \( E,F_1,F_2\) des espaces vectoriels normés, \( U\) ouvert dans \( E\) et des applications de classe \( C^r\) (\( r\geq 1\))
    \begin{subequations}
        \begin{align}
            f_1\colon U\to F_1\\
            f_2\colon U\to F_2\\
        \end{align}
    \end{subequations}
    et \( B\in\cL(F_1\times F_2,G)\). Alors l'application
    \begin{equation}
        \begin{aligned}
            \varphi\colon U&\to G \\
            x&\mapsto B\big( f_1(x),f_2(x) \big) 
        \end{aligned}
    \end{equation}
    est de classe \( C^r\) et
    \begin{equation}    \label{EqMNGBXWc}
        d\varphi_x(u)=\varphi\big( (df_1)_x(u),f_2(x) \big)+\varphi\big( f_1(x),(df_2)_x(u) \big).
    \end{equation}
\end{proposition}
\index{Leibnitz!applications entre espaces vectoriels normés}

\begin{proof}
    Par hypothèse \( B\) est continue (c'est la définition de l'espace \( \cL\)), et donc \(  C^{\infty}\) par le lemme \ref{LemFRdNDCd}. Par ailleurs la fonction \( f_1\times f_2\) est de classe \( C^r\) parce que \( f_1\) et \( f_2\) le sont et parce que la proposition \ref{PropOYtgIua} le dit. L'application composée \( B\circ(f_1\times f_2)\) est donc également de classe \( C^r\) par le théorème \ref{ThoAGXGuEt}.

    Il ne nous reste donc qu'à prouver la formule \ref{EqMNGBXWc}. En utilisant la différentielle du produit cartésien\footnote{Proposition \ref{PropOYtgIua}.} nous avons
    \begin{equation}
        f\big( B\circ(f_1\times f_2) \big)_x(h)=dB_{(f_1\times f_2)(x)}\big( (df_1)_x(h),(df_2)_x(h) \big).
    \end{equation}
    Nous développons cela en utilisant le lemme \ref{LemFRdNDCd} :
    \begin{subequations}
        \begin{align}
        d\big( B\circ(f_1\times f_2) \big)_x(h)&=dB_{\big( f_1(x),f_2(x) \big)}\big( (df_1)_x(h),(df_2)_x(h) \big)\\
        &=B\big( f_1(x),(df_2)_x(h) \big)+B\big( (df_1)_x(h),f_2(x) \big),
        \end{align}
    \end{subequations}
    comme souhaité.
\end{proof}

%--------------------------------------------------------------------------------------------------------------------------- 
\subsection{Différentielle partielle}
%---------------------------------------------------------------------------------------------------------------------------

\begin{definition}[Différentielle partielle]    \label{VJM_CtSKT}
    Soient \( E\), \( F\) et \( G\) des espaces vectoriels normés et une fonction \( f\colon E\times F\to G\). Nous définissons sa \defe{différentielle partielle}{différentielle!partielle} sur l'espace \( E\) par
    \begin{equation}
        \begin{aligned}
            d_1f_{(x_0,y_0)}\colon E&\to G \\
            u&\mapsto \Dsdd{ f(x_0+tu,y_0 }{t}{0} .
        \end{aligned}
    \end{equation}
    La différentielle \( d_2\) se définit de la même façon.
\end{definition}

\begin{proposition}[\cite{SNPdukn}] \label{PropLDN_nHWDF}
    Soient \( E_1\), \( E_2\) et \( F\) des espaces vectoriels normés, soit un ouvert \( U\subset E_1\times E_2\) et une fonction \( f\colon U\to F\).
    \begin{enumerate}
        \item   \label{ItemRDD_oPmXVi}
            Si \( f\) est différentiable alors les différentielles partielles existent et
            \begin{subequations}
                \begin{align}
                    d_1f_{(x_0,y_0)}(u)=df_{(x_0,y_0)}(u,0)\\
                    d_2f_{(x_0,y_0)}(v)=df_{(x_0,y_0)}(0,v)
                \end{align}
            \end{subequations}
            où \( u\in E_1\) et \( v\in E_2\).
        \item
            Si \( f\) est différentiable alors
            \begin{equation}
                df_{(x_0,y_0)}(u,v)=d_1f_{(x_,y_0)}(u)+d_2f_{(x_0,y_0)}(v).
            \end{equation}
    \end{enumerate}
\end{proposition}

\begin{proof}
    Nous posons \( \alpha=(x_0,y_0)\in U\) et
    \begin{equation}
        \begin{aligned}
            j_{\alpha}^{(1)}\colon E_1&\to E_1\times E_2 \\
            x&\mapsto (x,y_0). 
        \end{aligned}
    \end{equation}
    C'est une fonction de classe \(  C^{\infty}\) et 
    \begin{equation}
        (dj_{\alpha}^{(1)})_{x_0}(u)=\Dsdd{ j_{\alpha}^{(1)}(x_0+tu) }{t}{0}=\Dsdd{ (x_0+tu,y_0) }{t}{0}=(u,0).
    \end{equation}
    D'autre part 
    \begin{subequations}
        \begin{align}
            (d_1f)_{\alpha}(u)&=\Dsdd{ f(x_0+tu,y_0) }{t}{0}\\
            &=\Dsdd{ (f\circ j_{\alpha}^{(1)})(x_0+tu) }{t}{0}\\
            &=\big( d(f\circ j_{\alpha}^{(1)}) \big)_{x_0}(u).
        \end{align}
    \end{subequations}
    À ce moment nous utilisons la règle des différentielles composées \ref{ThoAGXGuEt} pour dire que
    \begin{equation}
        (d_1f)_{\alpha}(u)=df_{j_{\alpha}^{(1)}(x_0)}\circ (dj_{\alpha}^{(1)})_{x_0}(u)=df_{\alpha}(u,0).
    \end{equation}
    Voila qui prouve déjà le point \ref{ItemRDD_oPmXVi}.

    Pour la suite nous considérons les fonctions 
    \begin{equation}
        \begin{aligned}[]
            P_1(x,y)&=x,&&&J_1(u)&=(u,0),\\
            P_2(x,y)&=y,&&&J_2(v)&=(0,v)
        \end{aligned}
    \end{equation}
    et nous avons l'égalité évidente
    \begin{equation}
        J_1\circ P_1+J_2\circ P_2=\mtu
    \end{equation}
    sur \( E_1\times E_2\). En appliquant \( df_{\alpha}\) à cette dernière égalité, en appliquant à \( (u,v)\) et en utilisant la linéarité de \( df_{\alpha}\) nous trouvons
    \begin{subequations}
        \begin{align}
            df_{\alpha}(u,v)&=df_{\alpha}\big( (J_1\circ P_1)(u,v) \big)+df_{\alpha}\big( (J_2\circ P_2)(u,v) \big)\\
            &=df_{\alpha}(u,0)+df_{\alpha}(0,v)\\
            &=(d_1f)_{\alpha}(u)+(d_2f)_{\alpha}(v)
        \end{align}
    \end{subequations}
    où nous avons utilisé le point \ref{ItemRDD_oPmXVi} pour la dernière égalité.
\end{proof}

%--------------------------------------------------------------------------------------------------------------------------- 
\subsection{Formule des accroissements finis}
%---------------------------------------------------------------------------------------------------------------------------

\begin{proposition} \label{PropDQLhSoy}
    Soit \( E\) un espace vectoriel normé. Soient \( a<b\) dans \( \eR\) et deux fonctions
    \begin{subequations}
        \begin{align}
            f\colon \mathopen[ a , b \mathclose]\to E\\
            g\colon \mathopen[ a , b \mathclose]\to \eR
        \end{align}
    \end{subequations}
    continues sur \( \mathopen[ a , b \mathclose]\) et dérivables sur \( \mathopen] a , b \mathclose[\). Si pour tout \( t\in\mathopen] a , b \mathclose[\) nous avons \( \| f'(t) \|\leq g'(t)\) alors
        \begin{equation}
            \| f(b)-f(a) \|\leq g(b)-g(a).
        \end{equation}
\end{proposition}

\begin{proof}
    Soit \( \epsilon>0\) et la fonction
    \begin{equation}
        \begin{aligned}
            \varphi_{\epsilon}\colon \mathopen[ a , b \mathclose]&\to \eR \\
            t&\mapsto \| f(t)-f(a) \|-g(t)-\epsilon t. 
        \end{aligned}
    \end{equation}
    Cela est une fonction continue réelle à variable réelle. En particulier pour tout \( u\in\mathopen] a , b \mathclose[\) la fonction \( \varphi_{\epsilon}\) est continue sur le compact \( \mathopen[ u , b \mathclose]\) et donc y atteint son minimum en un certain point \( c\in\mathopen[ u , b \mathclose]\); c'est le bon vieux théorème de Weierstrass \ref{ThoWeirstrassRn}. Nous commençons par montrer que pour tout \( u\), ledit minimum ne peut être que \( b\). Pour cela nous allons montrer que si \( t\in\mathopen[ u , b [\), alors \( \varphi_{\epsilon}(s)<\varphi_{\epsilon}(t)\) pour un certain \( s>t\). Par continuité si \( s\) est proche de \( t\) nous avons
        \begin{equation}
            \left\|  \frac{ f(s)-f(t) }{ s-t }  \right\|-\frac{ \epsilon }{2}<\| f'(t) \|<g'(t)+\frac{ \epsilon }{2}=\frac{ g(s)-g(t) }{ s-t }+\frac{ \epsilon }{2}.
        \end{equation}
        Ces inégalités proviennent de la limite
        \begin{equation}
            \lim_{s\to t} \frac{ f(s)-f(t) }{ s-t }=f'(t),
        \end{equation}
        donc si \( s\) et \( t\) sont proches,
        \begin{equation}
            \left\| \frac{ f(s)-f(t) }{ s-t }-f'(t) \right\|
        \end{equation}
        est petit. Si \( s>t\) nous pouvons oublier des valeurs absolues et transformer l'inégalité en
        \begin{equation}
            \| f(s)-f(t) \|<g(s)-g(t)+\epsilon(s-t).
        \end{equation}
        Utilisant cela et l'inégalité triangulaire,
        \begin{subequations}
            \begin{align}
                \varphi_{\epsilon}(s)&\leq\| f(s)-f(t) \|+\| f(t)-f(a) \|-g(s)-\epsilon s\\
                &\leq g(s)-g(t)+\epsilon s-\epsilon t+\| f(t)-f(a) \|-g(s)-\epsilon s\\
                &=\varphi_{\epsilon}(t).
            \end{align}
        \end{subequations}
        Donc nous avons bien \( \varphi_{\epsilon}(s)<\varphi_{\epsilon}(t)\) avec l'inégalité stricte. Par conséquent pour tout \( u\in\mathopen] a , b \mathclose[\) nous avons \( \varphi_{\epsilon}(b)<\varphi_{\epsilon}(u)\) et en prenant la limite \( u\to a\) nous avons
        \begin{equation}
            \varphi_{\epsilon}(b)\leq \varphi_{\epsilon}(a).
        \end{equation}
        Cette inégalité donne immédiatement
        \begin{equation}
            \| f(b)-f(a) \|\leq g(b)-g(a)+\epsilon(b-a)
        \end{equation}
         pour tout \( \epsilon>0\) et donc
         \begin{equation}
            \| f(b)-f(a) \|\leq g(b)-g(a).
         \end{equation}
\end{proof}

\begin{theorem}[Théorème des accroissements finis]\label{ThoNAKKght}
    Soient \( E\) et \( F\) des espaces vectoriels normés, \( U \) ouvert dans \( E\) et une application différentiable \( f\colon U\to F\). Pour tout segment \( \mathopen[ a , b \mathclose]\subset U\) nous avons
    \begin{equation}
        \| f(b)-f(a) \|\leq\left( \sup_{x\in\mathopen[ a , b \mathclose]}\| df_x \| \right)\| b-a \|.
    \end{equation}
\end{theorem}
\index{théorème!accroissements finis}
Une version de ce théorème adaptée aux espaces de dimension finie est le théorème \ref{val_medio_2}.

\begin{proof}
    Nous prenons les applications
    \begin{equation}
        \begin{aligned}
            k\colon \mathopen[ 0 , 1 \mathclose]&\to E \\
            t&\mapsto f\big( (1-t)a+tb \big) 
        \end{aligned}
    \end{equation}
    et
    \begin{equation}
        \begin{aligned}
            g\colon \mathopen[ 0 , 1 \mathclose]&\to \eR \\
            t&\mapsto t\sup_{x\in\mathopen[ a , b \mathclose]}\| df_x \|\| b-a \|.
        \end{aligned}
    \end{equation}
    Pour tout \( t\) nous avons \( g'(t)=M\| b-a \|\) où il n'est besoin de dire ce qu'est \( M\). D'un autre côté nous avons aussi
    \begin{equation}
        \begin{aligned}[]
            k'(t)&=\lim_{\epsilon\to 0}\frac{ f\big( (1-t-\epsilon)a+(t+\epsilon)b \big)-f\big( (1-t)a+tb \big) }{ \epsilon }\\
            &=\Dsdd{ f\big( (1-t)a+tb+\epsilon(b-a) \big)  }{\epsilon}{0}\\
            &=df_{(1-t)a+tb}(b-a)
        \end{aligned}
    \end{equation}
    où nous avons utilisé l'hypothèse de différentiabilité de \( f\) sur \( \mathopen[ a , b \mathclose]\) et donc en \( (1-t)a+tb\). Nous avons donc
    \begin{equation}
        \| k'(t) \|\leq \| b-a \|\| df_{(1-t)a+tb} \|\leq M\| b-a \|=g'(t)
    \end{equation}
    La proposition \ref{PropDQLhSoy} est donc utilisable et
    \begin{equation}
        \| k(1)-k(0) \|=g(1)-g(0),
    \end{equation}
    c'est à dire
    \begin{equation}
        \| f(b)-f(a) \|=M\| b-a \|
    \end{equation}
    comme il se doit.
\end{proof}

\begin{proposition} \label{ProFSjmBAt}
    Soient \( E\) et \( F\) des espaces vectoriels normés, \( U \) ouvert dans \( E\) et une application \( f\colon U\to F\). Soient \( a,b\in U\) tels que \( \mathopen[ a , b \mathclose]\subset U\). Nous posons \( u=(b-a)/\| b-a \|\) et nous supposons que pour tout \( x\in\mathopen[ a , b \mathclose]\), la dérivée directionnelle
    \begin{equation}
        \frac{ \partial f }{ \partial u }(x)=\Dsdd{ f(x+tu) }{t}{0}
    \end{equation}
    existe. Nous supposons de plus que \( \frac{ \partial f }{ \partial u }(x)\) est continue en \( x=a\). Alors
    \begin{equation}
        \| f(b)-f(a) \|\leq\left( \sup_{x\in\mathopen[ a , b \mathclose]}\| \frac{ \partial f }{ \partial u }(x) \| \right)\| b-a \|.
    \end{equation}
\end{proposition}

\begin{proof}
    Nous posons évidemment 
    \begin{equation}
        M=\sup_{x\in\mathopen[ a , b \mathclose]}\| \frac{ \partial f }{ \partial u }(x) \| 
    \end{equation}
    et nous considérons les fonctions
    \begin{equation}
        k(t)=f\big( (1-t)a+tb \big)
    \end{equation}
    et
    \begin{equation}
        g(t)=tM\| b-a \|.
    \end{equation}
    Pour alléger les notations nous posons \( x=(1-t)a+tb\) et nous calculons avec un petit changement de variables dans la limite :
    \begin{equation}
        k'(t)=\Dsdd{  f\big( x+\epsilon(b-a) \big)  }{\epsilon}{0}=\| b-a \|\Dsdd{ f\big( x+\frac{ \epsilon }{ \| b-a \| }(b-a) \big) }{\epsilon}{0}=\| b-a \|\frac{ \partial f }{ \partial u }(x),
    \end{equation}
    et donc encore une fois nous avons
    \begin{equation}
        \| k'(t) \|\leq g'(t),
    \end{equation}
    ce qui donne
    \begin{equation}
        \| k(1)-k(0) \|=g(1)-g(0),
    \end{equation}
    c'est à dire
    \begin{equation}
        \| f(b)-f(a) \|\leq \sup_{x\in\mathopen[ a , b \mathclose]}\| \frac{ \partial f }{ \partial u }(x) \|\| b-a \|.
    \end{equation}
\end{proof}

\begin{theorem} \label{ThoOYwdeVt}
    Soient \( E,V\) deux espaces vectoriels normés, une application \( f\colon E\to V\), un point \( a\in E\) tel que pour tout \( u\in E\), la dérivée
    \begin{equation}
        \Dsdd{ f(x+tu) }{t}{0}
    \end{equation}
    existe pour tout \( x\in B(a,r)\) et est continue (par rapport à \( x\)) en \( x=a\). Nous supposons de plus que\quext{Je ne suis pas certain que cette hypothèse soit nécessaire, voir la question \ref{ItemLPrIWZhPg} de la page \pageref{ItemLPrIWZhPg}.}
    \begin{equation}
        \frac{ \partial f }{ \partial u }(a)=0
    \end{equation}
    pour tout \( u\in E\). Alors \( f\) est différentiable en \( a\) et
    \begin{equation}
        df_a=0
    \end{equation}
\end{theorem}

\begin{proof}
    Soit \( \epsilon>0\). Pourvu que \( \| h \|\) soit assez petit pour que \( a+h\in B(a,r)\), la proposition \ref{ProFSjmBAt} nous donne
    \begin{equation}
        \| f(a+h)-f(a) \|\leq \sup_{x\in\mathopen[ a , a+h \mathclose]}\| \frac{ \partial f }{ \partial u }(x) \|  |h |
    \end{equation}
    où \( u=h/\| h \|\). Par continuité de \( \partial_uf(x)\) en \( x=a\) et par le fait que cela vaut \( 0\) en \( x=a\), il existe un \( \delta>0\) tel que si \( \| h \|<\delta\) alors
    \begin{equation}
        \| \frac{ \partial f }{ \partial u }(a+h) \|\leq \epsilon.
    \end{equation}
    Pour de tels \( h\) nous avons
    \begin{equation}
        \| f(a+h)-f(a) \|\leq \epsilon\| h \|,
    \end{equation}
    ce qui prouve que l'application linéaire \( T(u)=0\) convient parfaitement pour faire fonctionner la définition \ref{DefKZXtcIT}.
%
%    Nous ne supposons plus que les dérivées directionnelles de \( f\) sont nulles en \( x=a\). Alors nous posons, pour \( x\in U\),
%    \begin{equation}    \label{EqCUgHXHy}
%        g(x)=f(x)-\Dsdd{ f(a+s(x-a)) }{s}{0}.
%    \end{equation}
%    Le fait que cette fonction soit bien définie est encore un coup de hypothèses sur les dérivées directionnelles de \( f\) qui sont bien définies autour de \( a\). Cette nouvelle fonction \( g\) satisfait à \( \frac{ \partial g }{ \partial v }(a)=0\) pour tout \( v\in E\) parce que
%    \begin{subequations}
%        \begin{align}
%            \frac{ \partial g }{ \partial v }(a)&=\Dsdd{ g(a+tv) }{t}{0}\\
%            &=\Dsdd{ f(a+tv)-\Dsdd{ f\big( a+s(tv) \big) }{s}{0} }{t}{0}\\
%            &=\frac{ \partial f }{ \partial v }(a)-\Dsdd{ t\frac{ \partial f }{ \partial v }(a) }{t}{0}\\
%            &=0.
%        \end{align}
%    \end{subequations}
%    Pour la dérivée par rapport à \( s\) nous avons effectué le changement de variables \( s\to ts\), ce qui explique la présence d'un \( t\) en facteur. La fonction \( g\) est donc différentiable en \( a\).
%
%
% Position 229262367
    % Attention : ce qui suit est faux. Mais il y a peut-être moyen d'adapter.
%\item[Dérivées non nulles]
%
%    Nous allons montrer que la fonction 
%    \begin{equation}
%        l(x)=\Dsdd{ f\big( a+s(x-a) \big) }{t}{0}
%    \end{equation}
%    est différentiable en \( x=a\), de différentielle \( T(u)=l(u+a)\). Cela fournira la différentiabilité de \( f\) parce que \eqref{EqCUgHXHy} donnerait alors \( f\) comme somme de deux fonctions différentiables.
%
%    En premier lieu nous devons montrer que \( T\) ainsi définie est linéaire.
%    
%    Notre but est donc de prouver que
%    \begin{equation}
%        \lim_{h \to 0}\frac{ \| l(x+h)-l(x)-l(h) \| }{ \| h \| }=0.
%    \end{equation}
%    Un premier pas est de calculer
%    \begin{subequations}
%        \begin{align}
%            l(x+h)-l(x)-l(h)&=\lim_{s\to 0}\frac{ f\big( s(x+h) \big)-f(0)-f(sx)+f(0)-f(sh)+f(0) }{ s }\\
%            &=\lim_{s\to 0}\frac{ f\big( s(x+h) \big)-f(sx)-f(sh)+f(0) }{ s }.
%        \end{align}
%    \end{subequations}
%    Ensuite nous étudions le numérateur en utilisant la proposition \ref{ProFSjmBAt}:
%    \begin{subequations}
%        \begin{align}
%            \| f\big( s(x+h) \big)-f(sx)-f(sh)+f(0) \|&\leq  \| f\big( s(x+h) \big)-f(sx)\| + \|f(sh)-f(0) \|  \\
%            &\leq \sup_{z\in\mathopen[ sx , sx+sh \mathclose]}\| \frac{ \partial f }{ \partial h }(z) \|\| sh \|\\
%            &\quad +\sup_{z\in\mathopen[ 0 , sh \mathclose]}\| \frac{ \partial f }{ \partial h }(z) \|\| sh \|.
%        \end{align}
%    \end{subequations}
%    La division par \( s\) se passe bien et nous avons
%    \begin{subequations}
%        \begin{align}
%            \| l(x+h)-l(x)-l(h) \|&\leq \lim_{s\to 0}  \sup_{z\in\mathopen[ sx , sx+sh \mathclose]}\| \frac{ \partial f }{ \partial h }(z) \|\| h \|+ \sup_{z\in\mathopen[ 0 , sh \mathclose]}\| \frac{ \partial f }{ \partial h }(z) \|\| h \|\\
%            &=2\| h \|\| \frac{ \partial f }{ \partial h }(0) \|        \label{SubeqVMMoSDH}\\
%            &=2\| h \|^2\| \frac{ \partial f }{ \partial u }(0) \|
%        \end{align}
%    \end{subequations}
%    où nous avons posé \( u=h/\| h \|\). Pour l'égalité \eqref{SubeqVMMoSDH} nous avons utilisé la continuité de \( \frac{ \partial f }{ \partial h }(z)\) en \( z=0\). Du coup
%    \begin{equation}
%        \lim_{y\to 0} \frac{ \| f(x+h)-f(x)-f(h) \| }{ \| h \| }=\lim_{h\to 0} 2\| h \|\| \frac{ \partial f }{ \partial u }(0) \|=0.
%    \end{equation}
%    Cela prouve que \( l\) est bien différentiable en \( x=0\).
%
%    \end{subproof}
%
\end{proof}

%--------------------------------------------------------------------------------------------------------------------------- 
\subsection{L'inverse, sa différentielle}
%---------------------------------------------------------------------------------------------------------------------------

Si \( E\) est un espace de Banach, nous sommes intéressé à l'espace \( \GL(E)\) des endomorphismes inversibles de \( E\) sur \( E\). Cet ensemble est métrique par la formule usuelle
\begin{equation}
    \| T \|=\sup_{\| x \|=1}\| T(x) \|_E.
\end{equation}

\begin{theorem}[Inverse dans \( \GL(E)\)\cite{laudenbach2000calcul,SNPdukn}]    \label{ThoCINVBTJ}
    Soient \( E\) et \( F\) des espaces vectoriels normés.
    \begin{enumerate}
        \item
        L'ensemble \( \GL(E)\) est ouvert dans \( \End(E)\).
    \item
        L'application inverse
    \begin{equation}
        \begin{aligned}
        i\colon \GL(E,F)&\to \GL(F,E) \\
        u&\mapsto u^{-1} 
        \end{aligned}
    \end{equation}
    est de classe \( C^{\infty}\) et
    \begin{equation}
        di_{u_0}(h)=-u_0^{-1}\circ h\circ u_0^{-1}
    \end{equation}
    pour tout \( h\in\End(E)\)
    \end{enumerate}
\end{theorem}
\index{différentielle!de $u\mapsto u^{-1}$}

\begin{proof}
Nous supposons que \( \GL(E,F)\) n'est pas vide, sinon ce n'est pas du jeu.
        \begin{subproof}

        \item[Cas de dimension finie]

            Si la dimension de \( E\) et \( F\) est finie, elles doivent être égales, sinon il n'y a pas de fonctions inversibles \( E\to F\). L'ensemble \( \GL(E,F)\) est donc naturellement \( \GL(n,\eR)\). Un élément de \( \eM(n,\eR)\) est dans \( \GL(n,\eR)\) si et seulement si son déterminant est non nul. Le déterminant étant une fonction continue (polynomiale) en les entrées de la matrice, l'ensemble \( \GL(n,\eR)\) est ouvert dans \( \eM(n,\eR)\).

            Même idée pour la régularité de la fonction \( i\colon \GL(n,\eR)\to \GL(n,\eR)\), \( X\mapsto X^{-1}\). Les entrées de \( X^{-1}\) sont les cofacteurs de \( X\) divisé par \( \det(X)\), et donc des polynômes en les entrées de \( X\) divisés par un polynôme qui ne s'annule pas sur \( \GL(n,\eR)\), et donc sur un ouvert autour de \( X\) et de \( X^{-1}\). Bref, tout est \(  C^{\infty}\).

            Le reste de la preuve parle de la dimension infinie.

        \item[Ouvert autour de l'identité]
            
        Nous commençons par prouver que \( B(\mtu,1)\subset \GL(E)\). Pour cela il suffit de remarquer que si \( \| u \|<1\) alors le lemme \ref{PropQAjqUNp} nous donne un inverse de \( (1+u)\) en la personne de \( \sum_{k=0}^{\infty}(-u)^k\).

    \item[Ouvert en général]

        Soit maintenant \( u_0\in\GL(E)\). Si \( \| u \|<\frac{1}{ \| u_0^{-1} \| }\) alors \( \| u_0^{-1}u \|<1\), ce qui signifie que
        \begin{equation}
            \mtu+u_0^{-1}u
        \end{equation}
    est inversible. Mais \( u_0+u=u_0(\mtu+u_0^{-1}u)\), donc \( u_0+u\in\GL(E)\) ce qui signifie que
    \begin{equation}
    B\left( u_0,\frac{1}{ \| u_0^{-1} \| } \right)\subset \GL(E).
    \end{equation}

    \item[Différentielle en l'identité]

    Nous commençons par prouver que \( di_{\mtu}(u)=-u\). Pour cela nous posons 
    \begin{equation}
        \alpha(h)=\sum_{k=2}^{\infty}(-1)^kh^k
    \end{equation}
    et nous calculons
    \begin{equation}
    di_{\mtu}(u)=\Dsdd{ i(\mtu+tu) }{t}{0}=\Dsdd{ \mtu-tu+\alpha(tu) }{t}{0}.
    \end{equation}
    Il suffit de prouver que \( \Dsdd{ \alpha(tu) }{t}{0}=0\) pour conclure que \( di_{\mtu}(u)=-u\). Pour cela, nous remarquons que \( \alpha(0)=0\) et donc que
    \begin{subequations}
        \begin{align}
        \Dsdd{ \alpha(tu) }{t}{0}&=\lim_{t\to 0} \frac{ \alpha(tu)-\alpha(0) }{ t }\\
        &=\lim_{t\to 0} \sum_{k=2}^{\infty}(-1)^k\frac{ (tu)^k }{ t }\\
        &=-\lim_{t\to 0} u\sum_{k=1}^{\infty}(-1)^kt^ku^k.
        \end{align}
    \end{subequations}
    La norme de ce qui est dans la limite est majorée par
    \begin{equation}
    \| u \|\sum_{k=1}^{\infty}\| tu \|^k=\| u \|\left( \frac{1}{ 1-\| tu \| }-1 \right),
    \end{equation}
    et cela tend vers zéro lorsque \( t\to\infty\). Nous avons utilisé la somme \ref{EqRGkBhrX} de la série géométrique. Nous avons bien prouvé que \( di_{\mtu}(u)=-u\).

    \item[Différentielle en général]
    Soit maintenant \( u_0\in\GL(E)\) et \( h\in\End(E)\) tel que \( u_0+h\in \GL(E)\); par le premier point, il suffit de prendre \( \| h \|\) suffisamment petit. Vu que \( u_0+h=u_0(\mtu+u_0^{-1}h)\) nous avons
    \begin{equation}
        (u_0+h)^{-1}=(\mtu+u_0^{-1}h)^{-1}u_0^{-1}.
    \end{equation}
    Nous pouvons donc calculer
    \begin{equation}
        (u_0+h)^{-1}=\big( \mtu-u_0^{-1}h+\alpha(u_0^{-1}h) \big)u_0^{-1}=u_0^{-1}-u_0^{-1}hu_0^{-1}+\alpha(u_0^{-1}h)u_0^{-1},
    \end{equation}
    et ensuite
    \begin{equation}
        di_{u_0}(h)=\Dsdd{ i(u_0+th) }{t}{0}=\Dsdd{ u_0^{-1}-tu_0^{-1}hu_0^{-1}+\alpha(tu_0^{-1}h)u_0^{-1} }{t}{0},
    \end{equation}
    mais nous avons déjà vu que
    \begin{equation}
        \Dsdd{ \alpha(th) }{t}{0}=0,
    \end{equation}
    donc
    \begin{equation}
        di_{u_0}(h)=-u_0^{-1}hu_0^{-1}
    \end{equation}
    Cela donne la différentielle de l'application inverse.

    \item[Continuité de l'inverse]

        L'application \( i\) est continue parce que différentiable.
    \item[L'inverse est \(  C^{\infty}\)]

        Nous allons écrire la fonction inverse comme une composée. Soient les applications
        \begin{equation}
            \begin{aligned}
                B\colon \cL(F,E)\times \cL(F,E)&\to \cL\big( \cL(E,F),\cL(F,E) \big) \\
                B(\psi_1,\psi_2)(A)&= -\psi_1\circ A\circ\psi_2
            \end{aligned}
        \end{equation}
        et
        \begin{equation}
            \begin{aligned}
                \Delta\colon \cL(F,E)&\to \cL(F,E)\times \cL(F,E) \\
                \varphi&\mapsto (\varphi,\varphi) 
            \end{aligned}
        \end{equation}
        Nous avons alors 
        \begin{equation}
            di=B\circ\Delta\circ i.
        \end{equation}
        L'application \( \Delta\) est de classe \(  C^{\infty}\). Nous devons voir que \( B\) l'est aussi. Pour le voir nous commençons par prouver qu'elle est bornée :
        \begin{equation}
            \begin{aligned}[]
                \| B \|&=\sup_{\| \psi_1 \|,\| \psi_2 \|=1}\| B(\psi_1,\psi_2) \|_{\aL\big( L(E,F),L(F,E) \big)}\\
                &=\sup_{  \| \psi_1 \|,\| \psi_2 \|=1 }\sup_{\| A \|=1}\| \psi_1\circ A\circ\psi_2 \|_{L(F,E)}\\
                &\leq \sup_{\| \psi_1 \|,\| \psi_2 \|=1}\sup_{\| A \|=1}\| \psi_1 \|\| A \|\| \psi_2 \|\\
                &\leq 1.
            \end{aligned}
        \end{equation}
        Donc \( B\) est bien bornée et par conséquent continue. Une application bilinéaire continue est \(  C^{\infty}\) par le lemme \ref{LemFRdNDCd}. La décomposition \( di=B\circ \Delta\circ i\) nous donne donc que \( i\in C^{\infty}\) dès que \( i\) est continue, ce que nous avions déjà montré.
        \end{subproof}
\end{proof}

%---------------------------------------------------------------------------------------------------------------------------
\subsection{Normes de matrices et d'applications linéaires}
%---------------------------------------------------------------------------------------------------------------------------
\label{subsecNomrApplLin}


\begin{theorem}
    La norme $2$ d'une matrice peut se calculer de la manière suivante :
    \begin{equation}
        \|A\|_2=\sqrt{\rho(A{^t}A)}
    \end{equation}
\end{theorem}

\begin{proposition} \label{PropMAQoKAg}
    La fonction
    \begin{equation}
        \begin{aligned}
            f\colon \eM(n,\eR)\times \eM(n,\eR)&\to \eR \\
            (X,Y)&\mapsto \tr(X^tY) 
        \end{aligned}
    \end{equation}
    est un produit scalaire sur \( \eM(n,\eR)\).
\end{proposition}
\index{trace!produit scalaire sur \( \eM(n,\eR)\)}
\index{produit!scalaire!sur \( \eM(n,\eR)\)}

\begin{proof}
    Il faut vérifier la définition \ref{DefVJIeTFj}.
    \begin{itemize}
        \item La bilinéairité est la linéarité de la trace.
        \item La symétrie de \( f\) est le fait que \( \tr(A^t)=\tr(A)\).
        \item L'application \( f\) est définie positive parce que si \( X\in \eM\), alors \( X^tX\) est symétrique définie positive, donc diagonalisable avec des nombres positifs sur la diagonale. La trace étant un invariant de similitude, nous avons \( f(X,X)=\tr(X^tX)\geq 0\). De plus si \( \tr(X^tX)=0\), alors \( X^tX=0\) (pour la même raison de diagonalisation). Mais alors \( \| Xu \|=0\) pour tout \( u\in E\), ce qui signifie que \( X=0\).
    \end{itemize}
\end{proof}

\begin{example}
	Soit $m=n$, un point $\lambda$ dans $\eR$ et $T_{\lambda}$ l'application linéaire définie par $T_{\lambda}(x)=\lambda x$. La norme de $T_{\lambda}$ est alors
\[
\|T_{\lambda}\|_{\mathcal{L}}=\sup_{\|x\|_{\eR^m}\leq 1}\|\lambda x\|_{\eR^n}= |\lambda|.
\]
Notez que $T_{\lambda}$ n'est rien d'autre que l'homothétie de rapport $\lambda$ dans $\eR^m$.
\end{example}

\begin{example}
	Considérons la rotation $T_{\alpha}$ d'angle $\alpha$ dans $\eR^2$. Elle est donnée par l'équation matricielle
	\begin{equation}
		T_{\alpha}\begin{pmatrix}
			x	\\ 
			y	
		\end{pmatrix}=\begin{pmatrix}
			\cos\alpha	&	\sin\alpha	\\ 
			-\sin\alpha	&	\cos\alpha	
		\end{pmatrix}\begin{pmatrix}
			x	\\ 
			y	
		\end{pmatrix}=\begin{pmatrix}
			\cos(\alpha)x+\sin(\alpha)y	\\ 
			-\sin(\alpha)x+\cos(\alpha)y	
		\end{pmatrix}
	\end{equation}
	Étant donné que cela est une rotation, c'est une isométrie : $\| T_{\alpha}x \|=\| x \|$. En ce qui concerne la norme de $T_{\alpha}$ nous avons
	\begin{equation}
		\| T_{\alpha} \|=\sup_{x\in\eR^2}\frac{ \| T_{\alpha}(x) \| }{ \| x \| }=\sup_{x\in\eR^2}\frac{ \| x \| }{ \| x \| }=1.
	\end{equation}
	Toutes les rotations dans le plan ont donc une norme $1$. La même preuve tient pour toutes les rotations en dimension quelconque. 
\end{example}

%TODO : le théorème de fuite des compacts qui dit qu'une solution de y'=f(y,t) cesse d'exister seulement si elle tend vers +- infini.

\begin{example}
  Soit $m=n$, un point $b$ dans $\eR^m$ et $T_b$ l'application linéaire définie par $T_b(x)=b\cdot x$ (petit exercice : vérifiez qu'il s'agit vraiment d'une application linéaire).  La norme de $T_b$ satisfait les inégalités suivantes 
 \[
\|T_b\|_{\mathcal{L}}=\sup_{\|x\|_{\eR^m}\leq 1}\|b\cdot x\|_{\eR^n}\leq \sup_{\|x\|_{\eR^m}\leq 1}\|b \|_{\eR^n}\|x\cdot x\|_{\eR^n}\leq\|b \|_{\eR^n},
\]
\[
\|T_b\|_{\mathcal{L}}=\sup_{\|x\|_{\eR^m}\leq 1}\|b\cdot x\|_{\eR^n}\geq \left\|b\cdot \frac{b}{\|b \|_{\eR^n}}\right\|_{\eR^n}=\|b \|_{\eR^n},
\]
donc $\|T_b\|_{\mathcal{L}}=\|b \|_{\eR^n}$.
\end{example}

Une inégalité que nous utiliserons quelque fois dans la suite, y compris dans la proposition qui suit.
\begin{lemma}		\label{LemAvmajAfoisv}
	Soit $T$ une application linéaire de $\eR^m$ vers $\eR^n$. Alors
	\begin{equation}
		\| Av \|_n\leq \| A \|_{\aL}\| v \|_m.
	\end{equation}
	pour tout $v\in\eR^m$.
\end{lemma}

\begin{proof}
	Étant donné que le supremum d'un ensemble est plus grand ou égal à tous les éléments qui le compose,
	\begin{equation}
		\| A \|_{\aL(\eR^m,\eR^n)}=\sup_{x\in\eR^m}\frac{ \| Ax \| }{ \| x \| }\geq\frac{ \| Av \| }{ \| v \| },
	\end{equation}
	d'où le résultat.
\end{proof}

\begin{proposition}
  Soit $T_1$ dans $\mathcal{L}(\eR^m, \eR^n)$ et $T_2$ dans $\mathcal{L}(\eR^n, \eR^p)$ . Alors l'application composée $T_2\circ T_1 $ est dans $\mathcal{L}(\eR^m, \eR^p)$ et sa norme satisfait
  \begin{equation}  \label{EqFwTvwI}
\|T_2\circ T_1 \|_{\mathcal{L}}\leq\|T_1\|_{\mathcal{L}} \|T_2\|_{\mathcal{L}}.
  \end{equation}
\end{proposition}
\begin{proof}
  \begin{itemize}
  \item $T_2\circ T_1 $ est dans $\mathcal{L}(\eR^m, \eR^p)$ : soient $x,\, y$ dans $\eR^m$ et $a,\, b$ dans $\eR$ . 
    \begin{equation}\nonumber
      \begin{aligned}
       T_2&\circ T_1 (ax+by)= T_2\left(T_1(ax+by)\right)=T_2(aT_1(x)+bT_1(y))=\\
&= aT_2\left(T_1(x)\right)+ bT_2\left(T_1(y)\right) = aT_2\circ T_1(x)+ bT_2\circ T_1(y). 
      \end{aligned}
    \end{equation}  
\item
	On veut une estimation de la norme de $T_2\circ T_1 $ :
\[
\|T_2\circ T_1 \|_{\mathcal{L}}= \sup_{x\in\eR^m}\frac{\left\|T_2\left(T_1(x)\right)\right\|_{\eR^p}}{\|x\|_{\eR^m}}\leq  \sup_{x\in\eR^m}\frac{\|T_2\|_{\mathcal{L}}\left\|\left(T_1(x)\right)\right\|_{\eR^p}}{\|x\|_{\eR^m}} =\|T_1\|_{\mathcal{L}} \|T_2\|_{\mathcal{L}}.
\]
  \end{itemize}
\end{proof}

\begin{proposition}
    Une application linéaire de \( \eR^m\) dans \( \eR^n\) est continue.
\end{proposition}

\begin{proof}
      Soit $x$ un point dans $\eR^m$. Nous devons vérifier l'égalité
      \begin{equation}
       \lim_{h\to 0_m}T(x+h)=T(x).
      \end{equation}
    Cela revient à prouver que $\lim_{h\to 0_m}T(h)=0$, parce que $T(x+h)=T(x)+T(h)$. Nous pouvons toujours majorer $\|T(h)\|_n$ par $\|T\|_{\mathcal{L}(\eR^m,\eR^n)}\| h \|_{\eR^m}$ (lemme \ref{LemAvmajAfoisv}). Quand $h$ s'approche de $ 0_m $ sa norme $\|h\|_m$ tend vers $0$, ce que nous permet de conclure parce que nous savons que de toutes façons, $\| T \|_{\aL}$ est fini.
\end{proof}
