% This is part of (everything) I know in mathematics
% Copyright (c) 2011-2013,2016-2017
%   Laurent Claessens
% See the file fdl-1.3.txt for copying conditions.

%--------------------------------------------------------------------------------------------------------------------------- 
\subsection{Exemples : intégration par partie}
%---------------------------------------------------------------------------------------------------------------------------

Les fonctions trigonométries offrent de nombreux moment d'intégration par partie.

%--------------------------------------------------------------------------------------------------------------------------- 
\subsection{Très modeste approximation de \( \pi\)}
%---------------------------------------------------------------------------------------------------------------------------
\label{SUBSECooPQVKooAOyxOe}

Nous sommes en droit de vouloir une valeur approchée de \( \pi\). 
\begin{lemma}       \label{LEMooJWSGooExmtDA}
    Nous avons l'approximation numérique
    \begin{equation}
        2\sqrt{ 2 }<\pi<4.
    \end{equation}
\end{lemma}

\begin{proof}
    Grace au lemme \ref{LEMooIGNPooPEctJy} nous savons que la fonction \( \sin\) passe de \( 0\) à \( \sqrt{ 2 }/2\) sur un intervalle de taille \( \pi/4\) avec une dérivé majorée par \( 1\). Par conséquent
    \begin{equation}
        \frac{ \pi }{ 4 }>\frac{ \sqrt{ 2 } }{2}
    \end{equation}
    et donc\footnote{Sérieusement, êtes vous capables de trouver une approximation de \( \sqrt{ 2 }\) en ne vous basant que sur des choses vues jusqu'ici ?}
    \begin{equation}
        \pi>2\sqrt{ 2 }\simeq 2.82
    \end{equation}
    De plus la fonction \( \sin\) passe de \( 0\) à \( \sqrt{ 2 }/2\) sur un intervalle de taille \( \pi/4\) avec une dérivée majorée par \( \sqrt{ 2 }/2\), donc
    \begin{equation}
        \frac{ \pi }{ 4 }<\frac{ \sqrt{ 2 }/2 }{ \sqrt{ 2 }/2 },
    \end{equation}
    ce qui donne
    \begin{equation}
        \pi<4.
    \end{equation}
\end{proof}

Pour avoir une meilleur approximation de \( \pi\), nous pouvons remarquer que \( \pi\in\mathopen] 2.82 , 4 \mathclose[\), et que cet intervalle est suffisamment petit pour ne pas recouvrir l'intervalle correspondant pour \( 2\pi\). L'équation \( \cos(x)=-1\) possède donc une unique solution dans cet intervalle (et cette solution est \( \pi\)). Nous pouvons donc faire une dichotomie pour trouver la valeur de \( \pi\), pourvu que nous ayons une façon d'évaluer des valeurs de \( \cos(x)\) de façon pas trop ridicule.

%--------------------------------------------------------------------------------------------------------------------------- 
\subsection{Cercle trigonométrique}
%---------------------------------------------------------------------------------------------------------------------------

Le \href{http://fr.wikiversity.org/wiki/Trigonométrie/Cosinus_et_sinus_dans_le_cercle_trigonométrique}{cercle trigonométrique} est le cercle de rayon $1$ représenté à la figure \ref{LabelFigCercleTrigono}. Sa longueur est $2\pi$.
\newcommand{\CaptionFigCercleTrigono}{Le cercle trigonométrique.}
\input{auto/pictures_tex/Fig_CercleTrigono.pstricks}

Nous verrons plus tard que la longueur de l'arc de cercle intercepté par un angle $\theta$ est égal à $\theta$. Les radians sont donc l'unité d'angle les plus adaptés au calcul de longueurs sur le cercle. 

%TODO : remettre ce lien après le fork
%Voir exercice \ref{exoGeomAnal-0034}.


%---------------------------------------------------------------------------------------------------------------------------
\subsection{Les fonctions tangente et arc tangente}
%---------------------------------------------------------------------------------------------------------------------------

\begin{definition}
    La fonction \defe{tangente}{tangente} est :
    \begin{equation}
        \tan(x)=\frac{ \sin(x) }{ \cos(x) }
    \end{equation}
    où \( \sin\) et \( \cos\) sont de la définition \ref{PROPooZXPVooBjONka}.
\end{definition}
La fonction tangente n'est pas définie sur les points de la forme \( x=\frac{ \pi }{2}+k\pi\), \( k\in \eZ\). Une interprétation géométrique, qui justifie le nom, est donnée sur la figure \ref{LabelFigTgCercleTrigono}.
\newcommand{\CaptionFigTgCercleTrigono}{Interprétation géométrique de la fonction tangente. La tangente de l'angle $\theta$ est positive (et un peu plus grande que $1$) tandis que celle de la tangente de l'angle $\varphi$ est négative.}
\input{auto/pictures_tex/Fig_TgCercleTrigono.pstricks}

\begin{proposition}
    La fonction
    \begin{equation}
        \begin{aligned}
        \tan\colon \mathopen] -\frac{ \pi }{ 2 } , \frac{ \pi }{2} \mathclose[&\to \eR \\
            x&\mapsto \tan(x) 
        \end{aligned}
    \end{equation}
    est une bijection.
\end{proposition}

\begin{proof}
    Le cosinus ne s'annulant pas sur l'intervalle donné, la fonction est bien définie. Nous avons
    \begin{equation}
        \lim_{x\to \pi/2^-} \tan(x)=+\infty
    \end{equation}
    parce que la limite du sinus est \( 1\) est celle du cosinus est zéro par les valeurs positives. Le même raisonnement donne la limite en \( -\pi/2\) qui vaut \( -\infty\). Le théorème des valeurs intermédiaires\footnote{Théorème \ref{ThoValInter}.} dit que la fonction tangente est alors surjective sur \( \eR\).

    Par ailleurs en utilisant les règles de calcul comme la dérivation du quotient \ref{PROPooOUZOooEcYKxn}\ref{ITEMooMUNQooLiKffz} nous trouvons
    \begin{equation}
        \tan'(x)=\tan^2(x)+1,
    \end{equation}
    ce qui nous donne une dérivée partout strictement positive, et donc une fonction strictement croissante et donc injective.
\end{proof}

Le graphe de la fonction tangente est sur la figure \ref{LabelFigPVJooJDyNAg}. % From file PVJooJDyNAg
\newcommand{\CaptionFigPVJooJDyNAg}{Le graphe de la fonction tangente.}
\input{auto/pictures_tex/Fig_PVJooJDyNAg.pstricks}

En ce qui concerne la bijection réciproque nous avons le théorème suivant.
\begin{theorem}     \label{THOooUSVGooOAnCvC}
    La fonction
    \begin{equation}
        \begin{aligned}
        \arctan\colon \eR&\to \left] -\frac{ \pi }{2} , \frac{ \pi }{2} \right[ \\
            x&\mapsto \arctan(x) 
        \end{aligned}
    \end{equation}
    nommée \defe{arc tangente}{arc tangente} est
    \begin{enumerate}
        \item
            impaire et strictement croissante sur \( \eR\).
        \item       \label{ITEMooMNHLooOVhIIb}
            dérivable sur \( \eR\) de dérivée
            \begin{equation}
                \arctan'(x)=\frac{1}{ 1+x^2 }.
            \end{equation}
    \end{enumerate}
\end{theorem}

\begin{proof}
    La fonction sinus étant impaire (visible sur son développement de définition \eqref{EQooCMRFooCTtpge}), et la fonction cosinus étant paire, la fonction tangente est impaire et sa réciproque l'est tout autant.

    La fonction arc tangente est également dérivable (donc continue) par la proposition \ref{PropMRBooXnnDLq} parce que la fonction tangente l'est. Notons qu'ici nous nous sommes restreint à \( \mathopen] -\pi/2 , \pi/2 \mathclose[\). Sinon, le résultat est faux.

    La formule proposée pour la dérivée provient également de la proposition \ref{PropMRBooXnnDLq} et de la dérivée de la tangente :
\end{proof}

\begin{lemma}       \label{LEMooJKIUooEMMOrs}
    Nous avons la valeur remarquable
    \begin{equation}
        \arctan(1/\sqrt{ 3 })=\frac{ \pi }{ 6 }.
    \end{equation}
\end{lemma}

Le nombre \( \arctan(x_0)\) se calcule en cherchant l'angle \( \theta\in\mathopen[ -\frac{ \pi }{2} , \frac{ \pi }{2} \mathclose]\) dont la tangente vaut \( x_0\). Nous obtenons le tableau de valeurs suivant :
\begin{equation*}
    \begin{array}[]{|c|c|c|c|c|}
        \hline
        x&0&\frac{1}{ \sqrt{3} }&1&\sqrt{3}\\
        \hline
        \arctan(x)&0&\frac{ \pi }{ 6 }&\frac{ \pi }{ 4 }&\frac{ \pi }{ 3 }\\
        \hline
    \end{array}
\end{equation*}

En ce qui concerne la représentation graphique de la fonction \( x\mapsto\arctan(x)\), elle s'obtient «en retournant» la partie entre \( -\frac{ \pi }{2}\) et \( \frac{ \pi }{ 2 }\) du graphique de la fonction tangente :
\begin{center}
   \input{auto/pictures_tex/Fig_UQZooGFLNEq.pstricks}
\end{center}

%--------------------------------------------------------------------------------------------------------------------------- 
\subsection{La fonction arc sinus}
%---------------------------------------------------------------------------------------------------------------------------

Nous voulons étudier la fonction
\begin{equation}
    \begin{aligned}
        \sin\colon \eR&\to \mathopen[ -1 , 1 \mathclose] \\
        x&\mapsto \sin(x) 
    \end{aligned}
\end{equation}
et sa réciproque éventuelle.

La fonction sinus est continue sur \( \eR\) mais n'est pas bijective : elle prend une infinité de fois chaque valeur de \( J=\mathopen[ -1 , 1 \mathclose]\). Pour définir une bijection réciproque de la fonction sinus en utilisant le théorème \ref{ThoKBRooQKXThd}, nous devons donc choisir un intervalle à partir duquel la fonction sinus est monotone. Nous choisissons l'intervalle
\begin{equation}
    I=\mathopen[ -\frac{ \pi }{ 2 } , \frac{ \pi }{2} \mathclose].
\end{equation}
La fonction
\begin{equation}
    \begin{aligned}
        \sin\colon \mathopen[ -\frac{ \pi }{2} , \frac{ \pi }{2} \mathclose]&\to \mathopen[ -1 , 1 \mathclose] \\
        x&\mapsto \sin(x) 
    \end{aligned}
\end{equation}
est une bijection croissante et continue. Nous avons donc le résultat suivant.
\begin{theorem}[Définition et propriétés de arc sinus]
    Nous nommons \defe{arc sinus}{arc sinus} la bijection inverse de la fonction \( \sin\colon I\to J\). La fonction
    \begin{equation}
        \begin{aligned}
            \arcsin\colon \mathopen[ -1 , 1 \mathclose]&\to \mathopen[ -\frac{ \pi }{2} , \frac{ \pi }{2} \mathclose] \\
            x&\mapsto \arcsin(x) 
        \end{aligned}
    \end{equation}
    ainsi définie est
    \begin{enumerate}
        \item
            continue et strictement croissante;
        \item
            impaire : pour tout \( x\in\mathopen[ -1 , 1 \mathclose]\) nous avons \( \arcsin(-x)=-\arcsin(x)\).
    \end{enumerate}
\end{theorem}

\begin{proof}
    Nous prouvons le fait que \( \arcsin\) est impaire. Un élément de l'ensemble de définition de \( \arcsin\) est de la forme \( y=\sin(x)\) avec \( x\in\mathopen[ -\pi/2 , \pi/2 \mathclose]\). La relation \eqref{EqHQRooNmLYbF} s'écrit dans notre cas
    \begin{equation}    \label{EqVUWooUwVxVp}
        x=\arcsin\big( \sin(x) \big).
    \end{equation}
    Nous écrivons d'une part cette équation avec \( -x\) au lieu de \( x\) :
    \begin{equation}    \label{EqRLYooIwOvSz}
        -x=\arcsin\big( \sin(-x) \big)=\arcsin\big( -\sin(x) \big)=\arcsin(-y);
    \end{equation}
    et d'autre part nous multiplions \eqref{EqVUWooUwVxVp} par \( -1\) :
    \begin{equation}    \label{EqTGIooDeRYyT}
        -x=-\arcsin\big( \sin(x) \big)=-\arcsin(y).
    \end{equation}
    En égalisant les valeurs \eqref{EqRLYooIwOvSz} et \eqref{EqTGIooDeRYyT} nous trouvons
    \begin{equation}
        \arcsin(-y)=-\arcsin(y),
    \end{equation}
    ce qui signifie que \( \arcsin\) est une fonction impaire.
\end{proof}
Notons que cette preuve repose sur le fait que tout élément de l'ensemble de définition de la fonction arc sinus peut être écrit sous la forme \( \sin(x)\) pour un certain \( x\).

Si \( x_0\in\mathopen[ -1 , 1 \mathclose]\) est donné, calculer \( \arcsin(x_0)\) revient à trouver un angle \( \theta_0\) dans \( \mathopen[ -\frac{ \pi }{2} , \frac{ \pi }{2} \mathclose]\) pour lequel \( \sin(\theta_0)=x_0\). Un tel angle sera forcément unique.

\begin{remark}
  La définition de arc sinus découle du choix de l'intervalle $I$, qui est une convention. Il aurait été possible de faire un choix différent : pourriez vous trouver la réciproque de la fonction sinus sur l'intervalle $[\pi/2, 3\pi/2]$ ? Le mieux est de l'écrire comme une translatée de arc sinus, en utilisant le fait que sinus est une fonction périodique. 
\end{remark}

\begin{example}
    Pour calculer \( \arcsin(1)\), il faut chercher un angle entre \( -\frac{ \pi }{2}\) et \( \frac{ \pi }{ 2 }\) ayant \( 1\) pour sinus : résoudre \( \sin(\theta)=1\). La solution est \( \theta=\frac{ \pi }{2}\) et nous avons donc \( \arcsin(1)=\frac{ \pi }{2}\).
\end{example}

À l'aide des valeurs remarquables de la fonction sinus nous obtenons le tableau suivant de valeurs remarquables pour l'arc sinus.
\begin{equation*}
    \begin{array}[]{|c|c|c|c|c|c|}
        \hline
        x&0&\frac{ 1 }{2}&\frac{ \sqrt{2} }{2}&\frac{ \sqrt{3} }{2}&1\\
          \hline
          \arcsin(x)&0&\frac{ \pi }{ 6 }&\frac{ \pi }{ 4 }&\frac{ \pi }{ 3 }&\frac{ \pi }{ 2 }\\ 
          \hline 
           \end{array}
\end{equation*}
Les autres valeurs remarquables peuvent être déduites du fait que l'arc sinus est une fonction impaire.

En ce qui concerne la dérivabilité de la fonction arc sinus, en application de la proposition \ref{PropMRBooXnnDLq} elle est dérivable en tout \( y=\sin(x)\) tel que \( \sin'(x)\neq 0\), c'est à dire tel que \( \cos(x)\neq 0\). Or \( \cos(x)=0\) pour \( x=\pm\frac{ \pi }{2}\), ce qui correspond à \( y=\sin(\pm\frac{ \pi }{2})=\pm 1\). La fonction arc sinus est donc dérivable sur \( \mathopen] -1 , 1 \mathclose[\). Nous avons donc la propriété suivante pour la dérivabilité.

\begin{proposition}
    La fonction arc sinus est continue sur \( \mathopen[ -1 , 1 \mathclose]\) et dérivable sur \( \mathopen] -1 , 1 \mathclose[\). Pour tout \( y\in\mathopen] -1 , 1 \mathclose[\), la dérivée est donnée par la formule \eqref{EqWWAooBRFNsv}, qui dans ce cas s'écrit
        \begin{equation}
            \arcsin'(y)=\frac{1}{ \cos\big( \arcsin(y) \big) }=\frac{1}{ \sqrt{1-y^2} }.
        \end{equation}
\end{proposition}
La dernière égalité viens du fait que si $x=\arcsin(y)$ alors $y = \sin(x)$ et $\cos(x)= \sqrt{1-\sin^2(x)} = \sqrt{1-y^2}$. 

Pour comprendre la dernière égalité, remarquer que dans le dessin suivant, \( \theta=\arcsin(y)\), donc $y = \sin(\theta)$, et \( x=\cos(\theta)\).
\begin{center}
    \input{auto/pictures_tex/Fig_BIFooDsvVHb.pstricks}
\end{center}

Notons enfin que le graphe de la fonction arc sinus est donné à la figure \ref{LabelFigFGRooDhFkch}. % From file FGRooDhFkch
\newcommand{\CaptionFigFGRooDhFkch}{Le graphe de la fonction \( x\mapsto \arcsin(x)\)}
\input{auto/pictures_tex/Fig_FGRooDhFkch.pstricks}

%--------------------------------------------------------------------------------------------------------------------------- 
\subsection{La fonction arc cosinus}
%---------------------------------------------------------------------------------------------------------------------------

Nous voulons étudier la fonction
\begin{equation}
        \cos\colon \eR\to \mathopen[ -1 , 1 \mathclose] 
\end{equation}
et son éventuelle réciproque. Encore une fois il n'est pas possible d'en prendre la réciproque globale parce que ce n'est pas une bijection; ne fut-ce que parce qu'elle est périodique (proposition \ref{PROPooFRVCooKSgYUM}). Nous choisissons de considérer l'intervalle \( \mathopen[ 0 , \pi \mathclose]\) sur lequel la fonction cosinus est continue et strictement monotone décroissante.

Nous avons alors le résultat suivant :

\begin{propositionDef}     \label{PROPooZOZHooSMoYQD}
    Pour définir la fonction arcsinus.  
    
    \begin{enumerate}
        \item
    La fonction
    \begin{equation}
            \cos\colon \mathopen[ 0 , \pi \mathclose]\to \mathopen[ -1 , 1 \mathclose] 
    \end{equation}
    est une bijection continue strictement décroissante. 
    \item
    Sa bijection réciproque est la fonction 
    \begin{equation}
            \arccos\colon \mathopen[ -1 , 1 \mathclose]\to \mathopen[ 0 , \pi \mathclose] \\
    \end{equation}
    nommée \defe{arc cosinus}{arc cosinus}.
    \item
        La fonction arc cosinus est continue, strictement décroissante.
    \item
        Elle est dérivable et pour tout \( y\in\mathopen] -1 , 1 \mathclose[\), sa dérivée est donnée par
    \begin{equation}
        \arccos'(y)=\frac{1}{ -\sin\big( \arccos(y) \big) }=\frac{ -1 }{ \sqrt{1-y^2} }.
    \end{equation}
    \end{enumerate}
\end{propositionDef}

\begin{proof}
    La fonction cosinus est continue et même de classe \(  C^{\infty}\) par la proposition \ref{PROPooZXPVooBjONka}. Elle est strictement décroissant parce que sa dérivée (\( -\sin\)) y est strictement positive (strictement à dans l'intérieur du domaine).

    Le fait que arc cosinus soit une bijection continue strictement monotone est dans le théorème de la bijection \ref{ThoKBRooQKXThd}. La dérivabilité et la formule sont de la proposition \ref{PropMRBooXnnDLq}.
\end{proof}

Pour \( y_0\in\mathopen[ -1 , 1 \mathclose]\), trouver la valeur de \( \arccos(y_0)\) revient à résoudre l'équation \( \cos(x_0)=y_0\). Cela nous permet de construire une tableau de valeurs :
\begin{equation*}
    \begin{array}[]{|c|c|c|c|c|c|c|c|c|c|}
        \hline
        x&-1&-\frac{ \sqrt{3} }{2}&-\frac{ \sqrt{2} }{2}&-\frac{ 1 }{2}&0&\frac{ 1 }{2}&\frac{ \sqrt{2} }{2}&\frac{ \sqrt{3} }{2}&1\\
          \hline
          \arccos(x)&\pi&\frac{ 5\pi }{ 6 }&\frac{ 3 }{ 4 }\pi&\frac{ 2 }{ 3 }\pi&\frac{ 1 }{2}\pi&\frac{ \pi }{ 3 }&\frac{1}{ 4 }\pi&\frac{1}{ 6 }\pi&0\\
          \hline 
           \end{array}
\end{equation*}

\begin{remark}
    Certes la fonction cosinus est paire (vue sur \( \eR\)), mais la fonction arc cosinus ne l'est pas car elle est une bijection entre \(\mathopen[ -1 , 1 \mathclose]\) et \(\mathopen[ 0 , \pi \mathclose]\).
\end{remark}

\begin{example}
    Cherchons \( \arccos(\frac{ 1 }{2})\). Il faut trouver un angle \( \theta\in\mathopen[ 0 , \pi \mathclose]\) tel que \( \cos(\theta)=\frac{ 1 }{2}\). La solution est \( \theta=\frac{ \pi }{ 3 }\). Donc \( \arccos(\frac{ 1 }{2})=\frac{ \pi }{ 3 }\).

    Il n'est cependant pas immédiat d'en déduire la valeur de \( \arccos(-\frac{ 1 }{2})\). En effet \( \theta=\arccos(-\frac{ 1 }{2})\) si et seulement si \( \cos(\theta)=-\frac{ 1 }{2}\) avec \( \theta\in\mathopen[ 0 , \pi \mathclose]\). La solution est \( \theta=\frac{ 2\pi }{ 3 }\).
\end{example}

En ce qui concerne la représentation graphique, il suffit de tracer la fonction cosinus entre \( 0\) et \( \pi\) puis de prendre le symétrique par rapport à la droite \( y=x\).

\begin{center}
    \input{auto/pictures_tex/Fig_GMIooJvcCXg.pstricks}
\end{center}

%--------------------------------------------------------------------------------------------------------------------------- 
\subsection{Une meilleure approximation de \( \pi\)}
%---------------------------------------------------------------------------------------------------------------------------

Nous avions laissé le nombre \( \pi\) avec l'approximation assez minable de \( 2\sqrt{ 2 }<\pi<4\) en le lemme \ref{LEMooJWSGooExmtDA}. Nous pouvons maintenant faire nettement mieux.

Le lemme \ref{LEMooJKIUooEMMOrs} donne
\begin{equation}
    \arctan(1/\sqrt{ 3 })=\pi/6
\end{equation}
et l'idée est de donner un développement de \( \arctan\) autour de zéro, de l'évaluer en \( 1/\sqrt{ 3 }\) et d'égaliser le résultat à \( \pi/6\). Tout cela donne lieux à des calcules peut-être fastidieux, mais comme un gars l'a fait dès l'an 1424\cite{ooOMUNooGROVUu} pour trouver \( 16\) décimales corrects, nous faisons comme si c'était facile.

Le développement en série de Taylor \ref{SubEqsDevTauil} d'arc tangente autour de \( x=0\) est donné par
\begin{equation}
    \arctan(x)=\sum_{k=0}^{\infty}\frac{ (-1)^{k}x^{2k+1} }{ 2k+1 },
\end{equation}
valable pour \( x\in \mathopen] -1 , 1 \mathclose[\). Avec cela nous avons
\begin{equation}
    \arctan(\frac{1}{ \sqrt{ 3 } })=\sum_{k=0}^{\infty}\frac{ (-1)^k }{ (2k+1)3^k }\times \frac{1}{ \sqrt{ 3 } }=\frac{ \pi }{ 6 },
\end{equation}
et donc
\begin{equation}
    \pi=\frac{ 6 }{ \sqrt{ 3 } }\sum_{k=0}^{\infty}\frac{ (-1)k }{ (2k+1)3^k }.
\end{equation}

Pour donner une idée du fait que ça fonctionne pas mal, voici le calcul pour quelque termes :
\lstinputlisting{tex/sage/sageSnip012.sage}
Calculer \( 5\) termes donne déjà \( 3.15\). Et on est à \( 10^{-6}\) de la bonne réponse avec \( 20\) termes. Et avec $58$ termes, on n'est à \( 10^{-16}\).

\begin{probleme}
    Pour bien faire, il faudrait étudier le reste et donner un encadrement.
\end{probleme}

%---------------------------------------------------------------------------------------------------------------------------
\subsection{Les coordonnées polaires}
%---------------------------------------------------------------------------------------------------------------------------

On a vu qu'un point $M$ dans $\eR^2$ peut être représenté par ses abscisses $x$ et ses ordonnées $y$. Nous pouvons également déterminer le même point $M$ en donnant un angle et une distance comme montré sur la figure \ref{LabelFigJWINooSfKCeA}.
\newcommand{\CaptionFigJWINooSfKCeA}{Un point en coordonnées polaires est donné par sa distance à l'origine et par l'angle qu'il faut avec l'horizontale.}
\input{auto/pictures_tex/Fig_JWINooSfKCeA.pstricks}


Le même point $M$ peut être décrit indifféremment avec les coordonnées $(x,y)$ ou bien avec $(r,\theta)$.

\begin{remark}
	L'angle $\theta$ d'un point n'étant a priori défini qu'à un multiple de $2\pi$ près, nous convenons de toujours choisir un angle $0\leq\theta<2\pi$. Par ailleurs l'angle $\theta$ n'est pas défini si $(x,y)=(0,0)$.

	La coordonnée $r$ est toujours positive.
\end{remark}

En utilisant la trigonométrie, il est facile de trouver le changement de variable qui donne $(x,y)$ en fonction de $(r,\theta)$:
\begin{subequations}		\label{EqrthetaxyPoal}
	\begin{numcases}{}
		x=r\cos(\theta)\\
		y=r\sin(\theta).
	\end{numcases}
\end{subequations}

%///////////////////////////////////////////////////////////////////////////////////////////////////////////////////////////
\subsubsection{Transformation inverse : théorie}
%///////////////////////////////////////////////////////////////////////////////////////////////////////////////////////////

Voyons la question inverse : comment retrouver $r$ et $\theta$ si on connait $x$ et $y$ ? Tout d'abord,
\begin{equation}
	r=\sqrt{x^2+y^2}
\end{equation}
parce que la coordonnée $r$ est la distance entre l'origine et $(x,y)$. Comment trouver l'angle ? Nous supposons $(x,y)\neq (0,0)$. Si $x=0$, alors le point est sur l'axe vertical et nous avons
\begin{equation}
	\theta=\begin{cases}
		\pi/2	&	\text{si }y>0\\
		3\pi/2	&	 \text{si }y<0
	\end{cases}
\end{equation}
Notez que si $y<0$, conformément à notre convention $\theta\geq 0$, nous avons noté $\frac{ 3\pi }{2}$ et non $-\frac{ \pi }{ 2 }$.

Supposons maintenant le cas général avec $x\neq 0$. Les équations \eqref{EqrthetaxyPoal} montrent que
\begin{equation}
	\tan(\theta)=\frac{ y }{ x }.
\end{equation}
Nous avons donc
\begin{equation}
	\theta=\tan^{-1}\left( \frac{ y }{ x } \right).
\end{equation}
La fonction inverse de la fonction tangente est celle définie plus haut.

%///////////////////////////////////////////////////////////////////////////////////////////////////////////////////////////
\subsubsection{Transformation inverse : pratique}
%///////////////////////////////////////////////////////////////////////////////////////////////////////////////////////////

Le code suivant utilise \href{http://www.sagemath.org}{Sage}.

\lstinputlisting{tex/frido/calculAngle.py}

Son exécution retourne :
\begin{verbatim}
(sqrt(2), 1/4*pi)
(sqrt(5), pi - arctan(1/2))
(6, 1/6*pi)
\end{verbatim}
Notez que ce sont des valeurs \emph{exactes}. Ce ne sont pas des approximations, Sage travaille de façon symbolique.

Voici un tableau qui rappelle les valeurs à retenir pour les fonctions sinus, cosinus et tangente.\label{PGooIMQFooTnBdIl}
\begin{equation*}
    \begin{array}[]{|c|c|c|c|}
      \hline
      x&\sin(x)&\cos(x)&\tan(x)\\
      \hline
      0&0&1&0\\
      \hline
      \pi/6&1/2&\sqrt{3}/2&\sqrt{3}/3\\
      \hline
      \pi/6&1/2&\sqrt{3}/2&\sqrt{3}/3\\
      \hline
      \pi/4&\sqrt{2}/2&\sqrt{2}/2&1\\
      \hline
      \pi/3&\sqrt{3}/2&1/2&\sqrt{3}\\
      \hline
      \pi/2&1&0&\text{N.D.}\\
      \hline
    \end{array}
\end{equation*}
où «N.D.»  signifie «non défini».

Rappelons le graphe de la fonction sinus :
\begin{center}
   \input{auto/pictures_tex/Fig_TWHooJjXEtS.pstricks}
\end{center}
celui de la fonction cosinus :
\begin{center}
   \input{auto/pictures_tex/Fig_JJAooWpimYW.pstricks}
\end{center}


    \begin{lemma}
      Pour toute valeur de $x\in \eR$ on a $|\sin(x)|\leq |x|$. 
    \end{lemma}

    \begin{proof}
        Nous séparons des cas en fonction des valeurs.
    \begin{itemize}
    \item Si $0\leq x\leq \pi/2$ alors le sinus de $x$ est la longueur du cathète verticale du triangle rectangle de sommets $O = (0,0)$, $A = (\cos(x), \sin(x))$ et $B = (\cos(x), 0)$. Le triangle de sommets $A$, $B$ et $C = (1, 0)$ est aussi rectangle et nous savons que chacun des cathètes ne peut pas être plus long que l'hypoténuse. Donc $\sin(x)$ est inférieur à la longueur du segment $AC$. À son tour le segment $AC$ ne peut pas être plus long que l'arc de cercle $\wideparen{A0C}$, car le chemin le plus court entre deux points du plan est toujours donné par un morceau de droite. La longueur de l'arc de cercle $\frown{AC}$ est \emph{par définition} la mesure en radiants de l'angle $\widehat{AOC}$, qui est $x$ et on a l'inégalité $\sin(x)\leq x$. 
    \item Si $-\pi/2\leq x\leq 0$ le m\^eme raisonnement que au point précédent permet de conclure que $\sin(x)\leq |x|$.
    \item Nous savons par ailleurs que la fonction sinus prend ses valeurs dans l'intervalle $[-1,1]$ et donc pour tout $x$ tel que $|x|\geq \pi/2 \equiv 1,57\ldots$ on a forcement $|\sin(x)|\leq |x|$.  
    \end{itemize}
    \end{proof}

\begin{proposition}
    Nous avons la limite
    \begin{equation}\label{sinsurx}
      \lim_{x\to 0} \frac{\sin(x)}{x} = 1.
    \end{equation}
\end{proposition}

\begin{proof}
    On commence par observer que la fonction $g(x)=\frac{\sin(x)}{x}$ est un rapport entre deux fonction impaires et est donc une fonction paire. Nous pouvons alors nous réduire à considérer le cas où $x$ est positif. La première étape de cette preuve nous dit que $g(x)\leq 1$ pour tout $x\in\eR^{+,*}$. 

    Nous voulons étudier le comportement de $g$ dans un voisinage de $0$. Nous pouvons alors supposer que $x$ soit inférieur à $\pi/2$. Soit $D = (1, \tan (x))$. On voit très bien dans le dessin que l'aire du triangle de sommets $O$, $D$ et $C$ est supérieure à l'aire du secteur circulaire de sommets $O$, $A$ et $C$. Ces deux aires peuvent \^etre calculées très facilement et nous obtenons
    \begin{equation*}
      \frac{\sin(x)}{2\cos(x)} \geq \frac{x}{2}.
    \end{equation*}
    À partir de cette dernière inégalité nous pouvons écrire 
    \begin{equation*}
      1\geq \frac{\sin(x)}{x}\geq \cos(x).
    \end{equation*}
    En prenant la limite lorsque $x$ tend vers $0$ dans les trois membres de l'inégalité la règle de l'étau nous permet d'obtenir la limite remarquable  \eqref{sinsurx}. 
\end{proof}

%---------------------------------------------------------------------------------------------------------------------------
\subsection{Forme polaire ou trigonométrique des nombres complexes}
%---------------------------------------------------------------------------------------------------------------------------

Dans le plan de Gauss, le module d'un complexe $z$ représente la distance entre $0$ et $z$. On appelle \Defn{argument} de $z$ (noté $\arg z$) l'angle (déterminé à $2\pi$ près) entre le demi-axe des réels positifs et la demi-droite qui part de $0$ et passe par $z$. Le module et l'argument d'un complexe permettent de déterminer univoquement ce complexe puisqu'on a la formule
\[z = a + bi = \module z \left( \cos(\arg(z)) + i \sin(\arg(z)) \right)\]

L'argument de $z$ se détermine via les formules 
\[\frac a {\module z} = \cos(\arg(z)) \quad \frac b {\module z} = \sin(\arg(z))\]
ou encore par la formule
\[
\frac b a = \tan(\arg(z)) \quad \text{en vérifiant le quadrant.}
\]
La vérification du quadrant vient de ce que la tangente ne détermine l'angle qu'à $\pi$ près.

%--------------------------------------------------------------------------------------------------------------------------- 
\subsection{Angle entre deux vecteurs}
%---------------------------------------------------------------------------------------------------------------------------

Si $a$ et $b$ sont des réels, l'inégalité $| a |\leq b$ peut se développer en une double inégalité
\begin{equation}
	-b\leq a\leq b.
\end{equation}
L'inégalité de Cauchy-Schwarz \eqref{EQooZDSHooWPcryG} devient alors
\begin{equation}
	-\| X \|\| Y \|\leq X\cdot Y\leq\| X \|\| Y \|.
\end{equation}
Si $X\neq 0$ et $Y\neq 0$, nous avons alors
\begin{equation}
	-1\leq\frac{ X\cdot Y }{ \| X \|\| Y \| }\leq 1.
\end{equation}
Il existe donc par la proposition \ref{PROPooZOZHooSMoYQD} un angle $\theta\in\mathopen[ 0 , \pi \mathclose]$ tel que
\begin{equation}		\label{eqDefAngleVect}
	\cos(\theta)=\frac{ X\cdot Y }{ \| X \|\| Y \| }.
\end{equation}

\begin{definition}      \label{DEFooSVDZooPWHwFQ}
L'angle ainsi défini est l'\defe{angle}{angle!entre vecteurs} entre $X$ et $Y$. La définition \eqref{eqDefAngleVect} est souvent utilisée sous la forme
\begin{equation}		\label{eqPropCosThet}
	X\cdot Y=\| X \|\| Y \|\cos(\theta).
\end{equation}
\index{cosinus!angle entre deux vecteurs}
\end{definition}

Notez que les angles sont toujours des angles plus petits ou égaux à \unit{180}{\degree}.

La longueur de la projection du point $P$ sur la droite horizontale va naturellement être égale à $\cos(\theta)$. En effet, si nous notons $X$ un vecteur horizontal de norme $1$, cette projection est donné par $P\cdot X$. Mais en reprenant l'équation \eqref{eqPropCosThet}, nous voyons que
\begin{equation}
	P\cdot X=\| P \|\| X \|\cos(\theta),
\end{equation}
tandis qu'ici nous avons $\| P \|=\| X \|=1$.

Nous appelons $\sin(\theta)$ la longueur de la projection sur l'axe vertical.

Quelques dessins nous convainquent que 
\begin{equation}
	\begin{aligned}[]
		\sin(\theta+2\pi)&=\sin(\theta)&\cos(\theta+2\pi)&=\sin(\theta),\\
		\sin(\theta+\frac{ \pi }{2})&=\cos(\theta)&\cos(\theta+\frac{ \pi }{2})&=-\sin(\theta),\\
		\sin(\pi-\theta)&=\sin(\theta)&\cos(\pi-\theta)&=-\cos(\theta).
	\end{aligned}
\end{equation}
Le théorème de Pythagore nous montre aussi l'importante relation
\begin{equation}
	\sin^2(\theta)+\cos^2(\theta)=1.
\end{equation}

Quelques valeurs remarquables pour les sinus et cosinus :
\begin{equation}
	\begin{aligned}[]
		\sin 0&=0,&\sin\frac{ \pi }{ 6 }&=\frac{ 1 }{2},&\sin\frac{ \pi }{ 4 }&=\frac{ \sqrt{2} }{2},&\sin\frac{ \pi }{ 3 }&=\frac{ \sqrt{3} }{2},&\sin\frac{ \pi }{2}&=1,&\sin\pi&=0\\
		\cos 0&=1,&\cos\frac{ \pi }{ 6 }&=\frac{ \sqrt{3} }{2},&\cos\frac{ \pi }{ 4 }&=\frac{ \sqrt{2} }{2},&\cos\frac{ \pi }{ 3 }&=\frac{ 1 }{2},&\cos\frac{ \pi }{2}&=0,&\cos\pi&=-1
	\end{aligned}
\end{equation}

Nous pouvons prouver simplement que $\sin(\unit{30}{\degree})=\frac{ 1 }{2}$ et $\cos(\unit{30}{\degree})=\frac{ \sqrt{3} }{2}$ en s'inspirant de la figure \ref{LabelFigGVDJooYzMxLW}. % From file GVDJooYzMxLW
\newcommand{\CaptionFigGVDJooYzMxLW}{Un triangle équilatéral de côté $1$.}
\input{auto/pictures_tex/Fig_GVDJooYzMxLW.pstricks}

%--------------------------------------------------------------------------------------------------------------------------- 
\subsection{Aire du parallélogramme}
%---------------------------------------------------------------------------------------------------------------------------

\begin{remark}      \label{RemaAireParalProdVect}
    Le nombre $\| a \|\| b \|\sin(\theta)$ est l'aire du parallélogramme formé par les vecteurs $a$ et $b$, comme cela se voit sur la figure \ref{LabelFigBNHLooLDxdPA}. % From file BNHLooLDxdPA
\newcommand{\CaptionFigBNHLooLDxdPA}{Calculer l'aire d'un parallélogramme.}
\input{auto/pictures_tex/Fig_BNHLooLDxdPA.pstricks}
\end{remark}

\begin{normaltext}      \label{NORMooWWOKooWzScnZ}
Si les vecteurs $a$, $b$ et $c$ ne sont pas coplanaires, alors la valeur absolue du produit mixte (voir équation \eqref{EqProduitMixteDet}) $a\cdot(b\times c)$ donne le volume du parallélépipède construit sur les vecteurs $a$, $b$ et $c$.

En effet si $\varphi$ est l'angle entre $b\times c$ et $a$, alors la hauteur du parallélépipède vaut $\| a \|\cos(\varphi)$ parce que la direction verticale est donnée par $b\times c$, et la hauteur est alors la «composante verticale» de $a$. Par conséquent, étant donné que $\| b\times c \|$ est l'aire de la base, le volume du parallélépipède vaut\footnote{Le calcul de ce volume mériterait une certaine réflexion, surtout à partir du moment où nous avons décidé de définir les fonctions trigonométriques à partir de son développement (définition \ref{PROPooZXPVooBjONka}).}
\begin{equation}
    V=\| b\times c\|  \| a \|\cos(\varphi).
\end{equation}
Or cette formule est le produit scalaire de $a$ par $b \times c$; ce dernier étant donné par le déterminant de la matrice formée des composantes de $a$, $b$ et $c$ grâce à la formule \eqref{EqProduitMixteDet}.
\end{normaltext}

La valeur absolue du déterminant 
\begin{equation}        \label{EqDeratb}
    \begin{vmatrix}
        a_1    &   a_2    \\ 
        b_1    &   b_2    
    \end{vmatrix}
\end{equation}
est l'aire du parallélogramme déterminé par les vecteurs $\begin{pmatrix}
    a_1    \\ 
    a_2    
\end{pmatrix}$ et $\begin{pmatrix}
    b_1    \\ 
    b_2    
\end{pmatrix}$. En effet, d'après la remarque \ref{RemaAireParalProdVect}, l'aire de ce parallélogramme est donnée par la norme du produit vectoriel
\begin{equation}
    \begin{pmatrix}
        a_1    \\ 
        a_2    \\ 
        0    
    \end{pmatrix}\times
    \begin{pmatrix}
          b_1  \\ 
        b_2    \\ 
        0    
    \end{pmatrix}=\begin{vmatrix}
        e_x    &   e_y    &   e_z    \\
        a_1    &   a_2    &   0    \\
        b_1    &   b_2    &   0
    \end{vmatrix}=
    \begin{vmatrix}
        a_1    &   a_2    \\ 
        b_1    &   b_2    
    \end{vmatrix}e_z,
\end{equation}
donc la norme $\| a\times b \|$ est bien donnée par la valeur absolue du déterminant \eqref{EqDeratb}.

%+++++++++++++++++++++++++++++++++++++++++++++++++++++++++++++++++++++++++++++++++++++++++++++++++++++++++++++++++++++++++++ 
\section{Angles orientés, rotations}
%+++++++++++++++++++++++++++++++++++++++++++++++++++++++++++++++++++++++++++++++++++++++++++++++++++++++++++++++++++++++++++

Nous avons défini les rotations planes en \ref{DEFooFUBYooHGXphm}, et montré que pour les rotations vectorielles,
nous avions en réalité le groupe \( \SO(2)\) en le corollaire \ref{CORooVYUJooDbkIFY}.

\begin{lemma}       \label{LEMooHRESooQTrpMz}
    Toute rotation admet une matrice de la forme (dans la base canonique)
    \begin{equation}
        \begin{pmatrix}
            \cos(\theta)    &   -\sin(\theta)    \\ 
            \sin(\theta)    &   \cos(\theta)    
        \end{pmatrix}
    \end{equation}
    avec \( \theta\in\mathopen[ 0 , 2\pi \mathclose[\).
\end{lemma}

\begin{proof}
    Soit une matrice \( A=\begin{pmatrix}
        a    &   b    \\ 
        c    &   d    
    \end{pmatrix}\) et imposons qu'elle soit dans \( \SO(2)\). Le fait que \( A\) soit orthogonale impose
    \begin{equation}
        \begin{pmatrix}
            a    &   c    \\ 
            b    &   d    
        \end{pmatrix}\begin{pmatrix}
            a    &   b    \\ 
            c    &   d    
        \end{pmatrix}=\begin{pmatrix}
            a^2+c^2    &   ab+cd    \\ 
            ab+cd    &   b^2+d^2    
        \end{pmatrix}=\begin{pmatrix}
            1    &   0    \\ 
            0    &   1    
        \end{pmatrix}.
    \end{equation}
    Nous en déduisons l'existence et unicité\quext{Entre autres par la proposition \ref{PROPooKSGXooOqGyZj}, mais il faudra être plus précis.} de \( \theta\) et \( \alpha\) dans \( \mathopen[ 0 , 2\pi \mathclose[\) tels que \( a=\cos(\theta)\), \( c=\sin(\theta)\), \( b=\cos(\alpha)\), \( d=\sin(\alpha)\).

    La matrice doit de plus être spéciale, c'est à dire de déterminant \( 1\) : \( ad-cb=1\). Cela donne l'équation trigonométrique
    \begin{equation}
        \cos(\theta)\sin(\alpha)-\sin(\theta)\cos(\alpha)=1
    \end{equation}
    qui a pour solution\quext{Si quelqu'un peut préciser l'unicité, je suis preneur.} \( \alpha=\frac{ \pi }{2}+\theta\), qui donne \( \cos(\alpha)=-\sin(\theta)\) et \( \sin(\alpha)=\cos(\theta)\) et donc
    \begin{equation}
        A=\begin{pmatrix}
            \cos(\theta)    &   -\sin(\theta)    \\ 
            \sin(\theta)    &   \cos(\theta)    
        \end{pmatrix}
    \end{equation}
    qui est bien une matrice de rotation.
\end{proof}

\begin{proposition}[\cite{ooGEXYooMTrOdH}]      \label{PROPooDWIMooQPkobw}
    Si \( u\) et \( v\) sont des vecteurs unitaires\footnote{De norme \( 1\).} de \( \eR^2\) alors il existe une unique rotation\footnote{Définition \ref{}} \( f\) telle que \( f(u)=v\).
\end{proposition}

Notons l'unicité. Nous ne faisons pas de différences entre \( R_{\theta}\) et \( R_{\theta+2\pi}\) et les autres \( R_{\theta+2k\pi}\). En particulier si une rotation \( T\) est donnée, dire «\( T=R_{\theta}\)» ne définit pas un nombre \( \theta\) de façon univoque. Par contre ça définit une classe modulo \( 2\pi\), c'est à dire un élément \( \theta\in \eR/2\pi\).

Nous avons déjà défini le groupe \( \SO(2)\) en la définition \ref{DEFooJLNQooBKTYUy} et nous avons déterminé ses matrices dans \( \eR^2\) en le lemme \ref{LEMooHRESooQTrpMz}. Pour rappel, la \emph{définition} d'une rotation est d'être un élément de \( \SO(2)\).

La proposition \ref{PROPooDWIMooQPkobw} donne une application
\begin{equation}
    T\colon S^1\times S^1\to \SO(2).
\end{equation}
Et nous avons une relation d'équivalence sur \( S^1\times S^1\) donnée par \( (u,v)\sim(u',v')\) si et seulement si il existe \( g\in\SO(2)\) telle que \( g(u)=u'\) et \( g(v)=v'\). 

\begin{definition}[Angle orienté\cite{ooGEXYooMTrOdH}]      \label{DEFooVBKIooWlHvod}
    Les classes de \( S^1\times S^1\) pour cette relation d'équivalence sont les \defe{angles orientés de vecteurs}{angle!orienté de vecteurs}. Nous notons \( [u,v]\) la classe de \( (u,v)\).
\end{definition}

\begin{proposition}     \label{PROPooIWJQooGQJBWR}
    Nous avons \( T(u,v)=T(u',v')\) si et seulement si \( (u,v)\sim(u',v')\).
\end{proposition}

\begin{proof}
    En utilisant la commutativité du groupe \( \SO(2)\) nous avons équivalence entre les affirmations suivantes :
    \begin{itemize}
        \item \( (u,v)\sim (u',v')\)
        \item \( T(u,u')=T(v',v')\)
        \item \( T(u,u')\circ T(u',v)=T(v,v')\circ T(u',v)\)
        \item
            \( T(u,v)=T(u',v')\).
    \end{itemize}
\end{proof}

\begin{proposition}
    Nous avons une bijection
    \begin{equation}
        \begin{aligned}
            S\colon \frac{ S^1\times S^1 }{ \sim }&\to \SO(2) \\
            [u,v]&\mapsto T(u,v). 
        \end{aligned}
    \end{equation}
\end{proposition}

\begin{proof}
    En plusieurs points.
    \begin{subproof}
    \item[\( S\) est bien définie]
        En effet si \( [u,v]=[z,t]\) alors \( T(u,v)=T(z,t)\).
    \item[Injectif]
        Si \( S[u,v]=S[z,t]\) alors \( T(u,v)=T(z,t)\), qui implique \( (u,v)\sim (z,t)\) par la proposition \ref{PROPooIWJQooGQJBWR}.
    \item[Surjectif]
        Nous avons \( R_{\theta}=T(u,R_{\theta}u)\).
    \end{subproof}
\end{proof}

\begin{definition}[Somme d'angles orientés\cite{ooGEXYooMTrOdH}]
    Si \( [u,v]\) et \( [z,t]\) sont des angles orientés, nous définissons la somme par
    \begin{equation}
        [u,v]+[z,t]=S^{-1}\Big( S[u,v]\circ S[z,t] \Big).
    \end{equation}
\end{definition}

\begin{lemma}       \label{LEMooWISVooYsStJp}
    Quelque propriétés des angles plats liées à la somme.
    \begin{enumerate}
        \item
            \( (S^1\times S^1)/\sim\) est un groupe commutatif.
        \item       \label{ITEMooBKTFooWbEvIU}
            Relations de Chasles :
            \begin{equation}
                [u,v]+[v,w]=[u,w].
            \end{equation}
        \item
            \( -[u,v]=[v,u]\).
    \end{enumerate}
\end{lemma}

\begin{proof}
    Pour la relation de Chasles, ça se base sur la propriété correspondante sur \( T\) :
    \begin{subequations}
        \begin{align}
            [u,v]+[v,w]&=S^{-1}\Big( T(u,v)\circ T(v,w) \Big)\\
            &=S^{-1}\big( T(u,w) \big)\\
            &=[u,w].
        \end{align}
    \end{subequations}
    Pour l'inverse, la vérification est que
    \begin{equation}
        [u,v]+[v,u]=[u,u]=0.
    \end{equation}
\end{proof}

\begin{definition}
    La \defe{mesure}{mesure!angle entre vecteurs} de l'angle orienté \( [u,v]\) est \( [\theta]_{2\pi}\) si \( T[u,v]=R_{\theta}\).
\end{definition}
Notons dans cette définition qu'écrire \( T[u,v]=R_{\theta}\) dans \( \SO(2)\) ne définit pas \( \theta\), mais seulement sa classe modulo \( 2\pi\). C'est pour cela que la mesure de l'angle orienté n'est également définie que modulo \( 2\pi\).

Pour la suite nous allons nous intéresser à des vecteurs qui ont, dans l'idée, un point de départ et un point d'arrivée. Si \( A,B\in \eR^2\) nous notons
\begin{equation}
    \vect{ AB }=\frac{ B-A }{ \| B-A \| }.
\end{equation}
C'est le vecteur unitaire dans la direction «de \( B\) vers $A$».

\begin{theorem}[Théorème de l'angle inscrit\cite{ooRGSCooNgALYH}]       \label{THOooQDNKooTlVmmj}
    Soit un cercle \( \Gamma\) de centre \( O\) et trois points distincts \( A,B,M\in \Gamma\). Alors
    \begin{equation}
        2(\vect{ MA },\vect{ MB })\in (\vect{ OA },\vect{ OB })_{2\pi}
    \end{equation}
    où l'indice \( 2\pi\) indique la classe modulo \( 2\pi\).
\end{theorem}

\begin{proof}
    Le triangle \( MOA\) est isocèle en \( O\), donc les angles à la base sont égaux. Et de plus la somme des angles est dans \( [\pi]_{2\pi}\). Bon, entre nous, nous savons que la somme des angles est exactement \( \pi\), mais comme nous n'avons pas défini les angles autrement que modulo \( \pi\), nous ne pouvons pas dire mieux. Donc
    \begin{equation}
        2(\vect{ AB },\vect{ AO })+(\vect{ OB },\vect{ OA })\in [\pi]_{2\pi}.
    \end{equation}
    Il faut être sûr de l'orientation de tout cela. Le nombre \( (\vect{ AB },\vect{ AO })\) est l'angle qui sert à amener \( \vect{ AB }\) sur \( \vect{ AO }\). Vu que nous l'avons choisit dans le sens trigonométrique, il faut bien prendre les autres dans le sens trigonométrique et utiliser \( (\vect{ OA }, \vect{ OB })\) et non \( (\vect{ OB },\vect{ OA })\).

\begin{center}
   \input{auto/pictures_tex/Fig_YQIDooBqpAdbIM.pstricks}
\end{center}

De la même manière sur le triangle \( MOB\) nous écrivons
\begin{equation}
    2(\vect{ MB },\vect{ MO })+(\vect{ OM },\vect{ OB })\in[\pi]_{2\pi}.
\end{equation}
Nous faisons la différence entre les deux équations en nous souvenant que \( -(\vect{ MB },\vect{ MO })=(\vect{ MO },\vect{ MB })\) et les relations de Chasles du lemme \ref{LEMooWISVooYsStJp}\ref{ITEMooBKTFooWbEvIU} nous avons :
\begin{equation}
    2(\vect{ MA },\vect{ MB })+(\vect{ OB },\vect{ OA })\in[0]_{2\pi}.
\end{equation}
\end{proof}

\begin{normaltext}
    Comment exprimer le fait qu'un angle orienté soit égal à \( \theta\) modulo \( \pi\) alors que les angles orientés sont des classes modulo \( 2\pi\) ? Nous ne pouvons certainement pas écrire
    \begin{equation}
        (u,v)=[\theta]_{\pi}
    \end{equation}
    parce que \( (u,v)\) est un élément de \( S^1\times S^1\) alors que \( [\theta]_{\pi}\) est un ensemble de nombres. Nous pouvons écrire
    \begin{equation}
        [u,v]\subset [\theta]_{\pi}.
    \end{equation}
    C'est cohérent parce que nous avons des deux côtés des ensembles de nombres. Les opérations permises sont l'égalité ou l'inclusion. L'égalité entre les deux ensembles n'est pas possible parce que la différence minimale ente deux éléments dans \( [u,v]\) est \( 2\pi\) alors que celle dans \( [\theta]_{\pi}\) est \( \pi\).

    Si \( u\) et \( v\) forment un angle droit, nous avons
    \begin{equation}
        [u,v]=\{ \frac{ \pi }{2}+2k\pi \}_{k\in \eZ}.
    \end{equation}
    Et cela est bien un sous-ensemble de \( [\pi/2]_{\pi}\).

    Pour exprimer que la différence entre deux angles orientés diffèrent de \( \pi\) nous devrions écrire :
    \begin{equation}
        [u,v]\subset[a,b]_{\pi}
    \end{equation}
    où le membre de droite signifie la classe modulo \( \pi\) d'un représentant de \( [a,b]\). 

    Nous allons cependant nous permettre d'écrire
    \begin{equation}
        [u,v]=[a,b]_{\pi}
    \end{equation}
    voire carrément
    \begin{equation}
        (u,v)=(a,b)_{\pi}.
    \end{equation}
    Cette dernière égalité devant être comprise comme voulant dire que l'angle pour passer de \( u\) à \( v\) est soit le même que celui pour alle de \( a\) à \( b\) soit ce dernier plus \( \pi\).
\end{normaltext}

\begin{theorem}[\cite{ooRGSCooNgALYH}]      \label{THOooUDUGooTJKDpO}
    Soient \( 4\) points distincts du plan \( A,B,C,D\). Ils sont alignés ou cocycliques\footnote{C'est à dire sur un même cercle.} si et seulement si
    \begin{equation}
        (\vect{ CA },\vect{ CB })=(\vect{ DA },\vect{ DB })_{\pi}.
    \end{equation}
\end{theorem}

Nous allons seulement démontrer l'implication directe.
\begin{proof}
    Si les quatre points sont alignés nous avons \( [\vect{ CA },\vect{ CB }]=[0]_{2\pi}\) et \( [\vect{ DA },\vect{ DB }]=[0]_{2\pi}\). En particulier nous avons
    \begin{equation}
        [\vect{ CA },\vect{ CB }]=[\vect{ DA },\vect{ DB }]
    \end{equation}
    et a fortiori l'égalité modulo \( \pi\) au lieu de \( 2\pi\).

    Nous nous relâchons en termes de notations. Si les quatre points sont cocycliques, nous pouvons utiliser le théorème de l'angle inscrit \ref{THOooQDNKooTlVmmj} dans les triangles \( ABC\) et \( ADB\) :
    \begin{subequations}
        \begin{align}
            2(\vect{ CA },\vect{ CB })=(\vect{ OA },\vect{ OB })_{2\pi}\\
            2(\vect{ DA },\vect{ DB })=(\vect{ OA },\vect{ OB })_{2\pi},
        \end{align}
    \end{subequations}
    ce qui donne \(  2(\vect{ CA },\vect{ CB })=2(\vect{ DA },\vect{ DB })_{2\pi}  \) et donc
    \begin{equation}
        (\vect{ CA },\vect{ CB })=(\vect{ DA },\vect{ DB })_{\pi}.
    \end{equation}

    Comme annoncé, nous ne faisons pas la preuve dans l'autre sens; elle peut être trouvée dans~\cite{ooRGSCooNgALYH}.
\end{proof}

%--------------------------------------------------------------------------------------------------------------------------- 
\subsection{Angles et nombres complexes}
%---------------------------------------------------------------------------------------------------------------------------
\label{SUBSECooKNUVooUBKaWm}

Les nombres complexes peuvent être repérés par une norme et un angle, ce qui en fait un terrain propice à l'utilisation des angles orientés. Nous en ferons d'ailleurs usage dans \( \hat\eC=\eC\cup\{ \infty \}\) pour parler d'alignement, de cocyclicité et de birapport dans la proposition \ref{PROPooSGCJooLnOLCx}.

Soient deux éléments \( z_1,z_2\in \eC\). Nous les écrivons sous la forme \( z_1=r_1 e^{i\theta_1}\) et \( z_2=r_2 e^{i\theta_2}\); remarquons que cela ne définit \( \theta_i\) qu'à \( 2\pi\) près. Nous avons
\begin{equation}
    [z_1,z_2]=[\theta_2-\theta_1]_{2\pi}.
\end{equation}

Soient maintenant \( a,b,c,d\in \eC\). Nous écrivons \( \vect{ ab }\) le vecteur unitaire dans le sens «de \( a\) vers \( b\)», c'est à dire un multiple positif bien choisi du nombre \( b-a\). Nous notons \( \theta_{ab}\) l'argument du nombre complexe \( b-a\), et nous avons encore
\begin{equation}
    [\vect{ ab },\vect{ cd }]=[\theta_{ab}-\theta_{cd}].
\end{equation}

Avec toutes ces notations, ce qui est bien est que les produits et quotients de nombres complexes se comportent très bien par rapport aux angles : l'argument de \( a/b\) est \( \theta_a-\theta_b\) et en particulier l'argument de 
\begin{equation}
    \frac{ a-b }{ c-d }
\end{equation}
est dans la classe de l'angle orienté
\begin{equation}
    [\vect{ ba },\vect{ dc }].
\end{equation}
