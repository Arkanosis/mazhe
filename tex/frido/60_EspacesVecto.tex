% This is part of Mes notes de mathématique
% Copyright (c) 2011-2017
%   Laurent Claessens, Carlotta Donadello
% See the file fdl-1.3.txt for copying conditions.


%+++++++++++++++++++++++++++++++++++++++++++++++++++++++++++++++++++++++++++++++++++++++++++++++++++++++++++++++++++++++++++
\section{Diagonalisation}
%+++++++++++++++++++++++++++++++++++++++++++++++++++++++++++++++++++++++++++++++++++++++++++++++++++++++++++++++++++++++++++

Ici encore \( \eK\) est un corps commutatif.

%---------------------------------------------------------------------------------------------------------------------------
\subsection{Matrices semblables}
%---------------------------------------------------------------------------------------------------------------------------

\begin{definition}[matrices semblables] \label{DefCQNFooSDhDpB}
    Sur l'ensemble \( \eM_n(\eK)\) des matrices \( n\times n\) à coefficients dans \(\eK\) nous introduisons la relation d'équivalence \( A\sim B\) si et seulement s'il existe une matrice \( P\in\GL(n,\eK)\) telle que \( B=P^{-1}AP\). Deux matrices équivalentes en ce sens sont dites \defe{semblables}{semblables!matrices}.
\end{definition}

Le polynôme caractéristique est un invariant sous les similitudes. En effet si \( P\) est une matrice inversible,
\begin{subequations}
    \begin{align}
        \chi_{PAP^{-1}}&=\det(PAP^{-1}-\lambda X)\\
        &=\det\big( P^{-1}(PAP^{-1}-\lambda X)P^{-1} \big)\\
        &=\det(A-\lambda X).
    \end{align}
\end{subequations}

La permutation de lignes ou de colonnes ne sont pas de similitudes, comme le montrent les exemples suivants :
\begin{equation}
    \begin{aligned}[]
        A&=\begin{pmatrix}
            1    &   2    \\ 
            3    &   4    
        \end{pmatrix}&
        B&=\begin{pmatrix}
            2    &   1    \\ 
            4    &   3    
        \end{pmatrix}.
    \end{aligned}
\end{equation}
Nous avons \( \chi_A=x^2-5x-2\) tandis que \( \chi_B=x^2-5x+2\) alors que le polynôme caractéristique est un invariant de similitude.

%---------------------------------------------------------------------------------------------------------------------------
\subsection{Endomorphismes diagonalisables}
%---------------------------------------------------------------------------------------------------------------------------

\begin{definition}  \label{DefCNJqsmo}
    Une matrice est \defe{diagonalisable}{diagonalisable} si elle est semblable à une matrice diagonale.
\end{definition}

\begin{lemma}
    Une matrice triangulaire supérieure avec des \( 1\) sur la diagonale n'est diagonalisable que si elle est diagonale (c'est à dire si elle est la matrice unité).
\end{lemma}

\begin{proof}
    Si \( A\) est une matrice triangulaire supérieure de taille \( n\) telle que \( A_{ii}=1\), alors \( \det(A-\lambda\mtu)=(1-\lambda)^n\), ce qui signifie que \( \Spec(A)=\{ 1 \}\). Pour la diagonaliser, il faudrait une matrice \( P\in\GL(n,\eK)\) telle que \( \mtu=P^{-1}AP\), ce qui est uniquement possible si \( A=\mtu\).
\end{proof}

\begin{lemma}       \label{LemgnaEOk}
    Soit \( F\) un sous-espace stable par \( u\). Soit une décomposition du polynôme minimal
    \begin{equation}
        \mu_u=P_1^{n_1}\ldots P_r^{n_r}
    \end{equation}
    où les \( P_i\) sont des polynômes irréductibles unitaires distincts. Si nous posons \( E_i=\ker P_i^{n_i}\), alors
    \begin{equation}
        F=(F\cap E_1)\oplus\ldots \oplus(F\cap E_r).
    \end{equation}
\end{lemma}

\begin{theorem}     \label{ThoDigLEQEXR}
    Soit \( E\), un espace vectoriel de dimension \( n\) sur le corps commutatif \( \eK\) et \( u\in\End(E)\). Les propriétés suivantes sont équivalentes.
    \begin{enumerate}
        \item\label{ItemThoDigLEQEXRiv}
            L'endomorphisme \( u\) est diagonalisable.
        \item       \label{ItemThoDigLEQEXRi}
            Il existe un polynôme \( P\in\eK[X]\) non constant, scindé sur \(\eK\) dont toutes les racines sont simples tel que \( P(u)=0\).
        \item\label{ItemThoDigLEQEXRii}
            Le polynôme minimal \( \mu_u\) est scindé sur \(\eK\) et toutes ses racines sont simples\footnote{Le polynôme \emph{caractéristique}, lui, n'a pas spécialement ses racines simples; il peut encore être de la forme
            \begin{equation}
                \chi_u(X)=\prod_{i=1}^r(X-\lambda_i)^{\alpha_i},
        \end{equation}
        mais alors \( \dim(E_{\lambda_i})=\alpha_i\). }.
        \item\label{ItemThoDigLEQEXRiii}
            Tout sous-espace de \( E\) possède un supplémentaire stable par \( u\).
        \item       \label{ITEMooZNJFooEiqDYp}
            Dans une base adaptée, la matrice de \( u\) est diagonale et les éléments diagonaux sont ses valeurs propres.
    \end{enumerate}
\end{theorem}
\index{diagonalisable!et polynôme minimum scindé}

\begin{proof}
    Plein d'implications à prouver.
    \begin{subproof}
    \item[\ref{ItemThoDigLEQEXRi} implique \ref{ItemThoDigLEQEXRii}] Étant donné que \( P(u)=0\), il est dans l'idéal des polynôme annulateurs de \( u\), et le polynôme minimal \( \mu_u\) le divise parce que l'idéal des polynôme annulateurs est généré par \( \mu_u\) par le théorème \ref{ThoCCHkoU}.

    \item[\ref{ItemThoDigLEQEXRii} implique \ref{ItemThoDigLEQEXRiv}] Étant donné que le polynôme minimal est scindé à racines simples, il s'écrit sous forme de produits de monômes tous distincts, c'est à dire
    \begin{equation}
        \mu_u(X)=(X-\lambda_1)\ldots(X-\lambda_r)
    \end{equation}
    où les \( \lambda_i\) sont des éléments distincts de \( \eK\). Étant donné que \( \mu_u(u)=0\), le théorème de décomposition des noyaux (théorème \ref{ThoDecompNoyayzzMWod}) nous enseigne que
    \begin{equation}
        E=\ker(u-\lambda_1)\oplus\ldots\oplus\ker(u-\lambda_r).
    \end{equation}
    Mais \( \ker(u-\lambda_i)\) est l'espace propre \( E_{\lambda_i}(u)\). Donc \( u\) est diagonalisable.

\item[\ref{ItemThoDigLEQEXRiv} implique \ref{ItemThoDigLEQEXRiii}] Soit \( \{ e_1,\ldots, e_n \}\) une base qui diagonalise \( u\), soit \( F\) un sous-espace de \( E\) un \( \{ f_1,\ldots, f_r \}\) une base de \( F\). Par le théorème \ref{ThoBaseIncompjblieG} (qui généralise le théorème de la base incomplète), nous pouvons compléter la base de \( F\) par des éléments de la base \( \{ e_i \}\). Le complément ainsi construit est invariant par \( u\).

\item[\ref{ItemThoDigLEQEXRiii} implique \ref{ItemThoDigLEQEXRiv}] En dimension un, tout endomorphisme est diagonalisable, nous supposons donc que \( \dim E=n\geq 2\). Nous procédons par récurrence sur le nombre de vecteurs propres connus de \( u\). Supposons avoir déjà trouvé \( p\) vecteurs propres \( e_1,\ldots, e_p\) de \( u\). Considérons \( H\), un hyperplan qui contient les vecteurs \( e_1,\ldots, e_p\). Soit \( F\) un supplémentaire de \( H\) stable par \( u\); par construction \( \dim F=1\) et si \( e_{p+1}\in F\), il doit être vecteur propre de \( u\).

\item[\ref{ItemThoDigLEQEXRiv} implique \ref{ItemThoDigLEQEXRi}] Nous supposons maintenant que \( u\) est diagonalisable. Soient \( \lambda_1,\ldots, \lambda_r\) les valeurs propres deux à deux distinctes, et considérons le polynôme
    \begin{equation}
        P(x)=(X-\lambda_1)\ldots (X-\lambda_r).
    \end{equation}
    Alors \( P(u)=0\). En effet si \( e_i\) est un vecteur propre pour la valeur propre \( \lambda_i\), 
    \begin{equation}
        P(u)e_i=\prod_{j\neq i}(u-\lambda_j)\circ(u-\lambda_i)e_i=0
    \end{equation}
    par le lemme \ref{LemQWvhYb}. Par conséquent \( P(u)\) s'annule sur une base.

\item[\ref{ITEMooZNJFooEiqDYp} implique \ref{ItemThoDigLEQEXRi}]
    Si la matrice \( A\) est diagonale alors le polynôme \( P=\prod_{i=1}^n(A-A_{ii}\mtu)\) est annulateur de \( A\).
        \item[\ref{ItemThoDigLEQEXRii} implique \ref{ITEMooZNJFooEiqDYp}]
            le polynôme minimal de \( u\) s'écrit 
            \begin{equation}
                \mu=(X-\lambda_1)\ldots(X-\lambda_r),
            \end{equation}
            et les espaces $E_i$ du lemme \ref{LemgnaEOk} sont les espaces propres \( E_i=\ker(u-\lambda_i)\). Nous avons donc une somme directe
            \begin{equation}
                E=E_1\oplus\ldots\oplus E_r.
            \end{equation}
            Dans chacun des espaces propres, $u$ a une matrice diagonale avec la valeur propre correspondante sur la diagonale. Une base de \( E\) constituée d'une base de chacun des espaces propres est donc une base comme nous en cherchons.
    \end{subproof}
\end{proof}

\begin{corollary}       \label{CorQeVqsS}
    Si \( u\) est diagonalisable et si \( F\) est une sous-espace stable par \( u\), alors
    \begin{equation}
        F=\bigoplus_{\lambda}E_{\lambda}(u)\cap F
    \end{equation}
    où \( E_{\lambda}(u)\) est l'espace propre de \( u\) pour la valeur propre \( \lambda\). En particulier la restriction de \( u\) à \( F\), \( u|_F\) est diagonalisable.
\end{corollary}

\begin{proof}
    Par le théorème \ref{ThoDigLEQEXR}, le polynôme \( \mu_u\) est scindé et ne possède que des racines simples. Notons le
    \begin{equation}
        \mu_u(X)=(X-\lambda_1)\ldots (X-\lambda_r).
    \end{equation}
    Les espaces \( E_i\) du lemme \ref{LemgnaEOk} sont maintenant les espaces propres.

    En ce qui concerne la diagonalisabilité de \( u|_F\), notons que nous avons une base de \( F\) composée de vecteurs dans les espaces \( E_{\lambda}(u)\). Cette base de \( F\) est une base de vecteurs propres de \( u\).
\end{proof}

\begin{lemma}
    Soit \( E\) un \( \eK\)-espace vectoriel et \( u\in\End(E)\). Si \( \Card\big( \Spec(u) \big)=\dim(E)\) alors \( u\) est diagonalisable.
\end{lemma}

\begin{proof}
    Soient \( \lambda_1,\ldots, \lambda_n\) les valeurs propres distinctes de \( u\). Nous savons que les espaces propres correspondants sont en somme directe (lemme \ref{LemjcztYH}). Par conséquent \( \Span\{ E_{\lambda_i}(u) \}\) est de dimension \( n\) est \( u\) est diagonalisable.
\end{proof}

Voici un résultat de diagonalisation simultanée. Nous donnerons un résultat de trigonalisation simultanée dans le lemme \ref{LemSLGPooIghEPI}.
\begin{proposition}[Diagonalisation simultanée]     \label{PropGqhAMei}
    Soit \( (u_i)_{i\in I}\) une famille d'endomorphismes qui commutent deux à deux.
    \begin{enumerate}
        \item       \label{ItemGqhAMei}
            Si \( i,j\in I\) alors tout sous-espace propre de \( u_i\) est stable par \( u_j\). Autrement dit \( u_j\big(E_{\lambda}(u)\big)\subset E_{\lambda}(u)\).
        \item
            Si les \( u_i\) sont diagonalisables, alors ils le sont simultanément.
    \end{enumerate}
\end{proposition}
\index{diagonalisation!simultanée}

\begin{proof}
    Supposons que \( u_i\) et \( u_j\) commutent et soit \( x\) un vecteur propre de \( u_i\) : \( u_ix=\lambda x\). Nous montrons que \( u_jx\in E_{\lambda}(u)\). Nous avons
    \begin{equation}
        u_i\big( u_j(x) \big)=u_j\big( u_i(x) \big)=\lambda u_j(x).
    \end{equation}
    Par conséquent \( u_j(x)\) est vecteur propre de \( u_i\) de valeur propre \( \lambda\).

    Montrons maintenant l'affirmation à propos des endomorphismes simultanément diagonalisables. Si \( \dim E=1\), le résultat est évident. Nous supposons également qu'aucun des \( u_i\) n'est multiple de l'identité. Nous effectuons une récurrence sur la dimension.

    Soit \( u_0\) un des \( u_i\) et considérons ses valeurs propres deux à deux distinctes \( \lambda_1,\ldots, \lambda_r\). Pour chaque \( k\) nous avons
    \begin{equation}
        E_{\lambda_k}(u_0)\neq E,
    \end{equation}
    sinon \( u_0\) serait un multiple de l'identité. Par contre le fait que \( u_0\) soit diagonalisable permet de décomposer \( E\) en espaces propres de \( u_0\) :
    \begin{equation}
        E=\bigoplus_{k}E_{\lambda_k}(u_0).
    \end{equation}
    Ce que nous allons faire est de simultanément diagonaliser les \( (u_i)_{i\in I}\) sur chacun des \( E_{\lambda_k}\) séparément. Par le point \ref{ItemGqhAMei}, nous avons \( u_i\colon E_{\lambda_k}(u_0)\to E_{\lambda_k}(u_0)\), et nous pouvons considérer la famille d'opérateurs
    \begin{equation}
        \left( u_i|_{E_{\lambda_k}(u_0)} \right)_{i\in I}.
    \end{equation}
    Ce sont tous des opérateurs qui commutent et qui agissent sur un espace de dimension plus petite. Par hypothèse de récurrence nous avons une base de \( E_{\lambda_k}(u_0)\) qui diagonalise tous les \( u_i\).
\end{proof}

\begin{example}     \label{ExewINgYo}
    Soit un espace vectoriel sur un corps \( \eK\). Un opérateur \defe{involutif}{involution} est un opérateur différent de l'identité dont le carré est l'identité. Typiquement une symétrie orthogonale dans \( \eR^3\). Le polynôme caractéristique d'une involution est \( X^2-1=(X+1)(X-1)\).
    
    Tant que \( 1\neq -1\), \( X^1-1\) est donc scindé à racines simples et les involutions sont diagonalisables (\ref{ThoDigLEQEXR}). Cependant si le corps est de caractéristique \( 2\), alors \( X^2-1=(X+1)^2\) et l'involution n'est plus diagonalisable.

    Par exemple si le corps est de caractéristique \( 2\), nous avons
    \begin{subequations}
        \begin{align}
            A&=\begin{pmatrix}
                1    &   1    \\ 
                0    &   1    
            \end{pmatrix}\\
            A^1&=\begin{pmatrix}
                1    &   2    \\ 
                0    &   1    
            \end{pmatrix}=\begin{pmatrix}
                1    &   0    \\ 
                0    &   1    
            \end{pmatrix}.
        \end{align}
    \end{subequations}
    Ce \( A\) est donc une involution mais n'est pas diagonalisable.
\end{example}

%---------------------------------------------------------------------------------------------------------------------------
\subsection{Diagonalisation : cas complexe}
%---------------------------------------------------------------------------------------------------------------------------

\begin{lemma}[Théorème spectral hermitien]      \label{LEMooVCEOooIXnTpp}
    Pour un opérateur hermitien\footnote{Définition \ref{DEFooKEBHooWwCKRK}.},
    \begin{enumerate}
        \item
            le spectre est réel,
        \item
            deux vecteurs propres à des valeurs propres distinctes sont orthogonales\footnote{Pour la forme \eqref{EqFormSesqQrjyPH}.}.
    \end{enumerate}
\end{lemma}
\index{spectre!matrice hermitienne}

\begin{proof}
    Soit \( v\) un vecteur de valeur propre \( \lambda\). Nous avons d'une part 
    \begin{equation}
        \langle Av, v\rangle =\lambda\langle v, v\rangle =\lambda\| v \|^2,
    \end{equation}
    et d'autre part, en utilisant le fait que \( A\) est hermitien,
    \begin{equation}
        \langle Av, v\rangle =\langle v, A^*v\rangle =\langle v, Av\rangle =\bar\lambda\| v \|^2,
    \end{equation}
    par conséquent \( \lambda=\bar\lambda\) parce que \( v\neq 0\).

    Soient \( \lambda_i\) et \( v_i\) (\( i=1,2\)) deux valeurs propres de \( A\) avec leurs vecteurs propres correspondants. Alors d'une part
    \begin{equation}
        \langle Av_1, v_2\rangle =\lambda_1\langle v_1, v_2\rangle ,
    \end{equation}
    et d'autre part
    \begin{equation}
        \langle Av_1, v_2\rangle =\langle v_1, Av_2\rangle =\lambda_2\langle v_1, v_2\rangle .
    \end{equation}
    Nous avons utilisé le fait que \( \lambda_2\) était réel. Par conséquent, soit \( \lambda_1=\lambda_2\), soit \( \langle v_1, v_2\rangle =0\).
\end{proof}

\begin{remark}      \label{REMooMLBCooTuKFmz}
    Un opérateur de la forme \( A^*A\) est évidemment hermitien. De plus ses valeurs propres sont toutes positives parce que si \( A^*Ax=\lambda v\) alors
    \begin{equation}
        0\leq \langle Av, Av\rangle =\langle A^*Av, v\rangle =\lambda\langle v, v\rangle .
    \end{equation}
    Donc \( \lambda\geq 0\).
\end{remark}

Le lemme suivant est utile en soi et dit que toute matrice complexe est trigonalisable. Une démonstration alternative passant par le polynôme caractéristique sera présentée dans la remarque \ref{RemXFZTooXkGzQg} utilisant la proposition \ref{PropKNVFooQflQsJ}.
\begin{lemma}[Lemme de Schur complexe\cite{NormHKNPKRqV}]  \label{LemSchurComplHAftTq}
    Si \( A\in\eM(n,\eC)\), il existe une matrice unitaire \( U\) telle que \( UAU^{-1}\) soit triangulaire supérieure.
\end{lemma}
\index{lemme!Schur complexe}
%TODO : Le lemme de Schur est souvent énoncé en disant que si p est une représentation irréductible, alors les seuls endomorphismes de V commutant avec tous les p(g) sont les multiples de l'idenditié. Quel est le lien avec ceci ?

\begin{proof}
    Étant donné que \( \eC\) est algébriquement clos, nous pouvons toujours considérer un vecteur propre \( v_1\) de \( A\), de valeur propre \( \lambda_1\). Nous pouvons utiliser un procédé de Gram-Schmidt pour construire une base orthonormée \( \{ v,u_2,\ldots, u_n \}\) de \( \eR^n\), et la matrice (unitaire)
    \begin{equation}
        Q=\begin{pmatrix}
             \uparrow   &   \uparrow    &       &   \uparrow    \\
             v   &   u_2    &   \cdots    &   u_n    \\ 
             \downarrow   &   \downarrow    &       &   \downarrow
         \end{pmatrix}.
    \end{equation}
    Nous avons \( Q^{-1}AQe_1=Q^{-1} Av=\lambda Q^{-1} v=\lambda e_1\), par conséquent la matrice \( Q^{-1} AQ\) est de la forme
    \begin{equation}
        Q^{-1}AQ=\begin{pmatrix}
            \lambda_1    &   *    \\ 
            0    &   A_1    
        \end{pmatrix}
    \end{equation}
    où \( *\) représente une ligne quelconque et \( A_1\) est une matrice de \( \eM(n-1,\eC)\). Nous pouvons donc répéter le processus sur \( A_1\) et obtenir une matrice triangulaire supérieure (nous utilisons le fait qu'un produit de matrices orthogonales est une matrice orthogonale).  
\end{proof}
En particulier les matrices hermitiennes, anti-hermitiennes et unitaires sont trigonalisables par une matrice unitaire, qui peut être choisie de déterminant \( 1\).

\begin{corollary}   \label{CorUNZooAZULXT}
    Le polynôme caractéristique\footnote{Définition \ref{DefOWQooXbybYD}.} sur \( \eC\) d'une matrice s'écrit sous la forme
    \begin{equation}
        \chi_A(X)=\prod_{i=1}^r(X-\lambda_i)^{m_i}
    \end{equation}
    où les \( \lambda_i\) sont les valeurs propres distinctes de \( A\) et \( m_i\) sont les multiplicités correspondantes.
\end{corollary}
\index{polynôme!caractéristique}

\begin{proof}
    Le lemme \ref{LemSchurComplHAftTq} nous donne l'existence d'une base de trigonalisation; dans cette base les valeurs propres de \( A\) sont sur la diagonale et nous avons 
    \begin{equation}
        \chi_A(X)=\det(A-X\mtu)=\det\begin{pmatrix}
            X-\lambda_1    &   *    &   *    \\
            0    &   \ddots    &   *    \\
            0    &   0    &   X-\lambda_r
        \end{pmatrix},
    \end{equation}
    qui vaut bien le produit annoncé.
\end{proof}

\begin{theorem}[Théorème spectral pour les matrices normales\cite{LecLinAlgAllen,OMzxpxE,HOQzXCw}]\index{théorème!spectral!matrices normales}  \index{diagonalisation!cas complexe}  \label{ThogammwA}
    Soit \( A\in\eM(n,\eC)\) une matrice de valeurs propres \( \lambda_1,\ldots, \lambda_n\) (non spécialement distinctes). Alors les conditions suivantes sont équivalentes :
    \begin{enumerate}
        \item   \label{ItemJZhFPSi}
            \( A\) est normale,
        \item   \label{ItemJZhFPSii}
            \( A\) se diagonalise par une matrice unitaire,
        \item
            \( \sum_{i,j=1}^n| A_{ij} |^2=\sum_{j=1}^n| \lambda_j |^2\),
        \item
            il existe une base orthonormale de vecteurs propres de \( A\).
    \end{enumerate}
\end{theorem}

\begin{proof}
    Nous allons nous contenter de prouver \ref{ItemJZhFPSi}\( \Leftrightarrow\)\ref{ItemJZhFPSii}.
    %TODO : le reste.

    Soit \( Q\) la matrice unitaire donnée par la décomposition de Schur (lemme \ref{LemSchurComplHAftTq}) : \( A=QTQ^{-1}\). Étant donné que \( A\) est normale nous avons
    \begin{equation}
        QTT^*Q^{-1}=QT^*TQ^{-1},
    \end{equation}
    ce qui montre que \( T\) est également normale. Or une matrice triangulaire supérieure normale est diagonale. En effet nous avons \( T_{ij}=0\) lorsque \( i>j\) et
    \begin{equation}
        (TT^*)_{ii}=(T^*T)_{ii}=\sum_{k=1}^n| T_{ki} |^2=\sum_{k=1}^n| T_{ik} |^2.
    \end{equation}
    Écrivons cela pour \( i=1\) en tenant compte de \( | T_{k1} |^2=0\) pour \( k=2,\ldots, n\),
    \begin{equation}
        | T_{11} |^2=| T_{11} |^2+| T_{12} |^2+\cdots+| T_{1n} |^2,
    \end{equation}
    ce qui implique que \( T_{11}\) est le seul non nul parmi les \( T_{1k}\). En continuant de la sorte avec \( i=2,\ldots, n\) nous trouvons que \( T\) est diagonale.

    Dans l'autre sens, si \( A\) se diagonalise par une matrice unitaire, \( UAU^*=D\), nous avons
    \begin{equation}
        DD^*=UAA^*U^*
    \end{equation}
    et 
    \begin{equation}
        D^*D=UA^*AU^*,
    \end{equation}
    qui ce prouve que \( A\) est normale.
\end{proof}

%---------------------------------------------------------------------------------------------------------------------------
\subsection{Diagonalisation : cas réel}
%---------------------------------------------------------------------------------------------------------------------------

\begin{lemma}[Lemme de Schur réel]  \label{LemSchureRelnrqfiy}
    Soit \( A\in\eM(n,\eR)\). Il existe une matrice orthogonale \( Q\) telle que \( Q^{-1}AQ\) soit de la forme
    \begin{equation}        \label{EqMtrTSqRTA}
        QAQ^{-1}=\begin{pmatrix}
            \lambda_1    &   *    &   *    &   *    &   *\\  
            0    &   \ddots    &   \ddots    &   \ddots    &   \vdots\\  
            0    &   0    &   \lambda_r    &   *    &   *\\  
            0    &   0    &   0    &   \begin{pmatrix}
                a_1    &   b_1    \\ 
                c_1    &   d_1    
            \end{pmatrix}&   *\\  
            0    &   0    &  0     &   0    &   \begin{pmatrix}
                a_s    &   b_s    \\ 
                c_s    &   d_s    
            \end{pmatrix}
        \end{pmatrix}.
    \end{equation}
    Le déterminant de \( A\) est le produit des déterminants des blocs diagonaux et les valeurs propres de \( A\) sont les \( \lambda_1,\ldots, \lambda_r\) et celles de ces blocs.
\end{lemma}
\index{lemme!Schur réel}

\begin{proof}
    Si la matrice \( A\) a des valeurs propres réelles, nous procédons comme dans le cas complexe. Cela nous fournit le partie véritablement triangulaire avec les valeurs propres \( \lambda_1,\ldots, \lambda_r\) sur la diagonale. Supposons donc que \( A\) n'a pas de valeurs propres réelles. Soit donc \( \alpha+i\beta \) une valeur propre (\( \beta\neq 0\)) et \( u+iv\) un vecteur propre correspondant où \( u\) et \( v\) sont des vecteurs réels. Nous avons
    \begin{equation}
        Au+iAv=A(u+iv)=(\alpha+i\beta)(u+iv)=\alpha u-\beta v+i(\alpha v+\beta v),
    \end{equation}
    et en égalisant les parties réelles et imaginaires,
    \begin{subequations}
        \begin{align}
            Au&=\alpha u-\beta v\\
            Av&=\alpha v+\beta u.
        \end{align}
    \end{subequations}
    Sur ces relations nous voyons que ni \( u\) ni \( v\) ne sont nuls. De plus \( u\) et \( v\) sont linéairement indépendants (sur \( \eR\)), en effet si \( v=\lambda u\) nous aurions \( Au=\alpha u-\beta\lambda u=(\alpha-\beta\lambda)u\), ce qui serait une valeur propre réelle alors que nous avions supposé avoir déjà épuisé toutes les valeurs propres réelles.

    Étant donné que \( u\) et \( v\) sont deux vecteurs réels non nuls et linéairement indépendants, nous pouvons trouver une base orthonormée \( \{ q_1,q_2 \}\) de \( \Span\{ u,v \}\). Nous pouvons étendre ces deux vecteurs en une base orthonormée \( \{ q_1,q_2,q_3,\ldots, q_n \}\) de \( \eR^n\). Nous considérons à présent la matrice orthogonale dont les colonnes sont formées de ces vecteurs : \( Q=[q_1\,q_2\,\ldots q_n]\).

    L'espace \( \Span\{ e_1,e_2 \}\) est stable par \( Q^{-1} AQ\), en effet nous avons
    \begin{equation}
        Q^{-1} AQe_1=Q^{-1} Aq_1=Q^{-1}(aq_1+bq_2)=ae_1+be_2.
    \end{equation}
    La matrice \( Q^{-1}AQ\) est donc de la forme
    \begin{equation}
        Q^{-1} AQ=\begin{pmatrix}
            \begin{pmatrix}
                \cdot    &   \cdot    \\ 
                \cdot    &   \cdot    
            \end{pmatrix}&   C_1    \\ 
            0    &   A_1    
        \end{pmatrix}
    \end{equation}
    où \( C_1\) est une matrice réelle \( 2\times (n-1)\) quelconque et \( A_1\) est une matrice réelle \( (n-2)\times (n-2)\). Nous pouvons appliquer une récurrence sur la dimension pour poursuivre.

    Notons que si \( A\) n'a pas de valeurs propres réelles, elle est automatiquement d'ordre pair parce que les valeurs propres complexes viennent par couple complexes conjuguées.

    En ce qui concerne les valeurs propres, il est facile de voir en regardant \eqref{EqMtrTSqRTA} que les valeurs propres sont celles des blocs diagonaux. Étant donné que \( QAQ^{-1}\) et \( A\) ont même polynôme caractéristique, ce sont les valeurs propres de \( A\).
\end{proof}

\begin{theorem}[Théorème spectral, matrice symétrique\cite{KXjFWKA}] \label{ThoeTMXla}
    Une matrice symétrique réelle,
    \begin{enumerate}
        \item
            a un spectre contenu dans \( \eR\)
        \item
            est diagonalisable par une matrice orthogonale.
    \end{enumerate}
    Si \( M\) est une matrice symétrique réelle alors \( \eR^n\) possède une base orthonormée de vecteurs propres de \( M\).
\end{theorem}
\index{diagonalisation!cas réel}
\index{rang!diagonalisation}
\index{endomorphisme!diagonalisation}
\index{spectre!matrice symétrique réelle}
\index{théorème!spectral!matrice symétrique}

\begin{proof}
    Soit \( A\) une matrice réelle symétrique. Si \( \lambda\) est une valeur propre complexe pour le vecteur propre complexe \( v\), alors d'une part \( \langle Av, v\rangle =\lambda\langle v, v\rangle \) et d'autre part \( \langle Av, v\rangle =\langle v, Av\rangle =\bar\lambda\langle v, v\rangle \). Par conséquent \( \lambda=\bar\lambda\).
    
    Le lemme de Schur réel \ref{LemSchureRelnrqfiy} donne une matrice orthogonale qui trigonalise \( A\). Les valeurs propres étant toutes réelles, la matrice \( QAQ^{-1}\) est même triangulaire (il n'y a pas de blocs dans la forme \eqref{EqMtrTSqRTA}). Prouvons que \( QAQ^{-1}\) est symétrique :
    \begin{equation}
        (QAQ^{-1})^t=(Q^{-1})^tA^tQ^t=QA^tQ^{-1}=QAQ^{-1}
    \end{equation}
    où nous avons utilisé le fait que \( Q\) était orthogonale (\( Q^{-1}=Q^t\)) et que \( A\) était symétrique (\( A^t=A\)). Une matrice triangulaire supérieure symétrique est obligatoirement une matrice diagonale.

    En ce qui concerne la base de vecteurs propres, soit \( \{ e_i \}_{i=1,\ldots, n}\) la base canonique de \( \eR^n\) et \( Q\) une matrice orthogonale e telle que \( A=Q^tDQ\) avec \( D\) diagonale. Nous posons \( f_i=Q^te_i\) et en tenant compte du fait que \( Q^t=Q^{-1}\) nous avons \( Af_i=Q^tDQQ^te_i=Q^t\lambda_i e_i=\lambda_if_i\). Donc les \( f_i\) sont des vecteurs propres de \( A\). De plus ils sont orthonormés parce que
    \begin{equation}
        \langle f_i, f_j\rangle =\langle Q^te_i, Q^te_j\rangle =\langle e_i, Q^tQe_j\rangle =\langle e_i, e_j\rangle =\delta_{ij}.
    \end{equation}
\end{proof}
Le théorème spectral pour les opérateurs auto-adjoints sera traité plus bas parce qu'il a besoin de choses sur les formes bilinéaires, théorème \ref{ThoRSBahHH}.
% et les choses sur la dégénérescences utilisent le théorème spectral, cas réel. Donc l'enchaînement est très loumapotiste.

\begin{remark}  \label{RemGKDZfxu}
    Une matrice symétrique est diagonalisable par une matrice orthogonale. Nous pouvons en réalité nous arranger pour diagonaliser par une matrice de \( \SO(n)\). Plus généralement si \( A\) est une matrice diagonalisable par une matrice \( P\in\GL^+(n,\eR)\) alors elle est diagonalisable par une matrice de \( \GL^-(n,\eR)\) en changeant le signe de la première ligne de \( P\). Et inversement.

    En effet, si nous avons \( P^tDP=A\), alors en notant \( *\) les quantités qui ne dépendent pas de \( a\), \( b\) ou~\( c\),
    \begin{equation}
        \begin{aligned}[]
        \begin{pmatrix}
            a    &   *    &   *    \\
            b    &   *    &   *    \\
            c    &   *    &   *
        \end{pmatrix}
        \begin{pmatrix}
            \lambda_1    &       &       \\
                &   \lambda_2    &       \\
                &       &   \lambda_3
            \end{pmatrix}
            \begin{pmatrix}
                a    &   b    &   c    \\
                *    &   *    &   *    \\
                *    &   *    &   *
            \end{pmatrix}&=
        \begin{pmatrix}
            a    &   *    &   *    \\
            b    &   *    &   *    \\
            c    &   *    &   *
        \end{pmatrix}
        \begin{pmatrix}
            \lambda_1a    &   \lambda_1b    &   \lambda_1c    \\
            *    &   *    &   *    \\
            *    &   *    &   *
        \end{pmatrix}\\
        &=\begin{pmatrix}
            \lambda_1 a^2+*   &   \lambda_1ab+*    &   \lambda_1ac  +*  \\
            \ldots    &   \ldots    &   \ldots    \\
            \ldots    &   \ldots    &   \ldots
        \end{pmatrix}.
        \end{aligned}
    \end{equation}
    Nous voyons donc que si nous changeons les signes de \( a\), \( b\) et \( c\) en même temps, le résultat ne change pas.
\end{remark}

\begin{definition}[Matrice définie positive, opérateur définit positif]    \label{DefAWAooCMPuVM}
    Un opérateur sur un espace vectoriel sur \( \eC\) ou \( \eR\) est \defe{définit positif}{opérateur!définit positif} si toutes ses valeurs propres sont réelles et strictement positives.  Il est \defe{semi-définie positive}{semi-définie positive} si ses valeurs propres sont réelles positives ou nulles.
\end{definition}
Afin d'éviter l'une ou l'autre confusion, nous disons souvent \emph{strictement} définie positive pour positive.

Nous notons \( S^+(n,\eR)\)\nomenclature[A]{\( S^+(n,\eR)\)}{matrices symétriques semi-définies positives} l'ensemble des matrices réelles \( n\times n\) semi-définies positives. L'ensemble \( S^{++}(n,\eR)\)\nomenclature[A]{\( S^{++}(n,\eR)\)}{matrices symétriques strictement définies positives} est l'ensemble des matrices symétriques strictement définies positives.

\begin{remark}
    Nous ne définissons pas la notion de matrice définie positive pour une matrice non symétrique.
\end{remark}

\begin{lemma}   \label{LemWZFSooYvksjw}
    La matrice symétrique \( M\) est strictement définie positive si et seulement si \( \langle x, Mx\rangle >0\) pour tout \( x\) non nul dans \( \eR^n\).
\end{lemma}

\begin{proof}
    Soit \( \{ e_i \}_{i=1,\ldots, n}\) une base orthonormée de vecteurs propres de \( M\) dont l'existence est assurée par le théorème spectral \ref{ThoeTMXla}. Nous nommons \( x_i\) les coordonnées de \( x\) dans cette base. Alors,
    \begin{equation}
        \langle x,Mx \rangle =\sum_{i,j}x_i\langle e_i, x_jMe_j\rangle =\sum_{i,j}x_ix_j\langle e_i, \lambda_je_j\rangle =\sum_{ij}x_ix_j\lambda_j\delta_{ij}=\sum_i\lambda_ix_i^2
    \end{equation}
    où les \( \lambda_i\) sont les valeurs propres de \( M\). Cela est strictement positif pour tout \( x\) si et seulement si tous les \( \lambda_i\) sont positifs.
\end{proof}

\begin{corollary}
    Une matrice symétrique strictement définie positive est inversible.
\end{corollary}

\begin{proof}
    Si \( Ax=0\) alors \( \langle Ax, x\rangle =0\). Mais dans le cas d'une matrice strictement définie positive, cela implique \( x=0\) par le lemme \ref{LemWZFSooYvksjw}.
\end{proof}

%--------------------------------------------------------------------------------------------------------------------------- 
\subsection{Pseudo-réduction simultanée}
%---------------------------------------------------------------------------------------------------------------------------

\begin{corollary}[Pseudo-réduction simultanée\cite{JMYQgLO}]  \label{CorNHKnLVA}
    Soient \( A,B\in \gS(n,\eR)\) avec \( A\) définie positive\footnote{Définition \ref{DefAWAooCMPuVM}.}. Alors il existe \( Q\in \GL(n,\eR)\) tell que \( Q^tBQ\) soit diagonale et \( Q^tAQ=\mtu\).
\end{corollary}

\begin{proof}
    Nous allons noter \( x\cdot y\) le produit scalaire usuel de \( \eR^n\) et \( \{ e_i \}_{i=1,\ldots, n}\) sa base canonique.

    Vu que \( A\) est définie positive, nous avons que l'expression\footnote{On peut aussi l'écrire de façon plus matricielle sous la forme \( \langle x, y\rangle =x^tAy\).} \( \langle x, y\rangle =x\cdot Ay\) est un produit scalaire sur \( \eR^n\). Autrement dit, \( E\) muni de cette forme bilinéaire symétrique est un espace euclidien, ce qui fait dire à la proposition \ref{PropUMtEqkb} qu'il existe une base de \( \eR^n\) orthonormée \( \{ f_i \}_{i=1,\ldots, n}\) pour ce produit scalaire, c'est à dire qu'il existe une matrice \( P\in \GL(n,\eR)\) telle que \( P^tAP=\mtu\). Ici, \( P\) est la matrice de changement de base de la base canonique à notre base orthonormée, c'est à dire la matrice qui fait \( Pe_i=f_i\) pour tout \( i\). Voyons cela avec un peu de détails.

    Pour savoir ce que valent les éléments de la matrice \( P^tAP\), nous nous souvenons que \( P^tAPe_j\) est un vecteur dont les coordonnées sont les éléments de la \( j\)\ieme colonne de \( P^tAP\). Nous avons donc \( (P^tAP)_{ij}=e_i\cdot P^tAPe_i\). Calculons :
    \begin{equation}
            (P^tAP)_{ij}=e_i\cdot P^tAPe_i
            =Pe_i\cdot APe_j
        =f_i\cdot Af_j
        =\langle f_i, f_j\rangle 
        =\delta_{ij}
    \end{equation}
    où nous avons utilisé le fait que \( A\) était auto-adjointe pour la passer de l'autre côté du produit scalaire (usuel). Au final nous avons effectivement \( P^tAP=\mtu\).

    La matrice \( P^tBP\) est une matrice symétrique, donc le théorème spectral \ref{ThoeTMXla} nous donne une matrice \( R\in \gO(n,\eR)\) telle que \( R^tP^tBPR\) soit diagonale. En posant maintenant \( Q=PR\) nous avons la matrice cherchée.
\end{proof}
Note : nous avons prouvé la pseudo-réduction simultanée comme corollaire du théorème de diagonalisation des matrices symétriques \ref{ThoeTMXla}. Il aurait aussi pu être vu comme un corollaire du théorème spectral \ref{ThoRSBahHH} sur les opérateurs auto-adjoints via son corollaire \ref{CorSMHpoVK}.


%+++++++++++++++++++++++++++++++++++++++++++++++++++++++++++++++++++++++++++++++++++++++++++++++++++++++++++++++++++++++++++
\section{Sous espaces caractéristiques}
%+++++++++++++++++++++++++++++++++++++++++++++++++++++++++++++++++++++++++++++++++++++++++++++++++++++++++++++++++++++++++++

% TODO : lire le blog de Pierre Bernard; en particulier celle-ci : http://allken-bernard.org/pierre/weblog/?p=2299

Lorsqu'un opérateur n'est pas diagonalisable, les valeurs propres jouent quand même un rôle important.

\begin{definition}  \label{DefFBNIooCGbIix}
    Soit \( E\) un \( \eK\)-espace vectoriel  \( f\in\End(E)\). Pour \( \lambda\in \eK\) nous définissons
    \begin{equation}
        F_{\lambda}(f)=\{ v\in E\tq (f-\lambda\mtu)^nv=0, n\in\eN \}
    \end{equation}
    et nous appelons ça un \defe{sous-espace caractéristique}{sous-espace!caractéristique} de \( f\).
\end{definition}
L'espace \( F_{\lambda}(f)\) est l'ensemble de nilpotence de l'opérateur \( f-\lambda\mtu\) et

\begin{lemma}   \label{LemBLPooHMAoyJ}
    L'ensemble \( F_{\lambda}(f)\) est non vide si et seulement si \( \lambda\) est une valeur propre de \( f\). L'espace \( F_{\lambda}(f)\) est invariant sous \( f\).
\end{lemma}

\begin{proof}
    Si \( F_{\lambda}(f)\) est non vide, nous considérons \( v\in F_{\lambda}(f)\) et \( n\) le plus petit entier non nul tel que \( (f-\lambda)^nv=0\). Alors \( (f-\lambda)^{n-1}v\) est un vecteur propre de \( f\) pour la valeur propre \( \lambda\). Inversement si \( v\) est une valeur propre de \( f\) pour la valeur propre \( \lambda\), alors \( v\in F_{\lambda}(f)\).

    En ce qui concerne l'invariance, remarquons que \( f\) commute avec \( f-\lambda\mtu\). Si \( x\in F_{\lambda}(f)\) il existe \( n\) tel que \( (f-\lambda\mtu)^nx=0\). Nous avons aussi
    \begin{equation}
        (f-\lambda\mtu)^nf(x)=f\big( (f-\lambda\mtu)^nx \big)=0,
    \end{equation}
    par conséquent \( f(x)\in F_{\lambda}(f)\).
\end{proof}

\begin{remark}  \label{RemBOGooCLMwyb}
    Toute matrice sur \( \eC\) n'est pas diagonalisable : nous en avons déjà donné une exemple simple en \ref{ExBRXUooIlUnSx}. Nous en voyons maintenant un moins simple. Considérons en effet l'endomorphisme \( f\) donné par la matrice
    \begin{equation}
        \begin{pmatrix}
            a&    \alpha    &   \beta    \\
            0    &   a    &   \gamma    \\
            0    &   0    &   b
        \end{pmatrix}
    \end{equation}
    où \( a\neq b\), \( \alpha\neq 0\), \( \beta\) et \( \gamma\) sont des nombres complexes quelconques.
    Son polynôme caractéristique est 
    \begin{equation}
        \chi_f(\lambda)=(a-\lambda)^2(b-\lambda),
    \end{equation}
    et les valeurs propres sont donc \( a\) et \( b\). Nous trouvons les vecteurs propres pour la valeur \( a\) en résolvant
    \begin{equation}
        \begin{pmatrix}
            a    &   \alpha    &   \beta    \\
            0    &   a    &   \gamma    \\
            0    &   0    &   b
        \end{pmatrix}\begin{pmatrix}
            x    \\ 
            y    \\ 
            z    
        \end{pmatrix}=\begin{pmatrix}
            ax    \\ 
            ay    \\ 
            az    
        \end{pmatrix}.
    \end{equation}
    L'espace propre \( E_a(f)\) est réduit à une seule dimension générée par \( (1,0,0)\). De la même façon l'espace propre correspondant à la valeur propre \( b\) est donné par 
    \begin{equation}
        \begin{pmatrix}
            \frac{1}{ b-a }\left( \beta+\frac{ \alpha\gamma }{ b-a } \right)    \\ 
            \frac{ \gamma }{ b-a }    \\ 
            1    
        \end{pmatrix}.
    \end{equation}
    Il n'y a donc pas trois vecteurs propres linéairement indépendants, et l'opérateur \( f\) n'est pas diagonalisable.

    Par contre nous pouvons voir que
    \begin{equation}
        (f-\alpha\mtu)^2\begin{pmatrix}
             0   \\ 
            1    \\ 
            0    
        \end{pmatrix}=
        \begin{pmatrix}
            a    &   \alpha    &   \beta    \\
            0    &   a    &   \gamma    \\
            0    &   0    &   b
        \end{pmatrix}
        \begin{pmatrix}
            \alpha    \\ 
            0    \\ 
            0    
        \end{pmatrix}-\begin{pmatrix}
            a\alpha    \\ 
            0    \\ 
            0    
        \end{pmatrix}=\begin{pmatrix}
            0    \\ 
            0    \\ 
            0    
        \end{pmatrix},
    \end{equation}
    de telle sorte que le vecteur \( (0,1,0)\) soit également dans l'espace caractéristique \( F_a(f)\).

    Dans cet exemple, la multiplicité algébrique de la racine \( a\) du polynôme caractéristique vaut \( 2\) tandis que sa multiplicité géométrique vaut seulement \( 1\).
\end{remark}

%--------------------------------------------------------------------------------------------------------------------------- 
\subsection{Théorèmes de décomposition}
%---------------------------------------------------------------------------------------------------------------------------

%TODO : Je crois qu'on peut remplacer l'hypothèse de corps algébriquement clos par le polynôme caractéristique scindé.
\begin{theorem}[Théorème spectral, décomposition primaire]\index{théorème!spectral}     \label{ThoSpectraluRMLok}
    Soit \( E\) espace vectoriel de dimension finie sur le corps algébriquement clos \( \eK\) et \( f\in\End(E)\). Alors
    \begin{equation}    \label{EqCTFHooBSGhYK}
        E=F_{\lambda_1}(f)\oplus\ldots\oplus F_{\lambda_k}(f)
    \end{equation}
    où la somme est sur les valeurs propres distinctes de \( f\).

    Les projecteurs sur les espaces caractéristique forment un système complet et orthogonal.
\end{theorem}
\index{décomposition!primaire}
\index{décomposition!spectrale}
\index{décomposition!sous-espaces caractéristiques}

\begin{proof}
    Soit \( P\) le polynôme caractéristique de \( f\) et une décomposition
    \begin{equation}
        P=(f-\lambda_1)^{\alpha_1}\ldots(f-\lambda_r)^{\alpha_r}
    \end{equation}
    en facteurs irréductibles. La le théorème de noyaux (\ref{ThoDecompNoyayzzMWod}) nous avons
    \begin{equation}        \label{EqDeFVSaYv}
        E=\ker(f-\lambda_1)^{\alpha_1}\oplus\ldots\oplus\ker(f-\lambda_r)^{\alpha_r}.
    \end{equation}
    Les projecteurs sont des polynômes en \( f\) et forment un système orthogonal. Il nous reste à prouver que \( \ker(f-\lambda_i)^{\alpha_i}=F_{\lambda_i}(f)\). L'inclusion
    \begin{equation}    \label{EqzmNxPi}
        \ker(f-\lambda_i)^{\alpha_i}\subset F_{\lambda_i}(f)
    \end{equation}
    est évidente. Nous devons montrer l'inclusion inverse. Prouvons que la somme des \( F_{\lambda_i}(f)\) est directe. Si \( v\in F_{\lambda_i}(f)\cap F_{\lambda_j}(f)\), alors il existe \( v_1=(f-\lambda_i)^nv\neq 0\) avec \( (f-\lambda_i)v_1=0\). Étant donné que \( (f-\lambda_i)\) commute avec \( (f-\lambda_j)\), ce \( v_1\) est encore dans \( F_{\lambda_j}(f)\) et par conséquent il existe \( w=(f-\lambda_j)^mv_1\) non nul tel que 
    \begin{subequations}
        \begin{numcases}{}
            (f-\lambda_i)w=0\\
            (f-\lambda_j)w=0.
        \end{numcases}
    \end{subequations}
    Ce \( w\) serait donc un vecteur propre simultané pour les valeurs propres \( \lambda_i\) et \( \lambda_j\), ce qui est impossible parce que les espaces propres sont linéairement indépendants. Les espaces \( F_{\lambda_i}\) sont donc en somme directe et
    \begin{equation}
        \sum_i\dim F_{\lambda_i}(f)\leq \dim E.
    \end{equation}
    En tenant compte de l'inclusion \eqref{EqzmNxPi} nous avons même
    \begin{equation}
        \dim E=\sum_i\dim\ker(f-\lambda_i)^{\alpha_i}\leq\sum_i F_{\lambda_i}(f)\leq \dim E.
    \end{equation}
    Par conséquent nous avons \( \dim\ker(f-\lambda_i)^{\alpha_i}=\dim F_{\lambda_i}(f)\) et l'égalité des deux espaces.
\end{proof}


\begin{probleme}
    Dans le cas où le corps n'est pas algébriquement clos, il paraît qu'il faut remplacer «diagonalisable» par «semi-simple».
\end{probleme}
%TODO : peut-être qu'il y a la réponse dans http://www.math.jussieu.fr/~romagny/agreg/dvt/endom_semi_simples.pdf

Si l'espace vectoriel est sur un corps algébriquement clos, alors les endomorphismes semi-simples\footnote{Définition \ref{DEFooBOHVooSOopJN}.} sont les endomorphismes diagonaux.


%TODO : Je crois qu'on peut remplacer l'hypothèse de corps algébriquement clos par le polynôme caractéristique scindé.
\begin{theorem}[Décomposition de Dunford] \label{ThoRURcpW}
    Soit \( E\) un espace vectoriel sur le corps algébriquement clos \( \eK\) et \( u\in\End(E)\) un endomorphisme de \( E\). 
    
    \begin{enumerate}
        \item
            
            L'endomorphisme \( u\) se décompose de façon unique sous la forme
            \begin{equation}
                u=s+n
            \end{equation}
            où \( s\) est diagonalisable, \( n\) est nilpotent et \( [s,n]=0\).
        \item
            Les endomorphismes \( s\) et \( n\) sont des polynômes en \( u\) et commutent avec \( u\).
        \item   \label{ItemThoRURcpWiii}
            Les parties \( s\) et \( n\) sont données par
            \begin{subequations}
                \begin{align}
                    s&=\sum_i\lambda_ip_i\\
                    n&=\sum_i(s-\lambda_i\mtu)p_i
                \end{align}
            \end{subequations}
            où les sommes sont sur les valeurs propres distinctes\footnote{C'est à dire sur les sous-espaces caractéristiques.} de \( f\) et où \( p_i\colon E\to F_{\lambda_i}(u)\) est la projection de \( E\) sur \( F_{\lambda_i}(u)\).
    \end{enumerate}
\end{theorem}
\index{décomposition!Dunford}
\index{Dunford!décomposition}
\index{réduction!d'endomorphisme}
\index{endomorphisme!sous-espace stable}
\index{polynôme!d'endomorphisme!décomposition de Dunford}
\index{endomorphisme!diagonalisable!Dunford}
\index{endomorphisme!nilpotent!Dunford}
%TODO : comprendre comment on calcule des exponentielles de matrices avec Dunford.

\begin{proof}
    Le théorème spectral \ref{ThoSpectraluRMLok} nous indique que
    \begin{equation}
        E=\bigoplus_iF_{\lambda_i}(f).
    \end{equation}
    Nous considérons l'endomorphisme \( s\) de \( E\) qui consiste à dilater d'un facteur \( \lambda\) l'espace caractéristique \( F_{\lambda}(f)\) :
    \begin{equation}
        s=\sum_i\lambda_ip_i
    \end{equation}
    où \( p_i\colon E\to F_{\lambda_i}(u)\) est la projection de \( E\) sur \( F_{\lambda_i}(u)\).

    Nous allons prouver que \( [s,f]=0\) et \( n=f-s\) est nilpotent. Cela impliquera que \( [s,n]=0\).

    Si \( x\in F_{\lambda}(f)\), alors nous avons \( sf(x)=\lambda f(x)\) parce que \( f(x)\in F_{\lambda}(f)\) tandis que \( fs(x)=f(\lambda x)=\lambda f(x)\). Par conséquent \( f\) commute avec \( s\).

    Pour montrer que \( f-s\) est nilpotent, nous en considérons la restriction
    \begin{equation}
        f-s\colon F_{\lambda}(f)\to F_{\lambda}(f).
    \end{equation}
    Cet opérateur est égal à \( f-\lambda\mtu\) et est par conséquent nilpotent.

    Prouvons à présent l'unicité. Soit \( u=s'+n'\) une autre décomposition qui satisfait aux conditions : \( s'\) est diagonalisable, \( n'\) est nilpotent et \( [n',s']=0\). Commençons par prouver que \( s'\) et \( n'\) commutent avec \( u\). En multipliant \( u=s'+n'\) par \( s'\) nous avons
    \begin{equation}
        s'u=s'^2+s'n'=s'^2+n's'=(s'+n')s'=us',
    \end{equation}
    par conséquent \( [u,s']=0\). Nous faisons la même chose avec \( n'\) pour trouver \( [u,n']=0\). Notons que pour obtenir ce résultat nous avons utilisé le fait que \( n'\) et \( s'\) commutent, mais pas leur propriétés de nilpotence et de diagonalisibilité.
    
    
    Si \( s'+n'=s+n\) est une autre décomposition, \( s'\) et \( n'\) commutent avec \( u\), et par conséquent avec tous les polynômes en \( u\). Ils commutent en particulier avec \( n\) et \( s\). Les endomorphismes \( s\) et \( s'\) sont alors deux endomorphismes diagonalisables qui commutent. Par la proposition \ref{PropGqhAMei}, ils sont simultanément diagonalisables. Dans la base de simultanée diagonalisation, la matrice de l'opérateur \( s'-s=n-n'\) est donc diagonale. Mais \( n-n'\) est également nilpotent, en effet si \( A\) et \( B\) sont deux opérateurs nilpotents,
    \begin{equation}
        (A+B)^n=\sum_{k=0}^n\binom{k}{n}A^kB^{n-k}.
    \end{equation}
    Si \( n\) est assez grand, au moins un parmi \( A^k\) ou \( B^{n-k}\) est nul.

    Maintenant que \( n-n'\) est diagonal et nilpotent, il est nul et \( n=n'\). Nous avons alors immédiatement aussi \( s=s'\).

\end{proof}

%--------------------------------------------------------------------------------------------------------------------------- 
\subsection{Diverses conséquences}
%---------------------------------------------------------------------------------------------------------------------------

\begin{theorem}
    Soit une matrice \( A\in \eM(n,\eC)\). On a que la suite \( (A^kx)\) tends vers zéro pour tout \( x\) si et seulement si \( \rho(A)<1\) où \( \rho(A)\)\index{rayon!spectral} est le rayon spectral de $A$
\end{theorem}
\index{décomposition!Dunford!exponentielle de matrice}

\begin{proof}
    Dans le sens direct, il suffit de prendre comme \( x\), un vecteur propre de \( A\). Dans ce cas nous avons \( A^kx=\lambda^kx\). Mais \( \lambda^kx\) ne tend vers zéro que si \( \lambda<1\). Donc toute les valeurs propres de \( A\) doivent être plus petite que \( 1\) et \( \rho(A)<1\).

    Pour l'autre sens nous utilisons la décomposition de Dunford (théorème \ref{ThoRURcpW}) : il existe une matrice inversible \( P\) telle que
    \begin{equation}
        A=P^{-1}(D+N)P
    \end{equation}
    où \( D\) est diagonale, \( N\) est nilpotente et \( [D,N]=0\). Étant donné que \( D+N\) est triangulaire, son polynôme caractéristique que
    \begin{equation}
        \chi_{D+N}(\lambda)=\prod_i D_{ii}-\lambda.
    \end{equation}
    Par similitude, c'est le même polynôme caractéristique que celui de \( A\) et nous savons alors que la diagonale de \( D\) contient les valeurs propres de \( A\).

    Par ailleurs nous avons
    \begin{subequations}
        \begin{align}
            A^k&=P^{-1}(D+N)^kP\\
            &=P^{-1}\sum_{j=0}^k{j\choose k}D^{j-k}N^jP\\
            &=P^{-1}\sum_{j=0}^{n-1}{j\choose k}D^{j-k}N^jP
        \end{align}
    \end{subequations}
    où nous avons utilité le fait que \( D\) et \( N\) commutent ainsi que \( N^{n-1}=0\) parce que \( N\) est nilpotente. Nous utilisons la norme matricielle usuelle, pour laquelle \( \| D \|=\rho(D)=\rho(A)\). Nous avons alors
    \begin{equation}
        \| (D+N)^k \|\leq \sum_{j=0}^k{j\choose k}\rho(D)^{k-j}\| N \|^j.
    \end{equation}
    Du coup si \( \rho(D)<1\) alors \( \| (D+N)^k \|\to 0\) (et c'est même un si et seulement si).
\end{proof}

Une application de la décomposition de Jordan est l'existence d'un logarithme pour les matrices. La proposition suivant va d'une certaine manière donner un logarithme pour les matrices inversibles complexes. Dans le cas des matrices réelles \( m\) telles que \( \| m-\mtu \|<1\), nous donnerons au lemme \ref{LemQZIQxaB} une formule pour le logarithme sous forme d'une série; ce logarithme sera réel.
\begin{proposition} \label{PropKKdmnkD}
    Toute matrice inversible complexe est une exponentielle.
\end{proposition}
\index{exponentielle!de matrice}
\index{décomposition!Jordan!et exponentielle de matrice}

\begin{proof}
    Soit \( A\in \GL(n,\eC)\); nous allons donner une matrice \( B\in \eM(n,\eC)\) telle que \( A=\exp(B)\). D'abord remarquons qu'il suffit de prouver le résultat pour une matrice par classe de similitude. En effet si \( A=\exp(B)\) et si \( M\) est inversible alors 
    \begin{subequations}    \label{EqqACuGK}
        \begin{align}
            \exp(MBM^{-1})&=\sum_k\frac{1}{ k! }(MBM^{-1})^k\\
            &=\sum_k\frac{1}{ k! }MB^kM^{-1}\\
            &=M\exp(B)M^{-1}.
        \end{align}
    \end{subequations}
    Donc \( MAM^{-1}=\exp(MBM^{-1})\). Nous pouvons donc nous contenter de trouver un logarithme pour les blocs de Jordan. Nous supposons donc que \( A=(\mtu+N)\) avec \( N^m=0\). 
    En nous inspirant de \eqref{EqweEZnV}, nous posons\footnote{Le logarithme d'un nombre n'est pas encore définit à ce moment, mais cela ne nous empêche pas de poser une définition ici pour une application des réels vers les matrices.}
    \begin{equation}
        D(t)=tN-\frac{ t^2 }{ 2 }N^2+\cdots +(-1)^m\frac{ t^{m-1} }{ m-1 }N^{m-1}
    \end{equation}
    et nous allons prouver que \(  e^{D(1)}=\mtu+N\). Notons que \( N\) étant nilpotente, cette somme ainsi que toutes celles qui viennent sont finies. Il n'y a donc pas de problèmes de convergences dans cette preuve (si ce n'est les passages des équations \eqref{EqqACuGK}).

    Nous posons \( S(t)= e^{D(t)}\) (la somme est finie), et nous avons
    \begin{equation}
        S'(t)=D'(t) e^{D(t)}
    \end{equation}
    Afin d'obtenir une expression qui donne \( S'\) en termes de \( S\), nous multiplions par \( (\mtu+tN)\) en remarquant que \( (\mtu+tN)D'(t)=N\) nous avons
    \begin{equation}
        (\mtu+tN)S'(t)=NS(t).
    \end{equation}
    En dérivant à nouveau,
    \begin{equation}    \label{EqKjccqP}
        (\mtu+tN)S''(t)=0.
    \end{equation}
    La matrice \( (\mtu+tN)\) est inversible parce que son noyau est réduit à \( \{ 0 \}\). En effet si \( (\mtu+tN)x=0\), alors \( Nx=-\frac{1}{ t }x\), ce qui est impossible parce que \( N\) est nilpotente. Ce que dit l'équation \eqref{EqKjccqP} est alors que \( S''(t)=0\). Si nous développons \( S(t)\) en puissances de \( t\) nous nous arrêtons au terme d'ordre \( 1\) et nous avons
    \begin{equation}
        S(t)=S(0)+tS'(0)=\mtu+tD'(0)=1+tN.
    \end{equation}
    En \( t=1\) nous trouvons \( S(1)=\mtu+N\). La matrice \( D(1)\) donnée est donc bien un logarithme de $\mtu+N$.
\end{proof}

\begin{proposition}[\cite{fJhCTE}]
    Si \( A\in \eM(n,\eC)\) est telle que \( \rho(A)<1\), alors \( A^n\to 0\).
\end{proposition}

\begin{proof}
    Nous nous plaçons dans une base des espaces caractéristiques de \( A\), c'est à dire que nous supposons que la matrice \( A\) a la forme
    \begin{equation}        \label{EqWMvkgLo}
        A=\begin{pmatrix}
            \lambda_1\mtu+N_1    &       &       \\
                &   \ddots    &       \\
                &       &   \lambda_s\mtu+N_s
        \end{pmatrix}
    \end{equation}
    où les \( \lambda_i\) sont les valeurs propres de \( A\) et les \( N_i\) sont nilpotentes. En effet nous savons que l'espace caractéristique \( F_{\lambda_i}\) est l'espace de nilpolence de \( A-\lambda_i\mtu\). Si nous notons \( A_i\) la restriction de \( A\) à cet espace, la matrice \( N_i=A_i-\lambda_i\mtu\) est nilpotente. Du coup \( A_i=\lambda_I\mtu+N_i\) et nous avons bien la décomposition \eqref{EqWMvkgLo}.

    Nous avons donc \( A^n\to 0\) si et seulement si \( (N_i+\lambda_i\mtu)^n\to 0\) pour tout \( i\). Soit donc \( N\) nilpotente et \( \lambda<1\) (parce que nous savons que toutes les valeurs propres de \( A\) sont inférieures à un). Nous avons
    \begin{equation}
            (\lambda\mtu+N)^n=\sum_{k=0}^n\binom{ n }{ k }\lambda^{n-k}N^{k}
            =\sum_{k=0}^{r-1}\binom{ n }{ k }\lambda^{n-k}N^{k}.
    \end{equation}
    Nous voyons que le nombre de termes dans la somme ne dépend pas de \( n\). De plus pour chacun de termes, la puissance de \( N\) ne dépend pas non plus de \( n\). Le terme
    \begin{equation}
        \binom{ n }{ k }\lambda^{n-k}\leq P(n)\lambda^{n-k}
    \end{equation}
    où \( P\) est un polynôme tend vers zéro lorsque \( n\) devient grand parce que c'est une cas polynôme fois exponentielle.
\end{proof}

%--------------------------------------------------------------------------------------------------------------------------- 
\subsection{Diagonalisabilité d'exponentielle}
%---------------------------------------------------------------------------------------------------------------------------

\begin{proposition}[\cite{fJhCTE}]      \label{PropCOMNooIErskN}
    Si \( A\in \eM(n,\eR)\) a un polynôme caractéristique scindé, alors \( A\) est diagonalisable si et seulement si \( e^A\) est diagonalisable.
\end{proposition}
\index{décomposition!Dunford!application}
\index{exponentielle!de matrice}
\index{diagonalisable!exponentielle}

\begin{proof}
    Si \( A\) est diagonalisable, alors il existe une matrice inversible \( M\) telle que \( D=M^{-1}AM\) soit diagonale (c'est la définition \ref{DefCNJqsmo}). Dans ce cas nous avons aussi \( (M^{-1}AM)^k=M^{-1}A^kM\) et donc \( M^{-1}e^AM=e^{M^{-1}AM}=e^D\) qui est diagonale.

    La partie difficile est donc le contraire. 
    
    \begin{subproof}
        \item[Qui est diagonalisable et comment ?]
            Nous supposons que \( e^A\) est diagonalisable et nous écrivons la décomposition de Dunford (théorème \ref{ThoRURcpW}) :
            \begin{equation}
                A=S+N
            \end{equation}
            où \( S\) est diagonalisable, \( N\) est nilpotente, \( [S,N]=0\). Nous avons besoin de prouver que \( N=0\).
    
            Les matrices \( A\) est \( S\) commutent; en passant au développement nous en déduisons que \( A\) et \( e^S\) commutent, puis encore en passant au développement que \( e^A\) et \( e^S\) commutent. Vu que \( S\) est diagonalisable, \( e^S\) l'est et par hypothèse \( e^A\) est également diagonalisable. Donc \( e^A\) et \( e^{-S}\) sont simultanément diagonalisables par la proposition \ref{PropGqhAMei}.

            Étant donné que \( A\) et \( S\) commutent, nous avons \( e^N=e^{A-S}=e^Ae^{-S}\), et nous en déduisons que \( e^N\) est diagonalisable vu que les deux facteurs \( e^A\) et \( e^{-S}\) sont simultanément diagonalisables.

        \item[Unipotence]

            Si \( r\) est le degré de nilpotence de \( N\), nous avons
            \begin{equation}    \label{EqQHjvLZQ}
                e^N-\mtu=N+\frac{ N^2 }{2}+\cdots +\frac{ N^{r-1} }{ (r-1)! }.
            \end{equation}
            Donc
            \begin{equation}
                (e^N-\mtu)^k=\left( N+\frac{ N^2 }{2}+\cdots +\frac{ N^{r-1} }{ (r-1)! } \right)^k
            \end{equation}
            où le membre de droite est un polynôme en \( N\) dont le terme de plus bas degré est de degré \( k\). Donc \( (e^N-\mtu)\) est nilpotente et \( e^N\) est unipotente.

            Si \( M\) est la matrice qui diagonalise \( e^N\), alors la matrice diagonale \( M^{-1}e^NM\) est tout autant unipotente que \( e^N\) elle-même. En effet,
            \begin{subequations}
                \begin{align}
                    (M^{-1}e^NM-\mtu)^r&=\sum_{k=0}^r\binom{ r }{ k }(-1)^{r-k}M^{-1}(e^N)^kM\\
                    &=M^{-1}\left( \sum_{k=0}^r\binom{ r }{ k }(-1)^{r-k}(e^N)^k \right)M\\
                    &=M^{-1}(e^N-\mtu)^rM\\
                    &=0.
                \end{align}
            \end{subequations}

            La matrice \( M^{-1}e^NM\) est donc une matrice diagonale et unipotente; donc \( M^{-1}e^NM=\mtu\), ce qui donne immédiatement que \( e^N=\mtu\).

        \item[Polynômes annulateurs]

            En reprenant le développement \eqref{EqQHjvLZQ} sachant que \( e^N=\mtu\), nous savons que
            \begin{equation}
                N+\frac{ N^2 }{2}+\cdots +\frac{ N^{r-1} }{ (r-1)! }=0.
            \end{equation}
            Dit en termes pompeux (mais non moins porteurs de sens), le polynôme
            \begin{equation}
                Q(X)=X+\frac{ X^2 }{2}+\cdots +\frac{ X^{r-1} }{ (r-1)! }
            \end{equation}
            est un polynôme annulateur de \( N\).
            
            La proposition \ref{PropAnnncEcCxj} stipule que le polynôme minimal d'un endomorphisme divise tous les polynômes annulateurs. Dans notre cas, \( X^r\) est un polynôme annulateur et donc le polynôme minimal de \( N\) est de la forme \( X^k\). Donc il est \( X^r\) lui-même.
            
            Nous avons donc \( X^r\divides Q\). Mais \( Q\) est un polynôme contenant le monôme \( X\) donc \( X^r\) ne peut diviser \( Q\) que si \( r=1\). Nous en concluons que \( X\) est un polynôme annulateur de \( N\). C'est à dire que \( N=0\).

        \item[Conclusion]

            Vu que Dunford\footnote{Théorème \ref{ThoRURcpW}.} dit que \( A=S+N\) et que nous venons de prouver que \( N=0\), nous concluons que \( A=S\) avec \( S\) diagonalisable.

    \end{subproof}
\end{proof}

%---------------------------------------------------------------------------------------------------------------------------
\subsection{Valeurs singulières}
%---------------------------------------------------------------------------------------------------------------------------

\begin{definition}
    Soit \( M\) une matrice \( m\times n\) sur \( \eK\) (\( \eK\) est \( \eR\) ou \( \eC\)). Un nombre réel \( \sigma\) est une \defe{valeur singulière}{valeur!singulière} de \( M\) s'il existent des vecteurs unitaires \( u\in \eK^m\), \( v\in \eK^n\) tels que
    \begin{subequations}
        \begin{align}
            Mv&=\sigma u\\
            M^*u&=\sigma v.
        \end{align}
    \end{subequations}
\end{definition}

\begin{theorem}[Décomposition en valeurs singulières]
    Soit \( M\in \eM(m\times n,\eK)\) où \( \eK=\eR,\eC\). Alors \( M\) se décompose en
    \begin{equation}
        M=ADB
    \end{equation}
    où
    il existe deux matrices unitaires \( A\in \gU(m\times m)\), \( B\in \gU(n\times n)\) et une matrice (pseudo)diagonale \( D\in \eM(m\times n)\) tels que
    \begin{enumerate}
        \item 
            \( A\in\gU(m\times m)\), \( B\in\gU(n\times n)\) sont deux matrices unitaires;,
        \item
            \( D\) est (pseudo)diagonale,
        \item
            les éléments diagonaux de \( D\) sont les valeurs singulières de \( M\),
        \item
            le nombre d'éléments non nuls sur la diagonale de \( D\) est le rang de \( M\).
    \end{enumerate}
\end{theorem}

\begin{corollary}
    Soit \( M\in \eM(n,\eC)\). Il existe un isomorphisme \( f\colon \eC^n\to \eC^n\) tel que \( fM\) soit autoadjoint.
\end{corollary}

\begin{proof}
    Si \( M=ADB\) est la décomposition de \( M\) en valeurs singulières, alors nous pouvons prendre \( f=\overline{ B }^tA^{-1}\) qui est une matrice inversible. Pour la vérification que ce \( f\) répond bien à la question, ne pas oublier que \( D\) est réelle, même si \( M\) ne l'est pas.
\end{proof}
