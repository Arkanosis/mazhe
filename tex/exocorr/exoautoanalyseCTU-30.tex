 % This is part of Analyse Starter CTU
% Copyright (c) 2014
%   Laurent Claessens,Carlotta Donadello
% See the file fdl-1.3.txt for copying conditions.

\begin{exercice}\label{exoautoanalyseCTU-30}


    On définit pour tout $n\in\mathbb{N}$ la fonction $f_n$ sur $\eR^+$ par $f_n(x)=
    x^n e^{-x}$. 
    
    Soit $a>0$, on pose :
       $ I_n(a) =\displaystyle \int_0^a f_n(x) \,\mathrm dx$

    \begin{enumerate}
        \item Déterminer $I_0(a)$.
        \item Montrer que $I_0(a)$ admet une limite lorsque $a\to+\infty$. 
        
        On notera cette limite
            $\displaystyle\int_0^{+\infty} f_0(x)\,\mathrm dx$.
        \item Déterminer une relation de récurrence entre $I_n(a)$ et $I_{n+1}(a)$.
        \item Montrer que pour tout $n$, $I_n(a)$ admet une limite lorsque $a$ tend vers $+\infty$.
        
            On notera cette limite $\displaystyle\int_0^{+\infty} f_n(x)\,\mathrm dx$.
        \item Démontrer que :
               $ \displaystyle\int_0^{+\infty} x^n e^{-x} \,\mathrm dx = n\;!$
       
    \end{enumerate}



\corrref{exoautoanalyseCTU-30}
\end{exercice}
