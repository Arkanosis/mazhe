% This is part of the Exercices et corrigés de CdI-2.
% Copyright (C) 2008, 2009, 2012
%   Laurent Claessens
% See the file fdl-1.3.txt for copying conditions.


\begin{corrige}{112}

%TODO: refaire le dessin
%Un petit graphe de la fonction est donné à la figure \ref{LabelFigexouud}
%\newcommand{\CaptionFigexouud}{Les parties non nulles de quelques unes des fonctions $f_n$.}
%\input{auto/pictures_tex/Fig_exouud.pstricks}

\begin{enumerate}
\item La suite converge sur $[0,1]$ vers la fonction
\begin{equation}
	f(x)=
\begin{cases}
	1	&	\text{si }x=0\\
	0	&	 \text{si }x\neq 0.
\end{cases}
\end{equation}

\item La convergence n'est pas uniforme parce que la limite n'est pas continue. Sur $]0,1]$, la convergence n'est pas uniforme non plus parce que $\forall n$, il existe $x\in]0,1]$ tel que $f_n(x)=\frac{ 1 }{2}$ et $f(x)=0$, à cause du théorème des valeurs intermédiaires.

\item Soit $a$, le minimum du compact $K$. On a que $| f_n(x) |\leq | f_n(a) |$ pour tout $x\in K$. Autrement dit,
\begin{equation}
	\| f_n \|_{\infty}\leq | f_n(a) |.
\end{equation}
Or, la suite numérique $| f_n(a) |$ tends vers zéro lorsque $n\to\infty$, donc nous avons convergence uniforme sur le compact $K$.
\end{enumerate}
\end{corrige}
