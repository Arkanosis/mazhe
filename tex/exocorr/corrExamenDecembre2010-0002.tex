% This is part of Exercices de mathématique pour SVT
% Copyright (c) 2011
%   Laurent Claessens et Carlotta Donadello
% See the file fdl-1.3.txt for copying conditions.

\begin{corrige}{ExamenDecembre2010-0002}

	\begin{enumerate}
		\item
			Nous mettons en évidence la plus grande puissance de $x$ au numérateur et au dénominateur. Attention : celle du dénominateur est $5$ :
			\begin{equation}
				\frac{ x^3+4x+13 }{ x+x^5+2 }=\frac{ x^3\left( 1+\frac{ 4 }{ x^2 }+\frac{ 12 }{ x^3 } \right) }{ x^5\left( \frac{1}{ x^4 }+1+\frac{ 2 }{ x^5 } \right) }.
			\end{equation}
			À ce moment, nous simplifions par $x^3$, et nous nous souvenons qu'à la limite $x\to\infty$, les termes du type $\frac{ 4 }{ x^2 }$ tendent vers $0$. Il reste donc $\lim_{x\to \infty} \frac{1}{ x^2 }=0$.

			La réponse est donc zéro.

		\item
			Ici, en remplaçant simplement $x$ par zéro dans l'expression, nous ne tombons sur aucune indétermination : $\frac{ 13 }{2}$.
		\item
			La limite très connue est $\lim_{x\to 0} \frac{ \sin(x) }{ x }=1$. Ici nous écrivons
			\begin{equation}
				\frac{ \sin^2(x) }{ x }=\sin(x)\frac{ \sin(x) }{ x }.
			\end{equation}
			Nous avons alors
			\begin{equation}
				\lim_{x\to 0} \sin(x)\frac{ \sin(x) }{ x }=\lim_{x\to 0} \sin(x)\lim_{x\to 0} \frac{ \sin(x) }{ x }
			\end{equation}
		\item
			En remplaçant $x$ par $5$, nous tombons sur l'indétermination $\frac{ 0 }{ 0 }$. Nous parvenons à la lever en factorisant le numérateur et en simplifiant :
			\begin{equation}
				\frac{ x^2-25 }{ x-5 }=\frac{(x-5)(x+5) }{ x-5 }=x+5.
			\end{equation}
			Maintenant, la limite vaut $\lim_{x\to 5} x+5=10$.
		\item
			Nous avons vu que, pour les limites en l'infini, l'exponentielle «avance plus vite que tout polynôme». La limite est donc $\infty$.
		\item
			En remplaçant nous trouvons $\frac{ 0 }{ -\infty }$. Cela n'est pas une indétermination : ça vaut $0$.
	\end{enumerate}

\end{corrige}
