% This is part of the Exercices et corrigés de CdI-2.
% Copyright (C) 2008, 2009, 2012
%   Laurent Claessens
% See the file fdl-1.3.txt for copying conditions.


\begin{corrige}{117}

\begin{enumerate}
\item La fonction $f_n(x)=nx e^{-nx^2}$ de l'exercice \ref{exo116} fait presque ce qu'il faut, mais ce serait bien que la formule \eqref{Eqintfexpemoin} diverge un peu plus vis-à-vis de $n$. Essayons donc $f_n(x)=n^2x e^{-nx^2}$. Par construction, nous avons
\begin{equation}
	\lim_{n\to\infty}\int_0^1f_n(x)dx=\infty,
\end{equation}
et, étant donné que pour tout $x$ nous avons $\lim_{n\to\infty}f(x)$=0, la permutation de la limite avec l'intégrale donne
\begin{equation}
	\int_0^1\lim_{n}f_n=0.
\end{equation}
Le simple fait que les deux intégrales ne coïncident pas prouve que la suite des $f_n$ ne converge pas uniformément sur $[0,1]$.

\item La suite de fonctions de l'exercice \ref{exo116} répond à la question.

\item
    % TODO : quand le dessin correspondant sera fait, remettre cette ligne.
    %En regardant les dessins de la figure \ref{LabelFigexouud}, ou en faisant les calculs
    Faisant les calculs, nous voyons que pour la suite de fonctions de l'exercice \ref{exo112}, nous avons
\begin{equation}
	\lim_{n\to\infty}\int_0^1f_n(x)dx=\int_0^1\lim_{n\to\infty}f_n(x)dx=0.
\end{equation}
Il est intéressant de remarquer ce, \emph{dans ce cas}, on peut permuter la limite et l'intégrale sans changer la valeur de l'intégrale.
\end{enumerate}

\end{corrige}

