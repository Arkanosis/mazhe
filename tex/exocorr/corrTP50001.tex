% This is part of the Exercices et corrigés de mathématique générale.
% Copyright (C) 2009
%   Laurent Claessens
% See the file fdl-1.3.txt for copying conditions.
\begin{corrige}{TP50001}

	\begin{enumerate}

		\item
			


	Montrons que la partie est génératrice, c'est à dire que tout polynôme de la forme
	\begin{equation}		\label{Eqpolyatrouver}
		ax^3+bx^2+cx+d
	\end{equation}
	peut être écrit sous la forme
	\begin{equation}		\label{EqPOlycofLL}
		\begin{aligned}[]
			&\lambda_1(1)\\
			&+\lambda_2(1-x)\\
			&+\lambda_3(1+x-x^2)\\
			&+\lambda_4(1+x+x^2-x^3).
		\end{aligned}
	\end{equation}
	Pour ce faire, nous égalisons les coefficients des termes en chaque puissance de $x$ entre le polynôme \eqref{Eqpolyatrouver} et \eqref{EqPOlycofLL}. Nous trouvons
	\begin{equation}
		\begin{aligned}[]
			x^3&\leadsto &a&=-\lambda_4\\
			x^2&\leadsto &b&=\lambda_4-\lambda_3\\
			x&\leadsto &c&=\lambda_4+\lambda_3-\lambda_2\\
			\text{terme indépendant}&\leadsto &d&=\lambda_4+\lambda_3+\lambda_3+\lambda_1,
		\end{aligned}
	\end{equation}
	d'où nous déduisons que
	\begin{equation}		\label{EqSolspolyTPc}
		\begin{aligned}[]
			\lambda_4&=-a\\
			\lambda_3&=-b-a\\
			\lambda_2&=-c-b-2a\\
			\lambda_3&=d+c+2b+4a.
		\end{aligned}
	\end{equation}
	Cela montre que chaque vecteur de l'espace considéré s'écrit de façon unique comme combinaison des vecteurs de la partie $B$. Ceci est exactement dire que $B$ est une base.


		\item
			Trouver les coefficients de $1+x+x^2+x^3$ est facile : il suffit de reprendre la solution \eqref{EqSolspolyTPc} avec $a=b=c=d=1$. Cela fait $\lambda_4=-1$, $\lambda_3=-2$, $\lambda_2=-4$, $\lambda_1=8$.

		\item
			C'est ici que les choses compliquées commencent. Il s'agit d'abord de voir les polynômes de base comme des vecteurs. Cela se fait par l'identification suivante :
			\begin{equation}
				\begin{aligned}[]
					1&=\begin{pmatrix}
						1	\\ 
						0	\\ 
						0	\\ 
						0	
					\end{pmatrix},
					&1-x&=\begin{pmatrix}
						0	\\ 
						1	\\ 
						1	\\ 
						1	
					\end{pmatrix},
					&1+x-x^2&=\begin{pmatrix}
						0	\\ 
						0	\\ 
						1	\\ 
						0	
					\end{pmatrix},
					&1+x+x^2-x^3&=\begin{pmatrix}
						0	\\ 
						0	\\ 
						0	\\ 
						1	
					\end{pmatrix}.
				\end{aligned}
			\end{equation}
			Maintenant nous devons voir comment la dérivation transforme ces vecteurs les uns en combinaisons des autres. En effet, dans la matrice de $\varphi$, la première colonne sera les composantes de l'image du premier vecteur, et ainsi de suite pour les autres colonnes. Nous avons
			\begin{equation}
				\begin{aligned}[]
					\varphi\begin{pmatrix}
						1	\\ 
						0	\\ 
						0	\\ 
						0	
					\end{pmatrix}&=(1)'=0=\begin{pmatrix}
						0	\\ 
						0	\\ 
						0	\\ 
						0	
					\end{pmatrix},\\
					\varphi\begin{pmatrix}
						0	\\ 
						1	\\ 
						0	\\ 
						0	
					\end{pmatrix}&=(1-x)'=-1=\begin{pmatrix}
						-1	\\ 
						0	\\ 
						0	\\ 
						0	
					\end{pmatrix},\\
					\varphi\begin{pmatrix}
						0	\\ 
						0	\\ 
						1	\\ 
						0	
					\end{pmatrix}&=(1+x-x^2)'=-2x+1=2(1-x)-1\\
					&=2\begin{pmatrix}
						0	\\ 
						1	\\ 
						0	\\ 
						0	
					\end{pmatrix}-\begin{pmatrix}
						1	\\ 
						0	\\ 
						0	\\ 
						0	
					\end{pmatrix}=\begin{pmatrix}
						-1	\\ 
						2	\\ 
						0	\\ 
						0	
					\end{pmatrix},\\
					\varphi\begin{pmatrix}
						0	\\ 
						0	\\ 
						0	\\ 
						1	
					\end{pmatrix}&=(1+x+x^2-x^3)'=3(1+x-x^2)+(1-x)-3\\
					&=3\begin{pmatrix}
						0	\\ 
						0	\\ 
						1	\\ 
						0	
					\end{pmatrix}+\begin{pmatrix}
						0	\\ 
						1	\\ 
						0	\\ 
						0	
					\end{pmatrix}-3\begin{pmatrix}
						1	\\ 
						0	\\ 
						0	\\ 
						0	
					\end{pmatrix}=\begin{pmatrix}
						-3	\\ 
						1	\\ 
						3	\\ 
						0	
					\end{pmatrix}.
				\end{aligned}
			\end{equation}
			La matrice de $\varphi$ consiste à mettre les images de $\varphi$ en colonnes, c'est à dire
			\begin{equation}
				\begin{pmatrix}
					 0	&	-1	&	-1	&	-3	\\
					 0	&	0	&	2	&	1	\\
					 0	&	0	&	0	&	3	\\ 
					 0	&	0	&	0	&	0	 
					  \end{pmatrix}.
			\end{equation}
			

	\end{enumerate}

\end{corrige}
