% This is part of the Exercices et corrigés de CdI-2.
% Copyright (C) 2008, 2009
%   Laurent Claessens
% See the file fdl-1.3.txt for copying conditions.


\begin{corrige}{_I-2-3}

L'intégrante est bornée sur l'intervalle $[a+\epsilon,b]$, donc nous nous bornons\footnote{Il n'y a pas que les intégrales qui sont bornées.} à étudier l'intégrale
\begin{equation}
	\int_a^{a+\epsilon}\left| \frac{ P(x) }{ Q(x) }  \right|dx.
\end{equation}
Étant donné que $Q(a)=0$, il existe un entier $q\geq 1$ et un polynôme $S(x)$ ne s'annulant pas en $a$ tels que
\begin{equation}
	Q(x)=(x-a)^qS(x).
\end{equation}
Par continuité, si $\epsilon$ est assez petit, le polynôme $S$ ne s'annule pas sur $[a,a+\epsilon]$. Si $M$ est une minoration de $| P(x)/S(x) |$ sur l'intervalle considéré, nous avons
\begin{equation}
	\int_a^{a+\epsilon}\left| \frac{ P(x) }{ Q(x) }  \right|dx  
	=\int_a^{a+\epsilon} \frac{ |P(x)| }{ (x-a)^q | S(x) | }dx 
	\geq M\int_a^{a+\epsilon}\frac{1}{ (x-a)^q }dx.
\end{equation}
Étant donné que, par construction, $q\geq 1$, cette dernière intégrale n'existe jamais. 

\end{corrige}
