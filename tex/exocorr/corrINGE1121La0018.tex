% This is part of the Exercices et corrigés de mathématique générale.
% Copyright (C) 2009-2010
%   Laurent Claessens
% See the file fdl-1.3.txt for copying conditions.


\begin{corrige}{INGE1121La0018}

	Calculons $Av$ afin de savoir la valeur propre associée au vecteur donné :
	\begin{equation}
		\begin{pmatrix}
			 2	&	1	&	-1	&	1	\\
			 1	&	0	&	1	&	1	\\
			 -1	&	1	&	2	&	1	\\ 
			 1	&	1	&	1	&	0	 
		 \end{pmatrix}
		 \begin{pmatrix}
			 -1	\\ 
			 0	\\ 
			 1	\\ 
			 0	
		 \end{pmatrix}
		 =
		 \begin{pmatrix}
			 -3	\\ 
			 0	\\ 
			 3	\\ 
			 0	
		 \end{pmatrix}.
	\end{equation}
	La valeur propre est donc $3$. Nous savons donc que $(\lambda-3)$ pourra être factorisé dans le polynôme caractéristique.

	Pour le reste de l'exercice c'est standard et c'est résolu de la façon suivante :

	\VerbatimInput[tabsize=3]{src_sage/exo65.sage}

	qui retourne

	\VerbatimInput[tabsize=3]{src_sage/exo65.txt}

	Un point délicat peut être la résolution du polynôme caractéristique. Nous trouvons
	\begin{equation}	\label{EqDHpolycar}
		(1+\lambda)\lambda^3-5\lambda^2+3\lambda+9=0,
	\end{equation}
	mais nous savons que $\lambda=3$ est une solution, donc nous pouvons diviser (Euclide) le membre de gauche de \eqref{EqDHpolycar} par $\lambda-3$ et trouver 
	\begin{equation}
		(1+\lambda)(\lambda-3)(\lambda^2-2\lambda-3)=0
	\end{equation}
	dont les solutions sont $3$ et $-1$, chacune étant de multiplicité deux.

	Conseil technique : on peut utiliser Gram-Schmidt séparément sur les deux vecteurs de valeur propre $-1$ et sur les deux de valeur propre $3$. En effet, ces deux «paquets» de vecteurs propres sont orthogonaux.

\end{corrige}
