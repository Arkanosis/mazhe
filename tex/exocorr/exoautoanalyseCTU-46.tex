 % This is part of Analyse Starter CTU
% Copyright (c) 2014
%   Laurent Claessens,Carlotta Donadello
% See the file fdl-1.3.txt for copying conditions.

\begin{exercice}\label{exoautoanalyseCTU-46}

Soit la fonction $f$ de la variable réelle $x$ définie par $f(x)=\arctan (x) - \arctan (3x)$.
\begin{enumerate}
    \item Déterminer l'ensemble de définition, les variations et les limites aux bords du domaine de la fonction $f$.
    \item En utilisant le point précédent, montrer que $f$ est une bijection de $I=\left]-\infty, -\dfrac{1}{\sqrt{3}}\right]$ vers un intervalle que l'on précisera.
    \item Démontrer que pour tout réel $x$, $f(x)=-\arctan \left(\dfrac{2x}{1+3x^2}\right)$. 
    \item Trouver une expression de la fonction réciproque de $f$ sur $I$ à partir de l'expression donné dans le point précedent. 
    \item La fonction $f$ réalise aussi une bijection de $\left] -\dfrac{1}{\sqrt{3}}, \dfrac{1}{\sqrt{3}}\right[$ vers $\left] -\dfrac{\pi}{6},\dfrac{\pi}{6}\right[$. Est-il possible d'utiliser entre ces deux intervalles la fonction réciproque trouvée au point précedent ? Motiver votre réponse.
\end{enumerate}

\corrref{exoautoanalyseCTU-46}
\end{exercice}
