\begin{exercice}\label{exoCalculDifferentiel0004}

On considère l'application $f : \eR ^2 \to \eR $ définie par
\begin{equation}
	f(x,y) =
	\begin{cases}
		\frac{x^3y}{x^4+y^2} 	&	\text{si }(x,y)\neq(0,0)\\
		0			&	 \text{si }(x,y)=(0,0)
	\end{cases}
\end{equation}

\begin{enumerate}
	\item
 Étudier   la continuité de $f$ au point $(x,y)=(0,0)$ (indication : remarquer que $2x^2y \le x^4+y^2$).
 \item
 Montrer que $f$ admet des dérivées partielles premières au point $(0,0)$ mais n'est pas différentiable en $(0,0)$.
\end{enumerate}


\corrref{CalculDifferentiel0004}
\end{exercice}
