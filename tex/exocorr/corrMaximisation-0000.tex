% This is part of the Exercices et corrigés de mathématique générale.
% Copyright (C) 2010
%   Laurent Claessens
% See the file fdl-1.3.txt for copying conditions.

\begin{corrige}{Maximisation-0000}

	Cette fois, les calculs ne sont pas compliqués. Le code Sage est :

	\VerbatimInput[tabsize=3]{src_sage/exo1011.sage}

	La sortie est :

	\VerbatimInput[tabsize=3]{src_sage/exo1011.txt}

	Il n'est pas capable de conclure, mais les informations sont intéressantes quand même et répondent à la première question.

	D'abord, notons que $f(0,0)=0$.

	Si nous prenons la droite\footnote{Notez que ce chemin ne donne pas toutes les droites lorsque $k$ varie : il manque la droite verticale $(0,t)$.} $(t,kt)$ et calculons $f$ dessus :
	\begin{equation}
		f(t,kt)=t^3-3kt^3=t^3(1-3k).
	\end{equation}
	Cela change de signe quand $t$ change de signe. Dans tout voisinage de $(0,0)$, il y a donc des points avec $f>0$ et des points avec $f<0$. Nous concluons que le point $(0,0)$ est un point de selle.

\end{corrige}
