% This is part of Exercices de mathématique pour SVT
% Copyright (c) 2011
%   Laurent Claessens et Carlotta Donadello
% See the file fdl-1.3.txt for copying conditions.

\begin{corrige}{ExamenDecembre2010-0005}

	\begin{enumerate}
		\item
			\begin{enumerate}
				\item
					Dire que $y(x)$ est une constante, c'est dire que $y(x)=C$ pour un certain $C\in\eR$. Dans ce cas $y'(x)=0$ et l'équation devient $0=-xC$ pour tout $x$. Le seul choix de constante $C$ qui convient est $C=0$.

					Par conséquent, la fonction $y(x)=0$ est une solution constante de l'équation $y'=xy$.
				\item
					Pour trouver la solution générale, nous écrivons $y'=dy/dx$ et nous mettons tous les $y$ à gauche et tous les $x$ à droite :
					\begin{equation}
						\begin{aligned}[]
							\frac{ dy }{ dx }&=-xy\\
							\frac{ dy }{ y }&=-xdx\\
							\ln(y)&=-\frac{ x^2 }{2}+C.
						\end{aligned}
					\end{equation}
					Pour obtenir la dernière ligne, nous avons intégré des deux côtés de l'équation. Nous prenons maintenant l'exponentielle des deux membres afin d'isoler le $y$ :
					\begin{equation}
						y(x)= e^{-\frac{ x^2 }{2}+C}= e^{-x^2/2} e^{C}=K e^{-x^2/2}
					\end{equation}
					où nous avons posé $K=e^C$.
			\end{enumerate}
		\item
			Si $y(x)=ax+b$, alors $y'(x)=a$. En mettant cela dans l'équation \eqref{nonhomog}, nous trouvons
			\begin{equation}
				a=-x(ax+b)+x,
			\end{equation}
			ou encore
			\begin{equation}	\label{EqPouraetbézuzc}
				-ax^2+(1-b)x-a=0.
			\end{equation}
			Attention : cette équation n'est pas une équation pour trouver $x$, mais une équation pour $a$ et $b$ qui doit être vérifiée pour tout $x$. Lorsque $x=0$, cette équation devient $a=0$. Nous devons donc avoir $a=0$. L'équation \eqref{EqPouraetbézuzc} devient alors
			\begin{equation}
				(1-b)x=0
			\end{equation}
			qui doit encore être vraie pour tout $x$. Avec $x=1$, nous avons la condition $b-1=0$ et donc $b=1$. Les valeurs de $a$ et $b$ telles que $P(x)=ax+b$ soit solution sont donc $a=0$ et $b=1$. C'est donc $P(x)=1$.
		\item
			Nous avons
			\begin{subequations}
				\begin{align}
					y(x)=1-3 e^{-x^2/2}
					y'(x)=-3(-x) e^{-x^2/2}=3x e^{-x^2/2}.
				\end{align}
			\end{subequations}
			En remplaçant ces valeurs de $y$ et $y'$ cela dans l'équation $y'=-xy+x$, nous avons
			\begin{equation}
				3x e^{-x^2/2}=-x\big( 1-3 e^{-x^2/2} \big)+x.
			\end{equation}
			En distribuant le $x$ dans le membre de droite, nous trouvons que cette équation est bien vérifiée pour tout $x$.

			En remplaçant $x$ par $\sqrt{\ln(3)}$ dans la formule de $y(x)$, nous avons
			\begin{equation}
				\begin{aligned}[]
					y\big( \sqrt{\ln(3)} \big)&=1-3 e^{-\big( \sqrt{\ln(2)} \big)^2/2}\\
					&=1-3 e^{-\ln(3)/2}\\
					&=1-3\sqrt{ e^{-\ln(3)}}
				\end{aligned}
			\end{equation}
			où nous avons utilisé le fait que $ e^{a/2}=\sqrt{ e^{a}}$. Maintenant en appliquant la formule $ e^{-a}=\frac{1}{  e^{a} }$ avec $a=\ln(3)$, nous trouvons
			\begin{equation}
				y\big( \sqrt{\ln(3)} \big)=1-3\sqrt{\frac{ 1 }{ 3 }}=1-\sqrt{\frac{ 9 }{ 3 }}=1-\sqrt{3}.
			\end{equation}
	\end{enumerate}

\end{corrige}
