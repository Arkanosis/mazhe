% This is part of Exercices et corrections de MAT1151
% Copyright (C) 2010
%   Laurent Claessens
% See the file LICENCE.txt for copying conditions.

\begin{corrige}{SerieQuatre0004}

	Calculons la quantité de nombres réels positifs représentables, et multiplions la par deux. Le nombre zéro ne pose pas de problèmes parce qu'il n'est de toutes façons pas représentable.

	Pour un $e$ donné, les nombres représentables sont les sommes possibles
	\begin{equation}
		\sum_{j=1}^ta_jb^{-j},
	\end{equation}
	c'est à dire toutes les combinaisons valides des $a_j$. Le coefficient $a_1$ peut prendre les valeurs entre $1$ et $b-1$, tandis que les autres vont de $0$ à $b-1$. Ici, on a $b=4$ et $t=2$, c'est à dire les combinaisons $10$, $11$, $12$, $13$, $20$, $21$, $22$, $23$, $30$, $31$, $32$, $33$ y a donc $12$ possibilités.
	
	En ce qui concerne $e$, il y a les possibilités de $L=-2$ à $U=2$, c'est à dire $5$ possibilités. En tout nous pouvons donc écrire $12\times 5=60$ nombres positifs, et donc $120$ nombres en tout.

\end{corrige}
