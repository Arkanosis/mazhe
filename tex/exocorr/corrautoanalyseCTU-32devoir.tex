% This is part of Analyse Starter CTU
% Copyright (c) 2014
%   Laurent Claessens,Carlotta Donadello
% See the file fdl-1.3.txt for copying conditions.

\begin{corrige}{autoanalyseCTU-32devoir}

\begin{enumerate}
\item[(3)]
  \begin{enumerate}
  \item En intégrant des deux c\^otés de l'équation nous obtenons  $\displaystyle \frac{y^3}{3}=\frac{x^3}{3}+C$, donc la forme explicite de la la solution générale de cette équation est $\displaystyle y=\left(x^3+C\right)^{1/3}$.
  \item La solution particulière qui satisfait la condition $\phi_1(0)=0$ est $\phi_1(x)=x$ car la constante $C$ doit \^etre nulle et $\left(x^3\right)^{1/3}=x$.
La solution particulière qui satisfait la condition $\phi_2(0)=1$ est $\phi_2(x)=\left(x^3+1\right)^{1/3}$, celle qui satisfait la condition $\phi_3(0)=-1$ est $\phi_3(x)=\left(x^3-1\right)^{1/3}$. 
  \end{enumerate}
\end{enumerate}


\end{corrige}   
