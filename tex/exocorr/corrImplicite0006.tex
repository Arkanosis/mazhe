% This is part of Exercices et corrigés de CdI-1
% Copyright (c) 2011,2014
%   Laurent Claessens
% See the file fdl-1.3.txt for copying conditions.

\begin{corrige}{Implicite0006}

	La définition de la fonction $y(x)$ est que 
	\begin{equation}		\label{EqNNDerImpliSix}
		y^2+\sin(xy)-1=0
	\end{equation}
	pour tout $x$.
	
	Nous vérifions que $y(0)=1$ résous bien cette équation et que le théorème de la fonction implicite s'applique dans un voisinage de $x=0$ et $y=1$.
	
	Nous dérivons cette équation par rapport à $x$ (qui apparaît dans le $y$) :
	\begin{equation}		\label{EqDerrSixImpli}
		y'\big( 2y+x\cos(xy) \big)=-y\cos(xy).
	\end{equation}
	Pour savoir le coefficient directeur de la tangente au point $(0,1)$, nous devons poser $x=0$ et $y=1$ dans cette équation. En tenant compte du fait que $\cos(0)=1$, nous trouvons
	\begin{equation}
		y'(0)=\frac{1}{ 2 }.
	\end{equation}
	La droite recherchée est celle qui passe par le point $(0,1)$ et qui a $-1/2$ comme coefficient directeur\footnote{Si vous ne savez pas comment en déduire une équation paramétrique ou cartésienne de la droite, demandez à madame Aude.}.

	Trouver une tangente horizontale est un petit peu plus subtil. Si nous posons $y'(x)=0$ dans la relation \eqref{EqDerrSixImpli}, nous trouvons l'équation suivante pour $x$~:
	\begin{equation}		\label{EqRechercheTgHorSix}
		y(x)\cos\big( xy(x) \big)=0.
	\end{equation}
	La première chose que nous voulons faire est de trouver un $x$ tel que $y(x)=0$. Hélas, la relation de définition \eqref{EqNNDerImpliSix} donne $0=-1$ lorsqu'on pose $y=0$. Cela montre que la fonction $y$ ainsi définie ne passe jamais par zéro.

	La seconde chose à faire pour annuler le membre de gauche de \eqref{EqRechercheTgHorSix} est d'essayer de trouver $x$ tel que $xy(x)=\frac{ \pi }{ 2 }$. Cela demanderais $y=\pi/2x$. Encore une fois, cet essai échoue si on remplace dans l'équation \eqref{EqNNDerImpliSix}.

	Le troisième essai est de chercher un $x$ tel que $xy(x)=-\pi/2$. Cette fois, en remplaçant dans \eqref{EqNNDerImpliSix}, ça fonctionne. Nous trouvons
	\begin{equation}
		\frac{ \pi^2 }{ 4x^2 }=2, 
	\end{equation}
	ce qui fait $x=\pm\frac{ \pi }{ 2\sqrt{2} }$.

	Nous ne sommes, cependant, pas sorti de l'auberge pour autant. En effet, nous avons prouvé que la fonction $y(x)$ existait dans un voisinage de $x=0$ et $y=1$. Rien ne prouve que ce voisinage va jusqu'à $\pm\pi/2\sqrt{2}$. Nous devons donc rappliquer le théorème de la fonction implicite pour prouver que la relation \eqref{EqNNDerImpliSix} définit bien une fonction $y(x)$ dans un voisinage de $(\tilde x,\tilde y)=(\pi/2\sqrt{2},-\sqrt{2})$. Il est d'abord vrai que $F(\tilde x,\tilde y)=0$, et ensuite,
	\begin{equation}
		\frac{ \partial F }{ \partial y }(\tilde x,\tilde y)=2\tilde y+\tilde x\cos(\tilde x\tilde y)=-2\sqrt{2}\neq 0,
	\end{equation}
	donc la fonction $y$ est bien définie au voisinage du point que nous avons sélectionné.
\end{corrige}
