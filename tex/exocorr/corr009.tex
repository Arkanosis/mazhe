\begin{corrige}{009}


When $C=\eR$, a vector field is just a smooth function $\eR\to\eR$ (be sure that you deeply understand this assertion !). If $g_X$ is the function of the vector field $X$ and $f_Y$ the one of $Y$, the answer is given by 
\begin{equation}
  f(x)=\frac{ g_Y(x) }{ g_X(x) }
\end{equation}
which is smooth as quotient of smooth functions.

We turn our attention to the case $C=S^1$. The formula, for each $x\in C$
\[ 
  Y_x=f(x)X_x
\]
completely defines $f$ because $X$ doesn't vanishes. The problem is to prove that it is smooth (extensions and all that). The natural way to begin is to say that $X$ and $Y$ admit smooth extensions $\hat X$ and $\hat Y$ in terms of which, we should be able to build an extension $\hat f$. The problem is that you have no guarantee that $\hat X$ and $\hat Y$ are parallel outside $C$. 

One can build normal vectors to the curve and translate $X$ and $Y$ along these lines. Then one gets smooth parallel extensions. Such a construction was given in the course during the proof of the lemma just below the inverse function theorem. From there, one defines the extension of $f$ by
\[ 
  \hat Y_z=\hat f(z)\hat X_z
\]
for all $z$ in a neighbourhood of $C$. But how to prove that it is smooth ? Maybe it is possible. I don't know, but this formula seems to define a smooth function since $f$ is a quantity which relates two smooth quantities each other.


A possible way to solve the exercise is to use results of exercise \ref{exo006}. Let $\dpt{ \xi_X }{ S^1 }{ \Cyl }$ be the map associted to $X$, and $\xi_Y$ the one associated with $Y$. They can be written under the form
\begin{subequations}
\begin{align}
   \xi_X(x)&=(.,.,h_X(x))\\
   \xi_Y(x)&=(.,.,h_Y(x))
\end{align}
\end{subequations}
where the dots represent some non essential functions. The functions $h_X$ and $h_Y$ are smooth and we define
\[ 
  \hat f(z)=\frac{ \hat h_Y(z) }{ \hat h_X(z) }
\]
which is smooth and a correct extension of $f$.

Now a general curve $C$ is diffeomorphic of $\eR$ or $S^1$. We denote by $\dpt{ \alpha }{ C }{ E }$ the diffeomorphism where $E$ denotes $\eR$ or $S^1$. The map $\alpha$ allows us to transform a complicated problem on $C$ into a solved problem on $E$. We consider $\tilde X=d\alpha X$ and $\tilde Y=d\alpha Y$, two vector fields on $E$. Then there exists $\dpt{ \tilde f }{ E }{ \eR }$ such that 

\begin{equation} \label{eq_tildeffx}
\tilde Y_x=\tilde f(x)\tilde X_x
\end{equation}
for each $x\in E$. 

Let us summarize the functions that we have at hand:
\[
\begin{split}
	\alpha\colon C&\to E\\
	\tilde f\colon E&\to \eR
\end{split}
\]
and we are searching for 
\[ 
  \dpt{ f }{ C }{ \eR }. 
\]
Obvioulsly the candidate is $f=\tilde f\circ\alpha$ which is smooth because it is a composition of smooth functions. In order to check that it is the right function, we have to prove that, for all $x\in C$, $Y_x=(\tilde f\circ\alpha)(x)X_x$.

The defining property of $\tilde f$ is 
\[ 
  d\alpha_x Y_x=\tilde f(\alpha(x))d\alpha_xX_x.
\]
Since $\alpha$ is diffeomorphic, $d\alpha_x$ is a linear injective map for each $x$; so we can ``simplify'' both sides by $d\alpha_x$.


\end{corrige}
