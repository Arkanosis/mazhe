% This is part of the Exercices et corrigés de mathématique générale.
% Copyright (C) 2009
%   Laurent Claessens
% See the file fdl-1.3.txt for copying conditions.
\begin{corrige}{Lineraire0014}

	On commence par calculer les produits à part :
	\begin{equation}
		\begin{aligned}[]
			5(-3\sqrt{2},\frac{1}{ 10 },2)&=(-14\sqrt{2},\frac{ 2 }{2},10)\\
			7(\sqrt{2},0,2)&=(7\sqrt{2},0,14),
		\end{aligned}
	\end{equation}
	et puis on fait la somme
	\begin{equation}
		(-14\sqrt{2},\frac{ 2 }{2},10)-(2,-1,\frac{ 3 }{ 5 })+=(7\sqrt{2},0,14)=(-8\sqrt{2}-2,\frac{ 3 }{ 2 },\frac{ 117 }{ 5 }).
	\end{equation}

\end{corrige}
