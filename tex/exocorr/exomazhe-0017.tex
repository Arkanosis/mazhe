% This is part of Analyse Starter CTU
% Copyright (c) 2014,2016
%   Laurent Claessens,Carlotta Donadello
% See the file fdl-1.3.txt for copying conditions.


\begin{exercice}\label{exomazhe-0017}

Soit $f$ la fonction $\displaystyle f(x) = \frac{e^x - 2}{e^x+1}$. 
\begin{enumerate}
\item Préciser l'ensemble de définition, les variations et les limites aux bords du domaine de $f$.
\item Montrer que pour tout $x$ dans l'ensemble de définition de $f$ on a 
  \begin{equation} \label{EQooCHQSooPtlbVZ}
    f'(x) = -\frac{1}{3}\left(f^2(x) + f(x)-2\right).
  \end{equation}
\item Montrer que $f$ réalise une bijection de son domaine vers un intervalle que l'on précisera.
\item Soit $g$ la bijection réciproque de $f$. Quel est l'intervalle de définition de $g$ ? 
\item Rappeler la formule qui donne la dérivée de la fonction réciproque.
\item Utiliser l'équation \eqref{EQooCHQSooPtlbVZ} pour montrer que la dérivée de $g$ est $g'(y)=-\frac{3}{y^2+y-2}$.
\item Déterminer l'expression explicite de $g$. 
\end{enumerate}


\corrref{mazhe-0017}
\end{exercice}
