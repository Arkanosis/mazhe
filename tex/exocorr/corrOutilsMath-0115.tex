% This is part of Outils mathématiques
% Copyright (c) 2011
%   Laurent Claessens
% See the file fdl-1.3.txt for copying conditions.

\begin{corrige}{OutilsMath-0115}

    En ce qui concerne le domaine, pour chaque $x$, la coordonnée $y$ varie de $\sin(x)+1$ à $\sin(x)+2$. Le domaine est donc celui de la figure \ref{LabelFigExSinLarge}.

    \newcommand{\CaptionFigExSinLarge}{Le domaine d'intégration}
    \input{pictures_tex/Fig_ExSinLarge.pstricks}

    La formule d'intégrale à utiliser est
    \begin{equation}
        \int_Df\,dS=\int_0^{\pi}\int_1^2 \tilde f(x,h)\| T_x\times T_h \|\,dh\,dx
    \end{equation}
    où $T_x=(1,\cos(x))$ et $T_h=(0,1)$. Nous avons donc $T_x\times T_h=e_z$ et par conséquent $\| T_x\times T_h \|=1$. 

    En ce qui concerne la fonction, nous devons exprimer $f(x,y)=y\cos(x)$ en termes de $x$ et $y$. Le changement de variables $(x,y)\to(x,h)$ est donné par
    \begin{subequations}
        \begin{numcases}{}
            x=x\\
            y=\sin(x)+h,
        \end{numcases}
    \end{subequations}
    par conséquent
    \begin{equation}
        \tilde f(x,h)=\big( \sin(x)+h \big)\cos(x).
    \end{equation}
    Nous restons donc avec l'intégrale 
    \begin{equation}
        \int_Df\,dS=\int_0^{\pi}\int_1^2\big( \sin(x)+h \big)\cos(x)\,dh\,dx
    \end{equation}
    à calculer. Cette intégrale est nulle parce que la fonction de $x$ à intégrer est antisymétrique par rapport à $\pi/2$. Si on ne voit pas cela, il y a moyen de calculer l'intégrale en suivant les mêmes pas que ceux décrits dans la correction de l'exercice \ref{exoOutilsMath-0098}.

\end{corrige}
