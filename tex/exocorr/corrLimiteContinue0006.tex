\begin{corrige}{LimiteContinue0006}

  \begin{enumerate}
  \item La fonction à considérer est le rapport de deux fonction continues $f_1=\ln (x+e^y)$ et  $f_2=\sqrt{x^2+y^2}$. Les deux fonctions sont bornées en $(1,0)$, et $f_2$ ne s'annule pas au voisinage de $(1,0)$. La limite donc existe et est le rapport entre les deux limites : $\ln 2 $.
    \item Passez aux coordonnés polaires. La limite n'existe pas. 

      \item 
		En passant aux coordonnés polaires, nous avons
		\begin{equation}
			\frac{ (x^2+y^2)^2 }{ x^2-y^2 }=\frac{ r^4 }{ r^2\big( \cos^2(\theta)-\sin^2(\theta) \big) }=\frac{ r^2 }{ \cos^2(\theta)-\sin^2(\theta) }=\frac{ r^2 }{ \cos(2\theta) }.
		\end{equation}
		Certes, \emph{pour chaque $\theta$}, cette fraction tend vers zéro lorsque $r$ tend vers zéro. Mais faire la limite avec $\theta$ constant revient à faire la limite le long d'une droite. Cependant, si en même temps que faire $r\to 0$, nous prenons un angle qui tend vers $45$ degrés, nous faisons tendre $\cos(2\theta)$ vers zéro. Cela reviendrait à suivre une spirale.

		Nous voyons que la fonction n'a pas de limite en suivant le chemin $r(t)=t$ et $\theta(t)$ tel que $\cos\big( 2\theta(t) \big)=t^3$. Est-ce qu'un tel chemin existe ? Oui parce que si nous prenons $\theta(t)=\frac{ \arccos(t^3) }{2}$, nous avons bien $\cos\big( 2\theta(t) \big)=t^3$. Ce chemin est tracé à la figure \ref{LabelFigSpiraleLimite}.
\newcommand{\CaptionFigSpiraleLimite}{Un chemin possible le long duquel calculer une limite. $r(t)=t$, $\theta(t)=\arccos(t^3)/2$.}
\input{auto/pictures_tex/Fig_SpiraleLimite.pstricks}

		
		Pour rappel, nous sommes intéressés par $t\to 0$, donc le fait que $\arccos(t^3)$ n'existe pas pour $| t |>1$ ne nous embête pas.

        \item Dans ce point et le suivant il faut utiliser les développements asymptotiques (de Taylor) pour le calcul des limites.
          
          La fonction $\cos(xy)$  admet le développement suivant : 
          \[
          \cos xy= 1-(xy)^2+o((xy)^3).
          \]
          Nous écrivons alors la limite sous la forme
          \begin{equation}
            \lim_{(x,y)\to (0,0)} \frac{1-\cos(xy)}{y^2}=\lim_{(x,y)\to (0,0)} \frac{(xy)^2}{y^2}=\lim_{(x,y)\to (0,0)} x^2=0.
          \end{equation}
       \item
		Il existe des fonctions $a\in o(x^2)$, $b\in o(y^4)$ et $c\in o(x^4)$ telles que
		\begin{subequations}
			\begin{align}
				\sin(x)&=x+a(x)\\
				\cos(y)&=1-y^2+b(y)\\
				\cosh(x)&=1+x^2+c(x).
			\end{align}
		\end{subequations}
		Pourquoi développer le numérateur à l'ordre $1$ et le dénominateur à l'ordre $2$ ? Parce que le dénominateur est une différence. Nous développons jusqu'à l'ordre qu'il faut pour avoir quelque chose de non nul. En l'occurrence dans $\cos(y)-\cosh(x)$, à l'ordre zéro nous avons $1-1=0$ tandis qu'à l'ordre $2$ nous avons $-(x^2+y^2)$ qui est non nul. Nous écrivons donc la fonction donnée sous la forme
		\begin{equation}
			f(x,y)=-\frac{ x+a(x) }{ x^2+y^2+b(y)+c(x) }.
		\end{equation}
		À ce moment, nous comprenons que la limite ne va pas exister parce que le degré du dénominateur est plus grand que celui du numérateur. Pour le montrer rigoureusement, prenons le chemin $\gamma(t)=(t,t)$ :
		\begin{equation}
			f(t,t)=-\frac{ t+a(t) }{ 2t^2+b(t)+c(t) }=-\frac{ 1+\frac{ a(t) }{ t } }{ 2t+\frac{ b(t) }{ t }+\frac{ c(t) }{ t } }.
		\end{equation}
		Lorsque nous faisons la limite $t\to 0$, nous avons, par construction, 
		\begin{subequations}
			\begin{align}
				\lim_{t\to 0}\frac{ a(t) }{ t }=\lim_{t\to 0} \frac{ b(t) }{ t }=\lim_{t\to 0} \frac{ c(t) }{ t }=0,
			\end{align}
		\end{subequations}
		et par conséquent la limite $\lim_{t\to 0} f(t,t)=\frac{1}{ 0 }$ n'existe pas.
          
  \end{enumerate}


\end{corrige}
