% This is part of Exercices de mathématique pour SVT
% Copyright (c) 2010
%   Laurent Claessens et Carlotta Donadello
% See the file fdl-1.3.txt for copying conditions.

\begin{corrige}{TD4-0001}

	\begin{enumerate}
		\item
			\newcommand{\CaptionFigGrapheunsurunmoinsx}{Le graphe de la fonction $x\mapsto\frac{1}{ 1-x }$.}
			\input{auto/pictures_tex/Fig_Grapheunsurunmoinsx.pstricks}
			La fonction $f(x)=\frac{1}{ 1-x }$ est représentée sur la figure \ref{LabelFigGrapheunsurunmoinsx}. Elle a deux intervalles de monotone (dé)croissance : $\mathopen] -\infty , 1 \mathclose[$ et $\mathopen] 1 , \infty \mathclose[$. Elle va donc être bijective sur ces deux intervalles.

			Sur l'intervalle $\mathopen] -\infty , 1 \mathclose[$, la fonction $f$ prend toutes les valeurs de $\mathopen] 0 , \infty \mathclose[$, tandis que sur l'intervalle $\mathopen] 1 , \infty \mathclose[$, elle prend les valeurs $\mathopen] -\infty , 0 \mathclose[$; en formules,
			\begin{subequations}
				\begin{align}
					f\big( \mathopen] -\infty , 1 \mathclose[ \big)&=\mathopen] 0 , \infty \mathclose[\\
					f\big( \mathopen] 1 , \infty \mathclose[ \big)&=\mathopen] -\infty , 0 \mathclose[.
				\end{align}
			\end{subequations}

			Trouver l'application inverse revient à trouver pour quel $x$ nous avons $f(y)=y$. L'équation à résoudre est
			\begin{equation}
				y=\frac{1}{ 1-x },
			\end{equation}
			à résoudre par rapport à $x$ pour un $y$ donné. La solution est
			\begin{equation}
				x=\frac{ y-1 }{ y }.
			\end{equation}
			Donc $f^{-1}(y)=\frac{ y-1 }{ y }$.

			Une autre façon de voir cela est d'écrire la définition de la fonction réciproque :
			\begin{equation}
				f\big( f^{-1}(y) \big)=y,
			\end{equation}
			et donc
			\begin{equation}
				\frac{1}{ 1-f^{-1}(y) }=y,
			\end{equation}
			et résoudre cette équation par rapport à $f^{-1}(y)$ en fonction de $y$.

		\item
			Cette fois, le graphe est plus difficile à tracer, donc nous allons calculer la dérivée pour savoir sur quel(s) intervalle(s) la fonction est monotone. Nous avons
			\begin{subequations}
				\begin{align}
					f(x)&=x^2-2x+1\\
					f'(x)&=2x-2.
				\end{align}
			\end{subequations}
			La dérivée est donc positive pour $x>1$ et négative sur $x<1$. Elle va donc être une bijection sur $\mathopen] -\infty , 1 \mathclose]$ et sur $\mathopen[ 1 , \infty [$. Il reste à voir sur quelle est l'image de ces intervalles par $f$. Étant donné que $f(1)=0$, nous avons
			\begin{subequations}
				\begin{align}
					f\big( \mathopen[ 1 , \infty [ \big)&=\mathopen[ 0 , \infty [\\
					f\big( \mathopen] -\infty , 1 \mathclose] \big)&=\mathopen[ 0 , \infty [.
				\end{align}
			\end{subequations}
			
			Pour information, nous avons tracé la fonction à la figure \ref{LabelFigExoCUd}.
			\newcommand{\CaptionFigExoCUd}{La fonction de l'exerice \ref{exoTD4-0001}.\ref{ItemexoTD1ii}. Remarquer la symétrie autour du sommet, comme toute fonction du second degré. Pour un $y$ donné, il y a \emph{deux} $x$ sur lesquels fa fonction vaut $y$.}
\input{auto/pictures_tex/Fig_ExoCUd.pstricks}

			Cette fois pour trouver la fonction inverse, il faut résoudre l'équation $f(x)=y$, c'est à dire
			\begin{equation}
				\begin{aligned}[]
					x^2-2x+1=y\\
					x^2-2x+(1-y)=0.
				\end{aligned}
			\end{equation}
			Cette équation est à résoudre par rapport à $x$ en fonction de $y$. La solution est 
			\begin{equation}
				x=1\pm\sqrt{y}.
			\end{equation}
			La question est de savoir s'il faut choisir le signe plus ou le signe moins.
			
			Si nous choisissons le signe moins, nous obtenons $x\leq 1$, et donc nous tombons dans l'intervalle de bijection de gauche, tandis qu'avec le signe plus, nous avons $x\geq 1$, et donc nous sommes dans l'intervalle de droite.

			La fonction inverse de 
			\begin{equation}
				\begin{aligned}
					f\colon \mathopen[ 1 , \infty [&\to \mathopen[ 0 , \infty [ \\
					x&\mapsto x^2-2x+1 
				\end{aligned}
			\end{equation}
			est $f^{-1}(y)=1-\sqrt{y}$, tandis que la fonction inverse de
			\begin{equation}
				\begin{aligned}
					f\colon \mathopen] -\infty , 1 ]&\to \mathopen[ 0 , \infty [ \\
					x&\mapsto x^2-2x+1 
				\end{aligned}
			\end{equation}
			est $f^{-1}(y)=1+\sqrt{y}$.
	\end{enumerate}

\end{corrige}
