% This is part of the Exercices et corrigés de mathématique générale.
% Copyright (C) 2009
%   Laurent Claessens
% See the file fdl-1.3.txt for copying conditions.
\begin{corrige}{Lineraire0026}

	Il faut essayer de trouver des constantes $a,b,c\in\eR$ telles que la fonction 
	\begin{equation}
		b e^{x}+b\sin(x)+c\cos(x)
	\end{equation}
	soit nulle, c'est à dire telle que cette somme soit zéro pour tout $x$. Cela n'est pas possible (et donc la partie est libre). En effet, comme $ e^{x}\to\infty$, de toutes façons, il faut $a=0$. Maintenant, en $x=0$, $\sin(x)=0$ et $\cos(x)=1$, nous devons donc avoir $c=0$. De là, $b=0$.

\end{corrige}
