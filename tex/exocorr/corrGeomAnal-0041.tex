\begin{corrige}{GeomAnal-0041}

    \begin{enumerate}
        \item
            Par définition de l'action d'une matrice sur un vecteur, nous avons
            \begin{equation}
                b_i=\sum_ka_{ik}x_k.
            \end{equation}
            En prenant la valeur absolue, nous trouvons
            \begin{equation}
                | b_i |\leq\sum_k| a_{ik} | |x_k |\leq\sum_k| a_{ik} |
            \end{equation}
            parce que \( | x_k |\leq 1\).

            Pour chaque \( x\in\eR^n\) tel que \( \| x \|_{\infty}=1\) nous avons
            \begin{equation}
                \| Ax \|_{\infty}=\max_{1\leq i\leq n}| b_i |\leq \max_{1\leq i\leq n}\sum_k| a_{ik} |.
            \end{equation}
            Donc en ce qui concerne la norme de \( A\), nous avons
            \begin{equation}
                \| A \|_{\infty}=\sup_{\| x \|_{\infty}=1}\| Ax \|_{\infty},
            \end{equation}
            mais tous les éléments de l'ensemble sur lequel le supremum est pris sont plus petit que \( \max\sum_k| a_{ik} |\).

        \item
            Nous passons à présent à l'inégalité inverse.
            \begin{enumerate}
                \item
                    Le vecteur \( x^*\) est fait exprès pour mettre un signe moins lorsque \( a_{i^*j}\) est négatif et un signe plus là où \( a_{i^*j}\) est positif. L'effet est donc de rendre \( a_{i^*j}x^*_j\) toujours positif.
                \item
                    Par définitions,
                    \begin{equation}
                        \| Ax^* \|_{\infty}=\max_{i\in\{ 1,\ldots,n \}}| (Ax^*)_i |=\max_i\left| \sum_k a_{ik}x^*_k \right| .
                    \end{equation}
                    Nous pouvons majorer en remplaçant \( i\) par \( i^*\) en effet, le membre de droite prend un maximum sur tous les \( i\) possibles alors que \( i^*\) est un \( i\) possible. Nous avons alors
                    \begin{equation}
                        \| Ax^* \|_{\infty}\geq\left| \sum_ka_{i^*k}x^*_k \right| =\left| \sum_k| a_{i^*k} | \right| =\sum_k| a_{i^*k} |.
                    \end{equation}
                    Maintenant la définition de \( i^*\) nous permet de dire
                    \begin{equation}
                        \| Ax^* \|_{\infty}\geq \max_{i\in\{ 1,\ldots,n \}}\sum_k| a_{ik} |.
                    \end{equation}
                    Par conséquent
                    \begin{equation}
                        \| A \|_{\infty}\geq\max_{i\in\{ 1,\ldots,n \}}\sum_k| a_{ik} |.
                    \end{equation}
            \end{enumerate}
    \end{enumerate}
    Ayant obtenu les inégalités dans les deux sens, les quantités sont égales.

\end{corrige}
