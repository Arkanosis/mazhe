% This is part of the Exercices et corrigés de mathématique générale.
% Copyright (C) 2009
%   Laurent Claessens
% See the file fdl-1.3.txt for copying conditions.
\begin{exercice}\label{exoLineraire0042}

	Exercice 19, page 93. On considère la transformation linéaire $A$ de $\eR^3$ dont la matrice dans la base canonique est
	\begin{equation}
		\begin{pmatrix}
			1	&	1	&	-2\\ 
			2	&	1	&	-2\\ 
			3	&	1	&	-4	  
		\end{pmatrix}.
	\end{equation}
	\begin{enumerate}

		\item
			Chercher les valeurs propres et les vecteurs propres de $A$.
		\item
			Cette matrice est-elle diagonalisable ? Si oui, donner la matrice d'un changement de base qui permet de la diagonaliser.
		\item
			On donne le vecteur $v_1=(1,1,2)$. Calculer $v_2=Av_1$ et $v_3=A^2v_1$.
		\item
			Montrer que les vecteurs $v_1$, $v_2$ et $v_3$ forment une base  de $\eR^3$.
		\item
			Donner les composantes du vecteur propre de $A$ de norme $2\sqrt{6}$ correspondant à la plus grande valeur propre de $A$ dans la base $\{ v_1,v_2,v_3 \}$.
	\end{enumerate}
	
\corrref{Lineraire0042}
\end{exercice}
