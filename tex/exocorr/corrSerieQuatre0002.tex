% This is part of Exercices et corrections de MAT1151
% Copyright (C) 2010,2016
%   Laurent Claessens
% See the file LICENCE.txt for copying conditions.

\begin{corrige}{SerieQuatre0002}

	Ici, comme dans l'exercice \ref{exoSerieQuatre0001}, le $s=0$ signifie qu'on parle de nombre positifs. La formule utilisée pour les nombres à virgule flottante est :
	\begin{equation}        \label{EqRepreFlotNOrm}
		x=(-1)^sb^e\sum_{j=1}^ta_jb^{-j}.
	\end{equation}
    Cela pourrait ne pas être la même que celle donnée dans la définition \ref{DEFooLYONooBNskty}.
	
	\begin{enumerate}

		\item
			Nous écrivons le nombre en suivant la formule \eqref{EqRepreFlotNOrm}:
			\begin{equation}
				10^2\cdot\big( 3\cdot 10^{-1}+4\cdot 10^{-2} \big)=10^2\cdot 0.34=34.
			\end{equation}

		\item
			Le premier $1$ est le $a_1$, et le denier est $a_5$. La formule \eqref{EqRepreFlotNOrm} devient donc
			\begin{equation}
				2^6\cdot(1\cdot 2^{-1}+1\cdot 2^{-5})=2^6\big( \frac{1}{ 2 }+\frac{1}{ 2^5 } \big)=2^5+2=34.
			\end{equation}	
	\end{enumerate}

\end{corrige}
