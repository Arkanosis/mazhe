% This is part of Outils mathématiques
% Copyright (c) 2011
%   Laurent Claessens
% See the file fdl-1.3.txt for copying conditions.

\begin{exercice}\label{exoOutilsMath-0122}

    Soient \( a>0\), \( b>0\), \( c>0\) et soit l'ellipsoïde 
    \begin{equation}
        E=\left\{ (x,y,z);\frac{ x^2 }{ a^2 }+\frac{ y^2 }{ b^2 }+\frac{ z^2 }{ c^2 }\leq 1 \right\}.
    \end{equation}
    \begin{enumerate}
        \item
            On considère l'application
            \begin{equation}
                T\colon \begin{pmatrix}
                    u    \\ 
                    v    \\ 
                    w    
                \end{pmatrix}\in\eR^3\to \begin{pmatrix}
                    x    \\ 
                    y    \\ 
                    z    
                \end{pmatrix}=\begin{pmatrix}
                    au    \\ 
                    bv    \\ 
                    cw    
                \end{pmatrix}\in\eR^3.
            \end{equation}
            Monter que lorsque \( (u,v,w)\) décrit la boule unité de \( \eR^3\), alors \( T(u,v,w)\) décrit l'ellipsoïde \( E\).
        \item
            Calculer le déterminant de la matrice jacobienne de \( T\). Calculer le volume de l'ellipsoïde \( E\).
    \end{enumerate}
    

\corrref{OutilsMath-0122}
\end{exercice}
