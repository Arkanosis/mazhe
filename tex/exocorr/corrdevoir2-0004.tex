\begin{corrige}{devoir2-0004}

Il y a deux façons de résoudre cet exercice. La première consiste à trouver d'abord une formule explicite pour $F$, l'utiliser pour calculer le gradient $\nabla F(x,y)$ et enfin évaluer cette fonction au point $(\pi, 3)$. La deuxième consiste à appliquer ce qu'on a étudié à la section 4.3.3 du poly sur la différentielle des fonctions composées. Il nous faut alors calculer les matrices  $\nabla g(f(\pi,3))$ et $\nabla f(\pi, 3)$ et les multiplier entre elles \textbf{dans cet ordre}.

Le résultat est 
\begin{equation}
 \nabla F(\pi, 3) =
 \begin{pmatrix}
   0 & 2\cos(3)\sin(3)\\
   -2(\pi+3) & -2(\pi+3)
 \end{pmatrix}.
\end{equation}
  
\end{corrige}
