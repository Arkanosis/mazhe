% This is part of Exercices de mathématique pour SVT
% Copyright (c) 2010
%   Laurent Claessens et Carlotta Donadello
% See the file fdl-1.3.txt for copying conditions.

\begin{corrige}{TD3-0014}

  \begin{enumerate}
  \item La suite $(u_n)_{n\in\eN}$ est définie par une formule de récurrence du type $u_{n+1}=g(u_n)$, donc si sa limite est un nombre ce nombre sera un point fixe de la fonction qui appara\^{i}t dans la  définition de la suite. 

La fonction qu'on doit considérer est alors 
\[
g(x)= a\frac{x^2}{x^2+b^2}-Ex.
\] 
Les points fixes de $g$ satisfont l'équation
\[
\ell = a\frac{\ell^2}{\ell^2+b^2}-E\ell, 
\]
qui est équivalente à 
\[
(E+1)\ell (\ell^2+b^2)-a \ell^2 =0,
\]
et encore 
\[
\ell \left((E+1)(\ell^2+b^2)-a \ell\right)=0.
\]
Le produit de deux facteurs est zéro si et seulement si au moins un entre les deux facteurs est nul. Les solutions sont  alors $0$, $\ell_1$ et $\ell_2$, où $\ell_1 $ et $\ell_2$ sont les racines du polynôme $(E+1)(\ell^2+b^2)-a \ell$.
\[
\ell_1 =\frac{a+\sqrt{a^2-4(E+1)^2b^2}}{2(E+1)}, \qquad\qquad\ell_2 =\frac{a-\sqrt{a^2-4(E+1)^2b^2}}{2(E+1)}.
\]
\item       \label{Itemtzziqtrois}
    Supposons maintenant que $E=0$ et que $a^2>4b^2$. On aura alors 
\[
\ell_1 =\frac{a+\sqrt{a^2-4b^2}}{2}, \qquad\qquad\ell_2 =\frac{a-\sqrt{a^2-4b^2}}{2}.
\]
 Notez que la deuxième de ces hypothèses veut dire simplement que les racines $\ell_1$ et $\ell_2$ sont bien deux nombres réels distincts. En fait, il est facile de voir que $0<\ell_2<\ell_1$.  

On a  
\[
\frac{au_n}{u_n^2+b^2}< 1 \quad\textrm{si et seulement si } au_n< u_n^2+b^2, 
\]
ce qui peut s'écrire aussi comme
\[
u_n^2-au_n+b^2>0.
\]
Cette dernière condition est satisfaite pour $u_n< \ell_2$ ou $u_n> \ell_1$. Il nous faut alors démontrer que si le premier terme de la suite $u_0$ est entre $[0,\ell_2]$ touts les termes suivantes sont contenus dans le même intervalle.

Notons d'abord que si $0<u_0<\ell_2$ alors $\displaystyle 0<\frac{a u_0}{u_0^2+b^2}<1$ et par conséquent $0<u_1<u_0$. Supposons maintenant que les termes $u_n$ soient dans l'intervalle $[0,\ell_2]$ pour  tous les indices $n$ entre $0$ et $k$ alors on aura 
\[
0<u_{k+1}=\frac{au_k}{u_k^2+b^2} u_k< u_k<\ell_2.
\]

\noindent\textbf{Méthode alternative :} la condition que $t\mapsto  \frac{a t^2}{t^2+b^2}$ est croissante implique que la suite $(u_n)_{n\in\eN}$ est monotone. Autrement dit, cela implique que soit  
\[
\frac{au_n}{u_n^2+b^2}< 1 \quad\textrm{pour tout } n, 
\]
soit
\[
\frac{au_n}{u_n^2+b^2}> 1 \quad\textrm{pour tout } n, 
\]
Il nous faut simplement vérifier laquelle entre les deux inégalités est vérifiée lorsque $n=0$. Notez que l'égalité est vérifié si et seulement si $u_n=\ell_1$ ou $u_n=\ell_2$. Dans ce deux cas la suite devient constante. 

\item
    Les candidats limites de la suite se trouvent en résolvant l'équation
    \begin{equation}
        x=\frac{ ax^2 }{ x^2+b^2 }.
    \end{equation}
    La première solution est \( x=0\). Pour trouver les autres solutions, nous simplifions l'équation par \( x\) pour trouver \( ax=x^2+b^2\) ou encore \( x^2-ax+b^2\) dont les solutions sont
    \begin{equation}
        \frac{ a\pm\sqrt{a^2-4b^2} }{ 2 }.
    \end{equation}
    Montrons que le seul candidat possible est zéro. D'abord par la question \ref{Itemtzziqtrois} nous savons que
    \begin{equation}
        \frac{ u_{n+1} }{ u_n }=\frac{ a u_{n} }{ u_n^2+b^2 }<1.
    \end{equation}
    La suite est donc décroissante. D'autre part à part zéro, le plus petit candidat limite est 
    \begin{equation}
        \frac{ a-\sqrt{a^2-4b^2} }{ 2 },
    \end{equation}
    mais par hypothèse, nous avons
    \begin{equation}
        0<x<\frac{ a-\sqrt{a^2-4b^2} }{ 2 },
    \end{equation}
    ce qui signifie que la suite commence déjà en dessous de ce candidat limite. Une suite décroissante ne peut pas converger vers une limite plus grande que son premier terme. Donc le seul candidat limite est zéro.

\item Dans les nouvelles conditions $a>2$, $b=1$ et $E> (a-2)/2$ nous avons
  \begin{equation}
    	u_{n+1}=a\frac{ u_n^2 }{ u_n^2+1 }-Eu_n= \left(a\frac{ u_n }{ u_n^2+1 }-E\right)u_n.
  \end{equation}
  \begin{description}
      \item[Décroissante] 
Il nous suffit de prouver que pour tout indice $n$ dans $\eN$ la quantité 
\[
a\frac{ u_n }{ u_n^2+1 }-E
\]
est inférieure à $1$. L'hypothèse sur $E$ nous permet d'écrire
\[
  a\frac{ u_n }{ u_n^2+1 }-E<  a\frac{ u_n }{ u_n^2+1 }-\frac{(a-2)}{2}.
\]
En mettant au même dénominateur, nous avons
\begin{equation}
    a\frac{ u_n }{ u_n^2+1 }-E<\frac{ 2au_n-au_n^2-a+2u_n^2+2 }{ 2(u_n^2+1) }=-a\frac{ (u_n-1)^2 }{ 2(u_n^2+1) }+1<1.
\end{equation}

        \item[bornée]
            Nous montrons maintenant que la suite est bornée inférieurement par zéro.

            PROBLÈME : avec $E=1000$, \( a=3\), \( u_0=1\), c'est un contre exemple !!

  \end{description}

  \end{enumerate}

\end{corrige}
