% This is part of (almost) Everything I know in mathematics
% Copyright (c) 2011,2015
%   Laurent Claessens
% See the file fdl-1.3.txt for copying conditions.

\begin{corrige}{mazhe-0001}

    Une paramétrisation du cylindre autour de l'axe $z$ est
    \begin{equation}
        \phi(\theta,z)=\begin{pmatrix}
            R\cos\theta    \\ 
            R\sin\theta    \\ 
            z    
        \end{pmatrix}.
    \end{equation}
    Les vecteurs tangents sont
    \begin{equation}
        \begin{aligned}[]
            T_{\theta}&=\begin{pmatrix}
                -R\sin\theta    \\ 
                R\cos\theta    \\ 
                0    
            \end{pmatrix},
            T_z&=\begin{pmatrix}
                0    \\ 
                0    \\ 
                1    
            \end{pmatrix}.
        \end{aligned}
    \end{equation}
    Le vecteur normal est donc
    \begin{equation}
        T_{\theta}\times T_z=R\cos(\theta)e_x+R\sin(\theta)e_y.
    \end{equation}
    C'est un vecteur dirigé vers l'extérieur.

    Le champ de vecteurs considéré est constant : $F(\theta,z)=e_x$. Nous avons donc
    \begin{equation}
        F(\theta,z)\cdot(T_{\theta}\times T_z)=R\cos(\theta)
    \end{equation}
    et le flux vaut
    \begin{equation}
        \Phi=\int_0^{2\pi}d\theta\int_0^hR\cos(\theta)dz=0.
    \end{equation}
    
    En ce qui concerne les couvercles haut au bas, ils sont paramétrés par
    \begin{equation}
        \begin{aligned}[]
            \phi_1(r,\theta)&=\begin{pmatrix}
                R\cos(\theta)    \\ 
                R\sin(\theta)    \\ 
                h    
            \end{pmatrix},
            \phi_2(r,\theta)&=\begin{pmatrix}
                R\cos(\theta)    \\ 
                R\sin(\theta)    \\ 
                0    
            \end{pmatrix}.
        \end{aligned}
    \end{equation}
    Les vecteurs normaux correspondants sont dans la direction de $e_z$, de façon que le produit scalaire avec $F(r,\theta)$ soit nul. Le flux total est donc nul.

    Regardons maintenant le cylindre le long de l'axe $x$. Une paramétrisation est
    \begin{equation}
        \phi(\theta,x)=\begin{pmatrix}
            x    \\ 
            R\cos(\theta)    \\ 
            R\sin(\theta)    
        \end{pmatrix},
    \end{equation}
    et le vecteurs tangents sont
    \begin{equation}
        \begin{aligned}[]
            T_{\theta}&=\begin{pmatrix}
                0    \\ 
                -R\sin\theta    \\ 
                R\cos\theta    
            \end{pmatrix},
            T_x&=\begin{pmatrix}
                1    \\ 
                0    \\ 
                0    
            \end{pmatrix}.
        \end{aligned}
    \end{equation}
    Le vecteur normal est alors donné par
    \begin{equation}
        T_{\theta}\times T_x=R\cos(\theta)e_y+R\sin(\theta)e_z.
    \end{equation}
    Nous avons par conséquent $F(\theta,x)\cdot (T_{\theta}\times T_x)=0$. Pas de flux par le côté du cylindre.

    Regardons les «couvercles». Le premier est donné par la paramétrisation
    \begin{equation}
        \phi_1(r,\theta)=\begin{pmatrix}
            0    \\ 
            r\cos(\theta)    \\ 
            r\sin(\theta)    
        \end{pmatrix}.
    \end{equation}
    Le vecteur normal serait $T_r\times T_{\theta}=re_x$, et le flux
    \begin{equation}
        \Phi=\int_0^{2\pi}d\theta\int_0^Rr\,dr=\pi R^2.
    \end{equation}
    
    Le second couvercle est donné par
    \begin{equation}
        \phi_2(r,\theta)=\begin{pmatrix}
            h    \\ 
            r\cos(\theta)    \\ 
            r\sin(\theta)    
        \end{pmatrix}.
    \end{equation}
    Le vecteur normal est encore $re_x$, et le flux est à nouveau $\pi R^2$.

    Le flux total serait donc $2\pi R^2$.

    Cela n'est pas possible parce que tous les vecteurs qui «rentrent» d'un côté doivent «sortir» de l'autre côté. L'erreur est le le premier vecteur normal est un vecteur qui pointe vers l'intérieur du cylindre, tandis que le second pointe vers l'extérieur. Si nous choisissons, par convention, de prendre uniquement les vecteurs extérieurs, il faut changer le vecteur normal du premier couvercle en $-re_x$. Le premier flux vaudra donc
    \begin{equation}
        -\pi R^2,
    \end{equation}
    de telle sorte que le flux total sera nul.

\end{corrige}
