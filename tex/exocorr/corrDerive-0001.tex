% This is part of Outils mathématiques
% Copyright (c) 2011,2016
%   Laurent Claessens
% See the file fdl-1.3.txt for copying conditions.

\begin{corrige}{Derive-0001}

Le programme suivant par Sage résout l'exercice:
\lstinputlisting{tex/sage/corrDerive_0001.sage}

Le résultat est :
\VerbatimInput[tabsize=3]{tex/sage/corrDerive_0001.txt}

Notez
\begin{enumerate}
	\item
		la syntaxe \verb+a**b+ pour signifier $a^b$;
	\item
		la syntaxe \verb+abs(x)+ pour la valeur absolue;
	\item
		la fonction \verb+log+ dans Sage\footnote{Comme dans tous les systèmes sérieux.} est le logarithme \emph{népérien}, c'est à dire en base $e$. Le logarithme que vous utilisez pour définir le PH d'une solution acide, lui, est en base $10$ !
\end{enumerate}

Donnons quelques détails sur certains.
\begin{enumerate}
	\item
		Pour calculer la dérivée de la tangente, on écrit
		\begin{equation}
			\tan(x)=\frac{ \sin(x) }{ \cos(x) },
		\end{equation}
		et ensuite on utilise la formule du quotient :
		\begin{equation}
			\tan'(x)=\frac{ \cos(x)\cos(x)-\sin(x)\big( -\sin(x) \big) }{ \cos^2(x) }=\frac{1}{ \cos^2(x) }.
		\end{equation}
		Notez que cela est bien égal à $\tan^2(x)+1$. Parfois la même fonction peut s'écrire de plusieurs façons différentes.
	\item
		La fonction $2^x$ n'est pas à confondre avec $x^2$. Le truc pour trouver la dérivée de $2^x$ est de se souvenir que le logarithme est la fonction inverse de l'exponentielle. Donc
		\begin{equation}
			2^x= e^{\ln(2^x)}= e^{x\ln(2)}.
		\end{equation}
		Nous avons utilisé le fait que $\ln(a^b)=b\ln(a)$. Par conséquent,
		\begin{equation}
			(2^x)'=\ln(2) e^{x\ln(2)}=\ln(2)2^{x}.
		\end{equation}
		
		Cette astuce de faire $f(x)= e^{\ln (f(x)) }$ est très classique pour calculer la dérivée de fonctions dans lesquelles le $x$ apparaît dans une puissance.
	\item
		La fonction 
		\begin{equation}
			f(x)=\frac{1}{ | x^2-4 | }
		\end{equation}
		requiert une attention particulière à cause de la valeur absolue. La fonction valeur absolue est définie par
		\begin{equation}
			\abs(x)=\begin{cases}
				x	&	\text{si }x\geq 0\\
				-x	&	 \text{si }x<0.
			\end{cases}
		\end{equation}
		Donc nous avons
		\begin{equation}		\label{EqDerivDeValAbs}
			\abs'(x)=\begin{cases}
				1	&	\text{si }x>0\\
				-1	&	 \text{si }x<0.
			\end{cases}
		\end{equation}
		Notez que cette fonction n'a pas de dérivée en zéro parce que son graphe fait un angle ! La formule \eqref{EqDerivDeValAbs} peut être écrite sous la forme
		\begin{equation}
			\abs'(x)=\frac{ x }{ | x | }.
		\end{equation}
		C'est cette dernière formule qui est manifestement utilisée par Sage. La dérivée de la fonction, après simplification, est donc
		\begin{equation}
			f'(x)=\begin{cases}
				\frac{ -2x }{ (x^2+4)^2 }	&	\text{si }-2<x<2\\
				\frac{ 2x }{ (x^2+4)^2 }	&	 \text{sinon}.
			\end{cases}
		\end{equation}
		La fonction n'est pas dérivable aux point $x=2$ et $x=-2$.
\end{enumerate}

\end{corrige}
