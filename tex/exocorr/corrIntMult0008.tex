% This is part of Exercices et corrigés de CdI-1
% Copyright (c) 2011
%   Laurent Claessens
% See the file fdl-1.3.txt for copying conditions.

\begin{corrige}{IntMult0008}

	Nommons les paraboloïdes $P_1\equiv z=x^2+y^2$ et $P_2\equiv \frac{ 1 }{2}(x^2+y^2)+2$, et passons en coordonnées cylindriques :
	\begin{equation}
		\begin{aligned}[]
			P_1&\equiv z=r^2\\
			P_2&\equiv z=\frac{ 1 }{2}r^2+2.
		\end{aligned}
	\end{equation}
	La paraboloïde $P_2$ est plus haute que la paraboloïde $P_1$ jusqu'à $r=2$ (solution de l'équation $r^2=\frac{ 1 }{2}r^2+2$). Nous allons donc calculer le volume contenu entre les deux paraboloïdes en faisant la différence
	\begin{equation}
		\int_{r\leq 2}P_2-\int_{r\leq 2}P_1,
	\end{equation}
	c'est à dire
	\begin{equation}
		\int_0^{2\pi}d\theta\int_0^2dr\int_0^{(r^2/2)+2}dz\cdot r-\int_0^{2\pi}d\theta\int_0^2dr\int_0^{r^2}dz\cdot r=12\pi-8\pi=4\pi.
	\end{equation}
	
\end{corrige}
