% This is part of Exercices et corrections de MAT1151
% Copyright (C) 2010
%   Laurent Claessens
% See the file LICENCE.txt for copying conditions.

\begin{corrige}{SerieUn0003}

	Un exemple assez classique de fonction dont la dérivée n'est pas bornée sans pour autant que la fonction aie un comportement immoral\footnote{Penser à $x\mapsto x\sin(1/x)$.} est $x\mapsto\sqrt{x}$. Afin d'avoir une fonction définie sur $\eR$ tout entier, nous regardons la fonction
	\begin{equation}
		x(d)=\sqrt{|d|}.
	\end{equation}
	Si nous considérons maintenant $d_0=0$ et n'importe quel $\eta$, nous avons
	\begin{equation}
		\frac{ | x(d)-x(d_0) | }{ | d-d_0 | }=\frac{ \sqrt{d} }{ d }=\frac{1}{ \sqrt{d} }.
	\end{equation}
	Il n'est pas possible de trouver un $K$ qui majore ce rapport. Le problème est donc mal conditionné.

	Attention : dans ce calcul nous avons supposé $d>0$. Pensez à adapter au cas $d<0$.

\end{corrige}
