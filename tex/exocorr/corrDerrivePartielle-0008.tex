% This is part of the Exercices et corrigés de mathématique générale.
% Copyright (C) 2010
%   Laurent Claessens
% See the file fdl-1.3.txt for copying conditions.

\begin{corrige}{DerrivePartielle-0008}

	Juste pour rester général, nous allons noter $1000=A$. Ainsi nous verrons si le résultat dépend du nombre à partager. Si $x,$ $y$ et $z$ sont les nombres, nous avons
	\begin{equation}
		z=A-x-y
	\end{equation}
	et nous devons maximiser $f(x,y)=xyz=xy(A-x-y)$.

	Cela se fait avec les méthodes habituelles et la fonction

	\VerbatimInput[tabsize=3]{src_sage/exo105.sage}

	Notez la présence de la ligne
	\begin{verbatim}
	assume(A>0)
	\end{verbatim}
	qui dit à Sage que $A$ est un nombre positif (ici c'est 1000).

	La sortie est :

	\VerbatimInput[tabsize=3]{src_sage/exo105.txt}

	Il faut faire un peu le ménage dans toutes les solutions trouvées. Les points avec $x$ ou $y$ qui sont nuls sont à rejeter parce qu'ils ne vont certainement pas maximiser le produit $xyz$. Reste la solution $x=A/3$, $y=A/3$, qui est un maximum local.

	La solution du problème est donc $(x,y,z)=(\frac{ A }{ 3 }, \frac{ A }{ 3 },\frac{ A }{ 3 })$. Il faut couper en trois parts égales, et cela ne dépend pas du nombre choisi.

\end{corrige}
