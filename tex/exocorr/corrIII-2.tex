% This is part of the Exercices et corrigés de mathématique générale.
% Copyright (C) 2009
%   Laurent Claessens
% See the file fdl-1.3.txt for copying conditions.
\begin{corrige}{2}

Nous avons $p(t)=a 2^{bt}$, et nous mesurons
\begin{equation}
	\begin{aligned}[]
		p(2)=a 2^{2b}=640&&\text{et}&&p(3)=a 2^{3b}=5120.
	\end{aligned}
\end{equation}
Ce qu'il nous faut, c'est trouver les nombres $a$ et $b$ qui satisfont les équations 
\begin{subequations}
\begin{numcases}{}
	a 2^{2b}=640\\
	a 2^{3b}=5120.
\end{numcases}
\end{subequations}
Afin de résoudre ce système, la subtilité est de prendre le rapport des deux équations pour éliminer le $a$ :
\begin{equation}
	\frac{ p(2) }{ p(3) }=\frac{ a 2^{2b} }{ a 2^{3b} }=\frac{ 640 }{ 5120 }=\frac{ 1 }{ 8 }.
\end{equation}
Cela fait $2^{2b-3b}=1/8$, ou encore $2^{-b}=1/8$, ce qui revient à $2^b=8$. Nous trouvons $b=3$. L'équation $a 2^{2b}=64$ devient alors $a 2^{6}=640$, et donc $a=10$. L'équation pour $p(t)$ est maintenant
\begin{equation}
	p(t)=10\cdot 2^{3t},
\end{equation}
et en particulier,
\begin{equation}
	p(t+T)=2^{3T}p(t),
\end{equation}
et donc on résous l'équation $2^{3T}=2$ pour trouver le temps de doublement de population : $T=\frac{1}{ 3 }$.


\end{corrige}
