% This is part of the Exercices et corrigés de CdI-2.
% Copyright (C) 2008, 2009
%   Laurent Claessens
% See the file fdl-1.3.txt for copying conditions.


\begin{corrige}{_II-1-09}

Nous suivons la méthode expliquée en \ref{SubSecRicatti}. La première chose à faire est de voir si nous pouvons deviner des solutions particulières.

\begin{enumerate}

\item 
Ici, nous voyons tout de suite que $y_1(t)=\sin(t)$ est une solution. Nous posons donc
\begin{equation}
	y(t)=\sin(t)+\frac{1}{ u },
\end{equation}
et nous trouvons l'équation différentielle
\begin{equation}		\label{EqII109DiffPouru}
	u'+\sin(t)u=1.
\end{equation}
Note : en suivant les notations du rappel théorique, nous avons
\begin{equation}
	\begin{aligned}[]
		a(t)	&=1,\\
		b(t)	&=-\sin(t),\\
		c(t)	&=\cos(t).
	\end{aligned}
\end{equation}
L'équation \eqref{EqII109DiffPouru} est du type de \eqref{EqDiffExempleVarCst}. Dans ces notations, nous cherchons donc $K$ qui vérifie l'équation $K'(t)=g(t)/u_H(t)$, c'est à dire
\begin{equation}
	K'(t)=-\frac{ 1 }{  e^{\cos(t)} }.
\end{equation}
Cette primitive n'est absolument pas simple à calculer. La solution à notre équation différentielle est donc 
\begin{equation}
	z(t)=- e^{\cos(t)}\int_0^t e^{-\cos(u)}du.
\end{equation}
Pour la petite histoire, l'\href{http://integrals.wolfram.com/index.jsp}{intégrateur en ligne} de Wolfram \href{http://reference.wolfram.com/mathematica/tutorial/IntegralsThatCanAndCannotBeDone.html}{ne trouve pas} de forme pour cette intégrale.


\end{enumerate}

\end{corrige}
