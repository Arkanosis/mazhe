\begin{exercice}\label{exoCourbesSurfaces0007}

On considère une courbe plane $([\theta_1, \theta _2], \vec{\gamma})$ définie, en coordonnées polaires, par $\vec{\gamma} (\theta ) = \big( r( \theta) \cos \theta, r( \theta) \sin \theta \big) $, où $r$ est une fonction $C^1$ sur $[\theta_1, \theta _2]$. 

Quelques applications de la formule.

\begin{enumerate}
	\item
		Est-ce que $ \vec{\gamma}$ est rectifiable ? 
	\item
		Montrer que la longueur de cet arc est donné par
		\begin{equation}		\label{EqFormDemExotLpola}
			\ell( \vec{\gamma}) = \int_{\theta_1}^{\theta _2} \sqrt{ r^2 ( \theta ) + (r')^2 ( \theta ) }\, d \theta. 
		\end{equation}
		
	\item
		Calculer la longueur de la courbe d'équation (en coordonnées polaires) $ r( \theta ) = \cos ^2 \theta$, $ \theta \in [0, \pi]$. 

	\item
		Calculer la longueur de la cardioïde, définie en coordonnées polaires par $ r( \theta ) = 1 - \cos \theta$, où $ \theta \in [- \pi,  \pi]$.  
		
\end{enumerate}

\corrref{CourbesSurfaces0007}
\end{exercice}
