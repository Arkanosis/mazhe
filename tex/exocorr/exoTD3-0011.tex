% This is part of Exercices de mathématique pour SVT
% Copyright (C) 2010
%   Laurent Claessens et Carlotta Donadello
% See the file fdl-1.3.txt for copying conditions.

\begin{exercice}\label{exoTD3-0011}

	Les lapins de Fibonacci.

	On considère des couples de lapins tels que, chaque mois, chaque couple donne naissance à un nouveau couple qui devient lui-même productif dès l'âge de deux mois. Après $n$ mois, leur nombre $u_n$ est donné par la suite
	\begin{equation}
		\begin{cases}
			u_{n+2}=u_{n+1}+u_n	&	\forall n\in\eN\\
			u_0=0\\
			u_1=1.	&	 
		\end{cases}
	\end{equation}
	\begin{enumerate}
		\item
			Montrer que pour tout $n\in\eN$,
			\begin{equation}
				u_n=\frac{1}{ \sqrt{5} }\Big[ \left( \frac{ 1+\sqrt{5} }{2} \right)^n-\left( \frac{ 1-\sqrt{5} }{2} \right)^n \Big].
			\end{equation}
		\item
			Justifier, dans ce cas très idéalisé, le modèle ci-dessus. Expliquer pourquoi a-t-on $u_{n+2}=u_{n+1}+u_n$.
	\end{enumerate}

\corrref{TD3-0011}
\end{exercice}
