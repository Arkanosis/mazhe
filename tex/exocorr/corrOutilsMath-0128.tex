% This is part of Outils mathématiques
% Copyright (c) 2011
%   Laurent Claessens
% See the file fdl-1.3.txt for copying conditions.

\begin{corrige}{OutilsMath-0128}

    \begin{enumerate}
        \item
            La vérification se fait en dérivant.
        \item
            Le calcul de l'intégrale est
            \begin{equation}
                \begin{aligned}[]
                    I&=\int_0^1dx\int_{ e^{x}}^{ e^{2x}}x\ln(y)dy\\
                    &=\int_0^1(2x^2 e^{2x}-x e^{2x}-x^2 e^{x}+xe^x).
                \end{aligned}
            \end{equation}
            Calculons séparément les quatre primitives à savoir pour cette intégrale. Elles se font toutes par partie selon le modèle \( A=\int x e^{x}dx\). On pose \( u=x\), \( dv=e^x\), et nous avons
            \begin{equation}
                A=\int x e^{x}dx=xe^x-\int e^x=xe^x-e^x.
            \end{equation}
            Ensuite,
            \begin{equation}
                \int x^2e^x=x^2e^x-2\int xe^x=x^2e^x-2xe^x+2e^x.
            \end{equation}
            De la même façon,
            \begin{equation}
                \int xe^{2x}dx=e^{2x}\left( \frac{ x }{2}-\frac{1}{ 4 } \right)
            \end{equation}
            et
            \begin{equation}
                \int x^2 e^{2x}dx=\frac{ x^2 }{2} e^{2x}-\int 2x\frac{ 1 }{2} e^{2x}= e^{2x}\left( \frac{ x^2 }{2}-\frac{ x }{2}+\frac{1}{ 4 } \right).
            \end{equation}
            En remettant les bouts ensemble, nous trouvons
            \begin{equation}
                I=\frac{ e^2 }{ 4 }-e+\frac{ 9 }{ 4 }.
            \end{equation}

        \item
            Le dessin est à la figure \ref{LabelFigratrap}.
            \newcommand{\CaptionFigratrap}{Le dessin de l'exerice \ref{exoOutilsMath-0128}.}
            \input{auto/pictures_tex/Fig_ratrap.pstricks}
        \item
            L'intégrale à calculer (en n'oubliant pas le jacobien) est
            \begin{subequations}
                \begin{align}
                    \int_a^bdr\int_0^{2\pi}d\theta r\ln(r^2)&=2\pi\int_a^b\ln(r^2)rdr\\
                    &=2\pi\frac{ 1 }{2}\int_{a^2}^{b^2}\ln(u)du\\
                    &=\pi\left[ u\ln(u)-u \right]_{a^2}^{b^2}.
                \end{align}
            \end{subequations}

    \end{enumerate}

\end{corrige}
