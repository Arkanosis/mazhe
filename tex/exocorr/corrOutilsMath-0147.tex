% This is part of Outils mathématiques
% Copyright (c) 2012
%   Laurent Claessens
% See the file fdl-1.3.txt for copying conditions.

\begin{corrige}{OutilsMath-0147}

    Les vecteurs demandés sont
    \begin{subequations}
        \begin{align}
            \overrightarrow{AB}&=B-A=(2,3)-(1,1)=(1,2)\\
            \overrightarrow{AC}&=(5,2)-(1,1)=(4,1).
        \end{align}
    \end{subequations}
    L'aire du parallélogramme est donnée par la norme du produit vectoriel :
    \begin{equation}
        (1,2)\times (4,1)=\begin{vmatrix}
            e_x    &   e_y    &   e_z    \\
            1    &   2    &   0    \\
            4    &   1    &   0
        \end{vmatrix}=-7e_z.
    \end{equation}
    L'aire demandée vaut donc \( 7\).

\end{corrige}
