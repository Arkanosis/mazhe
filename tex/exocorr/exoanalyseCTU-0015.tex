% This is part of Analyse Starter CTU
% Copyright (c) 2014
%   Laurent Claessens,Carlotta Donadello
% See the file fdl-1.3.txt for copying conditions.

\begin{exercice}\label{exoanalyseCTU-0015}


Dans tout le problème on se place dans l'intervalle $I = ]0, +\infty[$. 

On considère l'équation différentielle suivante
\begin{equation}\label{nonhomC}
  xy' + y= x^2.
\end{equation}
\begin{enumerate}
\item De quel type est l'équation \eqref{nonhomC} ?
\item Vérifier que $y(x) = \frac{1}{3} x^2 + \frac{1}{x}$ est l'unique solution de \eqref{nonhomC} qui satisfait la condition initiale $y(1) = \frac{4}{3}$.
\item  Vérifier que $y(x) = \frac{1}{3} x^2$ est l'unique solution de \eqref{nonhomC} qui satisfait la condition initiale $y(1) = \frac{1}{3}$.
\item Quel rapport y a-t-il entre l'équation \eqref{nonhomC} et  l'équation différentielle 
  \begin{equation}\label{homC}
    xy' + y= 0 \quad ?
  \end{equation}
\item Trouver l'unique solution de \eqref{homC} qui satisfait la condition initiale $y(1) = 1$.   
\end{enumerate}


\corrref{analyseCTU-0015}
\end{exercice}
