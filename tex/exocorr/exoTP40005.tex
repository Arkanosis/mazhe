% This is part of the Exercices et corrigés de mathématique générale.
% Copyright (C) 2009
%   Laurent Claessens
% See the file fdl-1.3.txt for copying conditions.
\begin{exercice}\label{exoTP40005}

	Une équipe de biologistes et de géographes part en expédition vers une île lointaine. Leur navire devra parcourir $d$ kilomètres. Parmi les coûts du voyage, il y a ceux du carburant et ceux du personnel. Le coût par heure en carburant est directement proportionnel au carré de la vitesse, de la forme $kv^2$. Quant au coût pas heure en personnel, il est évidemment indépendant de la vitesse, soit $p$ le cout par heure en personnel.

	Calculer, en fonction des constantes $k$ et $p$, la vitesse $v$ en $\unit{}{\kilo\meter\per\hour}$ à laquelle il faut naviguer pour minimiser le coût total du voyage.

\corrref{TP40005}
\end{exercice}
