% This is part of Exercices et corrections de MAT1151
% Copyright (C) 2010
%   Laurent Claessens
% See the file LICENCE.txt for copying conditions.

\begin{corrige}{SerieUn0004}

	Proche de zéro, la dérivé de $\cos(d)$ est petite. Nous allons donc prendre $d_0$ proche de $0$ et un $\epsilon$ et choisir $\eta$ suffisamment petit pour que 
	\begin{equation}
		\frac{ | \cos(d)-\cos(d_0) | }{ | d-d_0 | }<\epsilon
	\end{equation}
	tant que $| d-d_0 |<\epsilon$. Dans ce cas, nous avons que $K^{\eta}_{abs}(d_0)<\epsilon$, et donc que 
	\begin{equation}
		K_{rel}^{\eta}(d)=\epsilon\frac{ d }{ \cos(d) }.
	\end{equation}
	Lorsque $d$ est proche de zéro, la fraction reste proche de zéro.

	Ce que nous concluons est que $K^{\eta}_{rel}(d)$ peut être rendu aussi petit que l'on veut (prendre $\epsilon$ petit) en choisissant $d$ dans un petit voisinage de $0$ et $\eta$ suffisamment petit pour ne pas déborder du voisinage.

\end{corrige}
