% This is part of Exercices et corrigés de CdI-1
% Copyright (c) 2011
%   Laurent Claessens
% See the file fdl-1.3.txt for copying conditions.

\begin{corrige}{IntMult0011}

	Le volume d'un domaine est l'intégrale de la fonction constante $1$ sur le domaine. Nous passons aux coordonnées cylindriques \og autour de $x$\fg
	\begin{equation}
		\left\{
		\begin{array}{ll}
			x=x\\
			y=r\cos(\theta)\\
			z=r\sin(\theta)
		\end{array}
		\right.,
	\end{equation}
	dont le jacobien vaut $r$. Nous avons donc à calculer :
	\begin{equation}
		V=\int_a^bdx\int_0^{2\pi}d\theta\int_0^{f(x)}rdr=2\pi\int_a^b\frac{1}{ 2 }f(x)^2=\pi\int_a^bf^2,
	\end{equation}
	formule que certains étudiants ont probablement déjà vue.

\end{corrige}
