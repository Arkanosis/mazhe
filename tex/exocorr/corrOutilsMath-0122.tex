% This is part of Outils mathématiques
% Copyright (c) 2011
%   Laurent Claessens
% See the file fdl-1.3.txt for copying conditions.

\begin{corrige}{OutilsMath-0122}

    \begin{enumerate}
        \item
            Prouvons que \( T\) est une bijection entre la boule unité et \( E\). Si \( (x,y,z)\in E\), alors l'unique élément de \( \eR^3\) dont l'image par \( T\) est \( (x,y,z)\) est le vecteur \( \begin{pmatrix}
                x/a    \\ 
                y/b    \\ 
                z/c    
            \end{pmatrix}\). En effet,
            \begin{equation}
                T\begin{pmatrix}
                    x/a    \\ 
                    y/b    \\ 
                    z/c    
                \end{pmatrix}=\begin{pmatrix}
                    x    \\ 
                    y    \\ 
                    z    
                \end{pmatrix}.
            \end{equation}
            De plus si \( (x,y,z)\in E\), on a 
            \begin{equation}
                \frac{ x^2 }{ a^2 }+\frac{ y^2 }{ b^2 }+\frac{ z^2 }{ c^2 }\leq 1,
            \end{equation}
            c'est à dire que le point \( \begin{pmatrix}
                x/a    \\ 
                y/b    \\ 
                z/c    
            \end{pmatrix}\) est dans la boule unité.

        \item
            La matrice jacobienne de \( T\) est la matrice
            \begin{equation}
                J=\begin{pmatrix}
                    \frac{ \partial x }{ \partial u }    &   \frac{ \partial x }{ \partial v }    &   \frac{ \partial x }{ \partial w }    \\
                    \frac{ \partial y }{ \partial u }    &   \frac{ \partial y }{ \partial v }    &   \frac{ \partial y }{ \partial w }    \\
                    \frac{ \partial z }{ \partial u }    &   \frac{ \partial z }{ \partial v }    &   \frac{ \partial z }{ \partial w }    
                \end{pmatrix}=\begin{pmatrix}
                    a    &   0    &   0    \\
                    0    &   b    &   0    \\
                    0    &   0    &   c
                \end{pmatrix}.
            \end{equation}
            Son déterminant vaut \( abc\).

            Afin de trouver le volume de \( E\), nous pouvons utiliser les sphériques modifiées de l'exercice \ref{exoOutilsMath-0113}. 

            Une façon plus rapide est d'utiliser la formule du changement de variable du théorème \ref{ThoChamDeVarIntDDfOM}. Par rapport aux notations de ce théorème, nous avons \( f(x,y,z)=1\). Nous avons donc
            \begin{equation}
                V=\int_E1\,dxdudz=\int_{T(Boule)}1\,dxdydz=\int_{Boule}| J_T(u,v,w) |dudvdw.
            \end{equation}
            Ici nous venons de voir que \( J_T(u,v,w)\) était la constante \( abc\) qui peut sortir de l'intégrale. Donc
            \begin{equation}
                V=abc\int_{Boule}1\,dudvdw=abc\frac{ 4\pi }{ 3 }.
            \end{equation}
            La dernière intégrale était simplement le volume de la sphère.

    \end{enumerate}
    

\end{corrige}
