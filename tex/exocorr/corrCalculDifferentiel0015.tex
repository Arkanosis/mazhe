\begin{corrige}{CalculDifferentiel0015}

	Les deux premières égalités sont respectivement la proposition \ref{PropDiffLineaire} et le lemme \ref{LemDiffProsuid}. En ce qui concerne la différentielle de la puissance $n$ de $f$, nous procédons par récurrence. Si la formule est valable pour $n$, alors
	\begin{equation}
		\begin{aligned}[]
			F(f^{n+1})(x_0)&=D(ff^n)(x_0)\\
			&=f(x_0)Df^n(x_0)+f^n(x_0)Df(x_0)\\
			&=f(x_0)nf^{n-1}(x_0)Df(x_0)+f^n(x_0)Df(x_0)\\
			&=nf^n(x_0)Df(x_0)+f^n(x_0)Df(x_0)\\
			&=(n+1)f^n(x_0)Df(x_0).
		\end{aligned}
	\end{equation}

\end{corrige}
