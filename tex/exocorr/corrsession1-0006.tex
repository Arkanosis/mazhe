% This is part of Analyse Starter CTU
% Copyright (c) 2014
%   Laurent Claessens,Carlotta Donadello
% See the file fdl-1.3.txt for copying conditions.

\begin{corrige}{session1-0006}

L'équation n'est pas linéaire mais \`a variables séparables. Si $y(x)$ est égale \`a $\pm 1$ pour un $x\ in \mathbb{R}$ alors $y' = 0$ et la solution est constante. Ce cas ne nous intéresse pas car ces solutions ne satisfont pas la condition initiale imposée. Supposons donc  que $y$ soit diffèrent de $\pm 1$. Nous pouvons  procéder comme il suit : d'abord  
\begin{align*}
  y'=2x(-1+y^2) \Rightarrow \frac{y'}{(-1+y^2)}  = 2x,
\end{align*}
ensuite, nous cherchons les primitives (par rapport \`a $x$) des deux cotes de l'équation
\begin{align*}
  &\int \frac{y'}{(-1+y^2)}\, dx = \int 2x \, dx , \\
  &\int \frac{1}{(y-1)(y+1)}\, dy = x^2 +C,  \qquad \text{pour } C\in\mathbb{R} \\
  &\frac{1}{2}\int \frac{1}{(y-1)}-\frac{1}{(y+1)}\, dy = x^2 +C, \\
 &\ln\left(\sqrt{\frac{|y-1|}{|y+1|}}\right) = x^2 +C ,\\
&\frac{y-1}{y+1} = Ke^{2x^2}, \qquad \text{pour } K \in\mathbb{R}\setminus\{0\} . 
\end{align*}
\`A partir d'ici on peut trouver une expression analytique explicite pour la solution générale de l'équation différentielle   
\begin{equation*}
 \mathcal{Y} = \left\{ y(x) = \frac{1+Ke^{2x^2}}{1-Ke^{2x^2}}, \qquad  K\in\mathbb{R}\right\}\cup\{y(x)=-1\}.
\end{equation*}
La valeur de $K$ qui correspond \`a la condition initiale est trouvée par substitution et vaut $K= -1/3$.


\end{corrige}
