% This is part of Exercices de mathématique pour SVT
% Copyright (C) 2010
%   Laurent Claessens et Carlotta Donadello
% See the file fdl-1.3.txt for copying conditions.

\begin{corrige}{TD3-0006}

	Nous notons $N$ la quantité totale de micro-organismes (c'est à dire $N=10000$), puis $X_n$ la quantité des micro-organismes phosphorescents au jour $n$, et $Y_n$ la quantité de ceux qui ne le sont pas, nous avons
	\begin{subequations}
		\begin{align}
			X_0&=\frac{N}{ 4 }\\
			Y_0&=\frac{ 3N }{ 4 },
		\end{align}
	\end{subequations}
	et la loi d'évolution est
	\begin{subequations}
		\begin{align}
			X_{n+1}=X_n-\frac{ 20 X_n }{ 100 }+\frac{ 5Y_n }{ 100 }\\
			Y_{n+1}=Y_n+\frac{ 20 X_n }{ 100 }-\frac{ 5Y_n }{ 100 }.
		\end{align}
	\end{subequations}
	En simplifiant les fractions,
	\begin{equation}		\label{EqunevolXnTn}
		X_{n+1}=\frac{ 4 }{ 5 }X_n+\frac{1}{ 20 }Y_n.
	\end{equation}
	Étant donné que la quantité totale de micro-organismes est constante et vaut $N$, nous avons toujours $X_n+Y_n=N$. Nous pouvons donc remplacer $Y_n$ par $(N-X_n)$ dans l'équation \eqref{EqunevolXnTn} pour trouver 
	\begin{equation}
		X_{n+1}=\frac{ 3 }{ 4 }X_n+\frac{ N }{ 20 }.
	\end{equation}
	Et nous sommes maintenant dans un modèle, pour $X_n$, du même type que celui de l'exercice \ref{exoTD3-0005} avec $a=\frac{ 3 }{ 4 }$, $b=\frac{ N }{ 20 }$ et $x=X_0=\frac{ N }{ 4 }$. L'évolution de $X_n$ est donc donnée par
	\begin{equation}		\label{EqrefXnnform}
		\begin{aligned}[]
			X_n&=\left( \frac{ 3 }{ 4 } \right)^n\left( X_0+\frac{ N/20 }{ \frac{ 3 }{ 4 }-1 } \right)-\frac{ N/20 }{ \frac{ 3 }{ 4 }-1 }\\
			&=\left( \frac{ 3 }{ 4 } \right)^n\left( \frac{ N }{ 4 }-4\frac{ N }{ 20 } \right)+4\frac{ N }{ 20 }\\
			&=\left( \frac{ 3 }{ 4 } \right)^n\frac{ N }{ 20 }+\frac{ N }{ 5 }
		\end{aligned}
	\end{equation}
	où nous avons utilisé le fait que $X_0=\frac{ N }{ 4 }$. À partir de cette expression, nous pouvons retrouver $Y_n$ simplement en calculant $X_n$ et en faisant $Y_n=N-X_n$.

	Pour trouver la population de type $X$ au bout de $1$ jour, il suffit de poser $n=1$ dans l'expression \eqref{EqrefXnnform}. Le résultat est $\frac{ 19 }{ 80 }N=2375$. Au bout de deux jours il en reste $X_2=2281.25$, $X_5=2118.65$ et $X_10=2028.16$.


\end{corrige}
