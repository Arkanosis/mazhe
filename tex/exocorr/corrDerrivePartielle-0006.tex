% This is part of the Exercices et corrigés de mathématique générale.
% Copyright (C) 2010
%   Laurent Claessens
% See the file fdl-1.3.txt for copying conditions.

\begin{corrige}{DerrivePartielle-0006}

	La première dérivée est standard :
	\begin{equation}
		w'(x)=\frac{ \partial f }{ \partial u }\big( x,g(x) \big)+\frac{ \partial f }{ \partial v }\Big( x,g(x) \Big)g'(x)
	\end{equation}
	où nous avons implicitement utilisé les faits que $\frac{ \partial x }{ \partial x }=1$ et $\frac{ \partial g }{ \partial x }=g'(x)$ parce que $g$ est une authentique fonction de une seule variable.

	Pour calculer $w''(x)$, nous devons calculer la dérivée de tout cela. Allons y petit bout par petit bout. D'abord
	\begin{equation}
		\left[ \frac{ \partial f }{ \partial u }\Big( x,g(x) \Big) \right]'=\frac{ \partial^2f  }{ \partial u^2 }\Big( x,g(x) \Big)+\frac{ \partial^2f }{ \partial v\partial u }\Big( x,g(x) \Big)g'(x).
	\end{equation}
	Ensuite, pour le second terme nous utilisons la règle de Leibnitz (càd $(uv)'=u'v+v'u$) :
	\begin{equation}
		\begin{aligned}[]
			\left[  \frac{ \partial f }{ \partial v }\Big( x,g(x) \Big)g'(x) \right]'&=\left[ \frac{ \partial f }{ \partial v }\Big( x,g(x) \Big) \right]'g'(x)\\
			&\quad +\frac{ \partial f }{ \partial v }\Big( x,g(x) \Big)g''(x).
		\end{aligned}
	\end{equation}
	Enfin,
	\begin{equation}
		\begin{aligned}[]
			\left[ \frac{ \partial f }{ \partial v }\Big( x,g(x) \Big) \right]'&=\frac{ \partial  }{ \partial u }\left( \frac{ \partial f }{ \partial v } \right)\Big( x,g(x) \Big)\\
			&\quad +\frac{ \partial  }{ \partial v }\left( \frac{ \partial f }{ \partial v } \right)\Big( x,g(x) \Big)g'(x)\\
			&=\frac{ \partial^2f  }{ \partial u\partial v }\Big( x,g(x) \Big)+\frac{ \partial^2f }{ \partial v^2 }\Big( x,g(x) \Big)g'(x).
		\end{aligned}
	\end{equation}
	La réponse est obtenue en recollant tous les morceaux.
	

\end{corrige}
