% This is part of Exercices de mathématique pour SVT
% Copyright (C) 2010
%   Laurent Claessens et Carlotta Donadello
% See the file fdl-1.3.txt for copying conditions.



\begin{exercice}\label{exoTD3-0001}

	Vrai ou faux ?
	\begin{enumerate}
		\item
			Une suite est toujours soit majorée soit minorée.
		\item
			Toute suite convergente est monotone.
		\item
			Toute suite convergente est bornée.
		\item
			Si une suite est monotone et bornée, alors elle converge.
		\item
			Si une suite converge, alors elle est monotone et bornée.
		\item
			Une suite positive qui converge vers zéro est décroissante à partir d'un certain rang.
		\item
			Si la suite $(| x_n |)_{n\in\eN}$ converge vers $\ell$, la suite $(x_n)_{n\in\eN}$ converge vers $\ell$ ou vers $-\ell$.
		\item
			Si la suite $(x_n)_{n\in\eN}$ converge vers $\ell$, la suite $(x_{2n+n^2})$ converge vers $\ell$.
		\item
			Si la suite $(x_n-y_n)_{n\in\eN}$ converge vers $0$, les suites $(x_n)_{n\in\eN}$ et $(y_n)_{n\in\eN}$ convergent vers la même limite.
		\item
			Le produit d'une suite qui converge vers $0$ et d'une suite quelconque converge vers $0$.
		\item
			Toute suite encadrée par deux suites convergentes est convergente.
		\item
			La différence de deux suites équivalentes converge vers $0$.
		\item
			Le quotient de deux suites équivalentes non nulle tend vers $1$.
	\end{enumerate}

\corrref{TD3-0001}
\end{exercice}
