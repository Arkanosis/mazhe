% This is part of Exercices et corrigés de CdI-1
% Copyright (c) 2011
%   Laurent Claessens
% See the file fdl-1.3.txt for copying conditions.

\begin{corrige}{0031}

Pas de corrections pour cet exercice. Ça fait appel à des séries et tout ça \ldots

\marginpar{\tiny La marge est trop étroite pour répondre à ces questions, mais c'est une chose à creuser.} Au delà de l'exercice de calcul qu'il représente, cet énoncé donne l'occasion de réfléchir au sens \sout{de la vie} de ce qu'on fait quand on démontre quelque chose. Le point crucial de cet exercice est \og à partir des définitions\fg{}. Mais quelle est la définition de $\cos(x)$ ? Quelle est la définition de \og un angle de $x$ radians\fg{} ? Quelle est la définition de la longueur d'un arc de courbe ou de l'aire d'une surface définie par une telle courbe ?

\end{corrige}
