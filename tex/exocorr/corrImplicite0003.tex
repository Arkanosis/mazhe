% This is part of Exercices et corrigés de CdI-1
% Copyright (c) 2011
%   Laurent Claessens
% See the file fdl-1.3.txt for copying conditions.

\begin{corrige}{Implicite0003}

	Nous avons $f\big( (x,y),z \big)=ze^z-x-y$. Évidemment, $z(0,0)=0$, et pour que le théorème de la fonction implicite fonctionne, il faut que
	\begin{equation}
		\frac{ \partial F }{ \partial z }\big( (0,0),0 \big)=(ze^z+e^z)_{x=y=z=0}\neq 0.
	\end{equation}
	Cette condition est vérifiée. Le développement de $z(x,y)$ à l'ordre $2$ autour de $(0,0)$ est donné par
	\begin{equation}
		z(0,0)+x\partial_xz(0,0)+y\partial_yz(0,0)+\frac{x^2}{ 2 }\partial^2_xz(0,0)+\frac{ y^2 }{ 2 }\partial^2_yz(0,0)+xy\partial^2_{xy}z(0,0).
	\end{equation}
	Tout l'exercice se réduit donc à calculer les dérivées partielles de $z(x,y)$ par rapport à $x$ et à $y$ jusqu'à l'ordre $2$, et de les évaluer en $(0,0)$.
	La relation de définition de $z(x,y)$ est
	\begin{equation}		\label{EqDefZxyImpII}
		z(x,y)e^{z(x,y)}-x-y=0.
	\end{equation}
	Nous en extrayons la valeur de $z(0,0)=0$. Pour le reste, nous dérivons la relation $F\big( x,y,z(x,y) \big)=0$, et nous évaluons en $(0,0)$. Par exemple pour $\partial_xz(0,0)$, nous commençons par écrire
	\begin{equation}
		(\partial_1F)(0,0,z(0,0))+(\partial_3F)(0,0,z(0,0,0))\frac{ \partial z }{ \partial x }(0,0)=0.
	\end{equation}
	Dans cette équation, $\partial_1F$ dénote la dérivée de $F$ par rapport à sa première variable. Celle-là est toujours égale à $-1$ parce que $F(x,y,z)=ze^z-x-y$. D'autre part,
	\begin{equation}
		(\partial_3F)(x,y,z)=ze^z+e^z,
	\end{equation}
	donc, en isolant $\partial_xz(0,0)$, nous avons
	\begin{equation}
		\frac{ \partial z }{ \partial x }(0,0)=\frac{1}{ e^{z(0,0)}(z(0,0)+1) }=1.
	\end{equation}
	Le même genre de raisonnements amène les autres dérivées. Pour la dérivée par rapport à $y$, nous écrivons
	\begin{equation}
		\frac{ \partial F }{ \partial x_1 } \underbrace{\frac{ \partial x_1 }{ \partial y }}_{=0}+\frac{ \partial F }{ \partial x_2 }\underbrace{\frac{ \partial x_2 }{ \partial y }}_{=-1}+\underbrace{\frac{ \partial F }{ \partial x_3 }}_{=ze^z+e^z}\frac{ \partial z }{ \partial y }=0.
	\end{equation}
	Les résultats sont
	\begin{equation}
		\begin{aligned}[]
			\frac{ \partial z }{ \partial x }(0,0)&=1\\
			\frac{ \partial z }{ \partial y }(0,0)&=1\\
			\frac{ \partial^2 z }{ \partial x^2 }(0,0)&=-2
		\end{aligned}
	\end{equation}
\end{corrige}
