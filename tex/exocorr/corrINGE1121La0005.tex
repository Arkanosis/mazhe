% This is part of the Exercices et corrigés de mathématique générale.
% Copyright (C) 2009-2010,2014
%   Laurent Claessens
% See the file fdl-1.3.txt for copying conditions.


\begin{corrige}{INGE1121La0005}

	Afin de vérifier que $W$ est un espace vectoriel, il faut faire trois choses
	\begin{enumerate}

		\item
			Prouver que $(0,0,0,0)\in W$. La vérification est immédiate: $0+2\cdot 0+0+3\cdot 0=0$.
		\item
			Si $x$ et $y$ sont dans $W$, il faut que $x+y$ soient dans $W$. Testons donc l'équation de définition de $W$ sur le vecteur
			\begin{equation}
				x+y=(x_1+y_1,x_2+y_2,x_3+y_3,x_4+y_4).
			\end{equation}
			Il faut vérifier que la combinaison suivante est nulle :
			\begin{equation}
				(x_1+y_1)+2(x_2+y_2)+(x_3+y_3)+3(x_4+y_4)
			\end{equation}
			En regroupant les termes en $x$ et en $y$, on obtient
			\begin{equation}
				\underbrace{x_1+2x_2+x_3+3x_4}_{x\in W\Rightarrow =0}+\underbrace{y_1+2y_2+y_3+3y_4}_{y\in W\Rightarrow =0}.
			\end{equation}
		\item
			Si $x\in W$, il faut que $\lambda x\in W$ pour tout $\lambda\in\eR$. La démonstration est similaire à la précédente. Il suffit d'écrire la condition pour le vecteur $\lambda x$ et de mettre $\lambda$ en évidence pour obtenir zéro.

	\end{enumerate}

	Les points du plan sont les points $(x_1,x_2,x_3,x_4)\in\eR^4$ qui vérifient l'équation
	\begin{equation}
		x_1+2x_2+x_3+3x_4=0.
	\end{equation}
	La matrice de cette équation est la toute bête matrice
	\begin{equation}
		\begin{pmatrix} 
			1	&	2	&	1	&	3	
		\end{pmatrix},
	\end{equation}
	dont le rang est un. La dimension de l'espace de solutions d'un système de rang un dans $\eR^4$ est $3$. L'espace $W$ est donc de dimension $3$.

	Supposons avoir trouvé une base orthogonale ${w_1,w_2,w_3}$ de $W$. Les vecteurs de $W^{\perp}$ perpendiculaires à $w_1$, $w_2$ et $w_3$. Ils sont donc linéairement indépendants de ceux de $W$ (par la proposition \ref{PropVectsOrthLibres}). Si $w'_1,\cdots,w'_l$ sont une base orthonormale de $W^{\perp}$, nous avons alors $3+l$ vecteurs linéairement indépendants dans $\eR^4$. Nous en déduisons que $l=1$. Donc $\dim(W^{\perp})=1$.

	Trouvons une base de $W$. Par définition les vecteurs de $W$ s'écrivent sous la forme
	\begin{equation}
		(-2x_2-x_3-3x_4,x_2,x_3,x_4)
	\end{equation}
	où $x_2$, $x_3$ et $x_4$ sont des paramètres. Pour trouver trois vecteurs de base de cet espace, posons successivement $x_2=1$, $x_3=1$ et $x_3=1$. Les vecteurs trouvés sont
	\begin{equation}		\label{EqLesTroisvi}
		\begin{aligned}[]
			v_1&=(-2,1,0,0)\\
			v_2&=(-1,0,1,0)\\
			v_3&=(-3,0,0,1).
		\end{aligned}
	\end{equation}
	Pour vérification que ces trois vecteurs sont bien linéairement indépendants, nous pouvons les mettre dans une matrice
	\begin{equation}
		\begin{pmatrix}
			 -2	&	-1	&	-3	\\
			 1	&	0	&	0	\\
			 0	&	1	&	0	\\ 
			 0	&	0	&	1	 
			  \end{pmatrix}
	\end{equation}
	et puis vérifier que le rang de cette matrice est bien $3$.

	Nous savons que $W^{\perp}$ est de dimension $1$. En trouver une base revient à trouver un vecteur perpendiculaire aux trois vecteurs $v_i$ donnés en \eqref{EqLesTroisvi}. Nous pouvons bien entendu chercher ce vecteur sous la forme $x=(x_1,x_2,x_3,x_4)$ et résoudre le système d'équation donné par
	\begin{subequations}
		\begin{numcases}{}
			\langle x, v_1\rangle =0\\
			\langle x, v_2\rangle =0\\
			\langle x, v_3\rangle =0
		\end{numcases}
	\end{subequations}
	Heureusement, il y a un «truc» pour trouver. Il suffit de prendre l'équation de $W$ et de prendre comme vecteur $x$ le vecteur dont les coordonnées sont les coefficients dans l'équation de $W$. Ici, l'équation de $W$ est 
	\begin{equation}
		x_1+2x_2+x_3+3x_4=0.
	\end{equation}
	Je prétends donc que le vecteur $(1,2,1,3)$ est perpendiculaire aux trois. La vérification est aisée.

\end{corrige}
