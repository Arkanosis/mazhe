\begin{exercice}\label{exoCourbesSurfaces0010}

Même exercice pour l'hyperbole d'équation paramétrique $\vec{h}(t) = (\sinh t, \cosh t)$, $ t \in \eR$ et le point $ A = (0, 1)$. 

Pour rappel, les fonctions $\sinh$ et $\cosh$ sont les \defe{sinus hyperbolique}{hyperbolique!sinus} et \defe{cosinus hyperbolique}{hyperbolique!cosinus} et sont définis par
\begin{equation}
	\begin{aligned}[]
		\sinh(x)=\frac{  e^{x}- e^{-x} }{ 2 }&&\text{et}&&\cosh(x)=\frac{  e^{x}+ e^{-x} }{ 2 }.
	\end{aligned}
\end{equation}
Une de leurs propriétés fondamentales est que pour tout $x$, nous avons $\cosh^2(x)-\sinh^2(x)=1$.

\corrref{CourbesSurfaces0010}
\end{exercice}
