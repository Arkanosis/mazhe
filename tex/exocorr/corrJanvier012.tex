% This is part of the Exercices et corrigés de mathématique générale.
% Copyright (C) 2009
%   Laurent Claessens
% See the file fdl-1.3.txt for copying conditions.
\begin{corrige}{Janvier012}

Soit $f : P \to \{0, 1, \ldots, n-1\}$ la fonction qui a chaque personne (les éléments
de $P$) associe son nombre d'amis présents à la soirée.

Supposons d'abord que tout le monde ait au moins un ami. Alors $f$ est
à valeurs dans $\{1, \ldots, n-1\}$ qui contient $n-1$ éléments. Or il
y a $n$ personnes dans $P$, donc $f$ n'est pas injective (principe des
tiroirs). Ceci montre qu'au moins deux personnes ont le même nombre
d'amis dans ce cas.

Si par contre un des convives n'a aucun ami présent à la soirée, alors
soit il n'est pas seul dans ce cas (auquel cas il y a bien deux
personnes qui ont le même nombre d'amis : zéro), soit il est seul dans
ce cas, et les $n-1$ personnes restantes ont toutes au moins un ami,
ce qui nous ramène au premier cas.


\end{corrige}
