\begin{corrige}{LimiteContinue0004}

	Pour calculer $\lim_{x\to 0} f(x,y)$, nous coupons la fonction en deux termes :
	\begin{equation}	\label{EqLC4fffssusxusy}
		f(x,y)=x\sin\frac{1}{ x }\sin\frac{1}{ y }+y\sin\frac{1}{ x }\sin\frac{1}{ y }.
	\end{equation}
	Ne pas oublier que $y$ est une simple constante différente de zéro; en particulier $\sin\frac{1}{ y }$ et $y\sin\frac{1}{ y }$ sont deux simples constantes.

	Le premier terme du membre de droite de \eqref{EqLC4fffssusxusy} a une limite pour $x\to 0$. En effet,
	\begin{equation}
		0\leq| x\sin\frac{1}{ x } |<| x |
	\end{equation}
	parce que $\sin\frac{1}{ x }<1$. Par conséquent, $\lim_{x\to 0} x\sin\frac{1}{ x }=0$. Le second terme de \eqref{EqLC4fffssusxusy} n'a par contre pas de limite lorsque $x\to 0$ parce que $\sin\frac{1}{ x }$ oscille sans fin entre $-1$ et $1$.

	Nous en déduisons que $\lim_{x\to 0} f(x,y)$ n'existe pas. 

	En ce qui concerne la limite simultanée,
	\begin{equation}
		0\leq| f(x,y) |=| x+y | |\sin\frac{1}{ x } | |\sin\frac{1}{ y } |\leq| x+y |\to 0.
	\end{equation}
	La fonction $| f(x,y) |$ est donc coincée entre $0$ et la fonction $| x+y |$. Évidement, chacune de ces deux fonctions tend vers zéro lorsque $(x,y)\to(0,0)$.

	Notez que pour montrer que les limites itérées n'existaient pas, ce qui a joué est que nous avions deux termes dont un avait une limite et l'autre pas. Si nous avions deux termes dont aucun n'avait une limite, nous n'aurions pas pu conclure.

\end{corrige}
