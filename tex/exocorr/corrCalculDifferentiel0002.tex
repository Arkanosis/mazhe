\begin{corrige}{CalculDifferentiel0002}

	En ce qui concerne les ensembles de définition, les règles les plus courantes sont
	\begin{enumerate}
		\item
			$\ln(x)$ demande $x>0$ (stricte);
		\item
			$1/x$ demande $x\neq 0$;
		\item
			$\sqrt{x}$ demande $x\geq 0$ (non stricte).
	\end{enumerate}
	C'est parti \ldots
	\begin{enumerate}
		\item
			Il n'y a pas de conditions d'existence. Les dérivées partielles sont
			\begin{subequations}
				\begin{align}
					\frac{ \partial f }{ \partial x }&=\frac{ y }{ 1+(xy)^2 }\\
					\frac{ \partial f }{ \partial y }&=\frac{ x }{ 1+(xy)^2 }.
				\end{align}
			\end{subequations}
		\item
			L'arc tangente est bien définie pour tous les réels. La seule condition est donc $x\neq 0$. Les dérivées partielles sont
			\begin{subequations}
				\begin{align}
					\frac{ \partial f }{ \partial x }&=\frac{ -y }{ x^2+y^2 }\\
					\frac{ \partial f }{ \partial y }&=\frac{ x }{ x^2+y^2 }.
				\end{align}
			\end{subequations}
			Pour rappel, $(\arctan(x))'=\frac{1}{ 1+x^2 }$.
		\item
			Nous avons les fractions $x/y$ et $z/y$. Toutes deux demandent $y\neq 0$. C'est la seule condition. Le domaine est donc tout $\eR^2$ moins la droite $y=0$. Les dérivées partielles sont
			\begin{subequations}
				\begin{align}
					\frac{ \partial f }{ \partial x }&=\frac{1}{ y } e^{x/y},\\
					\frac{ \partial f }{ \partial y }&=-\frac{ x }{ y^2 } e^{x/y}-\frac{ z }{ y^2 } e^{z/y},\\
					\frac{ \partial f }{ \partial z }&=\frac{1}{ y } e^{z/y}.
				\end{align}
			\end{subequations}
		\item
			Le domaine est $\eR^2$. Les dérivées sont
			\begin{equation}
				\begin{aligned}[]
					\frac{ \partial f }{ \partial x }&=2x\sin(y)\\
					\frac{ \partial f }{ \partial y }&=x^2\cos(y).
				\end{aligned}
			\end{equation}
			Cette fonction est différentiable en tant que produit de fonctions différentiables.
		\item
			Condition : $1-x^2-y^2\geq 0$, c'est à dire $x^2+y^2\leq 1$, c'est à dire la boule centrée en $(0,0)$ et de rayon $1$, y compris le bord. Les dérivées sont
			\begin{subequations}
				\begin{align}
					\frac{ \partial f }{ \partial x }&=\frac{ -x }{ \sqrt{1-x^2-y^2} }\\
					\frac{ \partial f }{ \partial y }&=\frac{ -y }{ \sqrt{1-x^2-y^2} }.
				\end{align}
			\end{subequations}
			Il faut vérifier la différentiabilité sur le bord du domaine, c'est à dire sur les points de la forme $(a,b)$ avec $a^2+b^2=1$. Étant donné que les dérivées partielles ne sont pas continues en ces points, la fonction n'est pas différentiable sur le bord.
		\item
			Condition : $x+y>0$. C'est le demi-plan strictement au dessus de la droite $y=-x$. $\partial_xf(x,y)=1/(x+y)$ et $\partial_yf(x,y)=1/(x+y)$.

			Le domaine de la fonction étant ouvert (ne contient pas la droite), ça n'a pas de sens de se demander si la fonction est différentiable sur les points de la droite $x=-y$. En chaque point du domaine, il existe un voisinage sur lequel les dérivées partielles sont continues, donc la fonction est différentiables (proposition \ref{Diff_totale}).
	\end{enumerate}

\end{corrige}
