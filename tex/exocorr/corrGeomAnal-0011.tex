\begin{corrige}{GeomAnal-0011}

  Démontrer que la fonction $f:\eR^2\to \eR$
  \begin{equation}
  f(x,y)=\left\{  \begin{array}{ll}
      (x^2+y^2)\sin\left(\frac{1}{x^2+y^2}\right)\quad &  \textrm{si } (x,y)\neq(0,0),\\
0, &  \textrm{si } (x,y)=(0,0)
    \end{array}\right.
  \end{equation}
est différentiable à tout point de $\eR^2$ (et par conséquence est continue partout), mais que ses dérivées partielles ne sont pas définies en $(0,0)$.

Cet exercice est assez difficile. Sur $\eR^2\setminus (0,0)$ on peut utiliser le la proposition \ref{Diff_totale}. Pour prouver la différentiabilité en l'origine il faut utiliser la définition.   

\end{corrige}
