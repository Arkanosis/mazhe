% This is part of Exercices et corrigés de CdI-1
% Copyright (c) 2011
%   Laurent Claessens
% See the file fdl-1.3.txt for copying conditions.

\begin{corrige}{0086}

D'abord, on voit que cette fonction est bien définie, puisque la
distance est toujours minorée par $0$ donc l'infimum existe toujours.

Pour la continuité, prenons $x \in X$ et $\epsilon > 0$. On veut trouver $\delta>0$ tel que
\[d(x,y) < \delta \donc \abs{\inf_{a \in A} d(a,x) - \inf_{a\in A} d(a,y)} < \epsilon\]
Pour tout
$a\in A$, on a successivement
\begin{align*}
d(x,a) \leq d(a,y) + d(x,y) \donc& \inf_{a\in A} d(x,a) \leq \inf_{a\in A} \big(d(a,y) + d(x,y)\big)\\
\donc& \inf_{a \in A} d(x,a) \leq \inf_{a\in A} d(a,y) + d(x,y)\\
\donc& \inf_{a \in A} d(x,a) - \inf_{a\in A} d(a,y) \leq d(x,y)
\end{align*}
et de même en échangeant les rôles de $x$ et $y$, ce qui conduit à
\[\inf_{a \in A} d(y,a) - \inf_{a\in A} d(a,x) \leq d(x,y)\]
donc on en déduit la majoration
\begin{equation}		\label{eqex13}
	\abs{\inf_{a \in A} d(y,a) - \inf_{a\in A} d(a,x)} \leq d(x,y)
\end{equation}
dès lors le choix $\delta = \epsilon$ conduit à l'inégalité désirée.

\end{corrige}
