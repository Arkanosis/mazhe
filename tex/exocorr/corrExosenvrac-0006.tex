\begin{corrige}{Exosenvrac-0006}
 
L'équation $x'=tx^2$ peut se résoudre par séparation de variables parce que le membre de droite de l'équation est le produit d'une fonction de $t$, à l'occurrence l'identité, et une fonction qui ne dépend explicitement que de $x$ : $x^2$. 

La seule solution constante de l'équation est $x(t)=0$. 

Si $x$ n'est pas nulle on peut écrire l'équation sous la forme 
\begin{equation}
  \frac{1}{x^2} x' = t,
\end{equation}
 et intégrer par rapport à $t$ les deux côtés. Suite au changement de variable classique $x=x(t)$ dans l'intégrale à gauche, on obtient 
 \begin{equation}
   \int \frac{1}{x^2} \,dx = \int t \, dt,
 \end{equation}
et donc
\begin{equation}
    -\frac{1}{x}= \frac{ t^2 }{2}+C, \qquad C\in\mathbb{R}.
\end{equation}
La solution générale de l'équation est alors $\displaystyle x(t)= -\frac{1}{\frac{ t^2 }{2}+C}$. 

La solution du problème de Cauchy
\begin{equation}
  \begin{cases}
    x'=tx^2,\\
    x(2)=1, 
  \end{cases}
\end{equation} 
n'est pas la solution constante. Il faut alors supposer qu'elle soit de la forme $\displaystyle x(t)= -\frac{1}{\frac{ t^2 }{2}+C}$ et calculer $\displaystyle x(2)= -\frac{1}{2+C}$. La condition $\displaystyle  -\frac{1}{2+C}=1$ nous permet de trouver la valeur de la constante $C$ qui détermine l'unique solution du problème de Cauchy. Ici : $C = -3$.
\end{corrige}
