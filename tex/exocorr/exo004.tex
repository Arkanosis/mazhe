\begin{exercice}\label{exo004}
Consider  the set
\[ 
  E=\left\{ 
\begin{pmatrix}
\cos\theta&\sin\theta\\
-\sin\theta&\cos\theta
 \end{pmatrix},
 \theta\in\eR\right\}.
\]


\begin{enumerate}
\item It this set an embedded curve in $\eR^4$ ?
\item Write a general tangent vector at $\theta=0$. Is it surprising ? \label{enum004ii}
\end{enumerate}

Now, fix a $v\in\eR^2$, and let $\dpt{ f_v }{ E }{ \eR^2 }$ be defined by
\[ 
  f_v(\theta)=\theta v
\]
with a slight abuse of notation between $\theta$ and the element of $E$ which is defined by $\theta$. You have to guess the product in the right hand side.

\begin{enumerate}
\item Give an explicit form of $(df_v)_{\mtu}$ where $\mtu$ is the $2\times 2$ unit matrix. 
\item Can you give a geometric interpretation as ``infinitesimal'' rotation ? Does it help to answer point \ref{enum004ii} ?
\end{enumerate}

\corrref{004}

For general culture, remark that $E$ is a group for the matrix product and that the anti-symmetric matrices form an algebra over $\eR$ for the product $A\cdot B:=AB-BA$ where the dot denotes the algebra product and the product in the right hand side is the usual matrix product. 

The elements of $E$ depend smoothly on one parameter, turning $E$ into what is called a \defe{Lie group}{} of dimension 1. The set of antisymmetric matrices (which form the tangent space to $E$ at the identity), endowed with the dot product, forms the \defe{Lie algebra}{} associated with the Lie group $E$. Note that in this case, the dot product always gives zero. This is common to all Lie algebras of dimension 1.

More generally, consider a curve $x(t)$ in $O(n)$  (the group of orthogonal matrices) such that $x(0)=\mtu$. For each $t$, one has $x(t)x(t)^T=\mtu$. Taking the derivative of this equation with respect to $t$ at $t=0$, Leibnitz rule yields  $\dot x(0)x(0)^T+x(0)\dot x(0)^T=0$, or, taking $x(0)=\mtu$ into account,
\[ 
  \dot x(0)+\dot x(0)^T=0.
\]
This proves that $\dot x(0)$ is an antisymmetric matrix. So the Lie algebra of the Lie group of orthogonal matrices is the algebra of antisymmetric matrices. When $n\geq 3$, the algebra product $AB-BA$ is no longer identically zero.

\end{exercice}
