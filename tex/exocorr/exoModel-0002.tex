% This is part of Agregation : modélisation
% Copyright (c) 2011
%   Laurent Claessens
% See the file fdl-1.3.txt for copying conditions.

\begin{exercice}\label{exoModel-0002}

    Soit \( (X_1,\ldots,X_n)\), un échantillon de loi parente uniforme sur \( \mathopen[ 0 , \theta \mathclose]\) avec \( \theta>0\).

    \begin{enumerate}
        \item
            Montrer que la fonction de vraisemblance est donnée par
            \begin{equation}
                L(x_1,\ldots,x_n;\theta)=\frac{1}{ \theta^n }\mtu_{\mathopen[ 0 , \infty [}\big( \min(x_1,\ldots,x_n) \big)\mtu_{\mathopen[ \max(x_1,\ldots,x_n) , \infty [}(\theta).
            \end{equation}
            
        \item
            Déterminer l'estimateur du maximum de vraisemblance de \( \theta\).
    \end{enumerate}

\corrref{Model-0002}
\end{exercice}
