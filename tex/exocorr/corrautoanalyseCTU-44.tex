% This is part of Analyse Starter CTU
% Copyright (c) 2014
%   Laurent Claessens,Carlotta Donadello
% See the file fdl-1.3.txt for copying conditions.

\begin{corrige}{autoanalyseCTU-44}



On considère la fonction $f$ définie par $f(x)=\dfrac{\cos (x)}{1-x}$.
\begin{enumerate}
\item $\Dom_f = \eR\setminus\{1\}$.
\item Le développement de la fonction cosinus est connu
  \begin{equation*}
    \cos(x) = 1-\frac{x^2}{2} + \frac{x^4}{4!} + \ldots.
  \end{equation*}
Pour trouver le développement au voisinage de $0$ de la fonction $f$ nous utilisons la règle pour calculer le développement d'un rapport
\begin{equation*}
        \begin{array}[]{ccccccccccc|c}
            &1&&&-&\frac{x^2}{ 2 }&&&+&\frac{x^4}{ 24 }& &1-x \\
            \cline{12-12}
            -\Big( &1&-&x&\Big)&&&& && &1+x+\frac{x^2}{2}+\frac{x^3}{2}\\
            \cline{2-6}
            & &+ &x&-&\frac{x^2}{ 2 }&&& & & &  \\
            &&-\Big( +&x&-&x^2&\Big)&& & & \\
            \cline{3-7}
            & & & & +&\frac{ x^2}{ 2 }&&&+&\frac{x^4}{ 24 }& &  \\
            & & &  &-\Big(  &\frac{ x^2 }{ 2 }&-&\frac{x^3}{ 2 }&\Big)& && \\
            \cline{5-10}
            & & & & & &+ &\frac{x^3}{2} &+&\frac{x^4}{ 24 } && \\
            & & & & & &- \Big( +&\frac{x^3}{2} &-&\frac{x^4}{ 2 } && \\
            \cline{7-10}
            & & & & & &&&+&\frac{13x^4}{ 24 } && \\
        \end{array}
    \end{equation*}
Le développement limité  à l'ordre 3 en 0 de la fonction $f$ est $\displaystyle 1+x+\frac{x^2}{2}+\frac{x^3}{2} + x^3\alpha(x)$.
\begin{remark}
  Il était possible aussi de calculer ce développement comme le produit entre le développement de $\cos$ et de $\dfrac{1}{1-x}$ au voisinage de $0$. 
\end{remark}
\item Par la formule de Taylor-Young on a que $f'(0) = 1$, $f''(0) = 1$ et $f^{(3)}(0) = 3$.
\item L'équation de la tangente  $T$ à $(C)$ au point d'abscisse 0 est $y = 1+x$. L'équation de la tangente est toujours l'approximation d'ordre 1 de la fonction $f$ au voisinage de $0$.


\end{enumerate}

\end{corrige}   
