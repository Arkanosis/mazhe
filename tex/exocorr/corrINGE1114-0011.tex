% This is part of Un soupçon de physique, sans être agressif pour autant
% Copyright (C) 2006-2009
%   Laurent Claessens
% See the file fdl-1.3.txt for copying conditions.


\begin{corrige}{SerieUn0011}

	Nommons $B=\{ \frac{1}{ x }\tq x\in A \}$. Nous allons procéder en plusieurs étapes. D'abord, nous montrons que $1/\inf A$ est un majorant de $B$, et ensuite, nous montrerons que $1/\inf A$ est plus petit que tous les majorants.
	
	Pour tout $x$ dans $a$, nous avons $\inf A\leq x$, et donc
	\begin{equation}
		\frac{1}{ \inf A }\geq \frac{1}{ x }.
	\end{equation}
	Étant donné que $\inf A>0$, nous pouvons inverser les deux membres en retournant le sens de l'inégalité~:
	\begin{equation}
		\frac{1}{ \inf A }\geq y
	\end{equation}
	pour tout $y\in B$, ce qui signifie que $1/\inf A$ est un majorant de $B$.

	Soit maintenant $m$, un majorant de $B$. Nous voulons prouver que $m\geq 1/\inf A$, ou encore que $1/m\leq \inf A$. Étant donné que $m$ est un majorant de $B$, nous avons que pour tout $x\in A$,
	\begin{equation}
		\begin{aligned}[]
			m		&\geq\frac{1}{ x }\\
			\frac{1}{ m }	&\leq x\\
			\frac{1}{ m }	&\leq\inf A.
		\end{aligned}
	\end{equation}

\end{corrige}
