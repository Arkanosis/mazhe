% This is part of Outils mathématiques
% Copyright (c) 2011
%   Laurent Claessens
% See the file fdl-1.3.txt for copying conditions.

\begin{corrige}{Derive-0003}

	Nous avons déjà vu dans la correction de l'exercice \ref{exoDerive-0004} que pour un cercle
	\begin{equation}
		y'(x)=\frac{ -x }{ \sqrt{R^2-x^2} }.
	\end{equation}
	Un point général du cercle a pour abscisse $x=R\cos(\theta)$. En remplaçant nous trouvons le coefficient directeur suivant pour la tangente :
	\begin{equation}
		y'\big( R\cos(\theta) \big)=-\frac{1}{ \tan(\theta) }.
	\end{equation}
	Par conséquent une droite perpendiculaire à la tangente aurait comme coefficient directeur le nombre $\tan(\theta)$. Or cela est bien le coefficient directeur du rayon qui joint le point $(0,0)$ au point $\big( R\cos(\theta),R\sin(\theta) \big)$.

\end{corrige}
