% This is part of Un soupçon de physique, sans être agressif pour autant
% Copyright (C) 2006-2009,2012
%   Laurent Claessens
% See the file fdl-1.3.txt for copying conditions.


\begin{corrige}{INGE11140028}

	\begin{enumerate}

		\item
			Cette fonction est une très bonne candidate à l'utilisation de l'étau. En effet, nous avons une fonction trigonométrique ($\sin(x)\cos(x)$) multipliée par une fonction qu'on connait bien ($1/x$). Pour tout $x>0$, nous avons les inégalités
			\begin{equation}
				-\frac{1}{ x }\leq\frac{ \sin(x)\cos(x) }{ x }\leq \frac{1}{ x }
			\end{equation}
			Les deux fonctions $\pm\frac{1}{ x }$ tendent vers zéro lorsque $x$ va vers l'infini; par conséquent la fonction au milieu tend vers zéro.
			Note que l'on a utilisé le fait que $-1\leq \sin(x)\cos(x)\leq 1$.

		\item \label{ItemB40028ex}

                %TODO : refaire le dessin
			%La fonction est tracée sur la figure \ref{LabelFiggraphefonctionsisisi}
			%\newcommand{\CaptionFiggraphefonctionsisisi}{Tracé de la fonction pour l'exercie \ref{exoINGE11140028}\ref{ItemB40028ex}.}
			%\input{pictures_tex/Fig_graphefonctionsisisi.pstricks}
			Nous commençons par faire les manipulations suivantes :
			\begin{equation}
				\frac{ x+\sin(x) }{ x-\sin(x) }=\frac{ x\left( 1+\frac{ \sin(x) }{ x } \right) }{ x\left( 1-\frac{ \sin(x) }{ x } \right) }=\frac{ 1+\frac{ \sin(x) }{ x } }{ 1-\frac{ \sin(x) }{ x } }.
			\end{equation}
			En utilisant la règle de l'étau, nous savons que $\lim_{x\to \infty} \frac{ \sin(x) }{ x }=0$. Par conséquent nous avons
			\begin{equation}
				\lim_{x\to \infty} \frac{ x+\sin(x) }{ x-\sin(x) }=1.
			\end{equation}
			Remarque que, à côté de fonctions qui tendent vers l'infini, le sinus peut être considéré comme une constante.

	\end{enumerate}

\end{corrige}
