\begin{corrige}{EspVectoNorme0007}

	\begin{enumerate}
		\item
			La proposition \ref{PropBorneSupInf} nous indique que l'ensemble $\{ \| x-y \|\tq y\in A \}$ aura un infimum si et seulement s'il est borné par le bas. Cela est certainement le cas parce que c'est un ensemble de nombres positifs, qui accepte donc zéro pour minorant.

		\item
			Si $x$ n'est pas dans $\bar A$, il existe une boule de rayon $\delta$ autour de $x$ qui n'intersecte pas $A$. Or, par définition,
			\begin{equation}
				B(x,\delta)=\{ y\in\eR^N\tqs\| y-x \|<\delta \},
			\end{equation}
			donc tous les points de $A$ sont à distance au moins égale à $\delta$ de $x$ (les points à distance moins de $\delta$ seraient dans la boule). Donc $\delta$ est un minorant de l'ensemble $\{ \| x-y \|\tqs y\in A \}$, et l'infimum de cet ensemble ne peut pas avoir zéro comme infimum.

			Montrons à présent que les points $x$ de $\bar A$ sont tels que $d(x,A)=0$. Si $x\in\bar A$, nous avons une suite $(x_n)$ dans $A$ qui converge vers $x$. Cette convergence signifie que $\lim\| x-x_n \|=0$ (ne pas confondre $\lim x_n=x$ qui est une limite dans $\eR^N$ et $\lim\| x_n-x \|=0$ qui est une limite dans $\eR$).

			Pour tout $\varepsilon$, il existe donc un $x_n$ dans $A$ tel que $\| x_n-x \|\leq\varepsilon$, et donc $\inf_{y\in A}\| x-y \|\leq\varepsilon$. Cela prouve que cet infimum est zéro (parce que nous savons par ailleurs qu'il est positif ou nul).
			
	\end{enumerate}

\end{corrige}
