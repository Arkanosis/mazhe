\begin{exercice}\label{exoTD6A-0003}
  \begin{enumerate}
  \item Vérifier que le système d'équation différentielles 
\label{systWVHnZx}
\begin{subequations}
    \begin{numcases}{}
        \frac{ dx }{ dt }=  y\\
        \frac{ dy }{ dt }=-x
    \end{numcases}
\end{subequations}
    est équivalent à l'équation différentielle de deuxième degré 
    \begin{equation}\label{eq2deg}
      \frac{d^2x}{dt^2}= -x , 
    \end{equation}
    au sens que $x(t)=g(t)$ est une solution de \eqref{eq2deg} si et seulement si $x(t)=g(t),\, y(t)=g'(t)$ est une solution de \eqref{systWVHnZx}. 
  \item Trouver les solutions constantes de \eqref{systWVHnZx}.
  \item Vérifier que la fonction $x(t)= A\cos(t)+ B\sin(t)$ est une solution de \eqref{eq2deg} pour tous $A$, $B$ dans $\mathbb{R}$. Nous acceptons que \emph{toutes} les solutions de l'équation \eqref{systWVHnZx} sont de cette forme.
  \item Dessiner sur le plan $x$-$y$ les trajectoires des solutions $(x(t), y(t))$.
  \item Trouver toutes le solutions de \eqref{eq2deg} telles que $x(0)= 1$. Trouver sur le plan $x$-$y$ les trajectoires correspondantes. Comment faut-il faire pour choisir une unique solution de \eqref{eq2deg} ? Discuter. 
  \end{enumerate}
  
\corrref{TD6A-0003}
  
\end{exercice}
