% Ce fichier contient les choses faites avec l' "ancienne" décomposition d'Iwasawa.

%+++++++++++++++++++++++++++++++++++++++++++++++++++++++++++++++++++++++++++++++++++++++++++++++++++++++++++++++++++++++++++
\section{Some complements}
%+++++++++++++++++++++++++++++++++++++++++++++++++++++++++++++++++++++++++++++++++++++++++++++++++++++++++++++++++++++++++++

\subsection{A first brute force computation}
%/////////////////////////////////////////////

Let us use the ``old''{} Iwasawa decomposition for a little demonstrative and inessential computation. The exponentiations from $\sA\oplus\sN$ to $AN$ is given at page \pageref{pg:exp_AN}. Remark that a matrix of $\SOun$ leave unchanged the first column of $\SOdn$: 
\[
\begin{pmatrix}
  a&.&.&.\\
  b&.&.&.\\
  c&.&.&.\\
  d&.&.&.\\
\end{pmatrix}
\begin{pmatrix}
 1&0&0&0\\
 0&.&.&.\\
 0&.&.&.\\
 0&.&.&.
\end{pmatrix}=
\begin{pmatrix}
  a&.&.&.\\
  b&.&.&.\\
  c&.&.&.\\
  d&.&.&.\\
\end{pmatrix}.
\]
Thus, in order to compute the orbit $[AN]$ of $[\mtu]$, one can begin to compute a general matrix of $AN$ and impose conditions on the first column (it will not be affected by the equivalence classes). For example, in order to see if $[-\mtu]\in [\SO(2)]$ belongs to the orbit $[AN]$, we compute:\label{PgExplAN}
\begin{equation}\label{eq:gene_R}			
   \begin{pmatrix}
  \cosh\xi &    0      & \sinh\xi &    0  \\
    0       & \cosh\eta &    0      & \sinh\eta \\
  \sinh\xi &    0      & \cosh\xi &    0     \\
    0       & \sinh\eta &    0      & \cosh\eta \\
     \end{pmatrix}
     \begin{pmatrix}
                 1-2ab & b+a & 2ab   & b-a \\
 		 -b-a  & 1   & b+a   &  0   \\
		 -2ab  & b+a & 1+2ab &  b-a \\
		  b-a  &  0  &  a-b  &  1  
	      \end{pmatrix}	
\end{equation}
and we impose the first column to be $( -1,0,0,0)$:
\begin{subequations}\label{eq:S_14}
\begin{align}
  (1-2ab)\cosh\xi&-(2ab)\sinh\xi=-1   \label{eq:S1}\\
  (-b-a)\cosh\eta&+(b-a)\sinh\eta=0    \label{eq:S2}\\
  (1-2ab)\sinh\xi&-(2ab)\cosh\xi=0     \label{eq:S3}\\
  (-b-a)\sinh\eta&+(b-a)\cosh\eta=0.  \label{eq:S4}
\end{align}
\end{subequations}
It is easy to see that these equations doesn't accept any solutions. Indeed, the sum of equations \eqref{eq:S2} and \eqref{eq:S4} gives
\[
   -2a(\cosh\eta+\sinh\eta)=0.
\]
The possibility $a=0$ gives $\cosh\xi=-1$ in \eqref{eq:S1}; but the hyperbolic cosine is always bigger than one. Then $\cosh\eta=-\sinh\eta$. In this case, \eqref{eq:S4} gives $2b\cosh\eta=0$; since $\cosh\eta=0$ is not possible, $b$ must be zero. But with $b=0$, \eqref{eq:S1} gives once again $\cosh\xi=-1$.

Now we have to see that the matrices $e^{V_i}$ and $e^{W_j}$ doesn't change the result\label{pg:influence_V_W}. Since $W_j\in\sH$, it is clear that they will not change any result. If we compute $e^{cV_5}$ for example ($V_5^3=0$), we find
\begin{equation}		\label{EqExpDeV}
   e^{cV_5}=
\begin{pmatrix}
1+c^2/2	&		.	&	 -c^2/2	&	.	&	c\\
.		&	1	&	.	&	.	&	.\\
c^2/2		&	.	&	1-c^2/2	&	.	&	c\\
.		&	.	&	.	&	1	&	.\\
c		&	.	&	-c	&	.	&	1
\end{pmatrix}
\end{equation}
By multiply it by matrices of $AN$, we find
\[
\begin{pmatrix}
 &.&.&.&0\\
 &.&.&.&0\\
 &.&.&.&0\\
 &.&.&.&0\\
 &.&.&.&1\\ 
\end{pmatrix}
\begin{pmatrix}
 1+c^2/2&.&.&.&.\\
 0&.&.&.&.\\
 c^2/2&.&.&.&.\\
 0&.&.&.&.\\
 c&.&.&.&.
\end{pmatrix}
=
\begin{pmatrix}
 .&.&.&.&.\\
 .&.&.&.&.\\
 .&.&.&.&.\\
 .&.&.&.&.\\
 c&.&.&.&.
\end{pmatrix}.
\]
Then $c=0$ if we want it to be equal to $\pm\mtu$. As far as the matrices $W_j$ are concerned, it is even simpler: $W_j\in\sH$; then it will not affect the classes.

All that suppose that $N$ can \emph{globally} be written under the form
\[
  e^{aM}e^{bN}\prod_{i=5}^{n}e^{c_iV_i}\prod_{j=5}^{n}e^{c_jW_j}
\]
It is locally true from lemma \ref{lem:decomp}.


\subsection{Search for \texorpdfstring{$Z(K)$}{ZK}}
%////////////////////////////

Now we are going to find the center of $K$. From the explicit form \eqref{eq:K_H_SO}, we see that $Z(K)=\SO(2)$. Let us show it more abstractly. We consider the decomposition $\sG=\sH\oplus\sQ$ of $\sodn$ with respect to the involution $\sigma$:
\[
   \iK=\iKQ\oplus\iKH;
\]
This is an expression of the fact that $\SOdn/\SO(1,n)$ is a symmetric space. Since $\dim Z(K)=1$, it is a subset of $\iKQ$ or of $\iKH$ because $Z(K)$ is $\sigma$

On the other hand, we know\quext{Faudra un peu voir pourquoi} that a Cartan involution can be written as $\theta=\Ad(\exp Z)$ for a $Z\in \mZ(\iK)$. Then
\begin{equation}
\begin{split}
  0=[\sigma,\theta](X)&=\sigma\Ad(e^Z)X-\Ad(e^Z)\sigma X\\
                      &=\sigma e^{\ad Z}X-\Ad(e^Z)\sigma X\\
		      &=\Ad(e^{\sigma Z})\sigma X-\Ad(e^Z)X.
\end{split}
\end{equation}
Since $G$ has $\pm\mtu$ as center, this implies $e^{\sigma Z}=e^Z$. Then $e^Z\in H$. We can't however conclude that $z\in\sH$.

\subsection{The same with the ``old'' Iwasawa decomposition}
%//////////////////////////////////////////////////////////////////////

A general matrix of $\tsR$ is $mM+nN+tH_1+uH_2$; with the change of variable $x=n+m$, $y=n-m$, $a=t+u$ and $b=u-t$, it is
\begin{equation}
r=
\begin{pmatrix}
  0  & x & a  & y\\
  -x & 0 & x  & b\\
  a  & x & 0  & y\\
  y  & b & -y & 0
 \end{pmatrix}.
\end{equation}

Now we want to explicitly compute $\sR_z$ for a general matrix $z$ in the ``$\SO(2)$''\ part of $\SOdn$, i.e. for
\[
   z=
\begin{pmatrix}
  \cos u  & \sin u \\
  -\sin u & \cos u \\
    &  & 1  & 0\\
    &  & 0  & 1
 \end{pmatrix}.
\]
The result is 
\begin{equation}
\begin{split}
  \Ad&(z)\sR=\\
&\begin{pmatrix}
  0      & x      & a\cos u+x\sin u & y\cos u+b\sin u\\
  -x     & 0      & x\cos u-a\sin u & b\cos u-y\sin u\\
  a\cos u+x\sin u   & x\cos u -a\sin u &   0   & y\\
  y\cos u+b\sin u  & b\cos u-y\sin u &   -y  & 0
 \end{pmatrix}.   
\end{split}
\end{equation}
We are interested in the projection of these matrices with respect to $\sH$ (see equation \eqref{eq:gene_H}); then in order to see if $\sR_z=\sR$, we have to compare
\[
\begin{pmatrix}
  0      & x & a\cos u+x\sin u & y\cos u+b\sin u\\
  -x     &  &        & \\
  a\cos u+x\sin u  &  &        & \\
  y \cos u+b\sin u  &  &        & 
 \end{pmatrix}   
\]
with
\[
\begin{pmatrix}
  0  & x' & a'& y'\\
  -x' &  &  & \\
  a'  &  &  & \\
  y'  &  &  & 
 \end{pmatrix}
\]
By working on $a,b,x$ and $y$, we can easily fit the first matrix on the second
one if $c\neq 0$. In other words, $\sR_z\neq\sR$ only if the $\SO(2)$
part of $z$ is the rotation of an angle of $\pm\pi/2$. All that we can say now is that $\cos u=0$ is  not in the $\tR$-orbit of $\mfo$.

Up to here we have not taken the matrices $V_i$ and $W_j$ into account. It is rather easy to see that they don't change anything because (taking $i=j=5$ for sake of notational simplicity)
\[
  V_i+W_i\simeq
\begin{pmatrix}
0&0&0&0&1\\
0\\
0\\
0\\
1
\end{pmatrix}
\]
but
\[
  \Ad(z)(V_i+W_i)\simeq
\begin{pmatrix}
0&0&0&0&\sin u+\cos u\\
0\\
0\\
0\\
\sin u+\cos u
\end{pmatrix}.
\]

\subsection{A non-open orbit and a precision}\label{subsec:precision}
%-------------------------------------------

We consider the situation where $[\sigma,\theta]=0$ and the Iwasawa decomposition of $\SOdn$ for which $R_H$ is a subgroup of $R$ ($R_H$ is the ``$AN$''{} of $\SOun$). We denote by $\mfo$ the identity class: $\mfo=[e]=eH$. Since $\sH=\sR_H\oplus\sK_{\sH}$,
\[
   T_{\mfo}(R\mfo)=\pr_{\sH}(\sR)=(\sR+\sH)/\sH
\]
where the sum $\sR+\sH$ is not a direct sum. But $\sR+\sH=\sR+(\sR_H\oplus\sK_{\sH})=\sR\oplus\sK_H$ because $\sK_H\subset\sK$. Thus
\[
   T_{\mfo}(R\mfo)=(\sR\oplus\sK_H)/(\sR_H\oplus\sK_H)=\sR/\sR_H.
\]
\label{pg:subt_tilde}In the case $AdS_3$, $\dim(R\mfo)=6-3=3<4$; so that the identity orbit is not open. It is important to note that it contradicts the result of page \pageref{pg:mfo_ouvert}. The reason is that the latter was obtained with the ``old''{} Iwasawa decomposition.

From now the ``old''\ decomposition will be denoted by a tilde: 
\[
\lG=\tsA\oplus\tsN\oplus\sK.
\]
In this case, there exists a $k_0\in K$ such that $\Ad(k_0^{-1})\tsA=\sA$.  Moreover with this $k_0$, we have $\tA=k_0Ak_0^{-1}$ and $\tN=k_0Nk_0^{-1}$. This result can be found in \cite{Helgason} and maybe\quext{Faut que tu v\'erifies.} in \cite{Wisser}\quext{Quand tu auras refusion\'e, il faudra r\'ef\'erentier ta transcription de Wisser.}.

With these precisions, the previous computations are still relevant because  
\begin{equation}
   \tR[g]=k_0Rk_0^{-1}[g],
\end{equation}
which assures that the $\tR$-orbit of $[g]$ is the $R$-orbit of $[k_0^{-1} g]$ translated by
$k_0$. Since the translation is a diffeomorphism, the openness is not affected by the
translation. So the open $\tR$-orbit of $\mfo$ stated at page
\pageref{pg:mfo_ouvert} is now the open $R$-orbit of $k_0^{-1}$.

Now let us find out this famous $k_0$ matrix. Since the orbits are defined by the elements of the center of $K$ (i.e. a bloc-diagonal matrix in $\SOdn$ with a $\SO(2)$ matrix in the upper left corner and $\mtu$ anywhere else), we guess that $k_0$ is such a matrix. Before to compute it, we take a change of basis in $\tsA$: we consider $\frac{1}{2}(H_1+H_2)$ and $\frac{1}{2}(H_2-H_1)$ instead of $H_1$ and $H_2$. If we consider $d=\begin{pmatrix} 1&0\\0&0 \end{pmatrix}$, the problem is to find a matrix $z\in \SO(2)$ such that
\[
   \begin{pmatrix}
     z&0\\
     0&\mtu
   \end{pmatrix}
   \begin{pmatrix}
     0&d\\
     d&0
   \end{pmatrix}
   \begin{pmatrix}
     z^{-1}&0\\
     0&\mtu
   \end{pmatrix}
   =
   \begin{pmatrix}
    0&0&0&0\\
    0&0&1&0\\
    0&1&0&0\\
    0&0&0&0
   \end{pmatrix}.
\]
We can easily find that the only solution is 
$z=\begin{pmatrix}
     0&-1\\
     1&0
   \end{pmatrix}$.
Then
\begin{equation}
  k_0=
  \begin{pmatrix}
    0 &1&&\\
    -1&0&&\\
    &&1&0\\
    &&0&1
   \end{pmatrix}.
\end{equation}
  
\subsection{The same with the  ``old'' Iwasawa decomposition}
%/////////////////////////////////////////////////////////////////////

This useless point shows how to get the same conclusion with a bad choice of Iwasawa decomposition. If we consider some $z$ in the centrer of $K$ ($\SO(2)$) and $\tR_z=z\tR z^{-1}$, then the $\tR$-orbit of $\mfo$ is the $\tR$-orbit of $[z^{-1}]$ translated by $z$ and the $R$-orbit of $k_0z^{-1}$ translated by $zk_0^{-1}$:
\begin{equation}
   \tR_z=z\tR z^{-1}=zk_0 Rk_0^{-1} z^{-1}.
\end{equation}
Thus the study of the openness of the $\tR_z$-orbit of $\mfo$ is the study of the openness of the $R$-orbit of $k_0z^{-1}$. Proposition \ref{tho:pr_ouvert} allow us to perform this study by means of the surjectivity of $\dpt{\pr_{\sH}}{\tilde{\sR}_z}{\sQ}$. 

We begin by compute the matrices of $\sR_z$: $zH_1z^{-1}$, $zH_2z^{-1}$,
$zMz^{-1}$, $zNz^{-1}$, $zV_iz^{-1}$, $zW_jz^{-1}$. We make the following change of
basis in $\tsR$:
\begin{subequations}
\begin{align}
  H_1,H_2&\to \frac{1}{2}(H_1+H_2),\frac{1}{2}(H_2-H_1),\\
  M,N&\to\frac{1}{2}(M+N),\frac{1}{2}(M-N).
\end{align}
\end{subequations}
The result is that with
\begin{equation}
 z=\begin{pmatrix}S&0\\0&\mtu\end{pmatrix}\;\text{where}\; S=\begin{pmatrix}c&s\\-s&c\end{pmatrix}, 
\end{equation} 
and $c^2+s^2=1$, we have
\begin{equation}
H_{1z}=
\begin{pmatrix}
 &   & c &0\\
 &   &-s &0\\
c & -s &  &\\
0 & 0  &  &\\
\end{pmatrix},\quad
H_{2z}=
\begin{pmatrix}
 &   &  &s\\
 &   &  &c\\
 &   &  &0\\
s & c  & 0 &0\\
\end{pmatrix};
\end{equation}
\begin{equation}
M_z=
\begin{pmatrix}
0  & 1 & s &0\\
-1 & 0 & c &0\\
s  & c & 0 &0\\
0  & 0 & 0 &0\\
\end{pmatrix},\quad
N_z=
\begin{pmatrix}
  &  &  &-c\\
  &  &  &s\\
  &  &  &-1\\
-c & s & 1 &0\\
\end{pmatrix};
\end{equation}
\begin{subequations}
\begin{align}
   V_{iz}&=c(E_{1i}+E_{i1})-s(E_{2i}+E_{i2})+E_{3i}-E_{i3},\\
   W_{jz}&=s(E_{1j}+E_{j1})+c(E_{2j}+E_{j2})+E_{4j}-E_{j4}.
\end{align}  
\end{subequations}

The matrices we \emph{really} need are the projections of these one with respect to $\sH$. We denote it by  symbols with a line: 
\begin{subequations}
\begin{align}
\overline{H}_{1z}&=
\begin{pmatrix}
 &   &c  &0\\
 &   &0 &0\\
c & 0  &  &\\
0 & 0  &  &\\
\end{pmatrix},
&\overline{H}_{2z}&=
\begin{pmatrix}
 &   & 0 &s\\
 &   & 0 &0\\
0 & 0  &  &\\
s & 0  &  &\\
\end{pmatrix},\\
\overline{M}_z&=
\begin{pmatrix}
0  & 1 & s &0\\
-1 & 0 & 0 &0\\
s  & 0 & 0 &0\\
0  & 0 & 0 &0\\
\end{pmatrix},
&\overline{N}_z&=
\begin{pmatrix}
  &  &  &-c\\
  &  &  &0\\
  &  &  &0\\
-c & 0 & 0 &0\\
\end{pmatrix}\\
  \overline{V}_{iz}&=c(E_{1i}+E_{i1})
 &\overline{W}_{jz}&=s(E_{1j}+E_{j1}).
\end{align}
\end{subequations}

An explicit study shows that $\dpt{\pr_{\sH}}{\tsR_z}{\sQ}$ is surjective if and only if $c\neq 0$. Then the open $R$-orbits are the ones of $k_0^{-1} z^{-1}$ with $c\neq 0$:
\begin{equation}
\begin{pmatrix}
  0 & -1 & 0 & 0\\
  1 & 0  & 0 & 0\\
  0 & 0  & 1 & 0\\
  0 & 0  & 0 & 1\\
\end{pmatrix}
\begin{pmatrix}
  c & -s & 0 & 0\\
  s & c  & 0 & 0\\
  0 & 0  & 1 & 0\\
  0 & 0  & 0 & 1\\
\end{pmatrix}
=
\begin{pmatrix}
  -s & -c & 0 & 0\\
  c  & -s & 0 & 0\\
  0  &  0 & 1 & 0\\
  0  & 0  & 0 & 1\\
\end{pmatrix}
\end{equation}
Consequently, the not open $R$-orbits are the ones of $[\pm\mtu_{\SO(2)}]$.

\subsection{Orbits as homogeneous spaces}\index{homogeneous!space}
%----------------------------------------
A homogeneous space is a space with a transitive homeomorphism group. We consider the orbit $\mO=R[z]$ ($z\in \SO(2)$) which is a homogeneous space because the action of $G$ is trivially transitive. Theorem \ref{tho:homeo_action} assures us that $\mO$ can be written as $R/S$ where $S$ is the group which fixes some point in $\mO$.

We will firstly work out the homogeneous structure of the open $\tsR$-orbits; the matrices of $\tsR$ which leave invariant the point $[\mtu]$ of $G/H$ are the ones of the product \eqref{eq:gene_R} which satisfy some equations that are almost the same as \eqref{eq:S_14}:
\begin{subequations}
\begin{align}
  (1-2ab)\cosh\xi&-(2ab)\sinh\xi=1   \\
  (-b-a)\cosh\eta&+(b-a)\sinh\eta=0   \\
  (1-2ab)\sinh\xi&-(2ab)\cosh\xi=0    \\
  (-b-a)\sinh\eta&+(b-a)\cosh\eta=0.  
\end{align}
\end{subequations}
It is easy to see that the solutions are given by $a=b=\xi=0$ and no constraint on $\eta$. The set $S$ is thus given by
\begin{equation}
S\leadsto
\begin{pmatrix}
 1 & 0 & 0 & 0 \\
 0 & \cosh\eta & 0 & \sinh\eta \\
 0 & 0 & 1 & 0 \\
 0 & \sinh\eta & 0 & \cosh\eta \\
\end{pmatrix}.
\end{equation}
which corresponds\footnote{More precisely, because of the quotient by $H$, we had derived a necessary characterization of $S$, not a sufficient one. But we will soon see that in facts this is also a sufficient condition.} to $ e^{t(H_{1}-H_{2})}$ via the change of variable \eqref{eq:chm_xi_eta}. Since this is a matrix of $H$, this leaves $[\mtu]$ unchanged. As far as the matrices $V_i$ and $W_j$ are concerned, the reasoning of page \pageref{pg:influence_V_W} still holds.

So the part of $\tsR$ which leaves $\mtu$ unchanged is $\tsR\cap H$ (this is not really amazing). Now recall that $\tsR=k_0Rk_0^{-1}$. If $\tilr\in\tsR$ fixes $\mtu$, then $k_0^{-1}\tilr k_0\in R$ fixes $k_0^{-1}$, and if $r\in R$ fixes $k_0^{-1}$, then $k_0rk_0^{-1}\in\tsR$ fixes $\mtu$, so that $k_0rk_0^{-1}\in\tR\cap H$.

Then $r\in R$ fixes $[k_0^{-1}]$ if and only if $r\in k_0^{-1}(\tsR\cap H)k_0$. On the other hand, for the closed orbits the stabilizer in $R$ is $R\cap H$: $r\in R$ fixes $[\mtu]$ if $r[\mtu]=[\mtu]$, i.e. $[r]=[\mtu]$, which needs a $h\in H$ such that $r=h$.

