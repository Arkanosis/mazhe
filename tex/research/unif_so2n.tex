\section{Deformation of \texorpdfstring{$\SOdn$}{SO2n}}   \label{SecUnifSOdn}
%+++++++++++++++++++++++++++++++++++++++++++++++++++++++++++++++++

\subsection{Applying the extension lemma with old Iwasawa}
%----------------------------------------

One purpose of this section is to prove that
\[
  \SO(2,n)\stackrel{\sA\oplus\sN}{=}SU(1,1)\oplus SU(1,n-1).
\]
For that purpose, we will decompose the Iwasawa algebra of $\so(2,n)$ as a (symplectic) direct sum and we will compare the root spaces decomposition of each of the two parts with the ones of $\gsu(1,1)$ and $\gsu(1,n)$. Hence prescription on $\mathfrak{s}_1$ is to be two dimensional and to contains one and only one element of $\sA$. This condition will impose a change of variable in the ``new'' Iwasawa.

The first point is to decompose the Lie algebra $\sR:=\mA\oplus\mN$ (from the Iwasawa decomposition of
 $\sodn$) into two parts $\mfs_1$ and $\mfs_2$ such that, as Lie algebras,
\[
  \sR=\mfs_1\oplus_{\rho}\mfs_2
\]
where $\rho$ is the adjoint action in $\so(2,n)$. Recall that
\[
   \mA\oplus\mN=\{H_1,H_2,M,N,V_i,W_j\}.
\]
From symplectic considerations which will appear further, we want $\mfs_1$ and $\mfs_2$ to be even dimensional.
One can easily remark that
\decompss{H_1,N}{H_2,M,V_i,W_j}{}
works. What one has to check is that $\mfs_1$ and $\mfs_2$ are closed under $\ad$: $[s,t]\in\mfs_i$ if $s$, $t\in\mfs_i$, and that $\mfs_1$ acts on $\mfs_2$: $[A,s]\in\mfs_2$ if $A\in\mfs_1$ and $s\in\mfs_2$. This is a rather strong constraint.

Let us now explore systematically the possibilities. In a first time, we will not pay attention to the symplectic part. The question is to explicitly find all the possibilities of $\mfs_i$ such that $\mA\oplus\mN$ is semi-direct product of $\mfs_1$ and $\mfs_2$.

As notational convention, when one write $E=\{W_i\}$, we mean that \emph{all} the $W_i$ are in the set $E$. If we want to say that \emph{one particular} $W_a$ is in $E$, we use the indices $a,b,c\ldots$.

If $H_1\in\mfs_2$, one can only finds $H_2$ and $M$ in $\mfs_1$ because in
\decompss{W_i,N,V_i,\ldots}{H_1,\ldots}{,}
the Lie algebra $\mfs_1$ doesn't acts on $\mfs_2$. So with $H_1\in\mfs_2$, the only possibility is
\decompss{H_2,M}{H_1,V_i,W_i,N}{.}
But $\mfs_2$ is not closed for $\ad$. First conclusion: $H_1\in\mfs_1$.

Let us consider the case $H_2\in\mfs_2$. In this case, one can only put $H_1$ and $N$ in $\mfs_1$: $H_2\in\mfs_2$ implies
\decompss{H_1,N}{H_2,M,V_i,W_j}{.}
From now, the question becomes ``who can belong to $\mfs_1$ in the same time as $H_1$ ?``
 
The first step is to show that $M$ can't. Let us consider $H_1$, $M\in\mfs_1$. Since $[V_i,W_j]=\delta_{ij}M$, in order for $\mfs_2$ to be closed for $\ad$, one has to put some $V_a$ and (or) $W_b$ in $\mfs_1$. If $W_a\in\mfs_1$, $V_a$ must also belongs to $\mfs_1$ because $\mfs_1$ must acts on $\mfs_2$. For the same reason, $V_a\in\mfs_A$ implies $W_a\in\mfs_1$. Thus, $M\in\mfs_1$ imply at least
 \[
    \{H_1,M,V_i,W_j\}\subset \mfs_1.
 \]
Now, it is also clear that $N\in\mfs_2$ is not possible because of the action: $[W_i,N]=-2V_i$. We are left with
\decompss{H_1,M,N,V_i,W_j}{H_2}{,}
but we want even dimensional spaces. Conclusion: $H_1$, $M\in\mfs_1$ is not possible.

Our second point is to show that $H_1,V_a\in\mfs_1$ is also not possible. For, let us consider that one actually has $H_1,V_a\in\mfs_1$, ant let us explore the consequences. It is clear that $W_a$ and $N$ can't be in $\mfs_2$ in the same time. If $W_a\in\mfs_1$, $M$ must also be in $\mfs_1$ in order to close under $\ad$. But we just see that it was impossible. On the other hand, if $N$ belongs to $\mfs_1$, since $[N,W_a]=2V_a\in\mfs_1$, $W_a$ can't belongs to $\mfs_2$. Then we are left with the precedent case: $W_a\in\mfs_1$. We conclude that $H_1,V_a\in\mfs_1$ is not possible.
From now, one sees that the only possibility with $H_1$, $N\in\mfs_1$ is 
\decompss{H_1,N}{H_2,M,V_i,W_j}{.}

The two last cases to explore are $H_2$ or (and)  $W_a$ in $\mfs_1$ with $H_1$. If $W_a\in\mfs_1$, then $H_2\in\mfs_1$ because of the action of $\mfs_1$ on $\mfs_2$ and $[W_a,H_2]=-W_a\in\mfs_1$. On the other hand we had yet seen that $V_i\in\mfs_2$. So we are left with
\decompss{H_1,H_2,W_a,W_b}{M,N,W_{\neq a,b},V_i}{,}
and more generally, one can take in $\mfs_1$ any even number of $W_i$.

One checks that the latest possibility (which is a special case of the previous) works:
\decompss{H_1,H_2}{M,N,W_i,V_i}{.}

The decomposition of $\mA\oplus\mN$ as semi-direct product of $\mfs_1$ and $\mfs_2$ are:
\decompss{H_1,H_2,\underbrace{W_a,\ldots,W_b}_{\text{even or zero}}}{M,N,W_{\ldots},V_i}{,}
and
\decompss{H_1,N}{H_2,M,V_i,W_j}{.}


\subsubsection*{The symplectic conditions}
%////////////////////////////////////////

Let us now turn our attention to the symplectic matter. There are two symplectic constraint on the choice of the $\mfs_i$. The first one is that each one must be a \defe{symplectic Lie algebra}{symplectic!Lie algebra}: if $\Omega_i$ is the symplectic $2$-form on $\mfs_i$, then for any $x$, $y$, $z\in\mfs_i$,
\begin{equation}\label{eq:symple_Lie}
\Omega_i([x,y],z)+\Omega_i([y,z],x)+\Omega_i([z,x],y)=0.
\end{equation}

The second one is the fact that the action of $\mfs_1$ on $\mfs_2$ must be symplectic in the sense that 
$\forall X\in\mfs_1,\, \ad X\in\mfsp (\Omega_2)$. So one has to check that 
\begin{equation}
   \Omega_2\big(  \Ad_s A,\Ad_s B   \big)=\Omega_2(A,B)
\end{equation}
for any $s\in S_1$ and $A$, $B\in\mfs_2$. This fact was crucially used in the proof of proposition \ref{prop:Darboux}.

We first check that
\decompss{H_1,H_2,W_a,\ldots,W_b}{M,N,W_{\ldots},V_i}{}
doesn't works.
Indeed, one must have
\[
   \Oexp{H_1}{M}{N}=\Omega_2(M,N),
\]
but $[H_1,M]=0$, $[H_1,N]=2N$ and $\exp(\ad H_1)=id+[H_1,.]+\ldots$, then
\[
  \Oexp{H_1}{M}{N}=\Omega_2(M,e^2N),
\]
so that $\Omega_2(M,N)=0$.
Note that this is more general: $\Omega_2(M,s)=0$ for all $s\in\mfs_2$ such that $[H_1,s]=\alpha s$ because
\[
   \Omega_2(M,s)\stackrel{!}{=}\Omega_2(e^{\ad H_1}M,e^{\ad H_1}s)=\Omega_2(M,e^{\alpha}s).
\]
This imposes
\begin{equation}
\begin{split}
   \Omega_2(M,V_i)&=0\\
   \Omega_2(M,N)&=0\\
   \Omega_2(M,W_i)&=0.
\end{split}
\end{equation}
Thus $\Omega_2$ is degenerate. Now, we check that the second works:
\decompss{H_1,N}{H_2,M,V_i,W_j}{.}
The condition \eqref{eq:symple_Lie} gives
\begin{equation}
\begin{split}
   \Omega_2(V,M)&=0\\
   \Omega_2(W,M)&=0\\
   2\Omega_2(V,W)+\Omega_2(M,H_2)&=0.
\end{split}
\end{equation}
When $s\in\mfs_2$ is such that $[H_1,s]=\alpha s$, $\Omega_2(H_2,s)=0$. Now the matrix of $\Omega_2$ looks like
\begin{equation}
\Omega_2=\left(
\begin{array}{c|c|c|c|c}
 & H_2 & M & V_i & W_j \\ 
 \hline
H_2 & 0 &  & 0 & 0 \\ 
\hline
M &  & 0 & 0 & 0 \\ 
\hline
V_i & 0 & 0 & 0 &  \\ 
\hline
W_j & 0 & 0 &  & 0
\end{array}
\right)
\end{equation}
We have to check that the last entries keep free.
\begin{equation} \begin{split}
   \Omega_2(M,H_2)&\stackrel{!}{=}\Oexp{H_1}{M}{H_2}\\
                  &\stackrel{!}{=}\Oexp{N}{M}{H_2}.
\end{split}
\end{equation}
Since $[H_1,M]=[H_1,H_2]=[N,M]=[N,H_2]=0$, these two conditions are fulfilled. On the other hand,
\begin{equation}
\begin{split}
   \Omega_2(V_a,W_b)&\stackrel{!}{=}\Oexp{H_1}{V_a}{W_b}\\
                    &=\Omega_2(eV_A,e^{-1} W_b)\\
		    &=\Omega_2(V_a,W_b)\\
		    &\stackrel{!}{=}\Oexp{N}{V_a}{W_b}\\
		    &=\Omega_2(V_a,W_b+2V_b).
\end{split}
\end{equation}
Thus $\Omega_2(V_a,W_b)$ has no constraints and $\Omega_2(V_a,V_b)=0$. Finally, it is easy to see that
\begin{equation}
\begin{split}
\Omega_2(W_a,W_b)&\stackrel{!}{=}\Oexp{H_1}{W_a}{W_b}\\
                 &=e^{-2}\Omega_2(W_a,W_b),
\end{split}
\end{equation}
so that $\Omega_2(W_a,W_b)=0$. Now, one can write the possible form for $\Omega_2$ as
\begin{equation}
\Omega_2=\left(
\begin{array}{c|c|c|c|c|c|c}
 & H_2 & M & V_1 & V_2 & W_1 & W_2 \\ 
 \hline
H_2 & 0 & a & 0 & 0 & 0 & 0 \\ 
\hline
M & -a & 0 & 0 & 0 & 0 & 0 \\ 
\hline
V_1 & 0 & 0 & 0 & 0 & a/2 & a/2 \\ 
\hline
V_2 & 0 & 0 & 0 & 0 & a/2 & a/2 \\ 
\hline
W_1 & 0 & 0 & -a/2 & -a/2 & 0 & 0  \\ 
\hline
W_2 & 0 & 0 & -a/2 & -a/2 & 0 & 0  \\ 
\end{array}
\right)
\end{equation}



\subsection{Decomposition as split extension}
%------------------------------------------

When we try to decompose $\sA\oplus\sN$ as symplectic direct sum with a strict respect to chosen basis matrices, there are only two possibilities:
\begin{equation}
 \begin{split}
  \mfs_1&=\{ J_1,J_2,\overbrace{V_a,\ldots,V_b}^{\text{even}}  \}\\
\mfs_2&=\{ L,M,W_i,V_{\ldots}  \}  
\end{split}   
\end{equation}
and
\begin{equation}
 \begin{split}
  \mfs_1&=\{ J_2,\overbrace{V_a,\ldots,V_b}^{\text{odd}}  \}\\
\mfs_2&=\{ J_1,W_j,M,L,V_{\ldots}  \}  .
\end{split}   
\end{equation}
In the first possibility, $\mathfrak{s}_{1}$ is not symplectic (because it is abelian); while the second one only works in low dimensional cases: there must not be any $V_a$. This fact leads us to consider  the change of basis \eqref{EqChmHJ} in $\sA$: $H_1=J_1-J_2$ and $H_2=J_1+J_2$.

If $H_1\in\mathfrak{s}_2$, then $L,V_i,W_i\in\mathfrak{s}_2$ because $\mathfrak{s}_1$ must act on $\mathfrak{s}_2$. Hence $M\in\mathfrak{s}_2$ and $H_2$ remains alone in $\mathfrak{s}_1$. That proves that $H_1\in\mathfrak{s}_1$. If we suppose that $H_2\in\mathfrak{s}_2$, we find
\begin{equation}  \label{eq_HLss}
 \begin{split}
  \mfs_1&=\{ H_1,L  \}\\
\mfs_2&=\{ H_2,V_i,W_j,M  \}.
\end{split}   
\end{equation}
The case $H_1,H_2\in\mathfrak{s}_1$ leads to
\begin{equation}   \label{Eq_HHVaMLW}
 \begin{split}
	\mfs_1&=\{ H_1,H_2,\overbrace{V_a,\ldots V_b}^{\text{even}}  \}\\
	\mfs_2&=\{ M,L,W_i,V_{\text{others}}  \}.
\end{split}   
\end{equation}
The symplectic condition excludes the second decomposition. Indeed for each $s$ such that $[H_1,s]=\alpha s$ (i.e. $s=V_i,W_j,L$), we have
\[ 
  \Omega_2\big(  e^{ad H_1}M, e^{\ad H_1}s \big)= e^{\alpha}\Omega_2(M,s)\stackrel{!}{=}\Omega_2(M,s).
\]
Hence $\Omega_2(M,s)=0$. This proves that the decomposition \eqref{Eq_HHVaMLW} imposes the symplectic form $\Omega_2$ to be degenerate. We are left with decomposition \eqref{eq_HLss}.

Root space decomposition of $SU(1,n)$ can be found on pages 314--315 of \cite{Knapp}: it has $\dim\sA=1$, $\dim\sG_{2f}=1$ and $\dim\sG_f=2(n-1)$. In $\mathfrak{s}_2$, we have $V_i\in\sG_1$, $W_j\in\sG_1$, $M\in\sG_2$, and when we look at $AdS_l=\SO(2,l-1)/\SO(1,l-1)$, we have $l-3$ matrices $V_i$ and $W_j$. Therefore $\mathfrak{s}_2$ is nothing else than the $\sA\oplus\sN$ of $\mathfrak{su}(1,l-2)$ (recall $l\geq3$). The analysis shows that $\mathfrak{s}_1$ is the $\sA\oplus\sN$ of $\mathfrak{su}(1,1)$.

\subsection{Conclusion and perspectives}
%----------------------------------------

For our $AdS_l$ black hole, the algebra of the group which defines the singularity is the split extension 
\[ 
  (\sA\oplus\sN)_{\so(2,l-1)}=(\sA\oplus\sN)_{\mathfrak{su}(1,1)}\oplus_{\ad}(\sA\oplus\sN)_{\mathfrak{su}(1,l-2)}.
\]
A deformation of the corresponding groups is given in the article \cite{Biel-Massar}, and the extension lemma \ref{EXT} yields now an oscillatory integral universal deformation formula for proper actions of the Iwasawa subgroup of $\SO(2,l-1)$. That remark provides an alternative way to deform the black hole to the one presented in section \ref{SecGpStructOuvertOrb}.

The availability of a quantization of $AdS_l$ by action of $AN$ is an opportunity to embed our black hole toy model in the framework of noncommutative geometry. Indeed, the quantization of $AdS_l$ is the data of the anti de Sitter manifold and the action of the group $AN$; that is precisely the data which defines the black hole of chapter \ref{ChapBHinAdS}. So we would be able to ``see'' the causal issue from the data of the deformed spectral triple. Remark that a causal structure (in the physical meaning of the term) is a special property of \emph{pseudo}-Riemannian manifolds for which spectral geometry does not exist yet.

An important remaining problem with that method is the fact that the extension lemma does not assure the existence of a stable functional space for the new product. So there is still a lot of analytic work to be done.

