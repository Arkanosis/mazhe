\section{Some of my favourite mistakes}
%+++++++++++++++++++++++++++++++++++++

In this appendix, we explain a lot of mistakes and false ideas that one can commit.

\begin{erreur}\label{err:decomp}
The reasoning at page \pageref{pg:X_t} which deduce $X(t)=e$ from $[X(t)]=cst$ is not a particular case of lemma \ref{lem:decomp}, nor a re-proof of it. Here, we decompose $X(t)=gk(t)$ with $k(t)$ in the ``right space''\ from arguments which are true in our case but not in general.
\end{erreur}

\begin{erreur}\label{err:Intt_Aut}
Since $\ad(\lA)$ is a subalgebra of $\partial(\lA)$, it is the Lie algebra of an unique Lie subgroup of $\Aut(\lA)$ from theorem \ref{tho:gp_alg}. Then we deduce that $\Int(\lA)$ is a Lie subgroup of $\Aut(\lA)$. There is a topological problem: $\Int(\lA)$ is defined as an analytic subgroup of $\GL(\lA)$. There is no evidence that the topology of the subgroup of $\Aut(\lA)$ given by \ref{tho:gp_alg} is the one of $\Int(\lA)$.

So theorem \ref{tho:gp_alg} is not applicable to find relations between $\Int(\lA)$ and $\Aut(\lA)$.
\end{erreur}

\begin{erreur}\label{err:gross}
An easy (but false) counter example. If $G$ is a Lie group and $H$, a closed Lie subgroup of $G$ on which we consider the trivial topology (the open subsets of $H$ are $H$ itself and $\emptyset$). Then it is clear that $\dpt{\iota}{H}{H'}$ is not continuous, so that $H$ is not a topological subgroup. So the second item of theorem \ref{tho:H_ferme} is false.

In the definition of a \emph{Lie} subgroup, we want $H$ to be a submanifold. But proposition \ref{prop:topo_sub_manif} says that a submanifold has at least the induced topology. So a Lie subgroup with the trivial topology doesn't exist. See also the example at page \pageref{pg:ex_topo_Lie}.
\end{erreur}

\begin{erreur}\label{err:gp_meme_alg}
We can say that $e^Xe^Y$ can be written as $e^Z$ for a certain $Z\in\lG=\lH$ given by the Campbell-Backer-Hausdorff formula. Since the Lie algebras are the same, the CBH formula of $G$ and $H$ are the same. Then we can forget the assumption which says that $H$ is a subgroup of $G$. Thus two groups which have the same algebra are the same on a neighbourhood of $e$. This is false because there are no guaranties for $X$, $Y\in A$ that $[X,Y]$ still belongs to $A$.
\end{erreur}

\begin{erreur}\label{err:f_dege}
If $\lG\hperp=\lG$, then $f$ is degenerate. But we said in the assumptions that $f$ were nondegenerate. So $\lG=0$. Be careful: we had shown that \emph{if $f$ is nondegenerate}, then $f=\lambda B$. Here we pose $f=\tr$ and we have to show that it is nondegenerate in order to prove that $B=\lambda\tr$.
\end{erreur}

\begin{erreur}
\begin{proposition}  \label{ProperrProdInvarDiffeo}
Let $\phi\colon G_{1}\to G_{2}$ be a Lie group diffeomorphism such that
\begin{equation} \label{Eqerrphideformdiff}
  \phi\circ L_{g}=L_{g'}\circ\phi.
\end{equation}
If $K$ is a left invariant kernel on $G_{1}$, then $K'=\phi^*K$ is a left invariant kernel on $G_{2}$.
\end{proposition}

\begin{proof}
Just a computation using left invariance of $K$ :
\[ 
\begin{split}
  L_{g}^*K'&=(\phi\circ L_{g})^*K\\
		&=(L_{g'}\circ\phi)^*K\\
		&=\phi^*L_{g'}^*K\\
		&=\phi^*K\\
		&=K'.
\end{split}  
\]

\end{proof}

The mistake is the fact that equation \eqref{Eqerrphideformdiff} should better be written
\[ 
  \phi\circ L_{g}(h)=L_{g'}\circ\phi(h)
\]
where $g'$ depends on $h$.  But the kernel is a \emph{three} points function. 

\end{erreur}
