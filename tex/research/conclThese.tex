
%+++++++++++++++++++++++++++++++++++++++++++++++++++++++++++++++++++++++++++++++++++++++++++++++++++++++++++++++++++++++++++ 
\section{Conclusion}
%+++++++++++++++++++++++++++++++++++++++++++++++++++++++++++++++++++++++++++++++++++++++++++++++++++++++++++++++++++++++++++

In a first time we defined a black hole in anti de Sitter space. This construction is not related to any metric divergence but is a dimensional generalization of a causal black hole whose singularity is dictated by causal issues. The originality of our approach lies in the fact that our method uses essentially group theoretical and symmetric spaces techniques. That result should be generalisable to any semisimple symmetric space.

Then we proved that the physical domain of the black hole (the non singular part) is equivalent to a group in the sense that there exists a group which acts freely and transitively by diffeomorphisms. So we identify the group with the manifold and it is easy to prove that the latter group is a split extension of an Heisenberg group which happens to be quantizable by a twisted pull-back of a previously known quantization of $\SU(1,n)$. 

We also proved two somewhat out of subject small results. The first one is the fact that a deformation of the half-plane by Unterberger can be transported to a deformation of the Iwasawa subgroup of $\SL(2,\eR)$ which can in turn deform (by the group action method) the dual of its Lie algebra. We  showed however that that deformation is not universal; indeed we pointed out two different actions of the Iwasawa subgroup of $\SL(2,\eR)$ on $AdS_2$ for which the deformation by group action method reveals to be unable to even multiply two compactly supported functions. An interesting question is to know the precise point in the construction of Unterberger which makes his product non universal.

The second small result is a proof of concept for quantization of the Iwasawa subgroup of $\SO(2,n)$ by the method of the extension lemma. We wrote $\SO(2,n)$ as a symplectic split extension of $\SU(1,n)$ by $\SU(1,1)$. The extension lemma then provided a kernel on $\SO(2,n)$ because kernels were known on $\SU(1,1)$ and $\SU(1,n)$. Is that quantization equivalent in some sense to the one that we performed in the main line of the black hole deformation ? That question still has to be solved.

As a final remark, I want to point out that the major challenge of this century is not quantization, but global warming.
