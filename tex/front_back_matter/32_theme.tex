
\InternalLinks{déterminant}
    \begin{enumerate}
    \item
        Les \( n\)-formes alternées forment un espace de dimension \( 1\), proposition \ref{ProprbjihK}.
    \item
        Déterminant d'une famille de vecteurs \ref{DEFooODDFooSNahPb}.
    \item
        Déterminant d'un endomorphisme \ref{DefCOZEooGhRfxA}.
        \item
            Des interprétations géométriques du déterminant sont dans la section \ref{SECooSQRDooGifgQi}.
        \item
            Le déterminant de Vandermonde est à la proposition \ref{PropnuUvtj}. Il est utilisé à divers endroits :
\begin{enumerate}
    \item
        Pour prouver que \( \tr(u^p)=0\) pour tout \( p\) si et seulement si \( u\) est nilpotente (lemme \ref{LemzgNOjY}).
    \item
        Pour prouver qu'un endomorphisme possédant \( \dim(E)\) valeurs propres distinctes est cyclique (proposition \ref{PropooQALUooTluDif}).
\end{enumerate}

   \end{enumerate}

