% This is part of Outils mathématiques
% Copyright (c) 2011
%   Laurent Claessens
% See the file fdl-1.3.txt for copying conditions.

\begin{corrige}{OutilsMath-0116}

    Des trois variables $r$, $\theta$ et $z$, la plus facile à comprendre est $z$. Cette variables correspond au $z$ usuel. Le problème revient donc à comprendre quelle forme géométrique nous avons à chaque hauteur $z$. Pour chaque $z$ et $r$, nous avons un cercle de rayon
    \begin{equation}
        \frac{ R }{ h }zr
    \end{equation}
    situé à la hauteur $z$. Lorsque $r$ varie de $0$ à $1$, ce rayon varie de $0$ à $\frac{ R }{ h }z$. Notre volume $V$ consiste donc en un disque plein de rayon $Rz/h$ à chaque hauteur. Lorsque $z=0$, ce rayon est $0$ et lorsque $z=h$, ce rayon devient $R$. Nous avons donc un cône de hauteur $h$ posé sur la pointe.

    Le volume est donné par
    \begin{equation}
        V=\int_0^1dr\int_0^{2\pi}d\theta\int_0^hdz| T_z\cdot(T_r\times T_{\theta}) |
    \end{equation}
    où
    \begin{equation}
        \begin{aligned}[]
            T_r&=\frac{ \partial \phi }{ \partial r }=\begin{pmatrix}
                \frac{ R }{ h }z\cos(\theta)    \\ 
                \frac{ R }{ h }z\sin(\theta)    \\ 
                0    
            \end{pmatrix}\\
            T_{\theta}&=\frac{ \partial \phi }{ \partial \theta }=\begin{pmatrix}
                -\frac{ R }{ h }zr\sin(\theta)    \\ 
                \frac{ R }{ h }zr\cos(\theta)    \\ 
                0    
            \end{pmatrix}\\
            T_z&=\frac{ \partial \phi }{ \partial z }=\begin{pmatrix}
                \frac{ R }{ h }r\cos(\theta)    \\ 
                \frac{ R }{ h }r\sin(\theta)    \\ 
                1    
            \end{pmatrix}.
        \end{aligned}
    \end{equation}
    Le calcul donne
    \begin{equation}
        \begin{aligned}[]
            T_z\cdot(T_r\times T_{\theta})
            &=\begin{vmatrix}
                \frac{ R }{ h }z\cos\theta    &   \frac{ R }{ h }z\sin\theta    &   0    \\
                -\frac{ R }{ h }zr\sin\theta    &   \frac{ R }{ h }zr\cos\theta    &   0    \\
                \frac{ R }{ h }r\cos\theta    &   \frac{ R }{ h }r\sin\theta    &   1    
            \end{vmatrix}\\
            &=\frac{ R^3 }{ h^3 }
            \begin{vmatrix}
                z\cos\theta    &   z\sin\theta    &   0    \\
                -zr\sin\theta    &   zr\cos\theta    &   0    \\
                r\cos\theta    &   r\sin\theta    &   h/R    
            \end{vmatrix}\\
            &=\frac{ R^2 }{ h^2 }\big( z^2r\cos^2(\theta)+z^2r\sin^2(\theta) \big)\\
            &=\frac{ R^2 }{ h^2 }z^2r.
        \end{aligned}
    \end{equation}
    L'intégrale qui donne le volume est au final :
    \begin{equation}
        V=\int_0^1\int_0^{2\pi}\int_0^h\frac{ R^2 }{ h^2 }z^2r\,dz\,d\theta\,dr=\frac{ \pi R^2h }{ 3 }.
    \end{equation}

\end{corrige}
