% This is part of Mes notes de mathématique
% Copyright (c) 2015-2016
%   Laurent Claessens
% See the file fdl-1.3.txt for copying conditions.

%+++++++++++++++++++++++++++++++++++++++++++++++++++++++++++++++++++++++++++++++++++++++++++++++++++++++++++++++++++++++++++ 
\section*{Ce cours à l'agrégation ?}
%+++++++++++++++++++++++++++++++++++++++++++++++++++++++++++++++++++++++++++++++++++++++++++++++++++++++++++++++++++++++++++

Peut-on utiliser ce cours pour \textbf{les oraux d'\href{http://agreg.org/}{agrégation}} (de mathématiques) ?  Cela est une question qui m'est arrivée quelques fois.  

Le règlement interdit d'apporter avec soi une version imprimée chez soi, et oblige de n'utiliser que des ressources commercialisées. Cela fait que le Frido \emph{tel que vous l'avez sous les yeux} n'est pas utilisable à l'agrégation. Pour utiliser le Frido, vous devrez payer; j'en suis le premier désolé.

Une version sera prochainement commercialisée sur \href{http://www.thebookedition.com/fr/}{thebookedition.com}. Le Frido sera divisé en trois volumes entre \( 500\) et \( 600\) pages chacun, pour un prix entre \( 20\) et \( 25\) euros chacun (plus frais d'envoi).

Vous pouvez déjà télécharger les volumes en pdf :
\begin{itemize}
    \item 
 \url{http://laurent.claessens-donadello.eu/pdf/lefrido-vol1.pdf}
    \item 
 \url{http://laurent.claessens-donadello.eu/pdf/lefrido-vol2.pdf}
    \item 
 \url{http://laurent.claessens-donadello.eu/pdf/lefrido-vol3.pdf}
\end{itemize}
Ces fichiers ne seront pas mis à jour (forcément) pour vous permettre de travailler l'agrégation sur votre ordinateur avec exactement le texte dont vous disposerez lors des oraux si vous décidiez de les acheter (ou si vous décidez quelqu'un de les acheter pour vous -- votre université par exemple).

Lorsque ce sera dûment commercialisé, la grande question sera : avez-vous le droit de demander à votre université d'imprimer \emph{ces pdf-là} et de les mettre dans la malle ? Le règlement dit que le document doit être commercialisé, pas que le candidat doit passer par la voie commerciale pour se le procurer \ldots

Deux mots sur mon modèle économique. J'ai choisi de ne pas faire de bénéfice sur les ventes : à chaque copie vendue sur internet, je gagne zéro\footnote{En hexadécimal, ça se note \( 0\).} euros. Mon modèle économique est le don. Si vous réussissez l'agrégation et que vous pensez que le Frido y est pour quelque chose, n'hésitez pas à donner si votre situation vous le permet.

\vfill

J'accepte les donations.

% Moralement, vous devriez considérer ce fichier comme une section invariante
% au sens de la licence FDL.
% Autrement dit, je ne serais pas content que ce fichier soit modifié.

\begin{description}
\item[IBAN] FR76 3000 4004 0600 0035 2497 784
\item[BIC] BNPAFRPPBSC
\end{description}

