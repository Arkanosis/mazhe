% This is part of Exercices de mathématique pour SVT
% Copyright (C) 2010
%   Laurent Claessens et Carlotta Donadello
% See the file fdl-1.3.txt for copying conditions.

\begin{corrige}{TD3-0002}

	\begin{enumerate}
		\item
			Pour $u_0$, nous avons $u_0=x=xa^0$; pour rappel, $a^0=1$ pour tout $a$.

			Supposons que la formule soit vraie pour $u_k$, c'est à dire que nous avons $u_k=xa^k$ pour un certain $k$. Dans ce cas, nous avons
			\begin{equation}
				u_{k+1}=au_{k}=axa^k=xa^{k+1}
			\end{equation}
			parce que $aa^k=a^{k+1}$. Donc la formule proposée est également correcte pour $u_{k+1}$. Par récurrence, elle est donc correcte pour tous les $k$.
		\item
			Nous avons
			\begin{equation}
				u_n=\frac{ 1000 }{ 2^n }.
			\end{equation}
			Il faut donc savoir à partir de quel $n$ nous avons $2^n\geq 1000$. La réponse est $n=10$ ($2^{10}=1024$).
		\item
			Dans le cas présent, $u_n=22^n=2^{n+1}$. Cette suite dépasse $1000$ pour $n+1=10$, c'est à dire $n=9$.
	\end{enumerate}

\end{corrige}
