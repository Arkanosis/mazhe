% This is part of Exercices et corrigés de CdI-1
% Copyright (c) 2011
%   Laurent Claessens
% See the file fdl-1.3.txt for copying conditions.

\begin{corrige}{Devel0001}

\begin{enumerate}

\item
Pour la première vague, nous avons
\begin{enumerate}

\item
$f(x)=x^4$ fait l'affaire parce que $\frac{ x^4 }{ x^3 }\to 0$

\item
Sous-entendu, nous demandons une fonction dans $o\big( | x-1 | \big)$ pour $x\to 1$. La fonction $f(x)=(x-1)^2$ fonctionne.

\item
La fonction $f(x)=x\sin(x)$.

\end{enumerate}

\item
Pour la deuxième vague,
\begin{enumerate}

\item
Nous avons
\begin{equation}
	\lim_{x\to 0} \frac{ xf(x) }{ x^2 }=\lim_{x\to 0} \frac{ f(x) }{ x }=0, 
\end{equation}
donc VRAI.

\item
Nous avons $\lim_{x\to 0} f(x)/x^2\neq 0$, mais $x^2\in o(x)$, donc FAUX.

\item
En utilisant l'hypothèse $\lim_{x\to 0} f(x)/x^3$, nous trouvons
\begin{equation}
	\begin{aligned}[]
		\lim_{x\to 0} \frac{ f(x) }{ x^2 }&=\lim_{x\to 0} x\frac{ f(x) }{ x }=0\\
		\lim_{x\to 0} \frac{ f(x) }{ x }&=\lim_{x\to 0} x^2\frac{ f(x) }{ x^3 }=0\\
		\lim_{x\to 0} \frac{ f(x) }{ 1 }&=\lim_{x\to 0} x^3\frac{ f(x) }{ x^3 }=0.
\end{aligned}
\end{equation}
\end{enumerate}
\end{enumerate}
\end{corrige}
