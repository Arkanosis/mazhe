%++++++++++++++++++++++++++++++++++++++++++++++++++++++++++++++++++++++++++++++++++++++++++++++++++++++++++++++++++++++++++++
\section{Directed sets and net}

A \defe{directed set}{directed set} is a pre-ordered set (i.e. a set with a reflexive and transitive binary relation) such that every pair of elements has an upper bound. As a consequence, when $a_1,\ldots a_n$ are elements of the directed set $A$, then there exists a $a\in A$ such that $a\geq a_i$.

A \defe{net}{net} is a map $A\to X$ from a directed set to a topological space. We denote by $x_{\alpha}$ the element of $X$ which corresponds to $\alpha\in A$.

As example, if $S$ is any set, the set $A$ of finite subsets of $S$ with the inclusion is an example of net.

There is a notion of \defe{convergence}{convergence!of a net} of net. We say that the net $\alpha\mapsto x_\alpha$ converges to $x$ and we write $x_{\alpha}\to x$ if and only if for every open set $\mU\subseteq X$ containing $x$, there exists a $\alpha\in A$ such that $\alpha'\geq \alpha$ implies $x_{\alpha'}\in\mU$.

So a topology implies a convergence notion for nets, as well as for sequences. However, there exists different topologies which have the same notion of convergence of sequences, but two topologies having the same notion of convergence of nets are the same.


%+++++++++++++++++++++++++++++++++++++++++++++++++++++++++++++++++++++++++++++++++++++++++++++++++++++++++++++++++++++++++++
\section{Homotopy group}

Let $X$ be a topological space with a base point $b$, and $S^n$ be the $n$-sphere. The $n$th \defe{group of homotopy}{group!homotopy}\index{homotopy group} on the point $b$ of $X$ is\nomenclature{$\pi_n(X,b)$}{Homotopy group of $X$ on the base point $b$}
\begin{equation}
\pi_n(X,b)=\{ \text{homotopy classes of maps $f\colon S^n\to X$ such that $f(a)=b$} \}.
\end{equation}
The classes are taken up to homotopy, i.e. continuous deformations. In an equivalent way, $\pi_n(X,b)$ can be seen as the set of classes of maps $p\colon [0,1]^n\to X$ which sent the whole border of the cube to $b$.

%+++++++++++++++++++++++++++++++++++++++++++++++++++++++++++++++++++++++++++++++++++++++++++++++++++++++++++++++++++++++++++
					\section{Covering spaces}
%+++++++++++++++++++++++++++++++++++++++++++++++++++++++++++++++++++++++++++++++++++++++++++++++++++++++++++++++++++++++++++

Let $X$ be a topological space. A topological space $C$ is a \defe{covering}{covering!of a topological space} of $X$ if we have a continuous surjective map $\rho\colon C\to X$ such that $\forall x\in X$, there exists an open neighbourhood $\mU$ such that $\rho^{-1}(\mU)$ is an union of disjoint open sets $S_i$ on which the restricted map $\rho|_{S_i}\colon S_i\to \mU$ is an homeomorphism.

\begin{proposition}[lifting property]\index{lifting property!covering space}
Let $\rho\colon C\to X$ be a covering and $\gamma\colon [0,1]\to X$, a continuous map. Let $c\in \rho^{-1}\big( \gamma(0) \big)$. Then there exists one unique path $\sigma$ in $C$ such that $\sigma\circ\rho=\gamma$ and $\sigma(0)=0$.
\end{proposition}
\begin{proof}
No proof.
\end{proof}
If $x$ and $y$ in $X$ are connected by a path, the lifted path provides a bijection between the fibres $\rho^{-1}(x)$ and $\rho^{-1}(y)$.

%---------------------------------------------------------------------------------------------------------------------------
					\subsection{Universal covering}
%---------------------------------------------------------------------------------------------------------------------------

One says that a covering $q\colon D\to X$ is \defe{universal}{universal!covering}\index{covering!universal} if $D$ is simply connected. The following proposition states that an universal covering is a covering that covers all other coverings.

\begin{proposition}
Let $q\colon D\to X$ be an universal covering, and $\rho\colon C\to X$ be a covering of $X$ with $C$ being connected. Then there exists a covering map $f\colon D\to C$ such that $\rho\circ f=q$.
\end{proposition}
 
The following proposition states that the universal covering is essentially unique.

\begin{proposition}
Let $q_i\colon D_i\to X$ (with $i=1,2$) be two universal coverings of the topological space $X$. Then there exists an homeomorphism $f\colon D_1\to D_2$ such that $q_2\circ f=q_1$.
\end{proposition}
 
%---------------------------------------------------------------------------------------------------------------------------
					\subsection{Monodromy action}
%---------------------------------------------------------------------------------------------------------------------------
\label{sssMonodromyact}

Let $\rho\colon C\to X$ be a covering with $C$ being connected and locally arc connected. First, that shows that $X$ has these two properties too. Now, let $x\in X$ and $c\in\rho^{-1}(x)$ and a path $\gamma\colon [0,1]\to X$ with $\gamma(0)=\gamma(1)=x$. By the lifting property, that path lifts to an unique path in $C$ starting at $c$, while it is not guarantee that the lifted path will \emph{end} at $x$. One only knows that the lifted path will end in $\rho^{-1}(x)$.

It turns out that the end point of the lifted path only depends on the class of $\gamma$ in $\pi(X,x)$. Thus we define an action if $\pi(X,x)$ on the fibre over $x$. This is the \defe{monodromy action}{monodromy action}. Notice that by taking that action pointwise on $X$, the group $\pi(X)$ acts on $C$.

