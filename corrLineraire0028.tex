% This is part of the Exercices et corrigés de mathématique générale.
% Copyright (C) 2009
%   Laurent Claessens
% See the file fdl-1.3.txt for copying conditions.
\begin{corrige}{Lineraire0028}

	\begin{enumerate}

		\item
			

			Pour voir qu'ils sont orthogonaux : produit scalaire.
			\begin{equation}	
				x\cdot y=\begin{pmatrix}
					1/3	\\ 
					2/3	\\ 
					-2/3	
				\end{pmatrix}\cdot 
				\begin{pmatrix}
					2/3	\\ 
					1/3	\\ 
					2/3	
				\end{pmatrix}=
				\frac{1}{ 9 }+\frac{ 2 }{ 9 }-\frac{ 4 }{ 9 }=0,
			\end{equation}
			donc ils sont orthogonaux.

		\item
			Le produit vectoriel de $x$ et $y$ donne un vecteur perpendiculaire à la fois à $x$ et à $y$:
			\begin{equation}
				x\times y=\begin{vmatrix}
					e_1	&	e_2	&	e_3	\\
					1/3	&	2/3	&	-2/3	\\
					2/3	&	1/3	&	2/3
				\end{vmatrix}=
				2e_1-2e_2-e_3.
			\end{equation}
	\end{enumerate}

\end{corrige}
