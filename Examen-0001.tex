\begin{exercice}\label{exoExamen-0001}
  \begin{enumerate}
  \item Tracer le graphe de la fonction $f(x)=x$ pour $x$ dans l'intervalle $[0,5]$.
    \item Calculer l'intégrale $\displaystyle \int_{0}^{5} f(x)\, dx$, qui correspond à l'aire de la région entre l'axe des $x$ et le graphe de $f$.
      \item Tracer le graphe de la fonction \emph{partie entière} $x\mapsto [x]$, définie par $[x]=x-(\textrm{le nombre entier le plus grand qui est plus petit de } x)$. Exemple : $[5,67]=5$, $[2]=2$, $[0,34]=0$.
        \item Calculer l'intégrale $\displaystyle \int_{0}^{5} [x]\, dx$. Conseil : écrire cette intégrale comme la somme de $5$ intégrales $\int_{0}^{1}\ldots+ \ldots +\int_{4}^{5}\ldots$. 
          \item Tracer le graphe de la fonction \emph{mantisse}, $m(x)=x-[x]$, pour $x$ entre $0$ et $5$.
          \item Calculer l'intégrale entre $0$ et $5$ de la fonction \emph{mantisse}, $m(x)=x-[x]$. Vérifier graphiquement que l'intégrale que vous venez de calculer est égale à l'aire entre les graphes de $f(x)=x$ et de $g(x)=[x]$.
        \begin{equation}
          
        \end{equation}
  \end{enumerate}
\end{exercice}
