% This is part of the Exercices et corrigés de CdI-2.
% Copyright (C) 2008, 2009
%   Laurent Claessens
% See the file fdl-1.3.txt for copying conditions.


\begin{corrige}{_I-3-4}

Nous allons utiliser le critère de Weierstrass, théorème \ref{ThoCritWeiIntUnifCv}.

\begin{enumerate}
\item
	 La fonction $t\mapsto\frac{ 10 }{ t^2 }$ majore $| \sin(xt)/t^2 |$, or l'intégrale $\int_0^{\infty}\frac{ 10 }{ t^2 } dt$ existe.
\item 

	Le théorème \ref{ThoInDerrtCvUnifFContinue} affirme que si l'intégrale était uniformément convergente, alors la fonction
	\begin{equation}
		I(x)=\int_0^{\infty}x e^{-xt}dt
	\end{equation}
	serait continue. Nous allons vérifier cela. D'abord, si $x=0$, nous avons évidement $I(0)=0$. Si $x\neq 0$, alors
	\begin{equation}
		I(x)=\lim_{T\to\infty}\int_0^{T}x e^{-xt}dt=\lim_{T\to\infty}\big( - e^{-xT}+1 \big)=1.
	\end{equation}
	Cette fonction $I$ n'est donc pas continue en $0$ et il ne peut donc pas y avoir de convergence uniforme sur l'intervalle $[0,1]$.

	Ce résultat peut également être vu directement sur la définition. Pour avoir uniforme convergence, il faudrait que $\forall\epsilon>0$ $\exists T_0$ (indépendant de $x$) tel que $T\geq T_0$ implique
	\begin{equation}
		\left| \int_T^{\infty} x e^{-xt}dt\right|<\epsilon
	\end{equation}
pour tout $x$. Or, nous savons que
\begin{equation}
	 \int_T^{\infty} x e^{-xt}dt=
		\begin{cases}
	 e^{-xT}	&	\text{si }x\neq 0\\
	0	&	 \text{si }x=0,
\end{cases}
\end{equation}
donc $\sup_{x\in[0,1]}\int_01^{\infty}x e^{-xt}dt=1$, et un $T_0$ convenable ne peut pas être trouvé.

	Nous avons toutefois convergence uniforme sur tout compact de $]0,1]$ parce que si $x_0$ est le minimum du compact, alors $T_0=-\ln(\epsilon)/x_{0}$ fonctionne.

\item 
	Nous pouvons majorer la norme de $\ln(xt)$ de façon indépendante de $x$ de la façon suivante :
	\begin{equation}
		| \ln(xt) |=| \ln(x)+\ln(t) |\leq| \ln(x) |+| \ln(t) |\leq\ln(3)-\ln(t)=\ln(3/t)
	\end{equation}
	où nous avons utilisé le fait que $|\ln(t)|=-\ln(t)$ dans l'intervalle considéré. Maintenant, pour $t\in[0,1]$, nous avons $\ln(3/t)<3/t$, et au final, nous avons
	\begin{equation}
		| \ln(xt) |^{1/3}<\frac{ 3^{1/3} }{ t^{1/3} },
	\end{equation}
	tandis que
	\begin{equation}
		\int_0^1\frac{ 3^{1/3} }{ t^{1/3} }
	\end{equation}
	existe.

\item
	La norme $| \sin(xt)/(1+t^2) |$ se majore par $1/(1+t^2)$ dont l'intégrale existe.
\item
	L'intégrale n'existe pas en $x=0$ parce que l'intégrale de $t/(1+t^2)$ ne converge pas. Notez la différence avec l'exercice précédent où l'absence d'un $t$ au numérateur faisait converger l'intégrale. En ce qui concerne le cas $x\neq 0$, nous utilisons le critère d'Abel avec $\varphi(x,t)=\cos(xt)$ et $\psi(x,t)=1/(1+t^2)$.

Nous avons
\begin{equation}
	|\int_0^T\cos(xt)dt |=\frac{1}{ | x | }\sin(xT)\leq \frac{1}{ | x | }.
\end{equation}
Si $x$ est restreint à un compact de ne contenant pas $0$, alors cette quantité peut être bornée uniformément en $x$ par un certain nombre $M$. D'autre part, nous avons que $\lim_{t\to\infty}\psi(x,t)=0$ de façon uniforme en $x$ (parce que $x$ n'apparaît même pas dans la fonction).

Le critère d'Abel conclut à l'uniforme convergence sur tout compact ne contenant pas $0$.

\item 
Si nous posons 
\begin{equation}
	F(x)=\int_0^{\infty}\frac{ \sin(xt) }{ t }dt,
\end{equation}
nous trouvons $F(0)=0$ et, via le changement de variable $u=xt$ : $F(x)=F(1)$ quand $x>0$, et $F(x)=F(-1)$ pour $x<0$. Donc, si nous voulons que l'intégrale converge uniformément, il faut que $F(1)=F(-1)=0$. Nous verrons cependant que
\begin{equation}		\label{EqIntSinSurt}
	\int_0^{\infty}\frac{ \sin(t) }{ t }dt=\frac{ \pi }{ 2 }
\end{equation}
dans lemme 2, page III.21 (attention : preuve difficile). L'intégrale ne convergera donc uniformément sur aucun intervalle contenant zéro.

Sans utiliser le résultat \eqref{EqIntSinSurt}, nous pouvons utiliser le critère d'Abel avec
\begin{equation}
	\begin{aligned}[]
		\varphi(x,t)	&=\sin(xt),	&	\psi(x,t)	&=\frac{1}{ t }.
	\end{aligned}
\end{equation}
Nous obtenons facilement que
\begin{equation}
	|\int_0^T\varphi(x,t)dt |\leq\frac{1}{ | x | }M
\end{equation}
si $M$ est choisit assez grand. Cette quantité peut être majorée de façon uniforme par rapport à $x$ quand $x$ est restreint à un compact ne contenant pas zéro. Le critère d'Abel fournit donc la convergence uniforme sur tout compact ne contenant pas zéro.

\end{enumerate}

\end{corrige}
