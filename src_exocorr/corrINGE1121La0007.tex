% This is part of the Exercices et corrigés de mathématique générale.
% Copyright (C) 2009-2010
%   Laurent Claessens
% See the file fdl-1.3.txt for copying conditions.


\begin{corrige}{INGE1121La0007}

	Pour calculer le déterminant, il faut échelonner la matrice de façon usuelle et faire quelques mises en évidence en utilisant des produits remarquables :
	\begin{equation}
		\begin{aligned}[]
			\det\begin{pmatrix}
				1	&	a	&	a^2	\\
				1	&	b	&	b^2	\\
				1	&	c	&	c^2
			\end{pmatrix}&=
			\det\begin{pmatrix}
				1	&	a	&	a	\\
				0	&	b-a	&	a^2-b^2	\\
				0	&	c-a	&	c^2-a^2
			\end{pmatrix}\\
			&=
			(b-a)(c-a)\det\begin{pmatrix}
				1	&	a	&	a^2	\\
				0	&	1	&	-(a+b)	\\
				0	&	1	&	-(c+b)
			\end{pmatrix}\\
			&=
			(b-a)(c-a)\det\begin{pmatrix}
				1	&	a	&	a^2	\\
				0	&	1	&	-(a+b)	\\
				0	&	0	&	-c+a
			\end{pmatrix}\\
			&=
			(b-a)(c-a)(a-c)\det\begin{pmatrix}
				1	&	a	&	a^2	\\
				0	&	1	&	-(a+b)	\\
				0	&	0	&	1
			\end{pmatrix}
		\end{aligned}
	\end{equation}
	Le déterminant de la dernière matrice est $1$ parce que le déterminant d'une matrice triangulaire supérieure est égale au produit de ses éléments diagonaux.

\end{corrige}
