% This is part of the Exercices et corrigés de mathématique générale.
% Copyright (C) 2009
%   Laurent Claessens
% See the file fdl-1.3.txt for copying conditions.
\begin{corrige}{Lineraire0018}

	En général, il faut trouver les coefficients $\lambda_1$, $\lambda_2$ et $\lambda_3$ tels que
	\begin{equation}
		\lambda_1\begin{pmatrix}
			1	\\ 
			1	\\ 
			1	
		\end{pmatrix}+
		\lambda_2\begin{pmatrix}
			3	\\ 
			5	\\ 
			2	
		\end{pmatrix}+
		\lambda_3\begin{pmatrix}
			1	\\ 
			3	\\ 
			0	
		\end{pmatrix}
		=
		\begin{pmatrix}
			a	\\ 
			b	\\ 
			c	
		\end{pmatrix}.
	\end{equation}
	Cela revient à résoudre le système
	\begin{equation}
		\left\{
		\begin{array}{ll}
			\lambda_1+3\lambda_2+\lambda_3=a\\
			\lambda_1+5\lambda_2+3\lambda_3=b\\
			\lambda_1+2\lambda_2=c
		\end{array}
		\right.
	\end{equation}
	Nous résolvons cela en termes de matrices:
	\begin{equation}
		\left(\begin{array}{ccc|c}
			 1	&	3	&	1	&	a	\\
			  1	&	5	&	3	&	b\\
			   1	&	2	&	0	&	c	 
		   \end{array}\right)
		   \sim
		   \left(\begin{array}{ccc|c}
			    0	&	1	&	1	&	a-c	\\
			     0	&	2	&	2	&	b-a\\
			      1	&	2	&	0	&	c	 
		      \end{array}\right)
	\end{equation}
	Au niveau des coefficients, la seconde ligne est juste le double de la première, donc pour avoir une solution, il faut absolument que le terme indépendant de la seconde soit également la moitié : $b-a=2(a-c)$. Nous pouvons continuer et tomber sur la matrice suivante :
	\begin{equation}
		\left(\begin{array}{ccc|c}
			 0	&	0	&	0	&	3a-b-2c	\\
			  0	&	2	&	2	&	b-a\\
			   1	&	2	&	0	&	c	 
		   \end{array}\right).
	\end{equation}
	Lorsqu'on regarde le vecteur
	\begin{equation}
		\begin{pmatrix}
			a	\\ 
			b	\\ 
			c	
		\end{pmatrix}
		=
		\begin{pmatrix}
			1	\\ 
			2	\\ 
			3	
		\end{pmatrix},
	\end{equation}
	on a que la condition $3a-b-2c=3-2-6\neq 0$, donc ce vecteur ne peut pas être écrit comme combinaison des trois vecteurs donnés. Avec l'autre, par contre, ça fonctionne mieux :
	\begin{equation}
		\begin{pmatrix}
			a	\\ 
			b	\\ 
			c	
		\end{pmatrix}=
		\begin{pmatrix}
			0	\\ 
			-2	\\ 
			1	
		\end{pmatrix},
	\end{equation}
	donne le système
	\begin{equation}
		\left\{
		\begin{array}{ll}
			\lambda_1+2\lambda_2=1\\
			2 \lambda_2+2\lambda_3=-2.
		\end{array}
		\right.
	\end{equation}
	Comme ce sont deux équations avec 3 inconnues, il n'y a pas d'espoir de trouver une seule solution. Nous résolvons donc par rapport à $\lambda_1$ et $\lambda_3$ en laissant $\lambda_2$ comme paramètre. Nous trouvons $\lambda_1=1-2\lambda_2$ et $\lambda_3=-1-\lambda_2$.

	Pour chaque $\lambda_2$, cela donne un triple $(\lambda_1,\lambda_2,\lambda_3)$ tel que
	\begin{equation}
		\lambda_1\begin{pmatrix}
			1	\\ 
			1	\\ 
			1	
		\end{pmatrix}
		+\lambda_2\begin{pmatrix}
			3	\\ 
			5	\\ 
			2	
		\end{pmatrix}+
		\lambda_3
		\begin{pmatrix}
			1	\\ 
			3	\\ 
			0	
		\end{pmatrix}
		=
		\begin{pmatrix}
			0	\\ 
			-2	\\ 
			1	
		\end{pmatrix}.
	\end{equation}
	Prenez par exemple, $\lambda_2=0$, on trouve $\lambda_1=1$ et $\lambda_3=-1$.

\end{corrige}
