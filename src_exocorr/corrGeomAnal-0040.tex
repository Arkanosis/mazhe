\begin{corrige}{GeomAnal-0040}

    Nous allons montrer que les ensembles sur lesquels ont prend le supremum sont en réalité les mêmes :
    \begin{equation}
        \underbrace{\left\{ \frac{ \| Ax \|_p }{ \| x \|_p\tq x\neq 0 } \right\}}_{A}=\underbrace{\left\{ \| Ax \|_p\tq \| x \|_p=1 \right\}}_{B}.
    \end{equation}
    Attention : ce sont des sous-ensembles de réels; pas de sous-ensembles de \( \eM(\eR)\) ou des sous-ensembles de \( \eR^n\).

    Pour la première inclusion, prenons un élément de \( A\), et prouvons qu'il est dans \( B\). C'est à dire que nous prenons \( x\in\eR^n\) et nous considérons le nombre \( \| Ax \|_p/\| x \|_p\). Le vecteur \( y=x/\| x \|\) est un vecteur de norme $1$, donc la norme de \( Ay\) est un élément de \( B\), mais
    \begin{equation}
        \| Ay \|_p=\frac{ \| Ax \|_p }{ \| x \|_p }.
    \end{equation}
    Nous avons donc \( A\subset B\).

    L'inclusion \( B\subset A\) est immédiate.

\end{corrige}
