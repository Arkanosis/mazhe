% This is part of the Exercices et corrigés de CdI-2.
% Copyright (C) 2008,2009,2015
%   Laurent Claessens
% See the file fdl-1.3.txt for copying conditions.


\begin{corrige}{_I-2-2}

\begin{enumerate}
\item Si $\beta>1$, nous devons étudier l'intégrale $\int_1^{\infty}\left| \frac{ \sin(x) }{ x^{\beta} }  \right|dx$. La fonction $g(x)=2/x^{\beta}$ majore l'intégrante alors que l'intégrale $\int_1^{\infty}g(x)dx$ converge.

\item Si $\beta<0$, calculons la différence entre deux termes successifs de la suite
\begin{equation}		\label{EqInACaclDiff}
	I_n=\int_1^{n\pi}x^{-\beta}\sin(x)dx.
\end{equation}
Selon la définition \eqref{EqDEfConvergeZeroInftX} de la convergence d'une intégrale et le fait qu'une suite numérique ne peut converger que si elle est de Cauchy, pour que $I$ converge, la différence $I_{n+1}-I_n$ doit tendre vers zéro lorsque $n$ tend vers l'infini. (voir le théorème de la page 46 de première année : dans $\eR$ ou $\eR^n$, une suite est convergente si et seulement si elle est de Cauchy) Supposons pour simplifier que $n$ est pair (donc $\sin(x)$ est positive)
\begin{equation}
	I_n-I_{n+1}=\int_{n\pi}^{(n+1)\pi}\frac{ \sin(x) }{ x^{\beta} }dx>(n\pi)^{-\beta}\int_{n\pi}^{(n+1)\pi}\sin(x)dx=2(n\pi)^{-\beta}.
\end{equation}
Cela ne converge pas vers zéro quand $n\to\infty$. Donc l'intégrale $I$ ne converge pas quand $\beta\leq 0$. Elle n'existe donc pas non plus.


\end{enumerate}
Pour résumer, nous avons toutes les situations possibles :
\begin{itemize}
\item intégrale qui existe (et donc converge),
\item intégrale qui n'existe pas et qui converge\footnote{La terminologie \og existe\fg{} provient du cas de dimension plus que un, et n'est effectivement pas très heureuse ici.}
\item intégrale qui ne converge pas (et qui n'existe donc pas).
\end{itemize}

\end{corrige}
