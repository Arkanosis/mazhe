% This is part of Analyse Starter CTU
% Copyright (c) 2014
%   Laurent Claessens,Carlotta Donadello
% See the file fdl-1.3.txt for copying conditions.

\begin{corrige}{analyseCTU-0101}

   \begin{enumerate}
      \item Pour tout $y \in [-1,1]$ nous avons 
        \begin{equation*}
          \left[\cos (\arcsin(y)) \right]^2 = 1 -  \left[\sin (\arcsin(y)) \right]^2 = 1- y^2,
        \end{equation*}
et analoguement 
 \begin{equation*}
          \left[ \sin(\arccos(y))\right]^2 = 1 -  \left[\cos(\arccos(y))\right]^2 = 1- y^2.
        \end{equation*}
 Les deux expression sont donc \'equivalentes. 
        \item
            \begin{enumerate}
            \item L'ensemble de définition $\arctan$ est $\eR$, mais l'ensemble de  définition $1/x$ est $\eR\setminus\{0\}$. Par consequent l'ensemble de définition de $f$ est $\eR\setminus\{0\}$.
              \item Nous pouvons calculer la dérivée de $f$, on a  
                \begin{equation*}
                  f'(y) = \frac{1}{1+y^2} + \frac{1}{1+\left(\frac{1}{y}\right)^2}\frac{-1}{y^2} = \frac{1}{1+y^2} - \frac{1}{1+y^2} = 0,
                \end{equation*}
pour tout $y$ dans le domaine de $f$.
              \item On peut calculer $f(-1)$, en sachant que sa valeur est la valeur de $f$ pour tout \( y<0\),
            \begin{equation}
                \arctan(-1)+\arctan(\frac{1}{ -1 })= 2 \left(-\frac{ \pi }{ 4 }\right) = -\frac{ \pi }{ 2 }.
            \end{equation}
          \item La fonction $f$ n'est pas constante sur son domaine, car elle prend des valeurs diff\'erents dans les intervalles $]0,+\infty[$ et $]-\infty, 0[$. Pour $y>0$ on a en effet $f(y) = f(1) = \pi/2$.  
            \end{enumerate}
    \end{enumerate}

\end{corrige}
