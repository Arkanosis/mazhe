% This is part of Exercices de mathématique pour SVT
% Copyright (C) 2010
%   Laurent Claessens et Carlotta Donadello
% See the file fdl-1.3.txt for copying conditions.

\begin{corrige}{DS2010bis-0002}

	Chaque fonction a sa particularité qu'il faut reconnaître.
	\begin{enumerate}
		\item
			La fonction $\cos(x)$ elle-même est celle qui vaut $1$ en $0$, qui s'annule en $\frac{ \pi }{2}$ et qui oscille. C'est donc le graphe \ref{LabelFigExercicebissscosinus}.
		\item
			La fonction $\cos(x+\frac{ \pi }{2})$ est la même que la fonction cosinus, mais décalée de $\frac{ \pi }{2}$ vers la gauche. C'est le graphe  \ref{LabelFigExercicebissssinus}.
		\item
			La fonction $\cos( e^{x})$ est une fonction qui oscille de plus en plus vite parce que ce qui se trouve dans le cosinus (c'est à dire  $ e^{x}$) monte de plus en plus vite. Le graphe qui correspond est \ref{LabelFigExercicebissscosex}.
		\item
			Le graphe de la fonction $\cos(x)+1$ est le même que celui de $\cos(x)$, mais décalé de $1$ vers le haut. C'est le graphe \ref{LabelFigExercicebissscosxplusun}.
		\item
			La fonction $\cos(4x)$ oscille quatre fois plus vite que le cosinus (parce que $4x$ avance $4$ fois plus vite que $x$). C'est donc le graphe \ref{LabelFigExercicebissscosquattrox}.
		\item
			La fonction $| \cos(x) |$ est la même que $\cos(x)$ sauf que partout où $\cos(x)$ est négatif, il devient positif. C'est le graphe \ref{LabelFigExercicebisssvalabsoluecosinus}
		\item
			La fonction $\sqrt{\cos(x)}$ est reconnaissable au fait qu'elle n'est pas définie là où le cosinus est négatif. Il y a donc des «trous» dans son domaine. C'est le graphe \ref{LabelFigExercicebissssqrtcos}
	\end{enumerate}
	

\end{corrige}
