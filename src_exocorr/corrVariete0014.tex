% This is part of Exercices et corrigés de CdI-1
% Copyright (c) 2011
%   Laurent Claessens
% See the file fdl-1.3.txt for copying conditions.

\begin{corrige}{Variete0014}

 Soit $D$ ouvert borné dans l'espace, dont le bord est une
  variété de dimension $2$ ayant un champ de vecteurs unitaire normal
  $\nu$. Soit $G$ un champ de vecteurs constant et $c : \partial D
  \to \eR$ la fonction définie sur $\partial D$ qui associe au point
  $v\in\partial D$ le cosinus de l'angle entre $G(v)$ et $\nu(v)$.

  Par définition du produit scalaire, on sait que 
  \begin{equation*}
    c(v) = \frac{\scalprod{G(v)}{\nu(v)}}{\norme{G(v)}\norme{\nu(v)}} = \frac{\scalprod{G(v)}{\nu(v)}}{\norme{G(v)}}
  \end{equation*}
  où $\norme{G(v)}$ est une constante (puisque le champ $G$ est
  constant). Dès lors nous avons
  \begin{equation*}
    \iint_{\partial D} c(v) = \frac1{\norme{G}} \iint_{\partial D}
    \scalprod{G(v)}{\nu(v)} 
  \end{equation*}
  où la dernière intégrale est, par définition, le flux de $G$ au
  travers de $\partial D$. Comme $G$ est constant, sa divergence vaut
  $0$, et l'intégrale est donc nulle par le théorème de la divergence.


\end{corrige}
