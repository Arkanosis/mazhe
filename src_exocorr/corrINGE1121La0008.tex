% This is part of the Exercices et corrigés de mathématique générale.
% Copyright (C) 2009-2010
%   Laurent Claessens
% See the file fdl-1.3.txt for copying conditions.


\begin{corrige}{INGE1121La0008}

	Lorsque $n=1$, la proposition est évidente. Supposons maintenant que la proposition soit vraie pour un certain entier $k$, et prouvons qu'alors c'est vrai pour $k+1$.

	Nous devons calculer
	\begin{equation}		\label{EqRecur18INGE}
		A^{k+1}=A^kA=\begin{pmatrix}
			1	&	k	\\ 
			0	&	1	
		\end{pmatrix}
		\begin{pmatrix}
			1	&	1	\\ 
			0	&	1	
		\end{pmatrix}.
	\end{equation}
	Pour écrire cela, nous avons utilisé l'hypothèse de récurrence 
	\begin{equation}
		A^k=\begin{pmatrix}
			1	&	k	\\ 
			0	&	1	
		\end{pmatrix}.
	\end{equation}
	En effectuant le produit matriciel dans \eqref{EqRecur18INGE}, nous trouvons
	\begin{equation}
		A^{k+1}=\begin{pmatrix}
			1	&	k+1	\\ 
			0	&	1	
		\end{pmatrix},
	\end{equation}
	ce qui est bien ce que nous voulions.

\end{corrige}
