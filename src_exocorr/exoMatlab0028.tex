\begin{exercice}\label{exoMatlab0028}

Un restaurateur aimerait connaître \emph{le nombre de cuivre}, c'est-à-dire l'épaisseur idéale d'une tranche de viande pour ses gyros. Pour ce faire, il a effectué des tests de découpe à différentes épaisseurs et les a fait goûter à un échantillon de 3 personnes représentatives de la société belge.
Après avoir obtenu ses données et effectué de long calculs, il est arrivé au fait que le nombre de cuivre devait être égal au déterminant de la matrice
\[ A= (v^t.v)^2 + 42(v^t.v) - 2I \]
où le vecteur $v$ contient les épaisseurs allant de $0.7$ à $8.76$ millimètres en exactement 13 valeurs extrémités comprises, et où
$I$ est la matrice identité de genre $13\times 13$.

Bonne âme que vous êtes, vous décidez de l'aider, et vous
\begin{enumerate}
\item construisez en une opération le vecteur $v$ ;
% allant de $3.7$ à $8.76$ en exactement 13 valeurs.
\item construisez la matrice $A$ en un tournemain ;
%=(a_{ij})$ de genre $13\times 13$, définie par
%\[ A= (v^t.v)^2 - 2I \]
%où $I$ est la matrice identité de genre $13\times 13$.
\item donnez fièrement la valeur du nombre de cuivre \footnote{et vous rendez compte qu'il s'est royalement planté dans ses calculs et que vous ne connaîtrez jamais la valeur du nombre de cuivre.}.
\end{enumerate}

\corrref{Matlab0028}
\end{exercice}
