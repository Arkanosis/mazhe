\begin{corrige}{controlecontinu0007}

La méthode la plus rapide pour compléter l'exercice est de calculer d'abord  les matrices jacobiennes de $f$ et $g$, ou pour mieux dire, le gradient de $f$ et le jacobien de $g$. 

\begin{equation}
  \nabla f(x,y) = 
  \begin{pmatrix}
    \ln(yx^2)+ 2 & \frac{x}{y}
  \end{pmatrix},
\end{equation}
\begin{equation}
  \frac{d }{dt} g(t)=
  \begin{pmatrix}
    1\\ e^t
  \end{pmatrix}. 
\end{equation}

La fonction $F=g\circ f$ est une fonction de $\mathbb{R}^2$ dans $\mathbb{R}^2$. Sa matrice jacobienne doit alors être une matrice $2\times 2$. On la trouve par la formule

\begin{equation}
  \begin{aligned}
    &\nabla F(\pi/2, 1)= \nabla g(f(\pi/2, 1))\cdot \nabla f(\pi/2, 1) =  \\
    &\begin{pmatrix}
      1\\ \left(\frac{\pi}{2}\right)^\pi
    \end{pmatrix}\cdot \begin{pmatrix}
    \ln(\pi^2/4)+ 2 & \frac{\pi}{2}
  \end{pmatrix}=\\
    &
    \begin{pmatrix}
      \ln(\pi^2/4)+ 2 & \frac{\pi}{2}\\
     \left(\frac{\pi}{2}\right)^\pi (\ln(\pi^2/4)+ 2) & \left(\frac{\pi}{2}\right)^{\pi+1}
    \end{pmatrix}
  \end{aligned}
\end{equation}

\end{corrige}


