% This is part of Exercices et corrigés de CdI-1
% Copyright (c) 2011
%   Laurent Claessens
% See the file fdl-1.3.txt for copying conditions.

\begin{corrige}{OutilsMath-0084}

    En coordonnées cartésiennes le champ donné est
    \begin{equation}
        F(x,y,z)=\frac{1}{ r^2 }(-ye_x+xe_y+4e_z).
    \end{equation}
    Étant donné que nous avons vu que $e_{\theta}=\frac{1}{ r }(-ye_x+xe_y)$, nous avons
    \begin{equation}
        F(r,\theta,z)=\frac{1}{ r^2 }(re_{\theta}+4e_z).
    \end{equation}
    Nous avons donc $F_r=0$, $F_{\theta}=1/r$ et $F_z=4/r^2$. En utilisant la formule nous trouvons
    \begin{equation}
        \nabla\cdot F=0.
    \end{equation}
    Cela est logique étant donné que ce champ de vecteurs décrit un mouvement qui est la superposition d'un écoulement à vitesse constante le long de l'axe $z$ et d'une rotation autour de cet axe. Les écoulements à vitesse constante et les rotations sont des mouvements de fluides incompressibles.

\end{corrige}
