% This is part of Exercices et corrigés de CdI-1
% Copyright (c) 2011
%   Laurent Claessens
% See the file fdl-1.3.txt for copying conditions.

\begin{corrige}{0033}

\begin{enumerate}
\item Notons
  \begin{equation*}
    p(x) = a_0 + a_1 x + a_2 x^2 + a_3 x^3 + \cdots + a_n x^n
  \end{equation*}
  avec $a_n \neq 0$, de sorte que (en appliquant les règles de
  calculs)
  \begin{equation*}
    \limite[x > 0] x 0 \frac{p(x)}{x^2} =
    \begin{cases}
        \pm\infty & \text{si }a_0 \text{ ou } a_1 \text{est non-nul}\\
      a_2 & \text{si }a_0 = a_1 = 0
    \end{cases}
  \end{equation*}
	Dans le premier cas, le signe $\pm$ dépend des signes de $a_0$ et $a_1$.
  De la même manière (en mettant le terme de plus haut degré en
  évidence pour pouvoir calculer la limite)
  \begin{equation*}
    \limite[x > 0] x {+\infty} \frac{p(x)}{x^2} =
    \begin{cases}
        +\infty & \text{si }n > 2 \text{et} a_n>0\\
        -\infty & \text{si }n > 2 \text{ et } a_n<0\\
      a_2 & \text{si }n = 2
    \end{cases}
  \end{equation*}
  Ceci montre que la fonction $p(x) \div x$ prendra des valeurs
  arbitrairement grande (en valeur absolue) en s'approchant de $0$ et
  de $+\infty$, sauf si $a_0 = a_1 = 0$ et $n = 2$, c'est-à-dire si
  $p(x) = a_2 x^2$.

\item Par exemple $x^2 {\frac{\sin(x)+ 5}{2}}$.
\end{enumerate}

\end{corrige}
