\begin{exercice}\label{exoSC_serie1-0003}

	Quelques manipulations de matrices.

	\begin{enumerate}

		\item			
			Construire une matrice $A=(a_{ij})$ de genre $6\times 6$, définie par
			\begin{equation}
				A=I+u^tu/4
			\end{equation}
			où $u=(1,2,3,4,5,6)$.
		\item
			Ajouter $2$ à l'élément $a_{23}$ et multiplier par\footnote{Matlab donne-t-il le logarithme en base $e$ ou en base $10$ ?} $\ln(2)$ la deuxième colonne de $A$; on appellera $B$ la nouvelle matrice ainsi obtenue.
		\item
			Calculer la matrice inverse de $B$ et vérifier que le produit $BB^{-1}$ donne (approximativement) l'identité.
		\item
			Résoudre le système $Ax=u^t$ et vérifier que la colonne $x$ obtenue est bien solution.
	\end{enumerate}
	

\corrref{SC_serie1-0003}
\end{exercice}
