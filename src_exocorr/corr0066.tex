% This is part of Exercices et corrigés de CdI-1
% Copyright (c) 2011
%   Laurent Claessens
% See the file fdl-1.3.txt for copying conditions.

\begin{corrige}{0066}

\begin{enumerate}

\item 
$\sum_k(1/3^k)$.
C'est une série de puissances avec $q=1/3$. Étant donné que $| q |<1$, nous avons convergence absolue.

\item
$\sum_k(1/\ln(k)^k)$. Cette somme est toute désignée pour faire fonctionner le critère de la racine. Nous avons
\begin{equation}
	\sqrt[k]{\frac{1}{ \ln(k)^k }}=\frac{1}{ \ln(k) },
\end{equation}
 et bien entendu, la limite supérieure de cette suite est zéro. Donc, convergence.

\begin{alternative}
	Nous pouvons aussi dire que
\begin{equation}
	\frac{1}{ \ln(k)^k }<\frac{1}{ \ln(3)^k },
\end{equation}
et maintenant nous pouvons comparer avec une série de puissance de raison $1/\ln(3)$.

Remarquez que la majoration $1/\ln(k)^k<1/\ln(2)^k$ n'est pas suffisante parce que $\ln(2)<1$. En effet, le logarithme est croissant, et $\ln(e)=1$, alors que $e>2$ en vertu de l'exercice \ref{exo0020}\ref{Item0020beborne}.
\end{alternative}

\item
$\sum_k(k!/k^k)$.
Une fois n'est pas coutume, nous avons une suite qui croît plus vite que la factorielle. Appliquons le critère du quotient:
\begin{equation}
\frac{ a_{k+1} }{ a_k }=\frac{ (k+)! }{ (k+1)^{k+1} }\frac{ k^k }{ k! }=\frac{ (k+1)k^k }{ (k+1)^{k+1} }=\frac{ k^k }{ (k+1)^k }=\left( \frac{ k }{ k+1 } \right)^k.	
\end{equation}
Le truc est maintenant de voir que cette suite n'est pas loin d'être la suite qui définit le nombre $e$, pour rappel
\begin{equation}
	\left( 1+\frac{1}{ k } \right)^k\to e.
\end{equation}
Nous avons
\begin{equation}
	\left( \frac{ k }{ k+1 } \right)^k=\left( \frac{ k+1 }{ k } \right)^{-k}=\left[ \left( \frac{ k+1 }{ k } \right)^k \right]^{-1}\to 1/e<1.
\end{equation}
Donc la série converge absolument.

\item
$\sum_k(1/k(k+1))$.
Chaque terme de cette série est plus petit que $\frac{1}{ k^2 }$. Or la série des $1/k^2$ converge, donc le critère de comparaison donne la convergence.

\item
$\sum_k(1/(k^2-\cos(k)))$. D'abord, remarquons que tous les termes de cette série sont positif. Ensuite, Nous avons
\begin{equation}
	\frac{1}{ | k^2-\cos(k) | }<\frac{1}{ | k^2-1 | }<\frac{1}{ (k-1)^2 }.
\end{equation}
La dernière inégalité est due au fait que $k^2-1=(k+1)(k-1)$.

\item
$\sum_k(k^3 e^{-3k})$.
Nous faisons le quotient :
\begin{equation}
	\frac{ (k+1)^4 e^{-3(k+1)} }{ k^3 e^{-3k} }= e^{-3}\left( \frac{ k+1 }{ k } \right)^3\to e^{-3}<1,
\end{equation}
donc la série converge.

\item
$\sum_k \big( (-1)^k\ln(k) \big)/k$. C'est une série alternée construite sur une série dont le terme général tend vers zéro. Il y a donc convergence. Il faut voir maintenant si la convergence est absolue ou non. Cela est vite réglé par comparaison :
\begin{equation}
	\sum_{k=1}^{\infty}\frac{ \ln(k) }{ k }>\sum_k\frac{1}{ k },
\end{equation}
qui diverge. Il y a donc convergence simple mais pas absolue.

\item
$\sum_k\big( (-1)^k/k^a \big)$.
Lorsque $a<0$, la série diverge. Cela ne veut pas dire, cependant, qu'elle tend vers $\pm \infty$. Lorsque $a>0$, il y a convergence simple par le critère des séries alternées. Il faut encore étudier dans quel cas il y a convergence absolue. Pour ce faire, nous regardons la série non alternée
\begin{equation}
	\sum_{k=1}^{\infty}\frac{1}{ k^a },
\end{equation}
qui converge si et seulement si $a>1$. En résumé, nous avons
\begin{enumerate}
\item Diverge si $a<0$,
\item Converge simplement mais pas absolument si $0\leq a\leq 1$,
\item converge absolument quand $a>1$.
\end{enumerate}

\item
$\sum_k(\cos(k)+i\sin(k))/k^2$.
La norme du numérateur vaut tout le temps $1$, donc la série converge absolument, et donc simplement.

\item
La série converge absolument quand $a>1$. Si $a\leq 0$, alors la série diverge parce que le terme général ne tend pas vers zéro. Lorsque $0<a\leq 1$, il n'y a pas de convergence absolue parce que nous tombons sur la série de Riemann \eqref{EqSerRiem}. Par contre, le critère d'Abel assure la convergence simple sur cet intervalle. En résumé :
\begin{enumerate}
\item si $a\leq 0$, alors la série diverge,
\item si $0<a\leq 1$, il y a convergence simple,
\item si $a>1$, alors il y a convergence absolue.
\end{enumerate}

\end{enumerate}


\end{corrige}
