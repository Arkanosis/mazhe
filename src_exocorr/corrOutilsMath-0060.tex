% This is part of Exercices et corrigés de CdI-1
% Copyright (c) 2011
%   Laurent Claessens
% See the file fdl-1.3.txt for copying conditions.

\begin{corrige}{OutilsMath-0060}

    La difficulté de cet exercice est de paramétrer le chemin. Cela se fait en quatre morceaux :
    \begin{equation}
        \begin{aligned}[]
            \sigma_1(t)&=(t,0)\\
            \sigma_2(t)&=(1,t)\\
            \sigma_3(t)&=(1-t,0)\\
            \sigma_4(t)&=(0,1-t).
        \end{aligned}
    \end{equation}
    L'intégrale se fait donc en quatre parties. La première est
    \begin{equation}
        \begin{aligned}[]
            I_1&=\int_0^1 F(t,0)\cdot\begin{pmatrix}
                1    \\ 
                0    
            \end{pmatrix}dt\\
            &=\int_0^1\begin{pmatrix}
                t^2    \\ 
                0    
            \end{pmatrix}\cdot\begin{pmatrix}
                1    \\ 
                0    
            \end{pmatrix}dt\\
            &=\int_0^1 t^2dt\\
            &=\frac{1}{ 3 }.
        \end{aligned}
    \end{equation}
    La seconde :
    \begin{equation}
        \begin{aligned}[]
            I_2&=\int_0^1 F(1,t)\cdot\begin{pmatrix}
                0    \\ 
                1    
            \end{pmatrix}dt\\
            &=\int_0^1tdt\\
            &=\frac{1}{ 2 }.
        \end{aligned}
    \end{equation}
    La troisième :
    \begin{equation}
        I_3=\int_0^1F(1-t,0)\cdot\begin{pmatrix}
            -1    \\ 
            0    
        \end{pmatrix}=-\int_0^1(1-t)^2dt=-\frac{1}{ 3 }.
    \end{equation}
    Et la quatrième :
    \begin{equation}
        I_4=\int_0^1 F(0,1-t)\cdot \begin{pmatrix}
            0    \\ 
            -1    
        \end{pmatrix}dt=0.
    \end{equation}
    Au final, l'intégrale cherchée vaut
    \begin{equation}
        \frac{1}{ 3 }+\frac{ 1 }{2}-\frac{1}{ 3 }=\frac{1}{ 2 }.
    \end{equation}

\end{corrige}
