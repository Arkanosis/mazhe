% This is part of the Exercices et corrigés de mathématique générale.
% Copyright (C) 2009
%   Laurent Claessens
% See the file fdl-1.3.txt for copying conditions.
\begin{corrige}{Janvier009}


\begin{enumerate}
\item
La dérivée de la fonction $f$ est donnée par
  \begin{equation*}
    f^\prime(x) = \lim_{h\to 0} \frac{f(x+h) - f(x)}{h}
  \end{equation*}
\item Soit $x \in \eR$, alors
  \begin{equation*}
    \begin{split}
      (fg)^\prime(x) &= \lim_{h\to 0} \frac{(fg)(x+h) -
        (fg)(x)}{h}\\
      &= \lim_{h\to 0} \frac{f(x+h)g(x+h) - f(x) g(x+h) + f(x)
        g(x+h)
        - f(x)g(x)}{h}\\
      &= \lim_{h\to 0} \frac{(f(x+h) - f(x)) g(x+h) + f(x) (g(x+h)
        - g(x))}{h}\\
      &= f^\prime(x)g(x) + f(x) g^\prime(x)
    \end{split}
  \end{equation*}
en ayant utilisé la règle \og la limite du produit est le produit des limites lorsque celles-ci existent\fg.
\end{enumerate}

\end{corrige}
