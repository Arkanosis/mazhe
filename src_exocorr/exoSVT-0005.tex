% This is part of Exercices de mathématique pour SVT
% Copyright (c) 2011
%   Laurent Claessens and Carlotta Donadello
% See the file fdl-1.3.txt for copying conditions.

\begin{exercice}\label{exoSVT-0005}

    Nous vous suggérons de regarder la vidéo

    \url{http://www.youtube.com/watch?v=XYjUUGQhehs}

    \noindent qui explique comment s'obtient l'équation différentielle du circuit électrique \( RC\).

    Le résultat est que si on impose une tension \( V\) aux bornes du circuit comprenant une résistance \( R\) et un condensateur de capacité \( C\), l'équation qui régit la tension \( u_C\) aux bornes du condensateur est
    \begin{equation}    \label{EqnmKooo}
        V=RC\frac{ du_C }{ dt }+u_C.
    \end{equation}
    La charge sur le condensateur est donnée par \( q(t)=Cu_C(t)\).

    Déterminer la fonction \( u_C(t)\) en résolvant l'équation différentielle \eqref{EqnmKooo}. En déduire la fonction \( q(t)\) en supposant que \( q(0)=0\). Calculer \( \lim_{t\to \infty} q(t)\).

\corrref{SVT-0005}
\end{exercice}
