% This is part of the Exercices et corrigés de CdI-2.
% Copyright (C) 2008, 2009
%   Laurent Claessens
% See the file fdl-1.3.txt for copying conditions.


\begin{corrige}{1182}

\begin{enumerate}
\item Lorsque $x\neq 0$, nous avons que
\begin{equation}
	\lim_{n\to\infty}\frac{ n^2x }{ 1+n^2x^2 }=\frac{1}{ x },
\end{equation}
et $f_n(0)=0$ pour tout $n$. Par conséquent, la limite de la suite des fonctions $f_n$ est
\begin{equation}
	f(x)=
\begin{cases}
	\frac{1}{ x }	&	\text{si $x\neq 0$}\\
	0	&	 \text{si $x=0$}.
\end{cases}
\end{equation}
\item 
Le fait que la limite soit discontinue fait qu'il n'y a pas convergence uniforme sur $\eR$ (plus généralement, il n'y a pas convergence uniforme sur aucun ouvert contenant zéro).

\item Nous avons que $(f_n-f)(x)=-\frac{1}{ x(n^2x^2+1) }\to-\infty$ pour $x\to 0$, donc il n'y a pas de convergence uniforme sur $]0,\infty[$.

\item Étant donné que sur $]0,\infty[$, la fonction $| f_{n}-f |$ est décroissante, si $a$ est le minimum du compact $K$, alors
\begin{equation}
	\sup_{x\in K}| f_n-f |=\frac{ 1 }{ a(1+n^2a^2) }.
\end{equation}
Pour cette raison, $\lim_{n \to\infty}\| f_n-f \|_{\infty}=0$, et on a une convergence uniforme sur tout compact de $]0,\infty[$.
\end{enumerate}


\end{corrige}
