% This is part of the Exercices et corrigés de mathématique générale.
% Copyright (C) 2009-2010
%   Laurent Claessens
% See the file fdl-1.3.txt for copying conditions.


\begin{exercice}\label{exoFoncDeuxVar0018}

	%\let\Oldtheenumi\theenumi
	%\renewcommand{\theenumi}{\arabic{enumi}}
	(INGE1121, 8.8)Calculer les limites suivantes ou monter qu'elles n'existent pas.
	\begin{multicols}{2}
		\begin{enumerate}
			
			\item %2
				\[ 
					\lim_{(x,y)\to(0,0)}\frac{ 2xy^2 }{ x^4+2y^4 }.
				\]
			\item %3
				\[ 
					\lim_{(x,y)\to(2,3)}\frac{ x+y }{ 2-y }	
				\]
			\item	%4
				\[ 
					\lim_{(x,y,z)\to(0,0,0)}\frac{ 2xy+yz }{ x^2+y^2+z^2 }.	
				\]
			\item	%5
				\[ 
					\lim_{(x,y)\to(\frac{ \pi }{2},2)}\frac{ xy+1 }{ 2+\cos(x) }.
				\]
			\item	%6
				\[ 
					\lim_{(x,y,z)\to(0,0,0)}\frac{ 2x^2+y^2-z^2 }{ x^2+y^2+z^2 }
				\]
			\item	%7
				\[ 
					\lim_{(x,y)\to(1,-1)}\frac{ y^2+x }{ (x-1)(y+2) }
				\]
			\item	%8
				\[ 
					\lim_{(x,y,z)\to(0,0,0)}\frac{ x^2+y^3+2z^3 }{ xyz^2 }.
				\]
			\item	%9
				\[ 
					\lim_{(x,y)\to(0,0)}\frac{ x^4-y^4 }{ x^2+y^2 }.
				\]
			\item	%12
				\[ 
					\lim_{(x,y)\to(0,0)}\frac{ xy^2 }{ x^2+2y^2 }.
				\]
			\item	%13
				\[ 
					\lim_{(x,y)\to(0,0)}\frac{ x^3-2x^2y+3y^2x-y^3 }{ x^2+y^2 }.
				\]
			\item	%18
				\[ 
				\lim_{(x,y)\to(0,0)}\frac{ x^2y^3 }{ y^5-x^5 }.
				\]
			\item	%19
				\[ 
				\lim_{(x,y,z)\to(0,0,0)}\frac{ x^2-z^2 }{ x^3-z^3 }.
				\]
			\item	%20
				\[ 
					\lim_{(x,y)\to(1,-1)} e^{-xy^2}.
				\]
			\item	%22
				\[ 
					\lim_{(x,y)\to(0,0)}\ln\sqrt{1-x^2-y^2}
				\]
			\item	%23
				\[ 
					\lim_{(x,y)\to(0,0)}\frac{ x^2-2xy+2y^2 }{ x^2+2y^2 }.
				\]
			\item %24
				\[ 
					\lim_{(x,y)\to(2,-1)}\ln\frac{ 1+x+3y }{ 3y^2-x }
				\]
					
			\item %26
				\[ 
					\lim_{(x,y)\to(0,0)}\frac{  e^{xy}\sin^2(xy) }{ xy }.
				\]
			\item	%27
				\[ 
					\lim_{(x,y)\to(-2,1)}\frac{ x^2-4x+4 }{ xy+2y-x-2 }
				\]
			\item	%28
				\[ 
					\lim_{(x,y)\to(0,0)}\exp\left( \frac{ -1 }{ x^2+y^2 } \right)	
				\]

		\end{enumerate}

	\end{multicols}
	%\let\theenumi\theenumi\Oldtheenumi
	
\corrref{FoncDeuxVar0018}
\end{exercice}
