\begin{corrige}{CourbesSurfaces0006}

	\begin{enumerate}
		\item
			Comme expliqué à la page \pageref{PgGrqFnGamma}, nous voyons le graphe de la fonction $y=f(x)$ comme le chemin $\gamma(t)=\big( t,f(t) \big)$. Nous avons donc
			\begin{equation}
					\begin{aligned}[]
					\gamma(t)&=\big( t,\sqrt{t}(1-\frac{ t }{ 3 }) \big)\\
					\gamma'(t)&=\big( 1,\frac{ 1-t }{ 2\sqrt{t} } \big),
				\end{aligned}
			\end{equation}
			et par conséquent l'élément de longueur à intégrer est
			\begin{equation}
				\| \gamma'(t) \|=\sqrt{1+\frac{ (1-t)^2 }{ 4t }}=\frac{ t+1 }{ 2\sqrt{t} },
			\end{equation}
			et la longueur de l'arc considéré vaut
			\begin{equation}
				\int_0^3\| \gamma'(t) \|=\left[ \frac{1}{ 3 }t^{3/2}+t^{1/2} \right]_{t=0}^{3}=\frac{1}{ 3 }3^{3/2}+\sqrt{3}=2\sqrt{3}.
			\end{equation}
		
		\item
			En calculant les dérivées, nous trouvons
			\begin{equation}
				\| \gamma'(t) \|^2=18\big( 1-\sin(t)\sin(3t)-\cos(t)\cos(3t) \big)
			\end{equation}
			que l'on peut simplifier en utilisant des formules de trigonométrie%\footnote{$2\cos(x)\cos(y)=\cos(x-y)+\cos(x+y)$ et la formule du même type pour les sinus.} en
			\begin{equation}
				\| \gamma'(t) \|^2=36\sin^2(t).
			\end{equation}
			Nous allons devoir intégrer la racine carrée de cela entre $0$ et $2\pi$. Il ne faut pas dire que $\sqrt{\sin^2(t)}=\sin(t)$ et ensuite calculer
			\begin{equation}
				\sqrt{36}\int_0^{2\pi}\sin(t)dt,
			\end{equation}
			parce que $\sin(t)$ est négatif entre $\pi$ et $2\pi$. Ce qu'il faut faire est $\sqrt{\sin^2(t)}=| \sin(t) |$ et profiter du fait que $|\sin(t)|$ est symétrique par rapport à $\pi$ :
			\begin{equation}
				l=2\sqrt{36}\int_0^{\pi}\sin(t)dt=4\sqrt{36}=24.
			\end{equation}


	\end{enumerate}

\end{corrige}
