% This is part of Analyse Starter CTU
% Copyright (c) 2014
%   Laurent Claessens,Carlotta Donadello
% See the file fdl-1.3.txt for copying conditions.

\begin{corrige}{autoanalyseCTU-43}

  \begin{enumerate}
  \item[(1)] Nous pouvons soit utiliser la formule générale de Taylor-Young, \eqref{EqTJRooUbsyzJ}, soit se ramener à un développement limité au voisinage de zéro par une translation, ce qui veut dire, développer  au voisinage de zéro la fonction $g(x) = e^{1+x}$. Les deux méthodes sont absolument équivalentes.  

Nous allons essayer la première, car il y a des exemples de la deuxième méthode dans le chapitre sur les développements limités. Remarquez que $(e^x)^{(m)} = e^x$ pour tout $m\in\eN^*$, donc on pourra mettre à facteur $e$, qui est la valeur de la dérivée de l'exponentielle calculée en $x=1$. Il faut d'abord appliquer la formule et ensuite faire tous les calculs nécessaires à simplifier au maximum l'expression obtenue 
\begin{equation*}
  \begin{aligned}
    e^{x} = & e\left(1+(x-1) + \frac{(x-1)^2}{2} + \frac{(x-1)^3}{3!} + \frac{(x-1)^4}{4!} \right)  + (x-1)^4\alpha(x-1) \\
    &  =e\left(1+(x-1) + \frac{x^2+2x+1}{2} + \frac{x^3-3x^2 +3x-1}{6} + \frac{x^4-4x^3+6x^2-4x+1}{24} \right) \\
    &\qquad\qquad + (x-1)^4\alpha(x-1) \\
    &  =e\left( \frac{12-4+1}{24} + \frac{3x-x}{6} + \frac{x^2}{4} + \frac{x^4}{24}\right)  + (x-1)^4\alpha(x-1) \\
    &  =e\left( \frac{3}{8} + \frac{x}{3} + \frac{x^2}{4} + \frac{x^4}{24}\right)  + (x-1)^4\alpha(x-1) .
  \end{aligned}
\end{equation*}
 \item[(2)] Le développement à trouver est égale au développement autour de zéro de la fonction $x\mapsto \sin(x+\pi/2)$. On sait que $\sin(x+\pi/2) = \cos(x)\sin(\pi/2) + \cos(\pi/2)\sin(x) = \cos(x)$, et le développement de la fonction cosinus est connu (voir le tableau des développements dans le cours).
  \item[(3)] Il faudra utiliser ici la règle pour trouver le développement d'un rapport entre fonctions. Nous pouvons commencer par calculer les développements à l'ordre 3 en 0 des fonction $\ln(1+x)$ et $\sqrt{1+x}$. Ces deux développements sont dans la liste des développements à conn\"{i}tre, mais on les rappelle ici pour plus de lisibilité :
    \begin{equation*}
      \begin{aligned}
        \ln(x+1) &= x -\frac{x^2}{2} + \frac{x^3}{3} -\frac{x^4}{4} + \ldots ;\\
        \sqrt{1+x} & = (1+x)^{1/2}  = 1+ \frac{x}{2} -\frac{x^2}{8} + \frac{x^3}{16} + \ldots .
      \end{aligned}
    \end{equation*}
    Nous allons donc procéder à une division pour déterminer le développement du rapport $\dfrac{\ln(1+x)}{\sqrt{1+x}}$. 
    \begin{equation*}
        \begin{array}[]{ccccccccccc|c}
            &x&-&\frac{x^2}{ 2 }&+&\frac{x^3}{ 3 }& -& \frac{x^4}{4}&&&&1+\frac{ x }{2}-\frac{x^2}{ 8 }+ \frac{x^3}{16}\\
            \cline{12-12}
            -\Big( &x&+&\frac{ x^2 }{2}&-&\frac{x^3}{ 8 }&+& \frac{x^4}{16}&\Big) && &x-x^2+\frac{23x^3}{ 24 }\\
            \cline{2-8}
            & &- &x^2&+&\frac{11x^3}{ 24 }&- & \frac{5x^4}{16}& & && \\
            &&-\Big( - &x^2&-&\frac{x^3}{ 2 }&+&\frac{x^4}{ 8 }&\Big) & & \\
            \cline{4-10}
            & & & & &\frac{ 23x^3}{ 24 }&-&\frac{7x^4}{ 16 }& & & \\
            & & &  &-\Big(  &\frac{23 x^3 }{ 24 }&+&\frac{x^4}{ 48 }&\Big)& && \\
            \cline{6-10}
            & & & & & &- &\frac{11x^4}{24} &&&& \\
        \end{array}
    \end{equation*}
 Le développement cherché est donc $\dfrac{\ln(1+x)}{\sqrt{1+x}}=x-x^2+\frac{23x^3}{ 24 }+ x^3\alpha(x)$.
  \end{enumerate}
 
\end{corrige}   
