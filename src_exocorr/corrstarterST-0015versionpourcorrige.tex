% This is part of Analyse Starter CTU
% Copyright (c) 2014,2016
%   Laurent Claessens,Carlotta Donadello
% See the file fdl-1.3.txt for copying conditions.

\begin{corrige}{starterST-0015versionpourcorrige}
\begin{enumerate}
\item[(4)] Nous allons simplement utiliser les définitions de $\sinh$ et  $\cosh$. 
  \begin{equation*}
    \begin{aligned}
      \text{sinh} (x+y)& = \frac{e^{x+y}-e^{-(x+y)}}{2} ; \\
      \text{sinh}(x) \text{cosh}(y)+&\text{cosh}(x)\text{sinh}(y) =\frac{e^{x}-e^{-x}}{2}\frac{e^{y}+e^{-y}}{2}\frac{e^{x}+e^{-x}}{2}\frac{e^{y}-e^{-y}}{2}\\
      &=\frac{e^{x+y}+e^{x-y}-e^{-x+y}-e^{-(x+y)}}{4} +\frac{e^{x+y}-e^{x-y}+e^{-x+y}-e^{-(x+y)}}{4} \\
      &=\frac{e^{x+y}-e^{-(x+y)}}{2}. 
    \end{aligned}
  \end{equation*}
  L'autre égalité peut se montrer de façon analogue.
\item[(5)] Par le point précédent, en prenant $y=x$ : $\text{cosh}(2x)=\text{cosh}^2(x)+\text{sinh}^2(x) $ et $\text{sinh}(2x)= 2\text{cosh}(x)\text{sinh}(x) $.
 \item[(6)] 
   \begin{equation*}
     \begin{aligned}
       f(x)=&\cosh\Big(\ln(x+\sqrt{x^2-1})\Big)\\
       &=\frac{e^{\ln(x+\sqrt{x^2-1})}+e^{-\ln(x+\sqrt{x^2-1})}}{2} = \frac{1}{2}\left(x+\sqrt{x^2-1} + \frac{1}{x+\sqrt{x^2-1}}\right) =x
     \end{aligned}
   \end{equation*}
\end{enumerate}

\end{corrige}
