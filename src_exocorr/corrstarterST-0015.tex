% This is part of Analyse Starter CTU
% Copyright (c) 2014
%   Laurent Claessens,Carlotta Donadello
% See the file fdl-1.3.txt for copying conditions.

\begin{corrige}{starterST-0015}


\begin{enumerate}
\item Les fonctions  $\text{sinh}$ et  $\text{cosh}$ sont définies sur $\eR$ tout entier. La fonction $\sinh$ est impaire, car 
  \begin{equation*}
    \sinh(-x) = \frac{e^{-x}-e^{-(-x)}}{2} = \frac{e^{-x}-e^{x}}{2} = -\sinh(x).
  \end{equation*}
En procédant de la m\^eme manière on trouve que $\cosh$ est paire.  
\item 
\[
\sinh'(x) = \frac{e^{x}-\left(-e^{-x}\right)}{2} = \frac{e^{x}+e^{-x}}{2}  = \cosh(x),
\]
\[
\cosh'(x) = \frac{e^{x}+\left(-e^{-x}\right)}{2} = \frac{e^{x}-e^{-x}}{2}  = \sinh(x).
\]
\item On calcule directement en utilisant les formules \eqref{defcoshetsinh}. Pour tout $x\in\mathbb{R}$ 
\[
\cosh^2 (x) - \sinh^2 (x) = \left(\frac{e^{x}+e^{-x}}{2}\right)^2 -\left(\frac{e^{x}-e^{-x}}{2}\right)^2 = \frac{1}{4} \left[(e^{2x}+e^{-2x}+2) - (e^{2x}+e^{-2x}-2)\right] = 1.
\]
par la définition on savait déjà que la fonction $\cosh$ est strictement positive, cette nouvelle relation nous dit que $\cosh(x)>1$ pour tout $x\in\eR$.
\item[(4)] Nous allons simplement utiliser les définitions de $\text{sinh}$ et  $\text{cosh}$. 
  \begin{equation*}
    \begin{aligned}
      \text{sinh} (x+y)& = \frac{e^{x+y}-e^{-(x+y)}}{2} ; \\
      \text{sinh}(x) \text{cosh}(y)+&\text{cosh}(x)\text{sinh}(y) =\frac{e^{x}-e^{-x}}{2}\frac{e^{y}+e^{-y}}{2}\frac{e^{x}+e^{-x}}{2}\frac{e^{y}-e^{-y}}{2}\\
      &=\frac{e^{x+y}+e^{x-y}-e^{-x+y}-e^{-(x+y)}}{4} +\frac{e^{x+y}-e^{x-y}+e^{-x+y}-e^{-(x+y)}}{4} \\
      &=\frac{e^{x+y}-e^{-(x+y)}}{2}. 
    \end{aligned}
  \end{equation*}
  L'autre égalité peut se montrer de façon analogue.
\item[(5)] Par le point précédent, en prenant $y=x$ : $\text{cosh}(2x)=\text{cosh}^2(x)+\text{sinh}^2(x) $ et $\text{sinh}(2x)= 2\text{cosh}(x)\text{sinh}(x) $.
 \item[(6)] 
   \begin{equation*}
     \begin{aligned}
       f(x)=&\cosh\Big(\ln(x+\sqrt{x^2-1})\Big)\\
       &=\frac{e^{\ln(x+\sqrt{x^2-1})}+e^{-\ln(x+\sqrt{x^2-1})}}{2} = \frac{1}{2}\left(x+\sqrt{x^2-1} + \frac{1}{x+\sqrt{x^2-1}}\right)=x
     \end{aligned}
   \end{equation*}
%\item Étudier les variations des deux fonctions et en tracer une représentation graphique. 
%\item Démontrer les formules suivantes :
%\begin{enumerate}
%\item $\text{sinh} (x+y)=\text{sinh}(x) \text{cosh}(y)+\text{cosh}(x)\text{sinh}(y)$ ;
%\item $\text{cosh} (x+y)=\text{cosh}(x) \text{cosh}(y)+\text{sinh}(x)\text{sinh}(y)$.
%\end{enumerate}
%\item Donner des expressions de $\text{cosh}(2x)$ et $\text{sinh}(2x)$  en fonction de $\text{cosh}(x)$ et $\text{sinh}(x)$.
%\item Simplifier l'expression $f(x)=\cosh\Big(\ln(x+\sqrt{x^2-1})\Big)$.
\end{enumerate}

\end{corrige}
