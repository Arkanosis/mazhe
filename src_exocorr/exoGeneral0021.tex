% This is part of the Exercices et corrigés de mathématique générale.
% Copyright (C) 2009-2011
%   Laurent Claessens
% See the file fdl-1.3.txt for copying conditions.
\begin{exercice}\label{exoGeneral0021}

Exercice 6, page 48. Soit $F(x)=\int_{-\infty}^xe^{-t^2}dt$. Parmi les affirmations suivantes, lesquelles sont vraies ? Justifier.

\begin{multicols}{2}
\begin{enumerate}

\item
$F$ est croissante
\item
$F$ est décroissante

\item
$F$ n'est ni croissante ni décroissante

\item
$F$ ne s'annule jamais

\item
$F$ est impaire

\item
$F$ est paire

\item
$F$ admet un maximum en $x=0$

\item
$F$ admet un minimum en $x=0$

\item
$F$ admet un point d'inflexion en $x=0$.

\end{enumerate}
\end{multicols}

Classer par ordre croissant les nombres suivants :
\begin{equation}
	\begin{aligned}[]
		0,&&1,&&\int_2^3 e^{-x^2}dx,&&\int_0^1 e^{-x^2}dx,&&\int_{-3}^{-2} e^{-x^2}dx.
	\end{aligned}
\end{equation}

\corrref{General0021}
\end{exercice}
