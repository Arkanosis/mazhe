% This is part of the Exercices et corrigés de mathématique générale.
% Copyright (C) 2009-2010
%   Laurent Claessens
% See the file fdl-1.3.txt for copying conditions.


\begin{corrige}{FoncDeuxVar0012}

	L'énoncé «étudier la continuité de» revient à demander «vérifier si la limite est égale à la valeur». Dans le cadre de cet exercice, nous devons simplement calculer les limites des fonctions et voir si ces limites sont égales à la valeur donnée.

	\begin{enumerate}

		\item
			Regardons la ligne $x=y$, c'est à dire le chemin $(x,y)=(t,t)$. La limite de $f$ le long de ce chemin vaut
			\begin{equation}
				\lim_{t\to 0}\frac{ 2t^2 }{ t^2+t^2 }=1,
			\end{equation}
			tandis que le long du chemin $x=0$ (c'est à dire $(0,t)$) nous avons
			\begin{equation}
				\lim_{t\to 0}\frac{ 0 }{ t^2 }=0.
			\end{equation}
			Les deux limites n'étant pas égales, la limite de $f$ pour $(x,y)\to (0,0)$ n'existe pas et \emph{a forciori} la fonction n'est pas continue.
		\item
			Elle n'est pas continue parce que la limite n'existe même pas. Prenez par exemple le chemin vertical $x=0$. Le long de ce chemin, la fonction vaut $f(0,y)=\frac{1}{ y^2 }$ qui n'a certainement pas de limites pour $y\to 0$.

		\item
			En valeur absolue, $\cos(1/y)$ est borné par $1$, donc 
			\begin{equation}
				0\leq | f(x,y) |\leq | y |
			\end{equation}
			et donc elle tend vers zéro lorsque $(x,y)\to(0,0)$. Cette fonction est donc continue.

	\end{enumerate}

\end{corrige}
