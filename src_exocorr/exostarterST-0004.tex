% This is part of Analyse Starter CTU
% Copyright (c) 2014
%   Laurent Claessens,Carlotta Donadello
% See the file fdl-1.3.txt for copying conditions.

\begin{exercice}\label{exostarterST-0004}

Soit $f(x) = x(x-2)(x+3)$. 

\begin{enumerate}
\item Tracer la représentation graphique de $f$ et donner son tableau de variations.
\item Trouver un intervalle $I_1$ sur lequel $f$ est monotone croissante.
\item Trouver un intervalle $I_2$ sur lequel $f$ est monotone décroissante.
\item Trouver un intervalle $I_3$ et un intervalle $J$ tels que $f$ soit une bijection de $I_3$ dans $J$.

\end{enumerate}

Remarque : si cet exercice vous parait difficile, vous pouvez le faire d'abord pour la fonction $g(x) = x^2$. 
  

\corrref{starterST-0004}
\end{exercice}
