% This is part of Outils mathématiques
% Copyright (c) 2012
%   Laurent Claessens
% See the file fdl-1.3.txt for copying conditions.

\begin{corrige}{OutilsMath-0102}

    \begin{enumerate}
        \item
            %TODO : faire le dessin.
            Le mieux est de faire un dessin. Le domaine d'intégration est le triangle de sommets \( (0,0),(\frac{ 4 }{ 3 },\frac{ 4 }{ 3 }),(4,0)\). Nous le décomposons en deux parties. Celle de gauche est
            \begin{subequations}
                \begin{numcases}{}
                    x\colon 0\to \frac{ 4 }{ 3 }\\
                    y\colon 0\to x.
                \end{numcases}
            \end{subequations}
            et celle de droite est
            \begin{subequations}
                \begin{numcases}{}
                    x\colon \frac{ 4 }{ 3 }\to 4\\
                    y\colon 0\to 2-\frac{ x }{2}.
                \end{numcases}
            \end{subequations}
            L'intégrale à calculer est donc
            \begin{equation}
                \int_0^{4/3}\int_0^x\,x\,dydx+\int_{4/3}^4\int_0^{2-x/2}x\,dydx=\frac{ 64 }{ 81 }+\frac{ 320 }{ 81 }=\frac{ 128 }{ 27 }.
            \end{equation}
            \begin{verbatim}
sage: f(x,y)=x
sage: f.integrate(y,0,x).integrate(x,0,4/3)
(x, y) |--> 64/81
sage: f.integrate(y,0,2-x/2).integrate(x,4/3,4)  
(x, y) |--> 320/81
sage: 64/81+320/81
128/27
            \end{verbatim}
            <++>
    \end{enumerate}
    <++>

\end{corrige}
