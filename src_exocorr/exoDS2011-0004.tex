\begin{exercice}\label{exoDS2011-0004}

Considérons  courbe paramétrée $(]0,\pi/2], \gamma_{pol}(\theta))$, où la fonction $\gamma_{pol}$ est la fonction de $]0,\pi/2]$ dans $\eR^2$  $\gamma_{pol}(\theta)=(r(\theta)=2\cos(\theta), \theta)$ (coordonnées polaires).
\begin{enumerate}
\item Trouver une paramétrisation du même arc géométrique de la forme $\gamma_{cart}(x)=(x, y(x))$ (coordonnées cartésiennes). Conseil : essayez d'écrire d'abord l'équation cartésienne de la courbe.
\item La courbe est une partie d'un cercle. Trouver son centre, son rayon et dessiner la courbe dans le plan cartésien $x$-$y$. 
\item Déterminer la longueur de la courbe. Vous pouvez utiliser quelconque méthode qui vous vient à l'esprit quitte à la justifier.
\item[\textbf{Bonus :}] Dessiner la courbe dans le demi-plan $\rho$-$\theta$.    
\end{enumerate}
\corrref{DS2011-0004}
\end{exercice}
