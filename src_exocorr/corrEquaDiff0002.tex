% This is part of the Exercices et corrigés de mathématique générale.
% Copyright (C) 2009-2010
%   Laurent Claessens
% See the file fdl-1.3.txt for copying conditions.
\begin{corrige}{EquaDiff0002}

\begin{enumerate}

\item
L'équation est
\begin{equation}
	y'=\frac{ -4y }{ x }.
\end{equation}
En posant $y=xz$ et $y'=z+xz'$, nous trouvons
\begin{equation}
	\begin{aligned}[]
		z+xz'&=\frac{ -4xz }{ x }=-4z\\
		xz'&=-5z\\
		\frac{ z' }{ z }&=-\frac{ 5 }{ x },
	\end{aligned}
\end{equation}
ce qui fait $\ln(z)=-5\ln(x)+C$, que l'on remet dans la variable $y$ :
\begin{equation}
	y=Kx^{-4}.
\end{equation}


\item
En divisant par $dx$, l'équation devient $(2x+3y)+(y-x)y'=0$, que l'on remet sous la forme
\begin{equation}
	y'=\frac{ 3y+2x }{ x-y }.
\end{equation}
Comme indiqué dans la méthode générale pour ce genre d'équations, il faut poser $u=y/x$, c'est à dire $y=ux$ et $y'=u+xu'$, ce qui donne
\begin{equation}
	\begin{aligned}[]
		u+xu'&=\frac{ 3ux+2x }{ x-ux }=\frac{ 3u+2 }{ 1-u }\\
		xu'&=\frac{ 3u+2 }{ 1-u }-u\\
		x\frac{ du }{ dx }&=\frac{ u^2+2u+2 }{ 1-u }.
	\end{aligned}
\end{equation}
Ici, la subtilité est de remettre tous les $u$ d'un côté et tous les $x$ de l'autre, y compris les $du$ et $dx$. Ce que l'on obtient est
\begin{equation}
	\frac{ 1-u }{ u^2+2u+2 }du=\frac{ dx }{ x },
\end{equation}
qui peut être intégré\footnote{Voir le rappel \ref{subsecCarreDenoPar}.} des deux côtés :
\begin{equation}
	\ln(x)+K=2\arctan(u+1)-\frac{ 1 }{2}\ln(u^2+2u+2).
\end{equation}
Nous remettons maintenant les $y$ au lieu des $u$ et, en utilisant les propriétés des logarithmes, 
\begin{equation}
	K=\ln(y^2+2xy+2x^2)-4\arctan\left( \frac{ y+x }{ 2 } \right).
\end{equation}
Il est à remarquer qu'il n'est pas possible d'isoler $y$ dans cette expression. Nous ne pouvons donc pas donner la solution sous le forme explicite $y=y(x)$. Cela arrive souvent dans le cadre des équations différentielles.

\end{enumerate}

\end{corrige}
