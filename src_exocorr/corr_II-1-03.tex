% This is part of the Exercices et corrigés de CdI-2.
% Copyright (C) 2008, 2009
%   Laurent Claessens
% See the file fdl-1.3.txt for copying conditions.


\begin{corrige}{_II-1-03}

En suivant le notations usuelles, nous avons $u(t)=t^3+t$ et $f(y)= e^{-y}$. La solution générale est donnée par
\begin{equation}
	G\big( y(t) \big)=U(t)+C
\end{equation}
$U$ est une primitive de $u$ et $G$ est une primitive de $1/f$. En l'occurrence,
\begin{equation}
	y(t)=\ln\left( \frac{ t^4 }{ 4 }+\frac{ t^2 }{ 2 }+C \right).
\end{equation}
Lorsque $C=e-\frac{ 3 }{ 4 }$, nous avons $y(1)=1$, et la solution est
\begin{equation}
	y(t)=\ln\left( \frac{ t^4+2t^2+4e-3 }{ 4 } \right),
\end{equation}
qui existe sur tout $\eR$.

\end{corrige}
