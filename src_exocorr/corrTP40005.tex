% This is part of the Exercices et corrigés de mathématique générale.
% Copyright (C) 2009
%   Laurent Claessens
% See the file fdl-1.3.txt for copying conditions.
\begin{corrige}{TP40005}

	Le coût total de l'expédition est le coût du personnel plus celui du carburant. Afin d'exprimer le tout en fonction de $v$, il faut se rendre compte que la durée de l'expédition est la distance à parcourir divisée par la vitesse : $d/v$. À partir de là, la prix du carburant est
	\begin{equation}
		kv^2\left( \frac{ d }{ v } \right)=kdv,
	\end{equation}
	et le prix du personnel est
	\begin{equation}
		p\left( \frac{ d }{ v } \right),
	\end{equation}
	ce qui donne le coût total en fonction de la vitesse sous la forme
	\begin{equation}
		C(v)=kdv+\frac{ pd }{ v }.
	\end{equation}
	Pour en trouver le maximum, il faut trouver la valeur de $v$ où la dérivée de cette fonction s'annule. La dérivée vaut
	\begin{equation}
		C'(v)=kd-\frac{ pd }{ v^2 },
	\end{equation}
	et cela vaut zéro pour
	\begin{equation}
		v=\pm\sqrt{\frac{ p }{ k }}.
	\end{equation}
	La solution négative est évidement à rejeter. La vitesse optimale est donc
	\begin{equation}
		v=\sqrt{p/k}.
	\end{equation}
	Il est intéressant de noter que cette vitesse ne dépend pas de la distance à parcourir.

\end{corrige}
