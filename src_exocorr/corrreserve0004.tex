% This is part of Exercices et corrigés de CdI-1
% Copyright (c) 2008,2009,2012,2013
%   Laurent Claessens
% See the file fdl-1.3.txt for copying conditions.

\begin{corrige}{reserve0004}

\begin{enumerate}

\item
$ty'+y=ty^2$. 

Évidement, $y=0$ est solution. Si $y(t_0)\neq 0$, nous pouvons chercher une solution non triviale dans un voisinage de $t_0$. En vertu de la méthode générale exposée en \ref{SubSecBernh}, nous posons $z=1/y$, ce qui donne $y'=-y^2z'$. Après aménagements (diviser par $y^2$), nous trouvons
\begin{equation}		\label{EqII107Lin}
	-tz'+z=t,
\end{equation}
qui est une équation linéaire. La solution générale de l'équation homogène associée est
\begin{equation}
	z_H(t)=At.
\end{equation}
Nous utilisons maintenant la méthode de variation des constantes pour trouver la solution générale de \eqref{EqII107Lin}. Nous posons donc $z(t)=A(t)t$ que nous remplaçons dans l'équation de départ $ty'+y=ty^2$. Nous tombons sur l'équation
\begin{equation}
	A'=-\frac{1}{ t }
\end{equation}
dont la solution générale est $A(t)=\ln\left( \frac{ C }{ t } \right)$. En définitive, la solution à notre problème est
\begin{equation}
	z(t)=t\ln\left( \frac{ C }{ t } \right),
\end{equation}
et donc
\begin{equation}
	y(t)=\frac{1}{ t\ln\left( \frac{ C }{ t } \right) }.
\end{equation}
N'oublions pas de mentionner que $y=0$ est aussi solution (qui correspond à $C\to 0$).

\item
$y'=-\frac{1}{ t }y+\frac{ \ln| t | }{ t }y^2$.

Nous posons $z=1/y$, et nous récrivons l'équation sous la forme
\begin{equation}
	z'-\frac{1}{ t }z=-\frac{ \ln| t | }{ t }
\end{equation}
qui est une équation linéaire. L'équation linéaire est 
\begin{equation}
	z_H'=\frac{ z_H }{ t }
\end{equation}
et la solution est $z_h=At$. Nous utilisons encore la méthode de la variation des constantes en posant $z(t)=A(t)t$. L'équation différentielle à laquelle doit satisfaire $A(t)$ est alors
\begin{equation}
	A'=-\frac{ \ln(t) }{ t^2 }.
\end{equation}
Trouver une primitive de $\frac{ \ln| t | }{ t^2 }$ n'est pas trop aisé (ça se fait par partie), mais la solution est
\begin{equation}
	A(t)=\frac{1}{ t }\big( \ln(t)+1 \big)+Ct,
\end{equation}
donc
\begin{equation}
	z(t)=\ln(t)+1+Ct.
\end{equation}

\item
$y-\cos(t)y'=\cos(t)\big( 1-\sin(t) \big)y^2$.

Encore une fois, $y=0$ est solution. En posant $z=1/y$, nous trouvons l'équation
\begin{equation}		\label{EqII107EqpourZ}
	z+\cos(t)z'=\cos(t)\big(1-\sin(t)\big)
\end{equation}
à laquelle $z$ doit satisfaire. L'équation homogène est
\begin{equation}
	z_H'=-\frac{ z_H }{ \cos(t) }.
\end{equation}
Nous résolvons cette équation en utilisant la méthode des équations à variable séparées de la section \ref{Secvarsep}. Nous posons donc
\begin{equation}		\label{EqufUGII107}
	\begin{aligned}[]
		u(t)	&=\frac{1}{ \cos(t) }, \\
		f(z)	&=-z,\\
		U(t)	&=\ln\left[ \tan\left( \frac{ \pi }{ 4 }+\frac{ t }{ 2 } \right) \right]	&\text{(voir formulaire)},\\
		G(z)	&=\ln\left( \frac{1}{ z } \right).
	\end{aligned}
\end{equation}
La solution $z_H$ est donnée par l'équation
\begin{equation}
	\ln\left( \frac{1}{ z } \right)=\ln\left[ K\tan\left( \frac{ \pi }{ 4 }+\frac{ t }{ 2 } \right) \right],
\end{equation}
c'est à dire
\begin{equation}
	z_H(t)=\frac{ K }{ \tan\left( \frac{ \pi }{ 4 }+\frac{ t }{ 2 } \right) }.
\end{equation}
Nous appliquons maintenant la méthode de variation des constantes sur cette solution afin de trouver la solution générale de l'équation \eqref{EqII107EqpourZ}. En utilisant la règle de Leibnitz, $z'=K'z_H+Kz'_H$, nous trouvons
\begin{equation}
	\frac{ K }{ \tan\left( \frac{ \pi }{ 4 }+\frac{ t }{ 2 } \right) }+\cos(t)\left( \frac{ K' }{  \tan\left( \frac{ \pi }{ 4 }+\frac{ t }{ 2 } \right) }-\frac{ K }{ 2\sin^2 \left( \frac{ \pi }{ 4 }+\frac{ t }{ 2 } \right)  } \right)=\cos(t)\big( 1-\sin(t) \big).
\end{equation}
Malgré leurs apparences, les deux termes en $K$ se simplifient. En effet, en vertu de l'équation $z_H'=\frac{ -z_H }{ \cos(t) }$, nous avons
\begin{equation}
	\frac{ -K }{ 2\sin^2\left( \frac{ \pi }{ 4 }+\frac{ t }{ 2 } \right)}=\frac{ -K }{ \cos(t)\tan\left( \frac{ \pi }{ 4 }+\frac{ t }{ 2 } \right) }.
\end{equation}
Le travail de voir quel est le lien entre $\sin^2\left( \frac{ \pi }{ 4 }+\frac{ t }{ 2 } \right)$, $\tan\left( \frac{ \pi }{ 4 }+\frac{ t }{ 2 } \right)$ et $\cos(t)$ est en réalité fait dans votre formulaire au moment où vous l'avez utilisé pour intégrer $u$ pour obtenir le $U(t)$ de \eqref{EqufUGII107}.

Après cette simplification durement méritée, nous trouvons l'équation suivante pour $K(t)$ :
\begin{equation}		\label{EqFracII107exoVVprb}
	\frac{ K' }{ \tan\left( \frac{ \pi }{ 4 }+\frac{ t }{ 2 } \right) }=1-\sin(t).
\end{equation}
Résoudre cela revient à trouver la primitive de
\begin{equation}
\big( 1-\sin(t) \big) \tan\left( \frac{ \pi }{ 4 }+\frac{ t }{ 2 } \right),
\end{equation}
ce qui est relativement compliqué. La réponse est
\begin{equation}
	\begin{aligned}[]
		K(t)	&=\ln \left(\sin \left({{2\,x+\pi}\over{4}}\right)+1\right)+\ln  \left(\sin \left({{2\,x+\pi}\over{4}}\right)-1\right)\\
			&\quad+2\,\ln \sec  \left({{2\,x+\pi}\over{4}}\right)+2\,\sin ^2\left({{2\,x+\pi}\over{4 }}\right)
	\end{aligned}
\end{equation}
Nous pouvons un peu simplifier en utilisant le fait que $\ln(a+b)+\ln(a-b)=\ln(a^2-b^2)$ :
\begin{equation}
	\begin{aligned}[]
		K(t)	=\ln\left(-\cos^2 \left({{2\,x+\pi}\over{4}}\right)\right)
			+2\,\ln \sec  \left({{2\,x+\pi}\over{4}}\right)+2\,\sin ^2\left({{2\,x+\pi}\over{4 }}\right).
	\end{aligned}
\end{equation}
Il me semble toutefois qu'il faudrait prendre des valeurs absolues pour les logarithmes.

\end{enumerate}

\end{corrige}
