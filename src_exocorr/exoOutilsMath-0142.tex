% This is part of Outils mathématiques
% Copyright (c) 2012
%   Laurent Claessens
% See the file fdl-1.3.txt for copying conditions.

\begin{exercice}\label{exoOutilsMath-0142}

    Soit le domaine \( D\) défini par
    \begin{subequations}
        \begin{align}
            0\leq x+y\leq 5\\
            -2\leq x-y\leq 3.
        \end{align}
    \end{subequations}
    Calculer \( \int_D(x^2-y^2)dxdy\).

    Note : \( x^2-y^2\) est un produit remarquable. L'auteur de ces lignes vous recommande chaudement d'y penser et d'effectuer un changement de variables (ou de choisir une paramétrisation) sensé.
    

\corrref{OutilsMath-0142}
\end{exercice}
