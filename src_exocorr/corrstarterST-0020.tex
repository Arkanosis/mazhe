% This is part of Analyse Starter CTU
% Copyright (c) 2014
%   Laurent Claessens,Carlotta Donadello
% See the file fdl-1.3.txt for copying conditions.

\begin{corrige}{starterST-0020}

 \begin{enumerate}
 \item La solution cherchée est une fonction polynomiale de degré au plus $3$. Il s'agit donc d'une fonction de la forme $y(x) = ax^3 + b x^2 + cx + d$ avec $a$, $b$, $c$, $d$ réels à déterminer. Nous calculons la dérivée de cette fonction, $y'(x) = 3ax^2 + 2b x + c$, et ensuite nous faisons la substitution dans l'équation 
   \[
   \left(3ax^2 + 2b x + c\right) - 4\left(ax^3 + b x^2 + cx + d\right) = 4x^3-15x^2+2x-3.
   \]
   Cette équation ne peut \^etre vérifiée que si les coefficients des termes du m\^eme degré sont égaux. Pour trouver les bonnes valeurs de $a$, $b$, $c$, $d$ dans $\eR$ il suffit donc d'écrire un système comme le suivant 
   \begin{equation*}
     \begin{cases}
       -4a = 4, \quad & \text{ ici on impose que les coefficient des termes de degré 3 soient égaux ; } \\
       3a -4b  = -15, \quad & \text{ ici on impose que les coefficient des termes de degré 2 soient égaux ; } \\
       2b -4c = 2, \quad & \text{ ici on impose que les coefficient des termes de degré 1 soient égaux ; } \\
       c-4d = -3, \quad & \text{ ici on impose que les coefficient des termes de degré 0 soient égaux. }
     \end{cases}
   \end{equation*}
   
   On a alors $a = -1$, $b = 3$, $c = 1$ et $d = 1$. La solution polynomiale de l'équation différentielle est  $y(x) = -x^3 + 3 x^2 + x +1$.  
 \item Ici aussi il faut procéder par substitution. La dérivée de $y(x) = a\cos (x)+ b\sin (x)$ est $y'(x) = -a\sin (x)+ b\cos (x)$, donc on obtient
   \[
   -a\sin (x)+ b\cos (x)  - 2\left( a\cos (x)+ b\sin (x)\right) = \sin(x)
   \]
      
      Cette équation doit \^etre satisfaite pour tout valeur de $x$. En particulier, si $x$ vaut $0$ alors tous les termes avec $\sin(x)$ sont nuls et si $x$ vaut $\pi/2$ alors tous les termes avec $\cos(x)$ sont nuls. Cela permet d'écrire un système de deux équations pour déterminer les deux inconnues $a$ et $b$ :
      \begin{equation*}
        \begin{cases}
          b  - 2 a = 0 \quad & \text{ ici on prend }x= 0   \\
          -a -2b = 1 \quad & \text{ ici on prend }x= \pi/2  
        \end{cases}
      \end{equation*}
 \end{enumerate}
 
\end{corrige}
