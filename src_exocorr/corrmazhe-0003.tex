% This is part of (almost) Everything I know in mathematics
% Copyright (c) 2016
%   Laurent Claessens
% See the file fdl-1.3.txt for copying conditions.

\begin{corrige}{mazhe-0003}

    Pour \( z_1 \), en arithmétique exacte :
    \begin{equation}
        z_1=\frac{ 0.5\times 10^{20} }{ 0.5\times 10^{20} }+\frac{ 1.5\times 10^{20} }{ 1\times 10^{20} }=0.25\times 10^{1}.
    \end{equation}
    Le calcul exact de \( z_2\) donne la même chose.

    \begin{subproof}

        \item[Calcul de \( z_1\)]
        
            Les deux valeurs sont mémorisables et la différence \( x-y\)<++> se fait sans erreurs de cancellation. Idem pour la somme \( x+y\). Idem pour les divisions.

        \item[Calcul de \( z_2\)]

            Pour faire \( x^2\), c'est pas possible parce que c'est de l'ordre de \( 10^{40}\) alors que nous sommes en précision simple. Idem pour le produit \( xy\).

    \end{subproof}
    Morale : \( z_2\) donne un \info{overflow} alors que \( z_1\) fonctionne de façon exacte.

    \begin{remark}
        En réalité le \( z_1\) n'est pas tout à fait calculable de façon exacte sur la machine parce qu'elle doit d'abord convertir en binaire, ce qui n'est pas toujours possible. Mais sur notre machine qui fonctionne en base \( 10\), il n'y a pas de problèmes.
    \end{remark}

\end{corrige}
