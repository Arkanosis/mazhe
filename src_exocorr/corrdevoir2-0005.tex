\begin{corrige}{devoir2-0005}
  
  \begin{enumerate}
  \item Les dérivées partielles de $f$ en l'origine sont 
    \begin{equation}
      \partial_x f(0,0)=\lim_{h\to 0}\frac{f(h,0)-f(0,0)}{h}= \lim_{h\to 0} \frac{0}{h (|h|+0)} = 0, 
    \end{equation} et 
    \begin{equation}
      \partial_y f(0,0)=\lim_{h\to 0}\frac{f(0,h)-f(0,0)}{h}= \lim_{h\to 0} \frac{0}{h (0+ h^2)} = 0. 
    \end{equation}
  \item Si $f$ est différentiable en l'origine alors sa fonction différentielle est nulle. Cela veut dire que  $f$ est différentiable en l'origine si et seulement si 
    \[
    \lim_{(x,y)\to (0,0)} \frac{|f(x,y)|}{\sqrt{x^2+y^2}}=0. 
    \] 
    En fait, il est facile de voir que cette limite n'existe pas. Pour le démontrer nous allons utiliser la méthode des chemins. Soit $\lambda$ un nombre réel.  
    \begin{equation}
      \lim_{t\to 0} \frac{|f(t,\lambda t)|}{|t|\sqrt{1+\lambda}}=\lim_{t\to 0} \frac{|\lambda t^2|}{||t|+\lambda^2 t^2||t|\sqrt{1+\lambda}}. 
    \end{equation}
    Le terme dominant dans la limite est $\displaystyle \frac{|\lambda t^2|}{t^2\sqrt{1+\lambda}}= \frac{|\lambda|}{\sqrt{1+\lambda}}$. Comme cette quantité dépends de $\lambda$ la limite n'existe pas et la fonction $f$ n'est pas différentiable en l'origine. 
 \end{enumerate}
\end{corrige}
