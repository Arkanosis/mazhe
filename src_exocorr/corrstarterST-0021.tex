% This is part of Analyse Starter CTU
% Copyright (c) 2014
%   Laurent Claessens,Carlotta Donadello
% See the file fdl-1.3.txt for copying conditions.

\begin{corrige}{starterST-0021}

  \begin{enumerate}
  \item Soient $y_1$ et $y_2$ deux solutions de l'équation \eqref{nonhom}. La fonction différence $y_h= y_1-y_2$ est dérivable et sa dérivée est donné par $y_h'= y_1'-y_2'$, ce qui implique que 
    \begin{equation*}
     y_h'= y_1'-y_2' = (y_1+x) -(y_2+x) = y_1-y_2  = y_h. 
    \end{equation*}
  \item On remplace $y$ dans l'équation \eqref{nonhom} par une fonction de la forme  $y_p = ax+b$, où $a$ et $b$ sont des nombres réels, et on obtient des conditions sur $a $ et $b$
    \begin{equation*}
      y_p'= y_p+x \:\Rightarrow \: a= (a+1)x+b  , \qquad\text{pour tout }x\in\eR.
    \end{equation*}
On a alors que $a=b$ et $b=-1$, donc $y_p(x) = -x-1$.
  \item La solution générale de l'équation \eqref{nonhom}, $\mathcal{Y}$, est donnée par la somme de $y_p$ et de la solution générale de l'équation $y' = y$. Cette dernière équation a comme solution générale $\mathcal{Y}_h = \left\{ Ce^x\: : \: C\in\eR\right\}$, donc on a $\mathcal{Y} = \left\{ Ce^x-x-1\: : \: C\in\eR\right\}$.
  \end{enumerate}

  

\end{corrige}
