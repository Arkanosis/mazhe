% This is part of the Exercices et corrigés de mathématique générale.
% Copyright (C) 2009-2011
%   Laurent Claessens
% See the file fdl-1.3.txt for copying conditions.
\begin{corrige}{General0027}

La parabole \og sort\fg{} du disque aux points $(\pm\sqrt{3},1)$. La partie plus petite est celle qui est au dessus de l'axe $Ox$. En faisant tourner autour de $Ox$, nous calculons 
\begin{equation}
	V=2\pi\int_{0}^{\sqrt{3}}\left( (4-x^2)-\frac{ x^2 }{ 9 } \right)=\frac{ 28\sqrt{3} }{ 5 }.
\end{equation}

Pour calculer la rotation dans l'autre sens, il faut regarder sa feuille dans l'autre sens, c'est à dire inverser $x$ et $y$ dans les formules. De $y=\frac{ x^2 }{ 3 }$, nous tirons la nouvelle fonction $y=\sqrt{3x}$, et nous intégrons ça entre $1$ et $0$ :
\begin{equation}
	V=\pi\int_2^1(4-x^2)+\pi\int_1^03x=\frac{ 19\pi }{ 6 }.
\end{equation}

\end{corrige}
