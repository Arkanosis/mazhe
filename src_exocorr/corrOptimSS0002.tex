% This is part of Exercices et corrigés de CdI-1
% Copyright (c) 2011
%   Laurent Claessens
% See the file fdl-1.3.txt for copying conditions.

\begin{corrige}{OptimSS0002}

La définition de la dérivée sur le bord du domaine de définition (voir page 74) est
\begin{equation}
	f'(b)=\lim_{\substack{\epsilon\to 0\\\epsilon>0}}\frac{ f(b-\epsilon)-f(b) }{ \epsilon }.
\end{equation}
L'hypothèse dit que ce nombre est strictement positif, donc $\exists\epsilon_0$ tel que $\epsilon<\epsilon_0$ implique
\begin{equation}
	\frac{ f(b-\epsilon)-f(b) }{ \epsilon }>0,
\end{equation}
 et donc $f(b-\epsilon)-f(b)>0$, ce qui fait que $f$ est un maximum.

\end{corrige}
