% This is part of Exercices et corrigés de CdI-1
% Copyright (c) 2011
%   Laurent Claessens
% See the file fdl-1.3.txt for copying conditions.

\begin{exercice}\label{exoDevel0001}

Afin de se familiariser avec la notion de fonction de type $o(x^n)$.
\begin{enumerate}
\item Donner un exemple de fonction satisfaisant : 
\begin{enumerate}
\item $f(x)\in o(|x^3|)$,  \item $f(x)\in o(|x-1|)$,  \item $f(x)\in o(\sin x)$.\end{enumerate}

\item Vrai ou faux? (Justifier ou donner un contre-exemple):
\begin{enumerate}
\item Si $f(x)\in o(x)$ alors $xf(x)\in o(x^2)$.
\item Si $f(x)\in o(x)$ alors $\f{f(x)}{x}\in o(x)$.
\item Si $f(x)\in o(x^3)$ alors $f(x)\in o(x^2)$, $f(x)\in o(x)$, et $f(x)\in o(1)$.
\end{enumerate}
\end{enumerate}


\corrref{Devel0001}
\end{exercice}
