% This is part of Exercices et corrigés de CdI-1
% Copyright (c) 2011
%   Laurent Claessens
% See the file fdl-1.3.txt for copying conditions.

\begin{exercice}\label{exo0088}

Nous considérons l'ensemble $\bar\eR=\eR\cup\{\infty\}$. Un sous ensemble de $\bar\eR$ qui ne contient pas $\infty$ est dit ouvert si et seulement s'il est ouvert pour $\eR$, et un sous ensemble contenant $\infty$ sera ouvert si son complémentaire est compact dans $\eR$.

\begin{enumerate}

\item
Prouver que ces ouverts définissent bien une topologie sur $\bar\eR$.

\item
Prouver que $\bar\eR$ est un espace compact.

\end{enumerate}
Rendez vous compte que $\eR$ est non compact, mais qu'on l'a rendu compact en \emph{ajoutant} un point en plus. L'espace topologique ainsi défini est le \defe{\href{http://fr.wikipedia.org/wiki/Compactifié_d'Alexandroff}{compactifié}}{Compactifié} d'Alexandroff de l'ensemble des réels $\eR$.

\corrref{0088}
\end{exercice}
