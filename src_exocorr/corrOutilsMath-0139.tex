% This is part of Outils mathématiques
% Copyright (c) 2012
%   Laurent Claessens
% See the file fdl-1.3.txt for copying conditions.

\begin{corrige}{OutilsMath-0139}

    La surface est paramétrée par
    \begin{equation}
        \phi(x,y)=\begin{pmatrix}
            x    \\ 
            y    \\ 
            1-x-y    
        \end{pmatrix},
    \end{equation}
    les bornes étant données par 
    \begin{subequations}
        \begin{numcases}{}
            x\colon 0\to 2\\
            y\colon -x\to x.
        \end{numcases}
    \end{subequations}
    Pourquoi \( x\) part de zéro au lieu de \( -2\) ? Parce que pour \( x<0\), il n'y a pas de \( y\) vérifiant la condition \( -x<y<x\).
    L'intégrale à calculer est celle de la fonction \( f\) sur la surface paramétrée par \( \phi\), c'est à dire
    \begin{equation}
        \int_{\phi}f=\int_{0}^2dx\int_{-x}^xdy\,xy^2\| \frac{ \partial \phi }{ \partial x }\times\frac{ \partial \phi }{ \partial y } \|.
    \end{equation}
    L'élément de surface est calculé par
    \begin{equation}
        dS=\begin{vmatrix}
            e_x    &   e_y    &   e_z    \\
            1    &   0    &   -1    \\
            0    &   1    &   -1
        \end{vmatrix}=\begin{pmatrix}
            1    \\ 
            1    \\ 
            1    
        \end{pmatrix}.
    \end{equation}
    La norme vaut \( \sqrt{3}\) et l'intégrale recherchée est
    \begin{equation}
        \int_{\phi}f=\sqrt{3}\int_{0}^2\int_{-x}^x xy^2dydx=\sqrt{3}\frac{ 64 }{ 15 }.
    \end{equation}

\end{corrige}
