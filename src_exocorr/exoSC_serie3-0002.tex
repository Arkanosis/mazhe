\begin{exercice}\label{exoSC_serie3-0002}

	Le tableau ci-dessous donne la force électromotrice $E$ (en volts) d'une pile, en fonction de la température absolue $T$ en $\kelvin$.
	\[
		\begin{array}{|c|ccccc|}
			\hline
			\text{T}& 290	&	300	&	310	&	320	&	330 \\
			\hline
			\text{E}& 1.15053&	1.14950	&	1.14788	&	1.14656	&	1.14527  \\
			\hline
		\end{array}
	\]
	On estime que les valeurs de la force électromotrice peuvent être approchées par les valeurs d'un polynôme du troisième degré
	\begin{equation}
		E=aT^3+bT^2+cT+d.
	\end{equation}
	\begin{enumerate}

		\item
			Déterminer les constantes $a$, $b$, $c$ et $d$.
		\item
			Représenter, dans un même diagramme, le graphe du polynôme et les points correspondant aux données.
		\item
			Estimer la valeur de $E$ pour $T=\unit{316}{\kelvin}$.

	\end{enumerate}
	

\corrref{SC_serie3-0002}
\end{exercice}
