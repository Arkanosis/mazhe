% This is part of Exercices et corrigés de CdI-1
% Copyright (c) 2011
%   Laurent Claessens
% See the file fdl-1.3.txt for copying conditions.

\begin{corrige}{OutilsMath-0074}

    Nous commençons par trouver le plan passant par $(0,0,0)$ parallèle aux vecteurs $b-a$ et $c-a$. Les vecteurs 
    \begin{equation}
        \begin{aligned}[]
            \begin{pmatrix}
                1    \\ 
                1    \\ 
                2    
            \end{pmatrix},&&\begin{pmatrix}
                3    \\ 
                -2    \\ 
                1    
            \end{pmatrix}
        \end{aligned}
    \end{equation}
    doivent appartenir à $z=ax+by$. Le système est
    \begin{subequations}
        \begin{numcases}{}
            2=a+b\\
            1=3a-2b,
        \end{numcases}
    \end{subequations}
    et la solution est le plan $z=x+y$.

    Maintenant nous trouvons le plan $z=x+y+c$ qui passe par $(4,0,1)$. La solution est $c=-3$. Le plan cherché est donc
    \begin{equation}
        z=x+y-3.
    \end{equation}
    Pour vérification, si $f(x,y)=x+y-3$, nous avons bien $f(1,2)=0$, $f(0,1)=-2$ et $f(4,0)=1$.

\end{corrige}
