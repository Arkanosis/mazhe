% This is part of Analyse Starter CTU
% Copyright (c) 2014
%   Laurent Claessens,Carlotta Donadello
% See the file fdl-1.3.txt for copying conditions.

\begin{exercice}\label{exostarterST-0010}

\begin{enumerate}
\item Montrer que la fonction $\sin$ est bijective de $I=[-\dfrac{\pi}2,\dfrac{\pi}2]$ sur $J=[-1,1]$. La bijection réciproque s'appelle $\arcsin$ (on lit arcsinus), préciser son ensemble de définition.
\item Préciser les valeurs de $\arcsin(0)$,\quad $\arcsin(0,5)$,\quad $\arcsin\dfrac{\sqrt2}2$,\quad $\arcsin\dfrac{\sqrt3}2$,\quad $\arcsin(-0,5)$,\quad $\arcsin(-\dfrac{\sqrt{3}}2)$.
\item Résoudre  les équations : {\bfseries  a/} $\arcsin (x) = \dfrac{\pi}{4}$   \quad {\bfseries  b/} $\arcsin (x) = \dfrac{3\pi}{4}$
\item $\sin \cfrac{3\pi}4=\dfrac{\sqrt2}2$. Que vaut $\arcsin(\dfrac{\sqrt{2}}2)$ ? Peut-on comparer $\arcsin(\sin x)$ et $x$ ?
\item Montrer que $\arcsin$ est dérivable sur $\mathopen] -1 , 1 \mathclose[$ avec
			\begin{equation}
				\big( \arcsin(x) \big)'=\frac{1}{ \sqrt{1-x^2} }.
			\end{equation}
			Conseil : utiliser la formule $\cos^2(t)+\sin^2(t)=1$ pour tout $t\in\eR$.
\item Représenter la fonction $\arcsin$. 
\end{enumerate}


\corrref{starterST-0010}
\end{exercice}
