% This is part of Exercices et corrigés de CdI-1
% Copyright (c) 2011
%   Laurent Claessens
% See the file fdl-1.3.txt for copying conditions.

\begin{corrige}{OutilsMath-0097}

    \begin{enumerate}
        \item
            Le potentiel $V(x,y,z)$ doit vérifier les deux équations
            \begin{equation}
                \begin{aligned}[]
                    \frac{ \partial V }{ \partial x }&=6xy+4x^3y^4\\
                    \frac{ \partial V }{ \partial y }&=3x^2+4x^4y^3.
                \end{aligned}
            \end{equation}
            Il est facile de voir que la fonction $V(x,y,z)=3x^2y+x^4y^4$ satisfait à ces deux équations en même temps.
        \item
            Étant donné que le champ dérive d'un potentiel, son rotationnel est nul :
            \begin{equation}
                \nabla\times F=\nabla\times(\nabla V)=0.
            \end{equation}
        \item
            Le rotationnel du champ proposé est donné par
            \begin{equation}
                \nabla\times F=\begin{pmatrix}
                    e_x    &   e_y    &   e_z    \\
                    \partial_x    &   \partial_y    &   \partial_y    \\
                    y\sin(xy)    &   -x\sin(xy)    &   0
                \end{pmatrix}=-2xy\cos(xy)+2\sin(xy).
            \end{equation}
        \item
            Le fait que le rotationnel de $F$ soit nul implique que $F$ ne peut pas être un champ de gradients.
    \end{enumerate}

\end{corrige}
