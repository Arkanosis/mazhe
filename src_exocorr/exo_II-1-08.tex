% This is part of the Exercices et corrigés de CdI-2.
% Copyright (C) 2008, 2009
%   Laurent Claessens
% See the file fdl-1.3.txt for copying conditions.


\begin{exercice}\label{exo_II-1-08}

Lorsqu'on étudie le mouvement d'un point matériel \emph{pesant} soumis à une résistance de l'air fonction de la vitesse, on peut montrer que l'équation différentielles de l'\href{http://fr.wikipedia.org/wiki/Hodographe}{hodographe} des vitesses est 
\begin{equation}
	\frac{1}{ r }\frac{ dr }{ d\theta }=\frac{ \sin\theta+R(r) }{ \cos\theta }
\end{equation}
où $r,\theta$ sont les coordonnées polaires et $R$ est la fonction caractérisant le type de résistance.

Résoudre l'équation différentielle pour $R(r)=kr^{\alpha}$. Quelle est la nature de l'hodographe des vitesses lorsque $R(r)=kr$ ?

\corrref{_II-1-08}
\end{exercice}
