% This is part of Exercices et corrigés de CdI-1
% Copyright (c) 2011
%   Laurent Claessens
% See the file fdl-1.3.txt for copying conditions.

\begin{exercice}\label{exo0023}

Déterminez si les limites suivantes existent et dans l'affirmative calculez les en utilisant, s'il y a lieu, la règle de l'Hospital ou la règle de l'étau. 
\begin{multicols}{2}
\begin{enumerate}
	\item $ \lim_{x \rightarrow  +\infty} \frac{x+1}{x^2+2}$
	\item $ \lim_{x \rightarrow  +\infty} \frac{\sin(x)}{x} $
	\item $ \lim_{x \rightarrow  0} \frac{\sin(x)}{x} $
	\item $ \lim_{x \rightarrow  +\infty}  \frac{x ^n}{e ^x} $
	\item $ \lim_{x \rightarrow  +\infty} (1 + \frac{a}{x})^x $
	\item $ \lim_{x \rightarrow  0} (\frac{1}{\sin(x)} - \frac{1}{x} )$
	\item $ \lim_{x \rightarrow  +\infty} \cos( 2 \pi x) $
	\item\label{Item0023h} $ \lim_{x \rightarrow  +\infty} \frac{x}{\sin(x)+2} + \ln(x)\cos(x) $
	\item $ \lim_{x \rightarrow  +\infty} \frac{ \ln(x)(\sin(x) + 2)}{x} $
	\item $ \lim_{x \rightarrow  +\infty} x ^\frac{1}{x} $
\end{enumerate}
\end{multicols}
%
(Sont présentés ici différents types de problèmes auxquels on peut être confronté lors du calcul de limites de fonctions. Cet exercice est un exercice de drill : n'essayez pas de justifier \emph{à fond} chaque étape du calcul.)
%

\corrref{0023}
\end{exercice}
