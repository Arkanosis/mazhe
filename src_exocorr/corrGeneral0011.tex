% This is part of the Exercices et corrigés de mathématique générale.
% Copyright (C) 2009-2011
%   Laurent Claessens
% See the file fdl-1.3.txt for copying conditions.
\begin{corrige}{General0011}

En règle générale, pour calculer des limites de fonctions qui se présentent sous la forme $f(x)^{g(x)}$, il faut faire la manipulation suivante :
\begin{equation}	\label{EqManipExpLnlimite}
	f(x)^{g(x)}= e^{\ln\big( f(x)^{g(x)} \big)}= e^{g(x)\ln\big( f(x) \big)}.
\end{equation}
À partir de là, nous devons calculer la limite de $g(x)\ln\big( f(x) \big)$, et puis prendre l'exponentielle en justifiant par le faire que l'exponentielle est une application continue (et donc commute avec la limite).

\begin{enumerate}

\item
\item
\item
\item
\item
En vertu de la manipulation \eqref{EqManipExpLnlimite}, nous devons calculer
\begin{equation}		\label{EqPassageE}
	\lim_{x\to 0} \sin(x)\ln(x)=\lim_{x\to 0} \frac{ \ln(x) }{ 1/\sin(x) }=\lim_{x\to 0} \frac{ 1/x }{ -\cos(x)/\sin(x)^2 }=-\lim_{x\to 0} \frac{1}{ x }\frac{ \sin(x)^2 }{ \cos(x) }.
\end{equation}
Nous éliminons un des sinus et le $\frac{1}{ x }$ en utilisant la limite $\lim_{x\to 0} \frac{ \sin(x) }{ x }=1$. La limite \eqref{EqPassageE} vaut en définitive zéro, de telle sorte que
\begin{equation}
	\lim_{x\to_> 0} x^{\sin(x)}= e^{0}=1.
\end{equation}

\item
\item
\item
\item
\item

Nous avons
\begin{equation}
	\lim_{x\to \infty} \left( \frac{ 2 }{ x }+1 \right)^x=\lim_{x\to \infty} \exp\left( x\ln(\frac{ 2 }{ x }+1) \right).
\end{equation}
La limite à calculer est donc la suivante qui se traite par l'Hospital :
\begin{equation}
	\lim_{x\to \infty} x\ln\left( \frac{ 2 }{ x }+1 \right)=\lim_{x\to \infty} \frac{  (-2/x^2)/\big( \frac{ 2 }{ x }+1 \big)     }{ -1/x^2 }=2.
\end{equation}
La limite recherchée est donc $e^2$.

\item
Après avoir fait le coup de l'exponentielle, nous devons calculer la limite
\begin{equation}
	\lim_{x\to 1} (x-1)\ln\big( \ln(x) \big)=\lim_{x\to 1} =\lim_{x\to } \frac{ \ln\big( \ln(x) \big) }{ \frac{1}{ x-1 } }=-\lim_{x\to } \frac{ (x-1)^2 }{ x\ln(x) }=\lim_{x\to 1} \frac{ 2(x-1) }{ 1/x }=0.
\end{equation}
La réponse attendue est donc $ e^{0}=1$.

\item
\item
\item
\item


\end{enumerate}

\end{corrige}
