% This is part of the Exercices et corrigés de mathématique générale.
% Copyright (C) 2009
%   Laurent Claessens
% See the file fdl-1.3.txt for copying conditions.
\begin{corrige}{Inter0012}

À tous les coups, la formule à utiliser est la formule \eqref{EqLongArcCourbe}, la seule vraie difficulté est de calculer l'intégrale.

\begin{enumerate}

\item
Ici, $y'(x)=-2x/(1-x^2)$, et nous devons calculer l'intégrale
\begin{equation}
	\begin{aligned}[]
		l&=\int_0^{1/2}\sqrt{ 1+\frac{ 4x^2 }{ (1-x^2)^2 } }\\
			&=\int_0^{1/2}\sqrt{\frac{ (x^2+1)^2 }{ (1-x_2)^2 }}\\
			&=\int_0^{1/2}\frac{ x^2+1 }{ 1-x^2 }.
	\end{aligned}
\end{equation}
La division euclidienne de $x^2+1$ par $-x_2+1$ donne que
\begin{equation}
	\frac{ x^2+1 }{ 1-x^2 }=-1+\frac{ 2 }{ 1-x^2 },
\end{equation}
de telle sorte que la longueur demandée vaut
\begin{equation}
	\begin{aligned}[]
		l&=\int_0^{1/2}-1+\int_0^{1/2}\frac{ 2 }{ 1-x^2 }\\
			&=\left[ \ln(1+x)+1-x-\ln(1-x) \right]^{1/2}_0\\
			&=\ln(3)-\frac{ 1 }{2}.
	\end{aligned}
\end{equation}


\item
La longueur d'arc de $y=x^{3/2}$ avec $0\leq x\leq 5$. L'intégrale à calculer est
\begin{equation}
	l=\int_0^5\sqrt{1+\left( \frac{ 3 }{ 2 }x^{1/2} \right)^{2}}dx=\int_0^5\sqrt{1+\frac{ 9x }{ 4 }}dx.
\end{equation}
Cette intégrale s'effectue en faisant le changement de variables $t=1+\frac{ 9x }{ 4}$, $dx=\frac{ 4 }{ 9 }dx$, ce qui amène à
\begin{equation}
	\int\sqrt{t}\frac{ 4 }{ 9 }dt=\frac{ 4 }{ 9 }\frac{ t^{3/2} }{ 3/2 }=\frac{ 8 }{ 27 }\left( 1+\frac{ 9x }{ 4 } \right)^{3/2},
\end{equation}
et donc
\begin{equation}
	l=\frac{ 8 }{ 27 }\left[ \left( 1+\frac{ 9x }{ 4 } \right)^{3/2} \right]_0^5=\frac{ 335 }{ 27 }.
\end{equation}



\end{enumerate}

\end{corrige}
