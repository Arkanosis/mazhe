% This is part of Exercices et corrigés de CdI-1
% Copyright (c) 2011
%   Laurent Claessens
% See the file fdl-1.3.txt for copying conditions.

\begin{corrige}{OutilsMath-0098}

    Nous avons d'abord
    \begin{equation}
        \sigma'(t)=\begin{pmatrix}
            -3\cos^2(t)\sin(t)    \\ 
            3\sin^2(t)\cos(t)    
        \end{pmatrix}.
    \end{equation}
    Par conséquent,
    \begin{equation}
        \| \sigma'(t) \|^2=9\cos^2(t)\sin^2(t)\big( \cos^2(t)+\sin^2(t) \big)=9\cos^2(t)\sin^2(t).
    \end{equation}
    En ce qui concerne la longueur,
    \begin{equation}
        l(\sigma)=\int_0^{\pi/2}3\sin(t)\cos(t)dt.
    \end{equation}
    Cette intégrale se règle avec le changement de variables $u=\sin(t)$, $du=\cos(t)dt$. Lorsque $t=0$, $u=0$ et lorsque $t=\pi/2$, $u=1$. Par conséquent
    \begin{equation}
        l(\sigma)=3\int_0^1 udu=\frac{ 3 }{2}.
    \end{equation}

    Une autre façon de résoudre l'intégrale est d'écrire $\sin(t)\cos(t)=\frac{ \sin(2t) }{2}$. Nous nous retrouvons alors avec
    \begin{equation}
        l(\sigma)=\frac{ 3 }{ 2 }\int_0^{\pi/2}\sin(2t)dt.
    \end{equation}
    Une primitive de $\sin(2t)$ est facile à trouver : c'est $-\cos(2t)/2$.

    Une troisième façon de calculer
    \begin{equation}
        I=\int_0^{\pi/2}\cos(t)\sin(t)dt
    \end{equation}
    est de faire par parties :
    \begin{equation}
        \begin{aligned}[]
            u&=\cos(t)&u'&=-\sin(t)\\
            v'&=\sin(t)&v&=-\cos(t).
        \end{aligned}
    \end{equation}
    De la sorte,
    \begin{equation}
        I=\left[ -\cos^2(t) \right]_0^{\pi/2}-\int_0^{\pi/2}\sin(t)\cos(t)dt,
    \end{equation}
    c'est à dire $I=-1-I$, et par conséquent $I=-1/2$.
\end{corrige}
