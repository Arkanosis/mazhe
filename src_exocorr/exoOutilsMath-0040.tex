% This is part of Exercices et corrigés de CdI-1
% Copyright (c) 2011
%   Laurent Claessens
% See the file fdl-1.3.txt for copying conditions.

\begin{exercice}\label{exoOutilsMath-0040}

    Une personne au haut d'une colline de hauteur $h_0>0$ lance verticalement une masse avec une vitesse (verticale) $v_0>0$. La hauteur de la masse en fonction du temps est donnée par
    \[ 
        h(t)=h_0-\frac{ gt^2 }{2}+v_0t.
    \]
    \begin{enumerate}
        \item
            Pour quelle valeur de $t$ la hauteur est-elle maximale ?
        \item
            Quelle est la hauteur maximale atteinte ?
    \end{enumerate}
    Les réponses peuvent évidemment dépendre de $h_0$, $g$ et $v_0$.

\corrref{OutilsMath-0040}
\end{exercice}
