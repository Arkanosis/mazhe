% This is part of Exercices de mathématique pour SVT
% Copyright (c) 2011
%   Laurent Claessens et Carlotta Donadello
% See the file fdl-1.3.txt for copying conditions.

\begin{exercice}\label{exoECdecembre2010-0001}

	Modèle malthusien contrôlé. Soit la suite $(u_n)_{n\in\eN}$ définie par
	\begin{equation}\nonumber
		\begin{cases}
			u_{n+1}=au_n+b,	&	\textrm{pour tous } n\in\eN, \, n>0\\
			u_0=x
		\end{cases}
	\end{equation}
	où $a\neq 1$, $b$ et $x\geq 0$ sont des nombres réels.
	\begin{enumerate}
        \item Calculer les premiers $4$ termes de la suite si $a=2$ et $x=10$.
		\item
			Montrer que $u_n=a^n(x+\frac{ b }{ a-1 })-\frac{ b }{ a-1 }$ pour tout $n\in\eN$.
		%\item
		%	En supposant que $a=2$ et $x=100$ trouver $b$ tel que la population se stabilise, c'est à dire telle que la population tende vers une valeur finie ou nulle.
	\end{enumerate}


\corrref{ECdecembre2010-0001}
\end{exercice}
