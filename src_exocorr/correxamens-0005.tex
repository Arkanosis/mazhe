% This is part of Mes notes de mathématique
% Copyright (c) 2012
%   Laurent Claessens
% See the file fdl-1.3.txt for copying conditions.

\begin{corrige}{examens-0005}


    L'intégrale sur le rectangle \( \mathopen[ 0 , 1 \mathclose]\times \mathopen[ 0 , 2 \mathclose]\) revient à dire \( x\colon 0\to 1\) et \( y\colon 0\to 2\), ce qui revient à calculer l'intégrale
    \begin{equation}
        \int_0^1dx\int_0^2dy(x^2y^2+xy^3).
    \end{equation}
    \begin{verbatim}
sage: f(x,y)=x**2*y**2+x*y**3
sage: f.integrate(x,0,1).integrate(y,0,2)
(x, y) |--> 26/9
    \end{verbatim}
    Le résultat est \( 26/9\).

    En ce qui concerne la seconde intégrale, la droite qui joint \( (0,1)\) à \( (1,0)\) est la droite \( y=-x+1\). Nous intégrons donc avec \( x\colon 0\to 1\) et \( y\colon 0\to -x+1\).
    \begin{verbatim}
    sage: f(x,y)=x**2*y
    sage: f.integrate(y,0,-x+1)
    (x, y) |--> 1/2*(x^2 - 2*x + 1)*x^2
    sage: f.integrate(y,0,-x+1).integrate(x,0,1)
    (x, y) |--> 1/60
    \end{verbatim}
    La réponse est donc \( 1/60\).


\end{corrige}
