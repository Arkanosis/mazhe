% This is part of the Exercices et corrigés de mathématique générale.
% Copyright (C) 2009
%   Laurent Claessens
% See the file fdl-1.3.txt for copying conditions.
\begin{corrige}{Lineraire0016}

	\begin{enumerate}

		\item
			Oui parce que deux vecteurs dans $\eR^3$, il devraient être multiples l'un de l'autre pour ne pas être libres.
		\item
			Oui, base canonique

		\item
			Oui : tout les vecteurs sont libres avec $(0,0,0)$.
		\item
			On utilise la critère du déterminant (voir exercice \ref{exoLineraire0015}). Ici nous avons
			\begin{equation}
				\begin{vmatrix}
					2	&	1	&	4	\\
					1	&	4	&	9	\\
					-3	&	0	&	-3
				\end{vmatrix}
				=-3(1\cdot 9-4\cdot 4)=0.
			\end{equation}
			Donc le système n'est pas libre.

		\item
			Plus de $3$ vecteur dans $\eR^3$ ne peuvent pas être libres.
		\item
			Le déterminant
			\begin{equation}
				\begin{vmatrix}
					1	&	0	&	1	\\
					0	&	-5	&	7	\\
					1	&	0	&	-2
				\end{vmatrix}
			\end{equation}
			est non nul, donc la partie est libre.

		\item
			Un seul vecteur, c'est toujours libre.

	\end{enumerate}
	

\end{corrige}
