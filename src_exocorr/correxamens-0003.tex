% This is part of Mes notes de mathématique
% Copyright (c) 2012
%   Laurent Claessens
% See the file fdl-1.3.txt for copying conditions.

\begin{corrige}{examens-0003}

    \begin{enumerate}
        \item
            Le rotationnel est calculé de façon usuelle :
            \begin{equation}
                \nabla\times F=\begin{vmatrix}
                    e_x    &   e_y    &   e_z    \\
                    \partial_x    &   \partial_y    &   \partial_z    \\
                    y\sin(xy)    &   -x\sin(xy)    &   0
                \end{vmatrix}=\big( -2xy\cos(xy)-2\sin(xy) \big)e_z.
            \end{equation}
        \item
            Le fait que le rotationnel ne soit pas nul implique que ce champ ne dérive pas d'un potentiel (n'est pas un champ de gradient).

        \item

            Nous devons trouver une fonction \( f(x,y)\) telle que
            \begin{subequations}
                \begin{numcases}{}
                    \frac{ \partial f }{ \partial x }=6xy+4x^3y^4\\
                    \frac{ \partial f }{ \partial y }=3x^2+4x^4y^3.     \label{EqsubTUWvAA}
                \end{numcases}
            \end{subequations}
            En intégrant la première équation par rapport à \( x\), nous trouvons
            \begin{equation}    \label{EqSBVRsY}
                f(x,y)=3x^2y+x^4y^4+C(y)
            \end{equation}
            où \( C(y)\) est une fonction de \( y\) à déterminer par la deuxième équation. Nous dérivons la fonction \eqref{EqSBVRsY} par rapport à \( y\) et nous confrontons le résultat avec la prescription \eqref{EqsubTUWvAA} :
            \begin{equation}
                \frac{ \partial f }{ \partial y }=3x^2+4x^4y^3+C'(y)
            \end{equation}
            soit être égal à 
            \begin{equation}
                3x^4+4x^4y^3.
            \end{equation}
            Nous voyons que prendre \( C(y)=0\) fait l'affaire. La réponse est donc
            \begin{equation}
                f(x,y)=3x^2y+x^4y^4.
            \end{equation}
            
    \end{enumerate}

\end{corrige}
