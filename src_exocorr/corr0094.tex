% This is part of Exercices et corrigés de CdI-1
% Copyright (c) 2011
%   Laurent Claessens
% See the file fdl-1.3.txt for copying conditions.

\begin{corrige}{0094}

\begin{enumerate}

\item
La fonction est continue sur un compact, donc uniformément continue.

\item
Cette fonction est une restriction d'une fonction uniformément continue.

\item
Nous considérons la fonction
\begin{equation}
	\begin{aligned}
		f\colon \mathopen]0,1\mathclose]&\to  \eR \\
		x&\mapsto x\sin(\frac{1}{ x }). 
	\end{aligned}
\end{equation}
Étant donné que $\lim_{x\to 0}f(x)=0$ (le prouver), la fonction
\begin{equation}
	\begin{aligned}
		g\colon \mathopen[0,1\mathclose]&\to  \eR \\
		x&\mapsto 
		\begin{cases}
			f(x) 	&	\text{si }x\neq 0\\
			0	&	 \text{si }x=0
\end{cases}
	\end{aligned}
\end{equation}
est continue sur $[0,1]$ et y est donc uniformément continue. Maintenant, la fonction $f$ est la restriction de $g$ au domaine $\mathopen]0,1\mathclose]$, et y est donc uniformément continue.


\item
Vue sur l'intervalle $[a,1]$, la fonction $x\mapsto\sin(\frac{1}{ x })$ est continue sur un compact, et donc uniformément continue.o

\item
Pour tout $\delta>0$, il existe un choix de $x$ et $y$ avec $| x-y |\leq\delta$, mais avec $\frac{1}{ x }=\frac{ \pi }{ 2 }+2k\pi$ et $\frac{1}{ y }=\frac{ 3\pi }{2}+2k'\pi$. Pour un tel choix, nous avons
\begin{equation}
	\left| \sin(\frac{1}{ x })-\sin(\frac{1}{ y }) \right| =2,
\end{equation}
donc nous n'avons pas uniforme convergence.

\item
Cette fonction n'est pas uniformément continue parce que si $\epsilon$ est donné, nous allons montrer qu'aucun choix de $\delta$ ne convient. Il suffit en effet de prendre $x$ assez grand pour que $| x^2-(x+\delta)^2 |>2\pi$, et alors, dans la boule $B(x,\delta)$, il y a un endroit où la fonction $\sin(x^2)$ prend la valeur $1$ et un endroit où elle prend la valeur $-1$.

\item
La fonction est continue sur un compact.

\item
 Nous savons que la dérivée de $\sqrt{x}$ est bornée sur $\mathopen[1,\infty[$, donc la fonction y est uniformément continue en vertu de l'exercice \ref{exo0091}.

\item
La fonction $\sqrt{x}$ étant uniformément continue sur $[0,2]$ et sur $\mathopen[1,\infty[$, elle est uniformément continue sur $\mathopen[0,\infty[$ par le principe des intervalles chevauchant, exercice \ref{exo0092}.

\item
En utilisant l'astuce (habituelle) $x^x= e^{\ln(x^x)}= e^{x\ln(x)}$, nous pouvons montrer que
\begin{equation}
	\lim_{x\to 0^+} x^x=1.
\end{equation}
Nous pouvons donc prolonger par continuité la fonction $x^x$ sur l'intervalle compact $\mathopen[0,1\mathclose]$ en disant qu'elle vaut $1$ en $x=0$. Cette prolongation est uniformément continue, et donc la fonction de départ est uniformément continue.

\item
Étant donné que $\lim_{x\to 0}\ln(x)=-\infty$, pour tout $\delta$, il existe $x$ et $y$ tels que $| x-y |<\delta$ et avec $| \ln(x)-\ln(y) |>1$.

\item
Cette fonction est uniformément continue parce que sa dérivée est bornée.

\end{enumerate}

\end{corrige}
