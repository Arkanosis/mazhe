% This is part of Exercices et corrigés de CdI-1
% Copyright (c) 2011
%   Laurent Claessens
% See the file fdl-1.3.txt for copying conditions.

\begin{exercice}\label{exoVariete0004}

Trouver les extrema de la fonction $d:\eR^6 \rightarrow \eR: (v,v') \mapsto \|v-v'\|^2$, c'est à dire
\begin{equation}
	(x,y,z,x',y',z') \mapsto (x-x')^2+(y-y')^2+(z- z')^2
\end{equation}
relativement au plan $S :=\{ (x,y,z,x',y',z') \in R^6 \mid x=y=z, x'=1, y'=0 \}$.

Aide : reconsidérer le même problème relativement à l'ensemble $S\cap C$ où
\begin{equation}
	C :=\{ (x,y,z,x',y',z') \in R^6 \mid (x-x')^2+(y-y')^2+(z- z')^2<10 \}
\end{equation}
qui est borné et à sa fermeture $\overline{S \cap C}$ qui est compact mais n'est plus une variété, pour prouver que le point que vous obtenez par la méthode de Lagrange est bien un minimum.


\corrref{Variete0004}
\end{exercice}
