% This is part of Exercices de mathématique pour SVT
% Copyright (C) 2010,2016
%   Laurent Claessens et Carlotta Donadello
% See the file fdl-1.3.txt for copying conditions.

\begin{exercice}\label{exoTD3-0007}

	Modèle à ressources limitées (compétition). Soit la suite $(u_n)_{n\in\eN}$ définie par
	\begin{equation}
		\begin{cases}
			u_{n+1}=u_nf(u_n)	&	\forall n\in \eN_0\\
			u_0=x,
		\end{cases}
	\end{equation}
	où $x\geq$ est un nombre réel et $f\colon \mathopen[ 0 , \infty [\to \mathopen[ 0 , a \mathclose]$ est la fonction définie par
	\begin{equation}
		f(y)=\begin{cases}
			a	&	\text{si }y<2\\
			\frac{ 2a }{ y }	&	 \text{si }y\geq 2.
		\end{cases}
	\end{equation}
	Déterminer la limite de la suite $(u_n)_{n\in\eN}$ dans les cas suivants.
	\begin{enumerate}
		\item
			$x=3$ et $a=\frac{1}{ 2 }$. Conseil : commencer par montrer que $0<u_n<2$ pour tout $n\geq 1$.
		\item
			$x=1$ et $a=2$.
	\end{enumerate}

\corrref{TD3-0007}
\end{exercice}
