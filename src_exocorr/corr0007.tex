% This is part of Exercices et corrigés de CdI-1
% Copyright (c) 2011
%   Laurent Claessens
% See the file fdl-1.3.txt for copying conditions.

\begin{corrige}{0007}

\begin{enumerate}
\item La suite $k\mapsto \frac{1}{ k }$ est monotone décroissante, bornée vers le bas par zéro, elle est donc convergente. Il n'est pas compliqué de prouver qu'elle converge vers zéro.
\item Le candidat limite est $0$, et en effet,
\begin{equation}
	\left\|  \frac{ (-1)^k }{ k }-0 \right\|=\left| \frac{ (-1)^k }{ k }\right|=\frac{1}{ k } \to 0,
\end{equation}
donc le critère de convergence s'applique.
\item Étant donné que $| i^n |=1$ dans $\eC$, le raisonnement du point précédent s'applique.
\item $\frac{1}{ k^2 }$ est une sous suite de $\frac{1}{ k }$ qui converge,
\item $k\mapsto\frac{1}{ k+3 }$ est également une sous suite (décalée de $3$).
\end{enumerate}


\end{corrige}
