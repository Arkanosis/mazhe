% This is part of the Exercices et corrigés de mathématique générale.
% Copyright (C) 2009-2011
%   Laurent Claessens
% See the file fdl-1.3.txt for copying conditions.
\begin{corrige}{General0021}

\begin{enumerate}

\item
VRAI : sa dérivée est positive. On peut aussi justifier en disant que c'est la surface sous une courbe toujours positive, donc plus on prend un domaine large, plus la surface est grande.

\item
FAUX
\item
FAUX
\item
VRAI parce que la fonction est strictement croissante avec $\lim_{x\to\infty}f(x)=0$.
\item
FAUX : une fonction croissante n'est ni impaire, ni paire
\item
FAUX
\item
FAUX : fonction croissante
\item
FAUX
\item
Calculons la dérivée seconde. D'abord, $F'(t)= e^{-t^2}$, ensuite $F''(t)=-2t e^{-t^2}$. La dérivée seconde change effectivement de signe en $t=0$, donc il y a un point d'inflexion.

\end{enumerate}

Pour classer les nombres dans l'ordre, remarquons que la fonction est une surface sous une fonction positive, donc les nombres proposés sont toujours positifs (donc zéro est le plus petit). D'autre part, la fonction est toujours plus petite que $1$, donc les intégrales sur des domaines de taille $1$ sont toujours plus petites que $1$. Reste à classer les autres nombres.

Parce que la fonction est paire, la surface entre $-3$ et $-2$ est la même qu'entre $2$ et $3$. De plus, la fonction est décroissante lorsqu'on s'éloigne de $0$, donc la surface proche de zéro est plus grande que les deux autres.

\end{corrige}
