% This is part of the Exercices et corrigés de CdI-2.
% Copyright (C) 2008, 2009
%   Laurent Claessens
% See the file fdl-1.3.txt for copying conditions.


\begin{corrige}{_I-1-15}

Afin de majorer la différence $| f_k(x)-f_{k-1}(x) |=| f\big( f_{k-1}(x) \big)-f\big( f_{k-2}(x) \big) |$, nous utilisons la formule des accroissements finis sur $f$ : pour tout $x_1$, $x_2\in[a,b]$, il existe un $x^*\in[x_1,x_2]$ tel que $f(x_2)-f(x_1)=f'(x^*)(x_2-x_1)$. Nous obtenons successivement 
\begin{equation}
	\begin{aligned}[]
		| f_k(x)-f_{k-1}(x) |	&=	\Big| f\big( f_{k-1}(x) \big)-f\big( f_{k-2}(x) \big) \Big|\\
					&=	\Big| f'(x^*)\big( f_{k-2}(x)-f_{k-1}(x) \big)  \Big|\\
					&\leq	\lambda\big| f_{k-2}(x)-f_{k-1}(x) \big|\\
					&\leq	\lambda^2\big| f_{k-3}(x)-f_{k-2}(x) \big|\\
					&\qquad\vdots	\\
					&\leq	\lambda^{k-1}\big| f_{k-k}(x)-f_{k-(k-1)}(x) \big|.
	\end{aligned}
\end{equation}
où $\lambda=\sup_{[a,b]}| f'(x) |$. Nous avons donc la majoration
\begin{equation}
	| f_k(x)-f_{k-1}(x)|\leq \lambda^{k-1}| f(x)-x |.
\end{equation}
Si $M$ est le maximum de $| f(x)-x |$ sur le compact $[a,b]$, alors
\begin{equation}
	| f_k(x)-f_{k-1}(x)|\leq M\lambda^{k-1},
\end{equation}
mais si $\lambda<1$, la série $\sum_k\lambda^k$ converge. Le critère de Weierstrass (théorème \ref{ThoCritWeierstrass}) montre alors que la série
\begin{equation}		\label{EqSeriegfkmoinsfkmoinsun}
	g(x)=\sum_{k=1}^{\infty}\big( f_k(x)-f_{k-1}(x) \big)
\end{equation}
converge uniformément sur $[a,b]$. La suite des sommes partielles de la série \eqref{EqSeriegfkmoinsfkmoinsun} est donnée par $f_1-f_0$, $-f_0+f_2$, $-f_0+f_3$,\ldots c'est à dire que
\begin{equation}
	g(x)=-x+\lim_{n\to\infty}f_n(x).
\end{equation}
On peut exprimer ce résultat autrement en disant que $g(x)$ est la limite de ses sommes partielles qui sont données par $g_n(x)=-x+f_n(x)$. Nous avons démontré que le membre de gauche a une limite uniforme, donc le membre de droite possède également une limite uniforme. Ceci prouve que la suite de fonctions $\{ f_n(x) \}$ converge uniformément vers une certaine fonction $C(x)$.

Afin de prouver que la fonction $C(x)$ est constante, nous majorons la différence de $f_n$ prise en deux points différents :
\begin{equation}
	\begin{aligned}[]
		\big| f_n(x_1)-f_n(x_2)  \big|	&=	\Big|  f\big( f_{n-1}(x_1) \big) - f\big(f_{n-1}(x_2) \big)   \Big|\\
						&= \big| f'(x_1^*)\big|	\big|  f_{n-1}(x_1) - f_{n-1}(x_2)  \big|\\
						&\qquad\vdots\\
						&=\big| f'(x_1^*)\big|\ldots	\big| f'(x_n^*)\big|| x_1-x_2  \big|
	\end{aligned}
\end{equation}
où les points $n_i^*$ sont des points donnés par le théorème des accroissements finis. La norme $| x_1-x_{2} |$ est évidement majorée par $| b-a |$, tandis que les nombres $| f'(x^*_j) |$ sont tous majorés par $\lambda<1$. Au final, nous avons
\begin{equation}
	\big| f_n(x_1)-f_n(x_2)  \big|\leq \lambda^n| b-a |.
\end{equation}
En prenant $n$ assez grand, cela peut être rendu aussi petit que souhaité. La fonction $C(x)$ est donc constante.

L'unicité de $C$ en tant que point fixe de $f$ est facile : en effet, si $F$ est un point fixe de $f$, alors $f_1(F)=f_2(F)=\ldots$ Donc le point $F$ est un point fixe de la limite des $f_n$. Donc seule la limite vers laquelle $f_n$ tend peut être un point fixe de $f$.

Prouvons maintenant que $C$ est effectivement un point fixe de $f$. Par définition, nous avons $f_n(C)=f\big( f_{n-1}(C) \big)$. Soit $\epsilon>0$. En vertu de la continuité des fonctions $f_n$, il existe un $N$ tel que $n>N$ implique
\begin{equation}
	\begin{aligned}[]
		f_n(C)=C+\epsilon_1\\
		f_{n-1}(C)=C+\epsilon_2
	\end{aligned}
\end{equation}
avec $| \epsilon_1 |$ et $|\epsilon_2|$ plus petits que $\epsilon$. Pour un tel $n$, nous avons alors $C+\epsilon_1=f(C+\epsilon_2)$. Par la continuité de $f$, il existe un $\epsilon_3$ avec $| \epsilon_3 |<\epsilon$ tel que $f(C+\epsilon_2)=f(C)+\epsilon_3$ (choisir $n$ suffisamment grand pour que $\epsilon_2$ soit suffisamment petit. Nous avons donc
\begin{equation}
	f(C)=C+\epsilon_1+\epsilon_3
\end{equation}
où $\epsilon_1$ et $\epsilon_3$ peuvent être arbitrairement petits.

\end{corrige}
