\begin{corrige}{Exosenvrac-0015A}

    \begin{enumerate}
        \item
            La fonction \( f_1\) est partout strictement croissante parce que sa dérivée, \( f'_1(x)=3x^2\), est partout positive. Notez que \( f'(0)=0\), mais comme cela n'arrive qu'\`a un seul point et la d\'eriv\'ee ne change pas de signe \`a droite et \`a gauche de $x=0$,  la fonction reste strictement croissante.
            
            La fonction réciproque se trouve en résolvant par rapport à \( x\) l'équation
            \begin{equation}
                y=x^3+1.
            \end{equation}
            La solution est
            \begin{equation}
                x=\sqrt[3]{y-1}.
            \end{equation}
        \item
            La fonction \( f_2\) demande immédiatement \( x\neq 2\). En ce qui concerne la dérivée,
            \begin{equation}
                f'_2(x)=\frac{1}{ x-1 }-\frac{ x }{ (x-2)^2 }=\frac{ -2 }{ (x-2)^2 }.
            \end{equation}
            La fonction est toujours décroissante (sauf évidemment en \( x=2\) où la fonction n'existe pas).
            En ce qui concerne la fonction réciproque, il faut résoudre
            \begin{equation}
                y=\frac{ x }{ x-2 }
            \end{equation}
            pour obtenir \( y\) en fonction de \( x\). Nous avons successivement
            \begin{subequations}
                \begin{align}
                    y&=\frac{ x }{ x-2 }\\
                    y(x-2)&=x\\
                    yx-2y-x&=0\\
                    x(y-1)&=2y\\
                    x&=\frac{ 2y }{ y-1 }
                \end{align}
            \end{subequations}
            La fonction réciproque est donc
            \begin{equation}
                f_2^{-1}(y)=\frac{ 2y }{ y-1 }.
            \end{equation}
    \end{enumerate}

\end{corrige}
