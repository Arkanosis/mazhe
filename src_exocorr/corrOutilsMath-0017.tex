% This is part of Exercices et corrigés de CdI-1
% Copyright (c) 2011,2015
%   Laurent Claessens
% See the file fdl-1.3.txt for copying conditions.

\begin{corrige}{OutilsMath-0017}



	\newcommand{\CaptionFigExoProjection}{Pour l'exercice \ref{exoOutilsMath-0017}. À partir du point $A$, il faut trouver quel vecteur tombe perpendiculairement à la droite contenant $w$.}
\input{pictures_tex/Fig_ExoProjection.pstricks}

Étant donné que la projection est linéaire, nous allons cherche d'abord à déterminer \( \pr_{(1,0,1)}(1,1,1)\); ensuite il suffira de multiplier la réponse par \( 1/\sqrt{3}\).

	La projection dont nous parlons sera certainement un multiple de $(1,0,1)$. Tout l'exercice se réduit à savoir quel multiple. Soit
	\begin{equation}
		P(\lambda)=\lambda\begin{pmatrix}
			1	\\ 
			0	\\ 
			1	
		\end{pmatrix}.
	\end{equation}
	Il faut trouver $\lambda$ de telle manière que le vecteur qui joint $(1,1,1)$ à $P(\lambda)$ soit perpendiculaire à $(1,0,1)$, voir figure \ref{LabelFigExoProjection}.
	
	
	Si nous nommons $v(\lambda)$ le vecteur qui va de $(1,1,1)$ à $P(\lambda)$, nous avons
	\begin{equation}
		v(\lambda)=\begin{pmatrix}
			1	\\ 
			1	\\ 
			1	
		\end{pmatrix}-\begin{pmatrix}
			\lambda	\\ 
			0	\\ 
			\lambda	
		\end{pmatrix}=\begin{pmatrix}
			1-\lambda	\\ 
			1	\\ 
			1-\lambda	
		\end{pmatrix}
	\end{equation}
	Nous devons donc résoudre l'équation
	\begin{equation}
		v(\lambda)\cdot\begin{pmatrix}
			1	\\ 
			0	\\ 
			1	
		\end{pmatrix}=0,
	\end{equation}
	c'est à dire $(1-\lambda)+(1-\lambda)=0$, et par conséquent $\lambda=1$. La projection de \( (1,1,1)\) est finalement
	\begin{equation}
		\pr_{(1,0,1)}(1,1,1)=P(1)=\begin{pmatrix}
			1	\\ 
			0	\\ 
			1	
		\end{pmatrix}
	\end{equation}
    et
	\begin{equation}
        \pr_{(1,0,1)}\frac{1}{ \sqrt{3} }\begin{pmatrix}
            1    \\ 
            1    \\ 
            1    
        \end{pmatrix}=\frac{1}{ \sqrt{3} }P(1)=\frac{1}{ \sqrt{3} }\begin{pmatrix}
			1	\\ 
			0	\\ 
			1	
		\end{pmatrix}
	\end{equation}

\end{corrige}
