% This is part of Agregation : modélisation
% Copyright (c) 2011
%   Laurent Claessens
% See the file fdl-1.3.txt for copying conditions.

\begin{corrige}{Model-0003}

    La densité de la variable aléatoire conjointe \( (X_1,\ldots,X_n)\) au point \( (x_1,\ldots,x_n)\) est le produit des densités, donc
    \begin{equation}
        \prod_{i=1}^n\gamma_{m,\sigma}(x_1,\ldots,x_n)=\frac{1}{ \sigma^n(2\pi)^{n/2} }\exp-\frac{ 1 }{2}\left( \sum_i\left( \frac{ x_i-m }{ \sigma } \right)^2 \right).
    \end{equation}
    Étant donné que le but est de minimiser, nous pouvons oublier le facteur et passer au logarithme :
    \begin{equation}
        L_0(\bar x)=-\frac{ 1 }{2}\sum_{i=0}^n\left( \frac{ x_i-m }{ \sigma } \right)^2.
    \end{equation}
    Nous pouvons également supprimer le \( \frac{ 1 }{2}\) et le \( 1/\sigma^2\). La fonction à minimiser devient
    \begin{equation}
        L(x_1,\ldots,x_n)=-\sum_i(x_i-m)^2,
    \end{equation}
    dont la dérivée vaut \( 2nm-2\sum_ix_i\). Par conséquent nous avons un minimum avec
    \begin{equation}
        m=\frac{1}{ n }\sum_{i=1}^nx_i.
    \end{equation}

\end{corrige}
