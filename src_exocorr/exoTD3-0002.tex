% This is part of Exercices de mathématique pour SVT
% Copyright (C) 2010
%   Laurent Claessens et Carlotta Donadello
% See the file fdl-1.3.txt for copying conditions.



\begin{exercice}\label{exoTD3-0002}

	Modèle malthusien (ou géométrique). Soit la suite $(u_n)_{n\in\eN}$ définie par
	\begin{equation}
		\begin{cases}
			u_{n+1}=au_n	&	\text{$\forall n\in\eN_0$}\\
			u_0=x,
		\end{cases}
	\end{equation}
	où $a$ et $x\geq 0$ sont deux nombres réels.

	\begin{enumerate}
		\item
			Montrer que $u_n=xa^n$ pour tout $n\in\eN_0$.
		\item
			Soient $a=\frac{ 1 }{2}$ et $x=1000$, en supposant que $n$ représente un nombre de mois et $u_n$ un nombre d'individus, dans combien de temps la population sera inférieure à $1$ (cas d'extinction) ?
		\item
			Soient $a=2$ et $x=2$, en supposant que $n$ représente un nombre de mois et $u_n$ un nombre d'individus, dans combien de temps la population atteindra $1000$ individus ?

	\end{enumerate}

\corrref{TD3-0002}
\end{exercice}
