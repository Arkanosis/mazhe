\begin{exercice}\label{exo_I-1-15}

Soit $f\colon [a,b]\to [a,b]$, une fonction appartenant à $C^1\big( [a,b] \big)$, et soit $\{ f_n \}$, la suite des fonctions itérées définies par
\begin{equation}
	\begin{aligned}[]
	f_0(x)&=x,		&	f_n(x)=f\big( f_{n-1}(x) \big).
	\end{aligned}
\end{equation}
\begin{enumerate}
\item Montrer que si $\sup_{x\in [a,b]}| f'(x) |<1$, alors la série
\begin{equation}
	\sum_{k=1}^{\infty}\big( f_k(x)-f_{k-1}(x) \big)
\end{equation}
converge uniformément.
\item En déduire que la suite des fonctions $ f_n$ converge uniformément vers une fonction constante $C$, que $f(C)=C$ et que $C$ est unique.
\end{enumerate}
(Indication : utiliser la formule des accroissements finis).

\corrref{_I-1-15}
\end{exercice}
% This is part of the Exercices et corrigés de CdI-2.
% Copyright (C) 2008, 2009
%   Laurent Claessens
% See the file fdl-1.3.txt for copying conditions.


