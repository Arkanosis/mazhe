% This is part of Exercices et corrigés de CdI-1
% Copyright (c) 2011
%   Laurent Claessens
% See the file fdl-1.3.txt for copying conditions.

\begin{exercice}\label{exoEqsDiff0010}


Lorsqu'on étire un ressort, la force de rappel est proportionnelle à l'allongement du ressort. Nous allons décrire le mouvement d'une masse $m$ fixée à l'extrémité d'un ressort qu'on étire d'une longueur $y_0$ et qu'on lâche. On suppose qu'il n'y a aucun effet d'amortissement. On rappelle que l'on a le théorème fondamental de la dynamique $F=m\gamma$, liant l'accélération $\gamma$ et la somme des forces $F$ appliquées à la masse $m$.
\begin{enumerate}
\item Les forces en présence sont: le poids $(mg)$ et la force de rappel du ressort $-ky$. On suppose que la position de l'extrémité du ressort est $y$ relativement à la position d'équilibre du ressort avec la masse en son extrémité. Établir que l'on a l'équation différentielle $my''\;=\;-ky$, et la résoudre pour obtenir la position $y(t)$ de la masse à l'instant $t$. Quel type de mouvement obtient on?
\item On suppose maintenant que la masse est plongée dans un liquide (amortisseur) et que la force de friction est proportionnelle à la vitesse et vaut $-cy'$. Écrire l'équation régissant le mouvement de la masse, puis résoudre en distinguant 3 cas suivant la valeur de $c^2-4km$. Quel type de mouvement obtient on dans chacun des cas?
\end{enumerate}

\end{exercice}
