\begin{exercice}\label{exo_II-1-18}

Considérons la famille de droites données sous forme paramétrique ($z$ est le paramètre)
\begin{subequations}
	\begin{numcases}{}
		x=z+4\\
		y=6z+C
	\end{numcases}
\end{subequations}
Laquelle de ces droites passe par le point $(9,32)$ ?

Résoudre les équations différentielles
\begin{enumerate}
\item $t=\frac{ a+by'^2 }{ y' }$
\item $y=\frac{ a+by'^2 }{ y' }$
\end{enumerate}
où $a$ et $b$ sont des constantes réelles. Discuter, pour chacune des équations, le nombre de solutions au problème de Cauchy $y(t_0)=y_0$; comparer en particulier les cas $a=b=1$ et $a=b=-1$.

\corrref{_II-1-18}
\end{exercice}
% This is part of the Exercices et corrigés de CdI-2.
% Copyright (C) 2008, 2009
%   Laurent Claessens
% See the file fdl-1.3.txt for copying conditions.


