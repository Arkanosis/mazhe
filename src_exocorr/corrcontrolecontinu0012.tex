\begin{corrige}{controlecontinu0012}

    La paramétrisation «usuelle»  du cercle est 
    \begin{equation}
        \sigma(t)=(R\cos(t),R\sin(t)).
    \end{equation}
    Les formules qui donnent la paramétrisation normale sont les formules \eqref{EqFomVPcogammaN} et \eqref{EqFomVPcoordnorm}. Nous devons donc  d'abord calculer
    \begin{equation}
        \phi(t)=\int_0^{2\pi R}\| \sigma'(u) \|du=Rt.
    \end{equation}
    Ensuite, \( \gamma_N(s)=(\sigma\circ\phi^{-1})(s)\) où 
    \begin{equation}
        \phi^{-1}(s)=\frac{ s }{ R }.       
    \end{equation}
    Donc la paramétrisation normal est
    \begin{equation}
        \gamma_N(s)=\sigma\left( \frac{ s }{ R } \right)=\begin{pmatrix}
            R\cos\left( \frac{ s }{ R } \right)    \\ 
            R\sin\left( \frac{ s }{ R } \right).    
        \end{pmatrix}
    \end{equation}

\end{corrige}
