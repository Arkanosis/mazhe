\begin{corrige}{EspVectoNorme0001}

	\begin{enumerate}
		\item
			D'abord, en utilisant l'inégalité triangulaire, 
			\begin{equation}
				\| tx+(1-t)y \|\leq \| tx \|+\big\| (1-t)y \big\|.
			\end{equation}
			Ensuite, étant donné que $t\in\mathopen[ 0 , 1 \mathclose]$, nous avons $t>0$ et $1-t>0$, donc nous pouvons les sortir de la norme sans mettre de valeur absolue (voir le point \ref{ItemDefNormeii} de la définition \ref{DefNorme} et le fait que nous sachions, par ailleurs, que la norme euclidienne est une norme).

			Notez que ce point est valable pour toute norme : nous n'avons utilisé que des propriétés de définition des normes.


		\item
			Ce point est valable pour toute les normes en vertu de la proposition \ref{PropNmNNm}.
            % TODO : être plus explicite.

		\item
			Utilisons le fait que la norme euclidienne découle du produit scalaire\footnote{Nous mentionnons ce fait autour de la définition \ref{DefNormeEucleApp}.} : $\| x \|^2=x\cdot x$. Nous avons donc
			\begin{equation}
				\begin{aligned}[]
					\| x-y \|^2+\| x+y \|^2&=(x-y)\cdot (x-y)+(x+y)\cdot(x+y)\\
					&=x\cdot x-x\cdot y-y\cdot x+y\cdot y\\
					&\quad +x\cdot x+x\cdot y+y\cdot x+y\cdot y\\
					&=2x\cdot x+2y\cdot y\\
					&=2\| x \|^2+2\| y \|^2.
				\end{aligned}
			\end{equation}
			
	\end{enumerate}

\end{corrige}
