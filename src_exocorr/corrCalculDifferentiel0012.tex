\begin{corrige}{CalculDifferentiel0012}

	\begin{enumerate}
		\item
			Cette équation décrit le mouvement d'une particule dans le plan dont nous savons qu'elle ne subit aucune accélération dans la direction $x$ et dont nous n'avons aucune information sur le mouvement dans la direction $y$.

			L'équation
			\begin{equation}
				\frac{ \partial  }{ \partial x }\left( \frac{ \partial f }{ \partial x } \right)=0
			\end{equation}
			signifie que la fonction $\frac{ \partial f }{ \partial x }(x,y)$ ne dépend pas de $x$. Il existe donc une fonction $\psi$ telle que 
			\begin{equation}
				\frac{ \partial f }{ \partial x }(x,y)=\psi(y).
			\end{equation}
			En intégrant les deux membres par rapport à $x$ nous trouvons
			\begin{equation}
				f(x,y)=x\psi(y)+\varphi(y).
			\end{equation}
			Dans cette formule, $\varphi$ est la constante d'intégration. Vu que nous avons effectué une intégrale par rapport à $x$, cette constante d'intégration ne peut pas dépendre de $x$. Elle peut cependant dépendre de $y$.

			Note : la fonction $\varphi(y)$ est n'importe quelle fonction  de $y$, et peut en particulier avoir une partie constante. Les fonctions suivantes sont bonnes :
			\begin{equation}
				\begin{aligned}[]
					f(x,y)&=xy+1\\
                    f(x,y)&=\cos(y)	&\text{c'est à  dire }\psi(y)=0 \text{ et } \varphi(y)=\cos(y)
				\end{aligned}
			\end{equation}
		
		\item

			Nous avons $\frac{ \partial  }{ \partial x }\left( \frac{ \partial f }{ \partial y } \right)=0$, et donc $\partial_yf=\psi(y)$ pour une certaine fonction. En intégrant par rapport à $y$ nous trouvons
			\begin{equation}
				f(x,y)=\int\psi(y)dy+\varphi(x)+C
			\end{equation}
			où $\varphi(x)+C$ est la constante d'intégration. Étant donné que $\int\psi(y)dy$ peut être n'importe quelle fonction (nous ne sommes intéressés qu'aux fonctions $C^2$), nous avons la forme générale
			\begin{equation}
				f(x,y)=\psi(y)+\varphi(x).
			\end{equation}
            Les fonctions qui répondent à la question sont donc des fonctions qui ont une partie en \( x\) et une partie en \( y\) complètement séparées. En autres exemples, les fonctions suivantes fonctionnent :
			\begin{equation}
				\begin{aligned}[]
					f(x,y)&=\cos(y)+\sin(x)-4\\
					g(x,y)&=\frac{ x+y }{ xy }.
				\end{aligned}
			\end{equation}
			La seconde peut paraître étonnante, mais ce n'est rien d'autre que $\frac{1}{ x }+\frac{1}{ y }$.
		\item
			Nous avons
			\begin{equation}
				\frac{ \partial  }{ \partial x }\left( \frac{ \partial f }{ \partial x } \right)=\cos(x+y),
			\end{equation}
			et par conséquent,
			\begin{equation}
				\frac{ \partial f }{ \partial x }=\int \cos(x+y)dx=\sin(x+y)+\psi(y)
			\end{equation}
			où $\psi$ est la constante d'intégration par rapport à $x$. En intégrant encore,
			\begin{equation}
				f(x,y)=-\cos(x+y)+x\psi(y)+\varphi(y).
			\end{equation}
	\end{enumerate}
\end{corrige}
