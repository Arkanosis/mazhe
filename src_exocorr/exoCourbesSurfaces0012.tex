\begin{exercice}\label{exoCourbesSurfaces0012}

Soient $ (I, \vec{f})$ et $(J, \vec{g})$ deux courbes de classe $C^1$ équivalentes dans $ \eR^N$, c'est-à-dire il existe $ \varphi : J \to I $ bijective, $C^1$, avec $\varphi^{-1} : I \to J$ de classe $C^1$ telle que $\vec{g}(t) = \vec{f} ( \varphi(t)) $ pour chaque $ t \in J$. 

\begin{enumerate}
	\item

 Soit $ M_0 = \vec{g}(t_0) = \vec{f} ( \varphi(t_0))$ un point fixé et soit $M = \vec{g}(t) = \vec{f} ( \varphi(t))$ un point quelconque sur la courbe. Montrer que la longueur de l'arc  orienté de la courbe  $ (I, \vec{f})$ compris entre $M_0$ et $M$ est la même que la longueur de l'arc  orienté de la courbe  $ (J, \vec{g})$ compris entre $M_0$ et $M$.  En d'autres termes, en notant par $a_{M_0}^{\vec{f}}(M) $, respectivement $a_{M_0}^{\vec{g}}(M) $ l'abscisse curviligne de $M$ sur les deux courbes, on a $a_{M_0}^{\vec{g}}(M) =  a_{M_0}^{\vec{f}}(M)$, et plus explicitement :
 \begin{equation}
a_{M_0}^{\vec{g}}(\vec{g}(t) ) =  a_{M_0}^{\vec{f}}(\vec{f} ( \varphi(t))).
\end{equation}

\item
 On note par $ \vec{\tau}_{\vec f}(M)$ et  $ \vec{\tau}_{\vec g}(M)$ les vecteurs tangents unitaires en $M$ à $ (I, \vec{f})$, respectivement à  $(J, \vec{g})$. Montrer que 
 \begin{equation}
	 \begin{aligned}[]
		\vec{\tau}_{\vec f}(M) &=  \vec{\tau}_{\vec g}(M) & \text{si les deux courbes ont la même orientation}\\
		\vec{\tau}_{\vec f}(M)& =  - \vec{\tau}_{\vec g}(M) & \text{si les deux courbes ont des orientations opposées.}
	 \end{aligned}
 \end{equation}

\item
 Montrer que $ (I, \vec{f})$ et $(J, \vec{g})$ ont la même courbure au point $M$. 

		
\end{enumerate}

\corrref{CourbesSurfaces0012}
\end{exercice}
