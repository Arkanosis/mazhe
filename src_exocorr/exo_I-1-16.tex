% This is part of the Exercices et corrigés de CdI-2.
% Copyright (C) 2008, 2009
%   Laurent Claessens
% See the file fdl-1.3.txt for copying conditions.


\begin{exercice}\label{exo_I-1-16}

% Je n'ai pas de corrigés de cet exercice :
{\bf Théorème de Borel} Soient $\{ C_n \}$, une suite quelconque de nombres réels et $f$ une fonction $C^{\infty}$ de $\eR$ dans $[0,1]$, égale à $1$ dans $[-\frac{ 1 }{2},\frac{1}{ 2 }]$ et à $0$ dans le complémentaire de $[-1,1]$. On considère la série
\begin{equation}
	u(x)=\sum_{n=0}^{\infty}f\left( \frac{ x }{ t_n } \right)\frac{ x^n }{ n! }C_n.
\end{equation}
\begin{enumerate}
\item Montrer que par un choix convenable de la suite des nombres $t_n>0$, la série converge pour tout $x\in\eR$ et que pour tout entier $m\geq 1$, la série des dérivées $m$ièmes
\begin{equation}
	\frac{ d^m }{ dx^m }\left[ f\left( \frac{ x }{ t_n } \right)\frac{ x^n }{ n! } \right]C_n
\end{equation}
est uniformément convergente sur $\eR$.

\item En déduire que la fonction $u$ est $C^{\infty}$ sur $\eR$ et que 
\begin{equation}
	\left.\frac{ d^m }{ dx^m }u\right|_{x=0}=C_m
\end{equation}
pour tout $m\in\eN$.
\end{enumerate}

\corrref{_I-1-16}
\end{exercice}
