\begin{corrige}{IntegralesMultiples0001}

	Nous décomposons l'intégrale double en deux intégrales simples en utilisant le théorème de Fubini (voir les théorèmes \ref{fub} et \ref{ThoSectionINte}) :
	\begin{equation}
		I=\iint_{\mathopen[ 0 , 1 \mathclose]\times\mathopen[ 0 , 1 \mathclose]}\frac{1}{ (x+y+1)^2 }dxdy= \int_{\mathopen[ 0 , 1 \mathclose]}\left[ \int_{\mathopen[ 0 , 1 \mathclose]}\frac{1}{ (x+y+1)^2 }dx \right]dy.
	\end{equation}
	La première intégrale à faire est l'intégrale $\int_{\mathopen[ 0 , 1 \mathclose]}\frac{1}{ (x+y+1)^2 }dx$. Cela est une intégrale par rapport à $x$ dans laquelle nous devons considérer $y$ comme constante. Cela est donc une intégrale de la forme $\int_{0}^1\frac{1}{ (x+a)^2 }$ avec $a=y+1$. Ce type d'intégrale s'effectue en posant $t=x+a$.

	Nous posons donc  $t=x+y+1$, et nous calculons :
	\begin{equation}
		\int_0^1\frac{1}{ (x+y+1)^2 }dx=\int_{y+1}^{y+2}\frac{1}{ t^2 }dt=\left[ -\frac{1}{ t } \right]_{y+1}^{y+2}=\frac{1}{ y+1 }-\frac{1}{ y+2 }.
	\end{equation}
	Maintenant nous pouvons poursuivre :
	\begin{equation}
		I=\int_0^1\left( \frac{1}{ y+1 }-\frac{1}{ y+2 } \right)dy=2\ln(2)-\ln(3).
	\end{equation}
	Pour obtenir cela, nous avons intégré séparément les deux termes, en utilisant les changements de variables $t_1=y+2$ et $t_2=y+1$. Ne pas oublier que $\int \frac{dx}{ x }=\ln(x)$.

\end{corrige}
