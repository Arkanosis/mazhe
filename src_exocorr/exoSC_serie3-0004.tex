\begin{exercice}\label{exoSC_serie3-0004}

	Le tableau ci-dessous donne la conductivité molaire $\Lambda$ (en \reciprocal\ohm\centi\square\meter\per\mole) de l'acide chlorhydrique, en fonction de la concentration $c$ (en \mole\per\deci\cubic\meter),
	\[
		\begin{array}{|c|cccccc|}
			\hline
			c& 0.0005& 0.001	&0.005&0.01&	0.02	&	0.05 \\
			\hline
			\Lambda&   422.74 & 421.36 & 415.80 & 412.24 & 407.24 & 399.09  \\
			\hline
		\end{array}
	\]
	On considère que la relation entre $\Lambda$ et $c$ est donnée approximativement par une formule du type
	\begin{equation}	\label{EqLLamazau}
		\Lambda=a_0+a_1c^{1/2}.
	\end{equation}
	Trouver les coefficient $a_0$ et $a_1$ à partir des données et représenter dans un même diagramme les valeurs donnes et le graphe de la fonction définie par \eqref{EqLLamazau}.

	\emph{Indication} : Le second membre de \eqref{EqLLamazau} peut être vu comme un polynôme du premier degré en $c^{1/2}$.

\corrref{SC_serie3-0004}
\end{exercice}
