\begin{corrige}{LimiteContinue0002}

	\begin{enumerate}
		\item
			Si nous considérons le chemin $\gamma(t)=(t,t)$, nous avons $(f\circ\gamma)(t)=\frac{ t }{ 2t^2 }=\frac{1}{ 2t }$, et la limite $t\to 0$ n'existe pas. En vertu du corollaire \ref{CorMethodeChemoinNegatif}, la limite de $f$ en $(0,0)$ n'existe pas.

			Ceci est un cas typique d'un fait qui arrive souvent : lorsqu'on a une fraction de polynômes dont le dénominateur est de plus haut degré que le numérateur, alors il n'y a pas de limites (parce que le dénominateur tend plus vite vers zéro que le numérateur). Attention toutefois à ne pas en faire une règle générale.

		\item
			Nous voyons que le degré du numérateur est plus haut que celui du dénominateur. Cela nous incite à croire que la limite va exister et valoir $0$. En effet, en passant aux coordonnées polaires,
			\begin{equation}
				f(r\cos\theta,r\sin\theta)=r(\cos\theta+2\sin\theta)^3<r3^3=27r.
			\end{equation}
			La limite de cela lorsque $r\to 0$ est vaut $0$. La limite de la fonction est donc zéro.

			Notez qu'il est important de borner $f(r\cos\theta,r\sin\theta)$ par un nombre qui ne dépend pas de $\theta$ avant de tester la limite $r\to 0$. Ici, nous avons borné $\cos\theta+2\sin\theta$ par $3$, et $27r$ est le $s_r$ de la proposition \ref{PropMethodePolaire}.

	\end{enumerate}

\end{corrige}
