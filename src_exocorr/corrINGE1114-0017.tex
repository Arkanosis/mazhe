% This is part of Un soupçon de physique, sans être agressif pour autant
% Copyright (C) 2006-2009,2012
%   Laurent Claessens
% See the file fdl-1.3.txt for copying conditions.


\begin{corrige}{SerieUn0017}

	Supposons avoir réussi pour $n$, et regardons ce que devient la formule proposée avec $n+1$~:
	\begin{equation}
		\underbrace{\frac{1}{ 1\cdot 2}+\cdots+\frac{1}{ n(n+1) }}_{\text{$=n/(n+1)$ par hyp. de récurrence}}+\frac{1}{ (n+1)(n+2) }.
	\end{equation}
	En mettant au même dénominateur, nous trouvons
	\begin{equation}
		\frac{ n(n+2)+1 }{ (n+1)(n+2) }=\frac{ n^2+2n+1 }{ (n+1)(n+2) }=\frac{ n+1 }{ n+2 },
	\end{equation}
	comme il le fallait.

\end{corrige}
