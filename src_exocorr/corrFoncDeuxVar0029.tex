% This is part of the Exercices et corrigés de mathématique générale.
% Copyright (C) 2010
%   Laurent Claessens
% See the file fdl-1.3.txt for copying conditions.

\begin{corrige}{FoncDeuxVar0029}

	Les corrigés sont créés par le script Sage \verb+exo101.sage+

	\VerbatimInput[tabsize=3]{src_sage/exo101.sage}

	Des réponses :
	
	\begin{enumerate}

		\item
			\VerbatimInput[tabsize=3]{src_sage/exo101A.txt}

			Ici nous voyons que Sage a du mal à calculer la matrice Hessienne en $(0,0)$. En effet, nous tombons sur une division par zéro. Pour résoudre l'exercice, il faut se rendre compte que la fonction $(x,y)\mapsto\sqrt{x^2+y^2}$ est toujours positive et est nulle seulement au point $(0,0)$. Donc $f$ est toujours plus petite ou égale à deux tandis que $f(0,0)=2$. Le point est donc un maximum global.
		\item
			\VerbatimInput[tabsize=3]{src_sage/exo101B.txt}

			Petite note sur la façon dont on trouve les points critiques. Le système est
			\begin{subequations}
				\begin{numcases}{}
					3x^2+3y^2-15=0\\
					6xy-12=0.
				\end{numcases}
			\end{subequations}
			De la seconde équation, nous isolons $x$ : $x=2/y$. Précisons qu'il n'y a pas de solutions avec $y=0$. En remettant le $x$ trouvé en fonction de $y$ dans la première équation nous trouvons
			\begin{equation}
				\frac{ 12 }{ y^2 }+3y^2-15=0,
			\end{equation}
			ce qui revient à l'équation bicarrée 
			\begin{equation}
				3y^4-15y^2+12=0.
			\end{equation}
			En posant $u=y^2$ nous avons $u=4$ ou $u=1$ (résolution d'une équation du second degré pour $u$). Nous avons alors les quatre possibilités $y=\pm 2$ et $y=\pm 1$. Pour chacune de ces possibilités, la formule $x=2/y$ fournit le $x$ correspondant.

		\item
			\VerbatimInput[tabsize=3]{src_sage/exo101C.txt}
			

	\end{enumerate}

\end{corrige}
