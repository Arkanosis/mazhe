% This is part of the Exercices et corrigés de mathématique générale.
% Copyright (C) 2010
%   Laurent Claessens
% See the file fdl-1.3.txt for copying conditions.

\begin{corrige}{FoncDeuxVar0028}

	Certains corrigés de cet exercice ont étés réalisés par Sage. Le script utilisé est \verb+exo103.sage+

	\VerbatimInput[tabsize=3]{src_sage/exo103.sage}

	Des réponses :

	\begin{enumerate}

		\item	%1
			\VerbatimInput[tabsize=3]{src_sage/exo103A.txt}
		\item
		\item
		\item
		\item
		\item
		\item
		\item	%8
			\VerbatimInput[tabsize=3]{src_sage/exo103H.txt}

		\item
		\item
		\item
		\item
		\item
		\item
		\item
		\item
		\item	%17
			\VerbatimInput[tabsize=3]{src_sage/exo103Q.txt}

			Ici, Sage n'est pas capable de résoudre les équations qui annulent le jacobien. Les équations à résoudre sont pourtant faciles :
			\begin{subequations}
				\begin{numcases}{}
					e^{x}\cos(y)=0\\
					e^{x}\sin(y)=0
				\end{numcases}
			\end{subequations}
			Étant donné que l'exponentielle ne s'annule jamais, il faudrait avoir en même temps $\cos(y)=0$ et $\sin(y)=0$, ce qui est impossible. La fonction n'a donc aucun extrema local.
		\item
		\item
		\item
			

	\end{enumerate}

\end{corrige}
