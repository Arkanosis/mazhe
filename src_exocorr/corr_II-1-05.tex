% This is part of the Exercices et corrigés de CdI-2.
% Copyright (C) 2008, 2009,2015-2016
%   Laurent Claessens
% See the file fdl-1.3.txt for copying conditions.


\begin{corrige}{_II-1-05}

\begin{enumerate}

	\item $\sin(2t)y'+y^2=1$. 
Ceci est une équation à variables séparée qui s'écrit sous forme plus traditionnelle
\begin{equation}
	y'=\frac{1}{ \sin(t) }(1-y^2).
\end{equation}
En suivant les notations du cours de première, nous avons $u(t)=1/\sin(t)$ et $f(y)=1-y^2$. Si $1-y^2=0$, nous trouvons les deux solutions constantes $y=\pm 1$, sinon nous pouvons continuer la méthode et trouver
\begin{equation}
	\begin{aligned}[]
		U(t)	&=\frac{ 1 }{2}\ln(\tan(t))\\
		G(t)	&=\int\frac{ dy }{ 1-y^2 }=-\frac{ 1 }{2}\ln\left( \frac{ y+1 }{ y-1 } \right).
	\end{aligned}
\end{equation}
L'équation implicite donnant $y$ est donc
\begin{equation}
	-\frac{ 1 }{2}\ln\left( \frac{ y+1 }{ y-1 } \right)=\frac{ 1 }{2}\ln(\tan t)+C.
\end{equation}

	\item $y'+y/(t+1)=\sin(t)$. 
C'est une équation linéaire, dont on cherche d'abord la solution générale de l'homogène associée. Cette solution est
\begin{equation}
	y_H(t)=\frac{ K }{ t+1 }.
\end{equation}
Nous appliquons maintenant le technique de la variation des constantes pour trouver la solution générale de l'équation proposée. Maintenant $K=K(t)$, et nous avons
\begin{equation}
	y'(t)=\frac{ K'(t) }{ t+1 }-\frac{ K(t) }{ (t+1)^2 }
\end{equation}
que nous remettons dans l'équation de départ pour trouver une équation différentielle pour $K(t)$. Ce que nous trouvons est
\begin{equation}
	K'(t)=\sin(t)(t+1),
\end{equation}
dont la solution se ramène au calcul d'une primitive. Le résultat est
\begin{equation}
	K(t)=\sin(t)-\cos(t)(t+1)+C,
\end{equation}
ce qui donne la réponse finale :
\begin{equation}
	y(t)=\frac{ \sin(t)+C }{ t+1 }-\cos(t).
\end{equation}

	\item $y'-ay/t=e^tt^a$.
Ceci est une équation différentielle linéaire qui peut aussi s'écrire $y't^{-a}-at^{a-1}y=e^t$, ce qui met l'équation de départ sous la forme
\begin{equation}
	(t^{-a}y)'=e^t,
\end{equation}
ce qui donne tout de suite $t^{-a}y=e^t+C$, et donc la solution
\begin{equation}
	y(t)=(e^t+C)t^a.
\end{equation}

	\item $yy'+(1+y^2)\sin(t)=0$.
Cela est une équation à variables séparée, mais elle peut être simplifiée en remarquant que $yy'=(y^2)'/2$. Nous avons alors
\begin{equation}
	\frac{ 1 }{2}\frac{ (y^2)' }{ 1+y^2 }=-\sin(t),
\end{equation}
que nous récrivons
\begin{equation}
	\frac{ (1+y^2)' }{ 1+y^2 }=-2\sin(t).
\end{equation}
Cela entraîne que la solution est donnée par l'équation
\begin{equation}
	1+y^2=C e^{2\cos(t)}.
\end{equation}

\item
$y''+2y'+y= e^{-t}\ln(t)$.
C'est une équation linéaire à coefficients constants, dont l'équation homogène est
\begin{equation}
	y''+2y'+y=0.
\end{equation}
Son polynôme caractéristique est $r^2+2r+1=0$, dont l'unique solution est $r=-1$, de multiplicité $2$. En vertu de la théorie générale la solution générale à l'équation homogène est 
\begin{equation}
	y_H=(At+B) e^{-t}
\end{equation}
pour certaines constantes $A$ et $B$. 

Maintenant, nous utilisons la méthode de la variation des constantes pour trouver la solution générale de l'équation non homogène. Nous posons donc
\begin{equation}
	y(t)=\big( A(t)t+B(t) \big) e^{-t}.
\end{equation}
Toujours en vertu de ce qui a été vu en première, les fonctions $A$ et $B$ ne sont pas complètement indépendantes, mais peuvent être choisies de façon à vérifier la condition \eqref{EqVarCstSubtil}
\begin{equation}		\label{EqII105ConditionVC}
	A't+B'=0.
\end{equation}
Cette condition va considérablement simplifier le calcul qui suit. D'abord, nous avons
\begin{equation}
	y'=(A't+A+B'-At-B) e^{-t},
\end{equation}
dans lequel nous utilisons la condition \eqref{EqII105ConditionVC} pour trouver
\begin{equation}
	y'=(A-At-B) e^{-t}.
\end{equation}
En dérivant encore une fois, il vient
\begin{equation}
	y''=(A'+A't-2A-B'+At+B) e^{-t},
\end{equation}
dans laquelle il n'y a pas de dérivées secondes de $A$ et $B$, grâce à l'utilisation de \eqref{EqII105ConditionVC}. Nous pouvons maintenant écrire l'équation $y''+2y'+y= e^{-t}\ln| t |$ qui est maintenant une équation différentielle pour $A$ et $B$. Étant donné que \eqref{EqII105ConditionVC} tient toujours, nous avons en réalité le système
\begin{subequations}
\begin{numcases}{}
	A'-A't-B'=\ln| t |   \\   
	A't+B'=0.  
\end{numcases}
\end{subequations}
En introduisant la seconde équation dans la première, nous trouvons $A'=\ln| t |$, dont nous déduisons
\begin{equation}
	A(t)=t\big( \ln| t |-1 \big)+A_0.
\end{equation}
La seconde équation nous dit par ailleurs que $B'=-A't=-t\ln| t |$, dont l'intégration donne
\begin{equation}
	B(t)=-\frac{ t^2 }{ 4 }\big( 2\ln| t |-1 \big)+B_0.
\end{equation}
\end{enumerate}

\end{corrige}
