% This is part of the Exercices et corrigés de CdI-2.
% Copyright (C) 2008, 2009
%   Laurent Claessens
% See the file fdl-1.3.txt for copying conditions.


\begin{corrige}{_I-1-10}

\begin{enumerate}
\item D'abord, $f_n(0)=0$ pour tout $n$, donc $f(0)=0$. Maintenant, pour $p\in\eN$ et $x\in]\frac{1}{ 2^p },\frac{1}{ 2^{p-1} }]$, si $n\geq p$, alors
\begin{equation}
	\sum_{k=1}^{n}f_k(x)=\frac{1}{ p },
\end{equation}
donc $\lim_{n\to\infty}\left| \sum_{k=1}^n f_k(x)-\frac{1}{ p } \right|=0$, et nous avons donc convergence ponctuelle vers la fonction
\begin{equation}
	f(x)=
\begin{cases}
	\frac{1}{ p }	&	\text{si }\frac{1}{ 2^p }<x\leq \frac{1}{ 2^{p-1} }\\
	0	&	 \text{en }x=0.
\end{cases}
\end{equation}

\item Pour la convergence uniforme, remarquer que
\begin{equation}
	\left| f(x)-\sum_{k=1}^nf_k(x)\right|<\frac{1}{ n+1 },
\end{equation}
donc $\forall\epsilon > 0$, $\exists N$ tel que $n>N$ implique
\begin{equation}
	\|  f-\sum_{k=1}^nf_k \|_{\infty}<\epsilon,
\end{equation}
en l'occurrence, le $N$ qui fonctionne est donné par $\frac{1}{ N+1 }<\epsilon$. Nous avons donc convergence uniforme.

\item Étant donné que $\| f_k \|_{\infty}=1/k$, la série numérique $\sum_{k=1}^{\infty}\| f_k \|=\sum_{k=1}^{\infty}\frac{1}{ k }$ diverge et nous n'avons pas de convergence normale.
\end{enumerate}

Nous avons donc un exemple du fait que la convergence uniforme n'implique pas la convergence normale.

\end{corrige}
