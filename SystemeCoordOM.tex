% This is part of Mes notes de mathématique
% Copyright (c) 2011-2012
%   Laurent Claessens
% See the file fdl-1.3.txt for copying conditions.

%+++++++++++++++++++++++++++++++++++++++++++++++++++++++++++++++++++++++++++++++++++++++++++++++++++++++++++++++++++++++++++
\section{Les coordonnées cartésiennes}
%+++++++++++++++++++++++++++++++++++++++++++++++++++++++++++++++++++++++++++++++++++++++++++++++++++++++++++++++++++++++++++

Dans le plan, le choix de deux axes gradués perpendiculaires permet de repérer un point $M$ à partir de deux nombres réels $x$ et $y$, nommés \defe{abscisse}{abscisse} et \defe{ordonnées}{ordonnées} que l'on nomme les \defe{coordonnées cartésiennes}{coordonnées cartésiennes} de $M$.

Quelque exemples sur la figure \ref{LabelFigDefinitionCartesiennes}.
\newcommand{\CaptionFigDefinitionCartesiennes}{Quelque points en coordonnées cartésiennes.}
\input{Fig_DefinitionCartesiennes.pstricks}

Nous représentons alors le point par le couple $(x,y)$, et on identifie le point $M$ avec le vecteur $\overrightarrow{OM}$ qui relie l'origine $O$ des axes et le point $M$. Étant donné qu'il faut deux nombres pour repérer un point dans le plan, l'ensemble de tous les points s'identifie avec l'ensemble $\eR\times\eR=\eR^2$.

Dans l'espace, nous pouvons faire la même chose avec trois axes au lieu de deux. Les points sont alors donnés par des triples $(x,y,z)$. Étant donné qu'il faut trois nombres pour repérer un point dans l'espace, l'ensemble de tous les points de l'espace s'identifie avec l'ensemble $\eR\times\eR\times\eR=\eR^3$.

%+++++++++++++++++++++++++++++++++++++++++++++++++++++++++++++++++++++++++++++++++++++++++++++++++++++++++++++++++++++++++++
\section{Opérations sur les vecteurs}
%+++++++++++++++++++++++++++++++++++++++++++++++++++++++++++++++++++++++++++++++++++++++++++++++++++++++++++++++++++++++++++

L'\defe{addition}{addition de vecteurs} de deux vecteurs est définie par
\begin{equation}
	\begin{pmatrix}
		x	\\ 
		y	\\ 
		z	
	\end{pmatrix}+\begin{pmatrix}
		x'	\\ 
		y'	\\ 
		z'	
	\end{pmatrix}=\begin{pmatrix}
		x+x'	\\ 
		y+y'	\\ 
		z+z'	
	\end{pmatrix}.
\end{equation}
La \defe{multiplication}{multiplication de vecteurs par un scalaire} d'un vecteur $(x,y,z)$ par le scalaire $\lambda\in\eR$ est définie par
\begin{equation}
	\lambda\begin{pmatrix}
		x	\\ 
		y	\\ 
		z	
	\end{pmatrix}=\begin{pmatrix}
		\lambda x	\\ 
		\lambda y	\\ 
		\lambda z	
	\end{pmatrix}.
\end{equation}

%---------------------------------------------------------------------------------------------------------------------------
\subsection{Le produit scalaire}
%---------------------------------------------------------------------------------------------------------------------------

Le \defe{produit scalaire}{produit scalaire} entre deux vecteurs est défini par
\begin{equation}
	\begin{pmatrix}
		x	\\ 
		y	\\ 
		z	
	\end{pmatrix}\cdot\begin{pmatrix}
		x'	\\ 
		y'	\\ 
		z'	
	\end{pmatrix}=xx'+yy'+zz'.
\end{equation}
Nous utilisons souvent la notation compacte
\begin{equation}
	X=\begin{pmatrix}
		x	\\ 
		y	\\ 
		z	
	\end{pmatrix}
\end{equation}
et par conséquent nous écrivons le produit scalaire $X\cdot X'$.

\begin{proposition}[Propriétés du produit scalaire]
	Si $X$ et $Y$ sont des vecteurs de $\eR^3$, alors
	\begin{description}
		\item[Symétrie] $X\cdot Y=Y\cdot X$;
		\item[Linéarité] $(\lambda X+\mu X')\cdot Y=\lambda(X\cdot Y)+\mu(X'\cdot Y)$ pour tout $\lambda$ et $\mu$ dans $\eR$;
		\item[Défini positif] $X\cdot X\geq 0$ et $X\cdot X=0$ si et seulement si $X=0$.
	\end{description}
\end{proposition}
Note : lorsque nous écrivons $X=0$, nous voulons voulons dire $X=\begin{pmatrix}
	0	\\ 
	0	\\ 
	0	
\end{pmatrix}$.

%---------------------------------------------------------------------------------------------------------------------------
\subsection{Projection et angles}
%---------------------------------------------------------------------------------------------------------------------------

\begin{definition}
	La \defe{norme}{norme!vecteur} du vecteur $X$, notée $\| X \|$, est définie par 
	\begin{equation}
		\| X \|=\sqrt{X\cdot X}=\sqrt{x^2+y^2+z^2}
	\end{equation}
	si $X=(x,y,z)$. Cette norme sera parfois nommée «norme euclidienne».
\end{definition}
Cette définition est motivée par le théorème de Pythagore. Le nombre $X\cdot X$ est bien la longueur de la «flèche» $X$. Plus intrigante est la définition suivante :
\begin{definition}
	Deux vecteurs $X$ et $Y$ sont \defe{orthogonaux}{orthogonal!vecteur} si $X\cdot Y=0$. 
\end{definition}
Cette définition de l'orthogonalité est motivée par la proposition suivante.

\begin{proposition}		\label{PropProjScal}
	Si nous écrivons $\pr_Y$  l'opération de projection sur la droite qui sous-tend $Y$, alors nous avons
	\begin{equation}
		\| \pr_YX \|=\frac{ X\cdot Y }{ \| Y \| }.
	\end{equation}
\end{proposition}

\begin{proof}
	Les vecteurs $X$ et $Y$ sont des flèches dans l'espace. Nous pouvons choisir un système d'axe orthogonal tel que les coordonnées de $X$ et $Y$ soient
	\begin{equation}
		\begin{aligned}[]
			X&=\begin{pmatrix}
				x	\\ 
				y	\\ 
				0	
			\end{pmatrix},
			&Y&=\begin{pmatrix}
				l	\\ 
				0	\\ 
				0	
			\end{pmatrix}
		\end{aligned}
	\end{equation}
	où $l$ est la longueur du vecteur $Y$. Pour ce faire, il suffit de mettre le premier axe le long de $Y$, le second dans le plan qui contient $X$ et $Y$, et enfin le troisième axe dans le plan perpendiculaire aux deux premiers.

	Un simple calcul montre que $X\cdot Y=xl+y\cdot 0+0\cdot 0=xl$. Par ailleurs, étant donné la figure \ref{LabelFigProjectionScalaire}, nous avons $\| \pr_YX \|=x$.
	\newcommand{\CaptionFigProjectionScalaire}{La longueur de la projection de $X=(x,y)$ sur $Y$ est $x$.}
	\input{Fig_ProjectionScalaire.pstricks}
	Par conséquent,
	\begin{equation}
		\| \pr_YX \|=\frac{ X\cdot Y }{ l }=\frac{ X\cdot Y }{ \| Y \| }.
	\end{equation}
\end{proof}

\begin{corollary}
	Si la norme de $Y$ est $1$, alors le nombre $X\cdot Y$ est la longueur de la projection de $X$ sur $Y$.
\end{corollary}

\begin{proof}
	Poser $\| Y \|=1$ dans la proposition \ref{PropProjScal}.
\end{proof}

Nous sommes maintenant en mesure de déterminer, pour deux vecteurs quelconques $u$ et $v$, la projection orthogonale de $u$ sur $v$. Ce sera le vecteur $\bar u$ parallèle à $v$ tel que $u-\bar u$ est orthogonal à $v$. Nous avons donc
\begin{equation}
    \bar u=\lambda v
\end{equation}
et 
\begin{equation}
    (u-\lambda v)\cdot v=0.
\end{equation}
La seconde équation donne $u\cdot v-\lambda v\cdot v=0$, ce qui fournit $\lambda$ en fonction de $u$ et $v$ :
\begin{equation}
    \lambda=\frac{ u\cdot v }{ \| v \|^2 }.
\end{equation}
Nous avons par conséquent
\begin{equation}
    \bar u=\frac{ u\cdot v }{ \| v \|^2 }v.
\end{equation}
Armés de cette interprétation graphique du produit scalaire, nous comprenons pourquoi nous disons que deux vecteurs sont orthogonaux lorsque leur produit scalaire est nul.

Nous pouvons maintenant savoir quel est le coefficient angulaire d'une droite orthogonale à une droite donnée. En effet, supposons que la première droite soit parallèle au vecteur $X$ et la seconde au vecteur $Y$. Les droites seront perpendiculaires si $X\cdot Y=0$, c'est à dire si
\begin{equation}
	\begin{pmatrix}
		x_1	\\ 
		y_1	
	\end{pmatrix}\cdot\begin{pmatrix}
		y_1	\\ 
		y_2	
	\end{pmatrix}=0.
\end{equation}
Cette équation se développe en 
\begin{equation}		\label{Eqxuyukljsca}
	x_1y_1=-x_2y_2.
\end{equation}
Le coefficient angulaire de la première droite est $\frac{ x_2 }{ x_1 }$. Isolons cette quantité dans l'équation \eqref{Eqxuyukljsca} :
\begin{equation}
	\frac{ x_2 }{ x_1 }=-\frac{ y_1 }{ y_2 }.
\end{equation}
Donc le coefficient angulaire de la première est l'inverse et l'opposé du coefficient angulaire de la seconde.

\begin{example}
	Soit la droite $d\equiv y=2x+3$. Le coefficient angulaire de cette droite est $2$. Donc le coefficient angulaire d'une droite perpendiculaires doit être $-\frac{ 1 }{ 2 }$.
\end{example}

\begin{theorem}[Inégalité de Cauchy-Schwarz]\index{Cauchy-Schwarz}\index{inégalité!Cauchy-Schwarz}      \label{ThoCauchySchwarzIneg}
	Si $X$ et $Y$ sont des vecteurs, alors
	\begin{equation}
		| X\cdot Y |\leq\| X \|\| Y \|.
	\end{equation}
\end{theorem}

\begin{proof}
	Étant donné que les deux membres de l'inéquation sont positifs, nous allons travailler en passant au carré afin d'éviter les racines carrés dans le second membre.
	
	Nous considérons la fonction
	\begin{equation}
		\varphi(t)=\| X+tY \|=(X+tY)\cdot(X+tY)=X\cdot X+tX\cdot Y+tY\cdot X+t^2Y\cdot Y.
	\end{equation}
	En ordonnant les termes selon les puissance de $t$,
	\begin{equation}
		\varphi(t)=\| Y \|^2t^2+2(X\cdot Y)t+\| X \|^2.
	\end{equation}
	Cela est un polynôme du second degré en $t$. Par conséquent le discriminant\footnote{Le fameux $b^2-4ac$.} doit être négatif. Nous avons donc
	\begin{equation}
		4(X\cdot Y)^2-4\| X \|^2\| Y \|^2\leq 0,
	\end{equation}
	ce qui donne immédiatement
	\begin{equation}
		(X\cdot Y)^2\leq\| X \|^2\| Y^2 \|.
	\end{equation}
	
\end{proof}

\begin{proof}[Preuve alternative]
	La preuve peut également être donnée en ne faisant pas référence au produit scalaire. Il suffit d'écrire toutes les quantités en termes des coordonnées de $X$ et $Y$. Si nous posons
	\begin{equation}
		\begin{aligned}[]
			X&=\begin{pmatrix}
				x_1	\\ 
				x_2	\\ 
				x_2	
			\end{pmatrix},
			&Y&=\begin{pmatrix}
				y_1	\\ 
				y_2	\\ 
				y_3	
			\end{pmatrix},
		\end{aligned}
	\end{equation}
	l'inégalité à prouver devient
	\begin{equation}
		(x_1y_1+x_2y_2+x_3y_3)^2\leq (x_1^2+x_2^2+x_3^2)(y_1^2+y_2^2+y_3^2).
	\end{equation}
	Nous considérons la fonction
	\begin{equation}
		\varphi(t)=(x_1+ty_1)^2+(x_2+ty_2)^2+(x_3+ty_3)^2
	\end{equation}
	En tant que norme, cette fonction est évidement positive pour tout $t$. En regroupant les termes de chaque puissance de $t$, nous avons
	\begin{equation}
		\varphi(t)=(y_1^2+y_2^2+y_3^2)t^2+2(x_1y_1+x_2y_2+x_3y_3)t+(x_1^2+x_2^2+x_3^2).
	\end{equation}
	Cela est un polynôme du second degré en $t$. Par conséquent le discriminant doit être négatif. Nous avons donc
	\begin{equation}
		4(x_1y_1+x_2y_2+x_3y_3)^2-(x_1^2+x_2^2+x_3^2)(y_1^2+y_2^2+y_3^2)\leq 0.
	\end{equation}
	La thèse en découle aussitôt.
\end{proof}

\begin{proposition}
	La norme euclidienne a les propriétés suivantes :
	\begin{enumerate}
		\item
			Pour tout vecteur $X$ et réel $\lambda$,  $\| \lambda X \|=| \lambda |\| X \|$. Attention à ne pas oublier la valeur absolue !
		\item
			Pour tout vecteurs $X$ et $Y$, $\| X+Y \|\leq \| X \|+\| Y \|$.
	\end{enumerate}
\end{proposition}

\begin{proof}
	Le premier point est l'exercice \ref{exoOutilsMath-0002}. Pour le second, nous avons les inégalités suivantes :
	\begin{subequations}
		\begin{align}
			\| X+Y \|^2&=\| X \|^2+\| Y \|^2+2X\cdot Y\\
			&\leq\| X \|^2+\| Y \|^2+2|X\cdot Y|\\
			&\leq\| X \|^2+\| Y \|^2+2\| X \|\| Y \|\\
			&=\big( \| X \|+\| Y \| \big)^2
		\end{align}
	\end{subequations}
	Nous avons utilisé d'abord la majoration $| x |\geq x$ qui est évident pour tout nombre $x$; et ensuite l'inégalité de Cauchy-Schwarz.
\end{proof}

%+++++++++++++++++++++++++++++++++++++++++++++++++++++++++++++++++++++++++++++++++++++++++++++++++++++++++++++++++++++++++++
\section{Angle entre deux vecteurs}
%+++++++++++++++++++++++++++++++++++++++++++++++++++++++++++++++++++++++++++++++++++++++++++++++++++++++++++++++++++++++++++

Si $a$ et $b$ sont des réels, l'inégalité $| a |\leq b$ peut se développer en une double inégalité
\begin{equation}
	-b\leq a\leq b.
\end{equation}
L'inégalité de Cauchy-Schwarz devient alors
\begin{equation}
	-\| X \|\| Y \|\leq X\cdot Y\leq\| X \|\| Y \|.
\end{equation}
Si $X\neq 0$ et $Y\neq 0$, nous avons alors
\begin{equation}
	-1\leq\frac{ X\cdot Y }{ \| X \|\| Y \| }\leq 1.
\end{equation}
Il existe donc un angle $\theta\in\mathopen[ 0 , \pi \mathclose]$ tel que
\begin{equation}		\label{eqDefAngleVect}
	\cos(\theta)=\frac{ X\cdot Y }{ \| X \|\| Y \| }.
\end{equation}
L'angle ainsi défini est l'\defe{angle}{angle!entre vecteurs} entre $X$ et $Y$. La définition \eqref{eqDefAngleVect} est souvent utilisée sous la forme
\begin{equation}		\label{eqPropCosThet}
	X\cdot Y=\| X \|\| Y \|\cos(\theta).
\end{equation}

Notez que les angles sont toujours des angles plus petits ou égaux à \unit{180}{\degree}.

%+++++++++++++++++++++++++++++++++++++++++++++++++++++++++++++++++++++++++++++++++++++++++++++++++++++++++++++++++++++++++++
\section{Le cercle trigonométrique}
%+++++++++++++++++++++++++++++++++++++++++++++++++++++++++++++++++++++++++++++++++++++++++++++++++++++++++++++++++++++++++++


Le \href{http://fr.wikiversity.org/wiki/Trigonométrie/Cosinus_et_sinus_dans_le_cercle_trigonométrique}{cercle trigonométrique} est le cercle de rayon $1$ représenté à la figure \ref{LabelFigCercleTrigono}. Sa longueur est $2\pi$.
\newcommand{\CaptionFigCercleTrigono}{Le cercle trigonométrique.}
\input{Fig_CercleTrigono.pstricks}

Nous verrons plus tard que la longueur de l'arc de cercle intercepté par un angle $\theta$ est égal à $\theta$. Les radians sont donc l'unité d'angle les plus adaptés au calcul de longueurs sur le cercle. %Voir exercice \ref{exoGeomAnal-0034}. % L'exercice exoGeomAnal-0034 est dans le cours de géométrie analytique et non ici. On ne peut donc pas décommenter avant une possible fusion.

%---------------------------------------------------------------------------------------------------------------------------
\subsection{Les fonctions sinus et cosinus}
%---------------------------------------------------------------------------------------------------------------------------

La longueur de la projection du point $P$ sur la droite horizontale va naturellement être égale à $\cos(\theta)$. En effet, si nous notons $X$ un vecteur horizontal de norme $1$, cette projection est donné par $P\cdot X$. Mais en reprenant l'équation \eqref{eqPropCosThet}, nous voyons que
\begin{equation}
	P\cdot X=\| P \|\| X \|\cos(\theta),
\end{equation}
tandis qu'ici nous avons $\| P \|=\| X \|=1$.

Nous appelons $\sin(\theta)$ la longueur de la projection sur l'axe vertical.

Quelque dessins nous convainquent que 
\begin{equation}
	\begin{aligned}[]
		\sin(\theta+2\pi)&=\sin(\theta)&\cos(\theta+2\pi)&=\sin(\theta),\\
		\sin(\theta+\frac{ \pi }{2})&=\cos(\theta)&\cos(\theta+\frac{ \pi }{2})&=-\sin(\theta),\\
		\sin(\pi-\theta)&=\sin(\theta)&\cos(\pi-\theta)&=-\cos(\theta).
	\end{aligned}
\end{equation}
Le théorème de Pythagore nous montre aussi l'importante relation
\begin{equation}
	\sin^2(\theta)+\cos^2(\theta)=1.
\end{equation}

Quelque valeurs remarquables pour les sinus et cosinus :
\begin{equation}
	\begin{aligned}[]
		\sin 0&=0,&\sin\frac{ \pi }{ 6 }&=\frac{ 1 }{2},&\sin\frac{ \pi }{ 4 }&=\frac{ \sqrt{2} }{2},&\sin\frac{ \pi }{ 3 }&=\frac{ \sqrt{3} }{2},&\sin\frac{ \pi }{2}&=1,&\sin\pi&=0\\
		\cos 0&=1,&\cos\frac{ \pi }{ 6 }&=\frac{ \sqrt{3} }{2},&\cos\frac{ \pi }{ 4 }&=\frac{ \sqrt{2} }{2},&\cos\frac{ \pi }{ 3 }&=\frac{ 1 }{2},&\cos\frac{ \pi }{2}&=0,&\cos\pi&=-1
	\end{aligned}
\end{equation}
Voir l'exercice \ref{exoOutilsMath-0003}.

%---------------------------------------------------------------------------------------------------------------------------
\subsection{La fonction tangente}
%---------------------------------------------------------------------------------------------------------------------------

La définition de la \defe{tangente}{tangente} est :
\begin{equation}
	\tan\theta=\frac{ \sin\theta }{ \cos\theta }.
\end{equation}
Cette fonction a une interprétation géométrique donnée sur la figure \ref{LabelFigTgCercleTrigono}.
\newcommand{\CaptionFigTgCercleTrigono}{Interprétation géométrique de la fonction tangente. La tangente de l'angle $\theta$ est positive (et un peu plus grande que $1$) tandis que celle de la tangente de l'angle $\varphi$ est négative.}
\input{Fig_TgCercleTrigono.pstricks}

La restriction de la fonction tangente à l'intervalle $\mathopen] -\frac{ \pi }{2} , \frac{ \pi }{2} \mathclose[$ est une bijection vers $\eR$. Nous avons donc une fonction inverse
\begin{equation}
	\begin{aligned}
		\tan^{-1}\colon \eR&\to \mathopen] -\frac{ \pi }{2} , \frac{ \pi }{2} \mathclose[ \\
		x&\mapsto \text{$y$ tel que $\tan(y)=x$.}
	\end{aligned}
\end{equation}
Notez que cette définition, bien qu'elle ait un sens, ne dit pas comment \emph{calculer} le nombre $\tan^{-1}(x)$ pour un nombre $x$ donné\footnote{Si vous utilisez votre calculatrice, n'oubliez pas que les formules que vous connaissez ne sont valables qu'en radian.}.


%---------------------------------------------------------------------------------------------------------------------------
\subsection{Les coordonnées polaires}
%---------------------------------------------------------------------------------------------------------------------------

On a vu qu'un point $M$ dans $\eR^2$ peut être représenté par ses abscisses $x$ et ses ordonnées $y$. Nous pouvons également déterminer le même point $M$ en donnant un angle et une distance comme montré sur la figure \ref{LabelFigCoordPolaires}.
\newcommand{\CaptionFigCoordPolaires}{Un point en coordonnées polaires est donné par sa distance à l'origine et par l'angle qu'il faut avec l'horizontale.}
\input{Fig_CoordPolaires.pstricks}
Le même point $M$ peut être décrit indifféremment avec les coordonnées $(x,y)$ ou bien avec $(r,\theta)$.

\begin{remark}
	L'angle $\theta$ d'un point n'étant a priori défini qu'à un multiple de $2\pi$ près, nous convenons de toujours choisir un angle $0\leq\theta<2\pi$. Par ailleurs l'angle $\theta$ n'est pas défini si $(x,y)=(0,0)$.

	La coordonnée $r$ est toujours positive.
\end{remark}

En utilisant la trigonométrie, il est facile de trouver le changement de variable qui donne $(x,y)$ en fonction de $(r,\theta)$:
\begin{subequations}		\label{EqrthetaxyPoal}
	\begin{numcases}{}
		x=r\cos(\theta)\\
		y=r\sin(\theta).
	\end{numcases}
\end{subequations}

%///////////////////////////////////////////////////////////////////////////////////////////////////////////////////////////
\subsubsection{Transformation inverse : théorie}
%///////////////////////////////////////////////////////////////////////////////////////////////////////////////////////////

Voyons la question inverse : comment retrouver $r$ et $\theta$ si on connais $x$ et $y$ ? Tout d'abord,
\begin{equation}
	r=\sqrt{x^2+y^2}
\end{equation}
parce que la coordonnée $r$ est la distance entre l'origine et $(x,y)$. Comment trouver l'angle ? Nous supposons $(x,y)\neq (0,0)$. Si $x=0$, alors le point est sur l'axe vertical et nous avons
\begin{equation}
	\theta=\begin{cases}
		\pi/2	&	\text{si $y>0$}\\
		3\pi/2	&	 \text{si $y<0$.}
	\end{cases}
\end{equation}
Notez que si $y<0$, conformément à notre convention $\theta\geq 0$, nous avons noté $\frac{ 3\pi }{2}$ et non $-\frac{ \pi }{ 2 }$.

Supposons maintenant le cas général avec $x\neq 0$. Les équations \eqref{EqrthetaxyPoal} montrent que
\begin{equation}
	\tan(\theta)=\frac{ y }{ x }.
\end{equation}
Nous avons donc
\begin{equation}
	\theta=\tan^{-1}\left( \frac{ y }{ x } \right).
\end{equation}
La fonction inverse de la fonction tangente est celle définie plus haut.


%///////////////////////////////////////////////////////////////////////////////////////////////////////////////////////////
\subsubsection{Transformation inverse : pratique}
%///////////////////////////////////////////////////////////////////////////////////////////////////////////////////////////

Le code suivant utilise \href{http://www.sagemath.org}{Sage}.

\VerbatimInput[tabsize=3]{calculAngle.py}

Son exécution retourne :
\begin{verbatim}
(sqrt(2), 1/4*pi)
(sqrt(5), pi - arctan(1/2))
(6, 1/6*pi)
\end{verbatim}
Notez que ce sont des valeurs \emph{exactes}. Ce ne sont pas des approximations, ce logiciel travaille de façon symbolique ! Merci donc de jeter vos vieilles calculatrices à la poubelle\footnote{Pensez au recyclage : c'est plein de métaux lourds !} : c'est de la technologie qui n'a plus cours en 2011.

%+++++++++++++++++++++++++++++++++++++++++++++++++++++++++++++++++++++++++++++++++++++++++++++++++++++++++++++++++++++++++++
\section{Coordonnées cylindriques et sphériques}
%+++++++++++++++++++++++++++++++++++++++++++++++++++++++++++++++++++++++++++++++++++++++++++++++++++++++++++++++++++++++++++

Les \defe{coordonnées cylindriques}{coordonnées!cylindrique} sont un perfectionnement des coordonnées polaires. Il s'agit simplement de donner le point $(x,y,z)$ en faisant la conversion $(x,y)\mapsto(r,\theta)$ et en gardant le $z$. Les formules de passage sont
\begin{subequations}
	\begin{numcases}{}
		x=r\cos(\theta)\\
		y=r\sin(\theta)\\
		z=z.
	\end{numcases}
\end{subequations}
Voir les exercices \ref{exoOutilsMath-0005} et \ref{exoOutilsMath-0006}.

Les \defe{coordonnées sphériques}{coordonnées!sphériques} sont ce qu'on appelle les «méridiens» et «longitudes» en géographie. Les formules de transformation sont 
\begin{subequations}		\label{SubEqsCoordSphe}
	\begin{numcases}{}
		x=\rho\sin(\theta)\cos(\varphi)\\
		y=\rho\sin(\theta)\sin(\varphi)\\
		z=\rho\cos(\theta)
	\end{numcases}
\end{subequations}
avec $0\leq\theta\leq\pi$ et $0\leq\varphi<2\pi$.

\begin{remark}
	Attention : d'un livre à l'autre les conventions sur les noms des angles changent. N'essayez donc pas d'étudier par cœur des formules concernant les coordonnées sphériques trouvées autre part. Par exemple sur le premier dessin de \href{http://fr.wikipedia.org/wiki/Coordonnées_sphériques}{wikipédia}, l'angle $\varphi$ est noté $\theta$ et l'angle $\theta$ est noté $\Phi$. Mais vous noterez que sur cette même page, les convention de noms de ces angles changent plusieurs fois.
\end{remark}

Trouvons le changement inverse, c'est à dire trouvons $\rho$, $\theta$ et $\varphi$ en termes de $x$, $y$ et $z$. D'abord nous avons
\begin{equation}
	\rho=\sqrt{x^2+y^2+z^2}.
\end{equation}
Ensuite nous savons que
\begin{equation}
	\cos(\theta)=\frac{ z }{ \rho }
\end{equation}
détermine de façon unique\footnote{Le problème $\rho=0$ ne se pose pas; pourquoi ?} un angle $\theta\in\mathopen[ 0 , \pi \mathclose]$. Dès que $\rho$ et $\theta$ sont connus, nous pouvons poser $r=\rho\sin\theta$ et alors nous nous trouvons avec les équations
\begin{subequations}
	\begin{numcases}{}
		x=r\cos(\varphi)\\
		y=r\sin(\varphi),
	\end{numcases}
\end{subequations}
qui sont similaires à celles déjà étudiées dans le cas des coordonnées polaires.

% TODO: Ajouter un texte sur les équations de plan, et pourquoi ax+by+cz+d=0 est perpendiculaire au vecteur (a,b,c).

%+++++++++++++++++++++++++++++++++++++++++++++++++++++++++++++++++++++++++++++++++++++++++++++++++++++++++++++++++++++++++++
\section{Déterminant et produit vectoriel}
%+++++++++++++++++++++++++++++++++++++++++++++++++++++++++++++++++++++++++++++++++++++++++++++++++++++++++++++++++++++++++++

%---------------------------------------------------------------------------------------------------------------------------
\subsection{Quelque propriétés du déterminant}
%---------------------------------------------------------------------------------------------------------------------------

Une \defe{matrice}{matrice} $2\times 2$ est un tableau de nombres
\begin{equation}
    \begin{pmatrix}
        a    &   b    \\ 
        c    &   d    
    \end{pmatrix}.
\end{equation}
Le \defe{déterminant}{déterminant} de cette matrice est le nombre
\begin{equation}
    \begin{vmatrix}
          a  &   b    \\ 
        c    &   d    
    \end{vmatrix}=ad-cb.
\end{equation}
Nous verrons plus tard\footnote{Et dans les années à venir.} que ce nombre contient énormément d'informations sur la matrice. Il détermine entre autres le nombre de solutions que va avoir le système d'équations linéaires associé à la matrice.

Pour une matrice $3\times 3$, nous avons le même concept, mais un peu plus compliqué. Le déterminant de la matrice
\begin{equation}
    \begin{pmatrix}
        a_{11}    &   a_{12}    &   a_{13}    \\
        a_{21}    &   a_{22}    &   a_{23}    \\
        a_{31}    &   a_{32}    &   a_{33}    
    \end{pmatrix}
\end{equation}
est le nombre
\begin{equation}
    \begin{vmatrix}
        a_{11}    &   a_{12}    &   a_{13}    \\
        a_{21}    &   a_{22}    &   a_{23}    \\
        a_{31}    &   a_{32}    &   a_{33}    
    \end{vmatrix}=
    a_{11}\begin{vmatrix}
        a_{22}  &   a_{23}    \\ 
        a_{32}    &   a_{33}    
    \end{vmatrix}+
    a_{12}\begin{vmatrix}
        a_{21}  &   a_{23}    \\ 
        a_{31}    &   a_{33}
    \end{vmatrix}+
    a_{13}\begin{vmatrix}
        a_{21}  &   a_{22}    \\ 
        a_{31}    &   a_{32}
    \end{vmatrix}.
\end{equation}


\begin{proposition}
    Si on permute deux lignes ou deux colonnes d'une matrice, alors le déterminant change de signe.
\end{proposition}

\begin{proposition}
    Si on multiplie une ligne ou une colonne d'une matrice par un nombre $\lambda$, alors le déterminant est multiplié par $\lambda$.
\end{proposition}

\begin{proposition}
    Si deux lignes ou deux colonnes sont proportionnelles, alors le déterminant est nul.
\end{proposition}

\begin{proposition}
    Si on ajoute à une ligne une combinaison linéaire des autres lignes, alors le déterminant ne change pas (idem pour les colonnes).
\end{proposition}

%---------------------------------------------------------------------------------------------------------------------------
\subsection{Produit vectoriel}
%---------------------------------------------------------------------------------------------------------------------------

Une application importante du déterminant $3\times 3$ est qu'il détermine le \defe{produit vectoriel}{produit!vectoriel} entre deux vecteurs. Pour cela nous introduisons les vecteurs de base
\begin{equation}
    \begin{aligned}[]
        e_x&=\begin{pmatrix}
            1    \\ 
            0    \\ 
            0    
        \end{pmatrix}
        ,&e_y=\begin{pmatrix}
            0    \\ 
            1    \\ 
            0    
        \end{pmatrix},&e_z&=\begin{pmatrix}
            0    \\ 
            0    \\ 
            1    
        \end{pmatrix}.
    \end{aligned}
\end{equation}
Ensuite, si $v$ et $w$ sont des vecteurs dans $\eR^3$, nous définissons
\begin{equation}
    \begin{aligned}[]
        \begin{pmatrix}
            v_x    \\ 
            v_y    \\ 
            v_z    
        \end{pmatrix}\times\begin{pmatrix}
            w_x    \\ 
            w_y    \\ 
            w_z    
        \end{pmatrix}=
        \begin{vmatrix}
              e_x  &   e_y    &   e_z    \\
              v_x  &   v_y    &   v_z    \\
              w_x  &   w_y    &   w_z    \\
        \end{vmatrix}&=
        (v_yw_z-w_yvz)e_x\\
        &-(v_xw_z-w_xvz)e_y\\
        &+(v_xw_y-w_xvy)e_z\in\eR^3
    \end{aligned}
\end{equation}

Ce produit vectoriel peut aussi être écrit sous la forme
\begin{equation}        \label{EqProdVectEspilonijk}
    v\times w=\sum_{i,j,k}\epsilon_{ijk}v_iw_j1_k
\end{equation}
où $\epsilon_{ijk}$ est défini par $\epsilon_{xyz}=1$ et ensuite $\epsilon_{ijk}$ est $1$ ou $-1$ suivant que la permutation des $x$, $y$ et $z$ est paire ou impaire.

Un grand intérêt du produit vectoriel est qu'il fournit un vecteur qui est simultanément perpendiculaire aux deux vecteurs donnés.
\begin{proposition}
    Le vecteur $v\times w$ est perpendiculaire à $v$ et à $w$.
\end{proposition}

\begin{proposition}
    Le produit vectoriel est une opération antisymétrique, c'est à dire
    \begin{equation}
        v\times w=-w\times v.
    \end{equation}
    En particulier $v\times v=0$ pour tout vecteur $v\in\eR^3$.
\end{proposition}

\begin{proposition}
    Le produit vectoriel est linéaire. Pour tout vecteurs $a$, $b$, $c$ et pour tout nombre $\alpha$ et $\beta$ nous avons
    \begin{equation}
        \begin{aligned}[]
            a\times (\alpha b +\beta c)&=\alpha(a\times b)+\beta(a\times c)\\
            (\alpha a+\beta b)\times c&=\alpha(a\times c)+\beta(b\times c).
        \end{aligned}
    \end{equation}
\end{proposition}

Les trois vecteurs de base $e_x$, $e_y$ et $e_y$ ont des produits vectoriels faciles à retenir :
\begin{equation}
    \begin{aligned}[]
        e_x\times e_y&=e_z\\
        e_y\times e_z&=e_x\\
        e_z\times e_x&=e_y
    \end{aligned}
\end{equation}

\begin{example}
    Calculons le produit vectoriel $v\times w$ avec
    \begin{equation}
        \begin{aligned}[]
            v&=\begin{pmatrix}
                3    \\ 
                -1    \\ 
                1    
            \end{pmatrix}&w=\begin{pmatrix}
                1    \\ 
                2    \\ 
                -1    
            \end{pmatrix}.
        \end{aligned}
    \end{equation}
    Les vecteurs s'écrivent sous la forme $v=3e_x-e_y+e_z$ et $w=e_x+2e_y-e_z$. Le produit vectoriel s'écrit
    \begin{equation}
        \begin{aligned}[]
            (3e_x-e_y+e_z)\times (e_x+2e_y-e_z)&=6e_x\times e_y-3e_x\times e_z\\
                                &\quad -e_y\times e_x + e_y\times e_z\\
                                &\quad + e_z\times e_x + 2e_z\times e_y\\
                                &=6e_z+3e_y+e_z+e_x+e_y-2e_x\\
                                &=-e_x+4e_y+7e_z.
        \end{aligned}
    \end{equation}
\end{example}

%---------------------------------------------------------------------------------------------------------------------------
\subsection{Produit mixte}
%---------------------------------------------------------------------------------------------------------------------------

Si $a$, $b$ et $c$ sont trois vecteurs, leur \defe{produit mixte}{produit!mixte} est le nombre $a\cdot(b\times c)$. En écrivant le produit vectoriel sous forme de somme de trois déterminants $2\times 2$, nous avons
\begin{equation}
    \begin{aligned}[]
        a\cdot& (b\times c)\\&=(a_1e_x+a_2e_y+a_3e_z)\cdot\left(
        \begin{vmatrix}
            b_2    &   b_3    \\ 
            c_2    &   c_3    
        \end{vmatrix}e_x-\begin{vmatrix}
            b_1    &   b_3    \\ 
            c_1    &   c_3    
        \end{vmatrix}e_y+\begin{vmatrix}
            b_1    &   b_2    \\ 
            c_1    &   c_2    
        \end{vmatrix}\right)\\
        &=a_1\begin{vmatrix}
            b_2    &   b_3    \\ 
            c_2    &   c_3    
        \end{vmatrix}-a_2\begin{vmatrix}
            b_1    &   b_3    \\ 
            c_1    &   c_3    
        \end{vmatrix}+a_3\begin{vmatrix}
            b_1    &   b_2    \\ 
            c_1    &   c_2    
        \end{vmatrix}\\
        &=\begin{vmatrix}
            a_1    &   a_2    &   a_3    \\
            b_1    &   b_2    &   b_3    \\
            c_1    &   c_2    &   c_3
        \end{vmatrix}.
    \end{aligned}
\end{equation}
Le produit mixte s'écrit donc sous forme d'un déterminant. Nous retenons cette formule:
\begin{equation}        \label{EqProduitMixteDet}
    a\cdot (b\times c)=\begin{vmatrix}
        a_1    &   a_2    &   a_3    \\
        b_1    &   b_2    &   b_3    \\
        c_1    &   c_2    &   c_3
    \end{vmatrix}.
\end{equation}


\begin{proposition}
    Le produit vectoriel $a\times b$ est un vecteur orthogonal à $a$ et $b$.
\end{proposition}

\begin{proof}
    Vérifions que $a\perp (a\times b)$. Pour cela, nous calculons $a\cdot (a\times b)$, c'est à dire le produit mixte
    \begin{equation}
        a\cdot(a\times b)=\begin{vmatrix}
            a_1    &   a_2    &   a_3    \\
            a_1    &   a_2    &   a_3    \\
            b_1    &   b_2    &   b_3
        \end{vmatrix}=0.
    \end{equation}
    L'annulation de ce déterminant est due au fait que deux de ses lignes sont égales.
\end{proof}

\begin{proposition}     \label{PropNormeProdVectoabsint}
    Nous avons
    \begin{equation}
        \| a\times b \|=\| a \|\| b \|\sin(\theta)
    \end{equation}
    où $\theta\in\mathopen[ 0.\pi \mathclose]$ est l'angle formé par $a$ et $b$.
\end{proposition}

\begin{proof}
    En utilisant la décomposition du produit vectoriel, nous avons
    \begin{equation}
        \begin{aligned}[]
            \| a\times b \|^2&=\begin{vmatrix}
                a_2    &   a_3    \\ 
                b_2    &   b_3    
            \end{vmatrix}^2+\begin{vmatrix}
                a_1    &   a_3    \\ 
                b_1    &   b_3    
            \end{vmatrix}^2+\begin{vmatrix}
                a_1    &   a_2    \\ 
                b_1    &   b_2    
            \end{vmatrix}^2\\
            &=(a_2b_3-b_2a_3)^2+(a_1b_3-a_3b_1)^2+(a_1b_2-a_2b_1)^2\\
            &=(a_1^2+a_2^2+a_3^2)(b_1^2+b_2^2+b_3^2)-(a_1b_1+a_2b_2+a_3b_3)^2\\
            &=\| a \|^2\| b \|^2-(a\cdot b)^2\\
            &=\| a \|^2\| b \|^2-\| a \|^2\| b \|^2\cos^2(\theta)\\
            &=\| a \|^2\| b \|^2\big( 1-\cos^2(\theta) \big)\\
            &=\| a \|^2\| b \|^2\sin^2(\theta).
        \end{aligned}
    \end{equation}
    D'où le résultat.
\end{proof}

\begin{remark}      \label{RemaAireParalProdVect}
    Le nombre $\| a \|\| b \|\sin(\theta)$ est l'aire du parallélogramme formé par les vecteurs $a$ et $b$, comme cela se voit sur la figure \ref{LabelFigParallelogramme}.
    \newcommand{\CaptionFigParallelogramme}{Calculer l'aire d'un parallélogramme.}
    \input{Fig_Parallelogramme.pstricks}
\end{remark}

%---------------------------------------------------------------------------------------------------------------------------
\subsection{Interprétation géométrique du déterminant}
%---------------------------------------------------------------------------------------------------------------------------

%///////////////////////////////////////////////////////////////////////////////////////////////////////////////////////////
\subsubsection{Déterminant de dimension deux}
%///////////////////////////////////////////////////////////////////////////////////////////////////////////////////////////

La valeur absolue du déterminant 
\begin{equation}        \label{EqDeratb}
    \begin{vmatrix}
        a_1    &   a_2    \\ 
        b_1    &   b_2    
    \end{vmatrix}
\end{equation}
est l'aire du parallélogramme déterminé par les vecteurs $\begin{pmatrix}
    a_1    \\ 
    a_2    
\end{pmatrix}$ et $\begin{pmatrix}
    b_1    \\ 
    b_2    
\end{pmatrix}$. En effet, d'après la remarque \ref{RemaAireParalProdVect}, l'aire de ce parallélogramme est donnée par la norme du produit vectoriel
\begin{equation}
    \begin{pmatrix}
        a_1    \\ 
        a_2    \\ 
        0    
    \end{pmatrix}\times
    \begin{pmatrix}
          b_1  \\ 
        b_2    \\ 
        0    
    \end{pmatrix}=\begin{vmatrix}
        e_x    &   e_y    &   e_z    \\
        a_1    &   a_2    &   0    \\
        b_1    &   b_2    &   0
    \end{vmatrix}=
    \begin{vmatrix}
        a_1    &   a_2    \\ 
        b_1    &   b_2    
    \end{vmatrix}e_z,
\end{equation}
donc la norme $\| a\times b \|$ est bien donnée par la valeur absolue du déterminant \eqref{EqDeratb}.

%///////////////////////////////////////////////////////////////////////////////////////////////////////////////////////////
\subsubsection{Déterminant de dimension trois}
%///////////////////////////////////////////////////////////////////////////////////////////////////////////////////////////

Si les vecteurs $a$, $b$ et $c$ sont sont pas coplanaires, alors la valeur absolue du produit mixte (voir équation \eqref{EqProduitMixteDet}) $a\cdot(b\times c)$ donne le volume du parallélépipède construit sur les vecteurs $a$, $b$ et $c$.

En effet si $\varphi$ est l'angle entre $b\times c$ et $a$, alors la hauteur du parallélépipède vaut $\| a \|\cos(\varphi)$. En effet la direction verticale est donnée par $b\times c$, et la hauteur est alors la «composante verticale» de $a$. Par conséquent, étant donné que $\| b\times c \|$ est la surface de la base, le volume du parallélépipède vaut
\begin{equation}
    V=\| b\times c\|  \| a \|\cos(\varphi).
\end{equation}
Or cette formule est le produit scalaire de $a$ par $b \times c$; ce dernier étant donné par le déterminant de la matrice formée des composantes de $a$, $b$ et $c$ grâce à la formule \eqref{EqProduitMixteDet}.

