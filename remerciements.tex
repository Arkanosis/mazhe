%+++++++++++++++++++++++++++++++++++++++++++++++++++++++++++++++++++++++++++++++++++++++++++++++++++++++++++++++++++++++++++
\section{Auteurs, contributeurs, sources et remerciements}
%+++++++++++++++++++++++++++++++++++++++++++++++++++++++++++++++++++++++++++++++++++++++++++++++++++++++++++++++++++++++++++

Les remerciements sont mis dans l'ordre chronologique approximatif.

\begin{description}
    \item[Les personnes qui ont mis du code \LaTeX] 
        \begin{enumerate}
            \item
                Nicolas Richard et Ivik Swan pour les parties des exercices et rappels de calcul différentiel et intégral (Université libre de Bruxelles, 2003-2004) qui leurs reviennent.
            \item
                Carlotta Donadello pour la partie géométrie analytique : topologie dans \( \eR^n\), courbes, intégrales, limites. (Université de Franche-Comté 2010-2012)
            \item
                Lilian Besson pour m'avoir signalé un paquet de fautes et quelques points pas clairs en statistiques.
        \end{enumerate}
    \item[Ceux à qui je dois beaucoup plus que un ou deux développements]
        \begin{enumerate}
            \item 
                Arnaud Girand\cite{KXjFWKA} pour avoir mis des tonnes de développements bien faits en ligne. Une bonne vingtaine de résultats un peu partout dans ces notes viennent de lui.
            \item
                Le site \url{http://www.les-mathematiques.net} m'a donné les preuves de nombreux résultats.
            \item
                Pierre Monmarché\cite{PAXrsMn} pour son document en ligne tout plein de développements. 
        \end{enumerate}
    \item[Autres contributions et aides]


\begin{enumerate}
    \item
        Plein de monde pour diverses contributions à des énoncés d'exercices. Pierre Bieliavsky pour les énoncés d'analyse numérique (MAT1151 à Louvain la Neuve 2009-2010). Jonathan Di Cosmo pour certaines corrections de MAT1151. François Lemeux, exercices sur les normes de matrices et correction de coquilles. Martin Meyer et Mustapha Mokhtar-Kharroubi pour certains exercices de outils mathématiques (surtout ceux des DS et examens).
    \item 
    \item
        Mihai Bostan nous a donné ses notes manuscrites de son cours présentiel de géométrie analytique 2009-2010. (presque) Toute la structure du cours de géométrie analytique lui est due (qui est maintenant fondue un peu partout dans les chapitres d'analyse).
    \item
         Mustapha Mokhtar-Kharroubi pour la matière et la structure du cours d'outils mathématiques.
    \item
        Les étudiants de géométrie analytique en CTU 2010-2011 ont détecté d'innombrables coquilles.  Les étudiants du cours présentiel de géométrie analytique 2011-2012 ont signalé un certain nombre d'incorrections dans les exercices et les corrigés.  Les agrégatifs de Besançon 2011-2012 pour leurs plans et leurs développements.
    \item
        Tous les contributeurs du Wikipédia francophone (et aussi un peu l'anglophone) doivent être remerciés. J'en ai pompé des quantités astronomiques; des articles utilisés sont cités à divers endroits du texte, mais ce n'est absolument pas exhaustif. Il n'y a à peu près pas un résultat important de ces notes dont je n'aie pas lu la page Wikipédia, et souvent plusieurs pages connexes.
    \item
        Des centaines de profs qui ont mis leurs polys sur internet. Des dizaines d'entre eux ont des sites dédiés à l'agrégation. Ils sont cités en bibliographie aux endroits où ils sont utilisés. Merci à toutes et à tous d'avoir codé et publié. Encore plus merci à celles et ceux qui ont pris la peine de préciser une licence et merci spécial pour les licences libres (quant à ceux dont la licence libre couvre également le code \LaTeX\ publié \ldots).
    \item
        Le forum usenet de math, en particulier pour la construction des corps fini dans la fil « Vérifier qu'un polynôme est primitif » initié le 20 décembre 2011.
    \item
        Les intervenants du fil «\href{http://www.les-mathematiques.net/phorum/read.php?2,302266}{Antisymétrisation, alterné, déterminant et caractéristique}» sur \texttt{les-mathematiques.net} m'ont bien aidé pour la section sur les déterminants \ref{SecGYzHWs} (bien qu'ils ne le savent pas).
    \item
        Plouf qui m'a signalé une coquille dans le fil \href{http://passeurdesciences.blog.lemonde.fr/2014/01/24/la-selection-scientifique-de-la-semaine-numero-106}{la-selection-scientifique-de-la-semaine-numero-106}.
    \item
        Xavier Mauquoy pour l'énoncé et la preuve du théorème \ref{THOooYXJIooWcbnbm}.
\end{enumerate}
\end{description}


J'ai souvent donné entre parenthèse à côté des mots « théorème », « lemme » ou « proposition » une ou plusieurs références vers les sources de la preuve que je donne. Ce sont parfois des liens vers la bibliographie; parfois aussi des liens hypertexte vers des sites, des blogs, etc. Tous ces gens ont fait du bon boulot. Sans toute cette « communauté », l'internet serait mort\footnote{Cette dernière phrase doit être comprise comme un appel à ne pas utiliser Moodle et autres iCampus pour diffuser vos cours de math, ou en tout cas pas comme moyen exclusif.}.

\begin{center}
Le photocopiage ne tuera pas ce livre. Téléchargez, imprimez, photocopiez, diffusez; c'est cela qui fait vivre un livre.
\end{center}

