% This is part of the Exercices et corrigés de CdI-2.
% Copyright (C) 2008, 2009
%   Laurent Claessens
% See the file fdl-1.3.txt for copying conditions.


\begin{corrige}{_I-3-10}

Certains résultats de cet exercice sont aussi à voir dans le cours, à la page I.78.

\begin{enumerate}
\item 

Affin d'étudier la $n$ième dérivée de $\Gamma(x)$, nous divisons l'intégrale en deux parties :
\begin{equation}
	I_n(x)=\int_0^1\frac{ \partial^n }{ \partial x^n }\left(  e^{-t}t^{x-1} \right)dt=\int_0^1 e^{-t} t^{x-1} (\ln(t))^n dt,
\end{equation}
et 
\begin{equation}
	J_n(x)=\int_0^{\infty}\frac{ \partial^n }{ \partial x^n }\left(  e^{-t}t^{x-1} \right)dt=\int_0^{\infty} e^{-t} t^{x-1} (\ln(t))^n dt,
\end{equation}
avec $n\geq 0$. Affin d'étudier la convergence de $I$ et de $J$, nous posons
\begin{equation}
	L=\lim_{\substack{t\to 0\\t>0}} |  e^{-t}t^{x-1}(\l t)^n |t^{\alpha}= \lim_{\substack{t\to 0\\t>0}} t^{\alpha+x-1}| \ln t |^n=(-1)^n\lim_{\substack{t\to 0\\t>0}} (\ln t)^nt^{\alpha+x-1},
\end{equation}
et nous utilisons le critère des fonctions tests, corollaire \ref{CorCritFonsTest}. Dès que $\alpha\leq 1-x$, nous avons $L=\infty$. Donc si $x\leq0$, nous prenons $\alpha=1$ et nous concluons à ce que $I_n(x)$ n'existe pas. Si, par contre, $\alpha+x-1>0$, alors $L=0$. Donc dès que $x>0$, nous pouvons trouver un $\alpha<1$ tel que $\alpha+x-1>0$, et donc $I_n(x)$ existe quand $x>0$.

En conclusion, $I_n(x)$ existe si et seulement si $x>0$. Maintenant que nous savons l'existence de $I_n(x)$, nous étudions sa convergence quand $x>0$.

Sur un compact dont le minimum est $\epsilon$, nous avons
\begin{equation}
	|  e^{-t}t^{x-1}(\ln t)^n |< e^{-t}t^{\epsilon-1}| \ln t |^n,
\end{equation}
mais l'intégrale du membre de droite entre $0$ et $1$ existe (c'est $I_n(\epsilon)$), donc $I_n(x)$ converge uniformément sur tout compact de $]0,\infty[$. En conséquence de quoi, nous avons
\begin{enumerate}
\item $I_n(x)$ est continue,
\item $I_n(x)=\frac{ d^n }{ dx^n }\left( I_0(x) \right)$
\end{enumerate}
pour tout $n\geq 1$. Cela prouve que $I_0(x)$ est $ C^{\infty}$ sur $]0,\infty[$.

Nous passons maintenant à l'étude de $J_n(x)$. En utilisant l'astuce \eqref{EqExpDecrtPlusVite}, nous avons
\begin{equation}
	\lim_{t\to\infty}|  e^{-t}t^{x-1}(\ln t)^n |t^{\alpha}=0,
\end{equation}
de telle sorte que $J_n(x)$ existe. Son type de convergence est étudiée sur un compact en $x$ dont le maximum est $A$. Si $t\geq 1$, nous avons
\begin{equation}
	 e^{-t}t^{x-1}(\ln t)^n\leq  e^{-t}t^{A-1}(\ln t)^n.
\end{equation}

\item
Le calcul de $\Gamma(x+1)$ revient au le calcul de
\begin{equation}
	\int_0^{\infty} e^{-t}t^x,
\end{equation}
qui peut s'intégrer par partie en posant $u=t^x$ et $dv= e^{-t}dt$. Le résultat est
\begin{equation}
	\Gamma(x+1)=0+x\int_0^{\infty} e^{-t}t^{x-1}dt=x\Gamma(x).
\end{equation}

\item
Le calcul de $\Gamma(\frac{ 1 }{2})$ est un simple changement de variable dans l'intégrale. En posant $u=t^{1/2}$, nous trouvons
\begin{equation}
\Gamma(\frac{ 1 }{2})=\int_0^{\infty} e^{-t}t^{-\frac{ 1 }{2}}dt=2\int_0^{\infty} e^{-u^2}du.
\end{equation}

\item
Nous avons
\begin{equation}
	\Gamma(\frac{ 1 }{2})^2=\Gamma(\frac{ 1 }{2})\Gamma(\frac{ 1 }{2})=4\int_0^{\infty} e^{-u^2}du\int_0^{\infty} e^{-v^2}dv.
\end{equation}
La fonction $(u,v)\mapsto e^{-(u^2+v^2)}$ étant intégrable sur $[0,\infty[\times [0,\infty[$, le théorème de Fubini \ref{ThoFubini}, et en particulier la formule \eqref{EqFubiniFactori} s'appliquent et nous avons
\begin{equation}
	I=\frac{1}{ 4 }\Gamma(\frac{ 1 }{2})^2=\iint_{\substack{u>0\\v>0}} e^{-(u^2+v^2)}dudv.
\end{equation}
Cette intégrale se calcule en effectuant le changement de variable polaire : $u=r\cos\theta$, $v=r\sin\theta$ :
\begin{equation}
	I=\iint_{\substack{r>0\\0\leq\theta\leq\frac{ \pi }{2}}} e^{-r^2}rdrd\theta=\int_0^{\pi/2}\left( \int_0^{\infty} e^{-r^2}rdr \right)d\theta
		=\frac{ \pi }{2}\int_0^{\infty}\frac{ 1 }{2} e^{-u}du=\frac{ \pi }{ 4 }
\end{equation}
où nous avons effectué un changement de variable $u=r^2$ pour effectuer la dernière intégrale. La formule $\Gamma(\frac{ 1 }{2})=\sqrt{\pi}$ est maintenant prouvée.

\item
Il suffit de remarquer que l'intégrale que nous devons calculer n'est autre que $\Gamma(\frac{ 3 }{ 2 })$. En utilisant la formule \eqref{EqGammaFacto}, nous trouvons
\begin{equation}
	\int_0^{\infty} e^{-t}\sqrt{t}dt=\Gamma( \frac{ 1 }{ 2 }+1 )=\frac{ 1 }{2}\Gamma(\frac{ 1 }{ 2 })=\frac{ \sqrt{\pi} }{2}.
\end{equation}

\item
Un premier changement de variable $u=\ln x$ donne
\begin{equation}
	I=\int_0^1x(\ln x)^{-1/3}dx=\int_{-\infty}^0 e^{u}u^{-1/3} e^{u}du=-\int_0^{\infty} e^{2u}u^{-1/3}.
\end{equation}
À partir de là, le changement de variable $t=-2u$ fourni la solution.

\end{enumerate}


\end{corrige}

