% This is part of Mes notes de mathématique
% Copyright (c) 2011-2012
%   Laurent Claessens
% See the file fdl-1.3.txt for copying conditions.

Sur les espaces projectifs : \cite{ProjRolland}.

L'espace projectif de \( E\) est l'ensemble des droites vectorielles de \( E\).
\begin{definition}
    Soit \( \eK\) un corps et \( E\) un espace vectoriel de dimension finie sur \( \eK\). Nous définissons sur \( E\setminus\{ 0 \}\) la relation d'équivalence \( u\sim v\) si et seulement si \( u=\lambda v\) pour un certain \( \lambda\in\eK\). Cette relation est la relation de \defe{colinéarité}{colinéarité}. L'ensemble des classes d'équivalence de \( \sim\) est l'\defe{espace projectif}{espace!projectif}\index{projectif!espace} de \( E\) et sera noté \( P(E)\)\nomenclature[G]{\( P(E)\)}{l'espace projectif de $E$}.
\end{definition}

Si \( \dim E=2\), l'ensemble \( P(E)\) est la \defe{droite projective}{droite!projective}\index{projectif!droite}, et si \( \dim E=3\) nous parlons du \defe{plan projectif}{plan!projectif}\index{projectif!plan}.

Étant donné que tous les \( \eK\)-espaces vectoriels de dimensions \( n+1\) sont isomorphes à\( \eK^{n+1}\), nous noterons \( P_n(\eK)\) ou \( P_n\) l'espace projectif \( P(\eK^{n+1})\).

\begin{example}
    Si \( n=1\) et \( \eK=\eR\), l'espace projectif est l'ensemble des droites vectorielles dans le plan usuel. Il y en a une pour chaque point du type \( (x,1)\) avec \( x\in\eR\) et ensuite une horizontale, passant par le point \( (1,0)\). Nous avons donc
    \begin{equation}
        P_1(\eR)=\{ (1,0) \}\cup\{ (x,1)\tq x\in \eR \}.
    \end{equation}
    Le point \( (1,0)\) est dit «point à l'infini».
\end{example}

%+++++++++++++++++++++++++++++++++++++++++++++++++++++++++++++++++++++++++++++++++++++++++++++++++++++++++++++++++++++++++++
\section{Sous espaces projectifs}
%+++++++++++++++++++++++++++++++++++++++++++++++++++++++++++++++++++++++++++++++++++++++++++++++++++++++++++++++++++++++++++

Un \defe{sous espace projectif}{projectif!sous espace} de \( P(E)\) est une partie de la forme \( P(F)\) où \( F\) est un sous espace vectoriel de \( E\).

\begin{proposition}     \label{PropuqpWVx}
    Si \( F\) et \( G\) sont des sous espaces vectoriels de \( E\), alors
    \begin{equation}
        P(F)\cap P(G)=P(F\cap G)
    \end{equation}
    et nous avons
    \begin{equation}        \label{EqNAdWfN}
        \dim P(F)+\dim P(G)=\dim P(F+G)+\dim P(F\cap G).
    \end{equation}
\end{proposition}

\begin{proof}
    Nous avons 
    \begin{equation}
        P(F)=\{ [v]\tq v\in F \}
    \end{equation}
    où les crochets signifient la classe par rapport à la relation de colinéarité. Nous avons alors
    \begin{equation}
        P(F)\cap P(G)=\{ [v]\tq v\in F\cap G \}=P(F\cap G).
    \end{equation}
    Cela prouve le premier point.

    En ce qui concerne l'équation \eqref{EqNAdWfN}, en considérant \( \dim P(E)=\dim E-1\) nous devons prouver l'égalité
    \begin{equation}
        \dim F+\dim G=\dim (F+G)+\dim(F\cap G)
    \end{equation}
    concernant les dimensions des espaces vectoriels usuelles. Si nous considérons une base de \( E\) telle que \( B_1=\{ e_1,\ldots, e_{k_1} \}\) est une base de \( F\cap G\), \( B_2=\{ e_{k_1+1},\ldots, e_{k_2} \}\) complète \( B_1\) en une base de \( F\) et \( B_3=\{ e_{k_2+1},\ldots, e_n \}\) complète \( B_1\cup B_2\) en une base de \( G\).

    Nous avons alors
    \begin{subequations}
        \begin{align}
            \dim F+\dim G&=2\Card(B_1)+\Card(B_2)+\Card(b_3)\\
            \dim(F+G)&=\Card(B_1)+\Card(b_2)+\Card(B_3)\\
            \dim(F\cap G)&=\Card(B_1).
        \end{align}
    \end{subequations}
    De là la relation \eqref{EqNAdWfN} se déduit immédiatement.    
\end{proof}

\begin{theorem}[incidence]\index{théorème!incidence}
    Soient \( F\) et \( F\) deux sous espaces vectoriels de \( E\) tels que 
    \begin{equation}
        \dim P(F)+\dim P(G)\geq \dim P(E).
    \end{equation}
    Alors \( P(F)\cap P(G)\neq \emptyset\).
\end{theorem}

\begin{proof}
    En utilisant les hypothèses et la proposition \ref{PropuqpWVx} nous avons
    \begin{equation}
        \dim P(E)+\dim P(G)=\dim P(F+G)+\dim P(F\cap G)\geq \dim P(E).
    \end{equation}
    En passant aux espaces vectoriels correspondants,
    \begin{equation}
        \dim(F+G)+\dim(F\cap G)\geq \dim(E)+1.
    \end{equation}
    Mais nous avons aussi \( \dim(F+G)\leq \dim(E)\) et par conséquent \( \dim(F\cap G)\geq 1\). Au final, \( \dim P(F\cap G)\geq 0\). Cela prouve que \( P(F\cap G)\) contient au moins un élément (nous rappelons que lorsqu'un espace projectif contient un seul élément, sa dimension est zéro).
\end{proof}

\begin{example}
    Soient les plans \( \Pi_1\equiv x=0\) et \( \Pi_2\equiv y=0\). Nous avons
    \begin{subequations}
        \begin{align}
            P(\Pi_1)&=\{ [0,y,1] \}\cup\{ [0,1,0] \}\\
            P(\Pi_2)&=\{ [x,0,1] \}\cup\{ [1,0,0] \}
        \end{align}
    \end{subequations}
    où le crochet signifie la classe pour la colinéarité. Ces deux droites projectives ont comme point d'intersection le point \( [0,0,1]\).
\end{example}

\begin{definition}
    Un \defe{hyperplan projectif}{projectif!hyperplan} est un sous espace projectif de \( P(E)\) de la forme \( P(V)\) où \( V\) est un hyperplan de \( E\).
\end{definition}

\begin{proposition}
    Soit \( H=P(V)\) un hyperplan projectif de \( P(E)\) et soit \( m\) hors de \( H\). Alors toute droite projective passant par \( m\) coupe \( H\) en un et un seul point.
\end{proposition}

\begin{proof}
    Si \( \dim E=n\) nous avons \( \dim V=n-1\). Soit \( d=P(D)\) une droite projective passant par \( m\), c'est à dire que \( D\) est de dimension \( 2\) dans \( E\). Si \( D\subset V\) alors \( m\in P(D)\subset P(V)\); or nous avons demandé que \( m\) soit hors de \( P(V)\). Par conséquent \( D\) n'est pas inclus à \( V\) et en particulier \( \dim(D+V)=\dim(E)\).

    Nous recopions la formule \eqref{EqNAdWfN} pour notre cas :
    \begin{equation}
        \underbrace{\dim d}_{=1}+\underbrace{\dim H}_{=n-2}=\underbrace{\dim P(D+V)}_{=n-1}+\dim P(D\cap V).
    \end{equation}
    Nous avons donc \( \dim P(D\cap V)=0\), ce qui signifie que l'ensemble \( P(D\cap V)=P(D)\cap P(V)=d\cap H\) contient un et un seul point.
\end{proof}

%+++++++++++++++++++++++++++++++++++++++++++++++++++++++++++++++++++++++++++++++++++++++++++++++++++++++++++++++++++++++++++
\section{Espace projectifs comme «complétés» d'espaces affines}
%+++++++++++++++++++++++++++++++++++++++++++++++++++++++++++++++++++++++++++++++++++++++++++++++++++++++++++++++++++++++++++

Soit \( E\) un espace vectoriel de dimension \( 2\) et \( P(E)\) la droite projective correspondante, et soit \( \{ e_1,e_2 \}\) une base de \( E\). Nous considérons la droite affine \( d\equiv y=1\). Nous avons la bijection
\begin{equation}        \label{EqvrfDLz}
    \begin{aligned}
        \phi\colon d\cup\{ \infty \}&\to P(E) \\
        (x,1)&\mapsto \text{la droite vectorielle passant par \( (x,1)\)} \\
        \infty&\mapsto \text{la droite vectorielle passant par \( (1,0)\)}.
    \end{aligned}
\end{equation}

\begin{lemma}
    Si nous munissons l'ensemble \( d\cup\{ \infty \}\) de la topologie compactifiée d'Alexandroff, la bijection \eqref{EqvrfDLz} est un homéomorphisme.
\end{lemma}

Soient maintenant les plans affins dans l'espace vectoriel \( E\) de dimension \( 3\)
\begin{subequations}
    \begin{align}
        \Pi_1\equiv z&=0\\
        \Pi_2\equiv z&=1.
    \end{align}
\end{subequations}
Une droite (vectorielle) de \( E\) coupe \( \Pi_2\) en un et un seul point, sauf si elle est contenue dans \( \Pi_1\). Nous avons donc une bijection
\begin{equation}
    \begin{aligned}
        \phi\colon P(E)&\to \Pi_2\cup P(\Pi_1) \\
        d&\mapsto \begin{cases}
            \Pi_2\cap d    &   \text{si cette intersection est non vide}\\
            d    &    \text{sinon.}
        \end{cases}
    \end{aligned}
\end{equation}
La droite projective \( P(\Pi_1)\) est la droite à l'infini du plan projectif \( P(E)\). Nous voyons que le plan projectif \( P(E)\) peut être vu comme un plan affine \( (\Pi_2)\) «complété»  par une droite affine \( P(\Pi_1)\). Cette dernière droite est elle-même une droite affine complétée par un point à l'infini.

Nous pouvons généraliser cette démarche en considérant un espace affine \( \affE\) de direction \( E\) sur le corps \( \eK\). Nous construisons \( F=E\times \eK\) et nous considérons un repère affine sur \( F\) tel que \( E\equiv x_{n+1}=0\). Nous pouvons donc identifier \( \affE\) à l'hyperplan affine d'équation \( x_{n+1}=1\) dans \( F\).

Une droite vectorielle de \( F\) non contenue dans \( E\) coupe \( \affE\) en un unique point; nous avons donc une bijection
\begin{equation}
    \affE\cup P(E)\to P(F).
\end{equation}
Dans ce cadre, \( P(E)\) est l'hyperplan à l'infini et nous disons que \( P(E)\) est la \defe{complétion projective}{complétion!projective}\index{projectif!complétion} de \( \affE\).

\begin{definition}
    Soit \( E\) un espace vectoriel de dimension \( 3\). Nous disons que \( d\subset P(E)\) est une \defe{droite projective}{projectif!droite} de \( P(E)\) si \( d=P(D)\) pour une plan vectoriel \( D\subset E\).
\end{definition}

\begin{example}
    Nous considérons les plans affins
    \begin{subequations}
        \begin{align}
            \Pi_1&\equiv z=0\\
            \Pi_2&\equiv z=1
        \end{align}
    \end{subequations}
    et nous avons la bijection
    \begin{equation}
        P(E)=\Pi_2\cup P(\Pi_1).
    \end{equation}
    Un plan affine \( D\) a deux possibilités : soit il coupe \( \Pi_2\) en une droite, soit il est égal à \( \Pi_1\). Si \( D\cap\Pi_2=d\) (\( d\) est une droite affine), alors nous avons
    \begin{equation}
        P(D)=d\cup\{ \infty_D \},
    \end{equation}
    ce qui justifie la terminologie comme quoi \( P(D)\) est une droite dans \( P(E)\).
\end{example}

Soit \( E\) un espace vectoriel de dimension \( 3\) et le plan projectif \( P(E)\). Nous avons deux types de droites projectives :
\begin{enumerate}
    \item
        D'abord nous avons la droite à l'infini, donnée\footnote{Dans notre représentation usuelle du plan projectif \( z=1\).} par \( P(z=0)\).
    \item
        Ensuite nous avons toutes les droites affines du plan \( z=1\). Chacune de ces droites est complétée par un point à l'infini. 
\end{enumerate}

\begin{example}     \label{ExempMyTmFp}
    Étudions un peu le second type de droites. D'abord si deux droites sont parallèles, leurs points à l'infini sont identiques. Prenons par exemple les droites \( d=\{ z=1,x=1 \}\) et \( d'=\{ z=1,x=2 \}\). Elles décrivent les directions des vecteurs
    \begin{equation}
        \begin{aligned}[]
            \begin{pmatrix}
                1    \\ 
                 y   \\ 
                1    
            \end{pmatrix}&&\text{et}&&
            \begin{pmatrix}
                2    \\ 
                y    \\ 
                1    
            \end{pmatrix}.
        \end{aligned}
    \end{equation}
    En normalisant, ce sont les vecteurs
    \begin{equation}
        \begin{aligned}[]
            \frac{1}{ \sqrt{2+y^2} }\begin{pmatrix}
                1    \\ 
                y    \\ 
                1    
            \end{pmatrix}&&\text{et}&&
            \frac{1}{ \sqrt{5+y^2} }\begin{pmatrix}
                2    \\ 
                y    \\ 
                1    
            \end{pmatrix},
        \end{aligned}
    \end{equation}
    et toutes deux tendent vers le vecteur \( (0,1,0)\) pour \( y\to\infty\).
\end{example}

\begin{lemma}
    Deux droites d'un plan projectif ont toujours une intersection.
\end{lemma}

\begin{proof}
    Si les deux droites sont des droites affines non parallèles, le résultat est évident. Si elles sont parallèles, alors l'intersection est donnée par le point à l'infini comme indiqué dans l'exemple \ref{ExempMyTmFp}.

    Supposons que \( d\) est la droite à l'infini tandis que \( d'\) est une droite affine. Dans notre représentation usuelle du plan affine, la droite à l'infini \( d\) a contient les vecteurs \( (1,y,0)\) et le point à l'infini \( (0,1,0)\). La droite affine \( d'\) a pour équation paramétriques
    \begin{subequations}
        \begin{numcases}{}
            x=at+c\\
            y=bt+d\\
            z=1.
        \end{numcases}
    \end{subequations}
    Les directions données par la droite \( d'\) sont donc
    \begin{equation}
        \frac{1}{ a^2t^2+b^2t^2+c^2+d^2}\begin{pmatrix}
            at+c    \\ 
            bt+d    \\ 
            1
        \end{pmatrix}
    \end{equation}
    Son point à l'infini est la direction du vecteur \( (a,b,0)\), qui est bien un point de la droite à l'infini (éventuellement son point à l'infini\footnote{D'accord, aller chercher le point à l'infini de la droite à l'infini, c'est chercher loin, mais n'empêche que ça existe.}).
\end{proof}

La plupart du temps nous considérons le plan projectif comme étant le plan affine \( z=1\) de l'espace affine de dimension \( 3\) complété par la droite affine \( x=1,z=0\), elle-même complétée par le point \( (0,1,0)\). Ce n'est évidemment pas la seule manière. Tout plan peut être considéré comme le plan à l'infini et pour une droite projective, tout point peut être considéré comme point à l'infini.

Sur la figure \ref{LabelFigChoixInfinissLabelSubFigChoixInfini0}, le point à l'infini est la direction \( (1,0)\) tandis que la direction \( (1,1)\) n'a rien de spécial. À l'inverse sur la figure \ref{LabelFigChoixInfinissLabelSubFigChoixInfini1}, la direction à l'infini est \( (1,1)\) tandis que la direction \( (1,0)\) est une direction usuelle.

%The result is on figure \ref{LabelFigChoixInfini}.
\newcommand{\CaptionFigChoixInfini}{Deux façons de voir la droite projective. Étant donné que les points de la droite projective doivent être interprétés comme des directions (des classes d'équivallence), en réalité les deux dessins représentent les mêmes ensembles.}
\input{Fig_ChoixInfini.pstricks}

\begin{remark}
    Du point de vue de la topologie, si nous mettons celle de la compactification d'Alexandroff, tous les points de la droite projective sont équivalents.

    Du point de vue de la géométrie différentielle, c'est la même chose. En effet nous pouvons mettre sur la droite projective un système de deux cartes en pensant aux angles. La première sur \( \mathopen] -a , a \mathclose[\) avec par exemple \( a<\pi/4\). La seconde carte serait \( \mathopen] a/2 , \pi \mathclose[\). Dans ce cas la direction \( \theta=0\) semble jouer un rôle spécial, mais il n'en est rien.

    Nous pouvons également considérer les cartes \( \mathopen] \pi/4-a , \pi/4+a \mathclose[\) et \( \mathopen] \pi/4+a/2 , 5\pi/4 \mathclose[\). Dans ces cartes, c'est plutôt le point \( \theta=\pi/4\) qui semble différent (encore qu'il soit bien centré dans une carte).
\end{remark}


%+++++++++++++++++++++++++++++++++++++++++++++++++++++++++++++++++++++++++++++++++++++++++++++++++++++++++++++++++++++++++++
\section{Théorème de Pappus}
%+++++++++++++++++++++++++++++++++++++++++++++++++++++++++++++++++++++++++++++++++++++++++++++++++++++++++++++++++++++++++++

\begin{theorem}     \index{théorème!Pappus!affine}
    Soient deux droites \( d\) et \( d'\) dans un plan affine. Soient \( A,B,C\in d\) et \( A',B',C'\in d'\) tels que \( AB'\parallel BA'\) et \( BC'\parallel B'C\). Alors \( AC'\parallel A'C\).
\end{theorem}

\begin{proof}
    Si \( d\) et \( d'\) ne sont pas parallèles nous considérons \( o\), le point d'intersection. Les relations de parallélisme des hypothèses impliquent qu'il existe \( \lambda_1\) et \( \lambda_2\) tels que
    \begin{subequations}
        \begin{numcases}{}
            A=\lambda_1 B\\
            B'=\lambda_1 A'
        \end{numcases}
    \end{subequations}
    et
    \begin{subequations}
        \begin{numcases}{}
            B'=\lambda_2 C'\\
            C=\lambda_2 B.
        \end{numcases}
    \end{subequations}
    En substituant nous trouvons
    \begin{subequations}
        \begin{numcases}{}
            C=\frac{ \lambda_2 }{ \lambda_1 }A\\
            A'=\frac{ \lambda_2 }{ \lambda_1 }C',
        \end{numcases}
    \end{subequations}
    ce qui implique que \( A'C\parallel AC'\).

    Si les droites \( d\) et \( d'\) sont parallèles, alors nous avons les translations
    \begin{subequations}
        \begin{numcases}{}
            B=A+x\\
            A'=B'+x
        \end{numcases}
    \end{subequations}
    et
    \begin{subequations}
        \begin{numcases}{}
            B=C+y\\
            C'=B'+y,
        \end{numcases}
    \end{subequations}
    ce qui montre que
    \begin{subequations}
        \begin{numcases}{}
            C=A+x-y\\
            A'=C'+x-y,
        \end{numcases}
    \end{subequations}
    et donc que \( A'C\parallel AC'\).
\end{proof}

Le théorème suivant est une version projective.
\begin{theorem}     \index{théorème!Pappus!projectif}
    Soient \( d\) et \( d'\) deux droites projectives d'un plan projectif. Soient \( A,B,C\in d\) et \( A',B',C'\in d'\). Alors les points \( B'C\cap C'B\), \( C'A\cap A'C\) et \( A'B\cap B'A\) sont alignés.
\end{theorem}

\begin{proof}
    Soient \( E=BC'\cap C'B\) et \( E'=C'A\cap A'C\). Ces deux points existent parce que deux droites projectives distinctes ont toujours un unique point d'intersection. Nous allons prendre \( EE'\) comme droite à l'infini et prouver que le point \( A'B\cap B'A\) est dessus. Étant donné que le point d'intersection de \( B'C\) et \( C'B\) est à l'infini nous avons \( B'C\parallel C'B\) (cela est un exemple de la flexibilité de la notion de parallélisme en géométrie projective). De la même façon nous avons \( C'A\parallel A'C\).

    Par le théorème de Pappus affine nous avons alors \( A'B\parallel B'A\) et par conséquent le point d'intersection est sur la droite à l'infini, c'est à dire sur la droite \( EE'\).
\end{proof}

%+++++++++++++++++++++++++++++++++++++++++++++++++++++++++++++++++++++++++++++++++++++++++++++++++++++++++++++++++++++++++++
\section{Homographies}
%+++++++++++++++++++++++++++++++++++++++++++++++++++++++++++++++++++++++++++++++++++++++++++++++++++++++++++++++++++++++++++

%---------------------------------------------------------------------------------------------------------------------------
\subsection{Homographies}
%---------------------------------------------------------------------------------------------------------------------------

\begin{definition}
    Soient \( E\) et \( F\) deux espaces vectoriels avec leurs projections naturelles
    \begin{subequations}
        \begin{align}
            \pi_E\colon E\setminus\{ 0 \}&\to P(E)\\
            \pi_F\colon F\setminus\{ 0 \}&\to P(F).
        \end{align}
    \end{subequations}
    Une application \( g\colon P(E)\to P(F)\) est une \defe{homographie}{homographie} si il existe un isomorphisme d'espaces vectoriels \( \tilde g\colon E\to F\) tel que le diagramme
    \begin{equation}
        \xymatrix{%
        E\setminus\{ 0 \} \ar[r]^{\tilde g}\ar[d]_{\pi_E}        &   F\setminus\{ 0 \}\ar[d]^{\pi_F}\\
           P(E) \ar[r]_{g}   &   P(F)
           }
    \end{equation}
    commute.
\end{definition}

\begin{lemma}
    Si \( \tilde g\colon E\to F\) est linéaire et si \( \ker\tilde g=\{ 0 \}\) alors l'application \( g\) définie par
    \begin{equation}        \label{EqRlGIJW}
        g\big( \pi_E(v) \big)=\pi_F\big( \tilde g(v) \big)
    \end{equation}
    est une homographie.
\end{lemma}

\begin{proof}
    Nous devons simplement vérifier que l'équation \eqref{EqRlGIJW} définit bien une application. Soient \( v,w\in E\) tels que \( \pi_Ev=\pi_Ew\); nous devons montrer que 
    \begin{equation}        \label{EqmoIUkH}
        \pi_F\tilde gv=\pi_F\tilde gw.
    \end{equation}
    L'équation \eqref{EqmoIUkH} sera vérifiée si et seulement si il existe \( \lambda\in\eR\) tel que \( \tilde gv=\lambda\tilde gw\), c'est à dire si et seulement si \( \tilde g(v-\lambda w)=0\). Étant donné que nous supposons que le noyau de \( \tilde g\) est réduit à \( \{ 0 \}\), l'équation \eqref{EqmoIUkH} sera vérifiée si et seulement si \( v=\lambda w\), ce qui signifie exactement \( \pi_E(v)=\pi_E(w)\).
\end{proof}

La proposition suivante donne les premières propriétés des homographies.
\begin{proposition}
    Quelque propriétés des homographies.
    \begin{enumerate}
        \item
            Une homographie est bijective.
        \item
            Si deux espaces projectifs sont homographes, alors ils ont même dimension.
        \item
            L'ensemble des homographies \( P(E)\to P(F)\) est un groupe (pour la composition).
        \item
            Une homographie conserve l'alignement des points.
    \end{enumerate}
\end{proposition}

\begin{proof}
    Nous considérons une homographie \( g\colon P(E)\to P(F)\), et \( \tilde g\) l'isomorphisme d'espaces vectoriels correspondant.
    \begin{enumerate}
        \item
            Pour l'injectivité, si \( g\big( [v] \big)=g\big( [w] \big)\) alors en utilisant la définition d'une homographie, \( \pi_F\tilde gv=\pi_F\tilde gw\), ce qui implique que \( \tilde gv=\lambda\tilde gw\), et donc \( v=\lambda w\), ce qui signifie \( [v]=[w]\).

            Pour la surjectivité, un élément général de \( P(F)\) prend la forme \( \pi_F\tilde gv\) pour un certain \( v\in E\). Nous avons \( g\big( \pi_Ev \big)=\pi_F\tilde gv\). Par conséquent l'élément \( \pi_F\tilde gv\) est bien dans l'image de \( g\).

        \item
            Une homographie \( P(E)\to P(F)\) n'existe que si il existe un isomorphisme \( E\to F\). Les dimensions sont donc automatiquement égales.
        \item
            Il suffit de vérifier que l'application
            \begin{equation}
                \begin{aligned}
                    \varphi\colon P(E)&\to P(E) \\
                    \pi_F\tilde gv&\mapsto \pi_Ev 
                \end{aligned}
            \end{equation}
            est bien définie et donne l'inverse de \( g\).
        \item
            Soient les points \( A,B,C\) alignés dans \( P(E)\); ils correspondent à des directions de \( E\) qui sont données par des vecteurs situés sur la même droite affine. Autrement dit, il existe trois points \( a,b,c\in E\) situés sur la même droite affine tels que \( A,B,C=\pi_E(a,b,c)\). Les images par \( g\) sont données par \( \pi_F\tilde ga\), \( \pi_F\tilde gb\), et \( \pi_F\tilde gc\).

            Étant donné qu'un isomorphisme d'espaces vectoriels conserve l'alignement affin, les points \( \tilde ga\), \( \tilde gb\) et \( \tilde gc\) sont alignés dans \( F\). Cela implique que les projections par \( \pi_F\) sont alignés dans \( P(F)\).
    \end{enumerate}
\end{proof}

%---------------------------------------------------------------------------------------------------------------------------
\subsection{Le groupe projectif}
%---------------------------------------------------------------------------------------------------------------------------

\begin{definition}
    Le groupe des homographies de l'espace \( P(E)\) est le \defe{groupe projectif}{groupe!projectif}\index{projectif!groupe}, noté \( \PGL(E)\).\nomenclature[G]{\( \PGL(E)\)}{groupe projectif}
\end{definition}

Nous avons une surjection naturelle
\begin{equation}        \label{EqpqNEfe}
    \begin{aligned}
         \GL(E)&\to \PGL(E) \\
        \tilde g&\mapsto g 
    \end{aligned}
\end{equation}
qui s'avère être un morphisme de groupes.

\begin{proposition}
    Nous avons l'isomorphisme de groupes
    \begin{equation}
        \frac{ \GL(E) }{\{  \text{homothéties} \}}\simeq \PGL(E).
    \end{equation}
    
\end{proposition}

\begin{proof}
    Nous devons prouver que le noyau de l'application \eqref{EqpqNEfe} est constitué des homothéties. Considérons un automorphisme d'espace vectoriel \( f\colon E\to E\) dont l'homographie associée est l'identité, et prouvons que \( f\) est une homothétie. Nous avons le diagramme commutatif suivant :
    \begin{equation}
        \xymatrix{%
        E\setminus\{ 0 \} \ar[r]^{f}\ar[d]_{\pi_E}        &   E\setminus\{ 0 \}\ar[d]^{\pi_E}\\
           P(E) \ar[r]_{\id}   &   P(E)\ar@{~>}[ul]^{da}.
           }
    \end{equation}
    Pour tout vecteur \( v\in E\) nous avons \( \pi_E(v)=\pi_E\big( f(v) \big)\). Cela implique qu'il existe \( \lambda\in\eR\) tel que \( f(v)=\lambda v\). Tous les vecteurs de \( E\) sont donc des vecteurs propres de \( f\). Cela n'est possible que si toutes les valeurs propres sont identiques, c'est à dire que \( f\) est une homothétie.
\end{proof}
<++>

