% This is part of Mes notes de mathématique
% Copyright (c) 2011-2012
%   Laurent Claessens
% See the file fdl-1.3.txt for copying conditions.

L'espace projectif de \( E\) est l'ensemble des droites vectorielles de \( E\).
\begin{definition}
    Soit \( \eK\) un corps et \( E\) un espace vectoriel de dimension finie sur \( \eK\). Nous définissons sur \( E\setminus\{ 0 \}\) la relation d'équivalence \( u\sim v\) si et seulement si \( u=\lambda v\) pour un certain \( \lambda\in\eK\). Cette relation est la relation de \defe{colinéarité}{colinéarité}. L'ensemble des classes d'équivalence de \( \sim\) est l'\defe{espace projectif}{espace!projectif}\index{projectif!espace} de \( E\) et sera noté \( P(E)\)\nomenclature[A]{\( P(E)\)}{l'espace projectif de $E$}.
\end{definition}

Si \( \dim E=2\), l'ensemble \( P(E)\) est la \defe{droite projective}{droite!projective}\index{projectif!droite}, et si \( \dim E=3\) nous parlons du \defe{plan projectif}{plan!projectif}\index{projectif!plan}.

Étant donné que tous les \( \eK\)-espaces vectoriels de dimensions \( n+1\) sont isomorphes à\( \eK^{n+1}\), nous noterons \( P_n(\eK)\) ou \( P_n\) l'espace projectif \( P(\eK^{n+1})\).

\begin{example}
    Si \( n=1\) et \( \eK=\eR\), l'espace projectif est l'ensemble des droites vectorielles dans le plan usuel. Il y en a une pour chaque point du type \( (x,1)\) avec \( x\in\eR\) et ensuite une horizontale, passant par le point \( (1,0)\). Nous avons donc
    \begin{equation}
        P_1(\eR)=\{ (1,0) \}\cup\{ (x,1)\tq x\in \eR \}.
    \end{equation}
    Le point \( (1,0)\) est dit «point à l'infini».
\end{example}

%+++++++++++++++++++++++++++++++++++++++++++++++++++++++++++++++++++++++++++++++++++++++++++++++++++++++++++++++++++++++++++
\section{Sous espaces projectifs}
%+++++++++++++++++++++++++++++++++++++++++++++++++++++++++++++++++++++++++++++++++++++++++++++++++++++++++++++++++++++++++++

Un \defe{sous espace projectif}{projectif!sous espace} de \( P(E)\) est une partie de la forme \( P(F)\) où \( F\) est un sous espace vectoriel de \( E\).

\begin{proposition}     \label{PropuqpWVx}
    Si \( F\) et \( G\) sont des sous espaces vectoriels de \( E\), alors
    \begin{equation}
        P(F)\cap P(G)=P(F\cap G)
    \end{equation}
    et nous avons
    \begin{equation}        \label{EqNAdWfN}
        \dim P(F)+\dim P(G)=\dim P(F+G)+\dim P(F\cap G).
    \end{equation}
\end{proposition}

\begin{proof}
    Nous avons 
    \begin{equation}
        P(F)=\{ [v]\tq v\in F \}
    \end{equation}
    où les crochets signifient la classe par rapport à la relation de colinéarité. Nous avons alors
    \begin{equation}
        P(F)\cap P(G)=\{ [v]\tq v\in F\cap G \}=P(F\cap G).
    \end{equation}
    Cela prouve le premier point.

    En ce qui concerne l'équation \eqref{EqNAdWfN}, en considérant \( \dim P(E)=\dim E-1\) nous devons prouver l'égalité
    \begin{equation}
        \dim F+\dim G=\dim (F+G)+\dim(F\cap G)
    \end{equation}
    concernant les dimensions des espaces vectoriels usuelles. Si nous considérons une base de \( E\) telle que \( B_1=\{ e_1,\ldots, e_{k_1} \}\) est une base de \( F\cap G\), \( B_2=\{ e_{k_1+1},\ldots, e_{k_2} \}\) complète \( B_1\) en une base de \( F\) et \( B_3=\{ e_{k_2+1},\ldots, e_n \}\) complète \( B_1\cup B_2\) en une base de \( G\).

    Nous avons alors
    \begin{subequations}
        \begin{align}
            \dim F+\dim G&=2\Card(B_1)+\Card(B_2)+\Card(b_3)\\
            \dim(F+G)&=\Card(B_1)+\Card(b_2)+\Card(B_3)\\
            \dim(F\cap G)&=\Card(B_1).
        \end{align}
    \end{subequations}
    De là la relation \eqref{EqNAdWfN} se déduit immédiatement.    
\end{proof}

\begin{theorem}[incidence]\index{théorème!incidence}
    Soient \( F\) et \( F\) deux sous espaces vectoriels de \( E\) tels que 
    \begin{equation}
        \dim P(F)+\dim P(G)\geq \dim P(E).
    \end{equation}
    Alors \( P(F)\cap P(G)\neq \emptyset\).
\end{theorem}

\begin{proof}
    En utilisant les hypothèses et la proposition \ref{PropuqpWVx} nous avons
    \begin{equation}
        \dim P(E)+\dim P(G)=\dim P(F+G)+\dim P(F\cap G)\geq \dim P(E).
    \end{equation}
    En passant aux espaces vectoriels correspondants,
    \begin{equation}
        \dim(F+G)+\dim(F\cap G)\geq \dim(E)+1.
    \end{equation}
    Mais nous avons aussi \( \dim(F+G)\leq \dim(E)\) et par conséquent \( \dim(F\cap G)\geq 1\). Au final, \( \dim P(F\cap G)\geq 0\). Cela prouve que \( P(F\cap G)\) contient au moins un élément (nous rappelons que lorsqu'un espace projectif contient un seul élément, sa dimension est zéro).
\end{proof}

\begin{example}
    Soient les plans \( \Pi_1\equiv x=0\) et \( \Pi_2\equiv y=0\). Nous avons
    \begin{subequations}
        \begin{align}
            P(\Pi_1)&=\{ [0,y,1] \}\cup\{ [0,1,0] \}\\
            P(\Pi_2)&=\{ [x,0,1] \}\cup\{ [1,0,0] \}
        \end{align}
    \end{subequations}
    où le crochet signifie la classe pour la colinéarité. Ces deux droites projectives ont comme point d'intersection le point \( [0,0,1]\).
\end{example}

\begin{definition}
    Un \defe{hyperplan projectif}{projectif!hyperplan} est un sous espace projectif de \( P(E)\) de la forme \( P(V)\) où \( V\) est un hyperplan de \( E\).
\end{definition}

\begin{proposition}
    Soit \( H=P(V)\) un hyperplan projectif de \( P(E)\) et soit \( m\) hors de \( H\). Alors toute droite projective passant par \( m\) coupe \( H\) en un et un seul point.
\end{proposition}

\begin{proof}
    Si \( \dim E=n\) nous avons \( \dim V=n-1\). Soit \( d=P(D)\) une droite projective passant par \( m\), c'est à dire que \( D\) est de dimension \( 2\) dans \( E\). Si \( D\subset V\) alors \( m\in P(D)\subset P(V)\); or nous avons demandé que \( m\) soit hors de \( P(V)\). Par conséquent \( D\) n'est pas inclus à \( V\) et en particulier \( \dim(D+V)=\dim(E)\).

    Nous recopions la formule \eqref{EqNAdWfN} pour notre cas :
    \begin{equation}
        \underbrace{\dim d}_{=1}+\underbrace{\dim H}_{=n-2}=\underbrace{\dim P(D+V)}_{=n-1}+\dim P(D\cap V).
    \end{equation}
    Nous avons donc \( \dim P(D\cap V)=0\), ce qui signifie que l'ensemble \( P(D\cap V)=P(D)\cap P(V)=d\cap H\) contient un et un seul point.
\end{proof}

%+++++++++++++++++++++++++++++++++++++++++++++++++++++++++++++++++++++++++++++++++++++++++++++++++++++++++++++++++++++++++++
\section{Espace projectifs comme «complétés» d'espaces affines}
%+++++++++++++++++++++++++++++++++++++++++++++++++++++++++++++++++++++++++++++++++++++++++++++++++++++++++++++++++++++++++++

Soit \( E\) un espace vectoriel de dimension \( 2\) et \( P(E)\) la droite projective correspondante, et soit \( \{ e_1,e_2 \}\) une base de \( E\). Nous considérons la droite affine \( d\equiv y=1\). Nous avons la bijection
\begin{equation}        \label{EqvrfDLz}
    \begin{aligned}
        \phi\colon d\cup\{ \infty \}&\to P(E) \\
        (x,1)&\mapsto \text{la droite vectorielle passant par \( (x,1)\)} \\
        \infty&\mapsto \text{la droite vectorielle passant par \( (1,0)\)}.
    \end{aligned}
\end{equation}

\begin{lemma}
    Si nous munissons l'ensemble \( d\cup\{ \infty \}\) de la topologie compactifiée d'Alexandroff, la bijection \eqref{EqvrfDLz} est un homéomorphisme.
\end{lemma}

Soient maintenant les plans affins dans l'espace vectoriel \( E\) de dimension \( 3\)
\begin{subequations}
    \begin{align}
        \Pi_1\equiv z&=0\\
        \Pi_2\equiv z&=1.
    \end{align}
\end{subequations}
Une droite (vectorielle) de \( E\) coupe \( \Pi_2\) en un et un seul point, sauf si elle est contenue dans \( \Pi_1\). Nous avons donc une bijection
\begin{equation}
    \begin{aligned}
        \phi\colon P(E)&\to \Pi_2\cup P(\Pi_1) \\
        d&\mapsto \begin{cases}
            \Pi_2\cap d    &   \text{si cette intersection est non vide}\\
            d    &    \text{sinon.}
        \end{cases}
    \end{aligned}
\end{equation}
La droite projective \( P(\Pi_1)\) est la droite à l'infini du plan projectif \( P(E)\). Nous voyons que le plan projectif \( P(E)\) peut être vu comme un plan affine \( (\Pi_2)\) «complété»  par une droite affine \( P(\Pi_1)\). Cette dernière droite est elle-même une droite affine complétée par un point à l'infini.

Nous pouvons généraliser cette démarche en considérant un espace affine \( \affE\) de direction \( E\) sur le corps \( \eK\). Nous construisons \( F=E\times \eK\) et nous considérons un repère affine sur \( F\) tel que \( E\equiv x_{n+1}=0\). Nous pouvons donc identifier \( \affE\) à l'hyperplan affine d'équation \( x_{n+1}=1\) dans \( F\).

Une droite vectorielle de \( F\) non contenue dans \( E\) coupe \( \affE\) en un unique point; nous avons donc une bijection
\begin{equation}
    \affE\cup P(E)\to P(F).
\end{equation}
Dans ce cadre, \( P(E)\) est l'hyperplan à l'infini et nous disons que \( P(E)\) est la \defe{complétion projective}{complétion!projective}\index{projectif!complétion} de \( \affE\).

\begin{definition}
    Soit \( E\) un espace vectoriel de dimension \( 3\). Nous disons que \( d\subset P(E)\) est une \defe{droite projective}{projectif!droite} de \( P(E)\) si \( d=P(D)\) pour une plan vectoriel \( D\subset E\).
\end{definition}

\begin{example}
    Nous considérons les plans affins
    \begin{subequations}
        \begin{align}
            \Pi_1&\equiv z=0\\
            \Pi_2&\equiv z=1
        \end{align}
    \end{subequations}
    et nous avons la bijection
    \begin{equation}
        P(E)=\Pi_2\cup P(\Pi_1).
    \end{equation}
    Un plan affine \( D\) a deux possibilités : soit il coupe \( \Pi_2\) en une droite, soit il est égal à \( \Pi_1\). Si \( D\cap\Pi_2=d\) (\( d\) est une droite affine), alors nous avons
    \begin{equation}
        P(D)=d\cup\{ \infty_D \},
    \end{equation}
    ce qui justifie la terminologie comme quoi \( P(D)\) est une droite dans \( P(E)\).
\end{example}


