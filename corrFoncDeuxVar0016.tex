% This is part of the Exercices et corrigés de mathématique générale.
% Copyright (C) 2009-2010
%   Laurent Claessens
% See the file fdl-1.3.txt for copying conditions.


\begin{corrige}{FoncDeuxVar0016}

	Le domaine de définition de $f$ est manifestement l'ensemble de $(x,y)$ tels que $x\neq -y$. L'équation de la courbe de niveau de hauteur $C$ est
	\begin{equation}
		\frac{ x^2+y^2 }{ x+y }=C,
	\end{equation}
	c'est à dire
	\begin{equation}
		x^2-Cx+y^2-Cy=0.
	\end{equation}
	Afin de mettre cela sous une forme qui ressemble plus à un cercle, nous reformons les carrés parfait à partir de $x^2+Cx$ et $y^2+Cy$. Pour cela, nous cherchons $a$ et $b$ tels que
	\begin{equation}	\label{EqCourbeCxSeize}
		x^2-Cx=(x+a)^2+b.
	\end{equation}
	Égalisant les termes de degrés égaux en $x$, nous trouvons $a=-C/2$ et $b=-C^2/4$. L'équation de la courbe de niveau \eqref{EqCourbeCxSeize} se met donc sous la forme
	\begin{equation}
		\big( x-\frac{ C }{ 2 } \big)^2-\frac{ C^2 }{ 4 }+\big( y-\frac{ C }{ 2 } \big)^2-\frac{ C^2 }{ 4 }=0
	\end{equation}
	Cette dernière équation se récrit avantageusement sous la forme
	\begin{equation}
		\left( x-\frac{ C }{2} \right)^2+\left( y-\frac{ C }{2} \right)^2=\frac{ C^2 }{2},
	\end{equation}
	ce qui est le cercle de rayon $\frac{ | C | }{  \sqrt{2} }$ et de centre $(\frac{ C }{2},\frac{ C }{2})$. Notez la valeur absolue dans le rayon.

    %TODO : refaire les dessins
	%\newcommand{\CaptionFigExoHuitSept}{Quelques courbes de niveau pour l'exercice \ref{exoFoncDeuxVar0016}.}
	%\input{Fig_ExoHuitSept.pstricks}
	%Quelques courbes de niveau sont tracées sur la figure \ref{LabelFigExoHuitSept}. Notez qu'elles passent toutes par $(0,0)$. En effet, le centre $(C/2,C/2)$ est à une distance
	\begin{equation}
		\sqrt{ \left( \frac{ C }{ 2 } \right)^2+\left( \frac{ C }{ 2 } \right)^2 }=\frac{ | C | }{ \sqrt{2} }
	\end{equation}
	de l'origine. Il est donc hors de question que la limite existe quand $(x,y)\to(0,0)$.

\end{corrige}
