% This is part of Exercices et corrigés de CdI-1
% Copyright (c) 2011-2012
%   Laurent Claessens
% See the file fdl-1.3.txt for copying conditions.

\begin{exercice}\label{exoTP20090002}

Soit $T : A \subset {\eR}^n \to A$ une application. Supposons qu'il existe un entier $p > 1$ tel que $T^p \pardef \underbrace{T \circ T \circ \ldots \circ T}_{\text{$p$ fois}}$ soit une application contractante.

\begin{enumerate}
\item
Si $A$ est fermé, prouver à l'aide du théorème de Banach que $T$ possède un unique point fixe.
\item
Montrer qu'il existe un ensemble $A\subset \eR^2$ et une application $T : A \to A$ qui n'est pas une contraction, mais telle que $T^2$ est une contraction.
\end{enumerate}

\begin{remark}
    Cet énoncé est utilisé dans la démonstration du théorème de Cauchy-Lipschitz \ref{ThokUUlgU}.
\end{remark}

\corrref{TP20090002}
\end{exercice}
