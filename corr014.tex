\begin{corrige}{014}
We parametrize the surface by $x$ and $\theta$:
\[ 
  f(x,\theta)=(x,f(x)\cos\theta,f(x)\sin\theta).
\]
In this parametrization, the basis vectors are given by
\begin{subequations}
\begin{align}
  e_x&=(1,f'(x)\cos\theta,f'(x)\sin\theta)\\
	e_{\theta}&=(0,-f(x)\sin\theta,f(x)\cos(\theta)),
\end{align}
\end{subequations}
and the matrix $g$ is given by
\[ 
  g=
\begin{pmatrix}
1+f'(x)^2&0\\0&f(x)^2
\end{pmatrix}
\]
Christoffel symbols are obtained by formula
\[ 
  \Gamma_{ij}^{k}=\frac{ 1 }{2}g^{lk}(\partial_ig_{jl}+\partial_jg_{li}-\partial_lg_{ij})
\]
where the $g^{lk}$ are elements of the \emph{inverse} matrix of $g$. The non zero symbols are
\begin{equation}
\Gamma_{\theta\theta}^{x}=-\frac{ ff' }{ 1+{f'}^2 },\quad\Gamma_{\theta x}^{\theta}=\Gamma_{x\theta}^{\theta}=\frac{ f' }{ f },\quad\Gamma_{xx}^{x}=\frac{ f'f'' }{ 1+{f'}^2 }.
\end{equation}
Covariant derivatives of basis vectors are
\begin{align}
\nabla_xe_x&=\Gamma_{xx}^{x}e_x&\nabla_{\theta}e_x&=\Gamma_{\theta x}^{\theta}e_{\theta}\\
\nabla_xe_{\theta}&=\Gamma_{x\theta}^{\theta}e_{\theta}&\nabla_{\theta}e_{\theta}&=\Gamma_{\theta\theta}^{x}e_x.
\end{align}
As far as Riemann tensor and related curvature issues are concerned, we do not need to compute $\nabla_x\nabla_xe_i$ and $\nabla_{\theta}\nabla_{\theta}e_i$. Computations are as follows:
\[ 
\begin{split}
\nabla_x\nabla_{\theta}e_x&=\partial_x(\Gamma_{\theta x}^{\theta})e_{\theta}+\Gamma_{x\theta}^{\theta}\Gamma_{x\theta}^{\theta}e_{\theta}\\
	&=\frac{ f'' }{ f }e_{\theta}.
\end{split}  
\]
Other results ---up to personal faults--- are
\begin{equation}
\begin{aligned}
\nabla_x\nabla_{\theta}e_{\theta}&=\Big( \frac{ 3f{f'}^2f'' }{ (1+{f'}^2)^2 }=\frac{ {f'}^2+ff'' }{ 1+{f'}^2 } \Big)e_x\\
\nabla_{\theta}\nabla_xe_x&=\Big( \frac{ f'f'' }{ 1+{f'}^2 } \Big)\frac{ f' }{ f }e_{\theta}\\
\nabla_{\theta}\nabla_xe_{\theta}&=-\frac{ f' }{ f }\frac{ ff' }{ 1+{f'}^2 }e_x.
\end{aligned}
\end{equation}

\paragraph{Isometry criterion} All this part is ``without proof''.

We consider the following tensor:
\[ 
  R(X,Y)=\nabla_X\nabla_Y-\nabla_Y\nabla_X-\nabla_{[X,Y]}
\]
and the \defe{Ricci curvature}{} pointwise defined by
\[ 
  r(Y,Z)=\tr\big[ X\mapsto R(X,Y)Z \big].
\]
It is a trace in the sense of the trace of a matrix of a linear operator in a vector space: $R(.,Y)Z\colon T_x\Sigma^f\to T_x\Sigma^f$. For a general $A\colon V\to V$, it is defined by
\[ 
  \tr A=\sum_i\scald{ e_i }{ Ae_i }
\]
where $\{ e_i \}$ is a basis of $V$. We define the \defe{scalar curvature}{} by
\[ 
  \rho=\sum_{a=1}^2r(e_a,e_a).
\]
This defines a real valued function $\rho^f\colon \Sigma^f\to \eR$  for the surface $\Sigma^f$ and a corresponding one $\rho^g\colon \Sigma^g\to \eR$ for $\Sigma^g$. From symmetry of the problem, these functions do not depend on the angular part of the parametrization of $\Sigma^f$ and $\Sigma^g$. So one can only consider the restrictions
\begin{subequations}
\begin{align}
\rho^f&\colon ]a,b[\to \eR,\\
\rho^g&\colon ]c,d[\to \eR.
\end{align}
\end{subequations}
The criterion for $\Sigma^f$ and $\Sigma^g$ to be isometric is the existence of a diffeomorphism $\varphi\colon ]a,b[\to ]c,d[$ such that 
\[ 
  \rho^f=\rho^g\circ\varphi.
\]
Explicit proof of existence of such a $\varphi$ can be very hard.


\end{corrige}
