La section \ref{SecGraphesFonc} donne des techniques de dessin de fonction à plusieurs variables. Le but de cette section sera d'être capable d'esquisser les graphes des fonctions, et de reconnaître les équations qui décrivent les courbes et surfaces les plus courantes.

Les limites de suites et de fonctions sont introduites dans la section \ref{SecLimitesRn}. Les notions sont très similaires à ceux à propos des limites dans les espaces vectoriels normés; il est conseillé de relire la partie sur les limites du chapitre \ref{ChapEspVectNorm} après avoir lu ce chapitre.

Dédiée aux exercices, la section \ref{SecCalculLimite} introduira diverses techniques de calcul de limites. Une grande partie de la difficulté du chapitre est d'être capable d'utiliser ces techniques en comprenant le lien avec les définitions et théorèmes généraux quant aux limites. Nous aborderons les techniques suivantes :
\begin{description}
	\item[La règle de l'étau] Il s'agira de trouver la limite d'une fonction en montrant qu'elle est comprise entre deux fonctions dont on sait la limite.
	\item[La méthode des chemins] Cette méthode nous permettra de vérifier très vite lorsqu'une limite \emph{n'existe pas}. Il faut être capable de calculer la limite d'une fonction le long des chemins classiques : les droites, les paraboles, ou en coordonnées polaires.
	\item[Coordonnées polaires] Très utiles pour calculer des limites de fractions de polynômes, cette technique est intimement liée à la définition de la limite en termes de boules.
	\item[Développement asymptotique] Nous pouvons parfois transformer des fonctions compliquée (surtout trigonométriques) en des polynômes en utilisant un développement asymptotique (Taylor). La condition est être capable de prouver que l'erreur commise en remplaçant la fonction par le polynôme ne change pas la valeur de la limite. Bien qu'il existe une théorie qui permette de faire cela en plusieurs variables, nous allons nous contenter d'utiliser des développements en une seule variable. 

\end{description}
Seules les méthodes des coordonnées polaires et des chemins seront réellement nouvelles et feront l'objet d'une partie théorique importante. La méthode des développements asymptotiques sera seulement utilisée à une seule variable (bien qu'une théorie existe à plusieurs variables) et sera plus considéré comme une «astuce de calcul» que comme une théorie à part entière. La règle de l'étau, quant à elle, sera seulement un perfectionnement à plusieurs variables de ce que l'on connait déjà à une seule variable.

La section \ref{SecFonctionsSurCompacts} parle de fonctions sur des compacts. Elle complète certains résultats développés dans la section \ref{Sect_fonctions}. Nous donnons en particulier le résultat comme quoi une fonction continue sur un compact est bornée. Ce résultat sera utilisé de nombreuses fois dans les autres cours. Après avoir lu ce chapitre, vous devriez immédiatement penser à «borné» lorsque vous lisez les mots «continue» et «compact» dans la même phrase.

L'uniforme continuité est le sujet de la section \ref{SecUnifContinue}. Il s'agit d'une classe de fonctions continues qui ont la propriété que pour chaque $\varepsilon$, il existe un $\delta$ \emph{ne dépendant pas de $a$} tel que $\| x-a \|<\delta$ implique $\| f(x)-f(a) \|\leq \varepsilon$.

En ce qui concerne les exercices, les limites à calculer feront souvent appel aux développements expliqués dans l'appendice \ref{AppSecTaylorR}. La théorie contenue dans cet appendice ne fait pas partie de la matière, mais il faut être capable d'utiliser le théorème \ref{ThoTaylor} lors du calcul pratique de limites. Surtout les différentes variations données dans l'exemple \ref{ExempleUtlDev}.

Tout le chapitre fait partie de la matière sauf
\begin{enumerate}
	\item
		la démonstration du théorème \ref{ThoBolzanoWeierstrassRn};
	\item
		la démonstration du théorème \ref{ThoWeirstrassRn};
	\item
		la démonstration du théorème \ref{ThoHeineContinueCompact}.
\end{enumerate}
