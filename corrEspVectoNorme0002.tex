\begin{corrige}{EspVectoNorme0002}

	\begin{enumerate}
		\item

			Prenons $(x,y)\in A$, et tâchons de trouver une boule autour de $(x,y)$ qui soit contenue dans $A$. Par définition, $x>0$ et $y>0$. Donc si nous prenons $r=\min\{ x,y \}/2$, la boule $B\big( (x,y),r \big)$ est encore contenue dans $A$.

			Notez que l'ensemble des boules du type $\mO_{(x,y)}=B\big( (x,y),\frac{ \min\{ x,y \} }{ 2 } \big)$ est un recouvrement de $A$ par des ouverts.

		\item		
			L'adhérence est constituée des points qui «touchent» presque l'ensemble. Intuitivement, nous devinons que l'adhérence de $A$ va être l'ensemble des points $(x,y)$ tels que $x\geq 0$ et $y\geq 0$. D'abord, un point $(a,b)$ avec $a<0$ n'est pas dans $\bar A$ parce qu'il existe une boule autour de $(a,b)$ telle que $x<0$ pour tout $(x,y)$ dans la boule (même chose pour les points avec $y<0$).

			Ensuite, prouvons que les points de la forme $(0,y)$ et $(x,0)$ avec $x,y\geq 0$ sont dans $\bar A$. Pour cela, rien de tel qu'une suite. La suite $(\frac{1}{ n },y)$ avec $y\geq 0$ est contenue dans $A$, et sa limite est clairement le point $(0,y)$. Nous en concluons que $(0,y)$ est un point de $\bar A$.

			De la même façon, la suite $(x,\frac{1}{ n })$ montre que le point $(x,0)$ est dans $\bar A$. Donc
			\begin{equation}
				\bar A=A\cup\{ (x,0)\tqs x\geq 0 \}\cup\{ (0,y)\tqs y\geq 0\}.
			\end{equation}
			En particulier, le point $(0,0)$ est dans $\bar A$.
			
		\item
			En ce qui concerne la frontière, nous utilisons la caractérisation $\partial A=\bar A\setminus\Int(A)$. Étant donné que $A$ est ouvert, $\Int(A)=A$. Les points qui sont dans $\bar A$ et pas dans $A$ sont les points avec $x=0$ ou $y=0$. Donc
			\begin{equation}
				\partial A=\{ (x,0)\tqs x\geq 0 \}\cup\{ (0,y)\tqs y\geq 0\}.
			\end{equation}

	\end{enumerate}

	L'adhérence peut aussi être trouvée en utilisant la proposition \ref{PropovlAxBbarAbraB}. Nous avons $A=\mathopen] 0 , \infty \mathclose[\times\mathopen] 0 , \infty \mathclose[$, et par conséquent, la fermeture de $A$ est la produit des fermetures :
	\begin{equation}
		\bar A=\mathopen[ 0 , \infty [\times\mathopen[ 0 , \infty [.
	\end{equation}
	Nous insistons sur le fait que la fermeture de $\mathopen] 0 , \infty \mathclose[$ n'est pas $\mathopen[ 0 , \infty \mathclose]$. Ce dernier ensemble n'est pas une partie de $\eR$.
\end{corrige}
