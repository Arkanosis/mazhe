% This is part of (almost) Everything I know in mathematics
% Copyright (c) 2010-2014
%   Laurent Claessens
% See the file fdl-1.3.txt for copying conditions.

\begin{corrige}{mazhe-0017}

\begin{enumerate}
\item L'ensemble de définition de $f$ est $\eR$, les limites aux bords du domaine sont 
  \begin{equation*}
    \lim_{x\to -\infty} f(x) = -2 , \qquad \lim_{x\to +\infty} f(x) = 1.
  \end{equation*}
La d\'eriv\'ee de $f$ est 
\begin{equation*}
  f'(x) = \frac{e^x(e^x+1)-e^x(e^x-2)}{(e^x+1)^2} =  \frac{3e^x}{(e^x+1)^2},
\end{equation*}
qui est une fonction strictement positive. La fonction $f$ est continue et $f'$ est strictement positive, $f$ est donc strictement croissante entre les valeurs $-2$ et $1$. 
\item Nous pouvons calculer explicitement 
  \begin{align*} 
 &    -\frac{1}{3}\left(f^2(x) + f(x)-2\right) = -\frac{1}{3} \frac{(e^x-2)^2 + (e^x-2)(e^x+1)-2(e^x+1)^2}{(e^x+1)^2} \\
&= -\frac{1}{3}\frac{-9e^x}{(e^x+1)^2} = \frac{3e^x}{(e^x+1)^2}.
  \end{align*}
Cette expression est identique \`a l'expression de $f'$ trouvée au point pr\'ec\'edent. 
\item La fonction $f$ est d\'efinie sur $\eR$, continue et  strictement croissante entre les valeurs $-2$ et $1$. Elle réalise donc une bijection de $\eR$ vers $]-2,1[$. 
\item L'intervalle de définition de $g$ est $]-2,1[$. 
\item La formule qui donne la dérivée de la fonction réciproque est 
  \begin{equation*}
    g'(y) = \frac{1}{f'(g(y))}.
  \end{equation*}
\item  Nous avons 
  \begin{equation*}
    g'(y) = \frac{1}{f'(g(y))} =\frac{-3}{\left(f^2(g(y)) + f(g(y))-2\right)} = \frac{-3}{y^2 +y-2}.
  \end{equation*}
\item Pour trouver l'expression explicite de $g$ nous trouvons d'abord l'ensemble des primitives de $g'$
 \begin{equation*}
   \int g'(y)\,dy =\int \frac{-3}{y^2 +y-2}\,dy = \int \frac{-3}{(y+2)(y-1)}\,dy
 \end{equation*}
On a 
\begin{equation*}
  \frac{-3}{(y+2)(y-1)} = \frac{1}{(y+2)} - \frac{1}{(y-1)},
\end{equation*}
et donc 
\begin{equation*}
 \int g'(y)\,dy = \int\frac{1}{(y+2)} \,dy - \int \frac{1}{(y-1)}\,dy = \ln(|y+2|) - \ln(|y-1|) +C. 
\end{equation*}
Il faut trouver maintenant la valeur de la constante $C$ qui correspond \`a $g$ (car $g$ est une primitive particulière de $g'$). Il suffit de calculer la valeur de $f$ \`a un point, par exemple $f(0) = -1/2$. Ensuite forcement $g(-1/2) = 0$, donc 
\begin{equation*}
  \ln(|-1/2+2|) - \ln(|-1/2-1|) +C =0
\end{equation*}
et on trouve $C =\ln(1) =0 $. La fonction $g$ est $g(y) =\ln(|y+2|) - \ln(|y-1|) $.
\end{enumerate}
\end{corrige}
