% This is part of (almost) Everything I know in mathematics
% Copyright (C) 2009-2010
%   Laurent Claessens
% See the file fdl-1.3.txt for copying conditions.

Ce chapitre est un foure-tout de choses que l'on peut faire avec Sage.

\begin{example}     \label{ExBCRXooDVUdcf}
	Calculer la limite 
			\begin{equation}
				\lim_{x\to\infty}\frac{ \sin(x)\cos(x) }{ x }
			\end{equation}


            \begin{verbatim}
			var('x')
			f(x)=sin(x)*cos(x)/x
			limit(f(x),x=oo)
            \end{verbatim}
	La première ligne déclare que la lettre \texttt{x} désignera une variable. Pour la troisième ligne, notez que l'infini est écrit par deux petits \og o\fg.
\end{example}

\begin{example}     \label{ExCWDRooKxnjGL}
    Quelque limites et graphes avec Sage.

    \begin{enumerate}

		\item
			$\lim_{x\to 0} \frac{ \sin(\alpha x) }{ \sin(\beta x) }$.

			Pour effectuer cet exercice avec Sage, il faut taper les lignes suivantes~:

            
\begin{verbatim}
sage: var('x,a,b')
(x, a, b)
sage: f(x)=sin(a*x)/sin(b*x)
sage: limit( f(x),x=0  )
a/b
\end{verbatim}

			Noter qu'il faut déclarer les variables \texttt{x}, \texttt{a} et \texttt{b}.

		\item
			$\lim_{x\to \pm\infty} \frac{ \sqrt{x^2+1}-x }{ x-2 }$

            \begin{verbatim}
sage: f(x)=(sqrt(x**2+1))/(x-2)
sage: limit(f(x),x=oo)
1
sage: limit(f(x),x=-oo)
-1
            \end{verbatim}

			Noter la commande pour la racine carré~: \texttt{sqrt}. Étant donné que cette fonction diverge en $x=2$, si nous voulons la tracer, il faut procéder en deux fois :

            \begin{verbatim}
sage: plot(f,(-100,1.9))
Launched png viewer for Graphics object consisting of 1 graphics primitive
sage: plot(f,(2.1,100))
Launched png viewer for Graphics object consisting of 1 graphics primitive
            \end{verbatim}
			La première ligne trace de $-100$ à $1.9$ et la seconde de $2.1$ à $100$. Ces graphiques vous permettent déjà de voir les limites. Attention : ils ne sont pas des \emph{preuves} ! Mais ils sont de sérieux indices qui peuvent vous inspirer dans vos calculs.

	\end{enumerate}
\end{example}

\begin{example} \label{exJMGTooZcZYNy}
    

Calculer les dérivées partielles $\partial_xf$, $\partial_yf$, $\partial^2_xf$, $\partial^2_{xy}f$, $\partial^2_{yx}f$ et $\partial^2_yf$ des fonctions suivantes.
\begin{multicols}{2}
\begin{enumerate}
\item
$2x^3+3x^2y-2y^2$
\item
$\ln(xy^2)$
\item 
$\tan(x/y)$
\item 
$\frac{ xy^2 }{ x+y }$ 

\end{enumerate}	
\end{multicols}


Le script Sage suivant (\verb+exoDV002.sage+) résout l'exercice : 


\VerbatimInput[tabsize=3]{exoDV002.sage}

La sortie est :

\VerbatimInput[tabsize=3]{exoDV002.txt}

\end{example}

\begin{example}\label{exKGDIooVefujD}

\end{example}
<++>

