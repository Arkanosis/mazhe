% This is part of the Exercices et corrigés de mathématique générale.
% Copyright (C) 2009-2010
%   Laurent Claessens
% See the file fdl-1.3.txt for copying conditions.


\begin{corrige}{INGE1121La0001}

	Étant donné que l'espace $V$ est donné avec $3$ paramètre, il peut être au plus de dimension $3$. Nous trouvons $3$ vecteurs de cet espace en posant successivement $\alpha=1$, $\beta=1$ et $\gamma=1$ en laissant à chaque fois les deux autres paramètres à zéro. Les trois vecteurs trouvés sont:
	\begin{equation}
		\begin{aligned}[]
			v_1&=(1,1,2,0)\\
			v_2&=(0,1,1,1)\\
			v_3&=(0,0,-2,2)
		\end{aligned}
	\end{equation}
	Affin de vérifier que ces trois vecteurs sont linéairement indépendants, nous les mettons dans une matrice et nous vérifions que le rang de la matrice est bien $3$:
	\begin{equation}
		A=\begin{pmatrix}
			1	&	0	&	0	\\
			1	&	1	&	0	\\
			2	&	1	&	-2	\\
			0	&	1	&	2
		\end{pmatrix}.
	\end{equation}
	Le rang de cette matrice est bien $3$ parce que par exemple le déterminant de la sous matrice
	\begin{equation}
		\begin{pmatrix}
			1	&	0	&	0	\\
			1	&	1	&	0	\\
			2	&	1	&	-2
		\end{pmatrix}
	\end{equation}
	est non nul.

	Les vecteurs $v_i$ trouvé forment donc une base de $V$. Nous trouvons une base orthogonale en appliquant la méthode de Gram-Schmidt. Le résultat est
	\begin{equation}
		w_1=(1,1,2,0)\\
		w_2=(-1/2,1/2,0,1)\\
		w_1=(2/3,2/3,-2/3,2),
	\end{equation}
	dont les normes sont respectivement $\sqrt{6}$, $\sqrt{3/2}$ et $4/\sqrt{3}$. Une base orthonormée est donnée par les vecteurs $w_i$ divisés par leurs normes.

\end{corrige}
