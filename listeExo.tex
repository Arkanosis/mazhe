% This is part of Exercices et corrigés de CdI-1
% Copyright (c) 2011
%   Laurent Claessens
% See the file fdl-1.3.txt for copying conditions.

%+++++++++++++++++++++++++++++++++++++++++++++++++++++
\section{Supremum, maximum}

\Exo{0001}
\Exo{0002}
\Exo{0003}
\Exo{00035}


\Exo{0004}
\Exo{0005}


%++++++++++++++++++++++++++++++++++++++++++++++++++++
\section{Suites}

\Exo{0006}
\Exo{0007}
\Exo{0008}
\Exo{0009}
\Exo{0010}
\Exo{0011}
\Exo{0012}
\Exo{0014}

\Exo{0015}
\Exo{0018}
\Exo{0019}
\Exo{0020}


\subsection{Suites définies par récurrence}

\Exo{0021}
\Exo{0022}


\section{Calcul de limites}
\label{SecCalcLimFHtQNu}

\Exo{0023}
\Exo{0025}
\Exo{0026}
\Exo{0027}

\subsection{Limites à deux variables}

\Exo{0028}
\Exo{0029}
\Exo{0030}

\Exo{LimSup0001}

%+++++++++++++++++++++++++++++++++++++++++++++++++++++++++++++++++++++++++++++++++++++++++++++++++++++++++++++++++++++++++++
					\section{Limite et continuité}
%+++++++++++++++++++++++++++++++++++++++++++++++++++++++++++++++++++++++++++++++++++++++++++++++++++++++++++++++++++++++++++

\Exo{continueSupl1}
\Exo{continueSupl2}


\Exo{0031}
\Exo{0032}
\Exo{0033}
\Exo{0034}
\Exo{0035}


\Exo{0036}
\Exo{reserve0001}
\Exo{0037}
\Exo{0038}
\Exo{0039}
\Exo{0040}

\Exo{continueSup0003}
\Exo{continueSup0004}
\Exo{continueSup0005}


%+++++++++++++++++++++++++++++++++++++++++++++++++++++++++++++++++++++++++++++++++++++++++++++++++++++++++++++++++++++++++++
					\section{Dérivées partielles et différentiabilité}
%+++++++++++++++++++++++++++++++++++++++++++++++++++++++++++++++++++++++++++++++++++++++++++++++++++++++++++++++++++++++++++


\Exo{0041}
\Exo{0042}
\Exo{0043}
\Exo{0044}
\Exo{0045}
\Exo{0046}
\Exo{0047}
\Exo{0048}
\Exo{0049}
\Exo{0050}
\Exo{0051}
\Exo{0052}
\Exo{0053}
\Exo{0054}
\Exo{0055}
\Exo{0056}
\Exo{0057}
\Exo{0058}
\Exo{0059}
\Exo{0060}
\Exo{0061}

%+++++++++++++++++++++++++++++++++++++++++++++++++++++++++++++++++++++++++++++++++++++++++++++++++++++++++++++++++++++++++++
					\section{Séries et séries de puissances}
%+++++++++++++++++++++++++++++++++++++++++++++++++++++++++++++++++++++++++++++++++++++++++++++++++++++++++++++++++++++++++++

\Exo{0062}
\Exo{0063}
\Exo{0064}
\Exo{0065}
\Exo{0066}
\Exo{0067}


%+++++++++++++++++++++++++++++++++++++++++++++++++++++++++++++++++++++++++++++++++++++++++++++++++++++++++++++++++++++++++++
					\section{Exercices de topologie}
%+++++++++++++++++++++++++++++++++++++++++++++++++++++++++++++++++++++++++++++++++++++++++++++++++++++++++++++++++++++++++++

Si $A_n$ est une suite d'ensemble, le symbole
\begin{equation}
	\bigcap_{n=1}^{\infty}A_n
\end{equation}
désigne l'ensemble des éléments qui sont dans $A_n$ pour tout $n\in\eN$. Remarquez que l'infini \emph{n'est pas} un élément de $\eN$ ! L'intersection se fait donc de $n=1$ à l'infini; l'infini non compris.

Prenons comme exemple le cas du point \ref{ItemF0072} de l'exercice \ref{exo0072}. Étant donné que $A_n=\mathopen]-\frac{1}{ n },\frac{1}{ n }\mathclose[$, on pourrait croire que $A_{\infty}=\mathopen]0,0\mathclose[=\emptyset$, et que par conséquent, l'intersection $\cap_{n=1}^{\infty}$ est vide.

%---------------------------------------------------------------------------------------------------------------------------
					\subsection{Exercices ultra basiques}
%---------------------------------------------------------------------------------------------------------------------------


\Exo{0071}
\Exo{0072}
\Exo{0073}
\Exo{0074}
\Exo{0075}
\Exo{0076}
\Exo{0077}
\Exo{0078}
\Exo{0079}
\Exo{0080}

%---------------------------------------------------------------------------------------------------------------------------
					\subsection{Exercices simplement basiques}
%---------------------------------------------------------------------------------------------------------------------------

Les exercices qui suivent ne seront en principe pas vus aux séances (faute de temps, plus que faute d'envie), mais ils sont certainement très intéressants à regarder pour celles et ceux qui désirent en savoir un peu plus sur la topologie.

\Exo{0081}
\Exo{0082}
\Exo{0083}
\Exo{0084}
\Exo{0085}
\Exo{0086}
\Exo{0087}
\Exo{0088}
\Exo{0089}


%+++++++++++++++++++++++++++++++++++++++++++++++++++++++++++++++++++++++++++++++++++++++++++++++++++++++++++++++++++++++++++
					\section{Fonctions d'une variable réelle (suite)}
%+++++++++++++++++++++++++++++++++++++++++++++++++++++++++++++++++++++++++++++++++++++++++++++++++++++++++++++++++++++++++++

\Exo{0090}
\Exo{0091}
\Exo{0092}
\Exo{0093}
\Exo{0094}
\Exo{0095}
\Exo{0096}
\Exo{0097}
\Exo{0098}
\Exo{0099}
\Exo{0100}


%+++++++++++++++++++++++++++++++++++++++++++++++++++++++++++++++++++++++++++++++++++++++++++++++++++++++++++++++++++++++++++
					\section{Développements de Taylor et Maclaurin}
%+++++++++++++++++++++++++++++++++++++++++++++++++++++++++++++++++++++++++++++++++++++++++++++++++++++++++++++++++++++++++++

\Exo{Devel0001}
\Exo{Devel0002}
\Exo{Devel0003}
\Exo{Devel0004}

\Exo{Devel0009}

\Exo{Devel0005}
\Exo{Devel0006}
\Exo{Devel0007}
\Exo{Devel0008}


\Exo{reserve0002}

%+++++++++++++++++++++++++++++++++++++++++++++++++++++++++++++++++++++++++++++++++++++++++++++++++++++++++++++++++++++++++++
					\section{Optimisation sans contraintes}
%+++++++++++++++++++++++++++++++++++++++++++++++++++++++++++++++++++++++++++++++++++++++++++++++++++++++++++++++++++++++++++

\Exo{OptimSS0001}
\Exo{OptimSS0002}
\Exo{OptimSS0003}
\Exo{OptimSS0004}
\Exo{OptimSS0005}
\Exo{OptimSS0006}


%+++++++++++++++++++++++++++++++++++++++++++++++++++++++++++++++++++++++++++++++++++++++++++++++++++++++++++++++++++++++++++
					\section{Équations différentielles}
%+++++++++++++++++++++++++++++++++++++++++++++++++++++++++++++++++++++++++++++++++++++++++++++++++++++++++++++++++++++++++++


%---------------------------------------------------------------------------------------------------------------------------
					\subsection{Équations différentielles du premier ordre}
%---------------------------------------------------------------------------------------------------------------------------

\Exo{EqsDiff0001}
\Exo{EqsDiff0002}
\Exo{EqsDiff0003}
\Exo{EqsDiff0004}
\Exo{EqsDiff0005}

%---------------------------------------------------------------------------------------------------------------------------
					\subsection{Équations différentielles du second ordre}
%---------------------------------------------------------------------------------------------------------------------------

\Exo{EqsDiff0006}
\Exo{EqsDiff0007}
\Exo{EqsDiff0008}
\Exo{EqsDiff0009}

%---------------------------------------------------------------------------------------------------------------------------
					\subsection{Équations différentielles : modélisation}
%---------------------------------------------------------------------------------------------------------------------------

\Exo{EqsDiff0010}
\Exo{EqsDiff0011}
\Exo{EqsDiff0012}

\Exo{EqsDiff0013}
\Exo{EqsDiff0014}
\Exo{EqsDiff0015}
\Exo{EqsDiff0016}



%+++++++++++++++++++++++++++++++++++++++++++++++++++++++++++++++++++++++++++++++++++++++++++++++++++++++++++++++++++++++++++
					\section{Intégrales multiples}
%+++++++++++++++++++++++++++++++++++++++++++++++++++++++++++++++++++++++++++++++++++++++++++++++++++++++++++++++++++++++++++

%%%%%%%%%%%%%%%%%%%%%%%%%
%
% Tous les exercices de cette section ont été repris dans OutilsMath le 3 avril 2011.
%
%%%%%%%%%%%%%%%%%%%%%%%%

\Exo{IntMult0001}
\Exo{IntMult0002}

Calculer le volume ou la surface d'un domaine revient à intégrer la fonction constante $1$ sur le domaine. Si nous effectuons un changement de variables, le jacobien intervient toutefois.

\Exo{IntMult0003}
\Exo{IntMult0004}
\Exo{IntMult0005}
\Exo{IntMult0006}
\Exo{IntMult0007}
\Exo{IntMult0008}
\Exo{IntMult0009}
% Il n'y a plus de IntMult0010 parce qu'il traitait de l'intégrale gausienne et a été 
% intégré aux notes d'agrégation. 4 août 2012.

\Exo{IntMult0011}
\Exo{IntMult0012}
\Exo{IntMult0013}


%+++++++++++++++++++++++++++++++++++++++++++++++++++++++++++++++++++++++++++++++++++++++++++++++++++++++++++++++++++++++++++
					\section{Théorème de la fonction implicite}
%+++++++++++++++++++++++++++++++++++++++++++++++++++++++++++++++++++++++++++++++++++++++++++++++++++++++++++++++++++++++++++


\Exo{Implicite0001}                                                                
\Exo{Implicite0002}                                                                
\Exo{Implicite0003}                                                                
\Exo{Implicite0004}                                                                
\Exo{Implicite0005}                                                                
\Exo{Implicite0006}                                                                
\Exo{Implicite0007}                                                                
\Exo{Implicite0008}                                                                
\Exo{Implicite0009}                                                                


%+++++++++++++++++++++++++++++++++++++++++++++++++++++++++++++++++++++++++++++++++++++++++++++++++++++++++++++++++++++++++++
\section{Variétés et extrema liés}
%+++++++++++++++++++++++++++++++++++++++++++++++++++++++++++++++++++++++++++++++++++++++++++++++++++++++++++++++++++++++++++

\Exo{Variete0001}                                                                
\Exo{Variete0002}                                                                
\Exo{Variete0003}                                                                
\Exo{Variete0004}                                                                
\Exo{Variete0005}                                                                

%+++++++++++++++++++++++++++++++++++++++++++++++++++++++++++++++++++++++++++++++++++++++++++++++++++++++++++++++++++++++++++
\section{Intégrales curvilignes}
%+++++++++++++++++++++++++++++++++++++++++++++++++++++++++++++++++++++++++++++++++++++++++++++++++++++++++++++++++++++++++++


\Exo{Variete0006}
\Exo{Variete0007}                                                                
\Exo{Variete0008}                                                                
\Exo{Variete0009}                                                                


\Exo{Variete0010}       % repris en version très allégée dans OutilsMath. Dans OutilsMath, il y aura un corrigé                                                              
\Exo{Variete0011}                                                                


%+++++++++++++++++++++++++++++++++++++++++++++++++++++++++++++++++++++++++++++++++++++++++++++++++++++++++++++++++++++++++++
\section{Intégrales de surface, Stokes et Green}
%+++++++++++++++++++++++++++++++++++++++++++++++++++++++++++++++++++++++++++++++++++++++++++++++++++++++++++++++++++++++++++

\Exo{Variete0012}                                                                
\Exo{Variete0013}                                                                
\Exo{Variete0014}                                                                
\Exo{Variete0015}                                                                
\Exo{Variete0016}                                                                
\Exo{Variete0017}                                                                



\Exo{Variete0018}                                                                
\Exo{Variete0019}                                                                
\Exo{Variete0020}         

\begin{center}
	Bonnes vacances !
\end{center}
