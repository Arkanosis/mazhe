% This is part of Exercices et corrigés de CdI-1
% Copyright (c) 2011
%   Laurent Claessens
% See the file fdl-1.3.txt for copying conditions.

\begin{corrige}{OutilsMath-0059}

    Si $F$ était le gradient de la fonction $f$, alors le rotationnel de $F$ serait nul : $\nabla\times(\nabla f)=0$. Calculons le rotationnel de $F$ :
    \begin{equation}
        \nabla\times F=\begin{vmatrix}
           1_x &   1_y    &   1_z    \\
            \partial_x    &   \partial_y    &   \partial_z    \\
            x^2\sin(xy)    &   y^2+x\sin(y^2)    &   0
        \end{vmatrix}=
        \big( \sin(y^2)-x^3\cos(xy) \big)1_z.
    \end{equation}
    Cela n'est donc pas nul et nous en déduisons qu'il n'existe pas de fonctions $f$ telles que $F=\nabla f$.

\end{corrige}
