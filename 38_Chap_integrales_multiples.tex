% This is part of Mes notes de mathématique
% Copyright (c) 2009-2015
%   Laurent Claessens
% See the file fdl-1.3.txt for copying conditions.

%+++++++++++++++++++++++++++++++++++++++++++++++++++++++++++++++++++++++++++++++++++++++++++++++++++++++++++++++++++++++++++
\section{Trucs et astuces de calcul d'intégrales}
%+++++++++++++++++++++++++++++++++++++++++++++++++++++++++++++++++++++++++++++++++++++++++++++++++++++++++++++++++++++++++++

Afin d'alléger le texte de calculs parfois un peu longs, nous regroupons ici les intégrales à une variable que nous devons utiliser dans les autres parties du cours.

\begin{enumerate}
	\item	\label{ItemIntegrali}
		L'intégrale
		\begin{equation}
			\boxed{I=\int x\ln(x)dx=\frac{ x^2 }{2}\big( \ln(x)-\frac{ 1 }{2} \big)}
		\end{equation}
		se fait par partie en posant
		\begin{equation}
			\begin{aligned}[]
				u&=\ln(x),		& dv&=x\,dx\\
				du&=\frac{1}{ x }\,dx,	& v&=\frac{ x^2 }{2},
			\end{aligned}
		\end{equation}
		et ensuite
		\begin{equation}
			I=\ln(x)\frac{ x^2 }{2}-\int\frac{ x }{2}=\frac{ x^2 }{2}\big( \ln(x)-\frac{ 1 }{2} \big).
		\end{equation}
		
	\item	
		L'intégrale
		\begin{equation}
			\boxed{I=\int x\ln(x^2)dx=x^2\ln(x)-\frac{ x^2 }{2}.}
		\end{equation}
		En utilisant le fait que $\ln(u^2)=2\ln(u)$, nous retombons sur une intégrale du type \ref{ItemIntegrali} :
		\begin{equation}
			I=x^2\ln(x)-\frac{ x^2 }{2}.
		\end{equation}
	\item
		L'intégrale
		\begin{equation}		\label{EqTrucIntxlnxsqpun}
			\boxed{I=\int x\ln(1+x^2)dx=\frac{ 1 }{2}\ln(x^2+1)(x^2+1)-x^2-\frac{ 1 }{2}}
		\end{equation}
		se traite en posant $v=1+x^2$ de telle sorte à avoir $dx=\frac{ dv }{ 2x }$ et donc
		\begin{equation}
			I=\frac{ 1 }{2}\ln(x^2+1)(x^2+1)-x^2-\frac{ 1 }{2}.
		\end{equation}
		
	\item
		L'intégrale
		\begin{equation}
			I=\int \cos(\theta)\sin(\theta)\ln\left( 1+\frac{1}{ \cos^2(\theta) } \right)\,d\theta
		\end{equation}
		demande le changement de variable $u=\cos(\theta)$, $d\theta=-\frac{ du }{ \sin(\theta) }$. Nous tombons sur l'intégrale
		\begin{equation}
			I=-\int u\ln\left( \frac{ 1+u^2 }{ u^2 } \right)=-\int u\ln(1+u^2)+\int u\ln(u^2),
		\end{equation}
		qui sont deux intégrales déjà faites. Nous trouvons
		\begin{equation}
			I=-\frac{ 1 }{2}\ln\left( \frac{ \sin^2(\theta)-1 }{ \sin^2(\theta)-2 } \right)\sin^2(\theta)-\ln\big( \sin^2(\theta)-2 \big)+\frac{ 1 }{2}\ln\big( \sin^2(\theta)-1 \big)
		\end{equation}
	
	\item
		L'intégrale
		\begin{equation}
			\boxed{\int \frac{ r^3 }{ 1+r^2 }dr=\frac{ r^2 }{2}-\frac{ 1 }{2}\ln(r^2+1).}
		\end{equation}
		commence par faire la division euclidienne de $r^3$ par $r^2+1$; ce que nous trouvons est $r^3=(r^2+1)r-r$. Il reste à intégrer
		\begin{equation}
			\int \frac{ r^3 }{ 1+r^2 }dr=\int r\,dr-\int\frac{ r }{ 1+r^2 }dr.
		\end{equation}
		La fonction dans la seconde intégrale est $\frac{ r }{ 1+r^2 }=\frac{ 1 }{2}\frac{ f'(r) }{ f(r) }$ où $f(r)=1+r^2$, et donc $\int \frac{ r }{ 1+r^2 }=\frac{ 1 }{2}\ln(1+r^2)$. Au final,
		\begin{equation}
			I=\frac{ 1 }{2}r^2-\frac{ 1 }{2}\ln(r^2+1).
		\end{equation}


	\item	
		L'intégrale
		\begin{equation}	\label{EqTrucIntsxcxdx}
			\boxed{I=\int \cos(\theta)\sin(\theta)d\theta=\frac{ \sin^2(\theta) }{ 2 }}
		\end{equation}
		se traite par le changement de variable $u=\sin(\theta)$, $du=\cos(\theta)d\theta$, et donc
		\begin{equation}
			\int\cos(\theta)\sin(\theta)d\theta=\int udu=\frac{ u^2 }{2}=\frac{ \sin^2(\theta) }{ 2 }.
		\end{equation}
	\item
		L'intégrale
		\begin{equation}	\label{EqTrucsIntsqrtAplusu}
			\boxed{\int\sqrt{1+x^2}dx=\frac{ x }{2}\sqrt{1+x^2}+\frac{ 1 }{2}\arcsinh(x)}
		\end{equation}
		s'obtient en effectuant le changement de variable $u=\sinh(\xi)$.

    \item
        L'intégrale
        \begin{equation}        \label{EqTrucIntcossqsinsq}
            \boxed{ \int\cos^2(x)\sin^2(x)dx=\frac{ x }{ 8 }-\frac{ \sin(4x) }{ 32 } }
        \end{equation}
        s'obtient à coups de formules de trigonométrie. D'abord, $\sin(t)\cos(t)=\frac{ 1 }{2}\sin^2(2t)$ fait en sorte que la fonction à intégrer devient 
        \begin{equation}
            f(x)=\frac{1}{ 4 }\sin^2(x).
        \end{equation}
        Ensuite nous utilisons le fait que $\sin^2(t)=(1-\cos(2t))/2$ pour transformer la formule à intégrer en
        \begin{equation}
            f(x)=\frac{ 1-\cos(4x) }{ 8 }.
        \end{equation}
        Cela s'intègre facilement en posant $u=4x$, et le résultat est
        \begin{equation}
            \int f(x)dx=\frac{ x }{ 8 }-\frac{ \sin(4x) }{ 32 }.
        \end{equation}

    \item

        La fonction 
        \begin{equation}
            \sinc(x)=\frac{ \sin(x) }{ x }
        \end{equation}
        est le \defe{sinus cardinal}{sinus cardinal} de \( x\). Nous allons montrer que
        \begin{equation}    \label{EqKNOmLEd}
            \boxed{  \int_0^{\infty}\big| \sinc(x) \big|dx=\infty  }.
        \end{equation}
        D'abord nous avons
        \begin{equation}
            \int_{(n-1)\pi}^{n\pi}\frac{ \big| \sin(t) \big| }{ t }dt\geq \int_{(n-1)\pi}^{n\pi}\frac{ \big| \sin(t) \big| }{ n\pi }dt,
        \end{equation}
        mais par périodicité,
        \begin{equation}
            \int_{(n-1)\pi}^{n\pi}\big| \sin(t) \big|dt=\int_0^{\pi}\sin(t)dt=2.
        \end{equation}
        Par conséquent
        \begin{equation}
            \int_0^{n\pi}\big| \sinc(t) \big|dt\geq \frac{ 2 }{ \pi }\sum_{k=1}^n\frac{1}{ k },
        \end{equation}
        ce qui diverge lorsque \( n\to \infty\).

\end{enumerate}

%---------------------------------------------------------------------------------------------------------------------------
\subsection{Reformer un carré au dénominateur}
%---------------------------------------------------------------------------------------------------------------------------
\label{subsecCarreDenoPar}

Lorsqu'on a un second degré au dénominateur, le bon plan est de reformer un carré parfait. Par exemple : 
\begin{equation}
	x^2+2x+2=(x+1)^2+1.
\end{equation}
Ensuite, le changement de variable $t=x+1$ est pratique parce que cela donne $t^2+1$ au dénominateur.

Cherchons
\begin{equation}
	I=\int \frac{ 1-x }{ x^2+2x+2 }dx=\int\frac{ 1-x }{ (x+1)^2+1 }dx=\int\frac{ 1-(t-1) }{ t^2+1 }
\end{equation}
où nous avons fait le changement de variable $t=x+1$, $dt=dx$. L'intégrale se coupe maintenant en deux parties :
\begin{equation}
	I=\int\frac{ -t }{ t^2+1 }+\int \frac{ 2 }{ t^2+1 }.
\end{equation}
La seconde est dans les formulaires et vaut 
\begin{equation}
	2\arctan(t)=2\arctan(x+1),
\end{equation}
tandis que la première est presque de la forme $f'/f$ :
\begin{equation}
	\int\frac{ t }{ t^2+1 }=\frac{ 1 }{2}\int \frac{ 2t }{ t^2+1 }=\frac{ 1 }{2}\ln(t^1+1)=\frac{ 1 }{2}\ln(u^2+2u+2).
\end{equation}


%+++++++++++++++++++++++++++++++++++++++++++++++++++++++++++++++++++++++++++++++++++++++++++++++++++++++++++++++++++++++++++ 
\section{Ellipsoïde de John-Loewer}
%+++++++++++++++++++++++++++++++++++++++++++++++++++++++++++++++++++++++++++++++++++++++++++++++++++++++++++++++++++++++++++

Soit \( q\) une forme quadratique sur \( \eR^n\) ainsi que \( \mB\) une base orthonormée de \( \eR^n\) dans laquelle la matrice de  \( q\) est diagonale. Dans cette base, la forme \( q\) est donnée par la proposition \ref{PropFWYooQXfcVY} :
\begin{equation}
    q(x)=\sum_i\lambda_ix_i
\end{equation}
où les \( \lambda_i\) sont les valeurs propres de \( q\).

Plus généralement nous notons \( mat_{\mB}(q)\)\nomenclature[A]{\( mat_{\mB}(q)\)}{matrice de \( q\) dans la base \( \mB\)} la matrice de \( q\) dans la base \( \mB\) de \( \eR^n\).

\begin{proposition} \label{PropOXWooYrDKpw}
    Soit \( \mB\) une base orthonormée de \( \eR^n\) et l'application\footnote{L'ensemble \( Q(E)\) est l'ensemble des formes quadratiques sur \( E\).}
    \begin{equation}
        \begin{aligned}
            D\colon Q(\eR^n)&\to \eR \\
            q&\mapsto \det\big( mat_{\mB}(q) \big) .
        \end{aligned}
    \end{equation}
    Alors :
    \begin{enumerate}
        \item
            La valeur et \( D\) ne dépend pas du choix de la base orthonormée \( \mB\).
        \item
            La fonction \( D\) est donnée par la formule \( D(q)=\prod_i\lambda_i\) où les \( \lambda_i\) sont les valeurs propres de \( q\).
        \item
            La fonction \( D\) est continue.
    \end{enumerate}
\end{proposition}

\begin{proof}
    Soit \( q\) une forme quadratique sur \( \eR^n\). Nous considérons \( \mB\) une base de diagonalisation de \( q\) :
    \begin{equation}
        q(x)=\sum_i\lambda_ix_i
    \end{equation}
    où les \( x_i\) sont les composantes de \( x\) dans la base \( \mB\). Par définition, la matrice \( mat_{\mB}(q)\) est la matrice diagonale contenant les valeurs propres de \( q\).

    Nous considérons aussi \( \mB_1\), une autre base orthonormées de \( \eR^n\). Nous notons \( S=mat_{\mB_1}(q)\); étant symétrique, cette matrice se diagonalise par une matrice orthogonale : il existe \( P\in\gO(n,\eR)\) telle que
    \begin{equation}
        S=P mat_{\mB}(q)P^t;
    \end{equation}
    donc \( \det(S)=\det(PP^t)\det\big( \diag(\lambda_1,\ldots, \lambda_n) \big)=\lambda_1\ldots\lambda_n\). Ceci prouve en même temps que \( D\) ne dépend pas du choix de la base et que sa valeur est le produit des valeurs propres.

    Passons à la continuité. L'application déterminant \( \det\colon S_n(\eR^n)\to \eR\) est continue car polynôme en les composantes. D'autre par l'application \( mat_{\mB}\colon Q(\eR^n)\to S_n(\eR)\) est continue par la proposition \ref{PropFSXooRUMzdb}. L'application  \( D\) étant la composée de deux applications continues, elle est continue.
\end{proof}

\begin{proposition}[Ellipsoïde de John-Loewner\cite{KXjFWKA}]   \label{PropJYVooRMaPok}
    Soit \( K\) compact dans \( \eR^n\) et d'intérieur non vide. Il existe une unique ellipsoïde\footnote{Définition \ref{DefOEPooqfXsE}.} (pleine) de volume minimal contenant \( K\).
\end{proposition}
\index{déterminant!utilisation}
\index{extrema!volume d'un ellipsoïde}
\index{convexité!utilisation}

\begin{proof}
    Nous subdivisons la preuve en plusieurs parties.
    \begin{subproof}
        \item[À propos de volume d'un ellipsoïde]

            Soit \( \ellE\) un ellipsoïde. La proposition \ref{PropWDRooQdJiIr} et son corollaire \ref{CorKGJooOmcBzh} nous indiquent que 
            \begin{equation}
                \ellE=\{ x\in \eR^n\tq q(x)\leq 1 \}
            \end{equation}
            pour une certaine forme quadratique strictement définie positive \( q\). De plus il existe une base orthonormée \( \mB=\{ e_1,\ldots, e_n \}\) de \( \eR^n\) telle que 
            \begin{equation}    \label{EqELBooQLPQUj}
                q(x)=\sum_{i=1}^na_ix_i^2
            \end{equation}
            où \( x_i=\langle e_i, x\rangle \) et les \( a_i\) sont tous strictement positifs. Nous nommons \( \ellE_q\) l'éllipsoïde associée à la forme quadratique \( q\) et \( V_q\) son volume que nous allons maintenant calculer\footnote{Le volume ne change pas si nous écrivons l'inégalité stricte au lieu de large dans le domaine d'intégration; nous le faisons pour avoir un domaine ouvert.} :
            \begin{equation}
                V_q=\int_{\sum_ia_ix_i^2<1}dx
            \end{equation}
            Cette intégrale est écrite de façon plus simple en utilisant le \( C^1\)-difféomorphisme
            \begin{equation}
                \begin{aligned}
                    \varphi\colon \ellE_q&\to B(0,1) \\
                    x&\mapsto \Big( x_1\sqrt{a_1},\ldots, x_n\sqrt{a_n} \Big). 
                \end{aligned}
            \end{equation}
            Le fait que \( \varphi\) prenne bien ses valeurs dans \( B(0,1)\) est un simple calcul : si \( x\in\ellE_q\), alors
            \begin{equation}
                \sum_i\varphi(x)_i^2=\sum_ia_ix_i^2<1.
            \end{equation}
            Cela nous permet d'utiliser le théorème de changement de variables \ref{ThomFeRCi} :
            \begin{equation}
                V_q=\int_{\sum_ia_ix_i^2<1}dx=\frac{1}{ \sqrt{a_1\ldots a_n} }\int_{B(0,1)}dx.
            \end{equation}
            %TODO : le volume de la sphère dans \eR^n. Mettre alors une référence ici.
            La dernière intégrale est le volume de la sphère unité dans \( \eR^n\); elle n'a pas d'importance ici et nous la notons \( V_0\). La proposition \ref{PropOXWooYrDKpw} nous permet d'écrire \(V_q\) sous la forme
            \begin{equation}
                V_q=\frac{ V_0 }{ \sqrt{D(q)} }.
            \end{equation}
            
        \item[Existence de l'ellipsoïde]

            Nous voulons trouver un ellipsoïde contenant \( K\) de volume minimal, c'est à dire une forme quadratique \( q\in Q^{++}(\eR^n)\) telle que
            \begin{itemize}
                \item \( D(q)\) soit maximal
                \item \( q(x)\leq 1\) pour tout \( x\in K\).
            \end{itemize}
            Nous considérons l'ensemble des candidats semi-définis positifs.
            \begin{equation}
                A=\{ q\in Q^+\tq q(x)\leq 1\forall x\in K \}.
            \end{equation}
            Nous allons montrer que \( A\) est convexe, compact et non vide dans \( Q(\eR^n)\); il aura ainsi un maximum de la fonction continue \( D\) définie sur \( Q(\eR^n)\). Nous montrerons ensuite que le maximum est dans \( Q^{++}\). L'unicité sera prouvée à part.

            \begin{subproof}
            \item[Non vide]
                L'ensemble \( K\) est compact et donc borné par \( M>0\). La forme quadratique \( q\colon x\mapsto \| x \|^2/M^2\) est dans \( A\) parce que si \( x\in K\) alors 
                \begin{equation}
                    q(x)=\frac{ \| x \|^2 }{ M^2 }\leq 1.
                \end{equation}
            \item[Convexe]
                Soient \( q,q'\in A\) et \( \lambda\in\mathopen[ 0 , 1 \mathclose]\). Nous avons encore \( \lambda q+(1-\lambda)q'\in Q^+\) parce que 
                \begin{equation}
                    \lambda q(x)+(1-\lambda)q'(x)\geq 0
                \end{equation}
                dès que \( q(x)\geq 0\) et \( q'(x)\geq 0\).
            D'autre part si \( x\in K\) nous avons
            \begin{equation}
                \lambda q(x)+(1-\lambda)q'(x)\leq \lambda+(1-\lambda)=1.
            \end{equation}
            Donc \( \lambda q+(1-\lambda)q'\in A\).

        \item[Fermé]

            Pour rappel, la topologie de \( Q(\eR^n)\) est celle de la norme \eqref{EqZYBooZysmVh}. Nous considérons une suite \( (q_n)\) dans \( A\) convergeant vers \( q\in Q(\eR^n)\) et nous allons prouver que \( q\in A\), de sorte que la caractérisation séquentielle de la fermeture (proposition \ref{PropLFBXIjt}) conclue que \( A\) est fermé. En nommant \( e_x\) le vecteur unitaire dans la direction \( x\) nous avons
            \begin{equation}
                \big| q(x) \big|=\big| \| x \|^2q(e_x) \big|\leq \| x \|^2N(q),
            \end{equation}
            de sorte que notre histoire de suite convergente  donne pour tout \( x\) :
            \begin{equation}
                \big| q_n(x)-q(x) \big|\leq \| x \|^2N(q_n-q)\to 0.
            \end{equation}
            Vu que \( q_n(x)\geq 0\) pour tout \( n\), nous devons aussi avoir \( q(x)\geq 0\) et donc \( q\in Q^+\) (semi-définie positive). De la même manière si \( x\in K\) alors \( q_n(x)\leq 1\) pour tout \( n\) et donc \( q(x)\leq 1\). Par conséquent \( q\in A\) et \( A\) est fermé.

        \item[Borné]

            La partie \( K\) de \( \eR^n\) est borné et d'intérieur non vide, donc il existe \( a\in K\) et \( r>0\) tel que \( \overline{ B(a,r) }\subset K\). Si par ailleurs \( q\in A\) et \( x\in\overline{ B(0,r) }\) nous avons \( a+x\in K\) et donc \( q(a+x)\leq 1\). De plus \( q(-a)=q(a)\leq 1\), donc
            \begin{equation}
                \sqrt{q(x)}=\sqrt{q\big( x+a-a \big)}\leq \sqrt{q(x+a)}+\sqrt{q(-a)}\leq 2
            \end{equation}
            par l'inégalité de Minkowski \ref{PropACHooLtsMUL}. Cela prouve que si \( x\in\overline{ B(0,r) }\) alors \( q(x)\leq 4\). Si par contre \( x\in\overline{ B(0,1) }\) alors \( rx\in\overline{ B(0,r) } \) et 
            \begin{equation}
                0\leq q(x)=\frac{1}{ r^2 }q(rx)\leq \frac{ 4 }{ r^2 },
            \end{equation}
            ce qui prouve que \( N(q)\leq \frac{ 4 }{ r^2 }\) et que \( A\) est borné.


            \end{subproof}

            L'ensemble \( A\) est compact parce que fermé et borné, théorème de Borel-Lebesgue \ref{ThoXTEooxFmdI}. L'application continue \( D\colon Q(\eR^n)\to \eR\) de la proposition \ref{PropOXWooYrDKpw} admet donc un maximum sur le compact \( A\). Soit \( q_0\) ce maximum.

            Nous montrons que \( q_0\in Q^{++}(\eR^d)\). Nous savons que l'application \( f\colon x\mapsto \frac{ \| x \|^2 }{ M^2 }\) est dans \( A\) et que \( D(f)>0\). Vu que \( q_0\) est maximale pour \( D\), nous avons
            \begin{equation}
                D(q_0)\geq D(f)>0.
            \end{equation}
            Donc \( q_0\in Q^{++}\).

        \item[Unicité]

            Si il existe une autre ellipsoïde de même volume que celle associée à la forme quadratique \( q_0\), nous avons une forme quadratique \( q\in Q^{++}\) telle que \( q(x)\leq 1\) pour tout \( x\in K\). C'est à dire que nous avons \( q_0,q\in A\) tels que \( D(q_0)=D(q)\).

            Nous considérons la base canonique \( \mB_c\) de \( \eR^n\) et nous posons \( S=mat_{\mB_c}(q)\), \( S_0=mat_{\mB_c}(q_0)\). Étant donné que \( A\) est convexe, \( (q_0+q)/2\in A\) et nous allons prouver que cet élément de \( A\) contredit la maximalité de \( q_0\). En effet
            \begin{equation}
                D\left( \frac{ q+q_0 }{ 2 }\right)=\det\left( \frac{ S+S_0 }{2} \right)
            \end{equation}
            Nous allons utiliser le lemme \ref{LemXOUooQsigHs} qui dit que le logarithme est log-concave sous la forme de l'équation \eqref{EqSPKooHFZvmB} avec \( \alpha=\beta=\frac{ 1 }{2}\) :
            \begin{equation}    \label{eqBHJooYEUDPC}
                D\left( \frac{ q+q_0 }{ 2 }\right)=\det\left( \frac{ S+S_0 }{2} \right)>\sqrt{\det(S)}\sqrt{\det(S_0)}=\det(S_0)=D(q_0).
            \end{equation}
            Nous avons utilisé le fait que \( D(q_0)=D(q)\) qui signifie que \( \det(S_0)=\det(S)\). L'inéquation \eqref{eqBHJooYEUDPC} contredit la maximalité de \( D(q_0)\) et donne donc l'unicité.
    \end{subproof}
\end{proof}
% This is part of Mes notes de mathématique
% Copyright (c) 2011-2014
%   Laurent Claessens
% See the file fdl-1.3.txt for copying conditions.

%+++++++++++++++++++++++++++++++++++++++++++++++++++++++++++++++++++++++++++++++++++++++++++++++++++++++++++++++++++++++++++
\section{Rappel sur les intégrales usuelles}
%+++++++++++++++++++++++++++++++++++++++++++++++++++++++++++++++++++++++++++++++++++++++++++++++++++++++++++++++++++++++++++

%TODO : l'utilisation des macros \og et \fg ne se justifie plus : les enlever.

Soit une fonction
\begin{equation}
    \begin{aligned}
        f\colon \mathopen[ a , b \mathclose]\subset\eR&\to \eR^+ \\
        x&\mapsto f(x) .
    \end{aligned}
\end{equation}
L'intégrale de $f$ sur le segment $\mathopen[ a , b \mathclose]$, notée $\int_a^bf(x)dx$ est le nombre égal à l'aire de la surface située entre le graphe de $f$ et l'axe des $x$, comme indiqué à la figure \ref{LabelFigKKLooMbjxdI}. % From file KKLooMbjxdI
\newcommand{\CaptionFigKKLooMbjxdI}{L'intégrale de $f$ entre $a$ et $b$ représente la surface sous la fonction.}
\input{Fig_KKLooMbjxdI.pstricks}

\begin{definition}
    Si $f$ est une fonction de une variable à valeurs réelles, une \defe{primitive}{primitive} de $f$ est une fonction $F$ telle que $F'=f$.
\end{definition}

Toute fonction continue admet une primitive.

\begin{theorem}[Théorème fondamental du caclul intégral]
    Si $f$ est une fonction positive et continue, et si $F$ est une primitive de $f$, alors
    \begin{equation}
        \int_a^bf(x)dx=F(b)-F(a).
    \end{equation}
\end{theorem}

\begin{remark}
    Si $f$ est une fonction continue par morceaux, l'intégrale de $f$ se calcule comme la somme des intégrales de ses morceaux. Plus précisément si nous avons $a=x_0<x_1<\ldots<x_n=b$ et si $f$ est continue sur $\mathopen] x_i , x_{i+1} \mathclose[$ pour tout $i$, alors nous posons
    \begin{equation}
        \int_a^bf(x)dx=\int_{x_0}^{x_1}f(x)dx+\int_{x_1}^{x_2}f(x)dx+\ldots+\int_{x_{n-1}}^{n_n}f(x)dx.
    \end{equation}
    Sur chacun des morceaux, l'intégrale se calcule normalement en passant par une primitive.
\end{remark}

%+++++++++++++++++++++++++++++++++++++++++++++++++++++++++++++++++++++++++++++++++++++++++++++++++++++++++++++++++++++++++++
\section{Intégration de fonction à deux variables}
%+++++++++++++++++++++++++++++++++++++++++++++++++++++++++++++++++++++++++++++++++++++++++++++++++++++++++++++++++++++++++++

%---------------------------------------------------------------------------------------------------------------------------
\subsection{Intégration sur un domaine rectangulaire}
%---------------------------------------------------------------------------------------------------------------------------
\label{PgRapIntMultFubiniRect}

Soit une fonction positive
\begin{equation}
    \begin{aligned}
        f\colon \mathopen[ a , b \mathclose]\times\mathopen[ c , d \mathclose]&\to \eR^+ \\
        (x,y)&\mapsto f(x,y). 
    \end{aligned}
\end{equation}

L'intégrale de $f$ sur le rectangle $\mathopen[ a , b \mathclose]\times\mathopen[ c , d \mathclose]$ est le volume sous le graphe de la fonction. C'est à dire le volume de l'ensemble
\begin{equation}
    \{ (x,y,z)\tq (x,y)\in\mathopen[ a , b \mathclose]\times\mathopen[ c , d \mathclose], z\leq f(x,y) \}.
\end{equation}

\begin{theorem}[Théorème de Fubini]
    Soit une fonction $f\colon \eR^2\to \eR$ une fonction continue par morceaux sur $\mR=\mathopen[ a , b \mathclose]\times\mathopen[ c , d \mathclose]$. Alors
    \begin{equation}
        \int_{\mR}f(x,y)dxdy=\int_a^b\left[ \int_c^df(x,y)dy \right]dx=\int_c^d\left[ \int_a^bf(x,y)dx \right]dy.
    \end{equation}
\end{theorem}
\index{théorème!Fubini!version compacte dans \( \eR^2\)}

En pratique, nous utilisons le théorème de Fubini pour calculer les intégrales sur des rectangles.

\begin{example}
    Nous voudrions intégrer la fonction $f(x,y)-4+x^2+y^2$ sur le rectangle de la figure \ref{LabelFigVNBGooSqMsGU}. % From file VNBGooSqMsGU
\newcommand{\CaptionFigVNBGooSqMsGU}{Intégration sur un rectangle}
\input{Fig_VNBGooSqMsGU.pstricks}

    L'ensemble sur lequel nous intégrons est donné par le produit cartésien d'intervalles $E=[0,1]\times[0,2]$. Le théorème de Fubini montre que nous pouvons intégrer séparément sur l'intervalle horizontal et vertical :
    \begin{equation}
    	\int_{E=[0,1]\times[0,2]}f=\int_{[0,1]}\left( \int_{[0,2]}(4-x^2-y^2)dy \right)dx.
    \end{equation}
    Ces intégrales sont maintenant des intégrales usuelles qui s'effectuent en calculant des primitives :
    \begin{equation}
        \begin{aligned}[]
            \int_0^1\int_0^2(4-x^2-y^2)dy\,dx&=\int_0^1\left[ 4y-x^2y-\frac{ y^3 }{ 3 } \right]_0^2dx\\
            &=\int_0^1(8-2x^2-\frac{ 8 }{ 3 })dx\\
            &=\left[ \frac{ 16x }{ 3 }-\frac{ 2x^3 }{ 3 } \right]_0^1\\
            &=\frac{ 14 }{ 3 }.
        \end{aligned}
    \end{equation}
    Avec Sage, on peut faire comme ceci :

    \begin{verbatim}
----------------------------------------------------------------------
| Sage Version 4.6.1, Release Date: 2011-01-11                       |
| Type notebook() for the GUI, and license() for information.        |
----------------------------------------------------------------------
sage: f(x,y)=4-x**2-y**2                  
sage: f.integrate(y,0,2).integrate(x,0,1)
(x, y) |--> 14/3

    \end{verbatim}

\end{example}

%---------------------------------------------------------------------------------------------------------------------------
\subsection{Intégration sur un domaine non rectangulaire}
%---------------------------------------------------------------------------------------------------------------------------
\label{PgRapIntMultFubiniTri}

Nous voulons maintenant intégrer la fonction $f(x,y)=x^2+y^2$ sur le triangle de la figure \ref{LabelFigCURGooXvruWV}. % From file CURGooXvruWV
\newcommand{\CaptionFigCURGooXvruWV}{Intégration sur un triangle}
\input{Fig_CURGooXvruWV.pstricks}

Étant donné que $y$ varie de $0$ à $2$ et que \emph{pour chaque $y$}, la variable $x$ varie de $0$ à $y$, nous écrivons l'intégrale sur le triangle sous la forme :
\begin{equation}
	\int_{\text{triangle}}(x^2+y^2)dx dy=\int_0^2\left( \int_0^y(x^2+y^2)dx \right)dy.
\end{equation}

Il existe principalement deux types de domaines non rectangulaires : les «horizontales» et les «verticales», voir figure \ref{LabelFigHCJPooHsaTgI}. % From file HCJPooHsaTgI
\newcommand{\CaptionFigHCJPooHsaTgI}{Deux types de surfaces. Nous avons tracé un rectangle qui contient chacune des deux surfaces. L'intégrale sur un domaine sera l'intégrale sur le rectangle de la fonction qui vaut zéro en dehors du domaine.}
\input{Fig_HCJPooHsaTgI.pstricks}

Les surfaces horizontales sont de la forme 
\begin{equation}
    D=\{ (x,y)\tq x\in\mathopen[ a , b \mathclose],\varphi_1(x)\leq y\leq \varphi_2(x) \}
\end{equation}
où $\varphi_1$ et $\varphi_2$ sont les deux fonctions qui bornent le domaine. Le domaine $D$ est la région comprise entre les graphes de $\varphi_1$ et $\varphi_2$. Pour un tel domaine nous avons
\begin{equation}
    \iint_Df(x,y)dxdy=\int_a^bdx\int_{\varphi_1(x)}^{\varphi_2(x)}f(x,y)dy.
\end{equation}

Les surfaces verticales sont de la forme 
\begin{equation}
    D=\{ (x,y)\tq y\in\mathopen[ c , d \mathclose],\psi_1(y)\leq x\leq \psi_2(y) \}
\end{equation}
où $\varphi_1$ et $\varphi_2$ sont les deux fonctions qui bornent le domaine. Le domaine $D$ est la région comprise entre les graphes de $\varphi_1$ et $\varphi_2$. Dans ces cas nous avons
\begin{equation}
    \iint_Df=\int_c^d dy\int_{\psi_1(y)}^{\psi_2(y)} f(x,y)dx.
\end{equation}

\begin{proposition}
    L'aire du domaine $D$ vaut l'intégrable de la fonction $f(x,y)=1$ sur $D$ :
    \begin{equation}
        Aire(D)=\iint_Ddxdy.
    \end{equation}
\end{proposition}

\begin{proof}
    Supposons que le domaine soit du type «horizontal». En utilisant le théorème de Fubini avec $f(x,y)=1$ nous avons
    \begin{equation}
        \iint_Ddxdy=\int_a^b\left[ \int_{\varphi_1(x)}^{\varphi_2(x)}dy \right]dx=\int_a^b\big[ \varphi_2(x)-\varphi_1(x) \big].
    \end{equation}
    Cela représente l'aire sous $\varphi_2$ moins l'aire sous $\varphi_1$, et par conséquent l'aire contenue entre les deux.
\end{proof}

\begin{example}
    Cherchons la surface du disque de centre $(0,0)$ et de rayon $1$ dessinée à la figure \ref{LabelFigCMMAooQegASg}. % From file CMMAooQegASg
\newcommand{\CaptionFigCMMAooQegASg}{En bleu, la fonction $\sqrt{r^2-x^2}$ et en rouge, la fonction $-\sqrt{r^2-x^2}$.}
\input{Fig_CMMAooQegASg.pstricks}

    Le domaine est donné par $\varphi_1(x)\leq y\leq \varphi_2(x)$ et $x\in\mathopen[ -r ,r \mathclose]$ où $\varphi_1(x)=-\sqrt{r^2-x^2}$ et $\varphi_2(x)=\sqrt{r^2-x^2}$. L'aire est donc donnée par
    \begin{equation}
        A=\int_{-r}^r\big[ \varphi_2(x)-\varphi_1(x) \big]dx=2\int_{-r}^r\sqrt{r^2-x^2}dx=4\int_0^r\sqrt{r^2-x^2}.
    \end{equation}
    Nous effectuons le premier changement de variables $x=ru$, donc $dx=rdu$. En ce qui concerne les bornes, si $x=0$, alors $u=0$ et si $x=r$, alors $u=1$. L'intégrale à calculer devient
    \begin{equation}
        A=4\int_0^1\sqrt{r^2-r^2u^2}rdu=4r^2\int_0^1\sqrt{1-u^2}du.
    \end{equation}
    Cette dernière intégrale se calcule en posant
    \begin{equation}
        \begin{aligned}[]
            u&=\sin(t)&du&=\cos(t)dt\\
            u&=0&t&=0\\
            u&=1&t&=\pi/2.
        \end{aligned}
    \end{equation}
    Nous avons
    \begin{equation}
        A=4r^2\int_0^{\pi/2}\sqrt{1-\sin^2(t)}\cos(t)dt=4r^2\int_0^{\pi/2}\cos^2(t)dt.
    \end{equation}
    En utilisant la formule $2\cos^2(x)=1+\cos(2x)$, nous avons
    \begin{equation}
        A=4r^2\int_0^{\pi/2}\frac{ 1+\cos(2t) }{ 2 }dt=\pi r^2.
    \end{equation}
\end{example}

%---------------------------------------------------------------------------------------------------------------------------
\subsection{Changement de variables}
%---------------------------------------------------------------------------------------------------------------------------

Comme dans les intégrales simples, il y a souvent moyen de trouver un changement de variables qui simplifie les expressions.  Le domaine $E=\{ (x,y)\in\eR^2\tq x^2+y^2<1 \}$ par exemple s'écrit plus facilement $E=\{ (r,\theta)\tq r<1 \}$ en coordonnées polaires. Le passage aux coordonnées polaire permet de transformer une intégration sur un domaine rond à une intégration sur le domaine rectangulaire $\mathopen]0,2\pi\mathclose[\times\mathopen]0,1\mathclose[$. La question est évidement de savoir si nous pouvons écrire
\begin{equation}
	\int_Ef=\int_{0}^{2\pi}\int_0^1f(r\cos\theta,r\sin\theta)drd\theta.
\end{equation}
Hélas ce n'est pas le cas. Il faut tenir compte du fait que le changement de base dilate ou contracte certaines surfaces.

Soit $\varphi\colon D_1\subset\eR^2\to D_2\subset \eR^2$ une fonction bijective de classe $C^1$ dont l'inverse est également de classe $C^1$. On désigne par $x$ et $y$ ses composantes, c'est à dire que
\begin{equation}
    \varphi(u,v)=\begin{pmatrix}
        x(u,v)    \\ 
        y(u,v)    
    \end{pmatrix}
\end{equation}
avec $(u,v)\in D_1$.

\begin{theorem}     \label{ThoChamDeVarIntDDf}
    Soit une fonction continue $f\colon D_2\to \eR$. Alors
    \begin{equation}
        \iint_{\varphi(D_1)}f(x,y)dxdy=\iint_{D_1}f\big( x(u,v),y(u,v) \big)| J_{\varphi}(u,v) |dudv
    \end{equation}
    où $J_{\varphi}$ est le Jacobien de $\varphi$.
\end{theorem}
Pour rappel,
\begin{equation}
    J_{\varphi}(u,v)=\det\begin{pmatrix}
        \frac{ \partial x }{ \partial u }    &   \frac{ \partial x }{ \partial v }    \\ 
        \frac{ \partial y }{ \partial u }    &   \frac{ \partial u }{ \partial v }    
    \end{pmatrix}.
\end{equation}
Ne pas oublier de prendre la valeur absolue lorsqu'on utilise le Jacobien dans un changement de variables.

%///////////////////////////////////////////////////////////////////////////////////////////////////////////////////////////
\subsubsection{Le cas des coordonnées polaires}
%///////////////////////////////////////////////////////////////////////////////////////////////////////////////////////////

La fonction qui donne les coordonnées polaires est
\begin{equation}
    \begin{aligned}
        \varphi\colon \eR^+\times\mathopen] 0 , 2\pi \mathclose[&\to \eR^2 \\
        (r,\theta)&\mapsto\begin{pmatrix}
            r\cos(\theta)    \\ 
            r\sin(\theta)    
        \end{pmatrix}.
    \end{aligned}
\end{equation}
Son Jacobien vaut
\begin{equation}
    J_{\varphi}(r,\theta)=\det\begin{pmatrix}
        \frac{ \partial x(r,\theta) }{ \partial r }    &   \frac{ \partial x(r,\theta) }{ \partial \theta }    \\ 
        \frac{ \partial y(r,\theta) }{ \partial r }    &   \frac{ \partial y(r,\theta) }{ \partial \theta }    
    \end{pmatrix}=
    \begin{vmatrix}
        \cos(\theta)    &   -r\sin(\theta)    \\ 
        \sin(\theta)    &   r\cos(\theta)    
    \end{vmatrix}=r.
\end{equation}

\begin{example}
    Calculons la surface du disque $D$ de rayon $R$. Nous devons calculer
    \begin{equation}
        \iint_Ddxdy.
    \end{equation}
    Pour passer au polaires, nous savons que le disque est décrit par 
    \begin{equation}
        D=\{ (r,\theta)\tq 0\leq r\leq R,0\leq\theta\leq 2\pi \}.
    \end{equation}
    Nous avons donc
    \begin{equation}
        \iint_Ddxdy=\iint_{D}r\,drd\theta=\int_0^{2\pi}\int_0^Rr\,drd\theta=2\pi\int_0^Rr\,dr=\pi R^2.
    \end{equation}
\end{example}

\begin{example}     \label{ExpmfDtAtV}
    Montrons comment intégrer la fonction $f(x,y)=\sqrt{1-x^2-y^2}$ sur le domaine délimité par la droite $y=x$ et le cercle $x^2+y^2=y$, représenté sur la figure \ref{LabelFigHFAYooOrfMAA}. Pour trouver le centre et le rayon du cercle $x^2+y^2=y$, nous commençons par écrire $x^2+y^2-y=0$, et ensuite nous reformons le carré : $y^2-y=(y-\frac{ 1 }{2})^2-\frac{1}{ 4 }$.

\newcommand{\CaptionFigHFAYooOrfMAA}{Passage en polaire pour intégrer sur un morceau de cercle.}
\input{Fig_HFAYooOrfMAA.pstricks}

    Le passage en polaire transforme les équations du bord du domaine en
    \begin{equation}
        \begin{aligned}[]
            \cos(\theta)&=\sin(\theta)\\
            r^2&=r\sin(\theta).
        \end{aligned}
    \end{equation}
    L'angle $\theta$ parcours donc $\mathopen] 0 , \pi/4 \mathclose[$, et le rayon, pour chacun de ces $\theta$ parcours $\mathopen] 0 , \sin(\theta) \mathclose[$. La fonction à intégrer se note maintenant $f(r,\theta)=\sqrt{1-r^2}$. Donc l'intégrale à calculer est
    \begin{equation}		\label{PgOMRapIntMultFubiniBoutCercle}
        \int_{0}^{\pi/4}\left( \int_0^{\sin(\theta)}\sqrt{1-r^2}r\,rd \right).
    \end{equation}
    Remarquez la présence d'un $r$ supplémentaire pour le jacobien.

    Notez que les coordonnées du point $P$ sont $(1,1)$.
\end{example}

En pratique, lors du passage en coordonnées polaires, le «$dxdy$» devient «$r\,drd\theta$».

%///////////////////////////////////////////////////////////////////////////////////////////////////////////////////////////
\subsubsection{Les coordonnées cylindriques}
%///////////////////////////////////////////////////////////////////////////////////////////////////////////////////////////

En ce qui concerne les coordonnées cylindriques, le Jacobien est donné par
\begin{equation}
    J(r,\theta,z)=\begin{vmatrix}
        \frac{ \partial x }{ \partial r }    &   \frac{ \partial x }{ \partial \theta }    &   \frac{ \partial x }{ \partial z }    \\
        \frac{ \partial y }{ \partial r }    &   \frac{ \partial y }{ \partial \theta }    &   \frac{ \partial y }{ \partial z }    \\
        \frac{ \partial z }{ \partial r }    &   \frac{ \partial z }{ \partial \theta }    &   \frac{ \partial z }{ \partial z }    
    \end{vmatrix}=
    \begin{vmatrix}
        \cos\theta    &   -r\sin\theta    &   0    \\
        \sin\theta    &   r\cos\theta    &   0    \\
        0    &   0    &   1
    \end{vmatrix}=r.
\end{equation}
Nous avons donc $dx\,dy\,dz=r\,dr\,d\theta\,dz$.

%///////////////////////////////////////////////////////////////////////////////////////////////////////////////////////////
\subsubsection{Coordonnées sphériques}
%///////////////////////////////////////////////////////////////////////////////////////////////////////////////////////////

Le calcul est un peu plus long :
\begin{equation}
    \begin{aligned}[]
        J(\rho,\theta,\varphi)&=\begin{vmatrix}
            \frac{ \partial x }{ \partial \rho }    &   \frac{ \partial x }{ \partial \theta }    &   \frac{ \partial x }{ \partial \varphi }    \\
            \frac{ \partial y }{ \partial \rho }    &   \frac{ \partial y }{ \partial \theta }    &   \frac{ \partial y }{ \partial \varphi }    \\
            \frac{ \partial z }{ \partial \rho }    &   \frac{ \partial z }{ \partial \theta }    &   \frac{ \partial z }{ \partial \varphi }    
        \end{vmatrix}\\ 
        &=
        \begin{vmatrix}
            \sin\theta\cos\varphi    &   \rho\cos\theta\cos\varphi    &   -\rho\sin\theta\sin\varphi    \\
            \sin\theta\sin\varphi    &   \rho\cos\theta\sin\varphi    &   -\rho\sin\theta\cos\varphi    \\
            \cos\theta               &   -\rho\sin\theta              &   0
        \end{vmatrix}\\
        &=\rho^2\sin\theta.
    \end{aligned}
\end{equation}
Donc 
\begin{equation}
    dx\,dy\,dz=\rho^2\sin(\theta)\,d\rho\,d\theta\,d\varphi.
\end{equation}

%///////////////////////////////////////////////////////////////////////////////////////////////////////////////////////////
\subsubsection{Un autre système utile}
%///////////////////////////////////////////////////////////////////////////////////////////////////////////////////////////

Un changement de variables que l'on voit assez souvent est
\begin{subequations}
    \begin{numcases}{}
        u=x+y\\
        v=x-y.
    \end{numcases}
\end{subequations}
Afin de calculer son jacobien, il faut d'abord exprimer $x$ et $y$ en fonctions de $u$ et $v$ :
\begin{subequations}
    \begin{numcases}{}
        x=(u+v)/2\\
        y=(u-v)/2.
    \end{numcases}
\end{subequations}
La matrice jacobienne est
\begin{equation}
    \begin{pmatrix}
        \frac{ \partial x }{ \partial u }    &   \frac{ \partial x }{ \partial v }    \\ 
        \frac{ \partial y }{ \partial u }    &   \frac{ \partial y }{ \partial v }    
    \end{pmatrix}=
    \begin{pmatrix}
        \frac{ 1 }{2}    &   \frac{ 1 }{2}    \\ 
        \frac{ 1 }{2}    &   -\frac{ 1 }{2}    
    \end{pmatrix}.
\end{equation}
Le déterminant vaut $-\frac{1}{ 2 }$. Nous avons donc
\begin{equation}
    dxdy=\frac{ 1 }{2}dudv.
\end{equation}
Nous insistons sur le fait que c'est $\frac{ 1 }{2}$ et non $-\frac{ 1 }{2}$ qui intervient parce que que la formule du changement de variable demande d'introduire la \emph{valeur absolue} du jacobien.

\begin{example}
    Calculer l'intégrale de la fonction $f(x,y)=x^2-y^2$ sur le domaine représenté sur la figure \ref{LabelFigVWFLooPSrOqz}. % From file VWFLooPSrOqz
\newcommand{\CaptionFigVWFLooPSrOqz}{Un domaine qui s'écrit étonnament bien avec un bon changement de coordonnées.}
\input{Fig_VWFLooPSrOqz.pstricks}
    
    Les droites qui délimitent le domaine d'intégration sont
    \begin{equation}
        \begin{aligned}[]
            y&=-x+2\\
            y&=x-2\\
            y&=x\\
            y&=-x
        \end{aligned}
    \end{equation}
    Le domaine est donc donné par les équations
    \begin{subequations}
        \begin{numcases}{}
            y+x<2\\
            y-x>-2\\
            y-x<0 \\
            y+x>0.
        \end{numcases}
    \end{subequations}
    En utilisant le changement de variables $u=x+y$, $v=x-y$ nous trouvons le domaine $0<u<2$, $0<v<2$. En ce qui concerne la fonction, $f(x,y)=(x+y)(x-y)$ et par conséquent
    \begin{equation}
        f(u,v)=uv.
    \end{equation}
    L'intégrale à calculer est simplement
    \begin{equation}
        \int_0^2\int_0^2 uv\,dudv=\int_0^2 u\,du\left[ \frac{ v^2 }{ 2 } \right]_0^2=2\int_0^2u\,du=4.
    \end{equation}
    
\end{example}




%+++++++++++++++++++++++++++++++++++++++++++++++++++++++++++++++++++++++++++++++++++++++++++++++++++++++++++++++++++++++++++
\section{Les intégrales triples}
%+++++++++++++++++++++++++++++++++++++++++++++++++++++++++++++++++++++++++++++++++++++++++++++++++++++++++++++++++++++++++++

Les intégrales triples fonctionnent exactement de la même manière que les intégrales doubles. Il s'agit de déterminer sur quelle domaine les variables varient et d'intégrer successivement par rapport à $x$, $y$ et $z$. Il est autorisé de permuter l'ordre d'intégration\footnote{En toute rigueur, cela n'est pas vrai, mais nous ne considérons seulement des cas où cela est autorisé.} à condition d'adapter les domaines d'intégration. 

\begin{example}
    Soit le domaine parallélépipédique rectangle 
    \begin{equation}
        R=\mathopen[ 0 , 1 \mathclose]\times \mathopen[ 1 , 2 \mathclose]\times\mathopen[ 0 , 4 \mathclose].
    \end{equation}
    Pour intégrer la fonction $f(x,y,z)=x^2y\sin(z)$ sur $R$, nous faisons
    \begin{equation}
        \begin{aligned}[]
            I&=\iiint_Rx^2y\sin(z)\,dxdydz\\
            &=\int_0^1dx\int_1^2dy\int_0^4x^2y\sin(z)dz\\
            &=\int_0^1dx\int_1^2 x^2y(1-\cos(4))dy\\
            &=\int_0^1\frac{ 3 }{2}(1-\cos(4))x^2dx\\
            &=\frac{ 1 }{2}\big( 1-\cos(4) \big).
        \end{aligned}
    \end{equation}
    
    \begin{verbatim}
----------------------------------------------------------------------
| Sage Version 4.6.1, Release Date: 2011-01-11                       |
| Type notebook() for the GUI, and license() for information.        |
----------------------------------------------------------------------
sage: f(x,y,z)=x**2*y*sin(z)                                                                                                                                                            
sage: f.integrate(x,0,1).integrate(y,1,2).integrate(z,0,4)                                                                                                                               
(x, y, z) |--> -1/2*cos(4) + 1/2
    \end{verbatim}
\end{example}


\begin{example}
    Soit $D$ la région délimitée par le plan $x=0$, $y=0$, $z=2$ et la surface d'équation
    \begin{equation}
        z=x^2+y^2.
    \end{equation}
    Cherchons à calculer $\iiint_Dx\,dx\,dy\,dz$. Ici, un dessin indique que le volume considéré est $z\geq x^2+y^2$. Il y a plusieurs façon de décrire cet ensemble. Une est celle-ci :
    \begin{equation}
        \begin{aligned}[]
            z&\colon 0\to 2\\
            x&\colon 0\to \sqrt{z}\\
            y&\colon 0\to \sqrt{z-x^2}.
        \end{aligned}
    \end{equation}
    Cela revient à dire que $z$ peut prendre toutes les valeurs de $0$ à $2$, puis que pour chaque $z$, la variable $x$ peut aller de $0$ à $\sqrt{z}$, mais que pour chaque $z$ et $x$ fixés, la variable $y$ ne peut pas dépasser $\sqrt{z-x^2}$. En suivant cette méthode, l'intégrale à calculer est
    \begin{equation}
        \int_0^2dz\int_0^{\sqrt{z}}dx\int_0^{\sqrt{z-x^2}}f(x,y,z)dy.
    \end{equation}
    \begin{verbatim}
----------------------------------------------------------------------
| Sage Version 4.6.1, Release Date: 2011-01-11                       |
| Type notebook() for the GUI, and license() for information.        |
----------------------------------------------------------------------
sage: f(x,y,z)=x
sage: assume(z>0)
sage: assume(z-x**2>0)
sage: f.integrate(y,0,sqrt(z-x**2)).integrate(x,0,sqrt(z)).integrate(z,0,2)
(x, y, z) |--> 8/15*sqrt(2)
    \end{verbatim}
    Notez qu'il a fallu aider Sage en lui indiquant que $z>0$ et $z-x^2>0$.

    Une autre paramétrisation serait
    \begin{equation}
        \begin{aligned}[]
            x&\colon 0\to \sqrt{2}\\
            y&\colon 0\to \sqrt{2-x^2}\\
            z&\colon x^2+y^2\to 2.
        \end{aligned}
    \end{equation}
    \begin{verbatim}
----------------------------------------------------------------------
| Sage Version 4.6.1, Release Date: 2011-01-11                       |
| Type notebook() for the GUI, and license() for information.        |
----------------------------------------------------------------------
sage: f(x,y,z)=x
sage: assume(2-x**2>0)
sage: f.integrate(y,0,sqrt(z-x**2)).integrate(x,0,sqrt(z)).integrate(z,0,2)
(x, y, z) |--> 8/15*sqrt(2)
    \end{verbatim}

    Écrivons le détail de cette dernière intégrale :
    \begin{equation}
        \begin{aligned}[]
            I&=\int_0^{\sqrt{2}}dx\int_0^{\sqrt{2-x^2}}dy\int_{x^2+y^2}^2xdz\\
            &=\int_0^{\sqrt{2}}dx\int_0^{\sqrt{2-x^2}}x(2-x^2-y^2)dy\\
            &=\int_0^{\sqrt{2}}dx\,x\left[ (2-x^2)y-\frac{ y^3 }{ 3 } \right]_0^{\sqrt{2-x^2}}\\
            &=\int_0^{\sqrt{2}}\frac{ 2 }{ 3 }x(2-x^2)^{3/2}dx.
        \end{aligned}
    \end{equation}
    Ici nous effectuons le changement de variable $u=x^2$, $du=2xdx$. Ne pas oublier de changer les bornes de l'intégrale :
    \begin{equation}
        I=\frac{1}{ 3 }\int_0^2(2-u)^{3/2}du.
    \end{equation}
    Le changement de variable $t=2-u$, $dt=-du$ fait venir (attention aux bornes !!)
    \begin{equation}
        I=-\frac{1}{ 3 }\int_2^0t^{3/2}dt=\frac{1}{ 3 }\left[ \frac{ t^{5/2} }{ 5/2 } \right]_0^2=\frac{ 8 }{ 15 }\sqrt{2}.
    \end{equation}
       
\end{example}

%---------------------------------------------------------------------------------------------------------------------------
\subsection{Volume}
%---------------------------------------------------------------------------------------------------------------------------

Parmi le nombreuses interprétations géométriques de l'intégrale triple, notons celle-ci :
\begin{proposition}
    Soit $D\subset \eR^3$. Le volume de $D$ est donné par 
    \begin{equation}
        Vol(D)=\iiint_D dxdydz.
    \end{equation}
    C'est à dire l'intégrale de la fonction $f(x,y,z)=1$ sur $D$.
\end{proposition}
Suivant les points de vue, cette proposition peut être considérée comme une \emph{définition}] du volume.

\begin{example}     \label{ExemVolSphCart}
    Calculons le volume de la sphère de rayon $R$. Le domaine de variation des variables $x$, $y$ et $z$ pour la sphère est
    \begin{equation}
        \begin{aligned}[]
            x&\colon -R\to R\\
            y&\colon -\sqrt{R^2-x^2}\to \sqrt{R^2-x^2}\\
            z&\colon -\sqrt{R^2-x^2-y^2}\to \sqrt{R^2-x^2-y^2}.
        \end{aligned}
    \end{equation}
    Par conséquent nous devons calculer l'intégrale
    \begin{equation}
        V=\int_{-R}^Rdx\int_{-\sqrt{R^2-x^2}}^{\sqrt{R^2-x^2}}dy\int_{-\sqrt{R^2-x^2-y^2}}^{\sqrt{R^2-x^2-y^2}}dz.
    \end{equation}
    La première intégrale est simple :
    \begin{equation}
        V=2\int_{-R}^Rdx\int_{-\sqrt{R^2-x^2}}^{\sqrt{R^2-x^2}}\sqrt{R^2-x^2-y^2}dy.
    \end{equation}
    Afin de simplifier la notation, nous posons $a=R^2-x^2$. Ceci n'est pas un changement de variables : juste une notation provisoire le temps d'effectuer l'intégration sur $y$. Étudions donc
    \begin{equation}
        I=\int_{-\sqrt{a}}^{\sqrt{a}}\sqrt{a-y^2}dy,
    \end{equation}
    ce qui est la surface du demi-disque de rayon $\sqrt{a}$. Nous avons donc
    \begin{equation}
        I=\frac{ \pi a }{ 2 }=\frac{ \pi }{ 2 }(R^2-x^2),
    \end{equation}
    et
    \begin{equation}
        V=2\int_{-R}^R\frac{ \pi }{ 2 }(R^2-x^2)dx=\pi\left[ R^2x-\frac{ x^3 }{ 3 } \right]_{-R}^R=\frac{ 4 }{ 3 }\pi R^3.
    \end{equation}    
\end{example}

\begin{example}
    Nous pouvons calculer le volume de la sphère en utilisant les coordonnées sphériques. Les bornes des variables pour la sphère de rayon $R$ sont
    \begin{equation}
        \begin{aligned}[]
            \rho&\colon 0\to R\\
            \theta&\colon 0\to \pi\\
            \varphi&\colon 0\to 2\pi.
        \end{aligned}
    \end{equation}
    En n'oubliant pas le jacobien $\rho^2\sin(\theta)$, l'intégrale à calculer est
    \begin{equation}
        V=\int_0^Rd\rho\int_0^{2\pi}d\varphi\int_0^{\pi}\rho^2\sin(\theta)d\theta
    \end{equation}
    L'intégrale sur $\varphi$ fait juste une multiplication par $2\pi$. Celle sur $\rho$ vaut
    \begin{equation}
        \int_0^R\rho^2d\rho=\frac{ R^3 }{ 3 }.
    \end{equation}
    L'intégrale sur $\theta$ donne
    \begin{equation}
        \int_0^{\pi}\sin(\theta)d\theta=[-\cos(\theta)]_{0}^{\pi}=2.
    \end{equation}
    Le tout fait par conséquent
    \begin{equation}
        V=\frac{ 4 }{ 3 }\pi R^3.
    \end{equation}
    Sans contestes, le passage aux coordonnées sphériques a considérablement simplifié le calcul par rapport à celui de l'exemple \ref{ExemVolSphCart}.
\end{example}


%+++++++++++++++++++++++++++++++++++++++++++++++++++++++++++++++++++++++++++++++++++++++++++++++++++++++++++++++++++++++++++
\section{Un petit peu plus formel}
%+++++++++++++++++++++++++++++++++++++++++++++++++++++++++++++++++++++++++++++++++++++++++++++++++++++++++++++++++++++++++++

%---------------------------------------------------------------------------------------------------------------------------
\subsection{Intégration sur un domaine non rectangulaire}
%---------------------------------------------------------------------------------------------------------------------------

\newcommand{\CaptionFigPONXooXYjEot}{Intégrer sur des domaines plus complexes.}
\input{Fig_PONXooXYjEot.pstricks}

La méthode de Fubini ne fonctionne plus sur un domaine non rectangulaire tel que celui de la figure \ref{LabelFigPONXooXYjEot}. Nous allons donc utiliser une astuce. Considérons le domaine \begin{equation}
	E=\{ (x,y)\in\eR^2\tq a<x<b\text{ et } \alpha(x)<y<\beta(x) \}
\end{equation}
représenté sur la figure \ref{LabelFigPONXooXYjEot}. Nous considérons la fonction
\begin{equation}
	\tilde f(x,y)=\begin{cases}
	f(x,y)	&	\text{si $(x,y)\in E$}\\
	0	&	 \text{sinon.}
\end{cases}
\end{equation}
Ensuite intégrons $\tilde f$ sur un rectangle qui englobe la surface à intégrer à l'aide de Fubini. Étant donné que $\tilde f=f$ sur la surface et que $\tilde f$ est nulle en dehors, nous avons
\begin{equation}
	\int_Ef=\int_E\tilde f=\int_{\text{rectangle}}\tilde f=\int_a^b\left( \int_{\alpha(x)}^{\beta(x)}f(x,y)dy \right)dx.
\end{equation}

Dans le cas de l'intégrale de $f(x,y)=x^2+y^2$ sur le triangle de la figure \ref{LabelFigCURGooXvruWV}, nous avons
\begin{equation}
	\int_{\text{triangle}}(x^2+y^2)dx dy=\int_0^2\left( \int_0^y(x^2+y^2)dx \right)dy.
\end{equation}

\begin{remark}
    Le nombre $\iint_{D}f(x,y)dxdy$ ne dépend pas du choix du rectangle englobant $D$.
\end{remark}

En pratique, nous calculons l'intégrale en utilisant une extension du théorème de Fubini :
\begin{theorem}
    Soit $f\colon D\subset\eR^2\to \eR$ une fonction continue où $D$ est un domaine de type vertical ou horizontal.
    \begin{enumerate}
        \item
            Si $D$ est vertical, alors
            \begin{equation}
                \iint_Df=\int_a^b\left[ \int_{\varphi_1(x)}^{\varphi_2(x)}f(x,y)dy \right]dx.
            \end{equation}
        \item
            Si $D$ est horizontal, alors
            \begin{equation}
                \iint_Df=\int_c^d\left[ \int_{\psi_1(y)}^{\psi_2(y)}f(x,y)dx \right]dy.
            \end{equation}
    \end{enumerate}
    
\end{theorem}

%---------------------------------------------------------------------------------------------------------------------------
\subsection{Changement de variables}
%---------------------------------------------------------------------------------------------------------------------------


Le théorème du changement de variable est le suivant.
\begin{theorem}
Soit $g\colon A\to B$ un difféomorphisme. Soient $F\subset B$ un ensemble mesurable et borné et $f\colon F\to \eR$ une fonction bornée et intégrable. Supposons que $g^{-1}(F)$ soit borné et que $Jg$ soit borné sur $g^{-1}(F)$. Alors
\begin{equation}
    \int_Ff(x)dy=\int_{g^{-1}(F)}f\big( g(x) \big)| Jg(x) |dx
\end{equation}
\end{theorem}
Pour rappel, $Jg$ est le déterminant de la matrice \href{http://fr.wikipedia.org/wiki/Matrice_jacobienne}{jacobienne} (aucun lien de \href{http://fr.wikipedia.org/wiki/Jacob}{parenté}) donnée par
\begin{equation}
	Jg=\det\begin{pmatrix}
	\partial_xg_1	&	\partial_yg_1	\\ 
	\partial_xg_2	&	\partial_tg_2	
\end{pmatrix}.
\end{equation}
Un \defe{difféomorphisme}{difféomorphisme} est une application $g\colon A\to B$ telle que $g$ et $g^{-1}\colon B\to A$ soient de classe $C^1$.

%///////////////////////////////////////////////////////////////////////////////////////////////////////////////////////////
					\subsubsection{Coordonnées polaires}
%///////////////////////////////////////////////////////////////////////////////////////////////////////////////////////////

Les coordonnées polaires sont données par le difféomorphisme
\begin{equation}
	\begin{aligned}
		g\colon \mathopen]0,\infty\mathclose[\times\mathopen]0,2\pi\mathclose[ &\to\eR^2\setminus D\\
		(r,\theta)&\mapsto \big( r\cos(\theta),r\sin(\theta) \big)
	\end{aligned}
\end{equation}
où $D$ est la demi droite $y=0$, $x\geq 0$. Le fait que les coordonnées polaires ne soient pas un difféomorphisme sur tout $\eR^2$ n'est pas un problème pour l'intégration parce que le manque de difféomorphisme est de mesure nulle dans $\eR^2$. Le jacobien est donné par
\begin{equation}
	Jg=\det\begin{pmatrix}
	\partial_rx	&	\partial_{\theta}x	\\ 
	\partial_ry	&	\partial_{\theta}y
\end{pmatrix}=\det\begin{pmatrix}
	\cos(\theta)	&	-r\sin(\theta)	\\ 
	\sin(\theta)	&	r\cos(\theta)	
\end{pmatrix}=r.
\end{equation}

%///////////////////////////////////////////////////////////////////////////////////////////////////////////////////////////
					\subsubsection{Coordonnées sphériques}
%///////////////////////////////////////////////////////////////////////////////////////////////////////////////////////////
\label{SubSubCoordSpJxhMwm}

Les coordonnées sphériques sont données par
\begin{equation}		\label{OMEqChmVarSpherique}
	\left\{
\begin{array}{lllll}
x=r\cos\theta\sin\varphi	&			&r\in\mathopen] 0 , \infty \mathclose[\\
y=r\sin\theta\sin\varphi	&	\text{avec}	&\theta\in\mathopen] 0 , 2\pi \mathclose[\\
z=r\cos\varphi			&			&\phi\in\mathopen] 0 , \pi \mathclose[.
\end{array}
\right.
\end{equation}
Le jacobien associé est $Jg(r,\theta,\varphi)=-r^2\sin\varphi$. Rappelons que ce qui rentre dans l'intégrale est la valeur absolue du jacobien.

Si nous voulons calculer le volume de la sphère de rayon $R$, nous écrivons donc
\begin{equation}
	\int_0^Rdr\int_{0}^{2\pi}d\theta\int_0^{\pi}r^2 \sin(\phi)d\phi=4\pi R=\frac{ 4 }{ 3 }\pi R^3.
\end{equation}
Ici, la valeur absolue n'est pas importante parce que lorsque $\phi\in\mathopen] 0,\pi ,  \mathclose[$, le sinus de $\phi$ est positif.

Des petits malins pourraient remarquer que le changement de variable \eqref{OMEqChmVarSpherique} est encore une paramétrisation de $\eR^3$ si on intervertit le domaine des angles : 
\begin{equation}
	\begin{aligned}[]
		\theta&\colon 0 \to \pi\\
		\phi	&\colon 0\to 2\pi,
	\end{aligned}
\end{equation}
alors nous paramétrons encore parfaitement bien la sphère, mais hélas
\begin{equation}		\label{EqOMVolumeIncorrectSphere}
	\int_0^Rdr\int_{0}^{\pi}d\theta\int_0^{2\pi}r^2 \sin(\phi)d\phi=0.
\end{equation}
Pourquoi ces «nouvelles» coordonnées sphériques sont-elles mauvaises ? Il y a que quand l'angle $\phi$ parcours $\mathopen] 0 , 2\pi \mathclose[$, son sinus n'est plus toujours positif, donc la \emph{valeur absolue} du jacobien n'est plus $r^2\sin(\phi)$, mais $r^2\sin(\phi)$ pour les $\phi$ entre $0$ et $\pi$, puis $-r^2\sin(\phi)$ pour $\phi$ entre $\pi$ et $2\pi$. Donc l'intégrale \eqref{EqOMVolumeIncorrectSphere} n'est pas correcte. Il faut la remplacer par
\begin{equation}
	\int_0^Rdr\int_{0}^{\pi}d\theta\int_0^{\pi}r^2 \sin(\phi)d\phi- \int_0^Rdr\int_{0}^{\pi}d\theta\int_{\pi}^{2\pi}r^2 \sin(\phi)d\phi = \frac{ 4 }{ 3 }\pi R^3
\end{equation}

%+++++++++++++++++++++++++++++++++++++++++++++++++++++++++++++++++++++++++++++++++++++++++++++++++++++++++++++++++++++++++++
					\section{Primitives et surfaces}
%+++++++++++++++++++++++++++++++++++++++++++++++++++++++++++++++++++++++++++++++++++++++++++++++++++++++++++++++++++++++++++

Soit $f\colon \eR\to \eR$, une fonction continue, et $x\in\eR$. Pour chaque $x\in\eR$, nous pouvons considérer le nombre $F(x)$ défini par
\begin{equation}
	F(x)=\int_a^x f(t)dt.
\end{equation}

%The result is on figure \ref{LabelFigVSZRooRWgUGu}. % From file VSZRooRWgUGu
\newcommand{\CaptionFigVSZRooRWgUGu}{Surface sous une courbe}
\input{Fig_VSZRooRWgUGu.pstricks}

La fonction $F$ ainsi définie a deux importantes propriétés :
\begin{enumerate}

\item
C'est une primitive de $f$,
\item
Elle donne la surface en dessous de $f$ entre les points $a$ et $x$, voir la figure \ref{LabelFigVSZRooRWgUGu}.

\end{enumerate}

Notons que tant que $f$ est positive, la surface est croissante.

La manière de calculer la surface comprise entre deux fonctions est dessinée à la figure \ref{LabelFigQOBAooZZZOrl}. % From file QOBAooZZZOrl
\newcommand{\CaptionFigQOBAooZZZOrl}{Le calcul de la surface comprise entre deux fonctions.}
\input{Fig_QOBAooZZZOrl.pstricks}
%See also the subfigure \ref{LabelFigQOBAooZZZOrlssLabelSubFigQOBAooZZZOrl0}
%See also the subfigure \ref{LabelFigQOBAooZZZOrlssLabelSubFigQOBAooZZZOrl1}
%See also the subfigure \ref{LabelFigQOBAooZZZOrlssLabelSubFigQOBAooZZZOrl2}

La surface entre les deux fonctions $y_1(x)$ et $y_2(x)$ se calcule comme suit.
\begin{enumerate}

\item
On calcule les intersections entre $y1$ et $y_2$. Notons $a$ et $b$ les ordonnées obtenues.
\item
La surface demandée est la différence entre la surface sous la fonction $y_1$ (la plus grande) et la surface sous la fonction $y_2$ (la plus petite), donc
\begin{equation}
	S=\int_{a}^by_1-\int_a^by_1.
\end{equation}

\end{enumerate}

%---------------------------------------------------------------------------------------------------------------------------
					\subsection{Longueur d'arc de courbe}
%---------------------------------------------------------------------------------------------------------------------------

La longueur de l'arc de courbe de la fonction $y=f(x)$ entre les abscisses $x_0$ et $x_1$ est donné par la formule
\begin{equation}		\label{EqLongArcCourbe}
	l(x_0,x_1)=\int_{x_0}^{x_1}\sqrt{1+y'(t)^2}dt.
\end{equation}

Lorsque la courbe est donnée sous forme paramétrique
\begin{subequations}
\begin{numcases}{}
	x=x(t)\\
	y=y(t),
\end{numcases}
\end{subequations}
alors la formule devient
\begin{equation}		\label{EqLongArcParam}
	l(t_1,t_2)=\int_{t_1}^{t_2}\sqrt{\dot x(t)^2+\dot y(t)^2}dt,
\end{equation}
où $\dot x(t)=x'(t)$.

%---------------------------------------------------------------------------------------------------------------------------
					\subsection{Aire de révolution}
%---------------------------------------------------------------------------------------------------------------------------

Pour savoir l'aire engendrée par la ligne $y=f(x)$ entre $a$ et $b$ autour de l'axe $Ox$, on utilise la formule
\begin{equation}
	S=2\pi\int_a^b\sqrt{1+f'(x)^2}f(x)dx.
\end{equation}

\section{Fonctions réelles de deux variables réelles}

Une \textbf{fonction réelle de 2 variables réelles} est une fonction $f : A \subset \eR^2 \to \eR : (x,y) \mapsto z = f(x,y)$.

Le \textbf{graphe de $f$}, noté $\Graphe f$, est un sous-ensemble de $\eR^3$:\[\Graphe f = \{(x,y,z) \in \eR^3 \mid (x,y) \in A \text{ et } z = f(x,y)\}\]

Les \textbf{courbes de niveau} de la fonction $f$ sont obtenues en posant $f(x,y)=\lambda$.

%---------------------------------------------------------------------------------------------------------------------------
\subsection{Limites de fonctions à deux variables}
%---------------------------------------------------------------------------------------------------------------------------

Ici nous n'allons pas entrer dans tous les détails, mais simplement mentionner les quelque techniques les plus courantes. 

\begin{theorem}		\label{ThoLimiteCompose}
	Soient deux fonctions $f\colon \eR^n\to \eR^p$ et $g\colon \eR^p\to \eR^q$. Si $a$ est un point adhérent au domaine de $g\circ f$ et si
	\begin{equation}
		\begin{aligned}[]
			\lim_{x\to a}f(x)&=b\\
			\lim_{y\to b}g(y)&=c,
		\end{aligned}
	\end{equation}
	alors 
	\begin{equation}
		\lim_{x\to a}(g\circ f)(x)=c.
	\end{equation}
\end{theorem}

Les techniques usuelles sont
\begin{enumerate}

	\item
		La règle de l'étau. Cette technique demande un peu plus d'imagination parce qu'il faut penser à un «truc» différent pour chaque exercice. En revanche, la justification est facile : il y a un théorème qui dit que ça marche.

	\item
		Lorsqu'on applique la règle de l'étau, penser à
		\begin{equation}
			| x |=\sqrt{x^2}\leq\sqrt{x^2+y^2}.
		\end{equation}
		Cela permet de majorer le numérateur. Attention : ce genre de majoration ne fonctionnent qu'au numérateur : agrandir le dénominateur ferait diminuer la fraction.

	\item
		Il n'est pas vrai que
		\begin{equation}
			| x |=\sqrt{x^2}\leq\sqrt{x^4}\leq\sqrt{x^4+2y^4}.
		\end{equation}
		En effet, si $x$ est petit, alors $x^2>x^4$, et non le contraire.

\end{enumerate}

Une technique très efficace pour les limites $(x,y)\to (0,0)$ est le passage aux coordonnées polaires. Il s'agit de poser
\begin{subequations}
	\begin{numcases}{}
		x=r\cos(\theta)\\
		y=r\sin(\theta)
	\end{numcases}
\end{subequations}
et puis de faire la limite $r\to 0$.

Si la limite obtenue {\bf ne dépend pas de $\theta$}, alors c'est la limite cherchée. L'exercice suivant en donne des exemples.
\Exo{mazhe-0000}

%---------------------------------------------------------------------------------------------------------------------------
\subsection{Dérivées partielles}
%---------------------------------------------------------------------------------------------------------------------------

La \defe{dérivée partielle}{dérivée partielle} par rapport à $x$ au point $(x,y)$ est notée
\begin{equation}
	\frac{\partial f}{\partial x}(x,y) 
\end{equation}
et se calcule en dérivant $f$ par rapport  à $x$ en considérant que $y$ est constante.

De la même manière, la dérivée partielle par rapport à $y$ au point $(x,y)$ est notée
\begin{equation}
	\frac{\partial f}{\partial y}(x,y) 
\end{equation}
et se calcule en dérivant $f$ par rapport  à $y$ en considérant que $x$ est constante.

Pour les dérivées partielles secondes,
\begin{itemize}
\item $f''_{xx} (x,y) = (f'_x)'_x = \frac{\partial^2 f}{\partial x^2}(x,y) = \frac{\partial}{\partial x}(\frac{\partial f}{\partial x})$.
\item $f''_{yy} (x,y) = (f'_y)'_y = \frac{\partial^2 f}{\partial y^2}(x,y) = \frac{\partial}{\partial y}(\frac{\partial f}{\partial y})$.
\item $f''_{xy} (x,y) = (f'_x)'_y  = (f'_y)'_x = f''_{yx} (x,y) \text{ ou } \frac{\partial^2 f}{\partial x \partial y}(x,y) = \frac{\partial}{\partial x}(\frac{\partial f}{\partial y})  = \frac{\partial}{\partial y}(\frac{\partial f}{\partial x}) =\frac{\partial^2 f}{\partial y \partial x}(x,y)$.
\end{itemize}

%---------------------------------------------------------------------------------------------------------------------------
\subsection{Différentielle et accroissement}
%---------------------------------------------------------------------------------------------------------------------------

La \defe{différentielle totale}{différentielle!totale} de $f$ au point $(a,b)$ est donnée, quand elle existe (!), par la formule
\begin{equation}
	df(a,b) = \frac{\partial f}{\partial x}(a,b)dx + \frac{\partial f}{\partial y}(a,b) dy.
\end{equation}

De la même façon que la formule des accroissements finis disait que $f(x+a)\simeq f(x)+af'(x)$, en deux dimensions nous avons que l'\defe{accroissement}{accroissement} approximatif de $f$ au point $(a,b)$ pour des accroissements $\Delta x$ et $\Delta y$ est 
\begin{equation}
	f(x+\Delta x,y+\Delta y)=f(x,y)+\Delta x\frac{ \partial f }{ \partial x }(x,y)+\Delta y\frac{ \partial f }{ \partial y }(x,y).
\end{equation}

%TODO : pour l'index, l'expression régulière suivante aide :
% grep "defe{[A-Za-z ]*}{[A-Z]" *.tex
Le \defe{plan tangent}{plan!tangent} au graphe de $f$ au point $\big(a,b,f(a,b)\big)$ est 
\begin{equation}
	T_{(a,b)}(x,y) = f(a,b) + \frac{\partial f}{\partial x}(a,b) (x-a) + \frac{\partial f}{\partial y}(a,b) (y-b)
\end{equation}
essayez d'écrire l'équation de la droite tangente au graphe de $f(x)$ au point $x=a$ en terme de la dérivée de $f$, et comparez votre résultat à cette formule.

Un des principaux théorèmes pour tester la différentiabilité d'une fonction est le suivant.

\begin{theorem}		\label{ThoProuverDiffable}
	Soit une fonction $f\colon \eR^m\to \eR^p$. Si les dérivées partielles existent dans un voisinage de $a$ et donc continues en $a$, alors $f$ est différentiable en $a$.
\end{theorem}
Le plus souvent, nous prouvons qu'une fonction est différentiable en calculant les dérivées partielles et en montrant qu'elles sont continues.

%---------------------------------------------------------------------------------------------------------------------------
\subsection{Recherche d'extrema locaux}
%---------------------------------------------------------------------------------------------------------------------------

\begin{enumerate}
\item Rechercher les points critiques, càd les $(x,y)$ tels que
\[\begin{cases} \frac{\partial f}{\partial x}(x,y) = 0 \\ \frac{\partial f}{\partial y}(x,y) = 0 \end{cases} \]
En effet, si $(x_0,y_0)$ est un extrémum local de $f$, alors $\frac{\partial f}{\partial x}(x_0,y_0) = 0 = \frac{\partial f}{\partial y}(x_0,y_0)$.
\item Déterminer la nature des points critiques: «test» des dérivées secondes:
\[\text{On pose }H(x_0,y_0) = \frac{\partial^2 f}{\partial x^2}(x_0,y_0)\frac{\partial f^2}{\partial y^2}(x_0,y_0) - \left(\frac{\partial^2 f}{\partial x\partial y}(x_0,y_0)\right)^2\]
\begin{enumerate}
\item Si $H(x_0,y_0) > 0$ et $\frac{\partial^2 f}{\partial x^2}(x_0,y_0) > 0 \Longrightarrow (x_0,y_0)$ est un minimum local de $f$.
\item Si $H(x_0,y_0) > 0$ et $\frac{\partial^2 f}{\partial x^2}(x_0,y_0) < 0 \Longrightarrow (x_0,y_0)$ est un maximum local de $f$.
\item Si $H(x_0,y_0) < 0 \Longrightarrow f$ a un point de selle en $(x_0,y_0)$.
\item Si $H(x_0,y_0) = 0 \Longrightarrow$ on ne peut rien conclure.
\end{enumerate}
\end{enumerate}

\textbf{Dérivation implicite:} Soit $F(x,f(x)) = 0$ la représentation implicite d'une fonction $y=f(x)$ alors \[y' = f'(x) = - \frac{F'_x}{F'_y}.\]

%+++++++++++++++++++++++++++++++++++++++++++++++++++++++++++++++++++++++++++++++++++++++++++++++++++++++++++++++++++++++++++
\section{Méthode de Gauss pour résoudre des systèmes d'équations linéaires}
%+++++++++++++++++++++++++++++++++++++++++++++++++++++++++++++++++++++++++++++++++++++++++++++++++++++++++++++++++++++++++++


Pour résoudre un système d'équations linéaires, on procède comme suit:
\begin{enumerate}
\item Écrire le système sous forme matricielle. \[\text{p.ex. } \begin{cases} 2x+3y &= 5 \\ x+2y &= 4 \end{cases} \Leftrightarrow \left(\begin{array}{cc|c} 2 & 3 & 5 \\ 1 & 2 & 4 \end{array}\right) \]
\item Se ramener à une matrice avec un maximum de $0$ dans la partie de gauche en utilisant les transformations admissibles:
\begin{enumerate}
\item Remplacer une ligne par elle-même + un multiple d'une autre;
\[\text{p.ex. } \left(\begin{array}{cc|c} 2 & 3 & 5 \\ 1 & 2 & 4 \end{array}\right)  \stackrel{L_1  - 2. L_2 \mapsto L_1'}{\Longrightarrow} \left(\begin{array}{cc|c} 0 & -1 & -3 \\ 1 & 2 & 4 \end{array}\right) \]
\item Remplacer une ligne par un multiple d'elle-même;
\[\text{p.ex. } \left(\begin{array}{cc|c} 0 & -1 & -3 \\ 1 & 2 & 4 \end{array}\right)  \stackrel{-L_1  \mapsto L_1'}{\Longrightarrow} \left(\begin{array}{cc|c} 0 & 1 & 3 \\ 1 & 2 & 4 \end{array}\right) \]
\item Permuter des lignes.
\[\text{p.ex. } \left(\begin{array}{cc|c} 0 & 1 & 3 \\ 1 & 0 & -2 \end{array}\right)  \stackrel{L_1  \mapsto L_2' \text{ et } L_2  \mapsto L_1'}{\Longrightarrow} \left(\begin{array}{cc|c} 1 & 0 & -2 \\ 0 & 1 & 3  \end{array}\right) \]
\end{enumerate}
\item Retransformer la matrice obtenue en système d'équations.
\[\text{p.ex. }  \left(\begin{array}{cc|c} 1 & 0 & -2 \\ 0 & 1 & 3  \end{array}\right) \Leftrightarrow \begin{cases} x &= -2 \\ y &= 3 \end{cases}  \]
\end{enumerate}

\begin{remark}
\begin{itemize}
\item Si on obtient une ligne de zéros, on peut l'enlever:
\[\text{p.ex. }  \left(\begin{array}{ccc|c} 3 & 4 & -2 & 2 \\ 4 & -1 & 3 & 0 \\ 0 & 0 & 0 & 0 \end{array}\right) \Leftrightarrow  \left(\begin{array}{ccc|c} 3 & 4 & -2 & 2 \\ 4 & -1 & 3 & 0 \end{array}\right) \]
\item Si on obtient une ligne de zéros suivie d'un nombre non-nul, le système d'équations n'a pas de solution:
\[\text{p.ex. }  \left(\begin{array}{ccc|c} 3 & 4 & -2 & 2 \\ 4 & -1 & 3 & 0 \\ 0 & 0 & 0 & 7 \end{array}\right) \Leftrightarrow  \begin{cases} \cdots \\ \cdots \\ 0x + 0y + 0z = 7 \end{cases} \Rightarrow \textbf{Impossible} \]
\item Si on moins d'équations que d'inconnues, alors il y a une infinité de solutions qui dépendent d'un ou plusieurs paramètres:
\[\text{p.ex. }  \left(\begin{array}{ccc|c} 1 & 0 & -2 & 2 \\ 0 & 1 & 3 & 0 \end{array}\right) \Leftrightarrow  \begin{cases} x - 2z = 2 \\ y + 3z = 0 \end{cases} \Leftrightarrow  \begin{cases} x = 2 + 2\lambda \\ y = -3\lambda \\ z = \lambda \end{cases} \]
\end{itemize}
\end{remark}
