% This is part of Exercices et corrigés de CdI-1
% Copyright (c) 2011,2014
%   Laurent Claessens
% See the file fdl-1.3.txt for copying conditions.

\begin{corrige}{0079}


Soit $E = \{ (x,y,z) \tq x^2 \leq 3\}$. Prouvons que $E$ est
fermé, car si $(p_k)$ est une suite de points ($p_k = (x_k,y_k,z_k)$)
de $E$ (c'est à dire $(x_k)^2 \leq 3$) convergente dans $\eR^3$ vers une limite
$p = (x,y,z)$, alors en particulier la suite $x_k$ tend vers
$x$. Comme $(x_k)^2 \leq 3$, on peut passer à la limite dans
l'inégalité et donc $x^2 \leq 3$, ce qui prouve que $p = (x,y,z)$ est
dans $E$, et donc $\adh E \subset E$. Comme par ailleurs $E \subset
\adh E$, on a bien l'égalité.

\begin{remark}
  De manière plus générale, si $f : \eR^n \to \eR^m$ est continue,
  et si $F \subset \eR^m$ est un fermé de $\eR^m$, alors l'ensemble
  \begin{equation*}
    f^{-1}(F) \pardef \{ p \in\eR^n \tq f(p) \in F \}
  \end{equation*}
  est un ensemble fermé dans $\eR^n$. Par ailleurs, si on suppose $F$
  ouvert au lieu de fermé, alors $f^{-1}(F)$ est ouvert.

  Dans ce cas ci, $f : \eR^3 \to \eR : (x,y,z) \mapsto x^2$ est
  continue, et $F = \mathopen[3,+\infty\mathclose[$ est
  fermé. L'ensemble $E$ est bien l'ensemble des points $p \in \eR^3$
  tels que $f(p) \in F$.
\end{remark}

Prouvons que $\interieur E = \{ (x,y,z) \tq x^2 < 3\}$. Tout
d'abord, par la remarque ci-dessus, on observe que $\{ (x,y,z) \tq
x^2 < 3\}$ est ouvert (image réciproque de l'ouvert
$\mathopen]3,+\infty\mathclose]$) et donc est inclus dans l'intérieur
de $E$. Par ailleurs, les points $(x,y,z)$ vérifiant $x^2 = 3$ ne sont
pas dans l'intérieur~: si $B((3,y,z),\epsilon)$ désigne la boule de
centre $(3,y,z)$ et de rayon $\epsilon$, le point
$(3+\sfrac\epsilon2,y,z)$ appartient à cette boule mais n'est pas dans
$E$ ; donc aucune boule centrée en $(3,y,z)$ ne peut être contenue
entièrement dans $E$. Ce qui prouve le résultat.

Prouvons que $E$ est connexe par arcs. Si $(x,y,z)$ et $(a,b,c)$ sont
des points de $E$, alors le chemin
\begin{equation*}
  \gamma : [0,1] \to \eR^3 : t \mapsto t (x,y,z) + (1-t) (a,b,c)
\end{equation*}
est continu (il paramétrise le segment de droite joignant $(a,b,c)$ à
$(x,y,z)$). La condition pour être dans $E$ s'écrit $\abs x \leq 3$ et
$\abs a \leq 3$. Par l'inégalité triangulaire, on sait que
\begin{equation*}
  \abs{tx+(1-t)a} \leq \abs t \abs x + \abs{1-t} \abs a \leq t \sqrt3
  + (1-t) \sqrt 3 = \sqrt 3
\end{equation*}
où on a utilisé le fait que $t \geq 0$ et $1-t \geq 0$ (car $t \in
[0,1]$). Ceci prouve que $\gamma(t) \in E$ pour tout $t \in [0,1]$, et
on a donc bien un chemin continu reliant $(x,y,z)$ et $(a,b,c)$
contenu dans $E$.

\end{corrige}
