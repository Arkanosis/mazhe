\begin{corrige}{DS2011-0004}
  
    En termes du paramètre \( \theta\), la courbe s'écrit
    \begin{equation}
        \gamma(\theta)=\begin{pmatrix}
            r(\theta)\cos(\theta)    \\ 
            r(\theta)\sin(\theta)    
        \end{pmatrix}=
        \begin{pmatrix}
            2\cos^2(\theta)    \\ 
            2\cos(\theta)\sin(\theta)    
        \end{pmatrix}.
    \end{equation}
    Étant donné que sur \( \theta\in\mathopen[ 0 , \pi/2 \mathclose]\) la fonction \( 2\cos^2(\theta)\) est injective, il est possible d'utiliser \( x x\) comme paramètre. Nous avons \( x=2\cos^2(\theta)\), c'est à dire
    \begin{subequations}
        \begin{align}
            \cos(\theta)=\sqrt{\frac{ x }{2}}\\
            \sin(\theta)=\sqrt{1-\frac{ x }{2}}.
        \end{align}
    \end{subequations}
    Pour la seconde ligne, nous avons utilisé la formule \( \sin(\theta)=\sqrt{1-\cos^2(\theta)}\). Une fois de plus, cette formule se justifie par le fait que \( \theta\in\mathopen[ 0 , \pi/2 \mathclose]\) et que \( \sin(\theta)\) y est toujours positive. Nous avons au final
    \begin{equation}
        \gamma(x)=\begin{pmatrix}
            x    \\ 
            2\sqrt{\frac{ x }{2}\left( 1-\frac{ x }{2} \right)}.    
        \end{pmatrix}
    \end{equation}
    avec \( x\in\mathopen[ 0 , 2 \mathclose]\). Note : préciser l'intervalle de \( x \) n'est pas facultatif.

    De nombreux étudiants ont cru malin d'écrire
    \begin{equation}
        \theta=\arccos\sqrt{\frac{ x }{ 2 }}
    \end{equation}
    et
    \begin{equation}        \label{EqCartfgxygammaqnn}
        \gamma(x)=\begin{pmatrix}
            x    \\ 
            2\sqrt{\frac{ x }{2}}\sin\left( \arccos\sqrt{\frac{ x }{2}} \right)
        \end{pmatrix}.
    \end{equation}
    
   
    Quelque remarques à propos de cette idée :
    \begin{enumerate}
        \item
            Ce n'était pas la réponse attendue parce qu'elle est triviale. Si vous avez une fonction en termes du paramètre \( t\) donnée par
            \begin{equation}
                \gamma(t)=\begin{pmatrix}
                    f(t)    \\ 
                    g(t)    
                \end{pmatrix},
            \end{equation}
            vous pouvez toujours poser \( t=f^{-1}(t)\) et puis répondre
            \begin{equation}
                \gamma(x)=\begin{pmatrix}
                    x    \\ 
                    g\big( f^{-1}(x) \big)    
                \end{pmatrix}.
            \end{equation}
            La réponse en \( \arccos\) n'était pas attendue, mais elle reste néanmoins correcte. Soyons généreux. Pour un prochain (ou examen), essayez de ne pas recommencer. De notre côté, nous essayerons de formuler la question de façon à vous empêcher de répondre ainsi ;)

        \item
            Lorsque vous utiliser la fonction inverse d'une fonction non injective (par exemple cosinus), vous \emph{devez} préciser les domaines ! Dans le cas présent, vous devez considérer la fonction
            \begin{equation}
                \arccos\colon \mathopen[ 0 , 1 \mathclose]\to \mathopen[ 0 , \frac{ \pi }{2} \mathclose],
            \end{equation}
            et justifier que lorsque \( x\in\mathopen[ 0 , 2 \mathclose]\), la fonction \( \sqrt{\frac{ x }{ 2 }}\) reste entre \( 0\) et \( 1\).
    \end{enumerate}

    Pour comprendre de quel cercle il s'agit, nous pouvons partir de l'équation \eqref{EqCartfgxygammaqnn} et calculer \( r^2\) en sommant les carrés des deux composantes. Le résultat est que
    \begin{equation}
        x^2+y^2=x^2+4\frac{ x }{2}\left( 1-\frac{ x }{2} \right),
    \end{equation}
    et l'équation de notre cercle est \( x^2+y^2=2x\). Pour trouver le centre et le rayon, le mieux est de compléter le carré :
    \begin{equation}
        x^2-2x=(x-1)^2-1
    \end{equation}
    et par conséquent
    \begin{equation}
        (x-1)^2+y^2=1.
    \end{equation}
    Cela est l'équation d'un cercle de centre \( (1,0)\) et de rayon \( 1\). Étant donné que dans le contexte de cet exercice nous avons \( y\geq 0\)\footnote{Pouvez vous justifier cela en lisant l'énoncé ?}, nous n'avons en réalité que la moitié supérieure du cercle. Sa longueur vaut donc \( \pi\).

    Il y avait aussi moyen de calculer la longueur en utilisant une des équations paramétriques (polaires, ou cartésiennes) et en considérant la formule correspondante.

\end{corrige}
