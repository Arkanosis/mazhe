% This is part of Analyse Starter CTU
% Copyright (c) 2014
%   Laurent Claessens,Carlotta Donadello
% See the file fdl-1.3.txt for copying conditions.

\begin{corrige}{autoanalyseCTU-34bis}

La différence entre les deux points de cet exercice est que dans l'énoncé du  premier on a détaillé chaque passage de la résolution, alors que l'énoncé du deuxième est plus synthétique. En pratique, comme on verra, on demande essentiellement la m\^eme chose.  

\begin{enumerate}
\item 
    \begin{enumerate}
    \item L'équation homogène $(H)$ associée à $(E)$ sur $I$ est 
      \begin{equation*}
        y'-\dfrac{3}{x}y=0. 
      \end{equation*}
      Sa solution générale est donnée par la formule \eqref{solgeneqlinordre1}
      \begin{equation*}
        \mathcal{Y}_h=\left\{Ke^{\int\frac{3}{x}\, dx} = Kx^3\,:\, K\in\eR, \: x\in I \right\}.
      \end{equation*}
    \item La méthode de variation de la constante consiste à remplacer la constante $K$ dans $\mathcal{Y}_h$ par une fonction $x\mapsto K(x)$ à déterminer et ensuite injecter la fonction ``candidate'' solution $y_p = K(x)x^3$ dans $(E)$ à la place de l'inconnue $y$. Nous avons donc 
      \begin{equation*}
        K'(x)x^3 + 3 K(x) x^2 -\dfrac{3}{x} (K(x)x^3)  = x,
      \end{equation*}
      c'est à dire 
      \begin{equation*}
        K'(x)x^3  = x, \:\text{ ou encore } K'(x)  = \frac{1}{x^2}. 
      \end{equation*}
      L'ensemble des primitives de $1/x^2$ est $\mathcal{P} = -1/x + C$, avec $C\in \eR$ donc une  solution particulière $y_p$ est $-x^2$ (il suffit de prendre $C=0$). 
    \item L'ensemble des solutions de $(E)$ sur l'intervalle $I$ est la somme entre la solution générale $\mathcal{Y}_h$ et $y_p$.
      \begin{equation}\label{solgenexo34bis1}
        \mathcal{Y}=\left\{ -x^2 + Kx^3\,:\, C\in\eR, \: x\in I \right\}.
      \end{equation} 
      \begin{remark}
        On peut aussi écrire $ \mathcal{Y}$ comme le produit entre l'ensemble des primitives de $K'$ et la fonction $x^3$, on écrirait alors $\displaystyle\mathcal{Y}=\left\{ \left(-\frac{1}{x} + C\right)x^3 \,:\, C\in\eR, \: x\in I \right\}.$
      \end{remark}
    \item On doit simplement trouver la bonne valeur de $K$ dans \eqref{solgenexo34bis1}. Si $x=1$ on a $y = -1^2 + K\cdot 1^3 = -1 +K$ Donc $y(1) =2$ si et seulement si $K = 3$. 
    \end{enumerate}
  \item Résoudre  sur $]0\,;\,+\infty[$ l'équation différentielle $xy'+2y=\frac{x}{x^2+1}$. Ici aussi nous devons trouver d'abord la solution générale de l'équation homogène associé $xy' + 2y =0$, qui est, par la formule \eqref{solgeneqlinordre1}
      \begin{equation*}
        \mathcal{Y}_h=\left\{Ke^{-\int\frac{2}{x}\, dx} = \frac{K}{x^2}\,:\, K\in\eR, \: x\in I \right\}.
      \end{equation*}
      Ensuite nous appliquons la méthode de variation de la constante pour déterminer les solutions de l'équation de départ. Il faut injecter dans l'équation la ``candidate'' solution $\displaystyle \frac{K(x)}{x^2}$, ce qui donne
      \begin{equation*}
        x\left( \frac{K'(x)}{x^2} - 2\frac{K(x)}{x^3}\right) + 2\frac{K}{x^2} =\frac{x}{x^2+1}, 
      \end{equation*}
      \begin{equation*}
        K'(x)=\frac{x^2}{x^2+1}, 
      \end{equation*}
      Pour trouver les primitives de $K'$ nous pouvons utiliser le fait que $\displaystyle \frac{x^2}{x^2+1} =1 - \frac{1}{x^2+1}$, et obtenir $K(x) = x - \arctan(x) +C$ pour $C \in\eR$. La solution générale de l'équation est donc   
      \begin{equation*}
        \mathcal{Y}=\left\{\frac{x-\arctan(x)+ C}{x^2} + \,:\, C\in\eR, \: x\in I\right \}.
      \end{equation*}
\end{enumerate}


\end{corrige}   
