\begin{exercice}\label{exoSC_serie1-0004}

	Quelques exercices sur les matrices.
	\begin{enumerate}

		\item
			Donner des instructions (les plus simples possibles) pour produire la matrice $A$ de genre $10\times 10$ ayant la forme suivante :
			\begin{equation}
				A=\begin{pmatrix}
					\pi	&	0	&	0	&	\cdots	&	-1\\	
					0	&	\pi	&	0	&	\cdots	&	0\\	
					0	&	0	&	\pi	&	\cdots	&	0\\	
					\vdots	&	\ddots	&	\ddots	&	\ddots	&	\vdots\\	
					1	&	\cdots	&	0	&	0	&	\pi
				\end{pmatrix}
			\end{equation}.
			Remarque : les éléments représentés par des pointillés sont tous nuls sauf sur la diagonale principale de $A$.
		\item
			Calculer les trois premiers éléments de la diagonale principale de $A^{-1}$ et $A^5$.
	\end{enumerate}

\corrref{SC_serie1-0004}
\end{exercice}
