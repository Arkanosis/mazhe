
\thispagestyle{empty}

Plusieurs versions et extensions de ce document.
\begin{description}

    \item[La version courante] 

        Vous trouverez une version dédiée à l'agrégation régulièrement mise à jour à l'adresse suivante :
        \begin{center}
            \url{http://laurent.claessens-donadello.eu/pdf/lefrido.pdf}
        \end{center}

        \notbool{isEnVolume}{
    \item[Préliminaire de ce qui sera commercialisé]

        Le Frido est commercialisé en trois volumes sur \href{http://www.thebookedition.com/fr/}{thebookedition.com}. Vous pouvez aussi télécharger les fichiers.
        \begin{center}
     \url{http://laurent.claessens-donadello.eu/pdf/lefrido-vol1.pdf}\\
     \url{http://laurent.claessens-donadello.eu/pdf/lefrido-vol2.pdf}\\
     \url{http://laurent.claessens-donadello.eu/pdf/lefrido-vol3.pdf}
        \end{center}
        Ces fichiers là ne seront évidemment \textit{pas} mis à jour. Ils vous servent à préparer l'agrégation : ils contiennent exactement ce dont vous disposerez (si vous achetez ou si vous convainquez quelque un de l'acheter pour vous).
        
    \item[Pour les étudiants]

        Un texte ne contenant que (et tout) ce qui est destiné aux étudiants que j'ai eu (biologistes, agronomes, physiciens, etc) :
        \begin{center}
        \url{http://laurent.claessens-donadello.eu/pdf/enseignement.pdf}
        \end{center}
    }{}

    \item[La version la plus complète]

        Une version plus complète, comprenant le Frido, des exercices ainsi que de la mathématique de niveau recherche :
        \begin{center}
        \url{http://laurent.claessens-donadello.eu/pdf/mazhe.pdf}
        \end{center}

    \item[Tout ce qu'il faut savoir pour recompiler soi-même]
        Pour savoir comment recompiler ce document à l'identique, il faut lire
        \begin{center}
            \url{https://github.com/LaurentClaessens/mazhe}\\
            \url{http://laurent.claessens-donadello.eu/pdf/readme.pdf}
        \end{center}

\end{description}
