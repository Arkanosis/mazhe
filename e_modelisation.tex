% This is part of Agregation : modélisation
% Copyright (c) 2011
%   Laurent Claessens
% See the file fdl-1.3.txt for copying conditions.

\documentclass[a4paper,12pt]{book}
%\documentclass[a4paper,12pt,draft]{book}



\usepackage{ifthen}

\usepackage{latexsym}
\usepackage{amsfonts}
\usepackage{amsmath}
\usepackage{amsthm}
\usepackage{amssymb}
\usepackage{bbm}
\usepackage{mathrsfs}           

\usepackage{pstricks,pst-eucl,pstricks-add,calc,pst-math,catchfile}   % Les dépendances de phystricks.
\usepackage{graphicx}                   % Pour l'inclusion d'image en pfd.

\newcommand{\EpsOrPdfincludegraphics}[2][]{%
        \ifpdf
            \includegraphics[#1]{#2.png}
        \else
            \includegraphics[#1]{#2.eps}
        \fi
        }

\usepackage{subfigure}

\usepackage{fancyvrb}
\usepackage{stmaryrd}       % Pour le \obslash
\usepackage{xstring}        % Utilisé pour les références vers wikipédia
\usepackage{cases}
\usepackage{multicol}
\usepackage[all]{xy}


\usepackage[cdot,thinqspace,amssymb]{SIunits} 
 % L'option amssymb sert à éviter un conflit avec la commande \square de amssymb. Note qu'elle n'est plus accessible. Si tu en as besoin, faudra RTFM
%ftp://ftp.belnet.be/packages/ctan/macros/latex/contrib/SIunits/SIunits.pdf

\usepackage[nottoc]{tocbibind}
\usepackage[ps2pdf]{hyperref}       %Doit être appelé en dernier.
\hypersetup{colorlinks=true,linkcolor=blue}


%%%%%%%%%%%%%%%%%%%%%%%%%%
%
%   Trucs mathématiques
%
%%%%%%%%%%%%%%%%%%%%%%%%

% ENSEMBLES DE NOMBRES
\newcommand{\eR}{\mathbbm{R}}
\newcommand{\eZ}{\mathbbm{Z}}
\newcommand{\eC}{\mathbbm{C}}
\newcommand{\eQ}{\mathbbm{Q}}
\newcommand{\eN}{\mathbbm{N}}

% ENSEMBLES de fonctions
\newcommand{\aL}{\mathcal{L}}       % Les applications linéaires
\newcommand{\aC}{\mathcal{C}}       % Les fonctions C^1, C^2 etc

% AUTRES
\newcommand{\sdS}{\mathcal{S}}      % L'ensemble des subdivisions d'un intervalle.



\newcommand{\mF}{\mathcal{F}}
\newcommand{\mG}{\mathcal{G}}
\newcommand{\mI}{\mathcal{I}}
\newcommand{\mL}{\mathcal{L}}


\newcommand{\mtu}{\mathbbm{1}}              % La matrice unité
\newcommand{\caract}{\mathbbm{1}}    % Characteristic function of a set


%\newcommand{\efrac}[2]{\frac{ \displaystyle #1 }{\displaystyle #2 }}
%%%%%%%%%%%%%%%%%%%%%%%%%%
%
%   Numérotations en tout genre
%
%%%%%%%%%%%%%%%%%%%%%%%%

\setcounter{tocdepth}{3}        % Profondeur de la table des matièes
\setcounter{secnumdepth}{2}     % Profondeur dans le texte

%%%%%%%%%%%%%%%%%%%%%%%%%%
%
%   Les lignes magiques pour le texte en français.
%
%%%%%%%%%%%%%%%%%%%%%%%%

\usepackage[utf8]{inputenc}
\usepackage[T1]{fontenc}

\usepackage{textcomp}
\usepackage{lmodern}


    \usepackage[margin=2cm]{geometry} 
\usepackage[english,frenchb]{babel}

\usepackage[fr]{exocorr}
%%%%%%%%%%%%%%%%%%%%%%%%%%
%
%   Les théorèmes et choses attenantes
%
%%%%%%%%%%%%%%%%%%%%%%%%


\newcounter{numtho}
\newcounter{numprob}

\makeatletter
\@addtoreset{numtho}{chapter}
\@addtoreset{CountExercice}{section}
\makeatother

\newtheoremstyle{MyTheorems}%
        {9pt}{9pt}%
        {\itshape}%
        {}%
        {\bfseries}{.}%
        {\newline}%
        {}%
\newtheoremstyle{MyExamples}%
        {9pt}{9pt}%
        {}%
        {}%
        {\bfseries}{.}%
        {\newline}%
        {}%
\newtheoremstyle{MyRemarks}%
        {9pt}{9pt}%
        {}%
        {}%
        {\bfseries}{.}%
        {\newline}%
        {}%

\theoremstyle{MyExamples}   %\newtheorem{exemple}[numtho]{Exemple}      % Pour unification, ne plus utiliser
                            \newtheorem{example}[numtho]{Exemple}

\theoremstyle{MyRemarks}    \newtheorem{remark}[numtho]{Remarque}

                \newtheorem{amusement}[numtho]{Amusement}
                \newtheorem{erreur}[numtho]{Error}
                \newtheorem{probleme}[numprob]{\fbox{\bf Problèmes et choses à faire}}

\theoremstyle{MyTheorems}
            \newtheorem{lemma}[numtho]{Lemme}
            \newtheorem{corollary}[numtho]{Corollaire}
            \newtheorem{theorem}[numtho]{Théorème}      
            \newtheorem{definition}[numtho]{Définition}      
            \newtheorem{proposition}[numtho]{Proposition}      

\renewcommand{\thenumtho}{\thechapter.\arabic{numtho}}
% La numérotation des équations change dans les corrigés
\renewcommand{\theequation}{\thechapter.\arabic{equation}}
\renewcommand{\theCountExercice}{\arabic{section}.\arabic{CountExercice}}       % Ce compteur est défini dans SystemeCorr.sty
\newcommand{\defe}[2]{\textbf{#1}\index{#2}}

%\newcounter{CounterExample}
%\renewcommand{\theCounterExample}{\thechapter.\arabic{CounterExample}}
%\newenvironment{exemple}{\noindent{\bf Exemple \theCounterExample} }{\hfill $\triangle$}
%\renewcommand{\theenumi}{(\roman{enumi})}


%%%%%%%%%%%%%%%%%%%%%%%%%%
%
%   Les macros qui font des choses
%
%%%%%%%%%%%%%%%%%%%%%%%%

\newcommand{\tq}{\text{ tel que }}
\newcommand{\tqs}{\text{ tels que }}
\newcommand{\mA}{\mathcal{A}}
\newcommand{\mO}{\mathcal{O}}
\newcommand{\quext}[1]{ \footnote{\textsf{#1}}  }

\newcommand{\Borelien}{\mathcal{B}\text{or}}       % Les boréliens
\newcommand{\tribA}{\mathcal{A}}            % Une tribu A
\newcommand{\tribF}{\mathcal{F}}            % Une tribu F

\newcommand{\statS}{\mathcal{S}}            % Un modèle statistique

\newcommand{\dB}{\mathscr{B}}       % la distribution de Bernoulli
\newcommand{\dE}{\mathscr{E}}       % la distribution exponentielle
\newcommand{\dG}{\mathscr{G}}       % la distribution géométrique.
\newcommand{\dN}{\mathscr{N}}       % la distribution normale.
\newcommand{\hL}{\mathscr{L}}       
\newcommand{\dP}{\mathscr{P}}       % la distribution de Poisson.


%%%%%%%%%%%%%%%%%%%%%%%%%%
%
%   Bibliographie, index et liste des notations
%
%%%%%%%%%%%%%%%%%%%%%%%%

    \usepackage{makeidx}



\usepackage[nottoc]{tocbibind}      % Le paquetage qui fait en sorte que la biblio soit inclue correctement dans la table des matières.
\usepackage[refpage]{nomencl}
\renewcommand{\nomname}{Liste des notations}
%
%   Comment introduire des éléments dans l'index des notations.
%
% La liste des tags à mettre pour bien classer mes notations est :
% T     pour la topologie et théorie des ensembles
%
% La syntaxe est facile, par exemple 
%       $\SL(2,\eR)$\nomenclature[G]{$\SL(2,\eR)$}{Le groupe de matrices deux par deux de déterminant 1.}
\renewcommand{\nomgroup}[1]{%
    \ifthenelse{\equal{#1}{T}}{\item[\textbf{Topologie et théorie des ensembles}]}{}%
    \ifthenelse{\equal{#1}{C}}{\item[\textbf{Courbes paramétrées}]}{}%
}

%%%%%%%%%%%%%%%%%%%%%%%%%%
%
%   DeclareMathOperator
%
%%%%%%%%%%%%%%%%%%%%%%%%

\DeclareMathOperator{\signe}{sgn}
\DeclareMathOperator{\Vol}{Vol}
\DeclareMathOperator{\Int}{Int}     % Intérieur d'un ensemble.
\DeclareMathOperator{\Diam}{Diam}   
\DeclareMathOperator{\id}{Id}   
\DeclareMathOperator{\Graph}{Graph} 
\DeclareMathOperator{\pr}{\texttt{proj}}
\DeclareMathOperator{\dom}{dom}
\DeclareMathOperator{\arctg}{arctg}
\DeclareMathOperator{\cotg}{cotg}
\DeclareMathOperator{\cosec}{cosec}
\DeclareMathOperator{\arcsinh}{arcsinh}

\DeclareMathOperator{\Var}{Var}     % Variance d'une variable aléatoire.
\DeclareMathOperator{\Cov}{Cov}     % la covariance.


%%%%%%%%%%%%%%%%%%%%%%%%%%%%%%%%%%%%%%
%
% les petis yeux 
%
%%%%%%%%%%%%%%%%%%%%%%%%%%%%%%%%%%%%%%%%%%%%%

\newcommand{\coolexo}{$\circledast\circledast$}
\newcommand{\boringexo}{$\circleddash\circleddash$}
\newcommand{\minsyndical}{$\odot\odot$}
\newcommand{\mortelexo}{$\obslash\oslash$}
