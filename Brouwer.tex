% This is part of Mes notes de mathématique
% Copyright (c) 2011-2012
%   Laurent Claessens
% See the file fdl-1.3.txt for copying conditions.

%+++++++++++++++++++++++++++++++++++++++++++++++++++++++++++++++++++++++++++++++++++++++++++++++++++++++++++++++++++++++++++
\section{Théorèmes de Brouwer et Schauder}
%+++++++++++++++++++++++++++++++++++++++++++++++++++++++++++++++++++++++++++++++++++++++++++++++++++++++++++++++++++++++++++

\begin{proposition}
    Soit \( f\colon \mathopen[ 0 , 1 \mathclose]\to \mathopen[ 0 , 1 \mathclose]\) une fonction continue. Alors \( f\) accepte un point fixe.
\end{proposition}

\begin{proof}
    En effet si nous considérons \( g(x)=f(x)-x\) alors nous avons \( g(0)=f(0)\geq 0\) et \( g(1)=f(1)-1\leq 0\). Si \( g(0)\) ou \( g(1)\) est nul, la proposition est démontrée; nous supposons donc que \( g(0)>0\) et \( g(1)<0\). La proposition découle à présent du théorème des valeurs intermédiaires.
\end{proof}

%---------------------------------------------------------------------------------------------------------------------------
\subsection{Formes différentielles}
%---------------------------------------------------------------------------------------------------------------------------

Nous allons donner une toute petite introduction aux formes différentielles sur des variétés compactes.

\begin{lemma}[\cite{SpindelGeomDoff}]       \label{LemdwLGFG}
    Soit \( \omega\) une \( k\)-forme sur \( \eR^n\) et \( f\), une fonction \( C^{\infty}\) sur \( \eR^n\). Alors \( d(f^*\omega)=f^*d\omega\).
\end{lemma}

\begin{proof}
    Nous effectuons la preuve par récurrence sur le degré de la forme. Soit d'abord une \( 0\)-forme, c'est à dire une fonction \( g\colon \eR^n\to \eR\). Nous avons
    \begin{equation}
        d(d^*g)X=d(g\circ f)X=(dg\circ df)X=dg\big( df X \big)=(f^*dg)(X).
    \end{equation}
    
    Supposons maintenant que le résultat soit exact pour toute les \( p-1\) formes et montrons qu'il reste valable pour les \( p\)-formes. Par linéarité de la différentielle nous pouvons nous contenter de considérer la forme différentielle
    \begin{equation}
        \omega=g\,dx^1\wedge\ldots dx^p
    \end{equation}
    où \( g\) est une fonction \(  C^{\infty}\). Pour soulager les notations nous allons noter \( dx^I=dx^1\wedge\ldots dx^{p-1}\). Nous avons
    \begin{subequations}
        \begin{align}
            d(f^*\omega)&=d\big( f^*(gdx^I\wedge dx^p) \big)\\
            &=d\big( f^*(gdx^I)\wedge f^*dx^p \big)\\
            &=d\big( f^*(gdx^I)\big)\wedge f^*dx^p+(-1)^{p-1}f^*(gdx^I)\wedge(f^*dx^p)  \label{gnAnSt}\\
            &=f^*\big( d(gdx^I) \big)\wedge f^*dx^p      \label{xZrfjZ}\\
            &=f^*\big( d(gdx^I)\wedge dx^p \big)\\
            &=f^*d\omega        \label{loWUji}
        \end{align}
    \end{subequations}
    Justifications : \eqref{gnAnSt} est la formule de Leibnitz. \eqref{xZrfjZ} est parce que le second terme est nul : \( d(f^*dx^p)=f^*(d^2x^p)=0\). Nous avons utilisé l'hypothèse de récurrence et le fait que \( d^2=0\). L'étape \eqref{loWUji} est une utilisation à l'envers de la règle de Leibnitz en tenant compte que \( d^2x^p=0\).
\end{proof}

Soit \( M\) une variété de dimension \( n\) et \( \omega\) une \( n\)-forme différentielle
\begin{equation}
    \omega_p=f(p)dx_1\wedge\ldots\wedge dx_n.
\end{equation}
 Si \( (U,\varphi)\) est une carte (\( U\subset\eR^n\) et \( \varphi\colon U\to M\)) alors nous définissons
\begin{equation}
    \int_{\varphi(U)}\omega=\int_{U}f\big( \varphi(x) \big)dx_1\ldots dx_n.     
\end{equation}
Lorsque nous voulons intégrer sur une partie plus grande qu'une carte nous utilisons une partition de l'unité.
\begin{lemma}   \label{LemGPmRGZ}
    Soit \( \{ U_i \}\) un recouvrement de \( M\) par un nombre fini d'ouverts\footnote{Si \( M\) n'est pas compacte, alors il faut utiliser une version un peu plus élaborée du lemme\cite{SpindelGeomDoff}.}. Alors il existe une famille de fonctions \( f_i\in  C^{\infty}(M)\) telle que
    \begin{enumerate}
        \item
            \( \supp f_i\subset U_i\),
        \item
            pour tout \( i\), nous avons \( f_i\geq 0\),
        \item
            pour tout \( p\in M\) nous avons \( \sum_i f_i(p)=1\).
    \end{enumerate}
\end{lemma}
La famille \( (f_i)\) est une \defe{partition de l'unité}{partition!de l'unité} subordonnée au recouvrement \( \{ U_i \}\). Si \( \{ f_i \}\) est une partition de l'unité subordonnée à un atlas de \( M\) nous définissons
\begin{equation}
    \int_M\omega=\sum_i\int_{U_i}f\omega.
\end{equation}
Il est possible de montrer que cette définition ne dépend pas du choix de la partition de l'unité.

\begin{remark}
    Nous ne définissons pas d'intégrale de \( k\)-forme différentielle sur une variété de dimension \( n\neq k\). Le seul cas où cela se fait est le cas de \( 0\)-formes (les fonctions), mais cela n'est pas vraiment un cas particulier vu que les \( 0\)-formes sont associées aux \( n\)-formes de façon évidente.
\end{remark}

%---------------------------------------------------------------------------------------------------------------------------
\subsection{Théorème de Brouwer}
%---------------------------------------------------------------------------------------------------------------------------

Nous commençons par énoncer et démontrer le théorème de Brouwer dans le cas des fonctions \(  C^{\infty}\) en utilisant le théorème de Stockes.
\begin{proposition}     \label{PropDRpYwv}
    Soit \( B\) la boule fermée de centre \( 0\) et de rayon \( 1\) de \( \eR^n\) et \( f\colon B\to B\) une fonction \(  C^{\infty}\). Alors \( f\) admet un point fixe.
\end{proposition}

\begin{proof}
    Supposons que \( f\) ne possède pas de points fixes. Alors pour tout \( x\in B\) nous considérons la ligne droite partant de \( x\) dans la direction de \( f(x)\) (cette droite existe parce que \( x\) et \( f(x)\) sont supposés distincts). Cette ligne intersecte \( \partial B\) en un point que nous appelons \( F(x)\). La fonction \( F\) ainsi définie vérifie deux propriétés :
    \begin{enumerate}
        \item
            elle est \(  C^{\infty}\) parce que \( f\) l'est;
        \item
            elle est l'identité sur \( \partial B\).
    \end{enumerate}
    La suite de la preuve consiste à montrer qu'une telle rétraction sur \( B\) ne peut pas exister\footnote{Notons qu'il n'existe pas non plus de rétractions continues sur \( B\), mais pour le montrer il faut utiliser d'autres méthodes que Stockes, ou alors présenter les choses dans un autre ordre.}.

    Nous considérons une forme de volume \( \omega\) sur \( \partial B\) : l'intégrale de \( \omega\) sur \( \partial B\) est la surface de \( \partial B\) qui est non nulle. Nous avons alors
    \begin{equation}
        0<\int_{\partial B}\omega
        =\int_{\partial B}F^*\omega
        =\int_Bd(F^*\omega)
        =\int_Bd^*(d\omega)
        =0
    \end{equation}
    Justifications :
    \begin{itemize}
        \item 
            L'intégrale \( \int_{\partial B}\omega\) est la surface de \( \partial B\) et est donc non nulle.
        \item
            La fonction \( F\) est l'identité sur \( \partial B\). Nous avons donc \( \omega=F^*\omega\).
        \item
            Théorème de Stockes.
        \item
            Le lemme \ref{LemdwLGFG}.
        \item
            La forme \( \omega\) est de volume, par conséquent de degré maximum et \( d\omega=0\).
    \end{itemize}
\end{proof}

Un des points délicats est de se ramener au cas de fonctions \( C^{\infty}\). Pour la régularisation par convolution, voir \cite{AllardBrouwer}; pour celle utilisant le théorème de Weierstrass, voir \cite{KuttlerTopInAl}.
\begin{theorem}[Brouwer]\index{théorème!Brouwer}\label{ThoRGjGdO}
    Soit \( B\) la boule fermée de centre \( 0\) et de rayon \( 1\) de \( \eR^n\) et \( f\colon B\to B\) une fonction continue. Alors \( f\) admet un point fixe.
\end{theorem}

\begin{proof}
    Nous commençons par définir une suite de fonctions
    \begin{equation}
        f_k(x)=\frac{ f(x) }{ 1+\frac{1}{ k } }.
    \end{equation}
    Nous avons \( \| f_k-f \|_{\infty}\leq \frac{1}{ 1+k }\) où la norme est la norme uniforme sur \( B\). Par le théorème de Weierstrass \ref{ThoWmAzSMF} il existe une suite de fonctions \(  C^{\infty}\) \( g_k\) telles que
    \begin{equation}
        \|  g_k-f_k\|_{\infty}\leq\frac{1}{ 1+k }.
    \end{equation}
    Vérifions que cette fonction \( g_k\) soit bien une fonction qui prend ses valeurs dans \( B\) :
    \begin{subequations}
        \begin{align}
            \| g_k(x) \|&\leq \| g_k(x)-f_k(x) \|+\| f_k(x) \|\\
            &\leq \frac{1}{ 1+k }+\frac{ \| f(x) \| }{ 1+\frac{1}{ k } }\\
            &\leq \frac{1}{ 1+k}+\frac{1}{ 1+\frac{1}{ k } }\\
            &=1.
        \end{align}
    \end{subequations}
    Par la version \(  C^{\infty}\) du théorème (proposition \ref{PropDRpYwv}), \( g_k\) admet un point fixe que l'on nomme \( x_k\).

    Étant donné que \( x_k\) est dans le compact \( B\), quitte à prendre une sous suite nous supposons que la suite \( (x_k)\) converge vers un élément \( x\in B\). Nous montrons maintenant que \( x\) est un point fixe de \( f\) :
    \begin{subequations}
        \begin{align}
            \| f(x)-x \|&=\| f(x)-g_k(x)+g_k(x)-x_k+x_k-x \|\\
            &\leq \| f(x)-g_k(x) \| +\underbrace{\| g_k(x)-x_k \|}_{=0}+\| x_k-x \|\\
            &\leq \frac{1}{ 1+k }+\| x_k-x \|.
        \end{align}
    \end{subequations}
    En prenant le limite \( k\to\infty\) le membre de droite tend vers zéro et nous obtenons \( f(x)=x\).
\end{proof}

%---------------------------------------------------------------------------------------------------------------------------
\subsection{Théorème de Schauder et équations différentielles}
%---------------------------------------------------------------------------------------------------------------------------

Une conséquence du théorème de Brouwer est le théorème de Schauder qui est valide en dimension infinie.

\begin{theorem}[Théorème de Schauder\cite{LeDretSc}]\index{théorème!Schauder}       \label{ThovHJXIU}
    Soit \( E\), un espace vectoriel normé, \( K\) un convexe compact de \( E\) et \( f\colon K\to K\) une fonction continue. Alors \( f\) admet un point fixe.
\end{theorem}

\begin{proof}
    Étant donné que \( f\colon K\to K\) est continue, elle y est uniformément continue. Si nous choisissons \( \epsilon\) alors il existe \( \delta>0\) tel que 
    \begin{equation}
        \| f(x)-f(y) \|\leq \epsilon
    \end{equation}
    dès que \( \| x-y \|\leq \delta\). La compacité de \( K\) permet de choisir un recouvrement fini par des ouverts de la forme
    \begin{equation}    \label{EqKNPUVR}
        K\subset \bigcup_{1\leq i\leq p}B(x_j,\delta)
    \end{equation}
    où \( \{ x_1,\ldots, x_p \}\subset K\). Nous considérons maintenant \( L=\Span\{ f(x_j)\tq 1\leq j\leq p \}\) et
    \begin{equation}
        K^*=K\cap L.
    \end{equation}
    Le fait que \( K\) et \( L\) soient convexes implique que \( K^*\) est convexe. L'ensemble \( K^*\) est également compact parce qu'il s'agit d'une partie fermée de \( K\) qui est compact (lemme \ref{LemnAeACf}). Notons en particulier que \( K^*\) est contenu dans un espace vectoriel de dimension finie, ce qui n'est pas le cas de \( K\).

    Nous allons à présent construire une sorte de partition de l'unité subordonnée au recouvrement \eqref{EqKNPUVR} sur \( K\) (voir le lemme \ref{LemGPmRGZ}). Nous commençons par définir
    \begin{equation}
        \psi_j(x)=\begin{cases}
            0    &   \text{si \( \| x-x_j \|\geq \delta\)}\\
            1-\frac{ \| x-x_j \| }{ \delta }    &    \text{sinon}.
        \end{cases}
    \end{equation}
    pour chaque \( 1\leq j\leq p\). Notons que \( \psi_j\) est une fonction positive, nulle en-dehors de \( B(x_j,\delta)\). En particulier la fonction suivante est bien définie :
    \begin{equation}
        \varphi_j(x)=\frac{ \psi_j(x) }{ \sum_{k=1}^p\psi_k(x) }
    \end{equation}
    et nous avons \( \sum_{j=1}^p\varphi_j(x)=1\). Les fonctions \( \varphi_j\) sont continues sur \( K\) et nous définissons finalement
    \begin{equation}
        g(x)=\sum_{j=1}^p\varphi_j(x)f(x_j).
    \end{equation}
    Pour chaque \( x\in K\), l'élément \( g(x)\) est une combinaison des éléments \( f(x_j)\in K^*\). Étant donné que \( K^*\) est convexe et que la somme des coefficients \( \varphi_j(x)\) vaut un, nous avons que \( g\) prend ses valeurs dans \( K^*\) par la proposition \ref{PropPoNpPz}.

    Nous considérons seulement la restriction \( g\colon K^*\to K^*\) qui est continue sur un compact contenu dans un espace vectoriel de dimension finie. Le théorème de Brouwer nous enseigne alors que \( g\) a un point fixe (proposition \ref{ThoRGjGdO}). Nous nommons \( y\) ce point fixe. Notons que \( y\) est fonction du \( \epsilon\) choisit au début de la construction, via le \( \delta\) qui avait conditionné la partition de l'unité.

    Nous avons
    \begin{subequations}        \label{EqoXuTzE}
        \begin{align}
            f(y)-y&=f(y)-g(y)\\
            &=\sum_{j=1}^p\varphi_j(y)f(y)-\sum_{j=1}^p\varphi_j(y)f(x_j)\\
            &=\sum_{j=1}^p\varphi(j)(y)\big( f(y)-f(x_j) \big).
        \end{align}
    \end{subequations}
    Par construction, \( \varphi_j(y)\neq 0\) seulement si \( \| y-x_j \|\leq \delta\) et par conséquent seulement si \( \| f(y)-f(x_j) \|\leq \epsilon\). D'autre par nous avons \( \varphi_j(y)\geq 0\); en prenant la norme de \eqref{EqoXuTzE} nous trouvons
    \begin{equation}
        \| f(y)-y \|\leq \sum_{j=1}^p\| \varphi_j(y)\big( f(y)-f(x_j) \big) \|\leq \sum_{j=1}^p\varphi_j(y)\epsilon=\epsilon.
    \end{equation}
    Nous nous souvenons maintenant que \( y\) était fonction de \( \epsilon\). Soit \( y_m\) le \( y\) qui correspond à \( \epsilon=2^{-m}\). Nous avons alors
    \begin{equation}
        \| f(y_m)-y_m \|\leq 2^{-m}.
    \end{equation}
    L'élément \( y_m\) est dans \( K^*\) qui est compact, donc quitte à choisir une sous suite nous pouvons supposer que \( y_m\) est une suite qui converge vers \( y^*\in K\)\footnote{Notons que même dans la sous suite nous avons \( \| f(y_m)-y_m \|\leq 2^{-m}\), avec le même «\( m\)» des deux côtés de l'inégalité.}. Nous avons les majorations
    \begin{equation}
        \| f(y^*)-y^* \|\leq \| f(y^*)-f(y_m) \|+\| f(y_m)-y_m \|+\| y_m-y^* \|.
    \end{equation}
    Si \( m\) est assez grand, les trois termes du membre de droite peuvent être rendus arbitrairement petits, d'où nous concluons que
    \begin{equation}
        f(y^*)=y^*
    \end{equation}
    et donc que \( f\) possède un point fixe.
\end{proof}

Ce théorème permet de démontrer une version du théorème de Cauchy-Lipschitz (théorème \ref{ThokUUlgU}) sans la condition Lipschitz, mais alors sans unicité de la solution. Notons que de ce point de vue nous sommes dans la même situation que la différence entre le théorème de Brouwer et celui de Picard : hors hypothèse de type «contraction», point d'unicité.

\begin{theorem}[Cauchy-Arzela\cite{ClemKetl}]\index{théorème!Cauchy-Arzela}
    Nous considérons le système d'équation différentielles
    \begin{subequations}        \label{EqTXlJdH}
        \begin{numcases}{}
            y'=f(t,y)\\
            y(t_0)=y_0.
        \end{numcases}
    \end{subequations}
    avec \( f\colon U\to \eR^n\), continue où \( U\) est ouvert dans \( \eR\times \eR^n\). Alors il existe un voisinage fermé \( V\) de \( t_0\) sur lequel une solution \( C^1\) du problème \eqref{EqTXlJdH} existe.
\end{theorem}

\begin{proof}[Idée de la démonstration]
    Nous considérons \( M=\| f \|_{\infty}\) et \( K\), l'ensemble des fonctions \( M\)-Lipschitz sur \( U\). Nous prouvons que \( (K,\| . \|_{\infty})\) est compact. Ensuite nous considérons l'application
    \begin{equation}
        \begin{aligned}
            \Phi\colon K&\to K \\
            \Phi(f)(t)&=x_0+\int_{t_0}^tf\big( u,f(u) \big)du. 
        \end{aligned}
    \end{equation}
    Après avoir prouvé que \( \Phi\) était continue, nous concluons qu'elle a un point fixe par le théorème de Schauder \ref{ThovHJXIU}.
\end{proof}


%+++++++++++++++++++++++++++++++++++++++++++++++++++++++++++++++++++++++++++++++++++++++++++++++++++++++++++++++++++++++++++
\section{Théorème de Markov-Kakutani et mesure de Haar}
%+++++++++++++++++++++++++++++++++++++++++++++++++++++++++++++++++++++++++++++++++++++++++++++++++++++++++++++++++++++++++++

\begin{definition}
    Soit \( G\) un groupe topologique. Une \defe{mesure de Haar}{mesure!de Haar} sur \( G\) est une mesure \( \mu\) telle que 
    \begin{enumerate}
        \item
            \( \mu(gA)=\mu(A)\) pour tout mesurable \( A\) et tout \( g\in G\),
        \item
            \( \mu(K)<\infty\) pour tout compact \( K\subset G\).
    \end{enumerate}
    Si de plus le groupe \( G\) lui-même est compact nous demandons que la mesure soit normalisée : \( \mu(G)=1\).
\end{definition}

Le théorème suivant nous donne l'existence d'une mesure de Haar sur un groupe compact.
\begin{theorem}[Markov-Katutani\cite{BeaakPtFix}]\index{théorème!Markov-Takutani}   \label{ThoeJCdMP}
    Soit \( E\) un espace vectoriel normé et \( K\), une partie non vide, convexe, fermée et bornée de \( E'\). Soit \( T\colon K\to K\) une application continue. Alors \( T\) a un point fixe.
\end{theorem}

\begin{proof}
    Nous considérons un point \( x_0\in K\) et la suite
    \begin{equation}
        x_n=\frac{1}{ n+1 }\sum_{i=0}^n T^ix_0.
    \end{equation}
    La somme des coefficients devant les \( T^i(x_0)\) étant \( 1\), la convexité de \( K\) montre que \( x_n\in K\). Nous considérons l'ensemble
    \begin{equation}
        C=\bigcap_{n\in \eN}\overline{ \{ x_m\tq m\geq n \} }.
    \end{equation}
    Le lemme \ref{LemooynkH} indique que \( C\) n'est pas vide, et de plus il existe une sous suite de \( (x_n)\) qui converge vers un élément \( x\in C\). Nous avons
    \begin{equation}
        \lim_{n\to \infty} x_{\sigma(n)}(v)=x(v)
    \end{equation}
    pour tout \( v\in E\). Montrons que \( x\) est un point fixe de \( T\). Nous avons
    \begin{subequations}
        \begin{align}
            \| (Tx_{\sigma(k)}-x_{\sigma(k)})v \|&=\Big\| T\frac{1}{ 1+\sigma(k) }\sum_{i=0}^{\sigma(k)}T^ix_0(v)-\frac{1}{ 1+\sigma(k) }\sum_{i=0}^{\sigma(k)}T^ix_0(v) \Big\|\\
            &=\Big\| \frac{1}{ 1+\sigma(k) }\sum_{i=0}^{\sigma(k)}T^{i+1}x_0(v)-T^ix_0(v) \Big\|\\
            &=\frac{1}{ 1+\sigma(k) }\big\| T^{\sigma(k)+1}x_0(v)-x_0(v) \big\|\\
            &\leq\frac{ 2M }{ \sigma(k)+1 }
        \end{align}
    \end{subequations}
    où \( M=\sum_{y\in K}\| y(v) \|<\infty\) parce que \( K\) est borné. En prenant \( k\to\infty\) nous trouvons
    \begin{equation}
        \lim_{k\to \infty} \big( Tx_{\sigma(k)}-x_{\sigma(k)} \big)v=0,
    \end{equation}
    ce qui signifie que \( Tx=x\) parce que \( T\) est continue.
\end{proof}

Le théorème suivant est une conséquence du théorème de Markov-Katutani.
\begin{theorem}\index{mesure!Haar}
    Si \( G\) est un groupe topologique compact possédant une base dénombrable de topologie alors \( G\) accepte une unique mesure de Haar normalisée. De plus elle est unimodulaire :
    \begin{equation}
        \mu(Ag)=\mu(gA)=\mu(A)
    \end{equation}
    pour tout mesurables \( A\subset G\) et tout élément \( g\in G\).
\end{theorem}

%+++++++++++++++++++++++++++++++++++++++++++++++++++++++++++++++++++++++++++++++++++++++++++++++++++++++++++++++++++++++++++
\section{Prolongement de fonctions}
%+++++++++++++++++++++++++++++++++++++++++++++++++++++++++++++++++++++++++++++++++++++++++++++++++++++++++++++++++++++++++++

Sources : \cite{RasclAnaFonc,MaurayAnalSpec}

\begin{lemma}   \label{LemdCOMQM}
    Soit \( E\), un espace vectoriel normé complet et \( (A_n)\) une suite emboité de fermés non vides dont le diamètre tend vers zéro. Alors l'intersection \( \bigcap_{n\in \eN}A_n\) contient exactement un point.
\end{lemma}

\begin{proof}
    Si l'intersection contenait deux points distincts \( a\) et \( b\), alors nous aurions \( \diam(A_n)\geq\| a-b \|\), ce qui contredirait la limite.

    Soit une suite \( (x_n)\) avec \( x_k\in A_k\) pour tout \( k\in \eN\). C'est une suite de Cauchy. En effet si \( \epsilon>0\), considérons \( N\) tel que \( \diam(A_N)<\epsilon\). Dans ce cas dès que \( n,m>N\) nous avons \( x_n,x_m\in A_{N}\) et donc \( \| x_n-x_m \|\leq \epsilon\). La suite \( x_n\) converge donc vers un élément dans \( E\).

    Nous devons montrer que \( x\in A_k\) pour tout \( k\). La queue de suite \( (x_n)_{n\geq k}\) est une suite de Cauchy dans \( A_k\) qui converge donc vers un élément de \( A_k\) (ici nous utilisons le fait que \( A_k\) est fermé). Par unicité de la limite, cette dernière doit être \( x\). Par conséquent \( x\in\bigcap_{n\in \eN}A_n\).
\end{proof}

\begin{theorem}[\cite{MaurayAnalSpec}]      \label{ThoCaMpKO}
    Soient \( X\) et \( Y\) des espaces vectoriels normés. Pour une application linéaire \( f\colon X\to Y\), les assertions suivantes sont équivalentes :
    \begin{enumerate}
        \item
            \( f\) est continue sur \( X\),
        \item
            \( f\) est continue en un point de \( X\),
        \item
            \( f\) est bornée.
    \end{enumerate}
\end{theorem}

\begin{proposition}
    Soit \( X\) un espace normé et \( A\) une partie dense de \( X\). Soit \( F\) un espace de Banach. Toute application linéaire continue \( f\colon A\to F\) se prolonge de façon unique en une application linéaire continue \( \tilde f\colon X\to F\). De plus \( \| \tilde f \|=\| f \|\).
\end{proposition}

\begin{proof}
    Soit \( x\in X\) et la suite d'ensemble
    \begin{equation}
        A_n=\{ y\in A\tq \| x-y \|\leq 2^{-n}\}.
    \end{equation}
    Étant donné que \( A\) est dense, ces ensembles sont tous non vides. De plus \( \diam A_n\to 0\) parce que si \( y,y'\in A_n\) alors
    \begin{equation}
        \| y-y' \|\leq\| y-x \|+\| x-y' \|\leq 2^{-n+1}.
    \end{equation}
    Vu que \( f\) est bornée, la suite d'ensembles \( f(A_n)\) est une suite emboitée d'ensembles non vides de \( X\). De plus leur diamètre tend vers zéro. En effet si \( z,z'\in f(A_n)\), nous posons \( z=f(y)\), \( z'=f(y')\) et nous avons
    \begin{equation}
        \| z-z' \|\leq \| f(y)-f(x) \|+\| f(x)-f(y') \|\leq \| f \|\big( \| y-x \|+\| x-y' \| \big),
    \end{equation}
    ce qui montre que \( \diam f(A_n)\leq \| f \|2^{-n+1}\).  Notons que nous avons utilisé la linéarité de \( f\). Par le lemme \ref{LemdCOMQM}, l'intersection \( \bigcap_{n\in \eN}\overline{ f(A_n) }\) contient exactement un point. Nous posons
    \begin{equation}
        S(x)=\bigcap_{n\in \eN}\overline{ f(A_n) }.
    \end{equation}
    Nous allons montrer que l'application \( x\mapsto S(x)\) ainsi définie est l'application que nous cherchons. 

    Nous commençons par montrer que pour toute suite \( y_k\to x\) avec \( y_k\in A\) nous avons 
    \begin{equation}    \label{EqBnRZxW}
        f(y_k)\to S(x).
    \end{equation}
    Pour cela nous considérons \( n_0\in \eN\) et \( k_0\) tel que \( y_{k_0}\in A_{n_0}\). Avec cela nous avons
    \begin{equation}
        \| f(y_k)-S(x) \|\leq \diam(A_{n_0})\leq \| f \|2^{-n_0+1}.
    \end{equation}
    Pour montrer que \( S\) est linéaire, nous considérons deux suites dans \( A\) : \( y_k\to x\) et \( y'_k\to x'\) ainsi que la somme \( y_k+y'k\to x+x'\). Nous écrivons la relation \eqref{EqBnRZxW} pour ces trois suites :
    \begin{subequations}
        \begin{align}
            f(y_k)\to S(x)\\
            f(y'_k)\to S(x')\\
            f(y_k+y'_x)\to S(x+x').
        \end{align}
    \end{subequations}
    Cependant, étant donné que \( f\) est linéaire, pour tout \( k\) nous avons \( f(y_k+y'_k)=f(y_k)+f(y'_k)\) et par conséquent
    \begin{equation}
        f(y_k+y'_k)\to S(x)+S(x').
    \end{equation}
    Par unicité de la limite, \( S(x+x')=S(x)+S(x')\). Le même genre de raisonnement montre que \( S(\lambda x)=\lambda S(x)\). L'application \( S\) est donc linéaire.

    En ce qui concerna la continuité, nous avons
    \begin{equation}
            \| S(x) \|=\lim\| f(y_k) \|\leq \| f \|\| \lim y_k \|=\| f \|\| x \|,
    \end{equation}
    donc \( \| S \|\leq \| f \|\), c'est à dire que \( S\) est borné et donc continue parce que linéaire (théorème \ref{ThoCaMpKO}).

    Nous montrons maintenant que \( S\) prolonge \( f\). Si \( x\in A\), alors nous avons \( \bigcap_{n\in \eN}f(A_n)=f(x)\), et donc \( S(x)=f(x)\). Cela montre du même coup que \( \| f \|\leq \| S \|\) et que par conséquent \( \| f \|=\| S \|\).

    Passons à la partie sur l'unicité. Soient donc \( S\) et \( T\), deux prolongements continus de \( f\) sur \( X\). Soit \( x\in X\) et une suite \( x_n\to x\) dans \( A\). Par continuité nous avons \( T(x_n)\to T(x)\) et \( S(x_n)\to S(x)\). Étant donné que par ailleurs pour tout \( n\) nous avons \( S(x_n)=T(x_n)\), l'unicité de la limite montre que \( T(x)=S(x)\).
\end{proof}

\begin{definition}
    Soit une application \( f\colon X\to Y\). Le \defe{module de continuité}{module!de continuité} de \( f\) est la fonction \( \omega_f\colon \eR\to \eR\) définie comme suit. On pose \( \omega_f(x)=0\) pour \( x\leq 0\) et si \( h>0\),
    \begin{equation}
        \omega_f(h)=\sup_{\substack{x,y\in X\\d_X(x,y)<h}} d_Y\big( f(x),f(y) \big).
    \end{equation}
\end{definition}

\begin{lemma}   \label{LemeERapq}
    Une fonction \( f\) est uniformément continue si et seulement si son module de continuité est continue en zéro.
\end{lemma}

Dans la même veine nous avons ce résultat.
\begin{theorem}[\cite{ZHDEie}]      \label{ThoPVFQMi}
    Soient \( E\) et \( F\), deux espaces métriques complets ainsi que \( A\) dense dans \( E\). Si \( u\colon A\to F\) est uniformément continue, alors elle se prolonge de façon unique en une fonction continue \( \tilde u\colon E\to F\). De plus ce prolongement est uniformément continu.
\end{theorem}

\begin{proof}
    Soit \( x\in E\setminus A\) et une suite \( (x_n)\) contenue dans \( A\) et convergente vers \( x\). Nous voulons définir
    \begin{equation}
        \tilde u(x)=\lim_{n\to \infty} u(x_n)
    \end{equation}
    mais pour ce faire nous devons prouver que la suite \( \big( u(x_n) \big)\) converge dans \( F\) et que la limite ne dépend pas de la suite choisie parmi les suites de \( A\) qui convergent (dans \( E\)) vers \( x\).

    Commençons par montrer que \( \big( u(x_n) \big)\) est de Cauchy dans \( F\). Pour cela nous prenons \( \epsilon>0\) et \( \eta>0\) telle que \( d_E(a,b)<\eta\) implique \( d_F\big( u(a),u(b) \big)<\epsilon\) (uniforme continuité de \( u\)). Après, il suffit de choisir \( N\) tel que pour tout \( n,m>N\) nous ayons \( d(x_m,x_n)<\eta\) (parce que \( u_n\) est de Cauchy). Avec tout ça nous avons 
    \begin{equation}
        d_F\big( u(x_m),u(x_n) \big)<\epsilon,
    \end{equation}
    ce qui signifie que \( \big( u(x_n) \big)\) est de Cauchy et donc convergente dans \( F\). 
    
    Nous voulons montrer maintenant que si \( (x_n)\) et \( (y_n)\) sont deux suites dans \( A\) convergentes vers \( x\) alors \( \lim_{n\to \infty} u(x_n)=\lim_{n\to \infty} u(y_n)\). Pour cela nous considérons la suite \( z=(x_1,y_1,x_2,y_2,\ldots)\). Nous avons évidemment \( z_n\to x\), et donc \( u(z_n)\) converge dans \( F\) par ce qui a été dit plus haut. Mais \( u(x_n)\) et \( u(y_n)\) en sont deux sous-suites convergentes. Donc leurs limites sont égales.

    Il reste à montrer que ce \( \tilde u\) est continue et uniformément continue. Pour cela nous utilisons le module de continuité et le lemme \ref{LemeERapq}. Étant donné que \( \tilde u\) prolonge \( u\) nous avons 
    \begin{equation}        \label{EqFRYqON}
        \omega_{\tilde u}(h)\geq \omega_u(h).
    \end{equation}
    Soit \( h>0\) et \( \epsilon>0\); soit aussi \( x,y\in E\) tels que \( d(x,y)<h\). Nous prenons des suites \( (a_n)\to x\) et \( (y_n)\to y\) tout en choisissant \( n\) assez grand pour avoir \( d_E(a_n,b_n)<h\). Nous avons
    \begin{equation}
        d_F\big( \tilde u(x),\tilde u(y) \big)\leq d_F\big( \tilde u(x),u(a_n) \big)+d\big( u(a_n),u(b_n) \big)+d_F\big( u(b_n),\tilde u(y) \big).
    \end{equation}
    Si \( n\) est assez grand, par construction de \( \tilde u\), le premier et le dernier terme sont plus petits que \( \epsilon\). Par définition du module de continuité nous avons d'autre part \( d_F\big( u(a_n),u(b_n) \big)\leq \omega_u(h)\). Du coup
    \begin{equation}
        d_F\big( \tilde u(x),\tilde u(y) \big)\leq \omega_u(h)+2\epsilon.
    \end{equation}
    Si nous prenons le supremum sur les \( x\) et \( y\) vérifiant \( d_E(x,y)<h\), à gauche nous obtenons \( \omega_{\tilde u}(h)\) tandis que le membre de droite ne dépend pas de \( x\) et\( y\). Donc pour tout \( \epsilon\), nous avons
    \begin{equation}
        \omega_{\tilde u}(h)\leq \omega_u(h)+2\epsilon.
    \end{equation}
    En comparaison avec \eqref{EqFRYqON}, nous trouvons
    \begin{equation}
        \omega_{\tilde u(h)}\leq \omega_u(h).
    \end{equation}
    Les fonction s\( u\) et \( \tilde u\) ayant le même module de continuité, le lemme \ref{LemeERapq} nous enseigne que l'une est uniformément continue si et seulement si l'autre l'est. Vu que \( u\) est uniformément continue par hypothèse, le prolongement \( \tilde u\) est uniformément continu.
\end{proof}

Une conséquence du théorème de prolongement est le théorème suivant qui permet de compléter un espace métrique.
\begin{theorem}
    Tout espace métrique se plonge par une isométrie à image dense dans un espace métrique complet. De plus ce dernier est unique à isométrie près.
\end{theorem}
Cet espace est nommé le \defe{complété}{complété!espace métrique}.

%+++++++++++++++++++++++++++++++++++++++++++++++++++++++++++++++++++++++++++++++++++++++++++++++++++++++++++++++++++++++++++
					\section{Un petit extra}
%+++++++++++++++++++++++++++++++++++++++++++++++++++++++++++++++++++++++++++++++++++++++++++++++++++++++++++++++++++++++++++

Soit $f$ une fonction de $\eR$ dans $\eR$. Supposons que 
\begin{enumerate}

\item		\label{ItemExtrai}
$f(1)=1$,

\item		\label{ItemExtraii}
$f(x+y)=f(x)+f(y)$ pour tout réels $x$ et $y$.

\end{enumerate}
Nous pouvons montrer\footnote{et toi, tu le peux ?} que la seule fonction {\it continue} qui possède ces propriétés est la fonction identité $f(x)=x$ pour tout $x\in\eR$.

De la même manière, il est aisé de voir que les seules applications linéaires de $\Rn$ dans $\Rn$ sont de la forme 
\begin{equation}
	f(x)=Ax
\end{equation}
pour une constante réelle $A$. Une question naturelle qu'on peut alors se poser est la suivante: 
\begin{quote} 
	Est-il possible de définir une fonction non continue ayant les propriétés \ref{ItemExtrai} et \ref{ItemExtraii} ?
\end{quote}
En fait, il est possible de démontrer que si $E$ est un espace vectoriel de dimension finie, alors toute application linéaire $f:E\rightarrow  F$ (où $F$ est un espace vectoriel) sera continue. Ceci ne reste plus vrai si l'espace vectoriel $E$ est de dimension infinie. Donc une manière de trouver une réponse positive à la question posée plus haut, serait de voir $\Rn$ comme espace vectoriel de dimension infinie. Après un peu de réflexion, la réponse est venue à nous (merci à Nicolas et à Samuel). 



Si nous admettons l'\href{http://fr.wikipedia.org/wiki/Axiome_du_choix}{axiome du choix}, alors nous pouvons appliquer le théorème de Zorn et nous savons que tout espace vectoriel admet une base. En particulier, l'ensemble des réels vu comme espace vectoriel sur $\Qn$ admet une base, i.e. $\exists (e_i)_{i\in I}$  des éléments de $\Rn$ tels que tout réel s'écrit comme combinaison linéaire à coefficients rationnels  de ces $e_i$, i.e.
\begin{equation}
	\forall r \in \Rn, \exists (\lambda_i)_{i\in I} \text{ des éléments de } \Qn \text{ tels que  } r = \sum_{i\in I} \lambda_i e_i.
\end{equation}
Utilisons cette base pour définir une fonction $h$ de la manière suivante.
\begin{equation}
\forall i \in I, \mbox{ on définit } h(e_i) = \alpha_i
\end{equation}
 où les $\alpha_i$ doivent être bien choisis dans $\Rn$. Pour satisfaire la propriété \ref{ItemExtrai}, choisissons sans perte de généralité $e_1 = 1$ et $h(e_1) = 1$.  Ajoutons à cette propriété la linéarité en imposant que 
\begin{equation}
h(\sum \lambda_i e_i) = \sum \lambda_i \alpha_i.
\end{equation}
Les équations (1) et (2) nous permettent de voir que, moyennant le choix des $\alpha_i$, la fonction $h$ est bien définie sur $\Rn$ et linéaire. Il est clair que si nous prenons par exemple
$$\alpha_i=e_i\;\forall i \in I$$ 
nous obtenons que la fonction $h$ est en fait la fonction identité sur $\Rn$. Par contre, si nous définissons la fonction $h$ comme satisfaisant la propriété (2) et si nous choisissons les $\alpha_i$ dans (1) de la manière suivante 
\begin{equation}
	\begin{aligned}[]
		h(e_1)	&= e_2\\
		h(e_2)	&= e_1\\
		h(e_i)	&= e_i	&&\forall i\in I\setminus\{ 1,2 \}
	\end{aligned}
\end{equation}
alors la fonction ainsi obtenue est  linéaire et bien définie mais n'est plus l'identité. Donc nous avons trouvé une application linéaire de $\Rn$ dans $\Rn$ qui n'est pas continue.  

\begin{exercice}
 Trouver d'autres exemples d'applications linéaires non continues (pas nécessairement des transformations de $\Rn$). 		
\end{exercice}
