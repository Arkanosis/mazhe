%+++++++++++++++++++++++++++++++++++++++++++++++++++++++++++++++++++++++++++++++++++++++++++++++++++++++++++++++++++++++++++
\section{Définitions, quelque exemples remarquables}        \label{SecDeExCPar}
%+++++++++++++++++++++++++++++++++++++++++++++++++++++++++++++++++++++++++++++++++++++++++++++++++++++++++++++++++++++++++++

\begin{definition}
    Un \defe{arc paramétré}{arc!paramétré} dans $\eR^p$ est un couple $(I,\gamma)$ où $I$ est un intervalle de $\eR$ et $\gamma$ est une application continue de $I$ dans $\eR^p$. Nous disons que $(I,\gamma)$ est un arc paramétré \defe{compact}{compact!arc paramétré} (ou un \defe{chemin}{chemin!dans $\eR^p$} dans $\eR^p$) lorsque $I$ est compact dans $\eR$. 
\end{definition}
L'intervalle $I$ d'un arc paramétré compact est toujours de la forme $[a,b]$, étant donné que tous les intervalles compacts de $\eR$ sont de cette forme. Un \defe{sous arc}{sous arc} de $(I,\gamma)$ est un arc de la forme $(I_0,\gamma)$ avec $I_0\subset I$.


Le grand avantage des arcs paramétrés par rapports aux graphes de fonctions est le le graphe peut «faire des retours en arrière», ou bien des auto intersections. Outre les deux exemples typiques de la la figure \ref{LabelFigExempleArcParam}, un exemple classique est la droite verticale. Les fonctions $y=ax+b$ permettent de décrire toutes les droites, sauf les droites verticales. Dans le cadre des courbes paramétrées, les droites verticales et horizontales sont sur pied d'égalité. Quelque exemples classiques :
\begin{description}
    \item[Droite horizontale] Une droite horizontale à la hauteur $a$ est donnée par la courbe paramétrée $\gamma(t)=(t,a)$, avec $t\in I=\eR$.
    \item[Droite verticale] Une droite verticale à la distance $b$ de l'origine est donnée par la courbe paramétrée $\gamma(t)=(b,t)$, avec $t\in I=\eR$.
    \item[Graphe d'une fonction]\label{PgGrqFnGamma} Le graphe d'une fonction $f\colon \eR\to \eR$ est donné par l'arc $\gamma(t)=\big( t,f(t) \big)$. 
    \item[Un cercle] Le cercle de rayon $R$ est donné par l'arc $\gamma(t)=\big( R\cos(t),R\sin(t) \big)$.
\end{description}

\newcommand{\CaptionFigExempleArcParam}{Des exemples d'arcs paramétrées. Ceux ne sont pas des graphes.}
\input{Fig_ExempleArcParam.pstricks}

\begin{remark}
    Afin d'alléger la notation, nous allons le plus souvent désigner l'arc $(I,\gamma)$ simplement par la fonction $\gamma$. Il est cependant toujours \emph{très} important de savoir sur quel intervalle nous considérons le chemin. Cela dépendra le plus souvent du contexte, et nous indiquerons l'intervalle $I$ explicitement lorsqu'une ambigüité est à craindre.

    Par exemple, lorsque nous considérons le cercle $\gamma(t)=\big( R\cos(t),R\sin(t) \big)$, le plus souvent l'intervalle de variation de $t$ sera $I=\mathopen[ 0 , 2\pi \mathclose]$. Par contre, si nous considérons la droite $\gamma(t)=(t,2t)$, l'intervalle de variation de $t$ sera naturellement $I=\eR$.
\end{remark}

%+++++++++++++++++++++++++++++++++++++++++++++++++++++++++++++++++++++++++++++++++++++++++++++++++++++++++++++++++++++++++++
\section{Longueur d'arc}        \label{SecLongArc}
%+++++++++++++++++++++++++++++++++++++++++++++++++++++++++++++++++++++++++++++++++++++++++++++++++++++++++++++++++++++++++++

Nous voulons définir et étudier la notion de \wikipedia{fr}{Arc_rectifiable}{longueur} d'un arc paramétré. Pour cela, le plus raisonnable est d'approcher l'arc par des petits segments de droites (dont les longueurs sont évidentes), et d'extraire la «meilleure» approximation.

Une des notions clefs pour la suite est celle de subdivision d'intervalles. Cette notion sera encore utilisée par la suite à propos des intégrales.
\begin{definition}      \label{DefSubdivisionIntervalle}
    Si $I$ est un intervalle d'extrêmes $a$ et $b$ avec $a<b$, nous appelons \defe{subdivision finie}{subdivision!d'un intervalle} de $I$ un choix de nombres $t_i$ tels que
    \begin{equation}
        a=t_0<t_1<\ldots<t_n=b.
    \end{equation}
    Nous disons qu'une subdivision $\sigma'$ est \defe{plus fine}{fine!subdivision} que la subdivision $\sigma$ si l'ensemble des points de $\sigma$ est inclus à celui des points de $\sigma'$. Dans ce cas, la subdivision $\sigma'$ est un \defe{raffinement}{raffinement} de $\sigma$. Nous désignons par $\sdS(I)$ l'ensemble des subdivisions finies de l'intervalle $I$.
\end{definition}
Dans la suite, toutes les subdivisions que nous considérons seront des subdivisions finies. Aussi nous parlerons simplement de \emph{subdivisions} sans préciser. Nous allons souvent noter $\sigma=(t_i)_{i=1}^n$ pour désigner la subdivision formée par les nombres $t_i$. Il faut garder en tête que dans une subdivision, les nombres \emph{sont ordonnés}.

\newcommand{\CaptionFigCourbeRectifiable}{La longueur d'un découpage. La somme des longueurs des segments droits est facile à calculer.}
\input{Fig_CourbeRectifiable.pstricks}
\begin{definition}
    
    Soit un arc paramétré compact $(I,\gamma)$ et une subdivision $\sigma=(t_i)_{i=1}^n$ de $I=[a,b]$. À partir de $\gamma$ et du découpage $\sigma$ nous définissons le nombre (voir figure \ref{LabelFigCourbeRectifiable})
    \begin{equation}        \label{Eqlsigmagammasss}
        l_{\sigma}(\gamma)=\sum_{i=1}^n\big\| \gamma(t_i)-\gamma(t_{i-1}) \big\|.
    \end{equation}
    On appelle \defe{longueur}{longueur!d'un arc paramétré compact} de l'arc $\gamma$ le nombre
    \begin{equation}
        l(\gamma)=\sup_{\sigma}l_{\sigma}\in\mathopen[ 0 , \infty \mathclose].
    \end{equation}
    Nous disons que $\gamma$ est \defe{rectifiable}{rectifiable} lorsque $l(\gamma)<\infty$.
\end{definition}
Lorsque nous voulons spécifier sur quel intervalle nous considérons l'arc, nous noterons $l(I,\gamma)$ au lieu de $l(\gamma)$ pour être plus précis.

Par l'inégalité triangulaire, si $\sigma_1$ est plus fine que $\sigma$, nous avons
\begin{equation}
    l_{\sigma}(\gamma)\leq l_{\sigma_1}(\gamma),
\end{equation}
Comme cela peut être vu sur la figure \ref{LabelFigArcLongueurFinesse}.
\newcommand{\CaptionFigArcLongueurFinesse}{Il est visible que la longueur donnée par l'approximation par des petits segments (verts) est plus longue et plus précise que celle donnée par les longs segments (rouge).}
\input{Fig_ArcLongueurFinesse.pstricks}
%TODO : questa figura e' invisibile quando stampiamo il pdf. 

Dans la vie réelle, il est souvent difficile et peu pratique de calculer le supremum «à la main». C'est pourquoi nous allons travailler à exprimer la longueur d'un arc à l'aide d'une intégrale (théorème \ref{ThoLongueurIntegrale}).

\begin{lemma}
    Nous avons $l(\gamma)=0$ si et seulement si $\gamma(t)$ est un vecteur constant.
\end{lemma}

\begin{proof}
    Si l'application $\gamma(t)$ est constante, le résultat est évident. Supposons maintenant que $\gamma$ ne soit pas constante. Cela signifie qu'il existe $t_1$ et $t_2$ dans $I$ tels que $\gamma(t_1)\neq \gamma(t_2)$. Dans ce cas, si nous prenons le découpage $\sigma=\{ a,t_1,t_2,b \}$, la somme \eqref{Eqlsigmagammasss} contient au moins le terme non nul $\| \gamma(t_2)-\gamma(t_1) \|$, et donc $l_{\sigma}(\gamma)>0$. Par définition du supremum, nous avons alors $l(\gamma)\geq l_{\sigma}(\gamma)>0$.
\end{proof}

\begin{proposition}     \label{Propletautredecop}
    Soit $(I,\gamma)$ un arc paramétré compact.
    \begin{enumerate}
        \item
            Si $\gamma'=(I',\gamma)$ avec $I'\subset I$, alors $l(\gamma')\leq l(\gamma)$.
        \item
            Soit $c\in\mathopen[ a , b \mathclose]$, et considérons les arcs $\gamma_1=\big( \mathopen[ a , c \mathclose],\gamma \big)$ et $\gamma_2=\big( \mathopen[ c , b \mathclose],\gamma \big)$. Alors 
            \begin{equation}
                l(\gamma)=l(\gamma_1)+l(\gamma_2).
            \end{equation}
            En particulier, $\gamma$ est rectifiable si et seulement si $\gamma_1$ et $\gamma_2$ le sont.
    \end{enumerate}
\end{proposition}

\begin{proof}
    \begin{enumerate}
        \item
            Nous notons $I=\mathopen[ a , b \mathclose]$ et $I'=\mathopen[ a' , b' \mathclose]$. Étant donné que $I'\subset I$, nous avons
            \begin{equation}
                a\leq a'<b'\leq b.
            \end{equation}
            Pour chaque subdivision $\sigma_0:a'=t_0<t_1<\ldots<t_n=b'$ de $I'$, nous pouvons construire une subdivision de $I$ en «ajoutant» les points $a$ et $b$, c'est à dire
            \begin{equation}
                \sigma:a\leq t_0<\ldots<t_n\leq b.
            \end{equation}
            Si nous calculons $l_{\sigma}(\gamma)$, nous avons tous les termes qui arrivent dans $l_{\sigma_0}(\gamma')$ plus le premier et dernier terme : $\| \gamma(t_0)-\gamma(a) \|$ et $\| \gamma(b)-\gamma(t_n)\|$. Nous avons donc
            \begin{equation}
                l_{\sigma_0}(\gamma')\leq l_{\sigma}(\gamma)\leq\sup_{\sigma}l_{\sigma}(\gamma)=l(\gamma).
            \end{equation}
            Étant donné que pour toute subdivision $\sigma_0$ nous avons $l_{\sigma_0}(\gamma')\leq l(\gamma)$, en prenant le supremum sur les subdivision $\sigma_0$ de $I'$, nous avons comme annoncé
            \begin{equation}
                l(\gamma')\leq l(\gamma).
            \end{equation}
        \item
            Soit $\sigma=\{ t_i \}$ une subdivision de $\mathopen[ a , b \mathclose]$. Nous considérons les subdivision $\sigma_1$ et $\sigma_2$ définies comme suit:
            \begin{equation}
                \begin{aligned}[]
                    \sigma_1&:\{ t_i\tq t_i< c \}\cup\{ c \},\\
                    \sigma_2&:\{ t_i\tq t_i> c \}\cup\{ c \}.
                \end{aligned}
            \end{equation}
            L'inégalité triangulaire implique que
            \begin{equation}
                l_{\sigma}(\gamma)\leq l_{\sigma\cup\{ c \}}(\gamma)=l_{\sigma_1}(\gamma_1)+l_{\sigma_2}(\gamma_2)\leq l(\gamma_1)+l(\gamma_2).
            \end{equation}
            Nous avons donc 
            \begin{equation}    \label{EqIneglglglgud}
                l(\gamma)\leq l(\gamma_1)+l(\gamma_2).
            \end{equation}

            Nous prouvons maintenant l'inégalité inverse. Soit $\varepsilon>0$. Étant donné que $l(\gamma_1)$ est le supremum des quantités $l_{\sigma_1}(\gamma_1)$ lorsque $\sigma_1$ parcours toutes les subdivision possibles, il existe une partition $\sigma_1^{\varepsilon}$ telle que (idem pour $\gamma_2$)
            \begin{equation}        \label{EqAllsigmaepsgammaufd}
                \begin{aligned}[]
                    l_{\sigma_1^{\varepsilon}}(\gamma_1)+\frac{ \varepsilon }{2}>l(\gamma_1),\\
                    l_{\sigma_2^{\varepsilon}}(\gamma_2)+\frac{ \varepsilon }{2}>l(\gamma_2),
                \end{aligned}
            \end{equation}
            où $\sigma_1^{\varepsilon}$ est une subdivision de $\mathopen[ a , c \mathclose]$ et $\sigma_2^{\varepsilon}$ en est une de $\mathopen[ c , b \mathclose]$. En faisant la somme des deux équations \eqref{EqAllsigmaepsgammaufd}, nous trouvons
            \begin{equation}
                l(\gamma_1)+l(\gamma_2)<l_{\sigma_1^{\varepsilon}}(\gamma_1)+l_{\sigma_2^{\varepsilon}}(\gamma_2)+\varepsilon=l_{\sigma_1^{\varepsilon}\cup\sigma_2^{\varepsilon}}(\gamma)\leq l(\gamma)+\varepsilon.
            \end{equation}
            L'inégalité $l(\gamma_1)+l(\gamma_2)<l(\gamma)+\varepsilon$ étant valable pour tout $\varepsilon$, nous avons
            \begin{equation}
                l(\gamma_1)+l(\gamma_2)\leq l(\gamma).
            \end{equation}
            Cette inégalité, combinée avec l'inégalité \eqref{EqIneglglglgud}, donne bien $l(\gamma)=l(\gamma_1)+l(\gamma_2)$.
    \end{enumerate}
\end{proof}

%+++++++++++++++++++++++++++++++++++++++++++++++++++++++++++++++++++++++++++++++++++++++++++++++++++++++++++++++++++++++++++
\section{Abscisse curviligne}
%+++++++++++++++++++++++++++++++++++++++++++++++++++++++++++++++++++++++++++++++++++++++++++++++++++++++++++++++++++++++++++

\begin{definition}
    Soit $(I,\gamma)$ un arc rectifiable compact avec $I=\mathopen[ a , b \mathclose]$. L'application
    \begin{equation}
        \begin{aligned}
            \varphi\colon \mathopen[ a , b \mathclose]&\to \eR^+ \\
            t&\mapsto l\big( \mathopen[ a , t \mathclose],\gamma \big) 
        \end{aligned}
    \end{equation}
    est la \defe{longueur d'arc}{longueur d'arc} de $\gamma$.
\end{definition}
Cette fonction nous permet de calculer la distance (suivant la courbe) entre deux point arbitraires parce que si $a\leq t<u\leq b$, nous avons
\begin{equation}
    l\big( [t,u],\gamma \big)=\varphi(u)-\varphi(t).
\end{equation}
En effet,
\begin{equation}
    \varphi(u)-\varphi(t)=l\big( [a,u],\gamma \big)-l\big( [a,t],\gamma \big),
\end{equation}
mais en utilisant la proposition \ref{Propletautredecop}, nous avons
\begin{equation}
    l\big( [a,u],\gamma \big)=l\big( [a,t],\gamma \big)+l\big( [t,u],\gamma \big).
\end{equation}

\begin{proposition}
    La longueur d'arc d'un arc rectifiable compact est une fonction continue et croissante.
\end{proposition}

\begin{proof}
    Soit $(I,\gamma)$ un arc paramétré rectifiable compact avec $I=[a,b]$. Afin de montrer que $\varphi$ est croissante, prenons $t\in I$ ainsi que $h>0$ et montrons que $\varphi(t+h)\geq \varphi(t)$. La proposition \ref{Propletautredecop} implique que 
    \begin{equation}
        l\big( \mathopen[ a , t+h \mathclose],\gamma \big)=l\big( \mathopen[ a ,t  \mathclose],\gamma \big)+l\big( \mathopen[ t , t+h \mathclose],\gamma \big),
    \end{equation}
    c'est à dire 
    \begin{equation}
        \varphi(t+h)=\varphi(t)+l\big( \mathopen[ t , t+h \mathclose],\gamma \big)\geq \varphi(t).
    \end{equation}

    Pour la continuité, soit $t$ fixé dans $\mathopen[ a , b \mathclose]$ et $\varepsilon>0$. Il nous faut démontrer qu'il existe $\eta>0$ tel que si $s$ est dans $[0,\eta]$ alors 
\[
|\varphi(t+s)-\varphi(t)|\leq \varepsilon, \qquad \forall t \in [a,b].
\] 
Étant donné que $l\big( \mathopen[ t , b \mathclose],\gamma \big)$ est le supremum des $l_{\sigma}\big( \mathopen[ t , b \mathclose],\gamma \big)$, il existe une subdivision $\sigma:t,t_1,\cdots,t_{n-1},b$ telle que
    \begin{equation}
        l_{\sigma}\big( \mathopen[ t , b \mathclose],\gamma \big)>l\big( \mathopen[ t , b \mathclose],\gamma \big)-\frac{ \varepsilon }{2}=\varphi(b)-\varphi(t)-\frac{ \varepsilon }{2}.
    \end{equation}
    La continuité de $\gamma$ implique qu'il existe un $\eta$ tel que
    \begin{equation}
        s\in\mathopen[ 0 , \eta \mathclose]\Rightarrow\| \gamma(t+s)-\gamma(t) \|<\frac{ \varepsilon }{2}
    \end{equation}
    Quitte à prendre $\eta$ encore plus petit, nous supposons que $t+\eta<t_1$. Soit $s\in\mathopen[ 0 , \eta \mathclose]$ et considérons la subdivision de $\mathopen[ t , b \mathclose]$ donnée par $\sigma'=\sigma\cup\{ t+s \}$. Étant donné que $\sigma'$ est plus fine que $\sigma$, le nombre $l_{\sigma}\big( \mathopen[ t , b \mathclose],\gamma \big)$ est inférieur ou égal à $l_{\sigma'}\big( \mathopen[ t , b \mathclose],\gamma \big)$. Nous avons donc les inégalités
    \begin{equation}
        \begin{aligned}[]
            \varphi(b)-\varphi(t)-\frac{ \varepsilon }{2}&\leq l_{\sigma}\big( \mathopen[ t , b \mathclose],\gamma \big)\\
            &\leq l_{\sigma'}\big( \mathopen[ t , b \mathclose],\gamma \big)\\
            &= \big\| \gamma(t+s)-\gamma(t) \big\|+l_{\sigma'\setminus\{ t \}}\big( \mathopen[ t+s , b \mathclose]\gamma \big)\\
            &\leq\| \gamma(t+s)-\gamma(t) \|+\varphi(b)-\varphi(t+s)\\
            &\leq \frac{ \varepsilon }{2}+\varphi(b)-\varphi(t+s).
        \end{aligned}
    \end{equation}
    Au final, nous avons trouvé que
    \begin{equation}
        \varphi(t+s)-\varphi(t)\leq\varepsilon,
    \end{equation} 
    ce qui prouve que $\varphi$ est continue au point $t$.
\end{proof}

En guise de paramètre sur un arc, nous pouvons utiliser la longueur d'arc elle-même. En effet si $(I,\gamma)$ est un arc de longueur $l$, nous pouvons donner le même arc avec le couple $\big( \mathopen[ 0 , l \mathclose],g \big)$ où $g$ est la fonction qui au réel $s$ fait correspondre l'élément $\gamma\big( \varphi^{-1}(s) \big)$ de $\eR^n$. Dire
\begin{equation}
    P=(\gamma\circ\varphi^{-1})(s)
\end{equation}
revient à dire que le point $P$ est le point sur la courbe sur lequel on tombe après avoir marché une distance $s$ sur la courbe.

Nous allons revenir sur ce «changement de paramètre» plus tard, en particulier dans la section \ref{SecArcGeometrique}.


\begin{theorem}     \label{ThoLongueurIntegrale}
    Soit $(I,\gamma)$ un arc paramétré compact de classe $\mathcal{C}^1$. Alors $\gamma$ est rectifiable et
    \begin{equation}        \label{EqLongGammalInt}
        l(\gamma)=\int_a^b\| \gamma'(t) \|dt,
    \end{equation}
    où $I=\mathopen[ a , b \mathclose]$.
\end{theorem}

\begin{proof}
    Si $\sigma=\{ t_i \}$ est une subdivision de l'intervalle $\mathopen[ a , b \mathclose]$, alors
    \begin{equation}
        \begin{aligned}[]
            l_{\sigma}(\gamma)&=\sum_{i=1}^n\| \gamma(t_i)-\gamma(t_{i-1}) \|\\
                &=\sum_{i=1}^n\| \int_{t_{i-1}}^{t_i}\gamma'(t)dt \|\\
                &\leq\sum_{i=1}^n\int_{t_{i-1}}^{t_i}\| \gamma'(t) \|dt\\
                &=\int_a^b\| \gamma'(t) \|dt.
        \end{aligned}
    \end{equation}
    Cela prouve déjà que 
    \begin{equation}        \label{Eq_0208lsigsigmmintifp}
        l(\gamma)=\sup_{\sigma}l_{\sigma}(\gamma)\leq\int_a^b\| \gamma'(t) \|dt.
    \end{equation}
    Nous devons maintenant prouver l'inégalité inverse.

    Notons $\varphi$ l'abscisse curviligne $\varphi(t)=l\big( \mathopen[ a , t \mathclose],\gamma \big)$. Cette dernière vérifie
    \begin{equation}
        \varphi(t+h)-\varphi(t)=l\big( \mathopen[ t , t+h \mathclose],\gamma \big)\geq \| \gamma(t+h)-\gamma(t) \|,
    \end{equation}
    et en particulier
    \begin{equation}     \label{Eq_0208intervpvpintfrach}
        \left\| \frac{ \gamma(t+h)-\gamma(t) }{ h } \right\|\leq \frac{ \varphi(t+h)-\varphi(t) }{ h }.
    \end{equation}
    D'autre part, en utilisant \eqref{Eq_0208lsigsigmmintifp} sur le segment $\mathopen[ t , t+h \mathclose]$, nous avons
    \begin{equation}
        \varphi(t+h)-\varphi(t)=l\big( \mathopen[ t , t+h \mathclose],\gamma \big)\leq\int_{t}^{t+h}\| \gamma'(u) \|du.
    \end{equation}
    Cela nous permet de continuer l'inéquation \eqref{Eq_0208intervpvpintfrach} en
    \begin{equation}
        \left\| \frac{ \gamma(t+h)-\gamma(t) }{ h } \right\|\leq\frac{ \varphi(t+h)-\varphi(t) }{ h }\leq\frac{1}{ h }\int_t^{t+h}\| \gamma'(u) \|du.
    \end{equation}
    Prenons la limite $h\to 0$. À gauche nous reconnaissons la formule de la dérivée, et nous obtenons $\| \gamma'(t) \|$; au centre nous avons $\varphi'(t)$ et à droite, si $n(u)$ représente une primitive de la fonction $u\mapsto\| \gamma'(u) \|$,
    \begin{equation}
        \lim_{h\to 0}\frac{ n(t+h)-n(t) }{ h }=n'(t)=\| \gamma'(t) \|.
    \end{equation}
    Au final, 
    \begin{equation}
        \| \gamma'(t) \|\leq \varphi'(t)\leq\| \gamma'(t) \|,
    \end{equation}
    c'est à dire $\varphi'(t)=\| \gamma'(t) \|$ et donc
    \begin{equation}
        \varphi(t)-\varphi(a)=\int_a^t\| \gamma'(u) \|du.
    \end{equation}
    Par construction de la longueur d'arc, $\varphi(a)=0$ et en posant $t=b$ nous obtenons la relation recherchée:
    \begin{equation}
        l(\gamma)=\varphi(b)=\int_a^b\| \gamma'(u) \|du.
    \end{equation}
\end{proof}

\begin{remark}  \label{RemLongIntUn}
    Nous verrons avec la définition \eqref{EqhJGRcb} que la longueur de la courbe est l'intégrale de la fonction constante \( 1\) le long de la courbe.

    Nous verrons que cela est un fait général : l'intégrale de la fonction constante \( 1\) sur quelque chose est la mesure du quelque chose. Tellement que nous en ferons la définition \ref{DefMesureInt}.
\end{remark}

\begin{example}
Soient donc $a$ et $b$ deux points de $\eR^m$, et $\gamma$ la droite joignant $a$ à $b$, c'est à dire
\begin{equation}
    \gamma(t)=(1-t)a+tb
\end{equation}
avec $t\in\mathopen[ 0 , 1 \mathclose]$. Le théorème \ref{ThoLongueurIntegrale} nous enseigne que la longueur de ce chemin est
\begin{equation}
    l\big( [0,1],\gamma \big)=\int_0^1\| \gamma'(t) \|dt=\int_0^1\| -a+b \|=\| b-a \|,
\end{equation}
qui est bien la distance entre $a$ et $b$.
\end{example}

%+++++++++++++++++++++++++++++++++++++++++++++++++++++++++++++++++++++++++++++++++++++++++++++++++++++++++++++++++++++++++++
\section{Élément de longueur}
%+++++++++++++++++++++++++++++++++++++++++++++++++++++++++++++++++++++++++++++++++++++++++++++++++++++++++++++++++++++++++++

%---------------------------------------------------------------------------------------------------------------------------
\subsection{Élément de longueur : cartésiennes}
%---------------------------------------------------------------------------------------------------------------------------

Étant donné que la longueur d'arc d'une courbe paramétrée $(I,\gamma)$ est donnée par l'intégrale de $\| \gamma'(t) \|$, il est naturel d'appeler le nombre $\| \gamma'(t) \|\,dt$, \defe{l'élément de longueur}{longueur!élément de} de la courbe $\gamma$ au point $\gamma(t)$.

En coordonnées cartésiennes dans le plan, une courbe paramétré est donnée par 
\begin{equation}
    \gamma(t)=\big( x_1(t),x_2(t) \big),
\end{equation}
et l'élément de longueur est
\begin{equation}        \label{EqElLongCart}
    \| x'(t) \|\, dt =\sqrt{(x_1')^2+(x_2')^2} \, dt.
\end{equation}

%---------------------------------------------------------------------------------------------------------------------------
\subsection{Élément de longueur : polaires (1)}
%---------------------------------------------------------------------------------------------------------------------------

En coordonnées polaires, une courbe est donnée par
\begin{equation}
    \gamma(t)=\big( \rho(t),\theta(t) \big),
\end{equation}
et le passage aux cartésiennes se fait via les formules
\begin{subequations}
    \begin{numcases}{}
        x(t)=\rho(t)\cos\big( \theta(t) \big)\\
        y(t)=\rho(t)\sin\big( \theta(t) \big).
    \end{numcases}
\end{subequations}
L'élément de longueur se trouve directement en remplaçant $x(t)$ et $y(t)$ dans la formule \eqref{EqElLongCart}. Les dérivées sont données par
\begin{equation}
    \begin{aligned}[]
        x'(t)&=\rho'(t)\cos\theta(t)-\rho(t)\theta'(t)\sin\theta(t)\\
        y'(t)&=\rho'(t)\sin\theta(t)+\rho(t)\theta'(t)\cos\theta(t),
    \end{aligned}
\end{equation}
et un calcul montre que
\begin{equation}        \label{EqElLongEnPolaires}
    \big( x'(t) \big)^2+\big( y'(t) \big)^2=\big( \rho'(t) \big)^2+\big( \rho(t) \big)^2\big( \theta'(t) \big)^2.
\end{equation}

Nous reviendrons plus en détail sur le concept de changement de paramétrisation (ici, les polaires) à la section \ref{SecArcGeometrique}.

%---------------------------------------------------------------------------------------------------------------------------
\subsection{Élément de longueur : polaires (2)}
%---------------------------------------------------------------------------------------------------------------------------

Parfois on utilise $\theta$ comme paramètre. L'équation de la courbe est alors donnée en coordonnées polaires sous la forme
\begin{equation}        \label{Eqgenereformepolaire}
    \rho(\theta)=f(\theta),
\end{equation}
où $f$ est une fonction réelle et  il faut comprendre que nous parlons de la courbe $\big( \rho(\theta),\theta \big)$ en coordonnées polaires. En coordonnées cartésiennes, cette courbe est donnée par
\begin{subequations}        \label{EqPolaireSemiGen}
    \begin{numcases}{}
        x(t)=\rho(t)\cos(t)\\
        y(t)=\rho(t)\sin(t)
    \end{numcases}
\end{subequations}
avec $t$ qui parcours le plus souvent l'intervalle $\mathopen[ 0 , 2\pi \mathclose]$. Notez qu'il se peut que le domaine ne soit pas toujours $\mathopen[ 0 , 2\pi \mathclose]$; cela peut dépendre des circonstances. Quoi qu'il en soit, la donnée du domaine fait partie de la donnée d'une courbe, et il ne peut donc pas y avoir d'équivoques à ce niveau.

Nous utilisons à nouveau la formule \eqref{EqElLongCart} en y mettant les valeurs \eqref{EqPolaireSemiGen} :
\begin{subequations}
    \begin{numcases}{}
        x'(t)=\rho'(t)\cos(t)-\rho(t)\sin(t)\\
        y'(t)=\rho'(t)\sin(t)+\rho(t)\cos(t),
    \end{numcases}
\end{subequations}
et
\begin{equation}        \label{EqElemOngPOldeux}
    \big( x'(t) \big)^2+\big( y'(t) \big)^2=\rho'(t)^2+\rho(t)^2.
\end{equation}
% position 55702
%Si vous avez bien compris ce passage, vous pouvez jeter un œil à l'exercice \ref{exoGeomAnal-0004}.

\begin{remark}
    N'oubliez pas, en utilisant ces formules, que ce qui rentre dans l'intégrale est la \emph{racine carré} de $(x')^2+(y')^2$.
\end{remark}

\begin{example}     \label{ExempleLongCercle}
    Calculons la circonférence du cercle. En coordonnées polaires, le graphe du cercle correspond à l'équation
    \begin{equation}
        \big( \rho(t),\theta(t) \big)=(R,t)
    \end{equation}
    où $R$ est constante (le rayon du cercle) et $t$ va de $0$ à $2\pi$. En substituant dans l'équation \eqref{EqElLongEnPolaires}, l'élément de longueur à intégrer est seulement
    \begin{equation}
        \sqrt{R^2}=R
    \end{equation}
    parce que $\rho'(t)=0$ et $\theta'(t)=1$. La longueur du cercle est alors directement donnée par
    \begin{equation}
        l=\int_0^{2\pi}Rdt=2\pi R.
    \end{equation}
    
    Nous pouvions aussi faire le calcul en coordonnées cartésiennes. Alors la courbe est donnée par les équations
    \begin{equation}
        \begin{aligned}[]
            x(t)&=R\cos(t)\\
            y(t)&=R\sin(t)
        \end{aligned}
    \end{equation}
    et $t\in\mathopen[ 0 , 2\pi \mathclose]$. La circonférence du cercle est alors 
    \begin{equation}
        l=\int_0^{2\pi}\sqrt{R^2\sin^2(t)+R^2\cos^2(t)}\,dt=\int_0^{2\pi}R\,dt=2\pi R.
    \end{equation}
\end{example}
\begin{remark}
  Il faut bien comprendre que quand on parle de courbes paramétrées en  coordonnées cartésiennes on pense à une courbe dont le paramètre est, par exemple, $t$ et les équations de la courbe sont $(x(t), y(t))$. Cela ne veut pas dire que $x$ ou $y$ soit le paramètre. La cas où $x$ ou $y$ est le paramètre est un cas particulier qui est possible seulement pour certaines courbes et notamment pour les graphes. Le cercle de rayon $1$ n'est pas un graphe, donc si on veut utiliser $x$ ou $y$ comme paramètre il faut d'abord découper la courbe en deux morceaux, par exemple, la moitié inférieure ($y<0$) et la moitié supérieure ($y>0$).  
\end{remark}
\begin{example}     \label{ExCycloLong}
    Une \defe{cycloïde}{cycloïde!longueur} est une courbe paramétrée par
    \begin{subequations}
        \begin{numcases}{}
            x(t)=a(t-\sin(t))\\
            y(t)=a(1-\cos(t))
        \end{numcases}
    \end{subequations}
    avec $a>0$ et $t\in\eR$. Comme montré sur la figure \ref{LabelFigCycloideA}, la cycloïde donne lieu à un graphe périodique. Il est possible de montrer (le faire) que le premier arc correspond à $t\in\mathopen[ 0 , 2\pi \mathclose]$. Nous voulons donc calculer la longueur de l'arc sur cet intervalle.
    \newcommand{\CaptionFigCycloideA}{La cycloïde de paramètre $a=1$ entre $0$ et $4\pi$.}
    \input{Fig_CycloideA.pstricks}

    Nous avons $x'(t)=a(1-\cos(t))$ et $y'(t)=a\sin(t)$, de telle façon à ce que
    \begin{equation}    \label{Eq_0508dlcycloide}
        \sqrt{(x')^2+(y')^2}=a\sqrt{2-2\cos(t)}=a\sqrt{4\sin^2\left( \frac{ t }{ 2 } \right)}=2a\Big| \sin\frac{ t }{2} \Big|.
    \end{equation}
    La longueur est donc donnée par
    \begin{equation}
        \int_0^{2\pi}2a| \sin\frac{ t }{2} | dt=4a\int_0^{\pi}\sin(t)dt=8a.
    \end{equation}
    
\end{example}

\begin{example}
    La \defe{cardioïde}{cardioïde} est la courbe donnée par
    \begin{equation}        \label{EqCardioide}
        \rho(\theta)=a(1+\cos(\theta)).
    \end{equation}
    avec $\theta\in\mathopen[ -\pi , \pi \mathclose]$. Le nom de cette courbe provient de son graphe illustré à la figure \ref{LabelFigCardioid}.
    \newcommand{\CaptionFigCardioid}{Une cardioïde, $\rho=1+\cos(\theta)$.}
    \input{Fig_Cardioid.pstricks}

    L'équation \eqref{EqCardioide} est donnée sous la forme \eqref{Eqgenereformepolaire}, c'est à dire que $\theta(t)=t$ et $\theta'(t)=1$, et par conséquent l'élément de longueur est donné par
    \begin{equation}
        \begin{aligned}[]
            (\rho')^2+(\rho)^2&=\big( -a\sin(\theta) \big)^2+a^2\big( 1+\cos(\theta) \big)^2\\
                    &=a^2\sin^2(\theta)+a^2\big( 1+2\cos(\theta)+\cos^2(\theta) \big)\\
                    &=a^2\big( 1+1+2\cos(\theta) \big)\\
                    &=2a^2\big( 1+\cos(\theta) \big)\\
                    &=4a^2\cos^2\frac{ \theta }{2}.
        \end{aligned}
    \end{equation}
    La longueur d'arc est donc donnée par
    \begin{equation}
        l=\int_{-\pi}^{\pi}2a\cos\frac{ \theta }{2}d\theta=2a\int_{-\pi/2}^{\pi/2}\cos(t)2dt=8a.
    \end{equation}
\end{example}

%---------------------------------------------------------------------------------------------------------------------------
\subsection{Approximation de la longueur par des cordes}
%---------------------------------------------------------------------------------------------------------------------------

\begin{definition}
    Soit un arc paramétré $(I,\gamma)$. Un point $t\in I$ est dit \defe{régulier}{régulier!point d'un arc} si $\gamma'(t)\neq 0$, et il est dit \defe{critique}{critique!point d'un arc} si $\gamma'(t)=0$. Le point $t\in I$ est dit \defe{\href{http://c.caignaert.free.fr/chapitre15/node1.html}{birégulier}}{birégulier!point sur une courbe} si les vecteurs $\gamma'(t)$ et $\gamma''(t)$ sont linéairement indépendants et non nuls. 
    
    Par extension, nous dirons également que le point $\gamma(t)$ lui-même est régulier, critique ou birégulier. Un arc est dit \emph{régulier}\index{régulier!arc} lorsque tous ses points sont réguliers.
\end{definition}

Nous savons que la longueur d'une courbe est donné par le supremum sur toutes les subdivision de la longueur des cordes correspondantes. De plus l'inégalité triangulaire nous enseigne que plus la subdivision est fine, plus la longueur sera grande. Il est donc naturel de penser que sur un petit intervalle, la longueur de la courbe ne doit pas être très différente de la longueur de la corde correspondante. 

La proposition suivante est un énoncé précis et quantitatif de ce fait.
\begin{proposition}
    Soit $(I,\gamma)$ un arc de classe $\mathcal{C}^1$ et $t_0\in I$ un point régulier (c'est à dire $\gamma'(t_0)\neq 0$). Alors pour tout $\varepsilon>0$, il y a un $\delta>0$ tel que on trouve  $t,t'\in I\cap(t_0,\delta)$ tels que
    \begin{equation}
        \left| \int_t^{t'}\| \gamma'(u) \|du-\| \gamma(t)-\gamma(t') \| \right| \leq 2\varepsilon| t-t' |.
    \end{equation}
\end{proposition}
Intuitivement, cette proposition signifie qu'au voisinage de $t_0$, la longueur d'arc est équivalente à celle de la corde. 

\begin{proof}
    Par la continuité de $\gamma'$ (parce que $\gamma$ est $\mathcal{C}^1$), pour tout $\varepsilon$, il existe un $\delta$ tel que 
    \begin{equation}
        | t-t_0 |<\delta\Rightarrow\big\| \gamma'(t)-\gamma'(t_0) \big\|\leq \varepsilon.
    \end{equation}
    Nous considérons la fonction 
    \begin{equation}
        u\mapsto \gamma(u)-\gamma(t_0)-(u-t_0)\gamma'(t_0),
    \end{equation}
    dont la dérivée (par rapport à $u$) est
    \begin{equation}
        \gamma'(u)-\gamma'(t_0).
    \end{equation}
    Nous y appliquons la formule des accroissements finis entre $t$ et $t'$ choisis dans $I\cap\mathopen] t_0-\delta , t_0+\delta \mathclose[$. Il existe un $u$ entre $t$ et $t'$ tel que
    \begin{equation}
        \begin{aligned}[]
            \big\| \gamma(t)-\gamma(t_0)-(t-t_0)\gamma'(t_0)&-\gamma(t')+\gamma(t_0)+(t'-t_0)\gamma'(t_0) \big\|\\
                    &=| t-t' | \|\gamma'(u)-\gamma'(t_0) \|\\
                    &\leq\varepsilon| t-t' |.
        \end{aligned}
    \end{equation}
    En simplifiant ce qui peut être simplifié dans le membre de gauche, nous trouvons
    \begin{equation}
        \big\| \gamma(t)-\gamma(t')-(t-t')\gamma'(t_0) \big\|\leq\varepsilon| t-t' |.
    \end{equation}
    Le membre de gauche peut être minoré en utilisant la proposition \ref{PropNmNNm} :
    \begin{equation}        \label{Eq0308ffttttftt}
        \Big| \| \gamma(t)-\gamma(t')\| -\|(t-t')\gamma'(t_0) \| \Big|\leq\varepsilon| t-t' |.
    \end{equation}
    D'autre part, les inégalités \eqref{EqNleqNNleqNvqlqbs} montrent que
    \begin{equation} \label{EqNleqNNleqNvqlqbsgamma}
        -\| \gamma'(u)-\gamma'(t_0) \|\leq \| \gamma'(u) \|-\| \gamma'(t_0) \|\leq\| \gamma'(u)-\gamma'(t_0) \|.
    \end{equation}
    Si de plus $u$ est compris entre $t$ et $t'$, ces inégalités sont encore coincées entre $-\varepsilon$ et $\varepsilon$. En intégrant \eqref{EqNleqNNleqNvqlqbsgamma} par rapport à $u$ entre $t$ et $t'$, nous obtenons
    \begin{equation}
        \left| \int_t^{t'}\big\| \gamma'(u) \big\|-(t-t')\big\| \gamma'(t_0) \big\| \right| \leq\varepsilon| t-t' |.
    \end{equation}
    Afin d'alléger les notations pour la ligne suivante, nous notons $A$ le nombre positif $\int_t^{t'}\| \gamma'(u) \|du$. Nous avons
    \begin{equation}        \label{Eq0308Afffgelleqinegs}
        \begin{aligned}[]
        \Big| A-\| \gamma(t)-\gamma(t')\| \Big| &=\Big| A-| t-t' |\,\| \gamma'(t_0) \|+| t-t' |\,\| \gamma'(t_0) \|-\| \gamma(t)-\gamma(t') \| \Big| \\
                &\leq\Big|  A-| t-t' |\,\| \gamma'(t_0) \|  \Big|+\Big| | t-t' |\,\| \gamma'(t_0) \|-\| \gamma(t)-\gamma(t') \|    \Big|.
        \end{aligned}
    \end{equation}
    L'équation \eqref{Eq0308ffttttftt} montre que le second terme est plus petit ou égal à $\varepsilon| t-t' |$. En ce qui concerne le premier terme, étant donné que $A$ est positif,
    \begin{equation}
        \Big|   A-| t-t' |\,\| \gamma'(t_0) \|   \Big| \leq\Big|  A-(t-t')\,\| \gamma'(t_0) \|  \Big|\leq \varepsilon| t-t' |.
    \end{equation}
    Au final, l'inéquation \eqref{Eq0308Afffgelleqinegs} donne
    \begin{equation}
            \Big| A-\| \gamma(t)-\gamma(t')\| \Big| \leq 2\varepsilon\,| t-t' |,
    \end{equation}
    ce qu'il fallait démontrer.
\end{proof}

%+++++++++++++++++++++++++++++++++++++++++++++++++++++++++++++++++++++++++++++++++++++++++++++++++++++++++++++++++++++++++++
\section{Arc géométrique}
%+++++++++++++++++++++++++++++++++++++++++++++++++++++++++++++++++++++++++++++++++++++++++++++++++++++++++++++++++++++++++++
\label{SecArcGeometrique}

\begin{definition}      \label{DefAcrEquiva}
    Soient $(I,\gamma)$ et $(J,g)$ deux arcs paramétrés de classe $\mathcal{C}^k$. On dit qu'il sont \defe{équivalents}{equivalence@équivalence!arcs paramétrés} si il existe une bijection $\theta\colon I\to J$ de classe $\mathcal{C}^k$, d'inverse de classe $\mathcal{C}^k$ telle que $g=\gamma\circ\theta$. Nous notons $\gamma\sim g$\nomenclature[C]{$\gamma\sim g$}{Équivalence d'arcs paramétrés} lorsque $\gamma$ et $g$ sont équivalents (les ensembles $I$ et $J$ sont sous-entendus).
\end{definition}

Le passage d'une paramétrisation $(I,\gamma)$ à une autre $(J,g)$ se fait selon le diagramme suivant:
\begin{equation}
    \xymatrix{%
    I \ar[r]^{\gamma}   &   \eR^n\\
    J \ar[ru]_{g}\ar[u]^{\theta}    
       }
\end{equation}

\begin{proposition}
    La relation donnée dans la définition \ref{DefAcrEquiva} est une relation d'équivalence.
\end{proposition}

\begin{proof}
    Les trois points d'une relation d'équivalence se vérifient en utilisant le fait que $\theta$ est inversible, et que l'inverse $\theta^{-1}$ jouit des mêmes propriétés de continuité ($\mathcal{C}^k$) que $\theta$. 
    \begin{description}
        \item[Réflexivité] Nous avons $\gamma\sim \gamma$ avec $\theta=\id$.
        \item[Symétrie] Si $\gamma\sim g$, alors nous avons une application $\theta$ telle que $g=\gamma\circ\theta$, et donc $\gamma=g\circ\theta^{-1}$, ce qui montre que $g\sim \gamma$.
        \item[Transitivité] Si $\gamma\sim g$ et $g\sim h$ avec $g=\gamma\circ\theta$ et $h=g\circ\omega$, alors $h=\gamma\circ(\theta\circ\omega)$, ce qui montre que $\gamma\sim h$.
    \end{description}
\end{proof}
Si les arcs $(I,\gamma)$ et $(J,g)$ sont équivalents, les images dans $\eR^n$ sont identiques, et décrivent donc «le même dessin». Nous allons préciser cette notion plus loin. Pour cette raison les classes d'équivalences sont appelées des \defe{arcs géométriques}{arc!géométriques} (de classe $\mathcal{C}^k$).

Si $\Gamma$ est une arc géométrique, ses représentants sont dits des \defe{paramétrages admissibles}{paramétrages!admissible} ou, plus simplement \emph{paramétrisation}. On dit que l'application $\theta\colon J\to I$ est un \defe{changement de variable}{changement de variable}. Nous disons que un arc géométrique est \emph{compact} quand ses représentants sont compacts.

%Voir l'exercice \ref{exoGeomAnal-0001} position 31124

\begin{lemma}       \label{LemChamVarsStriMomnot}
    Dans le cas d'un arc $\mathcal{C}^1$, les changements de variables sont strictement monotones (croissants ou décroissants).
\end{lemma}

\begin{proof}
    Nous considérons $(I,\gamma)$ et $(J,g)$, deux paramétrisations différentes du même arc géométrique, et $\theta\in \mathcal{C}^1(J,I)$ le changement de variable. Nous allons noter $t$ la variable sur $I$ et $s$ la variable sur $J$. Par définition, $\theta\big( \theta^{-1}(t) \big)=t$, et par conséquent,
    \begin{equation}
        \theta'\big( \theta^{-1}(t) \big)(\theta^{-1})'(t)=1.
    \end{equation}
    En particulier $\theta'\big( \theta^{-1}(t) \big)$ ne s'annule pas pour aucune valeur de $t$. Mais $\theta^{-1}(t)$ peut prendre n'importe quelle valeur dans $J$, donc nous avons $\theta'(s)\neq 0$ pour tout $s\in J$. Cela signifie bien que $\theta$ est strictement monotone. En effet, $\theta'$ étant continue, elle ne peut pas changer de signe sans passer par zéro (théorème des valeurs intermédiaires).
\end{proof}

\begin{theorem}     \label{ThoLongArcGeom}
    La longueur d'un arc est indépendante de sa paramétrisation, c'est à dire que les représentants d'un arc géométrique compact de classe $\mathcal{C}^1$ ont même longueur. 
\end{theorem}
Nous nommons \defe{longueur}{longueur!arc géométrique} d'un arc géométrique la longueur commune de tous ses représentants. On dit que l'arc géométrique est \defe{rectifiable}{rectifiable!arc géométrique} si sa longueur est $<\infty$.

\begin{proof}
    Nous utilisons les mêmes notations que celles du lemme \ref{LemChamVarsStriMomnot}. Nous savons déjà que le changement de variable $\theta \colon J\to I$ est strictement monotone. Supposons que $\theta$ soit croissante.
%    (voir exercice \ref{exoGeomAnal-0002}). Position 23657
    En effectuant un changement de variable dans l'intégrale qui définit la longueur nous avons
    \begin{equation}
        \begin{aligned}[]
            l(\gamma)&=\int_I\| \gamma'(t) \|dt\\
                &=\int_J\| \gamma'\big( \theta(s) \big) \|\theta'(s)ds\\
                &=\int_J\| \gamma'\big( \theta(s) \big)\theta'(s) \|ds\\
                &=\int_J\| \frac{ d }{ ds }(\gamma\circ\theta)(s) \|ds\\
                &=\int_J\| g'(s) \|ds\\
                &=l(J,g).
        \end{aligned}
    \end{equation}
\end{proof}

%---------------------------------------------------------------------------------------------------------------------------
\subsection{Abscisse curviligne et paramétrisation normale}     \label{SubSecAbsCurv}
%---------------------------------------------------------------------------------------------------------------------------

\begin{definition}
    Soit $(I,\gamma)$ un arc paramétré continu rectifiable. Nous appelons \defe{abscisse curviligne}{abscisse curviligne} de $\gamma$ toute application $\phi\colon I\to \eR$ telle que pour tout $t,t'\in I$ avec $t<t'$, nous ayons
    \begin{equation}
        l\big( \mathopen[ t,t'  \mathclose],\gamma\big) = \big|  \phi(t')-\phi(t) \big|.
    \end{equation}
    Si il existe un $t_0\in I$ tel que $\phi(t_0)=0$, alors nous disons que $t_0$ est l'\defe{origine}{origine!abscisse curviligne} de l'abscisse $\phi$.

    Un arc paramétré $(I,\gamma_N)$ continu rectifiable est dit \defe{normal}{normal!arc paramétré} si l'identité est une abscisse curviligne. Dans ce cas, pour tout choix de $t$ et $t'$ dans $I$ avec $t<t'$, nous avons 
    \begin{equation}
        l\big( \mathopen[ t , t' \mathclose],\gamma_N \big)=t'-t.
    \end{equation}
\end{definition}

%
% Une abscisse curviligne est une fonction qui vérifie cette propriété. Les abscisse curvilignes sont notées \phi.
% Le nom de longueur d'arc est réservé à l'abscisse curviligne qui commence en 0. C'est celle définie plus haut.
% La longueur d'arc est notée \varphi.
%

\begin{example}     \label{ExCerlceRadNorm}
    Le cercle unitaire est donné par l'arc
    \begin{equation}
        \gamma(t)=\big( \cos(t),\sin(t) \big)
    \end{equation}
    et $t\in\mathopen[ 0 , 2\pi \mathclose]$. Pour tout choix de $t$ et $t'$ dans $\mathopen[ 0 , 2\pi \mathclose]$, nous avons
    \begin{equation}
        l\big( \mathopen[ t , t' \mathclose],\gamma \big)=\int_t^{t'}\sqrt{\sin^2(u)+\cos^2(u)}du=t'-t.
    \end{equation}
    Les angles exprimés en radians forment dont une paramétrisation normale du cercle de rayon~$1$. Voir aussi l'exercice \ref{exoCourbesSurfaces0008}.
    % position 28183
    %Voir aussi les exercices \ref{exoGeomAnal-0003} et \ref{exoCourbesSurfaces0008}.
\end{example}

\begin{lemma}
    Pour un arc paramétré compact, la longueur d'arc est une abscisse curviligne.
\end{lemma}

\begin{proof}
    Par définition de la longueur d'arc $\varphi$, nous avons
    \begin{equation}
        \varphi(t')-\varphi(t)=l\big( [a,t'],\gamma \big)-l\big( [a,t],\gamma \big)=\diamondsuit.
    \end{equation}
    Supposons pour fixer les idées que $t'>t$. En utilisant la proposition \ref{Propletautredecop}, nous avons
    \begin{equation}
        l\big( [a,t'],\gamma \big)=l\big( [a,t],\gamma \big)+l\big( [t,t'],\gamma \big),
    \end{equation}
    et donc après simplification de deux termes,
    \begin{equation}
        \diamondsuit=l\big( [t,t'],\gamma \big),
    \end{equation}
    ce qui est précisément la propriété demandée pour être une abscisse curviligne.
\end{proof}

\begin{proposition}     \label{PropExisteChmNorm}
    Pour tout arc paramétré $\mathcal{C}^1$ sans points critiques, il existe un changement de coordonnées qui rend l'arc normal.
\end{proposition}

\begin{proof}
    Soit $(I,\gamma)$ un arc de classe $\mathcal{C}^1$. Nous devons montrer qu'il existe un intervalle $J$ et une application $\theta\colon J\to I$ de classe $\mathcal{C}^1$ et d'inverse $\mathcal{C}^1$ tel que l'arc $(J,\gamma_N)$ soit $\mathcal{C}^1$ où $\gamma_N=\gamma\circ\theta$.

    Si $I=\mathopen[ a ,b \mathclose]$, nous considérons la fonction
    \begin{equation}        \label{EqDevVarPhi}
        \begin{aligned}
            \phi\colon I&\to \eR^+ \\
            t&\mapsto \int_a^t\| \gamma'(u) \|du. 
        \end{aligned}
    \end{equation}
    Étant définie par l'intégrale d'une fonction $\mathcal{C}^0$, la fonction $\phi$ est $\mathcal{C}^1$, et nous avons $\phi'(t)=\| \gamma'(t) \|>0$ pour tout $t\in I$. Vue comme application $\phi\colon \mathopen[ a , b \mathclose]\to \mathopen[ 0 , l(\gamma) \mathclose]$, l'application $\phi$ est bijective et d'inverse $\mathcal{C}^1$. Voyons cela point par point.
    \begin{enumerate}
        \item
            La fonction $\phi$ est injective parce que strictement croissante.
        \item
            Elle est surjective parce que $\phi(a)=0$ et $\phi(b)=l(\gamma)$.
        \item
            La continuité de l'inverse est plus délicate. Soit $l\in\mathopen[ 0 , l(\gamma) \mathclose]$ et $\varepsilon>0$. Pour prouver la continuité de $\phi^{-1}$ en $s$, nous devons trouver un $\delta$ tel que
            \begin{equation}
                | s-s' |<\delta\Rightarrow\big| \phi^{-1}(s)-\phi^{-1}(s') \big|<\varepsilon.
            \end{equation}
            Étant donné que $s$ et $s'$ sont dans l'image de $\phi$, nous considérons les uniques $t$ et $t'$ tels que $s=\phi(t)$ et $s'=\phi(t')$. La quantité $\phi(t)-\phi(t')$ devient
            \begin{equation}        \label{EqCondvpemuCont}
                \int_a^t\big\| \gamma'(u) \big\|du-\int_a^{t'}\big\| \gamma'(u) \big\|du=\int_{t}^{t'}\big\| \gamma'(u) \big\|du.
            \end{equation}
            D'autre part, $\phi^{-1}(s)=t$ et $\phi^{-1}(s')=t'$, donc la condition  \eqref{EqCondvpemuCont} devient
            \begin{equation}
                |   \int_{t'}^t\big\| \gamma'(u) \big\|du  |\leq\delta\Rightarrow | t-t' |<\varepsilon.
            \end{equation}
            Cela revient à la continuité des fonctions définies par des intégrales.
        \item 
            La dérivée de son inverse est donnée par\footnote{Pour obtenir cette formule, dérivez les deux membres de l'équation $\phi\big( \phi^{-1}(s) \big)=s$.} 
            \begin{equation}
                (\phi^{-1})'(s)=\frac{1}{\phi'\big( \phi^{-1}(s) \big)}.
            \end{equation}
            Nous avons vu que $\phi^{-1}$ et $\phi'$ étaient continues. La fonction $(\phi^{-1})'$ étant exprimée en termes de ces deux fonctions elle est également continue.
    \end{enumerate}

    Nous considérons l'arc paramétré $(J,\gamma_N)$ avec $J=\mathopen[ 0 , l(\gamma) \mathclose]$ et 
    \begin{equation}
        \gamma_N(s)=(\gamma\circ\phi^{-1})(s).
    \end{equation}
    Nous montrons maintenant que cette nouvelle paramétrisation est normale. Soient $0\leq s\leq s'\leq l(\gamma)$,
    \begin{equation}
        \begin{aligned}[]
            l\big( \mathopen[ s , s' \mathclose],g \big)&=\int_s^{s'}\big\| \gamma_N'(u) \big\|du\\
            &=\int_{\phi^{-1}(s)}^{\phi^{-1}(s')}\big\| (\gamma_N'\circ\phi)(t) \big\|\phi'(t)dt\\
            &=\int_{\phi^{-1}(s)}^{\phi^{-1}(s')}\big\| (\gamma_N\circ\phi)'(t) \big\|dt\\
            &=\int_{\phi^{-1}(s)}^{\phi^{-1}(s')}\big\| \gamma'(t) \big\|dt\\
            &=\int_{0}^{\phi^{-1}(s')}\big\| \gamma'(t) \big\|\,dt -\int_0^{\phi^{-1}(s)}\big\| \gamma'(t) \big\|\,dt \\
            &=\phi\big( \phi^{-1}(s') \big)-\phi\big( \phi^{-1}(s) \big)\\
            &=s'-s,
        \end{aligned}
    \end{equation}
    ce qui prouve que la paramétrisation $(J,\gamma_N)$ est normale.
\end{proof}

Nous retenons que la paramétrisation normale de $\gamma$ est donnée par $(J,\gamma_N)$ avec $J=\mathopen[ 0 , l(\gamma) \mathclose]$ et 
\begin{equation}        \label{EqFomVPcogammaN}
    \gamma_N(s)=(\gamma\circ\phi^{-1})(s)
\end{equation}
où
\begin{equation}        \label{EqFomVPcoordnorm}
    \begin{aligned}
        \phi\colon I&\to \eR^+ \\
        t&\mapsto \int_a^t\| \gamma'(u) \|du. 
    \end{aligned}
\end{equation}
Notons aussi que $\phi$ est une fonction croissante, étant l'intégrale d'une fonction positive.

\begin{example}
    Trouvons les coordonnées normales pour la cycloïde\index{cycloïde!coordonnées normales} donnée par
    \begin{equation}
        \begin{aligned}[]
            x(t)&=a(t-\sin(t)),\\
            y(t)&=a(1-\cos(t))
        \end{aligned}
    \end{equation}
    et $t\in\mathopen] 0 , 2\pi \mathclose[$. Relire l'exemple \ref{ExCycloLong}.
    
    D'abord nous trouvons $\phi$ avec la formule \eqref{EqFomVPcoordnorm} avec $a=0$. En utilisant le bout de calcul \eqref{Eq_0508dlcycloide}, nous avons
    \begin{equation}
        \phi(t)=2a\int_0^t\sin\frac{ u }{2}du=4a\left( 1-\cos\frac{t}{2} \right).
    \end{equation}
    Pour trouver $\phi^{-1}(s)$, nous résolvons l'équation
    \begin{equation}
        s=\phi\big( \phi^{-1}(s) \big)
    \end{equation}
    par rapport à $\phi^{-1}(s)$. Dans un premier temps, nous trouvons
    \begin{equation}
        1-\frac{ s }{ 4a }=\cos\frac{ \phi^{-1}(s) }{ 2 },
    \end{equation}
    donc $\frac{ \phi^{-1}(s) }{2}=\arccos(\frac{ 4a-s }{ 4a })$, et finalement
    \begin{equation}
        \phi^{-1}(s)=2\arccos\left(\frac{ 4a-s }{ 4a }\right).
    \end{equation}
    Il nous reste à injecter cela dans les expressions de $x(t)$ et $y(t)$ pour trouver $(\gamma_N)_x(s)$ et $(\gamma_N)_y(s)$. D'abord,
    \begin{equation}
        (\gamma_N)_x(s)=a\big[ \phi^{-1}(s)-\sin\big( \phi^{-1}(s) \big) \big].
    \end{equation}
    Nous utilisons maintenant la formule trigonométrique $\sin(x)=2\sin\frac{ x }{ 2 }\cos\frac{ x }{2}$ afin de simplifier les expression :
    \begin{equation}
        \begin{aligned}[]
            (\gamma_N)_x&=a\Big[ 2\arccos\left( \frac{ 4a-s }{ 4a } \right)-2\sin\big( \arccos\left( \frac{ 4a-s }{ 4a } \right) \big)\cos\big( \arccos\left( \frac{ 4a-s }{ 4a } \right) \big) \Big]\\
            &=a\Big[ 2\arccos\left( \frac{ 4a-s }{ 4a } \right)-\frac{ 4a-s }{ 2a } \sqrt{1-\left( \frac{ 4a-s }{ 4a } \right)^2}\Big]\\
            &=2a\arccos\left( \frac{ 4a-s }{ 4a } \right)-\sqrt{8as-s^2}\,\frac{ 4a-s }{ 8a }
        \end{aligned}
    \end{equation}
    où nous avons utilisé la formule $\sin\big( \arccos(x) \big)=\sqrt{1-x^2}$. Ensuite, pour obtenir $(\gamma_N)_y$ nous devons calculer
    \begin{equation}
        (\gamma_N)_y(s)=a\big[ 1-\cos\big( \phi^{-1}(s) \big) \big].
    \end{equation}
    Encore une fois, il est intéressant d'exprimer le cosinus en termes des angles divisés par deux : $\cos(x)=\cos^2\frac{ x }{2}-\sin^2\frac{ x }{2}$.
    \begin{equation}
        \begin{aligned}[]
            (\gamma_N)_y&=a\Big[ 1-\cos^2\frac{ \phi^{-1}(s) }{2}+\sin^2\frac{ \phi^{-1}(s) }{2} \Big]\\
            &=a\Big[ 2-2\cos^2\frac{ \phi^{-1}(s) }{2} \Big]\\
            &=2a\Big[ 1-\left( \frac{ 4a-s }{ 4a } \right)^2 \Big].
        \end{aligned}
    \end{equation}
    Dans cette paramétrisation, $s\in\mathopen] 0 , 8a \mathclose[$.
\end{example}

\begin{example}
    La cardioïde $\rho(\theta)=a\big(1+\cos(\theta)\big)$ avec $\theta$ entre $-\pi$ et $\pi$. Avant d'utiliser la formule \eqref{EqFomVPcoordnorm}, nous devons trouver l'élément de longueur de la cardioïde. Étant donné la façon dont l'équation de la cardioïde nous est donnée, l'élément de longueur est donné par\footnote{Nous vous déconseillons d'étudier cette formule par cœur. Sachez cependant la retrouver assez vite.} \eqref{EqElemOngPOldeux} :
    \begin{equation}
        \begin{aligned}[]
            \| \gamma'(u) \|^2&=a^2\sin^2(u)+a^2(1+\cos(u))^2\\
                &=2a^2\big( 1+\cos(u) \big),
        \end{aligned}
    \end{equation}
    et par conséquent\footnote{L'utilisation stricte de la formule \eqref{EqFomVPcoordnorm} demanderait d'intégrer à partir de $-\pi$. Pour plus de simplicité, nous intégrons à partir de zéro, et nous verrons plus tard comment adapter l'intervalle du nouveau paramètre.}
    \begin{equation}
        \begin{aligned}[]
            \phi(t)&=\int_0^t\sqrt{2a^2\big( 1+\cos(u) \big)}du\\
            &=\int_0^t\sqrt{2a^2\left( 1+\cos^2\frac{ u }{2}-\sin^2\frac{ u }{2} \right)}du\\
            &=2a\int_0^t\cos\frac{ u }{2}du\\
            &=4a\sin\frac{ t }{2}.
        \end{aligned}
    \end{equation}
    Pour trouver l'inverse, nous résolvons $\phi\big( \phi^{-1}(s) \big)=s$ par rapport à $\phi^{-1}(s)$ :
    \begin{equation}
        \begin{aligned}[]
            4a\sin\left( \frac{ \phi^{-1}(s) }{2} \right)&=s,\\
            \phi^{-1}(s)&=2\arcsin\left( \frac{ s }{ 4a } \right).
        \end{aligned}
    \end{equation}
    
    Avant d'écrire trop brutalement $\gamma_N(s)=(\gamma\circ\phi^{-1})(s)$, il faut comprendre comment est $\gamma$. Nous avons reçu la courbe sous forme polaire, c'est à dire
    \begin{equation}
        \gamma(t)=\big( \gamma_r(t),\gamma_{\theta}(t) \big)=\Big( a\big( 1+\cos(t) \big),t \Big).
    \end{equation}
    C'est comme cela qu'il faut comprendre la donnée $\rho(\theta)=a\big( 1+\cos(\theta) \big)$. Maintenant la formule $\gamma_N(s)=(\gamma\circ\phi^{-1})(s)$ devient
    \begin{subequations}
        \begin{numcases}{}
            (\gamma_N)_r(s)=\gamma_r\big( \phi^{-1}(s) \big)\\
            (\gamma_N)_{\theta}(s)=\gamma_{\theta}\big( \phi^{-1}(s) \big).
        \end{numcases}
    \end{subequations}
    Étant donné que $\gamma_{\theta}(t)=t$, la seconde est facile :
    \begin{equation}
        (\gamma_N)_{\theta}(s)=2\arcsin\left( \frac{ s }{ 4a } \right).
    \end{equation}
    Pour la première,
    \begin{equation}
        (\gamma_N)_r(s)=a\big[ 1+\cos\big( 2\arcsin\frac{ s }{ 4a } \big) \big]=\frac{ 16a^2-s^2 }{ 8a }.
    \end{equation}
    Nous écrivons donc le nouveau paramétrage en coordonnées polaires sous la forme
    \begin{equation}
        \left( \frac{ 16a^2-s^2 }{ 8a },2\arcsin\frac{ s }{ 4a } \right).
    \end{equation}
    La question qui arrive maintenant est de savoir quel intervalle parcours la nouvelle variable $s$. D'après le résultat de l'exemple \ref{EqCardioide}, la longueur de la cardioïde est de $8a$ et nous avons donc $s\in\mathopen[ 0 , 8a \mathclose]$. Cependant, la condition d'existence de $\arcsin$ nous interdit d'avoir $s$ plus grand que $4a$ en valeur absolue. Où est le problème ?

    Le problème est que nous avons changé l'origine de notre paramètre en donnant $\phi(t)$ comme une intégrale à partir de $0$ au lieu de $-\pi$. Cela se voit en regardant de quel point nous partons : en $s=0$ nous sommes sur le point $(2a,0)$ tandis qu'avec le paramètre original, c'est à dire $\theta\in\mathopen[ -\pi , \pi \mathclose]$, nous avons pour $\theta=-\pi$ le point $(0,-\pi)$.

    Il se passe donc que si nous commençons à parcourir la cardioïde avec $s=0$, nous partons du milieu, et nous ne parcourons donc pas tout. Étant donné que le «premier» point de la cardioïde est le point $(0,-\pi)$, le paramètre $s$ commence en $s=-4a$, et nous avons comme intervalle :
    \begin{equation}
        s\in\mathopen[ -4a , 4a \mathclose],
    \end{equation}
    ce qui est en accord avec le conditions d'existence.
\end{example}

Quel enseignement tirer de cet exemple ? Lorsqu'on calcule $\phi(t)$ pour trouver les coordonnées normales, il y a deux solutions.
\begin{enumerate}
    \item
        Utiliser strictement la formule $\phi(t)=\int_a^t\| \gamma'(u) \|du$, en prenant bien comme borne de départ le point de départ de la paramétrisation de $\gamma$. À ce moment la coordonnée normale construite aura $\mathopen[ 0 , l(\gamma) \mathclose]$ comme intervalle de variation.
    \item
        Faire commencer l'intervalle d'intégration en zéro (ou ailleurs). Un bon choix peut simplifier quelque calculs, mais alors il faudra bien choisir la valeur de départ de la nouvelle coordonnées pour que le «premier» point de la courbe soit correct. Dans ce cas, la longueur de l'intervalle sera quand même $l(\gamma)$. Il n'y a donc pas de problèmes pour trouver la valeur du bout de l'intervalle de variation du paramètre normal.
\end{enumerate}
Dans tous les cas, il faut bien préciser l'intervalle de variation du paramètre lorsqu'on donne une courbe paramétrée.

%---------------------------------------------------------------------------------------------------------------------------
\subsection{Tangente à une courbe paramétrée}
%---------------------------------------------------------------------------------------------------------------------------

\begin{definition}
    Soit $(I,\gamma)$ un arc paramétré de classe $\mathcal{C}^k$ avec $k\geq 1$. Nous disons que la courbe admet une \defe{tangente}{tangente} en $\gamma(t_0)\in\eR^n$ lorsque les deux conditions suivantes sont remplies
    \begin{enumerate}
        \item
            $\gamma(t)\neq \gamma(t_0)$ pour tout $t$ dans un voisinage de $t_0$;
        \item
            la direction de la droite qui passe par $\gamma(t)$ et $\gamma(t_0)$ admet une limite lorsque $t\to t_0$.
    \end{enumerate}
    Dans ce cas, la tangente sera la droite passant par le point $\gamma(t_0)$ et dont la direction est donnée par la limite.
\end{definition}
Dans cette définition, par \defe{direction}{direction} d'une droite, nous entendons le vecteur de norme $1$ parallèle à celle-ci sans tenir compte du signe. La tangente sera donc la droite passant par $\gamma(t_0)$ et parallèle au vecteur
\begin{equation}
    \lim_{t\to t_0}\frac{ \gamma(t)-\gamma(t_0) }{ \| \gamma(t)-\gamma(t_0) \| }. 
\end{equation}
Évidement si nous avions écrit $\gamma(t_0)-\gamma(t)$, ça n'aurait pas changé la droite. Par abus de langage, nous parlerons souvent de «la direction $u$» même lorsque $u$ n'est pas de norme $1$.

Formellement, une direction est une classe d'équivalence de vecteurs pour la relation $u\sim v$ si il existe $\lambda\neq 0$ tel que $u=\lambda v$, mais nous n'aurons pas besoin de cette précision ici.

Sans surprises, la tangente est à peu près toujours donnée par la dérivée lorsqu'elle existe. Plus précisément nous avons le
\begin{theorem}
    Soit $(I,\gamma)$, un arc paramétré de classe $\mathcal{C}^k$ ($k\geq 1$) et $t_0\in I$ tel que
    \begin{equation}
        \gamma'(t_0)=\gamma''(t_0)=\ldots=\gamma^{(q-1)}(t_0)=0
    \end{equation}
    et
    \begin{equation}
        \gamma^{(q)}(t_0)\neq 0
    \end{equation}
    pour un entier $1\leq q\leq k$. Alors $\gamma$ admet une tangente en $\gamma(t_0)$ de direction $\gamma^{(q)}(t_0)$.
\end{theorem}

\begin{proof}
    
    Le développement de $\gamma(t_0)$ en série de Taylor autour de $t$ jusqu'à l'ordre $q$ est
    \begin{equation}        \label{EqDevTaylfttzq}
        \begin{aligned}[]
            \gamma(t)&=\gamma(t_0)+\gamma'(t_0)| t-t_0 |+\frac{ \gamma't(t_0) }{2}| t-t_0 |^2+\ldots +\frac{ \gamma^{(q)}(t_0) }{ q! }| t-t_0 |^q\\
                &\quad+\varepsilon(t)| t-t_0 |^q
        \end{aligned}
    \end{equation}
    où $\varepsilon$ est une application $\varepsilon\colon \eR\to \eR^n$ telle que $\lim_{t\to t_0} \varepsilon(t)=0$. En utilisant les hypothèses, nous éliminons la majorité des termes dans le développement \eqref{EqDevTaylfttzq} :
    \begin{equation}
        \gamma(t)-\gamma(t_0)=\frac{1}{ q! }\gamma^{(q)}(t_0)| t-t_0 |^q+\varepsilon(t)| t-t_0 |^q.
    \end{equation}
    La direction de la droite qui joint $\gamma(t)$ à $\gamma(t_0)$ est donc donnée par
    \begin{equation}
        \frac{ \gamma(t)-\gamma(t_0) }{ \| \gamma(t)-\gamma(t_0) \| }=\frac{ \frac{1}{ q! }\gamma^{(q)}(t_0)| t-t_0 |^q+\varepsilon(t)| t-t_0 |^q }{ \| \frac{1}{ q! }\gamma^{(q)}(t_0)| t-t_0 |^q+\varepsilon(t)| t-t_0 |^q\|  }
    \end{equation}
    et la limite lorsque $t\to t_0$ donne $\gamma^{(q)}(t_0)$ comme direction de la tangente.

\end{proof}

Lorsque le théorème s'applique, le vecteur
\begin{equation}
    \tau=\frac{ \gamma^{(q)}(t_0) }{ \| \gamma^{(q)}(t_0) \| }
\end{equation}
est appelé le \defe{vecteur unitaire tangent}{vecteur!unitaire tangent} en $\gamma(t_0)$ à l'arc paramétré $\gamma$.


\begin{corollary}       \label{CorTgSoCun}
    Si $(I,\gamma)$ est un arc paramétré de classe $\mathcal{C}^1$ régulier (c'est à dire $\gamma'(t)\neq 0$ pour tout $t$) alors l'arc admet une tangente en tout point et le vecteur unitaire de la tangente est donné par
    \begin{equation}
        \tau(t)=\frac{ \gamma'(t) }{ \| \gamma'(t) \| },
    \end{equation}
    pour tout $t$ dans $I$.
\end{corollary}

\begin{corollary}       \label{CorUnitTgtaugpnorma}
    Si $\gamma=(J,\gamma_N)$ est un arc paramétré de classe $\mathcal{C}^1$, normal, alors le vecteur unitaire de la tangente au point $\gamma_N(s)$ est donné par $\tau(s)=\gamma_N'(s)$.
\end{corollary}

\begin{proof}
    Nous devons démontrer que dans le cas d'une paramétrisation normale nous avons $\| \gamma_N'(s) \|=1$ pour tout $s$. Par définition,
    \begin{equation}
        l\big( \mathopen[ s , s' \mathclose],g \big)=\int_s^{s'}\| \gamma_N'(u) \|du=s'-s.
    \end{equation}
    Par conséquent,
    \begin{equation}
        \lim_{h\to 0} \frac{1}{ h }\int_s^{s+h}\| \gamma_N'(u) \|du=\lim_{y\to 0} \frac{ s+h-s }{ h }=1.
    \end{equation}
    Cela implique que $\| \gamma_N'(s) \|=1$, et donc en particulier que $(J,\gamma_N)$ est un arc régulier. Le corollaire précédent montre alors que $\tau(s)=\gamma_N'(s)/\| \gamma_N'(s) \|=\gamma_N'(s)$.
\end{proof}

\begin{example}
    Considérons la courbe $\gamma(t)=(t^2,t^3)$, et cherchons la tangente en $t_0=0$. En dérivant nous avons successivement 
    \begin{equation}
        \begin{aligned}[]
            \gamma(t)&=(t^2,t^3)\\
            \gamma'(t)&=(2t,3t^2)\\
            \gamma''(t)&=(2,6t).
        \end{aligned}
    \end{equation}
    En posant $t=0$, nous trouvons que $\gamma'(0)=0$ mais $\gamma''(0)=(2,0)\neq 0$. Le théorème nous dit donc que la direction de la tangente est horizontale. Nous pouvons faire le calcul directement :
    \begin{equation}
        \frac{ \gamma(t)-\gamma(t_0) }{ \| \gamma(t)-\gamma(t_0) \| }=\frac{ (t^2,t^3) }{ \sqrt{t^4+t^6} }=\frac{ (t^2,t^3) }{ t^2\sqrt{1+t^2} }=\frac{ (1,t) }{ \sqrt{1+t^2} },
    \end{equation}
    dont la limite \( t\to 0\) est bien le vecteur horizontal $(1,0)$.

    La figure \ref{LabelFigParamTangente} montre quelque tangente, c'est à dire quelque vecteurs dans la direction $\gamma'(t)$ (pour les $t\neq 0$, il ne faut pas aller à la dérivée seconde). Nous remarquons que de part et d'autres du sommet, les vecteurs ne sont pas dirigés dans le même sens. \emph{En tant que vecteurs} de norme $1$, ces vecteurs n'ont pas de limites quand $t\to 0$. Ce sont bien les \emph{directions} qui ont une limite, parce que la direction ne tient pas compte du sens.
    \newcommand{\CaptionFigParamTangente}{Quelque tangentes de la courbe $\gamma(t)=(t^2,t^3)$.}
    \input{Fig_ParamTangente.pstricks}
 
\end{example}

%+++++++++++++++++++++++++++++++++++++++++++++++++++++++++++++++++++++++++++++++++++++++++++++++++++++++++++++++++++++++++++
\section{Repère de Frenet}      \label{SecFrenet}
%+++++++++++++++++++++++++++++++++++++++++++++++++++++++++++++++++++++++++++++++++++++++++++++++++++++++++++++++++++++++++++

Dans cette section, nous ne considérons que des courbes dans $\eR^3$.

\begin{proposition}     \label{Proptausclataupzero}
    Soit $\gamma=(J,\gamma_N)$ un arc paramétré normal de classe $\mathcal{C}^2$. Alors pour toute valeur de $s$ dans $J$, nous avons
    \begin{equation}
        \tau(s)\cdot\tau'(s)=0
    \end{equation}
    où $\tau(s)=\gamma_N'(s)$. C'est à dire que la dérivée seconde est perpendiculaire à la dérivée première.
\end{proposition}

\begin{proof}
    La paramétrisation étant normale, nous avons 
    \begin{equation}
        \| \gamma_N'(s) \|^2=\sum_{i=1}^nx'_i(s)^2=1;
    \end{equation}
    ce qui implique, en dérivant les deux membres, que
    \begin{equation}
        0=2\sum_{i=1}^nx_i'(s)x''_i(s),
    \end{equation}
    c'est à dire exactement $\gamma_N'(s)\cdot \gamma_N''(s)=0$; d'où la thèse.
\end{proof}

\begin{remark}
    Si nous n'utilisons pas des coordonnées normales, la proposition \ref{Proptausclataupzero} n'est pas spécialement vraie. Prenons par exemple la courbe qui donne la parabole :
    \begin{subequations}
        \begin{align}
            \gamma(t)&=(t,t^2)\\
            \gamma'(t)&=(1,2t)\\
            \gamma''(t)&=(0,2)
        \end{align}
    \end{subequations}
    Nous avons $\gamma'(t)\cdot \gamma''(t)=4t$. Par conséquent, la dérivée seconde n'est la normale à la courbe que en $t=0$. Cela est une propriété très intéressante des coordonnées normales : la dérivée seconde d'une coordonnées normale donne un vecteur normal à la courbe, c'est à dire perpendiculaire à la tangente.
\end{remark}

\begin{definition}      \label{DefCourbureNormleUnit}
    Soit $\gamma=(J,\gamma_N)$ un arc paramétré normal de classe $\mathcal{C}^2$. La \defe{normale principale}{normale!principale} est le vecteur $\tau'(s)$. Le \defe{vecteur unitaire normal}{unitaire!normale principale}\index{vecteur!unitaire normal} est le vecteur\nomenclature[C]{$\nu(s)$}{Vecteur unitaire de la normale principale}
    \begin{equation}
        \nu(s)=\frac{ \tau'(s) }{ \| \tau'(s) \| }=\frac{ \gamma_N''(s) }{ \| \gamma_N''(s) \| }.
    \end{equation}
    La \defe{courbure}{courbure} au point $\gamma_N(s)$ est le réel\nomenclature[C]{$c(s)$}{rayon de courbure}
    \begin{equation}
        c(s)=\| \tau'(s) \|=\| \gamma_N''(s) \|.
    \end{equation}
    Le \defe{rayon de courbure}{rayon!de courbure} est le réel
    \begin{equation}
        R(s)=\frac{1}{ c(s) }=\frac{1}{ \| \gamma_N''(s) \| }.
    \end{equation}
\end{definition}

Par la proposition \ref{Proptausclataupzero}, nous avons $\nu(s)\cdot\tau(s)=0$. En combinant toutes les formules, nous avons les différentes expressions suivantes pour le vecteur normal unitaire :
\begin{equation}        \label{Eq0908nufractauRc}
    \nu(s)=\frac{ \gamma_N''(s) }{ c(s) }=\frac{ \tau'(s) }{ \| \tau'(s) \| }=\frac{ \tau'(s) }{ c(s) }=R(s)\tau'(s)=R(s)\gamma_N''(s).
\end{equation}

\begin{proposition}
    La fonction courbure s'écrit $c=\| \gamma_N'\times \gamma_N'' \|$.
\end{proposition}

\begin{proof}
    Par le point \ref{ItemPropScalMixtLiniv} de la proposition \ref{PropScalMixtLin}, nous avons
    \begin{equation}
        \langle \gamma_N', \gamma_N''\rangle^2 + \| \gamma_N'\times \gamma_N'' \|^2=\| \gamma_N' \|^2\| \gamma_N'' \|^2=\| \gamma_N'' \|^2
    \end{equation}
    parce que, la paramétrisation étant normale, $\| \gamma_N' \|=1$. Mais $\langle \gamma_N', \gamma_N''\rangle =0$, donc il reste $\| \gamma_N'\times \gamma_N'' \|^2=\| \gamma_N'' \|^2$, d'où
    \begin{equation}        \label{Eqcsnormgpgpps}
        c(s)=\| \gamma_N''(s) \|=\| \gamma_N'(s)\times \gamma_N''(s) \|
    \end{equation}
    pour chaque $s$ dans $J$.
\end{proof}

\begin{definition}
    Soit $s$ un point birégulier (c'est à dire $\gamma_N'(s)\neq 0$ et $\gamma_N''(s)\neq 0$) de l'arc normal $\gamma=(J,\gamma_N)$. Le \defe{vecteur unitaire de la binormale}{binormale} est le vecteur\nomenclature[C]{$\beta(s)$}{Vecteur unitaire de la binormale}
    \begin{equation}
        \beta(s)=\tau(s)\times\nu(s)
    \end{equation}
\end{definition}

Par leurs définitions, $\tau$ et $\nu$ sont unitaires, tandis que la proposition \ref{Proptausclataupzero} montre qu'ils sont également orthogonaux. Les propriétés du produit vectoriel font que $\beta$ est également unitaire, et simultanément orthogonal à $\tau$ et à $\nu$.

\begin{definition}
    Le repère orthonormal $\{ \gamma_N(s),\tau(s),\beta(s) \}$ est le \defe{repère de Frenet}{repère de Frenet} au point $\gamma_N(s)$.
\end{definition}

\begin{lemma}
    Le  vecteur unitaire normal est donné par $\nu(s)=\beta(s)\times \tau(s)$.
\end{lemma}

\begin{proof}
    Ceci est une application de la formule d'expulsion \eqref{EqFormExpluxxx} et de l'orthonormalité de la base de Frenet :
    \begin{equation}
        \beta\times\tau=(\tau\times\nu)\times\tau=-(\nu\cdot\tau)\tau+(\tau\cdot\tau)\nu=\nu.
    \end{equation}
\end{proof}

%---------------------------------------------------------------------------------------------------------------------------
\subsection{Torsion}
%---------------------------------------------------------------------------------------------------------------------------

Décomposons le vecteur $\beta'(s)$ dans la base de Frenet. Pour cela nous allons utiliser la proposition \ref{PropScalCompDec} et montrer que $\beta'(s)\cdot \tau(s)=\beta'(s)\cdot\beta(s)=0$, ce qui voudra dire que, dans la base de Frenet, les composantes de $\beta'$ le long de $\tau$ et $\beta$ sont nulles. Le vecteur $\beta'$ sera donc colinéaire à $\nu$.

D'abord, étant donné que la norme de $\beta(s)$ est constante par rapport à $s$, nous avons
\begin{equation}
    0=\frac{ d }{ ds }\| \beta(s) \|^2=2\beta'(s)\cdot\beta(s).
\end{equation}
Ensuite, nous dérivons la définition $\beta(s)=\tau(s)\times\nu(s)$ en utilisant la formule de Leibnitz \eqref{EqFormLeibProdscalVect} :
\begin{equation}
    \beta'(s)=\tau'(s)\times\nu(s)+\tau(s)\times\nu'(s).
\end{equation}
Mais $\tau'(s)=\gamma_N''(s)$ tandis que $\nu(s)=\frac{ \gamma_N''(s) }{ \| \gamma_N''(s) \| }$, de telle sorte que $\tau'(s)\times\nu(s)=0$. Nous restons donc avec $\beta'(s)=\tau(s)\times\nu'(s)$, ce qui prouve que $\beta'(s)$ est perpendiculaire à $\tau(s)$ et donc que $\beta'(s)\cdot\tau(s)=0$.

Le vecteur $\beta'(s)$ est donc un multiple de $\nu(s)$. Nous notons $t(s)$\nomenclature[C]{$t(s)$}{Torsion} le facteur de proportionnalité : 
\begin{equation}
    \beta'(s)=t(s)\nu(s).
\end{equation}

\begin{definition}      \label{DefTorsion}
    Soit $\gamma=(J,\gamma_N)$ un arc paramétré normal de classe $\mathcal{C}^3$. La \defe{torsion}{torsion} de $\gamma$ au point $\gamma_N(s)$ est le réel
    \begin{equation}
        t(s)=\| \beta'(s) \|=\| \tau(s)\times\nu'(s) \|.
    \end{equation}
    Lorsque $t(s)\neq 0$, le réel $T(s)=\frac{1}{ t(s) }$ est le \defe{rayon de torsion}{rayon!de torsion} de $\gamma$ en $\gamma_N(s)$.
\end{definition}

Étant donné que pour chaque $s$, l'ensemble $\{ \tau(s),\nu(s),\beta(s) \}$ est une base, il est naturel de vouloir décomposer leurs dérivées dans cette base. D'abord, par définition de $c$ et de $t$, nous avons
\begin{equation}
    \begin{aligned}[]
        \tau'(s)&=c(s)\nu(s)\\
        \beta'(s)&=t(s)\nu(s).
    \end{aligned}
\end{equation}
Il reste à décomposer $\nu'(s)$. Définissons $\alpha_{\tau}$, $\alpha_{\nu}$ et $\alpha_{\beta}$ (qui peuvent dépendre de $s$) par
\begin{equation}
    \nu'(s)=\alpha_{\tau}\tau(s)+\alpha_{\nu}\nu(s)+\alpha_{\beta}\beta(s).
\end{equation}
En vertu de la proposition \ref{PropScalCompDec}, nous avons
\begin{equation}
    \begin{aligned}[]
        \alpha_{\tau}=\langle \nu'(s), \tau(s)\rangle&=-\langle \nu(s), \tau'(s)\rangle =-\langle \nu(s), c(s)\nu(s)\rangle =-c(s) ,\\
        \alpha_{\nu}=\langle \nu'(s),  \nu(s)\rangle &=0,\\
        \alpha_{\beta}=\langle \nu'(s), \beta(s)\rangle &=-\langle \nu(s), \beta'(s)\rangle =-t(s),
    \end{aligned}
\end{equation}
où nous avons utilisé le fait que $\langle \nu(s), \nu(s)\rangle =\| \nu(s) \|^2=1$. Si nous mettons ces résultats sous forme matricielle, nous avons les \defe{formules de Frenet}{Frenet!formules} :
\begin{equation}
    \begin{pmatrix}
        \tau'(s)    \\ 
        \nu'(s) \\ 
        \beta'(s)   
    \end{pmatrix}=
    \begin{pmatrix}
        0   &   c(s)    &   0   \\
        -c(s)   &   0   &   -t(s)   \\
        0   &   t(s)    &   0
    \end{pmatrix}
    \begin{pmatrix}
        \tau(s) \\ 
        \nu(s)  \\ 
        \beta(s)    
    \end{pmatrix}.
\end{equation}


\begin{proposition}
    Si $s$ est un point birégulier, alors la torsion est donnée par
    \begin{equation}
        t(s)=-\frac{ (\gamma_N'\times \gamma_N'')\times \gamma_N''' }{ \| \gamma_N'(s)\times \gamma_N''(s) \|^2 }.
    \end{equation}
\end{proposition}

\begin{proof}
    Par l'équation \eqref{Eq0908nufractauRc}, nous avons $\gamma_N''(s)=c'(s)\nu(s)$, et par conséquent
    \begin{equation}
        \gamma_N'''(s)=c'(s)\nu(s)+c(s)\nu'(s)=c'(s)\nu(s)+c(s)\big[ -c(s)\tau(s)-t(s)\beta(s) \big],
    \end{equation}
    où nous avons utilisé la formule de Frenet pour $\nu'(s)$. Par ailleurs, sachant le corollaire \ref{CorUnitTgtaugpnorma} et la formule de Frenet pour $\tau'$, nous avons
    \begin{equation}
        \gamma_N'\times \gamma_N''=\tau(s)\times \tau'(s)=\tau(s)\times c(s)\nu(s)=c(s)\beta(s).
    \end{equation}
    En combinant les deux dernières équations, et en se souvenant que la base de Frenet et orthonormale,
    \begin{equation}
        (\gamma_N'\times \gamma_N'')\cdot \gamma_N'''(s)=-c(s)^2t(s),
    \end{equation}
    et donc, en remplaçant $c(s)$ par la formule \eqref{Eqcsnormgpgpps},
    \begin{equation}
        t(s)=-\frac{  (\gamma_N'\times \gamma_N'')\cdot \gamma_N'''   }{ \| \gamma_N'\times \gamma_N'' \|^2 }.
    \end{equation}
\end{proof}

%+++++++++++++++++++++++++++++++++++++++++++++++++++++++++++++++++++++++++++++++++++++++++++++++++++++++++++++++++++++++++++
\section{Hors des coordonnées normales}
%+++++++++++++++++++++++++++++++++++++++++++++++++++++++++++++++++++++++++++++++++++++++++++++++++++++++++++++++++++++++++++

\begin{remark}      \label{Remfougnormoupad}
    Notons que la définition de $\tau$ est donnée pour tout arc $\mathcal{C}^1$ régulier $(I,\gamma)$ par $\tau(t)=\gamma'(t)/\| \gamma'(t) \|$. La propriété $\tau=\gamma_N'$ n'est valable que lorsque la paramétrisation est normale. Les autres définitions ont toutes été données dans le cas d'une paramétrisation normale.
\end{remark}

La remarque \ref{Remfougnormoupad} nous incite à exprimer toute la base de Frenet en terme de $\gamma$ lorsque la paramétrisation n'est pas normale. Étant donné que nous pouvons toujours faire le changement de variable $\gamma(t)=\gamma_N\big( \phi(t) \big)$ (proposition \ref{PropExisteChmNorm}), il est possible d'exprimer les vecteurs $\tau$, $\nu$ et $\beta$ ainsi que les réels $c$ et $t$ en fonction de $\gamma$ et de ses dérivées. 

Nous allons maintenant travailler à écrire les formules. 

Pour plus de facilité, nous collectons les définitions. Afin d'alléger la notation, nous n'exprimons pas explicitement les dépendances en $s$ :
\begin{description}
    \item[Vecteur unitaire tangent] 
        Par le corollaire \ref{CorUnitTgtaugpnorma}, $\tau$ est donné par $\tau=\gamma_N'$.
    \item[Vecteur unitaire normal] 
        Par la définition \ref{DefCourbureNormleUnit}, $\nu$ est donné par
        $\nu=\frac{ \tau' }{ \| \tau' \| }$.
    \item[Vecteur unitaire de la binormale] 
        Par la définition \ref{DefCourbureNormleUnit}, $\beta$ est donné par
            $\beta=\tau\times\nu$.
    \item[Courbure] 
        Par la définition \ref{DefCourbureNormleUnit}, $c$ est donné par
            $c=\| \tau' \|$.
    \item[Torsion]
        Par la définition \ref{DefTorsion}, $t$ est donné par
            $t=\| \beta' \|$.
\end{description}


Le schéma du changement de variable est
\begin{equation}        \label{EqDiagIJstgvpR}
    \xymatrix{%
    t\in I \ar[r]^{f}\ar[d]_{\phi}      &   \eR^3\\
    s\in J \ar[ru]_{g}  &   
       }
\end{equation}
La difficulté ne sera pas d'éliminer $\gamma_N$ de toutes les formule, mais bien de se débarrasser des fonctions $\phi$ qui arrivent quand nous exprimons $\gamma_N$ en termes de $\gamma$, et en particulier lorsque nous voulons exprimer les dérivées de $\gamma_N$ en termes de $\gamma$ et de ses dérivées.

Regardons d'abord comment les dérivées de $\gamma_N$ s'expriment en termes de $\gamma$. En utilisant le fait que $\gamma_N(s)=(\gamma\circ\phi^{-1})(s)$ et que $\| \gamma_N'(s) \|=1$, nous avons
\begin{equation}        \label{EqgpNgpnNnr}
    \gamma_N'(s)=\frac{ \gamma_N'(s) }{ \| \gamma_N'(s) \| }
    =\frac{ (\gamma\circ\phi^{-1})'(s) }{ \| (\gamma\circ\phi^{-1})'(s) \| }
    =\frac{ \gamma'\big( \phi^{-1}(s) \big)   (\phi^{-1})'(s)   }{ \| \gamma'\big( \phi^{-1}(s) \big) \|  |(\phi^{-1})'(s) |}
    =\frac{ \gamma'(t) }{ \| \gamma'(t) \| }
\end{equation}
où nous avons utilisé le fait que $\phi^{-1}$ étant croissante (parce que l'inverse d'une fonction croissante est croissante), $(\phi^{-1})'(s)=| (\phi^{-1})'(s) |$. Pourquoi écrivons nous $| \phi^{-1}(s) |$ et non $\| \phi^{-1}(s) \|$ ?

Pour la dérivée seconde, nous dérivons la relation \eqref{EqgpNgpnNnr} :
\begin{equation}
    \gamma_N''(s)=\frac{ \gamma''\big( \phi^{-1}(s) \big)(\phi^{-1})'(s) }{ \| \gamma'\big( \phi^{-1}(s) \big) \| }+\gamma'\big( \phi^{-1}(s) \big)\frac{ d }{ ds }\Big[ \| \gamma'\big( \phi^{-1}(s) \big) \| \Big].
\end{equation}
Le petit calcul suivant va nous permettre de simplifier cette expression :
\begin{equation}        \label{Eavpemuetfpnorm}
    (\phi^{-1})'(s)=(\phi^{-1})'\big( \phi(t) \big)=\frac{1}{ \phi'(t) }=\frac{1}{ \| \gamma'(t) \| }.
\end{equation}
Donc
\begin{equation}
    \gamma_N''(s)=\frac{ \gamma''(t) }{ \| \gamma'(t) \|^2 }+\gamma'(t)\frac{ d }{ ds }\Big[ \| \gamma'(t) \| \Big]
\end{equation}
où il est entendu que $t=\phi^{-1}(s)$. Avec cette expression, nous ne nous sommes pas encore débarrassés de la fonction $\phi$, mais nous allons voir que cela nous sera suffisant.

Pour le vecteur unitaire tangent $\tau(s)$, nous avons donc immédiatement
\begin{equation}        \label{EqTauavect}
    \tau(s)=\gamma_N'(s)=\frac{ \gamma'(t) }{ \| \gamma'(t) \| }.
\end{equation}
Ici encore il est sous-entendu que le $t$ dans le membre de droite est lié au $s$ du membre de gauche par $t=\phi^{-1}(s)$. Il est donc naturel de nous demander si nous avons gagné quelque chose, étant donné que la formule \eqref{EqTauavect} contient encore la fonction $\phi$.

Géométriquement, le vecteur $\tau(s)$ est le vecteur normal unitaire de la courbe au point $\gamma_N(s)$. En utilisant les relations du diagramme \eqref{EqDiagIJstgvpR}, nous avons en réalité $\gamma_N(s)=\gamma_N\big( \phi(t) \big)=\gamma(t)$. Le vecteur $\frac{ \gamma'(t) }{ \| \gamma'(t) \| }$ représente donc le vecteur normal tangent au point $\gamma(t)$.

Pour calculer la courbure, nous devons d'abord calculer le produit vectoriel
\begin{equation}        \label{eqProdvectogpgpp}
    \begin{aligned}[]
        \gamma_N'(s)\times \gamma_N''(s) &=  \frac{ \gamma'(t) }{ \| \gamma'(t) \| }\times \left( \frac{ \gamma''(t) }{ \| \gamma'(t) \|^2 }+\gamma'(t)\frac{ d }{ ds }\Big[ \| \gamma'(t) \| \Big] \right)\\
        &=\frac{ \gamma'(t)\times \gamma''(t) }{ \| \gamma'(t) \|^3 }
    \end{aligned}
\end{equation}
parce que le deuxième terme dans la parenthèse est un multiple de $\gamma'(t)$, de telle sorte à ce que son produit vectoriel avec $\gamma'(t)/\| \gamma'(t) \|$ soit nul. En prenant la norme,
\begin{equation}        \label{EqCourburetermf}
    c(s)=\frac{ \| \gamma'(t)\times \gamma''(t) \| }{ \| \gamma'(t) \|^3 }.
\end{equation}
Encore une fois, cette équation nous enseigne que la courbure au point $\gamma(t)\in\eR^3$ est donnée par le membre de droite, qui ne dépend que de $t$.

Le vecteur unitaire binormal est donné par $\beta(s)=\tau(s)\times \nu(s)$. En utilisant \eqref{EqTauavect} et \eqref{Eq0908nufractauRc},
\begin{equation}
    \beta(s)=\tau(s)\times\nu(s)=\gamma_N'(s)\times \frac{ \gamma_N''(s) }{ c(s) }.
\end{equation}
Les formules \eqref{eqProdvectogpgpp} pour le produit vectoriel et \eqref{EqCourburetermf} pour la courbure donnent ensuite
\begin{equation}
    \beta(s)=\frac{ \gamma'(t)\times \gamma''(t) }{ \| \gamma'(t) \|^3 }\cdot\frac{1}{ c(s) }=\frac{ \gamma'(t)\times \gamma''(t) }{ \|  \gamma'(t)\times \gamma''(t)  \| }.
\end{equation}
Cela donne le vecteur unitaire binormal au point $\gamma(t)$ en terme de $\gamma'(t)$ et $\gamma''(t)$.

La torsion demande d'utiliser la dérivée troisième de $\gamma_N$. Nous avons
\begin{equation}
    \begin{aligned}[]
        \gamma_N'''(s)&=(\gamma\circ\phi^{-1})'''(s)\\
        &=\Big( \gamma'\big( \phi^{-1}(s) \big)(\phi^{-1})'(s) \Big)''\\
        &=\Big( \gamma''\big( \phi^{-1}(s) \big)(\phi^{-1})'(s)^2+\gamma'\big( \phi^{-1}(s) \big)(\phi^{-1})''(s) \Big)'\\
        &=\gamma'''\big( \phi^{-1}(s) \big)(\phi^{-1})'(s)^3+ v\\
        &=\frac{ \gamma'''\big( \phi^{-1}(s) \big) }{ \| \gamma'(t) \|^3 }+v&&\text{par \eqref{Eavpemuetfpnorm}}
    \end{aligned}
\end{equation}
où $v$ est un élément de $\langle \gamma''\big( \phi^{-1}(s) \big),\gamma'\big( \phi^{-1}(s) \big)\rangle$. Le vecteur $v$ est donc perpendiculaire à $\gamma'\times \gamma''$ et donc à $\gamma_N'\times \gamma_N''$ à cause de la relation \eqref{eqProdvectogpgpp} qui montre que $\gamma'\times \gamma''$ est parallèle à $\gamma_N'\times \gamma_N''$. De ce fait, lorsque nous calculons $(\gamma_N'\times \gamma_N'')\cdot \gamma_N'''$, la partie $v$ de $\gamma_N'''$ n'entre pas en ligne de compte.

Nous avons donc le calcul suivant, en remplaçant les diverses occurrences de $\gamma_N'\times \gamma_N''$ par sa valeur \eqref{eqProdvectogpgpp} en termes de $\gamma$,
\begin{equation}
    \begin{aligned}[]
        t(s)&=-\frac{ (\gamma_N'\times \gamma_N'')\cdot \gamma_N''' }{ \| \gamma_N'\times \gamma_N'' \|^2 }\\
        &=-\frac{ (\gamma_N'\times \gamma_N'')\cdot \gamma'''(t) }{ \| \gamma_N'\times \gamma_N'' \|^2\,\| \gamma'(t) \|^2 }\\
        &=-\frac{ (\gamma'\times \gamma'')\cdot \gamma''' }{ \| \gamma'\times \gamma'' \|^2 }.
    \end{aligned}
\end{equation}
Dans cette expression, il est sous-entendu que tous les $\gamma_N$ sont fonctions de $s$ et tous les $\gamma$ sont fonction de $t$ où $s$ et $t$ sont liés par $s=\phi(t)$.

Ce que nous avons prouvé est le 
\begin{theorem}
    Pour tout représentant $(I,\gamma)$, les éléments métriques $(\tau,\nu,\beta,c,t)$ au point $\gamma(t)$ s'expriment en fonction de $\gamma(t)$, $\gamma'(t)$, $\gamma''(t)$ et $\gamma'''(t)$.
\end{theorem}

%+++++++++++++++++++++++++++++++++++++++++++++++++++++++++++++++++++++++++++++++++++++++++++++++++++++++++++++++++++++++++++
\section{Tracer des courbes paramétriques dans $\eR^2$}     \label{SecTracerParmCourbe}
%+++++++++++++++++++++++++++++++++++++++++++++++++++++++++++++++++++++++++++++++++++++++++++++++++++++++++++++++++++++++++++

Nous allons maintenant voir comment les concepts introduits nous aident à effectivement tracer des courbes dans le plan. Les courbes que nous regardons sont de la forme $\gamma(t)=\big( x(t),y(t) \big)$, et nous supposons que ces fonctions soient suffisamment régulières (disons trois fois continument dérivables). Nous ne supposons pas que la courbe soit donnée en coordonnées normales, en particulier, $\gamma''(t)$ n'est pas le vecteur normal en $\gamma(t)$.

La notion clef qui va jouer est le \defe{cercle osculateur}{osculateur (cercle)} de la courbe $\gamma$ au point $\gamma(t)$. Sans rentrer dans les détails, disons que c'est le cercle qui «colle» le mieux possible la courbe. Le rayon de ce cercle est le rayon de courbure :
\begin{equation}
    R(t)=\frac{ \| \gamma(t) \|^3 }{ \| \gamma'(t)\times\gamma''(t) \| }.
\end{equation}
En pratique, le produit vectoriel se calcule comme ceci :
\begin{equation}
    \gamma'(t)\times\gamma''(t)=\begin{vmatrix}
        e_x &   e_y &   e_z \\
        x'(t)   &   y'(t)   &   0   \\
        x''(t)  &   y''(t)  &   0
    \end{vmatrix}=(x'y''-x''y')e_z.
\end{equation}
Le centre du cercle osculateur va se trouver quelque part sur la normale. Le vecteur normal est donné par
\begin{equation}
    n(t)=J\frac{\gamma'(t) }{ \| \gamma'(t) \| }
\end{equation}
où $J$ est la rotation d'angle $\frac{ \pi }{2}$ :
\begin{equation}
    J\begin{pmatrix}
        x'(t)   \\ 
        y'(t)   
    \end{pmatrix}=
    \begin{pmatrix}
        0   &   1   \\ 
        -1  &   0   
    \end{pmatrix}\begin{pmatrix}
        x'(t)   \\ 
        y'(t)   
    \end{pmatrix}=\begin{pmatrix}
        y'(t)   \\ 
        -x'(t)  
    \end{pmatrix}.
\end{equation}
Cela nous laisse deux possibilités pour le centre du cercle osculateur : $\gamma(t)+R(t)n(t)$ ou bien $\gamma(t)-R(t)n(t)$. Il faut savoir de quel côté de la courbe est situé le centre du cercle osculateur. Il faut choisir le côté de la concavité, c'est à dire le côté de la dérivée seconde.

\newcommand{\CaptionFigQuelCote}{De quel coté de $\gamma'(t)$ se trouvent $n(t)$ et $-n(t)$ ?}
\input{Fig_QuelCote.pstricks}

La difficulté maintenant est de savoir qui de $n(t)$ ou $-n(t)$ est du côté de $\gamma''(t)$. Il faut savoir si $n(t)$ est du même côté de la droite tangente que $\gamma''(t)$ ou non. Par construction, si nous regardons la figure  \ref{LabelFigQuelCote}, le vecteur $n(t)$ sera toujours à gauche de $\gamma'(t)$. Le fait que $\gamma''(t)$ soit à gauche ou à droite de $\gamma'(t)$ est donné par le signe du produit vectoriel $\gamma'(t)\times \gamma''(t)$. Si ce produit vectoriel est positif, il faut choisir $-n(t)$ et si il est négatif, il faut choisir $n'(t)$.

Le truc pour obtenir le signe de $x'y''-x''y'$ est de faire
\begin{equation}
    \frac{ (\gamma'\times\gamma'')\cdot e_z}{\| \gamma'\times\gamma'' \|}.
\end{equation}

Le centre de courbure sera donc situé à la position
\begin{equation}
    \Omega(t)=\gamma(t)-n(t)\frac{ \| \gamma(t) \|^3 }{ \| \gamma'(t)\times\gamma''(t) \|^2 } (\gamma'\times\gamma'')\cdot e_z
\end{equation}
Nous pouvons écrire cela plus explicitement en nous souvenant que $\gamma'\times\gamma''=(x'y''-x''y')e_z$, par conséquent $\frac{ (\gamma'\times\gamma'')\cdot e_z}{\| \gamma'\times\gamma'' \|^2}=\frac{1}{ x'y''-x''y' }$. Nous avons
\begin{subequations}
    \begin{align}
        \Omega_x(t)&=x(t)-y'(t)\frac{ x'^2+y'^2 }{ x'y''-x''y' }\\
        \Omega_y(t)&=y(t)+x'(t)\frac{ x'^2+y'^2 }{ x'y''-x''y' }.
    \end{align}
\end{subequations}

Quelque exemples de cercles osculateurs sont sur la figure \ref{LabelFigOsculateur}.
\newcommand{\CaptionFigOsculateur}{Exemple de cercles osculateurs.}
\input{Fig_Osculateur.pstricks}





% TODO : Écrire quelque chose sur les points de rebroussement et d'inflexion, ainsi que sur les asymptotes.
%   Quand ce sera fait, il y a des choses à décommenter dans l'exerice exoCourbesSurfaces0002.tex
