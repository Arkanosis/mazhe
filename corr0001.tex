% This is part of Exercices et corrigés de CdI-1
% Copyright (c) 2011
%   Laurent Claessens
% See the file fdl-1.3.txt for copying conditions.

\begin{corrige}{0001}
 
    \newcounter{bidon}
\setcounter{bidon}{1}

Par convention, si le supremum n'existe pas, nous disons qu'il vaut l'infini. Cela est logique pour la raison suivante : étant donné que toute partie majorée de $\eR$ a un supremum, le fait de ne pas en avoir signifie que la partie considérée n'est pas majorée. Dans ce cas, il est logique de dire que son supremum est infini. (bien que cela ne soit pas vrai au sens strict de la définition)

\[ 
\begin{array}{lcccc}
\text{num}		&	\sup	&	\inf	&	\max	&	\min	\\
\hline
\text{\alph{bidon}}	&	36		&	10		&	36		&	NAN		\\	\stepcounter{bidon}
\text{\alph{bidon}}	&	36		&	10		&	NAN		&	10		\\	\stepcounter{bidon}
\text{\alph{bidon}}	&	\infty		&	5		&	NAN		&	NAN		\\	\stepcounter{bidon}
\text{\alph{bidon}}	&	9		&	8		&	NAN		&	NAN		\\	\stepcounter{bidon}
\text{\alph{bidon}}	&	1		&	-1		&	NAN		&	NAN		\\	\stepcounter{bidon}
\text{\alph{bidon}}	&	1/2		&	0		&	1/2		&	NAN		\\	\stepcounter{bidon}
\text{\alph{bidon}}	&	1		&	0		&	1		&	NAN		\\	\stepcounter{bidon}
\text{\alph{bidon}}	&	1		&	-1		&	NAN		&	NAN		\\	\stepcounter{bidon}
\text{\alph{bidon}}	&	1		&	-1		&	1		&	-1		\\	\stepcounter{bidon}
\text{\alph{bidon}}	&	\sin{\pi/3}	&	\sin{4\pi/3}	&	\sin{\pi/3}	&	\sin{4\pi/3}	\\	\stepcounter{bidon}
\text{\alph{bidon}}	&	1		&	0		&	NAN		&	0		\\	\stepcounter{bidon}
\text{\alph{bidon}}	&	\infty		&	0		&	NAN		&	NAN		\\	\stepcounter{bidon}
\text{\alph{bidon}}	&	\infty		&	-\infty		&	NAN		&	NAN		\\	\stepcounter{bidon}
\text{\alph{bidon}}	&	1		&	-2		&	1		&	NAN		\\	\stepcounter{bidon}
\text{\alph{bidon}}	&	2		&	0		&	NAN		&	0
\end{array}
\]

\end{corrige}
