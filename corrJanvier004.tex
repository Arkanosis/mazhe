% This is part of the Exercices et corrigés de mathématique générale.
% Copyright (C) 2009,2013
%   Laurent Claessens
% See the file fdl-1.3.txt for copying conditions.
\begin{corrige}{Janvier004}


\begin{enumerate}
  \item
Nous pouvons par exemple donner le \defe{critère de la racine}{critère!de la racine}.
 Soit $\sum_k a_k$ une série dont les termes sont
  positifs. Supposons que la limite
  \begin{equation*}
    \lim_{k \to \infty} \sqrt[k]{a_k}
  \end{equation*}
  existe et vaut $\alpha$, alors si $\alpha < 1$, la
  série converge, et si $\alpha > 1$ la série diverge.

\item Cette série converge par application du critère de la racine ;
  en effet, dans ce cas ci $a_k = \frac 1 {\ln(k)^k}$ et donc
  \begin{equation*}
    \lim_{k\to\infty} \sqrt[k]{a_k} = \lim_{k\to\infty} \frac 1
    {\ln(k)} = 0 < 1.
  \end{equation*}
\end{enumerate}

\end{corrige}
