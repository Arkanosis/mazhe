% This is part of Mes notes de mathématique
% Copyright (c) 2011-2012
%   Laurent Claessens, Carlotta Donadello
% See the file fdl-1.3.txt for copying conditions.

%+++++++++++++++++++++++++++++++++++++++++++++++++++++++++++++++++++++++++++++++++++++++++++++++++++++++++++++++++++++++++++
\section{Parties libres, génératrices et bases}
%+++++++++++++++++++++++++++++++++++++++++++++++++++++++++++++++++++++++++++++++++++++++++++++++++++++++++++++++++++++++++++
 
\begin{definition}
	Un sous-ensemble $B=\{v_1,\ldots,v_q\}$ de $\eR^m$ est une \defe{base}{base} de $\eR^m$ s'il satisfait les conditions suivantes
\begin{itemize}
	\item $B$ est \defe{libre}{libre}, c'est à dire
\[
\sum_{i=1}^{q}a_i v_i=0_{m} \quad\Leftrightarrow\quad a_i=0, \forall i=1,\ldots,q.
\]
\item $B$ est \defe{générateur}{générateur}, c'est à dire que pour tout $x$ dans $\eR^m$ il existe un ensemble de coefficients $\{a_i\in\eR, i=1,\ldots,n\}$ tel que
\[\sum_{i=1}^{q}a_i v_i=x.\]
\end{itemize}
\end{definition}
Il existe une infinité de bases de $\eR^m$. On peut démontrer que le cardinal de toute base de $\eR^m$ est $m$, c'est à dire que toute base de $\eR^m$ possède exactement $m$ éléments.

La base de $\eR^m$ qu'on dit \defe{canonique}{canonique!base}\index{base!canonique de $\eR^m$} (c.à.d. celle qu'on utilise tout le temps) est $\mathcal{B}=\{e_1,\ldots, e_m\}$, où le vecteur $e_j$ est 
\begin{equation}\nonumber
  e_j=
\begin{array}{cc}
  \begin{pmatrix}
    0\\\vdots\\0\\1\\ 0\\\vdots\\0
  \end{pmatrix} & 
  \begin{matrix}
    \quad\\\quad\\\leftarrow\textrm{j-ème} \quad\\\quad\\\quad\\
  \end{matrix}
\end{array}.
\end{equation}
La composante numéro $j$ de $e_i$ est $1$ si $i=j$ et $0$ si $i\neq j$. Cela s'écrit $(e_i)_j=\delta_{ij}$ où $\delta$ est le \defe{symbole de Kronecker}{Kronecker} défini par
\begin{equation}
	\delta_{ij}=\begin{cases}
		1	&	\text{si $i=j$}\\
		0	&	 \text{si $i\neq j$.}
	\end{cases}
\end{equation}
Les éléments de la base canonique de $\eR^m$ peuvent donc être écrits $e_i=\sum_{k=1}^m\delta_{ik}e_k$.


%+++++++++++++++++++++++++++++++++++++++++++++++++++++++++++++++++++++++++++++++++++++++++++++++++++++++++++++++++++++++++++
\section{Produit scalaire}
%+++++++++++++++++++++++++++++++++++++++++++++++++++++++++++++++++++++++++++++++++++++++++++++++++++++++++++++++++++++++++++

\begin{definition}      \label{DefYNWUFc}
	Soient $u$ et $v$, deux vecteurs de $\eR^m$. Le \defe{produit scalaire}{produit!scalaire} de $u$ et $v$, noté $\langle u, v\rangle $ ou $u\cdot v$ est le réel
	\begin{equation}		\label{EqDefProdScalsumii}
		\langle u, v\rangle =\sum_{k=1}^m u_kv_k=u_1v_1+u_2v_2+\ldots+u_mv_n.
	\end{equation}
\end{definition}

Calculons par exemple le produit scalaire de deux vecteurs de la base canonique : $\langle e_i, e_j\rangle $. En utilisant la formule de définition et le fait que $(e_i)_k=\delta_{ik}$, nous avons
\begin{equation}
	\langle e_i, e_j\rangle =\sum_{k=1}^m\delta_{ik}\delta_{jk}.
\end{equation}
Nous pouvons effectuer la somme sur $k$ en remarquant qu'à cause du $\delta_{ik}$, seul le terme avec $k=i$ n'est pas nul. Effectuer la somme revient donc à remplacer tous les $k$ par des $i$ :
\begin{equation}
	\langle e_i, e_j\rangle =\delta_{ii}\delta_{ji}=\delta_{ji}.
\end{equation}

Une des propriétés intéressantes du produit scalaire est qu'il permet de décomposer un vecteur dans une base, comme nous le montre la proposition suivante.

\begin{proposition}		\label{PropScalCompDec}
	Si nous notons $v_i$ les composantes du vecteur $v$, c'est à dire si $v=\sum_{i=1}^m v_ie_i$, alors nous avons $v_j=\langle v, e_j\rangle $.
\end{proposition}

\begin{proof}
	\begin{equation}		\label{Eqvejscalcomp}
		v\cdot e_j=\sum_{i=1}^m\langle v_ie_i, e_j\rangle =\sum_{i=1}^mv_i\langle e_i, e_j\rangle =\sum_{i=1}^mv_i\delta_{ij}
	\end{equation}
	En effectuant la somme sur $i$ dans le membre de droite de l'équation \eqref{Eqvejscalcomp}, tous les termes sont nuls sauf celui où $i=j$; il reste donc
	\begin{equation}
		v\cdot e_j=v_j.
	\end{equation}
\end{proof}

Le produit scalaire ne dépend en réalité pas de la base orthogonale choisie. 

\begin{lemma}
	Si $\{ e_i \}$ est la base canonique, et si $\{ f_i \}$ est une autre base orthonormale, alors si $u$ et $v$ sont deux vecteurs de $\eR^m$, nous avons
	\begin{equation}
		\sum_i u_iv_j=\sum_iu'_iv'_j
	\end{equation}
	où $u_i$ sont les composantes de $u$ dans la base $\{ e_i \}$ et $u'_i$ sont celles dans la base $\{ f_i \}$.
\end{lemma}

\begin{proof}
	La preuve demande un peu d'algèbre linéaire. Étant donné que $\{ f_i \}$ est une base orthonormale, il existe une matrice $A$ orthogonale ($AA^t=\mtu$) telle que $u'_i=\sum_jA_{ij}u_j$ et idem pour $v$. Nous avons alors
	\begin{equation}
		\begin{aligned}[]
			\sum_iu'_iv'_j&=\sum_i\left( \sum_jA_{ij} u_j\right)\left( \sum_k A_{ik}v_k \right)\\
			&=\sum_{ijk}A_{ij}A_{ik}u_jv_k\\
			&=\sum_{jk}\underbrace{\sum_i(A^t)_{ji}A_{ik}}_{=\delta_{jk}}u_jv_k\\
			&=\sum_{jk}\delta_{jk}u_jv_k\\
			&=\sum_ku_jv_k.
		\end{aligned}
	\end{equation}	
\end{proof}

Cette proposition nous permet de réellement parler du produit scalaire entre deux vecteurs de façon intrinsèque sans nous soucier de la base dans laquelle nous regardons les vecteurs.

Nous dirons que deux vecteurs sont \defe{orthogonaux}{orthogonal} lorsque leur produit scalaire est nul. Nous écrivons que $u\perp v$ lorsque $\langle u, v\rangle =0$.
\begin{definition}	\label{DefNormeEucleApp}
	La \defe{norme euclidienne}{norme euclidienne!dans $\eR^m$} d'un élément de $\eR^m$ est définie par $\| u \|=\sqrt{u\cdot u}$.
\end{definition}

Cette définition est motivée par le fait que le produit scalaire $u\cdot u$ donne exactement la norme usuelle donnée par le théorème de Pythagore :
\begin{equation}
	u\cdot u=\sum_{i=1}^mu_iu_i=\sum_{i=1}^m u_i^2=u_1^2+u_2^2+\ldots+u_m^2.
\end{equation}

Le fait que $e_i\cdot e_j=\delta_{ij}$ signifie que la base canonique est \defe{orthonormée}{orthonormé}, c'est à dire que les vecteurs de la base canonique sont orthogonaux deux à deux et qu'ils ont tout $1$ comme norme.

\begin{lemma}\label{LemSclNormeXi}
	Pour tout $u\in\eR^m$, il existe un $\xi\in\eR^m$ tel que $\| u \|=\xi\cdot u$ et $\| \xi \|=1$.
\end{lemma}

\begin{proof}
	Vérifions que le vecteur $\xi=u/\| u \|$ ait les propriétés requises. D'abord $\| \xi \|=1$ parce que $u\cdot u=\| u \|^2$. Ensuite
	\begin{equation}
		\xi\cdot u=\frac{ u\cdot u }{ \| u \| }=\frac{ \| u \|^2 }{ \| u \| }=\| u \|.
	\end{equation}
\end{proof}

\begin{theorem}[Inégalité de Cauchy-Schwarz]\index{Cauchy-Schwarz}\index{inégalité!Cauchy-Schwarz}      \label{ThoAYfEHG}
	Si $X$ et $Y$ sont des vecteurs, alors
	\begin{equation}
		| X\cdot Y |\leq\| X \|\| Y \|.
	\end{equation}
\end{theorem}

\begin{proof}
	Étant donné que les deux membres de l'inéquation sont positifs, nous allons travailler en passant au carré afin d'éviter les racines carrés dans le second membre.
	
	Nous considérons la fonction
	\begin{equation}
		\varphi(t)=\| X+tY \|=(X+tY)\cdot(X+tY)=X\cdot X+tX\cdot Y+tY\cdot X+t^2Y\cdot Y.
	\end{equation}
	En ordonnant les termes selon les puissance de $t$,
	\begin{equation}
		\varphi(t)=\| Y \|^2t^2+2(X\cdot Y)t+\| X \|^2.
	\end{equation}
	Cela est un polynôme du second degré en $t$. Par conséquent le discriminant\footnote{Le fameux $b^2-4ac$.} doit être négatif. Nous avons donc
	\begin{equation}
		4(X\cdot Y)^2-4\| X \|^2\| Y \|^2\leq 0,
	\end{equation}
	ce qui donne immédiatement
	\begin{equation}
		(X\cdot Y)^2\leq\| X \|^2\| Y^2 \|.
	\end{equation}
	
\end{proof}


%+++++++++++++++++++++++++++++++++++++++++++++++++++++++++++++++++++++++++++++++++++++++++++++++++++++++++++++++++++++++++++
\section{Produit vectoriel}
%+++++++++++++++++++++++++++++++++++++++++++++++++++++++++++++++++++++++++++++++++++++++++++++++++++++++++++++++++++++++++++

\begin{definition}
	Soient $u$ et $v$, deux vecteurs de $\eR^3$. Le \defe{produit vectoriel}{produit!vectoriel} de $u$ et $v$ est le vecteur $u\times v$ défini par 
	\begin{equation}
		\begin{aligned}[]
		u\times v&=\begin{vmatrix}
			e_1	&	e_2	&	e_3	\\
			u_1	&	u_2	&	u_3	\\
			v_1	&	v_2	&	v_3
		\end{vmatrix}\\
		&=
		(u_2v_3-u_3v_2)e_1+(u_3v_1-u_1v_3)e_2+(u_1v_2-u_2v_1)e_3
		\end{aligned}
	\end{equation}
	où les vecteurs $e_1$, $e_2$ et $e_3$ sont les vecteurs de la base canonique de $\eR^3$.
\end{definition}
La notion de produit vectoriel est propre à $\eR^3$; il n'y a pas de généralisation simple aux espaces $\eR^m$.

Nous n'allons pas nous attarder sur les nombreuses propriétés du produit vectoriel. Les principales sont résumées dans la proposition suivante.
\begin{proposition}
	Si $u$ et $v$ sont des vecteurs de $\eR^3$, alors le vecteur $u\times v$ est l'unique vecteur qui est perpendiculaire à $u$ et $v$ en même temps, de norme égal à la surface du parallélogramme construit sur $u$ et $v$ et tel que les vecteurs $u$, $v$, $u\times v$ forment une base dextrogyre.
\end{proposition}
La chose importante à retenir est que le produit vectoriel permet de construire un vecteur simultanément perpendiculaire à deux vecteurs donnés. Le vecteur $u\times v$ est donc linéairement indépendant de $u$ et $v$. En pratique, si $u$ et $v$ sont déjà linéairement indépendants, alors le produit vectoriel permet de compléter une base de $\eR^3$.

À l'aide du produit vectoriel et du produit scalaire, nous construisons le \defe{produit mixte}{produit!mixte} de trois vecteurs de $\eR^3$ par la formule
\begin{equation}
	(u\times v)\cdot w=\begin{vmatrix}
			u_1	&	u_2	&	u_3	\\
			v_1	&	v_2	&	v_3	\\
			w_1	&	w_2	&	w_3	
	\end{vmatrix}.
\end{equation}

Pourquoi nous ne considérons pas la combinaison $(u\cdot v)\times w$ ?

\begin{proposition}		 \label{PropScalMixtLin}
	Les applications produit scalaire, vectoriel et mixte sont multilinéaires. Spécifiquement, nous avons les propriétés suivantes.
	\begin{enumerate}
		\item
			Les applications produit scalaire et vectoriel sont bilinéaires. Le produit mixte est trilinéaire.
		\item
			Le produit vectoriel est antisymétrique, c'est à dire $u\times v=-v\times u$.
		\item
			Nous avons $u\times v=0$ si et seulement si $u$ et $v$ sont colinéaires, c'est à dire si et seulement si l'équation $\alpha u+\beta v=0$ a une solution différente de la solution triviale $(\alpha,\beta)=(0,0)$.
		\item		\label{ItemPropScalMixtLiniv}
			Pour tout $u$ et $v$ dans $\eR^3$, nous avons
			\begin{equation}
				\langle u, v\rangle^2 +\| u\times v \|^2=\| u \|^2\| v \|^2
			\end{equation}
		\item
			Par rapport à la dérivation, le produit scalaire et vectoriel vérifient une règle de Leibnitz. Soit $I$ un intervalle de $\eR$, et si $u$ et $u$ sont dans $C^1(I,\eR^3)$, alors
			\begin{equation}		\label{EqFormLeibProdscalVect}
				\begin{aligned}[]
					\frac{ d }{ dt }\big( u(t)\cdot v(t) \big)&=\big( u'(t)\cdot v(t) \big)+\big( u(t)\cdot v'(t) \big)\\
					\frac{ d }{ dt }\big( u(t)\times v(t) \big)&=\big( u'(t)\times v(t) \big)+\big( u(t)\times v'(t) \big).
				\end{aligned}
			\end{equation}
		\end{enumerate}
\end{proposition}

Les deux formules suivantes, qui mêlent le produit scalaire et le produit vectoriel, sont souvent utiles en analyse vectorielle :
\begin{equation}
	\begin{aligned}[]
		(u\times v)\cdot w&=u\cdot(v\times w)\\
		(u\times v)\times w&=-(v\cdot w)u+(u\cdot w)v		\label{EqFormExpluxxx}
	\end{aligned}
\end{equation}
pour tout vecteurs $u$, $v$ et $w$ dans $\eR^3$. Nous les admettons sans démonstration. La seconde formule est parfois appelée \defe{formule d'expulsion}{formule d'expulsion}.

%+++++++++++++++++++++++++++++++++++++++++++++++++++++++++++++++++++++++++++++++++++++++++++++++++++++++++++++++++++++++++++
\section{Parties libres, génératrices et bases}
%+++++++++++++++++++++++++++++++++++++++++++++++++++++++++++++++++++++++++++++++++++++++++++++++++++++++++++++++++++++++++++

\begin{definition}
    Si \( E\) est un espace vectoriel, une partie finie \( (u_i)_{1\leq i\leq n}\) de \( E\) est \defe{libre}{libre!partie} si l'égalité
    \begin{equation}
        a_1 u_1+\ldots +a_nu_n=0
    \end{equation}
    implique \( a_i=0\) pour tout \( i\).

    Une partie infinie est libre si toute ses parties finies le sont.
\end{definition}
La définition de liberté dans le cas des parties infinies a son importance lorsqu'on parle d'espaces vectoriels de dimension infinies (en dimension finie, aucune partie infinie n'est libre) parce que cela fera une différence entre une base algébrique et une base hilbertienne par exemple.

\begin{theorem} \label{ThoBaseIncompjblieG}
    Soit \( E\) un espace vectoriel de dimension finie et \( \{ e_i \}_{i\in I}\) une partie génératrice de \( E\).

    \begin{enumerate}
        \item
            Il existe \( J\subset I\) tel que \( \{ e_i \}_{i\in J}\) est une base. Autrement dit : de toute partie génératrice nous pouvons extraire une base.
        \item
            Soit \( \{ f_1,\ldots, f_l \}\) une partie libre. Alors nous pouvons la compléter en utilisant des éléments \( e_i\). C'est à dire qu'il existe \( J\subset I\) tel que \( \{ f_k \}\cup\{ e_i \}_{i\in J}\) soit une base.
    \end{enumerate}
\end{theorem}

Soit \( F\) un sous-espace vectoriel de l'espace vectoriel \( E\). La \defe{codimension}{codimension} de \( F\) dans \( E\) est
\begin{equation}
    \codim_E(F)=\dim(E/F).
\end{equation}

Le théorème suivant est valable également en dimension infinie; ce sera une des rares incursions en dimension infinie de ce chapitre.
\begin{theorem}[Théorème du rang]\index{théorème!du rang}       \label{ThoGkkffA}
       Soient \( E\) et \( F\) deux espaces vectoriels (de dimensions finies ou non) et soit \( f\colon E\to F\) une application linéaire. Alors le rang de \( f\) est égal à la codimension du noyau, c'est à dire
       \begin{equation}
           \rang(f)+\dim\ker f=\dim E.
       \end{equation}

       Dans le cas de dimension infinie afin d'éviter les problèmes d'arithmétique avec l'infini nous énonçons le théorème en disant que si \( (u_s)_{s\in S}\) est une base de \( \ker f\) et si \( \big( f(v_t) \big)_{t\in T}\) est une base de \( \Image(f)\) alors  \( (u_s)_{s\in s}\cup (v_t)_{t\in T}\) est une base de \( E\).
\end{theorem}

\begin{proof}
    Nous devons montrer que 
    \begin{equation}
          (u_s)_{s\in S}\cup (v_t)_{t\in T}
    \end{equation}
    est libre et générateur.

    Soit \( x\in E\). Nous définissons les nombres \( x_t\) par la décomposition de \( f(x)\) dans la base \( \big( f(v_t) \big)\) :
    \begin{equation}
        f(x)=\sum_{t\in T}x_tf(v_t).
    \end{equation}
    Ensuite le vecteur \( x=\sum_tx_tv_t\) est dans le noyau de \( f\), par conséquent nous le décomposons dans la base \( (u_s)\) :
    \begin{equation}
        x-\sum_tx_tv_t=\sum_s\in S x_su_s.
    \end{equation}
    Par conséquent
    \begin{equation}
        x=\sum_sx_su_s+\sum_tx_tv_t.
    \end{equation}
    
    En ce qui concerne la liberté nous écrivons
    \begin{equation}
        \sum_tx_tv_t+\sum_sx_su_s=0.
    \end{equation}
    En appliquant \( f\) nous trouvons que 
    \begin{equation}
        \sum_tx_tf(v_t)=0
    \end{equation}
    et donc que les \( x_t\) doivent être nuls. Nous restons avec \( \sum_sx_su_s=0\) qui à son tour implique que \( x_s=0\).
\end{proof}
Un exemple d'utilisation de ce théorème en dimension infinie sera donné dans le cadre du théorème de Fréchet-Riesz, théorème \ref{ThoQgTovL}.

\begin{proposition}[\cite{RombaldiO}]   \label{PropTVKbxU}
    Soit \( E\), un espace vectoriel sur un corps infini et \( (F_k)_{k=1,\ldots, r}\), des sous-espaces vectoriels de \( E\) tels que \( \bigcup_{i=1}^rF_i=E\). Alors \( E=F_k\) pour un certain \( k\).

    Autrement dit, l'union finie de sous-espaces propres ne peut être égal à l'espace complet.
\end{proposition}

%---------------------------------------------------------------------------------------------------------------------------
\subsection{Dualité}
%---------------------------------------------------------------------------------------------------------------------------

\begin{definition}
    Si \( E\) est un espace vectoriel, le \defe{dual}{dual} de \( E\) est l'ensemble des formes linéaires sur \( E\). Le \defe{dual topologique}{dual!topologique} est l'ensemble des formes linéaires continues.

    En dimension infinies, ces deux notions ne coïncident pas.
\end{definition}
%TODO : trouver un exemple où ça ne coïncide pas.
% Si je me souviens bien, pour les opérateurs linéaires, borné est équivalent à continu. Il faudra chercher de ce côté.


Soit \( E\), un espace vectoriel, et \( F\) une sous-espace de \( E\). L'\defe{orthogonal}{orthogonal!sous-espace} de \( F\) est la partie \( F^{\perp}\subset E^*\) donnée par
\begin{equation}    \label{Eqiiyple}
    F^{\perp}=\{ \alpha\in E^*\tq \forall x\in F,\alpha(x)=0 \}.
\end{equation}
Cette définition d'orthogonal via le dual n'est pas du pur snobisme. En effet, la définition «usuelle» qui ne parle pas de dual,
\begin{equation}
    F^{\perp}=\{ y\in E\tq \forall x\in F,y\cdot x=0 \},
\end{equation}
demande la donnée d'un produit scalaire. Évidemment dans le cas de \( \eR^n\) munie du produit scalaire usuel et de l'identification usuelle entre \( \eR^n\) et \( (\eR^n)^*\) via une base, les deux notions d'orthogonal coïncident.

Si \( B\subset E^*\), on note \( B^o\)\nomenclature[G]{\( B^o\)}{orthogonal dans le dual} son orthogonal :
\begin{equation}
    B^o=\{ x\in E\tq \omega(x)=0\,\forall \omega\in B \}.
\end{equation}
Notons qu'on le note \( B^o\) et non \( B^{\perp}\) parce qu'on veut un peu s'abstraire du fait que \( (E^*)^*=E\). Du coup on impose que \( B\) soit dans un dual et on prend une notation précise pour dire qu'on remonte au pré-dual et non qu'on va au dual du dual.

La définition \eqref{Eqiiyple} est intrinsèque : elle ne dépend que de la structure d'espace vectoriel.

Si \( f\colon E\to F\) est une application linéaire entre deux espaces vectoriels, la \defe{transposée}{transposée} est l'application \( f^t\colon F^*\to E^*\) donnée par
\begin{equation}
    f^t(\omega)(x)=\omega\big( f(x) \big).
\end{equation}
pour tout \( \omega\in F^*\) et \( x\in E\).

\begin{proposition}[\cite{DualMarcSAge}]
    Si \( f\) est une application linéaire entre les espaces vectoriels \( E\) et \( F\), alors nous avons
    \begin{equation}
        \Image(f^t)=\ker(f)^{\perp}.
    \end{equation}
\end{proposition}

\begin{proof}
\end{proof}

%+++++++++++++++++++++++++++++++++++++++++++++++++++++++++++++++++++++++++++++++++++++++++++++++++++++++++++++++++++++++++++
\section{Déterminants}
%+++++++++++++++++++++++++++++++++++++++++++++++++++++++++++++++++++++++++++++++++++++++++++++++++++++++++++++++++++++++++++

\begin{proposition}
    Soit \( E\), un \( \eK\)-espace vectoriel de dimension \( n\). L'espace des \( n\)-formes multilinéaires alternées sur \( E\) est de dimension \( 1\).
\end{proposition}
Une preuve sur \href{http://www.les-mathematiques.net/b/e/d/node4.php}{les-mathematiques.net}.

\begin{lemma}   \label{LemcDOTzM}
    Soit \( \eK\) un corps fini autre que \( \eF_2\)\footnote{Je ne comprends pas très bien à quel moment joue cette hypothèse.}, soit un groupe abélien \( M\) et un morphisme \( \varphi\colon \GL(n,\eK)\to M\). Alors il existe un unique morphisme \( \delta\colon \eK^*\to M\) tel que \( \varphi=\delta\circ\det\).
\end{lemma}

\begin{proof}
    D'abord le groupe dérivé de \( \GL(n,\eK)\) est \( \SL(n,\eK)\) parce que les éléments de \( D\big( \GL(n,\eK) \big)\) sont de la forme \( ghg^{-1}h^{-1}\) dont le déterminant est \( 1\). De plus le groupe \( \SL(n,\eK)\) est normal dans \( \GL(n,\eK)\). Par conséquent \( \GL(n,\eK)/\SL(n,\eK)\) est un groupe et nous pouvons définir l'application relevée
    \begin{equation}
        \tilde \varphi\colon \frac{ \GL(n,\eK) }{ \SL(n,\eK) }\to M
    \end{equation}
    vérifiant \( \varphi=\tilde \varphi\circ\pi\) où \( \pi\) est la projection. 

    Nous pouvons faire la même chose avec l'application
    \begin{equation}
        \det\colon \GL(n,\eK)\to \eK^*
    \end{equation}
    qui est un morphisme de groupes dont le noyau est \( \SL(n,\eK)\). Cela nous donne une application
    \begin{equation}
        \tilde \det\colon \frac{ \Gl(n,\eK) }{ \SL(n,\eK) }\to \eK^*
    \end{equation}
    telle que \( \det=\tilde \det\circ\pi\). Cette application \( \tilde \det\) est un isomorphisme. En effet elle est surjective parce que le déterminant l'est et elle est injective parce que son noyau est précisément ce par quoi on prend le quotient. Par conséquent \( \tilde \det \) possède un inverse et nous pouvons écrire
    \begin{equation}
        \varphi=\tilde \varphi\circ\tilde \det^{-1}\circ\tilde \det\circ\pi.
    \end{equation}
    État donné que \( \tilde \det\circ\pi=\det\), nous avons alors \( \varphi=\delta\circ\det\) avec \( \delta=\tilde \varphi\circ\tilde \det^{-1}\). Ceci conclu la partie existence de la preuve.

    En ce qui concerne l'unicité, nous considérons \( \delta'\colon \eK^*\to M\) telle que \( \varphi=\delta'\circ\det\). Pour tout \( u\in \GL(n,\eK)\) nous avons \( \delta'(\det(u))=\varphi(u)=\delta(\det(u))\). L'application \( \det\) étant surjective depuis \( \GL(n,\eK)\) vers \( \eK^*\), nous avons \( \delta'=\delta\).
\end{proof}

\begin{theorem}
    Soit \( p\geq 3\) un nombre premier et \( V\), un \( \eF_p\)-espace vectoriel de dimension finie \( n\). Pour tout \( u\in\GL(V)\) nous avons
    \begin{equation}
        \epsilon(u)=\left(\frac{\det(u)}{p}\right).
    \end{equation}
\end{theorem}
Ici \( \epsilon\) est la signature de \( u \) vue comme une permutation des éléments de \( \eF_p\).

\begin{proof}
    Commençons par prouver que
    \begin{equation}
        \epsilon\colon \GL(V)\to \{ -1,1 \}.
    \end{equation}
    est un morphisme. Si nous notons \( \bar u\in S(V)\) l'élément du groupe symétrique correspondant à la matrice \( u\in \GL(V)\), alors nous avons \( \overline{ uv }=\bar u\circ\bar v\), et la signature étant un homomorphisme (proposition \ref{ProphIuJrC}), 
    \begin{equation}
        \epsilon(uv)=\epsilon(\bar u\circ\bar v)=\epsilon(\bar u)\epsilon(\bar v).
    \end{equation}
    Par ailleurs \( \{ -1,1 \}\) est abélien, donc le lemme \ref{LemcDOTzM} s'applique et nous pouvons considérer un morphisme \( \delta\colon \eF_p^*\to \{ -1,1 \}\) tel que \( \epsilon=\delta\circ\det\).

    Nous allons utiliser le lemme \ref{Lemoabzrn} pour montrer que \( \delta\) est le symbole de Legendre. Pour cela il nous faudrait trouver un \( x\in \eF_p^*\) tel que \( \delta(x)=-1\). Étant donné que \( \det\) est surjective, nous cherchons ce \( x\) sous la forme \( x=\det(u)\). Par conséquent nous aurions
    \begin{equation}
        \delta(x)=(\delta\circ\det)(u)=\epsilon(u),
    \end{equation}
    et notre problème revient à trouver une matrice \( u\in\GL(V)\) dont la permutation associée soit de signature \( -1\).

    Soit \( n=\dim V\); en conséquence de la proposition \ref{PropHfrNCB}\ref{ItemiEFRTg}, l'espace \( \eE_q=\eF_{p^n}\) est un \( \eF_p\)-espace vectoriel de dimension \( n\) et est donc isomorphe en tant qu'espace vectoriel à \( V\). Étant donné que \( \eF_q\) est un corps fini, nous savons que \( \eF_q^*\) est un groupe cyclique à \( q-1\) éléments. Soit \( y\), un générateur de \( \eF_q^*\) et l'application
    \begin{equation}
        \begin{aligned}
            \beta\colon \eF_q&\to \eF_q \\
            x&\mapsto yx. 
        \end{aligned}
    \end{equation}
    Cela est manifestement \( \eF_p\)-linéaire (ici \( y\) et \( x\) sont des classes de polynômes et \( \eF_p\) est le corps des coefficients). L'application \( \beta\) fixe zéro et à part zéro, agit comme le cycle
    \begin{equation}
        (1,y,y^2,\ldots, y^{q-2}).
    \end{equation}
    Nous savons qu'un cycle de longueur \( n\) est de signature \( (-1)^{n+1}\). Ici le cycle est de longueur \( q-1\) qui est pair (pare que \( p\geq 3\)) et par conséquent, l'application \( \beta\) est de signature \( -1\).
\end{proof}

%+++++++++++++++++++++++++++++++++++++++++++++++++++++++++++++++++++++++++++++++++++++++++++++++++++++++++++++++++++++++++++++
\section{Applications linéaires}
%+++++++++++++++++++++++++++++++++++++++++++++++++++++++++++++++++++++++++++++++++++++++++++++++++++++++++++++++++++++++++++++

%---------------------------------------------------------------------------------------------------------------------------
\subsection{Définition}
%---------------------------------------------------------------------------------------------------------------------------

\begin{definition}
	Une application $T: \eR^m\to\eR^n$ est dite \defe{linéaire}{linéaire (application)} si 
\begin{itemize}
\item $T(x+y)=T(x)+T(y)$ pour tout $x$ et $y$ dans $\eR^m$,  
\item $T(\lambda x)=\lambda T(x)$ pour tout $\lambda$ dans $\eR^m$ et $\lambda$ dans $\eR$.
\end{itemize}
\end{definition}

Si $V$ et $W$ sont deux espaces vectoriels réels, nous définissons de la même manière une application linéaire de $V$ dans $W$ comme étant une application $T\colon V\to W$ telle que $T(v_1+v_2)=T(v_1)+T(v_2)$ et $T(\lambda v)=\lambda T(v)$ pour tout $v,v_1,v_2$ dans $V$ et pour tout réel $\lambda$.

L'ensemble des applications linéaires de $\eR^m$ vers $\eR^n$ est noté $\mathcal{L}(\eR^m, \eR^n)$, et plus généralement nous notons $\aL(V,W)$\nomenclature{$\aL(V,W)$}{Ensemble des applications linéaires de $V$ dans $W$} l'ensemble des applications linéaires de $V$ dans $W$. 

\begin{example}
Soit $m=n=1$. Pour tout $b$ dans $\eR$ la fonction $T_b(x)= bx$ est une application linéaire de $\eR$ dans $\eR$. En effet,
\begin{itemize}
\item  $T_b(x+y)= b(x+y)= bx + by = T_b(x)+T_b(y)$,
\item $T_b(ax)=b(ax)= abx = a T_b(x)$.
\end{itemize}
De la même façon on peut montrer que la fonction $T_{\lambda}$ définie par $T_{\lambda}(x)=bx$ est un application linéaire de $\eR^m$ dans $\eR^m$ pour tout $\lambda$ dans $\eR$ et $m$ dans $\eN$.
\end{example}

\begin{example}\label{ex_affine}
	Soit $m=n$. On fixe $\lambda$ dans $\eR$ et $v$ dans $\eR^m$. L'application $U_{\lambda}$ de $\eR^m$ dans $\eR^m$ définie par $U_{\lambda}(x)=\lambda x+v$ n'est pas une application linéaire, parce que 
\[
U_{\lambda}(ax)=\lambda(ax)+v\neq \lambda(bx+v)=a U_{\lambda}(x).
\]
\end{example}

\begin{example}\label{exampleT_A}
	Soit $A$ une matrice fixée de $\mathcal{M}_{n\times m}$\nomenclature{$\mathcal{M}_{n\times m}$}{l'ensemble des matrices $n\times m$}. La fonction $T_A\colon \eR^m\to \eR^n$ définie par $T_A(x)=Ax$ est une application linéaire. En effet, 
\begin{itemize}
\item  $T_A(x+y)= A(x+y)= Ax + Ay = T_A(x)+T_A(y)$,
\item $T_A(ax)=A(ax)= a(Ax) = a T_A(x)$.
\end{itemize}
\end{example}

On peut observer que, si on identifie $\mathcal{M}_{1\times 1}$ et $\eR$, on obtient le premier exemple comme cas particulier.

\begin{proposition}
 Toute application linéaire $T$ de $\eR^m$ dans $\eR^n$ s'écrit de manière unique par rapport aux bases canoniques de $\eR^m$ et $\eR^n$ sous la forme
\[
T(x)=Ax,
\]
avec $A$ dans $\mathcal{M}_{n\times m}$.
\end{proposition}

\begin{proof}
  Soit $x$ un vecteur dans $\eR^m$. On peut écrire $x$ sous la forme $ x=\sum_{i=1}^{m}x_i e_i$. Comme $T$ est une application linéaire on a
\[
T(x)=\sum_{i=1}^{m}x_iT(e_i).
\]
Les images de la base de $\eR^m$, $T(e_j), \, j=1,\ldots,m$, sont des éléments de $\eR^n$, donc on peut les écrire sous la forme de vecteurs
\[
T(e_i)=
\begin{pmatrix}
  a_{1i}\\
\vdots\\
a_{ni}
\end{pmatrix}.
\] 
On obtient alors
\[
T(x)=\sum_{i=1}^{m}x_iT(e_i)=\sum_{i=1}^{m}x_i\begin{pmatrix}
  a_{1i}\\
\vdots\\
a_{ni}
\end{pmatrix}=
\begin{pmatrix}
  a_{11} \ldots a_{1m}\\
\vdots \ddots \vdots\\
 a_{n1} \ldots a_{nm}\\
\end{pmatrix}
\begin{pmatrix}
  x_1\\
\vdots\\
x_m
\end{pmatrix}=Ax.
\]
\end{proof}


\begin{definition}
  Une application $S: \eR^m\to\eR^n$ est dite \defe{affine}{affine (application)} si elle est la somme d'une application linéaire et d'une application constante. Autrement dit, $S$ est affine s'il existe $T: \eR^m\to\eR^n$, linéaire, telle que $S(x)-T(x)$ soit un vecteur constant dans $\eR^n$. 
\end{definition}

\begin{example}
	Les exemples les plus courants d'applications affines sont les droites et les plans ne passant pas par l'origine.
	\begin{description}
		\item[Les droites] Une droite dans $\eR^2$ (ou $\eR^3$) qui ne passe pas par l'origine est le graphe d'une fonction de la forme $s(x)=ax+b$ (ou $s(t)=u x +v$, avec $u$ et $v$  dans $\eR^2$). On reconnait ici la fonction de l'exemple \ref{ex_affine}.
			
		\item[Les plans]
			De la même façon nous savons que tout plan qui ne passe pas par l'origine dans $\eR^3$ est le graphe d'une application affine, $P(x,y)= (a,b)^T\cdot(x,y)^T+(c,d)^T$.
	\end{description}
\end{example}




%+++++++++++++++++++++++++++++++++++++++++++++++++++++++++++++++++++++++++++++++++++++++++++++++++++++++++++++++++++++++++++
\section{Espace vectoriel normé}
%+++++++++++++++++++++++++++++++++++++++++++++++++++++++++++++++++++++++++++++++++++++++++++++++++++++++++++++++++++++++++++

La valeur absolue est essentielle pour introduire les notions de limite et de continuité pour les fonctions d'une variable. En fait nous disons que la fonction $f\colon \eR\to \eR$ est continue au point $a$ lorsque pour tout $\varepsilon$, il existe un $\delta$ tel que
\begin{equation}
	| x-a |\leq\delta \Rightarrow | f(x)-f(a) |\leq \varepsilon.
\end{equation}
La quantité $| x-a |$ donne la «distance» entre $x$ et $a$; la définition de la continuité signifie que pour tout $\varepsilon$, il existe un $\delta$ tel que si $a$ et $x$ sont au plus à la distance $\delta$ l'un de l'autre, alors $f(x)$ et $f(a)$ ne seront éloigné au plus d'une distance $\varepsilon$.

La valeur absolue, dans $\eR$, nous sert donc à mesurer des distances entre les nombres. Les principales propriétés de la valeur absolue sont :
\begin{enumerate}

	\item
		$| x |=0$ implique $x=0$,
	\item
		$| \lambda x |=| \lambda | |x |$,
	\item
		$| x+y |\leq | x |+| y |$

\end{enumerate}
pour tout $x,y\in\eR$ et $\lambda\in\eR$.

Afin de donner une notion de limite pour les fonctions de plusieurs variables, nous devons trouver un moyen de définir les notion de <<taille>> d'un vecteur et de distance entre deux points de $\eR^n$, avec $n>1$. La notion de <<taille>> doit satisfaire propriétés analogues à celles de la valeur absolue. 

La premier notion de <<taille>> pour un vecteur de $\eR^2$ que nous vient à l'esprit est la longueur du segment entre l'origine et l'extrémité libre du vecteur. Cela peut être calculée à l'aide du théorème de Pythagore : 
\begin{equation}
  \textrm{taille de } (a,b) = \sqrt{a^2+b^2}.
\end{equation}
Nous pouvons introduire une la notion de distance entre les éléments de $\eR^2$ de façon similaire :
\begin{equation}
	d\big((a_x,a_y),(b_x,b_y)\big)=\sqrt{  (a_x-b_x)^2+(a_y-b_y)^2  }.
\end{equation}
Cette définition a l'air raisonnable; est-elle mathématiquement correcte ? Peut-elle jouer le rôle de la valeur absolue dans $\eR^2$ ? Est-elle la seule définitions possibles de <<taille>> et distance en $\eR^2$ ?  


%+++++++++++++++++++++++++++++++++++++++++++++++++++++++++++++++++++++++++++++++++++++++++++++++++++++++++++++++++++++++++++
\section{Normes et distances}\label{Sect_definition}
%+++++++++++++++++++++++++++++++++++++++++++++++++++++++++++++++++++++++++++++++++++++++++++++++++++++++++++++++++++++++++++

Nous voulons formaliser les notions de «taille» et de distance dans $\eR^n$, et plus généralement dans un espace vectoriel $V$ de dimension finie. Pour cela nous nous inspirons des propriétés de la valeur absolue.
\begin{definition}		\label{DefNorme}
	Soit $V$ un espace vectoriel réel. Une \defe{norme}{norme!définition} est une application $N\colon V\to \eR^+$ vérifiant les axiomes 
	\begin{enumerate}

		\item
			$N(0_V)=0$, et $N(x)=0$ implique $x=0_V$;
		\item\label{ItemDefNormeii}
			$N(\lambda x)=| \lambda |N(x)$ pour tout $\lambda\in\eR$ et $x\in V$;
		\item\label{ItemDefNormeiii}
			$N(x+y)\leq N(x)+N(y)$ pour tout $x,y\in V$. Cette propriété est appelée \defe{inégalité triangulaire}{inégalité!triangulaire}.
	\end{enumerate}
	Ici et dans la suite, $0_V$ désigne l'élément zéro de l'espace $V$.
\end{definition}
En prenant $\lambda=-1$ dans la propriété \ref{ItemDefNormeii}, nous trouvons immédiatement que $N(-x)=N(x)$.

\begin{proposition}		\label{PropNmNNm}
	Toute norme $N$ sur l'espace vectoriel $V$ vérifie l'inégalité
	\begin{equation}
		\big| N(x)-N(y) \big|\leq N(x-y)
	\end{equation}
	pour tout $x,y\in V$.
\end{proposition}
	
\begin{proof}
	Nous avons, en utilisant le point \ref{ItemDefNormeiii} de la définition \ref{DefNorme},
	\begin{subequations}
		\begin{align}
			N(x)&=N(x-y+y)\leq N(x-y)+N(y),	\label{subEqNNNxxyyya}\\
			N(y)&=N(y-x+x)\leq N(y-x)+N(x).	\label{subEqNNNxxyyyb}
		\end{align}
	\end{subequations}
	Supposons d'abord que $N(x)\geq N(y)$. Dans ce cas, en utilisant \eqref{subEqNNNxxyyya},
	\begin{equation}
		\big| N(x)-N(y) \big|=N(x)-N(y)\leq N(x-y)+N(y)-N(y)=N(x-y).
	\end{equation}
	Si par contre $N(x)\leq N(y)$, alors nous utilisons \eqref{subEqNNNxxyyyb} et nous trouvons
	\begin{equation}
		\big| N(x)-N(y) \big|=N(y)-N(x)\leq N(y-x)+N(x)-N(x)=N(y-x).
	\end{equation}
	Dans les deux cas, nous avons retrouvé l'inégalité annoncée.
\end{proof}
Cette proposition signifie aussi que
\begin{equation}	\label{EqNleqNNleqNvqlqbs}
	-N(x-y)\leq N(x)-N(y)\leq N(x-y).
\end{equation}

Afin de suivre une notation proche de celle de la valeur absolue, à partir de maintenant, la norme d'un vecteur $v$ sera notée $\| v\|$ au lieu de $N(v)$.
\begin{definition}		\label{DefEVNetDistance}
	Un espace vectoriel $V$ muni d'une norme est une \defe{espace vectoriel normé}{normé!espace vectoriel}, et on écrit $(V,\| . \|)$. La \defe{distance induite}{distance (d'une norme)} par la norme entre les points $a$ et $b$ de $V$ est le nombre $d(a,b)=\| a-b \|$.

	Si $A$ est une partie de $V$ et si $x\in V$, nous disons que la \defe{distance}{distance!point et ensemble} entre $A$ et $x$ est le nombre
	\begin{equation}		\label{EqdefDistaA}
		d(x,A)=\inf_{a\in A}d(x,a).
	\end{equation}
\end{definition}
%The result is on the figure \ref{LabelFigDistanceEnsemble}
\newcommand{\CaptionFigDistanceEnsemble}{La distance entre $x$ et $A$ est donnée par la distance entre $x$ et $p$. Les distances entre $x$ et les autres points de $A$ sont plus grandes que $d(x,p)$.}
\input{Fig_DistanceEnsemble.pstricks}

Il est possible de définir de nombreuses normes sur $\eR^n$. Citons en quelque unes. Les normes $\| . \|_{L^p}$ ($p\in\eN$) sont définies de la façon suivante :
\begin{equation}		\label{EqDeformeLp}
	\| x \|_{L^p}=\Big( \sum_{i=1}^n| x_i |^p\Big)^{1/p},
\end{equation}
pour tout $x=(x_1,\ldots,x_n)\in\eR^n$. Parmi ces normes, celles qui seront le plus souvent utilisées dans ces notes sont
\begin{equation}
	\begin{aligned}[]
		\| x \|_{L^1}&=\sum_{i=1}^n| x_i |,\\
		\| x \|_{L^2}&=\Big( \sum_{i=1}^n| x_i |^2 \Big)^{1/2}.
	\end{aligned}
\end{equation}
La norme $L^2$ est la \defe{norme euclidienne}{norme!euclidienne}. Nous définissons également la \defe{norme supremum}{norme!supremum} par
\begin{equation}
	\| x \|_{\infty}=\sup_{1\leq i\leq n}| x_i |.
\end{equation}
Nous admettons sans démonstration que les fonctions $\| . \|_{L^p}\colon \eR^n\to \eR^+$ sont bien des normes.

\newcommand{\CaptionFigDistanceEuclide}{La \emph{norme} euclidienne induit la \emph{distance} euclidienne. D'où son nom. Le point $C$ est construit aux coordonnées $(A_x,B_y)$.}
\input{Fig_DistanceEuclide.pstricks}

Soient $A=(A_x,A_y)$ et $B=(B_x,B_y)$ deux éléments de $\eR^2$. La distance\footnote{Ne pas confondre «distance» et «norme».} euclidienne entre $A$ et $B$ est donnée par $\| A-B \|_2$. En effet, sur la figure \ref{LabelFigDistanceEuclide}, la distance entre les points $A$ et $B$ est donnée par
\begin{equation}
	| AB |^2=| AC |^2+| CB |^2=| A_x-B_x |^2+| A_y-B_y |^2,
\end{equation}
par conséquent,
\begin{equation}
	| AB |=\sqrt{| A_x-B_x |^2+| A_y-B_y |^2}=\| A-B \|_2.
\end{equation}

\begin{remark}
	Si $A$, $B$ et $C$ sont trois points dans le plan $\eR^2$, alors l'inégalité triangulaire $| AB |\leq| AC |+| CB |$ est précisément la propriété \ref{ItemDefNormeiii} de la norme (définition \ref{DefNorme}). En effet l'inégalité triangulaire s'exprime de la façon suivante en terme de la norme $\| . \|_2$ :
	\begin{equation}	\label{EqNDeuxAmBNNdd}
		\| A-B \|_2\leq \| A-C \|_2+\| C-B \|_2.
	\end{equation}
	En notant $u=A-C$ et $v=C-B$, l'équation \eqref{EqNDeuxAmBNNdd} devient exactement la propriété de définition de la norme :
	\begin{equation}
		\| u+v \|_2\leq \| u \|_2+\| v \|_2.
	\end{equation}
	Ceci explique pourquoi cette propriété des norme est appelée «inégalité triangulaire».
\end{remark}

Les distances que nous avons vues jusqu'à présent sont des distances définies à partir d'une norme. La définition suivante donne une notion générale de distance sur un espace vectoriel \( V\).

\begin{definition}
    Soit \( V\) un espace vectoriel. Une \defe{distance}{distance} sur \( V\) est une application \( d\colon V\times V\to \eR\) telle que
    \begin{enumerate}
        \item
            \( d(x,y)\geq 0\) pour tout \( x,y\in V\);
        \item
            \( d(x,y)=0\) si et seulement si \( x=y\);
        \item
            \( d(x,y)=d(y,x)\) pour tout \( x,y\in V\);
        \item
            \( d(x,y)\leq d(x,z)+d(z,y)\) pour tout \( x,y,z\in V\).
    \end{enumerate}
    La dernière condition est l'inégalité triangulaire. Le nombre \( d(x,y)\) est la \emph{distance} entre \( x\) et \( y\).
\end{definition}
Toute distance définit une norme en posant \( \| v \|=d(v,0)\).

%+++++++++++++++++++++++++++++++++++++++++++++++++++++++++++++++++++++++++++++++++++++++++++++++++++++++++++++++++++++++++++
\section{Boules et sphères}\label{Sect_boules}
%+++++++++++++++++++++++++++++++++++++++++++++++++++++++++++++++++++++++++++++++++++++++++++++++++++++++++++++++++++++++++++

\begin{definition}
	Soit $(V,\| . \|)$, un espace vectoriel normé, $a\in V$ et $r>0$. Nous allons abondamment nous servir des ensembles suivants :
	\begin{enumerate}

		\item
			la \defe{boule ouverte}{boule!ouverte} $B(a,r)=\{ x\in V\tq \| x-a \|<r \}$;
		\item
			la \defe{boule fermée}{boule!fermée} $\bar B(a,r)=\{ x\in V\tq \| x-a \|\leq r \}$;
		\item
			la \defe{sphère}{sphère} $S(a,r)=\{ x\in V\tq \| x-a \|=r \}$.

	\end{enumerate}
\end{definition}
Les différences entre ces trois ensembles sont très importantes. D'abord, les \emph{boules} sont pleines tandis que la \emph{sphère} est creuse. En comparant à une pomme, la boule ouverte serait la pomme «sans la peau», la boule fermée serait «avec la peau» tandis que la sphère serait seulement la peau. Nous avons
\begin{equation}
	\bar B(a,r)=B(a,r)\cup S(a,r).
\end{equation}

\begin{definition}
	Une partie $A$ de $V$ est dite \defe{bornée}{borné!partie de $V$} si il existe un réel $R$ tel que $A\subset B(0_V,R)$.
\end{definition}
Une partie est donc bornée si elle est contenue dans une boule de rayon fini.

\begin{example}
	Dans $\eR$, les boules sont  les intervalles ouverts et fermés tandis que la sphère est donnée par les points extrêmes des intervalles :
	\begin{equation}
		\begin{aligned}[]
			B(a,r)&=\mathopen] a-r , a+r \mathclose[,\\
			\bar B(a,r)&=\mathopen[ a-r , a+b \mathclose],\\
			S(a,r)&=\{ a-r,a+r \}.
		\end{aligned}
	\end{equation}
\end{example}

\begin{example}
	Si nous considérons $\eR^2$, la situation est plus riche parce que nous avons plus de normes. Essayons de voir les sphères de centre $(0,0)\in\eR^2$ et de rayon $r$ pour les normes $\| . \|_1$, $\| . \|_2$ et $\| . \|_{\infty}$.

	Pour la norme $\| . \|_1$, la sphère de rayon $r$ est donnée par l'équation
	\begin{equation}
		| x |+| y |=r.
	\end{equation}
	Pour la norme $\| . \|_2$, l'équation de la sphère de rayon $r$ est
	\begin{equation}
		\sqrt{x^2+y^2}=r,
	\end{equation}
	et pour la norme supremum, la sphère de rayon $r$ a pour équation
	\begin{equation}
		\max\{ | x |,| y | \}=r.
	\end{equation}
	Elles sont dessinées sur la figure \ref{LabelFigLesSpheres}
\newcommand{\CaptionFigLesSpheres}{Les sphères de rayon $1$ pour les trois normes classiques.}
\input{Fig_LesSpheres.pstricks}
\end{example}

\newcommand{\CaptionFigBoulePtLoin}{Le point $P$ est un peu plus loin que $x$, en suivant la même droite.}
\input{Fig_BoulePtLoin.pstricks}

\begin{proposition}		\label{PropBoitPtLoin}
	Soient $V$ un espace vectoriel normé, $a$ dans $V$ et $x$ tel que $d(a,x)=r$, c'est à dire $x\in S(a,r)$. Dans ce cas, toute boule centrée en $x$ contient un point $P$ tel que $d(P,a)>r$ et un point $Q$ tel que $d(Q,a)<r$.
\end{proposition}

\begin{proof}
	Soit une boule de rayon $\delta$ autour de $x$. Le but est de trouver un point $P$ tel que $d(P,a)>r$ et $d(P,x)<\delta$. Pour cela, nous prenons $P$ sur la même droite que $x$ (en partant de $a$), mais juste «un peu plus loin» (voir figure \ref{LabelFigBoulePtLoin}). Plus précisément, nous considérons le point
	\begin{equation}
		P=x+\frac{ v }{ N }
	\end{equation}
	où $v=x-a$ et $N$ est suffisamment grand pour que $d(x,P)$ soit plus petit que $\delta$. Cela est toujours possible parce que
	\begin{equation}
		d(P,x)=\| P-x \|=\frac{ \| v \| }{ N }
	\end{equation}
	peut être rendu aussi petit que l'on veut par un choix approprié de $N$. Montrons maintenant que $d(a,P)>d(a,x)$ :
	\begin{equation}
		\begin{aligned}[]
			d(a,P)&=\| a-x-\frac{ v }{ N }\| \\
			&=\| a-x+\frac{ a }{ N }-\frac{ x }{ N } \|\\
			&=\| \big( 1+\frac{1}{ N }(a-x) \big) \|\\
			&>\| a-x \|=d(a,x).
		\end{aligned}
	\end{equation}
	Nous laissons en exercice le soin de trouver un point $Q$ tel que $d(Q,a)<r$ et $d(Q,x)<\delta$.
\end{proof}

%+++++++++++++++++++++++++++++++++++++++++++++++++++++++++++++++++++++++++++++++++++++++++++++++++++++++++++++++++++++++++++
\section{Topologie}\label{Sect_topologie}
%+++++++++++++++++++++++++++++++++++++++++++++++++++++++++++++++++++++++++++++++++++++++++++++++++++++++++++++++++++++++++++

%---------------------------------------------------------------------------------------------------------------------------
\subsection{Ouverts, fermés, intérieur et adhérence}
%---------------------------------------------------------------------------------------------------------------------------

\begin{definition}
	Soit $(V,\| . \|)$ un espace vectoriel normé et $A$, une partie de $V$. Un point $a$ est dit \defe{intérieur}{intérieur!point} à $A$ si il existe une boule ouverte centrée en $a$ et contenue dans $A$.

	On appelle \defe{l'intérieur}{intérieur!d'un ensemble} de $A$ l'ensemble des points qui sont intérieurs à $A$. Nous notons $\Int(A)$ l'intérieur de $A$.
\end{definition}
Notons que $\Int(A)\subset A$ parce que si $a\in\Int(A)$, nous avons $B(a,r)\subset A$ pour un certain $r$ et en particulier $a\in A$.

\begin{example}
	Trouver l'intérieur d'un intervalle dans $\eR$ consiste à «ouvrir là où c'est fermé». 
	\begin{enumerate}

		\item
			$\Int\big(\mathopen[ 0 , 1 [\big)=\mathopen] 0 , 1 \mathclose[$. 
			
			Prouvons d'abord que $\mathopen] 0,1  \mathclose[\subset\Int(\mathopen[ 0 , 1 [)$. Si $a\in\mathopen] 0 , 1 \mathclose[$, alors $a$ est strictement supérieur à $0$ et strictement inférieur à $1$. Dans ce cas, la boule de centre $a$ et de rayon $\frac{ \min\{ a,1-a \} }{ 2 }$ est contenue dans $\mathopen] 0 , 1 \mathclose[$ (voir figure \ref{LabelFigIntervalle}). Cela prouve que $a$ est dans l'intérieur de $\mathopen[ 0 , 1 [$.

\newcommand{\CaptionFigIntervalle}{Trouver le rayon d'une boule autour de $a$. Une boule qui serait centrée en $a$ avec un rayon strictement plus petit à la fois de $a$ et de $1-a$ est entièrement contenue dans le segment $\mathopen] 0 , 1 \mathclose[$.}
\input{Fig_Intervalle.pstricks}

			Prouvons maintenant que $\Int\big( \mathopen[ 0 , 1 [ \big)\subset\mathopen] 0 , 1 \mathclose[$. Vu que l'intérieur d'un ensemble est inclus à l'ensemble, nous savons déjà que $\Int\big( \mathopen[ 0 , 1 [ \big)\subset\mathopen[ 0 , 1 [$. Nous devons donc seulement montrer que $0$ n'est pas dans l'intérieur de $\mathopen[ 0 , 1 [$. C'est le cas parce que toute boule du type $B(0,r)$ contient le point $-r/2$ qui n'est pas dans $\mathopen[ 0 , 1 [$.

		\item
			$\Int\Big( \mathopen[ 0 , \infty [ \Big)=\mathopen] 0 , \infty \mathclose[$.
		\item
			$\Int\big( \mathopen] 2 , 3 \mathclose[ \big)=\mathopen] 2 , 3 \mathclose[$.

	\end{enumerate}
	
\end{example}

\begin{example}			\label{ExempleIntBoules}
	Les intérieurs des boules et sphères sont importantes à savoir.
	\begin{enumerate}
		\item 
			$\Int\big( B(a,r) \big)=B(a,r)$. Si $x\in B(a,r)$, nous avons $d(a,x)<r$. Alors la boule $B\big(x,r-d(x,a)\big)$ est incluse à $B(a,r)$, et donc $x$ est dans l'intérieur de $B(a,r)$. Conseil : faire un dessin.
		\item
			$\Int\big( \bar B(a,r) \big)=B(a,r)$. Par le point précédent, la boule $B(a,r)$ est certainement dans l'intérieur de la boule fermée. Il reste à montrer que les points de $\bar B(a,r)$ qui ne sont pas dans $B(a,r)$ ne sont pas dans l'intérieur. Ces points sont ceux dont la distance à $a$ est \emph{égale} à $r$. Le résultat découle alors de la proposition \ref{PropBoitPtLoin}.
			
		\item
			$\Int\big( S(a,r) \big)=\emptyset$. Si $x\in S(a,r)$, toute boule centrée en $a$ contient des points qui ne sont pas à distance $r$ de $a$.
			
			Notez que la sphère est un exemple d'ensemble non vide mais d'intérieur vide.
	\end{enumerate}
\end{example}


\begin{definition}
	Une partie $A$ de l'espace vectoriel normé $(V,\| . \|)$ est dite \defe{ouverte}{ouvert} si chacun de ses points est intérieur. La partie $A$ est donc ouverte si $A\subset\Int(A)$. Par convention, nous disons que l'ensemble vide $\emptyset$ est ouvert.

	Une partie est dite \defe{fermée}{fermé} si son complémentaire est ouvert. La partie $A$ est donc fermée si $V\setminus A$ est ouverte.
\end{definition}

Remarque : un ensemble $A$ est ouvert si et seulement si $\Int(A)=A$.

\begin{definition}
	Une partie $A$ de l'espace vectoriel normé $V$ est dite \defe{compacte}{compact} si elle est fermée et bornée.
\end{definition}

Nous verrons tout au long de ce cours que les ensembles compacts, et les fonctions définies sur ces ensembles ont de nombreuses propriétés agraables.

\begin{example}		\label{ExempleFermeIntevrR}
	En ce qui concerne les intervalles de $\eR$,
	\begin{itemize}
		\item $\mathopen] 1 , 2 \mathclose[$ est ouvert;
		\item $\mathopen[ 3,  4 \mathclose]$ est fermé;
		\item $\mathopen[ 5 , 6 [$ n'est ni ouvert ni fermé;
	\end{itemize}
	Les intervalles fermés de $\eR$ sont toujours compacts.
\end{example}

\begin{proposition}		\label{PropTopologieAx}
	Soit $V$ un espace vectoriel normé.
	\begin{enumerate}
		\item
			L'ensemble $V$ lui-même et le vide sont à la fois fermées et ouvertes.
		\item
			Toute union d'ouverts est ouverte.
		\item
			Toute intersection \emph{finie} d'ouverts est ouverte.
		\item		\label{ItemPropTopologieAxiv}
			Le vide et $V$ sont les seules parties de $V$ à être à la fois fermées et ouvertes.
	\end{enumerate}
\end{proposition}

\begin{proof}
	L'ingrédient principal de cette démonstration est que si $a$ est un point d'un ouvert $\mO$, alors il existe une boule autour de $a$ contenue dans $\mO$ parce que $a$ doit être dans l'intérieur de $\mO$.
	\begin{enumerate}

		\item
			Nous avons déjà dit que, par définition, l'ensemble vide est ouvert. Cela implique que $V$ lui-même est fermé (parce que son complémentaire est le vide). De plus, $V$ est ouvert parce que toutes les boules sont inclues à $V$. Le vide est alors fermé (parce que son complémentaire est $V$).
		\item
			Soit une famille $(\mO_i)_{i\in I}$ d'ouverts\footnote{L'ensemble $I$ avec lequel nous «numérotons» les ouverts $\mO_i$ est \emph{quelconque}, c'est à dire qu'il peut être $\eN$, $\eR$, $\eR^n$ ou n'importe quel autre ensemble, fini ou infini.}, et l'union
			\begin{equation}
				\mO=\bigcup_{i\in I}\mO_i.
			\end{equation}
			Soit maintenant $a\in\mO$. Nous devons prouver qu'il existe une boule centrée en $a$ entièrement contenue dans $\mO$. Étant donné que $a\in\mO$, il existe $i\in I$ tel que $a\in\mO_i$ (c'est à dire que $a$ est au moins dans un des $\mO_i$). Par hypothèse l'ensemble $\mO_i$ est ouvert et donc tous ses points (en particulier $a$) sont intérieurs; il existe donc une boule $B(a,r)$ centrée en $a$ telle que $B(a,r)\subset\mO_i\subset\mO$.
		
		\item
			Soit une famille finie d'ouverts $(\mO_k)_{k\in\{ 1,\ldots,n \}}$, et $a\in\mO$ où
			\begin{equation}
				\mO=\bigcap_{k=1}^n\mO_k.
			\end{equation}
			Vu que $a$ appartient à chaque ouvert $\mO_k$, nous pouvons trouver, pour chacun de ces ouverts, une boule $B(a,r_k)$ contenue dans $\mO_k$. Chacun des $r_k$ est strictement positif, et nous n'en avons qu'un nombre fini, donc le nombre $r=\min\{ r_1,\ldots,r_n \}$ est strictement positif. La boule $B(a,r)$ est inclue dans toutes les autres (parce que $B(a,r)\subset B(a,r')$ lorsque $r\leq r'$), par conséquent
			\begin{equation}
				B(a,r)\subset\bigcap_{k=1}^nB(a,r_k)\subset\bigcap_{k=1}^n\mO_k=\mO,
			\end{equation}
			c'est à dire que la boule de rayon $r$ est une boule centrée en $a$ contenue dans $\mO$, ce qui fait que $a$ est intérieur à $\mO$.
		\item
			Nous acceptons ce point sans démonstration. 
	\end{enumerate}
	
\end{proof}

La proposition dit que toute intersection \emph{finie} d'ouvert est ouverte. Il est faux de croire que cela se généralise aux intersections infinies, comme le montre l'exemple suivant :
\begin{equation}
	\bigcap_{i=1}^{\infty}\mathopen] -\frac{1}{ n } , \frac{1}{ n } \mathclose[=\{ 0 \}.
\end{equation}
Chacun des ensembles $\mathopen] -\frac{1}{ n } , \frac{1}{ n } \mathclose[$ est ouvert, mais le singleton $\{ 0 \}$ est fermé (pourquoi ?).

Nous reportons à la proposition \ref{PropBorneSupInf} la preuve du fait que tout ensemble borné de $\eR$ possède un infimum et un supremum.



\begin{definition}
	L'ensemble des ouverts de $V$ est la \defe{topologie}{topologie} de $V$. La topologie dont nous parlons ici est dite \defe{induite}{induite!topologie} par la norme $\| . \|$ de $V$ (parce que cette norme définit la notion de boule et qu'à son tour la notion de boule définit la notion d'ouverts). Un \defe{voisinage}{voisinage} de $a$ dans $V$ est un ensemble contenant un ouvert contenant $a$.
\end{definition}

Il existe de nombreuses topologies sur un espace vectoriel donné, mais certaines sont plus fameuses que d'autres. Dans le cas de $V=\eR^n$, la topologie \defe{usuelle}{topologie!usuelle sur $\eR^n$} est celle induite par la norme euclidienne. Lorsque nous parlons de boules, de fermés, de voisinages ou d'autres notions topologiques (y compris de convergence, voir plus bas) dans $\eR^n$, nous sous-entendons toujours la topologie de la norme euclidienne.

\begin{example}
	Les ensemble suivants sont des voisinages de $3$ dans $\eR$ :
	\begin{itemize}
		\item
			$\mathopen] 1 , 5 \mathclose[$;
		\item
			$\mathopen[ 0 , 10 \mathclose]$;
		\item
			$\eR$.
	\end{itemize}
	Les ensembles suivants ne sont pas des voisinages de $3$ dans $\eR$ :
	\begin{itemize}
		\item 
			$\mathopen] 1 , 3 \mathclose[$;
		\item
			$\mathopen] 1 , 3 \mathclose]$;
		\item
			$\mathopen[ 0 , 5 [\setminus\{ 3 \}$.
	\end{itemize}
\end{example}

\begin{proposition}
	Dans un espace vectoriel normé,
	\begin{enumerate}
		\item
			toute intersection de fermés est fermée;
		\item
			toute union \emph{finie} de fermés est fermée.
	\end{enumerate}
\end{proposition}
Encore une fois, l'hypothèse de finitude de l'intersection est indispensable comme le montre l'exemple suivant :
\begin{equation}
	\bigcup_{n=1}^{\infty}\mathopen[ -1+\frac{1}{ n } , 1-\frac{1}{ n } \mathclose]=\mathopen] -1 , 1 \mathclose[.
\end{equation}
Chacun des intervalles dont on prend l'union est fermé tandis que l'union est ouverte.

\begin{definition}
	Soit $A$, une partie de l'espace vectoriel normé $V$. Un point $a\in V$ est dit \defe{adhérent}{adhérence} à $A$ dans $V$ si pour tout $\varepsilon>0$,
	\begin{equation}
		B(a,\varepsilon)\cap A\neq\emptyset.
	\end{equation}
	Nous notons $\bar A$ l'ensemble des points adhérents à $a$ et nous disons que $\bar A$ est l'adhérence de $A$. L'ensemble $\bar A$ sera aussi souvent nommé \defe{fermeture}{fermeture} de l'ensemble $A$.
\end{definition}
Un point peut être adhérent à $A$ sans faire partie de $A$, et nous avons toujours $A\subset\bar A$.

\begin{example}
	La terminologie «fermeture» de $A$ pour désigner $\bar A$ provient de deux origines.
	\begin{enumerate}
		\item
			L'ensemble $\bar A$ est le plus petit fermé contenant $A$. Cela signifie que si $B$ est un fermé qui contient $A$, alors $\bar A\subset A$. Nous acceptons cela sans preuve.
            % position 25804
            %Nous allons prouver cette affirmation dans l'exercice \ref{exoGeomAnal-0008}.
		\item
			Pour les intervalles dans $\eR$, trouver $\bar A$ revient à fermer les extrémités qui sont ouvertes, comme on en a parlé dans l'exemple \ref{ExempleFermeIntevrR}.
	\end{enumerate}
\end{example}

\begin{example}
	Dans $\eR$, l'infimum et le supremum d'un ensemble sont des points adhérents. En effet si $M$ est le supremum de $A\subset\eR$, pour tout $\varepsilon$, il existe un $a\in A$ tel que $a>M-\varepsilon$, tandis que $M>a$. Cela fait que $a\in B(M,\varepsilon)$, et en particulier que pour tout rayon $\varepsilon$, nous avons $B(M,\varepsilon)\cap A\neq\emptyset$.

	Le même raisonnement montre que l'infimum est également dans l'adhérence de $A$.
\end{example}

\begin{example}		\label{ParlerEncoredeF}
	Il ne faut pas conclure de l'exemple précédent qu'un point limite ou adhérent est automatiquement un minimum ou un maximum. En effet, si nous regardons l'ensemble formé par les points de la suite $x_n=(-1)^n/n$, le nombre zéro est un point adhérent et une limite, mais pas un infimum ni un maximum.
\end{example}

\begin{lemma}
	Si $B$ est une partie fermée de $V$, alors $B=\bar B$.
\end{lemma}

\begin{proof}
	Supposons qu'il existe $a\in\bar B$ tel que $a\notin B$. Alors il n'y a pas d'ouverts autour de $a$ qui soit contenu dans $\complement B$. Cela prouve que $\complement B$ n'est pas ouvert, et par conséquent que $B$ n'est pas fermé. Cela est une contradiction qui montre que tout point de $\bar B$ doit appartenir à $B$ lorsque $B$ est fermé.
\end{proof}

\begin{example}
	Au niveau des intervalles dans $\eR$, prendre l'adhérence consiste à «fermer là où c'est ouvert». Attention cependant à ne pas fermer l'intervalle en l'infini.
	\begin{enumerate}
		\item
			$\overline{ \mathopen[ 0 , 2 [ }=\mathopen[ 0 , 2 \mathclose]$.
		\item
			$\overline{ \mathopen] 3 , \infty \mathopen] }=\mathopen[ 3 , \infty [$.
	\end{enumerate}
\end{example}

\begin{proposition}
	Soit $V$ un espace vectoriel normé et $a\in V$. Les trois conditions suivantes sont équivalentes :
	\begin{enumerate}
		\item
			$a\in\bar A$;
		\item
			il existe une suite d'éléments $x_n$ dans $A$ qui converge vers $a$;
		\item
			$d(a,A)=0$.
	\end{enumerate}
\end{proposition}
Notez que dans cette proposition, nous ne supposons pas que $a$ soit dans $A$.

\begin{proposition}		\label{PropComleIntBar}
	Pour toute partie $A$ d'un espace vectoriel normé nous avons
	\begin{enumerate}
		\item
			$V\setminus\bar A=\Int(V\setminus A)$,
		\item
			$V\setminus\Int(A)=\overline{ V\setminus A }$.
	\end{enumerate}
\end{proposition}

En utilisant les notations du complémentaire (appendice \ref{AppComplement}), les deux points de la proposition se récrivent
\begin{enumerate}
	\item
		$\complement \bar A=\Int(\complement A)$,
	\item\label{ItemLemPropComplementiv}
		$\complement\Int(A)=\overline{ \complement A }$.
\end{enumerate}

\begin{proof}
	Nous avons $a\in V\setminus\bar A$ si et seulement si $a\notin\bar A$. Or ne pas être dans $\bar A$ signifie qu'il existe un rayon $\varepsilon$ tel que la boule $B(a,\varepsilon)$ n'intersecte pas $A$. Le fait que la boule $B(a,\varepsilon)$ n'intersecte pas $A$ est équivalent à dire que $B(a,\varepsilon)\subset V\setminus A$. Or cela est exactement la définition du fait que $a$ est à l'intérieur de $V\setminus A$. Nous avons donc montré que $a\in V\setminus \bar A$ si et seulement si $a\in\Int(V\setminus A)$. Cela prouve la première affirmation.

	Pour prouver la seconde affirmation, nous appliquons la première au complémentaire de $A$ : $\complement(\overline{ \complement A })=\Int(\complement\complement A)$. En prenant le complémentaire des deux membres nous trouvons successivement
	\begin{equation}
		\begin{aligned}[]
			\complement\complement(\overline{ \complement A })&=\complement\Int(\complement\complement A),\\
			\overline{ \complement A }&=\complement\Int(A),
		\end{aligned}
	\end{equation}
	ce qu'il fallait démontrer.
\end{proof}

Attention à ne pas confondre $\complement \bar A$ et $\overline{ \complement A }$. Ces deux ensembles ne sont pas égaux. En effet, en tant que complément d'un fermé, l'ensemble $\complement \bar A$ est certainement ouvert, tandis que, en tant que fermeture, l'ensemble $\overline{ \complement A }$ est fermé. Pouvez-vous trouver des exemples d'ensembles $A$ tels que $\complement \bar A=\overline{ \complement A }$ ?

\begin{proposition}
	Soient $A$ et $B$ deux parties de l'espace vectoriel normé $V$.
	\begin{enumerate}
		\item
			Pour les inclusions, si $A\subset B$, alors $\Int(A)\subset\Int(B)$ et $\bar A\subset\bar B$.
		\item
			Pour les unions, $\overline{ A\cup B }=\overline{ A }\cup\overline{ B }$ et $\overline{ A\cap B }\subset\bar A\cap\bar B$.
		\item
			Pour les intersections, $\Int(A)\cap\Int(B)=\Int(A\cap B)$ et $\Int(A)\cup\Int(B)\subset\Int(A\cup B)$.
	\end{enumerate}
\end{proposition}

\begin{proof}
	\begin{enumerate}
		\item
			Si $a$ est dans l'intérieur de $A$, il existe une boule autour de $a$ contenue dans $A$. Cette boule est alors contenue dans $B$ et donc est une boule autour de $a$ contenue dans $B$, ce qui fait que $a$ est dans l'intérieur de $B$. Si maintenant $a$ est dans l'adhérence de $A$, toute boule centrée en $a$ contient un élément de $A$ et donc un élément de $B$, ce qui prouve que $a$ est dans l'adhérence de $B$.
		\item
			Nous avons $A\subset A\cup B$ et donc, en utilisant le premier point, $\bar A\subset\overline{ A\cup B }$. De la même manière, $\bar B\subset\overline{ A\cup B }$. En prenant l'union, $\bar A\cup\bar B\subset\overline{ A\cup B }$.

			Réciproquement, soit $a\in\overline{ A\cup B }$ et montrons que $a\in\bar A\cup\bar B$. Supposons par l'absurde que $a$ ne soit ni dans $\bar A$ ni dans $\bar B$. Il existe donc des rayon $\varepsilon_1$ et $\varepsilon_2$ tels que
			\begin{equation}
				\begin{aligned}[]
					B(a,\varepsilon_1)\cap A&=\emptyset,\\
					B(a,\varepsilon_2)\cap B&=\emptyset.
				\end{aligned}
			\end{equation}
			En prenant $r=\min\{ \varepsilon_1,\varepsilon_2 \}$, la boule $B(a,r)$ est inclue aux deux boules citées et donc n'intersecte ni $A$ ni $B$. Donc $a\notin\overline{ A\cup B }$, d'où la contradiction.

		\item
			Si nous appliquons le second point à $\complement A$ et $\complement B$, nous trouvons
			\begin{equation}
				\overline{ \complement A\cup\complement B }=\overline{ \complement A}\cup\overline{ \complement B}.
			\end{equation}
			En utilisant les propriétés du lemme \ref{LemPropsComplement}, le membre de gauche devient
			\begin{equation}	\label{Eq2707CACBCAB}
				\overline{ \complement A\cup\complement B }=\overline{ \complement(A\cap B) }=\complement\Int(A\cap B),
			\end{equation}
			tandis que le membre de droite devient
			\begin{equation}		\label{Eq2707cAcBACAACB}
				\overline{ \complement A }\cup\overline{ \complement B }=\complement\Int(A)\cup\complement\Int(A)=\complement\Big( \Int(A)\cap\Int(B) \Big).
			\end{equation}
			En égalisant le membre de droite de \eqref{Eq2707CACBCAB} avec celui de \eqref{Eq2707cAcBACAACB} et en passant au complémentaire nous trouvons
			\begin{equation}
				\Int(A\cap B)=\Int(A)\cap\Int(B),
			\end{equation}
			comme annoncé.

			La dernière affirmation provient du fait que $\Int(A)\subset\Int(A\cup B)$ et de la propriété équivalente pour $B$.
	\end{enumerate}
\end{proof}

\begin{remark}
	Nous avons prouvé que $\overline{ A\cap B }\subset\bar A\cap\bar B$. Il arrive que l'inclusion soit stricte, comme dans l'exemple suivant. Si nous prenons $A=\mathopen[ 0 , 1 \mathclose]$ et $B=\mathopen] 1 , 2 \mathclose]$, nous avons $A\cap B=\emptyset$ et donc $\overline{ A\cap B }=\emptyset$. Par contre nous avons $\bar A\cap\bar B=\{ 1 \}$.
\end{remark}

\begin{definition}
	La \defe{frontière}{frontière} d'un sous-ensemble $A$ de l'espace vectoriel normé $V$ est l'ensemble des points $a\in V$ tels que
	\begin{equation}
		\begin{aligned}[]
			B(a,r)\cap A&\neq \emptyset,\\
			B(a,r)\cap \complement A&\neq \emptyset,
		\end{aligned}
	\end{equation}
	pour tout rayon $r$. En d'autres termes, toute boule autour de $a$ contient des points de $A$ et des points de $\complement A$. La frontière de $A$ se note $\partial A$\nomenclature[T]{$\partial A$}{La frontière de l'ensemble $A$}.
\end{definition}

\begin{proposition}		\label{PropDescFrpbsmI}
	La frontière d'une partie $A$ d'un espace vectoriel normé $V$ s'exprime sous la forme
	\begin{equation}
		\partial A=\bar A\setminus\Int(A).
	\end{equation}
\end{proposition}

\begin{proof}
	Le fait pour un point $a$ de $V$ d'appartenir à $\bar A$ signifie que toute boule centrée en $a$ intersecte $A$. De la même façon, le fait de ne pas appartenir à $\Int(A)$ signifie que toute boule centrée en $a$ intersecte $\complement A$.
\end{proof}

La description de la frontière donnée par la proposition \ref{PropDescFrpbsmI} est celle qu'en pratique nous utilisons le plus souvent. Dans certains textes, elle est prise comme définition de la frontière.

\begin{lemma}
	La frontière de $A$ peut également s'exprimer des façons suivantes :
	\begin{equation}
		\partial A= \bar A\cap\complement\Int(A)=\bar A\cap\overline{ \complement A },
	\end{equation}
\end{lemma}

\begin{proof}
	En partant de $\partial A=\bar A\setminus \Int(A)$, la première égalité est une application de la propriété \ref{ItemLemPropComplementiii} du lemme \ref{LemPropsComplement}. La seconde égalité est alors la proposition \ref{PropComleIntBar}.
\end{proof}

\begin{example}
	Dans $\eR$, la frontière d'un intervalle est la paire constituée des points extrêmes. En effet
	\begin{equation}
		\partial\mathopen[ a , b [=\overline{ \mathopen[ a , b [ }\setminus\Int\big( \mathopen[ a , b [ \big)=\mathopen[ a , b \mathclose]\setminus\mathopen] a , b \mathclose[=\{ a,b \}.
	\end{equation}

	Toujours dans $\eR$ nous avons
	\begin{equation}
		\partial\eR=\bar\eR\setminus\Int(\eR)=\eR\setminus\eR=\emptyset,
	\end{equation}
	et
	\begin{equation}
		\partial\eQ=\bar\eQ\setminus\Int(\eQ)=\eR\setminus\emptyset=\eR.
	\end{equation}
\end{example}

\begin{example}
	Dans $\eR^n$, nous avons
	\begin{equation}
		\partial B(a,r)=\partial\bar B(a,r)=S(a,r).
	\end{equation}
	La première égalité provient du fait que pour tout ensemble, nous ayons $\partial A=\partial\bar A$. Nous cherchons donc $\partial\bar B(a,r)$. Évidement, la fermeture de cet ensemble est lui-même (parce qu'il est déjà fermé), nous avons donc $\partial\bar B(a,r)=\bar B(a,r)\setminus\Int\big( \bar B(a,r) \big)$. Nous avons déjà vu dans l'exemple \ref{ExempleIntBoules} que $\Int\big( \bar B(a,r) \big)=B(a,r)$.

	Nous avons donc
	\begin{equation}
		\partial\bar B(a,r)=\bar B(a,r)\setminus B(a,r)=S(a,r).
	\end{equation}

\end{example}

%---------------------------------------------------------------------------------------------------------------------------
\subsection{Point isolé, point d'accumulation}
%---------------------------------------------------------------------------------------------------------------------------

\begin{definition}
	Soit $D$, une partie de $V$.
	\begin{enumerate}
		\item
			Un point $a\in D$ est dit \defe{isolé}{isolé!point dans un espace vectoriel normé} dans $D$ relativement à $V$ si il existe un $\varepsilon>0$ tel que
			\begin{equation}
				B(a,\varepsilon)\cap D=\{ a \}.
			\end{equation}
		\item
			Un point $a\in V$ est un \defe{point d'accumulation}{accumulation!dans espace vectoriel normé} de $D$ si pour tout $\varepsilon>0$,
			\begin{equation}
				\Big( B(a,\varepsilon)\setminus\{ a \}\Big)\cap D\neq \emptyset.
			\end{equation}
	\end{enumerate}
\end{definition}

\newcommand{\CaptionFigAccumulationIsole}{L'ensemble décrit par l'équation \eqref{Eq2807BouleIso}. Le point $P$ est un point isolé de $D$, tandis que  les points $S$ et $Q$ sont des points d'accumulation.}
\input{Fig_AccumulationIsole.pstricks}

\begin{example}
	Considérons la partie suivante de $\eR^2$ :
	\begin{equation}	\label{Eq2807BouleIso}
		D=\{ (x,y)\tq x^2+y^2<1\}\cup\{ (1,1) \}.
	\end{equation}
	Comme on peut le voir sur la figure \ref{LabelFigAccumulationIsole}, le point $P=(1,1)$ est un point isolé de $D$ parce qu'on peut tracer une boule autour de $P$ sans inclure d'autres points de $D$ que $P$ lui-même. Le point $Q=(-1,0)$ est un point d'accumulation de $D$ parce que toute boule autour de $Q$ contient des points de $D$.

    Le point $S$, étant un point intérieur, est un point d'accumulation : toute boule autour de $S$ intersecte $D$.
    
    Notez cependant que le point $Q$ lui-même n'est pas dans $D$ parce que l'inégalité qui définit $D$ est stricte.
\end{example}

\begin{remark}
    À propos de la position des points d'accumulation et des points isolés.
    \begin{enumerate}
        \item
            Les points intérieurs sont tous des points d'accumulation.
        \item
            Les points isolés ne sont jamais intérieurs.
        \item
            Certains points d'accumulation ne font pas partie de l'ensemble. Par exemple le point $1$ est un point d'accumulation de $E=\mathopen] 0 , 1 \mathclose[$.
        \item
            Les points de la frontière sont soit d'accumulation soit isolés.
    \end{enumerate}
\end{remark}


\begin{example}
	Tous les points de $\eR$ sont des points d'accumulation de $\eQ$ parce que dans toute boule autour d'un réel, on peut trouver un nombre rationnel.
\end{example}

\begin{remark}
	L'ensemble des points d'accumulation d'un ensemble n'est pas exactement son adhérence. En effet, un point isolé dans $A$ est dans l'adhérence de $A$, mais n'est pas un point d'accumulation de $A$.
\end{remark}

%+++++++++++++++++++++++++++++++++++++++++++++++++++++++++++++++++++++++++++++++++++++++++++++++++++++++++++++++++++++++++++
\section{Convergence de suites}\label{Sect_suites}
%+++++++++++++++++++++++++++++++++++++++++++++++++++++++++++++++++++++++++++++++++++++++++++++++++++++++++++++++++++++++++++

Nous disons qu'une suite réelle $(x_n)$ converge\footnote{Voir la définition \ref{DefLimiteSuiteNum} pour plus de détail.} vers $\ell$ lorsque pour tout $\varepsilon$, il existe un $M$ tel que
\begin{equation}
	n>N\Rightarrow | x_n-\ell |\leq\varepsilon.
\end{equation}
Le concept fondamental de cette définition est la notion de valeur absolue qui permet de donner la «distance» entre deux réels. Dans un espace vectoriel normé quelconque, cette notion est généralisée par la distance associée à la norme (définition \ref{DefEVNetDistance}). Nous pouvons donc facilement définir le concept de convergence d'une suite dans un espace vectoriel normé.

\begin{definition}		\label{DefCvSuiteEGVN}
	Soit une suite $(x_n)$ dans un espace vectoriel normé $V$. Nous disons qu'elle est \defe{convergente}{convergence!dans un espace vectoriel normé} si il existe un élément $\ell\in V$ tel que
	\begin{equation}
		\forall \varepsilon>0,\,\exists N\in\eN\tq n\geq N\Rightarrow \| x_n-l \|<\varepsilon.
	\end{equation}
	Dans ce cas, $\ell$ est appelé la \defe{limite}{limite!suite} de la suite $(x_n)$.
\end{definition}




\begin{lemma}		\label{LemLimAbarA}
	Soit $(x_n)$ une suite convergente contenue dans un ensemble $A\subset V$. Alors la limite $x_n$ appartient à $\bar A$.
\end{lemma}

\begin{proof}
	Supposons que nous ayons une partie $A$ de $V$, et une suite $(x_n)$ dont la limite $\ell$ se trouve hors de $\bar A$. Dans ce cas, il existe un $r>0$ tel que\footnote{Une autre manière de dire la même chose : si $\ell\notin\bar A$, alors $d(\ell,A)>0$.} $B(\ell,r)\cap A=\emptyset$. Si tous les éléments $x_n$ de la suite sont dans $A$, il n'y en a donc aucun tel que $d(x_n,\ell)=\| x_n-\ell \|<r$. Cela contredit la notion de convergence $x_n\to \ell$.
\end{proof}

Nous avons déjà mentionné dans l'exemple \ref{ParlerEncoredeF} que zéro était un point adhérent à l'ensemble $F=\{ (-1)^n/n\tq n\in\eN_0 \}$. Nous savons maintenant que $0$ étant la limite de la suite, il est automatiquement adhérent à l'ensemble des éléments de la suite.

\begin{corollary}		\label{CorAdhEstLim}
	Soit $a$ un point de l'adhérence d'une partie $A$ de $V$. Alors il existe une suite d'éléments dans $A$ qui converge vers $a$.
\end{corollary}

\begin{proof}
	Si $a\in A$, alors nous pouvons prendre la suite constante $x_n=a$. Si $a$ n'est pas dans $A$, alors $a$ est dans $\partial A$, et pour tout $n$, il existe un point de $A$ dans la boule $B(a,\frac{1}{ n })$. Si nous nommons $x_n$ ce point, la suite ainsi construite est une suite contenue dans $A$ et qui converge vers $a$ (ce dernier point est laissé à la sagacité du lecteur ou de la lectrice).
\end{proof}

En termes savants, ce corollaire signifie que la fermeture $\bar A$ est composé de $A$ plus de toutes les limites de toutes les suites contenues dans $A$.


\begin{proposition}		\label{PropSuiteCompactSScv}
	Si $K$ est une partie compacte de $V$ et si $(x_n)$ est une suite contenue dans $K$, alors $(x_n)$ possède une sous-suite convergente.
\end{proposition}

Nous ne donnons pas de preuves de cette proposition, étant donné qu'une preuve sera donnée dans le cas particulier de $V=\eR^m$ pour le théorème \ref{ThoBolzanoWeierstrassRn}. Cette preuve fonctionne ici mot à mot en remplaçant $\eR^m$ par $V$ en en réfléchissant un peu sur le concept de «composante».

%+++++++++++++++++++++++++++++++++++++++++++++++++++++++++++++++++++++++++++++++++++++++++++++++++++++++++++++++++++++++++++
\section{Fonctions}		\label{Sect_fonctions}
%+++++++++++++++++++++++++++++++++++++++++++++++++++++++++++++++++++++++++++++++++++++++++++++++++++++++++++++++++++++++++++

Soient $(V,\| . \|_V)$ et $(W,\| . \|_W)$ deux espaces vectoriels normés, et une fonction $f$ de $V$ dans $W$. Il est maintenant facile de définir les notions de limites et de continuité pour de telles fonctions en copiant les définitions données pour les fonctions de $\eR$ dans $\eR$ en changeant simplement les valeurs absolues par les normes sur $V$ et $W$.

En nous inspirant de la définition \ref{DefLimiteFonction}, nous écrivons
\begin{definition}		\label{LimiteDansEVN}
	Soit $f\colon V\to W$ une fonction de domaine \( \Domaine(f)\subset V\) et soit $a$ un point d'accumulation de $\Domaine(f)$. Nous disons que $f$ \defe{admet une limite}{limite!espace vectoriel normé} en $a$ si il existe un élément $\ell\in W$ tel que pour tout $\varepsilon>0$, il existe un $\delta>0$ tel que pour tout $x\in \Domaine(f)$,
    \begin{equation}        \label{EqDefLimzxmasubV}
		0<\| x-a \|_V<\delta\,\Rightarrow\,\| f(x)-\ell \|_W<\varepsilon.
	\end{equation}
	Dans ce cas, nous écrivons $\lim_{x\to a} f(x)=\ell$ et nous disons que $\ell$ est la \defe{limite}{limite} de $f$ lorsque $x$ tend vers $a$.
\end{definition}

\begin{remark}
    Le fait que nous limitions la formule \eqref{EqDefLimzxmasubV} aux \( x\) dans le domaine de \( f\) n'est pas anodin. Considérons la fonction \( f(x)=\sqrt{x^2-4}\), de domaine \( | x |\geq 2\). Nous avons
    \begin{equation}
        \lim_{x\to 2} \sqrt{x^2-4}=0.
    \end{equation}
    Nous ne pouvons pas dire que cette limite n'existe pas en justifiant que la limite à gauche n'existe pas. Les points \( x<2\) sont hors du domaine de \( f\) et ne comptent dons pas dans l'appréciation de l'existence de la limite.

    Vous verrez plus tard que ceci provient de la \wikipedia{fr}{Topologie_induite}{topologie induite} de \( \eR\) sur l'ensemble \( \mathopen[ 2 , \infty [\).
\end{remark}

\begin{definition}\label{DefContDansEVN}
	Une fonction $f\colon D\subset V\to W$ entre deux espaces vectoriels normés $V$ et $W$ est dite \defe{continue}{continue!fonction sur espace vectoriel normé} au point $a\in\bar D$ si $f(x)$ admet une limite pour $x$ tendant vers $a$ et si $\lim_{x\to a} f(x)=f(a)$.
\end{definition}

Une caractérisation très importante des fonctions continues est que l'image inverse d'un ouvert par une fonction continue est ouverte.

\begin{theorem}		\label{ThoContiueImageInvOUvert}
	Soient $V$ et $W$ deux espaces vectoriels normés. Une fonction $f$ de $V$ vers $W$ est continue si et seulement si pour tout ouvert $\mO$ dans $W$, l'ensemble $f^{-1}(\mO)$ est ouvert dans $V$.
\end{theorem}

\begin{proof}
	Supposons d'abord que $f$ est continue. Soit $\mO$ un ouvert de $W$, et prouvons que $f^{-1}(\mO)$ est ouvert. Pour cela, nous allons prouver qu'autour de chaque point $x$ de $f^{-1}(\mO)$, il existe une boule contenue dans $f^{-1}(\mO)$. Nous notons $y=f(x)\in\mO$. Étant donné que $\mO$ est ouvert dans $W$, il existe un rayon $r$ tel que
	\begin{equation}
		B_W\big( f(x),r \big)\subset\mO.
	\end{equation}
	Nous avons ajouté l'indice $W$ pour nous rappeler que c'est une boule dans $W$. Mais la continuité de $f$ implique qu'il existe un rayon $\delta$ tel que $\| x-a \|_V<\delta$ implique $\big\| f(x)-f(a) \big\|_W<r$. Ayant choisit un tel $\delta$, nous savons que si $a\in B_V(x,\delta)$, alors $f(a)\in B_W\big( f(x),r \big)\subset \mO$. Dans ce cas, $a\in f^{-1}(\mO)$. Nous avons donc montré que $B_V(x,\delta)\subset f^{-1}(\mO)$, ce qui prouve que $f^{-1}(\mO)$ est ouvert.

	Supposons maintenant que pour tout ouvert $\mO$ de $W$, l'ensemble $f^{-1}(\mO)$ est ouvert. Nous allons montrer qu'alors $f$ est continue. Soit $x\in V$ et $\varepsilon>0$. Nous devons trouver $\delta$ tel que $0<\| x-a \|_V<\delta$ implique $\| f(a)-f(x) \|_W<\varepsilon$.

	Considérons la boule ouverte $\mO=B_W\big( f(x),\varepsilon \big)$, et son image inverse $f^{-1}(\mO)$ qui est également ouverte par hypothèse. Étant donné que $f(x)\in\mO$, nous avons évidemment $x\in f^{-1}(\mO)$ et donc il existe une boule centrée en $x$ et contenue dans $f^{-1}(\mO)$. Soit $\delta$ le rayon de cette boule :
	\begin{equation}
		B_V\big( x,\delta \big)\subset f^{-1}(\mO).
	\end{equation}
	Par définition de l'image inverse, nous avons aussi $g\big( B_V(x,\delta) \big)\subset\mO$. En récapitulant,
	\begin{equation}
		\| x-a \|_V<\delta\Rightarrow a\in B_V(x,\delta)\Rightarrow f(a)\in\mO=B_W\big( f(x),\varepsilon \big)\Rightarrow\| f(a)-f(x) \|_W<\varepsilon.
	\end{equation}
	Ceci conclu la preuve.
\end{proof}

\begin{remark}
	Cette propriété des fonctions continues est tellement importante qu'elle est souvent prise comme définition de la continuité.
\end{remark}

Un résultat important dans la théorie des fonctions sur les espaces vectoriels normés est qu'une fonction continue sur un compact est bornée et atteint ses bornes. Ce résultat sera (dans d'autres cours) énormément utilisé pour trouver des maxima et minima de fonctions. Le théorème exact est le suivant.

\begin{theorem}		\label{WeierstrassEVN}
	Soit $K\subset V$ une partie compacte (fermée et bornée) d'un espace vectoriel normé $v$. Si $f\colon K\subset V\to \eR$ est une fonction continue, alors $f$ est bornée, et atteint ses bornes. 
	
	C'est à dire qu'il existe $x_0\in K$ tel que $f(x_0)=\inf\{ f(x)\tq x\in K \}$ ainsi que $x_1$ tel que $f(x_1)=\sup\{ f(x)\tq x\in K \}$.
\end{theorem}

Ce résultat sera prouvé dans le théorème \ref{ThoWeirstrassRn} dans le cas particulier de $V=\eR^n$. La preuve qui sera donné à ce moment peut être recopiée (presque) mot à mot en remplaçant $\eR^m$ par $V$. Nous n'allons donc pas donner de démonstration de ce théorème ici. Nous allons par contre donner la preuve d'un résultat un peu plus général.

\begin{proposition}		\label{PropContinueCompactBorne}
	Soient $V$ et $W$ deux espaces vectoriels normés. Soit $K$, une partie compacte de $V$, et $f\colon K\to W$, une fonction continue. Alors l'image $f(K)$ est compacte dans $W$.
\end{proposition}

\begin{proof}
	Nous allons prouver que $f(K)$ est fermée et bornée.
	\begin{description}
		\item[$f(K)$ est fermé] Nous allons prouver que si $(y_n)$ est une suite convergente contenue dans $f(K)$, alors la limite est également contenue dans $f(K)$. Dans ce cas, nous aurons que l'adhérence de $f(K)$ est contenue dans $f(K)$ et donc que $f(K)$ est fermé. Pour chaque $n\in\eN$, le vecteur $y_n$ appartient à $f(K)$ et donc il existe un $x_n\in K$ tel que $f(x_n)=y_n$. La suite $(x_n)$ ainsi construite est une suite dans le fermé $K$ et possède donc une sous-suite convergente (proposition \ref{PropSuiteCompactSScv}). Notons $(x'_n)$ cette sous-suite convergente, et $a$ sa limite : $\lim(x'_n)=a\in K$. Le fait que la limite soit dans $K$ provient du fait que $K$ est fermé.

			Nous pouvons considérer la suite $f(x'_n)$ dans $W$. Cela est une sous-suite de la suite $(y_n)$, et nous avons $\lim f(x'_n)=a$ parce que $f$ est continue. Par conséquent nous avons
			\begin{equation}
				f(a)=\lim f(x'_n)=\lim y_n.
			\end{equation}
			Cela prouve que la limite de $(y_n)$ est dans $f(K)$ et par conséquent que $f(K)$ est fermé.

		\item[$f(K)$ est borné]
			Si $f(K)$ n'est pas borné, nous pouvons trouver une suite $(x_n)$ dans $K$ telle que
			\begin{equation}		\label{EqfxnWgeqn}
				\| f(x_n) \|_W>n
			\end{equation}
			Mais par ailleurs, l'ensemble $K$ étant compact (et donc fermé), nous avons une sous-suite $(x'_n)$ qui converge dans $K$. Disons $\lim(x'_n)=a\in K$. 
			
			Par la continuité de $f$ nous avons alors $f(a)=\lim f(x'_n)$, et donc
			\begin{equation}
				| f(a) |=\lim | f(x'_n) |.
			\end{equation}
			La suite $f(x'_n)$ est alors une suite bornée, ce qui n'est pas possible au vu de la condition \eqref{EqfxnWgeqn} imposée à la suite de départ $(x_n)$.
	\end{description}
\end{proof}

\begin{corollary}	\label{CorFnContinueCompactBorne}
	Une fonction $f\colon K\to \eR$ où $K$ est une partie compacte d'un espace vectoriel normé est toujours bornée.
\end{corollary}

\begin{proof}
	En effet, la proposition \ref{PropContinueCompactBorne} montre que $f(K)$ est compact et donc borné.
\end{proof}


%+++++++++++++++++++++++++++++++++++++++++++++++++++++++++++++++++++++++++++++++++++++++++++++++++++++++++++++++++++++++++++
\section{Produit d'espaces vectoriels normés}\label{sec_prod}
%+++++++++++++++++++++++++++++++++++++++++++++++++++++++++++++++++++++++++++++++++++++++++++++++++++++++++++++++++++++++++++

%---------------------------------------------------------------------------------------------------------------------------
\subsection{Norme}
%---------------------------------------------------------------------------------------------------------------------------

Soient $V$ et $W$ deux espaces vectoriels normés. On appelle \defe{espace produit}{produit d'espaces vectoriels normés} de $V$ et $W$ le produit cartésien $V\times W$ 
\begin{equation}
V\times W=\{(v,w)\,|\, v\in V,\, w\in W\},
\end{equation}
muni de la norme $\|\cdot \|_{V\times W}$
\begin{equation}	\label{EqNormeVxWmax}
	\|(v,w) \|_{V\times W}=\max\{\|v\|_{V},\|w\|_W\}.
\end{equation}
Il est presque immédiat de vérifier que le produit cartésien $V\times W$ est un espace vectoriel pour les opération de somme et multiplication par les scalaires définies composante par composante. C'est à dire,  si $(v_1,w_1)$, $(v_2,w_2)$ sont dans $V\times W$ et $a$, $b$ sont des scalaires, alors  
\begin{equation}
 a (v_1,w_1)+ b(v_2,w_2)=(av_1,aw_1)+ (bv_2,bw_2)=(av_1+bv_2,aw_1+bw_2).
\end{equation}

\begin{lemma}
	L'opération $\|\cdot \|_{V\times W}\colon V\times W\to \eR$ est une norme.
\end{lemma}

\begin{proof}
	On doit vérifier les trois conditions de la définition \ref{DefNorme}.
	\begin{itemize}
		\item Soit $(v,w)$ dans $V\times W$ tel que $\|(v,w)\|_{V\times W}=\max\{\|v\|_{V},\|w\|_W\}=0$. Alors $\|v\|_V=0$ et $\|w\|_W=0$, donc $v=0_V$ et $w=0_W$. Cela implique $(v,w)=(0_v,0_w)=0_{V\times W}$. 
		\item Pour tout $a$ dans $\eR$ et $(v,w)$ dans $V\times W$,  la norme $\|a (v,w)\|_{V\times W}$ est donnée par  $\max\{\|av\|_{V},\|aw\|_W\}$. On peut factoriser $\|av\|_{V}=|a|\|v\|_{V}$ et $\|aw\|_W=|a|\|w\|_W$ et donc $\|a (v,w)\|_{V\times W}=|a|\max\{\|v\|_{V},\|w\|_W\}=|a|\|(v,w)\|_{V\times W}$.
		\item Soient $(v_1,w_1)$ et $(v_2,w_2)$ dans $V\times W$. 
		\begin{equation}
			\begin{aligned}
				\|(v_1,w_1)+(v_2,w_2)\|_{V\times W}&=\max\{\|v_1+v_2\|_{V},\|w_1+w_2\|_W\}\leq\\
				&\leq \max\{\|v_1\|_V+\|v_2\|_{V},\|w_1\|_W+\|w_2\|_W\}\leq\\
				&\leq\max\{\|v_1\|_V,\|w_1\|_W\}+ \max\{\|v_2\|_{V},\|w_2\|_W\}=\\
				&=\|(v_1,w_1)\|_{V\times W}+\|(v_2,w_2)\|_{V\times W}.
			\end{aligned}
		\end{equation}
	\end{itemize} 
\end{proof}
On remarque tout de suite que la norme $\|\cdot\|_\infty$ sur $\eR^2$ est la norme de l'espace produit $\eR\times\eR$. En outre cette définition nous permet de trouver plusieurs nouvelles normes dans les espaces $\eR^p$. Par exemple, si nous écrivons $\eR^4$ comme $\eR^2\times \eR^2$ on peut munir $\eR^4$ de la norme produit
\[
\|(x_1,x_2,x_3,x_4)\|_{\infty, 2}=\max\{\|(x_1,x_2)\|_\infty, \|(x_3,x_4)\|_2\}. 
\]    
Les applications de projection de l'espace produit $V\times W$ vers les espaces <<facteurs>>, $V$ $W$ sont notées $\pr_V$ et $\pr_W$ et sont définies par
\begin{equation}
	\begin{aligned}
		\pr_V\colon V\times W&\to V \\
		(v,w)&\mapsto v 
	\end{aligned}
\end{equation}
et
\begin{equation}
	\begin{aligned}
		\pr_W\colon V\times W &\to W \\
		(v,w)&\mapsto w. 
	\end{aligned}
\end{equation}
Les inégalités suivantes sont évidentes
\begin{equation}
	\begin{aligned}[]
		\|\pr_V(v,w)\|_V&\leq \|(v,w)\|_{V\times W} \\
		\|\pr_W(v,w)\|_W&\leq \|(v,w)\|_{V\times W}.
	\end{aligned}
\end{equation}
La topologie de l'espace produit est induite par les topologies des espaces <<facteurs>>. La construction est faite en deux passages : d'abord nous disons que une partie $A\times B$ de $V\times W$ est ouverte si $A$ et $B$ sont des parties ouvertes de $V$ et de $W$ respectivement.  Ensuite nous définissons que une partie quelconque de $V\times W$ est ouverte si elle est une intersection finie ou une réunion de parties ouvertes de $V\times W$ de la forme $A\times B$. 

Ce choix de topologie donne deux propriétés utiles de l'espace produit 
\begin{enumerate}
	\item
		Les projections sont des \defe{applications ouvertes}{application!ouverte}. Cela veut dire que l'image par $\pr_V$ (respectivement $\pr_W$) de toute partie ouverte de $V\times W$ est une partie ouverte de $V$ (respectivement $W$). 
	\item 
		Pour toute partir $A$ de $V$ et $B$ de $W$, nous avons $\Int (A\times B)=\Int A\times \Int B$.\label{PgovlABeqbAbB}
\end{enumerate}
Une propriété moins facile a prouver est que pour toute partie $A$ de $V$ et $B$ de $W$ nous avons  $\overline{A\times B}=\bar{A}\times \bar{B}$. Voir le lemme \ref{LemCvVxWcvVW}.
% position 26329
%et l'exercice \ref{exoGeomAnal-0009}.
  
Ce que nous avons dit jusqu'ici est valable pour tout produit d'un nombre fini d'espaces vectoriels normés. En particulier, pour tout $m>0$  l'espace  $\eR^m$ peut être considéré comme le produit de $m$ copies de $\eR$. 

\begin{example}
	Si $V$ et $W$ sont deux espaces vectoriels, nous pouvons considérer le produit $E=V\times W$. Les projections $\pr_V$ et $\pr_W$\nomenclature{$\pr_V$}{projection de $V\times W$ sur $V$}, définies dans la section \ref{sec_prod}, sont des applications linéaires. 

	En effet, la projection $\pr_V\colon V\times W\to V$ est donnée par $\pr_V(v,w)=v$. Alors,
	\begin{equation}
		\begin{aligned}[]
			\pr_V\big( (v,w)+(v',w') \big)&=\pr_V\big( (v+v'),(w+w') \big)\\
			&=v+v'\\
			&=\pr_V(v,w)+\pr_V(v',w'),
		\end{aligned}
	\end{equation}
	et
	\begin{equation}
		\pr_V\big( \lambda(v,w) \big)=\pr_V\big( (\lambda v,\lambda w) \big)=\lambda v=\lambda\pr_V(v,w).
	\end{equation}
	Nous laissons en exercice le soin d'adapter ces calculs pour montrer que $\pr_W$ est également une projection.
\end{example}


%---------------------------------------------------------------------------------------------------------------------------
\subsection{Suites}
%---------------------------------------------------------------------------------------------------------------------------

Nous allons maintenant parler de suites dans $V\times W$. Nous noterons $(v_n,w_n)$ la suite dans $V\times W$ dont l'élément numéro $n$ est le couple $(v_n,w_n)$ avec $v_n\in V$ et $w_n\in W$. La notions de convergence de suite découle de la définition de la norme via la définition usuelle \ref{DefCvSuiteEGVN}. Il se fait que dans le cas des produits d'espaces, la convergence d'une suite est équivalente à la convergence des composantes. Plus précisément, nous avons le lemme suivant.
\begin{lemma}		\label{LemCvVxWcvVW}
	La suite $(v_n,w_n)$ converge vers $(v,w)$ dans $V\times W$ si et seulement les suites $(v_n)$ et $(w_n)$ convergent séparément vers $v$ et $w$ respectivement dans $V$ et $W$. 
\end{lemma}

\begin{proof}
	Pour le sens direct, nous devons étudier le comportement de la norme de $(v_n,w_n)-(v,w)$ lorsque $n$ devient grand. En vertu de la définition de la norme dans $V\times W$ nous avons
	\begin{equation}
		\Big\| (v_n,w_n)-(v,w) \Big\|_{V\times W}=\max\big\{ \| v_n-v \|_V,\| w_n-w \|_W \big\}.
	\end{equation}
	Soit $\varepsilon>0$. Par définition de la convergence de la suite $(v_n,w_n)$, il existe un $N\in\eN$ tel que $n>N$ implique
	\begin{equation}
		\max\big\{ \| v_n-v \|_V,\| w_n-w \|_W \big\}<\varepsilon,
	\end{equation}
	et donc en particulier les deux inéquations
	\begin{subequations}
		\begin{align}
			\| v_n-v \|&<\varepsilon\\
			\| w_n-w \|&<\varepsilon.
		\end{align}
	\end{subequations}
	De la première, il ressort que $(v_n)\to v$, et de la seconde que $(w_n)\to w$.

	Pour le sens inverse, nous avons pour tout $\varepsilon$ un $N_1$ tel que $\| v_n-v \|_V\leq\varepsilon$ pour tout $n>N_1$ et un $N_2$ tel que $\| w_n-w \|_W\leq\varepsilon$ pour tout $n>N_2$. Si nous posons $N=\max\{ N_1,N_2 \}$ nous avons les deux inégalités simultanément, et donc
	\begin{equation}
		\max\big\{ \| v_n-v \|_V,\| w_n-w \|_W \big\}<\varepsilon,
	\end{equation}
	ce qui signifie que la suite $(v_n,w_n)$ converge vers $(v,w)$ dans $V\times W$.
\end{proof}

\begin{remark}		\label{RemTopoProdPasRm}
	Il faut remarquer que la norme \eqref{EqNormeVxWmax} est une norme \emph{par défaut}. C'est la norme qu'on met quand on ne sait pas quoi mettre. Or il y a au moins un cas d'espace produit dans lequel on sait très bien quelle norme prendre : les espaces $\eR^m$. La norme qu'on met sur $\eR^2$ est
	\begin{equation}
		\| (x,y) \|=\sqrt{x^2+y^2},
	\end{equation}
	et non la norme «par défaut» de $\eR^2=\eR\times\eR$ qui serait
	\begin{equation}
		\| (x,y) \|=\max\{ | x |,| y | \}.
	\end{equation}
	Les théorèmes que nous avons donc démontré à propos de $V\times W$ ne sont donc pas immédiatement applicables au cas de $\eR^2$.

	Cette remarque est valables pour tous les espaces $\eR^m$. À moins de mention contraire explicite, nous ne considérons jamais la norme par défaut \eqref{EqNormeVxWmax} sur un espace $\eR^m$.
\end{remark}

Étant donné la remarque \ref{RemTopoProdPasRm}, nous ne savons pas comment calculer par exemple la fermeture du produit d'intervalle $\mathopen] 0,1 ,  \mathclose[\times\mathopen[ 4 , 5 [$. Il se fait que, dans $\eR^m$, les fermetures de produits sont quand même les produits de fermetures.

\begin{proposition}		\label{PropovlAxBbarAbraB}
	Soit $A\subset\eR^m$ et $B\subset\eR^m$. Alors dans $\eR^{m+n}$ nous avons $\overline{ A\times B }=\bar A\times \bar B$.
\end{proposition}

La démonstration risque d'être longue; nous ne la faisons pas ici.



%+++++++++++++++++++++++++++++++++++++++++++++++++++++++++++++++++++++++++++++++++++++++++++++++++++++++++++++++++++++++++++
\section{Équivalence des normes}\label{normes_equiv}
%+++++++++++++++++++++++++++++++++++++++++++++++++++++++++++++++++++++++++++++++++++++++++++++++++++++++++++++++++++++++++++

Au premier coup d'œil, les notions dont nous parlons dans ce chapitre ont l'air très générales. Nous prenons en effet n'importe quel espace vectoriel $V$ de dimension finie, et nous le munissons de n'importe quelle norme (rien que dans $\eR^m$ nous en avons définis une infinité par l'équation \eqref{EqDeformeLp}). À partir de ces données, nous définissons les boules, la topologie, l'adhérence, etc.

%---------------------------------------------------------------------------------------------------------------------------
\subsection{En dimension finie}
%---------------------------------------------------------------------------------------------------------------------------

Dans $\eR^n$, les normes $\| . \|_{L^1}$, $\| . \|_{L^2}$ et $\| . \|_{\infty}$ ne sont pas égales. Cependant elles ne sont pas complètement indépendante au sens où l'on sent bien que si un vecteur sera grand pour une norme, il sera également grand pour les autres normes; les normes «vont dans le même sens». Cette notion est précisée par le concept de norme équivalente (voir appendice \ref{appEquivalence}).

\begin{definition}		\label{DefEquivNorm}
	Deux normes $N_1$ et $N_2$ sur $\eR^m$ sont \defe{\wikipedia{fr}{Norme_équivalente}{équivalentes}}{equivalence@équivalence!norme} si il existe deux nombres réels strictement positifs $k_1$ et $k_2$ tels que
	\begin{equation}
		k_1N_1(x)\leq N_2(x)\leq k_2 N_1(x),
	\end{equation}
	pour tout $x$ dans $\eR^m$. Dans ce cas nous écrivons que $N_1\sim N_2$.
\end{definition}
Il est possible de démontrer que cette notion est une relation d'équivalence sur l'ensemble des normes existantes sur $\eR^m$.

\begin{proposition}
    Nous avons les équivalences de normes $\| . \|_{L^1}\sim\| . \|_{L^2}$, $\| . \|_{L^1}\sim\| . \|_{\infty}$ et $\| . \|_{L^2}\sim\| . \|_{\infty}$. Plus précisément nous avons les inégalités
    \begin{subequations}
        \begin{align}
            \| x \|_2&\leq \| x \|_1\leq\sqrt{n}\| x \|_2,  \label{EqEquivdui}\\
            \| x \|_{\infty}&\leq \| x \|_1\leq n \| x \|_{\infty},\\
            \| x \|_{\infty}&\leq \| x \|_2\leq \sqrt{n}\| x \|_{\infty}.\label{EqEquivduiii}
        \end{align}
    \end{subequations}
\end{proposition}

\begin{proof}
    En mettant au carré la première inégalité \eqref{EqEquivdui}, nous voyons que nous devons vérifier l'inégalité
    \begin{equation}
        | x_1 |^2+\ldots+| x_n |^2\leq\big( | x_1 |+\ldots+| x_n | \big)^2
    \end{equation}
    qui est vraie parce que le membre de droite est égal au carré de chaque terme plus les double produits. La seconde inégalité \eqref{EqEquivdui} provient de l'inégalité de Cauchy-Schwarz (théorème \ref{ThoAYfEHG}) sur les vecteurs
    \begin{equation}
        \begin{aligned}[]
            v&=\begin{pmatrix}
                1/n    \\ 
                \vdots    \\ 
                1/n    
            \end{pmatrix},
            &w&=\begin{pmatrix}
                | x_1 |    \\ 
                \vdots    \\ 
                | x_n |    
            \end{pmatrix}.
        \end{aligned}
    \end{equation}
    Nous trouvons 
    \begin{equation}
        \frac{1}{ n }\sum_i| x_i |\leq\sqrt{b\cdot\frac{1}{ n }}\sqrt{\sum_i| x_i |^2},
    \end{equation}
    et par conséquent
    \begin{equation}
        \sum_i| x_i |\leq\sqrt{n}\| x \|_2.
    \end{equation}
    
    La première inégalité \eqref{EqEquivduiii} se démontre en remarquant que si \( a\) et \( b\) sont positifs, \( a\leq\sqrt{a^2+b}\). En appliquant cela à \( a=\max_i| x_i |\), nous avons
    \begin{equation}
        \max_i| x_i |\leq\sqrt{ | x_1 |^2+\ldots+| x_n |^2  }
    \end{equation}
    parce que \( \max_i| x_i |\) est évidemment un des termes de la somme. Pour la seconde inégalité \eqref{EqEquivduiii}, nous avons
    \begin{equation}
        \sqrt{\sum_k| x_k |^2}\leq\left( \sum_k\max_i| x_i |^2 \right)^{1/2}=\sqrt{n}\| x \|_{\infty}.
    \end{equation}
    Pour obtenir cette inégalité, nous avons remplacé tous les termes \( | x_k |\) par le maximum.
\end{proof}

En réalité, toutes les normes \( \| . \|_{L^p}\) et \( \| . \|_{\infty}\) sont équivalentes et, plus généralement, nous avons le résultat suivant, très étonnant à première vue, et en réalité assez difficile à prouver :
\begin{theorem}[\cite{TrenchRealAnalisys}]		\label{ThoNormesEquiv}
	Sur un espace vectoriel de dimension finie, toutes les normes (pas seulement les normes $L^p$ que nous avons définies sur $\eR^m$) sont équivalentes.
\end{theorem}
% TODO : la preuve est à la page 583 de Trench.

%---------------------------------------------------------------------------------------------------------------------------
\subsection{Contre-exemple en dimension infinie}
%---------------------------------------------------------------------------------------------------------------------------
\label{SubSecPOlynomesCE}

Lorsque nous considérons des espaces vectoriels de dimension infinie, les choses ne sons plus aussi simples. Nous voyons ici sur l'exemple de l'espace des polynômes que le théorème \ref{ThoNormesEquiv} n'est plus valable si on enlève l'hypothèse de dimension finie.

On considère l'ensemble des fonctions polynômiales à coefficients réels sur  l'intervalle $[0,1]$.
\begin{equation}
\mathcal{P}_\eR([0,1])=\{p:[0,1]\to \eR\,|\, p : x\mapsto a_0+a_1 x +a_2 x^2 + \ldots, \, a_i\in\eR,\,\forall i\in \eN\}.
\end{equation}
Cet ensemble, muni des opérations usuelles de somme entre polynômes et multiplications par les scalaires, est un espace vectoriel.  

Sur $\mathcal{P}(\eR)$ on définit les normes suivantes 
\begin{equation}
\begin{aligned}
&\|p\|_\infty=\sup_{x\in[0,1]}\{p(x)\},\\
&\|p\|_1 =\int_0^1|p(x)|\, dx,\\
&\|p\|_2 =\left(\int_0^1|p(x)|^2\, dx\right)^{1/2}.\\
\end{aligned}
\end{equation}
Les inégalités suivantes sont  immédiates
\begin{equation}
\begin{aligned}
&\|p\|_1 =\int_0^1|p(x)|\, dx\leq \|p\|_\infty,\\
&\|p\|_2 =\left(\int_0^1|p(x)|^2\, dx\right)^{1/2}\leq \|p\|_\infty,\\
\end{aligned}
\end{equation}
mais la norme $\|\cdot\|_\infty$ n'est  équivalente ni à $\|\cdot\|_1$, ni à $\|\cdot\|_2$. Soit $p_k(x)= x^k$. Alors
\begin{equation}
\begin{aligned}
&\|p_k\|_\infty=1,\\
&\|p_k\|_1 =\int_0^1x^k\, dx=  \frac{1}{k+1},\\
&\|p_k\|_2 =\left(\int_0^1x^{2k}\, dx\right)^{1/2}=\sqrt{\frac{1}{2k+1}}.
\end{aligned}
\end{equation}
Pour $k\to \infty$ les normes $\|p_k\|_1$, $\|p_k\|_2$ tendent vers zéro, alors que la norme $\|p_k\|_\infty$ est constante, donc les normes ne sont pas équivalentes parce que il n'existe pas un nombre positif $m$ tel que 
\begin{equation}
\begin{aligned}
& m \|p_k\|_\infty\leq \|p_k\|_1 ,\\
& m \|p_k\|_\infty\leq \|p_k\|_2 ,\\
\end{aligned}
\end{equation}
uniformément pour tout $k$ dans $\eN$.

%---------------------------------------------------------------------------------------------------------------------------
\section{Norme opérateur}
%+++++++++++++++++++++++++++++++++++++++++++++++++++++++++++++++++++++++++++++++++++++++++++++++++++++++++++++++++++++++++++
\label{SeckwyQjK}


Soit \( E\) un espace vectoriel (pas spécialement de dimension finie). Une  \defe{norme}{norme} sur $E$ est une application $\| . \|\colon E\to \eR$ telle que
\begin{enumerate}
		\label{PgDefNorme}
	\item
		$\| v \|=0$ seulement si $A=0$,
	\item
		$\| \lambda v \|=| \lambda |\cdot\| v \|$,
	\item
		$\| v+w \|\leq\| v \|+\| w \|$

\end{enumerate}
pour tout $v,w\in E$ et pour tout $\lambda\in\eR$.

\begin{definition}
	Soit $A$ une application linéaire entre espaces vectoriels réels normés. On définit sa \defe{\wikipedia{fr}{Norme_d'opérateur}{norme opérateur}}{norme!opérateur} comme le nombre
	\begin{equation}\label{EqThUCEJ}
		|A|_{\mbox{op}}:=\sup_{|x|=1}\{|\alpha(x)|\}.
	\end{equation}
\end{definition}
où dans le membre de droite, la norme est celle choisie sur \( E\). On l'écrit aussi souvent \( \| A \|_{\infty}\) parce que cette norme donne lieu à la \defe{topologie forte}{topologie!forte} sur l'espace des opérateurs.

La proposition suivante est valable également en dimension infinie. C'est elle qui montre que le produit scalaire est continu dans un espace de Hilbert par exemple.
\begin{proposition}     \label{PropmEJjLE}
    Soient \( V\) et \( W\) deux espaces vectoriels et \( T\colon V\to W\) une application linéaire bornée. Alors elle est continue.
\end{proposition}

\begin{proof}
    Pour tout \( x,y\in V\) nous avons
    \begin{equation}
        \| T(x)-T(y) \|=\| T(x-y) \|\leq \| T \|\| x-y \|.
    \end{equation}
    En particulier si \( x_n\) est une suite qui converge vers \( x\) alors
    \begin{equation}
        0\leq \| T(x_n)-T(x) \|\leq \{ T \}\| x-x_n \|\to 0
    \end{equation}
    et \( T\) est continue.
\end{proof}

La topologie forte n'est pas la seule possible. Il existe aussi par exemple la \defe{topologie faible}{topologie!faible} donnée par la notion de convergence \( A_i\to A\) si et seulement si \( A_ix\to Ax\) pour tout \( x\in E\).Il faut noter que la topologie faible n'est pas une topologie métrique. Cela même si la condition \( A_ix\to Ax\), elle, est métrique vu qu'elle est écrite dans \( E\).

et que dans le cas où \( E\) est de dimension infinie, elle est réellement différente de la topologie forte. Nous verrons à la sous-section \ref{subsecaeSywF} que dans le cas des projections sur un espaces de Hilbert, l'égalité
\begin{equation}
    \sum_{i=1}^{\infty}\pr_{u_i}=\id
\end{equation}
est vraie pour la topologie faible, mais pas pour la topologie forte.

\begin{definition}
    Une \defe{norme matricielle}{norme!matricielle} est une norme sur \( \eM(n,\eC)\) telle que pour toute matrice \( A\) et \( B\), 
    \begin{equation}
        \| AB \|\leq \| A \|\| B \|.
    \end{equation}
\end{definition}
La norme opérateur est une norme matricielle.

\begin{proposition}
    Pour tout norme matricielle, le rayon spectral d'une matrice sur \( \eC\) est toujours plus petit que sa norme. C'est à dire que nous avons toujours \( \rho(A)\leq \| A \|\) pour toute norme matricielle \( \| . \|\).
\end{proposition}

%---------------------------------------------------------------------------------------------------------------------------
\subsection{Normes de matrices}
%---------------------------------------------------------------------------------------------------------------------------
De bonnes choses peuvent être lues dans \cite{BrunelleMatricielle}.

L'ensemble de toutes les matrices de taille \( n\times n\) est un espace vectoriel de dimension \( n^2\) (voir exercice \ref{exoEspVectoNorme0009}). Nous pouvons donc y appliquer toute la théorie que nous venons de développer. Plusieurs normes sont envisageables.

\begin{definition}
    Soient $A,B\in \eM_n(\eR)$. On dit qu'une application $\| . \|: \eM_n(\eR)\to\eR$ est une \defe{norme matricielle}{norme matricielle} si
\begin{enumerate}
\item $\| A \|\geq 0\ \forall A\in \eM_n(\eR)$ et $\| A \|=0$ si et seulement si $A=0$
\item $\| \beta A\|=| \beta |\|B\|$ pour tous $\beta\in\eR$ et \( A\in\eM(\eR)\)
\item $\|A+B\|\leq\|A\|+\|B\|$ pour tous $A,B\in \eM_n(\eR)$
\item $\|AB\|\leq\|A\|\, \|B\|$ pour tous $A,B\in \eM_n(\eR)$
\end{enumerate}
\end{definition}

Remarquons que par rapport à la définition \ref{DefNorme}, nous ajoutons la condition que \( \| AB \|\leq \| A \|\| B \|\). Comme vous le verrez (ou pas) dans les années à venir, cela correspond à une condition pour obtenir une algèbre de Banach.

\begin{example}     \label{ExemdefnormpMrt}
    Pour chaque norme sur \( \eR^n\), nous pouvons définir une norme correspondante sur \( \eM_n(\eR)\), appelée \defe{norme opérateur}{norme!opérateur}. Si \( \| . \|\) est une norme sur \( \eR^n\), nous définissons \( \| A \|\) par
    \begin{equation}
        \|A\|=\sup_{\|x\|\neq 0}\frac{\|Ax\|}{\|x\|}
    \end{equation}
    En particulier, cela donne lieu à toutes les normes \( \| A \|_p\) qui correspondent aux normes \( \| . \|_p\) sur \( \eR^n\).
\end{example}

\begin{lemma}
    Cette norme peut aussi être écrite sous la forme
    \begin{equation}
        \| A \|_p=\sup_{\|x\|_p=1}\|Ax\|_p.
    \end{equation}
\end{lemma}

La preuve est l'exercice \ref{exoGeomAnal-0040}.


\begin{definition}
    Le \defe{\wikipedia{en}{Spectral_radius}{rayon spectral}}{rayon spectral} d'une matrice carrée $A$, noté $\rho(A)$, est défini de la manière suivante :
    \begin{equation}
        \rho(A)=\max_i|\lambda_i|
    \end{equation}
    où les $\lambda_i$ sont les valeurs propres de $A$.
\end{definition}

\begin{theorem}
    La norme $2$ d'une matrice peut se calculer de la manière suivante $$\|A\|_2=\sqrt{\rho(A{^t}A)}$$
\end{theorem}

%---------------------------------------------------------------------------------------------------------------------------
\subsection{Norme d'une application linéaire}
%---------------------------------------------------------------------------------------------------------------------------
\label{subsecNomrApplLin}

Nous pouvons munir $\aL(\eR^m, \eR^n)$ d'une structure d'espace vectoriel sur $\eR$ en définissant la somme et le produit par un scalaire de la façon suivante. Si $T$ et $U$ sont des élément de $\aL(\eR^m,\eR^m)$ et si $\lambda$ est un réel, nous définissons les éléments $T+U$ et $\lambda T$ par
\begin{enumerate}
	\item
		$(T+U)(x)=T(x)+U(x)$;
	\item
		$(\lambda T)(x)=\lambda T(x)$
\end{enumerate}
pour tout $x$ in $\eR^m$. Nous définissons exactement de la même manière la structure d'espace vectoriel sur $\aL(V,W)$ lorsque $V$ et $W$ sont deux espaces vectoriels.

Nous pouvons de plus définir une norme (au sens de la définition \ref{DefNorme}) sur $\aL(\eR^m,\eR^n)$ afin d'obtenir un espace vectoriel normé.
\begin{definition}		\label{DefNormeAppLineaire}
	Le nombre
	\begin{equation}
		\|T\|_{\mathcal{L}}=\sup_{x\in\eR^m}\frac{\|T(x)\|_{\eR^n}}{\|x\|_{\eR^m}}=\sup_{\|x\|_{\eR^m}\leq 1}\|T(x)\|_{\eR^n}
	\end{equation}
	est la \defe{norme}{norme!d'application linéaire} de $T$. De la même manière, si $T\in\aL(V,W)$ nous définissons
	\begin{equation}
		\| T \|_{\aL}=\sup_{v\in V}\frac{ \| T(v) \|_W }{ \| V \|_V }.
	\end{equation}
\end{definition}

Nous vérifions que l'application $\| . \|$ de $\aL(\eR^m,\eR^n)$ dans $\eR$ ainsi définie est effectivement une norme.
\begin{enumerate}
\item $\|T\|_{\mathcal{L}}=0$ signifie que $\|T(x)\|_{\eR^n}=0$ pour tout $x$ dans $\eR^m$. Comme  $\|\cdot\|_{\eR^n}$ est une norme on conclut que $T(x)=0_{n}$ pour tout $x$ dans $\eR^m$, donc $T$ est l'application nulle. 
\item Pour tout $a$ dans $\eR$ et tout  $T$ dans $\mathcal{L}(\eR^m, \eR^n)$ on a 
\[
\|aT\|_{\mathcal{L}}=\sup_{\|x\|_{\eR^m}\leq 1}\|aT(x)\|_{\eR^n}=|a|\sup_{\|x\|_{\eR^m}\leq 1}\|T(x)\|_{\eR^n}=|a|\|T\|_{\mathcal{L}}.
\]
\item Pour tous $T_1$ et $T_2$ dans $\mathcal{L}(\eR^m, \eR^n)$ on a 
  \begin{equation}\nonumber
    \begin{aligned}
       \|T_1+ T_2\|_{\mathcal{L}}&=\sup_{\|x\|_{\eR^m}\leq 1}\|T_1(x)+T_2(x)\|_{\eR^n}\leq\\
 &\leq\sup_{\|x\|_{\eR^m}\leq 1}\|T_1(x)\|_{\eR^n} +\sup_{\|x\|_{\eR^m}\leq 1}\|T_2(x)\|_{\eR^n}\\
 &=\|T_1\|_{\mathcal{L}}+\|T_2\|_{\mathcal{L}}.
    \end{aligned}
  \end{equation}
\end{enumerate}
\emph{Mutatis mutandis} la même preuve tient pour $\aL(V,W)$.

Le fait que la norme d'une application linéaire est toujours finie est une conséquence du corollaire \ref{CorFnContinueCompactBorne} et du fait que l'ensemble $\{ \| x \|\leq 1 \}$ est compact. Par conséquent la fonction
\begin{equation}
	x\mapsto \frac{ \| T(x) \|_{\eR^n} }{ \| x \|_{\eR^m} }
\end{equation}
est une fonction continue et est donc bornée sur le compact donné par la condition $\| x \|\leq 1$. Le supremum est donc un nombre réel fini.

\begin{example}
	Soit $m=n$, un point $\lambda$ dans $\eR$ et $T_{\lambda}$ l'application linéaire définie par $T_{\lambda}(x)=\lambda x$. La norme de $T_{\lambda}$ est alors
\[
\|T_{\lambda}\|_{\mathcal{L}}=\sup_{\|x\|_{\eR^m}\leq 1}\|\lambda x\|_{\eR^n}= |\lambda|.
\]
Notez que $T_{\lambda}$ n'est rien d'autre que l'homothétie de rapport $\lambda$ dans $\eR^m$.
\end{example}

\begin{example}
	Considérons la rotation $T_{\alpha}$ d'angle $\alpha$ dans $\eR^2$. Elle est donnée par l'équation matricielle
	\begin{equation}
		T_{\alpha}\begin{pmatrix}
			x	\\ 
			y	
		\end{pmatrix}=\begin{pmatrix}
			\cos\alpha	&	\sin\alpha	\\ 
			-\sin\alpha	&	\cos\alpha	
		\end{pmatrix}\begin{pmatrix}
			x	\\ 
			y	
		\end{pmatrix}=\begin{pmatrix}
			\cos(\alpha)x+\sin(\alpha)y	\\ 
			-\sin(\alpha)x+\cos(\alpha)y	
		\end{pmatrix}
	\end{equation}
	Étant donné que cela est une rotation, c'est une isométrie : $\| T_{\alpha}x \|=\| x \|$. En ce qui concerne la norme de $T_{\alpha}$ nous avons
	\begin{equation}
		\| T_{\alpha} \|=\sup_{x\in\eR^2}\frac{ \| T_{\alpha}(x) \| }{ \| x \| }=\sup_{x\in\eR^2}\frac{ \| x \| }{ \| x \| }=1.
	\end{equation}
	Toutes les rotations dans le plan ont donc une norme $1$. La même preuve tient pour toutes les rotations en dimension quelconque. 
\end{example}

\begin{example}
  Soit $m=n$, un point $b$ dans $\eR^m$ et $T_b$ l'application linéaire définie par $T_b(x)=b\cdot x$ (petit exercice : vérifiez qu'il s'agit vraiment d'une application linéaire).  La norme de $T_b$ satisfait les inégalités suivantes 
 \[
\|T_b\|_{\mathcal{L}}=\sup_{\|x\|_{\eR^m}\leq 1}\|b\cdot x\|_{\eR^n}\leq \sup_{\|x\|_{\eR^m}\leq 1}\|b \|_{\eR^n}\|x\cdot x\|_{\eR^n}\leq\|b \|_{\eR^n},
\]
\[
\|T_b\|_{\mathcal{L}}=\sup_{\|x\|_{\eR^m}\leq 1}\|b\cdot x\|_{\eR^n}\geq \left\|b\cdot \frac{b}{\|b \|_{\eR^n}}\right\|_{\eR^n}=\|b \|_{\eR^n},
\]
donc $\|T_b\|_{\mathcal{L}}=\|b \|_{\eR^n}$.
\end{example}

Une inégalité que nous utiliserons quelque fois dans la suite, y compris dans la proposition qui suit.
\begin{lemma}		\label{LemAvmajAfoisv}
	Soit $T$ une application linéaire de $\eR^m$ vers $\eR^n$. Alors
	\begin{equation}
		\| Av \|_n\leq \| A \|_{\aL}\| v \|_m.
	\end{equation}
	pour tout $v\in\eR^m$.
\end{lemma}

\begin{proof}
	Étant donné que le supremum d'un ensemble est plus grand ou égal à tous les éléments qui le compose,
	\begin{equation}
		\| A \|_{\aL(\eR^m,\eR^n)}=\sup_{x\in\eR^m}\frac{ \| Ax \| }{ \| x \| }\geq\frac{ \| Av \| }{ \| v \| },
	\end{equation}
	d'où le résultat.
\end{proof}

\begin{proposition}
  Toute application linéaire $T$ de $\eR^m$ dans $\eR^n$ est continue. 
\end{proposition}
\begin{proof}
  Soit $x$ un point dans $\eR^m$. Nous devons vérifier l'égalité
\[
\lim_{h\to 0_m}T(x+h)=T(x).
\]
Cela revient à prouver que $\lim_{h\to 0_m}T(h)=0$, parce que $T(x+h)=T(x)+T(h)$. Nous pouvons toujours majorer $\|T(h)\|_n$ par $\|T\|_{\mathcal{L}(\eR^m,\eR^n)}\| h \|_{\eR^m}$ (lemme \ref{LemAvmajAfoisv}). Quand $h$ s'approche de $ 0_m $ sa norme $\|h\|_m$ tend vers $0$, ce que nous permet de conclure parce que nous savons que de toutes façons, $\| T \|_{\aL}$ est fini.
\end{proof}

\begin{proposition}
  Soit $T_1$ dans $\mathcal{L}(\eR^m, \eR^n)$ et $T_2$ dans $\mathcal{L}(\eR^n, \eR^p)$ . Alors l'application composée $T_2\circ T_1 $ est dans $\mathcal{L}(\eR^m, \eR^p)$ et sa norme satisfait
\[
\|T_2\circ T_1 \|_{\mathcal{L}}\leq\|T_1\|_{\mathcal{L}} \|T_2\|_{\mathcal{L}}.
\]
\end{proposition}
\begin{proof}
  \begin{itemize}
  \item $T_2\circ T_1 $ est dans $\mathcal{L}(\eR^m, \eR^p)$ : soient $x,\, y$ dans $\eR^m$ et $a,\, b$ dans $\eR$ . 
    \begin{equation}\nonumber
      \begin{aligned}
       T_2&\circ T_1 (ax+by)= T_2\left(T_1(ax+by)\right)=T_2(aT_1(x)+bT_1(y))=\\
&= aT_2\left(T_1(x)\right)+ bT_2\left(T_1(y)\right) = aT_2\circ T_1(x)+ bT_2\circ T_1(y). 
      \end{aligned}
    \end{equation}  
\item
	On veut une estimation de la norme de $T_2\circ T_1 $ :
\[
\|T_2\circ T_1 \|_{\mathcal{L}}= \sup_{x\in\eR^m}\frac{\left\|T_2\left(T_1(x)\right)\right\|_{\eR^p}}{\|x\|_{\eR^m}}\leq  \sup_{x\in\eR^m}\frac{\|T_2\|_{\mathcal{L}}\left\|\left(T_1(x)\right)\right\|_{\eR^p}}{\|x\|_{\eR^m}} =\|T_1\|_{\mathcal{L}} \|T_2\|_{\mathcal{L}}.
\]
  \end{itemize}
\end{proof}


%---------------------------------------------------------------------------------------------------------------------------
\subsection{Espaces d'opérateurs}
%---------------------------------------------------------------------------------------------------------------------------

Soit \( E\), un espace vectoriel. La \defe{topologie \( *\)-faible}{topologie!$*$-faible} sur l'ensemble des opérateurs \( E\to E\) est la topologie de la convergence \( T_n\to T\) si et seulement si \( T_nv\to Tv\) pour tout \( v\in E\).


%++++++++++++++++++++++++++++++++++++++++++++++++++++++++++++++++++++++++++++++++++++++++++++++++++++++++++++++++++++++++++++++++++++++++
\section{Espaces de polynômes}		\label{SecEspacePolynomes}
%++++++++++++++++++++++++++++++++++++++++++++++++++++++++++++++++++++++++++++++++++++++++++++++++++++++++++++++++++++++++++++++++++++++++
 
Dans cette section nous abandonnons pour quelques minutes l'espace $\eR^m$ et considérons plus attentivement l'espace des fonctions polynômiales $\mathcal{P}_{\eR}$, que nous avons déjà rencontré dans la section \ref{normes_equiv}, et ses sous-espaces $\mathcal{P}_{\eR}^k$, pour $k$ dans $\eN_0$. 

Pour chaque $k>0$ donné nous définissons
\begin{equation}
\mathcal{P}_\eR^k=\{p:\eR\to \eR\,|\, p : x\mapsto a_0+a_1 x +a_2 x^2 + \ldots+a_k x^k, \, a_i\in\eR,\,\forall i=0,\ldots,k\}.
\end{equation}   
Il est facile de se convaincre que la somme de deux polynômes de degré inférieur ou égal à $k$ est encore un polynôme de degré inférieur ou égal à $k$. En outre il est clair que la multiplication par un scalaire ne peut pas augmenter le degré d'un polynôme. L'ensemble $\mathcal{P}_\eR^k$ est donc un espace vectoriel muni des opérations héritées de $\mathcal{P}_{\eR}$. 

La base canonique de l'espace $\mathcal{P}_\eR^k$ est donné par les monômes $\mathcal{B}=\{x\mapsto x^j \,|\, j=0, \ldots, k\}$. Le fait que cela soit une base est vraiment facile à démontrer et est un exercice très utile si vous ne l'avez pas encore vu dans un cours précédent. 

Nous allons maintenant étudier trois application linéaires de $\mathcal{P}_\eR^k$ vers des autres espaces vectoriels
\begin{description}
  \item[L'isomorphisme canonique  $\phi:\mathcal{P}_\eR^k \to\eR^{k+1}$] Nous définissons $\phi$ par les relations suivantes
\[
\phi(x^j)=e_{j+1}, \qquad \forall j\in\{0,\dots, k\}. 
\]
Cela veut dire que pour tout $p$ dans $\mathcal{P}_\eR^k$, avec $p(x)=a_0+a_1 x +a_2 x^2 + \ldots+a_k x^K$, l'image de $p$ par $\phi$ est 
\[
\phi(p)=\phi\left(\sum_{j=0}^k a_j x^j\right)=\sum_{j=0}^k a_j e_{j+1}.
\]
\begin{example} Soit $k=5$ on a 
  \begin{equation}
    \phi(-8-7x-4x^2+4x^3+2x^5)=
  \begin{pmatrix}
    -8\\
    -7\\
    -4\\
    4\\
    0\\
    2
  \end{pmatrix}.
  \end{equation}
\end{example}
Cette application est clairement bijective et respecte les opérations d'espace vectoriel, donc elle est un isomorphisme d'espaces vectoriels. L'existence d'un isomorphisme entre $\mathcal{P}_\eR^k$  et $\eR^{k+1}$ est un cas particulier du théorème qui dit que  pour chaque $m$ dans $\eN_0$ fixée, tous les espaces vectoriels sur $\eR$ de dimension $m$ sont isomorphes à $\eR^m$. Vous connaissez peut être déjà ce théorème depuis votre cours d'algèbre linéaire.  
    \item[La dérivation $d: \mathcal{P}_\eR^k \to \mathcal{P}_\eR^{k-1}$] L'application de dérivation $d$ fait exactement ce qu'on s'attend d'elle 
\[
d(x^0)=d(1)=0, \qquad d(x^j)=j x^{j-1}, \quad \forall j\in\{1,\dots, k\}. 
\]
Cette application n'est pas injective, parce que l'image de $p$ ne dépend pas de la valeur de $a_0$, donc si deux polynômes sont les mêmes à une constante près ils auront la même image par $d$.
\begin{example} Soit $k=3$ on a 
  \begin{equation}
    d(-8-12x+4x^3)= -12 (1) + 4 (3x^2) = -12+12 x^2.
    \end{equation}

    Noter que $d(-30-12x+4x^3)=d(-8-12x+4x^3)$. Cela confirme, comme mentionné plus haut que la dérivée n'est pas injective.
\end{example}
      \item[L'intégration $I: \mathcal{P}_\eR^k \to \mathcal{P}_\eR^{k+1}$] Nous pouvons définir une application que est <<à une constante prés>> l'application inverse de la dérivation
        \begin{equation}
          I(p)= \int_0^x p(t) \,dt.
        \end{equation}
Il faut comprendre que dans l'intégral la variable $t$ est simplement la variable d'intégration. La <<vraie>> variable de la fonction image de $p$ sera $x$ !
 
Comme d'habitude nous écrivons explicitement l'action de $I$ sur les éléments de la base canonique
\[
I(x^j)=\int_0^x t^k \,dt= \frac{x^{j+1}}{j+1}.
\] 
\begin{example} Soit $k=4$ on a 
  \begin{equation}
    I(6+2x+x^2+x^4)= 6x+x^2+\frac{x^3}{3}+\frac{x^5}{5}.
    \end{equation}
\end{example}
Remarquez que, étant donné que dans la définition de $I$ nous avons décidé d'intégrer entre zéro et $x$, tous les polynômes dans $\mathcal{P}_\eR^{k+1}$ qui sont l'image par $I$ d'un polynôme de $\mathcal{P}_\eR^{k}$ ont $a_0=0$. Cela veut dire que nous pouvons générer toute l'image de $I$ en utilisant un sous-ensemble de la base canonique de $\mathcal{P}_\eR^{k+1}$,  en particulier $\mathcal{B}_1=\{x\mapsto x^j \,|\, j=1, \ldots, k\}\subset \mathcal{B}$ nous suffira. Cela n'est guère surprenant, parce que l'image par une application linéaire d'un espace vectoriel de dimension finie ne peut pas être un espace de dimension supérieure. 
\end{description}

Les applications de dérivation et intégration correspondent évidemment à des application linéaires de $\mathcal{P}_\eR$ dans lui-même. 

%++++++++++++++++++++++++++++++++++++++++++++++++++++++++++++++++++++++++++++++++++++++++++++++++++++++++++++++++++++++++++++++++++++++++
\section{Applications multilinéaires}
%++++++++++++++++++++++++++++++++++++++++++++++++++++++++++++++++++++++++++++++++++++++++++++++++++++++++++++++++++++++++++++++++++++++++

\begin{definition}
 	Une application $T: \eR^{m_1}\times \ldots \times\eR^{m_k}\to\eR^p $ est dite $k$-linéaire si pour tout $X=(x_1, \ldots,x_k)$ dans $ \eR^{m_1}\times \ldots \times\eR^{m_k}$ les applications $x_i\mapsto T(x_1, \ldots, x_i,\ldots,x_k)$ sont linéaires pour tout $i$ dans $\{1,\ldots,k\}$, c'est à dire
	\begin{equation}
		\begin{aligned}[]
			T(\cdot,x_2, \ldots, x_i,\ldots,x_k)&\in \mathcal{L}(\eR^{m_1}, \eR^p),\\
			T(x_1,\cdot, \ldots, x_i,\ldots,x_k)&\in \mathcal{L}(\eR^{m_2}, \eR^p),\\
						& \vdots\\
			T(x_1, \ldots, x_i,\ldots,x_{k-1},\cdot)&\in \mathcal{L}(\eR^{m_k}, \eR^p).\\
		\end{aligned}
	\end{equation}
	En particulier lorsque $k=2$, nous parlons d'applications \defe{bilinéaires}{bilinéaire}. Vous pouvez deviner ce que sont les applications \emph{tri}linéaire ou \emph{quadri}linéaire.
\end{definition}

L'ensemble des applications $k$-linéaires de $ \eR^{m_1}\times \ldots \times\eR^{m_k}$ dans $\eR^p$ est noté $\mathcal{L}(\eR^{m_1}\times \ldots \times\eR^{m_k}, \eR^p)$ ou $\mathcal{L}(\eR^{m_1}, \ldots,\eR^{m_k}; \eR^p)$.
\begin{example}
  Soit $A$ une matrice avec $m$ lignes et $n$ colonnes. L'application bilinéaire de $\eR^m\times \eR^n$ dans $\eR$ associée à $A$ est définie par
\[
T_A(x,y)= x^TAy=\sum_{i,j}a_{i,j}x_i y_j, \qquad \forall x\in \eR^m, \, y \in \eR^n.
\]
\end{example}

\begin{definition}
	La norme sur l'espace $\mathcal{L}(\eR^{m_1}\times \ldots \times\eR^{m_k}, \eR^p)$ des fonction $k$-linéaires et continues est donnée par le meilleur $L$ possible, plus précisément elle est  définie par 
	\begin{equation}
		\|T\|_{\mathcal{L}(\eR^{m_1}\times \ldots \times\eR^{m_k}, \eR^p)}=\sup\{ \|T(u_1, \ldots,u_k)\|_p\,\vert\,\|u_i\|_{m_i}\leq 1, i=1,\ldots, k \}.
	\end{equation}
\end{definition}

\begin{proposition}
  L'application $k$-linéaire  $T: \eR^{m_1}\times \ldots \times\eR^{m_k}\to\eR^p $ est continue si et seulement s'il existe $L\geq 0$, réel, tel que
  \begin{equation}\label{limitatezza}
     \|T(x_1, \ldots,x_k)\|_p\leq L \|x_1\|_{m_1}\cdots\|x_k\|_{m_k}, \qquad \forall x_i\in\eR^{m_i},\,\forall i \in \{1,\ldots, k\}.
  \end{equation}
\end{proposition}

\begin{proof}
  Pour simplifier l'exposition nous nous limitons au cas $k=2$. On adopte la notation $T(x,y)=x*y$

Supposons que l'inégalité \eqref{limitatezza} soit satisfaite. 
\begin{equation}\label{LimImplCont}
  \begin{aligned}
    \|x*y-x_0*y_0\|_p&=\|(x-x_0)*y-x_0*(y-y_0)\|_p\leq\\
&\leq \|(x-x_0)*y\|_p+\|x_0*(y-y_0)\|_p\leq\\
&\leq L\|x-x_0\|_m\|y\|_n + L\|x_0\|_m\|y-y_0\|_n.
  \end{aligned}
\end{equation}
Si $x\to x_0$ et $y\to y_0$  on voit que $T$ est continue en passant à la limite aux deux côtes de l'inégalité \eqref{LimImplCont}.

Soit $T$ continue en $(0_m,0_n)$. Évidemment $0_m*0_n=0_p$, donc il existe $\delta>0$ tel que si $x$ est dans la boule de rayon $\delta$ centrée en $0_m$ et  $y$ est dans la boule de rayon $\delta$ centrée en $0_n$ alors $\|x*y\|_p\leq 1$. Soient maintenant  $x$ dans $\eR^m\setminus\{ 0_m\}$ et $y$ dans $\eR^n\setminus\{ 0_n\}$
\begin{equation}
  \begin{aligned}
    x*y=&\left(\frac{\|x\|_m}{\delta}\frac{\delta x}{\|x\|_m}\right)*\left(\frac{\|y\|_n}{\delta}\frac{\delta y}{\|y\|_n}\right)=\\
&=\frac{\|x\|_m\|y\|_n}{\delta^2} \left(\frac{\delta x}{\|x\|_m}\right)*\left(\frac{\delta y}{\|y\|_n}\right).
  \end{aligned}
 \end{equation}
On remarque que $\delta x/\|x\|_m$ est dans la boule de rayon $\delta$ centrée en $0_m$ et que $\delta y/\|y\|_n$ est dans la boule de rayon $\delta$ centrée en $0_n$. On conclut 
\[
 x*y\leq \frac{\|x\|_m\|y\|_n}{\delta^2}.
\]
Il faut prendre $L=1/\delta^2$.
\end{proof}
\begin{proposition}\label{isom_isom}
  On définit les fonctions
  \begin{equation}
    \begin{array}{rccc}
      \omega_g: & \mathcal{L}(\eR^{m}\times\eR^{n}, \eR^p)&\to &\mathcal{L}(\eR^{m}, \mathcal{L}(\eR^{n}, \eR^p)),\\
      \omega_d: & \mathcal{L}(\eR^{m}\times\eR^{n}, \eR^p)&\to &\mathcal{L}(\eR^{n}, \mathcal{L}(\eR^{m}, \eR^p)),
    \end{array}
  \end{equation}
par 
\[
\omega_g(T)(x)=T(x,\cdot), \qquad \forall x\in\eR^m,
\]
et
\[
\omega_d(T)(y)=T(\cdot, y), \qquad \forall y\in\eR^n.
\]
Les fonctions $\omega_g$ et $\omega_d$ sont des isomorphismes qui préservent les normes.    
\end{proposition}




%+++++++++++++++++++++++++++++++++++++++++++++++++++++++++++++++++++++++++++++++++++++++++++++++++++++++++++++++++++++++++++
\section{Endomorphismes}
%+++++++++++++++++++++++++++++++++++++++++++++++++++++++++++++++++++++++++++++++++++++++++++++++++++++++++++++++++++++++++++

%---------------------------------------------------------------------------------------------------------------------------
\subsection{Polynôme caractéristique}
%---------------------------------------------------------------------------------------------------------------------------

Soit \( A\) un anneau commutatif et \( \eK\), un corps commutatif. L'injection canonique \( A\to A[X]\) se prolonge en une injection
\begin{equation}
   \eM(A)\to\eM\big( A[X] \big).
\end{equation}
Si \( u\in\eM_n(A)\), nous définissons le \defe{polynôme caractéristique de \( u\)}{polynôme!caractéristique}\index{caractéristique!polynôme} :
\begin{equation}
    \chi_u(X)=\det(X\mtu_n-u).
\end{equation} 
Ce faisons nous assimilons la matrice \( u\) et l'endomorphisme \( u\colon E\to E\) qu'elle définit. 

\begin{lemma}
    Si \( u\) est un endomorphisme
    \begin{equation}
        I_u=\{ P\in \eK[X] \tq P(u)=0\}
    \end{equation}
    n'est pas vide.
\end{lemma}

\begin{proof}
    Nous avons un morphisme d'algèbre
    \begin{equation}
        \begin{aligned}
            \varphi_u\colon\eK[X]&\to \End(E) \\
            P&\mapsto P(u). 
        \end{aligned}
    \end{equation}
    Cet endomorphisme ne peut pas être injectif parce que \(\eK[X]\) est de dimension infinie tandis que \( \End(E)\) est de dimension finie. Il possède donc un noyau, c'est à dire qu'il existe \( P\in\eK[X]\) tel que \( P(X)=0\).
\end{proof}

\begin{definition}
    Le \defe{polynôme minimal}{polynôme!minimal!d'un endomorphisme}\index{minimal!polynôme!d'endomorphisme} de \( u\) est le générateur unitaire de \( I_u\). C'est le polynôme unitaire de plus petit degré qui annule \( u\). Nous le notons \( \mu_u\)\nomenclature[A]{\( \mu_A\)}{polynôme minimal de \( A\)} :
\begin{equation}
    \mu_u(u)=0.
\end{equation}
\end{definition}

\begin{lemma}
    Le polynôme \( \chi_u\) est unitaire et de degré \( n\).
\end{lemma}

\begin{lemma}       \label{LemjcztYH}
    Soit \( u\) un endomorphisme et \( E_{\lambda}(u)\)\nomenclature[A]{\( E_{\lambda}(u)\)}{Espace propre de \( u\)} ses espaces propres. La somme des \( V_{\lambda}\) est directe.
\end{lemma}

\begin{proof}
    Soit \( v_i\in V_{\lambda_i}\) un choix de vecteurs propres de \( u\). Si la somme n'est pas directe, nous pouvons considérer une combinaison linéaire des \( v_i\) qui soit nulle :
    \begin{equation}
        v_1+\ldots+v_p=0.
    \end{equation}
    Appliquons \( (A-\lambda_1\mtu)\) à cette égalité :
    \begin{equation}
        (\lambda_2-\lambda_1)v_1+\ldots+(\lambda_p-\lambda_1)v_p=0.
    \end{equation}
    En appliquant encore successivement les opérateurs \( (A-\lambda_i\mtu)\) nous réduisons le nombre de termes jusqu'à obtenir \( v_p=0\).
\end{proof}


\begin{theorem}     \label{ThoNhbrUL}
    Soit \( E\) un \(\eK\)-espace vectoriel de dimension finie \( n\) et un endomorphisme \( u\in\End(E)\). Alors
    \begin{enumerate}
        \item
            Le polynôme caractéristique divise \( (\mu_u)^n\) dans \(\eK[X]\).
        \item
            Les polynômes caractéristiques et minimaux ont mêmes facteurs irréductibles dans \(\eK[X]\).
        \item
            Les polynômes caractéristiques et minimaux ont mêmes racines dans \(\eK[X]\).
        \item
            Le polynôme caractéristique est scindé si et seulement si le polynôme minimal est scindé.
    \end{enumerate}
\end{theorem}


Si \( \lambda\in\eK\) est une racine de \( \chi_u\), l'ordre de l'annulation est la \defe{multiplicité algébrique}{multiplicité!algébrique d'une valeur propre} de la valeur propre \( \lambda\) de \( u\).

\begin{theorem}
    Soit \( u\in\End(E)\) et \( \lambda\in\eK\). Les conditions suivantes sont équivalentes
    \begin{enumerate}
        \item\label{ItemeXHXhHi}
            \( \lambda\in\Spec(u)\)
        \item\label{ItemeXHXhHii}
            \( \chi_u(\lambda)=0\)
        \item\label{ItemeXHXhHiii}
            \( \mu_u(\lambda)=0\).
    \end{enumerate}
\end{theorem}

\begin{proof}
    \ref{ItemeXHXhHi} \( \Leftrightarrow\) \ref{ItemeXHXhHii}. Dire que \( \lambda\) est dans le spectre de \( u\) signifie que l'opérateur \( u-\lambda\mtu\) n'est pas inversible, ce qui est équivalent à dire que \( \det(u-\lambda\mtu)\) est nul ou encore que \( \lambda\) est une racine du polynôme caractéristique de \( u\). 

    \ref{ItemeXHXhHii} \( \Leftrightarrow\) \ref{ItemeXHXhHiii}. Cela est une application directe du théorème \ref{ThoNhbrUL} qui précise que le polynôme caractéristique a les mêmes racines dans \(\eK\) que le polynôme minimal.
\end{proof}

\begin{lemma}
    Une matrice triangulaire supérieure avec des \( 1\) sur la diagonale n'est diagonalisable que si elle est diagonale (c'est à dire si elle est la matrice unité).
\end{lemma}

\begin{proof}
    Si \( A\) est une matrice triangulaire supérieure de taille \( n\) telle que \( A_{ii}=1\), alors \( \det(A-\lambda\mtu)=(1-\lambda)^n\), ce qui signifie que \( \Spec(A)=\{ 1 \}\). Pour la diagonaliser, il faudrait une matrice \( P\in\GL(n,\eK)\) telle que \( \mtu=P^{-1}AP\), ce qui est uniquement possible si \( A=\mtu\).
\end{proof}


%---------------------------------------------------------------------------------------------------------------------------
\subsection{Matrices semblables}
%---------------------------------------------------------------------------------------------------------------------------

Sur l'ensemble \( \eM_n(\eK)\) des matrices \( n\times n\) à coefficients dans \(\eK\) nous introduisons la relation d'équivalence \( A\sim B\) si et seulement si il existe une matrice \( P\in\GL(n,\eK)\) telle que \( B=P^{-1}AP\). Deux matrices équivalentes en ce sens sont dites \defe{semblables}{semblables!matrices}.

Le polynôme caractéristique est un invariant sous les similitudes. En effet si \( P\) est une matrice inversible,
\begin{subequations}
    \begin{align}
        \chi_{PAP^{-1}}&=\det(PAP^{-1}-\lambda X)\\
        &=\det\big( P^{-1}(PAP^{-1}-\lambda X)P^{-1} \big)\\
        &=\det(A-\lambda X).
    \end{align}
\end{subequations}

La permutation de lignes ou de colonnes ne sont pas de similitudes, comme le montrent les exemples suivants :
\begin{equation}
    \begin{aligned}[]
        A&=\begin{pmatrix}
            1    &   2    \\ 
            3    &   4    
        \end{pmatrix}&
        B&=\begin{pmatrix}
            2    &   1    \\ 
            4    &   3    
        \end{pmatrix}.
    \end{aligned}
\end{equation}
Nous avons \( \chi_A=x^2-5x-2\) tandis que \( \chi_B=x^2-5x+2\) alors que le polynôme caractéristique est un invariant de similitude.


%---------------------------------------------------------------------------------------------------------------------------
\subsection{Polynômes d'endomorphismes}
%---------------------------------------------------------------------------------------------------------------------------

Soit \( u\in\End(E)\) où \( E\) est un \( \eK\)-espace vectoriel. Nous considérons l'application
\begin{equation}
    \begin{aligned}
        \varphi_u\colon \eK[X]&\to \End(E) \\
        P&\mapsto P(u). 
    \end{aligned}
\end{equation}
L'image de \( \varphi_u\) est un sous-espace vectoriel. En effet si \( A=\varphi_u(P)\) et \( B=\varphi_u(Q)\), alors \( A+B=\varphi_u(P+Q)\) et \( \lambda A=(\lambda P)(u)\). En particulier c'est un espace fermé.

L'exponentielle \(  e^{x}\) n'est pas un polynôme en \( x\), mais nous avons les résultat marrant suivant.
\begin{proposition}
    Si \( u\) est un endomorphisme, alors \( \exp(u)\) est un polynôme d'endomorphisme\footnote{Nan, mais j'te jure : \( \exp\) n'est pas une polynôme, mais $\exp(u)$ est un polynôme d'endomorphisme.}.
\end{proposition}

\begin{proof}
    Étant donné que l'image de \( \varphi_u\) est un fermé dans \( \End(E)\), il suffit de montrer que la série
    \begin{equation}
        \sum_{k=0}^{\infty}\frac{ \varphi_u(X)^k }{ k! }
    \end{equation}
    converge dans \( \End(E)\) pour qu'elle converge dans \( \Image(\varphi_u)\). Pour ce faire nous nous rappelons de la norme opérateur \eqref{ExemdefnormpMrt} et de la propriété fondamentale \( \| A^k \|\leq \| A \|^k\). En notant \( A=\varphi_u(X)\),
    \begin{equation}
        \left\| \sum_{k=n}^m\frac{ A^k }{ k! } \right\|\leq \sum_{k=n}^m\frac{ \| A^k \| }{ k! }\leq \sum_{k=n}^m\frac{ \| A \|^k }{ k! },
    \end{equation}
    ce qui est une morceau du développement de \(  e^{\| A \|}\). La limite \( n\to\infty\) est donc zéro par la convergence de l'exponentielle réelle. La suite des sommes partielles de  $e^{A}$ est donc de Cauchy. La série converge donc parce que nous sommes dans un espace vectoriel réel de dimension finie (\( \End(E)\)).
\end{proof}
% TODO : démontrer que pour la norme opérateur, \| AB \|\leq \|  A \| \| B \|.
% TODO : et tant qu'on y est, justifier la convergence de la série de l'exponentielle réelle.

\begin{remark}
    Pourquoi \( \exp(u)\) est-il un polynôme d'endomorphisme alors que \( \exp\) n'est pas un polynôme ? Lorsque nous disons que la fonction \( x\mapsto \exp(x)\) n'est pas un polynôme, nous sommes en train de localiser la fonction \( \exp\) à l'intérieur de l'espace de toutes les fonctions \( \eR\to \eR\), c'est à dire à l'intérieur d'un espace de dimension infinie. Au contraire lorsqu'on parle de \( \exp(u)\) et qu'on le compare aux endomorphismes \( P(u)\), nous sommes en train de repérer \( \exp(u)\) à l'intérieur de l'espace des matrices qui est de dimension finie. Il n'est donc pas étonnant que l'on parvienne moins à faire la distinction.

    Si par contre nous considérons \( \exp\) en tant que application \( \exp\colon \End(E)\to \End(E)\), ce n'est pas dans un polynôme.
\end{remark}

Soit \( u\) un endomorphisme d'un \( \eK\)-espace vectoriel \( E\) et \( P\), un polynôme. Nous disons que \( P\) est un polynôme \defe{annulateur}{polynôme!annulateur} de \( u\) si \( P(u)=0\) en tant que endomorphisme de \( E\).

\begin{lemma}       \label{LemQWvhYb}
    Si \( P\) et \( Q\) sont des polynômes dans \( \eK[X]\) et si \( u\) est un endomorphisme d'un \( \eK\)-espace vectoriel \( E\), nous avons
    \begin{equation}
        (PQ)(u)=P(u)\circ Q(u).
    \end{equation}
\end{lemma}

\begin{proof}
    Si \( P=\sum_i a_iX^i\) et \( Q=\sum_j b_jX^j\), alors le coefficient de \( X^k\) dans \( PQ\) est
    \begin{equation}        \label{EqCoefGPyVcv}
        \sum_la_lb_{k-l}.
    \end{equation}
    Par conséquent \( (PQ)(u)\) contient \( \sum_la_lb_{k-l}u^k\). Par ailleurs \( P(u)\circ Q(u)\) est donné par
    \begin{equation}
        \sum_ia_iu^i\left( \sum_jb_ju^j \right)(x)=\sum_{ij}a_ib_ju^{i+j}(x).
    \end{equation}
    Le coefficient du terme en \( u^k\) est bien le même que celui donné par \eqref{EqCoefGPyVcv}.
\end{proof}

\begin{theorem}[Décomposition des noyaux ou lemme des noyaux]       \label{ThoDecompNoyayzzMWod}
    Soit \( u\) un endomorphisme du \( \eK\)-espace vectoriel \( E\). Soit \( P\in\eK[X]\) un polynôme tel que \( P(u)=0\). Nous supposons que \( P\) s'écrive comme le produit \( P=P_1\ldots P_n\) de polynômes deux à deux étrangers. Alors
    \begin{equation}
        E=\ker P_1(u)\oplus\ldots\oplus\ker P_n(u).
    \end{equation}
    De plus les projecteurs associés à cette décomposition sont des polynômes en \( u\).
\end{theorem}

\begin{proof}
    Nous posons 
    \begin{equation}
        Q_i=\prod_{j\neq i}P_i.
    \end{equation}
    Par le lemme \ref{LemuALZHn} ces polynômes sont étrangers entre eux et le théorème de Bézout (théorème \ref{ThoBezoutOuGmLB}) donne l'existence de polynômes \( R_i\) tels que
    \begin{equation}
        R_1Q_1+\ldots+R_nQ_n=1.
    \end{equation}
    Si nous appliquons cette égalité à \( u\) et ensuite à \( x\in E\) nous trouvons
    \begin{equation}        \label{EqqVcpUy}
        \sum_{i=1}^n(R_iQ_i)(u)(x)=x,
    \end{equation}
    et en particulier si nous posons \( E_i=\Image\big(P_iQ_i(u)\big)\) nous avons
    \begin{equation}
        E=\sum_{i=1}^nE_i.
    \end{equation}
    Cette dernière somme n'est éventuellement pas une somme directe. Si \( i\neq j\), alors \( Q_iQ_j\) est multiple de \( P\) et nous avons, en utilisant le lemme \ref{LemQWvhYb}, 
    \begin{equation}
        (R_iQ_i)(u)\circ (R_jQ_j)(u)=\big( R_iQ_iR_jQ_j \big)(u)=S_{ij}(u)\circ P(u)=0
    \end{equation}
    où \( S_{ij}\) est un polynôme. 

    Nous pouvons voir \( E\) comme un \( \eK\)-module et appliquer le théorème \ref{ThoProjModpAlsUR}. Les opérateurs \( R_iQ_i(u)\) ont l'identité comme somme et sont orthogonaux, et nous avons donc la décomposition en somme directe :
    \begin{equation}
        E=\bigoplus_{i=1}^nR_iQ_i(u)E.
    \end{equation}

    Afin de terminer la preuve, nous devons montrer que \( R_iQ_i(u)E=\ker P_i(u)\). D'abord nous avons
    \begin{equation}
        P_iR_iQ_i(u)=(R_iP)(u)=R_i(u)\circ P(u)=0,
    \end{equation}
    par conséquent \( \Image(R_iQ_i(u))\subset \ker P_i(u)\). Pour obtenir l'inclusion inverse, nous reprenons l'équation \eqref{EqqVcpUy} avec \( x\in\ker P_i(u)\). Elle se réduit à
    \begin{equation}
        (R_iQ_i)(u)x=x.
    \end{equation}
    Par conséquent \( x\in\Image\big( R_iQ_i(u) \big)\).
\end{proof}

\begin{corollary}   \label{CorKiSCkC}
    Soit \( E\), un \( \eK\)-espace vectoriel de dimension finie et \( f\), un endomorphisme semi-simple dont la décomposition du polynôme minimal \( \mu_f\) en facteurs irréductibles sur \( \eK[X]\) est \( \mu_f=M_1^{\alpha_1}\cdots M_r^{\alpha_r}\). Si \( F\) est un sous-espace stable par \( f\), alors
    \begin{equation}
        F=\bigoplus_{i=1}^r\ker M_i^{\alpha_i}(f)\cap F
    \end{equation}
\end{corollary}

\begin{proof}
    Nous posons \( E_i=\ker M_i^{\alpha_i}(f)\) et \( F_i=E_i\cap F\). Les polynômes \( M_i^{\alpha_i}\) sont deux à deux étrangers et \( \mu_f(f)=0\), donc le lemme des noyaux (\ref{ThoDecompNoyayzzMWod}) s'applique et
    \begin{equation}
        E=E_1\oplus\ldots\oplus E_r.
    \end{equation}
    Nous pouvons décomposer \( x\in F\) en termes de cette somme :
    \begin{equation}     \label{EqbBbrdi}
        x=x_1+\ldots +x_r
    \end{equation}
    avec \( x_i\in E_i\). Toujours selon le lemme des noyaux, les projections sur les espaces \( E_i\) sont des polynômes en \( f\). Par conséquent \( F\) est stable sous toutes ces projections \( \pr_i\colon E\to E_i\), et en appliquant \( \pr_i\) à \eqref{EqbBbrdi}, \( \pr_i(x)=x_i\). Vu que \( x\in F\), le membre de gauche est encore dans \( F\) et \( x_i\in E_i\cap F\). Nous avons donc
    \begin{equation}
        F\subset\bigoplus_{i=1}^rF_i.
    \end{equation}
    L'inclusion inverse est immédiate parce que \( F_i\subset F\) pour chaque \( i\).
\end{proof}

%---------------------------------------------------------------------------------------------------------------------------
\subsection{Polynôme minimal ponctuel}
%---------------------------------------------------------------------------------------------------------------------------

\begin{definition}  \label{Decyyumy}
    Soit \( E\), un espace vectoriel et \( f\colon E\to E\) un endomorphisme de \( E\). Pour chaque \( x\in E\) nous considérons l'idéal
    \begin{equation}
        I_{f,x}=\{ P\in \eK[X]\tq P(f)x=0 \}.
    \end{equation}
    C'est l'ensemble des polynômes qui annulent \( f\) en \( x\). Le générateur unitaire de \( I_{f,x}\) est le \defe{polynôme minimal ponctuel}{polynôme!minimal!ponctuel}\index{polynôme!minimal!relativement à un point} de \( f\) en \( x\). Il sera noté \( \mu_{f,x}\).
\end{definition}
Ces définitions sont légitimées par les faits suivants. L'idéal \( I_{f,x}\) n'est pas réduit à \( \{ 0 \}\) parce que le polynôme minimal de \( f\) fait partie de \( I_{f,x}\). C'est le théorème \ref{ThoCCHkoU} qui nous assure l'existence d'un unique générateur unitaire dans~\( I_{f,x}\). 

\begin{lemma}\label{LemSYsJJj}
    Soit \( f\colon E\to E\) un endomorphisme de l'espace vectoriel \( E\). Il existe un élément \( x\in E\) tel que \( \mu_{f,x}=\mu_f\).
\end{lemma}

\begin{proof}
    Nous savons que pour tout \( x\in E\), \( \mu_f\in I_{f,x}\), donc le polynôme \( \mu_{f,x}\) divise \( \mu_f\) pour tous les \( x\). Nous en déduisons que l'ensemble
    \begin{equation}
        \{ \mu_{f,x}\tq x\in E \}
    \end{equation}
    est en réalité un ensemble fini, sinon \( \mu_f\) ne serait pas un polynôme. Soient donc les points \( x_1,\ldots, x_l\) tels que
    \begin{equation}
        \{ \mu_{f,x}\tq x\in E \}=\{ \mu_{f,x_1},\ldots, \mu_{f,x_l} \}.
    \end{equation}
    Étant donné que \( x\in \ker\mu_{f,x}\) nous avons \( \mu_{f,x}\in I_{f,x}\) et donc \( \mu_{f,x}(f)x=0\). Par conséquent
    \begin{equation}
        E=\bigcup_{1\leq i\leq l}\ker\mu_{f,x_i(f)}.
    \end{equation}
    En vertu de la proposition \ref{PropTVKbxU}, un des termes de l'union doit être l'espace \( E\) entier. Il existe donc un \( x_i\) tel que
    \begin{equation}
        E=\ker\big( \mu_{f,x_i}(f) \big).
    \end{equation}
    Le polynôme \( \mu_{f,x_i}\) annule \( f\) et est donc divisé par le polynôme minimal de \( f\). Nous avons donc montré que \( \mu_{f,_{x_i}}\) divise et est divisé par \( \mu_f\). Par conséquent \( \mu_f=\mu_{f,x_i}\).
\end{proof}

\begin{lemma}   \label{LemrFINYT}
    Si le polynôme minimal d'un endomorphisme est irréductible, alors il est semi-simple.
\end{lemma}

\begin{proof}
    Soit \( f\), un endomorphisme dont le polynôme minimal est irréductible et \( F\), un sous-espace stable par \( f\). Nous devons en trouver un supplémentaire stable. Si \( F=E\), il n'y a pas de problèmes. Sinon nous considérons \( u_1\in E\setminus F\) et
    \begin{equation}
        E_{u_1}=\{ P(f)u_1\tq P\in \eK[X] \},
    \end{equation}
    qui est un espace stable par \( f\). 

    Montrons que \( E_{u_1}\cap F=\{ 0 \}\). Pour cela nous regardons l'idéal
    \begin{equation}
        I_{u_1}=\{ P\in \eK[X]\tq P(f)u_1=0 \}.
    \end{equation}
    Cela est un idéal non réduit à \( \{ 0 \}\) parce que le polynôme minimal de \( f\) par exemple est dans \( I_{u_1}\). Soit \( P_{u_1}\) un générateur unitaire de \( I_{u_1}\). Étant donné que \( \mu_f\in I_{u_1}\), nous avons que \( P_{u_1}\) divise \( \mu_f\) et donc \( P_{u_1}=\mu_f\) parce que \( \mu_f\) est irréductible par hypothèse.

    Soit \( y\in E_{u_1}\cap F\). Par définition il existe \( P\in\eK[X]\) tel que \( y=P(f)u_1\) et si \( y\neq 0\), ce la signifie que \( P\notin I_{u_1}\), c'est à dire que \( P_{u_1} \) ne divise pas \( P\). Étant donné que \( P_{u_1}\) est irréductible cela implique que \( P_{u_1}\) et \( P\) sont premiers entre eux (ils n'ont pas d'autre \( \pgcd\) que \( 1\)).

    Nous utilisons maintenant Bézout (théorème \ref{ThoBezoutOuGmLB}) qui nous donne \( A,B\in \eK[X]\) tels que 
    \begin{equation}
        AP+BP_{u_1}=1.
    \end{equation}
    Nous appliquons cette égalité à \( f\) et puis à \( u_1\):
    \begin{equation}
        u_1=A(f)\circ \underbrace{P(f)u_1}_{=y}+B(f)\circ \underbrace{P_{u_1}(u_1)}_{=0}=A(f)y.
    \end{equation}
    Mais \( y\in F\), donc \( A(f)y\in F\). Nous aurions donc \( u_1\in F\), ce qui est impossible par choix. Nous avons maintenant que l'espace \( E_{u_1}\oplus F\) est stable sous \( f\). Si cet espace est \( E\) alors nous arrêtons. Sinon nous reprenons le raisonnement avec \( E_{u_1}\oplus F\) en guise de \( F\) et en prenant \( u_2\in E\setminus(E_{u_1}\oplus F)\). Étant donné que \( E\) est de dimension finie, ce procédé s'arrête à un certain moment et nous aurons
    \begin{equation}
        E=F\oplus E_{u_1}\oplus\ldots\oplus E_{u_k}
    \end{equation}
    où chacun des \( E_{u_i}\) sont stables.
\end{proof}

\begin{theorem} \label{ThoFgsxCE}
    Un endomorphisme est semi-simple si et seulement si son polynôme minimal est produit de polynômes irréductibles distincts deux à deux.
\end{theorem}

\begin{proof}

    Supposons que \( f\) soit semi-simple et que son polynôme minimal soit donné par \( \mu_f=M_1^{\alpha_1}\ldots M_r^{\alpha_r}\) où les \( M_i\) sont des polynômes irréductibles deux à deux distincts. Nous devons montrer que \( \alpha_i=1\) pour tout \( i\). Soit \( i\) tel que \( \alpha_i\geq 1\) et \( N\in \eK[X]\) tel que \( \mu_f=M^2N\) où l'on a noté \( M=M_i\). Nous étudions l'espace
    \begin{equation}
        F=\ker M(f)
    \end{equation}
    qui est stable par \( f\), et qui possède donc un supplémentaire \( S\) également stable par \( f\). Nous allons montrer que \( MN\) est un polynôme annulateur de \( f\).

    D'abord nous prenons \( x\in S\). Étant donné que \( F\) est le noyau de \( M(f)\),
    \begin{equation}
        M(f)\big( MN(f)x \big)=\mu_f(f)x=0,
    \end{equation}
    ce qui signifie que \( MN(f)x\in F\). Mais vu que \( S\) est stable par \( f\) nous avons aussi que \( MN(f)x\in S\). Finalement \( MN(f)x\in F\cap S=\{ 0 \}\). Autrement dit, \( MN(f)\) s'annule sur \( S\).

    Prenons maintenant \( y\in F\). Nous avons
    \begin{equation}
        MN(f)=N(f)\big( M(f)y \big)=0
    \end{equation}
    parce que \( y\in F=\ker M(f)\).

    Nous avons prouvé que \( MN(f)\) s'annule partout et donc que \( MN(f)\) est un polynôme annulateur de \( f\), ce qui contredit la minimalité de \( \mu_f=M^2N\).

    Nous passons au sens inverse. Soit \( m_f=M_1\ldots M_r\) une décomposition du polynôme minimal de l'endomorphisme \( f\) en irréductibles distincts deux à deux. Soit \( F\) un sous-espace vectoriel stable par \( f\). Nous notons
    \begin{equation}
        E_i=\ker(M_i(f))
    \end{equation}
    et \( f_i=f|_{E_i}\). Par le lemme \ref{CorKiSCkC} nous avons
    \begin{equation}
        F=\bigoplus_{i=1}^r(F\cap E_i).
    \end{equation}
    Les espaces \( E_i\) sont stables par \( f\) et étant donné que \( M_i\) est irréductible, il est le polynôme minimal de \( f_i\). En effet, \( M_i\) est annulateur de \( f_i\), ce qui montre que le minimal de \( f_i\) divise \( M_i\). Mais \( M_i\) étant irréductible, \( M_i\) est le polynôme minimal. Étant donné que \( \mu_{f_i}=M_i\), l'endomorphisme \( f_i\) est semi-simple par le lemme \ref{LemrFINYT}.

    L'espace \( F\cap E_i\) étant stable par l'endomorphisme semi-simple \( f_i\), il possède un supplémentaire stable que nous notons \( S_i\)~:
    \begin{equation}
        E_i=S_i\oplus(F\cap E_i).
    \end{equation}
    Étant donné que sur chaque \( S_i\) nous avons \( f|_{S_i}=f_i\), l'espace \( S=S_1\oplus\ldots\oplus S_r\) est stable par \( f\). Du coup nous avons
    \begin{subequations}
        \begin{align}
            E&=E_1\oplus\ldots\oplus E_r\\
            &=\big( S_1\oplus(F\cap E_1) \big)\oplus\ldots\oplus\big( S_r\oplus(F\cap E_r) \big)\\
            &=\big( \bigoplus_{i=1}^rS_i \big)\oplus\big( \bigoplus_{i=1}^rF\cap E_i \big)\\
            &=S\oplus F,
        \end{align}
    \end{subequations}
    ce qui montre que \( F\) a bien un supplémentaire stable par \( f\) et donc que \( f\) est semi-simple.
\end{proof}

\begin{proposition}     \label{PropAnnncEcCxj}
    Si \( P\) est un polynôme tel que \( P(u)=0\), alors le polynôme minimal \( \mu_u\) divise \( P\).
\end{proposition}

\begin{proof}
    L'ensemble \( I=\{ Q\in \eK[X]\tq Q(u)=0 \}\) est un idéal par le lemme \ref{LemQWvhYb}. Le polynôme minimal de \( u\) est un élément de degré plus bas dans \( I\) et par conséquent \( I=(\mu_u)\) par le théorème \ref{ThoCCHkoU}. Nous concluons que \( \mu_u\) divise tous les éléments de \( I\).
\end{proof}

\begin{lemma}
    Soit \( f\) un endomorphisme cyclique d'un espace vectoriel \( E\) de dimension finie et \( y\), un vecteur cyclique de \( f\). Alors le polynôme minimal de \( f\) en \( y\) est le polynôme minimal de \( f\).
\end{lemma}

\begin{proof}
    En utilisant les notations de la définition \ref{Decyyumy}, nous devons démontrer que \( \mu_{f,y}=\mu_f\). Bien entendu, \( \mu_f\in I_{y,f}\), donc \( \mu_{f,y}\) divise \( \mu_f\). Montrons que \( \mu_{f,y}\) est un polynôme annulateur de \( f\). Dans ce cas \( \mu_f\) divisera \( \mu_{f,y}\) et le lemme sera démontré.

    Le vecteur \( y\) étant cyclique, tout élément de \( E\) s'écrit sous la forme \( x=P(f)y\) où \( P\) est un polynôme (de degré égal à la dimension de \( E\)). En utilisant le lemme \ref{LemQWvhYb} nous avons
    \begin{equation}
            \mu_{f,y}(f)x=\big( \mu_{f,y}(f)\circ P(f) \big)y
            =\big( P(f)\circ \mu_{f,y}(f) \big)y
            =0.
    \end{equation}
\end{proof}

Si \( f\) est un endomorphisme de l'espace vectoriel \( E\) et si \( x\in E\), nous notons 
\begin{equation}
    E_{f,x}=\Span\{ f^k(x)\tq k\in \eN \}.
\end{equation}

\begin{proposition}[\cite{RombaldiO}]\label{PropNrZGhT}
    Soit \( f\), un endomorphisme de \( E\) et \( x\in E\). Alors
    \begin{enumerate}
        \item
            L'espace \( E_{f,x}\) est stable par \( f\).
        \item\label{ItemfzKOCo}
            L'espace \( E_{f,x}\) est de dimension
            \begin{equation}
                p_{f,x}=\dim E_{f,x}=\deg(\mu_{f,x})
            \end{equation}
            où \( \mu_{f,x}\) est le générateur unitaire de \( I_{f,x}\).
        \item   \label{ItemKHNExH}
            Le polynôme caractéristique de \( f|_{E_{f,x}}\) est \( \mu_{f,x}\).
        \item   \label{ItemHMviZw}
            Nous avons
            \begin{equation}
                \chi_{f|_{E_{f,x}}}(f)x=\mu_{f,x}(f)x=0.
            \end{equation}
    \end{enumerate}
\end{proposition}

\begin{proof}
    Le fait que \( E_{f,x}\) soit stable par \( f\) est classique. Le point \ref{ItemHMviZw} est un une application du point \ref{ItemKHNExH}. Les deux gros morceaux sont donc les points \ref{ItemfzKOCo} et \ref{ItemKHNExH}.

    Étant donné que \( \mu_{f,x}\) est de degré minimal dans \( I_{f,x}\), l'ensemble
    \begin{equation}
        B=\{ f^k(x)\tq 0\leq k\leq p_{f,x}-1 \}
    \end{equation}
    est libre. En effet une combinaison nulle des vecteurs de \( B\) donnerait un polynôme en \( f\) de degré inférieur à \( p_{f,x}\) annulant \( x\). Nous écrivons
    \begin{equation}
        \mu_{f,x}(X)=X^{p_{f,x}}-\sum_{i=0}^{p_{f,x}-1}a_iX^k. 
    \end{equation}
    Étant donné que \( \mu_{f,x}(f)x=0\) et que la somme du membre de droite est dans \( \Span(B)\), nous avons \( f^{p_{f,x}}(x)\in\Span(B)\). Nous prouvons par récurrence que \( f^{p_{f,x}+k}(x)\in\Span(B)\). En effet en appliquant \( f^k\) à l'égalité
    \begin{equation}
        0=f^{p_{f,x}}(x)-\sum_{i=0}^{p_{f,x}-1}a_if^i(x)
    \end{equation}
    nous trouvons
    \begin{equation}
        f^{p_{f,x}+k}(x)=\sum_{i=0}^{p_{f,x}-1}a_if^{i+k}(x),
    \end{equation}
    alors que par hypothèse de récurrence le membre de droite est dans \( \Span(B)\). L'ensemble \( B\) est alors générateur de \( E_{f,x}\) et donc une base d'icelui. Nous avons donc bien \( \dim(E_{f,x})=p_{f,x}\).

    Nous montrons maintenant que \( \mu_{f,x}\) est annulateur de \( f\) au point \( x\). Nous savons que
    \begin{equation}
        \mu_{f,x}(f)x=0.
    \end{equation}
    En y appliquant \( f^k\) et en profitant de la commutativité des polynômes sur les endomorphismes (proposition \ref{LemQWvhYb}), nous avons
    \begin{equation}
        0=f^k\big( \mu_{f,x}(f)x \big)=\mu_{f,x}(f)f^k(x),
    \end{equation}
    de telle sorte que \( \mu_{f,x}(f)\) est nul sur \( B\) et donc est nul sur \( E_{f,x}\). Autrement dit,
    \begin{equation}
        \mu_{f,x}\big( f|_{E_{f,x}} \big)=0.
    \end{equation}
    Montrons que \( \mu_{f,x}\) est même minimal pour \( f|_{E_{f,x}}\). Sot \( Q\), un polynôme non nul de degré \( p_{f,x}-1\) annulant \( f|_{E_{f,x}}\). En particulier \( Q(f)x=0\), alors qu'une telle relation signifierait que \( B\) est un système lié, alors que nous avons montré que c'était un système libre. Nous concluons que \( \mu_{f,x}\) est le polynôme minimal de \( f|_{E_{f,x}}\).
\end{proof}

\begin{theorem}[Cayley-Hamlilton]\index{théorème!Cayley-Hamilton}   \label{ThoCalYWLbJQ}
    Le polynôme caractéristique est un polynôme annulateur.
\end{theorem}

\begin{proof}
    Nous devons prouver que \( \chi_f(f)x=0\) pour tout \( x\in E\). Pour cela nous nous fixons un \( x\in E\), nous considérons l'espace \( E_{f,x}\) et \( \chi_{f,x}\), le polynôme caractéristique de \( f|_{E_{f,x}}\). Étant donné que \( E_{f,x}\) est stable par \( f\), le polynôme caractéristique de \( f|_{E_{j,x}}\) divise \( \chi_f\), c'est à dire qu'il existe un polynôme \( Q_x\) tel que
    \begin{equation}
        \chi_f=Q_x\chi_{f,x},
    \end{equation}
    et donc aussi
    \begin{equation}
        \chi_f(f)x=Q_x(f)\big( \chi_{f,x}(f)x \big)=0
    \end{equation}
    parce que la proposition \ref{PropNrZGhT} nous indique que \( \chi_{f,x}\) est un polynôme annulateur de \( f|_{E_{f,x}}\).
\end{proof}

Le polynôme de Cayley-Hamilton donne un moyen de calculer l'inverse d'un endomorphisme inversible pourvu que l'on sache son polynôme caractéristique. En effet, supposons que
\begin{equation}
    \chi_f(X)=\sum_{k=0}^na_kX^k.
\end{equation}
Nous aurons alors
\begin{equation}
    0=\chi_f(f)=\sum_{k=0}^na_kf^k.
\end{equation}
Nous appliquons \( f^{-1}\) à cette dernière égalité en sachant que \( f^{-1}(0)=0\) :
\begin{equation}
    0=a_0f^{-1}+\sum_{k=1}^na_kf^{k-1},
\end{equation}
et donc
\begin{equation}
    u^{-1}=-\frac{1}{ \det(f) }\sum_{k=1}^na_kf^{k-1}
\end{equation}
où nous avons utilisé le fait que \( a_0=\chi_f(0)=\det(f)\).

\Exo{reserve0004}

%+++++++++++++++++++++++++++++++++++++++++++++++++++++++++++++++++++++++++++++++++++++++++++++++++++++++++++++++++++++++++++
\section{Diagonalisation}
%+++++++++++++++++++++++++++++++++++++++++++++++++++++++++++++++++++++++++++++++++++++++++++++++++++++++++++++++++++++++++++

Ici encore \( \eK\) est un corps commutatif.

%---------------------------------------------------------------------------------------------------------------------------
\subsection{Endomorphismes diagonalisables}
%---------------------------------------------------------------------------------------------------------------------------

\begin{lemma}       \label{LemgnaEOk}
    Soit \( F\) un sous-espace stable par \( u\). Soit une décomposition du polynôme minimal
    \begin{equation}
        \mu_u=P_1^{n_1}\ldots P_r^{n_r}
    \end{equation}
    où les \( P_i\) sont des polynômes irréductibles unitaires distincts. Si nous posons \( E_i=\ker P_i^{n_i}\), alors
    \begin{equation}
        F=(F\cap E_1)\oplus\ldots \oplus(F\cap E_r).
    \end{equation}
\end{lemma}

\begin{theorem}     \label{ThoDigLEQEXR}
    Soit \( E\), un espace vectoriel de dimension \( n\) sur le corps commutatif \( \eK\) et \( u\in\End(E)\). Les propriétés suivantes sont équivalentes.
    \begin{enumerate}
        \item       \label{ItemThoDigLEQEXRi}
            Il existe un polynôme \( P\in\eK[X]\) non constant, scindé sur \(\eK\) dont toutes les racines sont simples tel que \( P(u)=0\).
        \item\label{ItemThoDigLEQEXRii}
            Le polynôme minimal \( \mu_u\) est scindé sur \(\eK\) et toutes ses racines sont simples
        \item\label{ItemThoDigLEQEXRiii}
            Tout sous-espace de \( E\) possède un supplémentaire stable par \( u\).
        \item\label{ItemThoDigLEQEXRiv}
            L'endomorphisme \( u\) est diagonalisable.
    \end{enumerate}

\end{theorem}

\begin{proof}
    \ref{ItemThoDigLEQEXRi}\( \Rightarrow\)\ref{ItemThoDigLEQEXRii}. Étant donné que \( P(u)=0\), il est dans l'idéal des polynôme annulateurs de \( u\), et le polynôme minimal \( \mu_u\) le divise parce que l'idéal des polynôme annulateurs est généré par \( \mu_u\) par le théorème \ref{ThoCCHkoU}.

    \ref{ItemThoDigLEQEXRii}\( \Rightarrow\)\ref{ItemThoDigLEQEXRiv}. Étant donné que le polynôme minimal est scindé à racines simples, il s'écrit sous forme de produits de monômes tous distincts, c'est à dire
    \begin{equation}
        \mu_u(X)=(X-\lambda_1)\ldots(X-\lambda_r)
    \end{equation}
    où les \( \lambda_i\) sont des éléments distincts de \( \eK\). Étant donné que \( \mu_u(u)=0\), le théorème de décomposition des noyaux (théorème \ref{ThoDecompNoyayzzMWod}) nous enseigne que
    \begin{equation}
        E=\ker(u-\lambda_1)\oplus\ldots\oplus\ker(u-\lambda_r).
    \end{equation}
    Mais \( \ker(u-\lambda_i)\) est l'espace propre \( E_{\lambda_i}(u)\). Donc \( u\) est diagonalisable.

    \ref{ItemThoDigLEQEXRiv}\( \Rightarrow\)\ref{ItemThoDigLEQEXRiii}. Soit \( \{ e_1,\ldots, e_n \}\) une base qui diagonalise \( u\), soit \( F\) un sous-espace de \( E\) un \( \{ f_1,\ldots, f_r \}\) une base de \( F\). Par le théorème \ref{ThoBaseIncompjblieG} (qui généralise le théorème de la base incomplète), nous pouvons compléter la base de \( F\) par des éléments de la base \( \{ e_i \}\). Le complément ainsi construit est invariant par \( u\).

    \ref{ItemThoDigLEQEXRiii}\( \Rightarrow\)\ref{ItemThoDigLEQEXRiv}. En dimension un, tout endomorphisme est diagonalisable, nous supposons donc que \( \dim E=n\geq 2\). Nous procédons par récurrence sur le nombre de vecteurs propres connus de \( u\). Supposons avoir déjà trouvé \( p\) vecteurs propres \( e_1,\ldots, e_p\) de \( u\). Considérons \( H\), un hyperplan qui contient les vecteurs \( e_1,\ldots, e_p\). Soit \( F\) un supplémentaire de \( H\) stable par \( u\); par construction \( \dim F=1\) et si \( e_{p+1}\in F\), il doit être vecteur propre de \( u\).

    \ref{ItemThoDigLEQEXRiv}\( \Rightarrow\)\ref{ItemThoDigLEQEXRi}. Nous supposons maintenant que \( u\) est diagonalisable. Soient \( \lambda_1,\ldots, \lambda_r\) les valeurs propres deux à deux distinctes, et considérons le polynôme
    \begin{equation}
        P(x)=(X-\lambda_1)\ldots (X-\lambda_r).
    \end{equation}
    Alors \( P(u)=0\). En effet si \( e_i\) est un vecteur propre pour la valeur propre \( \lambda_i\), 
    \begin{equation}
        P(u)e_i=\prod_{j\neq i}(u-\lambda_j)\circ(u-\lambda_i)e_i=0
    \end{equation}
    par le lemme \ref{LemQWvhYb}. Par conséquent \( P(u)\) s'annule sur une base.
\end{proof}

\begin{corollary}
    Si \( u\) est diagonalisable et si \( F\) est une sous-espace stable par \( u\), alors
    \begin{equation}
        F=\sum_{\lambda}E_{\lambda}(u)\cap F
    \end{equation}
    où \( E_{\lambda}(u)\) est l'espace propre de \( u\) pour la valeur propre \( \lambda\). En particulier la restriction de \( u\) à \( F\), \( u|_F\) est diagonalisable.
\end{corollary}

\begin{proof}
    Par le théorème \ref{ThoDigLEQEXR}, le polynôme \( \mu_x\) est scindé et ne possède que des racines simples. Notons le
    \begin{equation}
        \mu_u(X)=(X-\lambda_1)\ldots (X-\lambda_r).
    \end{equation}
    Les espaces \( E_i\) du lemme \ref{LemgnaEOk} sont maintenant les espaces propres.

    En ce qui concerne la diagonalisabilité de \( u|_F\), notons que nous avons une base de \( F\) composée de vecteurs dans les espaces \( E_{\lambda}(u)\). Cette base de \( F\) est une base de vecteurs propres de \( u\).
\end{proof}

\begin{lemma}
    Soit \( E\) un \( \eK\)-espace vectoriel et \( u\in\End(E)\). Si \( \Card\big( \Spec(u) \big)=\dim(E)\) alors \( u\) est diagonalisable.
\end{lemma}

\begin{proof}
    Soient \( \lambda_1,\ldots, \lambda_n\) les valeurs propres distinctes de \( u\). Nous savons que les espaces propres correspondants sont en somme directe (lemme \ref{LemjcztYH}). Par conséquent \( \Span\{ E_{\lambda_i}(u) \}\) est de dimension \( n\) est \( u\) est diagonalisable.
\end{proof}

\begin{proposition}     \label{PropGqhAMei}
    Soit \( (u_i)_{i\in I}\) une famille d'endomorphismes qui commutent deux à deux.
    \begin{enumerate}
        \item       \label{ItemGqhAMei}
            Si \( i,j\in I\) alors tout sous-espace propre de \( u_i\) est stable par \( u_j\). Autrement dit \( u_j\big(E_{\lambda}(u)\big)\subset E_{\lambda}(u)\).
        \item
            Si les \( u_i\) sont diagonalisables, alors ils le sont simultanément.
    \end{enumerate}
\end{proposition}

\begin{proof}

    Supposons que \( u_i\) et \( u_j\) commutent et soit \( x\) un vecteur propre de \( u_i\) : \( u_ix=\lambda x\). Nous montrons que \( u_jx\in E_{\lambda}(u)\). Nous avons
    \begin{equation}
        u_i\big( u_j(x) \big)=u_j\big( u_i(x) \big)=\lambda u_j(x).
    \end{equation}
    Par conséquent \( u_j(x)\) est vecteur propre de \( u_i\) de valeur propre \( \lambda\).

    Montrons maintenant l'affirmation à propos des endomorphismes simultanément diagonalisables. Si \( \dim E=1\), le résultat est évident. Nous supposons également qu'aucun des \( u_i\) n'est multiple de l'identité. Nous effectuons une récurrence sur la dimension.

    Soit \( u_0\) un des \( u_i\) et considérons ses valeurs propres deux à deux distinctes \( \lambda_1,\ldots, \lambda_r\). Pour chaque \( k\) nous avons
    \begin{equation}
        E_{\lambda_k}(u_0)\neq E,
    \end{equation}
    sinon \( u_0\) serait un multiple de l'identité. Par contre nous avons
    \begin{equation}
        E=\bigoplus_{k}E_{\lambda_k}(u_0).
    \end{equation}
    Par le point \ref{ItemGqhAMei}, nous avons \( u_i\colon E_{\lambda_k}(u_0)\to E_{\lambda_k}(u_0)\), et nous pouvons considérer la famille d'opérateurs
    \begin{equation}
        \left( u_i|_{E_{\lambda_k}(u_0)} \right)_{i\in I}.
    \end{equation}
    Ce sont tous des opérateurs qui commutent et qui agissent sur un espace de dimension plus petite. Par hypothèse de récurrence nous avons une base de \( E_{\lambda_k}(u_0)\) qui diagonalise tous les \( u_i\).
\end{proof}

\begin{example}     \label{ExewINgYo}
    Soit un espace vectoriel sur un corps \( \eK\). Un opérateur \defe{involutif}{involution} est un opérateur différent de l'identité dont le carré est l'identité. Typiquement une symétrie orthogonale dans \( \eR^3\). Le polynôme caractéristique d'une involution est \( X^2-1=(X+1)(X-1)\).
    
    Tant que \( 1\neq -1\), \( X^1-1\) est donc scindé à racines simples et les involutions sont diagonalisables (\ref{ThoDigLEQEXR}). Cependant si le corps est de caractéristique \( 2\), alors \( X^2-1=(X+1)^2\) et l'involution n'est plus diagonalisable.

    Par exemple si le corps est de caractéristique \( 2\), nous avons
    \begin{subequations}
        \begin{align}
            A&=\begin{pmatrix}
                1    &   1    \\ 
                0    &   1    
            \end{pmatrix}\\
            A^1&=\begin{pmatrix}
                1    &   2    \\ 
                0    &   1    
            \end{pmatrix}=\begin{pmatrix}
                1    &   0    \\ 
                0    &   1    
            \end{pmatrix}.
        \end{align}
    \end{subequations}
    Ce \( A\) est donc une involution mais n'est pas diagonalisable.
\end{example}

\begin{proposition} \label{PropleGdaT}
    Soit \( p\) un nombre premier et \( P\) un élément de \( \eF_p[X]\). L'anneau \( \eF_p[X]/(P)\) est intègre si et seulement si \( P\) est irréductible dans \( \eF_p[X]\).
\end{proposition}

\begin{proof}
    Supposons que \( P\) soit réductible dans \( \eF_p[X]\), c'est à dire qu'il existe \( Q,R\in \eF_p[X]\) tels que \( P=QR\). Dans ce cas, \( \bar Q\) est diviseur de zéro dans \( \eF_p[X]/(P)\) parce que \( \bar Q\bar R=0\).

    Nous supposons maintenant que \( \eF_p[X]/(P)\) ne soit pas intègre : il existe des polynômes \( R,Q\in \eF_p[X]\) tels que \( \bar Q\bar R=0\). Dans ce cas le polynôme \( QR\) est le produit de \( P\) par un polynôme : \( QR=PA\). Tous les facteurs irréductibles de \( A \) étant soit dans \( Q\) soit dans \( R\), il est possible de modifier un peu \( Q\) et \( R\) pour obtenir \( QR=P\), ce qui signifie que \( P\) n'est pas irréductible.
\end{proof}

\begin{theorem}[Théorème des deux carrés]   \label{ThospaAEI}
    Un nombre premier est somme de deux carrés si et seulement si \( p=2\) ou \( p=1\mod 4\).
\end{theorem}
\begin{remark}
    Il n'est pas dit que les nombres dans \( [1]_4\) sont premiers (\( 9=8+1\) ne l'est pas par exemple). Le théorème signifie que (à part \( 2\)), si un nombre premier est dans \( [1]_4\) alors il est somme de deux carrés, et inversement, si un nombre premier est somme de deux carrés, il est dans \( [1]_4\).
\end{remark}

\begin{proof}
    Nous notons \( \Sigma=\{ a^2+b^2\tq a,b\in \eN \}\). Soit \( p\) un nombre premier dans \( \Sigma\). Si \( a=2k\), alors \( a^2=4k^2\) et \( a^2=0\mod 4\). Si au contraire \( a\) est impair, \( a=2k+1\) et \( a^2=4k^2+1+4k=1\mod 4\). La même chose est valable pour \( b\). Par conséquent, \( a^2+b^2\) est automatiquement \( [0]_4\), \( [1]_4\) ou \( [2]_4\). Évidemment les nombres de la forme \( 0\mod 4\) ne sont pas premiers; parmi les \( 2\mod 4\), seul \( p=2\) est premier (et vaut \( 1^2+1^2\)).

    Nous avons démontré que les seuls premiers de la forme \( a^2+b^2\) sont \( p=2\) et les \( p=1\mod 4\). Il reste à faire le contraire : démontrer que si un nombre premier \( p\) vaut \( 1\mod 4\), alors il est premier. Nous considérons l'anneau
    \begin{equation}
        \eZ[i]=\{ a+bi\tq a,b\in \eZ \}.
    \end{equation}
    puis l'application
    \begin{equation}
        \begin{aligned}
            N\colon \eZ[i]&\to \eN \\
            a+bi&\mapsto a^2+b^2. 
        \end{aligned}
    \end{equation}
    Un peu de calcul dans \( \eC\) montre que pour tout \( z,z'\in \eZ[i]\), \( N(zz')=N(z)N(z')\).

    Déterminons maintenant les éléments inversibles de \( \eZ[i]\). Si \( z\in \eZ[i]^*\), alors il existe \( z'\in \eZ[i]^*\) tel que \( zz'=1\). Dans ce cas nous aurions
    \begin{equation}
        1=N(zz')=N(z)N(z'),
    \end{equation}
    ce qui est uniquement possible avec \( N(z)=N(z')=1\), c'est à dire \( z=\pm 1\) ou \( z=\pm i\). Nous avons donc
    \begin{equation}
        \eZ[i]^*=\{ \pm 1,\pm i \}.
    \end{equation}
    Nous montrons que \( \eZ[i]\) est un anneau euclidien en montrant que \( N\) est un stathme. Soient \( t,t\in \eZ[i]\setminus\{ 0 \}\) et 
    \begin{equation}
        \frac{ z }{ t }=x+iy
    \end{equation}
    dans \( \eC\). Nous considérons \( q=a+bi\) où \( a\) et \( b\) sont les entiers les plus proches de \( x\) et \( y\). Si il y a \emph{ex aequo}, on prend au hasard\footnote{Dans l'exemple \ref{ExwqlCwvV}, nous prenions toujours l'inférieur parce que le stathme tenait compte de la positivité.}. Alors nous avons
    \begin{equation}
        | \frac{ z }{ t }-q |\leq \frac{ | 1+i | }{ 2 }=\frac{ \sqrt{2} }{2}<1.
    \end{equation}
    On pose \( r=z-qt\) qui est bien un élément de \( \eZ[i]\). De plus
    \begin{equation}
        | r |=| z-qt |=| t | |\frac{ z }{ t }-q |<| t |,
    \end{equation}
    c'est à dire que \( | r |^2<| t |^2\) et donc \( N(r)<N(t)\). L'anneau \( \eZ[i]\) étant euclidien, il est principal (proposition \ref{Propkllxnv}).

    Pour la suite, nous allons d'abord montrer que \( p\in\Sigma\) si et seulement si \( p\) n'est pas irréductible dans \( \eZ[i]\), puis nous allons voir quels sont les irréductibles de \( \eZ[i]\).

    Soit \( p\), un nombre premier dans \( \Sigma\). Si \( p=a^2+b^2\), alors nous avons \( p=(a+ib)(a-bi)\), mais étant donné que \( p\) est premier, nous avons \( a\neq 0\) et \( b\neq 0\). Du coup \( p\) n'est pas inversible dans \( \eZ[i]\), mais il peut être écrit comme le produit de deux non inversibles. Le nombre \( p\) est donc non irréductible dans \( \eZ[i]\).

    Dans l'autre sens, nous supposons que \( p\) est un nombre premier non irréductible dans \( \eZ[i]\). Nous avons alors \( p=zz'\) avec ni \( z\) ni \( z'\) dans \( \{ \pm 1,\pm i \}\). En appliquant \( N\) nous avons
    \begin{equation}
        p^2=N(p)=N(z)N(z').
    \end{equation}
    Vu que \( p\) est premier, cela est uniquement possible avec \( N(z)=N(z')=p\) (avoir \( N(z)=1\) est impossible parce que cela dirait que \( z\) est inversible). Si \( z=a+ib\), alors \( p=N(z)=a^2+b^2\), et donc \( p\in \Sigma\).

    Nous cherchons maintenant les éléments irréductibles de \( \eZ[i]\). Nous savons déjà que \( \eZ[i]\) est un anneau principal et n'est pas un corps; la proposition \ref{PropomqcGe} s'applique donc et \( p\) sera non irréductible si et seulement si l'idéal \( (p)\) sera non premier. Le fait que \( (p)\) soit un idéal non premier implique que le quotient \( \eZ[i]/(p)\) est non intègre (c'est la définition d'un idéal premier). Nous cherchons donc les nombres premiers pour lesquels le quotient \( \eZ[i]/(p)\) n'est pas intègre.

    Nous commençons par écrire le quotient \( \eZ[i]/(p)\) sous d'autres formes. D'abord en remarquant que si \( I\) et \( J\) sont deux idéaux, on a \( (\eA/I)/J\simeq (\eA/J)/I\), du coup, en tenant compte du fait que \( \eZ[i]=\eZ[X]/(X^2+1)\), nous avons
    \begin{equation}
        \eZ[i]/(p)=(\eZ[X]/(p))/(X^2+1)=\eF_p[X]/(X^2+1).
    \end{equation}
    Nous avons donc équivalence des propositions suivantes :
    \begin{subequations}
        \begin{align}
            p\in\Sigma\\
            \eF_p[X]/(X^2+1)\text{ n'est pas intègre}\\
            X^2+1\text{ n'est pas irréductible dans \( \eF_p\)} \label{EqZkdrvh}\\
            \text{\( X^2+1\) admet une racine dans \( \eF_p\)}\\
            -1\in (\eF_p^*)^2\\
            \exists y\in \eF_p^*\tq y^2=-1.
        \end{align}
    \end{subequations}
    Le point \eqref{EqZkdrvh} vient de la proposition \ref{PropleGdaT}. Maintenant nous utilisons le fait que \( p\) soit un premier impair (parce que le cas de \( p=2\) est déjà complètement traité), donc \( (p-1)/2\in \eN\) et nous avons, pour le \( y\) de la dernière ligne,
    \begin{equation}
        (-1)^{(p-1)/2}=(y^2)^{(p-1)/2}=y^{p-1}=1
    \end{equation}
    parce que dans \( \eF_p\) nous avons \( y^{(p-1)}=1\) par le petit théorème de Fermat (théorème \ref{ThoOPQOiO}). Du coup \( p\) doit vérifier
    \begin{equation}
        1=(-1)^{(p-1)/2},
    \end{equation}
    c'est à dire \( \frac{ p-1 }{2}=0\mod 2\) ou encore \( p=1\mod 4\).

\end{proof}

%---------------------------------------------------------------------------------------------------------------------------
\subsection{Théorème de Burnside}
%---------------------------------------------------------------------------------------------------------------------------

\begin{lemma}       \label{LemwXXzIt}
    Soit \( P\), un polynôme sur \( \eK\). Une racine de \( P\) est une racine simple si et seulement si elle n'est pas racine de \( P'\).
\end{lemma}

\begin{lemma}
    Un endomorphisme \( u\colon E\to E\) est nilpotent si et seulement si \( u^p\) est de trace nulle pour tout \( p\) entre \( 1\) et \( \dim E\).
\end{lemma}

\begin{theorem}     \label{ThoBurnsideoPuCtS}
    Toute représentation d'un groupe d'exposant fini sur \( \eC^n\) a une image finie.
\end{theorem}

Dans le cas d'un groupe abélien, la démonstration est facile. Étant donné que \( G\) est d'exposant fini, il existe \( \alpha\in \eN^*\) tel que \( g^{\alpha}=e\) pour tout \( g\in G\). Le polynôme \( P(X)=X^{\alpha}-1\) est scindé à racines simples. En effet tout polynôme sur \( \eC\) est scindé. Le fait qu'il soit à racines simples provient du lemme \ref{LemwXXzIt} parce que si \( a^{\alpha}=1\), alors il n'est pas possible d'avoir \( \alpha a^{\alpha-1}=0\).

Par ailleurs \( P(g)=0\). Le fait que nous ayons un polynôme annulateur de \( g\) scindé à racines simples implique que \( g\) est diagonalisable (théorème \ref{ThoDigLEQEXR}). Le fait que \( G\) soit abélien montre qu'il existe une base de \( \eC^n\) dans laquelle tous les éléments de \( G\) sont diagonaux. Nous devons par conséquent montrer qu'il existe un nombre fini de matrices de la forme
\begin{equation}
    \begin{pmatrix}
        \lambda_1    &       &       \\
            &   \ddots    &       \\
            &       &   \lambda_n
    \end{pmatrix}.
\end{equation}
Nous savons que \( \lambda_i^{\alpha}=1\) parce que \( g^{\alpha}=\mtu\), par conséquent chacun des \( \lambda_i\) est une racine de l'unité dont il n'existe qu'un nombre fini.

Le résultat reste vrai si \( G\) n'est pas abélien, mais la preuve devient plus compliquée. C'est le \wikipedia{fr}{Théorème_de_Burnside_(problème_de_1902)}{théorème de Burnside}.

%---------------------------------------------------------------------------------------------------------------------------
\subsection{Diagonalisation : cas complexe}
%---------------------------------------------------------------------------------------------------------------------------

Nous considérons maintenant le cas de l'espace \( E=\eC^n\) comme espace vectoriel de dimension \( n\) sur \( \eC\). Il est muni d'une forme sesquilinéaire
\begin{equation}    \label{EqFormSesqQrjyPH}
    \langle x, y\rangle =\sum_{k=1}^nx_k\bar y_k
\end{equation}
pour tout \( x,y\in\eC^n\).
\begin{lemma}
    Pour un opérateur hermitien,
    \begin{enumerate}
        \item
            le spectre est réel,
        \item
            deux vecteurs propres à des valeurs propres distinctes sont orthogonales\footnote{Pour la forme \eqref{EqFormSesqQrjyPH}.}.
    \end{enumerate}
\end{lemma}

\begin{proof}
    Soit \( v\) un vecteur de valeur propre \( \lambda\). Nous avons d'une part 
    \begin{equation}
        \langle Av, A\rangle =\lambda\langle v, v\rangle =\lambda\| v \|^2,
    \end{equation}
    et d'autre part, en utilisant le fait que \( A\) est hermitien,
    \begin{equation}
        \langle Av, v\rangle =\langle v, A^*v\rangle =\langle v, Av\rangle =\bar\lambda\| v \|^2,
    \end{equation}
    par conséquent \( \lambda=\bar\lambda\) parce que \( v\neq 0\).

    Soient \( \lambda_i\) et \( v_i\) (\( i=1,2\)) deux valeurs propres de \( A\) avec leurs vecteurs propres correspondants. Alors d'une part
    \begin{equation}
        \langle Av_1, v_2\rangle =\lambda_1\langle v_1, v_2\rangle ,
    \end{equation}
    et d'autre part
    \begin{equation}
        \langle Av_1, v_2\rangle =\langle v_1, Av_2\rangle =\lambda_2\langle v_1, v_2\rangle .
    \end{equation}
    Nous avons utilisé le fait que \( \lambda_2\) était réel. Par conséquent, soit \( \lambda_1=\lambda_2\), soit \( \langle v_1, v_2\rangle =0\).
\end{proof}

La preuve de Schur provient de \cite{NormHKNPKRqV}.

\begin{lemma}[Lemme de Schur complexe]\index{lemme!Schur complexe}  \label{LemSchurComplHAftTq}
    Si \( A\in\eM(n,\eC)\), il existe une matrice unitaire \( U\) telle que \( UAU^{-1}\) soit triangulaire supérieure.
\end{lemma}

\begin{proof}
    Étant donné que \( \eC\) est algébriquement clos, nous pouvons toujours considérer un vecteur propre \( v_1\) de \( A\), de valeur propre \( \lambda_1\). Nous pouvons utiliser un procédé de Gram-Schmidt pour construire une base orthonormée \( \{ v,u_2,\ldots, u_n \}\) de \( \eR^n\), et la matrice (unitaire)
    \begin{equation}
        Q=\begin{pmatrix}
             \uparrow   &   \uparrow    &       &   \uparrow    \\
             v   &   u_2    &   \cdots    &   u_n    \\ 
             \downarrow   &   \downarrow    &       &   \downarrow
         \end{pmatrix}.
    \end{equation}
    Nous avons \( Q^{-1}AQe_1=Q^{-1} Av=\lambda Q^{-1} v=\lambda e_1\), par conséquent la matrice \( Q^{-1} AQ\) est de la forme
    \begin{equation}
        Q^{-1}AQ=\begin{pmatrix}
            \lambda_1    &   *    \\ 
            0    &   A_1    
        \end{pmatrix}
    \end{equation}
    où \( *\) représente une ligne quelconque et \( A_1\) est une matrice de \( \eM(n-1,\eC)\). Nous pouvons donc répéter le processus sur \( A_1\) et obtenir une matrice triangulaire supérieure (nous utilisons le fait qu'un produit de matrices orthogonales est orthogonales).  
\end{proof}
En particulier les matrices hermitiennes, anti-hermitiennes et unitaires sont trigonables par une matrice unitaire, qui peut être choisie de déterminant \( 1\).


Le théorème suivant et la preuve proviennent de \cite{LecLinAlgAllen}, \wikipedia{en}{Spectral_theorem}{wikipedia} et \href{http://planetmath.org/encyclopedia/TheoremForNormalTriangularMatrices.html}{PlanetMath}.
\begin{theorem}[Théorème spectral pour les matrices normales]\index{théorème!spectral!matrices normales}
    Soit \( A\in\eM(n,\eC)\) une matrice de valeurs propres \( \lambda_1,\ldots, \lambda_n\) (non spécialement distinctes). Alors les conditions suivantes sont équivalentes :
    \begin{enumerate}
        \item   \label{ItemJZhFPSi}
            \( A\) est normale,
        \item   \label{ItemJZhFPSii}
            \( A\) se diagonalise par une matrice unitaire,
        \item
            \( \sum_{i,j=1}^n| A_{ij} |^2=\sum_{j=1}^n| \lambda_j |^2\),
        \item
            il existe une base orthonormale de vecteurs propres de \( A\).
    \end{enumerate}
\end{theorem}

\begin{proof}
    Nous allons nous contenter de prouver \ref{ItemJZhFPSi}\( \Leftrightarrow\)\ref{ItemJZhFPSii}. Soit \( Q\) la matrice unitaire donnée par la décomposition de Schur (lemme \ref{LemSchurComplHAftTq}) : \( A=QTQ^{-1}\). Étant donné que \( A\) est normale nous avons
    \begin{equation}
        QTT^*Q^{-1}=QT^*TQ^{-1},
    \end{equation}
    ce qui montre que \( T\) est également normale. Or une matrice triangulaire supérieure normale est diagonale. En effet nous avons \( T_{ij}=0\) lorsque \( i>j\) et
    \begin{equation}
        (TT^*)_{ii}=(T^*T)_{ii}=\sum_{k=1}^n| T_{ki} |^2=\sum_{k=1}^n| T_{ik} |^2.
    \end{equation}
    Écrivons cela pour \( i=1\) en tenant compte de \( | T_{k1} |^2=0\) pour \( k=2,\ldots, n\),
    \begin{equation}
        | T_{11} |^2=| T_{11} |^2+| T_{12} |^2+\ldots+| T_{1n} |^2,
    \end{equation}
    ce qui implique que \( T_{11}\) est le seul non nul parmi les \( T_{1k}\). En continuant de la sorte avec \( i=2,\ldots, n\) nous trouvons que \( T\) est diagonale.

    Dans l'autre sens, si \( A\) se diagonalise par une matrice unitaire, \( UAU^*=D\), nous avons
    \begin{equation}
        DD^*=UAA^*U^*
    \end{equation}
    et 
    \begin{equation}
        D^*D=UA^*AU^*,
    \end{equation}
    qui ce prouve que \( A\) est normale.
\end{proof}

%---------------------------------------------------------------------------------------------------------------------------
\subsection{Diagonalisation : cas réel}
%---------------------------------------------------------------------------------------------------------------------------

\begin{lemma}[Lemme de Schur réel]\index{lemme!Schur réel}  \label{LemSchureRelnrqfiy}
    Soit \( A\in\eM(n,\eR)\). Il existe une matrice orthogonale \( Q\) telle que \( Q^{-1}AQ\) soit de la forme
    \begin{equation}        \label{EqMtrTSqRTA}
        QAQ^{-1}=\begin{pmatrix}
            \lambda_1    &   *    &   *    &   *    &   *\\  
            0    &   \ddots    &   \ddots    &   \ddots    &   \vdots\\  
            0    &   0    &   \lambda_r    &   *    &   *\\  
            0    &   0    &   0    &   \begin{pmatrix}
                a_1    &   b_1    \\ 
                c_1    &   d_1    
            \end{pmatrix}&   *\\  
            0    &   0    &  0     &   0    &   \begin{pmatrix}
                a_s    &   b_s    \\ 
                c_s    &   d_s    
            \end{pmatrix}
        \end{pmatrix}.
    \end{equation}
    Le déterminant de \( A\) est le produit des déterminants des blocs diagonaux et les valeurs propres de \( A\) sont les \( \lambda_1,\ldots, \lambda_r\) et celles de ces blocs.
\end{lemma}

\begin{proof}
    Si la matrice \( A\) a des valeurs propres réelles, nous procédons comme dans le cas complexe. Cela nous fournit le partie véritablement triangulaire avec les valeurs propres \( \lambda_1,\ldots, \lambda_r\) sur la diagonale. Supposons donc que \( A\) n'a pas de valeurs propres réelles. Soit donc \( \alpha+i\beta \) une valeur propre (\( \beta\neq 0\)) et \( u+iv\) un vecteur propre correspondant où \( u\) et \( v\) sont des vecteurs réels. Nous avons
    \begin{equation}
        Au+iAv=A(u+iv)=(\alpha+i\beta)(u+iv)=\alpha u-\beta v+i(\alpha v+\beta v),
    \end{equation}
    et en égalisant les parties réelles et imaginaires,
    \begin{subequations}
        \begin{align}
            Au&=\alpha u-\beta v\\
            Av&=\alpha v+\beta u.
        \end{align}
    \end{subequations}
    Sur ces relations nous voyons que ni \( u\) ni \( v\) ne sont nuls. De plus \( u\) et \( v\) sont linéairement indépendants (sur \( \eR\)), en effet si \( v=\lambda u\) nous aurions \( Au=\alpha u-\beta\lambda u=(\alpha-\beta\lambda)u\), ce qui serait une valeur propre réelle alors que nous avions supposé avoir déjà épuisé toutes les valeurs propres réelles.

    Étant donné que \( u\) et \( v\) sont deux vecteurs réels non nuls et linéairement indépendants, nous pouvons trouver une base orthonormée \( \{ q_1,q_2 \}\) de \( \Span\{ u,v \}\). Nous pouvons étendre ces deux vecteurs en une base orthonormée \( \{ q_1,q_2,q_3,\ldots, q_n \}\) de \( \eR^n\). Nous considérons à présent la matrice orthogonale dont les colonnes sont formées de ces vecteurs : \( Q=[q_1\,q_2\,\ldots q_n]\).

    L'espace \( \Span\{ e_1,e_2 \}\) est stable par \( Q^{-1} AQ\), en effet nous avons
    \begin{equation}
        Q^{-1} AQe_1=Q^{-1} Aq_1=Q^{-1}(aq_1+bq_2)=ae_1+be_2.
    \end{equation}
    La matrice \( Q^{-1}AQ\) est donc de la forme
    \begin{equation}
        Q^{-1} AQ=\begin{pmatrix}
            \begin{pmatrix}
                \cdot    &   \cdot    \\ 
                \cdot    &   \cdot    
            \end{pmatrix}&   C_1    \\ 
            0    &   A_1    
        \end{pmatrix}
    \end{equation}
    où \( C_1\) est une matrice réelle \( 2\times (n-1)\) quelconque et \( A_1\) est une matrice réelle \( (n-2)\times (n-2)\). Nous pouvons appliquer une récurrence sur la dimension pour poursuivre.

    Notons que si \( A\) n'a pas de valeurs propres réelles, elle est automatiquement d'ordre pair parce que les valeurs propres complexes viennent par couple complexes conjuguées.

    En ce qui concerne les valeurs propres, il est facile de voir en regardant \eqref{EqMtrTSqRTA} que les valeurs propres sont celles des blocs diagonaux. Étant donné que \( QAQ^{-1}\) et \( A\) ont même polynôme caractéristique, ce sont les valeurs propres de \( A\).
\end{proof}

\begin{theorem} \label{ThoeTMXla}
    Le spectre d'une matrice symétrique réelle est réel. Les matrices symétriques sont diagonalisables par une matrice orthogonale.
\end{theorem}

\begin{proof}
    Soit \( A\) une matrice réelle symétrique. Si \( \lambda\) est une valeur propre complexe pour le vecteur propre complexe \( v\), alors d'une part \( \langle Av, v\rangle =\lambda\langle v, v\rangle \) et d'autre part \( \langle Av, v\rangle =\langle v, Av\rangle =\bar\lambda\langle v, v\rangle \). Par conséquent \( \lambda=\bar\lambda\).
    
    Le lemme de Schur réel \ref{LemSchureRelnrqfiy} donne une matrice orthogonale qui trigonalise \( A\). Les valeurs propres étant toutes réelles, la matrice \( QAQ^{-1}\) est même triangulaire (il n'y a pas de blocs dans la forme \eqref{EqMtrTSqRTA}). Prouvons que \( QAQ^{-1}\) est symétrique :
    \begin{equation}
        (QAQ^{-1})^t=(Q^{-1})^tA^tQ^t=QA^tQ^{-1}=QAQ^{-1}
    \end{equation}
    où nous avons utilisé le fait que \( Q\) était orthogonale (\( Q^{-1}=Q^t\)) et que \( A\) était symétrique (\( A^t=A\)). Une matrice triangulaire supérieure symétrique est obligatoirement une matrice diagonale.
\end{proof}

%+++++++++++++++++++++++++++++++++++++++++++++++++++++++++++++++++++++++++++++++++++++++++++++++++++++++++++++++++++++++++++
\section{Espaces de matrices}
%+++++++++++++++++++++++++++++++++++++++++++++++++++++++++++++++++++++++++++++++++++++++++++++++++++++++++++++++++++++++++++

\begin{lemma}
    Les groupes \( \gU(n)\) et \( \SU(n)\) sont connexes par arcs.
\end{lemma}

\begin{proof}
    Soit \( A\), une matrice unitaire et \( Q\) une matrice unitaire qui diagonalise \( A\). Étant donné que les valeurs propres arrivent par paires complexes conjuguées,
    \begin{equation}
        QAQ^{-1}=\begin{pmatrix}
            e^{i\theta_1}    &       &       &       &   \\  
            &    e^{-i\theta_1}    &       &       &   \\  
            &       &    \ddots    &       &   \\  
            &       &       &    e^{i\theta_r}    &   \\  
            &       &       &       &        e^{-i\theta_r}
        \end{pmatrix}.
    \end{equation}
    Le chemin \( U(t)\) obtenu en remplaçant \( \theta_i\) par \( t\theta_i\) avec \( t\in\mathopen[ 0 , 1 \mathclose]\) joint \( QAQ^{-1}\) à l'identité. Par conséquent \( Q^{-1}U(t)Q\) joint \( A\) à l'unité.
\end{proof}

\begin{theorem}
    Les matrices \wikipedia{fr}{Endomorphisme_normal}{normales} forment un espace connexe par arc.
\end{theorem}

\begin{proof}
    Soit \( A\) une matrice normale, et \( U\) une matrice unitaire qui diagonalise \( A\). Nous considérons \( U(t)\), un chemin qui joint \( \mtu\) à \( U\) dans \( \gU(n)\). Pour chaque \( t\), la matrice
    \begin{equation}
        A(t)=U(t)^{-1} AU(t)
    \end{equation}
    est normale. Nous avons donc trouvé un chemin dans les matrices normales qui joint \( A\) à une matrice diagonale. Il est à présent facile de la joindre à l'identité.

    Toutes les matrices normales étant connexes à l'identité, l'ensemble des matrices normales est connexe.
\end{proof}

\begin{proposition}     \label{PropDigDensVxzPuo}
    Les matrices diagonalisables sont denses dans \( \eM(n,\eC)\).
\end{proposition}

\begin{proof}
    D'après le lemme de Schur \ref{LemSchurComplHAftTq}, une matrice de \( \eM(n,\eC)\) est de la forme
    \begin{equation}
        A=Q\begin{pmatrix}
            \lambda_1    &   *    &   *    \\
              0  &   \ddots    &   *    \\
            0    &   0    &   \lambda_n
        \end{pmatrix}Q^{-1}.
    \end{equation}
    Les valeurs propres sont sur la diagonale. La matrice est diagonalisable si les éléments de la diagonales sont tous différents. Il suffit maintenant de considérer \( n\) suites \( (\epsilon^{(r)}_k)_{k\in\eN}\) convergentes vers zéro telles que pour chaque \( k\) les nombres \( \lambda_r+\epsilon^{(r)}_k\) soient tous différents. La suite de matrices
    \begin{equation}
        A_k=Q\begin{pmatrix}
            \lambda_1+\epsilon^{(1)}_k    &   *    &   *    \\
              0  &   \ddots    &   *    \\
              0    &   0    &   \lambda_n+\epsilon^{(n)}_k
        \end{pmatrix}Q^{-1}.
    \end{equation}
    est alors diagonalisable pour tout \( k\) et nous avons \( \lim_{k\to \infty} A_k=A\).
\end{proof}

\begin{proposition}
    Si \( A\in\eM(n,\eC)\) alors
    \begin{equation}
        e^{\tr(A)}=\det( e^{A}).
    \end{equation}
\end{proposition}

\begin{proof}
    Le résultat est un simple calcul pour les matrices diagonalisable. Si \( A\) n'est pas diagonalisable, nous considérons une suite de matrices diagonalisables \( A_k\) dont la limite est \( A\) (proposition \ref{PropDigDensVxzPuo}). La suite
    \begin{equation}
        a_k= e^{\tr(A_k)}
    \end{equation}
    converge vers \(  e^{\tr(A)}\) tandis que la suite 
    \begin{equation}
        b_k=\det( e^{A_k})
    \end{equation}
    converge vers \( \det( e^{A})\). Mais nous avons \( a_k=b_k\) pour tout \( k\); les limites sont donc égales.
\end{proof}

\begin{lemma}
    Les formes linéaires sur \( \eM(n,\eR)\) sont les applications de la forme
    \begin{equation}
        \begin{aligned}
            f_A\colon \eM_n(\eR)&\to \eR \\
            M&\mapsto \tr(AM). 
        \end{aligned}
    \end{equation}
\end{lemma}

\begin{proof}
    Nous considérons l'application
    \begin{equation}
        \begin{aligned}
            f\colon \eM(n,\eR)&\to \eM(n,\eR)' \\
            A&\mapsto f_A 
        \end{aligned}
    \end{equation}
    et nous voulons prouver que c'est une bijection. Étant donné que nous sommes en dimension finie, nous avons égalité des dimensions de \( \eM_n(\eR)\) et \( \eM_n(\eR)'\), et il suffit de prouver que \( f\) est injective. Soit donc \( A\) telle que \( f_A=0\). Nous l'appliquons à la matrice \( (E_{ij})_{kl}=\delta_{ik}\delta_{jl}\) :
    \begin{subequations}
        \begin{align}
            0&=f_A(E_{ij})\\
            &=\sum_{k}(AE_{ij})_{kk}\\
            &=\sum_{kl}A_{kl}\delta_{il}\delta_{jk}\\
            &=A_{ij}.
        \end{align}
    \end{subequations}
    Donc \( A=0\).
\end{proof}

\begin{theorem}
    L'enveloppe convexe de \( O(n)\) dans \( \eM_n(\eR)\) est la boule unité pour la norme induite de \( \| . \|_2\) sur \( \eR^n\).
\end{theorem}

%+++++++++++++++++++++++++++++++++++++++++++++++++++++++++++++++++++++++++++++++++++++++++++++++++++++++++++++++++++++++++++
\section{Sous espaces caractéristiques}
%+++++++++++++++++++++++++++++++++++++++++++++++++++++++++++++++++++++++++++++++++++++++++++++++++++++++++++++++++++++++++++

Sources : \cite{MneimneReduct} et \wikipedia{fr}{Décomposition_de_Dunford}{divers articles sur wikipédia}.

Lorsqu'un opérateur n'est pas diagonalisable, les valeurs propres jouent quand même un rôle important.

Soit \( E\) un \( \eK\)-espace vectoriel et \( f\in\End(E)\). Pour \( \lambda\in \eK\) nous définissons
\begin{equation}
    F_{\lambda}(f)=\{ v\in E\tq (f-\lambda\mtu)^nv=0, n\in\eN \}.
\end{equation}
C'est l'ensemble de nilpotence de l'opérateur \( f-\lambda\mtu\).

\begin{lemma}
    L'ensemble \( F_{\lambda}(f)\) est non vide si et seulement si \( \lambda\) est une valeur propre de \( f\). L'espace \( F_{\lambda}(f)\) est invariant sous \( f\).
\end{lemma}

\begin{proof}
    Si \( F_{\lambda}(f)\) est non vide, nous considérons \( v\in F_{\lambda}(f)\) et \( n\) le plus petit entier non nul tel que \( (f-\lambda)^nv=0\). Alors \( (f-\lambda)^{n-1}v\) est un vecteur propre de \( f\) pour la valeur propre \( \lambda\). Inversement si \( v\) est une valeur propre de \( f\) pour la valeur propre \( \lambda\), alors \( v\in F_{\lambda}(f)\).

    En ce qui concerne l'invariance, remarquons que \( f\) commute avec \( f-\lambda\mtu\). Si \( x\in F_{\lambda}(f)\) il existe \( n\) tel que \( (f-\lambda\mtu)^nx=0\). Nous avons aussi
    \begin{equation}
        (f-\lambda\mtu)^nf(x)=f\big( (f-\lambda\mtu)^nx \big)=0,
    \end{equation}
    par conséquent \( f(x)\in F_{\lambda}(f)\).
\end{proof}

\begin{remark}
    Toute matrice sur \( \eC\) n'est pas diagonalisable. Considérons en effet l'endomorphisme \( f\) donné par la matrice
    \begin{equation}
        \begin{pmatrix}
            a&    \alpha    &   \beta    \\
            0    &   a    &   \gamma    \\
            0    &   0    &   b
        \end{pmatrix}
    \end{equation}
    où \( a\neq b\), \( \alpha\neq 0\), \( \beta\) et \( \gamma\) sont des nombres complexes quelconques.
    Son polynôme caractéristique est 
    \begin{equation}
        \chi_f(\lambda)=(a-\lambda)^2(b-\lambda)
    \end{equation}
    de telle façon à ce que les valeurs propres soient \( a\) et \( b\). Nous trouvons les vecteurs propres pour la valeur \( a\) en résolvant
    \begin{equation}
        \begin{pmatrix}
            a    &   \alpha    &   \beta    \\
            0    &   a    &   \gamma    \\
            0    &   0    &   b
        \end{pmatrix}\begin{pmatrix}
            x    \\ 
            y    \\ 
            z    
        \end{pmatrix}=\begin{pmatrix}
            ax    \\ 
            ay    \\ 
            az    
        \end{pmatrix}.
    \end{equation}
    L'espace propre \( E_a(f)\) est réduit à une seule dimension générée par \( (1,0,0)\). De la même façon l'espace propre correspondant à la valeur propre \( b\) est donné par 
    \begin{equation}
        \begin{pmatrix}
            \frac{1}{ b-a }\left( \beta+\frac{ \alpha\gamma }{ b-a } \right)    \\ 
            \frac{ \gamma }{ b-a }    \\ 
            1    
        \end{pmatrix}.
    \end{equation}
    Il n'y a donc pas trois vecteurs propres linéairement indépendants, et l'opérateur \( f\) n'est pas diagonalisable.

    Par contre nous pouvons voir que
    \begin{equation}
        (f-\alpha\mtu)^2\begin{pmatrix}
             0   \\ 
            1    \\ 
            0    
        \end{pmatrix}=
        \begin{pmatrix}
            a    &   \alpha    &   \beta    \\
            0    &   a    &   \gamma    \\
            0    &   0    &   b
        \end{pmatrix}
        \begin{pmatrix}
            \alpha    \\ 
            0    \\ 
            0    
        \end{pmatrix}-\begin{pmatrix}
            a\alpha    \\ 
            0    \\ 
            0    
        \end{pmatrix}=\begin{pmatrix}
            0    \\ 
            0    \\ 
            0    
        \end{pmatrix},
    \end{equation}
    de telle sorte que le vecteur \( (0,1,0)\) soit également dans l'espace caractéristique \( F_a(f)\).

    Dans cet exemple, la multiplicité algébrique de la racine \( a\) du polynôme caractéristique vaut \( 2\) tandis que sa multiplicité géométrique vaut seulement \( 1\).
\end{remark}

Le théorème suivant est aussi appelé le théorème de \defe{décomposition primaire}{décomposition!primaire}.


\begin{theorem}[Théorème spectral, décomposition primaire]\index{théorème!spectral}     \label{ThoSpectraluRMLok}
    Soit \( E\) espace vectoriel de dimension finie sur le corps algébriquement clos \( \eK\) et \( f\in\End(E)\). Alors
    \begin{equation}
        E=F_{\lambda_1}(f)\oplus\ldots\oplus F_{\lambda_k}(f)
    \end{equation}
    où la somme est sur les valeurs propres distinctes de \( f\).

    Les projecteurs sur les espaces caractéristique forment un système complet et orthogonal.
\end{theorem}

\begin{proof}
    Soit \( P\) le polynôme caractéristique de \( u\) et une décomposition
    \begin{equation}
        P=(u-\lambda_1)^{\alpha_1}\ldots(u-\lambda_r)^{\alpha_r}
    \end{equation}
    en facteurs irréductibles. La le théorème de noyaux (\ref{ThoDecompNoyayzzMWod}) nous avons
    \begin{equation}        \label{EqDeFVSaYv}
        E=\ker(u-\lambda_1)^{\alpha_1}\oplus\ldots\oplus\ker(u-\lambda_r)^{\alpha_r}.
    \end{equation}
    Les projecteurs sont des polynômes en \( u\) et forment un système orthogonal. Il nous reste à prouver que \( \ker(u-\lambda_i)^{\alpha_i}=F_{\lambda_i}(u)\). L'inclusion
    \begin{equation}    \label{EqzmNxPi}
        \ker(u-\lambda_i)^{\alpha_i}\subset F_{\lambda_i}(u)
    \end{equation}
    est évidente. Nous devons montrer l'inclusion inverse. Prouvons que la somme des \( F_{\lambda_i}(u)\) est directe. Si \( v\in F_{\lambda_i}(u)\cap F_{\lambda_j}(u)\), alors il existe \( v_1=(u-\lambda_i)^nv\neq 0\) avec \( (u-\lambda_i)v_1=0\). Étant donné que \( (u-\lambda_i)\) commute avec \( (u-\lambda_j)\), ce \( v_1\) est encore dans \( F_{\lambda_j}(u)\) et par conséquent il existe \( w=(u-\lambda_j)^mv_1\) non nul tel que 
    \begin{subequations}
        \begin{numcases}{}
            (u-\lambda_i)w=0\\
            (u-\lambda_j)w=0.
        \end{numcases}
    \end{subequations}
    Ce \( w\) serait donc un vecteur propre simultané pour les valeurs propres \( \lambda_i\) et \( \lambda_j\), ce qui est impossible parce que les espaces propres sont linéairement indépendants. Les espaces \( F_{\lambda_i}\) sont donc en somme directe et
    \begin{equation}
        \sum_i\dim F_{\lambda_i}(u)\leq \dim E.
    \end{equation}
    En tenant compte de l'inclusion \eqref{EqzmNxPi} nous avons même
    \begin{equation}
        \dim E=\sum_i\dim\ker(u-\lambda_i)^{\alpha_i}\leq\sum_i F_{\lambda_i}(u)\leq \dim E.
    \end{equation}
    Par conséquent nous avons \( \dim\ker(u-\lambda_i)^{\alpha_i}=\dim F_{\lambda_i}(u)\) et l'égalité des deux espaces.
    
\end{proof}

Le théorème suivant généralise le théorème de diagonalisabilité \ref{ThoDigLEQEXR} au cas où le polynôme minimum est seulement scindé.

\begin{probleme}
    \begin{enumerate}
\item 
    Dans le cas où le corps n'est pas algébriquement clos, il paraît qu'il faut remplacer «diagonalisable» par «semi-simple».
    \end{enumerate}
\end{probleme}

\begin{definition}
    Un endomorphisme d'un espace vectoriel est \defe{semi-simple}{semi-simple!endomorphisme} si tout sous-espace stable par \( u\) possède un supplémentaire stable.
\end{definition}
Si l'espace vectoriel est sur un corps algébriquement clos, alors les endomorphismes semi-simples sont les endomorphismes diagonaux.


\begin{theorem}[Décomposition de Dunford]\index{décomposition!Dunford}\index{Dunford!décomposition} \label{ThoRURcpW}
    Soit \( E\) un espace vectoriel sur le corps algébriquement clos \( \eK\) et \( u\in\End(E)\) un endomorphisme de \( E\). Alors \( u\) se décompose de façon unique sous la forme
    \begin{equation}
        u=s+n
    \end{equation}
    où \( s\) est diagonalisable, \( n\) est nilpotent et \( [s,n]=0\).

    De plus \( s\) et \( n\) sont des polynômes en \( u\) et commutent avec \( u\).
\end{theorem}

\begin{proof}
    Le théorème spectral \ref{ThoSpectraluRMLok} nous indique que
    \begin{equation}
        E=\bigoplus_iF_{\lambda_i}(f).
    \end{equation}
    Nous considérons l'endomorphisme \( s\) de \( E\) qui consiste à dilater d'un facteur \( \lambda\) l'espace caractéristique \( F_{\lambda}(f)\) :
    \begin{equation}
        s=\sum_i\lambda_ip_i
    \end{equation}
    où \( p_i\colon E\to F_{\lambda_i}(u)\) est la projection de \( E\) sur \( F_{\lambda_i}(u)\).

    Nous allons prouver que \( [s,f]=0\) et \( n=f-s\) est nilpotent. Cela impliquera que \( [s,n]=0\).

    Si \( x\in F_{\lambda}(f)\), alors nous avons \( sf(x)=\lambda f(x)\) parce que \( f(x)\in F_{\lambda}(f)\) tandis que \( fs(x)=f(\lambda x)=\lambda f(x)\). Par conséquent \( f\) commute avec \( s\).

    Pour montrer que \( f-s\) est nilpotent, nous en considérons la restriction
    \begin{equation}
        f-s\colon F_{\lambda}(f)\to F_{\lambda}(f).
    \end{equation}
    Cet opérateur est égal à \( f-\lambda\mtu\) et est par conséquent nilpotent.

    Prouvons à présent l'unicité. Soit \( u=s'+n'\) une autre décomposition qui satisfait aux conditions : \( s'\) est diagonalisable, \( n'\) est nilpotent et \( [n',s']=0\). Commençons par prouver que \( s'\) et \( n'\) commutent avec \( u\). En multipliant \( u=s'+n'\) par \( s'\) nous avons
    \begin{equation}
        s'u=s'^2+s'n'=s'^2+n's'=(s'+n')s'=us',
    \end{equation}
    par conséquent \( [u,s']=0\). Nous faisons la même chose avec \( n'\) pour trouver \( [u,n']=0\). Notons que pour obtenir ce résultat nous avons utilisé le fait que \( n'\) et \( s'\) commutent, mais pas leur propriétés de nilpotence et de diagonalisibilité.
    
    
    Si \( s'+n'=s+n\) est une autre décomposition, \( s'\) et \( n'\) commutent avec \( u\), et par conséquent avec tous les polynômes en \( u\). Ils commutent en particulier avec \( n\) et \( s\). Les endomorphismes \( s\) et \( s'\) sont alors deux endomorphismes diagonalisables qui commutent. Par la proposition \ref{PropGqhAMei}, ils sont simultanément diagonalisables. Dans la base de simultanée diagonalisation, la matrice de l'opérateur \( s'-s=n-n'\) est donc diagonale. Mais \( n-n'\) est également nilpotent, en effet si \( A\) et \( B\) sont deux opérateurs nilpotents,
    \begin{equation}
        (A+B)^n=\sum_{k=0}^n\binom{k}{n}A^kB^{n-k}.
    \end{equation}
    Si \( n\) est assez grand, au moins un parmi \( A^k\) ou \( B^{n-k}\) est nul.

    Maintenant que \( n-n'\) est diagonal et nilpotent, il est nul et \( n=n'\). Nous avons alors immédiatement aussi \( s=s'\).
\end{proof}


%---------------------------------------------------------------------------------------------------------------------------
\subsection{Calcul de l'exponentielle d'une matrice}
%---------------------------------------------------------------------------------------------------------------------------

Nous reprenons l'exemple de \cite{MneimneReduct}. Soit \( A\) une matrice dont le polynôme minimum s'écrit
\begin{equation}
    P(X)=(X-1)^2(X-2).
\end{equation}
Par le théorème \ref{ThoDecompNoyayzzMWod} de décomposition des noyaux nous avons
\begin{equation}
    E=\ker(A-1)^2\oplus\ker(A-2).
\end{equation}
En suivant les notations de ce théorème nous avons \( P_1(X)=(X-1)^2\), \( P_2(X)=X-2\) et
\begin{subequations}
    \begin{align}
        Q_1(X)&=X-2\\
        Q_2(X)&=(X-1)^2.
    \end{align}
\end{subequations}
Les polynômes \( R_i\) dont l'existence est assurée par le théorème de Bézout sont
\begin{equation}
    \begin{aligned}[]
        R_1(X)&=-X\\
        R_2(X)&=1.
    \end{aligned}
\end{equation}
Nous avons
\begin{equation}
    R_1Q_1+R_2Q_2=1.
\end{equation}
Le projecteur \( p_i\) sur \( \ker P_i\) est \( R_iQ_i\) :
\begin{equation}
    \begin{aligned}[]
        p_1&=-A(A-2)=\pr_{\ker(u-1)^2}\\
        p_2&=(A-1)^2=\pr_{\ker(u-2)}.
    \end{aligned}
\end{equation}
Passons maintenant au calcul de l'exponentielle. Nous avons évidemment
\begin{equation}
    e^A=e^Ap_1+e^Ap_2.
\end{equation}
Étant donné que \( p_1\) est le projecteur sur le noyau de \( (A-1)^2\), nous avons
\begin{equation}
    e^Ap_1=ee^{A-1}p_1=ep_1+e(u-1)1=ep_1=-Ae(A-2).
\end{equation}
En effet \( e^{A-1}p_1=\sum_{k=0}^{\infty}(A-1)^k\circ p_1\). De la même façon nous avons
\begin{equation}
    e^Ap_2=e^2e^{A-2}p_2=e^2p_2=e^2(A-1)^2.
\end{equation}
Au final,
\begin{equation}
    e^A=-Ae(A-2)+e^2(A-1)^2.
\end{equation}

\begin{theorem}
    Soit une matrice \( A\in \eM(n,\eC)\). On a que la suite \( (A^kx)\) tends vers zéro pour tout \( x\) si et seulement si \( \rho(A)<1\) où \( \rho(A)\)\index{rayon!spectral} est le rayon spectral de $A$
\end{theorem}

\begin{proof}
    Dans le sens direct, il suffit de prendre comme \( x\), un vecteur propre de \( A\). Dans ce cas nous avons \( A^kx=\lambda^kx\). Mais \( \lambda^kx\) ne tend vers zéro que si \( \lambda<1\). Donc toute les valeurs propres de \( A\) doivent être plus petite que \( 1\) et \( \rho(A)<1\).

    Pour l'autre sens nous utilisons la décomposition de Dunford (théorème \ref{ThoRURcpW}) : il existe une matrice inversible \( P\) telle que
    \begin{equation}
        A=P^{-1}(D+N)P
    \end{equation}
    où \( D\) est diagonale, \( N\) est nilpotente et \( [D,N]=0\). Étant donné que \( D+N\) est triangulaire, son polynôme caractéristique que
    \begin{equation}
        \chi_{D+N}(\lambda)=\prod_i D_{ii}-\lambda.
    \end{equation}
    Par similitude, c'est le même polynôme caractéristique que celui de \( A\) et nous savons alors que la diagonale de \( D\) contient les valeurs propres de \( A\).

    Par ailleurs nous avons
    \begin{subequations}
        \begin{align}
            A^k&=P^{-1}(D+N)^kP\\
            &=P^{-1}\sum_{j=0}^k{j\choose k}D^{j-k}N^jP\\
            &==P^{-1}\sum_{j=0}^{n-1}{j\choose k}D^{j-k}N^jP
        \end{align}
    \end{subequations}
    où nous avons utilité le fait que \( D\) et \( N\) commutent ainsi que \( N^{n-1}=0\) parce que \( N\) est nilpotente. Nous utilisons la norme matricielle usuelle, pour laquelle \( \| D \|=\rho(D)=\rho(A)\). Nous avons alors
    \begin{equation}
        \| (D+N)^k \|\leq \sum_{j=0}^k{j\choose k}\rho(D)^{k-j}\| N \|^j.
    \end{equation}
    Du coup si \( \rho(D)<1\) alors \( \| (D+N)^k \|\to 0\) (et c'est même un si et seulement si).
\end{proof}


%---------------------------------------------------------------------------------------------------------------------------
\subsection{Valeurs singulières}
%---------------------------------------------------------------------------------------------------------------------------

\begin{definition}
    Soit \( M\) une matrice \( m\times n\) sur \( \eK\) (\( \eK\) est \( \eR\) ou \( \eC\)). Un nombre réel \( \sigma\) est une \defe{valeur singulière}{valeur!singulière} de \( M\) si il existent des vecteurs unitaires \( u\in \eK^m\), \( v\in \eK^n\) tels que
    \begin{subequations}
        \begin{align}
            Mv&=\sigma u\\
            M^*u&=\sigma v.
        \end{align}
    \end{subequations}
\end{definition}

\begin{theorem}[Décomposition en valeurs singulières]
    Soit \( M\in \eM(m\times n,\eK)\) où \( \eK=\eR,\eC\). Alors \( M\) se décompose en
    \begin{equation}
        M=ADB
    \end{equation}
    où
    il existe deux matrices unitaires \( A\in \gU(m\times m)\), \( B\in \gU(n\times n)\) et une matrice (pseudo)diagonale \( D\in \eM(m\times n)\) tels que
    \begin{enumerate}
        \item 
            \( A\in\gU(m\times m)\), \( B\in\gU(n\times n)\) sont deux matrices unitaires;,
        \item
            \( D\) est (pseudo)diagonale,
        \item
            les éléments diagonaux de \( D\) sont les valeurs singulières de \( M\),
        \item
            le nombre d'éléments non nuls sur la diagonale de \( D\) est le rang de \( M\).
    \end{enumerate}
\end{theorem}

\begin{corollary}
    Soit \( M\in \eM(n,\eC)\). Il existe un isomorphisme \( f\colon \eC^n\to \eC^n\) tel que \( fM\) soit autoadjoint.
\end{corollary}

\begin{proof}
    Si \( M=ADB\) est la décomposition de \( M\) en valeurs singulières, alors nous pouvons prendre \( f=\overline{ B }^tA^{-1}\) qui est une matrice inversible. Pour la vérification que ce \( f\) répond bien à la question, ne pas oublier que \( D\) est réelle, même si \( M\) ne l'est pas.
\end{proof}

%+++++++++++++++++++++++++++++++++++++++++++++++++++++++++++++++++++++++++++++++++++++++++++++++++++++++++++++++++++++++++++
\section{Matrice compagnon et endomorphismes cycliques}
%+++++++++++++++++++++++++++++++++++++++++++++++++++++++++++++++++++++++++++++++++++++++++++++++++++++++++++++++++++++++++++

%---------------------------------------------------------------------------------------------------------------------------
\subsection{Matrice compagnon}
%---------------------------------------------------------------------------------------------------------------------------

Soit le polynôme \( P=X^n-a_{n-1}X^{n-1}-\ldots-a_1X-a_0\) dans \( \eK[X]\). La \defe{matrice compagnon}{matrice!compagnon} de \( P\) est la matrice\nomenclature[A]{\( C(P)\)}{matrice compagnon} donnée par
\begin{equation}
    C(P)=\begin{pmatrix}
        0    &   \cdots    &   \cdots    &   0    &   a_0\\  
        1    &   0    &       &   \vdots    &   a_1\\  
        0    &   \ddots    &   \ddots    &   \vdots    &   \vdots\\  
        \vdots    &   \ddots    &   \ddots    &   0    &   a_{n-2}\\  
        0    &   \cdots    &   0    &   1    &   a_{n-1}    
    \end{pmatrix}
\end{equation}
si \( n\geq 2\) et par \( (a_0)\) si \( n=1\). Si \( f\) est l'endomorphisme associé à la matrice \( C(P)\) nous avons
\begin{equation}
    f(e_i)=\begin{cases}
        e_{i+1}    &   \text{si \( i<n\)}\\
        (a_0,\ldots, a_{n-1})    &    \text{si \( i=n\)}.
    \end{cases}
\end{equation}
Cet endomorphisme est conçu pour vérifier \( P(f)e_1=0\).

\begin{lemma}[\cite{RapportArgreg2011}] \label{LemkVNisk}
    Soit \( P\), un polynôme sur un corps commutatif \( \eK\). Si \( f\) est l'endomorphisme associé à la matrice compagnon de \( P\), alors \( P\) est la polynôme caractéristique de \( f\). En d'autres termes, \( P=\chi_f\).
\end{lemma}

\begin{proof}
    La propriété \( P(f)e_1=0\) nous indique que le polynôme minimal ponctuel de \( f\) en \( e_1\) divise \( P\). L'ensemble des puissances de \( f\) appliquées à \( e_1\), \( \big( f^i(e_1) \big)_{i=1,\ldots, n-1}\) est libre, donc le polynôme minimal ponctuel en \( e_1\) est de degré \( n\) au minimum. En reprenant les notations du théorème \ref{ThoCCHkoU}, nous avons \( I_{e_1}=(P)\) parce que \( P\) est de degré minimum dans \( I_{e_1}\) et \( \chi_f\in I_{e_1}\).

    Donc \( P\) divise \( \chi_f\) et est de degré égal à celui de \( \chi_f\). Étant donné qu'ils sont tous deux unitaires, ils sont égaux.
\end{proof}

\begin{remark}  \label{RemmQjZOA}
    Les matrices compagnons ne sont pas les seules dont le polynôme caractéristique est égal au polynôme minimal. En fait les matrices sont le polynôme caractéristique est égale au polynôme minimal sont denses dans les matrices. En effet une matrice dont le polynôme minimal n'est pas égal au polynôme caractéristique a un polynôme caractéristique avec une racine double. Il est possible, en modifiant arbitrairement peu la matrice de séparer la racine double en deux racines distinctes.
\end{remark}

\begin{definition}[Matrices, endomorphismes et vecteurs cycliques]
    Une matrice est \defe{cyclique}{cyclique!matrice}\index{matrice!cyclique} si elle est semblable à une matrice compagnon. Un endomorphisme \( f\colon E\to E\) est \defe{cyclique}{cyclique!endomorphisme}\index{endomorphisme!cyclique} si il existe un vecteur \( x\in E\) tel que \( \{ f^k(x)\tq k=1,\ldots, n-1 \}\) est une base de \( E\). Un vecteur ayant cette propriété est un \defe{vecteur cyclique}{vecteur!cyclique} pour \( f\).
\end{definition}

\begin{lemma}
    Un endomorphisme est cyclique si et seulement si sa matrice associée est cyclique.
\end{lemma}

\begin{lemma}   \label{LemSGmdnE}
    Une matrice est cyclique si et seulement si ses polynômes minimal et caractéristiques coïncident.
\end{lemma}

\begin{lemma}   \label{LemAGZNNa}
    Si \( f\colon E\to E\) est un endomorphisme cyclique et si \( y\) est un vecteur cyclique de \( f\), alors le polynôme minimal de \( f\) est égal au polynôme minimal de \( f\) au point \( y\) : \( \mu_{f}=\mu_{f,y}\).
\end{lemma}

\begin{proof}
    Montrons que \( \mu_{f,y}\) est un polynôme annulateur de \( f\), ce qui prouvera que \( \mu(f)\) divise \( \mu_{f,y}\). Étant donné que \( y\) est cyclique, tout élément de \( E\) s'écrit sous la forme \( x=Q(f)y\). Prenons un polynôme \( P\) annulateur de \( f\) en \( y\) : \( P(f)y=0\). Nous montrons que \( P\) est alors un polynôme annulateur de \( f\). En effet, nous avons
    \begin{equation}
        P(f)x=\big( P(f)\circ Q(f) \big)y=\big( Q(f)\circ P(f) \big)y=0
    \end{equation}
    où nous avons utilisé le lemme \ref{LemQWvhYb}.
\end{proof}

%---------------------------------------------------------------------------------------------------------------------------
\subsection{Réduction de Frobenius}
%---------------------------------------------------------------------------------------------------------------------------

\begin{theorem}[Réduction de Frobenius \cite{AutourFrobCompa,Vialivs}]      \index{réduction!Frobénius}\index{Frobénius!réduction}
    Soit \( E\), un \( \eK\)-espace vectoriel où \( \eK\) est \( \eR\) ou \( \eC\), et \( f\in \End(E)\). Alors il existe une suite de sous-espaces \( E_1,\ldots, E_r\) stables par \( f\) tels que
    \begin{enumerate}
        \item   \label{ItemmpwjnSs}
            \( E=\bigoplus_{i=1}^rE_i\);
        \item
            pour chaque \( E_i\), l'endomorphisme restreint \( f_i=f|_{E_i}\) est cyclique;
        \item
            si \( \mu_i\) est le polynôme minimal de \( f_i\) alors \( \mu_{i+1}\) divise \( \mu_i\);
    \end{enumerate}
    Une telle décomposition vérifie automatiquement \( \mu_1=\mu_f\) et \( \mu_1\cdots \mu_r=\chi_f\), et la suite \( (\mu_i)_{i=1,\ldots, r}\) ne dépend que de \( f\) et non du choix de la décomposition du point \ref{ItemmpwjnSs}.
\end{theorem}

Les polynômes \( \mu_i\) sont les \defe{invariants de similitude}{invariant!de similitude} de l'endomorphisme \( f\).

\begin{proof}
    Nous commençons par montrer que si une telle décomposition existe, alors
    \begin{subequations}    \label{subEqzcGouz}
        \begin{align}
            \chi_f=\prod_{i=1}^r\mu_i  \label{EqTaxsvb}\\
            \mu_f=\mu_1
        \end{align}
    \end{subequations}
    où \( \chi_f\) est le polynôme caractéristique de \( f\) et \( \mu_f\) est le polynôme minimal\footnote{Cette partie de la preuve provient de \cite{MoncetIVS}.}. D'abord le polynôme caractéristique de \( f\) devra être égal au produit des polynômes caractéristique des \( f|_{E_i}\), mais ces derniers endomorphismes étant cycliques, leurs polynôme caractéristiques sont égaux à leurs polynômes minimaux (lemme \ref{LemSGmdnE}). Cela prouve l'égalité \eqref{EqTaxsvb}. Ensuite tous les \( \mu_i\) doivent diviser le polynôme minimal, donc \( \ppcm(\mu_1,\ldots, \mu_r)\) divise \(\mu_f\). Cependant le polynôme minimal ne doit contenir une et une seule fois chacun des facteurs irréductibles du polynôme caractéristique, et chacun de ces facteurs sont dans les polynômes \( \mu_i\). Par conséquent \( \ppcm(\mu_1,\ldots, \mu_r)=\mu_f\). Mais par ailleurs \( \mu_1=\ppcm(\mu_1,\ldots, \mu_r)\), donc \( \mu_1=\mu_f\).
    
    Mais le produit des \( \mu_i\) est le polynôme caractéristique, donc tous les facteurs irréductibles du polynôme minimal sont dans les \( \mu_i\); cela signifie que \( \mu_f=\ppcm(\mu_1,\ldots, \mu_r)\).

    Soit \( d\), le degré du polynôme minimal de \( f\) et \( y\in E\) tel que \( \mu_f=\mu_{f,y}\) (voir lemme \ref{LemSYsJJj}). Le plus petit espace stable sous \( f\) contenant \( y\) est
    \begin{equation}
        E_y=\Span\{ y,f(y),\ldots, f^{d-1}(y) \}.
    \end{equation}
    Nous notons \( e_i=f^{i-1}(y)\). Notons que les vecteurs donnés forment bien une base de \( E_y\) parce que si les \( e_i\) n'était pas linéairement indépendants, alors soit \( \sum_ka_ke_k=0\). Donc ce cas nous aurions
    \begin{equation}
        \big( \sum_ka_kX^k \big)(f)y=0,
    \end{equation}
    ce qui contredirait la minimalité de \( \mu_{f,y}\).

    La difficulté du théorème est de trouver un complément de \( E_y\) qui soit également stable sous \( f\). Nous commençons par étendre\footnote{Pour autant que j'aie compris, cette extension manque dans \cite{AutourFrobCompa}. Corrigez moi si je me trompe.} \( \{ e_1,\ldots, e_d \}\) en une base \( \{ e_1,\ldots, e_n \}\) de \( E\). Ensuite nous allons montrer que
    \begin{equation}
        E=E_y\oplus F
    \end{equation}
    avec
    \begin{equation}
        F=\{ x\in E\tq  e^*_d\big( f^k(x) \big)=0\forall k\in \eN \}.
    \end{equation}
    Par construction, \( F\) est invariant sous \( f\). Montrons pour commencer que \( E_y\cap F=\{ 0 \}\). Un élément de \( E_y\) s'écrit
    \begin{equation}
        z=a_1e_1+\ldots +a_ke_k
    \end{equation}
    avec \( k\leq d\). Étant donné que \( f\) décale les vecteurs de base, nous avons \( e^*_d\big( f^{d-k}(z) \big)=a_k\). Du coup \( z\in F\) si et seulement si \( a_1=\ldots=a_d=0\), c'est à dire que \( E_y\cap F=\{ 0 \}\).

    Nous montrons maintenant que \( \dim F=n-d\). Pour cela nous considérons l'application
    \begin{equation}
        \begin{aligned}
            T\colon \eK[F]&\to E^* \\
            g&\mapsto e^*_d\circ g. 
        \end{aligned}
    \end{equation}
    Cette application est injective. En effet un élément général de \( \eK[f]\) est
    \begin{equation}
        g=a_1\id+a_2f+\ldots +a_pf^{p-1}
    \end{equation}
    avec \( p\leq d\). Si \( T(g)=0\), alors nous avons en particulier
    \begin{equation}
        0=T(g)e_{_d-p+1}=e^*_d(a_1e_{d-p+1}+a_2e_{d-p+2}+\ldots +a_pe_d)=a_p.
    \end{equation}
    Donc \( a_p=0\) et en appliquant maintenant \( T(g)\) à \( e_{d-p}\) nous obtenons \( a_{p-1}=0\). Au final nous trouvons que \( g=0\) et donc que \( T\) est injective.

    Étant donné que \( \dim\eK[f]=d\) et que \( T\) est injective, \( \dim\Image(T)=d\). Nous regardons l'orthogonal de l'image :
    \begin{subequations}
        \begin{align}
            (\Image(T))^{\perp}&=\{ x\in E\tq T(g)x=0\forall g\in\eK[f] \}\\
            &=\{ x\in E\tq e^*_d\big( g(x) \big)=0\forall g\in \eK[f] \}\\
            &=F.
        \end{align}
    \end{subequations}
    Par conséquent \( F^{\perp}=\Image(T)\). Vu que \( \dim\Image(T)=d\), nous avons donc \( \dim F=n-d\) et il est établi que \( E=E_y\oplus F\). 

    Nous avons donc trouvé \( F\), stable par \( f\) et tel que \( E=E_y\oplus F\). Nous devons maintenant nous assurer que cette décomposition tombe bien pour les polynômes minimaux. Si \( P_1\) est le polynôme minimal de \( f|_{E_yj}\), alors par le lemme \ref{LemAGZNNa} nous avons \( P_1=\mu_{f,y}=\mu_f\) parce que \( f|_{E_y}\) est cyclique sur \( E_y\). Mettons \( P_2\), le polynôme minimal de \( f|_F\). Étant attendu que \( F\) est stable par \( f\), le polynôme \( P_2\) divise \( P_1\). En recommençant la construction sur \( F\), nous construisons un nouvel espace \( F'\) stable sous \( F\) et vérifiant \( \mu_{f|_{F'}}=P_2\), etc.

    Nous passons maintenant à la partie unicité du théorème. Soient deux suites \( F_1,\ldots, F_r\) et \( G_1,\ldots, G_s\) de sous-espaces stables par \( f\) et vérifiant
    \begin{enumerate}
        \item
            \( E=\bigoplus_{i=1}^rF_i\),
        \item
            \( f|_{F_i}\) est cyclique,
        \item
            \( \mu_{f|_{F_{i+1}}}\) divise \( \mu_{f|_{F_i}}\),
    \end{enumerate}
    et, \emph{mutatis mutandis}, les mêmes conditions pour la famille \( \{ G_i \}\). Nous posons \( P_i=\mu_{f_{F_i}}\) et \( Q_i=\mu_{f|_{G_i}}\). Nous allons montrer par récurrence que \( P_i=Q_i\) et \( \dim F_i=\dim G_i\). Il ne sera cependant pas garanti que \( F_i=G_i\). D'abord, \( P_1=Q_1\) parce qu'ils sont tous deux égaux à \( \mu_f\) par les relations \eqref{subEqzcGouz}. Nous supposons que \( P_i=Q_i\) pour \( i\leq 1\leq j-1\) et nous tentons de montrer que \( P_j=Q_j\).

    Nous avons 
    \begin{equation}    \label{EqMrCtZO}
        P_j(f)=P_j(f)|_{F_1}\oplus\ldots\oplus P_j(f)|_{F_{j-1}}.
    \end{equation}
    En effet étant donné que \( P_{j+k}\) divise \( P_j\), nous avons\footnote{En vertu du lemme \ref{LemQWvhYb}.} \( P_{j}(f)=A(f)\circ P_{j+k}(f)\), mais \( P_{j+k}(f)F_{j+k}=0\), donc \( P_j(f)F_{j+k}=0\). Les espaces \( G_i\) n'ayant a priori aucun rapport avec les polynômes \( P_i\), nous écrivons
    \begin{equation}    \label{EqJreLiO}
        P_j(f)=P_j(f)|_{G_1}\oplus\ldots\oplus P_j(f)|_{G_{j-1}}\oplus P_j(f)|_{G_j}\oplus\ldots\oplus P_j(f)|_{G_s}.
    \end{equation}
    Pour \( 1\leq i\leq j-1\), nous avons supposé \( P_i=Q_i\). Étant donné que \( f|_{F_i}\) est semblable à \( C_{_i}\) et \( f|_{G_i}\) est semblable à \( C_{Q_i}\), la matrice de \( f|_{E_i}\) est semblable à la matrice de \( f|_{G_i}\). En particulier,
    \begin{equation}
        \dim P_j(f)F_i=\dim P_j(f)G_i.
    \end{equation}
    En prenant les dimensions des images dans les égalités \eqref{EqMrCtZO} et \eqref{EqJreLiO}, nous trouvons que
    \begin{equation}
        P_j(f)|_{G_j}=\ldots=P_j(f)|_{G_s}=0.
    \end{equation}
    Par conséquent \( P_j\in I_{f|G_j}\) et donc \( P_j\) divise \( Q_j\), qui est générateur de \( I_{f|_{G_j}}\). La situation étant symétrique entre \( P\) et \( Q\), nous montrons de même que \( Q_j\) divise \( P_j\) et donc que \( P_j=Q_j\).

    Ceci achève la démonstration du théorème de réduction de Frobenius.

\end{proof}


Sous forme matricielle, ce théorème dit que toute matrice est semblable à une matrice de la forme bloc-diagonale
\begin{equation}
    f=\begin{pmatrix}
        C_{\mu_1}    &       &       \\
            &   \ddots    &       \\
            &       &   C_{\mu_r}
    \end{pmatrix}
\end{equation}

\begin{remark}
    Si nous travaillons sur \( \eR\), la réduite de Frobenius restera une matrice réelle, même si les valeurs propres sont complexes. En effet le procédé de Frobenius ne regarde absolument pas les valeurs propres, mais seulement les facteurs irréductibles du polynôme caractéristique. La réduite de Frobenius ne tente pas de résoudre ces polynômes, mais se contente d'en utiliser les matrices compagnon.

    La situation sera différente dans le cas de la forme normal de Jordan.
\end{remark}

%---------------------------------------------------------------------------------------------------------------------------
\subsection{Forme normale de Jordan}
%---------------------------------------------------------------------------------------------------------------------------

Il existe une preuve directe de la réduction de Jordan ne nécessitant pas la réduction de Frobenius\cite{LecLinAlgAllen}. Cette dernière passe par les espaces caractéristiques\footnote{Aussi appelés «espaces propres généralisés».} et est à mon avis plus compliquée que la démonstration de Frobenius elle-même. Nous allons donc nous contenter de donner la réduction de Jordan comme un cas particulier de Frobenius.

\begin{theorem}[Réduction de Jordan]\index{réduction!Jordan}\index{Jordan!réduction}
    Soit \( E\) un espace vectoriel sur \( \eK\), et \( f\in\End(E)\) un endomorphisme dont le polynôme caractéristique \( \chi_f\) est scindé\footnote{C'est pour cette hypothèse que \( \eK=\eR\) n'est pas le bon cadre.}. Il existe une base de \( E\) dans laquelle la matrice de \( f\) s'écrit sous la forme
    \begin{equation}
        M=\begin{pmatrix}
            J_{n_1}(\lambda_1)    &       &       \\
                &   \ddots    &       \\
                &       &   J_{n_k}(\lambda_k)
        \end{pmatrix}
    \end{equation}
    où les \( \lambda_i\) sont les valeurs propres de \( f\) (avec éventuelle répétitions) et \( J_n(\lambda)\) représente le bloc \( n\times n\)
    \begin{equation}
        J_n(\lambda)=\begin{pmatrix}
            \lambda    &   1    &       &       &   \\  
                &   \lambda    &   1    &       &   \\  
                &       &   \lambda    &       &   \\  
                &       &       &   \ddots    &   1\\  
                &       &       &       &   \lambda    
        \end{pmatrix}.
    \end{equation}
    En d'autres termes, \( J_n(\lambda)_{ii}=\lambda\) et \( J_n(\lambda)_{i-1,i}=1\).    
\end{theorem}

\begin{proof}
    Nous commençons par le cas où \( f\) est nilpotente; nous notons \( M\) sa matrice. Dans ce cas la seule valeur propre est zéro et le polynôme caractéristique est \( X^m\) pour un certain \( m\). Nous savons par le lemme \ref{LemkVNisk} que (la matrice de) \( f\) est semblable à sa matrice compagnon. En l'occurrence pour \( f\) nous avons
    \begin{equation}
        C_{X^m}=\begin{pmatrix}
             0   &       &       &  0     \\
             1   &   \ddots    &       &   \vdots    \\
                &   \ddots    &   \ddots    &    \vdots   \\ 
                &       &   1    &   0     
         \end{pmatrix}.
    \end{equation}
    Ensuite le changement de base (qui est une similitude) \( (e_1,\ldots, e_n)\mapsto(e_n,\ldots, e_1)\) montre que \( C_{X^m}\) est semblable à un bloc de Jordan \( J_m(0)\).

    Supposons à présent que \( f\) ne soit pas nilpotente. Par l'hypothèse de polynôme caractéristique scindé, nous supposons que \( f\) a \( m\) valeurs propres distinctes et que son polynôme caractéristique est
    \begin{equation}
        \chi_f=(X-\lambda_1)^{l_1}\ldots (X-\lambda_m)^{l_m}.
    \end{equation}
    Le lemme des noyaux (théorème \ref{ThoDecompNoyayzzMWod}) nous enseigne que
    \begin{equation}
        E=\bigoplus_{i=1}^m\underbrace{\ker(f-\mu_i\mtu)^{l_i}}_{F_i}.
    \end{equation}
    La restriction de \( f-\lambda_i\mtu\) à \( F_i\) est par construction un endomorphisme nilpotent, et donc peut s'écrire comme un bloc de Jordan avec des zéros sur la diagonale. En utilisant la décomposition
    \begin{equation}
        f|_{F_i}=(f-\lambda_i\mtu)|_{F_i}+\lambda_i\mtu_{F_i},
    \end{equation}
    nous voyons que \( f|_{F_i}\) s'écrit comme un bloc de Jordan avec \( \lambda_i\) sur la diagonale.
\end{proof}

\begin{remark}
    Nous pouvons calculer la forme normale de Jordan pour une matrice complexe ou réelle, mais dans les deux cas nous devons nous attendre à obtenir une matrice complexe parce que les valeurs propres d'une matrice réelle peuvent être complexes. Cependant nous demandons que le polynôme caractéristique de \( f\) soit scindé sur \( \eK\). En pratique, la décomposition de Jordan n'est garantie que sur les corps algébriquement clos, c'est à dire sur \( \eC\).

    La suite des invariants de similitude sur laquelle repose Frobenius, elle, est disponible sur tout corps, y compris \( \eR\).
\end{remark}

Une application de la décomposition de Jordan est l'existence d'un logarithme pour les matrices.
\begin{proposition}
    Toute matrice inversible complexe est une exponentielle\index{exponentielle!de matrice}.
\end{proposition}

\begin{proof}
    Soit \( A\in \GL(n,\eC)\); nous allons donner une matrice \( B\in \eM(n,\eC)\) telle que \( A=\exp(B)\). D'abord remarquons qu'il suffit de prouver le résultat pour une matrice par classe de similitude. En effet si \( A=\exp(B)\) et si \( M\) est inversible alors 
    \begin{subequations}    \label{EqqACuGK}
        \begin{align}
            \exp(MBM^{-1})&=\sum_k\frac{1}{ k! }(MBM^{-1})^k\\
            &=\sum_k\frac{1}{ k! }MB^kM^{-1}\\
            &=M\exp(B)M^{-1}.
        \end{align}
    \end{subequations}
    Donc \( MAM^{-1}=\exp(MBM^{-1})\). Nous pouvons donc nous contenter de trouver un logarithme pour les blocs de Jordan. Nous supposons donc que \( A=(\mtu+N)\) avec \( N^m=0\). En nous inspirant de \eqref{EqweEZnV}, nous posons
    \begin{equation}
        D(t)=tN-\frac{ t^2 }{ 2 }N^2+\ldots +(-1)^m\frac{ t^{m-1} }{ m-1 }N^{m-1}
    \end{equation}
    et nous allons prouver que \(  e^{D(1)}=\mtu+N\). Notons que \( N\) étant nilpotente, cette somme ainsi que toutes celles qui viennent sont finies. Il n'y a donc pas de problèmes de convergences dans cette preuve (si ce n'est les passages des équations \eqref{EqqACuGK}).

    Nous posons \( S(t)= e^{D(t)}\) (la somme est finie), et nous avons
    \begin{equation}
        S'(t)=D'(t) e^{D(t)}
    \end{equation}
    Afin d'obtenir une expression qui donne \( S'\) en termes de \( S\), nous multiplions par \( (\mtu+tN)\) en remarquant que \( (\mtu+tN)D'(t)=N\) nous avons
    \begin{equation}
        (\mtu+tN)S'(t)=NS(t).
    \end{equation}
    En dérivant à nouveau,
    \begin{equation}    \label{EqKjccqP}
        (\mtu+tN)S''(t)=0.
    \end{equation}
    La matrice \( (\mtu+tN)\) est inversible parce que son noyau est réduit à \( \{ 0 \}\). En effet si \( (\mtu+tN)x=0\), alors \( Nx=-\frac{1}{ t }x\), ce qui est impossible parce que \( N\) est nilpotente. Ce que dit l'équation \eqref{EqKjccqP} est alors que \( S''(t)=0\). Si nous développons \( S(t)\) en puissances de \( t\) nous nous arrêtons au terme d'ordre \( 1\) et nous avons
    \begin{equation}
        S(t)=S(0)+tS'(0)=\mtu+tD'(0)=1+tN.
    \end{equation}
    En \( t=1\) nous trouvons \( S(1)=\mtu+N\). La matrice \( D(1)\) donnée est donc bien un logarithme de $\mtu+N$.
\end{proof}

%+++++++++++++++++++++++++++++++++++++++++++++++++++++++++++++++++++++++++++++++++++++++++++++++++++++++++++++++++++++++++++
\section{Mini introduction au produit tensoriel}
%+++++++++++++++++++++++++++++++++++++++++++++++++++++++++++++++++++++++++++++++++++++++++++++++++++++++++++++++++++++++++++
\label{SeOOpHsn}

%---------------------------------------------------------------------------------------------------------------------------
\subsection{Définitions}
%---------------------------------------------------------------------------------------------------------------------------

Soit \( E\), un espace vectoriel de dimension finie. Si \( \alpha\) et \( \beta\) sont deux formes linéaires sur un espace vectoriel \( E\), nous définissons \( \alpha\otimes \beta\) comme étant la \( 2\)-forme donnée par
\begin{equation}
    (\alpha\otimes \beta)(u,v)=\alpha(u)\beta(v).
\end{equation}
Si \( a\) et \( b\) sont des vecteurs de \( E\), ils sont vus comme des formes sur \( E\) via le produit scalaire et nous avons
\begin{equation}
    (a\otimes b)(u,v)=(a\cdot u)(b\cdot v).
\end{equation}
Cette dernière équation nous incite à pousser un peu plus loin la définition de \( a\otimes b\) et de simplement voir cela comme la matrice de composantes
\begin{equation}
    (a\otimes b)_{ij}=a_ib_j.
\end{equation}
Cette façon d'écrire a l'avantage de ne pas demander de se souvenir qui est une vecteur ligne, qui est un vecteur colonne et où il faut mettre la transposée. Évidemment \( (a\otimes b)\) est soit \( ab^t\) soit \( a^tb\) suivant que \( a\) et \( b\) soient ligne ou colonne.

\begin{lemma}   \label{LemMyKPzY}
    Soient \( x,y\in E\) et \( A,B\) deux opérateurs linéaires sur \( E\) vus comme matrices. Alors
    \begin{equation}        \label{EqXdxvSu}
        (Ax\otimes By)=A(x\otimes y)B^t.
    \end{equation}
\end{lemma}

\begin{proof}
    Calculons la composante \( ij\) de la matrice \( (Ax\otimes By)\). Nous avons
    \begin{subequations}
        \begin{align}
            (Ax\otimes By)_{ij}&=(Ax)_i(By)_j\\
            &=\sum_{kl}A_{ik}x_kB_{jl}y_l\\
            &=A_{ik}(x\otimes y)_{kl}B_{jl}\\
            &=\big( A(x\otimes y)B^t \big)_{ij}.
        \end{align}
    \end{subequations}
\end{proof}
