% This is part of Mes notes de mathématique
% Copyright (c) 2011-2013
%   Laurent Claessens
% See the file fdl-1.3.txt for copying conditions.

%+++++++++++++++++++++++++++++++++++++++++++++++++++++++++++++++++++++++++++++++++++++++++++++++++++++++++++++++++++++++++++
\section{Théorèmes de Brouwer et Schauder}
%+++++++++++++++++++++++++++++++++++++++++++++++++++++++++++++++++++++++++++++++++++++++++++++++++++++++++++++++++++++++++++

%--------------------------------------------------------------------------------------------------------------------------- 
\subsection{Brouwer}
%---------------------------------------------------------------------------------------------------------------------------
\label{SecZCCmMnQ}

\begin{proposition}
    Soit \( f\colon \mathopen[ a , b \mathclose]\to \mathopen[ a , b \mathclose]\) une fonction continue. Alors \( f\) accepte un point fixe.
\end{proposition}

\begin{proof}
    En effet si nous considérons \( g(x)=f(x)-x\) alors nous avons \( g(a)=f(a)-a\geq 0\) et \( g(b)=f(b)-b\leq 0\). Si \( g(a)\) ou \( g(b)\) est nul, la proposition est démontrée; nous supposons donc que \( g(a)>0\) et \( g(b)<0\). La proposition découle à présent du théorème des valeurs intermédiaires \ref{ThoValInter}.
\end{proof}

\begin{example}
    La fonction \( x\mapsto\cos(x)\) est continue entre \( \mathopen[ -1 , 1 \mathclose]\) et \( \mathopen[ -1 , 1 \mathclose]\). Elle admet donc un point fixe. Par conséquent il existe (au moins) une solution à l'équation \( \cos(x)=x\).
\end{example}

Nous allons donner plusieurs versions et preuves.
\begin{enumerate}
    \item
        Dans \( \eR^n\) en version \( C^{\infty}\) via le théorème de Stockes, proposition \ref{PropDRpYwv}.
    \item
        Dans \( \eR^n\) en version continue, en s'appuyant sur le cas \( C^{\infty}\) et en faisant un passage à la limite, théorème \ref{ThoRGjGdO}.
    \item
        Dans \( \eR^2\) via l'homotopie, théorème \ref{ThoLVViheK}. Oui, c'est très loin. Et c'est normal parce que ça va utiliser la formule de l'indice qui est de l'analyse complexe\footnote{On aime bien parce que ça ne demande pas Stokes, mais quand même hein, c'est pas gratos non plus.}.
\end{enumerate}

\begin{proposition}[Brouwer dans \( \eR^n\) version \(  C^{\infty}\) via Stokes]     \label{PropDRpYwv}
    Soit \( B\) la boule fermée de centre \( 0\) et de rayon \( 1\) de \( \eR^n\) et \( f\colon B\to B\) une fonction \(  C^{\infty}\). Alors \( f\) admet un point fixe.
\end{proposition}
\index{point fixe!Brouwer}

\begin{proof}
    Supposons que \( f\) ne possède pas de points fixes. Alors pour tout \( x\in B\) nous considérons la ligne droite partant de \( x\) dans la direction de \( f(x)\) (cette droite existe parce que \( x\) et \( f(x)\) sont supposés distincts). Cette ligne intersecte \( \partial B\) en un point que nous appelons \( g(x)\). Prouvons que cette fonction est \( C^k\) dès que \( f\) est \( C^k\) (y compris avec \( k=\infty\)).

   Le point \( g(x) \) est la solution du système
    \begin{subequations}
        \begin{numcases}{}
        g(x)-f(x)=\lambda\big( x-f(x) \big)\\
        \| g(x) \|^2=1\\
        \lambda\geq 0.
        \end{numcases}
    \end{subequations}
    En substituant nous obtenons l'équation
    \begin{equation}
        P_x(\lambda)=\| \lambda\big( x-f(x) \big)+f(x) \|^2-1=0,
    \end{equation}
    ou encore
    \begin{equation}
        \lambda^2\| x-f(x) \|^2+2\lambda\big( x-f(x) \big)\cdot f(x)+\| f(x) \|^2-1=0.
    \end{equation}
    En tenant compte du fait que \( \| f(x)<1 \|\) (pare que les images de \( f\) sont dans \( \mB\)), nous trouvons que \( P_x(0)\leq 0\) et \( P_x(1)\leq 0\). De même \( \lim_{\lambda\to\infty} P_x(\lambda)=+\infty\). Par conséquent le polynôme de second degré \( P_x\) a exactement deux racines distinctes \( \lambda_1\leq 0\) et \( \lambda_2\geq 1\). La racine que nous cherchons est la seconde. Le discriminant est strictement positif, donc pas besoin d'avoir peur de la racine dans
    \begin{equation}
        \lambda(x)=\frac{ -\big( x-f(x) \big)\cdot f(x)+\sqrt{   \Delta_x  } }{ \| x-f(x) \|^2 }
    \end{equation}
    où 
    \begin{equation}
        \Delta_x=4\Big( \big( x-f(x) \big)\cdot f(x) \Big)^2-4\| x-f(x) \|^2\big( \| f(x) \|^2-1 \big).
    \end{equation}
    Notons que la fonction \( \lambda(x)\) est \( C^k\) dès que \( f\) est \( C^k\); et en particulier elle est \( C^{\infty}\) si \( f\) l'est.

    En résumé la fonction \( g\) ainsi définie vérifie deux propriétés :
    \begin{enumerate}
        \item
            elle est \(  C^{\infty}\);
        \item
            elle est l'identité sur \( \partial B\).
    \end{enumerate}
    La suite de la preuve consiste à montrer qu'une telle rétraction sur \( B\) ne peut pas exister\footnote{Notons qu'il n'existe pas non plus de rétractions continues sur \( B\), mais pour le montrer il faut utiliser d'autres méthodes que Stokes, ou alors présenter les choses dans un autre ordre.}.

    Nous considérons une forme de volume \( \omega\) sur \( \partial B\) : l'intégrale de \( \omega\) sur \( \partial B\) est la surface de \( \partial B\) qui est non nulle. Nous avons alors
    \begin{equation}
        0<\int_{\partial B}\omega
        =\int_{\partial B}g^*\omega
        =\int_Bd(g^*\omega)
        =\int_Bg^*(d\omega)
        =0
    \end{equation}
    Justifications :
    \begin{itemize}
        \item 
            L'intégrale \( \int_{\partial B}\omega\) est la surface de \( \partial B\) et est donc non nulle.
        \item
            La fonction \( g\) est l'identité sur \( \partial B\). Nous avons donc \( \omega=g^*\omega\).
        \item
            Le lemme \ref{LemdwLGFG}.
        \item
            La forme \( \omega\) est de volume, par conséquent de degré maximum et \( d\omega=0\).
    \end{itemize}
\end{proof}

Un des points délicats est de se ramener au cas de fonctions \( C^{\infty}\). Pour la régularisation par convolution, voir \cite{AllardBrouwer}; pour celle utilisant le théorème de Weierstrass, voir \cite{KuttlerTopInAl}.
\begin{theorem}[Brouwer dans \( \eR^n\) version continue]\label{ThoRGjGdO}
    Soit \( B\) la boule fermée de centre \( 0\) et de rayon \( 1\) de \( \eR^n\) et \( f\colon B\to B\) une fonction continue. Alors \( f\) admet un point fixe.
\end{theorem}
\index{théorème!Brouwer}

\begin{proof}
    Nous commençons par définir une suite de fonctions
    \begin{equation}
        f_k(x)=\frac{ f(x) }{ 1+\frac{1}{ k } }.
    \end{equation}
    Nous avons \( \| f_k-f \|_{\infty}\leq \frac{1}{ 1+k }\) où la norme est la norme uniforme sur \( B\). Par le théorème de Weierstrass \ref{ThoWmAzSMF} il existe une suite de fonctions \(  C^{\infty}\) \( g_k\) telles que
    \begin{equation}
        \|  g_k-f_k\|_{\infty}\leq\frac{1}{ 1+k }.
    \end{equation}
    Vérifions que cette fonction \( g_k\) soit bien une fonction qui prend ses valeurs dans \( B\) :
    \begin{subequations}
        \begin{align}
            \| g_k(x) \|&\leq \| g_k(x)-f_k(x) \|+\| f_k(x) \|\\
            &\leq \frac{1}{ 1+k }+\frac{ \| f(x) \| }{ 1+\frac{1}{ k } }\\
            &\leq \frac{1}{ 1+k}+\frac{1}{ 1+\frac{1}{ k } }\\
            &=1.
        \end{align}
    \end{subequations}
    Par la version \(  C^{\infty}\) du théorème (proposition \ref{PropDRpYwv}), \( g_k\) admet un point fixe que l'on nomme \( x_k\).

    Étant donné que \( x_k\) est dans le compact \( B\), quitte à prendre une sous suite nous supposons que la suite \( (x_k)\) converge vers un élément \( x\in B\). Nous montrons maintenant que \( x\) est un point fixe de \( f\) :
    \begin{subequations}
        \begin{align}
            \| f(x)-x \|&=\| f(x)-g_k(x)+g_k(x)-x_k+x_k-x \|\\
            &\leq \| f(x)-g_k(x) \| +\underbrace{\| g_k(x)-x_k \|}_{=0}+\| x_k-x \|\\
            &\leq \frac{1}{ 1+k }+\| x_k-x \|.
        \end{align}
    \end{subequations}
    En prenant le limite \( k\to\infty\) le membre de droite tend vers zéro et nous obtenons \( f(x)=x\).
\end{proof}

%---------------------------------------------------------------------------------------------------------------------------
\subsection{Théorème de Schauder et équations différentielles}
%---------------------------------------------------------------------------------------------------------------------------

Une conséquence du théorème de Brouwer est le théorème de Schauder qui est valide en dimension infinie.

\begin{theorem}[Théorème de Schauder\cite{LeDretSc}]\index{théorème!Schauder}       \label{ThovHJXIU}
    Soit \( E\), un espace vectoriel normé, \( K\) un convexe compact de \( E\) et \( f\colon K\to K\) une fonction continue. Alors \( f\) admet un point fixe.
\end{theorem}
\index{théorème!Schauder}
\index{point fixe!Schauder}

\begin{proof}
    Étant donné que \( f\colon K\to K\) est continue, elle y est uniformément continue. Si nous choisissons \( \epsilon\) alors il existe \( \delta>0\) tel que 
    \begin{equation}
        \| f(x)-f(y) \|\leq \epsilon
    \end{equation}
    dès que \( \| x-y \|\leq \delta\). La compacité de \( K\) permet de choisir un recouvrement fini par des ouverts de la forme
    \begin{equation}    \label{EqKNPUVR}
        K\subset \bigcup_{1\leq i\leq p}B(x_j,\delta)
    \end{equation}
    où \( \{ x_1,\ldots, x_p \}\subset K\). Nous considérons maintenant \( L=\Span\{ f(x_j)\tq 1\leq j\leq p \}\) et
    \begin{equation}
        K^*=K\cap L.
    \end{equation}
    Le fait que \( K\) et \( L\) soient convexes implique que \( K^*\) est convexe. L'ensemble \( K^*\) est également compact parce qu'il s'agit d'une partie fermée de \( K\) qui est compact (lemme \ref{LemnAeACf}). Notons en particulier que \( K^*\) est contenu dans un espace vectoriel de dimension finie, ce qui n'est pas le cas de \( K\).

    Nous allons à présent construire une sorte de partition de l'unité subordonnée au recouvrement \eqref{EqKNPUVR} sur \( K\) (voir le lemme \ref{LemGPmRGZ}). Nous commençons par définir
    \begin{equation}
        \psi_j(x)=\begin{cases}
            0    &   \text{si \( \| x-x_j \|\geq \delta\)}\\
            1-\frac{ \| x-x_j \| }{ \delta }    &    \text{sinon}.
        \end{cases}
    \end{equation}
    pour chaque \( 1\leq j\leq p\). Notons que \( \psi_j\) est une fonction positive, nulle en-dehors de \( B(x_j,\delta)\). En particulier la fonction suivante est bien définie :
    \begin{equation}
        \varphi_j(x)=\frac{ \psi_j(x) }{ \sum_{k=1}^p\psi_k(x) }
    \end{equation}
    et nous avons \( \sum_{j=1}^p\varphi_j(x)=1\). Les fonctions \( \varphi_j\) sont continues sur \( K\) et nous définissons finalement
    \begin{equation}
        g(x)=\sum_{j=1}^p\varphi_j(x)f(x_j).
    \end{equation}
    Pour chaque \( x\in K\), l'élément \( g(x)\) est une combinaison des éléments \( f(x_j)\in K^*\). Étant donné que \( K^*\) est convexe et que la somme des coefficients \( \varphi_j(x)\) vaut un, nous avons que \( g\) prend ses valeurs dans \( K^*\) par la proposition \ref{PropPoNpPz}.

    Nous considérons seulement la restriction \( g\colon K^*\to K^*\) qui est continue sur un compact contenu dans un espace vectoriel de dimension finie. Le théorème de Brouwer nous enseigne alors que \( g\) a un point fixe (proposition \ref{ThoRGjGdO}). Nous nommons \( y\) ce point fixe. Notons que \( y\) est fonction du \( \epsilon\) choisit au début de la construction, via le \( \delta\) qui avait conditionné la partition de l'unité.

    Nous avons
    \begin{subequations}        \label{EqoXuTzE}
        \begin{align}
            f(y)-y&=f(y)-g(y)\\
            &=\sum_{j=1}^p\varphi_j(y)f(y)-\sum_{j=1}^p\varphi_j(y)f(x_j)\\
            &=\sum_{j=1}^p\varphi(j)(y)\big( f(y)-f(x_j) \big).
        \end{align}
    \end{subequations}
    Par construction, \( \varphi_j(y)\neq 0\) seulement si \( \| y-x_j \|\leq \delta\) et par conséquent seulement si \( \| f(y)-f(x_j) \|\leq \epsilon\). D'autre par nous avons \( \varphi_j(y)\geq 0\); en prenant la norme de \eqref{EqoXuTzE} nous trouvons
    \begin{equation}
        \| f(y)-y \|\leq \sum_{j=1}^p\| \varphi_j(y)\big( f(y)-f(x_j) \big) \|\leq \sum_{j=1}^p\varphi_j(y)\epsilon=\epsilon.
    \end{equation}
    Nous nous souvenons maintenant que \( y\) était fonction de \( \epsilon\). Soit \( y_m\) le \( y\) qui correspond à \( \epsilon=2^{-m}\). Nous avons alors
    \begin{equation}
        \| f(y_m)-y_m \|\leq 2^{-m}.
    \end{equation}
    L'élément \( y_m\) est dans \( K^*\) qui est compact, donc quitte à choisir une sous suite nous pouvons supposer que \( y_m\) est une suite qui converge vers \( y^*\in K\)\footnote{Notons que même dans la sous suite nous avons \( \| f(y_m)-y_m \|\leq 2^{-m}\), avec le même «\( m\)» des deux côtés de l'inégalité.}. Nous avons les majorations
    \begin{equation}
        \| f(y^*)-y^* \|\leq \| f(y^*)-f(y_m) \|+\| f(y_m)-y_m \|+\| y_m-y^* \|.
    \end{equation}
    Si \( m\) est assez grand, les trois termes du membre de droite peuvent être rendus arbitrairement petits, d'où nous concluons que
    \begin{equation}
        f(y^*)=y^*
    \end{equation}
    et donc que \( f\) possède un point fixe.
\end{proof}

Ce théorème permet de démontrer une version du théorème de Cauchy-Lipschitz (théorème \ref{ThokUUlgU}) sans la condition Lipschitz, mais alors sans unicité de la solution. Notons que de ce point de vue nous sommes dans la même situation que la différence entre le théorème de Brouwer et celui de Picard : hors hypothèse de type «contraction», point d'unicité.

\begin{theorem}[Cauchy-Arzela\cite{ClemKetl}]
    Nous considérons le système d'équation différentielles
    \begin{subequations}        \label{EqTXlJdH}
        \begin{numcases}{}
            y'=f(t,y)\\
            y(t_0)=y_0.
        \end{numcases}
    \end{subequations}
    avec \( f\colon U\to \eR^n\), continue où \( U\) est ouvert dans \( \eR\times \eR^n\). Alors il existe un voisinage fermé \( V\) de \( t_0\) sur lequel une solution \( C^1\) du problème \eqref{EqTXlJdH} existe.
\end{theorem}
\index{théorème!Cauchy-Arzela}

\begin{proof}[Idée de la démonstration]
    Nous considérons \( M=\| f \|_{\infty}\) et \( K\), l'ensemble des fonctions \( M\)-Lipschitz sur \( U\). Nous prouvons que \( (K,\| . \|_{\infty})\) est compact. Ensuite nous considérons l'application
    \begin{equation}
        \begin{aligned}
            \Phi\colon K&\to K \\
            \Phi(f)(t)&=x_0+\int_{t_0}^tf\big( u,f(u) \big)du. 
        \end{aligned}
    \end{equation}
    Après avoir prouvé que \( \Phi\) était continue, nous concluons qu'elle a un point fixe par le théorème de Schauder \ref{ThovHJXIU}.
\end{proof}

%+++++++++++++++++++++++++++++++++++++++++++++++++++++++++++++++++++++++++++++++++++++++++++++++++++++++++++++++++++++++++++
\section{Théorème de Markov-Kakutani et mesure de Haar}
%+++++++++++++++++++++++++++++++++++++++++++++++++++++++++++++++++++++++++++++++++++++++++++++++++++++++++++++++++++++++++++

\begin{definition}
    Soit \( G\) un groupe topologique. Une \defe{mesure de Haar}{mesure!de Haar} sur \( G\) est une mesure \( \mu\) telle que 
    \begin{enumerate}
        \item
            \( \mu(gA)=\mu(A)\) pour tout mesurable \( A\) et tout \( g\in G\),
        \item
            \( \mu(K)<\infty\) pour tout compact \( K\subset G\).
    \end{enumerate}
    Si de plus le groupe \( G\) lui-même est compact nous demandons que la mesure soit normalisée : \( \mu(G)=1\).
\end{definition}

Le théorème suivant nous donne l'existence d'une mesure de Haar sur un groupe compact.
\begin{theorem}[Markov-Katutani\cite{BeaakPtFix}]\index{théorème!Markov-Takutani}   \label{ThoeJCdMP}
    Soit \( E\) un espace vectoriel normé et \( K\), une partie non vide, convexe, fermée et bornée de \( E'\). Soit \( T\colon K\to K\) une application continue. Alors \( T\) a un point fixe.
\end{theorem}

\begin{proof}
    Nous considérons un point \( x_0\in K\) et la suite
    \begin{equation}
        x_n=\frac{1}{ n+1 }\sum_{i=0}^n T^ix_0.
    \end{equation}
    La somme des coefficients devant les \( T^i(x_0)\) étant \( 1\), la convexité de \( K\) montre que \( x_n\in K\). Nous considérons l'ensemble
    \begin{equation}
        C=\bigcap_{n\in \eN}\overline{ \{ x_m\tq m\geq n \} }.
    \end{equation}
    Le lemme \ref{LemooynkH} indique que \( C\) n'est pas vide, et de plus il existe une sous suite de \( (x_n)\) qui converge vers un élément \( x\in C\). Nous avons
    \begin{equation}
        \lim_{n\to \infty} x_{\sigma(n)}(v)=x(v)
    \end{equation}
    pour tout \( v\in E\). Montrons que \( x\) est un point fixe de \( T\). Nous avons
    \begin{subequations}
        \begin{align}
            \| (Tx_{\sigma(k)}-x_{\sigma(k)})v \|&=\Big\| T\frac{1}{ 1+\sigma(k) }\sum_{i=0}^{\sigma(k)}T^ix_0(v)-\frac{1}{ 1+\sigma(k) }\sum_{i=0}^{\sigma(k)}T^ix_0(v) \Big\|\\
            &=\Big\| \frac{1}{ 1+\sigma(k) }\sum_{i=0}^{\sigma(k)}T^{i+1}x_0(v)-T^ix_0(v) \Big\|\\
            &=\frac{1}{ 1+\sigma(k) }\big\| T^{\sigma(k)+1}x_0(v)-x_0(v) \big\|\\
            &\leq\frac{ 2M }{ \sigma(k)+1 }
        \end{align}
    \end{subequations}
    où \( M=\sum_{y\in K}\| y(v) \|<\infty\) parce que \( K\) est borné. En prenant \( k\to\infty\) nous trouvons
    \begin{equation}
        \lim_{k\to \infty} \big( Tx_{\sigma(k)}-x_{\sigma(k)} \big)v=0,
    \end{equation}
    ce qui signifie que \( Tx=x\) parce que \( T\) est continue.
\end{proof}

Le théorème suivant est une conséquence du théorème de Markov-Katutani.
\begin{theorem}
    Si \( G\) est un groupe topologique compact possédant une base dénombrable de topologie alors \( G\) accepte une unique mesure de Haar normalisée. De plus elle est unimodulaire :
    \begin{equation}
        \mu(Ag)=\mu(gA)=\mu(A)
    \end{equation}
    pour tout mesurables \( A\subset G\) et tout élément \( g\in G\).
\end{theorem}
\index{mesure!de Haar}

%+++++++++++++++++++++++++++++++++++++++++++++++++++++++++++++++++++++++++++++++++++++++++++++++++++++++++++++++++++++++++++
\section{Méthode de Newton}
%+++++++++++++++++++++++++++++++++++++++++++++++++++++++++++++++++++++++++++++++++++++++++++++++++++++++++++++++++++++++++++

L'objectif de la méthode de Newton est d'évaluer une racine \( a\) de l'équation \( f(x)=0\) lorsque nous avons déjà une approximation \( x_0\) de \( a\).

%---------------------------------------------------------------------------------------------------------------------------
\subsection{Points fixes attractifs et répulsifs}
%---------------------------------------------------------------------------------------------------------------------------

Soit \( I\) un intervalle fermé de \( \eR\) et \( \varphi\colon I\to I\) une application \( C^1\). Soit \( a\) un point fixe de \( \varphi\). Nous disons que \( a\) est \defe{attractif}{point fixe!attractif}\index{attractif!point fixe} si il existe un voisinage \( V\) de \( a\) tel que pour tout \( x_0\in V\) la suite \( x_{n+1}=\varphi(x_n)\) converge vers \( a\). Le point \( a\) sera dit \defe{répulsif}{répulsif!point fixe} si il existe un voisinage \( V\) de \( a\) tel que pour tout \( x_0\in V\) la suite \( x_{n+1}=\varphi(x_n)\) diverge.

\begin{lemma}[\cite{DemaillyNum}]
    Si \( | \varphi'(a) |<1\) alors \( a\) est attractif et la convergence est au moins exponentielle.

    Si \( | \varphi'(a) |>1\) alors \( a\) est répulsif et la divergence est au moins exponentielle.
\end{lemma}

\begin{proof}
    Si \( | \varphi'(a)<1 |\) alors il existe \( k\) tel que \( | \varphi'(a) |<k<1\) et par continuité il existe un voisinage \( V\) de \( a\) dans lequel \( | \varphi'(x) |<k\) pour tout \( x\in V\). En utilisant le théorème des accroissements finis nous avons
    \begin{equation}
        | x_n-a |=\big| f(x_{n-1}-a) \big|\leq k| x_{n-1}-a |
    \end{equation}
    et par récurrence
    \begin{equation}
        | x_n-a |\leq k^n| x_0-a |.
    \end{equation}

    Le cas \( | \varphi'(a)>1 |\) se traite de façon similaire.
\end{proof}

\begin{remark}
    Dans le cas \(| \varphi'(a) |=1\), nous ne pouvons rien conclure. Si \( \varphi(x)=\sin(x)\) nous avons \( \sin(x)<x\) et le point \( a=0\) est attractif. A contrario, si \( \varphi(x)=\sinh(x)\) nous avons \( |\sinh(x)|>|x|\) et le point \( a=0\) est répulsif.
\end{remark}

%---------------------------------------------------------------------------------------------------------------------------
\subsection{Méthode de Newton}
%---------------------------------------------------------------------------------------------------------------------------
Nous parlons un petit peu de méthode de Newton en dimension \( 1\) dans la sous-section \ref{SubSecMethodeNewton}.

\begin{lemma}       \label{LemXdObnV}
    Soient \( A\) et \( B\) deux matrices inversibles telles que la matrice \( (A+\epsilon B)\) soit inversible pour tout \( \epsilon\) assez petit. Alors il existe une matrice \( X(\epsilon)\) telle que
    \begin{equation}
        (A+\epsilon B)^{-1}=(A^{-1}+\epsilon X)
    \end{equation}
    et telle que \( \lim_{\epsilon\to 0}X(\epsilon)=-A^{-1} BA^{-1}\).
\end{lemma}

\begin{proof}
    Le candidat matrice \( X\) est relativement simple à trouver en écrivant
    \begin{equation}
        (A+\epsilon B)(A^{-1}+\epsilon X)=\mtu+\epsilon AX+\epsilon BA^{-1}+\epsilon^2BX.
    \end{equation}
    En imposant que cela soit \( \mtu\), nous trouvons
    \begin{equation}
        X(\epsilon)=-(A+\epsilon B)^{-1} BA^{-1}.
    \end{equation}
    La matrice \( X(\epsilon)\) étant un inverse à droite de \( (A+\epsilon B)\), son déterminant est non nul et \( X\) est inversible. Par conséquent elle est également inversible au sens usuel. Le calcul de la limite est direct :
    \begin{equation}
        \lim_{\epsilon\to 0}-(A+\epsilon B)^{-1} BA^{-1}=A^{-1} BA^{-1}
    \end{equation}
    parce que l'inverse est une fonction continue sur \( \eM(n,\eR)\).
\end{proof}

\begin{remark}
    Un calcul naïf nous permet de trouver le même résultat de façon plus heuristique. En effet un développement usuel (dans \( \eR\)) est
    \begin{equation}
        \frac{1}{ a+\epsilon b }=\frac{1}{ a }-\frac{ \epsilon b }{ a^2 }+\ldots
    \end{equation}
    Si nous récrivons cela avec des matrices, nous écrivons (attention : passage heuristique!) :
    \begin{equation}
        (A+\epsilon B)^{-1}=A^{-1}-\epsilon A^{-1} BA^{-1}+\ldots
    \end{equation}
    Notons le choix de généraliser \( b/a^2\) par \( a^{-1} ba^{-1}\). Dans les réels les deux écritures sont équivalentes, mais pas dans les matrices.

    Étudions si \( A^{-1}-\epsilon A^{-1}BA^{-1}\) est bien un inverse à \( \epsilon^2\) près de \( (A+\epsilon B)\) :
    \begin{equation}
        (A+\epsilon B)(A^{-1}+\epsilon A^{-1} BA^{-1})=1-\epsilon BA^{-1}+\epsilon BA^{-1}-\epsilon^2BA^{-1}BA^{-1}=1-\epsilon^2BA^{-1} BA^{-1}.
    \end{equation}
    Par conséquent, à des termes en \( \epsilon^2\) près la matrice \( A^{-1}-\epsilon A^{-1}BA^{-1}\) est bien un inverse de \( A+\epsilon B\).
\end{remark}

\begin{theorem}[Méthode de Newton\cite{ChambertNewton}]\label{ThoHGpGwXk}
    Soit \( f\colon \eR^n\to \eR^n\) une application de classe \( C^2\) et un point \( a\in \eR^n\) tel que \( f(a)=0\). Nous supposons que \( df_a\) est inversible.

    Alors il existe un voisinage \( V\) de \( a\) tel que pour tout \( x_0\in V\) la suite définie par récurrence
    \begin{equation}
        x_{n+1}=x_n-(df_a)^{-1}\big( f(x_n) \big)
    \end{equation}
    converge vers \( a\). De plus la vitesse est quadratique au sens où il existe \( C>1\) tel que 
    \begin{equation}        \label{EqtkiDXt}
        \| x_n-a \|\leq C^{-1-2^n}.
    \end{equation}
\end{theorem}
\index{Newton!méthode}
\index{méthode!Newton}
\index{formule!Taylor!utilisation}
\index{convergence!rapidité}
\index{suite!définie par itération}

\begin{proof}
    Étant donné que \( df_a\) est inversible et que \( df\) est continue, l'application \( df_x\) est continue\footnote{Nous pouvons voir \( df\) comme l'application qui à \( x\) fait correspondre la matrice \( df_x\in\eM(n,\eR)\). Cette application étant continue et la non inversibilité d'une matrice étant donnée par l'annulation du déterminant, les matrices inversibles forment un ouvert dans l'ensemble des matrices.} pour tout \( x\) dans un voisinage de \( a\). Nous prenons \( r>0\) tel que \( df_x\) est inversible pour tout \( x\in B(a,r)\).

    Nous considérons la fonction 
    \begin{equation}
        \begin{aligned}
                F\colon B(a,r)&\to \eR^n \\
                x&\mapsto x-(df_x)^{-1}\big( f(x) \big). 
            \end{aligned}
        \end{equation}
        Cela est une application \( C^1\). La clef est de montrer que l'application de \( F\) à un point \( a+h\) rapproche de \( a\) pourvu que \( h\) soit assez petit. Nous avons la formule suivante :
        \begin{equation}        \label{EqyDLQeE}
            F(a+h)-F(a)=h-\big( df_{a+h} \big)^{-1}\big( f(a+h) \big).
        \end{equation}
        Nous allons maintenant utiliser un développement de Taylor par rapport à \( h\) en suivant la formule \eqref{EquQtpoN}. Nous avons
        \begin{equation}
            f(a+h)=f(a)+df_a(h)+\| h \|^2\xi(h)
        \end{equation}
        où \( \xi\colon \eR^n\to \eR^n\) est une fonction qui tend vers une constante lorsque \( h\to 0\). Nous avons aussi
        \begin{equation}
            df_{a+h}=df_a+\| h \|\tau(h)
        \end{equation}
        où \( \tau\colon \eR^n\to \eM(n,\eR)\) est une application qui tend vers une constante lorsque \( h\to 0\). En ce qui concerne l'inverse nous utilisons le lemme\footnote{Pour l'inversibilité de \( \| h \|\tau(h)\), notons que \( df_a\) est inversible et que par hypothèse la somme \( df_a+\| h \|\tau(h)\) est inversible.} \ref{LemXdObnV} :
        \begin{equation}
            \big( df_a+\| h \|\tau(h) \big)^{-1}=(df_a)^{-1}+\| h \|A(h)
        \end{equation}
        où \( A\) est une autre matrice fonction de \(h\) qui tend vers une constante lorsque \( h\) tend vers zéro. En substituant le tout dans \eqref{EqyDLQeE} nous trouvons
        \begin{equation}
            F(a+h)-F(a)=\| h \|^2(df_a)^{-1}\xi(h)+\| h \|\big( A(h)\circ df_a \big)(h)+\| h \|^3A(h)\xi(h).
        \end{equation}
        En ce qui concerne la norme nous utilisons le fait que si \( T\) est un opérateur, \( \| Tx \|\leq \| T \|\| x \|\). Nous trouvons
        \begin{subequations}
            \begin{align}
                \| F(a+h)-F(a) \|&\leq \| h \|^2\| (df_a)^{-1} \|\| \xi(h) \|+\| h \|^2\| A(h)\circ df_a \|+\| h \|^3\| A(h) \|\| \xi(h) \|\\
                &=\| h \|^2\alpha(h)
            \end{align}
        \end{subequations}
    pour une certaine fonction \( \alpha\colon \eR^n\to \eR\) qui tend vers une constante lorsque \( h\to 0\). 

    En posant \( C=\lim_{h\to 0}\alpha(h) \) nous avons la majoration
    \begin{equation}        \label{EqSYiuYF}
        \| F(x)-a \|\leq C\| x-a \|^2.
    \end{equation}
    Nous pouvons également supposer que \( C>1\). Affin de prouver la vitesse de convergence \eqref{EqtkiDXt}, nous allons encore redéfinir \( r\) en demandant \( r<1/C^2\). De cette manière nous avons
    \begin{equation}
        \| x_0-a \|\leq \frac{1}{ C^2 }
    \end{equation}
    et
    \begin{equation}
        \| x_{n+1}-a \|=\| F(x_n)-a \|\leq C\| x_n-a \|^2\leq C\big( C^{-1-2^n} \big)^2=C^{-1-2^{n+1}}.
    \end{equation}

\end{proof}

\begin{remark}
    La valeur de la constate \( C\) a été fixée par l'équation \eqref{EqSYiuYF}. Certes nous pouvons toujours choisir \( C\) plus grand affin d'augmenter la vitesse de convergence, mais le point de départ \( x_0\) devant être dans une boule de taille \( 1/C^2\) autour de \( a\), demander \( C \) plus grand revient à demander un point de départ plus précis.
\end{remark}

%+++++++++++++++++++++++++++++++++++++++++++++++++++++++++++++++++++++++++++++++++++++++++++++++++++++++++++++++++++++++++++
\section{Prolongement de fonctions et complétion d'espaces métriques}
%+++++++++++++++++++++++++++++++++++++++++++++++++++++++++++++++++++++++++++++++++++++++++++++++++++++++++++++++++++++++++++

Sources : \cite{RasclAnaFonc}

\begin{lemma}   \label{LemdCOMQM}
    Soit \( E\), un espace vectoriel normé complet et \( (A_n)\) une suite emboité de fermés non vides dont le diamètre tend vers zéro. Alors l'intersection \( \bigcap_{n\in \eN}A_n\) contient exactement un point.
\end{lemma}

\begin{proof}
    Si l'intersection contenait deux points distincts \( a\) et \( b\), alors nous aurions \( \diam(A_n)\geq\| a-b \|\), ce qui contredirait la limite.

    Soit une suite \( (x_n)\) avec \( x_k\in A_k\) pour tout \( k\in \eN\). C'est une suite de Cauchy. En effet si \( \epsilon>0\), considérons \( N\) tel que \( \diam(A_N)<\epsilon\). Dans ce cas dès que \( n,m>N\) nous avons \( x_n,x_m\in A_{N}\) et donc \( \| x_n-x_m \|\leq \epsilon\). La suite \( x_n\) converge donc vers un élément dans \( E\).

    Nous devons montrer que \( x\in A_k\) pour tout \( k\). La queue de suite \( (x_n)_{n\geq k}\) est une suite de Cauchy dans \( A_k\) qui converge donc vers un élément de \( A_k\) (ici nous utilisons le fait que \( A_k\) est fermé). Par unicité de la limite, cette dernière doit être \( x\). Par conséquent \( x\in\bigcap_{n\in \eN}A_n\).
\end{proof}

\begin{theorem}[\cite{MaurayAnalSpec}]      \label{ThoCaMpKO}
    Soient \( X\) et \( Y\) des espaces vectoriels normés. Pour une application linéaire \( f\colon X\to Y\), les assertions suivantes sont équivalentes :
    \begin{enumerate}
        \item
            \( f\) est continue sur \( X\),
        \item
            \( f\) est continue en un point de \( X\),
        \item
            \( f\) est bornée.
    \end{enumerate}
\end{theorem}

\begin{proposition} \label{PropTTiRgAq}
    Soit \( X\) un espace normé et \( A\) une partie dense de \( X\). Soit \( F\) un espace de Banach. Toute application linéaire continue \( f\colon A\to F\) se prolonge de façon unique en une application linéaire continue \( \tilde f\colon X\to F\). De plus \( \| \tilde f \|=\| f \|\).
\end{proposition}

\begin{proof}
    Soit \( x\in X\) et la suite d'ensemble
    \begin{equation}
        A_n=\{ y\in A\tq \| x-y \|\leq 2^{-n}\}.
    \end{equation}
    Étant donné que \( A\) est dense, ces ensembles sont tous non vides. De plus \( \diam A_n\to 0\) parce que si \( y,y'\in A_n\) alors
    \begin{equation}
        \| y-y' \|\leq\| y-x \|+\| x-y' \|\leq 2^{-n+1}.
    \end{equation}
    Vu que \( f\) est bornée, la suite d'ensembles \( f(A_n)\) est une suite emboitée d'ensembles non vides de \( X\). De plus leur diamètre tend vers zéro. En effet si \( z,z'\in f(A_n)\), nous posons \( z=f(y)\), \( z'=f(y')\) et nous avons
    \begin{equation}
        \| z-z' \|\leq \| f(y)-f(x) \|+\| f(x)-f(y') \|\leq \| f \|\big( \| y-x \|+\| x-y' \| \big),
    \end{equation}
    ce qui montre que \( \diam f(A_n)\leq \| f \|2^{-n+1}\).  Notons que nous avons utilisé la linéarité de \( f\). Par le lemme \ref{LemdCOMQM}, l'intersection \( \bigcap_{n\in \eN}\overline{ f(A_n) }\) contient exactement un point. Nous posons
    \begin{equation}
        S(x)=\bigcap_{n\in \eN}\overline{ f(A_n) }.
    \end{equation}
    Nous allons montrer que l'application \( x\mapsto S(x)\) ainsi définie est l'application que nous cherchons. 

    Nous commençons par montrer que pour toute suite \( y_k\to x\) avec \( y_k\in A\) nous avons 
    \begin{equation}    \label{EqBnRZxW}
        f(y_k)\to S(x).
    \end{equation}
    Pour cela nous considérons \( n_0\in \eN\) et \( k_0\) tel que \( y_{k_0}\in A_{n_0}\). Avec cela nous avons
    \begin{equation}
        \| f(y_k)-S(x) \|\leq \diam(A_{n_0})\leq \| f \|2^{-n_0+1}.
    \end{equation}
    Pour montrer que \( S\) est linéaire, nous considérons deux suites dans \( A\) : \( y_k\to x\) et \( y'_k\to x'\) ainsi que la somme \( y_k+y'k\to x+x'\). Nous écrivons la relation \eqref{EqBnRZxW} pour ces trois suites :
    \begin{subequations}
        \begin{align}
            f(y_k)\to S(x)\\
            f(y'_k)\to S(x')\\
            f(y_k+y'_x)\to S(x+x').
        \end{align}
    \end{subequations}
    Cependant, étant donné que \( f\) est linéaire, pour tout \( k\) nous avons \( f(y_k+y'_k)=f(y_k)+f(y'_k)\) et par conséquent
    \begin{equation}
        f(y_k+y'_k)\to S(x)+S(x').
    \end{equation}
    Par unicité de la limite, \( S(x+x')=S(x)+S(x')\). Le même genre de raisonnement montre que \( S(\lambda x)=\lambda S(x)\). L'application \( S\) est donc linéaire.

    En ce qui concerna la continuité, nous avons
    \begin{equation}
            \| S(x) \|=\lim\| f(y_k) \|\leq \| f \|\| \lim y_k \|=\| f \|\| x \|,
    \end{equation}
    donc \( \| S \|\leq \| f \|\), c'est à dire que \( S\) est borné et donc continue parce que linéaire (théorème \ref{ThoCaMpKO}).

    Nous montrons maintenant que \( S\) prolonge \( f\). Si \( x\in A\), alors nous avons \( \bigcap_{n\in \eN}f(A_n)=f(x)\), et donc \( S(x)=f(x)\). Cela montre du même coup que \( \| f \|\leq \| S \|\) et que par conséquent \( \| f \|=\| S \|\).

    Passons à la partie sur l'unicité. Soient donc \( S\) et \( T\), deux prolongements continus de \( f\) sur \( X\). Soit \( x\in X\) et une suite \( x_n\to x\) dans \( A\). Par continuité nous avons \( T(x_n)\to T(x)\) et \( S(x_n)\to S(x)\). Étant donné que par ailleurs pour tout \( n\) nous avons \( S(x_n)=T(x_n)\), l'unicité de la limite montre que \( T(x)=S(x)\).
\end{proof}

\begin{definition}
    Soit une application \( f\colon X\to Y\). Le \defe{module de continuité}{module!de continuité} de \( f\) est la fonction \( \omega_f\colon \eR\to \eR\) définie comme suit. On pose \( \omega_f(x)=0\) pour \( x\leq 0\) et si \( h>0\),
    \begin{equation}
        \omega_f(h)=\sup_{\substack{x,y\in X\\d_X(x,y)<h}} d_Y\big( f(x),f(y) \big).
    \end{equation}
\end{definition}

\begin{lemma}   \label{LemeERapq}
    Une fonction \( f\) est uniformément continue si et seulement si son module de continuité est continu en zéro.
\end{lemma}

Dans la même veine que la proposition \ref{PropTTiRgAq} nous avons ce résultat.
\begin{theorem}[\cite{ZHDEie}]      \label{ThoPVFQMi}
    Soient \( E\) et \( F\), deux espaces métriques complets ainsi que \( A\) dense dans \( E\). Si \( u\colon A\to F\) est uniformément continue, alors elle se prolonge de façon unique en une fonction continue \( \tilde u\colon E\to F\). De plus ce prolongement est uniformément continu.
\end{theorem}
\index{densité!prolongement}
\index{prolongement!par densité}
\index{complétude}
\index{prolongement!de fonctions}

\begin{proof}
    Soit \( x\in E\setminus A\) et une suite \( (x_n)\) contenue dans \( A\) et convergente vers \( x\). Nous voulons définir
    \begin{equation}
        \tilde u(x)=\lim_{n\to \infty} u(x_n)
    \end{equation}
    mais pour ce faire nous devons prouver que la suite \( \big( u(x_n) \big)\) converge dans \( F\) et que la limite ne dépend pas de la suite choisie parmi les suites de \( A\) qui convergent (dans \( E\)) vers \( x\).

    Commençons par montrer que \( \big( u(x_n) \big)\) est de Cauchy dans \( F\). Pour cela nous prenons \( \epsilon>0\) et \( \eta>0\) telle que \( d_E(a,b)<\eta\) implique \( d_F\big( u(a),u(b) \big)<\epsilon\) (uniforme continuité de \( u\)). Après, il suffit de choisir \( N\) tel que pour tout \( n,m>N\) nous ayons \( d(x_m,x_n)<\eta\) (parce que \( u_n\) est de Cauchy). Avec tout ça nous avons 
    \begin{equation}
        d_F\big( u(x_m),u(x_n) \big)<\epsilon,
    \end{equation}
    ce qui signifie que \( \big( u(x_n) \big)\) est de Cauchy et donc convergente dans \( F\). 
    
    Nous voulons montrer maintenant que si \( (x_n)\) et \( (y_n)\) sont deux suites dans \( A\) convergentes vers \( x\) alors \( \lim_{n\to \infty} u(x_n)=\lim_{n\to \infty} u(y_n)\). Pour cela nous considérons la suite \( z=(x_1,y_1,x_2,y_2,\ldots)\). Nous avons évidemment \( z_n\to x\), et donc \( u(z_n)\) converge dans \( F\) par ce qui a été dit plus haut. Mais \( u(x_n)\) et \( u(y_n)\) en sont deux sous-suites convergentes. Donc leurs limites sont égales.

    Il reste à montrer que ce \( \tilde u\) est continue et uniformément continue. Pour cela nous utilisons le module de continuité et le lemme \ref{LemeERapq}. Étant donné que \( \tilde u\) prolonge \( u\) nous avons 
    \begin{equation}        \label{EqFRYqON}
        \omega_{\tilde u}(h)\geq \omega_u(h).
    \end{equation}
    Soit \( h>0\) et \( \epsilon>0\); soit aussi \( x,y\in E\) tels que \( d(x,y)<h\). Nous prenons des suites \( (a_n)\to x\) et \( (y_n)\to y\) tout en choisissant \( n\) assez grand pour avoir \( d_E(a_n,b_n)<h\). Nous avons
    \begin{equation}
        d_F\big( \tilde u(x),\tilde u(y) \big)\leq d_F\big( \tilde u(x),u(a_n) \big)+d\big( u(a_n),u(b_n) \big)+d_F\big( u(b_n),\tilde u(y) \big).
    \end{equation}
    Si \( n\) est assez grand, par construction de \( \tilde u\), le premier et le dernier terme sont plus petits que \( \epsilon\). Par définition du module de continuité nous avons d'autre part \( d_F\big( u(a_n),u(b_n) \big)\leq \omega_u(h)\). Du coup
    \begin{equation}
        d_F\big( \tilde u(x),\tilde u(y) \big)\leq \omega_u(h)+2\epsilon.
    \end{equation}
    Si nous prenons le supremum sur les \( x\) et \( y\) vérifiant \( d_E(x,y)<h\), à gauche nous obtenons \( \omega_{\tilde u}(h)\) tandis que le membre de droite ne dépend pas de \( x\) et\( y\). Donc pour tout \( \epsilon\), nous avons
    \begin{equation}
        \omega_{\tilde u}(h)\leq \omega_u(h)+2\epsilon.
    \end{equation}
    En comparaison avec \eqref{EqFRYqON}, nous trouvons
    \begin{equation}
        \omega_{\tilde u(h)}\leq \omega_u(h).
    \end{equation}
    Les fonctions \( u\) et \( \tilde u\) ayant le même module de continuité, le lemme \ref{LemeERapq} nous enseigne que l'une est uniformément continue si et seulement si l'autre l'est. Vu que \( u\) est uniformément continue par hypothèse, le prolongement \( \tilde u\) est uniformément continu.
\end{proof}

\begin{definition}
    Un \defe{plongement}{plongement} de l'espace topologique \( X\) dans \( Y\) est une application \( f\colon X\to Y\) telle que \( f\colon X\to f(X)\) soit un homéomorphisme.
\end{definition}

\begin{theorem}\label{ThoPHllyoB}
    Soit \( \tilde M\) un espace métrique complet et une application isométrique
    \begin{equation}
        f\colon A\to \tilde M
    \end{equation}
    où \( A\) est une partie dense d'un espace métrique \( M\) (pas spécialement complet). Alors \( f\) accepte une une unique extension isométrique
    \begin{equation}
        \tilde f\colon M\to \tilde M
    \end{equation}
    
    Supposons de plus que \( M\) soit complet\footnote{Il me semble que cette hypothèse manque dans \cite{JBRzHwn}.}. Alors \( \tilde f\colon M\to\tilde M\) est une bijection si et seulement si \( f(A)\) est dense dans \( \tilde M\).
\end{theorem}

\begin{proof}
    Nous commençons par prouver l'unicité. Soit \( \tilde f_1\) et \( \tilde f_2\), deux extensions de \( f\) et \( x\in M\). Si \( (a_n)\) est une suite dans \( A\) convergeant vers \( x\) (possible parce que \( A\) est dense dans \( M\)), alors nous avons
    \begin{equation}
        \tilde f_1(a_n)=\tilde f_2(a_n)
    \end{equation}
    et donc \( \tilde f_1(x)=\tilde f_2(x)\) par continuité (une application isométrique est continue (proposition \ref{PropLYMgVMJ})).

    Nous démontrons à présent l'existence.
    
    \begin{subproof}
    \item[Construction de \( \tilde f\)]
    Soit \( x\in M\) et \( (a_n)\) une suite dans \( A\) qui converge vers \( x\). Nous définissons
    \begin{equation}    \label{EqHEembqy}
        \tilde f(x)=\lim_{k\to \infty} f(a_k).
    \end{equation}
    Note : nous pouvons prouver que cette définition ne dépend pas du choix de la suite \( (a_n)\) convergeant vers \( x\), mais ce serait superflu parce que nous avons déjà prouvé l'unicité de \( \tilde f\). Par contre nous devons expliquer pourquoi la limite du membre de droite de \eqref{EqHEembqy} existe dans \( \tilde M\). D'abord la suite \( (a_n)\) est de Cauchy parce qu'elle est convergente (attention : \( M\) n'étant pas complet le fait d'être de Cauchy n'implique pas la convergence). Donc, étant donné que \( f\) est une isométrie, la suite \( \big( f(a_n) \big)\) est de Cauchy dans \( \tilde M\). Or ce dernier étant complet, la suite des images converge.
    
    Montrons que cette application \( \tilde f\colon M\to \tilde M\) répond à la question.

\item[\( \tilde f\) est isométrique]

        Soient \( a,b\in M\) et des suites dans \( A\) convergeant vers eux : \( a_n\to a\), \( b_n\to b\). Nous avons, par continuité de l'application distance,
        \begin{subequations}
            \begin{align}
                d\big( \tilde f(a),\tilde f(b) \big)&=\lim_{k\to \infty} d\big( \tilde f(a_k),\tilde f(b) \big)\\
                &=\lim_{k\to \infty}\lim_{l\to \infty}  d\big( \tilde f(a_k),\tilde f(b_l) \big)\\
                &=\lim_{k\to \infty}\lim_{l\to \infty}  d\big(  f(a_k), f(b_l) \big)\\
                &=\lim_{k\to \infty}\lim_{l\to \infty}  d\big( a_k),b_l\big)\\
                &=d(a,b).
            \end{align}
        \end{subequations}
        Cela prouve que \( \tilde f\) est une isométrie.
        
        Pour la suite nous supposons que \( M\) est complet. Notons tout de suite que \( \tilde f\) est injective parce qu'elle est isométrique.

    \item[Bijection (premier sens)]

        Nous supposons que \( \tilde f\colon M\to \tilde M\) est une bijection. Par l'absurde nous supposons que \( f(A)\) n'est pas dense dans \( \tilde M\), c'est à dire que nous avons un point \( x\in \tilde M\) et une boule n'intersectant par \( f(A)\) :
        \begin{equation}
            B(x,r)\cap f(A)=\emptyset.
        \end{equation}
        Étant donné que \( \tilde f\) a pour image des limites de suites dans \( f(A)\), l'image de \( \tilde f\) est contenue dans \( \overline{ f(A) }\). Donc si \( \tilde f\) est surjective, c'est que \( \tilde M\subset \overline{ f(A) }\) et donc que \( \overline{ f(A) }=\tilde M\). Cela prouve que si \( \tilde f\) est bijective, alors \( f(A)\) est dense dans \( \tilde M\).


    \item[Bijection (l'autre sens)]

        Nous supposons que \( \overline{ f(A) }=\tilde M\) et nous devons prouver que \( \tilde f\) est surjective. Soit \( x\in \tilde M\) et \( f(a_n)\) une suite dans \( f(A)\) qui converge vers \( x\); une telle suite existe parce que \( f(A)\) est dense dans \( \tilde M\). Cette suite est de Cauchy dans \( \tilde M\) parce que dans un espace métrique, une suite convergente est de Cauchy. La suite \( (a_n)\) est elle-même également de Cauchy parce que
        \begin{equation}
            d(a_n,a_m)=d\big( f(a_n),f(a_m) \big).
        \end{equation}
        Étant donné que \( (a_n)\) est de Cauchy dans \( M\), elle converge vers un élément que nous nommons \( a\in M\). Par continuité de \( f \) nous avons alors
        \begin{equation}
            \tilde f(a)=\lim_{k\to \infty} f(a_k)=x.
        \end{equation}
        Cela prouve que \( x\) est bien dans l'image de \( \tilde f\) et donc que \( \tilde f\) est surjective.
    \end{subproof}
\end{proof}

Une conséquence du théorème de prolongement est le théorème suivant qui permet de compléter un espace métrique.
\begin{theorem}[Complétion d'un espace métrique\cite{DOVXZwP,JBRzHwn}]\label{ThoKHTQJXZ}  \index{espace!complet}

    Tout espace métrique se plonge par une isométrie à image dense dans un espace métrique complet. De plus ce dernier est unique à isométrie près.

    Plus précisément, soit \( (M,d)\) un espace métrique. Il existe un espace métrique complet \( \tilde M\) muni d'un plongement isométrique \( \varphi\colon M\to \tilde M\) tel que \( \varphi(M)\) soit dense dans \( \tilde M\).

    Ce complété\index{complété!d'un espace métrique} de \( M\) est unique au sens suivant. Si \( \tilde M_1\) et \( \tilde M_2\) sont deux espaces métriques complets munis de plongements isométriques \( f_i\colon M\to \tilde M_1\) dont les images sont denses, alors il existe une bijection isométrique \( \phi\colon \tilde M_1\to \tilde M_2\) telle que \( \phi\circ f_1=f_2\).
\end{theorem}
\index{densité}
\index{complétude}

\begin{proof}
    Nous ne prouvons que l'existence.

    Soit \( C_M\) l'ensemble des suites de Cauchy de \( M\). Nous définissons
    \begin{equation}
        \begin{aligned}
            f\colon C_M\times C_M&\to \eR \\
            u,v&\mapsto \lim_{n\to \infty} d(u_n,v_n).
        \end{aligned}
    \end{equation}
    Notre première tâche est de nous assurer que cela est bien défini, c'est à dire que la limite existe toujours. En effet, si \( u\) et \( v\) sont des suites de Cauchy dans \( M\), nous avons
    \begin{equation}
        \left| d(u_n,v_n)-d(u_m,v_m) \right| \leq d(u_n,v_n)+d(u_m,v_m)\leq 2\epsilon
    \end{equation}
    dès que \( m\) et \( n\) sont assez grand. Cela prouve que la suite \( n\mapsto d(u_n,v_n)\) est de Cauchy dans \( \eR\). Par complétude de \( \eR\), elle converge\footnote{Ici nous utilisons la complétude de \( \eR\). Cette dernière doit donc être démontrée indépendamment. De plus nous ne pouvons pas définir \( \eR\) comme étant le complété de \( \eQ\) en utilisant ce théorème.}. 
    %TODO : lorsque ce sera fait, il faudra mettre ici une référence vers le théorème qui donne la complétude de la droite réelle.
    
    Nous considérons la relation d'équivalence \( u\sim v\) si et seulement si \( f(u,v)=0\). Nous posons \( \tilde M=C_M/\sim\) et nous y mettons la distance
    \begin{equation}    \label{EqDDLNRNF}
        d( [u],[v]  )=f(u,v)
    \end{equation}
    et nous devons encore vérifier que cela est bien défini. Prenons \( u'\sim u\) et \( v'\sim v\). Alors nous avons
    \begin{equation}
        d(u'_n,v'_n)\leq d(u'_n,u_n)+d(u_n,v_n)+d(v_n,v'_n),
    \end{equation}
    et donc 
    \begin{equation}
        d(u',v')=\lim_{n\to \infty} d(u'_n,v'n)\leq \lim_{n\to \infty} d(u_n,v_n)=d(u,v).
    \end{equation}
    Le même argument en inversant les primes et les non primes montre l'inégalité inverse. Donc \( d(u,v)=d(u',v')\) dans \( C_M\), et donc la distance \eqref{EqDDLNRNF} est bien définie sur \( \tilde M\).

    Affin de s'assurer que \( \tilde M\) répond bien à la question du théorème, il faut encore démontrer les points suivants :
    \begin{itemize}
        \item \( M\) se plonge isométriquement dans \( \tilde M\).
        \item l'image de \( M\) par le plongement est dense dans \( \tilde M\).
        \item \( \tilde M\) est complet.
    \end{itemize}

    Nous allons maintenant considérer l'application
    \begin{equation}
        \begin{aligned}
            \varphi\colon M&\to \tilde M \\
            x&\mapsto \text{la classe de la suite constante \( x\).} 
        \end{aligned}
    \end{equation}
    \begin{subproof}
        \item[Plongement isométrique]
        Nous allons montrer que cela est une isométrie bijective et que \( \varphi(M)\) est dense dans \( \tilde M\). Le fait que \( \varphi\) soit bijective entre \( M\) et \( \varphi(M)\) est évident. C'est une isométrie parce que
        \begin{equation}
            d\big( \varphi(x),\varphi(y) \big)=\lim_{n\to \infty} d\big(\varphi(x)_n,\varphi(y)_n\big)=d(x,y).
        \end{equation}
        
    \item[Densité]

        Soit \( [u]\in \tilde M\). Tous les termes \( u_n\) sont des éléments de \( M\). Nous considérons la suite dans \( \varphi(M)\) donnée par
        \begin{equation}
            a_n=\varphi(u_n)
        \end{equation}
        Chaque \( a_n\) est un élément\footnote{À partir de maintenant nous n'écrivons plus explicitement la classe d'équivalence.} de \( \tilde M\). Montrons que \( (a_n)\) converge dans \( \tilde M\) vers \( u\). Nous avons
        \begin{subequations}
            \begin{align}
                d(a_n,u)&=\lim_{k\to \infty} d\big( (a_n)_k,u_k \big)\\
                &=\lim_{k\to \infty} d(u_n,u_k)\\
                &=d(u_n,\ell)
            \end{align}
        \end{subequations}
        en notant \( \ell\) la limite de la suite \( (u_n)\). Ici nous avons utilisé le fait que la fonction distance était continue pour l'inverser avec la limite, par le théorème \ref{ThoLimSuite}. Nous avons alors
        \begin{equation}
            \lim_{n\to \infty} d(a_n,[u])=\lim_{n\to \infty} d(u_n,\ell)=0.
        \end{equation}
        
    \item[Complétude]
        Nous passons maintenant à la preuve du fait que \( \tilde M\) est complet. Soit \( (y_n)\) une suite de Cauchy dans \( \tilde M\). Soit \( \epsilon>0\); nous définissons \( K(n)\) par
        \begin{equation}
            d\big( (y_n)_k,(y_n)_l \big)<\epsilon
        \end{equation}
        dès que \( k,l\geq K(n)\). Cette définition fonctionne parce que pour chaque \( n\), \( y_n\) est une suite de Cauchy dans \( M\). Nous posons
        \begin{equation}
            x_n=(y_n)_{K(n)}\in M
        \end{equation}
        et nous allons montrer que \( (x_n)\) est de Cauchy dans \( M\) --donc est un élément de \( \tilde M\)-- et que \( y_k\to (x_n)\) dans \( \tilde M\). 
        
        Nous commençons par montrer que \( (x_n)\) est de Cauchy dans \( M\). Nous avons
        \begin{subequations}
            \begin{align}
                d(x_n,x_m)&=d\big( (y_n)_{K(n)},(y_m)_{K(m)} \big)\\
                &\leq d\big( (y_n)_{K(n)},(y_n)_l \big)+d\big( (y_n)_{l},(y_m)_l \big)+d\big( (y_m)_{l},(y_m)_{K(m)} \big)
            \end{align}
        \end{subequations}
        Nous choisissons \( n,m\) tels que \( d(y_n,y_m)<\epsilon\), ce qui nous permet de choisir \( l\) de telle façon à avoir \( d\big( (y_n)_k,(y_m)_k \big)<\epsilon\) pour tout \( k\geq l\). De plus, quitte à encore augmenter \( l\), nous supposons que \( l>K(m)\) et \( l>K(m)\). Avec ces choix nous voyons que \( d(x_n,x_m)<3\epsilon\), ce qui signifie que la suite \( (x_n)\) est de Cauchy dans \( M\).

        En ce qui concerne la convergence \( y_n\to (x)\), on a 
        \begin{subequations}
            \begin{align}
                d\big( y_n,(x) \big)&=\lim_{k\to \infty} d\big( (y_n)_k,(y_k)_{K(k)} \big)\\
                &\leq \lim_{k\to \infty} d\big( (y_n)_k,(y_n)_l \big)+\lim_{k\to \infty} d\big( (y_n)_l,(y_k)_{l} \big)+\lim_{k\to \infty} d\big( (y_k)_l,(y_k)_{K(k)} \big)    \label{EqABmqNwo}
            \end{align}
        \end{subequations}
        Nous devons trouver un \( n\) tel que si \( k\) est suffisamment grand, le tout est majoré par \( \epsilon\). Voici nos choix :
        \begin{itemize}
            \item \( n\) tel que \( d(y_n,y_m)<\epsilon\) dès que \( m\geq n\),
            \item \( k>n\),
            \item \( k>K(n)\),
            \item \( l>k\),
            \item \( l>K(k)\),
            \item \( l\) suffisamment grand pour que \( d\big( (y_n)_l,(y_k)_l \big)<\epsilon\).
        \end{itemize}
        Avec tous ces choix, les trois termes de \eqref{EqABmqNwo} sont plus petits que \( \epsilon\).

        Ceci prouve que \( \tilde M\) est complet.
    \end{subproof}
\end{proof}
%TODO : prouver l'unicité; c'est un grande partie le théorème précédent.
%TODO : la construction de la droite réelle et sa complétude.

\begin{theorem}[Principe du prolongement analytique]\label{ThoAVBCewB}
    Soit \( U\) un ouvert connexe. Si deux fonctions analytiques coïncident sur un sous-ensemble \( D\) de \( U\) contenant un point d'accumulation dans \( U\), alors elles sont égales sur \( U\).
\end{theorem}
\index{principe!prolongement analytique}
\index{connexité!prolongement analytique}
\index{prolongement!analytique}

%+++++++++++++++++++++++++++++++++++++++++++++++++++++++++++++++++++++++++++++++++++++++++++++++++++++++++++++++++++++++++++ 
\section{Théorème de Berstein}
%+++++++++++++++++++++++++++++++++++++++++++++++++++++++++++++++++++++++++++++++++++++++++++++++++++++++++++++++++++++++++++

\begin{theorem}[Théorème de Bernstein\cite{KXjFWKA}]
    Soit \( f\in C^0\big( \mathopen[ 0 , 1 \mathclose],\eC \big)\) et son module de continuité
    \begin{equation}
        \begin{aligned}
            \omega\colon \mathopen[ 0 , 1 \mathclose]&\to \eR \\
            h&\mapsto \sup\{ | f(u)-f(v) |\tq | u-v |< h \}. 
        \end{aligned}
    \end{equation}
    Pour \( n\geq 0\) nous définissons le \( n\Ieme\) \defe{polynôme de Bernstein}{polynôme!de Bernstein} de \( f\) par
    \begin{equation}
        B_n(f)(x)=\sum_{k=0}^{n}\binom{ n }{ k }x^k(1-x)^{n-k}f\left( \frac{ k }{ n } \right).
    \end{equation}
\end{theorem}
<++>

%+++++++++++++++++++++++++++++++++++++++++++++++++++++++++++++++++++++++++++++++++++++++++++++++++++++++++++++++++++++++++++
					\section{Un petit extra}
%+++++++++++++++++++++++++++++++++++++++++++++++++++++++++++++++++++++++++++++++++++++++++++++++++++++++++++++++++++++++++++

Soit $f$ une fonction de $\eR$ dans $\eR$. Supposons que 
\begin{enumerate}

\item		\label{ItemExtrai}
$f(1)=1$,

\item		\label{ItemExtraii}
$f(x+y)=f(x)+f(y)$ pour tout réels $x$ et $y$.

\end{enumerate}
Nous pouvons montrer\footnote{et toi, tu le peux ?} que la seule fonction {\it continue} qui possède ces propriétés est la fonction identité $f(x)=x$ pour tout $x\in\eR$.

De la même manière, il est aisé de voir que les seules applications linéaires de $\Rn$ dans $\Rn$ sont de la forme 
\begin{equation}
	f(x)=Ax
\end{equation}
pour une constante réelle $A$. Une question naturelle qu'on peut alors se poser est la suivante: 
\begin{quote} 
	Est-il possible de définir une fonction non continue ayant les propriétés \ref{ItemExtrai} et \ref{ItemExtraii} ?
\end{quote}
En fait, il est possible de démontrer que si $E$ est un espace vectoriel de dimension finie, alors toute application linéaire $f:E\rightarrow  F$ (où $F$ est un espace vectoriel) sera continue. Ceci ne reste plus vrai si l'espace vectoriel $E$ est de dimension infinie. Donc une manière de trouver une réponse positive à la question posée plus haut, serait de voir $\Rn$ comme espace vectoriel de dimension infinie. Après un peu de réflexion, la réponse est venue à nous (merci à Nicolas et à Samuel). 

Si nous admettons l'\href{http://fr.wikipedia.org/wiki/Axiome_du_choix}{axiome du choix}, alors nous pouvons appliquer le théorème de Zorn et nous savons que tout espace vectoriel admet une base. En particulier, l'ensemble des réels vu comme espace vectoriel sur $\Qn$ admet une base, i.e. $\exists (e_i)_{i\in I}$  des éléments de $\Rn$ tels que tout réel s'écrit comme combinaison linéaire à coefficients rationnels  de ces $e_i$, i.e.
\begin{equation}
	\forall r \in \Rn, \exists (\lambda_i)_{i\in I} \text{ des éléments de } \Qn \text{ tels que  } r = \sum_{i\in I} \lambda_i e_i.
\end{equation}
Utilisons cette base pour définir une fonction $h$ de la manière suivante.
\begin{equation}
\forall i \in I, \mbox{ on définit } h(e_i) = \alpha_i
\end{equation}
 où les $\alpha_i$ doivent être bien choisis dans $\Rn$. Pour satisfaire la propriété \ref{ItemExtrai}, choisissons sans perte de généralité $e_1 = 1$ et $h(e_1) = 1$.  Ajoutons à cette propriété la linéarité en imposant que 
\begin{equation}
h(\sum \lambda_i e_i) = \sum \lambda_i \alpha_i.
\end{equation}
Les équations (1) et (2) nous permettent de voir que, moyennant le choix des $\alpha_i$, la fonction $h$ est bien définie sur $\Rn$ et linéaire. Il est clair que si nous prenons par exemple
$$\alpha_i=e_i\;\forall i \in I$$ 
nous obtenons que la fonction $h$ est en fait la fonction identité sur $\Rn$. Par contre, si nous définissons la fonction $h$ comme satisfaisant la propriété (2) et si nous choisissons les $\alpha_i$ dans (1) de la manière suivante 
\begin{equation}
	\begin{aligned}[]
		h(e_1)	&= e_2\\
		h(e_2)	&= e_1\\
		h(e_i)	&= e_i	&&\forall i\in I\setminus\{ 1,2 \}
	\end{aligned}
\end{equation}
alors la fonction ainsi obtenue est linéaire et bien définie mais n'est plus l'identité. Donc nous avons trouvé une application linéaire de $\Rn$ dans $\Rn$ qui n'est pas continue.  

\begin{exercice}
 Trouver d'autres exemples d'applications linéaires non continues (pas nécessairement des transformations de $\Rn$). 		
\end{exercice}
