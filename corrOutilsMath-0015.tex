% This is part of Exercices et corrigés de CdI-1
% Copyright (c) 2011
%   Laurent Claessens
% See the file fdl-1.3.txt for copying conditions.

\begin{corrige}{OutilsMath-0015}

	Un point appartient au cercle si et seulement si sa distance à l'origine vaut $3$. L'équation du cercle est donc
	\begin{equation}
		r=3
	\end{equation}
	et pas de contraintes sur la coordonnée angulaire $\theta$.

	Morale : les coordonnée polaires sont très pratiques pour travailler sur des cercles centrés en l'origine.

\end{corrige}
