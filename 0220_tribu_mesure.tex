% This is part of Mes notes de mathématique
% Copyright (c) 2011-2014
%   Laurent Claessens
% See the file fdl-1.3.txt for copying conditions.

%+++++++++++++++++++++++++++++++++++++++++++++++++++++++++++++++++++++++++++++++++++++++++++++++++++++++++++++++++++++++++++
\section{Théorie de la mesure}
%+++++++++++++++++++++++++++++++++++++++++++++++++++++++++++++++++++++++++++++++++++++++++++++++++++++++++++++++++++++++++++
\label{SecSLOooeMaig}

\begin{definition}[\cite{MesureLebesgueLi}] \label{DefUMWoolmMaf}
    Une \defe{mesure extérieure}{mesure!extérieure} sur un ensemble \( S\) est une application \( m^*\colon \partP(S)\to \mathopen[ 0 , \infty \mathclose]\) telle que
    \begin{enumerate}
        \item
            \( m^*(\emptyset)=0\),
        \item
            Si \( A\subset B\) dans \( S\) alors \( m^*(A)\leq m^*(B)\)
        \item   \label{ItemARKooppZfDaiii}
            Si les \( A_n\) sont des parties de \( S\) alors
            \begin{equation}    \label{EqZLMooSxvaL}
                m^*\big( \bigcup_{n\in \eN}A_n \big)\leq \sum_{n\in \eN}m^*(A_n).
            \end{equation}
    \end{enumerate}
\end{definition}
La différence avec une mesure est que nous ne demandons pas que \eqref{EqZLMooSxvaL} soit une égalité lorsque les \( A_n\) sont disjoints.

%--------------------------------------------------------------------------------------------------------------------------- 
\subsection{Mesure et algèbre de parties}
%---------------------------------------------------------------------------------------------------------------------------

\begin{definition}[Algèbre de parties\cite{MesureLebesgueLi}]   \label{DefTCUoogGDud}
    Soit \( S\), un ensemble. Une classe \( \tribD\) de parties de \( S\) est une \defe{algèbre de parties}{algèbre!de parties} de \( S\) si
    \begin{enumerate}
        \item
            \( S\in\tribD\) et \( \emptyset\in\tribD\),
        \item
            si \( A\in\tribD\) alors \( A^c\in\tribD\),
        \item
            si \( A,B\in\tribD\) alors \( A\cup B\in\tribD\).
    \end{enumerate}
\end{definition}

Les algèbre de parties ne sont pas des classes si sauvages que ça; en témoigne le lemme suivant.
\begin{lemma}   \label{LemBFKootqXKl}
    Une algèbre de partie est stable par intersection (finie) et par différence ensembliste.
\end{lemma}

\begin{proof}
    Il suffit de remarquer que \( A\cap B=\big( A^c\cup B^c \big)^c\) et que \( A\setminus B=A\cap B^c\).
\end{proof}

\begin{definition}[Mesure sur une algèbre de parties]
    Soit \( S\) un ensemble et \( \tribD\) une algèbre de parties de \( S\). Une \defe{mesure}{mesure!sur algèbre de partie} sur \( (S,\tribD)\) est une application \( \mu\colon \tribD\to \mathopen[ 0 , \infty \mathclose]\) telle que
    \begin{enumerate}
        \item
            \( \mu(\emptyset)=0\),
        \item
            Si \( A_n\in\tribD\) sont des ensembles deux à deux disjoints et tels que \( \bigcup_nA_n\in\tribD\) alors
            \begin{equation}
                \mu\big( \bigcup_nA_n \big)=\sum_n\mu(A_n).
            \end{equation}
    \end{enumerate}

    La mesure est \defe{finie}{mesure!finie!sur algèbre de partie} si \( \mu(S)<\infty\) et \( \sigma\)-finie si il existe une suite \( (S_n)\) dans \( \tribD\) telle que \( S=\bigcup_nS_n\) et \( \mu(S_n)<\infty\).
\end{definition}

\begin{lemma}[\cite{MesureLebesgueLi}]  \label{LemZQUooMdCpq}
    Si \( \tribD\) est une algèbre de parties de \( S\) et si \( \mu\) est une mesure sur \( (S,\tribD)\) alors
    \begin{enumerate}
        \item
            si \( A,B\in\tribD\) avec \( A\subset B\) alors \( \mu(A)\leq \mu(B)\)
        \item   \label{ItemMFUooWCPNnii}
            si \( A_n\in\tribD\) et \( \bigcup_nA_n\in\tribD\) alors
            \begin{equation}
                \mu\big( \bigcup_nA_n \big)\leq\sum_n\mu(A_n).
            \end{equation}
    \end{enumerate}
\end{lemma}
La propriété \ref{ItemMFUooWCPNnii} est la \( \sigma\)-\defe{sous-additivité}{sous-additivité!sur algèbre de parties}.

\begin{proof}
    Si \( A\subset B\) alors \( B=A\cup(B\setminus A)\) avec \( A\) et \( B\setminus A\) disjoints donc
    \begin{equation}
        \mu(B)=\mu(A)+\mu(B\setminus A)\geq \mu(A).
    \end{equation}
    
    Pour la seconde, on passe par les compléments deux à deux : nous posons
    \begin{subequations}
        \begin{numcases}{}
            B_0=\emptyset\\
            B_{n}=A_n\setminus \bigcup_{k<n}B_k.
        \end{numcases}
    \end{subequations}
    Ces ensembles sont deux à deux disjoints et \( \bigcup_nB_n=\bigcup_nA_n\in\tribD\), donc
    \begin{equation}
        \mu\big( \bigcup_nA_n \big)=\sum_n\mu\big( A_n\setminus\bigcup_{k<n}B_k \big)\leq \sum_n\mu(A_n),
    \end{equation}
    où nous avons utilisé la première partie du lemme.
\end{proof}

\begin{proposition}[Mesure extérieure à partir d'une algèbre de parties\cite{MesureLebesgueLi}]    \label{PropIUOoobjfIB}
    Soit \( \tribD\) une algèbre de partie sur l'ensemble \( S\) et \( \mu\) une mesure sur \( (S,\tribD)\). Alors l'application
    \begin{equation}    \label{EqRNJooQrcoa}
        \begin{aligned}
            \mu^*\colon \partP(S)&\to \mathopen[ 0 , +\infty \mathclose] \\
            X&\mapsto \inf\{ \sum_n\mu(A_n)\tq A_n\in\tribD,X\subset\bigcup_nA_n \} 
        \end{aligned}
    \end{equation}
    est une mesure extérieure\footnote{Définition \ref{DefUMWoolmMaf}.} sur \( S\) et pour tout \( A\in\tribD\) nous avons \( \mu^*(A)=\mu(A)\).
\end{proposition}

\begin{proof}
    \begin{subproof}
    \item[La définition est bonne]
    Notons d'abord que la définition est bonne : l'ensemble sur lequel l'infimum est pris n'est pas vide : il suffit de prendre \( A_1=S\) et \( A_{n\geq 2}=\emptyset\).
    \item[Le vide]
        D'abord \( \mu^*(\emptyset)=0\) parce que \( \emptyset\in\tribD\). Prendre \( A_n=\emptyset\).
    \item[Inégalité d'inclusion]

        Soient \( X\subset Y\) dans \( \partP(S)\). Si \( Y\subset A\) alors \( X\subset A\), donc
        \begin{equation}
            \{ \sum_n\mu(A_n)\tq A_n\in\tribD,Y\subset\bigcup_nA_n \} \subset\{ \sum_n\mu(A_n)\tq A_n\in\tribD,X\subset\bigcup_nA_n \} ,
        \end{equation}
        ce qui prouve que \( \mu^*(X)\leq \mu^*(Y)\).
    \item[Inégalité par union dénombrable]

        Soit \( (X_n)_{n\in \eN}\) une suite de parties de \( S\). Si il existe \( n_0\) tel que \( \mu^*(X_{n_0})=\infty\) alors nous avons automatiquement \( \sum_n\mu^*(X_n)=\infty\) et l'inégalité demandée est évidente parce que n'importe quoi est plus petit ou égal à \( \infty\). Nous supposons donc que \( \mu^*(X_n)<\infty\) pour tout \( n\). 

        Soit \( \epsilon>0\). Pour tout \( n\geq 1\) il existe une suite \( (B_k^{(n)})_{k\in \eN}\) dans \( \tribD\) tells que \( X_n\subset\bigcup_kB_k^{(n)}\) et
        \begin{equation}
            \mu^*(X_n)+\frac{ \epsilon }{ 2^n }\geq \sum_k\mu(B_k^{(n)}).
        \end{equation}
        Étant donné que
        \begin{equation}
            \bigcup_nX_n\subset\bigcup_n\big( \bigcup_kB_k^{(n)} \big),
        \end{equation}
        nous avons
        \begin{equation}
            \mu^*\big( \bigcup_nX_n \big)\leq \sum_n\sum_k\mu^*(B_k^{(n)})\leq \sum_n\left( \mu^*(X_n)+\frac{ \epsilon }{ 2^n } \right)=\sum_n\mu^*(X_n)+\epsilon.
        \end{equation}
        Cette inégalité étant valable pour tout \( \epsilon\), nous avons bien
        \begin{equation}
            \mu^*\big( \bigcup_nX_n \big)=\sum_n\mu^*(X_n).
        \end{equation}

    \item[Restriction]

        Soit \( A\in\tribD\). Nous avons automatiquement \( \mu^*(A)\leq \mu(A)\) parce que \( \mu(A)\) est dans l'ensemble dont nous prenons l'infimum (prendre \( A_1=A\) et \( A_{n\geq 2}=\emptyset\)).

        En ce qui concerne l'inégalité inverse nous considérons une suite \( A_n\) dans \( \tribD\) telle que \( A\subset\bigcup_nA_n\). Vu que \( A\in\tribD\) et que \( \tribD\) est une algèbre de parties nous avons \( A\cap A_n\in \tribD\) et \( \bigcup_n(A\cap A_n)=A\in\tribD\). Par conséquent
        \begin{equation}
            \mu(A)=\mu\big( \bigcup_n(A\cap A_n) \big)\leq \sum_n\mu(A\cap A_n)\leq \sum_n\mu(A_n).
        \end{equation}
        Donc tous les éléments de l'ensemble sur lequel nous prenons l'infimum sont plus grands que \( \mu(A)\). Nous en déduisons que \( \mu^*(A)\geq \mu(A)\).

    \end{subproof}
\end{proof}

%--------------------------------------------------------------------------------------------------------------------------- 
\subsection{Mesure et tribu}
%---------------------------------------------------------------------------------------------------------------------------

\begin{definition}[Mesure]  \label{DefBTsgznn}
    Une \defe{\wikipedia{en}{Measure_space}{mesure}}{mesure} sur l'espace mesurable\footnote{Définition \ref{DefjRsGSy}.} \( (\Omega,\tribA)\) est une application \( \mu\colon \tribA\to \eR\cup\{ \infty \}\) telle que
    \begin{enumerate}
        \item
            \( \mu(A)\geq 0\) pour tout \( A\in\tribA\);
        \item
            \( \mu(\emptyset)=0\);
        \item   \label{ItemQFjtOjXiii}
            \( \mu\left( \bigcup_{i=0}^{\infty}A_i\right)=\sum_{i=0}^{\infty}\mu(A_i)\) si les \( A_i\) sont des éléments de \( \tribA\) deux à deux disjoints.
    \end{enumerate}
    Une mesure est \defe{\( \sigma\)-finie}{mesure!$\sigma$-finie} si il existe un recouvrement dénombrable de \( \Omega\) par des ensembles de mesure finie. Si la mesure est $\sigma$-finie, nous disons que l'espace \( (\Omega,\tribA,\mu)\) est un espace mesuré $\sigma$-fini.

    La mesure \( \mu\) sur \( \Omega\) est \defe{finie}{mesure!finie} si \( \mu(\Omega)<\infty\).
\end{definition}

Si \( (\Omega,\tribA,\mu)\) et \( (S,\tribF,\nu)\) sont deux espaces mesurés, alors nous notons
\begin{equation}
    (\Omega,\tribA,\mu)\subset (S,\tribF,\nu)
\end{equation}
lorsque \( \Omega\subset S\), \( \tribA\subset\tribF\) et pour tout \( A\in\tribA\), \( \mu(A)=\nu(A)\).

\begin{definition}[Ensemble mesurable]\label{DefHGsQxHB}
    Les éléments de \( \tribA\) sont les ensembles \defe{mesurables}{mesurable!ensemble} pour la mesure \( \mu\).
\end{definition}

Si la mesure des \( \sigma\)-finie, nous pouvons choisir le recouvrement croissant pour l'inclusion. En effet si \( (E_n)_{n\in \eN}\) est le recouvrement, il suffit de considérer \( F_n=\bigcup_{k\leq n}E_k\). Ces ensembles \( F_n\) forment tout autant un recouvrement dénombrable, mais il est évidemment croissant.

Le lemme suivant complète la propriété \ref{DefBTsgznn}\ref{ItemQFjtOjXiii} lorsque les ensembles ne sont pas disjoints.
\begin{lemma}\label{LemKKNtvee} \label{LemPMprYuC}
    Si \( A\subset B\) sont deux ensembles \( \mu\)-mesurables de mesure finie alors
    \begin{equation}
        \mu(B\setminus A)=\mu(B)-\mu(A)
    \end{equation}
    et en particulier
    \begin{equation}
        \mu(B)\geq \mu(A).
    \end{equation}

    Si \( (M_n)\) est une suite d'éléments de \( \tribA\) pas spécialement disjoints, alors
    \begin{equation}\label{EqWWFooYPCTt}
        \mu\big( \bigcup_kM_k \big)\leq \sum_{k}\mu(M_k).
    \end{equation}
\end{lemma}

\begin{proof}
    Vu que les ensembles \( B\setminus A\) et \( A\) sont disjoints par la propriété \ref{ItemQFjtOjXiii} de la définition de mesure nous avons
    \begin{equation}
        \mu\big( (B\setminus A)\cup A \big)=\mu(B\setminus A)+\mu(A)
    \end{equation}
    et donc
    \begin{equation}
        \mu(B)=\mu(B\setminus A)+\mu(A)
    \end{equation}
    comme demandé.

    Pour la seconde partie nous considérons la suite disjointe
    \begin{subequations}
        \begin{numcases}{}
            M'_0=\emptyset\\
            M'_k=M_k\setminus M'_{k-1}.
        \end{numcases}
    \end{subequations}
    Nous avons \( \bigcup_kM'_k=\bigcup_kM_k\). Le calcul suivant est alors immédiat :
    \begin{equation}
        \mu\big( \bigcup_kM_k \big)=\mu\big( \bigcup_kM'_k \big)=\sum_{k}\mu(M'_k)=\sum_k\mu(M_k\setminus M'_{k-1})\leq \sum_k\mu(M_k).
    \end{equation}
\end{proof}

\begin{lemma}[\cite{MonCerveau}]\label{LemAZGByEs}
    Résultats sur les unions croissantes d'ensembles mesurables dans \( (S,\tribA,\mu)\).
    \begin{enumerate}
        \item\label{ItemJWUooRXNPci}
            
        Si \( (A_k)\) est une suite croissante d'ensembles \( \mu\)-mesurables dont l'union est mesurable, alors
        \begin{equation}
            \lim_{n\to \infty} \mu(A_k)=\mu(\bigcup_kA_k).
        \end{equation}

    \item\label{ItemJWUooRXNPcii}
        Si \( K_n\) est une suite emboîtée d'éléments de \( \tribA\) tels que \( K_n\to S\). Si \( A\in\tribA\) alors
        \begin{equation}
            \lim_{n\to \infty} \mu(A\cap K_n)=\mu(A).
        \end{equation}

    \end{enumerate}
\end{lemma}

\begin{proof}
    Pour prouver \ref{ItemJWUooRXNPci}, nous faisons le coup de l'union télescopique, en posant \( A_0=\emptyset\) :
    \begin{equation}
        \bigcup_{k=1}^{\infty}A_k=\bigcup_{k=1}^{\infty}(A_k\setminus A_{k-1}).
    \end{equation}
    Les ensembles \( A_k\setminus A_{k-1}\) sont deux à deux disjoints, donc la propriété \ref{ItemQFjtOjXiii} de la définition d'une mesure donne
    \begin{subequations}
        \begin{align}
            \mu(\bigcup_{k=1}^{\infty}A_k)&=\mu\left( \bigcup_{k=1}^{\infty}(A_k\setminus A_{k-1}) \right)\\
            &=\sum_{k=1}^{\infty}\mu(A_k\setminus A_{k-1})\\
            &=   \sum_{k=1}^{\infty}\big( \mu(A_k)-\mu(A_{k-1}) \big)    \label{subEqMDRRorb}\\
            &=\lim_{k\to \infty} \mu(A_k)-\mu(A_0)\\
            &=\lim_{k\to \infty} \mu(A_k).
        \end{align}
    \end{subequations}
    où pour obtenir \ref{subEqMDRRorb}, nous avons utilisé le lemme \ref{LemPMprYuC}.

    Le point \ref{ItemJWUooRXNPcii} est une application du point \ref{ItemJWUooRXNPci}.
\end{proof}

\begin{definition}[mesure de comptage]
    Soit \( (S,\tribF)\) un ensemble mesurable. La \defe{mesure de comptage}{mesure!de comptage} sur \( (S,\tribF)\) est la mesure définie par
    \begin{equation}
        m(A)=\begin{cases}
            \Card(A)    &   \text{si \( A\) est fini}\\
            +\infty    &    \text{sinon}.
        \end{cases}
    \end{equation}
\end{definition}
Cette mesure est utilisée pour voir des séries comme des intégrales sur \( (\eN,\partP(\eN),m)\).

\begin{example}
    La mesure de comptage \( m\) sur \( \eN\) est \( \sigma\)-finie parce que \( E_n=\{ 0,\ldots, n \}\) est de mesure finie et \( \bigcup_{n\in \eN}E_n=\eN\).
\end{example}

\begin{example}
    L'intervalle \( I=\mathopen[ 0 , 1 \mathclose]\) muni de la tribu de toutes ses parties et de la mesure de comptage est un espace mesuré non \( \sigma\)-fini.
\end{example}

\begin{example}
    L'intégration à la Riemann n'est pas dans la théorie des espaces mesurés. En effet l'ensemble 
    \begin{equation}
        \tribA=\{   A\subset\mathopen[ 0 , 1 \mathclose]\tq  \text{\( \mtu_A\) est intégrable au sens de Riemann}   \}
    \end{equation}
    n'est pas une tribu. Par exemple les singletons en font partie tandis que \( \mathopen[ 0 , 1 \mathclose]\cap \eQ\) n'en fait pas partie malgé que ce soit une union dénombrable de singletons.
\end{example}

\begin{definition}
    Si \( \mu\) est une mesure nous disons qu'une propriété est vraie \( \mu\)-\defe{presque partout}{presque!partout} si elle est fausse seulement sur un ensemble de mesure nulle.
\end{definition}

Par exemple la fonction de Dirichlet est presque partout égale à la fonction \( 1\) (pour la mesure de Lebesgue).

\begin{definition}[fonction mesurable]
    Une application entre espace mesurés
    \begin{equation}
        f\colon (\Omega,\tribA)\to (\Omega',\tribA')
    \end{equation}
    est \defe{mesurable}{mesurable!application} si pour tout \( B\in\tribA'\), l'ensemble \( f^{-1}(B)\) est dans \( \tribA\).
\end{definition}
\index{application!mesurable}

\begin{lemma}   \label{LemIDITgAy}
    Une union dénombrable d'ensemble de mesure nulle est de mesure nulle.
\end{lemma}

\begin{proof}
    C'est une conséquence immédiate du point \ref{ItemQFjtOjXiii} de la définition d'une mesure : si les \( A_i\) sont de mesure nulle,
    \begin{equation}
        \mu\left( \bigcup_{i=1}^{\infty}A_i \right)\leq \mu(A_i)=0
    \end{equation}
\end{proof}

\begin{definition}
    Si \( (A_n)\) est une suite croissante d'ensembles alors la \defe{limite}{limite!d'ensembles} est
    \begin{equation}
        \lim_nA_n=\bigcup_{i=0}^{\infty}A_i.
    \end{equation}
    Si la suite est décroissante alors la limite est
    \begin{equation}
        \lim_nA_n=\bigcap_{i=0}^{\infty}A_i.
    \end{equation}
\end{definition}
\ifthenelse{\value{isAgreg}=0}{Pour une suite ni croissante ni décroissante d'ensembles, il y a la notion de limite inductive\footnote{\emph{direct limit} en anglais.} qui sera un peu traitée à la section \ref{SecDirectLimit}.}{}

\begin{proposition}[\cite{RArwFWJ}] \label{PropAFNPSsm}
    Soit \( \mu\) une mesure sur \( \Omega\) et \( (S_n)\) une suite croissante d'ensembles \( \mu\)-mesurables de \( \Omega\). Nous notons
    \begin{equation}
        S=\lim_nS_n.
    \end{equation}
    Alors pour tout ensemble mesurable\footnote{Définition \ref{DefHGsQxHB}} \( A\subset\Omega\) nous avons
    \begin{equation}
        \mu(A\cap S)=\lim_{n\to \infty} \mu(A\cap S_n).
    \end{equation}
\end{proposition}
Note : dans la référence le résultat fonctionne pour tout ensemble \( A\) (et non seulement les mesurables) parce que la définition de la mesurabilité est un peu différente.

\begin{proof}
    L'inégalité \( \lim\mu(A\cap S_n)\leq \mu(A\cap S)\) est simple à prouver. En effet pour tout \( n\) nous avons \( A\cap S_n\subset A\cap S\) et donc par le lemme \ref{LemKKNtvee} nous avons
    \begin{equation}
        \mu(A\cap S_n)\leq\mu(A\cap S).
    \end{equation}
    En passant à la limite (qui respecte les inégalités) nous avons l'inégalité.

    Nous passons à l'inégalité dans l'autre sens. D'abord si \( \mu(A\cap S_n)=\infty\) pour un certain \( n\), alors il cela vaut encore \( \infty\) pour tous les \( n\) suivants et la limite est \( \infty\) sans problèmes. Donc nous supposons que \( \mu(A\cap S_n)<\infty\) pour tout \( n\in \eN\). De plus, quitte à renommer les indices, nous pouvons supposer que \( S_0=\emptyset\).

    Un élément \( x\) est dans \( S\) si et seulement si il existe \( n\geq 0\) tel que \( x\in S_{n+1}\). En prenant le plus petit de ces \( n\) nous avons \( x\neq S_n\) (éventuellement \( n=0\)) et donc
    \begin{equation}
        S=\bigcup_{n=0}^{\infty}\big( S_{n+1}\setminus S_n \big).
    \end{equation}
    Par conséquent
    \begin{equation}
            A\cap S=A\cap\bigcup_{n=0}^{\infty}(S_{n+1}\setminus S_n)
            =\bigcup_{n=0}^{\infty}A\cap(S_{n+1}\setminus S_n)
    \end{equation}
    Étant donné que les ensembles \( A\cap(S_{n+1}\setminus S_n)\) sont disjoints,
    \begin{subequations}
        \begin{align}
            \mu(A\cap S)&=\sum_{n=0}^{\infty}\mu\big( A\cap(S_{n+1}\setminus S_n) \big)\\
            &=\sum_{n=0}^{\infty}\mu\Big( (A\cap S_{n+1})\setminus (A\cap S_n) \Big)\\
            &=\sum_{n=0}^{\infty}\big[ \mu(A\cap S_{n+1})-\mu(A\cap S_n) \big]\\
            &=\lim_{n\to \infty} \mu(A\cap S_{n+1})-\underbrace{\mu(A\cap S_0)}_{=0}\label{subeqLTvmTjO}\\
            &=\lim_{n\to \infty} \mu(A\cap S_n).
        \end{align}
    \end{subequations}
    Dans ce calcul nous avons utilisé plusieurs fois le fait que les \( S_n\) et \( A\) étaient mesurables (et la propriété de tribu qui dit que \( A\cap S_n\) est également mesurable) ainsi que le lemme \ref{LemKKNtvee}. Nous avons aussi utilisé la série télescopique dans \( \eR\) pour obtenir \eqref{subeqLTvmTjO}.
\end{proof}

\begin{definition}[\cite{PVWUyDH}]
    Soit \( E\) un ensemble. Un ensemble \( \tribD\) de parties de \( E\) est un \defe{\( \lambda\)-système}{$\lambda$-système} lorsqu'il vérifie les conditions suivantes :
    \begin{enumerate}
        \item
            si \( A,B\in\tribD\) avec \( A\subset B\) alors \( B\setminus A\in \tribD\),
        \item
            si \( (A_k)_{k\geq 1}\) est une suite croissante d'éléments de \( \tribD\) alors \( \bigcup_kA_k\in\tribD\).
    \end{enumerate}
\end{definition}
Note : une tribu est un \( \lambda\)-système.

\begin{lemma}[\cite{PVWUyDH}]
    Une intersection quelconque de \( \lambda\)-systèmes dans \( E\) est un \( \lambda\)-système dans \( E\).
\end{lemma}

\begin{proof}
    Soient \( \{ \tribD_l \}_{l\in L}\) des \( \lambda\)-systèmes indicés par un ensemble \( L\). Si \( A,B\in\bigcap_{l\in L}\tribD_l\) alors \( B\setminus A\in\tribD_l\) pour tout \( l\in L\) et donc \( A\setminus B\in\bigcap_{l\in L}\tribD_l\). De la même façon si \( (A_k)\) est une suite croissante dans \( \bigcap_{l\in L}\tribD_l\) alors pour tout \( l\in L\) nous avons \( \bigcup_kA_k\in\tribD_l\). Donc \( \bigcup_kA_k\in\bigcap_l\tribD_l\).
\end{proof}
Ce lemme est ce qui permet de définir le \( \lambda\)-système \defe{engendré}{engendré!$\lambda$-système} par une classe \( \tribA\) de parties de \( E\) : c'est l'intersection de tous les \( \lambda\)-systèmes de \( E\) contenant \( \tribA\).

\begin{lemma}[\cite{PVWUyDH}]   \label{LemLUmopaZ}
    Soit \( \tribC\) une classe de parties de \( E\) (contenant \( E\) lui-même) qui soit stable par intersection finie. Alors le \( \lambda\)-système engendré par \( \tribC\) coïncide avec la tribu engendrée par \( \tribC\).
\end{lemma}

\begin{proof}
    Nous notons \( \tribE\) le \( \lambda\)-système engendré par \( \tribC\) et \( \tribF\) la tribu engendrée par \( \tribC\). Étant donné que \( \tribF\) est un \( \lambda\)-système nous avons \( \tribE\subset\tribF\). Pour montrer l'inclusion inverse nous allons prouver que \( \tribE\) est une tribu.

    D'abord pour \( C\in\tribC\) nous posons
    \begin{equation}
        \mG_C=\{ A\subset \tribE\tq A\cap C\in\tribE \}.
    \end{equation}
    et pour \( F\in\tribE\),
    \begin{equation}
        \mH_F=\{ A\in\tribE\tq A\cap F\in\tribE \}.
    \end{equation}
    Nous allons montrer que \( \mG_C\) et \( \mH_F\) sont des \( \lambda\)-systèmes et que \( \mG_C=\mH_F=\tribE\).
    
    Nous commençons par \( \mG_C\). Si \( A,B\in\mG_C\) avec \( A\subset B\) alors
    \begin{equation}
        (B\setminus A)\cap C=\underbrace{(B\cap C)}_{\in\tribE}\setminus\underbrace{(A\cap C)}_{\in\tribE}.
    \end{equation}
    Vu que \( \tribE\) est un \( \lambda\)-système et que \( (A\cap C)\subset(B\cap C)\) nous avons bien \( (B\setminus A)\cap C\in\tribE\) et donc \( B\setminus A\in\mG_C\). Soit maintenant \( (A_k)\) une suite croissante dans \( \mG_C\). Nous avons
    \begin{equation}
        \big( \bigcup_{k=1}^{\infty}A_k \big)\cap C=\bigcup_{k=1}^{\infty}(A_k\cap C)
    \end{equation}
    qui est une union d'une suite croissante d'éléments de \( \tribE\). Donc \( \bigcup_{k=1}^{\infty}(A_k\cap C)\in\tribE\), ce qui signifie que \( \bigcup_{k=1}^{\infty}A_k\in\mG_C\). Cela termine la preuve du fait que \( \mG_C\) soit une \( \lambda\)-système. 

    Étant donné que \( \tribC\) est stable par intersection finie, si \( K\in\tribC\) nous avons \( C\cap K\in\tribC\), ce qui signifie que \( K\in\mG_C\). Nous avons donc \( \tribC\subset\mG_C\). Donc \( \mG_C\) est un \( \lambda\)-système vérifiant \( \tribC\subset\mG_C\subset\tribE\). Mais comme \( \tribE\) est le plus petit \( \lambda\)-système contenant \( \tribC\) nous avons en fait \( \mG_C=\tribE\).

    Nous montrons à présent que \( \mH_F\) est un \( \lambda\)-système. Si \( A,B\in\mH_F\) avec \( A\subset B\) alors \( (B\setminus A)\cap F=(B\cap F)\setminus(A\cap F)\). Vu que \( \tribE\) est une \( \lambda\)-système et que \( A\cap F\) et \( B\cap F\) sont dans \( \tribE\) avec \( A\cap F\subset B\cap F\), nous avons
    \begin{equation}
        (B\cap F)\setminus(A\cap F)\in\mH_F.
    \end{equation}
    Soit maintenant \( (A_k)_{k\geq 1}\) une suite croissante dans \( \mH_F\). Pour tout \( k\) nous avons \( A_k\cap F\in\tribE\), ce qui donne
    \begin{equation}
        \big( \bigcap_{k=1}^{\infty}A_k \big)\cap F=\bigcap_{k=1}^{\infty}(A_k\cap F)\in\tribE.
    \end{equation}
    Donc \( \mH_F\) est un \( \lambda\)-système vérifiant \( \tribC\subset\mH_F\subset\tribE\). Nous en concluons que pour tout \( C\in\tribC\) et pour tout \( F\in\tribE\),
    \begin{equation}
        \mG_C=\mH_F=\tribE.
    \end{equation}
    
    Nous allons maintenant prouver que \( \tribE\) est une tribu\footnote{Définition \ref{DefjRsGSy}.}.
    \begin{enumerate}
        \item
            Si \( F\in\tribE\) alors \( E\cap F=F\in\tribE\), ce qui signifie que \( E\in\mH_F=\tribE\).
        \item
            Si \( A\in \tribE\) alors \( E\setminus A\in\tribE\) parce que \( \tribE\) est un \( \lambda\)-système et \( E\in\tribE\). Donc \( \complement A\in\tribE\).
        \item
            Montrons que \( \tribE\) est stable par union finie en considérant \( A,B\in\tribE\). Vu que \( E\) est également un élément de \( \tribE\) nous avons
            \begin{equation}
                E\setminus(A\cup B)=(E\setminus A)\cap(E\setminus B)\in\tribE.
            \end{equation}
            Cela prouve que \( \complement( A\cup B)\in \tribE\). Par complémentarité nous avons aussi \( A\cup B\in\tribE\).
            
            Soient \( A_k\in\tribE\), et nommons \( B_p=A_1\cup\ldots\cup A_p\). Les ensembles \( B_p\) forment une suite croissante d'éléments de \( \tribE\). L'union est donc dans \( \tribE\) et ce dernier est au final stable par union dénombrable.
    \end{enumerate}
    
    Maintenant que \( \tribE\) est une tribu nous avons \( \tribF\subset\tribE\) parce que \( \tribF\) est la plus petite tribu contenant \( \tribC\). Nous en déduisons que \( \tribE=\tribF\), ce qu'il fallait démontrer.
\end{proof}

\begin{theorem}[Unicité des mesures\cite{PVWUyDH}] \label{ThoJDYlsXu}
    Soient \( \mu\) et \( \nu\), deux mesures sur \( (E,\tribA)\) et une classe \( \tribE\) de parties de \( E\) telles que
    \begin{enumerate}
        \item
            La tribu engendrée par \( \tribE\) soit \( \tribA\).
        \item
            si \( A,B\in\tribE\) alors \( A\cap B\in\tribE\)
        \item
            il existe une suite croissante \( (E_n)\) dans \( \tribE\) telle que \( E=\lim E_n\).
    \end{enumerate}
    Alors si les mesures \( \mu\) et \( \nu\) coïncident sur \( \tribE\), elles coïncident sur \( \tribA\) en entier.
\end{theorem}
\index{unicité!des mesures}

\begin{proof}
    Soit \( (E_n)\) la suite des hypothèses; nous considérons \( \mu_n\) et \( \nu_n\), les restrictions de \( \mu\) et \( \nu\) à \( E_n\), c'est à dire
    \begin{subequations}
        \begin{align}
        \mu_n(A)=\mu(A\cap E_n)\\
        \nu_n(A)=\nu(A\cap E_n).
        \end{align}
    \end{subequations}
    Vu que les \( E_n\) sont dans \( \tribE\subset\tribA\) ils sont mesurables au sens de \( \mu\) et \( \nu\). Par la proposition \ref{PropAFNPSsm}, pour tout \( A\in \tribE\) nous avons alors
    \begin{subequations}
        \begin{align}
            \lim_{n\to \infty} \mu_n(A)=\mu(A)\\
            \lim_{n\to \infty} \nu_n(A)=\nu(A)
        \end{align}
    \end{subequations}
    Nous devons donc seulement montrer que pour tout \( A\in\tribA\) et pour tout \( n\in\eN\), \( \mu_n(A)=\nu_n(A)\). Pour cela nous nous fixons un \( n\) et nous considérons la classe
    \begin{equation}
        \tribD=\{ A\in\tribA\tq\mu_n(A)=\nu_n(A) \}.
    \end{equation}
    Le but sera de prouver que \( \tribD=\tribA\).
    
    
    Par hypothèse \( A\cap E_n\in\tribE\) et donc
    \begin{equation}
        \mu(A\cap E_n)=\nu(A\cap E_n)<\infty,
    \end{equation}
    c'est à dire que \( \mu_n=\nu_n\) sur \( \tribE\). Par ailleurs, \( E\cap E_n=E_n\in\tribE\), donc
    \begin{equation}
        \mu_n(E)=\nu_n(E)<\infty.
    \end{equation}
    Par conséquent \( \mu_n=\nu_n\) sur la classe \( \tribE'=\tribE\cup\{ E \}\) : \( \tribE'\subset\tribD\).

    Montrons que \( \tribD\) est un \( \lambda\)-système. Soient \( A,B\in\tribD\) avec \( A\subset B\). Alors, étant donné que les mesures \( \mu_n\) et \( \nu_n\) sont finies, le lemme \ref{LemPMprYuC} nous donne
    \begin{subequations}
        \begin{align}
            \mu_n(B\setminus A)=\mu_n(B)-\mu_n(A)\\
            \nu_n(B\setminus A)=\nu_n(B)-\nu_n(A).
        \end{align}
    \end{subequations}
    Donc \( \mu_n(B\setminus A)=\nu_n(B\setminus A)\) et \( B\setminus A\in\tribD\).

    Soit par ailleurs une suite croissante \( (A_k)_{k\geq 1}\) d'éléments de \( \tribD\). En posant \( B_p=\bigcup_{k=1}^pA_k\), le lemme \ref{LemAZGByEs}\ref{ItemJWUooRXNPci} nous donne
    \begin{equation}
        \mu_n(\bigcup_{k=1}^{\infty}A_k)=\lim_{p\to \infty} \mu_n(A_p).
    \end{equation}
    Mais vu que pour chaque \( p\) nous avons \( \mu_n(A_p)=\nu_n(A_p)\), nous avons aussi
    \begin{equation}
        \mu_n(\bigcup_{p=1}^{\infty}A_p)=\nu_n(\bigcup_{p=1}^{\infty}A_p).
    \end{equation}
    Donc \( \tribD\) est bel et bien un \( \lambda\)-système contenant \( \tribE'\). Par le lemme \ref{LemLUmopaZ}, le \( \lambda\)-système engendré par \( \tribE'\) est égal à la tribu engendrée par \( \tribE'\), mais par hypothèse la tribu engendrée par \( \tribE\) est \( \tribA\), donc le \( \lambda\)-système engendré par \( \tribE'\) est \( \tribA\). Vu que \( \tribD\) est une \( \lambda\)-système contenant \( \tribE'\), nous avons alors \( \tribA\subset\tribD\) et donc \( \tribA=\tribD\), ce qu'il fallait.
\end{proof}

\begin{example}\label{ExDMPoohtNAj}
    La partie \( \tribE\) des intervalles de \( \eR\) de la forme \( \mathopen] a , b \mathclose[\) engendre les boréliens par la proposition \ref{PropNBSooraAFr}. Par conséquent pour vérifier que deux mesures sont égales sur les boréliens de \( \eR\) il suffit de prouver qu'elles sont égales sur les intervalles ouverts.
\end{example}

%--------------------------------------------------------------------------------------------------------------------------- 
\subsection{Mesure extérieure}
%---------------------------------------------------------------------------------------------------------------------------

Nous avons déjà défini la notion de mesure extérieure en la définition \ref{DefUMWoolmMaf}.

\begin{lemma}[\cite{MesureLebesgueLi}]  \label{LemULSooBgZLI}
    Soit \( (S,\tribF,\mu)\) un espace mesuré et \( X\subset S\). Alors
    \begin{equation}
        \inf_{\substack{A\in\tribF\\X\in A}}\mu(A)=\inf\{ \sum_n\mu(A_n)\tq A_n\in\tribF,X\subset\bigcup_kA_k \}.
    \end{equation}
\end{lemma}

\begin{proof}
    Pour montrer l'inégalité \( \geq\), nous remarquons qu'il y a plus d'éléments dans l'ensemble du second membre que dans le premier. En effet si \( A\in\tribF\) avec \( X\subset A\) alors dans le membre de gauche nous pouvons prendre \( A_1=A\) et \( A_{n\geq 1}=\emptyset\).

    Pour l'inégalité dans l'autre sens, nous montrons que tout élément de
    \begin{equation}    \label{EqZRAooBCPFk}
        \{ \sum_n\mu(A_n)\tq A_n\in\tribF,X\subset\bigcup_kA_k \}
    \end{equation}
    est plus grand qu'un élément de
    \begin{equation}    \label{EqYNMooNvCtS}
        \{ \mu(A)\tq A\in\tribF,X\subset A \}.
    \end{equation}
    En effet si \( A_n\in\tribF\) avec \( X\subset \bigcup_kA_k\) alors en posant \( A=\bigcup_kA_k\) nous avons \( A\in\tribF\) avec \( X\subset A\) ainsi que \( \mu(A)\leq\sum_n\mu(A_n)\). Cela prouve que l'élément \( \sum_n\mu(A_n)\) de \eqref{EqZRAooBCPFk} est plus grand que l'élément \( \mu(A)\) de \eqref{EqYNMooNvCtS}.
\end{proof}

\begin{proposition}[\cite{MesureLebesgueLi}]    \label{PropFDUooVxJaJ}
    Soit un espace mesuré \( (S,\tribF,\mu)\) et l'application\nomenclature[Y]{\( \mu^*\)}{La mesure extérieure associée à la mesure \( \mu\)}
    \begin{equation}
        \begin{aligned}
            \mu^*\colon \partP(S)&\to \mathopen[ 0 , \infty \mathclose] \\
            X&\mapsto \inf\{ \mu(A)\tq A\in\tribF,X\subset A \}. 
        \end{aligned}
    \end{equation}
    Alors \( \mu^*\) est une mesure extérieure sur \( S\) et sa restriction à \( \tribF\) est égale à \( \mu\).
\end{proposition}

Cela est un cas particulier de \ref{PropIUOoobjfIB} en utilisant \ref{LemULSooBgZLI}. Nous en donnons cependant une preuve directe, qui est presque identique à celle de \ref{PropIUOoobjfIB}, mais avec une ou deux simplifications.
%AFAIRE : montrer que c'est un cas particulier.

\begin{proof}
    Notons que la définition est bonne parce que l'ensemble sur lequel l'infimum est pris n'est pas vide : prendre \( A=S\).
    \begin{subproof}
    \item[Le vide]
        D'abord \( \mu^*(\emptyset)=O\) parce que \( \emptyset\in\tribF\).
    \item[Inégalité d'inclusion]

        Soient \( X\subset Y\) dans \( \partP(S)\). Si \( Y\subset A\) alors \( X\subset A\), donc
        \begin{equation}
            \inf \{ \mu(A)\tq A\in\tribF,X\subset A \}\leq \inf \{ \mu(A)\tq A\in\tribF,Y\subset A \},
        \end{equation}
        ce qui signifie que \( \mu^*(X)\leq \mu^*(Y)\).
    \item[Inégalité par union dénombrable]

        Soit \( (X_n)_{n\in \eN}\) une suite de parties de \( S\). Si il existe \( n_0\) tel que \( \mu^*(X_{n_0})=\infty\) alors nous avons automatiquement \( \sum_n\mu^*(X_n)=\infty\) et l'inégalité demandée est évidente parce que n'importe quoi est plus petit ou égal à \( \infty\). Nous supposons donc que \( \mu^*(X_n)<\infty\) pour tout \( n\). 

        Soit \( \epsilon>\) et par définition pour chaque \( n\), il existe un \( A_n\in \tribF\) tel que \( X_n\subset A_n\) et \( \mu(A_n)\leq \mu^*(X_n)+\frac{ \epsilon }{ 2^n }\). Bien entendu nous avons
        \begin{equation}
            \bigcup_nX_n\subset \bigcup_nA_n\in\tribF.
        \end{equation}
        Nous en déduisons que
        \begin{equation}
            \mu^*\big( \bigcup_nX_n \big)\leq\mu\big( \bigcup_nA_n \big).
        \end{equation}
        Mais \( (S,\tribF,\mu)\) étant un espace mesuré,
        \begin{equation}
            \mu\big( \bigcup_nA_n \big)\leq \sum_n\mu(A_n).
        \end{equation}
        Au final nous avons les inégalités
        \begin{equation}
            \mu^*\big( \bigcup_nX_n \big)\leq \mu\big( \bigcup_nA_n \big)\leq\sum_n\mu(A_n)\leq \sum_n\mu^*(X_n)+\epsilon\sum_n\frac{1}{ 2^n }=\sum_n\mu^*(X_n)+\epsilon.
        \end{equation}
        Cela étant vrai pour tout \( \epsilon\),
        \begin{equation}
            \mu^*\big( \bigcup_nX_n \big)\leq\sum_n\mu^*(X_n),
        \end{equation}
        ce qui prouve que \( \mu^*\) est une mesure extérieure.

    \item[Restriction]

    Supposons que \( X\in\tribF\). Alors si \( X\subset A\) nous avons \( \mu(X)\leq \mu(A)\); mais en même temps, \( \mu(X)\) est dans l'infimum qui définit \( \mu^*(X)\) donc
    \begin{equation}
        \mu^*(X)\leq\mu(X)\leq \inf \{ \mu(A)\tq A\in\tribF,X\subset A \}\leq \mu(X)\leq\mu^*(X).
    \end{equation}
    Donc nous avons égalité de tous les éléments de cette chaîne d'inégalité.
    \end{subproof}
\end{proof}

\begin{definition}  \label{DefTRBoorvnUY}
    Soit \( S\) un ensemble et \( m^*\) une mesure extérieure sur \( S\). Une partie \( A\subset X\) est \defe{$ m^*$-mesurable}{mesurable!au sens de $ m^*$} si pour tout \( X\subset S\),
    \begin{equation}
        m^*(X)=m^*(X\cap A)+m^*(X\cap A^c).
    \end{equation}
\end{definition}


\begin{remark}
    L'inégalité
    \begin{equation}
        m^*(X)\leq m^*(X\cap A)+m^*(X\cap A^c)
    \end{equation}
    étant toujours vraie, pour prouver qu'un ensemble est \( m^*\)-mesurable, il est suffisant de prouver l'inégalité inverse : 
    \begin{equation}
        m^*(X)\geq m^*(X\cap A)+m^*(X\cap A^c)
    \end{equation}
\end{remark}
La définition \ref{DefTRBoorvnUY} est motivée par la proposition suivante.

\begin{proposition} \label{PropOJFoozSKAE}
    Soit un espace mesuré \( (S,\tribF,\mu)\) et \( \mu^*\) la mesure extérieure qui va avec. Alors pour tous les éléments de \( \tribF\) sont \( \mu^*\)-mesurables. 
    
    En d'autres termes, pour tout \( A\in\tribF\) et tout \( X\subset S\) nous avons
    \begin{equation}
        \mu^*(X)=\mu^*(X\cap A)+\mu^*(X\cap A^c).
    \end{equation}
\end{proposition}

\begin{proof}
    Vu que \( X=(X\cap A)\cup(X\cap A^c)\), et que \( \mu^*\) est une mesure extérieure,
    \begin{equation}
        \mu^*(X)\leq \mu^*(X\cap A)+\mu^*(X\cap A^c).
    \end{equation}
    Nous devons montrer l'inégalité inverse.

    Soit \( B\in\tribF\) tel que \( X\subset B\). D'une part nous avons \( X\cap A\subset B\cap A\in\tribF\), donc
    \begin{equation}
        \mu^*(X\cap A)\leq \mu^*(B\cap A)= \mu(B\cap A). 
    \end{equation}
    Et d'autre part, \( X\cap A^c\subset B\cap A^c\in\tribF\), donc
    \begin{equation}
        \mu^*(X\cap A^c)\leq \mu(B\cap A^c). 
    \end{equation}
    En remettant ensemble,
    \begin{equation}    \label{EqLSMooTyHLB}
        \mu^*(X\cap A)+\mu^*(X\cap A^c)\leq \mu(B\cap A)+\mu(B\cap A^c)=\mu(B).
    \end{equation}
    La dernière égalité est le fait que \( B\cap A\) et \( B\cap A^c\) sont disjoints et que \( \mu\) est une mesure. L'inégalité \eqref{EqLSMooTyHLB} étant vraie pour tout \( B\in \tribF\) tel que \( X\subset B\), elle est encore vraie pour l'infimum :
    \begin{equation}
        \mu^*(X\cap A)+\mu^*(X\cap A^c)\leq \inf \{ \mu(B)\tq B\in\tribF,X\subset B \}=\mu^*(X).
    \end{equation}
    Nous avons donc prouvé que 
    \begin{equation}
        \mu^*(X\cap A)+\mu^*(X\cap A^c)\leq \mu^*(X).
    \end{equation}
\end{proof}

\begin{remark}
Notons la duplicité du vocabulaire. Les ensembles \( \mu\)-mesurables sont les éléments de \( \tribF\), qui sont a priori les seuls sur lesquels \( \mu\) est calculable\footnote{«calculable» au sens où \( \mu\) y vaut un nombre bien définit; après, que ce soit facile ou pas à calculer dans la pratique, c'est une autre histoire.}, alors que les \( \mu^*\)-mesurables sont les parties de \( S\) qui vérifient une certaine propriété (et \( \mu^*\) est calculable sur toutes les parties de \( S\)).
\end{remark}

%--------------------------------------------------------------------------------------------------------------------------- 
\subsection{Espace mesuré complet}
%---------------------------------------------------------------------------------------------------------------------------

\begin{definition}  \label{DefAVDoomkuXi}
    Soit un espace mesuré \( (X,\tribA,\mu)\). Une partie \( N\) de \( X\) est \defe{négligeable}{négligeable!partie d'un espace mesuré} pour \( \mu\) si il existe \( Y\in\tribA\) tel que \( N\subset Y\) et \( \mu(Y)=0\).
\end{definition}
Si \( \mu^*\) est la mesure extérieure associée à \( \mu\) et si \( N\) est \( \mu\)-négligeable alors \( \mu^*(N)=0\) parce que 
\begin{equation}
    \mu^*(N)\leq \mu^*(Y)=\mu(Y)=0
\end{equation}
pour un certain \( Y\) mesurable de mesure nulle contenant \( N\).

\begin{lemma}   \label{LemVKNooOCOQw}
    L'ensemble des parties négligeables est stable par union dénombrable.
\end{lemma}

\begin{proof}
    Si les ensembles \( N_i\) sont négligeables, alors pour chaque \( i\) nous avons \( Y_i\in\tribA\) tel que \( N_i\subset Y_i\) et \( \mu(Y_i)=0\). Alors bien entendu \( \bigcup_iN_i\subset \bigcup_iY_i\) et en utilisant \eqref{EqWWFooYPCTt},
    \begin{equation}
        \mu\big( \bigcup_iY_i \big)\leq \sum_i\mu(Y_i)=0.
    \end{equation}
\end{proof}

\begin{definition}  \label{DefBWAoomQZcI}
    L'espace mesuré \( (X,\tribF,\mu)\) est \defe{complet}{complet!espace mesuré} si tout ensemble \( \mu\)-négligeable est dans \( \tribF\).
\end{definition}

Notons que la proposition \ref{PropHYLooLgOCy} s'applique si \( (X,\tribF,\mu)\) est un espace mesuré et \( \tribN\) est l'ensemble des parties \( \mu\)-négligeables. C'est ce qui permet de donner le théorème suivant, que nous redémontrons de façon indépendante de la proposition \ref{PropHYLooLgOCy}.
\begin{theorem}[Complétion d'espace mesuré\cite{MesureLebesgueLi,DXTooFCLru,BOQoojbFpP}]   \label{thoCRMootPojn}
    Soit un espace mesuré \( (X,\tribF,\mu)\) et \( \tribN\) l'ensemble des parties \( \mu\)-négligeables de \( X\).
    \begin{enumerate}
        \item
            Les ensembles suivants sont égaux :
            \begin{subequations}
                \begin{align}
                    \tribA&=\{ A\subset X\tq\exists B,C\in\tribF\tq B\subset A\subset C,\mu(C\setminus B)=0 \}\\
                    \tribB&=\{ B\cup N\tq  B\in\tribF,N\in\tribN \}    \label{EqFJIoorxZNU}\\
                    \tribC&=\{ A\subset X\tq \exists B\in\tribF\tq A\Delta B\in \tribN \}.
                \end{align}
            \end{subequations}
            Ici \( A\Delta B\) est la différence symétrique de \( A\) et \( B\), définition \ref{DefBMLooVjlSG}.
        \item
            L'ensemble \( \hat\tribF=\tribA=\tribB=\tribC\) est une tribu.
        \item
            La définition 
            \begin{equation}
                \begin{aligned}
                    \mu'\colon \tribB&\to \mathopen[ 0 , \infty \mathclose] \\
                    A\cup N&\mapsto \mu(A) 
                \end{aligned}
            \end{equation}
            est cohérente.
        \item
            L'application \( \mu'\) ainsi définie est une mesure sur \( (X,\tribA)\).
        \item
            L'espace \( (X,\tribA,\mu')\) est complet.
        \item
            La mesure \( \mu'\) prolonge \( \mu\).
        \item   \label{thoCRMootPojnvii}
            La mesure \( \mu'\) est minimale au sens où toute mesure complète prolongeant \( \mu\) prolonge \( \mu'\).
    \end{enumerate}
\end{theorem}

\begin{proof}
    Commençons par prouver que les trois ensembles \( \tribA\), \( \tribB\) et \( \tribC\) sont égaux.
    \begin{subproof}
    \item[\( \tribA\subset\tribB\).]
        Soit \( A\in\tribA\). Alors nous avons des ensembles \( B,C\in\tribF \) tels que \( B\subset A\subset V\) avec \( \mu(C\setminus B)=0\). Alors nous avons aussi \( A=B\cup(C\setminus B)\), ce qui prouve que \( A\in\tribB\).
    \item[\( \tribB\subset\tribC\).] 
        Soit \( A\in\tribB\), c'est à dire que \( A=B\cup N\) avec \( B\in\tribF\) et \( N\in\tribN\). Nous avons évidemment \( A\cup B=A\) et donc
        \begin{equation}
            A\Delta B=(A\cup B)\setminus(A\cap B)=A\setminus(A\cap B)=(B\cup N)\setminus(A\cap B)\subset N.
        \end{equation}
        Pour comprendre la dernière inclusion, si \( x\) appartient à \( A=B\cup N\) sans être dans \( N\) alors \( x\in B\) et donc \( x\in A\cap B\). Par conséquent nous avons \( A\Delta B\subset N\) et donc \( A\Delta B\in\tribN\).
    \item[\( \tribC\subset\tribA\)]
        Soit donc \( A\in\tribC\); il existe \( B\in\tribF\) tel que \( A\Delta B\in\tribN\) ou encore, il existe \( D\in\tribF\) tel que \( A\Delta B\subset D\) avec \( \mu(D)=0\). Si nous posons \( B'=B\cap D^c\) et \( C'=B\cup D\) alors nous prétendons avoir
        \begin{equation}
            B'\subset A\subset C'.
        \end{equation}
        Et nous le prouvons. En effet si \( x\in B\cap D^c\) alors en remarquant que \( B\) se divise en 
        \begin{equation}
            B=(B\cap A)\cup\big(B\cap (A\Delta B)\big),
        \end{equation}
        et en nous souvenant que \( B\cap (A\Delta B)\subset D\), il vient que \( B\cap D^c\subset B\cap A\). Et en particulier \( x\in A\). D'autre part
        \begin{equation}
            A\subset B\cup(A\Delta B)\subset B\cup D.
        \end{equation}
        Nous avons donc bien \( B'\subset A\subset C'\). Par stabilité de la tribu \( \tribF\) sous les intersections et complémentaires nous avons aussi \( B',C'\in\tribF\). De plus
        \begin{equation}
            C'\setminus B'=(B\cup D)\setminus(B\cap D^c)\subset D,
        \end{equation}
         et donc
         \begin{equation}
             \mu(C'\setminus B')\leq \mu(D)=0.
         \end{equation}
    \end{subproof}

    Nous avons donc prouvé que \( \tribA\subset\tribB\subset\tribC\subset \tribA\), et donc que \( \tribA=\tribB=\tribC\). Nous allons donc maintenant noter \( \tribA\) indifféremment les trois ensembles. Nous prouvons à présent que c'est une tribu.

    \begin{subproof}

        \item[Tribu : le vide]
            
            Pas de problèmes à \( \emptyset\in\tribA\)

        \item[Tribu : complémentaire]
            
            Soit \( A\in\tribA\). Alors il existe \( B,C\in\tribF\) tels que \( B\subset A\subset C\) avec \( \mu(C\setminus B)=0\). En passant au complémentaire,
            \begin{equation}
                C^c\subset A^c\subset B^c.
            \end{equation}
            Mais \( B^c\setminus C^c=C\setminus B\), donc \( \mu(B^c\setminus C^c)=0\).

        \item[Tribu : union dénombrable] 

            Soit \( (A_n)\) des éléments de \( \tribA\). Pour chaque \( n\) nous avons des ensembles \( B_n,C_n\in\tribF\) tels que\( B_n\subset A_n\subset C_n\) avec \( \mu(C_n\setminus B_n)=0\). En ce qui concerne les unions nous avons
            \begin{equation}
                \bigcup_nB_n\subset \bigcup_nA_n\subset \bigcup_nC_n,
            \end{equation}
            et 
            \begin{equation}
                \big( \bigcup_nC_n\big)\setminus\big( \bigcup_nB_n\big)\subset \bigcup_n(C_n\setminus B_n).
            \end{equation}
            Par conséquent, en utilisant \eqref{EqWWFooYPCTt},
            \begin{equation}
                \mu\left( \big( \bigcup_nC_n\big)\setminus\big( \bigcup_nB_n\big)\right)\leq\mu\left(  \bigcup_n(C_n\setminus B_n)\right)\leq\sum_n\mu(C_n\setminus B_n)=0.
            \end{equation}
            Cela prouve que \( \bigcup_nA_n\in\tribA\), et donc que \( \tribA\) est une tribu.

        \item[Définition cohérente]

            Soient \( A,A'\in\tribF\) et \( N,N'\in\tribN\) tels que \( A\cup N=A'\cup N'\). Nous considérons \( Y,Y'\in\tribF\) tel que \( N\subset Y\), \( N'\subset Y'\) et \( \mu(Y)=\mu(Y')=0\). En vertu de \eqref{EqWWFooYPCTt} nous avons
            \begin{equation}
                \mu(A)\leq \mu(A\cup Y)\leq \mu(A'\cup Y\cup Y')\leq\mu(A')+\mu(Y)+\mu(Y')=\mu(A').
            \end{equation}
            En écrivant la même chose en échangeant les primes nous prouvons également \( \mu(A')\leq \mu(A)\). Au final \( \mu(A)=\mu(A')\), c'est à dire
            \begin{equation}
                \mu'(A\cup N)=\mu'(A'\cup N').
            \end{equation}
            La définition de \( \mu'\) est donc cohérente.
        \item[\( \mu'\) est une mesure]

            Le fait que \( \mu'\) soit positive et que \( \mu'(\emptyset)\) soit nul ne pose pas de problèmes. Il faut voir l'union dénombrable disjointe. Si les ensembles \( A_i=B_i\cup N_i\) sont disjoints, alors les \( B_i\) et le \( N_i\) sont tous disjoints deux à deux. De plus l'ensemble \( \bigcup_iN_i\) est négligeable parce que nous avons déjà vu que \( \tribN\) était stable par union dénombrable (\ref{EqWWFooYPCTt}). Donc
            \begin{equation}
                \mu'\left( \bigcup_i B_i\cup N_i \right)=\mu'\Big( \big( \bigcup_iB_i \big)\cup\underbrace{\big( \bigcup_iN_i \big)}_{\in\tribN} \Big)=\mu\big( \bigcup_iB_i \big)=\sum_u\mu(B_i)=\sum_i\mu'(B_i\cup N_i).
            \end{equation}
        \item[Espace complet]
            Un ensemble \( \mu'\)-négligeable est automatiquement \( \mu\)-négligeable. En effet si \( H\) est \( \mu'\)-négligeable, il existe \( B\in\tribF\) et \( N\in\tribN\) tels que \( H\subset B\cup N\) avec \( \mu(B)=0\). Vu que \( N\) est \( \mu\)-négligeable, il existe \( Y\in\tribF\) tel que \( N\subset Y\) et \( \mu(Y)=0\). Donc \( H\subset B\cup N\subset B\cup Y\) avec \( \mu(B\cup Y)=0\).

            Tous les ensembles \( \mu\)-négligeables faisant partie de \( \tribB\), tous les ensembles \( \mu'\)-négligeables font partie de \( \tribA\).
        \item[Prolongement]
            La mesure \( \mu'\) prolonge \( \mu\). En effet si \( A\in\tribF\) alors \( A=A\cup\emptyset\in\tribB\) et \( A\) est \( m'\)-mesurable. De plus \( \mu'(A)=\mu'(A\cup\emptyset)=\mu(A)\).
        \item[Minimalité]

            Soit un espace mesuré complet \( (X,\tribM,\nu)\) prolongeant \( (X,\tribF,\mu)\). Pour \( A\in\tribA\) nous devons prouver que \( A\in\tribM\) et que \( \mu'(A)=\nu(A)\). Il existe \( B\in\tribF\) et \( N\in\tribN\) tels que \( A=B\cup N\). Vu que \( N\) est \( \mu\)-négligeable, il est également \( \nu\)-négligeable et donc \( \nu\)-mesurable parce que \(\nu\) est complète : \( A\in\tribM\). En ce qui concerne l'égalité \( \mu'(A)=\nu(A)\) nous avons
            \begin{equation}
                \nu(B)\leq\nu(B\cup N)\leq \nu(B)+\nu(N)=\nu(B),
            \end{equation}
            donc \( \nu(A)=\nu(B\cup N)=\nu(B)=\mu(B)\). La dernière égalité est le fait que \( \nu\) prolonge \( \mu\). Mais par définition de \( \mu'\) nous avons aussi \( \mu'(A)=\mu'(B\cup N)=\mu(B)\). Au final \( \mu'(A)=\nu(A)=\mu(B)\).
    \end{subproof}
\end{proof}

\begin{definition}
    L'espace mesuré complet \( (X,\tribA,\mu')\) défini par le théorème \ref{thoCRMootPojn} est l'\defe{espace mesuré complétée}{espace!mesuré!complété} de \( (X,\tribF,\mu)\).

    Nous noterons le complété de \( (S,\tribF,\mu)\) par \( (S,\hat\tribF,\hat \mu)\)\nomenclature[Y]{\( (S,\hat\tribF,\hat\mu)\)}{complété de l'espace mesuré \( (S,\hat\tribF,\hat\mu)\)}
\end{definition}

\begin{theorem}[Carathéodory\cite{MesureLebesgueLi}] \label{ThoUUIooaNljH}
    Soit \( S\) un ensemble et \( m^*\) une mesure extérieure sur \( S\). Alors
    \begin{enumerate}
        \item   \label{RPPooHSWWsi}
            l'ensemble \( \tribM\) des parties \( m^*\)-mesurables est une tribu,
        \item
            la restriction de \( m^*\) est une mesure sur \( (S,\tribM)\),
        \item
            l'espace mesuré \( (S,\tribM,m^*)\) est complet\footnote{Définition \ref{DefBWAoomQZcI}.}.
    \end{enumerate}
\end{theorem}

\begin{proof}
    Une grosse partie de la preuve sera de prouver la stabilité de \( \tribM\) par union dénombrable quelconque; cela sera divisé en plusieurs parties.
    \begin{subproof}
    \item[Tribu : le vide]
        L'ensemble vide est \( m^*\)-mesurable.
    \item[Tribu : complémentaire]
        Soit \( A\in\tribM\) et \( X\in S\). La condition qui dirait \( A^c\in\tribM\) est :
        \begin{equation}
            m^*(X)=m^*(X\cap A^c)+m^*(X\cap A),
        \end{equation}
        qui est la même que celle qui dit que \( A\) est dans \( \tribM\).
    \item[Tribu : union finie]
        Soient \( A,B\in\tribM\) et \( X\subset S\). Alors, vu que \( m^*\) est une mesure extérieure,
        \begin{subequations}
            \begin{align}
                m^*(X)&\leq m^*\big( X\cap(A\cup B) \big)+m^*\big( X\cap (A\cup B)^x \big)\\
                &=m^*\big( (X\cap A)\cup(X\cap B) \big)+m^*\big( X\cap A^c\cap B^c \big).
            \end{align}
        \end{subequations}
        Mais nous pouvons écrire la première union sous forme d'une union disjointe de la façon suivante :
        \begin{equation}
            (X\cap A)\cup(X\cap B)=(X\cap A)\cup(X\cap B\cap A^c),
        \end{equation}
        ce qui donne 
        \begin{subequations}
            \begin{align}
                m^*(X)&\leq m^*(X\cap A)+m^*(X\cap B\cap A^c)+m^*(X\cap A^c\cap B^c)        \label{subeqLYNooRdrgCi}\\
                &=m^*(X\cap A)+m^*(X\cap A^c)\\
                &=m^*(X)
            \end{align}
        \end{subequations}
        parce que les deux dernier termes de \eqref{subeqLYNooRdrgCi} se somment à \( m^*(X\cap A^c)\) parce que \( B\in \tribM\). La dernière ligne est le fait que \( A\) soit \( m^*\)-mesurable.
    \item[Union finie disjointe]
        Soient \( \{ A_1,\ldots, A_n \}\) des éléments deux à deux disjoints de \( \tribM\). Nous allons maintenant prouver par récurrence que
        \begin{equation}    \label{EqBRIooAnPCd}
            m^*\Big( X\cap\big( \bigcup_{k=1}^nA_k \big) \Big)=\sum_{k=1}^nm^*(X\cap A_k).
        \end{equation}
        Si \( n=1\) le résultat est évident. Sinon, le fait que \( A_{n+1}\) soit \( m^*\)-mesurable donne
        \begin{equation}
            m^*\Big( X\cap\big( \bigcup_{k=1}^{n+1}A_k \big) \Big)=m^*\Big( X\cap\big( \bigcup_{k=1}^{n+1}A_k \big)\cap A_{n+1} \Big)+m^*\Big( X\cap\big( \bigcup_{k=1}^{n+1}A_k \big)\cap A_{n+1}^c \Big).
        \end{equation}
        Le fait que les \( A_k\) soient disjoints implique aussi que
        \begin{equation}
            X\cap\big( \bigcup_{k=1}^{n+1}A_k \big)\cap A_{n+1}=X\cap A_{n+1}
        \end{equation}
        et
        \begin{equation}
            X\cap\big( \bigcup_{k=1}^{n+1}A_k \big)\cap A_{n+1}^c=X\cap\big( \bigcup_{k=1}^nA_k \big)
        \end{equation}
        et donc
        \begin{subequations}
            \begin{align}
                m^*\Big( X\cap\big( \bigcup_{k=1}^{n+1}A_k \big) \Big)&=m^*(X\cap A_{n+1})+m^*\Big( X\cap\big( \bigcup_{k=1}^nA_k \big) \Big)\\
                &\stackrel{rec.}{=}m^*(X\cap A_{n+1})+\sum_{k=1}^nm^*(X\cap A_k)\\
                &=\sum_{k=1}^{n+1}m^*(X\cap A_k).
            \end{align}
        \end{subequations}
        La relation \eqref{EqBRIooAnPCd} est prouvée.

        Notons qu'en particularisant à \( X=S\) nous avons 
        \begin{equation}
            m^*\big( \bigcup_{k=1}^nA_k \big)=\sum_{k=1}^nm^*(A_k)
        \end{equation}
        dès que les \( A_k\) sont des éléments deux à deux disjoints de \( \tribM\).
        
    \item[Union dénombrable disjointe]

        Soit \( (A_n)_{n\in \eN}\) une suite d'éléments deux à deux disjoints dans \( \tribM\). Nous allons prouver les choses suivantes :
        \begin{itemize}
            \item \( \bigcup_nA_n\in\tribM\)
            \item \( m^*\big( \bigcup_nA_n \big)=\sum_nm^*(A_n)\)
        \end{itemize}
        où toutes les sommes et union sur \( n\) sont entre \( 1\) et \( \infty\).

        Nous posons \( A=\bigcup_kA_k\) et \( B_n=\bigcup_{k=1}^nA_k\). Nous savons que \( B_n\in\tribM\) pour tout \( n\) par le point précédent. Donc si \( X\in S\) nous avons
        \begin{subequations}
            \begin{align}   \label{EqGXLooRxqqg}
                m^*(X)&=m^*(X\cap B_n)+m^*(X\cap B_n^c)\\
                &=\sum_{k=1}^nm^*(X\cap A_k)+m^*(X\cap B_n^x)\\
                &\geq\sum_{k=1}^nm^*(X\cap A_k)+m^*(X\cap A^c)
            \end{align}
        \end{subequations}
        où nous avons utilisé la relation \eqref{EqBRIooAnPCd} sur les \( B_n\) ainsi que le fait que \( A^c\subset B_n^c\) (parce que \( B_n\subset A\)). L'inégalité \eqref{EqGXLooRxqqg} étant vraie pour tout \( n\), elle est vraie à la limite :
        \begin{subequations}
            \begin{align}
                m^*(X)&\geq \sum_{k=1}^{\infty}m^*(A\cap A_k)+m^*(X\cap A^c)\\
                &\geq m^*\Big( \bigcup_k(X\cap A_k) \Big)+m^*(X\cap A^c)\\
                &=m^*\Big( X\cap \big( \bigcup_kA_k \big) \Big)+m^*(X\cap A^c)\\
                &=m^*(X\cap A)+m^*(X\cap A^c),
            \end{align}
        \end{subequations}
        ce qui signifie que \( A\in\tribM\). La première des deux choses que nous voulions montrer est faite. En la particularisant à \( X=A\) et en tenant compte des faits que \( A\cap A_k=A_k\) et \( A\cap A^c=\emptyset\),
        \begin{equation}
            m^*(A)\geq \sum_{k=1}^{\infty}m^*(A\cap A_k)+m^*(A\cap A^c),
        \end{equation}
        c'est à dire que pour tout \( n\) nous avons
        \begin{equation}
            m^*\big( \bigcup_{k\in \eN}A_k \big)\geq \sum_{k=1}^nm^*(A_k).
        \end{equation}
        L'inégalité est encore vraie à la limite, et l'inégalité inverse étant toujours vraie pour une mesure extérieure,
        \begin{equation}
            m^*\big( \bigcup_{k\in \eN}A_k \big)=\sum_{k=1}^{\infty}m^*(A_k).
        \end{equation}

    \item[Union dénombrable quelconque]

        Soit maintenant une suite \( (A_n)_{n\in\eN}\) d'éléments de \( \tribM\) que nous ne supposons plus être disjoints. Nous nous ramenons au cas disjoint en posant
        \begin{subequations}
            \begin{numcases}{}
                B_1=A_1\\
                B_n=A_n\cap\big( \bigcup_{k=1}^{n-1}A_k \big)^c,
            \end{numcases}
        \end{subequations}
        c'est à dire que nous mettons dans \( B_n\) les éléments de \( A_n\) qui ne sont dans aucun des \( A_k\) précédents. Autrement dit, nous posons \( B_0=\emptyset\) et \( B_n=A_n\setminus B_{n-1}\). L'ensemble \( \tribM\) étant stable par réunion finie, par complément et par intersection finie nous avons \( B_n\in\tribM\). De plus les \( B_n\) sont disjoints, donc
        \begin{equation}
            \bigcup_{k=1}^{\infty}A_k=\bigcup_{k=1}^{\infty}B_k\in\tribM.
        \end{equation}
        La première égalité se justifie de la façon suivante : si \( x\in\bigcup_{k=1}^{\infty}A_k\) alors nous notons \( n_0\) le plus petit \( n\) tel que \( x\in A_n\) et alors \( x\in B_{n_0}\).
    \item[Espace complet]
        Nous prouvons à présent que \( (S,\tribM,m^*)\) est un espace mesuré complet. Soit \( N\) une partie \( m^*\)-négligeable de \( S\) et \( Y\in\tribM\) tel que \( m^*(Y)=0\) et \( N\subset Y\). D'abord \( m^*(N)=0\) parce que
        \begin{equation}
            m^*(N)\leq m^*(Y)=0.
        \end{equation}
        Si \( X\subset S\) nous avons
        \begin{subequations}
            \begin{align}
                X\cap N\subset   N&\Rightarrow m^*(X\cap N)=0\\
                X\cap N^c\subset X&\Rightarrow m^*(X\cap N^c)\leq m^*(X).
            \end{align}
        \end{subequations}
        Donc
        \begin{equation}
            m^*(X\cap N)+m^*(X\cap N^c)\leq m^*(X),
        \end{equation}
        ce qui montre que \( N\) est est \( m^*\)-mesurable.
    \end{subproof}
\end{proof}

\begin{normaltext}

Ce théorème nous pousse à adopter de la notation. Lorsqu'un espace mesuré \( (S,\tribF,\mu)\) est donné, nous noterons 
\begin{equation}
    (S,\tribM,\mu^*)   
\end{equation}
l'espace mesuré construit de la façon suivante. D'abord \( \mu^*\) est la mesure extérieure associée à \( \mu\) par la proposition \ref{PropFDUooVxJaJ}. Ensuite \( \tribM\) est la tribu des parties \( \mu^*\)-mesurables, qui est bien une tribu parce que \( \mu^*\) est une mesure extérieure (\ref{ThoUUIooaNljH}). La proposition \eqref{PropOJFoozSKAE} dit alors que \( \tribF\subset\tribM\). De plus \ref{ThoUUIooaNljH} nous explique que si \( A\in\tribF\) alors \( \mu(A)=\mu^*(A)\). Tout cela pour dire que
\begin{equation}    \label{EqXDPooKwWAF}
    (S,\tribF,\mu)\subset (S,\tribM,\mu^*).
\end{equation}
Et enfin, \ref{ThoUUIooaNljH} nous dit que l'espace mesuré \( (S,\tribM,\mu^*)\) est complet.
\end{normaltext}

\begin{example} \label{ExOIXoosScTC}
    Montrons un cas dans lequel \( (S,\tribM,\mu^*)\) n'est pas \( \sigma\)-fini. Soit \( S\) un ensemble non dénombrable et \( \tribF\) la tribu des parties de \( S\) qui sont soit fini ou dénombrables soit de complémentaire fini ou dénombrable. Nous y mettons la mesure
    \begin{equation}
        \mu(A)=\begin{cases}
            0    &   \text{si \( A\) est au plus dénombrable}\\
            \infty    &    \text{sinon}.
        \end{cases}
    \end{equation}
    Cette mesure n'est pas \( \sigma\)-finie parce qu'aucune union de dénombrables est non dénombrable. De plus \( (S,\tribF,\mu)\) est complet parce que toute partie contenue dans un ensemble fini ou dénombrable est fini ou dénombrable (\ref{PropQEPoozLqOQ}).

    \begin{subproof}
     \item[\( \tribF\) n'est pas \( \partP(S)\)]
    
        La tribu \( \tribF\) est différente de \( \partP(S)\). En effet \( S\) étant infini, il existe par \ref{PropVCSooMzmIX} une bijection \( \varphi\colon \{ 1,2 \}\times S\to S\). Alors l'ensemble \( \varphi\big( \{ 1 \}\times S \big)\) est non dénombrable et son complémentaire
        \begin{equation}
            \varphi\big( \{ 1 \}\times S \big)^c=\varphi\big( \{ 2 \}\times S \big)
        \end{equation}
        n'est as dénombrable non plus. Cet ensemble n'est donc pas de \( \tribF\).

    \item[\( \tribM\) est \( \partP(S)\)]

        En effet, soit \( A\subset S\); il faut prouver que pour tout \( X\subset S\) nous avons
        \begin{equation}
            \mu^*(X)=\mu^*(X\cap A)+\mu^*(X\cap A^c).
        \end{equation}
        Nous prouvons cela en séparant les cas suivant que \( X\) est dénombrable ou non.

        Si \( X\) est fini ou dénombrable, alors \( X\cap A\) et \( X\cap A^c\) le sont également et nous avons \( \mu^*(X)=\mu(X)=0\) ainsi que \( \mu^*(X\cap A)=\mu^*(X\cap A^c)=0\).
        
        Si au contraire \( X\) n'est pas dénombrable,
        \begin{equation}
            \mu^*(X)=\inf_{\substack{A\in\tribF\\X\subset A}}\mu(A)=\infty,
        \end{equation}
         parce que \( X\) n'étant pas dénombrable, l'ensemble \( A\) ne l'est pas non plus et \( \mu(A)=\infty\). Mais comme \( X\) n'est pas dénombrable, soit \( X\cap A\) soit \( X\cap A^c\) (soit les deux) n'est pas dénombrable non plus; par conséquent
         \begin{equation}
             \mu^*(X\cap A)+\mu^*(X\cap A^c)=\infty.
         \end{equation}
    \end{subproof}

    Par conséquent \( (S,\tribF,\mu) \neq (S,\tribM,\mu^*)\). Mais vu que \( (S,\tribF,\mu)\) est complété nous devons avoir \( (S,\tribF,\mu)=(S,\hat\tribF,\hat\mu)\). Tout cela pour dire que nous avons un exemple avec
    \begin{equation}
        (S,\tribM,\mu^*)\neq (S,\hat\tribF,\hat \mu).
    \end{equation}
\end{example}

Nous avons deux façons de créer un espace complet à partir de \( (S,\tribF,\mu)\).
\begin{enumerate}
    \item
        Partir de la mesure extérieure \( \mu^*\) et construire \( (S,\tribM,\mu^*)\).
    \item
        Partir des ensembles \( \mu\)-négligeables, construire \( \hat\tribF\) et ensuite \( (S,\hat\tribF,\hat\mu)\).
\end{enumerate}
Ces deux façons ne sont pas équivalentes en général comme le montre l'exemple \ref{ExOIXoosScTC}. Mais il sera montré par la proposition \ref{PropIIHooAIbfj} que si \( (S,\tribF,\mu)\) est \( \sigma\)-fini alors les deux sont équivalent.

\begin{lemma}   \label{LemAESoofkMpi}
    Soit \( (S,\tribF,\mu)\) un espace mesuré. Alors pour tout \( X\subset S\) tel que \( \mu^*(X)<\infty\) il existe \( A\in\tribF\) tel que \( X\subset A\) et \( \mu^*(X)=\mu(A)\).
\end{lemma}
C'est à dire que \( \mu^*\) a beau être défini sur toutes les parties de \( S\), ce qu'il faut rajouter pour être \( \mu\)-mesurable, c'est pas grand chose.

\begin{proof}
    Par définition de la mesure extérieure associée à \( \mu\) en tant qu'infimum, pour tout \( n\geq 1\), il existe \( A_n\in\tribF\) tel que \( X\subset A_n\) et \( \mu(A_n)\leq \mu^*(X)+\frac{1}{ 2^n }\). Nous posons \( A=\bigcap_{n\geq 1}A_n\) et nous vérifions que ce \( A\) fait l'affaire.

    D'abord \( A\in\tribF\) parce qu'une tribu est stable par union dénombrable. Ensuite pour tout \( n\geq 1\) nous avons
    \begin{equation}
        \mu(A)\leq \mu(A_n)\leq \mu^*(X)+\frac{1}{ 2^n },
    \end{equation}
    et à la limite \( \mu(A)\leq \mu^*(X)\). Mais \( X\subset A\) implique \( \mu^*(X)\leq \mu(A)\) parce que \( \mu^*(X)\) l'infimum d'un ensemble contenant \( \mu(A)\).
\end{proof}

\begin{lemma}       \label{LemOAEoocBDaO}
    Si l'espace mesuré \( (S,\tribF,\mu)\) est \( \sigma\)-fini alors l'espace mesuré \( (S,\tribM,\mu^*)\) est également \( \sigma\)-fini.
\end{lemma}

\begin{proof}
    Vu que \( (S,\tribF,\mu)\) est \( \sigma\)-fini, nous avons une suite croissante \( A_n\) d'éléments de \( \tribF\) tels que \( \bigcup_nA_n=S\) et telle que \( \mu(A_n)<\infty\) pour tout \( n\). Étant donné que \( \tribF\subset\tribM\), cette suite convient également pour montrer que \( (S,\tribM,\mu^*)\) est \( \sigma\)-fini parce que \( \mu^*(A_n)=\mu(A_n)<\infty\).
\end{proof}

La proposition suivante montre que si \( (S,\tribF,\mu)\) est \( \sigma\)-finie alors nous avons l'égalité.
\begin{proposition} \label{PropIIHooAIbfj}
    Soit \( (S,\tribF,\mu)\) un espace mesuré \( \sigma\)-fini, \( \mu^*\) la mesure extérieure associée et \( \tribM\) la tribu des ensembles \( \mu^*\)-mesurables\footnote{C'est bien une tribu par \ref{ThoUUIooaNljH}\ref{RPPooHSWWsi}.}. Alors 
    \begin{equation}
    (S,\tribM,\mu^*) = (S,\hat\tribF,\hat\mu).
    \end{equation}
\end{proposition}

\begin{proof}
    La proposition \ref{PropOJFoozSKAE} indique que tous les éléments de \( \tribF\) sont \( \mu^*\)-mesurables, c'est à dire que \( \tribF\subset \tribM\). Mais l'espace \( (S,\tribM,\mu^*)\) est complet par le théorème de Carathéodory \ref{ThoUUIooaNljH}, donc par minimalité du complété (\ref{thoCRMootPojn}\ref{thoCRMootPojnvii}),
    \begin{equation}
        (S,\hat\tribF,\hat\mu)\subset(S,\tribM,\mu^*)
    \end{equation}
    au sens où \( \hat\tribF\subset\tribM\) et si \( A\in\hat\tribF\) alors \( \hat\mu(A)=\mu^*(A)\). Notons que cette inclusion est vraie même si la mesure n'est pas \( \sigma\)-finie.

    Nous passons à l'inclusion inverse. Soit \( A\in\tribM\), c'est à dire que pour tout \( Y\subset S\) nous avons
    \begin{equation}    \label{EqTZAooTCdGg}
        \mu^*(Y)=\mu^*(Y\cap A)+\mu^*(Y\cap A^c).
    \end{equation}
    Nous allons montrer que \( A\in\hat\tribF\) en séparant les cas suivant que \( \mu^*(A)=\infty\) ou non.

    \begin{subproof}
        \item[Si \( \mu^*(A)<\infty\)] 
        
        Par le lemme \ref{LemAESoofkMpi}, il existe \( X\in\tribF\) tel que \( A\subset X\) et \( \mu^*(A)=\mu(X)\). Vu que \( (S,\tribF,\mu)\subset (S,\tribM,\mu^*)\) nous avons alors
        \begin{equation}    \label{EqKFQooQaont}
            \mu^*(A)=\mu(X)=\mu^*(X).
        \end{equation}
        Nous écrivons la relation \eqref{EqTZAooTCdGg} avec ce \( X\) en guise de \( Y\), et en nous souvenant que \( X\cap A=A\) et \( X\cap A^c=X\setminus A\) :
        \begin{equation}
            \mu^*(X)=\mu^*(A)+\mu^*(X\setminus A).
        \end{equation}
        En tenant compte de \eqref{EqKFQooQaont} et du fait que \( \mu^*(A)<\infty\), nous pouvons simplifier et trouver \( \mu^*(X\setminus A)=0\). Le lemme \ref{LemAESoofkMpi} nous donne alors \( B\in\tribF\) tel que \( X\setminus A\subset B\) et \( \mu(B)=\mu^*(X\setminus A)=0\), c'est à dire que \( X\setminus A\) est \( \mu\)-négligeable. Par conséquent \( X\setminus A\in\hat\tribF\). En écrivant
        \begin{equation}
            A=X\setminus(X\setminus A),
        \end{equation}
        nous avons écrit \( A\) comme différence de deux éléments de \( \hat\tribF\) et nous concluons que \( A\in\hat\tribF\).
        
        \item[Si \( \mu^*(A)<\infty\)] 

            Le lemme \ref{LemOAEoocBDaO} nous indique que \( (S,\tribM,\mu^*)\) est \( \sigma\)-fini et il existe donc une suite \( (S_n)_{n\geq 1}\) dans \( \tribM\) telle que \( \bigcup_nS_n=S\) et \( \mu^*(S_n)<\infty\). L'ensemble \( A\cap S_n\) est un élément de \( \tribM\) vérifiant
            \begin{equation}
                \mu^*(A\cap S_n)\leq \mu^*(A)<\infty,
            \end{equation}
            ce qui implique que \( A\cap S_n\in\hat\tribF\) par la première partie. Maintenant \( A=\bigcup_n(A\cap S_n)\in\hat\tribF\) par union dénombrable d'éléments de la tribu \( \hat\tribF\).
    \end{subproof}
\end{proof}
