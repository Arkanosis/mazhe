% This is part of Mes notes de mathématique
% Copyright (c) 2008-2015
%   Laurent Claessens
% See the file fdl-1.3.txt for copying conditions.

%+++++++++++++++++++++++++++++++++++++++++++++++++++++++++++++++++++++++++++++++++++++++++++++++++++++++++++++++++++++++++++
\section{Normes et distances}\label{Sect_definition}
%+++++++++++++++++++++++++++++++++++++++++++++++++++++++++++++++++++++++++++++++++++++++++++++++++++++++++++++++++++++++++++

%--------------------------------------------------------------------------------------------------------------------------- 
\subsection{Introduction : valeur absolue et norme}
%---------------------------------------------------------------------------------------------------------------------------

La valeur absolue est essentielle pour introduire les notions de limite et de continuité pour les fonctions d'une variable. En fait nous disons que la fonction $f\colon \eR\to \eR$ est continue au point $a$ lorsque pour tout $\varepsilon$, il existe un $\delta$ tel que
\begin{equation}
	| x-a |\leq\delta \Rightarrow | f(x)-f(a) |\leq \varepsilon.
\end{equation}
La quantité $| x-a |$ donne la «distance» entre $x$ et $a$; la définition de la continuité signifie que pour tout $\varepsilon$, il existe un $\delta$ tel que si $a$ et $x$ sont au plus à la distance $\delta$ l'un de l'autre, alors $f(x)$ et $f(a)$ ne seront éloigné au plus d'une distance $\varepsilon$.

La valeur absolue, dans $\eR$, nous sert donc à mesurer des distances entre les nombres. Les principales propriétés de la valeur absolue sont :
\begin{enumerate}

	\item
		$| x |=0$ implique $x=0$,
	\item
		$| \lambda x |=| \lambda | |x |$,
	\item
		$| x+y |\leq | x |+| y |$

\end{enumerate}
pour tout $x,y\in\eR$ et $\lambda\in\eR$.

Afin de donner une notion de limite pour les fonctions de plusieurs variables, nous devons trouver un moyen de définir les notion de <<taille>> d'un vecteur et de distance entre deux points de $\eR^n$, avec $n>1$. La notion de <<taille>> doit satisfaire propriétés analogues à celles de la valeur absolue. 

La premier notion de «taille» pour un vecteur de $\eR^2$ que nous vient à l'esprit est la longueur du segment entre l'origine et l'extrémité libre du vecteur. Cela peut être calculée à l'aide du théorème de Pythagore : 
\begin{equation}
  \textrm{taille de } (a,b) = \sqrt{a^2+b^2}.
\end{equation}
Nous pouvons introduire une la notion de distance entre les éléments de $\eR^2$ de façon similaire :
\begin{equation}
	d\big((a_x,a_y),(b_x,b_y)\big)=\sqrt{  (a_x-b_x)^2+(a_y-b_y)^2  }.
\end{equation}
Cette définition a l'air raisonnable; est-elle mathématiquement correcte ? Peut-elle jouer le rôle de la valeur absolue dans $\eR^2$ ? Est-elle la seule définitions possibles de «taille» et distance en $\eR^2$ ?  

%--------------------------------------------------------------------------------------------------------------------------- 
\subsection{Norme}
%---------------------------------------------------------------------------------------------------------------------------

Nous voulons formaliser les notions de «taille» et de distance dans $\eR^n$, et plus généralement dans un espace vectoriel $V$ de dimension finie. Pour cela nous nous inspirons des propriétés de la valeur absolue.
\begin{definition}		\label{DefNorme}
	Soit $V$ un espace vectoriel réel. Une \defe{norme}{norme!définition} est une application $N\colon V\to \eR^+$ vérifiant les axiomes 
	\begin{enumerate}

		\item
			$N(0_V)=0$, et $N(x)=0$ implique $x=0_V$;
		\item\label{ItemDefNormeii}
			$N(\lambda x)=| \lambda |N(x)$ pour tout $\lambda\in\eR$ et $x\in V$;
		\item\label{ItemDefNormeiii}
			$N(x+y)\leq N(x)+N(y)$ pour tout $x,y\in V$. Cette propriété est appelée \defe{inégalité triangulaire}{inégalité!triangulaire}.
	\end{enumerate}
	Ici et dans la suite, $0_V$ désigne l'élément zéro de l'espace $V$.
\end{definition}
En prenant $\lambda=-1$ dans la propriété \ref{ItemDefNormeii}, nous trouvons immédiatement que $N(-x)=N(x)$.

\begin{proposition}		\label{PropNmNNm}
	Toute norme $N$ sur l'espace vectoriel $V$ vérifie l'inégalité
	\begin{equation}
		\big| N(x)-N(y) \big|\leq N(x-y)
	\end{equation}
	pour tout $x,y\in V$.
\end{proposition}
	
\begin{proof}
	Nous avons, en utilisant le point \ref{ItemDefNormeiii} de la définition \ref{DefNorme},
	\begin{subequations}
		\begin{align}
			N(x)&=N(x-y+y)\leq N(x-y)+N(y),	\label{subEqNNNxxyyya}\\
			N(y)&=N(y-x+x)\leq N(y-x)+N(x).	\label{subEqNNNxxyyyb}
		\end{align}
	\end{subequations}
	Supposons d'abord que $N(x)\geq N(y)$. Dans ce cas, en utilisant \eqref{subEqNNNxxyyya},
	\begin{equation}
		\big| N(x)-N(y) \big|=N(x)-N(y)\leq N(x-y)+N(y)-N(y)=N(x-y).
	\end{equation}
	Si par contre $N(x)\leq N(y)$, alors nous utilisons \eqref{subEqNNNxxyyyb} et nous trouvons
	\begin{equation}
		\big| N(x)-N(y) \big|=N(y)-N(x)\leq N(y-x)+N(x)-N(x)=N(y-x).
	\end{equation}
	Dans les deux cas, nous avons retrouvé l'inégalité annoncée.
\end{proof}
Cette proposition signifie aussi que
\begin{equation}	\label{EqNleqNNleqNvqlqbs}
	-N(x-y)\leq N(x)-N(y)\leq N(x-y).
\end{equation}


Afin de suivre une notation proche de celle de la valeur absolue, à partir de maintenant, la norme d'un vecteur $v$ sera notée $\| v\|$ au lieu de $N(v)$. La proposition \ref{PropNmNNm} s'énoncera donc
\begin{equation}
\big| \| x \|-\| y \| \big|\leq \| x-y \|.
\end{equation}
\begin{definition}		\label{DefEVNetDistance}
	Un espace vectoriel $V$ muni d'une norme est une \defe{espace vectoriel normé}{normé!espace vectoriel}, et on écrit $(V,\| . \|)$. La \defe{distance induite}{distance (d'une norme)} par la norme entre les points $a$ et $b$ de $V$ est le nombre $d(a,b)=\| a-b \|$.

	Si $A$ est une partie de $V$ et si $x\in V$, nous disons que la \defe{distance}{distance!point et ensemble} entre $A$ et $x$ est le nombre
	\begin{equation}		\label{EqdefDistaA}
		d(x,A)=\inf_{a\in A}d(x,a).
	\end{equation}
\end{definition}
%The result is on the figure \ref{LabelFigDistanceEnsemble}
\newcommand{\CaptionFigDistanceEnsemble}{La distance entre $x$ et $A$ est donnée par la distance entre $x$ et $p$. Les distances entre $x$ et les autres points de $A$ sont plus grandes que $d(x,p)$.}
\input{Fig_DistanceEnsemble.pstricks}

Il est possible de définir de nombreuses normes sur $\eR^n$. Citons en quelque unes. 




Les normes $\| . \|_{L^p}$ ($p\in\eN$) sont définies de la façon suivante :
\begin{equation}		\label{EqDeformeLp}
	\| x \|_{L^p}=\Big( \sum_{i=1}^n| x_i |^p\Big)^{1/p},
\end{equation}
pour tout $x=(x_1,\ldots,x_n)\in\eR^n$. Parmi ces normes, celles qui seront le plus souvent utilisées dans ces notes sont
\begin{equation}
	\begin{aligned}[]
		\| x \|_{L^1}&=\sum_{i=1}^n| x_i |,\\
		\| x \|_{L^2}&=\Big( \sum_{i=1}^n| x_i |^2 \Big)^{1/2}.
	\end{aligned}
\end{equation}
La norme $L^2$ est la \defe{norme euclidienne}{norme!euclidienne}. Nous définissons également la \defe{norme supremum}{norme!supremum} par
\begin{equation}
	\| x \|_{\infty}=\sup_{1\leq i\leq n}| x_i |.
\end{equation}
Nous admettons sans démonstration que les fonctions $\| . \|_{L^p}\colon \eR^n\to \eR^+$ sont bien des normes.

\newcommand{\CaptionFigDistanceEuclide}{La \emph{norme} euclidienne induit la \emph{distance} euclidienne. D'où son nom. Le point $C$ est construit aux coordonnées $(A_x,B_y)$.}
\input{Fig_DistanceEuclide.pstricks}

Soient $A=(A_x,A_y)$ et $B=(B_x,B_y)$ deux éléments de $\eR^2$. La distance\footnote{Ne pas confondre «distance» et «norme».} euclidienne entre $A$ et $B$ est donnée par $\| A-B \|_2$. En effet, sur la figure \ref{LabelFigDistanceEuclide}, la distance entre les points $A$ et $B$ est donnée par
\begin{equation}
	| AB |^2=| AC |^2+| CB |^2=| A_x-B_x |^2+| A_y-B_y |^2,
\end{equation}
par conséquent,
\begin{equation}
	| AB |=\sqrt{| A_x-B_x |^2+| A_y-B_y |^2}=\| A-B \|_2.
\end{equation}

\begin{remark}
	Si $A$, $B$ et $C$ sont trois points dans le plan $\eR^2$, alors l'inégalité triangulaire $| AB |\leq| AC |+| CB |$ est précisément la propriété \ref{ItemDefNormeiii} de la norme (définition \ref{DefNorme}). En effet l'inégalité triangulaire s'exprime de la façon suivante en terme de la norme $\| . \|_2$ :
	\begin{equation}	\label{EqNDeuxAmBNNdd}
		\| A-B \|_2\leq \| A-C \|_2+\| C-B \|_2.
	\end{equation}
	En notant $u=A-C$ et $v=C-B$, l'équation \eqref{EqNDeuxAmBNNdd} devient exactement la propriété de définition de la norme :
	\begin{equation}
		\| u+v \|_2\leq \| u \|_2+\| v \|_2.
	\end{equation}
	Ceci explique pourquoi cette propriété des norme est appelée «inégalité triangulaire».
\end{remark}

Les distances que nous avons vues jusqu'à présent sont des distances définies à partir d'une norme. La définition suivante donne une notion générale de distance sur un espace vectoriel \( V\).

\begin{definition}
    Soit \( V\) un espace vectoriel. Une \defe{distance}{distance} sur \( V\) est une application \( d\colon V\times V\to \eR\) telle que
    \begin{enumerate}
        \item
            \( d(x,y)\geq 0\) pour tout \( x,y\in V\);
        \item
            \( d(x,y)=0\) si et seulement si \( x=y\);
        \item
            \( d(x,y)=d(y,x)\) pour tout \( x,y\in V\);
        \item
            \( d(x,y)\leq d(x,z)+d(z,y)\) pour tout \( x,y,z\in V\).
    \end{enumerate}
    La dernière condition est l'inégalité triangulaire. Le nombre \( d(x,y)\) est la \emph{distance} entre \( x\) et \( y\).
\end{definition}
Toute distance définit une norme en posant \( \| v \|=d(v,0)\).

%+++++++++++++++++++++++++++++++++++++++++++++++++++++++++++++++++++++++++++++++++++++++++++++++++++++++++++++++++++++++++++ 
\section{Produit scalaire}
%+++++++++++++++++++++++++++++++++++++++++++++++++++++++++++++++++++++++++++++++++++++++++++++++++++++++++++++++++++++++++++

\begin{definition}\label{DefVJIeTFj}
    Un \defe{produit scalaire}{produit!scalaire!en général} sur un espace vectoriel \( E\) est une forme bilinéaire symétrique définie positive.
\end{definition}

Étant donné que l'inégalité de Cauchy-Schwarz sera surtout utilisée dans le cas où un produit scalaire est bel et bien donné, nous l'énonçons et le démontrons avec des notations adaptée à l'usage. Le produit scalaire sera noté \( X\cdot Y\) pour \( b(X,Y)\) si \( b\) est la forme.
\begin{theorem}[Inégalité de Cauchy-Schwarz]      \label{ThoAYfEHG}
	Si $X$ et $Y$ sont des vecteurs, alors
	\begin{equation}
		| X\cdot Y |\leq\| X \|\| Y \|.
	\end{equation}
    Nous avons une égalité si et seulement si \( X\) et \( Y\) sont multiples l'un de l'autre.
\end{theorem}
\index{Cauchy-Schwarz}
\index{inégalité!Cauchy-Schwarz}

%TODO : mettre au point les notations.
\begin{proof}
	Étant donné que les deux membres de l'inéquation sont positifs, nous allons travailler en passant au carré afin d'éviter les racines carrés dans le second membre.

	Nous considérons le polynôme
	\begin{equation}
		P(t)=\| X+tY \|^2=(X+tY)\cdot(X+tY)=X\cdot X+tX\cdot Y+tY\cdot X+t^2Y\cdot Y.
	\end{equation}
	En ordonnant les termes selon les puissance de $t$,
	\begin{equation}
		P(t)=\| Y \|^2t^2+2(X\cdot Y)t+\| X \|^2.
	\end{equation}
	Cela est un polynôme du second degré en $t$. Par conséquent le discriminant\footnote{Le fameux $b^2-4ac$.} doit être négatif. Nous avons donc
	\begin{equation}
		\Delta=4(X\cdot Y)^2-4\| X \|^2\| Y \|^2\leq 0,
	\end{equation}
	ce qui donne immédiatement
	\begin{equation}
		(X\cdot Y)^2\leq\| X \|^2\| Y^2 \|.
	\end{equation}

    En ce qui concerne le cas d'égalité, si nous avons \( X\cdot Y=\| X \|\| Y \|\), alors le discriminant \( \Delta\) ci-dessus est nul et le polynôme \( P\) admet une racine double \( t_0\). Pour cette valeur nous avons
    \begin{equation}
        P(t_0)=| X+t_0Y |=0,
    \end{equation}
    ce qui implique \( X+t_0Y=0\) et donc que \( X\) et \( Y\) sont liés.
\end{proof}

Vu que nous allons voir un pâté d'espaces avec des produits scalaires, nous leur donnons un nom.
\begin{definition}\label{DefLZMcvfj}
    Un espace vectoriel \defe{euclidien}{euclidien!espace} est un espace vectoriel de dimension finie muni d'un produit scalaire.
\end{definition}

\begin{proposition} \label{PropEQRooQXazLz}
    Si \( x,y\mapsto x\cdot y\) est un produit scalaire sur un espace vectoriel \( E\), alors \( N(x)=\sqrt{x\cdot x}\) est une norme vérifiant l'identité du parallélogramme :
    \begin{equation}        \label{EqYCLtWfJ}
        \| x-y \|^2+\| x+y \|^2=2\| x \|^2+2\| y \|^2.
    \end{equation}
\end{proposition}

\begin{proof}

    Prouvons l'inégalité triangulaire\index{inégalité!triangulaire!produit scalaire}. Si \( x,y\in E\) nous avons
    \begin{equation}
        \| x+y \|=\sqrt{\| x \|^2+\| y \|^2+2x\cdot y}.
    \end{equation}
    Par l'inégalité de Cauchy-Schwartz, théorème \ref{ThoAYfEHG} nous avons aussi
    \begin{equation}
        \| x \|^2+\| y \|^2+2x\cdot y\leq \| x \|^2+\| y \|^2+2\| x \|\| y \|=\big( \| x \|+\| y \| \big)^2,
    \end{equation}
    donc
    \begin{equation}
        \| x+y \|\leq \sqrt{\big( \| x \|+\| y \| \big)^2}=\| x \|+\| y \|.
    \end{equation}

    La seconde assertion est seulement un calcul :
			\begin{equation}
				\begin{aligned}[]
					\| x-y \|^2+\| x+y \|^2&=(x-y)\cdot (x-y)+(x+y)\cdot(x+y)\\
					&=x\cdot x-x\cdot y-y\cdot x+y\cdot y\\
					&\quad +x\cdot x+x\cdot y+y\cdot x+y\cdot y\\
					&=2x\cdot x+2y\cdot y\\
					&=2\| x \|^2+2\| y \|^2.
				\end{aligned}
			\end{equation}
\end{proof}

Le produit scalaire permet de donner une norme via la formule suivante :
\begin{equation}
    \| x \|^2=x\cdot x.
\end{equation}

\begin{lemma}[\cite{KXjFWKA}]   \label{LemLPOHUme}
    Soit \( V\) un espace vectoriel muni d'un produit scalaire et de la norme associée. Si \( x,y\in V\) satisfont à \( \| x+y \|=\| x \|+\| y \|\), alors il existe \( \lambda\geq 0\) tel que \( x=\lambda y\).
\end{lemma}

\begin{proof}
    Quitte à raisonner avec \( x/\| x \|\) et \( y/\| y \|\), nous supposons que \( \| x \|=\| y \|=1\). Dans ce cas l'hypothèse signifie que \( \| x+y \|^2=4\). D'autre part en écrivant la norme en terme de produit scalaire,
    \begin{equation}
        \| x+y \|^2=\| x \|^2+\| y \|^2+2\langle x, y\rangle ,
    \end{equation}
    ce qui nous mène à affirmer que \( \langle x, y\rangle =1=\| x \|\| y \|\). Nous sommes donc dans le cas d'égalité de l'inégalité de Cauchy-Schwarz\footnote{Théorème \ref{ThoAYfEHG}.}, ce qui nous donne un \( \lambda\) tel que \( x=\lambda y\). Étant donné que \( \| x \|=\| y \|=1\) nous avons obligatoirement \( \lambda=\pm 1\), mais si \( \lambda=-1\) alors \( \langle x, y\rangle =-1\), ce qui est le contraire de ce qu'on a prétendu plus haut. Par soucis de cohérence, nous allons donc croire que \( \lambda=1\).
\end{proof}

%---------------------------------------------------------------------------------------------------------------------------
\subsection{Projection et angles}
%---------------------------------------------------------------------------------------------------------------------------

\begin{proposition}[Propriétés du produit scalaire]
	Si $X$ et $Y$ sont des vecteurs de $\eR^3$, alors
	\begin{description}
		\item[Symétrie] $X\cdot Y=Y\cdot X$;
		\item[Linéarité] $(\lambda X+\mu X')\cdot Y=\lambda(X\cdot Y)+\mu(X'\cdot Y)$ pour tout $\lambda$ et $\mu$ dans $\eR$;
		\item[Défini positif] $X\cdot X\geq 0$ et $X\cdot X=0$ si et seulement si $X=0$.
	\end{description}
\end{proposition}
Note : lorsque nous écrivons $X=0$, nous voulons voulons dire $X=\begin{pmatrix}
	0	\\ 
	0	\\ 
	0	
\end{pmatrix}$.


\begin{definition}
	La \defe{norme}{norme!vecteur} du vecteur $X$, notée $\| X \|$, est définie par 
	\begin{equation}
		\| X \|=\sqrt{X\cdot X}=\sqrt{x^2+y^2+z^2}
	\end{equation}
	si $X=(x,y,z)$. Cette norme sera parfois nommée «norme euclidienne».
\end{definition}
Cette définition est motivée par le théorème de Pythagore. Le nombre $X\cdot X$ est bien la longueur de la «flèche» $X$. Plus intrigante est la définition suivante :
\begin{definition}
	Deux vecteurs $X$ et $Y$ sont \defe{orthogonaux}{orthogonal!vecteur} si $X\cdot Y=0$. 
\end{definition}
Cette définition de l'orthogonalité est motivée par la proposition suivante.

\begin{proposition}		\label{PropProjScal}
	Si nous écrivons $\pr_Y$  l'opération de projection sur la droite qui sous-tend $Y$, alors nous avons
	\begin{equation}
		\| \pr_YX \|=\frac{ X\cdot Y }{ \| Y \| }.
	\end{equation}
\end{proposition}

\begin{proof}
	Les vecteurs $X$ et $Y$ sont des flèches dans l'espace. Nous pouvons choisir un système d'axe orthogonal tel que les coordonnées de $X$ et $Y$ soient
	\begin{equation}
		\begin{aligned}[]
			X&=\begin{pmatrix}
				x	\\ 
				y	\\ 
				0	
			\end{pmatrix},
			&Y&=\begin{pmatrix}
				l	\\ 
				0	\\ 
				0	
			\end{pmatrix}
		\end{aligned}
	\end{equation}
	où $l$ est la longueur du vecteur $Y$. Pour ce faire, il suffit de mettre le premier axe le long de $Y$, le second dans le plan qui contient $X$ et $Y$, et enfin le troisième axe dans le plan perpendiculaire aux deux premiers.

	Un simple calcul montre que $X\cdot Y=xl+y\cdot 0+0\cdot 0=xl$. Par ailleurs, nous avons $\| \pr_YX \|=x$. Par conséquent,
	\begin{equation}
		\| \pr_YX \|=\frac{ X\cdot Y }{ l }=\frac{ X\cdot Y }{ \| Y \| }.
	\end{equation}
\end{proof}

\begin{corollary}
	Si la norme de $Y$ est $1$, alors le nombre $X\cdot Y$ est la longueur de la projection de $X$ sur $Y$.
\end{corollary}

\begin{proof}
	Poser $\| Y \|=1$ dans la proposition \ref{PropProjScal}.
\end{proof}

Nous sommes maintenant en mesure de déterminer, pour deux vecteurs quelconques $u$ et $v$, la projection orthogonale de $u$ sur $v$. Ce sera le vecteur $\bar u$ parallèle à $v$ tel que $u-\bar u$ est orthogonal à $v$. Nous avons donc
\begin{equation}
    \bar u=\lambda v
\end{equation}
et 
\begin{equation}
    (u-\lambda v)\cdot v=0.
\end{equation}
La seconde équation donne $u\cdot v-\lambda v\cdot v=0$, ce qui fournit $\lambda$ en fonction de $u$ et $v$ :
\begin{equation}
    \lambda=\frac{ u\cdot v }{ \| v \|^2 }.
\end{equation}
Nous avons par conséquent
\begin{equation}
    \bar u=\frac{ u\cdot v }{ \| v \|^2 }v.
\end{equation}
Armés de cette interprétation graphique du produit scalaire, nous comprenons pourquoi nous disons que deux vecteurs sont orthogonaux lorsque leur produit scalaire est nul.

Nous pouvons maintenant savoir quel est le coefficient directeur d'une droite orthogonale à une droite donnée. En effet, supposons que la première droite soit parallèle au vecteur $X$ et la seconde au vecteur $Y$. Les droites seront perpendiculaires si $X\cdot Y=0$, c'est à dire si
\begin{equation}
	\begin{pmatrix}
		x_1	\\ 
		y_1	
	\end{pmatrix}\cdot\begin{pmatrix}
		y_1	\\ 
		y_2	
	\end{pmatrix}=0.
\end{equation}
Cette équation se développe en 
\begin{equation}		\label{Eqxuyukljsca}
	x_1y_1=-x_2y_2.
\end{equation}
Le coefficient directeur de la première droite est $\frac{ x_2 }{ x_1 }$. Isolons cette quantité dans l'équation \eqref{Eqxuyukljsca} :
\begin{equation}
	\frac{ x_2 }{ x_1 }=-\frac{ y_1 }{ y_2 }.
\end{equation}
Donc le coefficient directeur de la première est l'inverse et l'opposé du coefficient directeur de la seconde.

\begin{example}
	Soit la droite $d\equiv y=2x+3$. Le coefficient directeur de cette droite est $2$. Donc le coefficient directeur d'une droite perpendiculaires doit être $-\frac{ 1 }{ 2 }$.
\end{example}

\begin{proof}[Preuve alternative]
	La preuve peut également être donnée en ne faisant pas référence au produit scalaire. Il suffit d'écrire toutes les quantités en termes des coordonnées de $X$ et $Y$. Si nous posons
	\begin{equation}
		\begin{aligned}[]
			X&=\begin{pmatrix}
				x_1	\\ 
				x_2	\\ 
				x_2	
			\end{pmatrix},
			&Y&=\begin{pmatrix}
				y_1	\\ 
				y_2	\\ 
				y_3	
			\end{pmatrix},
		\end{aligned}
	\end{equation}
	l'inégalité à prouver devient
	\begin{equation}
		(x_1y_1+x_2y_2+x_3y_3)^2\leq (x_1^2+x_2^2+x_3^2)(y_1^2+y_2^2+y_3^2).
	\end{equation}
	Nous considérons la fonction
	\begin{equation}
		\varphi(t)=(x_1+ty_1)^2+(x_2+ty_2)^2+(x_3+ty_3)^2
	\end{equation}
	En tant que norme, cette fonction est évidement positive pour tout $t$. En regroupant les termes de chaque puissance de $t$, nous avons
	\begin{equation}
		\varphi(t)=(y_1^2+y_2^2+y_3^2)t^2+2(x_1y_1+x_2y_2+x_3y_3)t+(x_1^2+x_2^2+x_3^2).
	\end{equation}
	Cela est un polynôme du second degré en $t$. Par conséquent le discriminant doit être négatif. Nous avons donc
	\begin{equation}
		4(x_1y_1+x_2y_2+x_3y_3)^2-(x_1^2+x_2^2+x_3^2)(y_1^2+y_2^2+y_3^2)\leq 0.
	\end{equation}
	La thèse en découle aussitôt.
\end{proof}

\begin{proposition}
	La norme euclidienne a les propriétés suivantes :
	\begin{enumerate}
		\item
			Pour tout vecteur $X$ et réel $\lambda$,  $\| \lambda X \|=| \lambda |\| X \|$. Attention à ne pas oublier la valeur absolue !
		\item
			Pour tout vecteurs $X$ et $Y$, $\| X+Y \|\leq \| X \|+\| Y \|$.
	\end{enumerate}
\end{proposition}

\begin{proof}
	Nous ne prouvons pas le premier point.
    % TODO : faire la preuve
    Pour le second, nous avons les inégalités suivantes :
	\begin{subequations}
		\begin{align}
			\| X+Y \|^2&=\| X \|^2+\| Y \|^2+2X\cdot Y\\
			&\leq\| X \|^2+\| Y \|^2+2|X\cdot Y|\\
			&\leq\| X \|^2+\| Y \|^2+2\| X \|\| Y \|\\
			&=\big( \| X \|+\| Y \| \big)^2
		\end{align}
	\end{subequations}
	Nous avons utilisé d'abord la majoration $| x |\geq x$ qui est évident pour tout nombre $x$; et ensuite l'inégalité de Cauchy-Schwarz.
\end{proof}


%--------------------------------------------------------------------------------------------------------------------------- 
\subsection{Angle entre deux vecteurs}
%---------------------------------------------------------------------------------------------------------------------------

Si $a$ et $b$ sont des réels, l'inégalité $| a |\leq b$ peut se développer en une double inégalité
\begin{equation}
	-b\leq a\leq b.
\end{equation}
L'inégalité de Cauchy-Schwarz devient alors
\begin{equation}
	-\| X \|\| Y \|\leq X\cdot Y\leq\| X \|\| Y \|.
\end{equation}
Si $X\neq 0$ et $Y\neq 0$, nous avons alors
\begin{equation}
	-1\leq\frac{ X\cdot Y }{ \| X \|\| Y \| }\leq 1.
\end{equation}
Il existe donc un angle $\theta\in\mathopen[ 0 , \pi \mathclose]$ tel que
\begin{equation}		\label{eqDefAngleVect}
	\cos(\theta)=\frac{ X\cdot Y }{ \| X \|\| Y \| }.
\end{equation}
L'angle ainsi défini est l'\defe{angle}{angle!entre vecteurs} entre $X$ et $Y$. La définition \eqref{eqDefAngleVect} est souvent utilisée sous la forme
\begin{equation}		\label{eqPropCosThet}
	X\cdot Y=\| X \|\| Y \|\cos(\theta).
\end{equation}

Notez que les angles sont toujours des angles plus petits ou égaux à \unit{180}{\degree}.

%--------------------------------------------------------------------------------------------------------------------------- 
\subsection{Procédé de Gram-Schmidt}
%---------------------------------------------------------------------------------------------------------------------------

\begin{proposition}[Procédé de Gram-Schmidt]    \label{PropUMtEqkb}
    Un espace euclidien possède une base orthonormée.
\end{proposition}
\index{espace!euclidien}
\index{Gram-Schmidt}

\begin{proof}
    Soit \( E\) un espace euclidien et \( \{ v_1,\ldots, v_n \}\), une base quelconque de \( E\). Nous posons d'abord
    \begin{equation}
        \begin{aligned}[]
            f_1&=v_1,&e_1&=\frac{ f_1 }{ \| f_1 \| }.
        \end{aligned}
    \end{equation}
    Ensuite
    \begin{equation}
        \begin{aligned}[]
            f_2&=v_2-\langle v_2, e_1\rangle e_1,&e_2&=\frac{ f_2 }{ \| f_2 \| }.
        \end{aligned}
    \end{equation}
    Notons que \( \{ e_1,e_2 \}\) est une base de \( \Span\{ v_1,v_2 \}\). De plus elle est orthogonale :
    \begin{equation}
        \langle e_1, f_2\rangle =\langle e_1, v_2\rangle -\langle v_2, e_1\rangle \underbrace{\langle e_1, e_1\rangle}_{=1} =0.
    \end{equation}
    Le fait que \( \| e_1 \|=\| e_2 \|=1\) est par construction. Nous avons donc donné une base orthonormée de \( \Span\{ v_1,v_2 \}\).

    Nous continuons par récurrence en posant
    \begin{equation}
        \begin{aligned}[]
            f_k&=v_k-\sum_{i=1}^{k-1}\langle v_k, e_i\rangle e_i,&e_k&=\frac{ f_k }{ \| f_k \| }.
        \end{aligned}
    \end{equation}
    Pour tout \( j<k\) nous avons
    \begin{equation}
        \langle e_j, f_k\rangle =\langle e_j, v_k\rangle -\sum_{i=1}^{k-1}\langle v_k, e_i\rangle \underbrace{\langle e_i, e_j\rangle}_{=\delta_{ij}} =0
    \end{equation}
\end{proof}
Cet algorithme de Gram-Schmidt nous donne non seulement l'existence de bases orthonormée pour tout espace euclidien, mais aussi le moyen d'en construire à partir de n'importe quelle base.


%---------------------------------------------------------------------------------------------------------------------------
\section{Norme opérateur}
%+++++++++++++++++++++++++++++++++++++++++++++++++++++++++++++++++++++++++++++++++++++++++++++++++++++++++++++++++++++++++++
\label{SeckwyQjK}

Nous définissons ici la norme opérateur, et donnons des résultats généraux. D'autres résultats sur la norme opérateur :
\begin{enumerate}
    \item
        Le lemme à propos d'exponentielle de matrice \ref{LemQEARooLRXEef} donne :
        \begin{equation}
            \|  e^{tA} \|\leq P\big( | t | \big)\sum_{i=1}^r e^{t\real(\lambda_i)}.
        \end{equation}
\end{enumerate}

\begin{definition}[\cite{BrunelleMatricielle}]  \label{DefOYPooZIoWnI}
    Soit \( E\) un espace vectoriel (pas spécialement de dimension finie). Une  \defe{norme}{norme} sur $E$ est une application $\| . \|\colon E\to \eR$ telle que
    \begin{enumerate}
        \item
            $\| v \|=0$ seulement si $A=0$,
        \item
            $\| \lambda v \|=| \lambda |\cdot\| v \|$,
        \item
            $\| v+w \|\leq\| v \|+\| w \|$

    \end{enumerate}
    pour tout $v,w\in E$ et pour tout $\lambda\in\eR$.
\end{definition}

\begin{definition}  \label{DefDQRooVGbzSm}
    Si \( V\) et \( W\) sont des espaces vectoriels nous munissons \( \aL(V,W)\) d'une structure d'espace vectoriel en définissant la somme et le produit par un scalaire de la façon suivante. Si $T$ et $U$ sont des élément de $\aL(V,W)$ et si $\lambda$ est un réel, nous définissons les éléments $T+U$ et $\lambda T$ par
    \begin{enumerate}
        \item
            $(T+U)(x)=T(x)+U(x)$;
        \item
            $(\lambda T)(x)=\lambda T(x)$
    \end{enumerate}
    pour tout \( x\in V\).
\end{definition}

La proposition suivante donne une norme (au sens de la définition \ref{DefNorme}) sur $\aL(V,W)$ afin d'obtenir un espace vectoriel normé.
\begin{proposition}		\label{PropNormeAppLineaire}
    Soit le nombre
	\begin{equation}
        \|T\|_{\aL}=\sup_{x\in V}\frac{\|T(x)\|_{W}}{\|x\|_{V}}.
	\end{equation}
    \begin{enumerate}
        \item
            Si \( V\) est de dimension finie, alors \( \| T \|_{\aL}<\infty\).
        \item
            L'application \( T\mapsto\| T \|_{\aL}\) est une norme sur l'espace vectoriel des applications linéaires \( V\to W\).
        \item
            Nous avons la formule
            \begin{equation}    \label{EqFZPooIoecGH}
                \| T \|_{\aL}=\sup_{x\in V}\frac{\|T(x)\|_{W}}{\|x\|_{V}} =\sup_{\|x\|_{V}=1}\|T(x)\|_{W}
            \end{equation}
    \end{enumerate}
\end{proposition}
\index{norme!d'une application linéaire}

\begin{proof}
    Si \( V\) est de dimension finie alors l'ensemble $\{ \| x \|= 1 \}$ est compact par le théorème de Borel-Lebesgue \ref{ThoXTEooxFmdI}. Alors la fonction 
    \begin{equation}
        x\mapsto \frac{ \| T(x) \| }{ \| x \| }
    \end{equation}
    est une fonction continue sur un compact. Le corollaire \ref{CorFnContinueCompactBorne} nous dit alors qu'elle est bornée. Le supremum est donc un nombre réel fini.

    Nous vérifions que l'application $\| . \|$ de $\aL(V,W)$ dans $\eR$ ainsi définie est effectivement une norme.
    \begin{enumerate}
        \item
            $\|T\|_{\aL}=0$ signifie que $\|T(x)\|=0$ pour tout $x$ dans $V$. Comme  $\|\cdot\|_W$ est une norme nous concluons que $T(x)=0_{n}$ pour tout $x$ dans $V$, donc $T$ est l'application nulle. 
    \item
        Pour tout $a$ dans $\eR$ et tout  $T$ dans $\aL(V,W)$ nous avons
        \begin{equation}
            \|aT\|_{\mathcal{L}}=\sup_{\|x\|_{V}\leq 1}\|aT(x)\|_{W}=|a|\sup_{\|x\|_{V}\leq 1}\|T(x)\|_{W}=|a|\|T\|_{\mathcal{L}}.
        \end{equation}
    \item 
        Pour tous $T_1$ et $T_2$ dans $\aL(V,W)$ nous avons
      \begin{equation}\nonumber
        \begin{aligned}
           \|T_1+ T_2\|_{\mathcal{L}}&=\sup_{\|x\|\leq 1}\|T_1(x)+T_2(x)\|\leq\\
     &\leq\sup_{\|x\|\leq 1}\|T_1(x)\| +\sup_{\|x\|\leq 1}\|T_2(x)\|\\
     &=\|T_1\|\|T_2\|.
        \end{aligned}
      \end{equation}
    \end{enumerate}


    Enfin nous prouvons la formule alternative \eqref{EqFZPooIoecGH}. Nous allons montrer que les ensembles sur lesquels ont prend le supremum sont en réalité les mêmes :
    \begin{equation}
        \underbrace{\left\{ \frac{ \| Ax \| }{ \| x \| }\right\}_{x\neq 0}}_{A}=\underbrace{\left\{ \| Ax \|\tq \| x \|=1 \right\}}_{B}.
    \end{equation}
    Attention : ce sont des sous-ensembles de réels; pas de sous-ensembles de \( \eM(\eR)\) ou des sous-ensembles de \( \eR^n\).

    Pour la première inclusion, prenons un élément de \( A\), et prouvons qu'il est dans \( B\). C'est à dire que nous prenons \( x\in V\) et nous considérons le nombre \( \| Ax \|/\| x \|\). Le vecteur \( y=x/\| x \|\) est un vecteur de norme $1$, donc la norme de \( Ay\) est un élément de \( B\), mais
    \begin{equation}
        \| Ay \|=\frac{ \| Ax \| }{ \| x \| }.
    \end{equation}
    Nous avons donc \( A\subset B\).

    L'inclusion \( B\subset A\) est immédiate.
\end{proof}

\begin{definition}[Norme opérateur]          \label{DefNFYUooBZCPTr}
    Le nombre 
    \begin{equation}
        \| T \|_{\aL}=\sup_{x\in V}\frac{\|T(x)\|_{W}}{\|x\|_{V}} =\sup_{\|x\|_{V}=1}\|T(x)\|_{W}
    \end{equation}
    est la \defe{norme opérateur}{norme!d'application linéaire} de $T$. 
\end{definition}

La norme opérateur est liée à la continuité par la proposition \ref{PropmEJjLE}.

Lorsque nous considérons un espace vectoriel d'applications linéaires, nous considérons toujours\footnote{Sauf lorsque les événements nous forceront à trahir.} dessus la topologie liée à cette norme. 

%--------------------------------------------------------------------------------------------------------------------------- 
\subsection{Norme algébrique}
%---------------------------------------------------------------------------------------------------------------------------

\begin{definition}[Norme d'algèbre]  \label{DefJWRWQue}
    Si \( A\) est une algèbre\footnote{Définition \ref{DefAEbnJqI}.}, une \defe{norme d'algèbre}{norme!d'algèbre} sur \( A\) est une norme telle que pour toute \( u,v\in A\),
    \begin{equation}
        \| uv \|\leq \| u \|\| v \|.
    \end{equation}
\end{definition}
L'intérêt d'une norme d'algèbre est entre autres de mieux se comporter pour les séries, voir par exemple \ref{subsecEVnZXgf}.

\begin{proposition}
    Pour tout norme algébrique, le rayon spectral d'une matrice sur \( \eC\) est toujours plus petit que sa norme. C'est à dire que nous avons toujours \( \rho(A)\leq \| A \|\) pour toute norme algébrique \( \| . \|\).
\end{proposition}

\begin{definition}
    Le \defe{\wikipedia{en}{Spectral_radius}{rayon spectral}}{rayon spectral} d'une matrice carrée $A$, noté $\rho(A)$, est défini de la manière suivante :
    \begin{equation}
        \rho(A)=\max_i|\lambda_i|
    \end{equation}
    où les $\lambda_i$ sont les valeurs propres de $A$.
\end{definition}

\begin{example}     \label{ExemdefnormpMrt}
    Pour chaque norme sur \( \eR^n\), nous pouvons définir une norme correspondante sur \( \eM_n(\eR)\), appelée \defe{\wikipedia{fr}{Norme_d'opérateur}{norme opérateur}}{norme!opérateur}. Si \( \| . \|\) est une norme sur \( \eR^n\), nous définissons \( \| A \|\) par
    \begin{equation}\label{EqThUCEJ}
        \|A\|=\sup_{\|x\|\neq 0}\frac{\|Ax\|}{\|x\|}
    \end{equation}
    En particulier, cela donne lieu à toutes les normes \( \| A \|_p\) qui correspondent aux normes \( \| . \|_p\) sur \( \eR^n\). Cette norme est la norme \defe{subordonnée}{norme!subordonnée} à celle sur \( \eR^n\).
\end{example}

La norme opérateur est souvent écrite \( \| A \|_{\infty}\) parce que cette norme donne lieu à la \defe{topologie forte}{topologie!forte} sur l'espace des opérateurs. La topologie forte n'est pas la seule possible. Il existe aussi par exemple la \defe{topologie faible}{topologie!faible} donnée par la notion de convergence \( A_i\to A\) si et seulement si \( A_ix\to Ax\) pour tout \( x\in E\).

Il faut noter que la topologie faible n'est pas une topologie métrique. Cela même si la condition \( A_ix\to Ax\), elle, est métrique vu qu'elle est écrite dans \( E\).
%TODO : il faut mettre au clair quelle est vraiment la topologie faible à partir des ouverts.
et que dans le cas où \( E\) est de dimension infinie, elle est réellement différente de la topologie forte. Nous verrons à la sous-section \ref{subsecaeSywF} que dans le cas des projections sur un espaces de Hilbert, l'égalité
\begin{equation}
    \sum_{i=1}^{\infty}\pr_{u_i}=\id
\end{equation}
est vraie pour la topologie faible, mais pas pour la topologie forte.
\begin{definition}
    Si \( E\) et \( F\) sont deux espaces vectoriels normés nous notons \( \GL(E,F)\)\nomenclature[Y]{\( \GL(E,F)\)}{les bijections linéaires et continues} le sous-espace de \( \aL(E,F)\) des isométries linéaires de \( E\) vers \( F\). Un élément de \( \GL(E,F)\) est donc linéaire, continue et bijective.
\end{definition}

\begin{proposition} \label{PropEDvSQsA}
    Si \( E\) et \( F\) sont des espaces vectoriels normés alors la norme opérateur est une norme d'algèbre\footnote{Définition \ref{DefJWRWQue}.} sur \( \GL(E,F)\) :
    \begin{equation}
        \| AB \|\leq \| A \|\| B \|
    \end{equation}
    pour tout \( A,B\in\GL(E)\). De plus pour tout \( A\in \aL(E,F)\), et pour tout \( u\in E\) nous avons la majoration
    \begin{equation}
        \| Au \|\leq \| A \|\| u \|.
    \end{equation}
\end{proposition}

\begin{proof}
    Nous avons
    \begin{subequations}
        \begin{align}
            \| AB \|&=\sup_{x\in E}\frac{ \| ABx \| }{ \| x \| }\frac{ \| Bx \| }{ \| Bx \| }\\
            &=\sup_{x\in E}\frac{ \| A(Bx) \| }{ \| Bx \| }\frac{ \| Bx \| }{ \| x \| }\\
            &\leq \sup_{x\in E}\frac{ \| A(Bx) \| }{ \| Bx \| }\sup_{x\in E}\frac{ \| Bx \| }{ \| x \| }.
        \end{align}
    \end{subequations}
    Le premier facteur est égal à \( \| A \|\) parce que \( B\) est surjective. Le second est \( \| B \|\) par définition.

    Si \( u\in E\) alors
    \begin{equation}
        \| A \|=\sup_{x\in E}\frac{ \| Ax \| }{ \| x \| }\geq \frac{ \| Au \| }{ \| u \| },
    \end{equation}
    donc le résultat annoncé : \( \| Au \|\leq \| A \|\| u \|\).
\end{proof}
Notons qu'en réalité nous n'avons utilisé seulement le fait que \( B\) était surjective

\begin{lemma}   \label{LemWWXVSae}
Soit \( F\) un espace de Banach et deux suites \( A_k\to A\) et \( B_k\to B\) dans \( \aL(F,F)\). Alors \( A_k\circ B_k\to A\circ B\) dans \( \aL(F,F)\), c'est à dire
\begin{equation}
    \lim_{n\to \infty} (A_kB_k)=\left( \lim_{n\to \infty} A_k \right)\left( \lim_{n\to \infty} B_k \right).
\end{equation}
\end{lemma}

\begin{proof}
    Il suffit d'écrire
    \begin{equation}
        \| A_kB_k-AB \|\leq \| A_kB_k-A_kB \|+\| A_kB-AB \|.
    \end{equation}
    Le premier terme tend vers zéro pour \( k\to\infty\) parce que 
    \begin{subequations}
        \begin{align}
            \| A_kB_k-A_kB \|&=\| A_k(B_k-B) \|\\
            &\leq \| A_k \|\| B_k-B \|\to \| A \|\cdot 0\\
            &=0
        \end{align}
    \end{subequations}
    où nous avons utilisé la propriété fondamentale de la norme opérateur : la proposition \ref{PropEDvSQsA}. Le second terme tend également vers zéro pour la même raison.
\end{proof}

\begin{proposition}[Distributivité de la somme infinie] \label{PropQXqEPuG}
    Soient \( E\) un espace normé, une suite \( (u_k)\) dans \( \GL(E)\) ainsi que \( a\in\GL(E)\). Pourvu que la série \( \sum_{n=0}^{\infty}u_k\) converge nous avons
    \begin{equation}
        \left( \sum_{k=0}^{\infty}u_k \right)a=\sum_{k=0}^{\infty}(u_ka).
    \end{equation}
\end{proposition}

\begin{proof}
    Par définition de la somme infinie,
    \begin{equation}
        \spadesuit=\left( \sum_{k=0}^{\infty}u_k \right)a=\left( \lim_{n\to \infty} \sum_{k=0}^nu_k \right)a.
    \end{equation}
    Le lemme \ref{LemWWXVSae} appliqué à \( n\mapsto\sum_{k=0}^nu_k\) et à la suite constante \( a\) nous donne
    \begin{equation}    \label{EqOAoopjz}
        \spadesuit=\lim_{n\to \infty} \left( \sum_{k=0}u_ka \right),
    \end{equation}
    ce que nous voulions par distributivité de la somme finie : dans \eqref{EqOAoopjz}, le \( a\) est dans ou hors de la somme, au choix. L'important est qu'il soit dans la limite.
\end{proof}


\begin{lemma}       \label{LemCAIPooPMNbXg}
    Si \( E\) et \( F\) sont des espaces de Banach\quext{Je crois qu'il ne faut pas que \( E\) soit complet.}, l'espace \( \aL(E,F)\) est également de Banach.
\end{lemma}

\begin{proof}
    Soit \( (u_n)\) une suite de Cauchy dans \( \aL(E,F)\); si \( x\in E\) il existe \( N\) tel que si \( l,m>N\) alors \( \| a_l-a_m \|<\epsilon\), c'est à dire que pour tout \( \| x \|=1\) on a \( \| u_l(x)-u_n(x) \|<\epsilon\). Cela signifie que \( u_n(x)\) est une suite de Cauchy dans l'espace complet \( F\). Cette suite converge et nous pouvons définir l'application \( u\colon E\to F\) par
    \begin{equation}
        u(x)=\lim_{n\to \infty} u_n(x).
    \end{equation}
    Il suffit maintenant de prouver que \( u\) est linéaire, ce qui est une conséquence directe de la linéarité de la limite :
    \begin{equation}
        u(\alpha x+\beta y)=\lim_{n\to \infty} \big( \alpha u_n(x)+\beta u_n(y) \big).
    \end{equation}
\end{proof}

Le lemme \ref{LemPQFDooGUPBvF} donne une version plus simple de la proposition suivante.
\begin{proposition}     \label{PropQAjqUNp}
    Soit \( E\) un espace de Banach (espace vectoriel normé complet). Si \( A\in\aL(E,E)\) avec \( \| A \|<1\) pour la norme opérateur, alors \( (\mtu-A)\) est inversible et son inverse est donné par
    \begin{equation}
        (\mtu-A)^{-1}=\sum_{k=0}^{\infty}A^k.
    \end{equation}
    Le résultat tient aussi si \( A\) est nilpotente, même si sa norme n'est pas majorée par \( 1\).
\end{proposition}
\index{série!donnant \( (1-A)^{-1}\)}

\begin{proof}
    Étant donné que la norme opérateur est une norme algébrique (proposition \ref{PropEDvSQsA}), nous avons \( \| A^k \|\leq \| A \|^k\). Par conséquent la série \( \| A^k \|\) est majorée par la série géométrique qui converge. Par conséquent \( \sum_{k}A^k\) est une série absolument convergence et donc convergente par la proposition \ref{PropAKCusNM} et le fait que \( \aL(E)\) est complet (proposition \ref{LemCAIPooPMNbXg}).
    
    Montrons à présent que la somme est l'inverse de \( \mtu-A\) en utilisant le produit terme à terme autorisé par la proposition \ref{PropQXqEPuG} :
    \begin{equation}
        \sum_{k=0}^nA^k(\mtu-A)=\sum_{k=0}^n(A^k-A^{k+1})=\mtu-A^{n+1}.
    \end{equation}
    Par conséquent 
    \begin{equation}
        \| \mtu-\sum_{k=0}^nA^k(\mtu-A) \|=\| A^{n+1} \|\leq \| A \|^{n+1}\to 0.
    \end{equation}

    Si \( A\) est nilpotente, la convergence de \( \sum_{k=0}^{\infty}A^k\) ne pose pas de problèmes parce que la somme est finie. Le fait que cette somme soit \( (\mtu-A)^{-1}\) s'obtient de la même façon, mais il ne faut pas faire la dernière majoration : si \( A\) est nilpotente, il tombe sous le sens que \( \| A^{n+1} \|\to 0\), mais il n'est cependant pas vrai que \( \| A \|^{n+1}\) tende vers zéro.
\end{proof}

%---------------------------------------------------------------------------------------------------------------------------
\subsection{Normes de matrices et d'applications linéaires}
%---------------------------------------------------------------------------------------------------------------------------
\label{subsecNomrApplLin}


\begin{theorem}
    La norme $2$ d'une matrice peut se calculer de la manière suivante :
    \begin{equation}
        \|A\|_2=\sqrt{\rho(A{^t}A)}
    \end{equation}
\end{theorem}

\begin{proposition} \label{PropMAQoKAg}
    La fonction
    \begin{equation}
        \begin{aligned}
            f\colon \eM(n,\eR)\times \eM(n,\eR)&\to \eR \\
            (X,Y)&\mapsto \tr(X^tY) 
        \end{aligned}
    \end{equation}
    est un produit scalaire sur \( \eM(n,\eR)\).
\end{proposition}
\index{trace!produit scalaire sur \( \eM(n,\eR)\)}
\index{produit!scalaire!sur \( \eM(n,\eR)\)}

\begin{proof}
    Il faut vérifier la définition \ref{DefVJIeTFj}.
    \begin{itemize}
        \item La bilinéairité est la linéarité de la trace.
        \item La symétrie de \( f\) est le fait que \( \tr(A^t)=\tr(A)\).
        \item L'application \( f\) est définie positive parce que si \( X\in \eM\), alors \( X^tX\) est symétrique définie positive, donc diagonalisable avec des nombres positifs sur la diagonale. La trace étant un invariant de similitude, nous avons \( f(X,X)=\tr(X^tX)\geq 0\). De plus si \( \tr(X^tX)=0\), alors \( X^tX=0\) (pour la même raison de diagonalisation). Mais alors \( \| Xu \|=0\) pour tout \( u\in E\), ce qui signifie que \( X=0\).
    \end{itemize}
\end{proof}

\begin{example}
	Soit $m=n$, un point $\lambda$ dans $\eR$ et $T_{\lambda}$ l'application linéaire définie par $T_{\lambda}(x)=\lambda x$. La norme de $T_{\lambda}$ est alors
\[
\|T_{\lambda}\|_{\mathcal{L}}=\sup_{\|x\|_{\eR^m}\leq 1}\|\lambda x\|_{\eR^n}= |\lambda|.
\]
Notez que $T_{\lambda}$ n'est rien d'autre que l'homothétie de rapport $\lambda$ dans $\eR^m$.
\end{example}

\begin{example}
	Considérons la rotation $T_{\alpha}$ d'angle $\alpha$ dans $\eR^2$. Elle est donnée par l'équation matricielle
	\begin{equation}
		T_{\alpha}\begin{pmatrix}
			x	\\ 
			y	
		\end{pmatrix}=\begin{pmatrix}
			\cos\alpha	&	\sin\alpha	\\ 
			-\sin\alpha	&	\cos\alpha	
		\end{pmatrix}\begin{pmatrix}
			x	\\ 
			y	
		\end{pmatrix}=\begin{pmatrix}
			\cos(\alpha)x+\sin(\alpha)y	\\ 
			-\sin(\alpha)x+\cos(\alpha)y	
		\end{pmatrix}
	\end{equation}
	Étant donné que cela est une rotation, c'est une isométrie : $\| T_{\alpha}x \|=\| x \|$. En ce qui concerne la norme de $T_{\alpha}$ nous avons
	\begin{equation}
		\| T_{\alpha} \|=\sup_{x\in\eR^2}\frac{ \| T_{\alpha}(x) \| }{ \| x \| }=\sup_{x\in\eR^2}\frac{ \| x \| }{ \| x \| }=1.
	\end{equation}
	Toutes les rotations dans le plan ont donc une norme $1$. La même preuve tient pour toutes les rotations en dimension quelconque. 
\end{example}

%TODO : le théorème de fuite des compacts qui dit qu'une solution de y'=f(y,t) cesse d'exister seulement si elle tend vers +- infini.

\begin{example}
  Soit $m=n$, un point $b$ dans $\eR^m$ et $T_b$ l'application linéaire définie par $T_b(x)=b\cdot x$ (petit exercice : vérifiez qu'il s'agit vraiment d'une application linéaire).  La norme de $T_b$ satisfait les inégalités suivantes 
 \[
\|T_b\|_{\mathcal{L}}=\sup_{\|x\|_{\eR^m}\leq 1}\|b\cdot x\|_{\eR^n}\leq \sup_{\|x\|_{\eR^m}\leq 1}\|b \|_{\eR^n}\|x\cdot x\|_{\eR^n}\leq\|b \|_{\eR^n},
\]
\[
\|T_b\|_{\mathcal{L}}=\sup_{\|x\|_{\eR^m}\leq 1}\|b\cdot x\|_{\eR^n}\geq \left\|b\cdot \frac{b}{\|b \|_{\eR^n}}\right\|_{\eR^n}=\|b \|_{\eR^n},
\]
donc $\|T_b\|_{\mathcal{L}}=\|b \|_{\eR^n}$.
\end{example}

Une inégalité que nous utiliserons quelque fois dans la suite, y compris dans la proposition qui suit.
\begin{lemma}		\label{LemAvmajAfoisv}
	Soit $T$ une application linéaire de $\eR^m$ vers $\eR^n$. Alors
	\begin{equation}
		\| Av \|_n\leq \| A \|_{\aL}\| v \|_m.
	\end{equation}
	pour tout $v\in\eR^m$.
\end{lemma}

\begin{proof}
	Étant donné que le supremum d'un ensemble est plus grand ou égal à tous les éléments qui le compose,
	\begin{equation}
		\| A \|_{\aL(\eR^m,\eR^n)}=\sup_{x\in\eR^m}\frac{ \| Ax \| }{ \| x \| }\geq\frac{ \| Av \| }{ \| v \| },
	\end{equation}
	d'où le résultat.
\end{proof}

\begin{proposition}
    Une application linéaire de \( \eR^m\) dans \( \eR^n\) est continue.
\end{proposition}

\begin{proof}
      Soit $x$ un point dans $\eR^m$. Nous devons vérifier l'égalité
      \begin{equation}
       \lim_{h\to 0_m}T(x+h)=T(x).
      \end{equation}
    Cela revient à prouver que $\lim_{h\to 0_m}T(h)=0$, parce que $T(x+h)=T(x)+T(h)$. Nous pouvons toujours majorer $\|T(h)\|_n$ par $\|T\|_{\mathcal{L}(\eR^m,\eR^n)}\| h \|_{\eR^m}$ (lemme \ref{LemAvmajAfoisv}). Quand $h$ s'approche de $ 0_m $ sa norme $\|h\|_m$ tend vers $0$, ce que nous permet de conclure parce que nous savons que de toutes façons, $\| T \|_{\aL}$ est fini.
\end{proof}

Note : dans un espace de dimension infinie, la linéarité ne suffit pas pour avoir la continuité : il faut de plus être borné (ce que sont toutes les applications linéaires \( \eR^m\to\eR^n\)). Voir la proposition \ref{PropmEJjLE}.

%+++++++++++++++++++++++++++++++++++++++++++++++++++++++++++++++++++++++++++++++++++++++++++++++++++++++++++++++++++++++++++
\section{Produit vectoriel}
%+++++++++++++++++++++++++++++++++++++++++++++++++++++++++++++++++++++++++++++++++++++++++++++++++++++++++++++++++++++++++++

\begin{definition}
	Soient $u$ et $v$, deux vecteurs de $\eR^3$. Le \defe{produit vectoriel}{produit!vectoriel} de $u$ et $v$ est le vecteur $u\times v$ défini par 
	\begin{equation}
		\begin{aligned}[]
		u\times v&=\begin{vmatrix}
			e_1	&	e_2	&	e_3	\\
			u_1	&	u_2	&	u_3	\\
			v_1	&	v_2	&	v_3
		\end{vmatrix}\\
		&=
		(u_2v_3-u_3v_2)e_1+(u_3v_1-u_1v_3)e_2+(u_1v_2-u_2v_1)e_3
		\end{aligned}
	\end{equation}
	où les vecteurs $e_1$, $e_2$ et $e_3$ sont les vecteurs de la base canonique de $\eR^3$.
\end{definition}
La notion de produit vectoriel est propre à $\eR^3$; il n'y a pas de généralisation simple aux espaces $\eR^m$.

Nous n'allons pas nous attarder sur les nombreuses propriétés du produit vectoriel. Les principales sont résumées dans la proposition suivante.
\begin{proposition}
	Si $u$ et $v$ sont des vecteurs de $\eR^3$, alors le vecteur $u\times v$ est l'unique vecteur qui est perpendiculaire à $u$ et $v$ en même temps, de norme égal à la surface du parallélogramme construit sur $u$ et $v$ et tel que les vecteurs $u$, $v$, $u\times v$ forment une base dextrogyre.
\end{proposition}
La chose importante à retenir est que le produit vectoriel permet de construire un vecteur simultanément perpendiculaire à deux vecteurs donnés. Le vecteur $u\times v$ est donc linéairement indépendant de $u$ et $v$. En pratique, si $u$ et $v$ sont déjà linéairement indépendants, alors le produit vectoriel permet de compléter une base de $\eR^3$.

À l'aide du produit vectoriel et du produit scalaire, nous construisons le \defe{produit mixte}{produit!mixte} de trois vecteurs de $\eR^3$ par la formule
\begin{equation}
	(u\times v)\cdot w=\begin{vmatrix}
			u_1	&	u_2	&	u_3	\\
			v_1	&	v_2	&	v_3	\\
			w_1	&	w_2	&	w_3	
	\end{vmatrix}.
\end{equation}

Pourquoi nous ne considérons pas la combinaison $(u\cdot v)\times w$ ?

\begin{proposition}		 \label{PropScalMixtLin}
	Les applications produit scalaire, vectoriel et mixte sont multilinéaires. Spécifiquement, nous avons les propriétés suivantes.
	\begin{enumerate}
		\item
			Les applications produit scalaire et vectoriel sont bilinéaires. Le produit mixte est trilinéaire.
		\item
			Le produit vectoriel est antisymétrique, c'est à dire $u\times v=-v\times u$.
		\item
			Nous avons $u\times v=0$ si et seulement si $u$ et $v$ sont colinéaires, c'est à dire si et seulement si l'équation $\alpha u+\beta v=0$ a une solution différente de la solution triviale $(\alpha,\beta)=(0,0)$.
		\item		\label{ItemPropScalMixtLiniv}
			Pour tout $u$ et $v$ dans $\eR^3$, nous avons
			\begin{equation}
				\langle u, v\rangle^2 +\| u\times v \|^2=\| u \|^2\| v \|^2
			\end{equation}
		\item
			Par rapport à la dérivation, le produit scalaire et vectoriel vérifient une règle de Leibnitz. Soit $I$ un intervalle de $\eR$, et si $u$ et $u$ sont dans $C^1(I,\eR^3)$, alors
			\begin{equation}		\label{EqFormLeibProdscalVect}
				\begin{aligned}[]
					\frac{ d }{ dt }\big( u(t)\cdot v(t) \big)&=\big( u'(t)\cdot v(t) \big)+\big( u(t)\cdot v'(t) \big)\\
					\frac{ d }{ dt }\big( u(t)\times v(t) \big)&=\big( u'(t)\times v(t) \big)+\big( u(t)\times v'(t) \big).
				\end{aligned}
			\end{equation}
		\end{enumerate}
\end{proposition}

Les deux formules suivantes, qui mêlent le produit scalaire et le produit vectoriel, sont souvent utiles en analyse vectorielle :
\begin{equation}
	\begin{aligned}[]
		(u\times v)\cdot w&=u\cdot(v\times w)\\
		(u\times v)\times w&=-(v\cdot w)u+(u\cdot w)v		\label{EqFormExpluxxx}
	\end{aligned}
\end{equation}
pour tout vecteurs $u$, $v$ et $w$ dans $\eR^3$. Nous les admettons sans démonstration. La seconde formule est parfois appelée \defe{formule d'expulsion}{formule!d'expulsion (produit vectoriel)}.




%+++++++++++++++++++++++++++++++++++++++++++++++++++++++++++++++++++++++++++++++++++++++++++++++++++++++++++++++++++++++++++
\section{Boules et sphères}\label{Sect_boules}
%+++++++++++++++++++++++++++++++++++++++++++++++++++++++++++++++++++++++++++++++++++++++++++++++++++++++++++++++++++++++++++

\begin{definition}
	Soit $(V,\| . \|)$, un espace vectoriel normé, $a\in V$ et $r>0$. Nous allons abondamment nous servir des ensembles suivants :
	\begin{enumerate}

		\item
			la \defe{boule ouverte}{boule!ouverte} $B(a,r)=\{ x\in V\tq \| x-a \|<r \}$;
		\item
			la \defe{boule fermée}{boule!fermée} $\bar B(a,r)=\{ x\in V\tq \| x-a \|\leq r \}$;
		\item
			la \defe{sphère}{sphère} $S(a,r)=\{ x\in V\tq \| x-a \|=r \}$.

	\end{enumerate}
\end{definition}
Les différences entre ces trois ensembles sont très importantes. D'abord, les \emph{boules} sont pleines tandis que la \emph{sphère} est creuse. En comparant à une pomme, la boule ouverte serait la pomme «sans la peau», la boule fermée serait «avec la peau» tandis que la sphère serait seulement la peau. Nous avons
\begin{equation}
	\bar B(a,r)=B(a,r)\cup S(a,r).
\end{equation}

\begin{definition}
	Une partie $A$ de $V$ est dite \defe{bornée}{borné!partie de $V$} si il existe un réel $R$ tel que $A\subset B(0_V,R)$.
\end{definition}
Une partie est donc bornée si elle est contenue dans une boule de rayon fini.

\begin{example}
	Dans $\eR$, les boules sont  les intervalles ouverts et fermés tandis que la sphère est donnée par les points extrêmes des intervalles :
	\begin{equation}
		\begin{aligned}[]
			B(a,r)&=\mathopen] a-r , a+r \mathclose[,\\
			\bar B(a,r)&=\mathopen[ a-r , a+b \mathclose],\\
			S(a,r)&=\{ a-r,a+r \}.
		\end{aligned}
	\end{equation}
\end{example}

\begin{example}
	Si nous considérons $\eR^2$, la situation est plus riche parce que nous avons plus de normes. Essayons de voir les sphères de centre $(0,0)\in\eR^2$ et de rayon $r$ pour les normes $\| . \|_1$, $\| . \|_2$ et $\| . \|_{\infty}$.

	Pour la norme $\| . \|_1$, la sphère de rayon $r$ est donnée par l'équation
	\begin{equation}
		| x |+| y |=r.
	\end{equation}
	Pour la norme $\| . \|_2$, l'équation de la sphère de rayon $r$ est
	\begin{equation}
		\sqrt{x^2+y^2}=r,
	\end{equation}
	et pour la norme supremum, la sphère de rayon $r$ a pour équation
	\begin{equation}
		\max\{ | x |,| y | \}=r.
	\end{equation}
	Elles sont dessinées sur la figure \ref{LabelFigLesSpheres}
\newcommand{\CaptionFigLesSpheres}{Les sphères de rayon $1$ pour les trois normes classiques.}
\input{Fig_LesSpheres.pstricks}
\end{example}

\newcommand{\CaptionFigBoulePtLoin}{Le point $P$ est un peu plus loin que $x$, en suivant la même droite.}
\input{Fig_BoulePtLoin.pstricks}

\begin{proposition}		\label{PropBoitPtLoin}
	Soient $V$ un espace vectoriel normé, $a$ dans $V$ et $x$ tel que $d(a,x)=r$, c'est à dire $x\in S(a,r)$. Dans ce cas, toute boule centrée en $x$ contient un point $P$ tel que $d(P,a)>r$ et un point $Q$ tel que $d(Q,a)<r$.
\end{proposition}

\begin{proof}
	Soit une boule de rayon $\delta$ autour de $x$. Le but est de trouver un point $P$ tel que $d(P,a)>r$ et $d(P,x)<\delta$. Pour cela, nous prenons $P$ sur la même droite que $x$ (en partant de $a$), mais juste «un peu plus loin» (voir figure \ref{LabelFigBoulePtLoin}). Plus précisément, nous considérons le point
	\begin{equation}
		P=x+\frac{ v }{ N }
	\end{equation}
	où $v=x-a$ et $N$ est suffisamment grand pour que $d(x,P)$ soit plus petit que $\delta$. Cela est toujours possible parce que
	\begin{equation}
		d(P,x)=\| P-x \|=\frac{ \| v \| }{ N }
	\end{equation}
	peut être rendu aussi petit que l'on veut par un choix approprié de $N$. Montrons maintenant que $d(a,P)>d(a,x)$ :
	\begin{equation}
		\begin{aligned}[]
			d(a,P)&=\| a-x-\frac{ v }{ N }\| \\
			&=\| a-x+\frac{ a }{ N }-\frac{ x }{ N } \|\\
			&=\| \big( 1+\frac{1}{ N }(a-x) \big) \|\\
			&>\| a-x \|=d(a,x).
		\end{aligned}
	\end{equation}
	Nous laissons en exercice le soin de trouver un point $Q$ tel que $d(Q,a)<r$ et $d(Q,x)<\delta$.
\end{proof}

%+++++++++++++++++++++++++++++++++++++++++++++++++++++++++++++++++++++++++++++++++++++++++++++++++++++++++++++++++++++++++++
\section{Topologie}\label{Sect_topologie}
%+++++++++++++++++++++++++++++++++++++++++++++++++++++++++++++++++++++++++++++++++++++++++++++++++++++++++++++++++++++++++++

%---------------------------------------------------------------------------------------------------------------------------
\subsection{Ouverts, fermés, intérieur et adhérence}
%---------------------------------------------------------------------------------------------------------------------------

\begin{definition}
	Soit $(V,\| . \|)$ un espace vectoriel normé et $A$, une partie de $V$. Un point $a$ est dit \defe{intérieur}{intérieur!point} à $A$ si il existe une boule ouverte centrée en $a$ et contenue dans $A$.

	On appelle \defe{l'intérieur}{intérieur!d'un ensemble} de $A$ l'ensemble des points qui sont intérieurs à $A$. Nous notons $\Int(A)$ l'intérieur de $A$.
\end{definition}
Notons que $\Int(A)\subset A$ parce que si $a\in\Int(A)$, nous avons $B(a,r)\subset A$ pour un certain $r$ et en particulier $a\in A$.

\begin{example}
	Trouver l'intérieur d'un intervalle dans $\eR$ consiste à «ouvrir là où c'est fermé». 
	\begin{enumerate}

		\item
			$\Int\big(\mathopen[ 0 , 1 [\big)=\mathopen] 0 , 1 \mathclose[$. 
			
			Prouvons d'abord que $\mathopen] 0,1  \mathclose[\subset\Int(\mathopen[ 0 , 1 [)$. Si $a\in\mathopen] 0 , 1 \mathclose[$, alors $a$ est strictement supérieur à $0$ et strictement inférieur à $1$. Dans ce cas, la boule de centre $a$ et de rayon $\frac{ \min\{ a,1-a \} }{ 2 }$ est contenue dans $\mathopen] 0 , 1 \mathclose[$ (voir figure \ref{LabelFigIntervalleUn}). Cela prouve que $a$ est dans l'intérieur de $\mathopen[ 0 , 1 [$.

\newcommand{\CaptionFigIntervalleUn}{Trouver le rayon d'une boule autour de $a$. Une boule qui serait centrée en $a$ avec un rayon strictement plus petit à la fois de $a$ et de $1-a$ est entièrement contenue dans le segment $\mathopen] 0 , 1 \mathclose[$.}
\input{Fig_IntervalleUn.pstricks}

			Prouvons maintenant que $\Int\big( \mathopen[ 0 , 1 [ \big)\subset\mathopen] 0 , 1 \mathclose[$. Vu que l'intérieur d'un ensemble est inclus à l'ensemble, nous savons déjà que $\Int\big( \mathopen[ 0 , 1 [ \big)\subset\mathopen[ 0 , 1 [$. Nous devons donc seulement montrer que $0$ n'est pas dans l'intérieur de $\mathopen[ 0 , 1 [$. C'est le cas parce que toute boule du type $B(0,r)$ contient le point $-r/2$ qui n'est pas dans $\mathopen[ 0 , 1 [$.

		\item
			$\Int\Big( \mathopen[ 0 , \infty [ \Big)=\mathopen] 0 , \infty \mathclose[$.
		\item
			$\Int\big( \mathopen] 2 , 3 \mathclose[ \big)=\mathopen] 2 , 3 \mathclose[$.

	\end{enumerate}
	
\end{example}

\begin{example}			\label{ExempleIntBoules}
	Les intérieurs des boules et sphères sont importantes à savoir.
	\begin{enumerate}
		\item 
			$\Int\big( B(a,r) \big)=B(a,r)$. Si $x\in B(a,r)$, nous avons $d(a,x)<r$. Alors la boule $B\big(x,r-d(x,a)\big)$ est incluse à $B(a,r)$, et donc $x$ est dans l'intérieur de $B(a,r)$. Conseil : faire un dessin.
		\item
			$\Int\big( \bar B(a,r) \big)=B(a,r)$. Par le point précédent, la boule $B(a,r)$ est certainement dans l'intérieur de la boule fermée. Il reste à montrer que les points de $\bar B(a,r)$ qui ne sont pas dans $B(a,r)$ ne sont pas dans l'intérieur. Ces points sont ceux dont la distance à $a$ est \emph{égale} à $r$. Le résultat découle alors de la proposition \ref{PropBoitPtLoin}.
			
		\item
			$\Int\big( S(a,r) \big)=\emptyset$. Si $x\in S(a,r)$, toute boule centrée en $a$ contient des points qui ne sont pas à distance $r$ de $a$.
			
			Notez que la sphère est un exemple d'ensemble non vide mais d'intérieur vide.
	\end{enumerate}
\end{example}


\begin{definition}
	Une partie $A$ de l'espace vectoriel normé $(V,\| . \|)$ est dite \defe{ouverte}{ouvert} si chacun de ses points est intérieur. La partie $A$ est donc ouverte si $A\subset\Int(A)$. Par convention, nous disons que l'ensemble vide $\emptyset$ est ouvert.

	Une partie est dite \defe{fermée}{fermé} si son complémentaire est ouvert. La partie $A$ est donc fermée si $V\setminus A$ est ouverte.
\end{definition}

Remarque : un ensemble $A$ est ouvert si et seulement si $\Int(A)=A$.

\begin{definition}
	Une partie $A$ de l'espace vectoriel normé $V$ est dite \defe{compacte}{compact} si elle est fermée et bornée.
\end{definition}

Nous verrons tout au long de ce cours que les ensembles compacts, et les fonctions définies sur ces ensembles ont de nombreuses propriétés agraables.

\begin{example}		\label{ExempleFermeIntevrR}
	En ce qui concerne les intervalles de $\eR$,
	\begin{itemize}
		\item $\mathopen] 1 , 2 \mathclose[$ est ouvert;
		\item $\mathopen[ 3,  4 \mathclose]$ est fermé;
		\item $\mathopen[ 5 , 6 [$ n'est ni ouvert ni fermé;
	\end{itemize}
	Les intervalles fermés de $\eR$ sont toujours compacts.
\end{example}

\begin{proposition}		\label{PropTopologieAx}
	Soit $V$ un espace vectoriel normé.
	\begin{enumerate}
		\item
			L'ensemble $V$ lui-même et le vide sont à la fois fermées et ouvertes.
		\item
			Toute union d'ouverts est ouverte.
		\item
			Toute intersection \emph{finie} d'ouverts est ouverte.
		\item		\label{ItemPropTopologieAxiv}
			Le vide et $V$ sont les seules parties de $V$ à être à la fois fermées et ouvertes.
	\end{enumerate}
\end{proposition}

\begin{proof}
	L'ingrédient principal de cette démonstration est que si $a$ est un point d'un ouvert $\mO$, alors il existe une boule autour de $a$ contenue dans $\mO$ parce que $a$ doit être dans l'intérieur de $\mO$.
	\begin{enumerate}

		\item
			Nous avons déjà dit que, par définition, l'ensemble vide est ouvert. Cela implique que $V$ lui-même est fermé (parce que son complémentaire est le vide). De plus, $V$ est ouvert parce que toutes les boules sont inclues à $V$. Le vide est alors fermé (parce que son complémentaire est $V$).
		\item
			Soit une famille $(\mO_i)_{i\in I}$ d'ouverts\footnote{L'ensemble $I$ avec lequel nous «numérotons» les ouverts $\mO_i$ est \emph{quelconque}, c'est à dire qu'il peut être $\eN$, $\eR$, $\eR^n$ ou n'importe quel autre ensemble, fini ou infini.}, et l'union
			\begin{equation}
				\mO=\bigcup_{i\in I}\mO_i.
			\end{equation}
			Soit maintenant $a\in\mO$. Nous devons prouver qu'il existe une boule centrée en $a$ entièrement contenue dans $\mO$. Étant donné que $a\in\mO$, il existe $i\in I$ tel que $a\in\mO_i$ (c'est à dire que $a$ est au moins dans un des $\mO_i$). Par hypothèse l'ensemble $\mO_i$ est ouvert et donc tous ses points (en particulier $a$) sont intérieurs; il existe donc une boule $B(a,r)$ centrée en $a$ telle que $B(a,r)\subset\mO_i\subset\mO$.
		
		\item
			Soit une famille finie d'ouverts $(\mO_k)_{k\in\{ 1,\ldots,n \}}$, et $a\in\mO$ où
			\begin{equation}
				\mO=\bigcap_{k=1}^n\mO_k.
			\end{equation}
			Vu que $a$ appartient à chaque ouvert $\mO_k$, nous pouvons trouver, pour chacun de ces ouverts, une boule $B(a,r_k)$ contenue dans $\mO_k$. Chacun des $r_k$ est strictement positif, et nous n'en avons qu'un nombre fini, donc le nombre $r=\min\{ r_1,\ldots,r_n \}$ est strictement positif. La boule $B(a,r)$ est inclue dans toutes les autres (parce que $B(a,r)\subset B(a,r')$ lorsque $r\leq r'$), par conséquent
			\begin{equation}
				B(a,r)\subset\bigcap_{k=1}^nB(a,r_k)\subset\bigcap_{k=1}^n\mO_k=\mO,
			\end{equation}
			c'est à dire que la boule de rayon $r$ est une boule centrée en $a$ contenue dans $\mO$, ce qui fait que $a$ est intérieur à $\mO$.
		\item
			Nous acceptons ce point sans démonstration. 
	\end{enumerate}
   % TODO : trouver et mettre une preuve du dernier point.
	
\end{proof}

La proposition dit que toute intersection \emph{finie} d'ouvert est ouverte. Il est faux de croire que cela se généralise aux intersections infinies, comme le montre l'exemple suivant :
\begin{equation}
	\bigcap_{i=1}^{\infty}\mathopen] -\frac{1}{ n } , \frac{1}{ n } \mathclose[=\{ 0 \}.
\end{equation}
Chacun des ensembles $\mathopen] -\frac{1}{ n } , \frac{1}{ n } \mathclose[$ est ouvert, mais le singleton $\{ 0 \}$ est fermé (pourquoi ?).

Nous reportons à la proposition \ref{PropBorneSupInf} la preuve du fait que tout ensemble borné de $\eR$ possède un infimum et un supremum.



\begin{definition}
	L'ensemble des ouverts de $V$ est la \defe{topologie}{topologie} de $V$. La topologie dont nous parlons ici est dite \defe{induite}{induite!topologie} par la norme $\| . \|$ de $V$ (parce que cette norme définit la notion de boule et qu'à son tour la notion de boule définit la notion d'ouverts). Un \defe{voisinage}{voisinage} de $a$ dans $V$ est un ensemble contenant un ouvert contenant $a$.
\end{definition}

Il existe de nombreuses topologies sur un espace vectoriel donné, mais certaines sont plus fameuses que d'autres. Dans le cas de $V=\eR^n$, la topologie \defe{usuelle}{topologie!usuelle sur $\eR^n$} est celle induite par la norme euclidienne. Lorsque nous parlons de boules, de fermés, de voisinages ou d'autres notions topologiques (y compris de convergence, voir plus bas) dans $\eR^n$, nous sous-entendons toujours la topologie de la norme euclidienne.

\begin{example}
	Les ensemble suivants sont des voisinages de $3$ dans $\eR$ :
	\begin{itemize}
		\item
			$\mathopen] 1 , 5 \mathclose[$;
		\item
			$\mathopen[ 0 , 10 \mathclose]$;
		\item
			$\eR$.
	\end{itemize}
	Les ensembles suivants ne sont pas des voisinages de $3$ dans $\eR$ :
	\begin{itemize}
		\item 
			$\mathopen] 1 , 3 \mathclose[$;
		\item
			$\mathopen] 1 , 3 \mathclose]$;
		\item
			$\mathopen[ 0 , 5 [\setminus\{ 3 \}$.
	\end{itemize}
\end{example}

\begin{proposition}
	Dans un espace vectoriel normé,
	\begin{enumerate}
		\item
			toute intersection de fermés est fermée;
		\item
			toute union \emph{finie} de fermés est fermée.
	\end{enumerate}
\end{proposition}
Encore une fois, l'hypothèse de finitude de l'intersection est indispensable comme le montre l'exemple suivant :
\begin{equation}
	\bigcup_{n=1}^{\infty}\mathopen[ -1+\frac{1}{ n } , 1-\frac{1}{ n } \mathclose]=\mathopen] -1 , 1 \mathclose[.
\end{equation}
Chacun des intervalles dont on prend l'union est fermé tandis que l'union est ouverte.

\begin{definition}
	Soit $A$, une partie de l'espace vectoriel normé $V$. Un point $a\in V$ est dit \defe{adhérent}{adhérence} à $A$ dans $V$ si pour tout $\varepsilon>0$,
	\begin{equation}
		B(a,\varepsilon)\cap A\neq\emptyset.
	\end{equation}
	Nous notons $\bar A$ l'ensemble des points adhérents à $a$ et nous disons que $\bar A$ est l'adhérence de $A$. L'ensemble $\bar A$ sera aussi souvent nommé \defe{fermeture}{fermeture} de l'ensemble $A$.
\end{definition}
Un point peut être adhérent à $A$ sans faire partie de $A$, et nous avons toujours $A\subset\bar A$.

\begin{example}
	La terminologie «fermeture» de $A$ pour désigner $\bar A$ provient de deux origines.
	\begin{enumerate}
		\item
			L'ensemble $\bar A$ est le plus petit fermé contenant $A$. Cela signifie que si $B$ est un fermé qui contient $A$, alors $\bar A\subset A$. Nous acceptons cela sans preuve.
            % position 25804
            %Nous allons prouver cette affirmation dans l'exercice \ref{exoGeomAnal-0008}.
		\item
			Pour les intervalles dans $\eR$, trouver $\bar A$ revient à fermer les extrémités qui sont ouvertes, comme on en a parlé dans l'exemple \ref{ExempleFermeIntevrR}.
	\end{enumerate}
\end{example}

\begin{example}
	Dans $\eR$, l'infimum et le supremum d'un ensemble sont des points adhérents. En effet si $M$ est le supremum de $A\subset\eR$, pour tout $\varepsilon$, il existe un $a\in A$ tel que $a>M-\varepsilon$, tandis que $M>a$. Cela fait que $a\in B(M,\varepsilon)$, et en particulier que pour tout rayon $\varepsilon$, nous avons $B(M,\varepsilon)\cap A\neq\emptyset$.

	Le même raisonnement montre que l'infimum est également dans l'adhérence de $A$.
\end{example}

\begin{example}		\label{ParlerEncoredeF}
	Il ne faut pas conclure de l'exemple précédent qu'un point limite ou adhérent est automatiquement un minimum ou un maximum. En effet, si nous regardons l'ensemble formé par les points de la suite $x_n=(-1)^n/n$, le nombre zéro est un point adhérent et une limite, mais pas un infimum ni un maximum.
\end{example}

\begin{lemma}
	Si $B$ est une partie fermée de $V$, alors $B=\bar B$.
\end{lemma}

\begin{proof}
	Supposons qu'il existe $a\in\bar B$ tel que $a\notin B$. Alors il n'y a pas d'ouverts autour de $a$ qui soit contenu dans $\complement B$. Cela prouve que $\complement B$ n'est pas ouvert, et par conséquent que $B$ n'est pas fermé. Cela est une contradiction qui montre que tout point de $\bar B$ doit appartenir à $B$ lorsque $B$ est fermé.
\end{proof}

\begin{example}
	Au niveau des intervalles dans $\eR$, prendre l'adhérence consiste à «fermer là où c'est ouvert». Attention cependant à ne pas fermer l'intervalle en l'infini.
	\begin{enumerate}
		\item
			$\overline{ \mathopen[ 0 , 2 [ }=\mathopen[ 0 , 2 \mathclose]$.
		\item
			$\overline{ \mathopen] 3 , \infty \mathopen] }=\mathopen[ 3 , \infty [$.
	\end{enumerate}
\end{example}

\begin{proposition}
	Soit $V$ un espace vectoriel normé et $a\in V$. Les trois conditions suivantes sont équivalentes :
	\begin{enumerate}
		\item
			$a\in\bar A$;
		\item
			il existe une suite d'éléments $x_n$ dans $A$ qui converge vers $a$;
		\item
			$d(a,A)=0$.
	\end{enumerate}
\end{proposition}
Notez que dans cette proposition, nous ne supposons pas que $a$ soit dans $A$.

\begin{proposition}		\label{PropComleIntBar}
	Pour toute partie $A$ d'un espace vectoriel normé nous avons
	\begin{enumerate}
		\item
			$V\setminus\bar A=\Int(V\setminus A)$,
		\item
			$V\setminus\Int(A)=\overline{ V\setminus A }$.
	\end{enumerate}
\end{proposition}

En utilisant les notations du complémentaire (\ref{AppComplement}), les deux points de la proposition se récrivent
\begin{enumerate}
	\item
		$\complement \bar A=\Int(\complement A)$,
	\item\label{ItemLemPropComplementiv}
		$\complement\Int(A)=\overline{ \complement A }$.
\end{enumerate}

\begin{proof}
	Nous avons $a\in V\setminus\bar A$ si et seulement si $a\notin\bar A$. Or ne pas être dans $\bar A$ signifie qu'il existe un rayon $\varepsilon$ tel que la boule $B(a,\varepsilon)$ n'intersecte pas $A$. Le fait que la boule $B(a,\varepsilon)$ n'intersecte pas $A$ est équivalent à dire que $B(a,\varepsilon)\subset V\setminus A$. Or cela est exactement la définition du fait que $a$ est à l'intérieur de $V\setminus A$. Nous avons donc montré que $a\in V\setminus \bar A$ si et seulement si $a\in\Int(V\setminus A)$. Cela prouve la première affirmation.

	Pour prouver la seconde affirmation, nous appliquons la première au complémentaire de $A$ : $\complement(\overline{ \complement A })=\Int(\complement\complement A)$. En prenant le complémentaire des deux membres nous trouvons successivement
	\begin{equation}
		\begin{aligned}[]
			\complement\complement(\overline{ \complement A })&=\complement\Int(\complement\complement A),\\
			\overline{ \complement A }&=\complement\Int(A),
		\end{aligned}
	\end{equation}
	ce qu'il fallait démontrer.
\end{proof}

Attention à ne pas confondre $\complement \bar A$ et $\overline{ \complement A }$. Ces deux ensembles ne sont pas égaux. En effet, en tant que complément d'un fermé, l'ensemble $\complement \bar A$ est certainement ouvert, tandis que, en tant que fermeture, l'ensemble $\overline{ \complement A }$ est fermé. Pouvez-vous trouver des exemples d'ensembles $A$ tels que $\complement \bar A=\overline{ \complement A }$ ?

\begin{proposition}
	Soient $A$ et $B$ deux parties de l'espace vectoriel normé $V$.
	\begin{enumerate}
		\item
			Pour les inclusions, si $A\subset B$, alors $\Int(A)\subset\Int(B)$ et $\bar A\subset\bar B$.
		\item
			Pour les unions, $\overline{ A\cup B }=\overline{ A }\cup\overline{ B }$ et $\overline{ A\cap B }\subset\bar A\cap\bar B$.
		\item
			Pour les intersections, $\Int(A)\cap\Int(B)=\Int(A\cap B)$ et $\Int(A)\cup\Int(B)\subset\Int(A\cup B)$.
	\end{enumerate}
\end{proposition}

\begin{proof}
	\begin{enumerate}
		\item
			Si $a$ est dans l'intérieur de $A$, il existe une boule autour de $a$ contenue dans $A$. Cette boule est alors contenue dans $B$ et donc est une boule autour de $a$ contenue dans $B$, ce qui fait que $a$ est dans l'intérieur de $B$. Si maintenant $a$ est dans l'adhérence de $A$, toute boule centrée en $a$ contient un élément de $A$ et donc un élément de $B$, ce qui prouve que $a$ est dans l'adhérence de $B$.
		\item
			Nous avons $A\subset A\cup B$ et donc, en utilisant le premier point, $\bar A\subset\overline{ A\cup B }$. De la même manière, $\bar B\subset\overline{ A\cup B }$. En prenant l'union, $\bar A\cup\bar B\subset\overline{ A\cup B }$.

			Réciproquement, soit $a\in\overline{ A\cup B }$ et montrons que $a\in\bar A\cup\bar B$. Supposons par l'absurde que $a$ ne soit ni dans $\bar A$ ni dans $\bar B$. Il existe donc des rayon $\varepsilon_1$ et $\varepsilon_2$ tels que
			\begin{equation}
				\begin{aligned}[]
					B(a,\varepsilon_1)\cap A&=\emptyset,\\
					B(a,\varepsilon_2)\cap B&=\emptyset.
				\end{aligned}
			\end{equation}
			En prenant $r=\min\{ \varepsilon_1,\varepsilon_2 \}$, la boule $B(a,r)$ est inclue aux deux boules citées et donc n'intersecte ni $A$ ni $B$. Donc $a\notin\overline{ A\cup B }$, d'où la contradiction.

		\item
			Si nous appliquons le second point à $\complement A$ et $\complement B$, nous trouvons
			\begin{equation}
				\overline{ \complement A\cup\complement B }=\overline{ \complement A}\cup\overline{ \complement B}.
			\end{equation}
			En utilisant les propriétés du lemme \ref{LemPropsComplement}, le membre de gauche devient
			\begin{equation}	\label{Eq2707CACBCAB}
				\overline{ \complement A\cup\complement B }=\overline{ \complement(A\cap B) }=\complement\Int(A\cap B),
			\end{equation}
			tandis que le membre de droite devient
			\begin{equation}		\label{Eq2707cAcBACAACB}
				\overline{ \complement A }\cup\overline{ \complement B }=\complement\Int(A)\cup\complement\Int(A)=\complement\Big( \Int(A)\cap\Int(B) \Big).
			\end{equation}
			En égalisant le membre de droite de \eqref{Eq2707CACBCAB} avec celui de \eqref{Eq2707cAcBACAACB} et en passant au complémentaire nous trouvons
			\begin{equation}
				\Int(A\cap B)=\Int(A)\cap\Int(B),
			\end{equation}
			comme annoncé.

			La dernière affirmation provient du fait que $\Int(A)\subset\Int(A\cup B)$ et de la propriété équivalente pour $B$.
	\end{enumerate}
\end{proof}

\begin{remark}
	Nous avons prouvé que $\overline{ A\cap B }\subset\bar A\cap\bar B$. Il arrive que l'inclusion soit stricte, comme dans l'exemple suivant. Si nous prenons $A=\mathopen[ 0 , 1 \mathclose]$ et $B=\mathopen] 1 , 2 \mathclose]$, nous avons $A\cap B=\emptyset$ et donc $\overline{ A\cap B }=\emptyset$. Par contre nous avons $\bar A\cap\bar B=\{ 1 \}$.
\end{remark}

\begin{definition}
	La \defe{frontière}{frontière} d'un sous-ensemble $A$ de l'espace vectoriel normé $V$ est l'ensemble des points $a\in V$ tels que
	\begin{equation}
		\begin{aligned}[]
			B(a,r)\cap A&\neq \emptyset,\\
			B(a,r)\cap \complement A&\neq \emptyset,
		\end{aligned}
	\end{equation}
	pour tout rayon $r$. En d'autres termes, toute boule autour de $a$ contient des points de $A$ et des points de $\complement A$. La frontière de $A$ se note $\partial A$\nomenclature[T]{$\partial A$}{La frontière de l'ensemble $A$}.
\end{definition}

\begin{proposition}		\label{PropDescFrpbsmI}
	La frontière d'une partie $A$ d'un espace vectoriel normé $V$ s'exprime sous la forme
	\begin{equation}
		\partial A=\bar A\setminus\Int(A).
	\end{equation}
\end{proposition}

\begin{proof}
	Le fait pour un point $a$ de $V$ d'appartenir à $\bar A$ signifie que toute boule centrée en $a$ intersecte $A$. De la même façon, le fait de ne pas appartenir à $\Int(A)$ signifie que toute boule centrée en $a$ intersecte $\complement A$.
\end{proof}

La description de la frontière donnée par la proposition \ref{PropDescFrpbsmI} est celle qu'en pratique nous utilisons le plus souvent. Dans certains textes, elle est prise comme définition de la frontière.

\begin{lemma}
	La frontière de $A$ peut également s'exprimer des façons suivantes :
	\begin{equation}
		\partial A= \bar A\cap\complement\Int(A)=\bar A\cap\overline{ \complement A },
	\end{equation}
\end{lemma}

\begin{proof}
	En partant de $\partial A=\bar A\setminus \Int(A)$, la première égalité est une application de la propriété \ref{ItemLemPropComplementiii} du lemme \ref{LemPropsComplement}. La seconde égalité est alors la proposition \ref{PropComleIntBar}.
\end{proof}

\begin{example}
	Dans $\eR$, la frontière d'un intervalle est la paire constituée des points extrêmes. En effet
	\begin{equation}
		\partial\mathopen[ a , b [=\overline{ \mathopen[ a , b [ }\setminus\Int\big( \mathopen[ a , b [ \big)=\mathopen[ a , b \mathclose]\setminus\mathopen] a , b \mathclose[=\{ a,b \}.
	\end{equation}

	Toujours dans $\eR$ nous avons
	\begin{equation}
		\partial\eR=\bar\eR\setminus\Int(\eR)=\eR\setminus\eR=\emptyset,
	\end{equation}
	et
	\begin{equation}
		\partial\eQ=\bar\eQ\setminus\Int(\eQ)=\eR\setminus\emptyset=\eR.
	\end{equation}
\end{example}

%TODO : prouver que la boule fermée est la fermeture de la boule ouverte.

\begin{example}
	Dans $\eR^n$, nous avons
	\begin{equation}
		\partial B(a,r)=\partial\bar B(a,r)=S(a,r).
	\end{equation}

    Cela est un boulot pour la proposition \ref{PropBoitPtLoin}. Si \( x\in S(a,r)\) alors tout boule autour de \( x\) contient des points à distance strictement plus grande et plus petite que \( d(a,x)\), c'est à dire des points dans \( B(a,r)\) et hors de \( B(a,r)\). Cela prouve que les points de \( S(a,r)\) font partie de \( \partial B(a,r)\), c'est à dire que \( S(a,r)\subset \partial B(a,r)\); et idem pour \( \bar B(a,r)\). 

Pour prouver l'inclusion inverse, soit \( x\in \partial B(a,r)\). Vu que toute boule autour de \( x\) contient des points intérieurs à \( B(a,r)\), pour tout \( \epsilon>0\), \( d(a,x)-\epsilon< r \), c'est à dire que \( d(a,x)\leq r\). De la même manière toute boule autour de \( x\) contient des points hors de \( B(a,r)\) signifie que pour tout \( \epsilon\), \( d(a,x)+\epsilon>r\) ou encore que \( d(a,x)\geq r\). Les deux ensemble implique que \( d(a,x)=r\).
\end{example}

\begin{remark}
    Il serait toutefois faux de croire que \( \partial A=\partial \bar A\) pour toute partie \( A\) de \( \eR^n\). En effet si \( A=\eR\setminus\{ 0 \}\) nous avons \( \partial A=\{ 0 \}\) et \( \bar A=\eR\), donc \( \partial \bar A=\emptyset\).
\end{remark}

%---------------------------------------------------------------------------------------------------------------------------
\subsection{Point isolé, point d'accumulation}
%---------------------------------------------------------------------------------------------------------------------------

\begin{definition}
	Soit $D$, une partie de $V$.
	\begin{enumerate}
		\item
			Un point $a\in D$ est dit \defe{isolé}{isolé!point dans un espace vectoriel normé} dans $D$ relativement à $V$ si il existe un $\varepsilon>0$ tel que
			\begin{equation}
				B(a,\varepsilon)\cap D=\{ a \}.
			\end{equation}
		\item
			Un point $a\in V$ est un \defe{point d'accumulation}{accumulation!dans espace vectoriel normé} de $D$ si pour tout $\varepsilon>0$,
			\begin{equation}
				\Big( B(a,\varepsilon)\setminus\{ a \}\Big)\cap D\neq \emptyset.
			\end{equation}
	\end{enumerate}
\end{definition}

\newcommand{\CaptionFigAccumulationIsole}{L'ensemble décrit par l'équation \eqref{Eq2807BouleIso}. Le point $P$ est un point isolé de $D$, tandis que  les points $S$ et $Q$ sont des points d'accumulation.}
\input{Fig_AccumulationIsole.pstricks}

\begin{example}
	Considérons la partie suivante de $\eR^2$ :
	\begin{equation}	\label{Eq2807BouleIso}
		D=\{ (x,y)\tq x^2+y^2<1\}\cup\{ (1,1) \}.
	\end{equation}
	Comme on peut le voir sur la figure \ref{LabelFigAccumulationIsole}, le point $P=(1,1)$ est un point isolé de $D$ parce qu'on peut tracer une boule autour de $P$ sans inclure d'autres points de $D$ que $P$ lui-même. Le point $Q=(-1,0)$ est un point d'accumulation de $D$ parce que toute boule autour de $Q$ contient des points de $D$.

    Le point $S$, étant un point intérieur, est un point d'accumulation : toute boule autour de $S$ intersecte $D$.
    
    Notez cependant que le point $Q$ lui-même n'est pas dans $D$ parce que l'inégalité qui définit $D$ est stricte.
\end{example}

\begin{remark}
    À propos de la position des points d'accumulation et des points isolés.
    \begin{enumerate}
        \item
            Les points intérieurs sont tous des points d'accumulation.
        \item
            Les points isolés ne sont jamais intérieurs.
        \item
            Certains points d'accumulation ne font pas partie de l'ensemble. Par exemple le point $1$ est un point d'accumulation de $E=\mathopen] 0 , 1 \mathclose[$.
        \item
            Les points de la frontière sont soit d'accumulation soit isolés.
    \end{enumerate}
\end{remark}


\begin{example}
	Tous les points de $\eR$ sont des points d'accumulation de $\eQ$ parce que dans toute boule autour d'un réel, on peut trouver un nombre rationnel.
\end{example}

\begin{remark}
	L'ensemble des points d'accumulation d'un ensemble n'est pas exactement son adhérence. En effet, un point isolé dans $A$ est dans l'adhérence de $A$, mais n'est pas un point d'accumulation de $A$.
\end{remark}

%+++++++++++++++++++++++++++++++++++++++++++++++++++++++++++++++++++++++++++++++++++++++++++++++++++++++++++++++++++++++++++
\section{Équivalence des normes}
%+++++++++++++++++++++++++++++++++++++++++++++++++++++++++++++++++++++++++++++++++++++++++++++++++++++++++++++++++++++++++++
\label{normes_equiv}

Au premier coup d'œil, les notions dont nous parlons dans ce chapitre ont l'air très générales. Nous prenons en effet n'importe quel espace vectoriel $V$ de dimension finie, et nous le munissons de n'importe quelle norme (rien que dans $\eR^m$ nous en avons défini une infinité par l'équation \eqref{EqDeformeLp}). À partir de ces données, nous définissons les boules, la topologie, l'adhérence, etc.

%---------------------------------------------------------------------------------------------------------------------------
\subsection{En dimension finie}
%---------------------------------------------------------------------------------------------------------------------------

Dans $\eR^n$, les normes $\| . \|_{L^1}$, $\| . \|_{L^2}$ et $\| . \|_{\infty}$ ne sont pas égales. Cependant elles ne sont pas complètement indépendante au sens où l'on sent bien que si un vecteur sera grand pour une norme, il sera également grand pour les autres normes; les normes «vont dans le même sens». Cette notion est précisée par le concept de norme équivalente. 

\begin{definition}		\label{DefEquivNorm}
    Deux normes $N_1$ et $N_2$ sur $\eR^m$ sont \defe{\wikipedia{fr}{Norme_équivalente}{équivalentes}}{equivalence@équivalence!norme}\index{norme!équivalence}\index{équivalence!de norme} si il existe deux nombres réels strictement positifs $k_1$ et $k_2$ tels que
	\begin{equation}
		k_1N_1(x)\leq N_2(x)\leq k_2 N_1(x),
	\end{equation}
	pour tout $x$ dans $\eR^m$. Dans ce cas nous écrivons que $N_1\sim N_2$.
\end{definition}
Il est possible de démontrer que cette notion est une relation d'équivalence (définition \ref{DefHoJzMp}) sur l'ensemble des normes existantes sur $\eR^m$.

\begin{proposition} \label{PropLJEJooMOWPNi}
    Pour \( \eR^Nous\), nous avons les équivalences de normes $\| . \|_{L^1}\sim\| . \|_{L^2}$, $\| . \|_{L^1}\sim\| . \|_{\infty}$ et $\| . \|_{L^2}\sim\| . \|_{\infty}$. Plus précisément nous avons les inégalités
    \begin{enumerate}
        \item\label{ItemABSGooQODmLNi}
           $ \| x \|_2\leq \| x \|_1\leq\sqrt{n}\| x \|_2$
        \item\label{ItemABSGooQODmLNii}
            $\| x \|_{\infty}\leq \| x \|_1\leq n \| x \|_{\infty}$
        \item\label{ItemABSGooQODmLNiii}
            $\| x \|_{\infty}\leq \| x \|_2\leq \sqrt{n}\| x \|_{\infty}$
    \end{enumerate}
\end{proposition}

\begin{proof}
    En mettant au carré la première inégalité nous voyons que nous devons vérifier l'inégalité
    \begin{equation}
        | x_1 |^2+\ldots+| x_n |^2\leq\big( | x_1 |+\ldots+| x_n | \big)^2
    \end{equation}
    qui est vraie parce que le membre de droite est égal au carré de chaque terme plus les double produits. La seconde inégalité provient de l'inégalité de Cauchy-Schwarz (théorème \ref{ThoAYfEHG}) sur les vecteurs
    \begin{equation}
        \begin{aligned}[]
            v&=\begin{pmatrix}
                1/n    \\ 
                \vdots    \\ 
                1/n    
            \end{pmatrix},
            &w&=\begin{pmatrix}
                | x_1 |    \\ 
                \vdots    \\ 
                | x_n |    
            \end{pmatrix}.
        \end{aligned}
    \end{equation}
    Nous trouvons 
    \begin{equation}
        \frac{1}{ n }\sum_i| x_i |\leq\sqrt{b\cdot\frac{1}{ n }}\sqrt{\sum_i| x_i |^2},
    \end{equation}
    et par conséquent
    \begin{equation}
        \sum_i| x_i |\leq\sqrt{n}\| x \|_2.
    \end{equation}
    
    La première inégalité de \ref{ItemABSGooQODmLNiii}<++> se démontre en remarquant que si \( a\) et \( b\) sont positifs, \( a\leq\sqrt{a^2+b}\). En appliquant cela à \( a=\max_i| x_i |\), nous avons
    \begin{equation}
        \max_i| x_i |\leq\sqrt{ | x_1 |^2+\ldots+| x_n |^2  }
    \end{equation}
    parce que \( \max_i| x_i |\) est évidemment un des termes de la somme. Pour la seconde inégalité de \ref{ItemABSGooQODmLNiii}, nous avons
    \begin{equation}
        \sqrt{\sum_k| x_k |^2}\leq\left( \sum_k\max_i| x_i |^2 \right)^{1/2}=\sqrt{n}\| x \|_{\infty}.
    \end{equation}
    Pour obtenir cette inégalité, nous avons remplacé tous les termes \( | x_k |\) par le maximum.
\end{proof}

En réalité, toutes les normes \( \| . \|_{L^p}\) et \( \| . \|_{\infty}\) sont équivalentes et, plus généralement, nous avons le résultat suivant, très étonnant à première vue, et en réalité assez difficile à prouver :
\begin{theorem}[\cite{TrenchRealAnalisys}]		\label{ThoNormesEquiv}
	Sur un espace vectoriel de dimension finie, toutes les normes (pas seulement les normes $L^p$ que nous avons définies sur $\eR^m$) sont équivalentes.
\end{theorem}
% TODO : la preuve est à la page 583 de Trench.

\begin{corollary}
    Soit \( V\) un espace vectoriel de dimension finie et \( \| . \|_1\), \( \| . \|_2\) deux normes sur \( V\). Alors l'identité \( \id\colon V\to V\) est un isomorphisme d'espace topologique \( (V,\| . \|_1)\to (V,\| . \|_2)\).

    De plus les ouverts sont les mêmes : une partie de \( V\) est ouverte dans \( (V,\| . \|_1)\) si et seulement si elle est ouverte dans \( (V,\| . \|_2)\).
\end{corollary}

Le théorème d'équivalence de norme sera utilisé pour montrer que l'ensemble des formes quadratiques non dégénérées de signature \( (p,q)\) est ouvert dans l'ensemble des formes quadratiques, proposition \ref{PropNPbnsMd}. Plus généralement il est utilisé à chaque fois que l'on fait de la topologie sur les espaces de matrices en identifiant \( \eM(n,\eR)\) à \( \eR^{n^2}\), pour se rassurer en se disant que ce qu'on fait ne dépend pas de la norme choisie.

%---------------------------------------------------------------------------------------------------------------------------
\subsection{Contre-exemple en dimension infinie}
%---------------------------------------------------------------------------------------------------------------------------
\label{SubSecPOlynomesCE}

Lorsque nous considérons des espaces vectoriels de dimension infinie, les choses ne sons plus aussi simples. Nous voyons ici sur l'exemple de l'espace des polynômes que le théorème \ref{ThoNormesEquiv} n'est plus valable si on enlève l'hypothèse de dimension finie.

On considère l'ensemble des fonctions polynômiales à coefficients réels sur  l'intervalle $[0,1]$.
\begin{equation}
\mathcal{P}_\eR([0,1])=\{p:[0,1]\to \eR\,|\, p : x\mapsto a_0+a_1 x +a_2 x^2 + \ldots, \, a_i\in\eR,\,\forall i\in \eN\}.
\end{equation}
Cet ensemble, muni des opérations usuelles de somme entre polynômes et multiplications par les scalaires, est un espace vectoriel.  

Sur $\mathcal{P}(\eR)$ on définit les normes suivantes 
\begin{equation}
\begin{aligned}
&\|p\|_\infty=\sup_{x\in[0,1]}\{p(x)\},\\
&\|p\|_1 =\int_0^1|p(x)|\, dx,\\
&\|p\|_2 =\left(\int_0^1|p(x)|^2\, dx\right)^{1/2}.\\
\end{aligned}
\end{equation}
Les inégalités suivantes sont  immédiates
\begin{equation}
\begin{aligned}
&\|p\|_1 =\int_0^1|p(x)|\, dx\leq \|p\|_\infty,\\
&\|p\|_2 =\left(\int_0^1|p(x)|^2\, dx\right)^{1/2}\leq \|p\|_\infty,\\
\end{aligned}
\end{equation}
mais la norme $\|\cdot\|_\infty$ n'est  équivalente ni à $\|\cdot\|_1$, ni à $\|\cdot\|_2$. Soit $p_k(x)= x^k$. Alors
\begin{equation}
\begin{aligned}
&\|p_k\|_\infty=1,\\
&\|p_k\|_1 =\int_0^1x^k\, dx=  \frac{1}{k+1},\\
&\|p_k\|_2 =\left(\int_0^1x^{2k}\, dx\right)^{1/2}=\sqrt{\frac{1}{2k+1}}.
\end{aligned}
\end{equation}
Pour $k\to \infty$ les normes $\|p_k\|_1$, $\|p_k\|_2$ tendent vers zéro, alors que la norme $\|p_k\|_\infty$ est constante, donc les normes ne sont pas équivalentes parce que il n'existe pas un nombre positif $m$ tel que 
\begin{equation}
\begin{aligned}
& m \|p_k\|_\infty\leq \|p_k\|_1 ,\\
& m \|p_k\|_\infty\leq \|p_k\|_2 ,\\
\end{aligned}
\end{equation}
uniformément pour tout $k$ dans $\eN$.

%+++++++++++++++++++++++++++++++++++++++++++++++++++++++++++++++++++++++++++++++++++++++++++++++++++++++++++++++++++++++++++
\section{Suites}
%+++++++++++++++++++++++++++++++++++++++++++++++++++++++++++++++++++++++++++++++++++++++++++++++++++++++++++++++++++++++++++

%--------------------------------------------------------------------------------------------------------------------------- 
\subsection{Limites, convergence}
%---------------------------------------------------------------------------------------------------------------------------

Nous disons qu'une suite réelle $(x_n)$ converge\footnote{Voir la définition \ref{DefLimiteSuiteNum} pour plus de détail.} vers $\ell$ lorsque pour tout $\varepsilon$, il existe un $M$ tel que
\begin{equation}
	n>N\Rightarrow | x_n-\ell |\leq\varepsilon.
\end{equation}
Le concept fondamental de cette définition est la notion de valeur absolue qui permet de donner la «distance» entre deux réels. Dans un espace vectoriel normé quelconque, cette notion est généralisée par la distance associée à la norme (définition \ref{DefEVNetDistance}). Nous pouvons donc facilement définir le concept de convergence d'une suite dans un espace vectoriel normé.

\begin{definition}		\label{DefCvSuiteEGVN}
	Soit une suite $(x_n)$ dans un espace vectoriel normé $V$. Nous disons qu'elle est \defe{convergente}{convergence!dans un espace vectoriel normé} si il existe un élément $\ell\in V$ tel que
	\begin{equation}
		\forall \varepsilon>0,\,\exists N\in\eN\tq n\geq N\Rightarrow \| x_n-l \|<\varepsilon.
	\end{equation}
	Dans ce cas, $\ell$ est appelé la \defe{limite}{limite!suite} de la suite $(x_n)$.
\end{definition}


\begin{lemma}		\label{LemLimAbarA}
	Soit $(x_n)$ une suite convergente contenue dans un ensemble $A\subset V$. Alors la limite $x_n$ appartient à $\bar A$.
\end{lemma}

\begin{proof}
	Supposons que nous ayons une partie $A$ de $V$, et une suite $(x_n)$ dont la limite $\ell$ se trouve hors de $\bar A$. Dans ce cas, il existe un $r>0$ tel que\footnote{Une autre manière de dire la même chose : si $\ell\notin\bar A$, alors $d(\ell,A)>0$.} $B(\ell,r)\cap A=\emptyset$. Si tous les éléments $x_n$ de la suite sont dans $A$, il n'y en a donc aucun tel que $d(x_n,\ell)=\| x_n-\ell \|<r$. Cela contredit la notion de convergence $x_n\to \ell$.
\end{proof}

Nous avons déjà mentionné dans l'exemple \ref{ParlerEncoredeF} que zéro était un point adhérent à l'ensemble $F=\{ (-1)^n/n\tq n\in\eN_0 \}$. Nous savons maintenant que $0$ étant la limite de la suite, il est automatiquement adhérent à l'ensemble des éléments de la suite.

\begin{corollary}		\label{CorAdhEstLim}
	Soit $a$ un point de l'adhérence d'une partie $A$ de $V$. Alors il existe une suite d'éléments dans $A$ qui converge vers $a$.
\end{corollary}

\begin{proof}
	Si $a\in A$, alors nous pouvons prendre la suite constante $x_n=a$. Si $a$ n'est pas dans $A$, alors $a$ est dans $\partial A$, et pour tout $n$, il existe un point de $A$ dans la boule $B(a,\frac{1}{ n })$. Si nous nommons $x_n$ ce point, la suite ainsi construite est une suite contenue dans $A$ et qui converge vers $a$ (ce dernier point est laissé à la sagacité du lecteur ou de la lectrice).
\end{proof}

En termes savants, ce corollaire signifie que la fermeture $\bar A$ est composé de $A$ plus de toutes les limites de toutes les suites contenues dans $A$.

%--------------------------------------------------------------------------------------------------------------------------- 
\subsection{Critère de Cauchy}
%---------------------------------------------------------------------------------------------------------------------------

\begin{definition}
    Une suite \( (a_k)\) dans l'espace vectoriel normé \( V\) est \defe{de Cauchy}{suite!de Cauchy} si pour tout \( \epsilon\), il existe \( N\) tel que si \( n,m\geq N\) alors \( \| a_n-a_m \|\leq \epsilon\).
\end{definition}

\begin{lemma}
    Une suite de Cauchy dans un espace vectoriel normé admettant une sous-suite convergente est elle-même convergente vers la même limite.
\end{lemma}

\begin{proof}
    Soit \( (a_n)\) une suite de Cauchy dans un espace vectoriel normé \( E\) et \( \ell\) la limite d'une sous-suite de \( (a_n)\). Soit \( \epsilon>0\) et \( N\in \eN\) tel que \( \| a_m-a_p \|<\epsilon\) dès que \( m,p\geq N\). Nous allons montrer que si \( k>N\) alors \( \| a_k-\ell \|<2\epsilon\). Pour cela nous considérons un \( n>N\) tel que \( \| a_n-\ell \|\leq \epsilon\) et nous calculons
    \begin{equation}
        \| a_k-\ell \|\leq \| a_k-a_n \|+\| a_n-\ell \|\leq 2\epsilon.
    \end{equation}
\end{proof}

\begin{theorem}[Critère de Cauchy]  \label{ThoHGyzAva}
    Une suite réelle est convergence si et seulement si elle est de Cauchy. En particulier \( \eR\) est un espace complet.
\end{theorem}
\index{critère!de Cauchy}
\index{complétude!de \( \eR\)}

\begin{proof}
    Soit \( (a_n)\) une suite de Cauchy dans \( \eR\).   
\end{proof}
%TODO : à faire, complétude de R.

\begin{definition}
    Nous disons que deux suites \( (u_n)\) et \( (v_n)\) sont \defe{équivalentes}{equivalence@équivalence!de suites} si il existe une fonction \( \alpha\colon \eN\to \eR\) telle que
    \begin{enumerate}
        \item
            pour tout \( n\) à partir d'un certain rang, \( u_n=v_n\alpha(n)\)
        \item
            \( \alpha(n)\to 1\).
    \end{enumerate}
\end{definition}

\begin{lemma}
    Si les suites \( (u_n)\) et \( (v_n)\) sont équivalentes et si \( (v_n)\) admet une limite \( l\) différente de \( 1\), alors les suites \( (\ln u_n)\) et \( (\ln v_n)\) sont équivalentes.
\end{lemma}

\begin{proof}
    En effet si \( u_n=v_n\alpha(n)\) alors
    \begin{equation}
        \ln(u_n)=\ln(v_n)+\ln\big( \alpha(n) \big)=\ln(v_n)\left( 1+\frac{ \ln\big( \alpha(n) \big) }{ \ln(v_n) } \right),
    \end{equation}
    et comme \( \alpha(n)\to 1\), la parenthèse tend vers \( 1\).
\end{proof}

\begin{lemma}[formule de Stirling\cite{MEHuVnb}]        \label{LemCEoBqrP}
    Nous avons l'équivalence de suites
    \begin{equation}
        n!\sim \left( \frac{ n }{ e } \right)^n\sqrt{2\pi n}.
    \end{equation}
\end{lemma}
\index{formule!Stirling}

% Supprimé le 23 juin 2014 parce que la définition juste en dessous suffit.
%\begin{definition}[Convergence absolue] \label{DefDOaqApF}
%    Soit \( a_n\) une suite dans un espace normé \( E\). Nous disons que la série \( \sum_{k=0}^{\infty}a_k\) converge \defe{absolument}{convergence!absolue} dans \( E\) si la série \( \sum_{k=0}^{\infty}\| a_k \|\) converge dans \( \eR\).
%\end{definition}

%+++++++++++++++++++++++++++++++++++++++++++++++++++++++++++++++++++++++++++++++++++++++++++++++++++++++++++++++++++++++++++ 
\section{Séries}
%+++++++++++++++++++++++++++++++++++++++++++++++++++++++++++++++++++++++++++++++++++++++++++++++++++++++++++++++++++++++++++

\begin{definition}\label{DefGFHAaOL}
Soit \( (a_k)\) une suite dans un espace vectoriel normé \( (V,\| . \| )\). La \defe{série}{série!dans un espace vectoriel normé} associée, notée \( \sum_{k=0}^{\infty}a_k\) est la limite des \defe{sommes partielles}{somme!partielle}, c'est à dire la limite de la suite
    \begin{equation}
        s_k=\sum_{i=0}^ka_i
    \end{equation}
    si elle existe dans \( V\).

    Si une telle limite existe nous disons que \( \sum_{k=0}^{\infty}a_k\) \defe{converge}{convergence!série} dans \( V\). Si la limite de la suite des sommes partielles n'existe pas nous disons que la série \defe{diverge}{série!divergence}.
\end{definition}
\begin{remark}
    Si la limite de la suite des sommes partielles n'existe pas dans \( V\), alors elle peut parfois exister dans des extensions de \( V\). Par exemple une série de rationnels convergeant vers \( \sqrt{2}\) dans \( \eR\) ne converge pas dans \( \eQ\). Autre exemple : avec une bonne topologie sur \( \bar \eR\), une série peut ne pas converger dans \( \eR\) mais converger vers \( \pm\infty\) dans \( \bar \eR\).
\end{remark}

Dans le cas des espaces de fonctions, nous avons une norme importante : la norme uniforme définie par \( \| f \|_{\infty}=\sup\{ f(x) \}\) où le supremum est pris sur l'ensemble de définition de \( f\).
\begin{definition}[Convergence absolue] \label{DefVFUIXwU}
    Nous disons que la série \( \sum_{n=0}^{\infty}a_n\) dans l'espace vectoriel normé \( V\) \defe{converge absolument}{convergence!absolue} si la série \( \sum_{n=0}^{\infty}\| a_n \|\) converge dans \( \eR\).
\end{definition}

Dans le cas de l'espace des fonctions muni de la norme uniforme, nous parlons de convergence normale.

\begin{definition}[Convergence normale] \label{DefVBrJUxo}
    Une série de fonctions \( \sum_{n\in \eN}u_n \) converge \defe{normalement}{convergence!normale} si la série de nombres \( \sum_n\| u_n \|_{\infty}\) converge. C'est à dire si la série converge absolument pour la norme \( \| f \|_{\infty}\).
\end{definition}
Cela n'est rien d'autre que la convergence absolue dans l'espace des fonctions muni de la norme uniforme.

La convergence normale est à ne pas confondre avec la convergence uniforme. La somme \( \sum_nf_n\) \defe{converge uniformément}{convergence!uniforme!série de fonctions} vers la fonction \( F\) si la suite des sommes partielles converge uniformément, c'est à dire si 
\begin{equation}        \label{EqLNCJooVCTiIw}
    \lim_{N\to \infty} \| \sum_{n=1}^Nf_n-F \|_{\infty}=0.
\end{equation}

\begin{lemma}   \label{LemNEoaaPt}
    Si une série converge absolument, alors elle converge.
\end{lemma}

\begin{proof}
    Ici comme dans tout ce chapitre nous considérons un espace vectoriel normé de dimension finie; il est donc complet et nous pouvons utiliser le critère de Cauchy \ref{ThoHGyzAva} pour caractériser les suites convergentes. Nous notons \( s_n\) la \( n\ieme\) somme partielle de \( \sum_k a_k\). Nous allons montrer que c'est une suite de Cauchy et qu'elle converge donc; en supposant que \( m>n\),
    \begin{equation}
        \| s_n-s_m \|=\| \sum_{k=n+1}^m a_k \|\leq \sum_{k=n+1}^m\| a_k \|.
    \end{equation}
    Par hypothèse, la série converge absolument, ce qui signifie que la suite des sommes partielles de la série \( \sum_k\| a_k \|\) est de Cauchy et qu'il existe donc un \( N\) tel que si \( m,n>N\) alors \( \sum_{k=n+1}^m\| a_k \|\leq \epsilon\). Cela prouve que \( (s_n)\) est de Cauchy et donc que \( \sum_ka_k\) converge.
\end{proof}

\begin{proposition} \label{PropAKCusNM}
    Une série convergeant absolument dans un espace de Banach\footnote{Un espace vectoriel normé complet. Typiquement \( \eR\).} y converge au sens usuel.
\end{proposition}

\begin{proof}
    Soit \( (a_k)\) une suite dans un espace vectoriel normé complet dont la série converge absolument. Nous allons montrer que la suite des sommes partielles est de Cauchy. Cela suffira à montrer sa convergence par hypothèse de complétude.

    Nous avons
    \begin{equation}
        \| s_p-s_l \|=\| \sum_{k=l+1}^{p}a_k\|  \leq\sum_{k=l+1}^p\| a_k \|=\bar s_p-\bar s_l
    \end{equation}
    où \( \bar s_n=\sum_{k=0}^n \| a_k \|\) est la suite des sommes partielles de la série des normes (qui converge). Vu que la suite \( (\bar s_n)\) converge dans \( \eR\), elle y es de Cauchy par le théorème \ref{ThoHGyzAva}. Donc il existe un \( N\) tel que \( p,l>N\) implique
    \begin{equation}
        \| s_p-s_l \|=\bar s_p-\bar s_l\leq \epsilon.
    \end{equation}
    Cela signifie que \( (s_n)\) est une suite de Cauchy et donc convergente.
\end{proof}

\begin{remark}
    Nous savons que sur les espaces vectoriels de dimension finie toutes les normes sont équivalentes (théorème \ref{DefEquivNorm}). La notion de convergence de série ne dépend alors pas du choix de la norme. Il n'en est pas de même sur les espaces de dimension infinie. Une série peut converger pour une norme mais pas pour une autre.
\end{remark}
Lorsque nous verrons la convergence de séries, nous verrons que la convergence normale est la convergence absolue pour la norme uniforme.


\begin{proposition}\label{propnseries_propdebase}
Les principales propriétés de la somme définie par la limite \eqref{DefGFHAaOL} sont
  \begin{enumerate}
  \item Si une série converge absolument, alors elle converge simplement.
  \item Si la série est à termes positifs --c'est-à-dire pour tout indice $k$, $a_k \in \eR$ et $a_k \geq 0$-- il n'y a aucune différence entre convergence absolue et convergence simple.
  \item\label{point3-seriepropdebase} Si une série converge, son terme général doit tendre vers $0$.
\item 
Si la série converge alors la somme est associative
\item
Si la série converge absolument, alors la somme est commutative.
  \end{enumerate}
\end{proposition}

%TODO : la preuve des autres points
\begin{proof}
    
    \begin{enumerate}
        \item
            
  \item
  \item 
  \item
\item 
Associativité. Supposons que \( \sum_ka_k\) et \( \sum_kb_k\) convergent tous deux. Alors nous avons pour tout \( N\) :
\begin{equation}
    \sum_{k=0}^N(a_k+b_k)=\sum_{k=0}^Na_k+\sum_{k=0}^Nb_k.
\end{equation}
Mais si deux limites existent alors la somme commute avec la limite. C'est le cas pour la limite \( N\to \infty\), donc
\begin{equation}
    \lim_{N\to \infty} \sum_{k=1}^{\infty}(a_k+b_k)=\lim_{N\to \infty} \sum_{k=0}^{\infty}a_k+\lim_{N\to \infty} \sum_{k=0}^{\infty}b_k.
\end{equation}
\item
    \end{enumerate}
\end{proof}

%+++++++++++++++++++++++++++++++++++++++++++++++++++++++++++++++++++++++++++++++++++++++++++++++++++++++++++++++++++++++++++
\section{Série réelle}
%+++++++++++++++++++++++++++++++++++++++++++++++++++++++++++++++++++++++++++++++++++++++++++++++++++++++++++++++++++++++++++
\label{secseries}

La notion de série formalise le concept de somme infinie. L'absence de certaines propriétés de ces objets (problèmes de commutativité et même d'associativité) incitent à la prudence et montrent à quel point une définition précise est importante. 

\subsection{Rappels et définitions}

Nous avons déjà vu de nombreuses choses sur les séries.
\begin{enumerate}
    \item
        La définition de la somme d'une infinité de termes est donnée par la définition \ref{DefGFHAaOL}.
    \item
        La définition de la convergence absolue est la définition \ref{DefVFUIXwU}.
    \item
        Les propriétés générales de la proposition \ref{propnseries_propdebase}.
\end{enumerate}

\begin{remark}
    Vue comme somme infinie, l'associativité et la commutativité dans une série sont perdues. Néanmoins, il subsiste que
  \begin{enumerate}
  \item 
      si la série converge, on peut regrouper ses termes sans modifier la convergence ni la somme (associativité),
  \item
      si la série converge absolument, on peut modifier l'ordre des termes sans modifier la convergence ni la somme (commutativité).
  \end{enumerate}
\end{remark}
%TODO : il me semble qu'il y a une preuve de ça quelque part au début de Hilbert. Mettre une référence.
%TODO : donner des exemples de cette perte.

\subsection{Critères de convergence absolue}

Étant donné le terme général d'une série, il est souvent --dans les cas qui nous intéressent-- difficile de déterminer la somme de la série. L'exemple de la série géométrique est particulier, puisqu'on connait une formule pour chaque somme partielle, mais pour l'exemple des séries de Riemann il n'y a aucune formule simple pour un $\alpha$ général. D'où l'intérêt d'avoir des critères de convergence ne nécessitant aucune connaissance de l'éventuelle limite de la série.

\begin{lemma}[Critère de comparaison]   \label{LemgHWyfG}
Soient $\sum_i a_i$ et $\sum_j
b_j$ deux séries à termes positifs vérifiant
\begin{equation*}
  0 \leq a_i \leq b_i
\end{equation*}
alors
\begin{enumerate}
\item si $\sum_i a_i$ diverge, alors $\sum_j b_j$ diverge,
\item si $\sum_j b_j$ converge, alors $\sum_i a_i$ converge
  (absolument).
  \end{enumerate}
\end{lemma}


\begin{proposition}[Critère d'équivalence\cite{TrenchRealAnalisys}]
 Soient $\sum_i a_i$ et $\sum_j b_j$ deux séries à termes positifs. Supposons l'existence de la limite (éventuellement infinie) suivante
\begin{equation}
  \limite i \infty \frac{a_i}{b_i} = \alpha \in \eR \text{ ou $\alpha =
    \infty$.}
\end{equation}
Dans ce cas, nous avons
\begin{enumerate}
\item si $\alpha \neq 0$ et $\alpha\neq \infty$, alors
  \begin{equation}
    \sum_i a_i \text{~converge} \ssi \sum_j b_j\text{~converge,}
  \end{equation}
\item si $\alpha = 0$ et $\sum_j b_j$ converge, alors $\sum_i a_i$
  converge (absolument),
\item si $\alpha = +\infty$ et $\sum_j b_j$ diverge, alors $\sum_i
  a_i$ diverge.
\end{enumerate}
\end{proposition}

\begin{proof}
\begin{enumerate}
    \item
        Le fait que la suite $a_n/b_n$ converge vers $\alpha$ signifie que tant sa limite supérieure que sa limite inférieure convergent vers $\alpha$. En particulier la suite $\frac{ a_n }{ b_n }$ est bornée vers le haut et vers le bas. À partir d'un certain rang $N$, il existe $M$ tel que 
        \begin{equation}
            \frac{ a_n }{ b_n }<M
        \end{equation}
        et il existe $m$ tel que
        \begin{equation}
            \frac{ a_n }{ b_n }>m.
        \end{equation}
        Nous avons donc $a_n<Mb_n$ et $a_n>mb_n$. La série de $(a_n)$ converge donc si et seulement si la série de $(b_n)$ converge.
    \item
        Si $\alpha=0$, cela signifie que pour tout $\epsilon$, il existe un rang tel que $\frac{ a_n }{ b_n }<\epsilon$, et donc tel que $a_n<\epsilon b_k$. La suite de $(a_i)$ converge donc dès que la suite de $(b_i)$ converge.
    \item
        Pour tout $M$, il existe un rang dans la suite à partir duquel on a $\frac{ a_i }{ b_i }>M$, et donc $a_k>Mb_k$. Si la série de $(b_k)$ diverge, la série de $(a_k)$ doit également diverger.
\end{enumerate}
\end{proof}


\begin{proposition}[Critère du quotient\cite{KeislerElemCalculus}]     \label{PropOXKUooQmAaJX}
    Soit $\sum_i a_i$ une série. Supposons l'existence de la limite (éventuellement infinie) suivante
    \begin{equation}
      \limite i \infty \abs{\frac{a_{i+1}}{a_i}} = L\in \eR \text{ ou $L =
        \infty$.}
    \end{equation}
    Alors
    \begin{enumerate}
    \item si $L < 1$, la série converge absolument,
    \item si $L > 1$, la série diverge,
    \item si $L = 1$ le critère échoue : il existe des exemple de convergence et des exemples de divergence.
    \end{enumerate}
\end{proposition}

\begin{proof}
\begin{enumerate}
    \item
        Soit $b$ tel que $L<b<1$. À partir d'un certain rang $K$, on a $\left| \frac{ a_{i+1} }{ a_i } \right| <b$. En particulier,
        \begin{equation}
            | a_{K+1} |<b| a_K |,
        \end{equation}
        et pour $a_{K+2}$ nous avons
        \begin{equation}
            | a_{K+2} |<b| a_{K+1} |<b^2| a_K |.
        \end{equation}
        Au final,
        \begin{equation}
            | a_{K+n} |<b^n| a_K |.
        \end{equation}
        Étant donné que la série $\sum_{n\geq K}b^n$ converge (parce que $b<1$), la queue de suite $\sum_{i\geq K}a_i$ converge, et par conséquent la suite au complet converge.
    \item
        Si $L>1$, on a
        \begin{equation}
            | a_K |<| a_{K+1} |<| a_{K+2} |<\ldots
        \end{equation}
        Il est donc impossible que la suite $(a_i)$ converge vers zéro. La série ne peut donc pas converger.
    \item
        Par exemple la suite harmonique $a_n=\frac{1}{ n }$ vérifie $L=1$, mais la série ne converge pas. Par contre, la suite $a_n=\frac{ 1 }{ n^2 }$ vérifie aussi le critère avec $L=1$ tandis que la série $\sum_n\frac{1}{ n^2 }$ converge.
\end{enumerate}
\end{proof}


\begin{proposition}[Critère de la racine\cite{TrenchRealAnalisys}]
    Soit $\sum_i a_i$ une série, et considérons
    \begin{equation*}
      \limsup_{i \rightarrow \infty} \sqrt[i]{\abs{a_i}} = L \in \eR
      \text{ ou $L =
        \infty$.}
    \end{equation*}
    Alors
    \begin{enumerate}
    \item si $L < 1$, la série converge absolument,
    \item si $L> 1$, la série diverge,
    \item si $L = 1$ le critère échoue.
    \end{enumerate}
\end{proposition}

\begin{proof}
    \begin{enumerate}
        \item
            Si $L<1$, il existe un $r\in \mathopen] 0 , 1 \mathclose[$ tel que $| a_n |^{1/n}<r$ pour les grands $n$. Dans ce cas, $| a_n |<r^{n}$, et la série converge absolument parce que la série $\sum_nr^n$ converge du fait que $r<1$.
        \item
            Si $L>1$, il existe un $r>1$ tel que $| a_n |^{1/n}>r>1$. Cela fait que $| a_n |$ prend des valeurs plus grandes que $n$ pour une infinité de termes. Le terme général $a_n$ ne peut donc pas être une suite convergente. Par conséquent la suite diverge au sens où elle ne converge pas.

    \end{enumerate}
\end{proof}

%---------------------------------------------------------------------------------------------------------------------------
\subsection{Critères de convergence simple}
%---------------------------------------------------------------------------------------------------------------------------

Les critères de comparaison, d'équivalence, du quotient et de la racine sont des critères de convergence absolue. Pour conclure à une convergence simple qui n'est pas une convergence absolue, le critère d'Abel sera notre outil principal.  

\subsubsection{Critère d'Abel}

\begin{proposition}[Critère d'Abel]
    Soit la série $\sum_i c_iz_i$ avec
    \begin{enumerate}
        \item $(c_i)$ est une suite réelle décroissante qui tend vers zéro,
        \item $(z_i)$ est une suite dans $\eC$ dont la suite des sommes partielles est bornée dans $\eC$, c'est à dire qu'il existe un $M>0$ tel que pour tout $n$,
        \begin{equation}
            \left| \sum_{i=1}^nz_i \right| \leq M.
        \end{equation}
        Alors la série $\sum_ic_iz_i$ est convergente.
    \end{enumerate}
\end{proposition}
Remarquons que ce critère ne donne pas de convergence absolue.

\begin{corollary}[Critère des séries alternées]\index{critère!série alternée}       \label{CoreMjIfw}
    Si \( (a_n)\) est une suite décroissante à limite nulle, alors la série
  \begin{equation}
    \sum_{n=0}^\infty {(-1)}^n a_n
  \end{equation}
  converge simplement.
\end{corollary}

%--------------------------------------------------------------------------------------------------------------------------- 
\subsection{Exemples}
%---------------------------------------------------------------------------------------------------------------------------

\begin{example}[Série harmonique]
    La \defe{série harmonique}{série!harmonique} est
    \begin{equation}
        \sum_{i=1}^\infty \frac1i
    \end{equation}
    et diverge (possède une limite $+\infty$).
\end{example}

\begin{example}[Série géométrique] \label{ExZMhWtJS}
    La \defe{série géométrique}{série!géométrique} de raison $q \in \eC$ est
    \begin{equation}    \label{EqZQTGooIWEFxL}
        \sum_{i=0}^\infty q^i.
    \end{equation}
    Étudions la somme partielle \( S_N=1+q+q^2+\ldots +q^{n}\). Nous avons évidemment $S_N-zS_N=1-q^{N+1}$ et donc
    \begin{equation}    \label{EqASYTiCK}
        S_N=\sum_{n=0}^Nq^n=\frac{ 1-q^{N+1} }{ 1-q }.
    \end{equation}
    La limite \( \lim_{N\to \infty} S_N\) existe si et seulement si \( | q |\leq 1\) et dans ce cas nous avons
    \begin{equation}    \label{EqRGkBhrX}
        \sum_{n=0}^{\infty}q^n=\frac{ 1 }{ 1-q }.
    \end{equation}
    La convergence est absolue.
\end{example}

\begin{example}[Série de Riemann]
    Pour $\alpha \in \eR$, la \defe{série de Riemann}{série!Riemann}
    \begin{equation}        \label{EqSerRiem}
        \sum_{i=1}^\infty \frac1{i^\alpha}
    \end{equation}
    converge (absolument, puisque réelle et positive) si et seulement si $\alpha > 1$, et diverge sinon.
\end{example}

\begin{example}[Série exponentielle] \label{ExIJMHooOEUKfj}
    La série exponentielle est la série (pour \( t\in \eR\))
    \begin{equation}
        \exp(t)=\sum_{k=0}^{\infty}\frac{ t^k }{ k! }.
    \end{equation}
    Nous montrons qu'elle converge pour tout \( t\in \eR\). Si \( a_k=t^k/k!\) alors \( \frac{ a_{k+1} }{ a_k }=\frac{ t }{ k }\) dont la limite \( k\to \infty\) est zéro (quel que soit \( t\)). En vertu du critère du quotient \ref{PropOXKUooQmAaJX} la série exponentielle converge (absolument) pour tout \( t\in \eR\).

    Toutes les propriétés de la fonction de \( t\) ainsi définie sont dans le théorème \ref{ThoRWOZooYJOGgR}.
\end{example}
\index{exponentielle!convergence}

\begin{example}[Série arithméticogémétrique\cite{QXuqdoo}]
    La \defe{suite arithméticogémétrique}{suite!arithméticogéométrique} est une suite de la forme \( u_{n+1}=au_n+b\) avec \( a\) et \( b\) non nuls. Si elle possède une limite, cette dernière doit résoudre \( l=al+n\), et donc être égale à 
    \begin{equation}
        r=\frac{ b }{ 1-a }.
    \end{equation}
    Il n'est pas très compliqué de trouver le terme général de la suite en fonction de \( a\) et de \( b\). Il suffit de considérer la suite \( v_n=u_n-r\), et de remarquer que cette suite est géométrique :
    \begin{equation}
        v_{n+1}=av_n.
    \end{equation}
    Par conséquent \( v_n=a^nv_0\), ce qui donne pour la suite \( (u_n)\) la formule
    \begin{equation}
        u_n=a^n(u_0-r)+r.
    \end{equation}
\end{example}

%---------------------------------------------------------------------------------------------------------------------------
\subsection{Moyenne de Cesaro}
%---------------------------------------------------------------------------------------------------------------------------

Si \( (a_n)_{n\in \eN} \) est une suite dans \( \eR\) ou \( \eC\), alors sa \defe{moyenne de Cesaro}{moyenne!de Cesaro}\index{Cesaro!moyenn} est la limite (si elle existe) de la suite
\begin{equation}
    c_n=\frac{1}{ n }\sum_{k=1}^na_k.
\end{equation}
En un mot, c'est la limite des moyennes partielles.

\begin{lemma}       \label{LemyGjMqM}
    Si la suite \( (a_n)\) converge vers la limite \( \ell\) alors la suite admet une moyenne de Cesaro qui vaudra \( \ell\).
\end{lemma}

\begin{proof}
    Soit \( \epsilon>0\) et \( N\in \eN\) tel que \( | a_n-\ell |<\epsilon\) pour tout \( n>N\). En remarquant que
    \begin{equation}
        \frac{1}{ n }\sum_{k=1}^nk-\ell=\frac{1}{ n }\sum_{k=1}^n(a_k-\ell),
    \end{equation}
    nous avons
    \begin{subequations}
        \begin{align}
            | \frac{1}{ n }\sum_{k=1}^na_k-\ell |&\leq| \frac{1}{ n }\sum_{k=1}^N| a_k-\ell | |+\big| \frac{1}{ n }\sum_{k=N+1}^n\underbrace{| a_k-\ell |}_{\leq \epsilon} \big|\\
            &\leq \epsilon+\frac{ n-N-1 }{ n }\epsilon\\
            &\leq 2\epsilon.
        \end{align}
    \end{subequations}
    Dans ce calcul nous avons redéfinit \( N\) de telle sorte que le premier terme soit inférieur à \( \epsilon\).
\end{proof}

%--------------------------------------------------------------------------------------------------------------------------- 
\subsection{Écriture décimale d'un nombre}
%---------------------------------------------------------------------------------------------------------------------------

Soit \( b\geq 2\) un entier qui sera la base dans laquelle nous allons écrire les nombres. Nous considérons l'ensemble \( \eD_b\)\nomenclature[Y]{\( \eD_b\)}{l'ensemble de écritures décimales en base \( b\)} des suites dans \( \{ 0,1,\ldots, b-1 \}\) qui n'ont pas une queue de suite uniquement formée de \( b-1\). Autrement dit une suite \( (c_n)\) est dans \( \eD_b\) lorsque pour tout \( N\), il existe \( k>N\) tel que \( c_k\neq b-1\). Associé à cet ensemble nous considérons la fonction
\begin{equation}    \label{EqXXXooOTsCK}
    \begin{aligned}
        \varphi_b\colon \eD_b&\to \mathopen[ 0 , 1 [ \\
            c&\mapsto \sum_{n=1}^{\infty}\frac{ c_n }{ b^n }. 
    \end{aligned}
\end{equation}

\begin{lemma}
    La fonction \( \varphi_b\) est bien définie au sens où elle converge et prend ses valeurs dans \( \mathopen[ 0 , 1 [\).
\end{lemma}
    
\begin{proof}
    Tout se base sur la somme de la série géométrique \eqref{EqRGkBhrX} sous la forme
    \begin{equation}    \label{EqWZGooXJgwl}
        \sum_{k=0}^{\infty}\frac{1}{ b^k }=\frac{ b }{ b-1 }.
    \end{equation}
    La somme \eqref{EqXXXooOTsCK} est donc majorée par \( \sum_n\frac{ b-1 }{ b^n }\) qui converge.

    Pour prouver que l'image de \( \varphi_b\) est bien \( \mathopen[ 0 , 1 [\), nous savons qu'au moins un des \( c_n\) (en fait une infinité) est plus petit que \( b-1\), donc nous avons la majoration stricte\footnote{Notez que la somme \eqref{EqXXXooOTsCK} commence à un tandis que la série géométrique \eqref{EqWZGooXJgwl} commence à zéro.}
        \begin{equation}
            \varphi_b(c)<\sum_{n=1}^{\infty}\frac{ b-1 }{ b^n }=(b-1)\left( \sum_{n=1}^{\infty}\frac{1}{ b^n }-1 \right)=1
        \end{equation}
\end{proof}

Le fait d'introduire l'ensemble \( \eD\) au lieu de l'ensemble de toutes les suites est justifié par la proposition suivante. Elle explique pourquoi un nombre possède au maximum deux écritures décimales distinctes et que ces deux sont obligatoirement de la forme, par exemple en base \( 10\) :
\begin{equation}
    0.34599999999\ldots=0.34600000\ldots
\end{equation}
mais qu'un nombre commençant par \( 0.347\) ne peut pas être égal. C'est pour cela que dans la définition de \( \eD_b\) nous avons exclu les suites qui terminent par tout des \( b-1\).
\begin{proposition} \label{PropSAOoofRlQR}
    Soit la fonction
    \begin{equation}
        \begin{aligned}
            \varphi\colon \{ 0,\ldots, b-1 \}^{\eN}&\to \mathopen[ 0 , 1 [ \\
                x&\mapsto \sum_{n=1}^{\infty}\frac{ x_n }{ b^n }. 
        \end{aligned}
    \end{equation}
    Si \( \varphi(x)=\varphi(y)\) et si \( n_0\) est le plus petit entier tel que \( x_{n_0}\neq y_{n_0}\) alors soit
    \begin{equation}
        x_{n_0}-y_{n_0}=1
    \end{equation}
    et \( x_n=0\), \( y_n=b-1\) pour tout \( n>n_0\), soit le contraire : \( y_{n_0}-x_{n_0}=1\) avec \( y_n=0\) et \( x_n=b-1\) pour tout \( n>n_0\).
\end{proposition}

\begin{proof}
    Nous nous basons sur la formule (facilement dérivable depuis \eqref{EqWZGooXJgwl}) suivante :
    \begin{equation}
        \sum_{k=n_0+1}^{\infty}\frac{1}{ b^k }=\frac{1}{ b^{n_0+1} }\frac{ b }{ b-1 }.
    \end{equation}
    Nous avons 
    \begin{equation}
        0=\varphi(x)-\varphi(y)=\frac{ x_{n_0}-y_{n_0} }{ b^{n_0} }+\sum_{n=n_0+1}^{\infty}\frac{ x_n-y_n }{ b^n }\geq \frac{ x_{n_0}-y_{n_0} }{ b^{n_0} }-\sum_{n=n_0+1}^{\infty}\frac{ b-1 }{ b^n }=\frac{ x_{n_0}-y_{n_0}-1 }{ b^{n_0} }.
    \end{equation}
    Le dernier terme étant manifestement positif\footnote{C'est ici qu'intervient la subdivision entre le cas \( x_{n_0}-y_{n_0}=1\) ou le contraire. En effet si «ce dernier terme était manifestement \emph{négatif}», il aurait fallu majorer avec de \( 1-b\) au lieu de \( 1-b\).}, il est nul et nous avons \( x_{n_0}-y_{n_0}=1\).

    Nous avons donc maintenant
    \begin{equation}    \label{EqHWQoottPnb}
        0=\varphi(x)-\varphi(y)=\frac{1}{ b^{n_0} }+\sum_{n=n_0+1}^{\infty}\frac{ x_n-y_n }{ b^n }.
    \end{equation}
    Nous majorons la dernière somme de la façon suivante, en supposant que \( | x_n-y_n |\neq b-1\) pour un certain \( n>n_0\) :
    \begin{equation}
        \left| \sum_{n=n_0+1}^{\infty}\frac{ x_n-y_n }{ b^n } \right| \leq\sum_{n=n_0+1}^{\infty}\frac{ | x_n-y_n | }{ b^n }<\sum_{n=n_0+1}^{\infty}\frac{ b-1 }{ b^n }=\frac{1}{ b^{n_0} }.
    \end{equation}
    Étant donné cette inégalité stricte, l'équation \eqref{EqHWQoottPnb} ne peut pas être correcte (valoir zéro). Nous avons donc \( | x_n-b_n |=b-1\) pour tout \( n>n_0\). Donc pour chaque \( n>n_0\) nous avons soit \( x_n=0\) et \( y_n=b-1\), soit \( a_n=b-1\) et \( b_n=0\). Pour conclure il faut encore prouver que le choix doit être le même pour tout \( n\).

    Nous nous mettons dans le cas \( x_{n_0}-y_{n_0}=1\); dans ce cas nous avons bien l'égalité \eqref{EqHWQoottPnb} sans petites nuances de signes. Nous écrivons
    \begin{equation}
        \sum_{n=n_0+1}^{\infty}\frac{ x_n-y_n }{ b^n }=(b-1)\sum_{n=n_0+1}^{\infty}\frac{ (-1)^{s_n} }{ b^n }
    \end{equation}
    où \( s_n\) est pair ou impair suivant que \( x_n=0\), \( y_n=b-1\) ou le contraire. Si un des \( (-1)^{s_n}\) est pas \( -1\) alors nous avons l'inégalité stricte
    \begin{equation}
        (b-1)\sum_{n=n_0+1}^{\infty}\frac{ (-1)^{s_n} }{ b^n }>(b-1)\sum_{n=n_0+1}^{\infty}\frac{-1}{ b^n }=-\frac{1}{ b^{n_0} }.
    \end{equation}
    Dans ce cas il est impossible d'avoir \( \varphi(x)-\varphi(y)=0\). Nous en concluons que \( (-1)^{s_n}\) est toujours \( -1\), c'est à dire \( x_n-y_n=1-b\), ce qui laisse comme seule possibilité \( x_n=0\) et \( y_n=b-1\).
\end{proof}

\begin{theorem} \label{ThoRXBootpUpd}
    L'application \( \varphi_b\colon \eD_b\to \mathopen[ 0 , 1 [\) est bijective.
\end{theorem}

\begin{proof}
    En ce qui concerne l'injection, nous savons de la proposition \ref{PropSAOoofRlQR} que si \( \varphi_b(x)=\varphi_b(y)\) pour \( x,y\in\{ 0,\ldots, b-1 \}^{\eN}\), alors soit \( x\) soit \( y\) a une queue de suite composée uniquement de \( b-1\), ce qui est exclu dans \( \eD_b\). Nous en déduisons que \( \varphi_b\) est bien injective en prenant \( \eD_b\) comme ensemble départ.

    La partie lourde est la surjectivité. Nous prenons \( x\in \mathopen[ 0 , 1 [\) et nous allons construire par récurrence une suite \( a\in \eD_b\) telle que \( \varphi_b(a)=x\). Si il existe \( a_1\in\{ 0,\ldots, b-1 \}\) tel que \( x=a_1/b\) alors nous prenons la suite \( (a_1,0,\ldots, )\) et nous avons évidemment \( \varphi(a)=x\). Sinon il existe \( a_1\in\{ 0,\ldots, b-1 \}\) tel que
        \begin{equation}
            \frac{ a_1 }{ b }<x<\frac{ a_1+1 }{ b }
        \end{equation}
        parce que les autres possibilités pour \( x\) sont dans l'ensemble \( \mathopen[ 0 , 1 \mathclose[\setminus\{ \frac{ k }{ b } \}_{k=0,\ldots, b-1}\) que nous subdivisons en
        \begin{equation}
        \mathopen] 0 , \frac{1}{ b } \mathclose[\cup\mathopen] \frac{1}{ b } , \frac{ 2 }{ b } \mathclose[\cup\ldots\cup\mathopen] \frac{ b-1 }{ b } , 1 \mathclose[.
        \end{equation}
        Pour la récurrence nous supposons avoir trouvé \( a_1,\ldots, a_n\) tels que
        \begin{equation}
            \sum_{k=1}^n\frac{ a_k }{ b^k }< x<\sum_{k=1}^{n-1}\frac{ a_k }{ b^k }+\frac{ a_n+1 }{ b^n }.
        \end{equation}
    Encore une fois si il existe \( a_{n+1}\in\{ 0,\ldots, b-1 \}\) tel que \( \sum_{k=1}^{n+1}\frac{ a_k }{ b^k }=x\) alors nous prenons ce \( a_{n+1}\) et nous complétons la suite avec des zéros pour avoir \( \varphi(a)=x\). Sinon 
%nous subdivisions l'intervalle \( \mathopen]  \frac{ a_n }{ b^n }, \frac{ a_n }{ b^n }+\frac{ a_n+1 }{ b^n } \mathclose[\) (auquel nous retranchons les \( b\) nombres déjà traités) en
 %       \begin{equation}
 %       \mathopen] \frac{ a_n }{ b^n } , \frac{ a_n }{ b^n }+\frac{1}{ b^{n+1} } \mathclose[ \cup \mathopen] \frac{ a_n }{ b^n }+\frac{1}{ b^{n+1} } , \frac{ a_n }{ b^n }+\frac{2}{ b^{n+1} } \mathclose[\cup\ldots\cup\mathopen] \frac{ a_n }{ b^n }+\frac{ b-1 }{ b^{n+1} } , \frac{ a_n }{ b^n }+\frac{ 1 }{ b^n } \mathclose[.
 %       \end{equation}
        , pour simplifier les notations nous notons \( x'=x-\sum_{k=1}^{n}\frac{ a_k }{ b^k }\) et nous avons
        \begin{equation}
            0<x'<\frac{ a_n+1 }{ b^n }.
        \end{equation}
        Le nombre \( x'\) est forcément dans un des intervalles
        \begin{equation}
                \mathopen] \frac{ s }{ b^{n+1} } , \frac{ s+1 }{ b^{n+1} } \mathclose[
        \end{equation}
        avec \( s\in\{ 0,\ldots, b-1 \}\). Nous prenons le \( s\) correspondant à \( x'\) comme \( a_{n+1}\). Dans ce cas nous avons
        \begin{equation}
            \sum_{k=1}^{n+1}\frac{ a_k }{ b^k }< x<\sum_{k=1}^{n+1}\frac{ a_k }{ b^k }+\frac{1}{ b^{n+1} }.
        \end{equation}
        Note : les deux inégalités sont strictes. La première parce que si il y avait égalité, nous nous serions déjà arrêté en complétant avec des zéros. La seconde parce que 
        \begin{equation}
            \sum_{k=n+2}^{\infty}\frac{ a_k }{ b^k }\leq \sum_{k=n+2}^{\infty}\frac{ b-1 }{ b^k }=\frac{1}{ b^{n+1} }
        \end{equation}
        où l'égalité n'est possible que si \( a_k=b-1\) pour tout \( k\geq n+2\). Dans ce cas nous aurions eu
        \begin{equation}
            x=\sum_{k=1}^{n}\frac{ a_k }{ b^k }+\frac{ a_{n+1}+1 }{ b^{n+1} }
        \end{equation}
        et nous aurions choisit le nombre \( a_{n+1}\) autrement et complété la suite par des zéros à partir de là. Notons que cela prouve au passage que la suite que nous sommes en train de construire est bien dans \( \eD_b\) parce qu'elle ne contiendra pas de queue de suite composée de \( b-1\).

        Ceci termine la construction par récurrence de la suite \( a\in \eD_b\). Par construction nous avons pour tout \( N\geq 1\),
        \begin{equation}
            \sum_{k=1}^N\frac{ a_k }{ b^k }\leq x\leq \sum_{k=1}^N\frac{ a_k }{ b^k }+\frac{1}{ b^{N+1} }, 
        \end{equation}
        autrement dit : \( \varphi_b(a_1,\ldots, a_N)\in B(x,\frac{1}{ b^{N+1} })\). Nous avons donc bien convergence
        \begin{equation}
            \lim_{N\to \infty} \varphi_b(a_1,\ldots, a_N)=x
        \end{equation}
        et l'application \( \varphi_b\) est surjective.
\end{proof}

L'application \( \varphi_b^{-1}\colon \mathopen[ 0 , 1 [\to \eD_b\) est la \defe{décomposition décimale}{décimale!décomposition} en base \( b\) des nombres de \( \mathopen[ 0 , 1 [\).

Tout cela nous permet de montrer entre autres que \( \eR\) n'est pas dénombrable. Vu qu'il y a une bijection entre \( \mathopen[ 0 , 1 [\) et \( \eD_b\), il suffit de prouver que \( \eD_b\) est non dénombrable. De plus il suffit de démontrer que \( \eD_b\) est non dénombrable pour un entier \( b\geq 2\) donné.

\begin{proposition}[\cite{KZIoofzFLV}]  \label{PropNNHooYTVFw} 
    Il n'existe pas de surjection \( \eN\to \eD_b\). Autrement dit \( \eD_b\) est non dénombrable.
\end{proposition}

\begin{proof}
    Nous prenons \( b\neq 2\) pour des raisons qui seront claires plus tard. Soit \( f\colon \eN\to \eD_b\). Pour \( i\in \eN\) nous notons 
    \begin{equation}
        f(n)=(c_i^{(n)})_{i\geq 1},
    \end{equation}
    et nous définissons la suite
    \begin{equation}
        c_k=\begin{cases}
            0    &   \text{si \( c_k^{(k)}\neq 0\)}\\
            1    &    \text{si \( c_k^{(k)}=0\)}.
        \end{cases}
    \end{equation}
    Cela est une suite dans \( \eD_b\) parce que \( b\neq 2\) et que la suite ne contient que des \( 0\) et des \( 1\). Mais nous n'avons \( f(n)=c\) pour aucun \( n\in \eN\) parce que nous avons \( c_n\neq f(n)_n\).

    Si \( b=2\) alors nous savons que \( \eD_2\sim\mathopen[ 0 , 1 [\sim \eD_3\). Donc \( \eD_2\sim \eD_3\) et \( \eD_2\) ne peut pas plus être mis en bijection avec \( \eN\) que \( \eD_3\).
\end{proof}
\begin{remark}
    La preuve ne fonctionne pas en base \( b=2\) parce que rien n'empêche d'avoir une queue de \( 1\). Il y a alors toutefois moyen de se débrouiller en construisant la suite \( c\) de façon plus subtile. Si \( b=2\) et \( n\in \eN\) alors \( f(n)\) est une suite de \( 0\) et \( 1\) contenant une infinité de \( 0\) (parce qu'il n'y a pas de queue de suite ne contenant que des \( 1\)). Nous construisons alors \( c\) de la façon suivante : d'abord nous recopions \( f(0)\) jusqu'à son \emph{deuxième} zéro que nous changeons en \( 1\); nommons \( n_0\) le rang de ce deuxième zéro. Ensuite nous recopions les éléments de \( f(1) \) à partir du rang \( n_0+1\) jusqu'au second zéro que nous changeons en \( 1\), etc.

    Le fait de prendre le deuxième zéro nous garanti que la suite \( c\) n'aura pas de queue de suite ne contenant que des \( 1\).

    Notons que cette construction s'adapte à tout \( b\); il suffit de prendre le second terme qui n'est pas \( b-1\) et le remplacer par \( b-1\).
\end{remark}

\begin{corollary}
    L'ensemble \( \mathopen[ 0 , 1 [\) n'est pas dénombrable.
\end{corollary}

\begin{proof}
    L'ensemble \( \mathopen[ 0 , 1 [\) est en bijection avec \( \eD_b\) que nous venons de prouver n'être pas dénombrable.
\end{proof}

%+++++++++++++++++++++++++++++++++++++++++++++++++++++++++++++++++++++++++++++++++++++++++++++++++++++++++++++++++++++++++++ 
\section{Exponentielle de matrice}
%+++++++++++++++++++++++++++++++++++++++++++++++++++++++++++++++++++++++++++++++++++++++++++++++++++++++++++++++++++++++++++
%\label{subsecAOnIwQM}
\label{secAOnIwQM}

\begin{enumerate}
    \item
        En ce qui concerne la continuité, nous aurons évidemment besoin de théorie à propos de l'inversion de limites et de sommes. Nous en parlerons donc en \ref{subsecXNcaQfZ}.
    \item 
        Les séries entières de matrices seront traitées plus en détail autour de la proposition \ref{PropFIPooSSmJDQ}.
\end{enumerate}

\begin{proposition}     \label{PropPEDSooAvSXmY}
    Soit \( V\) un espace vectoriel de dimension finie et \( A\in\End(V)\). La série
    \begin{equation}
        \exp(A)=\mtu+A+\frac{ A^2 }{ 2 }+\frac{ A^3 }{ 3 }+\ldots =\sum_{k=1}^{\infty}\frac{ A^k }{ k! }.
    \end{equation}
    converge normalement dans \( \big( \End(V),\| . \|_{op} \big)\).  L'\defe{exponentielle}{exponentielle!de matrice} de la matrice \( A\) est cette matrice.
\end{proposition}

\begin{proof}
    Vu que la norme opérateur est une norme d'algèbre par la proposition \ref{PropEDvSQsA}, nous avons pour tout \( k\) la majoration \( \| A^k \|\leq \| A \|^k\). Nous avons donc
    \begin{equation}
        \sum_{k=0}^{\infty}\frac{ \| A^k \| }{ k! }\leq \sum_k\frac{ \| A \|^k }{ k! }.
    \end{equation}
    La dernière somme converge en vertu de la convergence de la série exponentielle donnée en exemple \ref{ExIJMHooOEUKfj}.
\end{proof}

Étant donné que c'est une limite, il y a une question de convergence et donc de topologie. C'est pour cela que nous ne pouvions pas introduire l'exponentielle de matrice avant d'avoir introduit la norme des matrices. La convergence de la série pour toute matrice sera prouvée au passage dans la proposition \ref{PropFMqsIE}.


La fonction exponentielle \(  x\mapsto e^{x}\) n'est pas un polynôme en \( x\), mais nous avons les résultat marrant suivant.
\begin{proposition} \label{PropFMqsIE}
    Si \( u\) est un endomorphisme, alors \( \exp(u)\) est un polynôme en \( u\)\footnote{Nan, mais j'te jure : \( \exp\) n'est pas un polynôme, mais $\exp(u)$ est un polynôme de \( u\).}.
\end{proposition}

\begin{proof}
    Nous considérons l'application
    \begin{equation}
        \begin{aligned}
            \varphi_u\colon \eK[X]&\to \End(E) \\
            P&\mapsto P(u)
        \end{aligned}
    \end{equation}
    Étant donné que l'image de \( \varphi_u\) est un fermé dans \( \End(E)\), il suffit de montrer que la série
    \begin{equation}
        \sum_{k=0}^{\infty}\frac{ \varphi_u(X)^k }{ k! }
    \end{equation}
    converge dans \( \End(E)\) pour qu'elle converge dans \( \Image(\varphi_u)\). Pour ce faire nous nous rappelons de la norme opérateur \eqref{ExemdefnormpMrt} et de la propriété fondamentale \( \| A^k \|\leq \| A \|^k\). En notant \( A=\varphi_u(X)\),
    \begin{equation}
        \left\| \sum_{k=n}^m\frac{ A^k }{ k! } \right\|\leq \sum_{k=n}^m\frac{ \| A^k \| }{ k! }\leq \sum_{k=n}^m\frac{ \| A \|^k }{ k! },
    \end{equation}
    ce qui est une morceau du développement de \(  e^{\| A \|}\). La limite \( n\to\infty\) est donc zéro par la convergence de l'exponentielle réelle. La suite des sommes partielles de  $e^{A}$ est donc de Cauchy. La série converge donc parce que nous sommes dans un espace vectoriel réel de dimension finie (\( \End(E)\)).
\end{proof}
% TODO : et tant qu'on y est, justifier la convergence de la série de l'exponentielle réelle.

\begin{remark}
    Pourquoi \( \exp(u)\) est-il un polynôme d'endomorphisme alors que \( \exp\) n'est pas un polynôme ? Lorsque nous disons que la fonction \( x\mapsto \exp(x)\) n'est pas un polynôme, nous sommes en train de localiser la fonction \( \exp\) à l'intérieur de l'espace de toutes les fonctions \( \eR\to \eR\), c'est à dire à l'intérieur d'un espace de dimension infinie. Au contraire lorsqu'on parle de \( \exp(u)\) et qu'on le compare aux endomorphismes \( P(u)\), nous sommes en train de repérer \( \exp(u)\) à l'intérieur de l'espace des matrices qui est de dimension finie. Il n'est donc pas étonnant que l'on parvienne moins à faire la distinction.

    Si par contre nous considérons \( \exp\) en tant qu'application \( \exp\colon \End(E)\to \End(E)\), ce n'est pas un polynôme.

    Si \( u\) et \( v\) sont des endomorphismes, nous aurons des polynômes \( P\) et \( Q\) tels que \( e^u=P(u)\) et \( e^v=Q(v)\); mais nous n'aurons en général évidemment pas \( P=Q\). En cela, \( \exp\) n'est pas un polynôme.
\end{remark}

%+++++++++++++++++++++++++++++++++++++++++++++++++++++++++++++++++++++++++++++++++++++++++++++++++++++++++++++++++++++++++++
\section{Sommes de familles infinies}
%+++++++++++++++++++++++++++++++++++++++++++++++++++++++++++++++++++++++++++++++++++++++++++++++++++++++++++++++++++++++++++

%---------------------------------------------------------------------------------------------------------------------------
\subsection{Convergence commutative}
%---------------------------------------------------------------------------------------------------------------------------

\begin{definition}
    Soit \( x_k\) une suite dans un espace vectoriel normé \( E\). Nous disons que la suite \defe{converge commutativement}{convergence!commutative} vers \( x\in E\) si \( \lim_{n\to \infty}\| x_n-x \| =0\) et si pour toute bijection \( \tau\colon \eN\to \eN\) nous avons aussi
    \begin{equation}
        \lim_{n\to \infty} \| x_{\tau(k)}-x \|=0.
    \end{equation}
    La notion de convergence commutative est surtout intéressante pour les séries. La somme
    \begin{equation}
        \sum_{k=0}^{\infty}x_k
    \end{equation}
    converge commutativement vers \( x\) si \( \lim_{N\to \infty} \| x-\sum_{k=0}^Nx_k \|=0\) et si pour toute bijection \( \tau\colon \eN\to \eN\) nous avons
    \begin{equation}
        \lim_{N\to \infty} \| x-\sum_{k=0}^Nx_{\tau(k)} \|=0.
    \end{equation}
\end{definition}

Nous démontrons maintenant qu'une série converge commutativement si et seulement si elle converge absolument.

Pour le sens inverse, nous avons la proposition suivante.
\begin{proposition}
    Soit \( \sum_{k=0}^{\infty}a_k\) une série réelle qui converge mais qui ne converge pas absolument. Alors pour tout \( b\in \eR\), il existe une bijection \( \tau\colon \eN\to \eN\) telle que \( \sum_{i=0}^{\infty}a_{\tau(i)}=b\).
\end{proposition}
Pour une preuve, voir \href{http://gilles.dubois10.free.fr/analyse_reelle/seriescomconv.html}{chez Gilles Dubois}.

\begin{proposition} \label{PopriXWvIY}
    Soit \( (a_i)_{i\in \eN}\) une suite dans \( \eC\) convergent absolument. Alors elle converge commutativement.
\end{proposition}

\begin{proof}
    Soit \( \epsilon>0\). Nous posons \( \sum_{i=0}^\infty a_i=a\) et nous considérons \( N\) tel que
    \begin{equation}
        | \sum_{i=0}^Na_i-a |<\epsilon.
    \end{equation}
    Étant donné que la série des \( | a_i |\) converge, il existe \( N_1\) tel que pour tout \( p,q>N_1\) nous ayons \( \sum_{i=p}^q| a_i |<\epsilon\). Nous considérons maintenant une bijection \( \tau\colon \eN\to \eN \). Prouvons que la série \( \sum_{i=0}^{\infty}| a_{\tau(i)} |\) converge. Nous choisissons \( M\) de telle sorte que pour tout \( n>M\), \( \tau(n)>N_1\); alors si \( p,q>M\) nous avons
    \begin{equation}
        \sum_{i=p}^q| a_{\tau(i)} |<\epsilon.
    \end{equation}
    Par conséquent la somme de la suite \( (a_{\tau(i)})\) converge. Nous devons montrer à présent qu'elle converge vers la même limite que la somme «usuelle» \( \lim_{N\to \infty} \sum_{i=0}^Na_i\).

    Soit \( n>\max\{ M,N \}\). Alors
    \begin{equation}
        \sum_{k=0}^na_{\tau(k)}-\sum_{k=0}^na_k=\sum_{k=0}^Ma_{\tau(k)}-\sum_{k=0}^Na_k+\underbrace{\sum_{M+1}^na_{\tau(k)}}_{<\epsilon}-\underbrace{\sum_{k=N+1}^na_k}_{<\epsilon}.
    \end{equation}
    Par construction les deux derniers termes sont plus petits que \( \epsilon\) parce que \( M\) et \( N\) sont les constantes de Cauchy pour les séries \( \sum a_{\tau(i)}\) et \( \sum a_i\). Afin de traiter les deux premiers termes, quitte à redéfinir \( M\), nous supposons que \( \{ 1,\ldots, N \}\subset \tau\{ 1,\ldots, M \}\); par conséquent tous les \( a_i\) avec \( i<N\) sont atteints par les \( a_{\tau(i)}\) avec \( i<M\). Dans ce cas, les termes qui restent dans la différence
    \begin{equation}
        \sum_{k=0}a_{\tau(k)}-\sum_{k=0}^Na_k
    \end{equation}
    sont des \( a_k\) avec \( k>N\). Cette différence est donc en valeur absolue plus petite que \( \epsilon\), et nous avons en fin de compte que
    \begin{equation}
        \left| \sum_{k=0}^na_{\tau(k)}-\sum_{k=0}^na_k \right| <\epsilon.
    \end{equation}
\end{proof}

\begin{proposition}     \label{PropyFJXpr}
    Soit \( \sum_{i=0}^{\infty}a_i\) une série qui converge mais qui ne converge pas absolument. Pour tout \( b\in \eR\), il existe une bijection \( \tau\colon \eN\to \eN\) telle que \( \sum_{i=}^{\infty}a_{\tau(i)}=b\).
\end{proposition}

Les propositions \ref{PopriXWvIY} et \ref{PropyFJXpr} disent entre autres qu'une série dans \( \eC\) est commutativement sommable si et seulement si elle est absolument sommable.

Soit \( (a_i)_{i\in I}\) une famille de nombres complexes indexée par un ensemble \( I\) quelconque. Nous allons nous intéresser à la somme \( \sum_{i\in I}a_i\).


Soit \( \{ a_i \}_{i\in I}\) des nombres positifs. Nous définissons la somme
\begin{equation}
    \sum_{i\in I}a_i=\sup_{\text{\( J\) fini}}\sum_{j\in J}a_j.
\end{equation}
Notons que cela est une définition qui ne fonctionne bien que pour les sommes de nombres positifs. Si \( a_i=(-1)^i\), alors selon la définition nous aurions \( \sum_i(-1)^i=\infty\). Nous ne voulons évidemment pas un tel résultat.

Dans le cas de familles de nombres réels positifs, nous avons une première définition de la somme. 
\begin{definition}  \label{DefHYgkkA}
Soit \( (a_i)_{i\in I}\) une famille de nombres réels positifs indexés par un ensemble quelconque \( I\). Nous définissons
\begin{equation}
    \sum_{i\in I}a_i=\sup_{\text{\( J\) fini dans \( I\)}}\sum_{j\in J}a_j.
\end{equation}
\end{definition}

\begin{definition}  \label{DefIkoheE}
    Si \( \{ v_i \}_{i\in I}\) est une famille de vecteurs dans un espace vectoriel normé indexée par un ensemble quelconque \( I\). Nous disons que cette famille est \defe{sommable}{famille!sommable} de somme \( v\) si pour tout \( \epsilon>0\), il existe un \( J_0\) fini dans \( I\) tel que pour tout ensemble fini \( K\) tel que \( J_0\subset K\) nous avons
    \begin{equation}
        \| \sum_{j\in K}v_j-v \|<\epsilon.
    \end{equation}
\end{definition}
Notons que cette définition implique la convergence commutative.

\begin{example}
    La suite \( a_i=(-1)^i\) n'est pas sommable parce que quel que soit \( J_0\) fini dans \( \eN\), nous pouvons trouver \( J\) fini contenant \( J_0\) tel que \( \sum_{j\in J}(-1)^j>10\). Pour cela il suffit d'ajouter à \( J_0\) suffisamment de termes pairs. De la même façon en ajoutant des termes impairs, on peut obtenir \( \sum_{j\in J'}(-1)^i<-10\).
\end{example}

\begin{example}
    De temps en temps, la somme peut sortir d'un espace. Si nous considérons l'espace des polynômes \( \mathopen[ 0 , 1 \mathclose]\to \eR\) muni de la norme uniforme, la somme de l'ensemble
    \begin{equation}
        \{ 1,-1,\pm\frac{ x^n }{ n! } \}_{n\in \eN}
    \end{equation}
    est zéro.

    Par contre la somme de l'ensemble \( \{ 1,\frac{ x^n }{ n! } \}_{n\in \eN}\) est l'exponentielle qui n'est pas un polynôme.
\end{example}

\begin{example}
    Au sens de la définition \ref{DefIkoheE} la famille
    \begin{equation}
        \frac{ (-1)^n }{ n }
    \end{equation}
    n'est pas sommable. En effet la somme des termes pairs est \( \infty\) alors que la somme des termes impairs est \( -\infty\). Quel que soit \( J_0\in \eN\), nous pouvons concocter, en ajoutant des termes pairs, un \( J\) avec \( J_0\subset J\) tel que \( \sum_{j\in J}(-1)^j/j\) soit arbitrairement grand. En ajoutant des termes négatifs, nous pouvons également rendre \( \sum_{j\in J}(-1)^j/j\) arbitrairement petit.
\end{example}

\begin{proposition} \label{PropVQCooYiWTs}
    Si \( (a_{ij})\) est une famille de nombres positifs indexés par \( \eN\times \eN\) alors
    \begin{equation}
        \sum_{(i,j)\in \eN^2}a_{ij}=\sum_{i=1}^{\infty}\Big( \sum_{j=1}^{\infty}a_{ij} \Big)
    \end{equation}
    où la somme de gauche est celle de la définition \ref{DefHYgkkA}.
\end{proposition}
%TODO : cette proposition peut être vue comme une application de Fubini pour la mesure de comptage. Le faire et référentier ici.

\begin{proof}
    Nous considérons \( J_{m,n}=\{ 0,\ldots, m \}\times \{ 0,\ldots, n \}\) et nous avons pour tout \( m\) et \( n\) :
    \begin{equation}
        \sum_{(i,j)\in \eN^2}a_{ij}\geq \sum_{(i,j)\in J_{m,n}}a_{ij}=\sum_{i=1}^m\Big( \sum_{j=1}^na_{ij} \Big).
    \end{equation}
    Si nous fixons \( m\) et que nous prenons la limite \( n\to \infty\) (qui commute avec la somme finie sur \( i\)) nous trouvons
    \begin{equation}
        \sum_{(i,j)\in \eN^2}a_{ij}\geq =\sum_{i=1}^m\Big( \sum_{j=1}^{\infty}a_{ij} \Big).
    \end{equation}
    Cela étant valable pour tout \( m\), c'est encore valable à la limite \( m\to \infty\) et donc
    \begin{equation}
        \sum_{(i,j)\in \eN^2}a_{ij}\geq \sum_{i=1}^{\infty}\Big( \sum_{j=1}^{\infty}a_{ij} \Big).
    \end{equation}
    
    Pour l'inégalité inverse, il faut remarquer que si \( J\) est fini dans \( \eN^2\), il est forcément contenu dans \( J_{m,n}\) pour \( m\) et \( n\) assez grand. Alors
    \begin{equation}
        \sum_{(i,j)\in J}a_{ij}\leq \sum_{(i,j)\in J_{m,n}}a_{ij}=\sum_{i=1}^m\sum_{j=1}^na_{ij}\leq \sum_{i=1}^{\infty}\Big( \sum_{j=1}^{\infty}a_{ij} \Big).
    \end{equation}
    Cette inégalité étant valable pour tout ensemble fini \( J\subset \eN^2\), elle reste valable pour le supremum.    
\end{proof}

La définition générale de la somme \ref{DefIkoheE} est compatible avec la définition usuelle dans les cas où cette dernière s'applique.
\begin{proposition}[commutative sommabilité]\label{PropoWHdjw}
    Soit \( I\) un ensemble dénombrable et une bijection \( \tau\colon \eN\to I\). Soit \( (a_i)_{i\in I}\) une famille dans un espace vectoriel normé. Alors
    \begin{equation}
        \sum_{k=0}^{\infty}a_{\tau(k)}=\sum_{i\in I}a_i
    \end{equation}
    dès que le membre de droite existe. Le membre de gauche est définit par la limite usuelle.
\end{proposition}

\begin{proof}
    Nous posons \( a=\sum_{i\in I}a_i\). Soit \( \epsilon>0\) et \( J_0\) comme dans la définition. Nous choisissons
    \begin{equation}
        N>\max_{j\in J_0}\{ \tau^{-1}(j) \}.
    \end{equation}
    En tant que sommes sur des ensembles finis, nous avons l'égalité
    \begin{equation}
        \sum_{k=0}^Na_{\tau(k)}=\sum_{j\in J_0}a_j
    \end{equation}
    où \( J\) est un sous-ensemble de \( I\) contenant \( J_0\). Soit \( J\) fini dans \( I\) tel que \( J_0\subset J\). Nous avons alors
    \begin{equation}
        \| \sum_{k=0}^Na_{\tau(k)}-a \|=\| \sum_{j\in J}a_j-a \|<\epsilon.
    \end{equation}
    Nous avons prouvé que pour tout \( \epsilon\), il existe \( N\) tel que \( n>N\) implique \( \| \sum_{k=0}^na_{\tau(k)}-a\| <\epsilon\).
\end{proof}

\begin{corollary}
    Nous pouvons permuter une somme dénombrable et une fonction continue. C'est à dire que si \( f\) est une fonction continue sur l'espace vectoriel normé \( E\) et \( (a_i)_{i\in I}\) une famille sommable dans \( E\) alors
    \begin{equation}
        f\left( \sum_{i\in I}a_i \right)=\sum_{i\in I}f(a_i).
    \end{equation}
\end{corollary}

\begin{proof}
    En utilisant une bijection \( \tau\) entre \( I\) et \( \eN\) avec la proposition \ref{PropoWHdjw} ainsi que le résultat connu à propos des sommes sur \( \eN\), nous avons
    \begin{equation}
        f\left( \sum_{i\in I}a_i \right)=f\left( \sum_{k=0}^{\infty}a_{\tau(k)} \right)=\sum_{k=0}^{\infty}f(a_{\tau(k)})=\sum_{i\in I}f(a_i).
    \end{equation}
\end{proof}

La proposition suivante nous enseigne que les sommes infinies peuvent être manipulée de façon usuelle.
\begin{proposition} \label{PropMpBStL}
    Soit \( I\) un ensemble dénombrable. Soient \( (a_i)_{i\in I}\) et \( (b_i)_{i\in I}\), deux familles de réels positifs telles que \( a_i<b_i\) et telles que \( (b_i)\) est sommable. Alors \( (a_i)\) est sommable.

    Si \( (a_i)_{i\in I}\) est une famille de complexes telle que \( (| a_i |)\) est sommable, alors \( (a_i)\) est sommable.
\end{proposition}

%+++++++++++++++++++++++++++++++++++++++++++++++++++++++++++++++++++++++++++++++++++++++++++++++++++++++++++++++++++++++++++
\section{Mini introduction aux nombres \texorpdfstring{p}{$p$}-adiques}
%+++++++++++++++++++++++++++++++++++++++++++++++++++++++++++++++++++++++++++++++++++++++++++++++++++++++++++++++++++++++++++

\subsection{La flèche d'Achille}\label{s:un}

C'est un grand classique que je donne ici juste comme introduction pour montrer que des série infinies peuvent donner des nombres finis de manière tout à fait intuitive.

Achille tire une flèche vers un arbre situé à $\unit{10}{\meter}$ de lui. Disons que la flèche avance à une vitesse constante de $\unit{1}{\meter\per\second}$. Il est clair que la flèche mettra $\unit{10}{\second}$ pour toucher l'arbre. En $\unit{5}{\second}$, elle aura parcouru la moitié de son chemin. On le note :
\[
\text{temps}=5s+\ldots
\]
Reste \( \unit{5}{\meter}\) à faire. En $\unit{2.5}{\second}$, elle aura fait la moitié de ce chemin chemin, soit $2.5m=\frac{10}{4}m$. On le note :
\[
\text{temps}=\frac{10}{2}s+\frac{10}{4}s+
\]
Reste $2.5m$ à faire. La moitié de ce trajet, soit $\frac{10}{8}m$, est parcouru en $\frac{10}{8}s$; on le note encore, mais c'est la dernière fois !

\[
\text{temps}=\frac{10}{2}s+\frac{10}{4}s+\frac{10}{8}s+
\]
En continuant ainsi à regarder la flèche qui parcours des demi-trajets puis des demi de demi-trajets et encore des demi de demi de demi-trajets, et en sachant que le temps total est $10s$, on trouve :
\[
10\left( \frac{1}{2}+\frac{1}{4}+\frac{1}{8}+\frac{1}{16}+\ldots  \right)=10.
\]
On doit donc croire que la somme jusqu'à l'infini des inverse des puissances de deux vaut $1$ :
\[
   \sum_{n=1}^{\infty}\frac{1}{2^n}=1.
\]
Cela peut être démontré à la loyale.

\subsection{La tortue et Achille}

Maintenant qu'on est convaincu que des sommes infinies peuvent représenter des nombres tout à fait normaux, passons à un truc plus marrant.

Achille, qui marche peinard à $\unit{10}{\meter\per\hour}$, part avec $1m$ d'avance sur une tortue qui avance à $\unit{1}{\meter\per\hour}$. Le temps que la tortue arrive au point de départ d'Achille, Achille aura parcouru $10m$, et le temps que la tortue mettra pour arriver à ce point, eh bien, Achille ne sera déjà plus là : il sera à $100m$. Si la tortue tient bon pendant un temps infini, et si l'on est confiant en le genre de raisonnements faits à la section \ref{s:un}, elle rattrapera Achille dans 
\[
1m+10m+100m+1000m+\ldots
\]
Autant dire que ça ne risque pas d'arriver. Et pourtant, mettons en équations : 
\begin{subequations}
    \begin{numcases}{}
        x_{\text{Achile}}(t)=1+10t\\
        x_{\text{tortue}}(t)=t.
    \end{numcases}
\end{subequations}
La tortue rejoints Achille au temps \( t\) tel que \( x_{\text{Achille}(t)}=x_{\text{tortue}}(t)\). Un mini calcul donne $t=-1/9$. Physiquement, c'est une situation logique. Peut-on en déduire une égalité mathématique du style de 
\[
1+10+100+1000+\ldots=-\frac{1}{9}\; ???
\]
Là où les choses deviennent jolies, c'est quand on cherche à voir ce que peut bien être la valeur d'un hypothétique $x=1+10+100+1000+\ldots$. En effet, logiquement on devrait avoir
\begin{equation*}
\begin{split}
\frac{x}{10}&=\frac{1}{10}+1+10+100+\ldots\\
            &=\frac{1}{10}+x.
\end{split}
\end{equation*}
Reste à résoudre l'équation du premier degré : $\frac{x}{10}=x+\frac{1}{10}$. Ai-je besoin de donner la solution ?

%---------------------------------------------------------------------------------------------------------------------------
\subsection{Dans les nombres \texorpdfstring{p}{$ p$}-adiques, c'est vrai}
%---------------------------------------------------------------------------------------------------------------------------

Nous nous proposons d'apprendre sur les nombres \( p\)-adiques juste ce qu'il faut pour montrer que l'égalité
\begin{equation}
    \sum_{k=0}^{\infty}10^k=-\frac{1}{ 9 }
\end{equation}
est vraie dans les nombres \( 5\)-adiques. Tout ce qu'il faut est sur \wikipedia{fr}{Nombre_p-adique}{wikipedia}.

Soit \( a\in \eN\) et \( p\), un nombre premier. La \defe{valuation}{valuation!$p$-adique} \( p\)-adique de \( a\) est l'exposant de \( p\) dans la décomposition de \( a\) en nombres premiers. On la note \( v_p(a)\). Pour un rationnel on définit
\begin{equation}
    v_p\left( \frac{ a }{ b } \right)=v_p(a)-v_p(b)
\end{equation}
La \defe{valeur absolue}{valeur absolue!$p$-adique} \( p\)-adique de \( r\in \eQ\) est 
\begin{equation}
    | r |_p=p^{-v_p(r)}.
\end{equation}
Nous posons \( | 0 |_p=0\). De là nous considérons la distance
\begin{equation}
    d_p(x,y)=| x-y |_p.
\end{equation}

\begin{lemma}
    L'espace \( (\eQ,d_p)\) est un espace métrique\footnote{Définition \ref{DefMVNVFsX}}.
\end{lemma}
\index{topologie!\( p\)-adique}

Nous considérons maintenant \( p=5\). Étant donné que \( a=5\cdot 2\) nous avons \( v_5(10)=1\) et
\begin{equation}
    v_5\left( \frac{1}{ 9 } \right)=v_5(1)-v_5(9)=0.
\end{equation}
Nous avons
\begin{equation}
    \sum_{k=0}^N10^k+\frac{1}{ 9 }=\frac{ 10^{N+1} }{ 9 }
\end{equation}
mais
\begin{equation}
    v_p\left( \frac{ 10^{N+1} }{ 9 } \right)=v_5(10^{N+1})-v_5(9)=N+1.
\end{equation}
Par conséquent
\begin{equation}
    d_5\big( \sum_{k=0}^N10^k,-\frac{1}{ 9 } \big)=| \frac{ 10^{N+1} }{ 9 } |_p=p^{-(N+1)}.
\end{equation}
En passant à la limite,
\begin{equation}
    \lim_{N\to \infty} d_5\big( \sum_{k=0}^N10^k,-\frac{1}{ 9 } \big)=0,
\end{equation}
ce qui signifie que\footnote{Voir la définition \ref{DefGFHAaOL} de la convergence d'une série dans un espace métrique.}
\begin{equation}
    \sum_{k=0}^{\infty}10^k=-\frac{1}{ 9 }.
\end{equation}
