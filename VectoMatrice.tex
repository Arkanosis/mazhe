% This is part of Mes notes de mathématique
% Copyright (c) 2011-2012
%   Laurent Claessens
% See the file fdl-1.3.txt for copying conditions.

%+++++++++++++++++++++++++++++++++++++++++++++++++++++++++++++++++++++++++++++++++++++++++++++++++++++++++++++++++++++++++++
\section{Parties libres, génératrices, bases et dimension}
%+++++++++++++++++++++++++++++++++++++++++++++++++++++++++++++++++++++++++++++++++++++++++++++++++++++++++++++++++++++++++++
 
\begin{definition}
	Un sous-ensemble $B=\{v_1,\ldots,v_q\}$ de $\eR^m$ est une \defe{base}{base} de $\eR^m$ s'il satisfait les conditions suivantes
\begin{itemize}
	\item $B$ est \defe{libre}{libre}, c'est à dire
\[
\sum_{i=1}^{q}a_i v_i=0_{m} \quad\Leftrightarrow\quad a_i=0, \forall i=1,\ldots,q.
\]
\item $B$ est \defe{générateur}{générateur}, c'est à dire que pour tout $x$ dans $\eR^m$ il existe un ensemble de coefficients $\{a_i\in\eR, i=1,\ldots,n\}$ tel que
\[\sum_{i=1}^{q}a_i v_i=x.\]
\end{itemize}
\end{definition}
Il existe une infinité de bases de $\eR^m$. On peut démontrer que le cardinal de toute base de $\eR^m$ est $m$, c'est à dire que toute base de $\eR^m$ possède exactement $m$ éléments.

La base de $\eR^m$ qu'on dit \defe{canonique}{canonique!base}\index{base!canonique de $\eR^m$} (c.à.d. celle qu'on utilise tout le temps) est $\mathcal{B}=\{e_1,\ldots, e_m\}$, où le vecteur $e_j$ est 
\begin{equation}\nonumber
  e_j=
\begin{array}{cc}
  \begin{pmatrix}
    0\\\vdots\\0\\1\\ 0\\\vdots\\0
  \end{pmatrix} & 
  \begin{matrix}
    \quad\\\quad\\\leftarrow\textrm{j-ème} \quad\\\quad\\\quad\\
  \end{matrix}
\end{array}.
\end{equation}
La composante numéro $j$ de $e_i$ est $1$ si $i=j$ et $0$ si $i\neq j$. Cela s'écrit $(e_i)_j=\delta_{ij}$ où $\delta$ est le \defe{symbole de Kronecker}{Kronecker} défini par
\begin{equation}
	\delta_{ij}=\begin{cases}
		1	&	\text{si $i=j$}\\
		0	&	 \text{si $i\neq j$.}
	\end{cases}
\end{equation}
Les éléments de la base canonique de $\eR^m$ peuvent donc être écrits $e_i=\sum_{k=1}^m\delta_{ik}e_k$.


\begin{definition}
    Si \( E\) est un espace vectoriel, une partie finie \( (u_i)_{1\leq i\leq n}\) de \( E\) est \defe{libre}{libre!partie} si l'égalité
    \begin{equation}
        a_1 u_1+\ldots +a_nu_n=0
    \end{equation}
    implique \( a_i=0\) pour tout \( i\).

    Une partie infinie est libre si toute ses parties finies le sont.
\end{definition}
La définition de liberté dans le cas des parties infinies a son importance lorsqu'on parle d'espaces vectoriels de dimension infinies (en dimension finie, aucune partie infinie n'est libre) parce que cela fera une différence entre une base algébrique et une base hilbertienne par exemple.

Un espace vectoriel est \defe{de type fini}{type!fini (espace vectoriel)} si il contient une partie génératrice finie. Nous verrons dans les résultats qui suivent que cette définition est en réalité inutile parce qu'une espace vectoriel sera de type fini si et seulement si il est de dimension finie.

\begin{lemma}       \label{LemytHnlD}
    Si \( E\) a une famille génératrice de cardinal \( n\), alors toute famille de \( n+1\) éléments est liée.
\end{lemma}

\begin{proof}
    Nous procédons par récurrence sur \( n\) -- qui n'est pas exactement la dimension d'un espace vectoriel fixé. Pour \( n=1\), nous avons \( E=\langle e\rangle\) et donc si \( v_1,v_2\in E\) nous avons \( v_1=\lambda_1 e\), \( v_2=\lambda_2e\) et donc \( \lambda_2v_1-\lambda_1v1=0\). Cela prouve le résultat dans le cas de la dimension \( 1\).

    Supposons maintenant que le résultat soit vrai pour \( k<n\), c'est à dire que pour tout espace vectoriel contenant une partie génératrice de cardinal \( k<n\), les parties de \( k+1\) éléments sont liées. Soit maintenant un espace vectoriel muni d'une partie génératrice \( G=\{ e_1,\ldots, e_n \}\) de \( n\) éléments, et montrons que toute partie \( V=\{ v_1,\ldots, v_{n+1} \}\) contenant \( n+1\) éléments est liée. Dans nos notations nous supposons que les \( e_i\) sont des vecteurs distincts et les \( v_i\) également. Nous les supposons également tous non nuls. Étant donné que \( \{ e_i \}\) est génératrice nous pouvons définir les nombres \( \lambda_{ij}\) par
    \begin{equation}
        v_i=\sum_{k=1}^n\lambda_{ij}e_j
    \end{equation}
    Vu que
    \begin{equation}
        v_{n+1}=\sum_{k=1}^n\lambda_{n+1,k}e_k\neq 0,
    \end{equation}
    quitte à changer la numérotation des \( e_i\) nous pouvons supposer que \( \lambda_{n+1,n}\neq 0\). Nous considérons les vecteurs
    \begin{equation}
        w_i=\lambda_{n+1,n}v_i-\lambda_{i,n}v_{n+1}.
    \end{equation}
    En calculant un peu,
    \begin{subequations}
        \begin{align}
            w_i&=\lambda_{n+1,n}\sum_k\lambda_{i,k}e_k-\lambda_{i,n}\sum_k\lambda_{n+1,k}e_k\\
            &=\sum_{k=1}^{n-1}\big( \lambda_{n+1,n}\lambda_{i,k}-\lambda_{i,n}\lambda_{n+1,} \big)e_k
        \end{align}
    \end{subequations}
    parce que les termes en \( e_n\) se sont simplifiés. Donc la famille \( \{ w_1,\ldots, w_n \}\) est une famille de \( n\) vecteurs dans l'espace vectoriel \( \Span\{ e_1,\ldots, e_{n-1} \}\); elle est donc liée par l'hypothèse de récurrence. Il existe donc des nombres \( \alpha_1,\ldots, \alpha_n\in \eK\) non tous nuls tels que
    \begin{equation}        \label{EqOQGGoU}
        0=\sum_{i=1}^n\alpha_iw_i=\sum_{i=1}^n\alpha_i\lambda_{n+1,n}v_i-\left( \sum_{i=1}^n\alpha_i\lambda_{i,n} \right)v_{n+1}.
    \end{equation}
    Vu que \( \lambda_{n+1,n}\neq 0\) et que parmi les \( \alpha_i\) au moins un est non nul, nous avons au moins un des produits \( \alpha_i\lambda_{n+1,n}\) qui est non nul. Par conséquent \eqref{EqOQGGoU} est une combinaison linéaire nulle non triviale des vecteurs de \( \{ v_1,\ldots, v_{n+1} \}\). Cette partie est donc liée.
\end{proof}

\begin{lemma}   \label{LemkUfzHl}
    Soit \( L\) une partie libre et \( G\) une partie génératrice. Soit \( B\) une partie maximale parmi les parties libres \( L'\) telles que \( L\subset L'\subset G\). Alors \( B\) est une base.
\end{lemma}
Qu'entend-t-on par «maximale» ? La partie \( B\) doit être libre, contenir \( L\), être contenue dans \( G\) et de plus avoir la propriété que \( \forall x\in G\setminus B\), la partie \( B\cup\{ x \}\) est liée.

\begin{proof}
    D'abord si \( G\) est une base, alors toutes les parties de \( G\) sont libres et le maximum est \( B=G\). Dans ce cas le résultat est évident. Nous supposons donc que \( G\) est liée.

    La partie \( B=\{ b_1,\ldots, b_l \}\) est libre parce qu'on l'a prise parmi les libres. Montrons que \( B\) est génératrice. Soit \( x\in G\setminus B\); par hypothèse de maximalité, \( B\cup\{ x \}\) est liée, c'est à dire qu'il existe des nombres \( \lambda_i\), \( \lambda_x\) non tous nuls tels que
    \begin{equation}    \label{EqxfkevM}
        \sum_{i=1}^l\lambda_ib_i+\lambda_xx=0.
    \end{equation}
    Si \( \lambda_x=0\) alors un de \( \lambda_i\) doit être non nul et l'équation \eqref{EqxfkevM} devient une combinaison linéaire nulle non triviale des \( b_i\), ce qui est impossible parce que \( B\) est libre. Donc \( \lambda_x\neq 0\) et
    \begin{equation}
        x=\frac{1}{ \lambda_x }\sum_{i=1}^l\lambda_ib_i.
    \end{equation}
    Donc tous les éléments de \( G\setminus B\) sont des combinaisons linéaires des éléments de \( B\), et par conséquent, \( G\) étant génératrice, tous les éléments de \( E\) sont combinaisons linéaires d'éléments de \( B\). 
\end{proof}

\begin{theorem} \label{ThonmnWKs}
    Soit \( E\) un espace vectoriel de type fini sur le corps \( \eK\).
    \begin{enumerate}
        \item   \label{ItemBazxTZ}
            Si \( L\) est une partie libre et si \( G\) est une partie génératrice contenant \( L\), alors il existe une base \( B\) telle que \( L\subset B\subset G\).
        \item
            Toutes les bases sont finies et ont même cardinal.
    \end{enumerate}
\end{theorem}
Notons que puisque \( E\) lui-même est générateur, le point \ref{ItemBazxTZ} implique que toute partie libre peut être étendue en une base.

\begin{proof}
    Vu que \( E\) est de type fini, il admet une partie génératrice \( G\) de cardinal fini \( n\). Donc une partie libre est de cardinal au plus \( n\) par le lemme \ref{LemytHnlD}. Soit \( L\), une partie libre contenue dans \( G\) (ça existe : par exemple \( L=\emptyset\)). La partie \( B\) maximalement libre contenue dans \( G\) et contenant \( L\) est une base par le lemme \ref{LemkUfzHl}.

    En ce qui concerne la seconde partie du théorème, soient \( B\) et \( B'\), deux bases. En particulier \( B\) est génératrice et \( B'\) est libre, donc le lemme \ref{LemytHnlD} indique que \( \Card(B')\leq \Card(B)\). Par symétrie on a l'inégalité inverse. Donc \( \Card(B)=\Card(B')\).
\end{proof}

Le théorème suivant est essentiellement une reformulation du théorème \ref{ThonmnWKs}.
\begin{theorem} \label{ThoBaseIncompjblieG}
    Soit \( E\) un espace vectoriel de dimension finie et \( \{ e_i \}_{i\in I}\) une partie génératrice de \( E\).

    \begin{enumerate}
        \item
            Il existe \( J\subset I\) tel que \( \{ e_i \}_{i\in J}\) est une base. Autrement dit : de toute partie génératrice nous pouvons extraire une base.
        \item
            Soit \( \{ f_1,\ldots, f_l \}\) une partie libre. Alors nous pouvons la compléter en utilisant des éléments \( e_i\). C'est à dire qu'il existe \( J\subset I\) tel que \( \{ f_k \}\cup\{ e_i \}_{i\in J}\) soit une base.
    \end{enumerate}
\end{theorem}

Soit \( F\) un sous-espace vectoriel de l'espace vectoriel \( E\). La \defe{codimension}{codimension} de \( F\) dans \( E\) est
\begin{equation}
    \codim_E(F)=\dim(E/F).
\end{equation}

Le théorème suivant est valable également en dimension infinie; ce sera une des rares incursions en dimension infinie de ce chapitre.
\begin{theorem}[Théorème du rang]\index{théorème!du rang}       \label{ThoGkkffA}
       Soient \( E\) et \( F\) deux espaces vectoriels (de dimensions finies ou non) et soit \( f\colon E\to F\) une application linéaire. Alors le rang de \( f\) est égal à la codimension du noyau, c'est à dire
       \begin{equation}
           \rang(f)+\dim\ker f=\dim E.
       \end{equation}

       Dans le cas de dimension infinie afin d'éviter les problèmes d'arithmétique avec l'infini nous énonçons le théorème en disant que si \( (u_s)_{s\in S}\) est une base de \( \ker f\) et si \( \big( f(v_t) \big)_{t\in T}\) est une base de \( \Image(f)\) alors  \( (u_s)_{s\in s}\cup (v_t)_{t\in T}\) est une base de \( E\).
\end{theorem}

\begin{proof}
    Nous devons montrer que 
    \begin{equation}
          (u_s)_{s\in S}\cup (v_t)_{t\in T}
    \end{equation}
    est libre et générateur.

    Soit \( x\in E\). Nous définissons les nombres \( x_t\) par la décomposition de \( f(x)\) dans la base \( \big( f(v_t) \big)\) :
    \begin{equation}
        f(x)=\sum_{t\in T}x_tf(v_t).
    \end{equation}
    Ensuite le vecteur \( x=\sum_tx_tv_t\) est dans le noyau de \( f\), par conséquent nous le décomposons dans la base \( (u_s)\) :
    \begin{equation}
        x-\sum_tx_tv_t=\sum_s\in S x_su_s.
    \end{equation}
    Par conséquent
    \begin{equation}
        x=\sum_sx_su_s+\sum_tx_tv_t.
    \end{equation}
    
    En ce qui concerne la liberté nous écrivons
    \begin{equation}
        \sum_tx_tv_t+\sum_sx_su_s=0.
    \end{equation}
    En appliquant \( f\) nous trouvons que 
    \begin{equation}
        \sum_tx_tf(v_t)=0
    \end{equation}
    et donc que les \( x_t\) doivent être nuls. Nous restons avec \( \sum_sx_su_s=0\) qui à son tour implique que \( x_s=0\).
\end{proof}
Un exemple d'utilisation de ce théorème en dimension infinie sera donné dans le cadre du théorème de Fréchet-Riesz, théorème \ref{ThoQgTovL}.

\begin{proposition}[\cite{RombaldiO}]   \label{PropTVKbxU}
    Soit \( E\), un espace vectoriel sur un corps infini et \( (F_k)_{k=1,\ldots, r}\), des sous-espaces vectoriels propres de \( E\) tels que \( \bigcup_{i=1}^rF_i=E\). Alors \( E=F_k\) pour un certain \( k\).

    Autrement dit, l'union finie de sous-espaces propres ne peut être égal à l'espace complet.
\end{proposition}

%+++++++++++++++++++++++++++++++++++++++++++++++++++++++++++++++++++++++++++++++++++++++++++++++++++++++++++++++++++++++++++
\section{Produit scalaire}
%+++++++++++++++++++++++++++++++++++++++++++++++++++++++++++++++++++++++++++++++++++++++++++++++++++++++++++++++++++++++++++

\begin{definition}      \label{DefYNWUFc}
	Soient $u$ et $v$, deux vecteurs de $\eR^m$. Le \defe{produit scalaire}{produit!scalaire} de $u$ et $v$, noté $\langle u, v\rangle $ ou $u\cdot v$ est le réel
	\begin{equation}		\label{EqDefProdScalsumii}
		\langle u, v\rangle =\sum_{k=1}^m u_kv_k=u_1v_1+u_2v_2+\ldots+u_mv_n.
	\end{equation}
\end{definition}

Calculons par exemple le produit scalaire de deux vecteurs de la base canonique : $\langle e_i, e_j\rangle $. En utilisant la formule de définition et le fait que $(e_i)_k=\delta_{ik}$, nous avons
\begin{equation}
	\langle e_i, e_j\rangle =\sum_{k=1}^m\delta_{ik}\delta_{jk}.
\end{equation}
Nous pouvons effectuer la somme sur $k$ en remarquant qu'à cause du $\delta_{ik}$, seul le terme avec $k=i$ n'est pas nul. Effectuer la somme revient donc à remplacer tous les $k$ par des $i$ :
\begin{equation}
	\langle e_i, e_j\rangle =\delta_{ii}\delta_{ji}=\delta_{ji}.
\end{equation}

Une des propriétés intéressantes du produit scalaire est qu'il permet de décomposer un vecteur dans une base, comme nous le montre la proposition suivante.

\begin{proposition}		\label{PropScalCompDec}
	Si nous notons $v_i$ les composantes du vecteur $v$, c'est à dire si $v=\sum_{i=1}^m v_ie_i$, alors nous avons $v_j=\langle v, e_j\rangle $.
\end{proposition}

\begin{proof}
	\begin{equation}		\label{Eqvejscalcomp}
		v\cdot e_j=\sum_{i=1}^m\langle v_ie_i, e_j\rangle =\sum_{i=1}^mv_i\langle e_i, e_j\rangle =\sum_{i=1}^mv_i\delta_{ij}
	\end{equation}
	En effectuant la somme sur $i$ dans le membre de droite de l'équation \eqref{Eqvejscalcomp}, tous les termes sont nuls sauf celui où $i=j$; il reste donc
	\begin{equation}
		v\cdot e_j=v_j.
	\end{equation}
\end{proof}

Le produit scalaire ne dépend en réalité pas de la base orthogonale choisie. 

\begin{lemma}
	Si $\{ e_i \}$ est la base canonique, et si $\{ f_i \}$ est une autre base orthonormale, alors si $u$ et $v$ sont deux vecteurs de $\eR^m$, nous avons
	\begin{equation}
		\sum_i u_iv_j=\sum_iu'_iv'_j
	\end{equation}
	où $u_i$ sont les composantes de $u$ dans la base $\{ e_i \}$ et $u'_i$ sont celles dans la base $\{ f_i \}$.
\end{lemma}

\begin{proof}
	La preuve demande un peu d'algèbre linéaire. Étant donné que $\{ f_i \}$ est une base orthonormale, il existe une matrice $A$ orthogonale ($AA^t=\mtu$) telle que $u'_i=\sum_jA_{ij}u_j$ et idem pour $v$. Nous avons alors
	\begin{equation}
		\begin{aligned}[]
			\sum_iu'_iv'_j&=\sum_i\left( \sum_jA_{ij} u_j\right)\left( \sum_k A_{ik}v_k \right)\\
			&=\sum_{ijk}A_{ij}A_{ik}u_jv_k\\
			&=\sum_{jk}\underbrace{\sum_i(A^t)_{ji}A_{ik}}_{=\delta_{jk}}u_jv_k\\
			&=\sum_{jk}\delta_{jk}u_jv_k\\
			&=\sum_ku_jv_k.
		\end{aligned}
	\end{equation}	
\end{proof}

Cette proposition nous permet de réellement parler du produit scalaire entre deux vecteurs de façon intrinsèque sans nous soucier de la base dans laquelle nous regardons les vecteurs.

Nous dirons que deux vecteurs sont \defe{orthogonaux}{orthogonal} lorsque leur produit scalaire est nul. Nous écrivons que $u\perp v$ lorsque $\langle u, v\rangle =0$.
\begin{definition}	\label{DefNormeEucleApp}
	La \defe{norme euclidienne}{norme!euclidienne!dans $\eR^m$} d'un élément de $\eR^m$ est définie par $\| u \|=\sqrt{u\cdot u}$.
\end{definition}

Cette définition est motivée par le fait que le produit scalaire $u\cdot u$ donne exactement la norme usuelle donnée par le théorème de Pythagore :
\begin{equation}
	u\cdot u=\sum_{i=1}^mu_iu_i=\sum_{i=1}^m u_i^2=u_1^2+u_2^2+\ldots+u_m^2.
\end{equation}

Le fait que $e_i\cdot e_j=\delta_{ij}$ signifie que la base canonique est \defe{orthonormée}{orthonormé}, c'est à dire que les vecteurs de la base canonique sont orthogonaux deux à deux et qu'ils ont tout $1$ comme norme.

\begin{lemma}\label{LemSclNormeXi}
	Pour tout $u\in\eR^m$, il existe un $\xi\in\eR^m$ tel que $\| u \|=\xi\cdot u$ et $\| \xi \|=1$.
\end{lemma}

\begin{proof}
	Vérifions que le vecteur $\xi=u/\| u \|$ ait les propriétés requises. D'abord $\| \xi \|=1$ parce que $u\cdot u=\| u \|^2$. Ensuite
	\begin{equation}
		\xi\cdot u=\frac{ u\cdot u }{ \| u \| }=\frac{ \| u \|^2 }{ \| u \| }=\| u \|.
	\end{equation}
\end{proof}

\begin{theorem}[Inégalité de Cauchy-Schwarz]\index{Cauchy-Schwarz}\index{inégalité!Cauchy-Schwarz}      \label{ThoAYfEHG}
	Si $X$ et $Y$ sont des vecteurs, alors
	\begin{equation}
		| X\cdot Y |\leq\| X \|\| Y \|.
	\end{equation}
\end{theorem}

\begin{proof}
	Étant donné que les deux membres de l'inéquation sont positifs, nous allons travailler en passant au carré afin d'éviter les racines carrés dans le second membre.
	
	Nous considérons la fonction
	\begin{equation}
		\varphi(t)=\| X+tY \|=(X+tY)\cdot(X+tY)=X\cdot X+tX\cdot Y+tY\cdot X+t^2Y\cdot Y.
	\end{equation}
	En ordonnant les termes selon les puissance de $t$,
	\begin{equation}
		\varphi(t)=\| Y \|^2t^2+2(X\cdot Y)t+\| X \|^2.
	\end{equation}
	Cela est un polynôme du second degré en $t$. Par conséquent le discriminant\footnote{Le fameux $b^2-4ac$.} doit être négatif. Nous avons donc
	\begin{equation}
		4(X\cdot Y)^2-4\| X \|^2\| Y \|^2\leq 0,
	\end{equation}
	ce qui donne immédiatement
	\begin{equation}
		(X\cdot Y)^2\leq\| X \|^2\| Y^2 \|.
	\end{equation}
	
\end{proof}


%+++++++++++++++++++++++++++++++++++++++++++++++++++++++++++++++++++++++++++++++++++++++++++++++++++++++++++++++++++++++++++
\section{Produit vectoriel}
%+++++++++++++++++++++++++++++++++++++++++++++++++++++++++++++++++++++++++++++++++++++++++++++++++++++++++++++++++++++++++++

\begin{definition}
	Soient $u$ et $v$, deux vecteurs de $\eR^3$. Le \defe{produit vectoriel}{produit!vectoriel} de $u$ et $v$ est le vecteur $u\times v$ défini par 
	\begin{equation}
		\begin{aligned}[]
		u\times v&=\begin{vmatrix}
			e_1	&	e_2	&	e_3	\\
			u_1	&	u_2	&	u_3	\\
			v_1	&	v_2	&	v_3
		\end{vmatrix}\\
		&=
		(u_2v_3-u_3v_2)e_1+(u_3v_1-u_1v_3)e_2+(u_1v_2-u_2v_1)e_3
		\end{aligned}
	\end{equation}
	où les vecteurs $e_1$, $e_2$ et $e_3$ sont les vecteurs de la base canonique de $\eR^3$.
\end{definition}
La notion de produit vectoriel est propre à $\eR^3$; il n'y a pas de généralisation simple aux espaces $\eR^m$.

Nous n'allons pas nous attarder sur les nombreuses propriétés du produit vectoriel. Les principales sont résumées dans la proposition suivante.
\begin{proposition}
	Si $u$ et $v$ sont des vecteurs de $\eR^3$, alors le vecteur $u\times v$ est l'unique vecteur qui est perpendiculaire à $u$ et $v$ en même temps, de norme égal à la surface du parallélogramme construit sur $u$ et $v$ et tel que les vecteurs $u$, $v$, $u\times v$ forment une base dextrogyre.
\end{proposition}
La chose importante à retenir est que le produit vectoriel permet de construire un vecteur simultanément perpendiculaire à deux vecteurs donnés. Le vecteur $u\times v$ est donc linéairement indépendant de $u$ et $v$. En pratique, si $u$ et $v$ sont déjà linéairement indépendants, alors le produit vectoriel permet de compléter une base de $\eR^3$.

À l'aide du produit vectoriel et du produit scalaire, nous construisons le \defe{produit mixte}{produit!mixte} de trois vecteurs de $\eR^3$ par la formule
\begin{equation}
	(u\times v)\cdot w=\begin{vmatrix}
			u_1	&	u_2	&	u_3	\\
			v_1	&	v_2	&	v_3	\\
			w_1	&	w_2	&	w_3	
	\end{vmatrix}.
\end{equation}

Pourquoi nous ne considérons pas la combinaison $(u\cdot v)\times w$ ?

\begin{proposition}		 \label{PropScalMixtLin}
	Les applications produit scalaire, vectoriel et mixte sont multilinéaires. Spécifiquement, nous avons les propriétés suivantes.
	\begin{enumerate}
		\item
			Les applications produit scalaire et vectoriel sont bilinéaires. Le produit mixte est trilinéaire.
		\item
			Le produit vectoriel est antisymétrique, c'est à dire $u\times v=-v\times u$.
		\item
			Nous avons $u\times v=0$ si et seulement si $u$ et $v$ sont colinéaires, c'est à dire si et seulement si l'équation $\alpha u+\beta v=0$ a une solution différente de la solution triviale $(\alpha,\beta)=(0,0)$.
		\item		\label{ItemPropScalMixtLiniv}
			Pour tout $u$ et $v$ dans $\eR^3$, nous avons
			\begin{equation}
				\langle u, v\rangle^2 +\| u\times v \|^2=\| u \|^2\| v \|^2
			\end{equation}
		\item
			Par rapport à la dérivation, le produit scalaire et vectoriel vérifient une règle de Leibnitz. Soit $I$ un intervalle de $\eR$, et si $u$ et $u$ sont dans $C^1(I,\eR^3)$, alors
			\begin{equation}		\label{EqFormLeibProdscalVect}
				\begin{aligned}[]
					\frac{ d }{ dt }\big( u(t)\cdot v(t) \big)&=\big( u'(t)\cdot v(t) \big)+\big( u(t)\cdot v'(t) \big)\\
					\frac{ d }{ dt }\big( u(t)\times v(t) \big)&=\big( u'(t)\times v(t) \big)+\big( u(t)\times v'(t) \big).
				\end{aligned}
			\end{equation}
		\end{enumerate}
\end{proposition}

Les deux formules suivantes, qui mêlent le produit scalaire et le produit vectoriel, sont souvent utiles en analyse vectorielle :
\begin{equation}
	\begin{aligned}[]
		(u\times v)\cdot w&=u\cdot(v\times w)\\
		(u\times v)\times w&=-(v\cdot w)u+(u\cdot w)v		\label{EqFormExpluxxx}
	\end{aligned}
\end{equation}
pour tout vecteurs $u$, $v$ et $w$ dans $\eR^3$. Nous les admettons sans démonstration. La seconde formule est parfois appelée \defe{formule d'expulsion}{formule!d'expulsion (produit vectoriel)}.

%+++++++++++++++++++++++++++++++++++++++++++++++++++++++++++++++++++++++++++++++++++++++++++++++++++++++++++++++++++++++++++
\section{Dualité}
%+++++++++++++++++++++++++++++++++++++++++++++++++++++++++++++++++++++++++++++++++++++++++++++++++++++++++++++++++++++++++++

\begin{definition}
    Si \( E\) est un espace vectoriel, le \defe{dual}{dual} de \( E\) est l'ensemble des formes linéaires sur \( E\). Le \defe{dual topologique}{dual!topologique} est l'ensemble des formes linéaires continues.

    En dimension infinies, ces deux notions ne coïncident pas.
\end{definition}
%TODO : trouver un exemple où ça ne coïncide pas.
% Si je me souviens bien, pour les opérateurs linéaires, borné est équivalent à continu. Il faudra chercher de ce côté.

%---------------------------------------------------------------------------------------------------------------------------
\subsection{Orthogonal}
%---------------------------------------------------------------------------------------------------------------------------

Soit \( E\), un espace vectoriel, et \( F\) une sous-espace de \( E\). L'\defe{orthogonal}{orthogonal!sous-espace} de \( F\) est la partie \( F^{\perp}\subset E^*\) donnée par
\begin{equation}    \label{Eqiiyple}
    F^{\perp}=\{ \alpha\in E^*\tq \forall x\in F,\alpha(x)=0 \}.
\end{equation}
Cette définition d'orthogonal via le dual n'est pas du pur snobisme. En effet, la définition «usuelle» qui ne parle pas de dual,
\begin{equation}
    F^{\perp}=\{ y\in E\tq \forall x\in F,y\cdot x=0 \},
\end{equation}
demande la donnée d'un produit scalaire. Évidemment dans le cas de \( \eR^n\) munie du produit scalaire usuel et de l'identification usuelle entre \( \eR^n\) et \( (\eR^n)^*\) via une base, les deux notions d'orthogonal coïncident.

Si \( B\subset E^*\), on note \( B^o\)\nomenclature[G]{\( B^o\)}{orthogonal dans le dual} son orthogonal :
\begin{equation}
    B^o=\{ x\in E\tq \omega(x)=0\,\forall \omega\in B \}.
\end{equation}
Notons qu'on le note \( B^o\) et non \( B^{\perp}\) parce qu'on veut un peu s'abstraire du fait que \( (E^*)^*=E\). Du coup on impose que \( B\) soit dans un dual et on prend une notation précise pour dire qu'on remonte au pré-dual et non qu'on va au dual du dual.

La définition \eqref{Eqiiyple} est intrinsèque : elle ne dépend que de la structure d'espace vectoriel.

\begin{proposition} \label{PropXrTDIi}
    Soit \( E\) un espace vectoriel, et \( F\) un sous-espace vectoriel. Alors nous avons
    \begin{equation}
        \dim F+\dim F^{\perp}=\dim E.
    \end{equation}
\end{proposition}

\begin{proof}
    Soit \( \{ e_1,\ldots, e_p \}\) une base de \( F\) que nous complétons en une base \( \{ e_1,\ldots, e_n \}\) de \( E\) par le théorème \ref{ThonmnWKs}. Soit \( \{ e_1^*,\ldots, e^*_n \}\) la base duale. Alors nous prouvons que \( \{ e^*_{p+1},\ldots, e_n^* \}\) est une base de \( F^{\perp}\). 
    
    Déjà c'est une partie libre en tant que partie d'une base.

    Ensuite ce sont des éléments de \( F^{perp}\) parce que si \( i\leq p\) et si \( k\geq 1\) nous avons \( e^*_{p+k}(e_i^*)=0\); donc oui, \( e^*_{p+k}\in F^{\perp}\).

    Enfin \( F^{\perp}\subset\Span\{ e_{p+1}^*,e_n^* \}\) parce que si \( \omega=\sum_{k=1}^n\omega_ke_k^*\), alors \( \omega(e_i)=\omega_i\), mais nous savons que si \( \omega\in F^{\perp}\), alors \( \omega(e_i)=0\) pour \( i\leq p\). Donc \( \omega=\sum_{i=p+1}^n\omega_ke^*_k\).
\end{proof}

%---------------------------------------------------------------------------------------------------------------------------
\subsection{Transposée}
%---------------------------------------------------------------------------------------------------------------------------

Si \( f\colon E\to F\) est une application linéaire entre deux espaces vectoriels, la \defe{transposée}{transposée} est l'application \( f^t\colon F^*\to E^*\) donnée par
\begin{equation}
    f^t(\omega)(x)=\omega\big( f(x) \big).
\end{equation}
pour tout \( \omega\in F^*\) et \( x\in E\).

\begin{lemma}
    Soit \( E\) muni de la base \( \{ e_i \}\) et \( F\) muni de la base \( \{ g_i \}\) et une application \( f\colon E\to F\). Si \( A\) est la matrice de \( f\) dans ces bases, alors \( A^t\) est la matrice de \( f^t\) dans les bases \( \{ e^*_i \}\) et \( \{ g^*_i \}\) de \( E^*\) et \( F^*\).
\end{lemma}

\begin{proof}
    Nous allons montrer que les formes \( f^t(g^*_i)\) et \( \sum_k(A^t)_{ik}g^*_k\) sont égales en les appliquant à un vecteur.

    Par définition de la matrice d'une application linéaire dans une base,
    \begin{equation}
        f^t(g_i^*)=\sum_j(f^t)_{ij}e^*j,
    \end{equation}
    et
    \begin{equation}
        f(e_k)=\sum_lA_{klg_l}.
    \end{equation}
    Du coup, si \( x=\sum_kx_ke_k\), nous avons
    \begin{equation}    \label{EqCzwftH}
        f^t(g_i^*)x=\sum_{kl}x_kg_i^*A_{kl}g_l=\sum_{kl}x_kA_{kl}\delta_{il}=\sum_k x_kA_{ki}=\sum_k(A^t)_{ik}x_k.
    \end{equation}
    D'autre part, 
    \begin{equation}    \label{EqWlQlrR}
        \sum_k(A^t)_{ik}g_k^*x=\sum_{kl}(A^t)_{ik}g^*_kx_le_l=\sum_k(A^t)_{ik}x_k.
    \end{equation}
    Le fait que \eqref{EqCzwftH} et \eqref{EqWlQlrR} donnent le même résultat prouve le lemme.
\end{proof}
En corollaire, les rangs de \( f\) et de \( f^t\) sont égaux parce que le rang est donné par la plus grande matrice carré de déterminant non nul. Nous prouvons cependant ce résultat de façon plus intrinsèque.

\begin{lemma}[\href{http://gilles.dubois10.free.fr/algebre_lineaire/dualite.html}{Gilles Dubois}]   \label{LemSEpTcW}
    Si \( f\colon E\to F\) est une application linéaire, alors
    \begin{equation}
        \rang(f)=\rang(f^t).
    \end{equation}
\end{lemma}

\begin{proof}
    Nous posons \( \dim\ker(f)=p\) et donc \( \rang(f)=n-p\). Soit \( \{ e_1,\ldots, e_p \}\) une base de \( \ker(f)\) que l'on complète en une base \( \{ e_1,\ldots, e_n \}\) de \( E\). Nous considérons maintenant les vecteurs
    \begin{equation}
        g_i=f(e_{p+i})
    \end{equation}
    pour \( i=1,\ldots, n-p\). C'est à dire que les \( g_i\) sont les images des vecteurs qui ne sont pas dans le noyau de \( f\). Prouvons qu'ils forment une famille libre. Si
    \begin{equation}
        \sum_{k=1}^{n-p}a_kf(e_{p+k})=0,
    \end{equation}
    alors \( f\big( \sum_ka_ke_{p+k} \big)=0\), ce qui signifierait que \( \sum_ka_ke_{p+k}\) se trouve dans le noyau de \( f\), ce qui est impossible par construction de la base \( \{ e_i \}_{i=1,\ldots, n}\). Étant donné que les vecteurs \( g_1,\ldots, g_{n-p}\) sont libres, nous les complétons en une base
    \begin{equation}
        \{ \underbrace{g_1,\ldots, g_{n-p}}_{\text{images}},\underbrace{g_{n-p+1},\ldots, g_r}_{\text{complétion}} \}
    \end{equation}
    de \( F\).

    Nous prouvons maintenant que \( \rang(f^t)\geq n-p\) en montrant que les formes \( \{ g_i^* \}_{i=1,\ldots, n-p}\) est libre (et donc l'espace image de \( f^f\) est au moins de dimension \( n-p\)). Pour cela nous prouvons que \( f^t(g_i^*)=e^*_{i+p}\). En effet
    \begin{equation}
        f^t(g^*_i)e_k=g_i^*(fe_k),
    \end{equation}
    Si \( k=1,\ldots, p\), alors \( fe_k=0\) et donc \( g_i^*(fe_k)=0\); si \( k=p+l\) alors
    \begin{equation}
        f^t(g_i^*)e_k=g_i^*(fe_{k+l})=g^*_i(g_l)=\delta_{i,l}=\delta_{i,k-p}=\delta_{k,i+p}.
    \end{equation}
    Donc \( f^t(g_i^*)=e^*_{i+p}\). Cela prouve que les formes \( f^t(g_i^*)\) sont libres et donc que
    \begin{equation}
        \rang(f^t)\geq n-p=\rang(f).
    \end{equation}
    En appliquant le même raisonnement à \( f^t\) au lieu de \( f\), nous trouvons
    \begin{equation}
        \rang\big( (f^t)^t \big)\geq \rang(f^t)
    \end{equation}
    et donc, vu que \( (f^t)^t=f\), nous obtenons \( \rang(f)=\rang(f^t)\).
    
\end{proof}

\begin{proposition}[\cite{DualMarcSAge}]
    Si \( f\) est une application linéaire entre les espaces vectoriels \( E\) et \( F\), alors nous avons
    \begin{equation}
        \Image(f^t)=\ker(f)^{\perp}.
    \end{equation}
\end{proposition}

\begin{proof}
    Soient donc l'application \( f\colon E\to F\) et sa transposée \( f^t\colon F^*\to E^*\). Nous commençons par prouver que \( \Image(f^{t})\subset(\ker f)^{\perp}\). Pour cela nous prenons \( \omega\in \Image(f^t)\), c'est à dire \( \omega=\alpha\circ f\) pour un certain élément \( \alpha\in F^*\). Si \( z\in\ker(f)\), alors \( \omega(z)=(\alpha\circ f)(z)=0\), c'est à dire que \( \omega\in (\ker f)^{\perp}\).

    Pour prouver qu'il y a égalité, nous n'allons pas démontrer l'inclusion inverse, mais plutôt prouver que les dimensions sont égales. Après, on sait que si \( A\subset B\) et si \( \dim A=\dim B\), alors \( A=B\). Nous avons
    \begin{subequations}
        \begin{align}
            \dim\big( \Image(f^t) \big)&=\rang(f^t)\\
            &=\rang(f)  &\text{lemme \ref{LemSEpTcW}}\\
            &=\dim(E)-\dim\ker(f)   &\text{théorème \ref{ThoGkkffA}}\\
            &=\dim\big( (\ker f)^{\perp} \big)  &\text{proposition \ref{PropXrTDIi}}.
        \end{align}
    \end{subequations}
\end{proof}


%---------------------------------------------------------------------------------------------------------------------------
\subsection{Polynômes de Lagrange}
%---------------------------------------------------------------------------------------------------------------------------

Soit \( E=\eR_n[X]\) l'ensemble des polynômes à coefficients réels de degré au plus \( n\). Soient les \( n+1\) réels distincts \( a_0,\ldots, a_n\). Nous considérons les formes linéaires associées \( f_i\in E^*\),
\begin{equation}
    f_i(P)=P(a_i).
\end{equation}
\begin{lemma}
    Ces formes forment une base de \( E^*\).
\end{lemma}

\begin{proof}
    Nous prouvons que l'orthogonal est réduit au nul :
    \begin{equation}
        \Span\{ f_0,\ldots, f_n \}^{\perp}=\{ 0 \}
    \end{equation}
    pour que la proposition \ref{PropXrTDIi} conclue. Si \( P\in\Span\{ f_i \}^{\perp}\), alors \( f_i(P)=0\) pour tout \( i\), ce qui fait que \( P(a_i)=0\) pour tout \( i=0,\ldots, n\). Un polynôme de degré au plus \( n\) qui s'annule en \( n+1\) points est automatiquement le polynôme nul.
\end{proof}

Les \defe{polynômes de Lagrange}{Lagrange!polynôme}\index{polynôme!Lagrange} sont les polynômes de la base (pré)duale de la base \( \{ f_i \}\).

\begin{proposition}
    Les polynômes de Lagrange sont donnés par
    \begin{equation}
        P_i=\prod_{k\neq i}\frac{ X-a_k }{ a_i-a_k }.
    \end{equation}
\end{proposition}

\begin{proof}
    Il suffit de vérifier que \( f_j(P_i)=\delta_{ij}\). Nous avons
    \begin{equation}
        f_j(P_i)=P_i(a_j)=\prod_{k\neq i}\frac{ a_j-a_k }{ a_i-a_k }.
    \end{equation}
    Si \( j\neq i\) alors un des termes est nul. Si au contraire \( i=j\), tous les termes valent \( 1\).
\end{proof}

%+++++++++++++++++++++++++++++++++++++++++++++++++++++++++++++++++++++++++++++++++++++++++++++++++++++++++++++++++++++++++++
\section{Déterminants}
%+++++++++++++++++++++++++++++++++++++++++++++++++++++++++++++++++++++++++++++++++++++++++++++++++++++++++++++++++++++++++++
\label{SecGYzHWs}

%  Lire http://www.les-mathematiques.net/phorum/read.php?2,302266

\begin{definition}
    Soit \( E\), un \( \eK\)-espace vectoriel. Une forme linéaire \defe{alternée}{forme linéaire!alternée}\index{alternée!forme linéaire} sur \( E\) est une application linéaire \( f\colon E\to \eK\) telle que \( f(v_1,\ldots, v_k)=0\) dès que \( v_i=v_j\) pour certains \( i\neq j\).
\end{definition}

\begin{lemma}   \label{LemHiHNey}
    Une forme linéaire alternée est antisymétrique. Si \( \eK\) est de caractéristique différente de \( 2\), alors une forme antisymétrique est alternée.
\end{lemma}

\begin{proof}
    Soit \( f\) une forme alternée; quitte à fixer toutes les autres variables, nous pouvons travailler avec une \( 2\)-forme et simplement montrer que \( f(x,y)=-f(y,x)\). Pour ce faire nous écrivons
    \begin{equation}
        0=f(x+y,x+y)=f(x,x)+f(x,y)+f(y,x)+f(y,y)=f(x,y)+f(y,x).
    \end{equation}
    
    Pour la réciproque, si \( f\) est antisymétrique, alors \( f(x,x)=-f(x,x)\). Cela montre que \( f(x,x)=0\) lorsque \( \eK\) est de caractéristique différente de deux.
\end{proof}

\begin{proposition}[\href{http://www.les-mathematiques.net/b/e/d/node5.php}{les-mathematiques.net}] \label{ProprbjihK}
    Soit \( E\), un \( \eK\)-espace vectoriel de dimension \( n\), où la caractéristique de \( \eK\) n'est pas deux. L'espace des \( n\)-formes multilinéaires alternées sur \( E\) est de \( \eK\)-dimension \( 1\).
\end{proposition}

\begin{proof}
    Soit \( \{ e_i \}\), une base de \( E\) et \( f\colon E\to \eK\) une \( n\)-forme linéaire alternée, puis \( (v_1,\ldots, v_n)\) des vecteurs de \( E\). Nous pouvons les écrire dans la base
    \begin{equation}
        v_j=\sum_{i=1}^n\alpha_{ij}e_i
    \end{equation}
    et alors exprimer \( f\) par
    \begin{subequations}
        \begin{align}
            f(v_1,\ldots, v_n)&=f\big( \sum_{i_1=1}^n\alpha_{1i_1}e_{i_1},\ldots, \sum_{i_n=1}^n\alpha_{ni_n}e_{i_n} \big)\\
            &=\sum_{i,j}\alpha_{1i_1}\ldots \alpha_{ni_n}f(e_{i_1},\ldots, e_{i_n}).
        \end{align}
    \end{subequations}
    Étant donné que \( f\) est alternée, les seuls termes de la somme sont ceux dont les \( i_k\) sont tous différents, c'est à dire ceux où \( \{ i_1,\ldots, i_n \}=\{ 1,\ldots, n \}\). Il y a donc un terme par élément du groupe des permutations \( S_n\) et
    \begin{equation}
        f(v_1,\ldots, v_n)=\sum_{\sigma\in S_n}\alpha_{\sigma(1)1}\ldots \alpha_{\sigma(n)n}f(e_{\sigma(1)},\ldots, e_{\sigma(n)}).
    \end{equation}
    En utilisant encore une fois le fait que la forme \( f\) soit alternée, \( f=f(e_1,\ldots, e_n)\Pi\) où
    \begin{equation}
        \Pi(v_1,\ldots, v_n)=\sum_{\sigma\in S_n}\epsilon(\sigma)\alpha_{\sigma(1)1}\ldots \alpha_{\sigma(n)n}.
    \end{equation}
    Pour rappel, la donnée des \( v_i\) est dans les nombres \( \alpha_{ij}\).
    
    L'espace des \( n\)-formes alternées est donc \emph{au plus} de dimension \( 1\). Pour montrer qu'il est exactement de dimension \( 1\), il faut et suffit de prouver que \( \Pi\) est alternée. Par le lemme \ref{LemHiHNey}, il suffit de prouver que cette forme est antisymétrique\footnote{C'est ici que joue l'hypothèse sur la caractéristique de \( \eK\).}. 

    Soient donc \( v_1,\ldots, v_n\) tels que \( v_i=v_j\). En posant \( \tau=(1i)\) et \( \tau'=(2j)\) et en sommant sur \( \sigma\tau\tau'\) au lieu de \( \sigma\), nous pouvons supposer que \( i=1\) et \( j=2\). Montrons que \( \Pi(v,v,v_3,\ldots, v_n)=0\) en tenant compte que \( \alpha_{i1}=\alpha_{i2}\) :
    \begin{subequations}
        \begin{align}
            \Pi(v,v,v_3,\ldots, v_n)&=\sum_{\sigma\in S_n}\epsilon(\sigma)\alpha_{\sigma(1)1}\alpha_{\sigma(2)2}\alpha_{\sigma(3)3}\ldots \alpha_{\sigma(n)n}\\
            &=\sum_{\sigma\in S_n}\epsilon(\sigma\tau)\alpha_{\sigma\tau(1)1}\alpha_{\sigma\tau(2)2}\alpha_{\sigma\tau(3)3}\ldots \alpha_{\sigma\tau(n)n}&\text{où \( \tau=(12)\)}\\
            &=-\sum_{\sigma\in S_n}\epsilon(\sigma)\alpha_{\sigma(1)1}\alpha_{\sigma(2)2}\alpha_{\sigma(3)3}\ldots \alpha_{\sigma(n)n} \\
            &=-\Pi(v,v,v_3,\ldots, v_n).
        \end{align}
    \end{subequations}
\end{proof}

\begin{proposition}[\cite{LoFdlw}]
    Soit \( n\geq 3\) et \( \eK\) un corps de caractéristique différente de \( 2\). Alors
    \begin{enumerate}
        \item
            le groupe dérivé de \( D(\GL(n,\eK))\) est \(\SL(n,\eK)\);  \index{groupe!dérivé!de \( \GL(n,\eK)\)}
        \item
            le groupe dérivé de \( \SL(n,\eK)\) est \( \SL(n,\eK)\).\index{groupe!dérivé!de \( \SL(n,\eK)\)}
    \end{enumerate}
\end{proposition}
La preuve utilise le fait que les transvections engendrent \( \SL(n,\eK)\) et que les transvections avec les dilatations engendrent \( \GL(n,\eK)\).

\begin{lemma}   \label{LemcDOTzM}
    Soit \( \eK\) un corps fini autre que \( \eF_2\)\footnote{Je ne comprends pas très bien à quel moment joue cette hypothèse.}, soit un groupe abélien \( M\) et un morphisme \( \varphi\colon \GL(n,\eK)\to M\). Alors il existe un unique morphisme \( \delta\colon \eK^*\to M\) tel que \( \varphi=\delta\circ\det\).
\end{lemma}

\begin{proof}
    D'abord le groupe dérivé de \( \GL(n,\eK)\) est \( \SL(n,\eK)\) parce que les éléments de \( D\big( \GL(n,\eK) \big)\) sont de la forme \( ghg^{-1}h^{-1}\) dont le déterminant est \( 1\).
    
    
    De plus le groupe \( \SL(n,\eK)\) est normal dans \( \GL(n,\eK)\). Par conséquent \( \GL(n,\eK)/\SL(n,\eK)\) est un groupe et nous pouvons définir l'application relevée
    \begin{equation}
        \tilde \varphi\colon \frac{ \GL(n,\eK) }{ \SL(n,\eK) }\to M
    \end{equation}
    vérifiant \( \varphi=\tilde \varphi\circ\pi\) où \( \pi\) est la projection. 

    Nous pouvons faire la même chose avec l'application
    \begin{equation}
        \det\colon \GL(n,\eK)\to \eK^*
    \end{equation}
    qui est un morphisme de groupes dont le noyau est \( \SL(n,\eK)\). Cela nous donne une application
    \begin{equation}
        \tilde \det\colon \frac{ \Gl(n,\eK) }{ \SL(n,\eK) }\to \eK^*
    \end{equation}
    telle que \( \det=\tilde \det\circ\pi\). Cette application \( \tilde \det\) est un isomorphisme. En effet elle est surjective parce que le déterminant l'est et elle est injective parce que son noyau est précisément ce par quoi on prend le quotient. Par conséquent \( \tilde \det \) possède un inverse et nous pouvons écrire
    \begin{equation}
        \varphi=\tilde \varphi\circ\tilde \det^{-1}\circ\tilde \det\circ\pi.
    \end{equation}
    État donné que \( \tilde \det\circ\pi=\det\), nous avons alors \( \varphi=\delta\circ\det\) avec \( \delta=\tilde \varphi\circ\tilde \det^{-1}\). Ceci conclu la partie existence de la preuve.

    En ce qui concerne l'unicité, nous considérons \( \delta'\colon \eK^*\to M\) telle que \( \varphi=\delta'\circ\det\). Pour tout \( u\in \GL(n,\eK)\) nous avons \( \delta'(\det(u))=\varphi(u)=\delta(\det(u))\). L'application \( \det\) étant surjective depuis \( \GL(n,\eK)\) vers \( \eK^*\), nous avons \( \delta'=\delta\).
\end{proof}

\begin{theorem}
    Soit \( p\geq 3\) un nombre premier et \( V\), un \( \eF_p\)-espace vectoriel de dimension finie \( n\). Pour tout \( u\in\GL(V)\) nous avons
    \begin{equation}
        \epsilon(u)=\left(\frac{\det(u)}{p}\right).
    \end{equation}
\end{theorem}
Ici \( \epsilon\) est la signature de \( u \) vue comme une permutation des éléments de \( \eF_p\).

\begin{proof}
    Commençons par prouver que
    \begin{equation}
        \epsilon\colon \GL(V)\to \{ -1,1 \}.
    \end{equation}
    est un morphisme. Si nous notons \( \bar u\in S(V)\) l'élément du groupe symétrique correspondant à la matrice \( u\in \GL(V)\), alors nous avons \( \overline{ uv }=\bar u\circ\bar v\), et la signature étant un homomorphisme (proposition \ref{ProphIuJrC}), 
    \begin{equation}
        \epsilon(uv)=\epsilon(\bar u\circ\bar v)=\epsilon(\bar u)\epsilon(\bar v).
    \end{equation}
    Par ailleurs \( \{ -1,1 \}\) est abélien, donc le lemme \ref{LemcDOTzM} s'applique et nous pouvons considérer un morphisme \( \delta\colon \eF_p^*\to \{ -1,1 \}\) tel que \( \epsilon=\delta\circ\det\).

    Nous allons utiliser le lemme \ref{Lemoabzrn} pour montrer que \( \delta\) est le symbole de Legendre. Pour cela il nous faudrait trouver un \( x\in \eF_p^*\) tel que \( \delta(x)=-1\). Étant donné que \( \det\) est surjective, nous cherchons ce \( x\) sous la forme \( x=\det(u)\). Par conséquent nous aurions
    \begin{equation}
        \delta(x)=(\delta\circ\det)(u)=\epsilon(u),
    \end{equation}
    et notre problème revient à trouver une matrice \( u\in\GL(V)\) dont la permutation associée soit de signature \( -1\).

    Soit \( n=\dim V\); en conséquence de la proposition \ref{PropHfrNCB}\ref{ItemiEFRTg}, l'espace \( \eE_q=\eF_{p^n}\) est un \( \eF_p\)-espace vectoriel de dimension \( n\) et est donc isomorphe en tant qu'espace vectoriel à \( V\). Étant donné que \( \eF_q\) est un corps fini, nous savons que \( \eF_q^*\) est un groupe cyclique à \( q-1\) éléments. Soit \( y\), un générateur de \( \eF_q^*\) et l'application
    \begin{equation}
        \begin{aligned}
            \beta\colon \eF_q&\to \eF_q \\
            x&\mapsto yx. 
        \end{aligned}
    \end{equation}
    Cela est manifestement \( \eF_p\)-linéaire (ici \( y\) et \( x\) sont des classes de polynômes et \( \eF_p\) est le corps des coefficients). L'application \( \beta\) fixe zéro et à part zéro, agit comme le cycle
    \begin{equation}
        (1,y,y^2,\ldots, y^{q-2}).
    \end{equation}
    Nous savons qu'un cycle de longueur \( n\) est de signature \( (-1)^{n+1}\). Ici le cycle est de longueur \( q-1\) qui est pair (pare que \( p\geq 3\)) et par conséquent, l'application \( \beta\) est de signature \( -1\).
\end{proof}

%---------------------------------------------------------------------------------------------------------------------------
\subsection{Déterminant de Vandermonde}
%---------------------------------------------------------------------------------------------------------------------------

\begin{proposition}[\cite{fJhCTE}]  \label{PropnuUvtj}
    Le \defe{déterminant de Vandermonde}{déterminant!Vandermonde}\index{Vandermonde (déterminant)} est le polynôme en \( n\) variables donné par
    \begin{equation}
        V(T_1,\ldots, T_n)=\det\begin{pmatrix}
             1   &   1    &   \ldots    &   1    \\
             T_1   &   T_2    &   \ldots    &   T_n    \\
             \vdots   &   \ddots    &   \ddots    &   \vdots    \\ 
             T_1^{n-1}   &   T_2^{n-1}    &   \ldots    &   T_n^{n-1}     
         \end{pmatrix}=\prod_{1\leq i<j\leq n}(T_j-T_i).
    \end{equation}
    Notez que l'inégalité du milieu est stricte (sinon d'ailleurs l'expression serait nulle).
\end{proposition}
Le déterminant de Vandermonde est entre autres utilisé pour prouver que \( \tr(u^p)=0\) pour tout \( p\) si et seulement si \( u\) est nilpotente (lemme \ref{LemzgNOjY}).

\begin{proof}
    Nous considérons le polynôme
    \begin{equation}
        f(X)=V(T_1,\ldots, T_{n-1},X)\in \big( \eK[T_1,\ldots, T_{n-1}] \big)[X].
    \end{equation}
    C'est un polynôme de degré au plus \( n-1\) en \( X\) et il s'annule aux points \( T_1,\ldots, T_{n-1}\). Par conséquent il existe \( \alpha\in \eK[T_1,\ldots, T_{n-1}]\) tel que
    \begin{equation}    \label{EqeVxRwO}
        f=\alpha(X-T_{n-1})\ldots(X-T_1).
    \end{equation}
    Nous trouvons \( \alpha\) en écrivant \( f(0)\). D'une part la formule \eqref{EqeVxRwO} nous donne
    \begin{equation}    \label{EqblwWMj}
        f(0)=\alpha(-1)^{n-1}T_1\ldots T_{n-1}.
    \end{equation}
    D'autre par la définition donne
    \begin{subequations}
        \begin{align}
            f(0)&=\det\begin{pmatrix}
                 1   &   \cdots    &   1    &   1    \\
                 T_1      &       &   T_{n-1}    &   0    \\
                 \vdots   &       &   \vdots    &   \vdots    \\ 
                 T_1^{n-1}   &   \cdots    &   T_{n-1}^{n-1}    &   0     
             \end{pmatrix}\\
             &=(-1)^{n-1}\det\begin{pmatrix}
                 T_1   &   \ldots    &   T_{n-1}    \\
                 \vdots   &   \ddots    &   \vdots    \\
                 T_1^{n-1}   &   \ldots    &   T_{n-1}^{n-1}
             \end{pmatrix}\\
             &=(-1)^{n-1}T_1\ldots T_{n-1}\det\begin{pmatrix}
                 1   &   \cdots    &   1    \\
                 \vdots   &   \ddots    &   \vdots    \\
                 T_1^{n-1}   &   \cdots    &   T_{n-1}^{n-1}
             \end{pmatrix}\\
             &=(-1)^{n-1}T_1\ldots T_{n-1}V(T_1,\ldots, T_{n-1})
        \end{align}
    \end{subequations}
    En égalisant avec \eqref{EqblwWMj}, nous trouvons \( \alpha=V(T_1,\ldots, T_{n-1})\), et donc
    \begin{equation}
        f=V(T_1,\ldots, T_{n-1})\prod_{j\leq n-1}(X-T_j)
    \end{equation}
    Enfin, une récurrence montre que
    \begin{subequations}
        \begin{align}
            V(T_1,\ldots, T_n)&=f(T_n)\\
            &=V(T_1,\ldots, T_{n-1})\prod_{j\leq n-1}(T_n-T_j)\\
            &=\prod_{k\leq n}\prod_{j\leq k-1}(T_k-T_j)\\
            &=\prod_{1\leq j<k\leq n}(T_i-T_j).
        \end{align}
    \end{subequations}
\end{proof}

%---------------------------------------------------------------------------------------------------------------------------
\subsection{Déterminant de Gram}
%---------------------------------------------------------------------------------------------------------------------------

Si \( x_1,\ldots, x_r\) sont des vecteurs d'un espace vectoriel, alors le \defe{déterminant de Gram}{déterminant!Gram}\index{Gram (déterminant)} est le déterminant
\begin{equation}
    G(x_1,\ldots, x_r)=\det\big( \langle x_i, x_j\rangle  \big).
\end{equation}
Notons que la matrice est une matrice symétrique.

\begin{proposition}\label{PropMsZhIK}
    Si \( F\) est un sous-espace vectoriel de base \( \{ x_1,\ldots, x_n \}\) et si \( x\) est un vecteur, alors le déterminant de Gram est un moyen de calculer la distance entre \( x\) et \( F\) par 
    \begin{equation}
        d(x,F)^2=\frac{ G(x,x_1,\ldots, x_n)}{ G(x_1,\ldots, x_n) }.
    \end{equation}
\end{proposition}

%---------------------------------------------------------------------------------------------------------------------------
\subsection{Déterminant de Cauchy}
%---------------------------------------------------------------------------------------------------------------------------

Soient des nombres \( a_i\) et \( b_i\) (\( i=1,\ldots, n\)) tels que \( a_i+b_j\neq 0\) pour tout couple \( (i,j)\). Le \defe{déterminant de Cauchy}{déterminant!de Cauchy}\index{Cauchy!déterminant} est 
\begin{equation}
    D_n=\det\left( \frac{1}{ a_i+b_j } \right).
\end{equation}

\begin{proposition}[\cite{RollandRobertjyYDzY}] \label{ProptoDYKA}
    Le déterminant de Cauchy est donné par la formule
    \begin{equation}
        D_n=\frac{ \prod_{i<j}(a_j-a_i)\prod_{i<j}(b_j-b_i) }{ \prod_{ij}(a_i+b_j) }.
    \end{equation}
\end{proposition}

%---------------------------------------------------------------------------------------------------------------------------
\subsection{Matrice de Sylvester}
%---------------------------------------------------------------------------------------------------------------------------
\label{subsecSQBJfr}

La définition est pompée de \wikipedia{fr}{Matrice_de_Sylvester}{wikipédia}.

Soient \( P\) et \( Q\) deux polynômes non nuls, de degrés respectifs \( m\) et \( n\) :
\begin{subequations}
    \begin{align}
        P(x)=p_0+p_1x+\ldots +p_mx^m\\
        Q(x)=q_0+q_1x+\ldots +q_mx^m.
    \end{align}
\end{subequations}
La \defe{matrice de Sylvester}{matrice!de Sylvester}\index{Sylvester (matrice)} associée à \( P\) et \( Q\) est la matrice carrée \( m+n\times m+n\) définie ainsi :
\begin{enumerate}
    \item
la première ligne est formée des coefficients de \( P\), suivis de 0 :
\begin{equation}
\begin{pmatrix} p_m & p_{m-1} & \cdots & p_1 & p_0 & 0 & \cdots & 0 \end{pmatrix} ;
\end{equation}
\item la seconde ligne s'obtient à partir de la première par permutation circulaire vers la droite ;
\item les $(n-2)$ lignes suivantes s'obtiennent en répétant la même opération ;
\item la ligne $(n+1)$ est formée des coefficients de \( Q\), suivis de 0 :
    \begin{equation}
    \begin{pmatrix} q_n & q_{n-1} & \cdots & q_1 & q_0 & 0 & \cdots & 0 \end{pmatrix} ;
    \end{equation}
    \item les $(m-1)$ lignes suivantes sont formées par des permutations circulaires.
\end{enumerate}

Ainsi dans le cas $m=4$ et $n=3$, la matrice obtenue est
\begin{equation}    \label{EqPEgtle}
S_{p,q}=\begin{pmatrix} 
p_4 & p_3 & p_2 & p_1 & p_0 & 0 & 0 \\
0 & p_4 & p_3 & p_2 & p_1 & p_0 & 0 \\
0 & 0 & p_4 & p_3 & p_2 & p_1 & p_0 \\
q_3 & q_2 & q_1 & q_0 & 0 & 0 & 0 \\
0 & q_3 & q_2 & q_1 & q_0 & 0 & 0 \\
0 & 0 & q_3 & q_2 & q_1 & q_0 & 0 \\
0 & 0 & 0 & q_3 & q_2 & q_1 & q_0 \\
\end{pmatrix}.
\end{equation}
Le déterminant de la matrice de Sylvester associée à \( P\) et \( Q\) est appelé le \defe{résultant}{résultant} de \( P\) et \( Q\) et noté \( R(P,Q)\)\nomenclature[A]{\( R(P,Q)\)}{résultat des polynômes \( P\) et \( Q\)}.

L'équation de Bézout \eqref{EqkbbzAi} peut être traitée avec une matrice de Sylvester. Soient \( P\) et \( Q\), deux polynômes donnés et à résoudre l'équation 
\begin{equation}    \label{EqSsyXOo}
    xP+yQ=0
\end{equation}
par rapport aux polynômes inconnus \( x\) et \( y\) dont les degrés sont \( \deg(x)<\deg(Q)\) et \( \deg(y)<\deg(P)\). Si nous notons \( \tilde x\) et \( \tilde y\) la liste des coefficients de \( x\) et \( y\) (dans l'ordre décroissant de degré), nous pouvons récrire l'équation \eqref{EqSsyXOo} sous la forme
\begin{equation}
    S_{PQ}^t\begin{pmatrix}
        \tilde x    \\ 
        \tilde y    
    \end{pmatrix}=0.
\end{equation}
Pour s'en convaincre, écrivons pour les polynômes de l'exemple \eqref{EqPEgtle} :
\begin{equation}
    \begin{pmatrix}
        p_4    &   0    &   0    &   q_3    &   0    &   0    &   0\\ 
        p_3    &   p_4    &   0    &   q_2    &   q_3    &   0    &   0\\ 
        p_2    &   p_3    &   p_4    &   q_1    &   q_2    &   q_3    &   0\\ 
        p_1    &   p_2    &   p_3    &   q_0    &   q_1    &   q_2    &   q_3\\ 
        p_0    &   p_1    &   p_2    &   0    &   q_0    &   q_1    &   q_2\\ 
        0    &   p_0    &   p_1    &   0    &   0    &   q_0    &   q_1\\ 
        0    &   0    &   p_0    &   0    &   0    &   0    &   q_0\\    
    \end{pmatrix}\begin{pmatrix}
        x_2    \\ 
        x_1  \\ 
        x_0  \\    
        y_3   \\ 
        y_2    \\ 
        y_1    \\ 
        y_0    
    \end{pmatrix}=
    \begin{pmatrix}
        x_2p_4+y_2q_3    \\ 
        p_3x_2+p_4x_1+q_2y_3+q_3y_2  \\ 
          \\    
           \\ 
        \vdots    \\ 
            \\ 
            
    \end{pmatrix}
\end{equation}
Nous voyons que sur la ligne numéro \( k\) (en partant du bas et en numérotant de à partir de zéro) nous avons les produits \( p_ix_j\) et \( q_iy_j\) avec \( i+j=k\). La colonne de droite représente donc bien les coefficients du polynôme \( xP+yQ\).


\begin{proposition}
    Le résultat de deux polynômes est non nul si et seulement si les deux polynômes sont premiers entre eux.
\end{proposition}

Un polynôme \( P\) a une racine double en \( a\) si et seulement si \( P\) et \( P'\) sont \( a\) comme racine commune, ce qui revient à dire que \( P\) et \( P'\) ne sont pas premiers entre eux. 

\begin{example}
    Si nous prenons \( P=aX^2+bX+c\) et \( P'=2aX+b\) alors la taille de la matrice de Sylvester sera \( 2+1=3\) et
    \begin{equation}
        S_{P,P'}=\begin{pmatrix}
              a  &   b    &   c    \\
            2a    &   b    &   0    \\
            0    &   2a    &   b
        \end{pmatrix}.
    \end{equation}
    Le résultant est alors
    \begin{equation}
        R(P,P')=-a(b^2-4ac).
    \end{equation}
    Donc un polynôme du second degré a une racine double si et seulement si \( b^2-4ac=0\). Cela est un résultat connu depuis longtemps mais qui fait toujours plaisir à revoir.
\end{example}

La matrice de Sylvester permet aussi de récrire l'équation de Bézout pour les polynômes; voir le théorème \ref{ThoBezoutOuGmLB} et la discussion qui s'ensuit.

%+++++++++++++++++++++++++++++++++++++++++++++++++++++++++++++++++++++++++++++++++++++++++++++++++++++++++++++++++++++++++++++
\section{Applications linéaires}
%+++++++++++++++++++++++++++++++++++++++++++++++++++++++++++++++++++++++++++++++++++++++++++++++++++++++++++++++++++++++++++++

%---------------------------------------------------------------------------------------------------------------------------
\subsection{Définition}
%---------------------------------------------------------------------------------------------------------------------------

\begin{definition}
	Une application $T: \eR^m\to\eR^n$ est dite \defe{linéaire}{linéaire (application)} si 
\begin{itemize}
\item $T(x+y)=T(x)+T(y)$ pour tout $x$ et $y$ dans $\eR^m$,  
\item $T(\lambda x)=\lambda T(x)$ pour tout $\lambda$ dans $\eR^m$ et $\lambda$ dans $\eR$.
\end{itemize}
\end{definition}

Si $V$ et $W$ sont deux espaces vectoriels réels, nous définissons de la même manière une application linéaire de $V$ dans $W$ comme étant une application $T\colon V\to W$ telle que $T(v_1+v_2)=T(v_1)+T(v_2)$ et $T(\lambda v)=\lambda T(v)$ pour tout $v,v_1,v_2$ dans $V$ et pour tout réel $\lambda$.

L'ensemble des applications linéaires de $\eR^m$ vers $\eR^n$ est noté $\mathcal{L}(\eR^m, \eR^n)$, et plus généralement nous notons $\aL(V,W)$\nomenclature{$\aL(V,W)$}{Ensemble des applications linéaires de $V$ dans $W$} l'ensemble des applications linéaires de $V$ dans $W$. 

\begin{example}
Soit $m=n=1$. Pour tout $b$ dans $\eR$ la fonction $T_b(x)= bx$ est une application linéaire de $\eR$ dans $\eR$. En effet,
\begin{itemize}
\item  $T_b(x+y)= b(x+y)= bx + by = T_b(x)+T_b(y)$,
\item $T_b(ax)=b(ax)= abx = a T_b(x)$.
\end{itemize}
De la même façon on peut montrer que la fonction $T_{\lambda}$ définie par $T_{\lambda}(x)=bx$ est un application linéaire de $\eR^m$ dans $\eR^m$ pour tout $\lambda$ dans $\eR$ et $m$ dans $\eN$.
\end{example}

\begin{example}\label{ex_affine}
	Soit $m=n$. On fixe $\lambda$ dans $\eR$ et $v$ dans $\eR^m$. L'application $U_{\lambda}$ de $\eR^m$ dans $\eR^m$ définie par $U_{\lambda}(x)=\lambda x+v$ n'est pas une application linéaire, parce que 
\[
U_{\lambda}(ax)=\lambda(ax)+v\neq \lambda(bx+v)=a U_{\lambda}(x).
\]
\end{example}

\begin{example}\label{exampleT_A}
	Soit $A$ une matrice fixée de $\mathcal{M}_{n\times m}$\nomenclature{$\mathcal{M}_{n\times m}$}{l'ensemble des matrices $n\times m$}. La fonction $T_A\colon \eR^m\to \eR^n$ définie par $T_A(x)=Ax$ est une application linéaire. En effet, 
\begin{itemize}
\item  $T_A(x+y)= A(x+y)= Ax + Ay = T_A(x)+T_A(y)$,
\item $T_A(ax)=A(ax)= a(Ax) = a T_A(x)$.
\end{itemize}
\end{example}

On peut observer que, si on identifie $\mathcal{M}_{1\times 1}$ et $\eR$, on obtient le premier exemple comme cas particulier.

\begin{proposition}
 Toute application linéaire $T$ de $\eR^m$ dans $\eR^n$ s'écrit de manière unique par rapport aux bases canoniques de $\eR^m$ et $\eR^n$ sous la forme
\[
T(x)=Ax,
\]
avec $A$ dans $\mathcal{M}_{n\times m}$.
\end{proposition}

\begin{proof}
  Soit $x$ un vecteur dans $\eR^m$. On peut écrire $x$ sous la forme $ x=\sum_{i=1}^{m}x_i e_i$. Comme $T$ est une application linéaire on a
\[
T(x)=\sum_{i=1}^{m}x_iT(e_i).
\]
Les images de la base de $\eR^m$, $T(e_j), \, j=1,\ldots,m$, sont des éléments de $\eR^n$, donc on peut les écrire sous la forme de vecteurs
\[
T(e_i)=
\begin{pmatrix}
  a_{1i}\\
\vdots\\
a_{ni}
\end{pmatrix}.
\] 
On obtient alors
\[
T(x)=\sum_{i=1}^{m}x_iT(e_i)=\sum_{i=1}^{m}x_i\begin{pmatrix}
  a_{1i}\\
\vdots\\
a_{ni}
\end{pmatrix}=
\begin{pmatrix}
  a_{11} \ldots a_{1m}\\
\vdots \ddots \vdots\\
 a_{n1} \ldots a_{nm}\\
\end{pmatrix}
\begin{pmatrix}
  x_1\\
\vdots\\
x_m
\end{pmatrix}=Ax.
\]
\end{proof}


\begin{definition}
  Une application $S: \eR^m\to\eR^n$ est dite \defe{affine}{affine (application)} si elle est la somme d'une application linéaire et d'une application constante. Autrement dit, $S$ est affine s'il existe $T: \eR^m\to\eR^n$, linéaire, telle que $S(x)-T(x)$ soit un vecteur constant dans $\eR^n$. 
\end{definition}

\begin{example}
	Les exemples les plus courants d'applications affines sont les droites et les plans ne passant pas par l'origine.
	\begin{description}
		\item[Les droites] Une droite dans $\eR^2$ (ou $\eR^3$) qui ne passe pas par l'origine est le graphe d'une fonction de la forme $s(x)=ax+b$ (ou $s(t)=u x +v$, avec $u$ et $v$  dans $\eR^2$). On reconnait ici la fonction de l'exemple \ref{ex_affine}.
			
		\item[Les plans]
			De la même façon nous savons que tout plan qui ne passe pas par l'origine dans $\eR^3$ est le graphe d'une application affine, $P(x,y)= (a,b)^T\cdot(x,y)^T+(c,d)^T$.
	\end{description}
\end{example}

%---------------------------------------------------------------------------------------------------------------------------
\subsection{Décomposition de Bruhat}
%---------------------------------------------------------------------------------------------------------------------------

Nous nommons \( E_{ij}\) la matrice remplie de zéros sauf à la case \( ij\) qui vaut \( 1\). Autrement dit
\begin{equation}
    (E_{ij})_{kl}=\delta_{ik}\delta_{jl}.
\end{equation}
\begin{definition}
    Une \defe{matrice de transvection}{transvection (matrice)}\index{matrice!de transvection} est une matrice de la forme
    \begin{equation}
        T_{ij}(\lambda)=\id+\lambda E_{ij}
    \end{equation}
    avec \( i\neq j\).

    Une \defe{matrice de dilatation}{matrice!dilatation}\index{dilatation (matrice)} est une matrice de la forme
    \begin{equation}
        D_i(\lambda)=\id+(\lambda-1)E_{ii}.
    \end{equation}
    Ici le \( (\lambda-1)\) sert à avoir \( \lambda\) et non \( 1+\lambda\). C'est donc une matrice qui dilate d'un facteur \( \lambda\) la direction \( i\) tout en laissant le reste inchangé.

    Si \( \sigma\) est une permutation (un élément du groupe symétrique \( S_n\)) alors la \defe{matrice de permutations}{matrice!de permutation}\index{permutation!matrice} associée est la matrice d'entrées
    \begin{equation}
        (P_{\sigma})_{ij}=\delta_{i\sigma(j)}.
    \end{equation}
\end{definition}

\begin{lemma}   \label{LemyrAXQs}
    La matrice \( T_{ij}(\lambda)A=(\mtu+\lambda E_{ij})A\) est la matrice \( A\) à qui on a effectué la substitution
    \begin{equation}
        L_i\to L_i+\lambda L_j.
    \end{equation}
    La matrice \( AT_{ij}(\lambda)\) est la substitution 
    \begin{equation}
        C_j\to C_j+\lambda C_i.
    \end{equation}

    La matrice \( AP_{\sigma}\) est la matrice \( A\) dans laquelle nous avons permuté les colonnes avec \( \sigma\).

    La matrice \( P_{\sigma}A\) est la matrice \( A\) dans laquelle nous avons permuté les lignes avec \( \sigma^{-1}\).
\end{lemma}

\begin{proof}
    Calculons la composante \( kl\) de la matrice \( E_{ij}A\) :
    \begin{subequations}
        \begin{align}
            (E_{ij}A)_{kl}&=\sum_m(E_{ij})_kmA_{ml}\\
            &=\sum_m\delta_{ik}\delta_{jm}A_{ml}\\
            &=\delta_{ik}A_{jl}.
        \end{align}
    \end{subequations}
    C'est donc la matrice pleine de zéros, sauf la ligne \( i\) qui est donnée par la ligne \( j\) de \( A\). Donc effectivement la matrice
    \begin{equation}
        A+\lambda E_{ij}A
    \end{equation}
    est la matrice \( A\) à laquelle on a substitué la ligne \( i\) par la ligne \( i\) plus \( \lambda\) fois la ligne \( j\).

    En ce qui concerne l'autre assertion sur les transvections, le calcul est le même et nous obtenons
    \begin{equation}
        (AE_{ij})=A_{ki}\delta_{jl}.
    \end{equation}

    Pour les matrices de permutations, nous avons 
    \begin{equation}
        (AP_{\sigma})_{kl}=A_{k\sigma(l)}
    \end{equation}
    et
    \begin{equation}
        (P_{\sigma}A)_{kl}=\sum_m\delta_{k\sigma(m)}A_{ml}=\sum_m\delta_{\sigma^{-1}(k)m}A_{ml}=A_{\sigma^{-1}(k)l}.
    \end{equation}
\end{proof}

\begin{theorem}[Décomposition de Bruhat]\index{Bruhat (décomposition)}\index{décomposition!Bruhat}    \label{ThoizlYJO}
    Soit \( \eK\) un corps; un élément \( M\in\GL(n,\eR)\) s'écrit sous la forme
    \begin{equation}
        M=T_1P_{\sigma}T_2
    \end{equation}
    où \( T_1\) et \( T_2\) sont des matrices triangulaires supérieures inversibles et où \( P_{\sigma}\) est une matrice de permutation \( \sigma\in S_n\). De plus il y a unicité de \( \sigma\).
\end{theorem}

\begin{proof}
    Afin de rendre les choses plus visuelles, nous nous permettons de donner des exemples au fur et à mesure de la preuve. Nous prenons l'exemple de la matrice
    \begin{equation}
        \begin{pmatrix}
            1    &   3    &   4    \\
            2    &   5    &   6    \\
            0    &   7    &   8
        \end{pmatrix}.
    \end{equation}
    \begin{subproof}
    \item[Existence]
        Soit \( M\in \GL(n,\eR)\); vu qu'elle est inversible, on a un indice \( i_1\) maximum tel que \( M_{i_1,1}\neq 0\). Nous changeons toutes les lignes jusque là, c'est à dire que nous faisons, pour \( 1\leq i< i_1\),
        \begin{equation}        \label{EqGHUbwR}
            L_i\to L_i-\frac{ M_{i1} }{ M_{i_11} }L_{i_1}.
        \end{equation}

        Nous avons donc obtenu une matrice dont la première colonne est nulle sauf la case numéro \( i_1\). L'opération \eqref{EqGHUbwR} revient à considérer la multiplication par la matrice de transvection
        \begin{equation}
            T_1^{(i)}=T_{ii_1}\left( -\frac{ M_{i1} }{ M_{i_11} } \right)
        \end{equation}
        pour tout \( i<i_1\). Pour rappel nous ne changeons que les lignes \emph{au-)dessus} de la \( i_1\). Du coup les matrices \( T^{(i)}_1\) sont triangulaires supérieures. Nous avons donc la nouvelle matrice \( M_1=\left( \prod_{i<i_1}T_1^{(i)} \right)M\) pour laquelle toute la première colonne est nulle sauf un élément.

        Dans le cas de l'exemple, le «pivot» sera la ligne \( (2,5,6)\) et la matrice se transforme à l'aide de la matrice \( T_1=T_{12}(-1/2)\) :
        \begin{equation}    \label{EqyjXIYf}
            \begin{pmatrix}
                1    &   -1/2    &   0    \\
                0    &   1    &   0    \\
                0    &   0    &   1
            \end{pmatrix}
            \begin{pmatrix}
                1    &   3    &   4    \\
                2    &   5    &   6    \\
                0    &   7    &   8
            \end{pmatrix}=
            \begin{pmatrix}
                0    &   1/2    &   1    \\
                2    &   5    &   6    \\
                0    &   7    &   8
            \end{pmatrix}.
        \end{equation}

    
    Maintenant nous faisons de même avec les colonnes (en renommant \( M\) la matrice obtenue à l'étape précédente) :
    \begin{equation}
        C_j\to C_j-\frac{ M_{i_1j} }{ M_{i_11} }C_1,
    \end{equation}
    qui revient à multiplier à droite par les matrices \( T_{1j}(\frac{ M_{i_1i} }{ M_{i_11} })\) avec \( j>1\). Encore une fois ce sont des matrices triangulaires supérieures.

    Dans l'exemple, pour traiter la seconde colonne, nous multiplions \eqref{EqyjXIYf} à droite par la matrice \( T_{12}(-5/2)\) :
    \begin{equation}
            \begin{pmatrix}
                0    &   1/2    &   1    \\
                2    &   5    &   6    \\
                0    &   7    &   8
            \end{pmatrix}
            \begin{pmatrix}
                1    &   -5/2    &   0    \\
                0    &   1    &   0    \\
                0    &   0    &   1
            \end{pmatrix}=
            \begin{pmatrix}
                0    &   1/2    &   1    \\
                2    &   0    &   6    \\
                0    &   7    &   8
            \end{pmatrix}.
    \end{equation}
    Appliquer encore la matrice \( T_{13}(-6/2)\) apporte la matrice
    \begin{equation}
        \begin{pmatrix}
            0    &   1/2    &   1    \\
            2    &   0    &   0    \\
            0    &   7    &   8
        \end{pmatrix}.
    \end{equation}
    Enfin nous multiplions la matrice obtenue par \( \frac{1}{ M_{i_11} }\mtu\) pour normaliser à \( 1\) l'élément «pivot» que nous avions choisit. Dans notre exemple nous multiplions par \( 1/2\) pour trouver
    \begin{equation}        \label{Eqduglwu}
        \begin{pmatrix}
            0    &   1/4    &   1/2    \\
            1    &   0    &   0    \\
            0    &   7/2    &   4
        \end{pmatrix}.
    \end{equation}

    La matrice obtenue jusqu'ici possède une ligne et une colonne de zéros avec un \( 1\) à leur intersection, et elle est de la forme
    \begin{equation}
        M'=T_1MT_2
    \end{equation}
    où \( T_1\) et \( T_2\) sont triangulaires supérieures et inversibles, produits de matrices de transvection (et d'une matrice scalaire pour la normalisation).

    Il reste à recommencer l'opération avec la seconde colonne (qui n'est pas toute nulle parce que le déterminant est encore non nul) puis la suivante etc. Dans notre exemple de l'équation \eqref{Eqduglwu}, nous éliminerions le \( 1/4\) et le \( 4\) en utilisant le \( 7/2\).

    Encore une fois tout cela se fait à l'aide de matrice supérieures parce qu'à chaque étape, les colonnes précédent le pivot sont déjà nulles (saut un \( 1\)) et ne doivent donc pas être touchées.

    À la fin de ce processus, ce qui reste est une matrice \( TMT'\) qui ne contient plus que un seul \( 1\) sur chaque ligne et chaque colonne, c'est à dire une matrice de permutation : \( P_{\sigma}=TMT'\) et donc
    \begin{equation}
        M=T^{-1}_{\sigma}(T')^{-1}.
    \end{equation}

        \item[Unicité]

            Soient \( \sigma,\sigma\in S_n'\) tels que \( T_1P_{\sigma}T_2=S_1P_{\tau}S_2\) avec \( T_i\) et \( S_i\) triangulaires supérieures et inversibles. En posant \( T=T_2S_2^{-1}\) et \( S=T_1^{-1}S_1\), nous avons
            \begin{equation}
                P_{\sigma}T=SP_{\tau}
            \end{equation}
            où \( S\) et \( T\) sont des matrices triangulaires supérieures et inversibles. Par les calculs de la preuve du lemme \ref{LemyrAXQs},
            \begin{subequations}
                \begin{numcases}{}
                    (P_{\sigma}T)_{kl}=T_{\sigma^{-1}(k)l}\\
                    (SP_{\tau})_{kl}=S_{k\tau(l)},
                \end{numcases}
            \end{subequations}
            et donc
            \begin{equation}    \label{EqKlmgOT}
                T_{\sigma^{-1}(k)l}=S_{k\tau(l)}.
            \end{equation}
            En écrivant cette équation avec \( k=\sigma(i)\) (nous rappelons que \( \sigma\) est bijective),
            \begin{equation}
                T_{il}=S_{\sigma(i)\tau(l)}.
            \end{equation}
            Nous savons que les termes diagonaux de \( T\) sont non nuls parce que \( T\) est triangulaire supérieure et inversible (donc pas de colonnes entières nulles). Nous avons donc, en prenant \( i=l=k\),
            \begin{equation}
                0\neq T_{kk}=S_{\sigma(k)\tau(k)}.
            \end{equation}
            La matrice étant triangulaire supérieure, cela implique 
            \begin{equation}    \label{EqEmiBTX}
                \sigma(k)\leq\tau(k).
            \end{equation}
            De la même manière en écrivant \eqref{EqKlmgOT} avec \( l=\tau^{-1}(i)\),
            \begin{equation}
                S_{ki}=T_{\sigma^{-1}(k)\tau^{-1}(i)}
            \end{equation}
            et donc
            \begin{equation}
                \sigma^{-1}(k)\leq \tau^{-1}(k).
            \end{equation}
            En écrivant cela avec \( k=\sigma(j)\), nous avons \( j\leq \tau^{-1}\sigma(j)\) et en appliquant enfin \( \tau\),
            \begin{equation}
                \tau(j)\leq \sigma(j).
            \end{equation}
            En comparant avec \eqref{EqEmiBTX}, nous avons \( \sigma=\tau\).
    \end{subproof}
\end{proof}

%++++++++++++++++++++++++++++++++++++++++++++++++++++++++++++++++++++++++++++++++++++++++++++++++++++++++++++++++++++++++++++++++++++++++
\section{Espaces de polynômes}		\label{SecEspacePolynomes}
%++++++++++++++++++++++++++++++++++++++++++++++++++++++++++++++++++++++++++++++++++++++++++++++++++++++++++++++++++++++++++++++++++++++++
 
Dans cette section nous abandonnons pour quelques minutes l'espace $\eR^m$ et considérons plus attentivement l'espace des fonctions polynômiales $\mathcal{P}_{\eR}$ et de ses sous-espaces $\mathcal{P}_{\eR}^k$, pour $k$ dans $\eN_0$. 

Pour chaque $k>0$ donné nous définissons
\begin{equation}
\mathcal{P}_\eR^k=\{p:\eR\to \eR\,|\, p : x\mapsto a_0+a_1 x +a_2 x^2 + \ldots+a_k x^k, \, a_i\in\eR,\,\forall i=0,\ldots,k\}.
\end{equation}   
Il est facile de se convaincre que la somme de deux polynômes de degré inférieur ou égal à $k$ est encore un polynôme de degré inférieur ou égal à $k$. En outre il est clair que la multiplication par un scalaire ne peut pas augmenter le degré d'un polynôme. L'ensemble $\mathcal{P}_\eR^k$ est donc un espace vectoriel muni des opérations héritées de $\mathcal{P}_{\eR}$. 

La base canonique de l'espace $\mathcal{P}_\eR^k$ est donné par les monômes $\mathcal{B}=\{x\mapsto x^j \,|\, j=0, \ldots, k\}$. Le fait que cela soit une base est vraiment facile à démontrer et est un exercice très utile si vous ne l'avez pas encore vu dans un cours précédent. 

Nous allons maintenant étudier trois application linéaires de $\mathcal{P}_\eR^k$ vers des autres espaces vectoriels
\begin{description}
  \item[L'isomorphisme canonique  $\phi:\mathcal{P}_\eR^k \to\eR^{k+1}$] Nous définissons $\phi$ par les relations suivantes
\[
\phi(x^j)=e_{j+1}, \qquad \forall j\in\{0,\dots, k\}. 
\]
Cela veut dire que pour tout $p$ dans $\mathcal{P}_\eR^k$, avec $p(x)=a_0+a_1 x +a_2 x^2 + \ldots+a_k x^K$, l'image de $p$ par $\phi$ est 
\[
\phi(p)=\phi\left(\sum_{j=0}^k a_j x^j\right)=\sum_{j=0}^k a_j e_{j+1}.
\]
\begin{example} Soit $k=5$ on a 
  \begin{equation}
    \phi(-8-7x-4x^2+4x^3+2x^5)=
  \begin{pmatrix}
    -8\\
    -7\\
    -4\\
    4\\
    0\\
    2
  \end{pmatrix}.
  \end{equation}
\end{example}

Cette application est clairement bijective et respecte les opérations d'espace vectoriel, donc elle est un isomorphisme d'espaces vectoriels. L'existence d'un isomorphisme entre $\mathcal{P}_\eR^k$  et $\eR^{k+1}$ est un cas particulier du théorème qui dit que  pour chaque $m$ dans $\eN_0$ fixée, tous les espaces vectoriels sur $\eR$ de dimension $m$ sont isomorphes à $\eR^m$. Vous connaissez peut être déjà ce théorème depuis votre cours d'algèbre linéaire.  
    \item[La dérivation $d: \mathcal{P}_\eR^k \to \mathcal{P}_\eR^{k-1}$] L'application de dérivation $d$ fait exactement ce qu'on s'attend d'elle 
\[
d(x^0)=d(1)=0, \qquad d(x^j)=j x^{j-1}, \quad \forall j\in\{1,\dots, k\}. 
\]
Cette application n'est pas injective, parce que l'image de $p$ ne dépend pas de la valeur de $a_0$, donc si deux polynômes sont les mêmes à une constante près ils auront la même image par $d$.

\begin{example} Soit $k=3$ on a 
  \begin{equation}
    d(-8-12x+4x^3)= -12 (1) + 4 (3x^2) = -12+12 x^2.
    \end{equation}

    Noter que $d(-30-12x+4x^3)=d(-8-12x+4x^3)$. Cela confirme, comme mentionné plus haut que la dérivée n'est pas injective.
\end{example}
      \item[L'intégration $I: \mathcal{P}_\eR^k \to \mathcal{P}_\eR^{k+1}$] Nous pouvons définir une application que est <<à une constante prés>> l'application inverse de la dérivation
        \begin{equation}
          I(p)= \int_0^x p(t) \,dt.
        \end{equation}
Il faut comprendre que dans l'intégral la variable $t$ est simplement la variable d'intégration. La <<vraie>> variable de la fonction image de $p$ sera $x$ !
 
Comme d'habitude nous écrivons explicitement l'action de $I$ sur les éléments de la base canonique
\begin{equation}
    I(x^j)=\int_0^x t^k \,dt= \frac{x^{j+1}}{j+1}.
\end{equation}

\begin{example} 
   Soit $k=4$ on a 
  \begin{equation}
    I(6+2x+x^2+x^4)= 6x+x^2+\frac{x^3}{3}+\frac{x^5}{5}.
    \end{equation}
\end{example}

Remarquez que, étant donné que dans la définition de $I$ nous avons décidé d'intégrer entre zéro et $x$, tous les polynômes dans $\mathcal{P}_\eR^{k+1}$ qui sont l'image par $I$ d'un polynôme de $\mathcal{P}_\eR^{k}$ ont $a_0=0$. Cela veut dire que nous pouvons générer toute l'image de $I$ en utilisant un sous-ensemble de la base canonique de $\mathcal{P}_\eR^{k+1}$,  en particulier $\mathcal{B}_1=\{x\mapsto x^j \,|\, j=1, \ldots, k\}\subset \mathcal{B}$ nous suffira. Cela n'est guère surprenant, parce que l'image par une application linéaire d'un espace vectoriel de dimension finie ne peut pas être un espace de dimension supérieure. 
\end{description}

Les applications de dérivation et intégration correspondent évidemment à des application linéaires de $\mathcal{P}_\eR$ dans lui-même. 

L'espace de tous les polynômes étant de dimension infinie, il peut servir de contre exemple assez simple. Dans la sous-section \ref{SubSecPOlynomesCE}, nous verrons que toutes les normes ne sont pas équivalentes sur l'espace des polynômes.



%---------------------------------------------------------------------------------------------------------------------------
\subsection{Polynômes symétriques, alternés ou semi-symétriques}
%---------------------------------------------------------------------------------------------------------------------------
\cite{fJhCTE}.

Soit \( \eK\) un corps de caractéristique différente\footnote{Le truc de la caractéristique deux est que \( a=-a\) n'implique pas \( a=0\).} de \(2\). Le groupe \( S_n\) agit sur l'anneau \( \eK[T_1,\ldots, T_n]\) par
\begin{equation}
    (\sigma\cdot f)(T_1,\ldots, T_n)=f\big( T_{\sigma(1)},\ldots, T_{\sigma(n)} \big).
\end{equation}
On peut vérifier que c'est un action.

\begin{definition}
    Un polynôme \( Q\) en \( n\) indéterminées est 
    \begin{enumerate}
        \item
            \defe{symétrique}{polynôme!symétrique}\index{symétrique!polynôme} si \( Q=\sigma\cdot Q\) pour tout \( \sigma\in S_n\);
        \item
            \defe{alterné}{polynôme!alterné}\index{alterné!polynôme} si \( \sigma\cdot Q=\epsilon(\sigma)Q\) pour tout \( \sigma\in S_n\);
        \item
            \defe{semi-symétrique}{semi-symétrique!polynôme}\index{polynôme!semi-symétrique} si \( \sigma\cdot Q=Q\) pour tout \( \sigma\in A_n\)
    \end{enumerate}
\end{definition}
Le polynôme \( T_1+T_2\) est symétrique; le polynôme \( T_1+T_2^2\) ne l'est pas. 

\begin{example}
    Le déterminant de Vandermonde (proposition \ref{PropnuUvtj}) est alterné, semi-symétrique et non symétrique. Le fait qu'il soit alterné est le fait qu'il soit un déterminant. Étant donné qu'il est alterné, il est semi-symétrique parce que sur \( A_n\), nous avons \( \epsilon=1\). Étant donné qu'il est alterné, il change de signe sous l'action des éléments impairs de \( S_n\) et n'est donc pas symétrique.
\end{example}

\begin{proposition} \label{PropUDqXax}
    Un polynôme semi-symétrique \( f\in \eK[T_1,\ldots, T_n]\) se décompose de façon unique en
    \begin{equation}
        f=P+VQ
    \end{equation}
    où \( P\) et \( Q\) sont deux polynômes symétriques.
\end{proposition}

\begin{proof}

    Nous commençons par prouver l'unicité en montrant que si \( f=PVQ\) avec \( P\) et \( Q\) symétrique, alors \( P\) et \( Q\) sont donnés par des formules explicites en termes de \( f\).


    Si \( \sigma_1\) et \( \sigma_2\) sont deux permutations impaires de \( \{ 1,\ldots, n \}\), alors \( \sigma_1\cdot f=\sigma_2\cdot f\) parce que l'élément \( \sigma_2^{-1}\sigma_1\) est pair (proposition \ref{ProphIuJrC}), de telle sorte que \( \sigma_2^{-1}\sigma_1\cdot f=f\). Nous posons donc \( g=\tau\cdot f\) où \( \tau\) est une permutation impaire quelconque -- par exemple une transposition.

    Vu que \( V\) est alternée et que \( \tau\) est une transposition nous avons
    \begin{equation}
        g=\tau\cdot f=P-VQ.
    \end{equation}
    Donc \( f+g=2P\) et \( f-g=2VQ\). Cela donne \( P\) et \( Q\) en terme de \( f\) et \( g\), et donc l'unicité.

    Attention : cela ne donne pas un moyen de prouver l'existence parce que rien ne prouve pour l'instant que \( f-g\) peut effectivement être écrit sous la forme \( VQ\), c'est à dire que \( f-g\) soit divisible par \( V\). C'est cela que nous allons nous atteler à démontrer maintenant.

    Nous commençons par prouver que \( f+g\) est symétrique et \( f-g\) alterné. Si \( \sigma\) est une transposition,
    \begin{equation}
        \sigma\cdot(f+g)=\sigma\cdot f+\sigma\tau\cdot f=g+f
    \end{equation}
    parce que \( \sigma\tau\) est pair. De la même façon,
    \begin{equation}
        \sigma\cdot(f-g)=g-f=\epsilon(\sigma)(f-g).
    \end{equation}
    Dans les deux cas nous concluons en utilisant le fait que toute permutation est un produit de transpositions (proposition \ref{PropPWIJbu}) et que \( \epsilon\) est un homomorphisme.

    Soient maintenant deux entiers \( h<k\) dans \( \{ 1,\ldots, n \}\) et l'anneau
    \begin{equation}
        \big( \eK[T_1,\ldots, \hat T_k,\ldots, T_n] \big)[T_k].
    \end{equation}
    Cet anneau contient le polynôme \( T_k-T_h\) où \( T_k\) est la variable et \( T_h\) est un coefficient. Nous faisons la division euclidienne de \( f-g\) par  \( T_k-T_h\) parce que nous avons dans l'idée de faire arriver le déterminant de Vandermonde et donc le produit de toutes les différences \( T_k-T_h\) :
    \begin{equation}    \label{EqSHdgrG}
        f-g=(T_k-T_h)q+r
    \end{equation}
    où \( \deg_{T_k}r<1\), c'est à dire que \( r\) ne dépends pas de \( T_k\). Nous revoyons maintenant l'égalité \eqref{EqSHdgrG} dans \( \eK[T_1,\ldots, T_n]\) et nous y appliquons la transposition \( \tau_{kh}\). Nous savons que \( \tau_{kh}(f-g)=-(f-g)\) et \( \tau_{kh}(T_k-T_h)=-(T_k-T_h)\), et donc
    \begin{equation}    \label{EqVOhjKB}
        -(f-g)=-(T_k-T_h)\tau_{kh}\cdot   q+\tau_{kh}\cdot r
    \end{equation}
    où \(\tau_{kh}\cdot r\) ne dépend pas de \( T_h\). Nous appliquons à \eqref{EqVOhjKB} l'application
    \begin{equation}
        \begin{aligned}
            t\alpha\colon \eK[T_1,\ldots, T_n]&\to \eK[T_1,\ldots, \hat T_k,\ldots, T_n] \\
            \alpha(PT_1,\ldots, \hat T_k,\ldots, T_n)&=P(T_1,\ldots, T_h,\ldots, T_n). 
        \end{aligned}
    \end{equation}
    Cette application vérifie \( \alpha\big( \tau_{kh}\cdot r \big)=\alpha(r)\) et nous avons
    \begin{equation}
        -\alpha(f-g)=\alpha(r).
    \end{equation}
    Puis en appliquant \( \alpha\) à la relation \( f-g=(T_k-T_h)q+r\), nous trouvons
    \begin{equation}
        \alpha(f-g)=\alpha(r),
    \end{equation}
    et par conséquent \( \alpha(r)=0\). Ici nous utilisons l'hypothèse de caractéristique différente de deux. Dire que \( \alpha(r)=0\), c'est dire que \( r\) est divisible par \( T_k-T_h\), mais \( r\) étant de degré zéro en \( T_k\), nous avons \( r=0\). Par conséquent \( T_k-T_h\) divise \( f-g\) pour tout \( h<k\), et nous pouvons définir un polynôme \( Q\) par
    \begin{equation}    \label{EqrnbgdA}
        f-g=2Q\prod_{h<k}\prod_{k\leq n}(T_k-T_h)=2Q(T_1,\ldots, T_n)V(T_1,\ldots, T_n),
    \end{equation}
    où nous avons utilisé la formule du déterminant de Vandermonde de la proposition \ref{PropnuUvtj}.

    Étant donné que \( f+g\) est un polynôme symétrique, nous allons aussi poser \( f+g=2P\) avec \( P\) symétrique.

    Montrons à présent que \( Q\) est un polynôme symétrique. Soit \( \sigma\in S_n\); vu que nous savons déjà que \( f-g\) est alternée, nous avons
    \begin{equation}    \label{EqpSPEyq}
        \sigma\cdot (f-g)=\epsilon(\sigma)(f-g)=\epsilon(\sigma)2QV,
    \end{equation}
    Mais en appliquant \( \sigma\) à l'équation \eqref{EqrnbgdA},
    \begin{subequations}
        \begin{align}
            \sigma\cdot (f-g)&=2(\sigma\cdot V)(T_1,\ldots, ,T_n)(\sigma\cdot Q)(T_1,\ldots,T_n)\\
            &=2\epsilon(\sigma)V(T_1,\ldots, T_n)(\sigma\cdot Q)(T_1,\ldots, T_n).
        \end{align}
    \end{subequations}
    En égalisant avec \eqref{EqpSPEyq} et en se souvenant que l'anneau \( \eK[T_1,\ldots, T_n]\) était intègre (théorème \ref{ThoBUEDrJ}), nous simplifions par \( 2\epsilon(\sigma)V\) pour obtenir
    \begin{equation}
        Q=\sigma\cdot Q,
    \end{equation}
    c'est à dire que \( Q\) est symétrique.

    Au final nous avons \( f+q=2P\) et \( f-g=2VQ\) avec \( P\) et \( Q\) symétriques. En faisant la somme,
    \begin{equation}
        f=P+VQ.
    \end{equation}
\end{proof}

%---------------------------------------------------------------------------------------------------------------------------
\subsection{Polynôme symétrique élémentaire}
%---------------------------------------------------------------------------------------------------------------------------

Le \( k\)ième \defe{polynôme symétrique élémentaire}{élémentaire!polynôme symétrique} à \( n\) inconnues est le polynôme est
\begin{equation}
    \sigma_k(T_1,\ldots, T_n)=\sum_{s\in F_k}\prod_{i=1}^kT_{s(i)}
\end{equation}
où \( F_k\) est l'ensemble des fonctions strictement croissantes \( \{ 1,2,\ldots, k \}\to\{ 1,2,\ldots, n \}\). Une autre façon de décrire ces polynômes élémentaires est
\begin{equation}
    \sigma_k=\sum_{1\leq i_1<\ldots<i_k\leq n}X_{i_1}\ldots X_{i_k}.
\end{equation}
Par exemple
\begin{subequations}
    \begin{align}
        \sigma_1(T_1,\ldots, T_n)&=T_1+T_2+\ldots +T_n\\
        \sigma_2(T_1,\ldots, T_n)&=T_1T_2+\ldots +T_1T_n+T_2T_3+\ldots +T_2T_n+\ldots +T_{n-1}T_n\\
        \sigma_n(T_1,\ldots, T_n)&=T_1\ldots T_n.
    \end{align}
\end{subequations}
En particulier, \( \sigma_2(x,y,z)=xy+yz+xz\).

\begin{theorem}[\cite{PoloPolSym}]  \label{TholReBiw}
    Si \( Q\) est un polynôme symétrique en \( T_1,\ldots, T_n\), alors il existe un et un seul polynôme \( P\) en \( n\) indéterminées tel que
    \begin{equation}
        Q(T_1,\ldots, T_n)=P\big( \sigma_1(T_1,\ldots, T_n),\ldots, \sigma_n(T_1,\ldots, T_n) \big).
    \end{equation}
\end{theorem}
%TODO : la preuve de ce théorème

\begin{example}
    Nous voulons décomposer \( P(x,y,z)=x^3+y^3+z^3\) en polynômes symétriques élémentaires, c'est à dire en
    \begin{subequations}
        \begin{numcases}{}
            \sigma_1=x+y+z\\
            \sigma_2=xy+xz+yz\\
            \sigma_3=xyz.
        \end{numcases}
    \end{subequations}
    Étant donné que \( P\) est de degré \( 3\), les seules combinaisons des \( \sigma_i\) qui peuvent intervenir sont \( \sigma_1^3\), \( \sigma_1\sigma_2\) et \( \sigma_3\). Étant donné que dans \( P\) le coefficient de \( x^3\) est un, il est obligatoire d'avoir un coefficient \( 1\) devant \( \sigma_1^3\). Nous le calculons :
    \begin{verbatim}
----------------------------------------------------------------------
| Sage Version 4.8, Release Date: 2012-01-20                         |
| Type notebook() for the GUI, and license() for information.        |
----------------------------------------------------------------------
sage: var('x,y,z')
(x, y, z)
sage: P=x**3+y**3+z**3  
sage: S1=x+y+z    
sage: S2=x*y+x*z+y*z
sage: S3=x*y*z
sage: (S1**3).expand()
x^3 + 3*x^2*y + 3*x^2*z + 3*x*y^2 + 6*x*y*z + 3*x*z^2 + y^3 + 3*y^2*z + 3*y*z^2 + z^3
sage: (S1**3-P).expand()
3*x^2*y + 3*x^2*z + 3*x*y^2 + 6*x*y*z + 3*x*z^2 + 3*y^2*z + 3*y*z^2
x^3 + 3*x^2*y + 3*x^2*z + 3*x*y^2 + 6*x*y*z + 3*x*z^2 + y^3 + 3*y^2*z + 3*y*z^2 + z^3
    \end{verbatim}
    Dans la différence \( \sigma_1^3-P\) nous voyons que le terme en \( xyz\) est \( 6xyz\); par conséquent nous savons que le coefficient de \( \sigma_3\) sera \( -6\). Il nous reste :
    \begin{verbatim}
sage: (S1**3+6*S3-P).expand()
3*x^2*y + 3*x^2*z + 3*x*y^2 + 12*x*y*z + 3*x*z^2 + 3*y^2*z + 3*y*z^2    
    \end{verbatim}
    que nous identifions facilement avec \( 3\sigma_1\sigma_2\). Nous avons donc
    \begin{equation}
        P=\sigma_1^3-3\sigma_1\sigma_2+3\sigma_3.
    \end{equation}
\end{example}


\begin{lemma}[\cite{fJhCTE}]    \label{LemSoXCQH}
    Soit \( \eK\) une extension de degré \( \delta\) de \( \eQ\) et \( P\in \eK[T_1,\ldots, T_m]\). Alors il existe \( \bar P\in \eQ[T_1,\ldots, T_m]\) tel que
    \begin{enumerate}
        \item
            $\deg\bar P=\delta\deg(P)$
        \item
            pour tout \( (z_1,\ldots, z_m)\in \eC^m\) tel que \( P(z_1,\ldots, z_m)=0\), on a \( \bar P(z_1,\ldots, z_m)=0\).
    \end{enumerate}
\end{lemma}

\begin{proof}
    En vertu de la proposition \ref{PropUmxJVw} et de l'exemple \ref{ExvQTyBl}, \( \eK\) est une extension séparable de \( \eQ\), et donc vérifie le théorème de l'élément primitif (\ref{ThoORxgBC}). Il existe \( \theta\in \eK\) tel que \( \eK=\eQ(\theta)\). Soit \( P_{\theta}\in\eQ[X]\) le polynôme minimal de \( \theta\). L'extension \( \eK\) étant de degré \( \delta\), et \( \theta\) étant un générateur, une base de \( \eK\) comme espace vectoriel sur \( \eQ\) est 
    \begin{equation}
        \{ 1,\theta,\ldots, \theta^{\delta-1} \}.
    \end{equation}
    Mais par ailleurs la proposition \ref{PropdsRAsk} nous indique qu'une base est donnée par
    \begin{equation}
        \{ 1,\theta,\ldots, \theta^{n-1} \}
    \end{equation}
    où \( n\) est le degré de \( P_{\theta}\). Donc \( P_{\theta}\) est de degré \( \delta\). Nous nommons \( \theta_1,\ldots, \theta_{\delta}\) les racines de \( P_{\theta}\) dans un corps de décomposition. Ici nous notons \( \theta=\theta_1\) et nous ne prétendons pas que \( \theta_k\in \eK\). Notons que ces \( \theta_i\) sont toutes des racines simples de \( P_{\theta}\), sinon nous aurions un facteur irréductible \( (X-\theta_k)^2\), et \( P_{\theta}\) ne serait pas irréductible sur \( \eQ\).

    Soit \( \sigma_k\) le morphisme canonique
    \begin{equation}
        \begin{aligned}
            \sigma_k\colon \eQ(\theta)&\to \eQ(\theta_k) \\
            \sum_i q_i\theta^i&\mapsto \sum_iq_i\theta_k^i 
        \end{aligned}
    \end{equation}
    Nous avons \( \sigma_1\colon \eK\to \eK\) qui est l'identité.

    Notons \( N\) le degré du polynôme \( P\in \eK[T_1,\ldots, T_m]\) dont il est question dans l'énoncé. Nous le décomposons alors en
    \begin{equation}
        P=\sum_{l=0}^N\sum_{i=1}^mc_{il}T_i^l
    \end{equation}
    avec \( c_{il}\in \eK\). Nous voyons \( c_{i,\cdot}\) comme un élément de \( \eK^m\) et donc nous écrivons\footnote{Il me semble qu'il manque la somme sur \( i\) dans \cite{fJhCTE}.}
    \begin{equation}
        P=\sum_{l=0}^N\sum_{i=1}^m c_l(\theta)_iT_i^l
    \end{equation}
    où \( c_l\in \eQ[X]^m\). Nous pouvons choisir \( \deg(c_l)<\delta\) parce que les puissances plus grandes de \( \theta\) ne génèrent rien de nouveau.

    Nous posons aussi
    \begin{equation}
        P^{\sigma_k}=\sum_{l,i} c_l(\theta_k)_iT_i^l\in \eQ(\theta_k)[T_1,\ldots, T_m],
    \end{equation}
    et \( \bar P=PP^{\sigma_2}\ldots P^{\sigma_k}\). Le coefficient de \( T_i^l\) dans \( \bar P\) est
    \begin{equation}
        \bar c_l(\theta_1,\ldots, \theta_{\delta})_i=\sum_{l_1+\ldots +l_{\delta}=l}c_{l_1}(\theta_1)_i\ldots c_{l_{\delta}}(\theta_{\delta})_i.
    \end{equation}
    Ce dernier est un polynôme en les \( \theta_k\) à coefficients dans \( \eQ\). Qui plus est, c'est un polynôme symétrique. En effet un terme contenant \( \theta_k^a\theta_l^b\) provenant de \( c_{l_i}(\theta_k)c_{l_j}(\theta_l)\) a un terme correspondant \( \theta_k^b\theta_l^a\) provenant de \( c_{l_j}(\theta_k)c_{l_i}(\theta_l)\).

    C'est donc le moment d'utiliser le théorème \ref{TholReBiw} à propos des polynômes symétriques élémentaires qui nous dit que les coefficients de \( \bar P\) sont en réalité des polynômes en ceux de \( P_{\theta}\) qui sont dans \( \eQ\). Donc \( \bar P\in \eQ[T_1,\ldots, T_m]\). Par ailleurs nous avons que
    \begin{equation}
        \deg(\bar P)=\delta \deg(P)
    \end{equation}
    parce que \( \bar P\) est le produit de \( \delta\) «copies»  de \( P\). De plus \( P=P^{\sigma_1}\) divise \( \bar P \) donc on a bien que si \( P(z)=0\) alors \( \bar P(z)=0\). Le polynôme \( \bar P\) est celui que nous cherchions. 
\end{proof}


%+++++++++++++++++++++++++++++++++++++++++++++++++++++++++++++++++++++++++++++++++++++++++++++++++++++++++++++++++++++++++++
\section{Polynômes cyclotomiques}
%+++++++++++++++++++++++++++++++++++++++++++++++++++++++++++++++++++++++++++++++++++++++++++++++++++++++++++++++++++++++++++

%---------------------------------------------------------------------------------------------------------------------------
\subsection{Définitions et propriétés}
%---------------------------------------------------------------------------------------------------------------------------

Pour \( n\in\eN^*\) nous considérons l'ensemble
\begin{equation}
    \Delta_n=\{  e^{2ki\pi/n}\tq 0\leq k\leq n-1,\pgcd(k,n)=1 \}.
\end{equation}
Voir \ref{SubSechZeTuL}. Le \defe{polynôme cyclotomique}{polynôme!cyclotomique} d'indice \( n\) est le polynôme
\begin{equation}    \label{EqLjGYKK}
    \phi_n(X)=\prod_{z\in\Delta_n}(X-z)
\end{equation}
où
\begin{equation}
    \Delta_n=\{  e^{2ik\pi/n}\tq 0\leq k\leq n-1\tq \pgcd(k,n)=1 \}.
\end{equation}
Le polynôme \( \phi_n\) est un polynôme unitaire de degré \( \varphi(n)\). Nous avons par exemple
\begin{subequations}
    \begin{align}
        \Delta_1&=\{ 1 \}\\
        \Delta_2&=\{ -1 \}\\
        \Delta_3&=\{  e^{2\pi i/3, e^{4\pi i/3}} \}
    \end{align}
\end{subequations}
et les premiers polynômes cyclotomiques sont donnés par
\begin{subequations}
    \begin{align}
        \phi_1(X)&=X-1\\
        \phi_2(X)&=X+1\\
        \phi_3(X)&=X^2+X+1.
    \end{align}
\end{subequations}
Pour le dernier nous avons utilisé le fait que \(  e^{6\pi i/3}=1\) et \(  e^{4\pi i/3+ e^{2\pi i/3}}=-1\).

\begin{proposition}     \label{PropUImYnL}
    Soient \( 1\leq m\leq n\) deux entiers et
    \begin{equation}
        T(X)=\frac{ X^n-1 }{ X^m-1 }\in \eZ(X).
    \end{equation}
    Soit \( \phi_n\) le \( n\)-ième polynôme cyclotomique. Alors
    \begin{enumerate}
        \item   \label{ItempnHhYk}
            \( X^n-1=\prod_{d\divides n}\phi_d(X)=\prod_{d\divides n}\prod_{z\in \Delta_d}(X-z)\),
        \item
            \( \phi_n\in \eZ[X]\),
        \item   \label{ItemhpDPKE}
            si \( m\divides n\) alors \( T\in \eZ[X]\),
        \item
            si \( m\divides n\) et si \( m<n\) alors \( \phi_n\) divise \( T\) dans \( \eZ[X]\).
    \end{enumerate}
\end{proposition}

\begin{proof}

    \begin{enumerate}
        \item
            La seconde égalité est seulement la définition \eqref{EqLjGYKK}. Nous ne devons que prouver la première. Notons juste pour le plaisir que dans le produit \( \prod_{d\divides n}\prod_{z\in\Delta_d}\), il y a bien \( n\) termes parce que \( \Card(\Delta_d)=\varphi(d)\) et \( \sum_{d\divides n}\varphi(d)=n\).

            Nous connaissons l'union disjointe \( \gU_n=\bigcup_{d\divides n}\Delta_d\) qui implique
            \begin{equation}
                \prod_{z\in \gU_n}(X-z)=\prod_{d\divides n}\prod_{z\in \Delta_d}(X-z)=\prod_{d\divides n}\phi_d(X),
            \end{equation}
            alors que par définition de \( \gU_n\) nous avons \( X^n-1=\prod_{z\in\gU_n}(X-z)\).

        \item

            Nous devons démontrer que les coefficients de \( \phi_n\) sont dans \( \eZ\) alors qu'ils sont a priori dans \( \eC\). Nous démontrons cela par récurrence. D'abord \( \phi_1(X)=X-1\), d'accord. Ensuite
            \begin{equation}
                X^{n+1}-1=\prod_{d\divides n+1}\phi_d(X)=\phi_{n+1}(X)\cdot\underbrace{\prod_{_{\substack{d\divides n+1\\d\leq n}}}\phi_d(X)}_{\in\eZ[X]\text{ par récurrence}}
            \end{equation}
            Le lemme \ref{LemzwkYdn} conclu que \( \phi_{n+1}\in \eZ[X]\). Nous avons vu \( \eZ\) comme sous anneau du corps \( \eC\).

        \item

            Si \( m\) divise \( n\) alors les diviseurs de \( n\) sont l'union des diviseurs de \( m\) et des diviseurs de \( n\) qui ne divisent pas \( m\). Soit
            \begin{equation}
                Q=\{\text{diviseurs de \( n\) ne divisant pas \( m\)} \}.
            \end{equation}
            Nous avons alors
            \begin{equation}
                X^n-1=\prod_{d\divides n}\phi_d(X)=\prod_{d\divides m}\phi_d(X)\cdot\prod_{q\in Q}\phi_q(X)=(X^m-1)\cdot\prod_{q\in Q}\phi_q(X).
            \end{equation}
            Nous avons donc
            \begin{equation}
                T(X)=\frac{ X^n-1 }{ X^m-1 }=\prod_{q\in Q}\phi_q(X)\in \eZ[X].
            \end{equation}
            
        \item

            Nous venons de montrer que
            \begin{equation}
                T=\prod_{q\in Q}\phi_q\in \eZ[X].
            \end{equation}
            Étant donné que \( m<n\) nous avons \( n\in Q\) et donc
            \begin{equation}
                T=\phi_n\cdot\prod_{q\in Q\setminus\{ n \}}\phi_q.
            \end{equation}
            Par conséquent \( \phi_n\) divise \( T\) dans \( \eZ[X]\).
        \end{enumerate}
\end{proof}

\begin{proposition}[\wikipedia{fr}{Polynôme_cyclotomique}{polynôme cyclotomique}] \label{PropoIeOVh}
    Les polynômes cyclotomiques sont irréductibles sur \( \eQ\).
\end{proposition}

\begin{proof}
    Pour rappel, nous savons déjà que pour tout \( n\in\eN\), \( \phi_n\in \eZ[X]\). Vu que les racines de \( \phi_n\) sont les racines primitives de l'unité, nous devons montrer que toutes les racines primitives de l'unité ont même polynôme minimal (qui sera alors \( \phi_n\)); en effet vu que ces polynômes divisent \( \phi_n\), si ils sont distincts, la proposition \ref{PropyMTEbH} s'applique et le produit des polynômes minimaux diviserait \( \phi_n\). Dans le cas inverse, \( \phi_n\) est polynôme minimal des racines primitives de l'unité et est donc irréductible. Soit donc \( \xi\), une telle racine primitive. Une autre racine primitive est de la forme \( \xi^l\) où \( l\) est un nombre premier tel que \( \pgcd(l,n)=1\).

    Soient \( f\) et \( g\), les polynômes minimaux dans \( \eZ[X]\) de \( \xi\) et \( \xi^l\). Nous allons montrer que \( f=g\) et donc que \( f=g=\phi_n\). Supposons par l'absurde que \( f\neq g\). Dans ce cas ils seraient des facteurs irréductibles distincts de \( \phi_n\) et il existerait un polynôme \( h\) tel que \( \phi_n=fgh\). A priori, \( h\in \eQ[X]\) parce que nous sommes justement en train de prouver que \( \phi_n\) est irréductible dans \( \eQ[X]\). Quoi qu'il en soit, le lemme de Gauss \ref{LemEfdkZw} nous montre que \( h\in \eZ[X]\) parce que \( \phi_n\), \( f\) et \( g\) ont des coefficients entiers. Nous avons
    \begin{equation}
        f(\xi)=g(\xi^l)=0.
    \end{equation}
    Considérons le polynôme \( \psi(X)=g(X^l)\). Ce polynôme \( \psi\) est dans \( \eZ[X]\) et \( \psi\) est annulateur de \( \xi\), donc \( f\) divise \( \psi\) en tant que polynôme minimal de \( \xi\). Il y a un polynôme unitaire à coefficients entiers (lemme de Gauss forever) \( k\) tel que
    \begin{equation}
        \psi=fk
    \end{equation}
    Nous considérons maintenant les projections sur \( \eF_l[X]\) : étant donné que \( \phi_n=fgh\), nous savons que \( \bar f\bar g\) divise \( \bar\phi_n\). En même temps, \( \bar f\) divise \( \bar \psi\). En utilisant le morphisme de Frobenius (c'est ici que la projection sur \( \eF_l\) joue), nous avons aussi
    \begin{equation}
        \bar\psi(X)=\bar g(X^l)=\bar g(X)^l.
    \end{equation}
    Par conséquent dire que \( \bar f\) divise \( \bar\psi\) revient à dire que \( \bar f(X)\) divise \( \bar g(X)^l\). En particulier tous facteur irréductible de \( \bar f\) divise \( \bar g\). Un facteur irréductible de \( \bar f\) serait donc à la fois dans \( \bar f\) et dans \( \bar g\) et donc deux fois (au moins) dans \( \bar\phi_n\) parce que \( \bar f\bar g\) divise \( \phi_n\). Dans un corps de décomposition de ce facteur, \( \phi_n\) aurait une racine double, alors que ce n'est pas le cas. Contradiction. Nous concluons que \( f=g\).
\end{proof}

\begin{theorem} \label{ThojCJpFW}
    Soit \( P\in \eZ[X]\) un polynôme unitaire irréductible non constant tel que toutes les racines dans \( \eC\) soient de module \( \leq 1\). Alors \( P=X\) ou \( P\) est un polynôme cyclotomique.
\end{theorem}

\begin{proof}
    Nous supposons que \( X\neq 0\), et nous notons \( P=\sum_ia_iX^i\). Étant donné que \( P\) est irréductible et différent de \( X\), nous avons \( a_0\neq 0\) (sinon \( x=0\) serait une racine). Nous allons montrer que les racines de \( P\) sont toutes des racines \( N\)-ièmes de l'unité (avec le même \( N\) pour toutes).

    Soient \( \{ \xi_i \}_{i=1,\ldots, d}\) les racines de \( P\); on a
    \begin{equation}
        P=\prod_{i=1}^d(X-\xi_i)
    \end{equation}
    avec \( \prod_{i=1}^d\xi_i=a_0\). Par hypothèse, \( | \xi_i |\leq 1\) et donc \( 0<| a_0 |\leq 1\). Vu que \( P\in \eZ[X]\) nous avons donc \( a_0=1\) et donc \( | \xi_i |=1\) pour tout \( i\).

    Nous introduisons les polynômes
    \begin{equation}
        g_q(X)=\prod_{i=1}^d\big( X-(\xi_i)^q \big),
    \end{equation}
    et en particulier \( g_1=P\), et nous développons
    \begin{equation}
        g_q(X)=X^n+C_{1,q}X^{n-1}+\ldots +C_{n,q}
    \end{equation}
    où
    \begin{equation}
        C_{k,q}=(-1)^k\sum_{1\leq i_1<\ldots<i_k\leq d}(\xi_{i_1}\ldots \xi_{i_k})^q.
    \end{equation}
    Nous introduisons aussi les polynômes
    \begin{equation}
        F_{k,q}(X_1,\ldots, X_n)=(-1)^k\sum_{1\leq i_1<\ldots< i_k\leq d}(X_{i_1}\ldots X_{i_k})^q
    \end{equation}
    qui sont des polynômes symétriques. Ils vérifient deux propriétés. La première est que
    \begin{equation}
        C_{r,q}=F_{r,q}(\xi_1,\ldots, \xi_n),
    \end{equation}
    et la seconde est que les polynômes \( F_{r,1}\) sont les polynômes symétriques élémentaires à un coefficients près. Le théorème \ref{TholReBiw} nous donne alors des polynômes \( G_{k,q}\in \eZ[X_1,\ldots, X_n]\) tels que
    \begin{equation}
        F_{k,q}(X_1,\ldots, X_n)=G_{k,q}\big( F_{1,1}(X_1,\ldots, X_n),\ldots, F_{k,1}(X_1,\ldots, X_n) \big).
    \end{equation}
    Nous savons que
    \begin{equation}
        | C_{k,q} |\leq \sum_{1\leq i_1<\ldots<i_k<d}1={d\choose k}.
    \end{equation}
    Donc \( g_q\) fait partie de l'ensemble fini des polynômes dans \( \eZ[q]\) dont tous les coefficients sont bornée en valeur absolue par 
    \begin{equation}
        \max_{k=1,\ldots, d}{d\choose k}.
    \end{equation}
    Il existe un certain nombre d'ensembles \( \{ \xi_i \}\) qui sont racines de polynômes vérifiant les conditions du théorème. À chacun de ces ensembles est associé une suite de polynômes \( g_q\) et donc des coefficients \( C_{k,q}\). Ce que nous avons vu est que l'ensemble de tous les coefficients \( C_{k,q}\) possibles (pour un choix donné des \( \{ \xi_i \}\)) est fini, en particulier, vu que \( C_{1,q}=\sum_i\xi_i^q\), pour chaque \( k\), l'ensemble
    \begin{equation}
        \{ \xi_k^q\tq q\in \eN \}.
    \end{equation}
    Par le principe des tiroirs, il existe \( q_1\) et \( q_2\) tels que \( \xi_k^{q_1}=\xi_k^{q_2}\). Ici, \( q_1\) et \( q_2\) dépendent de \( k\) et nous notons \( N_k=q_1-q_2\); nous avons donc \( \xi_k^{N_k}=1\).

    En posant \( N=\ppcm(N_1,\ldots, N_d)\), nous avons
    \begin{equation}
        \xi_k^N=1
    \end{equation}
    pour tout \( k\).

    Mais \( P\) est irréductible dans \( \eZ[X]\); si il a \( \pm 1\) comme racines, alors c'est que \( P=X+1\) ou \( P=X-1\) et ce sont des polynômes cyclotomiques. Si \( P\) n'a pas \( \pm 1\) parmi ses racines, alors \( P\) n'a pas de racines dans \( \eQ\) parce que \( \pm 1\) sont les seules racines de \( X^N-1\) dans \( \eQ\).

    Par conséquent \( P\) est un facteur irréductible de \( X^N-1\) dans \( \eQ[X]\). Mais étant donné que
    \begin{equation}
        X^N-1=\prod_{d\divides N}\phi_d(X),
    \end{equation}
    les polynômes cyclotomiques sont les seuls facteurs irréductibles de \( X^N-1\). Donc \( P\) est un polynôme cyclotomique.
\end{proof}

%---------------------------------------------------------------------------------------------------------------------------
\subsection{Nombres premiers}
%---------------------------------------------------------------------------------------------------------------------------

\begin{lemma}[\cite{naKXuR}]    \label{LemiAqLEn}
    Soit \( n\geq 1\). Il existe un nombre premier \( p\) et un entier \( a\) tels que
    \begin{enumerate}
        \item
            \( p\) divise \( \phi_n(a)\),
        \item
            \( p\) ne divise aucun de \( \phi_d(a)\) avec \( d\divides n\) et \( d\neq n\).
    \end{enumerate}
    De tels \( p\) et \( a\) vérifient automatiquement
    \begin{enumerate}
        \item
            \( p\) divise \( a^n-1\),
        \item
            \( p\) ne divise aucun des \( a^d-1\) pour \( d\divides n\), \( d\neq n\).
    \end{enumerate}
\end{lemma}

\begin{proof}
    Nous posons
    \begin{equation}
        B(X)=\prod_{_{\substack{d\divides n\\d\neq n}}}\phi_d(X),
    \end{equation}
    et nous commençons par montrer que \( \phi_n\) est premier avec \( B\). Nous avons \( X^n-1=B\phi_n\), donc \( B\) et \( \phi_n\) n'ont pas de racines communes (même pas dans \( \eC\)) parce que ce serait une racine double de \( X^n-1\). Notons que par définition \ref{EqLjGYKK}, les polynômes cyclotomiques sont scindés (dans \( \eC\)), donc en particulier les polynômes \( \phi_n\) et \( B\) sont scindés et dons premiers entre eux, dans \( \eC\) et a fortiori dans \( \eQ\). Par Bézout (corollaire \ref{CorimHyXy}), il existe \( U,V\in\eQ[X]\) tels que
    \begin{equation}
        U\phi_n+VB=1.
    \end{equation}
    Si nous prenons \( a\in \eZ\) tel que \( U'=aU\) et \( V'=aV\) soient tous deux dans \( \eZ[X]\), alors nous avons
    \begin{equation}    \label{EqCpNMEi}
        U'\phi_n+V'B=a,
    \end{equation}
    égalité dans \( \eZ[X]\). Quitte à prendre un multiple assez grand de \( a\), nous pouvons choisir \( a\) de telle sorte que \( | \phi_n(a) |\geq 2\). Nous prenons alors un nombre premier \( p\) divisant \( \phi_n(a)\). 

    Montrons que le \( a\) et le \( p\) ainsi construis satisfont aux exigences.

    Vu que \( X^n-1=B\phi_n\), si \( p\) divise \( \phi_n(a)\), il divise automatiquement \( a^n-1\) et donc \( [a^n]_p=1\), ce qui signifie entre autres que \( a\) et \( p\) sont premiers entre eux. Évaluons l'équation \eqref{EqCpNMEi} en~\( a\) :
    \begin{equation}
        U'(a)\phi_n(a)+V'(a)B(a)=a.
    \end{equation}
    Le nombre \( p\) ne divisant pas \( a\), mais divisant \( \phi_n(a)\), il ne peux pas diviser \( B(a)\)\footnote{C'est pour pouvoir dire ça que l'on a choisit \( V'\in \eZ[X]\) de telle sorte que \( V'(a)\) soit dans \( \eZ\)}. Étant donné que \( p\) ne divise pas \( B(a)\), il ne divise aucun des \( \phi_d(a)\) avec \( d\divides n\) et \( d\neq n\).

    Nous passons maintenant à la seconde partie de la preuve. Nous supposons avoir \( a\) et \( p\) tels que \( p\) soit un nombre premier divisant \( \phi_n(a)\) et tels que \( p\) ne divise aucun des \( \phi_d(a)\) avec \( d\divides n\), \( d\neq n\). Le fait de diviser \( \phi_n(a)\) entraine le fait de diviser \( a^n-1\) parce que \( \phi_n\) est un des facteurs de \( X^n-1\). Soit maintenant \( d\neq n\) divisant \( n\); nous avons
    \begin{equation}    \label{EqwTWcCu}
        X^d-1=\prod_{d'\divides d}\phi_{d'},
    \end{equation}
    et cela est une partie du produit
    \begin{equation}
        \prod_{\substack{d\divides n\\d\neq n}}\phi_d.
    \end{equation}
    Vu que \( p\) ne divise aucun des \( \phi_d(a)\) de ce dernier produit, a fortiori, il ne divise pas le produit \ref{EqwTWcCu}, et donc pas \( a^d-1\).
\end{proof}

\begin{lemma}       \label{LemrZnmpG}
    Si \( n\geq 1\), alors il existe un nombre premier dans \( [1]_n\), c'est à dire un nombre premier de la forme \( 1+kn\) avec \( k\in \eN^*\). 
\end{lemma}

\begin{proof}
    Soit \( n\geq 1\) et les nombres \( p,a\) donnés par le lemme \ref{LemiAqLEn}. Vu que \( p\) divise \( \phi_n(a)\), \( p\) divise \( a^n-1\) et donc \( [a]_p\) a un ordre qui divise \( n\) dans \( (\eZ/p\eZ)^*\) parce que \( [a]_p^n=[1]_p\).

    Prenons \( d\neq n\) divisant \( n\). Nous savons que
    \begin{equation}
        a^d-1=\prod_{d'\divides d}\phi_{d'}(a).
    \end{equation}
    
    Par construction de \( a\) et \( p\), nous avons
    \begin{equation}
        [\phi_{d'}(a)]_p\neq 0
    \end{equation}
    Vu que \( \eZ/p\eZ\) est intègre, le produit est également non nul, c'est à dire
    \begin{equation}
        \big[ \prod_{d'\divides d}\phi_{d'}(a) \big]_p\neq 0,
    \end{equation}
    et donc \( [a]_p^a\neq 1\). Nous avons donc montré que si \( d\neq n\) divise \( n\), alors nous avons en même temps
    \begin{equation}
        [a]_p^n=1
    \end{equation}
    et
    \begin{equation}
        [a]_p^d\neq 1.
    \end{equation}
    Cela prouve que \( [a]_p\) est d'ordre exactement \( n\). Oui, mais l'ordre de \( [a]_p\) doit diviser l'ordre du groupe \( \eZ/p\eZ\) qui est \( p-1\), donc \( n\) divise \( p-1\) et nous écrivons \( p=kn+1\) avec \( k\) entier.
\end{proof}

\begin{theorem}[Forme faible du théorème de Dirichlet \cite{fJhCTE}]    \label{ThoxwTjcl}   \index{nombre!premier}\index{Dirichlet!théorème (sur les nombres premiers)}\index{théorème!Dirichlet!forme faible}
    Pour tout \( n\geq 1\), il existe une infinité de nombres premiers dans \( [1]_n\).
\end{theorem}

\begin{proof}
    Le lemme \ref{LemrZnmpG} nous donne déjà l'existence de nombres premiers dans \( [1]_n\). Il faut maintenant voir qu'il y en a une infinité. Nous supposons qu'il y en ait seulement un nombre fini : \( p_1,\ldots, p_r\), et nous notons 
    \begin{equation}
        N=np_1\ldots p_r.
    \end{equation}
    Nous utilisons maintenant le lemme \ref{LemrZnmpG} avec ce \( N\), c'est à dire qu'on a un nombre premier de la forme
    \begin{equation}
        p=1+kN=1+knp_1\ldots p_r.
    \end{equation}
    Cela est un nombre premier plus grand que tous les \( p_i\) et de la forme \( 1+\lambda n\). Cela contredit l'exhaustivité de la liste \( p_1,\ldots, p_r\).
\end{proof}

%---------------------------------------------------------------------------------------------------------------------------
\subsection{Le jeu de la roulette}
%---------------------------------------------------------------------------------------------------------------------------
\label{pTqJLY}

Source : \cite{HEBOFl}.

Soit une roulette à \( n\) secteurs que nous voulons colorier en \( q\) couleurs. Nous voulons savoir le nombre de possibilités à rotations près. Soit d'abord \( E\) l'ensemble des coloriages possibles sans contraintes; il y a naturellement \( q^n\) possibilités. Sur l'ensemble \( E\), le groupe cyclique \( G\) des rotations d'angle \( 2\pi/n\) agit. Deux coloriages étant identiques si ils sont reliés par une rotation, la réponse à notre problème est donné par le nombre d'orbites de l'action de \( G\) sur \( E\) qui sera donnée par la formule de Burnside \ref{EqTUsblv}. 

Nous devons calculer \( \Card\big( \Fix(g) \big)\) pour tout \( g\in G\). Soit \( g\), un élément d'ordre \( d\) dans \( G\). Si \( g\) agit sur la roulette, chaque secteur a une orbite contenant \( d\) éléments. Autrement dit, \( g\) divise la roulette en \( n/d\) secteurs. Un élément de \( E\) appartenant à \( \Fix(g)\) doit colorier ces \( n/d\) secteurs de façon uniforme; il y a \( q^{n/d}\) possibilités.

Il reste à déterminer le nombre d'éléments d'ordre \( d\) dans \( G\). Un élément de \( G\) est donné par un nombre complexe de la forme \(  e^{2ik\pi/n}\). Les éléments d'ordre \( d\) sont les racines primitives\footnote{Une racine non primitive \( 8\)ième de l'unité est par exemple \( i\). Certes \( i^8=1\), mais \( i^4=1\) aussi. Le nombre \( i\) est d'ordre \( 4\).} \( d\)ièmes de l'unité. Nous savons que --par définition-- il y a \( \varphi(d)\) telles racines primitives de l'unité. Bref il y a \( \varphi(d)\) éléments d'ordre \( d\) dans \( G\). 

La formule de Burnside nous donne maintenant le nombre d'orbites :
\begin{equation}
    \frac{1}{ n }\sum_{d|n}\varphi(d)q^{n/d}.
\end{equation}
Cela est le nombre de coloriage possibles de la roulette à \( n\) secteurs avec \( q\) couleurs.

%---------------------------------------------------------------------------------------------------------------------------
\subsection{L'affaire du collier}
%---------------------------------------------------------------------------------------------------------------------------
\label{siOQlG}

Nous avons maintenant des perles de \( q\) couleurs différentes et nous voulons en faire un collier à \( n\) perles. Cette fois non seulement les rotations donnent des colliers équivalents, mais en outre les symétries axiales (il est possible de retourner un collier, mais pas une roulette). Le groupe agissant sur \( E\) est maintenant le groupe diédral\index{diédral}\index{groupe!diédral} \( D_n\) conservant un polygone a \( n\) sommets.

Nous devons séparer le cas \( n\) impair du cas \( n\) pair.

Si \( n\) est impair, alors les axes de symétries passent par un sommet par le milieu du côté opposé. Le groupe \( D_n\) contient \( n\) symétries axiales. Nous avons donc maintenant
\begin{equation}
    | G |=2n.
\end{equation}
Nous écrivons la formule de Burnside
\begin{equation}
    \Card(\Omega)=\frac{1}{ 2n }\sum_{g\in G}\Card\big( \Fix(g) \big).
\end{equation}
Si \( g\) est une rotation, le travail est déjà fait. Si \( g\) est une symétrie, nous avons le choix de la couleur du sommet par lequel passe l'axe et le choix de la couleur des \( (n-1)/2\) paires de sommets. Cela fait
\begin{equation}
    qq^{(n-1)/2}=q^{\frac{ n+1 }{2}}
\end{equation}
possibilités. Nous avons donc
\begin{equation}
    \Card(\Omega)=\frac{1}{ 2n }\left( \sum_{d|n}q^{n/d}\varphi(d)+nq^{\frac{ n+1 }{2}} \right).
\end{equation}

Si \( n\) est pair, le choses se compliquent un tout petit peu. En plus de symétries axiales passant par un sommet et le milieu du côté opposé, il y a les axes passant par deux sommets opposés. Pour colorier un collier en tenant compte d'une telle symétrie, nous pouvons choisir la couleur des deux perles par lesquelles passe l'axe ainsi que la couleur des \( (n-2)/2\) paires de perles. Cela fait en tout
\begin{equation}
    q^2q^{\frac{ n-2 }{2}}=q^{\frac{ n+2 }{2}}.
\end{equation}
Le groupe \( G\) contient \( n/2\) tels axes.

Notons que cette fois \( G\) ne contient plus que \( n/2\) symétries passant par un sommet et un côté. L'ordre de $G$ est donc encore \( 2n\). La formule de Burnside donne
\begin{equation}
    \Card(\Omega)=\frac{1}{ 2n }\left( \sum_{d\divides n}\varphi(d)q^{n/d}+\frac{ n }{2}q^{(n+2)/2}+\frac{ n }{2}q^{n/2} \right).
\end{equation}

%---------------------------------------------------------------------------------------------------------------------------
\subsection{Théorème de Wedderburn}
%---------------------------------------------------------------------------------------------------------------------------

\begin{theorem}[\href{http://www.les-mathematiques.net/d/a/w/node5.php}{Théorème de Wedderburn}]    \label{ThoMncIWA}
    Tout corps fini est commutatif.
\end{theorem}

\begin{proof}
    Soit \( \eK\) un corps fini et \( Z\), le centre de \( \eK\). Ce dernier est un corps fini et un sous corps de \( \eK\). Si \( q=\Card(Z)\) alors par le lemme \ref{LemobATFP} nous avons
    \begin{equation}
        \Card(\eK)=q^n
    \end{equation}
    pour un certain \( n\).

    Nous supposons maintenant que \( \eK\) est non commutatif. Dans ce cas \( Z\neq \eK\) et nous avons \( n\geq 2\). Nous considérons aussi
    \begin{equation}
        Z_x=\{ a\in \eK\tq ax=xa \}.
    \end{equation}
    Le centre \( Z\) est un sous corps de \( Z_x\), donc il existe \( d(x)\) tel  que
    \begin{equation}
        \Card(Z_x)=q^{d(x)}.
    \end{equation}
    De la même manière, \( Z_x\) est un sous corps de \( \eK\), donc il existe \( m(x)\) tel que
    \begin{equation}
        \Card(\eK)=\Card(Z_x)^{m(x)}.
    \end{equation}
    En mettant bout à bout nous avons
    \begin{equation}
        q^n=\Card(Z_x)^{m(x)}=q^{d(x)m(x)},
    \end{equation}
    et par conséquent \( n=d(x)m(x)\). Le point important à retenir est que \( d(x)\) divise \( n\) pour tout \( x\in \eK\).

    Nous considérons maintenant l'action adjointe du groupe \( \eK^*\) sur lui-même :
    \begin{equation}
        \varphi(k)x=kxk^{-1}.
    \end{equation}
    Nous notons \( \mO_x\) l'orbite de \( x\in \eK^*\) pour cette action, et \( \Stab(x)\) son stabilisateur. Nous avons
    \begin{equation}
        Z_y=\Stab(y)\cup\{ 0 \}
    \end{equation}
    parce que \( Z_y\) et \( \Stab(y)\) ont les mêmes définitions, sauf que \( \Stab(y)\) est dans \( \eK^*\) alors que \( Z_y\) est dans \( \eK\). Nous avons donc
    \begin{equation}
        \Card\big( \Stab(y) \big)=\Card(Z_y)-1=q^{d(y)}-1.
    \end{equation}
    Nous avons \( \Card(\mO_x)=1\) si et seulement si \( \mO_x=\{ x \}\) si et seulement si \( \Stab(x)=\eK^*\) si et seulement si \( z\in Z^*\). Soient \( z_0,\ldots, z_{q-1}\) les éléments de \( Z\) avec \( z_0=0\). Ce sont les éléments qui auront une orbite réduite à un point. Les orbites qui coupent \( Z^*\) sont
    \begin{equation}
        \{ z_1 \},\ldots, \{ z_{q-1} \}
    \end{equation}
    et il y en a \( q-1\). Soient \( \mO_{y_1},\ldots, \mO_{y_r}\), les autres orbites. Nous utilisons l'équation des classes \eqref{EqkgGmoq} :
    \begin{equation}
        \Card(\eK^*)=\Card(Z^*)+\sum_{i=1}^{r}\frac{ \Card(\eK^*) }{ \Card(\Stab(y_i)) },
    \end{equation}
    mais \( \Card(Z^*)=q-1\), \( \Card(\eK^*)=q^n-1\) et \( \Card\big( \Stab(y_i) \big)=q^{d(y_i)}-1\), donc
    \begin{equation}        \label{EqBPBDzE}
        q^n-1=(q-1)+\sum_{i=1}^{r}\frac{ q^n-1 }{ q^{d(y_i)}-1 }.
    \end{equation}
    Nous considérons la fraction rationnelle
    \begin{equation}        \label{EqATGciu}
        F(X)=(X^n-1)-\sum_{i=1}^{r}\frac{ X^n-1 }{ X^{d(y_i)}-1 }.
    \end{equation}
    Étant donné que \( d(y_i)\) divise \( n\), nous avons, contrairement aux apparences, que \( F\in \eZ[X]\) par la proposition \ref{PropUImYnL}\ref{ItemhpDPKE}.

    Nous pouvons exploiter un peu mieux la proposition \ref{PropUImYnL} en remarquant que \( d(y_i)<n\) parce que sinon \( \Card(Z_{y_i})=\Card(\eK)\), ce qui signifierait que \( y_i\in Z\), ce qui nous avions exclu. Par conséquent le polynôme cyclotomique \( \phi_n\) divise 
    \begin{equation}
        \frac{ X^n-1 }{ X^{d(y_i)}-1 }
    \end{equation}
    dans \( \eZ[X]\). Le polynôme cyclotomique \( \phi_n\) divise également \( X^n-1\) et par conséquent \( \phi_n\) divise \( F\). Il existe donc \( Q\in \eZ[X]\) tel que \( F=Q\phi_n\). En particulier en évaluant en \( q\) :
    \begin{equation}    \label{eqmoLdJy}
        F(q)=Q(q)\phi_n(q)=q-1.
    \end{equation}
    En effet nous avons \( F(q)=q-1\) par construction : comparer \eqref{EqBPBDzE} avec \eqref{EqATGciu}. Évidemment \( q\neq 1\) parce que si \( q=1\) alors \( \Card(\eK)=1\) et le théorème est trivial. Par ailleurs \( Q(q)\) est un entier (parce que \( Q\in \eZ[X]\) et \( q\in \eN\)) et \( Q(q)\neq 0\), parce qu'à droite de \eqref{eqmoLdJy} nous avons \( q-1\neq 0\). Nous avons donc \( | Q(q) |\geq 1\) et donc
    \begin{equation}
        | \phi_n(q) |\leq q-1.
    \end{equation}
    Par définition du polynôme cyclotomique nous avons
    \begin{equation}
        | \phi_n(q) |=\prod_{z\in\Delta_n}| q-z |.
    \end{equation}
    Étant donné que ce produit doit être inférieur à \( q-1\), au moins un des termes doit l'être : il existe \( z_0\in \Delta_n\) tel que \( | z_0-q |\leq q-1\). Étant donné que \( n\geq 2\) nous avons \( z_0\neq 1\).

    Mais d'autre part, comme indiqué sur la figure \ref{LabelFigtrigoWedd}, la distance entre \( z_0\) et \( q\) doit être strictement plus grande que \( q-1\) parce que \( q-1\) est le minimum de la distance entre le cercle trigonométrique et \( q\), et n'est atteint qu'en \( z=1\).
    \newcommand{\CaptionFigtrigoWedd}{Nous devons avoir \( | z_0-q |>q-1\).}
    \input{Fig_trigoWedd.pstricks}

    Nous avons ainsi obtenu une contradiction, et nous concluons que le corps \( \eK\) est commutatif.
\end{proof}



%++++++++++++++++++++++++++++++++++++++++++++++++++++++++++++++++++++++++++++++++++++++++++++++++++++++++++++++++++++++++++++++++++++++++
\section{Applications multilinéaires}
%++++++++++++++++++++++++++++++++++++++++++++++++++++++++++++++++++++++++++++++++++++++++++++++++++++++++++++++++++++++++++++++++++++++++

\begin{definition}
 	Une application $T: \eR^{m_1}\times \ldots \times\eR^{m_k}\to\eR^p $ est dite $k$-linéaire si pour tout $X=(x_1, \ldots,x_k)$ dans $ \eR^{m_1}\times \ldots \times\eR^{m_k}$ les applications $x_i\mapsto T(x_1, \ldots, x_i,\ldots,x_k)$ sont linéaires pour tout $i$ dans $\{1,\ldots,k\}$, c'est à dire
	\begin{equation}
		\begin{aligned}[]
			T(\cdot,x_2, \ldots, x_i,\ldots,x_k)&\in \mathcal{L}(\eR^{m_1}, \eR^p),\\
			T(x_1,\cdot, \ldots, x_i,\ldots,x_k)&\in \mathcal{L}(\eR^{m_2}, \eR^p),\\
						& \vdots\\
			T(x_1, \ldots, x_i,\ldots,x_{k-1},\cdot)&\in \mathcal{L}(\eR^{m_k}, \eR^p).\\
		\end{aligned}
	\end{equation}
	En particulier lorsque $k=2$, nous parlons d'applications \defe{bilinéaires}{bilinéaire}. Vous pouvez deviner ce que sont les applications \emph{tri}linéaire ou \emph{quadri}linéaire.
\end{definition}

L'ensemble des applications $k$-linéaires de $ \eR^{m_1}\times \ldots \times\eR^{m_k}$ dans $\eR^p$ est noté $\mathcal{L}(\eR^{m_1}\times \ldots \times\eR^{m_k}, \eR^p)$ ou $\mathcal{L}(\eR^{m_1}, \ldots,\eR^{m_k}; \eR^p)$.

\begin{example}
  Soit $A$ une matrice avec $m$ lignes et $n$ colonnes. L'application bilinéaire de $\eR^m\times \eR^n$ dans $\eR$ associée à $A$ est définie par
\[
T_A(x,y)= x^TAy=\sum_{i,j}a_{i,j}x_i y_j, \qquad \forall x\in \eR^m, \, y \in \eR^n.
\]
\end{example}

\begin{definition}
	La norme sur l'espace $\mathcal{L}(\eR^{m_1}\times \ldots \times\eR^{m_k}, \eR^p)$ des fonction $k$-linéaires et continues est donnée par le meilleur $L$ possible, plus précisément elle est  définie par 
	\begin{equation}
		\|T\|_{\mathcal{L}(\eR^{m_1}\times \ldots \times\eR^{m_k}, \eR^p)}=\sup\{ \|T(u_1, \ldots,u_k)\|_p\,\vert\,\|u_i\|_{m_i}\leq 1, i=1,\ldots, k \}.
	\end{equation}
\end{definition}

\begin{proposition}
  L'application $k$-linéaire  $T: \eR^{m_1}\times \ldots \times\eR^{m_k}\to\eR^p $ est continue si et seulement s'il existe $L\geq 0$, réel, tel que
  \begin{equation}\label{limitatezza}
     \|T(x_1, \ldots,x_k)\|_p\leq L \|x_1\|_{m_1}\cdots\|x_k\|_{m_k}, \qquad \forall x_i\in\eR^{m_i},\,\forall i \in \{1,\ldots, k\}.
  \end{equation}
\end{proposition}

\begin{proof}
  Pour simplifier l'exposition nous nous limitons au cas $k=2$. On adopte la notation $T(x,y)=x*y$

Supposons que l'inégalité \eqref{limitatezza} soit satisfaite. 
\begin{equation}\label{LimImplCont}
  \begin{aligned}
    \|x*y-x_0*y_0\|_p&=\|(x-x_0)*y-x_0*(y-y_0)\|_p\leq\\
&\leq \|(x-x_0)*y\|_p+\|x_0*(y-y_0)\|_p\leq\\
&\leq L\|x-x_0\|_m\|y\|_n + L\|x_0\|_m\|y-y_0\|_n.
  \end{aligned}
\end{equation}
Si $x\to x_0$ et $y\to y_0$  on voit que $T$ est continue en passant à la limite aux deux côtes de l'inégalité \eqref{LimImplCont}.

Soit $T$ continue en $(0_m,0_n)$. Évidemment $0_m*0_n=0_p$, donc il existe $\delta>0$ tel que si $x$ est dans la boule de rayon $\delta$ centrée en $0_m$ et  $y$ est dans la boule de rayon $\delta$ centrée en $0_n$ alors $\|x*y\|_p\leq 1$. Soient maintenant  $x$ dans $\eR^m\setminus\{ 0_m\}$ et $y$ dans $\eR^n\setminus\{ 0_n\}$
\begin{equation}
  \begin{aligned}
    x*y=&\left(\frac{\|x\|_m}{\delta}\frac{\delta x}{\|x\|_m}\right)*\left(\frac{\|y\|_n}{\delta}\frac{\delta y}{\|y\|_n}\right)=\\
&=\frac{\|x\|_m\|y\|_n}{\delta^2} \left(\frac{\delta x}{\|x\|_m}\right)*\left(\frac{\delta y}{\|y\|_n}\right).
  \end{aligned}
 \end{equation}
On remarque que $\delta x/\|x\|_m$ est dans la boule de rayon $\delta$ centrée en $0_m$ et que $\delta y/\|y\|_n$ est dans la boule de rayon $\delta$ centrée en $0_n$. On conclut 
\[
 x*y\leq \frac{\|x\|_m\|y\|_n}{\delta^2}.
\]
Il faut prendre $L=1/\delta^2$.
\end{proof}
\begin{proposition}\label{isom_isom}
  On définit les fonctions
  \begin{equation}
    \begin{array}{rccc}
      \omega_g: & \mathcal{L}(\eR^{m}\times\eR^{n}, \eR^p)&\to &\mathcal{L}(\eR^{m}, \mathcal{L}(\eR^{n}, \eR^p)),\\
      \omega_d: & \mathcal{L}(\eR^{m}\times\eR^{n}, \eR^p)&\to &\mathcal{L}(\eR^{n}, \mathcal{L}(\eR^{m}, \eR^p)),
    \end{array}
  \end{equation}
par 
\[
\omega_g(T)(x)=T(x,\cdot), \qquad \forall x\in\eR^m,
\]
et
\[
\omega_d(T)(y)=T(\cdot, y), \qquad \forall y\in\eR^n.
\]
Les fonctions $\omega_g$ et $\omega_d$ sont des isomorphismes qui préservent les normes.    
\end{proposition}




%+++++++++++++++++++++++++++++++++++++++++++++++++++++++++++++++++++++++++++++++++++++++++++++++++++++++++++++++++++++++++++
\section{Endomorphismes}
%+++++++++++++++++++++++++++++++++++++++++++++++++++++++++++++++++++++++++++++++++++++++++++++++++++++++++++++++++++++++++++

%---------------------------------------------------------------------------------------------------------------------------
\subsection{Polynôme caractéristique}
%---------------------------------------------------------------------------------------------------------------------------

Soit \( A\) un anneau commutatif et \( \eK\), un corps commutatif. L'injection canonique \( A\to A[X]\) se prolonge en une injection
\begin{equation}
   \eM(A)\to\eM\big( A[X] \big).
\end{equation}
Si \( u\in\eM_n(A)\), nous définissons le \defe{polynôme caractéristique de \( u\)}{polynôme!caractéristique}\index{caractéristique!polynôme} :
\begin{equation}    \label{Eqkxbdfu}
    \chi_u(X)=\det(X\mtu_n-u).
\end{equation} 
Ce faisons nous assimilons la matrice \( u\) et l'endomorphisme \( u\colon E\to E\) qu'elle définit. 

\begin{lemma}
    Si \( u\) est un endomorphisme
    \begin{equation}
        I_u=\{ P\in \eK[X] \tq P(u)=0\}
    \end{equation}
    n'est pas vide.
\end{lemma}

\begin{proof}
    Nous avons un morphisme d'algèbre
    \begin{equation}
        \begin{aligned}
            \varphi_u\colon\eK[X]&\to \End(E) \\
            P&\mapsto P(u). 
        \end{aligned}
    \end{equation}
    Cet endomorphisme ne peut pas être injectif parce que \(\eK[X]\) est de dimension infinie tandis que \( \End(E)\) est de dimension finie. Il possède donc un noyau, c'est à dire qu'il existe \( P\in\eK[X]\) tel que \( P(X)=0\).
\end{proof}

\begin{definition}
    Le \defe{polynôme minimal}{polynôme!minimal!d'un endomorphisme}\index{minimal!polynôme!d'endomorphisme} de \( u\) est le générateur unitaire de \( I_u\). C'est le polynôme unitaire de plus petit degré qui annule \( u\). Nous le notons \( \mu_u\)\nomenclature[A]{\( \mu_A\)}{polynôme minimal de \( A\)} :
\begin{equation}
    \mu_u(u)=0.
\end{equation}
\end{definition}

\begin{lemma}
    Le polynôme \( \chi_u\) est unitaire et de degré \( n\).
\end{lemma}

\begin{lemma}       \label{LemjcztYH}
    Soit \( u\) un endomorphisme et \( E_{\lambda}(u)\)\nomenclature[A]{\( E_{\lambda}(u)\)}{Espace propre de \( u\)} ses espaces propres. La somme des \( V_{\lambda}\) est directe.
\end{lemma}

\begin{proof}
    Soit \( v_i\in V_{\lambda_i}\) un choix de vecteurs propres de \( u\). Si la somme n'est pas directe, nous pouvons considérer une combinaison linéaire des \( v_i\) qui soit nulle :
    \begin{equation}
        v_1+\ldots+v_p=0.
    \end{equation}
    Appliquons \( (A-\lambda_1\mtu)\) à cette égalité :
    \begin{equation}
        (\lambda_2-\lambda_1)v_1+\ldots+(\lambda_p-\lambda_1)v_p=0.
    \end{equation}
    En appliquant encore successivement les opérateurs \( (A-\lambda_i\mtu)\) nous réduisons le nombre de termes jusqu'à obtenir \( v_p=0\).
\end{proof}


\begin{theorem}     \label{ThoNhbrUL}
    Soit \( E\) un \(\eK\)-espace vectoriel de dimension finie \( n\) et un endomorphisme \( u\in\End(E)\). Alors
    \begin{enumerate}
        \item
            Le polynôme caractéristique divise \( (\mu_u)^n\) dans \(\eK[X]\).
        \item
            Les polynômes caractéristiques et minimaux ont mêmes facteurs irréductibles dans \(\eK[X]\).
        \item
            Les polynômes caractéristiques et minimaux ont mêmes racines dans \(\eK[X]\).
        \item
            Le polynôme caractéristique est scindé si et seulement si le polynôme minimal est scindé.
    \end{enumerate}
\end{theorem}


Si \( \lambda\in\eK\) est une racine de \( \chi_u\), l'ordre de l'annulation est la \defe{multiplicité algébrique}{multiplicité!algébrique d'une valeur propre} de la valeur propre \( \lambda\) de \( u\).

\begin{theorem}
    Soit \( u\in\End(E)\) et \( \lambda\in\eK\). Les conditions suivantes sont équivalentes
    \begin{enumerate}
        \item\label{ItemeXHXhHi}
            \( \lambda\in\Spec(u)\)
        \item\label{ItemeXHXhHii}
            \( \chi_u(\lambda)=0\)
        \item\label{ItemeXHXhHiii}
            \( \mu_u(\lambda)=0\).
    \end{enumerate}
\end{theorem}

\begin{proof}
    \ref{ItemeXHXhHi} \( \Leftrightarrow\) \ref{ItemeXHXhHii}. Dire que \( \lambda\) est dans le spectre de \( u\) signifie que l'opérateur \( u-\lambda\mtu\) n'est pas inversible, ce qui est équivalent à dire que \( \det(u-\lambda\mtu)\) est nul ou encore que \( \lambda\) est une racine du polynôme caractéristique de \( u\). 

    \ref{ItemeXHXhHii} \( \Leftrightarrow\) \ref{ItemeXHXhHiii}. Cela est une application directe du théorème \ref{ThoNhbrUL} qui précise que le polynôme caractéristique a les mêmes racines dans \(\eK\) que le polynôme minimal.
\end{proof}

\begin{lemma}
    Une matrice triangulaire supérieure avec des \( 1\) sur la diagonale n'est diagonalisable que si elle est diagonale (c'est à dire si elle est la matrice unité).
\end{lemma}

\begin{proof}
    Si \( A\) est une matrice triangulaire supérieure de taille \( n\) telle que \( A_{ii}=1\), alors \( \det(A-\lambda\mtu)=(1-\lambda)^n\), ce qui signifie que \( \Spec(A)=\{ 1 \}\). Pour la diagonaliser, il faudrait une matrice \( P\in\GL(n,\eK)\) telle que \( \mtu=P^{-1}AP\), ce qui est uniquement possible si \( A=\mtu\).
\end{proof}


%---------------------------------------------------------------------------------------------------------------------------
\subsection{Matrices semblables}
%---------------------------------------------------------------------------------------------------------------------------

Sur l'ensemble \( \eM_n(\eK)\) des matrices \( n\times n\) à coefficients dans \(\eK\) nous introduisons la relation d'équivalence \( A\sim B\) si et seulement si il existe une matrice \( P\in\GL(n,\eK)\) telle que \( B=P^{-1}AP\). Deux matrices équivalentes en ce sens sont dites \defe{semblables}{semblables!matrices}.

Le polynôme caractéristique est un invariant sous les similitudes. En effet si \( P\) est une matrice inversible,
\begin{subequations}
    \begin{align}
        \chi_{PAP^{-1}}&=\det(PAP^{-1}-\lambda X)\\
        &=\det\big( P^{-1}(PAP^{-1}-\lambda X)P^{-1} \big)\\
        &=\det(A-\lambda X).
    \end{align}
\end{subequations}

La permutation de lignes ou de colonnes ne sont pas de similitudes, comme le montrent les exemples suivants :
\begin{equation}
    \begin{aligned}[]
        A&=\begin{pmatrix}
            1    &   2    \\ 
            3    &   4    
        \end{pmatrix}&
        B&=\begin{pmatrix}
            2    &   1    \\ 
            4    &   3    
        \end{pmatrix}.
    \end{aligned}
\end{equation}
Nous avons \( \chi_A=x^2-5x-2\) tandis que \( \chi_B=x^2-5x+2\) alors que le polynôme caractéristique est un invariant de similitude.

%---------------------------------------------------------------------------------------------------------------------------
\subsection{Polynômes d'endomorphismes}
%---------------------------------------------------------------------------------------------------------------------------

Soit \( u\in\End(E)\) où \( E\) est un \( \eK\)-espace vectoriel. Nous considérons l'application
\begin{equation}
    \begin{aligned}
        \varphi_u\colon \eK[X]&\to \End(E) \\
        P&\mapsto P(u). 
    \end{aligned}
\end{equation}
L'image de \( \varphi_u\) est un sous-espace vectoriel. En effet si \( A=\varphi_u(P)\) et \( B=\varphi_u(Q)\), alors \( A+B=\varphi_u(P+Q)\) et \( \lambda A=(\lambda P)(u)\). En particulier c'est un espace fermé.




Soit \( u\) un endomorphisme d'un \( \eK\)-espace vectoriel \( E\) et \( P\), un polynôme. Nous disons que \( P\) est un polynôme \defe{annulateur}{polynôme!annulateur} de \( u\) si \( P(u)=0\) en tant que endomorphisme de \( E\).

\begin{lemma}       \label{LemQWvhYb}
    Si \( P\) et \( Q\) sont des polynômes dans \( \eK[X]\) et si \( u\) est un endomorphisme d'un \( \eK\)-espace vectoriel \( E\), nous avons
    \begin{equation}
        (PQ)(u)=P(u)\circ Q(u).
    \end{equation}
\end{lemma}

\begin{proof}
    Si \( P=\sum_i a_iX^i\) et \( Q=\sum_j b_jX^j\), alors le coefficient de \( X^k\) dans \( PQ\) est
    \begin{equation}        \label{EqCoefGPyVcv}
        \sum_la_lb_{k-l}.
    \end{equation}
    Par conséquent \( (PQ)(u)\) contient \( \sum_la_lb_{k-l}u^k\). Par ailleurs \( P(u)\circ Q(u)\) est donné par
    \begin{equation}
        \sum_ia_iu^i\left( \sum_jb_ju^j \right)(x)=\sum_{ij}a_ib_ju^{i+j}(x).
    \end{equation}
    Le coefficient du terme en \( u^k\) est bien le même que celui donné par \eqref{EqCoefGPyVcv}.
\end{proof}

\begin{theorem}[Décomposition des noyaux ou lemme des noyaux]\index{lemme!des noyaux}       \label{ThoDecompNoyayzzMWod}
    Soit \( u\) un endomorphisme du \( \eK\)-espace vectoriel \( E\). Soit \( P\in\eK[X]\) un polynôme tel que \( P(u)=0\). Nous supposons que \( P\) s'écrive comme le produit \( P=P_1\ldots P_n\) de polynômes deux à deux étrangers. Alors
    \begin{equation}
        E=\ker P_1(u)\oplus\ldots\oplus\ker P_n(u).
    \end{equation}
    De plus les projecteurs associés à cette décomposition sont des polynômes en \( u\).
\end{theorem}
Ce lemme est utilisé pour prouver que toute représentation est décomposable en représentations irréductibles, proposition \ref{PropHeyoAN}.

\begin{proof}
    Nous posons 
    \begin{equation}
        Q_i=\prod_{j\neq i}P_i.
    \end{equation}
    Par le lemme \ref{LemuALZHn} ces polynômes sont étrangers entre eux et le théorème de Bézout (théorème \ref{ThoBezoutOuGmLB}) donne l'existence de polynômes \( R_i\) tels que
    \begin{equation}
        R_1Q_1+\ldots+R_nQ_n=1.
    \end{equation}
    Si nous appliquons cette égalité à \( u\) et ensuite à \( x\in E\) nous trouvons
    \begin{equation}        \label{EqqVcpUy}
        \sum_{i=1}^n(R_iQ_i)(u)(x)=x,
    \end{equation}
    et en particulier si nous posons \( E_i=\Image\big(P_iQ_i(u)\big)\) nous avons
    \begin{equation}
        E=\sum_{i=1}^nE_i.
    \end{equation}
    Cette dernière somme n'est éventuellement pas une somme directe. Si \( i\neq j\), alors \( Q_iQ_j\) est multiple de \( P\) et nous avons, en utilisant le lemme \ref{LemQWvhYb}, 
    \begin{equation}
        (R_iQ_i)(u)\circ (R_jQ_j)(u)=\big( R_iQ_iR_jQ_j \big)(u)=S_{ij}(u)\circ P(u)=0
    \end{equation}
    où \( S_{ij}\) est un polynôme. 

    Nous pouvons voir \( E\) comme un \( \eK\)-module et appliquer le théorème \ref{ThoProjModpAlsUR}. Les opérateurs \( R_iQ_i(u)\) ont l'identité comme somme et sont orthogonaux, et nous avons donc la décomposition en somme directe :
    \begin{equation}
        E=\bigoplus_{i=1}^nR_iQ_i(u)E.
    \end{equation}

    Afin de terminer la preuve, nous devons montrer que \( R_iQ_i(u)E=\ker P_i(u)\). D'abord nous avons
    \begin{equation}
        P_iR_iQ_i(u)=(R_iP)(u)=R_i(u)\circ P(u)=0,
    \end{equation}
    par conséquent \( \Image(R_iQ_i(u))\subset \ker P_i(u)\). Pour obtenir l'inclusion inverse, nous reprenons l'équation \eqref{EqqVcpUy} avec \( x\in\ker P_i(u)\). Elle se réduit à
    \begin{equation}
        (R_iQ_i)(u)x=x.
    \end{equation}
    Par conséquent \( x\in\Image\big( R_iQ_i(u) \big)\).
\end{proof}

\begin{corollary}   \label{CorKiSCkC}
    Soit \( E\), un \( \eK\)-espace vectoriel de dimension finie et \( f\), un endomorphisme semi-simple dont la décomposition du polynôme minimal \( \mu_f\) en facteurs irréductibles sur \( \eK[X]\) est \( \mu_f=M_1^{\alpha_1}\cdots M_r^{\alpha_r}\). Si \( F\) est un sous-espace stable par \( f\), alors
    \begin{equation}
        F=\bigoplus_{i=1}^r\ker M_i^{\alpha_i}(f)\cap F
    \end{equation}
\end{corollary}

\begin{proof}
    Nous posons \( E_i=\ker M_i^{\alpha_i}(f)\) et \( F_i=E_i\cap F\). Les polynômes \( M_i^{\alpha_i}\) sont deux à deux étrangers et \( \mu_f(f)=0\), donc le lemme des noyaux (\ref{ThoDecompNoyayzzMWod}) s'applique et
    \begin{equation}
        E=E_1\oplus\ldots\oplus E_r.
    \end{equation}
    Nous pouvons décomposer \( x\in F\) en termes de cette somme :
    \begin{equation}     \label{EqbBbrdi}
        x=x_1+\ldots +x_r
    \end{equation}
    avec \( x_i\in E_i\). Toujours selon le lemme des noyaux, les projections sur les espaces \( E_i\) sont des polynômes en \( f\). Par conséquent \( F\) est stable sous toutes ces projections \( \pr_i\colon E\to E_i\), et en appliquant \( \pr_i\) à \eqref{EqbBbrdi}, \( \pr_i(x)=x_i\). Vu que \( x\in F\), le membre de gauche est encore dans \( F\) et \( x_i\in E_i\cap F\). Nous avons donc
    \begin{equation}
        F\subset\bigoplus_{i=1}^rF_i.
    \end{equation}
    L'inclusion inverse est immédiate parce que \( F_i\subset F\) pour chaque \( i\).
\end{proof}

%---------------------------------------------------------------------------------------------------------------------------
\subsection{Polynôme minimal ponctuel}
%---------------------------------------------------------------------------------------------------------------------------

\begin{definition}  \label{Decyyumy}
    Soit \( E\), un espace vectoriel et \( f\colon E\to E\) un endomorphisme de \( E\). Pour chaque \( x\in E\) nous considérons l'idéal
    \begin{equation}
        I_{f,x}=\{ P\in \eK[X]\tq P(f)x=0 \}.
    \end{equation}
    C'est l'ensemble des polynômes qui annulent \( f\) en \( x\). Le générateur unitaire de \( I_{f,x}\) est le \defe{polynôme minimal ponctuel}{polynôme!minimal!ponctuel}\index{polynôme!minimal!relativement à un point} de \( f\) en \( x\). Il sera noté \( \mu_{f,x}\).
\end{definition}
Ces définitions sont légitimées par les faits suivants. L'idéal \( I_{f,x}\) n'est pas réduit à \( \{ 0 \}\) parce que le polynôme minimal de \( f\) fait partie de \( I_{f,x}\). C'est le théorème \ref{ThoCCHkoU} qui nous assure l'existence d'un unique générateur unitaire dans~\( I_{f,x}\). 

\begin{lemma}\label{LemSYsJJj}
    Soit \( f\colon E\to E\) un endomorphisme de l'espace vectoriel \( E\). Il existe un élément \( x\in E\) tel que \( \mu_{f,x}=\mu_f\).
\end{lemma}

\begin{proof}
    Nous savons que pour tout \( x\in E\), \( \mu_f\in I_{f,x}\), donc le polynôme \( \mu_{f,x}\) divise \( \mu_f\) pour tous les \( x\). Nous en déduisons que l'ensemble
    \begin{equation}
        \{ \mu_{f,x}\tq x\in E \}
    \end{equation}
    est en réalité un ensemble fini, sinon \( \mu_f\) ne serait pas un polynôme. Soient donc les points \( x_1,\ldots, x_l\) tels que
    \begin{equation}
        \{ \mu_{f,x}\tq x\in E \}=\{ \mu_{f,x_1},\ldots, \mu_{f,x_l} \}.
    \end{equation}
    Étant donné que \( x\in \ker\mu_{f,x}\) nous avons \( \mu_{f,x}\in I_{f,x}\) et donc \( \mu_{f,x}(f)x=0\). Par conséquent
    \begin{equation}
        E=\bigcup_{1\leq i\leq l}\ker\mu_{f,x_i(f)}.
    \end{equation}
    En vertu de la proposition \ref{PropTVKbxU}, un des termes de l'union doit être l'espace \( E\) entier. Il existe donc un \( x_i\) tel que
    \begin{equation}
        E=\ker\big( \mu_{f,x_i}(f) \big).
    \end{equation}
    Le polynôme \( \mu_{f,x_i}\) annule \( f\) et est donc divisé par le polynôme minimal de \( f\). Nous avons donc montré que \( \mu_{f,_{x_i}}\) divise et est divisé par \( \mu_f\). Par conséquent \( \mu_f=\mu_{f,x_i}\).
\end{proof}

\begin{lemma}   \label{LemrFINYT}
    Si le polynôme minimal d'un endomorphisme est irréductible, alors il est semi-simple.
\end{lemma}

\begin{proof}
    Soit \( f\), un endomorphisme dont le polynôme minimal est irréductible et \( F\), un sous-espace stable par \( f\). Nous devons en trouver un supplémentaire stable. Si \( F=E\), il n'y a pas de problèmes. Sinon nous considérons \( u_1\in E\setminus F\) et
    \begin{equation}
        E_{u_1}=\{ P(f)u_1\tq P\in \eK[X] \},
    \end{equation}
    qui est un espace stable par \( f\). 

    Montrons que \( E_{u_1}\cap F=\{ 0 \}\). Pour cela nous regardons l'idéal
    \begin{equation}
        I_{u_1}=\{ P\in \eK[X]\tq P(f)u_1=0 \}.
    \end{equation}
    Cela est un idéal non réduit à \( \{ 0 \}\) parce que le polynôme minimal de \( f\) par exemple est dans \( I_{u_1}\). Soit \( P_{u_1}\) un générateur unitaire de \( I_{u_1}\). Étant donné que \( \mu_f\in I_{u_1}\), nous avons que \( P_{u_1}\) divise \( \mu_f\) et donc \( P_{u_1}=\mu_f\) parce que \( \mu_f\) est irréductible par hypothèse.

    Soit \( y\in E_{u_1}\cap F\). Par définition il existe \( P\in\eK[X]\) tel que \( y=P(f)u_1\) et si \( y\neq 0\), ce la signifie que \( P\notin I_{u_1}\), c'est à dire que \( P_{u_1} \) ne divise pas \( P\). Étant donné que \( P_{u_1}\) est irréductible cela implique que \( P_{u_1}\) et \( P\) sont premiers entre eux (ils n'ont pas d'autre \( \pgcd\) que \( 1\)).

    Nous utilisons maintenant Bézout (théorème \ref{ThoBezoutOuGmLB}) qui nous donne \( A,B\in \eK[X]\) tels que 
    \begin{equation}
        AP+BP_{u_1}=1.
    \end{equation}
    Nous appliquons cette égalité à \( f\) et puis à \( u_1\):
    \begin{equation}
        u_1=A(f)\circ \underbrace{P(f)u_1}_{=y}+B(f)\circ \underbrace{P_{u_1}(u_1)}_{=0}=A(f)y.
    \end{equation}
    Mais \( y\in F\), donc \( A(f)y\in F\). Nous aurions donc \( u_1\in F\), ce qui est impossible par choix. Nous avons maintenant que l'espace \( E_{u_1}\oplus F\) est stable sous \( f\). Si cet espace est \( E\) alors nous arrêtons. Sinon nous reprenons le raisonnement avec \( E_{u_1}\oplus F\) en guise de \( F\) et en prenant \( u_2\in E\setminus(E_{u_1}\oplus F)\). Étant donné que \( E\) est de dimension finie, ce procédé s'arrête à un certain moment et nous aurons
    \begin{equation}
        E=F\oplus E_{u_1}\oplus\ldots\oplus E_{u_k}
    \end{equation}
    où chacun des \( E_{u_i}\) sont stables.
\end{proof}

\begin{theorem} \label{ThoFgsxCE}
    Un endomorphisme est semi-simple si et seulement si son polynôme minimal est produit de polynômes irréductibles distincts deux à deux.
\end{theorem}

\begin{proof}

    Supposons que \( f\) soit semi-simple et que son polynôme minimal soit donné par \( \mu_f=M_1^{\alpha_1}\ldots M_r^{\alpha_r}\) où les \( M_i\) sont des polynômes irréductibles deux à deux distincts. Nous devons montrer que \( \alpha_i=1\) pour tout \( i\). Soit \( i\) tel que \( \alpha_i\geq 1\) et \( N\in \eK[X]\) tel que \( \mu_f=M^2N\) où l'on a noté \( M=M_i\). Nous étudions l'espace
    \begin{equation}
        F=\ker M(f)
    \end{equation}
    qui est stable par \( f\), et qui possède donc un supplémentaire \( S\) également stable par \( f\). Nous allons montrer que \( MN\) est un polynôme annulateur de \( f\).

    D'abord nous prenons \( x\in S\). Étant donné que \( F\) est le noyau de \( M(f)\),
    \begin{equation}
        M(f)\big( MN(f)x \big)=\mu_f(f)x=0,
    \end{equation}
    ce qui signifie que \( MN(f)x\in F\). Mais vu que \( S\) est stable par \( f\) nous avons aussi que \( MN(f)x\in S\). Finalement \( MN(f)x\in F\cap S=\{ 0 \}\). Autrement dit, \( MN(f)\) s'annule sur \( S\).

    Prenons maintenant \( y\in F\). Nous avons
    \begin{equation}
        MN(f)=N(f)\big( M(f)y \big)=0
    \end{equation}
    parce que \( y\in F=\ker M(f)\).

    Nous avons prouvé que \( MN(f)\) s'annule partout et donc que \( MN(f)\) est un polynôme annulateur de \( f\), ce qui contredit la minimalité de \( \mu_f=M^2N\).

    Nous passons au sens inverse. Soit \( m_f=M_1\ldots M_r\) une décomposition du polynôme minimal de l'endomorphisme \( f\) en irréductibles distincts deux à deux. Soit \( F\) un sous-espace vectoriel stable par \( f\). Nous notons
    \begin{equation}
        E_i=\ker(M_i(f))
    \end{equation}
    et \( f_i=f|_{E_i}\). Par le lemme \ref{CorKiSCkC} nous avons
    \begin{equation}
        F=\bigoplus_{i=1}^r(F\cap E_i).
    \end{equation}
    Les espaces \( E_i\) sont stables par \( f\) et étant donné que \( M_i\) est irréductible, il est le polynôme minimal de \( f_i\). En effet, \( M_i\) est annulateur de \( f_i\), ce qui montre que le minimal de \( f_i\) divise \( M_i\). Mais \( M_i\) étant irréductible, \( M_i\) est le polynôme minimal. Étant donné que \( \mu_{f_i}=M_i\), l'endomorphisme \( f_i\) est semi-simple par le lemme \ref{LemrFINYT}.

    L'espace \( F\cap E_i\) étant stable par l'endomorphisme semi-simple \( f_i\), il possède un supplémentaire stable que nous notons \( S_i\)~:
    \begin{equation}
        E_i=S_i\oplus(F\cap E_i).
    \end{equation}
    Étant donné que sur chaque \( S_i\) nous avons \( f|_{S_i}=f_i\), l'espace \( S=S_1\oplus\ldots\oplus S_r\) est stable par \( f\). Du coup nous avons
    \begin{subequations}
        \begin{align}
            E&=E_1\oplus\ldots\oplus E_r\\
            &=\big( S_1\oplus(F\cap E_1) \big)\oplus\ldots\oplus\big( S_r\oplus(F\cap E_r) \big)\\
            &=\big( \bigoplus_{i=1}^rS_i \big)\oplus\big( \bigoplus_{i=1}^rF\cap E_i \big)\\
            &=S\oplus F,
        \end{align}
    \end{subequations}
    ce qui montre que \( F\) a bien un supplémentaire stable par \( f\) et donc que \( f\) est semi-simple.
\end{proof}

\begin{proposition}     \label{PropAnnncEcCxj}
    Si \( P\) est un polynôme tel que \( P(u)=0\), alors le polynôme minimal \( \mu_u\) divise \( P\).
\end{proposition}

\begin{proof}
    L'ensemble \( I=\{ Q\in \eK[X]\tq Q(u)=0 \}\) est un idéal par le lemme \ref{LemQWvhYb}. Le polynôme minimal de \( u\) est un élément de degré plus bas dans \( I\) et par conséquent \( I=(\mu_u)\) par le théorème \ref{ThoCCHkoU}. Nous concluons que \( \mu_u\) divise tous les éléments de \( I\).
\end{proof}

\begin{lemma}
    Soit \( f\) un endomorphisme cyclique d'un espace vectoriel \( E\) de dimension finie et \( y\), un vecteur cyclique de \( f\). Alors le polynôme minimal de \( f\) en \( y\) est le polynôme minimal de \( f\).
\end{lemma}

\begin{proof}
    En utilisant les notations de la définition \ref{Decyyumy}, nous devons démontrer que \( \mu_{f,y}=\mu_f\). Bien entendu, \( \mu_f\in I_{y,f}\), donc \( \mu_{f,y}\) divise \( \mu_f\). Montrons que \( \mu_{f,y}\) est un polynôme annulateur de \( f\). Dans ce cas \( \mu_f\) divisera \( \mu_{f,y}\) et le lemme sera démontré.

    Le vecteur \( y\) étant cyclique, tout élément de \( E\) s'écrit sous la forme \( x=P(f)y\) où \( P\) est un polynôme (de degré égal à la dimension de \( E\)). En utilisant le lemme \ref{LemQWvhYb} nous avons
    \begin{equation}
            \mu_{f,y}(f)x=\big( \mu_{f,y}(f)\circ P(f) \big)y
            =\big( P(f)\circ \mu_{f,y}(f) \big)y
            =0.
    \end{equation}
\end{proof}

Si \( f\) est un endomorphisme de l'espace vectoriel \( E\) et si \( x\in E\), nous notons 
\begin{equation}
    E_{f,x}=\Span\{ f^k(x)\tq k\in \eN \}.
\end{equation}

\begin{proposition}[\cite{RombaldiO}]\label{PropNrZGhT}
    Soit \( f\), un endomorphisme de \( E\) et \( x\in E\). Alors
    \begin{enumerate}
        \item
            L'espace \( E_{f,x}\) est stable par \( f\).
        \item\label{ItemfzKOCo}
            L'espace \( E_{f,x}\) est de dimension
            \begin{equation}
                p_{f,x}=\dim E_{f,x}=\deg(\mu_{f,x})
            \end{equation}
            où \( \mu_{f,x}\) est le générateur unitaire de \( I_{f,x}\).
        \item   \label{ItemKHNExH}
            Le polynôme caractéristique de \( f|_{E_{f,x}}\) est \( \mu_{f,x}\).
        \item   \label{ItemHMviZw}
            Nous avons
            \begin{equation}
                \chi_{f|_{E_{f,x}}}(f)x=\mu_{f,x}(f)x=0.
            \end{equation}
    \end{enumerate}
\end{proposition}

\begin{proof}
    Le fait que \( E_{f,x}\) soit stable par \( f\) est classique. Le point \ref{ItemHMviZw} est un une application du point \ref{ItemKHNExH}. Les deux gros morceaux sont donc les points \ref{ItemfzKOCo} et \ref{ItemKHNExH}.

    Étant donné que \( \mu_{f,x}\) est de degré minimal dans \( I_{f,x}\), l'ensemble
    \begin{equation}
        B=\{ f^k(x)\tq 0\leq k\leq p_{f,x}-1 \}
    \end{equation}
    est libre. En effet une combinaison nulle des vecteurs de \( B\) donnerait un polynôme en \( f\) de degré inférieur à \( p_{f,x}\) annulant \( x\). Nous écrivons
    \begin{equation}
        \mu_{f,x}(X)=X^{p_{f,x}}-\sum_{i=0}^{p_{f,x}-1}a_iX^k. 
    \end{equation}
    Étant donné que \( \mu_{f,x}(f)x=0\) et que la somme du membre de droite est dans \( \Span(B)\), nous avons \( f^{p_{f,x}}(x)\in\Span(B)\). Nous prouvons par récurrence que \( f^{p_{f,x}+k}(x)\in\Span(B)\). En effet en appliquant \( f^k\) à l'égalité
    \begin{equation}
        0=f^{p_{f,x}}(x)-\sum_{i=0}^{p_{f,x}-1}a_if^i(x)
    \end{equation}
    nous trouvons
    \begin{equation}
        f^{p_{f,x}+k}(x)=\sum_{i=0}^{p_{f,x}-1}a_if^{i+k}(x),
    \end{equation}
    alors que par hypothèse de récurrence le membre de droite est dans \( \Span(B)\). L'ensemble \( B\) est alors générateur de \( E_{f,x}\) et donc une base d'icelui. Nous avons donc bien \( \dim(E_{f,x})=p_{f,x}\).

    Nous montrons maintenant que \( \mu_{f,x}\) est annulateur de \( f\) au point \( x\). Nous savons que
    \begin{equation}
        \mu_{f,x}(f)x=0.
    \end{equation}
    En y appliquant \( f^k\) et en profitant de la commutativité des polynômes sur les endomorphismes (proposition \ref{LemQWvhYb}), nous avons
    \begin{equation}
        0=f^k\big( \mu_{f,x}(f)x \big)=\mu_{f,x}(f)f^k(x),
    \end{equation}
    de telle sorte que \( \mu_{f,x}(f)\) est nul sur \( B\) et donc est nul sur \( E_{f,x}\). Autrement dit,
    \begin{equation}
        \mu_{f,x}\big( f|_{E_{f,x}} \big)=0.
    \end{equation}
    Montrons que \( \mu_{f,x}\) est même minimal pour \( f|_{E_{f,x}}\). Sot \( Q\), un polynôme non nul de degré \( p_{f,x}-1\) annulant \( f|_{E_{f,x}}\). En particulier \( Q(f)x=0\), alors qu'une telle relation signifierait que \( B\) est un système lié, alors que nous avons montré que c'était un système libre. Nous concluons que \( \mu_{f,x}\) est le polynôme minimal de \( f|_{E_{f,x}}\).
\end{proof}

\begin{theorem}[Cayley-Hamlilton]\index{théorème!Cayley-Hamilton}   \label{ThoCalYWLbJQ}
    Le polynôme caractéristique est un polynôme annulateur.
\end{theorem}

\begin{proof}
    Nous devons prouver que \( \chi_f(f)x=0\) pour tout \( x\in E\). Pour cela nous nous fixons un \( x\in E\), nous considérons l'espace \( E_{f,x}\) et \( \chi_{f,x}\), le polynôme caractéristique de \( f|_{E_{f,x}}\). Étant donné que \( E_{f,x}\) est stable par \( f\), le polynôme caractéristique de \( f|_{E_{j,x}}\) divise \( \chi_f\), c'est à dire qu'il existe un polynôme \( Q_x\) tel que
    \begin{equation}
        \chi_f=Q_x\chi_{f,x},
    \end{equation}
    et donc aussi
    \begin{equation}
        \chi_f(f)x=Q_x(f)\big( \chi_{f,x}(f)x \big)=0
    \end{equation}
    parce que la proposition \ref{PropNrZGhT} nous indique que \( \chi_{f,x}\) est un polynôme annulateur de \( f|_{E_{f,x}}\).
\end{proof}

Le polynôme de Cayley-Hamilton donne un moyen de calculer l'inverse d'un endomorphisme inversible pourvu que l'on sache son polynôme caractéristique. En effet, supposons que
\begin{equation}
    \chi_f(X)=\sum_{k=0}^na_kX^k.
\end{equation}
Nous aurons alors
\begin{equation}
    0=\chi_f(f)=\sum_{k=0}^na_kf^k.
\end{equation}
Nous appliquons \( f^{-1}\) à cette dernière égalité en sachant que \( f^{-1}(0)=0\) :
\begin{equation}
    0=a_0f^{-1}+\sum_{k=1}^na_kf^{k-1},
\end{equation}
et donc
\begin{equation}
    u^{-1}=-\frac{1}{ \det(f) }\sum_{k=1}^na_kf^{k-1}
\end{equation}
où nous avons utilisé le fait que \( a_0=\chi_f(0)=\det(f)\).


%---------------------------------------------------------------------------------------------------------------------------
\subsection{Endomorphismes nilpotents}
%---------------------------------------------------------------------------------------------------------------------------

\begin{lemma}[\cite{fJhCTE}]   \label{LemzgNOjY}
    L'endomorphisme \( u\in\End(\eC^n)\) est nilpotent si et seulement si \( \tr(u^p)=0\) pour tout \( p\).
\end{lemma}

\begin{proof}
    Supposons que \( u\) est nilpotent. Alors ses valeurs propres sont toutes nulles et celles de \( u^p\) le sont également. La trace étant la somme des valeurs propres, nous avons alors tout de suite \( \tr(u^p)=0\).

    Supposons maintenant que \( \tr(u^p)=0\) pour tout \( p\). Le polynôme caractéristique \eqref{Eqkxbdfu} est
    \begin{equation}    \label{EqfnCqWq}
        \chi_u=(-1)^nX^{\alpha}(X-\lambda_1)^{\alpha_1}\ldots (X-\alpha_r)^{\alpha_r}.
    \end{equation}
    où les \( \lambda_i\) (\( i=1,\ldots, r\)) sont les valeurs propres non nulles distinctes de \( u\).

    Il est vite vu que le coefficient de \( X^{n-1}\) dans \( \chi_u\) est \( -\tr(u)\) parce que le coefficient de \( X^{n-1}\) se calcule en prenant tous les $X$ sauf une fois \( -\lambda_i\). D'autre part le polynôme caractéristique de \( u^p \) est le même que celui de \( u\), en remplaçant \( \lambda_i\) par \( \lambda_i^p\); cela est dû au fait que si \( v\) est vecteur propre de valeur propre \( \lambda\), alors \( u^pv=\lambda^pv\).

    Par l'équation \eqref{EqfnCqWq}, nous voyons que le coefficient du terme \( X^{n-1}\) dans les polynôme caractéristique est 
    \begin{equation}        \label{eqSoDSKH}
        0=\tr(u^p)=\alpha_1\lambda_1^p+\ldots +\alpha_r\lambda_r^p.
    \end{equation}
    Donc les nombres \( (\alpha_1,\ldots, \alpha_r)\) est une solution non triviale\footnote{Si \( \alpha_1=\ldots=\alpha_r=0\), alors les valeurs propres sont toutes nulles et la matrice est en réalité nulle dès le départ.} du système
    \begin{subequations}    \label{EqDpvTnu}
        \begin{numcases}{}
            \alpha_1X_1+\ldots +\lambda_rX_r=0\\
            \qquad\vdots\\
            \lambda^r_1X_1+\ldots +\lambda_r^rX_r=0.
        \end{numcases}
    \end{subequations}
    Cela sont les équations \eqref{eqSoDSKH} écrites avec \( p=1,\ldots, r\). Le déterminant de ce système est
    \begin{equation}
        \lambda_1\ldots\lambda_r\det\begin{pmatrix}
             1   &   \ldots    &   1    \\
             \lambda_1   &   \ldots    &   \lambda_1    \\
             \vdots   &       &   \vdots    \\ 
             \lambda_1^{r-1}   &   \ldots    &   \lambda_r^{r-1}
         \end{pmatrix}\neq 0,
    \end{equation}
    qui est un déterminant de Vandermonde (proposition \ref{PropnuUvtj}) valant
    \begin{equation}
        0=\lambda_1\ldots\lambda_r\prod_{1\leq i\leq j\leq r}(\lambda_i-\lambda_j).
    \end{equation}
    Étant donné que les \( \lambda_i\) sont distincts et non nuls, nous avons une contradiction et nous devons conclure que \( (\alpha_1,\ldots, \alpha_r)\) était une solution triviale du système \eqref{EqDpvTnu}.
\end{proof}

\begin{definition}  \label{DefkAXaWY}
    Si \( X\) est une partie d'une algèbre \( A\), alors l'\defe{algèbre engendrée}{algèbre engendrée} par \( X\) est l'intersection de toutes les sous-algèbres de \( A\) contenant \( X\).
\end{definition}

%+++++++++++++++++++++++++++++++++++++++++++++++++++++++++++++++++++++++++++++++++++++++++++++++++++++++++++++++++++++++++++
\section{Diagonalisation}
%+++++++++++++++++++++++++++++++++++++++++++++++++++++++++++++++++++++++++++++++++++++++++++++++++++++++++++++++++++++++++++

Ici encore \( \eK\) est un corps commutatif.

%---------------------------------------------------------------------------------------------------------------------------
\subsection{Endomorphismes diagonalisables}
%---------------------------------------------------------------------------------------------------------------------------

\begin{lemma}       \label{LemgnaEOk}
    Soit \( F\) un sous-espace stable par \( u\). Soit une décomposition du polynôme minimal
    \begin{equation}
        \mu_u=P_1^{n_1}\ldots P_r^{n_r}
    \end{equation}
    où les \( P_i\) sont des polynômes irréductibles unitaires distincts. Si nous posons \( E_i=\ker P_i^{n_i}\), alors
    \begin{equation}
        F=(F\cap E_1)\oplus\ldots \oplus(F\cap E_r).
    \end{equation}
\end{lemma}

\begin{theorem}     \label{ThoDigLEQEXR}
    Soit \( E\), un espace vectoriel de dimension \( n\) sur le corps commutatif \( \eK\) et \( u\in\End(E)\). Les propriétés suivantes sont équivalentes.
    \begin{enumerate}
        \item       \label{ItemThoDigLEQEXRi}
            Il existe un polynôme \( P\in\eK[X]\) non constant, scindé sur \(\eK\) dont toutes les racines sont simples tel que \( P(u)=0\).
        \item\label{ItemThoDigLEQEXRii}
            Le polynôme minimal \( \mu_u\) est scindé sur \(\eK\) et toutes ses racines sont simples
        \item\label{ItemThoDigLEQEXRiii}
            Tout sous-espace de \( E\) possède un supplémentaire stable par \( u\).
        \item\label{ItemThoDigLEQEXRiv}
            L'endomorphisme \( u\) est diagonalisable.
    \end{enumerate}

\end{theorem}

\begin{proof}
    \ref{ItemThoDigLEQEXRi}\( \Rightarrow\)\ref{ItemThoDigLEQEXRii}. Étant donné que \( P(u)=0\), il est dans l'idéal des polynôme annulateurs de \( u\), et le polynôme minimal \( \mu_u\) le divise parce que l'idéal des polynôme annulateurs est généré par \( \mu_u\) par le théorème \ref{ThoCCHkoU}.

    \ref{ItemThoDigLEQEXRii}\( \Rightarrow\)\ref{ItemThoDigLEQEXRiv}. Étant donné que le polynôme minimal est scindé à racines simples, il s'écrit sous forme de produits de monômes tous distincts, c'est à dire
    \begin{equation}
        \mu_u(X)=(X-\lambda_1)\ldots(X-\lambda_r)
    \end{equation}
    où les \( \lambda_i\) sont des éléments distincts de \( \eK\). Étant donné que \( \mu_u(u)=0\), le théorème de décomposition des noyaux (théorème \ref{ThoDecompNoyayzzMWod}) nous enseigne que
    \begin{equation}
        E=\ker(u-\lambda_1)\oplus\ldots\oplus\ker(u-\lambda_r).
    \end{equation}
    Mais \( \ker(u-\lambda_i)\) est l'espace propre \( E_{\lambda_i}(u)\). Donc \( u\) est diagonalisable.

    \ref{ItemThoDigLEQEXRiv}\( \Rightarrow\)\ref{ItemThoDigLEQEXRiii}. Soit \( \{ e_1,\ldots, e_n \}\) une base qui diagonalise \( u\), soit \( F\) un sous-espace de \( E\) un \( \{ f_1,\ldots, f_r \}\) une base de \( F\). Par le théorème \ref{ThoBaseIncompjblieG} (qui généralise le théorème de la base incomplète), nous pouvons compléter la base de \( F\) par des éléments de la base \( \{ e_i \}\). Le complément ainsi construit est invariant par \( u\).

    \ref{ItemThoDigLEQEXRiii}\( \Rightarrow\)\ref{ItemThoDigLEQEXRiv}. En dimension un, tout endomorphisme est diagonalisable, nous supposons donc que \( \dim E=n\geq 2\). Nous procédons par récurrence sur le nombre de vecteurs propres connus de \( u\). Supposons avoir déjà trouvé \( p\) vecteurs propres \( e_1,\ldots, e_p\) de \( u\). Considérons \( H\), un hyperplan qui contient les vecteurs \( e_1,\ldots, e_p\). Soit \( F\) un supplémentaire de \( H\) stable par \( u\); par construction \( \dim F=1\) et si \( e_{p+1}\in F\), il doit être vecteur propre de \( u\).

    \ref{ItemThoDigLEQEXRiv}\( \Rightarrow\)\ref{ItemThoDigLEQEXRi}. Nous supposons maintenant que \( u\) est diagonalisable. Soient \( \lambda_1,\ldots, \lambda_r\) les valeurs propres deux à deux distinctes, et considérons le polynôme
    \begin{equation}
        P(x)=(X-\lambda_1)\ldots (X-\lambda_r).
    \end{equation}
    Alors \( P(u)=0\). En effet si \( e_i\) est un vecteur propre pour la valeur propre \( \lambda_i\), 
    \begin{equation}
        P(u)e_i=\prod_{j\neq i}(u-\lambda_j)\circ(u-\lambda_i)e_i=0
    \end{equation}
    par le lemme \ref{LemQWvhYb}. Par conséquent \( P(u)\) s'annule sur une base.
\end{proof}

\begin{corollary}       \label{CorQeVqsS}
    Si \( u\) est diagonalisable et si \( F\) est une sous-espace stable par \( u\), alors
    \begin{equation}
        F=\sum_{\lambda}E_{\lambda}(u)\cap F
    \end{equation}
    où \( E_{\lambda}(u)\) est l'espace propre de \( u\) pour la valeur propre \( \lambda\). En particulier la restriction de \( u\) à \( F\), \( u|_F\) est diagonalisable.
\end{corollary}

\begin{proof}
    Par le théorème \ref{ThoDigLEQEXR}, le polynôme \( \mu_x\) est scindé et ne possède que des racines simples. Notons le
    \begin{equation}
        \mu_u(X)=(X-\lambda_1)\ldots (X-\lambda_r).
    \end{equation}
    Les espaces \( E_i\) du lemme \ref{LemgnaEOk} sont maintenant les espaces propres.

    En ce qui concerne la diagonalisabilité de \( u|_F\), notons que nous avons une base de \( F\) composée de vecteurs dans les espaces \( E_{\lambda}(u)\). Cette base de \( F\) est une base de vecteurs propres de \( u\).
\end{proof}

\begin{lemma}
    Soit \( E\) un \( \eK\)-espace vectoriel et \( u\in\End(E)\). Si \( \Card\big( \Spec(u) \big)=\dim(E)\) alors \( u\) est diagonalisable.
\end{lemma}

\begin{proof}
    Soient \( \lambda_1,\ldots, \lambda_n\) les valeurs propres distinctes de \( u\). Nous savons que les espaces propres correspondants sont en somme directe (lemme \ref{LemjcztYH}). Par conséquent \( \Span\{ E_{\lambda_i}(u) \}\) est de dimension \( n\) est \( u\) est diagonalisable.
\end{proof}

\begin{proposition}     \label{PropGqhAMei}
    Soit \( (u_i)_{i\in I}\) une famille d'endomorphismes qui commutent deux à deux.
    \begin{enumerate}
        \item       \label{ItemGqhAMei}
            Si \( i,j\in I\) alors tout sous-espace propre de \( u_i\) est stable par \( u_j\). Autrement dit \( u_j\big(E_{\lambda}(u)\big)\subset E_{\lambda}(u)\).
        \item
            Si les \( u_i\) sont diagonalisables, alors ils le sont simultanément.
    \end{enumerate}
\end{proposition}

\begin{proof}

    Supposons que \( u_i\) et \( u_j\) commutent et soit \( x\) un vecteur propre de \( u_i\) : \( u_ix=\lambda x\). Nous montrons que \( u_jx\in E_{\lambda}(u)\). Nous avons
    \begin{equation}
        u_i\big( u_j(x) \big)=u_j\big( u_i(x) \big)=\lambda u_j(x).
    \end{equation}
    Par conséquent \( u_j(x)\) est vecteur propre de \( u_i\) de valeur propre \( \lambda\).

    Montrons maintenant l'affirmation à propos des endomorphismes simultanément diagonalisables. Si \( \dim E=1\), le résultat est évident. Nous supposons également qu'aucun des \( u_i\) n'est multiple de l'identité. Nous effectuons une récurrence sur la dimension.

    Soit \( u_0\) un des \( u_i\) et considérons ses valeurs propres deux à deux distinctes \( \lambda_1,\ldots, \lambda_r\). Pour chaque \( k\) nous avons
    \begin{equation}
        E_{\lambda_k}(u_0)\neq E,
    \end{equation}
    sinon \( u_0\) serait un multiple de l'identité. Par contre nous avons
    \begin{equation}
        E=\bigoplus_{k}E_{\lambda_k}(u_0).
    \end{equation}
    Par le point \ref{ItemGqhAMei}, nous avons \( u_i\colon E_{\lambda_k}(u_0)\to E_{\lambda_k}(u_0)\), et nous pouvons considérer la famille d'opérateurs
    \begin{equation}
        \left( u_i|_{E_{\lambda_k}(u_0)} \right)_{i\in I}.
    \end{equation}
    Ce sont tous des opérateurs qui commutent et qui agissent sur un espace de dimension plus petite. Par hypothèse de récurrence nous avons une base de \( E_{\lambda_k}(u_0)\) qui diagonalise tous les \( u_i\).
\end{proof}

\begin{example}     \label{ExewINgYo}
    Soit un espace vectoriel sur un corps \( \eK\). Un opérateur \defe{involutif}{involution} est un opérateur différent de l'identité dont le carré est l'identité. Typiquement une symétrie orthogonale dans \( \eR^3\). Le polynôme caractéristique d'une involution est \( X^2-1=(X+1)(X-1)\).
    
    Tant que \( 1\neq -1\), \( X^1-1\) est donc scindé à racines simples et les involutions sont diagonalisables (\ref{ThoDigLEQEXR}). Cependant si le corps est de caractéristique \( 2\), alors \( X^2-1=(X+1)^2\) et l'involution n'est plus diagonalisable.

    Par exemple si le corps est de caractéristique \( 2\), nous avons
    \begin{subequations}
        \begin{align}
            A&=\begin{pmatrix}
                1    &   1    \\ 
                0    &   1    
            \end{pmatrix}\\
            A^1&=\begin{pmatrix}
                1    &   2    \\ 
                0    &   1    
            \end{pmatrix}=\begin{pmatrix}
                1    &   0    \\ 
                0    &   1    
            \end{pmatrix}.
        \end{align}
    \end{subequations}
    Ce \( A\) est donc une involution mais n'est pas diagonalisable.
\end{example}

\begin{theorem}[Burnside\cite{fJhCTE}]\label{ThooJLTit} \index{exposant}
    Un sous-groupe de \( \GL(n,\eC)\) est fini si et seulement si il est d'exposant fini.
\end{theorem}

\begin{proof}
    Soit \( G\) un sous-groupe de \( \GL(n,\eC)\). Si \( G\) est fini, l'ordre de ses éléments divise \( | G |\) (corollaire \ref{CorpZItFX}) au théorème de Lagrange et l'exposant est le PPCM qui est donc fini également.

    Nous supposons maintenant que l'ordre de \( G\) est fini. Nous notons \( e\) l'exposant de \( G\). En particulier, tous les éléments de \( G\) sont des racines du polynôme \( X^e-1\).
    
    \begin{subproof}
        \item[Générateurs]

            Le groupe \( G\) est une partie de \( \eM(n,\eC)\) dont nous considérons l'algèbre engendrée (définition \ref{DefkAXaWY}) \( \mG\). Soit \( C_1,\ldots, C_r\) une famille génératrice de \( \mG\) constituée d'éléments de \( G\) et la fonction
            \begin{equation}
                \begin{aligned}
                    \tau\colon G&\to \eC^r \\
                    A&\mapsto \big( \tr(AC_1),\ldots, \tr(AC_r) \big). 
                \end{aligned}
            \end{equation}

        \item[\( \tau\) est injective] Supposons que \( \tau(A)=\tau(B)\). Alors pour tout générateur \( C_i\) nous avons \( \tr(AC_i)=\tr(BC_i)\) et par linéarité de la trace, nous avons
            \begin{equation}    \label{EqnCYmKW}
                \tr(AM)=\tr(BM)
            \end{equation}
            pour tout \( M\in G\). Notons par ailleurs
            \begin{equation}
                N=AB^{-1}-\mtu,
            \end{equation}
            qui est diagonalisable parce que \( AB^{-1}\in G\) et donc est annulé par le polynôme \( X^e-1\) qui est scindé à racines simples. Du coup \( AB^{-1}\) est diagonalisable; posons \( PAB^{-1}P^{-1}=D\), alors \( P\big( AB^{-1}-\mtu \big)P^{-1}=D-\mtu\) qui est encore diagonale. Donc \( N\) est diagonalisable.

            Par ailleurs nous avons
            \begin{subequations}
                \begin{align}
                    \tr\big( (AB^{-1})^p \big)&=\tr\big( AB^{-1}(AB^{-1})^{p-1} \big)\\
                    &=\tr\big( BB^{-1}(AB^{-1})^{p-1} \big) &\text{\eqref{EqnCYmKW}}\\
                    &=\tr\big( (AB^{-1})^{p-1} \big).
                \end{align}
            \end{subequations}
            En continuant nous obtenons
            \begin{equation}
                \tr\big(  (AB^{-1})^p \big)=\tr(\mtu)=n.
            \end{equation}
            
            D'autre part, 
            \begin{equation}
                N^k=(AB^{-1}-\mtu)^k=\sum_{p=0}^k{p\choose k}(-1)^{k-p}(AB^{-1})^p
            \end{equation}
            En prenant la trace, et en tenant compte du fait que \( \tr\big( (AB^{-1})^p \big)=n\),
            \begin{equation}
                \tr(N^k)=\sum_{p=0}^k{p\choose k}(-1)^{k-p}n=n(1-1)^k=0.
            \end{equation}
            Donc la trace de \( N^k\) est nulle et le lemme \ref{LemzgNOjY} nous enseigne que \( N\) est alors nilpotente. Étant donné qu'elle est aussi diagonalisable, elle est nulle. Nous en concluons que \( AB^{-1}=\mtu\) et donc que \( A=B\). La fonction \( \tau\) est donc injective.

        \item[Nombre fini de valeurs]

            Les éléments de \( G\) sont annulés par \( X^e-1\) qui est un polynôme scindé à racines simples. Dons le polynôme minimal d'un élément de \( G\) est (a fortiori) scindé à racines simples et le théorème \ref{ThoDigLEQEXR} nous assure alors que ces éléments sont diagonalisables. Du coup les valeur propres des matrices de \( G\) sont des racines \( e\)ièmes de l'unité. Par conséquent les traces des éléments de \( G\) ne peuvent prendre qu'un nombre fini de valeurs : toutes les sommes de \( n\) racines \( e\)ièmes de l'unité. Mais vu que les \( C_i\) sont dans \( G\), nous avons
            \begin{equation}
                \Image(\tau)=\{ \tr(A)\tq A\in G \}^r,
            \end{equation}
            qui est un ensemble fini. Par conséquent \( G\) est fini parce que \( \tau\) est injective.
    \end{subproof}
\end{proof}


\begin{proposition} \label{PropleGdaT}
    Soit \( p\) un nombre premier et \( P\) un élément de \( \eF_p[X]\). L'anneau \( \eF_p[X]/(P)\) est intègre si et seulement si \( P\) est irréductible dans \( \eF_p[X]\).
\end{proposition}

\begin{proof}
    Supposons que \( P\) soit réductible dans \( \eF_p[X]\), c'est à dire qu'il existe \( Q,R\in \eF_p[X]\) tels que \( P=QR\). Dans ce cas, \( \bar Q\) est diviseur de zéro dans \( \eF_p[X]/(P)\) parce que \( \bar Q\bar R=0\).

    Nous supposons maintenant que \( \eF_p[X]/(P)\) ne soit pas intègre : il existe des polynômes \( R,Q\in \eF_p[X]\) tels que \( \bar Q\bar R=0\). Dans ce cas le polynôme \( QR\) est le produit de \( P\) par un polynôme : \( QR=PA\). Tous les facteurs irréductibles de \( A \) étant soit dans \( Q\) soit dans \( R\), il est possible de modifier un peu \( Q\) et \( R\) pour obtenir \( QR=P\), ce qui signifie que \( P\) n'est pas irréductible.
\end{proof}

\begin{theorem}[Théorème des deux carrés]   \label{ThospaAEI}
    Un nombre premier est somme de deux carrés si et seulement si \( p=2\) ou \( p=1\mod 4\).
\end{theorem}
\begin{remark}
    Il n'est pas dit que les nombres dans \( [1]_4\) sont premiers (\( 9=8+1\) ne l'est pas par exemple). Le théorème signifie que (à part \( 2\)), si un nombre premier est dans \( [1]_4\) alors il est somme de deux carrés, et inversement, si un nombre premier est somme de deux carrés, il est dans \( [1]_4\).
\end{remark}

\begin{proof}
    Nous notons \( \Sigma=\{ a^2+b^2\tq a,b\in \eN \}\). Soit \( p\) un nombre premier dans \( \Sigma\). Si \( a=2k\), alors \( a^2=4k^2\) et \( a^2=0\mod 4\). Si au contraire \( a\) est impair, \( a=2k+1\) et \( a^2=4k^2+1+4k=1\mod 4\). La même chose est valable pour \( b\). Par conséquent, \( a^2+b^2\) est automatiquement \( [0]_4\), \( [1]_4\) ou \( [2]_4\). Évidemment les nombres de la forme \( 0\mod 4\) ne sont pas premiers; parmi les \( 2\mod 4\), seul \( p=2\) est premier (et vaut \( 1^2+1^2\)).

    Nous avons démontré que les seuls premiers de la forme \( a^2+b^2\) sont \( p=2\) et les \( p=1\mod 4\). Il reste à faire le contraire : démontrer que si un nombre premier \( p\) vaut \( 1\mod 4\), alors il est premier. Nous considérons l'anneau
    \begin{equation}
        \eZ[i]=\{ a+bi\tq a,b\in \eZ \}.
    \end{equation}
    puis l'application
    \begin{equation}
        \begin{aligned}
            N\colon \eZ[i]&\to \eN \\
            a+bi&\mapsto a^2+b^2. 
        \end{aligned}
    \end{equation}
    Un peu de calcul dans \( \eC\) montre que pour tout \( z,z'\in \eZ[i]\), \( N(zz')=N(z)N(z')\).

    Déterminons maintenant les éléments inversibles de \( \eZ[i]\). Si \( z\in \eZ[i]^*\), alors il existe \( z'\in \eZ[i]^*\) tel que \( zz'=1\). Dans ce cas nous aurions
    \begin{equation}
        1=N(zz')=N(z)N(z'),
    \end{equation}
    ce qui est uniquement possible avec \( N(z)=N(z')=1\), c'est à dire \( z=\pm 1\) ou \( z=\pm i\). Nous avons donc
    \begin{equation}
        \eZ[i]^*=\{ \pm 1,\pm i \}.
    \end{equation}
    Nous montrons que \( \eZ[i]\) est un anneau euclidien en montrant que \( N\) est un stathme. Soient \( t,t\in \eZ[i]\setminus\{ 0 \}\) et 
    \begin{equation}
        \frac{ z }{ t }=x+iy
    \end{equation}
    dans \( \eC\). Nous considérons \( q=a+bi\) où \( a\) et \( b\) sont les entiers les plus proches de \( x\) et \( y\). Si il y a \emph{ex aequo}, on prend au hasard\footnote{Dans l'exemple \ref{ExwqlCwvV}, nous prenions toujours l'inférieur parce que le stathme tenait compte de la positivité.}. Alors nous avons
    \begin{equation}
        | \frac{ z }{ t }-q |\leq \frac{ | 1+i | }{ 2 }=\frac{ \sqrt{2} }{2}<1.
    \end{equation}
    On pose \( r=z-qt\) qui est bien un élément de \( \eZ[i]\). De plus
    \begin{equation}
        | r |=| z-qt |=| t | |\frac{ z }{ t }-q |<| t |,
    \end{equation}
    c'est à dire que \( | r |^2<| t |^2\) et donc \( N(r)<N(t)\). L'anneau \( \eZ[i]\) étant euclidien, il est principal (proposition \ref{Propkllxnv}).

    Pour la suite, nous allons d'abord montrer que \( p\in\Sigma\) si et seulement si \( p\) n'est pas irréductible dans \( \eZ[i]\), puis nous allons voir quels sont les irréductibles de \( \eZ[i]\).

    Soit \( p\), un nombre premier dans \( \Sigma\). Si \( p=a^2+b^2\), alors nous avons \( p=(a+ib)(a-bi)\), mais étant donné que \( p\) est premier, nous avons \( a\neq 0\) et \( b\neq 0\). Du coup \( p\) n'est pas inversible dans \( \eZ[i]\), mais il peut être écrit comme le produit de deux non inversibles. Le nombre \( p\) est donc non irréductible dans \( \eZ[i]\).

    Dans l'autre sens, nous supposons que \( p\) est un nombre premier non irréductible dans \( \eZ[i]\). Nous avons alors \( p=zz'\) avec ni \( z\) ni \( z'\) dans \( \{ \pm 1,\pm i \}\). En appliquant \( N\) nous avons
    \begin{equation}
        p^2=N(p)=N(z)N(z').
    \end{equation}
    Vu que \( p\) est premier, cela est uniquement possible avec \( N(z)=N(z')=p\) (avoir \( N(z)=1\) est impossible parce que cela dirait que \( z\) est inversible). Si \( z=a+ib\), alors \( p=N(z)=a^2+b^2\), et donc \( p\in \Sigma\).

    Nous cherchons maintenant les éléments irréductibles de \( \eZ[i]\). Nous savons déjà que \( \eZ[i]\) est un anneau principal et n'est pas un corps; la proposition \ref{PropomqcGe} s'applique donc et \( p\) sera non irréductible si et seulement si l'idéal \( (p)\) sera non premier. Le fait que \( (p)\) soit un idéal non premier implique que le quotient \( \eZ[i]/(p)\) est non intègre (c'est la définition d'un idéal premier). Nous cherchons donc les nombres premiers pour lesquels le quotient \( \eZ[i]/(p)\) n'est pas intègre.

    Nous commençons par écrire le quotient \( \eZ[i]/(p)\) sous d'autres formes. D'abord en remarquant que si \( I\) et \( J\) sont deux idéaux, on a \( (\eA/I)/J\simeq (\eA/J)/I\), du coup, en tenant compte du fait que \( \eZ[i]=\eZ[X]/(X^2+1)\), nous avons
    \begin{equation}
        \eZ[i]/(p)=(\eZ[X]/(p))/(X^2+1)=\eF_p[X]/(X^2+1).
    \end{equation}
    Nous avons donc équivalence des propositions suivantes :
    \begin{subequations}
        \begin{align}
            p\in\Sigma\\
            \eF_p[X]/(X^2+1)\text{ n'est pas intègre}\\
            X^2+1\text{ n'est pas irréductible dans \( \eF_p\)} \label{EqZkdrvh}\\
            \text{\( X^2+1\) admet une racine dans \( \eF_p\)}\\
            -1\in (\eF_p^*)^2\\
            \exists y\in \eF_p^*\tq y^2=-1.
        \end{align}
    \end{subequations}
    Le point \eqref{EqZkdrvh} vient de la proposition \ref{PropleGdaT}. Maintenant nous utilisons le fait que \( p\) soit un premier impair (parce que le cas de \( p=2\) est déjà complètement traité), donc \( (p-1)/2\in \eN\) et nous avons, pour le \( y\) de la dernière ligne,
    \begin{equation}
        (-1)^{(p-1)/2}=(y^2)^{(p-1)/2}=y^{p-1}=1
    \end{equation}
    parce que dans \( \eF_p\) nous avons \( y^{(p-1)}=1\) par le petit théorème de Fermat (théorème \ref{ThoOPQOiO}). Du coup \( p\) doit vérifier
    \begin{equation}
        1=(-1)^{(p-1)/2},
    \end{equation}
    c'est à dire \( \frac{ p-1 }{2}=0\mod 2\) ou encore \( p=1\mod 4\).

\end{proof}

%---------------------------------------------------------------------------------------------------------------------------
\subsection{Théorème de Burnside}
%---------------------------------------------------------------------------------------------------------------------------

\begin{lemma}       \label{LemwXXzIt}
    Soit \( P\), un polynôme sur \( \eK\). Une racine de \( P\) est une racine simple si et seulement si elle n'est pas racine de \( P'\).
\end{lemma}

\begin{lemma}
    Un endomorphisme \( u\colon E\to E\) est nilpotent si et seulement si \( u^p\) est de trace nulle pour tout \( p\) entre \( 1\) et \( \dim E\).
\end{lemma}

\begin{theorem}     \label{ThoBurnsideoPuCtS}
    Toute représentation d'un groupe d'exposant fini sur \( \eC^n\) a une image finie.
\end{theorem}

Dans le cas d'un groupe abélien, la démonstration est facile. Étant donné que \( G\) est d'exposant fini, il existe \( \alpha\in \eN^*\) tel que \( g^{\alpha}=e\) pour tout \( g\in G\). Le polynôme \( P(X)=X^{\alpha}-1\) est scindé à racines simples. En effet tout polynôme sur \( \eC\) est scindé. Le fait qu'il soit à racines simples provient du lemme \ref{LemwXXzIt} parce que si \( a^{\alpha}=1\), alors il n'est pas possible d'avoir \( \alpha a^{\alpha-1}=0\).

Par ailleurs \( P(g)=0\). Le fait que nous ayons un polynôme annulateur de \( g\) scindé à racines simples implique que \( g\) est diagonalisable (théorème \ref{ThoDigLEQEXR}). Le fait que \( G\) soit abélien montre qu'il existe une base de \( \eC^n\) dans laquelle tous les éléments de \( G\) sont diagonaux. Nous devons par conséquent montrer qu'il existe un nombre fini de matrices de la forme
\begin{equation}
    \begin{pmatrix}
        \lambda_1    &       &       \\
            &   \ddots    &       \\
            &       &   \lambda_n
    \end{pmatrix}.
\end{equation}
Nous savons que \( \lambda_i^{\alpha}=1\) parce que \( g^{\alpha}=\mtu\), par conséquent chacun des \( \lambda_i\) est une racine de l'unité dont il n'existe qu'un nombre fini.

Le résultat reste vrai si \( G\) n'est pas abélien, mais la preuve devient plus compliquée. C'est le \wikipedia{fr}{Théorème_de_Burnside_(problème_de_1902)}{théorème de Burnside}.

%---------------------------------------------------------------------------------------------------------------------------
\subsection{Diagonalisation : cas complexe}
%---------------------------------------------------------------------------------------------------------------------------

Nous considérons maintenant le cas de l'espace \( E=\eC^n\) comme espace vectoriel de dimension \( n\) sur \( \eC\). Il est muni d'une forme sesquilinéaire
\begin{equation}    \label{EqFormSesqQrjyPH}
    \langle x, y\rangle =\sum_{k=1}^nx_k\bar y_k
\end{equation}
pour tout \( x,y\in\eC^n\).
\begin{lemma}
    Pour un opérateur hermitien,
    \begin{enumerate}
        \item
            le spectre est réel,
        \item
            deux vecteurs propres à des valeurs propres distinctes sont orthogonales\footnote{Pour la forme \eqref{EqFormSesqQrjyPH}.}.
    \end{enumerate}
\end{lemma}

\begin{proof}
    Soit \( v\) un vecteur de valeur propre \( \lambda\). Nous avons d'une part 
    \begin{equation}
        \langle Av, A\rangle =\lambda\langle v, v\rangle =\lambda\| v \|^2,
    \end{equation}
    et d'autre part, en utilisant le fait que \( A\) est hermitien,
    \begin{equation}
        \langle Av, v\rangle =\langle v, A^*v\rangle =\langle v, Av\rangle =\bar\lambda\| v \|^2,
    \end{equation}
    par conséquent \( \lambda=\bar\lambda\) parce que \( v\neq 0\).

    Soient \( \lambda_i\) et \( v_i\) (\( i=1,2\)) deux valeurs propres de \( A\) avec leurs vecteurs propres correspondants. Alors d'une part
    \begin{equation}
        \langle Av_1, v_2\rangle =\lambda_1\langle v_1, v_2\rangle ,
    \end{equation}
    et d'autre part
    \begin{equation}
        \langle Av_1, v_2\rangle =\langle v_1, Av_2\rangle =\lambda_2\langle v_1, v_2\rangle .
    \end{equation}
    Nous avons utilisé le fait que \( \lambda_2\) était réel. Par conséquent, soit \( \lambda_1=\lambda_2\), soit \( \langle v_1, v_2\rangle =0\).
\end{proof}

La preuve de Schur provient de \cite{NormHKNPKRqV}.

\begin{lemma}[Lemme de Schur complexe]\index{lemme!Schur complexe}  \label{LemSchurComplHAftTq}
    Si \( A\in\eM(n,\eC)\), il existe une matrice unitaire \( U\) telle que \( UAU^{-1}\) soit triangulaire supérieure.
\end{lemma}

\begin{proof}
    Étant donné que \( \eC\) est algébriquement clos, nous pouvons toujours considérer un vecteur propre \( v_1\) de \( A\), de valeur propre \( \lambda_1\). Nous pouvons utiliser un procédé de Gram-Schmidt pour construire une base orthonormée \( \{ v,u_2,\ldots, u_n \}\) de \( \eR^n\), et la matrice (unitaire)
    \begin{equation}
        Q=\begin{pmatrix}
             \uparrow   &   \uparrow    &       &   \uparrow    \\
             v   &   u_2    &   \cdots    &   u_n    \\ 
             \downarrow   &   \downarrow    &       &   \downarrow
         \end{pmatrix}.
    \end{equation}
    Nous avons \( Q^{-1}AQe_1=Q^{-1} Av=\lambda Q^{-1} v=\lambda e_1\), par conséquent la matrice \( Q^{-1} AQ\) est de la forme
    \begin{equation}
        Q^{-1}AQ=\begin{pmatrix}
            \lambda_1    &   *    \\ 
            0    &   A_1    
        \end{pmatrix}
    \end{equation}
    où \( *\) représente une ligne quelconque et \( A_1\) est une matrice de \( \eM(n-1,\eC)\). Nous pouvons donc répéter le processus sur \( A_1\) et obtenir une matrice triangulaire supérieure (nous utilisons le fait qu'un produit de matrices orthogonales est orthogonales).  
\end{proof}
En particulier les matrices hermitiennes, anti-hermitiennes et unitaires sont trigonables par une matrice unitaire, qui peut être choisie de déterminant \( 1\).


Le théorème suivant et la preuve proviennent de \cite{LecLinAlgAllen}, \wikipedia{en}{Spectral_theorem}{wikipedia} et \href{http://planetmath.org/encyclopedia/TheoremForNormalTriangularMatrices.html}{PlanetMath}.
\begin{theorem}[Théorème spectral pour les matrices normales]\index{théorème!spectral!matrices normales}    \label{ThogammwA}
    Soit \( A\in\eM(n,\eC)\) une matrice de valeurs propres \( \lambda_1,\ldots, \lambda_n\) (non spécialement distinctes). Alors les conditions suivantes sont équivalentes :
    \begin{enumerate}
        \item   \label{ItemJZhFPSi}
            \( A\) est normale,
        \item   \label{ItemJZhFPSii}
            \( A\) se diagonalise par une matrice unitaire,
        \item
            \( \sum_{i,j=1}^n| A_{ij} |^2=\sum_{j=1}^n| \lambda_j |^2\),
        \item
            il existe une base orthonormale de vecteurs propres de \( A\).
    \end{enumerate}
\end{theorem}

\begin{proof}
    Nous allons nous contenter de prouver \ref{ItemJZhFPSi}\( \Leftrightarrow\)\ref{ItemJZhFPSii}. Soit \( Q\) la matrice unitaire donnée par la décomposition de Schur (lemme \ref{LemSchurComplHAftTq}) : \( A=QTQ^{-1}\). Étant donné que \( A\) est normale nous avons
    \begin{equation}
        QTT^*Q^{-1}=QT^*TQ^{-1},
    \end{equation}
    ce qui montre que \( T\) est également normale. Or une matrice triangulaire supérieure normale est diagonale. En effet nous avons \( T_{ij}=0\) lorsque \( i>j\) et
    \begin{equation}
        (TT^*)_{ii}=(T^*T)_{ii}=\sum_{k=1}^n| T_{ki} |^2=\sum_{k=1}^n| T_{ik} |^2.
    \end{equation}
    Écrivons cela pour \( i=1\) en tenant compte de \( | T_{k1} |^2=0\) pour \( k=2,\ldots, n\),
    \begin{equation}
        | T_{11} |^2=| T_{11} |^2+| T_{12} |^2+\ldots+| T_{1n} |^2,
    \end{equation}
    ce qui implique que \( T_{11}\) est le seul non nul parmi les \( T_{1k}\). En continuant de la sorte avec \( i=2,\ldots, n\) nous trouvons que \( T\) est diagonale.

    Dans l'autre sens, si \( A\) se diagonalise par une matrice unitaire, \( UAU^*=D\), nous avons
    \begin{equation}
        DD^*=UAA^*U^*
    \end{equation}
    et 
    \begin{equation}
        D^*D=UA^*AU^*,
    \end{equation}
    qui ce prouve que \( A\) est normale.
\end{proof}

%---------------------------------------------------------------------------------------------------------------------------
\subsection{Diagonalisation : cas réel}
%---------------------------------------------------------------------------------------------------------------------------

\begin{lemma}[Lemme de Schur réel]\index{lemme!Schur réel}  \label{LemSchureRelnrqfiy}
    Soit \( A\in\eM(n,\eR)\). Il existe une matrice orthogonale \( Q\) telle que \( Q^{-1}AQ\) soit de la forme
    \begin{equation}        \label{EqMtrTSqRTA}
        QAQ^{-1}=\begin{pmatrix}
            \lambda_1    &   *    &   *    &   *    &   *\\  
            0    &   \ddots    &   \ddots    &   \ddots    &   \vdots\\  
            0    &   0    &   \lambda_r    &   *    &   *\\  
            0    &   0    &   0    &   \begin{pmatrix}
                a_1    &   b_1    \\ 
                c_1    &   d_1    
            \end{pmatrix}&   *\\  
            0    &   0    &  0     &   0    &   \begin{pmatrix}
                a_s    &   b_s    \\ 
                c_s    &   d_s    
            \end{pmatrix}
        \end{pmatrix}.
    \end{equation}
    Le déterminant de \( A\) est le produit des déterminants des blocs diagonaux et les valeurs propres de \( A\) sont les \( \lambda_1,\ldots, \lambda_r\) et celles de ces blocs.
\end{lemma}

\begin{proof}
    Si la matrice \( A\) a des valeurs propres réelles, nous procédons comme dans le cas complexe. Cela nous fournit le partie véritablement triangulaire avec les valeurs propres \( \lambda_1,\ldots, \lambda_r\) sur la diagonale. Supposons donc que \( A\) n'a pas de valeurs propres réelles. Soit donc \( \alpha+i\beta \) une valeur propre (\( \beta\neq 0\)) et \( u+iv\) un vecteur propre correspondant où \( u\) et \( v\) sont des vecteurs réels. Nous avons
    \begin{equation}
        Au+iAv=A(u+iv)=(\alpha+i\beta)(u+iv)=\alpha u-\beta v+i(\alpha v+\beta v),
    \end{equation}
    et en égalisant les parties réelles et imaginaires,
    \begin{subequations}
        \begin{align}
            Au&=\alpha u-\beta v\\
            Av&=\alpha v+\beta u.
        \end{align}
    \end{subequations}
    Sur ces relations nous voyons que ni \( u\) ni \( v\) ne sont nuls. De plus \( u\) et \( v\) sont linéairement indépendants (sur \( \eR\)), en effet si \( v=\lambda u\) nous aurions \( Au=\alpha u-\beta\lambda u=(\alpha-\beta\lambda)u\), ce qui serait une valeur propre réelle alors que nous avions supposé avoir déjà épuisé toutes les valeurs propres réelles.

    Étant donné que \( u\) et \( v\) sont deux vecteurs réels non nuls et linéairement indépendants, nous pouvons trouver une base orthonormée \( \{ q_1,q_2 \}\) de \( \Span\{ u,v \}\). Nous pouvons étendre ces deux vecteurs en une base orthonormée \( \{ q_1,q_2,q_3,\ldots, q_n \}\) de \( \eR^n\). Nous considérons à présent la matrice orthogonale dont les colonnes sont formées de ces vecteurs : \( Q=[q_1\,q_2\,\ldots q_n]\).

    L'espace \( \Span\{ e_1,e_2 \}\) est stable par \( Q^{-1} AQ\), en effet nous avons
    \begin{equation}
        Q^{-1} AQe_1=Q^{-1} Aq_1=Q^{-1}(aq_1+bq_2)=ae_1+be_2.
    \end{equation}
    La matrice \( Q^{-1}AQ\) est donc de la forme
    \begin{equation}
        Q^{-1} AQ=\begin{pmatrix}
            \begin{pmatrix}
                \cdot    &   \cdot    \\ 
                \cdot    &   \cdot    
            \end{pmatrix}&   C_1    \\ 
            0    &   A_1    
        \end{pmatrix}
    \end{equation}
    où \( C_1\) est une matrice réelle \( 2\times (n-1)\) quelconque et \( A_1\) est une matrice réelle \( (n-2)\times (n-2)\). Nous pouvons appliquer une récurrence sur la dimension pour poursuivre.

    Notons que si \( A\) n'a pas de valeurs propres réelles, elle est automatiquement d'ordre pair parce que les valeurs propres complexes viennent par couple complexes conjuguées.

    En ce qui concerne les valeurs propres, il est facile de voir en regardant \eqref{EqMtrTSqRTA} que les valeurs propres sont celles des blocs diagonaux. Étant donné que \( QAQ^{-1}\) et \( A\) ont même polynôme caractéristique, ce sont les valeurs propres de \( A\).
\end{proof}

\begin{theorem} \label{ThoeTMXla}
    Le spectre d'une matrice symétrique réelle est réel. Les matrices symétriques sont diagonalisables par une matrice orthogonale.
\end{theorem}

\begin{proof}
    Soit \( A\) une matrice réelle symétrique. Si \( \lambda\) est une valeur propre complexe pour le vecteur propre complexe \( v\), alors d'une part \( \langle Av, v\rangle =\lambda\langle v, v\rangle \) et d'autre part \( \langle Av, v\rangle =\langle v, Av\rangle =\bar\lambda\langle v, v\rangle \). Par conséquent \( \lambda=\bar\lambda\).
    
    Le lemme de Schur réel \ref{LemSchureRelnrqfiy} donne une matrice orthogonale qui trigonalise \( A\). Les valeurs propres étant toutes réelles, la matrice \( QAQ^{-1}\) est même triangulaire (il n'y a pas de blocs dans la forme \eqref{EqMtrTSqRTA}). Prouvons que \( QAQ^{-1}\) est symétrique :
    \begin{equation}
        (QAQ^{-1})^t=(Q^{-1})^tA^tQ^t=QA^tQ^{-1}=QAQ^{-1}
    \end{equation}
    où nous avons utilisé le fait que \( Q\) était orthogonale (\( Q^{-1}=Q^t\)) et que \( A\) était symétrique (\( A^t=A\)). Une matrice triangulaire supérieure symétrique est obligatoirement une matrice diagonale.
\end{proof}
