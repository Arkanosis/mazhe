% This is part of Outils mathématiques
% Copyright (c) 2012
%   Laurent Claessens
% See the file fdl-1.3.txt for copying conditions.

\begin{corrige}{OutilsMath-0144}

La première intégrale est très simple. Il faut intégrer
\begin{equation}
    \int_0^1\left( \int_0^1 (x^3y^2+x^2y^3)dx  \right)dy.
\end{equation}

\begin{verbatim}
sage: f(x,y)=x**3*y**2+x**2*y**3                                                                                                                             
sage: f.integrate(x,0,1).integrate(y,0,1)
(x, y) |--> 1/6
\end{verbatim}
La réponse est donc \( 1/6\).

Pour la seconde, le domaine est le triangle de sommets \( (0,1)\), \( (0,0)\) et \( (1,0)\). Les bornes sont donc
\begin{subequations}
    \begin{numcases}{}
        x\colon 0\to 1\\
        y\colon 0\to 1-x.
    \end{numcases}
\end{subequations}
L'intégrale à calculer est alors
\begin{equation}
    \int_0^1dx\left( \int_0^{1-x} x^3y\,dy \right).
\end{equation}
Notez l'importance de l'ordre des bornes. Nous avons
\begin{verbatim}
sage: f(x,y)=x**3*y                                                                                                                                         
sage: f.integrate(y,0,1-x).integrate(x,0,1)                                                                                                                  
(x, y) |--> 1/120  
\end{verbatim}
La réponse est donc \( 1/120\).


\end{corrige}
