% This is part of Mes notes de mathématique
% Copyright (c) 2011-2013
%   Laurent Claessens
% See the file fdl-1.3.txt for copying conditions.

%+++++++++++++++++++++++++++++++++++++++++++++++++++++++++++++++++++++++++++++++++++++++++++++++++++++++++++++++++++++++++++
\section{Séries entières}
%+++++++++++++++++++++++++++++++++++++++++++++++++++++++++++++++++++++++++++++++++++++++++++++++++++++++++++++++++++++++++++

Source : \cite{RomainBoilEnt}. 

Dans cette section nous allons parler de séries complexes autant que de séries réelles. L'étude des propriétés à proprement parler complexes des séries entières (holomorphie) sera effectuée dans le chapitre dédié, voir le théorème \ref{ThomcPOdd} et ses conséquences.

Une \defe{série de puissance}{série!de puissance} est une série de la forme
\begin{equation}		\label{eqseriepuissance}
	\sum_{k=0}^{\infty}c_k(z-z_0)^k
\end{equation}
où $z_0\in \eC$ est fixé, $(c_k)$ est une suite complexe fixée, et $z$ est un paramètre complexe. Nous disons que cette série est \emph{centrée} en $z_0$.

\begin{definition}
    Une \defe{série entière}{série!entière} est une somme de la forme
    \begin{equation}
        \sum_{n=0}^{\infty}a_nz^n
    \end{equation}
    avec \( a_n,z\in\eC\).    
\end{definition}
Une série entière peut définir une fonction
\begin{equation}
    f(z)=\sum_na_nz^n.
\end{equation}
Le but de cette section est d'étudier des conditions sur la suite \( (a_n)\) qui assurent la continuité de \( f\) ou la possibilité de dériver ou intégrer la série terme à terme.


\begin{lemma}[Critère d'Abel]\index{critère!Abel}   \label{LemmbWnFI}
    Soit \( (a_n)\) une suite dans \( \eC\) et \( r>0\). Si la suite \( (a_nr^n)\) est bornée alors pour tout \( z\in B(0,r)\) la série \( \sum a_nz^n\) converge absolument.
\end{lemma}

\begin{proof}
    Soit \( M\in \eR\) tel que \( | a_n |r^n\leq M\) pour tout \( n\). Alors nous avons
    \begin{equation}
        | a_nz^n |=| a_n |r^n\big( \frac{ | z | }{ r } \big)^n\leq M\left( \frac{ | z | }{ r } \right)^n
    \end{equation}
    Si \( | z |<r\) alors nous tombons sur la série géométrique qui converge. Par le critère de comparaison la série \( \sum_{n=0}^{\infty}| a_nz^n |\) converge.
\end{proof}

\begin{definition}  \label{DefZWKOZOl}
    Soit \( \sum_{n\in \eN}a_nz^n\) une série entière. Le \defe{rayon de convergence}{rayon!de convergence} de cette série est le nombre
    \begin{equation}
        R=\sup\{ r\in \eR^+\tq \text{la suite \((a_nr^n)\) est bornée} \}\in\mathopen[ 0 , \infty \mathclose].
    \end{equation}
    Étant donné que cela est une propriété de la suite \( (a_n)\) et non réellement de la série \( \sum_ka_kz^k\), nous allons aussi dire que c'est le rayon de convergence de la suite \( (a_n)\).
\end{definition}
Le rayon de convergence d'une série ne dépend que des réels \( | a_n |\), même si à la base \( a_n\in \eC\).

\begin{theorem} \label{ThoLPWeIHE}
    Soit \( R>0\) le rayon de convergence de la somme \( \sum_na_nz^n\) et \( z\in \eC\).
    \begin{enumerate}
        \item
            Si \( | z |<R\) alors la série converge absolument.
        \item
            Si \( R<\infty\) et si \( | z |>R\) alors la suite \( (a_nz^n)\) n'est pas bornée et la série diverge.
    \end{enumerate}
\end{theorem}

\begin{proof}
    \begin{enumerate}
        \item
            Étant donné que \( | z |<R\), il existe \( r>0\) tel que \( | z |<r<R\). On a que \( (a_nr^n)\) est borné (parce que \( R\) est le supremum) et donc \( (a_n| z_n |)\) est bornée. Le critère d'Abel conclu.
        \item
            Par hypothèse la suite \( (a_n| z |^n)\) n'est pas bornée. La suite \( (a_nz^n)\) n'est donc pas bornée non plus et la série ne peut pas converger.
    \end{enumerate}
\end{proof}

Le théorème \ref{ThoLPWeIHE} parle bien de convergence absolue, et non de convergence normale. Pour chaque \( t\), la série \( \sum_k | a_nt^k |\) converge. Si par contre nous posons \( u_k(t)=a_kt^k\), nous n'avons a priori pas la convergence normale \( \sum_k\| u_k \|_{\infty}\), même pas si la norme est la norme supremum sur \( B(0,R)\)\quext{Il y aurait par contre bien convergence sur tout compact ? Cher lecteur, dites moi ce que vous en pensez}. Prenons comme exemple simplement \( a_k=1\) pour tout \( k\). Pour tout \( | t |<1\), la série \( \sum_k t^k\) converge absolument (série géométrique), mais nous aurions \( \| u_k \|_{\infty}=1\) et donc divergence évidente de \( \sum_k\| u_k \|_{\infty}\).

\begin{theorem}[Formule de Hadamard]\index{formule!Hadamard}\index{Hadamard!formule}		\label{ThoSerPuissRap}
Le rayon de convergence de la série entière \( \sum_n c_n z^n\) est donné par une des deux formules
\begin{equation}		\label{EqRayCOnvSer}
	\frac{1}{ R } =\limsup\sqrt[k]{| a_k |}
\end{equation}
ou
\begin{equation}		\label{EqAlphaSerPuissAtern}
	\frac{1}{ R }=\limite k \infty \abs{\frac{a_{k+1}}{a_k}}
\end{equation}
lorsque $a_k$ est non nul à partir d'un certain $k$.
\end{theorem}

Le disque $| z-z_0 |\leq R$ est le \defe{disque de convergence}{disque de convergence} de la série \( \sum_n a_n(z-z_0)^n\). Notons que le critère d'Abel ne dit rien pour les points tels que $| z-z_0 |=R$. Il faut traiter ces points au cas par cas. Et le pire, c'est qu'une série donnée peut converger pour certain des points sur le bord du disque, et diverger en d'autres. Le théorème d'Abel radial (théorème \ref{ThoLUXVjs}) nous donnera quelque informations sur le sujet.

Il y a un dessin à la figure \ref{LabelFigDisqueConv}.
\newcommand{\CaptionFigDisqueConv}{À l'intérieur du disque de convergence, la convergence est absolue. En dehors, la série diverge. Sur le cercle proprement dit, tout peut arriver.}
\input{Fig_DisqueConv.pstricks}

Si les suites \( a_n\) et \( b_n\) sont équivalentes, alors les séries correspondantes auront le même rayon de convergence. Cela ne signifie pas que sur le bord du disque de convergence, elles aient même comportement. Par exemple nous avons
\begin{equation}
    \frac{1}{ \sqrt{n} }\sim \frac{1}{ \sqrt{n} }+\frac{ (-1)^n }{ n }.
\end{equation}
En même temps, en \( z=-1\) la série 
\begin{equation}
    \sum_{n\geq 1}\frac{ z^n }{ \sqrt{n} }
\end{equation}
converge par le critère des séries alternées (corollaire \ref{CoreMjIfw}). Par contre la série
\begin{equation}
    \sum_{n\geq 1}\left( \frac{1}{ \sqrt{n} }+\frac{ (-1)^n }{ n } \right)z^n
\end{equation}
ne converge pas pour \( z=-1\).

\begin{example}
    Soit \( \alpha\in \eR\) et considérons la série \( \sum_{n\geq 1}a_nz^n\) où \( a_n\) est la \( n\)-ième décimale de \( \alpha\). Si \( \alpha\) est un nombre décimal limité, la suite \( (a_n)\) est finie et le rayon de convergence est infini. Sinon, pour tout \( N\) il existe un \( n>N\) tel que \( a_n\neq 0\) et la suite \( (a_n)\) ne tend pas vers zéro. Par conséquent la série
    \begin{equation}
        \sum_{n}a_nz^n
    \end{equation}
    diverge pour \( z=1\) et le rayon de convergence satisfait \( R\leq 1\). Nous avons aussi \( | a_n |\leq 9\), de telle manière à ce que la série soit bornée et par conséquent majorée en module par \( 9z^n\), ce qui signifie que \( R\geq 1\). 

    Nous déduisons alors \( R=1\).
\end{example}


%---------------------------------------------------------------------------------------------------------------------------
\subsection{Propriétés de la somme}
%---------------------------------------------------------------------------------------------------------------------------

\begin{theorem}     \label{ThokPTXYC}
    Soient \( \sum_na_nz^n\) et \( \sum b_nz^n\) deux séries de rayon de convergences respectivement \( R_a\) et \( R_b\).
    \begin{enumerate}
        \item   \label{IteWlajij}
            Si \( R_s\) est le rayon de convergence de \( \sum_n(a_n+b_n)z^n\), nous avons
            \begin{equation}
                R_s\geq \min\{ R_a,R_b \}
            \end{equation}
            et nous avons l'égalité si pour tout \( |z |\leq\min\{ R_a,R_b \}\), \( \sum (a_n+b_n)z^n=\sum_n a_nz^n+\sum_nb_nz^n\).
        \item
            Si \( \lambda\neq 0\) la série \( \sum_n(\lambda a_n)z^n\) a le même rayon de convergence que la série \( \sum_na_nz^n\) et si \( | z |<R_a\) nous avons
            \begin{equation}
                \sum_{n=0}^{\infty}(\lambda a_n)z^n=\lambda\sum_{n=0}^{\infty}a_nz^n.
            \end{equation}
        \item
            Le \defe{produit de Cauchy}{Cauchy!produit}\index{produit!de Cauchy} des deux séries est donné par
            \begin{equation}
                \sum_{n=0}^{\infty}\left( \sum_{i+j=n}a_ib_j \right)z^n=\sum_{n=0}^{\infty}\left( \sum_{k=0}^{n}a_kb_{n-k} \right)z^n.
            \end{equation}
            Si \( R_p\) est le rayon de convergence de ce produit nous avons
            \begin{equation}
                R_p\geq \min\{ R_a,R_b \}
            \end{equation}
            et si \( | z |<\min\{ R_a,R_b \}\) alors
            \begin{equation}
                \sum_{n=0}^{\infty}\left( \sum_{i+j=n}a_ib_j \right)z^n=\left( \sum_{n=0}^{\infty}a_nz^n \right)\left( \sum_{n=0}^{\infty}b_nz^n \right).
            \end{equation}
            
    \end{enumerate}
    
\end{theorem}

\begin{proof}
    Nous prouvons la partie sur le produit de Cauchy. En utilisant la propriété du produit de la somme par un scalaire nous avons
    \begin{subequations}
        \begin{align}
            \left( \sum_{n=0}^{\infty}a_nz^n \right)\left( \sum_{m=0}^{\infty}b_mz^m \right)&=\sum_{n=0}^{\infty}\left( \sum_{m=0}^{\infty}b_ma_nz^{m+n} \right)\\
            &=\lim_{N\to \infty} \lim_{M\to \infty} \sum_{n=0}^N\sum_{m=0}^Mb_ma_nz^{m+n}\\
            &=\lim_{N\to \infty} \lim_{M\to \infty} \sum_{k=0}^{N+M}\sum_{i+k=k}b_ia_jz^k\\
            &=\lim_{N\to \infty} \sum_{k=0}^{\infty}\sum_{i+k=k}b_ia_jz^k\\
            &=\sum_{k=0}^{\infty}\sum_{i+j=k}b_ia_jz^k.
        \end{align}
    \end{subequations}
\end{proof}

\begin{example}
    Montrons un produit de Cauchy dont le rayon de convergence est strictement plus grand que le minimum. D'abord nous considérons
    \begin{equation}
        A=1-z,
    \end{equation}
    c'est à dire \( a_0=1\), \( a_1=-1\), \( a_{n\geq 2}=0\) avec \( R_a=\infty\). Ensuite nous considérons
    \begin{equation}
        B=\sum_nz^n,
    \end{equation}
    c'est à dire \( B=(1-z)^{-1}\) et \( R_b=1\). Le produit de Cauchy de ces deux séries valant \( 1\), le rayon de convergence est infini.
\end{example}

\Exo{reserve0005}

\begin{theorem}
    Une série entière converge normalement sur tout disque fermé inclus au disque de convergence.
\end{theorem}

\begin{proof}
    Toute boule fermée inclue à \( B(0,R)\) est inclue à la boule \( \overline{ B(0,r) }\) pour un certain \( r<R\). Nous nous concentrons donc sur une telle boule fermée.

    Pour chaque \( n\) nous posons \( u_n(z)=a_nz^n\) que nous voyons comme une fonction sur \( \overline{ B(0,r) }\). Pour tout \( n\in \eN\) et tout \( z\in\overline{ B(0,r) }\) nous avons 
    \begin{equation}
        \| u_n \|_{\infty}\leq| a_nz^n |\leq | a_n |r^n.
    \end{equation}
    Étant donné que \( r<R\) la série \( \sum_n | a_n |r^n\) converge et la série \( \sum_n\| u_n \|\) est convergente. La série \( \sum_na_nz^n\) est alors normalement convergente.
\end{proof}

\begin{example}
    Encore une fois nous n'avons pas d'informations sur le comportement au bord. Par exemple la série \( \sum_nz^n\) a pour rayon de convergence \( R=1\), mais \( \sup_{z\in B(0,1)}| z^n |=1\) de telle façon à ce que nous n'avons pas de convergence normale sur la boule fermée.
\end{example}

La convergence normale n'est donc pas de mise sur tout l'intérieur du disque de convergence. La continuité, par contre est effective sur la boule. En effet si \( z_0\in B(0,R)\) alors il existe un rayon \( 0<r<R\) tel que \( B(z_0,r)\subset B(0,R)\). Sur \( B(z_0,r)\) nous avons convergence normale et donc continuité en \( z_0\).

La différence est que la continuité est une propriété locale tandis que la convergence normale est une propriété globale.

\begin{proposition}
    Soit \( f(z)=\sum_na_nz^n\) avec un rayon de convergence \( R\). Si \( \sum | a_n |R^n\) converge alors
    \begin{enumerate}
        \item
            la série \( \sum_na_nz^n\) converge normalement sur \( \overline{ B(0,R) }\),
        \item
            \( f\) est continue sur \( \overline{ B(0,R) }\).
    \end{enumerate}
\end{proposition}

\begin{proof}
    La conclusion est claire dans l'intérieur du disque de convergence. En ce qui concerne le bord, chacune des sommes partielles est une fonction continue. De plus nous avons \( \| u_n \|\leq | a_n |R^n\), dont la série converge. Par conséquent nous avons convergence normale sur le disque fermé.
\end{proof}

Le théorème suivant permet de donner, dans le cas de fonctions réelle, des informations sur la convergence en une des deux extrémités de l'intervalle de convergence.
\begin{theorem}[Convergene radiale de Abel]\index{Abel!convergence radiale} \label{ThoLUXVjs}
    Soit \( f(x)=\sum_na_nx^n\) une série réelle de rayon de convergence \( 0<R<\infty\).
    \begin{enumerate}
        \item
            Si \( \sum a_nR^n\) converge, alors \( f\) est continue sur \( \mathopen[ 0 , R \mathclose]\).
        \item
            Si \( \sum_na_n(-R)^n\) converge, alors \( f\) est continue sur \( \mathopen[ -R , 0 \mathclose]\).
    \end{enumerate}
\end{theorem}

\Exo{reserve0006}

Le résultat suivant permet d'identifier deux séries complexes lorsque leurs valeurs sur \( \eR\) sont identiques.
\begin{proposition}
    Soient les séries \( f(z)=\sum a_nz^n\) et \( g(z)=\sum b_n z^n\) convergentes dans \( B(0,R)\). Si \( f(x)=g(x)\) pour \( x\in \mathopen[ 0 , R [\) alors \( a_n=b_n\).
\end{proposition}

\begin{proof}
    Soit \( n_0\) le plus petit entier tel que \( a_{n_0}\neq b_{n_0}\). Pour tout \( z\in B(0,R)\) nous avons
    \begin{equation}
        f(z)-g(z)=\sum_{n=n_0}^{\infty}(a_n-b_n)z^n=z^{n_0}\varphi(z)
    \end{equation}
    où
    \begin{equation}
        \varphi(z)=\sum_{n\geq 0}(a_{n+n_0}-b_{n+n_0})z^n.
    \end{equation}
    Par le théorème \ref{ThokPTXYC}\ref{IteWlajij} le rayon de convergence de \( \varphi\) est plus grand que \( R\) et la fonction \( \varphi\) est continue en \( 0\). Étant donné que \( \varphi(0)=a_{n_0}-b_{n_0}\neq 0\) et que \( \varphi\) est continue nous avons un \( \rho\) tel que \( \varphi\neq 0\) sur \( B(0,\rho)\). Or cela n'est pas possible parce que au moins sur la partie réelle de cette dernière boule, \( \varphi\) doit être nulle.
\end{proof}

\begin{lemma}       \label{LemFVMaSD}
    Soit une série entière \( \sum a_nz^n\) de rayon de convergence \( R\). Les séries
    \begin{equation}
        \sum \frac{ a_n }{ n+1 }z^{n+1}
    \end{equation}
    et
    \begin{equation}
        \sum_{n\geq 1}na_nz^{n-1}
    \end{equation}
    ont même rayon de convergence \( R\).
\end{lemma}

Notons toutefois que nonobstant ce lemme, les séries dont il est question peuvent se comporter différemment sur le bord du disque de convergence. En effet la série
\begin{equation}
    \sum \frac{1}{ n }z^n
\end{equation}
diverge pour \( z=1\) alors que 
\begin{equation}
    \sum\frac{1}{ n(n+1) }z^{n+1}
\end{equation}
converge pour \( z=1\).

%---------------------------------------------------------------------------------------------------------------------------
\subsection{Dérivation, intégration}
%---------------------------------------------------------------------------------------------------------------------------

Les théorèmes de dérivation et d'intégration de séries de fonctions (théorèmes \ref{ThoCciOlZ} et \ref{ThoCSGaPY}) fonctionnent bien dans le cas des séries entières.

\begin{proposition} \label{PropfeFQWr}
    Soit la série entière $\sum a_nx^n$ de rayon de convergence \( R\). Pour tout segment \( \mathopen[ a , b \mathclose]\subset\mathopen] -R , R \mathclose[\) nous pouvons intégrer terme à terme :
    \begin{equation}
        \int_a^b\sum_{n=0}^{\infty}a_nx^ndt=\sum_{n=0}^{\infty}a_n\int_a^bt^ndt.
    \end{equation}
\end{proposition}

\begin{proof}
    Ceci est un cas particulier du théorème général \ref{ThoCciOlZ}. Notons que par le lemme \ref{LemFVMaSD}, la série entière qui intègre la série de \( f\) terme à terme a le même rayon de convergence que celui de \( f\).
\end{proof}

\begin{proposition}     \label{ProptzOIuG}
    Soit la série entière
    \begin{equation}
        f(x)=\sum_{n=0}^{\infty}a_n x^n
    \end{equation}
    de rayon de convergence \( R\). Alors la fonction \( f\) est \( C^1\) sur \( \mathopen] -R , R \mathclose[\) et se dérive terme à terme :
    \begin{equation}
        f'(x)=\sum_{n=1}^{\infty}na_nx^{n-1}
    \end{equation}
    pour tout \( x\in\mathopen] -R , R \mathclose[\).
\end{proposition}

\begin{proof}
    Nous savons que la série \( \sum_{n=1}^{\infty}na_nx^{n-1}\) a le même rayon de convergente que celui de la série \( f\). En particulier cette série des dérivées converge normalement sur tout compact dans \( \mathopen] -R , R \mathclose[\) et la somme est continue. Le théorème \ref{ThoCSGaPY} conclu.
\end{proof}

\begin{example}
    Montrons que la fonction
    \begin{equation}
        \begin{aligned}
            f\colon \eR_+\setminus\{ 0,1 \}&\to \eR \\
            x&\mapsto \frac{ \ln(x) }{ x-1 } 
        \end{aligned}
    \end{equation}
    admet un prolongement \( C^{\infty}\) sur \( \eR_+\setminus\{ 0 \}\).

    Nous allons étudier la fonction
    \begin{equation}
        f(x)=\frac{ \ln(1+x) }{ x }
    \end{equation}
    autour de \( x=0\). Le logarithme ne pose pas de problèmes à développer dans un voisinage :
    \begin{subequations}
        \begin{align}
            f(x)&=\frac{1}{ x }\sum_{n=1}^{\infty}\frac{ (-1)^{n+1} }{ n }x^n\\
            &=\sum_{n=1}^{\infty}\frac{ (-1)^{n+1} }{ n }x^{n-1}\\
            &=\sum_{n=0}^{\infty}\frac{ (-1)^k }{ k+1 }x^k.
        \end{align}
    \end{subequations}
    Cette série a un rayon de convergence égal à \( 1\), et donc définit sans problèmes une fonction \( C^{\infty}\) dans un voisinage de \( x=0\). Notons que par convention \( x^0=1\) même si \( x=0\).
\end{example}

%--------------------------------------------------------------------------------------------------------------------------- 
\subsection{Série génératrice d'une suite}
%---------------------------------------------------------------------------------------------------------------------------

Avant de commencer, une petite formule de dérivation toute simple que nous allons utiliser souvent :
\begin{equation}        \label{EqSOFdwhw}
    (z^k)^{(l)}=\begin{cases}
        0   &   \text{si \( l>k\)}\\
        \frac{ k! }{ (k-l)! }z^{k-l}    &    \text{sinon.}
    \end{cases}
\end{equation}

Soit \( u_n\) une suite telle que le rayon de convergence de
\begin{equation}
    f(z)=\sum_{n=0}^{\infty}u_nz^n
\end{equation}
soit strictement positif. Alors la série \( f\) est la \defe{série génératrice}{série!génératrice d'une suite} de la suite \( (u_n)\).

Grâce au théorème \ref{ProptzOIuG} nous pouvons la dériver terme à terme autour de \( z=0\). En utilisant la petite formule \eqref{EqSOFdwhw} nous trouvons
\begin{equation}    \label{EqNGhVCpP}
    f^{(l)}(z)=\sum_{n=l}^{\infty}u_n\frac{ n! }{ (n-l)! }z^{n-l},
\end{equation}
et donc
\begin{equation}
    u_l=\frac{ f^{(l)}(0) }{ l! }.
\end{equation}
D'où le nom de série génératrice. Cela est évidemment intéressant seulement si nous connaissons une autre forme pour \( f\) par ailleurs. 

Nous en utiliserons une pour déterminer les partitions d'un nombre en parts fixes, proposition \ref{PropWUFpuBR}.

%---------------------------------------------------------------------------------------------------------------------------
\subsection{Développement en série et Taylor}
%---------------------------------------------------------------------------------------------------------------------------

\begin{definition}  \label{DefwmRzKh}
    Soit une fonction \( f\colon \eC\to \eC\) et \( z_0\in \eC\). Nous disons que \( f\) est \defe{développable en série entière}{développable!en série entière} dans un voisinage de \( z_0\) si il existe une série \( \sum_n a_nz^n\) de rayon de convergence \( R>0\) et \( r\leq R\) tel que
    \begin{equation}
        f(z)=\sum_{n=0}^{\infty}a_n(z-z_0)^n
    \end{equation}
    pour tout \( z\in B(z_0,r)\).
\end{definition}

\begin{proposition}
    Si \( V\) est un ouvert dans \( \eC\) alors l'ensemble des fonctions \( V\to \eC\) développables en série entière forme une \( \eC\)-algèbre.
\end{proposition}

\begin{proof}
    Les séries entières passent aux sommes et aux produits en gardant des rayons de convergence non nuls.
\end{proof}

\begin{proposition} \label{ThoTGPtDj}
    Si \( f\) est développable en série entière à l'origine alors elle est \( C^{\infty}\) sur un voisinage de l'origine et le développement est celui de \defe{Taylor}{Taylor!série entière} :
    \begin{equation}
        f(x)=\sum_{n=0}^{\infty}\frac{ f^{(n)}(0) }{ n! }x^n
    \end{equation}
    pour tout \( x\) dans un voisinage de \( 0\).
\end{proposition}

\begin{proof}
    Si \( f(x)=\sum a_nx^n\), nous savons que \( f\) est \( C^1\) et que nous pouvons dériver terme à terme (au moins dans un voisinage). De plus le fait de dériver ne change pas le domaine. Par récurrence, la fonction est \( C^{\infty}\) sur le voisinage. En dérivant \( k\) fois la série \( \sum a_nx^n\) nous trouvons
    \begin{equation}
        f^{(k)}(x)=\sum_{n=k}^{\infty}n(n-1)\ldots (n-k+1)a_nx^{n-k}.
    \end{equation}
    En calculant en \( x=0\) nous trouvons
    \begin{equation}
        f^{(k)}(0)=k! a_k,
    \end{equation}
    d'où le terme général
    \begin{equation}
        a_k=\frac{ f^{(k)}(0) }{ k! }.
    \end{equation}
\end{proof}

Si \( f\) est une fonction et si la série
\begin{equation}
    T_f(x)=\sum_{n=0}^{\infty}\frac{ f^{(n)}(0) }{ n! }x^n
\end{equation}
converge, alors cette série est la \defe{série de Taylor}{série!Taylor} de \( f\).

\begin{remark}
    La série de Taylor d'une fonction n'est pas liée à sa fonction de façon aussi raide qu'on pourrait le croire. Même dans le cas d'une fonction \( C^{\infty}\) il peut arriver que \( T_f(x)\neq f(x)\).
    
    Il peut aussi arriver que \( f\) ne soit pas développable en série entières.
\end{remark}

\begin{example}
    Nous considérons la fonction
    \begin{equation}
        f(x)=\begin{cases}
            e^{-1/x^2}    &   \text{si \( x\neq 0\)}\\
            0    &    \text{si \( x=0\).}
        \end{cases}
    \end{equation}
    Nous avons
    \begin{equation}
        f'(x)=\begin{cases}
            \frac{ 2 }{ x^3 } e^{-1/x^2}    &   \text{si \( x\neq 0\)}\\
            0    &    \text{si \( x=0\)}.
        \end{cases}
    \end{equation}
    Note : pour la seconde ligne nous devons faire explicitement le calcul
    \begin{equation}
        f'(0)=\lim_{t\to 0} \frac{ f(t)-f(0) }{ t }=\lim_{y\to 0} \frac{1}{ t } e^{-1/t^2}=0.
    \end{equation}
    Plus généralement nous avons \( f^{(k)}(0)=0\), et par conséquent la série de Taylor converge (trivialement) vers la fonction identiquement nulle.

    Cette fonction n'est donc pas développable en série entière vu qu'il n'existe aucun voisinage de zéro sur lequel la série de \( f\) coïncide avec \( f\).
\end{example}

\begin{example}     \label{ExwobBAW}
    Développement de \( f(x)=\arctan(x)\). Nous savons que
    \begin{equation}
        f'(x)=\frac{1}{ 1+x^2 },
    \end{equation}
    alors que nous connaissons le développement
    \begin{equation}    \label{EqVmuaqT}
        \frac{1}{ 1-x }=\sum_{n=0}^{\infty}x^n
    \end{equation}
    pour tout \( x\in B(0,1)\). Nous avons donc successivement
    \begin{subequations}
        \begin{align}
            \frac{1}{ 1+x }&=\sum_{n=0}(-x)^n\\
            \frac{ 1 }{ 1+x^2 }&=\sum_{n=0}(-1)^nx^{2n}\\
            \arctan(x)&=\sum_{n=1}^{\infty}(-1)^n\frac{ x^{2n+1} }{ 2n+1 }+C.
        \end{align}
    \end{subequations}
    Notons que dans la dernière nous avons évité d'écrire la somme depuis \( n=0\) (qui serait un terme constat) et nous avons écris explicitement «\( +C\)». Étant donné que \( \arctan(0)=0\), nous devons poser \( C=0\) et donc
    \begin{equation}
        \arctan(x)=\sum_{n=1}^{\infty}(-1)^n\frac{ x^{2n+1} }{ 2n+1 }.
    \end{equation}
\end{example}

%---------------------------------------------------------------------------------------------------------------------------
\subsection{Resommer une série}
%---------------------------------------------------------------------------------------------------------------------------

Nous avons vu comment trouver la série correspondant à une fonction donnée. Un exercice difficile consiste à trouver la fonction qui correspond à une somme donnée. Pour des techniques de calculs de sommes, voir \cite{DAnSerEntiere}.

%///////////////////////////////////////////////////////////////////////////////////////////////////////////////////////////
\subsubsection{Les sommes du type \texorpdfstring{$ \sum_nP(n)x^n$}{P}}
%///////////////////////////////////////////////////////////////////////////////////////////////////////////////////////////

Pour calculer 
\begin{equation}
    \sum_{n=0}^{\infty}P(n)x^n
\end{equation}
où \( P\) est un polynôme de degré \( m\) nous commençons par écrire
\begin{equation}
    P(n)=\alpha_0+\alpha_1(n+1)+\alpha_2(n+1)(n+2)+\ldots +\alpha_m(n+1)\ldots (n+m).
\end{equation}
Nous décomposons alors la somme en \( m\) sommes de la forme
\begin{equation}
    \sum_{n=0}^{\infty}\alpha_k\frac{ (n+k)! }{ n! }x^n=\alpha_k\left( \sum_{n=0}^{\infty}x^{n+k} \right)^{(k)}.
\end{equation}
Effectuons par exemple\footnote{Je crois qu'ici il y a une faute de signe dans \cite{DAnSerEntiere}.}
\begin{equation}
    \sum_{n=0}^{\infty}x^{n+3}=\frac{1}{ 1-x }-1-x-x^2
\end{equation}
Notons que dans un usage pratique, ce terme devra être ensuite dérivé trois fois, de telle manière à ce que les termes «correctifs» n'interviennent pas. Cette méthode ne demande donc que de calculer les dérivées successives de \( 1/(1-x)\).

\begin{example}
    Calculons la fonction
    \begin{equation}
        f(x)=\sum_{n=0}^{\infty}n^3x^n.
    \end{equation}
    D'abord nous écrivons
    \begin{equation}
        n^3=-1+7(n+1)-6(n+1)(n+2)+(n+1)(n+2)(n+3).
    \end{equation}
    Nous avons 
    \begin{equation}
        \sum_{n=0}^{\infty}(n+1)x^n=\left( \sum_{n=0}^{\infty}x^{n+1} \right)'=\left( \frac{1}{ 1-x }-1 \right)'=\frac{1}{ (x-1)^2 }.
    \end{equation}
    De la même façon,
    \begin{subequations}
        \begin{align}
            \sum_n (n+1)(n+2)x^n&=\left( \sum x^{n+2} \right)''=\frac{ -2 }{ (x-1)^3 }\\
            \sum_n (n+1)(n+2)(n+3)x^n=\frac{ 6 }{ (x-1)^4 }.
        \end{align}
    \end{subequations}
    En remettant tout ensemble nous obtenons
    \begin{equation}
        \sum_{n=0}^{\infty}n^3x^n=-\frac{1}{ 1-x }+\frac{ 7 }{ (x-1)^2 }+\frac{ 12 }{ (x-1)^3 }+\frac{ 6 }{ (x-1)^4 }.
    \end{equation}

    Nous pouvons vérifier ce résultat en traçant les deux courbes et en remarquant qu'elles coïncident.
\begin{verbatim}
----------------------------------------------------------------------
| Sage Version 4.7.1, Release Date: 2011-08-11                       |
| Type notebook() for the GUI, and license() for information.        |
----------------------------------------------------------------------
sage: n=var('n')
sage: S(x)=sum(  [ n**3*x**n for n in range(0,30)  ]   )
sage: f(x)=-1/(1-x)+7/((x-1)**2)+12/((x-1)**3)+6/( (x-1)**4  )
sage: S(0.1)
0.214906264288980
sage: f(0.1)
0.214906264288981
sage: f.plot(-0.5,0.5)+S.plot(-0.5,0.5)
\end{verbatim}

\end{example}

%///////////////////////////////////////////////////////////////////////////////////////////////////////////////////////////
\subsubsection{Les sommes du type \texorpdfstring{$ \sum_nx^n/P(n)$}{P}}
%///////////////////////////////////////////////////////////////////////////////////////////////////////////////////////////

Si \( P(n)\) a des racines entières, nous pouvons le décomposer en fractions simples et utiliser la somme
\begin{equation}
    \sum_{n=1}^{\infty}\frac{ x^n }{ n }=-\ln(1-x).
\end{equation}
Nous avons par exemple
\begin{subequations}
    \begin{align}
        \sum_{n=0}^{\infty}\frac{1}{ n+1 }x^n&=\frac{1}{ x }\sum_{n=0}\frac{ x^{n+1} }{ n+1 }\\
        &=\frac{1}{ x }\sum_{n=1}^{\infty}\frac{ x^n }{ n }=-\frac{ \ln(1-x) }{ x }.
    \end{align}
\end{subequations}
Notez le changement de point de départ de la somme au passage.

Autre exemple :
\begin{subequations}
    \begin{align}
        \sum_{n=0}^{\infty}\frac{ x^n }{ n+3 }&=\frac{1}{ x^3 }\left( \sum_{n=1}^{\infty}\frac{ x^n }{ n }-x-\frac{ x^2 }{ 2 } \right)\\
        &=-\frac{ \ln(x-1) }{ x^3 }-\frac{1}{ x^2 }-\frac{1}{ 2x }.
    \end{align}
\end{subequations}

Si le polynôme possède des racines non entières, les choses se compliquent. 

\begin{example}
Calculons
\begin{equation}
    \sum_{n=0}^{\infty}\frac{ x^n }{ 2n+1 }.
\end{equation}
Si \( x\geq\), en posant \( t=\sqrt{x}\) nous trouvons
\begin{equation}
    \sum_{n=0}^{\infty}\frac{ x^n }{ 2n+1 }=\frac{1}{ t }\sum_{n=0}^{\infty}\frac{ t^{2n+1} }{ 2n+1 }.
\end{equation}
Étudions
\begin{equation}
    H(t)=\sum_{n=0}^{\infty}\frac{ t^{2n+1} }{ 2n+1 }.
\end{equation}
Nous avons 
\begin{equation}     \label{EqBuPjcM}
    H'(t)=\sum_{n=0}^{\infty}t^{2n}=\sum_{n=0}(t^2)^n=\frac{1}{ 1-t^2 }.
\end{equation}
Une primitive de cette fonction est
\begin{equation}
    \frac{ 1 }{2}\ln\left| \frac{ t+1 }{ t-1 } \right|.
\end{equation}
En \( t=0\), cette fonction vaut \( 0\) qui est la bonne valeur. Donc nous avons bien
\begin{equation}
    H(t)=\frac{ 1 }{2}\ln\left| \frac{ t+1 }{ t-1 } \right|.
\end{equation}

Notons que ce que l'équation \eqref{EqBuPjcM} nous dit est que \( H(t)\) est une primitive de \( 1/(1-t^2)\). Il faut choisir la bonne primitive en fixant une valeur.





Nous avons donc
\begin{equation}
    \sum_{n=0}^{\infty}\frac{ x^n }{ 2n+1 }=\frac{ 1 }{2\sqrt{x}}\ln\left| \frac{ \sqrt{x}+1 }{ \sqrt{x}-1 } \right| 
\end{equation}
pour \( x>0\). Nous devons encore trouver ce que cela vaut pour \( x<0\).

    Nous posons successivement \( X=-x\) puis \( g(X)=f(-X)\). Ce que nous devons calculer est
    \begin{equation}
        g(t)=\frac{1}{ t }\sum_{n=0}^{\infty}\frac{ (-1)^nt^{2n+1} }{ 2n+1 }.
    \end{equation}
    Si nous posons
    \begin{equation}
        h(t)=\sum \frac{ (-1)^nt^{2n+1} }{ 2n+1 },
    \end{equation}
    alors
    \begin{equation}
        h'(t)=\sum (-1)^nt^{2n}=\sum (-t^2)^n=\frac{1}{ 1+t^2 },
    \end{equation}
    par conséquent \( h(t)=\arctan(t)\) (cela avait déjà été déduit à l'envers dans l'exemple \ref{ExwobBAW}).

    Au final
    \begin{equation}        \label{EqIHlDjG}
        f(x)=\sum_{n=0}^{\infty}\frac{ x^n }{ 2n+1 }=\begin{cases}
            \frac{ 1 }{2\sqrt{x}}\ln\left| \frac{ \sqrt{x}+1 }{ \sqrt{x}-1 } \right|     &   \text{si \( x>0\)}\\
            \frac{ \arctan(\sqrt{-x}) }{ \sqrt{-x} }    &    \text{si \( x<0\)}\\
            1   &\text{si \( x=0\)}.
        \end{cases}
    \end{equation}
    Notons qu'elle est continue en zéro à gauche et à droite.

\end{example}

\begin{example}
Nous considérons l'exemple suivant :
\begin{equation}
    f(x)=\sum_{n=0}^{\infty}\frac{ x^n }{ 3n+2 }.
\end{equation}
Nous posons \( t=\sqrt[3]{x}\), et nous substituons :
\begin{equation}
    \frac{ x^n }{ 3n+2 }=\frac{ t^{3n} }{ 3n+2 }=\frac{1}{ t^2 }\frac{ t^{3n+2} }{ 3n+2 }.
\end{equation}
Nous devons étudier la fonction
\begin{equation}
    g(t)=\sum_{n=0}^{\infty}\frac{ t^{3n+2} }{ 3n+2 }
\end{equation}
Nous avons
\begin{equation}
    g'(t)=\sum_{n=0}t^{3n+1}=t\sum_{n=0}t^{3n}=\frac{ t }{ 1-t^3 }.
\end{equation}
Notons que \( g(0)=0\). 
\end{example}

\begin{example}
    Calculer le nombre
    \begin{equation}        \label{EqgUyKYe}
        \sum_{n=0}^{\infty}\frac{ (-1)^n }{ 2n+1 }.
    \end{equation}
    Nous aurions envie de dire que cela est \( f(-1)\) pour la fonction \( f\) donnée en \eqref{EqIHlDjG}. Le problème est que le rayon de convergence de \( f\) étant \( 1\), rien n'est garantit quand au fait que la fonction y soit continue en \( x=-1\). En particulier nous devons justifier le fait que
    \begin{equation}
        \lim_{x\to -1} \sum_n\frac{ x^n }{ 2n+1 }=\lim_{x\to -1} \frac{1}{ \sqrt{-x} }\arctan(\sqrt{-x}).
    \end{equation}
    Ce qui nous sauve est le critère d'Abel radial (théorème \ref{ThoLUXVjs}). En effet la série
    \begin{equation}        \label{EqAFrXRB}
        \sum\frac{ r^n }{ 2n+1 }
    \end{equation}
    étant convergente avec \( r=-1\), la série correspondante est continue sur \( \mathopen[ -1 , 0 \mathclose]\). Nous pouvons donc calculer la série \eqref{EqgUyKYe} en posant \( x=-1\) dans \eqref{EqIHlDjG} :
    \begin{equation}       
        \sum_{n=0}^{\infty}\frac{ (-1)^n }{ 2n+1 }=\frac{ \pi }{ 4 }.
    \end{equation}

    Note : la série \eqref{EqAFrXRB} ne converge pas avec \( r=1\). La fonction \( f\) n'est pas continue en \( x=1\).
\end{example}

\begin{example}     \label{ExGxzLlP}
    Nous avons
    \begin{equation}
        \sum_{n=1}^{\infty}nx^{n-1}=\frac{1}{ (1-x)^2 }.
    \end{equation}
    En effet si nous désignons par \( f\) la somme à gauche, nous trouvons que \( f=g'\) avec
    \begin{equation}
        g(x)=\sum_{n=1}^{\infty}x^n.
    \end{equation}
    Nous savons par ailleurs que \( g(x)=1/(1-x)\). Par conséquent
    \begin{equation}
        f(x)=\left( \frac{1}{ 1-x } \right)'=\frac{1}{ (1-x)^2 }.
    \end{equation}
\end{example}

%///////////////////////////////////////////////////////////////////////////////////////////////////////////////////////////
\subsubsection{Sage, primitives et logarithme complexe}
%///////////////////////////////////////////////////////////////////////////////////////////////////////////////////////////
\label{PgpXBuBh}

Attention : Sage pourrait nous induire en erreur si nous n'y prenions pas garde. En effet ce que vous ne savez pas mais que Sage sait, c'est que
\begin{equation}
    \ln(-1)=i\pi.
\end{equation}
Par conséquent Sage se permet de donner des primitives sans valeurs absolues dans le logarithme :
\begin{verbatim}
sage: f(x)=1/x
sage: f.integrate(x)
x |--> log(x)
\end{verbatim}
La primitive à laquelle on s'attend d'habitude est \( \ln(| x |)\). Ici la réponse est correcte parce que si \( x\) est négatif nous avons
\begin{equation}
    \ln(x)=\ln\big( (-1)| x | \big)=\ln(-1)+\ln(| x |).
\end{equation}
Cette fonction est donc décalée de la primitive usuelle seulement de la constante \( \ln(-1)\).

Un exemple plus élaboré :
\begin{verbatim}
sage: h(x)=1/(1-x**2)
sage: H=h.integrate(x)
sage: H
x |--> -1/2*log(x - 1) + 1/2*log(x + 1)
sage: H(0)
-1/2*I*pi
\end{verbatim}
    



\begin{example}
Encore une fois il faut faire attention en demandant la primitive à Sage :
\begin{verbatim}
----------------------------------------------------------------------
| Sage Version 4.7.1, Release Date: 2011-08-11                       |
| Type notebook() for the GUI, and license() for information.        |
----------------------------------------------------------------------
sage: f(x)=x/(1-x**3)
sage: F=f.integrate(x)
sage: F(0)
-1/3*I*pi - 1/3*sqrt(3)*arctan(1/3*sqrt(3))
\end{verbatim}
Cette fois la primitive proposée diffère de celle qu'on cherche de la constante complexe
\begin{equation}
    -\frac{ \pi }{ 3 }i.
\end{equation}
Mais il y a pire si nous voulons tracer. Nous voudrions définir la fonction \( F_2(x)=F(x)-F(0)\). Mathématiquement c'est bien de cette fonction que nous parlons, mais :
\begin{verbatim}
sage: F2(x)=F(x)-F(0)
sage: F2(x)
1/3*I*pi - 1/3*sqrt(3)*arctan(1/3*(2*x + 1)*sqrt(3)) + 
    +1/3*sqrt(3)*arctan(1/3*sqrt(3)) - 1/3*log(x - 1) + 1/6*log(x^2 + x + 1)
sage: F2.plot(x,-0.1,0.1)
verbose 0 (4101: plot.py, generate_plot_points) WARNING: When plotting, failed to evaluate function at 200 points.
verbose 0 (4101: plot.py, generate_plot_points) Last error message: 'unable to simplify to float approximation'
\end{verbatim}
Il refuse de tracer. Pourquoi ? La partie complexe de l'expression de \( F_2\) est mathématiquement nulle, mais elle est en deux parties :
\begin{equation}
    \frac{ \pi }{ 3 }+\text{la partie imaginaire de} -\frac{1}{ 3 }\ln(x-1).
\end{equation}
Lorsque Sage tente de tracer, il donne à \( x\) un certain nombre de valeurs et calcule une \emph{valeur approchée} de \( \ln(x-1)\). Cette dernière ne se simplifie pas avec le nombre \emph{exact} \( \pi/3\). Sage reste donc avec une partie imaginaire qu'il ne peut pas tracer.

Notez la nuance :
\begin{verbatim}
sage: ln(-0.1)
-2.30258509299405 + 3.14159265358979*I
sage: ln(-1/10)
I*pi + log(1/10)
\end{verbatim}
Du coup nous avons aussi
\begin{verbatim}
sage: F2(-0.1)
1/3*I*pi - 1/3*sqrt(3)*arctan(0.266666666666667*sqrt(3)) 
    + 1/3*sqrt(3)*arctan(1/3*sqrt(3)) - 0.0474885065133152 - 1.04719755119660*I
\end{verbatim}

\end{example}

%///////////////////////////////////////////////////////////////////////////////////////////////////////////////////////////
\subsubsection{Nombres de Bell}
%///////////////////////////////////////////////////////////////////////////////////////////////////////////////////////////

Ici nous montrerions bien le théorème \ref{ThoYFAzwSg} sur les nombres de Bell parce que c'est essentiellement un résultat sur les séries entières et leurs manipulations. Hélas, il demande un tout petit peu d'équation différentielle (presque rien). Donc il est postposé jusqu'en page \pageref{ThoYFAzwSg}.

%--------------------------------------------------------------------------------------------------------------------------- 
\subsection{Séries entières de matrices}
%---------------------------------------------------------------------------------------------------------------------------
\label{subsecEVnZXgf}

Nous nous proposons d'étudier des séries de la forme
\begin{equation}
    \sum_{k=0}^{\infty}a_kA^k
\end{equation}
où \( A\) est une matrice. L'essentiel de la théorie va rester. Nous considérons une norme matricielle (définition \ref{DefJWRWQue}), c'est à dire \( \| AB \|\leq \| A \|\| B \|\).

La notion de rayon de convergence de cette série reste la même : c'est la définition \ref{DefZWKOZOl} qui ne dépend que des coefficients \( a_k\) et pas du tout de ce qu'on met à côté dans la somme. Évidemment il faudra montrer que dans le cas des matrices, le nom «rayon de convergence» n'est pas usurpé.

\begin{proposition}
    Soit \( (a_n)\) une suite dans \( \eC\) de rayon de convergence \( R\) et \( A\in \eM(n,\eR)\) une matrice vérifiant \( \| A \|<R\). Alors la série
    \begin{equation}
        \sum_{k=0}^{\infty}a_kA^k
    \end{equation}
    converge absolument, c'est à dire que \( \sum_k\| a_kA^k \|<\infty\).
\end{proposition}

\begin{proof}
    Nous avons les majorations
    \begin{equation}
        \| a_n A^n\|\leq | a_n |\| A^n \|\leq | a_n |\| A \|^n.
    \end{equation}
    Par hypothèse \( \| A \|<R\) et \( R\) est un supremum, donc il existe \( r\) tel que \( \| A \|<r<R\) avec \( (a_nr^n)\) borné. Nommons \( M\) un majorant de la suite \( (a_nr^n)\). Alors nous avons
    \begin{equation}
        \| A_nA^n \|\leq | a_n |r^n\frac{ \| A \|^n }{ r^n }\leq M\left( \frac{ \| A \| }{ r } \right)^n.
    \end{equation}
    La série du membre de droite converge parce que c'est une série géométrique de raison plus petite que \( 1\); voir l'exemple \ref{exemplesseries}.
\end{proof}

\begin{proposition} \label{PropAMBXKgV}
    Soit \( (a_n)\) une suite dans \( \eC\) de rayon de convergence \( R\) et la fonction
    \begin{equation}
        \begin{aligned}
            f\colon \eM(n,\eR)&\to \eM(n,\eR) \\
            A&\mapsto \sum_{k=0}^{\infty}a_kA^k 
        \end{aligned}
    \end{equation}
    Alors
    \begin{enumerate}
        \item
            La différentielle de \( f\) sur \( B(0,R)\) est
            \begin{equation}    \label{EqRDVodDa}
                df_A(U)=\sum_{k=0}^{\infty}a_k\sum_{l=0}^{k-1}A^lUA^{k-1-l},
            \end{equation}
            c'est à dire que l'on peut différentier terme à terme. (Ici c'est \( A\) qui est dans \( B(0,R)\))
        \item
            La convergence de la somme \ref{EqRDVodDa} est absolue.
        \item
            La convergence de la somme \ref{EqRDVodDa} est normale sur tout compact.
        \item
            La fonction \( f\) est de classe \( C^1\) sur \( B(0,R)\), c'est à dire que la fonction \( A\mapsto df_A\) est continue.
    \end{enumerate}
\end{proposition}
Notons que \( df_A\) n'est pas tout à fait une série entière. Cependant, en ce qui concerne les normes, c'est tout comme si ça l'était.

\begin{proof}
    Nous posons \( u_k(A)=a_kA^k\), qui est une fonction de classe \(  C^{\infty}\) et dont la différentielle est donnée par
    \begin{equation}
        (du_k)_A(U)=\Dsdd{ u_k(A+tU) }{t}{0}=a_k\Dsdd{ (A+tU)^k }{t}{0};
    \end{equation}
    en distribuant le produit nous trouvons tout un tas de termes dont seuls ceux contenant exactement une fois \( tU\) ne vont pas s'annuler. Étant donné que \( U\) et \( A\) ne commutent pas nous avons l'expression un peu moche
    \begin{equation}
        (du_k)_A(U)=\sum_{l=0}^{k-1}a_kA^lUA^{k-1-l}.
    \end{equation}
    En ce qui concerne la norme, nous regardons celle de \( (du_k)_A\) pour un \( A\) fixé; c'est à dire que nous en regardons la norme opérateur :
    \begin{equation}
        \| (du_k)_A \|=\sup_{\| U \|=1}\| \sum_{l=0}^{k-1}a_kA^lUA^{k-1-l} \|\leq \sum_{l=0}^{k-1}| a_k |\| A \|^{l}\| A \|^{k-1-l}\leq k| a_k |\| A \|^{k-1}.
    \end{equation}
    Pour donner la convergence nous considérons un nombre \( r\) tel que \( \| A \|<r<R\), de telle sorte que la suite \( (a_nr^n)\) soit bornée par un nombre \( M\) et que nous puissions écrire
    \begin{equation}    \label{EqTGEwhnL}
        \| (du_k)_A \|\leq k| a_k |\| A \|^{k-1}=\frac{ k| a_k |\| A \|^k }{ \| A \| }=\frac{ k| a_k | }{ \| A \| }r^k\left( \frac{ \| A \| }{ r } \right)^k\leq \frac{ M }{ \| A \| }k\left( \frac{ \| A \| }{ r } \right)^k,
    \end{equation}
    dont la série converge. Nous avons donc convergence absolue de la série
    \begin{equation}
        \sum_{k=0}^{\infty}(du_k)_A.
    \end{equation}
    Passons à la convergence normale sur tout compact. Nous nous fixons \( r<R\) et nous nous intéressons à la norme de \( du_k\) sur \( \overline{ B(0,r) }\), c'est à dire
    \begin{equation}
        \| du_k \|_{\infty}=\sum_{x\in\overline{ B(0,r) }}\| (du_k)_A \|.
    \end{equation}
    Vu que \( \overline{ B(0,r) }\) est compact, ce supremum est un maximum et nous pouvons noter \( A_k\) la matrice qui le réalise. Nous réalisons alors les mêmes manipulations que pour \eqref{EqTGEwhnL} :
    \begin{equation}
        \| du_k \|_{\infty}=\| (du_k)_{A_k} \|\leq k| a_k |\| A_k \|^{k-1}\leq  k| a_k |r^{k-1}=\frac{1}{ r }k| a_k |r^k.
    \end{equation}
    Nous prenons maintenant \( r<r_0<R\) et \( M\), un majorant de \( (a_nr_0^n)\), de telle sorte qu'en multipliant et divisant par \( r_0^k\),
    \begin{equation}
        \| du_k \|_{\infty}\leq \frac{ k| a_k |r_0^k }{ r }\frac{ r^k }{ r_0^k }\leq \frac{ kM }{ r }\left( \frac{ r }{ r_0 } \right)^k,
    \end{equation}
    dont la série converge. Nous avons donc convergence normale sur tout compact. Par voie de \sout{fait} conséquences nous avons continuité de la série
    \begin{equation}
        \sum_{k=0}^{\infty}(du_k)_A
    \end{equation}
    et convergence vers \( df_A\) par le théorème \ref{ThoLDpRmXQ}.
\end{proof}

\begin{proposition} \label{PropQIIURAh}
    Si le rayon de convergence de la série \( u(A)=\sum_{k=0}^{\infty}a_kA^k\) est \( R\), alors 
    \begin{enumerate}
        \item
            elle converge normalement sur tout compact de \( B(0,R)\);
        \item
            la fonction \( u\) y est de classe \(  C^{\infty}\).
    \end{enumerate}
\end{proposition}

\begin{proof}
    Nous posons 
    \begin{equation}
        \begin{aligned}
            u_k\colon \eM(n,\eR)&\to \eM(n,\eR) \\
            A&\mapsto a_kA^k 
        \end{aligned}
    \end{equation}
    qui est évidemment une fonction de classe \(  C^{\infty}\). Nous étudions la \( j\Ieme\) différentielle en \( m\), pour \( k>j\) (dans une série, nous ne nous intéressons pas aux premiers termes). La \( j\Ieme\) différentielle appliquée à \( v_1\) appliquée à \( v_2\), etc s'exprime de la façon suivante :
    \begin{equation}
        (d^ju_k)_m(v_1,\ldots, v_j)=\frac{ d  }{ d t_1 }\ldots\frac{ d  }{ d t_j }\Big( u_k(m+t_1v_1+\ldots +t_jv_j)    \Big)_{t_i=0}.
    \end{equation}
    Dans le produit \( (m+t_1v_1+\ldots +t_jv_j)^k\), seuls les termes contenant exactement une fois chacun des \( t_i\) ne s'annulera pas après avoir fait la dérivée et évalué en \( t_i=0\). Combien de termes cela fait ? Parmi les \( k\) facteurs, il faut en placer \( j\) qui ne sont pas \( m\) (cela fait \( \binom{ k }{ j }\) possibilités), et puis il faut ordonner ces \( j\) termes, cela fait encore \( j!\) possibilités. Au final,
    \begin{equation}
        \| (d^ju_k)_m \|\leq | a_k | \binom{ k }{ j }j!\| m \|^{k-j}=| a_k |P(k)\| m \|^{k-j}
    \end{equation}
    où \( P(k)=\frac{ k! }{ (k-j)! }\) est un polynôme de degré \( j\).

    Afin d'étudier la convergence normale sur tout compact de la série des \( d^ju_k\), nous considérons \( r<r_0<R\) et nous allons prouver la convergence normale sur \( \overline{ B(0,r) }\). Vu que c'est un compact, il existe une matrice \( m_k\in\overline{ B(0,r) }\) telle que
    \begin{subequations}
        \begin{align}
            \| d^ju_k \|_{\infty}&=\| (d^ju_k)_{m_k} \|\\
            &\leq | a_k |P(k)\| m_k \|^{k-j}\\
            &\leq | a_k |P(k)r^{k-j}\\
            &=\frac{ | a_k |P(k) }{ r^j }r^k\\
            &=\frac{ | a_k |r_0^kP(k) }{ r^j }\left( \frac{ r }{ r_0 } \right)^k\\
            &\leq \frac{ M }{ r^j }P(k)\left( \frac{ r }{ r_0 } \right)^k
        \end{align}
    \end{subequations}
    où \( M\) est un majorant de \( a_nr^n\). Vu que \( r_0/r<1\), la somme sur \( k\) converge et nous avons convergence normale sur tout compact de
    \begin{equation}
        d^j\sum_{k=0}^{\infty}a_kA^k=\sum_{k=0}^{\infty}d^j(a_kA^k)
    \end{equation}
    avec un peu d'abus de notation.
\end{proof}

%--------------------------------------------------------------------------------------------------------------------------- 
\subsection{Exponentielle et logarithme de matrice}
%---------------------------------------------------------------------------------------------------------------------------
\label{subsecXNcaQfZ}

\begin{proposition} \label{PropXFfOiOb}
    L'application
    \begin{equation}
        \begin{aligned}
            \exp\colon \eM(n,\eR)&\to \eM(n,\eR) \\
            A&\mapsto \sum_{k=0}^{\infty}\frac{ A^k }{ k! } 
        \end{aligned}
    \end{equation}
    est une application de classe \(  C^{\infty}\). Sa différentielle en zéro est l'identité : \( d\exp_0=\id\).
\end{proposition}
\index{exponentielle!de matrice}

\begin{proof}
    En ce qui concerne la continuité, nous savons que le rayon de convergence de la suite \( \frac{1}{ k! }\) est infini; la proposition \ref{PropQIIURAh} conclu.

    Pour la différentielle, c'est la proposition \ref{PropAMBXKgV} qui nous permet d'écrire
    \begin{equation}
        d\exp_0(U)=\Dsdd{ \exp(tU) }{t}{0}=\Dsdd{ \sum_{k=0}^{\infty}\frac{ t^kU^k }{ k! } }{t}{0}=\left. \sum_{k=0}^{\infty}\frac{ kt^{k-1}U^k }{ k! }\right|_{t=0}=U
    \end{equation}
    parce que seul le terme \( k=1\) n'est pas nul.
\end{proof}

Nous avons vu par la proposition \ref{PropKKdmnkD} que toute matrice complexe inversible a un logarithme. Nous allons maintenant parler de logarithme de matrices réelles avec une condition sur la norme. La formule ci-dessous montre explicitement que le logarithme est réel.
\begin{equation}
    \begin{aligned}
        \ln\colon \{ A\in \eM(n,\eR)\tq \| A-\mtu \| <1 \}&\to \eM(n,\eR) \\
        A&\mapsto \sum_{k=0}^{\infty}(-1)^k\frac{ (A-\mtu)^{k+1} }{ k+1 }. 
    \end{aligned}
\end{equation}

\begin{lemma}   \label{LemQZIQxaB}
    Si \( \| m \|<1\) dans \( \eM(n,\eR)\), alors nous posons
    \begin{equation}    \label{EqIKgMabb}
        \ln(\mtu+m)=\sum_{k=0}^{\infty}(-1)^k\frac{ m^{k+1} }{ k+1 }.
    \end{equation}
    Cette fonction a les propriétés suivantes.
    \begin{enumerate}
        \item
            Elle est de classe \(  C^{\infty}\).        
        \item
            Elle est un bon logarithme au sens où
            \begin{equation}
                e^{\ln(\mtu+m)}=\mtu+m.
            \end{equation}
        \item
            Elle vérifie l'approximation
            \begin{equation}
                \ln(1+m)=m+\sigma(m)
            \end{equation}
            où \( \sigma\) a la propriété que
            \begin{equation}
                \lim_{k\to \infty} k\sigma\left( \frac{ m }{ k } \right)=0.
            \end{equation}
    \end{enumerate}
\end{lemma}
\index{logarithme!de matrice}
%TODO : le reste de la preuve, en particulier le point avec l'exponentielle.

\begin{proof}
    
    Le rayon de convergence de la suite \( a_k=\frac{ (-1)^k }{ k+1 }\) est \( 1\). Donc l'application donnée est \(  C^{\infty}\) sur \( B(0,1)\) par le théorème \ref{PropQIIURAh}.

    D'après la formule \eqref{EqIKgMabb} nous avons
    \begin{equation}
        \sigma(m)=\sum_{l=1}^{\infty}(-1)^l\frac{ m^{l+1} }{ l+1 }.
    \end{equation}
    Nous avons alors
    \begin{equation}
        k\sigma(\frac{ m }{ k })=\sum_{l=1}^{\infty}(-1)^l\frac{ m^{l+1} }{ k^l(l+1) },
    \end{equation}
    et donc
    \begin{equation}
        \| k\sigma(\frac{ m }{ k }) \|\leq \sum_{l=1}^{\infty}\frac{ \| m \|^{l+1} }{ k^l(l+1) }\leq\frac{1}{ k }\sum_{l=1}^{\infty}\frac{ \| m \|^{l+1} }{ l+1 }\stackrel{k\to\infty}{\to} 0
    \end{equation}
    Cela prouve la dernière assertion.   
\end{proof}

%+++++++++++++++++++++++++++++++++++++++++++++++++++++++++++++++++++++++++++++++++++++++++++++++++++++++++++++++++++++++++++ 
\section{Lemme de Borel}
%+++++++++++++++++++++++++++++++++++++++++++++++++++++++++++++++++++++++++++++++++++++++++++++++++++++++++++++++++++++++++++

%--------------------------------------------------------------------------------------------------------------------------- 
\subsection{Fonctions plateaux}
%---------------------------------------------------------------------------------------------------------------------------

Soient \( a<b<c<d\) dans \( \eR\). Nous voulons trouver une fonction \( f\in C^{\infty}(\eR)\) à valeurs positives telle que
\begin{enumerate}
    \item
        \( f(x)=1\) si \( x\in\mathopen[ b , c \mathclose]\)
    \item
        \( \supp(f)\subset\mathopen[ a , d \mathclose]\).
\end{enumerate}

Nous commençons par l'exemple classique de fonction \(  C^{\infty}\) qui n'est pas nulle partout :
\begin{equation}
    \varphi(x)=\begin{cases}
        e^{-1/x}    &   \text{si \( x>0\)}\\
        0    &    \text{sinon}.
    \end{cases}
\end{equation}
Il est facile de vérifier que \( \varphi\) est de classe \(  C^{\infty}\) parce que
\begin{equation}
    \lim_{x\to 0^+} \frac{  e^{-1/x} }{ P(x) }=0 
\end{equation}
pour tout polynôme \( P\). De plus c'est une fonction qui vaut zéro sur \( \mathopen] -\infty , 0 \mathclose]\). Ensuite nous construisons la fonction
\begin{equation}
    \psi_m(x)=1-\frac{ \int_0^x \varphi(t)dt }{ \int_0^m\varphi(t)dt}=\begin{cases}
        1    &   \text{si \( x<0\)}\\
        0    &    \text{si \( x>m\)}\\
        \text{positive} &\text{si \( x\in\mathopen[ 0 , m \mathclose]\)}.
    \end{cases}
\end{equation}
Cette fonction est encore de classe $ C^{\infty}$. À partir de là nous considérons les fonctions
\begin{subequations}
    \begin{align}
        f_1(x)&=\psi_{d-c}(x-c)=\begin{cases}
            1    &   \text{si \( x<c\)}\\
            0    &    \text{si \( x>d\)}\\
            \text{positive} &\text{si \( x\in\mathopen[ c , d \mathclose]\)}.
        \end{cases}\\
        f_2(x)&=\psi(b-a)(b-x)=\begin{cases}
            0    &   \text{si \( x<a\)}\\
            1    &    \text{si \( x>b\)}\\
            \text{positive} &\text{si \( x\in \mathopen[ a , b \mathclose]\)}.
        \end{cases},
    \end{align}
\end{subequations}
et finalement la fonction suivante répond à la question des fonctions plateaux sur \( \eR\) :
\begin{equation}
    f(x)=f_1(x)f_2(x).
\end{equation}
\index{fonction!définie par une intégrale}

Une variation sur le même thème est l'existence de fonctions infiniment dérivables à support compact, c'est à dire des fonctions dans \(  C^{\infty}_c(\eR^d)=\swD(\eR^d)\). Ces espaces ne sont pas vides, par exemple nous avons la fonction
\begin{equation}    \label{EqOBYNEMu}
    \xi(x)=\begin{cases}
        e^{-1/(1-| x |^2)}    &   \text{si \( x\in B(0,1)\)}\\
        0    &    \text{sinon}.
    \end{cases}
\end{equation}

%--------------------------------------------------------------------------------------------------------------------------- 
\subsection{Le lemme de Borel}
%---------------------------------------------------------------------------------------------------------------------------

\begin{lemma}[Lemme de Borel\cite{KXjFWKA}] \label{LemRENlIEL}
    Soit \( (a_n)\) une suite dans \( \eR\). Il existe une fonction \( u\in C^{\infty}(\eR)\) telle que \( u^{(k)}(0)=a_k\) pour tout \( k\geq 0\).
\end{lemma}
\index{lemme!Borel}
\index{prolongement!de fonctions!lemme de Borel}
\index{dérivabilité!lemme de Borel}

\begin{proof}
    Soit \( \varphi\in C^{\infty}_c(\eR)\) une fonction telle que \( \varphi(x)=1\) si \( | x |\leq \frac{ 1 }{2}\) et telle que \( \supp(\varphi)\subset\mathopen] -1 , 1 \mathclose[\).

    Nous commençons par considérer une suite de réels strictement positifs \( (\lambda_k)\) dont nous fixerons une valeur précise plus tard, et nous posons
    \begin{equation}
        f_k(x)=\varphi(\lambda_k x)\frac{ a_k }{ k! }x^k.
    \end{equation}
    Nous allons étudier la convergence et les propriétés de \( u(x)=\sum_{k=0}^{\infty}f_k(x)\). 

    Calculons (formellement) la \( m\Ieme\) dérivée de \( f_k\) :
    \begin{subequations}
        \begin{align}
            f_k^{(m)}(x)&=\frac{ a_k }{ k! }\sum_{l=0}^{m}\binom{ m }{ l }\lambda_k^{m-l}\varphi^{(m-l)}(\lambda_kx)(x^k)^{(l)}\\
            &=a_k\sum_{l=0}^{m}\binom{ l }{ m }\lambda_k^{m-l}\varphi^{(m-l)}(\lambda_kx)\frac{ x^{k-l} }{ (k-l)! }.
        \end{align}
    \end{subequations}
    Notons que nous travaillons à \( m\) fixé et que nous ne nous intéressons qu'aux termes avec \( k\) assez grand; nous pouvons donc supposer \( k\geq m\). De toutes façons pour \( \sum_{k=0}^mf_k\), on a la classe \(  C^{\infty}\), et la permutation de la somme avec tout ce qu'on veut. Vu que \( \varphi\) est continue à support compact nous pouvons poser
    \begin{equation}
        M_m=\max_{0\leq j\leq m}\| \varphi^{j} \|_{\infty}=\max_{0\leq j\leq m}\max_{x\in \eR}| \varphi^{(j)}(x) |.
    \end{equation}
    Nous continuons en nous fixant un \( x\in \eR\) et un \( k\geq m\).

    Si \( | x |>\frac{1}{ \lambda_k }\), alors \( \varphi^{(m)}(\lambda_kx)=0\) parce que \( \lambda_kx\) est strictement hors du support de \( \varphi\) qui est \( \mathopen] -1 , 1 \mathclose[\). Donc pour \( | x |>\frac{1}{ \lambda_k }\).

    Si par contre \( | x |\leq\frac{1}{ \lambda_k }\), nous avons les majorations
    \begin{subequations}
        \begin{align}
        | f^{(m)}_k(x) |&\leq  |a_k|\sum_{l=0}^{m}\binom{ l }{ m }|\lambda_k|^{m-l}\underbrace{\varphi^{(m-l)}(\lambda_kx)}_{\leq M_m}\frac{ 1 }{ (k-l)! }\underbrace{| x |^{k-l}}_{\leq (1/\lambda_k)^{k-l}}\\
        &\leq | a_k |M_m| \lambda_k |^{m-k}\frac{1}{ (k-m)! }\sum_{l=0}^{m}\binom{ m }{ l }\\
        &\leq \frac{ | a_k |M_m | \lambda_k |^{m-k}2^m }{ (k-m)! }\\
        &= \frac{ | a_k |M_m 2^m }{ (k-m)!  | \lambda_k |^{k-m} }       \label{EqQSPUaun}
        \end{align}
    \end{subequations}
    où pour faire disparaître la somme de coefficients binomiaux, nous avons remarqué que \( \sum_{l=0}^m\binom{ m }{ l }\) est le nombre total de termes dans le développement de \( (a+b)^m\), c'est à dire \( 2^m\). Nous voulons, pour \( m\) fixé, étudier la convergence de la somme de cela. Notons que le \( 2^m\) n'a en particulier strictement aucune importance parce qu'on travaille à \( m\) fixé.

    Nous fixons maintenant la valeur des \( \lambda_k\) :
    \begin{equation}
        \lambda_k=\max\{ | a_k |,1 \}.
    \end{equation}
    Avec cela, en nous souvenant que nous n'étudions que les termes \( k>m\), le dénominateur de \eqref{EqQSPUaun} est réellement croissant en \( k\), donc nous avons la majoration
    \begin{equation}
        | f^{(m)}_k(x) |\leq \frac{ M_m2^m }{ (k-m)! }.
    \end{equation}
    Au final nous avons
    \begin{equation}
        \| f_k^{(m)} \|_{\infty}\leq \frac{ 2^mM_m }{ (k-m)! }.
    \end{equation}
    Et la somme de cela converge sans difficultés. Donc la série
    \begin{equation}
        u(x)=\sum_{k=0}^{\infty}f_k^{(m)}(x)
    \end{equation}
    converge normalement et donc uniformément sur \( \eR\). Nous pouvons alors permuter la somme et la dérivation par le théorème \ref{ThoCSGaPY}. Donc
    \begin{equation}
        u^{(m)}=\sum_{k=0}^{\infty}f_k^{(m)}
    \end{equation}
    est continue. En particulier, pour évaluer en zéro, on peut faire
    \begin{equation}
        u^{(m)}(0)=\sum_{k=0}^{\infty}f_k^{(m)}(0).
    \end{equation}
    Nous avons
    \begin{equation}
        f_k(x)=\varphi(\lambda_kx)\frac{ a_k }{ k! }x^k.
    \end{equation}
    Pour calculer la dérivée en zéro, il suffit de la calculer sur un voisinage sur lequel \( \varphi(\lambda_kx)\) est la constante \( 1\); un tel voisinage existe pour tout \( k\). À ce moment le calcul est classique :
    \begin{equation}
        f_k^{(m)}(x)=\begin{cases}
            a_k    &   \text{si \( k=m\)}\\
            0    &    \text{sinon}.
        \end{cases}
    \end{equation}
    Finalement nous avons bien
    \begin{equation}
        u^{(m)}(0)=\sum_{k=0}^{\infty}f_k^{(m)}(0)=a_k.
    \end{equation}
    
\end{proof}

\begin{remark}
    Pour prouver le lemme de Borel, la première chose qui passe par la tête est la fonction toute simple
    \begin{equation}
        u(x)=\sum_{k=0}^{\infty}\frac{ a_k }{ k! }x^k.
    \end{equation}
    Évidemment si on calcule les dérivées successives de cette fonction, nous trouvons les bons résultats. Le problème est la convergence.  Rien qu'en prenant \( a_k=k!\), la série ne converge pour aucun \( x\) positifs. L'idée de multiplier chacun de \( f_k\) par une fonction plateau sur un petit intervalle autour de zéro a plusieurs avantages. D'abord on conserve les dérivées correctes parce qu'on ne touche pas à la valeur des \( f_k\) sur un petit voisinage. Ensuite cela ne modifie pas la continuité; et enfin en multipliant par \( \varphi(\lambda_kx)\), ça calme méchamment les divergences parce que \( \lambda_kx\) passe vite au-dessus de \( 1\) (et donc en dehors du support de \( \varphi\)) si \( \lambda_k\) est grand. D'où le fait qu'il soit normal que les \( \lambda_k\) soient de l'ordre des \( a_k\).
\end{remark}

%+++++++++++++++++++++++++++++++++++++++++++++++++++++++++++++++++++++++++++++++++++++++++++++++++++++++++++++++++++++++++++ 
\section{Intégrales convergeant uniformément}
%+++++++++++++++++++++++++++++++++++++++++++++++++++++++++++++++++++++++++++++++++++++++++++++++++++++++++++++++++++++++++++

%--------------------------------------------------------------------------------------------------------------------------- 
\subsection{Définition et propriété}
%---------------------------------------------------------------------------------------------------------------------------

Soit \( (\Omega,\mu)\) un espace mesuré. Nous disons que l'intégrale
\begin{equation}
    \int_{\Omega}f(x,\omega)d\mu(\omega)
\end{equation}
\defe{converge uniformément}{convergence!uniforme!intégrale} en \( x\) si pour tout \( \epsilon>0\), il existe un compact \( K_0\) tel que pour tout compact \( K\) tel que \( K_0\subset K\) nous avons
\begin{equation}
    \left| \int_{\Omega\setminus K}f(x,\omega)d\mu(\omega) \right| \leq \epsilon.
\end{equation}
Le point important est que le choix de \( K_0\) ne dépend pas de \( x\).

\begin{lemma}       \label{LemOgQdpJ}
    Soit
    \begin{equation}
        F(x)=\int_{\Omega}f(x,\omega)d\mu(\omega),
    \end{equation}
    une intégrale uniformément convergente. Pour chaque \( k\in \eN\) nous considérons un compact \( K_k\) tel que
    \begin{equation}
        \left| \int_{\Omega\setminus K_k}f(x,\omega)d\mu(\omega) \right| \leq\frac{1}{ k }.
    \end{equation}
    Alors la suite de fonctions \( F_k\) définie par
    \begin{equation}
        F_k(x)=\int_{K_k}f(x,\omega)d\mu(\omega)
    \end{equation}
    converge uniformément vers \( F\).
\end{lemma}

\begin{proof}
    Nous avons
    \begin{subequations}
        \begin{align}
            \big| F_k(x)-F(x) \big|&=\left| \int_{K_k}f(x,\omega)d\mu(\omega)-\int_{\Omega}f(x,\omega)d\mu(\omega) \right| \\
            &=| \int_{\Omega\setminus K_k}f(x,\omega)d\mu(\omega) |\\
            &\leq \frac{1}{ k }.
        \end{align}
    \end{subequations}
\end{proof}

%------------------------------------------------------------------------------------------------------------------------
\subsection{Critères de convergence uniforme}
%---------------------------------------------------------------------------------------------------------------------------

Afin de tester l'uniforme convergence d'une intégrale, nous avons le \defe{critère de Weierstrass}{critère!Weierstrass}:
\begin{theorem}		\label{ThoCritWeiIntUnifCv}
Soit $f(x,t)\colon [\alpha,\beta]\times[a,\infty[ \to \eR$, une fonction dont la restriction à toute demi-droite $x=cst$ est mesurable. Si $| f(x,t) |< \varphi(t)$ et $\int_a^{\infty}\varphi(t)dt$ existe, alors l'intégrale
\begin{equation}
	\int_0^{\infty}f(x,t)dt
\end{equation}
est uniformément convergente.
\end{theorem}

Le théorème suivant est le \defe{critère d'Abel}{critère!Abel pour intégrales} :
\begin{theorem}		\label{ThoAbelIntUnif}
	Supposons que $f(x,t)=\varphi(x,t)\psi(x,t)$ où $\varphi$ et $\psi$ sont bornée et intégrables en $t$ au sens de Riemann sur tout compact $[a,b]$, $b\geq a$. Supposons que :
	\begin{enumerate}
		\item $| \int_a^{T}\varphi(x,t)dt |\leq M$ où $M$ est indépendant de $T$ et de $x$,
		\item $\psi(x,t)\geq 0$,
		\item pour tout $x\in[\alpha,\beta]$, $\psi(x,t)$ est une fonction décroissante de $t$,
		\item les fonctions $x\mapsto \psi(x,t)$ convergent uniformément vers $0$ lorsque $t\to\infty$.
	\end{enumerate}
	Alors l'intégrale
	\begin{equation}
		\int_a^{\infty}f(x,t)dt
	\end{equation}
	est uniformément convergente.
\end{theorem}

%+++++++++++++++++++++++++++++++++++++++++++++++++++++++++++++++++++++++++++++++++++++++++++++++++++++++++++++++++++++++++++
\section{Fonctions définies par une intégrale}
%+++++++++++++++++++++++++++++++++++++++++++++++++++++++++++++++++++++++++++++++++++++++++++++++++++++++++++++++++++++++++++
\label{SecCHwnBDj}
\index{suite!de fonctions intégrables}
\index{fonction!définie par une intégrale}
\index{inversion de limite}

Soit \( (\Omega,\mu)\) un espace mesuré. Nous nous demandons dans quel cas l'intégrale
\begin{equation}
    F(x)=\int_{\Omega}f(x,\omega)d\omega
\end{equation}
définit une fonction \( F\) continue, dérivable ou autre. 

Dans la suite nous allons considérer des fonctions \( f\) à valeurs réelles. Quitte à passer aux composantes, nous pouvons considérer des fonctions à valeurs vectorielles. Par contre le fait que \( x\) soit dans \( \eR\) ou dans \( \eR^n\) n'est pas spécialement une chose facile à traiter.

%--------------------------------------------------------------------------------------------------------------------------- 
\subsection{Continuité sous l'intégrale}
%---------------------------------------------------------------------------------------------------------------------------
\index{continuité!fonction définie par une intégrale}

Nous allons présenter deux théorèmes donnant la continuité de \( F\).
\begin{enumerate}
    \item
        Si \( f\) est majorée par une fonction ne dépendant pas de \( x\), nous avons le théorème \ref{ThoKnuSNd},
    \item
        si l'intégrale est uniformément convergente, nous avons le théorème \ref{ThotexmgE}.
\end{enumerate}

\begin{theorem} \label{ThoKnuSNd}
    Soit \( (\Omega,\mu)\) est un espace mesuré, soit \( x_0\in \eR^m\) et \( f\colon U\times \Omega\to \eR\) où \( U\) est ouvert dans \( \eR^m\). Nous supposons que
    \begin{enumerate}
        \item
            La fonction \( f(x,.)\) est dans \( L^1(\Omega,\mu)\) pour tout \( x \in \eR^m\).
        \item
            La fonction \( f(.,\omega)\) est continue en \( x_0\) pour tout \( \omega\in\Omega\).
        \item       \label{ItemNAuYNG}
            Il existe une fonction \( G\in L^1(\Omega)\) telle que
            \begin{equation}
                | f(x,\omega) |\leq G(\omega)
            \end{equation}
            pour tout \( x\in U\).
    \end{enumerate}
    Alors la fonction 
    \begin{equation}
        \begin{aligned}
            F\colon U&\to \eR \\
            x&\mapsto \int_{\Omega}f(x,\omega)d\mu(\omega) 
        \end{aligned}
    \end{equation}
    est continue en \( x_0\).
\end{theorem}

\begin{proof}
    Soit \( (x_n)\) une suite convergente vers \( x_0\). Nous considérons la suite de fonctions \( f_n\colon \Omega\to \eR\) définies par
    \begin{equation}
        f_n(\phi)=f(x_n,\omega).
    \end{equation}
    sur qui nous pouvons utiliser le théorème de la convergence dominée (théorème \ref{ThoConvDomLebVdhsTf}) pour obtenir
    \begin{subequations}
        \begin{align}
            \lim_{n\to \infty} F(x_n)&=\lim_{n\to \infty} \int_{\Omega}f(x_n,\omega)d\mu(\omega)\\
            &=\int_{\Omega}\lim_{n\to \infty} f(x_n,\omega)d\mu(\omega)\\
            &=\int_{\Omega}f(x,\omega)d\mu(\omega)\\
            &=F(x).
        \end{align}
    \end{subequations}
    Nous avons utilisé la continuité de \( f(.,\omega)\).
\end{proof}


Si nous avons un peu de compatibilité entre la topologie et la mesure, alors nous pouvons utiliser l'uniforme convergence d'une intégrale pour obtenir la continuité d'une fonction définie par une intégrale.

\begin{theorem} \label{ThotexmgE}
    Soit \( (\Omega,\mu)\) un espace topologique mesuré tel que tout compact est de mesure finie. Soit une fonction \( f\colon \eR\times \Omega\to \eR\) telle que
    \begin{enumerate}
        \item
            Pour chaque \( x\in \eR\), la fonction \( f(x,.)\) est \( L^1(\Omega,\mu)\).
        \item
            Pour chaque \( \omega\in \Omega\), la fonction \( f(.,\omega)\) est continue en \( x_0\).
        \item
            L'intégrale
            \begin{equation}
                F(x)=\int_{\Omega}f(x,\omega)d\mu(\omega)
            \end{equation}
            est uniformément convergente.
    \end{enumerate}
    Alors la fonction \( F\) est continue en \( x_0\).
\end{theorem}

\begin{proof}
    Nous reprenons les notations du lemme \ref{LemOgQdpJ}. Les fonctions
    \begin{equation}
        F_k(x)=\int_{K_k}f(x,\omega)d\mu(\omega)
    \end{equation}
    existent parce que les fonctions \( f(x,.)\) sont dans \( L^1(\Omega)\). Montrons que les fonctions \( F_k\) sont continues. Soit une suite \( x_k\to x_0\) nous avons
    \begin{equation}
        \lim_{n\to \infty} F_k(x_n)=\lim_{n\to \infty} \int_{K_k}f(x_n,\omega)d\mu(\omega).
    \end{equation}
    Nous pouvons inverser la limite et l'intégrale en utilisant le théorème de la convergence dominée. Pour cela, la fonction \( f(x_n,\omega)\) étant continue sur le compact \( K_k\), elle y est majorée par une constante. Le fait que les compacts soient de mesure finie (hypothèse) implique que les constantes soient intégrales sur \( K_k\). Le théorème de la convergence dominée implique alors que
    \begin{equation}
        \lim_{n\to \infty} F_k(x_n)=\int_{K_k}\lim_{n\to \infty} f(x_n,\omega)d\mu(\omega)=\int_{K_k}f(x_0,\omega)d\mu(\omega)=F_k(x_0).
    \end{equation}
    Nous avons utilisé le fait que \( f(.,\omega)\) était continue en \( x_0\).

    Le lemme \ref{LemOgQdpJ} nous indique alors que la convergence \( F_k\to F\) est uniforme. Les fonctions \( F_k\) étant continues, la fonction \( F\) est continue.
\end{proof}


Pour finir, citons ce résultat concernant les fonctions réelles.
\begin{theorem}		\label{ThoInDerrtCvUnifFContinue}
    Nous considérons \( F(x)=\int_a^{\infty}f(x,t)dt\). Si \( f\) est continue sur $[\alpha,\beta]\times[a,\alpha[$ et l'intégrale converge uniformément, alors $F(x)$ est continue.
\end{theorem}

%---------------------------------------------------------------------------------------------------------------------------
\subsection{Dérivabilité sous l'intégrale}
%---------------------------------------------------------------------------------------------------------------------------
\index{dérivabilité!fonction définie par une intégrale}

Nous traitons à présent de la dérivabilité de la fonction \( F\) définie comme intégrale de \( f\). Pour cela nous allons suivre \cite{MesIntProbb} avec, il faut le noter, notamment des notations notablement différentes\footnote{Vous en êtes gracieusement notifiés.}.

\begin{theorem}[Dérivation sous le signe intégral\cite{MesIntProbb}]    \label{ThoMWpRKYp}
    Soit \( (\Omega,\mu)\) un espace mesuré et une fonction \( f\colon \eR\times \Omega\to \eR\) dont nous voulons étudier la dérivabilité en \(a\in \eR\). Nous supposons qu'il existe \( \delta>0\), \( A\) mesurable de mesure nulle dans \( \Omega\) tels que
    \begin{enumerate}
        \item
            \( f(x,\cdot)\) soit dans \( L^1(\Omega)\).
        \item
            L'application \( x\mapsto f(x,\omega)\) est dérivable pour tout \( x\in B(a,\delta)\) et pour tout \( \omega\in \complement A\).
        \item
            Il existe une fonction \( G\) intégrable sur \( \Omega\) telle que
            \begin{equation}
                \left| \frac{ \partial f }{ \partial x }(x,\omega) \right| \leq G(\omega)
            \end{equation}
            pour tout \( x\in B(a,\delta)\) et pour tout \( \omega\in\complement A\).
    \end{enumerate}
    Alors la fonction
    \begin{equation}
        F(x)=\int_{\Omega}f(x,\omega)d\mu(\omega)
    \end{equation}
    est dérivable en \( a\) et nous pouvons permuter la dérivée et l'intégrale :
    \begin{equation}
        F'(a)=\int_{\Omega}\frac{ \partial f }{ \partial x }(a,\omega)d\mu(\omega).
    \end{equation}
\end{theorem}

\begin{proof}
    Soit une suite \( (x_n)\) dans \( B(a,\delta)\) telle que \( x_n\neq a\) et \( x_n\to a\). Si la limite
    \begin{equation}
        \lim_{n\to \infty} \frac{ F(a)-F(x_n) }{ a-x_n }
    \end{equation}
    existe et ne dépend pas de la suite choisie, alors la fonction \( F\) est dérivable en \( a\) et sa dérivée vaut cette limite. Par linéarité de l'intégrale, nous devons étudier la limite
    \begin{equation}    \label{EqLIiralx}
        \lim_{n\to \infty} \int_{\Omega}\frac{ f(a,\omega)-f(x_n,\omega) }{ a-x_n }d\omega,
    \end{equation}
    montrer qu'elle existe, ne dépend pas de la suite choisie et vaut \( \int_{\Omega}\partial_xf(a,\omega)d\omega\). Nous sommes donc dans un problème d'inversion de limite et de dérivée pour lequel nous allons utiliser le théorème de la convergence dominée de Lebesgue. D'abord nous posons
    \begin{equation}    \label{EqAFOUbQB}
        g_n(\omega)=\frac{ f(x_n,\omega)-f(a,\omega) }{ x_n-a }.
    \end{equation}
    Cela est une suite de fonctions dans \( L^1(\Omega)\) parce qu'à la fois \( a\) et \( x_n\) sont dans \( B(a,\delta)\). De plus nous avons
    \begin{equation}
        \lim_{n\to \infty} g_n(\omega)=\frac{ \partial f }{ \partial x }(a,\omega)
    \end{equation}
    parce que nous savons que \( f\) est dérivable en \( a\) pour tout \( \omega\in\complement A\). En ce qui concerne la majoration de \( g_n\), nous utilisons le théorème des accroissements finis (théorème \ref{ThoAccFinisUneVariable}) sur le numérateur de \eqref{EqAFOUbQB}. Pour tout \( n\) et pour tout \( \omega\in \complement A\), il existe un \( \theta_{n,\omega}\) dans \( \mathopen] a , x_n \mathclose[\) tel que
        \begin{equation}
            f(x_n,\omega)-f(a,\omega)=\frac{ \partial f }{ \partial x }(\theta_{n,\omega},\omega)(x_n-a),
        \end{equation}
        donc
        \begin{equation}
            | g_n(\omega) |=\left| \frac{ \partial f }{ \partial x }(\theta_{n,\omega},\omega) \right| \leq G(\omega).
        \end{equation}
        La dernière inégalité provient des hypothèses. Le théorème de la convergence dominée de Lebesgue (théorème \ref{ThoConvDomLebVdhsTf}) nous permet alors de calculer la limite \eqref{EqLIiralx} :
        \begin{equation}
            \lim_{n\to \infty} \int_{\Omega}g_n(\omega)d\omega=\int_{\Omega}\lim_{n\to \infty} g_n(\omega)d\omega=\int_{\Omega}\frac{ \partial f }{ \partial x }(a,\omega)d\omega.
        \end{equation}
        Notons que l'existence de la dernière intégrale fait partie du théorème de la convergence dominée.

        Nous avons donc prouvé que la limite de gauche existait et ne dépendant pas de la suite choisie. Donc \( F\) est dérivable en \( a\) et la dérivée vaut cette limite :
        \begin{equation}
            F'(a)=\int_{\Omega}\frac{ \partial f }{ \partial x }(a,\omega)d\mu(\omega).
        \end{equation}
\end{proof}

\begin{theorem}
		Supposons $f$ continue et sa dérivée partielle $\frac{ \partial f }{ \partial x }$ continue sur $[\alpha,\beta]\times[a,\alpha[$. Supposons que $F(x)=\int_a^{\infty}f(x,t)dt$ converge et que $\int_a^{\infty}\frac{ \partial f }{ \partial x }dt$ converge uniformément. Alors $F$ est $C^1$ sur $[\alpha,\beta]$ et 
		\begin{equation}
			\frac{ dF }{ dx }=\int_a^{\infty}\frac{ \partial f }{ \partial x }dt.
		\end{equation}
\end{theorem}

En ce qui concerne les fonctions dans \( \eR^n\), il y a les  propositions \ref{PropDerrSSIntegraleDSD} et \ref{PropAOZkDsh} qui parlent de différentiabilité sous l'intégrale.

%--------------------------------------------------------------------------------------------------------------------------- 
\subsection{Absolue continuité}
%---------------------------------------------------------------------------------------------------------------------------

\begin{definition}      \label{DefAbsoluCont}
    Une fonction \( F\colon \eR\to \eR\) est \defe{absolument continue}{absolument continue} sur \( \mathopen[ a , b \mathclose]\) si il existe une fonction \( f\) sur \( \mathopen[ a , b \mathclose]\) telle que
    \begin{equation}
        F(x)=\int_a^xf(t)dt
    \end{equation}
    pour tout \( x\in\mathopen[ a , b \mathclose]\).
\end{definition}

\begin{theorem}     \label{ThoDerSousIntegrale}
    Soit \( A\) un ouvert de \( \eR\) et \( \Omega\), un espace mesuré. Soit une fonction \( f\colon A\times \Omega\to \eR\) et
    \begin{equation}
        F(x)=\int_{\Omega}f(x,\omega)d\omega.
    \end{equation}
    Nous supposons les points suivants.
    \begin{enumerate}
        \item
            La fonction \( f\) est mesurable en tant que fonction \( A\times\Omega\to \eR\). Pour chaque \( x\in A\), la fonction \( f(x,\cdot)\) est intégrable sur \( \Omega\).
        \item
            Pour presque tout \( \omega\in\Omega\), la fonction \( f(x,\omega)\) est une fonction absolument continue de \( x\).
        \item
            La fonction \( \frac{ \partial f }{ \partial x }\) est localement intégrable, c'est à dire que pour tout \( \mathopen[ a , b \mathclose]\subset A\),
            \begin{equation}
                \int_a^b\int_{\Omega}\left| \frac{ \partial f }{ \partial x }(x,\omega) \right| d\omega\,dx<\infty.
            \end{equation}
    \end{enumerate}
    Alors la fonction \( F\) est absolument continue et pour presque tout \( x\in A\), la dérivée est donné par
    \begin{equation}
        \frac{ d }{ dx }\int_{\Omega}f(x,\omega)d\omega=\int_{\Omega}\frac{ \partial f }{ \partial x }(x,\omega)d\omega.
    \end{equation}
\end{theorem}

La proposition suivante sera utilisée entre autres pour montrer que sous l'hypothèse d'une densité continue, la loi exponentielle est sans mémoire, proposition \ref{PropREXaIBg}.
\begin{proposition}		\label{PropDerrFnAvecBornesFonctions}
Soit $f(x,t)$ une fonction continue sur $[\alpha,\beta]\times[a,b]$, telle que $\frac{ \partial f }{ \partial x }$ existe et soit continue sur $]\alpha,\beta[\times[a,b]$. Soient $\varphi(x)$ et $\psi(x)$, des fonctions continues de $[\alpha,\beta]$ dans $\eR$ et admettant une dérivée continue sur $]\alpha,\beta [$. Alors la fonction
\begin{equation}
	F(x)=\int_{\varphi(x)}^{\psi(x)}f(x,t)dt
\end{equation}
admet une dérivée continue sur $]\alpha,\beta[$ et
\begin{equation}	\label{EqFormDerrFnAvecBorneNInt}
	\frac{ dF }{ dx }=\int_{\varphi(x)}^{\psi(x)}\frac{ \partial f }{ \partial x }(x,t)dt+f\big( x,\psi(x) \big)\cdot\frac{ d\psi }{ dx }- f\big( x,\varphi(x) \big)\cdot\frac{ d\varphi }{ dx }.
\end{equation}
\end{proposition}
%TODO : une preuve de ce théorème ? allons allons ...

L'exemple qui suit devrait pouvoir être rendu rigoureux en utilisant des distributions correctement.

\begin{example} \label{ExfYXeQg}
    Si \( g\) est une fonction continue, la fonction suivante est une primitive de \( g\) :
    \begin{equation}
        \int_0^xf(t)dt=\int_0^{\infty}f(t)\mtu_{t<x}(t)dt.
    \end{equation}
    Nous nous proposons de justifier \emph{de façon un peu heuristique} le fait que ce soit bien une primitive de \( g\) en considérant la fonction
    \begin{equation}
        f(t,x)=g(t)\mtu_{t<x}(t).
    \end{equation}
    Nous posons
    \begin{equation}
        F(x)=\int_0^{\infty}f(x,t)dt,
    \end{equation}
    et nous calculons \( F'\) en permutant la dérivée et l'intégrale\footnote{Ceci n'est pas rigoureux : il faudrait avoir un théorème à propos de distributions qui permet de le faire.}. D'abord,
    \begin{equation}
        f(t,x)=\begin{cases}
            g(t)    &   \text{si \( t\in \mathopen[ 0 , x \mathclose]\)}\\
            0    &    \text{sinon.}
        \end{cases}
    \end{equation}
    La dérivée de \( f\) par rapport à \( x\) est donnée par la distribution
    \begin{equation}
        \frac{ \partial f }{ \partial x }(t_0,x_0)=g(t_0)\delta(t_0-x_0).
    \end{equation}
    Donc
    \begin{equation}
        F'(x_0)=\int_0^{\infty}\frac{ \partial f }{ \partial x }(t,x_0)dt=\int_0^{\infty}g(t)\delta(t-x_0)=g(x_0),
    \end{equation}
    comme attendu.
\end{example}

Cet exemple est rendu rigoureux par la proposition suivante.
\begin{proposition} \label{PropJLnPpaw}
    Si \( f\in L^1(\eR)\), alors la fonction
    \begin{equation}
        F(x)=\int_{-\infty}^xf(t)dt
    \end{equation}
    est presque partout dérivable et pour les points où elle l'est nous avons \( F'(x)=f(x)\).
\end{proposition}
\index{fonction!définie par une intégrale}
%TODO : une preuve.

%--------------------------------------------------------------------------------------------------------------------------- 
\subsection{Différentiabilité sous l'intégrale}
%---------------------------------------------------------------------------------------------------------------------------

Le théorème suivant est restrictif sur l'ensemble d'intégration (qui doit être compact), mais accepte des fonctions de plusieurs variables, ce qui est un premier pas vers la différentiabilité.
\begin{proposition}		\label{PropDerrSSIntegraleDSD}
    Supposons $A\subset\eR^m$ ouvert et $B\subset\eR^n$ compact. Nous considérons une fonction \( f\colon A\times B\to \eR\). Si pour un $i\in\{ i,\ldots,n \}$, la dérivée partielle $\frac{ \partial f }{ \partial x_i }$ existe dans $A\times B$ et est continue, alors la fonction
    \begin{equation}
        F(x)=\int_Bf(x,t)dt
    \end{equation}
    admet une dérivée partielle dans la direction \( x_i\) sur \( A\). Cette dérivée partielle y est continue et
    \begin{equation}
        \frac{ \partial F }{ \partial x_i }(a)=\int_B\frac{ \partial f }{ \partial x_i }(a,t)dt,
    \end{equation}
    pour tout \( a\) dans l'ouvert \( A\).
\end{proposition}
\index{fonction!définie par une intégrale}
\index{inversion de limite}
L'égalité signifie que l'on peut \og dériver sous le signe intégral\fg.

\begin{proof}
    Nous procédons en plusieurs étapes.
    \begin{subproof}
    \item[\( F\) est dérivable]
            
        Nous voulons prouver que \( \frac{ \partial F }{ \partial x_i }(a,t)\) existe. Pour cela nous posons
        \begin{equation}
            g_l(t)=\frac{ f(a_1,\ldots, a_i+\epsilon_l,\ldots, a_n,t)-f(a_1,\ldots, a_k,\ldots, a_n,t) }{ \epsilon_l }
        \end{equation}
        où \( \epsilon_l\) est une suite de nombres tendant vers zéro. La fonction \( f\) est dérivable dans la direction \( x_i\) si et seulement si \( \lim_{l\to \infty}g_l(t) \) existe et ne dépend pas du choix de la suite. À ce moment, la valeur de la dérivée partielle sera cette limite. Dans notre cas, nous savons que \( f\) admet une dérivée partielle dans la direction \( x_i\) et donc nous avons
        \begin{equation}
            \frac{ \partial f }{ \partial x_i }(a,t)=\lim_{l\to \infty} g_l(t).
        \end{equation}
        
        De la même façon pour \( F\) nous avons
        \begin{equation}
            \frac{ \partial F }{ \partial x_i }=\lim_{l\to \infty} \int_{B}g_l(t)dt.
        \end{equation}
        Sous-entendu : si la limite de droite ne dépend pas de la suite choisie, alors \( \frac{ \partial F }{ \partial x_i }\) existe et vaut cette limite.

        Vu la continuité de \( f\), le seul point à vérifier pour le théorème de la convergence dominée de Lebesgue est l'existence d'une fonction intégrable de \( t\) majorant \( g_l\). Pour cela le théorème de accroissements finis (théorème \ref{ThoAccFinisUneVariable}) appliqué à la fonction \( \epsilon\mapsto f(a_n,\ldots, a_i+\epsilon,\ldots, a_n)\) nous dit que
        \begin{equation}
            f(a_1,\ldots, a_i+\epsilon_l,\ldots, a_n,t)-f(a_1,\ldots, a_i,\ldots, a_n,t)=\epsilon_l\frac{ \partial f }{ \partial x_i }(a_1,\ldots, \theta,\ldots, a_n,t)
        \end{equation}
        pour un certain \( \theta\in B(a_i,\epsilon_l)\). Notons que ce \( \theta\) dépend de \( t\) mais pas de \( l\). Vu que \( \partial_if\) est continue par rapport à ses deux variables, si \( K\) est un voisinage compact autour de \( a\), il existe \( M>0\) tel que
        \begin{equation}    \label{EqMXqviPC}
            \left| \frac{ \partial f }{ \partial x_i }(x,t) \right| < M
        \end{equation}
        pour tout \( x\in K\) et tout \( t\in B\). La valeur de \( \frac{ \partial f }{ \partial x_i }(a_1,\ldots, \theta,\ldots, a_n,t)\) est donc bien majorée par rapport à \( \theta\) et par rapport à \( t\) en même temps par une constante qui n'a pas de mal à être intégrée sur le compact \( B\).
        
        Le théorème de la convergence dominée (théorème \ref{ThoConvDomLebVdhsTf}) s'applique donc bien et nous avons
        \begin{equation}
            \lim_{l\to \infty} \int_Bg_l(t)dt=\int_B\lim_{l\to \infty} g_l(t)=\int_B\frac{ \partial f }{ \partial x_i }(a,t)dt.
        \end{equation}
        Le membre de droite ne dépendant pas de la suite \( \epsilon_l\) choisie, le membre de gauche est bien la dérivée de \( F\) par rapport à \( x_i\) et nous avons
        \begin{equation}
            \frac{ \partial F }{ \partial x_i }(a)=\int_B\frac{ \partial f }{ \partial x_i }(a,t)dt.
        \end{equation}
        Cela prouve la première partie de la proposition.

    \item[La dérivée est continue]

        Soit \( K\) un voisinage compact autour de \( a\) et \( U'\) un ouvert tel que \( a\in U'\subset K\). Nous avons encore la majoration \eqref{EqMXqviPC} sur \( U'\) et donc le théorème de continuité sous l'intégrale \ref{ThoKnuSNd} nous indique que la fonction
        \begin{equation}
            \begin{aligned}
                U'&\to \eR \\
                x&\mapsto \int_{B}\frac{ \partial f }{ \partial x_i }(x,t)dt 
            \end{aligned}
        \end{equation}
        est continue en \( a\).
        
    \end{subproof}
\end{proof}

Une conséquence de la proposition \ref{PropDerrSSIntegraleDSD} est que si elle fonctionne pour tous les \( i\), alors \( F\) est différentiable et même de classe \( C^1\), et la différentielle de \( F\) s'obtient comme intégrale de la différentielle de \( f\).

\begin{proposition}\label{PropAOZkDsh}
    Supposons $A\subset\eR^m$ ouvert et $B\subset\eR^n$ compact. Si pour tout $i\in\{ i,\ldots,n \}$, la dérivée partielle $\frac{ \partial f }{ \partial x_i }$ existe dans $A\times B$ et est continue, alors \( F\) est de classe \( C^1\) et
    \begin{equation}
        (dF)_a=\int_B(df_t)_adt
    \end{equation}
    où \( f_t(x)=f(x,t)\).
\end{proposition}

\begin{proof}
    En vertu de la proposition \ref{PropDerrSSIntegraleDSD}, toutes les dérivées partielles de \( F\) sont continues. Cela implique que \( F\) est de classe \( C^1\) par la proposition \ref{PropDerContCun} et que la différentielle s'écrive en terme des dérivées partielles avec la formule usuelle. Nous avons alors
    \begin{subequations}
        \begin{align}
            (dF)_a(u)&=\sum_k\frac{ \partial F }{ \partial x_k }(a)u_k\\
            &=\int_B\sum_k\frac{ \partial f }{ \partial x_k }(a,t)dt\\
            &=\int_B\sum_k\frac{ \partial f_t }{ \partial x_k }(a)u_kdt\\
            &=\int_B (df_t)_a(u)dt.
        \end{align}
    \end{subequations}
    Cela est la formule annoncée.
\end{proof}

Un autre théorème tourne autour du pot, et me semble inutile.
\begin{theorem} \label{ThoOLAQyRL}
    Soit \( (\Omega,\mu)\) un espace mesuré, une fonction \( f\colon \eR^n\times \Omega\to \eR\) et \( a\in \eR^n\). Nous considérons la fonction
    \begin{equation}
        F(x)=\int_{\Omega}f(x,\omega)d\mu(\omega).
    \end{equation}
    Pour chaque \( k=1,\ldots, n\) nous supposons avoir
    \begin{equation}
        \frac{ \partial F }{ \partial x_k }(a)=F_{|_k}'(a)=\int_{\Omega}\frac{ \partial f_{|_k} }{ \partial t }(a_k,\omega)d\mu(\omega)
    \end{equation}
    où \( F_{|_k}(t)=F(a_1,\ldots, t,\ldots, a_n)\) et \( f_{|_k}\) est définie de façon similaire.

    Nous supposons de plus que les fonctions \( \partial_{x_k}F\) sont continues.

    Alors \( F\) est de classe \( C^1\) et sa différentielle est donnée par
    \begin{equation}
        df_a=\int_{\Omega}(df_{\omega})_ad\omega
    \end{equation}
    où \( f_{\omega}\) est définie par \( f_{\omega}(x)=f(x,\omega)\).
\end{theorem}

\begin{proof}
    Étant donné que les dérivées partielles de \( F\) en \( a\) existent et sont continues, la proposition \ref{PropDerContCun} dit que \( F\) est différentiable et que
    \begin{equation}
        dF_a(u)=\sum_{k=1}^n\frac{ \partial F }{ \partial x_k }(a)u_k.
    \end{equation}
    La linéarité de l'intégrale et les hypothèses nous donnent alors
    \begin{subequations}
        \begin{align}
            df_a(u)&=\sum_{k=1}^n\frac{ \partial F }{ \partial x_k }(a)u_k\\
            &=\int_{\Omega}\sum_k\frac{ \partial f_{|_k} }{ \partial t }(a_k;\omega)u_kd\mu(\omega)\\
            &=\int_{\Omega}\sum_k\frac{ \partial f }{ \partial x_k }(a;\omega)u_kd\mu(\omega)\\
            &=\int_{\Omega}(df_{\omega})_a(u)d\mu(\omega),
        \end{align}
    \end{subequations}
    et donc \( df_a=\int_{\Omega}(df_{\omega})_ad\mu(\omega)\).
\end{proof}
Notons qu'en passant aux composantes, ce théorème fonctionne tout aussi bien pour des fonctions à valeurs dans un espace vectoriel normé de dimension finie plutôt que dans \( \eR\).

%+++++++++++++++++++++++++++++++++++++++++++++++++++++++++++++++++++++++++++++++++++++++++++++++++++++++++++++++++++++++++++ 
\section{Formes différentielles exactes et fermées}
%+++++++++++++++++++++++++++++++++++++++++++++++++++++++++++++++++++++++++++++++++++++++++++++++++++++++++++++++++++++++++++

Nous avons déjà parle de formes différentielles et de leurs intégrales sur un chemin dans la section \ref{SecFormDiffRappel}.

\begin{definition}  \label{DefEFKQmPs}
La forme différentielle $\omega$ est \defe{exacte}{forme!différentielle!exacte} si il existe une fonction $f$ telle que $\omega=df$; elle est dite \defe{fermée}{forme!différentielle!fermée} si $d\omega=0$.
\end{definition}

Une forme différentielle $\omega$ continue de degré $1$ sur $D$ est \defe{exacte}{forme!différentielle!exacte} si il existe $F\colon D\to \eR$ telle que $\omega=dF$. On dit que la forme est \defe{fermée}{forme!différentielle!fermée} si $d\omega=0$. Dire que la forme différentielle $\omega=fdx+gdy$ est fermée, c'est dire que
\begin{equation}
    \frac{ \partial g }{ \partial x }=\frac{ \partial f }{ \partial y }.
\end{equation}

\begin{theorem}        \label{ThoFermeExactFormRappel}
Supposons que $D\subset\eR^n$ soit un ouvert simplement connexe. Alors toute forme différentielle de degré $1$ et de classe $C^1$ sur $D$ qui est fermée est exacte.
\end{theorem}

Nous allons prouver ce théorème dans un cas un peu moins général : celui d'un domaine étoilé de \( \eR^2\) plutôt que simplement connexe de \( \eR^n\).

\begin{theorem} \label{ThoMSofFxL}
Soit $D\subset\eR^2$, une ouvert étoilé, et $\omega$, une $1$-forme fermée de classe $C^1$. Alors $\omega$ est exacte.
\end{theorem}
\begin{proof}

Soit $D\subset\eR^2$, un ouvert étoilé par rapport à l'origine. Soient $f\colon D\to \eR$, $g\colon D\to \eR$, des fonctions de classe $C^1$ telles que
\begin{equation}
	\frac{ \partial f }{ \partial y }=\frac{ \partial g }{ \partial x }
\end{equation}
sur $D$, et
\begin{equation}		\label{EqIMDefFformI33}
	F(x,y)=\int_0^1\big[  f(tx,ty)x+g(tx,ty)y  \big]dt
\end{equation}
pour tout $(x,y)\in D$. 

Étant donné que nous ne définissons $F(x,y)$ que pour des $(x,y)\in D$, la fonction $t\mapsto f(tx,ty)$ est $C^1$ sur tout le compact $[0,1]$ et aucune divergence de l'intégrale n'est à craindre. Nous sommes donc dans le cadre de la proposition \ref{PropDerrSSIntegraleDSD}, et nous pouvons dériver sous le signe intégral.

Nous calculons, en utilisant la règle de dérivation de fonctions composées
\begin{equation}		\label{EqIMI33dsdsFlolo}
	\begin{aligned}[]
		\frac{ \partial F }{ \partial x }(x,t)	&=\int_0^1\left[   f\frac{ \partial f }{ \partial x }(tx,ty)x+f(tx,ty)+t\frac{ \partial g }{ \partial x }(tx,ty)y  \right]dt\\
		&=\int_0^1\left[ t\Big( x\frac{ \partial f }{ \partial x }(tx,ty)+y\frac{ \partial f }{ \partial y }(tx,ty) \Big)+f(tx,ty) \right]dt
	\end{aligned}
\end{equation}
où nous avons utilisé l'hypothèse $\partial_yf=\partial_xg$. Ce qui se trouve dans la parenthèse n'est autre que $\partial_t\big( f(tx,ty) \big)$, plus précisément, si nous posons $\mF(x,y,t)=f(tx,ty)$, nous avons
\begin{equation}
	\frac{ \partial \mF }{ \partial t }(x,y,t)= x\frac{ \partial f }{ \partial x }(tx,ty)+y\frac{ \partial f }{ \partial y }(tx,ty).
\end{equation}
En recopiant le résultat \eqref{EqIMI33dsdsFlolo} en termes de $\mF$, nous avons
\begin{equation}
	\begin{aligned}[]
		\frac{ \partial F }{ \partial x }(x,t)	&=\int_0^1\left( t\frac{ \partial \mF }{ \partial t }(x,y,t)+\mF(x,y,t) \right)dt\\
		&=\int_0^1\partial_t\big( t\mF(x,y,t) \big)dt\\
		&=\big[ f\mF(x,y,t) \big]_0^1\\
		&=\mF(x,y,1)\\
		&=f(x,y).
	\end{aligned}
\end{equation}
Le résultat correspondant pour $\frac{ \partial F }{ \partial y }(x,y)=g(x,y)$ s'obtient de la même manière. Nous avons donc obtenu que
\begin{equation}		\label{EqIMFormI33Fffdd}
	\begin{aligned}[]
		\frac{ \partial F }{ \partial x }&=f,  &\text{et}&& \frac{ \partial F }{ \partial y }=g.
	\end{aligned}
\end{equation}
En ayant prouvé cela, nous avons prouvé que si $\omega=fdx+gdy$ avec $\partial_yf=\partial_xg$, alors $\omega=dF$ où $F$ est définie par \eqref{EqIMDefFformI33}.
\end{proof}

\begin{proof}[Démonstration alternative du théorème \ref{ThoMSofFxL}]
Nous posons $u=tx$ et $v=ty$, ainsi que $\mF(x,y,t)=f(u,v)$ et $\mG(x,y,t)=g(u,v)$. Avec cette notation, nous avons $F(x,y)=\int_0^1\big( x\mF(x,y,t)+y\mG(x,y,t) \big)dt$, et
\begin{equation}
	\begin{aligned}[]
		\frac{ \partial \mF }{ \partial x }&=\frac{ \partial f }{ \partial u }\frac{ \partial u }{ \partial x }+\frac{ \partial f }{ \partial v }\frac{ \partial v }{ \partial x }=t\frac{ \partial f }{ \partial u },\\
		\frac{ \partial \mG }{ \partial x }&=t\frac{ \partial g }{ \partial u }.
	\end{aligned}
\end{equation}
Ainsi,
\begin{equation}
	\begin{aligned}[]
		\frac{ \partial F }{ \partial x }	&=\int_0^1\left( x\frac{ \partial \mF }{ \partial x }+\mF+y\frac{ \partial G }{ \partial x } \right)dt\\
							&=\int_0^1\left( xt\frac{ \partial f }{ \partial u } +\mF+yt\frac{ \partial g }{ \partial u } \right)dt\\
							&=\int_0^1\left[  t\left( x\frac{ \partial f }{ \partial u }+y\frac{ \partial f }{ \partial v } \right)+\mF  \right]dt.
	\end{aligned}
\end{equation}
où nous avons utilisé le fait que, par hypothèse, $\frac{ \partial g }{ \partial u }=\frac{ \partial f }{ \partial v }$. Nous calculons par ailleurs que
\begin{equation}
	\frac{ \partial F }{ \partial t }=\frac{ \partial f }{ \partial u }\frac{ \partial u }{ \partial t }+\frac{ \partial f }{ \partial v }\frac{ \partial v }{ \partial t }=x\frac{ \partial f }{ \partial u }+y\frac{ \partial f }{ \partial v }.
\end{equation}
Donc, nous avons
\begin{equation}
	\frac{ \partial F }{ \partial x }=\int_0^1\left( t\frac{ \partial \mF }{ \partial t }+\mF \right)dt=\int_0^1\frac{ \partial  }{ \partial t }(t\mF)dt.
\end{equation}
Par conséquent,
\begin{equation}
	\frac{ \partial F }{ \partial x }=[t\mF]_0^1=\mF(x,y,1)=f(x,y).
\end{equation}
Le même genre de calculs fournit $\frac{ \partial F }{ \partial y }=g(x,y)$.
\end{proof}
    
%--------------------------------------------------------------------------------------------------------------------------- 
\subsection{Formes différentielles exactes et fermées}
%---------------------------------------------------------------------------------------------------------------------------

Considérons une fonction différentiable $f\colon D\to \eR$. Pour chaque $x\in D$, nous avons l'application différentielle
\begin{equation}
	\begin{aligned}
		df(x)\colon \eR^n&\to \eR \\
		v&\mapsto \sum_{i=1}^n\frac{ \partial f }{ \partial x_i }(x)v_i, 
	\end{aligned}
\end{equation}
c'est à dire que $df$ est une $1$-forme différentielle dont les composantes sont
\begin{equation}
	df(x)=\frac{ \partial f }{ \partial x_1 }(x)dx_1+\ldots+\frac{ \partial f }{ \partial x_n }(x)dx_n.
\end{equation}

Il est naturel de se demander si toutes les formes différentielles sont des différentielles de fonctions. Une réponse complète est délicate à établir, mais a d'innombrables conséquences en physique, notamment en ce qui concerne l'existence d'un potentiel vecteur pour le champ magnétique dans les équations de Maxwell.
\begin{definition}
	Deux classes importantes de formes différentielles sont à mettre en évidence
	\begin{enumerate}
		\item
			Une forme différentielle $\omega$ sur un ouvert $D\subset\eR^n$ est \defe{exacte}{forme!différentielle!exacte} si il existe une fonction différentiable $f\colon D\to \eR$ telle que
			\begin{equation}
				 \omega_x=df(x)
			\end{equation}
			pour tout $x\in D$.
		\item
			Une $1$-forme de classe $C^1$ sur l'ouvert $D$ est \defe{fermée}{forme!différentielle!fermée} si pour tout $i,j=1,\ldots n$, nous avons
			\begin{equation}
				\frac{ \partial a_i }{ \partial x_j }=\frac{ \partial a_j }{ \partial x_i }.
			\end{equation}
	\end{enumerate}
\end{definition}

\begin{proposition}
	Si $\omega$ est une $1$-forme exacte de classe $C^1$, alors $\omega$ est fermée.
\end{proposition}

\begin{proof}
	Le fait que $\omega$ soit exacte implique l'existence d'une fonction $f$ telle que $\omega=df$, c'est à dire
	\begin{equation}
		\omega_x=\sum_i a_i(x)dx_i=\sum_i\frac{ \partial f }{ \partial x_i }(x)dx_i,
	\end{equation}
	c'est à dire que $a_i(x)=\frac{ \partial f }{ \partial x_i }(x)$. L'hypothèse que $\omega$ est $C^1$ implique que $f$ est $C^2$, et donc que nous pouvons inverser l'ordre de dérivation pour les dérivées secondes $\partial^2_{ij}f=\partial^2_{ji}f$. Nous pouvons donc faire le calcul suivant :
	\begin{equation}
		\frac{ \partial a_i }{ \partial x_j }=\frac{ \partial  }{ \partial x_j }\frac{ \partial f }{ \partial x_i }=\frac{ \partial  }{ \partial x_i }\frac{ \partial f }{ \partial x_j }=\frac{ \partial a_j }{ \partial x_i },
	\end{equation}
	ce qu'il fallait démontrer.
\end{proof}

Ceci est assez pour les formes différentielles pour cette année.

%+++++++++++++++++++++++++++++++++++++++++++++++++++++++++++++++++++++++++++++++++++++++++++++++++++++++++++++++++++++++++++ 
\section{Théorème d'Abel angulaire}
%+++++++++++++++++++++++++++++++++++++++++++++++++++++++++++++++++++++++++++++++++++++++++++++++++++++++++++++++++++++++++++

\begin{theorem}[Abel angulaire\cite{KXjFWKA}]   \label{ThoTGjmeen}
    Soit \( \sum_{n}a_nz^n\) une série entière de rayon de convergence plus grand ou égal à \( 1\) et de somme \( f\). Soit \( \theta_0\in\mathopen[ 0 , \frac{ \pi }{2} \mathclose[\). Nous posons
    \begin{equation}
        \Delta_{\theta_0}=\{ z=1-\rho e^{i\varphi}\tq \rho>0,\varphi\in\mathopen[ \theta_0 , \theta_0 \mathclose],| z |<1 \}.
    \end{equation}
    Nous supposons de plus que \( \sum_na_n\) converge. Alors
    \begin{equation}
        \lim_{\substack{z\to1\\z\in\Delta_0}}f(z)=\sum_{k=0}^{\infty}a_k.
    \end{equation}
\end{theorem}
\index{Abel!angulaire}
\index{convergence!suite numérique!Abel angulaire}
\index{somme partielles!Abel angulaire}
\index{série!entière!Abel angulaire}

\begin{proof}

    Le résultat de ce théorème est que l'on peut calculer la limite \( z\to 1\) avec des chemins contenus dans un domaine de la forme de celui dessiné à la figure \ref{LabelFigJGuKEjH}. % From file JGuKEjH
    \newcommand{\CaptionFigJGuKEjH}{La zone dans laquelle peut être le chemin qui va vers \( z=1\).}
    \input{Fig_JGuKEjH.pstricks}

    De façon très classique nous posons
    \begin{equation}
        \begin{aligned}[]
            S&=\sum_{k=0}^{\infty}a_k&S_n&=\sum_{k=0}^na_k,
        \end{aligned}
    \end{equation}
    et \( R_n=S-S_n\). En particulier \( a_n=R_{n-1}-R_n\). 

    Le but du théorème est de montrer que \( \sum a_nz^n\) converge vers \( S\) lorsque \( z\) converge vers \( 1\) à l'intérieur de \( \Delta_{\theta_0}\). Pour cela nous calculons pour un \( N\) donné la différence \( \sum_{n=0}^{N}a_nz^n-S_N\) en triant les termes par ordre de \( R_n\), en isolant le terme \( R_0\) et le terme \( R_N\) :
    \begin{subequations}
        \begin{align}
            \sum_{n=0}^Na_nz^n-S_N&=\sum_{n=1}^Na_n(z^n-1)\\
            &=\sum_{n=1}^N(R_{n-1}-R_n)(z^n-1)\\
            &=R_0(z-1)+\sum_{n=1}^{N-1}R_n(z^{n+1}-1-z^n+1)+R_N(z^N-1)\\
            &=R_0(z-1)+\sum_{n=1}^{N-1}R_nz^n(z-1)+R_N(z^N-1)\\
            &=(z-1)\sum_{n=0}^{N-1}R_nz^n+R_N(z^N-1).
        \end{align}
    \end{subequations}
    Cela est valable pour tout \( N\) et \( | z |<1\). Nous avons donc
    \begin{equation}
        \sum_{n=0}^Na_nz^n-S_N=(z-1)\sum_{n=0}^{N-1}R_nz^n+R_N(z^N-1).
    \end{equation}
    Par hypothèse nous avons \( \lim_{N\to \infty} R_N=0\). Et de plus le membre de gauche converge parce que chacun des deux termes converge séparément. En passant à la limite nous avons pour tout \( | z |<1\) :
    \begin{equation}
        f(z)-S=(z-1)\sum_{n=0}^{\infty}R_nz^n.
    \end{equation}
    Nous voudrions étudier le comportement de la différence \( f(z)-S\) lorsque \( z\) tend vers \( 1\). Pour cela nous nous fixons \( \epsilon>0\) et \( N\geq 1\) tel que \( | R_n |<\epsilon\) dès que \( n\geq N\). Alors pour tout \( | z |<1\) nous avons
    \begin{subequations}
        \begin{align}
            | f(z)-S |&\leq | z-1 |\left( \sum_{n=0}^N| R_n | \underbrace{|z^n |}_{\leq 1} +\sum_{n=N+1}^{\infty}\underbrace{| R_n |}_{\leq \epsilon} |z^n | \right)\\
            &\leq | z-1 |\sum_{n=0}^N| R_n |+\epsilon\frac{ | z-1 | }{ 1-| z | }
        \end{align}
    \end{subequations}
    où nous avons utilisé la somme de la série géométrique \eqref{EqASYTiCK} et l'égalité \( | z^n |=| z |^n\). Avant de nous particulariser à \( z\in\Delta_{\theta_0}\) nous devons anticiper un problème au dénominateur en multipliant par le binôme conjugué :
    \begin{equation}
        \frac{ | z-1 | }{ 1-| z | }=\frac{ | z-1 |(1+| z |) }{ 1-| z |^2 }.
    \end{equation}
    C'est maintenant que nous nous particularisons à \( z\in\Delta_{\theta_0}\) en posant \( z=\rho e^{i\varphi}\) et en remarquant que \( | z |^2=1-2\rho\cos(\varphi)+\rho^2\). Nous avons le calcul suivant :
    \begin{subequations}
        \begin{align}
            \frac{ | z-1 | }{ 1-| z | }&=\frac{ \rho(1+| z |) }{ 2\rho\cos(\varphi)-\rho^2 }\\
            &=\frac{ 1+| z | }{ 2\cos(\varphi)-\rho}\\
            &\leq\frac{ 2 }{ 2\cos(\varphi)-\rho }\\
            &\leq\frac{ 2 }{ 2\cos(\varphi)-\cos(\theta_0) }\\
            &\leq\frac{ 2 }{ 2\cos(\theta_0)-\cos(\theta_0) }\\
            &=\frac{ 2 }{ \cos(\theta_0) }.
        \end{align}
    \end{subequations}
    Quelque justifications.
    \begin{itemize}
        \item Vu que nous avons dans l'idée de faire \( \rho\to 0\) nous supposons que \( \rho<\cos(\theta_0)\).
        \item Nous avons \( \cos(\varphi)>\cos(\theta_0)\) parce que \( z\) est dans \( \Delta_{\theta_0}\).
    \end{itemize}
    Nous avons donc, pour tout \( z\in\Delta_{\theta_0}\) que
    \begin{equation}
        | f(z)-S |\leq | z-1 |\sum_{n=0}^N| R_n |+\epsilon\frac{ 2 }{ \cos(\theta_0) }.
    \end{equation}
    Il suffit de prendre \( \rho\) assez petit pour que 
    \begin{equation}
        | z-1 |\sum_{n=0}^N| R_n |<\epsilon
    \end{equation}
    et nous avons
    \begin{equation}
        | f(z)-S |\leq \epsilon\left( 1+\frac{ 2 }{ \cos(\theta_0) } \right).
    \end{equation}
    Nous avons donc bien \( \lim_{\substack{z\to 1\\z\in\Delta_0}}f(z)=S\), comme nous le voulions.
\end{proof}

La réciproque du théorème d'Abel angulaire est que si \( f(z)=\sum_na_nz^n\) sur \( B(0,1)\) se prolonge par continuité en \( z=1\) alors cette prolongation se fait par \( f(1)=\sum_na_n\). Cela est faux comme le montre l'exemple suivant.

\begin{example}
    Nous considérons la série entière \( \sum_{n=0}^{\infty}(-1)^nz^n\) qui converge\footnote{C'est la série géométrique de raison \( -z\).} vers
    \begin{equation}
        f(z)=\frac{1}{ 1+z }
    \end{equation}
    sur \( B(0,1)\). De plus nous avons
    \begin{equation}
        \lim_{\substack{z\to 1\\    | z |<1}}\frac{1}{ 1+z }=\frac{ 1 }{2}.
    \end{equation}
    Donc la fonction converge bien vers quelque chose lorsque \( z\) tend vers \( 1\). La fonction \( f\) se prolonge par continuité en \( 1\). Pourtant la série es coefficients \( \sum_n(-1)^n\) ne converge pas.
\end{example}

Le théorème suivant donne une espèce d'inverse au théorème d'Abel angulaire. En effet il dit que si la série converge  en allant vers \( 1\) le long de l'axe réel, alors ça converge vers la somme des coefficients. Il faut cependant une hypothèse en plus sur les \( a_n\).
\begin{theorem}[Théorème taubérien faible\cite{KXjFWKA}]
    Soit \( \sum_na_nz^n\) une série entière de rayon de convergence \( 1\) et de somme \( f\). Nous supposons
    \begin{enumerate}
        \item
            Il existe \( S\in \eC\) tel que \( \lim_{\substack{x\to 1\\x\in\mathopen] -1 , 1 \mathclose[}}f(x)=S\).
            \item
                \( \lim_{n\to \infty} na_n=0\).
    \end{enumerate}
    Alors la série \( \sum_{n=0}^{\infty}a_n\) converge et vaut \( S\).
\end{theorem}
\index{théorème!taubérien faible}

\begin{proof}
    Nous notons \( S_n=\sum_{k=0}a_k\) et \( M=\sup_{k\geq 1}k| a_k |\), qui est fini par hypothèse. Pour \( x\in \mathopen] 0 , 1 \mathclose[\) et \( n\geq 0\) nous avons
    \begin{equation}
        S_n-f(x)=\sum_{k=1}^na_k-\sum_{k=1}^na_kx^k-\sum_{k=n+1}^{\infty}a_kx^k=\sum_{k=1}^na_k(1-x^k)-\sum_{k=n+1}^{\infty}a_kx^k.
    \end{equation}
    Nous utilisons la série géométrique sous la forme \( 1-x^k=(1-x)\sum_{i=0}^nx^i\) pour écrire
    \begin{subequations}
        \begin{align}
            S_n-f(x)&=\sum_{k=1}^na_k(1-x)\underbrace{\sum_{i=0}^{k-1}x^i}_{\leq k}-\sum_{k=n+1}^{\infty}a_kx^k\\
            &\leq\sum_{k=1}^nka_k(1-x)-\sum_{k=n+1}^{\infty}a_kx^k,
        \end{align}
    \end{subequations}
    donc en passant à la norme
    \begin{subequations}
        \begin{align}
            \big| S_n-f(x) \big|&\leq (1-x)Mn+\sum_{k=n+1}| a_k |x^k\\
            &\leq (1-x)Mn+\sum_{k=n+1}^{\infty}\underbrace{\frac{ k }{ n }| a_k |}_{\leq M/n}x^k\\
            &\leq (1-x)Mn+\frac{ M }{ n }\sum_{k=n+1}^{\infty}x^k\\
            &\leq (1-x)Mn+\frac{ M }{ n }\frac{1}{ 1-x }.
        \end{align}
    \end{subequations}
    Ce que nous cherchons à étudier est le comportement \( x\to 1\) et montrer que \( S_n\to S\), ce qui nous incite à calculer \( | S_n-f(1-\frac{ \epsilon }{n  }) |\) avec \( 0<\epsilon<1\) :
    \begin{equation}
        \big| S_n-f\big( 1-\frac{ \epsilon }{ n } \big) \big|\leq \epsilon M+\epsilon.
    \end{equation}
    Nous choisissons \( N_1\) tel que \( \frac{ M }{ n }\leq \epsilon^2\) dès que \( n\geq N_1\). En sus nous savons que 
    \begin{equation}
        \lim_{\epsilon\to 0}f(1-\epsilon)=S.
    \end{equation}
    Nous choisissons \( N_2\) de telle sorte à avoir
    \begin{equation}
        \left| f\left( 1-\frac{ \epsilon }{ n } \right)-S \right| <\epsilon,
    \end{equation}
    et en prenant \( n\geq\max(N_1,N_2)\) nous avons
    \begin{equation}
        | S_n-S |\leq \left| S_n-f\left( 1-\frac{ \epsilon }{ n } \right) \right| +\left| f\left( 1-\frac{ \epsilon }{ n } \right)-S \right|  \leq \epsilon M+2\epsilon.
    \end{equation}
    Il suffit de choisit \( \epsilon\) suffisamment petit (en particulier pour que \( \epsilon M\) soit petit) pour montrer que \( | S_n-S |\) est borné par un nombre arbitrairement petit.
\end{proof}
