% This is part of Exercices de mathématique pour SVT
% Copyright (c) 2010
%   Laurent Claessens et Carlotta Donadello
% See the file fdl-1.3.txt for copying conditions.

\begin{corrige}{TD6-0001}

	\begin{enumerate}
		\item
			Les primitives de la fonction $t\mapsto\sin(t)$ sont les fonctions $F(t)=-\cos(t)+C$ où $C$ est n'importe quelle constante. Il y a une primitive par choix de constante. Nous voulons sélectionner celle qui vérifie $F(\pi/2)=0$ :
			\begin{equation}
				F(\pi/2)=-\cos(\pi/2)+C=C.
			\end{equation}
			Afin que cela vaille zéro, nous avons besoin de $C=0$. La fonction $F$ dont nous parlons dans cet exercice est donc
			\begin{equation}
				F(t)=-\cos(t).
			\end{equation}
		\item
			Explicitons la condition $(y e^{F})'=1$ en utilisant la règle de dérivation de produit :
			\begin{equation}
				(y e^{F})=y' e^{F}+yF'e^F=e^F\big( y'+y\sin(t)\big).
			\end{equation}
			En imposant que cela soit $1$, nous trouvons la condition
			\begin{equation}
				e^{F(t)}\big( y'(t)+y(t)\sin(t) \big)=1,
			\end{equation}
			en passant $ e^{F(t)}$ de l'autre côté, et en remplaçant $F(t)$ par sa valeur $-\cos(t)$, nous trouvons
			\begin{equation}
				y'(t)+y(t)\sin(t)= e^{\cos(t)},
			\end{equation}
			c'est à dire l'équation de départ.
		\item
			Étant donné que $(y e^{F})'=1$, il existe une constante $C$ telle que $ye^F=t+C$ (c'est la primitive de $1$). Nous avons donc comme solution générale :
			\begin{equation}		\label{EqszzuyecttC}
				y(t)= e^{\cos(t)}(t+C).
			\end{equation}
		\item
			Parmi toutes les solutions données par l'équation \eqref{EqszzuyecttC}, nous devons sélectionner celle telle que $y(0)=0$. En posant $t=0$ dans la formule générale de $y(t)$, nous trouvons que
			\begin{equation}
				y(0)=0.
			\end{equation}
			Pour que cela soit zéro, il faut sélectionner $C=0$. La solution de l'équation de départ qui s'annule en zéro est donc
			\begin{equation}
				y(t)=t e^{\cos(t)}.
			\end{equation}
	\end{enumerate}

\end{corrige}
