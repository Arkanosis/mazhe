% This is part of Mes notes de mathématique
% Copyright (c) 2011-2014
%   Laurent Claessens
% See the file fdl-1.3.txt for copying conditions.

%+++++++++++++++++++++++++++++++++++++++++++++++++++++++++++++++++++++++++++++++++++++++++++++++++++++++++++++++++++++++++++ 
\section{Intégrale par rapport à une mesure}
%+++++++++++++++++++++++++++++++++++++++++++++++++++++++++++++++++++++++++++++++++++++++++++++++++++++++++++++++++++++++++++

Une mesure \( \mu\) sur un espace mesurable \( (\Omega,\tribA)\) permet de définir une fonctionnelle linéaire sur l'ensemble des fonctions mesurables \( \Omega\to \eR\). Cette fonctionnelle linéaire est l'intégrale que nous allons définir à présent.

Si \( Y\in \tribA\) et si \( f\) est une fonction simple nous définissons
\begin{equation}
    \int_Yfd\mu=\sum_ia_i\mu(Y\cap E_i).
\end{equation}
Pour une fonction \( \mu\)-mesurable générale \( f\colon \Omega\to \mathopen[ 0 , \infty \mathclose]\) nous définissons l'intégrale de \( f\) sur \( Y\) par
\begin{equation}        \label{EqDefintYfdmu}
    \int_Yfd\mu=\sup\Big\{ \int_Yhd\mu\,\text{où \( h\) est une fonction simple et mesurable telle que \( 0\leq h\leq f\)} \Big\}.
\end{equation}
Maintenant nous définissons
\begin{equation}
    \mu(f)=\int_{\Omega}f
\end{equation}
si \( f\) est une fonction mesurable sur \( \Omega\).

\begin{remark}
    Dans \( \eR^d\), quasiment toutes les fonctions et ensembles sont mesurables. En effet la construction d'ensembles non mesurables demande obligatoirement l'utilisation de l'axiome du choix; de tels ensembles doivent être construits «exprès pour». Il y a très peu de chances pour que vous tombiez sur un ensemble non mesurable de \( \eR^d\) sans que vous ne vous en rendiez compte.
\end{remark}

\begin{remark}
    «Mesurable» ne signifie pas «intégrable». Par exemple la fonction 
    \begin{equation}
        \begin{aligned}
            f\colon \eR&\to \bar\eR \\
            \omega&\mapsto\begin{cases}
            \frac{1}{ \omega }    &   \text{si $ \omega\neq 0$}\\
            \infty    &    \text{$\omega=0$}.
            \end{cases}
        \end{aligned}
    \end{equation}
    est mesurable, mais non intégrable.
\end{remark}

Le lemme suivant nous aide à détecter des fonctions presque partout nulles.
\begin{lemma}   \label{Lemfobnwt}
    Soit \( f\) une fonction mesurable positive ou nulle telle que
    \begin{equation}
        \int_{\Omega}fd\mu=0.
    \end{equation}
    Alors \( f=0\) \( \mu\)-presque partout.
\end{lemma}

\begin{proof}
    L'ensemble des points \( x\in\Omega\) tels que \( f(x)\neq 0\) peut s'écrire comme une union dénombrable disjointe :
    \begin{equation}
        \{ x\in\Omega\tq f(x)\neq 0 \}=\bigcup_{i=0}^{\infty}E_i
    \end{equation}
    avec
    \begin{subequations}
        \begin{align}
            E_0&=\{ x\in\Omega\tq f(x)>1 \}\\
            E_i&=\{ x\in\Omega\tq \frac{1}{ i+1 }\leq f(x)<\frac{1}{ i } \}.
        \end{align}
    \end{subequations}
    Si un des ensembles \( E_i\) est de mesure non nulle, alors nous pouvons considérer la fonction simple \( h(x)=\frac{1}{ i+1 }\mtu_{E_i}\) dont l'intégrale sur \( \Omega\) est strictement positive. Par conséquent le supremum de la définition \eqref{EqDefintYfdmu} est strictement positif.

    Nous savons donc que \( \mu(E_i)=0\) pour tout \( i\). Étant donné que la mesure d'une union disjointe dénombrable est égale à la somme des mesures, nous avons
    \begin{equation}
        \mu\{ x\in\Omega\tq f(x)\neq 0 \}=0,
    \end{equation}
    ce qui signifie que \( f\) est nulle \( \mu\)-presque partout.
\end{proof}

\begin{corollary}   \label{CorjLYiSm}
    Soit \( f\) une fonction mesurable sur l'espace mesuré \( (\Omega,\tribA,\mu)\) telle que
    \begin{equation}
        \int_{\Omega}f\mtu_{f>0}d\mu=0.
    \end{equation}
    Alors \( f\leq 0\) presque partout.
\end{corollary}

\begin{proof}
    Nous avons l'égalité d'ensembles
    \begin{equation}
        \{ f\mtu_{f>0}\neq 0 \}=\{ \mtu_{f>0}\neq 0 \}.
    \end{equation}
    Mais lemme \ref{Lemfobnwt} implique que \( f\mtu_{f>0}\) est nulle presque partout, c'est à dire que la mesure de l'ensemble du membre de gauche est nulle par conséquent
    \begin{equation}
        \mu\{ \mtu_{f>0}\neq 0 \}=0.
    \end{equation}
    Cela signifie que la fonction \( f\) est presque partout négative ou nulle.
\end{proof}

\begin{lemma}   \label{LemPfHgal}
    Soit \( f\) une fonction telle que \( | f(x)|\leq g(x) \) pour tout \( x\in\Omega\). Si \( g\) est intégrable, alors \( f\) est intégrable.
\end{lemma}

\begin{proof}
    Nous décomposons \( f\) en parties positives et négatives :
    \begin{subequations}
        \begin{align}
            A_+&=\{ x\in\Omega\tq f(x)>0 \}\\
            A_-&=\{ x\in\Omega\tq f(x)<0 \}.
        \end{align}
    \end{subequations}
    Nous posons \( f_+(x)=f(x)\mtu_{A_+}\) et \( f_-(x)=f(x)\mtu_{A_-}\). Nous avons \( f=f_+-f_-\) et
    \begin{equation}
        \int_{\Omega}f=\int_{A_+}f+\int_{A_-}f
    \end{equation}
    parce que \( \Omega=A_+\cup A_-\cup\{ x\in\Omega\tq f(x)=0 \}\). Si \( \varphi\) est une fonction simple qui majore \( f_+\) nous avons
    \begin{equation}
        \varphi(x)=\sum_{k}a_k\mtu_{E_k}(x)\leq f(x)\mtu_{A_+}(x)\leq g(x).
    \end{equation}
    Par conséquent le supremum qui définit \( \int f_+\) est inférieur au supremum qui définit \( \int g\). La fonction \( f_+\) est donc intégrable. La même chose est valable pour la fonction \( f_-\).
\end{proof}

%---------------------------------------------------------------------------------------------------------------------------
\subsection{Mesure dominée}
%---------------------------------------------------------------------------------------------------------------------------

Soient \( \mu\) et \( \nu\) deux mesures sur le même espace \( \Omega\) et la même tribu \( \tribA\). Nous disons que la mesure \( \mu\) est \defe{dominée}{dominée!mesure}\cite{PersoFeng} par \( \nu\) si pour tout ensemble mesurable \( A\), \( \nu(A)=0\) implique \( \mu(A)=0\).

La mesure \( \mu\) est \defe{portée}{portée!mesure} par l'ensemble \( E\in\tribA\) si pour tout \( A\in\tribA\), 
\begin{equation}
    \mu(A)=\mu(A\cap E).
\end{equation}

Nous écrivons que \( \mu\perp\nu\)\nomenclature[Y]{\( \mu\perp\nu\)}{mesures perpendiculaires} si il existe un ensemble \( E\in\tribA\) tel que \( \mu\) soit porté par \( E\) et \( \nu\) soit porté par \( \complement E\).

\begin{theorem}[Radon-Nikodym\cite{NikoLi}]\index{Radon-Nikodym}
    Soient \( \mu\) et \( m\) deux mesures \( \sigma\)-finies sur un espace métrisable \( (\Omega,\tribA)\).
    \begin{enumerate}
        \item
            Il existe un unique couple de mesures \( \mu_1\) et \( \mu_2\) telles que
            \begin{enumerate}
                \item
                    \( \mu=\mu_1+\mu_2\)
                \item
                    \( \mu_1\) est dominé par \( m\)
                \item
                    \( \mu_2\perp m\).
            \end{enumerate}
            Dans ce cas, les mesures \( \mu_1\) et \( \mu_2\) sont positives et \( \sigma\)-finies.
        \item
            À égalité \(  m\)-presque partout près, il existe une unique fonction mesurable positive \( f\) telle que pour tout mesurable \( A\),
            \begin{equation}
                \mu_1(A)=\int_Ad\mu_1=\int_{\Omega}\mtu_Afd m.
            \end{equation}
        \item
            À égalité \( m\)-presque partout près, il existe une unique fonction positive mesurable \( h\) telle que \( \mu_1=hm\).
    \end{enumerate}
\end{theorem}
%TODO : une preuve

\begin{corollary}   \label{CorZDkhwS}
    Si \( \mu\) es une mesure \( \sigma\)-finie dominée par la mesure \( \sigma\)-finie \( m\), alors \( \mu\) possède une unique fonction de densité.
\end{corollary}

\begin{corollary}       \label{CorDomDens}
    Soient \( \mu\) et \( m\), deux mesures positives \( \sigma\)-finies sur \( (\Omega,\tribA)\). Alors \( m\) domine \( \mu\) si et seulement si \( \mu\) possède une densité par rapport à \( m\).
\end{corollary}
 
\begin{proof}
    Si \( \mu\) est dominée par \( m\), alors la décomposition \( \mu=\mu+0\) satisfait le théorème de Radon-Nikodym. Par conséquent il existe une fonction \( f\) telle que
    \begin{equation}
        \mu(A)=\int_Afdm.
    \end{equation}
    Cette fonction est alors une densité pour \( \mu\) par rapport à \( m\).

    Pour la réciproque, nous supposons que \( \mu\) a une densité \( f\) par rapport à \( m\), et que \( A\) est une ensemble de \( m\)-mesure nulle :
    \begin{equation}
        m(A)=\int_{\Omega}\mtu_Adm=0.
    \end{equation}
    Cela signifie que la fonction \( \mtu_A\) est \( m\)-presque partout nulle. La fonction produit \( \mtu_Af\) est également nulle \( m\)-presque partout, et par conséquent
    \begin{equation}
        \mu(A)=\int_{\Omega}\mtu_Afdm=0.
    \end{equation}
\end{proof}

\begin{probleme}
    Est-ce que la démonstration de cela ne demande pas la convergence monotone d'une façon ou d'une autre ?
\end{probleme}

%--------------------------------------------------------------------------------------------------------------------------- 
\subsection{Primitive et intégrale}
%---------------------------------------------------------------------------------------------------------------------------

\begin{definition}
    Soit \( I\) un intervalle de \( \eR\) et une fonction \( f\colon I\to \eR\). La fonction \( F\colon I\to \eR\) est une \defe{primitive}{primitive!fonction} de \( f\) si \( F\) est dérivable sur \( I\) et si \( F'(x)=f(x)\) pour tout \( x\) dans \( I\).
\end{definition}

\begin{proposition}[Primitive et intégrale] \label{PropEZFRsMj}
    Soit \( f\) une fonction intégrable sur \( \mathopen[ a , b \mathclose]\) et \( x_0\) un point de continuité de \( f\). Nous considérons la fonction
    \begin{equation}
        \begin{aligned}
            F\colon \mathopen[ a , b \mathclose]&\to \eR \\
            x&\mapsto \int_{\mathopen[ a , x \mathclose]}f(t)dt.
        \end{aligned}
    \end{equation}
    Cette fonction est dérivable en \( x_0\) et \( F'(x_0)=f(x_0)\).
\end{proposition}

\begin{proof}
    Soit \( \epsilon>0\). Par continuité de \( f\) en \( x_0\), il existe une fonction \( \alpha\) telle que
    \begin{equation}
        f(x_0+h)=f(x_0)+\alpha(h)
    \end{equation}
    avec \( \lim_{h\to 0} \alpha(h)=0\). De plus il existe un \( \delta>0\) tel que \( \alpha(h)<\epsilon\) pour tout \( h<\delta\). À partir de maintenant nous ne considérons plus que de tels \( h\).

    Nous calculons la dérivée de \( F\) en \( x_0\). Pour cela,
    \begin{subequations}
        \begin{align}
            F(x_0+h)-F(x_0)&=\int_{x_0}^{x_0+h}f(t)dt\\
        &=\int_0^hf(x_0+t)dt\\
        &=\int_0^h\big[ f(x_0)+\alpha(t) \big]dt\\
        &=hf(x_0)+\int_0^{h}\alpha(t)dt.
        \end{align}
    \end{subequations}
    Nous avons donc, pour tout \( h<\delta\),
    \begin{equation}
        hf(x_0)-h\epsilon\leq F(x_0+h)-F(x_0)\leq hf(x_0)+h\epsilon.
    \end{equation}
    En divisant par \( h\) et en prenant la limite \( h\to 0\),
    \begin{equation}
        F'(x_0)\in B\big( f(x_0),\epsilon \big).
    \end{equation}
    Cela étant valable pour tout \( \epsilon>0\) nous en déduisons que
    \begin{equation}
        F'(x_0)=f(x_0).
    \end{equation}
\end{proof}

Ce petit résultat nous donne une façon «pratique» de calculer des intégrales en cherchant des primitives. Nous rappelons qu'en vertu du corollaire \ref{CorZeroCst}, une fonction ne possède qu'une seule primitive à constante près.

Le théorème suivant est à utiliser pour calculer des intégrales des fonctions réelle.
\begin{theorem}[Théorème fondamental du calcul intégral]
    Soit \( f\) une fonction continue sur l'intervalle \( I\subset \eR\) et \( F\) une primitive de \( f\) sur \( I\). Alors
    \begin{equation}
        \int_a^bf(t)dt=F(b)-F(a).
    \end{equation}
\end{theorem}

\begin{proof}
    Nous avons vu par la proposition \ref{PropEZFRsMj} que la fonction
    \begin{equation}
        \begin{aligned}
            \tilde F\colon I&\to \eR \\
            x&\mapsto  \int_a^xf(t)dt
        \end{aligned}
    \end{equation}
    était une primitive de \( f\); c'est même l'unique\footnote{Corollaire \ref{CorZeroCst}.} primitive de \( f\) sur \( I\) à s'annuler pour \( x=a\). Nous avons évidemment
    \begin{equation}
        \int_a^bf(t)dt=\tilde F(b).
    \end{equation}
    Si \( F\) est une primitive quelconque, il suffit de soustraire sa valeur en \( x=a\) : \( \tilde F(x)=F(x)-F(a)\) et donc
    \begin{equation}
        \int_a^bf(t)dt=\tilde F(b)=F(b)-F(a),
    \end{equation}
    comme il fallait le prouver.
\end{proof}

%--------------------------------------------------------------------------------------------------------------------------- 
\subsection{Théorème d'approximation}
%---------------------------------------------------------------------------------------------------------------------------

\begin{theorem}[Théorème d'approximation\cite{YHRSDGc}]     \label{ThoAFXXcVa}
    Soit \( (X,\tribB,\mu)\) un espace mesuré où \( \tribB\) sont les boréliens de \( X\). Soit \( A\in \tribB\) tel que \( A\subset W\) où \( W\) est un ouvert avec \( \mu(W)<\infty\). Soit aussi \( \epsilon>0\).
    \begin{enumerate}
        \item
            Il existe un fermé \( F\) et un ouvert \( V\) tels que \( \mu(V)<\infty\) et
            \begin{equation}
                F\subset A\subset V
            \end{equation}
            et \( \mu(V\setminus F)<\epsilon\).
        \item
            Il existe \( f\in C^0(X,\eR)\) nulle hors de \( W\) vérifiant \( 0\leq f\leq 1\) et
            \begin{equation}
                \int_X| \mtu_A-f |^pd\mu(x)<\epsilon.
            \end{equation}
    \end{enumerate}
\end{theorem}
% TODO : la preuve est dans la référence. Il faut replacer ce théorème après la définition de l'intégrale.

%--------------------------------------------------------------------------------------------------------------------------- 
\subsection{Mesure à densité}
%---------------------------------------------------------------------------------------------------------------------------

Si \( \mu\) est une mesure sur \( \eR^d\), une fonction \( f\colon \eR^d\to \eR\) est une \defe{densité}{densité d'une mesure} si pour tout \( A\subset\eR^d\) nous avons
\begin{equation}
    \mu(A)=\int_Af(x)dx
\end{equation}
où \( dx\) est la mesure de Lebesgue.

%+++++++++++++++++++++++++++++++++++++++++++++++++++++++++++++++++++++++++++++++++++++++++++++++++++++++++++++++++++++++++++ 
\section{Permuter limite et intégrale}
%+++++++++++++++++++++++++++++++++++++++++++++++++++++++++++++++++++++++++++++++++++++++++++++++++++++++++++++++++++++++++++

%--------------------------------------------------------------------------------------------------------------------------- 
\subsection{Convergence uniforme}
%---------------------------------------------------------------------------------------------------------------------------

\begin{proposition}[Permuter limite et intégrale]       \label{PropbhKnth}
    Soit \( f_n\to f\) uniformément sur un ensemble mesuré \( A\) de mesure finie. Alors si les fonctions \( f_n\) et \( f\) sont intégrables sur \( A\), nous avons
    \begin{equation}
        \lim_{n\to \infty} \int_A f_n=\int_A \lim_{n\to \infty} f_n.
    \end{equation}
\end{proposition}

\begin{proof}
    Notons \( f\) la limite de la suite \( (f_n)\). Pour tout \( n\) nous avons les majorations
    \begin{subequations}
        \begin{align}
            \left| \int_A f_n d\mu-\int_A fd\mu \right| &\leq \int_A| f_n-f |d\mu\\
            &\leq \int_A \| f_n-f \|_{\infty}d\mu\\
            &=\mu(A)\| f_n-f \|_{\infty}
        \end{align}
    \end{subequations}
    où \( \mu(A)\) est la mesure de \( A\). Le résultat découle maintenant du fait que \( \| f_n-f \|_{\infty}\to 0\).
\end{proof}
Il existe un résultat considérablement plus intéressant que cette proposition. En effet, l'intégrabilité de \( f\) n'est pas nécessaire. Cette hypothèse peut être remplacée soit par l'uniforme convergence de la suite (théorème \ref{ThoUnifCvIntRiem}), soit par le fait que les normes des \( f_n\) sont uniformément bornées (théorème de la convergence dominée de Lebesgue \ref{ThoConvDomLebVdhsTf}).

\begin{theorem}[\cite{BJblWiS}]			\label{ThoUnifCvIntRiem}
    La limite uniforme d'une suite de fonctions intégrables sur un borné est intégrable, et on peut permuter la limite et l'intégrale. 
    
    Plus précisément, soit \( A\) un ensemble de \( \mu\)-mesure finie et \( f_n\colon A\to \eR\) des fonctions intégrables sur \( A\). Si la limite \( f_n\to f\) est uniforme, alors \( f\) est intégrable sur \( A\) et nous pouvons inverser la limite et l'intégrale :
    \begin{equation}
        \lim_{n\to \infty} \int_A f_n=\int_A\lim_{n\to \infty} f_n.
    \end{equation}
\end{theorem}

\begin{proof}
    Soit \( \epsilon>0\) et \( n\) tel que \( \| f_n-f \|_{\infty}\leq \epsilon\) (ici la norme uniforme est prise sur \( A\)). Étant donné que \( f_n\) est intégrable sur \( A\), il existe une fonction simple \( \varphi_n\) qui minore \( f_n\) telle que
    \begin{equation}
        \left| \int_{A}\varphi_n-\int_A f_n \right| <\epsilon.
    \end{equation}
    La fonction \( \varphi_n+\epsilon\) est une fonction simple qui majore la fonction \( f\). Si \( \psi\) est une fonction simple qui minore \( f\), alors
    \begin{equation}
        \int_A\psi\leq\int_A\varphi_n+\epsilon\leq\int_A f_n+\epsilon\mu(A).
    \end{equation}
    Par conséquent le supremum qui définit \( \int_A f\) existe, ce qui montre que \( f\) est intégrable. Le fait qu'on puisse inverser la limite et l'intégrale est maintenant une conséquence de la proposition \ref{PropbhKnth}.
\end{proof}

\begin{remark}
    L'hypothèse sur le fait que \( A\) est de mesure finie est importante. Il n'est pas vrai qu'une suite uniformément convergente de fonctions intégrables est intégrables. En effet nous avons par exemple la suite
    \begin{equation}
        f_n(x)=\begin{cases}
            1/x    &   \text{si \( x<n\)}\\
            0    &    \text{sinon}
        \end{cases}
    \end{equation}
    qui converge uniformément vers \( f(x)=1/x\) sur \( A=\mathopen[ 1 , \infty [\). Le limite n'est cependant pas intégrable sur \( A\).
\end{remark}

%---------------------------------------------------------------------------------------------------------------------------
\subsection{Convergence monotone}
%---------------------------------------------------------------------------------------------------------------------------

\begin{theorem}[Théorème de la convergence monotone ou de Beppo-Levi\cite{mathmecaChoi}] \label{ThoConvMonFtBoVh}
    Soit un espace mesuré \( (\Omega,\tribA,\mu)\) et \( (f_n)\) une suite croissante de fonctions mesurables à valeurs dans \( \mathopen[ 0 , \infty \mathclose]\). Alors la limite ponctuelle \( \lim_{n\to \infty} f_n\) existe, est mesurable et
    \begin{equation}    \label{EqFHqCmLV}
        \lim_{n\to \infty} \int_{\Omega}f_nd\mu= \int_{\Omega}\lim_{n\to \infty} f_nd\mu,
    \end{equation}
    cette intégrable valant éventuellement \( \infty\).
\end{theorem}
\index{théorème!convergence!monotone}
\index{théorème!Beppo-Levi}

\begin{proof}
    La limite ponctuelle de la suite est la fonction à valeurs dans \( \mathopen[ 0 , \infty \mathclose]\) donnée par
    \begin{equation}
        f(x)=\lim_{n\to \infty} f_n(x).
    \end{equation}
    Ces limites existent parce que pour chaque \( x\) la suite \( f_n(x)\) est une suite numérique croissante. Nous notons
    \begin{equation}
        I_0=\int_{\Omega}fd\mu.
    \end{equation}
    Nous posons par ailleurs
    \begin{equation}
        I_n=\int_{\Omega}f_n.
    \end{equation}
    Cela est une suite numérique croissante qui a par conséquent une limite que nous notons \( I=\lim_{n\to \infty} I_n\). Notre objectif est de montrer que \( I=I_0\). D'abord par croissance de la suite, pour tous $n$ nous avons \( I_n\leq I_0\), par conséquent \( I\leq I_0\).

    Nous prouvons maintenant l'inégalité dans l'autre sens en nous servant de la définition \eqref{EqDefintYfdmu}. Soit une fonction simple \( h\) telle que \( h\leq f\), et une constante \( 0<C<1\). Nous considérons les ensembles
    \begin{equation}
        E_n=\{ x\in\Omega\tq f_n(x)\geq Ch(x) \}.
    \end{equation}
    Ces ensembles vérifient les propriétés \( E_n\subset E_{n+1}\) et \( \bigcup_{n=1}^{\infty}E_n=\Omega\). Pour chaque \( n\) nous avons les inégalités
    \begin{equation}
        \int_{\Omega}f_n\geq\int_{E_n}f_n\geq C\int_{E_n}h.
    \end{equation}
    Si nous prenons la limite \( n\to\infty\) dans ces inégalités,
    \begin{equation}
        \lim_{n\to \infty} \int_{\Omega}f_n\geq C\lim_{n\to \infty} \int_{E_n}h=C\int_{\Omega}h.
    \end{equation}
    Par conséquent \( \lim_{n\to \infty} \int f_n\geq C\int_{\Omega}h\). Mais étant donné que cette inégalité est valable pour tout \( C\) entre \( 0\) et \( 1\), nous pouvons l'écrire sans le \( C\) :
    \begin{equation}        \label{EqzAKEaU}
        \lim_{n\to \infty} \int_{\Omega}f_n\geq \int_{\Omega}h.
    \end{equation}
    Par définition, l'intégrale de \( f\) est donné par le supremum des intégrales de \( h\) où \( h\) est une fonction simple dominée par \( f\). En prenant le supremum sur \( h\) dans l'équation \eqref{EqzAKEaU} nous avons
    \begin{equation}
        \lim_{n\to \infty} \int_{\Omega}f_n\geq\int_{\Omega}f,
    \end{equation}
    ce qu'il nous fallait.
\end{proof}

\begin{remark}
    La proposition \ref{PropWBavIf} ainsi que le lemme \ref{LemYFoWqmS} montrent qu'une fonction mesurable peut-être écrite comme limite croissante de fonctions simples. Cela permet de démontrer des théorèmes en commençant par prouver sur les fonctions simples et en utilisant Beppo-Levi pour généraliser.
\end{remark}

\begin{remark}
    Une des raisons de demander la positivité des fonctions \( f_n\) est de n'avoir pas d'ambiguïté à parler d'intégrales qui valent \( \infty\). Si par exemple nous prenons \( \Omega=\mathopen[ 0 , 1 \mathclose]\) et que nous considérons
    \begin{equation}
        f_n(x)=\begin{cases}
            0    &   \text{si \( x\leq \frac{1}{ n }\)}\\
            \frac{1}{ x }    &    \text{sinon}.
        \end{cases}
    \end{equation}
    Ce sont des fonctions intégrables, mais la limite étant la fonction \( 1/x\), l'égalité \eqref{EqFHqCmLV} est une égalité entre deux intégrales valant \( \infty\).
\end{remark}

\begin{corollary}[Inversion de somme et intégrales] \label{CorNKXwhdz}
    Si \( (u_n)\) est une suite de fonctions mesurables positives ou nulles, alors
    \begin{equation}
        \sum_{i=0}^{\infty}\int u_i=\int\sum_{i=0}^{\infty}u_i.
    \end{equation}
\end{corollary}

\begin{proof}
    Nous considérons la suite des sommes partielles de \( (u_n)\) : \( f_n(x)=\sum_{i=0}^nu_n(x)\). Le théorème de la convergence monotone (théorème \ref{ThoConvMonFtBoVh}) implique que
    \begin{equation}
        \lim_{n\to \infty} \int f_n=\int\lim_{n\to \infty} f_n.
    \end{equation}
    Nous remplaçons maintenant \( f_n\) par sa valeur en termes des \( u_i\) et dans le membre de gauche nous permutons l'intégrale avec la somme finie :
    \begin{equation}
        \lim_{n\to \infty} \sum_{i=0}^{\infty}\int u_n=\int\sum_{i=0}^{\infty}u_n,
    \end{equation}
    ce qu'il fallait démontrer.
\end{proof}

\begin{lemma}[Lemme de Fatou]\index{lemme!Fatou}\index{Fatou}   \label{LemFatouUOQqyk}
    Soit \( (\Omega,\tribA,\mu)\) un espace mesuré et \( f_n\colon \Omega\to \mathopen[ 0 , \infty \mathclose]  \) une suite de fonctions mesurables. Alors la fonction \( f(x)=\liminf f_n(x)\) est mesurable et
    \begin{equation}
        \int_{\Omega}\liminf f_nd\mu\leq\liminf\int_{\Omega}fd\mu.
    \end{equation}
\end{lemma}
%TODO : pour la mesurabilité, il faudra citer un théorème du genre de celui fait avec le sup.

\begin{proof}
    Nous posons 
    \begin{equation}
        g_n(x)=\inf_{i\geq n}f_i(x).
    \end{equation}
    Cela est une suite croissance de fonctions positives mesurables telles que, par définition, 
    \begin{equation}
        \lim_{n\to \infty}g_n(x)=\liminf f_n(x).
    \end{equation}
    Nous pouvons y appliquer le théorème de la convergence monotone,
    \begin{equation}
        \lim_{n\to \infty} \int g_n(x)=\int\liminf f_n(x).
    \end{equation}
    Par ailleurs, pour chaque \( i\geq n\) nous avons
    \begin{equation}
        \int g_n\leq \int f_i,
    \end{equation}
    en passant à l'infimum nous avons
    \begin{equation}
        \int g_n\leq \inf_{i\geq n}\int f_i,
    \end{equation}
    et en passant à la limite nous avons
    \begin{equation}
        \int\liminf f_n=\lim_{n\to \infty} \int g_n\leq \lim_{n\to \infty} \inf_{i\geq n}\int f_i=\liminf_{i\to\infty}\inf f_i.
    \end{equation}
\end{proof}

L'inégalité donnée dans ce lemme n'est en général pas une égalité, comme le montre l'exemple suivant :
\begin{equation}
    f_i=\begin{cases}
        \mtu_{\mathopen[ 0 , 1 \mathclose]}    &   \text{si \( i\) est pair}\\
        \mtu_{\mathopen[ 1 , 2 \mathclose]}    &    \text{si \( i\) est impair}.
    \end{cases}
\end{equation}
Nous avons évidemment \( g_n(x)=0\) tandis que \( \int_{\mathopen[ 0 , 2 \mathclose]}f_i=1\) pour tout \( i\).

%---------------------------------------------------------------------------------------------------------------------------
\subsection{Convergence dominée de Lebesgue}
%---------------------------------------------------------------------------------------------------------------------------

\begin{theorem}[Convergence dominée de Lebesgue]        \label{ThoConvDomLebVdhsTf}
    Soit \( (f_n)_{n\in\eN}\) une suite de fonctions intégrables sur \( (\Omega,\tribA,\mu)\) à valeurs dans \( \eC\) ou \( \eR\). Nous supposons que  \( f_n\to f\) simplement sur \( \Omega\) presque partout et qu'il existe une fonction intégrable \( g\) telle que
    \begin{equation}
        | f_n(x) |< g(x) 
    \end{equation}
    pour presque\footnote{Si il n'y avait pas le «presque» ici, ce théorème serait à peu près inutilisable en probabilité ou en théorie des espaces \( L^p\), comme dans la démonstration du théorème de Fischer-Riesz \ref{ThoGVmqOro} par exemple.} tout \( x\in\Omega\) et pour tout \( n\in \eN\). Alors
    \begin{enumerate}
        \item
            \( f\) est intégrable,
        \item
           $\lim_{n\to \infty} \int_{\Omega}f_n=\int_\Omega f$,
        \item
            $\lim_{n\to \infty} \int_{\Omega}| f_n-f |=0$.
    \end{enumerate}
\end{theorem}
\index{théorème!convergence!dominée de Lebesgue}
\index{dominée!convergence (Lebesgue)}

\begin{proof}

    La fonction limite \( f\) est intégrable parce que \( | f |\leq g\) et \( g\) est intégrable (lemme \ref{LemPfHgal}). Par hypothèse nous avons
    \begin{equation}
        -g(x)\leq f_n(x)\leq g(x).
    \end{equation}
    En particulier la fonction \( g_n=f_n+g\) est positive et mesurable si bien que le lemme de Fatou (lemme \ref{LemFatouUOQqyk}) implique
    \begin{equation}
        \int_{\Omega}\liminf g_n\leq\liminf\int_{\Omega}g_n.
    \end{equation}
    Évidement nous avons \( \liminf g_n=f+g\), de telle sorte que
    \begin{equation}
        \int f+\int g\leq \liminf\int g_n=\liminf\int f_n+\int g,
    \end{equation}
    et le nombre \( \int g\) étant fini, nous pouvons le retrancher des deux côtés de l'inégalité :
    \begin{equation}
        \int f\leq\liminf\int f_n.
    \end{equation}
    Afin d'obtenir une minoration de \( \int f\) nous refaisons exactement le même raisonnement en utilisant la suite de fonctions \( k_n=-f_n\to k=-f\). Nous obtenons que
    \begin{equation}
        \int k\geq\liminf\int k_n=-\limsup\int f_n,
    \end{equation}
    et par conséquent
    \begin{equation}    \label{IneqsndMYTO}
        \liminf\int f_n\leq\int f\leq\limsup\int f_n.
    \end{equation}
    La limite supérieure étant plus grande ou égale à la limite inférieure, les trois quantités dans les inégalités \eqref{IneqsndMYTO} sont égales.

    Nous prouvons maintenant le troisième point. Soit la suite de fonctions
    \begin{equation}
        h_n(x)=| f_n(x)-f(x) |
    \end{equation}
    qui tend ponctuellement vers zéro. De plus
    \begin{equation}
    h_n(x)\leq | f_n(x) |+| f(x) |\leq 2g(x),
    \end{equation}
    ce qui prouve que les \( h_n\) majorés par une fonction intégrable. Donc
    \begin{equation}
        \lim_{n\to \infty} \int_{\Omega}| f_n-f |= \lim_{n\to \infty} \int_{\Omega}h_n(x)dx=\int_{\Omega}\lim_{n\to \infty} | f_n(x)-f(x) |=0
    \end{equation}
\end{proof}

\begin{remark}
    Lorsque nous travaillons sur des problèmes de probabilités, la fonction \( g\) peut être une constante parce que les constantes sont intégrables sur un espace de probabilité.
\end{remark}

\begin{corollary}       \label{CorCvAbsNormwEZdRc}
    Soit \( (a_i)_{i\in \eN}\) une suite numérique absolument convergente. Alors elle est convergente. Il en est de même pour les séries de fonctions si on considère la convergence ponctuelle.
\end{corollary}

\begin{proof}
    L'hypothèse est la convergence de l'intégrale \( \int_{\eN}| a_i |dm(i)\) où \( dm\) est la mesure de comptage. Étant donné que \( | a_i |\leq | a_i |\), la fonction \( a_i\) (fonction de \( i\)) peut jouer le rôle de \( g\) dans le théorème de la convergence dominée de Lebesgue (théorème \ref{ThoConvDomLebVdhsTf}).
\end{proof}
Nous utiliseront ce résultat pour montrer que la transformée de Fourier d'une fonction \( L^1(\eR^d)\) est continue (proposition \ref{PropJvNfj}).

\begin{proposition}[\cite{YHRSDGc}] \label{PropUXjnwLf}
    \begin{enumerate}
        \item
            Une fonction mesurable et positive est limite (simple) d'une suite croissante de fonctions étagées, mesurables et positives.
        \item
            Si \( f\colon \eR^d\to \bar \eR\) est mesurable, alors elle est limite (simple) de fonctions étagées \( f_n\) telles que \( | f_n |\leq | f |\).
    \end{enumerate}
\end{proposition}
%TODO : la preuve est dans le document cité.

%+++++++++++++++++++++++++++++++++++++++++++++++++++++++++++++++++++++++++++++++++++++++++++++++++++++++++++++++++++++++++++ 
\section{Deux gros théorèmes : Fubini et changement de variable}
%+++++++++++++++++++++++++++++++++++++++++++++++++++++++++++++++++++++++++++++++++++++++++++++++++++++++++++++++++++++++++++

%---------------------------------------------------------------------------------------------------------------------------
\subsection{Mesure produit}
%---------------------------------------------------------------------------------------------------------------------------

\begin{theorem}[\cite{NBoIEXO}\footnote{Modèle non contractuel : des notations et la définition de \( \lambda\)-système peuvent varier entre la référence et le présent texte.}]    \label{ThoCCIsLhO}
    Soient \( (\Omega_i,\tribA_i,\mu_i)\) (\( i=1,2\)) deux espaces mesurés \( \sigma\)-finie. Soit \( A\in\tribA_1\otimes \tribA_2\). Alors les fonctions\footnote{Voir la notation du lemme \ref{subEqCTtPccK}.}
    \begin{subequations}
        \begin{align}
            x\mapsto\mu_2\big( A_2(x) \big)\\
            y\mapsto\mu_1\big( A_1(y) \big)
        \end{align}
    \end{subequations}
    sont mesurables et
    \begin{equation}    \label{EqRKXwsQJ}
        \int_{\Omega_1}\mu_2\big( A_2(x) \big)d\mu_1(x)=\int_{\Omega_2}\mu_2\big( A_1(y) \big)d\mu_2(y).
    \end{equation}
\end{theorem}

\begin{proof}
    Nous supposons d'abord que \( \mu_1\) et \( \mu_2\) sont finies et nous notons \( \tribD\) le sous-ensemble de \( \tribA_1\otimes \tribA_2\) sur lequel le théorème est correct. Nous allons commencer par prouver que \( \tribD\) est un \( \lambda\)-système.

    \begin{subproof}
        \item[\( \lambda\)-système : différence ensembliste]
            Soient \( A,B\in\tribD\) avec \( A\subset B\). Nous avons
            \begin{subequations}
                \begin{align}
                    (B\setminus A)_1(y)&=\{ x\in \Omega_1\tq(x,y)\in B\setminus A \}\\
                    &=\{ x\in \Omega_1\tq(x,y)\in B\}\setminus\{ x\in \Omega_1\tq(x,y)\in  A \}\\
                    &=B_1(y)\setminus A_1(y).
                \end{align}
            \end{subequations}
            Vu que \( A_1(y)\subset B_1(y)\) et que les mesure sont finies le lemme \ref{LemPMprYuC} nous donne
            \begin{equation}
                \mu_1\big( (B\setminus A)_1(y) \big)=\mu_1\big( B_1(y) \big)-\mu_1\big( A_1(y) \big),
            \end{equation}
            et similairement pour \( 1\leftrightarrow 2\). Les deux fonctions (de \( y\)) à droite étant mesurables, nous avons la mesurabilité de la fonction \( y\mapsto \mu_1\big( (B\setminus A)_1(y) \big)\).

            Prouvons la formule intégrale en nous rappelant que la formule \eqref{EqRKXwsQJ} est supposée correcte pour \( A\) et \( B\) séparément :
            \begin{subequations}
                \begin{align}
                    \int_{\Omega_2}\mu_1\big( (B\setminus A)_1(y) \big)d\mu_2(y)&=\int_{\Omega_2}\mu_1\big( B_1(y) \big)d\mu_2(y)-\int_{\Omega_2}\mu_1\big( A_1(y) \big)d\mu_2(y)\\
                    &=\int_{\Omega_1}\mu_2\big( B_2(x) \big)d\mu_1(x)-\int_{\Omega_1}\mu_2\big( A_2(x) \big)d\mu_1(x)\\
                    &=\int_{\Omega_1}\mu_2\big( (B\setminus A)_2(x) \big)d\mu_1(x).
                \end{align}
            \end{subequations}
            
    
        \item[\( \lambda\)-système : limite de suite croissante]

            Soit \( (A_n)\) une suite croissante dans \( \tribD\); nous posons \( B_n=A_n\setminus A_{n-1}\) et \( A_0=\emptyset\) de telle sorte à travailler avec une suite d'ensembles disjoints qui satisfait \( \bigcup_nA_n=\bigcup_nB_n\). Vu que la suite est croissante nous avons \( A_{n-1}\subset A_n\) et donc \( B_n\in\tribD\) par le point déjà fait sur la différence ensembliste. Nous avons :
            \begin{subequations}
                \begin{align}
                    \mu_1\big( (\bigcup_nB_n)_1(y) \big)&=\{ x\in \Omega_1\tq (x,y)\in\bigcup_nB_n \}\\
                    &=\bigcup_n\{ x\in\Omega_1\tq (x,y)\in B_n \}\\
                    &=\bigcup_n (B_n)_1(y).
                \end{align}
            \end{subequations}
            Par conséquent, par la propriété \ref{ItemQFjtOjXiii} d'une mesure nous avons
            \begin{equation}
                \mu_1\big( (\bigcup_nB_n)_1(y) \big)=\sum_n\mu_1\big( (B_n)_1(y) \big).
            \end{equation}
            En tant que somme de fonctions positives et mesurables, la fonction
            \begin{equation}
                y\mapsto\sum_n\mu_1\big( (B_n)_1(y) \big)
            \end{equation}
            est mesurable par la proposition \ref{PropFYPEOIJ}. Il faut encore vérifier la formule intégrale. Le gros du boulot est de permuter une somme et une intégrale par le corollaire \ref{CorNKXwhdz} :
            \begin{subequations}
                \begin{align}
                    \int_{\Omega_2}\sum_n\mu_1\big( (B_n)_1(y) \big)d\mu_2(y)&=\sum_n\int_{\Omega_2}\mu_1\big( (B_n)_1(y) \big)d\mu_2(y)\\
                    &=\sum_n\int_{\Omega_1}\mu_2\big( (B_n)_2(x) \big)d\mu_1(x)\\
                    &=\int_{\Omega_1}\sum_n\mu_2\big( (B_n)_2(x) \big)d\mu_1(x)\\
                    &=\int_{\Omega_1}\mu_2\big( (\bigcup_nB_n)_1(y) \big)d\mu_1(x).
                \end{align}
            \end{subequations}
    \end{subproof}
    Maintenant que \( \tribD\) est un $\lambda$-système contenant les rectangles, le lemme \ref{LemLUmopaZ} dit que la tribu engendrée par \( \tribD\) (c'est à dire \( \tribA_1\otimes \tribA_2\)) est le $\lambda$-système \( \tribD\) lui-même.

    La preuve est finie dans le cas de mesures finies. Nous commençons maintenant à prouver dans le cas où les mesures \( \mu_1\) et \( \mu_2\) sont seulement \( \sigma\)-finies. Nous considérons des suites croissantes \( \Omega_{i,n}\to\Omega_i\) d'ensembles mesurables et de mesure finie : \( \mu_i(\Omega_{i,n})<\infty\). D'abord remarquons que
    \begin{equation}\label{EqNFuBzBF}
        \mu_2\Big( (A\cap \Omega_{1,j}\times E_{2,j})_2(x) \Big)=\mu_2\Big( A_2(x)\cap \Omega_{2,j} \Big)\mtu_{\Omega_{1,j}}.
    \end{equation}
    En effet,
    \begin{subequations}
        \begin{align}
            \heartsuit&=(A\cap\Omega_{1,j}\times E_{2,j})_2(x)\\
            &=\{ y\in\Omega_2\tq (x,y)\in A\cap \Omega_{1,j}\times E_{2,j} \}\\
            &=\{ y\in \Omega_2\tq (x,y)\in A\times \Omega_{2,j} \}\cap\{ y\in\Omega_2\tq (x,y)\in \Omega_{1,j}\times \Omega_{2,j} \}.
        \end{align}
    \end{subequations}
    Si \( y\in \Omega_{1,j}\) alors \( \{ y\in \Omega_2\tq (x,y)\in \Omega_{1,j}\times \Omega_{2,j} \}=\Omega_{2,j}\) et dans ce cas
    \begin{equation}
        \heartsuit=\{ y\in \Omega_2\tq (x,y)\in A\times \Omega_{2,j} \}\cap \Omega_{2,j}=A_2(x)\cap E_{2,j}.
    \end{equation}
    Et inversement, si \( x\notin \Omega_{1,j}\) alors \( \heartsuit=\emptyset\). Dans les deux cas nous avons \eqref{EqNFuBzBF}.

    Les ensembles \( A\cap \Omega_{1,j}\times \Omega_{2,j}\) étant de mesure finie, nous pouvons leur appliquer la première partie :
    \begin{equation}
        \int_{\Omega_1}\mu_2\Big( (A\cap\Omega_{1,j}\times \Omega_{2,j})_2(x) \Big)d\mu_1(x)=\int_{\Omega_2}\mu_1\Big( (A\cap\Omega_{1,j}\times \Omega_{2,j})_1(y) \Big)d\mu_2(u),
    \end{equation}
    ou encore
    \begin{equation}
        \int_{\Omega_1}\mu_2\Big( A_2(x)\cap \Omega_{2,j} \Big)\mtu_{\Omega_{1,j}}(x)d\mu_1(x)=\int_{\Omega_2}\mu_1\Big( A_1(y)\cap \Omega_{1,j} \Big)\mtu_{\Omega_{2,j}}(y)d\mu_2(y).
    \end{equation}
    Ce que nous avons dans ces intégrales sont (par rapport à \( j\)) des suites croissantes de fonction positives; nous pouvons donc permuter une limite et une intégrale. En sachant que si \( k\to \infty\), alors
    \begin{subequations}
        \begin{align}
            \mtu_{1,j}(x)\to 1\\
            \mu_2\big( A_2(x)\cap \Omega_2,j \big)\to\mu_2\big( A_2(x) \big),
        \end{align}
    \end{subequations}
    nous trouvons le résultat demandé.
\end{proof}

\begin{theorem}[\cite{FubiniBMauray,MesIntProbb}]   \label{ThoWWAjXzi}
    Soient \( \mu_i\) des mesures $\sigma$-finies sur \( (\Omega_i,\tribA_i)\) (\( i=1,2\)). Il existe une et une seule mesure, notée \( \mu_1\otimes \mu_2\), sur \( (\Omega_1\times\Omega_2,\tribA_1\otimes\tribA_2)\) telle que
    \begin{equation}    \label{EqOIuWLQU}
        (\mu_1\otimes\mu_2)(A_1\times A_2)=\mu_1(A_1)\mu_2(A_2)
    \end{equation}
    pour tout \( A_1\in \tribA_1\) et \( A_2\in\tribA_2\). Cette mesure est donnée par la formule\footnote{Voir les notations du lemme \ref{LemAQmWEmN}.}
    \begin{equation}   \label{EqDFxuGtH}
        (\mu_1\otimes \mu_2)(A)=\int_{\Omega_1}\mu_2\big( A_2(x) \big)d\mu_1(x)=\int_{\Omega_2}\mu_1\big( A_1(y) \big)d\mu_2(y).
    \end{equation}
    Cette mesure est la \defe{mesure produit}{mesure!produit} de \( \mu_1\) par \( \mu_2\).
\end{theorem}
\index{mesure!produit}

\begin{proof}
    L'ensemble des rectangles de \( \Omega_1\times \Omega_2\) engendre la tribu \( \tribA_1\otimes\tribA_2\), est fermé par intersection et contient une suite croissante d'ensembles \( P_n\times R_n\) de mesure finie (\( \mu(P_n\times R_n)<\infty\)) telle que \( P_n\times R_n\to \Omega_1\times \Omega_2\). Cette suite est donné par le fait que \( \mu_1\) et \( \mu_2\) sont \( \sigma\)-finies. En effet si \( (X_n)\) et \( (Y_n)\) sont des recouvrements dénombrables de \( \Omega_1\) et \( \Omega_2\) par des ensembles de mesure finie, en posant \( P_n=\bigcup_{k=1}^nX_n\) et \( R_n=\bigcup_{k=1}^nY_n\) nous avons bien une suite croissante de rectangles qui tendent vers \( \Omega_1\times \Omega_2\). Avec ces rectangles en main, le théorème \ref{ThoJDYlsXu} donne l'unicité.

    Nous passons à l'existence de la mesure. Le théorème \ref{ThoCCIsLhO} dit que ces formules ont un sens et sont correctes. Il suffit donc de prouver que dans le cas des rectangles, ces formules se réduisent à \eqref{EqOIuWLQU}. Soit donc \( A=X_1\times X_2\) avec \( X_i\in\tribA_i\). Alors
    \begin{equation}
        A_1(y)=\{ x\in\Omega_1\tq (x,y)\in X_1\times X_2 \}
    \end{equation}
    et
    \begin{equation}
        \mu_1\big( A_1(y) \big)=\mtu_{X_2}(y)\mu_1(X_1),
    \end{equation}
    donc
    \begin{subequations}
        \begin{align}
            (\mu_1\otimes\mu_2)(A)&=\int_{\Omega_2}\mu_1\big( A_1(y) \big)d\mu_2(y)\\
            &=\int_{\Omega_2}\mu_1(X_1)\mtu_{X_2}(y)d\mu_2(y)\\
            &=\mu_1(X_1)\int_{\Omega_2}\mtu_{X_2}(y)d\mu_2(y)\\
            &=\mu_1(X_1)\mu_2(X_2).
        \end{align}
    \end{subequations}
    Pour cela nous avons utilisé le fait que l'intégrale de la fonction caractéristique d'un ensemble mesurable est la mesure de cet ensemble.
\end{proof}

\begin{definition}[Produit d'espaces mesurés]  \label{DefUMlBCAO}
    Si \( (\Omega_i,\tribA_i,\mu_i)\) sont deux espaces mesurés, l'\defe{espace produit}{produit!espaces mesurés} est l'ensemble \( \Omega_1\times \Omega_2\) muni de la tribu produit \( \tribA_1\otimes \tribA_2\) de la définition \ref{DefTribProfGfYTuR} et de la mesure produit \( \mu_1\otimes \mu_2\) définie par le théorème \ref{ThoWWAjXzi}.
\end{definition}

%--------------------------------------------------------------------------------------------------------------------------- 
\subsection{Théorème de Fubini-Tonelli et de Fubini}
%---------------------------------------------------------------------------------------------------------------------------

Il existe plusieurs résultats similaires. 
\begin{itemize}
    \item
        le théorème de Fubini-Tonelli \ref{ThoWTMSthY} demande que la fonction soit mesurable et positive;
    \item
        le théorème de Fubini \ref{ThoFubinioYLtPI} demande que la fonction soit intégrable (mais pas spécialement positive);
    \item
        le corollaire \ref{CorTKZKwP} demande l'intégrabilité de la valeur absolue des intégrales partielles pour déduire que la fonction elle-même est intégrable.
\end{itemize}

%TODO : des démonstrations de ces trois théorèmes seraient les bienvenues.

Nous rappelons que \( \eR^n\) muni de la mesure de Lebesgue est un espace mesuré \( \sigma\)-fini, conformément à la définition \ref{DefBTsgznn}.

\begin{theorem}[Fubini-Tonelli\cite{NBoIEXO}]\label{ThoWTMSthY}
    Soient \( (\Omega_i,\tribA_i,\mu_i)\) deux espaces mesurés \( \sigma\)-finis, et \( (\Omega,\tribA,\mu)\) l'espace produit. Soit une fonction \( f\colon \Omega_1\times \Omega_2\to \eR\) une fonction mesurable et positive (valant éventuellement \( \infty\) à certains endroits)
    Alors :
    \begin{enumerate}
        \item
            Les fonction
            \begin{equation}
                F_1\colon x\mapsto \int_{\Omega_2}f(x,y)d\mu_2(y)
            \end{equation}
            et
            \begin{equation}
                F_2\colon y\mapsto \int_{\Omega_1}f(x,y)d\mu_1(x)
            \end{equation}
            sont mesurables.
        \item
            Toutes les intégrales imaginables existent et sont égales :
            \begin{subequations}    \label{EqJRVtOGx}
                \begin{align}
                    \iint_{\Omega_1\times \Omega_2}f(x,y)d(\mu_1\otimes \mu_2)(x,y)&=\int_{\Omega_1}\left[ \int_{\Omega_2}f(x,y)d\mu_2(y) \right]d\mu_1(x)\\
                &=\int_{\Omega_2}\left[ \int_{\Omega_1}f(x,y)d\mu_1(x) \right]d\mu_2(y).
                \end{align}
            \end{subequations}
    \end{enumerate}
\end{theorem}
\index{théorème!Fubini-Tonelli}

\begin{proof}
    Commençons par prouver le théorème dans le cas d'une fonction caractéristique d'un ensemble mesurable : \( f(x,y)=\mtu_{A}(x,y)\) pour un certain ensemble \( A\subset \Omega_1\times \Omega_2\). Dans ce cas,
    \begin{equation}
        F_1(x)=\int_{\Omega_2}\mtu_A(x,y)d\mu_2(y)=\int_{\omega_2}\mtu_{A_1(y)}(x)d\mu_2(y)=\mu_2\big( A_1(x) \big),
    \end{equation}
    et nous avons déjà vu au théorème \ref{ThoCCIsLhO} que cette fonction \( F_1\) était alors mesurable. En utilisant maintenant les égalités \eqref{EqDFxuGtH} ainsi que le fait que \( \mtu_A(x,y)=\mtu_{A_2(x)}(y)\) nous avons
    \begin{subequations}
        \begin{align}
            \iint_{\Omega_1\times \Omega_2}\mtu_A(x,y)d(\mu_1\otimes \mu_2)(x,y)&=(\mu_1\otimes \mu_2)(A)\\
            &=\int_{\Omega_1}\mu_2\big( A_2(x) \big)d\mu_1(x)\\
            &=\int_{\Omega_1}\left[   \int_{\Omega_2}\mtu_{A_2(x)}(y)d\mu_2(y)  \right]d\mu_1(x)\\
            &=\int_{\Omega_1}\left[ \int_{\Omega_2}\mtu_A(x,y)d\mu_2(y) \right]d\mu_1(x).
        \end{align}
    \end{subequations}
    Le théorème étant valable pour les fonctions caractéristiques, il est valable pour les fonctions simples (définition \ref{DefBPCxdel}) par linéarité de l'intégrale.

    Si \( f\) n'est pas une fonction simple, alors la proposition \ref{PropWBavIf} nous donne une suite croissante de fonctions simples et positives convergeant ponctuellement vers \( f\). La partie du théorème sur les fonctions simples dit que pour chaque \( n\) l'intégrale
    \begin{equation}
        \iint_{\Omega_1\times \Omega_2}f_n(x,y)d(\mu_1\otimes\mu_2)(x,y)
    \end{equation}
    peut être décomposée comme il faut en suivant la formule \eqref{EqJRVtOGx}. Il faut pouvoir permuter la limite et l'intégrale dans chacun de cas. D'abord le théorème de la convergence monotone \ref{ThoConvMonFtBoVh} appliqué à l'espace \( \Omega_1\times \Omega_2\) dit que
    \begin{equation}
        \lim_{n\to \infty} \iint_{\Omega_1\times \Omega_2}f_n(x,y)d(\mu_1\otimes \mu_2)(x,y)= \iint_{\Omega_1\times \Omega_2}f(x,y)d(\mu_1\otimes \mu_2)(x,y).
    \end{equation}
    Ensuite, pour chaque \( x\in\Omega_1\), les fonctions
    \begin{equation}
        \sigma_n(y)=\int_{\Omega_1}f_n(x,y)d\mu_1(x)
    \end{equation}
    forment une suite croissante de fonctions mesurables; nous leur appliquons encore le théorème de la convergence monotone :
    \begin{subequations}
        \begin{align}
            \lim_{n\to \infty} \int_{\Omega_2}\left[ \int_{\Omega_1}f_n(x,y)d\mu_1(x) \right]d\mu_2(y)&=\lim_{n\to \infty} \int_{\Omega_2}\sigma_n(y)d\mu_2(y)\\
            &=\int_{\Omega_2}\left[\lim_{n\to \infty} \int_{\Omega_1}f_n(x,y)d\mu_1(x)\right]d\mu_2(y)\\
            &=\int_{\Omega_2}\left[ \int_{\Omega_1}f(x,y)d\mu_1(x) \right]d\mu_2(y)
        \end{align}
    \end{subequations}
    où nous avons utilisé une seconde fois Beppo-Levi.
\end{proof}

\begin{theorem}[Fubini\cite{MesIntProbb}]\label{ThoFubinioYLtPI}
    Soient \( (\Omega_i,\tribA_i,\mu_i)\) deux espaces mesurés \( \sigma\)-finis, et \( (\Omega,\tribA,\mu)\) l'espace produit. Soit 
    \begin{equation}
        f\in L^1\big( (\Omega,\tribA),\eR \big),
    \end{equation}
    c'est à dire une fonction à valeurs réelles mesurable et intégrable sur \( \Omega\). Alors :
    \begin{enumerate}
        \item
            Pour presque tout \( x\in \Omega_1\), la fonction \( y\mapsto f(x,y)\) est \( L^1(\Omega_2)\).
        \item
            Si nous posons
            \begin{equation}
                \varphi_f(x)=\int_{\Omega_2}f(x,y)d\mu_2(y);
            \end{equation}
            alors \( \varphi_f\in L^1(\Omega_1)\).
        \item   \label{ItemQMWiolgiii}
            Nous avons la formule d'inversion d'intégrale
            \begin{subequations}
                \begin{align}
                \int_{\Omega}fd(\mu_1\otimes \mu_2)&=\int_{\Omega_1}\varphi_fd\mu_1\\
                &=\int_{\Omega_1}\left[ \int_{\Omega_2}f(x,y)d\mu_2(y) \right]d\mu_1(x)\\
                &=\int_{\Omega_2}\left[ \int_{\Omega_1}f(x,y)d\mu_1(x) \right]d\mu_2(y).
                \end{align}
            \end{subequations}
    \end{enumerate}
\end{theorem}
\index{théorème!Fubini!espace mesuré}

Si la fonction \( (x,y)\mapsto f(x)g(y)\) satisfait aux hypothèse du théorème de Fubini alors
\begin{equation}    \label{EqTJEEsJW}
    \int_{\Omega_1\times \Omega_2} f(x)g(y)dx\otimes dy=\left( \int_{\Omega_1}f(x)dx \right)\left( \int_{\Omega_2}g(y)dy \right).
\end{equation}
Le théorème de Fubini est souvent utilisé sous cette forme.

\begin{corollary}\label{CorTKZKwP}
    Soient \( (\Omega_i,\tribA_i,\mu_i)\) deux espaces mesurés \( \sigma\)-finis, et \( (\Omega,\tribA,\mu)\) l'espace produit\footnote{Définition \ref{DefUMlBCAO}.}. Soit une fonction mesurable \( f\colon \Omega\to \eR\). Alors les conditions suivantes sont équivalentes
    \begin{enumerate}
        \item
            \( f\in L^1(\Omega)\),
        \item
            \begin{equation}
                \int_{\Omega_1}\left[ \int_{\Omega_2}| f |d\mu_2 \right]d\mu_1 <\infty,
            \end{equation}
        \item
            \begin{equation}
                \int_{\Omega_2}\left[ \int_{\Omega_1}| f |d\mu_1 \right]d\mu_2 <\infty.
            \end{equation}
    \end{enumerate}
\end{corollary}
En pratique, lorsqu'on ne sait pas a priori si \( f\) est intégrable sur \( \Omega_1\times \Omega_2\), nous testons l'intégrabilité en chaine de \( | f |\), et si c'est bon, alors nous savons que \( f\) est intégrable sur le produit et qu'on peut permuter les intégrales.

\begin{example}
    Nous montrons que le théorème ne tient pas si une des deux mesures n'est pas \( \sigma\)-finie. Soit \( I=\mathopen[ 0 , 1 \mathclose]\). Nous considérons l'espace mesuré
    \begin{equation}
        (I,\Borelien(I),\lambda)
    \end{equation}
    où \( \Borelien(I)\) est la tribu des boréliens sur \( I\) et \( \lambda\) est la mesure de Lebesgue (qui est $\sigma$-finie). D'autre part nous considérons l'espace mesuré
    \begin{equation}
        (I,\partP(I),m)
    \end{equation}
    où \( \partP(I)\) est l'ensemble des parties de \( I\) et \( m\) est la mesure de comptage. Cette dernière n'est pas $\sigma$-finie parce que les seuls ensembles de mesure finie pour la mesure de comptage sont des ensembles finis, or une union dénombrable d'ensemble finis ne peut pas recouvrir l'intervalle \( I\).

    Nous allons montrer que dans ce cadre, l'intégrale de la fonction indicatrice de la diagonale sur \( I^2\) ne vérifie pas le théorème de Fubini. Étant donné que \( \Borelien(I)\subset\partP(I)\) nous avons
    \begin{equation}
        \Borelien(I^2)\subset\Borelien(I)\otimes\partP(I).
    \end{equation}
    Soit \( \Delta=\{ (x,x)\tq x\in I \}\). La fonction
    \begin{equation}
        \begin{aligned}
            g\colon I^2&\to \eR \\
            (x,y)&\mapsto x-y 
        \end{aligned}
    \end{equation}
    est continue et \( \Delta=g^{-1}(\{ 0 \})\) est donc fermé dans \( I^2\). L'ensemble \( \Delta\) est donc un borélien de \( I^2\) et par conséquent un élément de la tribu \( \Borelien(I)\otimes\partP(I)\). La fonction indicatrice \( \mtu_{\Delta}\) est alors mesurable pour l'espace mesuré
    \begin{equation}
        (I\times I,\Borelien(I)\otimes\partP(I),\lambda\otimes m).
    \end{equation}
    Pour \( x\) fixé nous avons
    \begin{equation}
        \mtu_{\Delta}(x,y)=\begin{cases}
            1    &   \text{si \( y= x\)}\\
            1    &    \text{si \( y\neq x\)}
        \end{cases}=\mtu_{\{ x \}}(y),
    \end{equation}
    et donc
    \begin{subequations}
        \begin{align}
            A_1&=\int_I\left( \int_I\mtu_{\Delta}(x,y)dm(y) \right)d\lambda(x)\\
            &=\int_I\left( \int_I\mtu_{\{ x \}}(y)dm(y) \right)d\lambda(x)\\
            &=\int_I\Big( m(\{ x \}) \Big)d\lambda(x)\\
            &=\int_I 1d\lambda(x)\\
            &=1.
        \end{align}
    \end{subequations}
    Par contre le support de \( \mtu_{\Delta}\) étant de mesure nulle pour la mesure de Lebesgue, nous avons
    \begin{equation}
        \int_I\mtu_{\Delta}(x,y)d\lambda(x)=0
    \end{equation}
    et par conséquent
    \begin{equation}
        A_2=\int_I\left( \int_I\mtu_{\Delta}(x,y)d\lambda(x) \right)dm(y)=0.
    \end{equation}
    Nous voyons donc que le théorème de Fubini ne s'applique pas.
\end{example}

\begin{example} \label{ExrgMIni}
    Le théorème de Fubini est utilisé dans le calcul de l'intégrale gaussienne
    \begin{equation}
        G=\int_{\eR} e^{-x^2}dx,
    \end{equation}
    alors que la fonction \( x\mapsto  e^{-x^2}\) n'a pas de primitives parmi les fonctions élémentaires.

    Par symétrie nous pouvons nous contenter de calculer
    \begin{equation}
        G_+=\int_0^{\infty} e^{-x^2}dx.
    \end{equation}
    L'astuce est de passer par l'intermédiaire
    \begin{subequations}
        \begin{align}
            H&=\int_{\eR^+\times\eR^+} e^{-(x^2+y^2)}dxdy       \label{EqIntFausasub}\\
            &=\int_{\eR^+}\left( \int_{\eR^+} e^{-x^2} e^{-y^2}dx \right)dy\\
            &=\left( \int_{\eR^+} e^{-x^2} dx\right)^2\\
            &=G_+^2
        \end{align}
    \end{subequations}
    L'intégrale \eqref{EqIntFausasub} se calcule en passant aux coordonnées polaires et le résultat est \( H=\frac{ \pi }{ 4 }\). Nous avons alors \( G=\frac{ \sqrt{\pi} }{ 2 }\) et
    \begin{equation}
        \int_{\eR} e^{-x^2}=\sqrt{\pi}.
    \end{equation}
\end{example}

\begin{example}
    Une variante, qui n'applique pas Fubini sur un domaine non borné. Nous commençons par écrire
\begin{equation}
	I=\int_{-\infty}^{+\infty} e^{-x^2} dx := \lim_{R \to +\infty} \int_{-R}^{+R} e^{-x^2} dx 
\end{equation}
et puis nous faisons le calcul
\begin{equation}		\label{EqCalculInteeemoisxcar}
	\begin{aligned}[]
		I^2 &= \lim_{R \to +\infty} \left( (\int_{-R}^{+R} e^{-x^2} dx)( \int_{-R}^{+R} e^{-y^2} dy) \right) \\
		&= \lim_{R \to +\infty} \left( \iint_{K_R}e^{-(x^2+y^2)} dx dy \right) \\
		&= \lim_{R \to +\infty} \left( \iint_{C_R}e^{-(x^2+y^2)} dx dy \right) 
	\end{aligned}
\end{equation}
où $K$ est le carré de demi côté $R$ centré à l'origine et de côtés parallèles aux axes et $C_R$ est le cercle de rayon $R$ centré à l'origine.

	La première étape à justifier est simplement l'application de Fubini. Pour le passage de l'intégrale du carré vers le cercle, définissons
	\begin{equation}
		\begin{aligned}[]
			I_K(r)&=\int_{K_r}f,&I_C(r)&=\int_{C_r}f
		\end{aligned}
	\end{equation}
	où $K_r$ est la carré de demi côté $r$ et $C_r$ est le cercle de rayon $r$. Le demi côté du carré inscrit à $C_r$ est $\sqrt{2}$, donc pour tout $r$ nous avons
	\begin{equation}
		I_K(\sqrt{2}r)\leq I_C(r)<I_K(r),
	\end{equation}
	et en prenant la limite, nous avons évidement
	\begin{equation}
		\lim_{r\to \infty}I_K(\sqrt{2}r)=\lim_{r\to\infty}I_K(r),
	\end{equation}
	de telle façon à ce que cette limite soit également égale à $\lim_{r\to\infty}I_C(t)$.


    Il ne reste qu'à calculer la dernière intégrale sur le cercle en passant aux coordonnées polaires :
	\begin{equation}
        \iint_{C_R} e^{-(x^2+y^2)}dxdy=\int_0^{2\pi}d\theta\int_0^Rr e^{-r^2}dr=\pi(1- e^{-R^2}).
	\end{equation}
	La limite donne $\pi$, nous en déduisons que
    \begin{equation}    \label{EqFDvHTg}
		\int_{-\infty}^{\infty} e^{-x^2}dx=\sqrt{\pi}.
	\end{equation}

\end{example}

\begin{example} \label{ExempInversSumIntFub}   \index{mesure!de comptage}
    Le théorème de Fubini-Tonelli nous permet également d'inverser des sommes et des séries. En effet une somme n'est rien d'autre qu'une intégrale pour la mesure de comptage :
    \begin{equation}
        \sum_{n=0}^{\infty}a_n=\int_{\eN}a_ndm(n).
    \end{equation}
    Considérons une suite de fonctions \( f_n\colon \eR^d\to \eR\) \emph{positives}, la quantité
    \begin{equation}    \label{EqAcalculParFubIntSum}
        I=\sum_{n=0}^{\infty}\int_{\eR^n}f_n(x)dx
    \end{equation}
    et les espaces mesurés \( (\eN,\partP(\eN),m)\), \( (\eR^n,\Borelien(\eR^n),\lambda)\) où \( \lambda\) est la mesure de Lebesgue. En écrivant la formule \eqref{EqAcalculParFubIntSum}, nous supposons que pour chaque \( n\), la fonction \( f_n\) est intégrable sur \( \eR^d\) et que le résultat soit sommable. Nous pouvons la récrire sous la forme
    \begin{equation}
        \int_{\eN}\left( \int_{\eR^n}f(n,x)dx \right)dm(n)
    \end{equation}
    avec la notation évidente \( f(n,x)=f_n(x)\). Prouvons que la fonction \( f\colon \eN\times\eR^d\to \eR\) ainsi définie est une fonction mesurable pour l'espace mesuré
    \begin{equation}
        \big( \eN\times\eR^d,\partP(\eN)\otimes\Borelien(\eR^d),m\otimes\lambda \big).
    \end{equation}
    Si \( A\subset\eR\), nous avons
    \begin{equation}
        f^{-1}(A)=\bigcup_{n\in\eN}\{ n \}\times f_n^{-1}(A).
    \end{equation}
    Chacun des ensembles dans l'union appartient à la tribu \( \partP(\eN)\times\Borelien(\eR^d)\) tandis que les tribus sont stables sous les unions dénombrables. La fonction \( f\) est donc mesurable. Comme nous avons supposé que \( f\) était positive, le théorème de Fubini-Tonelli s'applique et nous avons
    \begin{equation}
        I=\int_{\eR^d}\left( \int_{\eN}f(n,x)dm(n) \right)dx=\int_{\eR^d}\sum_{n\in \eN}f_n(x)dx.
    \end{equation}
\end{example}

%---------------------------------------------------------------------------------------------------------------------------
\subsection{Changement de variables dans une intégrale}
%---------------------------------------------------------------------------------------------------------------------------

\begin{theorem} \label{ThomFeRCi}
    Soit \( \mO\) un ouvert de \( \eR^n\) et \( \mO'\) un ouvert de \( \eR^m\). Soit \( \varphi\colon \mO\to \mO'\) un difféomorphisme \( C^1\). Si \( f\colon \mO\to \eR\) est une fonction mesurable, positive et intégrable, alors
    \begin{equation}
        \int_{\mO}f(u)du=\int_{\mO'}f\big( \varphi^{-1}(v) \big)| J_{\varphi^{-1}}(v) |dv.
    \end{equation}
\end{theorem}
