% This is part of Mes notes de mathématique
% Copyright (c) 2011-2014
%   Laurent Claessens
% See the file fdl-1.3.txt for copying conditions.

%+++++++++++++++++++++++++++++++++++++++++++++++++++++++++++++++++++++++++++++++++++++++++++++++++++++++++++++++++++++++++++
\section{Théorie de la mesure}
%+++++++++++++++++++++++++++++++++++++++++++++++++++++++++++++++++++++++++++++++++++++++++++++++++++++++++++++++++++++++++++

%---------------------------------------------------------------------------------------------------------------------------
\subsection{Espaces mesurables et mesurés}
%---------------------------------------------------------------------------------------------------------------------------

\begin{definition}[\cite{ProbaDanielLi}]  \label{DefjRsGSy}
    Si \( \Omega\) est un ensemble, un ensemble \( \tribA\) de sous-ensembles de \( \Omega\) est une \defe{tribu}{tribu} si 
    \begin{enumerate}
        \item
            \( \Omega\in\tribA\);
        \item
            \( \complement A\in A\) pour tout \( A\in\tribA\);
        \item
            si \( (A_i)_{i\in I}\) est un ensemble au plus dénombrable d'éléments de \( \tribA\), alors \( \sup_{n\geq 1}A_n=\bigcup_{i\in I}A_i\in\tribA\).
    \end{enumerate}
    Le couple \( (\Omega,\tribA)\) est alors un \defe{espace mesuré}{espace!mesuré}.
\end{definition}

\begin{lemma}   \label{LemBWNlKfA}
    Opérations ensemblistes sur les tribus.
    \begin{enumerate}
        \item
    Une tribu est stable par intersections au plus dénombrables.
\item
    Une tribu est stable par différence ensembliste.
    \end{enumerate}
\end{lemma}

\begin{proof}
    Soit \( (A_i)_{i\in I}\) une famille au plus dénombrable d'éléments de la tribu \( \tribA\). Nous devons prouver que \( \bigcap_{i\in I}A_i\) est également un élément de \( \tribA\). Pour cela nous passons au complémentaire :
    \begin{equation}
        \complement\left( \bigcap_{i\in I}A_i \right)=\bigcup_{i\in I}\complement A_i.
    \end{equation}
    La définition d'une tribu implique que le membre de droite est un élément de la tribu. Par stabilité d'une tribu par complémentaire, l'ensemble \( \bigcap_{i\in I}A_i\) est également un élément de la tribu.

    La seconde assertion est immédiate à partir de la première parce que \( A\setminus B=A\cap \complement B\).
\end{proof}

\begin{definition}
    La tribu des \defe{boréliens}{boréliens}, notée \( \Borelien(\eR^d)\) est la tribu engendrée par les ouverts de \( \eR^d\). 
\end{definition}
C'est la tribu que nous utiliserons (presque) tout le temps sur \( \eR^d\).

\begin{definition}
    Soit \( (E,\tribA)\) et \( (F,\tribF)\) deux espaces mesurés. Une fonction \( f\colon E\to F\) est \defe{mesurable}{mesurable!fonction} si pour tout \( \mO\in \tribF\), l'ensemble \( f^{-1}(\mO)\) est dans \( \tribA\).

    Une fonction à valeurs dans \( \eR^d\) est \defe{borélienne}{borélienne!fonction}\index{fonction!borélienne} si elle est mesurable pour la tribu des boréliens sur \( \eR^d\). Plus explicitement, \( f\colon (\Omega,\tribA)\to (\eR^d,\Borelien(\eR^d))\) est borélienne si pour tout \( \mO\in\Borelien\) nous avons \( f^{-1}(\mO)\in\tribA\).
\end{definition}
Si \( \tribA\) est une tribu sur un ensemble \( E\), nous notons \( m(\tribA)\)\nomenclature[P]{\( m(\tribA)\)}{Ensemble des fonctions \( \tribA\)-mesurables} l'ensemble des fonctions qui sont \( \tribA\)-mesurables.

\begin{remark}
    Le plus souvent les fonctions que nous considérons sont à valeurs réelles. La tribu considérée sur \( \eR\) est presque toujours celle des ensembles mesurables au sens de Lebesgue. Étant donné qu'il est franchement difficile de créer des ensembles non mesurables au sens de Lebesgue, il est franchement difficile de créer des fonction non mesurables à valeurs réelles. L'hypothèse de mesurabilité est donc toujours satisfaite dans les cas pratiques.

    Cependant en probabilités, la tribu considérée sur \( \eR^n\) pour les variables aléatoires est celle des boréliens.
\end{remark}

\begin{definition}  \label{DefBTsgznn}
    Une \defe{\wikipedia{en}{Measure_space}{mesure}}{mesure} sur l'espace mesurable \( (\Omega,\tribA)\) est une application \( \mu\colon \tribA\to \eR\cup\{ \infty \}\) telle que
    \begin{enumerate}
        \item
            \( \mu(A)\geq 0\) pour tout \( A\in\tribA\);
        \item
            \( \mu(\emptyset)=0\);
        \item   \label{ItemQFjtOjXiii}
            \( \mu\left( \bigcup_{i=0}^{\infty}A_i\right)=\sum_{i=0}^{\infty}\mu(A_i)\) si les \( A_i\) sont des éléments de \( \tribA\) deux à deux disjoints.
    \end{enumerate}
    Une mesure est \defe{\( \sigma\)-finie}{mesure!$\sigma$-finie} si il existe un recouvrement dénombrable de \( \Omega\) par des ensembles de mesure finie. Si la mesure est $\sigma$-finie, nous disons que l'espace \( (\Omega,\tribA,\mu)\) est un espace mesuré $\sigma$-fini.

    La mesure \( \mu\) sur \( \Omega\) est \defe{finie}{mesure!finie} si \( \mu(\Omega)<\infty\).
\end{definition}

\begin{definition}[Ensemble mesurable]\label{DefHGsQxHB}
    Les éléments de \( \tribA\) sont les ensembles \defe{mesurables}{mesurable!ensemble} pour la mesure \( \mu\).
\end{definition}

Si la mesure des \( \sigma\)-finie, nous pouvons choisir le recouvrement croissant pour l'inclusion. En effet si \( (E_n)_{n\in \eN}\) est le recouvrement, il suffit de considérer \( F_n=\bigcup_{k\leq n}E_k\). Ces ensembles \( F_n\) forment tout autant un recouvrement dénombrable, mais il est évidemment croissant.

\begin{lemma}\label{LemKKNtvee} \label{LemPMprYuC}
    Si \( A\subset B\) sont deux ensembles \( \mu\)-mesurables alors
    \begin{equation}
        \mu(B\setminus A)=\mu(B)-\mu(A)
    \end{equation}
    et en particulier
    \begin{equation}
        \mu(B)\geq \mu(A).
    \end{equation}
\end{lemma}

\begin{proof}
    Vu que les ensembles \( B\setminus A\) et \( A\) sont disjoints par la propriété \ref{ItemQFjtOjXiii} de la définition de mesure nous avons
    \begin{equation}
        \mu\big( (B\setminus A)\cup A \big)=\mu(B\setminus A)+\mu(A)
    \end{equation}
    et donc
    \begin{equation}
        \mu(B)=\mu(B\setminus A)+\mu(A)
    \end{equation}
    comme demandé.
\end{proof}

\begin{example}
    La mesure de comptage \( m\) sur \( \eN\) est \( \sigma\)-finie parce que \( E_n=\{ 0,\ldots, n \}\) est de mesure finie et \( \bigcup_{n\in \eN}E_n=\eN\).
\end{example}

\begin{example}
    La mesure de Lebesgue sur \( \eR^n\) est \( \sigma\)-finie parce que les boules de rayon \( n\) forment un ensemble dénombrable d'ensembles de mesures finies dont l'union est évidemment tout \( \eR^n\).

    L'intervalle \( I=\mathopen[ 0 , 1 \mathclose]\) muni de la tribu de toutes ses parties et de la mesure de comptage n'est pas un espace mesuré \( \sigma\)-fini.
\end{example}

\begin{example}
    L'intégration à la Riemann n'est pas dans la théorie des espaces mesurés. En effet l'ensemble 
    \begin{equation}
        \tribA=\{   A\subset\mathopen[ 0 , 1 \mathclose]\tq  \text{\( \mtu_A\) est intégrable au sens de Riemann}   \}
    \end{equation}
    n'est pas une tribu. Par exemple les singletons en font partie tandis que \( \mathopen[ 0 , 1 \mathclose]\cap \eQ\) n'en fait pas partie alors que c'est une union dénombrable de singletons.
\end{example}

\begin{definition}
    Si \( \mu\) est une mesure nous disons qu'une propriété est vraie \( \mu\)-\defe{presque partout}{presque partout} si elle est fausse seulement sur un ensemble de mesure nulle.
\end{definition}

Par exemple la fonction de Dirichlet est presque partout égale à la fonction \( 1\) (pour la mesure de Lebesgue).


\begin{definition}
    Une application entre espace mesurés
    \begin{equation}
        f\colon (\Omega,\tribA)\to (\Omega',\tribA')
    \end{equation}
    est \defe{mesurable}{mesurable!application} si pour tout \( B\in\tribA'\), l'ensemble \( f^{-1}(B)\) est dans \( \tribA\).
\end{definition}

Si \( \mu\) est une mesure sur \( \eR^d\), une fonction \( f\colon \eR^d\to \eR\) est une \defe{densité}{densité d'une mesure} si pour tout \( A\subset\eR^d\) nous avons
\begin{equation}
    \mu(A)=\int_Af(x)dx
\end{equation}
où \( dx\) est la mesure de Lebesgue.

%--------------------------------------------------------------------------------------------------------------------------- 
\subsection{Généralités}
%---------------------------------------------------------------------------------------------------------------------------

\begin{lemma}   \label{LemIDITgAy}
    Une union dénombrable d'ensemble de mesure nulle est de mesure nulle.
\end{lemma}

\begin{proof}
    C'est une conséquence immédiate du point \ref{ItemQFjtOjXiii} de la définition d'une mesure : si les \( A_i\) sont de mesure nulle,
    \begin{equation}
        \mu\left( \bigcup_{i=1}^{\infty}A_i \right)\leq \mu(A_i)=0
    \end{equation}
\end{proof}

\begin{definition}
    Si \( (A_n)\) est une suite croissante d'ensembles alors la \defe{limite}{limite!d'ensembles} est
    \begin{equation}
        \lim_nA_n=\bigcup_{i=0}^{\infty}A_i.
    \end{equation}
    Si la suite est décroissante alors la limite est
    \begin{equation}
        \lim_nA_n=\bigcap_{i=0}^{\infty}A_i.
    \end{equation}
\end{definition}
\ifthenelse{\value{isAgreg}=0}{Pour une suite ni croissante ni décroissante d'ensembles, il y a la notion de limite inductive\footnote{\emph{direct limit} en anglais.} qui sera un peu traitée à la section \ref{SecDirectLimit}.}{}

\begin{proposition}[\cite{RArwFWJ}] \label{PropAFNPSsm}
    Soit \( \mu\) une mesure sur \( \Omega\) et \( (S_n)\) une suite croissante d'ensembles \( \mu\)-mesurables de \( \Omega\). Nous notons
    \begin{equation}
        S=\lim_nS_n.
    \end{equation}
    Alors pour tout ensemble mesurable\footnote{Définition \ref{DefHGsQxHB}} \( A\subset\Omega\) nous avons
    \begin{equation}
        \mu(A\cap S)=\lim_{n\to \infty} \mu(A\cap S_n).
    \end{equation}
\end{proposition}
Note : dans la référence le résultat fonctionne pour tout ensemble \( A\) (et non seulement les mesurables) parce que la définition de la mesurabilité est un peu différente.

\begin{proof}
    L'inégalité \( \lim\mu(A\cap S_n)\leq \mu(A\cap S)\) est simple à prouver. En effet pour tout \( n\) nous avons \( A\cap S_n\subset A\cap S\) et donc par le lemme \ref{LemKKNtvee} nous avons
    \begin{equation}
        \mu(A\cap S_n)\leq\mu(A\cap S).
    \end{equation}
    En passant à la limite (qui respecte les inégalités) nous avons l'inégalité.

    Nous passons à l'inégalité dans l'autre sens. D'abord si \( \mu(A\cap S_n)=\infty\) pour un certain \( n\), alors il cela vaut encore \( \infty\) pour tous les \( n\) suivants et la limite est \( \infty\) sans problèmes. Donc nous supposons que \( \mu(A\cap S_n)<\infty\) pour tout \( n\in \eN\). De plus, quitte à renommer les indices, nous pouvons supposer que \( S_0=\emptyset\).

    Un élément \( x\) est dans \( S\) si et seulement si il existe \( n\geq 0\) tel que \( x\in S_{n+1}\). En prenant le plus petit de ces \( n\) nous avons \( x\neq S_n\) (éventuellement \( n=0\)) et donc
    \begin{equation}
        S=\bigcup_{n=0}^{\infty}\big( S_{n+1}\setminus S_n \big).
    \end{equation}
    Par conséquent
    \begin{equation}
            A\cap S=A\cap\bigcup_{n=0}^{\infty}(S_{n+1}\setminus S_n)
            =\bigcup_{n=0}^{\infty}A\cap(S_{n+1}\setminus S_n)
    \end{equation}
    Étant donné que les ensembles \( A\cap(S_{n+1}\setminus S_n)\) sont disjoints,
    \begin{subequations}
        \begin{align}
            \mu(A\cap S)&=\sum_{n=0}^{\infty}\mu\big( A\cap(S_{n+1}\setminus S_n) \big)\\
            &=\sum_{n=0}^{\infty}\mu\Big( (A\cap S_{n+1})\setminus (A\cap S_n) \Big)\\
            &=\sum_{n=0}^{\infty}\big[ \mu(A\cap S_{n+1})-\mu(A\cap S_n) \big]\\
            &=\lim_{n\to \infty} \mu(A\cap S_{n+1})-\underbrace{\mu(A\cap S_0)}_{=0}\label{subeqLTvmTjO}\\
            &=\lim_{n\to \infty} \mu(A\cap S_n).
        \end{align}
    \end{subequations}
    Dans ce calcul nous avons utilisé plusieurs fois le fait que les \( S_n\) et \( A\) étaient mesurables (et la propriété de tribu qui dit que \( A\cap S_n\) est également mesurable) ainsi que le lemme \ref{LemKKNtvee}. Nous avons aussi utilisé la série télescopique dans \( \eR\) pour obtenir \eqref{subeqLTvmTjO}.
\end{proof}

\begin{definition}[\cite{PVWUyDH}]
    Soit \( E\) un ensemble. Une partie \( \tribD\) de \( E\) est un \( \lambda\)-système si
    \begin{enumerate}
        \item
            pour tout \( A,B\in\tribD\) avec \( A\subset B\) implique \( B\setminus A\in \tribD\),
        \item
            si \( (A_k)_{k\geq 1}\) est une suite croissante d'éléments de \( \tribD\) alors \( \bigcup_kA_k\in\tribD\).
    \end{enumerate}
\end{definition}
Note : une tribu est une \( \lambda\)-système.

\begin{lemma}[\cite{PVWUyDH}]
    Une intersection quelconque de \( \lambda\)-systèmes dans \( E\) est un \( \lambda\)-système dans \( E\).
\end{lemma}

\begin{proof}
    Soient \( \{ \tribD_l \}_{l\in L}\) des \( \lambda\)-systèmes indicés par un ensemble \( L\). Si \( A,B\in\bigcap_{l\in L}\tribD_l\) alors \( B\setminus A\in\tribD_l\) pour tout \( l\in L\) et donc \( A\setminus B\in\bigcap_{l\in L}\tribD_l\). De la même façon si \( (A_k)\) est une suite croissante dans \( \bigcap_{l\in L}\tribD_l\) alors pour tout \( l\in L\) nous avons \( \bigcup_kA_k\in\tribD_l\). Donc \( \bigcup_kA_k\in\bigcap_l\tribD_l\).
\end{proof}
Le lemme est ce qui permet de définir le \( \lambda\)-système \defe{engendré}{engendré!$\lambda$-système} par une classe \( \tribA\) de parties de \( E\) : c'est l'intersection de tous les \( \lambda\)-systèmes de \( E\) contenant \( \tribA\).

\begin{lemma}[\cite{PVWUyDH}]
    Soit \( \tribC\) une classe de parties de \( E\) (contenant \( E\) lui-même) qui soit stable par intersection finie. Alors le \( \lambda\)-système engendré par \( \tribC\) coïncide avec la tribu engendrée par \( \tribC\).
\end{lemma}

\begin{proof}
    Nous notons \( \tribE\) le \( \lambda\)-système engendré par \( \tribC\) et \( \tribF\) la tribu engendrée par \( \tribC\). Étant donné que \( \tribF\) est un \( \lambda\)-système nous avons \( \tribE\subset\tribF\). Pour montrer l'inclusion inverse nous allons prouver que \( \tribE\) est une tribu.

    D'abord pour \( C\in\tribC\) nous posons
    \begin{equation}
        \mG_C=\{ A\subset E\tq A\cap C\in\tribE \}.
    \end{equation}
    et pour \( F\in\tribE\),
    \begin{equation}
        \mH_F=\{ A\in\tribE\tq A\cap F\in\tribE \}.
    \end{equation}
    Nous allons montrer que \( \mG_C\) et \( \mH_F\) sont des \( \lambda\)-systèmes et que \( \mG_C=\mH_F=\tribE\).
    
    Nous commençons par \( \mG_C\). Si \( A,B\in\mG_C\) avec \( A\subset B\) alors
    \begin{equation}
        (B\setminus A)\cap C=\underbrace{(B\cap C)}_{\in\tribE}\setminus\underbrace{(A\cap C)}_{\in\tribE}.
    \end{equation}
    Vu que \( \tribE\) est un \( \lambda\)-système et que \( (A\cap C)\subset(B\cap C)\) nous avons bien \( (B\setminus A)\cap C\in\tribE\) et donc \( B\setminus A\in\mG_C\). Soit maintenant \( (A_k)\) une suite croissante dans \( \mG_C\). Nous avons
    \begin{equation}
        \big( \bigcup_{k=1}^{\infty}A_k \big)\cap C=\bigcup_{k=1}^{\infty}(A_k\cap C)
    \end{equation}
    qui est une union d'une suite croissante d'éléments de \( \tribE\). Donc \( \bigcup_{k=1}^{\infty}(A_k\cap C)\in\tribE\), ce qui signifie que \( \bigcup_{k=1}^{\infty}A_k\in\mG_C\). Cela termine la preuve du fait que \( \mG_C\) soit une \( \lambda\)-système.
\end{proof}

\begin{theorem}[\cite{PVWUyDH}]
    Soient \( \mu\) et \( \nu\), deux mesures sur \( (E,\tribA)\) et une classe \( \tribE\) de parties de \( E\) telles que
    \begin{enumerate}
        \item
            La tribu engendrée par \( \tribE\) soit \( \tribA\).
        \item
            pour tout \( A\in \tribE\), \( \mu(A)=\nu(A)\)
        \item
            si \( A,B\in\tribE\) alors \( A\cap B\in\tribE\)
        \item
            il existe une suite croissante \( (E_n)\) dans \( \tribE\) telle que \( E=\lim E_n\).
    \end{enumerate}
    Alors les mesures \( \mu\) et \( \nu\) coïncident sur \( \tribA\) en entier.
\end{theorem}

\begin{proof}
    Soit \( (E_n)\) la suite des hypothèses; nous considérons \( \mu_n\) et \( \nu_n\), les restrictions de \( \mu\) et \( \nu\) à \( E_n\), c'est à dire
    \begin{subequations}
        \begin{align}
        \mu_n(A)=\mu(A\cap E_n)\\
        \nu_n(A)=\nu(A\cap E_n).
        \end{align}
    \end{subequations}
    Par la proposition \ref{PropAFNPSsm}, vu que les \( E_n\) sont dans \( \tribE\subset\tribA\) ils sont mesurables au sens de \( \mu\) et \( \nu\). Par la proposition \ref{PropAFNPSsm}, pour tout \( A\in \tribE\) nous avons
    \begin{subequations}
        \begin{align}
            \lim_{n\to \infty} \mu_n(A)=\mu(A)\\
            \lim_{n\to \infty} \nu_n(A)=\nu(A)
        \end{align}
    \end{subequations}
    <++>
\end{proof}
<++>

\begin{theorem}[Théorème d'approximation\cite{YHRSDGc}]     \label{ThoAFXXcVa}
    Soit \( (X,\tribB,\mu)\) un espace mesuré où \( \tribB\) sont les boréliens de \( X\). Soit \( A\in \tribB\) tel que \( A\subset W\) où \( W\) est un ouvert avec \( \mu(W)<\infty\). Soit aussi \( \epsilon>0\).
    \begin{enumerate}
        \item
            Il existe un fermé \( F\) et un ouvert \( V\) tels que \( \mu(V)<\infty\) et
            \begin{equation}
                F\subset A\subset V
            \end{equation}
            et \( \mu(V\setminus F)<\epsilon\).
        \item
            Il existe \( f\in C^0(X,\eR)\) nulle hors de \( W\) vérifiant \( 0\leq f\leq 1\) et
            \begin{equation}
                \int_X| \mtu_A-f |^pd\mu(x)<\epsilon.
            \end{equation}
    \end{enumerate}
\end{theorem}
% TODO : la preuve est dans la référence. Il faut replacer ce théorème après la définition de l'intégrale.

% TODO : les mesures à densité doivent être après les intégrales.

%---------------------------------------------------------------------------------------------------------------------------
\subsection{Théorème de récurrence}
%---------------------------------------------------------------------------------------------------------------------------

Soit \( X\) un espace mesurable, \( \mu\) une mesure finie sur \( X\) et \( \phi\colon X\to X\) une application mesurable préservant la mesure, c'est à dire que pour tout ensemble mesurable \( A\subset X\),
\begin{equation}
    \mu\big( \phi^{-1}(A) \big)=\mu(A).
\end{equation}
Si \( A\subset X\) est un ensemble mesurable, un point \( x\in A\) est dit \defe{récurrent}{récurrent!point d'un système dynamique} par rapport à \( A\) si et seulement si pour tout \( p\in \eN\), il existe \( k\geq p\) tel que \( \phi^k(x)\in A\).

\begin{theorem}[\wikipedia{fr}{Théorème_de_récurrence_de_Poincaré}{Théorème de récurrence de Poincaré}.]     \label{ThoYnLNEL}
    Si \( A\) est mesurable dans \( X\), alors presque tous les points de \( A\) sont récurrents par rapport à \( A\).
\end{theorem}

\begin{proof}
    Soit \( p\in \eN\) et l'ensemble
    \begin{equation}
        U_p=\bigcup_{k=p}^{\infty}\phi^{-k}(A)
    \end{equation}
    des points qui repasseront encore dans \( A\) après \( p\) itérations  de \( \phi\). C'est un ensemble mesurable en tant que union d'ensembles mesurables (pour rappel, les tribus sont stables par union dénombrable, comme demandé à la définition \ref{DefjRsGSy}), et nous avons donc
    \begin{equation}
        \mu(U_p)\leq \mu(X)<\infty.
    \end{equation}
    De plus \( U_p=\phi^{-p}(U_0)\), donc \( \mu(U_p)=\mu(U_0)\). Vu que \( U_p\subset U_p\), nous avons
    \begin{equation}
        \mu(U_0\setminus U_p)=0.
    \end{equation}
    Étant donné que \( A\subset U_0\) nous avons a fortiori que
    \begin{equation}
        \{ x\in A\tq x\notin U_p \}\subset U_0\setminus U_p,
    \end{equation}
    et donc
    \begin{equation}
        \mu\{ x\in A\tq x\notin U_p \}=0.
    \end{equation}
    Cela signifie exactement que l'ensemble des points \( x\) de \( A\) tels que aucun des \( \phi^k(x)\) avec \( k\geq p\) n'est dans \( A\) est de mesure nulle.
\end{proof}

%---------------------------------------------------------------------------------------------------------------------------
\subsection{Mesure produit}
%---------------------------------------------------------------------------------------------------------------------------

\begin{definition}      \label{DefTribProfGfYTuR}
    Si \( \tribA\) et \( \tribB\) sont deux tribus sur deux ensembles \( \Omega_1\) et \( \Omega_2\), nous définissons la \defe{tribu produit}{tribu!produit} \( \tribA\otimes\tribB\) comme étant la tribu engendrée par 
    \begin{equation}
        \{ A\times B\tq A\in\tribA,B\in\tribB \}.
    \end{equation}
\end{definition}

\begin{theorem}[\cite{FubiniBMauray,MesIntProbb}]
    Soient \( \mu_i\) des mesures $\sigma$-finies sur \( (\Omega_i,\tribA_i)\) (\( i=1,2\)). Il existe une et une seule mesure, notée \( \mu_1\otimes \mu_2\), sur \( (\Omega_1\times\Omega_2,\tribA_1\otimes\tribA_2)\) telle que
    \begin{equation}
        (\mu_1\otimes\mu_2)(A_1\times A_2)=\mu_1(A_1)\mu_2(A_2)
    \end{equation}
    pour tout \( A_1\in \tribA_1\) et \( A_2\in\tribA_2\).
\end{theorem}
\index{mesure!produit}

%---------------------------------------------------------------------------------------------------------------------------
\subsection{Intégrale par rapport à une mesure}
%---------------------------------------------------------------------------------------------------------------------------

\begin{lemma}[Limite croissante de fonctions étagées mesurables]    \label{LemYFoWqmS}
    Soit \( f\colon (\Omega,\tribA)\to \eR\) une fonction mesurable. Il existe une suite \( f_n\colon \Omega\to \eR\) de fonctions étagées telles que \( f_n\to f\) ponctuellement et \( | f_n |<f\).
\end{lemma}

\begin{proof}
    Nous considérons \( (q_n)\) une suite parcourant tous les rationnels\footnote{Nous rappelons que \( \eQ\) est dénombrable et dense dans \( \eR\).}.
    %TODO : démontrer ou référentier le dénombrable et le dense.
    Pour \( n\in \eN\) nous définissons la fonction
    \begin{equation}
        f_n(\omega)=\begin{cases}
            \max\{ q_i\tq i\leq n,\, q_i\leq f(\omega) \}    &   \text{si \( f(\omega)\geq 0\)}\\
            \min\{ q_i\tq i\leq n,\, q_i\geq f(\omega) \}    &    \text{si \( f(\omega)< 0\).}
        \end{cases}
    \end{equation}
    La fonction \( f_n\) est étagée parce qu'elle ne prend que \( n\) valeurs différentes. Nous avons aussi par construction que \( | f_n(\omega)|\leq |f(\omega) |\). Nous avons aussi pour tout \( \omega\in \Omega\) que \( f_n(\omega)\to f(\omega)\) parce que \( \eQ\) est dense dans \( \eR\).

    En ce qui concerne la mesurabilité de \( f_n\), les étages de \( f_n\) sont les ensembles de la forme \( \{ \omega\in \Omega\tq f(\omega)\in\mathopen[ a , b [ \}\) où \( a\) et \( b\) sont deux éléments de \( \{ q_1,\ldots, q_n \}\) qui sont consécutifs au sens de l'ordre dans \( \eQ\) (et non spécialement au sens de l'ordre d'apparition dans la suite), avec éventuellement \( b=\infty\) si \( a\) est le plus grand. Les ensembles \( \mathopen[ a , b [\) étant mesurables dans \( \eR\) et la fonction \( f\) étant mesurable par hypothèse, les ensembles \( f^{-1}\Big( \mathopen[ a , b [ \Big)\) sont mesurables dans \( (\Omega,\tribA)\).
\end{proof}

Une mesure \( \mu\) sur un espace mesurable \( (\Omega,\tribA)\) permet de définir une fonctionnelle linéaire sur l'ensemble des fonctions mesurables \( \Omega\to \eR\). Cette fonctionnelle linéaire est l'intégrale que nous allons définir à présent.

D'abord nous considérons les fonction \defe{simples}{simple!fonction}\index{fonction!simple}, c'est à dire les fonctions de la forme
\begin{equation}
    f=\sum_{i=1}^Na_i\caract_{E_i}
\end{equation}
où \( a_i\in\eR\) tandis que les \( E_i\) sont des ensembles \( \mu\)-mesurables. Si \( Y\in \tribA\) nous définissons
\begin{equation}
    \int_Yfd\mu=\sum_ia_i\mu(Y\cap E_i).
\end{equation}
Pour une fonction \( \mu\)-mesurable générale \( f\colon \Omega\to \mathopen[ 0 , \infty \mathclose]\) nous définissons l'intégrale de \( f\) sur \( Y\) par
\begin{equation}        \label{EqDefintYfdmu}
    \int_Yfd\mu=\sup\Big\{ \int_Yhd\mu\,\text{où \( h\) est une fonction simple et mesurable telle que \( 0\leq h\leq f\)} \Big\}.
\end{equation}
Maintenant nous définissons
\begin{equation}
    \mu(f)=\int_{\Omega}f
\end{equation}
si \( f\) est une fonction mesurable sur \( \Omega\).

\begin{remark}
    Dans \( \eR^d\), quasiment toutes les fonctions et ensembles sont mesurables. En effet la construction d'ensembles non mesurables demande obligatoirement l'utilisation de l'axiome du choix; de tels ensembles doivent être construits «exprès pour». Il y a très peu de chances pour que vous tombiez sur un ensemble non mesurable de \( \eR^d\) sans que vous ne vous en rendiez compte.

    Par exemple la variable aléatoire 
    \begin{equation}
        X(\omega)=\begin{cases}
            \frac{1}{ \omega }    &   \text{si $ \omega\neq 0$}\\
            \infty    &    \text{$\omega=0$}.
        \end{cases}
    \end{equation}
    est mesurable, mais non intégrable.
\end{remark}

Le lemme suivant nous aide à détecter des fonctions presque partout nulles.
\begin{lemma}   \label{Lemfobnwt}
    Soit \( f\) une fonction mesurable positive ou nulle telle que
    \begin{equation}
        \int_{\Omega}fd\mu=0.
    \end{equation}
    Alors \( f=0\) \( \mu\)-presque partout.
\end{lemma}

\begin{proof}
    L'ensemble des points \( x\in\Omega\) tels que \( f(x)\neq 0\) peut s'écrire comme une union dénombrable disjointe :
    \begin{equation}
        \{ x\in\Omega\tq f(x)\neq 0 \}=\bigcup_{i=0}^{\infty}E_i
    \end{equation}
    avec
    \begin{subequations}
        \begin{align}
            E_0&=\{ x\in\Omega\tq f(x)>1 \}\\
            E_i&=\{ x\in\Omega\tq \frac{1}{ i+1 }\leq f(x)<\frac{1}{ i } \}.
        \end{align}
    \end{subequations}
    Si un des ensembles \( E_i\) est de mesure non nulle, alors nous pouvons considérer la fonction simple \( h(x)=\frac{1}{ i+1 }\mtu_{E_i}\) dont l'intégrale sur \( \Omega\) est strictement positive. Par conséquent le supremum de la définition \eqref{EqDefintYfdmu} est strictement positif.

    Nous savons donc que \( \mu(E_i)=0\) pour tout \( i\). Étant donné que la mesure d'une union disjointe dénombrable est égale à la somme des mesures, nous avons
    \begin{equation}
        \mu\{ x\in\Omega\tq f(x)\neq 0 \}=0,
    \end{equation}
    ce qui signifie que \( f\) est nulle \( \mu\)-presque partout.
\end{proof}

\begin{corollary}   \label{CorjLYiSm}
    Soit \( f\) une fonction mesurable sur l'espace mesuré \( (\Omega,\tribA,\mu)\) telle que
    \begin{equation}
        \int_{\Omega}f\mtu_{f>0}d\mu=0.
    \end{equation}
    Alors \( f\leq 0\) presque partout.
\end{corollary}

\begin{proof}
    Nous avons l'égalité d'ensembles
    \begin{equation}
        \{ f\mtu_{f>0}\neq 0 \}=\{ \mtu_{f>0}\neq 0 \}.
    \end{equation}
    Mais lemme \ref{Lemfobnwt} implique que \( f\mtu_{f>0}\) est nulle presque partout, c'est à dire que la mesure de l'ensemble du membre de gauche est nulle par conséquent
    \begin{equation}
        \mu\{ \mtu_{f>0}\neq 0 \}=0.
    \end{equation}
    Cela signifie que la fonction \( f\) est presque partout négative ou nulle.
\end{proof}

\begin{lemma}   \label{LemPfHgal}
    Soit \( f\) une fonction telle que \( | f(x)|\leq g(x) \) pour tout \( x\in\Omega\). Si \( g\) est intégrable, alors \( f\) est intégrable.
\end{lemma}

\begin{proof}
    Nous décomposons \( f\) en parties positives et négatives :
    \begin{subequations}
        \begin{align}
            A_+&=\{ x\in\Omega\tq f(x)>0 \}\\
            A_-&=\{ x\in\Omega\tq f(x)<0 \}.
        \end{align}
    \end{subequations}
    Nous posons \( f_+(x)=f(x)\mtu_{A_+}\) et \( f_-(x)=f(x)\mtu_{A_-}\). Nous avons \( f=f_+-f_-\) et
    \begin{equation}
        \int_{\Omega}f=\int_{A_+}f+\int_{A_-}f
    \end{equation}
    parce que \( \Omega=A_+\cup A_-\cup\{ x\in\Omega\tq f(x)=0 \}\). Si \( \varphi\) est une fonction simple qui majore \( f_+\) nous avons
    \begin{equation}
        \varphi(x)=\sum_{k}a_k\mtu_{E_k}(x)\leq f(x)\mtu_{A_+}(x)\leq g(x).
    \end{equation}
    Par conséquent le supremum qui définit \( \int f_+\) est inférieur au supremum qui définit \( \int g\). La fonction \( f_+\) est donc intégrable. La même chose est valable pour la fonction \( f_-\).
\end{proof}

\begin{proposition} \label{PropWBavIf}
    Soit \( f\) une fonction positive \( \tribA\)-mesurable et bornée. Alors \( f\) est limite ponctuelle croissante de fonction simples.
\end{proposition}

\begin{proof}
    Soit \( \sigma_n=\{ a_0=0,\ldots, a_{r_n}=n \}\) une subdivision de \( \mathopen[ 0 , n \mathclose]\) en intervalles de taille plus petites que \( 1/n\) choisis de sorte que \( \sigma_{n-1}\subset\sigma_{n}\), et
    \begin{equation}
        f_n(x)=\begin{cases}
            0    &   \text{si \( f(x)>n\)}\\
            a_i    &    \text{sinon}
        \end{cases}
    \end{equation}
    où \( a_i\) est le plus grand élément de \( \sigma_n\) inférieur à \( f(x)\). La fonction \( f_n\) est simple et nous avons pour tout \( x\)
    \begin{equation}
        \lim_{n\to \infty} f_n(x)=f(x)
    \end{equation}
\end{proof}

%---------------------------------------------------------------------------------------------------------------------------
\subsection{Mesure dominée}
%---------------------------------------------------------------------------------------------------------------------------

Soient \( \mu\) et \( \nu\) deux mesures sur le même espace \( \Omega\) et la même tribu \( \tribA\). Nous disons que la mesure \( \mu\) est \defe{dominée}{dominée!mesure}\cite{PersoFeng} par \( \nu\) si pour tout ensemble mesurable \( A\), \( \nu(A)=0\) implique \( \mu(A)=0\).

La mesure \( \mu\) est \defe{portée}{portée!mesure} par l'ensemble \( E\in\tribA\) si pour tout \( A\in\tribA\), 
\begin{equation}
    \mu(A)=\mu(A\cap E).
\end{equation}

Nous écrivons que \( \mu\perp\nu\)\nomenclature[Y]{\( \mu\perp\nu\)}{mesures perpendiculaires} si il existe un ensemble \( E\in\tribA\) tel que \( \mu\) soit porté par \( E\) et \( \nu\) soit porté par \( \complement E\).

\begin{theorem}[Radon-Nikodym\cite{NikoLi}]\index{Radon-Nikodym}
    Soient \( \mu\) et \( m\) deux mesures \( \sigma\)-finies sur un espace métrisable \( (\Omega,\tribA)\).
    \begin{enumerate}
        \item
            Il existe un unique couple de mesures \( \mu_1\) et \( \mu_2\) telles que
            \begin{enumerate}
                \item
                    \( \mu=\mu_1+\mu_2\)
                \item
                    \( \mu_1\) est dominé par \( m\)
                \item
                    \( \mu_2\perp m\).
            \end{enumerate}
            Dans ce cas, les mesures \( \mu_1\) et \( \mu_2\) sont positives et \( \sigma\)-finies.
        \item
            À égalité \(  m\)-presque partout près, il existe une unique fonction mesurable positive \( f\) telle que pour tout mesurable \( A\),
            \begin{equation}
                \mu_1(A)=\int_Ad\mu_1=\int_{\Omega}\mtu_Afd m.
            \end{equation}
        \item
            À égalité \( m\)-presque partout près, il existe une unique fonction positive mesurable \( h\) telle que \( \mu_1=hm\).
    \end{enumerate}
\end{theorem}
%TODO : une preuve

\begin{corollary}   \label{CorZDkhwS}
    Si \( \mu\) es une mesure \( \sigma\)-finie dominée par la mesure \( \sigma\)-finie \( m\), alors \( \mu\) possède une unique fonction de densité.
\end{corollary}

\begin{corollary}       \label{CorDomDens}
    Soient \( \mu\) et \( m\), deux mesures positives \( \sigma\)-finies sur \( (\Omega,\tribA)\). Alors \( m\) domine \( \mu\) si et seulement si \( \mu\) possède une densité par rapport à \( m\).
\end{corollary}
 
\begin{proof}
    Si \( \mu\) est dominée par \( m\), alors la décomposition \( \mu=\mu+0\) satisfait le théorème de Radon-Nikodym. Par conséquent il existe une fonction \( f\) telle que
    \begin{equation}
        \mu(A)=\int_Afdm.
    \end{equation}
    Cette fonction est alors une densité pour \( \mu\) par rapport à \( m\).

    Pour la réciproque, nous supposons que \( \mu\) a une densité \( f\) par rapport à \( m\), et que \( A\) est une ensemble de \( m\)-mesure nulle :
    \begin{equation}
        m(A)=\int_{\Omega}\mtu_Adm=0.
    \end{equation}
    Cela signifie que la fonction \( \mtu_A\) est \( m\)-presque partout nulle. La fonction produit \( \mtu_Af\) est également nulle \( m\)-presque partout, et par conséquent
    \begin{equation}
        \mu(A)=\int_{\Omega}\mtu_Afdm=0.
    \end{equation}
\end{proof}

\begin{probleme}
    Est-ce que la démonstration de cela ne demande pas la convergence monotone d'une façon ou d'une autre ?
\end{probleme}

%---------------------------------------------------------------------------------------------------------------------------
\subsection{Changement de variables dans une intégrale}
%---------------------------------------------------------------------------------------------------------------------------

\begin{theorem} \label{ThomFeRCi}
    Soit \( \mO\) un ouvert de \( \eR^n\) et \( \mO'\) un ouvert de \( \eR^m\). Soit \( \varphi\colon \mO\to \mO'\) un difféomorphisme \( C^1\). Si \( f\colon \mO\to \eR\) est une fonction mesurable, positive et intégrable, alors
    \begin{equation}
        \int_{\mO}f(u)du=\int_{\mO'}f\big( \varphi^{-1}(v) \big)| J_{\varphi^{-1}}(v) |dv.
    \end{equation}
\end{theorem}

%--------------------------------------------------------------------------------------------------------------------------- 
\subsection{Théorème de Fubini-Tonelli et de Fubini}
%---------------------------------------------------------------------------------------------------------------------------

Il existe trois résultats. Le premier, le théorème de Fubini-Tonelli \ref{ThoWTMSthY} demande que la fonction soit mesurable et positive; le second, le théorème de Fubini \ref{ThoFubinioYLtPI} demande que la fonction soit intégrable (mais pas spécialement positive); et le troisième, le corollaire \ref{CorTKZKwP} demande l'intégrabilité de la valeur absolue des intégrales partielles pour déduire que la fonction elle-même est intégrable.

%TODO : des démonstrations de ces trois théorèmes seraient les bienvenues.

Nous rappelons que \( \eR^n\) muni de la mesure de Lebesgue est un espace mesuré \( \sigma\)-fini, conformément à la définition \ref{DefBTsgznn}.

\begin{theorem}[Fubini-Tonelli\cite{MesIntProbb}]\label{ThoWTMSthY}
    Soient \( (\Omega_i,\tribA_i,\mu_i)\) deux espaces mesurés \( \sigma\)-finis, et \( (\Omega,\tribA,\mu)\) l'espace produit. Soit une fonction mesurable
    \begin{equation}
        f\colon \Omega\to \eR^{+}\cup\{ +\infty \}.
    \end{equation}
    Alors :
    \begin{enumerate}
        \item
            Pour presque tout \( x\in\Omega_1\), la fonction \( y\mapsto f(x,y)\) est mesurable sur \( \Omega_2\).
        \item
            Si nous posons
            \begin{equation}
                \varphi_f(x)=\int_{\Omega_2}f(x,y)d\mu_2(y),
            \end{equation}
            alors \( \varphi_f\) est une fonction bien définie presque partout sur \( \Omega_1\) et \( \varphi\) est mesurable (à valeurs positives).
        \item
            Toutes les intégrales imaginables existent et sont égales :
            \begin{subequations}
                \begin{align}
                \int_{\Omega}fd(\mu_1\otimes \mu_2)&=\int_{\Omega_1}\varphi_fd\mu_1\\
                &=\int_{\Omega_1}\left[ \int_{\Omega_2}f(x,y)d\mu_2(y) \right]d\mu_1(x)\\
                &=\int_{\Omega_2}\left[ \int_{\Omega_1}f(x,y)d\mu_1(x) \right]d\mu_2(y).
                \end{align}
            \end{subequations}
    \end{enumerate}
\end{theorem}
\index{théorème!Fubini-Tonelli}

\begin{theorem}[Fubini\cite{MesIntProbb}]\label{ThoFubinioYLtPI}
    Soient \( (\Omega_i,\tribA_i,\mu_i)\) deux espaces mesurés \( \sigma\)-finis, et \( (\Omega,\tribA,\mu)\) l'espace produit. Soit 
    \begin{equation}
        f\in L^1\big( (\Omega,\tribA),\eR \big),
    \end{equation}
    c'est à dire une fonction à valeurs réelles mesurable et intégrable sur \( \Omega\). Alors :
    \begin{enumerate}
        \item
            Pour presque tout \( x\in \Omega_1\), la fonction \( y\mapsto f(x,y)\) est \( L^1(\Omega_2)\).
        \item
            Si nous posons
            \begin{equation}
                \varphi_f(x)=\int_{\Omega_2}f(x,y)d\mu_2(y);
            \end{equation}
            alors \( \varphi_f\in L^1(\Omega_1)\).
        \item   \label{ItemQMWiolgiii}
            Nous avons la formule d'inversion d'intégrale
            \begin{subequations}
                \begin{align}
                \int_{\Omega}fd(\mu_1\otimes \mu_2)&=\int_{\Omega_1}\varphi_fd\mu_1\\
                &=\int_{\Omega_1}\left[ \int_{\Omega_2}f(x,y)d\mu_2(y) \right]d\mu_1(x)\\
                &=\int_{\Omega_2}\left[ \int_{\Omega_1}f(x,y)d\mu_1(x) \right]d\mu_2(y).
                \end{align}
            \end{subequations}
    \end{enumerate}
\end{theorem}
\index{théorème!Fubini!espace mesuré}

Si la fonction \( (x,y)\mapsto f(x)g(y)\) satisfait aux hypothèse du théorème de Fubini alors
\begin{equation}    \label{EqTJEEsJW}
    \int_{\Omega_1\times \Omega_2} f(x)g(y)dx\otimes dy=\left( \int_{\Omega_1}f(x)dx \right)\left( \int_{\Omega_2}g(y)dy \right).
\end{equation}
Le théorème de Fubini est souvent utilisé sous cette forme.

\begin{corollary}\label{CorTKZKwP}
    Soient \( (\Omega_i,\tribA_i,\mu_i)\) deux espaces mesurés \( \sigma\)-finis, et \( (\Omega,\tribA,\mu)\) l'espace produit. Soit une fonction mesurable \( f\colon \Omega\to \eR\). Alors les conditions suivantes sont équivalentes
    \begin{enumerate}
        \item
            \( f\in L^1(\Omega)\),
        \item
            \begin{equation}
                \int_{\Omega_1}\left[ \int_{\Omega_2}| f |d\mu_2 \right]d\mu_1 <\infty,
            \end{equation}
        \item
            \begin{equation}
                \int_{\Omega_2}\left[ \int_{\Omega_1}| f |d\mu_1 \right]d\mu_2 <\infty.
            \end{equation}
    \end{enumerate}
\end{corollary}
En pratique, lorsqu'on ne sait pas a priori si \( f\) est intégrable sur \( \Omega_1\times \Omega_2\), nous testons l'intégrabilité en chaine de \( | f |\), et si c'est bon, alors nous savons que \( f\) est intégrable sur le produit et qu'on peut permuter les intégrales.

\begin{example}
    Nous montrons que le théorème ne tient pas si une des deux mesures n'est pas \( \sigma\)-finie. Soit \( I=\mathopen[ 0 , 1 \mathclose]\). Nous considérons l'espace mesuré
    \begin{equation}
        (I,\Borelien(I),\lambda)
    \end{equation}
    où \( \Borelien(I)\) est la tribu des boréliens sur \( I\) et \( \lambda\) est la mesure de Lebesgue (qui est $\sigma$-finie). D'autre part nous considérons l'espace mesuré
    \begin{equation}
        (I,\partP(I),m)
    \end{equation}
    où \( \partP(I)\) est l'ensemble des parties de \( I\) et \( m\) est la mesure de comptage. Cette dernière n'est pas $\sigma$-finie parce que les seuls ensembles de mesure finie pour la mesure de comptage sont des ensembles finis, or une union dénombrable d'ensemble finis ne peut pas recouvrir l'intervalle \( I\).

    Nous allons montrer que dans ce cadre, l'intégrale de la fonction indicatrice de la diagonale sur \( I^2\) ne vérifie pas le théorème de Fubini. Étant donné que \( \Borelien(I)\subset\partP(I)\) nous avons
    \begin{equation}
        \Borelien(I^2)\subset\Borelien(I)\otimes\partP(I).
    \end{equation}
    Soit \( \Delta=\{ (x,x)\tq x\in I \}\). La fonction
    \begin{equation}
        \begin{aligned}
            g\colon I^2&\to \eR \\
            (x,y)&\mapsto x-y 
        \end{aligned}
    \end{equation}
    est continue et \( \Delta=g^{-1}(\{ 0 \})\) est donc fermé dans \( I^2\). L'ensemble \( \Delta\) est donc un borélien de \( I^2\) et par conséquent un élément de la tribu \( \Borelien(I)\otimes\partP(I)\). La fonction indicatrice \( \mtu_{\Delta}\) est alors mesurable pour l'espace mesuré
    \begin{equation}
        (I\times I,\Borelien(I)\otimes\partP(I),\lambda\otimes m).
    \end{equation}
    Pour \( x\) fixé nous avons
    \begin{equation}
        \mtu_{\Delta}(x,y)=\begin{cases}
            1    &   \text{si \( y= x\)}\\
            1    &    \text{si \( y\neq x\)}
        \end{cases}=\mtu_{\{ x \}}(y),
    \end{equation}
    et donc
    \begin{subequations}
        \begin{align}
            A_1&=\int_I\left( \int_I\mtu_{\Delta}(x,y)dm(y) \right)d\lambda(x)\\
            &=\int_I\left( \int_I\mtu_{\{ x \}}(y)dm(y) \right)d\lambda(x)\\
            &=\int_I\Big( m(\{ x \}) \Big)d\lambda(x)\\
            &=\int_I 1d\lambda(x)\\
            &=1.
        \end{align}
    \end{subequations}
    Par contre le support de \( \mtu_{\Delta}\) étant de mesure nulle pour la mesure de Lebesgue, nous avons
    \begin{equation}
        \int_I\mtu_{\Delta}(x,y)d\lambda(x)=0
    \end{equation}
    et par conséquent
    \begin{equation}
        A_2=\int_I\left( \int_I\mtu_{\Delta}(x,y)d\lambda(x) \right)dm(y)=0.
    \end{equation}
    Nous voyons donc que le théorème de Fubini ne s'applique pas.
\end{example}

\begin{example} \label{ExrgMIni}
    Le théorème de Fubini est utilisé dans le calcul de l'intégrale gaussienne
    \begin{equation}
        G=\int_{\eR} e^{-x^2}dx,
    \end{equation}
    alors que la fonction \( x\mapsto  e^{-x^2}\) n'a pas de primitives parmi les fonctions élémentaires.

    Par symétrie nous pouvons nous contenter de calculer
    \begin{equation}
        G_+=\int_0^{\infty} e^{-x^2}dx.
    \end{equation}
    L'astuce est de passer par l'intermédiaire
    \begin{subequations}
        \begin{align}
            H&=\int_{\eR^+\times\eR^+} e^{-(x^2+y^2)}dxdy       \label{EqIntFausasub}\\
            &=\int_{\eR^+}\left( \int_{\eR^+} e^{-x^2} e^{-y^2}dx \right)dy\\
            &=\left( \int_{\eR^+} e^{-x^2} dx\right)^2\\
            &=G_+^2
        \end{align}
    \end{subequations}
    L'intégrale \eqref{EqIntFausasub} se calcule en passant aux coordonnées polaires et le résultat est \( H=\frac{ \pi }{ 4 }\). Nous avons alors \( G=\frac{ \sqrt{\pi} }{ 2 }\) et
    \begin{equation}
        \int_{\eR} e^{-x^2}=\sqrt{\pi}.
    \end{equation}
\end{example}

\begin{example}
    Une variante, qui n'applique pas Fubini sur un domaine non borné. Nous commençons par écrire
\begin{equation}
	I=\int_{-\infty}^{+\infty} e^{-x^2} dx := \lim_{R \to +\infty} \int_{-R}^{+R} e^{-x^2} dx 
\end{equation}
et puis nous faisons le calcul
\begin{equation}		\label{EqCalculInteeemoisxcar}
	\begin{aligned}[]
		I^2 &= \lim_{R \to +\infty} \left( (\int_{-R}^{+R} e^{-x^2} dx)( \int_{-R}^{+R} e^{-y^2} dy) \right) \\
		&= \lim_{R \to +\infty} \left( \iint_{K_R}e^{-(x^2+y^2)} dx dy \right) \\
		&= \lim_{R \to +\infty} \left( \iint_{C_R}e^{-(x^2+y^2)} dx dy \right) 
	\end{aligned}
\end{equation}
où $K$ est le carré de demi côté $R$ centré à l'origine et de côtés parallèles aux axes et $C_R$ est le cercle de rayon $R$ centré à l'origine.

	La première étape à justifier est simplement l'application de Fubini. Pour le passage de l'intégrale du carré vers le cercle, définissons
	\begin{equation}
		\begin{aligned}[]
			I_K(r)&=\int_{K_r}f,&I_C(r)&=\int_{C_r}f
		\end{aligned}
	\end{equation}
	où $K_r$ est la carré de demi côté $r$ et $C_r$ est le cercle de rayon $r$. Le demi côté du carré inscrit à $C_r$ est $\sqrt{2}$, donc pour tout $r$ nous avons
	\begin{equation}
		I_K(\sqrt{2}r)\leq I_C(r)<I_K(r),
	\end{equation}
	et en prenant la limite, nous avons évidement
	\begin{equation}
		\lim_{r\to \infty}I_K(\sqrt{2}r)=\lim_{r\to\infty}I_K(r),
	\end{equation}
	de telle façon à ce que cette limite soit également égale à $\lim_{r\to\infty}I_C(t)$.


    Il ne reste qu'à calculer la dernière intégrale sur le cercle en passant aux coordonnées polaires :
	\begin{equation}
        \iint_{C_R} e^{-(x^2+y^2)}dxdy=\int_0^{2\pi}d\theta\int_0^Rr e^{-r^2}dr=\pi(1- e^{-R^2}).
	\end{equation}
	La limite donne $\pi$, nous en déduisons que
    \begin{equation}    \label{EqFDvHTg}
		\int_{-\infty}^{\infty} e^{-x^2}dx=\sqrt{\pi}.
	\end{equation}

\end{example}

\begin{example} \label{ExempInversSumIntFub}   \index{mesure!de comptage}
    Le théorème de Fubini-Tonelli nous permet également d'inverser des sommes et des séries. En effet une somme n'est rien d'autre qu'une intégrale pour la mesure de comptage :
    \begin{equation}
        \sum_{n=0}^{\infty}a_n=\int_{\eN}a_ndm(n).
    \end{equation}
    Considérons une suite de fonctions \( f_n\colon \eR^d\to \eR\) \emph{positives}, la quantité
    \begin{equation}    \label{EqAcalculParFubIntSum}
        I=\sum_{n=0}^{\infty}\int_{\eR^n}f_n(x)dx
    \end{equation}
    et les espaces mesurés \( (\eN,\partP(\eN),m)\), \( (\eR^n,\Borelien(\eR^n),\lambda)\) où \( \lambda\) est la mesure de Lebesgue. En écrivant la formule \eqref{EqAcalculParFubIntSum}, nous supposons que pour chaque \( n\), la fonction \( f_n\) est intégrable sur \( \eR^d\) et que le résultat soit sommable. Nous pouvons la récrire sous la forme
    \begin{equation}
        \int_{\eN}\left( \int_{\eR^n}f(n,x)dx \right)dm(n)
    \end{equation}
    avec la notation évidente \( f(n,x)=f_n(x)\). Prouvons que la fonction \( f\colon \eN\times\eR^d\to \eR\) ainsi définie est une fonction mesurable pour l'espace mesuré
    \begin{equation}
        \big( \eN\times\eR^d,\partP(\eN)\otimes\Borelien(\eR^d),m\otimes\lambda \big).
    \end{equation}
    Si \( A\subset\eR\), nous avons
    \begin{equation}
        f^{-1}(A)=\bigcup_{n\in\eN}\{ n \}\times f_n^{-1}(A).
    \end{equation}
    Chacun des ensembles dans l'union appartient à la tribu \( \partP(\eN)\times\Borelien(\eR^d)\) tandis que les tribus sont stables sous les unions dénombrables. La fonction \( f\) est donc mesurable. La fonction \( f\) est donc mesurable. Comme nous avons supposé que \( f\) était positive, le théorème de Fubini-Tonelli s'applique et nous avons
    \begin{equation}
        I=\int_{\eR^d}\left( \int_{\eN}f(n,x)dm(n) \right)dx=\int_{\eR^d}\sum_{n\in \eN}f_n(x)dx.
    \end{equation}
\end{example}

%+++++++++++++++++++++++++++++++++++++++++++++++++++++++++++++++++++++++++++++++++++++++++++++++++++++++++++++++++++++++++++
\section{Forme différentielle et intégrale sur un chemin}
%+++++++++++++++++++++++++++++++++++++++++++++++++++++++++++++++++++++++++++++++++++++++++++++++++++++++++++++++++++++++++++
\label{SecFormDiffRappel}

Nous parlerons de formes différentielles exactes et fermées dans la section \ref{DefEFKQmPs}.

Une \defe{forme}{forme} sur un espace vectoriel $V$ est une application linéaire $\omega\colon V\to \eR$.

\begin{definition}
	Soit $D$, un ouvert dans $\eR^n$. Une $1$-\defe{forme différentielle}{forme!différentielle} $\omega$ sur $D$ est une application
	\begin{equation}
		\begin{aligned}
				\omega\colon D&\to (\eR^n)^* \\
				x&\mapsto \omega_x. 
			\end{aligned}
		\end{equation}
\end{definition}
Étant donné que $\{ dx_i \}$ est une base de $(\eR^n)^*$, pour chaque $x\in D$, il existe des uniques réels $a_i(x)$ tels que
\begin{equation}
	\omega_x=a_1(x)dx_1+\ldots+a_n(x)dx_n.
\end{equation}
Nous disons qu'une $1$-forme différentielle est \defe{continue}{continue!forme différentielle} si les fonctions $a_i$ sont continues. La forme sera $C^k$ quand les $a_i$ seront $C^k$.

\begin{remark}
	L'ensemble des $1$-formes différentielles forment un espace vectoriel avec les définitions
	\begin{equation}
		\begin{aligned}[]
			(\lambda\omega)_x(v)&=\lambda\omega_x(v)\\
			(\omega+\mu)_x(v)&=\omega_x(v)+\mu_x(v).
		\end{aligned}
	\end{equation}
\end{remark}

Une $1$-forme différentielle s'écrit toujours sous la forme
\begin{equation}
	\omega=\sum_i a_idx_i
\end{equation}
pour certaines fonctions $a_i$. Évidemment, ces fonctions $a_i$ peuvent être trouvées en appliquant $\omega$ aux éléments de la base canonique de $\eR^n$ :
\begin{equation}
	a_j(x)=\omega_x(e_j)
\end{equation}
parce que $\omega_x(e_j)=\sum_ia_i(x)dx_i(e_i)=\sum_ia_i(x)\delta_{ij}=a_j(x)$.


\begin{example}
    Un exemple type de forme différentielle est la différentielle d'une fonction $f\colon D\to \eR$. En effet, la différentielle d'une telle fonction est l'application linéaire
    \begin{equation}
        \begin{aligned}
            df\colon \eR^n&\to \eR \\
            v&\mapsto \frac{ \partial f }{ \partial x }v_x+\frac{ \partial f }{ \partial y }v_y. 
        \end{aligned}
    \end{equation}
\end{example}

Soit $D\subset\eR^n$. Par définition de la différentielle d'une $1$-forme, nous avons une formule de Leibnitz
\begin{equation}
    d(f\omega)=df\wedge\omega+fd\omega.
\end{equation}
En particulier,
\begin{equation}
    d(fdx)=df\wedge dx+f\underbrace{d(dx)}_{=0}=\frac{ \partial f }{ \partial x }dx\underbrace{dx\wedge dx}_{=0}+\frac{ \partial f }{ \partial y }dy\wedge dx. S
\end{equation}

Si $F\colon \eR^2\to \eR$ est une fonction $C^2$, sa différentielle est la forme
\begin{equation}
    dF=\frac{ \partial F }{ \partial x }dx+\frac{ \partial F }{ \partial y }dy.
\end{equation}
Si nous nommons $f$ et $g$ les fonctions $\partial_xF$ et $\partial_yF$, nous avons donc
\begin{equation}
    Df=fdx+gdy,
\end{equation}
qui vérifie
\begin{equation}
    \partial_yf=\partial_xg,
\end{equation}
parce que $\frac{ \partial f }{ \partial y }=\frac{ \partial^2F  }{ \partial x\partial y }=\frac{ \partial^2F  }{ \partial y\partial x }=\frac{ \partial g }{ \partial x }$. Ce que nous avons donc prouvé, c'est que 
\begin{lemma}
Si $fdx+gdy$ est la différentielle d'une fonction de classe $C^2$, alors $\partial_yf=\partial_xg$.
\end{lemma}

%---------------------------------------------------------------------------------------------------------------------------
\subsection{Forme différentielle}
%---------------------------------------------------------------------------------------------------------------------------

La formule d'intégration d'un champ de vecteur,
\begin{equation}
	\int_{\gamma}G=\int_{[a,b]}\langle G (\gamma(t)), \gamma'(t)\rangle dt,
\end{equation}
contient quelque chose d'intéressant : la combinaison $\langle G( \gamma(t) ), \gamma'(t)\rangle$. Cette combinaison sert à transformer le vecteur tangent $\gamma'(t)$ en un nombre en utilisant le produit scalaire avec le vecteur $G( \gamma(t) )$.

Si $G$ est un champ de vecteur sur $\eR^n$, et si $x\in\eR^n$, nous pouvons considérer, de façon un peu plus abstraite, l'application
\begin{equation}		\label{EqDefBemol}
	\begin{aligned}[]
		G^{\flat}_x\colon \eR^n&\to \eR \\
			v&\mapsto \langle G(x), v\rangle . 
	\end{aligned}
\end{equation}
Cela permet de compactifier la notation et écrire
\begin{equation}
	\int_{\gamma}G=\int_{[a,b]} G^{\flat}_{\gamma(t)}\big( \gamma'(t)\big) dt.
\end{equation}

Nous nous proposons maintenant d'étudier plus en détail ce qu'est l'objet $G^{\flat}$. La règle \eqref{EqDefBemol} dit que pour chaque $x$, l'application $G_x^{\flat}$ est une forme sur $\eR^n$, c'est à dire une application linéaire de $\eR^n$ vers $\eR$. Nous écrivons que
\begin{equation}
	G_x^{\flat}\in\big( \eR^n \big)^*.
\end{equation}
Nous connaissons la \defe{base duale}{base!duale} de $(\eR^n)^*$, ce sont les formes $e^*_i$ définies par $e^*_i(e_j)=\delta_{ij}$. Dans le cadre du cours d'analyse, nous allons noter ces formes\footnote{Parce que ce sont les différentielles des fonctions (projections)
\begin{equation}
	\begin{aligned}
			x_i\colon \eR^n&\to \eR \\
			x&\mapsto x_i 
		\end{aligned}
	\end{equation}
}
par $dx_i$ :
\begin{equation}
	\begin{aligned}[]
		e^*_1&=dx_1\colon v\mapsto v_1	\\
			&\vdots			\\
		e^*_n&=dx_n\colon v\mapsto v_n
	\end{aligned}
\end{equation}
Étant donné que ces $dx_i$ forment une base de l'espace vectoriel $(\eR^n)^*$, toute application linéaire $L\colon \eR^n\to \eR$ s'écrit
\begin{equation}
	\begin{aligned}[]
		Lv&=a_1v_1+\ldots+a_nv_n\\
			&=a_1dx_1(v)+\ldots+a_ndx_n(v).
	\end{aligned}
\end{equation}
Plus abstraitement, nous notons
\begin{equation}
	\begin{aligned}[]
		L&=a_1dx_1+\ldots+a_ndx_n\\
		&=\sum_{i=1}^na_idx_i.
	\end{aligned}
\end{equation}
L'application $L$ est une combinaison linéaire des $dx_i$ au sens de l'espace vectoriel $(\eR^n)^*$.

L'objet $G^{\flat}$ est la donnée, en chaque point de $D$, d'une telle forme sur $\eR^n$. 

Nous pouvons ainsi déterminer le développement de $G^{\flat}$ dans la base des $dx_i$ en faisant le calcul
\begin{equation}
	G_x^{\flat}(e_i)=\langle G(x), e_i\rangle =G_i(x),
\end{equation}
donc les composantes de $G^{\flat}$ dans la base $dx_i$ sont exactement les composantes de $G$ dans la base $e_i$ :
\begin{equation}
	G^{\flat}_x=G_1(x)dx_1+\ldots+G_n(x)dx_n.
\end{equation}

%///////////////////////////////////////////////////////////////////////////////////////////////////////////////////////////
\subsubsection{L'isomorphisme musical}
%///////////////////////////////////////////////////////////////////////////////////////////////////////////////////////////

Nous savons qu'un champ de vecteur $G$ produit la forme différentielle $G^{\flat}$. La construction inverse existe également. Si $\omega$ est une $1$-forme différentielle, nous pouvons définir le champ de vecteur $\omega^{\sharp}$ par la formule (implicite)
\begin{equation}
	\omega_x(v)=\langle \omega^{\sharp}(x), v\rangle 
\end{equation}
pour tout $v\in\eR^n$. Par définition, $(\omega^{\sharp})^{\flat}=\omega$. 

\begin{lemma}
    En composantes nous avons :
	\begin{equation}
		\omega^{\sharp}(x)=\big( a_1(x),\ldots,a_n(x) \big).
	\end{equation}
	Si $G$ est un champ de vecteurs, alors $(G^{\flat})^{\sharp}=G$.
\end{lemma}

%---------------------------------------------------------------------------------------------------------------------------
\subsection{Intégration d'une forme différentielle sur un chemin}
%---------------------------------------------------------------------------------------------------------------------------

Les formes intégrales que nous avons déjà vues sont celles de fonctions et de champs de vecteur sur des chemins. Si $\gamma\colon [a,b]\to \eR^n$ est le chemin, les formules sont
\begin{equation}
	\begin{aligned}[]
		\int_{\gamma}f&=\int_{[a,b]}f\big( \gamma(t) \big)\| \gamma'(t) \|dt\\
		\int_{\gamma}G&=\int_{[a,b]}\langle G\big( \gamma(t) \big), \gamma'(t)\rangle dt.
	\end{aligned}
\end{equation}
Dans les deux cas, le principe est que nous disposons de quelque chose (la fonction $f$ ou le vecteur $G$), et du vecteur tangent $\gamma'(t)$, et nous essayons d'en tirer un nombre que nous intégrons. Lorsque nous avons une $1$-forme, la façon de l'utiliser pour produire un nombre avec le vecteur tangent est évidement d'appliquer la $1$-forme au vecteur tangent. La définition suivante est donc naturelle.

\begin{definition}
	Soit $\gamma\colon [a,b]\to \eR^n$, un chemin de classe $C^1$ tel que son image est contenue dans le domaine $D$. Si $\omega$ es une $1$-forme différentielle sur $D$, nous définissons l'\defe{intégrale de $\omega$ le long de $\gamma$}{intégrale!d'une forme différentielle} le nombre
	\begin{equation}
		\begin{aligned}[]
			\int_{\gamma}\omega&=\int_a^b\omega_{\gamma(t)}\big( \gamma'(t) \big)dt\\
				&=\int_a^b\Big[ a_1\big( \gamma(t) \big)\gamma'_1(t)+\ldots +  a_n\big( \gamma(t) \big)\gamma'_n(t) \Big]dt.
		\end{aligned}
	\end{equation}
\end{definition}

Cette définition est une bonne définition parce que si on change la paramétrisation du chemin, on ne change pas la valeur de l'intégrale, c'est la proposition suivante.
\begin{proposition}
	Si $\gamma$ et $\beta$ sont des chemins équivalents, alors
	\begin{equation}
		\int_{\gamma}\omega=\int_{\beta}\omega,
	\end{equation}
	c'est à dire que l'intégrale est invariante sous les reparamétrisation du chemin.
\end{proposition}
\begin{proof}
	Deux chemins sont équivalents quand il existe un difféomorphisme $C^1$ $h\colon [a,b]\to [c,d]$ tel que $\gamma(t)=(\beta\circ h)(t)$. En remplaçant $\gamma$ par $(\beta\circ h)$ dans la définition de $\int_{\gamma}\omega$, nous trouvons
	\begin{equation}
		\int_a^b\omega_{\gamma(t)}\big( \gamma'(t) \big)dt=\int_a^b\omega_{(\beta\circ h)(t)}\big( (\beta\circ h)'(t) \big)dt.
	\end{equation}
	Un changement de variable $u=h(t)$ transforme cette dernière intégrale en $\int_{\beta}\omega$, ce qui prouve la proposition.
\end{proof}

\begin{remark}
	Si $\gamma$ est une somme de chemins, $\gamma=\gamma^{(1)}+\ldots+\gamma^{(n)}$, où chacun des $\gamma^{(i)}$ est un chemin, alors
	\begin{equation}
		\int_{\gamma}\omega=\sum_{i=1}^n\int_{\gamma_i}\omega
	\end{equation}
	parce que $\omega$ est linéaire.
\end{remark}

\begin{remark}
	Si $-\gamma$ est le chemin
	\begin{equation}
		\begin{aligned}
			- \gamma\colon [a,b]&\to \eR^n \\
			t&\mapsto \gamma\big( b-(t-a) \big),
		\end{aligned}
	\end{equation}
	alors
	\begin{equation}
		\int_{-\gamma}\omega=-\int_{\gamma}\omega,
	\end{equation}
	c'est à dire que si l'on parcours le chemin en sens inverse, alors on change le signe de l'intégrale.
\end{remark}

L'intégrale d'une forme différentielle sur un chemin est compatible avec l'intégrale déjà connue d'un champ de vecteur sur le chemin parce que si $G$ est un champ de vecteurs,
\begin{equation}
	\int_{\gamma}G^{\flat}=\int_{\gamma}G.
\end{equation}
En effet,
\begin{equation}
	\begin{aligned}[]
		\int_{\gamma G^{\flat}}	&=\int_a^b G_{\gamma(t)}^{\flat}(\gamma'(t))\\
					&=\int_a^b\big[ G_1( \gamma(t) )dx_1+\ldots G_n(\gamma(t))dx_n \big]\big( \gamma'_1(t),\ldots,\gamma'_n(t) \big)\\
					&=\int_{a}^b\langle G(\gamma(t)), \gamma'(t)\rangle \\
					&=\int_{\gamma}G.
	\end{aligned}
\end{equation}


\begin{proposition}
	Soit $\omega=df$, une $1$-forme exacte et continue sur le domaine $D$. Alors la valeur de $\int_{\gamma}df$ ne dépend que des valeurs de $f$ aux extrémités de $\gamma$.
\end{proposition}

\begin{proof}
	Nous avons
	\begin{equation}
		\begin{aligned}[]
			\int_{\gamma}\omega=\int_{\gamma}df&=\int_{a}^b\sum_{i=1}n\frac{ \partial f }{ \partial x_i }\big( \gamma(t) \big)\gamma'_i(t)dt\\
				&=\int_a^b\frac{ d }{ dt }\Big( (f\circ\gamma)(t) \Big)dt\\
				&=(f\circ\gamma)(b)-(f\circ\gamma(a)).
		\end{aligned}
	\end{equation}
\end{proof}

%---------------------------------------------------------------------------------------------------------------------------
\subsection{Interprétation physique : travail}
%---------------------------------------------------------------------------------------------------------------------------

\begin{definition}
	Une force $F\colon D\subset\eR^n\to \eR^n$ est \defe{\href{http://fr.wikipedia.org/wiki/Force_conservative}{conservative}}{Conservative} si elle dérive d'un potentiel, c'est à dire si il existe une fonction $V\in C^1(D,\eR)$ telle que 
	\begin{equation}
		F(x)=(\nabla V)(x).
	\end{equation}
\end{definition}
Étant donné que $F$ est un champ de vecteurs, nous avons une forme différentielle associée $F^{\flat}$,
\begin{equation}
	F^{\flat}_x\colon x\mapsto \langle F(x), v\rangle .
\end{equation}

\begin{lemma}
	Le champ $F$ est conservatif si et seulement si la $1$-forme différentielle $F^{\flat}$ est exacte.
\end{lemma}

\begin{proof}
	Supposons que la force $F$ soit conservative, c'est à dire qu'il existe une fonction $V$ telle que $F=\nabla V$. Dans ce cas, il est facile de prouver que $F^{\flat}$ est exacte et est donnée par $F_x^{\flat}=dV(x)$. En effet,
	\begin{equation}
		\begin{aligned}[]
			F_x^{\flat}(v)	&=\langle F(x), v\rangle \\
					&=F_1(x)v_1+\ldots+F_n(x)v_n\\
					&=\frac{ \partial V }{ \partial x_1 }(x)v_1+\ldots\frac{ \partial V }{ \partial x_n }(x)v_n\\
					&=dV(x)v.
		\end{aligned}
	\end{equation}
	
	Pour le sens inverse, supposons que $F^{\flat}$ soit exacte. Dans ce cas, nous avons une fonction $V$ telle que $F^{\flat}=dV$. Il est facile de prouver qu'alors, $F=\nabla V$.
\end{proof}
En résumé, nous avons deux façons équivalentes d'exprimer que la force $F$ dérive du potentiel $V$ :  soit nous disons $F=\nabla V$, soit nous disons $F^{\flat}=dV$.

\begin{proposition}
	Si $F$ est une force conservative, alors le \href{http://fr.wikipedia.org/wiki/Travail_d'une_force}{travail} de $F$ lors d'un déplacement ne dépend pas du chemin suivit.
\end{proposition}

\begin{proof}
	Le travail d'une force le long d'un chemin n'est autre que l'intégrale de la force le long du chemin, et le calcul est facile :
	\begin{equation}
		W_{\gamma}(F)=\int_{\gamma}F=\int_{\gamma}dV=V\big( \gamma(b) \big)-V\big( \gamma(a) \big).
	\end{equation}
	Donc si $\beta$ est un autre chemin tel que $\beta(a)=\gamma(a)$ et $\beta(b)=\gamma(b)$, nous avons $W_{\beta}(F)=W_{\gamma}(F)$.
\end{proof}

%+++++++++++++++++++++++++++++++++++++++++++++++++++++++++++++++++++++++++++++++++++++++++++++++++++++++++++++++++++++++++++
\section{Intégrale sur une variété}
%+++++++++++++++++++++++++++++++++++++++++++++++++++++++++++++++++++++++++++++++++++++++++++++++++++++++++++++++++++++++++++

%---------------------------------------------------------------------------------------------------------------------------
\subsection{Mesure sur une carte}
%---------------------------------------------------------------------------------------------------------------------------

Nous considérons dans cette section uniquement des variétés $M$ de dimension $2$ dans $\eR^3$.  Une particularité de $\eR^3$ (par rapport aux autres $\eR^n$) est qu'il existe le produit vectoriel. 

Si $v$, $w\in\eR^3$, alors le vecteur $v\times w$ est une vecteur normal au plan décrit par $v$ et $w$ qui jouit de l'importante propriété suivante :
\begin{equation}
	\text{aire du parallélogramme}=\| v\times w \|.
\end{equation}
L'aire du parallélogramme construit sur $v$ et $w$ est donnée par la norme du produit vectoriel. Afin de donner une mesure infinitésimale en un point $p\in M$, nous voudrions prendre deux vecteurs tangents à $M$ en $p$, et puis considérer la norme de leur produit vectoriel. Cette idée se heurte à la question du choix des vecteurs tangents à considérer.

Dans $\eR^2$, le choix est évident : nous choisissons $e_x$ et $e_y$, et nous avons $\|e_x\times e_y\|=1$. L'idée est donc de choisir une carte $F\colon W\to F(w)$ autour du point $p=F(w)$, et de choisir les vecteurs tangents qui correspondent à $e_x$ et $e_y$ via la carte, c'est à dire les vecteurs
\begin{equation}
	\begin{aligned}[]
		\frac{ \partial F }{ \partial x }(w),&&\text{et}&&\frac{ \partial F }{ \partial y }(w).
	\end{aligned}
\end{equation}
L'\defe{élément infinitésimal de surface}{élément de surface} sur $M$ au point $p=F(w)$ est alors défini par
\begin{equation}
	d\sigma_F=\|  \frac{ \partial F }{ \partial x }(w)\times\frac{ \partial F }{ \partial y }(w) \|dw,
\end{equation}
et si la partie $A\subset M$ est entièrement contenue dans $F(W)$, nous définissons la \defe{mesure}{mesure!dans une carte} de $A$ par
\begin{equation}		\label{EqDefMuDeuxDF}
	\mu_2(A)=\int_{F^{-1}(A)}d\sigma_F=\int_{F^{-1}(A)}\| \frac{ \partial F }{ \partial x }(w)\times\frac{ \partial F }{ \partial y }(w) \|dw.
\end{equation}
\begin{remark}
	Afin que cette définition ait un sens, nous devons prouver qu'elle ne dépend pas du choix de la carte $F$. En effet, les vecteurs $\partial_xF$ et $\partial_yF$ dépendent de la carte $F$, donc leur produit vectoriel (et sa norme) dépendent également de la carte $F$ choisie. Il faudrait donc un petit miracle pour que le nombre $\mu_2(A)$ donné par \eqref{EqDefMuDeuxDF} soit indépendant du choix de $F$.  Nous allons bientôt voir comme cas particulier du théorème \ref{ThoIntIndepF} que c'est en fait le cas. C'est à dire que si $F$ et $\tilde F$ sont deux cartes qui contiennent $A$, alors
	\begin{equation}
		\int_{F^{-1}(A)}d\sigma_F=\int_{\tilde F^{-1}(A)}d\sigma_{\tilde F}.
	\end{equation}
\end{remark}

%///////////////////////////////////////////////////////////////////////////////////////////////////////////////////////////
\subsubsection{Exemple : la mesure de la sphère}
%///////////////////////////////////////////////////////////////////////////////////////////////////////////////////////////

Nous nous proposons maintenant de calculer la surface de la sphère $S^2=x^2+y^2+z^2=R^2$. L'application $F\colon B( (0,0),R)\to R^3$ donnée par
\begin{equation}
	F(x,y)=\begin{pmatrix}
		x	\\ 
		y	\\ 
		\sqrt{R^2-x^2-y^2}	
	\end{pmatrix}
\end{equation}
est une carte pour une demi sphère. Ses dérivées partielles sont
\begin{equation}
	\begin{aligned}[]
		\frac{ \partial F }{ \partial x }&=\begin{pmatrix}
			1	\\ 
			0	\\ 
			-\frac{ x }{ \sqrt{R^2-x^2-y^2} }	
		\end{pmatrix},
		&\frac{ \partial F }{ \partial y }&=\begin{pmatrix}
			0	\\ 
			1	\\ 
			-\frac{ y }{ \sqrt{R^2-x^2-y^2} }	
		\end{pmatrix}.
	\end{aligned}
\end{equation}
Le produit vectoriel de ces deux vecteurs tangents donne
\begin{equation}
	\frac{ \partial F }{ \partial x }(x,y)\times\frac{ \partial F }{ \partial y }(x,y)=\frac{ x }{ \alpha }e_1+\frac{ y }{ \alpha }e_2+e_3
\end{equation}
où $\alpha=\sqrt{R^2-x^2-y^2}$. En calculant la norme, nous trouvons
\begin{equation}
	\| \frac{ \partial F }{ \partial x }(x,y)\times\frac{ \partial F }{ \partial y }(x,y)\| =\sqrt{  \frac{ R^2 }{ R^2-x^2-y^2 } },
\end{equation}
et en passant aux coordonnées polaires, nous écrivons l'intégrale \eqref{EqDefMuDeuxDF} sous la forme
\begin{equation}
	\int_B\| \partial_xF\times\partial_yF \|=\int_0^{2\pi}d\theta\int_0^R r\sqrt{  \frac{ R^2 }{ R^2-x^2-y^2 } }dr=2\pi R^2,
\end{equation}
qui est bien la mesure de la demi sphère.

%---------------------------------------------------------------------------------------------------------------------------
\subsection{Intégrale sur une carte}
%---------------------------------------------------------------------------------------------------------------------------

Nous pouvons maintenant définir l'intégrale d'une fonction sur une carte de la variété $M$.
\begin{definition}
	Soit $F\colon W\subset \eR^2\to \eR^3$, une carte pour une variété $M$. Soit $A$, une partie de $F(W)$ telle que $A=F(B)$ où $B\subset W$ est mesurable.  Soit encore $f\colon A\to \eR$, une fonction continue. L'\defe{intégrale}{intégrale!d'une fonction sur une carte} de $f$ sur $A$ est le nombre
	\begin{equation}	\label{EqDefIntDeuxDF}
		\int_Af=\int_Afd\sigma_F=\int_{F^{-1}(A)}(f\circ F)(w)\|  \frac{ \partial F }{ \partial x }(w)\times\frac{ \partial F }{ \partial y }(w) \| dw
	\end{equation}
\end{definition}

\begin{remark}
	L'intégrale \eqref{EqDefIntDeuxDF} n'est pas toujours bien définie. Étant donné que $F$ est $C^1$ et que $f$ est continue, l'intégrante est continue. L'intégrale sera donc bien définie par exemple lorsque $B$ est borné et si la fermeture $\bar A$ est un compact contenu dans $F(w)$.
\end{remark}

Le théorème suivant montre que le travail que nous avons fait jusqu'à présent ne dépend en fait pas du choix de carte $F$ effectué.

\begin{theorem}\label{ThoIntIndepF}
	Soient $F\colon W\to F(w)$ et $\tilde F\colon \tilde W\to \tilde F(\tilde W)$, deux cartes de la variété $M$. Soit une partie $A\subset F(W)\cap\tilde F(\tilde W)$ telle que $A=F(B)$ avec $B\subset W$ mesurable.  Alors $A=\tilde F(\tilde B)$ avec $\tilde B\subset\tilde W$ mesurable.

	Si $f$ est une fonction continue, et si $\int_Afd\sigma_F$ existe, alors $\int_Afd\sigma_{\tilde F}$ existe et
	\begin{equation}
		\int_Afd\sigma_F=\int_Afd\sigma_{\tilde F}.
	\end{equation}
\end{theorem}


%---------------------------------------------------------------------------------------------------------------------------
\subsection{Exemples}
%---------------------------------------------------------------------------------------------------------------------------

Intégrons la fonction $f(x,y,z)$ sur le carré $K=\mathopen] 0 , 1 \mathclose[\times \mathopen] 0 , 2 \mathclose[\times\{ 1 \}$. La première carte que nous pouvons utiliser est
\begin{equation}
	\begin{aligned}
		F\colon \mathopen] 0 , 1 \mathclose[\times\mathopen] 0 , 2 \mathclose[&\to K \\
		(x,y)&\mapsto (x,y,1). 
	\end{aligned}
\end{equation}
Nous trouvons aisément les vecteurs tangents qui forment l'élément de surface:
\begin{equation}
	\begin{aligned}[]
		\frac{ \partial F }{ \partial x }&=\begin{pmatrix}
			1	\\ 
			0	\\ 
			0	
		\end{pmatrix},
		&\frac{ \partial F }{ \partial y }&=\begin{pmatrix}
			0	\\ 
			1	\\ 
			0	
		\end{pmatrix},
	\end{aligned}
\end{equation}
donc $d\sigma_F=1\cdot dxdy$, et
\begin{equation}		\label{IntKSurcarrUn}
	\int_Kfd\sigma_F=\int_{\mathopen] 0 , 1 \mathclose[\times\mathopen] 0 , 2 \mathclose[}f(x,y,1)\cdot 1\cdot dxdy.
\end{equation}

Nous pouvons également utiliser la carte
\begin{equation}
	\begin{aligned}
		\tilde F\colon \mathopen] 0 , \frac{ 1 }{2} \mathclose[\times\mathopen] 0 , 6 \mathclose[&\to K \\
		(\tilde x,\tilde y)&\mapsto (2\tilde x,\frac{ \tilde y }{ 3 },1). 
	\end{aligned}
\end{equation}
Les vecteurs tangents sont maintenant
\begin{equation}
	\begin{aligned}[]
		\frac{ \partial \tilde F }{ \partial \tilde x }&=\begin{pmatrix}
			2	\\ 
			0	\\ 
			0	
		\end{pmatrix},
		&\frac{ \partial \tilde F }{ \partial \tilde y }&=\begin{pmatrix}
			0	\\ 
			1/3	\\ 
			0	
		\end{pmatrix},
	\end{aligned}
\end{equation}
de telle façon à ce que $d\sigma_{\tilde F}=\| \frac{ 2 }{ 3 }e_3 \|=\frac{ 2 }{ 3 }$. Cette fois, l'intégrale de $f$ sur $K$ s'écrit
\begin{equation}
	\int_Kfd\sigma_{\tilde F}=\int_{\mathopen] 0 , \frac{ 1 }{2} \mathclose[\times\mathopen] 0 , 6 \mathclose[}f\big( 2\tilde x,\frac{ \tilde y }{ 3 },1 \big)\cdot\frac{ 2 }{ 3 }\cdot d\tilde xs\tilde y.
\end{equation}
Conformément au théorème \ref{ThoIntIndepF}, cette dernière intégrale est égale à l'intégrale \eqref{IntKSurcarrUn} parce qu'il s'agit juste d'un changement de variable.

%---------------------------------------------------------------------------------------------------------------------------
\subsection{Orientation}
%---------------------------------------------------------------------------------------------------------------------------

Soient $F\colon W\to F(w)$ et $\tilde F\colon \tilde W\to \tilde F(\tilde W)$, deux cartes de la variété $M$. Nous pouvons considérer la fonction $h=\tilde F^{-1}\circ F$, définie uniquement sur l'intersection des cartes :
\begin{equation}
	h\colon F^{-1}\big( F(W)\cap\tilde F(\tilde W) \big)\to \tilde F^{-1}\big( F(W)\cap\tilde F(\tilde W) \big).
\end{equation}
Nous disons que $F$ et $\tilde F$ ont même \defe{orientation}{orientation} si
\begin{equation}
	J_h(w)>0
\end{equation}
pour tout $w\in  F^{-1}\big( F(W)\cap\tilde F(\tilde W) \big)$.

Considérons les deux carte suivantes pour le même carré:
\begin{equation}
	\begin{aligned}
		F\colon\mathopen] 0 , 1 \mathclose[\times \mathopen] 0 , 1 \mathclose[ &\to \eR^3 \\
		(x,y)&\mapsto (x,y,0) 
	\end{aligned}
\end{equation}
et
\begin{equation}
	\begin{aligned}
		\tilde F\colon\mathopen] 0 , \frac{ 1 }{2} \mathclose[\times\mathopen] 0 , \frac{1}{ 3 } \mathclose[ &\to \eR^3 \\
		(x,y)&\mapsto (2x,3y,0) 
	\end{aligned}
\end{equation}
Ici, $h(x,y)=\left( \frac{ x }{ 2 },\frac{ y }{ 3 } \right)$ et nous avons $J_h=\frac{1}{ 6 }>0$. Ces deux cartes ont même orientation. Notez que
\begin{equation}
	\frac{ \partial F }{ \partial x }\times\frac{ \partial F }{ \partial y }=e_3,
\end{equation}
tandis que
\begin{equation}
	\frac{ \partial \tilde F }{ \partial x }\times\frac{ \partial \tilde F }{ \partial y }=6e_3.
\end{equation}
Les vecteurs normaux à la paramétrisation pointent dans le même sens.

Si par contre nous prenons la paramétrisation
\begin{equation}
	\begin{aligned}
		G\colon \mathopen] 0,1 \mathclose[\times\mathopen] 0,1 ,  \mathclose[&\to \eR^2 \\
		(x,y)&\mapsto (x,(1-y),0), 
	\end{aligned}
\end{equation}
nous avons
\begin{equation}
	\frac{ \partial G }{ \partial x }\times\frac{ \partial G }{ \partial y }=-e_3,
\end{equation}
et si $g=G^{-1}\circ F$, alors $J_g=-1$.

L'orientation d'une carte montre donc si le vecteur normal à la surface pointe d'un côté ou de l'autre de la surface.

\begin{definition}
	Une variété $M$ est \defe{orientable}{orientable!variété} si il existe un atlas de $M$ tel que deux cartes quelconques ont toujours même orientation. Une variété est \defe{orientée}{variété !orientée} lorsque qu'un tel choix d'atlas est fait.
\end{definition}

\begin{proposition}
	Soit $M$, une variété orientable et un atlas orienté $\{ F_i\colon W_i\to \eR^3 \}$. Alors le vecteur unitaire
	\begin{equation}
		\frac{   \frac{ \partial F }{ \partial x }(x,y)\times\frac{ \partial F }{ \partial y }(x,y)   }{ \| \frac{ \partial F }{ \partial x }(x,y)\times\frac{ \partial F }{ \partial y }(x,y)\| }
	\end{equation}
	ne dépend pas du choix de $F$ parmi les $F_i$.
\end{proposition}


\begin{proof}
	Considérons deux cartes $F_1$ et $F_2$, ainsi que l'application $h=F_2^{-1}\circ F_1$. Écrivons le vecteur $\partial_x F_1\times\partial_yF_1$ en utilisant $F_1=F_2\circ h$. D'abord, par la règle de dérivation de fonctions composées,
	\begin{equation}
		\frac{ \partial (F_2\circ h) }{ \partial x }=\frac{ \partial F_2 }{ \partial x }\frac{ \partial h_1 }{ \partial x }+\frac{ \partial F_2 }{ \partial y }\frac{ \partial h_2 }{ \partial x }.
	\end{equation}
	Après avoir fait le même calcul pour $\frac{ \partial (F_2\circ h) }{ \partial y }$, nous pouvons écrire
	\begin{equation}
		\partial_x(F_2\circ h)\times\partial_y(F_2\circ h)=(\partial_xh_1\partial_xF_2+\partial_xh_2\partial_yF_2)\times(\partial_yh_1\partial_xF_2+\partial_yh_2\partial_yF_2).
	\end{equation}
	Dans cette expression, les facteurs $\partial_ih_j$ sont des nombres, donc ils se factorisent dans les produits vectoriels. En tenant compte du fait que $\partial_xF_2\times\partial_xF_2=0$ et $\partial_yF_2\times\partial_yF_2=0$, ainsi que de l'antisymétrie du produit vectoriel, l'expression se réduit à
	\begin{equation}
		\left( \frac{ \partial F_2 }{ \partial x }\times\frac{ \partial F_2 }{ \partial y } \right)(\partial_xh_1\partial_yh_2-\partial_xh_2\partial_yh_2).
	\end{equation}
	Par conséquent,
	\begin{equation}
		\frac{ \partial F_1 }{ \partial x }\times\frac{ \partial F_1 }{ \partial y } =\frac{ \partial (F_2\circ h) }{ \partial x }\times\frac{ \partial (F_2\circ h) }{ \partial y } =\left( \frac{ \partial F_2 }{ \partial x }\times\frac{ \partial F_2 }{ \partial y } \right)\det J_h.
	\end{equation}
	Donc, tant que $J_h$ est positif, les vecteurs unitaires correspondants au membre de gauche et de droite sont égaux.
\end{proof}

\begin{corollary}
	Si nous avons choisit un atlas orienté pour la variété $M$, nous avons une fonction continue $G\colon M\to \eR^3$ telle que $\| G(p) \|=1$ pour tout $p\in M$. Cette fonction est donnée par
	\begin{equation}		\label{DefCarteGOritn}
		G(F(x,y))=\frac{   \frac{ \partial F }{ \partial x }(x,y)\times\frac{ \partial F }{ \partial y }(x,y)   }{ \| \frac{ \partial F }{ \partial x }(x,y)\times\frac{ \partial F }{ \partial y }(x,y)\| }
	\end{equation}
	sur l'image de la carte $F$.
\end{corollary}

\begin{proof}
	La fonction $G$ est construite indépendamment sur chaque carte $F(W)$ en utilisant la formule \eqref{DefCarteGOritn}. Cette fonction est une fonction bien définie sur tout $M$ parce que nous venons de démontrer que sur $F_1(W_1)\cap F_2(W_2)$, les fonctions construites à partir de $F_1$ et à partir de $F_2$ sont égales.
\end{proof}

Il est possible que prouver, bien que cela soit plus compliqué, que la réciproque est également vraie.
\begin{proposition}
	Une variété $M$ de dimension $2$ dans $\eR^3$ est orientable si et seulement si il existe une fonction continue $G\colon M\to \eR^3$ telle que pour tout $p\in M$, le vecteur $G(p)$ soit de norme $1$ et normal à $M$ au point $p$.
\end{proposition}

%---------------------------------------------------------------------------------------------------------------------------
\subsection{Formes différentielles}
%---------------------------------------------------------------------------------------------------------------------------

Nous allons donner une toute petite introduction aux formes différentielles sur des variétés compactes.

\begin{lemma}[\cite{SpindelGeomDoff}]       \label{LemdwLGFG}
    Soit \( \omega\) une \( k\)-forme sur \( \eR^n\) et \( f\), une fonction \( C^{\infty}\) sur \( \eR^n\). Alors \( d(f^*\omega)=f^*d\omega\).
\end{lemma}

\begin{proof}
    Nous effectuons la preuve par récurrence sur le degré de la forme. Soit d'abord une \( 0\)-forme, c'est à dire une fonction \( g\colon \eR^n\to \eR\). Nous avons
    \begin{equation}
        d(d^*g)X=d(g\circ f)X=(dg\circ df)X=dg\big( df X \big)=(f^*dg)(X).
    \end{equation}
    
    Supposons maintenant que le résultat soit exact pour toute les \( p-1\) formes et montrons qu'il reste valable pour les \( p\)-formes. Par linéarité de la différentielle nous pouvons nous contenter de considérer la forme différentielle
    \begin{equation}
        \omega=g\,dx^1\wedge\ldots dx^p
    \end{equation}
    où \( g\) est une fonction \(  C^{\infty}\). Pour soulager les notations nous allons noter \( dx^I=dx^1\wedge\ldots dx^{p-1}\). Nous avons
    \begin{subequations}
        \begin{align}
            d(f^*\omega)&=d\big( f^*(gdx^I\wedge dx^p) \big)\\
            &=d\big( f^*(gdx^I)\wedge f^*dx^p \big)\\
            &=d\big( f^*(gdx^I)\big)\wedge f^*dx^p+(-1)^{p-1}f^*(gdx^I)\wedge(f^*dx^p)  \label{gnAnSt}\\
            &=f^*\big( d(gdx^I) \big)\wedge f^*dx^p      \label{xZrfjZ}\\
            &=f^*\big( d(gdx^I)\wedge dx^p \big)\\
            &=f^*d\omega        \label{loWUji}
        \end{align}
    \end{subequations}
    Justifications : \eqref{gnAnSt} est la formule de Leibnitz. \eqref{xZrfjZ} est parce que le second terme est nul : \( d(f^*dx^p)=f^*(d^2x^p)=0\). Nous avons utilisé l'hypothèse de récurrence et le fait que \( d^2=0\). L'étape \eqref{loWUji} est une utilisation à l'envers de la règle de Leibnitz en tenant compte que \( d^2x^p=0\).
\end{proof}

Soit \( M\) une variété de dimension \( n\) et \( \omega\) une \( n\)-forme différentielle
\begin{equation}
    \omega_p=f(p)dx_1\wedge\ldots\wedge dx_n.
\end{equation}
 Si \( (U,\varphi)\) est une carte (\( U\subset\eR^n\) et \( \varphi\colon U\to M\)) alors nous définissons
\begin{equation}
    \int_{\varphi(U)}\omega=\int_{U}f\big( \varphi(x) \big)dx_1\ldots dx_n.     
\end{equation}
Lorsque nous voulons intégrer sur une partie plus grande qu'une carte nous utilisons une partition de l'unité.
\begin{lemma}   \label{LemGPmRGZ}
    Soit \( \{ U_i \}\) un recouvrement de \( M\) par un nombre fini d'ouverts\footnote{Si \( M\) n'est pas compacte, alors il faut utiliser une version un peu plus élaborée du lemme\cite{SpindelGeomDoff}.}. Alors il existe une famille de fonctions \( f_i\in  C^{\infty}(M)\) telle que
    \begin{enumerate}
        \item
            \( \supp f_i\subset U_i\),
        \item
            pour tout \( i\), nous avons \( f_i\geq 0\),
        \item
            pour tout \( p\in M\) nous avons \( \sum_i f_i(p)=1\).
    \end{enumerate}
\end{lemma}
La famille \( (f_i)\) est une \defe{partition de l'unité}{partition!de l'unité} subordonnée au recouvrement \( \{ U_i \}\). Si \( \{ f_i \}\) est une partition de l'unité subordonnée à un atlas de \( M\) nous définissons
\begin{equation}
    \int_M\omega=\sum_i\int_{U_i}f\omega.
\end{equation}
Il est possible de montrer que cette définition ne dépend pas du choix de la partition de l'unité.

\begin{remark}
    Nous ne définissons pas d'intégrale de \( k\)-forme différentielle sur une variété de dimension \( n\neq k\). Le seul cas où cela se fait est le cas de \( 0\)-formes (les fonctions), mais cela n'est pas vraiment un cas particulier vu que les \( 0\)-formes sont associées aux \( n\)-formes de façon évidente.
\end{remark}

%---------------------------------------------------------------------------------------------------------------------------
\subsection{Intégrale d'une fonction sur une variété}
%---------------------------------------------------------------------------------------------------------------------------

Nous supposons à présent que $M$ est une variété compacte de dimension $2$ dans $\eR^3$. La compacité fait que $M$ possède un atlas contenant un nombre fini de cartes $F_i\colon W_i\to F_i(W_i)$. 

Si $A\subset M$ est tel que pour chaque $i$, $A\cap F_i(W_i)=F_i(V_i)$ pour une ensemble $V_i$ mesurable dans $\eR^2$, alors nous considérons
\begin{equation}
	A_1=A\cap F_1(W_2)=F_1(V_1).
\end{equation}
Ensuite, nous construisons $A_2$ en considérant $F_A(W_2)$ et en lui retranchant $A_1$ :
\begin{equation}
	A_2=\big( A\cap F_2(W_2) \big)\cap F_1(V_1).
\end{equation}
En continuant de la sorte, nous construisons la décomposition
\begin{equation}
	A=A_1\cup\ldots\cup A_p
\end{equation}
de $A$ en ouverts disjoints, chacun de ouverts $A_p$ étant compris dans une carte.

Il est possible de prouver que dans ce cas, la définition suivante a un sens et ne dépend pas du choix de l'atlas effectué.
\begin{definition}
	Si $f\colon A\to \eR$ est une fonction continue, alors l'\defe{intégrale}{intégrale!d'une fonction sur une variété} est le nombre
	\begin{equation}
		\int_Af=\sum_{i=1}^p\int_{A_i}fd\sigma_{F_i}.
	\end{equation}
\end{definition}

%+++++++++++++++++++++++++++++++++++++++++++++++++++++++++++++++++++++++++++++++++++++++++++++++++++++++++++++++++++++++++++ 
\section{Intégrales curvilignes}
%+++++++++++++++++++++++++++++++++++++++++++++++++++++++++++++++++++++++++++++++++++++++++++++++++++++++++++++++++++++++++++
\label{secintcurvi}

\subsection{Chemins de classe \texorpdfstring{$C^1$}{C1}}

Soit $p, q\in \eR^n$. Un \defe{chemin}{chemin} $C^1$ par morceaux joignant $p$ à $q$ est une application continue
\begin{equation}
  \gamma : [a,b] \to \eR^n \qquad \gamma(a) = p, \gamma(b) = q
\end{equation}
pour laquelle il existe une subdivision $a = t_0 < t_1 < \ldots < t_{r-1} < t_r = b$ telle que :
\begin{enumerate}
\item la restriction de $\gamma$ sur chaque ouvert $\mathopen]t_i,
  t_{i+1}\mathclose[$ est de classe $C^1$~;
\item pour tout $0 \leq i \leq r$, $\gamma^\prime$ possède une limite
  à gauche (sauf pour $i = 0$) et une limite à droite (sauf pour $i =
  r$) en $t_i$.
\end{enumerate}
Le \defe{chemin $\gamma$ est (globalement) de classe $C^1$}{Chemin!classe $C^2$} si la
subdivision peut être choisie de \og longueur\fg{} $r = 1$.

\begin{remark}
	Si $a$ et $b$ sont des points de
  $\eR^n$, on peut créer le chemin particulier
  \begin{equation*}
    \gamma : [0,1] \to \eR^n : t \mapsto (1-t)a + tb
  \end{equation*}
  qui relie ces points par un segment de droite.
\end{remark}

\subsection{Intégrer une fonction}

Soit $f : D \subset \eR^n \to \eR$ une fonction continue, et $\gamma
: [a,b] \to D$ un chemin $C^1$. On définit \Defn{l'intégrale de $f$
  sur $\gamma$} par
  \begin{equation}    \label{EqhJGRcb}
  \int_\gamma f d s = \int_\gamma f = \int_a^b f(\gamma(t)) \norme{\gamma^\prime(t)} d t.
\end{equation}

\begin{remark}
  Cette définition ne dépend pas de la paramétrisation choisie, ni du
  sens du chemin (échange entre point de départ et point d'arrivée).
\end{remark}

\begin{remark}      \label{RemiqswPd}
    Attention : les intégrales sur des chemins dans \( \eC\) ne sont la même chose. En effet \( \eC\) doit être souvent plutôt traité comme \( \eR\) que comme \( \eR^2\). Si \( \gamma\) est un chemin dans \( \eC\), l'intégrale
    \begin{equation}
        \int_{\gamma}f
    \end{equation}
    doit être comprise comme une généralisation de \( \int_a^bf(x)dx\) et non comme l'intégrale sur un chemin. La différence est qu'en retournant les bornes d'une intégrale usuelle sur \( \eR\) on change le signe, alors qu'en retournant un chemin dans \( \eR^2\), on ne change pas. Bref, la définition est que si \( \gamma\colon \mathopen[ a , b \mathclose]\to \eC\) est un chemin, alors
    \begin{equation}
        \int_{\gamma}f=\int_{\gamma}f(z)dz=\int_a^bf\big( \gamma(t) \big)\gamma'(t)dt.
    \end{equation}
\end{remark}


La formule qui donne la longueur d'un chemin est évidement l'intégrale de la fonction $1$ sur le chemin, c'est à dire
\begin{equation}
	L=\int_a^b\| \gamma'(t) \|dt.
\end{equation}
Si on veut savoir la longueur d'une courbe donnée sous la forme d'une fonction $y=y(x)$, un chemin qui trace la courbe est évidement donné par
\begin{equation}
	\gamma(t)=(t,y(t)),
\end{equation}
et le vecteur tangent au chemin est $\gamma'(t)=(1,y'(t))$. Donc
\begin{equation}
	\| \gamma'(t) \|=\sqrt{1+y'(t)^2},
\end{equation}
et 
\begin{equation}			\label{EqLongFonction}
	L=\int_a^b\sqrt{1+y'(t)^2}.
\end{equation}


\subsection{Intégrer un champ de vecteurs}
Un \Defn{champ de vecteur} est une application $G : \eR^n \to
\eR^n$. On définit l'intégrale de $G$ sur un chemin $\gamma : [a,b]
\to \eR^n$ par
\begin{equation*}
  \int_\gamma G \pardef \int_a^b \scalprod {G(\gamma(t))}{\gamma^\prime(t)} d t.
\end{equation*}

\begin{remark}
  Cette définition ne dépend pas de la paramétrisation choisie, mais
  le signe change selon le sens du chemin.
\end{remark}

%---------------------------------------------------------------------------------------------------------------------------
\subsection{Intégrer une forme différentielle sur un chemin}
%---------------------------------------------------------------------------------------------------------------------------

Une \defe{forme différentielle}{forme!différentielle} sur $\eR^n$ est une application
\begin{equation}
	\begin{aligned}
		\omega\colon \eR^n&\to (\eR^n)^* \\
		x&\mapsto \omega_x 
	\end{aligned}
\end{equation}
qui à chaque point $x$ de $\eR^n$ associe une forme linéaire $\omega_x: \eR^n \to \eR$.

On sait que $\{ d x_i \}_{1\leq i\leq n}$ est une base de
${(\eR^{n})}^{*}$, donc toute forme différentielle s'écrit
\begin{equation*}
  \omega_x = \sum_{i=0}^n a_i(x) d x_i
\end{equation*}
où $a_1,\ldots,a_n$ sont les \Defn{composantes de $\omega$} dans la
base usuelle, et sont des fonctions à valeurs réelles. Pour un vecteur
$v = (v_1,\ldots,v_n)$ on a donc par définition de $d x_i$
\begin{equation*}
  \omega_x v = \sum_{i=0}^n a_i(x) v_i.
\end{equation*}

L'intégrale d'une forme différentielle sur un chemin est définie par
\begin{equation}    \label{EqEFIZyEe}
    \int_\gamma \omega = \int_a^b \omega_{\gamma(t)}\gamma^\prime(t) d t
\end{equation}

\begin{remark}
  Cette définition ne dépend pas de la paramétrisation choisie, mais
  le signe change selon le sens du chemin.
\end{remark}


\subsection{Lien entre forme différentielle et champ vectoriel}
Si $G$ est un champ de vecteurs, on peut définir la forme différentielle
\begin{equation*}
  \omega^G : \eR^n \to {(\eR^n)}^\ast : x \mapsto \left\lbrack \omega^G_x :
  \eR^n \to \eR : v \mapsto \omega^G_x v = \scalprod {G(x)}v \right\rbrack
\end{equation*}
et réciproquement, si $\omega_x = \sum_i a_i(x)d x_i$ est une forme
différentielle on définit le champ de vecteurs
\begin{equation*}
  G^\omega(x) = (a_1(x),\ldots,a_n(x)).
\end{equation*}

Avec ces définitions, pour un chemin $\gamma$ donné on a
\begin{equation*}
  \int_\gamma \omega^G = \int_\gamma G^\omega
\end{equation*}

%---------------------------------------------------------------------------------------------------------------------------
\subsection{Intégrer un champs de vecteurs sur un bord en $2D$}
%---------------------------------------------------------------------------------------------------------------------------

Si $D\subset\eR^2$ est tel que $\partial D$ est une variété de dimension $1$ et tel que $D$ accepte un champ de vecteur normal extérieur unitaire $\nu$. Si nous voulons définir 
\begin{equation}
	\int_{\partial D}G,
\end{equation}
le mieux est de prendre une paramétrisation $\gamma\colon \mathopen[ 0 , 1 \mathclose]\to \eR^2$ et de calculer
\begin{equation}
	\int_0^1 \langle G_{\gamma(t)}, \frac{ \dot\gamma(t) }{ \| \dot\gamma(t) \| }\rangle dt.
\end{equation}
Hélas, cette définition ne fonctionne pas parce que son signe dépend du sens de la paramétrisation $\gamma$. Si la paramétrisation tourne dans l'autre sens, il y a un signe de différence.

Nous allons définir
\begin{equation}		\label{EqIntVectbordDeux}
	\int_{\partial D}G=\int_0^1\langle G_{\gamma(t)}, T(t)\rangle dt
\end{equation}
où $T(t)=\dot\gamma(t)/\| \dot\gamma(t) \|$ et où $\gamma$ est choisit de telle façon à ce que la rotation d'angle $\frac{ \pi }{ 2 }$ amène $\nu$ sur $T$. Cela fixe le choix de sens.

Ce choix de sens aura des répercussions dans l'application de la formule de Green et du théorème de Stokes.

%---------------------------------------------------------------------------------------------------------------------------
\subsection{Intégrer une forme différentielle sur un bord en $2D$}
%---------------------------------------------------------------------------------------------------------------------------

Nous n'allons pas chercher très loin :
\begin{equation}
	\int_{\partial D}\omega=\int_{\partial D}\omega^{\sharp},
\end{equation}
c'est à dire que l'intégrale de la forme différentielle est celle du champ de vecteur associé. Le membre de droite est définit par \eqref{EqIntVectbordDeux}, avec le choix d'orientation qui va avec.

%---------------------------------------------------------------------------------------------------------------------------
\subsection{Intégrer une forme différentielle sur un bord en $3D$}
%---------------------------------------------------------------------------------------------------------------------------

Nous allons maintenant intégrer une forme différentielle sur certains chemins fermés dans $\eR^3$. Soit $F(D)\subset\eR^3$, une variété de dimension $2$ dans $\eR^3$ où $F\colon D\subset\eR^2\to \eR^3$ est la carte. Nous supposons que $D$ vérifie les hypothèses de la formule de Green. Alors nous définissons
\begin{equation}		\label{EqDefIntTroisForBord}
	\int_{F(\partial D)}\omega = \int_{\partial D} F^*\omega
\end{equation}
où $F^*\omega$ est la forme différentielle définie sur $\partial D$ par $(F^*\omega)(v)=\omega\big( dF(v) \big)$.

Cette définition est très abstraite, mais nous n'allons, en pratique, jamais l'utiliser, grâce au théorème de Stokes.

%---------------------------------------------------------------------------------------------------------------------------
\subsection{Intégrer d'un champ de vecteurs sur un bord en $3D$}
%---------------------------------------------------------------------------------------------------------------------------

Encore une fois, nous n'allons pas chercher bien loin :
\begin{equation}
	\int_{F(\partial D)G}=\int_{F(\partial D)}G^{\flat}
\end{equation}
où $G^{\flat}$ est la forme différentielle associée au champ de vecteur. Le membre de droite est définit par l'équation \eqref{EqDefIntTroisForBord}.

%---------------------------------------------------------------------------------------------------------------------------
\subsection{Dérivées croisées et forme différentielle exacte}
%---------------------------------------------------------------------------------------------------------------------------

Nous considérons le problème suivant : trouver une fonction \( f\colon \eR^2\to \eR\) telle que
\begin{subequations}        \label{EqskfgfNr}
    \begin{numcases}{}
        \frac{ \partial f }{ \partial x }=a(x,y)\\
        \frac{ \partial f }{ \partial y }=b(x,y)
    \end{numcases}
\end{subequations}
où \( a\) et \( b\) sont des fonctions supposées suffisamment régulières. Nous savons que ce problème n'a pas de solutions lorsque
\begin{equation}
    \frac{ \partial a }{ \partial y }\neq\frac{ \partial b }{ \partial x }
\end{equation}
parce que cela impliquerait \( \partial^2_{xy}f\neq \partial^2_{yx}f\). Nous sommes en droit de nous demander si la condition
\begin{equation}
    \frac{ \partial a }{ \partial y }=\frac{ \partial b }{ \partial x }
\end{equation}
impliquerait qu'il existe une solution au problème \eqref{EqskfgfNr}. La réponse est oui, et nous allons brièvement la justifier. Pour plus de détails nous vous demandons de chercher un peu \href{http://www.bing.com/search?q=forme+diff\%C3\%A9rentielle+exacte+filetype\%3Apdf&form=QBRE&fit=all}{sur internet} les mots-clefs \emph{forme différentielles exacte}. Vous consulterez également avec profit \cite{DiffExact}.

\begin{proposition}
    Si \( a\) et \( b\) sont des fonctions qui satisfont à la condition
    \begin{equation}
        \frac{ \partial a }{ \partial y }=\frac{ \partial b }{ \partial x },
    \end{equation}
    alors la fonction
    \begin{equation}        \label{EqllhTaT}
        f(x,y)=\int_0^x a(t,0)dt+\int_0^yb(x,t)dt
    \end{equation}
    répond au problème
    \begin{subequations}     
        \begin{numcases}{}
            \frac{ \partial f }{ \partial x }=a(x,y)\\
            \frac{ \partial f }{ \partial y }=b(x,y)
        \end{numcases}
    \end{subequations}
\end{proposition}

La preuve qui suit n'en est pas complètement une parce qu'il manque des justification, notamment au moment de permuter la dérivée et l'intégrale.
\begin{proof}
    La clef de la preuve est le théorème fondamental de l'analyse :
    \begin{equation}
        \int_0^x \frac{ \partial f }{ \partial x }(t,y)dt=f(x,y)
    \end{equation}
    et son pendant par rapport à \( y\) :
    \begin{equation}
        \int_0^y \frac{ \partial f }{ \partial y }(x,t)dt=f(x,y).
    \end{equation}
    En appliquant ces version du théorème fondamental, nous obtenons immédiatement.
    \begin{equation}
        \frac{ \partial f }{ \partial y }=b(x,y).
    \end{equation}
    En ce qui concerne la dérivée par rapport à \( y\),
    \begin{subequations}
        \begin{align}
            \frac{ \partial f }{ \partial x }&=a(x,0)+\int_0^y\frac{ \partial b }{ \partial x }(x,t)dt\\
            &=a(x,0)+\int_0^y\frac{ \partial a }{ \partial y }(x,t)dt\\
            &=a(x,0)+[a(x,t)]_{t=0}^{t=y}\\
            &=a(x,y).
        \end{align}
    \end{subequations}
\end{proof}

En ce qui concerne l'unicité, supposons que \( f\) et \( g\) soient deux solutions au problème. L'équation
\begin{equation}
    \frac{ \partial f }{ \partial x }=a(x,y)=\frac{ \partial g }{ \partial x }
\end{equation}
implique que 
\begin{equation}
    f(x,y)=g(x,y)+C(y)
\end{equation}
où \( C\) est une fonction seulement de \( y\). L'autre équation implique
\begin{equation}
    f(x,y)=g(x,y)+D(x)
\end{equation}
où \( D\) est seulement une fonction de \( x\). En égalisant nous voyons que les fonctions \( C\) et \( D\) doivent être des constantes.

Par conséquent la fonction \( f\) est donnée à une constante près et en réalité la fonction \eqref{EqllhTaT} est suffisante pour répondre au problème de trouver toutes les fonctions dont les dérivées partielles sont données par les fonctions \( a\) et \( b\).

La fonction \( f\) ainsi créée est un \defe{potentiel}{potentiel} pour le champ de force
\begin{equation}
    F(x,y)=\begin{pmatrix}
        a(x,y)    \\ 
        b(x,y)  
    \end{pmatrix}.
\end{equation}
Notez que ce champ de vecteurs est le gradient de \( f\). La question initiale aurait donc pu être posée en les termes suivants : trouver une fonction \( f\) dont le gradient est donné par
\begin{equation}
    \nabla f=\begin{pmatrix}
        a(x,y)    \\ 
        b(x,y)    
    \end{pmatrix}.
\end{equation}
