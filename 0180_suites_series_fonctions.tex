% This is part of Mes notes de mathématique
% Copyright (c) 2011-2013
%   Laurent Claessens
% See the file fdl-1.3.txt for copying conditions.

Bonjour. Si vous êtes ici, c'est sans doute que vous cherchez le résultat qui dit que si \( \int f=0\) c'est que \( f=0\) dans un sens ou dans un autre. C'est le lemme \ref{Lemfobnwt}. Un lemme du genre dans \( L^2\) existe aussi pour \( \int f\varphi=0\) pour tout \( \varphi\). C'est le lemme \ref{LemDQEKNNf}. Et encore un pour \( L^p\) dans la proposition \ref{PropUKLZZZh}.

%+++++++++++++++++++++++++++++++++++++++++++++++++++++++++++++++++++++++++++++++++++++++++++++++++++++++++++++++++++++++++++
\section{Théorie de la mesure}
%+++++++++++++++++++++++++++++++++++++++++++++++++++++++++++++++++++++++++++++++++++++++++++++++++++++++++++++++++++++++++++

Pour la théorie de la mesure, voir entre autres \cite{ProbaDanielLi}.

%---------------------------------------------------------------------------------------------------------------------------
\subsection{Espaces mesurables et mesurés}
%---------------------------------------------------------------------------------------------------------------------------

\begin{definition}  \label{DefjRsGSy}
    Si \( \Omega\) est un ensemble, un ensemble \( \tribA\) de sous-ensembles de \( \Omega\) est une \defe{tribu}{tribu} si 
    \begin{enumerate}
        \item
            \( \Omega\in\tribA\);
        \item
            \( \complement A\in A\) pour tout \( A\in\tribA\);
        \item
            si \( (A_i)_{i\in I}\) est un ensemble au plus dénombrable d'éléments de \( \tribA\), alors \( \sup_{n\geq 1}A_n=\bigcup_{i\in I}A_i\in\tribA\).
    \end{enumerate}
    Le couple \( (\Omega,\tribA)\) est alors un \defe{espace mesuré}{espace!mesuré}.
\end{definition}

\begin{lemma}
    Une tribu est stable par intersections au plus dénombrables.
\end{lemma}

\begin{proof}
    Soit \( (A_i)_{i\in I}\) une famille au plus dénombrable d'éléments de la tribu \( \tribA\). Nous devons prouver que \( \bigcap_{i\in I}A_i\) est également un élément de \( \tribA\). Pour cela nous passons au complémentaire :
    \begin{equation}
        \complement\left( \bigcap_{i\in I}A_i \right)=\bigcup_{i\in I}\complement A_i.
    \end{equation}
    La définition d'une tribu implique que le membre de droite est un élément de la tribu. Par stabilité d'une tribu par complémentaire, l'ensemble \( \bigcap_{i\in I}A_i\) est également un élément de la tribu.
\end{proof}

La tribu que nous utiliserons toujours dans \( \eR^d\) est la tribu des \defe{boréliens}{boréliens}, notée \( \Borelien(\eR^d)\), qui est la tribu engendrée par les ouverts de \( \eR^d\). Une fonction \( f\colon (\Omega,\tribA)\to (\eR^d,\Borelien(\eR^d))\) est \defe{borélienne}{borélienne} si pour tout \( \mO\in\Borelien\), \( f^{-1}(\mO)\in\tribA\).

\begin{definition}
    Soit \( (E,\tribA)\) et \( (F,\tribF)\) deux espaces mesurés. Une fonction \( f\colon E\to F\) est \defe{mesurable}{mesurable!fonction} si pour tout \( \mO\in \tribF\), l'ensemble \( f^{-1}(\mO)\) est dans \( \tribA\).
\end{definition}
Si \( \tribA\) est une tribu sur un ensemble \( E\), nous notons \( m(\tribA)\)\nomenclature[P]{\( m(\tribA)\)}{Ensemble des fonctions \( \tribA\)-mesurables} l'ensemble des fonctions qui sont \( \tribA\)-mesurables.

\begin{remark}
    Le plus souvent les fonctions que nous considérons sont à valeurs réelles. La tribu considérée sur \( \eR\) est presque toujours celle des ensembles mesurables au sens de Lebesgue. Étant donné qu'il est franchement difficile de créer des ensembles non mesurables au sens de Lebesgue, il est franchement difficile de créer des fonction non mesurables à valeurs réelles. L'hypothèse de mesurabilité est donc toujours satisfaite dans les cas pratiques.

    Cependant en probabilités, la tribu considérée sur \( \eR^n\) pour les variables aléatoires est celle des boréliens.
\end{remark}

\begin{definition}  \label{DefBTsgznn}
    Une \defe{\wikipedia{en}{Measure_space}{mesure}}{mesure} sur l'espace mesurable \( (\Omega,\tribA)\) est une application \( \mu\colon \tribA\to \eR\cup\{ \infty \}\) telle que
    \begin{enumerate}
        \item
            \( \mu(A)\geq 0\) pour tout \( A\in\tribA\);
        \item
            \( \mu(\emptyset)=0\);
        \item   \label{ItemQFjtOjXiii}
            \( \mu\left( \bigcup_{i=0}^{\infty}A_i\right)=\sum_{i=0}^{\infty}\mu(A_i)\) si les \( A_i\) sont des éléments de \( \tribA\) deux à deux disjoints.
    \end{enumerate}
    Une mesure est \defe{\( \sigma\)-finie}{mesure!$\sigma$-finie} si il existe un recouvrement dénombrable de \( \Omega\) par des ensembles de mesure finie. Si la mesure est $\sigma$-finie, nous disons que l'espace \( (\Omega,\tribA,\mu)\) est un espace mesuré $\sigma$-fini.
\end{definition}

Si la mesure des \( \sigma\)-finie, nous pouvons choisir le recouvrement croissant pour l'inclusion. En effet si \( (E_n)_{n\in \eN}\) est le recouvrement, il suffit de considérer \( F_n=\bigcup_{k\leq n}E_k\). Ces ensembles \( F_n\) forment tout autant un recouvrement dénombrable, mais il est évidemment croissant.

\begin{example}
    La mesure de comptage \( m\) sur \( \eN\) est \( \sigma\)-finie parce que \( E_n=\{ 0,\ldots, n \}\) est de mesure finie et \( \bigcup_{n\in \eN}E_n=\eN\).
\end{example}

\begin{example}
    La mesure de Lebesgue sur \( \eR^n\) est \( \sigma\)-finie parce que les boules de rayon \( n\) forment un ensemble dénombrable d'ensembles de mesures finies dont l'union est évidemment tout \( \eR^n\).

    L'intervalle \( I=\mathopen[ 0 , 1 \mathclose]\) muni de la tribu de toutes ses parties et de la mesure de comptage n'est pas un espace mesuré \( \sigma\)-fini.
\end{example}

\begin{example}
    L'intégration à la Riemann n'est pas dans la théorie des espaces mesurés. En effet l'ensemble 
    \begin{equation}
        \tribA=\{   A\subset\mathopen[ 0 , 1 \mathclose]\tq  \text{\( \mtu_A\) est intégrable au sens de Riemann}   \}
    \end{equation}
    n'est pas une tribu. Par exemple les singletons en font partie tandis que \( \mathopen[ 0 , 1 \mathclose]\cap \eQ\) n'en fait pas partie alors que c'est une union dénombrable de singletons.
\end{example}

\begin{definition}
    Si \( \mu\) est une mesure nous disons qu'une propriété est vraie \( \mu\)-\defe{presque partout}{presque partout} si elle est fausse seulement sur un ensemble de mesure nulle.
\end{definition}

Par exemple la fonction de Dirichlet est presque partout égale à la fonction \( 1\) (pour la mesure de Lebesgue).

\begin{proposition}     \label{PropNCMToWI}
    Deux fonctions continue égales presque partout pour la mesure de Lebesgue\footnote{Définition \ref{DefKTzOlyH}.} sont égales.
\end{proposition}

\begin{proof}
    Soient \( f\) et \( g\) deux fonctions continues telles que \( f(x)=g(x)\) pour presque tout \( x\in D\). La fonction \( h=f-g\) est alors presque partout nulle et nous devons prouver qu'elle est nulle sur tout \( D\). La fonction \( h\) est continue; si \( h(a)\neq 0\) pour un certain \( a\in D\) alors \( h\) est non nulle sur un ouvert autour de \( a\) par continuité et donc est non nulle sur un ensemble de mesure non nulle.
\end{proof}

\begin{definition}
    Une application entre espace mesurés
    \begin{equation}
        f\colon (\Omega,\tribA)\to (\Omega',\tribA')
    \end{equation}
    est \defe{mesurable}{mesurable!application} si pour tout \( B\in\tribA'\), l'ensemble \( f^{-1}(B)\) est dans \( \tribA\).
\end{definition}

Si \( \mu\) est une mesure sur \( \eR^d\), une fonction \( f\colon \eR^d\to \eR\) est une \defe{densité}{densité d'une mesure} si pour tout \( A\subset\eR^d\) nous avons
\begin{equation}
    \mu(A)=\int_Af(x)dx
\end{equation}
où \( dx\) est la mesure de Lebesgue.

\begin{lemma}   \label{LemIDITgAy}
    Une union dénombrable d'ensemble de mesure nulle est de mesure nulle.
\end{lemma}

\begin{proof}
    C'est une conséquence immédiate du point \ref{ItemQFjtOjXiii} de la définition d'une mesure : si les \( A_i\) sont de mesure nulle,
    \begin{equation}
        \mu\left( \bigcup_{i=1}^{\infty}A_i \right)\leq \mu(A_i)=0
    \end{equation}
\end{proof}

\begin{theorem}[Théorème d'approximation\cite{YHRSDGc}]     \label{ThoAFXXcVa}
    Soit \( (X,\tribB,\mu)\) un espace mesuré où \( \tribB\) sont les boréliens de \( X\). Soit \( A\in \tribB\) tel que \( A\subset W\) où \( W\) est un ouvert avec \( \mu(W)<\infty\). Soit aussi \( \epsilon>0\).
    \begin{enumerate}
        \item
            Il existe un fermé \( F\) et un ouvert \( V\) tels que \( \mu(V)<\infty\) et
            \begin{equation}
                F\subset A\subset V
            \end{equation}
            et \( \mu(V\setminus F)<\epsilon\).
        \item
            Il existe \( f\in C^0(X,\eR)\) nulle hors de \( W\) vérifiant \( 0\leq f\leq 1\) et
            \begin{equation}
                \int_X| \mtu_A-f |^pd\mu(x)<\epsilon.
            \end{equation}
    \end{enumerate}
\end{theorem}

%---------------------------------------------------------------------------------------------------------------------------
\subsection{Théorème de récurrence}
%---------------------------------------------------------------------------------------------------------------------------

Soit \( X\) un espace mesurable, \( \mu\) une mesure finie sur \( X\) et \( \phi\colon X\to X\) une application mesurable préservant la mesure, c'est à dire que pour tout ensemble mesurable \( A\subset X\),
\begin{equation}
    \mu\big( \phi^{-1}(A) \big)=\mu(A).
\end{equation}
Si \( A\subset X\) est un ensemble mesurable, un point \( x\in A\) est dit \defe{récurrent}{récurrent!point d'un système dynamique} par rapport à \( A\) si et seulement si pour tout \( p\in \eN\), il existe \( k\geq p\) tel que \( \phi^k(x)\in A\).

\begin{theorem}[\wikipedia{fr}{Théorème_de_récurrence_de_Poincaré}{Théorème de récurrence de Poincaré}.]     \label{ThoYnLNEL}
    Si \( A\) est mesurable dans \( X\), alors presque tous les points de \( A\) sont récurrents par rapport à \( A\).
\end{theorem}

\begin{proof}
    Soit \( p\in \eN\) et l'ensemble
    \begin{equation}
        U_p=\bigcup_{k=p}^{\infty}\phi^{-k}(A)
    \end{equation}
    des points qui repasseront encore dans \( A\) après \( p\) itérations  de \( \phi\). C'est un ensemble mesurable en tant que union d'ensembles mesurables (pour rappel, les tribus sont stables par union dénombrable, comme demandé à la définition \ref{DefjRsGSy}), et nous avons donc
    \begin{equation}
        \mu(U_p)\leq \mu(X)<\infty.
    \end{equation}
    De plus \( U_p=\phi^{-p}(U_0)\), donc \( \mu(U_p)=\mu(U_0)\). Vu que \( U_p\subset U_p\), nous avons
    \begin{equation}
        \mu(U_0\setminus U_p)=0.
    \end{equation}
    Étant donné que \( A\subset U_0\) nous avons a fortiori que
    \begin{equation}
        \{ x\in A\tq x\notin U_p \}\subset U_0\setminus U_p,
    \end{equation}
    et donc
    \begin{equation}
        \mu\{ x\in A\tq x\notin U_p \}=0.
    \end{equation}
    Cela signifie exactement que l'ensemble des points \( x\) de \( A\) tels que aucun des \( \phi^k(x)\) avec \( k\geq p\) n'est dans \( A\) est de mesure nulle.
\end{proof}

%---------------------------------------------------------------------------------------------------------------------------
\subsection{Mesure produit}
%---------------------------------------------------------------------------------------------------------------------------

\begin{definition}      \label{DefTribProfGfYTuR}
    Si \( \tribA\) et \( \tribB\) sont deux tribus sur deux ensembles \( \Omega_1\) et \( \Omega_2\), nous définissons la \defe{tribu produit}{tribu!produit} \( \tribA\otimes\tribB\) comme étant la tribu engendrée par 
    \begin{equation}
        \{ A\times B\tq A\in\tribA,B\in\tribB \}.
    \end{equation}
\end{definition}

\begin{theorem}[\cite{FubiniBMauray,MesIntProbb}]
    Soient \( \mu_i\) des mesures $\sigma$-finies sur \( (\Omega_i,\tribA_i)\) (\( i=1,2\)). Il existe une et une seule mesure, notée \( \mu_1\otimes \mu_2\), sur \( (\Omega_1\times\Omega_2,\tribA_1\otimes\tribA_2)\) telle que
    \begin{equation}
        (\mu_1\otimes\mu_2)(A_1\times A_2)=\mu_1(A_1)\mu_2(A_2)
    \end{equation}
    pour tout \( A_1\in \tribA_1\) et \( A_2\in\tribA_2\).
\end{theorem}
\index{mesure!produit}

%---------------------------------------------------------------------------------------------------------------------------
\subsection{Intégrale par rapport à une mesure}
%---------------------------------------------------------------------------------------------------------------------------

\begin{lemma}[Limite croissante de fonctions étagées mesurables]    \label{LemYFoWqmS}
    Soit \( f\colon (\Omega,\tribA)\to \eR\) une fonction mesurable. Il existe une suite \( f_n\colon \Omega\to \eR\) de fonctions étagées telles que \( f_n\to f\) ponctuellement et \( | f_n |<f\).
\end{lemma}

\begin{proof}
    Nous considérons \( (q_n)\) une suite parcourant tous les rationnels\footnote{Nous rappelons que \( \eQ\) est dénombrable et dense dans \( \eR\).}.
    %TODO : démontrer ou référentier le dénombrable et le dense.
    Pour \( n\in \eN\) nous définissons la fonction
    \begin{equation}
        f_n(\omega)=\begin{cases}
            \max\{ q_i\tq i\leq n,\, q_i\leq f(\omega) \}    &   \text{si \( f(\omega)\geq 0\)}\\
            \min\{ q_i\tq i\leq n,\, q_i\geq f(\omega) \}    &    \text{si \( f(\omega)< 0\).}
        \end{cases}
    \end{equation}
    La fonction \( f_n\) est étagée parce qu'elle ne prend que \( n\) valeurs différentes. Nous avons aussi par construction que \( | f_n(\omega)|\leq |f(\omega) |\). Nous avons aussi pour tout \( \omega\in \Omega\) que \( f_n(\omega)\to f(\omega)\) parce que \( \eQ\) est dense dans \( \eR\).

    En ce qui concerne la mesurabilité de \( f_n\), les étages de \( f_n\) sont les ensembles de la forme \( \{ \omega\in \Omega\tq f(\omega)\in\mathopen[ a , b [ \}\) où \( a\) et \( b\) sont deux éléments de \( \{ q_1,\ldots, q_n \}\) qui sont consécutifs au sens de l'ordre dans \( \eQ\) (et non spécialement au sens de l'ordre d'apparition dans la suite), avec éventuellement \( b=\infty\) si \( a\) est le plus grand. Les ensembles \( \mathopen[ a , b [\) étant mesurables dans \( \eR\) et la fonction \( f\) étant mesurable par hypothèse, les ensembles \( f^{-1}\Big( \mathopen[ a , b [ \Big)\) sont mesurables dans \( (\Omega,\tribA)\).
\end{proof}

Une mesure \( \mu\) sur un espace mesurable \( (\Omega,\tribA)\) permet de définir une fonctionnelle linéaire sur l'ensemble des fonctions mesurables \( \Omega\to \eR\). Cette fonctionnelle linéaire est l'intégrale que nous allons définir à présent.

D'abord nous considérons les fonction \defe{simples}{simple!fonction}\index{fonction!simple}, c'est à dire les fonctions de la forme
\begin{equation}
    f=\sum_{i=1}^Na_i\caract_{E_i}
\end{equation}
où \( a_i\in\eR\) tandis que les \( E_i\) sont des ensembles \( \mu\)-mesurables. Si \( Y\in \tribA\) nous définissons
\begin{equation}
    \int_Yfd\mu=\sum_ia_i\mu(Y\cap E_i).
\end{equation}
Pour une fonction \( \mu\)-mesurable générale \( f\colon \Omega\to \mathopen[ 0 , \infty \mathclose]\) nous définissons l'intégrale de \( f\) sur \( Y\) par
\begin{equation}        \label{EqDefintYfdmu}
    \int_Yfd\mu=\sup\Big\{ \int_Yhd\mu\,\text{où \( h\) est une fonction simple et mesurable telle que \( 0\leq h\leq f\)} \Big\}.
\end{equation}
Maintenant nous définissons
\begin{equation}
    \mu(f)=\int_{\Omega}f
\end{equation}
si \( f\) est une fonction mesurable sur \( \Omega\).

\begin{remark}
    Dans \( \eR^d\), quasiment toutes les fonctions et ensembles sont mesurables. En effet la construction d'ensembles non mesurables demande obligatoirement l'utilisation de l'axiome du choix; de tels ensembles doivent être construits «exprès pour». Il y a très peu de chances pour que vous tombiez sur un ensemble non mesurable de \( \eR^d\) sans que vous ne vous en rendiez compte.

    Par exemple la variable aléatoire 
    \begin{equation}
        X(\omega)=\begin{cases}
            \frac{1}{ \omega }    &   \text{si $ \omega\neq 0$}\\
            \infty    &    \text{$\omega=0$}.
        \end{cases}
    \end{equation}
    est mesurable, mais non intégrable.
\end{remark}

Le lemme suivant nous aide à détecter des fonctions presque partout nulles.
\begin{lemma}   \label{Lemfobnwt}
    Soit \( f\) une fonction mesurable positive ou nulle telle que
    \begin{equation}
        \int_{\Omega}fd\mu=0.
    \end{equation}
    Alors \( f=0\) \( \mu\)-presque partout.
\end{lemma}

\begin{proof}
    L'ensemble des points \( x\in\Omega\) tels que \( f(x)\neq 0\) peut s'écrire comme une union dénombrable disjointe :
    \begin{equation}
        \{ x\in\Omega\tq f(x)\neq 0 \}=\bigcup_{i=0}^{\infty}E_i
    \end{equation}
    avec
    \begin{subequations}
        \begin{align}
            E_0&=\{ x\in\Omega\tq f(x)>1 \}\\
            E_i&=\{ x\in\Omega\tq \frac{1}{ i+1 }\leq f(x)<\frac{1}{ i } \}.
        \end{align}
    \end{subequations}
    Si un des ensembles \( E_i\) est de mesure non nulle, alors nous pouvons considérer la fonction simple \( h(x)=\frac{1}{ i+1 }\mtu_{E_i}\) dont l'intégrale sur \( \Omega\) est strictement positive. Par conséquent le supremum de la définition \eqref{EqDefintYfdmu} est strictement positif.

    Nous savons donc que \( \mu(E_i)=0\) pour tout \( i\). Étant donné que la mesure d'une union disjointe dénombrable est égale à la somme des mesures, nous avons
    \begin{equation}
        \mu\{ x\in\Omega\tq f(x)\neq 0 \}=0,
    \end{equation}
    ce qui signifie que \( f\) est nulle \( \mu\)-presque partout.
\end{proof}

\begin{corollary}   \label{CorjLYiSm}
    Soit \( f\) une fonction mesurable sur l'espace mesuré \( (\Omega,\tribA,\mu)\) telle que
    \begin{equation}
        \int_{\Omega}f\mtu_{f>0}d\mu=0.
    \end{equation}
    Alors \( f\leq 0\) presque partout.
\end{corollary}

\begin{proof}
    Nous avons l'égalité d'ensembles
    \begin{equation}
        \{ f\mtu_{f>0}\neq 0 \}=\{ \mtu_{f>0}\neq 0 \}.
    \end{equation}
    Mais lemme \ref{Lemfobnwt} implique que \( f\mtu_{f>0}\) est nulle presque partout, c'est à dire que la mesure de l'ensemble du membre de gauche est nulle par conséquent
    \begin{equation}
        \mu\{ \mtu_{f>0}\neq 0 \}=0.
    \end{equation}
    Cela signifie que la fonction \( f\) est presque partout négative ou nulle.
\end{proof}

\begin{lemma}   \label{LemPfHgal}
    Soit \( f\) une fonction telle que \( | f(x)|\leq g(x) \) pour tout \( x\in\Omega\). Si \( g\) est intégrable, alors \( f\) est intégrable.
\end{lemma}

\begin{proof}
    Nous décomposons \( f\) en parties positives et négatives :
    \begin{subequations}
        \begin{align}
            A_+&=\{ x\in\Omega\tq f(x)>0 \}\\
            A_-&=\{ x\in\Omega\tq f(x)<0 \}.
        \end{align}
    \end{subequations}
    Nous posons \( f_+(x)=f(x)\mtu_{A_+}\) et \( f_-(x)=f(x)\mtu_{A_-}\). Nous avons \( f=f_+-f_-\) et
    \begin{equation}
        \int_{\Omega}f=\int_{A_+}f+\int_{A_-}f
    \end{equation}
    parce que \( \Omega=A_+\cup A_-\cup\{ x\in\Omega\tq f(x)=0 \}\). Si \( \varphi\) est une fonction simple qui majore \( f_+\) nous avons
    \begin{equation}
        \varphi(x)=\sum_{k}a_k\mtu_{E_k}(x)\leq f(x)\mtu_{A_+}(x)\leq g(x).
    \end{equation}
    Par conséquent le supremum qui définit \( \int f_+\) est inférieur au supremum qui définit \( \int g\). La fonction \( f_+\) est donc intégrable. La même chose est valable pour la fonction \( f_-\).
\end{proof}

\begin{proposition} \label{PropWBavIf}
    Soit \( f\) une fonction positive \( \tribA\)-mesurable et bornée. Alors \( f\) est limite ponctuelle croissante de fonction simples.
\end{proposition}

\begin{proof}
    Soit \( \sigma_n=\{ a_0=0,\ldots, a_{r_n}=n \}\) une subdivision de \( \mathopen[ 0 , n \mathclose]\) en intervalles de taille plus petites que \( 1/n\) choisis de sorte que \( \sigma_{n-1}\subset\sigma_{n}\), et
    \begin{equation}
        f_n(x)=\begin{cases}
            0    &   \text{si \( f(x)>n\)}\\
            a_i    &    \text{sinon}
        \end{cases}
    \end{equation}
    où \( a_i\) est le plus grand élément de \( \sigma_n\) inférieur à \( f(x)\). La fonction \( f_n\) est simple et nous avons pour tout \( x\)
    \begin{equation}
        \lim_{n\to \infty} f_n(x)=f(x)
    \end{equation}
\end{proof}

%---------------------------------------------------------------------------------------------------------------------------
\subsection{Mesure dominée}
%---------------------------------------------------------------------------------------------------------------------------

Soient \( \mu\) et \( \nu\) deux mesures sur le même espace \( \Omega\) et la même tribu \( \tribA\). Nous disons que la mesure \( \mu\) est \defe{dominée}{dominée!mesure}\cite{PersoFeng} par \( \nu\) si pour tout ensemble mesurable \( A\), \( \nu(A)=0\) implique \( \mu(A)=0\).

La mesure \( \mu\) est \defe{portée}{portée!mesure} par l'ensemble \( E\in\tribA\) si pour tout \( A\in\tribA\), 
\begin{equation}
    \mu(A)=\mu(A\cap E).
\end{equation}

Nous écrivons que \( \mu\perp\nu\)\nomenclature[Y]{\( \mu\perp\nu\)}{mesures perpendiculaires} si il existe un ensemble \( E\in\tribA\) tel que \( \mu\) soit porté par \( E\) et \( \nu\) soit porté par \( \complement E\).

\begin{theorem}[Radon-Nikodym\cite{NikoLi}]\index{Radon-Nikodym}
    Soient \( \mu\) et \( m\) deux mesures \( \sigma\)-finies sur un espace métrisable \( (\Omega,\tribA)\).
    \begin{enumerate}
        \item
            Il existe un unique couple de mesures \( \mu_1\) et \( \mu_2\) telles que
            \begin{enumerate}
                \item
                    \( \mu=\mu_1+\mu_2\)
                \item
                    \( \mu_1\) est dominé par \( m\)
                \item
                    \( \mu_2\perp m\).
            \end{enumerate}
            Dans ce cas, les mesures \( \mu_1\) et \( \mu_2\) sont positives et \( \sigma\)-finies.
        \item
            À égalité \(  m\)-presque partout près, il existe une unique fonction mesurable positive \( f\) telle que pour tout mesurable \( A\),
            \begin{equation}
                \mu_1(A)=\int_Ad\mu_1=\int_{\Omega}\mtu_Afd m.
            \end{equation}
        \item
            À égalité \( m\)-presque partout près, il existe une unique fonction positive mesurable \( h\) telle que \( \mu_1=hm\).
    \end{enumerate}
\end{theorem}
%TODO : une preuve

\begin{corollary}   \label{CorZDkhwS}
    Si \( \mu\) es une mesure \( \sigma\)-finie dominée par la mesure \( \sigma\)-finie \( m\), alors \( \mu\) possède une unique fonction de densité.
\end{corollary}

\begin{corollary}       \label{CorDomDens}
    Soient \( \mu\) et \( m\), deux mesures positives \( \sigma\)-finies sur \( (\Omega,\tribA)\). Alors \( m\) domine \( \mu\) si et seulement si \( \mu\) possède une densité par rapport à \( m\).
\end{corollary}
 
\begin{proof}
    Si \( \mu\) est dominée par \( m\), alors la décomposition \( \mu=\mu+0\) satisfait le théorème de Radon-Nikodym. Par conséquent il existe une fonction \( f\) telle que
    \begin{equation}
        \mu(A)=\int_Afdm.
    \end{equation}
    Cette fonction est alors une densité pour \( \mu\) par rapport à \( m\).

    Pour la réciproque, nous supposons que \( \mu\) a une densité \( f\) par rapport à \( m\), et que \( A\) est une ensemble de \( m\)-mesure nulle :
    \begin{equation}
        m(A)=\int_{\Omega}\mtu_Adm=0.
    \end{equation}
    Cela signifie que la fonction \( \mtu_A\) est \( m\)-presque partout nulle. La fonction produit \( \mtu_Af\) est également nulle \( m\)-presque partout, et par conséquent
    \begin{equation}
        \mu(A)=\int_{\Omega}\mtu_Afdm=0.
    \end{equation}
\end{proof}

\begin{probleme}
    Est-ce que la démonstration de cela ne demande pas la convergence monotone d'une façon ou d'une autre ?
\end{probleme}

%---------------------------------------------------------------------------------------------------------------------------
\subsection{Changement de variables dans une intégrale}
%---------------------------------------------------------------------------------------------------------------------------

\begin{theorem} \label{ThomFeRCi}
    Soit \( \mO\) un ouvert de \( \eR^n\) et \( \mO'\) un ouvert de \( \eR^m\). Soit \( \varphi\colon \mO\to \mO'\) un difféomorphisme \( C^1\). Si \( f\colon \mO\to \eR\) est une fonction mesurable, positive et intégrable, alors
    \begin{equation}
        \int_{\mO}f(u)du=\int_{\mO'}f\big( \varphi^{-1}(v) \big)| J_{\varphi^{-1}}(v) |dv.
    \end{equation}
\end{theorem}

%--------------------------------------------------------------------------------------------------------------------------- 
\subsection{Théorème de Fubini-Tonelli et de Fubini}
%---------------------------------------------------------------------------------------------------------------------------

Il existe trois résultats. Le premier, le théorème de Fubini-Tonelli \ref{ThoWTMSthY} demande que la fonction soit mesurable et positive; le second, le théorème de Fubini \ref{ThoFubinioYLtPI} demande que la fonction soit intégrable (mais pas spécialement positive); et le troisième, le corollaire \ref{CorTKZKwP} demande l'intégrabilité de la valeur absolue des intégrales partielles pour déduire que la fonction elle-même est intégrable.

%TODO : des démonstrations de ces trois théorèmes seraient les bienvenues.

Nous rappelons au cas où que \( \eR^n\) muni de la mesure de Lebesgue est un espace mesuré \( \sigma\)-fini, conformément à la définition \ref{DefBTsgznn}.

\begin{theorem}[Fubini-Tonelli\cite{MesIntProbb}]\label{ThoWTMSthY}
    Soient \( (\Omega_i,\tribA_i,\mu_i)\) deux espaces mesurés \( \sigma\)-finis, et \( (\Omega,\tribA,\mu)\) l'espace produit. Soit une fonction mesurable
    \begin{equation}
        f\colon \Omega\to \eR^{+}\cup\{ +\infty \}.
    \end{equation}
    Alors :
    \begin{enumerate}
        \item
            Pour presque tout \( x\in\Omega_1\), la fonction \( y\mapsto f(x,y)\) est mesurable sur \( \Omega_2\).
        \item
            Si nous posons
            \begin{equation}
                \varphi_f(x)=\int_{\Omega_2}f(x,y)d\mu_2(y),
            \end{equation}
            alors \( \varphi_f\) est une fonction bien définie presque partout sur \( \Omega_1\) et \( \varphi\) est mesurable (à valeurs positives).
        \item
            Toutes les intégrales imaginables existent et sont égales :
            \begin{subequations}
                \begin{align}
                \int_{\Omega}fd(\mu_1\otimes \mu_2)&=\int_{\Omega_1}\varphi_fd\mu_1\\
                &=\int_{\Omega_1}\left[ \int_{\Omega_2}f(x,y)d\mu_2(y) \right]d\mu_1(x)\\
                &=\int_{\Omega_2}\left[ \int_{\Omega_1}f(x,y)d\mu_1(x) \right]d\mu_2(y).
                \end{align}
            \end{subequations}
    \end{enumerate}
\end{theorem}
\index{théorème!Fubini-Tonelli}

\begin{theorem}[Fubini\cite{MesIntProbb}]\label{ThoFubinioYLtPI}
    Soient \( (\Omega_i,\tribA_i,\mu_i)\) deux espaces mesurés \( \sigma\)-finis, et \( (\Omega,\tribA,\mu)\) l'espace produit. Soit 
    \begin{equation}
        f\in L^1\big( (\Omega,\tribA),\eR \big),
    \end{equation}
    c'est à dire une fonction à valeurs réelles mesurable et intégrable sur \( \Omega\). Alors :
    \begin{enumerate}
        \item
            Pour presque tout \( x\in \Omega_1\), la fonction \( y\mapsto f(x,y)\) est \( L^1(\Omega_2)\).
        \item
            Si nous posons
            \begin{equation}
                \varphi_f(x)=\int_{\Omega_2}f(x,y)d\mu_2(y);
            \end{equation}
            alors \( \varphi_f\in L^1(\Omega_1)\).
        \item   \label{ItemQMWiolgiii}
            Nous avons la formule d'inversion d'intégrale
            \begin{subequations}
                \begin{align}
                \int_{\Omega}fd(\mu_1\otimes \mu_2)&=\int_{\Omega_1}\varphi_fd\mu_1\\
                &=\int_{\Omega_1}\left[ \int_{\Omega_2}f(x,y)d\mu_2(y) \right]d\mu_1(x)\\
                &=\int_{\Omega_2}\left[ \int_{\Omega_1}f(x,y)d\mu_1(x) \right]d\mu_2(y).
                \end{align}
            \end{subequations}
    \end{enumerate}
\end{theorem}
\index{théorème!Fubini!espace mesuré}

Si la fonction \( (x,y)\mapsto f(x)g(y)\) satisfait aux hypothèse du théorème de Fubini alors
\begin{equation}    \label{EqTJEEsJW}
    \int_{\Omega_1\times \Omega_2} f(x)g(y)dx\otimes dy=\left( \int_{\Omega_1}f(x)dx \right)\left( \int_{\Omega_2}g(y)dy \right).
\end{equation}
Le théorème de Fubini est souvent utilisé sous cette forme.

\begin{corollary}\label{CorTKZKwP}
    Soient \( (\Omega_i,\tribA_i,\mu_i)\) deux espaces mesurés \( \sigma\)-finis, et \( (\Omega,\tribA,\mu)\) l'espace produit. Soit une fonction mesurable \( f\colon \Omega\to \eR\). Alors les conditions suivantes sont équivalentes
    \begin{enumerate}
        \item
            \( f\in L^1(\Omega)\),
        \item
            \begin{equation}
                \int_{\Omega_1}\left[ \int_{\Omega_2}| f |d\mu_2 \right]d\mu_1 <\infty,
            \end{equation}
        \item
            \begin{equation}
                \int_{\Omega_2}\left[ \int_{\Omega_1}| f |d\mu_1 \right]d\mu_2 <\infty.
            \end{equation}
    \end{enumerate}
\end{corollary}
En pratique, lorsqu'on ne sait pas a priori si \( f\) est intégrable sur \( \Omega_1\times \Omega_2\), nous testons l'intégrabilité en chaine de \( | f |\), et si c'est bon, alors nous savons que \( f\) est intégrable sur le produit et qu'on peut permuter les intégrales.

\begin{example}
    Nous montrons que le théorème ne tient pas si une des deux mesures n'est pas \( \sigma\)-finie. Soit \( I=\mathopen[ 0 , 1 \mathclose]\). Nous considérons l'espace mesuré
    \begin{equation}
        (I,\Borelien(I),\lambda)
    \end{equation}
    où \( \Borelien(I)\) est la tribu des boréliens sur \( I\) et \( \lambda\) est la mesure de Lebesgue (qui est $\sigma$-finie). D'autre part nous considérons l'espace mesuré
    \begin{equation}
        (I,\partP(I),m)
    \end{equation}
    où \( \partP(I)\) est l'ensemble des parties de \( I\) et \( m\) est la mesure de comptage. Cette dernière n'est pas $\sigma$-finie parce que les seuls ensembles de mesure finie pour la mesure de comptage sont des ensembles finis, or une union dénombrable d'ensemble finis ne peut pas recouvrir l'intervalle \( I\).

    Nous allons montrer que dans ce cadre, l'intégrale de la fonction indicatrice de la diagonale sur \( I^2\) ne vérifie pas le théorème de Fubini. Étant donné que \( \Borelien(I)\subset\partP(I)\) nous avons
    \begin{equation}
        \Borelien(I^2)\subset\Borelien(I)\otimes\partP(I).
    \end{equation}
    Soit \( \Delta=\{ (x,x)\tq x\in I \}\). La fonction
    \begin{equation}
        \begin{aligned}
            g\colon I^2&\to \eR \\
            (x,y)&\mapsto x-y 
        \end{aligned}
    \end{equation}
    est continue et \( \Delta=g^{-1}(\{ 0 \})\) est donc fermé dans \( I^2\). L'ensemble \( \Delta\) est donc un borélien de \( I^2\) et par conséquent un élément de la tribu \( \Borelien(I)\otimes\partP(I)\). La fonction indicatrice \( \mtu_{\Delta}\) est alors mesurable pour l'espace mesuré
    \begin{equation}
        (I\times I,\Borelien(I)\otimes\partP(I),\lambda\otimes m).
    \end{equation}
    Pour \( x\) fixé nous avons
    \begin{equation}
        \mtu_{\Delta}(x,y)=\begin{cases}
            1    &   \text{si \( y= x\)}\\
            1    &    \text{si \( y\neq x\)}
        \end{cases}=\mtu_{\{ x \}}(y),
    \end{equation}
    et donc
    \begin{subequations}
        \begin{align}
            A_1&=\int_I\left( \int_I\mtu_{\Delta}(x,y)dm(y) \right)d\lambda(x)\\
            &=\int_I\left( \int_I\mtu_{\{ x \}}(y)dm(y) \right)d\lambda(x)\\
            &=\int_I\Big( m(\{ x \}) \Big)d\lambda(x)\\
            &=\int_I 1d\lambda(x)\\
            &=1.
        \end{align}
    \end{subequations}
    Par contre le support de \( \mtu_{\Delta}\) étant de mesure nulle pour la mesure de Lebesgue, nous avons
    \begin{equation}
        \int_I\mtu_{\Delta}(x,y)d\lambda(x)=0
    \end{equation}
    et par conséquent
    \begin{equation}
        A_2=\int_I\left( \int_I\mtu_{\Delta}(x,y)d\lambda(x) \right)dm(y)=0.
    \end{equation}
    Nous voyons donc que le théorème de Fubini ne s'applique pas.
\end{example}

\begin{example} \label{ExrgMIni}
    Le théorème de Fubini est utilisé dans le calcul de l'intégrale gaussienne
    \begin{equation}
        G=\int_{\eR} e^{-x^2}dx,
    \end{equation}
    alors que la fonction \( x\mapsto  e^{-x^2}\) n'a pas de primitives parmi les fonctions élémentaires.

    Par symétrie nous pouvons nous contenter de calculer
    \begin{equation}
        G_+=\int_0^{\infty} e^{-x^2}dx.
    \end{equation}
    L'astuce est de passer par l'intermédiaire
    \begin{subequations}
        \begin{align}
            H&=\int_{\eR^+\times\eR^+} e^{-(x^2+y^2)}dxdy       \label{EqIntFausasub}\\
            &=\int_{\eR^+}\left( \int_{\eR^+} e^{-x^2} e^{-y^2}dx \right)dy\\
            &=\left( \int_{\eR^+} e^{-x^2} dx\right)^2\\
            &=G_+^2
        \end{align}
    \end{subequations}
    L'intégrale \eqref{EqIntFausasub} se calcule en passant aux coordonnées polaires et le résultat est \( H=\frac{ \pi }{ 4 }\). Nous avons alors \( G=\frac{ \sqrt{\pi} }{ 2 }\) et
    \begin{equation}
        \int_{\eR} e^{-x^2}=\sqrt{\pi}.
    \end{equation}
\end{example}

\begin{example}
    Une variante, qui n'applique pas Fubini sur un domaine non borné. Nous commençons par écrire
\begin{equation}
	I=\int_{-\infty}^{+\infty} e^{-x^2} dx := \lim_{R \to +\infty} \int_{-R}^{+R} e^{-x^2} dx 
\end{equation}
et puis nous faisons le calcul
\begin{equation}		\label{EqCalculInteeemoisxcar}
	\begin{aligned}[]
		I^2 &= \lim_{R \to +\infty} \left( (\int_{-R}^{+R} e^{-x^2} dx)( \int_{-R}^{+R} e^{-y^2} dy) \right) \\
		&= \lim_{R \to +\infty} \left( \iint_{K_R}e^{-(x^2+y^2)} dx dy \right) \\
		&= \lim_{R \to +\infty} \left( \iint_{C_R}e^{-(x^2+y^2)} dx dy \right) 
	\end{aligned}
\end{equation}
où $K$ est le carré de demi côté $R$ centré à l'origine et de côtés parallèles aux axes et $C_R$ est le cercle de rayon $R$ centré à l'origine.

	La première étape à justifier est simplement l'application de Fubini. Pour le passage de l'intégrale du carré vers le cercle, définissons
	\begin{equation}
		\begin{aligned}[]
			I_K(r)&=\int_{K_r}f,&I_C(r)&=\int_{C_r}f
		\end{aligned}
	\end{equation}
	où $K_r$ est la carré de demi côté $r$ et $C_r$ est le cercle de rayon $r$. Le demi côté du carré inscrit à $C_r$ est $\sqrt{2}$, donc pour tout $r$ nous avons
	\begin{equation}
		I_K(\sqrt{2}r)\leq I_C(r)<I_K(r),
	\end{equation}
	et en prenant la limite, nous avons évidement
	\begin{equation}
		\lim_{r\to \infty}I_K(\sqrt{2}r)=\lim_{r\to\infty}I_K(r),
	\end{equation}
	de telle façon à ce que cette limite soit également égale à $\lim_{r\to\infty}I_C(t)$.


    Il ne reste qu'à calculer la dernière intégrale sur le cercle en passant aux coordonnées polaires :
	\begin{equation}
        \iint_{C_R} e^{-(x^2+y^2)}dxdy=\int_0^{2\pi}d\theta\int_0^Rr e^{-r^2}dr=\pi(1- e^{-R^2}).
	\end{equation}
	La limite donne $\pi$, nous en déduisons que
    \begin{equation}    \label{EqFDvHTg}
		\int_{-\infty}^{\infty} e^{-x^2}dx=\sqrt{\pi}.
	\end{equation}

\end{example}

\begin{example} \label{ExempInversSumIntFub}   \index{mesure!de comptage}
    Le théorème de Fubini-Tonelli nous permet également d'inverser des sommes et des séries. En effet une somme n'est rien d'autre qu'une intégrale pour la mesure de comptage :
    \begin{equation}
        \sum_{n=0}^{\infty}a_n=\int_{\eN}a_ndm(n).
    \end{equation}
    Considérons une suite de fonctions \( f_n\colon \eR^d\to \eR\) \emph{positives}, la quantité
    \begin{equation}    \label{EqAcalculParFubIntSum}
        I=\sum_{n=0}^{\infty}\int_{\eR^n}f_n(x)dx
    \end{equation}
    et les espaces mesurés \( (\eN,\partP(\eN),m)\), \( (\eR^n,\Borelien(\eR^n),\lambda)\) où \( \lambda\) est la mesure de Lebesgue. En écrivant la formule \eqref{EqAcalculParFubIntSum}, nous supposons que pour chaque \( n\), la fonction \( f_n\) est intégrable sur \( \eR^d\) et que le résultat soit sommable. Nous pouvons la récrire sous la forme
    \begin{equation}
        \int_{\eN}\left( \int_{\eR^n}f(n,x)dx \right)dm(n)
    \end{equation}
    avec la notation évidente \( f(n,x)=f_n(x)\). Prouvons que la fonction \( f\colon \eN\times\eR^d\to \eR\) ainsi définie est une fonction mesurable pour l'espace mesuré
    \begin{equation}
        \big( \eN\times\eR^d,\partP(\eN)\otimes\Borelien(\eR^d),m\otimes\lambda \big).
    \end{equation}
    Si \( A\subset\eR\), nous avons
    \begin{equation}
        f^{-1}(A)=\bigcup_{n\in\eN}\{ n \}\times f_n^{-1}(A).
    \end{equation}
    Chacun des ensembles dans l'union appartient à la tribu \( \partP(\eN)\times\Borelien(\eR^d)\) tandis que les tribus sont stables sous les unions dénombrables. La fonction \( f\) est donc mesurable. La fonction \( f\) est donc mesurable. Comme nous avons supposé que \( f\) était positive, le théorème de Fubini-Tonelli s'applique et nous avons
    \begin{equation}
        I=\int_{\eR^d}\left( \int_{\eN}f(n,x)dm(n) \right)dx=\int_{\eR^d}\sum_{n\in \eN}f_n(x)dx.
    \end{equation}
\end{example}

%+++++++++++++++++++++++++++++++++++++++++++++++++++++++++++++++++++++++++++++++++++++++++++++++++++++++++++++++++++++++++++
\section{Suites de fonctions}
%+++++++++++++++++++++++++++++++++++++++++++++++++++++++++++++++++++++++++++++++++++++++++++++++++++++++++++++++++++++++++++
Source : \cite{TrenchRealAnalisys}.

\begin{definition}
    Nous disons qu'une suite de fonctions \( (f_n)\) définies sur un ensemble \( A\) \defe{converge uniformément}{convergence!uniforme} vers une fonction \( f\) si
    \begin{equation}
        \lim_{n\to \infty} \| f_n-f \|_A=0
    \end{equation}
    où \( \| g \|_A=\sup_{x\in A}\| g(x) \|\).
\end{definition}

\begin{theorem}[Limite uniforme de fonctions continues]			\label{ThoUnigCvCont}
    Soit \( A\), un ensemble mesuré et \( f_n\colon A\to \eR^n\), une suite de fonctions continues convergeant uniformément vers \( f\). Si les fonctions \( f_n\) sont toutes continues en \( x_0\in A\), alors \( f\) est continue en \( x_0\).
\end{theorem}

\begin{proof}
    Soit \( \epsilon>0\). Si \( x\in A\) nous avons, pour tout \( n\), la majoration
    \begin{subequations}
        \begin{align}
            \| f(x)-f(x_0) \|&\leq \| f(x)-f_n(x) \|+\| f_n(x)-f_n(x_0) \|+\| f_n(x_0)-f(x_0) \|\\
            &\leq\| f_n(x)-f_n(x_0) \|+2\| f_n-f \|_{\infty}.
        \end{align}
    \end{subequations}
    Grâce à l'uniforme convergence, nous considérons \(N\in \eN\) tel que \( \| f_n-f \|\leq \epsilon\) pour tout \( n\geq N\). Pour de tels \( n\), nous avons
    \begin{equation}
        \| f(x)-f(x_0) \|\leq 2\epsilon\| f_n-f \|+\| f_n(x)-f_n(x_0) \|.
    \end{equation}
    La continuité de \( f_n\) nous fournit un \( \delta>0\) tel que \( \| f_n(x_0)-f_n(x) \|<\epsilon\) dès que \( \| x-x_0 \|<\delta\). Pour ce \( \delta\), nous avons alors \( \| f(x)-f(x_0) \|<\epsilon\).
\end{proof}

\begin{proposition}[Permuter limite et intégrale]       \label{PropbhKnth}
    Soit \( f_n\to f\) uniformément sur un ensemble mesuré \( A\) de mesure finie. Alors si les fonctions \( f_n\) et \( f\) sont intégrables sur \( A\), nous avons
    \begin{equation}
        \lim_{n\to \infty} \int_A f_n=\int_A \lim_{n\to \infty} f_n.
    \end{equation}
\end{proposition}

\begin{proof}
    Notons \( f\) la limite de la suite \( (f_n)\). Pour tout \( n\) nous avons les majorations
    \begin{subequations}
        \begin{align}
            \left| \int_A f_n d\mu-\int_A fd\mu \right| &\leq \int_A| f_n-f |d\mu\\
            &\leq \int_A \| f_n-f \|_{\infty}d\mu\\
            &=\mu(A)\| f_n-f \|_{\infty}
        \end{align}
    \end{subequations}
    où \( \mu(A)\) est la mesure de \( A\). Le résultat découle maintenant du fait que \( \| f_n-f \|_{\infty}\to 0\).
\end{proof}
Il existe un résultat considérablement plus intéressant que cette proposition. En effet, l'intégrabilité de \( f\) n'est pas nécessaire. Cette hypothèse peut être remplacée soit par l'uniforme convergence de la suite (théorème \ref{ThoUnifCvIntRiem}), soit par le fait que les normes des \( f_n\) sont uniformément bornées (théorème de la convergence dominée de Lebesgue \ref{ThoConvDomLebVdhsTf}).

\begin{theorem}			\label{ThoUnifCvIntRiem}
    La limite uniforme d'une suite de fonctions intégrables sur un borné est intégrable, et on peut permuter la limite et l'intégrale. 
    
    Plus précisément, soit \( A\) un ensemble de \( \mu\)-mesure finie et \( f_n\colon A\to \eR\) des fonctions intégrables sur \( A\). Si la limite \( f_n\to f\) est uniforme, alors \( f\) est intégrable sur \( A\) et nous pouvons inverser la limite et l'intégrale :
    \begin{equation}
        \lim_{n\to \infty} \int_A f_n=\int_A\lim_{n\to \infty} f_n.
    \end{equation}
\end{theorem}
La preuve suivante est inspirée de celle fournie par \href{http://dubois.gilles.pagesperso-orange.fr/analyse_reelle/intlimites.html}{Gilles Dubois} dans le cas de l'intégrale de Riemann sur un intervalle compact.

\begin{proof}
    Soit \( \epsilon>0\) et \( n\) tel que \( \| f_n-f \|_{\infty}\leq \epsilon\) (ici la norme uniforme est prise sur \( A\)). Étant donné que \( f_n\) est intégrable sur \( A\), il existe une fonction simple \( \varphi_n\) qui minore \( f_n\) telle que
    \begin{equation}
        \left| \int_{A}\varphi_n-\int_A f_n \right| <\epsilon.
    \end{equation}
    La fonction \( \varphi_n+\epsilon\) est une fonction simple qui majore la fonction \( f\). Si \( \psi\) est une fonction simple qui minore \( f\), alors
    \begin{equation}
        \int_A\psi\leq\int_A\varphi_n+\epsilon\leq\int_A f_n+\epsilon\mu(A).
    \end{equation}
    Par conséquent le supremum qui définit \( \int_A f\) existe, ce qui montre que \( f\) est intégrable. Le fait qu'on puisse inverser la limite et l'intégrale est maintenant une conséquence de la proposition \ref{PropbhKnth}.
\end{proof}

\begin{remark}
    L'hypothèse sur le fait que \( A\) est de mesure finie est importante. Il n'est pas vrai qu'une suite uniformément convergente de fonctions intégrables est intégrables. En effet nous avons par exemple la suite
    \begin{equation}
        f_n(x)=\begin{cases}
            1/x    &   \text{si \( x<n\)}\\
            0    &    \text{sinon}
        \end{cases}
    \end{equation}
    qui converge uniformément vers \( f(x)=1/x\) sur \( A=\mathopen[ 1 , \infty [\). Le limite n'est cependant pas intégrable sur \( A\).
\end{remark}

\begin{theorem}		\label{ThoSerUnifDerr}
	Soit $U\subset\eR^n$ ouvert, $f_k\colon U\to \eR$ et $f_k$ de classe $C^1$. Supposons que $f_k$ converge simplement vers $f$ et que $\partial_if_k$ converge uniformément sur tout compact  vers une fonction $g_i$ pour $i=1,\ldots,n$. Alors $f$ est de classe $C^1$ et $\partial_if=g_i$. De plus, $f_k$ converge vers $f$ uniformément.
\end{theorem}

\begin{theorem}				\label{ThoSerCritAbel}
	Soit $\sum_{k=1}^{\infty}g_k(x)$, une série de fonctions complexes où $g_k(x)=\varphi_k(x)\psi_k(x)$. Supposons que
	\begin{enumerate}

		\item
			$\varphi_k\colon A\to \eC$ et $| \sum_{k=1}^K\varphi_k(x) |\leq M$ où $M$ est indépendant de $x$ et $K$,
		\item
			$\psi_k\colon A\to \eR$ avec $\psi_k(x)\geq 0$ et pour tout $x$ dans $A$, $\psi_{k+1}(x)\leq \psi_k(x)$, et enfin supposons que $\psi_k(x)$ converge uniformément vers $0$.

	\end{enumerate}
	Alors $\sum_{k=1}^{\infty}g_k$ est uniformément convergente.
\end{theorem}

\begin{theorem}		\label{ThoAbelSeriePuiss}
	Si la série de puissances (réelle) converge en $x=x_0+R$, alors elle converge uniformément sur $\mathopen[ x_0-R+\epsilon , x_0+R \mathclose]$ ($\epsilon>0$) vers une fonction continue.
\end{theorem}

%---------------------------------------------------------------------------------------------------------------------------
\subsection{Convergence de suites de fonctions}
%---------------------------------------------------------------------------------------------------------------------------

Nous considérons un espace normé \( (\Omega,\| . \|)\). Nous disons qu'une suite de fonctions \( f_n\) \defe{converge}{convergence!en norme} vers \( f\) pour la norme \( \| . \|\) si \( \forall \epsilon>0\), \( \exists N\) tel que \( n\geq N\) implique \( \| f_n-f \|<\epsilon\).

Dans le cas particulier de la norme 
\begin{equation}
    \| f \|_{\infty}=\sup_{x\in\Omega}| f(x) |,
\end{equation}
nous parlons que \defe{convergence uniforme}{convergence!uniforme!suite de fonctions}.

\begin{theorem}[Critère de Cauchy]  \label{ThoCauchyZelUF}
    Une suite de fonctions  \( (f_n)_{n\in\eN}\) sur \( \Omega\) converge en norme sur \( \Omega\) si et seulement si \( \forall\epsilon>0\), \( \exists N\) tel que
    \begin{equation}
        \| f_n-f_m \|<\epsilon
    \end{equation}
    pour \( n,m>N\).
\end{theorem}

\begin{corollary}       \label{CorCauchyCkXnvY}
    La série \( \sum f_n\) converge en norme sur \( \Omega\) si et seulement si \( \exists N\) tel que
    \begin{equation}
        \| f_n+\ldots+f_m \|\leq \epsilon
    \end{equation}
    pour tout \( n,m>N\).
\end{corollary}

\begin{proof}
    L'hypothèse montre que la suite des sommes partielles de la série \( \sum f_n\) vérifie le critère de Cauchy du théorème \ref{ThoCauchyZelUF}.
\end{proof}

\begin{definition}
    Nous disons qu'un sous ensemble \( A\) de \( \Omega\) est \defe{complet}{complet} si toute suite de Cauchy d'éléments de \( A\) converge vers un élément de \( A\).
\end{definition}

%---------------------------------------------------------------------------------------------------------------------------
\subsection{Convergence monotone}
%---------------------------------------------------------------------------------------------------------------------------

\begin{theorem}[Théorème de la convergence monotone ou de Beppo-Levi\cite{mathmecaChoi}] \label{ThoConvMonFtBoVh}
    Soit un espace mesuré \( (\Omega,\tribA,\mu)\) et \( (f_n)\) une suite croissante de fonctions mesurables à valeurs dans \( \mathopen[ 0 , \infty \mathclose]\). Alors la limite ponctuelle \( \lim_{n\to \infty} f_n\) existe, est mesurable et
    \begin{equation}    \label{EqFHqCmLV}
        \lim_{n\to \infty} \int_{\Omega}f_nd\mu= \int_{\Omega}\lim_{n\to \infty} f_nd\mu,
    \end{equation}
    cette intégrable valant éventuellement \( \infty\).
\end{theorem}
\index{théorème!convergence!monotone}
\index{théorème!Beppo-Levi}

\begin{proof}
    La limite ponctuelle de la suite est la fonction à valeurs dans \( \mathopen[ 0 , \infty \mathclose]\) donnée par
    \begin{equation}
        f(x)=\lim_{n\to \infty} f_n(x).
    \end{equation}
    Ces limites existent parce que pour chaque \( x\) la suite \( f_n(x)\) est une suite numérique croissante. Nous notons
    \begin{equation}
        I_0=\int_{\Omega}fd\mu.
    \end{equation}
    Nous posons par ailleurs
    \begin{equation}
        I_n=\int_{\Omega}f_n.
    \end{equation}
    Cela est une suite numérique croissante qui a par conséquent une limite que nous notons \( I=\lim_{n\to \infty} I_n\). Notre objectif est de montrer que \( I=I_0\). D'abord par croissance de la suite, pour tous $n$ nous avons \( I_n\leq I_0\), par conséquent \( I\leq I_0\).

    Nous prouvons maintenant l'inégalité dans l'autre sens en nous servant de la définition \eqref{EqDefintYfdmu}. Soit une fonction simple \( h\) telle que \( h\leq f\), et une constante \( 0<C<1\). Nous considérons les ensembles
    \begin{equation}
        E_n=\{ x\in\Omega\tq f_n(x)\geq Ch(x) \}.
    \end{equation}
    Ces ensembles vérifient les propriétés \( E_n\subset E_{n+1}\) et \( \bigcup_{n=1}^{\infty}E_n=\Omega\). Pour chaque \( n\) nous avons les inégalités
    \begin{equation}
        \int_{\Omega}f_n\geq\int_{E_n}f_n\geq C\int_{E_n}h.
    \end{equation}
    Si nous prenons la limite \( n\to\infty\) dans ces inégalités,
    \begin{equation}
        \lim_{n\to \infty} \int_{\Omega}f_n\geq C\lim_{n\to \infty} \int_{E_n}h=C\int_{\Omega}h.
    \end{equation}
    Par conséquent \( \lim_{n\to \infty} \int f_n\geq C\int_{\Omega}h\). Mais étant donné que cette inégalité est valable pour tout \( C\) entre \( 0\) et \( 1\), nous pouvons l'écrire sans le \( C\) :
    \begin{equation}        \label{EqzAKEaU}
        \lim_{n\to \infty} \int_{\Omega}f_n\geq \int_{\Omega}h.
    \end{equation}
    Par définition, l'intégrale de \( f\) est donné par le supremum des intégrales de \( h\) où \( h\) est une fonction simple dominée par \( f\). En prenant le supremum sur \( h\) dans l'équation \eqref{EqzAKEaU} nous avons
    \begin{equation}
        \lim_{n\to \infty} \int_{\Omega}f_n\geq\int_{\Omega}f,
    \end{equation}
    ce qu'il nous fallait.
\end{proof}

\begin{remark}
    Une des raisons de demander la positivité des fonctions \( f_n\) est de n'avoir pas d'ambiguïté à parler d'intégrales qui valent \( \infty\). Si par exemple nous prenons \( \Omega=\mathopen[ 0 , 1 \mathclose]\) et que nous considérons
    \begin{equation}
        f_n(x)=\begin{cases}
            0    &   \text{si \( x\leq \frac{1}{ n }\)}\\
            \frac{1}{ x }    &    \text{sinon}.
        \end{cases}
    \end{equation}
    Ce sont des fonctions intégrables, mais la limite étant la fonction \( 1/x\), l'égalité \eqref{EqFHqCmLV} est une égalité entre deux intégrales valant \( \infty\).
\end{remark}

\begin{corollary}[Inversion de somme et intégrales]
    Si \( (u_n)\) est une suite de fonctions mesurables positives ou nulles, alors
    \begin{equation}
        \sum_{i=0}^{\infty}\int u_i=\int\sum_{i=0}^{\infty}u_i.
    \end{equation}
\end{corollary}

\begin{proof}
    Nous considérons la suite des sommes partielles de \( (u_n)\) : \( f_n(x)=\sum_{i=0}^nu_n(x)\). Le théorème de la convergence monotone (théorème \ref{ThoConvMonFtBoVh}) implique que
    \begin{equation}
        \lim_{n\to \infty} \int f_n=\int\lim_{n\to \infty} f_n.
    \end{equation}
    Nous remplaçons maintenant \( f_n\) par sa valeur en termes des \( u_i\) et dans le membre de gauche nous permutons l'intégrale avec la somme finie :
    \begin{equation}
        \lim_{n\to \infty} \sum_{i=0}^{\infty}\int u_n=\int\sum_{i=0}^{\infty}u_n,
    \end{equation}
    ce qu'il fallait démontrer.
\end{proof}

\begin{lemma}[Lemme de Fatou]\index{lemme!Fatou}\index{Fatou}   \label{LemFatouUOQqyk}
    Soit \( (\Omega,\tribA,\mu)\) un espace mesuré et \( f_n\colon \Omega\to \mathopen[ 0 , \infty \mathclose]  \) une suite de fonctions mesurables. Alors la fonction \( f(x)=\liminf f_n(x)\) est mesurable et
    \begin{equation}
        \int_{\Omega}\liminf f_nd\mu\leq\liminf\int_{\Omega}fd\mu.
    \end{equation}
\end{lemma}

\begin{proof}
    Nous posons 
    \begin{equation}
        g_n(x)=\inf_{i\geq n}f_i(x).
    \end{equation}
    Cela est une suite croissance de fonctions positives mesurables telles que, par définition, 
    \begin{equation}
        \lim_{n\to \infty}g_n(x)=\liminf f_n(x).
    \end{equation}
    Nous pouvons y appliquer le théorème de la convergence monotone,
    \begin{equation}
        \lim_{n\to \infty} \int g_n(x)=\int\liminf f_n(x).
    \end{equation}
    Par ailleurs, pour chaque \( i\geq n\) nous avons
    \begin{equation}
        \int g_n\leq \int f_i,
    \end{equation}
    en passant à l'infimum nous avons
    \begin{equation}
        \int g_n\leq \inf_{i\geq n}\int f_i,
    \end{equation}
    et en passant à la limite nous avons
    \begin{equation}
        \int\liminf f_n=\lim_{n\to \infty} \int g_n\leq \lim_{n\to \infty} \inf_{i\geq n}\int f_i=\liminf_{i\to\infty}\inf f_i.
    \end{equation}
\end{proof}

L'inégalité donnée dans ce lemme n'est en général pas une égalité, comme le montre l'exemple suivant :
\begin{equation}
    f_i=\begin{cases}
        \mtu_{\mathopen[ 0 , 1 \mathclose]}    &   \text{si \( i\) est pair}\\
        \mtu_{\mathopen[ 1 , 2 \mathclose]}    &    \text{si \( i\) est impair}.
    \end{cases}
\end{equation}
Nous avons évidemment \( g_n(x)=0\) tandis que \( \int_{\mathopen[ 0 , 2 \mathclose]}f_i=1\) pour tout \( i\).

%---------------------------------------------------------------------------------------------------------------------------
\subsection{Convergence dominée de Lebesgue}
%---------------------------------------------------------------------------------------------------------------------------

\begin{theorem}[Convergence dominée de Lebesgue]        \label{ThoConvDomLebVdhsTf}
    Soit \( (f_n)_{n\in\eN}\) une suite de fonctions intégrables sur \( (\Omega,\tribA,\mu)\) à valeurs dans \( \eC\) ou \( \eR\). Nous supposons que  \( f_n\to f\) simplement sur \( \Omega\) presque partout et qu'il existe une fonction intégrable \( g\) telle que
    \begin{equation}
        | f_n(x) |< g(x) 
    \end{equation}
    pour presque\footnote{Si il n'y avait pas le «presque» ici, ce théorème serait à peu près inutilisable en probabilité ou en théorie des espaces \( L^p\), comme dans la démonstration du théorème de Fischer-Riesz \ref{ThoGVmqOro} par exemple.} tout \( x\in\Omega\) et pour tout \( n\in \eN\). Alors
    \begin{enumerate}
        \item
            \( f\) est intégrable,
        \item
           $\lim_{n\to \infty} \int_{\Omega}f_n=\int_\Omega f$,
        \item
            $\lim_{n\to \infty} \int_{\Omega}| f_n-f |=0$.
    \end{enumerate}
\end{theorem}
\index{théorème!convergence!dominée de Lebesgue}
\index{dominée!convergence (Lebesgue)}

\begin{proof}

    La fonction limite \( f\) est intégrable parce que \( | f |\leq g\) et \( g\) est intégrable (lemme \ref{LemPfHgal}). Par hypothèse nous avons
    \begin{equation}
        -g(x)\leq f_n(x)\leq g(x).
    \end{equation}
    En particulier la fonction \( g_n=f_n+g\) est positive et mesurable si bien que le lemme de Fatou (lemme \ref{LemFatouUOQqyk}) implique
    \begin{equation}
        \int_{\Omega}\liminf g_n\leq\liminf\int_{\Omega}g_n.
    \end{equation}
    Évidement nous avons \( \liminf g_n=f+g\), de telle sorte que
    \begin{equation}
        \int f+\int g\leq \liminf\int g_n=\liminf\int f_n+\int g,
    \end{equation}
    et le nombre \( \int g\) étant fini, nous pouvons le retrancher des deux côtés de l'inégalité :
    \begin{equation}
        \int f\leq\liminf\int f_n.
    \end{equation}
    Afin d'obtenir une minoration de \( \int f\) nous refaisons exactement le même raisonnement en utilisant la suite de fonctions \( k_n=-f_n\to k=-f\). Nous obtenons que
    \begin{equation}
        \int k\geq\liminf\int k_n=-\limsup\int f_n,
    \end{equation}
    et par conséquent
    \begin{equation}    \label{IneqsndMYTO}
        \liminf\int f_n\leq\int f\leq\limsup\int f_n.
    \end{equation}
    La limite supérieure étant plus grande ou égale à la limite inférieure, les trois quantités dans les inégalités \eqref{IneqsndMYTO} sont égales.

    Nous prouvons maintenant le troisième point. Soit la suite de fonctions
    \begin{equation}
        h_n(x)=| f_n(x)-f(x) |
    \end{equation}
    qui tend ponctuellement vers zéro. De plus
    \begin{equation}
    h_n(x)\leq | f_n(x) |+| f(x) |\leq 2g(x),
    \end{equation}
    ce qui prouve que les \( h_n\) majorés par une fonction intégrable. Donc
    \begin{equation}
        \lim_{n\to \infty} \int_{\Omega}| f_n-f |= \lim_{n\to \infty} \int_{\Omega}h_n(x)dx=\int_{\Omega}\lim_{n\to \infty} | f_n(x)-f(x) |=0
    \end{equation}
\end{proof}

\begin{remark}
    Lorsque nous travaillons sur des problèmes de probabilités, la fonction \( g\) peut être une constante parce que les constantes sont intégrables sur un espace de probabilité.
\end{remark}

\begin{corollary}       \label{CorCvAbsNormwEZdRc}
    Soit \( (a_i)_{i\in \eN}\) une suite numérique absolument convergente. Alors elle est convergente. Il en est de même pour les séries de fonctions si on considère la convergence ponctuelle.
\end{corollary}

\begin{proof}
    L'hypothèse est la convergence de l'intégrale \( \int_{\eN}| a_i |dm(i)\) où \( dm\) est la mesure de comptage. Étant donné que \( | a_i |\leq | a_i |\), la fonction \( a_i\) (fonction de \( i\)) peut jouer le rôle de \( g\) dans le théorème de la convergence dominée de Lebesgue (théorème \ref{ThoConvDomLebVdhsTf}).
\end{proof}
Nous utiliseront ce résultat pour montrer que la transformée de Fourier d'une fonction \( L^1(\eR^d)\) est continue (proposition \ref{PropJvNfj}).

\begin{proposition}[\cite{YHRSDGc}] \label{PropUXjnwLf}
    \begin{enumerate}
        \item
            Une fonction mesurable et positive est limite (simple) d'une suite croissante de fonctions étagées, mesurables et positives.
        \item
            Si \( f\colon \eR^d\to \bar \eR\) est mesurable, alors elle est limite (simple) de fonctions étagées \( f_n\) telles que \( | f_n |\leq | f |\).
    \end{enumerate}
\end{proposition}
%TODO : la preuve est dans le document cité.

%+++++++++++++++++++++++++++++++++++++++++++++++++++++++++++++++++++++++++++++++++++++++++++++++++++++++++++++++++++++++++++ 
\section{Séries de fonctions}
%+++++++++++++++++++++++++++++++++++++++++++++++++++++++++++++++++++++++++++++++++++++++++++++++++++++++++++++++++++++++++++

Les séries de fonctions sont des cas particuliers de suites, étant donné que, par définition,
\begin{equation}
    \sum_{n=1}^{\infty}f_n=\lim_{N\to \infty} \sum_{n=1}^{N}f_n.
\end{equation}

\begin{definition}  \label{DefQDrDqek}
    Une série de nombres \( \sum_{n=0}^{\infty}a_n\) converge \defe{absolument}{convergence!absolue} si la série $\sum_{n=0}^{\infty}| a_n |$ converge. Cette définition s'étend immédiatement aux séries dans n'importe quel espace normé.

    Une série de fonctions \( \sum_{n\in \eN}u_n \) converge \defe{normalement}{convergence!normale} si la série de nombre \( \sum_n\| u_n \|_{\infty}\) converge.
\end{definition}

La convergence normale est à ne pas confondre avec la convergence uniforme. La somme \( \sum_nf_n\) \defe{converge uniformément}{convergence!uniforme!série de fonctions} vers la fonction \( F\) si la suite des sommes partielles converge uniformément, c'est à dire si 
\begin{equation}
    \lim_{N\to \infty} \| \sum_{n=1}^Nf_n-F \|_{\infty}=0.
\end{equation}

\begin{lemma}
    Soient des fonctions \( u_n\colon \Omega\to \eC\). Si il existe une suite réelle positive \( (a_n)_{n\in \eN}\) telle que
    \begin{enumerate}
        \item
            pour tout \( z\in \Omega\) et pour tout \( n\in \eN\) nous avons \( | u_n(z) |\leq a_n\) (c'est à dire \( a_n\geq \| u_n \|_{\infty}\)),
        \item
            la somme \( \sum_{n}a_n\) converge,
    \end{enumerate}
    alors la série de fonctions \( \sum_{n=0}^{\infty}u_n\) converge normalement.
\end{lemma}

\begin{proof}
    Découle du lemme de comparaison \ref{LemgHWyfG}.
\end{proof}

\begin{proposition}     \label{PropUEMoNF}
    Soit \( (u_n)\) une suite de fonctions continues \( u_n\colon \Omega\subset\eC\to \eC\). Si la série \( \sum_nu_n\) converge normalement alors la somme est continue.
\end{proposition}

\begin{proof}
    Nous posons \( u(z)=\lim_{N\to \infty} \sum_{n=0}^N u_n(z)\), et nous vérifions que la fonction ainsi définie sur \( \Omega\) est continue. Soit \( z\in \Omega\) et prouvons la continuité de \( u\) au point \( z\). Pour tout \( z'\) dans un voisinage de \( z\) nous avons 
    \begin{subequations}
        \begin{align}
            \big| u(z)-u(z') \big|&=\left| \sum_{n=0}^{N}u_n(z)-\sum_{n=0}^{N}u_n(z')+\sum_{n=N+1}^{\infty}u_n(z)-\sum_{n=N+1}^{\infty}u_n(z') \right| \\
            &\leq \left| \sum_{n=0}^N u_n(z)-\sum_{n=0}^Nu_n(z') \right| +\sum_{n=N+1}^{\infty}| u_n(z) |+\sum_{n=N+1}^{\infty}| u_n(z') |.
        \end{align}
    \end{subequations}
    Étant donné que les sommes partielles sont continues, en prenant \( N\) suffisamment grand, le premier terme peut être rendu arbitrairement petit. Si \( N\) est suffisamment grand, le second terme est également petit. Par contre, cet argument ne tient pas pour le troisième terme parce que nous souhaitons une majoration pour tout \( z'\) dans une boule autour de \( z\). Nous devons donc écrire
    \begin{equation}
        \sum_{n=N}^{\infty}| u_n(z) |\leq \sum_{n=N+1}^{\infty}\| u_n \|_{\infty}.
    \end{equation}
    Ce dernier est arbitrairement petit lorsque \( N\) est grand. Notons que nous avons utilisé l'hypothèse de convergence normale.
\end{proof}

La même propriété, avec la même démonstration, tient dans le cas d'espaces vectoriels normée.
\begin{proposition} \label{PropOMBbwst}
    Soient \( E\) et \( F\), deux espaces vectoriels normés, \( \Omega\) une partie ouverte de \( E\) et une suite de fonctions \( u_n\colon \Omega\to F\) convergeant normalement sur \( \Omega\), c'est à dire que \( \sum_n\| u_n \|_{\infty}\) converge, la norme \( \| . \|_{\infty} \) devant être comprise comme la norme supremum sur \( \Omega\). Alors la fonction \( u=\sum_nu_n\) est continue sur \( \Omega\).
\end{proposition}

\begin{proof}
    Soit \( x,x'\in \Omega\) en supposant que \( \| x-x' \|\) est petit. Soit encore \( \epsilon>0\). Nous allons montrer la continuité en \( x\). Pour cela nous savons que pour tout \( N\) l'inégalité suivante est correcte :
    \begin{equation}
        \| u(x)-u(x') \|\leq \left\|  \sum_{n=0}^Nu_n(x)-\sum_{n=0}^{N}u_n(x') \right\|+\sum_{n=N+1}^{\infty}\| u_n(x) \|+\sum_{n=N+1}^{\infty}\| u_n(x') \|.
    \end{equation}
    Les deux derniers termes sont majorés par \( \sum_{n=N+1}^{\infty}\| u_n \|_{\infty}\) qui, par hypothèse, peut être rendu aussi petit que souhaité en choisissant \( N\) assez grand. Nous choisissons donc un \( N\) tel que ces deux termes soient plus petits que \( \epsilon\). Ce \( N\) étant fixé, la fonction \( \sum_{n=0}^{N}u_n\) est continue et nous pouvons choisir \( x'\) assez proche de \( x\) pour que le premier terme soit majoré par \( \epsilon\).
\end{proof}

\begin{theorem}			\label{ThoSerUnifCont}
	Si les $g_k$ sont continues et si $\sum g_k$ converge uniformément, alors $\sum g_k$ est continue.
\end{theorem}

\begin{theorem}[Critère de Weierstrass]\index{critère!Weierstrass!série de fonctions}		\label{ThoCritWeierstrass}
	Soit une suite de fonctions $f_k\colon A\to \eC$ telles que $| f_k(x) |\leq M_k\in\eR$, $\forall x\in A$. Si $\sum_{k=1}^{\infty}M_k$ converge, alors $\sum_{k=1}^{\infty}f_k$ converge absolument et uniformément.
\end{theorem}

\begin{proof}
    La convergence normale est facile : l'hypothèse dit que \( \| f_k \|_{\infty}\leq M_k\), et donc que
    \begin{equation}
        \sum_{k=1}^{\infty}\| f_k \|_{\infty}\leq \sum_kM_k<\infty.
    \end{equation}
    
    La convergence uniforme est à peine plus subtile. Nous nommons \( F\) la fonction somme. Pour tout \( x\) et pour tout \( N\), nous avons
    \begin{subequations}
        \begin{align}
            \left\| \sum_{n=1}^Nf_n(x)-F(x) \right\|&=\| \sum_{n=N}^{\infty}f_n(x) \|\\
            &\leq\sum_{n=N}^{\infty}\| f_k(x) \|\\
            &\leq \sum_{n=N}^{\infty}\| f_n \|_{\infty}.
        \end{align}
    \end{subequations}
    La convergence normale étant assurée, la série \( \sum_{n_1}^{\infty}\| f_n \|_{\infty}\) est finie, ce qui implique que la queue de somme \( \sum_{n=N}^{\infty}\| f_n \|_{\infty}\) tend vers zéro lorsque \( N\to \infty\). Pour tout \( \epsilon\), il existe donc un \( N\) (non dépendant de \( x\)) tel que
    \begin{equation}
        \| \sum_{n=1}^Nf_n(x)-F(x) \|\leq \epsilon.
    \end{equation}
    En prenant le supremum sur \( x\in A\) nous trouvons la convergence uniforme.
\end{proof}

\begin{remark}
    Il n'y a pas de critère correspondant pour les suites. Il n'est pas vrai que si \( \lim_{n\to \infty}\| f_n \| \) existe, alors \( \lim_{n\to \infty} f_n\) existe, comme le montre l'exemple
    \begin{equation}
        f_n(x)=\begin{cases}
            1    &   \text{si \( x\in\mathopen[ 0 , 1 \mathclose]\) et \( n\) est pair}\\
            1    &    \text{si \( x\in\mathopen[ 1 , 2 \mathclose]\) et \( n\) est impair}\\
             0   &    \text{sinon.}
        \end{cases}
    \end{equation}
\end{remark}

\begin{theorem}      \label{ThoCciOlZ}
    La somme uniforme de fonctions intégrables sur un ensemble de mesure fini est intégrable et on peut permuter la somme et l'intégrale.

    En d'autres termes, supposons que \( \sum_{n=0}^{\infty}f_n\) converge uniformément vers \( F\) sur \( A\) avec \( \mu(A)<\infty\). Si \( F\) et \( f_n\) sont des fonctions intégrables sur \( A\) alors
    \begin{equation}
        \int_AF(x)d\mu(x)=\sum_{n=0}^{\infty}\int_Af_n(x)d\mu(x).
    \end{equation}
\end{theorem}

\begin{proof}
    Ce théorème est une conséquence du théorème \ref{ThoUnifCvIntRiem}. En effet nous définissons la suite des sommes partielles
    \begin{equation}
        F_N=\sum_{n=0}^Nf_n.
    \end{equation}
    La limite \( \lim_{N\to \infty} F_N=F\) est uniforme. Par conséquent la fonction \( F\) est intégrable et
    \begin{equation}
        \int_A F=\lim_{N\to \infty} \int_AF_N=\lim_{N\to \infty} \int_A\sum_{n=0}^Nf_n=\lim_{N\to \infty} \sum_{n=0}^N\int_Af_n=\sum_{n=0}^{\infty}\int_Af_n.
    \end{equation}
    La première égalité est le théorème \ref{ThoUnifCvIntRiem}, les autres sont de simples manipulations rhétoriques.
\end{proof}


Le théorème suivant est une paraphrase du théorème de la convergence dominée de Lebesgue (\ref{ThoConvDomLebVdhsTf}).
\begin{theorem}     \label{ThoockMHn}
    Soient des fonctions \( (f_n)_{n\in \eN}\) telles que \( \sum_{n=0}^Nf_n\) soit intégrable sur \( (\Omega,\tribA,\mu)\) pour chaque \( N\). Nous supposons que la somme converge simplement vers
    \begin{equation}
        f(x)=\sum_{n=0}^{\infty}f_n(x)
    \end{equation}
    et qu'il existe une fonction \( g\) telle que
    \begin{equation}
        \left| \sum_{n=0}^Nf_n \right| <g
    \end{equation}
    pour tout \( N\in \eN\). Alors
    \begin{enumerate}
        \item
            \( \sum_{n=0}^{\infty}f_n\) est intégrable,
        \item
            on peut permuter somme et intégrale :
            \begin{equation}
                \lim_{N\to \infty} \int_{\Omega}\sum_{n=0}^Nf_nd\mu=\int_{\Omega}\sum_{n=0}^{\infty}f_n,
            \end{equation}
        \item
            \begin{equation}
                \lim_{N\to \infty} \int_{\Omega}\left| \sum_{n=0}^Nf_n-\sum_{n=0}^{\infty}f_n \right| =\lim_{N\to \infty} \int_{\Omega}\left| \sum_{n=N}^{\infty}f_n \right| =0.
            \end{equation}
    \end{enumerate}
\end{theorem}


\begin{theorem} \label{ThoCSGaPY}
    Soit \( f_n\) des fonctions \( C^1\mathopen[ a , b \mathclose]\) telles que
    \begin{enumerate}
        \item
            la série \( \sum_n f_n(x_0)\) converge pour un certain \( x_0\in\mathopen[ a , b \mathclose]\),
        \item
            la série des dérivées \( \sum_n f'_n\) converge uniformément sur \( \mathopen[ a , b \mathclose]\).
    \end{enumerate}
    Alors la série \( \sum_n f_n\) converge vers une fonction \( F\) et
    \begin{enumerate}
        \item
            La convergence est uniforme sur \( \mathopen[ a , b \mathclose]\).
        \item
            La fonction \( F\) est dérivable
        \item
            \( F'(x)=\sum_nf'_n(x)\).
    \end{enumerate}
\end{theorem}

\begin{lemma}
    Soient \( E\) et \( F\) deux espaces vectoriels normés. Si la suite \( (T_n))\) converge vers \( T\) dans \( \aL(E,F)\), alors pour tout \( v\in E\) nous avons
    \begin{equation}
        \left( \sum_{n=0}^{\infty}T_n \right)(v)=\sum_{n=0}^{\infty}T_n(v).
    \end{equation}
\end{lemma}

\begin{theorem}[\cite{DHdwZRZ}] \label{ThoLDpRmXQ}
    Soit \( E\) et \( F\), deux espaces vectoriels normés, \( \Omega\) un ouvert connexe par arcs de \( E\). Soit \( (u_n)\) une suite de fonctions \( u_n\colon \Omega\to F\) telle que
    \begin{enumerate}
        \item
            pour tout \( n\), la fonction \( u_n\) est de classe \( C^1\) sur \( \Omega\),
        \item
            la série \( \sum_nu_n\) converge simplement sur \( \Omega\),
        \item
            la série des différentielles \( \sum_n(du_n)\) converge normalement sur tout compact de \( \Omega\).
    \end{enumerate}
    Alors la somme \( u=\sum_nu_n\) est de classe \( C^1\) sur \( \Omega\) et sa différentielle est donnée par
    \begin{equation}
        du=\sum_{n=0}^{\infty}du_n.
    \end{equation}
\end{theorem}

\begin{proof}
    Pour chaque \( n\), la fonction \( du_n\colon \Omega\to \aL(E,F)\) est une fonction continue parce que \( u_n\) est de classe \( C^1\). La série convergeant normalement, la fonction \( \sum_{n=0}^{\infty}du_n\) est également continue par la proposition \ref{PropOMBbwst}. La difficulté de ce théorème est donc de prouver que cela est bien la différentielle de la fonction \( \sum_nu_n\).

    Soit \( a,x\in \Omega\) et \( \gamma\colon \mathopen[ 0 , 1 \mathclose]\to \Omega\) un chemin joignant \( a\) à \( x\). Nous considérons ce chemin en coordonnées normales et nous notons \( l\) sa longueur. Par définition \ref{EqEFIZyEe},
    \begin{equation}
        \clubsuit=\int_{\gamma}\sum_{n=0}^{\infty}du_n=\int_0^l\sum_n(du_n)_{\gamma(t)}\big( \gamma'(t) \big)dt
    \end{equation}
    Si nous notons \( f_n(t)=(du_n)_{\gamma(t)}\big( \gamma'(t) \big)\), sachant que la paramétrisation est normale (\( \| \gamma'(t) \|=1\)) nous avons\footnote{Histoire de ne pas s'embrouiller, il faut se rendre compte que \( \| du_n \|_{\infty}=\sup_{x\in \Omega}\| (du_n)_x \|\).}
    \begin{equation}
        \| f_n(t) \|\leq \|   (du_n)_{\gamma(t)}  \|\leq \| du_n \|_{\infty}.
    \end{equation}
    Or la série des \( \| du_n \|_{\infty}\) converge par hypothèse. L'intervalle \( \mathopen[ 0 , l \mathclose]\) étant compact, les fonctions \( f_n\) sont uniformément (en \( n\)) bornées par le nombre \( \sum_n\| du_n \|_{\infty}\) qui est intégrable sur \( \mathopen[ 0 , 1 \mathclose]\). Par la convergence dominée (théorème \ref{ThoConvDomLebVdhsTf}) nous permutons la somme et l'intégrale :
    \begin{equation}
        \clubsuit=\sum_{n=0}^{\infty}\int_0^l(du_n)_{\gamma(t)}\big( \gamma'(t) \big)dt=\sum_{n=0}^{\infty}u_n(x)-\sum_{n=0}^{\infty}u_n(a)=u(x)-u(a)
    \end{equation}
    où nous avons utilisé le théorème \ref{ThoUJMhFwU}. Jusqu'à présent nous avons montré que
    \begin{equation}
        u(x)=u(a)+\int_{\gamma}\sum_{n=0}^{\infty}du_n=u(a)+\int_0^l\sum_{n=0}^{\infty}(du_n)_{\gamma(t)}\big( \gamma'(t) \big)dt.
    \end{equation}
    Nous allons utiliser cela pour calculer \( du_x(v)\) selon la bonne vieille formule
    \begin{equation}
        du_x(v)=\Dsdd{ u(x+sv) }{s}{0}.
    \end{equation}
    Cela sera fait en considérant à nouveau un chemin \( \gamma_s \) joignant \( a\) à \( x+sv\) en paramétrisation normale; nous notons \( l_s\) sa longueur. Dans le calcul suivant, nous inversons la somme et l'intégrale de la même façon qu'avant. En piste maestro
    \begin{subequations}
        \begin{align}
            du_x(v)&=\frac{ d  }{ d s }\left.\int_0^{l_s}\sum_{n=0}^{\infty}(du_n)_{\gamma_s(t)}\big( \gamma'_s(t) \big)dt\right|_{s=0}\\
            &=\frac{ d  }{ d s }\left.\sum_{n=0}^{\infty}\int_{\gamma_s}du_n\right|_{s=0}\\
            &=\frac{ d  }{ d s }\sum_{n=0}^{\infty}\Big[ u_n\big( \gamma_s(l_s)\big)-u_n\big( \gamma_s(0) \big)  \Big]_{s=0}\\
            &=d\left( \sum_{n=0}^{\infty}u_n \right)_x(v).
        \end{align}
    \end{subequations}
\end{proof}


%---------------------------------------------------------------------------------------------------------------------------
\subsection{Théorème de Stone-Weierstrass}
%---------------------------------------------------------------------------------------------------------------------------

Comme presque tous les théorèmes importants, le théorème de Stone-Weierstrass possède de nombreuses formulations à divers degrés de généralité.

Le lemme suivant est une cas particulier du théorème \ref{ThoGddfas}, mais nous en donnons une démonstration indépendante afin d'isoler la preuve de la généralisation \ref{ThoWmAzSMF}. Une version pour les polynômes trigonométrique sera donnée dans le lemme \ref{LemXGYaRlC}.

\begin{lemma}       \label{LemYdYLXb}
    Il existe une suite de polynômes sur \( \mathopen[ 0 , 1 \mathclose]\) convergent uniformément vers la fonction racine carré.
\end{lemma}

\begin{proof}
    Nous donnons cette suite par récurrence :
    \begin{subequations}
        \begin{align}
            P_0(t)&=0\\
            P_{n+1}(t)&=P_n(t)+\frac{ 1 }{2}\big( t-P_n(t)^2 \big).
        \end{align}
    \end{subequations}
    Nous commençons par montrer que pour tout \( t\in \mathopen[ 0 , 1 \mathclose]\), \( P_n(t)\in\mathopen[ 0 , \sqrt{t} \mathclose]\). Pour \( P_0\), c'est évident. Ensuite nous avons
    \begin{subequations}
        \begin{align}
            P_{n+1}(t)-\sqrt{t}&=P_n(t)-\sqrt{t}+\frac{ 1 }{2}(t-P_n(t)^2)\\
            &=\big( P_n(t)-\sqrt{t} \big)\left( 1-\frac{ 1 }{2}\frac{ t-P_n(t)^2 }{ P_n(t)-\sqrt{t} } \right)\\
            &=\big( P_n(t)-\sqrt{t} \big)\left( 1-\frac{ \sqrt{t}+P_n(t) }{2} \right)\\
            &\leq 0
        \end{align}
    \end{subequations}
    parce que \( \sqrt{t} \leq 1\) et \( P_n(t)\leq 1\) par hypothèse de récurrence.

    Nous savons au passage que \( P_n(t)\) est une suite réelle croissante parce que \( t-P_n(t)^2\geq t-(\sqrt{t})^2=0\). La suite \( P_n(t)\) est donc croissante et majorée par \( \sqrt{t}\); elle converge donc. Les candidats limites sont déterminés par l'équation
    \begin{equation}
        \ell=\ell+\frac{ 1 }{2}(t-\ell^2),
    \end{equation}
    dont les solutions sont \( \ell=\pm\sqrt{t}\). La suite étant positive, nous avons une convergence ponctuelle de \( P_n\) vers la racine carré. Cette convergence prenant place sur un compact, elle est uniforme.
\end{proof}

\begin{lemma}           \label{LemUuxcqY}
    Soit \( K\), un compact de \( \eR\) et \( f_n\) une suite de fonctions sur \( K\) convergeant uniformément vers \( f\). Soit \( g\colon X\to K\) une fonction depuis un espace topologique \( K\). Alors \( f_n\circ g\) converge uniformément vers \( f\circ g\).
\end{lemma}

\begin{proof}
    En effet, pour tout \( x\in X\) nous avons
    \begin{equation}
        \| (f_n\circ g)-(f\circ g) \|_{\infty}=\sup_{x\in X} \| f_n\big( g(x) \big)-f\big( g(x) \big) \|\leq \| f_n-f \|_{\infty}.
    \end{equation}
    Par conséquent, si \( \epsilon\>0\) est donné, il suffit de choisir \( n\) de telle sorte à avoir \( \| f_n-f \|_{\infty}<\epsilon\) et nous avons \( \| (f_n\circ g)-(f\circ g) \|_{\infty}\leq \epsilon\).
\end{proof}

\begin{definition}
    Nous disons qu'une algèbre \( A\) de fonctions sur un espace \( X\) \defe{sépare les points}{sépare!les points} de \( X\) si pour tout \( x_1\neq x_2\) il existe \( g\in A\) telle que \( g(x_1)\neq g(x_2)\).
\end{definition}

Nous pouvons maintenant énoncer et démontrer une forme nettement plus générale du théorème de Stone-Weierstrass.
\begin{theorem}[Stone-Weierstrass\cite{MGecheleSW}] \label{ThoWmAzSMF}
    Soit \( X\), un espace compact et Hausdorff et \( A\) une sous algèbre de \( C(X,\eR)\) contenant une fonction constante non nulle. Alors \( A\) est dense dans \( \Big( C(X,\eR),\| . \|_{\infty}\Big)\) si et seulement si \( A\) sépare les points de \(X\).

    Nous pouvons remplacer \( \eR\) par \( \eC\) si de plus l'algèbre \( A\) est auto-adjointe : \( g\in A\) implique \( \bar g\in A\).
\end{theorem}
\index{théorème!Stone-Weierstrass}

\begin{proof}
    Nous allons écrire la démonstration en plusieurs étapes (dont la première est le lemme \ref{LemYdYLXb}).

    \begin{description}
        \item[Première étape] Pour tout \( x\neq y\in X\) et pour tout \( \alpha,\beta\in \eR\), il existe une fonction \( f\in A\) telle que \( f(x)=\alpha\) et \( f(y)=\beta\). 

            En effet, vu que \( A\) sépare les points nous pouvons considérer une fonction \( g\in A\) telle que \( g(x)\neq g(y)\) et ensuite poser
            \begin{equation}
                f(z)=\alpha+\frac{ \alpha-\beta }{ g(y)-g(x) }\big( g(z)-g(x) \big).
            \end{equation}
            Les constantes faisant partie de \( A\), cette fonction \( f\) est encore dans \( A\).

        \item[Seconde étape] Pour tout \( n\)-uples de fonctions \( f_1,\ldots, f_n\) dans \( \bar A\), les fonctions \( \min(f_1,\ldots, f_n)\) et \( \max(f_1,\ldots, f_n)\) sont dans \( \bar A\).

            Nous le démontrons pour \( n=2\); le reste allant évidemment par récurrence. Soient \( f,g\in \bar A\). Étant donné que
            \begin{subequations}
                \begin{align}
                    \max(f,g)&=\frac{ f+g }{2}+\frac{ | f-g | }{2}\\
                    \min(f,g)&=\frac{ f+g }{2}-\frac{ | f-g | }{2},
                \end{align}
            \end{subequations}
            if suffit de montrer que si \( f\in\bar A\) alors \( | f |\in \bar A\). Si \( f\) est nulle, c'est évident; supposons que \( f\neq 0\) et posons \( M=\| f \|_{\infty}\neq 0\). Pour tout \( x\in X\) nous avons
            \begin{equation}
                \frac{ f(x)^2 }{ M^2 }\in \mathopen[ 0 , 1 \mathclose].
            \end{equation}
            Nous considérons alors la suite
            \begin{equation}
                h_n=P_n\circ\frac{ f^2 }{ M^2 }
            \end{equation}
            où \( P_n\) est une suite de polynômes convergent uniformément vers la racine carré (voir lemme \ref{LemYdYLXb}). Le lemme \ref{LemUuxcqY} nous assure que \( h_n\) converge uniformément vers \( \frac{ | f | }{ M }\) dans \( C(X,\eR)\). Étant donné que \( \bar A\) est également une algèbre, \( h_n\) est dans \( \bar A\) pour tout \( n\) et la limite s'y trouve également (pour rappel, la fermeture \( \bar A\) est celle de la topologie de la convergence uniforme).

        \item[Troisième étape] Soit \( \epsilon>0\), \( f\in C(X,\eR)\) et \( x\in X\). Il existe une fonction \( g_x\in \bar A\) telle que 
            \begin{subequations}
                \begin{numcases}{}
                    g_x(x)=f(x)\\
                    g_x(y)\leq f(y)+\epsilon
                \end{numcases}
            \end{subequations}
            pour tout \( y\in X\).

            Soit \( z\in X\setminus\{ x \}\) et une fonction \( h_z\) telle que \( h_z(x)=f(x)\) et \( h_z(z)=f(z)\). Une telle fonction existe par une des étapes précédentes. Étant donné que \( f\) et \( h_z\) sont continues, il existe un voisinage ouvert \( V_z\) de \( z\) sur lequel
            \begin{equation}
                h_z(y)\leq f(y)+\epsilon
            \end{equation}
            pour tout \( y\in V_z\). Nous pouvons sélectionner un nombre fini de points \( z_1,\ldots, z_n\) tels que les ouverts \( V_{z_1},\ldots, V_{z_n}\) recouvrent \( X\) (parce que \( X\) est compact, de tout recouvrement par des ouverts, nous extrayons un sous recouvrement fini.). Nous posons 
            \begin{equation}
                g_x=\min(h_{z_1},\ldots, h_{z_n})\in \bar A.
            \end{equation}
            Si \( y\in X\), nous sélectionnons le \( i\) tel que \( h_{z_i}(y)\leq f(y)+\epsilon\) et nous avons
            \begin{equation}
                g_x(y)\leq h_{z_i}(y)\leq f(y)+\epsilon.
            \end{equation}
            
        \item[Étape \wikipedia{fr}{Final_Doom}{finale}] Soit \( \epsilon>0\) et \( f\in C(X,\eR)\). Pour chaque \( x\in X\) nous considérons une fonction \( g_x\in \bar A\) telle que
            \begin{subequations}
                \begin{numcases}{}
                    g_x(x)=f(x)\\
                    g_x(y)\leq f(y)+\epsilon
                \end{numcases}
            \end{subequations}
            pour tout \( y\in X\). Les fonctions \( f\) et \( g_x\) sont continues, donc il existe un voisinage ouvert \( W_x\) de \( x\) sur lequel
            \begin{equation}
                g_x(y)\geq f(y)-\epsilon.
            \end{equation}
            De ces \( W_x\) nous extrayons un sous recouvrement fini de \( X\) : \( W_{x_1},\ldots, W_{x_m}\) et nous posons
            \begin{equation}
                \varphi=\max(g_{x_1},\ldots, g_{x_n})\in \bar A.
            \end{equation}
            Si \( y\in X\), il existe un \( i\) tel que 
            \begin{equation}
                \varphi(y)\geq g_{x_i}(y)\geq f(y)-\epsilon.
            \end{equation}
            La première inégalité est le fait que \( \varphi\) est le maximum des \( g_{x_k}\), et la seconde est le choix de \( i\). Donc pour tout \( y\in X\) nous avons
            \begin{equation}        \label{EqJMxHaF}
                f(y)-\epsilon\leq \varphi(y)\leq f(y)+\epsilon.
            \end{equation}
            La première inégalité est ce que l'on vient de faire. La seconde est le fait que pour tout \( i\) nous ayons \( g_{x_i}(y)\leq f(y)+\epsilon\); le fait que \( \varphi\) soit le maximum sur les \( i\) ne change pas l'inégalité.

            Le fait que les inégalités \eqref{EqJMxHaF} soient vraies pour tout \( y\in X\) signifie que \( \| \varphi-f \|_{\infty}\leq \epsilon\), et donc que \( f\in \bar{\bar A}=\bar A\).
    \end{description}

    Tout cela prouve que \( C(X,\eR)\subset \bar A\). L'inclusion inverse est le fait que \( C(X,\eR)\) est fermé pour la norme \( \| . \|_{\infty}\), étant donné qu'une limite uniforme de fonctions continues est continue.

\end{proof}

Le théorème suivant est un des énoncés les plus classiques de Stone-Weierstrass. Il découle évidement du théorème général \ref{ThoWmAzSMF} (encore qu'il faut alors bien comprendre qu'il faut traiter la fonction \( x\mapsto \sqrt{x}\) séparément). Il en existe cependant une preuve indépendante.
%TODO : trouver cette preuve indépendante.
\begin{theorem}     \label{ThoGddfas}   \index{théorème!Stone-Weierstrass}
    Soit \( f\), une fonction continue de l'intervalle compact \( \mathopen[ a , b \mathclose]\) à valeurs dans \( \eR\). Alors pour tout \( \epsilon>0\), il existe un polynôme \( P\) tel que \( \| P-f \|_{\infty}<\epsilon\).

    Autrement dit, les polynômes sont denses dans \( C\mathopen[ a , b \mathclose]\) pour la norme uniforme.
\end{theorem}

\begin{corollary}   \label{CorRSczQD}
    Si \( X\subset \eR\) est compact et de mesure finie\footnote{Dans \( \eR\) cette hypothèse est évidemment superflue par rapport à l'hypothèse de compacité; mais ça suggère des généralisations \ldots}, alors l'ensemble des polynômes est denses dans \( \big( C(X,\eR),\| . \|_2 \big)\).
\end{corollary}

\begin{proof}
    Si \( f\) est une fonction dans \( C(X,\eR)\) et si \( \epsilon\geq 0\) est donné alors nous pouvons considérer un polynôme \( P\) tel que \( \| f-P \|_{\infty}\leq \epsilon\). Dans ce cas nous avons
    \begin{equation}
        \| f-P \|_2^2=\int_X| f(x)-P(x) |^2dx\leq \int_X\epsilon^2dx=\epsilon^2\mu(X)
    \end{equation}
    où \( \mu(X)\) est la mesure de \( X\) (finie par hypothèse).
\end{proof}

%---------------------------------------------------------------------------------------------------------------------------
\subsection{Théorème taubérien de Hardi-Littlewood}
%---------------------------------------------------------------------------------------------------------------------------

Un théorème \defe{taubérien}{taubérien}\index{théorème!taubérien} est un théorème qui compare les modes de convergence d'une série.

\begin{lemma}
    Si \( f\) et \( g\) sont des fonctions continues, alors \( s(x)=\max\{ f(x),g(x) \}\) est également une fonction continue.
\end{lemma}

\begin{proof}
    Soit \( x_0\) et prouvons que \( s\) est continue en \( x_0\). Si \( f(x_0)\neq g(x_0)\) (supposons \( f(x_0)>g(x_0)\) pour fixer les idées), alors nous avons un voisinage de \( x_0\) sur lequel \( f>g\) et alors \( s=f\) sur ce voisinage et la continuité provient de celle de \( f\).

    Si au contraire \( f(x_0)=g(x_0)=s(x_0)\) alors si \( (a_n)\) est une suite tendant vers \( x_0\), nous prenons \( N\) tel que \( \big| f(a_n)-f(x_0) \big|\leq \epsilon\) pour tout \( n>N\) et \( M\) tel que \( \big| g(a_n)-g(x_0) \big|\leq \epsilon\) pour tout \( n> M\). Alors pour tout \( n>\max\{ N,M \}\) nous avons
    \begin{equation}
        \big| s(a_n)-s(x_0) \big|\leq \epsilon,
    \end{equation}
    d'où la continuité de \( s\) en \( x_0\).
\end{proof}

La proposition suivante dit que si une fonction connaît un saut, alors on peut le lisser par une fonction continue.
\begin{proposition} \label{PropTIeYVw}
    Soit \( f\) continue sur \( \mathopen[ a , x_0 [\) et sur \( \mathopen[ x_0 , b \mathclose]\) avec \( f(x_0^-)<f(x_0)\). En particulier nous supposons que \( f(x^-)\) existe et est finie. Alors pour tout \( \epsilon>0\), il existe une fonction continue \( s\) telle que sur \( \mathopen[ a , b \mathclose]\) on ait \( s\leq f\) et
    \begin{equation}
        \int_a^bs(x)-f(x)\,dx\leq \epsilon.
    \end{equation}
\end{proposition}

\begin{proof}
    Nous notons \( A\) la taille du saut :
    \begin{equation}
        A=f(x_0)-f(x_0^-).
    \end{equation}
    Quitte à changer \( a\) et \( b\), nous pouvons supposer que
    \begin{equation}
        f(x)<f(x_0)+\frac{ A }{ 3 }
    \end{equation}
    pour \( x\in \mathopen[ a , x_0 [\) et 
    \begin{equation}
        f>f(x_0)+\frac{ 2A }{ 3 }
    \end{equation}
    pour \( x\in \mathopen[ x_0 , b \mathclose]\). C'est le théorème des valeurs intermédiaires qui nous permet de faire ce choix.

    Soit \( m(x)\) la droite qui joint le point \( \big( x_0-\epsilon, f(x_0-\epsilon) \big)\) au point \( \big( x_0,f(x_0^+) \big)\). Nous posons
    \begin{equation}
        s(x)=\begin{cases}
            f(x)    &   \text{si \( x<x_0-\epsilon\)}\\
            \max\{ m(x),f(x) \}    &   \text{si \( x_0-\epsilon\leq x\leq x_0\)}\\
            f(x)    &    \text{si $x>x_0$}.
        \end{cases}
    \end{equation}
    En vertu des différents choix effectués, c'est une fonction continue. En effet
    \begin{equation}
        s(x_0-\epsilon)=\max\{ f(x_0-\epsilon),f(x_0,\epsilon) \}=f(x_0-\epsilon)
    \end{equation}
    et 
    \begin{equation}
        s(x_0)=\max\{ m(x_0),f(x_0^+) \}=f(x_0^+)
    \end{equation}
    parce que \( m(x_0)=f(x_0^+)\). En ce qui concerne l'intégrale, si nous posons
    \begin{equation}
        M=\sup_{x,y\in \mathopen[ a , b \mathclose]}| f(x)-f(y) |,
    \end{equation}
    nous avons
    \begin{equation}
        \int_a^bs-f=\int_{x_0-\epsilon}^{x_0}s-f\leq \epsilon M.
    \end{equation}
\end{proof}

\begin{lemma}\label{LemauxrKN}
    Pour tout polynôme \( P\), nous avons la formule
    \begin{equation}
        \lim_{x\to 1^-} (1-x)\sum_{n=0}^{\infty}x^nP(x^n)=\int_0^1P(x)dx.
    \end{equation}
\end{lemma}

\begin{proof}
    D'abord pour \( P=1\), la formule se réduit à la série harmonique connue. Ensuite nous prouvons la formule pour le polynôme \( P=X^k\) et la linéarité fera le reste pour les autres polynômes. Nous avons
    \begin{equation}
        (1-x)\sum_nx^nx^{kn}=(1-x)\sum_n(x^{1+k})^n=\frac{ 1-x }{ 1-x^{1+k} }=\frac{1}{ 1+x+\ldots+x^k }.
    \end{equation}
    Donc
    \begin{equation}
        \lim_{x\to 1^-} (1-x)\sum_nx^nP(x^n)=\frac{1}{ 1+k }.
    \end{equation}
    Par ailleurs, c'est vite vu que
    \begin{equation}
        \int_0^1 x^kdx=\frac{1}{ k+1 }.
    \end{equation}
\end{proof}

\begin{theorem}[Hardy-Littlewood\cite{ytMOpe}]\index{théorème!Hardy-Littlewood}\index{Hardy-Littlewood (théorème)}      \label{ThoPdDxgP}
    Soit \( (a_n)\) une suite réelle telle que
    \begin{enumerate}
        \item
            \( \frac{ a_n }{ n }\) tends vers une constante,
        \item
            \( F(x)=\sum_{n=0}^{\infty}a_nx^n\) a un rayon de convergence \( \geq 1\),
        \item
            \( \lim_{x\to 1^-} F(x)=l\).
    \end{enumerate}
    Alors \( \sum_{n=0}^{\infty}a_n=l\).
\end{theorem}
\index{convergence!suite numérique}
\index{série!nombres}
\index{série!fonctions}
\index{limite!inversion}
\index{approximation!par polynômes}

\begin{proof}
    Quitte à prendre la suite \( b_0=a_0-l\) et \( b_n=a_n\), on peut supposer \( l=0\).

    Soit \( \Gamma\) l'ensemble des fonctions
    \begin{equation}
         \gamma\colon \mathopen[ 0 , 1 \mathclose]\to \eR 
    \end{equation}
    telles que 
    \begin{enumerate}
        \item
            $\sum_{n=0}^{\infty}a_n\gamma(x^n)$ converge pour \( 0\leq x<1\),
        \item
            \( \lim_{x\to 1^-} \sum_{n\geq 0}a_n\gamma(^n)=0\).
    \end{enumerate}
    Ce \( \Gamma\) est un espace vectoriel.
    \begin{subproof}
    \item[Les polynômes sont dans \( \Gamma\)]
        Soit \( \gamma(t)=t^s\). Pour \( 0\leq x<1\) nous avons
        \begin{equation}
            \sum_{n=0}^{\infty}a_n\gamma(x^n)=\sum_{n=0}^{\infty}a_nx^{ns}<\sum_{n=0}^{\infty}a_nx^n.
        \end{equation}
        Donc la condition de convergence est vérifiée. En ce qui concerne la limite,
        \begin{equation}
            \lim_{x\to 1^-} \sum_{n=0}^{\infty}a_nx^{ns}=\lim_{x\to 1^-} F(x^s)=0
        \end{equation}
        parce que par hypothèse, \( \lim_{x\to 1^-} F(x)=0\).

    \item[Définition de la fonction qui va donner la réponse]
        Nous considérons la fonction \( g=\mtu_{\mathopen[ \frac{ 1 }{2} , 1 \mathclose]}\), c'est à dire
        \begin{equation}
            g(t)=\begin{cases}
                0    &   \text{si \( 0\leq t<1/2\)}\\
                1    &    \text{si \( 1/2\leq t\leq 1\)}.
            \end{cases}
        \end{equation}
        Nous montrons que si \( g\in \gamma\), alors le théorème est terminé. Si \( 0\leq x\leq 1\), on a \( 0\leq x^n<1/2\) dès que
        \begin{equation}
            n>-\frac{ \ln(2) }{ \ln(x) }
        \end{equation}
        avec une note comme quoi \( \ln(x)<0\), donc la fraction est positive. Nous désignons par \( N_x\) la partie entière de ce \( n\) adapté à \( x\). L'idée est que la fonction  \( g(x^n)\) est la fonction indicatrice de \(0 \leq n\leq N_x\), et donc
        \begin{equation}
            \sum_{n\geq 0}a_ng(x^n)=\sum_{n=0}^{N_x}a_n.
        \end{equation}
        Mais si \( x\to 1^-\), alors \( N_x\to \infty\), donc
        \begin{equation}
            \lim_{N\to \infty} \sum_{n=0}^Na_n=\lim_{x\to 1^-} \sum_{n=0}^{N_x}a_n=\lim_{x\to 1^-} \sum_{n\in \eN}a_ng(x^n),
        \end{equation}
        et cela fait zéro si \( g\in \Gamma\).
        
    \item[Approximation de \( g\) par des polynômes]

        Nous considérons la fonction
        \begin{equation}
            h(t)=\frac{ g(t)-t }{ t(1-1) }=\begin{cases}
                \frac{1}{ t-1 }    &   \text{si \( t\in \mathopen[ 0 , 1/2 [\)}\\
                \frac{1}{ t }    &    \text{si \( t\in \mathopen[ 1/2 , 1 \mathclose]\)}.
            \end{cases}
        \end{equation}
        La seconde égalité est au sens du prolongement par continuité. La fonction \( h\) est une fonction non continue qui fait un saut de \( -2\) à \( 2\) en \( x=1/2\). En vertu de la proposition \ref{PropTIeYVw} (un peu adaptée), nous pouvons considérer deux fonctions continues \( s_1\) et \( s_2\) telles que
        \begin{equation}
            s_1\leq h\leq s_2
        \end{equation}
        et
        \begin{equation}
            \int_{0}^1s_2-s_1\leq \epsilon.
        \end{equation}
        Notons que l'inégalité \( s_1\leq s_2\) doit être stricte sur au moins un petit intervalle autour de \( x=1/2\). Soient \( P_1\) et \( P_2\), deux polynômes tels que \( \| P_1-s_1 \|_{\infty}\leq \epsilon\) et \( \| P_2-s_2 \|_{\infty}\leq \epsilon\) (ici la norme supremum est prise sur \( \mathopen[ 0 , 1 \mathclose]\)). C'est le théorème de Stone-Weierstrass (\ref{ThoGddfas}) qui nous permet de le faire.

        Nous posons aussi\footnote{À ce niveau, je crois qu'il y a une faute de frappe dans \cite{ytMOpe}.}
        \begin{subequations}
            \begin{align}
                Q_1=P_1+\epsilon\\
                Q_2=P_2-\epsilon.
            \end{align}
        \end{subequations}
        Nous avons
        \begin{equation}
            \int_0^1Q_1-Q_2\leq\int_0^1 Q_1-P_1+P_1-P_2+P_2-Q_2.
        \end{equation}
        Pour majorer cela, d'abord \( Q_1-P_1=P_2-Q2=\epsilon\), ensuite,
        \begin{equation}
            P_1-P_2=P_1-s_1+s_1-s_2+s_2-P_2
        \end{equation}
        dans lequel nous avons \( P_1-s_1\leq \epsilon\), \( s_2-P_2\leq \epsilon\) et \( \int_0^1s_1-s_2\leq\epsilon\). Au final, nous posons \( q=Q_2-Q_1\) et nous avons
        \begin{equation}
            \int_0^1q\leq 5\epsilon.
        \end{equation}
        Enfin nous posons aussi
        \begin{equation}
            R_i(x)=x+x(1-x)Q_i.
        \end{equation}
        Ces polynômes vérifient \( R_i(0)=0\), \( R_i(1)=1\) et
        \begin{equation}
            R_1\leq g\leq R_2
        \end{equation}
        parce que
        \begin{equation}
            Q_1\leq P_1\leq h\leq  P_2\leq Q_2
        \end{equation}
        et
        \begin{equation}
            t+t(1-t)Q_1\leq \underbrace{t+t(1-t)h(t)}_{g(t)}\leq t+t(1-t)Q_2.
        \end{equation}
        
    \item[Preuve que \( g\) est dans \( \Gamma\)]

        D'abord si \( 0\leq x<1\), \( x^N<\frac{ 1 }{2}\) pour un certain \( N\), et alors \( g(x^N)=0\). Du coup la série
        \begin{equation}
            \sum_{n=0}^{\infty}a_ng(x^n)=\sum_{n=0}^{N}a_n
        \end{equation}
        est une somme finie qui converge donc.

        D'autre part nous prenons \( M\) tel que \( | a_n |<\frac{ M }{ n }\) pour tout \( n\). Nous majorons \( \sum_{n \in \eN}a_ng(x^n)\) en utilisant \( R_1\). Mais vu que \( R_1\) est un polynôme, nous pouvons dire que \( | \sum_{n=0}^{\infty}a_nR_1(x^n) |\leq \epsilon\) en prenant \( x\in\mathopen[ \lambda , 1 [\) et \( \lambda\) assez grand. Nous avons :
        \begin{subequations}
            \begin{align}
                \left| \sum_{n=0}^{\infty}a_ng(x^n) \right| &\leq\left| \sum_{n=0}^{\infty}a_ng(x^n)-\sum_{n=0}^{\infty}a_nR_1(x^n) \right| +\underbrace{\left| \sum_{n=0}^{\infty}a_nR_1(x^n) \right|}_{\leq \epsilon} \\
                &\leq \epsilon+\sum_{n=0}^{\infty}| a_n |(g-R_1)(x^n)\\
                &\leq \epsilon+\sum_{n=0}^{\infty}| a_n |(R_2-R_1)(x^n)\\
                &\leq \epsilon+M\sum_{n=0}^{\infty}\frac{ x^n(1-x^n) }{ n }(Q_2-Q_1)(x^n)   &R_2-R_1=x(1-x)(Q_2-Q_1)\\
                &=\epsilon+M\sum_{n=0}^{\infty}\frac{ x^n(1-x^n) }{ n }q(x^n)\\
                &\leq \epsilon+M(1-x)\sum_nx^nq(x^n)   \label{subeqtZXDvu} 
            \end{align}
        \end{subequations}
        où la ligne \eqref{subeqtZXDvu} provient d'une majoration sauvage de \( 1/n\) par \( 1\) et de \( 1-x^n\) par \( 1-x\). Par le lemme \ref{LemauxrKN}, nous avons alors
        \begin{equation}
            \lim_{x\to 1^-} | \sum_na_ng(x^n) |\leq \epsilon+M\int_0^1q\leq 6\epsilon.
        \end{equation}
    \end{subproof}
\end{proof}

%---------------------------------------------------------------------------------------------------------------------------
\subsection{Théorème de Müntz}
%---------------------------------------------------------------------------------------------------------------------------

\begin{theorem}[Théorème de Müntz\cite{jqZSyG,oYGash}]  \label{ThoAEYDdHp}
    Soit \( C=C_0\big( \mathopen[ 0 , 1 \mathclose] \big)\), l'espace des fonctions continues sur \( \mathopen[ 0 , 1 \mathclose]\) muni de la norme \( \| . \|_{\infty}\) ou \( \| . \|_2\) et une suite \( (\alpha_n)\) strictement croissante de nombres positifs. Nous notons \( \phi_{\lambda}\) la fonction \( x\mapsto x^{\lambda}\).

    Alors l'espace \( \Span\{ \phi_{\alpha_n} \}\) est dense dans \( C\) si et seulement si \( \sum_{n=1}^{\infty}\frac{1}{ \alpha_n }\) diverge.
\end{theorem}

Nous prouvons le théorème pour la norme \( \| . \|_2\).
\begin{proof}
    Soit \( m\in \eR^+\); nous notons \( \Delta_N(m)\) la distance entre \( \phi_m\) et \( \Span\{ \phi_{\alpha_1},\ldots, \phi_{\alpha_N} \}\). Cette distance peut être évaluée avec le déterminant de Gram\index{déterminant!Gram} (proposition \ref{PropMsZhIK})
    \begin{equation}
        \Delta_N(m)^2=\frac{ G(\phi_m,\phi_{\alpha_1},\ldots, \phi_{\alpha_N}) }{ G(\phi_{\alpha_1},\ldots, \phi_{\alpha_N}) }.
    \end{equation}
    Pour calculer cela nous avons besoin des produits scalaires\footnote{C'est ici qu'on se particularise à la norme \( \| . \|_2\).}
    \begin{equation}
        \langle \phi_a, \phi_b\rangle =\int_0^1 x^{a+b}dx=\frac{1}{ a+b+1 }.
    \end{equation}
    Donc nous avons à calculer le déterminant
    \begin{equation}
        G(\phi_m,\phi_{\alpha_1},\ldots, \phi_{\alpha_N})=\det\begin{pmatrix}
            \frac{1}{ 2m+1 }   &   \frac{1}{ m+\alpha_1+1 }    &   \cdots    &   \frac{1}{ m+\alpha_N+1 }    \\
            \frac{1}{ m+\alpha_1+1 }   &   \frac{1}{ 2\alpha_1+1 }    &   \cdots    &   \frac{1}{ \alpha_1+\alpha_N+1 }    \\
             \vdots   &   \vdots    &   \ddots    &   \vdots    \\ 
             \frac{1}{ m+\alpha_N+1 }   &   \frac{1}{ \alpha_1+\alpha_N+1 }    &   \cdots    &   \frac{1}{ 2\alpha_N+1 }     
         \end{pmatrix}
    \end{equation}
    dans lequel nous reconnaissons un déterminant de Cauchy (proposition \ref{ProptoDYKA})\index{déterminant!Cauchy} en posant, dans \( \frac{1}{ \alpha_i+\alpha_j+1 }\), \( a_i=\alpha_i\) et \( b_j=\alpha_j+1\). Au passage nous nommons \( \alpha_0=m\) pour se simplifier les notations. Avec ces conventions, étant donné que \( b_j-b_i=a_j-a_i\), les facteurs des deux produits
    \begin{equation}
        \prod_{i<j}(a_j-a_i)\prod_{i<j}(b_j-b_i)
    \end{equation}
    sont les mêmes et donc le numérateur de \( G(\phi_m,\phi_{\alpha_1},\ldots, \phi_{\alpha_N})\) est donné par
    \begin{equation}
        \prod_{i<j}(\alpha_i-\alpha_j)^2\prod_i(\alpha_i-m)^2.
    \end{equation}
    En ce qui concerne le dénominateur, il faut prendre tous les couples \( (i,j)\) avec \( i\) et \( j\) éventuellement égaux à zéro. Nous décomposant cela en trois paquets. Le premier est \( (0,0)\); le second est \( (0,i)\) (chaque couple arrive en fait deux fois parce qu'il y a aussi \( (i,0)\)); et le troisième sont les \( i,j\) tous deux différents de zéro :
    \begin{equation}
        (2m+1)\prod_{ij}(\alpha_i+\alpha_j+1)\prod_i(\alpha_i+m+1)^2.
    \end{equation}
    Notons que dans le produit central, le carré est contenu dans le fait qu'on écrit \( \prod_{ij}\) et non \( \prod_{i<j}\). Nous avons donc
    \begin{equation}
        G(\phi_m,\phi_{\alpha_1},\ldots, \phi_{\alpha_N})=\frac{ \prod_{i<j}(\alpha_i-\alpha_j)^2\prod_i(\alpha_i-m)^2 }{ (2m+1)\prod_{ij}(\alpha_i+\alpha_j+1)\prod_i(\alpha_i+m+1)^2 }.
    \end{equation}
    
    Le calcul de \( G(\phi_{\alpha_1},\ldots, \phi_{\alpha_N})\) est plus simple\footnote{Je crois qu'il y a une faute de frappe dans le dénominateur de \cite{jqZSyG}.} :
    \begin{equation}
        G(\phi_{\alpha_1},\ldots, \phi_{\alpha_N})=\frac{ \prod_{i<j}(\alpha_i-\alpha_j)^2 }{ \prod_{ij}(\alpha_i+\alpha_j+1) }.    
    \end{equation}
    En divisant l'un par l'autre il ne reste que les facteurs comprenant \( m\) et en prenant la racine carré,
    \begin{equation}    \label{EqANiuNB}
        \Delta_N(m)=\frac{1}{ \sqrt{2m+1} }\prod_{i=1}^N\left| \frac{ \alpha_i-m }{ \alpha_i+m+1 } \right| .
    \end{equation}
    
    Nous passons maintenant à la preuve proprement dite. Supposons que \( V=\Span\{ \phi_{\alpha_i},i\in \eN \}\) est dense; alors nous avons en particulier que \( \phi_m\) peut être arbitrairement approché par les \( \phi_{\alpha_i}\), c'est à dire que
    \begin{equation}
        \lim_{N\to \infty} \Delta_N(m)=0
    \end{equation}
    Nous posons 
    \begin{equation}
        u_n=\ln\left( \frac{ \alpha_n-m }{ \alpha_n+m+1 } \right)
    \end{equation}
    et nous prouvons que la série \( \sum_nu_n\) diverge. En effet nous nous souvenons de la formule \( \ln(ab)=\ln(a)+\ln(b)\), de telle sorte que la \( N\)ième somme partielle de \( \sum_nu_n\) est
    \begin{equation}
        \ln\left( \frac{ \alpha_1-m }{ \alpha_1+m+1 }\cdot\ldots\cdot \frac{ \alpha_N-m }{ \alpha_N+m+1 } \right)=\ln\left( \sqrt{2m+1}\Delta_N(m) \right),
    \end{equation}
    qui tends vers \( -\infty\) lorsque \( N\to \infty\).

    Si la suite \( (\alpha_n)\) est majorée et plus généralement si nous n'avons pas \( \alpha_n\to \infty\), alors évidemment la série \( \sum_n\frac{1}{ \alpha_n }\) diverge. Nous supposons donc que \( \lim_{n\to \infty} \alpha_n=\infty\). Nous avons aussi\footnote{Je crois qu'il y a une faute de signe dans la dernière expression de \cite{oYGash}.}
    \begin{equation}
        u_n=\ln\left( \frac{ \alpha_n-m }{ \alpha_n+m+1 } \right)=\ln\left( 1-\frac{ 2m+1 }{ \alpha_n+m+1 } \right)\sim-\frac{ 2m+1 }{ \alpha_n }.
    \end{equation}
    Une justification est donné à l'équation \eqref{EqGICpOX}. Ce que nous avons surtout est
    \begin{equation}
        \sum_n u_n\sim -(2m+1)\sum_n\frac{1}{ \alpha_n }.
    \end{equation}
    Étant donné que la série de gauche diverge, celle de droite diverge\footnote{Nous utilisons le fait que si \( u_n\sum v_n\) en tant que suites et si \( \sum_nu_n\) diverge, alors \( \sum_nv_n\) diverge.}.

    Nous faisons maintenant le sens opposé : nous supposons que la série \( \sum_n1/\alpha_n\) diverge et nous nous posons
    \begin{equation}
        V=\Span\{ \phi_{\alpha_n}\tq n\in \eN \}.
    \end{equation}
    Si \( \alpha_n\to \infty\), alors il suffit de prouver que \( \phi_m\in \bar V\) pour tout \( m\) parce qu'un corollaire du théorème de Stone-Weierstrass \ref{CorRSczQD} montre que \( \Span\{ \phi_k\tq k\in \eN \}\) est dense dans \( C\) pour la norme \( \| . \|_2\). Nous avons :
    \begin{equation}
        u_n\sim\frac{ 2m+1 }{ \alpha }\to 0
    \end{equation}
    et alors \( \Delta_N(m)\to 0\). Dans ce cas nous avons immédiatement \( \phi_m\in \bar V\).

    Si par contre \( \alpha_n\) ne tend pas vers l'infini, nous repartons de l'expression \eqref{EqANiuNB}, nous posons \( \alpha=\sup_i\alpha_i\) et nous calculons :
    \begin{subequations}
        \begin{align}
            \sqrt{2m+1}\Delta_N(m)&=\prod_{i=1}^N\frac{ | \alpha_i-m | }{ \alpha_i+m+1 }\\
            &\leq \prod_{i=1}^N\frac{ \alpha_i+m }{ \alpha_i+m+1 }\\
            &=\prod_{i=1}^N\left( 1-\frac{ 1 }{ \alpha_i+m+1 } \right)\\
            &\leq \prod_{i=1}^N\left( 1-\frac{1}{ \alpha+m+1 } \right)\\
            &=\left( 1-\frac{1}{ \alpha+m+1 } \right)^N.
        \end{align}
    \end{subequations}
    Cette dernière expression tend vers \( 0\) lorsque \( N\to \infty\).
\end{proof}

\begin{example}
    Nous savons depuis le théorème \ref{ThonfVruT} que la somme des inverses des nombres premiers diverge.
\end{example}
