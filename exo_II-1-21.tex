\begin{exercice}\label{exo_II-1-21}

On considère l'équation différentielle linéaire du second ordre
\begin{equation}
	A(t)y''+B(t)y'+C(t)y=0.
\end{equation}

\begin{enumerate}
\item 
Montrer que si l'on possède une solution $y_1(t)$ de cette équation, on peut en déduire la solution générale par \og double quadrature\fg. S'inspirer de l'exercice \ref{exo_II-1-20}.


\item
Montrer que l'équation différentielle peut se ramener à la forme
\begin{equation}
	\big( p(t)y' \big)'+q(t)y=0
\end{equation}
et que si $y_1(t)$ en est une solution, sa solution générale est :
\begin{equation}
	y(t)=y_1(t)\Big(   C_1+C_2\int_{t_0}^t\frac{ d\xi }{ p(\xi)y_1^2(\xi) } \Big).
\end{equation}

\item 
Montrer que si l'on connaît une solution $y_1$ de l'équation
\begin{equation}
	\big( p(t)y' \big)'+q(t)y=0,
\end{equation}
on peut déterminer la solution générale de
\begin{equation}
	\big( p(t)y' \big)'+q(t)y=r(t).
\end{equation}

\item
Résoudre l'équation différentielle
\begin{equation}
	t(2-3t)(t-1)y''-2(2t-1)y'-2(1-3t)y=1.
\end{equation}

\end{enumerate}


\corrref{_II-1-21}
\end{exercice}
% This is part of the Exercices et corrigés de CdI-2.
% Copyright (C) 2008, 2009
%   Laurent Claessens
% See the file fdl-1.3.txt for copying conditions.


