% This is part of Exercices de mathématique pour SVT
% Copyright (C) 2010
%   Laurent Claessens et Carlotta Donadello
% See the file fdl-1.3.txt for copying conditions.

\begin{exercice}\label{exoTD3-0006}

	Dynamique des population : modélisation.

	On considère une population de dix mille micro-organismes. Parmi ces micro-organismes, certains portent un gène qui les rend phosphorescents. La population $X$ est celle portant ce gène et $Y$ est le reste de la population.

	Chaque jour, on mesure la taille de chaque population et on constate que $20\%$ des micro-organismes de $X$ mutent vers $Y$ alors que $5\%$ des micro-organismes de $Y$ mutent vers $X$.

	Sachant qu'au jour $0$, un quart des micro-organismes est phosphorescent et en supposant que le nombre de ces micro-organismes reste constant, quelle est la population de $X$ et de $Y$ au bout de $1$ jour, $2$ jours, $5$ jours et $10$ jours ?

\corrref{TD3-0006}
\end{exercice}
