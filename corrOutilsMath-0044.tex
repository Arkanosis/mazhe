% This is part of Exercices et corrigés de CdI-1
% Copyright (c) 2011
%   Laurent Claessens
% See the file fdl-1.3.txt for copying conditions.

\begin{corrige}{OutilsMath-0044}

    Les dérivées partielles sont données sont
    \begin{equation}
        \begin{aligned}[]
            \frac{ \partial f }{ \partial x }&=2x e^{\cos(y)}\\
            \frac{ \partial f }{ \partial y }&=-x^2\sin(y) e^{\cos(y)}.
        \end{aligned}
    \end{equation}
    En ce qui concerne la dérivée directionnelle, il faut d'abord comprendre que la «direction» d'angle $\pi/6$ est le vecteur
    \begin{equation}
        u=\begin{pmatrix}
            \cos\frac{ \pi }{ 6 }    \\ 
            \sin\frac{ \pi }{ 6 }    
        \end{pmatrix}
    \end{equation}
    et par conséquent,
    \begin{equation}
        \begin{aligned}[]
            \frac{ \partial f }{ \partial u }(x,y)&=u_1\frac{ \partial f }{ \partial x }+u_2\frac{ \partial f }{ \partial y }\\
            &=\cos\left( \frac{ \pi }{ 6 } \right)2x e^{\cos(y)}-\sin\left( \frac{ \pi }{ 6 } \right)x^2\sin(y) e^{\cos(y)}.
        \end{aligned}
    \end{equation}
    En remplaçant $x=1$ et $y=2$, nous trouvons
    \begin{equation}
        \frac{ \partial f }{ \partial u }(1,2)=\sqrt{3} e^{\cos(2)}-\frac{1}{ 2 }\sin(2) e^{\cos(2)}.
    \end{equation}

\end{corrige}
