% This is part of Exercices et corrigés de CdI-1
% Copyright (c) 2011
%   Laurent Claessens
% See the file fdl-1.3.txt for copying conditions.

\begin{corrige}{IntMult0009}

	\begin{enumerate}

		\item
			Il s'agit d'intégrer la fonction $(x,y)\mapsto  e^{\sqrt{x^2+y^2}}$ sur un quart de disque de rayon $1$. Le passage en polaire s'impose donc. L'intégrale à calculer est
			\begin{equation}
				I=\int_0^{\pi/2}d\theta\int_0^1r e^{r}dr.
			\end{equation}
			L'intégrale sur $r$ se fait par partie en posant $u=r$ et $dv=e^rdr$. Nous avons alors
			\begin{equation}
				I=\int_0^{\pi/2}[re^r-e^r]_0^1d\theta=\frac{ \pi }{ 2 }.
			\end{equation}
			
		\item
			Cette intégrale devient subitement plus facile si on intègre d'abord par rapport à $y$ et puis par rapport à $x$ :
			\begin{equation}
				\int_0^4\left( \int_{\sqrt{y}}^2 y e^{x^5}dx \right)dy=\int_0^2dx\int_0^{x^2}y e^{x^5}dy=\frac{ e^{32}-1 }{ 10 }.
			\end{equation}

	\end{enumerate}
	
\end{corrige}
