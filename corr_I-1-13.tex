% This is part of the Exercices et corrigés de CdI-2.
% Copyright (C) 2008, 2009
%   Laurent Claessens
% See the file fdl-1.3.txt for copying conditions.


\begin{corrige}{_I-1-13}

Nous posons $y(x)=\frac{ x-1 }{ x }$, et nous regardons la série
\begin{equation}		\label{EqSeryexotreize}
	\tilde f(y)=\sum_{n=1}^{\infty}\frac{1}{ n }y^n.
\end{equation}
Cette série de puissance converge absolument pour $| y |<1$, voir l'exemple 4 de la page 123bis du cours de première. Cette série converge également simplement en $y=-1$, par le corollaire de la page 123 du même cours\footnote{Un étudiant avait dit se souvenir qu'Abel s'appliquait seulement aux séries alternées; c'est ce corollaire (critère des séries alternées) qui l'a induit en erreur. En effet, Abel (proposition 5, page 122) est plus général, mais s'applique particulièrement bien aux séries alternées.}. Nous sommes dans le cas d'une série de puissance dont le disque de convergence est centré en $0$, et dont le rayon est $1$, mais qui converge (en plus) simplement sur un des bords du disque. Cela est le cadre du théorème \ref{ThoAbelSeriePuiss} qui nous permet de dire que pour tout $\epsilon>0$, la série \eqref{EqSeryexotreize} converge uniformément sur $[-1,1-\epsilon]$.

La fonction $\tilde f(y)$ est donc continue sur $[-1,1-\epsilon]$, et donc en particulier sur $[-1,0]$. Par ailleurs, la fonction $y(x)$ est continue en $x\neq 0$. En tant que composée de fonctions continues, la fonction $f(x)=\tilde f\big( y(x) \big)$ est continue sur $[\frac{ 1 }{2},1]$.

Nous la mettons la série des dérivées sous la forme d'une série de puissances :
\begin{equation}		\label{EqSerieDerrTreize}
	g(x)=\frac{1}{ x^2 }\sum_{n=1}^{\infty} \left( \frac{ x-1 }{ x } \right)^{n-1}.
\end{equation}
Afin d'éviter tout malentendu, nous insistons sur le fait que $g$ est la série des dérivée de la série $f$. Nous ne savons pas encore si $g$ existe (c'est à dire si elle converge), ni si sa somme est la dérivée de $f$. C'est cela que nous allons tenter d'établir maintenant.

Nous posons à nouveau $y(x)=\frac{ x-1 }{ x }$, et nous savons que la série de puissances $\sum_ny^n$ converge uniformément pour $y\in[-1+\epsilon,1-\epsilon]$ pour tout $\epsilon>0$. En repassant aux variables $x$, pour tout $\epsilon>0$, nous avons convergence uniforme de la série 
\begin{equation}
	\sum_{n=0}^{\infty} \left( \frac{ x-1 }{ x } \right)^n, 
\end{equation}
sur le compact $x\in[\frac{ 1 }{ 2-\epsilon },\frac{1}{ \epsilon }]$, ou, pour parler plus simplement, sur $x\in[\frac{ 1 }{2}+\epsilon,a]$ pour tout $\epsilon$ (petit) et $a$ (grand). Nous avons donc également convergence uniforme de la série des dérivées \eqref{EqSerieDerrTreize} sur le même intervalle. Maintenant, le théorème \ref{ThoSerUnifDerr} montre que la série des dérivée est bien la dérivée de la série, c'est à dire que 
\begin{equation}
	g(x)=f'(x)
\end{equation}
sur $]\frac{ 1 }{ 2 },1]$. Notez que la convergence uniforme \emph{sur tout compact} de la série des dérivées est suffisante.

Une bonne nouvelle est qu'il est possibles de sommer explicitement la série $\sum_ky^k$. En effet, il est montré à la page 115 du cours de première que $\sum_{k=0}^n z^k=\frac{ 1-z^{n+1} }{ 1-z }$, donc
\begin{equation}		\label{EqFormSomGeometrze}
	\sum_{k=0}^{\infty}y^n=\lim_{n\to\infty}\frac{1-y^{n+1}}{ 1-y }=\frac{1}{ 1-y },
\end{equation}
lorsque $| y |<1$. Du coup, nous avons simplement
\begin{equation}
	f'(x)=g(x)=\frac{1}{ x^2 }\sum_{n=1}^{\infty}\left( \frac{ x-1 }{ x } \right)^{n-1}=\frac{1}{ x^2 }\sum_{n=0}^{\infty}\left( \frac{ x-1 }{ x } \right)^n=\frac{1}{ x^2 }\left( \frac{1}{  1-\left( \frac{ x-1 }{ x } \right)  } \right)=\frac{1}{ x },
\end{equation}
donc la fonction $f$ a la forme simple $f(x)=\ln(x)+C$. Notez bien le petit jeu de variables de sommation. Au départ $g(x)$ est une somme qui part de $1$ avec un exposant $n-1$, et nous la transformons en une somme qui part de $0$ avec un exposant $n$. C'est cela qui nous permet d'appliquer la formule \eqref{EqFormSomGeometrze}.

 Étant donné que $f(1)=0$, nous avons
\begin{equation}
	f(x)=\ln(x)
\end{equation}
pour tout $x\in[\frac{ 1 }{2}+\epsilon,1]$. Mais nous avons vu que la fonction $f$ était continue sur $[\frac{ 1 }{2},1]$. Étant donné que $\ln(x)$ et $f(x)$ sont deux fonctions continues sur $[\frac{ 1 }{2},1]$ qui sont égales sur tout compact $[\frac{ 1 }{2}+\epsilon,1]$, nous déduisons que ces deux fonctions sont en réalité égales sur tout l'entièreté du compact $[\frac{ 1 }{2},1]$.

En particulier, en $x=\frac{ 1 }{2}$, nous avons
\begin{equation}
	f(\frac{ 1 }{2})=\sum_{n=1}^{\infty}\frac{1}{ n }(-1)^n=\ln(1/2)=-\ln(2).
\end{equation}
\end{corrige}
