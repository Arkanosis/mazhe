Dans ce chapitre nous allons voir comment calculer des intégrales sur des surfaces et volumes de $\eR^3$. De telles intégrales arrivent fréquemment en physique ou en mathématique, et il est important d'être capable de les manipuler correctement, ainsi que de pouvoir les calculer. 

La théorie définira la notion d'intégrale et donnera ensuite le théorème de Fubini (théorème \ref{fub}) qui permet de décomposer une intégrale sur un domaine de $\eR^3$ en plusieurs intégrales de fonctions réelles usuelles sur $\eR$.

Les changements de variables seront ensuite expliqués dans la section \ref{sec_coord}.

Nous n'envisagerons que des fonctions bornées sur des domaines bornées; nous n'étudierons donc pas des intégrales du type $\int_1^{\infty}\frac{1}{ x^2 }dx$ ou bien $\int_{S^2}\ln(y)\,dxdy$.

Dans ce chapitre, vous devez connaître les définitions, les exemples, mais vous ne devez connaître aucune démonstration.
