% This is part of Outils mathématiques
% Copyright (c) 2012
%   Laurent Claessens
% See the file fdl-1.3.txt for copying conditions.

\begin{corrige}{OutilsMath-0117}

    Nous utilisons la paramétrisation avec \( \rho=3\), c'est à dire
    \begin{equation}
        \phi(\theta,\varphi)=\begin{pmatrix}
            6\sin(\theta)\cos(\varphi)    \\ 
            12\sin(\theta)\sin(\varphi)    \\ 
            18\cos(\theta)    
        \end{pmatrix}
    \end{equation}
    avec \( \theta\in\mathopen[ 0 , \pi \mathclose]\) et \( \varphi\in\mathopen[ 0 , 2\pi \mathclose]\). La fonction à intégrer est \( f\big( \phi(\theta,\varphi) \big)=18| \cos(\theta) |\). Nous écrivons donc
    \begin{equation}
        \int_{\phi}f=18\int_0^{\pi}\int_0^{2\pi}| \cos(\theta) | |J_{\phi} |d\varphi d\theta.
    \end{equation}
    Pour déterminer \( J_{\phi}\) nous calculons le produit vectoriel
    \begin{equation}
        T_{\theta}\times T_{\varphi}=\begin{vmatrix}
            e_x    &   e_y    &   e_z    \\
            6\cos(\theta)\cos(\varphi)    &  12\cos(\theta)\sin(\varphi)     &   -18\sin(\theta)    \\
            -6\sin(\theta)\sin(\varphi)    &   12\sin(\theta)\cos(\varphi)    &   0
        \end{vmatrix}=216\sin(\theta)\begin{pmatrix}
            \sin(\theta)\cos(\varphi)    \\ 
            \sin(\theta)\sin(\varphi)    \\ 
            \cos(\theta)    
        \end{pmatrix}.
    \end{equation}
    

\end{corrige}
