% This is part of Exercices et corrigés de CdI-1
% Copyright (c) 2011
%   Laurent Claessens
% See the file fdl-1.3.txt for copying conditions.

%+++++++++++++++++++++++++++++++++++++++++++++++++++++++++++++++++++++++++++++++++++++++++++++++++++++++++++++++++++++++++++
\section{Quelque fautes usuelles}
%+++++++++++++++++++++++++++++++++++++++++++++++++++++++++++++++++++++++++++++++++++++++++++++++++++++++++++++++++++++++++++

Cette section a pour but de vous montrer un certain nombre de fautes que l'on rencontre régulièrement sur les feuilles d'examen. Les affirmations contenues dans cette sections sont donc fausses.

\begin{exercice}
Soit la fonction $f : \mathbb{R}^2\to \mathbb{R}$,
\begin{equation}
	f(x,y)=\begin{cases}
		\frac{ \sin(x^2+y^2+y^3) }{ x^2+y^2 }	&	\text{si $(x,y)\neq (0,0)$}\\
			1				&	 \text{si $(x,y)=(0,0)$}
	\end{cases}
\end{equation}

Étudier la continuité et la différentiabilité de $f$
\end{exercice}

Voici différente fautes qui ont été commises pour cette question.
\begin{enumerate}

\item
Nous avons $\frac{ \sin(y^3) }{ x^2+y^2 }\leq\sin(y^3)$, or $\lim_{y\to 0}\sin(y^3)=0$, donc $\lim_{(x,y)\to(0,0)}\frac{ \sin(y^3) }{ x^2+y^2 }=0 $. La fonction est donc continue en $(0,0)$.

\item
Étudions la limite lorsque $(x,y)$ tend vers $(0,0)$ :
\begin{equation}
	\begin{aligned}[]
		\lim_{ \begin{subarray}{l} (x,y)\to(0,0)\\x=\sqrt{y^3-y^2} \end{subarray} }\frac{ \sin(y^3) }{ x^2+y^2 }=\lim_{y\to 0} \frac{ \sin(y^3) }{ y^3-y^2+y^2 }=\lim_{y\to 0} \frac{ \sin(y^3) }{ y^3 }=1,
	\end{aligned}
\end{equation}
donc si la limite existe, elle est égale à un (parce qu'il existe un chemin sur lequel la limite vaut $1$).

\item
Affin de vérifier la différentiabilité en $(0,0)$, nous étudions l'existence de la limite suivante :
\begin{equation}
	\begin{aligned}[]
		\lim_{t\to 0} \frac{ f(0,0)-f\big( (0,0)+t(u,v) \big) }{ t }=\lim_{t\to 0} \frac{ \sin(t^3v^3) }{ t^2(v^2+u^2)t }=\lim_{t\to 0} \frac{ u^3 }{ (v^2+u^2) },
	\end{aligned}
\end{equation}
donc la limite existe et la fonction est différentiable en $(0,0)$.

\item
Petit calcul :
\begin{equation}
	\lim_{(x,y)\to (0,0)} \frac{ \frac{ \sin(y^3) }{ x^2+y^2 } }{ \sqrt{x^2+y^2} }=\lim_{(x,y)\to (0,0)} \frac{ \sin(y^3) }{ (x^2+y^2)\sqrt{x^2+y^2} }=\frac{ 0 }{ 0 },
\end{equation}
donc la limite n'existe pas.

\item
Soit $u=(u_1,u_2)$. Nous avons
\begin{equation}
	\frac{ \partial f }{ \partial u }(0,0)=
	\begin{cases}
	\lim_{t\to 0} \frac{ f(0,0)-f(0,0) }{ t }=0	&	\text{si $u=(0,0)$}\\
	\lim_{t\to 0} \frac{ f(tu_1,tu_2)-f(0,0) }{ t }	&	 \text{si $u\neq(0,0)$.}
\end{cases}
\end{equation}
Dans le second cas, nous trouvons
\begin{equation}
	\lim_{t\to 0} \frac{ \sin(t^3u_2^3) }{ t^2u_1^2+t^2u_2^2 }\cdot\frac{1}{ t }=\lim_{t\to 0} \left( \frac{ \sin(t^3u_1^3) }{ t^3u_1^3 }\cdot\frac{ u_1^3 }{ u_1^2+u_2^2 } \right)=\frac{ u_1^3 }{ u_1^2+u_2^2 }.
\end{equation}
Donc, la dérivée directionnelle existe dans toutes les directions, et donc $f$ est différentiable en $(0,0)$.


\end{enumerate}


