% This is part of Outils mathématiques
% Copyright (c) 2012
%   Laurent Claessens
% See the file fdl-1.3.txt for copying conditions.

\begin{corrige}{OutilsMath-0142}

    Le changement de variable qui s'invite est
    \begin{subequations}
        \begin{numcases}{}
            u=x+y\\
            v=x-y.
        \end{numcases}
    \end{subequations}
    Ce qui nous intéresse est surtout d'avoir \( x\) et \( y\) en fonction de \( u\) et \( v\), c'est à dire
    \begin{subequations}
        \begin{numcases}{}
            x=\frac{ u+v }{2}\\
            y=\frac{ u-v }{2}
        \end{numcases}
    \end{subequations}
    Nous utilisons donc la paramétrisation
    \begin{equation}
        \phi(u,v)=\begin{pmatrix}
            \frac{ u+v }{2}    \\ 
            \frac{ u-v }{ 2 }    
        \end{pmatrix}
    \end{equation}
    et les bornes \( u\colon 0\to 5\), \( v\colon -2\to 3\). Le jacobien en est donné par la norme du vecteur
    \begin{equation}
        \frac{ \partial \phi }{ \partial u }\times\frac{ \partial \phi }{ \partial v }=\begin{vmatrix}
            e_x    &   e_y    &   e_z    \\
            \frac{ 1 }{2}    &   \frac{ 1 }{2}    &   0    \\
            \frac{ 1 }{2}    &   -\frac{ 1 }{2}    &   0
        \end{vmatrix}=\begin{pmatrix}
            0    \\ 
            0    \\ 
            -1/2    
        \end{pmatrix}.
    \end{equation}
    Nous avons aussi \( f\big( \phi(u,v) \big)=uv\) par le produit remarquable \( x^2-y^2=(x+y)(x-y)=uv\). Bref, l'intégrale à calculer est
    \begin{equation}
        \int_{\phi}f=\int_0^5du\int_{-2}^3dv\frac{ 1 }{2}(uv)=\frac{ 125 }{8}.
    \end{equation}

\end{corrige}
