% This is part of Exercices et corrigés de CdI-1
% Copyright (c) 2011
%   Laurent Claessens
% See the file fdl-1.3.txt for copying conditions.

\documentclass[a4paper,12pt]{article}

\usepackage{latexsym}
\usepackage{amsfonts}
\usepackage{amsmath}
\usepackage{amsthm}
\usepackage{amssymb}
\usepackage{bbm}

\usepackage{hyperref}


\newtheoremstyle{PourExo}{9pt}{9pt}{}{}{\bfseries}{.}{\newline}{}
\theoremstyle{PourExo}\newtheorem{exercice}{Exercice}
\newcommand{\defe}[2]{ {\bf #1} }


\newcommand{\eR}{\mathbbm{R}}

%%%%%%%%%%%%%%%%%%%%%%%%%%
%
%   Les lignes magiques pour le texte en français.
%
%%%%%%%%%%%%%%%%%%%%%%%%

\usepackage[utf8]{inputenc}
\usepackage[T1]{fontenc}

\usepackage{textcomp}
\usepackage{lmodern}
\usepackage[a4paper]{geometry} 
\usepackage[english,frenchb]{babel}


\begin{document}

\title{Travail personnel 2008-2009}
\maketitle

\begin{exercice}

Un ensemble $\Omega\subset\eR^n$ est dit \defe{étoilé par rapport au point}{ensemble étoilé} $x_0\in\Omega$ si pour tout $y \in \Omega$, le segment
\begin{equation*}
	\{ (1-t) x_0 + t y \mid t \in  [0,1] \}
\end{equation*}
est inclus à $\Omega$. L'ensemble $\Omega$ est dit \defe{étoilé}{ensemble étoilé} s'il est étoilé par rapport à un de ses points.

\begin{enumerate}
\item
Soit $f\colon A\subset \eR\to\eR$ une fonction dérivable sur l'ensemble connexe $A$. Montrer que si $f'(a)=0$ pour tout $a\in A$, alors $f$ est constante sur $A$.

\item
Soit $f : \Omega \subset \eR^n \to \eR$ où $\Omega$ est étoilé. Montrer que si $f \in C^1(\Omega,{\eR})$ est telle que $df_a = 0$ pour tout $a \in \Omega$, alors $f$ est constante.

\end{enumerate}
\end{exercice}

\begin{exercice}
Soit $T : A \subset {\eR}^n \to A$ une application. Supposons qu'il existe $p > 1$ un entier tel que $T^p \stackrel{def}{=} \underbrace{T \circ T \circ \ldots \circ T}_{\text{$p$ fois}}$ soit une application contractante.

\begin{enumerate}
\item
À l'aide du théorème de Banach, prouver que $T$ possède un unique point fixe si $A$ est fermé.
\item
Montrer qu'il existe une application $T : A \to A$ qui n'est pas une contraction, mais telle que $T^2$ est une contraction.
\end{enumerate}
\end{exercice}

\begin{exercice}
Considérons un tore dans ${\eR}^3$, c'est-à-dire une surface engendrée par la rotation d'un cercle autour d'une droite coplanaire qui n'intersecte pas ce cercle. De manière équivalente, c'est l'ensemble des points à distance $r$ d'un cercle de rayon $R$, avec $0 < r < R$.
\begin{enumerate}
\item
Montrer que le tore est une variété de dimension $2$, et en donner un atlas.
\item
Écrire l'équation d'un plan tangent au tore en un point $(x_0, y_0, z_0)$.
\item
Calculer le volume intérieur au tore.
\end{enumerate}
\end{exercice}

\end{document}
