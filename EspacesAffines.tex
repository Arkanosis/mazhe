% This is part of Mes notes de mathématique
% Copyright (c) 2011-2012
%   Laurent Claessens
% See the file fdl-1.3.txt for copying conditions.

Ce chapitre provient principalement de \cite{Combes}.

\begin{definition}
    Un \defe{espace affine}{affine!espace}\index{espace!affine} est un ensemble \( \affE\) sur lequel le groupe\footnote{Voir exemple \ref{ExemMaKdwt}.} \( (E,+)\) agit à droite transitivement et librement.
\end{definition}

Étant donné que \( E\) est un groupe commutatif, l'action peut être vue indifféremment à gauche ou à droite. Si \( M\in\affE\) et si \( x\in E\) nous notons \( M+x\) au lieu de \( x\cdot M\) le résultat de l'action de \( x\) sur \( M\).

\begin{remark}  \label{RemAobrlX}
    Lorsque nous écrivons «\( M+x\)», le symbole plus n'est pas une loi de composition interne de \( \affE\), mais une action.
\end{remark}

Soient \( N,M\in\affE\). Par liberté et transitivité de l'action, il existe un unique \( x\in E\) tel que \( M+x=N\). Ce vecteur \( x\) sera noté \( MN\).

\begin{proposition}
    Si \( A,B,C\in\affE\) nous avons les égalités suivantes dans \( E\) :
    \begin{enumerate}
        \item
            \( AB+BC=AC\) (relations de Chasles)\index{relations!de Chasles}\index{Chasles},
        \item
            \( AA=0\),
        \item
            \( AB=-AB\).
    \end{enumerate}
\end{proposition}

Si \( E\) est un espace vectoriel, le groupe \( (E,+)\) agit sur \( E\) par l'action \( t_y(x)=y+x\). Utilisant cette action nous construisons l'\defe{espace affine canonique}{espace!affine!canonique}\index{canonique!espace affine} de \( E\). En particulier nous notons \( \affE_n(\eK)\) l'espace affine canonique de \( \eK^n\) vu comme espace vectoriel sur \( \eK\).

%+++++++++++++++++++++++++++++++++++++++++++++++++++++++++++++++++++++++++++++++++++++++++++++++++++++++++++++++++++++++++++
\section{Repères affines}
%+++++++++++++++++++++++++++++++++++++++++++++++++++++++++++++++++++++++++++++++++++++++++++++++++++++++++++++++++++++++++++

Soit \( E\) un \( \eK\)-espace vectoriel de dimension \( n\) et \( \affE\) un espace affine construit sur \( E\).
\begin{definition}
    Un multiplet \( (A,e_1,\ldots, e_n)\) où \( A\) est un point de \( \affE\) et \( \{ e_i \}\) est une base de \( E\) est un \defe{repère cartésien}{repère!cartésien!espace affine} de \( \affE\).
\end{definition}
Un tel repère donne une bijection
\begin{equation}
    \begin{aligned}
        \phi\colon \eK^n&\to \affE \\
        (x_1,\ldots, x_n)&\mapsto A+\sum_ix_ie_i. 
    \end{aligned}
\end{equation}
Ces nombres \( x_i\) sont les \defe{coordonnées}{coordonnées!dans un espace affine} du point \( A+\sum_ix_ie_i\) dans le repère \( (A,e_i)\).

Soient \( (A,e_i)\) et \( (A',e'_i)\) deux repères cartésiens pour l'espace affine \( \affE\). Soit \( (a_{ij})\) la matrice de changement de base entre \( \{ e_i \}\) et \( \{ e'_i \}\) dans \( E\). Nous voudrions trouver les \( x_i\) en termes des \( x'_i\).

Pour cela nous considérons un point \( M\) dans \( \affE\) et nous l'écrivons dans les deux bases. Cela fournit l'égalité
\begin{equation}        \label{EqcYfuMg}
    A+\sum_ix_ie_i=A'+\sum_ix'_ie'_i.
\end{equation}
Nous considérons les coordonnées \( (a_i)\) de \( A'\) dans le repère \( (A,e_i)\), c'est à dire
\begin{equation}    \label{EqZNwPHE}
    A'=A+\sum_ia_ie_i.
\end{equation}
En substituant \( e'_i=\sum_ka_{jk}e_k\) et \eqref{EqZNwPHE} dans \eqref{EqcYfuMg} nous trouvons
\begin{equation}
    \sum_kx_ke_k=\sum_ka_ke_k+\sum_{jk}a_{jk}x'_je_k,
\end{equation}
et par conséquent
\begin{equation}
    x_k=a_k+\sum_ja_{jk}x'_j.
\end{equation}

%+++++++++++++++++++++++++++++++++++++++++++++++++++++++++++++++++++++++++++++++++++++++++++++++++++++++++++++++++++++++++++
\section{Classification affine des conique}
%+++++++++++++++++++++++++++++++++++++++++++++++++++++++++++++++++++++++++++++++++++++++++++++++++++++++++++++++++++++++++++

Soit une conique \( f(x,y)=0\) avec
\begin{equation}
    f(x,y)=ax^2+2bxy+cy^2+2dx+2ey+f  
\end{equation}
dans le repère \( R=(A,e_i)\). La signature de la quadratique
\begin{equation}
    q(x,y)=ax^2+2bx+cy^2
\end{equation}
ne dépend pas de la base choisie et un changement de variables
\begin{subequations}
    \begin{numcases}{}
        \tilde x=\alpha x+\beta y\\
        \tilde y=\gamma x+\delta y
    \end{numcases}
\end{subequations}
peut nous amener dans trois cas :
\begin{equation}
    q(x,y)=\begin{cases}
        \tilde x^2+\tilde y^2    &   \text{genre ellipse}\\
        \tilde x^2-\tilde y^2    &    \text{genre hyperbole}\\
        \tilde x^2               &  \text{genre parabole}.   
    \end{cases}
\end{equation}
Dans le troisième cas, la matrice de \( q\) est de rang \( 1\).

Nous cherchons maintenant à savoir si un point \( I=(x_0,y_0)\) est un centre de symétrie de \( f(x,y)=0\). Pour cela nous choisissons le repère centré en \( I\), c'est à dire que nous posons 
\begin{subequations}
    \begin{numcases}{}
        x=x_0+\tilde x\\
        y=y_0+\tilde y.
    \end{numcases}
\end{subequations}
Un peu de calcul montre qu'alors la conique s'écrit
\begin{equation}
    f(x_0,y_0)+q(\tilde x,\tilde y)+(2ax_0+2by_0+2d)\tilde x+(2bx_0+2cy_0+2e)\tilde y=0.
\end{equation}
Le point \( I\) sera un centre de symétrique si les termes linéaires en \( \tilde x\) et \( \tilde y\) s'annulent, c'est à dire si
\begin{subequations}        \label{SyskhiOvW}
    \begin{numcases}{}
        ax_0+by_0+d=0\\
        bx_0+cy_0+e=0.
    \end{numcases}
\end{subequations}
Nous supposons que \( (d,e)\neq (0,0)\), sinon la conique de départ serait déjà centrée. Le déterminant du système \eqref{SyskhiOvW} est 
\begin{equation}
    \delta=ac-b^2.
\end{equation}
Si ce dernier est différent de zéro, le système possède une unique solution et la conique aura alors un unique centre de symétrie.

Si le déterminant du système est nul, il y a soit pas de centre de symétrie, soit une infinité. Dans le premier cas nous sommes en présence d'une parabole, et dans le second cas de deux droites parallèles.

\begin{example}
    Soit 
    \begin{equation}    \label{EqOgsEcz}
        f(x,y)=x^2+2xy-y^2-6x+2y-1=0
    \end{equation}
    donnée dans le repère affine \( R=(A,\{ e_i \})\). Nous commençons par étudier la signature de \( q(x,y)=x^2+2xy-y^2\) dont la matrice symétrique est
    \begin{equation}
        Q=\begin{pmatrix}
            1    &   1    \\ 
            1    &   -1    
        \end{pmatrix}.
    \end{equation}
    Son polynôme caractéristique est \( \lambda^2-2\) sont les racines sont \( \pm\sqrt{2}\). La signature est donc \( (1,-1)\) et nous sommes en présence d'une conique de genre hyperbole. Nous cherchons le centre en posant \( x=\tilde x+x_0\), \( y=\tilde y+y_0\). Le système à résoudre est
    \begin{subequations}
        \begin{numcases}{}
            x_0+y_0-3=0\\
            x_0-y_0+1=0,
        \end{numcases}
    \end{subequations}
    dont l'unique solution est \( (x_0,y_0)=(1,2)\). Nous considérons le repère centré en \( (x_0,y_0)\), c'est à dire le repère
    \begin{equation}
        R'=(I,\{ e_i \})
    \end{equation}
    avec \( I=A+x_0e_1+y_0e_2\) où \( A\) est l'origine du repère dans lequel l'équation \eqref{EqOgsEcz} était donnée.

    Par construction dans ce repère nous avons la conique
    \begin{equation}
        f(x_0,y_0)+q(\tilde x,\tilde y)=0,
    \end{equation}
    c'est à dire
    \begin{equation}
        \tilde x^2+2\tilde x\tilde y-\tilde y^2=0.
    \end{equation}
    Maintenant la nous avons une quadrique centrée nous voulons la mettre sous une forme plus canonique :
    \begin{equation}
        \left( \frac{1}{ \sqrt{2} }(\tilde x+\tilde y) \right)^2-\tilde y^2-1=0.
    \end{equation}
    Nous posons donc
    \begin{subequations}
        \begin{numcases}{}
            X=\frac{1}{ \sqrt{2} }(\tilde x+\tilde y)\\
            Y=\tilde y,
        \end{numcases}
    \end{subequations}
    et nous trouvons l'hyperbole
    \begin{equation}
        X^2-Y^2-1=0.
    \end{equation}
    Cela revient à faire le changement de base
    \begin{subequations}    \label{EqfiVwym}
        \begin{numcases}{}
            e'_1=\sqrt{2}e_1\\
            e'_2=-e_1+e_2.
        \end{numcases}
    \end{subequations}
    Pour rappel, les vecteurs de bases se transforment avec la matrice inverse des coefficients. Étant donné que
    \begin{equation}
        \begin{pmatrix}
            X    \\ 
            Y    
        \end{pmatrix}=\begin{pmatrix}
            1/\sqrt{2}    &   1/\sqrt{2}    \\ 
            0    &   1    
        \end{pmatrix}\begin{pmatrix}
            \tilde x    \\ 
            \tilde y    
        \end{pmatrix},
    \end{equation}
    nous avons
    \begin{equation}
        \begin{pmatrix}
            e'_1    \\ 
            e'_2    
        \end{pmatrix}=\begin{pmatrix}
            1/\sqrt{2}    &   1/\sqrt{2}    \\ 
            0    &   1    
        \end{pmatrix}^{-1}\begin{pmatrix}
            e_1    \\ 
            e_2    
        \end{pmatrix}.
    \end{equation}
    C'est de là que provient le changement \eqref{EqfiVwym}.
\end{example}

%+++++++++++++++++++++++++++++++++++++++++++++++++++++++++++++++++++++++++++++++++++++++++++++++++++++++++++++++++++++++++++
\section{Applications affines}
%+++++++++++++++++++++++++++++++++++++++++++++++++++++++++++++++++++++++++++++++++++++++++++++++++++++++++++++++++++++++++++

\begin{definition}
    Soient \( \affE\) et \( \affE'\) deux espaces affines sur les espaces vectoriels \( E\) et \( E'\) (sur le même corps \( \eK\)). Une application \( f\colon \affE\to \affE'\) est dite \defe{affine}{affine!application} si il existe une application linéaire \( u\colon E\to E'\) telle que pour tout \( M\in\affE\) on ait
    \begin{equation}    \label{EqMqIoWX}
        f(M+x)=f(M)+u(x).
    \end{equation}
\end{definition}
\begin{remark}
    La condition \eqref{EqMqIoWX} pour tout \( M\in\affE\) est équivalente à demander 
    \begin{equation}
        f\circ t_x=t_{u(x)}\circ f
    \end{equation}
    pour tout \( x\in E\).
\end{remark}

\begin{proposition}
    Soit \( f\) une application affine.
    \begin{enumerate}
        \item
            Une application linéaire vérifiant la condition \eqref{EqMqIoWX} est unique. Nous la noterons \( u_f\).
        \item
            L'application \( u_f\) est injective si et seulement si \( f\) est injective.
        \item
            L'application \( u_f\) est surjective si et seulement si \( f\) est surjective.
    \end{enumerate}
    Si \( \affE\) et \( \affE'\) ont même dimension finie, alors en plus \( f\) est injective si et seulement si \( f\) est surjective.
\end{proposition}

\begin{remark}
    La fonction linéaire \( u_f\) qui vérifie \( f(M+x)=f(M)+u_f(x)\) ne dépend pas du point \( M\). En effet si
    \begin{subequations}
        \begin{align}
            f(M+x)&=f(M)+u(x)\\
            f(N+y)&=f(N)+v(y).
        \end{align}
    \end{subequations}
    En effet si \( N=M+a\) nous avons d'une part que
    \begin{equation}
        f(N+y)=f(M+y+a)=f(M)+u(y+a),
    \end{equation}
    et d'autre part
    \begin{equation}
        f(N+y)=f(M+a)+v(y)=f(M)+u(a)+v(y).
    \end{equation}
    Par conséquent \( u=v\).
\end{remark}

\begin{proposition}
    Si \( f\colon \affE\to \affE'\) et \( g\colon \affE\to \affE''\) sont des applications affines, alors \( g\circ f\colon \affE\to \affE''\) est affine et \( u_{g\circ f}=u_g\circ u_f\).
\end{proposition}

\begin{proof}
    Si \( M\in\affE\) et \( x\in E\) nous avons
    \begin{equation}
        \begin{aligned}[]
            (g\circ f)(M+x)&=g\big( f(M)+u_f(x) \big)\\
            &=f\big( f(M) \big)+u_g\big( u_f(x) \big)\\
            &=(g\circ f)(M)+(u_g\circ u_f)(x).
        \end{aligned}
    \end{equation}
\end{proof}

\begin{theorem}
    Soient \( \affE\) et \( \affE'\) deux espaces affines de dimensions finies \( p\) et \( q\) sur \( \eK\). Soient les repères cartésiens \( R=(O,\{ e_i \})\) et \( R'=(O',\{ e_i' \})\). Une application \( f\colon \affE\to \affE'\) est affine si et seulement si il existe une matrice \( a\in\eM_{p,q}(\eK)\) et \( b\in \eK^q\) tels que
    \begin{equation}    \label{EqCmNHjs}
        f(x)=b+ax.
    \end{equation}
\end{theorem}

\begin{remark}
    L'équation \eqref{EqCmNHjs} est écrite en utilisant un abus de notation entre le vecteur \( x\in \eK^p\) et le point de \( \affE\) qui est représenté par \( x\) dans le repère \( (A,\{ e_i \})\).    
\end{remark}

%+++++++++++++++++++++++++++++++++++++++++++++++++++++++++++++++++++++++++++++++++++++++++++++++++++++++++++++++++++++++++++
\section{Isomorphismes}
%+++++++++++++++++++++++++++++++++++++++++++++++++++++++++++++++++++++++++++++++++++++++++++++++++++++++++++++++++++++++++++

\begin{definition}
    Un \defe{isomorphisme}{isomorphisme!espace affine} entre les espaces affines \( \affE\) \( \affE'\) est une application affine \( f\colon \affE\to \affE'\) inversible dont l'inverse est affine.
\end{definition}

\begin{proposition} \label{PropxtFeDE}
    Une application affine bijective est un isomorphisme. Si \( f\) est un isomorphisme d'espaces affines, alors \( u_{f^{-1}}=(u_f)^{-1}\).
\end{proposition}

\begin{proposition}
    Un espace affine de dimension finie \( n\) sur un corps \( \eK\) est isomorphe à l'espace affine canonique \( \affE_n(\eK)\).
\end{proposition}

\begin{proof}
    Si nous considérons le repère \( R=(A,\{ e_i \})\) de l'espace affine \( \affE\) alors l'application
    \begin{equation}
        \begin{aligned}
            \varphi\colon \eK^n&\to \affE \\
            (x_1,\ldots,x_n)&\mapsto A+\sum_ix_ie_i 
        \end{aligned}
    \end{equation}
    est un isomorphisme.
\end{proof}

%+++++++++++++++++++++++++++++++++++++++++++++++++++++++++++++++++++++++++++++++++++++++++++++++++++++++++++++++++++++++++++
\section{Sous espaces affines}
%+++++++++++++++++++++++++++++++++++++++++++++++++++++++++++++++++++++++++++++++++++++++++++++++++++++++++++++++++++++++++++

\begin{definition}
    Soit \( \affE\) un espace affine sur l'espace vectoriel \( E\). Un \defe{sous-espace affine}{affine!sous-espace} de \( \affE\) est une orbite de l'action d'un sous-espace vectoriel de \( E\).
\end{definition}

Si \( \affF\) est un sous ensemble de \( \affE\), il sera un sous-espace affine de \( \affE\) si et seulement si l'ensemble
\begin{equation}
    F=\{ AB\tq A,B\in\affF \}
\end{equation}
est un sous-espace vectoriel de \( E\). Dans ce cas nous disons que \( F\) est la \defe{direction}{direction!sous-espace affine} de \( \affF\). Si \( A\in\affF\), alors l'orbite de \( A\) sous \( F\) est \( \affF\). La \defe{dimension}{dimension!sous espace affine} de \( \affF\) est la dimension de sa direction.

Si \( \affF\) et \( \affG\) sont des sous-espaces affines de \( \affE\) de directions \( F\) et \( G\), nous disons que \( \affF\) est \defe{parallèle}{parallèle!sous-espaces affines} à \( \affG\) si \( F\subset G\).

\begin{proposition}
    Soit \( \affF\) un sous-espace affine de dimension \( k\) dans l'espace affine \( \affE\) de dimension \( n\). Alors il existe une application affine \( f\colon \affE\to \eK^{n-k}\) telle que \( \affF=f^{-1}(0)\).
\end{proposition}

\begin{proof}
    Soit \( F\) la direction de \( \affF\) et \( A\in\affF\). Nous considérons une base \( \{ e_i \}\) adaptée à \( F\) au sens \( \{ e_1,\ldots, e_k \}\) est une base de \( F\). Nous considérons maintenant le repère cartésien \( (A,\{ e_i \})\) avec \( A\in\affF\) et nous construisons l'application affine
    \begin{equation}
        \begin{aligned}
            f\colon \affE&\to \eK^{n-k} \\
            A+\sum_{i=1}^nx_ie_i&\mapsto \begin{pmatrix}
                x_{k+1}    \\ 
                \vdots    \\ 
                x_n    
            \end{pmatrix}.
        \end{aligned}
    \end{equation}
    Par construction nous avons \( f(M)=0\) si et seulement si \( M\in\affF\).
\end{proof}

\begin{proposition}[\cite{Combes}]      \label{PropomhBwi}
    Soit \( \sigma\) une partie de l'espace affine \( \affE\).
    \begin{enumerate}
        \item
            L'intersection de tous les sous-espaces affines contenant \( \sigma\) est un sous-espace affine, noté \( \affF\).
        \item
            Si \( A\in \sigma\), alors la direction de \( \affF\) est le sous-espace vectoriel 
            \begin{equation}        \label{EqnRAUfg}
                F=\Span\{ \overrightarrow{ AM }\tq M\in \sigma \}.
            \end{equation}
    \end{enumerate}
\end{proposition}
Le sous-espace affine donné par la proposition \ref{PropomhBwi} est le sous-espace affine \defe{engendré}{sous-espace!affine engendré par une partie} par la partie \( \sigma\).

\begin{proposition}
    Soit \( \affE\) un espace affine de dimension \( n\) sur \( \eK\), soit \( f\colon \affE\to \eK^r\) une fonction affine. Pour tout \( a=(a_1,\ldots, a_r)\in \eK^r\), l'ensemble \( f^{-1}(a)\) est un sous-espace affine de dimension \( \dim\ker(u_f)\).
\end{proposition}

\begin{proof}
    Nous considérons le repère \( (A,\{ e_i \})\) de \( \affE\). Étant donné que \( f\) est affine nous avons
    \begin{equation}
        f\big( A+\sum_ix_ie_i \big)=f(A)+u_f\big( \sum_ix_ie_i \big).
    \end{equation}
    Nous avons donc \( f\big( A+\sum_ix_ie_i \big)=a\) lorsque
    \begin{equation}
        u_f(\sum_ix_ie_i)=a-f(A).
    \end{equation}
    Nous avons donc
    \begin{equation}
        f^{-1}(a)=A+(u_f)^{-1}\big( a-f(A) \big),
    \end{equation}
    dont la dimension est le rang de \( (u_f)^{-1}=u_{f^{-1}}\) (proposition \ref{PropxtFeDE}). Le rang de \( (u_f)^{-1}\) est le dimension du noyau de \( u_f\).
\end{proof}

\begin{proposition}     \label{PropPoNpPz}
    Soit \( A\) un ensemble convexe dans un espace vectoriel et \( v_1,\ldots, v_n\) des éléments de \( A\). Alors toute combinaison
    \begin{equation}
        a_1v_1+\ldots +a_nv_n
    \end{equation}
    telle que \( a_1+\ldots +a_n=1\) et \( a_i\in\mathopen[ 0 , 1 \mathclose]\) appartient à \( A\).
\end{proposition}

\begin{proof}
    Nous prouvons la proposition pour \( n=3\). Nous devons trouver des nombres \( t_1,t_2\in \mathopen[ 0 , 1 \mathclose]\) tels que
    \begin{equation}
        t_2\big( t_1v_1+(1-t_1)v_2 \big)+(1-t_2)v_3=av_1+bv_2+cv_3.
    \end{equation}
    La réponse est immédiatement donnée par
    \begin{subequations}
        \begin{align}
            t_2a=1-c\\
            t_1=a/t_2.
        \end{align}
    \end{subequations}
    Étant donné que \( c\in \mathopen[ 0 , 1 \mathclose]\) nous avons \( t_2\in\mathopen[ 0 , 1 \mathclose]\). En ce qui concerne \( t_1\) nous avons
    \begin{equation}
        t_1=\frac{ a }{ t_2 }\leq \frac{ 1-c }{ 1-c }=1.
    \end{equation}
\end{proof}

%+++++++++++++++++++++++++++++++++++++++++++++++++++++++++++++++++++++++++++++++++++++++++++++++++++++++++++++++++++++++++++
\section{Barycentre}
%+++++++++++++++++++++++++++++++++++++++++++++++++++++++++++++++++++++++++++++++++++++++++++++++++++++++++++++++++++++++++++

Soit \( \affE\) un espace affine sur le \( \eK\)-espace vectoriel \( E\). Un couple \( (A,\lambda)\) avec \( A\in \affE\) et \( \lambda\in \eK\) est un \defe{point pondéré}{point!pondéré}.

\begin{lemma}[\cite{sZiwBQ}]        \label{LemtEwnSH}
    Soit une famille de points pondérés \( \{ (A_i,\lambda_i) \}_{i=1\ldots r}\). Si \( \sum_i\lambda_i\neq 0\), alors il existe un unique \( G\in \affE\) tel que
    \begin{equation}
        \sum_{i=1}^r\lambda_i\overrightarrow{ GA_i }=0.
    \end{equation}
\end{lemma}

Le point \( G\) donné par le lemme \ref{LemtEwnSH} est le \defe{barycentre}{barycentre} des points pondérés \( (A_i,\lambda_i)\).

Notons que l'on peut toujours supposer que \( \sum_i\lambda_i=1\) parce que le barycentre ne change pas lorsque tous les \( \lambda_i\) sont multipliés par un même nombre.

Le théorème suivant donné quelque caractérisations équivalentes du barycentre.
\begin{theorem}[\cite{sZiwBQ}]      \label{ThoIJVzxr}
    Soient \( \{ (A_i,\lambda_i) \}_{i=1,\ldots, r}\) une famille de points pondérés. Les conditions suivantes sur le point \( G\in \affE\) sont équivalentes.
    \begin{enumerate}
        \item
            Le point \( G\) est barycentre de la famille.
        \item
            Pour tout \( \alpha\in \eR^*\), \( \sum_i(\alpha\lambda_i)\overrightarrow{ GA_i }=0\).
        \item
            Il existe \( A\in\affE\) tel que \( \big( \sum_i\lambda_i \big)\overrightarrow{ AG }=\sum_i\lambda_i\overrightarrow{ AA_i }\).
        \item   \label{ItemEgOQBX}
            Pour tout \( B\in\affE\), nous avons \( \big( \sum_i\lambda_i \big)\overrightarrow{ BG }=\sum_i\lambda_i\overrightarrow{ BA_i }\).
    \end{enumerate}
\end{theorem}

\begin{definition}
    Si \( A,B\in \affE\), le \defe{segment}{segment!dans un espace affine} \( [AB]\) est l'ensemble des barycentres de \( A\) et \( B\) pondérés par des poids positifs (ouvert ou fermé suivant que l'on accepte que l'un ou l'autre des poids soit nul).
\end{definition}

Lorsque tous les \( \lambda_i\) sont égaux, nous parlons d'\defe{isobarycentre}{isobarycentre}. Autrement dit, l'isobarycentre des points \( A_i\) est le barycentre des points pondérés \( (A_i,1)\).

\begin{proposition}
    Une partie \( \affF\) des \( \affE\) est un sous-espace affine si et seulement si elle est stable par barycentrisation.
\end{proposition}

\begin{proof}
    Soit \( \affF\) une sous-espace affine de direction \( F\) et \( A_1,\ldots, A_n\) des points de \( \affF\). Nous devons voir que le barycentre des points \( A_i\) pondérés de n'importe quelles masses appartient à \( \affF\). Pour ce faire nous faisons appel à la caractérisation \ref{ItemEgOQBX} du théorème \ref{ThoIJVzxr} : pour tout \( B\in\affF\), 
    \begin{equation}
        \overrightarrow{ BG }=\sum_i\lambda_i\overrightarrow{ BA_i }.
    \end{equation}
    Vu que \( B\) et \( A_i\) sont dans \( \affF\), nous avons \( \overrightarrow{ BA_i\in F }\) et donc \( \overrightarrow{ BG }\in F\). Mais comme \( B\in\affF\), le point \( G\) est à son tour dans \( \affF\).

    Réciproquement, nous supposons que \( \affF\) est stable par barycentrisme. Nous voudrions montrer que l'ensemble
    \begin{equation}        \label{EqCmyWGi}
        F=\{ \overrightarrow{ AB }\tq A,B\in \affF \}
    \end{equation}
    est un sous-espace vectoriel. Soit \( A\in\affF\). Nous commençons par prouver que les vecteurs de la forme \( \overrightarrow{ AX }\) (\( X\in \affF\)) forment un espace vectoriel. Considérons \( \overrightarrow{ AX }+\overrightarrow{ AY }\) qui est un élément de \( E\); il existe donc \( V\in \affE\) tel que
    \begin{equation}
        \overrightarrow{ AV }=\overrightarrow{ AX }+\overrightarrow{ AY }.
    \end{equation}
    Par les relations de Chasles,
    \begin{equation}
        \overrightarrow{ AV }=\overrightarrow{ AV }+\overrightarrow{ VX }+\overrightarrow{ AV }+\overrightarrow{ VY },
    \end{equation}
    donc
    \begin{equation}
        0=\overrightarrow{ VX }-\overrightarrow{ VA }+\overrightarrow{ VY },
    \end{equation}
    ce qui prouve que \( V\) est un barycentre de \( X,A,Y\), et donc que \( V\in\affF\). De la même manière si \( W\in \affE\) est défini par \( \overrightarrow{ AW }=\mu \overrightarrow{ AX }\), alors
    \begin{equation}
        \overrightarrow{ AW }=\mu\overrightarrow{ AX }=\mu(\overrightarrow{ AW }+\overrightarrow{ WX }),
    \end{equation}
    ce qui signifie que
    \begin{equation}
        (1-\mu)\overrightarrow{ AW }+\mu\overrightarrow{ XW }=0
    \end{equation}
    et que \( W\) est un barycentre.

    Afin de montrer que \eqref{EqCmyWGi} est bien un espace vectoriel, nous devons considérer \( A,B,X,Y\in\affF\) et prouver que \( \overrightarrow{ AX }+\overrightarrow{ BY }\in F\). Nous avons
    \begin{subequations}
        \begin{align}
            \overrightarrow{ AX }+\overrightarrow{ BY }&=\overrightarrow{ AX }+\overrightarrow{ BA }+\overrightarrow{ AY }\\
            &=\overrightarrow{ AV }+\overrightarrow{ BA }   &\text{\( V\) est celui donné plus haut}\\
            &=\overrightarrow{ AV }-\overrightarrow{ AB }  \\
            &=\overrightarrow{ AV }+\overrightarrow{ AW }   &\text{\( W\) est donné par \( \mu=-1\).}\\
            &=\overrightarrow{ AV' }.
        \end{align}
    \end{subequations}
\end{proof}

\begin{proposition}[\cite{sZiwBQ}]      \label{PropBVbCOS}
    Soient \( A_0,\ldots, A_r\) des points de \( \affE\). L'ensemble des barycentres de ces points (avec des masses de somme \( 1\)) est le sous-espace affine engendré par les \( A_i\) que nous nommons \( \affF\).
\end{proposition}

\begin{proof}
    Soit \( G\) le barycentre associé aux poids \( \lambda_i\). Nous avons
    \begin{equation}
        G=A_0+\overrightarrow{ A_0G }=A_0+\sum_{i=1}^r\lambda_i\overrightarrow{ A_0A_i }.
    \end{equation}
    Notons que les vecteurs \( \overrightarrow{ A_0A_i }\) sont dans la direction du sous-espace affine engendré par les \( A_i\) par \eqref{EqnRAUfg}. Donc \( G\) est bien dans \( \affF\).

    Inversement si $X$ est dans \( \affF\), on a
    \begin{equation}
        X=A_0+\sum_i\lambda_i\overrightarrow{ A_0A_i }
    \end{equation}
    parce que \( \sum_i\lambda_i\overrightarrow{ A_0A_i }\) est un élément général de la direction de \( \affF\). Du coup
    \begin{equation}
        \overrightarrow{ A_0X }=\sum_i\lambda_i\overrightarrow{ A_0A_i },
    \end{equation}
    et en utilisant la relation de Chasles sur chacun des \( \overrightarrow{ A_0A_i }\),
    \begin{equation}
        \overrightarrow{ A_0X }=\sum_i\lambda_i\big( \overrightarrow{ A_0X }+\overrightarrow{ XA_i } \big).
    \end{equation}
    De là nous concluons que
    \begin{equation}
        \big( 1-\sum_i\lambda_i \big)\overrightarrow{ A_0X }+\sum_i\lambda_i\overrightarrow{ A_iX }=0,
    \end{equation}
    ce qui signifie précisément que \( X\) est un barycentre des \( A_i\).
\end{proof}

\begin{proposition}
    Soient \( r+1\) point \( A_0,\ldots, A_r\) dans \( \affE\). Le sous-espace affine engendré par les \( A_i\) est au plus de dimension \( r\).
\end{proposition}

\begin{proof}
    La direction de l'espace engendré \( \Aff\{ A_i \}\) est l'espace
    \begin{equation}
        \Span\{ \overrightarrow{ A_0A_i }_{i=1,\ldots, r} \}
    \end{equation}
    qui est engendré par \( r\) vecteurs et donc est au plus de dimension \( r\).
\end{proof}

%+++++++++++++++++++++++++++++++++++++++++++++++++++++++++++++++++++++++++++++++++++++++++++++++++++++++++++++++++++++++++++
\section{Repères, coordonnées cartésiennes et barycentriques}
%+++++++++++++++++++++++++++++++++++++++++++++++++++++++++++++++++++++++++++++++++++++++++++++++++++++++++++++++++++++++++++

\begin{definition}
    On dit que les points \( A_0,\ldots, A_r\in \affE\) sont \defe{affinement indépendants}{indépendance!affine} si le sous-espace affine engendré est de dimension \( r\).
\end{definition}

\begin{proposition}[\cite{sZiwBQ}]  \label{PropGAneHg}
    Pour \( r+1\) points \( A_0,\ldots, A_r\) dans \( \affE\), les propriétés suivantes sont équivalentes.
    \begin{enumerate}
        \item
            Les \( A_i\) sont affinement indépendants.
        \item
            Pour tout \( i=0,\ldots, r\), le point \( A_i\) n'est pas dans \( \Aff\{ A_0,\ldots, \hat A_i,\ldots, A_r \}\).
        \item\label{ItemrAzkIl}
            Les points \( A_0,\ldots, A_{r-1}\) sont affinement indépendants et \( A_r\notin\Aff\{ A_0,\ldots, A_{r+1} \}\).
        \item
            Il existe \( i\) tel que les vecteurs \( \overrightarrow{ A_kA_i }\) (\( k\in i\)) sont linéairement indépendants.
        \item\label{ItemFBfcuq}
            Pour tout \( i\in\{ 1,\ldots, r \}\), les vecteurs \( \overrightarrow{ A_kA_i }\) (\( k\neq i\)) sont linéairement indépendants.
    \end{enumerate}
\end{proposition}

Notons à propos de la condition \ref{ItemrAzkIl} que l'existence d'un \( i\) tel que \( A_i\notin\Aff\{ A_0,\ldots, \hat A_i,\ldots, A_r \}\) n'implique pas l'indépendance des \( r+1\) points. En effet dans \( \eR^2\) nous considérons les \( 4\) points \( A_0=(0,0)\), \( A_1=(1,0)\), \( A_2=(2,0)\) et \( A_3=(0,1)\). Évidemment le point \( A_3\) n'est pas dans l'espace engendré par les trois autres; il n'empêche que ces points ne sont pas affinement indépendants parce que la direction est de dimension \( 2\) au lieu de \( 3\).

\begin{definition}  \label{DefguuwEO}
    Soit \( \affE\) un espace affine de dimension \( n\) et \( \affF\) un sous-espace affine de dimension. Un \defe{repère affine}{repère!affine} de \( \affF\) est la donnée de \( k+1\) points affinement indépendants de \( \affF\).
\end{definition}
Si \( \{ A_0,\ldots, A_n \}\) est un repère affine, le point \( A_0\) est l'\defe{origine}{origine!repère affine}. C'est un choix complètement arbitraire; et c'est bien cet arbitraire qui nous amènera à considérer les coordonnées barycentriques au lieu des coordonnées cartésiennes.

Soit \( M\in \affE\); par définition nous avons
\begin{equation}
    M=A_0+\overrightarrow{ A_0M }.
\end{equation}
Mais nous savons que les vecteurs \( \overrightarrow{ A_0A_i }\) forment une base de \( E\), nous avons donc des nombres \( \lambda_i\) tels que
\begin{equation}
    \overrightarrow{ A_0M }=\sum_{i=1}^n\lambda_i\overrightarrow{ A_0A_i }.
\end{equation}
Les nombres \( \lambda_i\) ainsi construits sont les \defe{coordonnées cartésiennes}{coordonnées!cartésiennes!dans un espace affine} du point \( M\) dans le repère \( \{ A_0,\ldots, A_n \}\) d'origine \( A_0\).

À partir de ces coordonnées, le point \( M\in\affE\) se retrouve par la formule
\begin{equation}
    M=A_0+\sum_{i=1}^n\lambda_i\overrightarrow{ A_0A_i }.
\end{equation}

Les coordonnées barycentriques sont données par la proposition suivante.
\begin{proposition}[\cite{sZiwBQ}]
    Soient \( A_0,\ldots, A_r\) des points affinement indépendants dans \( \affE\) et \( \affF=\Aff\{ A_0,\ldots, A_r \}\). Tout point \( M\in\affF\) s'écrit de façon unique comme barycentre des \( A_i\) affectés de poids \( \lambda_i\) tels que \( \sum_{i=0}^r\lambda_i=1\).
\end{proposition}
Les nombres \( \lambda_i\) ainsi définis sont les \defe{coordonnées barycentriques}{coordonnées!barycentriques} de \( M\) dans le repère \( (A_0,\ldots, A_r)\) de \( \affF\).

\begin{proof}
    Nous avons vu plus haut (définition \ref{DefguuwEO}) que l'affine indépendance des points \( A_i\) assurait que \( (A_0,\ldots, A_r)\) était un repère de \( \affF\).

    En ce qui concerne l'existence de l'écriture de \( M\) comme barycentre, nous savons que les sous-espace affine sont exactement les ensembles de barycentres (proposition \ref{PropBVbCOS}), c'est à dire que si on a des points dans un sous-espace affine, alors les barycentres de ces points est encore dans le sous-espace affine.

    L'unicité est comme suit. Si \( M\) est barycentre des \( A_i\) avec poids \( \lambda_i\), nous écrivons la caractérisation \ref{ItemEgOQBX} du théorème \ref{ThoIJVzxr} avec \( B=A_0\) :
    \begin{equation}
        \overrightarrow{ A_0M }=\sum_{i=1}^r\lambda_i\overrightarrow{ A_0A_i }
    \end{equation}
    où la somme à droite s'étend a priori de \( 0\) à \( r\), mais comme \( \overrightarrow{ A_0A_0 }=0\), nous l'avons limitée à \( 1\). Si \( M\) s'écrit comme barycentre de deux façons différentes, nous aurions
    \begin{equation}
        \overrightarrow{ A_0M }=\sum_{i=1}^r\lambda_i\overrightarrow{ A_0A_i }=\sum_{i=1}^r\mu_i\overrightarrow{ A_0A_i }
    \end{equation}
    avec \( \sum_i\lambda_i=\sum_i\mu_i=1\). Étant donné que les points \( A_0,\ldots, A_r\) forment un repère, les vecteurs \( \overrightarrow{ A_0A_i }\) sont linéairement indépendants (point \ref{ItemFBfcuq} de la proposition \ref{PropGAneHg}) et donc \( \lambda_i=\mu_i\) pour \( i=1,\ldots, r\). La condition de somme des points égale à \( 1\) impose alors immédiatement \( \lambda_0=\mu_0\).
\end{proof}

\begin{example}
    Soient les points \( A=(3,1)\), \( B=(-1,2)\) et \( C=(0,-1)\) dans \( \eR^2\). Nous allons montrer qu'il forment un repère affine de \( \eR^2\). L'espace engendré par ces trois points est l'espace des
    \begin{equation}
        A+\alpha\overrightarrow{ AB }+\beta\overrightarrow{ AC },
    \end{equation}
    et la direction correspondante est l'espace vectoriel donné par \( \alpha\overrightarrow{ AB }+\beta\overrightarrow{ AC }\) qui est de dimension deux. Donc l'espace affine engendré par \( A\), \( B\) et \( C\) est de dimension \( 2\).
\end{example}

\begin{example}
    Dans le repère \( (A,B,C)\), quel est le point de coordonnées barycentriques \( (\frac{1}{ 6 },\frac{1}{ 3 },\frac{1}{ 2 })\) ? D'abord nous vérifions que
    \begin{equation}
        \frac{1}{ 6 }+\frac{1}{ 3 }+\frac{1}{ 2 }=1.
    \end{equation}
    Ensuite nous cherchons \( X\in \eR^2\) tel que
    \begin{equation}
        \frac{1}{ 6 }\overrightarrow{ AX }+\frac{1}{ 3 }\overrightarrow{ BX }+\frac{1}{ 2 }\overrightarrow{ CX }=0,
    \end{equation}
    c'est à dire
    \begin{equation}
        \frac{1}{ 6 }\begin{pmatrix}
            x-3    \\ 
            y-1    
        \end{pmatrix}+\frac{1}{ 3 }\begin{pmatrix}
            x+1    \\ 
            y-2    
        \end{pmatrix}+\frac{1}{ 2 }\begin{pmatrix}
            x    \\ 
            y+1    
        \end{pmatrix}=0.
    \end{equation}
    Nous trouvons immédiatement \( x=1/6\) et \( y=1/3\). Le point cherché est donc le point \( \begin{pmatrix}
        1/6    \\ 
        1/3    
    \end{pmatrix}\).
\end{example}

%---------------------------------------------------------------------------------------------------------------------------
\subsection{Équation de droite}
%---------------------------------------------------------------------------------------------------------------------------

Soit \( \affE\) un espace affine de dimension trois muni d'un repère barycentrique. Une droite est donnée par trois nombres : \( D=D(a,b,c)\) est l'ensemble des points dont les coordonnées barycentriques (normalisées) \( (x,y,z)\) vérifient \( ax+by+cz=0\). C'est un espace de dimension un parce qu'il y a aussi la condition \( x+y+z=1\).

La droite \( D(1,1,1,)\) n'existe pas parce que ce serait \( x+y+z=0\), qui est incompatible avec \( x+y+z=1\). 

Les droites \( D(a,b,c)\) et \( D(a',b',c')\) s'intersectent selon les solutions du système
\begin{subequations}
    \begin{numcases}{}
        x+y+z=1\\
        ax+by+cz=0\\
        a'x+b'y+c'z=0
    \end{numcases}
\end{subequations}
Donc deux droites affines ont un unique point d'intersection si et seulement si
\begin{equation}
    d=\begin{vmatrix}
        1    &   1    &   1    \\
        a    &   b    &   c    \\
        a'    &   b'    &   c'
    \end{vmatrix}\neq 0.
\end{equation}
Elles seront parallèles ou confondues si et seulement si \( d=0\).


