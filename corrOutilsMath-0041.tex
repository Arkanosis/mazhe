% This is part of Exercices et corrigés de CdI-1
% Copyright (c) 2011
%   Laurent Claessens
% See the file fdl-1.3.txt for copying conditions.

\begin{corrige}{OutilsMath-0041}

    \begin{enumerate}
        \item
            Commençons par $u=(0,1)$.
            \begin{equation}
                \begin{aligned}[]
                    \frac{ \partial f }{ \partial u }(x,y)&=\lim_{t\to 0} \frac{ f(x+0t,y+t)-f(x,y) }{ t }\\
                    &=\lim_{t\to 0} \frac{ x^2-(y+t)^2-(x^2-y^2) }{ t }\\
                    &=\lim_{t\to 0} \frac{ -2ty-t^2 }{ t }\\
                    &=-2y.
                \end{aligned}
            \end{equation}
        \item
           De la même manière, nous trouvons $\partial_uf(x,y)=2y$ lorsque $u=(0,1)$.
       \item
           Prenons maintenant $u=(1,1)/\sqrt{2}$. Nous avons
           \begin{equation}
               \begin{aligned}[]
                   \frac{ \partial f }{ \partial u }(x,y)&=\lim_{t\to 0} \frac{ f\big( x+\frac{ t }{ \sqrt{2} },y+\frac{ t }{ \sqrt{2} } \big)-f(x,y) }{ t }\\
                   &=\lim_{t\to 0} \frac{ \big( x^2+2\frac{ xt }{ \sqrt{2} }+\frac{ t }{2} \big)-\big( y^2+\frac{ 2yt }{ \sqrt{2} }+\frac{ t^2 }{2} \big) -x^2+y^2 }{ t }\\
                   &=\sqrt{2}(x-y).
               \end{aligned}
           \end{equation}
    \end{enumerate}
    
    \begin{remark}
        Cet exemple nous illustre deux principes plus généraux :
        \begin{enumerate}
            \item
                La dérivée directionnelle dans la direction $(1,0)$ est exactement la dérivée partielle $\partial_x$, et la dérivée directionnelle dans la direction $(0,1)$ est exactement la dérivée partielle $\partial_y$.
            \item
                La dérivée directionnelle dans une direction $(u_1,u_2)$ est la moyenne pondérée des dérivées partielles.
        \end{enumerate}
    \end{remark}

\end{corrige}
