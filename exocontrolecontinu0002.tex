\begin{exercice}\label{exocontrolecontinu0002}
  \begin{enumerate}
  \item Trouver le domaine de définition de la fonction 
    \begin{equation}
      f(x,y)=\frac{\sqrt{y-3}}{\ln (x+y)}.
    \end{equation}
    %\begin{enumerate}
    %\item $\displaystyle \frac{\ln (xy)}{x-2}$ ; 
    %\item $\displaystyle \frac{\sqrt{y-3}}{\ln (x+y)}$.
    %\end{enumerate}
  \item Tracer, si elles existent, les courbes de niveau de la fonction $\displaystyle f(x,y)= x^2+ (y-3)^2$ aux niveaux $-1$, $0$ et $1$.  
  \end{enumerate} 
\corrref{controlecontinu0002}
\end{exercice}
