% This is part of the Exercices et corrigés de CdI-2.
% Copyright (C) 2008, 2009, 2012
%   Laurent Claessens
% See the file fdl-1.3.txt for copying conditions.


\begin{corrige}{111}

% TODO : refaire le dessin
%Un petit graphe de la fonction est donné à la figure \ref{LabelFigexouuu}
%\newcommand{\CaptionFigexouuu}{Quelques unes des fonctions $f_n$.}
%\input{Fig_exouuu.pstricks}

\begin{enumerate}
\item 

La suite de fonction proposée converge (ponctuellement) vers la fonction
\begin{equation}
	f(x)=
\begin{cases}
	0	&	\text{si $x\neq 0$}\\
	1	&	 \text{si $x=0$}
\end{cases}
\end{equation}
\item En vertu du théorème \ref{ThoUnigCvCont}, si la convergence était uniforme, la fonction limite devrait être continue, ce qui n'est pas le cas. La convergence n'est donc pas uniforme sur $[-1,1]$.

\item

Soit $K$, un compact dans $]0,1]$, et appelons $a$ son minimum. Soit $\epsilon>0$, et choisissons $N$ tel que $f_n(a)<\epsilon$ dès que $n>N$. Étant donné que $f(x)=0$ sur $K$ et que toutes les fonctions en jeu sont positives, nous avons :
\begin{equation}
	\| f_n(x)-f(x) \|=f_n(x)\leq f_n(a)<\epsilon,
\end{equation}
ce qui prouve que la suite de fonction est uniformément convergente sur tout compact de $]0,1]$.

\end{enumerate}

\end{corrige}



