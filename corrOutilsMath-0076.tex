% This is part of Exercices et corrigés de CdI-1
% Copyright (c) 2011
%   Laurent Claessens
% See the file fdl-1.3.txt for copying conditions.

\begin{corrige}{OutilsMath-0076}

    Les vecteurs $N\times a$ et $N\times b$ sont dans le même sens. Il faut voir leur normes. En utilisant la formule de la proposition \ref{PropNormeProdVectoabsintOM}, la norme de $N\times a$ est $\sin(\theta_1)$ parce que $N$ et $a$ sont de norme $1$. Nous avons donc
    \begin{subequations}
        \begin{numcases}{}
           \| N\times a \|=\sin(\theta_1)\\
           \| N\times b \|=\sin(\theta_2)
        \end{numcases}
    \end{subequations}
    La loi de Snell montre alors que $\| N\times a \|n_1=\| N\times b \|n_2$.

\end{corrige}
