\begin{corrige}{CourbesSurfaces0009}

Nous cherchons le point \(\sigma(s)\) tel que la longueur d'arc entre \(\sigma(s)\) et \(A=(a,0)\) soit égale à \(s\). Si \(X\) et \(Y\) sont des points de l'ellipse, nous allons noter
\begin{equation}
    d(X,Y) 
\end{equation}
la longueur d'arc entre \(X\) et \(Y\), parcourue dans le sens trigonométrique.

La fonction qui joue un rôle fondamental dans cet exercice est la fonction
\begin{equation}
    \phi(t)=\int_0^t\| \gamma'(u) \|du.
\end{equation}
Il se fait que, dans le cas de l'ellipse, cette fonction est compliquée à calculer\footnote{Elle est même impossible à exprimer à l'aide des fonctions simples.}. Nous allons donc simplement l'appeler «\(\phi\)». Par définition et parce que \(\gamma(0)=A\),
\begin{equation}
    d(A,\gamma(t))=\phi(t).
\end{equation}
Par conséquent
\begin{equation}
    d\Big( A,\gamma\big( \phi^{-1}(s) \big) \Big)=\phi\big( \phi(s) \big)=s.
\end{equation}
La paramétrisation à choisir est donc
\begin{equation}
    \sigma_A=\gamma\circ\phi^{-1}.
\end{equation}
Cela est un résultat déjà trouvé dans le cours.

En ce qui concerne le point \(B\), nous avons \(t\in\mathopen[ 0 , 2\pi \mathclose]\) et
\begin{equation}
    \gamma_B(t)=\begin{pmatrix}
        a\cos(t+\frac{ \pi }{2})    \\ 
        b\sin(t+\frac{ \pi }{2})    
    \end{pmatrix}.
\end{equation}
Cela est la paramétrisation de l'ellipse dont le «premier» point est \(B\). Le lien entre \(\gamma_B\) et \(\gamma\) est
\begin{equation}
    \gamma_B(t)=\gamma(t+\frac{ \pi }{2}).
\end{equation}
La réponse peut donc être donné en considérant la fonction
\begin{equation}
    \phi_B(t)=\int_0^t\| \gamma_B'(u) \|du,
\end{equation}
et ensuite
\begin{equation}        \label{EqReponseSigmaB}
    \sigma_B(s)=\gamma_B\circ\phi_B^{-1}
\end{equation}

Pour le sport, nous allons exprimer cela en termes de \(\gamma\) et \(\phi\). En suivant le même raisonnement que précédemment, et en effectuant le changement de variable \(v=u+\pi/2\),
\begin{equation}
    \begin{aligned}[]
        d\big( B,\gamma_B(t) \big)&=\int_0^t\| \gamma'_B(u) \|du\\
        &=\int_0^t\| \gamma'_A\big( u+\frac{ \pi }{2} \big) \|du\\
        &=\int_{\pi/2}^{t+\pi/2}\| \gamma'(v) \|dv\\
        &=\phi\big( t+\frac{ \pi }{2} \big)-\phi\big( \frac{ \pi }{2} \big).
    \end{aligned}
\end{equation}
À partir de là, nous pouvons calculer \(\phi_B^{-1}\) en termes de l'inverse de \(\phi\) de la façon suivante. Nous avons
\begin{equation}
    t=\phi_B^{-1}(s)
\end{equation}
si et seulement si (en prenant \(\phi_B\) des deux côtés) :
\begin{equation}
    s=\phi(t+\frac{ \pi }{2})-\phi(\frac{ \pi }{2}).
\end{equation}
D'où nous déduisons successivement :
\begin{equation}
    \begin{aligned}[]
        s+\phi(\frac{ \pi }{2})=\phi(t+\frac{ \pi }{2})\\
        \phi^{-1}\left( s+\phi(\frac{ \pi }{2}) \right)=t+\frac{ \pi }{2}\\
        t=\phi^{-1}\left( s+\phi(\frac{ \pi }{2}) \right)-\frac{ \pi }{2}.
    \end{aligned}
\end{equation}
Nous avons donc prouvé que
\begin{equation}
    \phi_B^{-1}(s)=\phi^{-1}\left( s+\phi(\frac{ \pi }{2}) \right)-\frac{ \pi }{2}.
\end{equation}
Le sens de la réponse \eqref{EqReponseSigmaB} est que le point
\begin{equation}
    \gamma_B\big( \phi_B^{-1}(s) \big)
\end{equation}
est à la distance \(s\) du point \(B\). Autrement dit :
\begin{equation}
    d\Big( B, \gamma_B\big( \phi_B^{-1}(s) \big) \Big)=s.
\end{equation}
Cela peut s'exprimer maintenant en termes de \(\phi\) et de \(\gamma\) de la façon suivante :
\begin{equation}
    \begin{aligned}[]
        \gamma_B\big( \phi_B^{-1}(s) \big)&=\gamma\big( \phi_B^{-1}(s)+\frac{ \pi }{2} \big)\\
        &=\gamma\Big( \phi^{-1}\big( s+\phi(\frac{ \pi }{2}) \big) \Big)
    \end{aligned}
\end{equation}

\end{corrige}
