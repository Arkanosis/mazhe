% This is part of Mes notes de mathématique
% Copyright (c) 2011-2012
%   Laurent Claessens
% See the file fdl-1.3.txt for copying conditions.

%+++++++++++++++++++++++++++++++++++++++++++++++++++++++++++++++++++++++++++++++++++++++++++++++++++++++++++++++++++++++++++
\section{Complétude}
%+++++++++++++++++++++++++++++++++++++++++++++++++++++++++++++++++++++++++++++++++++++++++++++++++++++++++++++++++++++++++++

\begin{proposition}[Inégalité de Minkowski]     \label{PropInegMinkKUpRHg}
    Si \( 1\leq p<\infty\) et si \( f,g\in L^p(\Omega,\tribA,\mu)\) alors
    \begin{enumerate}
        \item
            \( \| f+g \|_p\leq \| f \|_p+\| g_p \|\)
        \item
            Il y a égalité si et seulement si les vecteurs \( f(x)\) et \( g(x)\) sont presque partout colinéaires : il existe \( \alpha,\beta\) tels que \( \alpha f+\beta g=0\) presque partout.
    \end{enumerate}
\end{proposition}

\begin{theorem}
    Pour \( 1\leq p<\infty\), l'espace \( L^p(\Omega,\tribA,\mu)\) est complet.
\end{theorem}

\begin{proof}
    Cette preuve provient de \cite{SuquetFourierProba}, disponible aussi sous forme d'application du \wikipedia{fr}{théorème_de_Riesz-Fischer}{théorème de Riesz-Fischer} sur wikipedia. 
    
    Soit \( (f_n)_{n\in\eN}\) une suite de Cauchy dans \( L^p\). Pour tout \( i\), il existe \( N_i\in\eN\) tel que $\| f_p-f_q \|_p\leq 2^{-i}$ pour tout \( p,q\geq N_i\). Nous considérons la sous suite \( g_i=f_{N_i}\), de telle sorte qu'en particulier
    \begin{equation}    \label{EqJLoDID}
        \|g_i-g_{i-1}\|_p\leq 2^{-i}.
    \end{equation}
    Pour chaque \( j\) nous considérons la somme télescopique
    \begin{equation}
        g_j=g_0+\sum_{i=1}^j(g_i-g_{i-1})
    \end{equation}
    et l'inégalité
    \begin{equation}
        | g_j |\leq | g_0 |+\sum_{i=1}^j| g_i-g_{i-1} |.
    \end{equation}
    Nous allons noter
    \begin{equation}        \label{EqSomPaFPQOWC}
        h_j=| g_0 |+\sum_{i=1}^j| g_i-g_{i-1} |.
    \end{equation}
    La suite de fonctions \( (h_j)\) ainsi définie est une suite croissante de fonctions positive qui converge donc (ponctuellement) vers une fonction \( h\) qui peut éventuellement valoir l'infini en certains points. Par continuité de la fonction \( x\mapsto x^p\) nous avons
    \begin{equation}
        \lim_{j\to \infty} h_j^p=h^p,
    \end{equation}
    puis par le théorème de la convergence monotone (théorème \ref{ThoConvMonFtBoVh}) nous avons
    \begin{equation}
        \lim_{j\to \infty} \int_{\Omega}h_j^pd\mu=\int_{\Omega}h^pd\mu.
    \end{equation}
    Utilisant à présent la continuité de la fonction \( x\mapsto x^{1/p}\) nous trouvons
    \begin{equation}
        \lim_{j\to \infty} \left( \int h_j^p \right)^{1/p}=\left( \int | h |^p \right)^{1/p}.
    \end{equation}
    Nous avons donc déjà montré que
    \begin{equation}
        \lim_{j\to \infty} \| h_j \|_p=\left( \int | h |^p \right)^{1/p}
    \end{equation}
    où, encore une fois, rien ne garantit à ce stade que l'intégrale à droite soit un nombre fini. En utilisant l'inégalité de Minkowski (proposition \ref{PropInegMinkKUpRHg}) et l'inégalité \eqref{EqJLoDID} nous trouvons
    \begin{equation}
        \|h_j\|_p\leq \|g_0\|_p+\sum_{i=1}^j\|g_i-g_{i-1}\|_p\leq \|g_0\|_p+1.
    \end{equation}
    En passant à la limite,
    \begin{equation}
        \left( \int| h |^p \right)^{1/p}=\lim_{j\to \infty}\|h_j\|_p \leq \|g_0\|_p+1<\infty.
    \end{equation}
    Par conséquent \( \int| h |^p\) est finie et
    \begin{equation}    \label{EqgLpdUPOBP}
        h\in L^p(\Omega,\tribA,\mu).
    \end{equation}
    En particulier, l'intégrale \( \int h\) est finie (parce que \( p\geq 1\)) et donc que \( h(x)<\infty\) pour presque tout \( x\in\Omega\).

    Nous savons que \( h(x)\) est la limite des sommes partielles \eqref{EqSomPaFPQOWC}, en particulier la série
    \begin{equation}
        \sum_{j=1}^{\infty}| g_i-g_{i-1} |
    \end{equation}
    converge ponctuellement. En vertu du corollaire \ref{CorCvAbsNormwEZdRc}, la série de terme général \( g_i-g_{i-1}\) converge ponctuellement. La suite \( g_i\) converge donc vers une fonction que nous notons \( g\). Par ailleurs la suite \( g_i\) est dominée par \( h\in L^p\), le théorème de la convergence dominée (théorème \ref{ThoConvDomLebVdhsTf}) implique que
    \begin{equation}
        \lim_{j\to \infty} \|g_j-g\|_p=0.
    \end{equation}
    Nous allons maintenant prouver que \( \lim_{n\to \infty\|f_n-g\|_p} =0\). Soit \( \epsilon>0\). Pour tout \( n\) et \( i\) nous avons
    \begin{equation}
        \|f_n-g\|_p=\|f_n-f_{N_i}+f_{N_i}-g\|_p\leq\|f_n-f_{N_i}\|_p+\|f_{N_i}-g\|_p.
    \end{equation}
    Pour rappel, \( f_{N_i}=g_i\). Si \(i\) et \( n\) sont suffisamment grands nous pouvons obtenir que chacun des deux termes est plus petit que \( \epsilon/2\).

    Il nous reste à prouver que \( g\in L^p(\Omega,\tribA,\mu)\). Nous avons déjà vu (équation \eqref{EqgLpdUPOBP}) que \( h\in L^p\), mais \( | g_i |\leq h^p\), par conséquent  \( g\in L^p\).

    Nous avons donc montré que la suite de Cauchy \( (f_n)\) converge vers une fonction de \( L^p\), ce qui signifie que \( L^p\) est complet.
\end{proof}

%+++++++++++++++++++++++++++++++++++++++++++++++++++++++++++++++++++++++++++++++++++++++++++++++++++++++++++++++++++++++++++
\section{L'espace \texorpdfstring{$L^2$}{$L^2$}}
%+++++++++++++++++++++++++++++++++++++++++++++++++++++++++++++++++++++++++++++++++++++++++++++++++++++++++++++++++++++++++++

Soit \( (\Omega,\tribA,\mu)\) un espace mesuré. Nous considérons l'opération
\begin{equation}    \label{DefProdScalLubrgTj}
    \langle f, g\rangle =\int_{\Omega}f(\omega)g(\omega)d\mu(\omega)
\end{equation}
et la norme associée
\begin{equation}
    \| f \|_2=\sqrt{\langle f, f\rangle }.
\end{equation}
Nous considérons l'ensemble
\begin{equation}
    \mL^2(\Omega,\mu)=\{ f\colon \Omega\to \eR\tq \| f \|_2<\infty \}
\end{equation}
et la relation d'équivalence \( f\sim g\) si et seulement si \( f(x)=g(x)\) pour \( \mu\)-presque tout \( x\). L'espace que nous considérons est
\begin{equation}
    L^2=\mL^2/\sim.
\end{equation}

\begin{lemma}
    La formule \eqref{DefProdScalLubrgTj} définit un produit scalaire sur \( L^2\).
\end{lemma}

\begin{proof}
    Nous devons d'abord montrer que la formule passe au quotient. Pour cela, nous considérons des fonctions \( \alpha\) et \( \beta\) nulles presque partout et nous regardons le produit de \( f_1=f+\alpha\) par \( g_1=g+\beta\) :
    \begin{equation}
        \langle f_1, g_1\rangle =\int fg+\beta f+\alpha g+ \alpha\beta.
    \end{equation}
    Les fonction \( \beta f\), \( \alpha g\) et \( \alpha\beta\) étant nulles presque partout, leur intégrale est nulle et nous avons bien \( \langle f_1, g_1\rangle =\langle f,g \rangle \). Nous pouvons donc considérer le produit sur l'ensemble des classes.

    Pour vérifier que la formule est un produit scalaire, le seul point non évidement est de prouver que \( \langle f, f\rangle =0\) implique \( f=0\). Cela découle du fait que
    \begin{equation}
        \langle f, f\rangle =\int_{\Omega}| f |^2.
    \end{equation}
    La fonction \( x\mapsto | f(x) |^2\) vérifie les hypothèses du lemme \ref{Lemfobnwt}. Par conséquent \( | f(x) |^2\) est presque partout nulle.
\end{proof}

%+++++++++++++++++++++++++++++++++++++++++++++++++++++++++++++++++++++++++++++++++++++++++++++++++++++++++++++++++++++++++++
\section{Espaces de Hilbert}
%+++++++++++++++++++++++++++++++++++++++++++++++++++++++++++++++++++++++++++++++++++++++++++++++++++++++++++++++++++++++++++

\begin{proposition}
    Si \( \hH\) est un espace de Hilbert réel, alors
    \begin{equation}
        \| x+y \|^2=\| x \|^2+\| y \|^2+2\langle x, y\rangle .
    \end{equation}
    Si \( \hH\) est un espace de Hilbert complexe alors
    \begin{equation}
        \| x+y \|^2=\| x \|^2+\| y \|^2+2\Reel\langle x, y\rangle .
    \end{equation}
    Dans les deux cas nous avons l'inégalité de Cauchy-Schwarz\index{inégalité!Cauchy-Schwarz}\index{Cauchy-Schwarz} :
    \begin{equation}
        | \langle x, y\rangle  |\leq \| x \|\| y \|.
    \end{equation}
\end{proposition}

%+++++++++++++++++++++++++++++++++++++++++++++++++++++++++++++++++++++++++++++++++++++++++++++++++++++++++++++++++++++++++++
\section{Théorème de la projection}
%+++++++++++++++++++++++++++++++++++++++++++++++++++++++++++++++++++++++++++++++++++++++++++++++++++++++++++++++++++++++++++

Sources : \cite{ProbCOndutetz} et \wikipedia{en}{Hilbert_projection_theorem}{théorème de projection} sur wikipédia.

\begin{theorem}[Projection sur partie fermée convexe]\index{théorème!projection!partie fermée convexe}
    \label{ThoProjOrthuzcYkz}
    Soit \( \hH\) un espace de Hilbert, et \( C\) un sous ensemble fermé convexe de \( \hH\). Pour chaque \( x\in \hH\), il existe un unique \( y\in C\) tel que
    \begin{enumerate}
        \item
            \( \| x-y \|=\inf\{ \| x-z \|\tq z\in C \}\),
        \item
            pour tout \( z\in C\), \( \langle x-y, z\rangle =0\).
    \end{enumerate}
\end{theorem}

\begin{proof}
    Nous nommons \( d\) l'infimum en question. 
    \begin{description}
        \item[Existence] 
    Soit \( (y_n)\) une suite dans \( C\) telle que 
    \begin{equation}
        \lim_{n\to \infty} \| x-y_n \|=\inf\{ \| x-y \|\tq z\in C \}=d.
    \end{equation}
    Nous allons montrer que cette suite peut être choisie de Cauchy. Elle convergera donc dans \( \hH\) parce que ce dernier est complet. Mais \( C\) étant supposé fermé dans \( \hH\), la limite appartiendra à \( C\). Soient \( r,s\in \eN\). D'abord nous avons
    \begin{subequations}
        \begin{align}
            \| y_r-y_r \|^2&=\langle y_r-y_s+x-x, y_r-y_s+x-x\rangle \\
            &=\| y_r-x \|^2+\| y_s-x \|^2-2\langle y_r-x, y_s-x\rangle .
        \end{align}
    \end{subequations}
    Ensuite,
    \begin{subequations}
        \begin{align}
            4\left\| \frac{ y_r+y_s }{2}-x \right\|^2&=\langle y_r+y_s-2x, y_r-y_s-2x\rangle \\
            &=\| y_r-x \|^2+\| y_s-x \|^2+2\langle y_r-x, y_s-x\rangle .
        \end{align}
    \end{subequations}
    Si nous égalisons les valeurs de \( 2\langle y_r-x, y_s-x\rangle \) nous trouvons
    \begin{equation}    \label{EqiqCyUa}
        \| y_r-y_s \|^2=-4\left\| \frac{ y_r+y_s }{2}-x \right\|^2+2\| y_r-x \|^2+2\| y_s-x \|^2.
    \end{equation}
    La distance infimum étant \( d\), nous pouvons choisir \( y_n\) de telle façon à avoir 
    \begin{equation}
        \| y_n-x \|\leq d+\frac{1}{ n }.
    \end{equation}
    D'autre part étant donné que \( C\) est convexe, \( (y_r+y_s)/2\) est dans \( C\) et nous avons
    \begin{equation}
        \left\| \frac{ y_r+y_s }{2}-x \right\| \leq d.
    \end{equation}
    En mettant ces majorations dans \eqref{EqiqCyUa} nous trouvons
    \begin{equation}
        \| y_r-y_s \|^2\leq -4d+2\left( d+\frac{1}{ r } \right)+2\left( d+\frac{1}{ s } \right)=\frac{1}{ r }+\frac{1}{ s }.
    \end{equation}
    La suite \( (y_n)\) est donc de Cauchy et la limite est un élément de \( C\). Prouvons que cet élément \( y\) réalise l'infimum. Pour cela nous avons les inégalités
    \begin{equation}
        d\leq \| x-y \|\leq\| x-y_n \|+\| y_n-y \|.
    \end{equation}
    En prenant le limite \( n\to\infty\) nous trouvons
    \begin{equation}
        d\leq \| x-y \|\leq d.
    \end{equation}
    
        \item[Unicité]
            Si \( y_1\) et \( y_2\) sont deux éléments de \( C\) qui minimisent la distance avec \( x\), nous avons
            \begin{equation}
                \| y_1-y_2 \|^2=-4\left\| \frac{ y_1+y_2j-x }{2} \right\|^2+2\| y_1-x \|^2+2\| y_2-x \|^2\leq -4d+2d+2d=0.
            \end{equation}
            Par conséquent \( y_1=y_2\).

    \end{description}
\end{proof}


\begin{theorem}[Projection orthogonale]\index{projection!orthogonale}   \index{théorème!projection!cas vectoriel}
    Soit \( K\) un sous espace vectoriel fermé et \( x\in\hH\). L'élément \( \pr_K(x)\) est l'unique élément de \( K\) tel que
    \begin{equation}
        x-y\in K^{\perp}.
    \end{equation}
    De plus l'application \( x\mapsto\pr_K(x)\) est linéaire et continue.
\end{theorem}

    L'élément \( y\) ainsi définit est la \defe{projection orthogonale}{projection!orthogonale} de \( x\) sur \( K\) et sera noté \( \pr_K(f)\).\nomenclature[Y]{\( \pr_K(x)\)}{projection orthogonale de \( x\) sur \( y\)}

\begin{proof}
            Soit \( z\) un élément de \( K\) tel que \( \langle z-x, a\rangle =0\) pour tout \( a\in K\). Nous avons
            \begin{subequations}
                \begin{align}
                    \| x-a \|^2&=\| z-x \|^2+\| a-z \|^2+2\underbrace{\langle z-x, a-z\rangle}_{=0}\\
                    &\geq \| z-x \|^2.
                \end{align}
            \end{subequations}
            Le produit scalaire est nul parce que \( a-z\in K\). La distance \( \| z-x \|\) est donc bien la plus petite distance entre \( x\) et les éléments de \( K\).

            Dans l'autre sens, nous supposons que \( y\in K\) minimise la distance à \( x\) dans \( K\). Par hypothèse pour tout \( a\) et pour tout \( \lambda\in \eR\), la différence
            \begin{equation}
                \| (y+\lambda a)-x \|^2-\| y-x \|^2
            \end{equation}
            est positive. En développant les produits scalaires nous trouvons la conditions suivante
            \begin{equation}
                \lambda^2\| a \|^2+2\lambda\langle a, y-x\rangle \geq 0
            \end{equation}
            qui doit être vraie pour tout \( \lambda\in\eR\). En tant que polynôme du second degré en \( \lambda\), cela n'aura pas deux racines réelles distinctes uniquement si \( \langle a, y-x\rangle =0\).
    
            Nous montrons maintenant la linéarité de la projection orthogonale. Soient \( x_1,x_2\in\hH\). L'élément \( y=\pr_Kx_1+\pr_Kx_2\) satisfait à la condition d'orthogonalité : pour tout \( z\in K\),
    \begin{equation}
        \langle x_1+x_2-\pr_Kx_1-\pr_Kx_2, z\rangle =\langle x_1-\pr_Kx_1, z\rangle +\langle x_2-\pr_Kx_2, z\rangle =0.
    \end{equation}
    Étant donné que \( K\) est un sous espace vectoriel, la condition de minimalité est automatiquement vérifiée (seconde partie du théorème \ref{ThoProjOrthuzcYkz}).
\end{proof}

\begin{proposition}
    Soit \( \hH=L^2(\Omega,\tribA,\mu)\), \( \tribF\) une sous tribu de \( \tribA\) et \( K\) l'ensemble de fonctions \( \tribF\)-mesurables dans \( L^2(\Omega,\tribA,\mu)\). Si \( f\in L^2(\Omega,\tribA,\mu)\) est positive, alors \( \pr_Kf\) est positive (presque partout).
\end{proposition}

\begin{proof}
    L'ensemble \( A=\{ \pr_Kf<0 \}\) est dans \( \tribF\). En effet 
    \begin{equation}
        A=(\pr_Kf)^{-1}\big( \mathopen] -\infty , 0 \mathclose[\big)
    \end{equation}
    alors que, par construction, \( \pr_Kf\) est \( \tribF\)-mesurable. La conséquent la fonction indicatrice \( \mtu_A\) est \( \tribF\)-mesurable (c'est à dire \( 1_A\in K\)) et nous avons
    \begin{equation}
        0\leq\int_{\Omega}f\mtu_A=\int_{\Omega}\pr_Kf\mtu_A\leq 0.
    \end{equation}
    Étant donné que nous avons supposé \( f\geq 0\) nous avons alors \( \mu(A)=0\). D'où le fait que \( \pr_Kf\) est presque partout positive.
\end{proof}

%+++++++++++++++++++++++++++++++++++++++++++++++++++++++++++++++++++++++++++++++++++++++++++++++++++++++++++++++++++++++++++
\section{Systèmes orthogonaux et bases}
%+++++++++++++++++++++++++++++++++++++++++++++++++++++++++++++++++++++++++++++++++++++++++++++++++++++++++++++++++++++++++++

Sources : \cite{HilbertLi}.


%+++++++++++++++++++++++++++++++++++++++++++++++++++++++++++++++++++++++++++++++++++++++++++++++++++++++++++++++++++++++++++
\section{Théorème de Kochen-Specker}
%+++++++++++++++++++++++++++++++++++++++++++++++++++++++++++++++++++++++++++++++++++++++++++++++++++++++++++++++++++++++++++

Le théorème suivant\cite{BainKochen} est \wikipedia{en}{Kochen-Specker_theorem}{central en mécanique quantique}.
\begin{theorem}[Kochen-Specker]
    Soit \( \hH\) un espace de Hilbert de dimension plus grande ou égale à trois. Soit \( \mA\) une famille d'opérateurs hermitiens sur \( \hH\). Une fonction réelle sur \( \mA\) ne peut pas satisfaire les deux propriétés suivantes en même temps
    \begin{enumerate}
        \item
            \( f(A+B)=f(A)+f(B)\)
        \item
            \( f(AB)=f(A)f(B)\)
    \end{enumerate}
    pour tout opérateurs \( A,B\in\mA\) possédants une base commune de vecteurs propres.
\end{theorem}

%TODO : la preuve. Il faudra élucider des points comme le fait que si H est un Hilbert séparable, alors il a
%       une base dénombrable.


