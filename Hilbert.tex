%+++++++++++++++++++++++++++++++++++++++++++++++++++++++++++++++++++++++++++++++++++++++++++++++++++++++++++++++++++++++++++
\section{Théorème de la projection}
%+++++++++++++++++++++++++++++++++++++++++++++++++++++++++++++++++++++++++++++++++++++++++++++++++++++++++++++++++++++++++++

Sources : \cite{ProbCOndutetz} et \wikipedia{en}{Hilbert_projection_theorem}{théorème de projection} sur wikipédia.

\begin{theorem}[Projection orthogonale]\index{projection!orthogonale}   \label{ThoProjOrthuzcYkz}
    Soit \( \hH\) un espace de Hilbert, et \( K\) un sous ensemble fermé convexe de \( \hH\). Pour chaque \( x\in \hH\), il existe un unique \( y\in K\) tel que
    \begin{enumerate}
        \item
            \( \| x-y \|=\inf\{ \| x-z \|\tq z\in K \}\),
        \item
            pour tout \( z\in K\), \( \langle x-y, z\rangle =0\).
    \end{enumerate}
    Si \( K\) est un sous espace vectoriel alors les deux conditions sont équivalentes. L'élément \( y\) ainsi définit est la \defe{projection orthogonale}{projection!orthogonale} de \( x\) sur \( K\) et sera noté \( \pr_K(f)\).\nomenclature[Y]{\( \pr_K(x)\)}{projection orthogonale de \( x\) sur \( y\)}
\end{theorem}

\begin{proof}
    Nous nommons \( d\) l'infimum en question. 
    \begin{description}
        \item[Existence] 
    Soit \( (y_n)\) une suite dans \( K\) telle que 
    \begin{equation}
        \lim_{n\to \infty} \| x-y_n \|=\inf\{ \| x-y \|\tq z\in K \}=d.
    \end{equation}
    Nous allons montrer que cette suite peut être choisie de Cauchy. Elle convergera donc dans \( \hH\) parce que ce dernier est complet. Mais \( K\) étant supposé fermé dans \( \hH\), la limite appartiendra à \( K\). Soient \( r,s\in \eN\). D'abord nous avons
    \begin{subequations}
        \begin{align}
            \| y_r-y_r \|^2&=\langle y_r-y_s+x-x, y_r-y_s+x-x\rangle \\
            &=\| y_r-x \|^2+\| y_s-x \|^2-2\langle y_r-x, y_s-x\rangle .
        \end{align}
    \end{subequations}
    Ensuite,
    \begin{subequations}
        \begin{align}
            4\left\| \frac{ y_r+y_s }{2}-x \right\|^2&=\langle y_r+y_s-2x, y_r-y_s-2x\rangle \\
            &=\| y_r-x \|^2+\| y_s-x \|^2+2\langle y_r-x, y_s-x\rangle .
        \end{align}
    \end{subequations}
    Si nous égalisons les valeurs de \( 2\langle y_r-x, y_s-x\rangle \) nous trouvons
    \begin{equation}    \label{EqiqCyUa}
        \| y_r-y_s \|^2=-4\left\| \frac{ y_r+y_s }{2}-x \right\|^2+2\| y_r-x \|^2+2\| y_s-x \|^2.
    \end{equation}
    La distance infimum étant \( d\), nous pouvons choisir \( y_n\) de telle façon à avoir 
    \begin{equation}
        \| y_n-x \|\leq d+\frac{1}{ n }.
    \end{equation}
    D'autre part étant donné que \( K\) est convexe, \( (y_r+y_s)/2\) est dans \( K\) et nous avons
    \begin{equation}
        \left\| \frac{ y_r+y_s }{2}-x \right\| \leq d.
    \end{equation}
    En mettant ces majorations dans \eqref{EqiqCyUa} nous trouvons
    \begin{equation}
        \| y_r-y_s \|^2\leq -4d+2\left( d+\frac{1}{ r } \right)+2\left( d+\frac{1}{ s } \right)=\frac{1}{ r }+\frac{1}{ s }.
    \end{equation}
    La suite \( (y_n)\) est donc de Cauchy et la limite est un élément de \( K\). Prouvons que cet élément \( y\) réalise l'infimum. Pour cela nous avons les inégalités
    \begin{equation}
        d\leq \| x-y \|\leq\| x-y_n \|+\| y_n-y \|.
    \end{equation}
    En prenant le limite \( n\to\infty\) nous trouvons
    \begin{equation}
        d\leq \| x-y \|\leq d.
    \end{equation}
    
        \item[Unicité]
            Si \( y_1\) et \( y_2\) sont deux éléments de \( K\) qui minimisent la distance avec \( x\), nous avons
            \begin{equation}
                \| y_1-y_2 \|^2=-4\left\| \frac{ y_1+y_2j-x }{2} \right\|^2+2\| y_1-x \|^2+2\| y_2-x \|^2\leq -4d+2d+2d=0.
            \end{equation}
            Par conséquent \( y_1=y_2\).

        \item[Le cas vectoriel]
            
            Soit \( z\) un élément de \( K\) tel que \( \langle z-x, a\rangle =0\) pour tout \( a\in K\). Nous avons
            \begin{subequations}
                \begin{align}
                    \| x-a \|^2&=\| z-x \|^2+\| a-z \|^2+2\underbrace{\langle z-x, a-z\rangle}_{=0}\\
                    &\geq \| z-x \|^2.
                \end{align}
            \end{subequations}
            Le produit scalaire est nul parce que \( a-z\in K\). La distance \( \| z-x \|\) est donc bien la plus petite distance entre \( x\) et les éléments de \( K\).

            Dans l'autre sens, nous supposons que \( y\in K\) minimise la distance à \( x\) dans \( K\). Par hypothèse pour tout \( a\) et pour tout \( \lambda\in \eR\), la différence
            \begin{equation}
                \| (y+\lambda a)-x \|^2-\| y-x \|^2
            \end{equation}
            est positive. En développant les produits scalaires nous trouvons la conditions suivante
            \begin{equation}
                \lambda^2\| a \|^2+2\lambda\langle a, y-x\rangle \geq 0
            \end{equation}
            qui doit être vraie pour tout \( \lambda\in\eR\). En tant que polynôme du second degré en \( \lambda\), cela n'aura pas deux racines réelles distinctes uniquement si \( \langle a, y-x\rangle =0\).
    \end{description}
\end{proof}

\begin{proposition}
    Soit \( K\) un sous espace vectoriel fermé de l'espace de Hilbert \( \hH\). La projection \( \pr_K\) est linéaire. 
\end{proposition}

\begin{proof}
    Soient \( x_1,x_2\in\hH\). L'élément \( y=\pr_Kx_1+\pr_Kx_2\) satisfait à la condition d'orthogonalité : pour tout \( z\in K\),
    \begin{equation}
        \langle x_1+x_2-\pr_Kx_1-\pr_Kx_2, z\rangle =\langle x_1-\pr_Kx_1, z\rangle +\langle x_2-\pr_Kx_2, z\rangle =0.
    \end{equation}
    Étant donné que \( K\) est un sous espace vectoriel, la condition de minimalité est automatiquement vérifiée (seconde partie du théorème \ref{ThoProjOrthuzcYkz}).
\end{proof}

\begin{proposition}
    Soit \( \hH=L^2(\Omega,\tribA,\mu)\), \( \tribF\) une sous tribu de \( \tribA\) et \( K\) l'ensemble de fonctions \( \tribF\)-mesurables dans \( L^2(\Omega,\tribA,\mu)\). Si \( f\in L^2(\Omega,\tribA,\mu)\) est positive, alors \( \pr_Kf\) est positive (presque partout).
\end{proposition}

\begin{proof}
    L'ensemble \( A=\{ \pr_Kf<0 \}\) est dans \( \tribF\). En effet 
    \begin{equation}
        A=(\pr_Kf)^{-1}\big( \mathopen] -\infty , 0 \mathclose[\big)
    \end{equation}
    alors que, par construction, \( \pr_Kf\) est \( \tribF\)-mesurable. La conséquent la fonction indicatrice \( \mtu_A\) est \( \tribF\)-mesurable (c'est à dire \( 1_A\in K\)) et nous avons
    \begin{equation}
        0\leq\int_{\Omega}f\mtu_A=\int_{\Omega}\pr_Kf\mtu_A\leq 0.
    \end{equation}
    Étant donné que nous avons supposé \( f\geq 0\) nous avons alors \( \mu(A)=0\). D'où le fait que \( \pr_Kf\) est presque partout positive.
\end{proof}

Le théorème suivant\cite{BainKochen} est \wikipedia{en}{Kochen-Specker_theorem}{central en mécanique quantique}.
\begin{theorem}[Kochen-Specker]
    Soit \( \hH\) un espace de Hilbert de dimension plus grande ou égale à trois. Soit \( \mA\) une famille d'opérateurs hermitiens sur \( \hH\). Une fonction réelle sur \( \mA\) ne peut pas satisfaire les deux propriétés suivantes en même temps
    \begin{enumerate}
        \item
            \( f(A+B)=f(A)+f(B)\)
        \item
            \( f(AB)=f(A)f(B)\)
    \end{enumerate}
    pour tout opérateurs \( A,B\in\mA\) possédants une base commune de vecteurs propres.
\end{theorem}

%TODO : la preuve. Il faudra élucider des points comme le fait que si H est un Hilbert séparable, alors il a
%       une base dénombrable.


