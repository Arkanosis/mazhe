% This is part of Mes notes de mathématique
% Copyright (c) 2011-2015
%   Laurent Claessens
% See the file fdl-1.3.txt for copying conditions.

%+++++++++++++++++++++++++++++++++++++++++++++++++++++++++++++++++++++++++++++++++++++++++++++++++++++++++++++++++++++++++++ 
\section{Intégrales convergeant uniformément}
%+++++++++++++++++++++++++++++++++++++++++++++++++++++++++++++++++++++++++++++++++++++++++++++++++++++++++++++++++++++++++++

%--------------------------------------------------------------------------------------------------------------------------- 
\subsection{Définition et propriété}
%---------------------------------------------------------------------------------------------------------------------------

Soit \( (\Omega,\mu)\) un espace mesuré. Nous disons que l'intégrale
\begin{equation}
    \int_{\Omega}f(x,\omega)d\mu(\omega)
\end{equation}
\defe{converge uniformément}{convergence!uniforme!intégrale} en \( x\) si pour tout \( \epsilon>0\), il existe un compact \( K_0\) tel que pour tout compact \( K\) tel que \( K_0\subset K\) nous avons
\begin{equation}
    \left| \int_{\Omega\setminus K}f(x,\omega)d\mu(\omega) \right| \leq \epsilon.
\end{equation}
Le point important est que le choix de \( K_0\) ne dépend pas de \( x\).

\begin{lemma}       \label{LemOgQdpJ}
    Soit
    \begin{equation}
        F(x)=\int_{\Omega}f(x,\omega)d\mu(\omega),
    \end{equation}
    une intégrale uniformément convergente. Pour chaque \( k\in \eN\) nous considérons un compact \( K_k\) tel que
    \begin{equation}
        \left| \int_{\Omega\setminus K_k}f(x,\omega)d\mu(\omega) \right| \leq\frac{1}{ k }.
    \end{equation}
    Alors la suite de fonctions \( F_k\) définie par
    \begin{equation}
        F_k(x)=\int_{K_k}f(x,\omega)d\mu(\omega)
    \end{equation}
    converge uniformément vers \( F\).
\end{lemma}

\begin{proof}
    Nous avons
    \begin{subequations}
        \begin{align}
            \big| F_k(x)-F(x) \big|&=\left| \int_{K_k}f(x,\omega)d\mu(\omega)-\int_{\Omega}f(x,\omega)d\mu(\omega) \right| \\
            &=| \int_{\Omega\setminus K_k}f(x,\omega)d\mu(\omega) |\\
            &\leq \frac{1}{ k }.
        \end{align}
    \end{subequations}
\end{proof}

%------------------------------------------------------------------------------------------------------------------------
\subsection{Critères de convergence uniforme}
%---------------------------------------------------------------------------------------------------------------------------

Afin de tester l'uniforme convergence d'une intégrale, nous avons le \defe{critère de Weierstrass}{critère!Weierstrass}:
\begin{theorem}		\label{ThoCritWeiIntUnifCv}
Soit $f(x,t)\colon [\alpha,\beta]\times[a,\infty[ \to \eR$, une fonction dont la restriction à toute demi-droite $x=cst$ est mesurable. Si $| f(x,t) |< \varphi(t)$ et $\int_a^{\infty}\varphi(t)dt$ existe, alors l'intégrale
\begin{equation}
	\int_0^{\infty}f(x,t)dt
\end{equation}
est uniformément convergente.
\end{theorem}

Le théorème suivant est le \defe{critère d'Abel}{critère!Abel pour intégrales} :
\begin{theorem}		\label{ThoAbelIntUnif}
	Supposons que $f(x,t)=\varphi(x,t)\psi(x,t)$ où $\varphi$ et $\psi$ sont bornée et intégrables en $t$ au sens de Riemann sur tout compact $[a,b]$, $b\geq a$. Supposons que :
	\begin{enumerate}
		\item $| \int_a^{T}\varphi(x,t)dt |\leq M$ où $M$ est indépendant de $T$ et de $x$,
		\item $\psi(x,t)\geq 0$,
		\item pour tout $x\in[\alpha,\beta]$, $\psi(x,t)$ est une fonction décroissante de $t$,
		\item les fonctions $x\mapsto \psi(x,t)$ convergent uniformément vers $0$ lorsque $t\to\infty$.
	\end{enumerate}
	Alors l'intégrale
	\begin{equation}
		\int_a^{\infty}f(x,t)dt
	\end{equation}
	est uniformément convergente.
\end{theorem}

\begin{remark}
    Étant donné que la fonction sinus est bornée, il est tentant de l'utiliser comme $\varphi$ dans le critère d'Abel (théorème \ref{ThoAbelIntUnif}). Hélas,
    \begin{equation}
        \int_0^T\sin(xt)=-\frac{ 1 }{ x }\big( \cos(xT)-\cos(x) \big),
    \end{equation}
    qui n'est pas bornée pour tout $x$ ! Poser $\varphi(x,t)=\sin(xt)$ \emph{ne fonctionne pas} pour assurer la convergence uniforme sur un intervalle qui contient des $x$ arbitrairement proches de $0$. Le critère d'Abel avec $\varphi(x,t)=\sin(xt)$ ne permet que de conclure à l'uniforme convergence \emph{sur tout compact} ne contenant pas $0$. Cela est toutefois souvent suffisant pour étudier la continuité ou la dérivabilité en se servant du fameux coup du compact.
\end{remark}


%+++++++++++++++++++++++++++++++++++++++++++++++++++++++++++++++++++++++++++++++++++++++++++++++++++++++++++++++++++++++++++
\section{Fonctions définies par une intégrale}
%+++++++++++++++++++++++++++++++++++++++++++++++++++++++++++++++++++++++++++++++++++++++++++++++++++++++++++++++++++++++++++
\label{SecCHwnBDj}
\index{suite!de fonctions intégrables}
\index{fonction!définie par une intégrale}
\index{permuter!limite et intégrale}

Soit \( (\Omega,\mu)\) un espace mesuré. Nous nous demandons dans quel cas l'intégrale
\begin{equation}
    F(x)=\int_{\Omega}f(x,\omega)d\omega
\end{equation}
définit une fonction \( F\) continue, dérivable ou autre. 

Dans la suite nous allons considérer des fonctions \( f\) à valeurs réelles. Quitte à passer aux composantes, nous pouvons considérer des fonctions à valeurs vectorielles. Par contre le fait que \( x\) soit dans \( \eR\) ou dans \( \eR^n\) n'est pas spécialement une chose facile à traiter.

%--------------------------------------------------------------------------------------------------------------------------- 
\subsection{Continuité sous l'intégrale}
%---------------------------------------------------------------------------------------------------------------------------
\index{continuité!fonction définie par une intégrale}

Nous allons présenter deux théorèmes donnant la continuité de \( F\).
\begin{enumerate}
    \item
        Si \( f\) est majorée par une fonction ne dépendant pas de \( x\), nous avons le théorème \ref{ThoKnuSNd},
    \item
        si l'intégrale est uniformément convergente, nous avons le théorème \ref{ThotexmgE}.
\end{enumerate}

\begin{theorem} \label{ThoKnuSNd}
    Soit \( (\Omega,\mu)\) est un espace mesuré, soit \( x_0\in \eR^m\) et \( f\colon U\times \Omega\to \eR\) où \( U\) est ouvert dans \( \eR^m\). Nous supposons que
    \begin{enumerate}
        \item
            La fonction \( f(x,.)\) est dans \( L^1(\Omega,\mu)\) pour tout \( x \in \eR^m\).
        \item
            La fonction \( f(.,\omega)\) est continue en \( x_0\) pour tout \( \omega\in\Omega\).
            %TODO : peut-être qu'on peut dire seulement pour presque tout omege dans Omega, voir la proposition \ref{prop:fdefint}.
        \item       \label{ItemNAuYNG}
            Il existe une fonction \( G\in L^1(\Omega)\) telle que
            \begin{equation}
                | f(x,\omega) |\leq G(\omega)
            \end{equation}
            pour tout \( x\in U\).
    \end{enumerate}
    Alors la fonction 
    \begin{equation}
        \begin{aligned}
            F\colon U&\to \eR \\
            x&\mapsto \int_{\Omega}f(x,\omega)d\mu(\omega) 
        \end{aligned}
    \end{equation}
    est continue en \( x_0\).
\end{theorem}
\index{permuter!limite et intégrale!espace mesuré}

\begin{proof}
    Soit \( (x_n)\) une suite convergente vers \( x_0\). Nous considérons la suite de fonctions \( f_n\colon \Omega\to \eR\) définies par
    \begin{equation}
        f_n(\phi)=f(x_n,\omega).
    \end{equation}
    sur qui nous pouvons utiliser le théorème de la convergence dominée (théorème \ref{ThoConvDomLebVdhsTf}) pour obtenir
    \begin{subequations}
        \begin{align}
            \lim_{n\to \infty} F(x_n)&=\lim_{n\to \infty} \int_{\Omega}f(x_n,\omega)d\mu(\omega)\\
            &=\int_{\Omega}\lim_{n\to \infty} f(x_n,\omega)d\mu(\omega)\\
            &=\int_{\Omega}f(x,\omega)d\mu(\omega)\\
            &=F(x).
        \end{align}
    \end{subequations}
    Nous avons utilisé la continuité de \( f(.,\omega)\).
\end{proof}


Si nous avons un peu de compatibilité entre la topologie et la mesure, alors nous pouvons utiliser l'uniforme convergence d'une intégrale pour obtenir la continuité d'une fonction définie par une intégrale.

\begin{theorem} \label{ThotexmgE}
    Soit \( (\Omega,\mu)\) un espace topologique mesuré tel que tout compact est de mesure finie. Soit une fonction \( f\colon \eR\times \Omega\to \eR\) telle que
    \begin{enumerate}
        \item
            Pour chaque \( x\in \eR\), la fonction \( f(x,.)\) est \( L^1(\Omega,\mu)\).
        \item
            Pour chaque \( \omega\in \Omega\), la fonction \( f(.,\omega)\) est continue en \( x_0\).
        \item
            L'intégrale
            \begin{equation}
                F(x)=\int_{\Omega}f(x,\omega)d\mu(\omega)
            \end{equation}
            est uniformément convergente.
    \end{enumerate}
    Alors la fonction \( F\) est continue en \( x_0\).
\end{theorem}
\index{permuter!limite et intégrale!espace mesuré}

\begin{proof}
    Nous reprenons les notations du lemme \ref{LemOgQdpJ}. Les fonctions
    \begin{equation}
        F_k(x)=\int_{K_k}f(x,\omega)d\mu(\omega)
    \end{equation}
    existent parce que les fonctions \( f(x,.)\) sont dans \( L^1(\Omega)\). Montrons que les fonctions \( F_k\) sont continues. Soit une suite \( x_k\to x_0\) nous avons
    \begin{equation}
        \lim_{n\to \infty} F_k(x_n)=\lim_{n\to \infty} \int_{K_k}f(x_n,\omega)d\mu(\omega).
    \end{equation}
    Nous pouvons inverser la limite et l'intégrale en utilisant le théorème de la convergence dominée. Pour cela, la fonction \( f(x_n,\omega)\) étant continue sur le compact \( K_k\), elle y est majorée par une constante. Le fait que les compacts soient de mesure finie (hypothèse) implique que les constantes soient intégrales sur \( K_k\). Le théorème de la convergence dominée implique alors que
    \begin{equation}
        \lim_{n\to \infty} F_k(x_n)=\int_{K_k}\lim_{n\to \infty} f(x_n,\omega)d\mu(\omega)=\int_{K_k}f(x_0,\omega)d\mu(\omega)=F_k(x_0).
    \end{equation}
    Nous avons utilisé le fait que \( f(.,\omega)\) était continue en \( x_0\).

    Le lemme \ref{LemOgQdpJ} nous indique alors que la convergence \( F_k\to F\) est uniforme. Les fonctions \( F_k\) étant continues, la fonction \( F\) est continue.
\end{proof}

Pour finir, citons ce résultat concernant les fonctions réelles.
\begin{theorem}		\label{ThoInDerrtCvUnifFContinue}
    Nous considérons \( F(x)=\int_a^{\infty}f(x,t)dt\). Si \( f\) est continue sur $[\alpha,\beta]\times[a,\alpha[$ et l'intégrale converge uniformément, alors $F(x)$ est continue.
\end{theorem}

%--------------------------------------------------------------------------------------------------------------------------- 
\subsection{Le coup du compact}
%---------------------------------------------------------------------------------------------------------------------------

Nous avons vu des fonctions définies par toute une série de processus de limite (suites, séries, intégrales). Une des questions centrales est de savoir si la fonction limite est continue, dérivable, intégrale, etc. étant donné que les fonctions sont continues.

Pour cela, nous inventons le concept de \emph{convergence uniforme}. Si la limite (série, intégrale) est uniforme, alors la fonction limite sera continue. Il arrive qu'une limite ne soit pas uniforme sur un intervalle ouvert $]0,1]$, et que nous voulions quand même prouver la continuité sur cet intervalle. C'est à cela que sert la notion de convergence uniforme \emph{sur tout compact}. En effet, la notion de continuité est une notion locale : savoir ce qu'il se passe dans un petit voisinage autour de $x$ est suffisant pour savoir la continuité en $x$ (idem pour sa dérivée).

Si nous avons uniforme convergence sur tout compact de $]0,1]$, mais pas uniforme convergence sur cet intervalle, la limite sera quand même continue sur $\mathopen] 0 , 1 \mathclose]$. En effet, si $x\in]0,1]$, il existe un ouvert autour de $x$ contenu dans un compact contenu dans $]0,1]$. L'uniforme convergence sur ce compact suffit à prouver la continuité en $x$.

Déduire la continuité sur un ouvert à partir de l'uniforme convergence sur tout compact de l'ouvert est appelé faire le \defe{coup du compact}{compact!le coup du}.


%---------------------------------------------------------------------------------------------------------------------------
\subsection{Dérivabilité sous l'intégrale}
%---------------------------------------------------------------------------------------------------------------------------
\index{dérivabilité!fonction définie par une intégrale}

Nous traitons à présent de la dérivabilité de la fonction \( F\) définie comme intégrale de \( f\).

\begin{theorem}[Dérivation sous le signe intégral\cite{MesIntProbb}]    \label{ThoMWpRKYp}
    Soit \( (\Omega,\mu)\) un espace mesuré et une fonction \( f\colon \eR\times \Omega\to \eR\) dont nous voulons étudier la dérivabilité en \(a\in \eR\). Nous supposons qu'il existe \( \delta>0\), \( A\) mesurable de mesure nulle dans \( \Omega\) tels que
    \begin{enumerate}
        \item
            \( f(x,\cdot)\) soit dans \( L^1(\Omega)\).
        \item
            L'application \( x\mapsto f(x,\omega)\) est dérivable pour tout \( x\in B(a,\delta)\) et pour tout \( \omega\in \complement A\).
        \item
            Il existe une fonction \( G\) intégrable sur \( \Omega\) telle que
            \begin{equation}
                \left| \frac{ \partial f }{ \partial x }(x,\omega) \right| \leq G(\omega)
            \end{equation}
            pour tout \( x\in B(a,\delta)\) et pour tout \( \omega\in\complement A\).
    \end{enumerate}
    Alors la fonction
    \begin{equation}
        F(x)=\int_{\Omega}f(x,\omega)d\mu(\omega)
    \end{equation}
    est dérivable en \( a\) et nous pouvons permuter la dérivée et l'intégrale :
    \begin{equation}
        F'(a)=\int_{\Omega}\frac{ \partial f }{ \partial x }(a,\omega)d\mu(\omega).
    \end{equation}
\end{theorem}
\index{permuter!dérivée et intégrale!dans \( \eR\)}

\begin{proof}
    Soit une suite \( (x_n)\) dans \( B(a,\delta)\) telle que \( x_n\neq a\) et \( x_n\to a\). Si la limite
    \begin{equation}
        \lim_{n\to \infty} \frac{ F(a)-F(x_n) }{ a-x_n }
    \end{equation}
    existe et ne dépend pas de la suite choisie, alors la fonction \( F\) est dérivable en \( a\) et sa dérivée vaut cette limite. Par linéarité de l'intégrale, nous devons étudier la limite
    \begin{equation}    \label{EqLIiralx}
        \lim_{n\to \infty} \int_{\Omega}\frac{ f(a,\omega)-f(x_n,\omega) }{ a-x_n }d\omega,
    \end{equation}
    montrer qu'elle existe, ne dépend pas de la suite choisie et vaut \( \int_{\Omega}\partial_xf(a,\omega)d\omega\). Nous sommes donc dans un problème d'inversion de limite et de dérivée pour lequel nous allons utiliser le théorème de la convergence dominée de Lebesgue. D'abord nous posons
    \begin{equation}    \label{EqAFOUbQB}
        g_n(\omega)=\frac{ f(x_n,\omega)-f(a,\omega) }{ x_n-a }.
    \end{equation}
    Cela est une suite de fonctions dans \( L^1(\Omega)\) parce qu'à la fois \( a\) et \( x_n\) sont dans \( B(a,\delta)\). De plus nous avons
    \begin{equation}
        \lim_{n\to \infty} g_n(\omega)=\frac{ \partial f }{ \partial x }(a,\omega)
    \end{equation}
    parce que nous savons que \( f\) est dérivable en \( a\) pour tout \( \omega\in\complement A\). En ce qui concerne la majoration de \( g_n\), nous utilisons le théorème des accroissements finis (théorème \ref{ThoAccFinis}) sur le numérateur de \eqref{EqAFOUbQB}. Pour tout \( n\) et pour tout \( \omega\in \complement A\), il existe un \( \theta_{n,\omega}\) dans \( \mathopen] a , x_n \mathclose[\) tel que
        \begin{equation}
            f(x_n,\omega)-f(a,\omega)=\frac{ \partial f }{ \partial x }(\theta_{n,\omega},\omega)(x_n-a),
        \end{equation}
        donc
        \begin{equation}
            | g_n(\omega) |=\left| \frac{ \partial f }{ \partial x }(\theta_{n,\omega},\omega) \right| \leq G(\omega).
        \end{equation}
        La dernière inégalité provient des hypothèses. Le théorème de la convergence dominée de Lebesgue (théorème \ref{ThoConvDomLebVdhsTf}) nous permet alors de calculer la limite \eqref{EqLIiralx} :
        \begin{equation}
            \lim_{n\to \infty} \int_{\Omega}g_n(\omega)d\omega=\int_{\Omega}\lim_{n\to \infty} g_n(\omega)d\omega=\int_{\Omega}\frac{ \partial f }{ \partial x }(a,\omega)d\omega.
        \end{equation}
        Notons que l'existence de la dernière intégrale fait partie du théorème de la convergence dominée.

        Nous avons donc prouvé que la limite de gauche existait et ne dépendant pas de la suite choisie. Donc \( F\) est dérivable en \( a\) et la dérivée vaut cette limite :
        \begin{equation}
            F'(a)=\int_{\Omega}\frac{ \partial f }{ \partial x }(a,\omega)d\mu(\omega).
        \end{equation}
\end{proof}

\begin{theorem}
		Supposons $f$ continue et sa dérivée partielle $\frac{ \partial f }{ \partial x }$ continue sur $[\alpha,\beta]\times[a,\alpha[$. Supposons que $F(x)=\int_a^{\infty}f(x,t)dt$ converge et que $\int_a^{\infty}\frac{ \partial f }{ \partial x }dt$ converge uniformément. Alors $F$ est $C^1$ sur $[\alpha,\beta]$ et 
		\begin{equation}
			\frac{ dF }{ dx }=\int_a^{\infty}\frac{ \partial f }{ \partial x }dt.
		\end{equation}
\end{theorem}

En ce qui concerne les fonctions dans \( \eR^n\), il y a les  propositions \ref{PropDerrSSIntegraleDSD} et \ref{PropAOZkDsh} qui parlent de différentiabilité sous l'intégrale.

%--------------------------------------------------------------------------------------------------------------------------- 
\subsection{Absolue continuité}
%---------------------------------------------------------------------------------------------------------------------------

\begin{definition}      \label{DefAbsoluCont}
    Une fonction \( F\colon \eR\to \eR\) est \defe{absolument continue}{absolument continue} sur \( \mathopen[ a , b \mathclose]\) si il existe une fonction \( f\) sur \( \mathopen[ a , b \mathclose]\) telle que
    \begin{equation}
        F(x)=\int_a^xf(t)dt
    \end{equation}
    pour tout \( x\in\mathopen[ a , b \mathclose]\).
\end{definition}

\begin{theorem}     \label{ThoDerSousIntegrale}
    Soit \( A\) un ouvert de \( \eR\) et \( \Omega\), un espace mesuré. Soit une fonction \( f\colon A\times \Omega\to \eR\) et
    \begin{equation}
        F(x)=\int_{\Omega}f(x,\omega)d\omega.
    \end{equation}
    Nous supposons les points suivants.
    \begin{enumerate}
        \item
            La fonction \( f\) est mesurable en tant que fonction \( A\times\Omega\to \eR\). Pour chaque \( x\in A\), la fonction \( f(x,\cdot)\) est intégrable sur \( \Omega\).
        \item
            Pour presque tout \( \omega\in\Omega\), la fonction \( f(x,\omega)\) est une fonction absolument continue de \( x\).
        \item
            La fonction \( \frac{ \partial f }{ \partial x }\) est localement intégrable, c'est à dire que pour tout \( \mathopen[ a , b \mathclose]\subset A\),
            \begin{equation}
                \int_a^b\int_{\Omega}\left| \frac{ \partial f }{ \partial x }(x,\omega) \right| d\omega\,dx<\infty.
            \end{equation}
    \end{enumerate}
    Alors la fonction \( F\) est absolument continue et pour presque tout \( x\in A\), la dérivée est donné par
    \begin{equation}
        \frac{ d }{ dx }\int_{\Omega}f(x,\omega)d\omega=\int_{\Omega}\frac{ \partial f }{ \partial x }(x,\omega)d\omega.
    \end{equation}
\end{theorem}

La proposition suivante sera utilisée entre autres pour montrer que sous l'hypothèse d'une densité continue, la loi exponentielle est sans mémoire, proposition \ref{PropREXaIBg}.
\begin{proposition}		\label{PropDerrFnAvecBornesFonctions}
Soit $f(x,t)$ une fonction continue sur $[\alpha,\beta]\times[a,b]$, telle que $\frac{ \partial f }{ \partial x }$ existe et soit continue sur $]\alpha,\beta[\times[a,b]$. Soient $\varphi(x)$ et $\psi(x)$, des fonctions continues de $[\alpha,\beta]$ dans $\eR$ et admettant une dérivée continue sur $]\alpha,\beta [$. Alors la fonction
\begin{equation}
	F(x)=\int_{\varphi(x)}^{\psi(x)}f(x,t)dt
\end{equation}
admet une dérivée continue sur $]\alpha,\beta[$ et
\begin{equation}	\label{EqFormDerrFnAvecBorneNInt}
	\frac{ dF }{ dx }=\int_{\varphi(x)}^{\psi(x)}\frac{ \partial f }{ \partial x }(x,t)dt+f\big( x,\psi(x) \big)\cdot\frac{ d\psi }{ dx }- f\big( x,\varphi(x) \big)\cdot\frac{ d\varphi }{ dx }.
\end{equation}
\end{proposition}
\index{permuter!dérivée et intégrale!dans \( \eR\) avec les bornes}
%TODO : une preuve de ce théorème ? allons allons ...

L'exemple qui suit devrait pouvoir être rendu rigoureux en utilisant des distributions correctement.

\begin{example} \label{ExfYXeQg}
    Si \( g\) est une fonction continue, la fonction suivante est une primitive de \( g\) :
    \begin{equation}
        \int_0^xf(t)dt=\int_0^{\infty}f(t)\mtu_{t<x}(t)dt.
    \end{equation}
    Nous nous proposons de justifier \emph{de façon un peu heuristique} le fait que ce soit bien une primitive de \( g\) en considérant la fonction
    \begin{equation}
        f(t,x)=g(t)\mtu_{t<x}(t).
    \end{equation}
    Nous posons
    \begin{equation}
        F(x)=\int_0^{\infty}f(x,t)dt,
    \end{equation}
    et nous calculons \( F'\) en permutant la dérivée et l'intégrale\footnote{Ceci n'est pas rigoureux : il faudrait avoir un théorème à propos de distributions qui permet de le faire.}. D'abord,
    \begin{equation}
        f(t,x)=\begin{cases}
            g(t)    &   \text{si \( t\in \mathopen[ 0 , x \mathclose]\)}\\
            0    &    \text{sinon.}
        \end{cases}
    \end{equation}
    La dérivée de \( f\) par rapport à \( x\) est donnée par la distribution
    \begin{equation}
        \frac{ \partial f }{ \partial x }(t_0,x_0)=g(t_0)\delta(t_0-x_0).
    \end{equation}
    Donc
    \begin{equation}
        F'(x_0)=\int_0^{\infty}\frac{ \partial f }{ \partial x }(t,x_0)dt=\int_0^{\infty}g(t)\delta(t-x_0)=g(x_0),
    \end{equation}
    comme attendu.
\end{example}

Cet exemple est rendu rigoureux par la proposition suivante.
\begin{proposition} \label{PropJLnPpaw}
    Si \( f\in L^1(\eR)\), alors la fonction
    \begin{equation}
        F(x)=\int_{-\infty}^xf(t)dt
    \end{equation}
    est presque partout dérivable et pour les points où elle l'est nous avons \( F'(x)=f(x)\).
\end{proposition}
\index{fonction!définie par une intégrale}
%TODO : une preuve.

%--------------------------------------------------------------------------------------------------------------------------- 
\subsection{Différentiabilité sous l'intégrale}
%---------------------------------------------------------------------------------------------------------------------------

Le théorème suivant est restrictif sur l'ensemble d'intégration (qui doit être compact), mais accepte des fonctions de plusieurs variables, ce qui est un premier pas vers la différentiabilité.
\begin{proposition}[Dérivation sous l'intégrale]		\label{PropDerrSSIntegraleDSD}
    Supposons $A\subset\eR^m$ ouvert et $B\subset\eR^n$ compact. Nous considérons une fonction \( f\colon A\times B\to \eR\). Si pour un $i\in\{ i,\ldots,n \}$, la dérivée partielle $\frac{ \partial f }{ \partial x_i }$ existe dans $A\times B$ et est continue, alors la fonction
    \begin{equation}
        F(x)=\int_Bf(x,t)dt
    \end{equation}
    admet une dérivée partielle dans la direction \( x_i\) sur \( A\). Cette dérivée partielle y est continue et
    \begin{equation}
        \frac{ \partial F }{ \partial x_i }(a)=\int_B\frac{ \partial f }{ \partial x_i }(a,t)dt,
    \end{equation}
    pour tout \( a\) dans l'ouvert \( A\).
\end{proposition}
\index{fonction!définie par une intégrale}
\index{permuter!dérivée et intégrale!\( \eR^n\)}

\begin{proof}
    Nous procédons en plusieurs étapes.
    \begin{subproof}
    \item[\( F\) est dérivable]
            
        Nous voulons prouver que \( \frac{ \partial F }{ \partial x_i }(a,t)\) existe. Pour cela nous posons
        \begin{equation}
            g_l(t)=\frac{ f(a_1,\ldots, a_i+\epsilon_l,\ldots, a_n,t)-f(a_1,\ldots, a_k,\ldots, a_n,t) }{ \epsilon_l }
        \end{equation}
        où \( \epsilon_l\) est une suite de nombres tendant vers zéro. La fonction \( f\) est dérivable dans la direction \( x_i\) si et seulement si \( \lim_{l\to \infty}g_l(t) \) existe et ne dépend pas du choix de la suite. À ce moment, la valeur de la dérivée partielle sera cette limite. Dans notre cas, nous savons que \( f\) admet une dérivée partielle dans la direction \( x_i\) et donc nous avons
        \begin{equation}
            \frac{ \partial f }{ \partial x_i }(a,t)=\lim_{l\to \infty} g_l(t).
        \end{equation}
        
        De la même façon pour \( F\) nous avons
        \begin{equation}
            \frac{ \partial F }{ \partial x_i }=\lim_{l\to \infty} \int_{B}g_l(t)dt.
        \end{equation}
        Sous-entendu : si la limite de droite ne dépend pas de la suite choisie, alors \( \frac{ \partial F }{ \partial x_i }\) existe et vaut cette limite.

        Vu la continuité de \( f\), le seul point à vérifier pour le théorème de la convergence dominée de Lebesgue est l'existence d'une fonction intégrable de \( t\) majorant \( g_l\). Pour cela le théorème de accroissements finis (théorème \ref{ThoAccFinis}) appliqué à la fonction \( \epsilon\mapsto f(a_n,\ldots, a_i+\epsilon,\ldots, a_n)\) nous dit que
        \begin{equation}
            f(a_1,\ldots, a_i+\epsilon_l,\ldots, a_n,t)-f(a_1,\ldots, a_i,\ldots, a_n,t)=\epsilon_l\frac{ \partial f }{ \partial x_i }(a_1,\ldots, \theta,\ldots, a_n,t)
        \end{equation}
        pour un certain \( \theta\in B(a_i,\epsilon_l)\). Notons que ce \( \theta\) dépend de \( t\) mais pas de \( l\). Vu que \( \partial_if\) est continue par rapport à ses deux variables, si \( K\) est un voisinage compact autour de \( a\), il existe \( M>0\) tel que
        \begin{equation}    \label{EqMXqviPC}
            \left| \frac{ \partial f }{ \partial x_i }(x,t) \right| < M
        \end{equation}
        pour tout \( x\in K\) et tout \( t\in B\). La valeur de \( \frac{ \partial f }{ \partial x_i }(a_1,\ldots, \theta,\ldots, a_n,t)\) est donc bien majorée par rapport à \( \theta\) et par rapport à \( t\) en même temps par une constante qui n'a pas de mal à être intégrée sur le compact \( B\).
        
        Le théorème de la convergence dominée (théorème \ref{ThoConvDomLebVdhsTf}) s'applique donc bien et nous avons
        \begin{equation}
            \lim_{l\to \infty} \int_Bg_l(t)dt=\int_B\lim_{l\to \infty} g_l(t)=\int_B\frac{ \partial f }{ \partial x_i }(a,t)dt.
        \end{equation}
        Le membre de droite ne dépendant pas de la suite \( \epsilon_l\) choisie, le membre de gauche est bien la dérivée de \( F\) par rapport à \( x_i\) et nous avons
        \begin{equation}
            \frac{ \partial F }{ \partial x_i }(a)=\int_B\frac{ \partial f }{ \partial x_i }(a,t)dt.
        \end{equation}
        Cela prouve la première partie de la proposition.

    \item[La dérivée est continue]

        Soit \( K\) un voisinage compact autour de \( a\) et \( U'\) un ouvert tel que \( a\in U'\subset K\). Nous avons encore la majoration \eqref{EqMXqviPC} sur \( U'\) et donc le théorème de continuité sous l'intégrale \ref{ThoKnuSNd} nous indique que la fonction
        \begin{equation}
            \begin{aligned}
                U'&\to \eR \\
                x&\mapsto \int_{B}\frac{ \partial f }{ \partial x_i }(x,t)dt 
            \end{aligned}
        \end{equation}
        est continue en \( a\).
        
    \end{subproof}
\end{proof}

Une conséquence de la proposition \ref{PropDerrSSIntegraleDSD} est que si elle fonctionne pour tous les \( i\), alors \( F\) est différentiable et même de classe \( C^1\), et la différentielle de \( F\) s'obtient comme intégrale de la différentielle de \( f\).

\begin{proposition}\label{PropAOZkDsh}
    Supposons $A\subset\eR^m$ ouvert et $B\subset\eR^n$ compact. Si pour tout $i\in\{ i,\ldots,n \}$, la dérivée partielle $\frac{ \partial f }{ \partial x_i }$ existe dans $A\times B$ et est continue, alors \( F\) est de classe \( C^1\) et
    \begin{equation}
        (dF)_a=\int_B(df_t)_adt
    \end{equation}
    où \( f_t(x)=f(x,t)\).
\end{proposition}
\index{permuter!différentielle et intégrale!\( \eR^n\)}

\begin{proof}
    En vertu de la proposition \ref{PropDerrSSIntegraleDSD}, toutes les dérivées partielles de \( F\) sont continues. Cela implique que \( F\) est de classe \( C^1\) par la proposition \ref{PropDerContCun} et que la différentielle s'écrive en terme des dérivées partielles avec la formule usuelle. Nous avons alors
    \begin{subequations}
        \begin{align}
            (dF)_a(u)&=\sum_k\frac{ \partial F }{ \partial x_k }(a)u_k\\
            &=\int_B\sum_k\frac{ \partial f }{ \partial x_k }(a,t)dt\\
            &=\int_B\sum_k\frac{ \partial f_t }{ \partial x_k }(a)u_kdt\\
            &=\int_B (df_t)_a(u)dt.
        \end{align}
    \end{subequations}
    Cela est la formule annoncée.
\end{proof}

Un autre théorème tourne autour du pot, et me semble inutile.
\begin{theorem} \label{ThoOLAQyRL}
    Soit \( (\Omega,\mu)\) un espace mesuré, une fonction \( f\colon \eR^n\times \Omega\to \eR\) et \( a\in \eR^n\). Nous considérons la fonction
    \begin{equation}
        F(x)=\int_{\Omega}f(x,\omega)d\mu(\omega).
    \end{equation}
    Pour chaque \( k=1,\ldots, n\) nous supposons avoir
    \begin{equation}
        \frac{ \partial F }{ \partial x_k }(a)=F_{|_k}'(a)=\int_{\Omega}\frac{ \partial f_{|_k} }{ \partial t }(a_k,\omega)d\mu(\omega)
    \end{equation}
    où \( F_{|_k}(t)=F(a_1,\ldots, t,\ldots, a_n)\) et \( f_{|_k}\) est définie de façon similaire.

    Nous supposons de plus que les fonctions \( \partial_{x_k}F\) sont continues.

    Alors \( F\) est de classe \( C^1\) et sa différentielle est donnée par
    \begin{equation}
        df_a=\int_{\Omega}(df_{\omega})_ad\omega
    \end{equation}
    où \( f_{\omega}\) est définie par \( f_{\omega}(x)=f(x,\omega)\).
\end{theorem}

\begin{proof}
    Étant donné que les dérivées partielles de \( F\) en \( a\) existent et sont continues, la proposition \ref{PropDerContCun} dit que \( F\) est différentiable et que
    \begin{equation}
        dF_a(u)=\sum_{k=1}^n\frac{ \partial F }{ \partial x_k }(a)u_k.
    \end{equation}
    La linéarité de l'intégrale et les hypothèses nous donnent alors
    \begin{subequations}
        \begin{align}
            df_a(u)&=\sum_{k=1}^n\frac{ \partial F }{ \partial x_k }(a)u_k\\
            &=\int_{\Omega}\sum_k\frac{ \partial f_{|_k} }{ \partial t }(a_k;\omega)u_kd\mu(\omega)\\
            &=\int_{\Omega}\sum_k\frac{ \partial f }{ \partial x_k }(a;\omega)u_kd\mu(\omega)\\
            &=\int_{\Omega}(df_{\omega})_a(u)d\mu(\omega),
        \end{align}
    \end{subequations}
    et donc \( df_a=\int_{\Omega}(df_{\omega})_ad\mu(\omega)\).
\end{proof}
Notons qu'en passant aux composantes, ce théorème fonctionne tout aussi bien pour des fonctions à valeurs dans un espace vectoriel normé de dimension finie plutôt que dans \( \eR\).

\begin{lemma}[Hadamard\cite{MVIooKLsjpa}]   \label{LemWNBooGPlIwT}
    Soit une fonction \( f\colon \eR^n\to \eR\) de classe \( C^p\) avec \( p\geq 1\). Pour tout \( a\in \eR^n\) il existe des fonctions \( g_1\),\ldots, \( g_n\) de classe \( C^{p-1}\) telles que
    \begin{equation}
        f(x)=f(a)+\sum_{i=1}^n(x_i-a_i)g_i(x).
    \end{equation}
\end{lemma}
\index{lemme!Hadamard}

\begin{proof}
    Vu que \( f\) est de classe \( C^1\), le théorème fondamental de l'analyse \ref{ThoRWXooTqHGbC} fonctionne et
    \begin{equation}    \label{EqZLTooVKmGln}
        f(x)-f(a)=\int_0^1\frac{ d }{ dt }\Big[ f\big( a+t(x-a) \big) \Big]dt=\int_0^1\sum_{i=1}^n\frac{ \partial f }{ \partial x_i }\big( a+t(x-a) \big)(x_i-a_i).
    \end{equation}
    Plus de détails : la fonction \( t\mapsto \frac{ d }{ dt }\Big[ f\big( a+t(x-a) \big) \Big]\) possède comme primitive la fonction \( F(t)=f\big( a+t(x-a) \big)\).

    Nous posons 
    \begin{equation}
        g_i(x)=\int_0^1\frac{ \partial f }{ \partial x_i }\big( a+t(x-a) \big)dt
    \end{equation}
    Le fait que l'intégrale existe est simplement le fait qu'il s'agit d'une fonction continue sur un compact et donc majorée par une constante. Pour voir que \( g_i\) est de classe \( C^{p-1}\) nous pouvons calculer \( \frac{ \partial g_i }{ \partial x_k }\) en permutant dérivée et intégrale par la proposition \ref{PropDerrSSIntegraleDSD} :
    \begin{equation}
        \frac{ \partial g_i }{ \partial x_k }(x)=\int_0^1\frac{ \partial  }{ \partial x_k }\left( \frac{ \partial f }{ \partial x_i }\big( a+t(x-a) \big) \right)dt=\int_0^1 t\frac{ \partial^2f }{ \partial x_k\partial x_i }\big( a+t(x-a) \big).
    \end{equation}
    Nous pouvons ainsi permuter \( p-1\) dérivées tout en gardant une fonction continue dans l'intégrale. Le théorème \ref{ThoKnuSNd} nous donne alors une fonction continue. Ainsi toutes les fonctions
    \begin{equation}
        \frac{ \partial^{p-1}g_i }{ \partial x_{i_1}\ldots\partial x_{i_{p-1}} }
    \end{equation}
    sont continues et \( g_i\) est de classe \( C^{p-1}\) par la proposition \ref{PropDYKooHvrfGw}.

    En repartant de \eqref{EqZLTooVKmGln} nous avons alors bien ce qui était annoncé :
    \begin{equation}
        f(x)=f(a)+\sum_{i=1}^ng_i(x)(x_i-a_i).
    \end{equation}
\end{proof}

\begin{corollary}
    Soit \( \phi\in\swD(\eR)\) tel que \( \phi^{(k)}(0)=0\) pour tout \( k\). Alors il existe une fonction \( \psi\in\swD(\eR)\) telle que 
    \begin{equation}
        \phi(x)=x^{n+1}\psi(x)
    \end{equation}
    pour tout \( x\in \eR\).
\end{corollary}

\begin{proof}
    En utilisant le lemme de Hadamard \ref{LemWNBooGPlIwT} avec \( a=0\), \( n=1\) et \( f(0)=0\), nous avons \( \phi(x)=xg_1(x)\) et comme \( g(0)=0\) nous avons \( g_1(x)=xg_2(x)\) et par conséquence \( \phi(x)=x^2g_2(x)\).

    En continuant ainsi autant de fois que l'on peut dériver \( \phi\), nous obtenons le résultat.
\end{proof}

%+++++++++++++++++++++++++++++++++++++++++++++++++++++++++++++++++++++++++++++++++++++++++++++++++++++++++++++++++++++++++++ 
\section{Formes différentielles exactes et fermées}
%+++++++++++++++++++++++++++++++++++++++++++++++++++++++++++++++++++++++++++++++++++++++++++++++++++++++++++++++++++++++++++

Nous avons déjà parlé de formes différentielles et de leurs intégrales sur un chemin dans la section \ref{SecFormDiffRappel}.

\begin{definition}  \label{DefEFKQmPs}
La forme différentielle $\omega$ est \defe{exacte}{forme!différentielle!exacte} si il existe une fonction $f$ telle que $\omega=df$; elle est dite \defe{fermée}{forme!différentielle!fermée} si $d\omega=0$.
\end{definition}

Dire que la forme différentielle $\omega=fdx+gdy$ est fermée, c'est dire que
\begin{equation}
    \frac{ \partial g }{ \partial x }=\frac{ \partial f }{ \partial y }.
\end{equation}

Il est naturel de se demander si toutes les formes différentielles sont des différentielles de fonctions. Une réponse complète est délicate à établir, mais a d'innombrables conséquences en physique, notamment en ce qui concerne l'existence d'un potentiel vecteur pour le champ magnétique dans les équations de Maxwell.

Le fait qu'une forme exacte soit fermée est relativement facile à établir; c'est la proposition suivante. La question plus délicate est la réciproque : sous quelles conditions une forme fermée est-elle exacte ?
\begin{proposition}
	Si $\omega$ est une $1$-forme exacte de classe $C^1$, alors $\omega$ est fermée.
\end{proposition}

\begin{proof}
	Le fait que $\omega$ soit exacte implique l'existence d'une fonction $f$ telle que $\omega=df$, c'est à dire
	\begin{equation}
		\omega_x=\sum_i a_i(x)dx_i=\sum_i\frac{ \partial f }{ \partial x_i }(x)dx_i,
	\end{equation}
	c'est à dire que $a_i(x)=\frac{ \partial f }{ \partial x_i }(x)$. L'hypothèse que $\omega$ est $C^1$ implique que $f$ est $C^2$, et donc que nous pouvons inverser l'ordre de dérivation pour les dérivées secondes $\partial^2_{ij}f=\partial^2_{ji}f$. Nous pouvons donc faire le calcul suivant :
	\begin{equation}
		\frac{ \partial a_i }{ \partial x_j }=\frac{ \partial  }{ \partial x_j }\frac{ \partial f }{ \partial x_i }=\frac{ \partial  }{ \partial x_i }\frac{ \partial f }{ \partial x_j }=\frac{ \partial a_j }{ \partial x_i },
	\end{equation}
	ce qu'il fallait démontrer.
\end{proof}

La réciproque est vraie sur un ouvert simplement connexe.
\begin{theorem}        \label{ThoFermeExactFormRappel}
Supposons que $D\subset\eR^n$ soit un ouvert simplement connexe. Alors toute forme différentielle de degré $1$ et de classe $C^1$ sur $D$ qui est fermée est exacte.
\end{theorem}

Nous allons prouver ce théorème dans un cas un peu moins général : celui d'un domaine étoilé de \( \eR^2\) plutôt que simplement connexe de \( \eR^n\).

\begin{theorem} \label{ThoMSofFxL}
Soit $D\subset\eR^2$, une ouvert étoilé, et $\omega$, une $1$-forme fermée de classe $C^1$. Alors $\omega$ est exacte.
\end{theorem}
\begin{proof}

Soit $D\subset\eR^2$, un ouvert étoilé par rapport à l'origine. Soient $f\colon D\to \eR$, $g\colon D\to \eR$, des fonctions de classe $C^1$ telles que
\begin{equation}
	\frac{ \partial f }{ \partial y }=\frac{ \partial g }{ \partial x }
\end{equation}
sur $D$, et
\begin{equation}		\label{EqIMDefFformI33}
	F(x,y)=\int_0^1\big[  f(tx,ty)x+g(tx,ty)y  \big]dt
\end{equation}
pour tout $(x,y)\in D$. 

Étant donné que nous ne définissons $F(x,y)$ que pour des $(x,y)\in D$, la fonction $t\mapsto f(tx,ty)$ est $C^1$ sur tout le compact $[0,1]$ et aucune divergence de l'intégrale n'est à craindre. Nous sommes donc dans le cadre de la proposition \ref{PropDerrSSIntegraleDSD}, et nous pouvons dériver sous le signe intégral.

Nous calculons, en utilisant la règle de dérivation de fonctions composées
\begin{equation}		\label{EqIMI33dsdsFlolo}
	\begin{aligned}[]
		\frac{ \partial F }{ \partial x }(x,t)	&=\int_0^1\left[   f\frac{ \partial f }{ \partial x }(tx,ty)x+f(tx,ty)+t\frac{ \partial g }{ \partial x }(tx,ty)y  \right]dt\\
		&=\int_0^1\left[ t\Big( x\frac{ \partial f }{ \partial x }(tx,ty)+y\frac{ \partial f }{ \partial y }(tx,ty) \Big)+f(tx,ty) \right]dt
	\end{aligned}
\end{equation}
où nous avons utilisé l'hypothèse $\partial_yf=\partial_xg$. Ce qui se trouve dans la parenthèse n'est autre que $\partial_t\big( f(tx,ty) \big)$, plus précisément, si nous posons $\mF(x,y,t)=f(tx,ty)$, nous avons
\begin{equation}
	\frac{ \partial \mF }{ \partial t }(x,y,t)= x\frac{ \partial f }{ \partial x }(tx,ty)+y\frac{ \partial f }{ \partial y }(tx,ty).
\end{equation}
En recopiant le résultat \eqref{EqIMI33dsdsFlolo} en termes de $\mF$, nous avons
\begin{equation}
	\begin{aligned}[]
		\frac{ \partial F }{ \partial x }(x,t)	&=\int_0^1\left( t\frac{ \partial \mF }{ \partial t }(x,y,t)+\mF(x,y,t) \right)dt\\
		&=\int_0^1\partial_t\big( t\mF(x,y,t) \big)dt\\
		&=\big[ f\mF(x,y,t) \big]_0^1\\
		&=\mF(x,y,1)\\
		&=f(x,y).
	\end{aligned}
\end{equation}
Le résultat correspondant pour $\frac{ \partial F }{ \partial y }(x,y)=g(x,y)$ s'obtient de la même manière. Nous avons donc obtenu que
\begin{equation}		\label{EqIMFormI33Fffdd}
	\begin{aligned}[]
		\frac{ \partial F }{ \partial x }&=f,  &\text{et}&& \frac{ \partial F }{ \partial y }=g.
	\end{aligned}
\end{equation}
En ayant prouvé cela, nous avons prouvé que si $\omega=fdx+gdy$ avec $\partial_yf=\partial_xg$, alors $\omega=dF$ où $F$ est définie par \eqref{EqIMDefFformI33}.
\end{proof}

\begin{proof}[Démonstration alternative du théorème \ref{ThoMSofFxL}]
Nous posons $u=tx$ et $v=ty$, ainsi que $\mF(x,y,t)=f(u,v)$ et $\mG(x,y,t)=g(u,v)$. Avec cette notation, nous avons $F(x,y)=\int_0^1\big( x\mF(x,y,t)+y\mG(x,y,t) \big)dt$, et
\begin{equation}
	\begin{aligned}[]
		\frac{ \partial \mF }{ \partial x }&=\frac{ \partial f }{ \partial u }\frac{ \partial u }{ \partial x }+\frac{ \partial f }{ \partial v }\frac{ \partial v }{ \partial x }=t\frac{ \partial f }{ \partial u },\\
		\frac{ \partial \mG }{ \partial x }&=t\frac{ \partial g }{ \partial u }.
	\end{aligned}
\end{equation}
Ainsi,
\begin{equation}
	\begin{aligned}[]
		\frac{ \partial F }{ \partial x }	&=\int_0^1\left( x\frac{ \partial \mF }{ \partial x }+\mF+y\frac{ \partial G }{ \partial x } \right)dt\\
							&=\int_0^1\left( xt\frac{ \partial f }{ \partial u } +\mF+yt\frac{ \partial g }{ \partial u } \right)dt\\
							&=\int_0^1\left[  t\left( x\frac{ \partial f }{ \partial u }+y\frac{ \partial f }{ \partial v } \right)+\mF  \right]dt.
	\end{aligned}
\end{equation}
où nous avons utilisé le fait que, par hypothèse, $\frac{ \partial g }{ \partial u }=\frac{ \partial f }{ \partial v }$. Nous calculons par ailleurs que
\begin{equation}
	\frac{ \partial F }{ \partial t }=\frac{ \partial f }{ \partial u }\frac{ \partial u }{ \partial t }+\frac{ \partial f }{ \partial v }\frac{ \partial v }{ \partial t }=x\frac{ \partial f }{ \partial u }+y\frac{ \partial f }{ \partial v }.
\end{equation}
Donc, nous avons
\begin{equation}
	\frac{ \partial F }{ \partial x }=\int_0^1\left( t\frac{ \partial \mF }{ \partial t }+\mF \right)dt=\int_0^1\frac{ \partial  }{ \partial t }(t\mF)dt.
\end{equation}
Par conséquent,
\begin{equation}
	\frac{ \partial F }{ \partial x }=[t\mF]_0^1=\mF(x,y,1)=f(x,y).
\end{equation}
Le même genre de calculs fournit $\frac{ \partial F }{ \partial y }=g(x,y)$.
\end{proof}

%+++++++++++++++++++++++++++++++++++++++++++++++++++++++++++++++++++++++++++++++++++++++++++++++++++++++++++++++++++++++++++ 
\section{Théorème d'Abel angulaire}
%+++++++++++++++++++++++++++++++++++++++++++++++++++++++++++++++++++++++++++++++++++++++++++++++++++++++++++++++++++++++++++

\begin{theorem}[Abel angulaire\cite{KXjFWKA}]   \label{ThoTGjmeen}
    Soit \( \sum_{n}a_nz^n\) une série entière de rayon de convergence plus grand ou égal à \( 1\) et de somme \( f\). Soit \( \theta_0\in\mathopen[ 0 , \frac{ \pi }{2} \mathclose[\). Nous posons
    \begin{equation}
        \Delta_{\theta_0}=\{ z=1-\rho e^{i\varphi}\tq \rho>0,\varphi\in\mathopen[ \theta_0 , \theta_0 \mathclose],| z |<1 \}.
    \end{equation}
    Nous supposons de plus que \( \sum_na_n\) converge. Alors
    \begin{equation}
        \lim_{\substack{z\to1\\z\in\Delta_0}}f(z)=\sum_{k=0}^{\infty}a_k.
    \end{equation}
\end{theorem}
\index{Abel!angulaire}
\index{convergence!suite numérique!Abel angulaire}
\index{somme partielles!Abel angulaire}
\index{série!entière!Abel angulaire}

\begin{proof}

    Le résultat de ce théorème est que l'on peut calculer la limite \( z\to 1\) avec des chemins contenus dans un domaine de la forme de celui dessiné à la figure \ref{LabelFigJGuKEjH}. % From file JGuKEjH
    \newcommand{\CaptionFigJGuKEjH}{La zone dans laquelle peut être le chemin qui va vers \( z=1\).}
    \input{Fig_JGuKEjH.pstricks}

    De façon très classique nous posons
    \begin{equation}
        \begin{aligned}[]
            S&=\sum_{k=0}^{\infty}a_k&S_n&=\sum_{k=0}^na_k,
        \end{aligned}
    \end{equation}
    et \( R_n=S-S_n\). En particulier \( a_n=R_{n-1}-R_n\). 

    Le but du théorème est de montrer que \( \sum a_nz^n\) converge vers \( S\) lorsque \( z\) converge vers \( 1\) à l'intérieur de \( \Delta_{\theta_0}\). Pour cela nous calculons pour un \( N\) donné la différence \( \sum_{n=0}^{N}a_nz^n-S_N\) en triant les termes par ordre de \( R_n\), en isolant le terme \( R_0\) et le terme \( R_N\) :
    \begin{subequations}
        \begin{align}
            \sum_{n=0}^Na_nz^n-S_N&=\sum_{n=1}^Na_n(z^n-1)\\
            &=\sum_{n=1}^N(R_{n-1}-R_n)(z^n-1)\\
            &=R_0(z-1)+\sum_{n=1}^{N-1}R_n(z^{n+1}-1-z^n+1)+R_N(z^N-1)\\
            &=R_0(z-1)+\sum_{n=1}^{N-1}R_nz^n(z-1)+R_N(z^N-1)\\
            &=(z-1)\sum_{n=0}^{N-1}R_nz^n+R_N(z^N-1).
        \end{align}
    \end{subequations}
    Cela est valable pour tout \( N\) et \( | z |<1\). Nous avons donc
    \begin{equation}
        \sum_{n=0}^Na_nz^n-S_N=(z-1)\sum_{n=0}^{N-1}R_nz^n+R_N(z^N-1).
    \end{equation}
    Par hypothèse nous avons \( \lim_{N\to \infty} R_N=0\). Et de plus le membre de gauche converge parce que chacun des deux termes converge séparément. En passant à la limite nous avons pour tout \( | z |<1\) :
    \begin{equation}
        f(z)-S=(z-1)\sum_{n=0}^{\infty}R_nz^n.
    \end{equation}
    Nous voudrions étudier le comportement de la différence \( f(z)-S\) lorsque \( z\) tend vers \( 1\). Pour cela nous nous fixons \( \epsilon>0\) et \( N\geq 1\) tel que \( | R_n |<\epsilon\) dès que \( n\geq N\). Alors pour tout \( | z |<1\) nous avons
    \begin{subequations}
        \begin{align}
            | f(z)-S |&\leq | z-1 |\left( \sum_{n=0}^N| R_n | \underbrace{|z^n |}_{\leq 1} +\sum_{n=N+1}^{\infty}\underbrace{| R_n |}_{\leq \epsilon} |z^n | \right)\\
            &\leq | z-1 |\sum_{n=0}^N| R_n |+\epsilon\frac{ | z-1 | }{ 1-| z | }
        \end{align}
    \end{subequations}
    où nous avons utilisé la somme de la série géométrique \eqref{EqASYTiCK} et l'égalité \( | z^n |=| z |^n\). Avant de nous particulariser à \( z\in\Delta_{\theta_0}\) nous devons anticiper un problème au dénominateur en multipliant par le binôme conjugué :
    \begin{equation}
        \frac{ | z-1 | }{ 1-| z | }=\frac{ | z-1 |(1+| z |) }{ 1-| z |^2 }.
    \end{equation}
    C'est maintenant que nous nous particularisons à \( z\in\Delta_{\theta_0}\) en posant \( z=\rho e^{i\varphi}\) et en remarquant que \( | z |^2=1-2\rho\cos(\varphi)+\rho^2\). Nous avons le calcul suivant :
    \begin{subequations}
        \begin{align}
            \frac{ | z-1 | }{ 1-| z | }&=\frac{ \rho(1+| z |) }{ 2\rho\cos(\varphi)-\rho^2 }\\
            &=\frac{ 1+| z | }{ 2\cos(\varphi)-\rho}\\
            &\leq\frac{ 2 }{ 2\cos(\varphi)-\rho }\\
            &\leq\frac{ 2 }{ 2\cos(\varphi)-\cos(\theta_0) }\\
            &\leq\frac{ 2 }{ 2\cos(\theta_0)-\cos(\theta_0) }\\
            &=\frac{ 2 }{ \cos(\theta_0) }.
        \end{align}
    \end{subequations}
    Quelque justifications.
    \begin{itemize}
        \item Vu que nous avons dans l'idée de faire \( \rho\to 0\) nous supposons que \( \rho<\cos(\theta_0)\).
        \item Nous avons \( \cos(\varphi)>\cos(\theta_0)\) parce que \( z\) est dans \( \Delta_{\theta_0}\).
    \end{itemize}
    Nous avons donc, pour tout \( z\in\Delta_{\theta_0}\) que
    \begin{equation}
        | f(z)-S |\leq | z-1 |\sum_{n=0}^N| R_n |+\epsilon\frac{ 2 }{ \cos(\theta_0) }.
    \end{equation}
    Il suffit de prendre \( \rho\) assez petit pour que 
    \begin{equation}
        | z-1 |\sum_{n=0}^N| R_n |<\epsilon
    \end{equation}
    et nous avons
    \begin{equation}
        | f(z)-S |\leq \epsilon\left( 1+\frac{ 2 }{ \cos(\theta_0) } \right).
    \end{equation}
    Nous avons donc bien \( \lim_{\substack{z\to 1\\z\in\Delta_0}}f(z)=S\), comme nous le voulions.
\end{proof}

La réciproque du théorème d'Abel angulaire est que si \( f(z)=\sum_na_nz^n\) sur \( B(0,1)\) se prolonge par continuité en \( z=1\) alors cette prolongation se fait par \( f(1)=\sum_na_n\). Cela est faux comme le montre l'exemple suivant.

\begin{example}
    Nous considérons la série entière \( \sum_{n=0}^{\infty}(-1)^nz^n\) qui converge\footnote{C'est la série géométrique de raison \( -z\).} vers
    \begin{equation}
        f(z)=\frac{1}{ 1+z }
    \end{equation}
    sur \( B(0,1)\). De plus nous avons
    \begin{equation}
        \lim_{\substack{z\to 1\\    | z |<1}}\frac{1}{ 1+z }=\frac{ 1 }{2}.
    \end{equation}
    Donc la fonction converge bien vers quelque chose lorsque \( z\) tend vers \( 1\). La fonction \( f\) se prolonge par continuité en \( 1\). Pourtant la série es coefficients \( \sum_n(-1)^n\) ne converge pas.
\end{example}

Le théorème suivant donne une espèce d'inverse au théorème d'Abel angulaire. En effet il dit que si la série converge  en allant vers \( 1\) le long de l'axe réel, alors ça converge vers la somme des coefficients. Il faut cependant une hypothèse en plus sur les \( a_n\).
\begin{theorem}[Théorème taubérien faible\cite{KXjFWKA}]
    Soit \( \sum_na_nz^n\) une série entière de rayon de convergence \( 1\) et de somme \( f\). Nous supposons
    \begin{enumerate}
        \item
            Il existe \( S\in \eC\) tel que \( \lim_{\substack{x\to 1\\x\in\mathopen] -1 , 1 \mathclose[}}f(x)=S\).
            \item
                \( \lim_{n\to \infty} na_n=0\).
    \end{enumerate}
    Alors la série \( \sum_{n=0}^{\infty}a_n\) converge et vaut \( S\).
\end{theorem}
\index{théorème!taubérien faible}

\begin{proof}
    Nous notons \( S_n=\sum_{k=0}a_k\) et \( M=\sup_{k\geq 1}k| a_k |\), qui est fini par hypothèse. Pour \( x\in \mathopen] 0 , 1 \mathclose[\) et \( n\geq 0\) nous avons
    \begin{equation}
        S_n-f(x)=\sum_{k=1}^na_k-\sum_{k=1}^na_kx^k-\sum_{k=n+1}^{\infty}a_kx^k=\sum_{k=1}^na_k(1-x^k)-\sum_{k=n+1}^{\infty}a_kx^k.
    \end{equation}
    Nous utilisons la série géométrique sous la forme \( 1-x^k=(1-x)\sum_{i=0}^nx^i\) pour écrire
    \begin{subequations}
        \begin{align}
            S_n-f(x)&=\sum_{k=1}^na_k(1-x)\underbrace{\sum_{i=0}^{k-1}x^i}_{\leq k}-\sum_{k=n+1}^{\infty}a_kx^k\\
            &\leq\sum_{k=1}^nka_k(1-x)-\sum_{k=n+1}^{\infty}a_kx^k,
        \end{align}
    \end{subequations}
    donc en passant à la norme
    \begin{subequations}
        \begin{align}
            \big| S_n-f(x) \big|&\leq (1-x)Mn+\sum_{k=n+1}| a_k |x^k\\
            &\leq (1-x)Mn+\sum_{k=n+1}^{\infty}\underbrace{\frac{ k }{ n }| a_k |}_{\leq M/n}x^k\\
            &\leq (1-x)Mn+\frac{ M }{ n }\sum_{k=n+1}^{\infty}x^k\\
            &\leq (1-x)Mn+\frac{ M }{ n }\frac{1}{ 1-x }.
        \end{align}
    \end{subequations}
    Ce que nous cherchons à étudier est le comportement \( x\to 1\) et montrer que \( S_n\to S\), ce qui nous incite à calculer \( | S_n-f(1-\frac{ \epsilon }{n  }) |\) avec \( 0<\epsilon<1\) :
    \begin{equation}
        \big| S_n-f\big( 1-\frac{ \epsilon }{ n } \big) \big|\leq \epsilon M+\epsilon.
    \end{equation}
    Nous choisissons \( N_1\) tel que \( \frac{ M }{ n }\leq \epsilon^2\) dès que \( n\geq N_1\). En sus nous savons que 
    \begin{equation}
        \lim_{\epsilon\to 0}f(1-\epsilon)=S.
    \end{equation}
    Nous choisissons \( N_2\) de telle sorte à avoir
    \begin{equation}
        \left| f\left( 1-\frac{ \epsilon }{ n } \right)-S \right| <\epsilon,
    \end{equation}
    et en prenant \( n\geq\max(N_1,N_2)\) nous avons
    \begin{equation}
        | S_n-S |\leq \left| S_n-f\left( 1-\frac{ \epsilon }{ n } \right) \right| +\left| f\left( 1-\frac{ \epsilon }{ n } \right)-S \right|  \leq \epsilon M+2\epsilon.
    \end{equation}
    Il suffit de choisit \( \epsilon\) suffisamment petit (en particulier pour que \( \epsilon M\) soit petit) pour montrer que \( | S_n-S |\) est borné par un nombre arbitrairement petit.
\end{proof}
%++++++++++++++++++++++++++++++++++++++++++++++++++++++++++++++++++++++++   
\section{Coordonnées polaires, cylindriques et sphériques}\label{sec_coord}
%++++++++++++++++++++++++++++++++++++++++++++++++++++++++++++++++++++++++   
\subsection{Coordonnées polaires}
Soit $T$ la fonction de $]0, +\infty[\times \eR$ dans $\eR^2\setminus\{(0,0)\}$ définie par
\begin{equation}
  \begin{array}{lccc}
    T: &]0, +\infty[\times \eR & \to & \eR^2\setminus\{(0,0)\}\\
 & (r, \theta)&\mapsto& (r\cos \theta, r \sin \theta),
  \end{array}
\end{equation}
Cette fonction est surjective. Elle est bijective sur chaque bande de la forme  $]0, +\infty[\times [a-\pi,a+\pi[$. Si $a=0$ l'inverse de $T$  est la fonction $T^{-1}(x,y)= (\sqrt{x^2+y^2}, \arctg (y/x))$. Soit $P=(x,y)$ un élément dans $\eR^2$, on dit que $r=\sqrt{x^2+y^2}$ est le rayon de $P$ et que $\theta=\arctg (y/x) $ est son argument principal. L'origine ne peut pas être décrite en coordonnées polaires parce que si son rayon est manifestement zéro, on ne peut pas lui associer une valeur univoque de l'angle $\theta$. 

\begin{example}
L'équation du cercle de rayon $a$ et centre $(0, 0)$ en coordonnées polaires est $r=a$. 
\end{example}

\begin{example}
	Une équation possible pour la demi-droite $x=y$, $x>0$,  est $\theta=\pi/4$.         
\end{example}

%++++++++++++++++++++++++++++++++++++++++++++++++++++++++++++++++++++++++   
\subsection{Coordonnées cylindriques}
%++++++++++++++++++++++++++++++++++++++++++++++++++++++++++++++++++++++++
Soit $T$ la fonction de $]0, +\infty[\times \eR^2$ dans $\eR^3\setminus\{(0,0,0)\}$ définie par
\begin{equation}
  \begin{array}{lccc}
    T: &]0, +\infty[\times \eR\times \eR & \to & \eR^3\setminus\{(0,0,0)\}\\
 & (r, \theta, z)&\mapsto& (r\cos \theta, r \sin \theta, z),
  \end{array}
\end{equation}
Cette fonction est surjective. Elle est bijective sur chaque bande de la forme  $]0, +\infty[\times [a-\pi,a+\pi[\times \eR$, $a$ dans $\eR$. Il n'y a presque rien de nouveau par rapport aux coordonnées polaires. Les coordonnées  cylindriques sont intéressantes si on décrit un objet invariant par rapport aux rotations autour de l'axe des $z$. 

\begin{example}
Il faut savoir ce que décrivent les équations les plus simples en coordonnées cylindriques, 
\begin{itemize}
\item $r\leq a$, pour $a$ constant dans  $]0, +\infty[$, est le cylindre de hauteur infinie qui a pour axe l'axe des $z$ et pour base le disque de rayon $a$ centré à l'origine, 
\item $r= a$ est  la surface du cylindre,
\item $\theta = b$ est un demi-plan ouvert et sa fermeture contient l'axe des $z$,
\item $z=c$ est un plan parallèle au plan $x$-$y$. 
\end{itemize}
\end{example}

\begin{example}
  Un demi-cône qui a  son sommet en l'origine et  pour axe l'axe des $z$ est décrit par $z=d r$.  Si $d$ est positif  il s'agit  de la moitié supérieure du cône, si $d<0$ de la moitié inférieure.
\end{example}

\begin{example}
 De même,  la sphère de rayon $a$ et centrée à l'origine est l'assemblage des calottes $z=\sqrt{a^2-r^2}$ et $z=-\sqrt{a^2-r^2}$. 
\end{example}
%++++++++++++++++++++++++++++++++++++++++++++++++++++++++++++++++++++++++   
\subsection{Coordonnées sphériques}
%++++++++++++++++++++++++++++++++++++++++++++++++++++++++++++++++++++++++
Soit $T$ la fonction de $]0, +\infty[\times \eR^2$ dans $\eR^3\setminus\{(0,0,0)\}$ définie par
\begin{equation}
  \begin{array}{lccc}
    T: &]0, +\infty[\times \eR\times \eR & \to & \eR^3\setminus\{(0,0,0)\}\\
 & (\rho, \theta, \phi)&\mapsto& (\rho\cos \theta\sin \phi, \rho \sin \theta\sin \phi, \rho\cos \phi),
  \end{array}
\end{equation}
Cette fonction est surjective. Elle est bijective sur chaque bande de la forme  $]0, +\infty[\times [a-\pi,a+\pi[\times [b-\pi/2, b+\pi/2[$, $a$ et $b$ dans $\eR$.  Si $a =0$ et $b=-\pi/2$ la fonction inverse $T^{-1}$ est donnée donnée
\begin{equation}
  \begin{array}{lccc}
    T: &\eR^3\setminus\{(0,0,0)\} & \to & ]0, +\infty[\times [-\pi,\pi[\times [0, \pi[\\
 & (x,y,z)&\mapsto& \left(\sqrt{x^2+y^2+z^2}, \arctg \frac{y}{x}, \arccos \left(\frac{z}{\sqrt{x^2+y^2+z^2}}\right)\right). 
  \end{array}
\end{equation}
Soit $ P$ un point dans $\eR^3$. L'angle $\phi$ est l'angle entre le demi-axe positif des $z$ et le vecteur $\overrightarrow{OP}$, $\rho$ est la norme de $\overrightarrow{OP}$ et $\theta$ est l'argument en coordonnées polaires de la projection de $\overrightarrow{OP}$ sur le plan $x$-$y$.  

\begin{remark}
	Dans la littérature, les angles $\theta$ et $\phi$ sont parfois inversés (voire, changent de nom, par exemple $\varphi$ au lieu de $\phi$). Il faut donc être très prudent lorsqu'on veut utiliser dans un cours des formules données dans un autre cours.
\end{remark}

\begin{example}
Il faut connaître le sens des équations plus simples, 
\begin{itemize}
\item $\rho\leq a$, pour $a$ constant dans  $]0, +\infty[$, est la boule fermée de rayon $a$ centrée à l'origine, 
\item $\rho= a$ est  la sphère de rayon $a$ centrée à l'origine,
\item $\theta = b$ est un demi-plan ouvert et sa fermeture contient l'axe des $z$,
\item $\phi= c$ est un demi-cône qui a  son sommet à l'origine et  pour axe l'axe des $z$.  Si $c$ est positif  il s'agit  de la moitié supérieure du cône, si $d<0$ de la moitié inférieure. 
\end{itemize}
 \end{example}

%++++++++++++++++++++++++++++++++++++++++++++++++++++++++++++++++++++++++++++++++++++++++++++++++++
\section{Changement de variables}
%++++++++++++++++++++++++++++++++++++++++++++++++++++++++++++++++++++++++++++++++++++++++++++++++++
\begin{theorem}		\label{ThoChmVarInt}
  Soient $U$ et $V$ deux ouverts bornés de $\eR^p$, $\phi$ un difféomorphisme de classe $\mathcal{C}^1$ de $U$ sur $V$ et $f$ une fonction intégrable de $V$ sur $\eR$. Alors nous avons la formule de changement de variables 
  \begin{equation}
    \int_{V}f(y)\, dy= \int_{U} f(\phi(x))\, \left| J_{\phi}(x)\right|\, dx,
  \end{equation}
  où $J_{\phi}$ est le déterminant de la matrice jacobienne\index{jacobienne} de $\phi$. 
\end{theorem}
Si $\phi$ est linéaire  alors le facteur $|J_{\phi}|$ est la mesure de l'image par $\phi$ d'une portion de $\eR^p$ de mesure $1$, sinon  $|J_{\phi}|$ est le rapport entre la mesure de l'image d'un élément infinitésimale de volume de $\eR^p$ et sa mesure originale. 
Soit $\phi(u,v)=g(u,v)e_1+h(u,v)e_2$ un difféomorphisme dans $\eR^2$. Soit $(x_0, y_0)$ l'image par $\phi$ de $(u_0,v_0)$. On considère le petit rectangle $R$ de sommets $(u_0,v_0)$, $(u_0+\Delta u,v_0)$, $(u_0+\Delta u,v_0+\Delta v)$ et $(u_0,v_0+\Delta v)$. L'image de $R$ n'est pas un rectangle en général, mais peut être bien approximée par le rectangle de sommets $(x_0,y_0)$, $(x_0 ,y_0)+ \phi_{u}\Delta u$, $(x_0 ,y_0)+\phi_{u}\Delta u +\phi_{v}\Delta v$ et  $(x_0 ,y_0)+ \phi_{v}\Delta v$ et son aire est $\| \phi_{u}\times \phi_{v}\| \Delta u\Delta v$. La valeur $|\phi_{u}\times \phi_{v}|$ est exactement $|J_{\phi}|$ 

\begin{example}
Soit $V$ la région trapézoïdale de sommets $(0,-1)$, $(1,0)$, $(2,0)$, $(0,-2)$, comme à la figure \ref{LabelFigZTTooXtHkcissLabelSubFigZTTooXtHkci0}. Calculons ensemble l'intégrale double  
\[
\int_{V}e^{\frac{x+y}{x-y}}\,dV,
\] 
avec le changement de variable $\psi(x,y)=(x+y,x-y)$. C'est à dire que nous considérons les nouvelles variables
\begin{subequations}
	\begin{numcases}{}
		u=x+y\\
		v=x-y.
	\end{numcases}
\end{subequations}
Il faut remarquer d'abord que le changement de variable proposé est dans le mauvais sens. On écrit alors $\phi(u,v)=\psi^{-1}(u,v)=\big((u+v)/2, (u-v)/2\big)$, c'est à dire
\begin{subequations}
	\begin{numcases}{}
		x=\frac{ u+v }{ 2 }\\
		y=\frac{ u-v }{2}.
	\end{numcases}
\end{subequations}
La région qui correspond à $V$ est $U$, le trapèze de sommets  $(-1,1)$, $(1,1)$, $(2,2)$ et $(-2,2)$, qu'on voit sur la figure \ref{LabelFigZTTooXtHkcissLabelSubFigZTTooXtHkci1} et qu'on décrit par
\[
U=\{ (u,v)\in\eR^2\,\vert\, 1\leq v\leq 2, \, -v\leq u\leq v\}.
\] 

% Celui-ci a été supprimée le 17 juillet 2014
%\ref{LabelFigexamplechangementvariables}
%\newcommand{\CaptionFigexamplechangementvariables}{Avant et après le changement de variables}
%\input{Fig_examplechangementvariables.pstricks}

%The result is on figure \ref{LabelFigZTTooXtHkci}. % From file ZTTooXtHkci
%See also the subfigure \ref{LabelFigZTTooXtHkcissLabelSubFigZTTooXtHkci0}
%See also the subfigure \ref{LabelFigZTTooXtHkcissLabelSubFigZTTooXtHkci1}
\newcommand{\CaptionFigZTTooXtHkci}{Avant et après le changement de variables}
\input{Fig_ZTTooXtHkci.pstricks}

On observe que $U$ est une région du premier type tandis que $V$ n'est pas du premier ou du deuxième type. Le déterminant de la  matrice  jacobienne de $\psi^{-1}$ est  $J_{\psi^{-1}}$,
\begin{equation}
 J_{\psi^{-1}}(u,v)= \left\vert\begin{array}{cc}
\frac{1}{2} & \frac{1}{2} \\
\frac{1}{2}  & -\frac{1}{2}
\end{array}\right\vert= -\frac{1}{2}.
\end{equation}
On a alors 
\[
\int_{V}e^{\frac{x+y}{x-y}}\,dV=\int_{U}e^{\frac{u}{v}}\,\frac{1}{2}\,dV=\int_1^2\int_{-v}^{v}e^{\frac{u}{v}}\,\frac{1}{2}\, du\,dv= \frac{3}{4}(e-e^{-1}).
\] 
\end{example}

\begin{example} 
\textbf{Coordonnées polaires : }On veut évaluer l'intégrale de la fonction $f(x,y)= x^2+y^2$ sur la région $V$ suivante :
\[
V=\{(x,y) \in \eR^2\,\vert\, x^2+y^2\leq 1,\, x>0,\, y>0\}.
\]
On peut faire le calcul directement,
\[
\int_{V}f(x,y)\, dV=\int_0^1\int_0^{\sqrt{1-x^2}}x^2+y^2\, dy\,dx=\int_0^1x^2\sqrt{1-x^2} + \frac{(1-x^2)^{3/2}}{3}\, dx  
\] 
mais c'est un peu ennuyeux. On peut simplifier beaucoup les calculs avec un changement de variables vers les coordonnées polaires. Dans ce cas, on sait bien que le difféomorphisme à utiliser est $\phi(r,\theta)=(r\cos \theta, r\sin\theta)$. Le jacobien  $J_{\phi}$ est
\begin{equation}
 J_{\phi}(r, \theta)= \left\vert\begin{array}{cc}
\cos \theta & \sin \theta \\
-r\sin \theta  & r\cos \theta
\end{array}\right\vert= r,
\end{equation}
qui est toujours positif. La fonction $f$ peut s'écrire comme $f(\phi(r,\theta))=r^2$ et $\phi^{-1}(V)=]0,1]\times]0, \pi/2[$.  
La formule du changement de variables nous donne
\[
\int_{V}f(x,y)\, dV=\int_0^{\pi/2}\int_0^{1}r^3 dr\,d\theta=\int_0^{\pi/2}\frac{1}{4}\,d\theta=\frac{\pi}{8}.  
\] 
\end{example}

\begin{example}
\textbf{Coordonnées cylindriques : }On veut calculer le volume de la région $A$ définie par  l'intersection entre la boule unité et le cylindre qui a pour base un disque de rayon $1/2$ centré en $(0, 1/2)$
\[
A=\{(x,y,z) \in\eR^3 \,\vert\, x^2+y^2+z^1\leq 1\}\cap\{(x,y,z) \in \eR^3\,\vert\, x^2+(y-1/2)^2\leq 1/4\}.
\]
On peut décrire $A$ en coordonnées cylindriques
\begin{equation}
  \begin{aligned}
    A=\Big\{(r,\theta,z) &\in ]0, +\infty[\times [-\pi,\pi[\times \eR\,\vert\,\\
& -\pi/2<\theta<\pi, \, 0<r\leq \sin\theta, \, -\sqrt{1-r^2}\leq z\leq\sqrt{1-r^2} \Big\}.
  \end{aligned}
\end{equation}
Le jacobien de ce changement de variables,  $J_{cyl}$, est
\begin{equation}
 J_{cyl}(r, \theta), z= \left\vert\begin{array}{ccc}
\cos \theta & \sin \theta & 0\\
-r\sin \theta  & r\cos \theta &0 \\
0&0&
\end{array}\right\vert= r,
\end{equation}
qui est toujours positif. Le volume de $A$ est donc
\[
\int_{\eR^3}\chi_{A}(x,y,z)\, dV=\int_{-\pi/2}^{\pi/2}\int_0^{\sin\theta}\int_{-\sqrt{1-r^2}}^{\sqrt{1-r^2}} r dz\,dr\,d\theta=\frac{2\pi}{8}+\frac{8}{9}.  
\] 
\end{example}

\begin{example}
\textbf{Volume d'un solide de révolution : }Soit $g:[a,b]\to\eR_+$ une fonction continue et positive. On dit que le solide $A$ décrit par
\[
A=\left\{(x,y,z)\in\eR^3\, \vert \, z\in[a,b], \,\sqrt{x^2+y^2}\leq g^2(z) \right\}
\]
est un solide de révolution. Afin de calculer son volume, on peut décrire $A$ en coordonnées cylindriques, 
\[
A=\left\{(r,\theta,z) \in ]0, +\infty[\times [-\pi,\pi[\times \eR\,\vert\, a\leq z\leq b, \, 0<r^2\leq g^2(z) \right\}.
\]
Le jacobien de ce changement de variables est  $J_{cyl}=r$, comme dans l'exemple précédent. Le volume de $A$ est donc
\[
\int_{\eR^3}\chi_{A}(x,y,z)\, dV=\int_a^{b}\int_{-\pi}^{\pi}\int_{0}^{g(z)} r  \,dr\,d\theta\, dz=\int_a^{b} \pi g^2(z) \, dz.
\] 
Cette formule peut être utilisée pour tout solide de révolution. 
\end{example}

\begin{example}
\textbf{Coordonnées sphériques : }On veut calculer le volume du cornet de glace  $A$ 
\[
A=\left\{(x,y,z)\in\eR^3\, \vert \, (x,y)\in \mathbb{S}^2, \,\sqrt{x^2+y^2}\leq z\leq \sqrt{1-x^2-y^2} \right\}. 
\]
On peut décrire $A$ en coordonnées sphériques. 
\[
A=\{(\rho,\theta,\phi) \in ]0, +\infty[\times [-\pi,\pi[\times [0,\pi[\,\vert\, 0<\phi\leq\pi/4, \, 0<\rho\leq 1 \}.
\]
Le jacobien de ce changement de variables  $J_{sph}$ est
\begin{equation}
 J_{sph}(\rho, \theta, \phi)= \left\vert\begin{array}{ccc}
\cos \theta \sin\phi & \sin \theta\sin\phi & \cos\phi\\
-\rho\sin \theta\sin\phi  & \rho\cos \theta\sin\phi & 0 \\
\rho\cos\theta\cos\phi&\rho\sin\theta\cos\phi& -\rho\sin\phi
\end{array}\right\vert= \rho^2\sin\phi,
\end{equation}
Le volume de $A$ est donc
\[
\int_{\eR^3}\chi_{A}(x,y,z)\, dV=\int_{-\pi}^{\pi}\int_0^{\pi/4}\int_{0}^{1}\rho^2\sin\phi \,d\rho\,d\phi\,d\theta=\frac{2\pi}{3}\left(1-\frac{1}{\sqrt{2}}\right).  
\] 
\end{example}

%---------------------------------------------------------------------------------------------------------------------------
\subsection{Récapitulatif des changements de variables}
%---------------------------------------------------------------------------------------------------------------------------

En pratique, nous retiendrons les formules suivantes:
%///////////////////////////////////////////////////////////////////////////////////////////////////////////////////////////
\subsubsection{Coordonnées polaires}
%///////////////////////////////////////////////////////////////////////////////////////////////////////////////////////////

\begin{subequations}
    \begin{numcases}{}
        x=r\cos\theta\\
        y=r\sin\theta
    \end{numcases}
\end{subequations}
avec \( r\in\mathopen] 0 , \infty \mathclose[\) et \( \theta\in\mathopen[ 0 , 2\pi [\). Le jacobien vaut \( r\).

%///////////////////////////////////////////////////////////////////////////////////////////////////////////////////////////
\subsubsection{Coordonnées cylindriques}
%///////////////////////////////////////////////////////////////////////////////////////////////////////////////////////////

\begin{subequations}
    \begin{numcases}{}
        x=r\cos\theta\\
        y=r\sin\theta\\
        z=z
    \end{numcases}
\end{subequations}
avec \( r\in\mathopen] 0 , \infty \mathclose[\), \( \theta\in\mathopen[ 0 , 2\pi [\) et \( z\in\eR\). Le jacobien vaut \( r\).

%///////////////////////////////////////////////////////////////////////////////////////////////////////////////////////////
\subsubsection{Coordonnées sphériques}
%///////////////////////////////////////////////////////////////////////////////////////////////////////////////////////////

\begin{subequations}
    \begin{numcases}{}
        x=\rho\cos\theta\sin\phi\\
        y=\rho\sin\theta\sin\phi\\
        z=\rho\cos\phi
    \end{numcases}
\end{subequations}
avec \( \rho\in\mathopen] 0 , \infty \mathclose[\), \( \theta\in\mathopen[ 0 , 2\pi [\) et \( \phi\in\mathopen[ 0 , \pi [\). Le jacobien vaut \( -\rho^2\sin(\phi)\). 

N'oubliez pas que lorsqu'on effectue un changement de variables dans une intégrale, la \emph{valeur absolue} du jacobien apparaît.

Cependant notre convention de coordonnées sphériques fait venir \( \sin(\phi)\) avec \( \phi\in\mathopen[ 0 , \pi [\); vu que le signe de \( \sin(\phi)\) y est toujours positif, cette histoire de valeur absolue est sans grandes conséquent. Ce n'est pas le cas de toutes les conventions possibles.

%---------------------------------------------------------------------------------------------------------------------------
					\subsection{Changement de variables}
%---------------------------------------------------------------------------------------------------------------------------

Le domaine $E=\{ (x,y)\in\eR^2\tq x^2+y^2<1 \}$ s'écrit plus facilement $E=\{ (r,\theta)\tq r<1 \}$ en coordonnées polaires. Le passage aux coordonnées polaire permet de transformer une intégration sur un domaine rond à une intégration sur le domaine rectangulaire $\mathopen]0,2\pi\mathclose[\times\mathopen]0,1\mathclose[$. La question est évidement de savoir si nous pouvons écrire
\begin{equation}
	\int_Ef=\int_{0}^{2\pi}\int_0^1f(r\cos\theta,r\sin\theta)drd\theta.
\end{equation}
Hélas, non; la vie n'est pas aussi simple.

\begin{theorem}
Soit $g\colon A\to B$ un difféomorphisme. Soient $F\subset B$ un ensemble mesurable et borné et $f\colon F\to \eR$ une fonction bornée et intégrable. Supposons que $g^{-1}(F)$ soit borné et que $Jg$ soit borné sur $g^{-1}(F)$. Alors
\begin{equation}
	\int_Ff(x)dy=\int_{g^{-1}(F)f\big( g(x) \big)}| Jg(x) |dx
\end{equation}
\end{theorem}
Pour rappel, $Jg$ est le déterminant de la matrice \href{http://fr.wikipedia.org/wiki/Matrice_jacobienne}{jacobienne} (aucun lien de \href{http://fr.wikipedia.org/wiki/Jacob}{parenté}) donnée par
\begin{equation}
	Jg=\det\begin{pmatrix}
	\partial_xg_1	&	\partial_yg_1	\\ 
	\partial_xg_2	&	\partial_tg_2	
\end{pmatrix}.
\end{equation}
Un \defe{difféomorphisme}{difféomorphisme} est une application $g\colon A\to B$ telle que $g$ et $g^{-1}\colon B\to A$ soient de classe $C^1$.

%///////////////////////////////////////////////////////////////////////////////////////////////////////////////////////////
					\subsubsection{Coordonnées polaires}
%///////////////////////////////////////////////////////////////////////////////////////////////////////////////////////////

Les coordonnées polaires sont données par le difféomorphisme
\begin{equation}
	\begin{aligned}
		g\colon \mathopen]0,\infty\mathclose[\times\mathopen]0,2\pi\mathclose[ &\to\eR^2\setminus D\\
		(r,\theta)&\mapsto \big( r\cos(\theta),r\sin(\theta) \big)
	\end{aligned}
\end{equation}
où $D$ est la demi droite $y=0$, $x\geq 0$. Le fait que les coordonnées polaires ne soient pas un difféomorphisme sur tout $\eR^2$ n'est pas un problème pour l'intégration parce que le manque de difféomorphisme est de mesure nulle dans $\eR^2$. Le jacobien est donné par
\begin{equation}
	Jg=\det\begin{pmatrix}
	\partial_rx	&	\partial_{\theta}x	\\ 
	\partial_ry	&	\partial_{\theta}y
\end{pmatrix}=\det\begin{pmatrix}
	\cos(\theta)	&	-r\sin(\theta)	\\ 
	\sin(\theta)	&	r\cos(\theta)	
\end{pmatrix}=r.
\end{equation}

\begin{example}    
    Montrons comment intégrer la fonction $f(x,y)=\sqrt{1-x^2-y^2}$ sur le domaine délimité par la droite $y=x$ et le cercle $x^2+y^2=y$, représenté sur la figure \ref{LabelFigQXyVaKD}. Pour trouver le centre et le rayon du cercle $x^2+y^2=y$, nous commençons par écrire $x^2+y^2-y=0$, et ensuite nous reformons le carré : $y^2-y=(y-\frac{ 1 }{2})^2-\frac{1}{ 4 }$.
    \newcommand{\CaptionFigQXyVaKD}{Passage en polaire pour intégrer sur un morceau de cercle.}
\input{Fig_QXyVaKD.pstricks}

    Le passage en polaire transforme les équations du bord du domaine en
    \begin{equation}
        \begin{aligned}[]
            \cos(\theta)&=\sin(\theta)\\
            r^2&=r\sin(\theta).
        \end{aligned}
    \end{equation}
    L'angle $\theta$ parcours donc $\mathopen] 0 , \pi/4 \mathclose[$, et le rayon, pour chacun de ces $\theta$ parcours $\mathopen] 0 , \sin(\theta) \mathclose[$. La fonction à intégrer se note maintenant $f(r,\theta)=\sqrt{1-r^2}$. Donc l'intégrale à calculer est
    \begin{equation}		\label{PgRapIntMultFubiniBoutCercle}
        \int_{0}^{\pi/4}\left( \int_0^{\sin(\theta)}\sqrt{1-r^2}r\,rd \right).
    \end{equation}
    Remarquez la présence d'un $r$ supplémentaire pour le jacobien.

    Notez que les coordonnées du point $P$ sont $(1,1)$.
\end{example}

%///////////////////////////////////////////////////////////////////////////////////////////////////////////////////////////
\subsubsection{Coordonnées sphériques}
%///////////////////////////////////////////////////////////////////////////////////////////////////////////////////////////

Les coordonnées sphériques sont données par
\begin{equation}		\label{EqChmVarSpherique}
	\left\{
\begin{array}{lllll}
x=r\cos\theta\sin\varphi	&			&r\in\mathopen] 0 , \infty \mathclose[\\
y=r\sin\theta\sin\varphi	&	\text{avec}	&\theta\in\mathopen] 0 , 2\pi \mathclose[\\
z=r\cos\varphi			&			&\phi\in\mathopen] 0 , \pi \mathclose[.
\end{array}
\right.
\end{equation}
Le jacobien associé est $Jg(r,\theta,\varphi)=-r^2\sin\varphi$. Rappelons que ce qui rentre dans l'intégrale est la valeur absolue du jacobien.

Si nous voulons calculer le volume de la sphère de rayon $R$, nous écrivons donc
\begin{equation}
	\int_0^Rdr\int_{0}^{2\pi}d\theta\int_0^{\pi}r^2 \sin(\phi)d\phi=4\pi R=\frac{ 4 }{ 3 }\pi R^3.
\end{equation}
Ici, la valeur absolue n'est pas importante parce que lorsque $\phi\in\mathopen] 0,\pi ,  \mathclose[$, le sinus de $\phi$ est positif.

Des petits malins pourraient remarquer que le changement de variable \eqref{EqChmVarSpherique} est encore une paramétrisation de $\eR^3$ si on intervertit le domaine des angles : 
\begin{equation}
	\begin{aligned}[]
		\theta&\colon 0 \to \pi\\
		\phi	&\colon 0\to 2\pi,
	\end{aligned}
\end{equation}
alors nous paramétrons encore parfaitement bien la sphère, mais hélas
\begin{equation}		\label{EqVolumeIncorrectSphere}
	\int_0^Rdr\int_{0}^{\pi}d\theta\int_0^{2\pi}r^2 \sin(\phi)d\phi=0.
\end{equation}
Pourquoi ces «nouvelles» coordonnées sphériques sont-elles mauvaises ? Il y a que quand l'angle $\phi$ parcours $\mathopen] 0 , 2\pi \mathclose[$, son sinus n'est plus toujours positif, donc la \emph{valeur absolue} du jacobien n'est plus $r^2\sin(\phi)$, mais $r^2\sin(\phi)$ pour les $\phi$ entre $0$ et $\pi$, puis $-r^2\sin(\phi)$ pour $\phi$ entre $\pi$ et $2\pi$. Donc l'intégrale \eqref{EqVolumeIncorrectSphere} n'est pas correcte. Il faut la remplacer par
\begin{equation}
	\int_0^Rdr\int_{0}^{\pi}d\theta\int_0^{\pi}r^2 \sin(\phi)d\phi- \int_0^Rdr\int_{0}^{\pi}d\theta\int_{\pi}^{2\pi}r^2 \sin(\phi)d\phi = \frac{ 4 }{ 3 }\pi R^3
\end{equation}

%---------------------------------------------------------------------------------------------------------------------------
					\subsection{Passage à la limite sous le signe intégral}
%---------------------------------------------------------------------------------------------------------------------------

Un autre résultat très important pour l'étude de l'intégrabilité est le théorème de la \defe{convergence dominée de Lebesgue}{}:
\begin{theorem}
	Soit $E\subset \eR^n$ un ensemble mesurable et $\{ f_k \}$, une suite de fonctions intégrables sur $E$ qui converge simplement vers une fonction $f\colon E\to \overline{ \eR }$. Supposons qu'il existe une fonction $g$ intégrable sur $E$ telle que pour tout $k$,
\begin{equation}
	| f(x) |\leq g(x)
\end{equation}
pour tout $x\in E$. Alors $f$ est intégrable sur $E$ et 
\begin{equation}
	\int_Ef=\lim_{k\to\infty}\int_Ef_k.
\end{equation}
\end{theorem}

%---------------------------------------------------------------------------------------------------------------------------
					\subsection{Théorème de Fubini et changement de variables}
%---------------------------------------------------------------------------------------------------------------------------

\begin{theorem}[Fubini]\label{ThoFubini}
Soit $(x,t)\mapsto f(x,y)\in\bar \eR$ une fonction intégrable sur $B_n\times B_m\subset\eR^{n+m}$ où $B_n$ et $B_m$ sont des ensembles mesurables de $\eR^n$ et $\eR^m$. Alors :
\begin{enumerate}
\item pour tout $x\in B_n$, sauf éventuellement en les points d'un ensemble $G\subset B_n$ de mesure nulle, la fonction $y\in B_m\mapsto f(x,y)\in\bar\eR$ est intégrable sur $B_m$
\item
la fonction
\begin{equation}
	x\in B_n\setminus G\mapsto\int_{B_m}f(x,y)dy\in\eR
\end{equation}
est intégrable sur $B_n\setminus G$

\item 
On a
\begin{equation}
	\int_{B_n\times B_m}f(x,y)dxdy=\int_{B_n}\left( \int_{B_m}f(x,y)dy \right)dx.
\end{equation}

\end{enumerate}
\end{theorem}
\index{théorème!Fubini!dans $ \eR^n$}
\index{Fubini!théorème!dans $ \eR^n$}


Notons en particulier que si $f(x,y)=\varphi(x)\phi(y)$, alors $\int_{B_m}\varphi(y)dy$ est une constante qui peut sortir de l'intégrale sur $B_n$, et donc
\begin{equation}		\label{EqFubiniFactori}
	\int_{B_n\times B_m}\varphi(x)\phi(y)dxdy=\int_{B_n}\varphi(x)dx\int_{B_m}\phi(y)dy.
\end{equation}

%---------------------------------------------------------------------------------------------------------------------------
					\subsection{Intégrale en dimension un}
%---------------------------------------------------------------------------------------------------------------------------

\begin{proposition}[Critère de comparaison]
Soit $f$ mesurable sur $]a,\infty[$ et bornée sur tout $]a,b]$, et supposons qu'il existe un $X_0\geq a$, tel que sur $]X_0,\infty[$,
\begin{equation}
	| f(x) |\leq g(x)
\end{equation}
où $g(x)$ est intégrable. Alors $f(x)$ est intégrable sur $]a,\infty[$.
\end{proposition}

\begin{corollary}[Critère d'équivalence]
Soient $f$ et $g$ des fonctions mesurables et positives ou nulles sur $]a,\infty[$, bornées sur tout $]a,b]$, telles que 
\begin{equation}
	\lim_{x\to\infty}\frac{ f(x) }{ g(x) }=L
\end{equation}
existe dans $\bar\eR$.
\begin{enumerate}
\item Si $L\neq\infty$ et $\int_{a}^{\infty}g(x)$ existe, alors $\int_a^{\infty}f(x)dx$ existe,
\item Si $L\neq 0$ et si $\int_a^{\infty}f(x)dx$ existe, alors $\int_a^{\infty}g(x)dx$ existe,
\end{enumerate}
\end{corollary}

\begin{corollary}[Critère des fonctions test]			\label{CorCritFonsTest}
Soit $f(x)$ une fonction mesurable et positive ou nulle sur $]a,\infty[$ et bornée pour tout $]a,b]$. Nous posons
\begin{equation}
	L(\alpha)=\lim_{x\to\infty}x^{\alpha}f(x),
\end{equation}
et nous supposons qu'elle existe.
\begin{enumerate}
\item Si il existe $\alpha>1$ tel que $L(\alpha)\neq\infty$, alors $\int_a^{\infty}f(x)dx$ existe,
\item Si il existe $\alpha\leq1$ et $L(\alpha)\neq 0$, alors $\int_a^{\infty}f(x)dx$ n'existe pas.
\end{enumerate}
\end{corollary}

\begin{corollary}		\label{CorAlphaLCasInteabf}
	Soit $f\colon ]a,b]\to \eR$ une fonction mesurable, positive ou nulle, et bornée sur $[a+\epsilon,b]$ $\forall\epsilon>0$. Si $\lim_{x\to a}(x-a)^{\alpha}f(x)=L$ existe, alors
	\begin{enumerate}
		\item Si $\alpha<1$ et $L\neq\infty$, alors $\int_a^bf(x)dx$ existe,
		\item Si $\alpha\geq 1$ et $L\neq 0$, alors $\int_a^bf(x)dx$ n'existe pas.
	\end{enumerate}
\end{corollary}

%---------------------------------------------------------------------------------------------------------------------------
					\subsection{Intégrales convergentes}
%---------------------------------------------------------------------------------------------------------------------------

\begin{definition}
    Soit $f$, une fonction mesurable sur $[a,\infty[$, bornée sur tout intervalle $[a,b]$. On dit que l'intégrale
    \begin{equation}
        \int_a^{\infty}f(x)dx
    \end{equation}
    \defe{converge}{intégrale!convergente} si la limite
    \begin{equation}		\label{EqDEfConvergeZeroInftX}
        \lim_{X\to\infty}\int_a^{X}f
    \end{equation}
    existe et est finie.
\end{definition}

%+++++++++++++++++++++++++++++++++++++++++++++++++++++++++++++++++++++++++++++++++++++++++++++++++++++++++++++++++++++++++++
\section{Trucs et astuces de calcul d'intégrales}
%+++++++++++++++++++++++++++++++++++++++++++++++++++++++++++++++++++++++++++++++++++++++++++++++++++++++++++++++++++++++++++

Afin d'alléger le texte de calculs parfois un peu longs, nous regroupons ici les intégrales à une variable que nous devons utiliser dans les autres parties du cours.

\begin{enumerate}
	\item	\label{ItemIntegrali}
		L'intégrale
		\begin{equation}
			\boxed{I=\int x\ln(x)dx=\frac{ x^2 }{2}\big( \ln(x)-\frac{ 1 }{2} \big)}
		\end{equation}
		se fait par partie en posant
		\begin{equation}
			\begin{aligned}[]
				u&=\ln(x),		& dv&=x\,dx\\
				du&=\frac{1}{ x }\,dx,	& v&=\frac{ x^2 }{2},
			\end{aligned}
		\end{equation}
		et ensuite
		\begin{equation}
			I=\ln(x)\frac{ x^2 }{2}-\int\frac{ x }{2}=\frac{ x^2 }{2}\big( \ln(x)-\frac{ 1 }{2} \big).
		\end{equation}
		
	\item	
		L'intégrale
		\begin{equation}
			\boxed{I=\int x\ln(x^2)dx=x^2\ln(x)-\frac{ x^2 }{2}.}
		\end{equation}
		En utilisant le fait que $\ln(u^2)=2\ln(u)$, nous retombons sur une intégrale du type \ref{ItemIntegrali} :
		\begin{equation}
			I=x^2\ln(x)-\frac{ x^2 }{2}.
		\end{equation}
	\item
		L'intégrale
		\begin{equation}		\label{EqTrucIntxlnxsqpun}
			\boxed{I=\int x\ln(1+x^2)dx=\frac{ 1 }{2}\ln(x^2+1)(x^2+1)-x^2-\frac{ 1 }{2}}
		\end{equation}
		se traite en posant $v=1+x^2$ de telle sorte à avoir $dx=\frac{ dv }{ 2x }$ et donc
		\begin{equation}
			I=\frac{ 1 }{2}\ln(x^2+1)(x^2+1)-x^2-\frac{ 1 }{2}.
		\end{equation}
		
	\item
		L'intégrale
		\begin{equation}
			I=\int \cos(\theta)\sin(\theta)\ln\left( 1+\frac{1}{ \cos^2(\theta) } \right)\,d\theta
		\end{equation}
		demande le changement de variable $u=\cos(\theta)$, $d\theta=-\frac{ du }{ \sin(\theta) }$. Nous tombons sur l'intégrale
		\begin{equation}
			I=-\int u\ln\left( \frac{ 1+u^2 }{ u^2 } \right)=-\int u\ln(1+u^2)+\int u\ln(u^2),
		\end{equation}
		qui sont deux intégrales déjà faites. Nous trouvons
		\begin{equation}
			I=-\frac{ 1 }{2}\ln\left( \frac{ \sin^2(\theta)-1 }{ \sin^2(\theta)-2 } \right)\sin^2(\theta)-\ln\big( \sin^2(\theta)-2 \big)+\frac{ 1 }{2}\ln\big( \sin^2(\theta)-1 \big)
		\end{equation}
	
	\item
		L'intégrale
		\begin{equation}
			\boxed{\int \frac{ r^3 }{ 1+r^2 }dr=\frac{ r^2 }{2}-\frac{ 1 }{2}\ln(r^2+1).}
		\end{equation}
		commence par faire la division euclidienne de $r^3$ par $r^2+1$; ce que nous trouvons est $r^3=(r^2+1)r-r$. Il reste à intégrer
		\begin{equation}
			\int \frac{ r^3 }{ 1+r^2 }dr=\int r\,dr-\int\frac{ r }{ 1+r^2 }dr.
		\end{equation}
		La fonction dans la seconde intégrale est $\frac{ r }{ 1+r^2 }=\frac{ 1 }{2}\frac{ f'(r) }{ f(r) }$ où $f(r)=1+r^2$, et donc $\int \frac{ r }{ 1+r^2 }=\frac{ 1 }{2}\ln(1+r^2)$. Au final,
		\begin{equation}
			I=\frac{ 1 }{2}r^2-\frac{ 1 }{2}\ln(r^2+1).
		\end{equation}


	\item	
		L'intégrale
		\begin{equation}	\label{EqTrucIntsxcxdx}
			\boxed{I=\int \cos(\theta)\sin(\theta)d\theta=\frac{ \sin^2(\theta) }{ 2 }}
		\end{equation}
		se traite par le changement de variable $u=\sin(\theta)$, $du=\cos(\theta)d\theta$, et donc
		\begin{equation}
			\int\cos(\theta)\sin(\theta)d\theta=\int udu=\frac{ u^2 }{2}=\frac{ \sin^2(\theta) }{ 2 }.
		\end{equation}
	\item
		L'intégrale
		\begin{equation}	\label{EqTrucsIntsqrtAplusu}
			\boxed{\int\sqrt{1+x^2}dx=\frac{ x }{2}\sqrt{1+x^2}+\frac{ 1 }{2}\arcsinh(x)}
		\end{equation}
		s'obtient en effectuant le changement de variable $u=\sinh(\xi)$.

    \item
        L'intégrale
        \begin{equation}        \label{EqTrucIntcossqsinsq}
            \boxed{ \int\cos^2(x)\sin^2(x)dx=\frac{ x }{ 8 }-\frac{ \sin(4x) }{ 32 } }
        \end{equation}
        s'obtient à coups de formules de trigonométrie. D'abord, $\sin(t)\cos(t)=\frac{ 1 }{2}\sin^2(2t)$ fait en sorte que la fonction à intégrer devient 
        \begin{equation}
            f(x)=\frac{1}{ 4 }\sin^2(x).
        \end{equation}
        Ensuite nous utilisons le fait que $\sin^2(t)=(1-\cos(2t))/2$ pour transformer la formule à intégrer en
        \begin{equation}
            f(x)=\frac{ 1-\cos(4x) }{ 8 }.
        \end{equation}
        Cela s'intègre facilement en posant $u=4x$, et le résultat est
        \begin{equation}
            \int f(x)dx=\frac{ x }{ 8 }-\frac{ \sin(4x) }{ 32 }.
        \end{equation}

    \item

        La fonction 
        \begin{equation}
            \sinc(x)=\frac{ \sin(x) }{ x }
        \end{equation}
        est le \defe{sinus cardinal}{sinus cardinal} de \( x\). Nous allons montrer que
        \begin{equation}    \label{EqKNOmLEd}
            \boxed{  \int_0^{\infty}\big| \sinc(x) \big|dx=\infty  }.
        \end{equation}
        D'abord nous avons
        \begin{equation}
            \int_{(n-1)\pi}^{n\pi}\frac{ \big| \sin(t) \big| }{ t }dt\geq \int_{(n-1)\pi}^{n\pi}\frac{ \big| \sin(t) \big| }{ n\pi }dt,
        \end{equation}
        mais par périodicité,
        \begin{equation}
            \int_{(n-1)\pi}^{n\pi}\big| \sin(t) \big|dt=\int_0^{\pi}\sin(t)dt=2.
        \end{equation}
        Par conséquent
        \begin{equation}
            \int_0^{n\pi}\big| \sinc(t) \big|dt\geq \frac{ 2 }{ \pi }\sum_{k=1}^n\frac{1}{ k },
        \end{equation}
        ce qui diverge lorsque \( n\to \infty\).

\end{enumerate}

%+++++++++++++++++++++++++++++++++++++++++++++++++++++++++++++++++++++++++++++++++++++++++++++++++++++++++++++++++++++++++++ 
\section{Ellipsoïde de John-Loewer}
%+++++++++++++++++++++++++++++++++++++++++++++++++++++++++++++++++++++++++++++++++++++++++++++++++++++++++++++++++++++++++++

Soit \( q\) une forme quadratique sur \( \eR^n\) ainsi que \( \mB\) une base orthonormée de \( \eR^n\) dans laquelle la matrice de  \( q\) est diagonale. Dans cette base, la forme \( q\) est donnée par la proposition \ref{PropFWYooQXfcVY} :
\begin{equation}
    q(x)=\sum_i\lambda_ix_i
\end{equation}
où les \( \lambda_i\) sont les valeurs propres de \( q\).

Plus généralement nous notons \( mat_{\mB}(q)\)\nomenclature[A]{\( mat_{\mB}(q)\)}{matrice de \( q\) dans la base \( \mB\)} la matrice de \( q\) dans la base \( \mB\) de \( \eR^n\).

\begin{proposition} \label{PropOXWooYrDKpw}
    Soit \( \mB\) une base orthonormée de \( \eR^n\) et l'application\footnote{L'ensemble \( Q(E)\) est l'ensemble des formes quadratiques sur \( E\).}
    \begin{equation}
        \begin{aligned}
            D\colon Q(\eR^n)&\to \eR \\
            q&\mapsto \det\big( mat_{\mB}(q) \big) .
        \end{aligned}
    \end{equation}
    Alors :
    \begin{enumerate}
        \item
            La valeur et \( D\) ne dépend pas du choix de la base orthonormée \( \mB\).
        \item
            La fonction \( D\) est donnée par la formule \( D(q)=\prod_i\lambda_i\) où les \( \lambda_i\) sont les valeurs propres de \( q\).
        \item
            La fonction \( D\) est continue.
    \end{enumerate}
\end{proposition}

\begin{proof}
    Soit \( q\) une forme quadratique sur \( \eR^n\). Nous considérons \( \mB\) une base de diagonalisation de \( q\) :
    \begin{equation}
        q(x)=\sum_i\lambda_ix_i
    \end{equation}
    où les \( x_i\) sont les composantes de \( x\) dans la base \( \mB\). Par définition, la matrice \( mat_{\mB}(q)\) est la matrice diagonale contenant les valeurs propres de \( q\).

    Nous considérons aussi \( \mB_1\), une autre base orthonormées de \( \eR^n\). Nous notons \( S=mat_{\mB_1}(q)\); étant symétrique, cette matrice se diagonalise par une matrice orthogonale : il existe \( P\in\gO(n,\eR)\) telle que
    \begin{equation}
        S=P mat_{\mB}(q)P^t;
    \end{equation}
    donc \( \det(S)=\det(PP^t)\det\big( \diag(\lambda_1,\ldots, \lambda_n) \big)=\lambda_1\ldots\lambda_n\). Ceci prouve en même temps que \( D\) ne dépend pas du choix de la base et que sa valeur est le produit des valeurs propres.

    Passons à la continuité. L'application déterminant \( \det\colon S_n(\eR^n)\to \eR\) est continue car polynôme en les composantes. D'autre par l'application \( mat_{\mB}\colon Q(\eR^n)\to S_n(\eR)\) est continue par la proposition \ref{PropFSXooRUMzdb}. L'application  \( D\) étant la composée de deux applications continues, elle est continue.
\end{proof}

\begin{proposition}[Ellipsoïde de John-Loewner\cite{KXjFWKA}]   \label{PropJYVooRMaPok}
    Soit \( K\) compact dans \( \eR^n\) et d'intérieur non vide. Il existe une unique ellipsoïde\footnote{Définition \ref{DefOEPooqfXsE}.} (pleine) de volume minimal contenant \( K\).
\end{proposition}
\index{déterminant!utilisation}
\index{extrema!volume d'un ellipsoïde}
\index{convexité!utilisation}

\begin{proof}
    Nous subdivisons la preuve en plusieurs parties.
    \begin{subproof}
        \item[À propos de volume d'un ellipsoïde]

            Soit \( \ellE\) un ellipsoïde. La proposition \ref{PropWDRooQdJiIr} et son corollaire \ref{CorKGJooOmcBzh} nous indiquent que 
            \begin{equation}
                \ellE=\{ x\in \eR^n\tq q(x)\leq 1 \}
            \end{equation}
            pour une certaine forme quadratique strictement définie positive \( q\). De plus il existe une base orthonormée \( \mB=\{ e_1,\ldots, e_n \}\) de \( \eR^n\) telle que 
            \begin{equation}    \label{EqELBooQLPQUj}
                q(x)=\sum_{i=1}^na_ix_i^2
            \end{equation}
            où \( x_i=\langle e_i, x\rangle \) et les \( a_i\) sont tous strictement positifs. Nous nommons \( \ellE_q\) l'éllipsoïde associée à la forme quadratique \( q\) et \( V_q\) son volume que nous allons maintenant calculer\footnote{Le volume ne change pas si nous écrivons l'inégalité stricte au lieu de large dans le domaine d'intégration; nous le faisons pour avoir un domaine ouvert.} :
            \begin{equation}
                V_q=\int_{\sum_ia_ix_i^2<1}dx
            \end{equation}
            Cette intégrale est écrite de façon plus simple en utilisant le \( C^1\)-difféomorphisme
            \begin{equation}
                \begin{aligned}
                    \varphi\colon \ellE_q&\to B(0,1) \\
                    x&\mapsto \Big( x_1\sqrt{a_1},\ldots, x_n\sqrt{a_n} \Big). 
                \end{aligned}
            \end{equation}
            Le fait que \( \varphi\) prenne bien ses valeurs dans \( B(0,1)\) est un simple calcul : si \( x\in\ellE_q\), alors
            \begin{equation}
                \sum_i\varphi(x)_i^2=\sum_ia_ix_i^2<1.
            \end{equation}
            Cela nous permet d'utiliser le théorème de changement de variables \ref{ThomFeRCi} :
            \begin{equation}
                V_q=\int_{\sum_ia_ix_i^2<1}dx=\frac{1}{ \sqrt{a_1\ldots a_n} }\int_{B(0,1)}dx.
            \end{equation}
            %TODO : le volume de la sphère dans \eR^n. Mettre alors une référence ici.
            La dernière intégrale est le volume de la sphère unité dans \( \eR^n\); elle n'a pas d'importance ici et nous la notons \( V_0\). La proposition \ref{PropOXWooYrDKpw} nous permet d'écrire \(V_q\) sous la forme
            \begin{equation}
                V_q=\frac{ V_0 }{ \sqrt{D(q)} }.
            \end{equation}
            
        \item[Existence de l'ellipsoïde]

            Nous voulons trouver un ellipsoïde contenant \( K\) de volume minimal, c'est à dire une forme quadratique \( q\in Q^{++}(\eR^n)\) telle que
            \begin{itemize}
                \item \( D(q)\) soit maximal
                \item \( q(x)\leq 1\) pour tout \( x\in K\).
            \end{itemize}
            Nous considérons l'ensemble des candidats semi-définis positifs.
            \begin{equation}
                A=\{ q\in Q^+\tq q(x)\leq 1\forall x\in K \}.
            \end{equation}
            Nous allons montrer que \( A\) est convexe, compact et non vide dans \( Q(\eR^n)\); il aura ainsi un maximum de la fonction continue \( D\) définie sur \( Q(\eR^n)\). Nous montrerons ensuite que le maximum est dans \( Q^{++}\). L'unicité sera prouvée à part.

            \begin{subproof}
            \item[Non vide]
                L'ensemble \( K\) est compact et donc borné par \( M>0\). La forme quadratique \( q\colon x\mapsto \| x \|^2/M^2\) est dans \( A\) parce que si \( x\in K\) alors 
                \begin{equation}
                    q(x)=\frac{ \| x \|^2 }{ M^2 }\leq 1.
                \end{equation}
            \item[Convexe]
                Soient \( q,q'\in A\) et \( \lambda\in\mathopen[ 0 , 1 \mathclose]\). Nous avons encore \( \lambda q+(1-\lambda)q'\in Q^+\) parce que 
                \begin{equation}
                    \lambda q(x)+(1-\lambda)q'(x)\geq 0
                \end{equation}
                dès que \( q(x)\geq 0\) et \( q'(x)\geq 0\).
            D'autre part si \( x\in K\) nous avons
            \begin{equation}
                \lambda q(x)+(1-\lambda)q'(x)\leq \lambda+(1-\lambda)=1.
            \end{equation}
            Donc \( \lambda q+(1-\lambda)q'\in A\).

        \item[Fermé]

            Pour rappel, la topologie de \( Q(\eR^n)\) est celle de la norme \eqref{EqZYBooZysmVh}. Nous considérons une suite \( (q_n)\) dans \( A\) convergeant vers \( q\in Q(\eR^n)\) et nous allons prouver que \( q\in A\), de sorte que la caractérisation séquentielle de la fermeture (proposition \ref{PropLFBXIjt}) conclue que \( A\) est fermé. En nommant \( e_x\) le vecteur unitaire dans la direction \( x\) nous avons
            \begin{equation}
                \big| q(x) \big|=\big| \| x \|^2q(e_x) \big|\leq \| x \|^2N(q),
            \end{equation}
            de sorte que notre histoire de suite convergente  donne pour tout \( x\) :
            \begin{equation}
                \big| q_n(x)-q(x) \big|\leq \| x \|^2N(q_n-q)\to 0.
            \end{equation}
            Vu que \( q_n(x)\geq 0\) pour tout \( n\), nous devons aussi avoir \( q(x)\geq 0\) et donc \( q\in Q^+\) (semi-définie positive). De la même manière si \( x\in K\) alors \( q_n(x)\leq 1\) pour tout \( n\) et donc \( q(x)\leq 1\). Par conséquent \( q\in A\) et \( A\) est fermé.

        \item[Borné]

            La partie \( K\) de \( \eR^n\) est borné et d'intérieur non vide, donc il existe \( a\in K\) et \( r>0\) tel que \( \overline{ B(a,r) }\subset K\). Si par ailleurs \( q\in A\) et \( x\in\overline{ B(0,r) }\) nous avons \( a+x\in K\) et donc \( q(a+x)\leq 1\). De plus \( q(-a)=q(a)\leq 1\), donc
            \begin{equation}
                \sqrt{q(x)}=\sqrt{q\big( x+a-a \big)}\leq \sqrt{q(x+a)}+\sqrt{q(-a)}\leq 2
            \end{equation}
            par l'inégalité de Minkowski \ref{PropACHooLtsMUL}. Cela prouve que si \( x\in\overline{ B(0,r) }\) alors \( q(x)\leq 4\). Si par contre \( x\in\overline{ B(0,1) }\) alors \( rx\in\overline{ B(0,r) } \) et 
            \begin{equation}
                0\leq q(x)=\frac{1}{ r^2 }q(rx)\leq \frac{ 4 }{ r^2 },
            \end{equation}
            ce qui prouve que \( N(q)\leq \frac{ 4 }{ r^2 }\) et que \( A\) est borné.


            \end{subproof}

            L'ensemble \( A\) est compact parce que fermé et borné, théorème de Borel-Lebesgue \ref{ThoXTEooxFmdI}. L'application continue \( D\colon Q(\eR^n)\to \eR\) de la proposition \ref{PropOXWooYrDKpw} admet donc un maximum sur le compact \( A\). Soit \( q_0\) ce maximum.

            Nous montrons que \( q_0\in Q^{++}(\eR^d)\). Nous savons que l'application \( f\colon x\mapsto \frac{ \| x \|^2 }{ M^2 }\) est dans \( A\) et que \( D(f)>0\). Vu que \( q_0\) est maximale pour \( D\), nous avons
            \begin{equation}
                D(q_0)\geq D(f)>0.
            \end{equation}
            Donc \( q_0\in Q^{++}\).

        \item[Unicité]

            Si il existe une autre ellipsoïde de même volume que celle associée à la forme quadratique \( q_0\), nous avons une forme quadratique \( q\in Q^{++}\) telle que \( q(x)\leq 1\) pour tout \( x\in K\). C'est à dire que nous avons \( q_0,q\in A\) tels que \( D(q_0)=D(q)\).

            Nous considérons la base canonique \( \mB_c\) de \( \eR^n\) et nous posons \( S=mat_{\mB_c}(q)\), \( S_0=mat_{\mB_c}(q_0)\). Étant donné que \( A\) est convexe, \( (q_0+q)/2\in A\) et nous allons prouver que cet élément de \( A\) contredit la maximalité de \( q_0\). En effet
            \begin{equation}
                D\left( \frac{ q+q_0 }{ 2 }\right)=\det\left( \frac{ S+S_0 }{2} \right)
            \end{equation}
            Nous allons utiliser le lemme \ref{LemXOUooQsigHs} qui dit que le logarithme est log-concave sous la forme de l'équation \eqref{EqSPKooHFZvmB} avec \( \alpha=\beta=\frac{ 1 }{2}\) :
            \begin{equation}    \label{eqBHJooYEUDPC}
                D\left( \frac{ q+q_0 }{ 2 }\right)=\det\left( \frac{ S+S_0 }{2} \right)>\sqrt{\det(S)}\sqrt{\det(S_0)}=\det(S_0)=D(q_0).
            \end{equation}
            Nous avons utilisé le fait que \( D(q_0)=D(q)\) qui signifie que \( \det(S_0)=\det(S)\). L'inéquation \eqref{eqBHJooYEUDPC} contredit la maximalité de \( D(q_0)\) et donne donc l'unicité.
    \end{subproof}
\end{proof}
% This is part of Mes notes de mathématique
% Copyright (c) 2011-2014
%   Laurent Claessens
% See the file fdl-1.3.txt for copying conditions.

%+++++++++++++++++++++++++++++++++++++++++++++++++++++++++++++++++++++++++++++++++++++++++++++++++++++++++++++++++++++++++++
\section{Rappel sur les intégrales usuelles}
%+++++++++++++++++++++++++++++++++++++++++++++++++++++++++++++++++++++++++++++++++++++++++++++++++++++++++++++++++++++++++++

%TODO : l'utilisation des macros \og et \fg ne se justifie plus : les enlever.

Soit une fonction
\begin{equation}
    \begin{aligned}
        f\colon \mathopen[ a , b \mathclose]\subset\eR&\to \eR^+ \\
        x&\mapsto f(x) .
    \end{aligned}
\end{equation}
L'intégrale de $f$ sur le segment $\mathopen[ a , b \mathclose]$, notée $\int_a^bf(x)dx$ est le nombre égal à l'aire de la surface située entre le graphe de $f$ et l'axe des $x$, comme indiqué à la figure \ref{LabelFigIntegraleSimple}.
\newcommand{\CaptionFigIntegraleSimple}{L'intégrale de $f$ entre $a$ et $b$ représente la surface sous la fonction.}
\input{Fig_IntegraleSimple.pstricks}

\begin{definition}
    Si $f$ est une fonction de une variable à valeurs réelles, une \defe{primitive}{primitive} de $f$ est une fonction $F$ telle que $F'=f$.
\end{definition}

Toute fonction continue admet une primitive.

\begin{theorem}[Théorème fondamental du caclul intégral]
    Si $f$ est une fonction positive et continue, et si $F$ est une primitive de $f$, alors
    \begin{equation}
        \int_a^bf(x)dx=F(b)-F(a).
    \end{equation}
\end{theorem}

\begin{remark}
    Si $f$ est une fonction continue par morceaux, l'intégrale de $f$ se calcule comme la somme des intégrales de ses morceaux. Plus précisément si nous avons $a=x_0<x_1<\ldots<x_n=b$ et si $f$ est continue sur $\mathopen] x_i , x_{i+1} \mathclose[$ pour tout $i$, alors nous posons
    \begin{equation}
        \int_a^bf(x)dx=\int_{x_0}^{x_1}f(x)dx+\int_{x_1}^{x_2}f(x)dx+\ldots+\int_{x_{n-1}}^{n_n}f(x)dx.
    \end{equation}
    Sur chacun des morceaux, l'intégrale se calcule normalement en passant par une primitive.
\end{remark}

%+++++++++++++++++++++++++++++++++++++++++++++++++++++++++++++++++++++++++++++++++++++++++++++++++++++++++++++++++++++++++++
\section{Intégration de fonction à deux variables}
%+++++++++++++++++++++++++++++++++++++++++++++++++++++++++++++++++++++++++++++++++++++++++++++++++++++++++++++++++++++++++++

%---------------------------------------------------------------------------------------------------------------------------
\subsection{Intégration sur un domaine rectangulaire}
%---------------------------------------------------------------------------------------------------------------------------
\label{PgRapIntMultFubiniRect}

Soit une fonction positive
\begin{equation}
    \begin{aligned}
        f\colon \mathopen[ a , b \mathclose]\times\mathopen[ c , d \mathclose]&\to \eR^+ \\
        (x,y)&\mapsto f(x,y). 
    \end{aligned}
\end{equation}

L'intégrale de $f$ sur le rectangle $\mathopen[ a , b \mathclose]\times\mathopen[ c , d \mathclose]$ est le volume sous le graphe de la fonction. C'est à dire le volume de l'ensemble
\begin{equation}
    \{ (x,y,z)\tq (x,y)\in\mathopen[ a , b \mathclose]\times\mathopen[ c , d \mathclose], z\leq f(x,y) \}.
\end{equation}

\begin{theorem}[Théorème de Fubini]
    Soit une fonction $f\colon \eR^2\to \eR$ une fonction continue par morceaux sur $\mR=\mathopen[ a , b \mathclose]\times\mathopen[ c , d \mathclose]$. Alors
    \begin{equation}
        \int_{\mR}f(x,y)dxdy=\int_a^b\left[ \int_c^df(x,y)dy \right]dx=\int_c^d\left[ \int_a^bf(x,y)dx \right]dy.
    \end{equation}
\end{theorem}
\index{théorème!Fubini!version compacte dans \( \eR^2\)}

En pratique, nous utilisons le théorème de Fubini pour calculer les intégrales sur des rectangles.


\begin{example}
    
    Nous voudrions intégrer la fonction $f(x,y)-4+x^2+y^2$ sur le rectangle de la figure \ref{LabelFigIntRectangle}.
    \newcommand{\CaptionFigIntRectangle}{Intégration sur un rectangle.}
    \input{Fig_IntRectangle.pstricks}
    L'ensemble sur lequel nous intégrons est donné par le produit cartésien d'intervalles $E=[0,1]\times[0,2]$. Le théorème de Fubini montre que nous pouvons intégrer séparément sur l'intervalle horizontal et vertical :
    \begin{equation}
    	\int_{E=[0,1]\times[0,2]}f=\int_{[0,1]}\left( \int_{[0,2]}(4-x^2-y^2)dy \right)dx.
    \end{equation}
    Ces intégrales sont maintenant des intégrales usuelles qui s'effectuent en calculant des primitives :
    \begin{equation}
        \begin{aligned}[]
            \int_0^1\int_0^2(4-x^2-y^2)dy\,dx&=\int_0^1\left[ 4y-x^2y-\frac{ y^3 }{ 3 } \right]_0^2dx\\
            &=\int_0^1(8-2x^2-\frac{ 8 }{ 3 })dx\\
            &=\left[ \frac{ 16x }{ 3 }-\frac{ 2x^3 }{ 3 } \right]_0^1\\
            &=\frac{ 14 }{ 3 }.
        \end{aligned}
    \end{equation}
    Avec Sage, on peut faire comme ceci :

    \begin{verbatim}
----------------------------------------------------------------------
| Sage Version 4.6.1, Release Date: 2011-01-11                       |
| Type notebook() for the GUI, and license() for information.        |
----------------------------------------------------------------------
sage: f(x,y)=4-x**2-y**2                  
sage: f.integrate(y,0,2).integrate(x,0,1)
(x, y) |--> 14/3

    \end{verbatim}

\end{example}

%---------------------------------------------------------------------------------------------------------------------------
\subsection{Intégration sur un domaine non rectangulaire}
%---------------------------------------------------------------------------------------------------------------------------
\label{PgRapIntMultFubiniTri}


Nous voulons maintenant intégrer la fonction $f(x,y)=x^2+y^2$ sur le triangle de la figure \ref{LabelFigIntTriangle}.
\newcommand{\CaptionFigIntTriangle}{Intégration sur un triangle.}
\input{Fig_IntTriangle.pstricks}

Étant donné que $y$ varie de $0$ à $2$ et que \emph{pour chaque $y$}, la variable $x$ varie de $0$ à $y$, nous écrivons l'intégrale sur le triangle sous la forme :
\begin{equation}
	\int_{\text{triangle}}(x^2+y^2)dx dy=\int_0^2\left( \int_0^y(x^2+y^2)dx \right)dy.
\end{equation}

Il existe principalement deux types de domaines non rectangulaires : les «horizontales» et les «verticales», voir figure \ref{LabelFigSurfaceHorizVerti}.

\newcommand{\CaptionFigSurfaceHorizVerti}{Deux types de surfaces. Nous avons tracé un rectangle qui contient chacune des deux surfaces. L'intégrale sur un domaine sera l'intégrale sur le rectangle de la fonction qui vaut zéro en dehors du domaine.}
\input{Fig_SurfaceHorizVerti.pstricks}
%See also the subfigure \ref{LabelFigSurfaceHorizVertissLabelSubFigSurfaceHorizVerti0}
%See also the subfigure \ref{LabelFigSurfaceHorizVertissLabelSubFigSurfaceHorizVerti1}

Les surfaces horizontales sont de la forme 
\begin{equation}
    D=\{ (x,y)\tq x\in\mathopen[ a , b \mathclose],\varphi_1(x)\leq y\leq \varphi_2(x) \}
\end{equation}
où $\varphi_1$ et $\varphi_2$ sont les deux fonctions qui bornent le domaine. Le domaine $D$ est la région comprise entre les graphes de $\varphi_1$ et $\varphi_2$. Pour un tel domaine nous avons
\begin{equation}
    \iint_Df(x,y)dxdy=\int_a^bdx\int_{\varphi_1(x)}^{\varphi_2(x)}f(x,y)dy.
\end{equation}

Les surfaces verticales sont de la forme 
\begin{equation}
    D=\{ (x,y)\tq y\in\mathopen[ c , d \mathclose],\psi_1(y)\leq x\leq \psi_2(y) \}
\end{equation}
où $\varphi_1$ et $\varphi_2$ sont les deux fonctions qui bornent le domaine. Le domaine $D$ est la région comprise entre les graphes de $\varphi_1$ et $\varphi_2$. Dans ces cas nous avons
\begin{equation}
    \iint_Df=\int_c^d dy\int_{\psi_1(y)}^{\psi_2(y)} f(x,y)dx.
\end{equation}

\begin{proposition}
    L'aire du domaine $D$ vaut l'intégrable de la fonction $f(x,y)=1$ sur $D$ :
    \begin{equation}
        Aire(D)=\iint_Ddxdy.
    \end{equation}
\end{proposition}

\begin{proof}
    Supposons que le domaine soit du type «horizontal». En utilisant le théorème de Fubini avec $f(x,y)=1$ nous avons
    \begin{equation}
        \iint_Ddxdy=\int_a^b\left[ \int_{\varphi_1(x)}^{\varphi_2(x)}dy \right]dx=\int_a^b\big[ \varphi_2(x)-\varphi_1(x) \big].
    \end{equation}
    Cela représente la surface sous $\varphi_2$ moins la surface sous $\varphi_1$, et par conséquent la surface contenue entre les deux.
\end{proof}

\begin{example}
    Cherchons la surface du disque de centre $(0,0)$ et de rayon $1$ dessinée à la figure \ref{LabelFigSurfaceCercle}.
    \newcommand{\CaptionFigSurfaceCercle}{En bleu, la fonction $\sqrt{r^2-x^2}$ et en rouge, la fonction $-\sqrt{r^2-x^2}$.}
    \input{Fig_SurfaceCercle.pstricks}

    Le domaine est donné par $\varphi_1(x)\leq y\leq \varphi_2(x)$ et $x\in\mathopen[ -r ,r \mathclose]$ où $\varphi_1(x)=-\sqrt{r^2-x^2}$ et $\varphi_2(x)=\sqrt{r^2-x^2}$. L'aire est donc donnée par
    \begin{equation}
        A=\int_{-r}^r\big[ \varphi_2(x)-\varphi_1(x) \big]dx=2\int_{-r}^r\sqrt{r^2-x^2}dx=4\int_0^r\sqrt{r^2-x^2}.
    \end{equation}
    Nous effectuons le premier changement de variables $x=ru$, donc $dx=rdu$. En ce qui concerne les bornes, si $x=0$, alors $u=0$ et si $x=r$, alors $u=1$. L'intégrale à calculer devient
    \begin{equation}
        A=4\int_0^1\sqrt{r^2-r^2u^2}rdu=4r^2\int_0^1\sqrt{1-u^2}du.
    \end{equation}
    Cette dernière intégrale se calcule en posant
    \begin{equation}
        \begin{aligned}[]
            u&=\sin(t)&du&=\cos(t)dt\\
            u&=0&t&=0\\
            u&=1&t&=\pi/2.
        \end{aligned}
    \end{equation}
    Nous avons
    \begin{equation}
        A=4r^2\int_0^{\pi/2}\sqrt{1-\sin^2(t)}\cos(t)dt=4r^2\int_0^{\pi/2}\cos^2(t)dt.
    \end{equation}
    En utilisant la formule $2\cos^2(x)=1+\cos(2x)$, nous avons
    \begin{equation}
        A=4r^2\int_0^{\pi/2}\frac{ 1+\cos(2t) }{ 2 }dt=\pi r^2.
    \end{equation}
\end{example}

%---------------------------------------------------------------------------------------------------------------------------
\subsection{Changement de variables}
%---------------------------------------------------------------------------------------------------------------------------

Comme dans les intégrales simples, il y a souvent moyen de trouver un changement de variables qui simplifie les expressions.  Le domaine $E=\{ (x,y)\in\eR^2\tq x^2+y^2<1 \}$ par exemple s'écrit plus facilement $E=\{ (r,\theta)\tq r<1 \}$ en coordonnées polaires. Le passage aux coordonnées polaire permet de transformer une intégration sur un domaine rond à une intégration sur le domaine rectangulaire $\mathopen]0,2\pi\mathclose[\times\mathopen]0,1\mathclose[$. La question est évidement de savoir si nous pouvons écrire
\begin{equation}
	\int_Ef=\int_{0}^{2\pi}\int_0^1f(r\cos\theta,r\sin\theta)drd\theta.
\end{equation}
Hélas ce n'est pas le cas. Il faut tenir compte du fait que le changement de base dilate ou contracte certaines surfaces.

Soit $\varphi\colon D_1\subset\eR^2\to D_2\subset \eR^2$ une fonction bijective de classe $C^1$ dont l'inverse est également de classe $C^1$. On désigne par $x$ et $y$ ses composantes, c'est à dire que
\begin{equation}
    \varphi(u,v)=\begin{pmatrix}
        x(u,v)    \\ 
        y(u,v)    
    \end{pmatrix}
\end{equation}
avec $(u,v)\in D_1$.

\begin{theorem}     \label{ThoChamDeVarIntDDf}
    Soit une fonction continue $f\colon D_2\to \eR$. Alors
    \begin{equation}
        \iint_{\varphi(D_1)}f(x,y)dxdy=\iint_{D_1}f\big( x(u,v),y(u,v) \big)| J_{\varphi}(u,v) |dudv
    \end{equation}
    où $J_{\varphi}$ est le Jacobien de $\varphi$.
\end{theorem}
Pour rappel,
\begin{equation}
    J_{\varphi}(u,v)=\det\begin{pmatrix}
        \frac{ \partial x }{ \partial u }    &   \frac{ \partial x }{ \partial v }    \\ 
        \frac{ \partial y }{ \partial u }    &   \frac{ \partial u }{ \partial v }    
    \end{pmatrix}.
\end{equation}
Ne pas oublier de prendre la valeur absolue lorsqu'on utilise le Jacobien dans un changement de variables.

%///////////////////////////////////////////////////////////////////////////////////////////////////////////////////////////
\subsubsection{Le cas des coordonnées polaires}
%///////////////////////////////////////////////////////////////////////////////////////////////////////////////////////////

La fonction qui donne les coordonnées polaires est
\begin{equation}
    \begin{aligned}
        \varphi\colon \eR^+\times\mathopen] 0 , 2\pi \mathclose[&\to \eR^2 \\
        (r,\theta)&\mapsto\begin{pmatrix}
            r\cos(\theta)    \\ 
            r\sin(\theta)    
        \end{pmatrix}.
    \end{aligned}
\end{equation}
Son Jacobien vaut
\begin{equation}
    J_{\varphi}(r,\theta)=\det\begin{pmatrix}
        \frac{ \partial x(r,\theta) }{ \partial r }    &   \frac{ \partial x(r,\theta) }{ \partial \theta }    \\ 
        \frac{ \partial y(r,\theta) }{ \partial r }    &   \frac{ \partial y(r,\theta) }{ \partial \theta }    
    \end{pmatrix}=
    \begin{vmatrix}
        \cos(\theta)    &   -r\sin(\theta)    \\ 
        \sin(\theta)    &   r\cos(\theta)    
    \end{vmatrix}=r.
\end{equation}

\begin{example}
    Calculons la surface du disque $D$ de rayon $R$. Nous devons calculer
    \begin{equation}
        \iint_Ddxdy.
    \end{equation}
    Pour passer au polaires, nous savons que le disque est décrit par 
    \begin{equation}
        D=\{ (r,\theta)\tq 0\leq r\leq R,0\leq\theta\leq 2\pi \}.
    \end{equation}
    Nous avons donc
    \begin{equation}
        \iint_Ddxdy=\iint_{D}r\,drd\theta=\int_0^{2\pi}\int_0^Rr\,drd\theta=2\pi\int_0^Rr\,dr=\pi R^2.
    \end{equation}
\end{example}

\begin{example}     \label{ExpmfDtAtV}
    Montrons comment intégrer la fonction $f(x,y)=\sqrt{1-x^2-y^2}$ sur le domaine délimité par la droite $y=x$ et le cercle $x^2+y^2=y$, représenté sur la figure \ref{LabelFigIntBoutCercle}. Pour trouver le centre et le rayon du cercle $x^2+y^2=y$, nous commençons par écrire $x^2+y^2-y=0$, et ensuite nous reformons le carré : $y^2-y=(y-\frac{ 1 }{2})^2-\frac{1}{ 4 }$.
    \newcommand{\CaptionFigIntBoutCercle}{Passage en polaire pour intégrer sur un morceau de cercle.}
    \input{Fig_IntBoutCercle.pstricks}
    %TODO : il faudra dupliquer cette figure parce qu'elle est utilisée autre part aussi.
    Le passage en polaire transforme les équations du bord du domaine en
    \begin{equation}
        \begin{aligned}[]
            \cos(\theta)&=\sin(\theta)\\
            r^2&=r\sin(\theta).
        \end{aligned}
    \end{equation}
    L'angle $\theta$ parcours donc $\mathopen] 0 , \pi/4 \mathclose[$, et le rayon, pour chacun de ces $\theta$ parcours $\mathopen] 0 , \sin(\theta) \mathclose[$. La fonction à intégrer se note maintenant $f(r,\theta)=\sqrt{1-r^2}$. Donc l'intégrale à calculer est
    \begin{equation}		\label{PgOMRapIntMultFubiniBoutCercle}
        \int_{0}^{\pi/4}\left( \int_0^{\sin(\theta)}\sqrt{1-r^2}r\,rd \right).
    \end{equation}
    Remarquez la présence d'un $r$ supplémentaire pour le jacobien.

    Notez que les coordonnées du point $P$ sont $(1,1)$.
\end{example}

En pratique, lors du passage en coordonnées polaires, le «$dxdy$» devient «$r\,drd\theta$».

%///////////////////////////////////////////////////////////////////////////////////////////////////////////////////////////
\subsubsection{Les coordonnées cylindriques}
%///////////////////////////////////////////////////////////////////////////////////////////////////////////////////////////

En ce qui concerne les coordonnées cylindriques, le Jacobien est donné par
\begin{equation}
    J(r,\theta,z)=\begin{vmatrix}
        \frac{ \partial x }{ \partial r }    &   \frac{ \partial x }{ \partial \theta }    &   \frac{ \partial x }{ \partial z }    \\
        \frac{ \partial y }{ \partial r }    &   \frac{ \partial y }{ \partial \theta }    &   \frac{ \partial y }{ \partial z }    \\
        \frac{ \partial z }{ \partial r }    &   \frac{ \partial z }{ \partial \theta }    &   \frac{ \partial z }{ \partial z }    
    \end{vmatrix}=
    \begin{vmatrix}
        \cos\theta    &   -r\sin\theta    &   0    \\
        \sin\theta    &   r\cos\theta    &   0    \\
        0    &   0    &   1
    \end{vmatrix}=r.
\end{equation}
Nous avons donc $dx\,dy\,dz=r\,dr\,d\theta\,dz$.

%///////////////////////////////////////////////////////////////////////////////////////////////////////////////////////////
\subsubsection{Coordonnées sphériques}
%///////////////////////////////////////////////////////////////////////////////////////////////////////////////////////////

Le calcul est un peu plus long :
\begin{equation}
    \begin{aligned}[]
        J(\rho,\theta,\varphi)&=\begin{vmatrix}
            \frac{ \partial x }{ \partial \rho }    &   \frac{ \partial x }{ \partial \theta }    &   \frac{ \partial x }{ \partial \varphi }    \\
            \frac{ \partial y }{ \partial \rho }    &   \frac{ \partial y }{ \partial \theta }    &   \frac{ \partial y }{ \partial \varphi }    \\
            \frac{ \partial z }{ \partial \rho }    &   \frac{ \partial z }{ \partial \theta }    &   \frac{ \partial z }{ \partial \varphi }    
        \end{vmatrix}\\ 
        &=
        \begin{vmatrix}
            \sin\theta\cos\varphi    &   \rho\cos\theta\cos\varphi    &   -\rho\sin\theta\sin\varphi    \\
            \sin\theta\sin\varphi    &   \rho\cos\theta\sin\varphi    &   -\rho\sin\theta\cos\varphi    \\
            \cos\theta               &   -\rho\sin\theta              &   0
        \end{vmatrix}\\
        &=\rho^2\sin\theta.
    \end{aligned}
\end{equation}
Donc 
\begin{equation}
    dx\,dy\,dz=\rho^2\sin(\theta)\,d\rho\,d\theta\,d\varphi.
\end{equation}

%///////////////////////////////////////////////////////////////////////////////////////////////////////////////////////////
\subsubsection{Un autre système utile}
%///////////////////////////////////////////////////////////////////////////////////////////////////////////////////////////

Un changement de variables que l'on voit assez souvent est
\begin{subequations}
    \begin{numcases}{}
        u=x+y\\
        v=x-y.
    \end{numcases}
\end{subequations}
Afin de calculer son jacobien, il faut d'abord exprimer $x$ et $y$ en fonctions de $u$ et $v$ :
\begin{subequations}
    \begin{numcases}{}
        x=(u+v)/2\\
        y=(u-v)/2.
    \end{numcases}
\end{subequations}
La matrice jacobienne est
\begin{equation}
    \begin{pmatrix}
        \frac{ \partial x }{ \partial u }    &   \frac{ \partial x }{ \partial v }    \\ 
        \frac{ \partial y }{ \partial u }    &   \frac{ \partial y }{ \partial v }    
    \end{pmatrix}=
    \begin{pmatrix}
        \frac{ 1 }{2}    &   \frac{ 1 }{2}    \\ 
        \frac{ 1 }{2}    &   -\frac{ 1 }{2}    
    \end{pmatrix}.
\end{equation}
Le déterminant vaut $-\frac{1}{ 2 }$. Nous avons donc
\begin{equation}
    dxdy=\frac{ 1 }{2}dudv.
\end{equation}
Nous insistons sur le fait que c'est $\frac{ 1 }{2}$ et non $-\frac{ 1 }{2}$ qui intervient parce que que la formule du changement de variable demande d'introduire la \emph{valeur absolue} du jacobien.

\begin{example}
    Calculer l'intégrale de la fonction $f(x,y)=x^2-y^2$ sur le domaine représenté sur la figure \ref{LabelFigExPolygone}.
    \newcommand{\CaptionFigExPolygone}{Un domaine qui s'écrit étonnament bien avec un bon changement de coordonnées.}
    \input{Fig_ExPolygone.pstricks}

    Les droites qui délimitent le domaine d'intégration sont
    \begin{equation}
        \begin{aligned}[]
            y&=-x+2\\
            y&=x-2\\
            y&=x\\
            y&=-x
        \end{aligned}
    \end{equation}
    Le domaine est donc donné par les équations
    \begin{subequations}
        \begin{numcases}{}
            y+x<2\\
            y-x>-2\\
            y-x<0 \\
            y+x>0.
        \end{numcases}
    \end{subequations}
    En utilisant le changement de variables $u=x+y$, $v=x-y$ nous trouvons le domaine $0<u<2$, $0<v<2$. En ce qui concerne la fonction, $f(x,y)=(x+y)(x-y)$ et par conséquent
    \begin{equation}
        f(u,v)=uv.
    \end{equation}
    L'intégrale à calculer est simplement
    \begin{equation}
        \int_0^2\int_0^2 uv\,dudv=\int_0^2 u\,du\left[ \frac{ v^2 }{ 2 } \right]_0^2=2\int_0^2u\,du=4.
    \end{equation}
    
\end{example}




%+++++++++++++++++++++++++++++++++++++++++++++++++++++++++++++++++++++++++++++++++++++++++++++++++++++++++++++++++++++++++++
\section{Les intégrales triples}
%+++++++++++++++++++++++++++++++++++++++++++++++++++++++++++++++++++++++++++++++++++++++++++++++++++++++++++++++++++++++++++

Les intégrales triples fonctionnent exactement de la même manière que les intégrales doubles. Il s'agit de déterminer sur quelle domaine les variables varient et d'intégrer successivement par rapport à $x$, $y$ et $z$. Il est autorisé de permuter l'ordre d'intégration\footnote{En toute rigueur, cela n'est pas vrai, mais nous ne considérons seulement des cas où cela est autorisé.} à condition d'adapter les domaines d'intégration. 

\begin{example}
    Soit le domaine parallélépipédique rectangle 
    \begin{equation}
        R=\mathopen[ 0 , 1 \mathclose]\times \mathopen[ 1 , 2 \mathclose]\times\mathopen[ 0 , 4 \mathclose].
    \end{equation}
    Pour intégrer la fonction $f(x,y,z)=x^2y\sin(z)$ sur $R$, nous faisons
    \begin{equation}
        \begin{aligned}[]
            I&=\iiint_Rx^2y\sin(z)\,dxdydz\\
            &=\int_0^1dx\int_1^2dy\int_0^4x^2y\sin(z)dz\\
            &=\int_0^1dx\int_1^2 x^2y(1-\cos(4))dy\\
            &=\int_0^1\frac{ 3 }{2}(1-\cos(4))x^2dx\\
            &=\frac{ 1 }{2}\big( 1-\cos(4) \big).
        \end{aligned}
    \end{equation}
    
    \begin{verbatim}
----------------------------------------------------------------------
| Sage Version 4.6.1, Release Date: 2011-01-11                       |
| Type notebook() for the GUI, and license() for information.        |
----------------------------------------------------------------------
sage: f(x,y,z)=x**2*y*sin(z)                                                                                                                                                            
sage: f.integrate(x,0,1).integrate(y,1,2).integrate(z,0,4)                                                                                                                               
(x, y, z) |--> -1/2*cos(4) + 1/2
    \end{verbatim}
\end{example}


\begin{example}
    Soit $D$ la région délimitée par le plan $x=0$, $y=0$, $z=2$ et la surface d'équation
    \begin{equation}
        z=x^2+y^2.
    \end{equation}
    Cherchons à calculer $\iiint_Dx\,dx\,dy\,dz$. Ici, un dessin indique que le volume considéré est $z\geq x^2+y^2$. Il y a plusieurs façon de décrire cet ensemble. Une est celle-ci :
    \begin{equation}
        \begin{aligned}[]
            z&\colon 0\to 2\\
            x&\colon 0\to \sqrt{z}\\
            y&\colon 0\to \sqrt{z-x^2}.
        \end{aligned}
    \end{equation}
    Cela revient à dire que $z$ peut prendre toutes les valeurs de $0$ à $2$, puis que pour chaque $z$, la variable $x$ peut aller de $0$ à $\sqrt{z}$, mais que pour chaque $z$ et $x$ fixés, la variable $y$ ne peut pas dépasser $\sqrt{z-x^2}$. En suivant cette méthode, l'intégrale à calculer est
    \begin{equation}
        \int_0^2dz\int_0^{\sqrt{z}}dx\int_0^{\sqrt{z-x^2}}f(x,y,z)dy.
    \end{equation}
    \begin{verbatim}
----------------------------------------------------------------------
| Sage Version 4.6.1, Release Date: 2011-01-11                       |
| Type notebook() for the GUI, and license() for information.        |
----------------------------------------------------------------------
sage: f(x,y,z)=x
sage: assume(z>0)
sage: assume(z-x**2>0)
sage: f.integrate(y,0,sqrt(z-x**2)).integrate(x,0,sqrt(z)).integrate(z,0,2)
(x, y, z) |--> 8/15*sqrt(2)
    \end{verbatim}
    Notez qu'il a fallu aider Sage en lui indiquant que $z>0$ et $z-x^2>0$.

    Une autre paramétrisation serait
    \begin{equation}
        \begin{aligned}[]
            x&\colon 0\to \sqrt{2}\\
            y&\colon 0\to \sqrt{2-x^2}\\
            z&\colon x^2+y^2\to 2.
        \end{aligned}
    \end{equation}
    \begin{verbatim}
----------------------------------------------------------------------
| Sage Version 4.6.1, Release Date: 2011-01-11                       |
| Type notebook() for the GUI, and license() for information.        |
----------------------------------------------------------------------
sage: f(x,y,z)=x
sage: assume(2-x**2>0)
sage: f.integrate(y,0,sqrt(z-x**2)).integrate(x,0,sqrt(z)).integrate(z,0,2)
(x, y, z) |--> 8/15*sqrt(2)
    \end{verbatim}

    Écrivons le détail de cette dernière intégrale :
    \begin{equation}
        \begin{aligned}[]
            I&=\int_0^{\sqrt{2}}dx\int_0^{\sqrt{2-x^2}}dy\int_{x^2+y^2}^2xdz\\
            &=\int_0^{\sqrt{2}}dx\int_0^{\sqrt{2-x^2}}x(2-x^2-y^2)dy\\
            &=\int_0^{\sqrt{2}}dx\,x\left[ (2-x^2)y-\frac{ y^3 }{ 3 } \right]_0^{\sqrt{2-x^2}}\\
            &=\int_0^{\sqrt{2}}\frac{ 2 }{ 3 }x(2-x^2)^{3/2}dx.
        \end{aligned}
    \end{equation}
    Ici nous effectuons le changement de variable $u=x^2$, $du=2xdx$. Ne pas oublier de changer les bornes de l'intégrale :
    \begin{equation}
        I=\frac{1}{ 3 }\int_0^2(2-u)^{3/2}du.
    \end{equation}
    Le changement de variable $t=2-u$, $dt=-du$ fait venir (attention aux bornes !!)
    \begin{equation}
        I=-\frac{1}{ 3 }\int_2^0t^{3/2}dt=\frac{1}{ 3 }\left[ \frac{ t^{5/2} }{ 5/2 } \right]_0^2=\frac{ 8 }{ 15 }\sqrt{2}.
    \end{equation}
       
\end{example}

%---------------------------------------------------------------------------------------------------------------------------
\subsection{Volume}
%---------------------------------------------------------------------------------------------------------------------------

Parmi le nombreuses interprétations géométriques de l'intégrale triple, notons celle-ci :
\begin{proposition}
    Soit $D\subset \eR^3$. Le volume de $D$ est donné par 
    \begin{equation}
        Vol(D)=\iiint_D dxdydz.
    \end{equation}
    C'est à dire l'intégrale de la fonction $f(x,y,z)=1$ sur $D$.
\end{proposition}
Suivant les points de vue, cette proposition peut être considérée comme une \emph{définition}] du volume.

\begin{example}     \label{ExemVolSphCart}
    Calculons le volume de la sphère de rayon $R$. Le domaine de variation des variables $x$, $y$ et $z$ pour la sphère est
    \begin{equation}
        \begin{aligned}[]
            x&\colon -R\to R\\
            y&\colon -\sqrt{R^2-x^2}\to \sqrt{R^2-x^2}\\
            z&\colon -\sqrt{R^2-x^2-y^2}\to \sqrt{R^2-x^2-y^2}.
        \end{aligned}
    \end{equation}
    Par conséquent nous devons calculer l'intégrale
    \begin{equation}
        V=\int_{-R}^Rdx\int_{-\sqrt{R^2-x^2}}^{\sqrt{R^2-x^2}}dy\int_{-\sqrt{R^2-x^2-y^2}}^{\sqrt{R^2-x^2-y^2}}dz.
    \end{equation}
    La première intégrale est simple :
    \begin{equation}
        V=2\int_{-R}^Rdx\int_{-\sqrt{R^2-x^2}}^{\sqrt{R^2-x^2}}\sqrt{R^2-x^2-y^2}dy.
    \end{equation}
    Afin de simplifier la notation, nous posons $a=R^2-x^2$. Ceci n'est pas un changement de variables : juste une notation provisoire le temps d'effectuer l'intégration sur $y$. Étudions donc
    \begin{equation}
        I=\int_{-\sqrt{a}}^{\sqrt{a}}\sqrt{a-y^2}dy,
    \end{equation}
    ce qui est la surface du demi-disque de rayon $\sqrt{a}$. Nous avons donc
    \begin{equation}
        I=\frac{ \pi a }{ 2 }=\frac{ \pi }{ 2 }(R^2-x^2),
    \end{equation}
    et
    \begin{equation}
        V=2\int_{-R}^R\frac{ \pi }{ 2 }(R^2-x^2)dx=\pi\left[ R^2x-\frac{ x^3 }{ 3 } \right]_{-R}^R=\frac{ 4 }{ 3 }\pi R^3.
    \end{equation}    
\end{example}

\begin{example}
    Nous pouvons calculer le volume de la sphère en utilisant les coordonnées sphériques. Les bornes des variables pour la sphère de rayon $R$ sont
    \begin{equation}
        \begin{aligned}[]
            \rho&\colon 0\to R\\
            \theta&\colon 0\to \pi\\
            \varphi&\colon 0\to 2\pi.
        \end{aligned}
    \end{equation}
    En n'oubliant pas le jacobien $\rho^2\sin(\theta)$, l'intégrale à calculer est
    \begin{equation}
        V=\int_0^Rd\rho\int_0^{2\pi}d\varphi\int_0^{\pi}\rho^2\sin(\theta)d\theta
    \end{equation}
    L'intégrale sur $\varphi$ fait juste une multiplication par $2\pi$. Celle sur $\rho$ vaut
    \begin{equation}
        \int_0^R\rho^2d\rho=\frac{ R^3 }{ 3 }.
    \end{equation}
    L'intégrale sur $\theta$ donne
    \begin{equation}
        \int_0^{\pi}\sin(\theta)d\theta=[-\cos(\theta)]_{0}^{\pi}=2.
    \end{equation}
    Le tout fait par conséquent
    \begin{equation}
        V=\frac{ 4 }{ 3 }\pi R^3.
    \end{equation}
    Sans contestes, le passage aux coordonnées sphériques a considérablement simplifié le calcul par rapport à celui de l'exemple \ref{ExemVolSphCart}.
\end{example}


%+++++++++++++++++++++++++++++++++++++++++++++++++++++++++++++++++++++++++++++++++++++++++++++++++++++++++++++++++++++++++++
\section{Un petit peu plus formel}
%+++++++++++++++++++++++++++++++++++++++++++++++++++++++++++++++++++++++++++++++++++++++++++++++++++++++++++++++++++++++++++

%---------------------------------------------------------------------------------------------------------------------------
\subsection{Intégration sur un domaine non rectangulaire}
%---------------------------------------------------------------------------------------------------------------------------

\newcommand{\CaptionFigIntEcourbe}{Intégrer sur des domaines plus complexes.}
\input{Fig_IntEcourbe.pstricks}

La méthode de Fubini ne fonctionne plus sur un domaine non rectangulaire tel que celui de la figure \ref{LabelFigIntEcourbe}. Nous allons donc utiliser une astuce. Considérons le domaine \begin{equation}
	E=\{ (x,y)\in\eR^2\tq a<x<b\text{ et } \alpha(x)<y<\beta(x) \}
\end{equation}
représenté sur la figure \ref{LabelFigIntEcourbe}. Nous considérons la fonction
\begin{equation}
	\tilde f(x,y)=\begin{cases}
	f(x,y)	&	\text{si $(x,y)\in E$}\\
	0	&	 \text{sinon.}
\end{cases}
\end{equation}
Ensuite intégrons $\tilde f$ sur un rectangle qui englobe la surface à intégrer à l'aide de Fubini. Étant donné que $\tilde f=f$ sur la surface et que $\tilde f$ est nulle en dehors, nous avons
\begin{equation}
	\int_Ef=\int_E\tilde f=\int_{\text{rectangle}}\tilde f=\int_a^b\left( \int_{\alpha(x)}^{\beta(x)}f(x,y)dy \right)dx.
\end{equation}

Dans le cas de l'intégrale de $f(x,y)=x^2+y^2$ sur le triangle de la figure \ref{LabelFigIntTriangle}, nous avons
\begin{equation}
	\int_{\text{triangle}}(x^2+y^2)dx dy=\int_0^2\left( \int_0^y(x^2+y^2)dx \right)dy.
\end{equation}

\begin{remark}
    Le nombre $\iint_{D}f(x,y)dxdy$ ne dépend pas du choix du rectangle englobant $D$.
\end{remark}

En pratique, nous calculons l'intégrale en utilisant une extension du théorème de Fubini :
\begin{theorem}
    Soit $f\colon D\subset\eR^2\to \eR$ une fonction continue où $D$ est un domaine de type vertical ou horizontal.
    \begin{enumerate}
        \item
            Si $D$ est vertical, alors
            \begin{equation}
                \iint_Df=\int_a^b\left[ \int_{\varphi_1(x)}^{\varphi_2(x)}f(x,y)dy \right]dx.
            \end{equation}
        \item
            Si $D$ est horizontal, alors
            \begin{equation}
                \iint_Df=\int_c^d\left[ \int_{\psi_1(y)}^{\psi_2(y)}f(x,y)dx \right]dy.
            \end{equation}
    \end{enumerate}
    
\end{theorem}

%---------------------------------------------------------------------------------------------------------------------------
\subsection{Changement de variables}
%---------------------------------------------------------------------------------------------------------------------------


Le théorème du changement de variable est le suivant.
\begin{theorem}
Soit $g\colon A\to B$ un difféomorphisme. Soient $F\subset B$ un ensemble mesurable et borné et $f\colon F\to \eR$ une fonction bornée et intégrable. Supposons que $g^{-1}(F)$ soit borné et que $Jg$ soit borné sur $g^{-1}(F)$. Alors
\begin{equation}
    \int_Ff(x)dy=\int_{g^{-1}(F)}f\big( g(x) \big)| Jg(x) |dx
\end{equation}
\end{theorem}
Pour rappel, $Jg$ est le déterminant de la matrice \href{http://fr.wikipedia.org/wiki/Matrice_jacobienne}{jacobienne} (aucun lien de \href{http://fr.wikipedia.org/wiki/Jacob}{parenté}) donnée par
\begin{equation}
	Jg=\det\begin{pmatrix}
	\partial_xg_1	&	\partial_yg_1	\\ 
	\partial_xg_2	&	\partial_tg_2	
\end{pmatrix}.
\end{equation}
Un \defe{difféomorphisme}{difféomorphisme} est une application $g\colon A\to B$ telle que $g$ et $g^{-1}\colon B\to A$ soient de classe $C^1$.

%///////////////////////////////////////////////////////////////////////////////////////////////////////////////////////////
					\subsubsection{Coordonnées polaires}
%///////////////////////////////////////////////////////////////////////////////////////////////////////////////////////////

Les coordonnées polaires sont données par le difféomorphisme
\begin{equation}
	\begin{aligned}
		g\colon \mathopen]0,\infty\mathclose[\times\mathopen]0,2\pi\mathclose[ &\to\eR^2\setminus D\\
		(r,\theta)&\mapsto \big( r\cos(\theta),r\sin(\theta) \big)
	\end{aligned}
\end{equation}
où $D$ est la demi droite $y=0$, $x\geq 0$. Le fait que les coordonnées polaires ne soient pas un difféomorphisme sur tout $\eR^2$ n'est pas un problème pour l'intégration parce que le manque de difféomorphisme est de mesure nulle dans $\eR^2$. Le jacobien est donné par
\begin{equation}
	Jg=\det\begin{pmatrix}
	\partial_rx	&	\partial_{\theta}x	\\ 
	\partial_ry	&	\partial_{\theta}y
\end{pmatrix}=\det\begin{pmatrix}
	\cos(\theta)	&	-r\sin(\theta)	\\ 
	\sin(\theta)	&	r\cos(\theta)	
\end{pmatrix}=r.
\end{equation}

%///////////////////////////////////////////////////////////////////////////////////////////////////////////////////////////
					\subsubsection{Coordonnées sphériques}
%///////////////////////////////////////////////////////////////////////////////////////////////////////////////////////////
\label{SubSubCoordSpJxhMwm}

Les coordonnées sphériques sont données par
\begin{equation}		\label{OMEqChmVarSpherique}
	\left\{
\begin{array}{lllll}
x=r\cos\theta\sin\varphi	&			&r\in\mathopen] 0 , \infty \mathclose[\\
y=r\sin\theta\sin\varphi	&	\text{avec}	&\theta\in\mathopen] 0 , 2\pi \mathclose[\\
z=r\cos\varphi			&			&\phi\in\mathopen] 0 , \pi \mathclose[.
\end{array}
\right.
\end{equation}
Le jacobien associé est $Jg(r,\theta,\varphi)=-r^2\sin\varphi$. Rappelons que ce qui rentre dans l'intégrale est la valeur absolue du jacobien.

Si nous voulons calculer le volume de la sphère de rayon $R$, nous écrivons donc
\begin{equation}
	\int_0^Rdr\int_{0}^{2\pi}d\theta\int_0^{\pi}r^2 \sin(\phi)d\phi=4\pi R=\frac{ 4 }{ 3 }\pi R^3.
\end{equation}
Ici, la valeur absolue n'est pas importante parce que lorsque $\phi\in\mathopen] 0,\pi ,  \mathclose[$, le sinus de $\phi$ est positif.

Des petits malins pourraient remarquer que le changement de variable \eqref{OMEqChmVarSpherique} est encore une paramétrisation de $\eR^3$ si on intervertit le domaine des angles : 
\begin{equation}
	\begin{aligned}[]
		\theta&\colon 0 \to \pi\\
		\phi	&\colon 0\to 2\pi,
	\end{aligned}
\end{equation}
alors nous paramétrons encore parfaitement bien la sphère, mais hélas
\begin{equation}		\label{EqOMVolumeIncorrectSphere}
	\int_0^Rdr\int_{0}^{\pi}d\theta\int_0^{2\pi}r^2 \sin(\phi)d\phi=0.
\end{equation}
Pourquoi ces «nouvelles» coordonnées sphériques sont-elles mauvaises ? Il y a que quand l'angle $\phi$ parcours $\mathopen] 0 , 2\pi \mathclose[$, son sinus n'est plus toujours positif, donc la \emph{valeur absolue} du jacobien n'est plus $r^2\sin(\phi)$, mais $r^2\sin(\phi)$ pour les $\phi$ entre $0$ et $\pi$, puis $-r^2\sin(\phi)$ pour $\phi$ entre $\pi$ et $2\pi$. Donc l'intégrale \eqref{EqOMVolumeIncorrectSphere} n'est pas correcte. Il faut la remplacer par
\begin{equation}
	\int_0^Rdr\int_{0}^{\pi}d\theta\int_0^{\pi}r^2 \sin(\phi)d\phi- \int_0^Rdr\int_{0}^{\pi}d\theta\int_{\pi}^{2\pi}r^2 \sin(\phi)d\phi = \frac{ 4 }{ 3 }\pi R^3
\end{equation}

%+++++++++++++++++++++++++++++++++++++++++++++++++++++++++++++++++++++++++++++++++++++++++++++++++++++++++++++++++++++++++++
\section{Intégrales curvilignes}
%+++++++++++++++++++++++++++++++++++++++++++++++++++++++++++++++++++++++++++++++++++++++++++++++++++++++++++++++++++++++++++

Nous savons maintenant commet intégrer des fonctions sur des volumes dans $\eR^3$ et sur des surfaces dans $\eR^2$. Nous savons également intégrer des champs de vecteurs sur des lignes dans $\eR^3$. Nous allons maintenant voir comment on intègre des fonctions sur des lignes et surfaces dans $\eR^3$.

Soit un chemin $\sigma\colon \mathopen[ a , b \mathclose]\to \eR^3$, et une fonction $f\colon \eR^3 \to \eR$ définie au moins sur l'image de $\sigma$. Nous définissons l'intégrale de $f$ sur $\sigma$ par
\begin{equation}
    \int_{\sigma}f\,ds=\int_a^bf\big( \sigma(t) \big)\| \sigma'(t) \|dt.
\end{equation}

\begin{remark}
    Nous désignons par «$\sigma$» autant la fonction que son image dans $\eR^3$.
\end{remark}


\begin{example}
    Soit l'hélice
    \begin{equation}
        \begin{aligned}
            \sigma\colon \mathopen[ 0 , 2\pi \mathclose]&\to \eR^3 \\
            t&\mapsto \begin{pmatrix}
                \cos(t)    \\ 
                \sin(t)    \\ 
                t    
            \end{pmatrix},
        \end{aligned}
    \end{equation}
    et la fonction $f(x,y,z)=x^2+y^2+z^2$. L'intégrale de $f$ sur $\sigma$ est
    \begin{equation}
        \begin{aligned}[]
            \int_{\sigma}f&=\int_0^{2\pi}(\cos^2t+\sin^2t+t^2)\| \sigma'(t) \|dt\\
            &=\int_0^{2\pi}(1+t^2)\sqrt{2}dt\\
            &=\sqrt{2}\left[ t+\frac{ t^3 }{ 3 } \right]_0^{2\pi}\\
            &=\sqrt{2}\left( 2\pi+\frac{ 8\pi^3 }{ 8 } \right).
        \end{aligned}
    \end{equation}
    
\end{example}

\begin{remark}
    Si $f=1$, alors nous tombons sur
    \begin{equation}
        \int_{\sigma}ds=\int_a^b\| \sigma'(t) \|dt,
    \end{equation}
    qui n'est autre que la longueur de la courbe. Encore une fois, l'intégrale de la fonction $1$ donne la «mesure» de l'ensemble.
\end{remark}

\begin{proposition}
    La valeur de l'intégrale de $f$ sur $\sigma$ ne dépend pas du paramétrage (équivalent ou pas) choisi.
\end{proposition}

\begin{proof}
    Soit $\varphi\colon \mathopen[ c , d \mathclose]\to \mathopen[ a , b \mathclose]$, une reparamétrisation de classe $C^1$, strictement monotone et $\gamma(s)=\sigma\big( \varphi(s) \big)$ avec $s\in\mathopen[ c , d \mathclose]$. En supposant que $\varphi'(s)\geq 0$, nous avons
    \begin{equation}
        \begin{aligned}[]
            I=\int_{\gamma}f&=\int_c^df\big( \gamma(s) \big)\| \gamma'(s) \|ds\\
            &=\int_c^df\Big( \sigma\big( \varphi(s) \big) \Big)\| \sigma'\big( \varphi(s) \big) \| |\varphi'(s) |ds.
        \end{aligned}
    \end{equation}
    Pour cette intégrale, nous posons $t=\varphi(s)$, et par conséquent $dt=\varphi'(s)ds$. Étant donné que $\varphi'(s)\geq 0$, nous pouvons supprimer les valeurs absolues, et obtenir
    \begin{equation}
        \begin{aligned}[]
            I&=\int_{\varphi(c)}^{\varphi(d)}f\big( \sigma(t) \big)\| \sigma'(t) \|dt\\
            &=\int_a^bf\big( \sigma(t) \big)\| \sigma'(t) \|dt\\
            &=\int_{\sigma}f.
        \end{aligned}
    \end{equation}

    Essayez de fait le cas $\varphi'(s)\leq 0$. 
\end{proof}

\begin{remark}
    Si $\sigma'(t)\neq 0$, nous pouvons considérer le vecteur unitaire tangent à la courbe :
    \begin{equation}
        T(t)=\frac{ \sigma'(t) }{ \| \sigma'(t) \| }.
    \end{equation}
    Si $F$ est un champ de vecteurs sur $\eR^3$, la circulation de $F$ le long de $\sigma$ sera donnée par 
    \begin{equation}
        \int_{\sigma}F\cdot ds=\int_a^b F\big( \sigma(t) \big)\cdot \sigma'(t)dt=\int_{a}^bF\big( \sigma(t) \big)\cdot\frac{ \sigma'(t) }{ \| \sigma'(t) \| }dt=\int_{\sigma} F\cdot T ds
    \end{equation}
    où dans la dernière expression, $F\cdot T$ est vu comme fonction $(x,y,z)\mapsto F(x,y,z)\cdot T(x,y,z)$. L'intégrale d'un champ de vecteurs sur une courbe n'est donc rien d'autre que l'intégrale de la composante tangentielle du champ de vecteurs.
\end{remark}

%+++++++++++++++++++++++++++++++++++++++++++++++++++++++++++++++++++++++++++++++++++++++++++++++++++++++++++++++++++++++++++
\section{Surfaces paramétrées}
%+++++++++++++++++++++++++++++++++++++++++++++++++++++++++++++++++++++++++++++++++++++++++++++++++++++++++++++++++++++++++++

De la même façon qu'un chemin dans $\eR^3$ est décrit comme une application $\sigma\colon \eR\to \eR^3$, une surface dans $\eR^3$ sera vue comme une application $\varphi\colon \eR^2\to \eR^3$. Une \defe{surface paramétrée}{surface paramétrée} dans $\eR^3$ est une application
\begin{equation}
    \begin{aligned}
        \varphi\colon D\subset\eR^2&\to \eR^3 \\
        (u,v)&\mapsto \varphi(u,v)=\begin{pmatrix}
            x(u,v)    \\ 
            y(u,v)    \\ 
            z(u,z)    
        \end{pmatrix}.
    \end{aligned}
\end{equation}
Nous allons parler de la «surface $\varphi$» pour désigner l'image de $\varphi$ dans $\eR^3$.

Si on fixe le paramètre $u-u_0$, alors l'application
\begin{equation}
    v\mapsto\varphi(u_0,v)
\end{equation}
est un chemin dans la surface. Un vecteur tangent à ce chemin sera tangent à la courbe :
\begin{equation}
    \frac{ \partial \varphi }{ \partial v }(u_0,v_0)=
    \begin{pmatrix}
        \frac{ \partial x }{ \partial v }(u_0,v_O)    \\ 
        \frac{ \partial y }{ \partial v }(u_0,v_O)    \\ 
        \frac{ \partial z }{ \partial v }(u_0,v_O)    
    \end{pmatrix}.
\end{equation}
De même, en fixant $v_0$, on considère le chemin
\begin{equation}
    u\mapsto\varphi(u,v_0).
\end{equation}
Le vecteur tangent à ce chemin est égalent tangent à la surface :
\begin{equation}
    \frac{ \partial \varphi }{ \partial u }=
    \begin{pmatrix}
        \frac{ \partial x }{ \partial u }(u_0,v_O)    \\ 
        \frac{ \partial y }{ \partial u }(u_0,v_O)    \\ 
        \frac{ \partial z }{ \partial u }(u_0,v_O)    
    \end{pmatrix}
\end{equation}

\begin{definition}      \label{DefSurfReguliere}
    Nous disons que la surface est \defe{régulière}{régulière!surface} si les vecteurs $\partial_u\varphi(u_0,v_0)$ et $\partial_v\varphi(u_0,v_0)$ sont non nuls et non colinéaires. 
\end{definition}
Si la surface est régulière, les vecteurs tangents à la paramétrisation forment le plan tangent à la surface au point $\varphi(u_0,v_0)$.

Un vecteur orthogonal à la surface (et donc au plan tangent) est donc donné par le produit vectoriel :
\begin{equation}
    n(u_0,v_0)=\frac{ \partial \varphi }{ \partial u }(u_0,v_0)  \times \frac{ \partial \varphi }{ \partial v }(u_0,v_0).
\end{equation}
L'équation du plan tangent est alors obtenue par
\begin{equation}        \label{EqPlanTgSurfaceParm}
    \begin{pmatrix}
        x-x_0    \\ 
        y-y_0    \\ 
        z-z_0    
    \end{pmatrix}\cdot n(u_0,v_0)=0
\end{equation}
où $x_0=x(u_0,v_0)$, $y_0=y(u_0,v_0)$, $z_0=z(u_0,v_0)$.

%---------------------------------------------------------------------------------------------------------------------------
\subsection{Graphe d'une fonction}
%---------------------------------------------------------------------------------------------------------------------------

Soit la fonction $f\colon D\subset\eR^2\to \eR$. Le graphe de $f$ est l'ensemble des points de la forme
\begin{equation}
    \big( x,y,f(x,y) \big)
\end{equation}
tels que $(x,y)\in D$. Cela est une surface paramétrée par
\begin{equation}
    \begin{aligned}
        \varphi\colon D&\to \eR^3 \\
        (x,y)&\mapsto \begin{pmatrix}
            x    \\ 
            y    \\ 
            f(x,y)    
        \end{pmatrix}.
    \end{aligned}
\end{equation}
Les vecteurs tangents sont
\begin{equation}
    \begin{aligned}[]
        \frac{ \partial \varphi }{ \partial x }&=\begin{pmatrix}
            1    \\ 
            0    \\ 
            \frac{ \partial f }{ \partial x }    
        \end{pmatrix},
        &\frac{ \partial \varphi }{ \partial y }&=\begin{pmatrix}
            0    \\ 
            1    \\ 
            \frac{ \partial \varphi }{ \partial y }    
        \end{pmatrix}.
    \end{aligned}
\end{equation}
La surface est donc partout régulière parce que ces deux vecteurs ne sont jamais nuls ou colinéaires. Un vecteur normal à cette surface au point $(x_0,y_0,f(x_0,y_0))$ est donné par le produit vectoriel
\begin{equation}
    n=\begin{vmatrix}
         e_x   &   e_y    &   e_z    \\
        1    &   0    &   \partial_xf(x_0,y_0)    \\
        0    & 1    &   \partial_yf(x_0,y_0)    
    \end{vmatrix}
    =-\frac{ \partial f }{ \partial x }(x_0,y_0)e_x-\frac{ \partial f }{ \partial y }(x_0,y_0)e_y+e_z.
\end{equation}
En suivant l'équation \eqref{EqPlanTgSurfaceParm}, nous avons l'équation suivante pour le plan :
\begin{equation}      
    \begin{pmatrix}
        x-x_0    \\ 
        y-y_0    \\ 
        z-z_0    
    \end{pmatrix}\cdot
    \begin{pmatrix}
        -\frac{ \partial f }{ \partial x }(x_0,y_0)    \\ 
        -\frac{ \partial f }{ \partial y }(x_0,y_0)    \\ 
        1
    \end{pmatrix}=0,
\end{equation}
c'est à dire
\begin{equation}
    -(x-x_0)\frac{ \partial f }{ \partial x }(x_0,y_0)-(y-y_0)\frac{ \partial f }{ \partial y }(x_0,y_0)+z-f(x_0,y_0)=0,
\end{equation}
ce qui revient à
\begin{equation}
    z-f(x_0,y_0)=\frac{ \partial f }{ \partial x }(x_0,y_0)(x-x_0)+\frac{ \partial f }{ \partial y }(x_0,y_0)(y-y_0).
\end{equation}
Nous retrouvons donc l'équation du plan tangent à un graphe.

\begin{example}
    La sphère de rayon $R$ peut être paramétrée par les angles sphériques :
    \begin{equation}
        \phi(\theta,\varphi)=\begin{pmatrix}
            R\sin\theta\cos\varphi    \\ 
            R\sin\theta\sin\varphi    \\ 
            R\cos\theta    
        \end{pmatrix}
    \end{equation}
    avec $(\theta,\varphi)\in\mathopen[ 0 , \pi \mathclose]\times \mathopen[ 0 , 2\pi \mathclose]$.

    Tentons d'en trouver le plan tangent au point $(x,y,z)=(R,0,0)$. Un petit dessin nous montre que c'est un plan vertical d'équation $x=R$. Montrons cela en utilisant la théorie que nous venons de découvrir. D'abord le point $(R,0,0)$ correspond à $\theta_0=\frac{ \pi }{ 2 }$ et $\varphi=0$. Les vecteurs tangents sont
    \begin{equation}        \label{EqTthetaSph}
        T_{\theta}=\frac{ \partial \phi }{ \partial \theta }(R,\frac{ \pi }{2},0)=\begin{pmatrix}
            R\cos\theta\cos\varphi    \\ 
            R\cos\theta\sin\varphi    \\ 
            -R\sin\theta    
        \end{pmatrix}=\begin{pmatrix}
            0    \\ 
            0    \\ 
            -R    
        \end{pmatrix},
    \end{equation}
    et
    \begin{equation}    \label{EqTvarphiSph}
        T_{\varphi}=\frac{ \partial \phi }{ \partial \varphi }(R,\frac{ \pi }{2},0)=\begin{pmatrix}
            -R\sin\theta\sin\varphi\\
            R\sin\theta\cos\varphi\\
            0
        \end{pmatrix}=\begin{pmatrix}
            0    \\ 
            R    \\ 
            0    
        \end{pmatrix}.
    \end{equation}
    Cela sont de toute évidence bien les deux vecteurs tangents à la sphère au point $(x,y,z)=(R,0,0)$. Le vecteur normal est
    \begin{equation}
        \begin{vmatrix}
            e_x    &   e_y    &   e_z    \\
            0    &   0    &   -R    \\
            0    &   R    &   0
        \end{vmatrix}=R^2e_x.
    \end{equation}
    Ici encore, nous avons le vecteur que nous attendions sur un dessin. L'équation du plan tangent est maintenant
    \begin{equation}
        \begin{pmatrix}
            x-R    \\ 
            y    \\ 
            z    
        \end{pmatrix}\cdot
        \begin{pmatrix}
            R^2    \\ 
            0    \\ 
            0    
        \end{pmatrix}=0,
    \end{equation}
    c'est à dire $R^2(x-R)=0$ et donc $x=R$.
\end{example}

%+++++++++++++++++++++++++++++++++++++++++++++++++++++++++++++++++++++++++++++++++++++++++++++++++++++++++++++++++++++++++++
\section{Intégrales de surface}
%+++++++++++++++++++++++++++++++++++++++++++++++++++++++++++++++++++++++++++++++++++++++++++++++++++++++++++++++++++++++++++

%---------------------------------------------------------------------------------------------------------------------------
\subsection{Aire d'une surface paramétrée}
%---------------------------------------------------------------------------------------------------------------------------

Lorsque nous avions vu la longueur d'une courbe paramétrée, nous avions pris comme «élément de longueur» la norme du vecteur tangent. Il est donc naturel de prendre comme «élément de surface» une petite surface que l'on peut construire à partir des deux vecteurs tangents à la surface.

Au point $\varphi(u_0,v_0)$, nous avons les deux vecteurs tangents
\begin{equation}
    \begin{aligned}[]
        T_u&=\frac{ \partial \varphi }{ \partial u }(u_0,v_0)&T_v&=\frac{ \partial \varphi }{ \partial v }(u_0,v_0).
    \end{aligned}
\end{equation}
L'élément de surface que nous pouvons construire à partir de ces deux vecteurs est la surface du parallélogramme, donnée par la norme du produit vectoriel :
\begin{equation}
    dS=\| T_u\times T_v \|.
\end{equation}

L'aire de la surface donné par $\varphi\colon D\subset\eR^2\to \eR^3$ sera donc donnée par
\begin{equation}
    Aire\big( \varphi(D) \big)=\iint_D\| T_u\times T_v \|du\,dv.
\end{equation}

\begin{example}
    Calculons l'aire de la sphère. Les vecteurs tangents ont déjà été calculés aux équations \eqref{EqTthetaSph} et \eqref{EqTvarphiSph} :
    \begin{equation}
        \begin{aligned}[]
            T_{\theta}&=\begin{pmatrix}
                R\cos\theta\cos\varphi    \\ 
                R\cos\theta\sin\varphi    \\ 
                -R\sin\theta    
            \end{pmatrix},
            &T_{\varphi}&=\begin{pmatrix}
                -R\sin\theta\sin\varphi    \\ 
                R\sin\theta\cos\varphi    \\ 
                0    
            \end{pmatrix}.
        \end{aligned}
    \end{equation}
    Le produit vectoriel vaut
    \begin{equation}
        \begin{aligned}[]
            T_{\theta}\times T_{\varphi}&=
            \begin{vmatrix}
                e_x    &   e_y    &   e_z    \\
             R\cos\theta\cos\varphi   &   R\cos\theta\sin\varphi    &   -R\sin\theta    \\
            -R\sin\theta\sin\varphi    &   R\sin\theta\cos\varphi    &   0
            \end{vmatrix}\\
            &=(R^2\sin^2\theta\cos\varphi)e_x+(R^2\sin^2\theta\sin\varphi)e_y\\
            &\quad +(R^2\cos\theta\sin\theta\cos^2\varphi+R^2\sin\theta\cos\theta \sin^2\varphi)e_z.
        \end{aligned}
    \end{equation}
    La norme demande quelque calculs et mises en évidences. Le résultat est :
    \begin{equation}        \label{EqProdVectTTSPh}
        \| T_{\theta}\times T_{\varphi} \|=R^2\sin\theta.
    \end{equation}
    L'aire de la sphère est donc donnée par
    \begin{equation}
        Aire=\int_0^{2\pi}d\varphi\int_0^{\pi} R^2\sin\theta d\theta=2\pi R^2[-\cos\theta]_0^{\pi}=4\pi R^2.
    \end{equation}
    
    Il est bon de se souvenir que, en coordonnées sphériques, 
    \begin{equation}
        \| T_{\theta}\times T_{\varphi} \|=R^2\sin\theta.
    \end{equation}
    Or nous savons que ce vecteur est dirigé dans le sens de $e_r$ parce que ce dernier est le vecteur qui est constamment dirigé radialement. En coordonnées sphériques nous avons donc
    \begin{equation}        \label{EqNormalEnSpeh}
        T_{\theta}\times T_{\varphi}=R^2\sin(\theta)e_r.
    \end{equation}
    
\end{example}

\begin{remark}
    L'équation \eqref{EqProdVectTTSPh} donne l'élément de surface pour la sphère. Notez que cela est justement l'expression du jacobien des coordonnées sphériques. Cela n'est évidemment pas une coïncidence.
\end{remark}

\begin{example}
    Nous pouvons donner l'aire du graphe d'une fonction quelconque. La surface est paramétrée par
    \begin{equation}
        \varphi(x,y)=\begin{pmatrix}
            x    \\ 
            y    \\ 
            f(x,y)    
        \end{pmatrix}.
    \end{equation}
    Les vecteurs tangents sont
    \begin{equation}
        \begin{aligned}[]
            T_x&=\begin{pmatrix}
                1    \\ 
                0    \\ 
                \partial_xf    
            \end{pmatrix},&T_y&=\begin{pmatrix}
                0    \\ 
                1    \\ 
                \partial_yf    
            \end{pmatrix}.
        \end{aligned}
    \end{equation}
    Le produit vectoriel est donné par
    \begin{equation}
        T_x\times T_y=\begin{vmatrix}
             e_x   &   e_y    &   e_z    \\
            1    &   0    &   \partial_xf    \\
            0    &   1    &   \partial_yf
        \end{vmatrix}=(-\partial_xf)e_x-(\partial_yf)e_y+e_z.
    \end{equation}
    L'élément de surface est par conséquent
    \begin{equation}
        dS=\sqrt{\left( \frac{ \partial f }{ \partial x } \right)^2+\left( \frac{ \partial f }{ \partial y } \right)^2+1},
    \end{equation}
    et la surface du graphe sera
    \begin{equation}
        Aire=\iint_D\sqrt{\left( \frac{ \partial f }{ \partial x }(x,y) \right)^2+\left( \frac{ \partial f }{ \partial y }(x,y) \right)^2+1}\,dx\,dy
    \end{equation}
    
\end{example}

%---------------------------------------------------------------------------------------------------------------------------
\subsection{Intégrale d'une fonction sur une surface}
%---------------------------------------------------------------------------------------------------------------------------

Si $S$ est une surface dans $\eR^3$ paramétrée par
\begin{equation}
    \begin{aligned}
        \varphi\colon D&\to \eR^3 \\
        (u,v)&\mapsto \varphi(u,v)\in S,
    \end{aligned}
\end{equation}
et si $f$ est une fonction $f\colon \eR^3\to \eR$ définie au moins sur $S$, l'intégrale de $f$ sur $S$ est logiquement définie par
\begin{equation}
    \int_S f\,dS=\iint_D f\big( \varphi(u,v) \big)\| T_u(u,v)\times T_v(u,v) \|dudv
\end{equation}
où $T_u=\frac{ \partial \varphi }{ \partial u }$ et $Y_v=\frac{ \partial \varphi }{ \partial v }$. La quantité
\begin{equation}
    \| T_u(u,v)\times T_v(u,v) \|dudv
\end{equation}
est appelé \defe{élément de surface}{element@élément!de surface}.

Encore une fois, si on prend $f=1$, alors on retrouve la surface de $S$ :
\begin{equation}
    \int_SdS=Aire(S).
\end{equation}

\begin{remark}
    Le nombre $\int_SfdS$ ne dépend pas de la paramétrisation choisie pour $S$.
\end{remark}

%---------------------------------------------------------------------------------------------------------------------------
\subsection{Aire d'une surface de révolution}
%---------------------------------------------------------------------------------------------------------------------------

Soit $\gamma$ une courbe dans le plan $xy$, paramétrée par
\begin{equation}
    \gamma(u)=\begin{pmatrix}
        x(u)    \\ 
        y(u)    \\ 
        0    
    \end{pmatrix}
\end{equation}
avec $u\in\mathopen[ a , b \mathclose]$. Nous supposons que la courbe est toujours positive, c'est à dire $y(u)>0$ pour tout $u$.

Nous voulons considérer la surface obtenue en effectuant une rotation de cette ligne autour de l'axe $X$. Chaque point de la courbe va parcourir un cercle de rayon $y(u)$ dans le plan $YX$ et centré en $(x(u),0,0)$. La surface est donc donnée par
\begin{equation}
    \varphi(u,\theta)=\begin{pmatrix}
        x(u)    \\ 
        y(u)\cos\theta    \\ 
        y(u)\sin\theta    
    \end{pmatrix}
\end{equation}
avec $(u,\theta)\in\mathopen[ a , b \mathclose]\times \mathopen[ 0 , 2\pi \mathclose]$. Notez que la courbe de départ correspond à $\theta=0$.

Les vecteurs tangents à la surface pour cette paramétrisation sont
\begin{equation}
    \begin{aligned}[]
        T_u&=\frac{ \partial \varphi }{ \partial u }=\begin{pmatrix}
            x'(u)    \\ 
            y'(u)\cos\theta    \\ 
            y'(u)\sin\theta    
        \end{pmatrix}&
        T_{\theta}&=\frac{ \partial \varphi }{ \partial \theta }=\begin{pmatrix}
            0    \\ 
            -y(u)\sin\theta    \\ 
            y(u)\cos\theta    
        \end{pmatrix}.
    \end{aligned}
\end{equation}
Le produit vectoriel de ces deux vecteurs vaut
\begin{equation}
    \begin{aligned}[]
        T_u\times T_{\theta}&=\begin{vmatrix}
            e_x    &   e_y    &   e_z    \\
            x'    &   y'\cos\theta    &   y'\sin\theta    \\
            0    &   -y\sin\theta    &   y\cos\theta
        \end{vmatrix}\\
        &=y'(u)y(u)\,e_x-x'(u)y(u)\cos\theta\, e_y+x'(u)y(u)\sin\theta\, e_z.
    \end{aligned}
\end{equation}
En ce qui concerne la norme :
\begin{equation}
    dS=\| T_u\times T_{\theta} \|=\sqrt{(y'y)^2+(x'y)^2}=| y(u) |\sqrt{y'(u)^2+x'(u)^2}.
\end{equation}
Étant donné que nous avons supposé que $y(u)>0$, nous pouvons supprimer les valeurs absolues, et l'aire de la surface de révolution devient :
\begin{equation}
    \begin{aligned}[]
        Aire(S)&=\int_0^{2\pi}d\theta\int_a^b y(u)\sqrt{x'(u)^2+y'(u)^2}du\\
        &=2\pi\int_a^b y(u)\sqrt{x'(u)^2+y'(u)^2}du.
    \end{aligned}
\end{equation}

\begin{example}
    Calculons la surface du cône de révolution de rayon (à la base) $R$ et de hauteur $h$. La courbe de départ est le segment droite qui part de $(0,0)$ et qui termine en $(R,h)$ de la figure \ref{LabelFigConeRevolution}.
    \newcommand{\CaptionFigConeRevolution}{En faisant tourner cette droite autour de l'axe $X$, nous obtenons un cône.}
    \input{Fig_ConeRevolution.pstricks}
    Ce segment peut être paramétré par
    \begin{equation}
        \gamma(u)=\begin{pmatrix}
            Ru    \\ 
            hu    \\ 
            0    
        \end{pmatrix}
    \end{equation}
    avec $u\in\mathopen[ 0 , 1 \mathclose]$. Cela donne $x(u)=Ru$, $y(u)=hu$ et par conséquent
    \begin{equation}
        Aire=2\pi\int_0^1hu\sqrt{R^2+h^2}=\pi h\sqrt{R^2+h^2}.
    \end{equation}
    Nous pouvons aussi exprimer ce résultat en fonction de l'angle, en sachant que $h=\sqrt{h^2+R^2}\sin(\alpha)$ :
    \begin{equation}
        Aire=\pi(R^2+h^2)\sin(\alpha).
    \end{equation}
    
\end{example}

\begin{example}
    Calculons la surface latérale du tore obtenu par révolution du cercle de la figure \ref{LabelFigToreRevolution}.                                                                                           
    \newcommand{\CaptionFigToreRevolution}{Si nous tournons ce cercle autour de l'axe $X$, nous obtenons un tore de rayon «externe» $a$ et de rayon «interne» $R$.}
    \input{Fig_ToreRevolution.pstricks}

    Le chemin qui détermine le cercle de départ est
    \begin{equation}
        \gamma(u)=\begin{pmatrix}
            R\cos(u)    \\ 
            a+R\sin(u)    \\ 
            0    
        \end{pmatrix},
    \end{equation}
    c'est à dire $x(u)=R\cos(u)$, $y(u)=a+R\sin(u)$ avec $u\in\mathopen[ 0 , 2\pi \mathclose]$. Nous avons donc l'aire
    \begin{equation}
        \begin{aligned}[]
            Aire&=2\pi\int_0^{2\pi}\big( a+R\sin(u) \big)R\,du\\
            &=2\pi R\big( 2\pi a+R[-\cos(u)]_0^{2\pi} \big)\\
            &=4\pi^2aR.
        \end{aligned}
    \end{equation}
\end{example}

%+++++++++++++++++++++++++++++++++++++++++++++++++++++++++++++++++++++++++++++++++++++++++++++++++++++++++++++++++++++++++++
\section{Flux d'un champ de vecteurs à travers une surface}
%+++++++++++++++++++++++++++++++++++++++++++++++++++++++++++++++++++++++++++++++++++++++++++++++++++++++++++++++++++++++++++

Nous voulons construire un moulin à eau. Comment placer les pales pour maximiser le travail de la pression de l'eau ? On n'a pas attendu l'invention du calcul intégral pour répondre à cette question. Trois paramètres rentrent en ligne de compte :
\begin{enumerate}
    \item
        plus il y a d'eau, plus ça pousse;
    \item
        plus la surface de la palle est grande, plus on va utiliser d'eau;
    \item
        plus la palle est perpendiculaire au courant, plus on va en profiter.
\end{enumerate}
Nous voyons sur la figure \ref{LabelFigMoulinEau} que lorsque la palle du moulin est inclinée, non seulement elle prend moins d'eau sur elle, mais qu'en plus elle la prend avec un moins bon angle : une partie de la force ne sert pas à la faire tourner.
\newcommand{\CaptionFigMoulinEau}{La partie rouge de la force est perdue si l'eau ne pousse pas perpendiculairement. De plus lorsque la palle est inclinée, elle prend moins d'eau sur elle.}
\input{Fig_MoulinEau.pstricks}
%See also the subfigure \ref{LabelFigMoulinEaussLabelSubFigMoulinEau0}
%See also the subfigure \ref{LabelFigMoulinEaussLabelSubFigMoulinEau1}


L'idée du flux d'un champ de vecteurs à travers une surface est de savoir quelle est la quantité «utile» de vecteurs qui traverse la surface. Ce sera simplement l'intégrale sur la surface de la composante du champ de vecteurs normale à la surface. Il reste deux problèmes à régler : le premier est de savoir quel est le vecteur normal à la surface, et le second est de savoir comment «sélectionner» la composante normale d'un champ de vecteurs $F$.

Le problème de trouver un vecteur normal est résolu par le produit vectoriel des vecteurs tangents. Si la surface est donnée par $\varphi\colon D\subset\eR^2\to \eR^3$, les vecteurs tangents sont $T_u=\partial_u\varphi(u,v)$ et  $T_v=\partial_v\varphi(u,v)$. Le normal de norme $1$ est donné par :
\begin{equation}
    n(u,v)=\frac{ T_u\times T_v }{ \| T_u\times T_v \| }.
\end{equation}

Si $p$ est un point de la surface $\varphi(D)$, la composante de $F(p)$ qui est normale à la surface au point $p$ est donnée par le produit scalaire
\begin{equation}
    F(p)_{\perp}=F(p)\cdot n(p).
\end{equation}
C'est ce nombre là que nous intégrons sur la surface. 

\begin{definition}
    Le \defe{flux du champ de vecteurs}{flux d'un champ de vecteurs} à travers la surface $S=\varphi(D)$ est
    \begin{equation}
        \int F\cdot dS=\int F \cdot n\,dudv.
    \end{equation}
\end{definition}

Une petite simplification se produit lorsqu'on veut calculer effectivement cette intégrale. En effet $F\cdot n$ est, en soi, une fonction sur $S$. Pour l'intégrer, il faut donc la multiplier par $\| T_u\times T_v \|$ (c'est la définition de l'intégrale d'une fonction sur une surface). Donc, étant donné que $n=(T_u\times T_v)/\| T_u\times T_v \|$, nous avons
\begin{equation}
    \int F\cdot dS=\iint_D F\big( \varphi(u,v) \big)\cdot (T_u\times T_v)\,dudv
\end{equation}
où $T_u=\frac{ \partial \phi }{ \partial u }$ et $T_v=\frac{ \partial \varphi }{ \partial v }$.


\begin{example}
    Soit le champ de vecteurs
    \begin{equation}
        F=\begin{pmatrix}
            2x    \\ 
            2y    \\ 
            2z    
        \end{pmatrix}.
    \end{equation}
    Calculons son flux au travers de la sphère de rayon $R$.

    Nous choisissons de paramétrer la sphère en coordonnées sphériques avec $\phi(\theta,\varphi)$. Nous pouvons reprendre le résultat \eqref{EqNormalEnSpeh} :
    \begin{equation}
        T_{\theta}\times T_{\varphi}=R^2\sin(\theta).
    \end{equation}
    Nous savons aussi que
    \begin{equation}
        F\big( \phi(\theta,\varphi) \big)=2e_r.
    \end{equation}
    L'intégrale à calculer est donc
    \begin{equation}
        I=\int_0^{\pi}d\theta\int_0^{2\pi}d\varphi\, 2e_r\cdot\big( R^2\sin(\theta)e_r \big).
    \end{equation}
    Vu que le produit scalaire $e_r\cdot e_r$ vaut $1$, nous calculons
    \begin{equation}
        \begin{aligned}[]
            I=4\pi R^2\int_0^{\pi}\sin(\theta)d\theta=8\pi R^2.
        \end{aligned}
    \end{equation}
    
\end{example}

\begin{example}
    Calculons le flux du champ de force de gravitation d'une masse au travers de la sphère de centre $R$ centrée autour la masse. À un coefficient constant près, le champ vaut
    \begin{equation}
        G(r,\theta,\varphi)=\frac{1}{ r^2 }e_r.
    \end{equation}
    Sur la sphère de rayon $R$, nous avons
    \begin{equation}
        G\big( \phi(\theta,\varphi) \big)=\frac{1}{ R^2 }e_r.
    \end{equation}
    L'intégrale est donc
    \begin{equation}
        \int_0^{\pi}d\theta\int_0^{2\pi}\frac{1}{ R^2 }e_r\cdot \big( R^2\sin(\theta)e_r \big)d\varphi=8\pi.
    \end{equation}
    Ce flux ne dépend pas de $R$.
\end{example}

\begin{example}
    Soit $S$ le disque de rayon $5$ placé horizontalement à la hauteur $12$. Calculer le flux du champ de vecteurs
    \begin{equation}
        F(x,y,z)=xe_x+ye_y+ze_z.
    \end{equation}
    Les équations de la surface sont $z=12$, $x^2+y^2\leq 25$. Nous prenons le paramétrage en coordonnées cylindriques :
    \begin{equation}
        \varphi(r,\theta)=\begin{pmatrix}
            r\cos(\theta)    \\ 
            r\sin(\theta)    \\ 
            12    
        \end{pmatrix}.
    \end{equation}
    Les vecteurs tangents sont
    \begin{equation}
        \begin{aligned}[]
            T_r=\frac{ \partial \varphi }{ \partial r }&=\begin{pmatrix}
                \cos\theta    \\ 
                \sin\theta    \\ 
                0    
            \end{pmatrix}&T_{\theta}=\frac{ \partial \varphi }{ \partial \theta }&=\begin{pmatrix}
                -r\sin\theta    \\ 
                r\cos\theta    \\ 
                0    
            \end{pmatrix}.
        \end{aligned}
    \end{equation}
    Le vecteur normal est alors
    \begin{equation}
        T_r\times T_{\theta}=re_z.
    \end{equation}
    Sur la surface, le champ de vecteurs s'écrit
    \begin{equation}
        F\big( \varphi(r,\theta) \big)=r\cos(\theta)e_x+r\sin(\theta)e_y+12e_z.
    \end{equation}
    Par conséquent
    \begin{equation}
        F\cdot(T_r\times T_{\theta})=12r.
    \end{equation}
    L'intégrale à calculer est
    \begin{equation}
        \begin{aligned}[]
            \int_0^5dr\int_0^{2\pi}12r\,d\theta&=12\cdot 2\pi\int_0^5r\,dr\\
            &=\frac{ 25 }{ 2 }24\pi\\
            &=300\pi.
        \end{aligned}
    \end{equation}
    
\end{example}

\clearpage

%+++++++++++++++++++++++++++++++++++++++++++++++++++++++++++++++++++++++++++++++++++++++++++++++++++++++++++++++++++++++++++
\section{Résumé des intégrales vues}
%+++++++++++++++++++++++++++++++++++++++++++++++++++++++++++++++++++++++++++++++++++++++++++++++++++++++++++++++++++++++++++

Nous sommes maintenant capables de revoir tous les types d'intégrales vues jusqu'ici de façon très cohérentes. Nous commencerons par les intégrales de fonctions et nous ferons ensuite les intégrales de champs de vecteurs.

%---------------------------------------------------------------------------------------------------------------------------
\subsection{L'intégrale d'une fonction sur les réels}
%---------------------------------------------------------------------------------------------------------------------------

Si $f\colon \mathopen[ a , b \mathclose]\subset\eR\to \eR$ est une fonction usuelle, sont intégrale est
\begin{equation}
    \int_a^bf(x)dx=F(b)-F(a)
\end{equation}
où $F$ est une primitive de $f$.

%---------------------------------------------------------------------------------------------------------------------------
\subsection{Intégrale d'une fonction sur un chemin}
%---------------------------------------------------------------------------------------------------------------------------

Si $f$ est une fonction sur $\eR^3$ et si $\sigma\colon \mathopen[ a , b \mathclose]\to \eR^3$ est un chemin dans $\eR^3$, l'intégrale de $f$ sur $\sigma$ est, par définition, 
\begin{equation}
    \int f\,d\sigma=\int_a^b f\big( \sigma(t) \big)\| \sigma'(t) \|dt.
\end{equation}

%---------------------------------------------------------------------------------------------------------------------------
\subsection{Intégrale d'une fonction sur une surface}
%---------------------------------------------------------------------------------------------------------------------------

Nous devons paramétrer la surface $S$ par une application $\varphi\colon D\subset\eR^2\to \eR^3$. À partir d'une telle paramétrisation, nous construisons un élément de surface en prenant le produit vectoriel des deux vecteurs tangents :
\begin{equation}
    dS=\frac{ \partial \varphi }{ \partial u }\times\frac{ \partial \varphi }{ \partial v }dudv.
\end{equation}
L'intégrale est
\begin{equation}        \label{EqDefIntSurffS}
    \int f\,dS=\iint_Df\big( \varphi(u,v) \big)\left\| \frac{ \partial \varphi }{ \partial u }\times\frac{ \partial \varphi }{ \partial v } \right\|dudv.
\end{equation}

Il ne faut pas rajouter de jacobien : la norme du produit vectoriel \emph{est} le jacobien.

\begin{remark}
    La formule \eqref{EqDefIntSurffS} est autant valable pour des surfaces dans $\eR^2$ que dans $\eR^3$. Si nous considérons une surface dans $\eR^2$, nous la voyons dans $\eR^3$ en ajoutant un zéro comme troisième composante.
\end{remark}

\begin{example}
    Les coordonnées polaires sont données par
    \begin{equation}
        \varphi(r,\theta)=\begin{pmatrix}
            r\cos\theta    \\ 
            r\sin\theta    \\ 
            0    
        \end{pmatrix}.
    \end{equation}
    Les vecteurs tangents à cette paramétrisation sont
    \begin{equation}
        \begin{aligned}[]
            T_r&=\frac{ \partial \varphi }{ \partial r }=\begin{pmatrix}
                \cos\theta    \\ 
                \sin\theta    \\ 
                0    
            \end{pmatrix},&T_{\theta}&=\frac{ \partial \varphi }{ \partial \theta }=\begin{pmatrix}
                -r\sin\theta    \\ 
                r\cos\theta    \\ 
                0    
            \end{pmatrix}.
        \end{aligned}
    \end{equation}
    Le vecteur normal est
    \begin{equation}
        \frac{ \partial \varphi }{ \partial r }\times\frac{ \partial \varphi }{ \partial \theta }=\begin{vmatrix}
            e_x    &   e_y    &   e_z    \\
            \cos\theta    &   \sin\theta    &   0    \\
            -r\sin\theta    &   r\cos\theta    &   0
        \end{vmatrix}=re_z.
    \end{equation}
    Nous trouvons donc que l'élément de surface est la norme de $re_z$, c'est à dire $r$, le jacobien connu.
\end{example}

%---------------------------------------------------------------------------------------------------------------------------
\subsection{Intégrale d'une fonction sur un volume}
%---------------------------------------------------------------------------------------------------------------------------

Si $V$ est un volume dans $\eR^3$, nous effectuons la même procédure : nous trouvons une paramétrisation, et nous formons un élément de volume avec les vecteurs tangents de la paramétrisation. Nous avons donc un volume déterminé par l'application
\begin{equation}
    \varphi\colon D\subset\eR^3\to \eR^3,
\end{equation}
et ses trois vecteurs tangents
\begin{equation}
    \begin{aligned}[]
        T_u&=\frac{ \partial \varphi }{ \partial u }\\
        T_v&=\frac{ \partial \varphi }{ \partial v }\\
        T_w&=\frac{ \partial \varphi }{ \partial w }.
    \end{aligned}
\end{equation}
Comment former un volume avec trois vecteurs ? Réponse : le produit mixte. L'intégrale de $f$ sur $V$ sera
\begin{equation}
    \int f\,dV=\iiint_D f\big( \varphi(u,v,w) \big)\left\| \frac{ \partial \varphi }{ \partial u }\cdot \left( \frac{ \partial \varphi }{ \partial v }\times\frac{ \partial \varphi }{ \partial w }\right) \right\|dudv.
\end{equation}

Encore une fois, le produit mixte \emph{est} le jacobien. Prenons les coordonnées sphériques :
\begin{equation}
    \begin{aligned}[]
        x(r,\theta,\varphi)&=r\sin(\theta)\cos(\varphi)\\
        y(r,\theta,\varphi)&=r\sin(\theta)\sin(\varphi)\\
        z(r,\theta,\varphi)&=r\cos(\theta)
    \end{aligned}
\end{equation}
Les trois vecteurs tangents seront
\begin{subequations}
    \begin{align}
        T_r&=\begin{pmatrix}
            \frac{ \partial x }{ \partial r }    \\ 
            \frac{ \partial y }{ \partial r }    \\ 
            \frac{ \partial z }{ \partial r }    
        \end{pmatrix}=\begin{pmatrix}
            \sin(\theta)\cos(\varphi)    \\ 
            \sin(\theta)\sin(\varphi)    \\ 
            \cos(\theta)    
        \end{pmatrix}\\
        T_{\theta}&=\begin{pmatrix}
            \frac{ \partial x }{ \partial \theta }    \\ 
            \frac{ \partial y }{ \partial \theta }    \\ 
            \frac{ \partial z }{ \partial \theta }    
        \end{pmatrix}=\begin{pmatrix}
            r\cos(\theta)\cos(\varphi)    \\ 
            r\cos(\theta)\sin(\varphi)    \\ 
            -r\sin(\theta)    
        \end{pmatrix}\\
        T_{\varphi}&=\begin{pmatrix}
            \frac{ \partial x }{ \partial \varphi }    \\ 
            \frac{ \partial y }{ \partial \varphi }    \\ 
            \frac{ \partial z }{ \partial \varphi }    
        \end{pmatrix}=\begin{pmatrix}
            -r\sin(\theta)\sin(\varphi)    \\ 
            r\sin(\theta)\cos(\varphi)    \\ 
            0
        \end{pmatrix}
        \end{align}
\end{subequations}
Nous avons vu que le produit mixte revient à mettre toutes les composantes dans une matrice. Ici nous avons donc
\begin{equation}
    \frac{ \partial \phi }{ \partial r }\cdot\left( \frac{ \partial \phi }{ \partial \theta }\times\frac{ \partial \phi }{ \partial \varphi } \right)=\begin{vmatrix}
        \frac{ \partial x }{ \partial r }    &   \frac{ \partial y }{ \partial r }    &   \frac{ \partial z }{ \partial r }    \\
        \frac{ \partial x }{ \partial \theta }    &   \frac{ \partial y }{ \partial \theta }    &   \frac{ \partial z }{ \partial \theta }    \\
        \frac{ \partial x }{ \partial \varphi }    &   \frac{ \partial y }{ \partial \varphi }    &   \frac{ \partial z }{ \partial \varphi }    
    \end{vmatrix}
\end{equation}
Cela est précisément le jacobien dont nous parlions plus haut.

%---------------------------------------------------------------------------------------------------------------------------
\subsection{Conclusion pour les fonctions}
%---------------------------------------------------------------------------------------------------------------------------

Lorsque nous intégrons une fonction sur un chemin, une surface ou un volume, la technique est toujours la même :
\begin{enumerate}
    \item
        Trouver une paramétrisation à une, deux ou trois variables.
    \item
        Dériver la paramétrisation par rapport à ses variables.
    \item
        Construire un élément de longueur, surface ou volume à partir des vecteurs que l'on a. Cela se fait en prenant la norme, le produit vectoriel ou le produit mixte.
\end{enumerate}

%---------------------------------------------------------------------------------------------------------------------------
\subsection{Circulation d'un champ de vecteurs}
%---------------------------------------------------------------------------------------------------------------------------

Pour les champs de vecteurs, nous faisons la même chose, mais au lieu de \emph{multiplier} par l'élément de longueur ou de surface, nous prenons le produit scalaire. Si nous considérons la courbe paramétrée $\sigma\colon \mathopen[ a , b \mathclose]\to \eR^3$ et le champ de vecteurs $F$, nous avons donc
\begin{equation}
    \int_{\sigma}F=\int F\cdot d\sigma=\int_a^bF\big( \sigma(t) \big)\cdot\sigma'(t)dt.
\end{equation}

%---------------------------------------------------------------------------------------------------------------------------
\subsection{Flux d'un champ de vecteurs}
%---------------------------------------------------------------------------------------------------------------------------

Si la surface $S\subset\eR^3$ est paramétrée par
\begin{equation}
    \begin{aligned}
        \phi\colon D\subset\eR^2&\to \eR^3 \\
        (u,v)&\mapsto \phi(u,v), 
    \end{aligned}
\end{equation}
et si $F$ est un champ de vecteurs, alors on a
\begin{equation}        \label{EqResIntFluxPhi}
    \int_SF=\int_S F\cdot dS=\iint_D F\big( \phi(u,v) \big)\cdot\left( \frac{ \partial \phi }{ \partial u }\times\frac{ \partial \phi }{ \partial v } \right)\,dudv.
\end{equation}

%---------------------------------------------------------------------------------------------------------------------------
\subsection{Conclusion pour les champs de vecteurs}
%---------------------------------------------------------------------------------------------------------------------------

La circulation et le flux ne représentent pas tout à fait la même chose. En effet pour la circulation, nous sélectionnons la composante \emph{tangente} à la courbe, c'est à dire la partie du vecteurs qui «circule» le long de la courbe. Une force perpendiculaire au mouvement ne travaille pas.

La situation est exactement le contraire pour le flux. Étant donné que le vecteur
\begin{equation}
    \frac{ \partial \phi }{ \partial u }\times\frac{ \partial \phi }{ \partial v }
\end{equation}
est normal à la surface, le fait de prendre le produit scalaire du champ de vecteurs avec lui sélectionne la composante \emph{normale} à la surface, c'est à dire la partie du vecteur qui traverse la surface.

%---------------------------------------------------------------------------------------------------------------------------
\subsection{Attention pour les surfaces fermées !}
%---------------------------------------------------------------------------------------------------------------------------

Si nous considérons une surface fermée, il faut faire attention à choisir une \emph{orientation}. Les vecteurs normaux doivent soit tous pointer vers l'intérieur soit tous vers l'extérieur. En effet, en tant que vecteur normal, nous avons choisi de prendre
\begin{equation}
    T_u\times T_v.
\end{equation}
Mais le vecteur $T_v\times T_u$ est tout aussi normal ! Il n'y a pas a priori de façon standard pour choisir l'un ou l'autre. Il faut juste être cohérent : il faut que si on divise la surface en plusieurs morceaux, tous les vecteurs pointent dans le même sens.

Notez que si vous faites un choix et que votre voisin fait le choix inverse, vous obtiendrez des réponses qui diffèrent d'un signe. Sans plus de précisions\footnote{Il faudrait définir ce qu'est une surface \emph{orientable} et choisir une orientation.}, les deux réponses sont correctes.

Un exemple de ce problème est donné dans l'exercice \ref{exoOutilsMath-0110}.
%+++++++++++++++++++++++++++++++++++++++++++++++++++++++++++++++++++++++++++++++++++++++++++++++++++++++++++++++++++++++++++
\section{Le théorème de Green}
%+++++++++++++++++++++++++++++++++++++++++++++++++++++++++++++++++++++++++++++++++++++++++++++++++++++++++++++++++++++++++++

Soit un champ de vecteurs
\begin{equation}
    F(x,y,z)=\begin{pmatrix}
        F_1(x,y,z)    \\ 
        F_2(x,y,z)    \\ 
        F_2(x,y,z)    
    \end{pmatrix}
\end{equation}
et un chemin $\sigma\colon \mathopen[ a , b \mathclose]\to \eR^3$ donné par
\begin{equation}
    \sigma(t)=\begin{pmatrix}
        x(t)    \\ 
        y(t)    \\ 
        z(t)    
    \end{pmatrix}.
\end{equation}
Nous avons défini la circulation de $F$ le long de $\sigma$ par
\begin{equation}
    \begin{aligned}[]
        \int_{\sigma}F\cdot d\sigma&=\int_a^bF\big( \sigma(t) \big)\cdot\sigma'(t)dt\\
        &=\int_a^b\Big[ F_1\big( \sigma(t) \big)x'(t)+F_2\big( \sigma(t) \big)y'(y)+F_3\big( \sigma(t) \big)z'(t)\Big]dt\\
        &=\int_{\sigma} F_1dx +F_2dy+F_3dz.
    \end{aligned}
\end{equation}
La dernière ligne est juste une notation compacte\footnote{Il y aurai beaucoup de choses à dire là-dessus, mais la vie est trop courte pour parler de formes différentielles, et c'est dommage.}. Elle sert à se souvenir qu'on va mettre $x'$ à côté de $F_1$, $y'$ à côté de $F_2$ et $z'$ à côté de $F_3$. L'avantage de cette notation est qu'on peut écrire d'autres combinaisons.

Si $f$ et $g$ sont deux fonctions sur $\eR^3$, nous pouvons écrire
\begin{equation}
    \int_{\sigma} fdy+gdz.
\end{equation}
Cela signifie
\begin{equation}
    \int_a^b \Big[ f\big( \sigma(t) \big)y'(t)+g\big( \sigma(t) \big)z'(t)\Big]dt.
\end{equation}

Soit $D$ une région du plan et $\sigma$, son contour que nous prenons, par convention\footnote{Il y aurait beaucoup de choses à dire sur ça aussi, mais\ldots}, dans l'orientation trigonométrique, comme indiqué sur la figure \ref{LabelFigContourGreen}. Nous supposons également que le domaine $D$ n'a pas de trous intérieurs.
\newcommand{\CaptionFigContourGreen}{Un contour avec son orientation.}
\input{Fig_ContourGreen.pstricks}

Nous notons par $\sigma=\partial D$ le bord de $D$, c'est à dire le contour dont nous venons de parler.

\begin{theorem}[Théorème de Green]
    Soient $P,Q\colon D\to \eR$ deux fonctions de classe $C^1$. Alors
    \begin{equation}        \label{EqThoGreen}
        \int_{\partial D} Pdx+Qdy=\iint_D\left( \frac{ \partial Q }{ \partial x }-\frac{ \partial P }{ \partial y } \right)dxdy.
    \end{equation}
\end{theorem}
Pour rappel, l'intégrale du membre de gauche signifie
\begin{equation}
    \int_a^b \Big[P\big( \sigma(t) \big)\sigma_x'(t)+Q\big( \sigma(t) \big)\sigma_y'(t)\Big]dt.
\end{equation}
Ce n'est d'ailleurs rien d'autre que l'intégrale du champ de vecteurs $\begin{pmatrix}
    P    \\ 
    Q    
\end{pmatrix}$.

\begin{corollary}
    L'aire du domaine $D$ est donnée par
    \begin{equation}
        A=\frac{ 1 }{2}\int_{\partial D}(xdy-ydx).
    \end{equation}
\end{corollary}

\begin{proof}
    L'intégrale $\int_{\partial D}(xdy-ydx)$ se traite avec le théorème de Green où l'on pose $P=-y$ et $Q=x$. Nous avons donc
    \begin{equation}
        \begin{aligned}[]
            \int_{\partial D} -ydx+xdy&=\iint_D\left( \frac{ \partial x }{ \partial x }-\frac{ \partial (-y) }{ \partial y } \right)dxdy\\
            &=\iint_D2\,dxdy.
        \end{aligned}
    \end{equation}
    La dernière ligne est bien le double de la surface.
\end{proof}

\begin{example}
    Calculons (encore une fois) l'aire du disque de rayon $R$. Il s'agit de calculer l'intégrale
    \begin{equation}
        I=\frac{ 1 }{2}\int_{\sigma}(xdt-ydx)
    \end{equation}
    où $\sigma$ est le cercle donné par
    \begin{equation}
        \sigma(t)=\begin{pmatrix}
            x(t)    \\ 
            y(t)    
        \end{pmatrix}=\begin{pmatrix}
            R\cos(t)    \\ 
            R\sin(t)    
        \end{pmatrix}
    \end{equation}
    Le calcul est
    \begin{equation}
        \begin{aligned}[]
            I&=\frac{ 1 }{2}\int_{0}^{2\pi} \underbrace{R\cos(\theta)}_{x}\underbrace{R\cos(\theta)}_{y'}-\underbrace{R\sin(\theta)}_{y}\underbrace{(-R\sin(\theta))}_{x'}\,d\theta\\
            &=\frac{ R^2 }{2}\int_{0}^{\pi}d\theta\\
            &=\pi R^2.
        \end{aligned}
    \end{equation}
\end{example}

\begin{example}
    Calculons l'aire de l'ellipse 
    \begin{equation}
        \frac{ x^2 }{ a^2 }+\frac{ y^2 }{ b^2 }\leq 1
    \end{equation}
    dont le bord est donné par
    \begin{subequations}
        \begin{numcases}{}
            x(t)=a\cos(t)\\
            y(t)=b\sin(t).
        \end{numcases}
    \end{subequations}
    Le terme $xdy$ devient $a\cos(t)b\cos(t)=ab\cos^2(t)$ et le terme $ydx$ devient $b\sin(t)(-a\sin(t))=-ab\sin^2(t)$. L'intégrale qui donne la surface est donc
    \begin{equation}
        \frac{ 1 }{2}\int_{\partial D}(xdy-ydx)=\frac{ 1 }{2}\int_0^{2\pi}ab=\pi ab.
    \end{equation}
\end{example}


Le théorème de Green peut être mis sous une autre forme.

\begin{theorem}[Théorème de Green, forme vectorielle]       \label{ThoGreenVecto}
    Si $G$ est un champ de vecteurs sur $D$, nous avons
    \begin{equation}        \label{EqGreenVecto}
        \int_{\partial D}G\cdot d\sigma=\iint_D(\nabla\times G)\cdot dS
    \end{equation}
    où le second membre est le flux de $\nabla\times G$ sur la surface $D$.
\end{theorem}

\begin{proof}
    Analysons le membre de droite. Nous savons que $D$ est une surface dans le plan $\eR^2$. Le vecteur normal à la surface est donc simplement le vecteur (constant) $e_z$. Le produit scalaire $(\nabla\times F)\cdot dS$ est donc $(\nabla\times F)\cdot e_z$ et se réduit à la troisième composante du rotationnel, c'est à dire
    \begin{equation}
        \frac{ \partial F_2 }{ \partial x }-\frac{ \partial F_1 }{ \partial y }.
    \end{equation}
    Cela est bien le membre de droite de l'équation \eqref{EqThoGreen}. Le membre de gauche de cette dernière est bien le membre de gauche de \eqref{EqGreenVecto}.
\end{proof}

\begin{example}     \label{ExempleGreenSqL}
    Soit le champ de vecteurs $F(x,y)=\begin{pmatrix}
        xy^2    \\ 
        y+x    
    \end{pmatrix}$, et soit à calculer
    \begin{equation}
        \iint_D\nabla\times F\cdot dS
    \end{equation}
    où $D$ est la région comprise entre les courbes $y=x^2$ et $y=x$ pour $x\geq 0$ (voir la figure \ref{LabelFigContourSqL}).
    \newcommand{\CaptionFigContourSqL}{Le contour d'intégration pour l'exemple \ref{ExempleGreenSqL}.}
    \input{Fig_ContourSqL.pstricks}

    Nous pouvons calculer cette intégrale directement en calculant le rotationnel de $F$:
    \begin{equation}
        \nabla\times F=\begin{pmatrix}
            0    \\ 
            0    \\ 
            1-2xy    
        \end{pmatrix}.
    \end{equation}
    Par conséquent l'intégrale à effectuer est
    \begin{equation}
        I=\int_0^1 dx\int_{x^2}^x(1-2xy)dy=\frac{1}{ 12 }.
    \end{equation}
    \begin{verbatim}
 ----------------------------------------------------------------------
| Sage Version 4.6.1, Release Date: 2011-01-11                       |
| Type notebook() for the GUI, and license() for information.        |
----------------------------------------------------------------------
sage: f(x,y)=1-2*x*y
sage: f.integrate(y,x**2,x).integrate(x,0,1)
(x, y) |--> 1/12
    \end{verbatim}
    
    L'autre façon de calculer l'intégrale est d'utiliser le théorème de Green et de calculer la circulation de $F$ le long de $\partial D$ :
    \begin{equation}
        I=\int_{\partial D}F\cdot \sigma.
    \end{equation}
    Le chemin $\sigma=\partial D$ est composé de la parabole $y=x^2$ et du segment de droite $x=y$. Attention : il faut respecter l'orientation. Nous avons
    \begin{equation}
        \sigma_1(t)=(t,t^2)
    \end{equation}
    et
    \begin{equation}
        \sigma_2(t)=(1-t,1-t).
    \end{equation}
    Notez bien que le second chemin est $(1-t,1-t)$ et non $(t,t)$ parce qu'il faut le parcourir dans le bon sens (voir le dessin).

    Commençons par le premier chemin :
    \begin{equation}
        \begin{aligned}[]
            \sigma_1(t)&=(t,t^2)\\
            \sigma_1'(t)&=(1,2t)\\
            F\big( \sigma_1(t) \big)&=\begin{pmatrix}
                t^5    \\ 
                t+t^2    
            \end{pmatrix},
        \end{aligned}
    \end{equation}
    et par conséquent
    \begin{equation}
        F\big( \sigma_1(t) \big)\cdot \sigma_1'(t)=t^5+2t^2+2t^3,
    \end{equation}
    et le premier morceau de la circulation vaut
    \begin{equation}
        \int_{\sigma_1} F\cdot d\sigma_1=\int_0^1 t^5+2t^2+2t^3=\frac{ 4 }{ 3 }.
    \end{equation}
    
    Pour le second chemin :
    \begin{equation}
        \begin{aligned}[]
            \sigma_2(t)=(1-t,1-t)\\
            \sigma_2'(t)=(-1,-1)\\
            F\big( \sigma_2(t) \big)=\begin{pmatrix}
                (1-t)^3    \\ 
                2(1-t)    
            \end{pmatrix}.
        \end{aligned}
    \end{equation}
    Par conséquent
    \begin{equation}
        F\big( \sigma_2(t) \big)\cdot \sigma_2(t)=-(1-t)^2-2(1-t).
    \end{equation}
    Le second morceau de la circulation est par conséquent
    \begin{equation}
        \int_0^1-(1-t)^2-2(1-t)dt=-\frac{ 5 }{ 4 }.
    \end{equation}
    La circulation de $F$ le long de $\sigma$ est donc égale à
    \begin{equation}
        \frac{ 4 }{ 3 }-\frac{ 5 }{ 4 }=\frac{1}{ 12 }.
    \end{equation}
    Comme prévu, nous obtenons le même résultat.
\end{example}


%+++++++++++++++++++++++++++++++++++++++++++++++++++++++++++++++++++++++++++++++++++++++++++++++++++++++++++++++++++++++++++
\section{Théorème de la divergence dans le plan}
%+++++++++++++++++++++++++++++++++++++++++++++++++++++++++++++++++++++++++++++++++++++++++++++++++++++++++++++++++++++++++++

%---------------------------------------------------------------------------------------------------------------------------
\subsection{La convention de sens de parcours}
%---------------------------------------------------------------------------------------------------------------------------

Soient $D$, un domaine dans le plan et une paramétrisation
\begin{equation}
    \begin{aligned}
        \sigma\colon \mathopen[ a , b \mathclose]&\to \eR^2 \\
        t&\mapsto \begin{pmatrix}
            x(t)    \\ 
            y(t)    
        \end{pmatrix},
    \end{aligned}
\end{equation}
une paramétrisation du bord $\partial D$ de $D$. La normale à $\sigma$ est perpendiculaire à la tangente, donc la normale extérieure de norme $1$ vaut
\begin{equation}
    \begin{aligned}[]
        n&=\frac{ \big( y'(t),-x'(t) \big) }{ \sqrt{ \big( x'(t)\big)^2+\big( y'(t) \big)^2  } }&\text{ou}&&n&-=\frac{ \big( y'(t),-x'(t) \big) }{ \sqrt{ \big( x'(t)\big)^2+\big( y'(t) \big)^2  } }.
    \end{aligned}
\end{equation}
Comment faire le choix ?

Nous prenons comme convention que le sens \emph{du chemin} doit être tel que le vecteur normal extérieur soit
\begin{equation}
        n=\frac{ \big( y'(t),-x'(t) \big) }{ \sqrt{ \big( x'(t)\big)^2+\big( y'(t) \big)^2  } }.
\end{equation}
Donc si le chemin $\sigma$ donne lieu à un vecteur $n$ pointant vers l'intérieur, il faut utiliser le chemin qui va dans le sens contraire : $\tilde \sigma(t)=\sigma(1-t)$.

Les vecteurs tangents et normaux d'un contour sont dessinés sur la figure \ref{LabelFigContourTgNDivergence}.
\newcommand{\CaptionFigContourTgNDivergence}{Le champ de vecteurs tangents est dessiné en rouge tandis qu'en vert nous avons le champ de vecteurs normaux extérieurs.}
\input{Fig_ContourTgNDivergence.pstricks}

%---------------------------------------------------------------------------------------------------------------------------
\subsection{Théorème de la divergence}
%---------------------------------------------------------------------------------------------------------------------------

\begin{theorem}[Théorème de la divergence]
    Soit $F$ un champ de vecteurs sur $\eR^2$. Le flux de $F$ à travers le bord de $D$ est égal à l'intégrale de la divergence de $F$ sur $D$. En formule :
    \begin{equation}
        \int_{\partial D} F\cdot n\,d\sigma=\iint_D\nabla\cdot F\,dxdy.
    \end{equation}
\end{theorem}

\begin{remark}
    Tant $F\cdot n$ que $\nabla\times F$ sont des fonctions. Le membre de gauche est donc l'intégrale d'une fonction sur un chemin et le membre de droite est l'intégrale d'une fonction sur une surface.
\end{remark}

\begin{proof}
    Notre convention de sens de parcours du chemin permet d'écrire le produit scalaire $F\cdot n$ sous la forme suivante :
    \begin{equation}
        \begin{aligned}[]
            F\cdot n&=\frac{1}{ \| \sigma' \| }\begin{pmatrix}
                F_x    \\ 
                F_y    
            \end{pmatrix}\cdot \begin{pmatrix}
                y'    \\ 
                -x'    
            \end{pmatrix}\\
            &=\frac{1}{ \| \sigma' \| }(F_xy'-F_yx')\\
            &=\frac{1}{ \| \sigma' \| }\begin{pmatrix}
                -F_y    \\ 
                F_x    
            \end{pmatrix}\cdot \begin{pmatrix}
                x'    \\ 
                y'    
            \end{pmatrix}\\
            &=\frac{1}{ \| \sigma' \| }\begin{pmatrix}
                -F_y    \\ 
                F_x    
            \end{pmatrix}\cdot \sigma'.
        \end{aligned}
    \end{equation}

    Par conséquent, la \emph{fonction}
    \begin{equation}
        F\cdot n
    \end{equation}
    est la même que la \emph{fonction} 
    \begin{equation}
        \frac{1}{ \| \sigma' \| }\begin{pmatrix}
            -F_y    \\ 
            F_x    
        \end{pmatrix}\cdot \sigma'.
    \end{equation}
    L'intégrale de cette dernière fonction sur le chemin $\sigma$ est 
    \begin{equation}
        \begin{aligned}[]
            I&=\int_{\sigma} F\cdot n\\
            &=\int_{\sigma}\frac{1}{ \| \sigma' \| }\begin{pmatrix}
                -F_y    \\ 
                F_x    
            \end{pmatrix}\cdot \sigma'\\
            &=  \int_a^b\frac{1}{ \| \sigma'(t)\| }\begin{pmatrix}
                -F_y\big( \sigma(t) \big)    \\ 
                F_x\big( \sigma(t) \big)
            \end{pmatrix}
            \cdot\sigma'(t)\| \sigma'(t) \|dt\\
            &=
            \int_a^b\begin{pmatrix}
                -F_y    \\ 
                F_x    
            \end{pmatrix}\cdot \sigma'(t)dt.
        \end{aligned}
    \end{equation}
    Cette dernière intégrale est la circulation du champ de vecteurs $\begin{pmatrix}
        -F_y    \\ 
        F_x    
    \end{pmatrix}$ sur le chemin $\sigma$. Le théorème de Green \ref{ThoGreenVecto} nous enseigne que la circulation le long d'un chemin est égale au flux du rotationnel à travers la surface. Par conséquent,
    \begin{equation}
        I=\iint_D\left( \nabla\times\begin{pmatrix}
            -F_y    \\ 
            F_x    
        \end{pmatrix}\right)\cdot dS=\iint_D\nabla\cdot F\, dxdy
    \end{equation}
    

\end{proof}

%+++++++++++++++++++++++++++++++++++++++++++++++++++++++++++++++++++++++++++++++++++++++++++++++++++++++++++++++++++++++++++
\section{Théorème de Stokes}
%+++++++++++++++++++++++++++++++++++++++++++++++++++++++++++++++++++++++++++++++++++++++++++++++++++++++++++++++++++++++++++

Nous nous mettons maintenant dans $\eR^3$, et nous y considérons une surface paramétrée $S$ donc le bord est $\partial S$. 

\begin{theorem}[Théorème de Stokes]
    Alors le flux du rotationnel de $F$ à travers $S$ est égal à la circulation de $F$ le long du bord. En formule :
    \begin{equation}
        \iint_S\nabla\times F\cdot dS=\int_{\partial S} F\cdot d\sigma.
    \end{equation}
\end{theorem}

Nous pouvons nous donner une idée du pourquoi ce théorème est vrai. D'abord, si la surface est plate, cela est exactement le théorème de Green \ref{ThoGreenVecto}. Supposons maintenant que le bord reste plat, mais que la surface se déforme un petit peu. Le chemin
\begin{equation}
    \sigma(t)=\begin{pmatrix}
        \cos(t)    \\ 
        \sin(t)    \\ 
        0    
    \end{pmatrix}
\end{equation}
est tout autant le bord du disque plat de rayon $1$ que celui de la demi-sphère
\begin{equation}
    \phi(x,y)=\begin{pmatrix}
        x    \\ 
        y    \\ 
        \sqrt{1-x^2-y^2}    
    \end{pmatrix}.
\end{equation}
Le champ de vecteur que nous considérons est $G=\nabla\times F$. Il a un certain flux à travers le disque plat, et ce plus est égal à la circulation de $F$ sur $\sigma$. Quel est le flux de $G$ à travers la demi-sphère ? Étant donné que $\nabla\cdot G=\nabla\cdot(\nabla\times F)=0$, le champ de vecteurs $G$ est incompressible, de telle façon à ce que tout ce qui rentre dans la demi-sphère doit en sortir. Le flux de $G$ à travers la demi-sphère doit par conséquent être égal à celui à travers le disque plat.


\begin{example}

    Soit $C$ l'intersection entre le cylindre $x^2+y^2=1$ et le plan $x+y+z=1$. Calculer la circulation de
    \begin{equation}
        F(x,y,z)=\begin{pmatrix}
            -y^3    \\ 
            x^3    \\ 
            -z^3    
        \end{pmatrix}
    \end{equation}
    le long de $C$. 

    Au lieu de calculer directement
    \begin{equation}
        \int_{C}F\cdot d\sigma,
    \end{equation}
    nous allons calculer
    \begin{equation}
        \int_S\nabla\times F\cdot dS
    \end{equation}
    où $S$ est une surface dont $C$ est le bord. Cette intégrale est à calculer avec la formule \eqref{EqResIntFluxPhi}.

    La première chose à faire est de trouver une surface dont le bord est $C$ et en trouver une paramétrisation $\phi$. Le plus simple est de prendre le graphe du plan sur le cercle $x^2+y^2+1$. Une paramétrisation de cette surface est simplement
    \begin{equation}
        \begin{aligned}
            \phi\colon D&\to \eR^3 \\
            (x,y)&\mapsto \begin{pmatrix}
                x    \\ 
                y    \\ 
                1-x-y    
            \end{pmatrix}
        \end{aligned}
    \end{equation}
    où $D$ est le disque de rayon $1$. Étant donné que cela paramètre le plan $x+y+z-1=0$, le vecteur normal est $n=e_x+e_y+z_z$. Nous pouvons cependant calculer ce vecteur normal en suivant la recette usuelle. D'abord les vecteurs tangents sont
    \begin{equation}
        \begin{aligned}[]
            \frac{ \partial \phi }{ \partial x }&=\begin{pmatrix}
                1    \\ 
                0    \\ 
                -1    
            \end{pmatrix},
            &\frac{ \partial \phi }{ \partial y }&=\begin{pmatrix}
                0    \\ 
                1    \\ 
                -1    
            \end{pmatrix}.
        \end{aligned}
    \end{equation}
    Et le vecteur normal est donné par le produit vectoriel :
    \begin{equation}
        \begin{aligned}[]
            n&=\frac{ \partial \phi }{ \partial x }\times\frac{ \partial \phi }{ \partial y }\\
            &=\begin{vmatrix}
                e_x    &   e_y    &   e_z    \\
                1    &   0    &   -1    \\
                0    &   1    &   -1
            \end{vmatrix}\\
            &=e_x+e_y+z_z.
        \end{aligned}
    \end{equation}

    Ensuite, le rotationnel de $F$ est donné par
    \begin{equation}
        \nabla\times F=3(x^2+y^2)e_z.
    \end{equation}
    Par conséquent,
    \begin{equation}
        \nabla\times F\cdot\left( \frac{ \partial \phi }{ \partial x }\times\frac{ \partial \phi }{ \partial y } \right)=3(x^2+y^2).
    \end{equation}
    L'intégrale à calculer est donc
    \begin{equation}
        \begin{aligned}[]
            \iint_S\nabla\times F\cdot dS&=\iint_D(\nabla\times F)\big( \phi(x,y) \big)\cdot\left( \frac{ \partial \phi }{ \partial x }\times\frac{ \partial \phi }{ \partial y } \right)dxdy\\
            &=3\int_D(x^2+y^2)dxdy.
        \end{aligned}
    \end{equation}
    Cette dernière intégrale est l'intégrale d'une fonction sur le disque de rayon $1$. Elle s'effectue en passant aux coordonnées polaires :
    \begin{equation}
        3\int_D(x^2+y^2)dxdy=\int_0^{2\pi}d\theta\int_0^1(r^2)r\,dr=\frac{ 3\pi }{2}.
    \end{equation}
\end{example}

%+++++++++++++++++++++++++++++++++++++++++++++++++++++++++++++++++++++++++++++++++++++++++++++++++++++++++++++++++++++++++++
\section{Théorème de Gauss}
%+++++++++++++++++++++++++++++++++++++++++++++++++++++++++++++++++++++++++++++++++++++++++++++++++++++++++++++++++++++++++++

Soit $V$ une partie de $\eR^3$ délimitée par une surface $S$ sur laquelle nous considérons la normale extérieure. Soit $F$ un champ de vecteurs sur $\eR^3$.

\begin{theorem}[Théorème de la divergence ou de Gauss]
    Le flux d'un champ de vecteur $F$ à travers une surface fermée est égale à l'intégrale de la divergence sur le volume correspondant :
    \begin{equation}
        \int_{\partial V} F\cdot dS=\iiint_V\nabla\cdot F\,dxdydz.
    \end{equation}
\end{theorem}

Ce théorème signifie que la quantité de fluide qui s'accumule dans le volume (le flux est ce qui rentre moins ce qui sort) est égal à l'intégrale de $\nabla\cdot F$ sur le volume, alors que nous savons que, localement, la quantité $\nabla\cdot F(x,y,z)$ est la quantité de fluide qui s'accumule au point $(x,y,z)$.

\begin{remark}
    Ce théorème ne fonctionne qu'avec des surfaces fermées. Essayer de l'appliquer au calcul de flux à travers des surfaces ouvertes n'a pas de sens parce qu'une surface ouverte ne délimite pas un volume.
\end{remark}

\begin{example}
    Calculer le flux du champ de vecteurs
    \begin{equation}
        F(x,y,z)=\begin{pmatrix}
            2x    \\ 
            y^2    \\ 
            z^2    
        \end{pmatrix}
    \end{equation}
    à travers la sphère de rayon $1$ centrée à l'origine. Nous utilisons le théorème de la divergence
    \begin{equation}
        \iint_S F\cdot n\,dS=\iiint_B\nabla \cdot F\,dxdydz
    \end{equation}
    où $S$ est la sphère et $B$ est la boule (la sphère pleine). La divergence de $F$ se calcule :
    \begin{equation}
        \nabla\cdot F=\frac{ \partial F_x }{ \partial x }+\frac{ \partial F_y }{ \partial y }+\frac{ \partial F_z }{ \partial z }=2+2x+2y.
    \end{equation}
    L'intégrale est donc en trois termes :
    \begin{equation}
        \begin{aligned}[]
            \iiint_B2=2\text{Volume(B)}=\frac{ 8\pi }{ 3 }\\
            \iiint_By\,dxdydz=0\\
            \iiint_Bz\,dxdydz=0.
        \end{aligned}
    \end{equation}
\end{example}

Dans certains cas le théorème de Gauss permet de simplifier le calcul de l'intégrale d'une fonction sur une surface.

\begin{example}
    Soit à calculer l'intégrale
    \begin{equation}
        I=\iint_{\partial B}(x^2+y+z)dS,
    \end{equation}
    c'est à dire l'intégrale de la fonction $x^2+y+z$ sur la sphère. Le vecteur normal à la sphère est
    \begin{equation}
        n=xe_x+ye_y+ze_z.
    \end{equation}
    Étant donné que nous sommes sur la sphère de rayon $1$, ce vecteur est même normé. La fonction que nous regardons n'est rien d'autre que $F\cdot n$ avec
    \begin{equation}
        F=\begin{pmatrix}
            x    \\ 
            1    \\ 
            1    
        \end{pmatrix}.
    \end{equation}
    Nous pouvons donc simplement intégrer $\nabla\cdot F$ sur toute la boule :
    \begin{equation}
        I=\iiint_{B}\nabla\cdot F\,dxdydz=\iiint_B 1\,dxdudz=\frac{ 4\pi }{ 3 }.
    \end{equation}
\end{example}
% This is part of the Exercices et corrigés de mathématique générale.
% Copyright (C) 2009-2013
%   Laurent Claessens
% See the file fdl-1.3.txt for copying conditions.

%+++++++++++++++++++++++++++++++++++++++++++++++++++++++++++++++++++++++++++++++++++++++++++++++++++++++++++++++++++++++++++
\section{Techniques d'intégration}
%+++++++++++++++++++++++++++++++++++++++++++++++++++++++++++++++++++++++++++++++++++++++++++++++++++++++++++++++++++++++++++

%---------------------------------------------------------------------------------------------------------------------------
\subsection{Reformer un carré au dénominateur}
%---------------------------------------------------------------------------------------------------------------------------
\label{subsecCarreDenoPar}

Lorsqu'on a un second degré au dénominateur, le bon plan est de reformer un carré parfait. Par exemple : 
\begin{equation}
	x^2+2x+2=(x+1)^2+1.
\end{equation}
Ensuite, le changement de variable $t=x+1$ est pratique parce que cela donne $t^2+1$ au dénominateur.

Cherchons
\begin{equation}
	I=\int \frac{ 1-x }{ x^2+2x+2 }dx=\int\frac{ 1-x }{ (x+1)^2+1 }dx=\int\frac{ 1-(t-1) }{ t^2+1 }
\end{equation}
où nous avons fait le changement de variable $t=x+1$, $dt=dx$. L'intégrale se coupe maintenant en deux parties :
\begin{equation}
	I=\int\frac{ -t }{ t^2+1 }+\int \frac{ 2 }{ t^2+1 }.
\end{equation}
La seconde est dans les formulaires et vaut 
\begin{equation}
	2\arctan(t)=2\arctan(x+1),
\end{equation}
tandis que la première est presque de la forme $f'/f$ :
\begin{equation}
	\int\frac{ t }{ t^2+1 }=\frac{ 1 }{2}\int \frac{ 2t }{ t^2+1 }=\frac{ 1 }{2}\ln(t^1+1)=\frac{ 1 }{2}\ln(u^2+2u+2).
\end{equation}

%+++++++++++++++++++++++++++++++++++++++++++++++++++++++++++++++++++++++++++++++++++++++++++++++++++++++++++++++++++++++++++
					\section{Primitives et surfaces}
%+++++++++++++++++++++++++++++++++++++++++++++++++++++++++++++++++++++++++++++++++++++++++++++++++++++++++++++++++++++++++++

Soit $f\colon \eR\to \eR$, une fonction continue, et $x\in\eR$. Pour chaque $x\in\eR$, nous pouvons considérer le nombre $F(x)$ défini par
\begin{equation}
	F(x)=\int_a^x f(t)dt.
\end{equation}


\newcommand{\CaptionFigSurfacePrimiteGeog}{Surface sous une courbe.}
\input{Fig_SurfacePrimiteGeog.pstricks}

La fonction $F$ ainsi définie a deux importantes propriétés :
\begin{enumerate}

\item
C'est une primitive de $f$,
\item
Elle donne la surface en dessous de $f$ entre les points $a$ et $x$, voir la figure \ref{LabelFigSurfacePrimiteGeog}.

\end{enumerate}

Notons que tant que $f$ est positive, la surface est croissante.

La manière de calculer la surface comprise entre deux fonctions est dessinée à la figure \ref{LabelFigSurfaceEntreCourbes}.
\newcommand{\CaptionFigSurfaceEntreCourbes}{Le calcul de la surface comprise entre deux fonctions.}
\input{Fig_SurfaceEntreCourbes.pstricks}

La surface entre les deux fonctions $y_1(x)$ et $y_2(x)$ se calcule comme suit.
\begin{enumerate}

\item
On calcule les intersections entre $y1$ et $y_2$. Notons $a$ et $b$ les ordonnées obtenues.
\item
La surface demandée est la différence entre la surface sous la fonction $y_1$ (la plus grande) et la surface sous la fonction $y_2$ (la plus petite), donc
\begin{equation}
	S=\int_{a}^by_1-\int_a^by_1.
\end{equation}

\end{enumerate}

%---------------------------------------------------------------------------------------------------------------------------
					\subsection{Longueur d'arc de courbe}
%---------------------------------------------------------------------------------------------------------------------------

La longueur de l'arc de courbe de la fonction $y=f(x)$ entre les abscisses $x_0$ et $x_1$ est donné par la formule
\begin{equation}		\label{EqLongArcCourbe}
	l(x_0,x_1)=\int_{x_0}^{x_1}\sqrt{1+y'(t)^2}dt.
\end{equation}

Lorsque la courbe est donnée sous forme paramétrique
\begin{subequations}
\begin{numcases}{}
	x=x(t)\\
	y=y(t),
\end{numcases}
\end{subequations}
alors la formule devient
\begin{equation}		\label{EqLongArcParam}
	l(t_1,t_2)=\int_{t_1}^{t_2}\sqrt{\dot x(t)^2+\dot y(t)^2}dt,
\end{equation}
où $\dot x(t)=x'(t)$.

%---------------------------------------------------------------------------------------------------------------------------
					\subsection{Aire de révolution}
%---------------------------------------------------------------------------------------------------------------------------

Pour savoir l'aire engendrée par la ligne $y=f(x)$ entre $a$ et $b$ autour de l'axe $Ox$, on utilise la formule
\begin{equation}
	S=2\pi\int_a^b\sqrt{1+f'(x)^2}f(x)dx.
\end{equation}

\section{Fonctions réelles de deux variables réelles}

Une \textbf{fonction réelle de 2 variables réelles} est une fonction $f : A \subset \eR^2 \to \eR : (x,y) \mapsto z = f(x,y)$.

Le \textbf{graphe de $f$}, noté $\Graphe f$, est un sous-ensemble de $\eR^3$:\[\Graphe f = \{(x,y,z) \in \eR^3 \mid (x,y) \in A \text{ et } z = f(x,y)\}\]

Les \textbf{courbes de niveau} de la fonction $f$ sont obtenues en posant $f(x,y)=\lambda$.

%---------------------------------------------------------------------------------------------------------------------------
\subsection{Limites de fonctions à deux variables}
%---------------------------------------------------------------------------------------------------------------------------

Ici nous n'allons pas entrer dans tous les détails, mais simplement mentionner les quelque techniques les plus courantes. 

\begin{theorem}		\label{ThoLimiteCompose}
	Soient deux fonctions $f\colon \eR^n\to \eR^p$ et $g\colon \eR^p\to \eR^q$. Si $a$ est un point adhérent au domaine de $g\circ f$ et si
	\begin{equation}
		\begin{aligned}[]
			\lim_{x\to a}f(x)&=b\\
			\lim_{y\to b}g(y)&=c,
		\end{aligned}
	\end{equation}
	alors 
	\begin{equation}
		\lim_{x\to a}(g\circ f)(x)=c.
	\end{equation}
\end{theorem}

Les techniques usuelles sont
\begin{enumerate}

	\item
		La règle de l'étau. Cette technique demande un peu plus d'imagination parce qu'il faut penser à un «truc» différent pour chaque exercice. En revanche, la justification est facile : il y a un théorème qui dit que ça marche.

	\item
		Lorsqu'on applique la règle de l'étau, penser à
		\begin{equation}
			| x |=\sqrt{x^2}\leq\sqrt{x^2+y^2}.
		\end{equation}
		Cela permet de majorer le numérateur. Attention : ce genre de majoration ne fonctionnent qu'au numérateur : agrandir le dénominateur ferait diminuer la fraction.

	\item
		Il n'est pas vrai que
		\begin{equation}
			| x |=\sqrt{x^2}\leq\sqrt{x^4}\leq\sqrt{x^4+2y^4}.
		\end{equation}
		En effet, si $x$ est petit, alors $x^2>x^4$, et non le contraire.

\end{enumerate}

Une technique très efficace pour les limites $(x,y)\to (0,0)$ est le passage aux coordonnées polaires. Il s'agit de poser
\begin{subequations}
	\begin{numcases}{}
		x=r\cos(\theta)\\
		y=r\sin(\theta)
	\end{numcases}
\end{subequations}
et puis de faire la limite $r\to 0$.

Si la limite obtenue {\bf ne dépend pas de $\theta$}, alors c'est la limite cherchée. Des exemples sont donnés dans les corrections de l'exercice \ref{exoFoncDeuxVar0010}.  % Oui, je sais, cet exercices n'est pas pour Outils math, donc il y a une référence qui manque.

%---------------------------------------------------------------------------------------------------------------------------
\subsection{Dérivées partielles}
%---------------------------------------------------------------------------------------------------------------------------

La \defe{dérivée partielle}{dérivée partielle} par rapport à $x$ au point $(x,y)$ est notée
\begin{equation}
	\frac{\partial f}{\partial x}(x,y) 
\end{equation}
et se calcule en dérivant $f$ par rapport  à $x$ en considérant que $y$ est constante.

De la même manière, la dérivée partielle par rapport à $y$ au point $(x,y)$ est notée
\begin{equation}
	\frac{\partial f}{\partial y}(x,y) 
\end{equation}
et se calcule en dérivant $f$ par rapport  à $y$ en considérant que $x$ est constante.

Pour les dérivées partielles secondes,
\begin{itemize}
\item $f''_{xx} (x,y) = (f'_x)'_x = \frac{\partial^2 f}{\partial x^2}(x,y) = \frac{\partial}{\partial x}(\frac{\partial f}{\partial x})$.
\item $f''_{yy} (x,y) = (f'_y)'_y = \frac{\partial^2 f}{\partial y^2}(x,y) = \frac{\partial}{\partial y}(\frac{\partial f}{\partial y})$.
\item $f''_{xy} (x,y) = (f'_x)'_y  = (f'_y)'_x = f''_{yx} (x,y) \text{ ou } \frac{\partial^2 f}{\partial x \partial y}(x,y) = \frac{\partial}{\partial x}(\frac{\partial f}{\partial y})  = \frac{\partial}{\partial y}(\frac{\partial f}{\partial x}) =\frac{\partial^2 f}{\partial y \partial x}(x,y)$.
\end{itemize}

%---------------------------------------------------------------------------------------------------------------------------
\subsection{Différentielle et accroissement}
%---------------------------------------------------------------------------------------------------------------------------

La \defe{différentielle totale}{différentielle!totale} de $f$ au point $(a,b)$ est donnée, quand elle existe (!), par la formule
\begin{equation}
	df(a,b) = \frac{\partial f}{\partial x}(a,b)dx + \frac{\partial f}{\partial y}(a,b) dy.
\end{equation}

De la même façon que la formule des accroissements finis disait que $f(x+a)\simeq f(x)+af'(x)$, en deux dimensions nous avons que l'\defe{accroissement}{accroissement} approximatif de $f$ au point $(a,b)$ pour des accroissements $\Delta x$ et $\Delta y$ est 
\begin{equation}
	f(x+\Delta x,y+\Delta y)=f(x,y)+\Delta x\frac{ \partial f }{ \partial x }(x,y)+\Delta y\frac{ \partial f }{ \partial y }(x,y).
\end{equation}

%TODO : pour l'index, l'expression régulière suivante aide :
% grep "defe{[A-Za-z ]*}{[A-Z]" *.tex
Le \defe{plan tangent}{plan!tangent} au graphe de $f$ au point $\big(a,b,f(a,b)\big)$ est 
\begin{equation}
	T_{(a,b)}(x,y) = f(a,b) + \frac{\partial f}{\partial x}(a,b) (x-a) + \frac{\partial f}{\partial y}(a,b) (y-b)
\end{equation}
essayez d'écrire l'équation de la droite tangente au graphe de $f(x)$ au point $x=a$ en terme de la dérivée de $f$, et comparez votre résultat à cette formule.

Un des principaux théorèmes pour tester la différentiabilité d'une fonction est le suivant.

\begin{theorem}		\label{ThoProuverDiffable}
	Soit une fonction $f\colon \eR^m\to \eR^p$. Si les dérivées partielles existent dans un voisinage de $a$ et donc continues en $a$, alors $f$ est différentiable en $a$.
\end{theorem}
Le plus souvent, nous prouvons qu'une fonction est différentiable en calculant les dérivées partielles et en montrant qu'elles sont continues.

%---------------------------------------------------------------------------------------------------------------------------
\subsection{Recherche d'extrema locaux}
%---------------------------------------------------------------------------------------------------------------------------

\begin{enumerate}
\item Rechercher les points critiques, càd les $(x,y)$ tels que
\[\begin{cases} \frac{\partial f}{\partial x}(x,y) = 0 \\ \frac{\partial f}{\partial y}(x,y) = 0 \end{cases} \]
En effet, si $(x_0,y_0)$ est un extrémum local de $f$, alors $\frac{\partial f}{\partial x}(x_0,y_0) = 0 = \frac{\partial f}{\partial y}(x_0,y_0)$.
\item Déterminer la nature des points critiques: «test» des dérivées secondes:
\[\text{On pose }H(x_0,y_0) = \frac{\partial^2 f}{\partial x^2}(x_0,y_0)\frac{\partial f^2}{\partial y^2}(x_0,y_0) - \left(\frac{\partial^2 f}{\partial x\partial y}(x_0,y_0)\right)^2\]
\begin{enumerate}
\item Si $H(x_0,y_0) > 0$ et $\frac{\partial^2 f}{\partial x^2}(x_0,y_0) > 0 \Longrightarrow (x_0,y_0)$ est un minimum local de $f$.
\item Si $H(x_0,y_0) > 0$ et $\frac{\partial^2 f}{\partial x^2}(x_0,y_0) < 0 \Longrightarrow (x_0,y_0)$ est un maximum local de $f$.
\item Si $H(x_0,y_0) < 0 \Longrightarrow f$ a un point de selle en $(x_0,y_0)$.
\item Si $H(x_0,y_0) = 0 \Longrightarrow$ on ne peut rien conclure.
\end{enumerate}
\end{enumerate}

\textbf{Dérivation implicite:} Soit $F(x,f(x)) = 0$ la représentation implicite d'une fonction $y=f(x)$ alors \[y' = f'(x) = - \frac{F'_x}{F'_y}.\]

%+++++++++++++++++++++++++++++++++++++++++++++++++++++++++++++++++++++++++++++++++++++++++++++++++++++++++++++++++++++++++++
\section{Méthode de Gauss pour résoudre des systèmes d'équations linéaires}
%+++++++++++++++++++++++++++++++++++++++++++++++++++++++++++++++++++++++++++++++++++++++++++++++++++++++++++++++++++++++++++


Pour résoudre un système d'équations linéaires, on procède comme suit:
\begin{enumerate}
\item Écrire le système sous forme matricielle. \[\text{p.ex. } \begin{cases} 2x+3y &= 5 \\ x+2y &= 4 \end{cases} \Leftrightarrow \left(\begin{array}{cc|c} 2 & 3 & 5 \\ 1 & 2 & 4 \end{array}\right) \]
\item Se ramener à une matrice avec un maximum de $0$ dans la partie de gauche en utilisant les transformations admissibles:
\begin{enumerate}
\item Remplacer une ligne par elle-même + un multiple d'une autre;
\[\text{p.ex. } \left(\begin{array}{cc|c} 2 & 3 & 5 \\ 1 & 2 & 4 \end{array}\right)  \stackrel{L_1  - 2. L_2 \mapsto L_1'}{\Longrightarrow} \left(\begin{array}{cc|c} 0 & -1 & -3 \\ 1 & 2 & 4 \end{array}\right) \]
\item Remplacer une ligne par un multiple d'elle-même;
\[\text{p.ex. } \left(\begin{array}{cc|c} 0 & -1 & -3 \\ 1 & 2 & 4 \end{array}\right)  \stackrel{-L_1  \mapsto L_1'}{\Longrightarrow} \left(\begin{array}{cc|c} 0 & 1 & 3 \\ 1 & 2 & 4 \end{array}\right) \]
\item Permuter des lignes.
\[\text{p.ex. } \left(\begin{array}{cc|c} 0 & 1 & 3 \\ 1 & 0 & -2 \end{array}\right)  \stackrel{L_1  \mapsto L_2' \text{ et } L_2  \mapsto L_1'}{\Longrightarrow} \left(\begin{array}{cc|c} 1 & 0 & -2 \\ 0 & 1 & 3  \end{array}\right) \]
\end{enumerate}
\item Retransformer la matrice obtenue en système d'équations.
\[\text{p.ex. }  \left(\begin{array}{cc|c} 1 & 0 & -2 \\ 0 & 1 & 3  \end{array}\right) \Leftrightarrow \begin{cases} x &= -2 \\ y &= 3 \end{cases}  \]
\end{enumerate}

\textbf{Remarques :} 
\begin{itemize}
\item Si on obtient une ligne de zéros, on peut l'enlever:
\[\text{p.ex. }  \left(\begin{array}{ccc|c} 3 & 4 & -2 & 2 \\ 4 & -1 & 3 & 0 \\ 0 & 0 & 0 & 0 \end{array}\right) \Leftrightarrow  \left(\begin{array}{ccc|c} 3 & 4 & -2 & 2 \\ 4 & -1 & 3 & 0 \end{array}\right) \]
\item Si on obtient une ligne de zéros suivie d'un nombre non-nul, le système d'équations n'a pas de solution:
\[\text{p.ex. }  \left(\begin{array}{ccc|c} 3 & 4 & -2 & 2 \\ 4 & -1 & 3 & 0 \\ 0 & 0 & 0 & 7 \end{array}\right) \Leftrightarrow  \begin{cases} \cdots \\ \cdots \\ 0x + 0y + 0z = 7 \end{cases} \Rightarrow \textbf{Impossible} \]
\item Si on moins d'équations que d'inconnues, alors il y a une infinité de solutions qui dépendent d'un ou plusieurs paramètres:
\[\text{p.ex. }  \left(\begin{array}{ccc|c} 1 & 0 & -2 & 2 \\ 0 & 1 & 3 & 0 \end{array}\right) \Leftrightarrow  \begin{cases} x - 2z = 2 \\ y + 3z = 0 \end{cases} \Leftrightarrow  \begin{cases} x = 2 + 2\lambda \\ y = -3\lambda \\ z = \lambda \end{cases} \]
\end{itemize}

%+++++++++++++++++++++++++++++++++++++++++++++++++++++++++++++++++++++++++++++++++++++++++++++++++++++++++++++++++++++++++++
\section{Matrices, applications linéaires et directions conservées}
%+++++++++++++++++++++++++++++++++++++++++++++++++++++++++++++++++++++++++++++++++++++++++++++++++++++++++++++++++++++++++++

Nous savons qu'une application \emph{linéaire} $A\colon \eR^3\to \eR^3$ est complètement définie par la donnée de son action sur les trois vecteurs de base, c'est à dire par la donnée de
\begin{equation}
	\begin{aligned}[]
		Ae_1,&&Ae_2&&\text{et}&&Ae_3.
	\end{aligned}
\end{equation}
Nous allons former la matrice de $A$ en mettant simplement les vecteurs $Ae_1$, $Ae_2$ et $Ae_3$ en colonne. Donc la matrice
\begin{equation}		\label{EqExempleALin}
	A=\begin{pmatrix}
		3	&	0	&	0	\\
		0	&	1	&	0	\\
		0	&	1	&	0
	\end{pmatrix}
\end{equation}
signifie que l'application linéaire $A$ envoie le vecteur $e_1$ sur $\begin{pmatrix}
	3	\\ 
	0	\\ 
	0	
\end{pmatrix}$, le vecteur $e_2$ sur $\begin{pmatrix}
	0	\\ 
	0	\\ 
	1	
\end{pmatrix}$ et le vecteur $e_3$ sur $\begin{pmatrix}
	0	\\ 
	1	\\ 
	0	
\end{pmatrix}$.
Pour savoir comment $A$ agit sur n'importe quel vecteur, on applique la règle de produit vecteur$\times$matrice :
\begin{equation}
	\begin{pmatrix}
		1	&	2	&	3	\\
		4	&	5	&	6	\\
		7	&	8	&	9
	\end{pmatrix}\begin{pmatrix}
		x	\\ 
		y	\\ 
		z	
	\end{pmatrix}=
	\begin{pmatrix}
		x+2y+3z	\\ 
		4x+5y+6z	\\ 
		7x+8y+9z	
	\end{pmatrix}.
\end{equation}

Une chose intéressante est de savoir quelles sont les directions invariantes de la transformation linéaire. Par exemple, on peut lire sur la matrice \eqref{EqExempleALin} que la direction $\begin{pmatrix}
	1	\\ 
	0	\\ 
	0	
\end{pmatrix}$ est invariante : elle est simplement multipliée par $3$. Dans cette direction, la transformation est juste une dilatation. Afin de savoir si $v$ est un vecteur d'une direction conservée, il faut voir si il existe un nombre $\lambda$ tel que $Av=\lambda v$, c'est à dire voir si $v$ est simplement dilaté.

L'équation $Av=\lambda v$ se récrit $(A-\lambda\mtu)v=0$, c'est à dire qu'il faut résoudre l'équation
\begin{equation}
	(A-\lambda\mtu)\begin{pmatrix}
		x	\\ 
		y	\\ 
		z	
	\end{pmatrix}=
	\begin{pmatrix}
		0	\\ 
		0	\\ 
		0	
	\end{pmatrix}.
\end{equation}
Nous savons qu'une telle équation ne peut avoir de solutions que si $\det(A-\lambda\mtu)=0$. La première étape est donc de trouver les $\lambda$ qui vérifient cette condition.


%---------------------------------------------------------------------------------------------------------------------------
\subsection{Comment trouver la matrice d'une symétrie donnée ?}
%---------------------------------------------------------------------------------------------------------------------------
\label{SubSecMtrSym}

Ceci est une FAQ (Faut Avoir Quompri). 

%///////////////////////////////////////////////////////////////////////////////////////////////////////////////////////////
\subsubsection{Symétrie par rapport à un plan}
%///////////////////////////////////////////////////////////////////////////////////////////////////////////////////////////

Comment trouver par exemple la matrice $A$ qui donne la symétrie autour du plan $z=0$ ? La définition d'une telle symétrie est que les vecteurs du plan $z=0$ ne bougent pas, tandis que les vecteurs perpendiculaires changent de signe. Ces informations vont permettre de trouver comment $A$ agit sur une base de $\eR^3$. En effet :
\begin{enumerate}

	\item
		Le vecteur $\begin{pmatrix}
			1	\\ 
			0	\\ 
			0	
		\end{pmatrix}$ est dans le plan $z=0$, donc il ne bouge pas,

	\item
		le vecteur $\begin{pmatrix}
			0	\\ 
			1	\\ 
			0	
		\end{pmatrix}$ est également dans le plan, donc il ne bouge pas non plus,

	\item
		et le vecteur $\begin{pmatrix}
			0	\\ 
			0	\\ 
			1	
		\end{pmatrix}$ est perpendiculaire au plan $z=0$, donc il va changer de signe.

\end{enumerate}
Cela nous donne directement les valeurs de $A$ sur la base canonique et nous permet d'écrire 
\begin{equation}
	A=\begin{pmatrix}
		1	&	0	&	0	\\
		0	&	1	&	0	\\
		0	&	0	&	-1
	\end{pmatrix}.
\end{equation}
Pour écrire cela, nous avons juste mit en colonne les images des vecteurs de base. Les deux premiers n'ont pas changé et le troisième a changé.

Et si maintenant on donne un plan moins facile que $z=0$ ? Le principe reste le même : il faudra trouver deux vecteurs qui sont dans le plan (et dire qu'ils ne bougent pas), et puis un vecteur qui est perpendiculaire au plan\footnote{Pour le trouver, penser au produit vectoriel.}, et dire qu'il change de signe.

Voyons ce qu'il en est pour le plan $x=-z$. Il faut trouver deux vecteurs linéairement indépendants dans ce plan. Prenons par exemple
\begin{equation}		\label{EqffudE}
	\begin{aligned}[]
		f_1&=\begin{pmatrix}
			0	\\ 
			1	\\ 
			0	
		\end{pmatrix},&f_2&=\begin{pmatrix}
			1	\\ 
			0	\\ 
			-1	
		\end{pmatrix}.
	\end{aligned}
\end{equation}
Nous avons 
\begin{equation}
	\begin{aligned}[]
		Af_1&=f_1\\
		Af_2&=f_2.
	\end{aligned}
\end{equation}
Afin de trouver un vecteur perpendiculaire au plan, calculons le produit vectoriel :
\begin{equation}
	f_3=f_1\times f_2=\begin{vmatrix}
		e_1	&	e_2	&	e_3	\\
		0	&	1	&	0	\\
		1	&	0	&	-1
	\end{vmatrix}=-e_1-e_3=\begin{pmatrix}
		-1	\\ 
		0	\\ 
		-1	
	\end{pmatrix}.
\end{equation}
Nous avons 
\begin{equation}
	Af_3=-f_3.
\end{equation}
Afin de trouver la matrice $A$, il faut trouver $Ae_1$, $Ae_2$ et $Ae_3$. Pour ce faire, il faut d'abord écrire $\{ e_1,e_2,e_3 \}$ en fonction de $\{ f_1,f_2,f_3 \}$. La première des équation \eqref{EqffudE} dit que 
\begin{equation}
	f_1=e_2.
\end{equation}
Ensuite, nous avons
\begin{equation}
	\begin{aligned}[]
		f_2&=e_1-e_3\\
		f_3&=-e_1-e_3.
	\end{aligned}
\end{equation}
La somme de ces deux équations donne $-2e_3=f_2+f_3$, c'est à dire
\begin{equation}
	e_3=-\frac{ f_2+f_3 }{ 2 }
\end{equation}
Et enfin, nous avons
\begin{equation}
	e_1=\frac{ f_2-f_3 }{ 2 }.
\end{equation}

Maintenant nous pouvons calculer les images de $e_1$, $e_2$ et $e_3$ en faisant
\begin{equation}
	\begin{aligned}[]
		Ae_1&=\frac{ Af_2-Af_3 }{ 2 }=\frac{1 }{2}\begin{pmatrix}
			0	\\ 
			0	\\ 
			-2	
		\end{pmatrix}=\begin{pmatrix}
			0	\\ 
			0	\\ 
			-1	
		\end{pmatrix},\\
		Ae_2&=Af_1=f_1=\begin{pmatrix}
			0	\\ 
			1	\\ 
			0	
		\end{pmatrix},\\
		Ae_3&=-\frac{ f_2-f_3 }{ 2 }=-\frac{ 1 }{2}\begin{pmatrix}
			2	\\ 
			0	\\ 
			0	
		\end{pmatrix}=\begin{pmatrix}
			-1	\\ 
			0	\\ 
			0	
		\end{pmatrix}.
	\end{aligned}
\end{equation}
La matrice $A$ s'écrit maintenant en mettant les trois images trouvées en colonnes :
\begin{equation}
	A=\begin{pmatrix}
		0	&	0	&	-1	\\
		0	&	1	&	0	\\
		-1	&	0	&	0
	\end{pmatrix}.
\end{equation}

%///////////////////////////////////////////////////////////////////////////////////////////////////////////////////////////
\subsubsection{Symétrie par rapport à une droite}
%///////////////////////////////////////////////////////////////////////////////////////////////////////////////////////////

Le principe est exactement le même : il faut trouver trois vecteurs $f_1$, $f_2$ et $f_3$ sur lesquels on connaît l'action de la symétrie. Ensuite il faudra exprimer $e_1$, $e_2$ et $e_3$ en termes de $f_1$, $f_2$ et $f_3$.

Le seul problème est de trouver les trois vecteurs $f_i$. Le premier est tout trouvé : c'est n'importe quel vecteur sur la droite. Pour les deux autres, il faut un peu ruser parce qu'il faut impérativement qu'ils soient perpendiculaire à la droite. Pour trouver $f_2$, on peut écrire
\begin{equation}
	f_2=\begin{pmatrix}
		1	\\ 
		0	\\ 
		x	
	\end{pmatrix},
\end{equation}
et puis fixer le $x$ pour que le produit scalaire de $f_2$ avec $f_1$ soit nul. Si il n'y a pas moyen (genre si $f_1$ a sa troisième composante nulle), essayer avec $\begin{pmatrix}
	x	\\ 
	1	\\ 
	0	
\end{pmatrix}$. Une fois que $f_2$ est trouvé (il y a des milliards de choix possibles), trouver $f_3$ est super facile : prendre le produit vectoriel entre $f_1$ et $f_2$.

%///////////////////////////////////////////////////////////////////////////////////////////////////////////////////////////
\subsubsection{En résumé}
%///////////////////////////////////////////////////////////////////////////////////////////////////////////////////////////


La marche à suivre est

\begin{enumerate}

	\item
		Trouver trois vecteurs $f_1$, $f_2$ et $f_3$ sur lesquels on connaît l'action de la symétrie. Typiquement : des vecteurs qui sont sur l'axe ou le plan de symétrie, et puis des perpendiculaires. Pour la perpendiculaire, penser au produit scalaire et au produit vectoriel.

	\item
		Exprimer la base canonique $e_1$, $e_2$ et $e_3$ en termes de $f_1$, $f_2$, $f_3$.

	\item
		Trouver $Ae_1$, $Ae_2$ et $Ae_3$ en utilisant leur expression en termes des $f_i$, et le fait que l'on connaisse l'action de $A$ sur les $f_i$.

	\item
		La matrice s'obtient en mettant les images des $e_i$ en colonnes.

\end{enumerate}

%+++++++++++++++++++++++++++++++++++++++++++++++++++++++++++++++++++++++++++++++++++++++++++++++++++++++++++++++++++++++++++
\section{Orthogonalité}
%+++++++++++++++++++++++++++++++++++++++++++++++++++++++++++++++++++++++++++++++++++++++++++++++++++++++++++++++++++++++++++

\begin{proposition}			\label{PropVectsOrthLibres}
	si $v_1,\cdots,v_k$ sont des vecteurs non nuls, orthogonaux deux à deux, alors ces vecteurs forment une famille libre.
\end{proposition}

