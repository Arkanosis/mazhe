% This is part of Mes notes de mathématique
% Copyright (c) 2011-2014
%   Laurent Claessens
% See the file fdl-1.3.txt for copying conditions.

%+++++++++++++++++++++++++++++++++++++++++++++++++++++++++++++++++++++++++++++++++++++++++++++++++++++++++++++++++++++++++++ 
\section{Intégrales convergeant uniformément}
%+++++++++++++++++++++++++++++++++++++++++++++++++++++++++++++++++++++++++++++++++++++++++++++++++++++++++++++++++++++++++++

%--------------------------------------------------------------------------------------------------------------------------- 
\subsection{Définition et propriété}
%---------------------------------------------------------------------------------------------------------------------------

Soit \( (\Omega,\mu)\) un espace mesuré. Nous disons que l'intégrale
\begin{equation}
    \int_{\Omega}f(x,\omega)d\mu(\omega)
\end{equation}
\defe{converge uniformément}{convergence!uniforme!intégrale} en \( x\) si pour tout \( \epsilon>0\), il existe un compact \( K_0\) tel que pour tout compact \( K\) tel que \( K_0\subset K\) nous avons
\begin{equation}
    \left| \int_{\Omega\setminus K}f(x,\omega)d\mu(\omega) \right| \leq \epsilon.
\end{equation}
Le point important est que le choix de \( K_0\) ne dépend pas de \( x\).

\begin{lemma}       \label{LemOgQdpJ}
    Soit
    \begin{equation}
        F(x)=\int_{\Omega}f(x,\omega)d\mu(\omega),
    \end{equation}
    une intégrale uniformément convergente. Pour chaque \( k\in \eN\) nous considérons un compact \( K_k\) tel que
    \begin{equation}
        \left| \int_{\Omega\setminus K_k}f(x,\omega)d\mu(\omega) \right| \leq\frac{1}{ k }.
    \end{equation}
    Alors la suite de fonctions \( F_k\) définie par
    \begin{equation}
        F_k(x)=\int_{K_k}f(x,\omega)d\mu(\omega)
    \end{equation}
    converge uniformément vers \( F\).
\end{lemma}

\begin{proof}
    Nous avons
    \begin{subequations}
        \begin{align}
            \big| F_k(x)-F(x) \big|&=\left| \int_{K_k}f(x,\omega)d\mu(\omega)-\int_{\Omega}f(x,\omega)d\mu(\omega) \right| \\
            &=| \int_{\Omega\setminus K_k}f(x,\omega)d\mu(\omega) |\\
            &\leq \frac{1}{ k }.
        \end{align}
    \end{subequations}
\end{proof}

%------------------------------------------------------------------------------------------------------------------------
\subsection{Critères de convergence uniforme}
%---------------------------------------------------------------------------------------------------------------------------

Afin de tester l'uniforme convergence d'une intégrale, nous avons le \defe{critère de Weierstrass}{critère!Weierstrass}:
\begin{theorem}		\label{ThoCritWeiIntUnifCv}
Soit $f(x,t)\colon [\alpha,\beta]\times[a,\infty[ \to \eR$, une fonction dont la restriction à toute demi-droite $x=cst$ est mesurable. Si $| f(x,t) |< \varphi(t)$ et $\int_a^{\infty}\varphi(t)dt$ existe, alors l'intégrale
\begin{equation}
	\int_0^{\infty}f(x,t)dt
\end{equation}
est uniformément convergente.
\end{theorem}

Le théorème suivant est le \defe{critère d'Abel}{critère!Abel pour intégrales} :
\begin{theorem}		\label{ThoAbelIntUnif}
	Supposons que $f(x,t)=\varphi(x,t)\psi(x,t)$ où $\varphi$ et $\psi$ sont bornée et intégrables en $t$ au sens de Riemann sur tout compact $[a,b]$, $b\geq a$. Supposons que :
	\begin{enumerate}
		\item $| \int_a^{T}\varphi(x,t)dt |\leq M$ où $M$ est indépendant de $T$ et de $x$,
		\item $\psi(x,t)\geq 0$,
		\item pour tout $x\in[\alpha,\beta]$, $\psi(x,t)$ est une fonction décroissante de $t$,
		\item les fonctions $x\mapsto \psi(x,t)$ convergent uniformément vers $0$ lorsque $t\to\infty$.
	\end{enumerate}
	Alors l'intégrale
	\begin{equation}
		\int_a^{\infty}f(x,t)dt
	\end{equation}
	est uniformément convergente.
\end{theorem}

\begin{remark}
    Étant donné que la fonction sinus est bornée, il est tentant de l'utiliser comme $\varphi$ dans le critère d'Abel (théorème \ref{ThoAbelIntUnif}). Hélas,
    \begin{equation}
        \int_0^T\sin(xt)=-\frac{ 1 }{ x }\big( \cos(xT)-\cos(x) \big),
    \end{equation}
    qui n'est pas bornée pour tout $x$ ! Poser $\varphi(x,t)=\sin(xt)$ \emph{ne fonctionne pas} pour assurer la convergence uniforme sur un intervalle qui contient des $x$ arbitrairement proches de $0$. Le critère d'Abel avec $\varphi(x,t)=\sin(xt)$ ne permet que de conclure à l'uniforme convergence \emph{sur tout compact} ne contenant pas $0$. Cela est toutefois souvent suffisant pour étudier la continuité ou la dérivabilité en se servant du fameux coup du compact.
\end{remark}


%+++++++++++++++++++++++++++++++++++++++++++++++++++++++++++++++++++++++++++++++++++++++++++++++++++++++++++++++++++++++++++
\section{Fonctions définies par une intégrale}
%+++++++++++++++++++++++++++++++++++++++++++++++++++++++++++++++++++++++++++++++++++++++++++++++++++++++++++++++++++++++++++
\label{SecCHwnBDj}
\index{suite!de fonctions intégrables}
\index{fonction!définie par une intégrale}
\index{inversion de limite}

Soit \( (\Omega,\mu)\) un espace mesuré. Nous nous demandons dans quel cas l'intégrale
\begin{equation}
    F(x)=\int_{\Omega}f(x,\omega)d\omega
\end{equation}
définit une fonction \( F\) continue, dérivable ou autre. 

Dans la suite nous allons considérer des fonctions \( f\) à valeurs réelles. Quitte à passer aux composantes, nous pouvons considérer des fonctions à valeurs vectorielles. Par contre le fait que \( x\) soit dans \( \eR\) ou dans \( \eR^n\) n'est pas spécialement une chose facile à traiter.

%--------------------------------------------------------------------------------------------------------------------------- 
\subsection{Continuité sous l'intégrale}
%---------------------------------------------------------------------------------------------------------------------------
\index{continuité!fonction définie par une intégrale}

Nous allons présenter deux théorèmes donnant la continuité de \( F\).
\begin{enumerate}
    \item
        Si \( f\) est majorée par une fonction ne dépendant pas de \( x\), nous avons le théorème \ref{ThoKnuSNd},
    \item
        si l'intégrale est uniformément convergente, nous avons le théorème \ref{ThotexmgE}.
\end{enumerate}

\begin{theorem} \label{ThoKnuSNd}
    Soit \( (\Omega,\mu)\) est un espace mesuré, soit \( x_0\in \eR^m\) et \( f\colon U\times \Omega\to \eR\) où \( U\) est ouvert dans \( \eR^m\). Nous supposons que
    \begin{enumerate}
        \item
            La fonction \( f(x,.)\) est dans \( L^1(\Omega,\mu)\) pour tout \( x \in \eR^m\).
        \item
            La fonction \( f(.,\omega)\) est continue en \( x_0\) pour tout \( \omega\in\Omega\).
            %TODO : peut-être qu'on peut dire seulement pour presque tout omege dans Omega, voir la proposition \ref{prop:fdefint}.
        \item       \label{ItemNAuYNG}
            Il existe une fonction \( G\in L^1(\Omega)\) telle que
            \begin{equation}
                | f(x,\omega) |\leq G(\omega)
            \end{equation}
            pour tout \( x\in U\).
    \end{enumerate}
    Alors la fonction 
    \begin{equation}
        \begin{aligned}
            F\colon U&\to \eR \\
            x&\mapsto \int_{\Omega}f(x,\omega)d\mu(\omega) 
        \end{aligned}
    \end{equation}
    est continue en \( x_0\).
\end{theorem}
\index{permuter!limite et intégrale!espace mesuré}

\begin{proof}
    Soit \( (x_n)\) une suite convergente vers \( x_0\). Nous considérons la suite de fonctions \( f_n\colon \Omega\to \eR\) définies par
    \begin{equation}
        f_n(\phi)=f(x_n,\omega).
    \end{equation}
    sur qui nous pouvons utiliser le théorème de la convergence dominée (théorème \ref{ThoConvDomLebVdhsTf}) pour obtenir
    \begin{subequations}
        \begin{align}
            \lim_{n\to \infty} F(x_n)&=\lim_{n\to \infty} \int_{\Omega}f(x_n,\omega)d\mu(\omega)\\
            &=\int_{\Omega}\lim_{n\to \infty} f(x_n,\omega)d\mu(\omega)\\
            &=\int_{\Omega}f(x,\omega)d\mu(\omega)\\
            &=F(x).
        \end{align}
    \end{subequations}
    Nous avons utilisé la continuité de \( f(.,\omega)\).
\end{proof}


Si nous avons un peu de compatibilité entre la topologie et la mesure, alors nous pouvons utiliser l'uniforme convergence d'une intégrale pour obtenir la continuité d'une fonction définie par une intégrale.

\begin{theorem} \label{ThotexmgE}
    Soit \( (\Omega,\mu)\) un espace topologique mesuré tel que tout compact est de mesure finie. Soit une fonction \( f\colon \eR\times \Omega\to \eR\) telle que
    \begin{enumerate}
        \item
            Pour chaque \( x\in \eR\), la fonction \( f(x,.)\) est \( L^1(\Omega,\mu)\).
        \item
            Pour chaque \( \omega\in \Omega\), la fonction \( f(.,\omega)\) est continue en \( x_0\).
        \item
            L'intégrale
            \begin{equation}
                F(x)=\int_{\Omega}f(x,\omega)d\mu(\omega)
            \end{equation}
            est uniformément convergente.
    \end{enumerate}
    Alors la fonction \( F\) est continue en \( x_0\).
\end{theorem}
\index{permuter!limite et intégrale!espace mesuré}

\begin{proof}
    Nous reprenons les notations du lemme \ref{LemOgQdpJ}. Les fonctions
    \begin{equation}
        F_k(x)=\int_{K_k}f(x,\omega)d\mu(\omega)
    \end{equation}
    existent parce que les fonctions \( f(x,.)\) sont dans \( L^1(\Omega)\). Montrons que les fonctions \( F_k\) sont continues. Soit une suite \( x_k\to x_0\) nous avons
    \begin{equation}
        \lim_{n\to \infty} F_k(x_n)=\lim_{n\to \infty} \int_{K_k}f(x_n,\omega)d\mu(\omega).
    \end{equation}
    Nous pouvons inverser la limite et l'intégrale en utilisant le théorème de la convergence dominée. Pour cela, la fonction \( f(x_n,\omega)\) étant continue sur le compact \( K_k\), elle y est majorée par une constante. Le fait que les compacts soient de mesure finie (hypothèse) implique que les constantes soient intégrales sur \( K_k\). Le théorème de la convergence dominée implique alors que
    \begin{equation}
        \lim_{n\to \infty} F_k(x_n)=\int_{K_k}\lim_{n\to \infty} f(x_n,\omega)d\mu(\omega)=\int_{K_k}f(x_0,\omega)d\mu(\omega)=F_k(x_0).
    \end{equation}
    Nous avons utilisé le fait que \( f(.,\omega)\) était continue en \( x_0\).

    Le lemme \ref{LemOgQdpJ} nous indique alors que la convergence \( F_k\to F\) est uniforme. Les fonctions \( F_k\) étant continues, la fonction \( F\) est continue.
\end{proof}

Pour finir, citons ce résultat concernant les fonctions réelles.
\begin{theorem}		\label{ThoInDerrtCvUnifFContinue}
    Nous considérons \( F(x)=\int_a^{\infty}f(x,t)dt\). Si \( f\) est continue sur $[\alpha,\beta]\times[a,\alpha[$ et l'intégrale converge uniformément, alors $F(x)$ est continue.
\end{theorem}

%--------------------------------------------------------------------------------------------------------------------------- 
\subsection{Le coup du compact}
%---------------------------------------------------------------------------------------------------------------------------

Nous avons vu des fonctions définies par toute une série de processus de limite (suites, séries, intégrales). Une des questions centrales est de savoir si la fonction limite est continue, dérivable, intégrale, etc. étant donné que les fonctions sont continues.

Pour cela, nous inventons le concept de \emph{convergence uniforme}. Si la limite (série, intégrale) est uniforme, alors la fonction limite sera continue. Il arrive qu'une limite ne soit pas uniforme sur un intervalle ouvert $]0,1]$, et que nous voulions quand même prouver la continuité sur cet intervalle. C'est à cela que sert la notion de convergence uniforme \emph{sur tout compact}. En effet, la notion de continuité est une notion locale : savoir ce qu'il se passe dans un petit voisinage autour de $x$ est suffisant pour savoir la continuité en $x$ (idem pour sa dérivée).

Si nous avons uniforme convergence sur tout compact de $]0,1]$, mais pas uniforme convergence sur cet intervalle, la limite sera quand même continue sur $\mathopen] 0 , 1 \mathclose]$. En effet, si $x\in]0,1]$, il existe un ouvert autour de $x$ contenu dans un compact contenu dans $]0,1]$. L'uniforme convergence sur ce compact suffit à prouver la continuité en $x$.

Déduire la continuité sur un ouvert à partir de l'uniforme convergence sur tout compact de l'ouvert est appelé faire le \defe{coup du compact}{compact!le coup du}.


%---------------------------------------------------------------------------------------------------------------------------
\subsection{Dérivabilité sous l'intégrale}
%---------------------------------------------------------------------------------------------------------------------------
\index{dérivabilité!fonction définie par une intégrale}

Nous traitons à présent de la dérivabilité de la fonction \( F\) définie comme intégrale de \( f\).

\begin{theorem}[Dérivation sous le signe intégral\cite{MesIntProbb}]    \label{ThoMWpRKYp}
    Soit \( (\Omega,\mu)\) un espace mesuré et une fonction \( f\colon \eR\times \Omega\to \eR\) dont nous voulons étudier la dérivabilité en \(a\in \eR\). Nous supposons qu'il existe \( \delta>0\), \( A\) mesurable de mesure nulle dans \( \Omega\) tels que
    \begin{enumerate}
        \item
            \( f(x,\cdot)\) soit dans \( L^1(\Omega)\).
        \item
            L'application \( x\mapsto f(x,\omega)\) est dérivable pour tout \( x\in B(a,\delta)\) et pour tout \( \omega\in \complement A\).
        \item
            Il existe une fonction \( G\) intégrable sur \( \Omega\) telle que
            \begin{equation}
                \left| \frac{ \partial f }{ \partial x }(x,\omega) \right| \leq G(\omega)
            \end{equation}
            pour tout \( x\in B(a,\delta)\) et pour tout \( \omega\in\complement A\).
    \end{enumerate}
    Alors la fonction
    \begin{equation}
        F(x)=\int_{\Omega}f(x,\omega)d\mu(\omega)
    \end{equation}
    est dérivable en \( a\) et nous pouvons permuter la dérivée et l'intégrale :
    \begin{equation}
        F'(a)=\int_{\Omega}\frac{ \partial f }{ \partial x }(a,\omega)d\mu(\omega).
    \end{equation}
\end{theorem}
\index{permuter!dérivée et intégrale!dans \( \eR\)}

\begin{proof}
    Soit une suite \( (x_n)\) dans \( B(a,\delta)\) telle que \( x_n\neq a\) et \( x_n\to a\). Si la limite
    \begin{equation}
        \lim_{n\to \infty} \frac{ F(a)-F(x_n) }{ a-x_n }
    \end{equation}
    existe et ne dépend pas de la suite choisie, alors la fonction \( F\) est dérivable en \( a\) et sa dérivée vaut cette limite. Par linéarité de l'intégrale, nous devons étudier la limite
    \begin{equation}    \label{EqLIiralx}
        \lim_{n\to \infty} \int_{\Omega}\frac{ f(a,\omega)-f(x_n,\omega) }{ a-x_n }d\omega,
    \end{equation}
    montrer qu'elle existe, ne dépend pas de la suite choisie et vaut \( \int_{\Omega}\partial_xf(a,\omega)d\omega\). Nous sommes donc dans un problème d'inversion de limite et de dérivée pour lequel nous allons utiliser le théorème de la convergence dominée de Lebesgue. D'abord nous posons
    \begin{equation}    \label{EqAFOUbQB}
        g_n(\omega)=\frac{ f(x_n,\omega)-f(a,\omega) }{ x_n-a }.
    \end{equation}
    Cela est une suite de fonctions dans \( L^1(\Omega)\) parce qu'à la fois \( a\) et \( x_n\) sont dans \( B(a,\delta)\). De plus nous avons
    \begin{equation}
        \lim_{n\to \infty} g_n(\omega)=\frac{ \partial f }{ \partial x }(a,\omega)
    \end{equation}
    parce que nous savons que \( f\) est dérivable en \( a\) pour tout \( \omega\in\complement A\). En ce qui concerne la majoration de \( g_n\), nous utilisons le théorème des accroissements finis (théorème \ref{ThoAccFinis}) sur le numérateur de \eqref{EqAFOUbQB}. Pour tout \( n\) et pour tout \( \omega\in \complement A\), il existe un \( \theta_{n,\omega}\) dans \( \mathopen] a , x_n \mathclose[\) tel que
        \begin{equation}
            f(x_n,\omega)-f(a,\omega)=\frac{ \partial f }{ \partial x }(\theta_{n,\omega},\omega)(x_n-a),
        \end{equation}
        donc
        \begin{equation}
            | g_n(\omega) |=\left| \frac{ \partial f }{ \partial x }(\theta_{n,\omega},\omega) \right| \leq G(\omega).
        \end{equation}
        La dernière inégalité provient des hypothèses. Le théorème de la convergence dominée de Lebesgue (théorème \ref{ThoConvDomLebVdhsTf}) nous permet alors de calculer la limite \eqref{EqLIiralx} :
        \begin{equation}
            \lim_{n\to \infty} \int_{\Omega}g_n(\omega)d\omega=\int_{\Omega}\lim_{n\to \infty} g_n(\omega)d\omega=\int_{\Omega}\frac{ \partial f }{ \partial x }(a,\omega)d\omega.
        \end{equation}
        Notons que l'existence de la dernière intégrale fait partie du théorème de la convergence dominée.

        Nous avons donc prouvé que la limite de gauche existait et ne dépendant pas de la suite choisie. Donc \( F\) est dérivable en \( a\) et la dérivée vaut cette limite :
        \begin{equation}
            F'(a)=\int_{\Omega}\frac{ \partial f }{ \partial x }(a,\omega)d\mu(\omega).
        \end{equation}
\end{proof}

\begin{theorem}
		Supposons $f$ continue et sa dérivée partielle $\frac{ \partial f }{ \partial x }$ continue sur $[\alpha,\beta]\times[a,\alpha[$. Supposons que $F(x)=\int_a^{\infty}f(x,t)dt$ converge et que $\int_a^{\infty}\frac{ \partial f }{ \partial x }dt$ converge uniformément. Alors $F$ est $C^1$ sur $[\alpha,\beta]$ et 
		\begin{equation}
			\frac{ dF }{ dx }=\int_a^{\infty}\frac{ \partial f }{ \partial x }dt.
		\end{equation}
\end{theorem}

En ce qui concerne les fonctions dans \( \eR^n\), il y a les  propositions \ref{PropDerrSSIntegraleDSD} et \ref{PropAOZkDsh} qui parlent de différentiabilité sous l'intégrale.

%--------------------------------------------------------------------------------------------------------------------------- 
\subsection{Absolue continuité}
%---------------------------------------------------------------------------------------------------------------------------

\begin{definition}      \label{DefAbsoluCont}
    Une fonction \( F\colon \eR\to \eR\) est \defe{absolument continue}{absolument continue} sur \( \mathopen[ a , b \mathclose]\) si il existe une fonction \( f\) sur \( \mathopen[ a , b \mathclose]\) telle que
    \begin{equation}
        F(x)=\int_a^xf(t)dt
    \end{equation}
    pour tout \( x\in\mathopen[ a , b \mathclose]\).
\end{definition}

\begin{theorem}     \label{ThoDerSousIntegrale}
    Soit \( A\) un ouvert de \( \eR\) et \( \Omega\), un espace mesuré. Soit une fonction \( f\colon A\times \Omega\to \eR\) et
    \begin{equation}
        F(x)=\int_{\Omega}f(x,\omega)d\omega.
    \end{equation}
    Nous supposons les points suivants.
    \begin{enumerate}
        \item
            La fonction \( f\) est mesurable en tant que fonction \( A\times\Omega\to \eR\). Pour chaque \( x\in A\), la fonction \( f(x,\cdot)\) est intégrable sur \( \Omega\).
        \item
            Pour presque tout \( \omega\in\Omega\), la fonction \( f(x,\omega)\) est une fonction absolument continue de \( x\).
        \item
            La fonction \( \frac{ \partial f }{ \partial x }\) est localement intégrable, c'est à dire que pour tout \( \mathopen[ a , b \mathclose]\subset A\),
            \begin{equation}
                \int_a^b\int_{\Omega}\left| \frac{ \partial f }{ \partial x }(x,\omega) \right| d\omega\,dx<\infty.
            \end{equation}
    \end{enumerate}
    Alors la fonction \( F\) est absolument continue et pour presque tout \( x\in A\), la dérivée est donné par
    \begin{equation}
        \frac{ d }{ dx }\int_{\Omega}f(x,\omega)d\omega=\int_{\Omega}\frac{ \partial f }{ \partial x }(x,\omega)d\omega.
    \end{equation}
\end{theorem}

La proposition suivante sera utilisée entre autres pour montrer que sous l'hypothèse d'une densité continue, la loi exponentielle est sans mémoire, proposition \ref{PropREXaIBg}.
\begin{proposition}		\label{PropDerrFnAvecBornesFonctions}
Soit $f(x,t)$ une fonction continue sur $[\alpha,\beta]\times[a,b]$, telle que $\frac{ \partial f }{ \partial x }$ existe et soit continue sur $]\alpha,\beta[\times[a,b]$. Soient $\varphi(x)$ et $\psi(x)$, des fonctions continues de $[\alpha,\beta]$ dans $\eR$ et admettant une dérivée continue sur $]\alpha,\beta [$. Alors la fonction
\begin{equation}
	F(x)=\int_{\varphi(x)}^{\psi(x)}f(x,t)dt
\end{equation}
admet une dérivée continue sur $]\alpha,\beta[$ et
\begin{equation}	\label{EqFormDerrFnAvecBorneNInt}
	\frac{ dF }{ dx }=\int_{\varphi(x)}^{\psi(x)}\frac{ \partial f }{ \partial x }(x,t)dt+f\big( x,\psi(x) \big)\cdot\frac{ d\psi }{ dx }- f\big( x,\varphi(x) \big)\cdot\frac{ d\varphi }{ dx }.
\end{equation}
\end{proposition}
\index{permuter!dérivée et intégrale!dans \( \eR\) avec les bornes}
%TODO : une preuve de ce théorème ? allons allons ...

L'exemple qui suit devrait pouvoir être rendu rigoureux en utilisant des distributions correctement.

\begin{example} \label{ExfYXeQg}
    Si \( g\) est une fonction continue, la fonction suivante est une primitive de \( g\) :
    \begin{equation}
        \int_0^xf(t)dt=\int_0^{\infty}f(t)\mtu_{t<x}(t)dt.
    \end{equation}
    Nous nous proposons de justifier \emph{de façon un peu heuristique} le fait que ce soit bien une primitive de \( g\) en considérant la fonction
    \begin{equation}
        f(t,x)=g(t)\mtu_{t<x}(t).
    \end{equation}
    Nous posons
    \begin{equation}
        F(x)=\int_0^{\infty}f(x,t)dt,
    \end{equation}
    et nous calculons \( F'\) en permutant la dérivée et l'intégrale\footnote{Ceci n'est pas rigoureux : il faudrait avoir un théorème à propos de distributions qui permet de le faire.}. D'abord,
    \begin{equation}
        f(t,x)=\begin{cases}
            g(t)    &   \text{si \( t\in \mathopen[ 0 , x \mathclose]\)}\\
            0    &    \text{sinon.}
        \end{cases}
    \end{equation}
    La dérivée de \( f\) par rapport à \( x\) est donnée par la distribution
    \begin{equation}
        \frac{ \partial f }{ \partial x }(t_0,x_0)=g(t_0)\delta(t_0-x_0).
    \end{equation}
    Donc
    \begin{equation}
        F'(x_0)=\int_0^{\infty}\frac{ \partial f }{ \partial x }(t,x_0)dt=\int_0^{\infty}g(t)\delta(t-x_0)=g(x_0),
    \end{equation}
    comme attendu.
\end{example}

Cet exemple est rendu rigoureux par la proposition suivante.
\begin{proposition} \label{PropJLnPpaw}
    Si \( f\in L^1(\eR)\), alors la fonction
    \begin{equation}
        F(x)=\int_{-\infty}^xf(t)dt
    \end{equation}
    est presque partout dérivable et pour les points où elle l'est nous avons \( F'(x)=f(x)\).
\end{proposition}
\index{fonction!définie par une intégrale}
%TODO : une preuve.

%--------------------------------------------------------------------------------------------------------------------------- 
\subsection{Différentiabilité sous l'intégrale}
%---------------------------------------------------------------------------------------------------------------------------

Le théorème suivant est restrictif sur l'ensemble d'intégration (qui doit être compact), mais accepte des fonctions de plusieurs variables, ce qui est un premier pas vers la différentiabilité.
\begin{proposition}[Dérivation sous l'intégrale]		\label{PropDerrSSIntegraleDSD}
    Supposons $A\subset\eR^m$ ouvert et $B\subset\eR^n$ compact. Nous considérons une fonction \( f\colon A\times B\to \eR\). Si pour un $i\in\{ i,\ldots,n \}$, la dérivée partielle $\frac{ \partial f }{ \partial x_i }$ existe dans $A\times B$ et est continue, alors la fonction
    \begin{equation}
        F(x)=\int_Bf(x,t)dt
    \end{equation}
    admet une dérivée partielle dans la direction \( x_i\) sur \( A\). Cette dérivée partielle y est continue et
    \begin{equation}
        \frac{ \partial F }{ \partial x_i }(a)=\int_B\frac{ \partial f }{ \partial x_i }(a,t)dt,
    \end{equation}
    pour tout \( a\) dans l'ouvert \( A\).
\end{proposition}
\index{fonction!définie par une intégrale}
\index{inversion de limite}
\index{permuter!dérivée et intégrale!\( \eR^n\)}

\begin{proof}
    Nous procédons en plusieurs étapes.
    \begin{subproof}
    \item[\( F\) est dérivable]
            
        Nous voulons prouver que \( \frac{ \partial F }{ \partial x_i }(a,t)\) existe. Pour cela nous posons
        \begin{equation}
            g_l(t)=\frac{ f(a_1,\ldots, a_i+\epsilon_l,\ldots, a_n,t)-f(a_1,\ldots, a_k,\ldots, a_n,t) }{ \epsilon_l }
        \end{equation}
        où \( \epsilon_l\) est une suite de nombres tendant vers zéro. La fonction \( f\) est dérivable dans la direction \( x_i\) si et seulement si \( \lim_{l\to \infty}g_l(t) \) existe et ne dépend pas du choix de la suite. À ce moment, la valeur de la dérivée partielle sera cette limite. Dans notre cas, nous savons que \( f\) admet une dérivée partielle dans la direction \( x_i\) et donc nous avons
        \begin{equation}
            \frac{ \partial f }{ \partial x_i }(a,t)=\lim_{l\to \infty} g_l(t).
        \end{equation}
        
        De la même façon pour \( F\) nous avons
        \begin{equation}
            \frac{ \partial F }{ \partial x_i }=\lim_{l\to \infty} \int_{B}g_l(t)dt.
        \end{equation}
        Sous-entendu : si la limite de droite ne dépend pas de la suite choisie, alors \( \frac{ \partial F }{ \partial x_i }\) existe et vaut cette limite.

        Vu la continuité de \( f\), le seul point à vérifier pour le théorème de la convergence dominée de Lebesgue est l'existence d'une fonction intégrable de \( t\) majorant \( g_l\). Pour cela le théorème de accroissements finis (théorème \ref{ThoAccFinis}) appliqué à la fonction \( \epsilon\mapsto f(a_n,\ldots, a_i+\epsilon,\ldots, a_n)\) nous dit que
        \begin{equation}
            f(a_1,\ldots, a_i+\epsilon_l,\ldots, a_n,t)-f(a_1,\ldots, a_i,\ldots, a_n,t)=\epsilon_l\frac{ \partial f }{ \partial x_i }(a_1,\ldots, \theta,\ldots, a_n,t)
        \end{equation}
        pour un certain \( \theta\in B(a_i,\epsilon_l)\). Notons que ce \( \theta\) dépend de \( t\) mais pas de \( l\). Vu que \( \partial_if\) est continue par rapport à ses deux variables, si \( K\) est un voisinage compact autour de \( a\), il existe \( M>0\) tel que
        \begin{equation}    \label{EqMXqviPC}
            \left| \frac{ \partial f }{ \partial x_i }(x,t) \right| < M
        \end{equation}
        pour tout \( x\in K\) et tout \( t\in B\). La valeur de \( \frac{ \partial f }{ \partial x_i }(a_1,\ldots, \theta,\ldots, a_n,t)\) est donc bien majorée par rapport à \( \theta\) et par rapport à \( t\) en même temps par une constante qui n'a pas de mal à être intégrée sur le compact \( B\).
        
        Le théorème de la convergence dominée (théorème \ref{ThoConvDomLebVdhsTf}) s'applique donc bien et nous avons
        \begin{equation}
            \lim_{l\to \infty} \int_Bg_l(t)dt=\int_B\lim_{l\to \infty} g_l(t)=\int_B\frac{ \partial f }{ \partial x_i }(a,t)dt.
        \end{equation}
        Le membre de droite ne dépendant pas de la suite \( \epsilon_l\) choisie, le membre de gauche est bien la dérivée de \( F\) par rapport à \( x_i\) et nous avons
        \begin{equation}
            \frac{ \partial F }{ \partial x_i }(a)=\int_B\frac{ \partial f }{ \partial x_i }(a,t)dt.
        \end{equation}
        Cela prouve la première partie de la proposition.

    \item[La dérivée est continue]

        Soit \( K\) un voisinage compact autour de \( a\) et \( U'\) un ouvert tel que \( a\in U'\subset K\). Nous avons encore la majoration \eqref{EqMXqviPC} sur \( U'\) et donc le théorème de continuité sous l'intégrale \ref{ThoKnuSNd} nous indique que la fonction
        \begin{equation}
            \begin{aligned}
                U'&\to \eR \\
                x&\mapsto \int_{B}\frac{ \partial f }{ \partial x_i }(x,t)dt 
            \end{aligned}
        \end{equation}
        est continue en \( a\).
        
    \end{subproof}
\end{proof}

Une conséquence de la proposition \ref{PropDerrSSIntegraleDSD} est que si elle fonctionne pour tous les \( i\), alors \( F\) est différentiable et même de classe \( C^1\), et la différentielle de \( F\) s'obtient comme intégrale de la différentielle de \( f\).

\begin{proposition}\label{PropAOZkDsh}
    Supposons $A\subset\eR^m$ ouvert et $B\subset\eR^n$ compact. Si pour tout $i\in\{ i,\ldots,n \}$, la dérivée partielle $\frac{ \partial f }{ \partial x_i }$ existe dans $A\times B$ et est continue, alors \( F\) est de classe \( C^1\) et
    \begin{equation}
        (dF)_a=\int_B(df_t)_adt
    \end{equation}
    où \( f_t(x)=f(x,t)\).
\end{proposition}
\index{permuter!différentielle et intégrale!\( \eR^n\)}

\begin{proof}
    En vertu de la proposition \ref{PropDerrSSIntegraleDSD}, toutes les dérivées partielles de \( F\) sont continues. Cela implique que \( F\) est de classe \( C^1\) par la proposition \ref{PropDerContCun} et que la différentielle s'écrive en terme des dérivées partielles avec la formule usuelle. Nous avons alors
    \begin{subequations}
        \begin{align}
            (dF)_a(u)&=\sum_k\frac{ \partial F }{ \partial x_k }(a)u_k\\
            &=\int_B\sum_k\frac{ \partial f }{ \partial x_k }(a,t)dt\\
            &=\int_B\sum_k\frac{ \partial f_t }{ \partial x_k }(a)u_kdt\\
            &=\int_B (df_t)_a(u)dt.
        \end{align}
    \end{subequations}
    Cela est la formule annoncée.
\end{proof}

Un autre théorème tourne autour du pot, et me semble inutile.
\begin{theorem} \label{ThoOLAQyRL}
    Soit \( (\Omega,\mu)\) un espace mesuré, une fonction \( f\colon \eR^n\times \Omega\to \eR\) et \( a\in \eR^n\). Nous considérons la fonction
    \begin{equation}
        F(x)=\int_{\Omega}f(x,\omega)d\mu(\omega).
    \end{equation}
    Pour chaque \( k=1,\ldots, n\) nous supposons avoir
    \begin{equation}
        \frac{ \partial F }{ \partial x_k }(a)=F_{|_k}'(a)=\int_{\Omega}\frac{ \partial f_{|_k} }{ \partial t }(a_k,\omega)d\mu(\omega)
    \end{equation}
    où \( F_{|_k}(t)=F(a_1,\ldots, t,\ldots, a_n)\) et \( f_{|_k}\) est définie de façon similaire.

    Nous supposons de plus que les fonctions \( \partial_{x_k}F\) sont continues.

    Alors \( F\) est de classe \( C^1\) et sa différentielle est donnée par
    \begin{equation}
        df_a=\int_{\Omega}(df_{\omega})_ad\omega
    \end{equation}
    où \( f_{\omega}\) est définie par \( f_{\omega}(x)=f(x,\omega)\).
\end{theorem}

\begin{proof}
    Étant donné que les dérivées partielles de \( F\) en \( a\) existent et sont continues, la proposition \ref{PropDerContCun} dit que \( F\) est différentiable et que
    \begin{equation}
        dF_a(u)=\sum_{k=1}^n\frac{ \partial F }{ \partial x_k }(a)u_k.
    \end{equation}
    La linéarité de l'intégrale et les hypothèses nous donnent alors
    \begin{subequations}
        \begin{align}
            df_a(u)&=\sum_{k=1}^n\frac{ \partial F }{ \partial x_k }(a)u_k\\
            &=\int_{\Omega}\sum_k\frac{ \partial f_{|_k} }{ \partial t }(a_k;\omega)u_kd\mu(\omega)\\
            &=\int_{\Omega}\sum_k\frac{ \partial f }{ \partial x_k }(a;\omega)u_kd\mu(\omega)\\
            &=\int_{\Omega}(df_{\omega})_a(u)d\mu(\omega),
        \end{align}
    \end{subequations}
    et donc \( df_a=\int_{\Omega}(df_{\omega})_ad\mu(\omega)\).
\end{proof}
Notons qu'en passant aux composantes, ce théorème fonctionne tout aussi bien pour des fonctions à valeurs dans un espace vectoriel normé de dimension finie plutôt que dans \( \eR\).

\begin{lemma}[Hadamard\cite{MVIooKLsjpa}]   \label{LemWNBooGPlIwT}
    Soit une fonction \( f\colon \eR^n\to \eR\) de classe \( C^p\) avec \( p\geq 1\). Pour tout \( a\in \eR^n\) il existe des fonctions \( g_1\),\ldots, \( g_n\) de classe \( C^{p-1}\) telles que
    \begin{equation}
        f(x)=f(a)+\sum_{i=1}^n(x_i-a_i)g_i(x).
    \end{equation}
\end{lemma}
\index{lemme!Hadamard}

\begin{proof}
    Vu que \( f\) est de classe \( C^1\), le théorème fondamental de l'analyse \ref{ThoRWXooTqHGbC} fonctionne et
    \begin{equation}    \label{EqZLTooVKmGln}
        f(x)-f(a)=\int_0^1\frac{ d }{ dt }\Big[ f\big( a+t(x-a) \big) \Big]dt=\int_0^1\sum_{i=1}^n\frac{ \partial f }{ \partial x_i }\big( a+t(x-a) \big)(x_i-a_i).
    \end{equation}
    Plus de détails : la fonction \( t\mapsto \frac{ d }{ dt }\Big[ f\big( a+t(x-a) \big) \Big]\) possède comme primitive la fonction \( F(t)=f\big( a+t(x-a) \big)\).

    Nous posons 
    \begin{equation}
        g_i(x)=\int_0^1\frac{ \partial f }{ \partial x_i }\big( a+t(x-a) \big)dt
    \end{equation}
    Le fait que l'intégrale existe est simplement le fait qu'il s'agit d'une fonction continue sur un compact et donc majorée par une constante. Pour voir que \( g_i\) est de classe \( C^{p-1}\) nous pouvons calculer \( \frac{ \partial g_i }{ \partial x_k }\) en permutant dérivée et intégrale par la proposition \ref{PropDerrSSIntegraleDSD} :
    \begin{equation}
        \frac{ \partial g_i }{ \partial x_k }(x)=\int_0^1\frac{ \partial  }{ \partial x_k }\left( \frac{ \partial f }{ \partial x_i }\big( a+t(x-a) \big) \right)dt=\int_0^1 t\frac{ \partial^2f }{ \partial x_k\partial x_i }\big( a+t(x-a) \big).
    \end{equation}
    Nous pouvons ainsi permuter \( p-1\) dérivées tout en gardant une fonction continue dans l'intégrale. Le théorème \ref{ThoKnuSNd} nous donne alors une fonction continue. Ainsi toutes les fonctions
    \begin{equation}
        \frac{ \partial^{p-1}g_i }{ \partial x_{i_1}\ldots\partial x_{i_{p-1}} }
    \end{equation}
    sont continues et \( g_i\) est de classe \( C^{p-1}\) par la proposition \ref{PropDYKooHvrfGw}.

    En repartant de \eqref{EqZLTooVKmGln} nous avons alors bien ce qui était annoncé :
    \begin{equation}
        f(x)=f(a)+\sum_{i=1}^ng_i(x)(x_i-a_i).
    \end{equation}
\end{proof}

\begin{corollary}
    Soit \( \phi\in\swD(\eR)\) tel que \( \phi^{(k)}(0)=0\) pour tout \( k\). Alors il existe une fonction \( \psi\in\swD(\eR)\) telle que 
    \begin{equation}
        \phi(x)=x^{n+1}\psi(x)
    \end{equation}
    pour tout \( x\in \eR\).
\end{corollary}

\begin{proof}
    En utilisant le lemme de Hadamard \ref{LemWNBooGPlIwT} avec \( a=0\), \( n=1\) et \( f(0)=0\), nous avons \( \phi(x)=xg_1(x)\) et comme \( g(0)=0\) nous avons \( g_1(x)=xg_2(x)\) et par conséquence \( \phi(x)=x^2g_2(x)\).

    En continuant ainsi autant de fois que l'on peut dériver \( \phi\), nous obtenons le résultat.
\end{proof}

%+++++++++++++++++++++++++++++++++++++++++++++++++++++++++++++++++++++++++++++++++++++++++++++++++++++++++++++++++++++++++++ 
\section{Formes différentielles exactes et fermées}
%+++++++++++++++++++++++++++++++++++++++++++++++++++++++++++++++++++++++++++++++++++++++++++++++++++++++++++++++++++++++++++

Nous avons déjà parlé de formes différentielles et de leurs intégrales sur un chemin dans la section \ref{SecFormDiffRappel}.

\begin{definition}  \label{DefEFKQmPs}
La forme différentielle $\omega$ est \defe{exacte}{forme!différentielle!exacte} si il existe une fonction $f$ telle que $\omega=df$; elle est dite \defe{fermée}{forme!différentielle!fermée} si $d\omega=0$.
\end{definition}

Dire que la forme différentielle $\omega=fdx+gdy$ est fermée, c'est dire que
\begin{equation}
    \frac{ \partial g }{ \partial x }=\frac{ \partial f }{ \partial y }.
\end{equation}

Il est naturel de se demander si toutes les formes différentielles sont des différentielles de fonctions. Une réponse complète est délicate à établir, mais a d'innombrables conséquences en physique, notamment en ce qui concerne l'existence d'un potentiel vecteur pour le champ magnétique dans les équations de Maxwell.

Le fait qu'une forme exacte soit fermée est relativement facile à établir; c'est la proposition suivante. La question plus délicate est la réciproque : sous quelles conditions une forme fermée est-elle exacte ?
\begin{proposition}
	Si $\omega$ est une $1$-forme exacte de classe $C^1$, alors $\omega$ est fermée.
\end{proposition}

\begin{proof}
	Le fait que $\omega$ soit exacte implique l'existence d'une fonction $f$ telle que $\omega=df$, c'est à dire
	\begin{equation}
		\omega_x=\sum_i a_i(x)dx_i=\sum_i\frac{ \partial f }{ \partial x_i }(x)dx_i,
	\end{equation}
	c'est à dire que $a_i(x)=\frac{ \partial f }{ \partial x_i }(x)$. L'hypothèse que $\omega$ est $C^1$ implique que $f$ est $C^2$, et donc que nous pouvons inverser l'ordre de dérivation pour les dérivées secondes $\partial^2_{ij}f=\partial^2_{ji}f$. Nous pouvons donc faire le calcul suivant :
	\begin{equation}
		\frac{ \partial a_i }{ \partial x_j }=\frac{ \partial  }{ \partial x_j }\frac{ \partial f }{ \partial x_i }=\frac{ \partial  }{ \partial x_i }\frac{ \partial f }{ \partial x_j }=\frac{ \partial a_j }{ \partial x_i },
	\end{equation}
	ce qu'il fallait démontrer.
\end{proof}

La réciproque est vraie sur un ouvert simplement connexe.
\begin{theorem}        \label{ThoFermeExactFormRappel}
Supposons que $D\subset\eR^n$ soit un ouvert simplement connexe. Alors toute forme différentielle de degré $1$ et de classe $C^1$ sur $D$ qui est fermée est exacte.
\end{theorem}

Nous allons prouver ce théorème dans un cas un peu moins général : celui d'un domaine étoilé de \( \eR^2\) plutôt que simplement connexe de \( \eR^n\).

\begin{theorem} \label{ThoMSofFxL}
Soit $D\subset\eR^2$, une ouvert étoilé, et $\omega$, une $1$-forme fermée de classe $C^1$. Alors $\omega$ est exacte.
\end{theorem}
\begin{proof}

Soit $D\subset\eR^2$, un ouvert étoilé par rapport à l'origine. Soient $f\colon D\to \eR$, $g\colon D\to \eR$, des fonctions de classe $C^1$ telles que
\begin{equation}
	\frac{ \partial f }{ \partial y }=\frac{ \partial g }{ \partial x }
\end{equation}
sur $D$, et
\begin{equation}		\label{EqIMDefFformI33}
	F(x,y)=\int_0^1\big[  f(tx,ty)x+g(tx,ty)y  \big]dt
\end{equation}
pour tout $(x,y)\in D$. 

Étant donné que nous ne définissons $F(x,y)$ que pour des $(x,y)\in D$, la fonction $t\mapsto f(tx,ty)$ est $C^1$ sur tout le compact $[0,1]$ et aucune divergence de l'intégrale n'est à craindre. Nous sommes donc dans le cadre de la proposition \ref{PropDerrSSIntegraleDSD}, et nous pouvons dériver sous le signe intégral.

Nous calculons, en utilisant la règle de dérivation de fonctions composées
\begin{equation}		\label{EqIMI33dsdsFlolo}
	\begin{aligned}[]
		\frac{ \partial F }{ \partial x }(x,t)	&=\int_0^1\left[   f\frac{ \partial f }{ \partial x }(tx,ty)x+f(tx,ty)+t\frac{ \partial g }{ \partial x }(tx,ty)y  \right]dt\\
		&=\int_0^1\left[ t\Big( x\frac{ \partial f }{ \partial x }(tx,ty)+y\frac{ \partial f }{ \partial y }(tx,ty) \Big)+f(tx,ty) \right]dt
	\end{aligned}
\end{equation}
où nous avons utilisé l'hypothèse $\partial_yf=\partial_xg$. Ce qui se trouve dans la parenthèse n'est autre que $\partial_t\big( f(tx,ty) \big)$, plus précisément, si nous posons $\mF(x,y,t)=f(tx,ty)$, nous avons
\begin{equation}
	\frac{ \partial \mF }{ \partial t }(x,y,t)= x\frac{ \partial f }{ \partial x }(tx,ty)+y\frac{ \partial f }{ \partial y }(tx,ty).
\end{equation}
En recopiant le résultat \eqref{EqIMI33dsdsFlolo} en termes de $\mF$, nous avons
\begin{equation}
	\begin{aligned}[]
		\frac{ \partial F }{ \partial x }(x,t)	&=\int_0^1\left( t\frac{ \partial \mF }{ \partial t }(x,y,t)+\mF(x,y,t) \right)dt\\
		&=\int_0^1\partial_t\big( t\mF(x,y,t) \big)dt\\
		&=\big[ f\mF(x,y,t) \big]_0^1\\
		&=\mF(x,y,1)\\
		&=f(x,y).
	\end{aligned}
\end{equation}
Le résultat correspondant pour $\frac{ \partial F }{ \partial y }(x,y)=g(x,y)$ s'obtient de la même manière. Nous avons donc obtenu que
\begin{equation}		\label{EqIMFormI33Fffdd}
	\begin{aligned}[]
		\frac{ \partial F }{ \partial x }&=f,  &\text{et}&& \frac{ \partial F }{ \partial y }=g.
	\end{aligned}
\end{equation}
En ayant prouvé cela, nous avons prouvé que si $\omega=fdx+gdy$ avec $\partial_yf=\partial_xg$, alors $\omega=dF$ où $F$ est définie par \eqref{EqIMDefFformI33}.
\end{proof}

\begin{proof}[Démonstration alternative du théorème \ref{ThoMSofFxL}]
Nous posons $u=tx$ et $v=ty$, ainsi que $\mF(x,y,t)=f(u,v)$ et $\mG(x,y,t)=g(u,v)$. Avec cette notation, nous avons $F(x,y)=\int_0^1\big( x\mF(x,y,t)+y\mG(x,y,t) \big)dt$, et
\begin{equation}
	\begin{aligned}[]
		\frac{ \partial \mF }{ \partial x }&=\frac{ \partial f }{ \partial u }\frac{ \partial u }{ \partial x }+\frac{ \partial f }{ \partial v }\frac{ \partial v }{ \partial x }=t\frac{ \partial f }{ \partial u },\\
		\frac{ \partial \mG }{ \partial x }&=t\frac{ \partial g }{ \partial u }.
	\end{aligned}
\end{equation}
Ainsi,
\begin{equation}
	\begin{aligned}[]
		\frac{ \partial F }{ \partial x }	&=\int_0^1\left( x\frac{ \partial \mF }{ \partial x }+\mF+y\frac{ \partial G }{ \partial x } \right)dt\\
							&=\int_0^1\left( xt\frac{ \partial f }{ \partial u } +\mF+yt\frac{ \partial g }{ \partial u } \right)dt\\
							&=\int_0^1\left[  t\left( x\frac{ \partial f }{ \partial u }+y\frac{ \partial f }{ \partial v } \right)+\mF  \right]dt.
	\end{aligned}
\end{equation}
où nous avons utilisé le fait que, par hypothèse, $\frac{ \partial g }{ \partial u }=\frac{ \partial f }{ \partial v }$. Nous calculons par ailleurs que
\begin{equation}
	\frac{ \partial F }{ \partial t }=\frac{ \partial f }{ \partial u }\frac{ \partial u }{ \partial t }+\frac{ \partial f }{ \partial v }\frac{ \partial v }{ \partial t }=x\frac{ \partial f }{ \partial u }+y\frac{ \partial f }{ \partial v }.
\end{equation}
Donc, nous avons
\begin{equation}
	\frac{ \partial F }{ \partial x }=\int_0^1\left( t\frac{ \partial \mF }{ \partial t }+\mF \right)dt=\int_0^1\frac{ \partial  }{ \partial t }(t\mF)dt.
\end{equation}
Par conséquent,
\begin{equation}
	\frac{ \partial F }{ \partial x }=[t\mF]_0^1=\mF(x,y,1)=f(x,y).
\end{equation}
Le même genre de calculs fournit $\frac{ \partial F }{ \partial y }=g(x,y)$.
\end{proof}

%+++++++++++++++++++++++++++++++++++++++++++++++++++++++++++++++++++++++++++++++++++++++++++++++++++++++++++++++++++++++++++ 
\section{Théorème d'Abel angulaire}
%+++++++++++++++++++++++++++++++++++++++++++++++++++++++++++++++++++++++++++++++++++++++++++++++++++++++++++++++++++++++++++

\begin{theorem}[Abel angulaire\cite{KXjFWKA}]   \label{ThoTGjmeen}
    Soit \( \sum_{n}a_nz^n\) une série entière de rayon de convergence plus grand ou égal à \( 1\) et de somme \( f\). Soit \( \theta_0\in\mathopen[ 0 , \frac{ \pi }{2} \mathclose[\). Nous posons
    \begin{equation}
        \Delta_{\theta_0}=\{ z=1-\rho e^{i\varphi}\tq \rho>0,\varphi\in\mathopen[ \theta_0 , \theta_0 \mathclose],| z |<1 \}.
    \end{equation}
    Nous supposons de plus que \( \sum_na_n\) converge. Alors
    \begin{equation}
        \lim_{\substack{z\to1\\z\in\Delta_0}}f(z)=\sum_{k=0}^{\infty}a_k.
    \end{equation}
\end{theorem}
\index{Abel!angulaire}
\index{convergence!suite numérique!Abel angulaire}
\index{somme partielles!Abel angulaire}
\index{série!entière!Abel angulaire}

\begin{proof}

    Le résultat de ce théorème est que l'on peut calculer la limite \( z\to 1\) avec des chemins contenus dans un domaine de la forme de celui dessiné à la figure \ref{LabelFigJGuKEjH}. % From file JGuKEjH
    \newcommand{\CaptionFigJGuKEjH}{La zone dans laquelle peut être le chemin qui va vers \( z=1\).}
    \input{Fig_JGuKEjH.pstricks}

    De façon très classique nous posons
    \begin{equation}
        \begin{aligned}[]
            S&=\sum_{k=0}^{\infty}a_k&S_n&=\sum_{k=0}^na_k,
        \end{aligned}
    \end{equation}
    et \( R_n=S-S_n\). En particulier \( a_n=R_{n-1}-R_n\). 

    Le but du théorème est de montrer que \( \sum a_nz^n\) converge vers \( S\) lorsque \( z\) converge vers \( 1\) à l'intérieur de \( \Delta_{\theta_0}\). Pour cela nous calculons pour un \( N\) donné la différence \( \sum_{n=0}^{N}a_nz^n-S_N\) en triant les termes par ordre de \( R_n\), en isolant le terme \( R_0\) et le terme \( R_N\) :
    \begin{subequations}
        \begin{align}
            \sum_{n=0}^Na_nz^n-S_N&=\sum_{n=1}^Na_n(z^n-1)\\
            &=\sum_{n=1}^N(R_{n-1}-R_n)(z^n-1)\\
            &=R_0(z-1)+\sum_{n=1}^{N-1}R_n(z^{n+1}-1-z^n+1)+R_N(z^N-1)\\
            &=R_0(z-1)+\sum_{n=1}^{N-1}R_nz^n(z-1)+R_N(z^N-1)\\
            &=(z-1)\sum_{n=0}^{N-1}R_nz^n+R_N(z^N-1).
        \end{align}
    \end{subequations}
    Cela est valable pour tout \( N\) et \( | z |<1\). Nous avons donc
    \begin{equation}
        \sum_{n=0}^Na_nz^n-S_N=(z-1)\sum_{n=0}^{N-1}R_nz^n+R_N(z^N-1).
    \end{equation}
    Par hypothèse nous avons \( \lim_{N\to \infty} R_N=0\). Et de plus le membre de gauche converge parce que chacun des deux termes converge séparément. En passant à la limite nous avons pour tout \( | z |<1\) :
    \begin{equation}
        f(z)-S=(z-1)\sum_{n=0}^{\infty}R_nz^n.
    \end{equation}
    Nous voudrions étudier le comportement de la différence \( f(z)-S\) lorsque \( z\) tend vers \( 1\). Pour cela nous nous fixons \( \epsilon>0\) et \( N\geq 1\) tel que \( | R_n |<\epsilon\) dès que \( n\geq N\). Alors pour tout \( | z |<1\) nous avons
    \begin{subequations}
        \begin{align}
            | f(z)-S |&\leq | z-1 |\left( \sum_{n=0}^N| R_n | \underbrace{|z^n |}_{\leq 1} +\sum_{n=N+1}^{\infty}\underbrace{| R_n |}_{\leq \epsilon} |z^n | \right)\\
            &\leq | z-1 |\sum_{n=0}^N| R_n |+\epsilon\frac{ | z-1 | }{ 1-| z | }
        \end{align}
    \end{subequations}
    où nous avons utilisé la somme de la série géométrique \eqref{EqASYTiCK} et l'égalité \( | z^n |=| z |^n\). Avant de nous particulariser à \( z\in\Delta_{\theta_0}\) nous devons anticiper un problème au dénominateur en multipliant par le binôme conjugué :
    \begin{equation}
        \frac{ | z-1 | }{ 1-| z | }=\frac{ | z-1 |(1+| z |) }{ 1-| z |^2 }.
    \end{equation}
    C'est maintenant que nous nous particularisons à \( z\in\Delta_{\theta_0}\) en posant \( z=\rho e^{i\varphi}\) et en remarquant que \( | z |^2=1-2\rho\cos(\varphi)+\rho^2\). Nous avons le calcul suivant :
    \begin{subequations}
        \begin{align}
            \frac{ | z-1 | }{ 1-| z | }&=\frac{ \rho(1+| z |) }{ 2\rho\cos(\varphi)-\rho^2 }\\
            &=\frac{ 1+| z | }{ 2\cos(\varphi)-\rho}\\
            &\leq\frac{ 2 }{ 2\cos(\varphi)-\rho }\\
            &\leq\frac{ 2 }{ 2\cos(\varphi)-\cos(\theta_0) }\\
            &\leq\frac{ 2 }{ 2\cos(\theta_0)-\cos(\theta_0) }\\
            &=\frac{ 2 }{ \cos(\theta_0) }.
        \end{align}
    \end{subequations}
    Quelque justifications.
    \begin{itemize}
        \item Vu que nous avons dans l'idée de faire \( \rho\to 0\) nous supposons que \( \rho<\cos(\theta_0)\).
        \item Nous avons \( \cos(\varphi)>\cos(\theta_0)\) parce que \( z\) est dans \( \Delta_{\theta_0}\).
    \end{itemize}
    Nous avons donc, pour tout \( z\in\Delta_{\theta_0}\) que
    \begin{equation}
        | f(z)-S |\leq | z-1 |\sum_{n=0}^N| R_n |+\epsilon\frac{ 2 }{ \cos(\theta_0) }.
    \end{equation}
    Il suffit de prendre \( \rho\) assez petit pour que 
    \begin{equation}
        | z-1 |\sum_{n=0}^N| R_n |<\epsilon
    \end{equation}
    et nous avons
    \begin{equation}
        | f(z)-S |\leq \epsilon\left( 1+\frac{ 2 }{ \cos(\theta_0) } \right).
    \end{equation}
    Nous avons donc bien \( \lim_{\substack{z\to 1\\z\in\Delta_0}}f(z)=S\), comme nous le voulions.
\end{proof}

La réciproque du théorème d'Abel angulaire est que si \( f(z)=\sum_na_nz^n\) sur \( B(0,1)\) se prolonge par continuité en \( z=1\) alors cette prolongation se fait par \( f(1)=\sum_na_n\). Cela est faux comme le montre l'exemple suivant.

\begin{example}
    Nous considérons la série entière \( \sum_{n=0}^{\infty}(-1)^nz^n\) qui converge\footnote{C'est la série géométrique de raison \( -z\).} vers
    \begin{equation}
        f(z)=\frac{1}{ 1+z }
    \end{equation}
    sur \( B(0,1)\). De plus nous avons
    \begin{equation}
        \lim_{\substack{z\to 1\\    | z |<1}}\frac{1}{ 1+z }=\frac{ 1 }{2}.
    \end{equation}
    Donc la fonction converge bien vers quelque chose lorsque \( z\) tend vers \( 1\). La fonction \( f\) se prolonge par continuité en \( 1\). Pourtant la série es coefficients \( \sum_n(-1)^n\) ne converge pas.
\end{example}

Le théorème suivant donne une espèce d'inverse au théorème d'Abel angulaire. En effet il dit que si la série converge  en allant vers \( 1\) le long de l'axe réel, alors ça converge vers la somme des coefficients. Il faut cependant une hypothèse en plus sur les \( a_n\).
\begin{theorem}[Théorème taubérien faible\cite{KXjFWKA}]
    Soit \( \sum_na_nz^n\) une série entière de rayon de convergence \( 1\) et de somme \( f\). Nous supposons
    \begin{enumerate}
        \item
            Il existe \( S\in \eC\) tel que \( \lim_{\substack{x\to 1\\x\in\mathopen] -1 , 1 \mathclose[}}f(x)=S\).
            \item
                \( \lim_{n\to \infty} na_n=0\).
    \end{enumerate}
    Alors la série \( \sum_{n=0}^{\infty}a_n\) converge et vaut \( S\).
\end{theorem}
\index{théorème!taubérien faible}

\begin{proof}
    Nous notons \( S_n=\sum_{k=0}a_k\) et \( M=\sup_{k\geq 1}k| a_k |\), qui est fini par hypothèse. Pour \( x\in \mathopen] 0 , 1 \mathclose[\) et \( n\geq 0\) nous avons
    \begin{equation}
        S_n-f(x)=\sum_{k=1}^na_k-\sum_{k=1}^na_kx^k-\sum_{k=n+1}^{\infty}a_kx^k=\sum_{k=1}^na_k(1-x^k)-\sum_{k=n+1}^{\infty}a_kx^k.
    \end{equation}
    Nous utilisons la série géométrique sous la forme \( 1-x^k=(1-x)\sum_{i=0}^nx^i\) pour écrire
    \begin{subequations}
        \begin{align}
            S_n-f(x)&=\sum_{k=1}^na_k(1-x)\underbrace{\sum_{i=0}^{k-1}x^i}_{\leq k}-\sum_{k=n+1}^{\infty}a_kx^k\\
            &\leq\sum_{k=1}^nka_k(1-x)-\sum_{k=n+1}^{\infty}a_kx^k,
        \end{align}
    \end{subequations}
    donc en passant à la norme
    \begin{subequations}
        \begin{align}
            \big| S_n-f(x) \big|&\leq (1-x)Mn+\sum_{k=n+1}| a_k |x^k\\
            &\leq (1-x)Mn+\sum_{k=n+1}^{\infty}\underbrace{\frac{ k }{ n }| a_k |}_{\leq M/n}x^k\\
            &\leq (1-x)Mn+\frac{ M }{ n }\sum_{k=n+1}^{\infty}x^k\\
            &\leq (1-x)Mn+\frac{ M }{ n }\frac{1}{ 1-x }.
        \end{align}
    \end{subequations}
    Ce que nous cherchons à étudier est le comportement \( x\to 1\) et montrer que \( S_n\to S\), ce qui nous incite à calculer \( | S_n-f(1-\frac{ \epsilon }{n  }) |\) avec \( 0<\epsilon<1\) :
    \begin{equation}
        \big| S_n-f\big( 1-\frac{ \epsilon }{ n } \big) \big|\leq \epsilon M+\epsilon.
    \end{equation}
    Nous choisissons \( N_1\) tel que \( \frac{ M }{ n }\leq \epsilon^2\) dès que \( n\geq N_1\). En sus nous savons que 
    \begin{equation}
        \lim_{\epsilon\to 0}f(1-\epsilon)=S.
    \end{equation}
    Nous choisissons \( N_2\) de telle sorte à avoir
    \begin{equation}
        \left| f\left( 1-\frac{ \epsilon }{ n } \right)-S \right| <\epsilon,
    \end{equation}
    et en prenant \( n\geq\max(N_1,N_2)\) nous avons
    \begin{equation}
        | S_n-S |\leq \left| S_n-f\left( 1-\frac{ \epsilon }{ n } \right) \right| +\left| f\left( 1-\frac{ \epsilon }{ n } \right)-S \right|  \leq \epsilon M+2\epsilon.
    \end{equation}
    Il suffit de choisit \( \epsilon\) suffisamment petit (en particulier pour que \( \epsilon M\) soit petit) pour montrer que \( | S_n-S |\) est borné par un nombre arbitrairement petit.
\end{proof}
%++++++++++++++++++++++++++++++++++++++++++++++++++++++++++++++++++++++++   
\section{Coordonnées polaires, cylindriques et sphériques}\label{sec_coord}
%++++++++++++++++++++++++++++++++++++++++++++++++++++++++++++++++++++++++   
\subsection{Coordonnées polaires}
Soit $T$ la fonction de $]0, +\infty[\times \eR$ dans $\eR^2\setminus\{(0,0)\}$ définie par
\begin{equation}
  \begin{array}{lccc}
    T: &]0, +\infty[\times \eR & \to & \eR^2\setminus\{(0,0)\}\\
 & (r, \theta)&\mapsto& (r\cos \theta, r \sin \theta),
  \end{array}
\end{equation}
Cette fonction est surjective. Elle est bijective sur chaque bande de la forme  $]0, +\infty[\times [a-\pi,a+\pi[$. Si $a=0$ l'inverse de $T$  est la fonction $T^{-1}(x,y)= (\sqrt{x^2+y^2}, \arctg (y/x))$. Soit $P=(x,y)$ un élément dans $\eR^2$, on dit que $r=\sqrt{x^2+y^2}$ est le rayon de $P$ et que $\theta=\arctg (y/x) $ est son argument principal. L'origine ne peut pas être décrite en coordonnées polaires parce que si son rayon est manifestement zéro, on ne peut pas lui associer une valeur univoque de l'angle $\theta$. 

\begin{example}
L'équation du cercle de rayon $a$ et centre $(0, 0)$ en coordonnées polaires est $r=a$. 
\end{example}

\begin{example}
	Une équation possible pour la demi-droite $x=y$, $x>0$,  est $\theta=\pi/4$.         
\end{example}

%++++++++++++++++++++++++++++++++++++++++++++++++++++++++++++++++++++++++   
\subsection{Coordonnées cylindriques}
%++++++++++++++++++++++++++++++++++++++++++++++++++++++++++++++++++++++++
Soit $T$ la fonction de $]0, +\infty[\times \eR^2$ dans $\eR^3\setminus\{(0,0,0)\}$ définie par
\begin{equation}
  \begin{array}{lccc}
    T: &]0, +\infty[\times \eR\times \eR & \to & \eR^3\setminus\{(0,0,0)\}\\
 & (r, \theta, z)&\mapsto& (r\cos \theta, r \sin \theta, z),
  \end{array}
\end{equation}
Cette fonction est surjective. Elle est bijective sur chaque bande de la forme  $]0, +\infty[\times [a-\pi,a+\pi[\times \eR$, $a$ dans $\eR$. Il n'y a presque rien de nouveau par rapport aux coordonnées polaires. Les coordonnées  cylindriques sont intéressantes si on décrit un objet invariant par rapport aux rotations autour de l'axe des $z$. 

\begin{example}
Il faut savoir ce que décrivent les équations les plus simples en coordonnées cylindriques, 
\begin{itemize}
\item $r\leq a$, pour $a$ constant dans  $]0, +\infty[$, est le cylindre de hauteur infinie qui a pour axe l'axe des $z$ et pour base le disque de rayon $a$ centré à l'origine, 
\item $r= a$ est  la surface du cylindre,
\item $\theta = b$ est un demi-plan ouvert et sa fermeture contient l'axe des $z$,
\item $z=c$ est un plan parallèle au plan $x$-$y$. 
\end{itemize}
\end{example}

\begin{example}
  Un demi-cône qui a  son sommet en l'origine et  pour axe l'axe des $z$ est décrit par $z=d r$.  Si $d$ est positif  il s'agit  de la moitié supérieure du cône, si $d<0$ de la moitié inférieure.
\end{example}

\begin{example}
 De même,  la sphère de rayon $a$ et centrée à l'origine est l'assemblage des calottes $z=\sqrt{a^2-r^2}$ et $z=-\sqrt{a^2-r^2}$. 
\end{example}
%++++++++++++++++++++++++++++++++++++++++++++++++++++++++++++++++++++++++   
\subsection{Coordonnées sphériques}
%++++++++++++++++++++++++++++++++++++++++++++++++++++++++++++++++++++++++
Soit $T$ la fonction de $]0, +\infty[\times \eR^2$ dans $\eR^3\setminus\{(0,0,0)\}$ définie par
\begin{equation}
  \begin{array}{lccc}
    T: &]0, +\infty[\times \eR\times \eR & \to & \eR^3\setminus\{(0,0,0)\}\\
 & (\rho, \theta, \phi)&\mapsto& (\rho\cos \theta\sin \phi, \rho \sin \theta\sin \phi, \rho\cos \phi),
  \end{array}
\end{equation}
Cette fonction est surjective. Elle est bijective sur chaque bande de la forme  $]0, +\infty[\times [a-\pi,a+\pi[\times [b-\pi/2, b+\pi/2[$, $a$ et $b$ dans $\eR$.  Si $a =0$ et $b=-\pi/2$ la fonction inverse $T^{-1}$ est donnée donnée
\begin{equation}
  \begin{array}{lccc}
    T: &\eR^3\setminus\{(0,0,0)\} & \to & ]0, +\infty[\times [-\pi,\pi[\times [0, \pi[\\
 & (x,y,z)&\mapsto& \left(\sqrt{x^2+y^2+z^2}, \arctg \frac{y}{x}, \arccos \left(\frac{z}{\sqrt{x^2+y^2+z^2}}\right)\right). 
  \end{array}
\end{equation}
Soit $ P$ un point dans $\eR^3$. L'angle $\phi$ est l'angle entre le demi-axe positif des $z$ et le vecteur $\overrightarrow{OP}$, $\rho$ est la norme de $\overrightarrow{OP}$ et $\theta$ est l'argument en coordonnées polaires de la projection de $\overrightarrow{OP}$ sur le plan $x$-$y$.  

\begin{remark}
	Dans la littérature, les angles $\theta$ et $\phi$ sont parfois inversés (voire, changent de nom, par exemple $\varphi$ au lieu de $\phi$). Il faut donc être très prudent lorsqu'on veut utiliser dans un cours des formules données dans un autre cours.
\end{remark}

\begin{example}
Il faut connaître le sens des équations plus simples, 
\begin{itemize}
\item $\rho\leq a$, pour $a$ constant dans  $]0, +\infty[$, est la boule fermée de rayon $a$ centrée à l'origine, 
\item $\rho= a$ est  la sphère de rayon $a$ centrée à l'origine,
\item $\theta = b$ est un demi-plan ouvert et sa fermeture contient l'axe des $z$,
\item $\phi= c$ est un demi-cône qui a  son sommet à l'origine et  pour axe l'axe des $z$.  Si $c$ est positif  il s'agit  de la moitié supérieure du cône, si $d<0$ de la moitié inférieure. 
\end{itemize}
 \end{example}

%++++++++++++++++++++++++++++++++++++++++++++++++++++++++++++++++++++++++++++++++++++++++++++++++++
\section{Changement de variables}
%++++++++++++++++++++++++++++++++++++++++++++++++++++++++++++++++++++++++++++++++++++++++++++++++++
\begin{theorem}		\label{ThoChmVarInt}
  Soient $U$ et $V$ deux ouverts bornés de $\eR^p$, $\phi$ un difféomorphisme de classe $\mathcal{C}^1$ de $U$ sur $V$ et $f$ une fonction intégrable de $V$ sur $\eR$. Alors nous avons la formule de changement de variables 
  \begin{equation}
    \int_{V}f(y)\, dy= \int_{U} f(\phi(x))\, \left| J_{\phi}(x)\right|\, dx,
  \end{equation}
  où $J_{\phi}$ est le déterminant de la matrice jacobienne\index{jacobienne} de $\phi$. 
\end{theorem}
Si $\phi$ est linéaire  alors le facteur $|J_{\phi}|$ est la mesure de l'image par $\phi$ d'une portion de $\eR^p$ de mesure $1$, sinon  $|J_{\phi}|$ est le rapport entre la mesure de l'image d'un élément infinitésimale de volume de $\eR^p$ et sa mesure originale. 
Soit $\phi(u,v)=g(u,v)e_1+h(u,v)e_2$ un difféomorphisme dans $\eR^2$. Soit $(x_0, y_0)$ l'image par $\phi$ de $(u_0,v_0)$. On considère le petit rectangle $R$ de sommets $(u_0,v_0)$, $(u_0+\Delta u,v_0)$, $(u_0+\Delta u,v_0+\Delta v)$ et $(u_0,v_0+\Delta v)$. L'image de $R$ n'est pas un rectangle en général, mais peut être bien approximée par le rectangle de sommets $(x_0,y_0)$, $(x_0 ,y_0)+ \phi_{u}\Delta u$, $(x_0 ,y_0)+\phi_{u}\Delta u +\phi_{v}\Delta v$ et  $(x_0 ,y_0)+ \phi_{v}\Delta v$ et son aire est $\| \phi_{u}\times \phi_{v}\| \Delta u\Delta v$. La valeur $|\phi_{u}\times \phi_{v}|$ est exactement $|J_{\phi}|$ 

\begin{example}
Soit $V$ la région trapézoïdale de sommets $(0,-1)$, $(1,0)$, $(2,0)$, $(0,-2)$, comme à la figure \ref{LabelFigZTTooXtHkcissLabelSubFigZTTooXtHkci0}. Calculons ensemble l'intégrale double  
\[
\int_{V}e^{\frac{x+y}{x-y}}\,dV,
\] 
avec le changement de variable $\psi(x,y)=(x+y,x-y)$. C'est à dire que nous considérons les nouvelles variables
\begin{subequations}
	\begin{numcases}{}
		u=x+y\\
		v=x-y.
	\end{numcases}
\end{subequations}
Il faut remarquer d'abord que le changement de variable proposé est dans le mauvais sens. On écrit alors $\phi(u,v)=\psi^{-1}(u,v)=\big((u+v)/2, (u-v)/2\big)$, c'est à dire
\begin{subequations}
	\begin{numcases}{}
		x=\frac{ u+v }{ 2 }\\
		y=\frac{ u-v }{2}.
	\end{numcases}
\end{subequations}
La région qui correspond à $V$ est $U$, le trapèze de sommets  $(-1,1)$, $(1,1)$, $(2,2)$ et $(-2,2)$, qu'on voit sur la figure \ref{LabelFigZTTooXtHkcissLabelSubFigZTTooXtHkci1} et qu'on décrit par
\[
U=\{ (u,v)\in\eR^2\,\vert\, 1\leq v\leq 2, \, -v\leq u\leq v\}.
\] 

% Celui-ci a été supprimée le 17 juillet 2014
%\ref{LabelFigexamplechangementvariables}
%\newcommand{\CaptionFigexamplechangementvariables}{Avant et après le changement de variables}
%\input{Fig_examplechangementvariables.pstricks}

%The result is on figure \ref{LabelFigZTTooXtHkci}. % From file ZTTooXtHkci
%See also the subfigure \ref{LabelFigZTTooXtHkcissLabelSubFigZTTooXtHkci0}
%See also the subfigure \ref{LabelFigZTTooXtHkcissLabelSubFigZTTooXtHkci1}
\newcommand{\CaptionFigZTTooXtHkci}{Avant et après le changement de variables}
\input{Fig_ZTTooXtHkci.pstricks}

On observe que $U$ est une région du premier type tandis que $V$ n'est pas du premier ou du deuxième type. Le déterminant de la  matrice  jacobienne de $\psi^{-1}$ est  $J_{\psi^{-1}}$,
\begin{equation}
 J_{\psi^{-1}}(u,v)= \left\vert\begin{array}{cc}
\frac{1}{2} & \frac{1}{2} \\
\frac{1}{2}  & -\frac{1}{2}
\end{array}\right\vert= -\frac{1}{2}.
\end{equation}
On a alors 
\[
\int_{V}e^{\frac{x+y}{x-y}}\,dV=\int_{U}e^{\frac{u}{v}}\,\frac{1}{2}\,dV=\int_1^2\int_{-v}^{v}e^{\frac{u}{v}}\,\frac{1}{2}\, du\,dv= \frac{3}{4}(e-e^{-1}).
\] 
\end{example}

\begin{example} 
\textbf{Coordonnées polaires : }On veut évaluer l'intégrale de la fonction $f(x,y)= x^2+y^2$ sur la région $V$ suivante :
\[
V=\{(x,y) \in \eR^2\,\vert\, x^2+y^2\leq 1,\, x>0,\, y>0\}.
\]
On peut faire le calcul directement,
\[
\int_{V}f(x,y)\, dV=\int_0^1\int_0^{\sqrt{1-x^2}}x^2+y^2\, dy\,dx=\int_0^1x^2\sqrt{1-x^2} + \frac{(1-x^2)^{3/2}}{3}\, dx  
\] 
mais c'est un peu ennuyeux. On peut simplifier beaucoup les calculs avec un changement de variables vers les coordonnées polaires. Dans ce cas, on sait bien que le difféomorphisme à utiliser est $\phi(r,\theta)=(r\cos \theta, r\sin\theta)$. Le jacobien  $J_{\phi}$ est
\begin{equation}
 J_{\phi}(r, \theta)= \left\vert\begin{array}{cc}
\cos \theta & \sin \theta \\
-r\sin \theta  & r\cos \theta
\end{array}\right\vert= r,
\end{equation}
qui est toujours positif. La fonction $f$ peut s'écrire comme $f(\phi(r,\theta))=r^2$ et $\phi^{-1}(V)=]0,1]\times]0, \pi/2[$.  
La formule du changement de variables nous donne
\[
\int_{V}f(x,y)\, dV=\int_0^{\pi/2}\int_0^{1}r^3 dr\,d\theta=\int_0^{\pi/2}\frac{1}{4}\,d\theta=\frac{\pi}{8}.  
\] 
\end{example}

\begin{example}
\textbf{Coordonnées cylindriques : }On veut calculer le volume de la région $A$ définie par  l'intersection entre la boule unité et le cylindre qui a pour base un disque de rayon $1/2$ centré en $(0, 1/2)$
\[
A=\{(x,y,z) \in\eR^3 \,\vert\, x^2+y^2+z^1\leq 1\}\cap\{(x,y,z) \in \eR^3\,\vert\, x^2+(y-1/2)^2\leq 1/4\}.
\]
On peut décrire $A$ en coordonnées cylindriques
\begin{equation}
  \begin{aligned}
    A=\Big\{(r,\theta,z) &\in ]0, +\infty[\times [-\pi,\pi[\times \eR\,\vert\,\\
& -\pi/2<\theta<\pi, \, 0<r\leq \sin\theta, \, -\sqrt{1-r^2}\leq z\leq\sqrt{1-r^2} \Big\}.
  \end{aligned}
\end{equation}
Le jacobien de ce changement de variables,  $J_{cyl}$, est
\begin{equation}
 J_{cyl}(r, \theta), z= \left\vert\begin{array}{ccc}
\cos \theta & \sin \theta & 0\\
-r\sin \theta  & r\cos \theta &0 \\
0&0&
\end{array}\right\vert= r,
\end{equation}
qui est toujours positif. Le volume de $A$ est donc
\[
\int_{\eR^3}\chi_{A}(x,y,z)\, dV=\int_{-\pi/2}^{\pi/2}\int_0^{\sin\theta}\int_{-\sqrt{1-r^2}}^{\sqrt{1-r^2}} r dz\,dr\,d\theta=\frac{2\pi}{8}+\frac{8}{9}.  
\] 
\end{example}
\begin{example}
\textbf{Volume d'un solide de révolution : }Soit $g:[a,b]\to\eR_+$ une fonction continue et positive. On dit que le solide $A$ décrit par
\[
A=\left\{(x,y,z)\in\eR^3\, \vert \, z\in[a,b], \,\sqrt{x^2+y^2}\leq g^2(z) \right\}
\]
est un solide de révolution. Afin de calculer son volume, on peut décrire $A$ en coordonnées cylindriques, 
\[
A=\left\{(r,\theta,z) \in ]0, +\infty[\times [-\pi,\pi[\times \eR\,\vert\, a\leq z\leq b, \, 0<r^2\leq g^2(z) \right\}.
\]
Le jacobien de ce changement de variables est  $J_{cyl}=r$, comme dans l'exemple précédent. Le volume de $A$ est donc
\[
\int_{\eR^3}\chi_{A}(x,y,z)\, dV=\int_a^{b}\int_{-\pi}^{\pi}\int_{0}^{g(z)} r  \,dr\,d\theta\, dz=\int_a^{b} \pi g^2(z) \, dz.
\] 
Cette formule peut être utilisée pour tout solide de révolution. 
\end{example}

\begin{example}
\textbf{Coordonnées sphériques : }On veut calculer le volume du cornet de glace  $A$ 
\[
A=\left\{(x,y,z)\in\eR^3\, \vert \, (x,y)\in \mathbb{S}^2, \,\sqrt{x^2+y^2}\leq z\leq \sqrt{1-x^2-y^2} \right\}. 
\]
On peut décrire $A$ en coordonnées sphériques. 
\[
A=\{(\rho,\theta,\phi) \in ]0, +\infty[\times [-\pi,\pi[\times [0,\pi[\,\vert\, 0<\phi\leq\pi/4, \, 0<\rho\leq 1 \}.
\]
Le jacobien de ce changement de variables  $J_{sph}$ est
\begin{equation}
 J_{sph}(\rho, \theta, \phi)= \left\vert\begin{array}{ccc}
\cos \theta \sin\phi & \sin \theta\sin\phi & \cos\phi\\
-\rho\sin \theta\sin\phi  & \rho\cos \theta\sin\phi & 0 \\
\rho\cos\theta\cos\phi&\rho\sin\theta\cos\phi& -\rho\sin\phi
\end{array}\right\vert= \rho^2\sin\phi,
\end{equation}
Le volume de $A$ est donc
\[
\int_{\eR^3}\chi_{A}(x,y,z)\, dV=\int_{-\pi}^{\pi}\int_0^{\pi/4}\int_{0}^{1}\rho^2\sin\phi \,d\rho\,d\phi\,d\theta=\frac{2\pi}{3}\left(1-\frac{1}{\sqrt{2}}\right).  
\] 
\end{example}

%---------------------------------------------------------------------------------------------------------------------------
\subsection{Récapitulatif des changements de variables}
%---------------------------------------------------------------------------------------------------------------------------

En pratique, nous retiendrons les formules suivantes:
%///////////////////////////////////////////////////////////////////////////////////////////////////////////////////////////
\subsubsection{Coordonnées polaires}
%///////////////////////////////////////////////////////////////////////////////////////////////////////////////////////////

\begin{subequations}
    \begin{numcases}{}
        x=r\cos\theta\\
        y=r\sin\theta
    \end{numcases}
\end{subequations}
avec \( r\in\mathopen] 0 , \infty \mathclose[\) et \( \theta\in\mathopen[ 0 , 2\pi [\). Le jacobien vaut \( r\).

%///////////////////////////////////////////////////////////////////////////////////////////////////////////////////////////
\subsubsection{Coordonnées cylindriques}
%///////////////////////////////////////////////////////////////////////////////////////////////////////////////////////////

\begin{subequations}
    \begin{numcases}{}
        x=r\cos\theta\\
        y=r\sin\theta\\
        z=z
    \end{numcases}
\end{subequations}
avec \( r\in\mathopen] 0 , \infty \mathclose[\), \( \theta\in\mathopen[ 0 , 2\pi [\) et \( z\in\eR\). Le jacobien vaut \( r\).

%///////////////////////////////////////////////////////////////////////////////////////////////////////////////////////////
\subsubsection{Coordonnées sphériques}
%///////////////////////////////////////////////////////////////////////////////////////////////////////////////////////////

\begin{subequations}
    \begin{numcases}{}
        x=\rho\cos\theta\sin\phi\\
        y=\rho\sin\theta\sin\phi\\
        z=\rho\cos\phi
    \end{numcases}
\end{subequations}
avec \( \rho\in\mathopen] 0 , \infty \mathclose[\), \( \theta\in\mathopen[ 0 , 2\pi [\) et \( \phi\in\mathopen[ 0 , \pi [\). Le jacobien vaut \( -\rho^2\sin(\phi)\). 

N'oubliez pas que lorsqu'on effectue un changement de variables dans une intégrale, la \emph{valeur absolue} du jacobien apparaît.

Cependant notre convention de coordonnées sphériques fait venir \( \sin(\phi)\) avec \( \phi\in\mathopen[ 0 , \pi [\); vu que le signe de \( \sin(\phi)\) y est toujours positif, cette histoire de valeur absolue est sans grandes conséquent. Ce n'est pas le cas de toutes les conventions possibles.

%---------------------------------------------------------------------------------------------------------------------------
					\subsection{Changement de variables}
%---------------------------------------------------------------------------------------------------------------------------

Le domaine $E=\{ (x,y)\in\eR^2\tq x^2+y^2<1 \}$ s'écrit plus facilement $E=\{ (r,\theta)\tq r<1 \}$ en coordonnées polaires. Le passage aux coordonnées polaire permet de transformer une intégration sur un domaine rond à une intégration sur le domaine rectangulaire $\mathopen]0,2\pi\mathclose[\times\mathopen]0,1\mathclose[$. La question est évidement de savoir si nous pouvons écrire
\begin{equation}
	\int_Ef=\int_{0}^{2\pi}\int_0^1f(r\cos\theta,r\sin\theta)drd\theta.
\end{equation}
Hélas, non; la vie n'est pas aussi simple.

\begin{theorem}
Soit $g\colon A\to B$ un difféomorphisme. Soient $F\subset B$ un ensemble mesurable et borné et $f\colon F\to \eR$ une fonction bornée et intégrable. Supposons que $g^{-1}(F)$ soit borné et que $Jg$ soit borné sur $g^{-1}(F)$. Alors
\begin{equation}
	\int_Ff(x)dy=\int_{g^{-1}(F)f\big( g(x) \big)}| Jg(x) |dx
\end{equation}
\end{theorem}
Pour rappel, $Jg$ est le déterminant de la matrice \href{http://fr.wikipedia.org/wiki/Matrice_jacobienne}{jacobienne} (aucun lien de \href{http://fr.wikipedia.org/wiki/Jacob}{parenté}) donnée par
\begin{equation}
	Jg=\det\begin{pmatrix}
	\partial_xg_1	&	\partial_yg_1	\\ 
	\partial_xg_2	&	\partial_tg_2	
\end{pmatrix}.
\end{equation}
Un \defe{difféomorphisme}{difféomorphisme} est une application $g\colon A\to B$ telle que $g$ et $g^{-1}\colon B\to A$ soient de classe $C^1$.

%///////////////////////////////////////////////////////////////////////////////////////////////////////////////////////////
					\subsubsection{Coordonnées polaires}
%///////////////////////////////////////////////////////////////////////////////////////////////////////////////////////////

Les coordonnées polaires sont données par le difféomorphisme
\begin{equation}
	\begin{aligned}
		g\colon \mathopen]0,\infty\mathclose[\times\mathopen]0,2\pi\mathclose[ &\to\eR^2\setminus D\\
		(r,\theta)&\mapsto \big( r\cos(\theta),r\sin(\theta) \big)
	\end{aligned}
\end{equation}
où $D$ est la demi droite $y=0$, $x\geq 0$. Le fait que les coordonnées polaires ne soient pas un difféomorphisme sur tout $\eR^2$ n'est pas un problème pour l'intégration parce que le manque de difféomorphisme est de mesure nulle dans $\eR^2$. Le jacobien est donné par
\begin{equation}
	Jg=\det\begin{pmatrix}
	\partial_rx	&	\partial_{\theta}x	\\ 
	\partial_ry	&	\partial_{\theta}y
\end{pmatrix}=\det\begin{pmatrix}
	\cos(\theta)	&	-r\sin(\theta)	\\ 
	\sin(\theta)	&	r\cos(\theta)	
\end{pmatrix}=r.
\end{equation}

\begin{example}    
    Montrons comment intégrer la fonction $f(x,y)=\sqrt{1-x^2-y^2}$ sur le domaine délimité par la droite $y=x$ et le cercle $x^2+y^2=y$, représenté sur la figure \ref{LabelFigQXyVaKD}. Pour trouver le centre et le rayon du cercle $x^2+y^2=y$, nous commençons par écrire $x^2+y^2-y=0$, et ensuite nous reformons le carré : $y^2-y=(y-\frac{ 1 }{2})^2-\frac{1}{ 4 }$.
    \newcommand{\CaptionFigQXyVaKD}{Passage en polaire pour intégrer sur un morceau de cercle.}
\input{Fig_QXyVaKD.pstricks}

    Le passage en polaire transforme les équations du bord du domaine en
    \begin{equation}
        \begin{aligned}[]
            \cos(\theta)&=\sin(\theta)\\
            r^2&=r\sin(\theta).
        \end{aligned}
    \end{equation}
    L'angle $\theta$ parcours donc $\mathopen] 0 , \pi/4 \mathclose[$, et le rayon, pour chacun de ces $\theta$ parcours $\mathopen] 0 , \sin(\theta) \mathclose[$. La fonction à intégrer se note maintenant $f(r,\theta)=\sqrt{1-r^2}$. Donc l'intégrale à calculer est
    \begin{equation}		\label{PgRapIntMultFubiniBoutCercle}
        \int_{0}^{\pi/4}\left( \int_0^{\sin(\theta)}\sqrt{1-r^2}r\,rd \right).
    \end{equation}
    Remarquez la présence d'un $r$ supplémentaire pour le jacobien.

    Notez que les coordonnées du point $P$ sont $(1,1)$.
\end{example}

%///////////////////////////////////////////////////////////////////////////////////////////////////////////////////////////
\subsubsection{Coordonnées sphériques}
%///////////////////////////////////////////////////////////////////////////////////////////////////////////////////////////

Les coordonnées sphériques sont données par
\begin{equation}		\label{EqChmVarSpherique}
	\left\{
\begin{array}{lllll}
x=r\cos\theta\sin\varphi	&			&r\in\mathopen] 0 , \infty \mathclose[\\
y=r\sin\theta\sin\varphi	&	\text{avec}	&\theta\in\mathopen] 0 , 2\pi \mathclose[\\
z=r\cos\varphi			&			&\phi\in\mathopen] 0 , \pi \mathclose[.
\end{array}
\right.
\end{equation}
Le jacobien associé est $Jg(r,\theta,\varphi)=-r^2\sin\varphi$. Rappelons que ce qui rentre dans l'intégrale est la valeur absolue du jacobien.

Si nous voulons calculer le volume de la sphère de rayon $R$, nous écrivons donc
\begin{equation}
	\int_0^Rdr\int_{0}^{2\pi}d\theta\int_0^{\pi}r^2 \sin(\phi)d\phi=4\pi R=\frac{ 4 }{ 3 }\pi R^3.
\end{equation}
Ici, la valeur absolue n'est pas importante parce que lorsque $\phi\in\mathopen] 0,\pi ,  \mathclose[$, le sinus de $\phi$ est positif.

Des petits malins pourraient remarquer que le changement de variable \eqref{EqChmVarSpherique} est encore une paramétrisation de $\eR^3$ si on intervertit le domaine des angles : 
\begin{equation}
	\begin{aligned}[]
		\theta&\colon 0 \to \pi\\
		\phi	&\colon 0\to 2\pi,
	\end{aligned}
\end{equation}
alors nous paramétrons encore parfaitement bien la sphère, mais hélas
\begin{equation}		\label{EqVolumeIncorrectSphere}
	\int_0^Rdr\int_{0}^{\pi}d\theta\int_0^{2\pi}r^2 \sin(\phi)d\phi=0.
\end{equation}
Pourquoi ces «nouvelles» coordonnées sphériques sont-elles mauvaises ? Il y a que quand l'angle $\phi$ parcours $\mathopen] 0 , 2\pi \mathclose[$, son sinus n'est plus toujours positif, donc la \emph{valeur absolue} du jacobien n'est plus $r^2\sin(\phi)$, mais $r^2\sin(\phi)$ pour les $\phi$ entre $0$ et $\pi$, puis $-r^2\sin(\phi)$ pour $\phi$ entre $\pi$ et $2\pi$. Donc l'intégrale \eqref{EqVolumeIncorrectSphere} n'est pas correcte. Il faut la remplacer par
\begin{equation}
	\int_0^Rdr\int_{0}^{\pi}d\theta\int_0^{\pi}r^2 \sin(\phi)d\phi- \int_0^Rdr\int_{0}^{\pi}d\theta\int_{\pi}^{2\pi}r^2 \sin(\phi)d\phi = \frac{ 4 }{ 3 }\pi R^3
\end{equation}

%---------------------------------------------------------------------------------------------------------------------------
					\subsection{Passage à la limite sous le signe intégral}
%---------------------------------------------------------------------------------------------------------------------------

Un autre résultat très important pour l'étude de l'intégrabilité est le théorème de la \defe{convergence dominée de Lebesgue}{}:
\begin{theorem}
	Soit $E\subset \eR^n$ un ensemble mesurable et $\{ f_k \}$, une suite de fonctions intégrables sur $E$ qui converge simplement vers une fonction $f\colon E\to \overline{ \eR }$. Supposons qu'il existe une fonction $g$ intégrable sur $E$ telle que pour tout $k$,
\begin{equation}
	| f(x) |\leq g(x)
\end{equation}
pour tout $x\in E$. Alors $f$ est intégrable sur $E$ et 
\begin{equation}
	\int_Ef=\lim_{k\to\infty}\int_Ef_k.
\end{equation}
\end{theorem}

%---------------------------------------------------------------------------------------------------------------------------
					\subsection{Théorème de Fubini et changement de variables}
%---------------------------------------------------------------------------------------------------------------------------

\begin{theorem}[Fubini]\label{ThoFubini}
Soit $(x,t)\mapsto f(x,y)\in\bar \eR$ une fonction intégrable sur $B_n\times B_m\subset\eR^{n+m}$ où $B_n$ et $B_m$ sont des ensembles mesurables de $\eR^n$ et $\eR^m$. Alors :
\begin{enumerate}
\item pour tout $x\in B_n$, sauf éventuellement en les points d'un ensemble $G\subset B_n$ de mesure nulle, la fonction $y\in B_m\mapsto f(x,y)\in\bar\eR$ est intégrable sur $B_m$
\item
la fonction
\begin{equation}
	x\in B_n\setminus G\mapsto\int_{B_m}f(x,y)dy\in\eR
\end{equation}
est intégrable sur $B_n\setminus G$

\item 
On a
\begin{equation}
	\int_{B_n\times B_m}f(x,y)dxdy=\int_{B_n}\left( \int_{B_m}f(x,y)dy \right)dx.
\end{equation}

\end{enumerate}
\end{theorem}
\index{théorème!Fubini!dans $ \eR^n$}
\index{Fubini!théorème!dans $ \eR^n$}


Notons en particulier que si $f(x,y)=\varphi(x)\phi(y)$, alors $\int_{B_m}\varphi(y)dy$ est une constante qui peut sortir de l'intégrale sur $B_n$, et donc
\begin{equation}		\label{EqFubiniFactori}
	\int_{B_n\times B_m}\varphi(x)\phi(y)dxdy=\int_{B_n}\varphi(x)dx\int_{B_m}\phi(y)dy.
\end{equation}

%---------------------------------------------------------------------------------------------------------------------------
					\subsection{Intégrale en dimension un}
%---------------------------------------------------------------------------------------------------------------------------

\begin{proposition}[Critère de comparaison]
Soit $f$ mesurable sur $]a,\infty[$ et bornée sur tout $]a,b]$, et supposons qu'il existe un $X_0\geq a$, tel que sur $]X_0,\infty[$,
\begin{equation}
	| f(x) |\leq g(x)
\end{equation}
où $g(x)$ est intégrable. Alors $f(x)$ est intégrable sur $]a,\infty[$.
\end{proposition}

\begin{corollary}[Critère d'équivalence]
Soient $f$ et $g$ des fonctions mesurables et positives ou nulles sur $]a,\infty[$, bornées sur tout $]a,b]$, telles que 
\begin{equation}
	\lim_{x\to\infty}\frac{ f(x) }{ g(x) }=L
\end{equation}
existe dans $\bar\eR$.
\begin{enumerate}
\item Si $L\neq\infty$ et $\int_{a}^{\infty}g(x)$ existe, alors $\int_a^{\infty}f(x)dx$ existe,
\item Si $L\neq 0$ et si $\int_a^{\infty}f(x)dx$ existe, alors $\int_a^{\infty}g(x)dx$ existe,
\end{enumerate}
\end{corollary}

\begin{corollary}[Critère des fonctions test]			\label{CorCritFonsTest}
Soit $f(x)$ une fonction mesurable et positive ou nulle sur $]a,\infty[$ et bornée pour tout $]a,b]$. Nous posons
\begin{equation}
	L(\alpha)=\lim_{x\to\infty}x^{\alpha}f(x),
\end{equation}
et nous supposons qu'elle existe.
\begin{enumerate}
\item Si il existe $\alpha>1$ tel que $L(\alpha)\neq\infty$, alors $\int_a^{\infty}f(x)dx$ existe,
\item Si il existe $\alpha\leq1$ et $L(\alpha)\neq 0$, alors $\int_a^{\infty}f(x)dx$ n'existe pas.
\end{enumerate}
\end{corollary}

\begin{corollary}		\label{CorAlphaLCasInteabf}
	Soit $f\colon ]a,b]\to \eR$ une fonction mesurable, positive ou nulle, et bornée sur $[a+\epsilon,b]$ $\forall\epsilon>0$. Si $\lim_{x\to a}(x-a)^{\alpha}f(x)=L$ existe, alors
	\begin{enumerate}
		\item Si $\alpha<1$ et $L\neq\infty$, alors $\int_a^bf(x)dx$ existe,
		\item Si $\alpha\geq 1$ et $L\neq 0$, alors $\int_a^bf(x)dx$ n'existe pas.
	\end{enumerate}
\end{corollary}

%---------------------------------------------------------------------------------------------------------------------------
					\subsection{Intégrales convergentes}
%---------------------------------------------------------------------------------------------------------------------------

\begin{definition}
    Soit $f$, une fonction mesurable sur $[a,\infty[$, bornée sur tout intervalle $[a,b]$. On dit que l'intégrale
    \begin{equation}
        \int_a^{\infty}f(x)dx
    \end{equation}
    \defe{converge}{intégrale!convergente} si la limite
    \begin{equation}		\label{EqDEfConvergeZeroInftX}
        \lim_{X\to\infty}\int_a^{X}f
    \end{equation}
    existe et est finie.
\end{definition}

%+++++++++++++++++++++++++++++++++++++++++++++++++++++++++++++++++++++++++++++++++++++++++++++++++++++++++++++++++++++++++++
\section{Trucs et astuces de calcul d'intégrales}
%+++++++++++++++++++++++++++++++++++++++++++++++++++++++++++++++++++++++++++++++++++++++++++++++++++++++++++++++++++++++++++

Afin d'alléger le texte de calculs parfois un peu longs, nous regroupons ici les intégrales à une variable que nous devons utiliser dans les autres parties du cours.

\begin{enumerate}
	\item	\label{ItemIntegrali}
		L'intégrale
		\begin{equation}
			\boxed{I=\int x\ln(x)dx=\frac{ x^2 }{2}\big( \ln(x)-\frac{ 1 }{2} \big)}
		\end{equation}
		se fait par partie en posant
		\begin{equation}
			\begin{aligned}[]
				u&=\ln(x),		& dv&=x\,dx\\
				du&=\frac{1}{ x }\,dx,	& v&=\frac{ x^2 }{2},
			\end{aligned}
		\end{equation}
		et ensuite
		\begin{equation}
			I=\ln(x)\frac{ x^2 }{2}-\int\frac{ x }{2}=\frac{ x^2 }{2}\big( \ln(x)-\frac{ 1 }{2} \big).
		\end{equation}
		
	\item	
		L'intégrale
		\begin{equation}
			\boxed{I=\int x\ln(x^2)dx=x^2\ln(x)-\frac{ x^2 }{2}.}
		\end{equation}
		En utilisant le fait que $\ln(u^2)=2\ln(u)$, nous retombons sur une intégrale du type \ref{ItemIntegrali} :
		\begin{equation}
			I=x^2\ln(x)-\frac{ x^2 }{2}.
		\end{equation}
	\item
		L'intégrale
		\begin{equation}		\label{EqTrucIntxlnxsqpun}
			\boxed{I=\int x\ln(1+x^2)dx=\frac{ 1 }{2}\ln(x^2+1)(x^2+1)-x^2-\frac{ 1 }{2}}
		\end{equation}
		se traite en posant $v=1+x^2$ de telle sorte à avoir $dx=\frac{ dv }{ 2x }$ et donc
		\begin{equation}
			I=\frac{ 1 }{2}\ln(x^2+1)(x^2+1)-x^2-\frac{ 1 }{2}.
		\end{equation}
		
	\item
		L'intégrale
		\begin{equation}
			I=\int \cos(\theta)\sin(\theta)\ln\left( 1+\frac{1}{ \cos^2(\theta) } \right)\,d\theta
		\end{equation}
		demande le changement de variable $u=\cos(\theta)$, $d\theta=-\frac{ du }{ \sin(\theta) }$. Nous tombons sur l'intégrale
		\begin{equation}
			I=-\int u\ln\left( \frac{ 1+u^2 }{ u^2 } \right)=-\int u\ln(1+u^2)+\int u\ln(u^2),
		\end{equation}
		qui sont deux intégrales déjà faites. Nous trouvons
		\begin{equation}
			I=-\frac{ 1 }{2}\ln\left( \frac{ \sin^2(\theta)-1 }{ \sin^2(\theta)-2 } \right)\sin^2(\theta)-\ln\big( \sin^2(\theta)-2 \big)+\frac{ 1 }{2}\ln\big( \sin^2(\theta)-1 \big)
		\end{equation}
	
	\item
		L'intégrale
		\begin{equation}
			\boxed{\int \frac{ r^3 }{ 1+r^2 }dr=\frac{ r^2 }{2}-\frac{ 1 }{2}\ln(r^2+1).}
		\end{equation}
		commence par faire la division euclidienne de $r^3$ par $r^2+1$; ce que nous trouvons est $r^3=(r^2+1)r-r$. Il reste à intégrer
		\begin{equation}
			\int \frac{ r^3 }{ 1+r^2 }dr=\int r\,dr-\int\frac{ r }{ 1+r^2 }dr.
		\end{equation}
		La fonction dans la seconde intégrale est $\frac{ r }{ 1+r^2 }=\frac{ 1 }{2}\frac{ f'(r) }{ f(r) }$ où $f(r)=1+r^2$, et donc $\int \frac{ r }{ 1+r^2 }=\frac{ 1 }{2}\ln(1+r^2)$. Au final,
		\begin{equation}
			I=\frac{ 1 }{2}r^2-\frac{ 1 }{2}\ln(r^2+1).
		\end{equation}


	\item	
		L'intégrale
		\begin{equation}	\label{EqTrucIntsxcxdx}
			\boxed{I=\int \cos(\theta)\sin(\theta)d\theta=\frac{ \sin^2(\theta) }{ 2 }}
		\end{equation}
		se traite par le changement de variable $u=\sin(\theta)$, $du=\cos(\theta)d\theta$, et donc
		\begin{equation}
			\int\cos(\theta)\sin(\theta)d\theta=\int udu=\frac{ u^2 }{2}=\frac{ \sin^2(\theta) }{ 2 }.
		\end{equation}
	\item
		L'intégrale
		\begin{equation}	\label{EqTrucsIntsqrtAplusu}
			\boxed{\int\sqrt{1+x^2}dx=\frac{ x }{2}\sqrt{1+x^2}+\frac{ 1 }{2}\arcsinh(x)}
		\end{equation}
		s'obtient en effectuant le changement de variable $u=\sinh(\xi)$.

    \item
        L'intégrale
        \begin{equation}        \label{EqTrucIntcossqsinsq}
            \boxed{ \int\cos^2(x)\sin^2(x)dx=\frac{ x }{ 8 }-\frac{ \sin(4x) }{ 32 } }
        \end{equation}
        s'obtient à coups de formules de trigonométrie. D'abord, $\sin(t)\cos(t)=\frac{ 1 }{2}\sin^2(2t)$ fait en sorte que la fonction à intégrer devient 
        \begin{equation}
            f(x)=\frac{1}{ 4 }\sin^2(x).
        \end{equation}
        Ensuite nous utilisons le fait que $\sin^2(t)=(1-\cos(2t))/2$ pour transformer la formule à intégrer en
        \begin{equation}
            f(x)=\frac{ 1-\cos(4x) }{ 8 }.
        \end{equation}
        Cela s'intègre facilement en posant $u=4x$, et le résultat est
        \begin{equation}
            \int f(x)dx=\frac{ x }{ 8 }-\frac{ \sin(4x) }{ 32 }.
        \end{equation}

    \item

        La fonction 
        \begin{equation}
            \sinc(x)=\frac{ \sin(x) }{ x }
        \end{equation}
        est le \defe{sinus cardinal}{sinus cardinal} de \( x\). Nous allons montrer que
        \begin{equation}    \label{EqKNOmLEd}
            \boxed{  \int_0^{\infty}\big| \sinc(x) \big|dx=\infty  }.
        \end{equation}
        D'abord nous avons
        \begin{equation}
            \int_{(n-1)\pi}^{n\pi}\frac{ \big| \sin(t) \big| }{ t }dt\geq \int_{(n-1)\pi}^{n\pi}\frac{ \big| \sin(t) \big| }{ n\pi }dt,
        \end{equation}
        mais par périodicité,
        \begin{equation}
            \int_{(n-1)\pi}^{n\pi}\big| \sin(t) \big|dt=\int_0^{\pi}\sin(t)dt=2.
        \end{equation}
        Par conséquent
        \begin{equation}
            \int_0^{n\pi}\big| \sinc(t) \big|dt\geq \frac{ 2 }{ \pi }\sum_{k=1}^n\frac{1}{ k },
        \end{equation}
        ce qui diverge lorsque \( n\to \infty\).

\end{enumerate}

%+++++++++++++++++++++++++++++++++++++++++++++++++++++++++++++++++++++++++++++++++++++++++++++++++++++++++++++++++++++++++++ 
\section{Ellipsoïde de John-Loewer}
%+++++++++++++++++++++++++++++++++++++++++++++++++++++++++++++++++++++++++++++++++++++++++++++++++++++++++++++++++++++++++++

Soit \( q\) une forme quadratique sur \( \eR^n\) ainsi que \( \mB\) une base orthonormée de \( \eR^n\) dans laquelle la matrice de  \( q\) est diagonale. Dans cette base, la forme \( q\) est donnée par la proposition \ref{PropFWYooQXfcVY} :
\begin{equation}
    q(x)=\sum_i\lambda_ix_i
\end{equation}
où les \( \lambda_i\) sont les valeurs propres de \( q\).

Plus généralement nous notons \( mat_{\mB}(q)\)\nomenclature[A]{\( mat_{\mB}(q)\)}{matrice de \( q\) dans la base \( \mB\)} la matrice de \( q\) dans la base \( \mB\) de \( \eR^n\).

\begin{proposition} \label{PropOXWooYrDKpw}
    Soit \( \mB\) une base orthonormée de \( \eR^n\) et l'application\footnote{L'ensemble \( Q(E)\) est l'ensemble des formes quadratiques sur \( E\).}
    \begin{equation}
        \begin{aligned}
            D\colon Q(\eR^n)&\to \eR \\
            q&\mapsto \det\big( mat_{\mB}(q) \big) .
        \end{aligned}
    \end{equation}
    Alors :
    \begin{enumerate}
        \item
            La valeur et \( D\) ne dépend pas du choix de la base orthonormée \( \mB\).
        \item
            La fonction \( D\) est donnée par la formule \( D(q)=\prod_i\lambda_i\) où les \( \lambda_i\) sont les valeurs propres de \( q\).
        \item
            La fonction \( D\) est continue.
    \end{enumerate}
\end{proposition}

\begin{proof}
    Soit \( q\) une forme quadratique sur \( \eR^n\). Nous considérons \( \mB\) une base de diagonalisation de \( q\) :
    \begin{equation}
        q(x)=\sum_i\lambda_ix_i
    \end{equation}
    où les \( x_i\) sont les composantes de \( x\) dans la base \( \mB\). Par définition, la matrice \( mat_{\mB}(q)\) est la matrice diagonale contenant les valeurs propres de \( q\).

    Nous considérons aussi \( \mB_1\), une autre base orthonormées de \( \eR^n\). Nous notons \( S=mat_{\mB_1}(q)\); étant symétrique, cette matrice se diagonalise par une matrice orthogonale : il existe \( P\in\gO(n,\eR)\) telle que
    \begin{equation}
        S=P mat_{\mB}(q)P^t;
    \end{equation}
    donc \( \det(S)=\det(PP^t)\det\big( \diag(\lambda_1,\ldots, \lambda_n) \big)=\lambda_1\ldots\lambda_n\). Ceci prouve en même temps que \( D\) ne dépend pas du choix de la base et que sa valeur est le produit des valeurs propres.

    Passons à la continuité. L'application déterminant \( \det\colon S_n(\eR^n)\to \eR\) est continue car polynôme en les composantes. D'autre par l'application \( mat_{\mB}\colon Q(\eR^n)\to S_n(\eR)\) est continue par la proposition \ref{PropFSXooRUMzdb}. L'application  \( D\) étant la composée de deux applications continues, elle est continue.
\end{proof}

\begin{proposition}[Ellipsoïde de John-Loewner\cite{KXjFWKA}]   \label{PropJYVooRMaPok}
    Soit \( K\) compact dans \( \eR^n\) et d'intérieur non vide. Il existe une unique ellipsoïde\footnote{Définition \ref{DefOEPooqfXsE}.} (pleine) de volume minimal contenant \( K\).
\end{proposition}
\index{déterminant!utilisation}
\index{extrema!volume d'un ellipsoïde}
\index{convexité!utilisation}

\begin{proof}
    Nous subdivisons la preuve en plusieurs parties.
    \begin{subproof}
        \item[À propos de volume d'un ellipsoïde]

            Soit \( \ellE\) un ellipsoïde. La proposition \ref{PropWDRooQdJiIr} et son corollaire \ref{CorKGJooOmcBzh} nous indiquent que 
            \begin{equation}
                \ellE=\{ x\in \eR^n\tq q(x)\leq 1 \}
            \end{equation}
            pour une certaine forme quadratique strictement définie positive \( q\). De plus il existe une base orthonormée \( \mB=\{ e_1,\ldots, e_n \}\) de \( \eR^n\) telle que 
            \begin{equation}    \label{EqELBooQLPQUj}
                q(x)=\sum_{i=1}^na_ix_i^2
            \end{equation}
            où \( x_i=\langle e_i, x\rangle \) et les \( a_i\) sont tous strictement positifs. Nous nommons \( \ellE_q\) l'éllipsoïde associée à la forme quadratique \( q\) et \( V_q\) son volume que nous allons maintenant calculer\footnote{Le volume ne change pas si nous écrivons l'inégalité stricte au lieu de large dans le domaine d'intégration; nous le faisons pour avoir un domaine ouvert.} :
            \begin{equation}
                V_q=\int_{\sum_ia_ix_i^2<1}dx
            \end{equation}
            Cette intégrale est écrite de façon plus simple en utilisant le \( C^1\)-difféomorphisme
            \begin{equation}
                \begin{aligned}
                    \varphi\colon \ellE_q&\to B(0,1) \\
                    x&\mapsto \Big( x_1\sqrt{a_1},\ldots, x_n\sqrt{a_n} \Big). 
                \end{aligned}
            \end{equation}
            Le fait que \( \varphi\) prenne bien ses valeurs dans \( B(0,1)\) est un simple calcul : si \( x\in\ellE_q\), alors
            \begin{equation}
                \sum_i\varphi(x)_i^2=\sum_ia_ix_i^2<1.
            \end{equation}
            Cela nous permet d'utiliser le théorème de changement de variables \ref{ThomFeRCi} :
            \begin{equation}
                V_q=\int_{\sum_ia_ix_i^2<1}dx=\frac{1}{ \sqrt{a_1\ldots a_n} }\int_{B(0,1)}dx.
            \end{equation}
            %TODO : le volume de la sphère dans \eR^n. Mettre alors une référence ici.
            La dernière intégrale est le volume de la sphère unité dans \( \eR^n\); elle n'a pas d'importance ici et nous la notons \( V_0\). La proposition \ref{PropOXWooYrDKpw} nous permet d'écrire \(V_q\) sous la forme
            \begin{equation}
                V_q=\frac{ V_0 }{ \sqrt{D(q)} }.
            \end{equation}
            
        \item[Existence de l'ellipsoïde]

            Nous voulons trouver un ellipsoïde contenant \( K\) de volume minimal, c'est à dire une forme quadratique \( q\in Q^{++}(\eR^n)\) telle que
            \begin{itemize}
                \item \( D(q)\) soit maximal
                \item \( q(x)\leq 1\) pour tout \( x\in K\).
            \end{itemize}
            Nous considérons l'ensemble des candidats semi-définis positifs.
            \begin{equation}
                A=\{ q\in Q^+\tq q(x)\leq 1\forall x\in K \}.
            \end{equation}
            Nous allons montrer que \( A\) est convexe, compact et non vide dans \( Q(\eR^n)\); il aura ainsi un maximum de la fonction continue \( D\) définie sur \( Q(\eR^n)\). Nous montrerons ensuite que le maximum est dans \( Q^{++}\). L'unicité sera prouvée à part.

            \begin{subproof}
            \item[Non vide]
                L'ensemble \( K\) est compact et donc borné par \( M>0\). La forme quadratique \( q\colon x\mapsto \| x \|^2/M^2\) est dans \( A\) parce que si \( x\in K\) alors 
                \begin{equation}
                    q(x)=\frac{ \| x \|^2 }{ M^2 }\leq 1.
                \end{equation}
            \item[Convexe]
                Soient \( q,q'\in A\) et \( \lambda\in\mathopen[ 0 , 1 \mathclose]\). Nous avons encore \( \lambda q+(1-\lambda)q'\in Q^+\) parce que 
                \begin{equation}
                    \lambda q(x)+(1-\lambda)q'(x)\geq 0
                \end{equation}
                dès que \( q(x)\geq 0\) et \( q'(x)\geq 0\).
            D'autre part si \( x\in K\) nous avons
            \begin{equation}
                \lambda q(x)+(1-\lambda)q'(x)\leq \lambda+(1-\lambda)=1.
            \end{equation}
            Donc \( \lambda q+(1-\lambda)q'\in A\).

        \item[Fermé]

            Pour rappel, la topologie de \( Q(\eR^n)\) est celle de la norme \eqref{EqZYBooZysmVh}. Nous considérons une suite \( (q_n)\) dans \( A\) convergeant vers \( q\in Q(\eR^n)\) et nous allons prouver que \( q\in A\), de sorte que la caractérisation séquentielle de la fermeture (proposition \ref{PropLFBXIjt}) conclue que \( A\) est fermé. En nommant \( e_x\) le vecteur unitaire dans la direction \( x\) nous avons
            \begin{equation}
                \big| q(x) \big|=\big| \| x \|^2q(e_x) \big|\leq \| x \|^2N(q),
            \end{equation}
            de sorte que notre histoire de suite convergente  donne pour tout \( x\) :
            \begin{equation}
                \big| q_n(x)-q(x) \big|\leq \| x \|^2N(q_n-q)\to 0.
            \end{equation}
            Vu que \( q_n(x)\geq 0\) pour tout \( n\), nous devons aussi avoir \( q(x)\geq 0\) et donc \( q\in Q^+\) (semi-définie positive). De la même manière si \( x\in K\) alors \( q_n(x)\leq 1\) pour tout \( n\) et donc \( q(x)\leq 1\). Par conséquent \( q\in A\) et \( A\) est fermé.

        \item[Borné]

            La partie \( K\) de \( \eR^n\) est borné et d'intérieur non vide, donc il existe \( a\in K\) et \( r>0\) tel que \( \overline{ B(a,r) }\subset K\). Si par ailleurs \( q\in A\) et \( x\in\overline{ B(0,r) }\) nous avons \( a+x\in K\) et donc \( q(a+x)\leq 1\). De plus \( q(-a)=q(a)\leq 1\), donc
            \begin{equation}
                \sqrt{q(x)}=\sqrt{q\big( x+a-a \big)}\leq \sqrt{q(x+a)}+\sqrt{q(-a)}\leq 2
            \end{equation}
            par l'inégalité de Minkowski \ref{PropACHooLtsMUL}. Cela prouve que si \( x\in\overline{ B(0,r) }\) alors \( q(x)\leq 4\). Si par contre \( x\in\overline{ B(0,1) }\) alors \( rx\in\overline{ B(0,r) } \) et 
            \begin{equation}
                0\leq q(x)=\frac{1}{ r^2 }q(rx)\leq \frac{ 4 }{ r^2 },
            \end{equation}
            ce qui prouve que \( N(q)\leq \frac{ 4 }{ r^2 }\) et que \( A\) est borné.


            \end{subproof}

            L'ensemble \( A\) est compact parce que fermé et borné, théorème de Borel-Lebesgue \ref{ThoXTEooxFmdI}. L'application continue \( D\colon Q(\eR^n)\to \eR\) de la proposition \ref{PropOXWooYrDKpw} admet donc un maximum sur le compact \( A\). Soit \( q_0\) ce maximum.

            Nous montrons que \( q_0\in Q^{++}(\eR^d)\). Nous savons que l'application \( f\colon x\mapsto \frac{ \| x \|^2 }{ M^2 }\) est dans \( A\) et que \( D(f)>0\). Vu que \( q_0\) est maximale pour \( D\), nous avons
            \begin{equation}
                D(q_0)\geq D(f)>0.
            \end{equation}
            Donc \( q_0\in Q^{++}\).

        \item[Unicité]

            Si il existe une autre ellipsoïde de même volume que celle associée à la forme quadratique \( q_0\), nous avons une forme quadratique \( q\in Q^{++}\) telle que \( q(x)\leq 1\) pour tout \( x\in K\). C'est à dire que nous avons \( q_0,q\in A\) tels que \( D(q_0)=D(q)\).

            Nous considérons la base canonique \( \mB_c\) de \( \eR^n\) et nous posons \( S=mat_{\mB_c}(q)\), \( S_0=mat_{\mB_c}(q_0)\). Étant donné que \( A\) est convexe, \( (q_0+q)/2\in A\) et nous allons prouver que cet élément de \( A\) contredit la maximalité de \( q_0\). En effet
            \begin{equation}
                D\left( \frac{ q+q_0 }{ 2 }\right)=\det\left( \frac{ S+S_0 }{2} \right)
            \end{equation}
            Nous allons utiliser le lemme \ref{LemXOUooQsigHs} qui dit que le logarithme est log-concave sous la forme de l'équation \eqref{EqSPKooHFZvmB} avec \( \alpha=\beta=\frac{ 1 }{2}\) :
            \begin{equation}    \label{eqBHJooYEUDPC}
                D\left( \frac{ q+q_0 }{ 2 }\right)=\det\left( \frac{ S+S_0 }{2} \right)>\sqrt{\det(S)}\sqrt{\det(S_0)}=\det(S_0)=D(q_0).
            \end{equation}
            Nous avons utilisé le fait que \( D(q_0)=D(q)\) qui signifie que \( \det(S_0)=\det(S)\). L'inéquation \eqref{eqBHJooYEUDPC} contredit la maximalité de \( D(q_0)\) et donne donc l'unicité.
    \end{subproof}
\end{proof}
