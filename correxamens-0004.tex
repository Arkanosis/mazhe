% This is part of Mes notes de mathématique
% Copyright (c) 2011-2012
%   Laurent Claessens
% See the file fdl-1.3.txt for copying conditions.

\begin{corrige}{examens-0004}

    \begin{enumerate}
        \item
            Un peu de calcul montre que le rotationnel est nul : \( \nabla \times F=0\).
        \item
            Pour la circulation, nous utilisons la formule \eqref{EqDeffvkZwhOM} :
            \begin{subequations}
                \begin{align}
                    \int_{\sigma}F&=\int_0^{2\pi} F\big( \sigma(t) \big)\cdot \sigma'(t)dt\\
                    &=\int_0^{2\pi}F\big( \cos(t),\sin(t) \big)\cdot\begin{pmatrix}
                        -\sin(t)    \\ 
                        \cos(t)    
                    \end{pmatrix}dt\\
                    &=\int_0^{2\pi}\begin{pmatrix}
                        \sin(t)    \\ 
                        -\cos(t)    
                    \end{pmatrix}\cdot
                    \begin{pmatrix}
                        -\sin(t)    \\ 
                        \cos(t)    
                    \end{pmatrix}dt\\
                    &=-2\pi.
                \end{align}
            \end{subequations}
        \item
            Si le champ dérivait d'un potentiel scalaire, son intégrale le long d'un chemin fermé (tel que \( \sigma\)) serait nulle.
        \item
            La conclusion est que le champ a une condition d'existence \( x^2+y^2\neq 0\), c'est à dire qu'au point \( (0,0)\), le champ n'existe pas. Le théorème «rotationnel nul \( \Rightarrow\) potentiel» n'est donc pas valide pour ce champ.

    \end{enumerate}

\end{corrige}
