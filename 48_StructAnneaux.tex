% This is part of Mes notes de mathématique
% Copyright (c) 2011-2016
%   Laurent Claessens
% See the file fdl-1.3.txt for copying conditions.

%+++++++++++++++++++++++++++++++++++++++++++++++++++++++++++++++++++++++++++++++++++++++++++++++++++++++++++++++++++++++++++
\section{Généralités}
%+++++++++++++++++++++++++++++++++++++++++++++++++++++++++++++++++++++++++++++++++++++++++++++++++++++++++++++++++++++++++++

Soit \( X\) un ensemble et un anneau $(A, +, \times)$. Nous considérons \( \Fun(X,A)\)\nomenclature[A]{\( \Fun(X,Y)\)}{les applications de \( X\) vers \( Y\)} l'ensemble des applications \( X\to A\). Cet ensemble devient un anneau avec les définitions
\begin{subequations}
    \begin{align}
        (f+g)(x)=f(x)+g(x)\\
        (fg)(x)=f(x)g(x).
    \end{align}
\end{subequations}
Cela est la \defe{structure canonique}{structure d'anneau canonique} d'anneau sur \( \Fun(X,A)\).

Le \defe{centralisateur}{centralisateur} de \( x\in A\) dans \( A\) est l'ensemble
\begin{equation}
    \{ y\in A\tq xy=yx \},
\end{equation}
le \defe{centre}{centre!d'un anneau} de \( A\) est
\begin{equation}
    \{ y\in A\tq xy=yx,\forall x\in A \}.
\end{equation}

Un élément \( a\in A\) est \defe{régulier à droite}{régulier à droite} si \( ba=0\) implique \( b=0\). Il est régulier à gauche si \( ab=0\) implique \( b=0\).

L'ensemble \( U(A)\)\nomenclature[A]{\( U(A)\)}{ensemble des inversibles} des éléments inversibles de \( A\) est un groupe pour la multiplication. Nous notons \( A^*=A\setminus\{ 0 \}\).

\begin{lemma}
    Si \( a\) et \( b\) commutent, nous avons la formule
    \begin{equation}        \label{Eqarpurmkbk}
        a^{r+1}-b^{r+1}=(a-b)(\sum_{k=0}^ra^{r-k}b^k).
    \end{equation}
\end{lemma}

\begin{proposition}
    Si \( a\) est un élément nilpotent de l'anneau \( A\), alors \( 1-a\) est inversible. Si \( a\) est nilpotent non nul, alors il est diviseur de zéro.
\end{proposition}

\begin{proof}
    Soit \( n\) le minimum tel que \( a^n=0\). En vertu de la formule \eqref{Eqarpurmkbk} nous avons
    \begin{equation}
        1=1-a^n=(1-a)(1+a+\ldots+a^{n-1})=(1+a+\ldots+a^{n-1})(1-a).
    \end{equation}
    La somme \( 1+a+\ldots+a^{n-1}\) est donc un inverse de \( (1-a)\).
\end{proof}

\begin{definition}
    Soit \( \eA\) un anneau et \( a,b\in \eA\). Nous disons que \( d\) est un \( \pgcd\)\index{pgcd} de \( a\) et \( b\) si tout diviseur commun de \( a\) et \( b\) divise \( d\).
\end{definition}

\begin{definition}
    Si \( A\) et \( B\) sont des anneaux, un \defe{morphisme}{morphisme!d'anneaux} est une application \( f\colon A\to B\) telle que pour tout \( x,y\in A\) nous ayons
    \begin{enumerate}
        \item
            \( f(x+y)=f(x)+f(y)\)
        \item
            \( f(xy)=f(x)f(y)\)
        \item
            \( f(1)=1\)
    \end{enumerate}
\end{definition}

Si \( f\) est un morphisme, nous avons \( f(0)=0\) et \( f(x)^{-1}=f(x^{-1})\).

%+++++++++++++++++++++++++++++++++++++++++++++++++++++++++++++++++++++++++++++++++++++++++++++++++++++++++++++++++++++++++++ 
\section{Binôme de Newton et morphisme de Frobenius}
%+++++++++++++++++++++++++++++++++++++++++++++++++++++++++++++++++++++++++++++++++++++++++++++++++++++++++++++++++++++++++++

\begin{proposition}     \label{PropBinomFExOiL}
Pour tout $x$, $y\in\eR$ et $n\in\eN$, nous avons
\begin{equation}        \label{EqNewtonB}
    (x+y)^n=\sum_{k=0}^n{n\choose k}x^{n-k}y^k
\end{equation}
où
\begin{equation}
    {n\choose k}=\frac{ n! }{ k!(n-k)! }
\end{equation}
sont les \defe{coefficients binomiaux}{coefficients binomiaux}.
\end{proposition}

La preuve qui suit provient de \href{http://fr.wikipedia.org/wiki/Formule_du_binôme_de_Newton}{wikipédia}.
\begin{proof}
    La preuve se fait par récurrence. La vérification pour $n=0$ se fait aisément pour peu que l'on se rappelle que \( x^0=1\) et que \( 0!=1\), ce qui donne entre autres \( {0\choose 0}=1\). 
    
    Supposons que la formule \eqref{EqNewtonB} soit vraie pour $n$, et prouvons la pour $n+1$. Nous avons
\begin{equation}        \label{EqBinTrav}
    \begin{aligned}[]
        (x+y)^{n+1} &=(x+y)\cdot  \sum_{k=0}^n{n\choose k}x^{n-k}y^k\\
                &= \sum_{k=0}^n{n\choose k}x^{n-k+1}y^k+\sum_{k=0}^n{n\choose k}x^{n-k}y^{k+1}\\
                &=x^{n+1}+ \sum_{k=1}^n{n\choose k}x^{n-k+1}y^k+\sum_{k=0}^{n-1}{n\choose k}x^{n-k}y^{k+1}+y^{n+1}.
    \end{aligned}
\end{equation}
La seconde grande somme peut être transformée en posant $i=k+1$ :
\begin{equation}
    \sum_{k=0}^{n-1}{n\choose k}x^{n-k}y^{k+1}  =\sum_{i=1}^n{n\choose i-1}x^{n-(i-1)}y^{i-1+1},
\end{equation}
dans lequel nous pouvons immédiatement renommer $i$ par $k$. En remplaçant dans la dernière expression de \eqref{EqBinTrav}, nous trouvons
\begin{equation}
    (x+y)^{n+1}=x^{n+1}+y^{n+1}+\sum_{k=1}^n\left[ {n\choose k}+{n\choose k-1} \right]x^{n-k+1}y^k.
\end{equation}
La thèse découle maintenant de la formule
\begin{equation}
    {n\choose k}+{n\choose k-1}={n+1\choose k}
\end{equation}
qui est vraie parce que
\begin{equation}
    \frac{ n! }{ k!(n-k)! }+\frac{ n! }{ (k-1)(n-k+1)! }=\frac{ n!(n-k+1)+n!k }{ k!(n-k+1)! }=\frac{ n!(n+1) }{  k!(n-k+1)!  },
\end{equation}
par simple mise au même dénominateur.
\end{proof}

%+++++++++++++++++++++++++++++++++++++++++++++++++++++++++++++++++++++++++++++++++++++++++++++++++++++++++++++++++++++++++++ 
\section{Décomposition en facteurs premiers}
%+++++++++++++++++++++++++++++++++++++++++++++++++++++++++++++++++++++++++++++++++++++++++++++++++++++++++++++++++++++++++++

Le théorème fondamental de l'arithmétique permet de décomposer des nombres en facteurs premiers.

\begin{theorem}[\cite{RATEooJuqgom}]        \label{ThoAJFJooAveRvY}
    Tout entier strictement positif peut être écrit comme un produit de nombres premiers d'une unique façon, à l'ordre près des facteurs.
\end{theorem}

\begin{proof}
    
    Soit \( n\) un entier positif. Nous prouvons l'existence d'une décomposition en facteurs premiers par récurrence. Le nombre \( n=1\) est le produit d'une famille finie de nombres premiers : la famille vide.

    Supposons que tout entier strictement inférieur à un certain entier \( n>1\) est produit de nombres premiers. Deux possibilités apparaissent pour $n$ : il est premier ou non. Si $n$ est premier, et donc produit d'un unique entier premier, à savoir lui-même, le résultat est vrai. Si \( n\) n'est pas premier, il se décompose sous la forme $kl$ avec $k$ et $l$ strictement inférieurs à $n$. Dans ce cas, l'hypothèse de récurrence implique que les entiers $k$ et $l$ peuvent s'écrire comme produits de nombres premiers. Leur produit aussi, ce qui fournit une décomposition de $n$ en produit de nombres premiers.  Par application du principe de récurrence, tous les entiers naturels peuvent s'écrire comme produit de nombres premiers.  

    Nous prouvons maintenant l'unicité. Prenons deux produits de nombres premiers qui sont égaux. Prenons n'importe quel nombre premier $p$ du premier produit. Il divise le premier produit, et, de là, aussi le second. La le lemme d'Euclide \ref{LemAXINooOeuMJZ}, $p$ doit alors diviser au moins un facteur dans le second produit. Mais les facteurs sont tous des nombres premiers eux-mêmes, donc $p$ doit être égal à un des facteurs du second produit. Nous pouvons donc simplifier par $p$ les deux produits. En continuant de cette manière, nous voyons que les facteurs premiers des deux produits coïncident précisément.
\end{proof}

En d'autres termes, pour tout entier \( n>1\), il existe une suite finie unique $(p_1, k_1)$,\ldots $(p_r, k_r)$ telle que :
\begin{enumerate}
    \item
les \( p_i\) sont des nombres premiers tels que, si $i < j$, alors $p_i < p_j$ ;
\item
les \( k_i\) sont des entiers naturels non nuls ;
\item
    \( n=\prod_{i=1}^rp_i^{k_i}\).
\end{enumerate}

%+++++++++++++++++++++++++++++++++++++++++++++++++++++++++++++++++++++++++++++++++++++++++++++++++++++++++++++++++++++++++++ 
\section{Le groupe des racines de l'unité}
%+++++++++++++++++++++++++++++++++++++++++++++++++++++++++++++++++++++++++++++++++++++++++++++++++++++++++++++++++++++++++++
\label{SecGJOLooWdMYVl}

\begin{definition}
    Une \defe{racine \( n\)\ieme de l'unité}{racine!de l'unité} est une racine du polynôme \( X^n-1\).
\end{definition}

Dans \( \eC\) nous avons au maximum \( n\) telles racines, et il est facile de voir qu'il y en a effectivement \( n\) distinctes données par les éléments du groupe multiplicatif
\begin{equation}        \label{EqIEAXooIpvFPe}
    \gU_n=\{  e^{2i\pi k/n}  \tq k=0,\ldots, n-1 \}
\end{equation}
\nomenclature[A]{\( \gU_n\)}{Le groupe des racines \( n\)\ieme de l'unité.}
Un des intérêts du groupe des racines est qu'il permet de factoriser \( X^n-1\), comme nous le verrons via les polynômes cyclotomiques dans le lemme \ref{LemKYGBooAwpOHD}.

\begin{lemma}       \label{LemWHQGooXyeJiw}
    L'ensemble \( U_n\) est un groupe cyclique\footnote{Définition \ref{DefHFJWooFxkzCF}.} d'ordre \( n\) généré par \( \xi= e^{2i\pi/n}\).
\end{lemma}

\begin{proof}
    Il y a les trois propriétés à vérifier pour que ce soit un groupe.
    \begin{subproof}
        \item[Neutre]
            Le nombre \( 1\) est une racine de l'unité.
    \item[Inverse]
        Si \( \omega\in U_n\) alors \( \omega^n=1\) et donc \( \omega\omega^{n-1}=1\), ce qui signifie que \( \omega^{n-1}\) est un inverse de \( \omega\). Il reste à voir que \( \omega^{n-1}\in U_n\). En effet \(  \big( \omega^{n-1} \big)^n=(\omega^n)^{n-1}=1^{n-1}=1  \).
    \item[Associativité]
        Cas particulier de l'associativité dans \( \eC\).
    \end{subproof}
    Le fait que ce soit un groupe cyclique contenant \( n\) éléments est la définition.
\end{proof}


Le lemme suivant donne les autres générateurs.
\begin{lemma}   \label{LemcFTNMa}
    Le nombre \( \xi^a\) est un générateur de \( \gU_n\) si et seulement si \( \pgcd(a,n)=1\).
\end{lemma}

\begin{proof}
    Si \( \pgcd(a,n)=1\) alors le théorème de Bézout \ref{ThoBuNjam} nous fournit des entiers \( u\) et \( v\) tels que \( ua+vn=1\). Alors nous avons
    \begin{equation}
        e^{2i\pi /n}= e^{2(ua+vn)i\pi/n}=( e^{2ai\pi/n})^u,
    \end{equation}
    ce qui signifie que \( \xi\) est dans le groupe engendré par \( \xi^a\), et par conséquent tout \( \gU_n\) est engendré.

    Pour l'implication inverse, nous utilisons Bézout dans le sens inverse. Soit \( \xi^a\) un générateur de \( \gU_n\). Alors il existe \( u\) tel que \( (\xi^a)^u=\xi\), donc \( \xi^{au-1}=1\), c'est à dire qu'il existe \( v\) tel que \( au-1=vn\). Cette dernière égalité implique que \( \pgcd(a,n)=1\).
\end{proof}

\begin{example}
Une conséquence tout à fait extraordinaire de ce lemme est que \( 7\) est générateur de \( \eZ/12\eZ\) (parce que \( \pgcd(7,12)=1\)). Or en solfège\index{solfège}, une quinte fait \( 7\) demi-tons, et une gamme en fait 12. Le cycle des quintes est donc générateur de la gamme\cite{YDXsAM}. Cela est un fait connu des pianistes\footnote{Même ceux qui ignorent le théorème de Bézout.} depuis des siècles.
\end{example}

\begin{proposition}[Intersection par deux]
    Les ensembles \( U_{\alpha}\) et \( U_{\beta}\) ont une intersection non réduite à \( \{ 1 \}\) si et seulement si \( \alpha\) et \( \beta\) ne sont pas premiers entre eux.
\end{proposition}

\begin{proof}
    Nous rappelons qu'une racine \( \alpha\)\ieme de l'unité peut s'écrire sous la forme \(  e^{2i\pi k/\alpha}\) avec \( 0\leq k<\alpha\).
    \begin{subproof}
    \item[Sens direct]
        Nous supposons que \( z\in U_{\alpha}\cap U_{\beta}\). Le fait que \( z\) soit une racine \( \alpha\)\ieme de l'unité implique qu'il existe un \( k<\alpha\) tel que \( z= e^{2i\pi k/\alpha}\). Mais si \( z\) est également une racine \( \beta\)\ieme de l'unité, alors \( z^{\beta}=1\), c'est à dire que \( k\beta/\alpha\) doit être un entier, soit \( l\) cet entier. Nous avons
        \begin{equation}
            k\beta=l\alpha.
        \end{equation}
        Le nombre \( \alpha\) divise \( k\beta\); et si nous supposons que \( \alpha\) et \( \beta \) étaient premiers entre eux, cela conduirait via le lemme de Gauss \ref{LemPRuUrsD} à dire que \( \alpha\) divise \( k\). Mais \( \alpha\) ne peut pas diviser \( k\) parce que nous avions supposé que \( k\) était strictement plus petit que \( \alpha\).
    \item[Sens réciproque]
        Nous supposons maintenant que \( \alpha\) et \( \beta\) ne sont pas premiers entre eux, et nous notons \( d\) leur \( \pgcd\). Nous nommons \( \alpha=d\alpha'\) et \( \beta=d\beta'\). Pour trouver une intersection entre \( U_{\alpha}\) et \( U_{\beta}\) nous devons trouver une valeur de \( k<\alpha\) telle que
        \begin{equation}
            ( e^{2i\pi k/\alpha})^{\beta}= e^{2i\pi k\beta/\alpha}=1,
        \end{equation}
        c'est à dire une valeur de \( k\) telle que \( k\beta/\alpha\) soit un entier. Mais \( k\beta/\alpha=k\beta'/\alpha'\) et par conséquent prendre \( k=\alpha'\) fonctionne. Surtout que par hypothèse \( d>1\) et donc \( k=\alpha'<\alpha\).
    \end{subproof}
\end{proof}

\begin{proposition}[Intersection : le cas général\cite{MonCerveau}]  \label{PropFDDHooEyYxBC}
    Soient des entiers positifs \( \alpha_1,\ldots, \alpha_p\). Nous avons
    \begin{equation}
        \bigcup_{i=1}^pU_{\alpha_i}=\{ 1 \}
    \end{equation}
    si et seulement si \( \pgcd(\alpha_1,\ldots, \alpha_p)=1\) (c'est à dire que les \( \alpha_i\) sont premiers dans leur ensemble).
\end{proposition}

\begin{proof}
    Nous le décomposons les \( \alpha_i\) en facteurs premiers\footnote{Théorème \ref{ThoAJFJooAveRvY}.} de la façon suivante : \( \alpha_i=\prod_{k\in \eN}p_k^{\alpha_i^{(k)}}\) où les \( p_k\) sont les nombres premiers. 
    
    \begin{subproof}
    \item[Caractérisation par une décomposition en facteurs premiers]
        Les éléments \( z\) différents de \( 1\) dans \( U_{\alpha_1}\) s'écrivent sous la forme
        \begin{equation}
            z= e^{2i\pi k/\alpha_1}
        \end{equation}
        avec \( 0<k<\alpha_1\).

        Pour tout \( i\neq 1\), le fait que \( z\in U_{\alpha_i}\cap U_{\alpha_1}\) se traduit par le fait que \( \big(  e^{2i\pi k/\alpha_1} \big)^{\alpha_i}=1\), c'est à dire que \( \alpha_ik/\alpha_1\) est entier, donc que \( \alpha_1\) divise \( k\alpha_i\). Par conséquent il existera un élément différent de \( 1\) dans l'intersection des \( U_{\alpha_i}\) si et seulement si il existe un entier \( k\) strictement compris entre \( 0\) et \( \alpha_1\) pour lequel \( \alpha_1\) divise tous les \( k\alpha_i\).

        Un entier \( 0<k<\alpha_1\) convient si et seulement si pour tout \( l\), la puissance de \( p_l\) dans la décomposition de \( k\) est au moins égale à
        \begin{equation}
            \alpha_1^{(l)}-\alpha_i^{(l)}
        \end{equation}
        pour tout \( l\).
    \item[Sens direct]
        L'hypothèse \( \pgcd(\alpha_1,\ldots, \alpha_p)\neq 1\) implique qu'il existe un \( l\) pour lequel tous les \( \alpha_i^{(l)}\) sont non nuls. Nous construisons le \( k\) voulu en prenant pour tout \( p_i\) la même puissance que celle dans \( \alpha_1\), sauf pour \( p_l\) pour lequel nous prenons la puissance \(  \alpha_1^{(l)}-\min_i\{   \alpha_i^{(l)} \} \). Le minimum en question est strictement positif, ce qui donne un \( k\) strictement inférieur à \( \alpha_1\).
    \item[Sens réciproque]
        Si \( \pgcd(\alpha_1,\ldots, \alpha_p)=1\) alors pour tout \( l\), il existe un \( i\) tel que \( \alpha_i^{(l)}=0\). Donc pour tout \( l\), la puissance de \( p_l\) dans la décomposition de \( k\) est au moins \( \alpha_1^{(l)}\). Cela implique que \( k\geq \alpha_1\), ce qui est impossible.
    \end{subproof}
\end{proof}

\begin{definition}\label{DefLYGTooFPOYGZ}
    Les générateurs de \( \gU_n\) sont les \defe{racines primitives}{racine!de l'unité!primitive}\footnote{parce qu'en prenant les puissances successives de l'une d'entre elles, nous retrouvons toutes les racines de l'unité, voir aussi la définition \ref{DefnPNCFO}.} de l'unité dans \( \eC\). Nous nommons \( \Delta_n\) leur ensemble :
\begin{equation}
    \Delta_n=\{  e^{2ki\pi/n}\tq 0\leq k\leq n-1,\pgcd(k,n)=1 \}.
\end{equation}
\end{definition}
Nous avons par exemple
\begin{subequations}
    \begin{align}
        \Delta_1&=\{ 1 \}\\
        \Delta_2&=\{  e^{\pi i} \}\\
        \Delta_4&=\{  e^{\pi i/2}, e^{3\pi i/2} \}.
    \end{align}
\end{subequations}
Notons que \( 1\in \Delta_d\) seulement avec \( d=1\).


%+++++++++++++++++++++++++++++++++++++++++++++++++++++++++++++++++++++++++++++++++++++++++++++++++++++++++++++++++++++++++++
\section{Fonction indicatrice d'Euler (première partie)}
%+++++++++++++++++++++++++++++++++++++++++++++++++++++++++++++++++++++++++++++++++++++++++++++++++++++++++++++++++++++++++++

Nous introduisons ici la fonction indicatrice d'Euler et ses liens basiques avec les racines de l'unité. Pour les propriétés plus avancées, voir \ref{subSecKGDFooAbETjs}.

%---------------------------------------------------------------------------------------------------------------------------
\subsection{Introduction par les racines de l'unité}
%---------------------------------------------------------------------------------------------------------------------------

\begin{definition}
La fonction \( \varphi\) donnée par
\begin{equation}    \label{EqEulerGqPsvi}
    \varphi(n)=\Card(\Delta_n)
\end{equation}
est l'\defe{indicatrice d'Euler}{indicatrice d'Euler}\index{Euler!indicatrice}.
\end{definition}
Si \( p\) est un nombre premier, alors \( \varphi(p)=p-1\).

\begin{lemma}       \label{LemKcpjee}
    Nous avons
    \begin{equation}        \label{EqpZuIyL}
        \gU_n=\bigcup_{d\divides n}\Delta_d
    \end{equation}
    et l'union est disjointe. Nous avons aussi la formule
    \begin{equation}        \label{EqTPHqgJ}
        n=\sum_{d\divides n}\varphi(d).
    \end{equation}
\end{lemma}

\begin{proof}
    À l'application \( x\mapsto  e^{2i\pi x}\) près, nous pouvons considérer
    \begin{equation}
        \Delta_d=\{ \frac{ k }{ d }\tq k=0,\ldots, d-1, \pgcd(k,d)=1 \},
    \end{equation}
    c'est à dire l'ensemble des fractions irréductibles dont le dénominateur est \( d\). L'union des \( \Delta_d\) sera donc disjointe.
    
    Toujours à l'application \( x\mapsto  e^{2i\pi x}\) près, le groupe \( \gU_n\) est donné par
    \begin{equation}
        \gU_n=\{ \frac{ k }{ n }\tq k=0,\ldots, n-1 \}.
    \end{equation}
    L'égalité \eqref{EqpZuIyL} revient maintenant à dire que toute fraction de la forme \( \frac{ k }{ n }\) s'écrit de façon irréductible avec un dénominateur qui divise \( n\).

    La relation \eqref{EqTPHqgJ} consiste à prendre le cardinal des deux côtés de \eqref{EqpZuIyL}. Nous avons \( \Card(\gU_n)=n\) et l'union étant disjointe, à droite nous avons la somme des cardinaux.


    Pour chaque diviseur \( d\) de \( n\) nous considérons l'ensemble
    \begin{equation}
        \Phi_n(d)=\{ n\in \eN\tq \pgcd(n,n)=\frac{ n }{ d } \}.
    \end{equation}
    Étant donné que tous les entiers entre \( 0\) et \( n\) ont un pgcd avec \( n\) qui est automatiquement un quotient de \( n\) nous avons
    \begin{equation}
        \{ 0,\ldots, n \}=\bigcup_{d\divides n}\Phi_n(d)
    \end{equation}
    où l'union est disjointe. Par ailleurs nous savons que si \( \pgcd(a,b)=1\), alors \( \pgcd(ka,kb)=k\). Donc si \( n\in \Delta(d)\), alors \( n\cdot \frac{ n }{ d }\) appartient à \( \Phi_n(d)\). En d'autres termes, \( a\mapsto \frac{ n }{ d }a\) est une bijection entre \( \Delta(d)\) et \( \Phi_n(d)\).

    Nous avons donc \( \Card(\Phi_n(d))=\Card(\Delta(d))=\varphi(d)\) et finalement
    \begin{equation}
        \Card\{ 1,\ldots, n \}=\sum_{d\divides n}\Card(\Phi_n(d))=\sum_{d\divides n}\varphi(d).
    \end{equation}
\end{proof}

\begin{lemma}
    Si \( p\) est un nombre premier, alors \( \varphi(p^n)=p^n-p^{n-1}\).
\end{lemma}

\begin{proof}
    Les éléments de \( \{ 1,\ldots,p^n \}\) qui ont un \( \pgcd\) différent de \( 1\) avec \( p^n\) sont des nombres qui s'écrivent sous la forme \( qp\) avec \( q\leq p^{n-1}\). Il y a évidemment \( p^{n-1}\) tels nombres.

    Par conséquent le cardinal de \( P_{p^n}\) est \( \varphi(p^{n})=p^n-p^{n-1}\).
\end{proof}

%--------------------------------------------------------------------------------------------------------------------------- 
\subsection{Générateurs}
%---------------------------------------------------------------------------------------------------------------------------

\begin{proposition}     \label{PropZnmuphiGensn}
    Soit \( n\in\eN^*\) et le groupe (additif) \( \eZ/n\eZ\). L'élément \( [x]_n\) est un générateur de \( \eZ/n\eZ\) si et seulement si \( x\in P_n\). En particulier \( \eZ/n\eZ\) est un groupe contenant \( \varphi(n)\) générateurs.
\end{proposition}

\begin{proof}
    Nous avons \( \gr\big( [1]_n \big)=\eZ/n\eZ\). L'élément \( [x]_n\) sera générateur si et seulement si il génère \( [1]_n \), c'est à dire si il existe \( p\) tel que \( p[x]_n=[1]_n\). Cette dernière égalité étant une égalité de classes dans \( \eZ/n\eZ\), elle sera vraie si et seulement si il existe \( q\) tel que
    \begin{equation}
        px+qn=1.
    \end{equation}
    Cela signifie entre autres que \( x\eZ+n\eZ=\eZ\), c'est à dire que \( \pgcd(x,n)=1\) et que \( x\in P_n\).
\end{proof}

%+++++++++++++++++++++++++++++++++++++++++++++++++++++++++++++++++++++++++++++++++++++++++++++++++++++++++++++++++++++++++++ 
\section{Idéal dans un anneau}
%+++++++++++++++++++++++++++++++++++++++++++++++++++++++++++++++++++++++++++++++++++++++++++++++++++++++++++++++++++++++++++

La définition d'un idéal dans un anneau est la définition \ref{DefooQULAooREUIU}.

\begin{definition}  \label{DefAJVTPxb}
    Un sous ensemble \( B\subset A\) d'un anneau est un \defe{sous anneau}{sous anneau} si
    \begin{enumerate}
        \item
            \( 1\in B\)
        \item
            \( B\) est un sous-groupe pour l'addition
        \item
            \( B\) est stable pour la multiplication.
    \end{enumerate}
\end{definition}

Lorsqu'un ensemble est idéal à gauche et à droite, nous disons que c'est un \defe{idéal bilatère}{idéal!bilatère}. Lorsque nous parlons d'idéal sans précisions, nous parlons d'idéal bilatère.

\begin{remark}
    Un idéal n'est pas toujours un anneau parce que l'identité pourrait manquer. Un idéal qui contient l'identité est l'anneau complet.
\end{remark}

\begin{example}
    L'ensemble \( 2\eZ\) est un idéal de \( \eZ\).
\end{example}

Soit \( A\), un anneau, \( I\) un idéal bilatère\footnote{Définition \ref{DefooQULAooREUIU}.} de \( A\). Nous considérons la relation d'équivalence \( x\sim y\) si et seulement si \( x-y\in I\). Dans ce cas, le quotient
\begin{equation}
    A/\sim=A/I
\end{equation}
est un anneau appelé \defe{anneau quotient}{anneau!quotient par un idéal}. La surjection \( A\to A/I\) est un morphisme.

\begin{proposition}
    Soient \( A\) et \( B\) des anneaux et un homomorphisme \( f\colon A\to B\). Nous considérons l'injection canonique \( j\colon f(A)\to B\) et la surjection canonique \( \phi\colon A\to A/\ker f\). Alors il existe un unique isomorphisme
    \begin{equation}
        \tilde f \colon A/\ker f\to f(A)
    \end{equation}
    tel que \( f=j\circ\tilde f\circ\phi\).

    \begin{equation}
        \xymatrix{%
        A \ar[r]^{f}\ar[d]_{\phi}        &   B\ar[d]^{j}\\
           A/\ker f \ar[r]_{\tilde f}   &   f(A)\subset B
           }
    \end{equation}
\end{proposition}

\begin{proposition}     \label{PropIJJIdsousphi}
    Soit \( I\), un idéal de \( A\) et \( \phi\colon A\to A/I\) la surjection canonique. Les idéaux de \( A/I\) sont les \( \phi(J)\) où \( J\) est un idéal de \( A\) contenant \( I\). De plus cette relation est bijective :
    \begin{equation}        \label{EqKbrizu}
        \{ \text{idéaux de \( A\) contenant \( I\)}\}\simeq\{ \text{idéaux de \( R/I\)} \}.
    \end{equation}
\end{proposition}

\begin{proof}
    Si \( I\subset J\) et si \( J \) est un idéal de \( A\), alors \( \phi(J)\) est un idéal dans \( A/I\). En effet un élément de \( \phi(J)\) est de la forme \( \phi(j)\) et un élément de \( A/I\) est de la forme \( \phi(i)\). Leur produit vaut
    \begin{equation}
        \phi(i)\phi(j)=\phi(ij)\in\phi(J).
    \end{equation}
    
    Soit maintenant \( K\), un idéal dans \( A/I\). Soit \( J=\phi^{-1}(K)\). Étant donné qu'un idéal doit contenir \( 0\) (parce qu'un idéal est un groupe pour l'addition), \( [0]\in K\) et par conséquent \( I\subset\phi^{-1}(K)\).
\end{proof}
% TODO : il faudrait dire à peu près ici qu'une des utilités de Z_2 est le groupe modulaire PSL(2,Z)=SL(2,Z)/Z_2

\begin{corollary}
    Les quotients de \( \eZ\) sont \( \eZ/n\eZ\).
\end{corollary}

\begin{proof}
    Tous les idéaux de \( \eZ\) sont de la forme \( n\eZ\). En effet en vertu de la proposition \ref{PropSsgpZestnZ}, les seuls sous-groupes de \( \eZ\) (en tant que groupe additif) sont les \( n\eZ\). Tous les idéaux sont donc de cette forme. De plus les \( n\eZ\) sont effectivement tous des idéaux : si \( a\in n\eZ\) et si \( k\in \eZ\) alors \( ak\in n\eZ\). Cela est donc un idéal.
\end{proof}

\begin{proposition}     \label{PropZpintssiprempUzn}
    Soit \( n\geq 2\) un entier et \( \phi\colon \eZ\to \eZ/n\eZ\), la surjection canonique. Nous noterons \( \tilde a=\phi(a)\). Alors l'ensemble des inversibles de \( \eZ/n\eZ\) est donné par
    \begin{equation}
        U(\eZ/n\eZ)=\phi(P_n)=\{ \tilde x\tq 0\leq x\leq n\tq\pgcd(x,n)=1 \}.
    \end{equation}
    où \( P_n\) est l'ensemble $P_n=\{ x\in\{ 0,\ldots,n-1 \}\tq\pgcd(x,n)=1 \}$. En particulier, \( \Card\big( U(\eZ/n\eZ) \big)=\varphi(n)\).
\end{proposition}

\begin{proof}
    Soit \( 0\leq x\leq n\) tel que \( \pgcd(x,n)=1\). Il existe donc \( p,q\in\eZ\) tels que \( px+qn=1\). En passant aux classes,
    \begin{equation}
        \tilde p\tilde x=\tilde 1,
    \end{equation}
    donc \( \tilde p\) est l'inverse de \( \tilde x\). Cela prouve que \( \phi(P_n)\subset U(\eZ/n\eZ)\).

    Nous prouvons maintenant l'inclusion inverse. Soit \( \tilde x\) et \( \tilde y\) inverses l'un de l'autre : $\tilde x\tilde y=\tilde 1$. Il existe donc \( q\in\eZ\) tel que \( xy-qn=1\), ce qui prouve que \( \pgcd(x,n)=1\).
\end{proof}

%+++++++++++++++++++++++++++++++++++++++++++++++++++++++++++++++++++++++++++++++++++++++++++++++++++++++++++++++++++++++++++ 
\section{Caractéristique}
%+++++++++++++++++++++++++++++++++++++++++++++++++++++++++++++++++++++++++++++++++++++++++++++++++++++++++++++++++++++++++++

L'application 
\begin{equation}
    \begin{aligned}
        \mu\colon \eZ&\to A \\
        n&\mapsto n\cdot 1_A 
    \end{aligned}
\end{equation}
est un morphisme d'anneaux. Le noyau de \( \mu\) étant un sous-groupe de \( \eZ\), il existe un et un seul \( p\in\eZ\) tel que \( \ker\mu=p\eZ\). Ce \( p\) est la \defe{caractéristique}{caractéristique!d'un anneau} de \( A\).

Par exemple la caractéristique que \( \eQ\) est zéro parce qu'aucun multiple de l'unité n'est nul.

À propos de diagonalisation en caractéristique \( 2\), voir l'exemple \ref{ExewINgYo}.

\begin{lemma}
    Si \( A\) est de caractéristique nulle, alors \( A\) est infini.
\end{lemma}

\begin{proof}
    En effet, \( \ker\mu=0\) implique que \( n1_A\neq  m1_A\) et par conséquent \( A\) est infini.
\end{proof}

\begin{lemma}       \label{LemHmDaYH}
    Si \( p\) est la caractéristique de l'anneau \( A\), alors nous avons l'isomorphisme d'anneaux
    \begin{equation}
         \eZ 1_A\simeq\eZ/p\eZ.
    \end{equation}
\end{lemma}

\begin{proof}
    L'isomorphisme est donné par l'application \( n1_A\mapsto \phi(n)\) si \( \phi\) est la projection canonique \( \eZ\to \eZ_p\).
\end{proof}

\begin{proposition}     \label{PropGExaUK}
    La caractéristique d'un anneau fini divise son cardinal.
\end{proposition}

\begin{proof}
    Si \( \eA\) est un anneau, le groupe \( \eZ\) agit sur \( \eA\) par
    \begin{equation}
        n\cdot a=a+n1_A.
    \end{equation}
    Chaque orbite de cette action est de la forme
    \begin{equation}
        \mO_a=\{ a+n1_A\tq n=0,\ldots, p-1 \}
    \end{equation}
    où \( p\) est la caractéristique de \( \eA\). Les orbites ont \( p\) éléments et forment une partition de \( \eA\), donc le cardinal de \( \eA\) est un multiple de \( p\).
\end{proof}

L'ensemble typique de caractéristique \( p\) est \( \eF_p=\eZ/p\eZ\).

\begin{example}
    Soit à factoriser \( X^p-1\) dans \( \eF_p\). Grâce au morphisme de Frobenius, nous avons immédiatement
    \begin{equation}
        X^p-1=(X-1)^p.
    \end{equation}
\end{example}

\begin{definition}
    Un élément \( a\) d'un anneau est dit \defe{nilpotent}{nilpotent} si \( a^r=0\) pour un certain \( r\). Il est dit \defe{unipotent}{unipotent} si \( a-1\) est nilpotent, c'est à dire si \( (a-1)^r=0\) pour un certain \( r\).
\end{definition}

\begin{proposition}     \label{Propqrrdem}
    Soit \( \eA\) un anneau commutatif de caractéristique première \( p\). Alors \( \sigma(x)=x^p\) est un automorphisme de l'anneau \( \eA\). Nous avons la formule
    \begin{equation}
        (a+b)^p=a^p+b^p
    \end{equation}
    pour tout \( a,b\in \eA\).
\end{proposition}

\begin{proof}
    Nous utilisons la formule du binôme de la proposition \ref{PropBinomFExOiL} et le fait que les coefficients binomiaux non extrêmes sont divisibles par \( p\) et donc nuls.
\end{proof}

\begin{proposition} \label{PropFrobHAMkTY}
    Soit \( \eA\) un anneau commutatif unitaire de caractéristique \( p\). L'application
    \begin{equation}
        \begin{aligned}
            \Frob_\eA\colon \eA&\to \eA \\
            x&\mapsto x^p 
        \end{aligned}
    \end{equation}
    est un automorphisme d'anneau unitaire.
\end{proposition}
Nous le nommons le \defe{morphisme de Frobenius}{morphisme!Frobenius}\index{Frobenius!morphisme}. Nous utiliserons aussi les itérés du morphisme de Frobenius : \( \Frob^k\colon x\mapsto x^{p^k}\).


%+++++++++++++++++++++++++++++++++++++++++++++++++++++++++++++++++++++++++++++++++++++++++++++++++++++++++++++++++++++++++++
\section{Modules}
%+++++++++++++++++++++++++++++++++++++++++++++++++++++++++++++++++++++++++++++++++++++++++++++++++++++++++++++++++++++++++++

Si \( \eA\) est un anneau et si \( (\modE,+)\) est un groupe commutatif, nous disons que \( \modE\) est un \defe{\wikipedia{fr}{Module_sur_un_anneau}{module}}{module!sur un anneau} à gauche sur \( \eA\) si nous avons une application \( \eA\times M\to M\) notée \( a\cdot x\) telle que
\begin{enumerate}
    \item
       $\alpha\cdot(x + y) = a\cdot x + a\cdot y$  (distributivité de $\cdot$ par rapport à l'addition dans $M$)
   \item $(a + b) \cdot x = a \cdot x + b \cdot x$ (distributivité de $\cdot$ par rapport à l'addition dans \( \eA\)). Remarque :  la loi $+$ du membre de gauche est celle de l'anneau $A$ et la loi $+$ du membre de droite est celle du groupe $M$
   \item $(ab) \cdot x = a \cdot (b \cdot x$)
   \item $1 \cdot x = x$ 
\end{enumerate}

Soit \( \modE\) un \( A\)-module et \( x=(x_i)_{i\in I}\) une famille d'éléments de \( \modE\), paramétrée par l'ensemble \( I\). Nous considérons l'application
\begin{equation}
    \begin{aligned}
        \mu_x\colon A^{(I)}&\to \modE \\
        (a_i)_{i\in I}&\mapsto \sum_{i\in I}a_ix_i.
    \end{aligned}
\end{equation}
Ici \( A^{(I)}\) désigne l'ensemble de toutes les applications \( I\to A\) de support fini.  

\begin{definition}      \label{DefBasePouyKj}
    À l'instar des espaces vectoriels, les modules ont une notion de partie libre, génératrice et de bases :
    \begin{enumerate}
        \item
            Si \( \mu_x\) est surjective, nous disons que \( x\) est une partie \defe{génératrice}{génératrice!partir d'un module}.
        \item
            Si \( \mu_x\) est injective, nous disons que la partie \( x\) est \defe{libre}{libre!partie d'un module}.
        \item
            Si \( \mu_x\) est bijective, nous disons que la partie \( x\) est une \defe{base}{base!d'un module}.
    \end{enumerate}
\end{definition}

\begin{definition}
    Soit \( \modE\) un module sur un anneau commutatif \( A\). Un \defe{projecteur}{projecteur!dans un module} est une application linéaire \( p\colon \modE\to \modE\) telle que \( p^2=p\).

    Une famille \( (p_i)_{i\in I}\) sur \( \modE\) est \defe{orthogonale}{orthogonal!famille de projecteurs} si \( p_i\circ p_j=0\) pour tout \( i\neq j\). La famille est \defe{complète}{complète!famille de projecteurs} si \( \sum_{i\in I}p_i=\mtu\).
\end{definition}

\begin{theorem}     \label{ThoProjModpAlsUR}
    Soient des sous modules \( \modE_1,\ldots,\modE_n\) du module \( \modE\) tels que \( \modE=\modE_1\oplus\ldots\oplus\modE_n\). Les applications \( p_i\) définies par
    \begin{equation}
        p_i(x_1+x_n)=x_i
    \end{equation}
    forment une famille orthogonale de projecteurs et \( p_1+\ldots +p_n=\id\).

    Inversement, si \( (p_1,\ldots, p_n)\) est une famille orthogonale de projecteurs dans un module \( \modE\) tel que \( \sum_{i=1}^np_i=\id\), alors
    \begin{equation}
        \modE=\bigoplus_{i=1}^np_i(\modE).
    \end{equation}
\end{theorem}

Un sous-ensemble \( \modF\subset\modE\) est un \defe{sous-module}{sous-module} si \( (\modF,+)\) est un sous-groupe de \( (\modE,+)\) et si \( a\cdot x\in \modF\) pour tout \( x\in \modF\) et pour tout \( a\in \eA\).

\begin{example}
    Un anneau \( \eA\) est lui-même un \( \eA\)-module et ses sous-modules sont les idéaux.
\end{example}

Un module est \defe{simple}{simple!module}\index{module!simple} ou \defe{irréductible}{irréductible!module}\index{module!irréductible} si il n'a pas d'autre sous-modules que \( \{ 0 \}\) et lui-même. Un module est \defe{indécomposable}{indécomposable!module}\index{module!indécomposable} si il ne peut pas être écrit comme somme directe de sous-modules.

Un module simple est a fortiori indécomposable. L'inverse n'est pas vrai comme le montre l'exemple suivant.

\begin{example}
    Soit \( \modE=\eC[X]/(X^2)\) vu comme \( \eC[X]\)-module. C'est le \( \eC[X]\)-module des polynômes de la forme \( aX+b\) avec \( a,b\in \eC\). L'ensemble des polynômes de la forme \( aX\) est un sous-module. Le module \( \modE\) n'est donc pas simple. Il est cependant indécomposable parce que \( \{ aX \}\) est le seul sous-module non trivial. En effet si \( \modF\) est un sous-module de \( \modE\) contenant \( aX+b\) avec \( b\neq 0\), alors \( \modF\) contient \( X(aX+b)=bX\) et donc contient tout \( \modE\).
\end{example}

%+++++++++++++++++++++++++++++++++++++++++++++++++++++++++++++++++++++++++++++++++++++++++++++++++++++++++++++++++++++++++++
\section{Anneau intègre}
%+++++++++++++++++++++++++++++++++++++++++++++++++++++++++++++++++++++++++++++++++++++++++++++++++++++++++++++++++++++++++++

Un élément \( a\neq 0\) est un \defe{diviseur de zéro à gauche}{diviseur!de zéro} si il existe \( x\neq 0\) tel que $xa=0$. L'élément \( a\) est un diviseur de zéro \defe{à droite}{diviseur!de zéro à droite} si il existe \( b\) tel que \( ab=0\). Un anneau est \defe{intègre}{intègre!anneau}\index{anneau!intègre} si il est non nul et ne possède pas de diviseurs de zéro.

Autrement dit, un anneau intègre est un anneau qui possède la propriété du produit nul : si \( ab=0\), alors soit \( a\) soit \( b\) est nul. Nous verrons en temps et en heure (proposition \ref{LemAnnCorpsnonInterdivzer}) qu'un corps est un anneau intègre.
\index{règle!du produit nul}

\begin{example}
    L'ensemble \( \eZ\) avec les opérations usuelles est un anneau intègre.
\end{example}

\begin{example}
    L'anneau \( \eZ/6\eZ\) n'est pas intègre parce que \( 3\cdot 2=0\) alors que ni \( 3\) ni \( 2\) ne sont nuls.
\end{example}

\begin{example}
    Nous verrons au théorème \ref{ThoBUEDrJ} que si \( \eA\) est intègre, alors l'anneau des polynômes sur \( \eA\) est intègre.
\end{example}

\begin{corollary}   \label{CorZnInternprem}
    L'anneau \( \eZ/n\eZ\) est intègre si et seulement si \( n\) est premier.
\end{corollary}

\begin{proof}
    Si \( n\) est premier, tous les éléments de \( \eZ/n\eZ\) sont inversibles parce que tous les éléments rentrent dans \( \phi(P_n)\). Donc \( \eZ/n\eZ\) est intègre.

    Si \( n\) n'est pas premier, il existe \( p,q\in\eN^*\) tels que \( pq=n\). Dans ce cas au niveau des classes nous avons \( \tilde p\tilde q=0\) avec \( \tilde p\neq 0\neq\tilde q\), ce qui montre que \( \eZ/n\eZ\) a des diviseurs de zéro et n'est pas intègre.
\end{proof}


\begin{lemma}       \label{LemCaractIntergernbrcartpre}
    La caractéristique d'un anneau intègre est zéro ou un nombre premier.
\end{lemma}

\begin{proof}
    Si \( A\) est intègre, alors \( \eZ 1_A\) est intègre (a fortiori), et \( \eZ_p\) est intègre parce qu'il est isomorphe à \( \eZ 1_A\). Mais nous savons que \( \eZ_p\) est intègre si et seulement si \( p\) est premier (proposition \ref{CorZnInternprem}).
\end{proof}

\begin{example}
    Il existe des corps dont la caractéristique n'est pas égale au cardinal (contrairement à ce que laisserait penser l'exemple des \( \eZ/p\eZ\)). En effet les matrices \( n\times n\) inversibles sur \( \eF_{3}\) forment un corps qui n'est pas de cardinal trois alors que la caractéristique est \( 3\) :
    \begin{equation}
        \begin{pmatrix}
            1    &       \\ 
                &   1    
            \end{pmatrix}+\begin{pmatrix}
                1    &       \\ 
                    &   1    
                \end{pmatrix}+\begin{pmatrix}
                    1    &       \\ 
                        &   1    
                \end{pmatrix}=0.
    \end{equation}
\end{example}

\begin{example}
    Si \( \eK\) est un corps de caractéristique \( 2\), alors l'égalité \( x=-x\) n'implique pas \( x=0\), vu que \( 2x=0\) est vérifiée pour tout \( x\). Cela se répercute sur un certain nombre de résultats. En caractéristique deux, une forme antisymétrique n'est pas toujours alternée. Voir le lemme \ref{LemHiHNey}.
\end{example}


\begin{lemma}\label{LemRmVTRq}
    Si \( \eA\) est un anneau intègre et si \( a,b\in \eA\) sont tels que \( a\divides b\) et \( b\divides a\), alors il existe un inversible \( u\in \eA\) tel que \( a=ub\).
\end{lemma}

\begin{proof}
    Les hypothèses à propos de la divisibilité nous indiquent que \( a=xb\) et \( b=ya\) pour certains \( x,y\in \eA\). Du coup,
    \begin{equation}
        b(1-yx)=0.
    \end{equation}
    Étant donné que \( \eA\) est intègre, cela montre que \( b=0\) ou \( 1-yx=0\). Si \( b=0\) nous avons immédiatement \( a=0\) et le lemme est prouvé. Si au contraire \( yx=1\), c'est que \( y\) et \( x\) sont inversibles et inverses l'un de l'autre.
\end{proof}

\begin{example}     \label{ExybCZyl}
    Si \( \eA\) est un anneau intègre, l'anneau \( \eA[X]\) des polynômes sur \( \eA\) est également intègre. En effet si \( P\) et \( Q\) sont deux polynômes non nuls, alors le coefficient du terme de plus haut degré de \( PQ\) est donné par le produit des coefficients de plus haut degré de \( P\) et \( Q\) qui est non nul parce que \( \eA\) est intègre.
\end{example}

%---------------------------------------------------------------------------------------------------------------------------
\subsection{PGCD et PPCM}
%---------------------------------------------------------------------------------------------------------------------------
Source : \cite{XPXxPl}.

Le théorème de Bézout aura lieu dans les anneaux principaux, corollaire \ref{CorimHyXy}.

Dans un anneau intègre, la relation de divisibilité s'exprime en termes d'idéaux par
\begin{equation}
    a\divides b\Leftrightarrow (b)\subset (a).
\end{equation}
Donc la divisibilité devient en réalité une relation d'ordre dont nous pouvons chercher un maximum et un minimum. Si \( S\) est une partie de \( \eA\), nous notons \( a\divides S\) pour exprimer que \( a\divides x\) pour tout \( x\in S\).

\begin{definition}\label{DefrYwbct}
    Soit \( \eA\), un anneau intègre et \( S\subset \eA\). Nous disons que \( \delta\in \eA\) est un \defe{PGCD}{PGCD!dans un anneau intègre} de \( S\) si
    \begin{enumerate}
        \item
            \( \delta\divides S\)
        \item
            si \( d\divides S\) alors\footnote{Il me semble qu'à ce niveau il y a une faute de frappe dans \cite{XPXxPl}.} \( d\divides \delta\).
    \end{enumerate}
    Nous disons que \( \mu\in \eA\) est un \defe{PPCM}{PPCM!dans un anneau intègre} de \( S\) si
    \begin{enumerate}
        \item
            \( S\divides \mu\),
        \item
            si \( S\divides m\), alors \( \mu\divides m\).
    \end{enumerate}
\end{definition}
Notons qu'il n'y a en général pas unicité du PGCD ou du PPCM d'un ensemble.

\begin{lemma}
    Soit \( \eA\) un anneau intègre et \( S\subset \eA\). Si \( \delta\) est un PGCD de \( S\), alors l'ensemble des PGCD de \( S\) est la classe d'association de \( \delta\).

    De la même façon si \( \mu\) est un PPCM de \( S\), alors l'ensemble des PPCM de \( S\) est la classe d'association de \( \mu\).
\end{lemma}

\begin{proof}
    Soit \( \delta\) un PGCD de \( S\) et \( u\) un inversible dans \( \eA\). Si \( x\in S\) nous avons \( \delta\divides x\) et donc \( x=a\delta\). Par conséquent \( x=au^{-1}u\delta\) et donc \( u\delta\) divise \( x\). De la même manière, si \( d\) divise \( x\) pour tout \( x\in S\), alors \( d\) divise \( \delta\) et donc \( \delta=ad\) et \( u\delta=aud\), ce qui signifie que \( d\) divise \( u\delta\).

    Dans l'autre sens nous devons prouver que si \( \delta'\) est une autre PGCD de \( S\), alors il existe un inversible \( u\in \eA\) tel que \( \delta'=u\delta\). Vu que \( \delta'\) divise \( x\) pour tout \( x\in S\), nous avons \( \delta'\divides \delta\), et symétriquement nous trouvons \( \delta\divides\delta'\). Par conséquent (lemme \ref{LemRmVTRq}), il existe un inversible \( u\) tel que \( \delta=u\delta'\).

    Le même type de raisonnement tient pour le PPCM.
\end{proof}

Si \( \delta\) est un PGCD de \( S\), nous dirons \emph{par abus de langage} que \( \delta\) est \emph{le} PGCD de \( S\), gardant en tête qu'en réalité toute sa classe d'association est PGCD. Nous noterons aussi, toujours par abus que \( \delta=\pgcd(S)\).

\begin{remark}
    La classe d'association d'un élément n'est pas toujours très grande. Les inversibles dans \( \eZ\) étant seulement \( \pm 1\), nous pouvons obtenir l'unicité du PGCD et du PPCM en imposant qu'ils soient positifs.

    Pour les polynômes, nous obtenons l'unicité en demandant que le PGCD soit unitaire.

    Dans les cas pratiques, il y a donc en réalité peu d'ambiguïté à parler du PGCD ou du PPCM d'un ensemble.
\end{remark}

%--------------------------------------------------------------------------------------------------------------------------- 
\subsection{Contenu d'un polynôme}
%---------------------------------------------------------------------------------------------------------------------------

\begin{definition}
    Le \defe{contenu}{contenu}\index{polynôme!contenu} du polynôme \( P=\sum_ia_iX^i\in\eK[X]\) 
    \begin{equation}
        c(P)=\pgcd(a_i).
    \end{equation}
    Une polynôme est dit \defe{primitif}{polynôme!primitif (au sens du contenu)} si \( c(P)=1\).
\end{definition}

\begin{lemma}[de Gauss\cite{KXjFWKA,OGaUHGn}]   \label{LemHULrVaF}
    Soient \( P,Q\in \eZ[X]\). Alors \( c(PQ)=c(P)c(Q)\).
\end{lemma}
\index{lemme!de Gauss!contenu de polynôme}

\begin{proof}
    
    Afin de fixer les notations, nous posons \( P=\sum_ia_iX^i\) et \( Q=\sum_jb_jY^j\).
    
    \begin{subproof}
        \item[Pour les polynômes primitifs]
    
            Nous commençons par supposer que \( c(P)=c(Q)=1\). Dans ce cas si \( c(PQ)\neq 1\), nous considérons un nombre premier \( p\) divisant \( c(PQ)\). Vu que le contenu de \( P\) et de \( Q\) sont \( 1\), le nombre \( p\) ne peut pas diviser tous leurs coefficients. Nous définissons \( i_0\) de façon que \( a_{i_0}\) soit le premier à ne pas être divisible par \( p\) et \( j_0\) de telle façon que \( b_{j_0}\) soit le premier à ne pas être divisible par \( p\). Autrement dit :
            \begin{equation}
                p\divides a_0,p\divides a_1,\ldots,p\divides a_{i_0-1},p\notdivides a_{i_0}
            \end{equation}
            et similaire pour \( j_O\). Donc \( p\) ne divise ni \( a_{i_0}\) ni \( b_{j_0}\). Nous nous demandons alors avec malice quel est le coefficient de \( X^{i_0+j_0}\) dans \( PQ\). La réponse est :
            \begin{equation}
                a_{i_0}b_{j_0}+\sum_{\substack{i+j=i_0+j_0\\i<i_0\text{ ou }j<j_0}}a_ib_j.
            \end{equation}
            Par définition \( p\) divise soit \( a_i\) soit \( b_j\) pour chacun des termes de la grande somme. Vu que \( p\) ne divise pas \( a_{i_0}b_{j_0}\), il ne divise pas le coefficient de \( X^{i_0+j_0}\) dans \( PQ\), alors que nous étions partis en disant que \( p\) divisait tous les coefficients de \( PQ\).

            Nous concluons donc que \( c(PQ)=1\).

        \item[Cas général]

            Si \( P\) et \( Q\) sont maintenant des polynômes sans conditions particulières dans \( \eZ[X]\), nous considérons \( P_1=\frac{ P }{ c(P) }\) et \( Q_1=\frac{ Q }{ c(Q) }\); ces deux polynômes sont primitifs et nous avons alors, en utilisant la première partie :
            \begin{equation}
                c(P_1Q_1)=1.
            \end{equation}
            Étant donné que
            \begin{equation}
                P_1Q_1=\frac{1}{ c(P)c(Q) }PQ,
            \end{equation}
            nous avons
            \begin{equation}
                c(PQ)=c(P)c(Q)c(P_1Q_1)=c(P)c(Q).
            \end{equation}
    \end{subproof}
\end{proof}

%+++++++++++++++++++++++++++++++++++++++++++++++++++++++++++++++++++++++++++++++++++++++++++++++++++++++++++++++++++++++++++
\section{Anneau factoriel}
%+++++++++++++++++++++++++++++++++++++++++++++++++++++++++++++++++++++++++++++++++++++++++++++++++++++++++++++++++++++++++++

\begin{definition}  \label{DefrXUixs}
    On dit que les éléments \( a\) et \( b\) d'un anneau sont \defe{associés}{associé}\index{éléments!associés dans un anneau} si il existe un élément \( u\) inversible dans \( \eA\) tel que \( a=ub\).
\end{definition}

\begin{example}
    Dans \( \eZ[i]\), les inversibles sont \( \pm 1\) et \( \pm i\). Donc les éléments associés à \( z\) sont \( z\), \( -z\), \( iz\) et \( -iz\).

    Notons au passage que la notion de divisibilité dans \( \eZ[i]\) n'est pas immédiatement intuitive. En effet bien que \( 7\) ne soit pas divisible par \( 2\), le nombre \( 7+6i\) est divisible par \( 2+i\) dans \( \eZ[i]\). En effet :
    \begin{equation}
        (2+i)(4+i)=7+6i.
    \end{equation}
\end{example}

\begin{definition}  \label{DeirredBDhQfA}
    Soit \( \eA\) un anneau commutatif intègre. Un élément \( a\in\eA\) est \defe{irréductible}{irréductible!dans un anneau} si \( a\) n'est pas inversible, mais si \( a=xy\), alors soit \( x\) soit \( y\) est inversible. Nous notons \( U(\eA)\) l'ensemble des éléments inversibles de \( \eA\).
\end{definition}

\begin{remark}
    Un corps n'a pas d'éléments irréductibles parce qu'à part zéro tous les éléments sont inversibles alors que \( 0\) n'est certainement pas irréductible vu que \( 0=0\cdot 0\).
\end{remark}

\begin{example}
    Les éléments irréductibles de l'anneau \( \eZ\) sont les nombres premiers. En effet les seuls inversibles de \( \eZ\) sont \( \pm 1\). Si \( p\) est premier et \( p=ab\) avec \( a,b\in \eZ\), alors nous avons soit \( a=\pm 1\) soit \( b=\pm 1\).
\end{example}

\begin{definition}
    Un anneau commutatif unitaire \( \eA\) est \defe{factoriel}{factoriel!anneau}\index{anneau!factoriel} si il vérifie les propriétés suivantes.
    \begin{enumerate}
        \item
            L'anneau \( \eA\) est intègre (pas de diviseurs de zéro).
        \item
            Si \( a\in \eA\) est non nul et non inversible alors il admet une décomposition \( a=p_1\ldots p_k\) où les \( p_i\) sont irréductibles.
        \item
            Si \( a=q_1\ldots q_m\) est une autre décomposition de \( a\) en irréductibles, alors \( m=k\) et il existe une permutation \( \sigma\in S_k\) telle que \( p_i\) et \( q_{\sigma(i)}\) soient associés.
    \end{enumerate}
\end{definition}

Un anneau factoriel permet de définir le \( \pgcd\) et le \( \ppcm\) de nombres. Soit une famille \( \{ a_n \}\) d'éléments de \( \eA\) qui se décomposent en irréductibles comme
\begin{equation}
    a_i=\prod_{k}p_k^{\alpha_{k,i}}.
\end{equation}
Nous définissons
\begin{equation}
    \pgcd\{ a_n \}=\prod_kp_k^{min_i\{ \alpha_{k,i} \}}.
\end{equation}
\begin{proposition}
    L'élément \( \pgcd\{ a_n \}\) est l'unique diviseur commun des \( a_i\) à être un multiple des autres.
\end{proposition}
% TODO : une preuve de ça.

Nous définissons aussi
\begin{equation}
    \ppcm\{ a_i \}=\prod_kp_k^{\max_i\{ \alpha_{k,i} \}}.
\end{equation}
Un anneau factoriel a une relation de préordre partiel\index{ordre!sur un anneau factoriel} donnée par \( a<b\) si \( a\) divise \( b\). En termes d'idéaux, cela donne l'ordre inverse de celui de l'inclusion : \( a<b\) si et seulement si \( (b)\subset (a)\).

\begin{example}
    L'anneau \( \eZ[i\sqrt{3}]\) n'est pas factoriel parce que
    \begin{equation}
        4=2\cdot 2=(1+i\sqrt{3})(1-i\sqrt{3})
    \end{equation}
    donnent deux décompositions distinctes de \( 4\) en irréductibles.
\end{example}
Nous allons voir dans l'exemple \ref{ExeDufyZI} que \( \eZ[i\sqrt{2}]\) est factoriel parce qu'il sera euclidien.

%+++++++++++++++++++++++++++++++++++++++++++++++++++++++++++++++++++++++++++++++++++++++++++++++++++++++++++++++++++++++++++
\section{Anneau principal}
%+++++++++++++++++++++++++++++++++++++++++++++++++++++++++++++++++++++++++++++++++++++++++++++++++++++++++++++++++++++++++++

Nous parlons de l'idéal des polynôme annulateurs dans le théorème \ref{ThoCCHkoU}.

\begin{definition}      \label{DefIdPrinpuMrbOq}
    Un idéal \( I\) dans \(\eA\) est \defe{principal à gauche}{idéal!principal!à gauche} si il existe \( a\in I\) tel que \( I=\eA a\). Il est \defe{principal à droite}{idéal!principal!à droite} si il existe \( a\in I\) tel que \( I=a\eA\). Nous disons qu'il est \defe{principal}{principal!idéal} si il est principal à gauche et à droite.

    Un anneau commutatif intègre est \defe{principal}{principal!anneau} si tous ses idéaux sont principaux.
\end{definition}

Un idéal \( I\) dans l'anneau \( \eA\) est \defe{maximal}{maximal!idéal}\index{idéal!maximal} si les seuls idéaux de \( \eA\) contenants \( I\) sont \( I\) et \( \eA\).

Souvent pour prouver qu'un anneau est principal, nous prouvons qu'il est euclidien (définition \ref{DefAXitWRL}) et nous utilisons la proposition \ref{Propkllxnv} qui dit qu'un anneau euclidien est principal.


\begin{definition}
    Nous disons qu'un idéal \( I\) dans \( \eA\) est \defe{premier}{premier!idéal} si \( \eA\) est un anneau commutatif intègre et si \( A/I\) est intègre.
\end{definition}

\begin{definition}[Idéal engendré par un élément]  \label{DefSKTooOTauAR}
    Si \( p\) est un élément d'un anneau \( \eA\) alors nous notons \( (p)\)\nomenclature[A]{\( (p)\)}{idéal engendré par \( p\)}\index{engendré!idéal dans un anneau} l'idéal dans \( \eA\) \defe{engendré}{engendré} par \( p\), c'est à dire \( p\eA\).
\end{definition}

\begin{proposition} \label{PropomqcGe}
    Soit \( \eA\) un anneau principal qui n'est pas un corps. Pour un idéal \( I\subset \eA\), les conditions suivantes sont équivalentes :
    \begin{enumerate}
        \item
            \( I\) est un idéal maximum;
        \item
            \( I\) est un idéal premier non nul;
        \item
            il existe \( p\) irréductible dans \( \eA\) tel que \( I=(p)\).
    \end{enumerate}
\end{proposition}

\begin{proposition}
    Un idéal \( I\) dans \( \eA\) est premier si et seulement si \( I\) est strictement inclus dans \( \eA\) et si pour tout \( a,b\in\eA\) tels que \( ab\in I\) nous avons \( a\in I\) ou \( b\in I\).
\end{proposition}

\begin{proposition}     \label{PropoTMMXCx}
    Si \( \eA\) est un anneau principal et si \( p\) est irréductible, alors \( \eA/p\) est un corps.
\end{proposition}

\begin{example}
    L'anneau \( \eZ\) est principal parce que ses seuls idéaux sont les \( n\eZ\) qui sont principaux : \( n\eZ\) est engendré par \( n\).
\end{example}

\begin{example}
    Les anneaux \( \eZ/n\eZ\) sont principaux. En effet les idéaux de \( \eZ/n\eZ\) seraient par la proposition \ref{PropIJJIdsousphi} des quotients d'idéaux de \( \eZ\) par \( (n)\). Donc les idéaux de \( \eZ/n\eZ\) sont les anneaux \( (\eZ/m\eZ\) avec \( m\) divisant \( n\).

    Par exemple si \( I=4\eZ\), on peut considérer l'idéal \( J=2\eZ\) qui contient \( 4\eZ\). Donc \( \eZ/2\eZ\) est un idéal de \( \eZ/4\eZ\).
\end{example}

\begin{theorem}\index{théorème!chinois!anneau principal}        \label{ThofPXwiM}
    Si \( \eA\) est un anneau principal et si \( p\) et \( q\) sont premiers entre eux dans \( \eA\), alors on a l'isomorphisme d'anneaux
    \begin{equation}
        \eA/pq\eA\simeq \eA/p\eA\times \eA/q\eA.
    \end{equation}
\end{theorem}
% TODO : trouver un preuve. Je parie que recopier la même que celle dans Z fonctionne très bien.

%---------------------------------------------------------------------------------------------------------------------------
\subsection{Bézout}
%---------------------------------------------------------------------------------------------------------------------------

\begin{theorem}[\cite{XPXxPl}]
    Toute partie \( S\) d'un anneau principal admet un PGCD et un PPCM. De plus
    \begin{equation}
        \begin{aligned}[]
            \delta=\pgcd(S)\Leftrightarrow (\delta)=\sum_{s\in S}(s)
            \mu=\ppcm(S)\Leftrightarrow (\mu)=\bigcap_{s\in S}(s)
        \end{aligned}
    \end{equation}
\end{theorem}

\begin{proof}
    Vu que l'anneau \( \eA\) est principal, tous ses idéaux sont principaux et donc engendrés par un seul élément. En particulier il existe \( \delta,\mu\in \eA\) tels que
    \begin{subequations}
        \begin{align}
            (\delta)&=\sum_{s\in S}(s)\\
            (\mu)&=\bigcap_{s\in S}(s)
        \end{align}
    \end{subequations}
    \begin{subproof}
    \item[PGCD]
        Montrons ce que \( \delta\) est un PGCD de \( S\). Pour tout \( x\in S\), nous avons \( (x)\subset (\delta)\), et donc \( \delta\divides x\). Par ailleurs si \( d\divides x\) pour tout \( x\in S\), nous avons \( (x)\subset (d)\) et donc 
        \begin{equation}
            \sum_{x\in S}(x)\subset (d),
        \end{equation}
        puis \( (\delta)\subset (d)\) et finalement \( d\divides \delta\).
        \item[PPCM]
            Si \( x\in S\) nous avons \( (\mu)\subset (x)\) et donc \( x\divides \mu\). D'autre part si \( x\divides m\) pour tout \( x\in S\), alors \( (m)\subset (x)\) et donc \( (m)\subset(\mu)\), finalement \( \mu\divides m\).
    \end{subproof}
\end{proof}

Nous disons que deux éléments d'un anneau principal sont \defe{premiers entre eux}{premier!deux éléments d'un anneau principal} si leur PGCD est \( 1\).

\begin{corollary}[Théorème de Bézout\cite{XPXxPl}]\index{Bézout!anneau principal}\label{CorimHyXy}
    Soit \( \eA\) un anneau principal. Deux éléments \( a,b\in \eA\) sont premiers entre eux si et seulement si il existe un couple \( u,v\in \eA\) tel que
    \begin{equation}
        ua+vb=1.
    \end{equation}
    À la place de \( 1\) on aurait pu écrire n'importe quel inversible.
\end{corollary}
\index{anneau!principal}

\begin{proof}
    Pour cette preuve, nous allons écrire \( \pgcd(a,b)\) l'ensemble de PGCD de \( a\) et \( b\), c'est à dire la classe d'association d'un PGCD.

    Si \( a\) et \( b\) sont premiers entre eux, alors
    \begin{equation}
        1\in\pgcd(a,b)=\sum_{x=a,b}(x)=(a)+(b).
    \end{equation}
    
    À l'inverse, si nous avons \( ua+vb=1\), alors \( 1\in (a)+(b)\), mais vu que \( (a)+(b)\) est un idéal principal, \( (1)=(a)+(b)\) et donc \( 1\in \pgcd(a,b)\).
\end{proof}

Le lemme de Gauss est une application immédiate de Bézout. Il y aura aussi un lemme de Gauss à propos de polynômes (lemme \ref{LemEfdkZw}), et une généralisation directe au théorème de Gauss, théorème \ref{ThoLLgIsig}.
\begin{lemma}[\href{http://ljk.imag.fr/membres/Bernard.Ycart/mel/ar/node6.html}{lemme de Gauss}]    \label{LemSdnZNX}
    Soit \( \eA\) un anneau principal et \( a,b,c\in \eA\) tels que \( a\) divise \( bc\). Si \( a\) est premier avec \( c\), alors \( a\) divise \( b\).
\end{lemma}
\index{lemme!Gauss!dans un anneau principal}

\begin{proof}
    Vu que \( a\) est premier avec \( c\), nous avons \( \pgcd(a,c)=1\) et Bézout (\ref{ThoBuNjam}) nous donne donc \( s,t\in \eA\) tels que \( sa+tc=1\). En multipliant par \( b\),
    \begin{equation}
        sab+tbc=b.
    \end{equation}
    Mais les deux termes du membre de gauche sont multiples de \( a\) parce que \( a\) divise \( bc\). Par conséquent \( b\) est somme de deux multiples de \( a\) et donc est multiple de \( a\).
\end{proof}
Un cas usuel d'utilisation est le cas de \( \eA=\eN^*\).

%---------------------------------------------------------------------------------------------------------------------------
\subsection{Anneau noetherien}
%---------------------------------------------------------------------------------------------------------------------------

Un anneau est dit \defe{noetherien}{anneau!noetherien} si toute suite croissante d'idéaux est stationnaire (à partir d'un certain rang). Montrer que tout anneau principal est noetherien est le premier pas pour montrer que tout anneau principal est factoriel.

\begin{lemma}
    Tout anneau principal est noetherien.
\end{lemma}

\begin{proof}
    Soit \( (J_n)\) une suite croissante d'idéaux et \( J\) la réunion. L'ensemble \( J\) est encore un idéal parce que les \( J_i\) sont emboités. Étant donné que l'idéal est principal nous pouvons prendre \( a\in J\) tel que \( J=(a)\). Il existe \( N\) tel que \( a\in J_N\). Alors pour tout \( n\geq N\) nous avons
    \begin{equation}
        J\subset J_N\subset J_n\subset J.
    \end{equation}
    La première inclusion est le fait que \( J=(a)\) et \( a\in J_N\). La seconde est la croissance des idéaux et la troisième est le fait que \( J\) est une union. Par conséquent pour tout \( n\geq N\) nous avons \( J_N=J_n=J\). La suite est par conséquent stationnaire.
\end{proof}

\begin{theorem}[\cite{FSwlnf}]
    Tout anneau principal est factoriel.
\end{theorem}

%+++++++++++++++++++++++++++++++++++++++++++++++++++++++++++++++++++++++++++++++++++++++++++++++++++++++++++++++++++++++++++
\section{Anneau euclidien}
%+++++++++++++++++++++++++++++++++++++++++++++++++++++++++++++++++++++++++++++++++++++++++++++++++++++++++++++++++++++++++++

\begin{definition}[\wikipedia{fr}{Anneau_euclidien}{Wikipédia}] \label{DefAXitWRL}
    Soit \( \eA\) un anneau intègre. Un \defe{stathme euclidien}{stathme euclidien} sur \( \eA\) est une application \( \alpha\colon \eA\setminus\{ 0 \}\to \eN\) tel que
    \begin{enumerate}
        \item
            \( \forall a,b\in \eA\setminus\{ 0 \}\), il existe \( q,r\in \eA\) tel que
            \begin{equation}
                a=bq+r
            \end{equation}
            et \( \alpha(r)<\alpha(b)\).
        \item
            Pour tout \( a,b\in \eA\setminus\{ 0 \}\), \( \alpha(b)\leq \alpha(ab)\).
    \end{enumerate}
    Un anneau est \defe{euclidien}{euclidien!anneau} si il accepte un stathme euclidien.
\end{definition}
Le stathme est la fonction qui donne le «degré» à utiliser dans la division euclidienne. La contrainte est que le degré du reste soit plus petit que le degré du dividende.

\begin{example} \label{ExwqlCwvV}
    Le stathme de \( \eN\) pour la division euclidienne usuelle est \( \alpha(n)=n\). Si \( a,b\in \eN\) nous écrivons
    \begin{equation}
        a=bq+r
    \end{equation}
    où \( q\) est l'entier le plus proche \emph{inférieur} à \( a/b\) (on veut que le reste soit positif) et \( r=a-bq\). Nous avons donc
    \begin{equation}
        r-b=a-b(q+1)<a-b\frac{ a }{ b }=0,
    \end{equation}
    ce qui montre que \( r<b\).
\end{example}

\begin{proposition}[\wikipedia{fr}{Anneau_euclidien}{Wikipédia}]\label{Propkllxnv}
    Un anneau euclidien est principal.
\end{proposition}

\begin{proof}
    Soit \( \eA\) un anneau principal et \( \alpha\) un stathme sur \( \eA\). Nous considérons un idéal \( I\) non nul de \( \eA\). Nous devons montrer que \( I\) est généré par un élément. En l'occurrence nous allons montrer que l'élément \( a\in I\setminus\{ 0 \}\) qui minimise \( \alpha(a)\) va générer. Soit \( x\in I\). Par construction, il existe \( q,r\in \eA\) tels que \( a=aq+r\) avec \( r=0\) ou \( \alpha(r)<\alpha(a)\). Étant donné que \( x,a\in I\), \( r\in I\). Si \( r\neq 0\), alors \( r\) contredirait la minimalité de \( \alpha(a)\). Donc \( r=0\) et \( x=aq\), ce qui signifie que \( I\) est principal.
\end{proof}

\begin{example} \label{ExeDufyZI}
    Prouvons que \( \eZ[i\sqrt{2}]\) est une anneau euclidien. Pour cela nous démontrons que
    \begin{equation}    \label{EqOZUIooZGmHWl}
        \begin{aligned}
            N\colon \eZ[i\sqrt{2}]&\to \eN \\
            a+bi\sqrt{2}&\mapsto a^2+2b^2 
        \end{aligned}
    \end{equation}
    est un stathme euclidien.    

    Soient \( z=a+bi\sqrt{2}\), \( t=a'+b'i\sqrt{2}\). Nous cherchons \( q\) et \( r\) tels que la division euclidienne s'écrive \( z=qt+r\). Soient \( \alpha,\beta\in \eQ\) tels que 
    \begin{equation}
        \frac{ z }{ t }=\alpha+\beta i\sqrt{2}.
    \end{equation}
    Nous désignons par \( \alpha+\epsilon_1\) et \( \beta+\epsilon_2\) les entiers les plus proches de \( \alpha\) et \( b\). Nous avons \( | \alpha |,| \beta |\leq \frac{ 1 }{2}\). Nous posons alors naturellement 
    \begin{equation}
        q=(\alpha+\epsilon_1)+(\beta+\epsilon_2)i\sqrt{2}
    \end{equation}
    et nous calculons \( r=z-qt\) :
    \begin{equation}
        2b'\epsilon_2-a'\epsilon_1+i\sqrt{2}\big( \epsilon_1b'-a'\epsilon_2 \big).
    \end{equation}
    Nous trouvons 
    \begin{equation}
        N(r)=a'^2\epsilon_1^2+4b'^2\epsilon_2^2+2a'^2\epsilon_1^2+2b'^2\epsilon_2^2\leq \frac{ 3 }{ 4 }a'^2+\frac{ 3 }{2}b'^2.
    \end{equation}
    D'autre part \( N(t)=a'^2+2b'^2\), et nous avons donc bien \( N(r)<N(t)\).

    En ce qui concerne la seconde propriété du stathme, un petit calcul montre que
    \begin{equation}
        N(zt)=(a^2+2b^2)(a'^2+2b'^2),
    \end{equation}
    et tant que \( t\neq 0\) nous avons bien \( N(zt)>N(z)\).
\end{example}

Notons en particulier que \( \eZ[i\sqrt{2}]\) est factoriel et principal.

\begin{example} \label{ExluqIkE}
    Décomposition en facteurs irréductibles dans \( \eZ[i\sqrt{2}]\). Les éléments inversibles de \( \eZ[i\sqrt{2}]\) sont \( \pm 1\), donc deux éléments \( a\) et \( b\) sont associés (définition \ref{DefrXUixs}) si et seulement si \( a=\pm b\).

    De plus si \( p\) est irréductible, alors \( -p\) est irréductible. Les éléments irréductibles de \( \eZ[i\sqrt{2}]\) arrivent donc par pairs d'éléments associés. Soit \( \{ p_i \}\) une sélection de un élément irréductible parmi chaque paire. Tout élément \( x\) de \( \eZ[i\sqrt{2}]\) peut alors être écrit \( x=\pm p_1^{\alpha_1}\ldots p_n^{\alpha_n}\). Ce fait va être pratique pour comparer des décomposition en facteurs irréductibles d'éléments.
\end{example}

Le lemme suivant fait en pratique partie de l'exemple \ref{ExmuQisZU}, mais nous l'isolons pour plus de clarté\footnote{Merci à \href{http://fr.wikipedia.org/wiki/Utilisateur:Marvoir}{Marvoir} pour m'avoir souligné le manque.}.
\begin{lemma}       \label{LemTScCIv}
    Si \( a\) et \( b\) sont deux éléments premiers entre eux de \( \eZ[i\sqrt{2}]\) tels que \( ab=y^3\) alors \( a\) et \( b\) sont des cubes (dans \( \eZ[i\sqrt{2}]\)).
\end{lemma}

\begin{proof}
    D'après l'exemple \ref{ExluqIkE} nous pouvons écrire
    \begin{subequations}
        \begin{align}
            y&=\pm p_1^{\sigma_1}\ldots p_n^{\sigma_n}\\
            a&=\pm p_1^{\alpha_1}\ldots p_n^{\alpha_n}\\
            b&=\pm p_1^{\beta_1}\ldots p_n^{\beta_n}
        \end{align}
    \end{subequations}
    où les \( p_i\) sont les irréductibles de \( \eZ[i\sqrt{2}]\) «modulo \( \pm 1\)» au sens où la liste des irréductibles est \( \{ p_i \}\cup\{ -p_i \}\) (union disjointe). Étant donné que \( a\) et \( b\) sont premiers entre eux, \( \alpha_i\) et \( \beta_i\) ne peuvent pas être non nuls en même temps alors que leur somme doit faire \( 3\sigma_i\). Nous avons donc pour chaque \( i\) soit \( \alpha_i=3\sigma_i\) soit \( \beta=3\sigma_i\) (et bien entendu si \( \sigma_i=0\) alors \( \alpha_i=\beta_i=0\)).

    Étant donné que \( \pm 1\) sont également deux cubes, \( a\) et \( b\) sont bien des cubes.

    Notons que nous avons utilisé de façon capitale le fait que \( \eZ[i\sqrt{2}]\) était factoriel.
\end{proof}

%---------------------------------------------------------------------------------------------------------------------------
\subsection{Équations diophantiennes}
%---------------------------------------------------------------------------------------------------------------------------
%TODO : il y a une équation diophantienne qui semple pas mal ici : http://fr.wikipedia.org/wiki/Entier_quadratique#x2_.2B_5.y2_.3D_p

\begin{example} \label{ExZPVFooPpdKJc}
    L'équation diophantienne
    \begin{equation}
        x^2=3y^2+8
    \end{equation}
    n'a pas de solutions. En effet si nous prenons l'équation modulo \( 3\) nous obtenons
    \begin{equation}
        x^2\mod 3=8\mod 3=2\mod 3.
    \end{equation}
    Or dans \( \eZ/3\eZ\), aucun élément ne vérifie \( x^2=2\) : \( 0^2=0\neq 2\), \( 1^2=1\neq 2\) et \( 2^2=4=1\neq 2\).
\end{example}

\begin{example}     \label{ExmuQisZU}
    Résolvons l'équation diophantienne\index{équation!diophantienne} 
    \begin{equation}
        x^2+2=y^3.
    \end{equation}
    Une première remarque est que \( x\) doit être impair. En effet si \( x=2k\), nous devons avoir \( y^3\) pair. Mais si un cube pair est divisible par \( 8\), donc \( y^3=8l\). L'équation devient \( 4k^2+2=8l^3\), c'est à dire \( 2k^2+1=4l^3\). Le membre de gauche est impair tandis que celui de droite est pair. Impossible.

    Nous pouvons écrire l'équation sous la forme \( x^2+2=(x+i\sqrt{2})(x-i\sqrt{2})\). Et nous considérons \( \eZ[i\sqrt{2}]\) muni de sons stathme \( N\) donné par \eqref{EqOZUIooZGmHWl}.

    L'élément \( i\sqrt{2}\) est irréductible parce que \( N(i\sqrt{2})=2\). Si nous avions \( i\sqrt{2}=pq\), alors nous aurions \( N(p)N(q)=2\), ce qui n'est possible que si \( N(p)\) ou \( N(q)\) égal à \( 1\).

    Nous prouvons maintenant que les éléments \( x+i\sqrt{2}\) et \( x-i\sqrt{2}\) sont premiers entre eux. Supposons que \( d\) soit un diviseur commun; alors il divise aussi la somme et la différence. Donc \( d\) divise à la fois \( 2x\) et \( 2i\sqrt{2}\).

    Étant donné que \( i\sqrt{2}\) est irréductible et que \( 2i\sqrt{2}=(-i\sqrt{2})^3\), les diviseurs de \( 2i\sqrt{2}\) sont les puissances de \( (-i\sqrt{2})\). Du coup nous devrions avoir \( d=(i\sqrt{2})^{\alpha}\) et donc
    \begin{equation}
        x=(i\sqrt{2})^{\beta}q
    \end{equation}
    pour un certain \( q\in\eZ[i\sqrt{2}]\). Dans ce cas nous avons \( N(x)=2^{\beta}N(q)\), mais nous avons déjà précisé que \( x\) ne pouvait pas être pair, donc \( \beta=0\) et nous avons \( d=1\).

    Vu que les nombres \( x\pm i\sqrt{2}\) sont premiers entre eux et que leur produit doit être un cube, ils doivent être séparément des cubes (lemme \ref{LemTScCIv}). Nous devons donc résoudre séparément \( x\pm i\sqrt{2}=y^3\).

    Cherchons les \( x\) et \( y\) entiers tels que \( x+i\sqrt{2}=y^3\). Si nous posons \( z=a+bi\sqrt{2}\), il suffit de calculer \( z^3\) :
    \begin{verbatim}
----------------------------------------------------------------------
| Sage Version 4.8, Release Date: 2012-01-20                         |
| Type notebook() for the GUI, and license() for information.        |
----------------------------------------------------------------------
sage: var('a,b')
(a, b)
sage: z=a+I*sqrt(2)*b
sage: (z**3).expand()
3*I*sqrt(2)*a^2*b - 2*I*sqrt(2)*b^3 + a^3 - 6*a*b^2
    \end{verbatim}
    En identifiant cela à \( x+i\sqrt{2}\) nous trouvons le système
    \begin{subequations}
        \begin{numcases}{}
            x=a^3-6ab^2\\
            1=3a^2b-2b^3
        \end{numcases}
    \end{subequations}
    où, nous le rappelons, \( x\), \( a\) et \( b\) sont des entiers. Le seconde équation montre que \( b\) doit être inversible : \( b(3a^2-2b^2)=1\). Il y a donc les possibilités \( b=\pm 1\). Pour \( b=1\) l'équation devient \( 3a^2-2=1\), c'est à dire \( a=\pm 1\). Pour \( b=-1\) l'équation devient \( 3a^2-2=-1\), impossible. En conclusion les possibilités sont
    \begin{subequations}
        \begin{align}
            (x,z)=(-5,1+i\sqrt{2})\\
            (x,z)=(5,-1+i\sqrt{2})\\
        \end{align}
    \end{subequations}
    Le travail avec \( x-i\sqrt{2}\) donne les mêmes résultats.

    Les deux solutions de l'équation \( x^2+2=y^3\) sont alors \( (5,3)\) et \( (-5,3)\).
\end{example}

%--------------------------------------------------------------------------------------------------------------------------- 
\subsection{Triplet pythagoriciens et équation de Fermat pour \texorpdfstring{$ n=4$}{n=4}}
%---------------------------------------------------------------------------------------------------------------------------

\begin{proposition}[Triplets pythagoriciens\cite{fJhCTE,HARRooBvzbXo}]  \label{PropXHMLooRnJKRi}
    Un triple \( (x,y,z)\in(\eN*)^3\) est solution de \( x^2+y^2=z^2\) si et seulement si il existe \( d\in \eN\) et \( u,v\in \eN^*\) premiers entre eux tels que
    \begin{subequations}        \label{subeqLVHFooVgWsFx}
        \begin{numcases}{}
            x=d(u^2-v^2)\\
            y=2duv\\
            z=d(u^2+v^2)
        \end{numcases}
    \end{subequations}
    ou
    \begin{subequations}
        \begin{numcases}{}
            x=2duv\\
            y=d(u^2-v^2)\\
            z=d(u^2+v^2)
        \end{numcases}
    \end{subequations}
    La différence entre les deux est seulement d'inverser les rôles de \( x\) et \( y\).
\end{proposition}

\begin{proof}
    Montrons d'abord que les formules proposées sont bien des solutions; nous vérifions \eqref{subeqLVHFooVgWsFx} :
    \begin{equation}
        x^2+y^2=d^2(u^2-v^2)+4d^2u^2v^2=d^2(u^2+v^2)^2,
    \end{equation}
    qui correspond bien au \( z^2\) proposé.

    Nous allons maintenant prouver la réciproque : toute solution est d'une des deux formes proposées. D'abord une solution \emph{primitive} est un triplet \( (x,y,z)\) d'entiers premiers dans leur ensemble. Déterminer ces solutions suffira parce que si \( (x,y,z)\) n'est pas une solution primitive, alors en posant \( k=\pgcd(x,y,z)\), le triplet \( \big( \frac{ x }{ k },\frac{ y }{ k },\frac{ z }{ k } \big)\) est primitif. Connaissant les triplet primitifs, nous obtenons tous les autres par simple multiplication.

    Soit donc \( (x,y,z)\) un triplet pythagoricien primitif. D'abord nous remarquons que si deux parmi \( x\), \( y\) et \( z\) sont divisible par un nombre, alors tous les trois sont divisibles par ce nombre\footnote{Parce que si \( k\) et \( k^2\) ont les mêmes facteurs premiers.}, donc les nombres \( x\), \( y\) et \( z\) sont premiers deux à deux.

    De plus les nombres \( x\) et \( y\) ne sont pas tous les deux impairs. En effet si \( x=2a+1\) nous avons \( x^2=4a^2+4a+1\in [1]_4\) (et idem pour \( y\)). Si \( x=4a+1\) et \( y=4b+1\) alors \( z^2=x^2+y^2\in [2]_4\), ce qui signifierait que \( z^2\) est pair et donc que \( z\) est pair. Et enfin si \( z\) est pair, \( z=2c\) alors \( z^2=4c^2\in [0]_4\). Contradiction. 

    Au final, quitte à inverser les rôles de \( x\) et \( y\), nous avons \( x\) pair, \( y\) impair, \( z\) impair.

    Cela dit, nous nous attaquons à l'équation. Nous avons \( x^2=(z+y)(z-y)\) et donc
    \begin{equation}
        \left( \frac{ x }{2} \right)^2=\left( \frac{ z+y }{2} \right)^2\left( \frac{ z-y }{ 2 } \right)^2.
    \end{equation}
    Vu que \( z\) et \( y\) sont premiers entre eux, les nombres \( z-y\) et \( z+y\) sont également premiers entre eux\footnote{Si \( z-y=kn\) et \( z+y=km\), faisant la somme et la différence on voit que \( y\) et \( z\) sont divisibles par \( k\).}. Donc les facteurs premiers (qui arrivent au moins au carré) de \( (x/2)^2\) sont chacun soit dans \( (z+y)/2\) soit dans \( (z-y)/2\). Nous en déduisons que ces derniers sont des carrés d'entiers. Nous posons
    \begin{equation}
        \begin{aligned}[]
            \frac{ z-y }{2}=u^2&&\frac{ z+y }{2}=v^2.
        \end{aligned}
    \end{equation}
    Bien entendu \( u\) et \( v\) sont non nuls parce que nous avons exclu la possibilité de triplets dont un élément serait nul. Avec tout cela nous avons \( (x/2)^2=u^2v^2\) et donc \( x=2uv\) puis par somme et différence :
    \begin{subequations}
        \begin{numcases}{}
            x=2yv\\
            y=v^2-u^2\\
            z=m^2+n^2,
        \end{numcases}
    \end{subequations}
    ce qu'il fallait.
\end{proof}

\begin{remark}
    Les solutions dans lesquelles \( x\), \( y\) ou \( z\) sont nuls sont faciles à classer. La solution \( (1,0,1)\) n'est pas dans les formes proposées. En effet elle ne peut pas être de la première forme : avoir \( y=0\) demanderait qu'un nombre parmi \( d\), \( u\) et \( v\) soit nul, ce qui est interdit. La solution \( (1,0,1) \) ne peut pas non plus être de la seconde forme parce que \( x\) y est pair.
\end{remark}

\begin{remark}
    Les solutions entières (positives) de l'équation \( x^2+y^2=z^2\) sont appelés \defe{triplets pythagoriciens}{triplet!pythagoricien}. Ils donnent toutes les possibilités de triangles rectangles dont les côtés ont des longueurs entières.
\end{remark}

\begin{proposition}[\cite{fJhCTE}]      \label{propFKKKooFYQcxE}
    Les équations \( x^4+y^4=z^2\) et \( x^2+y^4=z^4\) n'ont pas de solutions dans \( (\eN^*)^3\).
\end{proposition}
\index{équation!diophantienne}

\begin{proof}
    Si la première équation n'a pas de solutions, alors la seconde n'en n'a pas non plus parce que \( z^4\) est un carré. Nous nous concentrons donc sur l'équation \( x^4+y^4=z^2\) et nous supposons qu'il existe une solution \( (x,y,z)\) dans \( (\eN^*)^3\) que nous prenons avec \( z\) minimum. Du coup, les trois nombres \( x\), \( y\) et \( z\) sont premiers dans leur ensembles parce que une division par leur \( \pgcd\) donnerait une nouvelle solution qui contredirait la minimalité de \( z\).

    Nous posons \( x^4=\bar x^2\) et \( y^4=\bar y^2\). Ils vérifient l'équation \( \bar x^2+\bar y^2=z^2\) et par la proposition \ref{PropXHMLooRnJKRi}, il existe \( u,v\in \eN^*\) premiers entre eux tels que
    \begin{subequations}
        \begin{numcases}{}
            \bar x=2uv\\
            \bar y=u^2-v^2\\
            z=u^2+v^2.
        \end{numcases}
    \end{subequations}
    Si \( u\) est pair, alors \( v\) est impair (et inversement) parce que \( \pgcd(u,v)=1\).a. Si \( u\) est pair, alors \( u=2a\) et \( v=2b+1\), ce qui donne \( \bar y=4a^2-4b^2-4b-1\in[-1]_4\). Or nous avons déjà vu qu'un carré est dans \( [0]_4\) ou dans \( [1]_4\). Il faut donc que \( u\) soit impair et \( v\) pair.

    Nous pouvons donc écrire l'équation \( v^2+\bar y=u^2\) où \( u\), \( v\) et \( \bar y\) sont premiers dans leur ensemble (dans une égalité \( a+b=c\), si deux des nombres ont un diviseur commun, le troisième l'a aussi). Vu que \( \bar y=y^2\), le triplet \( (v,y,u)\) est un triplet pythagoriciens élémentaire. Il existe donc deux nombres \( r\) et \( s\) premiers entre eux tels que
    \begin{subequations}
        \begin{numcases}{}
            v=2rs\\
            y=r^2-s^2\\
            u=r^2+s^2.
        \end{numcases}
    \end{subequations}
    Avec cela, \( \bar x=2uv=2rs(r^2+s^2)\). Vu que \( \bar x\) est un carré et que \( r\), \( s\) et \( r^2+s^2\) sont premiers entre eux, ils doivent séparément être des carrés, ou alors \( rs(r^2+s^2)=2\). Voyons que cette deuxième possibilité est impossible. Un produit de deux entiers qui vaut \( 2\), c'est soit \( 1\times 2\), soit \( 2\times 1\). D'une part \( r^2+s^2=1\) est impossible parce que \( r\) et \( s\) sont non nuls, et d'autre part, demander \( r^2+s^2=2\) avec \( rs=1\), c'est demander \( r=s=1\), mais alors \( v=2\), \( y=0\) et \( u=2\), ce qui contredit le fait que \( u\) et \( v\) sont premiers entre eux. Bref, les nombres \( r\), \( s\) et \( r^2+s^2\) sont séparément des carrés :
    \begin{subequations}
        \begin{numcases}{}
            r=\alpha^2\\
            s=\beta^2\\
            r^2+s^2=\gamma^2.
        \end{numcases}
    \end{subequations}
    En mettant les deux premiers dans la troisième, nous avons \( \alpha^2+\beta^4=\gamma^2\). Donc \( (\alpha,\beta,\gamma)\) est une solution. Nous allons prouver que \( \gamma<z\), ce qui terminera la preuve. Nous avons :
    \begin{equation}
        z=u^2+v^2=r^2+s^2+4r^2s^2=\gamma^2+4r^2s^2.
    \end{equation}
    Cela prouve que \( \gamma^2<z\) et a fortiori que \( \gamma<z\).
\end{proof}

%--------------------------------------------------------------------------------------------------------------------------- 
\subsection{Lignes et colonnes de matrices}
%---------------------------------------------------------------------------------------------------------------------------

Nous nommons \( E_{ij}\) la matrice remplie de zéros sauf à la case \( ij\) qui vaut \( 1\). Autrement dit
\begin{equation}
    (E_{ij})_{kl}=\delta_{ik}\delta_{jl}.
\end{equation}
\begin{definition}
    Une \defe{matrice de transvection}{transvection (matrice)}\index{matrice!de transvection} est une matrice de la forme
    \begin{equation}
        T_{ij}(\lambda)=\id+\lambda E_{ij}
    \end{equation}
    avec \( i\neq j\).

    Une \defe{matrice de dilatation}{matrice!de dilatation}\index{dilatation (matrice)} est une matrice de la forme
    \begin{equation}
        D_i(\lambda)=\id+(\lambda-1)E_{ii}.
    \end{equation}
    Ici le \( (\lambda-1)\) sert à avoir \( \lambda\) et non \( 1+\lambda\). C'est donc une matrice qui dilate d'un facteur \( \lambda\) la direction \( i\) tout en laissant le reste inchangé.

    Si \( \sigma\) est une permutation (un élément du groupe symétrique \( S_n\)) alors la \defe{matrice de permutations}{matrice!de permutation}\index{permutation!matrice} associée est la matrice d'entrées
    \begin{equation}
        (P_{\sigma})_{ij}=\delta_{i\sigma(j)}.
    \end{equation}
\end{definition}

\begin{lemma}   \label{LemyrAXQs}
    La matrice \( T_{ij}(\lambda)A=(\mtu+\lambda E_{ij})A\) est la matrice \( A\) à qui on a effectué la substitution
    \begin{equation}
        L_i\to L_i+\lambda L_j.
    \end{equation}
    La matrice \( AT_{ij}(\lambda)\) est la substitution 
    \begin{equation}
        C_j\to C_j+\lambda C_i.
    \end{equation}

    La matrice \( AP_{\sigma}\) est la matrice \( A\) dans laquelle nous avons permuté les colonnes avec \( \sigma\).

    La matrice \( P_{\sigma}A\) est la matrice \( A\) dans laquelle nous avons permuté les lignes avec \( \sigma^{-1}\).
\end{lemma}

\begin{proof}
    Calculons la composante \( kl\) de la matrice \( E_{ij}A\) :
    \begin{subequations}
        \begin{align}
            (E_{ij}A)_{kl}&=\sum_m(E_{ij})_kmA_{ml}\\
            &=\sum_m\delta_{ik}\delta_{jm}A_{ml}\\
            &=\delta_{ik}A_{jl}.
        \end{align}
    \end{subequations}
    C'est donc la matrice pleine de zéros, sauf la ligne \( i\) qui est donnée par la ligne \( j\) de \( A\). Donc effectivement la matrice
    \begin{equation}
        A+\lambda E_{ij}A
    \end{equation}
    est la matrice \( A\) à laquelle on a substitué la ligne \( i\) par la ligne \( i\) plus \( \lambda\) fois la ligne \( j\).

    En ce qui concerne l'autre assertion sur les transvections, le calcul est le même et nous obtenons
    \begin{equation}
        (AE_{ij})=A_{ki}\delta_{jl}.
    \end{equation}

    Pour les matrices de permutations, nous avons 
    \begin{equation}
        (AP_{\sigma})_{kl}=A_{k\sigma(l)}
    \end{equation}
    et
    \begin{equation}
        (P_{\sigma}A)_{kl}=\sum_m\delta_{k\sigma(m)}A_{ml}=\sum_m\delta_{\sigma^{-1}(k)m}A_{ml}=A_{\sigma^{-1}(k)l}.
    \end{equation}
\end{proof}


%--------------------------------------------------------------------------------------------------------------------------- 
\subsection{Algorithme des facteurs invariants}
%---------------------------------------------------------------------------------------------------------------------------

\begin{proposition}[Algorithme des facteurs invariants\cite{KXjFWKA}]   \label{PropPDfCqee}
    Soit \( (\eA,\delta)\) un anneau euclidien muni de son stathme  et \( U\in \eM(n,m,\eA)\). Alors il existe \( d_1,\ldots, d_s\in \eA^*\) et des matrices \( P\in\GL(m,\eA)\), \( Q\in \GL(n,\eA)\) tels que nous ayons
    \begin{equation}
        U=P \begin{pmatrix}
            \begin{matrix}
                d_1    &       &       \\
                    &   \ddots    &       \\
                    &       &   d_s
            \end{matrix}&   0    \\ 
            0    &   0    
        \end{pmatrix}Q
    \end{equation}
    avec \( d_i\divides d_{i+1}\) pour tout \( i\).
\end{proposition}
\index{anneau!euclidien!facteurs invariants}
\index{algorithme!facteurs invariants}

\begin{proof}
    Nous allons donner la preuve plus ou moins sous forme d'algorithme.

    D'abord si \( U=0\) c'est bon, on a la réponse. Sinon, nous prenons l'élément \( (i_0,j_0)\) dont le stathme est le plus petit et nous l'amenons en \( (1,1)\) par les permutations
    \begin{equation}
        \begin{aligned}[]
            C_1&\leftrightarrow C_{j_0}\\
            L_1&\leftrightarrow L_{i_0}
        \end{aligned}
    \end{equation}
    Ensuite nous traitons la première colonne jusqu'à amener des zéros partout en dessous de \( u_{11}\) de la façon suivante : pour chaque ligne successivement nous calculons la division euclidienne
    \begin{equation}
        u_{i1}=qu_{11}+r_i,
    \end{equation}
    et nous faisons
    \begin{equation}
        L_i\to L_i-qL_1,
    \end{equation}
    c'est à dire que nous enlevons le maximum possible et il reste seulement \( r_i\) en \( u_{i1}\). Vu que le but est de ne laisser que des zéros dans la première colonne, si le reste n'est pas zéro nous ne sommes pas content\footnote{Si il est zéro, nous passons à la ligne suivante}. Dans ce cas nous permutons \( L_1\leftrightarrow L_i\), ce qui aura pour effet de strictement diminuer le stathme de \( u_{11}\) parce qu'on va mettre en \( u_{11}\) le nombre \( r_i\) dont le stathme est strictement plus petit que celui de \( u_{11}\).

    En faisant ce jeu de division euclidienne puis échange, on diminue toujours le stathme de \( u_{11}\), donc ça fini par s'arrêter, c'est à dire qu'à un certain moment la division euclidienne de \( u_{i1}\) par \( u_{11}\) va donner un reste zéro et nous serons content.

    Une fois la première colonne ramenée à la forme
    \begin{equation}
        C_1=\begin{pmatrix}
            u_{11}    \\ 
            0    \\ 
            \vdots    \\ 
            0    
        \end{pmatrix},
    \end{equation}
    nous faisons tout le même jeu avec la première ligne en faisant maintenant des sommes divisions et permutations de colonnes. Notons que ce faisant nous ne changeons plus la première colonne.

    En fin de compte nous trouvons une matrice\footnote{Nous nommons toujours par la même lettre \( U\) la matrice originale et la modifiée, comme il est d'usage en informatique.}
    \begin{equation}
        U=\begin{pmatrix}
            u_{11}   &   0    &   \ldots    &   0    \\
             0   &       &       &       \\
             \vdots   &       &   A    &       \\ 
             0   &       &       &        
         \end{pmatrix}
    \end{equation}
    Si l'élément \( u_{11}\) ne divise pas un des éléments de \( A\), disons \( a_{ij}\), alors nous faisons 
    \begin{equation}
        C_1\to C_1-C_j.
    \end{equation}
    Cela nous détruit un peu la première colonne, mais ne change pas \( u_{11}\). Nous avons maintnant
    \begin{equation}
        U=\begin{pmatrix}
            u_{11}   &   0    &   \ldots    &   0    \\
             0   &       &       &       \\
             *   &       &       &       \\ 
             u_{ij}   &       &   A    &       \\ 
             *   &       &       &       \\ 
             0   &       &       &        
         \end{pmatrix}
    \end{equation}
    Et nous refaisons tout le jeu depuis le début. Cependant lorsque nous allons nous attaquer à la ligne \( i\), \( u_{11}\) ne divisera pas \( u_{ij}\), ce qui donnera lieu à une division euclidienne et un échange \( L_1\leftrightarrow L_i\). Cet échange mettre \( r_i\) à la place de \( u_{11}\) et donc diminuera encore strictement le stathme. Encore une fois nous allons travailler jusqu'à avoir la matrice sous la forme
    \begin{equation}    \label{EqADcNVgI}
        U=\begin{pmatrix}
            u_{11}   &   0    &   \ldots    &   0    \\
             0   &       &       &       \\
             \vdots   &       &   A    &       \\ 
             0   &       &       &        
         \end{pmatrix},
    \end{equation}
    saut que cette fois le stathme de \( u_{11}\) est strictement plus petit que la fois précédente. Si \( u_{11}\) ne divise toujours pas tous les éléments de \( A\), nous recommençons encore et encore. En fin de compte nous finissons par avoir une matrice de la forme \eqref{EqADcNVgI} avec \( u_{11}\) qui divise tous les éléments de \( A\).

    Une fois que cela est fait, il faut continuer en recommençant tout sur la matrice \( A\). Nous avons maintenant
    \begin{equation}
        U=\begin{pmatrix}
            \begin{matrix}
                u_{11}  &       \\ 
                &   u_{22}    
            \end{matrix}&   0    \\ 
            0    &   B    
        \end{pmatrix}.
    \end{equation}
    Sous cette forme nous avons \( u_{11}\divides u_{22}\) et \( u_{11}\) divise tous les éléments de \( B\). En effet \( u_{11}\) divisant tous les éléments de \( A\), il divise toutes les combinaisons de ces éléments. Or tout l'algorithme ne consiste qu'à prendre des combinaisons d'éléments.

    Nous finissons donc bien sur une matrice comme annoncée. De plus n'ayant effectué que des combinaisons de lignes, nous avons seulement multiplié par des matrices inversibles (lemme \ref{LemyrAXQs}).
\end{proof}

%+++++++++++++++++++++++++++++++++++++++++++++++++++++++++++++++++++++++++++++++++++++++++++++++++++++++++++++++++++++++++++
\section{Anneaux des polynômes}
%+++++++++++++++++++++++++++++++++++++++++++++++++++++++++++++++++++++++++++++++++++++++++++++++++++++++++++++++++++++++++++

Soit \( A\) un anneau commutatif. Nous considérons \( \polyP\) l'ensemble des suites presque nulles d'éléments de \( A\), ce sont les suites \( (a_n)_{n\in\eN}\) telles que il existe \( N\) tel que \( a_i=0\) pour tout \( i>N\).

Cela est un \( A\)-module libre de base (définition \ref{DefBasePouyKj})
\begin{equation}
    (e_n)_k=\delta_{nk}.
\end{equation}
Si \( (a_n)_{n\in \eN}\) et \( (b_n)_{n\in\eN}\) sont des éléments de \( \polyP\), nous définissons le produit \( ab\) par
\begin{equation}
    (ab)_n=\sum_{p+q=n}a_pb_q.
\end{equation}
Cela est bien un élément de \( \polyP\) parce qu'il existe \( N\in\eN\) tel que \( a_n=b_n=0\) pour tout \( n\geq N\). Avec la somme et le produit par un scalaire, le module \( \polyP\) devient une \( A\)-algèbre commutative unitaire. L'unité est 
\begin{equation}
    e_0=(1,0,\ldots).
\end{equation}

\begin{definition}  \label{DefRGOooGIVzkx}
    En tant que \( A\)-algèbre, l'ensemble \( \polyP\) est l'\defe{algèbre des polynômes en une indéterminée}{algèbre!polynômes} à coefficients dans \( A\).
\end{definition}

Si nous posons que \( X=e_1\), et que nous prenons la convention \( X^0=1\), alors nous avons \( e_k=X^k\) et nous notons \( A[X]\)\nomenclature[A]{\( A[X]\)}{tous les polynômes de degré fini à coefficients dans \( A\)} l'anneau \( \polyP\) exprimé avec \( X\). Les éléments de la forme \( \lambda X^k\) avec \( \lambda\in A\) et \( k\in\eN\) sont des \defe{monômes}{monôme}. Nous allons aussi considérer\nomenclature[A]{\( A_n[X]\)}{les polynômes à coefficients dans \( A\) et de degré inférieur à \( n\)}
\begin{equation}
    A_n[X]=\{ P\in A[X]\tq \deg(P)\leq n \}.
\end{equation}
Cela est un sous module libre.

\begin{remark}  \label{RemLIOooXHePSd}
    L'ensemble \( A[X]\) est une algèbre et donc un espace vectoriel. Il possède un unique élément nul qui est celui dont tous les coefficients sont nuls; cela est immédiat par la construction en tant que suites presque nulles.

    Il n'y a a priori pas équivalence entre le fait d'être un polynôme nul et le fait de s'évaluer à zéro sur tous les éléments de \( A\). Cela sera discuté dans le théorème \ref{ThoLXTooNaUAKR} et l'exemple \ref{exVQBooBMPLkD}.
\end{remark}

\begin{theorem}     \label{ThoBUEDrJ}
    L'anneau \( A\) est intègre si et seulement si \( A[X]\) est intègre.
\end{theorem}

\begin{proof}
    Soient \( P\) et \( Q\) des éléments non nuls de \( A[X]\). Vu que l'anneau \( A\) est intègre, nous avons
    \begin{equation}
        \deg(PQ)=\deg(P)+\deg(Q)
    \end{equation}
    et le produit ne peut pas être nul. L'anneau \( A[X]\) est donc intègre.

    Si \( A[X]\) est intègre, \( A\) est intègre parce qu'il peut être vu comme sous anneau.
\end{proof}

\begin{remark}
    Si \( A\) n'est pas intègre, soit \( \alpha\beta=0\), alors \( (\alpha X)(\beta x)=0\) et le degré du produit n'est pas la somme des degrés.
\end{remark}

\begin{corollary}
    Si \( A\) est intègre, les inversibles de \( A[X]\) sont les éléments de \( U(A)\).
\end{corollary}

\begin{proof}
    Pour que \( Q\) soit inversible, il faut un \( P\) tel que \( PQ=1\). Mais l'anneau \( A\) étant intègre, les degrés s'additionnent. Par conséquent ils doivent être de degrés zéro et il faut que \( P,Q\in A\). Enfin pour qu'ils soient inversibles, ils doivent être dans \( U(A)\).
\end{proof}

La \defe{valuation}{valuation} de \( P\) du polynôme \( P=\sum_n a_nX^n\), notée \( \val(P)\), est 
\begin{equation}
    \val(P)=\min\{ n\tq a_n\neq 0 \}.
\end{equation}
Nous avons \( \val(P)\leq \deg(P)\) et \( \val(P)=\deg(P)\) si et seulement si \( P\) est un monôme. Si \( P=0\), nous convenons que \( \val(0)=\infty\) et \( \deg(0)=-\infty\).

\begin{proposition}     \label{PropqGZXvr}
    L'anneau \( \eK[X]\) des polynômes sur un corps commutatif \( \eK\) est factoriel.
\end{proposition}
%TODO : une preuve.

Le théorème suivant est une particularisation à \( \eK[X]\) du théorème chinois \ref{ThofPXwiM}.
\begin{theorem}[Théorème chinois]\index{théorème!chinois!anneau des polynômes}
    Si \( P\) et \( Q\) sont deux polynômes premiers entre eux, alors nous avons l'isomorphisme
    \begin{equation}
        \eK[X]/(P,Q)\simeq\eK[X]/(P)\times \eK[X]/(Q).
    \end{equation}
\end{theorem}
% TODO : s'assurer que c'est bien un icp du théorème chinois de plus haut.

%---------------------------------------------------------------------------------------------------------------------------
\subsection{Irréductibilité}
%---------------------------------------------------------------------------------------------------------------------------

\begin{theorem}[d'Alembert-Gauss]\index{théorème!d'Alembert-Gauss}      \label{ThovgyUuA}
    Tout polynôme non constant à coefficients complexes possède au moins une racine complexe.
\end{theorem}

\begin{definition}      \label{DefIrredfIqydS}
    Un polynôme est \defe{irréductible}{irréductible!polynôme}\index{polynôme!irréductible} lorsqu'il ne peut pas être écrit sous la forme de produits de polynômes de degré supérieurs à \( 1\).
\end{definition}

\begin{example}
    Si un polynôme \( P\in \eZ[X]\) n'a que des racines complexes, ça ne l'empêche pas d'être réductible sur \( \eZ\). La réductibilité ne signifie pas qu'on peut mettre des racines en évidence. Par exemple le polynôme \( P=X^4+3X^2+2\) est réductible sur \( \eZ\) en
    \begin{equation}
        P=(X^2+1)(X^2+2),
    \end{equation}
    mais n'a pas de racines dans \( \eZ\). Par contre, il est vrai que si on veut réduire plus, il faut sortir de \( \eZ\).

\end{example}

\begin{proposition}
    Un polynôme irréductible à coefficients réels est soit de degré un soit de degré \( 2\) avec un discriminant négatif.
\end{proposition}

\begin{proof}
    Soit un polynôme \( P\) à coefficients réels de degré plus grand que \( 1\). Alors le théorème de d'Alembert-Gauss (théorème \ref{ThovgyUuA}) implique l'existence d'une racine \( \alpha\). Il est facile de montrer que le conjugué complexe \( \bar \alpha\) est également racine. Par conséquent les polynômes \( (X-\alpha)\) et \( (X-\bar \alpha)\) divisent \( P\).

    Ces deux polynômes sont premiers entre eux parce que
    \begin{equation}
        a(X-\alpha)+b(X-\bar \alpha)=0
    \end{equation}
    implique \( a=b=0\). Par conséquent le produit 
    \begin{equation}
        X^2-(\alpha+\bar \alpha)X+\alpha\bar\alpha
    \end{equation}
    divise également \( P\). Ce dernier est un polynôme à coefficients réels de degré \( 2\). Donc tout polynôme de degré \( 3\) ou plus est réductible.
\end{proof}

\begin{definition}  \label{DefCPLSooQaHJKQ}
    Nous disons que \( P\in\eK[X]\) est \defe{scindé}{polynôme!scindé} sur \(\eK\) si il est produit dans \(\eK[X]\) de polynômes de degré~\( 1\). 
\end{definition}
Note : les constantes ne sont donc pas des polynôme scindés.

\begin{theorem}[Conséquence du \wikipedia{fr}{Lemme_de_Gauss_(polynômes)}{lemme de Gauss}]     \label{ThofiIpXg} 
    Soit \( \eA\) un anneau factoriel et \( \Frac(\eA)\) son corps des fractions. Un polynôme non constant \( P\in \eA[X]\) est irréductible (sur \( \eA\)) si et seulement si il est irréductible et primitif sur \( \Frac(\eA)[X]\). 
\end{theorem}
\index{primitif!polynôme}

Dans cet énoncé, un polynôme primitif est un polynôme dont le \( \pgcd\) des coefficients est \( 1\). Voir la remarque \ref{RemwwJbYP}. Notons qu'ici nous considérons des polynômes dont les coefficients sont dans un anneau et non dans un corps comme nous en avons l'habitude.

%---------------------------------------------------------------------------------------------------------------------------
\subsection{Division euclidienne}
%---------------------------------------------------------------------------------------------------------------------------

Le théorème suivant établit la \defe{division euclidienne}{division!euclidienne} dans \( \eA[X]\) du polynôme \( A\) par \( B\).
\begin{theorem}     \label{ThodivEuclPsFexf}
    Soit \( B\neq 0\) dans \( \eA[X]\) de coefficient dominant inversible dans \( \eA\). Pour tout \( A\in\eA[X]\), il existe \( Q,R\in \eA[X]\) tels que
    \begin{equation}
        A=BQ+R
    \end{equation}
    avec \( \deg(R)<\deg(B)\).

    Les polynômes \( Q\) et \( R\) sont déterminés de façon univoque par cette condition. 
    
\end{theorem}

\begin{definition}\label{DefMPZooMmMymG}
    Le polynôme \( Q\) est le \defe{quotient}{quotient} et \( R\) est le \defe{reste}{reste} de la division euclidienne de \( A\) par \( B\). Si le reste de la division de \( A\) par $B$ est nul on dit que \( B\) \defe{divise}{diviseur!polynôme} \( A\) et on note \( B\divides A\)\nomenclature[A]{\( B\divides A\)}{\( B\) divise \( A\)}. Autrement dit \( B\) divise \( A\) si il existe \( Q\) tel que \( A=BQ\).
\end{definition}

\begin{corollary}
    Si \( \eA\) est un anneau intègre, l'anneau des polynômes à coefficients dans \( \eA\) est un anneau euclidien et principal.
\end{corollary}

\begin{proof}
    Le théorème \ref{ThodivEuclPsFexf} montre que le degré est un stathme euclidien (définition \ref{DefAXitWRL}), ce qui fait que l'anneau des polynômes est un anneau euclidien et en particulier un anneau principal par la proposition \ref{Propkllxnv}. 
\end{proof}

\begin{definition}  \label{DefDSFooZVbNAX}
Deux polynômes \( P\) et \( Q\) sont dits \defe{étrangers}{etranger@étrangers!polynômes} entre eux si \( 1\) est un \( \pgcd\) de \( P\) et \( Q\). Un ensemble de polynômes \( (P_i)_{i\in I}\) est étranger \defe{dans leur ensemble}{étranger!dans leur ensemble} si \( 1\) est un \( \pgcd\) des \( P_i\).
    
Les polynômes \( P\) et \( Q\) sont \defe{premiers entre eux}{premier!deux polynômes entre eux} si les seuls diviseurs communs de \( P\) et \( Q\) sont les inversibles.
\end{definition}

%--------------------------------------------------------------------------------------------------------------------------- 
\subsection{Bézout}
%---------------------------------------------------------------------------------------------------------------------------

\begin{theorem}[Bézout] \label{ThoBezoutOuGmLB}     
    Les polynômes \( P_1,\ldots,P_n\) dans \( \eK[X]\) sont étrangers entre eux si et seulement si il existe des polynômes \( Q_1,\ldots,Q_n\in\eK[X]\) tels que
    \begin{equation}
        P_1Q_1+\ldots+P_nQ_n=1.
    \end{equation}
\end{theorem}
\index{Bézout!polynômes}
\index{théorème!Bézout!polynômes}

Deux polynômes \( P\) et \( Q\) ne sont donc pas premiers entre eux si il existe des polynômes \( x\) et \( y\) tels que l'identité de Bézout soit vérifiée :
\begin{equation}    \label{EqkbbzAi}
    xP+yQ=0;
\end{equation}
cette dernière pourra être écrite en termes de la matrice de Sylvester, voir sous-section \ref{subsecSQBJfr}.

\begin{lemma}       \label{LemuALZHn}
    Soient \( (P_i)_{i=1,\ldots,n}\in \eK[X]\) des polynômes étrangers deux à deux. Alors les polynômes \begin{equation} Q_i=\prod_{j\neq i}P_j \end{equation}
    sont étrangers entre eux\footnote{Et non juste deux à deux.}.
\end{lemma}

\begin{lemma}[\cite{SQxrsoL}]   \label{LemzwkYdn}
    Soit \( \eK\) un corps commutatif et \( \eA\subset \eK\) un sous anneau de \( \eK\). Soit \( \phi\in \eK[X]\). Si il existe \( Q\in \eA[X]\) unitaire tel que \( \phi Q\in \eA[X]\), alors \( \phi\in \eA[X]\).
\end{lemma}

\begin{lemma}   \label{LemUELTuwK}
    Quelques propriétés du PGCD dans les polynômes.
    \begin{enumerate}
        \item
            Soit \( P,Q,R\in \eK[X]\) des polynômes tels que \( P\) soit premier avec \( Q\). Alors
            \begin{equation}
                \pgcd(P,QR)=\pgcd(P,Q)\pgcd(P,R)
            \end{equation}
        \item
            En analogie avec le lemme \ref{LemiVqita}, nous avons
            \begin{equation}
                \pgcd(P,PQ+R)=\pgcd(P,R).
            \end{equation}
    \end{enumerate}
\end{lemma}
\index{PGCD!polynômes}
\begin{probleme}
    Est-ce que ce lemme \ref{LemUELTuwK} est correct ? J'aimerais en voir une preuve.
\end{probleme}

%---------------------------------------------------------------------------------------------------------------------------
\subsection{Idéaux}
%---------------------------------------------------------------------------------------------------------------------------

Soit \( P\in \eK[X]\) un polynôme. Nous notons \( (P)\) l'idéal engendré par \( P\) :
\begin{equation}        \label{EqDefxMkDtW}
    (P)=\{ PR\tq R\in\eK[X] \}.
\end{equation}

\begin{lemma}
    Nous avons
    \begin{enumerate}
        \item
            \( (P)\subset (Q)\) si et seulement si \( Q\) divise \( P\),
        \item
            \( (P)=(Q)\) si et seulement si \( P\) et \( Q\) sont multiples (non nuls) l'un de l'autre.
    \end{enumerate}
\end{lemma}

\begin{proof}
    Si \( (P)\subset (Q)\), en particulier \( P\in(Q)\) et il existe \( R\in\eK[X]\) tel que \( P=QR\), ce qui signifie que \( Q\) divise \( P\).

    Si les idéaux de \( P\) et de \( Q\) sont identiques, l'un divise l'autre et l'autre divise l'un. Ils sont donc multiples l'un de l'autre.
\end{proof}

\begin{theorem}     \label{ThoCCHkoU}
    Soit \( \eK\) un corps commutatif.
    \begin{enumerate}
        \item
            L'anneau \( \eK[X]\) est principal. 
        \item
            Si \( I\) est un idéal dans \( \eK[X]\) et si \( P\) est de degré minimal, alors \( (P)=I\).
        \item   \label{ITEMooASHKooZqkiCH}
            De plus si \( I\neq \{  0\}\), il existe un unique polynôme unitaire \( \mu\) tel que \( I=(\mu)\).
    \end{enumerate}
\end{theorem}

\begin{proof}
    L'anneau \( \eK[X]\) est commutatif et intègre (pas de diviseurs de zéro). Nous devons encore montrer que tous les idéaux sont principaux.

    Si \( I=\{ 0 \}\), le résultat est évident. Nous supposons donc \( I\) non nul. Soit \( P\) de degré minimum parmi les éléments de \( I\). Évidemment \( (P)\subset I\). Nous allons démontrer qu'en réalité \( (P)=I\).

    Soit \( A\in I\). Par le théorème \ref{ThodivEuclPsFexf} de la division euclidienne\footnote{Ici \( \eK\) est un corps et donc l'hypothèse d'inversibilité est automatiquement vérifiée.}, il existe \( Q\) et \( R\) dans \( \eK[X]\) tels que \( A=PQ+R\) avec \( \deg(R)<\deg(P)\). Étant donné que \( R=A-PQ\) nous avons \( R\in I\) et par conséquent \( R=0\) parce que \( P\) a été choisit de degré minimum dans \( I\). Nous avons donc \( A=PQ\) et \( I\subset (P)\).

    L'existence d'un polynôme unitaire qui génère \( I\) est obtenu en choisissant \( U=P/a_n\) où \( a_n\) est le coefficient du terme de plus haut degré.
\end{proof}
Nous voyons que n'importe quel polynôme de degré minimum dans un idéal génère l'idéal. Une importante conséquence du théorème \ref{ThoCCHkoU} que nous verrons plus bas est que tout polynôme annulateur d'un endomorphisme est divisé par le polynôme minimal (proposition \ref{PropAnnncEcCxj}).

%--------------------------------------------------------------------------------------------------------------------------- 
\subsection{Racines des polynômes}
%---------------------------------------------------------------------------------------------------------------------------
\begin{proposition} \label{PropHSQooASRbeA}
    Soit \( \eA\) un anneau et \( P\) un polynôme non nul dans \( \eA[X]\). Si \( \alpha\in \eA\) est une racine de \( P\) alors \( X-\alpha\) divise \( P\).
\end{proposition}

\begin{proof}
    Nous notons le polynôme \( \mu=X-\alpha\) par analogie avec le polynôme minimal dont il sera question dans la très semblable proposition \ref{PropXULooPCusvE}. 
    Vu que \( P\) possède une racine, il est de degré au moins égal à \( 1\) et nous pouvons effectuer la division euclidienne\footnote{Théorème \ref{ThodivEuclPsFexf}.} de \( P\) par \( \mu\) : il existe des polynômes \( Q\) et \( R\) tels que
    % Il faut laisser la coupure entre les deux lignes pour séparer deux références.
    \begin{equation}
        Q=Q\mu+R
    \end{equation}
    avec \( \deg(R)<\deg(\mu)\). Donc \( \deg(R)=0\) et il est constant. Il existe donc \( a\in \eA\) tel que \( S=Q\mu+a\). En évaluant cela en \( \alpha\),
    \begin{equation}
        S(\alpha)=Q(\alpha)\mu(\alpha)+a,
    \end{equation}
    nous voyons que \( a=0\), ce qui montre que \( S=Q\mu\) et que \( \mu\) divise \( S\).
\end{proof}

\begin{definition}[Racine simple et multiple d'un polynôme]
    Soit \( \eA\) un anneau ainsi qu'un polynôme \( P\in \eA[X]\) et \( \alpha\in \eA\). La \defe{multiplicité}{multiplicité!racine d'un polynôme} de \( \alpha\) par rapport à \( P\) est l'entier \( h\) tel que \( P\) est divisible par \( (X-\alpha)^h\) mais pas divisible par \( (X-\alpha)^{h+1}\).  Nous noterons \( \theta_{\alpha}(P)\)\nomenclature[A]{\( \theta_{\alpha}(P)\)}{la multiplicité de \( \alpha\) par rapport à \( P\)} la multiplicité de \( \alpha\) par rapport à \( P\).
\end{definition}

Pour une définition générale d'une racine simple de l'équation \( f(x)=0\), voir la définition \ref{DEFooXSOQooAnWqKM}.

La proposition \ref{PropHSQooASRbeA} nous indique que toute racine est de multiplicité au moins \( 1\).

\begin{proposition}     \label{PropahQQpA}
    L'élément \( \alpha\in \eA\) est une racine de multiplicité \( h\) du polynôme\( P\) si et seulement si il existe \( Q\in\eA[X]\) tel que \( P=(X-\alpha)^hQ\) avec \( Q(\alpha)\neq 0\).
\end{proposition}

\begin{lemma}       \label{LemIeLhpc}
    Soient \( P\) et \( Q\) des polynômes non nuls de \( \eA[X]\) et \( \alpha\in \eA\) de multiplicité \( p\) pour \( P\) et de multiplicité \( q\) pour \( Q\). Alors
    \begin{enumerate}
        \item
            \( \theta_{\alpha}(P+Q)\geq\ln\{ \theta_{\alpha}(P),\theta_{\alpha}(Q) \}\)
        \item
            si \( \theta_{\alpha}(P)\neq \theta_{\alpha}(Q)\), alors \( \theta_{\alpha}(P+Q)=\min\{ \theta_{\alpha}(P),\theta_{\alpha}(Q) \}\)
        \item
            \( \theta_{\alpha}(PQ)\geq \theta_{\alpha(P)}+\theta_{\alpha}(Q)\);
        \item       \label{ItemIeLhpciv}
            si \(\eA \) est intègre alors \( \theta_{\alpha}(PQ)= \theta_{\alpha}(P)+\theta_{\alpha}(Q)\);
    \end{enumerate}
\end{lemma}

\begin{theorem} \label{ThoSVZooMpNANi}
    Soit \( \eA\) un anneau intègre et \( P\in \eA[X]\setminus\{ 0 \}\), un polynôme de degré \( n\). Si \( \alpha_1,\ldots, \alpha_p\in\eA\) sont des racines deux à deux distinctes de multiplicités \( k_1,\ldots, k_p\), alors il existe \( Q\in \eA[X]\) tel que
    \begin{enumerate}
        \item
            \( Q(\alpha_i)\neq 0\);
        \item   \label{ItemJZZVooMogYLq}
            \( P=Q\prod_{i=1}^p(X-\alpha_i)^{k_i}\);
        \item
            Le degré de \( Q\) est \( n-p\).
    \end{enumerate}
    De plus la sommes des multiplicités des racines de \( P\) est au plus \( \deg(P)\).
\end{theorem}
\index{factorisation!de polynôme}

\begin{proof}
    Si \( p=1\), soit \( \alpha\) une racine de multiplicité \( k\) de \( P\). La définition de la multiplicité d'une racine nous dit que \( P\) est divisible par \( (X-\alpha)^k\) mais pas par \( (X-\alpha)^{k+1}\). Donc il existe \( Q\in \eA[X]\) tel que \( P=Q(X-\alpha)^k\). Il reste à voir que \( Q(\alpha)\neq 0\). Cela est une conséquence de la proposition \ref{PropHSQooASRbeA} : si \( Q(\alpha)\) était nul, on pourrait lui factoriser \( (X-\alpha)\) et donc avoir \( (X-\alpha)^{k+1}\) qui se factorise dans \( P\), ce qui n'est pas possible.

    Nous supposons que \( p\geq 2\) et nous effectuons une récurrence sur \( p\). Nous considérons donc les \( p-1\) premières racines \( \alpha_1,\ldots, \alpha_{p-1}\) et un polynôme \( R\in\eA[X]\) tel que \( R(\alpha_i)\neq 0\) pour \( i=1,\ldots, p-1\) et
    \begin{equation}
        P=\underbrace{(X-\alpha_1)^{k_1}\ldots (X-\alpha_{p-1})^{k_{p-1}}}_SR.
    \end{equation}
    Par hypothèse \( P(\alpha_p)=S(\alpha_p)R(\alpha_p)=0\). L'anneau \( \eA\) étant intègre, \( S(\alpha_p)\neq 0\) parce que \( \alpha_i\neq \alpha_p\) pour \( i\neq p\). Par conséquent, \( R(\alpha_p)=0\).
    
    Nous devons encore vérifier que la multiplicité \( \alpha_p\) est \( k_p\) par rapport à \( R\). Pour cela nous utilisons le point \ref{ItemIeLhpciv} du lemme \ref{LemIeLhpc} afin de dire que le degré de \( \alpha_p\) pour \( P=SR\) est \( k_p\). Par conséquent
    \begin{equation}
        R=(X-\alpha_p)^{k_p}T
    \end{equation}
    avec \( T(\alpha_p)\neq 0\) et enfin
    \begin{equation}
        P=\prod_{i=1}^p(X-\alpha_i)T.
    \end{equation}
    De plus \( T(\alpha_i)\neq 0\), sinon \( R(\alpha_i)\) serait nul.
\end{proof}

\begin{proposition}[\wikipedia{fr}{Critère_d'Eisenstein}{Critère d'Eisenstein}]
    Soit le polynôme \( P(X)=\sum_{k=0}^n a_nX^n\) dans \( \eZ[X]\). Nous supposons avoir un nombre premier \( p\) tel que
    \begin{enumerate}
        \item
            \( p\) divise tous les \( a_0,\ldots, a_{n-1}\),
        \item
            \( p\) ne divise pas \( a_n\),
        \item
            \( p^2\) ne divise pas \( a_0\).
    \end{enumerate}
    Alors \( P\) est irréductible dans \( \eQ[X]\).

    Si de plus \( P\) est primitif au sens du \( \pgcd\) alors \( P\) est irréductible dans \( \eZ[X]\).
\end{proposition}

\begin{proof}
    Nous considérons \( \bar P\) le polynôme réduit modulo \( p\), c'est à dire \( \bar P\in \eF_p[X]\). Étant donné que par hypothèse tous les coefficients sont multiples de \( p\) sauf \( a_n\), nous avons \( \bar P=cX^n\). Supposons par l'absurde que \( P=QR\) avec \( Q,R\in \eQ[X]\). Alors le lemme de Gauss (\ref{LemSdnZNX}) impose \( P,Q\in \eZ[X]\).

    Nous avons aussi, au niveau des réductions modulo \( p\) que $\bar Q\bar R=\bar P$. Or \( \bar P\) est un monôme, donc \( \bar Q\) et \( \bar R\) doivent également l'être. Donc \( \bar Q=dX^k\) et \( \bar R=eX^{n-k}\) et en particulier \( \bar Q(0)=\bar R(0)=0\), c'est à dire que \( Q(0)\) et \( R(0)\) sont divisibles par \( p\). Cela impliquerait que \( a_0=Q(0)R(0)\) soit divisible par \( p^2\), ce qui est exclu par les hypothèses. Donc \( P\) est irréductible.

    Supposons de surcroît que \( P\) est primitif au sens du \( \pgcd\). Il est donc irréductible et primitif sur \( \eQ[X]\) et une conséquence du lemme de Gauss (\ref{ThofiIpXg}) nous dit alors que \( P\) est irréductible sur \( \eZ[X]\).
\end{proof}

\begin{example}
    Soit le polynôme \( P(X)=3X^4+15 X^2+10\). Pour faire fonctionner le critère d'Eisenstein il nous faut un nombre premier \( p\) divisant \( 15\) et \( 10\), mais pas \( 3\) et dont le carré ne divise pas \( 10\). C'est vite vu que \( p=5\) fait l'affaire. Le polynôme \( P\) est donc irréductible sur \( \eQ[X]\).
\end{example}

%--------------------------------------------------------------------------------------------------------------------------- 
\subsection{Quelque identités}
%---------------------------------------------------------------------------------------------------------------------------

\begin{lemma}   \label{LemISPooHIKJBU}
    Quelques identités de polynômes.
    \begin{enumerate}
        \item   \label{ItemLTBooAcyMtN}
            Si \( n\) est impair, alors \( 1+X\) divise \( 1+X^n\).
        \item\label{ItemLTBooAcyMtNii}
            Pour tout \( n\) nous avons \( X^n-1=(X-1)(1+X+\ldots +X^{n-1})\).
        \item
            \( z^n-z_0^n=(z-z_0)\sum_{i=0}^{n-1}z^iz_{0}^{n-1-i}\).
    \end{enumerate}
\end{lemma}

\begin{proof}
    Le cas \( n=1\) est évident. Procédons alors par récurrence en considérant un nombre entier impair \( n\) :
    \begin{subequations}
        \begin{align}
            1+X^{n+2}&=1+X^n+X^{n+2}-X^n\\
                    &=(1+X)P+X^n(X^2-1)\\
                    &=(1+X)P+X^n(X+1)(X-1)\\
                    &=(1+X)\big( P+X^n(X-1) \big).
        \end{align}
    \end{subequations}
\end{proof}
