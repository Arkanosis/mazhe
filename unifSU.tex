%%%%%%%%%%%%%%%%%%%%%%%%%%
%
   \section{One dimensional split extension of Heisenberg algebras} \label{SecExtHeiz}
%
%%%%%%%%%%%%%%%%%%%%%%%%

\subsection{Introduction}
%------------------------

The one dimensional extensions of Heisenberg algebras are classified by triples $(\BX,\mu,d)$. The quantization in the case $(\id,0,\mu)$  reveals to be a particular case of the one studied in \cite{Biel-Massar}. The kernel of this quantization will be denoted by $K$. 

Then we will give a way to twist $K$ in order to obtain a kernel $K'$ on any extension of the form $(\BX,0,2)$. Quantizations of other extensions can be obtained by composing with Lie group isomorphisms. The kernel for an arbitrary extension is denoted by $K_{0}(\BX,\mu,d)$, or simply $K_{0}$ when there are no possible ambiguity.

When we will deal with the anti de Sitter situation, our starting point will be this $K_{0}$ that we will have to adapt to another symplectic form that $E^*$. 

\subsection{Reminder about a previous deformation} 
%--------------------------------------------------

Before to go on with the construction of a deformation of one dimensional split extensions of Heisenberg algebras, we mention a result due to Bieliavsky and Massar on deformation in $\SU(1,n)$. The product on the extension of Heisenberg algebra will be nothing but a transport of this one.

In the article \cite{Biel-Massar}, a formal universal deformation formula for the actions of the Iwasawa component $ R_0:= A_0\SUN$ of $\SU(1,n)$ is given in oscillatory integral form.  It has been observed in \cite{lcBBM} that this type of universal deformation formula is actually non-formal for proper actions on topological spaces. The precise framework  and statement are as follows.  Through the natural identification $ R_0=\SU(1,n)/U(n)$ induced by the Iwasawa decomposition of $\SU(1,n)$, the group $ R_0$ is endowed with a (family of) left invariant symplectic structure(s) $\omega$.  Denoting by $\sR_0:=sA_0\times\sN_0$ its Lie algebra, the map
\begin{equation} \label{DARBOUX}
	\sR_0\longrightarrow  R_0:(a,n)\mapsto\exp(a)\exp(n)
\end{equation}
turns out to be a global Darboux chart on $( R_0,\omega)$.  Setting $\lN_0=V\times\eR Z$ with table 
\[ 
[(x,z)\,,\,(x',z')]=\Omega_V(x,x')\,Z, 
\]
and  $\sR_0=\{(a,x,z)\,|\,,a,z\in\eR;x\in V\}$, one has the following result:

\begin{probleme}
Il faudra je crois unifier partout la notation $R_0$ en $\SUR_0$.
\end{probleme}


\begin{theorem}
For all non-zero $\theta\in\eR$, there exists a Fréchet function space $\swE_\theta$, $C^\infty_c( R_0)\subset\swE_\theta\subset C^\infty( R_0)$, such that, defining for all $u,v\in C^\infty_c( R_0)$  
\begin{equation}  \label{PRODUCT}
\begin{split}
(u\star_\theta v)(a_0,x_0,z_0)
		:=\frac{1}{\theta^{\dim  R_0}} \int_{  R_0\times  R_0}& \cosh(2(a_1-a_2))\,[\cosh(a_2- a_0)\cosh(a_0-a_1)\,]^{\dim  R_0-2}\\
&\times \exp \Big( \frac{2i}{\theta}\varphi(r_1,r_2,r_3)\Big)\\
		&\times u(a_1,x_1,z_1)\,v(a_2,x_2,z_2)\, da_1da_2dx_1dx_2dz_1dz_2;
\end{split}
\end{equation}
where 
\[ 
  \varphi(r_1,r_2,r_3)=S_V\big(\cosh(a_1-a_2)x_0, \cosh(a_2-a_0)x_1, \cosh(a_0-a_1)x_2\big) \bigoplus_{0,1,2}\sinh(2(a_0-a_1))z_2 
\]
with $S_V(x_0,x_1,x_2):=\Omega_V(x_0,x_1)+\Omega_V(x_1,x_2)+\Omega_V(x_2,x_0)$ is the phase for the Weyl product on $C^\infty_c(V)$ and $\bigoplus_{0,1,2}$ stands for cyclic summation\footnote{In \cite{Biel-Massar}, the exponent $\dim  R_0-2$ was forgotten in the expression of the amplitude of the oscillating kernel.}, one has: 

\begin{enumerate}

\item\label{tBMi} 
	 $u\star_\theta v$ is smooth and the map $ C^\infty_c( R_0)\times C^\infty_c( R_0) \to C^\infty( R_0)$ extends to an associative product on $\swE_\theta$. The pair $(\swE_\theta,\star_\theta)$ is a (pre-$C^\star$) Fréchet algebra.

\item\label{tBMii}
	 In coordinates $(a,x,z)$ the group multiplication law reads
\[ 
	L_{(a,x,z)}(a',x',z')=\left(a+a',e^{-a'}x+x',e^{-2a'}z+z'+\frac{1}{2}\Omega_V(x,x')e^{-a'}
\right).
\]
The phase and amplitude occurring in formula \eqref{PRODUCT} are both invariant under the left action $L: R_0\times  R_0\to  R_0$.

\item\label{tBMiii}
	 Formula \eqref{PRODUCT} admits a formal asymptotic expansion of the form:
 \begin{equation*}
	u\star_\theta v\sim \,uv\,+\,\frac{\theta}{2i}\{u,v\}\,+O(\theta^2); 
\end{equation*}
where $\{\,,\,\}$ denotes the symplectic Poisson bracket on $C^\infty( R_0)$ associated with $\omega$.  The full series yields an associative formal star product on $( R_0,\omega)$ denoted by $\tilde{\star}_\theta$. 
 \end{enumerate}
\label{BIMAS}
\end{theorem}

The setting and \ref{tBMi} and \ref{tBMii} may be found in \cite{Biel-Massar}, while \ref{tBMiii} is a straightforward adaptation  to $ R_0$ of \cite{lcBBM}.

