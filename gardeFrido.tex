% This is part of Mes notes de mathématique
% Copyright (c) 2011-2016
%   Laurent Claessens
% See the file fdl-1.3.txt for copying conditions.


% Ce fichier est inclu à 'gardeVolume' pour être compilé séparément
% pour la division en volumes.
% Il est donc inutile de faire des choses ici lorsque 'isBook' est true. 
% En effet 'isBook' sert à créer la version qui va être divisée en
% volumes, et pour cette division, une recompilation 
% spéciale de ce fichier sera faite avec 'isEnVolume' mis à 'true'.


\thispagestyle{empty}
\begin{center}
  \begin{minipage}{15cm}
    \hrule\par
    \vspace{2mm}
    \begin{center}
        \Huge \bfseries Le Frido \\  {\small Les quelques premières années de mathématiques}
    \normalsize
    \ifbool{isAgreg}{%
        \url{http://laurent.claessens-donadello.eu/pdf/lefrido.pdf}
    }{}%
    \ifbool{isEnVolume}{%
        \newline
        \url{http://laurent.claessens-donadello.eu/pdf/préliminaire-vol1.pdf}\\
        \url{http://laurent.claessens-donadello.eu/pdf/préliminaire-vol2.pdf}\\
        \url{http://laurent.claessens-donadello.eu/pdf/préliminaire-vol3.pdf}\\
        Versions préliminaires.
    }{}%
    \end{center}
    \hrule\par
  \end{minipage}\\
  \vspace{0.2cm}
  \ifbool{isEnVolume}{VOLUME \arabic{envolume}}{}
\end{center}

\vspace{2cm}

\begin{center}
    Laurent \textsc{Claessens}\\
    \notbool{isEnVolume}{\today}{2016}\\
    \notbool{isEnVolume}{ \texttt{\GitCommitHexsha} }{}

    \vspace{1cm}

\end{center}

\vfill

 \LogoEtLicence


% Créer une nouvelle branche git
% Copier tout dans un nouveau répertoire
% Supprimer la mise à jour automatique dans le scipt de mise en ligne.
% Supprimer le «Une version de ces notes est disponible dans la bibliothèque de l'agrégation» 
% Ajouter ici l'ISBN. Pour les révisions, mettre un nouvel ISBN et indiquer que c'est une révision.
% Pour l'ISBN:
% Coder en dur la date (càd enlever \today)
% Comme c'est pour imprimer, regarder si c'est pas mieux d'enlever l'option ``oneside'' de la classe book.
% Pour l'URL donnée juste en dessous du titre, ajouter l'année : mes_notes.pdf --> mes_notes-2015.pdf
% supprimer la liste des développements possibles.

% http://www.bnf.fr/fr/professionnels/s_informer_obtenir_isbn/s.qu_est_ce_que_isbn.html

% Il faut écrire l'ISBN au verso de la page de titre, au bas de la dernière page de couverture et au bas de la dernière page de la jaquette des livres ;

% Imprimer quelques pages d'essai pour voir si les couleurs des liens passent bien. En particulier :
% - équation
% - théorèmes
% - notes en bas de page
% - bibliographie
% - URL
% - href
% Il y a le fichier test_couleur pour ça.


% De temps en temps, il faut renvoyer une nouvelle version à
% http://megamaths.perso.neuf.fr/

%\clearpage

% Pour avoir l'ISBN en dos de couverture, ajouter ceci :
%\clearpage
%\vfill
%\LogoEtLicence
%\clearpage


