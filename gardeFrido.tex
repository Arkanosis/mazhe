% This is part of Mes notes de mathématique
% Copyright (c) 2011-2015
%   Laurent Claessens
% See the file fdl-1.3.txt for copying conditions.




\thispagestyle{empty}
\begin{center}
  \begin{minipage}{15cm}
    \hrule\par
    \vspace{2mm}
    \begin{center}
        \Huge \bfseries Le Frido \\  {\small Les quelque premières années de mathématique }     % Ici il y avait \par
    \normalsize
    \url{http://student.ulb.ac.be/~lclaesse/lefrido.pdf}
    \end{center}
    \hrule\par
  \end{minipage}\\
  \vspace{0.2cm}
  \ifnumequal{\value{envolume}}{0}{}{VOLUME \arabic{envolume}}
\end{center}

\vspace{2cm}

\begin{center}
    Laurent \textsc{Claessens}\\
    \today

    \vspace{1cm}

\end{center}

\vfill

  \ifnumequal{\value{envolume}}{0}{\LogoEtLicence}{\LicenceFDL}


% Créer une nouvelle branche git
% Copier tout dans un nouveau répertoire
% Supprimer la mise à jour automatique dans le scipt de mise en ligne.
% Supprimer le «Une version de ces notes est disponible dans la bibliothèque de l'agrégation» 
% Ajouter ici l'ISBN. Pour les révisions, mettre un nouvel ISBN et indiquer que c'est une révision.
% Pour l'ISBN:
% Coder en dur la date (càd enlever \today)
% Comme c'est pour imprimer, regarder si c'est pas mieux d'enlever l'option ``oneside'' de la classe book.
% Pour l'URL donnée juste en dessous du titre, ajouter l'année : mes_notes.pdf --> mes_notes-2015.pdf
% supprimer la liste des développements possibles.

% http://www.bnf.fr/fr/professionnels/s_informer_obtenir_isbn/s.qu_est_ce_que_isbn.html

% Il faut écrire l'ISBN au verso de la page de titre, au bas de la dernière page de couverture et au bas de la dernière page de la jaquette des livres ;

% Imprimer quelque pages d'essai pour voir si les couleurs des liens passent bien. En particulier :
% - équation
% - théorèmes
% - notes en bas de page
% - bibliographie
% - URL
% - href
% Il y a le fichier test_couleur pour ça.


% De temps en temps, il faut renvoyer une nouvelle version à
% http://megamaths.perso.neuf.fr/

%\clearpage

% Pour avoir l'ISBN en dos de couverture, ajouter ceci :
%\clearpage
%\vfill
%\LogoEtLicence
%\clearpage


% et supprimer ceci
\clearpage

\thispagestyle{empty}

Ce document existe en plusieurs versions.
\begin{itemize}
    \item
        La version de décembre \( 2013\) :
        \begin{center}
            \url{http://student.ulb.ac.be/~lclaesse/mes_notes-2013.pdf}
        \end{center}
        C'est cette version que vous devriez utiliser si vous comptez passer l'agrégation en $2014$.
        
        Le règlement de l'agrégation vous permet sans ambigüités\footnote{\url{http://agreg.org/Pratique/bibliotheque.html}} d'en apporter une copie, et vous n'avez pas d'autorisation à demander\footnote{\url{http://agreg.org/Pratique/faq.html}}. Cependant le jury m'a fait savoir par un courrier personnel (rien d'officiel donc; vous me croyez ou non) qu'ils n'accepteraient pas que des candidats amènent des versions de ce document imprimés par leurs soins. M'est avis qu'ils ne refuseront pas que vous en mettiez dans la malle\ldots Faites moi savoir la réponse du jury si vous leur posez la question.

    \item 
        La version préliminaire de celle qui aura un ISBN en décembre $2014$ :
        \begin{center}
        \url{http://student.ulb.ac.be/~lclaesse/mes_notes-2014.pdf}
        \end{center}

        C'est cette version que vous devriez utiliser si vous comptez passer l'agrégation en $2015$, sous les mêmes conditions que celles expliquées plus haut.
        
    \item

        La version \( 2012\) :  
        \begin{center}
        \url{http://student.ulb.ac.be/~lclaesse/mes_notes-2012.pdf}
        \end{center}

        Elle est disponible en \( 4\) exemplaires dans la bibliothèque de l'agrégation à Paris. D'où le fait que je croie qu'envoyer des versions plus récentes ne devrait pas poser de problèmes.

    \item

        Une version plus complète, comprenant à la fois de matières plus avancées et des exercices moins avancés : 
        \begin{center}
        \url{http://student.ulb.ac.be/~lclaesse/mazhe.pdf}
        \end{center}

        C'est la version que vous devriez utiliser si l'agrégation ne vous intéresse pas.

\end{itemize}


\vfill

\LogoEtLicence
\clearpage

