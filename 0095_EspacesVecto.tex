% This is part of Mes notes de mathématique
% Copyright (c) 2011-2015
%   Laurent Claessens, Carlotta Donadello
% See the file fdl-1.3.txt for copying conditions.

%+++++++++++++++++++++++++++++++++++++++++++++++++++++++++++++++++++++++++++++++++++++++++++++++++++++++++++++++++++++++++++
\section{Matrice compagnon et endomorphismes cycliques}
%+++++++++++++++++++++++++++++++++++++++++++++++++++++++++++++++++++++++++++++++++++++++++++++++++++++++++++++++++++++++++++

%---------------------------------------------------------------------------------------------------------------------------
\subsection{Matrice compagnon}
%---------------------------------------------------------------------------------------------------------------------------

Soit le polynôme \( P=X^n-a_{n-1}X^{n-1}-\ldots-a_1X-a_0\) dans \( \eK[X]\). La \defe{matrice compagnon}{matrice!compagnon} de \( P\) est la matrice\nomenclature[A]{\( C(P)\)}{matrice compagnon} donnée par
\begin{equation}
    C(P)=\begin{pmatrix}
        0    &   \cdots    &   \cdots    &   0    &   a_0\\  
        1    &   0    &       &   \vdots    &   a_1\\  
        0    &   \ddots    &   \ddots    &   \vdots    &   \vdots\\  
        \vdots    &   \ddots    &   \ddots    &   0    &   a_{n-2}\\  
        0    &   \cdots    &   0    &   1    &   a_{n-1}    
    \end{pmatrix}
\end{equation}
si \( n\geq 2\) et par \( (a_0)\) si \( n=1\). Si \( f\) est l'endomorphisme associé à la matrice \( C(P)\) nous avons
\begin{equation}
    f(e_i)=\begin{cases}
        e_{i+1}    &   \text{si \( i<n\)}\\
        (a_0,\ldots, a_{n-1})    &    \text{si \( i=n\)}.
    \end{cases}
\end{equation}
Cet endomorphisme est conçu pour vérifier \( P(f)e_1=0\).

\begin{lemma}[\cite{RapportArgreg2011}] \label{LemkVNisk}
    Un polynôme sur un corps commutatif est le polynôme caractéristique de sa matrice compagnon. En d'autres termes nous avons \( \chi_{C(P)}=P\).
\end{lemma}

\begin{proof}
    Nous notons \( f\) l'endomorphisme associé à \( C(P)\). La propriété \( P(f)e_1=0\) nous indique que le polynôme minimal ponctuel de \( f\) en \( e_1\) divise \( P\). L'ensemble des puissances de \( f\) appliquées à \( e_1\), \( \big( f^i(e_1) \big)_{i=1,\ldots, n-1}\) est libre, donc le polynôme minimal ponctuel en \( e_1\) est de degré \( n\) au minimum. En reprenant les notations du théorème \ref{ThoCCHkoU}, nous avons \( I_{e_1}=(P)\) parce que \( P\) est de degré minimum dans \( I_{e_1}\) et \( \chi_f\in I_{e_1}\).

    Donc \( P\) divise \( \chi_f\) et est de degré égal à celui de \( \chi_f\). Étant donné qu'ils sont tous deux unitaires, ils sont égaux.
\end{proof}

\begin{remark}  \label{RemmQjZOA}
    Les matrices compagnons ne sont pas les seules dont le polynôme caractéristique est égal au polynôme minimal. En fait les matrices dont le polynôme caractéristique est égale au polynôme minimal sont denses dans les matrices. En effet une matrice dont le polynôme minimal n'est pas égal au polynôme caractéristique a un polynôme caractéristique avec une racine double. Il est possible, en modifiant arbitrairement peu la matrice de séparer la racine double en deux racines distinctes.
\end{remark}

\begin{definition}[Matrices, endomorphismes et vecteurs cycliques]
    Une matrice est \defe{cyclique}{cyclique!matrice}\index{matrice!cyclique} si elle est semblable à une matrice compagnon. Un endomorphisme \( f\colon E\to E\) est \defe{cyclique}{cyclique!endomorphisme}\index{endomorphisme!cyclique} si il existe un vecteur \( x\in E\) tel que \( \{ f^k(x)\tq k=1,\ldots, n-1 \}\) est une base de \( E\). Un vecteur ayant cette propriété est un \defe{vecteur cyclique}{vecteur!cyclique} pour \( f\).
\end{definition}

\begin{lemma}\label{LemSGmdnE}
    Si \( A\) est la matrice de l'endomorphisme \( f\) alors nous avons équivalence des propriétés suivantes :
    \begin{enumerate}
        \item
            La matrice \( A\) est cyclique.
        \item
            L'endomorphisme \( f\) est cyclique.
        \item
            Le polynôme caractéristique de \( A\) est égal à son polynôme caractéristique.
    \end{enumerate}
\end{lemma}

\begin{lemma}   \label{LemAGZNNa}
    Si \( f\colon E\to E\) est un endomorphisme cyclique et si \( y\) est un vecteur cyclique de \( f\), alors le polynôme minimal de \( f\) est égal au polynôme minimal de \( f\) au point \( y\) : \( \mu_{f}=\mu_{f,y}\).
\end{lemma}

\begin{proof}
    Montrons que \( \mu_{f,y}\) est un polynôme annulateur de \( f\), ce qui prouvera que \( \mu(f)\) divise \( \mu_{f,y}\). Étant donné que \( y\) est cyclique, tout élément de \( E\) s'écrit sous la forme \( x=Q(f)y\). Prenons un polynôme \( P\) annulateur de \( f\) en \( y\) : \( P(f)y=0\). Nous montrons que \( P\) est alors un polynôme annulateur de \( f\). En effet, nous avons
    \begin{equation}
        P(f)x=\big( P(f)\circ Q(f) \big)y=\big( Q(f)\circ P(f) \big)y=0
    \end{equation}
    où nous avons utilisé le lemme \ref{LemQWvhYb}.
\end{proof}

%---------------------------------------------------------------------------------------------------------------------------
\subsection{Réduction de Frobenius}
%---------------------------------------------------------------------------------------------------------------------------

\begin{theorem}[Réduction de Frobenius \cite{AutourFrobCompa,Vialivs,MoncetIVS}]      \index{réduction!Frobénius}\index{Frobénius!réduction}
    Soit \( E\), un \( \eK\)-espace vectoriel où \( \eK\) est \( \eR\) ou \( \eC\), et \( f\in \End(E)\). Alors il existe une suite de sous-espaces \( E_1,\ldots, E_r\) stables par \( f\) tels que
    \begin{enumerate}
        \item   \label{ItemmpwjnSs}
            \( E=\bigoplus_{i=1}^rE_i\);
        \item
            pour chaque \( E_i\), l'endomorphisme restreint \( f_i=f|_{E_i}\) est cyclique;
        \item
            si \( \mu_i\) est le polynôme minimal de \( f_i\) alors \( \mu_{i+1}\) divise \( \mu_i\);
    \end{enumerate}
    Une telle décomposition vérifie automatiquement \( \mu_1=\mu_f\) et \( \mu_1\cdots \mu_r=\chi_f\), et la suite \( (\mu_i)_{i=1,\ldots, r}\) ne dépend que de \( f\) et non du choix de la décomposition du point \ref{ItemmpwjnSs}.
\end{theorem}

Les polynômes \( \mu_i\) sont les \defe{invariants de similitude}{invariant!de similitude} de l'endomorphisme \( f\).

\begin{proof}
    Nous commençons par montrer que si une telle décomposition existe, alors
    \begin{subequations}    \label{subEqzcGouz}
        \begin{align}
            \chi_f=\prod_{i=1}^r\mu_i  \label{EqTaxsvb}\\
            \mu_f=\mu_1
        \end{align}
    \end{subequations}
    où \( \chi_f\) est le polynôme caractéristique de \( f\) et \( \mu_f\) est le polynôme minimal. D'abord le polynôme caractéristique de \( f\) devra être égal au produit des polynômes caractéristique des \( f|_{E_i}\), mais ces derniers endomorphismes étant cycliques, leurs polynôme caractéristiques sont égaux à leurs polynômes minimaux (lemme \ref{LemSGmdnE}). Cela prouve l'égalité \eqref{EqTaxsvb}. Ensuite tous les \( \mu_i\) doivent diviser le polynôme minimal, donc \( \ppcm(\mu_1,\ldots, \mu_r)\) divise \(\mu_f\). Cependant le polynôme minimal doit contenir une et une seule fois chacun des facteurs irréductibles du polynôme caractéristique, et chacun de ces facteurs sont dans les polynômes \( \mu_i\). Par conséquent \( \ppcm(\mu_1,\ldots, \mu_r)=\mu_f\). Mais par ailleurs \( \mu_1=\ppcm(\mu_1,\ldots, \mu_r)\) parce qu'on a supposé \( \mu_{i+1}\divides \mu_i\), donc \( \mu_1=\mu_f\).
    
    Soit \( d\), le degré du polynôme minimal de \( f\) et \( y\in E\) tel que \( \mu_f=\mu_{f,y}\) (voir lemme \ref{LemSYsJJj}). Le plus petit espace stable sous \( f\) contenant \( y\) est
    \begin{equation}
        E_y=\Span\{ y,f(y),\ldots, f^{d-1}(y) \}.
    \end{equation}
    Nous notons \( e_i=f^{i-1}(y)\). Notons que les vecteurs donnés forment bien une base de \( E_y\) parce que si les \( e_i\) n'était pas linéairement indépendants, alors nous aurions des \( a_k\) tels que \( \sum_ka_ke_k=0\) et avec lesquels
    \begin{equation}
        \big( \sum_ka_kX^k \big)(f)y=0,
    \end{equation}
    ce qui contredirait la minimalité de \( \mu_{f,y}\).

    La difficulté du théorème est de trouver un complément de \( E_y\) qui soit également stable sous \( f\). Nous commençons par étendre\quext{Pour autant que j'aie compris, cette extension manque dans \cite{AutourFrobCompa}. Corrigez moi si je me trompe.} \( \{ e_1,\ldots, e_d \}\) en une base \( \{ e_1,\ldots, e_n \}\) de \( E\). Ensuite nous allons montrer que
    \begin{equation}
        E=E_y\oplus F
    \end{equation}
    avec
    \begin{equation}
        F=\{ x\in E\tq  e^*_d\big( f^k(x) \big)=0\forall k\in \eN \}.
    \end{equation}
    Par construction, \( F\) est invariant sous \( f\). Montrons pour commencer que \( E_y\cap F=\{ 0 \}\). Un élément de \( E_y\) s'écrit
    \begin{equation}
        z=a_1e_1+\ldots +a_ke_k
    \end{equation}
    avec \( k\leq d\). Étant donné que \( f\) décale les vecteurs de base, nous avons \( e^*_d\big( f^{d-k}(z) \big)=a_k\). Du coup \( z\in F\) si et seulement si \( a_1=\ldots=a_d=0\), c'est à dire que \( E_y\cap F=\{ 0 \}\).

    Nous montrons maintenant que \( \dim F=n-d\). Pour cela nous considérons l'application
    \begin{equation}
        \begin{aligned}
            T\colon \eK[F]&\to E^* \\
            g&\mapsto e^*_d\circ g. 
        \end{aligned}
    \end{equation}
    Cette application est injective. En effet un élément général de \( \eK[f]\) est
    \begin{equation}
        g=a_1\id+a_2f+\ldots +a_pf^{p-1}
    \end{equation}
    avec \( p\leq d\). Si \( T(g)=0\), alors nous avons en particulier
    \begin{equation}
        0=T(g)e_{_d-p+1}=e^*_d(a_1e_{d-p+1}+a_2e_{d-p+2}+\ldots +a_pe_d)=a_p.
    \end{equation}
    Donc \( a_p=0\) et en appliquant maintenant \( T(g)\) à \( e_{d-p}\) nous obtenons \( a_{p-1}=0\). Au final nous trouvons que \( g=0\) et donc que \( T\) est injective.

    Étant donné que \( \dim\eK[f]=d\) et que \( T\) est injective, \( \dim\Image(T)=d\). Nous regardons l'orthogonal de l'image :
    \begin{subequations}
        \begin{align}
            (\Image(T))^{\perp}&=\{ x\in E\tq T(g)x=0\forall g\in\eK[f] \}\\
            &=\{ x\in E\tq e^*_d\big( g(x) \big)=0\forall g\in \eK[f] \}\\
            &=F.
        \end{align}
    \end{subequations}
    Par conséquent \( F^{\perp}=\Image(T)\). Vu que \( \dim\Image(T)=d\), nous avons donc \( \dim F=n-d\) et il est établi que \( E=E_y\oplus F\). 

    Nous avons donc trouvé \( F\), stable par \( f\) et tel que \( E=E_y\oplus F\). Nous devons maintenant nous assurer que cette décomposition tombe bien pour les polynômes minimaux. Si \( P_1\) est le polynôme minimal de \( f|_{E_yj}\), alors par le lemme \ref{LemAGZNNa} nous avons \( P_1=\mu_{f,y}=\mu_f\) parce que \( f|_{E_y}\) est cyclique sur \( E_y\). Mettons \( P_2\), le polynôme minimal de \( f|_F\). Étant attendu que \( F\) est stable par \( f\), le polynôme \( P_2\) divise \( P_1\). En recommençant la construction sur \( F\), nous construisons un nouvel espace \( F'\) stable sous \( F\) et vérifiant \( \mu_{f|_{F'}}=P_2\), etc.

    Nous passons maintenant à la partie unicité du théorème. Soient deux suites \( F_1,\ldots, F_r\) et \( G_1,\ldots, G_s\) de sous-espaces stables par \( f\) et vérifiant
    \begin{enumerate}
        \item
            \( E=\bigoplus_{i=1}^rF_i\),
        \item
            \( f|_{F_i}\) est cyclique,
        \item
            \( \mu_{f|_{F_{i+1}}}\) divise \( \mu_{f|_{F_i}}\),
    \end{enumerate}
    et, \emph{mutatis mutandis}, les mêmes conditions pour la famille \( \{ G_i \}\). Nous posons \( P_i=\mu_{f_{F_i}}\) et \( Q_i=\mu_{f|_{G_i}}\). Nous allons montrer par récurrence que \( P_i=Q_i\) et \( \dim F_i=\dim G_i\). Il ne sera cependant pas garanti que \( F_i=G_i\). D'abord, \( P_1=Q_1\) parce qu'ils sont tous deux égaux à \( \mu_f\) par les relations \eqref{subEqzcGouz}. Nous supposons que \( P_i=Q_i\) pour \( i\leq 1\leq j-1\) et nous tentons de montrer que \( P_j=Q_j\).

    Nous avons 
    \begin{equation}    \label{EqMrCtZO}
        P_j(f)=P_j(f)|_{F_1}\oplus\ldots\oplus P_j(f)|_{F_{j-1}}.
    \end{equation}
    En effet étant donné que \( P_{j+k}\) divise \( P_j\), nous avons\footnote{En vertu du lemme \ref{LemQWvhYb}.} \( P_{j}(f)=A(f)\circ P_{j+k}(f)\), mais \( P_{j+k}(f)F_{j+k}=0\), donc \( P_j(f)F_{j+k}=0\). Les espaces \( G_i\) n'ayant a priori aucun rapport avec les polynômes \( P_i\), nous écrivons
    \begin{equation}    \label{EqJreLiO}
        P_j(f)=P_j(f)|_{G_1}\oplus\ldots\oplus P_j(f)|_{G_{j-1}}\oplus P_j(f)|_{G_j}\oplus\ldots\oplus P_j(f)|_{G_s}.
    \end{equation}
    Pour \( 1\leq i\leq j-1\), nous avons supposé \( P_i=Q_i\). Étant donné que \( f|_{F_i}\) est semblable à \( C_{_i}\) et \( f|_{G_i}\) est semblable à \( C_{Q_i}\), la matrice de \( f|_{E_i}\) est semblable à la matrice de \( f|_{G_i}\). En particulier,
    \begin{equation}
        \dim P_j(f)F_i=\dim P_j(f)G_i.
    \end{equation}
    En prenant les dimensions des images dans les égalités \eqref{EqMrCtZO} et \eqref{EqJreLiO}, nous trouvons que
    \begin{equation}
        P_j(f)|_{G_j}=\ldots=P_j(f)|_{G_s}=0.
    \end{equation}
    Par conséquent \( P_j\in I_{f|G_j}\) et donc \( P_j\) divise \( Q_j\), qui est générateur de \( I_{f|_{G_j}}\). La situation étant symétrique entre \( P\) et \( Q\), nous montrons de même que \( Q_j\) divise \( P_j\) et donc que \( P_j=Q_j\).

    Ceci achève la démonstration du théorème de réduction de Frobenius.

\end{proof}


Sous forme matricielle, ce théorème dit que toute matrice est semblable à une matrice de la forme bloc-diagonale
\begin{equation}
    f=\begin{pmatrix}
        C_{\mu_1}    &       &       \\
            &   \ddots    &       \\
            &       &   C_{\mu_r}
    \end{pmatrix}
\end{equation}

\begin{remark}
    Si nous travaillons sur \( \eR\), la réduite de Frobenius restera une matrice réelle, même si les valeurs propres sont complexes. En effet le procédé de Frobenius ne regarde absolument pas les valeurs propres, mais seulement les facteurs irréductibles du polynôme caractéristique. La réduite de Frobenius ne tente pas de résoudre ces polynômes, mais se contente d'en utiliser les matrices compagnon.

    La situation sera différente dans le cas de la forme normale de Jordan.
\end{remark}

%---------------------------------------------------------------------------------------------------------------------------
\subsection{Forme normale de Jordan}
%---------------------------------------------------------------------------------------------------------------------------

Il existe une preuve directe de la réduction de Jordan ne nécessitant pas la réduction de Frobenius\cite{LecLinAlgAllen}. Cette dernière passe par les espaces caractéristiques\footnote{Aussi appelés «espaces propres généralisés».} et est à mon avis plus compliquée que la démonstration de Frobenius elle-même. Nous allons donc nous contenter de donner la réduction de Jordan comme un cas particulier de Frobenius.

\begin{theorem}[Réduction de Jordan]        \label{ThoGGMYooPzMVpe}
    Soit \( E\) un espace vectoriel sur \( \eK\), et \( f\in\End(E)\) un endomorphisme dont le polynôme caractéristique \( \chi_f\) est scindé\footnote{C'est pour cette hypothèse que \( \eK=\eR\) n'est pas le bon cadre.}. Il existe une base de \( E\) dans laquelle la matrice de \( f\) s'écrit sous la forme
    \begin{equation}
        M=\begin{pmatrix}
            J_{n_1}(\lambda_1)    &       &       \\
                &   \ddots    &       \\
                &       &   J_{n_k}(\lambda_k)
        \end{pmatrix}
    \end{equation}
    où les \( \lambda_i\) sont les valeurs propres de \( f\) (avec éventuelle répétitions) et \( J_n(\lambda)\) représente le bloc \( n\times n\)
    \begin{equation}
        J_n(\lambda)=\begin{pmatrix}
            \lambda    &   1    &       &       &   \\  
                &   \lambda    &   1    &       &   \\  
                &       &   \lambda    &       &   \\  
                &       &       &   \ddots    &   1\\  
                &       &       &       &   \lambda    
        \end{pmatrix}.
    \end{equation}
    En d'autres termes, \( J_n(\lambda)_{ii}=\lambda\) et \( J_n(\lambda)_{i-1,i}=1\).    
\end{theorem}
\index{réduction!Jordan}
\index{Jordan!réduction}

\begin{proof}
    Nous commençons par le cas où \( f\) est nilpotente; nous notons \( M\) sa matrice. Dans ce cas la seule valeur propre est zéro et le polynôme caractéristique est \( X^m\) pour un certain \( m\). Nous savons par le lemme \ref{LemkVNisk} que (la matrice de) \( f\) est semblable à sa matrice compagnon. En l'occurrence pour \( f\) nous avons
    \begin{equation}
        C_{X^m}=\begin{pmatrix}
             0   &       &       &  0     \\
             1   &   \ddots    &       &   \vdots    \\
                &   \ddots    &   \ddots    &    \vdots   \\ 
                &       &   1    &   0     
         \end{pmatrix}.
    \end{equation}
    Ensuite le changement de base (qui est une similitude) \( (e_1,\ldots, e_n)\mapsto(e_n,\ldots, e_1)\) montre que \( C_{X^m}\) est semblable à un bloc de Jordan \( J_m(0)\).

    Supposons à présent que \( f\) ne soit pas nilpotente. Par l'hypothèse de polynôme caractéristique scindé, nous supposons que \( f\) a \( m\) valeurs propres distinctes et que son polynôme caractéristique est
    \begin{equation}
        \chi_f=(X-\lambda_1)^{l_1}\ldots (X-\lambda_m)^{l_m}.
    \end{equation}
    Le lemme des noyaux (théorème \ref{ThoDecompNoyayzzMWod}) nous enseigne que
    \begin{equation}
        E=\bigoplus_{i=1}^m\underbrace{\ker(f-\mu_i\mtu)^{l_i}}_{F_i}.
    \end{equation}
    La restriction de \( f-\lambda_i\mtu\) à \( F_i\) est par construction un endomorphisme nilpotent, et donc peut s'écrire comme un bloc de Jordan avec des zéros sur la diagonale. En utilisant la décomposition
    \begin{equation}
        f|_{F_i}=(f-\lambda_i\mtu)|_{F_i}+\lambda_i\mtu_{F_i},
    \end{equation}
    nous voyons que \( f|_{F_i}\) s'écrit comme un bloc de Jordan avec \( \lambda_i\) sur la diagonale.
\end{proof}

\begin{remark}
    Nous pouvons calculer la forme normale de Jordan pour une matrice complexe ou réelle, mais dans les deux cas nous devons nous attendre à obtenir une matrice complexe parce que les valeurs propres d'une matrice réelle peuvent être complexes. Cependant nous demandons que le polynôme caractéristique de \( f\) soit scindé sur \( \eK\). En pratique, la décomposition de Jordan n'est garantie que sur les corps algébriquement clos, c'est à dire sur \( \eC\).

    La suite des invariants de similitude sur laquelle repose Frobenius, elle, est disponible sur tout corps, y compris \( \eR\).
\end{remark}

%+++++++++++++++++++++++++++++++++++++++++++++++++++++++++++++++++++++++++++++++++++++++++++++++++++++++++++++++++++++++++++
\section{Mini introduction au produit tensoriel}
%+++++++++++++++++++++++++++++++++++++++++++++++++++++++++++++++++++++++++++++++++++++++++++++++++++++++++++++++++++++++++++
\label{SeOOpHsn}

%---------------------------------------------------------------------------------------------------------------------------
\subsection{Définitions}
%---------------------------------------------------------------------------------------------------------------------------

Soit \( E\), un espace vectoriel de dimension finie. Si \( \alpha\) et \( \beta\) sont deux formes linéaires sur un espace vectoriel \( E\), nous définissons \( \alpha\otimes \beta\) comme étant la \( 2\)-forme donnée par
\begin{equation}
    (\alpha\otimes \beta)(u,v)=\alpha(u)\beta(v).
\end{equation}
Si \( a\) et \( b\) sont des vecteurs de \( E\), ils sont vus comme des formes sur \( E\) via le produit scalaire et nous avons
\begin{equation}
    (a\otimes b)(u,v)=(a\cdot u)(b\cdot v).
\end{equation}
Cette dernière équation nous incite à pousser un peu plus loin la définition de \( a\otimes b\) et de simplement voir cela comme la matrice de composantes
\begin{equation}
    (a\otimes b)_{ij}=a_ib_j.
\end{equation}
Cette façon d'écrire a l'avantage de ne pas demander de se souvenir qui est une vecteur ligne, qui est un vecteur colonne et où il faut mettre la transposée. Évidemment \( (a\otimes b)\) est soit \( ab^t\) soit \( a^tb\) suivant que \( a\) et \( b\) soient ligne ou colonne.

\begin{lemma}   \label{LemMyKPzY}
    Soient \( x,y\in E\) et \( A,B\) deux opérateurs linéaires sur \( E\) vus comme matrices. Alors
    \begin{equation}        \label{EqXdxvSu}
        (Ax\otimes By)=A(x\otimes y)B^t.
    \end{equation}
\end{lemma}

\begin{proof}
    Calculons la composante \( ij\) de la matrice \( (Ax\otimes By)\). Nous avons
    \begin{subequations}
        \begin{align}
            (Ax\otimes By)_{ij}&=(Ax)_i(By)_j\\
            &=\sum_{kl}A_{ik}x_kB_{jl}y_l\\
            &=A_{ik}(x\otimes y)_{kl}B_{jl}\\
            &=\big( A(x\otimes y)B^t \big)_{ij}.
        \end{align}
    \end{subequations}
\end{proof}

%+++++++++++++++++++++++++++++++++++++++++++++++++++++++++++++++++++++++++++++++++++++++++++++++++++++++++++++++++++++++++++
\section{Formes bilinéaires et quadratiques}
%+++++++++++++++++++++++++++++++++++++++++++++++++++++++++++++++++++++++++++++++++++++++++++++++++++++++++++++++++++++++++++
\label{SecTQkRXIu}

%--------------------------------------------------------------------------------------------------------------------------- 
\subsection{Généralités}
%---------------------------------------------------------------------------------------------------------------------------

Les applications multilinéaires ont déjà été définies en la définition \ref{DefFRHooKnPCT}; nous donnons ici une définition plus explicite dans le cas des applications bilinéaires.
\begin{definition}
    Une \defe{forme bilinéaire}{forme!bilinéaire} sur un espace vectoriel \( E\) est une application \( b\colon E\times E\to \eK\) telle que
    \begin{enumerate}
        \item
            \( b(u,v)=b(v,u)\),
        \item
            \( b(u+v,w)=b(u,w)+b(v,w)\),
        \item
            \( b(\lambda u,v)=\lambda b(u,v)\)
    \end{enumerate}
    pour tout \( u,v,w\in E\) et \( \lambda\in \eK\) où \( \eK\) est une corps commutatif.
\end{definition}

\begin{definition}[\cite{RUAoonJAym}]   \label{DefBSIoouvuKR}
    Soit un espace vectoriel \( E\) et \( \eF\) un corps de caractéristique différente de \( 2\). Une \defe{forme quadratique}{forme!quadratique} sur \( E\) est une application \( q\colon V\to \eF\) pour laquelle il existe une forme bilinéaire symétrique \( b\colon V\times V\to \eF\) satisfaisant \( q(x)=b(x,x)\) pour tout \( x\in V\).

    L'ensemble des formes quadratiques réelles sur \( E\) est noté \( Q(E)\)\nomenclature[B]{\( Q(E)\)}{formes quadratiques réelles sur \( E\)}.
\end{definition}

\begin{lemma}
    Si \( q\) est une forme quadratique, il existe une unique forme bilinéaire \( b\) telle que \( q(x)=b(x,x)\).
\end{lemma}

\begin{proof}
    L'existence n'est pas en cause : c'est la définition d'une forme quadratique. Pour l'unicité, étant donné une forme quadratique, la forme bilinéaire \( b\) doit forcément vérifier l'\defe{identités de polarisation}{identité!polarisation}\index{polarisation (identité)} :
\begin{equation}    \label{EqMrbsop}
    b(x,y)=\frac{ 1 }{2}\big( q(x)+q(y)-q(x-y) \big).
\end{equation}
Elle est donc déterminée par \( q\).
\end{proof}
Notons la division par \( 2\) qui est le pourquoi de la demande de la caractéristique différente de \( 2\) pour \( \eF\) dans la définition de forme quadratique.

%--------------------------------------------------------------------------------------------------------------------------- 
\subsection{Topologie}
%---------------------------------------------------------------------------------------------------------------------------

La topologie considérée sur \( Q(E)\) est celle de la norme
\begin{equation}    \label{EqZYBooZysmVh}
    N(q)=\sup_{\| x \|_E=1}| q(x) |,
\end{equation}
qui du point de vue de \( S_n(\eR)\) est
\begin{equation}    
    N(A)=\sup_{\| x \|_E=1}| x^tAx |.
\end{equation}
Notons que à droite, c'est la valeur absolue usuelle sur \( \eR\).

%--------------------------------------------------------------------------------------------------------------------------- 
\subsection{Matrice associée}
%---------------------------------------------------------------------------------------------------------------------------

Si une base \( \{ e_i \}_{i=1,\ldots, n}\) de l'espace vectoriel \( E\) est donnée, la \defe{matrice associée}{matrice!associée à une forme quadratique}\index{forme!quadratique!matrice associée} à la forme bilinéaire \( b\) sur \( E\) est la matrice d'éléments
\begin{equation}
    B_{ij}=b(e_i,e_j).
\end{equation}
Notons que la matrice associée à une forme bilinéaire (ou quadratique associée) est uniquement valable pour une base donnée. Si nous changeons de base, la matrice change. Cependant lorsque nous travaillons sur \( \eR^n\), la base canonique est tellement canonique que nous allons nous permettre de parler de «la» matrice associée à une forme bilinéaire. 

Si \( B_{ij}\) est la matrice associée à la forme bilinéaire \( b\) alors la valeur de \( b(u,v)\) se calcule avec la formule
\begin{equation}
    b(x,y)=\sum_{i,j}B_{ij}x_iy_j
\end{equation}
lorsque \( x_i\) et \( y_j\) sont les coordonnées de \( x\) et \( y\) dans la base choisie.

\begin{proposition} \label{PropFSXooRUMzdb}
    Soit \( \{ e_i \}\) une base de \( E\). L'application
    \begin{equation}
        \begin{aligned}
            \phi\colon Q(E)&\to S(n,\eR) \\
            q&\mapsto \big(   b(e_i,e_j)   \big)_{i,j}
        \end{aligned}
    \end{equation}
    où \( b\) est forme bilinéaire associée à \( q\) est une bijection linéaire et continue.
\end{proposition}

\begin{proof}
    Si \( \phi(q)=\phi(q')\); alors
    \begin{equation}
        q(x)=\sum_{i,j}\phi(q)_{ij}x_ix_j=\sum_{i,j}\phi(q')_{ij}x_ix_j=q'(x).
    \end{equation}
    Donc \( q=q'\). L'application \( \phi\) est donc injective

    De plus elle est surjective parce que si \( B\in S(n,\eR)\) alors la forme quadratique
    \begin{equation}
        q(x)=\sum_{i,j}B_{ij}x_ix_j
    \end{equation}
    a évidemment \( B\) comme matrice associée. L'application \( \phi\) est donc surjective.

    Notre application \( \phi\) est de plus linéaire parce que l'association d'une forme quadratique à la forme bilinéaire associée est linéaire.

    En ce qui concerne la continuité, nous la prouvons en zéro en considérant une suite convergente \( q_n\stackrel{Q(E)}{\longrightarrow}0\). C'est à dire que
    \begin{equation}
        \sup_{\| x \|=1}| q_n(x) |\to 0.
    \end{equation}
    Nous rappelons l'identité de polarisation : 
    \begin{equation}
        b_n(x,y)=\frac{ 1 }{2}\big( q_n(x-y)-q(x)-q(y) \big).
    \end{equation}
    En ce qui concerne deux des trois termes, il n'y a pas de problèmes :
    \begin{equation}
        \big| \phi(q_n)_{ij} \big|=\big| b_n(e_i,e_j) \big|\leq\frac{ 1 }{2}\big| b_n(e_i-e_j) \big|+\frac{ 1 }{2}\big| q_n(e_i) \big|+\frac{ 1 }{2}\big| q_n(e_j) \big|.
    \end{equation}
    Si \( n\) est assez grand, nous avons tout de suite
    \begin{equation}
        \big| \phi(q_n)_{ij} \big|\leq \frac{ 1 }{2}\big| q_n(e_i-e_j) \big|+\epsilon.
    \end{equation}
    Nous définissons \( e_{ij}\) et \( \alpha_{ij}\) de telle sorte que \( e_i-e_j=\alpha_{ij}e_{ij}\) avec \( \| e_{ij} \|=1\). Si \( \alpha=\max\{ \alpha_{ij},1 \}\) alors nous avons
    \begin{equation}
        q_n(e_i-e_j)=\alpha_{ij}^2q_n(e_{ij})\leq \alpha^2q_n(e_{ij}).
    \end{equation}
    Il suffit maintenant de prendre \( n\) assez grand pour avoir \( \sup_{\| x \|=1}| q_n(x) |\leq \frac{ \epsilon }{ \alpha^2 }\) pour avoir
    \begin{equation}
        \big| \phi(q_n)_{ij} \big|\leq \frac{ \epsilon }{2}+\frac{ \epsilon }{ \alpha^2 }.
    \end{equation}
\end{proof}

\begin{proposition}\label{PropFWYooQXfcVY}
    Dans la base de diagonalisation de sa matrice associée, une forme quadratique a la forme
    \begin{equation}
        q(x)=\sum_i\lambda_ix_i^2
    \end{equation}
    où les \( \lambda_i\) sont les valeurs propres de la matrice associée à \( q\).
\end{proposition}

\begin{proof}
Soit \( q\) une forme quadratique et \( b\) la forme bilinéaire associée. Si \( \{ f_i \}\) est une base de diagonalisation de la matrice de \( b\) alors dans cette base nous avons
\begin{equation}
    q(x)=b(x,x)=\sum_{ij}x_ix_jb(f_i,f_j)=\sum_i\lambda_ix_i^2
\end{equation}
où les \( \lambda_i\) sont les valeurs propres de la matrice de \( b\).
\end{proof}
Notons que si nous choisissons une autre base de diagonalisation, les \( \lambda_i\) ne changement pas (à part l'ordre éventuellement). Cela pour dire que nous nous permettrons de parler des \defe{valeurs propres}{valeur propre!d'une forme quadratique} d'une forme quadratique comme étant les valeurs propres de la matrice associée.

%---------------------------------------------------------------------------------------------------------------------------
\subsection{Quelque mots à propos de matrices}
%---------------------------------------------------------------------------------------------------------------------------

Si $g$ est une application bilinéaire sur $\eR^2$, nous disons qu'elle est
\begin{enumerate}
\item
\defe{définie positive}{application!définie positive} si $g(u,u)\geq 0$ pour tout $u\in\eR^2$ et $g(u,u)=0$ si et seulement si $u=0$.
\item
\defe{semi-définie positive}{application!semi-définie positive} si $g(u,u)\geq 0$ pour tout $u\in\eR^2$. Nous dirons aussi parfois qu'elle est simplement «positive».
\end{enumerate}
Cela est évidemment à lier à la définition \ref{DefAWAooCMPuVM} : une application bilinéaires est définie positive si et seulement si sa matrice symétrique associée l'est.

\begin{proposition}     \label{PropcnJyXZ}
    Soit $M$, une matrice $2\times 2$ symétrique. Nous avons
    \begin{enumerate}
        \item
        $\det M>0$ et $\tr(M)>0$ implique $M$ définie positive,
        \item
        $\det M>0$ et $\tr(M)<0$ implique $M$ définie négative,
    \item   \label{ItemluuFPN}
        $\det M<0$ implique ni semi définie positive, ni définie négative 
        \item
        $\det M=0$ implique $M$ semi-définie positive ou semi-définie négative.
    \end{enumerate}
\end{proposition}

%--------------------------------------------------------------------------------------------------------------------------- 
\subsection{Dégénérescence}
%---------------------------------------------------------------------------------------------------------------------------

Soit \( b\), une forme bilinéaire symétrique non dégénérée  sur l'espace vectoriel \( E\) de dimension \( n\) sur \( \eK\) où \( \eK\) est un corps de caractéristique différente de \( 2\). Nous notons \( q\) la forme quadratique associée.

\begin{definition}
    Une forme bilinéaire est \defe{non dégénérée}{forme!bilinéaire!non dégénérée} \( b(x,z)=0\) pour tout \( z\) implique \( x=0\).
\end{definition}

\begin{lemma}   \label{LemyKJpVP}
    Soit \( b\) une forme bilinéaire non dégénérée. Si \( x\) et \( y\) sont tels que \( b(x,z)=b(y,z)\) pour tout \( z\), alors \( x=y\).
\end{lemma}

\begin{proof}
    C'est immédiat du fait de la linéarité en le premier argument et de la non-dégénérescence : si \( b(x,z)-b(y,z)=0\) alors
    \begin{equation}
        b(x-y,z)=0
    \end{equation}
    pour tout \( z\), ce qui implique \( x-y=0\).
\end{proof}

\begin{proposition}
    La forme bilinéaire \( b\) est non-dénénérée si et seulement si sa matrice associée est inversible.
\end{proposition}

\begin{proof}
    Nous savons que la matrice associée est symétrique et qu'elle peut donc être diagonalisée (théorème \ref{ThoeTMXla}). En nous plaçant dans une base de diagonalisation, nous devons prouver que la forme est non-dégénérée si et seulement si les éléments diagonaux de la matrice sont tous non nuls.

    Écrivons \( b(x,z)\) en choisissant pour \( z\) le vecteur de base \( e_k\) de composantes \( (e_k)_j=\delta_{kj}\) :
    \begin{equation}
            b(x,e_k)=\sum_{ij}x_i(e_k)_j
            =\sum_i b_{ik}x_i
            =b_{kk}x_k.
    \end{equation}
    Si \( b\) est dégénérée et si \( x\) est un vecteur non nul (disons que la composante \( x_i\) est non nulle) de \( E\) tel que \( b(x,z)=0\) pour tout \( z\in E\), alors \( b_{ii}=0\), ce qui montre que la matrice de \( b\) n'est pas inversible.

    Réciproquement si la matrice de \( b\) est inversible, alors tous les \( b_{kk}\) sont différents de zéro, et le seul vecteur \( x\) tel que \( b_{kk}x_k=0\) pour tout \( k\) est le vecteur nul.
\end{proof}


\begin{definition}[Isotropie]   \label{DefVKMnUEM}
    Un vecteur est \defe{isotrope}{isotrope (vecteur)} pour \( b\) si il est perpendiculaire à lui-même; en d'autres termes, \( x\) est isotrope si et seulement si \( b(x,x)=0\). Un sous-espace \( W\subset E\) est \defe{totalement isotrope}{isotrope!totalement} si pour tout \( x,y\in W\), nous avons \( b(x,y)=0\).

    Le \defe{cône isotrope}{isotrope!cône} de \( b\) est l'ensemble de ses vecteurs isotropes :
    \begin{equation}
        C(b)=\{ x\in E\tq b(x,x)=0 \}.
    \end{equation}
\end{definition}
Nous introduisons quelque notations. D'abord pour \( y\in E\) nous notons
\begin{equation}
    \begin{aligned}
        \Phi_y\colon E&\to \eR \\
        x&\mapsto b(x,y) 
    \end{aligned}
\end{equation}
et ensuite
\begin{equation}
    \begin{aligned}
        \Phi\colon E&\to E^* \\
        y&\mapsto \Phi_y. 
    \end{aligned}
\end{equation}
\begin{definition}
    Le fait pour une forme bilinéaire \( b\) d'être dégénérée signifie que l'application \( \Phi\) n'est pas injective. Le \defe{noyau}{noyau!d'une forme bilinéaire} de la forme bilinéaire est celui de \( \Phi\), c'est à dire
    \begin{equation}
        \ker(b)=\{ z\in E\tq b(z,y)=0\,\forall y\in E \}.
    \end{equation}
    Autrement dit, \( \ker(b)=E^{\perp}\) où le perpendiculaire est pris par rapport à \( b\).
\end{definition}
Notons tout de même que nous utilisons la notation \( \perp\) même si \( b\) est dégénérée et éventuellement pas positive; c'est à dire même si la formule \( (x,y)\mapsto b(x,y)\) ne fournit pas un produit scalaire.

\begin{proposition}[\cite{RTzQrdx}]     \label{PropHIWjdMX}
    Soit \( b\) une forme bilinéaire et symétrique. Alors
    \begin{enumerate}
        \item
            \( \ker(b)\subset C(b)\) (cône d'isotropie, définition \ref{DefVKMnUEM})
        \item
            si \( b\) est positive alors \( \ker(b)=C(b)\).
    \end{enumerate}
\end{proposition}

\begin{proof}
    \begin{enumerate}
        \item
            Si \( z\in\ker(b)\) alors pour tout \( y\in E\) nous avons \( b(z,y)=0\). En particulier pour \( y=z\) nous avons \( b(z,z,)=0\) et donc \( z\in C(b)\).
        \item
            Soit \( b\) positive et \( x\in C(b)\). Par l'inégalité de Cauchy-Schwarz (proposition \ref{ThoAYfEHG}) nous avons
            \begin{equation}
                | b(x,y) |\leq \sqrt{   b(x,x)b(y,y) }=0.
            \end{equation}
            Donc pour tout \( y\) nous avons \( b(x,y)=0\).
    \end{enumerate}
\end{proof}

%--------------------------------------------------------------------------------------------------------------------------- 
\subsection{Inégalité de Minkowski}
%---------------------------------------------------------------------------------------------------------------------------

Ce qui est couramment nommé «inégalité de Minkowski» est la proposition \ref{PropInegMinkKUpRHg} dans les espaces \( L^p\). Nous allons en donner ici un cas très particulier.

\begin{proposition} \label{PropACHooLtsMUL}
    Si \( q\) est une forme quadratique sur \( \eR^n\) et si \( x,y\in \eR^n\) alors
    \begin{equation}
        \sqrt{q(x+y)}\leq\sqrt{q(x)}+\sqrt{q(y)}.
    \end{equation}
\end{proposition}

\begin{proof}
    La proposition \ref{PropFWYooQXfcVY} nous permet de «diagonaliser» la forme quadratique \( q\). Quitte à ne plus avoir une base orthonormale, nous pouvons renormaliser les vecteurs de base pour avoir
    \begin{equation}
        q(x)=\sum_ix_i^2.
    \end{equation}
    Le résultat n'est donc rien d'autre que l'inégalité triangulaire pour la norme euclidienne usuelle, laquelle est démontrée dans la proposition \ref{PropEQRooQXazLz}.
\end{proof}

%--------------------------------------------------------------------------------------------------------------------------- 
\subsection{Ellipsoïde}
%---------------------------------------------------------------------------------------------------------------------------

\begin{lemma}   \label{LemYVWoohcjIX}
    Toute matrice peut être décomposée de façon unique en une partie symétrique et une partie antisymétrique. Cette décomposition est donnée par
\begin{equation}\label{subEqHIQooyhiWM}
    \begin{aligned}[]
            S&=\frac{ M+M^t }{ 2 },&A&=\frac{ M-M^t }{ 2 }
    \end{aligned}
\end{equation}
\end{lemma}

\begin{proof}
    L'existence est une vérification immédiate de \( S+A=M\) en utilisant \eqref{subEqHIQooyhiWM}. Pour l'unicité, si \( S+A=S'+A'\) alors \( S-S'=A-A'\). Mais \( S-S'\) est symétrique et \( A-A'\) est antisymétrique; l'égalité implique l'annulation des deux membres, c'est à dire \( S=S'\) et \( A=A'\).
\end{proof}

\begin{definition}  \label{DefOEPooqfXsE}
    Un \defe{ellipsoïde}{ellipsoïde} dans \( \eR^n\) centré en \( v\) est le lieu des points \( x\) vérifiant l'équation
    \begin{equation}\label{EqSNWooXfbTH}
        (x-v)^t M(x-v)=1
    \end{equation}
    où \( M\) est une matrice symétrique strictement définie positive\footnote{Définition \ref{DefAWAooCMPuVM}.}.

    Lorsque nous parlons d'ellipsoïde \emph{plein}, il suffit de changer l'égalité en une inégalité.
\end{definition}
Une autre façon d'écrire la relation \eqref{EqSNWooXfbTH} est d'écrire \( \langle (x-v),M(x,v)\rangle\) en utilisant le produit scalaire.

\begin{remark}
    Le fait que \( M\) soit symétrique n'est pas tout à fait obligatoire; la chose important est que toutes les valeurs propres soient strictement positives. En effet si \( M\) a toutes ses valeurs propres strictement positives, nous nommons \( S\) la partie symétrique de \( M\) et \( A\) la partie antisymétrique (lemme \ref{LemYVWoohcjIX}). Alors pour tout \( x\in \eR^n\) nous avons
    \begin{equation}
        x^tAx=\langle x, Ax\rangle =\langle A^tx,x \rangle =-\langle Ax, x\rangle =-\langle x,Ax\rangle ,
    \end{equation}
    donc \( x^tAx=0\). L'équation \( x^tMx=1\) est donc équivalente à \( x^tSx=1\) (elles ont les mêmes solutions).
    
    De plus \( S\) reste strictement définie positive parce que pour tout \( x\in \eR^n\) nous avons 
    \begin{equation}
        0<x^tMx=x^tSx.
    \end{equation}
\end{remark}

\begin{proposition}\label{PropWDRooQdJiIr}
    Si \( \ellE\) est un ellipsoïde centrée à l'origine, il existe une base de \( \eR^n\) dans laquelle son équation est :
    \begin{equation}
        \sum_{i=1}^n\frac{ x_i^2 }{ a_i^2 }=1.
    \end{equation}
\end{proposition}

\begin{proof}
    Nous avons une matrice symétrique strictement définie positive \( S\) telle que l'équation soit \( \langle x, Sx\rangle =1\). Le théorème spectral \ref{ThoeTMXla} nous fournit une base orthonormale \( \{ e_i \}\) dans laquelle \( Se_i=\lambda_ie_i\) avec \( \lambda_i>0\). En substituant dans l'équation \( \langle x, Sx\rangle =1\) nous trouvons l'équation
    \begin{equation}
        \sum_i\lambda_ix_i^2=1.
    \end{equation}
    En posant \( a_i=\frac{1}{ \sqrt{\lambda_i} }\), nous trouvons le résultat.  Cette définition des \( a_i\) est toujours possible parce que \( \lambda_i>0\).
\end{proof}

\begin{corollary}   \label{CorKGJooOmcBzh}
    Un ellipsoïde plein centré en l'origine admet une équation de la forme \( q(x)\leq 1\) où \( q\) est une forme quadratique strictement définie positive.
\end{corollary}
Pour rappel de notation, l'ensemble des formes quadratiques strictement définies positives sur l'espace vectoriel \( E\) est noté \( Q^{++}(E)\).

\begin{proof}
    Soit \( \{ e_i \}\) une base de \( \eR^n\) telle que l'ellipsoïde \( \ellE\) ait pour équation
    \begin{equation}
        \sum_{i=1}^n\frac{ x_i^2 }{ a_i^2 }\leq 1.
    \end{equation}
    Nous considérons la forme quadratique
    \begin{equation}
        \begin{aligned}
            q\colon \eR^n&\to \eR \\
            x&\mapsto \sum_{i=1}^n\frac{ \langle x, e_i\rangle^2 }{ a_i^2 }. 
        \end{aligned}
    \end{equation}
    Nous avons évidemment \( \ellE=\{ x\in \eR^n\tq q(x)\leq 1 \}\). De plus la forme \( q\) est strictement définie positive parce que dès que \( x\neq 0\), au moins un des produits scalaires \( \langle x, e_i\rangle \) est non nul et \( q(x)> 0\).
\end{proof}

%--------------------------------------------------------------------------------------------------------------------------- 
\subsection{Théorème spectral auto-adjoint}
%---------------------------------------------------------------------------------------------------------------------------

\begin{definition}
    Si \( E\) est un espace euclidien, un endomorphisme \( f\colon E\to E\) est \defe{auto-adjoint}{endomorphisme!auto-adjoint} si pour tout \( x,y\in E\) nous avons \( \langle x, f(y)\rangle=\langle f(x), Y\rangle  \). 
\end{definition}
L'ensemble des opérateurs auto-adjoints de \( E\) est noté \( \gS(E)\)\nomenclature[A]{\( \gS(E)\)}{Les opérateurs auto-adjoints de $E$}. Cette notation provient du fait que dans \( \eR^n\) muni du produit scalaire usuel, les opérateurs auto-adjoints sont les matrices symétriques.

\begin{theorem}[Théorème spectral auto-adjoint] \label{ThoRSBahHH}
    Un endomorphisme auto-adjoint d'un espace euclidien
    \begin{enumerate}
        \item
            est diagonalisable dans une base orthonormée,
        \item
            a son spectre réel.
    \end{enumerate}
\end{theorem}
\index{théorème!spectral!autoadjoint}
\index{diagonalisation!endomorphisme auto-adjoint}

\begin{proof}
    Nous procédons par récurrence sur la dimension de \( E\), et nous commençons par \( n=1\)\footnote{Dans \cite{KXjFWKA}, l'auteur commence avec \( n=0\) mais moi je n'en ai \wikipedia{en}{Vacuous_truth}{pas le courage.}.}. Soit donc \( f\colon E\to E\) avec \( \langle f(x), y\rangle =\langle x, f(y)\rangle \). Étant donné que \( f\) est également linéaire, il existe \( \lambda\in \eR\) tel que \( f(x)=\lambda x\) pour tout \( x\in E\). Tous les vecteurs de \( E\) sont donc vecteurs propres de \( f\).

    Passons à la récurrence. Nous considérons \( \dim(E)=n+1\) et \( f\in\gS(E)\). Nous considérons la forme bilinéaire symétrique \( \Phi_f\) et la forme quadratique associée \( \phi_f\). Pour rappel,
    \begin{subequations}
        \begin{align}
        \Phi_f(x,y)=\langle x, f(y)\rangle \\
        \phi_f(x)=\Phi_f(x,x).
        \end{align}
    \end{subequations}
    Et nous allons laisser tomber les indices \( f\) pour noter simplement \( \Phi\) et \( \phi\). Étant donné que \( \overline{ B(0,1) }\) est compacte et que \( \phi\) est continue, il existe \( x_0\in\overline{ B(0,1) }\) tel que 
    \begin{equation}
        \lambda=\phi(x_0)=\sup_{x\in\overline{ B(0,1) }}\phi(x).
    \end{equation}
    Notons aussi que \( \| x_0 \|=1\) : le maximum est pris sur le bord. Nous posons
    \begin{equation}
        g=\lambda\id-f
    \end{equation}
    ainsi que 
    \begin{equation}
        \Phi_1(x,y)=\langle x, g(y)\rangle .
    \end{equation}
    Cela est une forme bilinéaire et symétrique parce que
    \begin{equation}
        \Phi_1(y,x)=\langle y, g(x)\rangle =\langle g(y), x\rangle =\langle x, g(y)\rangle =\Phi_1(x,y)
    \end{equation}
    où nous avons utilisé le fait que \( g\) était auto-adjoint et la symétrie du produit scalaire. De plus \( \Phi_1\) est semi-définie positive parce que
    \begin{equation}
        \Phi_1(x,x)=\langle x, \lambda x-f(x)\rangle =\lambda\| x \|^2-\phi(x).
    \end{equation}
    Vu que \( \lambda\) est le maximum, nous avons tout de suite \( \Phi_1(x)\geq 0\) tant que \( \| x \|=1\). Et si \( x\) n'est pas de norme \( 1\), c'est le même prix parce qu'on se ramène à \( \| x \|=1\) en multipliant par un nombre positif. Attention cependant : 
    \begin{equation}
        \Phi_1(x_0,x_0)=\lambda\| x_0 \|^2-\phi(x_0)=0.
    \end{equation}
    Donc \( \Phi_1\) a un noyau contenant \( x_0\) par la proposition \ref{PropHIWjdMX}. Nous en déduisons que \( \Image(g)\neq E\) en effet, \( x_0\in\Image(g)^{\perp}\), mais nous avons la proposition \ref{PropXrTDIi} sur les dimensions : 
    \begin{equation}
        \dim E=\dim(\Image(g))+\dim( \Image(g)^{\perp}).
    \end{equation}
    Vu que \( \Image(g)^{\perp}\) est un espace vectoriel non réduit à \( \{ 0 \}\), la dimension de \( \Image(g)\) ne peut pas être celle de \( E\). L'endomorphisme \( g\) n'étant pas surjectif, il ne peut pas être injectif non plus parce que nous sommes en dimension finie; il existe donc \( e_1\in E\) tel que \( g(e_1)=0\) et tant qu'à faire nous choisissons \( \| e_1 \|=1\) (ici la norme est bien celle de l'espace euclidien considéré). Par définition,
    \begin{equation}
        f(e_1)=\lambda e_1,
    \end{equation}
    c'est à dire que \( \lambda\in\Spec(f)\). Et \( \phi\) étant une forme quadratique réelle nous avons \( \lambda\in \eR\).

    Nous posons à présent \( H=\Span\{ e_1 \}^{\perp}\). C'est un sous-espace stable par \( f\) parce que si \( x\in H\) alors
    \begin{equation}
        \langle e_1, f(x)\rangle =\langle f(e_1j),x\rangle =\lambda\langle e_1, x\rangle =0.
    \end{equation}
    Nous pouvons donc considérer la restriction de \( f\) à \( H\) : \( f_H\colon H\to H\). Cet endomorphisme est bilinéaire et symétrique sur l'espace \( H\) de dimension inférieure à celle de \( E\), donc la récurrence nous donne une base orthonormée
    \begin{equation}
        \{ e_2,\ldots, e_n \}
    \end{equation}
    de vecteurs propres de \( f_H\). De plus les valeurs propres sont réelles, toujours par récurrence. Donc
    \begin{equation}
        \Spec(f)=\{ \lambda \}\cup\Spec(f_H)\subset \eR.
    \end{equation}
    Notons pour être complet que si \( i\geq 2\) alors
    \begin{equation}
        \langle e_1, e_i\rangle =0
    \end{equation}
    parce que le vecteur \( e_i\) est par construction choisit dans l'espace \( H=e_1^{\perp}\). Nous avons donc bien une base orthonormée de \( E\) construite sur des vecteurs propres de \( f\).
\end{proof}

\begin{corollary}   \label{CorSMHpoVK}
    Soit \( E\) un espace vectoriel ainsi que \( \phi\) et \( \psi\) des formes quadratiques sur \( E\) avec \( \psi\) définie positive. Alors il existe une base \( \psi\)-orthonormale dans laquelle \( \phi\) est diagonale.
\end{corollary}

\begin{proof}
    Il suffit de considérer l'espace euclidien \( E\) muni du produit scalaire \( \langle x, y\rangle =\psi(x,y)\). Ensuite nous diagonalisons la matrice (symétrique) de \( \phi\) pour ce produit scalaire à l'aide du théorème \ref{ThoRSBahHH}.
\end{proof}

\begin{definition}      \label{DefYNWUFc}
    Dans le cas de \( V=\eR^m\) nous avons un produit scalaire canonique. Soient $u$ et $v$, deux vecteurs de $\eR^m$. Le \defe{produit scalaire}{produit!scalaire} de $u$ et $v$, noté $\langle u, v\rangle $ ou $u\cdot v$ est le réel
	\begin{equation}		\label{EqDefProdScalsumii}
		\langle u, v\rangle =\sum_{k=1}^m u_kv_k=u_1v_1+u_2v_2+\ldots+u_mv_n.
	\end{equation}
\end{definition}

Calculons par exemple le produit scalaire de deux vecteurs de la base canonique : $\langle e_i, e_j\rangle $. En utilisant la formule de définition et le fait que $(e_i)_k=\delta_{ik}$, nous avons
\begin{equation}
	\langle e_i, e_j\rangle =\sum_{k=1}^m\delta_{ik}\delta_{jk}.
\end{equation}
Nous pouvons effectuer la somme sur $k$ en remarquant qu'à cause du $\delta_{ik}$, seul le terme avec $k=i$ n'est pas nul. Effectuer la somme revient donc à remplacer tous les $k$ par des $i$ :
\begin{equation}
	\langle e_i, e_j\rangle =\delta_{ii}\delta_{ji}=\delta_{ji}.
\end{equation}

Une des propriétés intéressantes du produit scalaire est qu'il permet de décomposer un vecteur dans une base, comme nous le montre la proposition suivante.

\begin{proposition}		\label{PropScalCompDec}
	Si nous notons $v_i$ les composantes du vecteur $v$, c'est à dire si $v=\sum_{i=1}^m v_ie_i$, alors nous avons $v_j=\langle v, e_j\rangle $.
\end{proposition}

\begin{proof}
	\begin{equation}		\label{Eqvejscalcomp}
		v\cdot e_j=\sum_{i=1}^m\langle v_ie_i, e_j\rangle =\sum_{i=1}^mv_i\langle e_i, e_j\rangle =\sum_{i=1}^mv_i\delta_{ij}
	\end{equation}
	En effectuant la somme sur $i$ dans le membre de droite de l'équation \eqref{Eqvejscalcomp}, tous les termes sont nuls sauf celui où $i=j$; il reste donc
	\begin{equation}
		v\cdot e_j=v_j.
	\end{equation}
\end{proof}

Le produit scalaire ne dépend en réalité pas de la base orthogonale choisie. 

\begin{lemma}
	Si $\{ e_i \}$ est la base canonique, et si $\{ f_i \}$ est une autre base orthonormale, alors si $u$ et $v$ sont deux vecteurs de $\eR^m$, nous avons
	\begin{equation}
		\sum_i u_iv_j=\sum_iu'_iv'_j
	\end{equation}
	où $u_i$ sont les composantes de $u$ dans la base $\{ e_i \}$ et $u'_i$ sont celles dans la base $\{ f_i \}$.
\end{lemma}

\begin{proof}
	La preuve demande un peu d'algèbre linéaire. Étant donné que $\{ f_i \}$ est une base orthonormale, il existe une matrice $A$ orthogonale ($AA^t=\mtu$) telle que $u'_i=\sum_jA_{ij}u_j$ et idem pour $v$. Nous avons alors
	\begin{equation}
		\begin{aligned}[]
			\sum_iu'_iv'_j&=\sum_i\left( \sum_jA_{ij} u_j\right)\left( \sum_k A_{ik}v_k \right)\\
			&=\sum_{ijk}A_{ij}A_{ik}u_jv_k\\
			&=\sum_{jk}\underbrace{\sum_i(A^t)_{ji}A_{ik}}_{=\delta_{jk}}u_jv_k\\
			&=\sum_{jk}\delta_{jk}u_jv_k\\
			&=\sum_ku_jv_k.
		\end{aligned}
	\end{equation}	
\end{proof}

Cette proposition nous permet de réellement parler du produit scalaire entre deux vecteurs de façon intrinsèque sans nous soucier de la base dans laquelle nous regardons les vecteurs.

Nous dirons que deux vecteurs sont \defe{orthogonaux}{orthogonal} lorsque leur produit scalaire est nul. Nous écrivons que $u\perp v$ lorsque $\langle u, v\rangle =0$.
\begin{definition}	\label{DefNormeEucleApp}
	La \defe{norme euclidienne}{norme!euclidienne!dans $\eR^m$} d'un élément de $\eR^m$ est définie par $\| u \|=\sqrt{u\cdot u}$.
\end{definition}

Cette définition est motivée par le fait que le produit scalaire $u\cdot u$ donne exactement la norme usuelle donnée par le théorème de Pythagore :
\begin{equation}
	u\cdot u=\sum_{i=1}^mu_iu_i=\sum_{i=1}^m u_i^2=u_1^2+u_2^2+\ldots+u_m^2.
\end{equation}

Le fait que $e_i\cdot e_j=\delta_{ij}$ signifie que la base canonique est \defe{orthonormée}{orthonormé}, c'est à dire que les vecteurs de la base canonique sont orthogonaux deux à deux et qu'ils ont tout $1$ comme norme.

\begin{lemma}\label{LemSclNormeXi}
	Pour tout $u\in\eR^m$, il existe un $\xi\in\eR^m$ tel que $\| u \|=\xi\cdot u$ et $\| \xi \|=1$.
\end{lemma}

\begin{proof}
	Vérifions que le vecteur $\xi=u/\| u \|$ ait les propriétés requises. D'abord $\| \xi \|=1$ parce que $u\cdot u=\| u \|^2$. Ensuite
	\begin{equation}
		\xi\cdot u=\frac{ u\cdot u }{ \| u \| }=\frac{ \| u \|^2 }{ \| u \| }=\| u \|.
	\end{equation}
\end{proof}

%+++++++++++++++++++++++++++++++++++++++++++++++++++++++++++++++++++++++++++++++++++++++++++++++++++++++++++++++++++++++++++
\section{Espaces hermitiens}
%+++++++++++++++++++++++++++++++++++++++++++++++++++++++++++++++++++++++++++++++++++++++++++++++++++++++++++++++++++++++++++

\begin{definition}  \label{DefMZQxmQ}
Si \( E\) est un espace vectoriel sur \( \eC\), nous disons qu'une application \( \langle ., .\rangle \colon E\times E\to \eC\) est un \defe{produit scalaire hermitien}{produit!scalaire!hermitien}\index{hermitien!produit scalaire} si pour tout \( u,v\in E\) nous avons
\begin{enumerate}
    \item
        \( \langle u, v\rangle =\overline{ \langle v, u\rangle  }\)
    \item
        \( \lambda\langle u, v\rangle =\langle \lambda u, v\rangle =\langle u, \bar \lambda v\rangle \)
    \item
        \( \langle u, u\rangle \in \eR^+\) et \( \langle u, u\rangle =0\) si et seulement si \( u=0\).
\end{enumerate}
\end{definition}

%+++++++++++++++++++++++++++++++++++++++++++++++++++++++++++++++++++++++++++++++++++++++++++++++++++++++++++++++++++++++++++
\section{Coordonnées cylindriques et sphériques}
%+++++++++++++++++++++++++++++++++++++++++++++++++++++++++++++++++++++++++++++++++++++++++++++++++++++++++++++++++++++++++++

Les \defe{coordonnées cylindriques}{coordonnées!cylindrique} sont un perfectionnement des coordonnées polaires. Il s'agit simplement de donner le point $(x,y,z)$ en faisant la conversion $(x,y)\mapsto(r,\theta)$ et en gardant le $z$. Les formules de passage sont
\begin{subequations}
	\begin{numcases}{}
		x=r\cos(\theta)\\
		y=r\sin(\theta)\\
		z=z.
	\end{numcases}
\end{subequations}

Les \defe{coordonnées sphériques}{coordonnées!sphériques} sont ce qu'on appelle les «méridiens» et «longitudes» en géographie. Les formules de transformation sont 
\begin{subequations}		\label{SubEqsCoordSphe}
	\begin{numcases}{}
		x=\rho\sin(\theta)\cos(\varphi)\\
		y=\rho\sin(\theta)\sin(\varphi)\\
		z=\rho\cos(\theta)
	\end{numcases}
\end{subequations}
avec $0\leq\theta\leq\pi$ et $0\leq\varphi<2\pi$.

\begin{remark}
	Attention : d'un livre à l'autre les conventions sur les noms des angles changent. N'essayez donc pas d'étudier par cœur des formules concernant les coordonnées sphériques trouvées autre part. Par exemple sur le premier dessin de \href{http://fr.wikipedia.org/wiki/Coordonnées_sphériques}{wikipédia}, l'angle $\varphi$ est noté $\theta$ et l'angle $\theta$ est noté $\Phi$. Mais vous noterez que sur cette même page, les convention de noms de ces angles changent plusieurs fois.
\end{remark}

Trouvons le changement inverse, c'est à dire trouvons $\rho$, $\theta$ et $\varphi$ en termes de $x$, $y$ et $z$. D'abord nous avons
\begin{equation}
	\rho=\sqrt{x^2+y^2+z^2}.
\end{equation}
Ensuite nous savons que
\begin{equation}
	\cos(\theta)=\frac{ z }{ \rho }
\end{equation}
détermine de façon unique\footnote{Le problème $\rho=0$ ne se pose pas; pourquoi ?} un angle $\theta\in\mathopen[ 0 , \pi \mathclose]$. Dès que $\rho$ et $\theta$ sont connus, nous pouvons poser $r=\rho\sin\theta$ et alors nous nous trouvons avec les équations
\begin{subequations}
	\begin{numcases}{}
		x=r\cos(\varphi)\\
		y=r\sin(\varphi),
	\end{numcases}
\end{subequations}
qui sont similaires à celles déjà étudiées dans le cas des coordonnées polaires.

% TODO: Ajouter un texte sur les équations de plan, et pourquoi ax+by+cz+d=0 est perpendiculaire au vecteur (a,b,c).

%+++++++++++++++++++++++++++++++++++++++++++++++++++++++++++++++++++++++++++++++++++++++++++++++++++++++++++++++++++++++++++
\section{Méthode de Gauss pour résoudre des systèmes d'équations linéaires}
%+++++++++++++++++++++++++++++++++++++++++++++++++++++++++++++++++++++++++++++++++++++++++++++++++++++++++++++++++++++++++++

Pour résoudre un système d'équations linéaires, on procède comme suit:
\begin{enumerate}
\item Écrire le système sous forme matricielle. \[\text{p.ex. } \begin{cases} 2x+3y &= 5 \\ x+2y &= 4 \end{cases} \Leftrightarrow \left(\begin{array}{cc|c} 2 & 3 & 5 \\ 1 & 2 & 4 \end{array}\right) \]
\item Se ramener à une matrice avec un maximum de $0$ dans la partie de gauche en utilisant les transformations admissibles:
\begin{enumerate}
\item Remplacer une ligne par elle-même + un multiple d'une autre;
\[\text{p.ex. } \left(\begin{array}{cc|c} 2 & 3 & 5 \\ 1 & 2 & 4 \end{array}\right)  \stackrel{L_1  - 2. L_2 \mapsto L_1'}{\Longrightarrow} \left(\begin{array}{cc|c} 0 & -1 & -3 \\ 1 & 2 & 4 \end{array}\right) \]
\item Remplacer une ligne par un multiple d'elle-même;
\[\text{p.ex. } \left(\begin{array}{cc|c} 0 & -1 & -3 \\ 1 & 2 & 4 \end{array}\right)  \stackrel{-L_1  \mapsto L_1'}{\Longrightarrow} \left(\begin{array}{cc|c} 0 & 1 & 3 \\ 1 & 2 & 4 \end{array}\right) \]
\item Permuter des lignes.
\[\text{p.ex. } \left(\begin{array}{cc|c} 0 & 1 & 3 \\ 1 & 0 & -2 \end{array}\right)  \stackrel{L_1  \mapsto L_2' \text{ et } L_2  \mapsto L_1'}{\Longrightarrow} \left(\begin{array}{cc|c} 1 & 0 & -2 \\ 0 & 1 & 3  \end{array}\right) \]
\end{enumerate}
\item Retransformer la matrice obtenue en système d'équations.
\[\text{p.ex. }  \left(\begin{array}{cc|c} 1 & 0 & -2 \\ 0 & 1 & 3  \end{array}\right) \Leftrightarrow \begin{cases} x &= -2 \\ y &= 3 \end{cases}  \]
\end{enumerate}

\begin{remark}
\begin{itemize}
\item Si on obtient une ligne de zéros, on peut l'enlever:
\[\text{p.ex. }  \left(\begin{array}{ccc|c} 3 & 4 & -2 & 2 \\ 4 & -1 & 3 & 0 \\ 0 & 0 & 0 & 0 \end{array}\right) \Leftrightarrow  \left(\begin{array}{ccc|c} 3 & 4 & -2 & 2 \\ 4 & -1 & 3 & 0 \end{array}\right) \]
\item Si on obtient une ligne de zéros suivie d'un nombre non-nul, le système d'équations n'a pas de solution:
\[\text{p.ex. }  \left(\begin{array}{ccc|c} 3 & 4 & -2 & 2 \\ 4 & -1 & 3 & 0 \\ 0 & 0 & 0 & 7 \end{array}\right) \Leftrightarrow  \begin{cases} \cdots \\ \cdots \\ 0x + 0y + 0z = 7 \end{cases} \Rightarrow \textbf{Impossible} \]
\item Si on moins d'équations que d'inconnues, alors il y a une infinité de solutions qui dépendent d'un ou plusieurs paramètres:
\[\text{p.ex. }  \left(\begin{array}{ccc|c} 1 & 0 & -2 & 2 \\ 0 & 1 & 3 & 0 \end{array}\right) \Leftrightarrow  \begin{cases} x - 2z = 2 \\ y + 3z = 0 \end{cases} \Leftrightarrow  \begin{cases} x = 2 + 2\lambda \\ y = -3\lambda \\ z = \lambda \end{cases} \]
\end{itemize}
\end{remark}
