% This is part of Mes notes de mathématique
% Copyright (c) 2011-2014
%   Laurent Claessens, Carlotta Donadello
% See the file fdl-1.3.txt for copying conditions.

%+++++++++++++++++++++++++++++++++++++++++++++++++++++++++++++++++++++++++++++++++++++++++++++++++++++++++++++++++++++++++++
\section{Matrice compagnon et endomorphismes cycliques}
%+++++++++++++++++++++++++++++++++++++++++++++++++++++++++++++++++++++++++++++++++++++++++++++++++++++++++++++++++++++++++++

%---------------------------------------------------------------------------------------------------------------------------
\subsection{Matrice compagnon}
%---------------------------------------------------------------------------------------------------------------------------

Soit le polynôme \( P=X^n-a_{n-1}X^{n-1}-\ldots-a_1X-a_0\) dans \( \eK[X]\). La \defe{matrice compagnon}{matrice!compagnon} de \( P\) est la matrice\nomenclature[A]{\( C(P)\)}{matrice compagnon} donnée par
\begin{equation}
    C(P)=\begin{pmatrix}
        0    &   \cdots    &   \cdots    &   0    &   a_0\\  
        1    &   0    &       &   \vdots    &   a_1\\  
        0    &   \ddots    &   \ddots    &   \vdots    &   \vdots\\  
        \vdots    &   \ddots    &   \ddots    &   0    &   a_{n-2}\\  
        0    &   \cdots    &   0    &   1    &   a_{n-1}    
    \end{pmatrix}
\end{equation}
si \( n\geq 2\) et par \( (a_0)\) si \( n=1\). Si \( f\) est l'endomorphisme associé à la matrice \( C(P)\) nous avons
\begin{equation}
    f(e_i)=\begin{cases}
        e_{i+1}    &   \text{si \( i<n\)}\\
        (a_0,\ldots, a_{n-1})    &    \text{si \( i=n\)}.
    \end{cases}
\end{equation}
Cet endomorphisme est conçu pour vérifier \( P(f)e_1=0\).

\begin{lemma}[\cite{RapportArgreg2011}] \label{LemkVNisk}
    Un polynôme sur un corps commutatif est le polynôme caractéristique de sa matrice compagnon. En d'autres termes nous avons \( \chi_{C(P)}=P\).
\end{lemma}

\begin{proof}
    Nous notons \( f\) l'endomorphisme associé à \( C(P)\). La propriété \( P(f)e_1=0\) nous indique que le polynôme minimal ponctuel de \( f\) en \( e_1\) divise \( P\). L'ensemble des puissances de \( f\) appliquées à \( e_1\), \( \big( f^i(e_1) \big)_{i=1,\ldots, n-1}\) est libre, donc le polynôme minimal ponctuel en \( e_1\) est de degré \( n\) au minimum. En reprenant les notations du théorème \ref{ThoCCHkoU}, nous avons \( I_{e_1}=(P)\) parce que \( P\) est de degré minimum dans \( I_{e_1}\) et \( \chi_f\in I_{e_1}\).

    Donc \( P\) divise \( \chi_f\) et est de degré égal à celui de \( \chi_f\). Étant donné qu'ils sont tous deux unitaires, ils sont égaux.
\end{proof}

\begin{remark}  \label{RemmQjZOA}
    Les matrices compagnons ne sont pas les seules dont le polynôme caractéristique est égal au polynôme minimal. En fait les matrices dont le polynôme caractéristique est égale au polynôme minimal sont denses dans les matrices. En effet une matrice dont le polynôme minimal n'est pas égal au polynôme caractéristique a un polynôme caractéristique avec une racine double. Il est possible, en modifiant arbitrairement peu la matrice de séparer la racine double en deux racines distinctes.
\end{remark}

\begin{definition}[Matrices, endomorphismes et vecteurs cycliques]
    Une matrice est \defe{cyclique}{cyclique!matrice}\index{matrice!cyclique} si elle est semblable à une matrice compagnon. Un endomorphisme \( f\colon E\to E\) est \defe{cyclique}{cyclique!endomorphisme}\index{endomorphisme!cyclique} si il existe un vecteur \( x\in E\) tel que \( \{ f^k(x)\tq k=1,\ldots, n-1 \}\) est une base de \( E\). Un vecteur ayant cette propriété est un \defe{vecteur cyclique}{vecteur!cyclique} pour \( f\).
\end{definition}

\begin{lemma}\label{LemSGmdnE}
    Si \( A\) est la matrice de l'endomorphisme \( f\) alors nous avons équivalence des propriétés suivantes :
    \begin{enumerate}
        \item
            La matrice \( A\) est cyclique.
        \item
            L'endomorphisme \( f\) est cyclique.
        \item
            Le polynôme caractéristique de \( A\) est égal à son polynôme caractéristique.
    \end{enumerate}
\end{lemma}

\begin{lemma}   \label{LemAGZNNa}
    Si \( f\colon E\to E\) est un endomorphisme cyclique et si \( y\) est un vecteur cyclique de \( f\), alors le polynôme minimal de \( f\) est égal au polynôme minimal de \( f\) au point \( y\) : \( \mu_{f}=\mu_{f,y}\).
\end{lemma}

\begin{proof}
    Montrons que \( \mu_{f,y}\) est un polynôme annulateur de \( f\), ce qui prouvera que \( \mu(f)\) divise \( \mu_{f,y}\). Étant donné que \( y\) est cyclique, tout élément de \( E\) s'écrit sous la forme \( x=Q(f)y\). Prenons un polynôme \( P\) annulateur de \( f\) en \( y\) : \( P(f)y=0\). Nous montrons que \( P\) est alors un polynôme annulateur de \( f\). En effet, nous avons
    \begin{equation}
        P(f)x=\big( P(f)\circ Q(f) \big)y=\big( Q(f)\circ P(f) \big)y=0
    \end{equation}
    où nous avons utilisé le lemme \ref{LemQWvhYb}.
\end{proof}

%---------------------------------------------------------------------------------------------------------------------------
\subsection{Réduction de Frobenius}
%---------------------------------------------------------------------------------------------------------------------------

\begin{theorem}[Réduction de Frobenius \cite{AutourFrobCompa,Vialivs,MoncetIVS}]      \index{réduction!Frobénius}\index{Frobénius!réduction}
    Soit \( E\), un \( \eK\)-espace vectoriel où \( \eK\) est \( \eR\) ou \( \eC\), et \( f\in \End(E)\). Alors il existe une suite de sous-espaces \( E_1,\ldots, E_r\) stables par \( f\) tels que
    \begin{enumerate}
        \item   \label{ItemmpwjnSs}
            \( E=\bigoplus_{i=1}^rE_i\);
        \item
            pour chaque \( E_i\), l'endomorphisme restreint \( f_i=f|_{E_i}\) est cyclique;
        \item
            si \( \mu_i\) est le polynôme minimal de \( f_i\) alors \( \mu_{i+1}\) divise \( \mu_i\);
    \end{enumerate}
    Une telle décomposition vérifie automatiquement \( \mu_1=\mu_f\) et \( \mu_1\cdots \mu_r=\chi_f\), et la suite \( (\mu_i)_{i=1,\ldots, r}\) ne dépend que de \( f\) et non du choix de la décomposition du point \ref{ItemmpwjnSs}.
\end{theorem}

Les polynômes \( \mu_i\) sont les \defe{invariants de similitude}{invariant!de similitude} de l'endomorphisme \( f\).

\begin{proof}
    Nous commençons par montrer que si une telle décomposition existe, alors
    \begin{subequations}    \label{subEqzcGouz}
        \begin{align}
            \chi_f=\prod_{i=1}^r\mu_i  \label{EqTaxsvb}\\
            \mu_f=\mu_1
        \end{align}
    \end{subequations}
    où \( \chi_f\) est le polynôme caractéristique de \( f\) et \( \mu_f\) est le polynôme minimal. D'abord le polynôme caractéristique de \( f\) devra être égal au produit des polynômes caractéristique des \( f|_{E_i}\), mais ces derniers endomorphismes étant cycliques, leurs polynôme caractéristiques sont égaux à leurs polynômes minimaux (lemme \ref{LemSGmdnE}). Cela prouve l'égalité \eqref{EqTaxsvb}. Ensuite tous les \( \mu_i\) doivent diviser le polynôme minimal, donc \( \ppcm(\mu_1,\ldots, \mu_r)\) divise \(\mu_f\). Cependant le polynôme minimal doit contenir une et une seule fois chacun des facteurs irréductibles du polynôme caractéristique, et chacun de ces facteurs sont dans les polynômes \( \mu_i\). Par conséquent \( \ppcm(\mu_1,\ldots, \mu_r)=\mu_f\). Mais par ailleurs \( \mu_1=\ppcm(\mu_1,\ldots, \mu_r)\) parce qu'on a supposé \( \mu_{i+1}\divides \mu_i\), donc \( \mu_1=\mu_f\).
    
    Soit \( d\), le degré du polynôme minimal de \( f\) et \( y\in E\) tel que \( \mu_f=\mu_{f,y}\) (voir lemme \ref{LemSYsJJj}). Le plus petit espace stable sous \( f\) contenant \( y\) est
    \begin{equation}
        E_y=\Span\{ y,f(y),\ldots, f^{d-1}(y) \}.
    \end{equation}
    Nous notons \( e_i=f^{i-1}(y)\). Notons que les vecteurs donnés forment bien une base de \( E_y\) parce que si les \( e_i\) n'était pas linéairement indépendants, alors nous aurions des \( a_k\) tels que \( \sum_ka_ke_k=0\) et avec lesquels
    \begin{equation}
        \big( \sum_ka_kX^k \big)(f)y=0,
    \end{equation}
    ce qui contredirait la minimalité de \( \mu_{f,y}\).

    La difficulté du théorème est de trouver un complément de \( E_y\) qui soit également stable sous \( f\). Nous commençons par étendre\quext{Pour autant que j'aie compris, cette extension manque dans \cite{AutourFrobCompa}. Corrigez moi si je me trompe.} \( \{ e_1,\ldots, e_d \}\) en une base \( \{ e_1,\ldots, e_n \}\) de \( E\). Ensuite nous allons montrer que
    \begin{equation}
        E=E_y\oplus F
    \end{equation}
    avec
    \begin{equation}
        F=\{ x\in E\tq  e^*_d\big( f^k(x) \big)=0\forall k\in \eN \}.
    \end{equation}
    Par construction, \( F\) est invariant sous \( f\). Montrons pour commencer que \( E_y\cap F=\{ 0 \}\). Un élément de \( E_y\) s'écrit
    \begin{equation}
        z=a_1e_1+\ldots +a_ke_k
    \end{equation}
    avec \( k\leq d\). Étant donné que \( f\) décale les vecteurs de base, nous avons \( e^*_d\big( f^{d-k}(z) \big)=a_k\). Du coup \( z\in F\) si et seulement si \( a_1=\ldots=a_d=0\), c'est à dire que \( E_y\cap F=\{ 0 \}\).

    Nous montrons maintenant que \( \dim F=n-d\). Pour cela nous considérons l'application
    \begin{equation}
        \begin{aligned}
            T\colon \eK[F]&\to E^* \\
            g&\mapsto e^*_d\circ g. 
        \end{aligned}
    \end{equation}
    Cette application est injective. En effet un élément général de \( \eK[f]\) est
    \begin{equation}
        g=a_1\id+a_2f+\ldots +a_pf^{p-1}
    \end{equation}
    avec \( p\leq d\). Si \( T(g)=0\), alors nous avons en particulier
    \begin{equation}
        0=T(g)e_{_d-p+1}=e^*_d(a_1e_{d-p+1}+a_2e_{d-p+2}+\ldots +a_pe_d)=a_p.
    \end{equation}
    Donc \( a_p=0\) et en appliquant maintenant \( T(g)\) à \( e_{d-p}\) nous obtenons \( a_{p-1}=0\). Au final nous trouvons que \( g=0\) et donc que \( T\) est injective.

    Étant donné que \( \dim\eK[f]=d\) et que \( T\) est injective, \( \dim\Image(T)=d\). Nous regardons l'orthogonal de l'image :
    \begin{subequations}
        \begin{align}
            (\Image(T))^{\perp}&=\{ x\in E\tq T(g)x=0\forall g\in\eK[f] \}\\
            &=\{ x\in E\tq e^*_d\big( g(x) \big)=0\forall g\in \eK[f] \}\\
            &=F.
        \end{align}
    \end{subequations}
    Par conséquent \( F^{\perp}=\Image(T)\). Vu que \( \dim\Image(T)=d\), nous avons donc \( \dim F=n-d\) et il est établi que \( E=E_y\oplus F\). 

    Nous avons donc trouvé \( F\), stable par \( f\) et tel que \( E=E_y\oplus F\). Nous devons maintenant nous assurer que cette décomposition tombe bien pour les polynômes minimaux. Si \( P_1\) est le polynôme minimal de \( f|_{E_yj}\), alors par le lemme \ref{LemAGZNNa} nous avons \( P_1=\mu_{f,y}=\mu_f\) parce que \( f|_{E_y}\) est cyclique sur \( E_y\). Mettons \( P_2\), le polynôme minimal de \( f|_F\). Étant attendu que \( F\) est stable par \( f\), le polynôme \( P_2\) divise \( P_1\). En recommençant la construction sur \( F\), nous construisons un nouvel espace \( F'\) stable sous \( F\) et vérifiant \( \mu_{f|_{F'}}=P_2\), etc.

    Nous passons maintenant à la partie unicité du théorème. Soient deux suites \( F_1,\ldots, F_r\) et \( G_1,\ldots, G_s\) de sous-espaces stables par \( f\) et vérifiant
    \begin{enumerate}
        \item
            \( E=\bigoplus_{i=1}^rF_i\),
        \item
            \( f|_{F_i}\) est cyclique,
        \item
            \( \mu_{f|_{F_{i+1}}}\) divise \( \mu_{f|_{F_i}}\),
    \end{enumerate}
    et, \emph{mutatis mutandis}, les mêmes conditions pour la famille \( \{ G_i \}\). Nous posons \( P_i=\mu_{f_{F_i}}\) et \( Q_i=\mu_{f|_{G_i}}\). Nous allons montrer par récurrence que \( P_i=Q_i\) et \( \dim F_i=\dim G_i\). Il ne sera cependant pas garanti que \( F_i=G_i\). D'abord, \( P_1=Q_1\) parce qu'ils sont tous deux égaux à \( \mu_f\) par les relations \eqref{subEqzcGouz}. Nous supposons que \( P_i=Q_i\) pour \( i\leq 1\leq j-1\) et nous tentons de montrer que \( P_j=Q_j\).

    Nous avons 
    \begin{equation}    \label{EqMrCtZO}
        P_j(f)=P_j(f)|_{F_1}\oplus\ldots\oplus P_j(f)|_{F_{j-1}}.
    \end{equation}
    En effet étant donné que \( P_{j+k}\) divise \( P_j\), nous avons\footnote{En vertu du lemme \ref{LemQWvhYb}.} \( P_{j}(f)=A(f)\circ P_{j+k}(f)\), mais \( P_{j+k}(f)F_{j+k}=0\), donc \( P_j(f)F_{j+k}=0\). Les espaces \( G_i\) n'ayant a priori aucun rapport avec les polynômes \( P_i\), nous écrivons
    \begin{equation}    \label{EqJreLiO}
        P_j(f)=P_j(f)|_{G_1}\oplus\ldots\oplus P_j(f)|_{G_{j-1}}\oplus P_j(f)|_{G_j}\oplus\ldots\oplus P_j(f)|_{G_s}.
    \end{equation}
    Pour \( 1\leq i\leq j-1\), nous avons supposé \( P_i=Q_i\). Étant donné que \( f|_{F_i}\) est semblable à \( C_{_i}\) et \( f|_{G_i}\) est semblable à \( C_{Q_i}\), la matrice de \( f|_{E_i}\) est semblable à la matrice de \( f|_{G_i}\). En particulier,
    \begin{equation}
        \dim P_j(f)F_i=\dim P_j(f)G_i.
    \end{equation}
    En prenant les dimensions des images dans les égalités \eqref{EqMrCtZO} et \eqref{EqJreLiO}, nous trouvons que
    \begin{equation}
        P_j(f)|_{G_j}=\ldots=P_j(f)|_{G_s}=0.
    \end{equation}
    Par conséquent \( P_j\in I_{f|G_j}\) et donc \( P_j\) divise \( Q_j\), qui est générateur de \( I_{f|_{G_j}}\). La situation étant symétrique entre \( P\) et \( Q\), nous montrons de même que \( Q_j\) divise \( P_j\) et donc que \( P_j=Q_j\).

    Ceci achève la démonstration du théorème de réduction de Frobenius.

\end{proof}


Sous forme matricielle, ce théorème dit que toute matrice est semblable à une matrice de la forme bloc-diagonale
\begin{equation}
    f=\begin{pmatrix}
        C_{\mu_1}    &       &       \\
            &   \ddots    &       \\
            &       &   C_{\mu_r}
    \end{pmatrix}
\end{equation}

\begin{remark}
    Si nous travaillons sur \( \eR\), la réduite de Frobenius restera une matrice réelle, même si les valeurs propres sont complexes. En effet le procédé de Frobenius ne regarde absolument pas les valeurs propres, mais seulement les facteurs irréductibles du polynôme caractéristique. La réduite de Frobenius ne tente pas de résoudre ces polynômes, mais se contente d'en utiliser les matrices compagnon.

    La situation sera différente dans le cas de la forme normale de Jordan.
\end{remark}

%---------------------------------------------------------------------------------------------------------------------------
\subsection{Forme normale de Jordan}
%---------------------------------------------------------------------------------------------------------------------------

Il existe une preuve directe de la réduction de Jordan ne nécessitant pas la réduction de Frobenius\cite{LecLinAlgAllen}. Cette dernière passe par les espaces caractéristiques\footnote{Aussi appelés «espaces propres généralisés».} et est à mon avis plus compliquée que la démonstration de Frobenius elle-même. Nous allons donc nous contenter de donner la réduction de Jordan comme un cas particulier de Frobenius.

\begin{theorem}[Réduction de Jordan]\index{réduction!Jordan}\index{Jordan!réduction}
    Soit \( E\) un espace vectoriel sur \( \eK\), et \( f\in\End(E)\) un endomorphisme dont le polynôme caractéristique \( \chi_f\) est scindé\footnote{C'est pour cette hypothèse que \( \eK=\eR\) n'est pas le bon cadre.}. Il existe une base de \( E\) dans laquelle la matrice de \( f\) s'écrit sous la forme
    \begin{equation}
        M=\begin{pmatrix}
            J_{n_1}(\lambda_1)    &       &       \\
                &   \ddots    &       \\
                &       &   J_{n_k}(\lambda_k)
        \end{pmatrix}
    \end{equation}
    où les \( \lambda_i\) sont les valeurs propres de \( f\) (avec éventuelle répétitions) et \( J_n(\lambda)\) représente le bloc \( n\times n\)
    \begin{equation}
        J_n(\lambda)=\begin{pmatrix}
            \lambda    &   1    &       &       &   \\  
                &   \lambda    &   1    &       &   \\  
                &       &   \lambda    &       &   \\  
                &       &       &   \ddots    &   1\\  
                &       &       &       &   \lambda    
        \end{pmatrix}.
    \end{equation}
    En d'autres termes, \( J_n(\lambda)_{ii}=\lambda\) et \( J_n(\lambda)_{i-1,i}=1\).    
\end{theorem}

\begin{proof}
    Nous commençons par le cas où \( f\) est nilpotente; nous notons \( M\) sa matrice. Dans ce cas la seule valeur propre est zéro et le polynôme caractéristique est \( X^m\) pour un certain \( m\). Nous savons par le lemme \ref{LemkVNisk} que (la matrice de) \( f\) est semblable à sa matrice compagnon. En l'occurrence pour \( f\) nous avons
    \begin{equation}
        C_{X^m}=\begin{pmatrix}
             0   &       &       &  0     \\
             1   &   \ddots    &       &   \vdots    \\
                &   \ddots    &   \ddots    &    \vdots   \\ 
                &       &   1    &   0     
         \end{pmatrix}.
    \end{equation}
    Ensuite le changement de base (qui est une similitude) \( (e_1,\ldots, e_n)\mapsto(e_n,\ldots, e_1)\) montre que \( C_{X^m}\) est semblable à un bloc de Jordan \( J_m(0)\).

    Supposons à présent que \( f\) ne soit pas nilpotente. Par l'hypothèse de polynôme caractéristique scindé, nous supposons que \( f\) a \( m\) valeurs propres distinctes et que son polynôme caractéristique est
    \begin{equation}
        \chi_f=(X-\lambda_1)^{l_1}\ldots (X-\lambda_m)^{l_m}.
    \end{equation}
    Le lemme des noyaux (théorème \ref{ThoDecompNoyayzzMWod}) nous enseigne que
    \begin{equation}
        E=\bigoplus_{i=1}^m\underbrace{\ker(f-\mu_i\mtu)^{l_i}}_{F_i}.
    \end{equation}
    La restriction de \( f-\lambda_i\mtu\) à \( F_i\) est par construction un endomorphisme nilpotent, et donc peut s'écrire comme un bloc de Jordan avec des zéros sur la diagonale. En utilisant la décomposition
    \begin{equation}
        f|_{F_i}=(f-\lambda_i\mtu)|_{F_i}+\lambda_i\mtu_{F_i},
    \end{equation}
    nous voyons que \( f|_{F_i}\) s'écrit comme un bloc de Jordan avec \( \lambda_i\) sur la diagonale.
\end{proof}

\begin{remark}
    Nous pouvons calculer la forme normale de Jordan pour une matrice complexe ou réelle, mais dans les deux cas nous devons nous attendre à obtenir une matrice complexe parce que les valeurs propres d'une matrice réelle peuvent être complexes. Cependant nous demandons que le polynôme caractéristique de \( f\) soit scindé sur \( \eK\). En pratique, la décomposition de Jordan n'est garantie que sur les corps algébriquement clos, c'est à dire sur \( \eC\).

    La suite des invariants de similitude sur laquelle repose Frobenius, elle, est disponible sur tout corps, y compris \( \eR\).
\end{remark}

%+++++++++++++++++++++++++++++++++++++++++++++++++++++++++++++++++++++++++++++++++++++++++++++++++++++++++++++++++++++++++++ 
\section{Exponentielle de matrice}
%+++++++++++++++++++++++++++++++++++++++++++++++++++++++++++++++++++++++++++++++++++++++++++++++++++++++++++++++++++++++++++
%\label{subsecAOnIwQM}
\label{secAOnIwQM}

L'\defe{exponentielle}{exponentielle!de matrice} d'une matrice est la limite
\begin{equation}
    \exp(A)=\mtu+A+\frac{ A^2 }{ 2 }+\frac{ A^3 }{ 3 }+\ldots =\sum_{k=1}^{\infty}\frac{ A^k }{ k! }.
\end{equation}
Étant donné que c'est une limite, il y a une question de convergence et donc de topologie. C'est pour cela que nous ne pouvons pas introduire l'exponentielle de matrice avant d'avoir introduit la norme des matrices. La convergence de la série pour toute matrice sera prouvée au passage dans la proposition \ref{PropFMqsIE}.

En ce qui concerne la continuité, nous aurons évidemment besoin de théorie à propos de l'inversion de limites et de sommes. Nous en parlerons donc en \ref{subsecXNcaQfZ}.

La fonction exponentielle \(  x\mapsto e^{x}\) n'est pas un polynôme en \( x\), mais nous avons les résultat marrant suivant.
\begin{proposition} \label{PropFMqsIE}
    Si \( u\) est un endomorphisme, alors \( \exp(u)\) est un polynôme en \( u\)\footnote{Nan, mais j'te jure : \( \exp\) n'est pas un polynôme, mais $\exp(u)$ est un polynôme de \( u\).}.
\end{proposition}

\begin{proof}
    Nous considérons l'application
    \begin{equation}
        \begin{aligned}
            \varphi_u\colon \eK[X]&\to \End(E) \\
            P&\mapsto P(u)
        \end{aligned}
    \end{equation}
    déjà introduite en \eqref{EqOVKooeMJuv}. Étant donné que l'image de \( \varphi_u\) est un fermé dans \( \End(E)\), il suffit de montrer que la série
    \begin{equation}
        \sum_{k=0}^{\infty}\frac{ \varphi_u(X)^k }{ k! }
    \end{equation}
    converge dans \( \End(E)\) pour qu'elle converge dans \( \Image(\varphi_u)\). Pour ce faire nous nous rappelons de la norme opérateur \eqref{ExemdefnormpMrt} et de la propriété fondamentale \( \| A^k \|\leq \| A \|^k\). En notant \( A=\varphi_u(X)\) et en utilisant l'inégalité \eqref{EqFwTvwI},
    \begin{equation}
        \left\| \sum_{k=n}^m\frac{ A^k }{ k! } \right\|\leq \sum_{k=n}^m\frac{ \| A^k \| }{ k! }\leq \sum_{k=n}^m\frac{ \| A \|^k }{ k! },
    \end{equation}
    ce qui est une morceau du développement de \(  e^{\| A \|}\). La limite \( n\to\infty\) est donc zéro par la convergence de l'exponentielle réelle. La suite des sommes partielles de  $e^{A}$ est donc de Cauchy. La série converge donc parce que nous sommes dans un espace vectoriel réel de dimension finie (\( \End(E)\)).
\end{proof}
% TODO : et tant qu'on y est, justifier la convergence de la série de l'exponentielle réelle.

\begin{remark}
    Pourquoi \( \exp(u)\) est-il un polynôme d'endomorphisme alors que \( \exp\) n'est pas un polynôme ? Lorsque nous disons que la fonction \( x\mapsto \exp(x)\) n'est pas un polynôme, nous sommes en train de localiser la fonction \( \exp\) à l'intérieur de l'espace de toutes les fonctions \( \eR\to \eR\), c'est à dire à l'intérieur d'un espace de dimension infinie. Au contraire lorsqu'on parle de \( \exp(u)\) et qu'on le compare aux endomorphismes \( P(u)\), nous sommes en train de repérer \( \exp(u)\) à l'intérieur de l'espace des matrices qui est de dimension finie. Il n'est donc pas étonnant que l'on parvienne moins à faire la distinction.

    Si par contre nous considérons \( \exp\) en tant qu'application \( \exp\colon \End(E)\to \End(E)\), ce n'est pas un polynôme.

    Si \( u\) et \( v\) sont des endomorphismes, nous aurons des polynômes \( P\) et \( Q\) tels que \( e^u=P(u)\) et \( e^v=Q(v)\); mais nous n'aurons en général évidemment pas \( P=Q\). En cela, \( \exp\) n'est pas un polynôme.
\end{remark}

Nous reprenons l'exemple de \cite{MneimneReduct}. Soit \( A\) une matrice dont le polynôme minimum s'écrit
\begin{equation}
    P(X)=(X-1)^2(X-2).
\end{equation}
Par le théorème \ref{ThoDecompNoyayzzMWod} de décomposition des noyaux nous avons
\index{théorème!décomposition des noyaux!et exponentielle de matrice}
\begin{equation}
    E=\ker(A-1)^2\oplus\ker(A-2).
\end{equation}
En suivant les notations de ce théorème nous avons \( P_1(X)=(X-1)^2\), \( P_2(X)=X-2\) et
\begin{subequations}
    \begin{align}
        Q_1(X)&=X-2\\
        Q_2(X)&=(X-1)^2.
    \end{align}
\end{subequations}
Les polynômes \( R_i\) dont l'existence est assurée par le théorème de Bézout sont
\begin{equation}
    \begin{aligned}[]
        R_1(X)&=-X\\
        R_2(X)&=1.
    \end{aligned}
\end{equation}
Nous avons
\begin{equation}
    R_1Q_1+R_2Q_2=1.
\end{equation}
Le projecteur \( p_i\) sur \( \ker P_i\) est \( R_iQ_i\) :
\begin{equation}
    \begin{aligned}[]
        p_1&=-A(A-2)=\pr_{\ker(u-1)^2}\\
        p_2&=(A-1)^2=\pr_{\ker(u-2)}.
    \end{aligned}
\end{equation}
Passons maintenant au calcul de l'exponentielle. Nous avons évidemment
\begin{equation}
    e^A=e^Ap_1+e^Ap_2.
\end{equation}
Étant donné que \( p_1\) est le projecteur sur le noyau de \( (A-1)^2\), nous avons
\begin{equation}
    e^Ap_1=ee^{A-1}p_1=ep_1+e(u-1)1=ep_1=-Ae(A-2).
\end{equation}
En effet \( e^{A-1}p_1=\sum_{k=0}^{\infty}(A-1)^k\circ p_1\). De la même façon nous avons
\begin{equation}
    e^Ap_2=e^2e^{A-2}p_2=e^2p_2=e^2(A-1)^2.
\end{equation}
Au final,
\begin{equation}
    e^A=-Ae(A-2)+e^2(A-1)^2.
\end{equation}

\begin{theorem}
    Soit une matrice \( A\in \eM(n,\eC)\). On a que la suite \( (A^kx)\) tends vers zéro pour tout \( x\) si et seulement si \( \rho(A)<1\) où \( \rho(A)\)\index{rayon!spectral} est le rayon spectral de $A$
\end{theorem}
\index{décomposition!Dunford!exponentielle de matrice}

\begin{proof}
    Dans le sens direct, il suffit de prendre comme \( x\), un vecteur propre de \( A\). Dans ce cas nous avons \( A^kx=\lambda^kx\). Mais \( \lambda^kx\) ne tend vers zéro que si \( \lambda<1\). Donc toute les valeurs propres de \( A\) doivent être plus petite que \( 1\) et \( \rho(A)<1\).

    Pour l'autre sens nous utilisons la décomposition de Dunford (théorème \ref{ThoRURcpW}) : il existe une matrice inversible \( P\) telle que
    \begin{equation}
        A=P^{-1}(D+N)P
    \end{equation}
    où \( D\) est diagonale, \( N\) est nilpotente et \( [D,N]=0\). Étant donné que \( D+N\) est triangulaire, son polynôme caractéristique que
    \begin{equation}
        \chi_{D+N}(\lambda)=\prod_i D_{ii}-\lambda.
    \end{equation}
    Par similitude, c'est le même polynôme caractéristique que celui de \( A\) et nous savons alors que la diagonale de \( D\) contient les valeurs propres de \( A\).

    Par ailleurs nous avons
    \begin{subequations}
        \begin{align}
            A^k&=P^{-1}(D+N)^kP\\
            &=P^{-1}\sum_{j=0}^k{j\choose k}D^{j-k}N^jP\\
            &=P^{-1}\sum_{j=0}^{n-1}{j\choose k}D^{j-k}N^jP
        \end{align}
    \end{subequations}
    où nous avons utilité le fait que \( D\) et \( N\) commutent ainsi que \( N^{n-1}=0\) parce que \( N\) est nilpotente. Nous utilisons la norme matricielle usuelle, pour laquelle \( \| D \|=\rho(D)=\rho(A)\). Nous avons alors
    \begin{equation}
        \| (D+N)^k \|\leq \sum_{j=0}^k{j\choose k}\rho(D)^{k-j}\| N \|^j.
    \end{equation}
    Du coup si \( \rho(D)<1\) alors \( \| (D+N)^k \|\to 0\) (et c'est même un si et seulement si).
\end{proof}

Une application de la décomposition de Jordan est l'existence d'un logarithme pour les matrices. La proposition suivant va d'une certaine manière donner un logarithme pour les matrices inversibles complexes. Dans le cas des matrices réelles \( m\) telles que \( \| m-\mtu \|<1\), nous donnerons au lemme \ref{LemQZIQxaB} une formule pour le logarithme sous forme d'une série; ce logarithme sera réel.
\begin{proposition} \label{PropKKdmnkD}
    Toute matrice inversible complexe est une exponentielle.
\end{proposition}
\index{exponentielle!de matrice}
\index{décomposition!Jordan!et exponentielle de matrice}

\begin{proof}
    Soit \( A\in \GL(n,\eC)\); nous allons donner une matrice \( B\in \eM(n,\eC)\) telle que \( A=\exp(B)\). D'abord remarquons qu'il suffit de prouver le résultat pour une matrice par classe de similitude. En effet si \( A=\exp(B)\) et si \( M\) est inversible alors 
    \begin{subequations}    \label{EqqACuGK}
        \begin{align}
            \exp(MBM^{-1})&=\sum_k\frac{1}{ k! }(MBM^{-1})^k\\
            &=\sum_k\frac{1}{ k! }MB^kM^{-1}\\
            &=M\exp(B)M^{-1}.
        \end{align}
    \end{subequations}
    Donc \( MAM^{-1}=\exp(MBM^{-1})\). Nous pouvons donc nous contenter de trouver un logarithme pour les blocs de Jordan. Nous supposons donc que \( A=(\mtu+N)\) avec \( N^m=0\). En nous inspirant de \eqref{EqweEZnV}, nous posons
    \begin{equation}
        D(t)=tN-\frac{ t^2 }{ 2 }N^2+\ldots +(-1)^m\frac{ t^{m-1} }{ m-1 }N^{m-1}
    \end{equation}
    et nous allons prouver que \(  e^{D(1)}=\mtu+N\). Notons que \( N\) étant nilpotente, cette somme ainsi que toutes celles qui viennent sont finies. Il n'y a donc pas de problèmes de convergences dans cette preuve (si ce n'est les passages des équations \eqref{EqqACuGK}).

    Nous posons \( S(t)= e^{D(t)}\) (la somme est finie), et nous avons
    \begin{equation}
        S'(t)=D'(t) e^{D(t)}
    \end{equation}
    Afin d'obtenir une expression qui donne \( S'\) en termes de \( S\), nous multiplions par \( (\mtu+tN)\) en remarquant que \( (\mtu+tN)D'(t)=N\) nous avons
    \begin{equation}
        (\mtu+tN)S'(t)=NS(t).
    \end{equation}
    En dérivant à nouveau,
    \begin{equation}    \label{EqKjccqP}
        (\mtu+tN)S''(t)=0.
    \end{equation}
    La matrice \( (\mtu+tN)\) est inversible parce que son noyau est réduit à \( \{ 0 \}\). En effet si \( (\mtu+tN)x=0\), alors \( Nx=-\frac{1}{ t }x\), ce qui est impossible parce que \( N\) est nilpotente. Ce que dit l'équation \eqref{EqKjccqP} est alors que \( S''(t)=0\). Si nous développons \( S(t)\) en puissances de \( t\) nous nous arrêtons au terme d'ordre \( 1\) et nous avons
    \begin{equation}
        S(t)=S(0)+tS'(0)=\mtu+tD'(0)=1+tN.
    \end{equation}
    En \( t=1\) nous trouvons \( S(1)=\mtu+N\). La matrice \( D(1)\) donnée est donc bien un logarithme de $\mtu+N$.
\end{proof}

\begin{proposition}[\cite{fJhCTE}]
    Si \( A\in \eM(n,\eR)\) a un polynôme caractéristique scindé, alors \( A\) est diagonalisable si et seulement si \( e^A\) est diagonalisable.
\end{proposition}
\index{décomposition!Dunford!application}
\index{exponentielle!de matrice}
\index{diagonalisable!exponentielle}

\begin{proof}
    Si \( A\) est diagonalisable, alors il existe une matrice inversible \( M\) telle que \( D=M^{-1}AM\) soit diagonale (c'est la définition \ref{DefCNJqsmo}). Dans ce cas nous avons aussi \( (M^{-1}AM)^k=M^{-1}A^kM\) et donc \( M^{-1}e^AM=e^{M^{-1}AM}=e^D\) qui est diagonale.

    La partie difficile est donc le contraire. 
    
    \begin{subproof}
        \item[Qui est diagonalisable et comment ?]
            Nous supposons que \( e^A\) est diagonalisable et nous écrivons la décomposition de Dunford (théorème \ref{ThoRURcpW}) :
            \begin{equation}
                A=S+N
            \end{equation}
            où \( S\) est diagonalisable, \( N\) est nilpotente, \( [S,N]=0\). Nous avons besoin de prouver que \( N=0\).
    
            Les matrices \( A\) est \( S\) commutent; en passant au développement nous en déduisons que \( A\) et \( e^S\) commutent, puis encore en passant au développement que \( e^A\) et \( e^S\) commutent. Vu que \( S\) est diagonalisable, \( e^S\) l'est et par hypothèse \( e^A\) est également diagonalisable. Donc \( e^A\) et \( e^{-S}\) sont simultanément diagonalisables par la proposition \ref{PropGqhAMei}.

            Étant donné que \( A\) et \( S\) commutent, nous avons \( e^N=e^{A-S}=e^Ae^{-S}\), et nous en déduisons que \( e^N\) est diagonalisable vu que les deux facteurs \( e^A\) et \( e^{-S}\) sont simultanément diagonalisables.

        \item[Unipotence]

            Si \( r\) est le degré de nilpotence de \( N\), nous avons
            \begin{equation}    \label{EqQHjvLZQ}
                e^N-\mtu=N+\frac{ N^2 }{2}+\ldots +\frac{ N^{r-1} }{ (r-1)! }.
            \end{equation}
            Donc
            \begin{equation}
                (e^N-\mtu)^k=\left( N+\frac{ N^2 }{2}+\ldots +\frac{ N^{r-1} }{ (r-1)! } \right)^k
            \end{equation}
            où le membre de droite est un polynôme en \( N\) dont le terme de plus bas degré est de degré \( k\). Donc \( (e^N-\mtu)\) est nilpotente et \( e^N\) est unipotente.

            Si \( M\) est la matrice qui diagonalise \( e^N\), alors la matrice diagonale \( M^{-1}e^NM\) est tout autant unipotente que \( e^N\) elle-même. En effet,
            \begin{subequations}
                \begin{align}
                    (M^{-1}e^NM-\mtu)^r&=\sum_{k=0}^r\binom{ r }{ k }(-1)^{r-k}M^{-1}(e^N)^kM\\
                    &=M^{-1}\left( \sum_{k=0}^r\binom{ r }{ k }(-1)^{r-k}(e^N)^k \right)M\\
                    &=M^{-1}(e^N-\mtu)^rM\\
                    &=0.
                \end{align}
            \end{subequations}

            La matrice \( M^{-1}e^NM\) est donc une matrice diagonale et unipotente; donc \( M^{-1}e^NM=\mtu\), ce qui donne immédiatement que \( e^N=\mtu\).

        \item[Polynômes annulateurs]

            En reprenant le développement \eqref{EqQHjvLZQ} sachant que \( e^N=\mtu\), nous savons que
            \begin{equation}
                N+\frac{ N^2 }{2}+\ldots +\frac{ N^{r-1} }{ (r-1)! }=0.
            \end{equation}
            Dit en termes pompeux (mais non moins porteurs de sens), le polynôme
            \begin{equation}
                Q(X)=X+\frac{ X^2 }{2}+\ldots +\frac{ X^{r-1} }{ (r-1)! }
            \end{equation}
            est un polynôme annulateur de \( N\).
            
            La proposition \ref{PropAnnncEcCxj} stipule que le polynôme minimal d'un endomorphisme divise tous les polynômes annulateurs. Dans notre cas, \( X^r\) est un polynôme annulateur et donc le polynôme minimal de \( N\) est de la forme \( X^k\). Donc il est \( X^r\) lui-même.
            
            Nous avons donc \( X^r\divides Q\). Mais \( Q\) est un polynôme contenant le monôme \( X\) donc \( X^r\) ne peut diviser \( Q\) que si \( r=1\). Nous en concluons que \( X\) est un polynôme annulateur de \( N\). C'est à dire que \( N=0\).

        \item[Conclusion]

            Vu que Dunford dit que \( A=S+N\) et que nous venons de prouver que \( N=0\), nous concluons que \( A=S\) avec \( S\) diagonalisable.

    \end{subproof}
\end{proof}

\begin{proposition}[\cite{fJhCTE}]
    Si \( A\in \eM(n,\eC)\) est telle que \( \rho(A)<1\), alors \( A^n\to 0\).
\end{proposition}

\begin{proof}
    Nous nous plaçons dans une base des espaces caractéristiques de \( A\), c'est à dire que nous supposons que la matrice \( A\) a la forme
    \begin{equation}        \label{EqWMvkgLo}
        A=\begin{pmatrix}
            \lambda_1\mtu+N_1    &       &       \\
                &   \ddots    &       \\
                &       &   \lambda_s\mtu+N_s
        \end{pmatrix}
    \end{equation}
    où les \( \lambda_i\) sont les valeurs propres de \( A\) et les \( N_i\) sont nilpotentes. En effet nous savons que l'espace caractéristique \( F_{\lambda_i}\) est l'espace de nilpolence de \( A-\lambda_i\mtu\). Si nous notons \( A_i\) la restriction de \( A\) à cet espace, la matrice \( N_i=A_i-\lambda_i\mtu\) est nilpotente. Du coup \( A_i=\lambda_I\mtu+N_i\) et nous avons bien la décomposition \eqref{EqWMvkgLo}.

    Nous avons donc \( A^n\to 0\) si et seulement si \( (N_i+\lambda_i\mtu)^n\to 0\) pour tout \( i\). Soit donc \( N\) nilpotente et \( \lambda<1\) (parce que nous savons que toutes les valeurs propres de \( A\) sont inférieures à un). Nous avons
    \begin{equation}
            (\lambda\mtu+N)^n=\sum_{k=0}^n\binom{ n }{ k }\lambda^{n-k}N^{k}
            =\sum_{k=0}^{r-1}\binom{ n }{ k }\lambda^{n-k}N^{k}.
    \end{equation}
    Nous voyons que le nombre de termes dans la somme ne dépend pas de \( n\). De plus pour chacun de termes, la puissance de \( N\) ne dépend pas non plus de \( n\). Le terme
    \begin{equation}
        \binom{ n }{ k }\lambda^{n-k}\leq P(n)\lambda^{n-k}
    \end{equation}
    où \( P\) est un polynôme tend vers zéro lorsque \( n\) devient grand parce que c'est une cas polynôme fois exponentielle.
\end{proof}

\begin{proposition}
    Soit \( A\in\GL(n,\eC)\). La suite \( (A^k)_{k\in \eZ}\) est bornée si et seulement si \( A\) est diagonalisable et \( \Spec(A)\subset \gS^1\).
\end{proposition}

\begin{proof}
    Si \( A\) est diagonalisable avec les valeurs propres \( \lambda_i\) de norme \( 1\) dans \( \eC\), alors \( A^k\) est la matrice diagonale avec les \( \lambda_i^k\) sur la diagonale. Cela reste borné pour toute valeur entière de \( k\).

    En ce qui concerne l'autre sens, nous supposons encore que
    \begin{equation}
        A=\begin{pmatrix}
            \lambda_1\mtu+N_1    &       &       \\
                &   \ddots    &       \\
                &       &   \lambda_s\mtu+N_s
        \end{pmatrix},
    \end{equation}
    et nous regardons un des blocs. Nous voulons prouver que \( N=0\) et que \( | \lambda |=1\).
    
    Nous commençons par regarder ce qu'implique le fait que \( (\lambda \mtu+N)^n\) reste borné pour \( n>0\). En notant \( r\) l'ordre de nilpotence de \( N\), nous avons le développement
    \begin{equation}
        (\lambda\mtu+N)^n=\sum_{k=0}^{r-1}\binom{ n }{ k }N^k\lambda^{n-k}.
    \end{equation}
    Une matrice nilpotente s'écrit dans une base sous la forme
    \begin{equation}
        N=\begin{pmatrix}
             0   &   1    &       &       \\
                &   0    &   1    &       \\
                & &   \ddots   &   \ddots    &      \\ 
                &&       &   0    &   1     \\
                &&       &      &   0     
         \end{pmatrix}
    \end{equation}
    et effectuer \( A^k\) revient à décaler la diagonale de \( 1\). Donc la famille
    \begin{equation}
        \{ \mtu,N,\ldots, N^{r-1} \}
    \end{equation}
    est libre. Par conséquent la suite \( (\lambda\mtu+N)^n\) restera bornée si et seulement si chacun des termes 
    \begin{equation}    \label{EqXRDVDCM}
        \binom{ n }{ k }N^k\lambda^{n-k}
    \end{equation}
    reste borné. Le premier terme étant \( \lambda^n\mtu\), nous avons obligatoirement \( | \lambda |\leq 1\). Si \( | \lambda |<1\), alors le coefficient \( \binom{ n }{ k }\lambda^{n-k}\) tend vers zéro. Si \( | \lambda |=1\) par contre ce coefficient tend vers l'infini et la seule façon pour que \eqref{EqXRDVDCM} reste borné est que \( N=0\). Nous avons donc deux possibilités :
    \begin{itemize}
        \item \( | \lambda |<1\)
        \item \( | \lambda |=1\) et \( N=0\).
    \end{itemize}

    Nous nous tournons maintenant sur la contrainte que \( (\lambda\mtu+N)^n\) doive rester borné pour \( n<0\). Nous avons
    \begin{equation}
        \lambda\mtu+N=\lambda(\mtu+\lambda^{-1}N),
    \end{equation}
    et nous pouvons appliquer la proposition \ref{PropQAjqUNp} à l'opérateur nilpotent \( -\lambda^{-1} N\) pour avoir
    \begin{equation}
        (\mtu+\lambda^{-1}N)^{-1}=\mtu+\sum_{k=1}^{\infty}(-\lambda)^{-1}N^k.
    \end{equation}
    Ceci pour dire que \( (\lambda\mtu+N)^{-1}=\lambda^{-1}(\mtu+\lambda^{-1}N')\) pour une autre matrice nilpotente \( N'\). Le travail déjà fait, appliqué à \( \lambda^{-1}\) et \( N'\), nous donne deux possibilités :
    \begin{itemize}
        \item \( | \lambda^{-1} |<1\)
        \item \( | \lambda^{-1} |=1\) et \( N'=0\).
    \end{itemize}
    La possibilité \( | \lambda^{-1} |<1\) est exclue parce qu'elle impliquerait \( | \lambda |>1\) qui avait déjà été exclu. Il ne reste donc que la possibilité \( | \lambda |=1\) et \( N=N'=0\).
\end{proof}

%+++++++++++++++++++++++++++++++++++++++++++++++++++++++++++++++++++++++++++++++++++++++++++++++++++++++++++++++++++++++++++
\section{Mini introduction au produit tensoriel}
%+++++++++++++++++++++++++++++++++++++++++++++++++++++++++++++++++++++++++++++++++++++++++++++++++++++++++++++++++++++++++++
\label{SeOOpHsn}

%---------------------------------------------------------------------------------------------------------------------------
\subsection{Définitions}
%---------------------------------------------------------------------------------------------------------------------------

Soit \( E\), un espace vectoriel de dimension finie. Si \( \alpha\) et \( \beta\) sont deux formes linéaires sur un espace vectoriel \( E\), nous définissons \( \alpha\otimes \beta\) comme étant la \( 2\)-forme donnée par
\begin{equation}
    (\alpha\otimes \beta)(u,v)=\alpha(u)\beta(v).
\end{equation}
Si \( a\) et \( b\) sont des vecteurs de \( E\), ils sont vus comme des formes sur \( E\) via le produit scalaire et nous avons
\begin{equation}
    (a\otimes b)(u,v)=(a\cdot u)(b\cdot v).
\end{equation}
Cette dernière équation nous incite à pousser un peu plus loin la définition de \( a\otimes b\) et de simplement voir cela comme la matrice de composantes
\begin{equation}
    (a\otimes b)_{ij}=a_ib_j.
\end{equation}
Cette façon d'écrire a l'avantage de ne pas demander de se souvenir qui est une vecteur ligne, qui est un vecteur colonne et où il faut mettre la transposée. Évidemment \( (a\otimes b)\) est soit \( ab^t\) soit \( a^tb\) suivant que \( a\) et \( b\) soient ligne ou colonne.

\begin{lemma}   \label{LemMyKPzY}
    Soient \( x,y\in E\) et \( A,B\) deux opérateurs linéaires sur \( E\) vus comme matrices. Alors
    \begin{equation}        \label{EqXdxvSu}
        (Ax\otimes By)=A(x\otimes y)B^t.
    \end{equation}
\end{lemma}

\begin{proof}
    Calculons la composante \( ij\) de la matrice \( (Ax\otimes By)\). Nous avons
    \begin{subequations}
        \begin{align}
            (Ax\otimes By)_{ij}&=(Ax)_i(By)_j\\
            &=\sum_{kl}A_{ik}x_kB_{jl}y_l\\
            &=A_{ik}(x\otimes y)_{kl}B_{jl}\\
            &=\big( A(x\otimes y)B^t \big)_{ij}.
        \end{align}
    \end{subequations}
\end{proof}

%+++++++++++++++++++++++++++++++++++++++++++++++++++++++++++++++++++++++++++++++++++++++++++++++++++++++++++++++++++++++++++
\section{Formes bilinéaires et quadratiques}
%+++++++++++++++++++++++++++++++++++++++++++++++++++++++++++++++++++++++++++++++++++++++++++++++++++++++++++++++++++++++++++
\label{SecTQkRXIu}

%--------------------------------------------------------------------------------------------------------------------------- 
\subsection{Généralités}
%---------------------------------------------------------------------------------------------------------------------------

Les applications multilinéaires ont déjà été définies en la définition \ref{DefFRHooKnPCT}; nous donnons ici une définition plus explicite dans le cas des applications bilinéaires.
\begin{definition}
    Une \defe{forme bilinéaire}{forme!bilinéaire} sur un espace vectoriel \( E\) est une application \( b\colon E\times E\to \eK\) telle que
    \begin{enumerate}
        \item
            \( b(u,v)=b(v,u)\),
        \item
            \( b(u+v,w)=b(u,w)+b(v,w)\),
        \item
            \( b(\lambda u,v)=\lambda b(u,v)\)
    \end{enumerate}
    pour tout \( u,v,w\in E\) et \( \lambda\in \eK\) où \( \eK\) est une corps commutatif.
\end{definition}

\begin{definition}[\cite{RUAoonJAym}]   \label{DefBSIoouvuKR}
    Soit un espace vectoriel \( E\) et \( \eF\) un corps de caractéristique différente de \( 2\). Une \defe{forme quadratique}{forme!quadratique} sur \( E\) est une application \( q\colon V\to \eF\) pour laquelle il existe une forme bilinéaire symétrique \( b\colon V\times V\to \eF\) satisfaisant \( q(x)=b(x,x)\) pour tout \( x\in V\).

    L'ensemble des formes quadratiques réelles sur \( E\) est noté \( Q(E)\)\nomenclature[B]{\( Q(E)\)}{formes quadratiques réelles sur \( E\)}.
\end{definition}

\begin{lemma}
    Si \( q\) est une forme quadratique, il existe une unique forme bilinéaire \( b\) telle que \( q(x)=b(x,x)\).
\end{lemma}

\begin{proof}
    L'existence n'est pas en cause : c'est la définition d'une forme quadratique. Pour l'unicité, étant donné une forme quadratique, la forme bilinéaire \( b\) doit forcément vérifier l'\defe{identités de polarisation}{identité!polarisation}\index{polarisation (identité)} :
\begin{equation}    \label{EqMrbsop}
    b(x,y)=\frac{ 1 }{2}\big( q(x)+q(y)-q(x-y) \big).
\end{equation}
Elle est donc déterminée par \( q\).
\end{proof}
Notons la division par \( 2\) qui est le pourquoi de la demande de la caractéristique différente de \( 2\) pour \( \eF\) dans la définition de forme quadratique.

%--------------------------------------------------------------------------------------------------------------------------- 
\subsection{Topologie}
%---------------------------------------------------------------------------------------------------------------------------

La topologie considérée sur \( Q(E)\) est celle de la norme
\begin{equation}
    N(q)=\sup_{\| x \|_E=1}| q(x) |,
\end{equation}
qui du point de vue de \( S_n(\eR)\) est
\begin{equation}    
    N(A)=\sup_{\| x \|_E=1}| x^tAx |.
\end{equation}
Notons que à droite, c'est la valeur absolue usuelle sur \( \eR\).

%--------------------------------------------------------------------------------------------------------------------------- 
\subsection{Matrice associée}
%---------------------------------------------------------------------------------------------------------------------------

Si une base \( \{ e_i \}_{i=1,\ldots, n}\) de l'espace vectoriel \( E\) est donnée, la \defe{matrice associée}{matrice!associée à une forme quadratique}\index{forme!quadratique!matrice associée} à la forme bilinéaire \( b\) sur \( E\) est la matrice d'éléments
\begin{equation}
    B_{ij}=b(e_i,e_j).
\end{equation}
Notons que la matrice associée à une forme bilinéaire (ou quadratique associée) est uniquement valable pour une base donnée. Si nous changeons de base, la matrice change. Cependant lorsque nous travaillons sur \( \eR^n\), la base canonique est tellement canonique que nous allons nous permettre de parler de «la» matrice associée à une forme bilinéaire. 

Si \( B_{ij}\) est la matrice associée à la forme bilinéaire \( b\) alors la valeur de \( b(u,v)\) se calcule avec la formule
\begin{equation}
    b(x,y)=\sum_{i,j}B_{ij}x_iy_j
\end{equation}
lorsque \( x_i\) et \( y_j\) sont les coordonnées de \( x\) et \( y\) dans la base choisie.

\begin{proposition} \label{PropFSXooRUMzdb}
    Soit \( \{ e_i \}\) une base de \( E\). L'application
    \begin{equation}
        \begin{aligned}
            \phi\colon Q(E)&\to S(n,\eR) \\
            q&\mapsto \big(   b(e_i,e_j)   \big)_{i,j}
        \end{aligned}
    \end{equation}
    où \( b\) est forme bilinéaire associée à \( q\) est une bijection linéaire et continue.
\end{proposition}

\begin{proof}
    Si \( \phi(q)=\phi(q')\); alors
    \begin{equation}
        q(x)=\sum_{i,j}\phi(q)_{ij}x_ix_j=\sum_{i,j}\phi(q')_{ij}x_ix_j=q'(x).
    \end{equation}
    Donc \( q=q'\). L'application \( \phi\) est donc injective

    De plus elle est surjective parce que si \( B\in S(n,\eR)\) alors la forme quadratique
    \begin{equation}
        q(x)=\sum_{i,j}B_{ij}x_ix_j
    \end{equation}
    a évidemment \( B\) comme matrice associée. L'application \( \phi\) est donc surjective.

    Notre application \( \phi\) est de plus linéaire parce que l'association d'une forme quadratique à la forme bilinéaire associée est linéaire.

    En ce qui concerne la continuité, nous la prouvons en zéro en considérant une suite convergente \( q_n\stackrel{Q(E)}{\longrightarrow}0\). C'est à dire que
    \begin{equation}
        \sup_{\| x \|=1}| q_n(x) |\to 0.
    \end{equation}
    Nous rappelons l'identité de polarisation : 
    \begin{equation}
        b_n(x,y)=\frac{ 1 }{2}\big( q_n(x-y)-q(x)-q(y) \big).
    \end{equation}
    En ce qui concerne deux des trois termes, il n'y a pas de problèmes :
    \begin{equation}
        \big| \phi(q_n)_{ij} \big|=\big| b_n(e_i,e_j) \big|\leq\frac{ 1 }{2}\big| b_n(e_i-e_j) \big|+\frac{ 1 }{2}\big| q_n(e_i) \big|+\frac{ 1 }{2}\big| q_n(e_j) \big|.
    \end{equation}
    Si \( n\) est assez grand, nous avons tout de suite
    \begin{equation}
        \big| \phi(q_n)_{ij} \big|\leq \frac{ 1 }{2}\big| q_n(e_i-e_j) \big|+\epsilon.
    \end{equation}
    Nous définissons \( e_{ij}\) et \( \alpha_{ij}\) de telle sorte que \( e_i-e_j=\alpha_{ij}e_{ij}\) avec \( \| e_{ij} \|=1\). Si \( \alpha=\max\{ \alpha_{ij},1 \}\) alors nous avons
    \begin{equation}
        q_n(e_i-e_j)=\alpha_{ij}^2q_n(e_{ij})\leq \alpha^2q_n(e_{ij}).
    \end{equation}
    Il suffit maintenant de prendre \( n\) assez grand pour avoir \( \sup_{\| x \|=1}| q_n(x) |\leq \frac{ \epsilon }{ \alpha^2 }\) pour avoir
    \begin{equation}
        \big| \phi(q_n)_{ij} \big|\leq \frac{ \epsilon }{2}+\frac{ \epsilon }{ \alpha^2 }.
    \end{equation}
\end{proof}

\begin{proposition}\label{PropFWYooQXfcVY}
    Dans la base de diagonalisation de sa matrice associée, une forme quadratique a la forme
    \begin{equation}
        q(x)=\sum_i\lambda_ix_i^2
    \end{equation}
    où les \( \lambda_i\) sont les valeurs propres de la matrice associée à \( q\).
\end{proposition}

\begin{proof}
Soit \( q\) une forme quadratique et \( b\) la forme bilinéaire associée. Si \( \{ f_i \}\) est une base de diagonalisation de la matrice de \( b\) alors dans cette base nous avons
\begin{equation}
    q(x)=b(x,x)=\sum_{ij}x_ix_jb(f_i,f_j)=\sum_i\lambda_ix_i^2
\end{equation}
où les \( \lambda_i\) sont les valeurs propres de la matrice de \( b\).
\end{proof}
Notons que si nous choisissons une autre base de diagonalisation, les \( \lambda_i\) ne changement pas (à part l'ordre éventuellement). Cela pour dire que nous nous permettrons de parler des \defe{valeurs propres}{valeur propre!d'une forme quadratique} d'une forme quadratique comme étant les valeurs propres de la matrice associée.

%---------------------------------------------------------------------------------------------------------------------------
\subsection{Quelque mots à propos de matrices}
%---------------------------------------------------------------------------------------------------------------------------

Si $g$ est une application bilinéaire sur $\eR^2$, nous disons qu'elle est
\begin{enumerate}
\item
\defe{définie positive}{application!définie positive} si $g(u,u)\geq 0$ pour tout $u\in\eR^2$ et $g(u,u)=0$ si et seulement si $u=0$.
\item
\defe{semi-définie positive}{application!semi-définie positive} si $g(u,u)\geq 0$ pour tout $u\in\eR^2$. Nous dirons aussi parfois qu'elle est simplement «positive».
\end{enumerate}
Cela est évidemment à lier à la définition \ref{DefAWAooCMPuVM} : une application bilinéaires est définie positive si et seulement si sa matrice symétrique associée l'est.

\begin{proposition}     \label{PropcnJyXZ}
    Soit $M$, une matrice $2\times 2$ symétrique. Nous avons
    \begin{enumerate}
        \item
        $\det M>0$ et $\tr(M)>0$ implique $M$ définie positive,
        \item
        $\det M>0$ et $\tr(M)<0$ implique $M$ définie négative,
    \item   \label{ItemluuFPN}
        $\det M<0$ implique ni semi définie positive, ni définie négative 
        \item
        $\det M=0$ implique $M$ semi-définie positive ou semi-définie négative.
    \end{enumerate}
\end{proposition}

%--------------------------------------------------------------------------------------------------------------------------- 
\subsection{Dégénérescence}
%---------------------------------------------------------------------------------------------------------------------------

Soit \( b\), une forme bilinéaire symétrique non dégénérée  sur l'espace vectoriel \( E\) de dimension \( n\) sur \( \eK\) où \( \eK\) est un corps de caractéristique différente de \( 2\). Nous notons \( q\) la forme quadratique associée.

\begin{definition}
    Une forme bilinéaire est \defe{non dégénérée}{forme!bilinéaire!non dégénérée} \( b(x,z)=0\) pour tout \( z\) implique \( x=0\).
\end{definition}

\begin{lemma}   \label{LemyKJpVP}
    Soit \( b\) une forme bilinéaire non dégénérée. Si \( x\) et \( y\) sont tels que \( b(x,z)=b(y,z)\) pour tout \( z\), alors \( x=y\).
\end{lemma}

\begin{proof}
    C'est immédiat du fait de la linéarité en le premier argument et de la non-dégénérescence : si \( b(x,z)-b(y,z)=0\) alors
    \begin{equation}
        b(x-y,z)=0
    \end{equation}
    pour tout \( z\), ce qui implique \( x-y=0\).
\end{proof}

\begin{proposition}
    La forme bilinéaire \( b\) est non-dénénérée si et seulement si sa matrice associée est inversible.
\end{proposition}

\begin{proof}
    Nous savons que la matrice associée est symétrique et qu'elle peut donc être diagonalisée (théorème \ref{ThoeTMXla}). En nous plaçant dans une base de diagonalisation, nous devons prouver que la forme est non-dégénérée si et seulement si les éléments diagonaux de la matrice sont tous non nuls.

    Écrivons \( b(x,z)\) en choisissant pour \( z\) le vecteur de base \( e_k\) de composantes \( (e_k)_j=\delta_{kj}\) :
    \begin{equation}
            b(x,e_k)=\sum_{ij}x_i(e_k)_j
            =\sum_i b_{ik}x_i
            =b_{kk}x_k.
    \end{equation}
    Si \( b\) est dégénérée et si \( x\) est un vecteur non nul (disons que la composante \( x_i\) est non nulle) de \( E\) tel que \( b(x,z)=0\) pour tout \( z\in E\), alors \( b_{ii}=0\), ce qui montre que la matrice de \( b\) n'est pas inversible.

    Réciproquement si la matrice de \( b\) est inversible, alors tous les \( b_{kk}\) sont différents de zéro, et le seul vecteur \( x\) tel que \( b_{kk}x_k=0\) pour tout \( k\) est le vecteur nul.
\end{proof}


\begin{definition}[Isotropie]   \label{DefVKMnUEM}
    Un vecteur est \defe{isotrope}{isotrope (vecteur)} pour \( b\) si il est perpendiculaire à lui-même; en d'autres termes, \( x\) est isotrope si et seulement si \( b(x,x)=0\). Un sous-espace \( W\subset E\) est \defe{totalement isotrope}{isotrope!totalement} si pour tout \( x,y\in W\), nous avons \( b(x,y)=0\).

    Le \defe{cône isotrope}{isotrope!cône} de \( b\) est l'ensemble de ses vecteurs isotropes :
    \begin{equation}
        C(b)=\{ x\in E\tq b(x,x)=0 \}.
    \end{equation}
\end{definition}
Nous introduisons quelque notations. D'abord pour \( y\in E\) nous notons
\begin{equation}
    \begin{aligned}
        \Phi_y\colon E&\to \eR \\
        x&\mapsto b(x,y) 
    \end{aligned}
\end{equation}
et ensuite
\begin{equation}
    \begin{aligned}
        \Phi\colon E&\to E^* \\
        y&\mapsto \Phi_y. 
    \end{aligned}
\end{equation}
\begin{definition}
    Le fait pour une forme bilinéaire \( b\) d'être dégénérée signifie que l'application \( \Phi\) n'est pas injective. Le \defe{noyau}{noyau!d'une forme bilinéaire} de la forme bilinéaire est celui de \( \Phi\), c'est à dire
    \begin{equation}
        \ker(b)=\{ z\in E\tq b(z,y)=0\,\forall y\in E \}.
    \end{equation}
    Autrement dit, \( \ker(b)=E^{\perp}\) où le perpendiculaire est pris par rapport à \( b\).
\end{definition}
Notons tout de même que nous utilisons la notation \( \perp\) même si \( b\) est dégénérée et éventuellement pas positive; c'est à dire même si la formule \( (x,y)\mapsto b(x,y)\) ne fournit pas un produit scalaire.

\begin{proposition}[\cite{RTzQrdx}]     \label{PropHIWjdMX}
    Soit \( b\) une forme bilinéaire et symétrique. Alors
    \begin{enumerate}
        \item
            \( \ker(b)\subset C(b)\) (cône d'isotropie, définition \ref{DefVKMnUEM})
        \item
            si \( b\) est positive alors \( \ker(b)=C(b)\).
    \end{enumerate}
\end{proposition}

\begin{proof}
    \begin{enumerate}
        \item
            Si \( z\in\ker(b)\) alors pour tout \( y\in E\) nous avons \( b(z,y)=0\). En particulier pour \( y=z\) nous avons \( b(z,z,)=0\) et donc \( z\in C(b)\).
        \item
            Soit \( b\) positive et \( x\in C(b)\). Par l'inégalité de Cauchy-Schwarz (proposition \ref{ThoAYfEHG}) nous avons
            \begin{equation}
                | b(x,y) |\leq \sqrt{   b(x,x)b(y,y) }=0.
            \end{equation}
            Donc pour tout \( y\) nous avons \( b(x,y)=0\).
    \end{enumerate}
\end{proof}

%--------------------------------------------------------------------------------------------------------------------------- 
\subsection{Théorème spectral auto-adjoint}
%---------------------------------------------------------------------------------------------------------------------------

\begin{definition}
    Si \( E\) est un espace euclidien, un endomorphisme \( f\colon E\to E\) est \defe{auto-adjoint}{endomorphisme!auto-adjoint} si pour tout \( x,y\in E\) nous avons \( \langle x, f(y)\rangle=\langle f(x), Y\rangle  \). 
\end{definition}
L'ensemble des opérateurs auto-adjoints de \( E\) est noté \( \gS(E)\)\nomenclature[A]{\( \gS(E)\)}{Les opérateurs auto-adjoints de $E$}. Cette notation provient du fait que dans \( \eR^n\) muni du produit scalaire usuel, les opérateurs auto-adjoints sont les matrices symétriques.

\begin{theorem}[Théorème spectral auto-adjoint] \label{ThoRSBahHH}
    Un endomorphisme auto-adjoint d'un espace euclidien
    \begin{enumerate}
        \item
            est diagonalisable dans une base orthonormée,
        \item
            a son spectre réel.
    \end{enumerate}
\end{theorem}
\index{théorème!spectral!autoadjoint}
\index{diagonalisation!endomorphisme auto-adjoint}

\begin{proof}
    Nous procédons par récurrence sur la dimension de \( E\), et nous commençons par \( n=1\)\footnote{Dans \cite{KXjFWKA}, l'auteur commence avec \( n=0\) mais moi je n'en ai \wikipedia{en}{Vacuous_truth}{pas le courage.}.}. Soit donc \( f\colon E\to E\) avec \( \langle f(x), y\rangle =\langle x, f(y)\rangle \). Étant donné que \( f\) est également linéaire, il existe \( \lambda\in \eR\) tel que \( f(x)=\lambda x\) pour tout \( x\in E\). Tous les vecteurs de \( E\) sont donc vecteurs propres de \( f\).

    Passons à la récurrence. Nous considérons \( \dim(E)=n+1\) et \( f\in\gS(E)\). Nous considérons la forme bilinéaire symétrique \( \Phi_f\) et la forme quadratique associée \( \phi_f\). Pour rappel,
    \begin{subequations}
        \begin{align}
        \Phi_f(x,y)=\langle x, f(y)\rangle \\
        \phi_f(x)=\Phi_f(x,x).
        \end{align}
    \end{subequations}
    Et nous allons laisser tomber les indices \( f\) pour noter simplement \( \Phi\) et \( \phi\). Étant donné que \( \overline{ B(0,1) }\) est compacte et que \( \phi\) est continue, il existe \( x_0\in\overline{ B(0,1) }\) tel que 
    \begin{equation}
        \lambda=\phi(x_0)=\sup_{x\in\overline{ B(0,1) }}\phi(x).
    \end{equation}
    Notons aussi que \( \| x_0 \|=1\) : le maximum est pris sur le bord. Nous posons
    \begin{equation}
        g=\lambda\id-f
    \end{equation}
    ainsi que 
    \begin{equation}
        \Phi_1(x,y)=\langle x, g(y)\rangle .
    \end{equation}
    Cela est une forme bilinéaire et symétrique parce que
    \begin{equation}
        \Phi_1(y,x)=\langle y, g(x)\rangle =\langle g(y), x\rangle =\langle x, g(y)\rangle =\Phi_1(x,y)
    \end{equation}
    où nous avons utilisé le fait que \( g\) était auto-adjoint et la symétrie du produit scalaire. De plus \( \Phi_1\) est semi-définie positive parce que
    \begin{equation}
        \Phi_1(x,x)=\langle x, \lambda x-f(x)\rangle =\lambda\| x \|^2-\phi(x).
    \end{equation}
    Vu que \( \lambda\) est le maximum, nous avons tout de suite \( \Phi_1(x)\geq 0\) tant que \( \| x \|=1\). Et si \( x\) n'est pas de norme \( 1\), c'est le même prix parce qu'on se ramène à \( \| x \|=1\) en multipliant par un nombre positif. Attention cependant : 
    \begin{equation}
        \Phi_1(x_0,x_0)=\lambda\| x_0 \|^2-\phi(x_0)=0.
    \end{equation}
    Donc \( \Phi_1\) a un noyau contenant \( x_0\) par la proposition \ref{PropHIWjdMX}. Nous en déduisons que \( \Image(g)\neq E\) en effet, \( x_0\in\Image(g)^{\perp}\), mais nous avons la proposition \ref{PropXrTDIi} sur les dimensions : 
    \begin{equation}
        \dim E=\dim(\Image(g))+\dim( \Image(g)^{\perp}).
    \end{equation}
    Vu que \( \Image(g)^{\perp}\) est un espace vectoriel non réduit à \( \{ 0 \}\), la dimension de \( \Image(g)\) ne peut pas être celle de \( E\). L'endomorphisme \( g\) n'étant pas surjectif, il ne peut pas être injectif non plus parce que nous sommes en dimension finie; il existe donc \( e_1\in E\) tel que \( g(e_1)=0\) et tant qu'à faire nous choisissons \( \| e_1 \|=1\) (ici la norme est bien celle de l'espace euclidien considéré). Par définition,
    \begin{equation}
        f(e_1)=\lambda e_1,
    \end{equation}
    c'est à dire que \( \lambda\in\Spec(f)\). Et \( \phi\) étant une forme quadratique réelle nous avons \( \lambda\in \eR\).

    Nous posons à présent \( H=\Span\{ e_1 \}^{\perp}\). C'est un sous-espace stable par \( f\) parce que si \( x\in H\) alors
    \begin{equation}
        \langle e_1, f(x)\rangle =\langle f(e_1j),x\rangle =\lambda\langle e_1, x\rangle =0.
    \end{equation}
    Nous pouvons donc considérer la restriction de \( f\) à \( H\) : \( f_H\colon H\to H\). Cet endomorphisme est bilinéaire et symétrique sur l'espace \( H\) de dimension inférieure à celle de \( E\), donc la récurrence nous donne une base orthonormée
    \begin{equation}
        \{ e_2,\ldots, e_n \}
    \end{equation}
    de vecteurs propres de \( f_H\). De plus les valeurs propres sont réelles, toujours par récurrence. Donc
    \begin{equation}
        \Spec(f)=\{ \lambda \}\cup\Spec(f_H)\subset \eR.
    \end{equation}
    Notons pour être complet que si \( i\geq 2\) alors
    \begin{equation}
        \langle e_1, e_i\rangle =0
    \end{equation}
    parce que le vecteur \( e_i\) est par construction choisit dans l'espace \( H=e_1^{\perp}\). Nous avons donc bien une base orthonormée de \( E\) construite sur des vecteurs propres de \( f\).
\end{proof}

\begin{corollary}   \label{CorSMHpoVK}
    Soit \( E\) un espace vectoriel ainsi que \( \phi\) et \( \psi\) des formes quadratiques sur \( E\) avec \( \psi\) définie positive. Alors il existe une base \( \psi\)-orthonormale dans laquelle \( \phi\) est diagonale.
\end{corollary}

\begin{proof}
    Il suffit de considérer l'espace euclidien \( E\) muni du produit scalaire \( \langle x, y\rangle =\psi(x,y)\). Ensuite nous diagonalisons la matrice (symétrique) de \( \phi\) pour ce produit scalaire à l'aide du théorème \ref{ThoRSBahHH}.
\end{proof}

\begin{definition}      \label{DefYNWUFc}
    Dans le cas de \( V=\eR^m\) nous avons un produit scalaire canonique. Soient $u$ et $v$, deux vecteurs de $\eR^m$. Le \defe{produit scalaire}{produit!scalaire} de $u$ et $v$, noté $\langle u, v\rangle $ ou $u\cdot v$ est le réel
	\begin{equation}		\label{EqDefProdScalsumii}
		\langle u, v\rangle =\sum_{k=1}^m u_kv_k=u_1v_1+u_2v_2+\ldots+u_mv_n.
	\end{equation}
\end{definition}

Calculons par exemple le produit scalaire de deux vecteurs de la base canonique : $\langle e_i, e_j\rangle $. En utilisant la formule de définition et le fait que $(e_i)_k=\delta_{ik}$, nous avons
\begin{equation}
	\langle e_i, e_j\rangle =\sum_{k=1}^m\delta_{ik}\delta_{jk}.
\end{equation}
Nous pouvons effectuer la somme sur $k$ en remarquant qu'à cause du $\delta_{ik}$, seul le terme avec $k=i$ n'est pas nul. Effectuer la somme revient donc à remplacer tous les $k$ par des $i$ :
\begin{equation}
	\langle e_i, e_j\rangle =\delta_{ii}\delta_{ji}=\delta_{ji}.
\end{equation}

Une des propriétés intéressantes du produit scalaire est qu'il permet de décomposer un vecteur dans une base, comme nous le montre la proposition suivante.

\begin{proposition}		\label{PropScalCompDec}
	Si nous notons $v_i$ les composantes du vecteur $v$, c'est à dire si $v=\sum_{i=1}^m v_ie_i$, alors nous avons $v_j=\langle v, e_j\rangle $.
\end{proposition}

\begin{proof}
	\begin{equation}		\label{Eqvejscalcomp}
		v\cdot e_j=\sum_{i=1}^m\langle v_ie_i, e_j\rangle =\sum_{i=1}^mv_i\langle e_i, e_j\rangle =\sum_{i=1}^mv_i\delta_{ij}
	\end{equation}
	En effectuant la somme sur $i$ dans le membre de droite de l'équation \eqref{Eqvejscalcomp}, tous les termes sont nuls sauf celui où $i=j$; il reste donc
	\begin{equation}
		v\cdot e_j=v_j.
	\end{equation}
\end{proof}

Le produit scalaire ne dépend en réalité pas de la base orthogonale choisie. 

\begin{lemma}
	Si $\{ e_i \}$ est la base canonique, et si $\{ f_i \}$ est une autre base orthonormale, alors si $u$ et $v$ sont deux vecteurs de $\eR^m$, nous avons
	\begin{equation}
		\sum_i u_iv_j=\sum_iu'_iv'_j
	\end{equation}
	où $u_i$ sont les composantes de $u$ dans la base $\{ e_i \}$ et $u'_i$ sont celles dans la base $\{ f_i \}$.
\end{lemma}

\begin{proof}
	La preuve demande un peu d'algèbre linéaire. Étant donné que $\{ f_i \}$ est une base orthonormale, il existe une matrice $A$ orthogonale ($AA^t=\mtu$) telle que $u'_i=\sum_jA_{ij}u_j$ et idem pour $v$. Nous avons alors
	\begin{equation}
		\begin{aligned}[]
			\sum_iu'_iv'_j&=\sum_i\left( \sum_jA_{ij} u_j\right)\left( \sum_k A_{ik}v_k \right)\\
			&=\sum_{ijk}A_{ij}A_{ik}u_jv_k\\
			&=\sum_{jk}\underbrace{\sum_i(A^t)_{ji}A_{ik}}_{=\delta_{jk}}u_jv_k\\
			&=\sum_{jk}\delta_{jk}u_jv_k\\
			&=\sum_ku_jv_k.
		\end{aligned}
	\end{equation}	
\end{proof}

Cette proposition nous permet de réellement parler du produit scalaire entre deux vecteurs de façon intrinsèque sans nous soucier de la base dans laquelle nous regardons les vecteurs.

Nous dirons que deux vecteurs sont \defe{orthogonaux}{orthogonal} lorsque leur produit scalaire est nul. Nous écrivons que $u\perp v$ lorsque $\langle u, v\rangle =0$.
\begin{definition}	\label{DefNormeEucleApp}
	La \defe{norme euclidienne}{norme!euclidienne!dans $\eR^m$} d'un élément de $\eR^m$ est définie par $\| u \|=\sqrt{u\cdot u}$.
\end{definition}

Cette définition est motivée par le fait que le produit scalaire $u\cdot u$ donne exactement la norme usuelle donnée par le théorème de Pythagore :
\begin{equation}
	u\cdot u=\sum_{i=1}^mu_iu_i=\sum_{i=1}^m u_i^2=u_1^2+u_2^2+\ldots+u_m^2.
\end{equation}

Le fait que $e_i\cdot e_j=\delta_{ij}$ signifie que la base canonique est \defe{orthonormée}{orthonormé}, c'est à dire que les vecteurs de la base canonique sont orthogonaux deux à deux et qu'ils ont tout $1$ comme norme.

\begin{lemma}\label{LemSclNormeXi}
	Pour tout $u\in\eR^m$, il existe un $\xi\in\eR^m$ tel que $\| u \|=\xi\cdot u$ et $\| \xi \|=1$.
\end{lemma}

\begin{proof}
	Vérifions que le vecteur $\xi=u/\| u \|$ ait les propriétés requises. D'abord $\| \xi \|=1$ parce que $u\cdot u=\| u \|^2$. Ensuite
	\begin{equation}
		\xi\cdot u=\frac{ u\cdot u }{ \| u \| }=\frac{ \| u \|^2 }{ \| u \| }=\| u \|.
	\end{equation}
\end{proof}



%+++++++++++++++++++++++++++++++++++++++++++++++++++++++++++++++++++++++++++++++++++++++++++++++++++++++++++++++++++++++++++ 
\section{Sous-groupes du groupe linéaire}
%+++++++++++++++++++++++++++++++++++++++++++++++++++++++++++++++++++++++++++++++++++++++++++++++++++++++++++++++++++++++++++

\begin{lemma}[\cite{KXjFWKA}]       \label{LemOCtdiaE}
    Soit \( V\) un espace vectoriel de dimension finie muni d'une norme euclidienne \( \| . \|\). Soit \( K\) un compact convexe de \( V\) et \( G\), un sous groupe compact de \( \GL(V)\) tel que
    \begin{equation}
        u(K)\subset K
    \end{equation}
    pour tout \( u\in G\). Alors il existe \( a\in K\) tel que \( u(a)=a\) pour tout \( u\in G\).
\end{lemma}
\index{groupe!linéaire!sous-groupes compacts}
\index{compacité!sous-groupes du groupe linéaire}

\begin{proof}
    Avant de nous lancer dans la preuve, nous avons besoin d'un petit résultat.
    \begin{subproof}
        \item[Un pré-résultat]

        Nous commençons par prouver que si \( v\in \aL(V)\) vérifie \( v(K)\subset K\), alors \( v\) a un point fixe dans \( K\). Pour cela nous considérons \( x_0\in K\) et la suite
        \begin{equation}
            x_k=\frac{1}{ k+1 }\sum_{i=0}^kv^i(x_0).
        \end{equation}
        Étant donné que \( K\) est convexe et stable par \( v\), la suite \( (x_k)\) est contenue dans \( K\) et accepte une sous-suite convergente\footnote{C'est Bolzano-Weierstrass, théorème \ref{ThoBWFTXAZNH}.} que nous allons noter \( x_{\varphi(n)}\) avec \( \varphi\colon \eN\to \eN\) strictement croissante. Soit \( a\in K\) la limite :
        \begin{equation}
            \lim_{n\to \infty} x_{\varphi(n)}=a.
        \end{equation}
        Tant que nous y sommes nous pouvons aussi calculer \( v(x_k)\) :
        \begin{subequations}
            \begin{align}
                v(x_k)&=v\left( \frac{1}{ k+1 }\sum_{i=1}^kv^i(x_0) \right)\\
                &=\frac{1}{ k+1 }\sum_{i=0}^kv^{i+1}(x_0)\\
                &=x_k+\frac{1}{ k+1 }\Big( v^{k+1}(x_0)-x_0 \Big).      \label{EqUAfcaKG}
            \end{align}
        \end{subequations}
        La norme \( \| v^{k+1}(x_0)-x_0 \|\) est bornée par le diamètre de \( K\), donc en prenant la limite \( k\to \infty\) le second terme de \eqref{EqUAfcaKG} tend vers zéro. En prenant ces égalités en \( k=\varphi(n)\) et en prenant \( n\to\infty\), nous trouvons
        \begin{equation}
            v(a)=a,
        \end{equation}
        c'est à dire le résultat que nous voulions dans un premier temps.

    \item[Une norme sur \( V\)]

        Nous passons maintenant à la preuve du lemme. D'abord nous remarquons que le groupe \( G\) agit sur \( V\) par \( u\cdot x=u(x)\) et de plus, considérant la fonction continue
        \begin{equation}
            \begin{aligned}
                \alpha\colon G&\to V \\
                u&\mapsto u(x), 
            \end{aligned}
        \end{equation}
        nous voyons que les orbites de cette action sont compactes en tant qu'image par \( \alpha\) du compact \( G\) (théorème \ref{ThoImCompCotComp}). Nous posons
        \begin{equation}
            \begin{aligned}
                \nu\colon V&\to \eR^+ \\
                x&\mapsto \max_{u\in G}\| u(x) \|. 
            \end{aligned}
        \end{equation}
        Cette définition a un sens parce que l'orbite \( \{ u(x)\tq u\in G \}\) est compacte dans \( V\) et donc l'ensemble des normes est compact dans \( \eR\) et admet un maximum. De plus cela donne une norme sur \( V\) parce que nous vérifions les conditions de la définition \ref{DefNorme} :
        \begin{enumerate}
            \item
                Pour tout \( x,y\in V\) nous avons :
                \begin{equation}
                    \nu(x+y)=\max_{u\in G}\| u(x)+u(y) \|\leq \max_{u\in G}\left( \| u(x) \|+\| u(y) \| \right)\leq \nu(x)+\nu(y).
                \end{equation}
            \item
                Si \( \nu(x)=0\), alors l'égalité \( \max_{u\in G}\| u(x) \|=0\) nous enseigne que \( \| u(x) \|=0\) pour tout \( u\in G\) et donc en particulier avec \( u=\id\) nous trouvons \( x=0\).
            \item
                Pour tout \( \lambda\in \eR\) et \( x\in V\),
                \begin{equation}
                    \nu(\lambda x)=\max_{u\in G}\| u(\lambda x) \|=\max\| \lambda u(x) \|=\max| \lambda |\| u(x) \|=| \lambda |\nu(x).
                \end{equation}
        \end{enumerate}
        De plus la fonction \( \nu\) est constante sur les orbites de \( G\).

    \item[Un point fixe]

        Pour tout \( u\in G\) nous posons
        \begin{equation}
            F_u=\{ x\in K\tq u(x)=x \};
        \end{equation}
        par le pré-résultat, aucun de ces ensembles n'est vide. Ils sont de plus tous fermés par continuité de \( u\) (le complémentaire est ouvert). Nous devons prouver que \( \bigcap_{u\in G}F_u\neq \emptyset\) parce qu'une intersection serait un point fixe de tous les éléments de \( G\). Supposons donc que \( \bigcap_{u\in G}F_u=\emptyset\). Alors les complémentaires des \( F_u\) forment un recouvrement ouvert de \( K\) et nous pouvons en extraire un sous-recouvrement fini par compacité. Soient \( \{ u_i \}_{i=1,\ldots, p}\) les éléments qui réalisent ce recouvrement. Alors
        \begin{equation}
            \bigcap_{i=1}^pF_{u_i}=\emptyset.
        \end{equation}
        Nous considérons l'opérateur
        \begin{equation}
            v=\frac{1}{ p }\sum_{i=1}^pu_i\in\aL(V).
        \end{equation}
        Vu que \( K\) est convexe et stable sous chacun des \( u_i\), nous avons aussi \( v(K)\subset K\) et donc il existe \( a\in K\) tel que \( v(a)=a\). Pour ce \( a\), nous avons
        \begin{subequations}
            \begin{align}
                \nu\big( v(a) \big)&=\nu\left( \frac{1}{ p }\sum_{i=1}^pu_i(a) \right)      \label{EqDXSnwPb}\\
                &\leq \frac{1}{ p }\sum_{i=1}^p\nu\left( u_i(a) \right)\\
                &=\frac{1}{ p }\sum_{i=1}^p\nu(a)\\
                &=\nu(a)
            \end{align}
        \end{subequations}
        où nous avons utilisé la constance de \( \nu\) sur les orbites de \( G\). Par ailleurs nous savons que \( v(a)=a\), donc en réalité à gauche dans \eqref{EqDXSnwPb} nous avons \( \nu(a)\) et toutes les inégalités sont des égalités. Nous avons en particulier
        \begin{equation}        \label{EqBMjypoV}
                \nu\left( \sum_{i=1}^pu_i(a) \right) =\sum_{i=1}^p\nu\left( u_i(a) \right).
        \end{equation}
        Notons \( u_0\in G\) l'élément qui réalise le maximum de la définition de \( \nu\) pour le vecteur \( \sum_iu_i(a)\) :
        \begin{equation}
            \nu\left( \sum_i u_i(a) \right)=\| u_0\left( \sum_iu_i(a) \right) \|\leq\sum_i\| u_0u_i(a) \|\leq \sum_i\nu\big( u_i(a) \big).
        \end{equation}
        Mais nous venons de voir (équation \eqref{EqBMjypoV}) que l'expression de gauche est égale à celle de droite. Donc les inégalités sont des égalités et en particulier la première inégalité devient l'égalité
        \begin{equation}
            \| \sum_iu_0u_i(a)  \|=\sum_i\| u_0u_i(a) \|.
        \end{equation}
        En vertu du lemme \ref{LemLPOHUme}, il existe des nombres positifs \( \lambda_i\) tels que
        \begin{equation}
            u_0u_1(a)=\lambda_2u_0u_2(a)=\ldots =\lambda_pu_0u_p(a).
        \end{equation}
        Du fait que \( u_0\) est inversible nous avons aussi 
        \begin{equation}       \label{EqSTQwfIl}
            u_1(a)=\lambda_2u_2(a)=\ldots =\lambda_pu_p(a).
        \end{equation}
        Mais par constance de \( \nu\) sur les orbites nous avons \( \nu(u_i(a))=\nu(u_j(a))\) pour tout \( i\) et \( j\); en appliquant \( \nu\) à la série d'égalités \eqref{EqSTQwfIl}, nous trouvons que tous les \( \lambda_i\) doivent être égaux à \( 1\). En particulier
        \begin{equation}     
            u_1(a)=u_2(a)=\ldots =u_p(a).
        \end{equation}
        
        Nous récrivons maintenant l'équation \( v(a)=a\) avec la définition de \( v\) :
        \begin{equation}
            a=v(a)=\frac{1}{ p }\sum_{i=1}^pu_i(a)=u_j(a)
        \end{equation}
        pour n'importe quel \( j\). Donc
        \begin{equation}
            a\in\bigcap_{i=1}^pF_{u_i},
        \end{equation}
        ce qui contredit notre hypothèse de départ.
            
        \end{subproof}

\end{proof}

\begin{proposition}[\cite{NHXUsTa,KXjFWKA,RXvMqkd}]     \label{PropQZkeHeG}
    Soit \( G\) un sous-groupe compact de \( \GL(n,\eR)\). Alors 
    \begin{enumerate}
        \item
            Il existe une forme quadratique définie positive \( q\) sur \( \eR^n\) telle que \( G\subset \gO(q)\).
        \item
            Le groupe \( G\) est conjugué à un sous-groupe de \( \gO(n,\eR)\).
    \end{enumerate}
\end{proposition}
\index{groupe!action!utilisation}
\index{matrice!équivalence!dans le groupe linéaire}
\index{forme!quadratique!groupe orthogonal}
\index{groupe!orthogonal!d'une forme quadratique}
\index{endomorphisme!préservant une forme quadratique}

\begin{proof}
    Nous considérons le (pas tout à fait) morphisme de groupe
    \begin{equation}
        \begin{aligned}
            \rho\colon G&\to \GL\big( \gS(n,\eR) \big) \\
            u&\mapsto \rho_u\colon s\to  u^tsu,
        \end{aligned}
    \end{equation}
    et tant que nous y sommes à considérer, nous considérons l'ensemble
    \begin{equation}
        H=\{ M^tM\tq M\in G \}\subset \gS(n,\eR).
    \end{equation}
    Cet ensemble est constitué de matrices définies positives parce que si \( \langle M^tMx, x\rangle =0\), alors \(0= \langle Mx, Mx\rangle =\| Mx \|\), mais \( M\) étant inversible, cela implique que \( x=0\). Qui plus est cet ensemble est compact dans \( \GL(n,\eR)\) en tant qu'image du compact \( G\) par l'application continue \( M\mapsto M^tM\). L'enveloppe convexe \( K=\Conv(H)\) est alors également compacte par le théorème \ref{CorOFrXzIf}. Enfin nous considérons \( L=\rho(G)\), qui est un sous-groupe compact de \( \GL\big( \gS(n,\eR) \big)\) parce que \( \rho_u\rho_v=\rho_{vu}\in\rho(G)\). Nous remarquons que \( \rho_u\) étant linéaire, elle préserve les combinaisons convexes et donc pour tout \( u\in G\), \( \rho_u(K)\subset K\).

    Bref, \( L\) est un sous-groupe compact de \( \GL(n,\eR)\) préservant le compact \( K\) de \( \gS(n,\eR)\). Par le lemme \ref{LemOCtdiaE}, il existe \( s\in K\) tel que \( \rho_u(s)=s\) pour tout \( u\in G\). Ou encore :
    \begin{equation}
        u^tsu=s
    \end{equation}
    pour tout \( u\in G\). Fort de ce \( s\) bien particulier, nous considérons la forme quadratique associée : \( q(x)=x^tsx\). Cette forme est définie positive parce que \( s\) l'est. Nous avons \( G\subset \gO(q)\) parce que si \( u\in G\) alors
    \begin{equation}
        q\big( ux \big)=(ux)^tsux=x^t\underbrace{u^tsu}_{=s}x=q(x).
    \end{equation}
    Le premier point est prouvé.

    La matrice \( s\) est symétrique et définie positive. Elle peut donc être diagonalisée\footnote{Théorème \ref{ThoeTMXla}} en \( \diag(\lambda_1,\ldots, \lambda_n)\) avec \( \lambda_i>0\), et ensuite transformée en la matrice \( \mtu_n\) par la matrice \( \diag(1/\sqrt{\lambda_i})\). Nous avons donc une matrice \( a\in\GL(n,\eR)\) telle que \( a^tsa=\mtu_n\). Avec ça, si \( u\in G\), nous avons
    \begin{equation}
        (a^{-1}ua)^t(a^{-1} ua)=(a^{-1}ua)^t\mtu_n(a^{-1} ua)=a^tu^t(a^t)^{-1}a^tsaa^{-1}ua=a^tu^tsua=a^tsa=\mtu,
    \end{equation}
    ce qui prouve que \( a^{-1} ua\) est dans \( \gO(n,\eR)\), et donc que \( a^{-1} G a\subset \gO(n,\eR)\).
\end{proof}

%+++++++++++++++++++++++++++++++++++++++++++++++++++++++++++++++++++++++++++++++++++++++++++++++++++++++++++++++++++++++++++
\section{Isométries de l'espace euclidien}
%+++++++++++++++++++++++++++++++++++++++++++++++++++++++++++++++++++++++++++++++++++++++++++++++++++++++++++++++++++++++++++

Nous considérons l'espace affine euclidien \( A=\affE_n(\eR)\) modelé sur \( \eR^n\) avec sa métrique usuelle. Un premier grand résultat sera le théorème \ref{ThoDsFErq} qui dira que les isométries de cet espace sont des applications linéaires\footnote{Regardez un coup dans le second tome du Landau et Lifchitz voir comment ils démontrent que les transformations de Lorentz doivent être linéaires. Ça vois donnera une idée à quel point notre théorème est cool.}.

\begin{example}
    La forme quadratique \( q(x)=x_1^2+x_2^2\) donne la norme euclidienne. La forme bilinéaire associée est \( b(x,y)=x_1y_1+x_2y_2\), qui est le produit scalaire usuel.
\end{example}

Il ne faudrait pas déduire trop vite que la formule \( \| x \|^2=q(x)\) donne une norme dès que \( q\) est non dégénérée. En effet \( q\) peut ne pas être définie positive. La forme \( q(x)=x_1^2-x_2^2\) prend des valeurs positives et négatives. A fortiori \( d(x,y)=q(x-y)\) ne donne pas toujours une distance.

Nous allons cependant appeler \defe{isométrie}{isométrie!de forme quadratique} pour la forme \( q\) une application bijective \( f\colon V\to V\) telle que \( q(x-y)=q\big( f(x)-f(y) \big)\). Dans les cas où \( q\) donne une distance, alors c'est une isométrie au sens usuel.

\begin{lemma}   \label{LemewGJmM}
    Pour une application bijective \( f\colon E\to E\) telle que \( f(0)=0\), les conditions suivantes sont équivalentes: 
    \begin{enumerate}
        \item
            \( b\big( f(x),f(y) \big)=b(x,y)\) pour tout \( x,y\in E\);
        \item
            \( q\big( f(x)-f(y) \big)=q(x-y)\) pour tout \( x,y\in E\).
    \end{enumerate}
\end{lemma}

\begin{proof}
    Dans le sens direct, en posant \( x=y\) nous trouvons tout de suite \( q(f(x))=q(f)\); ensuite en utilisant la distributivité de \( b\),
    \begin{subequations}
        \begin{align}
            q\big( f(x)-f(y) \big)&=b\big( f(x)-f(y),f(x)-f(y) \big)\\
            &=q\big( f(x) \big)-2b\big( f(x),f(y) \big)+q\big( f(y) \big)\\
            &=q(x)+q(y)-2b(x,y)\\
            &=q(x-y).
        \end{align}
    \end{subequations}
    
    Dans l'autre sens, nous commençons par remarquer que l'hypothèse \( f(0)=0\) donne \( q(x)=q\big( f(x) \big)\). Ensuite nous utilisons l'identité de polarisation \eqref{EqMrbsop} :
    \begin{subequations}
        \begin{align}
            b\big( f(x),f(y) \big)&=\frac{ 1 }{2}\big[ q\big( f(x) \big)+q\big( f(y) \big)-q\big( f(x-y) \big) \big]\\
            &=\frac{ 1 }{2}\big[ q(x)+q(y)-q(x-y) \big]\\
            &=b(x,y).
        \end{align}
    \end{subequations}
\end{proof}

\begin{theorem}     \label{ThoDsFErq}
    Soit \( f\colon E\to E\) une bijection telle que
    \begin{equation}
        q(x-y)=q\big( f(x)-f(y) \big)
    \end{equation}
    pour tout \( x,y\in E\). Alors
    \begin{enumerate}
        \item
            si \( f(0)=0\), alors \( f\) est linéaire;
        \item
            si \( f(0)\neq 0\) alors \( f\) est affine.
    \end{enumerate}
\end{theorem}
La rédaction la preuve a bénéficié d'un coup de main de la part de \href{http://www.ilemaths.net/forum-sujet-500814.html}{GaBuZoMeu}. Une autre preuve, utilisant un peut plus d'indices et un peu plus de mots comme «tenseurs», peut être trouvée  \href{http://physics.stackexchange.com/questions/12664/proving-that-interval-preserving-transformations-are-linear}{ici}. Le fait que la preuve donnée soit tensorielle me fait penser que le résultat peut encore être généralisé.

\begin{proof}
    Si \( f(0)=0\), nous savons par le lemme \ref{LemewGJmM} que \( b\big( f(x),f(y) \big)=b(x,y)\). Soit \( z\in E\); étant donné que \( f\) est bijective nous pouvons considérer l'élément \( f^{-1}(z)\in E\) et calculer
    \begin{subequations}
        \begin{align}
            b\big( f(x+y),z \big)&=b\big( f(x+y),f(f^{-1}(z)) \big)\\
            &=b(x+y,f^{-1}(z))\\
            &=b(x,f^{-1}(z))+b(y,f^{-1}(z))\\
            &=b(f(x),z)+b(f(y),z)\\
            &=b\big( f(x)+f(y),z \big),
        \end{align}
    \end{subequations}
    donc \( f(x+y)=f(x)+f(y)\) par le lemme \ref{LemyKJpVP}. 

    De la même façon on trouve \( b\big( f(\lambda x),z \big)=b\big( \lambda f(x),z \big)\) qui prouve que \( f(\lambda x)=\lambda f(x)\) et donc que \( f\) est linéaire.

    Si \( f(0)\neq 0\), alors nous posons \( g(x)=f(x)-f(0)\) qui vérifie \( g(0)=0\) et
    \begin{equation}
        q\big( g(x)-g(y) \big)=q\big( f(x)-f(0)-f(y)+f(0) \big)=q(x-y).
    \end{equation}
    Nous pouvons donc appliquer le premier point à \( g\), déduire que \( g\) est linéaire et donc que \( f\) est affine.
\end{proof}

Maintenant nous savons que le groupe des isométries d'un espace quadratique \( (E,q)\) est un sous-groupe de \( \GL(E)\). Dans le cas de la métrique euclidienne, il est connu que ce sont les matrices orthogonales.

Nous pouvons maintenant avoir une discussion plus détaillée des groupes d'isométries de l'espace euclidien, parce que nous savons maintenant qu'elles sont des applications linéaires. Pour en savoir plus sur le groupe des isométries, il faut lire le théorème de Cartan-Dieudonné dans \cite{JGAdTA}.



\begin{lemma}[\cite{JGAdTA}]
    Si \( n\geq 3\), alors toute droite est intersection de deux plans non isotropes.
\end{lemma}

%--------------------------------------------------------------------------------------------------------------------------- 
\subsection{Théorème de Sylvester}
%---------------------------------------------------------------------------------------------------------------------------

% TODO : Il y a une démonstration sur wikipédia, à voir.

\begin{theorem}[de Sylvester]   \label{ThoQFVsBCk}
    Soit $Q$ une forme quadratique réelle de signature \( (p,q)\). Alors pour toute base orthonormée on a
    \begin{subequations}
        \begin{align}
            p&=\Card\{ i\tq Q(e_i)>0 \}\\
            q&=\Card\{ i\tq Q(e_i)<0 \}.
        \end{align}
    \end{subequations}
    Le rang de \( Q\) est \( p+q\).

    Si \( A\) est la matrice de \( Q\) dans une base, alors il existe une matrice inversible \( P\) telle que
    \begin{equation}
        P^tAP=\begin{pmatrix}
            -\mtu_q    &       &       \\
                &   \mtu_p    &       \\
                &       &   0
        \end{pmatrix}.
    \end{equation}
\end{theorem}
\index{théorème!Sylvester}
\index{rang}
\index{matrice!semblables}
\index{forme!quadratique}

%---------------------------------------------------------------------------------------------------------------------------
\subsection{Produit semi-direct}
%---------------------------------------------------------------------------------------------------------------------------

Les isométries directes\footnote{Directes au sens où nous ne considérons pas les retournements.} de cet espace sont données d'une part par les rotations de \( \SO(n)\) et d'autre part par les translations données par les vecteurs de \( \eR^n\). Plus précisément, un couple \( (v,\Lambda)\in \eR^n\times\SO(n)\) agit sur \( x\in A\) par
\begin{equation}
    (v,\Lambda)x=\Lambda x+v.
\end{equation}
La loi de composition est donnée par
\begin{subequations}
    \begin{align}
        (v,\Lambda)\cdot(v',\Lambda')x&=(v,\Lambda)(\Lambda'x+v')\\
        &=\Lambda\Lambda'x+\Lambda v'+v\\
        &=(\Lambda v'+v,\Lambda\Lambda')x.
    \end{align}
\end{subequations}
Nous avons donc, pour tout \( v,v'\in \eR^n\), \( \Lambda,\Lambda'\in\SO(n)\) la loi de groupe
\begin{equation}    \label{EqDiHcut}
        (v,\Lambda)\cdot(v',\Lambda')=(\Lambda v'+v,\Lambda\Lambda').
\end{equation}
    
Le groupe \( \SO(n)\) agit naturellement sur \( \eR^n\) par
\begin{equation}
    \begin{aligned}
        \phi\colon \SO(n)&\to \Aut(\eR^n) \\
        \Lambda&\mapsto \phi_{\Lambda}\colon v\to \Lambda v. 
    \end{aligned}
\end{equation}
Il est à noter qu'ici, \( \eR^n\) est vu comme l'ensemble des applications \( v\colon A\to A\), \( v(x)=x+a\). Voir aussi la remarque \ref{RemAobrlX}.

Nous pourrions alors présenter le groupe de isométries de \( A\) sous la forme du produit semi-direct
\begin{equation}
    \Iso^+(A)=\eR^n\times_{\phi}\SO(n).
\end{equation}
Plusieurs choses sont à vérifier :
\begin{enumerate}
    \item
        Pour chaque \( \Lambda\), l'application \( \phi_{\Lambda}\) est un automorphisme du groupe \( \eR^n\) (en tant qu'agissant sur \( A\)). Le fait que \( \phi_{\Lambda}\) soit une bijection n'est pas un problème. Nous devons vérifier que
        \begin{equation}
            \phi_{\Lambda}(v+w)=\phi_{\Lambda}(v)\circ\phi_{\Lambda}(w)
        \end{equation}
        en tant qu'égalité dans l'ensemble des isométries de \( A\). Nous la testons donc sur un élément \( x\in A\). D'une part
        \begin{equation}
            \phi_{\Lambda}(v+w)x=x+\Lambda(v+m),
        \end{equation}
        et d'autre part,
        \begin{equation}
            \phi_{\Lambda}(v)\circ\phi_{\Lambda}(w)x=\phi_{\Lambda}(v)\big( x+\Lambda w \big)=x+\Lambda w+\Lambda v.
        \end{equation}
    \item
        L'application \( \phi\colon \SO(n)\to \Aut(\eR^n)\) est un morphisme de groupe. Nous devons vérifier l'égalité
        \begin{equation}
            \phi_{\Lambda\Lambda'}=\phi_{\Lambda}\circ\phi_{\Lambda'}
        \end{equation}
        dans \( \Aut(\eR^n)\), c'est à dire que pour tout \( v\in \eR^n\) et \( x\in A\) nous devons avoir
        \begin{equation}
            \phi_{\Lambda\Lambda'}(v)x=\big( \phi_{\Lambda}\circ\phi_{\Lambda'}\big)(v)x.
        \end{equation}
        Le membre de gauche fait immédiatement \( x+\Lambda\Lambda'v\) tandis que le membre de droite vaut
        \begin{equation}
            \big( \phi_{\Lambda}\circ\phi_{\Lambda'}\big)(v)x=\big( \phi_{\Lambda}(\Lambda'v) \big)x=(\Lambda\Lambda'v)x=x+\Lambda\Lambda'v.
        \end{equation}
    \item
        La loi de groupe donnée par \( \phi\) sur \( \SO(n)\times \eR^n\) par la définition \eqref{EqDRgbBI} est bien la loi de groupe \eqref{EqDiHcut}. Cela est encore un calcul immédiat. L'utilisation de la définition \eqref{EqDRgbBI} donne
        \begin{equation}
            (v,\Lambda)\cdot(v',\Lambda')=(v+\phi_{\Lambda}(v'),\Lambda\Lambda')=(v+\Lambda v',\Lambda\Lambda'),
        \end{equation}
        qui est bien la formule \eqref{EqDiHcut}.
\end{enumerate}

%---------------------------------------------------------------------------------------------------------------------------
\subsection{Groupe diédral}
%---------------------------------------------------------------------------------------------------------------------------
\label{subsecHibJId}

%///////////////////////////////////////////////////////////////////////////////////////////////////////////////////////////
\subsubsection{Définition et générateurs : vue géométrique}
%///////////////////////////////////////////////////////////////////////////////////////////////////////////////////////////

Le \defe{groupe diédral}{groupe!diédral} \( D_n\)\nomenclature[R]{\( D_n\)}{groupe diédral} est le groupe des isométries de \( \eR^2\) laissant invariant un polygone régulier à \( n\) côtés. Il peut être vu comme le stabilisateur de l'ensemble
\begin{equation}
    \{  e^{2ik\pi/n},k=0,\ldots, n-1 \}
\end{equation}
dans le groupe des isométries affines de \( \eC^*\).
\index{groupe!agissant sur un ensemble!diédral}
\index{groupe!en géométrie}
\index{groupe!fini!diédral}
\index{groupe!permutation!diédral}
% TODO : prouver que les racines de l'unité forment un polygone régulier.

Si \( f\in D_n\), alors \( f( e^{2ik\pi/n}) \) doit être l'un des \(  e^{2ik'\pi/n}\), et vu que \( f\) conserve les longueurs dans \( \eC\), nous devons avoir
\begin{equation}
    1=d(0, e^{2ik\pi/n})=d\big( f(0), e^{2ik'\pi/n} \big).
\end{equation}
Donc \( f(0)\) est à l'intersection de tous les cercles de rayon \( 1\) centrés en les \(  e^{2ik\pi/n}\), ce qui montre que \( f(0))0\) (dès que \( n\geq 3\)). Par conséquent notre étude du groupe diédral ne doit prendre en compte que les isométries vectorielles de \( \eR^2\). En d'autres termes
\begin{equation}
    D_n\subset O(2,\eR).
\end{equation}

\begin{proposition}[\cite{tzHydF}]
    Le groupe \( D_n\) contient un sous groupe cyclique d'ordre \( 2\) et un sous groupe cyclique d'ordre \( n\).
\end{proposition}

\begin{proof}
    Si \( s\) est la réflexion d'axe \( \eR\), alors \( s\) est d'ordre \( 2\). De plus \( s\) est bien dans \( D_n\) parce que
    \begin{equation}    \label{EqSUshknP}
        s\big(  e^{2ki\pi/n} \big)= e^{2(n-k)i\pi/n}.
    \end{equation}

    De la même façon, la rotations d'angle \(2\pi/n\), que l'on note \( r\), agit sur les racines de l'unité et engendre un le groupe d'ordre \( n\) des rotations d'angle \(2 k\pi/n\).
\end{proof}

Notons que la conjugaison complexe ne fait pas spécialement partie du groupe \( D_n\). En effet pour \( n=3\) par exemple les points fixes sont \( A_1=(1,0)\), \( A_2=(-\frac{ 1 }{2},\frac{ \sqrt{3} }{2})\) et \( A_3=(\frac{ 1 }{2},-\frac{ \sqrt{3} }{2})\). La conjugaison complexe envoie évidemment \( A_1\) sur \( A_1\), mais pas du tout \( A_2\) sur \( A_3\).
%TODO : un dessin du triangle équilatéral serait pas mal ici.

\begin{proposition}[\cite{tzHydF}]
    Nous avons \( (sr)^2=\id\).
\end{proposition}

\begin{proof}
    Si \( z^n=1\), alors
    \begin{equation}
        (srsr)z=srs e^{2 i\pi/n}z=sr\big( e^{-2\pi i/n\bar z}\big)=s\bar z=z.
    \end{equation}
\end{proof}

\begin{proposition}[\cite{tzHydF}] \label{PropLDIPoZ}
    Le groupe diédral \( D_n\) est engendré par \( s\) et \( r\). De plus tous les éléments de \( D_n\) s'écrivent sous la forme \( s\circ r^m\).
\end{proposition}
\index{groupe!diédral!générateurs (preuve)}
\index{racine!de l'unité}
\index{géométrie!avec nombres complexes}
\index{géométrie!avec des groupes}
\index{isométrie!de l'espace euclidien \( \eR^2\)}

\begin{proof}
    Nous considérons les points \( A_0=1\) et \( A_k= e^{2ki\pi/n}\) avec \( k\in\{ 1,\ldots, n-1 \}\). Par convention, \( A_n=A_0\). L'action des éléments \( s\) et \( r\) sur ces points est
    \begin{subequations}
        \begin{align}
            r(A_k)&=A_{k+1}\\
            s(A_k)&=A_{n-k}.
        \end{align}
    \end{subequations}
    Cette dernière est l'équation \eqref{EqSUshknP}.
    
    Soit \( f\in D_n\). Étant donné que c'est une isométrie de \( \eR^2\) avec un point fixe (le point \( 0\)), \( f\) est soit une rotation soit une réflexion.
    %TODO : il faut démontrer ce point et mettre un lien vers ici.

    Supposons pour commencer que un des \( A_k\) est fixé par \( f\). Dans ce cas \( f\) a deux points fixes : \( O\) et \( A_k\) et est donc la réflexion d'axe \( (OA_k)\). Dans ce cas, nous avons \( f=s\circ r^{n-2k}\). En effet
    \begin{equation}
        s\circ r^{n-2k}(A_k)=s(A_{k+n-2k})=s(A_{n-k})=A_k.
    \end{equation}
    Donc \( O\) et \( A_k\) sont deux points fixes de l'isométrie \( f\); donc \( f\) est bien la réflexion sur le bon axe.

    Nous passons à présent au cas où \( f\) ne fixe aucun des \( A_k\). 
    \begin{enumerate}
        \item
            Supposons que \( f\) soit une rotation. Si \( f(A_k)=A_m\), alors l'angle de la rotation est 
            \begin{equation}
                \frac{ 2(m-k)\pi }{ n },
            \end{equation}
            et donc \( f=r^{m-k}\), qui est de la forme demandée.
        \item
            Supposons à présent que \( f\) soit une réflexion d'axe \( \Delta\). Cette fois, \( \Delta\) ne passe par aucun des points \( A_k\), par contre \( \Delta\) passe par \( 0\). Nous commençons par montrer que \( \Delta\) doit être la médiatrice d'un des côtés \( [A_p,A_{p+1}]\) du polygone. Vu que \( \Delta\) passe par \( O\) et n'est aucune des droites \( (OA_k)\), cette droite passe par l'intérieur d'un des triangles \( OA_pA_{p+1}\) et intersecte donc le côté correspondant.

            Notre tâche est de montrer que \( \Delta\) coupe \( [A_p,A_{p+1}]\) en son milieu. Dans ce cas, \( \Delta\) sera automatiquement perpendiculaire parce que le triangle \( OA_pA_{p+1}\) est isocèle en \( O\). Nommons \( l\) la longueur des côtés du polygone, \( P=\Delta\cap[A_p,A_{p+1}]\), \( x=d(A_p,P)\) et \( \delta=d(A_p,\Delta)\). Vu que \( f\) est la symétrie d'axe \( \Delta\), nous avons aussi \( d\big( f(A_p),\Delta \big)=\delta\) et \( d\big( A_p,f(A_p) \big)=2\delta\). D'autre part, par la définition de la distance, \( \delta<x\). Si \( x<\frac{ l }{2}\), alors \( \delta<\frac{ \delta }{2}\) et donc \( d\big( A_p,f(A_p) \big)<l\). Or cela est impossible parce que le polygone ne possède aucun sommet à distance plus courte que \( l\) de \( A_p\).

            De la même manière si \( x>\frac{ l }{2}\), nous raisonnons avec \( A_{p+1}\) pour obtenir une contradiction. Nous en concluons que la seule possibilité est \( x=\frac{ l }{2}\), et donc \( f(A_p)=A_{p+1}\). Montrons alors que \( f=s\circ r^{n-2p-1}\). Il faut montrer que c'est une réflexion qui envoie \( A_p\) sur \( A_{p+1}\). D'abord c'est une réflexion parce que
            \begin{equation}
                \det(sr^{n-2p-1})=\det(s)\det(r^{n-2p-1})=-1
            \end{equation}
            parce que \( \det(s)=-1\) alors que \( \det(r^k)=1\) parce que \( r\) est une rotation dans \( \SO(2)\). Ensuite nous avons
            \begin{equation}
                s\circ r^{n-2p-1}(A_p)=s(A_{p+n-2p-1})=s(A_{n-p-1})=A_{n-(n-p-1)}=A_{p+1}.
            \end{equation}

            Donc \( s\circ r^{n-2p-1}\) est bien une réflexion qui envoie \( A_p\) sur \( A_{p+1}\).

    \end{enumerate}
\end{proof}

\begin{corollary}   \label{CorWYITsWW}
La liste des éléments de \( D_n\) est 
\begin{equation}
    D_n=\{ 1,r,\ldots, r^{n-1},s,sr,\ldots, sr^{n-1} \}
\end{equation}
et \( | D_n |=2n\).
\end{corollary}

\begin{proof}
    Nous savons par la proposition \ref{PropLDIPoZ} que tous les élément de \( D_n\) s'écrivent sous la forme \( r^k\) ou \( sr^k\). Vu que \( r\) est d'ordre \( n\), il ne faut considérer que \( k\in\{ 1,\ldots, n-1 \}\). Les éléments \( 1\), \( r\),\ldots, \( r^{n-1}\) sont tous différents, et sont (pour des raisons de déterminant) tous différents des \( sr^k\). Les isométries \( sr^k\) sont toutes différentes entre elles pour essentiellement la même raison :
    \begin{equation}
        sr^k(A_p)=s(A_{p+k})=A_{n-p+k}
    \end{equation}
    donc si \( k\neq k'\), \( sr^k(A_p)\neq sr^{k'}(A_p)\). La liste des éléments de \( D_n\) est donc
    \begin{equation}
        D_n=\{ 1,r,\ldots, r^{n-1},s,sr,\ldots, sr^{n-1} \}
    \end{equation}
    et donc \( | D_n |=2n\).
\end{proof}

\begin{example}
    Nous considérons le carré \( ABCD\) dans \( \eR^2\) et nous cherchons les isométries de \( \eR^2\) qui laissent le carré invariant. Nous nommons les points comme sur la figure \ref{LabelFigIsomCarre}. La symétrie d'axe vertical est nommée \( s\) et la rotation de \( 90\) degrés est notée \( r\).
    \newcommand{\CaptionFigIsomCarre}{Le carré dont nous étudions le groupe diédral.}
    \input{Fig_IsomCarre.pstricks}

    Il est facile de vérifier que toutes les symétries axiales peuvent être écrites sous la forme \( r^is\). De plus le groupe engendré par \( s\) agit sur le groupe engendré par \( r\) parce que
    \begin{equation}
        (srs^{-1})(A,B,C,D)=sr(B,A,D,C)=s(A,D,C,B)=(B,C,D,A),
    \end{equation}
    c'est à dire \( srs^{-1}=r^{-1}\). Nous sommes alors dans le cadre du corollaire \ref{CoroGohOZ} et nous pouvons écrire que
    \begin{equation}
        D_4=\gr(r)\times_{\sigma}\gr(s).
    \end{equation}
\end{example}

%///////////////////////////////////////////////////////////////////////////////////////////////////////////////////////////
\subsubsection{Générateurs : vue abstraite}
%///////////////////////////////////////////////////////////////////////////////////////////////////////////////////////////

Nous allons montrer que \( D_n\) peut être décrit de façon abstraite en ne parlant que de ses générateurs. Nous considérons un groupe \( G\) engendré par des éléments \( a\) et \( b\) tels que
\begin{enumerate}
    \item
        \( a\) est d'ordre \( 2\),
    \item
        \( b\) est d'ordre \( n\) avec \( n\geq 3\),
    \item
        \( abab=e\).
\end{enumerate}
Nous allons prouver que ce groupe doit avoir la même liste d'éléments que celle du corollaire \ref{CorWYITsWW}.

\begin{proposition}[\cite{tzHydF}]
    Le groupe \( G\) n'est pas abélien.
\end{proposition}

\begin{proof}
    Nous savons que \( abab=e\), donc \( abab^{-1}=b^{-2}\), mais \( b^{-2}\neq e\) parce que \( b\) est d'ordre \( n>2\). Donc \( abab^{-1}\neq e\). En manipulant un peu :
    \begin{equation}
        e\neq abab^{-1}=(ab)(ba^{-1})^{-1}=(ab)(ba)^{-1}
    \end{equation}
    parce que \( a^{-1}=a\). Donc \( ab\neq ba\).
\end{proof}

\begin{lemma}[\cite{tzHydF}]        \label{LemKKXdqdL}
    Pour tout \( k\) entre \( 1\) et \( n-1\) nous avons
    \begin{equation}
        ab^ka=b^{-k}.
    \end{equation}
\end{lemma}

\begin{proof}
    Nous faisons la démonstration par récurrence. D'abord pour \( k=1\), nous devons avoir \( aba=b^{-1}\), ce qui est correct parce que par construction de \( G\) nous avons \( abab=e\). Ensuite nous supposons que le lemme tient pour \( k\) et nous regardons ce qu'il se passe avec \( k+1\) :
    \begin{equation}
            ab^{k+1}ba=ab^kba=\underbrace{ab^ka}_{b^{-k}}\underbrace{aba}_{b^{-1}}=b^{-k}b^{-1}=b^{-(k+1)}.
    \end{equation}
\end{proof}

\begin{proposition}
    L'élément \( a\) n'est pas une puissance de \( b\).
\end{proposition}

\begin{proof}
    Supposons le contraire : \( a=b^k\). Dans ce cas nous aurions
    \begin{equation}
        e=(ab)(ab)=b^{k+1}b^{k+1}=b^{2k+2}=b^{2k}b^2=a^2b^2=b^2,
    \end{equation}
    ce qui signifierait que \( b\) est d'ordre \( 2\), ce qui est exclu par construction.
\end{proof}

\begin{proposition}[\cite{tzHydF}]
    La liste des éléments de \( G\) est donnée par
    \begin{equation}
        G=\{ 1,b,\cdots,b^{n-1},a,ab,\ldots, ab^{n-1} \}.
    \end{equation}
\end{proposition}

\begin{proof}
    Étant donné que \( a\) n'est pas une puissance de \( b\), les éléments \( 1\), \( a\), \( b\),\ldots, \( b^{n-1}\) sont distincts. De plus si \( k\) et \( m=k+p\) sont deux éléments distincts de \( \{ 1,\ldots, n-1 \}\), nous avons \( ab^k\neq ab^m\) parce que si \( ab^k=ab^{k+p}\), alors \( a=ab^p\) avec \( p<n\), ce qui est impossible. Pour la même raison, \( ab^k\neq e\), et \( ab^k\neq b^m\).

    Au final les éléments \( 1,a,b,\ldots, b^{n-1},ab,\ldots, ab^{n-1}\) sont tous différents. Nous devons encore voir qu'il n'y en a pas d'autres.

    Par définition le groupe \( G\) est engendré par \( a\) et \( b\), donc tout élément \( x\in G\) s'écrit $x=a^{m_1}b^{k_1}\ldots a^{m_r}b^{k_r}$ pour un certain \( r\) et avec pour tout \( i\), \( k_i\in\{ 1,\ldots, n-1 \}\) (sauf \( k_r\) qui peut être égal à zéro) et \( m_i=1\), sauf \( m_1\) qui peut être égal à zéro. Donc
    \begin{equation}
        x=a^mb^{k_1}ab^{k_2}a\ldots b^{k_{r-1}}ab^{k_r}
    \end{equation}
    où \( m\) et \( k_r\) peuvent éventuellement être zéro. En utilisant le lemme \ref{LemKKXdqdL} sous la forme \( b^{k_i}a=ab^{-k_i}\), quitte à changer les valeurs des exposants, nous pouvons passer tous les \( a \) à gauche et tous les \( b\) à droite pour finir sous la forme \( x=a^kb^m\). 

    Donc non, il n'existe pas d'autres éléments dans \( G\) que ceux déjà listés.

\end{proof}

\begin{theorem}
    Les groupes \( G\) et \( D_n\) sont isomorphes.
\end{theorem}

\begin{proof}
        Nous utilisons l'application
    \begin{equation}
        \begin{aligned}
            \psi\colon G&\to D_n \\
            a^kb^m&\mapsto s^kr^m. 
        \end{aligned}
    \end{equation}
    C'est évidemment bien défini et bijectif, mais c'est également un homomorphisme parce que si nous calculons \( \psi\) sur un produit, nous devons comparer
    \begin{equation}        \label{EqBULPilp}
        \psi\big( a^{k_1}b^{m_1}a^{k_2}b^{m_2} \big)
    \end{equation}
    avec
    \begin{equation}        \label{EqIVEIphI}
        \psi\big( a^{k_1}b^{m_1}\big)\psi\big(a^{k_2}b^{m_2} \big)= s^{k_1}r^{m_1}s^{k_2}r^{m_2}.
    \end{equation}
    Vu que \( D_n\) et \( G\) ont les mêmes propriétés qui permettent de permuter \( a\) et \( b\) ou \( s\) et \( r\), l'expression à l'intérieur du \( \psi\) dans \eqref{EqBULPilp} se simplifie en \( a^kb^m\) avec les même \( k\) et \( n\) que l'expression à droite dans \eqref{EqIVEIphI} ne se simplifie en \( s^kr^m\).
\end{proof}

\begin{corollary}
    Toutes les propriétés démontrées pour \( G\) sont vraies pour \( D_n\). En particulier, avec quelque redites :
    \begin{enumerate}
        \item
            Le groupe \( D_n\) peut être défini comme étant le groupe engendré par un élément \( s\) d'ordre \( 2\) et un élément \( r\) d'ordre \( n-1\) assujettis à la relation \( srsr=e\).
        \item
            Le groupe \( D_n\) n'est pas abélien.
        \item
            Pour tout \( k\in\{ 1,\ldots, n-1 \}\) nous avons \( sr^ks=r^{-k}\).
        \item
            L'élément \( s\) ne peut pas être obtenu comme une puissance de \( r\).
        \item
            La liste des éléments de \( D_n\) est
            \begin{equation}
                D_n=\{ 1,r,\ldots, r^{n-1},s,sr,\ldots, sr^{n-1} \}
            \end{equation}
        \item
            Le groupe diédral \( D_n\) est d'ordre \( 2n\).
    \end{enumerate}
\end{corollary}

%///////////////////////////////////////////////////////////////////////////////////////////////////////////////////////////
\subsubsection{Classes de conjugaison}
%///////////////////////////////////////////////////////////////////////////////////////////////////////////////////////////
\label{subsubsecZQnBcgo}

Pour les classes de conjugaison du groupe diédral nous suivons \cite{HRIMAJJ}.

D'abord pour des raisons de déterminants\footnote{Vous notez qu'ici nous utilisons un argument qui utilise la définition de \( D_n\) comme isométries de \( \eR^2\). Si nous avions voulu à tout prix nous limiter à la définition «abstraite» en termes de générateurs, il aurait fallu trouver autre chose.}, les classes des éléments de la forme \( r^k\) et de la forme \( sr^k\) ne se mélangent pas. Nous notons \( C(x)\) la classe de conjugaison de \( x\), et \( y\cdot x=yxy^{-1}\).

Les relations que nous allons utiliser sont 
\begin{subequations}
    \begin{align}
        sr^ks=r^{-k}\\
        rs=sr^{-1}=sr^{n-1}.
    \end{align}
\end{subequations}

La classe de conjugaison qui ne rate jamais est bien entendu \( C(1)={1}\). Nous commençons les vraies festivités \( C(r^{m})\). D'abord \( r^k\cdot r^m=r^m\), ensuite
\begin{equation}
    (sr^k)\cdot r^m=sr^kr^mr^{-k}s^{-1}=sr^ms^{-1}=r^{-m}.
\end{equation}
Donc
\begin{equation}    \label{EqVFfFxgi}
    C(r^m)=\{ r^m,r^{-m} \}.
\end{equation}
À ce niveau il faut faire deux remarques. D'abord si \( m>\frac{ n }{2}\), alors \( C(r^m)\) est la classe de \( C^{n-m}\) avec \( n-m<\frac{ n }{2}\). Donc les classes que nous avons trouvées sont uniquement à lister avec \( m<\frac{ n }{2}\). Ensuite si \( m=\frac{ n }{2}\) alors \( r^m=r^{-m}\) et la classe est un singleton. Cela n'arrive que si \( n\) est pair.

Nous passons ensuite à \( C(s)\). Nous avons
\begin{equation}
    r^k\cdot s=r^ksr^{-k}=ssr^ksr^{-k}=sr^{-k}r^{-k}=sr^{n-2k},
\end{equation}
et
\begin{equation}
    (sr^k)\cdot s=\underbrace{sr^ks}_{r^{-k}}r^{-k}s^{-1}=r^{-2k}s=r^{n-2k}s=sr^{(n-1)(n-2k)}=sr^{n^2-2kn-n+2k}=sr^{2k}.
\end{equation}
donc
\begin{equation}
    C(s)=\{ sr^{n-2k},sr^{2k} \}_{k=0,\ldots, n-1}.
\end{equation}
Ici aussi l'écriture n'est pas optimale : peut-être que pour certains \( k\) il y a des doublons. Nous reportons l'écriture exacte à la discussion plus bas qui distinguera \( n\) pair de \( n\) impair. Notons juste que si \( n\) est pair, l'élément \( sr\) n'est pas dans la classe \( C(s)\).

Nous en faisons donc à présent le calcul en gardant en tête le fait qu'il n'a de sens que si \( n\) est pair. D'abord
\begin{equation}
    s\cdot (sr)=ssrs=rs=sr^{n-1}.
\end{equation}
Ensuite
\begin{equation}
    (sr^k)\cdot (sr)=sr^ksrr^{-k}s=r^{-2k+1}s=sr^{2k-1}.
\end{equation}
Avec \( k=\frac{ n }{2}\), cela rend \( s\cdot (sr)\), donc pas besoin de le recopier. Nous avons
\begin{equation}
    C(sr)=\{ sr^{2k-1} \}_{k=1,\ldots, n-1}.
\end{equation}

%///////////////////////////////////////////////////////////////////////////////////////////////////////////////////////////
\subsubsection{Le compte pour $ n$ pair}
%///////////////////////////////////////////////////////////////////////////////////////////////////////////////////////////
\label{SubsubsecROVmHuM}

Si \( n\) est pair, nous avons les classes
\begin{subequations}
    \begin{align}
        C(1)&=\{ 1 \}       &&&\text{\( 1\) élément}\\
        C(r^m)&=\{ r^m,r^{m-1} \}&\text{ pour }&0<m<\frac{ n }{2}   &\text{\( \frac{ n }{2}-1\) fois \( 2\) éléments}\\
        C(r^{n/2})&=\{ r^{n/2} \}   &&& \text{\( 1\) élément}\\ 
        C(s)&=\{ sr^{2k} \}_{k=0,\ldots, \frac{ n }{2}-1} &&& \text{\( \frac{ n }{2}\) éléments}\\
        C(sr)&=\{ sr^{2k+1} \}_{k=0,\ldots, \frac{ n }{2}-1} &&& \text{\( \frac{ n }{2}\) éléments}.
    \end{align}
\end{subequations}
Au total nous avons bien listé \( 2n\) éléments comme il se doit, dans \(  \frac{ n }{2}+3\) classes différentes.

%///////////////////////////////////////////////////////////////////////////////////////////////////////////////////////////
\subsubsection{Le compte pour $ n$ impair}
%///////////////////////////////////////////////////////////////////////////////////////////////////////////////////////////
\label{Subsubsec*GJIzDEP}

Si \( n\) est impair, nous avons les classes
\begin{subequations}
    \begin{align}
        C(1)&=\{ 1 \}       &&&\text{\( 1\) élément}\\
        C(r^m)&=\{ r^m,r^{m-1} \}&\text{ pour }&0<m<\frac{ n-1 }{2}   &\text{\( \frac{ n-1 }{2}\) fois \( 2\) éléments}\\
        C(s)&=\{ sr^k \}_{k=0,\ldots, n-1} &&& \text{\( n\) éléments}
    \end{align}
\end{subequations}
Au total nous avons bien listé \( 2n\) éléments comme il se doit, dans \(  \frac{ n+3 }{2}\) classes différentes.

%+++++++++++++++++++++++++++++++++++++++++++++++++++++++++++++++++++++++++++++++++++++++++++++++++++++++++++++++++++++++++++
\section{Espaces hermitiens}
%+++++++++++++++++++++++++++++++++++++++++++++++++++++++++++++++++++++++++++++++++++++++++++++++++++++++++++++++++++++++++++

\begin{definition}  \label{DefMZQxmQ}
Si \( E\) est un espace vectoriel sur \( \eC\), nous disons qu'une application \( \langle ., .\rangle \colon E\times E\to \eC\) est un \defe{produit scalaire hermitien}{produit!scalaire!hermitien}\index{hermitien!produit scalaire} si pour tout \( u,v\in E\) nous avons
\begin{enumerate}
    \item
        \( \langle u, v\rangle =\overline{ \langle v, u\rangle  }\)
    \item
        \( \lambda\langle u, v\rangle =\langle \lambda u, v\rangle =\langle u, \bar \lambda v\rangle \)
    \item
        \( \langle u, u\rangle \in \eR^+\) et \( \langle u, u\rangle =0\) si et seulement si \( u=0\).
\end{enumerate}
\end{definition}
