% This is part of the Exercices et corrigés de mathématique générale.
% Copyright (C) 2009-2010
%   Laurent Claessens
% See the file fdl-1.3.txt for copying conditions.


\begin{corrige}{FoncDeuxVar0011}

	L'énoncé est faux. Pensez à une fonction dont «tous les chemins sauf un» ont la même limite. Par exemple
	\begin{equation}
		f(x,y)=\begin{cases}
			1	&	\text{si $y\neq 0$}\\
			0	&	 \text{si $y=0$.}
		\end{cases}
	\end{equation}
	Cette fonction vaut $1$ partout sauf sur une droite. Il y a donc une infinité de chemins suivant lesquels on a une limite en $(0,0)$ qui vaut $1$ (tous les chemins qui ne passent pas par $y=0$). Mais on a un chemin pour lequel la limite est zéro (le chemin horizontal $y=0$).

	Donc cette fonction n'a pas de limite en $(0,0)$.

	Essayez de trouver d'autres exemples. Trouvez un exemple de fonction pour lequel on a une infinité de chemins pour lesquels la limite est $1$ et en même temps une infinité de chemins pour lesquels la limite est $\sqrt{2}$.

\end{corrige}
