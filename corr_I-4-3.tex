% This is part of the Exercices et corrigés de CdI-2.
% Copyright (C) 2008, 2009
%   Laurent Claessens
% See the file fdl-1.3.txt for copying conditions.


\begin{corrige}{_I-4-3}

La convergence uniforme implique l'uniforme sur tout compact, mais l'inverse n'est pas vrai comme le montre l'exemple
\begin{equation}
	f_n(x)=\begin{cases}
	1	&	\text{si $n<x<n+1$}\\
	0	&	 \text{sinon.}
\end{cases}
\end{equation}
La convergence uniforme n'implique pas la convergence en moyenne quadratique, comme nous l'avons vu à l'exercice \ref{exo_I-4-2}.

Affin de voir que la convergence $L^2$ n'implique pas la convergence uniforme sur tout compact, reprenons l'exemple de l'exercice \ref{exo114}. En comparant les graphes \ref{LabelFigexouuiv} et \ref{LabelFigexouuivbis}, la différence entre la limite et les fonctions $f_n$ est composée des deux triangles de base $\frac{1}{ n }$ et de hauteur $1$. La surface en-dessous de $(f_n-f)^2$ n'est donc autre que
\begin{equation}
	2\left( \frac{1}{ n }\frac{1}{ 2 } \right)^2=\frac{1}{ 2n^2 }\to 0.
\end{equation}
Il y a donc convergence au sens $L^2$, mais même pas uniforme sur tout compact, et \emph{a fortiori} pas uniforme.

\end{corrige}
