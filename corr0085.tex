% This is part of Exercices et corrigés de CdI-1
% Copyright (c) 2011
%   Laurent Claessens
% See the file fdl-1.3.txt for copying conditions.

\begin{corrige}{0085}

\begin{enumerate}
\item
 Il suffit de regarder le complémentaire, c'est-à-dire l'ensemble $S$ des $x$ vérifiant $f(x) \neq 0$. Vérifions que pour tout $x\in S$, on peut placer une boule ouverte centrée en $x$ complètement incluse à $S$.  Soit $\epsilon = \abs{f(x)}$. Alors il existe $\delta > 0$ tel que si $d(x,y) < \delta$, on a $d(f(x),f(y)) = \abs{f(x) - f(y)} < \epsilon$. La boule ouverte $\{y \in X \telque d(x,y) < \delta$ est alors une boule incluse dans $S$ : l'image d'un point $y$ ne peut pas être nulle, sinon l'inégalité $\abs{f(x) - f(y)} = \epsilon < \epsilon$ n'est pas satisfaite.

\item On utilise le premier point sur la fonction $g$ définie par $g(x) = f(x) - x$.
\end{enumerate}


\end{corrige}
