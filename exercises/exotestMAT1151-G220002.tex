\begin{exercice}\label{exotestMAT1151-G220002}

	Le travail d'une grue qui soulève un bloc de masse $m$ d'une hauteur $h$ à la surface de la Terre est donné par
	\begin{equation}
		W(h)=\int_{R}^{R+h}\frac{ GM m }{ r^2 }dr
	\end{equation}
	où $R=6.500.000$ mètres est le rayon de la Terre et le produit $GM$ vaut environ $4.144\cdot 10^{14}$

	Tracer le graphique du travail nécessaire pour monter une masse $m=1$ à la hauteur $h$ en fonction de $h$ entre $h=0$ et $h=10.000$. 

	\vspace{1cm}
Questions bonus (ne comptent pas pour des points)
\begin{enumerate}

	\item
		Quelles sont les unités de $W$ dans le SI ?
	\item
		Est-ce que vous êtes capables d'interpréter le résultat ?
	\item
		Essayez de tracer le graphique beaucoup plus loin que $h=10000$.
		

\end{enumerate}

\corrref{testMAT1151-G220002}
\end{exercice}
