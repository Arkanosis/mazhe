% This is part of Exercices de mathématique pour SVT
% Copyright (c) 2011
%   Laurent Claessens et Carlotta Donadello
% See the file fdl-1.3.txt for copying conditions.

\begin{exercice}\label{exoExamenDecembre2010-0001}

  \begin{enumerate}
  \item Tracer le graphe de la fonction $f(x)=x$ pour $x$ entre $0$ et $5$.
    \item Calculer l'intégrale $\displaystyle \int_{0}^{5} x\, dx$, qui correspond à l'aire de la région entre l'axe des $x$ et le graphe de $f$.
      \item Tracer le graphe de la fonction \emph{partie entière} $x\mapsto [x]$, définie par 
        \[[x]=\textrm{le plus grand nombre entier qui est plus petit de } x.
        \]
        Exemples : $[4.67]=4$, $[2]=2$, $[0.34]=0$.
        \item Calculer l'intégrale $\displaystyle \int_{0}^{5} [x]\, dx$. Conseil : écrire cette intégrale comme la somme de $5$ intégrales $\int_{0}^{1}\ldots+ \cdots +\int_{4}^{5}\ldots$. 
          \item Tracer le graphe de la fonction \emph{mantisse}, $m(x)=x-[x]$, pour $x$ entre $0$ et $5$.
          \item Calculer l'intégrale $\displaystyle \int_{0}^{5} x-[x]\, dx$.
         % \item Vérifier graphiquement que l'intégrale que vous venez de calculer, qui répresente l'aire de la région entre l'axe des $x$ et le graphe de $m$,  est égale à l'aire entre les graphes de $f(x)=x$ et de $g(x)=[x]$.
  \end{enumerate}
\corrref{ExamenDecembre2010-0001}
\end{exercice}
