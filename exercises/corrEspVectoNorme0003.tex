\begin{corrige}{EspVectoNorme0003}

	En général, lorsqu'un ensemble est donné par des inégalités, prendre la fermeture consiste à transformer les inégalités strictes en inégalités non strictes; prendre l'intérieur consiste à rendre stricte toutes les inégalités; la frontière consiste \emph{en gros} à transformer toutes les inégalités en égalités (nous allons voir que pour la frontière, c'est un peu plus de travail). Comprenez bien que cela n'est vrai que «en général». Il faut toujours bien regarder sur chaque exemple si il n'y a pas l'un ou l'autre point problématique.

	La proposition \ref{Propfaposfxposcont} sera une des clefs pour dire que si une inégalité stricte est satisfaite en un point, alors elle sera satisfaite en tout point dans un voisinage. Voir aussi l'exercice \ref{exoEspVectoNorme0008}.
	\begin{enumerate}
		\item
			Si un point $(x,y)\in\eR^2$ est tel que $x^2-5x+6<y$, alors dans une boule centrée en $(x,y)$ (de rayon $r_1$), l'inégalité reste vraie (parce que la fonction $x^2-5x+6-y$ est une fonction continue). De la même manière, si nous avons $y<2$ en $(x,y)$, alors nous avons encore l'inégalité dans une boule de rayon $r_2$. En prenant $r=\min\{ r_1,r_2 \}$, les deux inégalités restent vraies dans la boule de rayon $r$.

			Donc les points $(x,y)$ tels que $x^2 - 5x + 6 < y < 2$ sont dans l'intérieur de $A_1$.
			
			Pour les mêmes raisons, autour d'un point $(x,y)$ tel que $x^2-5x+6>y$, nous pouvons trouver une boule dans laquelle l'inégalité reste stricte. Ces points ne sont donc pas dans l'adhérence de $A_1$. Un point qui vérifie $x^2-5x+6= y= 2$ est par contre dans l'adhérence parce que dans toute boule, on pourra trouver un $x$ tel que $x^2-5x+6<y$, et un $y$. L'adhérence est donc donnée par les inéquations
			\begin{equation}
				\bar A_1\equiv x^2-5x+6\leq y\leq 2.
			\end{equation}
			
			La frontière est donnée par les points de l'adhérence qui ne sont pas dans l'intérieur de $A_1$. Attention : {\bf ne pas dire} que la frontière est alors donnée simplement en remplaçant les inégalités par des égalités : $\partial A_1\equiv x^2-5x+6= y= 2$. Quel est cet ensemble ?

			Trouver la frontière demande un peu plus de travail. Le point marqué sur la figure \ref{LabelFigAdhIntFr} est sur la frontière parce que toute boule intersecte l'intérieur et l'extérieur. Cela est dû au fait que, sur ce point, nous ayons $x^2-5x+6=y$ en même temps que $y<2$. Donc si on prend une boule assez petite, on conserve $y<2$, mais on obtient des points tels que $x^2-5x+6<y$. 

			En voyant le dessin, la chose à faire pour écrire la frontière est de trouver les deux points d'intersections entre la parabole et la droite horizontale. Ces points sont les points $(x,y)$ qui satisfont au système
			\begin{subequations}
				\begin{numcases}{}
					x^2-5x+6=y\\
					y=2.
				\end{numcases}
			\end{subequations}
			En substituant la seconde équation dans la première, il vient $x^2-5x+6=2$, ce qui nous donne à résoudre le polynôme du second degré $x^2-5x+4=0$. Les solutions sont $x=1$ et $x=4$, et les deux points d'intersections sont les points $P=(1,2)$ et $Q=(4,2)$. Les points de la frontière de $A_1$ sont donc donnés par 
			\begin{equation}
				\begin{aligned}[]
					\partial A_1&=\{ (x,y)\in\eR^2\tqs x^2-5x+6=y\text{ et } 1\leq x\leq 4 \}\\
						&\quad\cup\{ (x,y)\in\eR^2\tqs y=2\text{ et } 1\leq x\leq 4 \}.
				\end{aligned}
			\end{equation}
			
			\newcommand{\CaptionFigAdhIntFr}{En hachuré : l'intérieur; en trait plein : la frontière. L'adhérence est l'union des deux. Exercice \ref{exoEspVectoNorme0003},\ref{ItemExoEVN3i}.}
			\input{pictures_tex/Fig_AdhIntFr.pstricks}

			Notez que les points de la parabole qui sont sur la frontière ne font pas partie de l'ensemble $A_1$ lui-même, tandis que ceux de la frontière qui sont sur la droite horizontale en font partie sauf \( (4,2)\) et \( (1,2)\).
\newcommand{\CaptionFigAdhIntFrDeux}{Notez que le point d'angle fait partie de la frontière, mais pas de l'ensemble. Exercice \ref{exoEspVectoNorme0003},\ref{ItemExoEVN3ii}.}
\input{pictures_tex/Fig_AdhIntFrDeux.pstricks}

			L'intérieur de $A_1$ n'étant pas égal à $A_1$, cet ensemble n'est pas ouvert; de la même manière, vu que $\bar A_1\neq A_1$, l'ensemble n'est pas fermé. L'ensemble $A_1$ est par contre borné parce qu'il est contenu par exemple dans la boule de centre $(0,0)$ et de rayon $5$. Les points d'accumulation de \( A_1\) sont les points de sa fermeture.

		\item
			Pour les mêmes raisons que dans l'exercice précédent, l'intérieur est donné par
			\begin{equation}
				\Int(A_2)\equiv x+1<y<2x;
			\end{equation}
			L'adhérence est donnée par
			\begin{equation}
				\overline{ A_2 }\equiv x+1\leq y\leq 2x,
			\end{equation}
			Pour la frontière, les deux droites dont il est question dans la définition de $A_2$ (les droites $y=x+1$ et $y=2x$) se coupent en $x=1$ (refaire soi-même le dessin de la figure \ref{LabelFigAdhIntFrDeux}). Lorsque $x<1$, les conditions $x+1<y$ et $y<2x$ sont incompatibles : aucun point de $A_2$ n'est dans la partie $x<1$ du plan. Lorsque $x>1$, alors les points situés \emph{entre} les deux droites font partie de $A_2$. La frontière est donc donnée par ces deux droites pour $x\geq 1$. 

			Étant donné que $\Int(A_2)=A_2$, cet ensemble est ouvert (et donc pas fermé par la proposition \ref{PropTopologieAx}\ref{ItemPropTopologieAxiv}). Il n'est par contre pas borné parce qu'il contient des point $(x,y)$ avec des $x$ arbitrairement grands.

		\item
			L'ensemble $A_3$ est un petit segment de droite. Son intérieur est vide parce que toute boule centrée en un point de la droite intersecte l'extérieur de la droite. Son adhérence et sa frontière sont $A_3$ lui-même parce que nous considérons les valeurs de $t$ dans $\mathopen[ 0 , 1 \mathclose]$ qui est un intervalle fermé. Si l'intervalle avait été ouvert, l'adhérence et la frontière auraient été trouvés en fermant :
			\begin{equation}
                \overline{ \{ (t,2t)\tqs t\in\mathopen[ 0 , 1 [\, \}}=\{ (t,2t)\tqs t\in\mathopen[ 0 , 1 ] \}
			\end{equation}
			Étant donné que son adhérence est égal à lui-même, cet ensemble est fermé (et donc pas ouvert). Il est également borné parce qu'il est contenu dans une boule de rayon $3$.
		\item
			Dans $\eR$ nous savons que $\bar\eQ=\eR$, $\Int(\eQ)=\emptyset$ et $\partial\eQ=\eR$ parce que toute boule centrée en un rationnel contient un irrationnel, et inversement, toute boule centrée en un irrationnel contient un rationnel. Dans $\eR^2$ nous avons le même phénomène parce dans la boule $B\big( (p,q),r \big)$ avec $(p,q)\in\eQ\times\eQ$, se trouvent en particulier les points de la forme $(p,x)$ avec $x\in B(q,r)\subset\eR$. Évidement, certains de ces $x$ ne sont pas dans $\eQ$ et par conséquent, la boule $B\big( (p,q),r \big)$ contient les points $(p,x)\notin\eQ\times\eQ$.

			De la même manière, si $(x,y)$ est un point de $\eR^2$, dans toute boule centrée en $(x,y)$, il y aura un élément de $\eQ^2$.

			Par conséquent, $\Int(\eQ\times\eQ)=\emptyset$, $\overline{ \eQ\times\eQ }=\eR\times\eR$ et $\partial(\eQ\times\eQ)=\eR^2$.

			Il n'est ni ouvert ni fermé (parce qu'il n'est égal ni à son intérieur ni à sa fermeture). Il n'est pas borné non plus parce qu'il existe des nombres rationnels arbitrairement grands.

		\item
			La fonction $x\mapsto\sin(\frac{1}{ x })$ est une des fonctions dont le graphe doit être connu. La figure \ref{LabelFigAdhIntFrTrois} montre la situation. Comme d'habitude, il est fortement recommandé de refaire le dessin soi-même.
\newcommand{\CaptionFigAdhIntFrTrois}{Les points qui dont sur l'axe vertical entre $0$ et $3$ sont sur la frontière, mais pas dans l'ensemble $A_5$.}
\input{pictures_tex/Fig_AdhIntFrTrois.pstricks}

			L'ensemble $A_5$ est ouvert parce que les conditions $x\in\mathopen] 0 , 1 \mathclose[$ et $\sin\frac{1}{ x }<y<3$ sont des condition «ouvertes» au sens où si un point les vérifient, alors on peut trouver une boule dans lequel ces conditions restent vérifiées. Cela prouve que $\Int(A_5)=A_5$.

			La fermeture de $A_5$ contient en outre les points tels que $\sin\frac{1}{ x }=y$ entre $x=0$ et $x=1$ (les bornes étant incluses) ainsi que les points des trois segments de droites suivants:
			\begin{equation}
				\begin{aligned}[]
					\{ (0,y)\tqs y\in\mathopen[ -1 , 3 \mathclose] \}\\
					\{ (x,3)\tqs x\in\mathopen[ 0 , 1 \mathclose] \}\\
					\{ (1,y)\tqs y\in\mathopen[ \sin(1) , 3 \mathclose] \}.
				\end{aligned}
			\end{equation}

			La frontière est composée de ces trois segments et du graphe de la fonction $\sin\frac{1}{ x }$ entre $0$ et $1$.

			L'ensemble $A_5$ est borné parce qu'il est contenu par exemple dans la boule centrée en $(0,0)$ et de rayon $10$. Il est ouvert et donc pas fermé.

		\item
			L'ensemble $A_6$ est une union infinie de segments de droites verticaux, voir figure \ref{LabelFigAdhIntFrSix}
\newcommand{\CaptionFigAdhIntFrSix}{Le segment sur l'axe vertical entre $y=0$ et $y=1$ fait partie de l'adhérence et de la frontière, mais pas de l'ensemble $A_6$ lui-même.}
\input{pictures_tex/Fig_AdhIntFrSix.pstricks}
				L'intérieur est vide parce qu'autour de tout réel de la forme $\frac{1}{ n }$, il y a un réel qui n'est pas de cette forme. En ce qui concerne la frontière et l'adhérence, il s'agit de l'union de tous ces segments plus le segment en $x=0$.

			En effet, la boule de rayon $r$ autour du point $(0,y)$ le point $(\frac{1}{ n },y)$ avec $n$ assez grand pour que $\frac{1}{ n }<r$.

	\end{enumerate}

\end{corrige}
