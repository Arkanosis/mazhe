% This is part of the Exercices et corrigés de mathématique générale.
% Copyright (C) 2009-2010
%   Laurent Claessens
% See the file fdl-1.3.txt for copying conditions.

\begin{corrige}{INGE1121La0012}

	\begin{enumerate}

		\item
			Il s'agit d'abord de remplir la matrice $A$ correspondante à la forme quadratique. Le terme en $x_1x_3$ correspond aux cases $(1,3)$ et $(3,1)$ de la matrice. Comme le coefficient est $2$, nous mettons $1$ dans chaque case. De la même manière, le terme $2x_3x_4$ fait venir un $1$ dans les cases $(3,4)$ et $(4,3)$. La matrice est donc
			\begin{equation}
				A=\begin{pmatrix}
					 0	&	0	&	1	&	0	\\
					 0	&	0	&	0	&	0	\\
					 1	&	0	&	0	&	1	\\ 
					 0	&	0	&	1	&	0	 
					  \end{pmatrix}.
			\end{equation}
			Cette façon de construire la matrice fournit toujours une matrice symétrique. Cette dernière aura donc une base de vecteurs propres. Si nous avions mit simplement $2$ dans les cases $(1,3)$ et $(2,4)$, nous n'aurions pas eut la symétrie. Cela est l'intérêt de mettre la moitié dans chacune des deux cases.

			Nous pouvons maintenant nous lancer dans la recherche des valeurs propres de cette matrice :
			\begin{equation}
				\begin{aligned}[]
					\det(A-\lambda \mtu)&=\det\begin{pmatrix}
						 -\lambda	&	0	&	1	&	0	\\
						 0	&	-\lambda	&	0	&	0	\\
						 1	&	0	&	-\lambda	&	1	\\ 
						 0	&	0	&	1	&	-\lambda	 
						  \end{pmatrix}\\
						  &=
						  -\lambda\det
						  \begin{pmatrix}
							  -\lambda	&	1	&	0	\\
							  1	&	-\lambda	&	1	\\
							  0	&	1	&	-\lambda
						\end{pmatrix}\\
						&=
						-\lambda^2\det\begin{pmatrix}
							-\lambda	&	1	\\ 
							1	&	-\lambda	
						\end{pmatrix}
						-\lambda\det
						\begin{pmatrix}
							1	&	0	\\ 
							1	&	-\lambda	
						\end{pmatrix}\\
						&=-\lambda^2(\lambda^2-2).
				\end{aligned}
			\end{equation}
			Les valeurs propres sont donc $0$ (de multiplicité deux) et $\pm\sqrt{2}$. Nous savons donc qu'il existe des nouvelles variables $y_i$ telles que la forme quadratique s'écrive
			\begin{equation}		\label{exoLa0012quddiafy}
				\sqrt{2}(y_1^2-y_2^2).
			\end{equation}
			Afin d'exprimer ces nouvelles coordonnées en fonction des $x_i$, nous devons trouver les  \wikipedia{fr}{Valeur_propre_(synthèse)}{vecteurs propres} de la matrice.

			Pour la valeur propre $\sqrt{2}$, nous devons résoudre le système donné par la matrice
			\begin{equation}
				\begin{pmatrix}
					-\sqrt{2}	&	0	&	1	&	0	\\
					0	&	-\sqrt{2}	&	0	&	0	\\
					1	&	0	&	-\sqrt{2}	&	1	\\ 
					0	&	0	&	1	&	-\sqrt{2}	 
					  \end{pmatrix}
			\end{equation}
			La deuxième ligne dit tout de suite que $x_2=0$. En prenant $x_4=\mu$ comme paramètre, nous trouvons ensuite
			\begin{equation}
				x_3=\sqrt{2}\mu,
			\end{equation}
			et $-\sqrt{2}x_1+x_3=0$, ce qui donne $\sqrt{2}(\mu-x_1)=0$ et donc
			\begin{equation}
				x_1=\mu.
			\end{equation}
			Une base de l'espace propre pour la valeur propre $\sqrt{2}$ est alors donné par
			\begin{equation}
				\begin{pmatrix}
					1	\\ 
					0	\\ 
					\sqrt{2}	\\ 
					1	
				\end{pmatrix}.
			\end{equation}
			De la même manière, nous trouvons les espaces propres pour les valeurs $-\sqrt{2}$ et $0$. Les réponses sont
			\begin{equation}
				\begin{aligned}[]
					\sqrt{2}&\leadsto \begin{pmatrix}
						1	\\ 
						0	\\ 
						\sqrt{2}	\\ 
						1	
					\end{pmatrix},
					&
					-\sqrt{2}&\leadsto\begin{pmatrix}
						1	\\ 
						0	\\ 
						-\sqrt{2}	\\ 
						1	
					\end{pmatrix}\\
					0&\leadsto\begin{pmatrix}
						1	\\ 
						0	\\ 
						0	\\ 
						-1	
					\end{pmatrix},
					\begin{pmatrix}
						0	\\ 
						1	\\ 
						0	\\ 
						0	
					\end{pmatrix}.
				\end{aligned}
			\end{equation}
			Notez que tous ces vecteurs sont deux à deux orthogonaux (était-ce prévisible ?).

			Une erreur à ne pas faire est de croire que la matrice orthogonale qui diagonalise $A$ est la matrice obtenue en mettant ces quatre vecteurs en colonnes. En effet, ces vecteurs ne sont pas normés. Pour obtenir la bonne matrice, il faut les d'abord diviser chacun de ces vecteurs par leur norme. Ce que nous obtenons est
			\begin{equation}
				B=\begin{pmatrix}
					1/2	&	1/2	&	1/\sqrt{2}	&	0	\\
					 0	&	0	&	0	&	1	\\
					 -\sqrt{2}/2	&	\sqrt{2}/2	&	0	&	0	\\ 
					 1/2	&	1/2	&	-1/\sqrt{2}	&	0	 
				 \end{pmatrix}.
			\end{equation}
			
			La forme quadratique a la forme \eqref{exoLa0012quddiafy} dans les variables $Y$ données par $X=BY$, c'est à dire
			\begin{equation}
				\begin{pmatrix}
					x_1	\\ 
					x_2	\\ 
					x_3	\\ 
					x_4	
				\end{pmatrix}=
				\begin{pmatrix}
					1/2	&	1/2	&	1/\sqrt{2}	&	0	\\
					 0	&	0	&	0	&	1	\\
					 -\sqrt{2}/2	&	\sqrt{2}/2	&	0	&	0	\\ 
					 1/2	&	1/2	&	-1/\sqrt{2}	&	0	 
				 \end{pmatrix}
				 \begin{pmatrix}
					 y_1	\\ 
					 y_2	\\ 
					 y_3	\\ 
					 y_4	
				 \end{pmatrix}.
			\end{equation}
			Tu paries que si on substitue les valeurs
			\begin{equation}
				\begin{aligned}[]
					x_1&=\frac{ 1 }{2}(y_1+y_2)+\frac{ y_3 }{ \sqrt{2} }\\
					x_2&=y_4\\
					x_3&=\frac{1}{ \sqrt{2} }(y_2-y_1)\\
					x_4&=\frac{ 1 }{2}(y_1+y_2)-\frac{ y_3 }{ \sqrt{2} }
				\end{aligned}
			\end{equation}
			dans l'expression $2x_1x_3+2x_3x_4$, on obtient bien $\sqrt{2}(y_2^2-y_1^2)$ ?

			En ce qui concerne le \emph{genre} de la forme quadratique, il est indéterminé parce qu'il y a une valeur propre strictement négative et une strictement positive.

	\item
		La matrice est
		\begin{equation}
			A=
			\begin{pmatrix}
				0	&	1/2	&	1/2	\\
				1/2	&	0	&	1/2	\\
				1/2	&	1/2	&	0
			\end{pmatrix}.
		\end{equation}
		Ses valeurs et vecteurs propres sont
		\begin{equation}
			\begin{aligned}[]
				1&\to v_1=(1, 1, 0)\\
				-1/2&\to v_2=(1, 0, 1)\\
				-1/2&\to v_3=(1, -1, -1).
			\end{aligned}
		\end{equation}
		La matrice suivante est formée des vecteurs d'une base orthonormale de vecteurs propres :
		\begin{equation}
			B=
			\begin{pmatrix}
				\frac{1}{2} \, \sqrt{2} & \frac{1}{2} \, \sqrt{2} & 0 \\
				\frac{1}{3} \, \sqrt{\frac{3}{2}} & -\frac{1}{3} \, \sqrt{\frac{3}{2}} & \frac{2}{3} \, \sqrt{\frac{3}{2}} \\
				\frac{1}{3} \, \sqrt{3} & -\frac{1}{3} \, \sqrt{3} & -\frac{1}{3} \, \sqrt{3}
			\end{pmatrix}
		\end{equation}

		Étant donné les valeurs propres, la matrice n'a aucun genre particulier.
		

	\end{enumerate}

\end{corrige}
