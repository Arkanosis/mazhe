% This is part of Exercices et corrigés de CdI-1
% Copyright (c) 2011
%   Laurent Claessens
% See the file fdl-1.3.txt for copying conditions.

\begin{corrige}{OutilsMath-0024}

	En coordonnées cylindriques, il est souvent pratique de penser en termes de «tranches» horizontales. Ici à la hauteur $h$ nous avons un cercle de rayon $R(1-h)$. L'équation de ce cercle est
	\begin{subequations}
		\begin{numcases}{}
			r=(1-h)R\\
			z=h.
		\end{numcases}
	\end{subequations}
	Cela est pour une tranche. L'équation du cône est alors
	\begin{equation}
		r=(1-z)R
	\end{equation}
	avec $z\in\mathopen[ 0 , 1 \mathclose]$.

\end{corrige}
