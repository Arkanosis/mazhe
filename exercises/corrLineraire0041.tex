% This is part of the Exercices et corrigés de mathématique générale.
% Copyright (C) 2009
%   Laurent Claessens
% See the file fdl-1.3.txt for copying conditions.
\begin{corrige}{Lineraire0041}

	\begin{enumerate}

		\item
			


	Les deux vecteurs directeurs du plan sont conservés, et le vecteur perpendiculaire est retourné. Les conservés sont ceux qui sont vecteurs propres de valeur propre $1$, et ce sont
	\begin{equation}
		\begin{aligned}[]
			\begin{pmatrix}
				0	\\ 
				1	\\ 
				1	
			\end{pmatrix}&&\text{et}&&\begin{pmatrix}
				1	\\ 
				0	\\ 
				1	
			\end{pmatrix}.
		\end{aligned}
	\end{equation}
	Pour trouver un vecteur perpendiculaire au plan, il faut un vecteur perpendiculaire aux deux vecteurs directeurs du plan. Pour ce faire, on calcule le produit vectoriel. C'est
	\begin{equation}
		\begin{pmatrix}
			1	\\ 
			1	\\ 
			-1	
		\end{pmatrix}.
	\end{equation}
	Ce dernier est un vecteur propre de valeur propre $-1$.

\item
	Les vecteurs sur la droite sont conservés. À multiple près, il y en a un seul et c'est
	\begin{equation}
		\begin{pmatrix}
			1	\\ 
			-3	\\ 
			-2	
		\end{pmatrix}.
	\end{equation}
	Celui-là est de valeur propre $1$. Tous ceux qui lui sont perpendiculaires (le plan perpendiculaire à cette droite) sont de valeur propre $-1$.

	\end{enumerate}


\end{corrige}
