% This is part of the Exercices et corrigés de mathématique générale.
% Copyright (C) 2009-2010
%   Laurent Claessens
% See the file fdl-1.3.txt for copying conditions.


\begin{corrige}{INGE1121La0002}

	La première chose à faire est de trouver les vecteurs de l'espace $V$. Pour ce faire, nous résolvons le système donné par la matrice
	\begin{equation}
		\begin{pmatrix}
			1	&	1	&	1	&	-1	\\
			1	&	-2	&	2	&	0
		\end{pmatrix}.
	\end{equation}
	Le rang de cette matrice étant $2$, nous savons que l'espace des solutions sera de dimension donnée par
	\begin{equation}
		\dim V=\dim(\eR^4)-2=2.
	\end{equation}
	Il est assez facile de trouver deux vecteurs linéairement indépendants dans $V$, par exemple en posant $x_3=1$ et $x_4=0$ et puis $x_3=0$ et $x_4=1$. Les vecteurs ainsi trouvés sont
	\begin{equation}
		\begin{aligned}[]
			v_1=(-\frac{ 4 }{ 3 },\frac{1}{ 3 },1,0)\\
			v_2=(\frac{ 2 }{ 3 },\frac{1}{ 3 },0,1).
		\end{aligned}
	\end{equation}
	Ces deux vecteurs sont linéairement indépendants parce que la matrice qu'ils forment est de rang deux.

	La méthode de Gram-Schmidt appliqué à ces deux vecteurs fournit
	\begin{equation}
		\begin{aligned}[]
			w_1&=(-\frac{ 4 }{ 3 },\frac{1}{ 3 },1,0)\\
			w_2&=(\frac{ 4 }{ 13 },\frac{ 11 }{ 26 },\frac{ 7 }{ 26 },1),
		\end{aligned}
	\end{equation}
	et leurs normes sont $\sqrt{26}/3$ et $\sqrt{35/26}$.

\end{corrige}
