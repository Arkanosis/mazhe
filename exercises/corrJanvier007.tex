% This is part of the Exercices et corrigés de mathématique générale.
% Copyright (C) 2009
%   Laurent Claessens
% See the file fdl-1.3.txt for copying conditions.
\begin{corrige}{Janvier007}



Charly a cinq couleurs a disposition, et a huit éléments à repeindre.

En ce qui concerne les portières, sans tenir compte de la restriction, il a $5^4$ possibilités différentes. Or il faut qu'au moins une des portières soit verte, ce qui exclut les $4^4$ cas où cette couleur n'est pas utilisée. Soit $5^4 - 4^4$ possibilités.

Pour le pare-chocs, il a $4$ possibilités, et pour chacun des trois autres éléments, $5$ possibilités.

Au total, il y a donc $4 \times 5^3 \times (5^4 -4^4)$ possibilités pour barioler la voiture de Charles-Edouard.

\end{corrige}
