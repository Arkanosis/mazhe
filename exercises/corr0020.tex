% This is part of Exercices et corrigés de CdI-1
% Copyright (c) 2011,2015
%   Laurent Claessens
% See the file fdl-1.3.txt for copying conditions.

\begin{corrige}{0020}

Nous considérons la suite $x_k = {\left(1+\frac1k\right)}^k$
\begin{enumerate}
\item Montrons que $(x_k)$ est croissante. Pour cela, utilisons la
  formule du binôme de Newton :
  \begin{equation*}
    \begin{split}
      {\left(1+\frac1{k}\right)}^{k} &= \sum_{i=0}^{k} \binom{k}{i}
      1^{(1-i)} {\left(\frac{1}{k}\right)}^i\\
      &= \sum_{i=0}^{k} \frac{k\cdot (k-1)\cdot\ldots\cdot 2 \cdot 1
      }{i! \cdot
        (k-i)\cdot(k-i-1)\cdot\ldots\cdot 2\cdot 1} \frac{1}{k^i}\\
      &= \sum_{i=0}^{k} \frac{k\cdot(k-1)\cdot\ldots\cdot(k - i +
        1)}{i!}
      \frac{1}{k^i}\\
      &=\sum_{i=0}^{k} \frac{1}{i!} \cdot \frac{k}{k}\cdot
      \frac{k-1}{k}\cdot\frac{k-2}{k}\cdot\ldots\cdot
      \frac{k-(i-1)}{k}\\
      &=\sum_{i=0}^{k} \frac{1}{i!} \cdot 1\cdot \left(1 - \frac{1}{k}\right)\cdot\left(1 - \frac{2}{k}\right)\cdot\ldots\cdot\left(1 - \frac{i-1}{k}\right)\\
%
      % {\left(1+\frac1{k+1}\right)}^{k+1} &= \sum_{i=0}^{k+1} \binom
      % {k+1} i 1^{(k+1-i)} {\left(\frac{1}{k+1}\right)}^i\\
      % &=\sum_{i=0}^{k+1} \frac{(k+1)k(k-1)\ldots 1 }{i!
      %   (k+1-i)(k-i)(k-i-1)\ldots 1}
      % {\left(\frac{1}{k+1}\right)}^i\\
      % &=\sum_{i=0}^{k+1} \frac{(k+1)k(k-1)\ldots (k+1-(i-1))}{i!}
      % {\left(\frac{1}{k+1}\right)}^i\\
      % &=\sum_{i=0}^{k+1} \frac{1}{i!} \frac{(k+1)}{(k+1)}
      % \frac{k}{(k+1)}\frac{(k-1)}{(k+1)}\ldots
      % \frac{(k+1-(i-1))}{(k+1)}\\
      % %
      % &=\sum_{i=0}^{k+1} \frac{k+1}{k+1-i}\frac{k!}{i! (k-i)!}
      % {\left(\frac{1}{k+1}\right)}^i\\
      % &\leq\sum_{i=0}^{k} \frac{k+1}{k+1-i}\frac{k!}{i! (k-i)!}
      % {\left(\frac{1}{k+1}\right)}^i\\
      % &\leq\sum_{i=0}^{k} \frac{k+1}{k+1-i}\frac{k!}{i! (k-i)!}
      % {\left(\frac{1}{k}\right)}^i
    \end{split}
  \end{equation*}
	Afin de soulager la notation, écrivons
\begin{equation}
	 A_k(i)=\frac{1}{i!} \cdot 1\cdot \left(1 - \frac{1}{k}\right)\cdot\left(1 - \frac{2}{k}\right)\cdot\ldots\cdot\left(1 - \frac{i-1}{k}\right),
\end{equation}
de façon à avoir $x_k=\sum_{i=0}^kA_k(i)$. Manifestement, tant que $i<k+1$ (ce qui est toujours le cas dans les sommes considérées), $A_k(i)\geq 0$ et $A_{k+1}(i)-A_k(i)\geq 0$ pour tout $k$ et tout $i$. Donc
\begin{equation}
	x_{k+1}-x_k=\sum_{i=0}^k(A_{k+1}(i)-A_{k}(i))+A_{k+1}(k)\geq 0.
\end{equation}
Cela prouve que la suite est croissante.

  Par ailleurs, on observe que la suite est majorée par $3$ : on peut majorer les facteurs du type $1 - \frac \cdot k$ par $1$, et on en déduit que
\begin{equation}
	\begin{aligned}[]
      x_k \leq \sum_{i=0}^k \frac 1{i!}	&= \sum_{i=0}^k \frac1{1\cdot 2 \cdot 3 \cdot \ldots \cdot i}\\
					&\leq \sum_{i=0}^k \frac{1}{1 \cdot \underbrace{ 2 \cdot 2 \cdot \ldots \cdot 2}_{\text{$i-1$ fois}}} \\
					&= 1 + \sum_{i=1}^k \frac{1}{2^{i-1}} < 1 + 2 = 3
	\end{aligned}
\end{equation}
  où on utilise le fait que les sommes partielles de la série géométrique de raison $\frac{ 1 }{2}$ sont plus petites que la somme de cette série, c'est-à-dire $2$.

\item Le premier point prouve déjà que la limite, notée $e$ est
  inférieure à $3$. De plus, la suite étant croissante on a forcément
  \begin{equation*}
    e \geq x_1 = 2
  \end{equation*}
  ce qui est le résultat annoncé.

\item Dans l'ordre : $e$, $\sqrt e$, $\sqrt e$, $\sfrac 1 e$.

\begin{enumerate}
\item Nous pouvons écrire $\left( 1+\frac{1}{ k } \right)^{k+1}=\left( 1+\frac{1}{ k } \right)\cdot\left( 1+\frac{1}{ k } \right)^{k}$, et puis utiliser le produit de deux suites convergentes.
\item Il faut utiliser le fait que si $x_k$ est convergente, $\lim\sqrt{x_k}=\sqrt{\lim x_k}$. Nous allons prouver cela \og à la main\fg{}  mais sachez que c'est une conséquence de la continuité de la fonction $t\mapsto\sqrt{t}$. Si la suite $(x_k)$ converge vers $r\neq 0$, alors nous pouvons faire la manipulation suivante en supposant que $k$ est assez grand pour que $| x_k-r |\leq\epsilon$ :
\begin{equation}		\label{EqInegssqrtLimCOn}
	\begin{aligned}[]
		| \sqrt{x_k}-\sqrt{r} |	&=\left|  \frac{ (\sqrt{x_k}-\sqrt{r})(\sqrt{x_k}+\sqrt{r}) }{ \sqrt{x_k}+\sqrt{r} }  \right|\\
					&=\left| \frac{ x_k-r }{ \sqrt{x_k}+\sqrt{r} } \right| \\
					&\leq\frac{ | x_k-r | }{ \sqrt{r} }\\
					&\leq \frac{ \epsilon }{ \sqrt{r} }
	\end{aligned}
\end{equation}
où nous avons tenu compte du fait que $\sqrt{x_k}+\sqrt{r}\geq\sqrt{r}$. Pour tout $\epsilon>0$, il existe un $K$ tel que $k>K$ implique les inégalités \eqref{EqInegssqrtLimCOn}. Si $\epsilon'$ est donné, il suffit de prendre $\epsilon< \sqrt{r}\epsilon'$ pour obtenir
\begin{equation}
	| \sqrt{x_k}-\sqrt{r} |\leq\epsilon'.
\end{equation}
\item
Si nous comparons les suites
\begin{equation}
	\begin{aligned}[]
		x_k&=\left( 1+\frac{1}{ k } \right)^k&\text{et}&&y_k&=\left( 1+\frac{1}{ 2k } \right)^k,
	\end{aligned}
\end{equation}
nous voyons que $x_{2k}=\left( 1+\frac{1}{ 2k } \right)^{2k}$, et donc $y_k=\sqrt{x_{2k}}$. Or, il est évident que la limite de la suite $(x_{2k})$ est la même que la limite de la suite $(x_k)$.

\item 
Il faut essayer de transformer $1-\frac{1}{ k }$ en $1+\frac{1}{ k }$. Pour cela, on voit que
\begin{equation}
	y_k=\left( 1-\frac{1}{ k } \right)^k=\left( \frac{ k-1 }{ k } \right)^k=\left( \frac{ k }{ k-1 } \right)^{-k}.
\end{equation}
Étant donné que $k>k-1$, nous exprimons $\frac{ k }{ k-1 }$ sous la forme $1+x$, et nous trouvons
\begin{equation}
	y_k=\left( 1+\frac{1}{ k-1 } \right)^{-k}=\left( 1+\frac{1}{ k-1 } \right)^{-k+1}\left( 1+\frac{ 1 }{ k-1 } \right)^{-1}\to 1/e
\end{equation}
par la règle du produit de deux suites convergentes, et le fait que $\lim(1/x_k) = 1/\lim(x_k)$.
\end{enumerate}
\end{enumerate}
\end{corrige}
