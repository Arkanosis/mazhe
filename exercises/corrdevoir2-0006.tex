\begin{corrige}{devoir2-0006}

On commence par intégrer par rapport à la variable $x$ la dérivée partielle de $g$ par rapport à $x$. Cette procédure nous permet de trouver la fonction $f$ à moins d'une fonction $K_1$ qui ne dépend pas de $x$ et joue le rôle de <<constante d'intégration>>. Nous avons
\begin{equation}\label{parrapportax}
 f(x,y,z)= \int \partial_x g(x,y,z)\, dx= x^3y\sin(y z)+ 3y\ln(x)+K_1 (y,z).  
\end{equation} 
De même, en intégrant par rapport à $y$ et $z$ les dérivées partielles correspondantes on obtient
\begin{equation}\label{parrapportay}
f(x,y,z)= \int \partial_y g(x,y,z)\, dy= x^3y\sin(y z)+ 3y\ln(x)+K_2 (x,z), 
\end{equation} 
\begin{equation}\label{parrapportaz}
f(x,y,z)= \int \partial_z g(x,y,z)\, dz= x^3y\sin(y z)+ 3z^2+K_3 (x,y). 
\end{equation}  

Pour compléter notre exercice il nous faut maintenant trouver des fonctions $K_1$, $K_2$ et $K_3$ telles que les équations \eqref{parrapportax}, \eqref{parrapportay} et \eqref{parrapportaz} soient vérifiées simultanément.

On a alors $K_1=K_2=3z^2$  et $K_3= 3y\ln(x)$.
\end{corrige}
