% This is part of Analyse Starter CTU
% Copyright (c) 2014
%   Laurent Claessens,Carlotta Donadello
% See the file fdl-1.3.txt for copying conditions.

\begin{exercice}\label{exoanalyseCTU-0014}


\begin{enumerate}
\item 
Compléter le tableau suivant. 
  \begin{equation}
    \begin{array}{|c|c|c|}\hline 
      \textrm{Primitive } \int f(x)\, dx & \quad\textrm{Function } f(x)\quad &\quad \textrm{D\'eriv\'ee } f'(x)\quad \\\hline 
      \vspace{1mm}&&\\
      \quad\ldots\quad & \quad x^3 \quad&\quad \ldots \quad\\
      \vspace{1mm}&&\\\hline
      \vspace{1mm}&&\\
      \quad\ldots & x^{1/5} & \ldots\\
      \vspace{1mm}&&\\\hline 
      \vspace{1mm}&&\\
      \quad\ldots & \cos(x) & \ldots \\
      \vspace{1mm}&&\\\hline 
      \vspace{1mm}&&\\
      \quad\ldots & \quad \frac{1}{1+x^2}\quad & \ldots \\
      \vspace{1mm}&&\\\hline 
    \end{array}
  \end{equation}
  \item Calculer les intégrales suivantes
    \begin{enumerate}
    \item $\displaystyle \int_2^3 \frac{1}{(x-1)(x+2)} \, dx$ ;
    \item $\displaystyle \int_0^{1/2}\frac{1}{\sqrt{1-x^2}}  \, dx$ ;
    \item $\displaystyle \int_{0}^{\pi/2} e^{\cos(x)}\sin(x) \, dx$. Conseil : utiliser un changement de variable. 
    \end{enumerate}
\item Calculer, par parties,  les primitives suivantes
    \begin{enumerate}
    \item $\displaystyle \int \ln(x) \, dx$ ;
    \item $\displaystyle \int x\sin(x) \, dx$.
    \end{enumerate}
\end{enumerate}


\corrref{analyseCTU-0014}
\end{exercice}
