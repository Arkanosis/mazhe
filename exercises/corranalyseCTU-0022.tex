% This is part of Analyse Starter CTU
% Copyright (c) 2014
%   Laurent Claessens,Carlotta Donadello
% See the file fdl-1.3.txt for copying conditions.

\begin{corrige}{analyseCTU-0022}

    \begin{enumerate}
        \item
            Non : la fonction peut «faire un saut» et redescendre juste en \( b\). Par exemple la fonction
            \begin{equation}    \label{EqKZMooQyBmFN}
                f(x)=\begin{cases}
                    x    &   \text{si \( -1\leq x\leq 5\)}\\
                    x-8    &    \text{si \( x>5\)}
                \end{cases}
            \end{equation}
            est croissante sur \( \mathopen[ -1 , 5 \mathclose]\) et sur \( \mathopen] 5 , 10 \mathclose]\) mais \( f(4)=4\) alors que \( f(6)=-2\).

            Il est recommandé de faire un dessin de la fonction \eqref{EqKZMooQyBmFN}.

        \item
            Une fonction \( f\) est toujours surjective vers l'intervalle \( f(I)\). Nous devons donc chercher une fonction qui ne soit pas injective. La fonction sinus en est le parfait exemple. Nous pouvons donc proposer \( f_2(x)=\sin(x)\) et \( I=\eR\).
        \item
            Voici le graphique d'une fonction bijective de \( \mathopen[ 0 , 1 \mathclose]\) vers lui-même :
\begin{center}
   \input{Fig_GCNooKEbjWB.pstricks}
\end{center}
Le point blanc indique un point qui n'est pas sur le graphe. Ici \( f(1/2)=1\) et non \( f(1/2)=1/2\).
Soyez capable d'en donner une expression analytique.

    \end{enumerate}

\end{corrige}
