\begin{corrige}{devoir1-0001}
  \begin{enumerate}
  \item On commence par tracer la  parabole  $x=2y^2+4y+2$, la circonférence $x^2+y^2=4$ et les droites $y=-1.5$ et $y=1/2$. On voit tout de suite que l'aire délimitée par les quatre courbes est donnée par l'union de deux parties. Dans la première $\sqrt{4- y^2}\leq x\leq 2y^2+4y+2$, $y\in [0,0.5]$ et dans l'autre $2y^2+4y+2\leq x\leq \sqrt{4- y^2}$, $y\in [-1.5, 0]$. L'ensemble $A_1$ est contenu dans la deuxième, \ref{LabelFigLAfWmaN}. L'intérieur de $A_1$ est donné par $\Int(A_1) = \{ (x, y ) \in \eR^2 \; | \; 2y^2+4y+2<x< \sqrt{4- y^2},\, y\in ]-1.5, 0[ \}$, et sa frontière est l'union de 3 morceaux de courbe $\ell_1$, $\ell_2$, $\ell_3$:
      \begin{equation}
        \begin{aligned}
          &\ell_1=\{(x,y)\, |\, x=2y^2+4y+2,\, y\in [-1.5, 0] \}\\
          &\ell_2=\{(x,y)\, |\, x=\sqrt{4-y^2},\, y\in [-1.5, 0] \}\\
          &\ell_3=\{(x,y)\, |\, x\in [0.5, \sqrt{7/4}]\, y=-1.5 \}.
        \end{aligned}
      \end{equation}

%The result is on figure \ref{LabelFigLAfWmaN}. % From file LAfWmaN
\newcommand{\CaptionFigLAfWmaN}{}
\input{Fig_LAfWmaN.pstricks}

  \item L'ensemble $A_2$ est une courbe. Comme la fonction sinus est bornée $A_2$ est borné. L'intérieur de $A_2$ est vide, touts ses points sont de points de frontière. $A_2$ n'est pas fermé parce que il lui manque le morceau de droite verticale $I=\{0\}\times\{-1,1\}$, qui est partie de sa fermeture. En fait, lorsque $t$ tend vers $0$ la fonction sinus oscille de plus en plus vite et on peut trouver au moins  un point du type $(t,\sin(t))$ dans tout voisinage d'un point en $I$. Cet exemple est très important parce qu'il nous aide à comprendre l'expression \emph{point d'accumulation}.\ref{LabelFigYWxOAkh}. % From file YWxOAkh
\newcommand{\CaptionFigYWxOAkh}{}
\input{Fig_YWxOAkh.pstricks}

      
  \item L'ensemble $A_3$ n'est pas ouvert, ni fermé, ni borné dans la topologie de $\eR^2$. En fait, comme on a vu dans les exercices du cours, $\eQ$ a intérieur vide et sa fermeture est $\eR$. L'ensemble $\eN$, par contre est fermé et non borné. On peut remarquer que tous les points de $\eN$ sont points isolés. La fermeture de $A_3$ est alors $\eN\times \eR$ et son intérieur est vide. On peut dessiner la fermeture de cet ensemble comme une famille de droites verticales $x=n$, pour tout $n$ dans $\eN$.
    
  \end{enumerate}
\end{corrige}

% TODO : regarder si les figures de cette correction fonctionnent encore.


