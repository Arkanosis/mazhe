\begin{corrige}{Lineraire0003}

	Disons qu'au début, ils ont respectivement $x$, $y$ et $z$ écus. Lorsque le premier donne, il en donne $y$ au second et $z$ au troisième. Leurs avoirs sont donc
	\begin{equation}
		\begin{aligned}[]
			x-y-z\\
			2y\\
			2z.
		\end{aligned}
	\end{equation}
	À ce moment, le second donne $x-y-z$ au premier et $2z$ au troisième. Ils ont donc
	\begin{equation}
		\begin{aligned}[]
			2(x-y-z)\\
			2y-(x-y-z)-2z\\
			4z.
		\end{aligned}
	\end{equation}
	En simplifiant :
	\begin{equation}
		\begin{aligned}[]
			2(x-y-z)\\
			3y-x-z\\
			4z.
		\end{aligned}
	\end{equation}
	Maintenant, le troisième donne $2(x-y-z)$ au premier et $3y-x-z$ au second. Ils se retrouvent donc avec
	\begin{equation}
		\begin{aligned}[]
			4(x-y-z)\\
			2(3y-x-z)\\
			-x-y+7z.
		\end{aligned}
	\end{equation}
	Petite vérification : la somme des trois vaut bien $x+y+z$. Il faut maintenant juste maintenant résoudre le système donné en égalant ces trois quantités à $8$. La solution est donnée par $x=13$, $y=7$ et $z=4$.
\end{corrige}
% This is part of the Exercices et corrigés de mathématique générale.
% Copyright (C) 2009
%   Laurent Claessens
% See the file fdl-1.3.txt for copying conditions.
