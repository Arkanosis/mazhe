% This is part of the Exercices et corrigés de CdI-2.
% Copyright (C) 2008, 2009
%   Laurent Claessens
% See the file fdl-1.3.txt for copying conditions.


\begin{corrige}{114}


%TODO: refaire le dessin
%Un petit graphe de la fonction est donné à la figure \ref{LabelFigexouuiv}.
%\newcommand{\CaptionFigexouuiv}{Une des fonctions $f_n$ proposées.}
%\input{Fig_exouuiv.pstricks}

\begin{enumerate}
\item

%TODO: refaire le dessin et décommenter la correction
    %Il est facile de constater que la suite converge vers la fonction esquissée à la figure \ref{LabelFigexouuivbis}.
%\newcommand{\CaptionFigexouuivbis}{La limite de la suite de fonction de l'exercice \ref{corr114}.}
%\input{Fig_exouuivbis.pstricks}

\item Pour cause de discontinuité de la limite $f$ en $0$, nous n'avons pas de convergence uniforme, ni sur $\eR$, ni sur tout compact de $\eR$.
\end{enumerate}
\end{corrige}
