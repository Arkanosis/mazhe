% This is part of Exercices et corrigés de CdI-1
% Copyright (c) 2011
%   Laurent Claessens
% See the file fdl-1.3.txt for copying conditions.

\begin{exercice}\label{exoEqsDiff0005}

Soit $p(t)$ le nombre d'individus d'une population à l'instant $t$. Un modèle de population classique et très simple est celui régi par $p' = K p$ où $K \in \eR$ est le taux de croissance de la population (taux de natalité moins le taux de mortalité). La résolution de cette équation différentielle montre que la croissance de la population est exponentielle, ce qui est généralement satisfaisant tant qu'il n'y a pas de problèmes de surpopulation (territoire et ressource illimités).  Pour tenir compte de tels problèmes, on suppose plutôt que $p$ est régie par l'équation différentielle $p' = Kp(M-p)$ où $M \in \eR_0^+$ est le seuil de surpopulation.
\begin{enumerate}
\item
Sachant que $p(0) = p_0$ déterminez $p(t)$.
\item
Déterminez le comportement de $p(t)$ lorsque $t$ tend vers $+\infty $.
\item
Dessiner le champ de pentes correspondant à cette équation et en déduire l'allure des solutions sans utiliser la résolution de l'équation différentielle. Remarquez que $p=M$ est une solution stable, alors que $p=0$ est instable.
\end{enumerate}

\end{exercice}
