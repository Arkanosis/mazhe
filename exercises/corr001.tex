\begin{corrige}{001}
 
    Let us denote $E_1=\{(t,\sin(\frac{1}{t}))\}_{t\in]0,1[}$ (figure \ref{LabelFigTZCISko} ) and $E_2=\{(0,s)\}_{s\in[-1,1]}$. 
If one wants $E$ to be an embedded curve in $\eR^2$, one has to find maps $\dpt{ \varphi_{\alpha} }{ \mU_{\alpha} }{ E }$ such that the famous three conditions hold.

\newcommand{\CaptionFigTZCISko}{The graph of the function \( x\mapsto \sin(1/x).\)}
\input{pictures_tex/Fig_TZCISko.pstricks}

One of these maps must contain the point $a=(0,-1)$; let's say $\varphi(1)=a$. Consider the open set $B(a,r)$ of radius $r<\frac{ 1 }{2}$ around $a$. If $\varphi$ is continuous, there exists a $\delta$ such that $\epsilon\leq\delta$ implies $\varphi(1+\epsilon)\in B(a,r)$. 

It is possible to find an open neighbourhood $B'$ of $a$ which does not contain the oscillation on which $b$ lies. Let $b'=\varphi(1+\epsilon')\in B'$. Since $\varphi$ is continuous, the path $c\colon [\epsilon',\epsilon]\to C$ given by $c(t)=\varphi(1+t)$ is continuous (as path in $\eR^2$). A point of the image of $c$ has the form $\varphi(1+s)$ with $s\leq\epsilon\leq\delta$. So $c\big( [\epsilon',\epsilon] \big)\subset B(a,r)$.

But the fact that $c$ reaches $b'$ from $b$ which is not on the same oscillation forces at least one point of $c$ to be on the top of an oscillation and then to get out from $B(a,r)$. This contradicts the continuity.

\end{corrige}
