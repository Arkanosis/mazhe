% This is part of Exercices de mathématique pour SVT
% Copyright (c) 2010
%   Laurent Claessens et Carlotta Donadello
% See the file fdl-1.3.txt for copying conditions.

\begin{corrige}{TD5-a-0003}

	\begin{enumerate}
		\item
			$-2$;
		\item
			$2e^2-e$;
		\item
			$2\ln(2)-1$;
		\item
			$-\frac{ 1 }{2} e^{-\pi}-\frac{ 1 }{2}$;
		\item
			Étant donné que c'est une intégrale qui va revenir plus tard, essayons de la faire d'abord avec des bornes générales $a$ et $b$. Soit donc à calculer
			\begin{equation}
				\int_a^b\sin^2(x)dx.
			\end{equation}
			Nous faisons par partie :
			\begin{equation}
				\begin{aligned}[]
					f&=\sin(x)	&g'&=\sin(x)\\
					f'&=\cos(x)	&g&=-\cos(x).
				\end{aligned}
			\end{equation}
			Nous avons donc
			\begin{equation}		\label{EqRelccscintab}
				\int_a^b\sin^2(x)=\left[ -\sin(x)\cos(x) \right]_a^b+\int_a^b\cos^2(x)dx.
			\end{equation}
			Ajoutons la quantité $\int_a^b\sin^2(x)dx$ des deux côtés de cette équation. Dans le membre de gauche nous trouvons $2\int_a^b\sin^2(x)dx$. Dans le membre de droite, nous avons par contre
			\begin{equation}
				\left[ -\sin(x)\cos(x) \right]_a^b+\int_a^b\cos^2(x)dx+\int_a^b\sin^2(x)dx.
			\end{equation}
			La somme des deux intégrales est en réalité $\int_a^b\cos^2(x)+\sin^2(x)dx=\int_a^b1dx=b-a$.

			Au final, nous avons
			\begin{equation}
				2\int_a^b\sin^2(x)dx=\left[ -\sin(x)\cos(x) \right]_a^b+(b-a).
			\end{equation}
			À partir de là, nous pouvons trouver aussi (gratuitement) la valeur de $\int_a^b\cos^2(x)dx$ en substituant dans l'équation \eqref{EqRelccscintab}.


			Pour être plus schématique, ce que nous avons fait est la chose suivante. Si nous notons $A=\int_a^b\sin^2(x)dx$, $B=\int_a^b\cos^2(x)dx$ et $c=\left[ -\sin(x)\cos(x) \right]_a^b$, la relation \eqref{EqRelccscintab} nous dit que
			\begin{equation}
				A=c+B.
			\end{equation}
			Mais nous savons que
			\begin{equation}
				A+B=\int_a^b\sin^2(x)dx+\cos^2(x)dx=b-a.
			\end{equation}
			Nous devons donc simplement résoudre le système
			\begin{subequations}
				\begin{numcases}{}
					A=c+B\\
					A+B=b-a.
				\end{numcases}
			\end{subequations}
			Si nous trouvons $A$ et $B$ en termes de $c$ et $(b-a)$, nous avons trouvé toutes les intégrales qui nous intéressent. En substituant la valeur de $A$ donné par la première équation dans la seconde, nous trouvons
			\begin{equation}
				c+B+B=(b-a),
			\end{equation}
			c'est à dire 
			\begin{equation}
				B=\frac{ 1 }{2}(b-a)-\frac{ c }{2}.
			\end{equation}
			En remettant cette valeur de $B$ dans la première, nous trouvons la valeur de $A$ :
			\begin{equation}
				A=c+B=c+\frac{ 1 }{2}(b-a)-\frac{ c }{2}=\frac{ 1 }{2}(b-a)+\frac{ c }{2}.
			\end{equation}
			Au final, ce que nous avons trouvé est que
			\begin{subequations}
				\begin{align}
					\int_a^b\sin^2(x)dx&=\frac{ 1 }{2}(b-a)+\frac{ 1 }{2}\left[ -\sin(x)\cos(x) \right]_a^b		\label{subEqIntsincdxab}\\
					\int_a^b\cos^2(x)dx&=\frac{ 1 }{2}(b-a)-\frac{ 1 }{2}\left[ -\sin(x)\cos(x) \right]_a^b.	\label{subEqIntcoscdxab}
				\end{align}
			\end{subequations}
			Dans le cas de cet exercice, nous devons trouver $\int_0^{\pi/2}\cos^2(x)dx$, c'est à dire poser $a=0$ et $b=\pi/2$ dans l'équation \eqref{subEqIntcoscdxab} :
			\begin{equation}
				\int_0^{\pi/2}\cos^2(x)dx=\frac{ 1 }{2}\frac{ \pi }{ 2 }-\frac{ 1 }{2}\Big( -\sin(\pi/2)\underbrace{\cos(\pi/2)}_{=0}+\underbrace{\sin(0)}_{=0}\cos(0) \Big)=\frac{ \pi }{ 4 }.
			\end{equation}
		\item
            $-\frac{ \ln(2) }{ 3 }+\frac{ 8\ln(4) }{ 3 }-\frac{ 7 }{ 9 }=5\ln(2)-\frac{ 7 }{ 9 }$.
	\end{enumerate}

\end{corrige}
