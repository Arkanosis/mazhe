% This is part of Analyse Starter CTU
% Copyright (c) 2014
%   Laurent Claessens,Carlotta Donadello
% See the file fdl-1.3.txt for copying conditions.

\begin{corrige}{autoanalyseCTU-52}

 
\begin{enumerate}
\item L'ensemble de définition de $f$  est $\Dom_f = \{x\in\eR\text{ tels que } x\neq 0\text{ et } 1+x >0\} = ]-1,0[\cup]0,+\infty[$.
\item Il faut calculer la limite de $f$ lorsque $x\to 0$. Pour le faire nous avons besoin du théorème de de L'H\^opital
  \begin{equation*}
    \lim_{x\to 0} f(x) = \lim_{x\to 0}\frac{\ln(1+x)-x}{x^2} = \lim_{x\to 0}\frac{\frac{1}{1+x}-1}{2x} = \lim_{x\to 0}\frac{-x}{2x(1+x)} = -\frac{1}{2}.
  \end{equation*}
La fonction qui prolonge $f$ en $x=0$ est 
\begin{equation*}
  \tilde f(x) = 
  \begin{cases}
    f(x), &\qquad \text{si } x\neq 0 ;\\
    -\frac{1}{2}, &\qquad \text{si } x= 0. 
  \end{cases}
\end{equation*}
\end{enumerate}


\end{corrige}   
