% This is part of the Exercices et corrigés de CdI-2.
% Copyright (C) 2008, 2009
%   Laurent Claessens
% See the file fdl-1.3.txt for copying conditions.


\begin{corrige}{116}

La première chose à faire est d'un peu regarder à quoi ressemble la fonction, et en particulier, calculer sa dérivée qui va nous permetre de trouver le maximum de la fonction :
\begin{equation}
	f'_n(x)=n e^{-nx^2}(1-2nx^2).
\end{equation}
D'autre part, pour tout $x\in[0,1]$, on a évidement $\lim_{n\to\infty}f_n(x)=0$, donc $f_n\to 0$. Pour l'uniforme convergence, nous calculons
\begin{equation}
	\sup_{x\in[0,1]}| f_n(x)-f(x) |=\sup_{x\in[0,1]}| f_n(x) |=\max_{x\in[0,1]}f_n(x)= e^{-1/2}\sqrt{\frac{ n }{ 2 }}.
\end{equation}
Dans ce calcul, le suprémum de la valeur absolue est devenu le maximum sans valeur absolue parce que $f_n$ est positive et continue, or une fonction continue sur un compact atteint ses bornes. En conclusion, il n'y a pas convergence uniforme sur $[0,1]$.

Une primitive de $f_n$ est donnée par
\begin{equation}
	F_n(x)=-\frac{  e^{-nx^2} }{2},
\end{equation}
et on vérifie que 
\begin{equation}
	\begin{aligned}[]
		\lim_{n\to\infty}\int_0^1f_n(x)dx=\frac{ 1 }{2}	&&	\text{mais}	&&	\int_0^1\lim_{n\to\infty}f_n(x)dx=0.
	\end{aligned}
\end{equation}
Le fait qu'il n'y ait pas égalité nous permet de vérifier immédiatement que la suite ne converge pas uniformément, en vertu du théorème \ref{ThoUnifCvIntRiem}. En particulier, nous notons que
\begin{equation}		\label{Eqintfexpemoin}
	\int_0^1f_n(x)dx= \frac{ 1 }{2}(1- e^{-n}).
\end{equation}

\end{corrige}
