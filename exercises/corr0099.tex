% This is part of Exercices et corrigés de CdI-1
% Copyright (c) 2011
%   Laurent Claessens
% See the file fdl-1.3.txt for copying conditions.

\begin{corrige}{0099}

D'abord, remarquons que $f'(0)=1$, donc $f'>0$ sur un voisinage de $0$. Donc, si $\epsilon$ est dans ce voisinage, $f(\epsilon)>1$. Disons $f(\epsilon)=1+\delta$. En utilisant la propriété d'additivité, nous avons
\begin{equation}
	f(\epsilon+\epsilon)=f(\epsilon)^2=(1+\delta)^2>1,
\end{equation}
et plus généralement,
\begin{equation}
	f(n\epsilon)=(1+\delta)^n.
\end{equation}
Évidement, $(1+\delta)^n$ tend vers l'infini lorsque $n$ tend vers l'infini. Nous en déduisons que $\mathopen[1,\infty[\subset\Image(f)$.

Reste à voir que les valeurs entre $0$ et $1$ sont dans l'image et que les valeurs négatives n'y sont pas. Afin de voir que les valeurs entre $1$ et $0$ sont dans l'image de $f$, nous prenons le même argument que précédemment, mais en utilisant le fait que $f(-\epsilon)=1-\delta$ et que $(1-\delta)^n\to 0$ lorsque $n\to\infty$.

Montrons que $f(x)\leq0$ n'est pas possible. Bien entendu, $f(x)=0$ n'est pas possible parce que nous aurions, pour tout $y$ l'identité $f(x+y)=f(x)f(y)=0$, et donc $f=0$.

Si $f(x_0)<0$, alors il existe $x_1>x$ tel que $f(x_1)=0$ (parce que nous avons déjà montré que $f(n\epsilon)\to\infty$ quand $n\to\infty$). Prenons
\begin{equation}
	x_1=\inf\{ x>x_0\tq f(x)=0 \}.
\end{equation}
Étant donné que $f$ est continue sur $\mathopen[x_0,x_1\mathclose]$, il existe $c\in\mathopen]x_0,x_1\mathclose[$ tel que $f'(c)>0$. Mais,  par définition de $x_1$, nous avons $f(c)<0$, cela contredit la propriété $f'=f$.

\end{corrige}
