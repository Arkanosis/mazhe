% This is part of the Exercices et corrigés de mathématique générale.
% Copyright (C) 2009-2010
%   Laurent Claessens
% See the file fdl-1.3.txt for copying conditions.


\begin{corrige}{INGE1121La0021}

	\begin{enumerate}

		\item
			La matrice $A$ est symétrique. Il existe donc une matrice orthogonale qui la diagonalise.
		\item
		\item
		\item
			Le polynôme caractéristique est donné par
			\begin{equation}
				\det\begin{pmatrix}
					1-\lambda	&	2	\\ 
					1	&	2-\lambda	
				\end{pmatrix}=(1-\lambda)(2-\lambda)-2.
			\end{equation}
			Les solutions sont $\lambda_1=0$ et $\lambda_2=3$.

			Pour trouver le vecteur propre correspondant à la valeur zéro, on résous le système homogène de la matrice
			\begin{equation}
				\begin{pmatrix}
					1	&	2	\\ 
					1	&	2	
				\end{pmatrix}.
			\end{equation}
			La réponse est donnée par le vecteur $v_1=\begin{pmatrix}
				-2	\\ 
				1	
			\end{pmatrix}$ et tous ses multiples.

			Pour trouver les vecteurs de la valeur $3$, nous résolvons le système de
			\begin{equation}
				\begin{pmatrix}
					-2	&	2	\\ 
					1	&	-1	
				\end{pmatrix},
			\end{equation}
			et la réponse est $v_2=\begin{pmatrix}
				1	\\ 
				1	
			\end{pmatrix}$.

			Les deux vecteurs $v_1$ et $v_2$ n'étant pas orthogonaux, il n'est pas possible de trouver une matrice orthogonale qui diagonalise la matrice de départ.

	\end{enumerate}

\end{corrige}
