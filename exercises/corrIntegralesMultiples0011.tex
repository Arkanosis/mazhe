\begin{corrige}{IntegralesMultiples0011}

  \begin{enumerate}
  \item Nous passons aux coordonnées polaires $x=r\cos(\theta)$ et $y=r\sin(\theta)$. Le domaine d'intégration est alors $D=\{(r,\theta) \,:\, r\in[\pi,2\pi],\: \theta \in[0, 2\pi]\}$. Il s'agit d'une couronne circulaire. 
    \begin{equation}
      \begin{aligned}
        \int_{0}^{2\pi}\int_{\pi}^{2\pi} \sin(r) \,r \, dr d\theta=&\int_{0}^{2\pi}\left(\left[-r\cos(r)\right]_{\pi}^{2\pi}+\int_{\pi}^{2\pi}\cos(r)\right)\,d\theta=\\
        &= \int_{0}^{2\pi}-3\pi+ \left[\sin(r)\right]_{\pi}^{2\pi}\,d\theta=-6\pi^2.
      \end{aligned}
    \end{equation}
  \item Nous passons aux coordonnées elliptiques $x=ar\cos(\theta)$ et $y=br\sin(\theta)$. La description du domaine est alors très simple $D=[0,1]\times[0,2\pi]$. La fonction à intégrer devient $a^2r^2\cos^2(\theta)+b^2r^2\sin^2(\theta)$. Le jacobien du changement de variable est le détermiant
      \begin{equation}
          \begin{vmatrix}
              \frac{ \partial x }{ \partial r }  &   \frac{ \partial x }{ \partial \theta }    \\ 
              \frac{ \partial y }{ \partial r }  &   \frac{ \partial y }{ \partial \theta }    
          \end{vmatrix}=\begin{vmatrix}
              a\cos(\theta)  &   -ar\sin(\theta)    \\ 
              b\sin(\theta)  &   br\cos(\theta)    
          \end{vmatrix}=abr.
      \end{equation}
      On a 
    \begin{equation}
      \begin{aligned}
        &\int_{0}^{2\pi}\int_{0}^{1} \left(a^2r^2\cos^2(\theta)+b^2r^2\sin^2(\theta)\right) \,abr \, dr d\theta=\\
        &=ab\frac{1}{4}\int_{0}^{2\pi}\left(a^2\cos^2(\theta)+b^2\sin^2(\theta)\right)\,d\theta=\\
        &=ab\frac{a^2}{4}\int_{0}^{2\pi}\cos^2(\theta)d\theta+ \frac{b^2}{4} \int_{0}^{2\pi}\sin^2(\theta)d\theta\\
        &=ab\frac{(a^2+b^2)\pi}{4}.
      \end{aligned}
    \end{equation}
  \item  Il faut passer aux coordonnées cylindriques $x=r\cos(\theta)$, $y=r\sin(\theta)$ et $z=z$. Le domaine d'intégration est $D=\{(r,\theta, z)\,:\, r\in[0,1], \theta\in[0,2\pi], z\in[0,h]\}$. Le jacobien de cette transformation est $r$ comme pour les coordonnées polaires 
    \begin{equation}
      \begin{aligned}
        &\int_{0}^{2\pi}\int_{0}^{1} \int_{0}^{h} z \,r \,dz dr d\theta=\\
        &=\int_{0}^{2\pi}\int_0^1 \frac{rh^2}{2} \,dr d\theta=\\
        &=\frac{\pi h^2}{2}.
      \end{aligned}
    \end{equation}
  \item Le domaine d'intégration est le volume entre deux sphères centrées dans l'origine. Leur rayons respectifs sont $1$ et $2$. Il est donc naturel de travailler ici en coordonnées sphériques.  Le jacobien est $\rho^2\sin(\phi)$ et la fonction à intégrer est $\rho^{2\alpha}$. Remarque : toute puissance de $\rho$ est intégrable dans l'intervalle $[1,2]$. Cependant il faut écrire séparément les cas $\alpha=-3/2$ et $\alpha\neq -3/2$. Si $\alpha\neq -3/2$ alors
    \begin{equation}
      \begin{aligned}
        &\int_{0}^{\pi}\int_{0}^{2\pi} \int_{1}^{2} \rho^{2(\alpha+1)}\sin(\phi) \,d\rho d\theta d\phi=\\
        &=2\pi\left[\frac{\rho^{2(\alpha+1)+1}}{2(\alpha+1)+1}\right]_{1}^{2}\int_{0}^{\pi} \sin(\phi) d\phi=\\
        &=\frac{4\pi}{2(\alpha+1)+1} \left(2^{2(\alpha+1)+1}-1\right).
      \end{aligned}
    \end{equation}
    Si $\alpha=-3/2$ alors
    \begin{equation}
        \int_{0}^{\pi}\int_{0}^{2\pi} \int_{1}^{2} \rho^{-1}\sin(\phi) \,d\rho d\theta d\phi=4\pi\ln(2).
    \end{equation}
    \item Voir l'exercice \ref{exoIntegralesMultiples0010}.
    \item Le domaine d'intégration est l'huitième de la  sphère unitaire contenu dans le premier octant. On peut utiliser des coordonnées sphériques ou cylindriques. Ce corrigé est en coordonnées cylindriques, mais vous pouvez essayer les sphériques pour voir si vous obtenez le même résultat. On décrit le domaine d'intégration de la façon suivante :
      \begin{equation}
        \begin{array}{l}
          z\in[0,1]\\
          r\in[0,\sqrt{1-z^2}]\\
          \theta\in[0,\pi/2].
        \end{array}
      \end{equation}
      Le jacobien est simplement $r$ et l'intégrale devient
      \begin{equation}
        \begin{aligned}
          &\int_{0}^{1}\int_{0}^{\pi/2}\int^{\sqrt{1-z^2}}_{0} r^3z\cos(\theta)\sin(\theta)\, dr d\theta dz=\\
          &=\frac{1}{4}\int_{0}^{1}\int_{0}^{\pi/2}(1-z^2)^2z\cos(\theta)\sin(\theta)\, d\theta dz =\\
          &=\frac{1}{4}\int_{0}^{1}(1-z^2)^2z \,dz \,\int_{0}^{\pi/2}\cos(\theta)\sin(\theta)\, d\theta
        \end{aligned}
      \end{equation}
      Avec les changements de variables $\tilde z = 1-z^2$ et $t = \sin(\theta)$  on obtient enfin 
      \begin{equation}
        \frac{1}{4} \left[\frac{1}{6} \tilde z^3\right]_{0}^{1} \left[\frac{t^2}{2}\right]_{0}^{1}= \frac{1}{48}.
      \end{equation}
  \end{enumerate}

\end{corrige}
