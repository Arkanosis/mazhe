\begin{corrige}{4}

La fonction  $\ln(x^2+\frac{ 5 }{ 9 })$ est définie tant que $x^2+\frac{ 5 }{ 9 }>0$, c'est à dire toujours. La dérivée est
\begin{equation}
	f'(x)=\frac{ 2x }{ x^2+\frac{ 5 }{ 9 } },
\end{equation}
qui a le signe de $x$. Donc
\begin{enumerate}

\item $f(x)$ est croissante pour $x>0$,
\item $f(x)$ est décroissante pour $x<0$.
\end{enumerate}

Nous calculons la dérivée seconde :
\begin{equation}
	f''(x)=-\frac{ 162 x^2 -9}{ 81x^4+9x^2+25 }.
\end{equation}
Le dénominateur est toujours positif, donc cette dérivée a le signe de $9-162 x^2$. Il y a un point d'inflexion en
\begin{equation}
	x=\pm\frac{ 1 }{ 3\sqrt{2} }.
\end{equation}
Il y a un extrema en $f'(x)=0$, c'est à dire en $x=0$, et c'est un minimum (regarder la croissance pour le voir).

Il n'y a aucune asymptote parce que $f(x)\to \infty$ pour $x\to\pm\infty$, et 
\begin{equation}
	\lim_{x\to\infty}\frac{ f(x) }{ x }=0,
\end{equation}
voir page $65$ du cours.

%TODO : refaire le dessin
%La fonction est tracée à la figure \ref{LabelFigFonctionfIV}
%\newcommand{\CaptionFigFonctionfIV}{La fonction de l'exercice \ref{exo4}}.
%\input{Fig_FonctionfIV.pstricks}

\end{corrige}
