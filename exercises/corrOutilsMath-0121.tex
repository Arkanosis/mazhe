% This is part of Outils mathématiques
% Copyright (c) 2011
%   Laurent Claessens
% See the file fdl-1.3.txt for copying conditions.

\begin{corrige}{OutilsMath-0121}

    Le volume que l'on considère est paramétré par
    \begin{subequations}
        \begin{numcases}{}
            x\colon 0\to 1\\
            y\colon 0\to 1\\
            z\colon 0\to xy.
        \end{numcases}
    \end{subequations}
    \begin{enumerate}
        \item
            Le volume est alors donné par
            \begin{equation}
                V=\int_0^1dx\int_0^1dy\int_0^{xy}dz 1=\int_0^1dx\int_0^1 xy\,dy=\frac{1}{ 4 }.
            \end{equation}
            
        \item
            L'intégrale à calculer est
            \begin{equation}
                \begin{aligned}[]
                    \int_0^1dx\int_0^1dy\int_0^{xy}(x+y)&=\int_0^1dx\int_0^1dy(x+y)[z]_0^{xy}\\
                    &=\int_0^1dx\int_0^1 xy(x+y)dy\\
                    &=\frac{1}{ 3 }.
                \end{aligned}
            \end{equation}
            
    \end{enumerate}

\end{corrige}
