% This is part of Exercices et corrigés de CdI-1
% Copyright (c) 2011,2013
%   Laurent Claessens
% See the file fdl-1.3.txt for copying conditions.

\begin{exercice}\label{exoTP20090001}

Un ensemble $\Omega\subset\eR^n$ est dit \defe{étoilé par rapport à}{ensemble étoilé} $x_0\in\Omega$ si pour tout $y \in \Omega$, le segment
\begin{equation*}
	[x_0, y] \stackrel{def}{=} \{ (1-t) x_0 + t y \tq t \in  [0,1] \}
\end{equation*}
est inclus à $\Omega$. Un ensemble $\Omega$ étoilé par rapport à un de ses points est dit \defe{étoilé}{ensemble étoilé}.

\begin{enumerate}
\item
Soit $f\colon A\subset \eR\to\eR$ une fonction dérivable sur l'ensemble connexe $A$. Montrer que si $f'(a)=0$ pour tout $a\in A$, alors $f$ est constante sur $A$.

\item
Soit $f : \Omega \subset \eR^n \to \eR$ où $\Omega$ est étoilé. Montrer que si $f \in C^1(\Omega,{\eR})$ est telle que $df_a = 0$ pour tout $a \in \Omega$, alors $f$ est constante.

\end{enumerate}

\corrref{TP20090001}
\end{exercice}
