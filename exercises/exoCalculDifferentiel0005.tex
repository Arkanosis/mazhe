\begin{exercice}\label{exoCalculDifferentiel0005}

Soit $f:\eR^2\to\eR$ une fonction $C^1$. 
\begin{enumerate}
	\item
Montrer que si $f$ admet un minimum global au point $(x_0,y_0)$ (c'est à dire que pour tout  $(x,y)\in\eR^2$, nous avons $f(x,y)\geq f(x_0,y_0)$), alors
\begin{equation}
	\begin{aligned}[]
		\frac{\partial f}{\partial x}(x_0,y_0) =0 &&\text{ et } &&  \frac{\partial f}{\partial y}(x_0,y_0) =0.
	\end{aligned}
\end{equation}
Pour ce faire, considérer les fonctions d'une variable réelle $f_1(t) = f(x_0+t,y_0)$ et $f_2(t)  =f(x_0,y_0+t)$.
\item
 Montrer que $f(x,y) = x^2 + 2xy + y^2 +x$ n'admet pas de minimum.
\item
 Montrer que $f(x,y) = x^2 - y^2$ n'admet pas de minimum.
\item
 Montrer que si $f(x,y) = 2x^2 + 2xy + y^2 +x + 3y$ admet un minimum alors ce minimum est atteint au point $(x_0,y_0) = (1, -5/2)$. Montrer que $f$ admet effectivement un minimum en ce point. 

 Indice : remarquer que pour tous réels $h$ et $k$, $f(1+h,-5/2+k) -  f(1,-5/2) \geq 0$.
\end{enumerate}


\corrref{CalculDifferentiel0005}
\end{exercice}
