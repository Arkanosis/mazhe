% This is part of Exercices et corrigés de CdI-1
% Copyright (c) 2011,2015
%   Laurent Claessens
% See the file fdl-1.3.txt for copying conditions.

\begin{corrige}{0064}

Si $h_k$ désigne le $k$ième terme de la série harmonique, la somme proposée n'est autre que $\sum_k(h_k-h_k)$. Nous pouvons le réarranger en espaçant les $-h_k$ de telle manière à laisser la série monter de $1$ entre deux arrivée d'un terme négatif. Plus précisément,
\begin{equation}		\label{EqSerieArran0064}
	h_1,\ldots,h_{S_1},-h_1,h_{S_1+1},\ldots,h_{S_2},-h_2,h_{S_2+1},\ldots,h_{S_3},\ldots	
\end{equation}
où les indices $S_k$ sont définis de telle sorte à avoir
\begin{equation}
	(\sum_{i=S_k}^{S_{k+1}}h_i)-h_k>1.
\end{equation}
De cette façon, la série des sommes partielles de \eqref{EqSerieArran0064} est divergente.

\end{corrige}
