% This is part of Exercices et corrigés de CdI-1
% Copyright (c) 2011
%   Laurent Claessens
% See the file fdl-1.3.txt for copying conditions.

\begin{corrige}{OptimSS0006}

D'abord, remarquer que $f\equiv 0$ sur le bord du domaine. La fonction est également toujours positive à l'intérieur du domaine. Nous avons donc minimum global sur les bords, et nous ne recherchons que des minima locaux à l'intérieur. Les équations pour les points critiques sont
\begin{subequations}
\begin{numcases}{}
	\partial_xf(x,y)=y(y-2x+1)=0\\	
	\partial_yf(x,y)=(x-1)(2y-x)=0,
\end{numcases}
\end{subequations}
dont les solutions sont les points $(0,0)$, $(\frac{ 2 }{ 3 },\frac{ 1 }{ 3 })$, $(1,0)$, $(1,1)$. Parmi eux, seul $(\frac{ 2 }{ 3 },\frac{ 1 }{ 3 })$ n'est pas sur le bord, donc c'est lui le maximum local.

\end{corrige}
