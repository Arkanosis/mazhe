% This is part of Exercices et corrigés de CdI-1
% Copyright (c) 2011
%   Laurent Claessens
% See the file fdl-1.3.txt for copying conditions.

\begin{exercice}\label{exoEqsDiff0009}

La solution générale d'une équation différentielle du second ordre 
est typiquement une famille à deux paramètres de fonctions. Le 
problème de \og Cauchy\fg{}  consiste à chercher dans cette famille une 
fonction dont la valeurs en un point et la valeur de la dérivée en ce 
même point sont spécifiées. Le problème des \og valeur aux limites\fg{} 
consiste à chercher dans cette même famille une fonction dont les 
valeurs en deux points distincts sont spécifiées. Étudiez l'existence 
et l'unicité des solutions de ce problème dans le cadre des équations 
différentielles linéaires à coefficients constants. En particulier 
donnez un exemple de tel problème qui ne possède pas de solutions.

\end{exercice}
