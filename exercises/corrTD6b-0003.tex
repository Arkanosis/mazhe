% This is part of Exercices de mathématique pour SVT
% Copyright (c) 2010-2011
%   Laurent Claessens et Carlotta Donadello
% See the file fdl-1.3.txt for copying conditions.

\begin{corrige}{TD6b-0003}

    Ce sont les mêmes équations que pour l'exercice \ref{exoTD6b-0002}. Nous repartirons donc de la solution, et nous fixerons la constante.

    \begin{enumerate}
            

    \item

        En plaçant \( t=1\) dans la solution nous avons
        \begin{equation}
            y(1)=\ln\left( \frac{ 3 }{ 4 }+C \right),
        \end{equation}
        et par conséquent, pour avoir \( y(1)=1\), nous demandons
        \begin{equation}
            C=e-\frac{ 3 }{ 4 }.
        \end{equation}
        
    \item

        Nous avons \( y(0)=\tan(C)\). Demander \( y(0)=0\) demande \( \tan(C)=0\) c'est à dire \( C=k\pi\). Les solutions sont donc
        \begin{equation}
            y_k(t)=\tan(t+k\pi).
        \end{equation}
    \item
        C'est la même équation que la précédente. La condition sur \( C\) s'écrit \( \sin(C)=\cos(C)\), c'est à dire \( C=\frac{ \pi }{ 4 }+k\pi\). Les solutions sont donc
        \begin{equation}
            y_k(t)=\tan(t+k\pi+\frac{ \pi }{ 4 }).
        \end{equation}

    \item

        Pour cette équation nous n'étions pas parvenu à écrire une solution explicite, mais nous pouvons tout de même fixer la constante :
        \begin{equation}
            y(\pi/2)+ e^{y(\pi/2)}=\sin(\pi/2)+C,
        \end{equation}
        mais \( y(\pi/2)=3\) et \( \sin(\pi/2)=1\), donc
        \begin{equation}
            2+ e^{3}=C.
        \end{equation}
        L'équation qui donne \( y\) (et que nous ne savons pas résoudre) est alors
        \begin{equation}
            y+ e^{y}=\sin(t)+3+e^3.
        \end{equation}
        
    \item
        
        La solution en \( t=1\) est
        \begin{equation}
            y(1)=-\frac{ 1 }{ 1+C }.
        \end{equation}
        Pour obtenir \( y(1)=2\) nous avons besoin de \( C=-3/2\). La solution est donc
        \begin{equation}
            y(t)=-\frac{1}{ t-\frac{ 3 }{ 2 } }.
        \end{equation}
        
    \item

        Cette fois nous devons avoir
        \begin{equation}
            -\frac{1}{ 1+C }=0,
        \end{equation}
        ce qui est impossible. Voici donc un exemple d'équation différentielle avec condition initiale impossible.

    \item

        L'équation à résoudre pour \( C\) est
        \begin{equation}
            y(0)=\left( \frac{ 2 }{ 3 }C \right)^{-2/3}=-1.
        \end{equation}
        Cela est impossible. Notez que
        \begin{verbatim}
----------------------------------------------------------------------
| Sage Version 4.7.1, Release Date: 2011-08-11                       |
| Type notebook() for the GUI, and license() for information.        |
----------------------------------------------------------------------
sage: solve(x**(-2/3)==-1,x)
[x == I, x == -I]
        \end{verbatim}

        Sage trouve des solutions parmi les \wikipedia{fr}{Nombre_complexe}{nombres complexes}. Si vous voulez utiliser un ordinateur pour travailler, il faut pouvoir interpréter ses réponses !

    \item

        Ici nous devons résoudre
        \begin{equation}
            \left( \frac{ 2 }{ 3 }C \right)^{-2/3}=0,
        \end{equation}
        ce qui donne \( C=0\) et par conséquent
        \begin{equation}
            y(t)=\left( \frac{ 2t }{ 3 } \right)^{-2/3}.
        \end{equation}

    \end{enumerate}
\end{corrige}
