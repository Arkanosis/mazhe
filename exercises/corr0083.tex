% This is part of Exercices et corrigés de CdI-1
% Copyright (c) 2011
%   Laurent Claessens
% See the file fdl-1.3.txt for copying conditions.

\begin{corrige}{0083}

Afin de prouver que la boule est ouverte, nous allons utiliser le théorème \ref{ThoPartieOUvpartouv} : nous prenons un point $p\in B(x,r)$, et nous allons montrer qu'il existe une boule autour de $p$ qui est contenue dans $B(x,r)$.

Étant donné que $p\in B(x,r)$, nous avons $d(p,x)<r$. Prouvons que la boule $B\big(p,r-d(p,x)\big)$ est contenue dans $B(x,r)$. Pour cela, nous prenons $p'\in B\big(p,r-d(p,x)\big)$, et nous essayons de prouver que $p'\in B(x,r)$. En effet, en utilisant l'inégalité triangulaire,
\begin{equation}
	d(x,p')\leq d(x,p)+d(p,p')\leq d(x,p)+r-d(p,x)=r.
\end{equation}

\end{corrige}
