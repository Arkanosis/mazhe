% This is part of the Exercices et corrigés de mathématique générale.
% Copyright (C) 2009-2011
%   Laurent Claessens
% See the file fdl-1.3.txt for copying conditions.
\begin{corrige}{General0016}


La situation est représentée sur la figure \ref{LabelFigBateau}.

Cet exercice peut être résolu de façon simple en remarquant que le canot doit aller de $A$ vers $I$ puis de $I$ vers $B$. Ce trajet est le même que celui qui consiste à aller de $A$ vers $I$ puis de $I$ vers $B'$, si $B'$ est le symétrique de $B$ par rapport à la côte. Ce qu'il faut faire est donc simplement fixer $x$ pour que le trajet $AIB'$ soit une droite.

La réponse est $x=1$.

Sans cette astuce, la distance à parcourir, en fonction de $x$ s'exprime avec Pythagore :
\begin{equation}
	d(x)=\sqrt{x^2+9}+\sqrt{(4-x)^2+81},
\end{equation}
dont la dérivée est
\begin{equation}
	d'(x)=\frac{ x }{ \sqrt{x^2+9} }-\frac{ 4-x }{ \sqrt{(4-x)^2}+81 }.
\end{equation}
Après mise au même dénominateur, nous voyons que cela s'annule en $x=1$, comme précédemment déduit.

\newcommand{\CaptionFigBateau}{Petit dessin pour l'exercice \ref{exo0016}.}
\input{pictures_tex/Fig_Bateau.pstricks}

\end{corrige}
