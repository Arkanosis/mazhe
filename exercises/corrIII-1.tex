% This is part of the Exercices et corrigés de mathématique générale.
% Copyright (C) 2009
%   Laurent Claessens
% See the file fdl-1.3.txt for copying conditions.
\begin{corrige}{1}

\begin{enumerate}

\item
$f(x)=\ln(x^2+1)$, utiliser la formule $(\ln(u))'=\frac{ u' }{ u }$,
\begin{equation}
	f'(x)=\frac{ 2x }{ x^2+1 },
\end{equation}
$f'(2)=\frac{ 4 }{ 5 }$.

\item
$g(x)=x^x$, passer au logarithme :
 \begin{equation}
	x^x= e^{\ln(x^x)}= e^{x\ln(x)}.
\end{equation}
À partir de là, nous utilisons la formule $(e^u)'=u'e^u$ avec $u(x)=x\ln(x)$. Ce que nous trouvons est
\begin{equation}
	g'(x)=x^x(\ln(x)+1),
\end{equation}
$g'(e)=2 e^{e}$.

\item
$h(x)=x e^{\sin(x)}$, cela est un produit de $x$ par $ e^{\sin(x)}$. Chacun des deux facteurs est relativement facile à dériver. Ce que nous trouvons à la fin est 
\begin{equation}
	h'(x)=x\cos(x) e^{\sin(x)}+ e^{\sin(x)},
\end{equation}
$h'(\frac{ \pi }{ 6 })=\frac{ \pi\sqrt{3} }{ 12 }\sqrt{e}+\sqrt{e}$.

\item
$\ell(x)=\frac{ x }{ 5+\cos(x) }$, c'est une fraction sans surprises :
\begin{equation}
	\ell'(x)=\frac{ x\sin(x) }{ \big( \cos(x)+5 \big)^2 }+\frac{1}{ \cos(x)+5 },
\end{equation}
$\ell'(\frac{ \pi }{ 3 })=\frac{ 2\pi\sqrt{3} }{ 363 }+\frac{ 2 }{ 11 }$.

\end{enumerate}


\end{corrige}
