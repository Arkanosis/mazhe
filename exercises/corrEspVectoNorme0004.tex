\begin{corrige}{EspVectoNorme0004}
  \begin{enumerate}
  \item Si la suite $(x_n)_{n\in\eN}$ n'est pas bornée alors pour tout $M\in\eN$ il existe $n_M$ tel que $\|x_{n_M}\|>M$. La sous-suite $(x_{n_M})_{M\in\eN}$ ne converge pas dans $\eR$, donc $(x_n)_{n\in\eN}$ n'est pas convergente. Cela veut dire que être bornée est une condition nécessaire pour la convergence. 
    \item Soit $L\in\eR^N$ la limite de la suite $(x_n)_{n\in\eN}$. Alors pour chaque $\varepsilon >0$ fixé il existe un $\bar n$ tel que  si $n\geq \bar n$ 
      \begin{equation}
        \|x_n- L\|\leq \varepsilon.
      \end{equation}
      La suite $(\|x_n\|)_{n\in\eN}$ est convergente parce que pour tout $n\geq \bar n$ on a  
      \begin{equation}
        \left| \|x_{n}\|-\|L\|\right|\leq \|x_{n}-L\|\leq \varepsilon.
      \end{equation}
      On a utilisé la proposition \ref{PropNmNNm}.

      Notez que la suite $\| x_n \|$ peut être convergente sans que la suite $(x_n)$ soit elle-même convergente. Par exemple, la suite $x_n=(-1)^n$ n'est pas convergente, tandis que la suite des normes $\| x_n \|=1$ est constante, et donc convergente.
  \end{enumerate}
\end{corrige}
