% This is part of Exercices et corrigés de CdI-1
% Copyright (c) 2011
%   Laurent Claessens
% See the file fdl-1.3.txt for copying conditions.

\begin{corrige}{OutilsMath-0068}

    \begin{enumerate}
        \item
            Ce point est juste un calcul. Tout se simplifie.
        \item
            Nous devons avoir en même temps
            \begin{equation}
                \frac{ \partial V }{ \partial x }(x,y,z)=y(2x^2+1) e^{x^2+y^2}
            \end{equation}
            et
            \begin{equation}
                \frac{ \partial V }{ \partial y }(x,y,z)=x(1+2y^2) e^{x^2+y^2}.
            \end{equation}
            Cherchons un potentiel sous la forme
            \begin{equation}
                V(x,y)=f(x,y) e^{x^2+y^2}.
            \end{equation}
            Nous avons
            \begin{equation}
                \frac{ \partial V }{ \partial x }=\frac{ \partial f }{ \partial x } e^{x^2+y^2}+2x f e^{x^2+y^2}.
            \end{equation}
            Nous devons donc trouver $f$ telle que
            \begin{equation}
                \frac{ \partial f }{ \partial x }+2xf=y(2x^2+1).
            \end{equation}
            La fonction $f(x,y)=xy$ fonctionne.

            Il suffit donc de vérifier que le potentiel
            \begin{equation}
                V(x,y,z)=xy e^{x^2+y^2}
            \end{equation}
            est bien tel que $\nabla V=F$.

        \item
            Nous avons $V(1,0)=V(-1,0)=0$, par conséquent, quel que soit le chemin les reliant, nous avons toujours que la circulation le long d'un chemin joignant $(1,0)$ à $(-1,0)$ est nulle.
    \end{enumerate}

\end{corrige}
