% This is part of Exercices et corrigés de CdI-1
% Copyright (c) 2011
%   Laurent Claessens
% See the file fdl-1.3.txt for copying conditions.

\begin{corrige}{continueSupl1}

\begin{enumerate}
\item La fonction constante $f(x)=0$
\item La fonction
\begin{equation}
	f(x)=\begin{cases}
	1	&	\text{si $x=0$ ou si $x=1$}\\
	0	&	 \text{sinon.}
\end{cases}
\end{equation}

\item Impossible :  quelle que soit la valeur réelle mise en zéro, il y aura toujours un point où $f(x)$ sera plus grande que cette valeur dans tout voisinage de $x=0$.
\item La fonction
\begin{equation}		\label{EqSupl1Dirich}
	f(x)=\begin{cases}
	1	&	\text{si $x\in\eQ$}\\
	0	&	 \text{si $x\notin\eQ$.}
\end{cases}
\end{equation}
Celle-là, c'est un classique, ne l'oubliez pas !

\end{enumerate}

\end{corrige}
