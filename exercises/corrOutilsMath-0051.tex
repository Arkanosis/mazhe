% This is part of Exercices et corrigés de CdI-1
% Copyright (c) 2011
%   Laurent Claessens
% See the file fdl-1.3.txt for copying conditions.

\begin{corrige}{OutilsMath-0051}

    Le champ de force est donné par
    \begin{equation}
        F(x,y,z)=\begin{pmatrix}
            0    \\ 
            0    \\ 
            -mg    
        \end{pmatrix},
    \end{equation}
    et le chemin suivit est donné par 
    \begin{equation}
        \sigma(t)=\begin{pmatrix}
            \cos(t)    \\ 
            \sin(t)    \\ 
            t    
        \end{pmatrix}.
    \end{equation}
    La première chose à faire est de calculer la vitesse du chemin (c'est à dire sa dérivée) :
    \begin{equation}
        \sigma'(t)=\begin{pmatrix}
            -\sin(t)    \\ 
            \cos(t)    \\ 
            1    
        \end{pmatrix}.
    \end{equation}
    Ensuite, la quantité à intégrer est le produit scalaire $F\big( \sigma(t) \big)\cdot\sigma'(t)$ :
    \begin{equation}
        \begin{pmatrix}
            0    \\ 
            0    \\ 
            -mg    
        \end{pmatrix}\cdot\begin{pmatrix}
            -\sin(t)    \\ 
            \cos(t)    \\ 
            1    
        \end{pmatrix}=-mg.
    \end{equation}
    Par conséquent le travail est donné par
    \begin{equation}
        W=\int_0^h-mg dt=-mgh.
    \end{equation}

\end{corrige}
