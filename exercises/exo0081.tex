% This is part of Exercices et corrigés de CdI-1
% Copyright (c) 2011
%   Laurent Claessens
% See the file fdl-1.3.txt for copying conditions.

\begin{exercice}\label{exo0081}

Deux métriques à peine bizarres\ldots

\begin{enumerate}
\item
 Pour tout ensemble non vide $X$ (par exemple l'ensemble des habitants de Mars, l'ensemble des étoiles fixes dans la voie lactée, ou l'ensemble des bouteilles de bière à Bruxelles) nous définissons la \og métrique discrète\fg{}  par
\begin{equation*}
d(x,y) \,:=\,
\left\{
\begin{array}{cl} 
0 \quad &\text{si } x=y,  \\[.2em]
1 \quad &\text{si } x \neq y. 
\end{array} 
\right.
\end{equation*}
Expliquer pourquoi le nom est approprié. Montrer qu'il s'agit bien d'une métrique.
Décrire $B_r(x)$ pour tout $r>0$ et pour tout $x \in X$.

\item 
Dans $X = \eR^2$ nous regardons la \og métrique bureaucratique\fg{} (ou la \og métrique SNCF\fg)
définie par
\begin{equation*}
d(x,y) \,:=\,
\left\{
\begin{array}{ll} 
|x-y| \quad 
  &\text{si } x, y \text{ sont linéairement dépendants},  \\[.2em]
|x|+|y| \quad 
  &\text{si } x, y \text{ sont linéairement indépendants}.
\end{array} \right.
\end{equation*}
Expliquer pourquoi le nom est approprié. Montrer qu'il s'agit bien d'une métrique.
\end{enumerate}


\corrref{0081}
\end{exercice}
