% This is part of the Exercices et corrigés de mathématique générale.
% Copyright (C) 2009-2010
%   Laurent Claessens
% See the file fdl-1.3.txt for copying conditions.


\begin{corrige}{INGE1121La0006}

	\begin{enumerate}

		\item
			La matrice $C$ est triangulaire \emph{inférieure}, donc son déterminant n'est pas le produit des éléments diagonaux. Il y a une subtilité sur les signes. Si nous développons le déterminant selon la première ligne, seul le $1$ reste, mais il vient avec un signe :
			\begin{equation}
				\det\begin{pmatrix}
					0	&	0	&	0	&	0	&	0	&	1\\	
					0	&	0	&	0	&	0	&	2	&	1\\
					0	&	0	&	0	&	3	&	2	&	1\\
					0	&	0	&	4	&	3	&	2	&	1\\
					0	&	5	&	4	&	3	&	2	&	1\\
					6	&	5	&	4	&	3	&	2		1
				\end{pmatrix}=
				(-1)\det
				\begin{pmatrix}
					0	&	0	&	0	&	0	&	2	\\
					0	&	0	&	0	&	3	&	2	\\
					0	&	0	&	4	&	3	&	2	\\
					0	&	5	&	4	&	3	&	2	\\
					6	&	5	&	4	&	3	&	2				
				\end{pmatrix}
			\end{equation}
			Nous développons ce déterminant en suivant sa première ligne, et il ne reste que le $2$, qui vient avec un signe $+$. Notez que ce $2$, dans la matrice originale était sur une case qui aurait eut un signe moins.

			En continuant de la sorte, le déterminant à calculer est
			\begin{equation}
				(-1)\cdot 2\cdot(-3)\cdot 4\cdot(-5)\cdot 6=-720.
			\end{equation}
		
	\end{enumerate}

\end{corrige}
