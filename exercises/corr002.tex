\begin{corrige}{002}
The parametrization given in the question is not smooth because the derivative is discontinuous at $t=0$. Remark that tangent vectors in this parametrization must be parallel to $(1,-1)$ in the left part and to $(1,1)$ in the right part. The only way for a function (namely the vertical part of the derivative of the parametrization) to change sign without discontinuity is to vanish. One thus needs a parametrization whose derivative vanishes at $(0,0)$. This already answers the second question: this is not an embedded curve.
 
Consider the parametrization $\varphi(t)=\big( t^2, x(t) \big)$ where
\[ 
  x(t)=
\begin{cases}
 -t^2& \text{when $t\in\,]-1,0]$}\\
t^2  & \text{when $t\in\,]0,1[$}.
\end{cases}
\]
This parametrization doesn't work because the second derivative has a discontinuity at $t=0$. Try to convince yourself that a correct parametrization is given by
\[ 
  \varphi(t)=
\begin{cases}
(1,-1)f(t)&\text{if $t\in\,]-1,0]$}\\
(1,1)f(t) &\text{if $t\in\,]0,1[$}
\end{cases}
\]
where $\dpt{ f }{ \eR }{ \eR }$ is a function which gives $1$ at $1$ and $-1$ and vanishes at zero with all its derivatives. Such a function can be built from Cauchy's regularization function $\rho(x)=e^{-1/x^2}$.

The annihilation of \emph{all} derivatives of the parametrization at $(0,0)$ is required by the fact that on the left, a derivative of any order must be a multiple of $(1,-1)$ while on the right it must be a multiple of $(1,1)$.


\end{corrige}
