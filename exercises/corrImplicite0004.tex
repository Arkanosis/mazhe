% This is part of Exercices et corrigés de CdI-1
% Copyright (c) 2011,2015
%   Laurent Claessens
% See the file fdl-1.3.txt for copying conditions.

\begin{corrige}{Implicite0004}

	Cet exercice est particulièrement facile parce que l'équation
	\begin{equation}
		F(x,y,z)=(x^2+y^2+z^2-1,x^2+y^2-x)=(0,0)
	\end{equation}
	peut être résolue explicitement pour $Y(x)$ et $Z(x)$. En effet, la deuxième composante dit que
	\begin{equation}
		x^2+Y(x)^2-x=0,
	\end{equation}
	donc
	\begin{equation}
		Y^{(\pm)}(x)=\pm\sqrt{x-x^2}.
	\end{equation}
	Pour chacune de ces deux solutions possibles, l'équation pour la première composante devient
	\begin{equation}
		Z(x)^2=1-x,
	\end{equation}
    de telle façon à avoir
	\begin{equation}
		Z^{(\pm)}(x)=\pm\sqrt{1-x}.
	\end{equation}
	Il y a donc bien $4$ possibilités.

\end{corrige}
