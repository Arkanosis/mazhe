% This is part of Exercices et corrigés de CdI-1
% Copyright (c) 2011
%   Laurent Claessens
% See the file fdl-1.3.txt for copying conditions.

\begin{corrige}{IntMult0001}

\begin{enumerate}

\item
Comme expliqué dans les rappels (page \pageref{PgRapIntMultFubiniRect}), nous utilisons Fubini :
\begin{equation}
	\begin{aligned}[]
	\int_{[0,2]\times[0,1]}f(x,y)&=\int_0^2\int_0^1(4-x^2-y^2)\,dxdy\\
			&=\int_0^2\left[ 4x-\frac{ x^3 }{ 3 }-y^2x \right]_0^1dy\\
			&=\int_0^2\left( 4-\frac{1}{ 3 }-y^2 \right)dy\\
			&=\frac{ 14 }{ 3 }.
	\end{aligned}
\end{equation}

\item
Cet exercice a été fait à la page \pageref{PgRapIntMultFubiniTri}. L'intégrale à calculer est
\begin{equation}
	\int_0^2\left( \int_0^y(x^2+y^2)dx \right)dy=\frac{ 16 }{ 3 }.
\end{equation}

\item
Un dessin du domaine d'intégration est donné à la figure \ref{LabelFigIntDeuxCarres}. Une stratégie consiste à intégrer la fonction sur tout le grand carré, et ensuite soustraire l'intégrale sur le petit. Pour cela, nous utilisons la sous additivité de l'intégrale mentionnée au point $3$ de la page 436 du cours.
\newcommand{\CaptionFigIntDeuxCarres}{Un domaine d'intégration.}
\input{Fig_IntDeuxCarres.pstricks}
L'intégrale sur le grand carré vaut
\begin{equation}
		\int_{-2}^2\left( \int_{-2}^2 e^{x+y}dx \right)dy=\int_{-2}^2e^y[e^x]_{-2}^2dy=(e^2- e^{-2})^2,
\end{equation}
tandis que l'intégrale sur le petit carré central vaut
\begin{equation}
		\int_{-1}^1\left( \int_{-1}^1 e^xe^ydx \right)dy=\left( [e^x]_{-1}^1 \right)^2=\left( e-e^{-1} \right)^2.
\end{equation}
Au final,
\begin{equation}
	\int_Ef=(e^2-e^{-2})^2-(e-e^{-1})^2.
\end{equation}

\item
Nous commençons par intégrer verticalement. Le domaine va de $z=0$ à $z=1$. Pour chacun de ces $z$, la variable $y$ peut varier de $0$ à $1-z$, et pour chacun de ces $(z,y)$ fixés, la variable $x$ peut varier de $0$ à $1-z-y$, donc l'intégrale à calculer est
\begin{equation}		\label{EqLongCalzxyzInt}
	\begin{aligned}[]
		I	&=\int_0^1dz\big( \int_0^{1-z}dy\int_0^{1-z-y}dx\, (xyz) \big)\\
			&=\int_0^1dz\int_0^{1-z}\left[ \frac{ x^2 }{ 2 }yz \right]_0^{1-z-y}\\
			&=\frac{ 1 }{2}\int_0^1dz\int_{0}^{1-z}yz(z^2+2yz-2z+y^2-2y+1)dy\\
			&=\frac{ 1 }{2}\int_0^1dz\int_0^{1-z}[y(z^3-2z^2+z)+y^2(2z^2-2z)+y^3z]dy\\
			&=\frac{ 1 }{2}\int_0^1z(1-z)^4\left( \frac{ 1 }{2}+\frac{ 2 }{ 3 }+\frac{1}{ 4 } \right)dz\\
			&=\frac{ 17 }{ 24 }\int_0^1z(1-z)^4dz.
	\end{aligned}
\end{equation}
Cette dernière intégrale s'effectue en changeant de variable en posant $u=1-z$. Nous tombons\footnote{Je crois qu'il y a une faute de coefficient dans le calcul \eqref{EqLongCalzxyzInt}; en tout cas la réponse $1/720$ est correcte, parce que c'est ce que dit \href{http://fr.wikipedia.org/wiki/Maxima}{maxima}.\\
\texttt{integrate( integrate(integrate(x*y*z,x,0,1-z-y),y,0,1-z) ,z,0,1 ).}} sur
\begin{equation}
	I=\frac{ 17 }{ 24 }\int_0^1(u^4-u^5)du=\frac{1}{ 720 }.
\end{equation}

\item
Cet exercice est fait comme exemple à la page \pageref{PgOMRapIntMultFubiniBoutCercle}, l'intégrale à calculer est donnée à l'équation \eqref{PgOMRapIntMultFubiniBoutCercle}. Le résultat est
\begin{equation}
	I=\frac{ 3\pi-5\sqrt{2} }{ 36 }.
\end{equation}

\end{enumerate}

\end{corrige}
