\begin{corrige}{DS2011-0003}
  
    \begin{enumerate}
        \item
            \( 9,\sqrt{41},6\).
        \item
            Un vecteur qui a la même direction que \( (2,3)\) est un vecteur de la forme \( \lambda (2,3)\) pour un certain \( \lambda\). Nous fixons \( \lambda\) afin que la norme soit \( 1\), c'est à dire
            \begin{equation}
                \max\{ |2\lambda|,|3\lambda| \}=1
            \end{equation}
            Nous trouvons \( \lambda=1/3\) et par conséquent \( w_1=(\frac{ 2 }{ 3 },1)\).
        \item
            Nous avons les restrictions suivantes sur le domaine :
            \begin{subequations}
                \begin{numcases}{}
                    \ln(x+1)\neq 0\\
                    x+1>0\\
                    x^2-4\geq 0.
                \end{numcases}
            \end{subequations}
            L'intersection des trois conditions est \( x\geq 2\). Cela est donc le domaine de notre fonction. Les limites du domaine sont donc \( x=2\) et \( x=\infty\). Nous avons
            \begin{equation}
                \lim_{x\to 2} \begin{pmatrix}
                    x    \\ 
                    1/\ln(x+1)    \\ 
                    \sqrt{x^2-4}    
                \end{pmatrix}=\begin{pmatrix}
                    2    \\ 
                    1/\ln(3)    \\ 
                    9    
                \end{pmatrix}.
            \end{equation}
            La limite en l'infini n'existe pas parce que dans la définition de limite, nous demandons que le vecteur limite ait une norme finie. Autant pour les fonctions \( \eR\to \eR\), nous avons donné un sens à une expression du type \( \lim_{x\to \infty} f(x)=\infty\), dans le cas de fonctions \( \eR\to\eR^3\), nous n'avons pas donné de sens à des expressions du type
            \begin{equation}        \label{EqLimhqnsDS}
                \lim_{x\to 2} \begin{pmatrix}
                    x    \\ 
                    1/\ln(x+1)    \\ 
                    \sqrt{x^2-4}    
                \end{pmatrix}=\begin{pmatrix}
                    \infty\\
                    0\\
                    \infty
                \end{pmatrix}.
            \end{equation}
            La réponse \eqref{EqLimhqnsDS} est donc fausse.
    \end{enumerate}
\end{corrige}
