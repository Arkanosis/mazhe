% This is part of Mes notes de mathématique
% Copyright (c) 2012
%   Laurent Claessens
% See the file fdl-1.3.txt for copying conditions.

\begin{exercice}\label{exoreserve0005}

    Montrer que
    \begin{equation}
        \sum_{n=0}^{\infty}(n+1)x^n=\frac{1}{ (1-x)^2 }
    \end{equation}
    pour \( x\in\mathopen] -1 , 1 \mathclose[\).

\corrref{reserve0005}
\end{exercice}
