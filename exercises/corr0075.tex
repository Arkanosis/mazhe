% This is part of Exercices et corrigés de CdI-1
% Copyright (c) 2011
%   Laurent Claessens
% See the file fdl-1.3.txt for copying conditions.

\begin{corrige}{0075}

Montrons la première relation : soit $x \in \Omega$. Alors nous avons les équivalences suivantes~:
\begin{equation*}
  \begin{split}
    & x \in {\left(\bigcup_nA_n\right)}^c \iff \neg \left( x \in
      \bigcup_nA_n \right) \iff \neg \left( \exists n \tq x \in
      A_n \right)\\
    \iff & \forall n, \neg (x \in A_n)\iff \forall n, x \in
    \left(A_n\right)^c \iff x \in \bigcap_n \left(A_n\right)^c
  \end{split}
\end{equation*}
où $\neg$ désigne la négation logique (et $n$ est élément de $\eN$).

La deuxième relation se traite de façon similaire. On peut également la déduire de la première relation de la manière suivante~: pour montrer que ${\left(\cap_n A_n\right)}^c = \cup_n ({A_n}^c)$, il suffit de poser $B_n = A_n^c$, de prendre le complémentaire dans les deux membres et d'utiliser le fait que ${(X^c)}^c = X$ pour se ramener à la première relation.


\end{corrige}
