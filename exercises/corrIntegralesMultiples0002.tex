\begin{corrige}{IntegralesMultiples0002}

	La véritable difficulté de l'exercice est de décomposer le domaine correctement. Après, ce sont des intégrales à effectuer.
	\begin{enumerate}
		\item
			Dans le domaine, $x$ peut prendre n'importe quelle valeurs entre $0$ et $1$. Mais une fois que $x$ est fixé, la variable $y$ ne peut varier qu'entre $0$ et $1-x$. Nous décomposons donc l'intégrale de la façon suivante :
			\begin{equation}
				I_1=\int_0^1\left[ \int_0^{1-x}(x+y) e^{-x} e^{-y}dy \right]dx=\int_0^1\big( (x+1) e^{-x}-2 e^{-1} \big)dx=2-5 e^{-1}.
			\end{equation}
			
		\item
			La combinaison $x^2+y^2$ nous incite à passer en polaires. Le domaine est donné par les inéquations
			\begin{subequations}
				\begin{numcases}{}
					r^2<r\cos(\theta)\\
					r^2>r\sin(\theta).
				\end{numcases}
			\end{subequations}
			Étant donné que $r$ est toujours positif, on peut simplifier ces inéquations par $r$ sans toucher au sens des inégalités. Le domaine sera donc $\sin(\theta)<r$ et $r<\cos(\theta)$. Notons que quand $\sin(\theta)<0$, alors le domaine de variation de $r$ est $[0, \cos(\theta)[$ (et non $]\sin(\theta),\cos(\theta)[$). Il faudra donc intégrer séparément la partie du domaine avec $\sin(\theta)<0$ et celle avec $\sin(\theta)>0$. 

Si $\sin(\theta)<0$ la condition $\cos(\theta)>r\geq 0$ nous dit que $\theta$ varie entre $-\pi/2$ et $0$.

Si $\sin(\theta)>0$ la condition $\cos(\theta)>r>\sin(\theta)$ nous dit que $\theta$ varie entre $0$ et $\pi/4$.

L'intégrale à effectuer est donc
			\begin{equation}
				I_2=\underbrace{\int_{-\pi/2}^0\int_{0}^{\cos\theta} r^3\,d rd\theta}_{=\frac{ 3\pi }{ 64 }}+\underbrace{\int_0^{\pi/4}\int_{\sin\theta}^{\cos\theta}r^3\,drd\theta}_{=\frac{1}{ 8 }}=\frac{ 3\pi }{ 64 }+\frac{1}{ 8 }.
			\end{equation}

			Le dessin du domaine est lui aussi très intéressant à étudier. L'équation $x^2+y^2=x$ est un cercle parce que, en reformant le carré, nous avons successivement
			\begin{equation}
				\begin{aligned}[]
					x^2-x+y^2=0\\
					\left( x-\frac{ 1 }{2} \right)^2-\frac{1}{ 4 }+y^2=0\\
					\left( x-\frac{ 1 }{2} \right)^2+y^2=\left( \frac{ 1 }{2} \right)^2,
				\end{aligned}
			\end{equation}
			ce qui est l'équation du cercle de rayon $1/2$ et de centre $(\frac{ 1 }{2},0)$. Par conséquent l'inéquation $x^2+y^2<x$ correspond à l'intérieur de ce cercle.

			\newcommand{\CaptionFigDeuxCercles}{Domaine d'intégration pour l'exercice \ref{exoIntegralesMultiples0002}.\ref{ItemexoIntegralesMultiples0002ii}.}
			\input{Fig_DeuxCercles.pstricks}

			De la même façon, l'inéquation $x^2+y^2>y$ correspond à l'extérieur du cercle de rayon $\frac{ 1 }{2}$ et de centre $(0,\frac{ 1 }{2})$. Le domaine est dessiné sur la figure \ref{LabelFigDeuxCercles}. On y voit que pour la partie en dessous de $y=0$, il n'y a pas de contraintes sur le rayon (à part de rester dans le cercle).

%The result is on figure \ref{LabelFigIntTrois}
\newcommand{\CaptionFigIntTrois}{Domaine d'intégration pour l'exercice \ref{exoIntegralesMultiples0002}.\ref{ItemexoIntegralesMultiples0002iii}.}
\input{Fig_IntTrois.pstricks}

		\item
			Nous devons intégrer sur la partie du carré de côté $1$ qui ne se trouve pas dans le cercle. Nous pourrions, pour chaque $x$ entre $0$ et $1$, pour $y$ entre $\sqrt{1-x^2}$ et $1$, comme indiqué sur la figure \ref{LabelFigIntTroisssLabelMauvais}, mais nous pouvons aussi intégrer en polaires comme indiqué sur la figure \ref{LabelFigIntTroisssLabelBon}.

			Dans le cas des polaires, on intègre $\theta$ de $0$ à $\pi/2$, et puis pour $r$, le début de l'intégration est en $r=1$ tandis que la fin de l'intégration est donnée par l'intersection entre la droite d'angle $\theta$ et soit le côté vertical soit le côté horizontal du carré selon que $\theta$ soit plus petit ou plus grand que $\pi/4$.

			Pour $\theta<\pi/4$, si $\rho$ est la valeur de $r$ à laquelle la droite d'angle $\theta$ intersecte le côté vertical, et $y=\rho\sin\theta$ la hauteur à laquelle ça se passe, nous avons $\rho^2=1+y^2$, c'est à dire $\rho=1/\cos(\theta)$.

			Pour les angles entre $\pi/4$ et $\pi/2$, ce que nous obtenons est $\rho=1/\sin\theta$.

			L'intégrale à calculer en polaire est donc
			\begin{equation}
				I_3=\int_{0}^{\pi/4}\int_1^{1/\cos\theta} f\big( r\cos\theta,r\sin\theta \big)r\,drd\theta+\int_{\pi/4}^{\pi/2}\int_1^{1/\sin\theta} f\big( r\cos\theta,r\sin\theta \big)r\,drd\theta.
			\end{equation}
			Bonne chance.

			En cartésiennes, les choses se révèlent plus facile. En effet nous devons faire
			\begin{equation}
				I_3=\int_0^1\int_{\sqrt{1-x^2}}^1\frac{ xy }{ 1+x^2+y^2 }\,dydx.
			\end{equation}
			Il faut bien comprendre que nous commençons par intégrer par rapport à $y$ en traitant $x$ comment une «vulgaire» constante que nous pouvons même sortir de l'intégrale :
			\begin{equation}
				\begin{aligned}[]
					I_3&=\int_0^1 x\left[ \int_{\sqrt{1-x^2}}^1\frac{ y }{ (1+x^2)+y^2 }dy \right]dx\\
					&=\int_0^1 x \left[ \frac{ 1 }{2}\ln(x^2+y^2+1) \right]_{\sqrt{1-x^2}}^1dx\\
					&=\frac{ 1 }{2}\int_0^1\Big( x\ln(x^2+2)-x \ln(2) \Big) dx.
				\end{aligned}
			\end{equation}
			Nous avons utilisé la primitive classique $\int y/(a+y^2)=\ln(y^2+a)/2$ lorsque $a$ est positif, ce qui est le cas de $1+x^2$. À partir de là, nous sommes sur une intégrale usuelle, très similaire à \eqref{EqTrucIntxlnxsqpun}. Le résultat est :
			\begin{equation}
				I_3=\frac{ 3 }{ 4 }\ln\left( \frac{ 3 }{ 2 } \right)-\frac{1}{ 4 }.
			\end{equation}

			Une des morales de cet exercice est qu'il y a des situations dans lesquelles la façon de paramétrer le domaine change radicalement la difficulté du calcul effectif des intégrales. Ici c'est nettement plus simple en cartésiennes.

		\item
			Attention à l'ordre des intégrales. Ceci est faux :
			\begin{equation}
				I_4=\int_0^{\pi/2}\int_0^{1/2}\frac{1}{ x+xy+y+1 }dxdy.
			\end{equation}
			En effet dans cette écriture c'est $x$ qu'on intègre de $0$ à $\frac{ 1 }{2}$, or dans le domaine la variable $x$ est dans $\mathopen[ 0 , \frac{ \pi }{ 2 } \mathclose]$. Ce qui est correct est
			\begin{equation}
				I_4=\int_0^{1/2}\int_0^{\pi/2}\frac{1}{ x+xy+y+1 }dxdy.
			\end{equation}
			La première chose à faire est de factoriser le dénominateur : $x+xy+y+1=(x+1)(y+1)$. Ensuite, nous pouvons diviser l'intégrale en deux intégrales indépendantes en sortant $\frac{1}{ y+1 }$ de l'intégrale sur $x$ :
			\begin{equation}
				I_4=\int_0^{1/2}\frac{1}{ y+1 }dy\int_0^{\pi/2}\frac{1}{ x+1 }dx=\ln\left( \frac{ \pi }{ 2 }+1 \right)\ln\left( \frac{ 3 }{ 2 } \right).
			\end{equation}
			
		\item
			Une façon de paramétrer le domaine est le suivant :
			\begin{equation}
				\begin{aligned}[]
					z\colon 0\to 1\\
					y\colon 0\to \sqrt{1-z}\\
					x\colon 0\to \sqrt{1-z}.
				\end{aligned}
			\end{equation}
			Nous laissons donc aller $z$ de $0$ à $1$, et pour chacune des valeurs de $z$, nous contraignons $x$ et $y$ en fonction de $z$ par les équations $y^2+z\leq 1$ et $x^2+z\leq 1$. L'intégrale est donc, en se rappelant que $\int z\,dx=xz$,
			\begin{equation}
				\begin{aligned}[]
					I_4&=\int_0^1\int_0^{\sqrt{1-z}}\int_{0}^{\sqrt{1-z}}z\,dxdydz\\
					&=\int_0^1\int_0^{\sqrt{1-z}}[xz]_{x=0}^{x=\sqrt{1-z}}dydz\\
					&=\int_0^1\int_0^{\sqrt{1-z}}z\sqrt{1-z}\,dydz\\
					&=\int_0^1[yz\sqrt{1-z}]_0^{\sqrt{1-z}}\,dz\\
					&=\int_0^1 z(1-z)\,dz\\
					&=\frac{1}{ 6 }.
				\end{aligned}
			\end{equation}
			
		\item
			Nous avons envie d'utiliser une version un peu modifiée des coordonnées polaires : $x=ar\cos(\theta)$, $y=br\sin(\theta)$, de telle façon à épouser les formes du domaine. Pour intégrer, il faut comprendre et utiliser le théorème \ref{THOooUMIWooZUtUSg}. Ici,
			\begin{equation}
				\phi(r,\theta)=\big( ar\cos(\theta),br\sin(\theta) \big),
			\end{equation}
			et l'intégrale devient
			\begin{equation}
				\int_D f(x,y)dxdy=\int_{\phi^{-1}(D)}f\big( \phi(r,\theta) \big)| J_{\phi}(r,\theta) |drd\theta.
			\end{equation}
			Le jacobien est donné par
			\begin{equation}
				| J_{\phi}(r,\theta) |=\det\begin{pmatrix}
					\frac{ \partial \phi_1 }{ \partial r }	&	\frac{ \partial \phi_1 }{ \partial \theta }	\\ 
					\frac{ \partial \phi_2 }{ \partial r }	&	\frac{ \partial \phi_2 }{ \partial \theta }	
				\end{pmatrix}=
				\begin{pmatrix}
					a\cos(\theta)	&	-ar\sin(\theta)	\\ 
					b\sin(\theta)	&	br\cos(\theta)	
				\end{pmatrix}=
				abr.
			\end{equation}
			Notez que dans le cas particulier $a=b=1$, nous retrouvons le jacobien usuel $r$ des coordonnées polaires. Il faut trouver $\phi^{-1}(D)$. Pour parcourir le quart d'ellipse proposé, il faut $r\colon 0\to 1$ et $\theta\colon 0\to \pi/2$. Nous devons calculer
			\begin{equation}
				\begin{aligned}[]
					I_5&=\int_{[0,1]\times [0,\frac{ \pi }{2}]}(ar\cos\theta)(br\sin\theta) abr\,drd\theta\\
					&=a^2b^2\int_{0}^{\pi/2}\int_0^1r^3\cos\theta\sin\theta\,drd\theta\\
					&=\frac{ a^2b^2 }{ 8 }
				\end{aligned}
			\end{equation}
			où nous avons utilisé l'intégrale \eqref{EqTrucIntsxcxdx}.
			
	\end{enumerate}

\end{corrige}
