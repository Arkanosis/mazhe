% This is part of Exercices de mathématique pour SVT
% Copyright (c) 2010-2011,2015
%   Laurent Claessens et Carlotta Donadello
% See the file fdl-1.3.txt for copying conditions.

\begin{corrige}{TD4-0002}

	\begin{enumerate}
		\item
            La dérivée est cosinus. Or $\cos(x)$ est \emph{strictement} positive $x\in\mathopen] -\frac{ \pi }{2} , \frac{ \pi }{2} \mathclose[$. La fonction sinus est donc strictement croissante sur cet intervalle et par conséquent elle y est bijective. Nous avons donc prouvé que 
            \begin{equation}
                \sin\colon \mathopen] -\frac{ \pi }{2} , \frac{ \pi }{2} \mathclose[\to \mathopen] -1 , 1 \mathclose[ 
            \end{equation}
            est une bijection. Quid des derniers deux points ? \( \cos(-\pi/2)=-1\) et \( \cos(\pi/2)=1\), et ce sont les seuls points sur lesquels les valeurs \( 1\) et \( -1\) sont prises. Nous pouvons donc les ajouter au domaine de bijectivité.
		\item
			Pour calculer la dérivée d'une fonction inverse, le truc est dériver la définition en utilisant la règle de dérivation des fonctions composées. C'est à dire, pour calculer la dérivée de $f^{-1}$, nous écrivons
			\begin{equation}
				f\big( f^{-1}(y) \big)=y,
			\end{equation}
			et nous dérivons par rapport à $y$, ce qui donne
			\begin{equation}
				f'\big( f^{-1}(y) \big)(f^{-1})'(y)=1.
			\end{equation}
			De là, si nous savons la dérivée de $f$, nous pouvons trouver la dérivée de $f^{-1}$ en isolant $f^{-1}(y)$. 

			En ce qui concerne $\arcsin$, nous écrivons
			\begin{equation}
				\sin\big( \arcsin(x) \big)=x,
			\end{equation}
			et nous dérivons par rapport à $x$ :
			\begin{equation}			\label{eqcosasnpu}
				\cos\big( \arcsin(x) \big)\arcsin'(x)=1
			\end{equation}
            Oui, mais pour tout $\heartsuit$ entre \( -\pi/2\) et \( \pi/2\), nous avons $\cos(\heartsuit)=\sqrt{1-\sin^2(\heartsuit)}$, donc\footnote{Si \( \heartsuit\) n'est pas entre \( -\pi/2\) et \( \pi/2\), le cosinus peut être négatif, et la formule serait fausse.}
			\begin{equation}
				\cos\big( \underbrace{\arcsin(x)}_{\heartsuit} \big)=\sqrt{1-\sin^2\big( \underbrace{\arcsin(x)}_{\heartsuit} \big)}.
			\end{equation}
			En sachant que $\sin^2(\arcsin(x))=x^2$, nous avons donc
			\begin{equation}
				\cos\big( \arcsin(x) \big)=\sqrt{1-x^2}.
			\end{equation}
			En reportant cela dans l'équation \eqref{eqcosasnpu}, nous trouvons
			\begin{equation}
				\sqrt{1-x^2}\arcsin'(x)=1,
			\end{equation}
			et par conséquent
			\begin{equation}
				\arcsin'(x)=\frac{1}{ \sqrt{1-x^2} }.
			\end{equation}

		\item
			En ce qui concerne la tangente, nous devons savoir la formule
			\begin{equation}
				\tan'(x)=\frac{1}{ \cos^2(x) },
			\end{equation}
			et alors nous avons successivement, en dérivant et en isolant :
	\label{Eqsubatantancos}
    \begin{subequations}
        \begin{align}
					\tan\big(\arctan(x)\big)=x\\
					\tan'\big( \arctan(x) \big)\arctan'(x)=1\\
                    \arctan'(x)=\cos^2\big( \arctan(x) \big)    \label{EqsubatantancosddQZLe}.
        \end{align}
    \end{subequations}
			Il faut maintenant trouver ce que vaut $\cos\big( \arctan(x) \big)$.
            
            \begin{description}
                \item[Méthode conseillée] 
                        Nous utilisons la formule \( \cos^2(\heartsuit)+\sin^2(\heartsuit)=1\) avec \( \heartsuit=\arctan(x)\):
                        \begin{equation}
                            \cos^2\big( \arctan(x) \big)+\sin^2\big( \arctan(x) \big)=1.
                        \end{equation}
                        Nous divisons par \( \cos^2\big( \arctan(x) \big)\) pour faire apparaître la tangente:
                        \begin{equation}
                            \begin{aligned}[]
                                1+\frac{ \sin^2\big( \arctan(x) \big) }{  \cos^2\big( \arctan(x) \big)  }&=\frac{1}{  \cos^2\big( \arctan(x) \big)  }\\
                                1+\tan\big( \arctan(x) \big)^2&=\frac{1}{ \cos^2(\arctan(x)) }\\
                                1+x^2&=\frac{1}{ \cos^2\big( \arctan(x) \big) }.
                            \end{aligned}
                        \end{equation}
                        
                \item[Méthode alternative]

            Afin d'exprimer \( \cos\big( \arctan(x) \big)\),  il serait bon d'exprimer $\cos(x)$ en termes de $\tan(x)$, de façon que $\cos\big( \arctan(x) \big)$ s'exprime en termes de $\tan\big( \arctan(x) \big)=x$. Pour cela, nous partons de la définition de la tangente :
			\begin{equation}
				\tan(x)=\frac{ \sin(x) }{ \cos(x) }=\frac{ \sqrt{1-\cos^2(x)} }{ \cos(x) }.
			\end{equation}
			Par conséquent,
			\begin{equation}
				\tan^2(x)=\frac{ 1-\cos^2(x) }{ \sin(x) },
			\end{equation}
			et donc
			\begin{equation}
				\cos^2(x)\big( \tan^2(x)+1 \big)=1,
			\end{equation}
			et au final,
			\begin{equation}
				\cos^2(x)=\frac{1}{ \tan^2(x)+1 }.
			\end{equation}
			Nous retournons maintenant à notre dernière expression \eqref{Eqsubatantancos} :
			\begin{equation}
				\begin{aligned}[]
					\cos^2\big( \arctan(x) \big)=\frac{1}{ \tan^2\big( \arctan(x) \big)+1 }=\frac{1}{ x^2+1 }.
				\end{aligned}
			\end{equation}

            \end{description}
            
            Quelle que soit la méthode choisie, nous avons
            \begin{equation}
                \cos^2\big( \arctan(x) \big)=\frac{1}{ 1+x^2 },
            \end{equation}
            et nous concluons en reprenant la formule \eqref{EqsubatantancosddQZLe} :
            \begin{equation}
                \arctan'(x)=\frac{1}{ 1+x^2 }.
            \end{equation}
			
	\end{enumerate}
	

\end{corrige}
