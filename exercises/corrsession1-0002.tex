% This is part of Analyse Starter CTU
% Copyright (c) 2014
%   Laurent Claessens,Carlotta Donadello
% See the file fdl-1.3.txt for copying conditions.

\begin{corrige}{session1-0002}

  \begin{enumerate}
  \item $\displaystyle \int x^{-4} + x^{1/5}\, dx = -\frac{x^{-3}}{3} + \frac{5}{6}x^{6/5} +C $ ;
  \item $\displaystyle \int \ln(x)\, dx = x\ln(x) - \int x\times \frac{1}{x}\, dx = x\ln(x) -x +C$ ;
  \item
    \begin{equation*}
      \begin{aligned}
        \int & x\sin^2(x)\, dx = \int x\frac{1 - \cos(2x)}{2} \,dx \\
        &= \frac{x^2}{4} - \int \frac{x}{2}\cos\left(2x\right) \, dx \\
        &= \frac{x^2}{4} - \frac{x}{4}\sin\left(2x\right) + \int \frac{\sin\left(2x\right)}{4}\, dx \\
        &= \frac{x^2}{4} - \frac{x}{4}\sin\left(2x\right) - \frac{\cos\left(2x\right)}{8}+ C \:;
      \end{aligned}
    \end{equation*}
    \item $\displaystyle \int_{-\pi/4}^{\pi/4}\tan(x)\, dx = 0$ car $\tan$ est une fonction impaire et $[-\pi/4, \pi/4]$ est sym\'etrique par rapport \`a l'origine. Si on veut v\'erifier ce r\'esultat il suffit d'\'ecrire $\tan(x)$ comme $\frac{\sin(x)}{\cos(x)}$ et utiliser la formule pour les primitives des fonctions du type $U'/U$. On aura alors que les primitives de $\tan$ sont $-\ln(|\cos(x)|) + C$.  
  \item  On veut utiliser le changement de variable $t = e^x$. On a donc $dt = e^x dx$, ce qu'on peut \'ecrire aussi comme $\frac{1}{t}dt = dx$.    
    \begin{equation*}
      \begin{aligned}
        \int_{0}^{1}&\frac{1}{4e^x+e^{-x} }\, dx = \int_{e^0=1}^{e^1 =e} \frac{1}{4t + \frac{1}{t}} \frac{1}{t} \,dt\\
        & = \int_1^e \frac{1}{4t^2 +1} \, dt =\frac{1}{2}\left[\arctan(2t)\right]_{t=1}^{t=e} \\
        &= \frac{1}{2}\left(\arctan(2e)-\arctan(2)\right).
      \end{aligned}
    \end{equation*}
  \item  On veut utiliser le changement de variable $t = \arcsin(x)$, donc $dt = \frac{1}{\sqrt{1-x^2}} dx $. On peut \'ecrire l'int\'egrale comme il suit 
    \begin{equation*}
      \begin{aligned}
        \int_{0}^{1/2} & \sqrt{\frac{\arcsin(x)}{1-x^2}}\, dx = \int_{0}^{1/2}\sqrt{\arcsin(x)} \frac{1}{\sqrt{1-x^2}}\, dx\\
        &= \int_{\arcsin(0)=0}^{\arcsin(1/2) = \pi/6} \sqrt{t} \,dt = \left[\frac{2}{3}t^{3/2}\right]_0^{\pi/6} = \frac{2}{3}\left(\frac{\pi}{6}\right)^{3/2}. 
      \end{aligned}
    \end{equation*}
  \end{enumerate}

\end{corrige}
