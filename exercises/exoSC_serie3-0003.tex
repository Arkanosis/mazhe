\begin{exercice}\label{exoSC_serie3-0003}

	Le tableau ci-dessous donne les valeurs de l'énergie $E$ consommée par différents animaux dans la course, en liaison avec la masse $m$ de ces animaux.
	\[
		\begin{array}{|c|ccccccc|}
			\hline
			\text{Animal}& \text{souris}	&	\text{écureil}	&	\text{rat}	&	\text{chien (petit)}	&	\text{chien (gros)}	& \text{mouton}	&\text{cheval} \\
			\text{Masse $m$ (\gram)}& 21	& 236	&384&	2.6\times 10^3&	1.8\times 10^4	&	3.9\times 10^4	& 5.8\times 10^5 \\
			\hline
			\text{Énergie (cal\per\gram\per\kilo\meter)}& 13&	3.7&4.4&	1.7&	0.92&0.58&0.15\\
			\hline
		\end{array}
	\]
	Dans le plan $(\ln(E),\ln(m))$, trouver une droite qui passe approximativement par les points donnés; quelle est la pente de cette droite ? Représenter, dans un même diagramme, cette droite et les points correspondants aux données.

\corrref{SC_serie3-0003}
\end{exercice}
