% This is part of Analyse Starter CTU
% Copyright (c) 2014
%   Laurent Claessens,Carlotta Donadello
% See the file fdl-1.3.txt for copying conditions.

\begin{corrige}{starterST-0010-0011}
  \begin{enumerate}
  \item La fonction $\arcsin$ est une bijection définie sur l'intervalle $[-1,1]$ qui prend ses valeurs dans l'intervalle $\displaystyle \left[\frac{-\pi}{2}, \frac{\pi}{2}\right]$. Il existe donc une seule solution de la première équation, qui est $x = \sqrt{2}/2$. La deuxième équation, par contre, n'a pas de solution.   
  \item On peut toujours comparer $\arcsin(\sin (x))$ et $x$, mais la relation $\arcsin(\sin (x)) = x$ sera satisfaite uniquement si $x$ appartient à l'intervalle $\displaystyle \left[\frac{-\pi}{2}, \frac{\pi}{2}\right]$. Dans les autres cas, on saura seulement que soit $x = \arcsin(\sin (x)) + 2K\pi$ ou $x = - \arcsin(\sin (x)) + (2K+1)\pi$, avec $K\in\eZ$. 
\item La fonction $\arccos$ est une bijection définie sur l'intervalle $[-1,1]$ qui prend ses valeurs dans l'intervalle $\displaystyle \left[0, \pi\right]$. Il existe donc une seule solution de la première équation, qui est $x = \sqrt{2}/2$ et une seule solution de la deuxième,  qui est $x = -\sqrt{2}/2$.
  \item  On peut toujours comparer $\arccos(\cos (x))$ et $x$, mais la relation $\arccos(\cos (x)) = x$ sera satisfaite uniquement si $x$ appartient à l'intervalle $\displaystyle \left[0, \pi\right]$. Dans les autres cas, on saura seulement que soit $x = \arccos(\cos (x)) + 2K\pi$ ou $x = - \arccos(\cos (x)) + 2K\pi$, pour un $K\in\eZ$ à déterminer. 
  \end{enumerate}
\end{corrige}
