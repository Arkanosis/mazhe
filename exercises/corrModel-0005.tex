% This is part of Agregation : modélisation
% Copyright (c) 2011
%   Laurent Claessens
% See the file fdl-1.3.txt for copying conditions.

\begin{corrige}{Model-0005}

    Pour chaque observation \( x_i\) nous avons une densité gaussienne. Le produit donne
    \begin{equation}
        p\big( x_1,\ldots,x_n;(m,\sigma^2) \big)=\frac{1}{ \sigma^2(2\pi)^{n/2} }\exp\left[ -\frac{1}{ 2\sigma^2 }\sum_i(x_i-m)^2 \right].
    \end{equation}
    En passant au logarithme et en supprimant des facteurs inutiles à la minimisation,
    \begin{equation}
        L(m,\sigma)=-n\ln(\sigma)-\frac{1}{ 2\sigma^2 }\sum_i(x_i-m)^2.
    \end{equation}
    L'annulation de la dérivée par rapport à \( m\) donne immédiatement \( m=\frac{1}{ n }\sum_i x_i\). L'annulation de la dérivée par rapport à \( \sigma\) donne
    \begin{equation}
        -n\sigma^2+\sum_i(x_i-m)^2=0
    \end{equation}
    et donc
    \begin{equation}
        \sigma^2=\frac{1}{ n }\sum_i(x_i-\bar x_n).
    \end{equation}
    L'estimateur de maximum de vraisemblance du couple \( \theta=(m,\sigma^2)\) est donc
    \begin{equation}
        \hat\theta_n=\left( \frac{1}{ n }\sum_iX_i,\frac{1}{ n }\sum_i(X_i-\bar X_n)^2 \right).
    \end{equation}

\end{corrige}
