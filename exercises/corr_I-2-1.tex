% This is part of the Exercices et corrigés de CdI-2.
% Copyright (C) 2008, 2009
%   Laurent Claessens
% See the file fdl-1.3.txt for copying conditions.


\begin{corrige}{_I-2-1}

Nous donnons systématiquement deux preuves. La première utilise les fonctions test, tandis que la seconde utilise une majoration (ou minoration) explicite qui donne le résultat. Notez qu'à chaque fois, les deux méthodes reviennent au même.

\begin{enumerate}

\item Intuitivement, le résultat est clair parce que, pour des grands $x$, la fonction $f(x)=(1+2x^2)^{-1/2}$ se comporte comme $1/x$ (compter les degrés de $x$), alors que $1/x$ ne s'intègre pas vers l'infini. La réponse est donc que l'intégrale ne converge pas, et il faudra la comparer à une fonction de type $1/x$. 

En effet, nous utilisons le corollaire \ref{CorCritFonsTest} avec $\alpha=1$ :
\begin{equation}
	L=\lim_{x\to\infty}\frac{ x }{ \sqrt{1+2x^2} }=\frac{1}{ \sqrt{2} }\neq 0.
\end{equation}
L'intégrale n'existe donc pas.

{\bf Justification alternative}
Nous cherchons directement une fonction de type $a/x$ qui majore $f(x)$ pour les grands $x$.
Posons $g(x)=a/x$. Nous avons
\begin{equation}
	f(x)-g(x)=\frac{ x-a\sqrt{1+2x^2} }{ x\sqrt{1+2x^2} },
\end{equation}
dont le dénominateur est toujours positif ou nul dans l'ensemble considéré. Dès que $a<1/\sqrt{2}$, il existe un $x_0$ tel que $\forall x>x_0$, nous avons $f(x)>g(x)$. Pour prouver cela, remarquez que
\begin{equation}
	\lim_{x\to\infty}\big( x-a\sqrt{1+2x^2} \big)=\lim_{x\to\infty}x(1-\sqrt{2}a)=\pm\infty
\end{equation}
où le $\pm$ est fixé par la valeur de $a$. Nous utilisons le théorème \ref{ThoFnTestIntnnBorn} pour conclure à la non existence.

\item 

Nous avons $\lim_{x\to\infty}x^2P(x) e^{-x^2}=0$, d'où nous concluons à l'existence de l'intégrale.

{\bf Justification alternative} Sans perte de généralité, nous pouvons supposer que $P(x)$ est positif. En effet, pour $x$ assez grand, un polynôme ne change plus de signe, et nous pouvons éventuellement considérer $-P(x)$ au lieu de $P(x)$. Soit $Q$, un polynôme de degré deux plus haut que $P$, que nous prenons positif. Une propriété de l'exponentielle est que $\exists x_0>0$ tel que $x>x_0$ implique $ e^{-x^2}>Q(x)$ et $Q(x)>P(x)$, et donc
\begin{equation}
	\frac{ P(x) }{ Q(x) }>P(x) e^{-x^2}
\end{equation}
pour $x>x_0$. Mais, la fraction rationnelle $P/Q$ est intégrable sur $[x_0,\infty[$ parce qu'elle peut être majoré par une fonction de la forme $a/x^2$, comme dans l'exercice \ref{ExoPointPremierIdeuxUn}. Le théorème \ref{ThoFnTestIntnnBorn} conclut à l'intégrabilité de $f$ sur $[0,\infty[$.

\item

Pour tout $\alpha$ et $\beta$, nous avons
\begin{equation}
	\lim_{x\to\infty}x^{\alpha} e^{-x}\ln(x)^{\beta}=0,
\end{equation}
d'où l'existence de l'intégrale s'ensuit.

{\bf Justification alternative}
 La fonction $f(x)=x^{\alpha} e^{-x}\ln^{\beta}(x)$ est positive sur l'ensemble considéré. Étant donné que $\ln(x)<x$ (pour $x$ assez grand), lorsque $\beta>0$, nous pouvons majorer $f(x)$ par $ e^{-x}x^{\alpha+\beta}$. Une fois de plus, l'exponentielle fait son travail : la fonction $g(x)= e^{-x}x^{\alpha+\beta}$ est intégrable quel que soient $\alpha$ et $\beta$.

Si $\beta<0$, alors $\ln(x)^{\beta}$ est même majorée par $1$, et c'est la conclusion est encore plus simple.

\item\label{ItemDCorrI21}

Le changement de variable $t=1/\ln(x)$ mène à étudier l'existence de
\begin{equation}
	\int_{1/\ln(2)}^{\infty}t^{-1} e^{-1/2t}dt.
\end{equation}
Étant donné que $\lim_{t\to \infty}t^1t^{-1} e^{-1/2t}=1$, cette intégrale n'existe pas.

{\bf Justification alternative}
 La fonction $f(x)=\sqrt{x}/\ln(x)$ est bornée sur tout intervalle de la forme $[1+\epsilon,2]$; nous ne sommes donc intéressés que par l'intégrale sur $[1,1+\epsilon]$. Sur cet intervalle, nous avons $\ln(x)<x-1$, et donc 
\begin{equation}
	\frac{ \sqrt{x} }{ \ln(x) }>\frac{ \sqrt{x} }{ x-1 }>\frac{ 1 }{ 2 }\frac{ 1 }{ x-1 }
\end{equation}
où $\frac{ 1 }{2}$ est une minoration de $\sqrt{x}$ entre $1$ et $1+\epsilon$. L'intégrale de $1/(x-1)$ ne converge pas en $x=1$.

Notez qu'un changement de variable $t=\ln(x)$ fait tout aussi bien le travail : nous tombons sur
\begin{equation}
	\int_{0}^{\ln(2)}\frac{1}{ t } e^{3t/2}dt
\end{equation}
dans laquelle $ e^{3t/2}$ peut être minorée par $1$, alors que l'intégrale de $1/x$ en $x=0$ n'existe pas.

\item 

Le changement de variable $u=1/(e^x-1)$ donne
\begin{equation}
	\int_{e-1}^{\infty}\frac{1}{ 1+u }du,
\end{equation}
mais $\lim_{u\to\infty}u/(1+u)=1$, donc cette intégrale n'existe pas.

{\bf Justification alternative}
Lorsque $x$ est proche de zéro et si $a$ est suffisamment négatif, nous avons $ e^{x}-1<ax$. Donc, nous avons la majoration
\begin{equation}
	\frac{1}{  e^{x}-1 }>-\frac{1}{ ax },
\end{equation}
alors que la fonction $1/x$ n'est pas intégrable.

\item 

Prenons $\alpha$ quelconque et calculons
\begin{equation}
	L_p(\alpha)=\lim_{x\to\infty}x^{\alpha}\frac{ \ln(x) }{ (1+x^3)^{1/p} }=\lim_{x\to\infty}x^{\alpha-\frac{ 3 }{ p }}\ln(x)=\begin{cases}
	\infty	&	\text{si $\alpha-3/p\geq 0$}\\
	0	&	 \text{si $\alpha-3/p<0$.}
\end{cases}
\end{equation}
L'intégrale existera si $\alpha>1$ et si $L\neq\infty$, c'est à dire si $\alpha<3/p$.
L'intégrale existe donc si on a un $\alpha$ tel que $1<\alpha<\frac{ 3 }{ p }$. Au
contraire, l'intégrale n'existera pas s'il existe un $\alpha$ tel que $\frac{ 3 }{ p }\leq 1\leq\alpha\leq 1$. En d'autres termes, l'intégrale existe si $\frac{ 3 }{ p }>1$, et n'existe pas si $\frac{ 3 }{ p }\leq 1$.

En définitive, l'intégrale existera si et seulement si $p\in]0,3[$.

{\bf Justification alternative}
Prouvons que l'intégrale
\begin{equation}
	\int_2^{\infty}\frac{ \ln(x) }{ x^{1+\epsilon} }
\end{equation}
converge pour tout $\epsilon>0$. Pour tout $\alpha>0$, nous avons
\begin{equation}
	\lim_{x\to\infty}\frac{ \ln(x) }{ x^{\alpha} }=\lim_{x\to\infty}\frac{ 1/x }{ \alpha x^{\alpha-1} }=\lim_{x\to\infty}\frac{1}{ \alpha x^{\alpha} }=0,
\end{equation}
donc la fonction $x\mapsto \ln(x)/x^{\alpha}$ est bornée pour tout $\alpha >0$. Maintenant,
\begin{equation}
	\frac{ \ln(x) }{ x^{1+\epsilon} }=\frac{ \ln(x) }{ x^{\epsilon/2} }\cdot\frac{1}{ x^{1+\epsilon/2} }
\end{equation}
où le premier terme peut être majoré par une certaine constante $M$. Par conséquent,
\begin{equation}
	\frac{ \ln(x) }{ x^{1+\epsilon} }<M\frac{ 1 }{ x^{1+\epsilon/2} },
\end{equation}
dont le membre de droite a une intégrale qui converge sur $[2,\infty[$ pour tout $\epsilon>0$. Nous avons donc existence de l'intégrale lorsque $3/p >1$.

\item

Prenons $\alpha=\frac{ 3 }{ 2 }>1$. Nous avons
\begin{equation}
	\lim_{x\to\infty}\frac{ x^{\frac{ 3 }{ 2 }} }{ x^2-1 }=0,
\end{equation}
donc l'intégrale existe.

{\bf Justification alternative}
Il est évident que pour tout $M$, l'intégrale $\int_2^Mf(x)dx$ ne pose pas de problèmes. Nous sommes donc en train de chercher une fonction $g$ qui majore $f$ pour les grands $x$. Afin d'avoir la bonne puissance au dénominateur, il faut chercher $g(x)=a/x^2$. On vérifie que $a>1$ fait l'affaire. En effet, la quantité
\begin{equation}
	\frac{ a }{ x^2 }-\frac{ 1 }{ x^2-1 }=\frac{ x^2(a-1)-a }{ x^2(x-1)(x+1) }.
\end{equation}
est positive pour tout $x$ assez grand dès que $a>1$.

\end{enumerate}
\end{corrige}
