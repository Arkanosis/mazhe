% This is part of Exercices et corrigés de CdI-1
% Copyright (c) 2011
%   Laurent Claessens
% See the file fdl-1.3.txt for copying conditions.

\begin{corrige}{0003}

\begin{enumerate}
\item\label{itemCorr3a} Étant donné que $A\subset B$, nous avons que, si $s_B=\sup(B)$, alors $\forall x\in B$, $x\leq s_B$, et en particulier,
\begin{equation}		\label{EqForallsAsBgeq}
	\forall x\in A,\, x\leq s_B.
\end{equation}
Si maintenant, $s_A=\sup(A)$ et $s_A>s_B$, alors il existe un $\epsilon>0$ tel que $s_B=s_A-\epsilon$, et donc $\exists y\in A$ tel que $y>s_B$, ce qui contredirait \eqref{EqForallsAsBgeq}.

De la même manière, si $A\subset B$, alors $\inf(A)\geq \inf(B)$.

\item Non. Par exemple $A=\{ 2 \}$ et $B=\{ 1,2 \}$.

\item Nous avons
\begin{equation}
	\sup(A\cup B)=\max\{ \sup(A),\sup(B) \}.
\end{equation}
En effet, appelons $m$ ce maximum. Pour tout $x\in A\cup B$, nous avons $x\leq m$, et d'autre part, si $\epsilon>0$, alors $m-\epsilon$ est soit plus petit que $s_A$, soit plus petit que $s_B$ (soit les deux). Si $m<s_A$, alors il existe un $x\in A\subset A\cup B$ tel que $x>m-\epsilon$, et, de la la même manière, si $m<s_B$, alors il existe un $x\in B\subset A\cup B$ tel que $x>m-\epsilon$. 

Nous n'avons par contre pas de rapport direct entre $\sup(A)$, $\sup(B)$ et $\sup(A\cap B)$, comme le montre l'exemple $A=\{ 1,5 \}$ et $B=\{ 1,7 \}$. Par contre, nous avons que
\begin{equation}
	\sup(A\cap B)\leq\min\{ \sup(A),\sup(B) \}.
\end{equation}
En effet, en utilisant les résultats du point \ref{itemCorr3a}, et en tenant compte du fait que que $A\cap B\subset A$, nous avons $\sup(A)\geq\sup(A\cap B)$ et $\sup B\geq\sup(A\cap B)$.
\end{enumerate}

\end{corrige}
