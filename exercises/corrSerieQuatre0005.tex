% This is part of Exercices et corrections de MAT1151
% Copyright (C) 2010
%   Laurent Claessens
% See the file LICENCE.txt for copying conditions.

\begin{corrige}{SerieQuatre0005}

	Certes, le nombre $d$ rentre très bien dans la machine, mais nous allons voir que le nombre $1-d$, lui, rendre nettement plus mal. En effet, une des choses importantes vues au cours théorique est que
	\begin{equation}
		F(\hat d)\sim F(d)(1+K_{\text{rel}}(d)\rho_d)
	\end{equation}
	où $K_{\text{rel}}$ est le conditionnement relatif du problème $F$ tandis que $\rho_d$ est l'écart relatif entre $d$ et $\hat d$ : $\rho_d=(\hat d-d)/d$. Dans notre cas, nous avons
	\begin{equation}
		\rho_d=\frac{ 1.001-1.00098 }{ 1.00098 }=0.00001998\ldots\sim\frac{1}{ 50000 }.
	\end{equation}
	Par contre, l'erreur relative sur $F(d)$ est donnée par le nombre
	\begin{equation}		\label{EqEcartRelSurF}
		\frac{ F(\hat d)-F(d) }{ F-d }=K_{\text{rel}}(d)\rho_d.
	\end{equation}
	Nous devons évaluer $K_{\text{rel}}(d)$. D'abord nous calculons le conditionnement absolu
	\begin{equation}
		K_{\text{abs}}(d)\simeq| F'(d) |=| 1-d |'=1.
	\end{equation}
	Ensuite le relatif est donné par
	\begin{equation}
		K_{\text{rel}}(d)=\frac{ | d | }{ | 1-d | }=\frac{ 1.00098 }{ 1-1.00098 }\sim 1021.
	\end{equation}
	L'erreur relative sur $F(d)$ est donc environ $1000$ fois plus grande que celle sur $d$.

\end{corrige}
