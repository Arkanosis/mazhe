% This is part of the Exercices et corrigés de mathématique générale.
% Copyright (C) 2009
%   Laurent Claessens
% See the file fdl-1.3.txt for copying conditions.
\begin{corrige}{EquaDiff0011}

\begin{enumerate}

	\item
		Le polynôme caractéristique est
		\begin{equation}
			\lambda^2+10\lambda+16=0,
		\end{equation}
		dont les racines sont $\lambda_1=-8$ et $\lambda_2=-2$. La solution générale s'écrit donc
		\begin{equation}
			y(x)=a e^{-8x}+B e^{-2t}.
		\end{equation}
		Il faut maintenant fixer les paramètres $A$ et $B$ pour que $y(0)=0$ et $y'(0)=12$. Pour ce faire, nous calculons $y(x)$ :
		\begin{equation}
			y'(x)=-8A e^{-8x}-2B e^{-2x}.
		\end{equation}
		Nous avons donc les contraintes suivantes sur $A$ et $B$ : $y(0)=A+B=1$ et $y'(0)=-8A-2B=12$. Cela est un petit système de deux équations à deux inconnues que nous résolvons facilement :
		\begin{equation}
			\begin{aligned}[]
				A&=-\frac{ 7 }{ 3 }\\
				B&=\frac{ 10 }{ 3 }\\
			\end{aligned}
		\end{equation}
		La solution répondant aux conditions posées est donc
		\begin{equation}
			\frac{ 56 }{ 3 } e^{-8x}-\frac{ 20 }{ 3 } e^{-2x}.
		\end{equation}


	\item
		Les solutions du polynôme caractéristique sont $\lambda_1=3+i$ et $\lambda_2=3-i$. La première chose à faire est de trouver les solutions réelles qui en découlent. Elles sont
		\begin{equation}
			y(x)=A e^{3x}\sin(x)+B e^{3x}\sin(x).
		\end{equation}
		Il faut maintenant fixer $A$ et $B$ pour que $y(0)=1$ et $y'(0)=4$. Le système d'équation à résoudre est
		\begin{subequations}
			\begin{numcases}{}
				B=1\\
				3B+A=4.
			\end{numcases}
		\end{subequations}
		La réponse est $A=1$ et $B=1$.

	\item
		L'équation caractéristique est $\lambda^2+9=0$, et les racines sont $\lambda=\pm 3i$. Les solutions  de l'équation différentielle sont donc les fonctions de la forme
		\begin{equation}
			y(x)=A\cos(3x)+B\sin(3x).
		\end{equation}
		Étant donné que $y(\frac{ \pi }{ 3 })=0$, nous trouvons $-A=0$ et la condition $y'(\frac{ \pi }{ 3 })=1$ donne $-3B=0$, donc en réalité $A=B=0$ et la solution à l'équation répondant aux conditions posées est la fonction identiquement nulle :
		\begin{equation}
			y(x)=0.
		\end{equation}

\end{enumerate}

\end{corrige}
