% This is part of the Exercices et corrigés de mathématique générale.
% Copyright (C) 2009,2012
%   Laurent Claessens
% See the file fdl-1.3.txt for copying conditions.
\begin{corrige}{EquaDiff0016}


%TODO : refaire le dessin
%Un dessin se trouve à la figure \ref{LabelFigProbTgEqDiff}.
%\newcommand{\CaptionFigProbTgEqDiff}{La tengente et la droite qui lie à l'origine pour l'exercice \ref{exoEquaDiff0016}.}
%\input{pictures_tex/Fig_ProbTgEqDiff.pstricks}
Considérons un point $\big( x,y(x) \big)$ sur la courbe. Nous nommons $\beta$ l'angle que fait la droite joignant ce point à l'origine, par définition,
\begin{equation}
	\tan(\beta)=\frac{ y(x) }{ x }.
\end{equation}
Si $\alpha$ est l'angle que fait la tangente avec l'axe $Ox$, alors 
\begin{equation}
	\tan(\alpha)=y'(x).
\end{equation}
La condition imposée est $\alpha+\beta=90$. Un peu de trigonométrie montre que
\begin{equation}
	\tan(\beta)=\tan(90-\alpha)=\frac{1}{ \tan(\alpha) },
\end{equation}
et donc
\begin{equation}
	y'(x)=\frac{ x }{ f(x) }
\end{equation}
est l'équation différentielle à résoudre. Nous la remettons sous la forme
\begin{equation}
	\frac{ dy }{ dx }=\frac{ x }{ y }, 
\end{equation}
ou encore : $ydy=xdx$, d'où la solution
\begin{equation}
	y^2=x^2+C.
\end{equation}


\end{corrige}
