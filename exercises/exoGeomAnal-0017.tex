\begin{exercice}\label{exoGeomAnal-0017}

  \begin{enumerate}
 
  \item Expliquer pourquoi il n'existe pas une fonction $f: \eR^2\to \eR $ telle que 
    \begin{equation}
      \partial_x f(x,y)= 3x+y, \qquad \partial_y f(x,y)= 4\sin(xy).
    \end{equation}
 % \item Soit $g: \eR^2\to \eR $ la fonction
  %  \begin{equation}
   %   g(x,y)=\left\{
    %  \begin{array}{ll}
     %   \frac{xy}{\sqrt{x^2+y^2}}, & \textrm{si } (x,y)\neq (0,0)\\
      %  0, & \textrm{si } (x,y)= (0,0). 
      %\end{array}
      %\right.
    %\end{equation}
     % Calculer les dérivées partielles de la fonction $g$ en tout point de $\eR^2$, y compris au point $(0,0)$, si elles existent.
    \item Soit  $h: \eR^3\to \eR $ une fonction dont les dérivées partielles existent partout dans $\eR^3$. Nous savons que $\partial_z h$ est continue sur  $\eR^3\setminus \{(0,1,0)\}$ et que $\partial_x h$ et $\partial_y h$ sont continue partout dans $\eR^3$. Qu'est-ce que pouvons nous dire de la différentiabilité de $h$ ? 
  \end{enumerate}

\corrref{GeomAnal-0017}
\end{exercice}
