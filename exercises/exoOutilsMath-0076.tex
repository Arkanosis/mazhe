% This is part of Exercices et corrigés de CdI-1
% Copyright (c) 2011
%   Laurent Claessens
% See the file fdl-1.3.txt for copying conditions.

\begin{exercice}\label{exoOutilsMath-0076}

    Soient deux milieux d'indice de réfraction $n_1$ et $n_2$ et séparés par une surface plane perpendiculaire à un vecteur unitaire $\overline{ N }$. L'après la loi de Snell, on a 
    \begin{equation}
        \frac{ \sin(\theta_1) }{ \sin(\theta_2) }=\frac{ n_2 }{ n_1 }
    \end{equation}
    où $\theta_1$ et $\theta_2$ sont les angles d'incidence (par rapport à $\overline{ N }$) d'un rayon lumineux et son angle de réfraction.

    Vérifier que
    \begin{equation}
        n_1(\overline{ N }\times a)=n_2(\overline{ N }\times b)
    \end{equation}
    où $a$ et $b$ désignent deux vecteurs unitaires le long des rayons d'incidence et de réfraction.


    Voir la situation sur la figure \ref{LabelFigRefraction}.
\newcommand{\CaptionFigRefraction}{Angles de réfraction et de réflexion.}
\input{pictures_tex/Fig_Refraction.pstricks}

\corrref{OutilsMath-0076}
\end{exercice}
