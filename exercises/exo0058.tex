% This is part of Exercices et corrigés de CdI-1
% Copyright (c) 2011,2014
%   Laurent Claessens
% See the file fdl-1.3.txt for copying conditions.

\begin{exercice}\label{exo0058}


Soit
\[
f : \eR^2 \rightarrow \eR: (x,y) \rightarrow 4x^2 + y^2
\]
\begin{enumerate}
\item
Calculez les dérivées partielles $(\frac{\partial}{\partial x}f)(a,b)$
et $(\frac{\partial}{\partial y}f)(a,b)$ de la fonction $f$ par rapport
à sa première et par rapport à sa seconde variable au point $(a,b)$
Donnez en une interprétation géométrique
\item
Calculez l'équation du plan tangent au graphe de $f$ au point
$(a,b,f(a,b))$.
\item
Calculez la différentielle $(df)(a,b)$ de $f$ au point $(a,b)$. Quel lien
y a-t-il entre ce plan tangent et le graphe de $df(a,b)$ ?
\item
Calculez la dérivée directionnelle de $f$ au point $(1,2)$ dans la
direction $(\frac{1}{2},\frac{\sqrt{3}}{2})$.
\item
Calculez le gradient $\nabla f$ de $f$ au point $(a,b)$. Donnez en une
interprétation géométrique.
\item Dessinez quelques ensembles de niveau de la fonction $f$. Calculez
$\nabla f(1,1)$ et écrire l'équation de la tangente \`a la courbe
\[
\{ (x,y) \mid 4x^2+y^2=5 \}
\]
au point $(1,1)$.
\end{enumerate}



\corrref{0058}
\end{exercice}
