% This is part of Exercices de mathématique pour SVT
% Copyright (C) 2010
%   Laurent Claessens et Carlotta Donadello
% See the file fdl-1.3.txt for copying conditions.

\begin{exercice}[\minsyndical]\label{exoTD2A-2}
Il faut conna\^{i}tre les valeurs des limites suivantes :
\begin{description}
\item[Sinus $x$ sur $x$ :] 
  $\displaystyle \lim_{x\to 0}\frac{\sin x}{x}=1$ ;
\item[Nombre de Néper :] 
  $\displaystyle \lim_{x\to \pm\infty}\left(1+\frac{1}{x}\right)^x=e$.
\end{description}

Un adage fort utile est «l'exponentielle va plus vite et le logarithme va moins vite». Cela se traduit par les limites suivantes :
\begin{multicols}{2}
	\begin{enumerate}
	\item
		$\lim_{x\to 0} x\ln(x)=0$
	\item
		$\lim_{x\to 0} \sqrt{x}\ln(x)=0$
	\item
		$\lim_{x\to \infty} \displaystyle\frac{ \ln(x) }{ x }=0$
	\item
		$\lim_{x\to \infty} \displaystyle\frac{ \ln(x) }{ \sqrt{x} }=0$
	\item
		$\lim_{x\to \infty} \displaystyle\frac{  e^{x} }{ x }=\infty$
	\item
		$\lim_{x\to -\infty} xe^x=0$
	\end{enumerate}
\end{multicols}

En utilisant ces limites (dites \emph{remarquables}, en raison de leur importance) nous pouvons calculer des autres limites importantes :
\begin{enumerate}
\item 
  $\displaystyle \lim_{x\to 0}\frac{1-\cos x}{x^2}=\frac{1}{2}$ ;
\item
  $\displaystyle \lim_{x\to \pm\infty}\left(1+\frac{a}{x}\right)^x=e^a$ ;
\item 
  $\displaystyle \lim_{x\to 0}(1+x)^{\frac{1}{x}}=e$ ;
\item 
  $\displaystyle \lim_{x\to 0}\frac{\ln(1+x)}{x}=1$ ;
\item 
  $\displaystyle \lim_{x\to 0}\frac{e^x-1}{x}=1$.
\end{enumerate}
  
\end{exercice}
