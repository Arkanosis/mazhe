\begin{corrige}{devoir3-0004}

Soit $L$ la courbe de $\eR^n$ paramétrée par $t\to \textbf{r}(t)= (r_1(t), \ldots, r_n(t))$, pour $t$ dans l'intervalle $I$. 
Pour calculer la longueur de $L$  il  nous faut utiliser la formule 
\begin{equation}
    \textrm{Longueur de } L = \int_{I} |\vect{r'}(t)|\, dt= \int_{I} \sqrt{\left(\frac{d r_1(t)}{dt}\right)^2+\cdots +\left(\frac{d r_n(t)}{dt}\right)^2}\, dt.
\end{equation}

On obtient alors :  
\begin{enumerate}
    \item $\vect{r'}(t)= (2\cos(t), 2\sin(t), 2)$, et $t$ dans $[0,2\pi]$,
  \begin{equation}
   \textrm{Longueur de la première courbe }= \int_0^{2\pi}  \sqrt{4\cos^2(t)+4\sin^2(t) + 4}\, dt= 4\sqrt{2}\pi.  
  \end{equation}
\item $\vect{r'}(t)= (-e^{-t}\cos(t)-e^{-t}\sin(t), -e^{-t}\sin(t)+e^{-t}\cos(t), -e^{-t})$ et  $t$ dans $[0,2]$, 
  \begin{equation}
    \begin{aligned}
      &\textrm{Longueur de la deuxième courbe }=\\
& \int_0^{2}  \sqrt{e^{-2t}(\cos(t)+\sin(t))^2+e^{-2t}(\cos(t)-\sin(t))^2+ e^{-2t}} dt= \\
    &\int_0^{2}  \sqrt{3e^{-2t}}\, dt= \sqrt{3}(1-e^{2}).
    \end{aligned}
  \end{equation}
\item $\vect{r'}(t)= (2t, 4, -4/t)$ pour $t$ dans l'intervalle $[1,2]$.  
 \begin{equation}
    \begin{aligned}
     & \textrm{Longueur de la troisième courbe }= \int_1^{2}  \sqrt{4t^2+16+ \frac{16}{t^2}}\, dt= \\
    &\int_1^{2}  2t+\frac{4}{t}\, dt= 3+4\ln(2).
    \end{aligned}
  \end{equation}
\end{enumerate}
\end{corrige}
