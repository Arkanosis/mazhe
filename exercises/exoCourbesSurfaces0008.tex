\begin{exercice}\label{exoCourbesSurfaces0008}

On considère le cercle ${\mathcal{C}}_R$  de rayon $R$ centré à l'origine qui a l'équation paramétrique $f(t)   = (R \cos t, R \sin t) $, où $ t \in [0, 2 \pi]$.  

\begin{enumerate}
	\item
Déterminer l'abscisse curviligne du point $f(t)$ par rapport au point $A = (R, 0)$. Calculer la longueur de ${\mathcal{C}}_R$.
\item

Déterminer  le vecteur tangent unitaire $ {\tau}(t)$,  le vecteur unitaire normal $\nu(t)$ en $f(t)$, la courbure $c(t)$ et le rayon de courbure $R(t)$ de ${\mathcal{C}}_R$ en $f(t)$.

\item

	Le vecteur $\nu(t)$  pointe vers l'intérieur ou vers l'extérieur du disque de rayon $R$?  Que se passe-t-il lorsqu'on remplace $f$ par la paramétrisation équivalente 
	\begin{equation}
		g(t)   = \big(R \cos (-t), R \sin (-t)\big),
	\end{equation}
	où $t \in [0, 2 \pi]$?
		
\end{enumerate}

\corrref{CourbesSurfaces0008}
\end{exercice}
