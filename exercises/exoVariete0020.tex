% This is part of Exercices et corrigés de CdI-1
% Copyright (c) 2011,2014
%   Laurent Claessens
% See the file fdl-1.3.txt for copying conditions.

\begin{exercice}\label{exoVariete0020}

Considérons sur $\eR^3$ le volume $V:\eR^3\times\eR^3\times\eR^3\rightarrow \eR$ \[V(u,v,w)=\f{1}{3!}\sum_{\sigma\in S_3}\epsilon(\sigma)u_{\sigma(1)}v_{\sigma(1)}w_{\sigma(1)}\] où $\epsilon(\sigma)$ est la signature de la permutation $\sigma$. Soient $\eR>0$ et $0<a<R$. Soit \[M_a=\left\{ ( x_1, x_2, x_3)\; \in \eR^3 \; | \; \sum_{i=1}^3x_i^2 = R^2 \; {\rm et} \; x_3>-a\right\}\]
\begin{enumerate}
\item Démontrez que $M_a$ est une variété dans $(\eR^n)^3$.
\item Démontrez que, si l'on définit pour deux vecteurs tangents à $M_a$ au point $x\in M_a$: \[\omega_x(A,B)=V(x,A,B) \; \forall A, B \in T_x(M_a)\] on a que $\omega$ est une 2-forme différentielle sur $M_a$.
\item Calculer $\int_{M_a}\omega$.
\item Que vaut $\lim_{a\rightarrow  R} \int_{M_a}\omega$?
\item Retrouvez ce résultat par le théorème de Stokes.
\end{enumerate}


\corrref{Variete0020}
\end{exercice}
