% This is part of Analyse Starter CTU
% Copyright (c) 2014
%   Laurent Claessens,Carlotta Donadello
% See the file fdl-1.3.txt for copying conditions.

\begin{corrige}{starterST-0002}

    La fonction n'est pas monotone parce qu'elle est tantôt croissante (les parties \( x+1\) et \( 2x\)), tantôt décroissante (la partie \( 4-x\)).

    Le graphe de la fonction \( f\) est en trois parties :
\begin{center}
   \input{Fig_XTGooSFFtPu.pstricks}
\end{center}
Les points bleus représentent des points sur le graphe de la fonction. Le graphe se poursuit à l'infini tant à gauche qu'à droite.

Il est visible qu'elle n'est pas continue en \( x=-1\) et \( x=1\). La justification précise est que 
\begin{equation}
    \lim_{x\to -1} f(x)
\end{equation}
n'existe pas parce que \( \lim_{x\to -1^-} f(x)=\lim_{x\to -1} x+1=0\) alors que \( \lim_{x\to -1^+} f(x)=\lim_{x\to -1} 2x=-2\).

La fonction \( f\) admet deux antécédents de \( -1\) : \( x=\frac{ 1 }{2}\) et \( x=3\). Ils sont visibles sur le graphe aux points \( Z_1\) et \( Z_2\). Formellement, il s'agit de résoudre l'équation
\begin{equation}
    f(x)=1.
\end{equation}

\end{corrige}
