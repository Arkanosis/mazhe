% This is part of Exercices et corrections de MAT1151
% Copyright (C) 2010
%   Laurent Claessens
% See the file LICENCE.txt for copying conditions.

\begin{corrige}{SerieTrois0003}

	La stabilité de l'algorithme revient à la stabilité de tous les problèmes intermédiaires. Le problème numéro $n$ est le problème qui consiste à trouver $x_{n+1}$ en terme de la donnée de la fonction ($a$ et $b$). Dans notre cas, l'équation \eqref{EqFPourNewtonUn} s'écrit
	\begin{equation}		\label{EqProbNDeNewtonTT}
		F_n(x_{n+1},x_n,(a,b))=x_{n+1}-x_n+\frac{ x_n^2+ax_n+b }{ 2x_n+a }.
	\end{equation}
	La solution (unique) de ce problème est
	\begin{equation}		\label{EqSolNewtonTT}
		x_{n+1}(a,b)=x_n-\frac{ x_n^2+ax_n+b }{ 2x_n+a }
	\end{equation}
	Le problème \eqref{EqProbNDeNewtonTT} a une solution unique donnée par \eqref{EqSolNewtonTT}. Cette solution est une fonction de $a$ et $b$ après substitution de $x_n$ par sa valeur en termes de $x_{n-1}$ puis $x_{n-2}$, etc. En tant que composée de fonctions $C^1$ (chacune des fonctions $x_i(x_{i-1},a,b)$), la fonction $x_{n+1}(a,b)$ est $C^1$, de telle sorte que le résultat de l'exercice \ref{exoSerieUn0002} montre que le problème est stable. Notons que cela n'est pas vrai si $x_n$ est sur le sommet de la parabole, mais nous avons vu dans l'exercice \ref{exoSerieDeux0006} que si $x_0$ n'était pas sur le sommet, alors les $x_n$ n'y seraient pas non plus.

	Le conditionnement absolu peut être approximé par la norme du gradient de la fonction $x_{n+1}$ :
	\begin{equation}
		\begin{aligned}[]
			\frac{ \partial F_n }{ \partial a }&=\frac{ x_n^2-b }{ (2x_n+a)^2 }\\
			\frac{ \partial F_n }{ \partial b }&=\frac{1}{ 2x_n+a }.
		\end{aligned}
	\end{equation}
	En prenant la norme nous trouvons
	\begin{equation}
		K_{\text{abs}}\simeq \frac{ \sqrt{(x_n^2-b)^2+(2x_n+a)^2} }{ (2x_n+a)^2 },
	\end{equation}
	et donc nous avons le conditionnement relatif
	\begin{equation}\label{EqKReltroitroti}
		\begin{aligned}[]
			K_{\text{rel}}&=K_{\text{abs}}\cdot\frac{ \sqrt{a^2+b^2} }{ | x_{n+1}(a,b,x_n) | }\\
			&=\frac{ \sqrt{(x_n^2-b)^2+(2x_n+a)^2}\sqrt{a^2+b^2} }{ | 2x_n+a |\cdot| x_{n+1}(a,b,x_n) | }
		\end{aligned}
	\end{equation}
	où $x_{n+1}(a,b,x_n)$ est donné par la formule \eqref{EqSolNewtonTT}.

	Si l'algorithme converge\footnote{Voir exercice \ref{exoSerieTrois0004}}, alors les $x_n$ convergent vers la solution exacte. Or nous savons la solution exacte du problème depuis le jardin d'enfance; nous avons donc :
	\begin{equation}		\label{EqSilimExisteTroistrois}
		\lim x_n=\frac{ 1 }{2}\big( -a+\sqrt{a^2-4b} \big).
	\end{equation}
	
	Le conditionnement relatif asymptotique est obtenu en substituant \eqref{EqSilimExisteTroistrois} dans l'expression \eqref{EqKReltroitroti} parce que quand une suite a une limite, sa limite supérieure est égale à sa limite.

\end{corrige}
