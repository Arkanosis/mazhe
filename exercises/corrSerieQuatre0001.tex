% This is part of Exercices et corrections de MAT1151
% Copyright (C) 2010
%   Laurent Claessens
% See the file LICENCE.txt for copying conditions.

\begin{corrige}{SerieQuatre0001}

	Le $s=0$ indique qu'on a affaire à des nombres positifs. Dans le cas de la base $10$ ($b=10$), le nombre $[0012,200000]$ représente
	\begin{equation}
		0\cdot 10^{3}+0\cdot 10^2+1\cdot 10^1+2\cdot 10^0+2\cdot 10^{-1}+0\cdot 10^{-2}+\ldots+0\cdot 10^{-6}=12.2.
	\end{equation}
	C'est le $12.2$ usuel comme on apprend à l'écrire au jardin d'enfants.

	En base $3$, c'est le même jeu, sauf qu'on remplace les $10$ par des $3$. Nous avons donc
	\begin{equation}
		1\cdot 3^1+2\cdot 3^0+2\cdot 3^{-1}=3+2+\frac{ 2 }{ 3 }=\frac{ 17 }{ 3 }.
	\end{equation}

\end{corrige}
