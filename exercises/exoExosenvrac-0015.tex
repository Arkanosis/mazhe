\begin{exercice}\label{exoExosenvrac-0015}\textbf{ (6 points) }

Déterminer dans chaque cas le domaine de définition, la périodicité et/ou les symétries éventuelles et les limites aux extrêmes du domaine. Calculer ensuite la dérivée de la fonction, la où elle est définie.   
    
    \begin{enumerate}
    \item $\displaystyle f_1(x)= x^3-1$ ;
      \begin{enumerate}
      \item Domaine de définition : \ldots 
        \vspace{3mm}
      \item La fonction est périodique ? Si oui, trouver sa période.  
        \vspace{3mm}
      \item La fonction est paire? Impaire ? Expliquer.  
        \vspace{1cm}
      \item Limites aux extrêmes du domaine : \ldots
        \vspace{1cm}
      \item Dérivée : \ldots 
        \vspace{3mm}
      \end{enumerate}
    \item $\displaystyle f_2(x)= e^{\cos(x)}$ ;
      \begin{enumerate}
      \item Domaine de définition : \ldots 
        \vspace{3mm}
      \item La fonction est périodique ? Si oui, trouver sa période.  
        \vspace{3mm}
      \item La fonction est paire? Impaire ? Expliquer.  
        \vspace{1cm}
      \item Limites aux extrêmes du domaine : \ldots
        \vspace{1cm}
      \item Dérivée : \ldots 
        \vspace{3mm}
      \end{enumerate}
    \item $\displaystyle f_3(x)= \frac{x}{x-2}$ ;
      \begin{enumerate}
      \item Domaine de définition : \ldots 
        \vspace{3mm}
      \item La fonction est périodique ? Si oui, trouver sa période.  
        \vspace{3mm}
      \item La fonction est paire? Impaire ? Expliquer.  
        \vspace{1cm}
      \item Limites aux extrêmes du domaine : \ldots
        \vspace{1cm}
      \item Dérivée : \ldots 
        \vspace{3mm}
      \end{enumerate}      
    \end{enumerate}

\corrref{Exosenvrac-0015}

\end{exercice}
