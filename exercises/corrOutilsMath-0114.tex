% This is part of Outils mathématiques
% Copyright (c) 2011
%   Laurent Claessens
% See the file fdl-1.3.txt for copying conditions.

\begin{corrige}{OutilsMath-0114}

    Une paramétrisation du cylindre situé autour de l'axe $x$ est donnée par une adaptation des coordonnées cylindriques :
    \begin{equation}
        \phi(x,\theta)=\begin{pmatrix}
            x    \\ 
            2\cos\theta    \\ 
            2\sin\theta    
        \end{pmatrix}.
    \end{equation}
    
    Pour trouver le plan tangent, la meilleur façon est souvent de trouver deux vecteurs tangents, et d'en déduire un vecteur normal en prenant le produit vectoriel. Les vecteurs tangents à la paramétrisation sont 
    \begin{equation}
        \begin{aligned}[]
            T_{\theta}&=\begin{pmatrix}
                0    \\ 
                -2\sin\theta    \\ 
                2\cos\theta    
            \end{pmatrix},&
            T_x&=\begin{pmatrix}
                1    \\ 
                0    \\ 
                0    
            \end{pmatrix}.
        \end{aligned}
    \end{equation}
    Afin de savoir quels sont les vecteurs tangents \emph{au point demandé},  il faut trouver à quels $x$ et $\theta$ correspond le point $(-1,1,-\sqrt{3})$. Manifestement, $x=-1$. En ce qui concerne $\theta$, nous avons
    \begin{equation}
        \begin{aligned}[]
            2\cos\theta&=1\\
            2\sin\theta&=-\sqrt{3}.
        \end{aligned}
    \end{equation}
    Cela fait $\theta=-\pi/3$. Les vecteurs tangents (et donc les générateurs du plan tangent) sont alors
    \begin{equation}
        \begin{aligned}[]
            T_{\theta}(-1,-\frac{ \pi }{ 3 })&=\begin{pmatrix}
                0    \\ 
                \sqrt{3}    \\ 
                1    
            \end{pmatrix},&
            T_x(-1,-\frac{ \pi }{ 3 })&=\begin{pmatrix}
                1    \\ 
                0    \\ 
                0    
            \end{pmatrix}.
        \end{aligned}
    \end{equation}
    
    Un vecteur normal au cylindre au point $\phi(-1,-\pi/3)$ est donné par
    \begin{equation}
        \begin{vmatrix}
            e_x    &   e_y    &   e_z    \\
            0    &   \sqrt{3}    &   1    \\
            1    &   0    &   0
        \end{vmatrix}=e_y-\sqrt{3}e_z=\begin{pmatrix}
            0    \\ 
            1    \\ 
            -\sqrt{3}    
        \end{pmatrix}.
    \end{equation}
    L'équation du plan tangent est donc de la forme
    \begin{equation}
        y-\sqrt{3}z+d=0
    \end{equation}
    pour un certain $d$ à fixer. Nous le fixons en imposant que ce plan passe par le point $(-1,1,-\sqrt{3})$. Nous demandons donc que
    \begin{equation}
        1-\sqrt{3}(-\sqrt{3})+d=0,
    \end{equation}
    ce qui fait $d=-4$. Le plan est donc
    \begin{equation}
        y-\sqrt{3}z-4=0.
    \end{equation}

\end{corrige}
