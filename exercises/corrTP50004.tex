% This is part of the Exercices et corrigés de mathématique générale.
% Copyright (C) 2009
%   Laurent Claessens
% See the file fdl-1.3.txt for copying conditions.
\begin{corrige}{TP50004}

	Une matrice est inversible si et seulement si son déterminant est non nul. Ici, le déterminant vaut
	\begin{equation}
		\det(M)=a(a^2-b^2).
	\end{equation}
	La matrice $M$ sera donc non inversible si $a=0$ ou bien si $a=\pm b$. Dans tous les autres cas, elle est inversible.

	Pour les vecteurs et valeurs propres, nous regardons l'équation caractéristique
	\begin{equation}
		\begin{vmatrix}
			a-\lambda	&	0	&	b	\\
			b	&	a-\lambda	&	0	\\
			b	&	0	&	a-\lambda
		\end{vmatrix}=
		(a-\lambda)\big( (a-\lambda)^2-b^2 \big)=0.
	\end{equation}
	La première valeur propre évidente est $\lambda_1=a$. Les deux autres sont les solutions de $(a-\lambda)^2-b^2$, qui est une équation usuelle du second degré. Les solutions sont $a+b$ et $a-b$.  Au final :
	\begin{equation}
		\begin{aligned}[]
			\lambda_2&=a\\
			\lambda_2&=a+b\\
			\lambda_3&=a-b.
		\end{aligned}
	\end{equation}
	La proposition $14.8$ de la page 202 du cours dit que si une matrice a trois valeurs propres distinctes, alors elle est diagonalisable. Ici, c'est le cas tant que $b\neq 0$. Si $b=0$, alors les choses sont moins claires. Lorsque $a=0$, l'équation caractéristique devient
	\begin{equation}
		(a-\lambda)^3=0,
	\end{equation}
	dont la solution $\lambda=a$ est une racine triple. Il y a donc de fortes chances qu'il y ait trois vecteur propres correspondants. En effet, dans le cas $b=0$, la matrice $M$ se réduit à
	\begin{equation}
		M_{b=0}=
		\begin{pmatrix}
			a	&	0	&	0	\\
			0	&	a	&	0	\\
			0	&	0	&	a
		\end{pmatrix}.
	\end{equation}
	Là, c'est bien clair que les trois vecteurs de base $e_1$, $e_2$ et $e_3$ sont vecteurs propres de valeur propre $a$ (sauf si $a=0$). En résumé, la matrice $M$ est diagonalisable tout le temps sauf si $a=0$ et $b=0$ en même temps.


\end{corrige}
