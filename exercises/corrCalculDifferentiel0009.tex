\begin{corrige}{CalculDifferentiel0009}

	Nous allons montrer deux méthodes pour résoudre cet exercice.

	\begin{description}
		\item[Première méthode]
			Commençons par écrire le changement de variable de façon très explicite et dans les deux sens :
			\begin{subequations}
				\begin{align}
					x(u,v)&=u+v\\
					y(u,v)&=u-v
				\end{align}
			\end{subequations}
			et
			\begin{subequations}
				\begin{align}
					u(x,y)=\frac{ 1 }{2}(x+y)\\
					v(x,y)=\frac{ 1 }{2}(x-y)
				\end{align}
			\end{subequations}
			Nous introduisons la fonction $\tilde f$ qui est la fonction $f$ «dans les coordonnées $(u,v)$» :
			\begin{equation}
				\tilde f(u,v)=f\big( x(u,v),y(u,v) \big).
			\end{equation}
			La fonction $f$ s'exprime en terme de $\tilde f$ de la façon suivante :
			\begin{equation}
				f(x,y)=\tilde f\big( u(x,y),v(x,y) \big).
			\end{equation}
			Nous pouvons donc écrire les dérivées partielles de $f$ en termes de celles de $\tilde f$ en utilisant la règle de dérivation des fonctions composées :
			\begin{equation}
				\begin{aligned}[]
						\frac{ \partial f }{ \partial x }(x,y)&=\frac{ \partial \tilde f }{ \partial u }\big( u(x,y),v(x,y) \big)\frac{ \partial u }{ \partial x }(x,y)+\frac{ \partial \tilde f }{ \partial v }\big( u(x,y),v(x,y) \big)\frac{ \partial v }{ \partial x }(x,y)\\
						\frac{ \partial f }{ \partial y }(x,y)&=\frac{ \partial \tilde f }{ \partial u }\big( u(x,y),v(x,y) \big)\frac{ \partial u }{ \partial y }(x,y)+\frac{ \partial \tilde f }{ \partial v }\big( u(x,y),v(x,y) \big)\frac{ \partial v }{ \partial y }(x,y).
				\end{aligned}
			\end{equation}
			Les dérivées de $u$ et $v$ par rapport à $x$ et $y$ sont connues. Nous avons donc
			\begin{equation}
				\begin{aligned}[]
					\frac{ \partial f }{ \partial x }(x,y)&=\frac{ 1 }{2}\frac{ \partial \tilde f }{ \partial u }\big( u(x,y),v(x,y) \big)+\frac{ 1 }{2}\frac{ \partial \tilde f }{ \partial v }\big( u(x,y),v(x,y) \big)\\
					\frac{ \partial f }{ \partial y }(x,y)&=\frac{ 1 }{2}\frac{ \partial \tilde f }{ \partial u }\big( u(x,y),v(x,y) \big)-\frac{ 1 }{2}\frac{ \partial \tilde f }{ \partial v }\big( u(x,y),v(x,y) \big).
				\end{aligned}
			\end{equation}
			En substituant ces valeurs dans l'équation pour $f$, nous trouvons la condition suivante pour $\tilde f$ :
			\begin{equation}
				\frac{ \partial \tilde f }{ \partial v }\big( u(x,y),v(x,y) \big)=0,
			\end{equation}
			c'est à dire que $\tilde f$ ne dépend pas de $v$. Il existe donc une fonction $\psi$ telle que $\tilde f(u,v)=\psi(u)$. Cela donne, sur $f$, la forme suivante :
			\begin{equation}
				f(x,y)=\tilde f\big( u(x,y),v(x,y) \big)=\psi\big( u(x,y) \big)=\psi\big( (x+y)/2 \big).
			\end{equation}
			La conclusion est que la fonction $f$ ne peut être qu'une fonction de $x+y$, ou encore que les variables $x$ et $y$ ne peuvent apparaître que sous la combinaison $x+y$.

			Par exemple les fonctions suivantes sont bonnes :
			\begin{subequations}
				\begin{align}
					f(x,y)&=x+y\\
					f(x,y)&=\frac{ \cos(x+y) }{ (x+y)^2 }.
				\end{align}
			\end{subequations}
			Mais la fonction $f(x,y)=x^2+y^2$ n'est pas bonne.
	\end{description}
	

\end{corrige}
