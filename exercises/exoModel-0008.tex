% This is part of Agregation : modélisation
% Copyright (c) 2011
%   Laurent Claessens
% See the file fdl-1.3.txt for copying conditions.

\begin{exercice}\label{exoModel-0008}

    Soit \( (X_1,\ldots,X_n)\) un échantillon de loi parente \( \dN(\theta,1)\) avec \( \theta\in\{ \theta_0,\theta_1 \}\). Nous supposons \( \theta_0<\theta_1\). Nous voulons tester \( H_0=\{ \theta_0 \}\) contre \( H_1=\{ \theta_1 \}\). Nous proposons le test suivant. La variable de décision sera \( \bar X_n\) et la région de rejet sera
    \begin{equation}
        W=\{ (x_1,\ldots,x_n)\in\eR^n\tq\frac{1}{ n }\sum_ix_i>\frac{ \theta_1+\theta_2 }{ 2 } \}.
    \end{equation}
    
    \begin{enumerate}
        \item
            Donner le risque de première espèce de ce test.
        \item
            Soit \( 0<\alpha<1\). Pour quelle valeur de \( n\) le tests a-t-il un risque de première espèce égal à \( \alpha\) ?
        \item
            Donner la puissance du test.
    \end{enumerate}
    Le réponses peuvent être exprimées en termes de la fonction de répartition \( F\) de la loi normale centrée réduite.

\corrref{Model-0008}
\end{exercice}
