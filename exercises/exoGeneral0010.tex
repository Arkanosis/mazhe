% This is part of the Exercices et corrigés de mathématique générale.
% Copyright (C) 2009-2011
%   Laurent Claessens
% See the file fdl-1.3.txt for copying conditions.
\begin{exercice}\label{exoGeneral0010}

Calculer les limites suivantes ($\frac{ 0 }{ 0 }$, $\frac{ \infty }{ \infty }$, $\infty-\infty$,$0\cdot\infty$).
\begin{multicols}{2}
	
\begin{enumerate}

\item
$\lim_{x\to 2} \frac{ x^2+x-6 }{ x^2-4 }$.

\item
$\lim_{x\to \frac{ \pi }{ 4 }} \frac{ 1-\tan(x)}{\cos(2x)}$.

\item
$\lim_{x\to \frac{ \pi }{ 2 }} \frac{ \sec(x) }{ \tan(x) }$.

\item
$\lim_{x\to 0} \frac{ x^4-2x^3 }{ 2x-\sin(2x) }$.

\item
$\lim_{x\to \pm\infty} \frac{ x }{ \sqrt{1+x^2} }$.

\item
$\lim_{x\to 0} \frac{ a^x-b^x }{ x }$. 

\item
$\lim_{x\to 0} \frac{ e^x-e^{-x} }{ \sin(x) }$.

\item
$\lim_{x\to 1} \left( \frac{ x }{ x-1 }-\frac{1}{ \ln(x) } \right)$.


\item
$\lim_{x\to 0} \frac{ e^x-e^{\sin(x)} }{ x-\sin(x) }$.


\item
$\lim_{x\to 0} \frac{ x+\sin(2x) }{ x-\sin(2x) }$.



\item
$\lim_{x\to 0}\left( \frac{1}{ \sin(x) }-\frac{ \cos(x) }{ \sin(x) } \right)$.



\item
$\lim_{x\to 0} \frac{ \sqrt{x+9}-3 }{ x }$.



\item
$\lim_{x\to 0} \frac{ \tan(x) }{ x }$.


\item
$\lim_{x\to_{>} 0} x^m\ln(x)$.



\item
$\lim_{x\to_{<} \frac{ \pi }{ 2 }} \big( \tan(3x)-\tan(x)\big)$.


\item
$\lim_{x\to \infty} x\sin(a/x)$.

\item
$\lim_{x\to \frac{ \pi }{2}} \frac{ \ln\big( \sin(x) \big) }{ (\pi-2x)^2 }$.

\end{enumerate}
\end{multicols}
Ici, $a$ et $b$ sont des réels positifs, et $m$ est un entier positif.


\corrref{General0010}
\end{exercice}
