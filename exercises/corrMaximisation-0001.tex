% This is part of the Exercices et corrigés de mathématique générale.
% Copyright (C) 2010
%   Laurent Claessens
% See the file fdl-1.3.txt for copying conditions.

\begin{corrige}{Maximisation-0001}

	En ce qui concerne $\partial_yf$, il n'y a aucun problème parce que
	\begin{equation}
		\frac{ \partial f }{ \partial y }(0,0)=\lim_{t\to 0}\frac{ f(0,t)-f(0,0) }{ t }=0
	\end{equation}
	parce que $f(0,t)=f(0,0)=0$ pour tout $t\neq 0$.

	La dérivée par rapport à $x$ est plus douteuse parce que $f(t,0)$ existe (et vaut $0$) lorsque $a>0$ du fait que $y^a=0$ se trouve au numérateur, mais $f(t,0)$ n'existe pas lorsque $a<0$ parce que $y^a=0$ se trouve au dénominateur. La limite ne peut donc pas être calculée et la dérivée partielle $\partial_xf(0,0)$ n'existe pas.

\end{corrige}
