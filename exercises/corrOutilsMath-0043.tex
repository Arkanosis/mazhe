% This is part of Exercices et corrigés de CdI-1
% Copyright (c) 2011
%   Laurent Claessens
% See the file fdl-1.3.txt for copying conditions.

\begin{corrige}{OutilsMath-0043}

    Les dérivées partielles sont données par
    \begin{equation}
        \begin{aligned}[]
            \frac{ \partial f }{ \partial x }(x,y)&=y\cos(x+y)-xy\sin(x+y)\\
            \frac{ \partial f }{ \partial y }(x,y)&=x\cos(x+y)-xy\sin(x+y).
        \end{aligned}
    \end{equation}
    Étant donne que ce sont des fonctions continues, nous pouvons utiliser la formule de la dérivée directionnelle en termes des dérivées partielles :
    \begin{equation}
        \frac{ \partial f }{ \partial u }(a,b)=u_1\frac{ \partial f }{ \partial x }(a,b)+u_2\frac{ \partial f }{ \partial y }(a,b).
    \end{equation}
    Ici $(a,b)=\big( \frac{ \pi }{ 5 },\frac{ 2\pi }{ 15 } \big)$. Nous avons alors
    \begin{equation}
        \begin{aligned}[]
            \frac{ \partial f }{ \partial x }(\frac{ \pi }{ 5 },\frac{ 2\pi }{ 15 })&=\frac{ 2\pi }{ 15 }\cos(\frac{ \pi }{ 3 })-\frac{ 2\pi^2 }{ 75 }\sin(\frac{ \pi }{ 3 })=\frac{ \pi }{ 15 }-\frac{ \sqrt{3}\pi^2 }{ 75 }\\
            \frac{ \partial f }{ \partial y }(\frac{ \pi }{ 5 },\frac{ 2\pi }{ 15 })&=\frac{ \pi }{ 10 }-\frac{ \sqrt{3}\pi^2 }{ 75 }.
        \end{aligned}
    \end{equation}
    Donc
    \begin{equation}
        \frac{ \partial f }{ \partial u }(\frac{ \pi }{ 5 },\frac{ 2\pi }{ 15 })=\frac{1}{ \sqrt{2} }\left[ \frac{ \pi }{ 15 }-\frac{ \sqrt{3}\pi^2 }{ 75 }-\frac{ \pi }{ 10 }+\frac{ \sqrt{3}\pi^2 }{ 75 } \right].
    \end{equation}
    En calculant, le résultat est $-\frac{ \pi }{ 30\sqrt{2} }$.

\end{corrige}
