% This is part of the Exercices et corrigés de CdI-2.
% Copyright (C) 2008, 2009
%   Laurent Claessens
% See the file fdl-1.3.txt for copying conditions.


\begin{corrige}{119}


%TODO: refaire le dessin
%Les fonctions sont esquissées sur la figure \ref{LabelFigexouuix}.
%\newcommand{\CaptionFigexouuix}{Quelques unes des fonctions $f_n$.}
%\input{pictures_tex/Fig_exouuix.pstricks}

Remarquons d'abord que toutes les fonctions $f_{n}$ sont discontinues en $1-\frac{1}{ n+1 }$, et que pour tout $n$, nous avons $f_n(1)=0$. D'autre part, le domaine $[0,1-\frac{1}{ n+1 }]$ sur lequel $f_n$ est nulle est croissant (avec $n$) et tend vers $[0,1]$ entier. Donc, pontuellement, nous avons la convergence $f_n\to 0$.

La convergence uniforme sur $[0,1]$ demande d'avoir
\begin{equation}
	\lim_{n\to\infty}\| f_n-f \|_{\infty}=0.
\end{equation}
Le calcul de cette quantité est aisé :
\begin{equation}
	\lim_{n\to\infty}\| f_n-f \|_{\infty}=\lim_{n\to\infty}\sup_{x\in[0,1]}| f_n |=\lim_{n\to\infty}\left( \frac{1}{ \left( 1-\frac{ 1 }{ n+1 } \right)^n }-1 \right),
\end{equation}
où la limite se calcule en remarquant que $1-\frac{ 1 }{ n+1 }=\frac{ n }{ n+1 }=\left( 1+\frac{ 1 }{ n }\right)^{-1}$, et donc nous nous retrouvons avec la limite
\begin{equation}
	\lim_{n\to\infty}\left( 1+\frac{1}{ n } \right)^n-1=e-1\neq 0.
\end{equation}
Il n'y a donc pas convergence uniforme sur $[0,1]$.

La suite converge par contre uniformément sur tout compact de $[0,1[$. En effet, soit $b$ le maximum du compact $K\subset[0,1[$. Dès que $1-\frac{ 1 }{ n+1 }>b$, nous avons $f_{n}|_{K}=0$.

\end{corrige}
