% This is part of Exercices et corrigés de CdI-1
% Copyright (c) 2011
%   Laurent Claessens
% See the file fdl-1.3.txt for copying conditions.

\begin{corrige}{0095}

Par le théorème de Rolle, il y a un zéro de la dérivée entre deux zéros de la fonction. Comme la fonction possède $n$ zéros, la dérivée en possède \emph{au moins} $n-1$. Nous savons par ailleurs que la dérivée est un polynôme de degré $n-1$ et possède donc \emph{au plus} $n-1$ zéros. La combinaison des deux donne la conclusion.

Par récurrence, la dérivée $k$ième a $n-k$ zéros distincts.

Si $P$ est de degré $0$ ou $1$, alors elle est uniformément continue parce que linéaire (ou affine). Si le degré est plus grand ou égal à $2$, il faut encore montrer que le polynôme n'est pas uniformément continue.

Il y a au plus $n+(n-1)+\cdots+1$ point distincts où $P$ ou une de ses dérivées s'annule. Au delà de ces points, $P$ et ses dérivées ne changent plus de signe, et sont donc toutes positives ou toutes négatives. En particulier, $P'$ est une fonction qui tend soit vers l'infini soit vers moins l'infini (c'est ici que nous utilisons le fait que le degré est plus grand ou égal à $2$). Soit $x$ dans cette zone, et notons $m$, le minimum de $P'$ sur $B(x,\delta)$, alors
\begin{equation}
	| P(x)-P(x+\delta) |>m\delta,
\end{equation}
et il suffit de prendre $x$ assez grand pour que $m>\epsilon/\delta$ pour contredire l'uniforme continuité.

\end{corrige}
