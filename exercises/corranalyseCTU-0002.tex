% This is part of Analyse Starter CTU
% Copyright (c) 2014
%   Laurent Claessens,Carlotta Donadello
% See the file fdl-1.3.txt for copying conditions.

\begin{corrige}{analyseCTU-0002}

    Le fait que \( x\mapsto\ln(x)\) soit une primitive de \( \frac{1}{ x }\) se traduit par le fait que
    \begin{equation}
        \ln(x)=\int_1^x\frac{1}{ t }dt.
    \end{equation}
    La primitive de la fonction \( x\mapsto 1\frac{1}{ x }\) pour les \( x\) négatifs sera la fonction de \( x \) donnée par
    \begin{equation}
        \int_{-1}^{x}\frac{1}{ t }dt=\int_1^{-x}\frac{1}{ -u }(-du)=\int_1^{-x}\frac{1}{ u }du=\ln(-x).
    \end{equation}
    Dans ce calcul \( x\) est une constante négative.

    Donc pour les \( x\) négatifs, la primitive choisie de \( \frac{1}{ x }\) est la fonction \( x\mapsto\ln(-x)\).

    Pour toutes les valeurs de \( x\) nous avons que la primitive de \( \frac{1}{ x }\) est \( \ln(| x |)\).

\end{corrige}
