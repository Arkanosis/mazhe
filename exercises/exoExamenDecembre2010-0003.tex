% This is part of Exercices de mathématique pour SVT
% Copyright (c) 2011
%   Laurent Claessens et Carlotta Donadello
% See the file fdl-1.3.txt for copying conditions.

\begin{exercice}\label{exoExamenDecembre2010-0003}

On considère les trois suites suivantes, définies pour $n\geq 1$, 
\begin{equation}\nonumber
  \begin{array}{ccc}
   \displaystyle  u_n=\frac{1}{n}, &\displaystyle  v_n=\frac{1}{n+1}, & \displaystyle w_n=u_nv_n.
  \end{array}
\end{equation}
\begin{enumerate}
\item Écrire les premiers 4 termes de chaque suite.
\item Montrer que la suite $(u_n)_n$ est bornée et décroissante.  
\item Les trois suites convergent. Trouver la limite de chaque suite.
\item Monter que $\displaystyle w_n=\int_{n}^{n+1}\frac{1}{x^2}\, dx$, pour tout $n\geq 1$.
\item Esquisser le graphe de la fonction $1/x^2$ pour $x$ dans $[1,4]$. Donner une intérpretation graphique des intégrales  $w_1$, $w_2$, $w_3$. 
\item On définit la suite $(s_n)_n$ par $s_n= w_1+w_2+\ldots+ w_n$. Démontrer, par récurrence que $s_n=\int_{1}^{n+1}\frac{1}{x^2}\,dx$.
\item Calculer la limite de $(s_n)_n$. Conseil : calculer l'intégrale du point précédent.  
\end{enumerate}

\corrref{ExamenDecembre2010-0003}
\end{exercice}
