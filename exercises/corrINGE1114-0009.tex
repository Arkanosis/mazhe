% This is part of Un soupçon de physique, sans être agressif pour autant
% Copyright (C) 2006-2011
%   Laurent Claessens
% See the file fdl-1.3.txt for copying conditions.


\begin{corrige}{INGE1114-0009}

	Prouvons l'égalité \eqref{SubEqsupexoSerUnab}. Il n'y a que deux possibilités. Soit $a>b$, soit $a<b$. Si $a>b$, alors $| a-b |>0$ et nous pouvons supprimer les valeurs absolues. Ce que nous avons est alors
	\begin{equation}
		\frac{ 1 }{2}(a+b+a-b)=a,
	\end{equation}
	et effectivement, quand $a>b$, nous avons $\sup\{ a,b \}=a$. Par contre, si nous supposons que $a<b$, nous savons que $| a-b |=-(a-b)$ et donc
	\begin{equation}
		\frac{ 1 }{2}(a+b+| a-b |)=\frac{ 1 }{2}(a+b-a+b)=b.
	\end{equation}

\end{corrige}
