% This is part of Exercices et corrections de MAT1151
% Copyright (C) 2010
%   Laurent Claessens
% See the file LICENCE.txt for copying conditions.

\begin{exercice}\label{exoSerieQuatre0005}

Une machine représente les nombres en virgule flottante en base $b=10$ et avec $t=4$ chiffres significatifs. On considère le problème $x=F(d)=d-1$ avec $d=1,00098$. L'erreur relative après codage par $\hat{d}=\mbox{fl}(d)=0,1001.10^1$ est approximativement $\epsilon_d\simeq\frac{1}{50000}$. Or la résolution machine, livre une solution approchée $\hat{x}=1.10^{-3}$ dont l'erreur relative s'élève à 
\begin{equation}
	\epsilon_x=\left|\frac{F(\hat{d})-x}{x}\right|\simeq\frac{1}{50}.
\end{equation}
Expliquer ce phénomène de propagation en terme du conditionnement de $F$.

\corrref{SerieQuatre0005}
\end{exercice}
