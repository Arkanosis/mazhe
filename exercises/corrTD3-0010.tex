% This is part of Exercices de mathématique pour SVT
% Copyright (C) 2010-2011
%   Laurent Claessens et Carlotta Donadello
% See the file fdl-1.3.txt for copying conditions.



\begin{corrige}{TD3-0010}
  

  \begin{enumerate}
  \item 
      On démontre ce premier point par récurrence. D'abord par définition $u_0=1$, donc $u_0\in[0,1]$. Ensuite si $u_n\in[0,1]$ pour touts indices $n$ entre $0$ et $k$ alors on a 
            \[
            u_{k+1}=\frac{2u_k}{1+u_k}\frac{u_k}{u_k+s}\leq\frac{2}{1}\frac{1}{s}\leq 1.
            \] 
            La première inégalité est obtenue en observant que pour majorer une fraction il faut majorer le numérateur et minorer le dénominateur. Dans notre cas, le numérateur est toujours $\leq 2 \cdot 1$ parce que $u_k\leq 1$ et le dénominateur est supérieur à $s$, parce que $u_k\geq 0$.

    \item 
        
        Pour démontrer la décroissance, nous écrivons
        \begin{equation}
            \frac{ u_{n+1} }{ u_n }=\frac{ 2u_n }{ (1+u_n)(s+u_n) }\leq \frac{ 2u_n }{ (u_n+s) }
        \end{equation}
        parce que \( u_n\in\mathopen[ 0 , 1 \mathclose]\), donc en remplaçant dans le dénominateur \( (1+u_n)\) par \( 1\), la fraction diminue. De la même manière, si nous remplaçons dans le dénominateur \( u_n\) par zéro et \( s\) par \( 2\), nous agrandissons la fraction :
        \begin{equation}
            \frac{ 2u_n }{ u_n+s }\leq \frac{ 2u_n }{ 2 }=u_n\leq 1.
        \end{equation}
        Par conséquent la suite est décroissante.
        
        Une méthode alternative pour montrer la décroissance est de remarquer que la récurrence définissant la suite peut être écrite sous la forme
        \begin{equation}
            u_{n+1}=u_ng(u_n)
        \end{equation}
        avec
        \begin{equation}
            g(x)=\frac{2x}{(1+x)(x+s)}.
        \end{equation}
        Cette fonction est une fonction rationnelle. Pour la majorer il faut majorer son numérateur et minorer sont dénominateur, en sachant que $x\in[0,1]$. On obtient alors 
        \begin{equation}
            g(x)\leq \frac{2}{s}\leq 1.
        \end{equation}
        Nous avons donc \( u_{n+1}=g(u_n)u_n\leq u_n\leq 1\).

    \item 

        La suite est bornée et décroissante. Elle est donc convergente. Rien n'oblige cependant que la limite soit exactement zéro. Pour le montrer, nous allons chercher les candidats limites, et montrer que seul \( 0\) convient. Il y a plusieurs façons de procéder.
        
        \begin{enumerate}
            \item
        
        Pour cela, écrivons la suite sous la forme
        \begin{equation}
            u_{n+1}=u_ng(u_n)
        \end{equation}
        avec
        \begin{equation}
            g(x)=\frac{2x}{(1+x)(x+s)}.
        \end{equation}
        Le premier candidat limite est \( u=0\). Les autres s'obtiennent en résolvant l'équation \( g(x)=1\). D'abord nous remarquons que \( g(0)=0\). Ensuite, en passant à la dérivée, nous voyons que la fonction \( g\) est croissante :
        \begin{equation}
            g'(x)=\frac{ 2(x+1)(s+x)-2x(x+1)-2x(s+x) }{ (x+1)^2(x+s)^2 }.
        \end{equation}
        Le dénominateur est certainement positif. Le numérateur vaut \( -2x^2+2s\). Étant donné que nous considérons \( x\in\mathopen[ 0 , 1 \mathclose]\) et \( s>2\), nous avons \( g'(x)>0\). Le maximum de la fonction \( g\) entre \( 0\) et \( 1\) est alors
        \begin{equation}
            g(1)=\frac{1}{ s+1 }<1.
        \end{equation}
        La fonction $g$ ne passe donc jamais par \( 1\) entre \( x=0\) et \( x=1\).

        Le seul candidat limite est donc zéro. 
        
            \item

                La recherche directe des points fixe revient à résoudre l'équation
                \begin{equation}
                    x=\frac{ 2x^2 }{ (x+1)(x+s) }.
                \end{equation}
                La solution \( x=0\) est évidente. Cherchons les autres solutions. Elles reviennent à résoudre
                \begin{equation}
                    x^2+x(1+s-2)+s=0.
                \end{equation}
                Les racines sont
                \begin{equation}
                    x=\frac{ -(s-1)\pm\sqrt{ (s-1)^2-4s   } }{ 2 }.
                \end{equation}
                Ce qui se trouve dans la racine carrée est automatiquement plus petit que \( (s-1)^2\), par conséquent la racine carrée est plus petite que \( (s-1)\) et la combinaison \( -(s-1)\pm\sqrt{\cdots}\) est négative. Par conséquent les derniers candidats limites sont négatifs et donc à rejeter parce que nous savons que la suite est positive.

        \end{enumerate}
        Étant donné que nous savons que la suite est convergente, nous savons qu'elle converge vers zéro.

  \end{enumerate}

\end{corrige}
