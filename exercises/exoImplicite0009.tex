% This is part of Exercices et corrigés de CdI-1
% Copyright (c) 2011
%   Laurent Claessens
% See the file fdl-1.3.txt for copying conditions.

\begin{exercice}\label{exoImplicite0009}

Soit $\Phi:\eR \times \eR^2 \rightarrow \eR^2 : (t,v) \rightarrow \Phi(t,v) =:
\varphi_t(v)$ une famille à un paramètre de difféomorphismes de $\eR^2$
($\varphi_t : \eR^2 \rightarrow \eR^2 : v \rightarrow \varphi_t(v)$ est un
difféomorphisme de $\eR^2$ pour $t \in \eR$).
\begin{enumerate}
\item
Montrer que si $\varphi_0$ possède un point fixe $v_0$ et que le spectre
de $(d\varphi_0)(v_0)$ ne contient pas $1$ (le réel $1$ n'est pas une
valeur propre de l'opérateur $(d\varphi_0)(v_0)$ de $R^2$) alors il
existe $\varepsilon > 0$ tel que pour $t \in ]-\varepsilon, \varepsilon[$
le difféomorphisme $\varphi_t$ possède aussi un point fixe.
\item
Montrer en exhibant un exemple que l'hypothèse sur le spectre de
$(d\varphi_0)(v_0)$ est essentielle.
\end{enumerate}


\corrref{Implicite0009}
\end{exercice}
