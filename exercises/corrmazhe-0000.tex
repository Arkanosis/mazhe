% This is part of (almost) Everything I know in mathematics
% Copyright (c) 2010-2014
%   Laurent Claessens
% See the file fdl-1.3.txt for copying conditions.

\begin{corrige}{mazhe-0000}

	\begin{enumerate}

		\item
			Ici la méthode des chemins pour est particulièrement éclairante. Regardons d'abord la fonction sur la droite $x=y$. Nous avons
			\begin{equation}
				f(x,y)=\frac{ x-x }{ 2x }=0.
			\end{equation}
			Donc la fonction est nulle sur toute la ligne.

			Si nous regardons maintenant la ligne verticale $x=0$, nous avons
			\begin{equation}
				f(0,y)=\frac{ -y }{ y }=-1,
			\end{equation}
			donc la fonction vaut $-1$ sur toute la ligne verticale.
       %TODO : refaire la figure
%Regardez la figure \ref{LabelFigExoHuitUnINGE}

		\item

		\item
			Regardons la technique des coordonnées polaires. Nous remplaçons $x$ par $r\cos(\theta)$ et $y$ par $r\sin(\theta)$ :
			\begin{equation}
				f(r,\theta)=\frac{ r^4\cos(\theta)\sin^3(\theta) }{ r^2 }=r^2\cos(\theta)\sin^3(\theta).
			\end{equation}
			Cette fonction tend vers zéro quand $r\to 0$. Nous avons donc 
			\begin{equation}
				\lim_{(x,y)\to(0,0)}f(x,y)=0.
			\end{equation}

			Pour cet exercice nous pouvons aussi utiliser la règle de l'étau en écrivant d'abord
			\begin{equation}
				0\leq | f(x,y) |\leq\frac{ | x | |y^3 | }{ | x^2+y^2 | }.
			\end{equation}
			Mais on a que $| x |\leq\sqrt{x^2+y^2}$, $| y |\leq\sqrt{x^2+y^2}$ et $| x^2+y^2 |=\big( \sqrt{x^2+y^2} \big)^2$, donc
			\begin{equation}
				0\leq| f(x,y) |\leq \frac{ \sqrt{x^2+y^2}\big( \sqrt{x^2+y^2} \big)^3 }{ \big( \sqrt{x^2+y^2} \big)^2 }=\big( \sqrt{x^2+y^2} \big)^2\to 0.
			\end{equation}

		\item
			En passant aux polaires, nous avons
			\begin{equation}
				f(r,\theta)=\frac{ r\cos\theta\sin\big( r\sin\theta \big) }{ r }=\cos(\theta)\sin\big( r\sin\theta \big).
			\end{equation}
			La limite de cette dernière fonction lorsque $r\to 0$ vaut zéro.

			Une autre façon de procéder consiste à multiplier et diviser par $y$ de telle façon à faire apparaître $\sin(y)/y$ dont nous connaissons la limite :
			\begin{equation}
				f(x,y)=\frac{ \sin(y) }{ y }\cdot\frac{ xy }{ \sqrt{x^2+y^2} }.
			\end{equation}
			La limite du premier facteur est $1$, tandis que le second peut être traité de façon classique en prenant la valeur absolue et en majorant $| x |$ par $\sqrt{x^2+y^2}$.
			
	\end{enumerate}

	%\newcommand{\CaptionFigExoHuitUnINGE}{Sur toute la ligne rouge, la fonction vaut zéro, tandis que sur la ligne bleue elle vaut $-1$. Au point $(0,0)$, les deux sont inconciliables. Donc la limite n'existe pas.}
	%\input{pictures_tex/Fig_ExoHuitUnINGE.pstricks}

\end{corrige}
