% This is part of the Exercices et corrigés de mathématique générale.
% Copyright (C) 2009
%   Laurent Claessens
% See the file fdl-1.3.txt for copying conditions.
\begin{corrige}{EquaDiff0003}

\begin{enumerate}


\item
Nous commençons par résoudre l'équation homogène
\begin{equation}
	\frac{ y'_H }{ y_H }=\frac{1}{ x-2 }
\end{equation}
dont la solution générale est
\begin{equation}
	y_H(x)=C(x-2).
\end{equation}
Afin de résoudre l'équation non homogène, nous utilisons la méthode de variation des constantes, qui donne $C'=2(x-2)$ comme équation pour $C(x)$. La solution est $C(x)=x^2-4x+K$ et donc nous avons
\begin{equation}
	y(x)=(x^2-4x+K)(x-2).
\end{equation}

\item
L'équation homogène, $y'_H+y_H\cotg(x)=0$, se résous en sachant qu'une primitive de $\cotg(x)$ est $\ln\big(\sin(x)\big)$. La solution de l'homogène est donc
\begin{equation}
	y_H(x)=\frac{ K }{ \sin(x) }.
\end{equation}
La méthode de variation des constantes demande donc de substituer
\begin{equation}
	\begin{aligned}[]
		y(x)&=\frac{ K(x) }{ \sin(x) }\\
		y'(x)&=\frac{ K'(x) }{ \sin(x) }-\frac{ K(x)\cos(x) }{ \sin^2(x) }
	\end{aligned}
\end{equation}
dans l'équation non homogène. Encore une fois, les termes en $K$ non dérivés se simplifient et nous restons avec
\begin{equation}
	K'(x)=5\sin(x) e^{\cos(x)},
\end{equation}
donc la solution est $K(x)=-5 e^{\cos(x)}$. La solution au problème posé est donc
\begin{equation}
	y(x)=\frac{ -5 e^{\cos(x)} }{ \sin(x) }.
\end{equation}


\end{enumerate}


\end{corrige}
