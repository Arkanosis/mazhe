% This is part of the Exercices et corrigés de mathématique générale.
% Copyright (C) 2009-2010
%   Laurent Claessens
% See the file fdl-1.3.txt for copying conditions.


\begin{corrige}{INGE1121La0003}

	Ici, la tentation est d'appliquer la méthode de Gram-Schmidt. Cela n'est pas directement possible parce qu'avant d'appliquer Gram-Schmidt, il nous faut une partie libre. La matrice formée par les vecteurs donnés est de rang deux:
	\begin{equation}
		\begin{pmatrix}
			 1	&	1	&	-2	&	-3	\\
			 1	&	2	&	-3	&	-4	\\ 
			 0	&	-2	&	2	&	2	 
		\end{pmatrix}.
	\end{equation}
	En substituant les lignes $ L_2\to L_2-L_1$ et $L_3\to L_3/2$, nous trouvons que la deuxième et la troisième ligne deviennent les mêmes, de telle sorte qu'il reste
	\begin{equation}
		\begin{pmatrix}
			 1	&	1	&	-2	&	-3	\\
			 0	&	1	&	-1	&	-1	
		\end{pmatrix},
	\end{equation}
	dont le rang est deux.

	Nous savons donc maintenant que $\dim(E)=2$. Prenons donc deux vecteurs linéairement indépendants dans $E$ et appliquons leur Gram-Schmidt. Un choix possible est
	\begin{equation}
		\begin{aligned}[]
			v_1&=(1,1,0)\\
			v_2&=(1,2,-2)
		\end{aligned}
	\end{equation}
	dont le Gram-Schmidt est
	\begin{equation}
		\begin{aligned}[]
			w_1&=(1,1,0)\\
			w_2&=(-\frac{1}{ 2 },\frac{1}{ 2 },-2)
		\end{aligned}
	\end{equation}

\end{corrige}
