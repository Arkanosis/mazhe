% This is part of the Exercices et corrigés de mathématique générale.
% Copyright (C) 2009-2010
%   Laurent Claessens
% See the file fdl-1.3.txt for copying conditions.


\begin{corrige}{INGE1121La0009}

	Nous échelonons la matrice de façon usuelle. Tellement usuelle que nous en confions les calculs à Sage.

	\VerbatimInput[tabsize=3]{exo15.sage}

	D'un point de vue technique, remarquez que la numérotation des lignes et colonnes commence à zéro, et non à un ! C'est comme ça dans beaucoup de langages de programmation.

	La sortie est

	\VerbatimInput[tabsize=3]{exo15.txt}

	À partir d'ici, nous pouvons discuter un petit peu. D'abord, la dernière ligne donne deux cas
	\begin{enumerate}

		\item
			$s=10$. Dans ce cas, il y a deux sous cas
			\begin{enumerate}

				\item
					$-2y_1+2y_2+y_4=0$. Ici, on peut simplement barrer la dernière ligne, et continuer. Il y aura une infinité de solutions.

					
					En continuant, nous tombons sur la troisième ligne. Il y a une discussion à faire selon que $r=4$ ou non.
				\item $-2y_1+2y_2+y_4\neq 0$. Ici, la dernière ligne dit qu'il n'y a pas de solutions.

			\end{enumerate}
			

	\end{enumerate}
	
	La troisième ligne donne une discussion similaire selon que $r=4$ ou non en fonction de $-y_1-y_2+y_3$. Les cas avec des solutions sont
	\begin{enumerate}

		\item
			$s=10$, $r=4$, $-2y_1+2y_2+y_4=0$, $-y_1-y_2+y_3=0$. 
			Dans ce cas, le système devient
			\begin{equation}
				\left(\begin{array}{cccc|c}
					1	&	2	&	3	&	4	&	y_1	\\
					0	&	-5	&	-5	&	-9	&	-2y_1+y_2
				\end{array}\right).
			\end{equation}
			La solution de ce système comporte deux paramètres et est
			\begin{equation}		\label{EqSolsUn15}
				\begin{aligned}[]
					x_1&=-\frac{ 2 }{ 5 }\alpha-\beta+y_1+\frac{ 2 }{ 5 }y_2\\
					x_2&=-\frac{ 9 }{ 5 }\alpha-\beta+\frac{1}{ 5 }y_2\\
					x_3&=\beta\\
					x_4&=\alpha.
				\end{aligned}
			\end{equation}
			Notez que les $y$ ne sont pas des paramètres. Ce sont des données du problème.
		\item
			$s=10$, $r\neq 4$, $-2y_1+2y_2+y_4=0$, $-y_1-y_2+y_3\neq 0$. Dans ce cas, seule la dernière ligne doit être barrée dans le système. Il reste un système de trois équations. 
			\begin{equation}
				\left(\begin{array}{cccc|c}
					1	&	2	&	3	&	4	&	y_1	\\
					0	&	-5	&	-5	&	-9	&	-2y_1+y_2\\
					0	&	0	&	r-4	&	0	&	-y_1-y_2+y_3
				\end{array}\right).
			\end{equation}
			
			
			Les solutions sont les mêmes que celles \eqref{EqSolsUn15}, mais $x_3$ n'est plus un paramètre :
			\begin{equation}
				\beta=x_3=\frac{ -y_1-y_2+y_3 }{ r-4 }.
			\end{equation}
		\item
			$s\neq 10$, $r= 4$, $-2y_1+2y_2+y_4=\neq $, $-y_1-y_2+y_3=0$. Ce cas-ci est similaire au précédent.

		\item
			$s\neq 10$, $r\neq 4$, $-2y_1+2y_2+y_4\neq 0$, $-y_1-y_2+y_3\neq 0$. Ici, c'est un système complet de $4$ équations à $4$ inconnues à résoudre.

	\end{enumerate}

\end{corrige}
