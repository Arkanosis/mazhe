% This is part of Exercices et corrigés de CdI-1
% Copyright (c) 2011
%   Laurent Claessens
% See the file fdl-1.3.txt for copying conditions.

\begin{corrige}{EqsDiff0002}

\begin{enumerate}

\item
Nous manipulons un tout petit peu l'équation en utilisant $y'=dy/dt$ :
\begin{equation}
	\begin{aligned}[]
		e^yy'&=(t^3+t)\\
		e^ydy&=(t^3+t)dt,
	\end{aligned}
\end{equation}
que nous intégrons des deux côtés pour trouver $e^y=\frac{ t^4 }{ 4 }+\frac{ t^2 }{ 2 }+C$. De là, nous déduisons la solution générale :
\begin{equation}
	y(t)=\ln\left( \frac{ t^4 }{ 4 }+\frac{ t^2 }{ 2 }+C \right).
\end{equation}

\item
$y'=1+y^2$. Nous avons $u(t)=1$ et $f(y)=1+y^2$ qui ne s'annule jamais, donc $I=J=\eR$. Petit calcul :
\begin{equation}
	G(y)=\int\frac{dy}{ 1+y^2 }=\arctan(y).
\end{equation}
La solution est donc implicitement donnée par $\arctan\big( y(t) \big)=t+C$, et donc explicitement par
\begin{equation}
	y(t)=\tan(t+C).
\end{equation}
Notons que, dans ce cas ci, nous sommes parvenu à isoler $y(t)$ dans l'équation $G\big( y(t) \big)=U(t)+C$. Cela n'est pas toujours possible, comme nous le verrons dans d'autres exercices.


\item
Nous avons $u(t)=\cos(t)$ et $f(y)=\frac{1}{ 1+ e^{y} }$, et donc $U(t)=\sin(t)$ et $G(t)=y+e^y$. La solution se présente sous forme implicite
\begin{equation}
	y+e^y=\sin(t)+C.
\end{equation}
Il n'est pas possible de résoudre cette équation pour isoler $y$, donc nous devons nous contenter de cette forme. Toutefois, nous pouvons nous demander sur quel domaine cette formule définit correctement la fonction $y(t)$. Il faudrait trouver les domaines $I$ et $J$ telles que la fonction $z\mapsto z+e^z$ soit bijective entre $I$ et $J$.

Démontrons que la fonction $f\colon z\to z+e^z$ est bijective sur $\eR$. D'abord, elle est sur surjective parce que $\lim_{z\to \pm\infty} f(z)=\pm\infty$. Ensuite, elle est injective parce que sa dérivée est toujours strictement positive. Si nous notons $g$ son inverse, alors
\begin{equation}
	y(t)=g\big( \sin(t)+C \big)
\end{equation}
est la solution.

\item
$y'=y^2$. Nous avons $u(t)=1$, donc $I=\eR$ et $f(y)=y^2$, donc deux possibilités de domaines connexes où $f$ ne s'annule pas : $J_1=\eR_0^-$ et $J_2=\eR_0^+$. Nous trouvons immédiatement que
\begin{equation}
	\begin{aligned}[]
		G(y)&=-\frac{1}{ y }&&U(t)&=t,
	\end{aligned}
\end{equation}
donc
\begin{equation}		\label{EqSol105d}
	y(t)=\frac{ -1 }{ t+C }
\end{equation}
sont les solutions qui ne s'annulent pas.

Dès que $y(t_0)\neq 0$, nous avons $y(t)=-1/(t+C)$ sur un voisinage de $t_0$. Cependant, cette fonction ne s'annule jamais. Donc une fonction qui ne s'annule pas en un point ne s'annule jamais. La seule solution qui s'annule en un point est la solution identiquement nulle $y(t)=0$ pour tout $t$.

La solution \eqref{EqSol105d} est valable sur les intervalles $I_1=\mathopen]-\infty,-C\mathclose[$ et $I_2=\mathopen]-C,\infty,\mathclose[$. Une constante différente peut être choisie sur ces deux domaines, vu que nous n'avons de toutes façon pas la continuité en $t=-C$. 

La solution générale consiste à découper $\eR$ en une suite d'intervalles $I_k$ et de poser
\begin{equation}
	y(t)=\begin{cases}
	-\frac{ 1 }{ t+C_k }	\\
	\text{ou bien}\\
	0	
\end{cases}
\end{equation}
sur l'intervalle $I_k$. La seule contrainte sur le choix du découpage et des constantes $C_k$ est que $-C_k\notin I_k$. Il y a donc \emph{énormément} de solutions, si on n'impose pas d'être continue sur un intervalle maximum.

\item
$y'=y^{1/3}$. Nous avons $y'/y^{1/3}=1$, et donc, si $y(t_0)\neq 0$, alors sur un voisinage de $t_0$, la solution est donnée par $\frac{ 3 }{ 2 }y^{2/3}=t+C$,
\begin{equation}
	y(t)=\left( \frac{ 2t }{ 3 }+C \right)^{3/2}.
\end{equation}
L'étude du domaine de cette solution est intéressante. Ce domaine est a priori l'intervalle $A=\mathopen]-\frac{ 3C }{ 2 },\infty\mathclose[$. Pourquoi ne pas mettre le point $-3C/2$ dans le domaine ? Parce que $y=0$ en ce point, et nous avons dès le départ écarté les solutions avec $y=0$. 

Cependant, même si $y'$ n'est pas définit en $-3C/2$, il n'en reste pas moins que nous pouvons raccorder $y(t)$ avec la solution $y(t)\equiv 0$ en ce point :
\begin{equation}
	y(t)=\begin{cases}
	0	&					\text{si $t\leq -3C/2$}\\
	\left( \frac{ 2t }{ 3 }+C \right)^{3/2}	&	\text{si $t>-3C/2$.}
\end{cases}
\end{equation}
Cela est une solution continue de l'équation différentielle donnée, dont la dérivée n'existe pas en un seul point.

\item
Nous remettons l'équation sous la forme
\begin{equation}
	y'=-\frac{ 1+y^2 }{ y }\sin(t).
\end{equation}
Nous avons $f(y)=-\frac{ 1+y^2 }{ y }$ qui ne s'annule pas, mais qui n'est pas continue en $y=0$. Donc une fonction $y\colon \eR\to \eR^+_0$ ou $y\colon \eR\to \eR^-_0$ est solution de l'équation proposée si et seulement si
\begin{equation}
	\frac{ 1 }{2}\ln(y^2+1)=\cos(t)+C,
\end{equation}
c'est à dire
\begin{equation}		\label{EqSolGeneRacExpCis}
	y(t)=\pm\sqrt{K e^{2\cos(t)}-1}.
\end{equation}
Cela fait une double infinité de solutions : pour chaque $K\in\eR$ et pour chaque choix de $\pm$, nous avons une solution.

Ici encore, le domaine vaut le coup d'œil. Notons tout de suite que lorsque $K<0$, le domaine est vide. L'expression $ e^{2\cos(t)}$ varie entre $e^{-2}$ et $e^2$, donc
\begin{enumerate}

\item
Si $K<e^{-2}$, alors le domaine est vide.
\item
Si $K>e^2$, alors le domaine est tout $\eR$.
\item
Si $K\in\mathopen]e^{-2},e^2\mathclose[$, alors la solution n'existe pas pour tous les $t$, et le domaine est une suite d'intervalles.
\item
Si $k=e^2$, alors $y(t_0)=0$ lorsque $\cos(t_0)=-1$, mais la solution continue à exister après $t_0$, parce que l'expression sous la racine redevient strictement positive. Cependant, la dérivée de la fonction
\begin{equation}
	y(t)=\sqrt{K e^{2\cos(t)}-1}
\end{equation}
en le point $t_0$ n'existe pas  : elle vaut $-1$ à gauche et $1$ à droite (voir le calcul de la sous-section \ref{SubSecCalcLimHeuris}). Il y a cependant moyen d'écrire une solution dont la dérivée existe en faisant
\begin{equation}
	y(t)=\begin{cases}
	\sqrt{Ke^{2\cos(t)}-1}	&	\text{si $t\leq t_0$}\\
	-\sqrt{Ke^{2\cos(t)}-1}		&	 \text{si $t> t_0$.}
\end{cases}
\end{equation}
C'est à dire, en combinant plusieurs choix de signes $\pm$ à différents points de l'intervalle.

\end{enumerate}
\end{enumerate}

\end{corrige}
