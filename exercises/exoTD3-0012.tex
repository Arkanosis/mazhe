% This is part of Exercices de mathématique pour SVT
% Copyright (C) 2010
%   Laurent Claessens et Carlotta Donadello
% See the file fdl-1.3.txt for copying conditions.

\begin{exercice}\label{exoTD3-0012}

	Modèle logistique.

	Soit la suite $(u_n)_{n\in\eN}$ définie par
	\begin{equation}
		\begin{cases}
			u_{n+1}=2u_n(1-bu_n)	&	\text{$\forall n\in\eN_0$}\\
			u_0=x,
		\end{cases}
	\end{equation}
	où $b>0$ et $x\geq 0$ sont des nombres réels.
	\begin{enumerate}
		\item
			Montrer que si $x\in\{ 0,\frac{1}{ b } \}$, alors $\lim_{n\to\infty}u_n=0$.
		\item
			Montrer que si $x\in\mathopen] 0 , \frac{1}{ 2b } \mathclose]$, alors $\lim_{n\to\infty}u_n=\frac{1}{ 2b }$.
		\item
			Montrer que si $x\in\mathopen[ \frac{1}{ 2b } , \frac{1}{ b } [$, alors $\lim_{n\to\infty}u_n=\frac{1}{ 2b }$.
	\end{enumerate}

\corrref{TD3-0012}
\end{exercice}
