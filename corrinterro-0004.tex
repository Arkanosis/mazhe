% This is part of Exercices de mathématique pour SVT
% Copyright (C) 2010
%   Laurent Claessens et Carlotta Donadello
% See the file fdl-1.3.txt for copying conditions.

\begin{corrige}{interro-0004}

	Pour rappel la définition du logarithme est $\log_a(x)=y$ lorsque $a^y=x$. Encore pour rappel, nous avons $\sqrt{a}=a^{1/2}$, et $\frac{1}{ a }=a^{-1}$.

	\begin{enumerate}
		\item
			Nous aurons $\log_9(3)=y$ si $9^y=3$. Quelle valeur de $y$ ? Nous savons que $\sqrt{9}=3$, et donc $9^{1/2}=3$, c'est à dire que $\log_9(3)=\frac{ 1 }{2}$.
		\item
			Nous aurons $\log_{10}(100)=y$ si $10^y=100$. Or $10^2=100$, donc $\log_{10}(100)=2$.
		\item
			Ici, nous devons nous rendre compte que $\frac{1}{ 4 }=4^{-1}$. Donc nous restons avec la question : pour que $y$ est-ce que $2^y=4^{-1}$ ? La réponse est $y=-2$, donc $\log_2\frac{1}{ 4 }=-2$.

	\end{enumerate}

\end{corrige}
