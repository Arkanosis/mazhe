% This is part of the Exercices et corrigés de mathématique générale.
% Copyright (C) 2009, 2012
%   Laurent Claessens
% See the file fdl-1.3.txt for copying conditions.
\begin{corrige}{Lineraire0040}

	\begin{enumerate}

		\item
			
            %TODO : refaire le dessin
			%Il s'agit de la symétrie autour de l'axe vert de la figure \ref{LabelFigSymQC}.
            
            Le vecteur qui est sur cet axe ne bouge pas, tandis que celui qui est perpendiculaire change de signe. Ces deux vecteurs sont donc des vecteurs propres, de valeurs propres $1$ et $-1$ respectivement.
			%\newcommand{\CaptionFigSymQC}{Sous la symétrie par rapport à l'axe, le vecteur sur l'axe ne bouge pas, et celui perpendiculaire change de signe.}
			%\input{Fig_SymQC.pstricks}

		\item
			Dans une rotation, aucune direction n'est conservée, sauf cas exceptionnels comme $\alpha=2\pi$ (où rien ne bouge) ou bien $\alpha=\pi$ qui renverse tout.

		\item
			Tous les vecteurs sont renversés. En particulier $Ae_1=-e_1$ et $Ae_2=-e_2$. Donc la matrice est diagonale dans la base canonique, avec $-1$ comme valeur propre double.

	\end{enumerate}
	


\end{corrige}
