%+++++++++++++++++++++++++++++++++++++++++++++++++++++++++++++++++++++++++++++++++++++++++++++++++++++++++++++++++++++++++++
\section{Le théorème de Green}
%+++++++++++++++++++++++++++++++++++++++++++++++++++++++++++++++++++++++++++++++++++++++++++++++++++++++++++++++++++++++++++

Soit un champ de vecteurs
\begin{equation}
    F(x,y,z)=\begin{pmatrix}
        F_1(x,y,z)    \\ 
        F_2(x,y,z)    \\ 
        F_2(x,y,z)    
    \end{pmatrix}
\end{equation}
et un chemin $\sigma\colon \mathopen[ a , b \mathclose]\to \eR^3$ donné par
\begin{equation}
    \sigma(t)=\begin{pmatrix}
        x(t)    \\ 
        y(t)    \\ 
        z(t)    
    \end{pmatrix}.
\end{equation}
Nous avons défini la circulation de $F$ le long de $\sigma$ par
\begin{equation}
    \begin{aligned}[]
        \int_{\sigma}F\cdot d\sigma&=\int_a^bF\big( \sigma(t) \big)\cdot\sigma'(t)dt\\
        &=\int_a^b\Big[ F_1\big( \sigma(t) \big)x'(t)+F_2\big( \sigma(t) \big)y'(y)+F_3\big( \sigma(t) \big)z'(t)\Big]dt\\
        &=\int_{\sigma} F_1dx +F_2dy+F_3dz.
    \end{aligned}
\end{equation}
La dernière ligne est juste une notation compacte\footnote{Il y aurai beaucoup de choses à dire là-dessus, mais la vie est trop courte pour parler de formes différentielles, et c'est dommage.}. Elle sert à se souvenir qu'on va mettre $x'$ à côté de $F_1$, $y'$ à côté de $F_2$ et $z'$ à côté de $F_3$. L'avantage de cette notation est qu'on peut écrire d'autres combinaisons.

Si $f$ et $g$ sont deux fonctions sur $\eR^3$, nous pouvons écrire
\begin{equation}
    \int_{\sigma} fdy+gdz.
\end{equation}
Cela signifie
\begin{equation}
    \int_a^b \Big[ f\big( \sigma(t) \big)y'(t)+g\big( \sigma(t) \big)z'(t)\Big]dt.
\end{equation}

Soit $D$ une région du plan et $\sigma$, son contour que nous prenons, par convention\footnote{Il y aurait beaucoup de choses à dire sur ça aussi, mais\ldots}, dans l'orientation trigonométrique, comme indiqué sur la figure \ref{LabelFigContourGreen}. Nous supposons également que le domaine $D$ n'a pas de trous intérieurs.
\newcommand{\CaptionFigContourGreen}{Un contour avec son orientation.}
\input{Fig_ContourGreen.pstricks}

Nous notons par $\sigma=\partial D$ le bord de $D$, c'est à dire le contour dont nous venons de parler.

\begin{theorem}[Théorème de Green]
    Soient $P,Q\colon D\to \eR$ deux fonctions de classe $C^1$. Alors
    \begin{equation}        \label{EqThoGreen}
        \int_{\partial D} Pdx+Qdy=\iint_D\left( \frac{ \partial Q }{ \partial x }-\frac{ \partial P }{ \partial y } \right)dxdy.
    \end{equation}
\end{theorem}
Pour rappel, l'intégrale du membre de gauche signifie
\begin{equation}
    \int_a^b \Big[P\big( \sigma(t) \big)\sigma_x'(t)+Q\big( \sigma(t) \big)\sigma_y'(t)\Big]dt.
\end{equation}
Ce n'est d'ailleurs rien d'autre que l'intégrale du champ de vecteurs $\begin{pmatrix}
    P    \\ 
    Q    
\end{pmatrix}$.

\begin{corollary}
    L'aire du domaine $D$ est donnée par
    \begin{equation}
        A=\frac{ 1 }{2}\int_{\partial D}(xdy-ydx).
    \end{equation}
\end{corollary}

\begin{proof}
    L'intégrale $\int_{\partial D}(xdy-ydx)$ se traite avec le théorème de Green où l'on pose $P=-y$ et $Q=x$. Nous avons donc
    \begin{equation}
        \begin{aligned}[]
            \int_{\partial D} -ydx+xdy&=\iint_D\left( \frac{ \partial x }{ \partial x }-\frac{ \partial (-y) }{ \partial y } \right)dxdy\\
            &=\iint_D2\,dxdy.
        \end{aligned}
    \end{equation}
    La dernière ligne est bien le double de la surface.
\end{proof}

\begin{example}
    Calculons (encore une fois) l'aire du disque de rayon $R$. Il s'agit de calculer l'intégrale
    \begin{equation}
        I=\frac{ 1 }{2}\int_{\sigma}(xdt-ydx)
    \end{equation}
    où $\sigma$ est le cercle donné par
    \begin{equation}
        \sigma(t)=\begin{pmatrix}
            x(t)    \\ 
            y(t)    
        \end{pmatrix}=\begin{pmatrix}
            R\cos(t)    \\ 
            R\sin(t)    
        \end{pmatrix}
    \end{equation}
    Le calcul est
    \begin{equation}
        \begin{aligned}[]
            I&=\frac{ 1 }{2}\int_{0}^{2\pi} \underbrace{R\cos(\theta)}_{x}\underbrace{R\cos(\theta)}_{y'}-\underbrace{R\sin(\theta)}_{y}\underbrace{(-R\sin(\theta))}_{x'}\,d\theta\\
            &=\frac{ R^2 }{2}\int_{0}^{\pi}d\theta\\
            &=\pi R^2.
        \end{aligned}
    \end{equation}
\end{example}

\begin{example}
    Calculons l'aire de l'ellipse 
    \begin{equation}
        \frac{ x^2 }{ a^2 }+\frac{ y^2 }{ b^2 }\leq 1
    \end{equation}
    dont le bord est donné par
    \begin{subequations}
        \begin{numcases}{}
            x(t)=a\cos(t)\\
            y(t)=b\sin(t).
        \end{numcases}
    \end{subequations}
    Le terme $xdy$ devient $a\cos(t)b\cos(t)=ab\cos^2(t)$ et le terme $ydx$ devient $b\sin(t)(-a\sin(t))=-ab\sin^2(t)$. L'intégrale qui donne la surface est donc
    \begin{equation}
        \frac{ 1 }{2}\int_{\partial D}(xdy-ydx)=\frac{ 1 }{2}\int_0^{2\pi}ab=\pi ab.
    \end{equation}
\end{example}


Le théorème de Green peut être mis sous une autre forme.

\begin{theorem}[Théorème de Green, forme vectorielle]       \label{ThoGreenVecto}
    Si $G$ est un champ de vecteurs sur $D$, nous avons
    \begin{equation}        \label{EqGreenVecto}
        \int_{\partial D}G\cdot d\sigma=\iint_D(\nabla\times G)\cdot dS
    \end{equation}
    où le second membre est le flux de $\nabla\times G$ sur la surface $D$.
\end{theorem}

\begin{proof}
    Analysons le membre de droite. Nous savons que $D$ est une surface dans le plan $\eR^2$. Le vecteur normal à la surface est donc simplement le vecteur (constant) $e_z$. Le produit scalaire $(\nabla\times F)\cdot dS$ est donc $(\nabla\times F)\cdot e_z$ et se réduit à la troisième composante du rotationnel, c'est à dire
    \begin{equation}
        \frac{ \partial F_2 }{ \partial x }-\frac{ \partial F_1 }{ \partial y }.
    \end{equation}
    Cela est bien le membre de droite de l'équation \eqref{EqThoGreen}. Le membre de gauche de cette dernière est bien le membre de gauche de \eqref{EqGreenVecto}.
\end{proof}

\begin{example}     \label{ExempleGreenSqL}
    Soit le champ de vecteurs $F(x,y)=\begin{pmatrix}
        xy^2    \\ 
        y+x    
    \end{pmatrix}$, et soit à calculer
    \begin{equation}
        \iint_D\nabla\times F\cdot dS
    \end{equation}
    où $D$ est la région comprise entre les courbes $y=x^2$ et $y=x$ pour $x\geq 0$ (voir la figure \ref{LabelFigContourSqL}).
    \newcommand{\CaptionFigContourSqL}{Le contour d'intégration pour l'exemple \ref{ExempleGreenSqL}.}
    \input{Fig_ContourSqL.pstricks}

    Nous pouvons calculer cette intégrale directement en calculant le rotationnel de $F$:
    \begin{equation}
        \nabla\times F=\begin{pmatrix}
            0    \\ 
            0    \\ 
            1-2xy    
        \end{pmatrix}.
    \end{equation}
    Par conséquent l'intégrale à effectuer est
    \begin{equation}
        I=\int_0^1 dx\int_{x^2}^x(1-2xy)dy=\frac{1}{ 12 }.
    \end{equation}
    \begin{verbatim}
 ----------------------------------------------------------------------
| Sage Version 4.6.1, Release Date: 2011-01-11                       |
| Type notebook() for the GUI, and license() for information.        |
----------------------------------------------------------------------
sage: f(x,y)=1-2*x*y
sage: f.integrate(y,x**2,x).integrate(x,0,1)
(x, y) |--> 1/12
    \end{verbatim}
    
    L'autre façon de calculer l'intégrale est d'utiliser le théorème de Green et de calculer la circulation de $F$ le long de $\partial D$ :
    \begin{equation}
        I=\int_{\partial D}F\cdot \sigma.
    \end{equation}
    Le chemin $\sigma=\partial D$ est composé de la parabole $y=x^2$ et du segment de droite $x=y$. Attention : il faut respecter l'orientation. Nous avons
    \begin{equation}
        \sigma_1(t)=(t,t^2)
    \end{equation}
    et
    \begin{equation}
        \sigma_2(t)=(1-t,1-t).
    \end{equation}
    Notez bien que le second chemin est $(1-t,1-t)$ et non $(t,t)$ parce qu'il faut le parcourir dans le bon sens (voir le dessin).

    Commençons par le premier chemin :
    \begin{equation}
        \begin{aligned}[]
            \sigma_1(t)&=(t,t^2)\\
            \sigma_1'(t)&=(1,2t)\\
            F\big( \sigma_1(t) \big)&=\begin{pmatrix}
                t^5    \\ 
                t+t^2    
            \end{pmatrix},
        \end{aligned}
    \end{equation}
    et par conséquent
    \begin{equation}
        F\big( \sigma_1(t) \big)\cdot \sigma_1'(t)=t^5+2t^2+2t^3,
    \end{equation}
    et le premier morceau de la circulation vaut
    \begin{equation}
        \int_{\sigma_1} F\cdot d\sigma_1=\int_0^1 t^5+2t^2+2t^3=\frac{ 4 }{ 3 }.
    \end{equation}
    
    Pour le second chemin :
    \begin{equation}
        \begin{aligned}[]
            \sigma_2(t)=(1-t,1-t)\\
            \sigma_2'(t)=(-1,-1)\\
            F\big( \sigma_2(t) \big)=\begin{pmatrix}
                (1-t)^3    \\ 
                2(1-t)    
            \end{pmatrix}.
        \end{aligned}
    \end{equation}
    Par conséquent
    \begin{equation}
        F\big( \sigma_2(t) \big)\cdot \sigma_2(t)=-(1-t)^2-2(1-t).
    \end{equation}
    Le second morceau de la circulation est par conséquent
    \begin{equation}
        \int_0^1-(1-t)^2-2(1-t)dt=-\frac{ 5 }{ 4 }.
    \end{equation}
    La circulation de $F$ le long de $\sigma$ est donc égale à
    \begin{equation}
        \frac{ 4 }{ 3 }-\frac{ 5 }{ 4 }=\frac{1}{ 12 }.
    \end{equation}
    Comme prévu, nous obtenons le même résultat.
\end{example}


%+++++++++++++++++++++++++++++++++++++++++++++++++++++++++++++++++++++++++++++++++++++++++++++++++++++++++++++++++++++++++++
\section{Théorème de la divergence dans le plan}
%+++++++++++++++++++++++++++++++++++++++++++++++++++++++++++++++++++++++++++++++++++++++++++++++++++++++++++++++++++++++++++

%---------------------------------------------------------------------------------------------------------------------------
\subsection{La convention de sens de parcours}
%---------------------------------------------------------------------------------------------------------------------------

Soient $D$, un domaine dans le plan et une paramétrisation
\begin{equation}
    \begin{aligned}
        \sigma\colon \mathopen[ a , b \mathclose]&\to \eR^2 \\
        t&\mapsto \begin{pmatrix}
            x(t)    \\ 
            y(t)    
        \end{pmatrix},
    \end{aligned}
\end{equation}
une paramétrisation du bord $\partial D$ de $D$. La normale à $\sigma$ est perpendiculaire à la tangente, donc la normale extérieure de norme $1$ vaut
\begin{equation}
    \begin{aligned}[]
        n&=\frac{ \big( y'(t),-x'(t) \big) }{ \sqrt{ \big( x'(t)\big)^2+\big( y'(t) \big)^2  } }&\text{ou}&&n&-=\frac{ \big( y'(t),-x'(t) \big) }{ \sqrt{ \big( x'(t)\big)^2+\big( y'(t) \big)^2  } }.
    \end{aligned}
\end{equation}
Comment faire le choix ?

Nous prenons comme convention que le sens \emph{du chemin} doit être tel que le vecteur normal extérieur soit
\begin{equation}
        n=\frac{ \big( y'(t),-x'(t) \big) }{ \sqrt{ \big( x'(t)\big)^2+\big( y'(t) \big)^2  } }.
\end{equation}
Donc si le chemin $\sigma$ donne lieu à un vecteur $n$ pointant vers l'intérieur, il faut utiliser le chemin qui va dans le sens contraire : $\tilde \sigma(t)=\sigma(1-t)$.

Les vecteurs tangents et normaux d'un contour sont dessinés sur la figure \ref{LabelFigContourTgNDivergence}.
\newcommand{\CaptionFigContourTgNDivergence}{Le champ de vecteurst tangents est dessiné en rouge tandis qu'en vert nous avons le champ de vecteurs normaux extérieurs.}
\input{Fig_ContourTgNDivergence.pstricks}

%---------------------------------------------------------------------------------------------------------------------------
\subsection{Théorème de la divergence}
%---------------------------------------------------------------------------------------------------------------------------

\begin{theorem}[Théorème de la divergence]
    Soit $F$ un champ de vecteurs sur $\eR^2$. Le flux de $F$ à travers le bord de $D$ est égal à l'intégrale de la divergence de $F$ sur $D$. En formule :
    \begin{equation}
        \int_{\partial D} F\cdot n\,d\sigma=\iint_D\nabla\cdot F\,dxdy.
    \end{equation}
\end{theorem}

\begin{remark}
    Tant $F\cdot n$ que $\nabla\times F$ sont des fonctions. Le membre de gauche est donc l'intégrale d'une fonction sur un chemin et le membre de droite est l'intégrale d'une fonction sur une surface.
\end{remark}

\begin{proof}
    Notre convention de sens de parcours du chemin permet d'écrire le produit scalaire $F\cdot n$ sous la forme suivante :
    \begin{equation}
        \begin{aligned}[]
            F\cdot n&=\frac{1}{ \| \sigma' \| }\begin{pmatrix}
                F_x    \\ 
                F_y    
            \end{pmatrix}\cdot \begin{pmatrix}
                y'    \\ 
                -x'    
            \end{pmatrix}\\
            &=\frac{1}{ \| \sigma' \| }(F_xy'-F_yx')\\
            &=\frac{1}{ \| \sigma' \| }\begin{pmatrix}
                -F_y    \\ 
                F_x    
            \end{pmatrix}\cdot \begin{pmatrix}
                x'    \\ 
                y'    
            \end{pmatrix}\\
            &=\frac{1}{ \| \sigma' \| }\begin{pmatrix}
                -F_y    \\ 
                F_x    
            \end{pmatrix}\cdot \sigma'.
        \end{aligned}
    \end{equation}

    Par conséquent, la \emph{fonction}
    \begin{equation}
        F\cdot n
    \end{equation}
    est la même que la \emph{fonction} 
    \begin{equation}
        \frac{1}{ \| \sigma' \| }\begin{pmatrix}
            -F_y    \\ 
            F_x    
        \end{pmatrix}\cdot \sigma'.
    \end{equation}
    L'intégrale de cette dernière fonction sur le chemin $\sigma$ est 
    \begin{equation}
        \begin{aligned}[]
            I&=\int_{\sigma} F\cdot n\\
            &=\int_{\sigma}\frac{1}{ \| \sigma' \| }\begin{pmatrix}
                -F_y    \\ 
                F_x    
            \end{pmatrix}\cdot \sigma'\\
            &=  \int_a^b\frac{1}{ \| \sigma'(t)\| }\begin{pmatrix}
                -F_y\big( \sigma(t) \big)    \\ 
                F_x\big( \sigma(t) \big)
            \end{pmatrix}
            \cdot\sigma'(t)\| \sigma'(t) \|dt\\
            &=
            \int_a^b\begin{pmatrix}
                -F_y    \\ 
                F_x    
            \end{pmatrix}\cdot \sigma'(t)dt.
        \end{aligned}
    \end{equation}
    Cette dernière intégrale est la circulation du champ de vecteurs $\begin{pmatrix}
        -F_y    \\ 
        F_x    
    \end{pmatrix}$ sur le chemin $\sigma$. Le théorème de Green \ref{ThoGreenVecto} nous enseigne que la circulation le long d'un chemin est égale au flux du rotationnel à travers la surface. Par conséquent,
    \begin{equation}
        I=\iint_D\left( \nabla\times\begin{pmatrix}
            -F_y    \\ 
            F_x    
        \end{pmatrix}\right)\cdot dS=\iint_D\nabla\cdot F\, dxdy
    \end{equation}
    

\end{proof}

%+++++++++++++++++++++++++++++++++++++++++++++++++++++++++++++++++++++++++++++++++++++++++++++++++++++++++++++++++++++++++++
\section{Théorème de Stockes}
%+++++++++++++++++++++++++++++++++++++++++++++++++++++++++++++++++++++++++++++++++++++++++++++++++++++++++++++++++++++++++++

Nous nous mettons maintenant dans $\eR^3$, et nous y considérons une surface paramétrée $S$ donc le bord est $\partial S$. 

\begin{theorem}[Théorème de Stockes]
    Alors le flux du rotationnel de $F$ à travers $S$ est égal à la circulation de $F$ le long du bord. En formule :
    \begin{equation}
        \iint_S\nabla\times F\cdot dS=\int_{\partial S} F\cdot d\sigma.
    \end{equation}
\end{theorem}

Nous pouvons nous donner une idée du pourquoi ce théorème est vrai. D'abord, si la surface est plate, cela est exactement le théorème de Green \ref{ThoGreenVecto}. Supposons maintenant que le bord reste plat, mais que la surface se déforme un petit peu. Le chemin
\begin{equation}
    \sigma(t)=\begin{pmatrix}
        \cos(t)    \\ 
        \sin(t)    \\ 
        0    
    \end{pmatrix}
\end{equation}
est tout autant le bord du disque plat de rayon $1$ que celui de la demi-sphère
\begin{equation}
    \phi(x,y)=\begin{pmatrix}
        x    \\ 
        y    \\ 
        \sqrt{1-x^2-y^2}    
    \end{pmatrix}.
\end{equation}
Le champ de vecteur que nous considérons est $G=\nabla\times F$. Il a un certain flux à travers le disque plat, et ce plus est égal à la circulation de $F$ sur $\sigma$. Quel est le flux de $G$ à travers la demi-sphère ? Étant donné que $\nabla\cdot G=\nabla\cdot(\nabla\times F)=0$, le champ de vecteurs $G$ est incompressible, de telle façon à ce que tout ce qui rentre dans la demi-sphère doit en sortir. Le flux de $G$ à travers la demi-sphère doit par conséquent être égal à celui à travers le disque plat.


\begin{example}

    Soit $C$ l'intersection entre le cylindre $x^2+y^2=1$ et le plan $x+y+z=1$. Calculer la circulation de
    \begin{equation}
        F(x,y,z)=\begin{pmatrix}
            -y^3    \\ 
            x^3    \\ 
            -z^3    
        \end{pmatrix}
    \end{equation}
    le long de $C$. 

    Au lieu de calculer directement
    \begin{equation}
        \int_{C}F\cdot d\sigma,
    \end{equation}
    nous allons calculer
    \begin{equation}
        \int_S\nabla\times F\cdot dS
    \end{equation}
    où $S$ est une surface dont $C$ est le bord. Cette intégrale est à calculer avec la formule \eqref{EqResIntFluxPhi}.

    La première chose à faire est de trouver une surface dont le bord est $C$ et en trouver une paramétrisation $\phi$. Le plus simple est de prendre le graphe du plan sur le cercle $x^2+y^2+1$. Une paramétrisation de cette surface est simplement
    \begin{equation}
        \begin{aligned}
            \phi\colon D&\to \eR^3 \\
            (x,y)&\mapsto \begin{pmatrix}
                x    \\ 
                y    \\ 
                1-x-y    
            \end{pmatrix}
        \end{aligned}
    \end{equation}
    où $D$ est le disque de rayon $1$. Étant donné que cela paramètre le plan $x+y+z-1=0$, le vecteur normal est $n=e_x+e_y+z_z$. Nous pouvons cependant calculer ce vecteur normal en suivant la recette usuelle. D'abord les vecteurs tangents sont
    \begin{equation}
        \begin{aligned}[]
            \frac{ \partial \phi }{ \partial x }&=\begin{pmatrix}
                1    \\ 
                0    \\ 
                -1    
            \end{pmatrix},
            &\frac{ \partial \phi }{ \partial y }&=\begin{pmatrix}
                0    \\ 
                1    \\ 
                -1    
            \end{pmatrix}.
        \end{aligned}
    \end{equation}
    Et le vecteur normal est donné par le produit vectoriel :
    \begin{equation}
        \begin{aligned}[]
            n&=\frac{ \partial \phi }{ \partial x }\times\frac{ \partial \phi }{ \partial y }\\
            &=\begin{vmatrix}
                e_x    &   e_y    &   e_z    \\
                1    &   0    &   -1    \\
                0    &   1    &   -1
            \end{vmatrix}\\
            &=e_x+e_y+z_z.
        \end{aligned}
    \end{equation}

    Ensuite, le rotationnel de $F$ est donné par
    \begin{equation}
        \nabla\times F=3(x^2+y^2)e_z.
    \end{equation}
    Par conséquent,
    \begin{equation}
        \nabla\times F\cdot\left( \frac{ \partial \phi }{ \partial x }\times\frac{ \partial \phi }{ \partial y } \right)=3(x^2+y^2).
    \end{equation}
    L'intégrale à calculer est donc
    \begin{equation}
        \begin{aligned}[]
            \iint_S\nabla\times F\cdot dS&=\iint_D(\nabla\times F)\big( \phi(x,y) \big)\cdot\left( \frac{ \partial \phi }{ \partial x }\times\frac{ \partial \phi }{ \partial y } \right)dxdy\\
            &=3\int_D(x^2+y^2)dxdy.
        \end{aligned}
    \end{equation}
    Cette dernière intégrale est l'intégrale d'une fonction sur le disque de rayon $1$. Elle s'effectue en passant aux coordonnées polaires :
    \begin{equation}
        3\int_D(x^2+y^2)dxdy=\int_0^{2\pi}d\theta\int_0^1(r^2)r\,dr=\frac{ 3\pi }{2}.
    \end{equation}
\end{example}

%+++++++++++++++++++++++++++++++++++++++++++++++++++++++++++++++++++++++++++++++++++++++++++++++++++++++++++++++++++++++++++
\section{Théorème de Gauss}
%+++++++++++++++++++++++++++++++++++++++++++++++++++++++++++++++++++++++++++++++++++++++++++++++++++++++++++++++++++++++++++

Soit $V$ une partie de $\eR^3$ délimitée par une surface $S$ sur laquelle nous considérons la normale extérieure. Soit $F$ un champ de vecteurs sur $\eR^3$.

\begin{theorem}[Théorème de la divergence ou de Gauss]
    Le flux d'un champ de vecteur $F$ à travers une surface fermée est égale à l'intégrale de la divergence sur le volume correspondant :
    \begin{equation}
        \int_{\partial V} F\cdot dS=\iiint_V\nabla\cdot F\,dxdydz.
    \end{equation}
\end{theorem}

Ce théorème signifie que la quantité de fluide qui s'accumule dans le volume (le flux est ce qui rentre moins ce qui sort) est égal à l'intégrale de $\nabla\cdot F$ sur le volume, alors que nous savons que, localement, la quantité $\nabla\cdot F(x,y,z)$ est la quantité de fluide qui s'accumule au point $(x,y,z)$.

\begin{remark}
    Ce théorème ne fonctionne qu'avec des surfaces fermées. Essayer de l'appliquer au calcul de flux à travers des surfaces ouvertes n'a pas de sens parce qu'une surface ouverte ne délimite pas un volume.
\end{remark}

\begin{example}
    Calculer le flux du champ de vecteurs
    \begin{equation}
        F(x,y,z)=\begin{pmatrix}
            2x    \\ 
            y^2    \\ 
            z^2    
        \end{pmatrix}
    \end{equation}
    à travers la sphère de rayon $1$ centrée à l'origine. Nous utilisons le théorème de la divergence
    \begin{equation}
        \iint_S F\cdot n\,dS=\iiint_B\nabla \cdot F\,dxdydz
    \end{equation}
    où $S$ est la sphère et $B$ est la boule (la sphère pleine). La divergence de $F$ se calcule :
    \begin{equation}
        \nabla\cdot F=\frac{ \partial F_x }{ \partial x }+\frac{ \partial F_y }{ \partial y }+\frac{ \partial F_z }{ \partial z }=2+2x+2y.
    \end{equation}
    L'intégrale est donc en trois termes :
    \begin{equation}
        \begin{aligned}[]
            \iiint_B2=2\text{Volume(B)}=\frac{ 8\pi }{ 3 }\\
            \iiint_By\,dxdydz=0\\
            \iiint_Bz\,dxdydz=0.
        \end{aligned}
    \end{equation}
\end{example}

Dans certains cas le théorème de Gauss permet de simplifier le calcul de l'intégrale d'une fonction sur une surface.

\begin{example}
    Soit à calculer l'intégrale
    \begin{equation}
        I=\iint_{\partial B}(x^2+y+z)dS,
    \end{equation}
    c'est à dire l'intégrale de la fonction $x^2+y+z$ sur la sphère. Le vecteur normal à la sphère est
    \begin{equation}
        n=xe_x+ye_y+ze_z.
    \end{equation}
    Étant donné que nous sommes sur la sphère de rayon $1$, ce vecteur est même normé. La fonction que nous regardons n'est rien d'autre que $F\cdot n$ avec
    \begin{equation}
        F=\begin{pmatrix}
            x    \\ 
            1    \\ 
            1    
        \end{pmatrix}.
    \end{equation}
    Nous pouvons donc simplement intégrer $\nabla\cdot F$ sur toute la boule :
    \begin{equation}
        I=\iiint_{B}\nabla\cdot F\,dxdydz=\iiint_B 1\,dxdudz=\frac{ 4\pi }{ 3 }.
    \end{equation}
\end{example}
