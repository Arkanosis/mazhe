\begin{corrige}{CalculDifferentiel0021}

	\begin{enumerate}
		\item
			Cette fonction est différentiable partout. D'abord les dérivées partielles existent et sont continues sur $\eR^2\setminus\{ (0,0) \}$ parce que la fonction est une composée de fonctions dérivables et continues. En ce qui concerne les dérivées partielles en $(0,0)$, nous avons déjà mentionné qu'elle n'étaient pas continues. Le nombre $\partial_xf(0,0)$ existe pourtant :
			\begin{equation}
				\partial_xf(0,0)=\lim_{x\to 0} \frac{ f(x,0)-f(0,0) }{ | x | }=\lim_{x\to 0} | x |\sin\left( \frac{1}{ x^2 } \right)=0.
			\end{equation}
			Attention : le théorème \ref{Diff_totale} ne permet pas de conclure. 

			La dérivée partielle par rapport à $y$ se calcule de la même façon et nous avons $\partial_yf(0,0)=0$. Affin de savoir si la fonction est différentiable, nous testons directement avec la définition. Si la fonction est différentiable en $(0,0)$, la différentielle est zéro (parce que les dérivées partielles sont nulles), donc nous calculons
			\begin{equation}
				\lim_{(x,y)\to(0,0)}\frac{ \| f(x,y)-T(x,y) \| }{ \| (x,y) \| }
			\end{equation}
			avec $T(x,y)=0$. Le calcul est
			\begin{equation}
				\lim_{(x,y)\to(0,0)}(x^2+y^2)^{1/2}\sin\left( \frac{1}{ x^2+y^2 } \right)=\lim_{r\to 0} r\sin\left( \frac{1}{ r^2 } \right)=0
			\end{equation}
			La différentielle $df(0,0)$ est donc bien l'application nulle.

		\item
			Nous avons vu autour de l'équation \eqref{EqcddzuiiDifcy} que $\partial_xf(0,0)=1$ et $\partial_yf(0,0)=-1$. Par conséquent la différentielle de $f$ en $(0,0)$, si elle existe, vaut $T(x,y)=0$. Afin de voir si cela est bien la différentielle, nous calculons
			\begin{equation}
				\begin{aligned}[]
					\lim_{(x,y)\to(0,0)}\frac{ \frac{ x^3-y^3 }{ x^2-y^2 }-(x-y) }{ \sqrt{x^2+y^2} }&=\lim_{(x,y)\to(0,0)}\frac{ -xy^2+yx^2 }{ (x^2+y^2)^{3/2} }\\
					&=\lim_{r\to 0} r^3\frac{ \cos\theta\sin\theta(\cos\theta-\sin\theta) }{ r^3 }.
				\end{aligned}
			\end{equation}
			Cette dernière limite étant dépendante de $\theta$, nous en déduisons que la limite qui définit la différentielle n'existe pas. La fonction n'est donc pas différentiable en $(0,0)$.
	\end{enumerate}
	

\end{corrige}
