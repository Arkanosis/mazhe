% This is part of Exercices et corrigés de CdI-1
% Copyright (c) 2011
%   Laurent Claessens
% See the file fdl-1.3.txt for copying conditions.

\begin{exercice}\label{exoOutilsMath-0097}

    \begin{enumerate}
        \item
            Montrer que le champ de vecteurs sur $\eR^2$
            \begin{equation}
                F(x,y)=\begin{pmatrix}
                    6xy+4x^3y^4    \\ 
                    3x^2+4x^4y^3    
                \end{pmatrix}
            \end{equation}
            dérive d'un potentiel scalaire. Donner un tel potentiel.
        \item
            Quel est le rotationnel de $F$ ?
        \item
            Montrer que le champ de vecteurs sur $\eR^2$
            \begin{equation}
                F(x,y)=\begin{pmatrix}
                    y\sin(xy)    \\ 
                    -x\sin(xy)    
                \end{pmatrix}
            \end{equation}
            n'est pas un champ de gradients, c'est à dire qu'il ne dérive pas d'un potentiel scalaire.
    \end{enumerate}

\corrref{OutilsMath-0097}
\end{exercice}
