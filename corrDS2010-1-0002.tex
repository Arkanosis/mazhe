% This is part of Exercices de mathématique pour SVT
% Copyright (C) 2010,2014
%   Laurent Claessens et Carlotta Donadello
% See the file fdl-1.3.txt for copying conditions.

\begin{corrige}{DS2010-1-0002}

	Chaque graphique a sa petite particularité qui lui permet de se distinguer des autres.
	\begin{enumerate}
		\item
			Le graphique \ref{LabelFigExercicesslabellogarithme} est le graphique bien reconnaissable de $\ln(x)$ : en fait le domaine de cette fonction est $]0, +\infty[$ et $\ln (1)=0$.
		\item
			Le graphique \ref{LabelFigExercicesslabellnvalabsolue} est le même que le logarithme, mais symétrique par rapport à l'axe vertical. C'est donc $\ln(| x |)$.
		\item
			Le graphique \ref{LabelFigExercicesslabellnxplusun} est le même que $\ln(x)$, mais décalé de $1$ vers la gauche (il tend vers $-\infty$ en $-1$ au lieu de zéro). C'est donc le $\ln(x+1)$.
		\item
			Le graphique \ref{LabelFigExercicesslabelvalabsolueln} est le logarithme décalé de $1$ vers le haut, c'est donc $\ln(x)+1$.
		\item
			Le graphique \ref{LabelFigExercicesslabelunplusln} est $| \ln(x) |$ parce que c'est le même que le logarithme pour $x\geq 1$, et il est le symétrique par rapport à l'axe horizontal de $\ln(x)$ là où $\ln(x)$ est négatif.
		\item
			Le graphique \ref{LabelFigExercicesslabelsqrtln} est $\sqrt{\ln(x)}$. La façon de le voir est de remarquer que le graphe commence en $1$. Il n'existe pas avant $x=1$, ce qui signifie qu'il n'existe pas là où $\ln(x)$ est négatif.
		\item
			Le graphique \ref{LabelFigExercicesslabellnsqrt} est $\ln(x^2)$ parce qu'il est symétrique, mais plus pentu que $\ln(| x |)$.
			
	\end{enumerate}
	Notez que $\ln(x^2)$ et $\ln(| x |)$ se ressemblent parce que tant le carré que la valeur absolue on pour effet de «rendre positif les nombres négatifs». Cependant comme $x^2$ avance plus vite que $x$, c'est lui qui est le plus penché.

\end{corrige}


The result is on figure \ref{LabelFigACUooQwcDMZ}. % From file ACUooQwcDMZ
\newcommand{\CaptionFigACUooQwcDMZ}{<+Type your caption here+>}
\input{Fig_ACUooQwcDMZ.pstricks}
See also the subfigure \ref{LabelFigACUooQwcDMZssLabelSubFigACUooQwcDMZ0}
See also the subfigure \ref{LabelFigACUooQwcDMZssLabelSubFigACUooQwcDMZ1}
See also the subfigure \ref{LabelFigACUooQwcDMZssLabelSubFigACUooQwcDMZ2}
See also the subfigure \ref{LabelFigACUooQwcDMZssLabelSubFigACUooQwcDMZ3}
See also the subfigure \ref{LabelFigACUooQwcDMZssLabelSubFigACUooQwcDMZ4}
See also the subfigure \ref{LabelFigACUooQwcDMZssLabelSubFigACUooQwcDMZ5}
See also the subfigure \ref{LabelFigACUooQwcDMZssLabelSubFigACUooQwcDMZ6}

