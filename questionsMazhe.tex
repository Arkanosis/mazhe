% This is part of (almost) Everything I know in mathematics
% Copyright (c) 2014
%   Laurent Claessens
% See the file fdl-1.3.txt for copying conditions.

\newcommand{\refprob}[1]{\ref{#1}, page \pageref{#1}}

\section{Still some questions and problems}


\begin{enumerate}
\item \refprob{ProbNablades} Où vient le $\nabla$ dans la définition de WKB ?
 \item \refprob{ProbFibra} à voir que quand on a une courte suite exacte d'espaces qui s'injèctent les uns dans les autres, on a une longue suite exacte dans les espaces de cohomologie.
\item  \refprob{ProbAdJthetaj} En réalité, c'est sans doute l'avant-dernière ligne de la preuve du théorème 3.1 dans Semisimple Symplectic Symmetric spaces. Ou bien (i.2) de la proposition 4.1
\item \refprob{propCrtadeux} Encore le problème de $\theta$ comme automorphisme interne.

\item \refprob{ProbWeylMoy} Une référence pour l'équivalence entre Moyal et Weyl
\item \refprob{ProbEnonSLdef} Un énoncé exact du produit sur $\SL(2,\eR)$.
\item \refprob{Probintertw} Prouver que $\mT_{\theta}$ entrelace réellement $\rho_{\nu}$ et $dL$.

    \item
        A friend said me that the condition \eqref{eq:symple_Lie} means that the form is closed in the sense of Chevalley and thus its left invariant prolongation is closed in the sense of de Rahm.
    \item

	L’espace symetrique $M$ est vu comme $G/H$ ou $G =$ composante connexe à l’identite de $Aff(M) =$ larger connex group of affine transformation (théorème \ref{ThoGplugdSymssgpAff}), ou $Aff(M)=$groupe des transformations affines de $M$.

A-t-on forcement : composante connexe à l’identite de $Aff(M)  =$ larger connex group of $Aff(M)$ ?
(genre, pourrait-on avoir un ss-groupe connexe qui ne passe pas par l’identité, plus grand que tous ceux qui passent par l’identité ?) 

On sait que toutes les symetries $s_x$ sont des transformations affines, mais le contraire n’est pas vrai ; en particulier, $Aff(M)$  peut contenir des elements qui ne sont pas des symetries.  Ceci pour savoir la definition du theoreme  \ref{ThoGplugdSymssgpAff}, etait la meme que celle du theoreme 2.4 du papier de Bieliavsky, où il exprime $M=G/H$, avec cette fois $G =$ transvection group de $M$, qui est INCLUS dans $Aff(M)$. 

Dans ton theoreme \ref{ThoGplugdSymssgpAff}, $G$ be the LARGER connex group of affine transformation, tandis que chez Bieliavsky (i), le transvection group $G$ is the SMALLEST subgroup of Aff(M) which is transitive etc.

\end{enumerate}
