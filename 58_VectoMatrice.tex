% This is part of Mes notes de mathématique
% Copyright (c) 2011-2015
%   Laurent Claessens
% See the file fdl-1.3.txt for copying conditions.

%+++++++++++++++++++++++++++++++++++++++++++++++++++++++++++++++++++++++++++++++++++++++++++++++++++++++++++++++++++++++++++
\section{Méthode de Gauss pour résoudre des systèmes d'équations linéaires}
%+++++++++++++++++++++++++++++++++++++++++++++++++++++++++++++++++++++++++++++++++++++++++++++++++++++++++++++++++++++++++++


Pour résoudre un système d'équations linéaires, on procède comme suit:
\begin{enumerate}
\item Écrire le système sous forme matricielle. \[\text{p.ex. } \begin{cases} 2x+3y &= 5 \\ x+2y &= 4 \end{cases} \Leftrightarrow \left(\begin{array}{cc|c} 2 & 3 & 5 \\ 1 & 2 & 4 \end{array}\right) \]
\item Se ramener à une matrice avec un maximum de $0$ dans la partie de gauche en utilisant les transformations admissibles:
\begin{enumerate}
\item Remplacer une ligne par elle-même + un multiple d'une autre;
\[\text{p.ex. } \left(\begin{array}{cc|c} 2 & 3 & 5 \\ 1 & 2 & 4 \end{array}\right)  \stackrel{L_1  - 2. L_2 \mapsto L_1'}{\Longrightarrow} \left(\begin{array}{cc|c} 0 & -1 & -3 \\ 1 & 2 & 4 \end{array}\right) \]
\item Remplacer une ligne par un multiple d'elle-même;
\[\text{p.ex. } \left(\begin{array}{cc|c} 0 & -1 & -3 \\ 1 & 2 & 4 \end{array}\right)  \stackrel{-L_1  \mapsto L_1'}{\Longrightarrow} \left(\begin{array}{cc|c} 0 & 1 & 3 \\ 1 & 2 & 4 \end{array}\right) \]
\item Permuter des lignes.
\[\text{p.ex. } \left(\begin{array}{cc|c} 0 & 1 & 3 \\ 1 & 0 & -2 \end{array}\right)  \stackrel{L_1  \mapsto L_2' \text{ et } L_2  \mapsto L_1'}{\Longrightarrow} \left(\begin{array}{cc|c} 1 & 0 & -2 \\ 0 & 1 & 3  \end{array}\right) \]
\end{enumerate}
\item Retransformer la matrice obtenue en système d'équations.
\[\text{p.ex. }  \left(\begin{array}{cc|c} 1 & 0 & -2 \\ 0 & 1 & 3  \end{array}\right) \Leftrightarrow \begin{cases} x &= -2 \\ y &= 3 \end{cases}  \]
\end{enumerate}

\textbf{Remarques :} 
\begin{itemize}
\item Si on obtient une ligne de zéros, on peut l'enlever:
\[\text{p.ex. }  \left(\begin{array}{ccc|c} 3 & 4 & -2 & 2 \\ 4 & -1 & 3 & 0 \\ 0 & 0 & 0 & 0 \end{array}\right) \Leftrightarrow  \left(\begin{array}{ccc|c} 3 & 4 & -2 & 2 \\ 4 & -1 & 3 & 0 \end{array}\right) \]
\item Si on obtient une ligne de zéros suivie d'un nombre non-nul, le système d'équations n'a pas de solution:
\[\text{p.ex. }  \left(\begin{array}{ccc|c} 3 & 4 & -2 & 2 \\ 4 & -1 & 3 & 0 \\ 0 & 0 & 0 & 7 \end{array}\right) \Leftrightarrow  \begin{cases} \cdots \\ \cdots \\ 0x + 0y + 0z = 7 \end{cases} \Rightarrow \textbf{Impossible} \]
\item Si on moins d'équations que d'inconnues, alors il y a une infinité de solutions qui dépendent d'un ou plusieurs paramètres:
\[\text{p.ex. }  \left(\begin{array}{ccc|c} 1 & 0 & -2 & 2 \\ 0 & 1 & 3 & 0 \end{array}\right) \Leftrightarrow  \begin{cases} x - 2z = 2 \\ y + 3z = 0 \end{cases} \Leftrightarrow  \begin{cases} x = 2 + 2\lambda \\ y = -3\lambda \\ z = \lambda \end{cases} \]
\end{itemize}

%+++++++++++++++++++++++++++++++++++++++++++++++++++++++++++++++++++++++++++++++++++++++++++++++++++++++++++++++++++++++++++
\section{Orthogonalité}
%+++++++++++++++++++++++++++++++++++++++++++++++++++++++++++++++++++++++++++++++++++++++++++++++++++++++++++++++++++++++++++

\begin{proposition}			\label{PropVectsOrthLibres}
	si $v_1,\cdots,v_k$ sont des vecteurs non nuls, orthogonaux deux à deux, alors ces vecteurs forment une famille libre.
\end{proposition}

%---------------------------------------------------------------------------------------------------------------------------
\subsection{Décomposition de Bruhat}
%---------------------------------------------------------------------------------------------------------------------------

\begin{theorem}[Décomposition de Bruhat]\index{Bruhat (décomposition)}\index{décomposition!Bruhat}    \label{ThoizlYJO}
    Soit \( \eK\) un corps; un élément \( M\in\GL(n,\eR)\) s'écrit sous la forme
    \begin{equation}
        M=T_1P_{\sigma}T_2
    \end{equation}
    où \( T_1\) et \( T_2\) sont des matrices triangulaires supérieures inversibles et où \( P_{\sigma}\) est une matrice de permutation \( \sigma\in S_n\). De plus il y a unicité de \( \sigma\).
\end{theorem}
\index{groupe!permutation}
\index{groupe!linéaire}
\index{matrice}

\begin{proof}
    Afin de rendre les choses plus visuelles, nous nous permettons de donner des exemples au fur et à mesure de la preuve. Nous prenons l'exemple de la matrice
    \begin{equation}
        \begin{pmatrix}
            1    &   3    &   4    \\
            2    &   5    &   6    \\
            0    &   7    &   8
        \end{pmatrix}.
    \end{equation}
    \begin{subproof}
    \item[Existence]
        Soit \( M\in \GL(n,\eR)\); vu qu'elle est inversible, on a un indice \( i_1\) maximum tel que \( M_{i_1,1}\neq 0\). Nous changeons toutes les lignes jusque là, c'est à dire que nous faisons, pour \( 1\leq i< i_1\),
        \begin{equation}        \label{EqGHUbwR}
            L_i\to L_i-\frac{ M_{i1} }{ M_{i_11} }L_{i_1}.
        \end{equation}

        % position 29873-10689
        Nous avons donc obtenu une matrice dont la première colonne est nulle sauf la case numéro \( i_1\). L'opération \eqref{EqGHUbwR} revient à considérer la multiplication par la matrice de transvection
        \begin{equation}
            T_1^{(i)}=T_{ii_1}\left( -\frac{ M_{i1} }{ M_{i_11} } \right)
        \end{equation}
        pour tout \( i<i_1\). Pour rappel nous ne changeons que les lignes \emph{au-dessus} de la \( i_1\). Du coup les matrices \( T^{(i)}_1\) sont triangulaires supérieures. Nous avons donc la nouvelle matrice \( M_1=\left( \prod_{i<i_1}T_1^{(i)} \right)M\) pour laquelle toute la première colonne est nulle sauf un élément.

        Dans le cas de l'exemple, le «pivot» sera la ligne \( (2,5,6)\) et la matrice se transforme à l'aide de la matrice \( T_1=T_{12}(-1/2)\) :
        \begin{equation}    \label{EqyjXIYf}
            \begin{pmatrix}
                1    &   -1/2    &   0    \\
                0    &   1    &   0    \\
                0    &   0    &   1
            \end{pmatrix}
            \begin{pmatrix}
                1    &   3    &   4    \\
                2    &   5    &   6    \\
                0    &   7    &   8
            \end{pmatrix}=
            \begin{pmatrix}
                0    &   1/2    &   1    \\
                2    &   5    &   6    \\
                0    &   7    &   8
            \end{pmatrix}.
        \end{equation}

    
    Maintenant nous faisons de même avec les colonnes (en renommant \( M\) la matrice obtenue à l'étape précédente) :
    \begin{equation}
        C_j\to C_j-\frac{ M_{i_1j} }{ M_{i_11} }C_1,
    \end{equation}
    qui revient à multiplier à droite par les matrices \( T_{1j}(\frac{ M_{i_1i} }{ M_{i_11} })\) avec \( j>1\). Encore une fois ce sont des matrices triangulaires supérieures.

    Dans l'exemple, pour traiter la seconde colonne, nous multiplions \eqref{EqyjXIYf} à droite par la matrice \( T_{12}(-5/2)\) :
    \begin{equation}
            \begin{pmatrix}
                0    &   1/2    &   1    \\
                2    &   5    &   6    \\
                0    &   7    &   8
            \end{pmatrix}
            \begin{pmatrix}
                1    &   -5/2    &   0    \\
                0    &   1    &   0    \\
                0    &   0    &   1
            \end{pmatrix}=
            \begin{pmatrix}
                0    &   1/2    &   1    \\
                2    &   0    &   6    \\
                0    &   7    &   8
            \end{pmatrix}.
    \end{equation}
    Appliquer encore la matrice \( T_{13}(-6/2)\) apporte la matrice
    \begin{equation}
        \begin{pmatrix}
            0    &   1/2    &   1    \\
            2    &   0    &   0    \\
            0    &   7    &   8
        \end{pmatrix}.
    \end{equation}
    Enfin nous multiplions la matrice obtenue par \( \frac{1}{ M_{i_11} }\mtu\) pour normaliser à \( 1\) l'élément «pivot» que nous avions choisit. Dans notre exemple nous multiplions par \( 1/2\) pour trouver
    \begin{equation}        \label{Eqduglwu}
        \begin{pmatrix}
            0    &   1/4    &   1/2    \\
            1    &   0    &   0    \\
            0    &   7/2    &   4
        \end{pmatrix}.
    \end{equation}

    La matrice obtenue jusqu'ici possède une ligne et une colonne de zéros avec un \( 1\) à leur intersection, et elle est de la forme
    \begin{equation}
        M'=T_1MT_2
    \end{equation}
    où \( T_1\) et \( T_2\) sont triangulaires supérieures et inversibles, produits de matrices de transvection (et d'une matrice scalaire pour la normalisation).

    Il reste à recommencer l'opération avec la seconde colonne (qui n'est pas toute nulle parce que le déterminant est encore non nul) puis la suivante etc. Dans notre exemple de l'équation \eqref{Eqduglwu}, nous éliminerions le \( 1/4\) et le \( 4\) en utilisant le \( 7/2\).

    Encore une fois tout cela se fait à l'aide de matrice supérieures parce qu'à chaque étape, les colonnes précédent le pivot sont déjà nulles (saut un \( 1\)) et ne doivent donc pas être touchées.

    À la fin de ce processus, ce qui reste est une matrice \( TMT'\) qui ne contient plus que un seul \( 1\) sur chaque ligne et chaque colonne, c'est à dire une matrice de permutation : \( P_{\sigma}=TMT'\) et donc
    \begin{equation}
        M=T^{-1}_{\sigma}(T')^{-1}.
    \end{equation}

        \item[Unicité]

            Soient \( \sigma,\sigma\in S_n'\) tels que \( T_1P_{\sigma}T_2=S_1P_{\tau}S_2\) avec \( T_i\) et \( S_i\) triangulaires supérieures et inversibles. En posant \( T=T_2S_2^{-1}\) et \( S=T_1^{-1}S_1\), nous avons
            \begin{equation}
                P_{\sigma}T=SP_{\tau}
            \end{equation}
            où \( S\) et \( T\) sont des matrices triangulaires supérieures et inversibles. Par les calculs de la preuve du lemme \ref{LemyrAXQs},
            \begin{subequations}
                \begin{numcases}{}
                    (P_{\sigma}T)_{kl}=T_{\sigma^{-1}(k)l}\\
                    (SP_{\tau})_{kl}=S_{k\tau(l)},
                \end{numcases}
            \end{subequations}
            et donc
            \begin{equation}    \label{EqKlmgOT}
                T_{\sigma^{-1}(k)l}=S_{k\tau(l)}.
            \end{equation}
            En écrivant cette équation avec \( k=\sigma(i)\) (nous rappelons que \( \sigma\) est bijective),
            \begin{equation}
                T_{il}=S_{\sigma(i)\tau(l)}.
            \end{equation}
            Nous savons que les termes diagonaux de \( T\) sont non nuls parce que \( T\) est triangulaire supérieure et inversible (donc pas de colonnes entières nulles). Nous avons donc, en prenant \( i=l=k\),
            \begin{equation}
                0\neq T_{kk}=S_{\sigma(k)\tau(k)}.
            \end{equation}
            La matrice étant triangulaire supérieure, cela implique 
            \begin{equation}    \label{EqEmiBTX}
                \sigma(k)\leq\tau(k).
            \end{equation}
            De la même manière en écrivant \eqref{EqKlmgOT} avec \( l=\tau^{-1}(i)\),
            \begin{equation}
                S_{ki}=T_{\sigma^{-1}(k)\tau^{-1}(i)}
            \end{equation}
            et donc
            \begin{equation}
                \sigma^{-1}(k)\leq \tau^{-1}(k).
            \end{equation}
            En écrivant cela avec \( k=\sigma(j)\), nous avons \( j\leq \tau^{-1}\sigma(j)\) et en appliquant enfin \( \tau\),
            \begin{equation}
                \tau(j)\leq \sigma(j).
            \end{equation}
            En comparant avec \eqref{EqEmiBTX}, nous avons \( \sigma=\tau\).
    \end{subproof}
\end{proof}

%+++++++++++++++++++++++++++++++++++++++++++++++++++++++++++++++++++++++++++++++++++++++++++++++++++++++++++++++++++++++++++ 
\section{À propos de nombres premiers}
%+++++++++++++++++++++++++++++++++++++++++++++++++++++++++++++++++++++++++++++++++++++++++++++++++++++++++++++++++++++++++++

\begin{proposition} \label{PropleGdaT}
    Soit \( p\) un nombre premier et \( P\) un élément de \( \eF_p[X]\). L'anneau \( \eF_p[X]/(P)\) est intègre si et seulement si \( P\) est irréductible dans \( \eF_p[X]\).
\end{proposition}

\begin{proof}
    Supposons que \( P\) soit réductible dans \( \eF_p[X]\), c'est à dire qu'il existe \( Q,R\in \eF_p[X]\) tels que \( P=QR\). Dans ce cas, \( \bar Q\) est diviseur de zéro dans \( \eF_p[X]/(P)\) parce que \( \bar Q\bar R=0\).

    Nous supposons maintenant que \( \eF_p[X]/(P)\) ne soit pas intègre : il existe des polynômes \( R,Q\in \eF_p[X]\) tels que \( \bar Q\bar R=0\). Dans ce cas le polynôme \( QR\) est le produit de \( P\) par un polynôme : \( QR=PA\). Tous les facteurs irréductibles de \( A \) étant soit dans \( Q\) soit dans \( R\), il est possible de modifier un peu \( Q\) et \( R\) pour obtenir \( QR=P\), ce qui signifie que \( P\) n'est pas irréductible.
\end{proof}

%--------------------------------------------------------------------------------------------------------------------------- 
\subsection{Un peu de structure dans \texorpdfstring{$ \eZ[i]$}{Zi}}
%---------------------------------------------------------------------------------------------------------------------------

\begin{lemma}   \label{LemSCAlICY}
     L'application
     \begin{equation}
         \begin{aligned}
             N\colon \eZ[i]&\to \eN \\
             a+bi&\mapsto a^2+b^2 
         \end{aligned}
     \end{equation}
     est un stathme euclidien pour \( \eZ[i]\).
\end{lemma}
\index{stathme!sur \( \eZ[i]\)}

\begin{proof}
    Soient \( t,t\in \eZ[i]\setminus\{ 0 \}\) et 
    \begin{equation}
        \frac{ z }{ t }=x+iy
    \end{equation}
    dans \( \eC\). Nous considérons \( q=a+bi\) où \( a\) et \( b\) sont les entiers les plus proches de \( x\) et \( y\). Si il y a \emph{ex aequo}, on prend au hasard\footnote{Dans l'exemple \ref{ExwqlCwvV}, nous prenions toujours l'inférieur parce que le stathme tenait compte de la positivité.}. Alors nous avons
    \begin{equation}
        | \frac{ z }{ t }-q |\leq \frac{ | 1+i | }{ 2 }=\frac{ \sqrt{2} }{2}<1.
    \end{equation}
    On pose \( r=z-qt\) qui est bien un élément de \( \eZ[i]\). De plus
    \begin{equation}
        | r |=| z-qt |=| t | |\frac{ z }{ t }-q |<| t |,
    \end{equation}
    c'est à dire que \( | r |^2<| t |^2\) et donc \( N(r)<N(t)\). 
\end{proof}
Étant donné que \( \eZ[i]\) est euclidien, il est principal (proposition \ref{Propkllxnv}).

\begin{lemma}   \label{LemBMEIiiV}
    Les éléments inversibles de \( \eZ[i]\) sont \( \{ \pm 1,\pm i \}\).
\end{lemma}

\begin{proof}

    Déterminons les éléments inversibles de \( \eZ[i]\). Si \( z\in \eZ[i]^*\), alors il existe \( z'\in \eZ[i]^*\) tel que \( zz'=1\). Dans ce cas nous aurions
    \begin{equation}
        1=N(zz')=N(z)N(z'),
    \end{equation}
    ce qui est uniquement possible avec \( N(z)=N(z')=1\), c'est à dire \( z=\pm 1\) ou \( z=\pm i\). Nous avons donc
    \begin{equation}
        \eZ[i]^*=\{ \pm 1,\pm i \}.
    \end{equation}
\end{proof}

Nous notons \( \Sigma=\{ a^2+b^2\tq a,b\in \eN \}\). 
\begin{lemma}   \label{LemIBDPzMB}
    L'ensemble \( \Sigma\) est un sous-monoïde de \( \eN\).
\end{lemma}

\begin{proof}
    Il suffit de prouver que si \( m,n\in \Sigma\), alors le produit \( mn\) est également dans \( \Sigma\). Si \( N\) est le stathme euclidien sur \( \eZ[i]\), alors  \( n\in \Sigma\) si et seulement si il existe \( z\in \eZ[i]\) tel que \( N(z)=n\). Si \( z,z'\in \eZ[i]\), alors \( zz'\in \eZ[i]\) et de plus
    \begin{equation}
        N(zz')=N(z)N(z')=nm.
    \end{equation}
    Donc \( nm\) est l'image de \( zz'\) par \( N\), ce qui prouve que \( nm\in \Sigma\).
\end{proof}

\begin{theorem}[Théorème des deux carrés, version faible]   \label{ThospaAEI}
    Un nombre premier est somme de deux carrés si et seulement si \( p=2\) ou \( p\in[1]_4\).
\end{theorem}
\index{anneau!principal}
\index{nombre!premier}
\index{théorème!des deux carrés!version faible}

\begin{remark}
    Il n'est pas dit que les nombres dans \( [1]_4\) sont premiers (\( 9=8+1\) ne l'est pas par exemple). Le théorème signifie que (à part \( 2\)), si un nombre premier est dans \( [1]_4\) alors il est somme de deux carrés, et inversement, si un nombre premier est somme de deux carrés, il est dans \( [1]_4\).
\end{remark}

\begin{proof}
    Soit \( p\) un nombre premier dans \( \Sigma\). Si \( a=2k\), alors \( a^2=4k^2\) et \( a^2=0\mod 4\). Si au contraire \( a\) est impair, \( a=2k+1\) et \( a^2=4k^2+1+4k=1\mod 4\). La même chose est valable pour \( b\). Par conséquent, \( a^2+b^2\) est automatiquement \( [0]_4\), \( [1]_4\) ou \( [2]_4\). Évidemment les nombres de la forme \( 0\mod 4\) ne sont pas premiers; parmi les \( 2\mod 4\), seul \( p=2\) est premier (et vaut \( 1^2+1^2\)).

    Nous avons démontré que les seuls premiers de la forme \( a^2+b^2\) sont \( p=2\) et les \( p=1\mod 4\). Il reste à faire le contraire : démontrer que si un nombre premier \( p\) vaut \( 1\mod 4\), alors il est premier. Nous considérons l'anneau
    \begin{equation}
        \eZ[i]=\{ a+bi\tq a,b\in \eZ \}.
    \end{equation}
    puis l'application
    \begin{equation}
        \begin{aligned}
            N\colon \eZ[i]&\to \eN \\
            a+bi&\mapsto a^2+b^2. 
        \end{aligned}
    \end{equation}
    Un peu de calcul dans \( \eC\) montre que pour tout \( z,z'\in \eZ[i]\), \( N(zz')=N(z)N(z')\).


    Nous savons que les éléments inversibles de \( \eZ[i]\) sont \( \pm 1\) et \( \pm i\) (lemme \ref{LemBMEIiiV}).

    Le lemme \ref{LemSCAlICY} montre que \( \eZ[i]\) est un anneau euclidien parce que \( N\) est un stathme. L'anneau \( \eZ[i]\) étant euclidien, il est principal (proposition \ref{Propkllxnv}).

   

    Pour la suite, nous allons d'abord montrer que \( p\in\Sigma\) si et seulement si \( p\) n'est pas irréductible dans \( \eZ[i]\), puis nous allons voir quels sont les irréductibles de \( \eZ[i]\).

    Soit \( p\), un nombre premier dans \( \Sigma\). Si \( p=a^2+b^2\), alors nous avons \( p=(a+ib)(a-bi)\), mais étant donné que \( p\) est premier, nous avons \( a\neq 0\) et \( b\neq 0\). Du coup \( p\) n'est pas inversible dans \( \eZ[i]\), mais il peut être écrit comme le produit de deux non inversibles. Le nombre \( p\) est donc non irréductible dans \( \eZ[i]\).

    Dans l'autre sens, nous supposons que \( p\) est un nombre premier non irréductible dans \( \eZ[i]\). Nous avons alors \( p=zz'\) avec ni \( z\) ni \( z'\) dans \( \{ \pm 1,\pm i \}\). En appliquant \( N\) nous avons
    \begin{equation}
        p^2=N(p)=N(z)N(z').
    \end{equation}
    Vu que \( p\) est premier, cela est uniquement possible avec \( N(z)=N(z')=p\) (avoir \( N(z)=1\) est impossible parce que cela dirait que \( z\) est inversible). Si \( z=a+ib\), alors \( p=N(z)=a^2+b^2\), et donc \( p\in \Sigma\).

    Nous savons déjà que \( \eZ[i]\) est un anneau principal et n'est pas un corps; la proposition \ref{PropomqcGe} s'applique donc et \( p\) sera non irréductible si et seulement si l'idéal \( (p)\) sera non premier. Le fait que \( (p)\) soit un idéal non premier implique que le quotient \( \eZ[i]/(p)\) est non intègre (c'est la définition d'un idéal premier). Nous cherchons donc les nombres premiers pour lesquels le quotient \( \eZ[i]/(p)\) n'est pas intègre.

    Nous commençons par écrire le quotient \( \eZ[i]/(p)\) sous d'autres formes. D'abord en remarquant que si \( I\) et \( J\) sont deux idéaux, on a \( (\eA/I)/J\simeq (\eA/J)/I\), du coup, en tenant compte du fait que \( \eZ[i]=\eZ[X]/(X^2+1)\), nous avons
    \begin{equation}
        \eZ[i]/(p)=(\eZ[X]/(p))/(X^2+1)=\eF_p[X]/(X^2+1).
    \end{equation}
    Nous avons donc équivalence des propositions suivantes :
    \begin{subequations}
        \begin{align}
            p\in\Sigma\\
            \eF_p[X]/(X^2+1)\text{ n'est pas intègre}\\
            X^2+1\text{ n'est pas irréductible dans \( \eF_p\)} \label{EqZkdrvh}\\
            \text{\( X^2+1\) admet une racine dans \( \eF_p\)}\\
            -1\in (\eF_p^*)^2\\
            \exists y\in \eF_p^*\tq y^2=-1.
        \end{align}
    \end{subequations}
    Le point \eqref{EqZkdrvh} vient de la proposition \ref{PropleGdaT}. Maintenant nous utilisons le fait que \( p\) soit un premier impair (parce que le cas de \( p=2\) est déjà complètement traité), donc \( (p-1)/2\in \eN\) et nous avons, pour le \( y\) de la dernière ligne,
    \begin{equation}
        (-1)^{(p-1)/2}=(y^2)^{(p-1)/2}=y^{p-1}=1
    \end{equation}
    parce que dans \( \eF_p\) nous avons \( y^{(p-1)}=1\) par le petit théorème de Fermat (théorème \ref{ThoOPQOiO}). Du coup \( p\) doit vérifier
    \begin{equation}
        1=(-1)^{(p-1)/2},
    \end{equation}
    c'est à dire \( \frac{ p-1 }{2}=0\mod 2\) ou encore \( p=1\mod 4\).
\end{proof}

\begin{theorem}[Théorème des deux carrés\cite{KXjFWKA}]
    Soit \( n\geq 2\) un nombre dont nous notons
    \begin{equation}    \label{EqBMHTzCT}
        n=\prod_{p\in\pP}p^{v_p(n)}
    \end{equation}
    où \( \pP\) est l'ensemble des nombres premiers. Alors \( n\in \Sigma\) si et seulement si pour tout \( p\in\pP\cap[3]_4\), nous avons \( v_p(n)\in [0]_2\) (c'est à dire \( v_p(n)\) est pair).
\end{theorem}
\index{théorème!des deux carrés}
\index{nombre!premier!théorème des deux carrés}
\index{anneau!principal!utilisation}
%TODO : il y a un lien entre le théorème des deux carrés et les triplets pytagoritiens http://fr.wikipedia.org/wiki/Triplet_pythagoricien

\begin{proof}
    \begin{subproof}
    \item[Condition suffisante.]
        
        Le produit \eqref{EqBMHTzCT} est évidemment un produit fini que nous pouvons alors regrouper en quatre parties : \( \pP\cap[0]_4\), \( \pP\cap[1]_4\), \( \pP\cap[2]_4\) et \( \pP\cap[3]_4\).

        \begin{itemize}
            \item Il n'y a pas de nombres premiers dans \( [0]_4\).
            \item Les nombres premiers de \( [1]_4\) sont dans \( \Sigma\). Le produit d'éléments de \( \Sigma\) étant dans \( \Sigma\), nous avons
                \begin{equation}
                    \prod_{p\in\pP\cap[1]_4}p^{v_p(n)}\in \Sigma.
                \end{equation}
            \item
                Le seul nombre premier dans \( [2]_4\) est \( 2\). C'est un élément de \( \Sigma\).
            \item
                Le produit
                \begin{equation}
                    \prod_{p\in\pP\cap[3]_4}p^{v_p(n)}
                \end{equation}
                est par hypothèse un produit de carrés (\( v_p(n)\) est pair), et est donc un carré.
        \end{itemize}
        Au final le produit \( \prod_{p\in\pP}p^{v_p(n)}\) est un produit d'un carré par un élément de \( \Sigma\), ce qui est encore un élément de \( \Sigma\).

        Pour cette partie, nous avons utilisé et réutilisé le lemme \ref{LemIBDPzMB}.

    \item[Condition nécessaire.] 

        Soit \( p\), un nombre premier. Nous voulons montrer que
        \begin{equation}
            \{ v_p(n)\tq n\in \Sigma \}\subset [2]_2.
        \end{equation}
        Pour montrer cela nous allons procéder par récurrence sur les ensembles
        \begin{equation}
            E_k=\{ v_p(n)\tq n\in \Sigma \}\cap\{ 0,\ldots, k \}.
        \end{equation}
        Il est évident que les éléments de \( E_0\) sont pairs, vu qu'il n'y a que zéro, qui est pair.

        Supposons que \( E_k\subset[0]_2\), et montrons que \( E_{k+1}\subset[0]_2\). Soit un élément de \( E_{k+1}\), c'est à dire \( v_p(n)\leq k+1\) avec \( n=a^2+b^2\). Si \( v_p(n)=0\) alors l'affaire est réglée; sinon c'est que \( p\) divise \( n\). Mais dans \( \eZ[i]\) nous avons
        \begin{equation}
            n=a^2+b^2=(a+bi)(a-bi)
        \end{equation}
        Vu que \( \eZ[i]\) est principal, le lemme de Gauss \ref{LemSdnZNX} nous dit que si \( p\) divise \( n\), alors il doit diviser soit \( a+bi\), soit \( a-bi\) (et du coup en fait les deux). Nous avons alors \( p\divides a\) et \( p\divides b\) en même temps. Du coup
        \begin{equation}
            p^2\divides a^2+b^2=n.
        \end{equation}
        Posons \( a=pa'\) et \( b=pb'\) avec \( a',b'\in \eN\). Nous avons
        \begin{equation}
            \frac{ n }{ p^2 }=\frac{ p^2a'^2+p^2b'^2 }{ p^2 }=a'^2+b'^2\in \Sigma.
        \end{equation}
        Mais par construction,
        \begin{equation}
            v_p\left( \frac{ n }{ p^2 } \right)=v_p(n)-2<k.
        \end{equation}
        Donc \( v_p(\frac{ n }{ p^2 })\) est pair et du coup \( v_p(n)\) doit également être pair.

    \end{subproof}
\end{proof}

