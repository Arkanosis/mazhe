% This is part of Exercices de mathématique pour SVT
% Copyright (C) 2010-2011,2015
%   Laurent Claessens et Carlotta Donadello
% See the file fdl-1.3.txt for copying conditions.

\begin{corrige}{DS2010-1-0005}

	\begin{enumerate}
		\item
			Pour rappel, 
			\begin{equation}
				(-1)^n=\begin{cases}
					1	&	\text{si $n$ est pair}\\
					-1	&	 \text{si $n$ est impair.}
				\end{cases}
			\end{equation}
			Les premiers termes sont donc
			\begin{equation}
				\begin{aligned}[]
					u_1&=(-1)+1=0\\
					u_2&=(-1)^2+\frac{ 1 }{2}=\frac{ 3 }{ 2 }\\
					u_3&=(-1)^3+\frac{1}{ 3 }=-\frac{ 2 }{ 3 }\\
					u_4&=(-1)^4+\frac{1}{ 4 }=\frac{ 5 }{ 4 }.
				\end{aligned}
			\end{equation}
		\item
                  Comme $1/n$ est surement compris entre $0$ et $1$, la suite $(u_n)_n$ est majorée par $2$ et minorée par $-1$.

		  Méthode alternative : nous avons
			\begin{equation}
				\left| (-1)^n+\frac{1}{ n } \right| \leq | (-1)^n |+\left| \frac{1}{ n } \right| \leq 1+\frac{1}{ n }.
			\end{equation}
			Or $\frac{1}{ n }$ est une suite qui tend vers zéro, ce qui fait que $1+\frac{1}{ n }$ est bornée.
		\item
			La suite est croissante entre $u_1$ et $u_2$, et décroissante entre $u_2$ et $u_3$. Plus généralement, les termes pairs seront positifs tandis que les termes impairs seront négatifs; elle n'arrête donc pas de monter et descendre au dessus de zéro et en dessous de zéro.
		\item
		   La valeur absolue de $u_n$ est
                  \begin{equation}
				\left| (-1)^n+\frac{1}{ n } \right| =1+ \frac{(-1)^n }{ n }.
			\end{equation}
                  La suite $\frac{(-1)^n }{ n }$ tend vers $0$, comme on a vu au cours. La limite de $(|u_n|)_n$ est donc $1$.

                  Méthode alternative : si la suite $(| u_n |)$ converge, elle converge vers la même limite que n'importe quelle de ses sous-suites. La sous-suite des termes pairs est facile :
			\begin{equation}
				| u_{2n} |=\left| (-1)^{2n}+\frac{1}{ n } \right| =1+\frac{1}{ n },
			\end{equation}
			et cela tend vers $1$.
		\item
			\begin{equation}
				\begin{aligned}[]
					u_2&=\frac{ 3 }{2},	&u_1&=0,\\
					u_4&=\frac{ 5 }{4},	&u_3&=-\frac{ 2 }{ 3 },\\
					u_6&=\frac{ 7 }{6},	&u_5&=-\frac{ 4 }{ 5 }.
				\end{aligned}
			\end{equation}
		\item
			La suite des termes pairs est (rappel : pour tout $n$, le nombre $2n$ est pair et donc $(-1)^{2n}=1$)
			\begin{equation}
				u_{2n}=(-1)^{2n}+\frac{1}{ 2n }=1+\frac{1}{ 2n }.
			\end{equation}
			Cette suite tend vers $1$ parce que $\frac{1}{ 2n }$ tend vers zéro.

			La suite des termes impairs par contre vaut
			\begin{equation}
				u_{2n+1}=(-1)^{2n+1}+\frac{ 1 }{ 2n+1 }=-1+\frac{1}{ 2n+1 },
			\end{equation}
			et cette suite tend vers $-1$.

			Nous avons donc trouvé deux sous-suites de $(u_n)$ qui tendent vers des limites différentes. La suite $(u_n)$ elle-même ne converge donc pas.
	\end{enumerate}

\end{corrige}
