\begin{enumerate}

	\item
		Les corrigés sont rédigés pour \href{http://qtoctave.wordpress.com/}{Octave}. De petites différences avec Matlab existent.
	\item
		Les exercices des séances sont tirés des notes «Introduction au logicile Matlab» qui fut donné à Louvain-la-Neuve sous le nom BIR1200.  Les exercices des tests sont dûs à Laurent Claessens et Yannick Voglaire.
    \item
        Merci à J.J. pour m'avoir signalé que \texttt{VerbatimInput} créait des problèmes avec \texttt{hyperref}, puis à Tanguy Briançon et Jean-Côme Charpentier pour l'avoir résolu.
	\item
		Merci de me signaler toute erreur ou imprécision. Plus vous vous plaignez, plus les étudiants de l'année prochaine auront un document de qualité :)
\end{enumerate}

%Note : ces notes proviennent d'un cours que j'ai été \emph{obligé} de donner sur Matlab. De mon plein gré, je n'aurais pas enseigné Matlab, mais Sage\footnote{\url{http://www.sagemath.org}} dont le langage de programmation sous-jacent est python (c'est un \emph{vrai} langage). Je parle un peu de ce logiciel dans mes notes de mathématiques\footnote{\url{http://student.ulb.ac.be/~lclaesse/lefrido.pdf}}, et je vous conseille chaudement d'oublier Matlab.


