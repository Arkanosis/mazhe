\chapter{Matlab}

\begin{enumerate}

	\item
		Les corrigés sont rédigés pour \href{http://qtoctave.wordpress.com/}{Octave}. De petites différences avec Matlab existent.
	\item
		Les exercices des séances sont tirés des notes «Introduction au logicile Matlab» qui fut donné à Louvain-la-Neuve sous le nom BIR1200.  Les exercices des tests sont dûs à Laurent Claessens et Yannick Voglaire.
    \item
        Merci à J.J. pour m'avoir signalé que \texttt{VerbatimInput} créait des problèmes avec \texttt{hyperref}, puis à Tanguy Briançon et Jean-Côme Charpentier pour les avoir résolu.
	\item
		Merci de me signaler toute erreur ou imprécision. Plus vous vous plaignez, plus les étudiants de l'année prochaine auront un document de qualité :)
\end{enumerate}

\section{Exercices des séances}

\subsection{Bases et calcul matriciel}

\Exo{SC_serie1-0001}
\Exo{SC_serie1-0002}
\Exo{SC_serie1-0003}
\Exo{SC_serie1-0004}
\Exo{SC_serie1-0005}

\subsection{Vecteurs à éléments équidistants}

\Exo{SC_serie2-0001}
\Exo{SC_serie2-0002}
\Exo{SC_serie2-0003}
\Exo{SC_serie2-0004}
\Exo{SC_serie2-0005}

\subsection{Polynômes et approximations au sens des moindres carrés}

\Exo{SC_serie3-0001}
\Exo{SC_serie3-0002}
\Exo{SC_serie3-0003}
\Exo{SC_serie3-0004}

\subsection{Intégration numérique et résolution d'équations différentielles}

\Exo{SC_serie4-0001}
\Exo{SC_serie4-0002}
\Exo{SC_serie4-0003}
\Exo{SC_serie4-0004}

\subsection{Exercices variés}

\Exo{SC_serie5-0001}
\Exo{SC_serie5-0002}
\Exo{SC_serie5-0003}
\Exo{SC_serie5-0004}
\Exo{SC_serie5-0005}
\Exo{SC_serie5-0006}

\section{Anciens tests et examens}
\subsubsection{BIR1200 en 2009}

\Exo{Matlab0001}
\Exo{Matlab0002}
\Exo{Matlab0003}
\Exo{Matlab0004}
\Exo{Matlab0005}
\Exo{Matlab0006}
\Exo{Matlab0007}
\Exo{Matlab0008}
\Exo{Matlab0009}
\Exo{Matlab0010}
\Exo{Matlab0011}
\Exo{Matlab0012}
\Exo{Matlab0013}
\Exo{Matlab0014}
\Exo{Matlab0015}
\Exo{Matlab0016}
\Exo{Matlab0017}
\Exo{Matlab0018}
\Exo{Matlab0019}
\Exo{Matlab0020}
\Exo{Matlab0021}
\Exo{Matlab0022}
\Exo{Matlab0023}
\Exo{Matlab0024}
\Exo{Matlab0025}
\Exo{Matlab0026}
\Exo{Matlab0027}
\Exo{Matlab0028}
\Exo{Matlab0029}
\Exo{Matlab0030}
\Exo{Matlab0031}
\Exo{Matlab0032}
\Exo{Matlab0033}
\Exo{Matlab0034}
\Exo{Matlab0035}
\Exo{Matlab0036}
\Exo{Matlab0037}
\Exo{Matlab0038}
\Exo{Matlab0039}
\Exo{Matlab0040}

\Exo{LCexoMatlab0001}
\Exo{LCexoMatlab0002}
\Exo{LCexoMatlab0003}
\Exo{LCexoMatlab0004}
\Exo{LCexoMatlab0005}
\Exo{LCexoMatlab0006}
\Exo{LCexoMatlab0007}
\Exo{LCexoMatlab0008}


\Exo{testMAT1151-G210001}
\Exo{testMAT1151-G210002}
\Exo{testMAT1151-G210003}
\Exo{testMAT1151-G220001}
\Exo{testMAT1151-G220002}
\Exo{testMAT1151-G220003}
\Exo{testMAT1151-G310001}
\Exo{testMAT1151-G310002}
\Exo{testMAT1151-G310003}
\Exo{testMAT1151-G320001}
\Exo{testMAT1151-G320002}
\Exo{testMAT1151-G320003}

\section{Conseils généraux}
\subsection{Écriture d'une fonction}

Lorsque vous créez une fonction, veillez aux éléments suivants 
\begin{enumerate}

	\item
		La première ligne doit être de la forme
		\begin{verbatim}
		function retour=nom_de_fonction(x)
		\end{verbatim}
		La valeur de la fonction sera celle de la variable \verb+retour+ lorsqu'on arrivera à la fin de la fonction. De plus, cette fonction doit être sauvée dans le fichier \verb+nom_de_fonction.m+.

\end{enumerate}

Noms des fichiers
\begin{enumerate}
	\item
		Tenez vous en à l'alphabet latin et aux chiffres arabes (et non le contraire).
	\item
		Le nom de fichier ne peut pas \emph{commencer} par des chiffres
	\item
		Évitez absolument de mettre des caractères spéciaux qui peuvent avoir un sens mathématique : «(», «'»,«-»,«+»,\ldots
	\item
		Évitez de mélanger les majuscules et les minuscules. %Il existe encore des personnes qui utilisent Windows (qui est le seul système à ne pas faire la distinction). Certes ces personnes sont peu nombreuses, mais il en reste encore assez\footnote{En fait, toutes les personnes qui ont un ado dans la maison qui a envie de jouer eux Sims.} pour que cela pose des problèmes de temps en temps.
	\item
		Ne mettez pas de points dans le nom de vos fichiers (à part \verb+.m+).
\end{enumerate}

\subsection{Conception des fonctions}

Une fonction doit prendre un nombre (ou une matrice) en entrée et sortir un nombre à la fin. Une fonction doit seulement générer des nombres (ou des matrices). Vous ne devez pas mettre de commandes comme \verb+plot+ dans une fonction. Même si cela fonctionne de temps en temps, ce n'est pas une bonne idée. En principe, vous devez concevoir vos fonctions de telle façon qu'elles n'affichent rien.

Les commandes qui doivent \emph{afficher} des résultats se mettent dans un script (blank M-file). Créez un tel script par exercice, même si l'exercice se décompose en sous questions.

\subsection{Autres}

Si un résultat dépend d'un calcul intermédiaire, ne faites \emph{jamais} le calcul dans la fenêtre principale pour en copier-coller le résultat dans votre script. Faites faire le calcul dans votre script, et enregistrez le résultat dans une variable. Ainsi vous gardez toutes les décimales que Matlab avait calculées sans les afficher. Et cela, même si ladite réponse intermédiaire est un nombre entier.


