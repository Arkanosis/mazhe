\section{Topologie}
%+++++++++++++++++

Maintenant que nous avons vu des choses amusantes avec l'infini, nous devons passer à une partie mois drôle et un peu plus formelle. Tu verras plus tard que ce qui intéresse le physicien, c'est les dérivées. Mais hélas pour toi, ô lecteur poly-dégouté par les math, pour pouvoir au mieux tirer parti de la notion de dérivée, il faut d'abord savoir correctement manipuler des limites ainsi savoir déterminer si une fonction est continue ou non. Or le cadre naturel de l'étude de la continuité est la topologie.

Les concepts de topologie et de continuité ne seront pour ainsi dire jamais invoqués dans des problèmes concrets de physique\footnote{Si si : des limites en l'infini, tu vas en voir dans tes cours de physique.}. Ceci pour la simple raison que pratiquement toutes les fonctions qui arrivent en physique sont continues. Le physicien ne s'embarrasse donc jamais de savoir si les fonctions qu'il manipule sont continues ou non. D'ailleurs, une des seules façons de distinguer à coup sûr un physicien d'un mathématicien, c'est que le mathématicien vérifie la continuité de ses fonctions avant de travailler tandis que le physicien pas : le physicien suppose que tout va bien, et quand il a vraiment un problème, il va aller pleurer chez un ami mathématicien pour qu'il lui dise si il n'y a pas un problème de topologie caché quelque part.

\subsection{Maximum et supremum}
%+++++++++++++++++++++++++++++++

\subsubsection{Un tout petit peu de théorie\ldots}
%//////////////////////

Ce n'est un secret pour personne que $\eR$ est un \href{http://fr.wikipedia.org/wiki/Relation_d'ordre}{ensemble ordonné} : il y a des éléments plus grands que d'autres, et mieux : à chaque fois que je prends deux éléments différents dans $\eR$, il y en a un des deux qui est plus grand que l'autre. Il n'y a pas d'\emph{ex aequo} dans $\eR$.

  Si je regarde l'intervalle $I=[0,1]$, je peux même dire que $10$ est plus grand que tous les éléments de $I$. Nous disons que $10$ est un \emph{majorant} de $[0,1]$. La définition est la suivante.
\begin{definition}
Lorsque $A$ est un sous-ensemble de $\eR$, on dit que $s$ est un \defe{majorant}{Majorant} de $A$ si $s$ est plus grand que tous les éléments de $A$. En d'autres termes, si
\[
  \forall x\in A,\,s\geq x.
\]
\end{definition}
Je me permet d'insister sur le fait que l'inégalité n'est pas stricte. Ainsi, $1$ est un majorant de $[0,1]$. Dès qu'un ensemble a un majorant, il en a plein. Si $s$ majore l'ensemble $A$, alors évident $s+1$, $s+4$, $s+\pi^2$ majorent également $A$.
\begin{exemple}
Une petite galerie d'exemples de majorants.
\begin{itemize}
\item L'intervalle fermé $[4,8]$ admet entre autres $8$ et $130$ comme majorants,
\item l'intervalle ouvert $]4,8[$ admet également $8$ et $130$ comme majorants,
\item $7$ n'est pas un majorant de $[1,5]\cup]8,32]$,
\item $10/10$ majore les côtes qu'on peut obtenir à une interrogation,
\item l'intervalle $[4,\infty[$ n'a pas de majorants.
\end{itemize}
\end{exemple}
Maintenant nous allons voir le premier concept vraiment subtile de toute l'histoire de tes cours de math\footnote{Quoi ? Il y a déjà des trucs que tu avais trouvé compliqué ? Eh bien rassures-toi : ce qui suit n'est pas spécialement \emph{compliqué}; c'est \emph{subtil} !}.
\begin{definition}
Le \defe{supremum}{Supremum} d'un ensemble est le plus petit majorant. En d'autres terme, $s$ est un supremum de $A$ si tout nombre plus petit que $s$ ne majore pas $A$, ou encore,
\[
  \forall x<s,\exists y\in A\text{ tel que } y>x.
\]
Nous disons que $M$ est un \defe{maximum}{Maximum} de $A$ si $M$ est un supremum \emph{et} $M\in A$.
\end{definition}
Quand $s$ est un supremum de $A$, ça veut dire que le moindre pas vers la gauche que l'on fait à partir de $s$ (c'est à dire le moindre $\epsilon$), et on tombe dans $A$, ou tout au moins, il existe des éléments de $A$ qui sont plus grand que $s-\epsilon$.

\subsubsection{\ldots et quelque exemples}
%//////////////////////

En matières de notations, le maximum de l'ensemble $A$ est noté $\max A$, le supremum est noté $\sup A$. Le minimum et l'infimum sont notés $\min A$ et $\inf A$.

\begin{exemple}
Exemples de différence entre majorant, supremum et maximum.
\begin{itemize}
\item Le nombre $10$ est un supremum, majorant et maximum de l'intervalle fermé $[0,10]$,
\item Le nombre $10$ est un majorant et un supremum, mais pas un maximum de l'intervalle ouvert $]0,10[$,
\item Le nombre $136$ est un majorant, mais ni un maximum ni un supremum de l'intervalle $[0,10]$.
\end{itemize}
\end{exemple}

En utilisant les notations concises, ces différents cas s'écrivent ainsi :
\begin{align*}
10&=\max[0,10]=\sup[0,10]	& 10&=\sup[0,10[
\end{align*}


\begin{exemple}
Si on dit que un pont s'effondre à partir d'une charge de $10$ tonnes, alors $10$ tonnes est un \emph{supremum} des charges que le pont peut supporter : si on met $9,999999$ tonnes dessus, il tient encore le coup, mais si on ajoute un gramme, alors il s'effondre (on sort de l'ensemble des charges acceptables).
\end{exemple}

\begin{exemple}
Si on dit qu'un pont résiste jusqu'à $10$ tonnes, alors $10$ tonnes est un \emph{maximum} de la charge acceptable. Sur ce pont-ci, on peut ajouter le dernier gramme. Mais à partir de là, le moindre truc qu'on ajoute, il s'effondre.
\end{exemple}

\begin{exemple}
Lorsqu'on dit que $50\%$ est le minimum requis pour passer son année, alors $50\%$ est un supremum des côtes avec lesquelles on rate, mais pas un maximum. En effet, pour tout $\epsilon$ plus grand que zéro, il y a une côte plus grande que $50-\epsilon$ qui fait rater : la côte $(50-\epsilon/2)$ par exemple.
\end{exemple}

\Exo{206}

Maintenant il est important de se rendre compte d'une chose : un ensemble ne peut avoir qu'un seul maximum et supremum. Jusqu'à présent nous avons toujours dit \emph{un} supremum. À partir de maintenant nous pouvons dire \emph{le} supremum. La preuve de cela est assez simple.
\begin{proposition}
Si $A$ est un sous-ensemble de $\eR$ admettant un supremum, alors il n'a qu'un seul supremum; et si il accepte un maximum, il n'en accepte un seul, et le maximum est égal au supremum.
\end{proposition}

\begin{proof}
Commençons par l'affirmation concernant le supremum. Supposons que $x$ et $y$ soient tous les deux suprema différents de $A$. Étant donné que $x\neq y$, nous pouvons supposer que $x<y$, et donc, par définition du fait que $y$ est un supremum, il existe un élément de $A$ qui est plus grand que $x$. Cela contredit le fait que $x$ soit supremum. En conclusion, il ne peut pas y avoir deux suprema différents pour un même ensemble.

Étant donné qu'un maximum est un supremum, il ne peut pas y avoir deux maxima différents vu qu'il ne peut pas y avoir deux suprema différents.
\end{proof}

\Exo{207}

\subsection{Espaces métriques}
%-----------------------------

Nous allons présenter maintenant les bases de la topologie sur des espaces métriques en prenant $\eR$ et $\eR^2$ comme exemple principaux. La topologie est un des fondements de la mathématique et est une prolongation de la théorie des ensembles. Nous n'en trouvons hélas pas beaucoup d'application directes en physique.

Si $E$ est un ensemble quelconque, nous disons qu'une \defe{distance}{Distance} sur $E$ est une fonction $d\colon E\times E\to \eR^+$ telle que
\begin{description}
\item[Symétrie] $d(x,y)=d(y,x)$,
\item[Séparation] $d(x,y)=0$ ssi $x=y$. Insistons sur le fait que dans tous les cas, nous devons avoir $d(x,y)\geq 0$,
\item[Inégalité triangulaire] $d(x,z)\leq d(x,y)+d(y,z)$
\end{description}
pour tout $x$, $y$, $z\in E$. Un ensemble muni d'une loi de distance s'appelle un \href{http://fr.wikipedia.org/wiki/Espace_métrique}{espace métrique}.

Le premier exemple d'espace métrique que nous connaissons est $\eR$ muni de la distance usuelle ente deux nombres :
\begin{equation}
d(x,y)=| y-x |.
\end{equation}
Je me permet de te faire remarquer la valeur absolue.

\begin{exercice}
Que penses-tu de la formule $d(x,y)=y-x$ pour définir une distance sur $\eR$ ?
\end{exercice}

À partir de là, nous définissons la notion de \defe{boule ouverte}{Boule!ouverte} sur l'ensemble $E$ centrée au point $x$ et de rayon $r>0$ comme
\[
  B(x,r)=\{ y\in\eR\tq d(x,y)< r \}.
\]
La \defe{boule fermée}{Boule!fermée} centrée en $x$ et de rayon $r>0$ est définie par
\[
  \bar B(x,r)=\{ y\in\eR\tq d(x,y)\leq r \}.
\]
La différence est que dans la première l'inégalité est stricte.

\begin{theorem}		\label{ThoBoulOuvVois}
Une boule ouverte contient une boule ouverte autour de chacun de ses points.
\end{theorem}

\begin{proof}
Prenons $y\in B(x,r)$, et prouvons que la boule $B(y,r-d(x,y))$ est contenue dans $B(x,r)$. Première chose : $r-d(x,y)>0$ parce que $y$ est dans la boule ouverté centrée en $x$ et de rayon $r$. Pour prouver que  $B(y,r-d(x,y))\subset B(x,r)$, prenons un point dans le premier ensemble et montrons qu'il est dans le second ensemble.

Soit donc $z\in B\big(y,r-d(x,y)\big)$ et testons $d(x,z)$ que nous voudrions être plus petit que~$r$. Et, miracle, il l'est parce que
\begin{align*}
  d(x,z)	&\leq d(x,y)+d(y,z)&\text{inégalité triangulaire}\\
		&<d(x,y)+\big(r-d(x,y)\big)&\text{$z\in B\big(y,r-d(x,y)\big)$}\\
		&=r.
\end{align*}
Remarquez que la première inégalité n'est pas stricte, tandis que la seconde est stricte. Nous avons donc bien $d(x,z)<r$ (strictement) comme le demandé pour que $z$ soit dans la boule \emph{ouverte} de centre $x$ et de rayon $r$.
\end{proof}

Lorsque $x\in E$, nous disons qu'un \defe{voisinage}{Voisinage} de $x$ est n'importe quel sous-ensemble de $E$ qui contient une boule ouverte centrée en $x$. Nous disons qu'un ensemble est \defe{ouvert}{Ouvert} si il contient un voisinage de chacun de ses points. Évidement les boules ouvertes sont les prototypes d'ouverts par le théorème \ref{ThoBoulOuvVois}. Par convention, nous disons que l'ensemble vide est ouvert.

\begin{definition}
L'ensemble des boules ouvertes d'un espace métrique forment la \defe{topologie}{Topologie!métrique} de l'espace.
\end{definition}

Un ensemble est ouvert si et seulement si il contient une boule autour de chacun de ses points.

Nous allons dire qu'une partie $A$ d'un espace métrique est \defe{bornée}{Bornée} si il existe une boule\footnote{À titre d'exercice, je te laisse te convaincre que l'on peut dire boule \emph{ouverte} ou \emph{fermée} au choix sans changer la définition.} qui contient $A$.

\begin{lemma}  \label{LemSupOuvPas}
Le supremum d'un ensemble ouvert n'est pas dans l'ensemble (et n'est donc pas un maximum).
\end{lemma}

\begin{proof}
Soit $\mO$, un ensemble ouvert et $s$, son supremum. Si $s$ était dans $\mO$, on aurait un voisinage $B=B(s,r)$ de $s$ contenu dans $\mO$. Le point $s+r/2$ est alors à la fois dans $\mO$ et plus grand que $s$, ce qui contredit le fait que $s$ soit un supremum de $\mO$.
\end{proof}

\begin{exercice}
Par le même genre de raisonnements, montrez que l'union et l'intersection de deux ouverts sont encore des ouverts.
\end{exercice}

\begin{remark}
L'intersection d'une \emph{inifinté} d'ouverts n'est pas spécialement un ouvert comme le montre l'exemple suivant :
\[ 
  \mO_i=]1,2+\frac{ 1 }{ i }[.
\]
Tous les ensembles $\mO_i$ contiennent le point $2$ qui est donc dans l'intersection. Mais quel que soit le $\epsilon>0$ que l'on choisisse, le point $2+\epsilon$ n'est pas dans $\mO_{(1/\epsilon)+1}$. Donc aucun point au-delà de $2$ n'est dans l'intersection, ce qui prouve que $2$ ne possède pas de voisinages contenus dans $\cap_{i=1}^{\infty}\mO_i$.
\end{remark}

\begin{exercice}
Prouver que, quels que soient les ensembles $A$ et $B$ dans $\eR$, nous avons
\[ 
  \sup(A\cap B)\leq\sup A\leq\sup(A\cup B).
\]
\end{exercice}


\subsection{Connexité}
%----------------------

Dès qu'un ensemble est muni d'une métrique, nous pouvons définir les boules ouvertes, les voisinages et les sous-ensembles ouverts. Dès que l'on a identifié les sous-ensemble ouverts de $E$, nous disons que $E$ devient un \defe{espace topologique}{Espace topologique}. Nous allons maintenant un pas plus loin.

Nous voulons maintenant décrire ce qu'est un ensemble connexe. La notion intuitive d'un ensemble connexe est le fait que l'on puisse aller d'un point à l'autre sans sortir. En d'autres mots, nous voulons dire qu'un ensemble est connexe quand il est en un seul morceau. Nous avons donc envie de dire que le sous-ensemble $A$ de $E$ est connexe quand il ne peut pas être écrit comme une union disjointe de deux ensembles.

Cette définition ne peut pas fonctionner telle quelle, parce que tout ensemble peut être écrit comme l'union disjointe de deux ensembles. Il faut donc un peu contraindre le choix d'ensembles en lesquels on ne veut pas que $A$ se décompose. Il se faut que la bonne définition est la suivante :

\begin{definition}
 Lorsque $E$ est un espace topologique, nous disons qu'un sous-ensemble $A$ est \defe{non connexe}{Connexe} quand on peut trouver des ouverts $O_1$ et $O_2$ tels que
\begin{equation} 	\label{EqDefnnCon}
  A=(A\cap O_1)\cup (A\cap O_2),
\end{equation}
et tels que $A\cap O_1\neq\emptyset$, et $A\cap O_2\neq\emptyset$.
Si un sous-ensemble n'est pas non-connexe, alors on dit qu'il est connexe.
\end{definition}
Une autre façon d'exprimer la condition \eqref{EqDefnnCon} est de dire que $A$ n'est pas connexe quand il est contenu dans la réunion de deux ouverts disjoints qui intersectent tous les deux $A$.
\begin{figure}
\centering
\psset{xunit=1cm,yunit=1cm}
\begin{pspicture}(-4,-1.3)(3.2,2.7)
   %\psframe[linecolor=cyan](-4,-1.3)(3.2,2.7)
   \psset{PointSymbol=none,PointName=none}
	%\pspolygon[fillstyle=vlines,hatchcolor=red,linecolor=red](0,0)(0,2)(2.5,2)(1,-1)
	%\pspolygon[fillstyle=vlines,hatchcolor=red,linecolor=red](0,0)(0,2)(2.5,2)(1,-1)
	\pspolygon[fillstyle=vlines](0,0)(0,2)(2.5,2)(1,-1)
	\pspolygon[fillstyle=vlines](-2,0)(-3,0.4)(-2.5,2)
	\pstGeonode(0,1){A}(0,3){B}(0,2.5){C}
	\pstCircleOA[linecolor=red,Radius=\pstDistAB{A}{B}]{1.2,0.7}{}
	\pstCircleOA[linecolor=blue,Radius=\pstDistAB{A}{C}]{-2.5,1}{}
\end{pspicture}

\caption{La figure hachurée n'est pas connexe parce qu'on peut dessiner deux ouverts disjoints qui la sépare.}  \label{FigExnnConn}
\end{figure}
Le cas de la figure \ref{FigExnnConn} montre une surface dans $\eR^2$ qui est clairement non connexe. Mais il y a des cas nettement moins faciles à traiter. La figure \ref{FigConnPapi} montre deux parties de $\eR^2$ qui ne diffèrent que de un seul point. Est-ce que tu pourrais dire si il y en a un des deux qui est connexe ?
\begin{figure}
\centering
\psset{xunit=1cm,yunit=1cm}
\subfigure[Cet ensemble contient le point central.]{%
\begin{pspicture}(-2,-1)(2,1)
	\psset{PointSymbol=none, PointName=none}
	\pspolygon[fillstyle=vlines](0,0)(-2,1)(-2,-1)(2,1)(2,-1)
\end{pspicture}
%
}
\subfigure[Cet ensemble ne contient pas le point central.]{%
\begin{pspicture}(-2,-1)(2,1)
	\psset{PointSymbol=none, PointName=none}
	\pspolygon[fillstyle=vlines](0,0)(-2,1)(-2,-1)(2,1)(2,-1)
   \pstGeonode(0,0){A}(0.1,0){B}
	\pstCircleOA[fillstyle=solid,fillcolor=white,linecolor=black]{A}{B}
\end{pspicture}
}
\caption{Exemple de deux ensembles dont la connexité se joue à un point près.}	\label{FigConnPapi}
\end{figure}
Nous n'allons pas traiter plus avant cet exemple. Au lieu de cela, nous allons déterminer tous les sous-ensembles connexes de $\eR$. Pour cela nous avons besoin d'une définition précise de ce que l'on appelle un \emph{intervalle} dans~$\eR$.
\begin{definition}
	Un \defe{intervalle}{Intervalle} est une partie de $\eR$ telle que tout élément compris entre deux éléments de la partie soit dedans. En formule, la partie $I$ de $\eR$ est un intervalle si
	\[
	  \forall a,b\in I,(a\leq x\leq b)\Rightarrow x\in I.
	\]
\end{definition}
Cette définition englobe tous les exemples que tu connais d'intervalles ouverts, fermés avec ou sans infini : $[a,b]$, $[a,b[$, $]-\infty,a]$, \ldots

Une des nombreuses propositions qui vont servir à prouver le théorème des \href{http://fr.wikipedia.org/wiki/Théorème_des_valeurs_intermédiaires}{valeurs intermédiaires} (théorème numéro \ref{ThoValInter}) est la suivante.
\begin{proposition}	\label{PropInterssiConn}
	Une partie de $\eR$ est connexe si et seulement si c'est un intervalle.
\end{proposition}

\begin{proof}
	La preuve est en deux partie. D'abord nous démontrons que si un sous-ensemble de $\eR$ est connexe, alors c'est un intervalle; et ensuite nous démontrons que tout intervalle est connexe. Je te préviens que les deux parties sont difficiles, alors ouvres bien grand tes oreilles.

	Affin de prouver qu'un ensemble connexe est toujours un intervalle, nous allons prouver que si un ensemble n'est pas un intervalle, alors il n'est pas connexe. Prenons $A$, une partie de $\eR$ qui n'est pas un intervalle. Il existe donc $a$, $b\in A$ et un $x_0$ entre $a$ et $b$ qui n'est pas dans $A$. Comme le but est de prouver que $A$ n'est pas connexe, il faut couper $A$ en deux ouverts disjoints. L'élément $x_0$ qui n'est pas dans $A$ est le bon candidat pour effectuer cette coupure. Prenons $M$, un majorant de $A$ et $m$, un minorant de $A$, et définissons (voir figure \ref{FigChoixabxz})
	\begin{align*}
		\mO_1&=]m,x_0[\\
		\mO_2&=]x_0,M[.
	\end{align*}
	Si $A$ n'a pas de minorant, nous remplaçons la définition de $\mO_1$ par $]-\infty,x_0[$, et si $A$ n'a pas de majorant, nous remplaçons la définition de $\mO_2$ par $]x_0,\infty[$. Dans tous les cas, ce sont deux ensembles ouverts dont l'union recouvre tout $A$. En effet, $\mO_1\cup \mO_2$ contient tous les nombres entre un minorant de $A$ et un majorant sauf $x_0$, mais on sait que $x_0$ n'est pas dans $A$. Cela prouve que $A$ n'est pas connexe.
	\begin{figure}[ht]
	\centering
	\begin{pspicture}(-3.5,-0.5)(7.5,0.7)
	   %\psframe[linecolor=cyan](-3.5,-0.5)(7.5,0.7)
	   \psset{PointSymbol=none,PointName=none}
	   \pstGeonode[PointSymbol=*](0,0){Pa}(5,0){Pb}

	% Une petite manip pour élargir le segment Pa,Pb de façon symétrique en ne devant taper qu'une seule fois le facteur. 
	%   Le résultat est La,Lb
	   \pstHomO[HomCoef=1.3]{Pb}{Pa}[La]
	   \pstTransHom{La}{Pa}{Pb}{1}{Lb}

		\psline(La)(Lb)
		\pstMarquePoint{La}{0.3;180}{$A$}
		\pstMarquePoint{Pa}{0.3;90}{$a$}
		\pstMarquePoint{Pb}{0.3;90}{$b$}

	   \pstGeonode(0,0){O}(-3,0.5){Ea}(2,0.5){Eb}			% Les tailles des éllipses, plus un point O en (0,0).
		\psellipse[linecolor=red](Pa)(Ea)
		\psellipse[linecolor=blue](Pb)(Eb)
	   \pstGeonode[PointSymbol=*](3,0){Xz}
		\pstMarquePoint{Xz}{0.4;-90}{$x_0$}

	   \pstTransHom{O}{Eb}{Pb}{1}{mOb}
		\pstMarquePoint{mOb}{0,0}{ {\blue $\mO_2$}}
	   \pstTransHom{O}{Ea}{Pa}{1}{mOa}
		\pstMarquePoint{mOa}{0,0}{ {\red $\mO_1$}}

	\end{pspicture}

	\caption{Le point $x_0$ et les ouverts qui coupent en deux la partie $A$. Le fait déterminant dans la démonstration est que $x_0$ se trouve à la frontière entre $\mO_1$ et $\mO_2$}  \label{FigChoixabxz}
	\end{figure}

	Jusqu'à présent nous avons prouvé que si un ensemble n'est pas un intervalle, alors il ne peut pas être connexe. Pour remettre les choses à l'endroit, prenons un ensemble connexe, et demandons-nous si il peut être autre chose qu'un intervalle ? La réponse est \emph{non} parce que si il était autre chose, il ne serait pas connexe.

	Prouvons à présent que tout intervalle est connexe. Pour cela, nous refaisons le coup de \href{http://fr.wikipedia.org/wiki/Contraposée}{la contraposée}. Nous allons donc prendre une partie $A$ de $\eR$, supposer qu'elle n'est pas connexe et puis prouver qu'elle n'est alors pas un intervalle. Nous avons deux ouverts disjoints $\mO_1$ et $\mO_2$ tels que $A\subset \mO_1\cup \mO_2$. Prenons $a\in A_1$ et $b\in A_2$. Pour fixer les idées, on suppose que $a<b$. Maintenant, le jeu est de montrer qu'il existe une point $x_0$ entre $a$ et $b$ qui ne soit pas dans $A$ (cela montrerait que $A$ n'est pas un intervalle). Nous allons prouver que c'est le cas du point
	\[ 
	  x_0=\sup\{ x\in\mO_1\tq x<b \}.
	\]
	Étant donné que l'ensemble $\mA=\{ x\in\mO_1\tq x<b \}$ est ouvert\footnote{C'est l'intersection entre l'ouvert $\mO_1$ et l'ouvert $\{x\tq x<b \}$.}, le point $x_0$ n'est pas dans l'ensemble par le lemme \ref{LemSupOuvPas}. Nous avons donc
	\begin{itemize}
		\item soit $x_0$ n'est pas dans $\mO_1$,
		\item soit $x_0\leq b$,
		\item soit les deux en même temps.
	\end{itemize}
	Nous allons montrer qu'un tel $x_0$ ne peut pas être dans $A$. D'abord, remarquons que $\sup\mA\leq\sup\mO$ parce que $\mA$ est une intersection de $\mO$ avec quelque chose. Ensuite, il n'est pas possible que $x_0$ soit dans $\mO_2$ parce que tout élément de $\mO_2$ possède un voisinage contenu dans $\mO_2$. Un point de $\mO_2$ est donc toujours strictement plus grand que le supremum de $\mO_1$.

	Maintenant, remarque que si $x_0\leq b$, alors $x_0=b$, sinon $b$ serait un majorant de $\mA$ plus petit que $x_0$, ce qui n'est pas possible vu que $x_0$ est le supremum de $\mA$ et donc le plus petit majorant. Oui mais si $x_0=b$, c'est que $x_0\in\mO_2$, ce qu'on vient de montrer être impossible. Nous voila déjà débarrassé des deuxièmes et troisième possibilités. 

	Si la première possibilité est vraie, alors $x_0$ n'est pas dans $A$ parce qu'on a aussi prouvé que $x_0\notin\mO_2$. Or n'être ni dans $\mO_1$ ni dans $\mO_2$ implique de ne pas être dans $A$. Ce point $x_0=\sup\mA$ est donc hors de $A$.

	Oui, mais comme $a\in\mA$, on a obligatoirement que $x_0\geq a$. Mais par construction, on a aussi que $x_0\leq b$ (ici, l'inégalité est même stricte, mais ce n'est pas important). Donc
	\[ 
	  a\leq x_0\leq b
	\]
	avec $a$, $b\in A$, et $x_0\notin A$. Cela finit de prouver que $A$ n'est pas un intervalle.
\end{proof}

% This is part of Un soupçon de physique, sans être agressif pour autant
% Copyright (C) 2006-2009
%   Laurent Claessens
% See the file fdl-1.3.txt for copying conditions.


\section{Continuité}
%+++++++++++++++++++

Nous allons considérer trois approches différentes de la continuité. La première sera de définir la continuité de fonctions de $\eR$ vers $\eR$ au moyen du critère usuel. Ensuite, nous définiront la continuité des applications entre n'importes quels espaces métriques, et nous montrerons que les deux définitions sont équivalentes dans le cas des fonctions sur $\eR$ à valeurs réelles.

Enfin, un peu plus tard nous verrons que la continuité peut également être vue en termes de limites. Encore une fois nous verrons que dans le cas de fonctions de $\eR$ vers $\eR$ cette troisième approche est équivalentes aux deux premières.

\subsection{Approche analytique}
%-------------------------------

Une question qu'on peut se poser, c'est de savoir quand le graphe d'une fonction peut être dessiné sans lever le crayon. Regarde les exemples de la figure \ref{FigUncontDeuxpasC}. Le premier graphe semble pouvoir être dessiné sans lever son crayon. C'est une courbe en un seul morceau. La seconde par contre ne peut pas être dessinée sans lever le crayon. Elle est en deux morceaux.
\begin{figure}[ht] 
\centering
\subfigure[Cette fonction peut manifestement être tracée au crayon sans lever la main.]{%
\begin{pspicture}(-2.9,-1)(3.2,4.2)
  \psaxes[dotsep=1pt]{->}(0,0)(-2.9,0)(2.9,3.5)
	%\psframe[linecolor=yellow](-2.9,-1)(3.2,4.2)
	\psset{PointSymbol=none, PointName=none}

   %\pstGeonode[PosAngle={90,90,90}, CurveType=curve]
    %              (-3,1){A}(-2,2){B}(0,0.5){C}(1,1){D}(2,2){F}(3,0){G}
   \pstGeonode[PosAngle={90,90,90}, CurveType=curve]
                  (-3,1){A}(-2,2){B}(0,0.5){C}(2,2){F}(3,0){G}
\end{pspicture}
}
%
\subfigure[Cette fonction ne peut pas être tracée au crayon sans lever la main. Le point noir signifie que $f(1)=2$, et non $1$.]{%
\begin{pspicture}(-2.9,-1)(3.2,4.2)
  \psaxes[dotsep=1pt]{->}(0,0)(-2.9,0)(2.9,3.5)
	%\psframe[linecolor=yellow](-2.9,-1)(3.2,4.2)
	\psset{PointSymbol=none, PointName=none}

   \pstGeonode[PosAngle={90,90,90}, CurveType=curve, dotscale=2, PointSymbol={none,none,none,none}]
                  (-3,1){A}(-2,2){B}(0,0.5){C}(1,1){D}
   \pstGeonode[PosAngle={90,90,90}, CurveType=curve, dotscale=2,PointSymbol={none,none,none}]
                  (1,2){E}(2,3){F}(3,1){G}
	\psline[linestyle=dotted](D)(E)
	\pscircle[fillstyle=solid,fillcolor=white,linecolor=black](D){0.1}					% Le 0.1 est la taille du point. Si on change l'un, il faut changer l'autre.
	\pscircle[fillstyle=solid,fillcolor=black,linecolor=black](E){0.1}					% Par «l'autre», je veux dire : celui-ci.
\end{pspicture}	\label{subFigdiscontpasC}
}
%
\caption{Exemple d'une fonction continue et d'une fonction non continue.}\label{FigUncontDeuxpasC}
\end{figure}

Demandons nous ce exactement quelle propriété a la courbe de la figure \ref{subFigdiscontpasC} pour ne pas pouvoir être tracée sans lever le crayon. Mettons que tu dessines de gauche à droite. Dans ce cas, au fur et à mesure que tu avances en t'approchant de $x=1$, la courbe te mène vers le point $(1,1)$ tandis qu'en réalité, $f(1)=2$. C'est à dire que quand $x$ s'approche de $1$, eh bien $f(x)$ ne s'approche pas de $f(1)$.

Nous allons donc dire qu'une fonction est continue quand plus $x$ s'approche de $a$ en suivant la courbe, plus $f(x)$ s'approche de $f(a)$. Voici la définition précise.

\begin{definition}		\label{DefContinue}
Nous disons que la fonction $x\mapsto f(x)$ est \defe{continue en $a$}{Continue} si
\begin{equation}
 \forall \epsilon>0,\exists \delta\text{ tel que } \big(| x-a |\leq\delta\big)\Rightarrow | f(x)-f(a) |\leq \epsilon.
\end{equation}

\end{definition}

Nous allons maintenant étudier quelque conséquences de cette définition. Je te préviens tout de suite qu'il va y avoir certaines conséquences qui ne collent pas bien avec l'intuition d'un graphe qu'on peut tracer sans lever le crayon.

\begin{enumerate}
\item D'abord on voit que la continuité n'a été définie qu'en un point. On peut dire que la fonction $f$ est continue \emph{en tel point donné}, mais nous n'avons pas dit ce qu'est une fonction continue \emph{dans son ensemble}.

\item Si $I$ est un intervalle de $\eR$, on dit que $f$ est \defe{continue sur l'intervalle}{Continuité sur un intervalle} $I$ si elle est continue en chaque point de $I$.

\item Comme la définition de $f$ continue en $a$ fait intervenir $f(x)$ pour tous les $x$ pas trop loin de $a$, il faut au moins déjà que $f$ soit définie sur ces $x$. En d'autres termes, dire que $f$ est continue en $a$ demande que $f$ existe sur un intervalle autour de $a$. 

Ceci couplé à la définition précédente laisse penser qu'il est surtout intéressant d'étudier les fonctions qui sont continues sur un intervalle.

\item L'intuition comme quoi une fonction continue doit pouvoir être tracée sans lever la main correspond aux fonctions continues sur des intervalles. Au moins sur l'intervalle où elle est continue, elle est traçable en un morceau.
\end{enumerate}


Nous allons démontrer maintenant une série de petits résultats qui permettent de simplifier la démonstration de la continuité de toute une série de fonctions.
\begin{theorem}
Si la fonction $f$ est continue au point $a$, alors la fonction $\lambda f$ est également continue en $a$.
\end{theorem}

\begin{proof}
Soit $\epsilon>0$. Nous avons besoin d'un $\delta>0$ tel que pour chaque $x$ à moins de $\delta$ de $a$, la fonction $\lambda f$ soit à moins de $\epsilon$ de $(\lambda f)(a)=\lambda f(a)$. Étant donné que la fonction $f$ est continue en $a$, on sait déjà qu'il existe un $\delta_1$ (nous notons $\delta_1$ affin de ne pas confondre ce nombre dont on est sûr de l'existence avec le $\delta$ que nous sommes en train de chercher) tel que 
\[ 
  (| x-a |\leq \delta_1)\Rightarrow | f(x)-f(a) |\leq \epsilon_1.
\]
Hélas, ce $\delta_1$ n'est pas celui qu'il faut faut parce que nous travaillons avec $\lambda f$ au lieu de $f$, ce qui fait qu'au lieux d'avoir $| f(x)-f(a) |$, nous avons $| \lambda f(x)-\lambda f(a) |=| \lambda |\cdot | f(x)-f(a) |$.  Ce que $\delta_1$ fait avec $(\lambda f)$, c'est
\[ 
  (| x-a |\leq\delta_1)\Rightarrow  | (\lambda f)(x)- (\lambda f)(a)|\leq | \lambda |\epsilon_1.
\]
Ce que nous apprend la continuité de $f$, c'est que pour chaque choix de $\epsilon_1$, on a un $\delta_1$ qui fait cette implication. Comme cela est vrai pour chaque choix de $\epsilon_1$, essayons avec $\epsilon_1=\epsilon/| \lambda |$ pour voir ce que ça donne. Nous avons donc un $\delta_1$ qui fait
\[ 
  (| x-a |\leq\delta_1)\Rightarrow  | (\lambda f)(x)- (\lambda f)(a)|\leq | \lambda |\epsilon_1=\epsilon.
\]
Ce $\delta_1$ est celui qu'on cherchait. 
\end{proof}

\begin{theorem}
Si $f$ et $g$ sont deux fonctions continues en $a$, alors la fonction $f+g$ est également continue en $a$.
\end{theorem}

\begin{proof}
La continuité des fonctions $f$ et $g$ au point $a$ fait en sorte que pour tout choix de $\epsilon_1$ et $\epsilon_2$, il existe $\delta_1$ et $\delta_2$ tels que 
\[ 
  (| x-a |\leq \delta_1)\Rightarrow | f(x)-f(a) |\leq \epsilon_1.
\]
et
\[ 
  (| x-a |\leq \delta_2)\Rightarrow | g(x)-g(a) |\leq \epsilon_2.
\]
La quantité que nous souhaitons analyser est $| f(x)+g(x)-f(a)-g(a) |$. Tout le jeu de la démonstration de la continuité est de triturer cette expression pour en tirer quelque chose en termes de $\epsilon_1$ et $\epsilon_2$. Si nous supposons avoir prit $| x-a |$ plus petit en même temps que $\delta_1$ et que $\delta_2$, nous avons
\[
| f(x)+g(x)-f(a)-g(a) |\leq| f(x)-g(x) |+| g(x)-g(a) |\leq\epsilon_1+\epsilon_2 
\]
en utilisant la formule générale $| a+b |\leq | a |+| b |$. Maintenant si on choisit $\epsilon_1$ et $\epsilon_2$ tels que $\epsilon_1+\epsilon_2<\epsilon$, et les $\delta_1$, $\delta_2$ correspondants, on a que 
\[
| f(x)+g(x)-f(a)-g(a) |\leq\epsilon,
\]
pourvu que $| x-a |$ soit plus petit que $\delta_1$ et $\delta_2$. Le bon $\delta$ a prendre est donc le minimum de $\delta_1$ et $\delta_2$ qui eux-même sont donnés par un choix de $\epsilon_1$ et $\epsilon_2$ tels que $\epsilon_1+\epsilon_2\leq\epsilon$.
\end{proof}

Pour résumer ces deux théorèmes, on dit que si $f$ et $g$ sont continues en $a$, alors la fonction $\alpha f+\beta g$ est également continue en $a$ pour tout $\alpha$, $\beta\in\eR$.

Parmi les propriétés immédiates de la continuité d'une fonction, nous avons ceci qui est souvent bien utile.

\begin{corollary}
Si la fonction $f$ est continue en $a$ et si $f(a)>0$, alors $f$ est positive sur un intervalle autour de $a$.
\end{corollary}

\begin{proof}
Prenons $\epsilon<f(a)$ et voyons\footnote{ici, nous insistons sur le fait que nous prenons $\epsilon$ \emph{strictement} plus petit que $f(a)$.} ce que la continuité de $f$ en $a$ nous offre : il existe un $\delta$ tel que
\[ 
  (| x-a |\leq \delta)\Rightarrow | f(x)-f(a) |\leq\epsilon < f(a).
\]
Nous en retenons que sur un intervalle (de largeur $\delta$), nous avons $| f(x)-f(a) |\leq f(a)$. Par hypothèse, $f(a)>0$, donc si $f(x)<0$, alors la différence $f(x)-f(a)$ donne un nombre encore plus négatif que $-f(a)$, c'est à dire que $| f(x)-f(a) |>f(a)$, ce qui est contraire à ce que nous venons de démontrer. D'où la conclusion que $f(x)>0$.
\end{proof}


\subsection{La fonction la moins continue du monde}
%--------------------------------------------------

Si tu veux des exemples de fonctions qui ne sont pas continues, c'est pas compliqué : dessine n'importe quoi qui fait un saut comme sur la figure \ref{subFigdiscontpasC}. Mais il y a moyen de donner des exemples de fonctions encore plus sales. Celle-ci par exemple :
\[ 
  \chi_{\eQ}(x)=
\begin{cases}
	1 \text{ si $x\in\eQ$}\\
	0 \text{ sinon.}
\end{cases}
\]
Par exemple, $\chi_{\eQ}(0)=1$, et\footnote{Pour prouver que $\sqrt{2}$ n'est pas rationnel, c'est pas trop compliqué, mais pour prouver que $\pi$ ne l'est pas non plus, tu devras encore manger de la soupe.} $\chi_{\eQ}(\pi)=\chi_{\eQ}(\sqrt{2})=0$. Malgré que $\chi_{\eQ}(0)=1$, il n'existe \emph{aucun} voisinage de $1$ sur lequel la fonction reste proche de $1$, parce que tout voisinage va contenir au moins un irrationnel. À chaque millimètre, cette fonction fait une infinité de bonds !

Cette fonction n'est donc continue nulle part. 

Tu sais que l'interprétation usuelle de la continuité est la capacité à pouvoir dessiner la fonction sans lever le crayon. Donc il te semblerait logique que si une fonction est continue en un point, elle soit au mois plus ou moins sympathique autour du point. En fait, ce que tu espères secrètement, c'est que si une fonction est continue en un point, alors elle est continue au moins sur un voisinage du point. Hélas, cela est faux : regarde la fonction
\[ 
  f(x)=x\chi_{\eQ}(x)=
\begin{cases}
x\text{ si $x\in\eQ$}\\
0\text{ sinon.}
\end{cases}
\]
Cette fonction est continue en zéro. En effet, prenons $\delta>0$; il nous faut un $\epsilon$ tel que $| x |\leq\epsilon$ implique $f(x)\leq \delta$ parce que $f(0)=0$. Bon ben prendre simplement $\epsilon=\delta$ nous contente. Cette fonction est donc très facilement continue en zéro.

Et pourtant, dès que l'on s'écarte un tant soit peu de zéro, elle fait des bons une infinité de fois par millionième de millimètre ! Cette fonction est donc la plus discontinue du monde en tous les points saut un (zéro) où elle est une fonction continue !

Oui, les math recèlent quelque exemples de monstres de ce type qui heurtent l'intuition et qui nous rappellent qu'il faut être très prudent. Tant qu'on n'a pas une démonstration complète d'un fait, des choses incroyables peuvent arriver.

\subsection{Approche topologique}
%--------------------------------

Nous avons vu que sur tout ensemble métrique, nous pouvons définir ce qu'est un ouvert : c'est un ensemble qui contient une boule ouverte autour de chacun de ses points. Quand on est dans un ensemble ouvert, on peut toujours un peu se déplacer sans sortir de l'ensemble.

Le théorème suivant est une très importante caractérisation des fonctions continues (de $\eR$ dans $\eR$) en termes de topologie, c'est à dire en termes d'ouverts.

\begin{theorem}		\label{ThoContInvOuvert}
Si $I$ est un intervalle ouvert contenu dans $\dom f$, alors $f$ est continue sur $I$ si et seulement si pour tout ouvert $\mO$ dans $\eR$, l'image inverse $f|_I^{^{-1}}(\mO)$ est ouvert.
\end{theorem}

Par abus de langage, nous exprimons souvent cette condition par \og une fonction est continue si et seulement si l'image inverse de tout ouvert est un ouvert\fg.

\begin{proof}

Dans un premier temps, nous allons transformer le critère de continuité en termes de boules ouvertes, et ensuite, nous passeront à la démonstration proprement dite. Le critère de continuité de $f$ au point $x$ dit que
\begin{equation}		\label{EqDEfCOntAn}
  \forall \delta>0,\exists\,\epsilon>0\text{ tel que }\big( | x-a |< \epsilon \big)\Rightarrow| f(x)-f(a) |<\delta.
\end{equation}
Cette condition peut être exprimée sous la forme suivante :
\[ 
  \forall \delta>0,\exists\epsilon\text{ tel que } a\in B(x,\epsilon)\Rightarrow f(a)\in B\big( f(x),\delta \big),
\]
ou encore
\begin{equation}		\label{EqRedefContBoules}
  \forall \delta>0,\exists\epsilon\text{ tel que } f\big( B(x,\epsilon) \big)\subset B\big( f(x),\delta \big).
\end{equation}
Jusque ici, nous n'avons fait que du jeu de notations. Nous avons exprimé en termes de topologie des inégalités analytiques. Si tu veux, tu peux retenir cette condition \eqref{EqRedefContBoules} comme définition d'une fonction continue en $x$. Si tu choisit de vivre comme ça, tu dois être capable de retrouver \eqref{EqDEfCOntAn} à partir de \eqref{EqRedefContBoules}.
 
Passons maintenant à la démonstration proprement dite du théorème. Comme quasiment toutes les démonstrations de \og si et seulement si\fg, cette démonstrations est en deux parties. Une dans chaque sens.

D'abord, supposons que $f$ est continue sur $I$, et prenons $\mO$, un ouvert quelconque. Le but est de prouver que $f|_I^{-1}(\mO)$ est ouvert. Pour cela, nous prenons un point $x_0\in f|_I^{-1}(\mO)$ et nous allons trouver un ouvert autour ce ce point contenu dans $f|_I^{-1}(\mO)$. Nous écrivons $y_0=f(x_0)$. Évidement, $y_0\in\mO$, donc on a une boule autour de $y_0$ qui est contenue dans $\mO$, soit donc $\delta>0$ tel que
\[  
  B(y_0,\delta)\subset\mO.
\]
Par hypothèse, $f$ est continue en $x_0$, et nous pouvons donc y appliquer le critère \eqref{EqRedefContBoules}. Il existe donc $\epsilon>0$ tel que 
\[ 
  f\big( B(x_0,\epsilon) \big)\subset B\big( f(x_0),\delta \big)\subset\mO.
\]
Cela prouve que $B(x_0,\epsilon)\subset f|_I^{-1}(\mO)$.

Dans l'autre sens, maintenant. Nous prenons $x_0\in I$ et nous voulons prouver que $f$ est continue en $x_0$, c'est à dire que pour tout $\delta$ nous cherchons un $\epsilon$ tel que $f\big( B(x_0,\epsilon) \big)\subset B\big( f(x_0),\delta \big)$. Oui, mais $B\big( f(x_0),\delta \big)$ est ouverte, donc par hypothèse, $f|_I^{-1}\Big( B\big( f(x_0),\delta \big) \Big)$ est ouvert, inclue à $I$ et contient $x_0$. Donc il existe un $\epsilon$ tel que
\[ 
  B(x_0,\epsilon)\subset f|_I^{-1}\Big( B\big( f(x_0),\delta \big) \Big),
\]
et donc tel que 
\[ 
  f\big( B(x_0,\epsilon) \big)\subset B\big( f(x_0),\delta \big),
\]
ce qu'il fallait prouver.
\end{proof}


Avant de démontrer le théorème des valeurs intermédiaires, nous avons encore besoin d'un petit lemme.
\begin{lemma}	\label{LemConncontconn}
L'image d'un ensemble connexe par une fonction continue est connexe.
\end{lemma}

\begin{proof}
Tu sais quoi ? Nous allons encore faire la contraposée. Soit $A$ une partie de $\eR$ telle que $f(A)$ ne soit pas connexe. Nous allons prouver que $A$ elle-même n'est pas connexe. Dire que $f(A)$ n'est pas connexe, c'est dire qu'il existe $\mO_1$ et $\mO_2$, deux ouverts disjoints qui recouvrent $f(A)$. Je prétends que $f^{-1}(\mO_1)$ et $f^{-1}(\mO_2)$ sont ouverts, disjoints et qu'ils recouvrent $A$.
\begin{itemize}
\item Ces deux ensembles sont ouverts parce qu'ils sont images inverses d'ouverts par une fonction continue (théorème \ref{ThoContInvOuvert}).
\item Si $x\in f^{-1}(\mO_1)\cap f^{-1}(\mO_2)$, alors $f(x)\in \mO_1\cap\mO_2$, ce qui contredirait le fait que $\mO_1$ et $\mO_2$ sont disjoints. Il n'y a donc pas d'éléments dans l'intersection de $f^{-1}(\mO_1)$ et de $f^{-1}(\mO_2)$.
\item Si $f^{-1}(\mO_1)$ et $f^{-1}(\mO_2)$ ne recouvrent pas $A$, il existe un $x$ dans $A$ qui n'est dans aucun des deux. Dans ce cas, $f(x)$ est dans $f(A)$, mais n'est ni dans $\mO_1$, ni dans $\mO_2$, ce qui contredirait le fait que ces deux derniers recouvrent $f(A)$.
\end{itemize}
Nous déduisons que $A$ n'est pas connexe. Et donc le lemme.
\end{proof}

\begin{theorem}[Théorème des valeurs intermédiaires]		\label{ThoValInter}
Soit $f$, une fonction continue sur $[a,b]$, et supposons que $f(a)<f(b)$. Alors pour tout $y$ tel que $f(a)\leq y\leq f(b)$, il existe un $x$ entre $a$ et $b$ tel que $f(x)=y$.
\end{theorem}

\begin{proof}
Nous savons que $[a,b]$ est connexe pare que c'est un intervalle (proposition \ref{PropInterssiConn}). Donc $f\big( [a,b] \big)$ est connexe (lemme \ref{LemConncontconn}) et donc est un intervalle (à nouveau la proposition \ref{PropInterssiConn}). Étant donné que $f\big( [a,b] \big)$ est un intervalle, il contient toutes les valeurs intermédiaires entre n'importe quels deux de ses éléments. En particulier toutes les valeurs intermédiaires entre $f(a)$ et $f(b)$.
\end{proof}

\begin{corollary}		\label{CorImInterInter}
L'image d'un intervalle par une fonction continue est un intervalle.
\end{corollary}
La preuve est laissée à titre d'exercice.

Intuitivement, ce théorème est évident : si tu veux tracer une courbe qui commence en $-1$ et qui finit en $4$ sans lever ton crayon, tu devras bien passer par $0$, $1$, $3$, $3.56123$ et tous les intermédiaires. Mais comme tu le vois, ce résultat intuitivement évident est plutôt compliqué à prouver : ça demande des tonnes de subtilités en topologie\footnote{Il existe une preuve qui ne fait pas appel à la topologie, mais qui demande de savoir des propriétés avancées des limites de suites.}.
\begin{figure}
\centering
\begin{pspicture}(-0.5,-0.5)(5.5,5.5)
   %\psframe[linecolor=cyan](-0.5,-0.5)(5.5,5.5)
   \psset{PointName=none}
	\pstGeonode(0,0){A}(5,5){B}
   \psset{PointSymbol=none,PointName=none}
   \pstGeonode(0,3){Pg}(5,3){Pd}
	\psline[linecolor=green](Pg)(Pd)	
	\pscurve[linecolor=red](A)(4,1)(B)
	\pscurve[linecolor=cyan](A)(1,4)(B)
	\pscurve[linecolor=blue](A)(1,2)(2,1)(3,3.5)(4,3)(B)
	\pstMarquePoint{A}{0.3;-90}{$A$}
	\pstMarquePoint{B}{0.3;90}{$B$}
	\cnode[fillstyle=solid,fillcolor=black](A){0.5mm}{bla}
	\cnode[fillstyle=solid,fillcolor=black](B){0.5mm}{blo}
\end{pspicture}

\caption{Comment passer du point $A$ au point $B$ sans couper la ligne verte ? Avec une fonction continue, ce n'est pas possible.}  \label{FigContiValInter}
\end{figure}

\subsection{Exercices}
%---------------------

\Exo{210}
\Exo{211}
\Exo{209}
\Exo{208}


\subsection{Continuité de la racine carré}
%-----------------------------------------

Pourquoi nous intéresser particulièrement à cette fonction ? Parce qu'elle a une sale condition d'existence : son domaine de définition n'est pas ouvert. Or dans tous les théorèmes de continuité d'approche topologique que nous avons vus, nous avons donné des contions \emph{pour tout ouvert}. Nous nous attendons donc a avoir des difficultés avec la continuité de $\sqrt{x}$ en zéro.

Prenons $I$, n'importe quel intervalle ouvert dans $\eR^+$, et voyons que la fonction
\begin{equation}
\begin{aligned}
 f\colon \eR^+&\to \eR^+ \\ 
   x&\mapsto \sqrt{x} 
\end{aligned}
\end{equation}
est continue sur $I$. Remarque déjà que si $I$ est un ouvert dans $\eR^+$, il ne peut pas contenir zéro. Avant de nous lancer dans notre propos, nous prouvons un lemme qui fera tout le travail\footnote{C'est toujours ingrat d'être un lemme : on fait tout le travail et c'est toujours le théorème qui est nommé.}.

\begin{lemma}
Soit $\mO$, un ouvert dans $\eR^+$. Alors $\mO^2=\{ x^2\tq x\in\mO \}$ est également ouvert .
\end{lemma}

\begin{proof}
Un élément de $\mO^2$ s'écrit sous la forme $x^2$ pour un certain $x\in\mO$. Le but est de trouver un ouvert autour de $x^2$ qui soit contenu dans $\mO^2$. Étant donné que $\mO$ est ouvert, on a une boule centrée en $x$ contenue dans $\mO$. Nous appelons $\delta$ le rayon de cette boule :
\[ 
  B(x,\delta)\subset\mO.
\]
Étant donné que cet ensemble est connexe, nous savons par le lemme \ref{LemConncontconn} que $B(x,\delta)^2$ est également connexe (parce que la fonction $x\mapsto x^2$ est continue). Son plus grand élément est $(x+\delta)^2=x^2+\delta^2+2x\delta>x^2+\delta^2$, et son plus petit élément est $(x-\delta)^2=x^2+\delta^2-2x\delta$. 

Ce qui serait pas mal, c'est que ces deux bornes entourent $x^2$, de telle façon à ce qu'elles définissent un ouvert autour de $x^2$ qui soit dans $\mO^2$. Hélas, c'est pas gagné que $x^2+\delta^2-2x\delta$ soit plus petit que $x^2$. 

Heureusement, en fait c'est vrai parce que d'une part, du fait que $\mO\subset\eR^+$, on a $x>0$, et d'autre part, pour que $\mO$ soit positif, il faut que $\delta<x$. Donc on a évidement que $\delta<2x$, et donc que
\[ 
  x^2+\delta^2-2x\delta=x^2+\delta\underbrace{(\delta-2x)}_{<0}<x^2.
\]
Donc nous avons fini : l'ensemble
\[ 
  B(x,\delta)^2=]x^2+\delta^2-2x\delta,x^2+\delta^2+2x\delta[\subset\mO^2
\]
est un intervalle qui contient $x^2$, et donc qui contient une boule ouverte centrée en~$x^2$.

\end{proof}

Maintenant nous pouvons nous attaquer à la continuité de la racine carré sur tout ouvert positif en utilisant le théorème \ref{ThoContInvOuvert}. Soit $\mO$ n'importe quel ouvert de $\eR$, et prouvons que $f|_I^{-1}(\mO)$ est ouvert. Par définition,
\begin{equation}
  f|_I^{-1}(\mO)=\{ x\in I\tq \sqrt{x}\in\mO \}.
\end{equation}
Maintenant c'est un tout petit effort que de remarquer que $f|_I^{-1}(\mO)=\mO^2\cap I$. De là, on a gagné parce que $\mO^2$ et $I$ sont des ouverts. Or l'intersection de deux ouverts est ouvert. 

Nous n'en avons pas fini avec la fonction $\sqrt{x}$. Ce que nous avons fait est représenté à la figure \ref{FigSqrtpqsz}. Nous avons la continuité de la racine carré pour tous les réels strictement positifs. Il reste à pouvoir dire que la fonction est continue en zéro malgré qu'elle ne soit pas définie sur un ouvert autour de zéro. Encore de la mauvaise foi de mathématicien en perspective.
\begin{figure}[ht]
\begin{center}
	\psset{xunit=0.06cm,yunit=0.5cm}
\begin{pspicture}(-10,-1.5)(210,16)
   \psset{PointSymbol=none, PointName=none}
	%\psframe[linecolor=cyan](-10,-1.5)(210,16)
  	\psaxes[dotsep=1pt, Dx=50, Dy=2]{->}(0,0)(0,0)(210,16)

   \def\Fn{x sqrt}	
	\psplot[linecolor=blue,plotpoints=1000]{0}{200}{\Fn}

   \pstGeonode(10,1){A}(190,1){B}(210,1){C}
	\psline[linecolor=green](A)(B)
	\psline[linecolor=green,linestyle=dotted](B)(C)
	\pscircle[linecolor=red,fillstyle=solid,fillcolor=white](A){0.1}

\end{pspicture}
\end{center}
\caption{Nous avons prouvé la continuité de $x\mapsto\sqrt{x}$ pour tous les intervalles du type de celui représenté en vert. Remarque que cet intervalle ne contient pas le premier point : il est ouvert à sa gauche. Mais ce premier point peut être en réalité aussi près que l'on veut de zéro, sans toutefois l'atteindre. Bref, il ne nous manque la continuité qu'en $0$. Ce serait frustrant que cette fonction ne soit juste pas continue en ce point hein ?}\label{FigSqrtpqsz}
\end{figure}

Il est possible de dire que la racine carré est continue en $0$, malgré qu'elle ne soit pas définie sur un ouvert autour de $0$\ldots en tout cas pas un ouvert au sens que tu as en tête. Nous allons rentabiliser un bon coup notre travail sur les espaces métriques.

Nous pouvons définir la notion de boule ouverte sur n'importe quel espace métrique $A$ en disant que
\[ 
  B(x,r)=\{ y\in A\tq d(x,y)<r \}.
\]
\begin{definition}		\label{DefContMetrique}
Soit $f\colon A\to B$, une application entre deux espaces métriques. Nous disons que $f$ est \defe{continue}{Continue!sur espace métrique} au point $a\in A$ si $\forall \delta>0$, $\exists\epsilon>0$ tel que 
\begin{equation}
  f\big( B(a,\epsilon) \big)\subset B\big( f(a),\delta \big).
\end{equation}
\end{definition}
Tu reconnais évidement la condition \eqref{EqRedefContBoules}. Nous l'avons juste recopiée. Tu remarqueras cependant que cette définition généralise immensément la continuité que l'on avait travaillé à propos des fonctions de $\eR$ vers $\eR$. Maintenant tu peux prendre n'importe quel espace métrique et c'est bon.

Nous n'allons pas faire un tour complet des conséquences et exemples de cette définition. Au lieu de cela, nous allons juste montrer en quoi cette définition règle le problème de la continuité de la racine carré en zéro.

La fonction que nous regardons est 
\begin{equation}
\begin{aligned}
f \colon \eR^+&\to \eR^+ \\ 
   x&\mapsto \sqrt{x}.
\end{aligned}
\end{equation}
Mais cette fois, nous ne la voyons pas comme étant une fonction dont le domaine est une partie de $\eR$, mais comme fonction dont le domaine est $\eR^+$ vu comme un espace métrique en soi. Quelles sont les boules ouvertes dans $\eR^+$ autour de zéro ? Réponse : la boule ouverte de rayon $r$ autour de zéro dans $\eR^+$ est :
\[ 
  B(0,r)_{\eR^+}=\{ x\in\eR^+\tq d(x,0)<r \}=[0,r[.  
\]
Cet intervalle est un ouvert. Aussi incroyable que cela puisse paraître !

Testons la continuité de la racine carré en zéro dans ce contexte. Il s'agit de prendre $A=\eR^+$, $B=\eR^+$ et $a=0$ dans la définition \ref{DefContMetrique}. Nous avons que $B(\sqrt{0},\delta)=B(0,\delta)=[0,\delta[$ pour la topologie de $\eR^+$.

Il s'agit maintenant de trouver un $\epsilon$ tel que $f\big( B(0,\epsilon) \big)\subset [0,\delta[$. Par définition, nous avons que
\[ 
  f\big( B(0,\epsilon) \big)=[0,\sqrt{\epsilon}[,
\]
le problème revient dont à trouver $\epsilon$ tel que $\sqrt{\epsilon}\leq\delta$. Prendre $\epsilon<\delta^2$ fait l'affaire.


Donc voila. Au sens de la \href{http://fr.wikipedia.org/wiki/Topologie_induite}{topologie propre} à $\eR^+$, nous pouvons dire que la fonction racine carré est partout continue.
\subsection{Limites en des nombres}
%----------------------------------

Si tu regardes la fonction $f(x)=5x+3$, tu ne serais pas étonnée si je te disais par exemple que 
\begin{align}
\lim_{x\to 10}f(x)&=53&\text{et}&\lim_{x\to 0}f(x)=3.
\end{align}
En effet, plus $x$ est proche de $10$, plus $f(x)$ est proche de $53$ et plus $x$ est proche de $0$, plus $f(x)$ est proche de $3$. Pas grand chose de neuf sous le Soleil.

 Oui, mais l'intérêt d'introduire le concept de limite dans le cas de l'infini était qu'on ne peut pas bêtement calculer $f(\infty)$. Il fallait donc une astuce pour parler du comportement de $f$ quand on s'approche de l'infini.

Ici, $f(10)=53$ et $f(0)=3$, donc on ne voit pas très bien pourquoi il faudrait s'inquiéter et introduire une notion de limite. Malheureusement, si tu prenais la peine de regarder encore une fois la figure \ref{FigUnSurx}, tu verrais qu'on a quand même besoin d'une astuce pour décrire ce qu'on voudrait dire  : la fonction $1/x$ tend vers l'infini quand $x$ tend vers zéro. Je sais que c'est ce que tu veux dire, mais nous allons voir qu'en réalité c'est faux\footnote{Surtout si ce que tu voulais dire est quelque chose du genre \og je m'en fous !\fg.}. Cela va faire l'objet d'une subtilité très marrante. Mais faisons comme si de rien n'était et posons la définition suivante :
\begin{definition}		\label{DefInfNombre}
Lorsque $a\in\eR$, on dit que la fonction $f$ \defe{tend vers l'infini quand $x$ tend vers $a$}{} si
\[ 
  \forall M\in\eR,\exists \delta\tq (| x-a |\leq \delta )\Rightarrow f(x)\geq M\text{ quand $x\in\dom f$}.
\]
\end{definition}
Cela signifie que l'on demande que dès que $x$ est assez proche de $a$ (c'est à dire dès que $| x-a |\leq\delta$), alors $f(x)$ est plus grand que $M$, et que l'on peut trouver un $\delta$ qui fait ça pour n'importe quel $M$. Une autre façon de le dire est que pour toute hauteur $M$, on peut trouver un intervalle de largeur $\delta$ autour de $a$\footnote{C'est à dire un intervalle de la forme $[a-\delta,a+\delta]$.} tel que sur cet intervalle, la fonction $f$ est toujours plus grande que $M$.

Montrons sur un dessin pourquoi je disais que la fonction $x\to 1/x$ n'est pas de ce type. Regarde donc la figure \ref{FigUnsurxContreEx}.
\begin{figure}[ht]
\centering
\psset{xunit=0.6,yunit=0.6}
\begin{pspicture}(-11,-5)(11,5)
  \psaxes[dotsep=1pt,Dy=2,Dx=2]{->}(0,0)(-10.9,-4.9)(11,5)
	\psset{PointSymbol=none, PointName=none}
	\def\Fn{1 x div}
	\psplot[linecolor=red]{0.2}{10.3}{\Fn}
	\psplot[linecolor=red]{-10.3}{-0.2}{\Fn}
\end{pspicture}
\caption{La fonction $f(x)=1/x$. Le problème est que ça descend d'un côté et que ça monte de l'autre.}\label{FigUnsurxContreEx}
\end{figure}
Le problème est qu'il n'existe par exemple aucun intervalle autour de $0$ sur lequel $f$ serait toujours plus grande que $10$. En effet n'importe quel intervalle autour de $0$ contient au moins un nombre négatif. Or quand $x$ est négatif, $f$ n'est certainement pas plus grande que $10$. Nous y reviendrons.

Pour l'instant, montrons que la fonction $f(x)=1/x^2$ de la figure \ref{FigUnSurxCarr} est une fonction qui vérifie la définition \ref{DefInfNombre}. 
\begin{figure}[ht]
\centering
\psset{xunit=0.6,yunit=0.6,Dx=2,Dy=2}
\begin{pspicture}(-11,-1)(11,11)
  \psaxes[dotsep=1pt]{->}(0,0)(-10.9,0)(11,10)
	%\psframe[linecolor=cyan](-11,-1)(11,11)
	\psset{PointSymbol=none, PointName=none}
	\def\Fn{1 x 2 exp div}
	\psplot[linecolor=red]{0.30151}{10}{\Fn}
	\psplot[linecolor=red]{-10}{-0.30151}{\Fn}
\end{pspicture}
\caption{La fonction $f(x)=1/x^2$. Elle monte bien vers l'infini quand $x$ tend vers zéro; tant du côté des négatifs que du côté des positifs.}\label{FigUnSurxCarr}
\end{figure}
Avant de prendre n'importe quel $M$, prenons par exemple $100$. Nous avons besoin d'un intervalle autour de zéro sur lequel $f$ est toujours plus grande que $100$. C'est vite vu que $f(0.1)=f(-0.1)=100$, donc l'intervalle $[-\frac{ 1 }{ 10 },\frac{1}{ 10 }]$ est le bon. Partout dans cet intervalle, $f$ est plus grande que $100$. Partout ? Ben non : en $x=0$, la fonction n'est même pas définie, donc c'est un peu dur de dire qu'elle est plus grande que $100$. C'est pour cela que nous avons ajouté la condition \og quand $x\in\dom f$\fg{} dans la définition de la limite.

Prenons maintenant un $M\in\eR$ arbitraire, et trouvons un intervalle autour de $0$ sur lequel $f$ est toujours plus grande que $M$. La réponse est évidement l'intervalle de largeur $1/\sqrt{M}$, c'est à dire 
\[ 
  \left[ -\frac{ 1 }{ \sqrt{M} },\frac{ 1 }{ \sqrt{M} } \right].
\]


\section{Limite et continuité}
%++++++++++++++++++++++++++++++


\subsection{Limites quand tout va bien}
%--------------------------------------

D'abord définissons ce qu'on entend par la limite d'une fonction en un point quand il n'y a aucun infini en jeu.
\begin{definition}		\label{DefLimPointSansInfini}
 On dit que la fonction $f$ \defe{tend vers $b$ quand $x$ tend vers $a$}{} si 
\[ 
  \forall \epsilon>0,\exists\delta\tq (| x-a |\leq\delta)\Rightarrow | f(x)-b |\leq \epsilon\text{ quand $x\in\dom f$}.
\]
Dans ce cas, nous notons
\begin{equation}
\lim_{x\to a}f(x)=b.
\end{equation} 
\end{definition}

Commençons par un exemple très simple : prouvons que $\lim_{x\to 0}x=0$. C'est donc $a=b=0$ dans la définition. Prenons $\epsilon>0$, et trouvons un intervalle autour de zéro tel que partout dans l'intervalle, $x\leq \epsilon$. Bon ben c'est clair que $\delta=\epsilon$ fonctionne.

Plus compliqué maintenant, mais toujours sans surprises.

\begin{proposition}
\[ 
  \lim_{x\to 0}x^2=0.
\]

\end{proposition}

\begin{proof}
Soit $\epsilon>0$. On veut un intervalle de largeur $\delta$ autour de zéro tel que $x^2$ soit plus petit que $\epsilon$ sur cet intervalle. Cette fois-ci, le $\delta$ qui fonctionne est $\delta=\sqrt{\epsilon}$. En effet un élément de l'intervalle $[-\delta,\delta]$ est un $r$ de valeur absolue plus petite ou égale à $\delta$ : 
\[ 
| r |\leq\delta=\sqrt{\epsilon}.
\]
En prenant le carré de cette inégalité on a :
\[ 
  r^2\leq\epsilon,
\]
ce qu'il fallait prouver.
\end{proof}

\Exo{204}

Calculer et prouver des valeurs de limites, mêmes très simples, devient vite de l'arrachage de cheveux à essayer de trouver le bon $\delta$ en fonction de $\epsilon$ si on n'a pas quelque théorèmes généraux. Nous allons donc maintenant en prouver quelque-uns.

\begin{theorem}		\label{ThoLimLinMul}
	Si
	\begin{equation} \label{Eqhypmullimlin}
	  \lim_{x\to a}f(x)=b,
	\end{equation}
	alors
	\begin{equation} \label{Eqbutmultlim}
	  \lim_{x\to a}(\lambda f)(x)=\lambda b
	\end{equation}
	pour n'importe quel $\lambda\in\eR$.
\end{theorem}

\begin{proof}
Soit $\epsilon>0$. Affin de prouver la propriété \eqref{Eqbutmultlim}, il faut trouver un $\delta$ tel que pour tout $x$ dans $[a-\delta,a+\delta]$, on ait $| (\lambda f)(x)- \lambda b |\leq\epsilon$. Cette dernière inégalité est équivalente à $|\lambda|| f(x)-b |\leq\epsilon$. Nous devons donc trouver un $\delta$ tel que 
\begin{equation} 
| f(x)-b |\leq\frac{ \epsilon }{ | \lambda | }.
\end{equation}
soit vraie pour tout $x$ dans $[a-\delta,a+\delta]$. Mais l'hypothèse \eqref{Eqhypmullimlin} dit précisément qu'il existe un $\delta$ tel que pour tout $x$ dans $[a-\delta,a+\delta]$ on ait cette inégalité. 
\end{proof}

\begin{theorem}		\label{ThoLimLin}
	Si
	\begin{subequations}
	\begin{align}
		\lim_{x\to a}f(x)&=b_1\\
		\lim_{x\to a}g(x)&=b_2,
	\end{align}
	\end{subequations}
	alors
	\begin{equation}
		\lim_{x\to a}(f+g)(x)=b_1+b_2.
	\end{equation}
\end{theorem}

\begin{proof}
	Soit $\epsilon>0$. Par hypothèse, il existe $\delta_1$ tel que
	\begin{equation}	\label{Eqfbunepsdeux}
	  | f(x)-b_1 |\leq \frac{ \epsilon }{ 2 }
	\end{equation}
	dès que $| x-a |\leq\delta_1$. Il existe aussi $\delta_2$ tel que 
	\begin{equation} 	\label{Eqgbdeuxepsdeux}
	  | g(x)-b_2 |\leq \frac{ \epsilon }{ 2 }.
	\end{equation}
	dès que $| x-a |\leq \delta_2$. Tu notes l'astuce de prendre $\epsilon/2$ dans la définition de limite pour $f$ et $g$. Maintenant, ce qu'on voudrait c'est un $\delta$ tel que l'on ait $| (f+g)(x)-(b_1+b_2) |\leq \epsilon$ dès que $| x-a |\leq \delta$. Moi je dit que $\delta=\min\{ \delta_1,\delta_2 \}$ fonctionne. En effet, en utilisant l'inégalité $| a+b |\leq | a |+| b |$, nous trouvons :
	\begin{align}
	| (f+g)(x)-(b_1+b_2) |=| (f(x)-b_1)+(g(x)-b_2) |
			\leq | f(x)-b_1 |+| g(x)-b_2 |.		\label{Eqfplusgfbun}
	\end{align}
	Comme on suppose que $| x-a |\leq\delta$, on a évidement $| x-a |\leq\delta_1$, et donc l'équation \eqref{Eqfbunepsdeux} tient. Mais si $| x-a |\leq\delta$, on a aussi $| x-a |\leq\delta_2$, et donc l'équation  \eqref{Eqfbunepsdeux} tient également. Chacun des deux termes de \eqref{Eqfplusgfbun} est donc plus petits que $\epsilon/2$, et donc le tout est plus petit que $\epsilon$, ce qu'il fallait montrer.

\end{proof}

Une formule qui résume ces deux théorèmes est que
\begin{equation}	\label{EqLimLinRes}
	\lim_{x\to a}[\alpha f(x)+\beta g(x)]=\alpha\lim_{x\to a}f(x)+\beta\lim_{x\to a}g(x).
\end{equation}

\begin{lemma}		\label{LemLimMajorableVois}
	Si $\lim_{x\to a}f(x)=b$ avec $a$, $b\in\eR$, alors il existe un $\delta>0$ et un $M>0$ tels que 
	\[ 
		(| x-a |\leq\delta)\Rightarrow | f(x) |\leq M.
	\]

\end{lemma}

Ce que signifie ce lemme, c'est que quand la fonction $f$ admet une limite finie en un point, alors il est possible de majorer la fonction sur un intervalle autour du point.

\begin{proof}
	Cela va être démontré par l'absurde. Supposons qu'il n'existe pas de $\delta$ ni de $M$ qui vérifient la condition. Dans ce cas, pour tout $\delta$ et pour tout $M$, il existe un $x$ tel que $| x-a |\leq\delta$ et $| f(x) |> M$. Cela est valable pour tout $M$, donc prenons par exemple $b+1000$. Donc 
	\begin{equation}
	\forall\delta>0,\exists x\text{ tel que } | x-a |\leq\delta\text{ et }| f(x) |>b+1000.
	\end{equation}
	Cela signifie qu'aucun $\delta$ ne peut convenir dans la définition de $\lim_{x\to a}f(x)=b$, ce qui contredit les hypothèses.
\end{proof}

Dans le même ordre d'idée, on peut prouver que si la limite de la fonction en un point est positive, alors elle est positive autour ce ce point. Plus précisément, nous avons la
\begin{proposition}	\label{PropoLimPosFPos}
	Si $f$ est une fonction telle que $\lim_{x\to a}f(x)>0$, alors il existe un voisinage de $a$ sur lequel $f$ est positive.
\end{proposition}	

\begin{proof}
	Supposons que $\lim_{x\to a}f(x)=y_0$. Par la définition de la limite fait que si pour tout $x$ dans un voisinage autour de $a$, on ait $| f(x)-a |<\epsilon$. Cela est valable pour tout $\epsilon$, pourvu que le voisinage soit assez petit. Si je choisit un voisinage pour lequel $| f(x)-a |<\frac{ y_0 }{ 2 }$, alors sur ce voisinage, $f$ est positive.
\end{proof}

Cette propoition ne devrait pas être sans te rappeler le théorème \ref{ThoValInter} des valeurs intermédiaires. Et en effet, c'est la même idée : si on sait que la fonction est positive en un point, on en déduit la positivité autour du point. Cette proposition est toutefois un peu plus forte parce que l'on ne suppose pas que la fonction soit continue.

\begin{theorem}		\label{Tholimfgabab}
	Si
	\begin{align}
		\lim_{x\to a}f(x)&=b_1&\text{et}&&\lim_{x\to a}g(x)=b_2,
	\end{align}
	alors
	\begin{equation}
		\lim_{x\to a}(fg)(x)=b_1b_2.
	\end{equation}
\end{theorem}

\begin{proof}
	Soit $\epsilon>0$, et tentons de trouver un $\delta$ tel que $| f(x)g(x)-b_1b_2 |\leq \epsilon$ dès que $| x-a |\leq \delta$. Nous avons 
	\begin{equation}	\label{EqfgbunbdeuxMin}
	\begin{split}
	| f(x)g(x)-b_1b_2 |&=|  f(x)g(x)-b_1b_2 +f(x)b_2-f(x)b_2 |\\
			&=\left|   f(x)\big( g(x)-b_2 \big)+b_2\big( f(x)-b_1 \big)    \right|\\
			&\leq \left|  f(x)\big( g(x)-b_2 \big)  \right|+\left|  b_2\big( f(x)-b_1 \big)    \right|\\
			&= | f(x) | | g(x)-b_2  |+| b_2 | |f(x)-b_1 |.	
	\end{split}
	\end{equation}
	À la première ligne se trouve la subtilité de la démonstration : on ajoute et on enlève\footnote{Comme exercice, tu peux essayer de refaire la démonstration en ajoutant et enlevant $g(x)b_1$ à la place.} $f(x)b_2$. Maintenant nous savons par le lemme \ref{LemLimMajorableVois} que pour un certain $\delta_1$, la quantité $| f(x) |$ peut être majoré par un certain $M$ dès que $| x-a |\leq \delta_1$. Prenons donc un tel $\delta_1$ et supposons que $| x-a |\leq \delta_1$. Nous savons aussi que pour n'importe quel choix de $\epsilon_2$ et $\epsilon_3$, il existe des nombres $\delta_2$ et $\delta_3$ tels que $| f(x)-b_1 |\leq \epsilon_2$ et $| g(x)-b_1 |\leq \epsilon_3$ dès que $| x-a |\leq\delta_2$ et $| x-a |\leq\delta_3$. Dans ces conditions, la dernière expression \eqref{EqfgbunbdeuxMin} se réduit à
	\begin{equation}
	| f(x)g(x)-b_1b_2 |\leq M\epsilon_2+| b_2 |\epsilon_3.
	\end{equation}
	Pour terminer la preuve, il suffit de choisir $\epsilon_2$ et $\epsilon_3$ tels que $M\epsilon_2+| b_2 |\epsilon_3\leq\epsilon$, et puis prendre $\delta=\min\{ \delta_1,\delta_2,\delta_3 \}$.


	Remetons les choses dans l'ordre. L'on se donne $\epsilon$ au départ. La première chose est de trouver un $\delta_1$ qui permet de majorer $|f(x)|$ par $M$ selon le lemme \ref{LemLimMajorableVois}, et puis choisissons $\epsilon_2$ et $\epsilon_3$ tels que $M\epsilon_2+| b_2 |\epsilon_3\leq\epsilon$. Ensuite nous prenons, en vertu des hypothèses de limites pour $f$ et $g$, les nombres $\delta_2$ et $\delta_3$ tels que $| f(x)-b_1 |\leq \epsilon_2$ et $| g(x)-b_2 |\leq \epsilon_3$ dès que $| x-a |\leq \delta_2$ et $| x-a |\leq \delta_3$.

	Si avec tous ça on prend $\delta=\min\{ \delta_1,\delta_2,\delta_3 \}$, alors la majoration et les deux inégalités sont valables en même temps et au final
	\[ 
	  | f(x)g(x)-b_1b_2 |\leq M\epsilon_2+b_2\epsilon_3\leq \epsilon,
	\]
	ce qu'il fallait prouver.

\end{proof}

À l'aide de ces petits résultats, nous pouvons déjà calculer pas mal de limites. Nous pouvons déjà par exemple calculer les limites de tous les polynomes en tous les nombrs réels. En effet, nous savons la limite de la fonction $f(x)=x$. la fonction $x\mapsto x^2$ n'est rien d'autre que le produit de $f$ par elle-même. Donc
\[ 
  \lim_{x\to a}x^2=\big( \lim_{x\to a}x\big)\cdot\big( \lim_{x\to a}x \big)=a^2.
\]
De la même façon, nous trouvons facilement que 
\begin{equation}
 \lim_{x\to a}x^n=a^n.
\end{equation}

\begin{exercice}
En continuant ce petit jeu, prouvez que $\lim_{x\to a}(x^3-5x+8)=a^3-5a+8$, et puis que si $P(x)$ est n'importe quel polynome, alors $\lim_{x\to a}P(x)=P(a)$.
\end{exercice}

Là, tu as un peu l'impression que l'on dit à tous les coups que $\lim_{x\to a}f(x)=f(a)$. D'ailleurs si tu relis la définition \ref{DefLimPointSansInfini}, tu vois que c'est un peu logique que la limite d'une fonction en un point soit la valeur de la fonction en ce point. Qu'est-ce qu'on a gagné alors ?

D'abord, est-ce que c'est toujours vrai que $\lim_{x\to a}f(x)=a$ ? Pour toute fonction ? Vraimment ? Eh bien non. Cela n'est pas vrai pour toutes les fonctions. Ce n'est vrai que pour une catégorie particulière de fonctions. L'important théorème suivant nous dit sans surprises pour quelles fonctions c'est vrai.

\begin{theorem}[Limite et continuité]			\label{ThoLimCont}
La fonction $f$ est continue au point $a$ si et seulement si $\lim_{x\to a}f(x)=f(a)$.
\end{theorem}

\begin{proof}
Nous commençons par supposer que $f$ est continue en $a$, et nous prouvons que $\lim_{x\to a}f(x)=a$. Soit $\epsilon>0$; ce qu'il nous faut c'est un $\delta$ tel que $| x-a |\leq\delta$ implique $| f(x)-f(a) |\leq\epsilon$. Relis la définition \ref{DefContinue} de la continuité, et tu verras que l'hypothèse de continuité est \emph{exactement} l'existence d'un $\delta$ comme il nous faut.

Dans l'autre sens, c'est à dire prouver que $f$ est continue au point $a$ sous l'hypothèse que $\lim_{x\to a}f(x)=f(a)$, la preuve se fait de la même façon.
\end{proof}

Nous en déduisons que si nous voulons gagner quelque chose à parler de limites, il faut prendre des fonctions non continues. Prenons une fonction qui fait un saut comme celle tracée à la figure \ref{subFigdiscontpasC}. Pour se fixer les idées, prenons celle-ci :
\begin{equation}
f(x)=
\begin{cases}
2x&\text{si $x\in]\infty,2[$}\\
x/2&\text{si $x\in[2,\infty[$}
\end{cases}
\end{equation}  
qui est représentée à la figure \ref{FigFnDiscDeux}. Essayons de trouver la limite de cette fonction lorsque $x$ tend vers $2$.
\begin{figure}
\centering
\begin{pspicture}(-1,-1)(8,5)
   %\psframe[linecolor=cyan](-1,-1)(8,5)
   \psset{PointSymbol=none,PointName=none}
	
	\psaxes(0,0)(-0.9,-0.9)(7.9,4.9)
   \def\Fn{2 x mul}
   \newcommand{\Gn}{x 2 div}
	\psplot{-0.5}{2}{\Fn}
	\psplot{2}{7}{\Gn}
   \pstGeonode(2,4){A}(2,1){B}
	\psline[linestyle=dashed](A)(B)
	
	\pscircle[fillstyle=solid,fillcolor=white,linecolor=black](A){0.1}				
	\pscircle[fillstyle=solid,fillcolor=black,linecolor=black](B){0.1}			
\end{pspicture}
\caption{Une fonction discontinue en $2$.}  \label{FigFnDiscDeux}
\end{figure}
Étant donné que $f$ n'est pas continue en $2$, nous savons déjà que $\lim_{x\to 2}f(x)\neq f(2)$. Donc ce n'est pas $1$. Cette limite ne peut pas valoir $4$ non plus parce que si je prends n'importe quel $\epsilon$, la valeur de $f(2+\epsilon)$ est très proche de $2$, et donc ne peut pas s'approcher de $4$. En fait, tu peux facilement vérifier que \emph{aucun nombre ne vérifie la condition de limite pour $f$ en $2$}. Nous disons que la limite n'existe pas.

Pour résumer, les limites qui ne font pas intervenir l'infini ne servent à rien parce que
\begin{itemize}
\item si la fonction est continue, la limite est simplement la valeur de la fonction par le théorème \ref{ThoLimCont},
\item si la fonction fait un saut, alors la limite n'existe pas (nous n'avons pas prouvé cela en général, mais avoue que l'exemple est convainquant).
\end{itemize}
Nous avons même la proposition suivante :
\begin{proposition}		\label{PropExisteLimVql}
Si $f$ existe en $a$ (c'est à dire si $a\in\dom(f)$) et si $\lim_{x\to a}f(x)=b$, alors $f(a)=b$.
\end{proposition}

\begin{proof}
Du fait que $\lim_{x\to a}f(x)=b$, il découle que pour tout $\epsilon$, il existe un $\delta$ tel que $| x-a |\leq \delta$ implique $| f(x)-b |\leq \epsilon$. Il est évident que pour tout $\delta$, $| x-x |\leq \delta$, donc nous avons que 
\[ 
  | f(a)-b |\leq\epsilon
\]
pour tout $\epsilon$. Cela implique que $f(a)=b$.
\end{proof}
Notons toutefois que l'inverse de cette proposition n'est pas vraie : la figure \ref{FigFnDiscDeux} montre justement une fonction qui prend la valeur $1$ en $2$ sans que la limite en $2$ soit $1$. Quoi qu'il en soit, cette proposition achève de nous convaincre de l'inutilité d'étudier d'étudier les limites sans infinis : dès qu'on a une limite, à tous les coups c'est la valeur de la fonction \ldots heu \ldots en es-tu bien sûr ?

\subsection{Limites et prolongement}
%-----------------------------------

À tous les coups ? Non ! La proposition \ref{PropExisteLimVql} a une terrible limitation : il faut que la fonction existe au point considéré. Or si tu regardes bien la définition \ref{DefLimPointSansInfini}, tu verras que $\lim_{x\to a}f(x)$ peut très bien exister sans que $f(a)$ n'existe.

Voici maintenant le bonus dont on parlait à la page \pageref{PgBonusLimite}\ldots bon d'accord, nous sommes à la page \label{PgIciBonus}\pageref{PgIciBonus}; le bonus est un peu cher !

Reprenons l'exemple de la fonction \eqref{EqOrdiRefuse} que mon ordinateur refusait de calculer en zéro :
\begin{equation}
f(x)=\frac{ x+4 }{ 3x^2+10x-8 }=\frac{ x+4 }{ (x+4)\left( x-\frac{ 2 }{ 3 } \right) }.
\end{equation}
Cette fonction a une condition d'existence en $x=-4$. Et pourtant, tant que $x\neq 4$, cela a un sens de simplifier les $(x+4)$ et d'écrire
\[ 
  f(x)=\frac{ 1 }{ x-\frac{ 2 }{ 3 } }=\frac{ 3 }{ 3x-2 }.
\]
Étant donné que pour toute valeur de $x$ différente de $-4$, la fonction $f$ s'exprime de cette façon, nous avons que
\[ 
  \lim_{x\to -4}f(x)=\lim_{x\to -4}\left(\frac{ 3 }{ 3x-2 }\right).
\]
Oui, mais la fonction\footnote{Cette fonction $g$ n'est pas $f$ parce que $g$ a en plus l'avantage d'être définie en $-4$.} $g(x)=3/(3x-2)$ est continue en $-4$ et donc sa limite vaut sa valeur. Nous en déduisons que
\[ 
  \lim_{x\to -4}f(x)=-\frac{ 3 }{ 14 }.
\]
Que dire maintenant de la fonction ainsi définie ?
\begin{equation}
\tilde f(x)=
\begin{cases}
f(x)&\text{si $x\neq -4$}\\
-3/14&\text{si $x=-4$}.
\end{cases}
\end{equation}
Cette fonction est continue en $-4$ parce qu'elle y est égale à sa limite. Les étapes suivies pour obtenir ce résultat sont :
\begin{itemize}
\item Repérer un point où la fonction n'existe pas,
\item calculer la limite de la fonction en ce point, et en particulier vérifier que cette limite existe, ce qui n'est pas toujours le cas,
\item définir une nouvelle fonction qui vaut partout la même chose que la fonction originale, sauf au point considéré où l'on met la valeur de la limite.
\end{itemize}
\begin{figure}
\centering
   \psset{PointSymbol=none,PointName=none,xunit=1cm,yunit=1cm}
\begin{pspicture}(-5.5,-4)(5.5,4)
   	%\psframe[linecolor=cyan](-5.5,-4)(5.5,4)
	\psaxes(0,0)(-5,-3.9)(5,3.9)
   \newcommand{\Fn}{x 4 add  x 2 exp 3 mul x 10 mul add 8 sub div }
	\psplot[linecolor=blue]{-5}{0.57}{\Fn}
	\psplot[linecolor=blue]{0.75}{5}{\Fn}
	\psline[linecolor=red](0.583,-4)(0.6666,4)
\end{pspicture}
\caption{En $x=-4$, tu vois bien qu'il ne se passe en fait rien : on peut la prolonger. En $2/3$ par contre, elle part vers l'infini et il n'y a aucun espoir de la prolonger par continuité.}  \label{FigProloiuetnon}
\end{figure}
C'est ce qu'on appelle \defe{prolonger la fonction par continuité}{Prolongation par continuité} parce que la fonction résultante est continue. La prolongation de $f$ par continuité est donc en général définie par
\begin{equation}
\tilde f(x)=
\begin{cases}
f(x)			&\text{si $f(x)$ existe}\\
\lim_{y\to x}f(y)	&\text{si $f(x)$ si cette limite existe et est finie.}
\end{cases}
\end{equation}
Dans le cas que nous regardions (voir la figure \ref{FigProloiuetnon}),
\[ 
	f(x)=\frac{ x+4 }{ 3x^2+10x-8 },
\]
le prolongement par continuité est donné par
\begin{equation}
\tilde f =\frac{ 3 }{ 3x-2 }.
\end{equation}
Remarque que cette fonction n'est toujours pas définie en $x=2/3$. 


%+++++++++++++++++++++++++++++++++++++++++++++++++++++++++++++++++++++++++++++++++++++++++++++++++++++++++++++++++++++++++++
\section{Calcul de limites}
%+++++++++++++++++++++++++++++++++++++++++++++++++++++++++++++++++++++++++++++++++++++++++++++++++++++++++++++++++++++++++++

Un résultat pratique pour calculer des limites est la
\begin{proposition}		\label{PropChmVarLim}
Quand la limite existe, nous avons
\[ 
  \lim_{x\to a}f(x)=\lim_{\epsilon\to 0}f(a+\epsilon),
\]
ce qui correspond à un \og changement de variables\fg{} dans la limite.
\end{proposition}

\begin{proof}
Si $A=\lim_{x\to a}f(x)$, par définition,
\begin{equation}		\label{EqCondFaplusespLim}
\forall\epsilon'>0,\,\exists\delta\text{ tel que }| x-a |\leq\delta\Rightarrow| f(x)-A |\leq\epsilon'.
\end{equation}
La seule subtilité de la démonstration est de remarquer que si $| x-a |\leq\delta$, alors $x$ peut être écrit sous la forme $x=a+\epsilon$ pour un certain $| \epsilon |\leq\delta$. En remplaçant $x$ par $a+\epsilon$ dans la condition \ref{EqCondFaplusespLim}, nous trouvons 
\begin{equation}
\forall\epsilon'>0,\,\exists\delta\text{ tel que }| \epsilon |\leq\delta\Rightarrow| f(x+\epsilon)-A |\leq\epsilon',
\end{equation}
ce qui signifie exactement que $\lim_{\epsilon\to 0}f(x+\epsilon)=A$.	
\end{proof}

Il y a une petite différence de point de vue entre $\lim_{x\to a}f(x)$ et $\lim_{\epsilon\to 0}f(a+\epsilon)$. Dans le premier cas, on considère $f(x)$, et on regarde ce qu'il se passe quand $x$ se rapproche de $a$, tandis que dans le second, on considère $f(a)$, et on regarde ce qu'il se passe quand on s'éloigne un tout petit peu de $a$. Dans un cas, on s'approche très près de $a$, et dans l'autre on s'en éloigne un tout petit peu. Ces deux points de vue sont évidement équivalents, comme prouvé par la proposition \ref{PropChmVarLim}.

\Exo{212}

% Il y a des techniques de calcul de limites décrites sur le site
% http://bernard.gault.free.fr/terminale/limites/limite.html

%+++++++++++++++++++++++++++++++++++++++++++++++++++++++++++++++++++++++++++++++++++++++++++++++++++++++++++++++++++++++++++
					\section{Compacité}
%+++++++++++++++++++++++++++++++++++++++++++++++++++++++++++++++++++++++++++++++++++++++++++++++++++++++++++++++++++++++++++
%http://fr.wikipedia.org/wiki/Espace_compact
%http://fr.wikipedia.org/wiki/Théorème_de_Heine-Borel
%http://fr.wikipedia.org/wiki/Émile_Borel
%http://fr.wikipedia.org/wiki/Henri_Léon_Lebesgue

Aussi incroyable que cela puisse paraître, ce que nous allons faire va servir dans la démonstration de la formule \eqref{EqMRUAINT}. Soit $E$, un sous ensemble de $\eR$. Nous pouvons considérer les ouverts suivants : 
\begin{equation}
	\mO_x=B(x,1)
\end{equation}
pour chaque $x\in E$. Évidement,
\begin{equation}
	E\subseteq \bigcup_{x\in E}\mO_x.
\end{equation}
Cette union est très souvent énorme, et même infinie. Elle contient de nombreuses redondances. Si par exemple $E=[-10,10]$, l'élément $3\in E$ est contenu dans $\mO_{3.5}$, $\mO_{2.7}$ et bien d'autres. Pire : même si on enlève par exemple $\mO_2$ de la liste des ouverts, l'union de ce qui reste continue à être tout $E$. La question est : \emph{est-ce qu'on peut en enlever suffisamment pour qu'il n'en reste qu'un nombre fini ?}
\begin{definition}
Soit $E$, un sous ensemble de $\eR$. Une collection d'ouverts $\mO_i$ est un \defe{recouvrement}{Recouvrement} de $E$ si $E\subseteq \bigcup_{i}\mO_i$. Un sous ensemble $E$ de $\eR$ tel que de tout recouvrement par des ouverts, on peut extraire un sous-recouvrement fini est dit \defe{\href{http://fr.wikipedia.org/wiki/Espace_compact}{compact}}{Compact}.
\end{definition}

\begin{proposition}
Les ensembles compacts sont fermés et bornés.
\end{proposition}

\begin{proof}
Prouvons d'abord qu'un ensemble compact est borné. Pour cela, supposons que $K$ est un compact non borné vers le haut\footnote{Nous laissons à titre d'exercice le cas où $K$ est borné par le haut et pas par le bas.}. Donc il existe une suite infinie de nombres strictement croissante $x_1<x_2<\ldots$ tels que $x_i\in K$. Prenons n'importe quel recouvrement ouvert de la partie de $K$ plus petite ou égale à $x_1$, et complétons ce recouvrement par les ouverts $\mO_i=]x_{i-1},x_i[$. Le tout forme bien un recouvrement de $K$ par des ouverts. 

Il n'y a cependant pas moyen d'en tirer un sous recouvrement fini parce que si on ne prends qu'un nombre fini parmi les $\mO_i$, on en aura fatalement un maximum, disons $\mO_k$. Dans ce cas, les points $x_{k+1}$, $x_{k+1}$,\ldots ne seront pas dans le choix fini d'ouverts.

Cela prouve que $K$ doit être borné.

Pour prouver que $K$ est fermé, nous allons prouver que le complémentaire est ouvert. Et pour cela, nous allons prouver que si le complémentaire n'est pas ouvert, alors nous pouvons construire un recouvrement de $K$ dont on ne peut pas extraire de sous recouvrement fini.

Si $\eR\setminus K$ n'est pas ouvert, il possède un point, disons $x$, tel que tout voisinage de $x$ intersecte $K$. Soit $B(x,\epsilon_1)$, un de ces voisinages, et prenons $k_1\in K\cap B(x,\epsilon_1)$. Ensuite, nous prenons $\epsilon_2$ tel que $k_1$ n'est pas dans $B(x,\epsilon_1)$, et nous choisissons $k_2\in K\cap B(x,\epsilon_2)$. De cette manière, nous construisons une suite de $k_i\in K$ tous différents et de plus en plus proches de $x$. Prenons un recouvrement quelconque par des ouverts de la partie de $K$ qui n'est pas dans $B(x,\epsilon_1)$. Les nombres $k_i$ ne sont pas dans ce recouvrement.

Nous ajoutons à ce recouvrement les ensembles $\mO=]k_i,k_{i+1}[$. Le tout forme un recouvrement (infini) par des ouverts dont il n'y a pas moyen de tirer un sous recouvrement fini, pour exactement la même raison que la première fois.
\end{proof}

Le résultat suivant le théorème de \href{http://fr.wikipedia.org/wiki/Théorème_de_Heine-Borel}{Borel-Lebesgue}, et la démonstration vient de wikipédia.
\begin{theorem}[\href{http://fr.wikipedia.org/wiki/Émile_Borel}{borel}-\href{http://fr.wikipedia.org/wiki/Henri_Léon_Lebesgue}{Lebesgue}]	\label{ThoBOrelLebesgue}
	Les intervalles de la forme $[a,b]$ sont compacts.
\end{theorem}

\begin{proof}
	Soit $\Omega$, un recouvrement du segment $[a,b]$ par des ouverts, c'est à dire que
	\begin{equation}
		[a,b]\subseteq\bigcup_{\mO\in\Omega}\mO.
	\end{equation}
	Nous notons par $M$ le sous-ensemble de $[a,b]$ des points $m$ tels que l'intervalle $[a,m]$ peut être recouvert par un sous-ensemble fini de $\Omega$. C'est à dire que $M$ est le sous ensemble de $[a,b]$ sur lequel le théorème est vrai. Le but est maintenant de prouver que $M=[a,b]$.
	\begin{description}
		\item[$M$ est non vide] En effet, $a\in M$ parce que il existe un ouvert $\mO\in\Omega$ tel que $a\in\mO$. Donc $\mO$ tout seul recouvre l'intervalle $[a,a]$. 
		\item[$M$ est un intervalle] Soient $m_1$, $m_2\in M$. Le but est de montrer que si $m'\in[m_1,m_2]$, alors $m'\in M$. Il y a un sous recouvrement fini de l'intervalle $[a,m_2]$ (par définition de $m_2\in M$). Ce sous recouvrement fini recouvre évidement aussi $[a,m']$ parce que $[a,m']\subseteq [a,m_2]$, donc $m'\in M$.
		\item[$M$ est une ensemble ouvert] Soit $m\in M$. Le but est de prouver qu'il y a un ouvert autour de $m$ qui est contenu dans $M$. Mettons que $\Omega'$ soit un sous recouvrement fini qui contienne l'intervalle $[a,m]$. Dans ce cas, on a un ouvert $\mO\in\Omega'$ tel que $m\in\mO$. Tous les points de $\mO$ sont dans $M$, vu qu'ils sont tous recouverts par $\Omega'$. Donc $\mO$ est un voisinage de $m$ contenu dans $M$.
		\item[$M$ est un ensemble fermé] $M$ est un intervalle qui commence en $a$, en contenant $a$, et qui finit on ne sait pas encore où. Il est donc soit de la forme $[a,m]$, soit de la forme $[a,m[$. Nous allons montrer que $M$ est de la première forme en démontrant que $M$ contient son supremum $s$. Ce supremum est un élément de $[a,b]$, et donc il est contenu dans un des ouverts de $\Omega$. Disons $s\in\mO_s$. Soit $c$, un élément de $\mO_s$ strictement plus petit que $c$; étant donné que $s$ est supremum de $M$, cet élément $c$ est dans $M$, et donc on a un sous recouvrement fini $\Omega'$ qui recouvre $[a,c]$. Maintenant, le sous recouvrement constitué de $\Omega'$ et de $\mO_s$ est fini et recouvre $[a,s]$.
	\end{description}
	Nous pouvons maintenant conclure : le seul intervalle non vide de $[a,b]$ qui soit à la fois ouvert et fermé est $[a,b]$ lui-même, ce qui prouve que $M=[a,b]$, et donc que $[a,b]$ est compact.
\end{proof}
Note : il est également vrai que \emph{tous} les compacts de $\eR$ sont fermés et bornés, mais nous n'allons pas démontrer cela ici.

Une propriété très importante des compacts est la suivante :
\begin{theorem}		\label{ThoImCompCotComp}
L'image d'un compact par une fonction continue est un compact
\end{theorem}

\begin{proof}
	Soit $K\subset \eR$, un ensemble compact, et regardons $f(K)$; en particulier, nous considérons $\Omega$, un recouvrement de $f(K)$ par des ouverts. Nous avons que
	\begin{equation}
		f(K)\subseteq\bigcup_{\mO\in\Omega}\mO.
	\end{equation}
	Par construction, nous avons aussi
	\begin{equation}
		K\subseteq\bigcup_{\mO\in\Omega}f^{-1}(\mO),
	\end{equation}
	en effet, si $x\in K$, alors $f(x)$ est dans un des ouverts de $\Omega$, disons $f(x)\in \mO_0$, et évidemment, $x\in f^{-1}(\mO)$.  Les $f^{-1}(\mO)$ recouvrent le compact $K$, et donc on peut en choisir un sous-recouvrement fini, c'est à dire un choix de $\{ f^{-1}(\mO_1),\ldots,f^{-1}(\mO_n) \}$ tels que
	\begin{equation}
		K\subseteq \bigcup_{i=1}^nf^{-1}(\mO_i).
	\end{equation}
	Dans ce cas, nous avons que
	\begin{equation}
		f(K)\subseteq\bigcup_{i=1}^n\mO_i,
	\end{equation}
	ce qui prouve la compacité de $f(K)$.
\end{proof}

Par le théorème des valeurs intermédiaires, l'image d'un intervalle par une fonction continue est un intervalle, et nous avons l'importante propriété suivante des fonctions continues sur un compact.

\begin{theorem}
	Si $f$ est une fonction continue sur l'intervalle compact $[a,b]$. Alors $f$ est bornée sur $[a,b]$ et elle atteint ses bornes.
\end{theorem}

\begin{proof}
	Étant donné que $[a,b]$ est un intervalle compact, son image est également un intervalle compact, et donc est de la forme $[m,M]$. Ceci découle du théorème \ref{ThoImCompCotComp} et le corollaire \ref{CorImInterInter}. Le maximum de $f$ sur $[a,b]$ est la borne $M$ qui est bien dans l'image (parce que $[m,M]$ est fermé). Idem pour le minimum $m$.
\end{proof}

En préparant ces notes, j'ai fait une très jolie faute en essayant de prouver ce théorème\footnote{Qui ne fait pas de fautes en tapant à l'ordi couché sur son lit entre une et deux heures du matin ?}. Voici comment je comptait prouver le fait que $f$ est bornée. Supposons que $f$ ne soit bornée ni vers le haut, ni vers le bas, donc son image est $\eR$ qui est ouvert. Mais $f$ est continue sur $[a,b]$, donc l'image inverse de $\eR$ par $f$ doit être un ouvert. Oui, mais cette image inverse est exactement $[a,b]$ qui est fermé. Cela est une contradiction qui prouve que $f$ doit être bornée au moins soit vers le haut, soit vers le bas.

Peux-tu trouver la faute ?


\section{A mad tea party}	\label{PgMadTeaParty}
%++++++++++++++++++++++++

\textit{\og Reprenez donc un peu de thé\fg{} propose le Lièvre de Mars.}

\textit{\og Je n'ai rien pris du tout, je ne saurai donc reprendre de rien !\fg}

\textit{\og Vous voulez dire que vous ne sauriez reprendre de quelque chose\fg{} repartit le Chapelier.\\
 \og Quand il n'y a rien, ce n'est pas
facile d'en reprendre\fg.}

 \begin{itemize}

 \item Alors comme ça, vous êtes \href{http://fr.wikipedia.org/wiki/Manifestations_de_la_place_Tian'anmen}{étudiante}~?
 \item Oui, en mathématiques par exemple.
 \item Alors que vaut cette fraction : un sur deux sur trois sur quatre~?
 \item Eh bien ...
 \item Elle vaut deux tiers, la devança le Loir.
 \item Ou trois huitièmes si vous préférez, ajouta le Lièvre de Mars.
 \item Ou encore un sur vingt-quatre, affirma le  \href{http://fr.wikipedia.org/wiki/http://fr.wikipedia.org/wiki/Chapelier_fou_(Alice_au_pays_des_merveilles)}{Chapelier}.
 \item En fait, je crois que...
\item Aucune importance ! Dites-nous plutôt combien vous voulez de sucre dans votre thé~?
\item Deux ou trois, ça dépend de la taille de la tasse.
\item Certainement pas, car de toute façon, deux ou trois c'est pareil.
\item Parfaitement~! approuva le Loir en fixant Alice qui écarquillait les yeux.
\item Ce n'est pourtant pas ce qu'on m'a appris, fit celle-ci.
\item Pourtant, ce n'est pas compliqué à comprendre, en voici une démonstration des plus élémentaires. On sait que pour tout entier $n$ on a successivement
                \[ 
			(n+1)^2=n^2+2n+1
		\]
                \[
			(n+1)^2-2n-1=n^2
		\]
	Retranchons $n(2n+1)$ des deux côtés
                \[
			(n+1)^2-(n+1)(2n+1)=n^2-n(2n+1).
		\]
	Mézalor, en ajoutant $(2n+1)^2/4$, on obtient
		\[ 
                	(n+1)^2-(n+1)(2n+1)+\frac{(2n+1)^2}{4}=n^2-n(2n+1)+\frac{(2n+1)^2}{4}
		\]
	Soit
		\[
	                \left((n+1)-\frac{2n+1}{2}\right)^2=\left(n-\frac{2n+1}{2}\right)^2
		\]
	En passant à la racine carrée, on obtient
		\[ 
			(n+1)-\frac{2n+1}{2}=n-\frac{2n+1}{2}
		\]
	d'où
		\[ 
			n+1=n
		\]
Et si je prends $n=2$, j'ai aussitôt $3=2$
\item Alors, qu'est-ce que vous en dites~?
\item Je\ldots commença Alice.
\item D'ailleurs, cela prouve que tous les entiers sont égaux, la coupa le Lièvre de Mars.
\item Pas mal du tout ! Qu'en dites-vous mademoiselle la mathématicienne~?
\item Je vais vous dire tout de suite ce que j'en pense
\item Ah non ! Nous préférerions de loin que vous pensiez ce que vous allez nous dire.
\item C'est pareil ! grinça Alice qui commençait à en avoir assez.
\item Comment ça, c'est pareil~? Dire ce que l'on pense ce serait pareil que penser ce que l'on dit~? s'étrangla le Lièvre de Mars.
\item Incroyable ! Et manger ce qu'on voit ce serait pareil que voir ce qu'on mange~?
\item Mais...
\item Et respirer quand on dort pareil que dormir quand on respire~?
\item En logique, nous vous mettons 3 sur 5.
\item Autant dire moins que un.
\item C'est à dire zéro, puisque si $2=3$ alors $1=0$.
\item Parce que chez vous, 3 c'est moins que 1 ? s'indigna Alice.
\item On se demande ce qu'on vous apprend à l'école ! Bien sûr que oui ! Tenez, considérez
		\[ 
			f(x)=\frac{x^2+32}{2x^2+1}+\frac{|x|+1}{2x+51}
		\]
Eh bien il est facile de voir que cette fonction a pour limite $0$ en moins l'infini et $1$ en plus l'infini.
\item Je ne dis pas le contraire, protesta Alice.
\item Donc l'image par $f$ de $\eR$ est l'intervalle $]0,1[$, or $f(0)=3$, donc $3$ appartient à $]0,1[$ à ce titre : on a bien $3$ plus petit que $1$.
\item C'est de la folie pure, pensa Alice\ldots
\end{itemize}


\begin{remark}
Cette partie de thé de fous est volée avec quelque modifications mineures (dont les liens) et sans en avoir honte de la série de cours de math de Guillaume Connan, \href{http://gconnan.free.fr/}{Tehessin le rezeen}. À lire sans hésiter ! Le livre original de \href{http://fr.wikipedia.org/wiki/Lewis_Carroll}{Lewis Carol} d'où est tiré un très bon dessin animé est également à lire sans hésiter.
\end{remark}




