% This is part of Mes notes de mathématique
% Copyright (c) 2008-2014
%   Laurent Claessens
% See the file fdl-1.3.txt for copying conditions.

\subsection{Continuité}
%----------------------

Nous allons considérer trois approches différentes de la continuité. La première sera de définir la continuité de fonctions de $\eR$ vers $\eR$ au moyen du critère usuel. Ensuite, nous définiront la continuité des applications entre n'importes quels espaces métriques, et nous montrerons que les deux définitions sont équivalentes dans le cas des fonctions sur $\eR$ à valeurs réelles.

Enfin, un peu plus tard nous verrons que la continuité peut également être vue en termes de limites. Encore une fois nous verrons que dans le cas de fonctions de $\eR$ vers $\eR$ cette troisième approche est équivalentes aux deux premières.


La définition de fonction continue est la définition \ref{DefOLNtrxB}.


Nous allons donc dire qu'une fonction est continue quand plus $x$ s'approche de $a$ en suivant la courbe, plus $f(x)$ s'approche de $f(a)$. Voici la définition précise.
\begin{definition}      \label{DefContinue}
Nous disons que la fonction $x\mapsto f(x)$ est \defe{continue en $a$}{continue} si
\begin{equation}
 \forall \epsilon>0,\exists \delta\text{ tel que } \big(| x-a |\leq\delta\big)\Rightarrow | f(x)-f(a) |\leq \epsilon.
\end{equation}
\end{definition}


\begin{definition}		\label{DefFonctContinueRR}
	Soit une fonction $f\colon D\to \eR$ et un point $a$ dans $D$. Nous disons que $f$ est \defe{continue}{continue!fonction réelle} lorsque $f$ possède une limite en $a$ et $\lim_{x\to a} f(x)=f(a)$.
\end{definition}
En remplaçant $\ell$ par $f(a)$ dans la définition de la limite, nous exprimons la continuité de $f$ en $a$ par la façon suivante. Pour tout $\varepsilon>0$, il existe un $\delta>0$ tel que $\forall x\in D$,
\begin{equation}
	| x-a |<\delta\Rightarrow \big| f(x)-f(a) \big|<\varepsilon.
\end{equation}


Nous allons maintenant étudier quelque conséquences de cette définition. 

\begin{enumerate}
\item D'abord on voit que la continuité n'a été définie qu'en un point. On peut dire que la fonction $f$ est continue \emph{en tel point donné}, mais nous n'avons pas dit ce qu'est une fonction continue \emph{dans son ensemble}.

\item Si $I$ est un intervalle de $\eR$, on dit que $f$ est \defe{continue sur l'intervalle}{continuité!sur un intervalle} $I$ si elle est continue en chaque point de $I$.

\item Comme la définition de $f$ continue en $a$ fait intervenir $f(x)$ pour tous les $x$ pas trop loin de $a$, il faut au moins déjà que $f$ soit définie sur ces $x$. En d'autres termes, dire que $f$ est continue en $a$ demande que $f$ existe sur un intervalle autour de $a$. 

Ceci couplé à la définition précédente laisse penser qu'il est surtout intéressant d'étudier les fonctions qui sont continues sur un intervalle.

\item L'intuition comme quoi une fonction continue doit pouvoir être tracée sans lever la main correspond aux fonctions continues sur des intervalles. Au moins sur l'intervalle où elle est continue, elle est traçable en un morceau.
\end{enumerate}


Nous allons démontrer maintenant une série de petits résultats qui permettent de simplifier la démonstration de la continuité de fonctions.
\begin{theorem}
Si la fonction $f$ est continue au point $a$, alors la fonction $\lambda f$ est également continue en $a$.
\end{theorem}

\begin{proof}
Soit $\epsilon>0$. Nous avons besoin d'un $\delta>0$ tel que pour chaque $x$ à moins de $\delta$ de $a$, la fonction $\lambda f$ soit à moins de $\epsilon$ de $(\lambda f)(a)=\lambda f(a)$. Étant donné que la fonction $f$ est continue en $a$, on sait déjà qu'il existe un $\delta_1$ (nous notons $\delta_1$ afin de ne pas confondre ce nombre dont on est sûr de l'existence avec le $\delta$ que nous sommes en train de chercher) tel que 
\[ 
  (| x-a |\leq \delta_1)\Rightarrow | f(x)-f(a) |\leq \epsilon_1.
\]
Hélas, ce $\delta_1$ n'est pas celui qu'il faut faut parce que nous travaillons avec $\lambda f$ au lieu de $f$, ce qui fait qu'au lieu d'avoir $| f(x)-f(a) |$, nous avons $| \lambda f(x)-\lambda f(a) |=| \lambda |\cdot | f(x)-f(a) |$.  Ce que $\delta_1$ fait avec $(\lambda f)$, c'est
\[ 
  (| x-a |\leq\delta_1)\Rightarrow  | (\lambda f)(x)- (\lambda f)(a)|\leq | \lambda |\epsilon_1.
\]
Ce que nous apprend la continuité de $f$, c'est que pour chaque choix de $\epsilon_1$, on a un $\delta_1$ qui fait cette implication. Comme cela est vrai pour chaque choix de $\epsilon_1$, essayons avec $\epsilon_1=\epsilon/| \lambda |$ pour voir ce que ça donne. Nous avons donc un $\delta_1$ qui fait
\[ 
  (| x-a |\leq\delta_1)\Rightarrow  | (\lambda f)(x)- (\lambda f)(a)|\leq | \lambda |\epsilon_1=\epsilon.
\]
Ce $\delta_1$ est celui qu'on cherchait. 
\end{proof}

\begin{theorem}
Si $f$ et $g$ sont deux fonctions continues en $a$, alors la fonction $f+g$ est également continue en $a$.
\end{theorem}

\begin{proof}
La continuité des fonctions $f$ et $g$ au point $a$ fait en sorte que pour tout choix de $\epsilon_1$ et $\epsilon_2$, il existe $\delta_1$ et $\delta_2$ tels que 
\[ 
  (| x-a |\leq \delta_1)\Rightarrow | f(x)-f(a) |\leq \epsilon_1.
\]
et
\[ 
  (| x-a |\leq \delta_2)\Rightarrow | g(x)-g(a) |\leq \epsilon_2.
\]
La quantité que nous souhaitons analyser est $| f(x)+g(x)-f(a)-g(a) |$. Tout le jeu de la démonstration de la continuité est de triturer cette expression pour en tirer quelque chose en termes de $\epsilon_1$ et $\epsilon_2$. Si nous supposons avoir pris $| x-a |$ plus petit en même temps que $\delta_1$ et que $\delta_2$, nous avons
\[
| f(x)+g(x)-f(a)-g(a) |\leq| f(x)-g(x) |+| g(x)-g(a) |\leq\epsilon_1+\epsilon_2 
\]
en utilisant la formule générale $| a+b |\leq | a |+| b |$. Maintenant si on choisit $\epsilon_1$ et $\epsilon_2$ tels que $\epsilon_1+\epsilon_2<\epsilon$, et les $\delta_1$, $\delta_2$ correspondants, on a que 
\[
| f(x)+g(x)-f(a)-g(a) |\leq\epsilon,
\]
pourvu que $| x-a |$ soit plus petit que $\delta_1$ et $\delta_2$. Le bon $\delta$ a prendre est donc le minimum de $\delta_1$ et $\delta_2$ qui eux-même sont donnés par un choix de $\epsilon_1$ et $\epsilon_2$ tels que $\epsilon_1+\epsilon_2\leq\epsilon$.
\end{proof}

Pour résumer ces deux théorèmes, on dit que si $f$ et $g$ sont continues en $a$, alors la fonction $\alpha f+\beta g$ est également continue en $a$ pour tout $\alpha$, $\beta\in\eR$.

Parmi les propriétés immédiates de la continuité d'une fonction, nous avons ceci qui est souvent bien utile.

\begin{corollary}
Si la fonction $f$ est continue en $a$ et si $f(a)>0$, alors $f$ est positive sur un intervalle autour de $a$.
\end{corollary}

\begin{proof}
Prenons $\epsilon<f(a)$ et voyons\footnote{ici, nous insistons sur le fait que nous prenons $\epsilon$ \emph{strictement} plus petit que $f(a)$.} ce que la continuité de $f$ en $a$ nous offre : il existe un $\delta$ tel que
\[ 
  (| x-a |\leq \delta)\Rightarrow | f(x)-f(a) |\leq\epsilon < f(a).
\]
Nous en retenons que sur un intervalle (de largeur $\delta$), nous avons $| f(x)-f(a) |\leq f(a)$. Par hypothèse, $f(a)>0$, donc si $f(x)<0$, alors la différence $f(x)-f(a)$ donne un nombre encore plus négatif que $-f(a)$, c'est à dire que $| f(x)-f(a) |>f(a)$, ce qui est contraire à ce que nous venons de démontrer. D'où la conclusion que $f(x)>0$.
\end{proof}

\subsection{La fonction la moins continue du monde}
%--------------------------------------------------

Parmi les exemples un peu sales de fonctions non continues, il y a celle-ci :
\[ 
  \chi_{\eQ}(x)=
\begin{cases}
    1 \text{ si $x\in\eQ$}\\
    0 \text{ sinon.}
\end{cases}
\]
Par exemple, $\chi_{\eQ}(0)=1$, et\footnote{Pour prouver que $\sqrt{2}$ n'est pas rationnel, c'est pas trop compliqué, mais pour prouver que $\pi$ ne l'est pas non plus, il faudra encore manger de la soupe.} $\chi_{\eQ}(\pi)=\chi_{\eQ}(\sqrt{2})=0$. Malgré que $\chi_{\eQ}(0)=1$, il n'existe \emph{aucun} voisinage de $1$ sur lequel la fonction reste proche de $1$, parce que tout voisinage va contenir au moins un irrationnel. À chaque millimètre, cette fonction fait une infinité de bonds !

Cette fonction n'est donc continue nulle part. 

À partir de là, nous pouvons construire la fonction suivante qui n'est continue qu'en un point :
\[ 
  f(x)=x\chi_{\eQ}(x)=
\begin{cases}
x\text{ si $x\in\eQ$}\\
0\text{ sinon.}
\end{cases}
\]
Cette fonction est continue en zéro. En effet, prenons $\delta>0$; il nous faut un $\epsilon$ tel que $| x |\leq\epsilon$ implique $f(x)\leq \delta$ parce que $f(0)=0$. Bon ben prendre simplement $\epsilon=\delta$ nous contente. Cette fonction est donc très facilement continue en zéro.

Et pourtant, dès que l'on s'écarte un tant soit peu de zéro, elle fait des bons une infinité de fois par millionième de millimètre ! Cette fonction est donc la plus discontinue du monde en tous les points saut un (zéro) où elle est une fonction continue !

\subsection{Approche topologique}
%--------------------------------

Nous avons vu que sur tout ensemble métrique, nous pouvons définir ce qu'est un ouvert : c'est un ensemble qui contient une boule ouverte autour de chacun de ses points. Quand on est dans un ensemble ouvert, on peut toujours un peu se déplacer sans sortir de l'ensemble.

Le théorème suivant est une très importante caractérisation des fonctions continues (de $\eR$ dans $\eR$) en termes de topologie, c'est à dire en termes d'ouverts.

\begin{theorem}     \label{ThoContInvOuvert}
Si $I$ est un intervalle ouvert contenu dans $\dom f$, alors $f$ est continue sur $I$ si et seulement si pour tout ouvert $\mO$ dans $\eR$, l'image inverse $f|_I^{^{-1}}(\mO)$ est ouvert.
\end{theorem}

Par abus de langage, nous exprimons souvent cette condition par \og une fonction est continue si et seulement si l'image inverse de tout ouvert est un ouvert\fg.

\begin{proof}

Dans un premier temps, nous allons transformer le critère de continuité en termes de boules ouvertes, et ensuite, nous passeront à la démonstration proprement dite. Le critère de continuité de $f$ au point $x$ dit que
\begin{equation}        \label{EqDEfCOntAn}
  \forall \delta>0,\exists\,\epsilon>0\text{ tel que }\big( | x-a |< \epsilon \big)\Rightarrow| f(x)-f(a) |<\delta.
\end{equation}
Cette condition peut être exprimée sous la forme suivante :
\[ 
  \forall \delta>0,\exists\epsilon\text{ tel que } a\in B(x,\epsilon)\Rightarrow f(a)\in B\big( f(x),\delta \big),
\]
ou encore
\begin{equation}        \label{EqRedefContBoules}
  \forall \delta>0,\exists\epsilon\text{ tel que } f\big( B(x,\epsilon) \big)\subset B\big( f(x),\delta \big).
\end{equation}
Jusque ici, nous n'avons fait que du jeu de notations. Nous avons exprimé en termes de topologie des inégalités analytiques. Si tu veux, tu peux retenir cette condition \eqref{EqRedefContBoules} comme définition d'une fonction continue en $x$. Si tu choisit de vivre comme ça, tu dois être capable de retrouver \eqref{EqDEfCOntAn} à partir de \eqref{EqRedefContBoules}.
 
Passons maintenant à la démonstration proprement dite du théorème. 

D'abord, supposons que $f$ est continue sur $I$, et prenons $\mO$, un ouvert quelconque. Le but est de prouver que $f|_I^{-1}(\mO)$ est ouvert. Pour cela, nous prenons un point $x_0\in f|_I^{-1}(\mO)$ et nous allons trouver un ouvert autour ce ce point contenu dans $f|_I^{-1}(\mO)$. Nous écrivons $y_0=f(x_0)$. Évidement, $y_0\in\mO$, donc on a une boule autour de $y_0$ qui est contenue dans $\mO$, soit donc $\delta>0$ tel que
\[  
  B(y_0,\delta)\subset\mO.
\]
Par hypothèse, $f$ est continue en $x_0$, et nous pouvons donc y appliquer le critère \eqref{EqRedefContBoules}. Il existe donc $\epsilon>0$ tel que 
\[ 
  f\big( B(x_0,\epsilon) \big)\subset B\big( f(x_0),\delta \big)\subset\mO.
\]
Cela prouve que $B(x_0,\epsilon)\subset f|_I^{-1}(\mO)$.

Dans l'autre sens, maintenant. Nous prenons $x_0\in I$ et nous voulons prouver que $f$ est continue en $x_0$, c'est à dire que pour tout $\delta$ nous cherchons un $\epsilon$ tel que $f\big( B(x_0,\epsilon) \big)\subset B\big( f(x_0),\delta \big)$. Oui, mais $B\big( f(x_0),\delta \big)$ est ouverte, donc par hypothèse, $f|_I^{-1}\Big( B\big( f(x_0),\delta \big) \Big)$ est ouvert, inclus à $I$ et contient $x_0$. Donc il existe un $\epsilon$ tel que
\[ 
  B(x_0,\epsilon)\subset f|_I^{-1}\Big( B\big( f(x_0),\delta \big) \Big),
\]
et donc tel que 
\[ 
  f\big( B(x_0,\epsilon) \big)\subset B\big( f(x_0),\delta \big),
\]
ce qu'il fallait prouver.
\end{proof}

\begin{lemma}   \label{LemConncontconn}
L'image d'un ensemble connexe par une fonction continue est connexe.
\end{lemma}

\begin{proof}
Nous allons encore faire la contraposée. Soit $A$ une partie de $\eR$ telle que $f(A)$ ne soit pas connexe. Nous allons prouver que $A$ elle-même n'est pas connexe. Dire que $f(A)$ n'est pas connexe, c'est dire qu'il existe $\mO_1$ et $\mO_2$, deux ouverts disjoints qui recouvrent $f(A)$. Je prétends que $f^{-1}(\mO_1)$ et $f^{-1}(\mO_2)$ sont ouverts, disjoints et qu'ils recouvrent $A$.
\begin{itemize}
\item Ces deux ensembles sont ouverts parce qu'ils sont images inverses d'ouverts par une fonction continue (théorème \ref{ThoContInvOuvert}).
\item Si $x\in f^{-1}(\mO_1)\cap f^{-1}(\mO_2)$, alors $f(x)\in \mO_1\cap\mO_2$, ce qui contredirait le fait que $\mO_1$ et $\mO_2$ sont disjoints. Il n'y a donc pas d'éléments dans l'intersection de $f^{-1}(\mO_1)$ et de $f^{-1}(\mO_2)$.
\item Si $f^{-1}(\mO_1)$ et $f^{-1}(\mO_2)$ ne recouvrent pas $A$, il existe un $x$ dans $A$ qui n'est dans aucun des deux. Dans ce cas, $f(x)$ est dans $f(A)$, mais n'est ni dans $\mO_1$, ni dans $\mO_2$, ce qui contredirait le fait que ces deux derniers recouvrent $f(A)$.
\end{itemize}
Nous déduisons que $A$ n'est pas connexe. Et donc le lemme.
\end{proof}

\begin{theorem}[Théorème des valeurs intermédiaires]        \label{ThoValInter}
Soit $f$, une fonction continue sur $[a,b]$, et supposons que $f(a)<f(b)$. Alors pour tout $y$ tel que $f(a)\leq y\leq f(b)$, il existe un $x$ entre $a$ et $b$ tel que $f(x)=y$.
\end{theorem}
\index{connexité!théorème des valeurs intermédiaires}
\index{théorème!valeurs intermédiaires}

\begin{proof}
Nous savons que $[a,b]$ est connexe parce que c'est un intervalle (proposition \ref{PropInterssiConn}). Donc $f\big( [a,b] \big)$ est connexe (lemme \ref{LemConncontconn}) et donc est un intervalle (à nouveau la proposition \ref{PropInterssiConn}). Étant donné que $f\big( [a,b] \big)$ est un intervalle, il contient toutes les valeurs intermédiaires entre n'importe quels deux de ses éléments. En particulier toutes les valeurs intermédiaires entre $f(a)$ et $f(b)$.
\end{proof}

\begin{corollary}       \label{CorImInterInter}
L'image d'un intervalle par une fonction continue est un intervalle.
\end{corollary}
La preuve est laissée à titre d'exercice.

\subsection{Continuité de la racine carré, invitation à la topologie induite}
%-----------------------------------------

Pourquoi nous intéresser particulièrement à cette fonction ? Parce qu'elle a une sale condition d'existence : son domaine de définition n'est pas ouvert. Or dans tous les théorèmes de continuité d'approche topologique que nous avons vus, nous avons donné des contions \emph{pour tout ouvert}. Nous nous attendons donc a avoir des difficultés avec la continuité de $\sqrt{x}$ en zéro.

Prenons $I$, n'importe quel intervalle ouvert dans $\eR^+$, et voyons que la fonction
\begin{equation}
\begin{aligned}
 f\colon \eR^+&\to \eR^+ \\ 
   x&\mapsto \sqrt{x} 
\end{aligned}
\end{equation}
est continue sur $I$. Remarque déjà que si $I$ est un ouvert dans $\eR^+$, il ne peut pas contenir zéro. Avant de nous lancer dans notre propos, nous prouvons un lemme qui fera tout le travail\footnote{C'est toujours ingrat d'être un lemme : on fait tout le travail et c'est toujours le théorème qui est nommé.}.

\begin{lemma}
Soit $\mO$, un ouvert dans $\eR^+$. Alors $\mO^2=\{ x^2\tq x\in\mO \}$ est également ouvert .
\end{lemma}

\begin{proof}
Un élément de $\mO^2$ s'écrit sous la forme $x^2$ pour un certain $x\in\mO$. Le but est de trouver un ouvert autour de $x^2$ qui soit contenu dans $\mO^2$. Étant donné que $\mO$ est ouvert, on a une boule centrée en $x$ contenue dans $\mO$. Nous appelons $\delta$ le rayon de cette boule :
\[ 
  B(x,\delta)\subset\mO.
\]
Étant donné que cet ensemble est connexe, nous savons par le lemme \ref{LemConncontconn} que $B(x,\delta)^2$ est également connexe (parce que la fonction $x\mapsto x^2$ est continue). Son plus grand élément est $(x+\delta)^2=x^2+\delta^2+2x\delta>x^2+\delta^2$, et son plus petit élément est $(x-\delta)^2=x^2+\delta^2-2x\delta$. 

Ce qui serait pas mal, c'est que ces deux bornes entourent $x^2$, de telle façon à ce qu'elles définissent un ouvert autour de $x^2$ qui soit dans $\mO^2$. Hélas, c'est pas gagné que $x^2+\delta^2-2x\delta$ soit plus petit que $x^2$. 

Heureusement, en fait c'est vrai parce que d'une part, du fait que $\mO\subset\eR^+$, on a $x>0$, et d'autre part, pour que $\mO$ soit positif, il faut que $\delta<x$. Donc on a évidement que $\delta<2x$, et donc que
\[ 
  x^2+\delta^2-2x\delta=x^2+\delta\underbrace{(\delta-2x)}_{<0}<x^2.
\]
Donc nous avons fini : l'ensemble
\[ 
  B(x,\delta)^2=]x^2+\delta^2-2x\delta,x^2+\delta^2+2x\delta[\subset\mO^2
\]
est un intervalle qui contient $x^2$, et donc qui contient une boule ouverte centrée en~$x^2$.

\end{proof}

Maintenant nous pouvons nous attaquer à la continuité de la racine carré sur tout ouvert positif en utilisant le théorème \ref{ThoContInvOuvert}. Soit $\mO$ n'importe quel ouvert de $\eR$, et prouvons que $f|_I^{-1}(\mO)$ est ouvert. Par définition,
\begin{equation}
  f|_I^{-1}(\mO)=\{ x\in I\tq \sqrt{x}\in\mO \}.
\end{equation}
Maintenant c'est un tout petit effort que de remarquer que $f|_I^{-1}(\mO)=\mO^2\cap I$. De là, on a gagné parce que $\mO^2$ et $I$ sont des ouverts. Or l'intersection de deux ouverts est ouvert. 

Nous n'en avons pas fini avec la fonction $\sqrt{x}$. Nous avons la continuité de la racine carré pour tous les réels strictement positifs. Il reste à pouvoir dire que la fonction est continue en zéro malgré qu'elle ne soit pas définie sur un ouvert autour de zéro. 

Il est possible de dire que la racine carré est continue en $0$, malgré qu'elle ne soit pas définie sur un ouvert autour de $0$\ldots en tout cas pas un ouvert au sens que tu as en tête. Nous allons rentabiliser un bon coup notre travail sur les espaces métriques.

Nous pouvons définir la notion de boule ouverte sur n'importe quel espace métrique $A$ en disant que
\[ 
  B(x,r)=\{ y\in A\tq d(x,y)<r \}.
\]
\begin{definition}      \label{DefContMetrique}
Soit $f\colon A\to B$, une application entre deux espaces métriques. Nous disons que $f$ est \defe{continue}{continue!sur espace métrique} au point $a\in A$ si $\forall \delta>0$, $\exists\epsilon>0$ tel que 
\begin{equation}
  f\big( B(a,\epsilon) \big)\subset B\big( f(a),\delta \big).
\end{equation}
\end{definition}
Tu reconnais évidement la condition \eqref{EqRedefContBoules}. Nous l'avons juste recopiée. Tu remarqueras cependant que cette définition généralise immensément la continuité que l'on avait travaillé à propos des fonctions de $\eR$ vers $\eR$. Maintenant tu peux prendre n'importe quel espace métrique et c'est bon.

Nous n'allons pas faire un tour complet des conséquences et exemples de cette définition. Au lieu de cela, nous allons juste montrer en quoi cette définition règle le problème de la continuité de la racine carré en zéro.

La fonction que nous regardons est 
\begin{equation}
\begin{aligned}
f \colon \eR^+&\to \eR^+ \\ 
   x&\mapsto \sqrt{x}.
\end{aligned}
\end{equation}
Mais cette fois, nous ne la voyons pas comme étant une fonction dont le domaine est une partie de $\eR$, mais comme fonction dont le domaine est $\eR^+$ vu comme un espace métrique en soi. Quelles sont les boules ouvertes dans $\eR^+$ autour de zéro ? Réponse : la boule ouverte de rayon $r$ autour de zéro dans $\eR^+$ est :
\[ 
  B(0,r)_{\eR^+}=\{ x\in\eR^+\tq d(x,0)<r \}=[0,r[.  
\]
Cet intervalle est un ouvert. Aussi incroyable que cela puisse paraître !

Testons la continuité de la racine carré en zéro dans ce contexte. Il s'agit de prendre $A=\eR^+$, $B=\eR^+$ et $a=0$ dans la définition \ref{DefContMetrique}. Nous avons que $B(\sqrt{0},\delta)=B(0,\delta)=[0,\delta[$ pour la topologie de $\eR^+$.

Il s'agit maintenant de trouver un $\epsilon$ tel que $f\big( B(0,\epsilon) \big)\subset [0,\delta[$. Par définition, nous avons que
\[ 
  f\big( B(0,\epsilon) \big)=[0,\sqrt{\epsilon}[,
\]
le problème revient dont à trouver $\epsilon$ tel que $\sqrt{\epsilon}\leq\delta$. Prendre $\epsilon<\delta^2$ fait l'affaire.


Donc voila. Au sens de la \href{http://fr.wikipedia.org/wiki/Topologie_induite}{topologie propre} à $\eR^+$, nous pouvons dire que la fonction racine carré est partout continue.
\subsection{Limites en des nombres}
%----------------------------------

Si tu regardes la fonction $f(x)=5x+3$, tu ne serais pas étonnée si je te disais par exemple que 
\begin{align}
\lim_{x\to 10}f(x)&=53&\text{et}&\lim_{x\to 0}f(x)=3.
\end{align}
En effet, plus $x$ est proche de $10$, plus $f(x)$ est proche de $53$ et plus $x$ est proche de $0$, plus $f(x)$ est proche de $3$. Pas grand chose de neuf sous le Soleil.

Oui, mais l'intérêt d'introduire le concept de limite dans le cas de l'infini était qu'on ne peut pas bêtement calculer $f(\infty)$. Il fallait donc une astuce pour parler du comportement de $f$ quand on s'approche de l'infini.

Nous posons la définition suivante.
\begin{definition}      \label{DefInfNombre}
Lorsque $a\in\eR$, on dit que la fonction $f$ \defe{tend vers l'infini quand $x$ tend vers $a$}{} si
\[ 
  \forall M\in\eR,\exists \delta\tq (| x-a |\leq \delta )\Rightarrow f(x)\geq M\text{ quand $x\in\dom f$}.
\]
\end{definition}
Cela signifie que l'on demande que dès que $x$ est assez proche de $a$ (c'est à dire dès que $| x-a |\leq\delta$), alors $f(x)$ est plus grand que $M$, et que l'on peut trouver un $\delta$ qui fait ça pour n'importe quel $M$. Une autre façon de le dire est que pour toute hauteur $M$, on peut trouver un intervalle de largeur $\delta$ autour de $a$\footnote{C'est à dire un intervalle de la forme $[a-\delta,a+\delta]$.} tel que sur cet intervalle, la fonction $f$ est toujours plus grande que $M$.

Montrons sur un dessin pourquoi je disais que la fonction $x\to 1/x$ n'est pas de ce type.


Le problème est qu'il n'existe par exemple aucun intervalle autour de $0$ sur lequel $f$ serait toujours plus grande que $10$. En effet n'importe quel intervalle autour de $0$ contient au moins un nombre négatif. Or quand $x$ est négatif, $f$ n'est certainement pas plus grande que $10$. Nous y reviendrons.

Pour l'instant, montrons que la fonction $f(x)=1/x^2$ est une fonction qui vérifie la définition \ref{DefInfNombre}.  Avant de prendre n'importe quel $M$, prenons par exemple $100$. Nous avons besoin d'un intervalle autour de zéro sur lequel $f$ est toujours plus grande que $100$. C'est vite vu que $f(0.1)=f(-0.1)=100$, donc l'intervalle $[-\frac{ 1 }{ 10 },\frac{1}{ 10 }]$ est le bon. Partout dans cet intervalle, $f$ est plus grande que $100$. Partout ? Ben non : en $x=0$, la fonction n'est même pas définie, donc c'est un peu dur de dire qu'elle est plus grande que $100$. C'est pour cela que nous avons ajouté la condition \og quand $x\in\dom f$\fg{} dans la définition de la limite.

Prenons maintenant un $M\in\eR$ arbitraire, et trouvons un intervalle autour de $0$ sur lequel $f$ est toujours plus grande que $M$. La réponse est évidement l'intervalle de largeur $1/\sqrt{M}$, c'est à dire 
\[ 
  \left[ -\frac{ 1 }{ \sqrt{M} },\frac{ 1 }{ \sqrt{M} } \right].
\]

\subsection{Limites quand tout va bien}
%--------------------------------------

D'abord définissons ce qu'on entend par la limite d'une fonction en un point quand il n'y a aucun infini en jeu.
\begin{definition}      \label{DefLimPointSansInfini}
 On dit que la fonction $f$ \defe{tend vers $b$ quand $x$ tend vers $a$}{} si 
\[ 
  \forall \epsilon>0,\exists\delta\tq (| x-a |\leq\delta)\Rightarrow | f(x)-b |\leq \epsilon\text{ quand $x\in\dom f$}.
\]
Dans ce cas, nous notons
\begin{equation}
\lim_{x\to a}f(x)=b.
\end{equation} 
\end{definition}

Commençons par un exemple très simple : prouvons que $\lim_{x\to 0}x=0$. C'est donc $a=b=0$ dans la définition. Prenons $\epsilon>0$, et trouvons un intervalle autour de zéro tel que partout dans l'intervalle, $x\leq \epsilon$. Bon ben c'est clair que $\delta=\epsilon$ fonctionne.

Plus compliqué maintenant, mais toujours sans surprises.

\begin{proposition}
\[ 
  \lim_{x\to 0}x^2=0.
\]

\end{proposition}

\begin{proof}
Soit $\epsilon>0$. On veut un intervalle de largeur $\delta$ autour de zéro tel que $x^2$ soit plus petit que $\epsilon$ sur cet intervalle. Cette fois-ci, le $\delta$ qui fonctionne est $\delta=\sqrt{\epsilon}$. En effet un élément de l'intervalle $[-\delta,\delta]$ est un $r$ de valeur absolue plus petite ou égale à $\delta$ : 
\[ 
| r |\leq\delta=\sqrt{\epsilon}.
\]
En prenant le carré de cette inégalité on a :
\[ 
  r^2\leq\epsilon,
\]
ce qu'il fallait prouver.
\end{proof}


Calculer et prouver des valeurs de limites, mêmes très simples, devient vite de l'arrachage de cheveux à essayer de trouver le bon $\delta$ en fonction de $\epsilon$ si on n'a pas quelque théorèmes généraux. Nous allons donc maintenant en prouver quelque-uns.

\begin{theorem}     \label{ThoLimLinMul}
    Si
    \begin{equation} \label{Eqhypmullimlin}
      \lim_{x\to a}f(x)=b,
    \end{equation}
    alors
    \begin{equation} \label{Eqbutmultlim}
      \lim_{x\to a}(\lambda f)(x)=\lambda b
    \end{equation}
    pour n'importe quel $\lambda\in\eR$.
\end{theorem}

\begin{proof}
Soit $\epsilon>0$. Afin de prouver la propriété \eqref{Eqbutmultlim}, il faut trouver un $\delta$ tel que pour tout $x$ dans $[a-\delta,a+\delta]$, on ait $| (\lambda f)(x)- \lambda b |\leq\epsilon$. Cette dernière inégalité est équivalente à $|\lambda|| f(x)-b |\leq\epsilon$. Nous devons donc trouver un $\delta$ tel que 
\begin{equation} 
| f(x)-b |\leq\frac{ \epsilon }{ | \lambda | }.
\end{equation}
soit vraie pour tout $x$ dans $[a-\delta,a+\delta]$. Mais l'hypothèse \eqref{Eqhypmullimlin} dit précisément qu'il existe un $\delta$ tel que pour tout $x$ dans $[a-\delta,a+\delta]$ on ait cette inégalité. 
\end{proof}

\begin{theorem}     \label{ThoLimLin}
    Si
    \begin{subequations}
    \begin{align}
        \lim_{x\to a}f(x)&=b_1\\
        \lim_{x\to a}g(x)&=b_2,
    \end{align}
    \end{subequations}
    alors
    \begin{equation}
        \lim_{x\to a}(f+g)(x)=b_1+b_2.
    \end{equation}
\end{theorem}

\begin{proof}
    Soit $\epsilon>0$. Par hypothèse, il existe $\delta_1$ tel que
    \begin{equation}    \label{Eqfbunepsdeux}
      | f(x)-b_1 |\leq \frac{ \epsilon }{ 2 }
    \end{equation}
    dès que $| x-a |\leq\delta_1$. Il existe aussi $\delta_2$ tel que 
    \begin{equation}    \label{Eqgbdeuxepsdeux}
      | g(x)-b_2 |\leq \frac{ \epsilon }{ 2 }.
    \end{equation}
    dès que $| x-a |\leq \delta_2$. Tu notes l'astuce de prendre $\epsilon/2$ dans la définition de limite pour $f$ et $g$. Maintenant, ce qu'on voudrait c'est un $\delta$ tel que l'on ait $| (f+g)(x)-(b_1+b_2) |\leq \epsilon$ dès que $| x-a |\leq \delta$. Moi je dit que $\delta=\min\{ \delta_1,\delta_2 \}$ fonctionne. En effet, en utilisant l'inégalité $| a+b |\leq | a |+| b |$, nous trouvons :
    \begin{align}
    | (f+g)(x)-(b_1+b_2) |=| (f(x)-b_1)+(g(x)-b_2) |
            \leq | f(x)-b_1 |+| g(x)-b_2 |.     \label{Eqfplusgfbun}
    \end{align}
    Comme on suppose que $| x-a |\leq\delta$, on a évidement $| x-a |\leq\delta_1$, et donc l'équation \eqref{Eqfbunepsdeux} tient. Mais si $| x-a |\leq\delta$, on a aussi $| x-a |\leq\delta_2$, et donc l'équation  \eqref{Eqfbunepsdeux} tient également. Chacun des deux termes de \eqref{Eqfplusgfbun} est donc plus petits que $\epsilon/2$, et donc le tout est plus petit que $\epsilon$, ce qu'il fallait montrer.

\end{proof}

Une formule qui résume ces deux théorèmes est que
\begin{equation}    \label{EqLimLinRes}
    \lim_{x\to a}[\alpha f(x)+\beta g(x)]=\alpha\lim_{x\to a}f(x)+\beta\lim_{x\to a}g(x).
\end{equation}

\begin{lemma}       \label{LemLimMajorableVois}
    Si $\lim_{x\to a}f(x)=b$ avec $a$, $b\in\eR$, alors il existe un $\delta>0$ et un $M>0$ tels que 
    \[ 
        (| x-a |\leq\delta)\Rightarrow | f(x) |\leq M.
    \]

\end{lemma}

Ce que signifie ce lemme, c'est que quand la fonction $f$ admet une limite finie en un point, alors il est possible de majorer la fonction sur un intervalle autour du point.

\begin{proof}
    Cela va être démontré par l'absurde. Supposons qu'il n'existe pas de $\delta$ ni de $M$ qui vérifient la condition. Dans ce cas, pour tout $\delta$ et pour tout $M$, il existe un $x$ tel que $| x-a |\leq\delta$ et $| f(x) |> M$. Cela est valable pour tout $M$, donc prenons par exemple $b+1000$. Donc 
    \begin{equation}
    \forall\delta>0,\exists x\text{ tel que } | x-a |\leq\delta\text{ et }| f(x) |>b+1000.
    \end{equation}
    Cela signifie qu'aucun $\delta$ ne peut convenir dans la définition de $\lim_{x\to a}f(x)=b$, ce qui contredit les hypothèses.
\end{proof}

Dans le même ordre d'idée, on peut prouver que si la limite de la fonction en un point est positive, alors elle est positive autour ce ce point. Plus précisément, nous avons la
\begin{proposition} \label{PropoLimPosFPos}
    Si $f$ est une fonction telle que $\lim_{x\to a}f(x)>0$, alors il existe un voisinage de $a$ sur lequel $f$ est positive.
\end{proposition}   

\begin{proof}
    Supposons que $\lim_{x\to a}f(x)=y_0$. Par la définition de la limite fait que si pour tout $x$ dans un voisinage autour de $a$, on ait $| f(x)-a |<\epsilon$. Cela est valable pour tout $\epsilon$, pourvu que le voisinage soit assez petit. Si je choisit un voisinage pour lequel $| f(x)-a |<\frac{ y_0 }{ 2 }$, alors sur ce voisinage, $f$ est positive.
\end{proof}


\begin{theorem}     \label{Tholimfgabab}
    Si
    \begin{align}
        \lim_{x\to a}f(x)&=b_1&\text{et}&&\lim_{x\to a}g(x)=b_2,
    \end{align}
    alors
    \begin{equation}
        \lim_{x\to a}(fg)(x)=b_1b_2.
    \end{equation}
\end{theorem}

\begin{proof}
    Soit $\epsilon>0$, et tentons de trouver un $\delta$ tel que $| f(x)g(x)-b_1b_2 |\leq \epsilon$ dès que $| x-a |\leq \delta$. Nous avons 
    \begin{equation}    \label{EqfgbunbdeuxMin}
    \begin{split}
    | f(x)g(x)-b_1b_2 |&=|  f(x)g(x)-b_1b_2 +f(x)b_2-f(x)b_2 |\\
            &=\left|   f(x)\big( g(x)-b_2 \big)+b_2\big( f(x)-b_1 \big)    \right|\\
            &\leq \left|  f(x)\big( g(x)-b_2 \big)  \right|+\left|  b_2\big( f(x)-b_1 \big)    \right|\\
            &= | f(x) | | g(x)-b_2  |+| b_2 | |f(x)-b_1 |.  
    \end{split}
    \end{equation}
    À la première ligne se trouve la subtilité de la démonstration : on ajoute et on enlève\footnote{Comme exercice, tu peux essayer de refaire la démonstration en ajoutant et enlevant $g(x)b_1$ à la place.} $f(x)b_2$. Maintenant nous savons par le lemme \ref{LemLimMajorableVois} que pour un certain $\delta_1$, la quantité $| f(x) |$ peut être majoré par un certain $M$ dès que $| x-a |\leq \delta_1$. Prenons donc un tel $\delta_1$ et supposons que $| x-a |\leq \delta_1$. Nous savons aussi que pour n'importe quel choix de $\epsilon_2$ et $\epsilon_3$, il existe des nombres $\delta_2$ et $\delta_3$ tels que $| f(x)-b_1 |\leq \epsilon_2$ et $| g(x)-b_1 |\leq \epsilon_3$ dès que $| x-a |\leq\delta_2$ et $| x-a |\leq\delta_3$. Dans ces conditions, la dernière expression \eqref{EqfgbunbdeuxMin} se réduit à
    \begin{equation}
    | f(x)g(x)-b_1b_2 |\leq M\epsilon_2+| b_2 |\epsilon_3.
    \end{equation}
    Pour terminer la preuve, il suffit de choisir $\epsilon_2$ et $\epsilon_3$ tels que $M\epsilon_2+| b_2 |\epsilon_3\leq\epsilon$, et puis prendre $\delta=\min\{ \delta_1,\delta_2,\delta_3 \}$.


    Remetons les choses dans l'ordre. L'on se donne $\epsilon$ au départ. La première chose est de trouver un $\delta_1$ qui permet de majorer $|f(x)|$ par $M$ selon le lemme \ref{LemLimMajorableVois}, et puis choisissons $\epsilon_2$ et $\epsilon_3$ tels que $M\epsilon_2+| b_2 |\epsilon_3\leq\epsilon$. Ensuite nous prenons, en vertu des hypothèses de limites pour $f$ et $g$, les nombres $\delta_2$ et $\delta_3$ tels que $| f(x)-b_1 |\leq \epsilon_2$ et $| g(x)-b_2 |\leq \epsilon_3$ dès que $| x-a |\leq \delta_2$ et $| x-a |\leq \delta_3$.

    Si avec tous ça on prend $\delta=\min\{ \delta_1,\delta_2,\delta_3 \}$, alors la majoration et les deux inégalités sont valables en même temps et au final
    \[ 
      | f(x)g(x)-b_1b_2 |\leq M\epsilon_2+b_2\epsilon_3\leq \epsilon,
    \]
    ce qu'il fallait prouver.

\end{proof}

À l'aide de ces petits résultats, nous pouvons déjà calculer pas mal de limites. Nous pouvons déjà par exemple calculer les limites de tous les polynomes en tous les nombrs réels. En effet, nous savons la limite de la fonction $f(x)=x$. la fonction $x\mapsto x^2$ n'est rien d'autre que le produit de $f$ par elle-même. Donc
\[ 
  \lim_{x\to a}x^2=\big( \lim_{x\to a}x\big)\cdot\big( \lim_{x\to a}x \big)=a^2.
\]
De la même façon, nous trouvons facilement que 
\begin{equation}
 \lim_{x\to a}x^n=a^n.
\end{equation}

\begin{theorem}[Limite et continuité]           \label{ThoLimCont}
La fonction $f$ est continue au point $a$ si et seulement si $\lim_{x\to a}f(x)=f(a)$.
\end{theorem}

\begin{proof}
Nous commençons par supposer que $f$ est continue en $a$, et nous prouvons que $\lim_{x\to a}f(x)=a$. Soit $\epsilon>0$; ce qu'il nous faut c'est un $\delta$ tel que $| x-a |\leq\delta$ implique $| f(x)-f(a) |\leq\epsilon$. La définition \ref{DefContinue} de la continuité donne l'existence d'un $\delta$ comme il nous faut.

Dans l'autre sens, c'est à dire prouver que $f$ est continue au point $a$ sous l'hypothèse que $\lim_{x\to a}f(x)=f(a)$, la preuve se fait de la même façon.
\end{proof}

Nous en déduisons que si nous voulons gagner quelque chose à parler de limites, il faut prendre des fonctions non continues. Prenons une fonction qui fait un saut. Pour se fixer les idées, prenons celle-ci :
\begin{equation}    \label{EqnCtOEL}
f(x)=
\begin{cases}
2x&\text{si $x\in]\infty,2[$}\\
x/2&\text{si $x\in[2,\infty[$}
\end{cases}
\end{equation}  
Essayons de trouver la limite de cette fonction lorsque $x$ tend vers $2$. Étant donné que $f$ n'est pas continue en $2$, nous savons déjà que $\lim_{x\to 2}f(x)\neq f(2)$. Donc ce n'est pas $1$. Cette limite ne peut pas valoir $4$ non plus parce que si je prends n'importe quel $\epsilon$, la valeur de $f(2+\epsilon)$ est très proche de $2$, et donc ne peut pas s'approcher de $4$. En fait, tu peux facilement vérifier que \emph{aucun nombre ne vérifie la condition de limite pour $f$ en $2$}. Nous disons que la limite n'existe pas.

Pour résumer, les limites qui ne font pas intervenir l'infini ne servent à rien parce que
\begin{itemize}
\item si la fonction est continue, la limite est simplement la valeur de la fonction par le théorème \ref{ThoLimCont},
\item si la fonction fait un saut, alors la limite n'existe pas (nous n'avons pas prouvé cela en général, mais avoue que l'exemple est convainquant).
\end{itemize}
Nous avons même la proposition suivante :
\begin{proposition}     \label{PropExisteLimVql}
Si $f$ existe en $a$ (c'est à dire si $a\in\dom(f)$) et si $\lim_{x\to a}f(x)=b$, alors $f(a)=b$.
\end{proposition}

\begin{proof}
Du fait que $\lim_{x\to a}f(x)=b$, il découle que pour tout $\epsilon$, il existe un $\delta$ tel que $| x-a |\leq \delta$ implique $| f(x)-b |\leq \epsilon$. Il est évident que pour tout $\delta$, $| x-x |\leq \delta$, donc nous avons que 
\[ 
  | f(a)-b |\leq\epsilon
\]
pour tout $\epsilon$. Cela implique que $f(a)=b$.
\end{proof}
Notons toutefois que l'inverse de cette proposition n'est pas vraie : la fonction \eqref{EqnCtOEL} donne justement une fonction qui prend la valeur $1$ en $2$ sans que la limite en $2$ soit $1$. Quoi qu'il en soit, cette proposition achève de nous convaincre de l'inutilité d'étudier d'étudier les limites sans infinis : dès qu'on a une limite, à tous les coups c'est la valeur de la fonction \ldots heu \ldots en es-tu bien sûr ?


%---------------------------------------------------------------------------------------------------------------------------
\subsection{Discussion avec mon ordinateur}
%---------------------------------------------------------------------------------------------------------------------------

Voici un extrait de ce peut donner Sage. Nous lui donnons la fonction
\begin{equation}    \label{EqyEHTBZ}
    f(x)=\frac{ x+4 }{ 3x^2+10x-8 }.
\end{equation}
Cette fonction est faite exprès pour que le dénominateur s'annule en \( -4\). En fait \( 3x^2+10x-8=(x+4)(3x-2)\), et la fraction peut se simplifier en
\begin{equation}
    f(x)=\frac{1}{ 3x-2 }.
\end{equation}
Et avec cela nous écririons \( f(-4)=-\frac{1}{ 14 }\). Voyons comment cela passe dans Sage.

\begin{verbatim}
----------------------------------------------------------------------
| Sage Version 5.2, Release Date: 2012-07-25                         |
| Type "notebook()" for the browser-based notebook interface.        |
| Type "help()" for help.                                            |
----------------------------------------------------------------------
sage: f(x)=(x+4)/(3*x**2+10*x-8)                                                                                              
sage: f(-4)
---------------------------------------------------------------------------
ValueError                                Traceback (most recent call last)
ValueError: power::eval(): division by zero
\end{verbatim}
Il produit donc une erreur de division par zéro. Cela n'est pas étonnant. Pourtant si on lui demande, il est capable de simplifier. En effet :
\begin{verbatim}
sage: f.simplify_full()                                                                                                        
x |--> 1/(3*x - 2)                                                                                                                                           
sage: f.simplify_full()(-4)                                                                                                                                  
-1/14                                                                                                                                                        
\end{verbatim}

\subsection{Limites et prolongement}
%-----------------------------------

La proposition \ref{PropExisteLimVql} a une terrible limitation : il faut que la fonction existe au point considéré. Or en regardant bien la définition \ref{DefLimPointSansInfini}, nous remarquons que $\lim_{x\to a}f(x)$ peut très bien exister sans que $f(a)$ n'existe.

Reprenons l'exemple de la fonction \eqref{EqyEHTBZ} que mon ordinateur refusait de calculer en zéro :
\begin{equation}
f(x)=\frac{ x+4 }{ 3x^2+10x-8 }=\frac{ x+4 }{ (x+4)\left( x-\frac{ 2 }{ 3 } \right) }.
\end{equation}
Cette fonction a une condition d'existence en $x=-4$. Et pourtant, tant que $x\neq 4$, cela a un sens de simplifier les $(x+4)$ et d'écrire
\[ 
  f(x)=\frac{ 1 }{ x-\frac{ 2 }{ 3 } }=\frac{ 3 }{ 3x-2 }.
\]
Étant donné que pour toute valeur de $x$ différente de $-4$, la fonction $f$ s'exprime de cette façon, nous avons que
\[ 
  \lim_{x\to -4}f(x)=\lim_{x\to -4}\left(\frac{ 3 }{ 3x-2 }\right).
\]
Oui, mais la fonction\footnote{Cette fonction $g$ n'est pas $f$ parce que $g$ a en plus l'avantage d'être définie en $-4$.} $g(x)=3/(3x-2)$ est continue en $-4$ et donc sa limite vaut sa valeur. Nous en déduisons que
\[ 
  \lim_{x\to -4}f(x)=-\frac{ 3 }{ 14 }.
\]
Que dire maintenant de la fonction ainsi définie ?
\begin{equation}
\tilde f(x)=
\begin{cases}
f(x)&\text{si $x\neq -4$}\\
-3/14&\text{si $x=-4$}.
\end{cases}
\end{equation}
Cette fonction est continue en $-4$ parce qu'elle y est égale à sa limite. Les étapes suivies pour obtenir ce résultat sont :
\begin{itemize}
\item Repérer un point où la fonction n'existe pas,
\item calculer la limite de la fonction en ce point, et en particulier vérifier que cette limite existe, ce qui n'est pas toujours le cas,
\item définir une nouvelle fonction qui vaut partout la même chose que la fonction originale, sauf au point considéré où l'on met la valeur de la limite.
\end{itemize}
C'est ce qu'on appelle \defe{prolonger la fonction par continuité}{prolongement!par continuité} parce que la fonction résultante est continue. La prolongation de $f$ par continuité est donc en général définie par
\begin{equation}
\tilde f(x)=
\begin{cases}
f(x)            &\text{si $f(x)$ existe}\\
\lim_{y\to x}f(y)   &\text{si $f(x)$ si cette limite existe et est finie.}
\end{cases}
\end{equation}
Dans le cas que nous regardions, 
\[ 
    f(x)=\frac{ x+4 }{ 3x^2+10x-8 },
\]
le prolongement par continuité est donné par
\begin{equation}
\tilde f =\frac{ 3 }{ 3x-2 }.
\end{equation}
Remarque que cette fonction n'est toujours pas définie en $x=2/3$. 



\section{Continuité et dérivabilité}
\label{seccontetderiv}

On considère dans la suite une fonction $f : A \to \eR$, où $a \in A \subset \eR$ ; cependant, les notions de continuité et de dérivabilité se généralisent immédiatement au cas de fonctions à valeurs vectorielles ; la notion de continuité se généralise au cas des fonctions à plusieurs variables (la notion de dérivabilité est remplacée par celle de différentiabilité dans ce cadre).

\begin{definition}
    La fonction $f$ est \defe{dérivable}{dérivable} en \( a\) si $a \in
  \operatorname{int} A$ et si
  \begin{equation*}
    \lim_{\substack{x\rightarrow a\\x\neq a}} \frac{f(x)-f(a)}{x-a}
  \end{equation*}
  existe. On note alors cette quantité $f^\prime(a)$, c'est le nombre
  dérivé de $f$ en $a$. La \Defn{fonction dérivée} de $f$ est
  \begin{equation*}
    f^\prime : A^\prime \to \eR : a \mapsto f^\prime(a)
  \end{equation*}
  définie sur l'ensemble noté $A^\prime$ des points $a$ où $f$ est
  dérivable.
\end{definition}

\begin{example}
      Montrons que la fonction $f : \eR \to \eR : x\mapsto x$ est continue et dérivable. Exceptionnellement (bien qu'on sache que la dérivabilité implique la continuité), montrons ces deux assertions séparément.
      \begin{description}
      \item[Continuité] Pour prouver la continuité au point $a \in \eR$ nous devons montrer que
     \begin{equation}
       \limite x a x = a
     \end{equation}
     c'est-à-dire
     \begin{equation}
       \forall \epsilon > 0, \exists \delta > 0 :  \forall x \in \eR \abs{x-a} <
       \delta \Rightarrow \abs{x-a} < \epsilon
     \end{equation}
     ce qui est clair en prenant $\delta = \epsilon$.

      \item[Dérivabilité] Soit $a \in \eR$. Calculons la limite du quotient différentiel
        \begin{equation}
          \limite[x\neq a]{x}{a} \frac{x-a}{x-a} = \limite[x\neq a]x a 1 = 1
        \end{equation}
        ce qui prouve que $f$ est dérivable et que sa dérivée vaut $1$ en
        tout point $a$ de $\eR$.
      \end{description}

     On a donc montré que la fonction $x \mapsto x$ est continue, dérivable, et que sa dérivée vaut $1$ en tout point $a$ de son domaine.

\end{example}

\begin{proposition} \label{PropSFyxOWF}
    Une fonction dérivable sur un intervalle y est continue.
\end{proposition}

\begin{proof}
    Soit \( I\) un intervalle sur lequel la fonction \( f\) est dérivable, et soit \( x_0\in I\). Nous allons prouver la continuité de \( f\) en \( x_0\). Le fait que la limite
    \begin{equation}
        f'(x_0)=\lim_{h\to 0} \frac{ f(x_0+h)-f(x_0) }{ h }
    \end{equation}
    existe implique a fortiori que 
    \begin{equation}
        \lim_{h\to 0} f(x_0+h)-f(x_0)=0.
    \end{equation}
    Cela signifie la continuité de \( f\) en vertu du critère \ref{ThoLimCont}.
\end{proof}

%--------------------------------------------------------------------------------------------------------------------------- 
\subsection{Développement limité au premier ordre}
%---------------------------------------------------------------------------------------------------------------------------

Si une fonction est dérivable en \( a\) alors elle peut être approximée «au premier ordre» par une formule simple.
\begin{proposition}[Développement limité au premier ordre]  \label{PropUTenzfQ}
    Si \( f\) est dérivable en \( a\) alors nous avons la formule
    \begin{equation}
        f(a+h)=f(a)+hf'(a)+\alpha(h)
    \end{equation}
    pour une fonction \( \alpha\) telle que
    \begin{equation}
        \lim_{h\to 0} \frac{ \alpha(h) }{ h }=0.
    \end{equation}
\end{proposition}
\index{développement!limité!premier ordre}
Ce résultat sera généralisé pour des dérivées d'ordre supérieures avec les séries de Taylor, théorème \ref{ThoTaylor}.

\begin{proof}
    La fonction \( f\) étant dérivable en \( a\) nous avons l'existence de la limite suivante :
    \begin{equation}
        f'(a)=\lim_{h\to 0} \frac{ f(a+h)-f(a) }{ h },
    \end{equation}
    ce qui revient à dire qu'en définissant la fonction \( \beta\) par
    \begin{equation}
        f'(a)=\frac{ f(a+h)-f(a) }{ h }+\beta(h)
    \end{equation}
    alors \( \beta(h)\to 0\) lorsque \( h\to 0\). En multipliant par \( h\) et en nommant \( \alpha(h)=h\beta(h)\) nous trouvons le résultat :
    \begin{equation}
        f(a+h)=f(a)+hf'(a)+\alpha(h)
    \end{equation}
    avec 
    \begin{equation}
        \lim_{h\to 0} \frac{ \alpha(h) }{ h }=\lim_{h\to 0} \beta(h)=0.
    \end{equation}
\end{proof}

%+++++++++++++++++++++++++++++++++++++++++++++++++++++++++++++++++++++++++++++++++++++++++++++++++++++++++++++++++++++++++++
\section{Espace des fonctions continues}
%+++++++++++++++++++++++++++++++++++++++++++++++++++++++++++++++++++++++++++++++++++++++++++++++++++++++++++++++++++++++++++

\begin{definition}
    Soit \( I\), un intervalle de \( \eR\). L'\defe{oscillation}{oscillation!d'une fonction} sur \( I\) est le nombre
    \begin{equation}
        \omega_f(I)=\sup_{x\in I}f(x)-\inf_{x\in I}f(x).
    \end{equation}
\end{definition}
    Pour chaque \( x\) fixé, la fonction
    \begin{equation}
        x\mapsto \omega_f\big( B(x,\delta) \big)
    \end{equation}
    est une fonction positive, croissante et a donc une limite (pour \( \delta\to 0\)). Nous notons \( \omega_f(x)\) cette limite qui est l'\defe{oscillation}{oscillation!d'une fonction en un point} de \( f\) en ce point. Une propriété immédiate est que \( f\) est continue en \( x_0\) si et seulement si \( \omega_f(x_0)=0\).

    \begin{lemma}       \label{LemuaPbtQ}
    L'ensemble des points de discontinuité d'une fonction \( f\colon \eR\to \eR\) est une réunion dénombrable de fermés.
\end{lemma}

\begin{proof}
    Soit \( D\) l'ensemble des points de discontinuité de \( f\). Nous avons
    \begin{equation}
        D=\bigcup_{n=1}^{\infty}\{ x\tq \omega_f(x)\geq \frac{1}{ n } \}.
    \end{equation}
    Il nous suffit donc de montrer que pour tout \( \epsilon\), l'ensemble
    \begin{equation}
        \{ x\tq \omega_f(x)<\epsilon \}
    \end{equation}
    est ouvert. Soit en effet \( x_0\) dans cet ensemble. Il existe \( \delta\) tel que \( \omega_f\big( B(x_0,\delta) \big)<\epsilon\). Si \( x\in B(x_0,\delta)\), alors si on choisit \( \delta'\) tel que \( B(x,\delta')\subset B(x_0,\delta)\), nous avons \( \omega_f\big( B(x,\delta') \big)<\epsilon\), ce qui justifie que \( \omega_f(x)<\epsilon\) et donc que \( x\) est également dans l'ensemble considéré.
\end{proof}

\begin{theorem}
    L'ensemble des points de discontinuité d'une limite simple de fonctions continues est de première catégorie.
\end{theorem}

\begin{proof}
    Soit \( (f_n)\) une suite de fonctions convergent simplement vers \( f\). Nous devons écrire l'ensemble des points de discontinuité de \( f\) comme une union dénombrable d'ensembles tels que sur tout intervalle \( I\), aucun de ces ensembles n'est dense. Nous savons déjà par le lemme \ref{LemuaPbtQ} que l'ensemble des points de discontinuité  de \( f\) est donné par
    \begin{equation}
        D=\bigcup_{n=1}^{\infty}\{ x\tq \omega_f(x)\geq \frac{1}{  n } \}.
    \end{equation}
    Nous essayons donc de prouver que pour tout \( \epsilon\), l'ensemble 
    \begin{equation}
        F=\{ x\tq \omega_f(x)\geq \epsilon \}
    \end{equation}
    est nulle part dense. Soit
    \begin{equation}
        E_n=\bigcap_{i,j>n}\{ x\tq | f_i(x)-f_j(x) |<\epsilon \}.
    \end{equation}
    Nous montrons que cet ensemble est fermée en étudiant le complémentaire. Soit \( x\notin E_n\); alors il existe un couple \( (i,j)\) tel que
    \begin{equation}
        | f_i(x)-f_j(x) |>\epsilon.
    \end{equation}
    Par continuité, cette inégalité reste valide dans un voisinage de \( x\). Donc il existe un voisinage de \( x\) contenu dans \( \complement E_n\) et \( E_n\) est donc fermé.

    De plus nous avons \( E_n\subset E_{n+1}\) et \( \bigcup_nE_n=\eR\). Ce dernier point est dû au fait que pour tout \( x\), il existe \( N\) tel que \( i,j>N\) implique \( | f_i(x)-f_j(x) |\leq \epsilon\). Cela est l'expression du fait que la suite \( \big( f_n(x) \big)_{n\in \eN}\) est de Cauchy.

    Soit \( I\), un intervalle fermé de \( \eR\). Nous voulons trouver un intervalle \( J\subset I\) sur lequel \( f\) est continue. Nous écrivons \( I\) sous la forme 
    \begin{equation}
        I=\bigcup_{n=1}^{\infty}(E_n\cap I).
    \end{equation}
    Tous les ensembles \( J_n=E_n\cap I\) ne peuvent être nulle part dense en même temps (à cause du théorème de Baire \ref{ThoQGalIO}). Il existe donc un \( n\) tel que \( J_n\) contienne un ouvert \( J\). Le but est de montrer que \( f\) est continue sur \( J\). Pour ce faire, nous n'allons pas simplement majorer \( | f(x)-f(x_0) |\) par \( \epsilon\) lorsque \( | x-x_0 |\) est petit. Ce que nous allons faire est majorer l'oscillation de \( f\) sur \( B(x_0,\delta)\) lorsque \( \delta\) est petit. Pour cela nous prenons \( x_0\) et \( x\) dans \( J\) et nous écrivons
    \begin{equation}
        | f(x)-f(x_0) |\leq | f(x)-f_n(x) |+| f_n(x)-f_n(x_0) |.
    \end{equation}
    À ce niveau nous rappelons que \( n\) est fixé par le choix de \( J\), dans lequel \( \epsilon\) est déjà inclus. Nous choisissons évidemment \( | x-x_0 |\leq \delta\) de telle sorte que le second terme soit plus petit que \( \epsilon\) en vertu de la continuité de \( f_n\). Pour le premier terme, pour tout \( i,j\geq n\) nous avons
    \begin{equation}
        | f_i(x)-f_j(x) |<\epsilon.
    \end{equation}
    Si nous posons \( j=n\) et \( i\to\infty\), en tenant compte du fait que \( f_i\to f\) simplement,
    \begin{equation}
        | f(x)-f_n(x) |\leq \epsilon.
    \end{equation}
    Nous avons donc obtenu \( | f(x)-f_n(x_0) |\leq 2\epsilon\). Cela signifie que dans un voisinage de rayon \( \delta\) autour de \( x_0\), les valeurs extrêmes prises par \( f(x) \) sont \( f_n(x_0)\pm 4\epsilon\). Nous avons donc prouvé que pour tout \( \epsilon\), il existe \( \delta\) tel que
    \begin{equation}
        \omega_f\big( \mathopen[ x_0-\delta , x_0+\delta \mathclose] \big)\leq 4\epsilon.
    \end{equation}
    De là nous concluons que
    \begin{equation}
        \lim_{\delta\to 0}\omega_f\big( \mathopen[ x_0-\delta , x_0+\delta \mathclose] \big)=0,
    \end{equation}
    ce qui signifie que \( f\) est continue en \( x_0\).
\end{proof}

\begin{example}
    Une fonction discontinue sur \( \eQ\) et continue ailleurs. La fonction 
    \begin{equation}
        f(x)=\begin{cases}
            0    &   \text{si \( x\notin \eQ\)}\\
            \frac{1}{ q }    &    \text{si \( x=p/q\)}
        \end{cases}
    \end{equation}
    où par «\( x=p/q\)» nous entendons que \( p/q\) est la fraction irréductible.

    Cette fonction est discontinue sur \( \eQ\) parce que si \( q\in \eQ\) alors \( f(q)\neq 0\) alors que dans tous voisinage de \( q\) il existe un irrationnel sur qui la fonction vaudra zéro.

    Montrons que \( f\) est continue sur les irrationnels. Si \( x_0\notin \eQ\) alors \( f(x_0)=0\). Mais si on prend un voisinage suffisamment petit de \( x_0\), nous pouvons nous arranger pour que tous les rationnels aient un dénominateur arbitrairement grand. En effet si nous nous fixons un premier rayon \( r_0>0\) alors il existe un nombre fini de fractions de la forme \( 1\), \( \frac{ k }{2}\), \( \frac{ k }{ 3 }\),\ldots, \( \frac{ k }{ N }\) dans \( B(x_0,r_0)\). Il suffit maintenant de choisir \( 0<r\leq r_0\) tel que ces fractions soient toutes hors de \( B(x_0,r)\). Dans cette boule nous avons \( f<\frac{1}{ N }\). Du coup \( f\) est continue en \( x_0\).
\end{example}

%+++++++++++++++++++++++++++++++++++++++++++++++++++++++++++++++++++++++++++++++++++++++++++++++++++++++++++++++++++++++++++
\section{Limites à plusieurs variables}
%+++++++++++++++++++++++++++++++++++++++++++++++++++++++++++++++++++++++++++++++++++++++++++++++++++++++++++++++++++++++++++
\label{SecLimVarsPlus}

Prenons une fonction $f\colon \eR^n\to \eR$. Nous disons que
\begin{equation}
    \lim_{x\to x_0}f(x)=l\in\eR
\end{equation}
lorsque $\forall \epsilon>0$, $\exists\delta$ tel que $\| x-x_0 \|\leq\delta$ implique $| f(x)-l |\leq \epsilon$. 

Remarquez qu'ici, $x\in\eR^n$, et sachez distinguer $\| . \|$, la norme dans $\eR^n$ de $| . |$ qui est la valeur absolue dans $\eR$. Une autre façon d'exprimer cette définition est que l'ensemble des valeurs atteintes par $f$ dans une boule de rayon $\delta$ autour de $x_0$ n'est pas très loin de $l$. Nous définissons donc
\begin{equation}
    E_{\delta}=\{ f(x)\tq x\in B(x_0,\delta) \}.
\end{equation}
Notez que si $f$ n'est pas définie en $x_0$, il n'y a pas de valeurs correspondantes au centre de la boule dans $E_{\delta}$. Ceci est évidement la situation générique lorsqu'il y a une indétermination à lever dans le calcul de la limite. Nous avons alors que
\begin{equation}
    \lim_{x\to x_0}f(x)=l
\end{equation}
lorsque $\forall\epsilon>0$, $\exists\delta$ tel que 
\begin{equation}        \label{Eqvmoinsrapplimdeux}
    \sup\{ | v-l |\tq v\in E_{\delta} \}\leq\epsilon.
\end{equation}
Une façon classique de montrer qu'une limite n'existe pas, est de prouver que, pour tout $\delta$, l'ensemble $E_{\delta}$ contient deux valeurs constantes. Si par exemple $0\in E_{\delta}$ et $1\in E_{\delta}$ pour tout $\delta$, alors aucune valeur de $l$ (même pas $l=\pm\infty$) ne peut satisfaire à la condition \eqref{Eqvmoinsrapplimdeux} pour toute valeur de $\epsilon$.

Nous laissons à la sagacité de l'étudiant le soin d'adapter tout ceci pour le cas $\lim_{x\to x_0}f(x)=\pm\infty$.

La proposition suivante semble évidente, mais nous sera tellement
utile qu'il est préférable de l'expliciter~:
\begin{proposition}
Soit $f : D \to \eR$ une fonction dont le domaine
  s'écrit comme une réunion \emph{finie}
  \begin{equation*}
    D = \bigcup_{i=1}^k A_i
  \end{equation*}  
  où $k$ est un entier. Soit $a \in \adh D$ tel que $a \in \adh A_i$
  pour tout $i \leq k$, et soit $b \in \eR$. Alors, la limite
  \begin{equation*}
    \limite x a f(x)
  \end{equation*}
  existe et vaut $b$ si et seulement si chacune des limites
  \begin{equation*}
    \limite[x \in A_i] x a f(x)
  \end{equation*}
  existe et vaut $b$.
\end{proposition}

\begin{proof}On sait déjà que si la limite de $f : D \to \eR$
  existe, alors toute restriction à $A_i$ admet la même limite. Il
  suffit donc de prouver la réciproque.

  Par hypothèse, pour tout $i = 1 \ldots k$, nous savons que
  \begin{equation*}
    \forall \epsilon > 0\, \exists \delta_i > 0 \tq (x \in A_i)
    \text{ et }
    (\norme{x-a} < \delta_i) \Rightarrow \norme{f(x) - b} < \epsilon
  \end{equation*}

  Si $\epsilon$ est fixé, posons $\delta = \min_i\{\delta_i\}$. Nous
  savons alors que
  \begin{enumerate}
  \item pour tout $x \in D$, il existe $i$ tel que $x \in A_i$, et
  \item si $x$ vérifie $\norme{x-a} < \delta$, alors pour tout $i$,
    $\norme{x-a} < \delta_i$ par définition de $\delta$.
  \end{enumerate}
  
  On en déduit que si $x \in D$ et $\norme{x-a} < \delta$, alors il
  existe $i$ tel que $x \in A_i$ et $\norme{x-a} < \delta_i$, ce qui
  implique $\abs{f(x) - b} < \epsilon$ et prouve la continuité.
\end{proof}

\begin{example}
  \begin{enumerate}
  \item Pour qu'une fonction $f : \eR \to \eR$ admette une limite en
    $a \in \eR$, il faut et il suffit qu'elle y admette une limite à
    droite et une limite à gauche qui soient égales.

  \item Une suite $(x_k)$ admet une limite si et seulement si les
    sous suites $(x_{2k})$ et $(x_{2k+1})$ convergent vers la même
    limite. Ceci n'est pas une application directe de la proposition,
    mais la teneur est la même.
  \end{enumerate}
\end{example}

%+++++++++++++++++++++++++++++++++++++++++++++++++++++++++++++++++++++++++++++++++++++++++++++++++++++++++++++++++++++++++++
\section{Uniforme continuité}		\label{SecUnifContinue}
%+++++++++++++++++++++++++++++++++++++++++++++++++++++++++++++++++++++++++++++++++++++++++++++++++++++++++++++++++++++++++++

\begin{definition}
	Une partie $A\subset\eR^m$ est dite \defe{bornée}{bornée!partie de $\eR^m$} si il existe un $M>0$ tel que $A\subset B(0,M)$. Le \defe{diamètre}{diamètre} de la partie $A$ est\nomenclature[T]{$\Diam(A)$}{Diamètre de la partie $A$} le nombre
	\begin{equation}
		\Diam(A)=\sup_{x,y\in A}\| x-y \|\in\mathopen[ 0 , \infty \mathclose].
	\end{equation}
\end{definition}
Lorsque $A$ est borné, il existe un $M$ tel que $\| x \|\leq M$ pour tout $x\in A$.

\begin{lemma}
	Si $A$ est une partie non vide de $\eR^m$, alors $\Diam(A)=\Diam(\bar A)$.
\end{lemma}
Nous n'allons pas donner de démonstrations de ce lemme.


Si $(x_n)$ est une suite et $I$ est un sous-ensemble infini de $\eN$, nous désignons par $x_I$ la suite des éléments $x_n$ tels que $n\in I$. Par exemple la suite $x_{\eN}$ est la suite elle-même, la suite $x_{2\eN}$ est la suite obtenue en ne prenant que les éléments d'indice pair.

Les suites $x_I$ ainsi construites sont dites des \defe{sous-suites}{sous-suite} de la suite $(x_n)$.


Pour une fonction $f\colon D\subset\eR^m\to \eR$, la continuité au point $a$ signifie que pour tout $\varepsilon>0$,
\begin{equation}
	\exists\delta>0\tq 0<\| x-a \|<\delta\Rightarrow | f(x)-f(a) |<\varepsilon.
\end{equation}
Le $\delta$ qu'il faut choisir dépend évidement de $\varepsilon$, mais il dépend en général aussi du point $a$ où l'on veut tester la continuité. C'est à dire que, étant donné un $\varepsilon>0$, nous pouvons trouver un $\delta$ qui fonctionne pour certains points, mais qui ne fonctionne pas pour d'autres points.

Il peut cependant également arriver qu'un même $\delta$ fonctionne pour tous les points du domaine. Dans ce cas, nous disons que la fonction est uniformément continue sur le domaine.

\begin{definition}
	Une fonction $f\colon D\subset\eR^m\to \eR$ est dite \defe{uniformément continue}{continue!uniformément} sur $D$ si
	\begin{equation}	\label{EqConditionUnifCont}
		\forall\varepsilon>0,\,\exists\delta>0\tq\,\forall x,y\in D,\,\| x-y \|\leq\delta \Rightarrow| f(x)-f(a) |<\varepsilon.
	\end{equation}
\end{definition}

Il est intéressant de voir ce que signifie le fait de \emph{ne pas} être uniformément continue sur un domaine $D$. Il s'agit essentiellement de retourner tous les quantificateurs de la condition \eqref{EqConditionUnifCont} :
\begin{equation}	\label{EqConditionPasUnifCont}
	\exists\varepsilon>0\tq\forall\delta>0,\,\exists x,y\in D\tq \| x-y \|<\delta\text{ et }\big| f(x)-f(y) \big|>\varepsilon.
\end{equation}
Dans cette condition, les points $x$ et $y$ peuvent être fonction du $\delta$. L'important est que pour tout $\delta$, on puisse trouver deux points $\delta$-proches dont les images par $f$ ne soient pas $\varepsilon$-proches.

\begin{example}
	Prenons la fonction $f(x)=\frac{1}{ x }$, et demandons nous pour quel $\delta$ nous sommes sûr d'avoir
	\begin{equation}
		| f(a+\delta)-f(a) |=\left| \frac{1}{ a+\delta }-\frac{1}{ a } \right| <\varepsilon.
	\end{equation}
	Pour simplifier, nous supposons que $a>0$. Nous calculons
	\begin{equation}
		\begin{aligned}[]
			\frac{ 1 }{ a }-\frac{1}{ a+\delta }&<	\varepsilon\\
			\frac{ \delta }{ a(a+\delta) }&<\varepsilon\\
			\delta&<\varepsilon a^2+\varepsilon a\delta\\
			\delta(1-\varepsilon a)&<\varepsilon a^2\\
			\delta&<\frac{ \varepsilon a^2 }{ 1-\varepsilon a }.
		\end{aligned}
	\end{equation}
	Notons que, à $\varepsilon$ fixé, plus $a$ est petit, plus il faut choisir $\delta$ petit. La fonction $x\mapsto\frac{1}{ x }$ n'est donc pas uniformément continue. Cela correspond au fait que, proche de zéro, la fonction monte très vite. Une fonction uniformément continue sera une fonction qui ne montera jamais très vite.
\end{example}

\begin{proposition}
	Quelque propriétés des fonctions uniformément continues.
	\begin{enumerate}
		\item
			Toute application uniformément continue est continue;
		\item
			la composée de deux fonctions uniformément continues est uniformément continue;
		\item
			tout application lipschitzienne est uniformément continues.
	\end{enumerate}
\end{proposition}

Une fonction peut être uniformément continue sur un domaine et pas sur un autre. Le théorème suivant donne une importante indication à ce sujet.
\begin{theorem}[Heine]\index{théorème!Heine}\index{Heine (théorème)}		\label{ThoHeineContinueCompact}
	Une fonction continue sur un compact (fermé et borné) est uniformément continue.
\end{theorem}

La démonstration qui suit est valable pour une fonction \( f\colon \eR^n\to \eR^m\) et utilise le fait que le produit cartésien de compacts est compact. Dans le cas de fonctions sur \( \eR\), nous pouvons modifier la démonstration pour ne pas utiliser ce résultat; voir plus bas.
%TODO : trouver où se trouve la preuve du produit de compacts et la référentier ici.
\begin{proof}
	Nous allons prouver ce théorème par l'absurde. Nous commençons par écrire la condition \eqref{EqConditionPasUnifCont} qui exprime que $f$ n'est pas uniformément continue sur le compact \( K\) :
	\begin{equation}
		\exists\varepsilon>0\tq\forall\delta>0,\,\exists x,y\in K\tqs \| x-y \|<\delta\text{ et }\big| f(x)-f(y) \big|>\varepsilon.
	\end{equation}
	En particulier (en prenant $\delta=\frac{1}{ n }$ pour tout $n$), pour chaque $n$ nous pouvons trouver $x_n$ et $y_n$ dans $K$ qui vérifient simultanément les deux conditions suivantes :
	\begin{subequations}
		\begin{numcases}{}
			\| x_n-y_n \|<\frac{1}{ n }\\
			\big| f(x_n)-f(y_n) \big|>\varepsilon.	\label{EqCond3107fxfyepsppt}
		\end{numcases}
	\end{subequations}
    Nous insistons que c'est le même $\varepsilon$ pour chaque $n$. L'ensemble $K$ étant compact, l'ensemble \( K\times K \) est compact (proposition \ref{PropIYmxXuu}) et nous pouvons trouver une sous-suite convergente \emph{du couple} \( (x_n,y_n)\) dans \( K\times K\). Quitte à passer à ces sous-suites, nous  nous supposons que \( (x_n,y_n)\) converge dans \( K\times K\) et en particulier que les suites $(x_n)$ et $(y_n)$ sont convergentes. Étant donné que pour chaque $n$ elles vérifient $\| x_n-y_n \|<\frac{1}{ n }$, les limites sont égales :
	\begin{equation}
		\lim x_n=\lim y_n=x.
	\end{equation}
	L'ensemble $K$ étant fermé, la limite $x$ est dans $K$. Par continuité de $f$, nous avons finalement
	\begin{equation}
		\lim f(x_n)=\lim f(y_n)=f(x),
	\end{equation}
	mais alors 
	\begin{equation}
		\lim_{n\to\infty}\big| f(x_n)-f(y_n) \big|=0,
	\end{equation}
	ce qui est en contradiction avec le choix \eqref{EqCond3107fxfyepsppt}.

	Tout ceci prouve que $f(K)$ est bornée supérieurement et que $f$ atteint son supremum (qui est donc un maximum). Le fait que $f(K)$ soit borné inférieurement se prouve en considérant la fonction $-f$ au lieu de $f$.

\end{proof}

\begin{remark}
    Nous pouvons ne pas utiliser le fait que le produit de compacts est compact. Cela est particulièrement commode lorsqu'on considère des fonctions de \( \eR\) dans \( \eR\) parce que dans ce cadre nous ne pouvons pas supposer connue la notion de produit d'espace topologiques.

    Pour choisir les sous-suites \( (x_n)\) et \( (y_n)\), il suffit de prendre une sous-suite convergente de \( (x_n)\) et d'invoquer le fait que \( \| x_n-y_n \|\leq \frac{1}{ n }\). Les suites \( (x_n)\) et \( (y_n)\) étant adjacentes, la convergence de \( (x_n)\) implique la convergence de \( (y_n)\) vers la même limite.

    Il est donc un peu superflus de parler de la convergence du couple \( (x_n,y_n)\).
\end{remark}
