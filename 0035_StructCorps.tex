% This is part of Mes notes de mathématique
% Copyright (c) 2011-2013
%   Laurent Claessens
% See the file fdl-1.3.txt for copying conditions.

%+++++++++++++++++++++++++++++++++++++++++++++++++++++++++++++++++++++++++++++++++++++++++++++++++++++++++++++++++++++++++++
\section{Corps finis}
%+++++++++++++++++++++++++++++++++++++++++++++++++++++++++++++++++++++++++++++++++++++++++++++++++++++++++++++++++++++++++++
\label{SecCorpsFinizkAcbS}

%---------------------------------------------------------------------------------------------------------------------------
\subsection{Existence, unicité}
%---------------------------------------------------------------------------------------------------------------------------

Nous avons déjà défini le corps fini \( \eF_p\) lorsque \( p\) est un nombre premier dans la section \ref{subseccorpspremhBlYIv}. Le théorème suivant sert à définir \( \eF_{p^n}\)\nomenclature[A]{\( \eF_{p^n}\)}{corps fini à \( p^n\) éléments} lorsque \( p\) est premier.
\begin{theorem}     \label{ThoOzgSfy}
    Soit \( p\) un nombre premier, soit \( n\in \eN^*\) et \( q=p^n\). Alors il existe un unique corps \( \eK\) de cardinal \( q\). Ce corps est le corps de décomposition du polynôme \( X^q-X\) sur \( \eF_p\).
\end{theorem}

\begin{proof}
    Montrons l'unicité. Soit \( \eK\) un corps fini de cardinal \( q=p^n\). Le groupe multiplicatif \( \eK^*\) est de cardinal \( q-1\), et par le corollaire \ref{CorpZItFX} tous les éléments de \( \eK^*\) vérifient \( g^{q-1}=e\), c'est à dire que dans \( \eK[X]\), les éléments de \( \eK^*\) sont des racines du polynôme
    \begin{equation}
        X^{q-1}-1
    \end{equation}
    Par conséquent \( \eK\) est un corps de décomposition pour le polynôme \( Q(X)=X^q-X=X(X^{q-1}-1)\) parce que \( Q(X)=0\) dans \( \eK\). Il est unique par la proposition \ref{PropTMkfyM}.

    Montrons maintenant que le corps de décomposition de \( P=X^q-X\) sur \( \eF_p\) est un corps de cardinal \( q\). Pour ce faire nous considérons \( \eK\) ce corps de décomposition et \(\eE\), l'ensemble des racines de \( P\) dans \( \eK\). Nous allons montrer que \( \eE=\eK\) et que \( \eE\) est un corps contenant \( q\) éléments.

    Montrons que \( \eE\) est un corps. Pour \( \alpha,\beta\in \eE\) nous avons
    \begin{equation}
        (\alpha\beta)^q=\alpha^q\beta^q=\alpha\beta
    \end{equation}
    parce que \( \alpha^q=\alpha\). Le produit \( \alpha\beta\) est donc encore dans \( \eE\). Pour la somme,
    \begin{equation}
        (\alpha+\beta)^q=(\alpha+\beta)^{p^n}=\Big( (\alpha+\beta)^p \Big)^{p^{n-1}}=(\alpha^p+\beta^p)^{p^{n-1}}=\ldots=\alpha^{p^n}+\beta^{p^n}=\alpha+\beta.
    \end{equation}
    En ce qui concerne l'inverse,
    \begin{equation}
        (\alpha^{-1})^q=(\alpha^q)^{-1}=\alpha^{-1}.
    \end{equation}
    Donc \( \eE\) est un corps. Évidemment \( \eE\) est un corps de décomposition de \( P\) au sens où \( \eE\) est une extension de \( \eF_p\) sur lequel \( P\) est scindé (parce qu'il est scindé sur \( \eK\) et \( \eE\) est le sous corps de \( \eK\) contenant les racines de \( P\)) et tel que \( \eE=\eF_p(\{ \alpha_i \})\) où les \( \alpha_i\) sont les racines de \( P\). Notons que \( \eF_p\subset \eE\) parce que dans \( \eF_p\) on a \( x^q=x\).

    Par unicité, nous avons \( \eK=\eE\). Nous devons montrer que \( P\) possède exactement \( q\) racines distinctes, afin d'avoir \( \Card(\eE)=q\). Pour cela remarquons que 
    \begin{equation}
        P'(X)=qX^{q-1}-1=-1
    \end{equation}
    dans \( \eF_p\). En effet \( P\in\eF_p\) et \( q=0\) dans \( \eF_p\). Par conséquent \( P'\) ne s'annule pas et \( P\) n'a pas de racines doubles. Toutes les racines étant simples, il y en a exactement \( q\).

\end{proof}

Le théorème \ref{ThoOzgSfy} ne permet pas de \emph{construire} le corps à \( q=p^n\) éléments. Nous allons maintenant voir un certain nombre de résultats donnant des façons de construire. Ces résultats proviennent de \cite{MichelMerlecorpsfinis,GabrielPeyre,RodierCorpsFinis} et de \wikipedia{fr}{Théorème_de_l'élément_primitif}{wikipedia} 

\begin{proposition}[\cite{RodierCorpsFinis}]     \label{PropnfebjI}
    Soit \( \eK\) un corps fini. Alors le groupe multiplicatif \( \eK^*\) est cyclique.
\end{proposition}

\begin{proof}
    Soit \( \eK\) un corps ayant \( q\) éléments. Le groupe \( \eK^*\) en a \( q-1\), de telle façon à ce que l'ordre des éléments de \( \eK^*\) soient des diviseurs de \( q-1\); c'est le corollaire \ref{CorpZItFX}. Soit \( d\) un diviseur de \( q-1\) et
    \begin{subequations}
        \begin{align}
            H^*_d&=\{ x\,\text{d'ordre \( d\) dans \( \eK^*\)} \}\\
            H_d&=\{ \text{racines de \( X^d-1\) dans \( \eK\)} \}.
        \end{align}
    \end{subequations}
    Ici le polynôme \( X^d-1\) est vu dans \( \eK[X]\). Notons que nous avons automatiquement \( H^*_d\subset H_d\), mais l'inclusion inverse n'est pas assurée parce que les éléments d'ordre \( d/2\) par exemple sont aussi dans \( H_d\). Supposons \( H^*_d\neq \emptyset\) et considérons \( a\in H^*_d\). Alors l'application
    \begin{equation}
        \begin{aligned}
            \phi\colon \eZ/d\eZ&\to H_d \\
            n&\mapsto a^n 
        \end{aligned}
    \end{equation}
    est un isomorphisme d'anneaux. En effet étant donné que \( a\in H^*_d\subset H_d\), l'ensemble \( H_d\) contient le groupe cyclique engendré par \( a\). Ce dernier contient, par construction, \( d\) éléments. Mais \( \Card(H_d)\leq d\) parce que \( H_d\) est l'ensemble des racines d'un polynôme de degré \( d\). Par conséquent \( \Card(H_d)=d\) et l'ensemble \( H_d\) est bien engendré par \( a\) et \( \phi\) est bien un isomorphisme. Par conséquent tous les éléments de \( H^*_d\) sont des générateurs de \( H_d\).

    Inversement soit \( x\) un générateur de \( H_d\). L'ordre de \( H_d\) étant \( d\), l'ordre de \( x\) doit être un diviseur de \( d\). Supposons donc que \( x\) soit d'ordre \( d/k\). Dans ce cas nous devrions avoir \( \Card(H_d)=d/k\), ce qui contredit l'isomorphisme \( \phi\).

    En conclusion, \( H^*_d\) est l'ensemble des générateurs du groupe \( H_d\). Le nombre de générateurs de \( \eZ/d\eZ\) étant \( \varphi(d)\) par la proposition \ref{PropZnmuphiGensn}, et \( H_d\) étant isomorphe à \( \eZ/d\eZ\) nous avons
    \begin{equation}
        \Card(H^*_d)=\varphi(d).
    \end{equation}
    
    Par conséquent si \( H^*_d\) n'est pas vide, son cardinal est \( \varphi(d)\). Nous avons 
    \begin{subequations}
        \begin{align}
            q-1&=\Card(\eK^*)\\
            &=\Card\big( \bigcup_{d\divides q-1}H^*_d \big)\\
            &=\sum_{d\divides q-1}\Card(H^*_d)\\
            &\leq \sum_{d\divides q-1}\varphi(d)\\
            &=q-1
        \end{align}
    \end{subequations}
    où nous avons utilisé le lemme \ref{LemKcpjee}. Par conséquent pour tout \( d\) divisant \( q-1\) nous avons \( \Card(H^*_d)=\varphi(d)\) et il y a au moins un élément d'ordre \( q-1\) dans \( \eK\). Cet élément engendre \( \eK^*\) parce que \( \eK^*\) contient exactement \( q-1\) éléments. Par conséquent \( \eK\) est cyclique.
\end{proof}

\begin{corollary}   \label{CorpRUndR}
    Si \( p\) est un nombre premier, alors 
    \begin{equation}
        (\eZ/p\eZ)^*\simeq\eZ/(p-1)\eZ.
    \end{equation}
    L'isomorphisme est un isomorphisme de groupe (abéliens). À gauche multiplicatif et à droite additif.
\end{corollary}
\index{groupe!fini}
\index{corps!fini}
\index{anneau!\( \eZ/n\eZ\)}
\index{nombre!premier}
\index{isomorphisme!\( (\eZ/p\eZ)^*\simeq\eZ/(p-1)\eZ\)}

\begin{proof}
    La proposition \ref{PropnfebjI} nous enseigne que le le groupe multiplicatif d'un corps fini est cyclique et donc isomorphe à un certain \( \eZ/n\eZ\). Donc \( (\eZ/p\eZ)^*\) est un groupe cyclique d'ordre \( p\), et donc isomorphe à \( \eZ/n\eZ\) avec \( n=p-1\).
\end{proof}

Lorsque \( \eK\) est un corps les éléments du groupe \( \eK^*\) sont les \defe{éléments primitifs}{primitif!élément d'un corps} de \( \eK\).
\begin{proposition}     \label{propQRcUlq}
    Soit \( \eK\) un corps contenant \( q\) éléments. Alors
    \begin{enumerate}
        \item
            \( x^q=x\) pour tout \( x\in \eK\),
        \item
            \( X^q-X=\prod_{a\in \eK}(X-a)\).
    \end{enumerate}
\end{proposition}

\begin{proof}
    Le groupe \( \eK^*\) ayant \( q-1\) éléments, ses éléments vérifient \( a^{q-1}=1\) par le corollaire \ref{CorpZItFX} et par conséquent \( a^q=aa^{q-1}=a \).

    Soit \( a\in \eK\). Étant donné que \( a^q-a=0\), le polynôme \( (X-a)\) divise \( X^q-X\) dans \( \eK[X]\). Par conséquent 
    \begin{equation}
        \prod_{a\in \eK}(X-a)
    \end{equation}
    divise également \( X^q-X\). Les polynômes \( X^q-X\) et \( \prod_{a\in \eK}(X-a)\) étant deux polynômes unitaires de même degré, le fait que l'un divise l'autre montre qu'ils sont égaux.
\end{proof}

\begin{example}
    Soit \( \eK=\eQ\) et \( \eL=\eQ(\sqrt{2},\sqrt{3})\). Afin de montrer que \( \eL=\eQ(\alpha)\) avec \( \alpha=\sqrt{2}+\sqrt{3}\) nous devons montrer que \( \sqrt{2}\) et \( \sqrt{3}\) sont des polynômes en \( \alpha\).
\end{example}

Une conséquence du fait que \( x^q=x\) est qu'il ne fait pas regarder le théorème \ref{ThoLXTooNaUAKR} trop rapidement en disant «si il s'annule partout, alors c'est le polynôme nul». En effet dans un corps fini, «partout» n'est pas forcément très grand.

% Le saut de ligne au milieu de la phrase est utile pour séparer deux références.
\begin{example}\label{exVQBooBMPLkD}
    Si \( \eF_3=\eZ/3\eZ\) est le\footnote{Le singulier est justifié par le théorème \ref{ThoOzgSfy}, mais ça n'a pas d'importance ici.} corps à \( 3\)
    éléments, alors le polynôme \( P(X)=X^3-X\) s'évalue à zéro pour tout \( x\in \eF_3\) (proposition \ref{propQRcUlq}.) mais il n'est pas le polynôme nul.
\end{example}

%---------------------------------------------------------------------------------------------------------------------------
\subsection{Symboles de Legendre et carrés}
%---------------------------------------------------------------------------------------------------------------------------

Source : \cite{RecQuadVento}.

Nous disons que \( a\in \eF_p\) est un \defe{carré}{carré!dans un corps fini} si il existe \( b\in \eF_p\) tel que \( a=b^2\).

\begin{definition}
    Soit \( n\in \eN\) et \( p>2\) un nombre premier. Le \defe{symbole de Legendre}{symbole!de Legendre}\index{Legendre!symbole} par
    \begin{equation}
        \left( \frac{ n }{ p } \right)=\begin{cases}
            0    &   \text{si \( p\) divise \( n\)}\\
            1    &    \text{si \( n\) est un carré dans \( \eF_p\)}\\
            -1    &    \text{sinon}.
        \end{cases}
    \end{equation}
\end{definition}

Note que \( -1\) peut être un carré, et pas que dans \( \eC\). Par exemple dans \( \eF_5\) nous avons \( 4=-1\) et donc \( -1\) est un carré.

\begin{proposition} \label{PropcGsJjk}
    Soit un nombre premier \( p>2\). Le corps \( \eF_p^*\) contient autant de carrés que de non carrés. De plus pour tout \( n\in \eN\) nous avons
    \begin{equation}    \label{Eqbcugos}
        \left(\frac{n}{p}\right)=n^{(p-1)/2}\mod p.
    \end{equation}
\end{proposition}

\begin{proof}
    Nous considérons l'application 
    \begin{equation}
        \begin{aligned}
            \psi\colon \eF^*_p&\to \eF^*_p \\
            x&\mapsto x^2. 
        \end{aligned}
    \end{equation}
    C'est un morphisme de groupes multiplicatifs et \( \ker\psi=\{ -1,1 \}\). Étant donné que \( p>2\), nous avons alors
    \begin{equation}
        \Card(\ker\psi)=2
    \end{equation}
    parce que \( 1\neq -1\). Évidemment l'ensemble des carrés dans \( \eF^*_p\) est l'image de \( \psi\). Le premier théorème d'isomorphisme \ref{ThoPremierthoisomo}\ref{ItemWLCLdk} nous permet alors de conclure que
    \begin{equation}
        \Card(\Image(\psi))=\frac{ \Card(\eF^*_p) }{2}.
    \end{equation}
    Ceci prouve la première assertion.

    Par le petit théorème de Fermat (théorème \ref{ThoOPQOiO}), nous avons \( x^{p-1}=1\) pour tout \( x\in \eF^*_p\). Les \( (p-1)\) éléments de \( \eF^*_p\) sont donc tous racines d'un des deux polynômes
    \begin{equation}
        X^{(p-1)/2}=\pm 1.
    \end{equation}
    Mais chacun des deux ne peut avoir, au maximum, que \( (p-1)/2\) solutions. Ils ont donc chacun exactement \( (p-1)/2\) racines.

    Nous pouvons maintenant prouver la formule \eqref{Eqbcugos}. D'abord si \( n=0\), elle est évidente. Si \( n\) est un carré dans \( \eF_p\), nous posons \( n=x^2\) et nous avons
    \begin{equation}
        n^{(p-1)/2}=n^{p-1}=1=\left(\frac{n}{p}\right).
    \end{equation}
    Si \( n\) n'est pas un carré, c'est que \( n\) n'est pas une racine de \( X^{(p-1)/2}=1\). Le nombre \( n\) est alors une racine de \( X^{(p-1)/2}=-1\). Nous avons alors
    \begin{equation}
        n^{(p-1)/2}=-1=\left(\frac{n}{p}\right).
    \end{equation}
\end{proof}

\begin{corollary}   \label{CoruJosNz}
    Si \( a,b\in \eN\) et si \( p>2\) est un nombre premier, alors
    \begin{equation}
        \left(\frac{ab}{p}\right)=\left(\frac{a}{p}\right)\left(\frac{b}{p}\right).
    \end{equation}
\end{corollary}

\begin{proof}
    Par la formule \eqref{Eqbcugos},
    \begin{equation}
        \left(\frac{ab}{p}\right)=(ab)^{(p-1)/2}=a^{(p-1)/2}b^{(p-1)/2}=\left(\frac{a}{p}\right)\left(\frac{b}{p}\right).
    \end{equation}
\end{proof}

Soit un nombre premier \( q>2\) et \( \eA\), un anneau de caractéristique \( p\). Si \( \alpha\in \eA\) vérifie
\begin{equation}
    1+\alpha+\ldots+\alpha^{q-1}=0,
\end{equation}
nous définissons la \defe{somme de Gauss}{Gauss!somme de} par
\begin{equation}
    \tau=\sum_{x\in \eF_q}\left(\frac{i}{q}\right)\alpha^i=\sum_{x=1}^{q-1}\left(\frac{x}{q}\right)\alpha^i.
\end{equation}
Notons que la somme de Gauss dépend de \( q\) et du \( \alpha\) choisis.

\begin{proposition} \label{PropciRUov}
    Les sommes de gauss vérifient les propriétés suivantes.
    \begin{enumerate}
        \item
            $\tau^2=\left(\frac{-1}{q}\right)q$. Nous allons noter \( \epsilon(q)=\left(\frac{-1}{q}\right)\).

        \item
            Si \( \eA\) est de caractéristique \( p\geq 3\) et si \( p\neq q\) alors
            \begin{equation}    \label{EqxBNpJz}
                \tau^p=\left(\frac{p}{q}\right)\tau.
            \end{equation}
        \item
            Si \( \eA\) est de caractéristique \( p\) et si \( q\) est premier avec \( p\), alors \( \tau\) est inversible dans \( \eA\).
    \end{enumerate}
\end{proposition}

\begin{proof}
    D'abord nous notons que
    \begin{equation}
        \alpha^q-1=(\alpha-1)(1+\alpha+\ldots+\alpha^{q-1})=0
    \end{equation}
    par définition de \( \alpha\). Nous calculons
    \begin{subequations}
        \begin{align}
            \epsilon(q)\tau^2&=\epsilon(q)\sum_{x,y\in \eF_q}\left(\frac{x}{q}\right)\left(\frac{y}{q}\right)\alpha^{x+y}\\
            &=\sum_{x,y\in \eF_q}\left(\frac{-xy}{q}\right)\alpha^{x+y}. \label{EqlObFeo}\\
            &=\sum_{z\in \eF_q}\sum_{y\in \eF_q}\left(\frac{-(z-y)y}{q}\right)\alpha^{z}    \label{EqWyIhhk}\\
            &=\sum_{z\in \eF_q}s_z\alpha^z  \label{EqWoIszS}
        \end{align}
    \end{subequations}
    Justifications :
    \begin{itemize}
        \item 
            Pour obtenir \eqref{EqlObFeo} nous avons utilisé le corollaire \ref{CoruJosNz}. 
        \item
            \eqref{EqWyIhhk} est un changement de variable \( z=x+y\) dans la somme sur \( x\).
        \item
            Pour \eqref{EqWoIszS} nous avons posé
            \begin{equation}
                s_z=\sum_{y\in \eF_q}\left(\frac{-(z-y)y}{q}\right).
            \end{equation}
            
    \end{itemize}
    Nous avons 
    \begin{equation}
        s_0=\sum_{y\in \eF_q}\left(\frac{y^2}{q}\right).
    \end{equation}
    Dans cette somme, tous les termes sont \( 1\) sauf celui avec \( y=0\) qui vaut zéro. Nous avons donc \( s_0=q-1\). Voyons maintenant \( s_y\) avec \( y\neq 0\). L'application
    \begin{equation}
        \begin{aligned}
            \eF^*_q&\to \eF_q\setminus\{ 1 \} \\
            k&\mapsto 1-zy^{-1} 
        \end{aligned}
    \end{equation}
    étant une bijection nous pouvons effectuer le changement de variables \( t=y^{-1}z-1\) pour la somme sur \( y\) en notant \( y^{-1}\) l'inverse de \( y\) dans \( \eF^*_q\), nous trouvons alors
    \begin{subequations}
        \begin{align}
            \sum_{y\in \eF_q}\left(\frac{y(z-y)}{q}\right)&=\sum_{y\in \eF_q}\left(\frac{y^2(y^{-1}z-1)}{q}\right)\\
            &=\sum_{y\in \eF_q}\left(\frac{y^{-1}z-1}{q}\right)\\
            &=\sum_{t\in \eF_q\setminus\{ 1 \}}\left(\frac{t}{q}\right)\\
            &=\underbrace{\sum_{t\in \eF_q}\left(\frac{y}{q}\right)}_{=0}-\left(\frac{1}{1}\right)\\
            &=-1
        \end{align}
    \end{subequations}
    parce qu'il  y a autant de carrés que de non carrés dans \( \eF_q^*\) (proposition \ref{PropcGsJjk}). En résumé nous avons
    \begin{equation}
        \epsilon(q)\tau^2=\sum_{z\in \eF_q}s_z\alpha^z
    \end{equation}
    où
    \begin{equation}
        s_z=\begin{cases}
            q-1    &   \text{si \( z=0\)}\\
            -1    &    \text{sinon}.
        \end{cases}
    \end{equation}
    Cela donne
    \begin{equation}
        \epsilon(q)\tau^2=(q-1)-\underbrace{(\alpha+\ldots +\alpha^{q-1})}_{=-1}=q
    \end{equation}
    où nous avons utilisé l'hypothèse sur \( \alpha\). Donc \( \epsilon(q)\tau^2=q\), et étant donné que \( \epsilon(q)=\pm 1\) nous concluons
    \begin{equation}
        \tau^2=\epsilon(q)q.
    \end{equation}
    
    Nous prouvons maintenant la seconde partie. Vu que \( \eA\) est de caractéristique \( p\) en utilisant le fait que le morphisme de Frobenius est un morphisme,
    \begin{equation}
        \tau^p=\left( \sum_{x\in \eF_q}\left(\frac{x}{q}\right)\alpha^x \right)^p=\sum_{x\in \eF_q}\left(\frac{x}{q}\right)^p\alpha^{px}.
    \end{equation}
    Étant donné que \( \left(\frac{x}{q}\right)=\pm 1\) et que \( p\) est impair, nous avons
    \begin{equation}
        \left(\frac{x}{q}\right)^p=\left(\frac{x}{q}\right).
    \end{equation}
    Du coup nous avons
    \begin{equation}
        \left(\frac{p}{q}\right)\tau^p=\sum_{x\in \eF_p}\left(\frac{xp}{q}\right)\alpha^{px}.
    \end{equation}
    Mais \( p\) étant inversible dans \( \eF_q\), l'application \( x\mapsto px\) est une bijection et nous pouvons sommer sur \( px\) au lieu de \( x\) :
    \begin{equation}
        \left(\frac{p}{q}\right)\tau^p=\sum_{x\in \eF_p}\left(\frac{x}{q}\right)\alpha^x=\tau.
    \end{equation}
    Nous trouvons alors que
    \begin{equation}
        \tau^p=\left(\frac{p}{q}\right)\tau.
    \end{equation}

    Étant donné la formule du \( \tau^2\) que nous venons de démontrer, nous avons \( \tau^2=\pm q\). Les nombres \( p\) et \( q\) étant premiers entre eux, la relation de Bézout (théorème \ref{ThoBuNjam}) nous donne \( a\) et \( b\) tels que
    \begin{equation}
        ap+ba=1.
    \end{equation}
    Cela montre que \( b\) est un inverse de \( q\) modulo \( p\). Donc \( \tau^2\) est inversible, et il en découle que \( \tau\) lui-même est inversible.
\end{proof}

\begin{theorem}[Loi de réciprocité quadratique]\index{loi!réciprocité quadratique}  \label{ThoMiEiUm}
    Soient deux nombres premiers distincts \( p,q\geq 3\). Alors
    \begin{equation}
        \left(\frac{p}{q}\right)=(-1)^{\frac{ (p-1)(q-1) }{ 4 }}\left(\frac{q}{p}\right).
    \end{equation}
\end{theorem}
\index{corps!fini}

\begin{proof}
    Soit \( \phi_q\) le polynôme \( A+X+\ldots+X^{q-1}\) et l'anneau
    \begin{equation}
        \eA=\eF_p[X]/(\phi_q).
    \end{equation}
    Cela est un anneau de caractéristique \( p\) parce que son unité est le polynôme constant \( 1\). Nous nommons \( \alpha=X/(\phi_q)\), c'est à dire que \( \phi_q(\alpha)=0\) dans \( \eA\), et nous pouvons considérer la somme de Gauss
    \begin{equation}
        \tau=\sum_{i\in \eF_q}\left(\frac{i}{q}\right)\alpha^i.
    \end{equation}
    Notons que cela est un élément de \( \eA\) et plus précisément une (classe de) polynôme de degré zéro dans \( \eA\). Donc les coefficients de \( \alpha\) doivent être compris comme des éléments de \( \eF_p\).
    Nous savons (proposition \ref{PropciRUov}) que 
    \begin{equation}
        \tau^2=\left(\frac{-1}{q}\right),
    \end{equation}
    et en utilisant la formule \eqref{Eqbcugos} nous trouvons
    \begin{equation}
        \left(\frac{\tau^2}{p}\right)=(\tau^2)^{(p-1)/2}\mod p=\tau^{p-1}\mod p
    \end{equation}
    En réalité sur cette dernière ligne, nous ne devrions pas préciser le «\( \mod p\)» parce que, comme mentionné plus haut, ce sont des éléments de \( \eF_p\). En utilisant cela ainsi que \eqref{EqxBNpJz} nous avons
    \begin{equation}
        \underbrace{\left(\frac{\tau^2}{p}\right)}_{\tau^{p-1}}\tau=\tau^p=\left(\frac{p}{q}\right)\tau
    \end{equation}
    Vu que \( \tau\) est inversible, nous écrivons
    \begin{equation}
        \left(\frac{\tau^2}{p}\right)=\left(\frac{p}{q}\right)
    \end{equation}

    Nous utilisons maintenant la formule \eqref{Eqbcugos} sur le membre de gauche avec \( n=\tau^2=\left(\frac{-1}{q}\right)\) :
    \begin{equation}
        \left(\frac{p}{q}\right)=\left(\frac{-1}{q}\right)^{\frac{ q-1 }{2}}\left(\frac{q}{p}\right).
    \end{equation}
    Toujours avec la même formule nous pouvons substituer \( \left(\frac{-1}{q}\right)\) par \( (-1)^{(q-1)/2}\) et obtenir
    \begin{equation}
        \left(\frac{p}{q}\right)=(-1)^{\frac{ (q-1) }{2}\frac{ (p-1) }{2}}.
    \end{equation}

\end{proof}


\begin{lemma}\label{Lemoabzrn}
    Si \( p\) est un nombre premier \( p\geq 3\), alors le symbole de Legendre \( x\mapsto\left(\frac{x}{p}\right)\) est l'unique morphisme non trivial de \( \eF^*_p\) dans \( \{ -1,1 \}\).
\end{lemma}

\begin{proof}
    Le fait que le symbole de Legendre soit non trivial est simplement le fait qu'il y ait des carrés et des non carrés dans \( \eF_p^*\); voir la proposition \ref{PropcGsJjk}. Pour l'unicité, soit \( \alpha\colon \eF^*_p\to \{ -1,1 \}\) un morphisme surjectif (c'est à dire non trivial). Étant donné que 
    \begin{equation}
        \eF^*_p=\ker(\alpha)\cup-\ker(\alpha),
    \end{equation}
    le groupe \( \eF_p^*/\ker(\alpha)\) ne contient que deux éléments : \( [1]\) et \( [-1]\). Autrement dit, \( \ker(\alpha)\) est d'indice \( 2\) dans \( \eF_p^*\). 
    
    Or \( \eF_p^*\) ne possède qu'un seul sous-groupe d'indice \( 2\). En effet soit \( S\) un tel sous-groupe et \( a\), un générateur de \( \eF_p^*\) (qui est cyclique par la proposition \ref{PropnfebjI}), alors \( a^2\in S\) par le lemme \ref{PropubeiGX}. Par conséquent \( S\) contient le groupe des puissances paires de \( a\). Le groupe $S$ ne peut rien contenir de plus parce qu'il est d'indice \( 2\) et que l'ordre de \( \eF_p^*\) est pair.

    Bref, le sous-groupe \( \ker(\alpha)\) est l'unique sous-groupe d'indice \( 2\) dans \( \eF_p^*\). Mais la proposition \ref{PropcGsJjk} nous indique que \( | (\eF_p^*)^2 |=\frac{ p-1 }{2}\), c'est à dire que le groupe des carrés est d'indice \( 2\). Nous avons donc, par l'unicité,
    \begin{equation}
        \ker(\alpha)=(\eF_p^*)^2.
    \end{equation}
    Au final, pour \( y\in \eF_p^*\),
    \begin{equation}
        \alpha(y)=\begin{cases}
           1 &   \text{si \( y\) est un carré}\\
            -1    &    \text{sinon.}
        \end{cases}
    \end{equation}
    Cela est bien la définition des symboles de Legendre.
\end{proof}
 
\begin{proposition}
    Pour \( p\) premier nous avons
    \begin{equation}
        \left(\frac{2}{p}\right)=\begin{cases}
            1    &   \text{si \( p\in [1]_8\) ou \( p\in [7]_8\)}\\
            -1    &    \text{sinon}.
        \end{cases}
    \end{equation}
\end{proposition}

\begin{proof}
    Soit le polynôme 
    \begin{equation}
        X^4+1\in \eF_p[X]
    \end{equation}
    et \( \alpha\), une racine dans une fermeture de \( \eF_p\). Nous posons \( \theta=\alpha+\alpha^{-1}\) et nous calculons
    \begin{equation}
        \theta^2=(\alpha+\alpha^{-1})(\alpha+\alpha^{-1})=\alpha^2+2+(\alpha^2)^{-1}=\alpha^2+2-\alpha^2=2
    \end{equation}
    parce que \( \alpha^2\) étant \( -1\), nous avons \( (\alpha^2)^{-1}=-\alpha^2\). Bref, \( \theta^2=2\).

    Dire que \( 2\) est un carré modulo \( p\) revient à dire que \( \theta\) est dans \( \eF_p\). C'est à dire que pour calculer le symbole de Legendre \( \left(\frac{2}{p}\right)\), nous étudions pour quels \( p\), l'élément \( \theta\) est vraiment dans \( \eF_p\) et non seulement dans l'extension \( \eF_p(\alpha)\). En tenant compte de l'exemple \ref{ExLQhLhJ}, il faut distinguer deux cas : \( \alpha^p=\alpha\) et \( \alpha^p\neq \alpha\). Autrement dit, si \( \alpha^k=\alpha\) pour un certain nombre premier \( k\), alors le cas \( p=k\) est à traiter à part. La liste des puissances de \( \alpha\) est :
    \begin{equation}
        1,\alpha,\alpha^2,\alpha^3,-1,-\alpha,-\alpha^2,-\alpha^3,1,\alpha,\ldots
    \end{equation}
    Nous avons donc automatiquement \( \alpha^{9k}=\alpha\), mais \( p=9k\) est exclu parce que \( p\) est premier. Nous devons donc vérifier si une des possibilités
    \begin{subequations}
        \begin{align}
            \alpha^2=\alpha\\
            \alpha^3=\alpha\\
            -\alpha=\alpha\\
            -\alpha^3=\alpha
        \end{align}
    \end{subequations}
    est possible. Il est aisément vérifiable, au cas par cas, que ces possibilités sont toutes incompatibles avec \( \alpha^4=-1\). Nous avons donc certainement \( \alpha^p\neq \alpha\) et compte tenu de l'exemple \ref{ExLQhLhJ}, l'équation \( x^p=x\) caractérise les éléments de \( \eF_p\) dans \( \eF_p(\alpha)\).

    L'équation \( X^2=2\) a exactement deux solutions qui sont \( \pm\theta\). Nous avons donc \( 2\in \eF_p^2\) si et seulement si \( \theta\in \eF_p\) si et seulement si \( \theta^p=\theta\). Nous avons réduit notre problème à déterminer pour quels \( p\) nous avons \( \theta^p=\theta\). D'abord nous avons, par le morphisme de Frobenius,
    \begin{equation}
        \theta^p=(\alpha+\alpha^{-1})^p=\alpha^p+\alpha^{-p}.
    \end{equation}
    Nous pouvons maintenant conclure facilement. Un nombre premier étant impair (sauf \( p=2\) qui peut être traité à part), \( p\) est automatiquement dans un des ensembles \( [1]_8\), \( [3]_8\), \( [5]_8\) ou \( [7]_8\). Nous avons quatre petites vérifications à faire. Dans tous les cas \( \alpha^{8k}=1\). Si \( p=1+8k\), alors
    \begin{equation}
        \theta^p=\alpha^{1+8k}+(\alpha^{-1})^{1+8k}=\alpha+\alpha^{-1}=\theta,
    \end{equation}
    donc \( 2\) est un carré dans \( \eF_p\). Si \( p\in[3]_8\), alors \( \theta^p=\alpha^3+\alpha^{-3}\). Si cela était égal à \( \alpha+\alpha^{-1}\), alors nous aurions
    \begin{equation}
        \alpha^6+1=\alpha^4+\alpha^2,
    \end{equation}
    et donc \( \alpha^2=1\), ce qui est impossible. Les vérifications pour \( p\in [5]_8\) et \( p\in [7]_8\) sont du même style.

\end{proof}

%---------------------------------------------------------------------------------------------------------------------------
\subsection{Théorème de Chevalley-Warning}
%---------------------------------------------------------------------------------------------------------------------------

\begin{lemma}
    Soit \( \eK\) un corps de caractéristique \( p\) et de cardinal \( q\). Pour \( m\in \eN\) nous définissons
    \begin{equation}
        S_m=\sum_{x\in \eK}x^m.
    \end{equation}
    Alors nous avons
    \begin{equation}
        S_m\mod p=\begin{cases}
            -1     &   \text{si \( m\geq 1\) et \( m\) divisible par \( q-1\)}\\
            0    &    \text{sinon}.
        \end{cases}
    \end{equation}
\end{lemma}

\begin{proof}
    Si \( m=0\), alors \( x^0=1\) et \( S_m=q\). Par conséquent \( S_m\mod p=0\) parce que la caractéristique d'un corps divise son ordre (proposition \ref{PropGExaUK}). 

    Nous prenons maintenant \( m\geq 1\) et nous voyons séparément les cas où \( q-1\) divise \( m\) ou non. Si \( q-1\) divise \( m\), alors pour tout \( x\neq 0\) nous avons
    \begin{equation}
        x^m=x^{k(q-1)}=1
    \end{equation}
    parce que \( \eK^*\) est cyclique et \( x^{q-1}=1\) par le petit théorème de Fermat (théorème \ref{ThoOPQOiO}). Par conséquent nous avons
    \begin{equation}
        \sum_{x\in \eK}x^m=\sum_{x\in \eK^*}1=q-1.
    \end{equation}
    
    Si le nombre \( m\geq 1\) n'est pas divisible par \( q-1\) alors nous prenons un générateur \( y\) du groupe \( \eK^*\). Un tel élément vérifie \( y^m\neq 1\). En effet, si \( y\) vérifiait \( y^m=1\) alors cela signifierait que l'ordre de \( \eK^*\) est un diviseur de \( m\), ce qui n'est pas le cas ici parce que l'ordre de \( \eK^*\) est \( q-1\). Pour un tel \( y\), l'application
    \begin{equation}
        \begin{aligned}
            \varphi\colon \eK^*&\to \eK^* \\
            x&\mapsto yx 
        \end{aligned}
    \end{equation}
    est une bijection\footnote{Notons que nous n'avons pas réellement besoin que \( y\) soit un générateur. Nous n'utilisons seulement le fait que \( y^m\neq 1\) et \( y\neq 0\).}. En ce qui concerne l'injectivité, \( ya=yb\) implique \( a=b\). En ce qui concerne la surjectivité, si \( a\) est un générateur, si \( z=a^l\) et si \( y=a^k\), alors
    \begin{equation}
        z=\varphi(a^{l-k}).
    \end{equation}
    Nous pouvons maintenant faire le calcul.
    \begin{equation}
        S_m=\sum_{x\in \eK^*}x^n=\sum_{x\in \eK^*}(yx)^m=y^m\sum_{x\in \eK^*}x^m=y^mS_m.
    \end{equation}
    Étant donné que \( y^m\neq 1\), la seule solution est \( S_m=0\).
\end{proof}

\begin{theorem}[\wikipedia{en}{Chevalley–Warning_theorem}{Chevalley-Warning}]\index{théorème!Chevalley-Warning}        \label{ThoLTcYKk}
    Soit \( \eK\) un corps fini de cardinal \( q\) et de caractéristique \( p\). Soient \( P_1,\ldots, P_r\) des éléments de \( \eK[X_1,\ldots, X_n]\) tels que \( \sum_{i=1}^r\deg(P_i)<n\). Nous considérons l'ensemble des zéros communs à tous les polynômes :
    \begin{equation}
        V=\{ x\in \eK^n\tq P_1(x)=\ldots=P_r(x)=0 \}.
    \end{equation}
    Alors \( \Card(V)=0\mod p\).
\end{theorem}
\index{corps!fini}
\index{polynôme!à plusieurs indéterminées}
\index{polynôme!symétrique}

\begin{proof}
    Nous considérons le polynôme
    \begin{equation}
        P=\prod_{i=1}^r(1-P_i^{q-1}).
    \end{equation}
    Montrons que
    \begin{equation}
        P(x)=\begin{cases}
            1    &   \text{si \( x\in V\)}\\
            0    &    \text{sinon}.
        \end{cases}
    \end{equation}
    La première ligne est facile : étant donné que tous les \( P_i(x)\) sont nuls pour \( x\in V\), nous avons \( P(x)=1\). Si \( x\) n'est pas dans \( V\), alors nous avons un \( i\) tel que \( P_i(x)\in \eK^*\). Mais dans ce cas (toujours la cyclicité de \( \eK^*\)) nous avons \( P_i(x)^{q-1}=1\) et donc le produit est nul.

    En utilisant l'hypothèse sur le degré des \( P_i\), nous trouvons que
    \begin{equation}
        \deg(P)=\sum_{i=1}^r(q-1)\deg(P_i)<n(q-1).
    \end{equation}

    Pour un polynôme \( Q\in \eK[X_1,\ldots, X_n]\), nous définissons 
    \begin{equation}
        \int Q=\sum_{x\in \eK^n}Q(x).
    \end{equation}
    Nous avons immédiatement
    \begin{equation}
        \int P=\sum_{x\in \eK^n}P(x)=\sum_{x\in V}1=\Card(V)\mod p.
    \end{equation}
    Nous insistons sur le «modulo \( p\)» parce que dans la formule \( P(x)=1\), le membre de droite est le \( 1\) de \( \eK\); il est donc automatiquement modulo la caractéristique de \( \eK\).

    Il nous reste à prouver que \( \int P=0\). Pour cela nous décomposons 
    \begin{equation}        \label{EqHnUVlM}
        P=\sum_m c_mX_1^{m_1}\ldots X_n^{m_n}
    \end{equation}
    où la somme s'étend sur les \( m\in \eN^n\) tels que \( c_m\neq 0\). Nous avons
    \begin{subequations}
        \begin{align}
            \int P&=\sum_{\in \eK^n}\sum_mc_m x_1^{m_1}\ldots x_n^{m_n}\\
            &=\sum_{m}c_m\left( \sum_{x\in \eK^n}x_1^{m_1}\ldots x_n^{m_n} \right)\\
            &=\sum_m c_m S_{m_1}\ldots S_{m_n}.
        \end{align}
    \end{subequations}
    Le terme de plus haut degré dans la décomposition \eqref{EqHnUVlM} est celui du \( m\) tel que \( \sum_im_i\) est le plus grand, mais étant donné que nous savons que ce degré est plus petit que \( n(q-1)\), nous avons pour tous les \( m\) entrant dans la somme que
    \begin{equation}
        \sum_{i=1}^nm_i<n(q-1).
    \end{equation}
    En particulier pour tout \( m\in \eN^n\), il existe \( i\) tel que \( m_i<q-1\), et dans ce cas \( S_{m_i}=0\). Donc tous les termes de la somme
    \begin{equation}
        \sum_{m\in \eN^n}c_mS_{m_1}\ldots S_{m_n}
    \end{equation}
    ont un facteur nul.
\end{proof}

\begin{corollary}       \label{CorfuHNKz}
    Soit \( P_i\) des polynômes à \( n\) variables avec \( \sum_{i=1}^r\deg(P_i)<n\). Si les \( P_i\) n'ont pas de termes constants, alors ils ont un zéro commun non trivial.
\end{corollary}

\begin{proof}
    Nous reprenons les notations du théorème \ref{ThoLTcYKk}. Étant donné que les \( P_i\) n'ont pas de termes constants, \( 0\in V\), mais \( \Card(V)=0\mod p\). Par conséquent nous devons avoir \( \Card(V)>p\).
\end{proof}

\begin{example}
    Nous considérons les polynômes
    \begin{subequations}
        \begin{align}
            P_1(x,y,t,u)=xy+x+ux\\
            P_2(x,y,t,u)=x+y-3t.
        \end{align}
    \end{subequations}
    La somme de leurs degrés est \( 3\) et ce sont des polynômes à \( 4\) variables. Nous devons donc avoir, en vertu du corollaire \ref{CorfuHNKz}, des autres racines que la racine triviale \( (x,y,t,u)=(0,0,0,0)\).

    Le corollaire nous donne aussi une borne inférieure nombre de racines à chercher : plus que la caractéristique du corps sur lequel nous travaillons. Nous pouvons dire cela sans avoir la moindre idée de la façon dont on pourrait résoudre le système \( P_1=P_2=0\).
\end{example}

%---------------------------------------------------------------------------------------------------------------------------
\subsection{Théorème de l'élément primitif}
%---------------------------------------------------------------------------------------------------------------------------

% TODO : fusionner cette définition avec celle du degré.
\begin{definition}
    Soit \( \eK\) un corps. Une extension \( \eL\) de \( \eK\) est dite \defe{finie}{extension!de corps!finie} si \( \eL\) est un espace vectoriel de dimension finie sur \( \eK\).
\end{definition}
Notez que la définition d'extension finie ne suppose ni que \( \eK\) ni que \( \eL\) soient finis en tant qu'ensembles.

\begin{theorem}[de l'élément primitif]\index{théorème!élément primitif}
    Si \( \eK\) est un corps fini, toute extension finie de \( \eK\) est simple\footnote{Définition \ref{DefZCYIbve}.}.

    Si \( \eK\) est un corps quelconque alors toute extension séparable finie est simple.
\end{theorem}

\begin{proof}
    Nous ne donnons la preuve que dans le cas où \( \eK\) est fini. Dans ce cas nous savons par la proposition \ref{PropnfebjI} que le groupe \( \eK^*\) est cyclique. Si de plus \( \eL\) est une extension finie alors \( \eL\) est fini en tant qu'ensemble. Par conséquent \( \eL^*\) est un groupe cyclique. Si \( \alpha\) est un générateur de \( \eL\) alors \( \eL=\eK(\alpha)\) et l'extension est donc simple.

    Une preuve de l'assertion dans le cas où \( \eK\) est infini peut être trouvée sur wikipédia.
\end{proof}

\begin{definition}\label{DefvBFpsY}
    Soit \( P\) un polynôme irréductible de degré \( n\) sur \( \eF_p[X]\). L'\defe{ordre}{ordre!d'un polynôme} de \( P\) est
    \begin{equation}
        \min\{ k\tq P\divides X^k-1 \}.
    \end{equation}

    Soit \( p\), un nombre premier et \( P\) un polynôme unitaire irréductible de degré $n$ dans \( \eF_p[X]\). Nous disons que \( P\) est \defe{primitif}{primitif!polynôme}\index{polynôme!primitif} si les racines de \( P\) sont d'ordre \( p^n-1\) dans \( \eF_p[X]/P\).
\end{definition}

\begin{remark}  \label{RemwwJbYP}
    Si \( \eA\) est un anneau factoriel, \wikipedia{fr}{Polynôme}{il est souvent dit} qu'un polynôme \( P\in \eA[X]\) est primitif si le pgcd de ses coefficients est \( 1\). Cette notion de primitivité n'est apparemment pas reliée à celle de la définition \ref{DefvBFpsY} qui sera celle que nous retiendrons pour la suite. Notons au passage que dans un corps, tous les polynômes non nuls et unitaires sont primitifs au sens du pgcd des coefficients; cette notions de primitivité n'a donc pas d'intérêt dans notre cadre.

    Lorsque nous utiliserons la notion de polynôme primitif au sens du \( \pgcd\), nous le mentionnerons explicitement. C'est pas exemple le cas pour le théorème \ref{ThofiIpXg}.
\end{remark}

\begin{proposition}
    L'ordre d'un polynôme \( P\) vérifie les propriétés suivantes :
    \begin{enumerate}
        \item
            L'ordre de \( P\) est l'ordre multiplicatif de ses racines
        \item
            L'ordre de \( P\) divise \( p^n-1\).
    \end{enumerate}
\end{proposition}

\begin{lemma}       \label{LemZrUUOz}
    Soit \( p\) un nombre premier et \( P\) un polynôme irréductible unitaire de degré \( n\). Si \( \alpha,\beta\in \eF_p[X]/P\), alors \( (\alpha+\beta)^p=\alpha^p+\beta^p\).
\end{lemma}

\begin{proof}
    La preuve est exactement la preuve classique :
    \begin{equation}
        (\alpha+\beta)^p=\sum_k{k\choose p} \alpha^k\beta^{p-k}
    \end{equation}
    où les coefficients binomiaux sont dans \( \eF_p\) et donc nuls pour les \( k\) différents de \( p\) et de \( 0\).
\end{proof}
Cette proposition est encore vraie avec \( \alpha,\beta\in\eF_{p^n}\) et \( (\alpha+\beta)^{p^n}\).
%TODO : dire plus précisément, et prouver.

\begin{lemma}
    Si \( \alpha\in \eF_q\) est une racine d'ordre \( k\) de \( P\) (de degré \( n\)) alors les racines de \( X^k-1\) sont \( \{ \alpha^i\tq i=0,\ldots, k-1 \}\).
\end{lemma}

Nous serions donc intéressé à construire $\eF_{q}$ comme quotient de \( \eF_p[X]\) par un polynôme primitif. Le théorème suivant donne une description abstraite de \( \eF_q\) qui va nous servir de point de départ pour la construction.
\begin{theorem}[Théorème de l'élément primitif]    \label{ThoqSludu}
    Soit \( p\) un nombre premier, \( n\in \eN\) et \( q=p^n\). Soit \( \eK\) un corps à \( q\) éléments. Alors
    \begin{enumerate}
        \item
            Il existe \( \alpha\in \eK\) tel que \( \eK=\eF_p[\alpha]\).
        \item
            Il existe une polynôme irréductible \( P\in\eF_p[X]\) de degré \( n\) tel que
            \begin{equation}        \label{EqWlMhhm}
                \begin{aligned}
                    \phi\colon \eF_p[X]/(P)&\to \eK \\
                    \bar X&\mapsto \alpha 
                \end{aligned}
            \end{equation}
            soit un isomorphisme de corps.
    \end{enumerate}
    Soit \( \alpha\) et \( P\) choisis pour avoir les propriétés citées plus haut. Alors nous avons les propriétés suivantes.
    \begin{enumerate}
        \item
            \( P\) est primitif.
        \item
            \( P\) est scindé dans \( \eK\).
        \item
            L'ensemble des racines de \( P\) est \( \{ \alpha,\alpha^p,\ldots, \alpha^{p^{n-1}} \}\).
        \item
            Le polynôme \( P\) divise \( X^q-X\) dans \( \eF_p[X]\).
    \end{enumerate}
\end{theorem}
\index{théorème!élément primitif}

\begin{proof}
    Le corps \( \eK\) étant fini, il est cyclique par la proposition \ref{PropnfebjI}. Si \( \alpha\) un générateur de \( \eK^*\) alors
    \begin{equation}
        \eK=\eF_p[\alpha].
    \end{equation}
    Soit \( \ell\) le plus grand entier tel que l'ensemble
    \begin{equation}
        \{ 1,\alpha,\cdots,\alpha^{\ell-1} \}\subset\eK
    \end{equation}
    soit libre. Pour rappel \( \eK\) est une espace vectoriel sur \( \eF_p\). Il existe des \( a_i\in \eF_p\) tels que
    \begin{equation}
        \alpha^{\ell}+a_{\ell-1}\alpha^{\ell-1}+\ldots+a_0=0.
    \end{equation}
    De façon équivalente, il existe un polynôme unitaire \( P\in\eF_p[X]\) de degré \( \ell\) tel que \( P(\alpha)=0\). Étant donné que \( \alpha\) est générateur de \( \eK\),
    \begin{equation}
        \eK=\Span\{ 1,\alpha,\ldots, \alpha^{\ell-1} \}
    \end{equation}
    parce que \( \eK\) est généré par les puissances de \( \alpha\) alors que les puissances de \( \alpha\) plus hautes que \( \ell-1\) peuvent être générées par \( 1,\alpha,\ldots, \alpha^{\ell-1}\). L'espace \( \eK\) est donc un \( \eF_p\)-espace vectoriel de dimension \( \ell\); par conséquent
    \begin{equation}
        \Card(\eK)=p^n=q
    \end{equation}
    et \( \ell=n\).

    Montrons que \( P\) est irréductible dans \( \eF_p\). Si \( P\) était réductible dans \( \eF_p\), l'élément \( \alpha\in \eK\) serait une racine d'un des facteurs, c'est à dire qu'il serait racine d'un polynôme de degré inférieur à \( n\), ce qui contredirait le fait que 
    \begin{equation}
        \{ \alpha^{\ell-1},\ldots, 1 \}
    \end{equation}
    est libre.

    Montrons que l'application
    \begin{equation}
        \begin{aligned}
            \phi\colon \eF_p[X]/(P)&\to \eK \\
            \bar X&\mapsto \alpha 
        \end{aligned}
    \end{equation}
    est un isomorphisme. Pour l'injectivité, deux éléments \( Q_1,Q_2\in \eF_p[X]/(P)\) s'écrivent
    \begin{subequations}
        \begin{align}
            Q_1&=\sum_{k=0}^{n-1}a_k\bar X^k\\
            Q_2&=\sum_{k=0}^{n-1}b_k\bar X^k.
        \end{align}
    \end{subequations}
    Dans ce cas si \( \phi(Q_1)=\phi(Q_2)\) alors
    \begin{equation}
        \phi(Q_1)=\sum_{k=0}^{n-1}a_k\alpha^k=\phi(Q_2)=\sum_{k=0}^{n-1}b_k\alpha^k.
    \end{equation}
    Mais l'ensemble \( \{ 1,\alpha,\ldots, \alpha^{n-1} \}\) étant libre sur \( \eF_p\), cela implique \( a_k=b_k\). La surjectivité de \( \phi\) provient du fait que \( \alpha\) génère \( \eK\).

    Nous passons maintenant à la seconde partie de la démonstration. Soient \( \alpha\in \eK\) tel que \( \eK=\eF_p[\alpha]\) et \( P\in \eF_p[X]\) un polynôme irréductible de degré \( n\) tel que \( \alpha\mapsto \bar X\) soit un isomorphisme entre \( \eK\) et \( \eF_p[X]/(P)\).

    Le polynôme \( P\) est primitif parce que \( \alpha\) est d'ordre \( p^n\) dans \( \eK\) alors que \( \bar X\mapsto \alpha\) est un isomorphisme. Par conséquent \( \bar X\) est d'ordre \( p^n\) dans \( \eF_p[X]/P\).

    Nous commençons par prouver que l'ensemble
    \begin{equation}        \label{EqAcsQHL}
        \{ \alpha,\alpha^p,\alpha^{p^2},\ldots, \alpha^{p^{n-1}} \}
    \end{equation}
    est l'ensemble des racines distinctes de \( P\). Pour cela nous posons
    \begin{equation}
        P(X)=\sum_{k=0}^na_kX^k
    \end{equation}
    avec \( a_k\in\eF_p\). D'abord \( \alpha\) est une racine de \( P\). En effet
    \begin{equation}        \label{EqbTAmKG}
        P(\bar X)=\sum_ka_k\bar X^k=0
    \end{equation}
    parce que cette somme est calculée dans \( \eF_p[X]/(P)\). En appliquant l'isomorphisme \( \phi\) à l'égalité \eqref{EqbTAmKG} nous trouvons
    \begin{equation}
        0=\phi\big( P(\bar X) \big)=\sum_ka_k\phi(\bar X^k)=\sum_ka_k\alpha^k.
    \end{equation}
    Donc \( \alpha\) est bien une racine de \( P\) dans \( \eF_p[X]\). Nous devons montrer qu'il en est de même pour les autres puissances dans l'ensemble \eqref{EqAcsQHL}. Étant donné que pour tout \( x\) dans \( \eF_p\) nous avons \( x^p=x\), nous avons aussi
    \begin{equation}
        P(X^p)=\sum_ka_k(X^p)^k=\sum_ka_k^p(X^p)^k=\sum_k(a_kX^k)^p
    \end{equation}
    alors que nous savons que \( x\mapsto x^p\) est un automorphisme de \( \eF_p\) par la proposition \ref{PropFrobHAMkTY}. Par conséquent
    \begin{equation}
        P(X^p)=\sum_k(a_kX^k)^p=\left( \sum_k a_kX^k\right)^p=P(X)^p.
    \end{equation}
    Nous avons montré que si \( \beta\) est une racine de \( P\), alors \( \beta^p\) est également une racine de \( P\). Nous savons déjà que \( \alpha\) est une racine de \( P\), et que \( \alpha\) est également générateur de \( \eK\), c'est à dire que \( \alpha\) est d'ordre \( q-1\). Les puissances
    \begin{equation}
        \alpha,\alpha^p,\alpha^{p^2},\ldots, \alpha^{p^{n-1}}
    \end{equation}
    sont donc distinctes (\( \alpha^{p^n}=\alpha^q=1\)) et sont toutes des racines de \( P\). Étant donné que \( P\) est de degré \( n\) il ne peut pas y avoir d'autres racines. Nous concluons que l'ensemble
    \begin{equation}
        \{ \alpha,\alpha^p,\alpha^{p^2},\ldots, \alpha^{p^{n-1}}\}
    \end{equation}
    est l'ensemble des racines distinctes de \( P\) dans \( \eK\). Le polynôme \( P\) est alors scindé dans \( \eK[X]\).

    Le dernier point du théorème est de montrer que \( P\) divise \( X^q-X\). Pour cela nous allons montrer que toutes les racines de \( P\) sont des racines de \( X^q-X\). Soit \( \beta\) une racine de \( P\); il s'écrit \( \beta=\alpha^k\) pour un certain \( k\). Étant donné que \( \alpha^{q-1}=e=\alpha^{p^n-1}\),
    \begin{subequations}
        \begin{align}
            \beta^q&=(\alpha^{p^n})^k\\
            &=\left( \alpha^{p^n-1}\alpha \right)^k\\
            &=\left( \alpha^{q-1}\alpha \right)^k\\
            &=\alpha^k\\
            &=\beta.
        \end{align}
    \end{subequations}
    Cela signifie que \( \beta^q=\beta\) et donc que \( \beta\) est racine de \( X^q-X\).
\end{proof}

\begin{corollary}
    Le corps fini à \( q=p^n\) éléments est de caractéristique \( p\).
\end{corollary}

\begin{proof}
    Nous considérons le corps fini \( \eK\) à \( q\) éléments sous la forme \( \eK=\eF_p[X]/P\) comme indiqué par l'équation \eqref{EqWlMhhm}. Soit \( 1_q\) la classe du polynôme \( 1\) modulo \( P\), nous considérons le morphisme
    \begin{equation}
        \begin{aligned}
            \mu\colon \eZ&\to \eF_q \\
            n&\mapsto n1_q. 
        \end{aligned}
    \end{equation}
    Le noyau de cette application est \( \ker\mu=\eZ_p\) parce que \( p1_q=0\), les coefficients étant à comprendre dans \( \eF_p\).
\end{proof}

\begin{definition}  \label{DefnPNCFO}
    Soit \( P\), un polynôme de degré \( n\) et \( p\), un nombre premier. Un élément \( \alpha\in \eF_p[X]/(P)\) est une \defe{racine primitive}{racine!primitive}\index{primitif!racine} si les puissances de \( \alpha\) parcourent tout le groupe multiplicatif \( (\eF_p[X]/P)^*\).
\end{definition}

\begin{lemma}       \label{Lembcerei}
    Soit \( p\) un nombre premier et \( P\), un polynôme de degré \( n\). Si \( \alpha\in \eF_p[X]/P\) est une racine primitive de \( P\) alors les autres racines de \(P\) sont également primitives.
\end{lemma}

\begin{proof}
    Soit \( \alpha\in \eF_p[X]/P\) une racine primitive de \( P\). L'élément \( \alpha^p\) est également une racine parce que si \( P=\sum_ka_kX^k\),
    \begin{equation}
        P(\alpha^p)=\sum_k(a_k\alpha^k)^p=\big( \sum_ka_k\alpha^k \big)^p=0
    \end{equation}
    où nous avons utilisé le fait que \( a_k^p=a_k\) étant donné que \( a_k\in\eF_p\). Par hypothèse \( \alpha\) est une racine primitive; cela implique que les éléments \( \alpha,\alpha^p,\alpha^{p^2},\ldots,\alpha^{p^n-1}\) sont distincts dans \( \eF_p[X]/P\). Ces éléments constituent donc \emph{toutes} les racines de \( P\).

    Soit \( \beta=\alpha^{p^k}\) une racine de \( P\). Montrons que \( \alpha\) est une puissance de \( \beta\). Étant donné que \( (\eF_p[X]/P)^*\) est un groupe à \( p^n-1\) éléments, le corollaire \ref{CorpZItFX} indique que \( \alpha^{p^n}=\alpha\). En particulier avec \( r=p^{n-k}\) nous avons
    \begin{equation}
        \beta^r=\alpha^{rp^k}=\alpha^{p^n}=\alpha.
    \end{equation}
    Par suite toutes les puissances de \( \alpha\) sont des puissances de \( \beta\), ce qui implique que \( \beta\) est générateur du groupe cyclique \( (\eF_p[X]/P)^*\).
\end{proof}

\begin{lemma}       \label{LemkzWjse}
    Soit \( p\) un nombre premier et \( n\), un entier. Un polynôme de degré \( d\) irréductible dans \( \eF_p[X]\) divise \( X^{p^n}-X\) si et seulement si \( d\) divise \( n\).
\end{lemma}

\begin{theorem}
    Soient \( P\) et \( Q\) deux polynômes irréductibles de degré \( n\) dans \( \eF_p[X]\). Alors les quotients \( \eF_p[X]/P\) et \( \eF_p[X]/Q\) sont isomorphes en tant que corps.
\end{theorem}
En guise de démonstration de ce théorème, nous allons démontrer la proposition suivante.
\begin{proposition}      \label{PropCRPjZsp}
    Si \( \eK\) et \( \eL\) sont deux corps à \( q=p^n\) éléments, alors ils sont isomorphes.
\end{proposition}

\begin{proof}
    Soit \( a\) un élément primitif de \( \eK\) et \( P\) son polynôme minimal. Nous savons que \( \eK\simeq \eF_p[X]/P\) par le théorème de l'élément primitif \ref{ThoqSludu}. L'élément \( a\) est en particulier une racine de \( X^q-X\). Par ailleurs \( P\) divise \( X^q-X\) par le lemme \ref{LemkzWjse}.

    Nous avons aussi
    \begin{equation}
        X^q-X=\prod_{b\in \eL}(X-b)
    \end{equation}
    par la proposition \ref{propQRcUlq}. Étant donné que \( P\) divise \( X^q-X\), un des éléments de \( \eL\) annule \( P\). Soit \( b\in \eL\) tel que \( P(b)=0\). Soit \( Q\) le polynôme minimal de \( b\). Par définition nous avons que \( Q\) divise \( P\), mais \( P\) étant irréductible et unitaire nous avons immédiatement \( P=Q\). En particulier nous avons
    \begin{equation}
        \eF_p[X]/P\simeq\eF_p[X]/Q\simeq \eK.
    \end{equation}
    Nous montrons maintenant que \( \eF_p[X]/Q\simeq \eL\) par l'application
    \begin{equation}
        \begin{aligned}
            \phi\colon \eF_p[X]/Q&\to \eL \\
            \bar X&\mapsto b 
        \end{aligned}
    \end{equation}
    qui se prolonge en \( R(\bar X)\mapsto R(b)\) pour tout \( R\in \eF_p[X]\). Cette application est bien définie parce que \( Q(b)=0\). Elle est injective parce que \( R(b)=0\) ne peut pas avoir lieu avec \( R\in \eF_p[X]/Q\) parce que \( Q\) est le polynôme minimal de \( b\). La surjectivité vient alors du fait que les deux corps ont le même nombre d'éléments.
\end{proof}

%---------------------------------------------------------------------------------------------------------------------------
\subsection{Construction de $\eF_q$}
%---------------------------------------------------------------------------------------------------------------------------

Soit \( p\) un nombre premier et \( n\in \eN\). Nous souhaitons construire un corps à \( q=p^n\) éléments. Nous savons déjà que ce corps est unique (théorème \ref{ThoOzgSfy}) et que nous le notons \( \eF_q\). Le théorème \ref{ThoqSludu} nous incite à l'écrire sous la forme
\begin{equation}
    \eF_q=\eF_p[X]/(P)
\end{equation}
pour un certain polynôme irréductible \( P\in\eF_p[X]\).

%///////////////////////////////////////////////////////////////////////////////////////////////////////////////////////////
\subsubsection{La version du faignant}
%///////////////////////////////////////////////////////////////////////////////////////////////////////////////////////////

Nous pouvons construire le corps à \( p^n\) éléments en prenant le quotient de \( \eF_p[X]\) par n'importe quel polynôme irréductible de degré \( n\). Le résultat est le suivant.
\begin{proposition} \label{PropHfrNCB}
    Soit \( P\) une polynôme unitaire irréductible dans \( \eF_p[X]\). Nous posons \( \eK=\eF_p[X]/(P)\). Alors
    \begin{enumerate}
        \item
            \( \eK\) est un corps à \( q\) éléments.
        \item
            \( \alpha=\bar X\) est une racine de \( P\) dans \( \eK\).
        \item   \label{ItemiEFRTg}
            \( \eK=\eF_p[\alpha]\).
    \end{enumerate}
\end{proposition}

\begin{proof}
    \begin{enumerate}
        \item
            En vertu du corollaire \ref{CorsLGiEN}, \( \eK\) est un corps. Il est aussi un espace vectoriel de dimension \( n\) sur \( \eF_p\), et contient donc \( p^n=q\) éléments.
        \item
            Nous avons \( P(\bar X)=0\) par construction de \( \eK=\eF_p[X]/(P)\).
        \item
            En tant que quotient de \( \eF_p[X]\), les éléments de \( \eK\) sont des polynômes en \( \bar X\).
    \end{enumerate}
\end{proof}

%///////////////////////////////////////////////////////////////////////////////////////////////////////////////////////////
\subsubsection{La version plus élaborée}
%///////////////////////////////////////////////////////////////////////////////////////////////////////////////////////////

Construire \( \eF_q\) comme quotient de \( \eF_p[X]\) par un polynôme irréductible quelconque ne donne pas d'informations sur les générateurs de \( \eF_q^*\), et en particulier il n'est pas toujours vrai que \( \bar X\) est générateur.

\begin{example}
    Construisons \( \eF_4\). Le polynôme \( X^2+X+1\) est irréductible dans \( \eF_2\) parce qu'il n'a pas de racines (c'est vite vu : dans \( \eF_2\) il n'y a que deux candidats). Donc \( \eF_4=\eF_2[X]/(X^2+X+1)\).
\end{example}

\begin{remark}
    Le corps \( \eF_2\) n'est pas un sous-corps de \( \eC\) parce que leurs caractéristiques ne sont pas les mêmes. Une conséquence est que les racines de polynômes peuvent être très différentes. Par exemple le polynôme \( X^1+1\) accepte \( x=1\) comme racine dans \( \eF_2\) tandis qu'il a pour racines \( \pm i\) dans \( \eC\).

    En changeant de corps, les racines peuvent donc complètement changer. Ce n'est pas juste qu'il y a des racines dans l'un et pas dans l'autre.
\end{remark}

\begin{example}     \label{ExemWUdrcs}
    Cherchons à construire \( \eF_{16}\) comme quotient de \( \eF_2\) par un polynôme de degré \( 4\).
    \begin{verbatim}
----------------------------------------------------------------------
| Sage Version 4.7.1, Release Date: 2011-08-11                       |
| Type notebook() for the GUI, and license() for information.        |
----------------------------------------------------------------------
sage: x=polygen(GF(2))
sage: -x-1
x + 1
sage: Q=x**15-1
sage: Q.factor()
(x + 1) * (x^2 + x + 1) * (x^4 + x + 1) * (x^4 + x^3 + 1) * (x^4 + x^3 + x^2 + x + 1)
    \end{verbatim}
    Les polynômes candidats a avoir des racines génératrices sont donc au nombre de \( 3\):
    \begin{subequations}
        \begin{align}
            P_1&=X^4+X+1\\
            P_2&=X^4+X^3+1\\
            P_3&=X^4+X^3+X^2+X+1.
        \end{align}
    \end{subequations}
    Dans le quotient \( \eF_2[X]/P_3\), l'élément \( \bar X\) n'est pas générateur. En effet nous avons \( X^4=X^3+X^2+X+1\) et par conséquent les puissances successives de \( X\) sont
    \begin{subequations}
        \begin{align}
            X&\\
            X^2&\\
            X^3&\\
            X^4&=X^3+X^2+X+1\\
            1.
        \end{align}
    \end{subequations}
    La classe de \( X\) dans \( \eF_2[X]/P_3\) n'est donc pas génératrice du groupe \( (\eF_2[X]/P_3)^*\).

    Le polynôme \( P_1=X^4+X+1\) par contre est primitif parce que les puissances de \( X\) dans \( \eF_2[X]/P_1\) sont
    \begin{subequations}
        \begin{align}
            X\\
            X^2\\
            X^3\\
            X+1\\
            X^2+X\\
            X^3+X^2\\
            X+1+X^3\\
            X^2+1\\
            X^3+X\\
            X+1+X^2\\
            X^2+X+X^3\\
            X^3+X^2+X+1\\
            1+X^2+X^2\\
            1+X^3\\
            1
        \end{align}
    \end{subequations}
    Cela fait \( 15\) puissances distinctes, ce qui prouve que \( P_1\) est primitif. Nous verrons plus loin comment alléger un peu la vérification de la primitivité de \( P_1\).
\end{example}

\begin{proposition}     \label{PropNsLqWb}
    Soit \( P\) un polynôme irréductible unitaire primitif dans \( \eF_p[X]\). Nous considérons \( \eK=\eF_p[X]/P\) et \( \alpha=\bar X\in \eK\). Alors
    \begin{enumerate}
        \item
            Les racines de \( P\) sont \( \{ \alpha,\alpha^p,\ldots, \alpha^{p^{n-1}} \}\) et \( \alpha^q=\alpha\).
        \item
            \( P\) est le polynôme minimal de \( \alpha\).
        \item
            \( P\) est scindé dans \( \eK\).
        \item
            \( P\) divise \( X^q-X\) dans \( \eK\).
        \item
            La famille \( \{1, \alpha,\alpha^2,\ldots, \alpha^{n-1} \}\) est une base de \( \eK\) en tant qu'espace vectoriel sur \( \eF_p\).
        \item
            En tant qu'ensemble,
            \begin{equation}
                \eF_q=\{0, \alpha,\alpha^2,\alpha^3,\ldots, \alpha^{q-1} \},
            \end{equation}
            et les \( \alpha^k\) sont distincts pour \( k=1,\ldots, q-1\).
    \end{enumerate}
\end{proposition}

\begin{proof}
    La plupart des assertions sont des corollaires ou des paraphrases de résultats contenus dans les propositions précédentes.
    \begin{enumerate}
        \item
            L'assertion à propos des racines de \( P\) est contenue dans le lemme \ref{Lembcerei}. D'autre part le groupe \( (\eF_p[X]/P)^*\) est cyclique d'ordre \( q-1\). Par conséquent le corollaire \ref{CorpZItFX} indique que \( \alpha^{q-1}=1\) et donc \( \alpha^q=\alpha\).
        \item
            Soit \( \tilde P\) un polynôme annulateur de \( \alpha\). Nous voyons que si \( \beta\) est racine de \( \tilde P\) alors \( \beta^p\) est également racine de \( \tilde P\) en utilisant les techniques habituelles. Par conséquent toutes les racines de \( P\) sont racines de \( \tilde P\), ce qui implique que \( \tilde P\) est de degré au moins égal à celui de \( P\).
        \item
            Possédant \( n\) racines distinctes dans \( \eK\), le polynôme \( P\) est scindé.
        \item
            D'après le lemme \ref{propQRcUlq} un polynôme irréductible de degré \( n\) divise le polynôme \( X^{p^n}-X\). Une autre façon de montrer ce point est de remarquer que le polynôme \( P\) est scindé et que toutes ses racines sont également racines de \( X^q-X\).
        \item
            Une combinaison linéaire nulle entre les éléments de \( \{ 1,\alpha,\alpha^2,\ldots, \alpha^{n-1} \}\) serait un polynôme annulateur de degré \( n-1\) de \( \alpha\). Cet ensemble est donc libre. Par ailleurs un ensemble libre de \( n\) éléments dans un espace vectoriel de dimension \( n\) est générateur.
        \item
            Si \( \alpha^l=\alpha^k\) avec \( k<l\) et \( k,l\leq q\) alors nous avons \( \alpha^r=1\) avec \( r=l-k<q\), ce qui contredirait la primitivité de \( P\). Les éléments \( 0,\alpha,\ldots, \alpha^{q-1}\) étant distincts et au nombre de \( q\), ils forment tout l'ensemble \( \eF_q\).

    \end{enumerate}
\end{proof}

%---------------------------------------------------------------------------------------------------------------------------
\subsection{Exemple : étude de \texorpdfstring{$\eF_{16}$}{F16}}
%---------------------------------------------------------------------------------------------------------------------------

Dans cette sous section nous voulons construire \( \eF_{16}\). Nous considérons donc \( p=2\) et \( n=4\). Des polynôme irréductibles de degré \( 4\) dans \( \eF_2[X]\) ne sont pas très difficiles à trouver. Par exemple \( X^4+X^3+X^2+X+1\), plus généralement un polynôme contenant un nombre impair de termes non nuls dont le terme indépendant.

Les polynômes primitifs par contre doivent être trouvés parmi les diviseurs irréductibles de \( X^{15}-1\). Montrons que
\begin{equation}
    P=X^4+X^3+1
\end{equation}
est primitif. Nous posons \( \omega=\bar X\in \eF_2[X]/P\). L'ordre de \( \omega\) dans le groupe \( (\eF_2[X]/P)^* \) doit être un diviseur de \( 15\) et donc seulement \( 1\), \( 3\), \( 5\) ou \( 15\). Le fait que l'ordre ne soit ni \( 1\) ni \( 3\) est trivial parce que le degré de \( P\) est \( 4\). Montrons que l'ordre de \( \omega\) n'est pas \( 5\) non plus :
\begin{equation}
    \omega^5=\omega^4\omega=(\omega^3+1)\omega=\omega^4+\omega=\omega^3+\omega+1\neq 1.
\end{equation}
Dans ce calcul nous avons abondamment utilisé le fait que \( -1=1\).

À partir de maintenant nous posons \( \eK=\eF_2[X]/P\). Les racines de \( P\) sont \( \omega,\omega^2,\omega^4\) et \( \omega^8\). En effet si \( \beta\) est une racine de \( P\), alors \( \beta^2\) est une racine en vertu de 
\begin{equation}
    P(\beta^2)=(\beta^2)^4+(\beta^2)^3+1=(\beta^4)^2+(\beta^3)^2+1^2=(\beta^4+\beta^3+1)^2=0.
\end{equation}
Ici nous avons implicitement utilisé le lemme \ref{LemZrUUOz}. D'autre part \( P\) ne peut pas avoir plus de \( 4\) racines.

\begin{proposition}
    L'ensemble \( \{ \omega,\omega^2,\omega^4,\omega^8 \}\) est une base de \( \eF_{16}\) sur \( \eF_2\).
\end{proposition}

\begin{proof}
    Nous savons que \( \{ 1,\omega,\omega^2,\omega^3 \}\) est une base. En effet cet ensemble est libre (sinon \( \omega\) aurait un polynôme annulateur de degré \( 3\)) et générateur parce que l'espace engendré par \( 4\) vecteurs indépendants sur \( \eF_2\) contient \( 2^4=16\) éléments.

    Nous posons \( e_0=1\), \( e_1=\omega\), \( e_2=\omega^2\), \( e_3=\omega^3\) et \( f_1=\omega\), \( f_2=\omega^2\), \( f_3=\omega^4\), \( f_4=\omega^8\). En utilisant le calcul modulo \( \omega^4+\omega^3+1=0\) et \( 2=0\) nous trouvons
    \begin{subequations}
        \begin{align}
            f_1&=\omega\\
            f_2&=\omega^2\\
            f_3&=\omega^3+1\\
            f_4&=\omega^3+\omega^2+\omega.
        \end{align}
    \end{subequations}
    Ensuite nous montrons que les vecteurs \( e_i\) peuvent être construits comme combinaisons linéaires des vecteurs \( f_j\) :
    \begin{subequations}
        \begin{align}
            f_1+f_2+f_3+f_4&=e_0\\
            f_1&=e_1\\
            f_2&=e_2\\
            f_1+f_2+f_4&=e_3.
        \end{align}
    \end{subequations}
    Les quatre vecteurs \( f_j\) forment donc bien un base parce qu'ils sont générateurs d'un espace de dimension \( 4\).
\end{proof}

\begin{example}
    \begin{enumerate}
        \item
            Résoudre dans \( \eF_{16}\) l'équation \( x^5=a\) en discutant éventuellement en fonction de la valeur de \( a\).
        \item
            Montrer qu'il existe quatre éléments \( \gamma\in\eF_{16}\) tels que pour chacun d'eux l'ensemble \( B_{\gamma}=\{ \gamma,\gamma^2,\gamma^4,\gamma^8 \}\) est une base de \( \eF_{16}\) sur \( \eF_2\) telle que le produit de deux éléments de \( B_{\gamma}\) est soit un élement de \( B_{\gamma}\) soit \( 1\).
    \end{enumerate}

    C'est parti !

    \begin{enumerate}
        \item
            Si \( a=0\), alors \( x=0\) est la seule solution. Si \( a\neq 0 \) alors \( a\) est une puissance de \( \omega\); nous posons \( a=\omega^l\). Nous cherchons \( x\) sous la forme \( x=\omega^k\). L'équation à résoudre pour \( k\) est
            \begin{equation}
                \omega^{5k}=\omega^l
            \end{equation}
            où \( l\) est donné. Cette équation revient à 
            \begin{equation}
                5k=l\mod 15.
            \end{equation}
            Si \( l\) n'est pas un multiple de \( 5\), alors il n'y a pas de solutions. Il n'y a des solutions uniquement pour \( l=0,5,10\) et elles sont :
            \begin{equation}
                k=\begin{cases}
                    3,6,9,12     &   \text{si \( l\)=0}\\
                    1     &   \text{si \( l\)=5}\\
                    2     &   \text{si \( l\)=10}
                \end{cases}
            \end{equation}
        \item
            Nous cherchons \( \gamma\) sous la forme \( \gamma=\omega^k\). Parmi les nombreuses contraintes liées à l'énoncé nous devons avoir
            \begin{equation}
                \gamma^5=1,\gamma,\gamma^2,\gamma^4,\gamma^8.
            \end{equation}
            Les possibilités \( \gamma^5=\gamma,\gamma^2,\gamma^4,\gamma^5\) ne sont pas bonnes parce qu'elles impliqueraient que \( B_{\gamma}\) n'est pas une base. Reste à explorer \( \gamma^5=1\).

            Étant donné le premier point nous restons avec les possibilités
            \begin{equation}
                \gamma=1,\omega^3,\omega^6,\omega^9,\omega^{12}.
            \end{equation}
            Évidemment \( \gamma=1\) ne produit pas une base. Avec \( \gamma=\omega^3\) nous trouvons
            \begin{equation}
                B_{\gamma}=\{ \omega^3,\omega^6,\omega^{12},\omega^{24} \}=\{ \omega^3,\omega^6,\omega^{12},\omega^9 \}
            \end{equation}
            où nous avons utilisé le fait que \( \omega^k=\omega^{k\mod 15}\). En utilisant le fait que \( \omega^4=\omega^3+1\) nous trouvons
            \begin{subequations}
                \begin{align}
                    \omega^5&=\omega^3+\omega+1\\
                    \omega^6&=\omega^3+\omega^2+\omega+1\\
                    \omega^9&=\omega^2+1\\
                    \omega^{12}&=\omega+1.
                \end{align}
            \end{subequations}
            L'ensemble \( B_{\gamma}\) est alors formé des éléments
            \begin{subequations}
                \begin{align}
                    f_1&=\omega^3\\
                    f_2&=\omega^3+\omega^2+\omega+1\\
                    f_3&=\omega+1\\
                    f_4&=\omega^2+1.
                \end{align}
            \end{subequations}
            Il est assez simple de vérifier que cela est une base en remarquant que \( f_1+f_2+f_2+f_4=1\).

            Les possibilités \( \gamma=\omega^6,\omega^9,\omega^{12}\) produisent les mêmes ensemble \( B_{\gamma}\).
    \end{enumerate}
\end{example}

%--------------------------------------------------------------------------------------------------------------------------- 
\subsection{Polynômes irréductibles sur $\eF_q$}
%---------------------------------------------------------------------------------------------------------------------------

\begin{definition}  \label{DefWXBkOxg}
    La \defe{fonction de Möbius}{fonction!de Möbius} est la fonction \( \mu\colon \eN^*\to \{ -1,0,1 \}\) définie par
    \begin{equation}
        \mu(n)=\begin{cases}
            0    &   \text{si \( n\) est divisible par un carré différent de \( 1\),}\\
            1    &   \text{si \( n\) est le produit d'un nombre pair de nombres premiers distincts,}\\
            -1    &    \text{si \( n\) est le produit d'un nombre impair de nombres premiers distincts,}
        \end{cases}
    \end{equation}
\end{definition}

\begin{proposition}[\cite{POkXeBE}]
    Si \( m\) et \( n\) sont strictement positifs et premiers entre eux, alors
    \begin{equation}
        \mu(mn)=\mu(m)\mu(n).
    \end{equation}
    De plus nous avons
    \begin{equation}
        \sum_{d\divides n}\mu(d)=\begin{cases}
            1    &   \text{si \( n=1\)}\\
            0    &    \text{si \( n>1\)}
        \end{cases}
    \end{equation}
\end{proposition}
%TODO : une preuve.

\begin{proposition}[Formule d'inversion de Möbius\cite{POkXeBE}]    \label{PropLBZoIoO}
    Soient \( f,g\colon \eN\to \eC\) telles que pour tout \( n\geq 1\),
    \begin{equation}
        g(n)=\sum_{d\divides n}f(d).
    \end{equation}
    Alors
    \begin{equation}
        f(n)=\sum_{d\divides n}\mu\left( \frac{ n }{ d } \right)g(d)
    \end{equation}
    où \( \mu\) est la fonction de Möbius pour tout \( n\geq 1\).
\end{proposition}
\index{formule!inversion Möbius}

\begin{lemma}[\cite{ERWMpWo}]   \label{LemRGuWqNu}
    Soient \( P,Q\in \eK[X]\) ayant une racine commune dans une extension \( \eL\) de \( \eK\). Si \( P\) est irréductible, alors \( P\divides Q\).
\end{lemma}

\begin{proof}
    Si \( P\) ne divise pas \( Q\), alors \( P\) et \( Q\) sont premiers entre eux parce que dans la décomposition en irréductibles de \( Q\), il n'y a pas de \( P\) tandis que dans celle de \( P\), il n'y a que \( P\). Par conséquent, il existe \( a,b\in \eK\subset\eL\) tels que\footnote{Théorème de Bézout, \ref{ThoBezoutOuGmLB}.} \( aP+bQ=1\). Cette dernière égalité est encore valable dans \( \eL\) et donc rend impossible l'existence d'une racine commune.
\end{proof}

\begin{proposition}[\cite{ERWMpWo,KXjFWKA}] \label{PropVFNOvzZ}
    Soit \( p\) un nombre premier, \( n\geq 1\) et \( r\in \eN^*\). Nous notons \( q=p^r\), \( A(n,q)\), l'ensemble des polynômes unitaires irréductibles de degré \( n\) sur \( \eF_q\). Nous notons aussi \( I(n,q)=\Card\big( A(n,q) \big)\). Alors :
    \begin{enumerate}
        \item
            Le polynôme \( X^{q^n}-X\) se décompose en irréductibles de la façon suivante :
            \begin{equation}
                X^{q^n}-X=\prod_{d\divides n}\,\prod_{p\in A(d,q)}P.
            \end{equation}
        \item
            Le nombre d'irréductibles est donné par
            \begin{equation}
                I(n,q)=\frac{1}{ n }\sum_{d\divides n}\mu\left( \frac{ n }{ d } \right)q^d
            \end{equation}
            où \( \mu\) est la fonction de Möbius (définition \ref{DefWXBkOxg}).
        \item
            Nous avons l'équivalence de suite
            \begin{equation}
                I(n,q)\sim_{n\to\infty}\frac{ q^n }{ n }.
            \end{equation}
    \end{enumerate}
\end{proposition}
\index{polynôme!irréductible!sur $ \eF_q$}

\begin{proof}
    \begin{enumerate}
        \item
            Soit un diviseur \( d\) de \( n\) et \( P\in A(d,q)\). Montrons que \( P\) divise \( X^{q^n}-X\). Nous considérons la corps \( \eK=\eF_q[X]/(P)\), qui est une extension de degré \( \deg(P)\) de \( \eF_q\) parce qu'il s'agit des polynômes de degré au maximum \( \deg(P)\) a coefficients dans \( \eF_q\). Ce corps possède donc \( q^d\) éléments et est isomorphe à \( \eF_{q^d}\) par la proposition \ref{PropCRPjZsp}. Par construction dans \( \eK\), l'élément \( \alpha=[X]\) (la classe de \( X\) dans le quotient par \( P\)) est une racine de \( P\). Cet élément est également une racine de \( X^{q^d}-X\) parce que tout élément de \( \eF_{q^d}\) est une racine de ce polynôme. Ce dernier point est la proposition \ref{propQRcUlq}.

            Nous sommes donc dans la situation où \( P\) et \( X^{q^d}-X\) ont une racine commune dans l'extension \( \eF_q[X]/(P)\). Nous en déduisons que \( \alpha\) est aussi une racine de \( X^{q^n}-X\). En effet en utilisant le fait que \( \alpha^{q^d}=\alpha\), nous avons
            \begin{equation}
                \alpha^{q^n}=\alpha^{q^{kd}}=\alpha^{q^dq^{(k-1)d}}=\left( \alpha^{q^d} \right)^{q^{(k-1)d}}=\alpha^{q^{(k-1)d}},
            \end{equation}
            donc par récurrence, on a encore \( \alpha^{q^n}=\alpha\), et \( \alpha\) est racine de \( X^{q^n}-X\). Vu que \( P\) est irréductible, le lemme \ref{LemRGuWqNu} nous indique que \( P\) divise \( X^{q^n}-X\). Nous en déduisons que \( P\) divise \( X^{q^n}-X\).

            Étant donné que tous les éléments de \( A(d,q)\) divisent \( X^{q^n}-X\) et sont irréductibles, leur produit divise encore \( X^{q^n}-X\) :
            \begin{equation}
                \prod_{d\divides n}\prod_{P\in A(d,q)}P\divides X^{q^n}-X.
            \end{equation}
            
            Nous devons à présent montrer que tous les facteurs irréductibles de \( X^{q^n}-X\) sont dans un \( A(d,q)\) avec \( d\divides n\). Soit donc \( P\) un facteur irréductible de \( X^{q^n}-X\) de degré \( d\geq 1\). Nous posons encore \( \eK=\eF_q[X]/(P)\) et nous utilisons la propriété de multiplication sur les degrés (proposition \ref{PropGWazMpY}) :
            \begin{equation}
                [\eF_{q^n}:\eK][\eK:\eF_q]=[\eF_{q^n}:\eF_q]=n,
            \end{equation}
            donc \( [\eK:\eF_q]\), qui vaut \( \deg(P)\) est un diviseur de \( n\).

            Étant donné que \( X^{q^n}-X\) n'a que des racines simples sur \( \eF_{q^n}\) (à nouveau la proposition \ref{propQRcUlq}), dans sa décomposition en irréductibles sur \( \eF_q\), il n'a pas de facteurs carrés; il n'a donc qu'une fois chacun des \( P\in A(d,q)\) avec \( d\divides n\). Autrement dit, tous les facteurs irréductibles de \( X^{q^n}-X\) sont dans le produit \( \prod_{d\divides n}\prod_{P\in A(d,q)}P\) et donc \( X^{q^n}-X\) divise ce gros produit :
            \begin{equation}
                X^{q^n}-X\divides \prod_{d\divides n}\prod_{P\in A(d,q)}P.
            \end{equation}
            Ayant déjà obtenu la divisibilité inverse et les polynômes étant unitaires, nous avons égalité.

        \item

            Nous passons au degré dans l'expression que nous venons de démontrer :
            \begin{equation}
                q^n=\sum_{d\divides n}d\Card\big( A(d,q) \big)=\sum_{d\divides n}dI(d,q).
            \end{equation}
            Nous pouvons utiliser la formule d'inversion de Möbius (proposition \ref{PropLBZoIoO}) pour les fonctions \( g(n)=q^n\) et \( f(n)=dI(n,q)\). Nous écrivons alors
            \begin{equation}
                f(n)=\sum_{d\divides n}\mu\left( \frac{ n }{ d } \right)q^d,
            \end{equation}
            ou encore 
            \begin{equation}
                I(n,q)=\frac{1}{ n }\sum_{d\divides n}\mu\left( \frac{ n }{ d } \right)q^d,
            \end{equation}
            ce qu'il fallait.

        \item

            Nous posons 
            \begin{equation}
                r_n=\sum_{\substack{d\divides n\\d<n}}\mu\left( \frac{ n }{ d } \right)q^d,
            \end{equation}
            mais sachant que les diviseurs de \( n\), outre \( n\) lui-même, sont tous plus petit ou égal à \( n/2\) et qu'en valeur absolue, la fonction de Möbius est toujours plus petite ou égale à\quext{Dans \cite{KXjFWKA}, ma dernière inégalité arrive comme une égalité.} \( 1\),
            \begin{equation}
                | r_n |\leq\sum_{d=1}^{\lfloor n/2\rfloor}q^d=\frac{ q-q^{\lfloor n/2\rfloor} }{ 1-q }=q\frac{ q^{\lfloor n/2}-1 }{ q-1 }\leq \frac{ q^{\lfloor n/2 \rfloor+1} }{ q-1 }.
            \end{equation}
            D'autre part en reprenant la formule déjà prouvée,
            \begin{equation}
                I(n,q)=\frac{1}{ n }\sum_{d\divides n}\mu\left( \frac{ n }{ d } \right)q^d=\frac{1}{ n }\left( r_n+\mu\left( \frac{ n }{ n } \right)q^n \right)=\frac{ r_n+q^n }{ n }.
            \end{equation}
            Au numérateur, le plus haut degré en \( n\) est \( q^n\) parce que \( r_n\) est en \( q^{\lfloor n/2\rfloor}\). Donc nous avons bien l'équivalence de suite pour \( n\to \infty\) :
            \begin{equation}
                \frac{ q^n+r_n }{ n }\sim_{n\to\infty}\frac{ q^n }{ n }.
            \end{equation}
    \end{enumerate}
\end{proof}
