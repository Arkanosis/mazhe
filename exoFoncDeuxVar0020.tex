% This is part of the Exercices et corrigés de mathématique générale.
% Copyright (C) 2010
%   Laurent Claessens
% See the file fdl-1.3.txt for copying conditions.

\begin{exercice}\label{exoFoncDeuxVar0020}

	(INGE1121, 9.19) Calculer les dérivées partielles en $(0,0)$ de la fonction $F(x,y)=f\big( g(x,y),y \big)$ sachant que
	\begin{equation}
		\begin{aligned}[]
			g(0,0)&=1	& \frac{ \partial g }{ \partial x }(0,0)&=-1	&	\frac{ \partial g }{ \partial y }(0,0)&=3\\
			f(0,0)&=2	& \frac{ \partial f }{ \partial x }(0,0)&=-1	&	\frac{ \partial f }{ \partial y }(0,0)&=7\\
			f(2,0)&=3	& \frac{ \partial f }{ \partial x }(1,0)&=-2	&	\frac{ \partial f }{ \partial y }(1,0)&=8\\
			f(0,2)&=4	& \frac{ \partial f }{ \partial x }(0,1)&=-4	&	\frac{ \partial f }{ \partial y }(0,1)&=9.
		\end{aligned}
	\end{equation}

\corrref{FoncDeuxVar0020}
\end{exercice}
