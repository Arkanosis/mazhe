% This is part of Mes notes de mathématique
% Copyright (c) 2011-2012
%   Laurent Claessens
% See the file fdl-1.3.txt for copying conditions.

%+++++++++++++++++++++++++++++++++++++++++++++++++++++++++++++++++++++++++++++++++++++++++++++++++++++++++++++++++++++++++++
%+++++++++++++++++++++++++++++++++++++++++++++++++++++++++++++++++++++++++++++++++++++++++++++++++++++++++++++++++++++++++++
\section{Généralités}
%+++++++++++++++++++++++++++++++++++++++++++++++++++++++++++++++++++++++++++++++++++++++++++++++++++++++++++++++++++++++++++

Source : \cite{Tauvel}.

\begin{definition}
    Un \defe{anneau}{anneau} est un triple \( (A,+,\cdot)\) avec les conditions
    \begin{enumerate}
        \item
            \( (A,+)\) est un groupe abélien. Nous notons \( 0\) le neutre.
        \item
            La multiplication est associative et nous notons \( 1\) le neutre
        \item
            La multiplication est distributive par rapport à l'addition.
    \end{enumerate}
\end{definition}

\begin{remark}
    Un anneau est ce qu'on appelle «\emph{ring}» en anglais. Un corps est en anglais «\emph{field}».
\end{remark}

Soit \( X\) un ensemble et $A$ un anneau. Nous considérons \( \Fun(X,A)\)\nomenclature[A]{\( \Fun(X,Y)\)}{les applications de \( X\) vers \( Y\)} l'ensemble des applications \( X\to A\). Cet ensemble devient un anneau avec les définitions
\begin{subequations}
    \begin{align}
        (f+g)(x)=f(x)+g(x)\\
        (fg)(x)=f(x)g(x).
    \end{align}
\end{subequations}
Cela est la \defe{structure canonique}{structure d'anneau canonique} d'anneau sur \( \Fun(X,A)\).

Le \defe{centralisateur}{centralisateur} de \( x\in A\) dans \( A\) est l'ensemble
\begin{equation}
    \{ y\in A\tq xy=yx \},
\end{equation}
le \defe{centre}{centre!d'un anneau} de \( A\) est
\begin{equation}
    \{ y\in A\tq xy=yx\forall x\in A \}.
\end{equation}


Un élément \( a\in A\) est \defe{régulier à droite}{régulier à droite} \( ba=0\) implique \( b=0\). Il est régulier ) gauche si \( ab=0\) implique \( b=0\).

L'ensemble \( U(A)\)\nomenclature[A]{\( U(A)\)}{ensemble des inversibles} des éléments inversibles de \( A\) est un groupe pour la multiplication. Nous notons \( A^*=A\setminus\{ 0 \}\).

\begin{lemma}
    Si \( a\) et \( b\) commutent, nous avons la formule
    \begin{equation}        \label{Eqarpurmkbk}
        a^{r+1}-b^{r+1}=(a-b)(\sum_{k=0}^ra^{r-k}b^k).
    \end{equation}
\end{lemma}

\begin{proposition}
    Si \( a\) est un élément nilpotent de l'anneau \( A\), alors \( 1-a\) est inversible. Si \( a\) est nilpotent non nul, alors il est diviseur de zéro.
\end{proposition}

\begin{proof}
    Soit \( n\) le minimum tel que \( a^n=0\). En vertu de la formule \eqref{Eqarpurmkbk} nous avons
    \begin{equation}
        1=1-a^n=(1-a)(1+a+\ldots+a^{n-1})=(1+a+\ldots+a^{n-1})(1-a).
    \end{equation}
    La somme \( 1+a+\ldots+a^{n-1}\) est donc un inverse de \( (1-a)\).
\end{proof}

\begin{definition}
    Soit \( \eA\) un anneau et \( a,b\in \eA\). Nous disons que \( d\) est un \( \pgcd\)\index{pgcd} de \( a\) et \( b\) si tout diviseur commun de \( a\) et \( b\) divise \( d\).
\end{definition}

\begin{definition}
    Si \( A\) et \( B\) sont des anneaux, un \defe{morphisme}{morphisme!d'anneaux} est une application \( f\colon A\to B\) telle que pour tout \( x,y\in A\) nous ayons
    \begin{enumerate}
        \item
            \( f(x+y)=f(x)+f(y)\)
        \item
            \( f(xy)=f(x)f(y)\)
        \item
            \( f(1)=1\)
    \end{enumerate}
\end{definition}

Si \( f\) est un morphisme, nous avons \( f(0)=0\) et \( f(x)^{-1}=f(x^{-1})\).

%---------------------------------------------------------------------------------------------------------------------------
\subsection{Binôme de Newton et morphisme de Frobenius}
%---------------------------------------------------------------------------------------------------------------------------

\begin{proposition}     \label{PropBinomFExOiL}
Pour tout $x$, $y\in\eR$ et $n\in\eN$, nous avons
\begin{equation}        \label{EqNewtonB}
    (x+y)^n=\sum_{k=0}^n{n\choose k}x^{n-k}y^k
\end{equation}
où
\begin{equation}
    {n\choose k}=\frac{ n! }{ k!(n-k)! }
\end{equation}
sont les \defe{coefficients binomiaux}{Coefficients binomiaux}.
\end{proposition}

La preuve qui suit provient de \href{http://fr.wikipedia.org/wiki/Formule_du_binôme_de_Newton}{wikipédia}.
\begin{proof}
La preuve se fait par récurrence. La vérification pour $n=0$ et $n=1$ sont faciles. Supposons que la formule \eqref{EqNewtonB} soit vraie pour $n$, et prouvons la pour $n+1$. Nous avons
\begin{equation}        \label{EqBinTrav}
    \begin{aligned}[]
        (x+y)^{n+1} &=(x+y)\cdot  \sum_{k=0}^n{n\choose k}x^{n-k}y^k\\
                &= \sum_{k=0}^n{n\choose k}x^{n-k+1}y^k+\sum_{k=0}^n{n\choose k}x^{n-k}y^{k+1}\\
                &=x^{n+1}+ \sum_{k=1}^n{n\choose k}x^{n-k+1}y^k+\sum_{k=0}^{n-1}{n\choose k}x^{n-k}y^{k+1}+y^{n+1}.
    \end{aligned}
\end{equation}
La seconde grande somme peut être transformée en posant $i=k+1$ :
\begin{equation}
    \sum_{k=0}^{n-1}{n\choose k}x^{n-k}y^{k+1}  =\sum_{i=1}^n{n\choose i-1}x^{n-(i-1)}y^{i-1+1},
\end{equation}
dans lequel nous pouvons immédiatement renommer $i$ par $k$. En remplaçant dans la dernière expression de \eqref{EqBinTrav}, nous trouvons
\begin{equation}
    (x+y)^{n+1}=x^{n+1}+y^{n+1}+\sum_{k=1}^n\left[ {n\choose k}+{n\choose k-1} \right]x^{n-k+1}y^k.
\end{equation}
La thèse découle maintenant de la formule
\begin{equation}
    {n\choose k}+{n\choose k-1}={n+1\choose k}
\end{equation}
qui est vraie parce que
\begin{equation}
    \frac{ n! }{ k!(n-k)! }+\frac{ n! }{ (k-1)(n-k+1)! }=\frac{ n!(n-k+1)+n!k }{ k!(n-k+1)! }=\frac{ n!(n+1) }{  k!(n-k+1)!  },
\end{equation}
par simple mise au même dénominateur.
\end{proof}

\begin{proposition}     \label{Propqrrdem}
    Soit \( \eA\) un anneau commutatif de caractéristique première \( p\). Alors \( \sigma(x)=x^p\) est un automorphisme de l'anneau \( \eA\). Nous avons la formule
    \begin{equation}
        (a+b)^p=a^p+b^p
    \end{equation}
    pour tout \( a,b\in \eA\).
\end{proposition}

\begin{proof}
    Nous utilisons la formule du binôme de la proposition \ref{PropBinomFExOiL} et le fait que les coefficients binomiaux non extrêmes sont divisibles par \( p\) et donc nuls.
\end{proof}

\begin{proposition} \label{PropFrobHAMkTY}
    Soit \( \eA\) un anneau commutatif unitaire de caractéristique \( p\). L'application
    \begin{equation}
        \begin{aligned}
            \Frob_\eA\colon \eA&\to \eA \\
            x&\mapsto x^p 
        \end{aligned}
    \end{equation}
    est un automorphisme d'anneau unitaire.
\end{proposition}
Nous le nommons le \defe{morphisme de Frobenius}{morphisme!Frobenius}\index{Frobenius!morphisme}. Nous utiliserons aussi les itérés du morphisme de Frobenius : \( \Frob^k\colon x\mapsto x^{p^k}\).


%---------------------------------------------------------------------------------------------------------------------------
\subsection{Idéaux dans les anneaux}
%---------------------------------------------------------------------------------------------------------------------------

Soit \( A\), un anneau, \( I\) un idéal bilatère de \( A\). Nous considérons la relation d'équivalence \( x\sim y\) si et seulement si \( x-y\in I\). Dans ce cas, le quotient
\begin{equation}
    A/\sim=A/I
\end{equation}
est un anneau appelé \defe{anneau quotient}{anneau!quotient par un idéal}. La surjection \( A\to A/I\) est un morphisme.

\begin{proposition}
    Soient \( A\) et \( B\) des anneaux et un homomorphisme \( f\colon A\to B\). Nous considérons l'injection canonique \( j\colon f(A)\to B\) et la surjection canonique \( \phi\colon A\to A/\ker f\). Alors il existe un unique isomorphisme
    \begin{equation}
        \tilde f \colon A/\ker f\to f(A)
    \end{equation}
    tel que \( f=j\circ\tilde f\circ\phi\).

    \begin{equation}
        \xymatrix{%
        A \ar[r]^{f}\ar[d]_{\phi}        &   B\ar[d]^{j}\\
           A/\ker f \ar[r]_{\tilde f}   &   f(A)\subset B
           }
    \end{equation}
\end{proposition}

\begin{proposition}     \label{PropIJJIdsousphi}
    Soit \( I\), un idéal de \( A\) et \( \phi\colon A\to A/I\) la surjection canonique. Les idéaux de \( A/I\) sont les \( \phi(J)\) où \( J\) est un idéal de \( A\) contenant \( I\). De plus cette relation est bijective :
    \begin{equation}        \label{EqKbrizu}
        \{ \text{idéaux de \( A\) contenant \( I\)}\}\simeq\{ \text{idéaux de \( R/I\)} \}.
    \end{equation}
\end{proposition}

\begin{proof}
    Si \( I\subset J\) et si \( J \) est un idéal de \( A\), alors \( \phi(J)\) est un idéal dans \( A/I\). En effet un élément de \( \phi(J)\) est de la forme \( \phi(j)\) et un élément de \( A/I\) est de la forme \( \phi(i)\). Leur produit vaut
    \begin{equation}
        \phi(i)\phi(j)=\phi(ij)\in\phi(J).
    \end{equation}
    
    Soit maintenant \( K\), un idéal dans \( A/I\). Soit \( J=\phi^{-1}(K)\). Étant donné qu'un idéal doit contenir \( 0\) (parce qu'un idéal est un groupe pour l'addition), \( [0]\in K\) et par conséquent \( I\subset\phi^{-1}(K)\).
\end{proof}
% TODO : il faudrait dire à peu près ici qu'une des utilités de Z_2 est le groupe modulaire PSL(2,Z)=SL(2,Z)/Z_2

\begin{corollary}
    Les quotients de \( \eZ\) sont \( \eZ_n\)
\end{corollary}

\begin{proof}
    Nous avons déjà vu que les seuls idéaux de \( \eZ\) sont les \( n\eZ\).
\end{proof}

\begin{definition}
    Un sous ensemble \( B\subset A\) d'un anneau est un \defe{sous anneau}{sous anneau} si
    \begin{enumerate}
        \item
            \( 1\in B\)
        \item
            \( B\) est un sous-groupe pour l'addition
        \item
            \( B\) est stable pour la multiplication.
    \end{enumerate}
    Un sous ensemble \( I\subset A\) est un \defe{idéal}{idéal!dans un anneau} à gauche si
    \begin{enumerate}
        \item
            \( I\) est un sous-groupe pour l'addition
        \item
            si \( aI\subset I\) pour tout \( A\in A\).
    \end{enumerate}
\end{definition}
Lorsqu'un ensemble est idéal à gauche et à droite, nous disons que c'est un \defe{idéal bilatère}{idéal!bilatère}. Lorsque nous parlons d'idéal sans précisions, nous parlons d'idéal bilatère.

\begin{remark}
    Un idéal n'est pas toujours un anneau parce que l'identité pourrait manquer. Un idéal qui contient l'identité est l'anneau complet.
\end{remark}

\begin{example}
    L'ensemble \( 2\eZ\) est un idéal de \( \eZ\). Tous les idéaux de \( \eZ\) sont de la forme \( n\eZ\). En effet en vertu de la proposition \ref{PropSsgpZestnZ}, les seule sous-groupes de \( \eZ\) (en tant que groupe additif) sont les \( n\eZ\).
\end{example}



\begin{proposition}     \label{PropZpintssiprempUzn}
    Soit \( n\geq 2\) un entier et \( \phi\colon \eZ\to \eZ_n\), la surjection canonique. Nous noterons \( \tilde a=\phi(a)\). Alors
    \begin{equation}
        U(\eZ_n)=\phi(P_n)=\{ \tilde x\tq 0\leq x\leq n\tq\pgcd(x,n)=1 \}.
    \end{equation}
    où \( P_n\) est l'ensemble décrit par l'équation \eqref{EqDefPnEntierldeost}. En particulier, \( \Card\big( U(\eZ_n) \big)=\varphi(n)\).

\end{proposition}

\begin{proof}
    Soit \( 0\leq x\leq n\) tel que \( \pgcd(x,n)=1\). Il existe donc \( p,q\in\eZ\) tels que \( px+qn=1\). En passant aux classes,
    \begin{equation}
        \tilde p\tilde x=\tilde 1,
    \end{equation}
    donc \( \tilde p\) est l'inverse de \( \tilde x\). Cela prouve que \( \phi(P_n)\subset U(\eZ_n)\).

    Nous prouvons maintenant l'inclusion inverse. Soit \( \tilde x\) et \( \tilde y\) inverses l'un de l'autre : $\tilde x\tilde y=\tilde 1$. Il existe donc \( q\in\eZ\) tel que \( xy-qn=1\), ce qui prouve que \( \pgcd(x,n)=1\).

\end{proof}

\begin{corollary}   \label{CorZnInternprem}
    L'anneau \( \eZ_n\) est intègre si et seulement si \( n\) est premier.
\end{corollary}

\begin{proof}
    Si \( n\) est premier, tous les éléments de \( \eZ_n\) sont inversibles parce que tous les éléments rentrent dans \( \phi(P_n)\). Donc \( \eZ_n\) est intègre.

    Si \( n\) n'est pas premier, il existe \( p,q\in\eN^*\) tels que \( pq=n\). Dans ce cas au niveau des classes nous avons \( \tilde p\tilde q=0\) avec \( \tilde p\neq 0\neq\tilde q\), ce qui montre que \( \eZ_n\) a des diviseurs de zéro et n'est pas intègre.
\end{proof}

%---------------------------------------------------------------------------------------------------------------------------
\subsection{Caractéristique}
%---------------------------------------------------------------------------------------------------------------------------

L'application 
\begin{equation}
    \begin{aligned}
        \mu\colon \eZ&\to A \\
        n&\mapsto n\cdot 1_A 
    \end{aligned}
\end{equation}
est un morphisme d'anneaux. Le noyau de \( \mu\) étant un sous-groupe de \( \eZ\), il existe un et un seul \( p\in\eZ\) tel que \( \ker\mu=p\eZ\). Ce \( p\) est la \defe{caractéristique}{caractéristique!d'un anneau} de \( A\).

Par exemple la caractéristique que \( \eQ\) est zéro parce qu'aucun multiple de l'unité n'est nul.

À propos de diagonalisation en caractéristique \( 2\), voir l'exemple \ref{ExewINgYo}.

\begin{lemma}
    Si \( A\) est de caractéristique nulle, alors \( A\) est infini.
\end{lemma}

\begin{proof}
    En effet, \( \ker\mu=0\) implique que \( n1_A\neq  m1_A\) et par conséquent \( A\) est infini.
\end{proof}

\begin{lemma}       \label{LemHmDaYH}
    Si \( p\) est la caractéristique de l'anneau \( A\), alors nous avons l'isomorphisme d'anneaux
    \begin{equation}
         \eZ 1_A\simeq\eZ/p\eZ.
    \end{equation}
\end{lemma}

\begin{proof}
    L'isomorphisme est donné par l'application \( n1_A\mapsto \phi(n)\) si \( \phi\) est la projection canonique \( \eZ\to \eZ_p\).
\end{proof}

\begin{lemma}       \label{LemCaractIntergernbrcartpre}
    La caractéristique d'un anneau intègre est zéro ou un nombre premier.
\end{lemma}

\begin{proof}
    Si \( A\) est intègre, alors \( \eZ 1_A\) est intègre (a fortiori), et \( \eZ_p\) est intègre parce qu'il est isomorphe à \( \eZ A_A\). Mais nous savons que \( \eZ_p\) est intègre si et seulement si \( p\) est premier (proposition \ref{CorZnInternprem}).
\end{proof}

\begin{example}
    Il existe des corps dont la caractéristique n'est pas égale au cardinal (contrairement à ce que laisserait penser l'exemple des \( \eZ/p\eZ\)). En effet les matrices \( n\times n\) inversibles sur \( \eF_{3}\) forment un corps qui n'est pas de cardinal trois alors que la caractéristique est \( 3\) :
    \begin{equation}
        \begin{pmatrix}
            1    &       \\ 
                &   1    
            \end{pmatrix}+\begin{pmatrix}
                1    &       \\ 
                    &   1    
                \end{pmatrix}+\begin{pmatrix}
                    1    &       \\ 
                        &   1    
                \end{pmatrix}=0.
    \end{equation}
\end{example}

\begin{example}
    Si \( \eK\) est un corps de caractéristique \( 2\), alors l'égalité \( x=-x\) n'implique pas \( x=0\), vu que \( 2x=0\) est vérifiée pour tout \( x\). Cela se répercute sur un certain nombre de résultats. En caractéristique deux, une forme antisymétrique n'est pas toujours alternée. Voir le lemme \ref{LemHiHNey}.
\end{example}

\begin{proposition}     \label{PropGExaUK}
    La caractéristique d'un anneau fini divise son cardinal.
\end{proposition}

\begin{proof}
    Si \( \eA\) est un anneau, le groupe \( \eZ\) agit sur \( \eA\) par
    \begin{equation}
        n\cdot a=a+n1_A.
    \end{equation}
    Chaque orbite de cette action est de la forme
    \begin{equation}
        \mO_a=\{ a+n1_A\tq n=0,\ldots, p-1 \}
    \end{equation}
    où \( p\) est la caractéristique de \( \eA\). Les orbites ont \( p\) éléments et forment une partition de \( \eA\), donc le cardinal de \( \eA\) est un multiple de \( p\).
\end{proof}

L'ensemble typique de caractéristique \( p\) est \( \eF_p=\eZ/p\eZ\).
\begin{example}
    Soit à factoriser \( X^p-1\) dans \( \eF_p\). Grâce au morphisme de Frobenius, nous avons immédiatement
    \begin{equation}
        X^p-1=(X-1)^p.
    \end{equation}
\end{example}

%+++++++++++++++++++++++++++++++++++++++++++++++++++++++++++++++++++++++++++++++++++++++++++++++++++++++++++++++++++++++++++
\section{Modules}
%+++++++++++++++++++++++++++++++++++++++++++++++++++++++++++++++++++++++++++++++++++++++++++++++++++++++++++++++++++++++++++

Si \( \eA\) est un anneau et si \( (\modE,+)\) est un groupe commutatif, nous disons que \( \modE\) est un \defe{\wikipedia{fr}{Module_sur_un_anneau}{module}}{module!sur un anneau} à gauche sur \( \eA\) si nous avons une application \( \eA\times M\to M\) notée \( a\cdot x\) telle que
\begin{enumerate}
    \item
       $\alpha\cdot(x + y) = a\cdot x + a\cdot y$  (distributivité de $\cdot$ par rapport à l'addition dans $M$)
   \item $(a + b) \cdot x = a \cdot x + b \cdot x$ (distributivité de $\cdot$ par rapport à l'addition dans \( \eA\)). Remarque :  la loi $+$ du membre de gauche est celle de l'anneau $A$ et la loi $+$ du membre de droite est celle du groupe $M$
   \item $(ab) \cdot x = a \cdot (b \cdot x$)
   \item $1 \cdot x = x$ 
\end{enumerate}

Soit \( \modE\) un \( A\)-module et \( x=(x_i)_{i\in I}\) une famille d'éléments de \( \modE\), paramétrée par l'ensemble \( I\). Nous considérons l'application
\begin{equation}
    \begin{aligned}
        \mu_x\colon A^{(I)}&\to \modE \\
        (a_i)_{i\in I}&\mapsto \sum_{i\in I}a_ix_i.
    \end{aligned}
\end{equation}
Ici \( A^{(I)}\) désigne l'ensemble de toutes les applications \( I\to A\) de support fini.  

\begin{definition}      \label{DefBasePouyKj}
    À l'instar des espaces vectoriels, les modules ont une notion de partie libre, génératrice et de bases :
    \begin{enumerate}
        \item
            Si \( \mu_x\) est surjective, nous disons que \( x\) est une partie \defe{génératrice}{génératrice!partir d'un module}.
        \item
            Si \( \mu_x\) est injective, nous disons que la partie \( x\) est \defe{libre}{libre!partie d'un module}.
        \item
            Si \( \mu_x\) est bijective, nous disons que la partie \( x\) est une \defe{base}{base!d'un module}.
    \end{enumerate}
\end{definition}

\begin{definition}
    Soit \( \modE\) un module sur un anneau commutatif \( A\). Un \defe{projecteur}{projecteur!dans un module} est une application linéaire \( p\colon \modE\to \modE\) telle que \( p^2=p\).

    Une famille \( (p_i)_{i\in I}\) sur \( \modE\) est \defe{orthogonale}{orthogonal!famille de projecteurs} si \( p_i\circ p_j=0\) pour tout \( i\neq j\). La famille est \defe{complète}{complète!famille de projecteurs} si \( \sum_{i\in I}p_i=\mtu\).
\end{definition}

\begin{theorem}     \label{ThoProjModpAlsUR}
    Soient des sous modules \( \modE_1,\ldots,\modE_n\) du module \( \modE\) tels que \( \modE=\modE_1\oplus\ldots\oplus\modE_n\). Les applications \( p_i\) définies par
    \begin{equation}
        p_i(x_1+x_n)=x_i
    \end{equation}
    forment une famille orthogonale de projecteurs et \( p_1+\ldots +p_n=\id\).

    Inversement, si \( (p_1,\ldots, p_n)\) est une famille orthogonale de projecteurs dans un module \( \modE\) tel que \( \sum_{i=1}^np_i=\id\), alors
    \begin{equation}
        \modE=\bigoplus_{i=1}^np_i(\modE).
    \end{equation}
\end{theorem}

Un sous-ensemble \( \modF\subset\modE\) est un \defe{sous-module}{sous-!module} si \( (\modF,+)\) est un sous-groupe de \( (\modE,+)\) et si \( a\cdot x\in \modF\) pour tout \( x\in \modF\) et pour tout \( a\in \eA\).

\begin{example}
    Un anneau \( \eA\) est lui-même un \( \eA\)-module et ses sous-modules sont les idéaux.
\end{example}

Un module est \defe{simple}{simple!module}\index{module!simple} ou \defe{irréductible}{irréductible!module}\index{module!irréductible} si il n'a pas d'autre sous-modules que \( \{ 0 \}\) et lui-même. Un module est \defe{indécomposable}{indécomposable!module}\index{module!indécomposable} si il ne peut pas être écrit comme somme directe de sous-modules.

Un module simple est a fortiori indécomposable. L'inverse n'est pas vrai comme le montre l'exemple suivant.

\begin{example}
    Soit \( \modE=\eC[X]/(X^2)\) vu comme \( \eC[X]\)-module. C'est le \( \eC[X]\)-module des polynômes de la forme \( aX+b\) avec \( a,b\in \eC\). L'ensemble des polynômes de la forme \( aX\) est un sous-module. Le module \( \modE\) n'est donc pas simple. Il est cependant indécomposable parce que \( \{ aX \}\) est le seul sous-module non trivial. En effet si \( \modF\) est un sous-module de \( \modE\) contenant \( aX+b\) avec \( b\neq 0\), alors \( \modF\) contient \( X(aX+b)=bX\) et donc contient tout \( \modE\).
\end{example}

%+++++++++++++++++++++++++++++++++++++++++++++++++++++++++++++++++++++++++++++++++++++++++++++++++++++++++++++++++++++++++++
\section{Anneau intègre}
%+++++++++++++++++++++++++++++++++++++++++++++++++++++++++++++++++++++++++++++++++++++++++++++++++++++++++++++++++++++++++++

Un élément \( a\neq 0\) est un \defe{diviseur de zéro à gauche}{diviseur!de zéro} si il existe \( x\neq 0\) tel que $xa=0$. L'élément \( a\) est un diviseur de zéro \defe{à droite}{diviseur!de zéro à droite} si il existe \( b\) tel que \( ab=0\). Un anneau est \defe{intègre}{intègre!anneau}\index{anneau!intègre} si il est non nul et ne possède pas de diviseurs de zéro.

Autrement dit, un anneau intègre est un anneau qui possède la propriété du produit nul : si \( ab=0\), alors soit \( a\) soit \( b\) est nul.

\begin{example}
    L'ensemble \( \eZ\) avec les opérations usuelles est un anneau intègre.
\end{example}

\begin{example}
    L'anneau \( \eZ/6\eZ\) n'est pas intègre parce que \( 3\cdot 2=0\) alors que ni \( 3\) ni \( 2\) ne sont nuls.
\end{example}

\begin{example}
    Nous verrons au théorème \ref{ThoBUEDrJ} que si \( \eA\) est intègre, alors l'anneau des polynômes sur \( \eA\) est intègre.
\end{example}

\begin{lemma}\label{LemRmVTRq}
    Si \( \eA\) est un anneau intègre et si \( a,b\in \eA\) sont tels que \( a\divides b\) et \( b\divides a\), alors il existe un inversible \( u\in \eA\) tel que \( a=ub\).
\end{lemma}

\begin{proof}
    Les hypothèses à propos de la divisibilité nous indiquent que \( a=xb\) et \( b=ya\) pour certains \( x,y\in \eA\). Du coup,
    \begin{equation}
        b(1-yx)=0.
    \end{equation}
    Étant donné que \( \eA\) est intègre, cela montre que \( b=0\) ou \( 1-yx=0\). Si \( b=0\) nous avons immédiatement \( a=0\) et le lemme est prouvé. Si au contraire \( yx=1\), c'est que \( y\) et \( x\) sont inversibles et inverses l'un de l'autre.
\end{proof}

\begin{example}     \label{ExybCZyl}
    Si \( \eA\) est un anneau intègre, l'anneau \( \eA[X]\) des polynômes sur \( \eA\) est également intègre. En effet si \( P\) et \( Q\) sont deux polynômes non nuls, alors le coefficient du terme de plus haut degré de \( PQ\) est donné par le produit des coefficients de plus haut degré de \( P\) et \( Q\) qui est non nul parce que \( \eA\) est intègre.
\end{example}

%---------------------------------------------------------------------------------------------------------------------------
\subsection{PGCD et PPCM}
%---------------------------------------------------------------------------------------------------------------------------
Source : \cite{XPXxPl}.

Le théorème de Bézout aura lieu dans les anneaux principaux, corollaire \ref{CorimHyXy}.

Dans un anneau intègre, la relation de divisibilité s'exprime en termes d'idéaux par
\begin{equation}
    a\divides b\Leftrightarrow (b)\subset (a).
\end{equation}
Donc la divisibilité devient en réalité une relation d'ordre dont nous pouvons chercher un maximum et un minimum. Si \( S\) est une partie de \( \eA\), nous notons \( a\divides S\) pour exprimer que \( a\divides x\) pour tout \( x\in S\).

\begin{definition}\label{DefrYwbct}
    Soit \( \eA\), un anneau intègre et \( S\subset \eA\). Nous disons que \( \delta\in \eA\) est un \defe{PGCD}{PGCD!dans un anneau intègre} de \( S\) si
    \begin{enumerate}
        \item
            \( \delta\divides S\)
        \item
            si \( d\divides S\) alors\footnote{Il me semble qu'à ce niveau il y a une faute de frappe dans \cite{XPXxPl}.} \( d\divides \delta\).
    \end{enumerate}
    Nous disons que \( \mu\in \eA\) est un \defe{PPCM}{PPCM!dans un anneau intègre} de \( S\) si
    \begin{enumerate}
        \item
            \( S\divides \mu\),
        \item
            si \( S\divides m\), alors \( \mu\divides m\).
    \end{enumerate}
\end{definition}
Notons qu'il n'y a en général pas unicité du PGCD ou du PPCM d'un ensemble.

\begin{lemma}
    Soit \( \eA\) un anneau intègre et \( S\subset \eA\). Si \( \delta\) est un PGCD de \( S\), alors l'ensemble des PGCD de \( S\) est la classe d'association de \( \delta\).

    De la même façon si \( \mu\) est un PPCM de \( S\), alors l'ensemble des PPCM de \( S\) est la classe d'association de \( \mu\).
\end{lemma}

\begin{proof}
    Soit \( \delta\) un PGCD de \( S\) et \( u\) un inversible dans \( \eA\). Si \( x\in S\) nous avons \( \delta\divides x\) et donc \( x=a\delta\). Par conséquent \( x=au^{-1}u\delta\) et donc \( u\delta\) divise \( x\). De la même manière, si \( d\) divise \( x\) pour tout \( x\in S\), alors \( d\) divise \( \delta\) et donc \( \delta=ad\) et \( u\delta=aud\), ce qui signifie que \( d\) divise \( u\delta\).

    Dans l'autre sens nous devons prouver que si \( \delta'\) est une autre PGCD de \( S\), alors il existe un inversible \( u\in \eA\) tel que \( \delta'=u\delta\). Vu que \( \delta'\) divise \( x\) pour tout \( x\in S\), nous avons \( \delta'\divides \delta\), et symétriquement nous trouvons \( \delta\divides\delta'\). Par conséquent (lemme \ref{LemRmVTRq}), il existe un inversible \( u\) tel que \( \delta=u\delta'\).

    Le même type de raisonnement tient pour le PPCM.
\end{proof}

Si \( \delta\) est un PGCD de \( S\), nous dirons \emph{par abus de langage} que \( \delta\) est \emph{le} PGCD de \( S\), gardant en tête qu'en réalité toute sa classe d'association est PGCD. Nous noterons aussi, toujours par abus que \( \delta=\pgcd(S)\).

\begin{remark}
    La classe d'association d'un élément n'est pas toujours très grande. Les inversibles dans \( \eZ\) étant seulement \( \pm 1\), nous pouvons obtenir l'unicité du PGCD et du PPCM en imposant qu'ils soient positifs.

    Pour les polynômes, nous obtenons l'unicité en demandant que le PGCD soit unitaire.

    Dans les cas pratiques, il y a donc en réalité peu d'ambiguïté à parler du PGCD ou du PPCM d'un ensemble.
\end{remark}

%+++++++++++++++++++++++++++++++++++++++++++++++++++++++++++++++++++++++++++++++++++++++++++++++++++++++++++++++++++++++++++
\section{Anneau factoriel}
%+++++++++++++++++++++++++++++++++++++++++++++++++++++++++++++++++++++++++++++++++++++++++++++++++++++++++++++++++++++++++++

\begin{definition}  \label{DefrXUixs}
    On dit que les éléments \( a\) et \( b\) d'un anneau sont \defe{associés}{associé}\index{éléments!associés dans un anneau} si il existe un élément \( u\) inversible dans \( \eA\) tel que \( a=ub\).
\end{definition}

\begin{definition}  \label{DeirredBDhQfA}
    Soit \( \eA\) un anneau commutatif intègre. Un élément \( a\in\eA\) est \defe{irréductible}{irréductible!dans un anneau} si \( a\) n'est pas inversible, mais si \( a=xy\), alors soit \( x\) soit \( y\) est inversible. Nous notons \( U(\eA)\) l'ensemble des éléments inversibles de \( \eA\).
\end{definition}

\begin{remark}
    Un corps n'a pas d'éléments irréductibles parce qu'à part zéro tous les éléments sont inversibles alors que \( 0\) n'est certainement pas irréductible vu que \( 0=0\cdot 0\).
\end{remark}

\begin{example}
    Dans l'anneau \( \eZ\), le éléments irréductibles sont les nombres premiers. En effet les seuls inversibles de \( \eZ\) sont \( \pm 1\). Si \( p\) est premier et \( p=ab\) avec \( a,b\in \eZ\), alors nous avons soit \( a=\pm 1\) soit \( b=\pm 1\).
\end{example}

\begin{definition}
    Un anneau commutatif unitaire \( \eA\) est \defe{factoriel}{factoriel!anneau}\index{anneau!factoriel} si il vérifie les propriétés suivantes.
    \begin{enumerate}
        \item
            L'anneau \( \eA\) est intègre (pas de diviseurs de zéro).
        \item
            Si \( a\in \eA\) est non nul et non inversible alors il admet une décomposition \( a=p_1\ldots p_k\) où les \( p_i\) sont irréductibles.
        \item
            Si \( a=q_1\ldots q_m\) est une autre décomposition de \( a\) en irréductibles, alors \( m=k\) et il existe une permutation \( \sigma\in S_k\) telle que \( p_i\) et \( q_{\sigma(i)}\) soient associés.
    \end{enumerate}
\end{definition}

Un anneau factoriel permet de définir le \( \pgcd\) et le \( \ppcm\) de nombres. Soit une famille \( \{ a_n \}\) d'éléments de \( \eA\) qui se décomposent en irréductibles comme
\begin{equation}
    a_i=\prod_{k}p_k^{\alpha_{k,i}}.
\end{equation}
Nous définissons
\begin{equation}
    \pgcd\{ a_n \}=\prod_kp_k^{min_i\{ \alpha_{k,i} \}}.
\end{equation}
\begin{proposition}
    L'élément \( \pgcd\{ a_n \}\) est l'unique diviseur commun des \( a_i\) à être un multiple des autres.
\end{proposition}
% TODO : une preuve de ça.

Nous définissons aussi
\begin{equation}
    \ppcm\{ a_i \}=\prod_kp_k^{\max_i\{ \alpha_{k,i} \}}.
\end{equation}
Un anneau factoriel a une relation de préordre partiel\index{ordre!sur un anneau factoriel} donnée par \( a<b\) si \( a\) divise \( b\). En termes d'idéaux, cela donne l'ordre inverse de celui de l'inclusion : \( a<b\) si et seulement si \( (b)\subset (a)\).

\begin{example}
    L'anneau \( \eZ[i\sqrt{3}]\) n'est pas factoriel parce que
    \begin{equation}
        4=2\cdot 2=(1+i\sqrt{3})(1-i\sqrt{3})
    \end{equation}
    donnent deux décompositions distinctes de \( 4\) en irréductibles.
\end{example}
Nous allons voir dans l'exemple \ref{ExeDufyZI} que \( \eZ[i\sqrt{2}]\) est factoriel parce qu'il sera euclidien.

%+++++++++++++++++++++++++++++++++++++++++++++++++++++++++++++++++++++++++++++++++++++++++++++++++++++++++++++++++++++++++++
\section{Anneau principal}
%+++++++++++++++++++++++++++++++++++++++++++++++++++++++++++++++++++++++++++++++++++++++++++++++++++++++++++++++++++++++++++

Nous parlons de l'idéal des polynôme annulateurs dans le théorème \ref{ThoCCHkoU}.

\begin{definition}      \label{DefIdPrinpuMrbOq}
    Un idéal \( I\) dans \(\eA\) est \defe{principal à gauche}{idéal!principal!à gauche} si il existe \( a\in I\) tel que \( I=\eA a\). Il est \defe{principal à droite}{idéal!principal!à droite} si il existe \( a\in I\) tel que \( I=a\eA\). Nous disons qu'il est \defe{principal}{principal!idéal} si il est principal à gauche et à droite.

    Un anneau commutatif intègre est \defe{principal}{principal!anneau} si tous ses idéaux sont principaux.
\end{definition}

Un idéal \( I\) dans l'anneau \( \eA\) est \defe{maximal}{maximal!idéal}\index{idéal!maximal} si les seuls idéaux de \( \eA\) contenants \( I\) sont \( I\) et \( \eA\).


\begin{definition}
    Nous disons qu'un idéal \( I\) dans \( \eA\) est \defe{premier}{premier!idéal} si \( \eA\) est un anneau commutatif intègre et si \( A/I\) est intègre.
\end{definition}

\begin{proposition} \label{PropomqcGe}
    Soit \( \eA\) un anneau principal qui n'est pas un corps. Pour un idéal \( I\subset \eA\), les conditions suivantes sont équivalentes :
    \begin{enumerate}
        \item
            \( I\) est un idéal maximum;
        \item
            \( I\) est un idéal premier non nul;
        \item
            il existe \( p\) irréductible dans \( \eA\) tel que \( I=(p)\).
    \end{enumerate}
\end{proposition}
Ici, \( (p)\) est l'idéal dans \( \eA\) engendré par \( p\), c'est à dire \( p\eA\)\nomenclature[A]{\( (p)\)}{idéal engendré par \( p\)}.

\begin{proposition}
    Un idéal \( I\) dans \( \eA\) est premier si et seulement si \( I\) est strictement inclus dans \( \eA\) et si pour tout \( a,b\in\eA\) tels que \( ab\in I\) nous avons \( a\in I\) ou \( b\in I\).
\end{proposition}

\begin{proposition}     \label{PropoTMMXCx}
    Si \( \eA\) est un anneau principal et si \( p\) est irréductible, alors \( \eA/p\) est un corps.
\end{proposition}

\begin{example}
    L'anneau \( \eZ\) est principal parce que ses seuls idéaux sont les \( n\eZ\) qui sont principaux : \( n\eZ\) est engendré par \( n\).
\end{example}

\begin{example}
    Les anneaux \( \eZ/n\eZ\) sont principaux. En effet les idéaux de \( \eZ/n\eZ\) seraient par la proposition \ref{PropIJJIdsousphi} des quotients d'idéaux de \( \eZ\) par \( (n)\). Donc les idéaux de \( \eZ/n\eZ\) sont les anneaux \( (\eZ/m\eZ\) avec \( m\) divisant \( n\).

    Par exemple si \( I=4\eZ\), on peut considérer l'idéal \( J=2\eZ\) qui contient \( 4\eZ\). Donc \( \eZ/2\eZ\) est un idéal de \( \eZ/4\eZ\).
\end{example}

\begin{theorem}\index{théorème!chinois!anneau principal}        \label{ThofPXwiM}
    Si \( \eA\) est un anneau principal et si \( p\) et \( q\) sont premiers entre eux dans \( \eA\), alors on a l'isomorphisme d'anneaux
    \begin{equation}
        \eA/pq\eA\simeq \eA/p\eA\times \eA/q\eA.
    \end{equation}
\end{theorem}
% TODO : trouver un preuve. Je parie que recopier la même que celle dans Z fonctionne très bien.

%---------------------------------------------------------------------------------------------------------------------------
\subsection{Bézout}
%---------------------------------------------------------------------------------------------------------------------------

Source : \cite{XPXxPl}.

\begin{theorem}
    Toute partie \( S\) d'un anneau principal admet un PGCD et un PPCM. De plus
    \begin{equation}
        \begin{aligned}[]
            \delta=\pgcd(S)\Leftrightarrow (\delta)=\sum_{s\in S}(s)
            \mu=\ppcm(S)\Leftrightarrow (\mu)=\bigcap_{s\in S}(s)
        \end{aligned}
    \end{equation}
\end{theorem}

\begin{proof}
    Vu que l'anneau \( \eA\) est principal, tous ses idéaux sont principaux et donc engendrés par un seul élément. En particulier il existe \( \delta,\mu\in \eA\) tels que
    \begin{subequations}
        \begin{align}
            (\delta)&=\sum_{s\in S}(s)\\
            (\mu)&=\bigcap_{s\in S}(s)
        \end{align}
    \end{subequations}
    \begin{subproof}
    \item[PGCD]
        Montrons ce que \( \delta\) est un PGCD de \( S\). Pour tout \( x\in S\), nous avons \( (x)\subset (\delta)\), et donc \( \delta\divides x\). Par ailleurs si \( d\divides x\) pour tout \( x\in S\), nous avons \( (x)\subset (d)\) et donc 
        \begin{equation}
            \sum_{x\in S}(x)\subset (d),
        \end{equation}
        puis \( (\delta)\subset (d)\) et finalement \( d\divides \delta\).
        \item[PPCM]
            Si \( x\in S\) nous avons \( (\mu)\subset (x)\) et donc \( x\divides \mu\). D'autre part si \( x\divides m\) pour tout \( x\in S\), alors \( (m)\subset (x)\) et donc \( (m)\subset(\mu)\), finalement \( \mu\divides m\).
    \end{subproof}
\end{proof}

Nous disons que deux éléments d'un anneau principal sont \defe{premiers entre eux}{premier!deux éléments d'un anneau principal} si leur PGCD est \( 1\).

\begin{corollary}[Théorème de Bézout\cite{XPXxPl}]\index{Bézout!anneau principal}\label{CorimHyXy}
    Soit \( \eA\) un anneau principal. Deux éléments \( a,b\in \eA\) sont premiers entre eux si et seulement si il existe un couple \( u,v\in \eA\) tel que
    \begin{equation}
        ua+vb=1.
    \end{equation}
    À la place de \( 1\) on aurait pu écrire n'importe quel inversible.
\end{corollary}

\begin{proof}
    Pour cette preuve, nous allons écrire \( \pgcd(a,b)\) l'ensemble de PGCD de \( a\) et \( b\), c'est à dire la classe d'association d'un PGCD.

    Si \( a\) et \( b\) sont premiers entre eux, alors
    \begin{equation}
        1\in\pgcd(a,b)=\sum_{x=a,b}(x)=(a)+(b).
    \end{equation}
    
    À l'inverse, si nous avons \( ua+vb=1\), alors \( 1\in (a)+(b)\), mais vu que \( (a)+(b)\) est un idéal principal, \( (1)=(a)+(b)\) et donc \( 1\in \pgcd(a,b)\).
\end{proof}

%---------------------------------------------------------------------------------------------------------------------------
\subsection{Anneau noetherien}
%---------------------------------------------------------------------------------------------------------------------------

Un anneau est dit \defe{noetherien}{anneau!noetherien} si toute suite croissante d'idéaux est stationnaire (à partir d'un certain rang). Montrer que tout anneau principal est noetherien est le premier pas pour montrer que tout anneau principal est factoriel.

\begin{lemma}
    Tout anneau principal est noetherien.
\end{lemma}

\begin{proof}
    Soit \( (J_n)\) une suite croissante d'idéaux et \( J\) la réunion. L'ensemble \( J\) est encore un idéal parce que les \( J_i\) sont emboités. Étant donné que l'idéal est principal nous pouvons prendre \( a\in J\) tel que \( J=(a)\). Il existe \( N\) tel que \( a\in J_N\). Alors pour tout \( n\geq N\) nous avons
    \begin{equation}
        J\subset J_N\subset J_n\subset J.
    \end{equation}
    La première inclusion est le fait que \( J=(a)\) et \( a\in J_N\). La seconde est la croissance des idéaux et la troisième est le fait que \( J\) est une union. Par conséquent pour tout \( n\geq N\) nous avons \( J_N=J_n=J\). La suite est par conséquent stationnaire.
\end{proof}

\begin{theorem}[\cite{FSwlnf}]
    Tout anneau principal est factoriel.
\end{theorem}

%+++++++++++++++++++++++++++++++++++++++++++++++++++++++++++++++++++++++++++++++++++++++++++++++++++++++++++++++++++++++++++
\section{Anneau euclidien}
%+++++++++++++++++++++++++++++++++++++++++++++++++++++++++++++++++++++++++++++++++++++++++++++++++++++++++++++++++++++++++++

\begin{definition}[\wikipedia{fr}{Anneau_euclidien}{Wikipédia}]
    Soit \( \eA\) un anneau intègre. Un \defe{stathme euclidien}{stathme euclidien} sur \( \eA\) est une application \( \alpha\colon \eA\setminus\{ 0 \}\to \eN\) tel que
    \begin{enumerate}
        \item
            \( \forall a,b\in \eA\setminus\{ 0 \}\), il existe \( q,r\in \eA\) tel que
            \begin{equation}
                a=bq+r
            \end{equation}
            et \( \alpha(r)<\alpha(b)\).
        \item
            Pour tout \( a,b\in \eA\setminus\{ 0 \}\), \( \alpha(b)\leq \alpha(ab)\).
    \end{enumerate}
    Un anneau est \defe{euclidien}{euclidien!anneau} si il accepte un stathme euclidien.
\end{definition}
Le stathme est la fonction qui donne le «degré» à utiliser dans la division euclidienne. La contrainte est que le degré du reste soit plus petit que le degré du dividende.

\begin{example} \label{ExwqlCwvV}
    Le stathme de \( \eN\) pour la division euclidienne usuelle est \( \alpha(n)=n\). Si \( a,b\in \eN\) nous écrivons
    \begin{equation}
        a=bq+r
    \end{equation}
    où \( q\) est l'entier le plus proche \emph{inférieur} à \( a/b\) (on veut que le reste soit positif) et \( r=a-bq\). Nous avons donc
    \begin{equation}
        r-b=a-b(q+1)<a-b\frac{ a }{ b }=0,
    \end{equation}
    ce qui montre que \( r<b\).
\end{example}

\begin{proposition}[\wikipedia{fr}{Anneau_euclidien}{Wikipédia}]\label{Propkllxnv}
    Un anneau euclidien est principal.
\end{proposition}

\begin{proof}
    Soit \( \eA\) un anneau principal et \( \alpha\) un stathme sur \( \eA\). Nous considérons un idéal \( I\) non nul de \( \eA\). Nous devons montrer que \( I\) est généré par un élément. En l'occurrence nous allons montrer que l'élément \( a\in I\setminus\{ 0 \}\) qui minimise \( \alpha(a)\) va générer. Soit \( x\in I\). Par construction, il existe \( q,r\in \eA\) tels que \( a=aq+r\) avec \( r=0\) ou \( \alpha(r)<\alpha(a)\). Étant donné que \( x,a\in I\), \( r\in I\). Si \( r\neq 0\), alors \( r\) contredirait la minimalité de \( \alpha(a)\). Donc \( r=0\) et \( x=aq\), ce qui signifie que \( I\) est principal.
\end{proof}

\begin{example} \label{ExeDufyZI}
    Prouvons que \( \eZ[i\sqrt{2}]\) est une anneau euclidien. Pour cela nous démontrons que
    \begin{equation}
        \begin{aligned}
            N\colon \eZ[i\sqrt{2}]&\to \eN \\
            a+bi\sqrt{2}&\mapsto a^2+2b^2 
        \end{aligned}
    \end{equation}
    est un stathme euclidien.    

    Soient \( z=a+bi\sqrt{2}\), \( t=a'+b'i\sqrt{2}\). Nous cherchons \( q\) et \( r\) tels que la division euclidienne s'écrive \( z=qt+r\). Soient \( \alpha,\beta\in \eQ\) tels que 
    \begin{equation}
        \frac{ z }{ t }=\alpha+\beta i\sqrt{2}.
    \end{equation}
    Nous désignons par \( \alpha+\epsilon_1\) et \( \beta+\epsilon_2\) les entiers les plus proches de \( \alpha\) et \( b\). Nous avons \( | \alpha |,| \beta |\leq \frac{ 1 }{2}\). Nous posons alors naturellement 
    \begin{equation}
        q=(\alpha+\epsilon_1)+(\beta+\epsilon_2)i\sqrt{2}
    \end{equation}
    et nous calculons \( r=z-qt\) :
    \begin{equation}
        2b'\epsilon_2-a'\epsilon_1+i\sqrt{2}\big( \epsilon_1b'-a'\epsilon_2 \big).
    \end{equation}
    Nous trouvons 
    \begin{equation}
        N(r)=a'^2\epsilon_1^2+4b'^2\epsilon_2^2+2a'^2\epsilon_1^2+2b'^2\epsilon_2^2\leq \frac{ 3 }{ 4 }a'^2+\frac{ 3 }{2}b'^2.
    \end{equation}
    D'autre part \( N(t)=a'^2+2b'^2\), et nous avons donc bien \( N(r)<N(t)\).

    En ce qui concerne la seconde propriété du stathme, un petit calcul montre que
    \begin{equation}
        N(zt)=(a^2+2b^2)(a'^2+2b'^2),
    \end{equation}
    et tant que \( t\neq 0\) nous avons bien \( N(zt)>N(z)\).
\end{example}

Notons en particulier que \( \eZ[i\sqrt{2}]\) est factoriel et principal.

\begin{example} \label{ExluqIkE}
    Décomposition en facteurs irréductibles dans \( \eZ[i\sqrt{2}]\). Les éléments inversibles de \( \eZ[i\sqrt{2}]\) sont \( \pm 1\), donc deux éléments \( a\) et \( b\) sont associés (définition \ref{DefrXUixs}) si et seulement si \( a=\pm b\).

    De plus si \( p\) est irréductible, alors \( -p\) est irréductible. Les éléments irréductibles de \( \eZ[i\sqrt{2}]\) arrivent donc par pairs d'éléments associés. Soit \( \{ p_i \}\) une sélection de un élément irréductible parmi chaque paire. Tout élément \( x\) de \( \eZ[i\sqrt{2}]\) peut alors être écrit \( x=\pm p_1^{\alpha_1}\ldots p_n^{\alpha_n}\). Ce fait va être pratique pour comparer des décomposition en facteurs irréductibles d'éléments.
\end{example}

Le lemme suivant fait en pratique partie de l'exemple \ref{ExmuQisZU}, mais nous l'isolons pour plus de clarté\footnote{Merci à \href{http://fr.wikipedia.org/wiki/Utilisateur:Marvoir}{Marvoir} pour m'avoir souligné le manque.}.
\begin{lemma}       \label{LemTScCIv}
    Si \( a\) et \( b\) sont deux éléments premiers entre eux de \( \eZ[i\sqrt{2}]\) tels que \( ab=y^3\) alors \( a\) et \( b\) sont des cubes (dans \( \eZ[i\sqrt{2}]\)).
\end{lemma}

\begin{proof}
    D'après l'exemple \ref{ExluqIkE} nous pouvons écrire
    \begin{subequations}
        \begin{align}
            y&=\pm p_1^{\sigma_1}\ldots p_n^{\sigma_n}\\
            a&=\pm p_1^{\alpha_1}\ldots p_n^{\alpha_n}\\
            b&=\pm p_1^{\beta_1}\ldots p_n^{\beta_n}
        \end{align}
    \end{subequations}
    où les \( p_i\) sont les irréductibles de \( \eZ[i\sqrt{2}]\) «modulo \( \pm 1\)» au sens où la liste des irréductibles est \( \{ p_i \}\cup\{ -p_i \}\) (union disjointe). Étant donné que \( a\) et \( b\) sont premiers entre eux, \( \alpha_i\) et \( \beta_i\) ne peuvent pas être non nuls en même temps alors que leur somme doit faire \( 3\sigma_i\). Nous avons donc pour chaque \( i\) soit \( \alpha_i=3\sigma_i\) soit \( \beta=3\sigma_i\) (et bien entendu si \( \sigma_i=0\) alors \( \alpha_i=\beta_i=0\)).

    Étant donné que \( \pm 1\) sont également deux cubes, \( a\) et \( b\) sont bien des cubes.

    Notons que nous avons utilisé de façon capitale le fait que \( \eZ[i\sqrt{2}]\) était factoriel.
\end{proof}

%---------------------------------------------------------------------------------------------------------------------------
\subsection{Équations diophantiennes}
%---------------------------------------------------------------------------------------------------------------------------

\begin{example}
    L'équation diophantienne
    \begin{equation}
        x^2=3y^2+8
    \end{equation}
    n'a pas de solutions. En effet si nous prenons l'équation modulo \( 3\) nous obtenons
    \begin{equation}
        x^2\mod 3=8\mod 3=2\mod 3.
    \end{equation}
    Or dans \( \eZ/3\eZ\), aucun élément ne vérifie \( x^2=2\) : \( 0^2=0\neq 2\), \( 1^2=1\neq 2\) et \( 2^2=4=1\neq 2\).
\end{example}

\begin{example}     \label{ExmuQisZU}
    Résolvons l'équation diophantienne\index{équation!diophantienne} 
    \begin{equation}
        x^2+2=y^3.
    \end{equation}
    Une première remarque est que \( x\) doit être impair. En effet si \( x=2k\), nous devons avoir \( y^3\) pair. Mais si un cube pair est divisible par \( 8\), donc \( y^3=8l\). L'équation devient \( 4k^2+2=8l^3\), c'est à dire \( 2k^2+1=4l^3\). Le membre de gauche est impair tandis que celui de droite est pair. Impossible.

    Nous pouvons écrire l'équation sous la forme \( x^2+2=(x+i\sqrt{2})(x-i\sqrt{2})\).

    L'élément \( i\sqrt{2}\) est irréductible parce que \( N(i\sqrt{2})=2\). Si nous avions \( i\sqrt{2}=pq\), alors nous aurions \( N(p)N(q)=2\), ce qui n'est possible que si \( N(p)\) ou \( N(q)\) égal à \( 1\).

    Nous prouvons maintenant que les éléments \( x+i\sqrt{2}\) et \( x-i\sqrt{2}\) sont premiers entre eux. Supposons que \( d\) soit un diviseur commun; alors il divise aussi la somme et la différence. Donc \( d\) divise à la fois \( 2x\) et \( 2i\sqrt{2}\).

    Étant donné que \( i\sqrt{2}\) est irréductible et que \( 2i\sqrt{2}=(-i\sqrt{2})^3\), les diviseurs de \( 2i\sqrt{2}\) sont les puissances de \( (-i\sqrt{2})\). Du coup nous devrions avoir \( d=(i\sqrt{2})^{\alpha}\) et donc
    \begin{equation}
        x=(i\sqrt{2})^{\beta}q
    \end{equation}
    pour un certain \( q\in\eZ[i\sqrt{2}]\). Dans ce cas nous avons \( N(x)=2^{\beta}N(q)\), mais nous avons déjà précisé que \( x\) ne pouvait pas être pair, donc \( \beta=0\) et nous avons \( d=1\).

    Vu que les nombres \( x\pm i\sqrt{2}\) sont premiers entre eux et que leur produit doit être un cube, ils doivent être séparément des cubes (lemme \ref{LemTScCIv}). Nous devons donc résoudre séparément \( x\pm i\sqrt{2}=y^3\).

    Cherchons les \( x\) et \( y\) entiers tels que \( x+i\sqrt{2}=y^3\). Si nous posons \( z=a+bi\sqrt{2}\), il suffit de calculer \( z^3\) :
    \begin{verbatim}
----------------------------------------------------------------------
| Sage Version 4.8, Release Date: 2012-01-20                         |
| Type notebook() for the GUI, and license() for information.        |
----------------------------------------------------------------------
sage: var('a,b')                                                                                                                                            
(a, b)
sage: z=a+I*sqrt(2)*b
sage: (z**3).expand()
3*I*sqrt(2)*a^2*b - 2*I*sqrt(2)*b^3 + a^3 - 6*a*b^2
    \end{verbatim}
    En identifiant cela à \( x+i\sqrt{2}\) nous trouvons le système
    \begin{subequations}
        \begin{numcases}{}
            x=a^3-6ab^2\\
            1=3a^2b-2b^3
        \end{numcases}
    \end{subequations}
    où, nous le rappelons, \( x\), \( a\) et \( b\) sont des entiers. Le seconde équation montre que \( b\) doit être inversible : \( b(3a^2-2b^2)=1\). Il y a donc les possibilités \( b=\pm 1\). Pour \( b=1\) l'équation devient \( 3a^2-2=1\), c'est à dire \( a=\pm 1\). Pour \( b=-1\) l'équation devient \( 3a^2-2=-1\), impossible. En conclusion les possibilités sont
    \begin{subequations}
        \begin{align}
            (x,z)=(-5,1+i\sqrt{2})\\
            (x,z)=(5,-1+i\sqrt{2})\\
        \end{align}
    \end{subequations}
    Le travail avec \( x-i\sqrt{2}\) donne les mêmes résultats.

    Les deux solutions de l'équation \( x^2+2=y^3\) sont alors \( (5,3)\) et \( (-5,3)\).
\end{example}

%+++++++++++++++++++++++++++++++++++++++++++++++++++++++++++++++++++++++++++++++++++++++++++++++++++++++++++++++++++++++++++
\section{Anneaux des polynômes}
%+++++++++++++++++++++++++++++++++++++++++++++++++++++++++++++++++++++++++++++++++++++++++++++++++++++++++++++++++++++++++++

Soit \( A\) un anneau commutatif. Nous considérons \( \polyP\) l'ensemble des suites presque nulles d'éléments de \( A\), ce sont les suites \( (a_n)_{n\in\eN}\) telles que il existe \( N\) tel que \( a_i=0\) pour tout \( i>N\).

Cela est un \( A\)-module libre de base (définition \ref{DefBasePouyKj})
\begin{equation}
    (e_n)_k=\delta_{nk}.
\end{equation}
Si \( (a_n)_{n\in \eN}\) et \( (b_n)_{n\in\eN}\) sont des éléments de \( \polyP\), nous définissons le produit \( ab\) par
\begin{equation}
    (ab)_n=\sum_{p+q=n}a_pb_q.
\end{equation}
Cela est bien un élément de \( \polyP\) parce qu'il existe \( N\in\eN\) tel que \( a_n=b_n=0\) pour tout \( n\geq N\). Avec la somme et le produit par un scalaire, le module \( \polyP\) devient une \( A\)-algèbre commutative unitaire. L'unité est 
\begin{equation}
    e_0=(1,0,\ldots).
\end{equation}

\begin{definition}
    En tant que \( A\)-algèbre, l'ensemble \( \polyP\) est l'\defe{algèbre des polynômes en une indéterminée}{algèbre!polynômes} à coefficients dans \( A\).
\end{definition}

Si nous posons que \( X=e_1\), et que nous prenons la convention \( X^0=1\), alors nous avons \( e_k=X^k\) et nous notons \( A[X]\) l'anneau \( \polyP\) exprimé avec \( X\). Les éléments de la forme \( \lambda X^k\) avec \( \lambda\in A\) et \( k\in\eN\) sont des \defe{monômes}{monôme}. Nous allons aussi considérer
\begin{equation}\nomenclature[A]{\( A_n[X]\)}{les polynômes à coefficients dans \( A\) et de degré inférieur à \( n\)}
    A_n[X]=\{ P\in A[X]\tq \deg(P)\leq n \}.
\end{equation}
Cela est un sous module libre.

\begin{theorem}     \label{ThoBUEDrJ}
    L'anneau \( A\) est intègre si et seulement si \( A[X]\) est intègre.
\end{theorem}

\begin{proof}
    Soient \( P\) et \( Q\) des éléments non nuls de \( A[X]\). Vu que l'anneau \( A\) est intègre, nous avons
    \begin{equation}
        \deg(PQ)=\deg(P)+\deg(Q)
    \end{equation}
    et le produit ne peut pas être nul. L'anneau \( A[X]\) est donc intègre.

    Si \( A[X]\) est intègre, \( A\) est intègre parce qu'il peut être vu comme sous anneau.
\end{proof}

\begin{remark}
    Si \( A\) n'est pas intègre, soit \( \alpha\beta=0\), alors \( (\alpha X)(\beta x)=0\) et le degré du produit n'est pas la somme des degrés.
\end{remark}

\begin{corollary}
    Si \( A\) est intègre, les inversibles de \( A[X]\) sont les éléments de \( U(A)\).
\end{corollary}

\begin{proof}
    Pour que \( Q\) soit inversible, il faut un \( P\) tel que \( PQ=1\). Mais l'anneau \( A\) étant intègre, les degrés s'additionnent. Par conséquent ils doivent être de degrés zéro et il faut que \( P,Q\in A\). Enfin pour qu'ils soient inversibles, ils doivent être dans \( U(A)\).
\end{proof}

La \defe{valuation}{valuation} de \( P\) du polynôme \( P=\sum_n a_nX^n\), notée \( \val(P)\), est 
\begin{equation}
    \val(P)=\min\{ n\tq a_n\neq 0 \}.
\end{equation}
Nous avons \( \val(P)\leq \deg(P)\) et \( \val(P)=\deg(P)\) si et seulement si \( P\) est un monôme. Si \( P=0\), nous convenons que \( \val(0)=\infty\) et \( \deg(0)=-\infty\).

\begin{proposition}     \label{PropqGZXvr}
    L'anneau \( \eK[X]\) des polynômes sur un corps commutatif \( \eK\) est factoriel.
\end{proposition}

Le théorème suivant est une particuliarisation à \( \eK[X]\) du théorème chinois \ref{ThofPXwiM}.
\begin{theorem}[Théorème chinois]\index{théorème!chinois!anneau des polynômes}
    Si \( P\) et \( Q\) sont deux polynômes premiers entre eux, alors nous avons l'isomorphisme
    \begin{equation}
        \eK[X]/(P,Q)\simeq\eK[X]/(P)\times \eK[X]/(Q).
    \end{equation}
\end{theorem}
% TODO : s'assuer que c'est bien un icp du théorème chinois de plus haut.

%---------------------------------------------------------------------------------------------------------------------------
\subsection{Irréductibilité}
%---------------------------------------------------------------------------------------------------------------------------

\begin{theorem}[d'Alembert-Gauss]\index{théorème!d'Alembert-Gauss}      \label{ThovgyUuA}
    Tout polynôme non constant à coefficients complexes possède au moins une racine complexe.
\end{theorem}


\begin{definition}      \label{DefIrredfIqydS}
    Un polynôme est \defe{irréductible}{irréductible!polynôme} lorsqu'il ne peut pas être écrit sous la forme de produits de polynômes de degré supérieurs à \( 1\).
\end{definition}

\begin{proposition}
    Un polynôme irréductible à coefficients réels est soit de degré un soit de degré \( 2\) avec un discriminant négatif.
\end{proposition}

\begin{proof}
    Soit un polynôme \( P\) à coefficients réels de degré plus grand que \( 1\). Alors le théorème de d'Alembert-Gauss (théorème \ref{ThovgyUuA}) implique l'existence d'une racine \( \alpha\). Il est facile de montrer que le conjugué complexe \( \bar \alpha\) est également racine. Par conséquent les polynômes \( (X-\alpha)\) et \( (X-\bar \alpha)\) divisent \( P\).

    Ces deux polynômes sont premiers entre eux parce que
    \begin{equation}
        a(X-\alpha)+b(X-\bar \alpha)=0
    \end{equation}
    implique \( a=b=0\). Par conséquent le produit 
    \begin{equation}
        X^2-(\alpha+\bar \alpha)X+\alpha\bar\alpha
    \end{equation}
    divise également \( P\). Ce dernier est un polynôme à coefficients réels de degré \( 2\). Donc tout polynôme de degré \( 3\) ou plus est réductible.
\end{proof}

Nous disons que \( P\in\eK[X]\setminus\eK\) est \defe{scindé}{polynôme!scindé} sur \(\eK\) si il est produit dans \(\eK[X]\) de polynômes de degré \( 1\).

\begin{theorem}[Conséquence du \wikipedia{fr}{Lemme_de_Gauss_(polynômes)}{lemme de Gauss}]\index{primitif!polynôme}     \label{ThofiIpXg} 
    Soit \( \eA\) un anneau factoriel et \( \Frac(\eA)\) son corps des fractions. Un polynôme non constant \( P\in \eA[X]\) est irréductible (sur \( \eA\)) si et seulement si il est irréductible et primitif sur \( \Frac(\eA)[X]\). 
\end{theorem}
Dans cet énoncé, un polynôme primitif est un polynôme dont le \( \pgcd\) des coefficients est \( 1\). Voir la remarque \ref{RemwwJbYP}. Notons qu'ici nous considérons des polynômes dont les coefficients sont dans un anneau et non dans un corps comme nous en avons l'habitude.

%---------------------------------------------------------------------------------------------------------------------------
\subsection{Division euclidienne}
%---------------------------------------------------------------------------------------------------------------------------

Le théorème suivant établit la \defe{division euclidienne}{division!euclidienne} dans \( \eA[X]\) du polynôme \( A\) par \( B\).
\begin{theorem}     \label{ThodivEuclPsFexf}
    Soit \( B\neq 0\) dans \( \eA[X]\) de coefficient dominant inversible dans \( \eA\). Pour tout \( A\in\eA[X]\), il existe \( Q,R\in \eA[X]\) tels que
    \begin{equation}
        A=BQ+R
    \end{equation}
    avec \( \deg(R)<\deg(B)\).

    Les polynômes \( Q\) et \( R\) sont déterminés de façon univoque par cette condition. Le polynôme \( Q\) est le \defe{quotient}{quotient} et \( R\) est le \defe{reste}{reste} de la division euclidienne de \( A\) par \( B\).
\end{theorem}

\begin{definition}
Deux polynômes \( P\) et \( Q\) sont dits \defe{étrangers}{étrangers!polynômes} entre eux si \( 1\) est un \( \pgcd\) de \( P\) et \( Q\). Un ensemble de polynômes \( (P_i)_{i\in I}\) est étranger \defe{dans leur ensemble}{étranger!dans leur ensemble} si \( 1\) est un \( \pgcd\) des \( P_i\).
    
Les polynômes \( P\) et \( Q\) sont \defe{premiers entre eux}{premier!deux polynômes entre eux} si les seuls diviseurs communs de \( P\) et \( Q\) sont les inversibles.
\end{definition}

\begin{theorem}[Bézout] \label{ThoBezoutOuGmLB}     \index{Bézout!polynômes}
    Les polynômes \( P_1,\ldots,P_n\) dans \( \eK[X]\) sont étrangers entre eux si et seulement si il existe des polynômes \( Q_1,\ldots,Q_n\in\eK[X]\) tels que
    \begin{equation}
        P_1Q_1+\ldots+P_nQ_n=1.
    \end{equation}
\end{theorem}

Deux polynômes \( P\) et \( Q\) ne sont donc pas premiers entre eux si il existe des polynômes \( x\) et \( y\) tels que l'identité de Bézout soit vérifiée :
\begin{equation}    \label{EqkbbzAi}
    xP+yQ=0;
\end{equation}
cette dernière pourra être écrite en termes de la matrice de Sylvester, voir sous-section \ref{subsecSQBJfr}.

\begin{lemma}       \label{LemuALZHn}
    Soient \( (P_i)_{i=1,\ldots,n}\in \eK[X]\) des polynômes étrangers deux à deux. Alors les polynômes \begin{equation} Q_i=\prod_{j\neq i}P_j \end{equation}
    sont étrangers entre eux\footnote{Et non juste deux à deux.}.
\end{lemma}

\begin{lemma}   \label{LemzwkYdn}
    Soit \( \eK\) un corps commutatif et \( \eA\subset \eK\) un sous anneau de \( \eK\). Soit \( \phi\in \eK[X]\). Si il existe \( Q\in \eA[X]\) unitaire tel que \( \phi Q\in \eA[X]\), alors \( \phi\in \eA[X]\).
\end{lemma}
Une preuve peut être trouvée dans la page des lemmes pour le théorème de Wedderburn sur \href{http://www.les-mathematiques.net/d/a/w/node5.php}{les-mathematiques.net}.

%---------------------------------------------------------------------------------------------------------------------------
\subsection{Idéaux}
%---------------------------------------------------------------------------------------------------------------------------

Soit \( P\in \eK[X]\) un polynôme. Nous notons \( (P)\) l'idéal engendré par \( P\) :
\begin{equation}        \label{EqDefxMkDtW}
    (P)=\{ PR\tq R\in\eK[X] \}.
\end{equation}

\begin{lemma}
    Nous avons
    \begin{enumerate}
        \item
            \( (P)\subset (Q)\) si et seulement si \( Q\) divise \( P\),
        \item
            \( (P)=(Q)\) si et seulement si \( P\) et \( Q\) sont multiples (non nuls) l'un de l'autre.
    \end{enumerate}
\end{lemma}

\begin{proof}
    Si \( (P)\subset (Q)\), en particulier \( P\in(Q)\) et il existe \( R\in\eK[X]\) tel que \( P=QR\), ce qui signifie que \( Q\) divise \( P\).

    Si les idéaux de \( P\) et de \( Q\) sont identiques, l'un divise l'autre et l'autre divise l'un. Ils sont donc multiples l'un de l'autre.
\end{proof}

\begin{theorem}     \label{ThoCCHkoU}
    Soit \( \eK\) un corps commutatif.
    \begin{enumerate}
        \item
            L'anneau \( \eK[X]\) est principal. 
        \item
            Si \( I\) est un idéal dans \( \eK[X]\) et si \( P\) est de degré minimal, alors \( (P)=I\).
        \item
            De plus si \( I\neq \{  0\}\), il existe un unique polynôme unitaire \( \mu\) tel que \( I=(\mu)\).
    \end{enumerate}
\end{theorem}

\begin{proof}
    L'anneau \( \eK[X]\) est commutatif et intègre (pas de diviseurs de zéro). Nous devons encore montrer que tous les idéaux sont principaux.

    Si \( I=\{ 0 \}\), le résultat est évident. Nous supposons donc \( I\) non nul. Soit \( P\) de degré minimum parmi les éléments de \( I\). Évidemment \( (P)\subset I\). Nous allons démontrer qu'en réalité \( (P)=I\).

    Soit \( A\in I\). Par le théorème \ref{ThodivEuclPsFexf} de la division euclidienne\footnote{Ici \( \eK\) est un corps et donc l'hypothèse d'inversibilité est automatiquement vérifiée.}, il existe \( Q\) et \( R\) dans \( \eK[X]\) tels que \( A=PQ+R\) avec \( \deg(R)<\deg(P)\). Étant donné que \( R=A-PQ\) nous avons \( R\in I\) et par conséquent \( R=0\) parce que \( P\) a été choisit de degré minimum dans \( I\). Nous avons donc \( A=PQ\) et \( I\subset (P)\).

    L'existence d'un polynôme unitaire qui génère \( I\) est obtenu en choisissant \( U=P/a_n\) où \( a_n\) est le coefficient du terme de plus haut degré.
\end{proof}
Nous voyons que n'importe quel polynôme de degré minimum dans un idéal génère l'idéal. Une importante conséquence du théorème \ref{ThoCCHkoU} que nous verrons plus bas est que tout polynôme annulateur d'un endomorphisme est divisé par le polynôme minimal (proposition \ref{PropAnnncEcCxj}).

\begin{corollary}
    Soit \( P\in \eK[X]\) et \( a\in \eK\), une racine de \( P\). Alors le polynôme minimal de \( a\) dans \( \eK[X]\) divise \( P\). En d'autre termes, le polynôme minimal d'un élément divise tout polynôme annulateur.
\end{corollary}

\begin{proof}
    Nous considérons l'idéal
    \begin{equation}
        I=\{ Q\in \eK[X]\tq Q(a)=0 \}.
    \end{equation}
    Le fait que cela soit un idéal est simplement dû à la définition du produit : \( (PQ)(a)=P(a)Q(a)\). Par le théorème \ref{ThoCCHkoU}, le polynôme minimal \( \mu_a\) de \( a\) est dans \( I\) et qui plus est le génère : \( I=(\mu_a)\). Par conséquent tout polynôme annulateur de \( a\) est divisé par \( \mu_a\).
\end{proof}

%---------------------------------------------------------------------------------------------------------------------------
\subsection{Racines de polynômes}
%---------------------------------------------------------------------------------------------------------------------------

Soit \( \eA\) un anneau et \( P\in \eA[X]\) un polynôme et \( \alpha\in \eA\). Le \defe{degré}{degré!d'une racine} ou la \defe{multiplicité}{multiplicité!d'une racine} de \( \alpha\) par rapport à \( P\) est l'entier \( h\) tel que \( P\) est divisible par \( (X-\alpha)^h\) mais pas divisible par \( (X-\alpha)^{h+1}\).

Nous noterons \( \theta_{\alpha}(P)\)\nomenclature[A]{\( \theta_{\alpha}(P)\)}{l'ordre de \( \alpha\) par rapport à \( P\)} l'ordre de \( \alpha\) par rapport à \( P\).

\begin{proposition}     \label{PropahQQpA}
    L'élément \( \alpha\in \eA\) est d'ordre \( h\) par rapport à \( \) si et seulement si il existe \( Q\in\eA[X]\) tel que \( P(X-\alpha)^hQ\) avec \( Q(\alpha)\neq 0\).
\end{proposition}

\begin{lemma}       \label{LemIeLhpc}
    Soient \( P\) et \( Q\) des polynômes non nuls de \( \eA[X]\) et \( \alpha\in \eA\) d'ordre \( p\) pour \( P\) et d'ordre \( q\) pour \( Q\). Alors
    \begin{enumerate}
        \item
            \( \theta_{\alpha}(P+Q)\geq\ln\{ \theta_{\alpha}(P),\theta_{\alpha}(Q) \}\)
        \item
            si \( \theta_{\alpha}(P)\neq \theta_{\alpha}(Q)\), alors \( \theta_{\alpha}(P+Q)=\min\{ \theta_{\alpha}(P),\theta_{\alpha}(Q) \}\)
        \item
            \( \theta_{\alpha}(PQ)\geq \theta_{\alpha(P)}+\theta_{\alpha}(Q)\);
        \item       \label{ItemIeLhpciv}
            si \(\eA \) est intègre alors \( \theta_{\alpha}(PQ)= \theta_{\alpha}(P)+\theta_{\alpha}(Q)\);
    \end{enumerate}
\end{lemma}

\begin{theorem}
    Soit \( \eA\) un anneau intègre et \( P\in \eA[X]\setminus\{ 0 \}\), un polynôme de degré \( n\). Si \( \alpha_1,\ldots, \alpha_p\in\eA\) sont des racines deux à deux distinctes d'ordres \( k_1,\ldots, k_p\), alors il existe \( Q\in \eA[X]\) tel que
    \begin{enumerate}
        \item
            \( Q(\alpha_i)\neq 0\);
        \item
            \( P=Q\prod_{i=1}^p(X-\alpha_i)\);
    \end{enumerate}
    De plus la sommes des ordres des racines de \( P\) est au plus \( \deg(P)\).
\end{theorem}

\begin{proof}
    Si \( p=1\), alors le résultat est la proposition \ref{PropahQQpA}. Nous supposons que \( p\geq 2\) et nous effectuons une récurrence sur \( P\). Nous considérons donc pas \( p-1\) premières racines \( \alpha_1,\ldots, \alpha_{p-1}\) et un polynôme \( R\in\eA[X]\) tel que \( R(\alpha_i)\neq 0\) pour \( i=1,\ldots, p-1\) et
    \begin{equation}
        P=\underbrace{(X-\alpha_1)^{k_1}\ldots (X-\alpha_{p-1})^{k_{p-1}}}_SR.
    \end{equation}
    Par hypothèse \( P(\alpha_p)=S(\alpha_p)R(\alpha_p)=0\). L'anneau \( \eA\) étant intègre, \( S(\alpha_p)\neq 0\) parce que \( \alpha_i\neq \alpha_p\) pour \( i\neq p\). Par conséquent, \( R(\alpha_p)=0\).
    
    Nous devons encore vérifier que l'ordre de \( \alpha_p\) est \( k_p\) par rapport à \( R\). Pour cela nous utilisons le point \ref{ItemIeLhpciv} du lemme \ref{LemIeLhpc} afin de dire que le degré de \( \alpha_p\) pour \( P=SR\) est \( k_p\). Par conséquent
    \begin{equation}
        R=(X-\alpha_p)^{k_p}T
    \end{equation}
    avec \( T(\alpha_p)\neq 0\) et enfin
    \begin{equation}
        P=\prod_{i=1}^p(X-\alpha_i)T.
    \end{equation}
    De plus \( T(\alpha_i)\neq 0\), sinon \( R(\alpha_i)\) serait nul.
\end{proof}

\begin{proposition}[\wikipedia{fr}{Critère_d'Eisenstein}{Critère d'Eisenstein}]
    Soit le polynôme \( P(X)=\sum_{k=0}^n a_nX^n\) dans \( \eZ[X]\). Nous supposons avoir un nombre premier \( p\) tel que
    \begin{enumerate}
        \item
            \( p\) divise tous les \( a_0,\ldots, a_{n-1}\),
        \item
            \( p\) ne divise pas \( a_n\),
        \item
            \( p^2\) ne divise pas \( a_0\).
    \end{enumerate}
    Alors \( P\) est irréductible dans \( \eQ[X]\).

    Si de plus \( P\) est primitif au sens du \( \pgcd\) alors \( P\) est irréductible dans \( \eZ[X]\).
\end{proposition}

\begin{proof}
    Nous considérons \( \bar P\) le polynôme réduit modulo \( p\), c'est à dire \( \bar P\in \eF_p[X]\). Étant donné que par hypothèse tous les coefficients sont multiples de \( p\) sauf \( a_n\), nous avons \( \bar P=cX^n\). Supposons par l'absurde que \( P=QR\) avec \( Q,R\in \eQ[X]\). Alors le lemme de Gauss (\ref{LemSdnZNX}) impose \( P,Q\in \eZ[X]\).

    Nous avons aussi, au niveau des réductions modulo \( p\) que $\bar Q\bar R=\bar P$. Or \( \bar P\) est un monôme, donc \( \bar Q\) et \( \bar R\) doivent également l'être. Donc \( \bar Q=dX^k\) et \( \bar R=eX^{n-k}\) et en particulier \( \bar Q(0)=\bar R(0)=0\), c'est à dire que \( Q(0)\) et \( R(0)\) sont divisibles par \( p\). Cela impliquerait que \( a_0=Q(0)R(0)\) soit divisible par \( p^2\), ce qui est exclu par les hypothèses. Donc \( P\) est irréductible.

    Supposons de surcroît que \( P\) est primitif au sens du \( \pgcd\). Il est donc irréductible et primitif sur \( \eQ[X]\) et une conséquence du lemme de Gauss (\ref{ThofiIpXg}) nous dit alors que \( P\) est irréductible sur \( \eZ[X]\).
\end{proof}

\begin{example}
    Soit le polynôme \( P(X)=3X^4+15 X^2+10\). Pour faire fonctionner le critère d'Eisenstein il nous faut un nombre premier \( p\) divisant \( 15\) et \( 10\), mais pas \( 3\) et dont le carré ne divise pas \( 10\). C'est vite vu que \( p=5\) fait l'affaire. Le polynôme \( P\) est donc irréductible sur \( \eQ[X]\).
\end{example}
