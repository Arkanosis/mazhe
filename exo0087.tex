% This is part of Exercices et corrigés de CdI-1
% Copyright (c) 2011
%   Laurent Claessens
% See the file fdl-1.3.txt for copying conditions.

\begin{exercice}\label{exo0087}



Soient $(X,d_X)$, $(Y,d_Y)$ deux espaces métriques.  Une application $f \colon X \to Y$ est {\bf Lipschitzienne} s'il existe une constante $L \ge 0$ telle que 
\begin{equation}
	d_Y \bigl( f(x), f(x') \bigr) \,\le\, L \,d_X (x,x')
\end{equation}
pour tout  $x$, $x' \in X$. Dans ce cas, on dit que $f$ est {\bf $L$-Lipschitzienne}.


\begin{enumerate}
\item
Montrer qu'une application Lipschitzienne est continue.

\item
Montrer qu'une application $f \colon \RR \to \RR$, $x \mapsto ax+b$ est Lipschitzienne.  Quelle est la plus petite constante $L$ qui convienne?

\item
 Montrer que les fonctions $z \mapsto |z|$, $z \mapsto \overline z$, $z \mapsto {\rm Re\,} z$ et $z \mapsto {\rm Im\,} z$ sont Lipschitziennes.  Quelle sont les plus petites constantes $L$ qui conviennent?

\item 
Montrer que la fonction $d(A,\cdot)\colon X \to \eR$ de l'exercice \ref{exo0086} est Lipschitzienne.

\end{enumerate}

\corrref{0087}
\end{exercice}
