% This is part of Mes notes de mathématique
% Copyright (c) 2008-2014
%   Laurent Claessens
% See the file fdl-1.3.txt for copying conditions.

%+++++++++++++++++++++++++++++++++++++++++++++++++++++++++++++++++++++++++++++++++++++++++++++++++++++++++++++++++++++++++++
\section{Série réelle}
%+++++++++++++++++++++++++++++++++++++++++++++++++++++++++++++++++++++++++++++++++++++++++++++++++++++++++++++++++++++++++++
\label{secseries}

\subsection{Rappels et définitions}

La notion de série formalise le concept de somme infinie. L'absence de certaines propriétés de ces objets (problèmes de commutativité et même d'associativité) incitent à la prudence et montrent à quel point une définition précise est importante.

Soit ${(a_k)}_{k \geq k_0}$ une suite réelle. La \defe{série}{série} de terme général $(a_k)${série}, notée
\begin{math}
  \sum_{i=k_0}^\infty a_i,
\end{math}
est la suite ${(s_n)}_{n \geq k_0}$ dont les termes sont donnés par
\begin{equation*}
  s_n \pardef \sum_{i=k_0}^k a_i
\end{equation*}
et sont appelés les \defe{sommes partielles}{somme!partielle} de la série.

La série $\sum_{i=k_0}^\infty a_i$ \defe{converge}{série!convergence} si la suite $(s_n)$
converge vers un réel $s$. Sa limite est appelée la \defe{somme de la
série}{série!somme} et on note
\begin{equation}        \label{EqDefSommeLim}
  \sum_{i=k_0}^\infty a_i = \lim_{n\to\infty}\sum_{i=k_0}^na_i.
\end{equation}
Si la série ne converge pas, elle \defe{diverge}{série!divergence} et peut alors avoir
une limite infinie (uniquement si le terme général est réel, on note
alors $+\infty$ ou $-\infty$ sa limite) ou pas de limite.  La série
$\sum_{i=k_0}^\infty a_i$ \defe{converge absolument}{convergence!absolue} si la série
$\sum_{i=k_0}^\infty \abs{a_i}$ converge.

\begin{proposition}\label{propnseries_propdebase}
Les principales propriétés de la somme définie par la limite \eqref{EqDefSommeLim} sont
  \begin{enumerate}
  \item Si une série converge absolument, alors elle converge
    (simplement).
  \item Si la série est à termes positifs --c'est-à-dire pour tout
    indice $k$, $a_k \in \eR$ et $a_k \geq 0$-- il n'y a aucune
    différence entre convergence absolue et convergence simple.
  \item\label{point3-seriepropdebase} Si une série converge, son terme général doit tendre vers $0$.
\item 
Si la série converge alors la somme est associative
\item
Si la série converge absolument, alors la somme est commutative.
  \end{enumerate}
\end{proposition}

\begin{remark}Vue comme somme infinie, l'associativité et la
  commutativité dans une série sont perdues. Néanmoins, il subsiste
  que
  \begin{enumerate}
  \item si la série converge, on peut regrouper ses termes sans
    modifier la convergence ni la somme (associativité),
  \item si la série converge absolument, on peut modifier l'ordre des
    termes sans modifier la convergence ni la somme (commutativité).
  \end{enumerate}
\end{remark}
%TODO : il me semble qu'il y a une preuve de ça quelque part au début de Hilbert. Mettre une référence.

\begin{example}\label{exemplesseries}
\begin{enumerate}

\item
    La \defe{série harmonique}{série!harmonique} est
\begin{equation}
\sum_{i=1}^\infty \frac1i
\end{equation}
et diverge (possède une limite $+\infty$).

\item
    La \defe{série géométrique}{série!géométrique} de raison $q \in \eC$ est
\begin{equation}
\sum_{i=0}^\infty q^i.
\end{equation}
Étudions la somme partielle \( S_N=1+q+q^2+\ldots +q^{n}\). Nous avons évidemment $S_N-zS_N=1-q^{N+1}$ et donc
\begin{equation}    \label{EqASYTiCK}
    S_N=\sum_{n=0}^Nq^n=\frac{ 1-q^{N+1} }{ 1-q }.
\end{equation}
La limite \( \lim_{N\to \infty} S_N\) existe si et seulement si \( | q |\leq 1\) et dans ce cas nous avons
\begin{equation}
    \sum_{n=1}^{\infty}z^n=\frac{ 1 }{ 1-z }.
\end{equation}
La convergence est absolue.

\item
Pour $\alpha \in \RR$, la série de Riemann (ou Dirichlet)
\begin{equation}        \label{EqSerRiem}
\sum_{i=1}^\infty \frac1{i^\alpha}
\end{equation}
converge (absolument, puisque réelle et positive) si et seulement
si $\alpha > 1$, et diverge sinon.

\item La \defe{suite arithméticogémétrique}{suite!arithméticogéométrique} est une suite de la forme \( u_{n+1}=au_n+b\) avec \( a\) et \( b\) non nuls. Si elle possède une limite, cette dernière doit résoudre \( l=al+n\), et donc être égale à 
    \begin{equation}
        r=\frac{ b }{ 1-a }.
    \end{equation}
    Il n'est pas très compliqué de trouver le terme général de la suite en fonction de \( a\) et de \( b\). Il suffit de considérer la suite \( v_n=u_n-r\), et de remarquer que cette suite est géométrique :
    \begin{equation}
        v_{n+1}=av_n.
    \end{equation}
    Par conséquent \( v_n=a^nv_0\), ce qui donne pour la suite \( (u_n)\) la formule
    \begin{equation}
        u_n=a^n(u_0-r)+r.
    \end{equation}
    Pour plus de détails, vous pouvez consulter \cite{QXuqdoo}.

\end{enumerate}
\end{example}

\subsection{Critères de convergence absolue}

  Étant donné le terme général d'une série, il est souvent --dans les cas qui nous intéressent-- difficile de déterminer la somme de la série. L'exemple de la série géométrique est particulier, puisqu'on connait une formule pour chaque somme partielle, mais pour l'exemple des séries de Riemann il n'y a aucune formule simple pour un $\alpha$ général. D'où l'intérêt d'avoir des critères de convergence ne nécessitant aucune connaissance de l'éventuelle limite de la série.

\subsubsection{Critère de comparaison} 

\begin{lemma}[Critère de comparaison]   \label{LemgHWyfG}
Soient $\sum_i a_i$ et $\sum_j
b_j$ deux séries à termes positifs vérifiant
\begin{equation*}
  0 \leq a_i \leq b_i
\end{equation*}
alors
\begin{enumerate}
\item si $\sum_i a_i$ diverge, alors $\sum_j b_j$ diverge,
\item si $\sum_j b_j$ converge, alors $\sum_i a_i$ converge
  (absolument).
  \end{enumerate}
\end{lemma}

\subsubsection{Critère d'équivalence}
%\label{PgCritEquiv}

\begin{proposition}[\cite{TrenchRealAnalisys}]
 Soient $\sum_i a_i$ et $\sum_j b_j$ deux séries à termes positifs. Supposons l'existence de la limite (éventuellement infinie) suivante
\begin{equation}
  \limite i \infty \frac{a_i}{b_i} = \alpha \in \RR \text{ ou $\alpha =
    \infty$.}
\end{equation}
Dans ce cas, nous avons
\begin{enumerate}
\item si $\alpha \neq 0$ et $\alpha\neq \infty$, alors
  \begin{equation}
    \sum_i a_i \text{~converge} \ssi \sum_j b_j\text{~converge,}
  \end{equation}
\item si $\alpha = 0$ et $\sum_j b_j$ converge, alors $\sum_i a_i$
  converge (absolument),
\item si $\alpha = +\infty$ et $\sum_j b_j$ diverge, alors $\sum_i
  a_i$ diverge.
\end{enumerate}
\end{proposition}

\begin{proof}
\begin{enumerate}
    \item
        Le fait que la suite $a_n/b_n$ converge vers $\alpha$ signifie que tant sa limite supérieure que sa limite inférieure convergent vers $\alpha$. En particulier la suite $\frac{ a_n }{ b_n }$ est bornée vers le haut et vers le bas. À partir d'un certain rang $N$, il existe $M$ tel que 
        \begin{equation}
            \frac{ a_n }{ b_n }<M
        \end{equation}
        et il existe $m$ tel que
        \begin{equation}
            \frac{ a_n }{ b_n }>m.
        \end{equation}
        Nous avons donc $a_n<Mb_n$ et $a_n>mb_n$. La série de $(a_n)$ converge donc si et seulement si la série de $(b_n)$ converge.
    \item
        Si $\alpha=0$, cela signifie que pour tout $\epsilon$, il existe un rang tel que $\frac{ a_n }{ b_n }<\epsilon$, et donc tel que $a_n<\epsilon b_k$. La suite de $(a_i)$ converge donc dès que la suite de $(b_i)$ converge.
    \item
        Pour tout $M$, il existe un rang dans la suite à partir duquel on a $\frac{ a_i }{ b_i }>M$, et donc $a_k>Mb_k$. Si la série de $(b_k)$ diverge, la série de $(a_k)$ doit également diverger.
\end{enumerate}
\end{proof}

\subsubsection{Critère du quotient}

\begin{proposition}[\cite{KeislerElemCalculus}]
    Soit $\sum_i a_i$ une série. Supposons l'existence de la limite (éventuellement infinie) suivante
    \begin{equation}
      \limite i \infty \abs{\frac{a_{i+1}}{a_i}} = L\in \RR \text{ ou $L =
        \infty$.}
    \end{equation}
    Alors
    \begin{enumerate}
    \item si $L < 1$, la série converge absolument,
    \item si $L > 1$, la série diverge,
    \item si $L = 1$ le critère échoue : il existe des exemple de convergence et des exemples de divergence.
    \end{enumerate}
\end{proposition}

\begin{proof}
\begin{enumerate}
    \item
        Soit $b$ tel que $L<b<1$. À partir d'un certain rang $K$, on a $\left| \frac{ a_{i+1} }{ a_i } \right| <b$. En particulier,
        \begin{equation}
            | a_{K+1} |<b| a_K |,
        \end{equation}
        et pour $a_{K+2}$ nous avons
        \begin{equation}
            | a_{K+2} |<b| a_{K+1} |<b^2| a_K |.
        \end{equation}
        Au final,
        \begin{equation}
            | a_{K+n} |<b^n| a_K |.
        \end{equation}
        Étant donné que la série $\sum_{n\geq K}b^n$ converge (parce que $b<1$), la queue de suite $\sum_{i\geq K}a_i$ converge, et par conséquent la suite au complet converge.
    \item
        Si $L>1$, on a
        \begin{equation}
            | a_K |<| a_{K+1} |<| a_{K+2} |<\ldots
        \end{equation}
        Il est donc impossible que la suite $(a_i)$ converge vers zéro. La série ne peut donc pas converger.
    \item
        Par exemple la suite harmonique $a_n=\frac{1}{ n }$ vérifie $L=1$, mais la série ne converge pas. Par contre, la suite $a_n=\frac{ 1 }{ n^2 }$ vérifie aussi le critère avec $L=1$ tandis que la série $\sum_n\frac{1}{ n^2 }$ converge.
\end{enumerate}
\end{proof}

\subsubsection{Critère de la racine}

\begin{proposition}[\cite{TrenchRealAnalisys}]
    Soit $\sum_i a_i$ une série, et considérons
    \begin{equation*}
      \limsup_{i \rightarrow \infty} \sqrt[i]{\abs{a_i}} = L \in \RR
      \text{ ou $L =
        \infty$.}
    \end{equation*}
    Alors
    \begin{enumerate}
    \item si $L < 1$, la série converge absolument,
    \item si $L> 1$, la série diverge,
    \item si $L = 1$ le critère échoue.
    \end{enumerate}
\end{proposition}

\begin{proof}
    \begin{enumerate}
        \item
            Si $L<1$, il existe un $r\in \mathopen] 0 , 1 \mathclose[$ tel que $| a_n |^{1/n}<r$ pour les grands $n$. Dans ce cas, $| a_n |<r^{n}$, et la série converge absolument parce que la série $\sum_nr^n$ converge du fait que $r<1$.
        \item
            Si $L>1$, il existe un $r>1$ tel que $| a_n |^{1/n}>r>1$. Cela fait que $| a_n |$ prend des valeurs plus grandes que $n$ pour une infinité de termes. Le terme général $a_n$ ne peut donc pas être une suite convergente. Par conséquent la suite diverge au sens où elle ne converge pas.

    \end{enumerate}
\end{proof}

%---------------------------------------------------------------------------------------------------------------------------
\subsection{Critères de convergence simple}
%---------------------------------------------------------------------------------------------------------------------------

Les critères de comparaison, d'équivalence, du quotient et de la racine sont des critères de convergence absolue. Pour conclure à une convergence simple qui n'est pas une convergence absolue, le critère d'Abel sera notre outil principal.  

\subsubsection{Critère d'Abel}

\begin{proposition}[Critère d'Abel]
    Soit la série $\sum_i c_iz_i$ avec
    \begin{enumerate}
        \item $(c_i)$ est une suite réelle décroissante qui tend vers zéro,
        \item $(z_i)$ est une suite dans $\eC$ dont la suite des sommes partielles est bornée dans $\eC$, c'est à dire qu'il existe un $M>0$ tel que pour tout $n$,
        \begin{equation}
            \left| \sum_{i=1}^nz_i \right| \leq M.
        \end{equation}
        Alors la série $\sum_ic_iz_i$ est convergente.
    \end{enumerate}
\end{proposition}
Remarquons que ce critère ne donne pas de convergence absolue.

\begin{corollary}[Critère des séries alternées]\index{critère!série alternée}       \label{CoreMjIfw}
    Si \( (a_n)\) est une suite décroissante à limite nulle, alors la série
  \begin{equation}
    \sum_{n=0}^\infty {(-1)}^n a_n
  \end{equation}
  converge simplement.
\end{corollary}

%---------------------------------------------------------------------------------------------------------------------------
\subsection{Moyenne de Cesaro}
%---------------------------------------------------------------------------------------------------------------------------

Si \( (a_n)_{n\in \eN} \) est une suite dans \( \eR\) ou \( \eC\), alors sa \defe{moyenne de Cesaro}{moyenne!de Cesaro}\index{Cesaro!moyenn} est la limite (si elle existe) de la suite
\begin{equation}
    c_n=\frac{1}{ n }\sum_{k=1}^na_k.
\end{equation}
En un mot, c'est la limite des moyennes partielles.

\begin{lemma}       \label{LemyGjMqM}
    Si la suite \( (a_n)\) converge vers la limite \( \ell\) alors la suite admet une moyenne de Cesaro qui vaudra \( \ell\).
\end{lemma}

\begin{proof}
    Soit \( \epsilon>0\) et \( N\in \eN\) tel que \( | a_n-\ell |<\epsilon\) pour tout \( n>N\). En remarquant que
    \begin{equation}
        \frac{1}{ n }\sum_{k=1}^nk-\ell=\frac{1}{ n }\sum_{k=1}^n(a_k-\ell),
    \end{equation}
    nous avons
    \begin{subequations}
        \begin{align}
            | \frac{1}{ n }\sum_{k=1}^na_k-\ell |&\leq| \frac{1}{ n }\sum_{k=1}^N| a_k-\ell | |+\big| \frac{1}{ n }\sum_{k=N+1}^n\underbrace{| a_k-\ell |}_{\leq \epsilon} \big|\\
            &\leq \epsilon+\frac{ n-N-1 }{ n }\epsilon\\
            &\leq 2\epsilon.
        \end{align}
    \end{subequations}
    Dans ce calcul nous avons redéfinit \( N\) de telle sorte que le premier terme soit inférieur à \( \epsilon\).
\end{proof}


%+++++++++++++++++++++++++++++++++++++++++++++++++++++++++++++++++++++++++++++++++++++++++++++++++++++++++++++++++++++++++++
\section{Limite de fonction}
%+++++++++++++++++++++++++++++++++++++++++++++++++++++++++++++++++++++++++++++++++++++++++++++++++++++++++++++++++++++++++++
\label{SecLimiteFontion}

%---------------------------------------------------------------------------------------------------------------------------
\subsection{Définition}
%---------------------------------------------------------------------------------------------------------------------------

\begin{definition}[Limite d'une fonction]	\label{DefLimiteFonction}
	Soit une fonction $f\colon D\subset\eR\to \eR$ et $a$ un point d'accumulation de $D$. On dit que $f$ admet une \defe{limite}{limite!fonction} en $a$ si il existe un réel $\ell$ tel que 
	\begin{equation}\label{EqDefLimiteFonction}
		\forall\varepsilon>0,\,\exists\delta>0\tq \forall x\in D,\, 0<| x-a |<\delta\Rightarrow| f(x)-\ell |<\varepsilon.
	\end{equation}
\end{definition}

Si aucun nombre $\ell$ ne vérifie la condition de la définition, alors on dit que la fonction n'admet pas de limite en $a$. Lorsque $f$ possède la limite $\ell$ en $a$, nous notons
\begin{equation}
	\lim_{x\to a} f(x)=\ell.
\end{equation}

\begin{proposition}
	Soit une fonction $f\colon D\to \eR$. Si $a$ est un point d'accumulation de $D$ et si il existe une limite de $f$ en $a$, alors il en existe une seule. 
\end{proposition}

De façon équivalente, il ne peut pas exister deux nombres $\ell\neq\ell'$ vérifiant tout les deux la condition \eqref{EqDefLimiteFonction}.

\begin{proof}
	Soient $\ell$ et $\ell'$ deux limites de $f$ au point $a$. Par définition, pour tout $\varepsilon$ nous avons des nombres $\delta$ et $\delta'$ tels que
	\begin{equation}	\label{EqsContf2307Right}
		\begin{aligned}[]
			| x-a |<\delta&\Rightarrow \big| f(x)-\ell \big|<\varepsilon\\
			| x-a |<\delta'&\Rightarrow \big| f(x)-\ell' \big|<\varepsilon
		\end{aligned}
	\end{equation}
	Pour fixer les idées, supposons que $\delta<\delta'$ (le cas $\delta\geq\delta'$ se traite de la même manière).

	Étant donné que $a$ est un point d'accumulation du domaine $D$ de $f$, il existe un $x\in D$ tel que $| x-a |<\delta$. Évidemment, nous avons aussi $| x-a |<\delta'$. Les conditions \eqref{EqsContf2307Right} signifient alors que ce $x$ vérifie en même temps
	\begin{equation}
		| f(x)-\ell |<\varepsilon,
	\end{equation}
	et
	\begin{equation}
		| f(x)-\ell' |<\varepsilon.
	\end{equation}
	Afin de prouver que $\ell=\ell'$, nous allons maintenant calculer $| \ell-\ell' |$ et montrer que cette distance est plus petite que tout nombre. Nous avons (voir remarque \ref{RemTechniqueIneqs})
	\begin{equation}	\label{EqInesq2307ellellepr}
		| \ell-\ell' |=| \ell-f(x)+f(x)-\ell' |\leq | \ell-f(x) |+| f(x)-\ell' |<\varepsilon+\varepsilon.
	\end{equation}
	En résumé, pour tout $\varepsilon>0$ nous avons
	\begin{equation}
		| \ell-\ell' |<2\varepsilon,
	\end{equation}
	et donc $| \ell-\ell' |=0$, ce qui signifie que $\ell=\ell'$.
\end{proof}

\begin{remark}		\label{RemTechniqueIneqs}
	Les inégalités \eqref{EqInesq2307ellellepr} utilisent deux techniques très classiques en analyse qu'il convient d'avoir bien compris. La première est de faire
	\begin{equation}
		| A-B |=| A-C+C-B |.
	\end{equation}
	Il s'agit d'ajouter $-C+C$ dans la norme. Évidemment, cela ne change rien.

	La seconde technique est l'inégalité
	\begin{equation}
		| A+B |\leq| A |+| B |.
	\end{equation}
\end{remark}

\begin{example}
	Considérons la fonction $f(x)=2x$, et calculons la limite $\lim_{x\to 3} f(x)$. Vu que $f(3)=6$, nous nous attendons à avoir $\ell=6$. C'est ce que nous allons prouver maintenant. Pour chaque $\varepsilon>0$ nous devons trouver un $\delta>0$ tel que $| x-3 |<\delta$ implique $| f(x)-6 |<\varepsilon$. En remplaçant $f(x)$ par sa valeur en fonction de $x$ et avec quelque manipulations nous trouvons :
	\begin{equation}
		\begin{aligned}[]
			| f(x)-6 |&<\varepsilon\\
			| 2x-6 |&<\varepsilon\\
			2| x-3 |&<\varepsilon\\
			| x-3 |&<\frac{ \varepsilon }{2}
		\end{aligned}
	\end{equation}
	Donc dès que $| x-3 |<\frac{ \varepsilon }{2}$, nous avons $| f(x)-6 |<\varepsilon$. Nous posons donc $\delta=\frac{ \varepsilon }{2}$.

	Plus généralement, nous avons $\lim_{x\to a} f(x)=2a$, et cela se prouve en étudiant $| f(x)-2a |$ exactement de la même manière.
\end{example}

%---------------------------------------------------------------------------------------------------------------------------
\subsection{Propriétés de base}
%---------------------------------------------------------------------------------------------------------------------------

\begin{proposition}	\label{PropLimEstLineraure}
	La limite est une opération linéaire, c'est à dire que si $f$ et $g$ sont des fonctions qui admettent des limites en $a$ et si $\lambda$ est un nombre réel,
	\begin{enumerate}

		\item
			$\lim_{x\to a} (\lambda f)(x)=\lambda\lim_{x\to a} f(x)$,
		\item
			$\lim_{x\to a} (f+g)(x)=\lim_{x\to a} f(x)+\lim_{x\to a} g(x)$.
	\end{enumerate}
\end{proposition}
En combinant les deux propriétés de la proposition \ref{PropLimEstLineraure}, nous pouvons écrire
\begin{equation}
	\lim_{x\to a} (\lambda f+\mu g)(x)=\lambda\lim_{x\to a} f(x)+\mu\lim_{x\to a} g(x).
\end{equation}
pour toutes fonctions $f$ et $g$ admettant une limite en $a$ et pour tout réels $\lambda$ et $\mu$.

En plus d'être linéaire, la limite possède les deux propriétés suivantes.
\begin{proposition}
	Si $f$ et $g$ sont deux fonctions qui admettent une limite en $a$, alors
	\begin{equation}
		\lim_{x\to a} (fg)(x)=\lim_{x\to a} f(x)\cdot\lim_{x\to a} g(x).
	\end{equation}
	Si de plus $\lim_{x\to a} g(x)\neq 0$, alors
	\begin{equation}
		\lim_{x\to a} \frac{ f(x) }{ g(x) }=\frac{ \lim_{x\to a} f(x) }{ \lim_{x\to a} g(x) }.
	\end{equation}
\end{proposition}

%---------------------------------------------------------------------------------------------------------------------------
\subsection{Limites de fonctions}
%---------------------------------------------------------------------------------------------------------------------------

\begin{definition}\label{def_limite}
	Soit $f\colon D\subset\eR^m\to \eR$ une fonction et $a$ un point d'accumulation de $D$. On dit que $f$ possède une \defe{limite}{limite!fonction de plusieurs variables} si il existe un élément $\ell\in\eR$ tel que
	\begin{equation}		\label{Eq2807CondiionLimifnm}
		\forall\varepsilon>0,\,\exists\delta>0\tq 0<\| x-a \|<\delta\Rightarrow | f(x)-\ell |<\varepsilon.
	\end{equation}
	
	Pour une fonction $f\colon D\subset\eR^m\to \eR^n$, la définition est la même, sauf que nous remplaçons la valeur absolue par la norme dans $\eR^n$. Nous disons donc que $\ell$ est la limite de $f$ lorsque $x$ tend vers $a$, et nous notons $\lim_{x\to a} f(x)=\ell$ lorsque pour tout $\varepsilon>0$, il existe un $\delta>0$ tel que
	\begin{equation}		\label{EqDefLimRpRn}
		0<\| x-a \|_{\eR^m}<\delta\Rightarrow\,\| f(x)-\ell \|_{\eR^n}<\varepsilon.
	\end{equation}
\end{definition}

\begin{remark}
	Dans l'équation \eqref{EqDefLimRpRn}, nous avons explicitement écrit les normes $\| . \|_{\eR^m}$ et $\| . \|_{\eR^n}$. Dans la suite nous allons le plus souvent noter $\| . \|$ sans plus de précision. Il est important de faire l'exercice de bien comprendre à chaque fois de quelle norme nous parlons.
\end{remark}

\begin{remark}
	Il est important de remarquer à quel point les définitions \ref{def_limite}, \ref{LimiteDansEVN} et \ref{DefLimiteFonction} sont analogues. En réalité, la définition fondamentale est la définition de la limite dans les espaces vectoriels normés; les deux autres sont des cas particuliers, adaptés à $\eR$ et $\eR^m$. Il en sera de même pour les définitions de fonctions continues : il y aura une définition pour la continuité de fonctions entre espaces vectoriels normés, et ensuite une définition pour les fonctions de $\eR^m$ dans $\eR^n$ qui en sera un cas particulier.
\end{remark}

Tentons de comprendre ce que signifie qu'un nombre $\ell$ \emph{ne soit pas} la limite de $f$ lorsque $x\to a$. Il s'agit d'inverser la condition \eqref{Eq2807CondiionLimifnm}. Le nombre $\ell$ n'est pas une limite de $f$ pour $x\to a$ lorsque
\begin{equation}		\label{EqCaractNonLim}
	\exists\varepsilon>0\tq\,\forall\delta>0,\,\exists x\tq 0<\| x-a \|<\delta\text{ et }\| f(x)-\ell \|>\varepsilon,
\end{equation}
c'est à dire qu'il existe un certain seuil $\varepsilon$ tel qu'on a beau s'approcher aussi proche qu'on veut de $a$ (distance $\delta$), on trouvera toujours un $x$ tel que $f(x)$ n'est pas $\varepsilon$-proche de $\ell$.

\begin{lemma}[Unicité de la limite]
	Si $\ell$ et $\ell'$ sont deux limites de $f(x)$ lorsque $x$ tend vers $a$, alors $\ell=\ell'$.
\end{lemma}

\begin{proof}
	Soit $\varepsilon>0$. Nous considérons $\delta$ tel que $\| f(x)-\ell \|<\varepsilon$ pour tout $x$ tel que $\| x-a \|<\delta$. De la même manière, nous prenons $\delta'$ tel que $\| x-a \|<\delta'$ implique $\| f(x)-\ell' \|<\varepsilon$. Pour les $x$ tels que $\| x-a \|$ est plus petit que $\delta$ et $\delta'$ en même temps, nous avons
	\begin{equation}
		\| \ell-\ell' \|=\| \ell-f(x)+f(x)-\ell' \|\leq\| \ell-f(x) \|+\| f(x)-\ell' \|<2\varepsilon,
	\end{equation}
	et donc $\| \ell-\ell' \|=0$ parce que c'est plus petit que $2\varepsilon$ pour tout $\varepsilon$.
\end{proof}

Le concept de limite appelle immédiatement celui de continuité.
\begin{definition}
	Soit $f\colon D\subset\eR^m\to \eR^n $ et $a\in D$. On dit que $f$ est \defe{continue}{continuité} en $a$ lorsque la limite $\lim_{x\to a} f(x)$ existe et est égale à $f(a)$.

	On dit que $f$ est continue sur une partie $A\subset D$ si elle est continue en tous les points de $a$.
\end{definition}

La continuité peut évidement être récrite avec une formule du même type que celle de la limite.
\begin{proposition}
	La fonction $f\colon D\subset\eR^m\to \eR^n$ est continue en $a\in D$ si et seulement si
	\begin{equation}
		\forall\varepsilon,\,\exists\delta>0\tq x\in D\cap B(a,\delta)\Rightarrow \| f(x)-f(a) \|<\varepsilon.
	\end{equation}
\end{proposition}


\begin{theorem}
	Une fonction $f$ de $\eR^m$ vers $\eR^n$ est continue si et seulement si pour tout ouvert $\mO$ de $\eR^n$, l'image inverse $f^{-1}(\mO)$ est ouverte dans $\eR^m$.
\end{theorem}

\begin{proof}
	Ce théorème est un cas particulier du théorème \ref{ThoContiueImageInvOUvert}. Il suffit de remplacer $V$ par $\eR^m$ et $W$ par $\eR^n$.
\end{proof}

Quasiment toutes les propriétés des limites ont un équivalent concernant la continuité.
\begin{proposition}	\label{PropLimParcompos}
	Soit $f\colon D\subset\eR^m\to \eR^n$. Nous avons 
	\begin{equation}
		\lim_{x\to a} f(x)=\ell
	\end{equation}
	si et seulement si 
	\begin{equation}
		\lim_{x\to a} f_i(x)=\ell_i
	\end{equation}
	pour tout $i\in\{ 1,\ldots,n \}$ où $f_i(x)$ dénote la $i$-ème composante de $f(x)$ et $\ell_i$ la $i$-ème composante de $\ell\in\eR^n$.
\end{proposition}
Cette proposition revient à dire que la convergence d'une fonction est équivalente à la convergence de chacune de ses composantes.

\begin{proof}
	L'élément clef de la preuve est le fait que pour tout vecteur $u\in\eR^p$, nous ayons l'inégalité
	\begin{equation}	\label{Equilequnorme}
		| u_i |\leq\sqrt{\sum_{k=1}^p| u_k |^2}=\| u \|.
	\end{equation}
	La norme (dans $\eR^p$) d'un vecteur est plus grande ou égale à la valeur absolue de chacune de ses composantes.

	Supposons que nous ayons une fonction dont chacune des composantes a une limite en $a$ : $\lim_{x\to a} f_i(x)=\ell_i$. Montrons que dans ce cas la fonction $f$ tend vers $\ell$. Si nous considérons $\varepsilon>0$, par définition de la limite de chacune des fonctions $f_i$, il  existent des $\delta_i$ tels que
	\begin{equation}
		\| x-a \|_{\eR^m}<\delta_i\Rightarrow | f_i(x)-\ell_i |<\varepsilon.
	\end{equation}
	Notez que la norme à gauche est une norme dans $\eR^m$ et que celle à droite est une simple valeur absolue dans $\eR$. Considérons $\delta=\min\{ \delta_i \}_{i=1,\ldots n}$. Si $\| x-a \|<\delta$, alors
	\begin{equation}
		\| f(x)-\ell \|=\sqrt{\sum_{i=1}^n| f_i(x)-\ell_i |^2}<\sqrt{\sum_{i=1}^n\varepsilon^2}=\sqrt{n\varepsilon^2}=\sqrt{n}\varepsilon.
	\end{equation}
	Nous voyons qu'en choisissant les $\delta_i$ tels que $| f_i(x)-\ell_i |<\varepsilon$, nous trouvons $\| f(x)-\ell \|<\sqrt{n}\varepsilon$. Afin d'obtenir $\| f(x)-\ell \|<\varepsilon$, nous choisissons donc les $\delta_i$ de telle manière a avoir $| f_i(x)-\ell_i |<\varepsilon/\sqrt{n}$.

	Nous avons donc prouvé que la limite composante par composante impliquait la limite de la fonction. Nous devons encore prouver le sens inverse.

	Supposons donc que $\lim_{x\to a} f(x)=\ell$, et prouvons que nous ayons $\lim_{x\to a} f_i(x)=\ell_i$ pour chaque $i$. Soit $\varepsilon>0$ et $\delta>0$ tel que $\| x-a \|<\delta$ implique $\| f(x)-\ell \|<\varepsilon$. Avec ces choix, nous avons
	\begin{equation}
		| f_i(x)-\ell_i |\leq\| f(x)-\ell \|<\varepsilon
	\end{equation}
	où nous avons utilisé la majoration \eqref{Equilequnorme} avec $f(x)-\ell$ en guise de $u$.
\end{proof}

De même, pour la continuité nous avons la proposition suivante :
\begin{proposition}
	Soit une fonction $f\colon D\subset\eR^m\to \eR^n$ et $a\in D$. La fonction $f$ est continue en $a$ si et seulement si chacune de ses composantes l'est, c'est à dire si et seulement si chacune des fonctions $f_i\colon D\to \eR$ est continue en $a$.
\end{proposition}
Essayez de prouver cette proposition directement par la définition de la continuité, en suivant pas à pas la démonstration de la proposition \ref{PropLimParcompos}.

\begin{proposition}		\label{Propfaposfxposcont}
	Soit $f\colon \eR^m\to \eR$ et $a$, un point du domaine de $f$ telle que $f(a)>0$. Alors il existe un rayon $r$ tel que $f(x)>0$ pour tout $x$ dans $B(a,r)$.
\end{proposition}
Cette proposition signifie que si la fonction est strictement positive en un point, alors elle restera strictement positive en tous les points «pas trop loin».

\begin{proof}
	Prenons $\varepsilon=f(a)/2$ dans la définition de la continuité. Il existe donc un rayon $\delta$ tel que pour tout $x$ dans $B(a,\delta)$,
	\begin{equation}
		| f(x)-f(a) |\leq \frac{ f(a) }{2},
	\end{equation}
	en d'autres termes, $f(x)\in B\big( f(a),\frac{ f(a) }{ 2 } \big)$. Évidement aucun nombre négatif ne fait partie de cette dernière boule lorsque $f(a)$ est strictement positif.
\end{proof}

\begin{corollary}		\label{CorfneqzOuvert}
	Si $f\colon \eR^m\to \eR$ est une fonction continue, alors l'ensemble
	\begin{equation}
		A=\{ x\in\eR^m\tqs f(x)\neq 0 \}
	\end{equation}
	est ouvert.
\end{corollary}

\begin{proof}
	Soit $x\in A$. Si $x>0$ (le cas $x<0$ est laissé en exercice), alors il existe une boule autour de $x$ sur laquelle $f$ reste strictement positive (proposition \ref{Propfaposfxposcont}). Cette boule est donc contenue dans $A$. Étant donné qu'autour de chaque point de $A$ nous pouvons trouver une boule contenue dans $A$, ce dernier est ouvert.
\end{proof}

\begin{example} \label{ExBNOQEWe}
    Soit  $GL_n(\eR)$ l'ensemble des matrices $n \times n$ inversibles.   Nous allons montrer que $GL_n(\eR)$ est un ouvert de $ \eR^{n^2}$. L'identification entre les vecteurs et les matrices consiste simplement à «déplier» la matrice pour en faire un vecteur. Par exemple, en dimension deux,
	\begin{equation}
		\begin{pmatrix}
			1	&	2	\\ 
			3	&	4	
		\end{pmatrix}\mapsto
		\begin{pmatrix}
			1	\\ 
			2	\\ 
			3	\\ 
			4	
		\end{pmatrix}\in\eR^4.
	\end{equation}
	En dimension $3$,
	\begin{equation}
		\begin{aligned}[]
			\begin{pmatrix}
				1	&	2	&	3	\\
				4	&	5	&	6	\\
				7	&	8	&	9
			\end{pmatrix}
			\mapsto
			\begin{pmatrix}
				1	\\ 
				2	\\ 
				3	\\ 
				4	\\ 
				5	\\ 
				6	\\ 
				7	\\ 
				8	\\ 
				9	
			\end{pmatrix}\in\eR^9.
		\end{aligned}
	\end{equation}
	
	Une matrice est inversible si et seulement si son déterminant est non nul. Or le déterminant est un polynôme en les composantes de la matrice. En dimension deux, nous avons
	\begin{equation}
		\det\begin{pmatrix}
			a	&	b	\\ 
			c	&	d	
		\end{pmatrix}=ad-bc,
	\end{equation}
	mais en écriture «dépliée», nous pouvons aussi bien écrire
	\begin{equation}
		\det\begin{pmatrix}
			a	\\ 
			b	\\ 
			c	\\ 
			d	
		\end{pmatrix}=ad-bc.
	\end{equation}
	En dimension $3$, le déterminant est donc un polynôme des $9$ variables qui apparaissent dans le vecteur «déplié». En général, dans $\eR^{n^2}$, nous considérons donc le polynôme $\det\colon \eR^{n^2}\to \eR$ qui à un vecteur $X\in\eR^{n^2}$ fait correspondre le déterminant de la matrice obtenue en «repliant» le vecteur $X$.

	Donc dans $\eR^{n^2}$, l'ensemble des matrices inversibles est donné par l'ensemble des vecteurs sur lesquels le polynôme $\det$ ne s'annule pas, c'est à dire
	\begin{equation}
		\{ X\in\eR^{n^2}\tqs \det(X)\neq 0 \}.
	\end{equation}
	Mais le déterminant est un polynôme, et donc une fonction continue. Cet ensemble est par conséquence ouvert par le corollaire \ref{CorfneqzOuvert}.
\end{example}



La proposition suivante montre que la limite peut «passer à travers» les fonctions continues.
\begin{proposition}[limite de fonction composée]		\label{PropLimCompose}
	Soit $f\colon \eR^n\to \eR^q$ et $g\colon \eR^m\to \eR^n$ telles que
	\begin{subequations}
		\begin{align}
			\lim_{x\to a} g(x)&= p		\label{EqLimCompHypa}\\
			\lim_{y\to p} f(y)&= q		\label{EqLimCompHypb}
		\end{align}
	\end{subequations}
	Alors nous avons $\lim_{x\to a} (f\circ g)(x)=q$. 
\end{proposition}

\begin{proof}
	Comme presque toute preuve à propos de limite ou de continuité, nous commençons par choisir $\varepsilon>0$. Nous devons montrer qu'il existe un $\delta$ tel que $\| x-a \|\leq \delta$ implique $\| f\big( g(x) \big)-q \|\leq \varepsilon$.

	La limite \eqref{EqLimCompHypb} impose l'existence d'un $\tilde\delta$ tel que $\| y-p \|\leq\tilde\delta$ implique $\| f(y)-q \|\leq\varepsilon$, tandis que la limite \eqref{EqLimCompHypa} donne un $\delta$ tel que $\| x-a \|\leq\delta$ implique $\| g(x)-p \|\leq\tilde\delta$ (nous avons pris $\tilde\delta$ en guise de $\varepsilon$ dans la définition de la limite pour $g$).

	Avec ces choix, si $\| x-a \|\leq \delta$, alors $\| g(x)-p \|\leq\tilde\delta$, et par conséquent,
	\begin{equation}
		\| f\big( g(x) \big)-q \|\leq\varepsilon,
	\end{equation}
	ce que nous voulions.
\end{proof}

De façon pragmatique, la proposition \ref{PropLimCompose} nous fournit une formule pour les limites de fonctions composée :
\begin{equation}		\label{Eqlimfgvomp}
	\lim_{x\to a} (f\circ g)(x)=\lim_{y\to \lim_{x\to a} g(x)}f(y)
\end{equation}
lorsque $f$ est continue.

\begin{remark}
	La formule \eqref{Eqlimfgvomp} ne peut pas être utilisée à l'envers. Il existe des cas où $\lim_{x\to a} (g\circ f)(x)=q$, et $\lim_{x\to a} f(x)=p$ sans pour autant avoir $\lim_{y\to q} g(y)=q$. Par exemple
	\begin{subequations}
		\begin{align}
			g(x)&=\begin{cases}
				2	&	\text{si $x\geq0$,}\\
				0	&	 \text{si $x<0$}\\
			\end{cases}\\
			f(x)&=| x |.
		\end{align}
	\end{subequations}
	Nous avons $(g\circ f)(x)=2$ pour tout $x$, ainsi que $\lim_{x\to 0} f(x)=0$, mais la limite $\lim_{y\to 0} g(y)$ n'existe pas.
\end{remark}


\begin{theorem}[Caractérisation de la limite par les suites]		\label{ThoLimSuite}
	Une fonction $f\colon D\subset\eR^m\to \eR^n$ admet une limite $\ell$ en un point d'accumulation $a$ de $D$ si et seulement si pour toute suite $(x_n)$ dans $D\setminus\{ a \}$ convergente vers $a$, la suite $\big( f(x_n) \big)$ dans $\eR^n$ converge vers $\ell$.
\end{theorem}

\begin{proof}
	Supposons d'abord que la fonction ait une limite $\ell$ lorsque $x\to a$, et considérons une suite $(x_n)$ dans $D\setminus\{ a \}$ convergente vers $a$. Nous devons montrer que la suite $y_n=f(x_n)$ converge vers $\ell$, c'est à dire que si nous choisissons $\varepsilon>0$ nous devons montrer qu'il existe un $N$ tel que $n>N$ implique $\| y_n-\ell  \|=\| f(x_n)-\ell \|<\varepsilon$. 
	
	Nous avons deux hypothèses. La première est la convergence de la fonction et la seconde est la convergence de la suite $(x_n)$. L'hypothèse de convergence de la fonction nous dit que (le $\varepsilon$ a déjà été choisit dans le paragraphe précédent)
	\begin{equation}
		\exists\delta\tq\,0<\| x-a \|<\delta\Rightarrow\| f(x)-\ell \|<\varepsilon.
	\end{equation}
	Une fois choisit ce $\delta$ qui «va avec» le $\varepsilon$ qui a été choisit précédemment, la définition de la convergence de la suite nous enseigne que
	\begin{equation}
		\exists N\tq n>N\Rightarrow\| x_n-a \|<\delta.
	\end{equation}
	Récapitulons ce que nous avons fait. Nous avons choisit un $\varepsilon$, et puis nous avons construit un $N$. Lorsque $n>N$, nous avons $\| x_n-a \|<\delta$. Mais alors, par construction de ce $\delta$, nous avons $\| f(x_n)-\ell \|<\varepsilon$. Au final, $n>N$ implique bien $\| y_n-\ell \|<\varepsilon$, ce qu'il nous fallait.

	Nous supposons maintenant que la fonction $f$ \emph{ne} converge \emph{pas} vers $\ell$, et nous allons construire une suite d'éléments $x_n$ qui converge vers $a$ sans que $(y_n)=f(x_n)$ ne converge vers $\ell$. La fonction $f$ vérifie la condition \eqref{EqCaractNonLim}. Nous prenons donc un $\varepsilon$ tel que $\forall \delta$, il existe un $x$ qui vérifie \emph{en même temps} les deux conditions
	\begin{subequations}
		\begin{numcases}{}
			0<\| x-a \|<\delta\\
			\| f(x)-\ell \|>\varepsilon.
		\end{numcases}
	\end{subequations}
	Un tel $x$ existe pour tout choix de $\delta$. Choisissons un $n$ arbitraire et $\delta=\frac{1}{ n }$. Nous nommons $x_n$ le $x$ correspondant à ce choix de $n$. La suite $(x_n)$ ainsi construite converge vers $a$ parce que 
	\begin{equation}
		\| x_n-a \|<\delta_n=\frac{1}{ n },
	\end{equation}
	donc dès que $n$ est grand, $\| x_n-a \|$ est petit. Mais la suite $y_n=f(x_n)$ ne converge pas vers $\ell$ parce que
	\begin{equation}
		\| f(x_n)-\ell \|>\varepsilon
	\end{equation}
	pour tout $n$. La suite $y_n$ ne s'approche donc jamais à moins d'une distance $\varepsilon$ de $\ell$.
\end{proof}

Nous avons déjà vu par le corollaire \ref{CorFHbMqGGyi} qu'une suite croissante et bornée était convergente. Il en va de même pour les fonctions.
\begin{proposition} \label{PropMTmBYeU}
    Si la fonction réelle \( f\colon I=\mathopen[ a , b [\to \eR\) est croissante et bornée, alors la limite
    \begin{equation}
        \lim_{x\to b} f(x)
    \end{equation}
    existe et est finie.
\end{proposition}

\begin{proof}

    Commençons par prouver que si \( (x_n)\) est une suite dans \( I\) convergent vers \( b\), alors \( f(x_n)\) est une suite convergente. Dans un second temps nous allons prouver que si \( (x_n)\) et \( (x'_n)\) sont deux suites convergent vers \( b\), alors les suites convergentes \( f(x_n)\) et \( f(x'_n)\) convergent vers la même limite. Alors le critère critère séquentiel de la limite d'une fonction conclura (proposition \ref{PropFnContParSuite}).

    Nous pouvons extraire de \( x_n\) une sous-suite croissante \( (x_{\alpha(n)})\). Alors la suite \( f\big( x_{\alpha(n)} \big)\) est une suite croissante et majorée, donc convergente par le corollaire \ref{CorFHbMqGGyi}\footnote{En gros nous sommes en train de dire que toute la théorie des fonctions convexes est un vulgaire corollaire de Bolzano-Weierstrass.}. Nommons \( \ell\) la limite et montrons qu'elle est aussi limite de \( f\) sur la suite originale.

    Pour tout \( \epsilon>0\), il existe \( K\) tel que si \( n>K\) alors \( \big| f\big( x_{\alpha(n)} \big)-\ell \big|<\epsilon\). Soit \( K'\) tel que pour tout \( n>K'\) nous ayons \( x_n>x_{\alpha(K')}\). Cela est possible parce que la suite est bornée par \( b\) et converge vers \( b\) : il suffit de prendre \( K'\) de telle sorte que \( | x_n-b |\leq | x_{\alpha(n)}-b |\). Si \( n>K'\) alors \( x_n>x_{\alpha(K)}\) et
    \begin{equation}
        f(x_n)\geq f(x_{\alpha(n)})\geq \ell-\epsilon;
    \end{equation}
    en résumé si \( n>K\) alors \( | f(x_n)-\ell |<\epsilon\). Cela prouve que \( f(x_n)\to\ell\).

    Soit maintenant une autre suite \( (x'_n)\) convergent également vers \( b\). Comme nous venons de le voir la suite \( f(x'_n)\) est convergente et nous nommons \( \ell'\) la limite. Si nous considérons \( (x''_n)\) la suite «alternée» (\( x_1,x'_1,x_2,x'_2,\cdots\)) alors nous avons encore une suite convergente vers \( b\) et donc \( f(x''_n)\to \ell'\).

    Mais étant donné que \( f(x_n)\) et \( f(x'_n)\) sont des sous-suites, elles doivent converger vers la même valeur. Donc \( \ell=\ell'=\ell''\).
\end{proof}

%TODO : écrire un truc sur la limite à gauche et la limite pour la topologie induite.

%---------------------------------------------------------------------------------------------------------------------------
\subsection{Règles simples de calcul}
%---------------------------------------------------------------------------------------------------------------------------

Les opérations simples passent à la limite, sauf la division pour laquelle il faut faire attention au dénominateur.
\begin{proposition}     \label{PropOpsSimplesLimites}
    Soient \( f\) et \( g\) deux fonctions telles que \( \lim_{x\to a} f(x)=\alpha\) et \( \lim_{x\to a} g(x)=\beta\). Alors
    \begin{enumerate}
        \item
            \( \lim_{x\to a} f(x)+g(x)=\alpha+\beta\),
        \item
            \( \lim_{x\to a} f(x)g(x)=\alpha\beta\),
        \item
            si il existe un voisinage de \( a\) sur lequel \( g\) ne s'annule pas, alors \( \lim_{x\to a} \frac{ f(x) }{ g(x) }=\frac{ \alpha }{ \beta }\).
    \end{enumerate}
\end{proposition}

%---------------------------------------------------------------------------------------------------------------------------
\subsection{Règle de l'étau}
%---------------------------------------------------------------------------------------------------------------------------

Une première façon de calculer la limite d'une fonction est de la «\wikipedia{en}{Squeeze_theorem}{coincer}» entre deux fonctions dont nous connaissons la limite. Le théorème, que nous acceptons sans démonstration, est le suivant :
\begin{theorem}		\label{ThoRegleEtau}
	Soit $\mO$, un ouvert de $\eR^m$ contenant le point $a$. Soient $f$, $g$ et $h$, trois fonctions définies sur $\mO$ (éventuellement pas en $a$ lui-même). Supposons que pour tout $x\in\mO$ (à part éventuellement $a$), nous ayons les inégalités
	\begin{equation}
		g(x)\leq f(x)\leq h(x).
	\end{equation}
	Supposons de plus que
	\begin{equation}
		\lim_{x\to a} g(x)=\lim_{x\to a} h(x)=\ell.
	\end{equation}
	Alors la limite $\lim_{x\to a} f(x)$ existe et vaut $\ell$.
\end{theorem}

Nous insistons sur le fait que les deux fonctions entre lesquelles nous coinçons $f$ doivent tendre vers \emph{la même} valeur.

Cette méthode est très pratique lorsqu'on a des fonctions trigonométriques qui se factorisent parce qu'elles sont toujours majorables par $1$.
\begin{example}
	Prouvons que la fonction $f(x)=x\sin(x)$ tend vers zéro lorsque $x$ tend vers $0$. D'abord, nous coinçons la fonction entre deux fonctions connues :
	\begin{equation}
		0\leq| x\sin(x) |=| x | |\sin(x) |\leq | x |.
	\end{equation}
	Donc $| x\sin(x) |$ est coincé entre $g(x)=0$ et $h(x)=| x |$. Ces deux fonctions tendent vers $0$ lorsque $x\to 0$, et donc $f(x)$ tend vers zéro.
\end{example}


\begin{example}
	Prouver la continuité en $(0,0)$ de la fonction
	\begin{equation}
		f(x,y)=\begin{cases}
			\frac{ x | y | }{ \sqrt{x^2+y^2} }	&	\text{si $(x,y)\neq (0,0)$}\\
			0	&	 \text{sinon.}
		\end{cases}
	\end{equation}
	Considérons une suite $(x_n,y_n)\in\eR^2$ qui tend vers $(0,0)$. Étant donné que $\frac{ | y | }{ \sqrt{x^2+y^2} }<1$ pour tout $x$ et $y$, nous avons
	\begin{equation}
		0\leq | f(x_n,y_n) |=\left| \frac{ x_n | y_n | }{ \sqrt{x_n^2+y_n^2} } \right| \leq | x_n |\to 0.
	\end{equation}
	Donc nous avons
	\begin{equation}
		\lim_{(x,y)\to(0,0)}f(x,y)=0=f(0,0),
	\end{equation}
	ce qui prouve que la fonction est continue en $(0,0)$ par la proposition \ref{PropFnContParSuite}. Nous avons utilisé la règle de l'étau (théorème \ref{ThoRegleEtau}).
\end{example}

Nous notons \( f\sim g\)\nomenclature[Y]{\( f\sim g\)}{fonctions ayant des limites équivalentes} pour \( x\to a\) lorsque \( \lim_{x\to a} \frac{ f(x) }{ g(x) }=1\). Cela signifie que \( f\) et \( g\) tendent vers la même limite, à la même vitesse. Par exemple nous avons \( \ln(1-x)\sim -x\) pour \( x\to 0\) parce que
\begin{equation}    \label{EqGICpOX}
    \lim_{x\to 0} -\frac{ \ln(1-x) }{ x }=\lim_{x\to 0} -\frac{ \frac{ -1 }{ 1-x } }{ 1 }=\lim_{x\to 0} \frac{1}{ 1-x }=1
\end{equation}
où nous avons utilisé la règle de l'Hospital.

%---------------------------------------------------------------------------------------------------------------------------
\subsection{Méthode des chemins}
%---------------------------------------------------------------------------------------------------------------------------

Lorsque la limite n'existe pas, il y a une façon en général assez simple de le savoir, c'est la \defe{méthode des chemin}{méthode!des chemins}.

\newcommand{\CaptionFigMethodeChemin}{Sur toute la droite $y=-x$, la fonction vaut $-1/2$, tandis que sur toute la droite $y=x/2$, elle vaut $\frac{2}{ 5 }$. Il est donc impossible que la fonction ait une limite en $(0,0)$, parce que dans toute boule autour de zéro, il y aura toujours un point de chacune de ces deux droites.}
	\input{Fig_MethodeChemin.pstricks}

\begin{example}		\label{ExFNExempleMethodeTrigigi}
	Considérons la fonction
	\begin{equation}
		f(x,y)=\frac{ xy }{ x^2+y^2 },
	\end{equation}
	et remarquons que, quelle que soit la valeur de $y$, cette fonction est nulle lorsque $x=0$. De la même manière, nous voyons que si $x=y$, alors la fonction vaut\footnote{En fait ce que nous sommes en train de faire est de poser $\theta=\pi/2$ et $\theta=\pi/4$ dans \eqref{Eq2807fpolairerhodeuxcossin}.} $\frac{ 1 }{2}$. 

	Il est impossible que la fonction ait une limite en $(0,0)$ parce qu'on ne peut pas trouver un $\ell$ dont on s'approche à la fois en suivant la ligne $x=0$ et la ligne $x=y$.

	Deux autres chemins avec encore deux autres valeurs sont dessinés sur la figure \ref{LabelFigMethodeChemin}.

\end{example}

Nous pouvons formaliser cet exemple en utilisant le théorème \ref{ThoLimSuite}. Considérons les deux suites $x_n=(0,\frac{1}{ n })$ et $y_n=(\frac{1}{ n },\frac{1}{ n })$. Ce sont deux suites dans $\eR^2$ qui tendent vers $(0,0)$. Si la fonction $f$ convergeait vers $\ell$, alors nous aurions au moins
\begin{subequations}\label{Eq3007Lixxyyell}
	\begin{align}
		\lim f(x_n)&=\ell\\
		\lim f(y_n)&=\ell,
	\end{align}
\end{subequations}
mais nous savons que pour tout $n$, $f(x_n)=f(0,\frac{1}{ n })=0$ et $f(y_n)=f(\frac{1}{ n },\frac{1}{ n })=\frac{1}{ 2 }$. Il n'y a donc aucun nombre $\ell$ qui vérifie les deux équations \eqref{Eq3007Lixxyyell} parce que $\lim f(x_n)=0$ et $\lim f(y_n)=\frac{ 1 }{2}$.

Tout ceci est formalisé et généralisé dans la proposition suivante.
\begin{proposition}
	Soit $f\colon D\subset\eR^m\to \eR^n$ et $a$ un point d'adhérence de $D$. Alors nous avons
	\begin{equation}
		\lim_{x\to a} f(x)=\ell
	\end{equation}
	si et seulement si pour toute fonction $\gamma\colon \eR\to \eR^m$ telle que $\lim_{t\to 0} \gamma(t)=a$, nous avons
	\begin{equation}
		\lim_{t\to 0} (f\circ\gamma)(t)=\ell.
	\end{equation}	
\end{proposition}

\begin{corollary}	\label{CorMethodeChemin}
	Soient $f\colon D\subset\eR^m\to \eR^n$ et $a$ un point d'accumulation de $D$. Si nous avons deux fonctions $\gamma_1,\gamma_2\colon \eR\to \eR^m$ telles que
	\begin{equation}
		\lim_{t\to 0} \gamma_1(t)=\lim_{t\to 0} \gamma_2(t)=a
	\end{equation}
	tandis que
	\begin{equation}
		\lim_{t\to 0} (f\circ \gamma_1)(t)\neq\lim_{t\to 0} (f\circ \gamma_2)(t),
	\end{equation}
	ou bien que l'une des deux limites n'existe pas, alors la limite de $f(x)$ lorsque $x\to a$ n'existe pas.
\end{corollary}

\begin{corollary}	\label{CorMethodeChemoinNegatif}
	Soient $f\colon D\subset\eR^m\to \eR^n$ et $a$ un point d'accumulation de $D$. Si il existe une fonction $\gamma\colon \eR\to \eR^m$ avec $\gamma(0)=a$ telle que la limite $\lim_{t\to 0} (f\circ\gamma)(t)$ n'existe pas, alors la limite $\lim_{x\to a} f(x)$ n'existe pas.
\end{corollary}

En ce qui concerne le calcul de limites, la méthode des chemins peut être utilisé de trois façons :
\begin{enumerate}
	\item
		Dès que l'on trouve une fonction $\gamma\colon \eR\to \eR^m$ telle que $\lim_{t\to 0} (f\circ \gamma)(t)=\ell$, alors nous savons que \emph{si la limite $\lim_{x\to a} f(x)$ existe}, alors cette limite vaut $\ell$.
	\item
		Dès que l'on a trouvé deux fonctions $\gamma_i$ qui tendent vers $a$, mais dont les limites de $\lim_{t\to 0} (f\circ\gamma_i)(t)$ sont différentes, alors la limite $\lim_{x\to a} f(x)$ n'existe pas.
	\item
		Dès qu'on trouve une chemin le long duquel il n'y a pas de limite, alors la limite n'existe pas (corollaire \ref{CorMethodeChemoinNegatif}).
\end{enumerate}
La méthode des chemins ne permet donc pas de de calculer une limite quand elle existe. Elle permet uniquement de la «deviner», ou bien de prouver que la limite n'existe pas.

\begin{example}
	Soit à calculer
	\begin{equation}	\label{Eq3007ExempleLimiche}
		\lim_{(x,y)\to(0,0)}\frac{ x-y }{ x+y }.
	\end{equation}
	Si nous prenons le chemin $\gamma_1(t)=(t,t)$, nous avons bien $\lim_{t\to 0} \gamma_1(t)=(0,0)$, et nous avons
	\begin{equation}
		\lim_{t\to 0} (f\circ\gamma_1)(t)=\lim_{t\to 0} \frac{ t-t }{ t+t }=0.
	\end{equation}
	Donc si la limite \eqref{Eq3007ExempleLimiche} existait, elle vaudrait obligatoirement $0$. Mais si nous considérons $\gamma_2(t)=(0,t)$, nous avons
	\begin{equation}
		(f\circ\gamma_2)(t)=\frac{ -t }{ t }=-1,
	\end{equation}
	donc si la limite existe, elle doit obligatoirement valoir $-1$. Ne pouvant être égale à $0$ et à $-1$ en même temps, la limite \eqref{Eq3007ExempleLimiche} n'existe pas.
\end{example}

%---------------------------------------------------------------------------------------------------------------------------
\subsection{Méthode des coordonnées polaires}
%---------------------------------------------------------------------------------------------------------------------------

La proposition suivante exprime la définition de la limite en d'autres termes, et va être pratique dans le calcul de certaines limites.
\begin{proposition}		\label{PropMethodePolaire}
	Soit $f\colon D\subset\eR^m\to \eR^n$, $a$ un point d'accumulation de $D$ et $\ell\in \eR^n$. Nous définissons
	\begin{equation}
		E_r=\{ f(x)\tq x\in B(a,r)\cap D \},
	\end{equation}
	et
	\begin{equation}
		s_r=\sup\{ \| v-\ell \|\tq v\in E_r \}.
	\end{equation}
	Alors nous avons $\lim_{x\to a} f(x)=\ell$ si et seulement si $\lim_{r\to 0} s_r=0$.
\end{proposition}

Dans cette proposition, $E_r$ représente l'ensemble des valeurs atteintes par $f$ dans un rayon $r$ autour de $a$. Le nombre $s_r$ sélectionne, parmi toutes ces valeurs, celle qui est la plus éloignée de $\ell$ et donne la distance. En d'autres termes, $s_r$ est la distance maximale entre $f(x)$ et $\ell$ lorsque $x$ est à une distance au maximum $r$ de $a$.

Lorsque nous avons affaire à une fonction $f\colon \eR^2\to \eR$, cette proposition nous permet de calculer facilement les limites en passant aux coordonnées polaires.

\begin{example}		\label{ExempleMethodeTrigigi}
	Reprenons la fonction de l'exemple \ref{ExFNExempleMethodeTrigigi}:
	\begin{equation}
		f(x,y)=\frac{ xy }{ x^2+y^2 }.
	\end{equation}
	Son domaine est $\eR^2\setminus\{ (0,0) \}$. Nous voulons calculer $\lim_{(x,y)\to(0,0)}f(x,y)$. Écrivons la définition de $E_r$~:
	\begin{equation}
		E_r=\{ f(x,y)\tq (x,y)\in B\big( (0,0),r \big) \}.
	\end{equation}
	Les points de la boule sont, en coordonnées polaires, les points de la forme $(\rho,\theta)$ avec $\rho<r$. La chose intéressante est que $f(\rho,\theta)$ est relativement simple (plus simple que la fonction départ). En effet en remplaçant tous les $x$ par $\rho\cos(\theta)$ et tous les $y$ par $\rho\sin(\theta)$, et en utilisant le fait que $\cos^2(\theta)+\sin^2(\theta)=1$, nous trouvons
	\begin{equation}		\label{Eq2807fpolairerhodeuxcossin}
		f(\rho,\theta)=\frac{ \rho^2\cos(\theta)\sin(\theta) }{ \rho^2 }=\cos(\theta)\sin(\theta).
	\end{equation}
	Cela signifie que
	\begin{equation}
		E_r=\{ \cos(\theta)\sin(\theta)\tq\theta\in\mathopen[ 0 , 2\pi [ \}.
	\end{equation}
	Prenons $\ell$ quelconque. Le nombre $s_r$ est le supremum des
	\begin{equation}
		\| \ell-\cos(\theta)\sin(\theta) \|
	\end{equation}
	lorsque $\theta$ parcours $\mathopen[ 0 , 2\pi \mathclose]$. Nous ne sommes pas obligés calculer la valeur exacte de $s_r$. Ce qui compte ici est que $s_r$ ne vaut certainement pas zéro, et ne dépend pas de $r$. Donc il est impossible d'avoir $\lim_{r\to 0} s_r=0$, et la fonction donnée n'a pas de limite en $(0,0)$.
\end{example}

Nous pouvons retenir cette règle pour calculer les limites lorsque $(x,y)\to(0,0)$ de fonctions $f\colon \eR^2\to \eR$ :
\begin{enumerate}
	\item
		passer en coordonnées polaires, c'est à dire remplacer $x$ par $\rho\cos(\theta)$ et $y$ par $\rho\sin(\theta)$;
	\item
		nous obtenons une fonction $g$ de $\rho$ et $\theta$. Si la limite $\lim_{r\to 0} g(r,\theta)$ n'existe pas ou dépend de $\theta$, alors la fonction n'a pas de limite. Si on peut majorer $g$ par une fonction ne dépendant pas de $\theta$, et que cette fonction a une limite lorsque $r\to 0$, alors cette limite est la limite de la fonction.
\end{enumerate}

La vraie difficulté de la technique des coordonnées polaire est de trouver le supremum de $E_r$, ou tout au moins de montrer qu'il est borné par une fonction qui a une limite qui ne dépend pas de $\theta$. Une de situations classiques dans laquelle c'est facile est lorsque la fonction se présente comme une fonction de $r$ multiplié par une fonction de $\theta$. 

\begin{example}		\label{Exemplexyxsqysq}
	Soit à calculer la limite
	\begin{equation}
		\lim_{(x,y)\to(0,0)}xy\left( \frac{ x^2-y^2 }{ x^2+y^2 }\right).
	\end{equation}
	Le passage aux coordonnées polaires donne
	\begin{equation}
		f(r,\theta)=r^2\sin\theta\cos\theta(\cos^2\theta-\sin^2\theta).
	\end{equation}
	Déterminer le supremum de cela est relativement difficile. Mais nous savons que de toutes façons, la quantité $\sin\theta\cos\theta(\cos^2\theta-\sin^2\theta)$ est bornée par $1$. Donc
	\begin{equation}
		\| f(r,\theta) \|\leq r^2.
	\end{equation}
	Maintenant la règle de l'étau montre que $\lim_{(x,y)\to(0,0)}f(x,y)$ est zéro.

	La situation vraiment gênante serait celle avec une fonction de $\theta$ qui risque de s'annuler dans un dénominateur.
\end{example}

\begin{example}
	Soit à calculer
	\begin{equation}
		\lim_{(x,y)\to(0,0)}\frac{ x^2+y^2 }{ x-y }.
	\end{equation}
	Le passage en polaires donne
	\begin{equation}
		f(r,\theta)=\frac{ r^2 }{ r\big( \cos(\theta)-\sin(\theta) \big) }=\frac{ r }{ \cos(\theta)-\sin(\theta) }.
	\end{equation}
	Certes \emph{pour chaque $\theta$} nous avons $\lim_{r\to 0} f(r,\theta)=0$, mais il ne faut pas en déduire trop vite que la limite $\lim_{(x,y)\to(0,0)}f(x,y)$ vaut zéro parce que prendre la limite $r\to 0$ avec $\theta$ fixé revient à prendre la limite le long de la droite d'angle $\theta$.

	Il n'est pas possible de majorer $f(r,\theta)$ par une fonction ne dépendant pas de $\theta$ parce que cette fonction tend vers l'infini lorsque $\theta\to\pi/4$. Est-ce que cela veut dire que la limite n'existe pas ? Cela veut en tout cas dire que la méthode des coordonnées polaires ne parvient pas à résoudre l'exercice. Pour conclure, il faudra encore un peu travailler.

    Nous pouvons essayer de calculer le long d'un chemin plus général \( (r(t),\theta(t))\). Choisissons \( r(t)=t\) puis cherchons \( \theta(t)\) de telle sorte à avoir 
    \begin{equation}        \label{EqICrDSe}
        \cos\theta(t)-\sin\theta(t)=t^2.
    \end{equation}
    Le mieux serait de résoudre cette équation pour trouver \( \theta(t)\). Mais en réalité il n'est pas nécessaire de résoudre : montrer qu'il existe une solution suffit. Nous pouvons supposer que \( t^2<1\). Pour \( \theta=\pi/4\) nous avons \( \cos(\theta)-\sin(\theta)=0\) et pour \( \theta=0\) nous avons \( \cos(\theta)-\sin(\theta)=1\). Le théorème des valeurs intermédiaires nous enseigne alors qu'il existe une valeur de \( \theta\) qui résout l'équation \eqref{EqICrDSe}.

    %TODO : le citer lorsque le fork sera fait.
    Pour être rigoureux, nous devons aussi montrer que la fonction \( \theta(t)\) est continue. Pour cela il faudrait utiliser le \wikipedia{fr}{Théorème_des_fonctions_implicites}{théorème de la fonction implicite}. Nous verrons dans l'exemple \ref{ExmeASDLAf} comment s'en sortir sans théorème de la fonction implicite, au prix de plus de calculs.
\end{example}

\begin{example}
	Considérons encore la fonction 
	\begin{equation}
		f(x,y)=\frac{ x^2+y^2 }{ x-y }.
	\end{equation}
	Une mauvaise idée pour prouver que la limite n'existe pas pour $(x,y)\to(0,0)$ est de considérer le chemin $(t,t)$. En effet, la fonction n'existe pas sur ce chemin. Or la méthode des chemins parle uniquement de chemins contenus dans le domaine de la fonction.
\end{example}

\begin{example}     \label{ExmeASDLAf}
	Revenons encore et toujours sur la fonction 
	\begin{equation}
		(x^2+y^2)/(x-y).
	\end{equation}
	Nous prouvons que la limite n'existe pas en trouvant des chemins le long desquels les limites sont différentes. Si nous essayons le chemin \( (t,kt)\) avec \( k\) constant, nous trouvons
    \begin{equation}
        f(t,kt)=\frac{ t(1+k^2) }{ 1-k }.
    \end{equation}
    La limite \( t\to 0\) est hélas toujours \( 0\). Nous ne pouvons donc pas conclure.

    Nous allons maintenant utiliser la même technique que celle utilisée en coordonnées polaires. Vous noterez que dans ce cas, travailler en cartésiennes donne lieu à des calculs plus longs.  L'astuce consiste à prendre \( k\) non constant et à chercher par exemple \( k(t)\) de façon à avoir
    \begin{equation}
        \frac{ 1+k(t)^2 }{ 1-k(t) }=\frac{1}{ t }.
    \end{equation}
    Avec une telle fonction, la fonction \( t\mapsto f(t,tk(t))\) serait la constante \( 1\). L'équation à résoudre pour \( k\) est
    \begin{equation}
        tk^2+k+(t-1)=0,
    \end{equation}
    et les solutions sont
    \begin{equation}
        k(t)=\frac{ -1\pm\sqrt{1-4t(t-1)} }{ 2t }.
    \end{equation}
    Nous proposons donc les chemins
    \begin{equation}
        \begin{pmatrix}
            x    \\ 
            y    
        \end{pmatrix}=\begin{pmatrix}
            t    \\ 
            \frac{ -1\pm\sqrt{1-4t(t-1)}    }{2}
        \end{pmatrix}
    \end{equation}
    Nous devons vérifier deux points. D'abord que ce chemin est bien défini, et ensuite que \( tk(t)\) tend bien vers zéro lorsque \( t\to 0\) (sinon \( (t,k(t)t)\)) n'est pas un chemin passant par \( (0,0)\). Lorsque \( t\) est petit, ce qui se trouve sous la racine est proche de \( 1\) et ne pose pas de problèmes. Ensuite,
    \begin{equation}
        \lim_{t\to 0} tk(t)=\frac{ -1\pm 1 }{ 2 }.
    \end{equation}
    En choisissant le signe \( +\), nous trouvons un chemin qui nous convient. 

    Ce que nous avons prouvé est que
    \begin{equation}
        f\left( t,   \frac{ -1\pm\sqrt{1-4t(t-1)}    }{2}\right)=1
    \end{equation}
    pour tout \( t\). Le long de ce chemin, la limite de \( f\) est donc \( 1\). Cette limite est différente des limites obtenues le long de chemins avec \( k\) constant. La limite \( \lim_{(x,y)\to (0,0)} f(x,y)\) n'existe donc pas.
\end{example}

\begin{example}\label{seno}
	Considérons la fonction (figure \ref{LabelFigsenotopologo})

	\begin{equation}
		f(x,y)=\begin{cases}
			\sqrt{x^2+y^2}\sin\frac{1}{ x^2+y^2 }	&	\text{si $(x,y)\neq(0,0)$}\\
			0	&	 \text{si $(x,y)=(0,0)$},
		\end{cases}
	\end{equation}
	et cherchons la limite $(x,y)\to(0,0)$. Le passage en coordonnées polaires donne
	\begin{equation}		\label{EqFoncRho2907}
		f(\rho,\theta)=\rho\sin\frac{1}{ \rho }.
	\end{equation}
	Pour calculer la limite de cela lorsque $\rho\to 0$, nous remarquons que
	\begin{equation}
		0\leq|\rho\sin\frac{1}{ \rho }|\leq\rho
	\end{equation}
	parce que $\sin(\frac{1}{ \rho })\leq 1$ quel que soit $\rho$. Or évidement $\lim_{\rho\to 0} \rho=0$, donc la limite de la fonction \eqref{EqFoncRho2907} est zéro et ne dépend pas de $\theta$. Nous en concluons que $\lim_{(x,y)\to(0,0)}f(x,y)=0$.
\end{example}
\newcommand{\CaptionFigsenotopologo}{La fonction de l'exemple \ref{seno}.}
\input{Fig_senotopologo.pstricks}

%---------------------------------------------------------------------------------------------------------------------------
\subsection{Méthode du développement asymptotique}
%---------------------------------------------------------------------------------------------------------------------------

Nous savons que nous pouvons développer certaines fonctions en série grâce au développement de Taylor (théorème \ref{ThoTaylor}). Lorsque nous avons une limite à calculer, nous pouvons remplacer certaines parties de la fonction à traiter par la formule \eqref{subeqfTepseqb}. Cela est très utile pour comparer des fonctions trigonométrique à des polynômes.

\begin{example}		\label{ExamLimSinxxa}
	La limite $\lim_{x\to 0} \frac{ \sin(x) }{ x }=1$ est bien connue. Une manière de la prouver des d'écrire
	\begin{equation}
		\sin(x)=x+h(x)
	\end{equation}
	avec $h\in o(x)$, c'est à dire $\lim_{x\to 0} h(x)/x=0$. Alors nous avons
	\begin{equation}
		\lim_{x\to 0} \frac{ \sin(x) }{ x }=\lim_{x\to 0} \frac{ x+h(x) }{ x }=\lim_{x\to 0} \frac{ x }{ x }+\lim_{x\to 0} \frac{ h(x) }{ x }=1.
	\end{equation}
\end{example}

L'utilisation de la proposition \ref{PropLimCompose} permet d'utiliser cette technique dans le cadre de limites à plusieurs variables. Reprenons l'exemple \ref{ExamLimSinxxa} un tout petit peu modifié :

\begin{example}
	Soit à calculer $\lim_{(x,y)\to(0,0)}f(x,y)$ où
	\begin{equation}
		f(x,y)=\frac{ \sin(xy) }{ xy }.
	\end{equation}
	La première chose à faire est de voir $f$ comme la composée de fonctions $f=f_1\circ f_2$ avec
	\begin{equation}
		\begin{aligned}
			f_1\colon \eR&\to \eR \\
			t&\mapsto \frac{ \sin(t) }{ t } 
		\end{aligned}
	\end{equation}
	et
	\begin{equation}
		\begin{aligned}
			f_2\colon \eR^2&\to \eR \\
			(x,y)&\mapsto xy. 
		\end{aligned}
	\end{equation}
	 Étant donné que $\lim_{(x,y)\to(0,0)}f_2(x,y)=0$, nous avons $\lim_{(x,y)\to(0,0)}f(x,y)=\lim_{t\to 0} f_1(t)=1$.
\end{example}

