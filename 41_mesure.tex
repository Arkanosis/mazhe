% This is part of Mes notes de mathématique
% Copyright (c) 2011-2015
%   Laurent Claessens, Carlotta Donadello
% See the file fdl-1.3.txt for copying conditions.

%+++++++++++++++++++++++++++++++++++++++++++++++++++++++++++++++++++++++++++++++++++++++++++++++++++++++++++++++++++++++++++ 
\section{Applications mesurables}
%+++++++++++++++++++++++++++++++++++++++++++++++++++++++++++++++++++++++++++++++++++++++++++++++++++++++++++++++++++++++++++

%--------------------------------------------------------------------------------------------------------------------------- 
\subsection{Propriétés}
%---------------------------------------------------------------------------------------------------------------------------

\begin{definition}[Fonction mesurable] \label{DefQKjDSeC}
    Soit \( (E,\tribA)\) et \( (F,\tribF)\) deux espaces mesurés. Une fonction \( f\colon E\to F\) est \defe{mesurable}{mesurable!fonction} si pour tout \( \mO\in \tribF\), l'ensemble \( f^{-1}(\mO)\) est dans \( \tribA\).
\end{definition}

\begin{definition}[Fonction borélienne]     \label{DefHHIBooNrpQjs}
    Une application \( f\colon (\Omega,\tribA)\to (\eR^d,\Borelien(\eR^d))\) est \defe{borélienne}{borélienne!fonction}\index{fonction!borélienne} si elle est mesurable, c'est à dire si pour tout \( B\in\Borelien(\eR^d)\) nous avons \( f^{-1}(A)\in\tribA\).

    Si rien n'est précisé, une application entre deux espaces topologiques est borélienne lorsqu'elle est mesurable en considérant la tribu borélienne sur \emph{les deux} espace.
\end{definition}
Si \( \tribA\) est une tribu sur un ensemble \( E\), nous notons \( m(\tribA)\)\nomenclature[P]{\( m(\tribA)\)}{Ensemble des fonctions \( \tribA\)-mesurables} l'ensemble des fonctions qui sont \( \tribA\)-mesurables.

Le plus souvent lorsque nous parlerons de fonctions \( f\colon X\to Y\) où \( Y\) est un espace topologique, nous considérons la tribu borélienne sur \( Y\). Ce sera en particulier le cas dans la théorie de l'intégration.

\begin{remark}
    Lorsque nous considérons des fonctions à valeurs réelles \( f\colon X\to \eR\) nous utiliserons toujours la tribu borélienne sur \( \eR\). Pour \( X\), cela peut dépendre des contextes. En théorie de l'intégration, nous mettrons sur \( X\) la tribu des ensembles mesurables au sens de Lebesgue sur \( X\), \emph{tout en gardant celle des boréliens sur l'ensemble d'arrivée}.

    Pour toute la partie sur l'intégration, une fonction \( f\colon \eR^n\to \eR^m\) sera mesurable si pour tout borélien \( A\) de \( \eR^m\) l'ensemble \( f^{-1}(A)\) est Lebesgue-mesurable dans \( \eR^n\).

    Étant donné qu'il est franchement difficile de créer des ensembles non mesurables au sens de Lebesgue, il est franchement difficile de créer des fonction non mesurables à valeurs réelles. L'hypothèse de mesurabilité est donc toujours satisfaite dans les cas pratiques.
\end{remark}

\begin{proposition}     \label{PROPooEFHKooARJBwW}
    Soient \( (S_i,\tribF_i)\) (\( i=1,2,3\)) des espaces mesurables et des fonctions mesurables \( f\colon S_1\to S_2\) et \( g\colon S_2\to S_3\). Alors la fonction \( g\circ f\colon S_1\to S_3\) est mesurable.
\end{proposition}

\begin{proof}
    Soit \( B\in\tribF_3\). Alors
    \begin{equation}
        (g\circ f)^{-1}(B)=f^{-1}\big( g^{-1}(B) \big)\in f^{-1}(\tribF_2)\subset\tribF_1.
    \end{equation}
\end{proof}

%--------------------------------------------------------------------------------------------------------------------------- 
\subsection{D'une tribu à l'autre}
%---------------------------------------------------------------------------------------------------------------------------


\begin{lemma}[\cite{TribuLi}]       \label{LemooVDXJooZNYelH}
    Soit une application \( f\colon S_1\to S_2\) et une tribu \( \tribF_2\) sur \( S_2\). Alors \( f^{-1}(\tribF_2)\) est une tribu sur \( S_1\)
\end{lemma}

\begin{proof}
    Il faut prouver les trois propriétés de la définition \ref{DefjRsGSy} d'une tribu.
    \begin{enumerate}
        \item
            D'abord \( f\) est définit sur tout \( S_1\), donc \( f^{-1}(S_2)=S_1\) alors que \( S_2\in \tribF_2\).
        \item
            Soit \( A\in f^{-1}(\tribF_2)\), c'est à dire \( A=f^{-1}(B)\) pour un certain \( B\in \tribF_2\). En ce qui concerne le complémentaire :
            \begin{equation}
                A^c=f^{-1}(B)^c=S_1\setminus f^{-1}(B)=f^{-1}(S_2\setminus B)=f^{-1}(B^c).
            \end{equation}
        \item
            Si \( (A_i)_{i\in \eN}\) sont des éléments de \( f^{-1}(\tribF_2)\) avec \( A_i=f^{-1}(B_i)\) alors
            \begin{equation}
                \bigcup_iA_i=\bigcup_if^{-1}(B_i)=f^{-1}\big( \bigcup_iB_i \big).
            \end{equation}
            Ce qui est dans la dernière parenthèse est dans \( \tribF_2\) parce que cette dernière est une tribu.
    \end{enumerate}
\end{proof}

\begin{lemma}[\cite{TribuLi}]       \label{LemJYKBooBSXBXJ}
    Soit une application \( f\colon S_1\to S_2\) et \( \tribF\) une tribu de \( S_1\). Alors
    \begin{enumerate}
        \item
            L'ensemble
            \begin{equation}
                \tribF_f=\{  B\subset S_2\tq f^{-1}(B)\in \tribF  \}
            \end{equation}
            est une tribu sur \( S_2\).
        \item
            C'est la plus grande tribu de \( S_2\) pour laquelle \( f\) est mesurable.
    \end{enumerate}
\end{lemma}

\begin{proof}
    Encore les trois propriétés à vérifier.
    \begin{enumerate}
        \item
            \( S_2\in\tribF\), sont \( S_1=f^{-1}(S_2)\in \tribF_f\).
        \item
            Si \( A\in \tribF_f\) alors \( A=f^{-1}(B)\) pour un certain \( B\in \tribF\). Nous avons alors aussi \( B^c\in \tribF\) et donc 
            \begin{equation}
                f^{-1}(B^c)=f^{-1}(B)^c=A^c.
            \end{equation}
            Par conséquent \( A^c\) est dans \( \tribF_f\).
        \item
            Si \( (A_i)\) sont des éléments de \( \tribF_f\) avec \( A_i=f^{-1}(B_i)\) pour \( B_i\in \tribF\) alors \( \bigcup_iB_i\in\tribF\) et
            \begin{equation}
                f^{-1}\big( \bigcup_iB_i \big)=\bigcup_if^{-1}(B_i)\in\tribF_f.
            \end{equation}
    \end{enumerate}
    En ce qui concerne la maximalité, si \( R\subset S_2\) n'est pas dans \( \tribF_f\) alors \( f^{-1}(R)\) n'est pas dans \( \tribF\) et donc \( f\) ne serait pas mesurable.
\end{proof}

\begin{definition}[Tribu engendrée] \label{DefNOJWooLGKhmJ}
    Soit une application \( f\colon S_1\to S_2\) et \( \tribF\) une tribu de \( S_1\). Alors conformément au lemme \ref{LemJYKBooBSXBXJ} l'ensemble
            \begin{equation}
                \tribF_f=\{  B\subset S_2\tq f^{-1}(B)\in \tribF  \}
            \end{equation}
            est la \defe{tribu engendrée}{tribu!engendrée!par une application}.
\end{definition}

\begin{lemma}[Lemme de transfert]       \label{LemOQTBooWGYuDU}
    Soit \( f\colon S_1\to S_2\) une application et une classe \( \tribC\) de parties de \( S_2\). Alors
    \begin{equation}
        \sigma\big( f^{-1}(\tribC) \big)=f^{-1}\big( \sigma(\tribC) \big).
    \end{equation}
\end{lemma}
\index{lemme!de transfert}

\begin{proof}
    Vu que \( \sigma(\tribC)\) es tune tribu, dans \( S_2\) alors le lemme \ref{LemJYKBooBSXBXJ} dit que \( f^{-1}\big( \sigma(\tribC) \big)\) est une tribu qui contient en particulier \(  f^{-1}(\tribC) \). Nous en déduisons que \( \sigma\big( f^{-1}(\tribC) \big)\subset f^{-1}\big( \sigma(\tribC) \big)\).

    Réciproquement. Dans \( S_1\) nous avons la tribu \( \sigma\big( f^{-1}(\tribC) \big)\). Nous pouvons alors considérer la tribu
    \begin{equation}
        \tribF_f=\{ B\subset S_2\tq f^{-1}(B)\in\sigma\big( f^{-1}(\tribC) \big) \}.
    \end{equation}
    Montrons que \( \tribC\subset \tribF_f\). Lorsque \( B\in \tribC\) nous avons \( f^{-1}(B)\in f^{-1}(\tribC)\subset\sigma\big( f^{-1}(\tribC) \big)\). Du coup \( B\in \tribF_f\). Nous avons alors, en passant aux tribus engendrées :
    \begin{equation}
        \sigma(\tribC)\subset\sigma(\tribF_f)=\tribF_f.
    \end{equation}
    Si maintenant \( B\in\sigma(\tribC)\), nous avons \( f^{-1}(B)\in \sigma\big( f^{-1}(\tribC) \big)\), ce qui signifie que
    \begin{equation}
        f^{-1}\big( \sigma(\tribC) \big)\subset\sigma\big( f^{-1}(\tribC) \big).
    \end{equation}
\end{proof}

Le théorème suivant est important pour prouver qu'une application est mesurable. En effet, il permet de ne tester si une application est mesurable uniquement sur une partie génératrice de la tribu d'arrivé\footnote{Typiquement les ouverts pour les boréliens.}.
\begin{theorem}     \label{ThoECVAooDUxZrE}
    Soient des espaces mesurables \( ( S_1,\tribF_1 )\) et \( (S_2,\tribF_2)\) ainsi qu'une application \( f\colon S_1\to S_2\). Si il existe un ensemble de parties \( \tribC\) de \( S_2\) tel que
    \begin{itemize}
        \item \( \sigma(\tribC)=\tribF_2\)
        \item \( f^{-1}(B) \in \tribF_1 \) pour tout \( B\in \tribC\)
    \end{itemize}
    alors \( f\) est mesurable.
\end{theorem}

\begin{proof}
    Par hypothèse, \( \sigma(\tribC)=\tribF_2\) et \( f^{-1}(\tribC)\subset \tribF_1\) et nous pouvons utiliser le lemme de transfert \ref{LemOQTBooWGYuDU} :
    \begin{equation}
        \sigma\big( f^{-1}(\tribC) \big)=f^{-1}\big( \sigma(\tribC) \big)
    \end{equation}
    qui s'écrit ici
    \begin{equation}
        \sigma\big( f^{-1}(\tribC) \big)=f^{-1}(\tribF_2).
    \end{equation}
    Mais vu que \( f^{-1}(\tribC)\subset \tribF_1\), nous avons aussi \( \sigma\big( f^{-1}(\tribC) \big)\subset \tribF_1\), ce qui signifie que
    \begin{equation}
        f^{-1}(\tribF_2)\subset \tribF_1.
    \end{equation}
    Cela est exactement le fait que \( f\) soit mesurable.
\end{proof}

Le théorème suivant est très important parce qu'en pratique c'est souvent lui, en conjonction avec la proposition \ref{PropooLNBHooBHAWiD} qui permet de déduire qu'une fonction est borélienne.
\begin{theorem}[\cite{TribuLi}]     \label{ThoJDOKooKaaiJh}
    Soient \( X\) et \( Y\) deux espaces topologiques. Alors toute application continue \( f\colon X\to Y\) est borélienne\footnote{Définition \ref{DefHHIBooNrpQjs}.}.
\end{theorem}

\begin{proof}
    Pour vérifier que \( f\) est borélienne, nous devons prouver que \( f^{-1}(B)\) est borélien pour tout borélien \( B\) de \( Y\). Heureusement, le théorème \ref{ThoECVAooDUxZrE} nous permet de limiter la vérification aux \( B\) appartenant à une classe engendrant les boréliens de \( Y\).

    La classe en question est toute trouvée : ce sont les ouverts. Si \( \mO\) est un ouvert de \( Y\) alors \( f^{-1}(\mO)\) est un ouvert de \( X\) et donc un borélien de \( X\).
\end{proof}

Le théorème suivant donne une importante compatibilité entre l'induction de tribu et l'induction de topologie : la tribu induite à partir des boréliens sur un sous-espace topologique est la tribu des boréliens pour la topologie induite.
\begin{theorem}[\cite{TribuLi}]     \label{ThoSVTHooChgvYa}
    Soit \( X\), un espace topologique et \( Y\subset X\) une partie munie de la topologie induite. Alors
    \begin{equation}
        \Borelien(Y)=\Borelien(X)_Y
    \end{equation}
    où \( \Borelien(X)_Y\) est la tribu sur \( Y\) induite de \( \Borelien(X)\) par la définition \ref{DefDHTTooWNoKDP}.
\end{theorem}

\begin{proof}
    Nous notons \( \tau_X\) et \( \tau_Y\) les topologies de \( X\) et \( Y\). 
    \begin{subproof}
        \item[\( \Borelien(Y)\subset\Borelien(X)_Y\)]
            Si \( A\in \tau_Y\) alors \( A=Y\cap \Omega\) pour un \( \Omega\in \tau_X\). Mais vu que \(\Omega\) est un ouvert de \( X\), il est un borélien de \( X\), ce qui donne que \( Y\cap\Omega\) est un élément de \( \Borelien(X)_Y\). Cela prouve que \( \tau_Y\subset\Borelien(X)_Y\), c'est à dire que \( \Borelien(X)_Y\) est une tribu sur \( Y\) contenant les ouverts de \( Y\). Nous avons donc
            \begin{equation}
                \Borelien(X)\subset\Borelien(X)_Y.
            \end{equation}
        \item[Réciproquement]
            L'application \( \id\colon (Y,\tau_Y)\to (X,\tau_X)\) est continue parce que si \( \Omega\) est ouvert de \( X\) alors \( \id^{-1}(\Omega)=\Omega\cap Y\in \tau_Y\). Par conséquent l'identité est une application borélienne (théorème \ref{ThoJDOKooKaaiJh}), ce qui signifie que \( \id^{-1}\big( \Borelien(X) \big)\subset\Borelien(Y)\), ou encore que si \( B\in\Borelien(X)\), alors \( \id^{-1}(B)=B\cap Y\in\Borelien(Y)\). Cela signifie que 
            \begin{equation}
                \Borelien(X)_Y\subset \Borelien(Y).
            \end{equation}
    \end{subproof}
\end{proof}

\begin{corollary}       \label{CorooMJQYooFfwoTd}
    Si \( U\) est un borélien de l'espace topologique \( X\), alors les boréliens de \( U\) sont les boréliens de \( X\) inclus à \( U\) :
    \begin{equation}
        \Borelien(U)=\{ B\in\Borelien(X)\tq B\subset U \}.
    \end{equation}
\end{corollary}

\begin{proof}
    Si \( B'\in\Borelien(U)\), le théorème \ref{ThoSVTHooChgvYa} donne un borélien \( B\in\Borelien(X)\) tel que \( B'=B\cap U\). Mais \( U\) étant borélien de \( X\), l'intersection \( B\cap U\) est encore un borélien de \( X\).
\end{proof}
Ce corollaire s'applique en particulier lorsque \( U\) est un ouvert.

%--------------------------------------------------------------------------------------------------------------------------- 
\subsection{Mesure image}
%---------------------------------------------------------------------------------------------------------------------------

Le produit d'une mesure par une fonction est définit par la propriété \ref{PropooVXPMooGSkyBo}.

\begin{propositionDef}[Mesure image\cite{TribuLi}]     \label{PropJCJQooAdqrGA}
    Soient \( (S_1,\tribF_1)\) et \( (S_2,\tribF_2)\) des espaces mesurables. Soit \( \varphi\colon S_1\to S_2\) une application mesurable. Si \( m_1\) est une mesure positive sur \( S_1\) alors l'application définie par
    \begin{equation}
        m_2(A_2)=m_1\big( \varphi^{-1}(A_2) \big)
    \end{equation}
    est une mesure positive sur \( (S_2,\tribF_2)\).

    La mesure \( m_2\) ainsi définie est la \defe{mesure image}{mesure!image} de \( m_1\) par l'application \( \varphi\). Elle est notée \( \varphi(m_1)\).
\end{propositionDef}

\begin{proof}
    Il y a deux choses à vérifier pour avoir une mesure positive\footnote{Définition \ref{DefBTsgznn}}. D'abord pour l'ensemble vide :
    \begin{equation}
        m_2(\emptyset)=m_1\big( \varphi^{-1}(\emptyset) \big)=m_1(\emptyset)=0.
    \end{equation}
    Ensuite pour l'additivité. Soient \( A_n\) dans \( \tribF_2\) des parties deux à deux disjointes et telles que \( \bigcup_nA_n\in\tribF_2\). Alors nous avons
    \begin{subequations}
        \begin{align}
            m_2\big( \bigcup_nA_n \big)&=m_1\Big( \varphi^{-1}(\bigcup_nA_n) \Big)\\
            =&m_1\big( \bigcup_n\varphi^{-1}(A_n) \big)\\
            &=\sum_nm_1\big( \varphi(A_n) \big)\\
            &=\sum_nm_2(A_n).
        \end{align}
    \end{subequations}
\end{proof}

\begin{lemma}
    Soient deux espaces mesurables \( (S_1,\tribF_1)\) et \( (S_2,\tribF_2)\) ainsi que deux mesures \( \mu\) et \( \nu\) sur \( (S_1,\tribF_1)\). Si \( \varphi\colon S_1\to S_2\) est mesurable et si \( \mu\leq \nu\) alors \( \varphi(\mu)\leq \varphi(\nu)\).
\end{lemma}

\begin{proof}
    Soit \( B\) mesurable dans \( (S_2,\tribF_2)\) (c'est à dire \( B\in \tribF_2\)). Alors
    \begin{equation}
        \varphi(\mu)(B)=\mu\big( \varphi^{-1}(B) \big)\leq\nu\big( \varphi^{-1}(B) \big)=\varphi(\nu)(B).
    \end{equation}
\end{proof}

Il est naturel de se demander comment il faut intégrer par rapport à une mesure image. La réponse sera dans le théorème \ref{THOooVADUooLiRfGK}.

%--------------------------------------------------------------------------------------------------------------------------- 
\subsection{Régularité d'une mesure}
%---------------------------------------------------------------------------------------------------------------------------

Certaines mesures ont de la compatibilité avec la topologie. Nous allons étudier ça.

\begin{theorem}[\cite{TribuLi}]     \label{ThoPKGEooVrpsGU}
    Soit \( X\) un espace métrique et \( m\) une mesure positive bornée sur \( \big(X,\Borelien(X)\big)\). Alors si \( B\) est un borélien,
    \begin{enumerate}
        \item
            Régularité extérieure : \( m(B)=\inf\{ m(\Omega)\text{où \( \Omega\) est un ouvert contenant \( B\)} \}\)
        \item
            Régularité intérieure : \( m(B)=\sup\{ m(F) \text{où \( F\) est un fermé, \( F\subset B\)} \}\).
    \end{enumerate}
\end{theorem}

\begin{proof}
    Soit \( \tribF\) l'ensemble des \( B\in\Borelien(X)\) tels que pour tout \( \epsilon>0\), il existe \( \Omega_{\epsilon}\) ouvert et \( F_{\epsilon}\) fermé tels que \( F_{\epsilon}\subset B\subset \Omega_{\epsilon}\) et \( m(\Omega_{\epsilon}\setminus F_{\epsilon})\leq \epsilon\). Nous allons montrer que cela est une tribu contenant les ouverts. Comme cela est inclus à la tribu borélienne, nous en déduirons que \( \tribF=\Borelien(X)\).
    \begin{subproof}
        \item[\( \tribF\) contient les ouverts]
            Soit \( \Omega\) un ouvert de \( X\). Alors \( \Omega^c\) est fermé et \( d(x,\Omega^c)=0\) si et seulement si \( x\in \Omega^c\) par la proposition \ref{PropGULUooNzqZKj}. Nous pouvons donc écrire
            \begin{equation}
                 \Omega^c=\bigcap_{n\geq 1}\{ x\in X\tq d(x,\Omega^c)<\frac{1}{ n } \}.
            \end{equation}
            En passant au complémentaire et en posant \( F_n=\{ x\in X\tq d(x,\Omega^c)\geq \frac{1}{ n } \}\) nous avons
            \begin{equation}
                \Omega=\bigcup_{n\geq 1}F_n.
            \end{equation}
            Chacun des \( F_n\) est fermé parce que \( F_n\) est l'image réciproque du fermé \( \mathopen[ \frac{1}{ n } , \infty \mathclose[\) par l'application \( x\mapsto d(x,\Omega^c)\) qui est continue. De plus les \( F_n\) forment une suite croissante, donc le lemme \ref{LemAZGByEs} nous assure que \( m(\Omega)=\lim_{n\to \infty}m(F_n)\). Et le lemme \ref{LemPMprYuC} que \( m(\Omega\setminus F_n)=m(\Omega)-m(F_n)\).
                
                Soit \( \epsilon>0\). Il existe alors \( n_{\epsilon}\geq 1\) tel que 
                \begin{equation}
                    m(\Omega\setminus F_n)=m(\Omega)-m(F_n)\leq \epsilon.
                \end{equation}
                Bref si \( \Omega\) est ouvert nous considérons \( \Omega_{\epsilon}=\Omega\) et \( F_{\epsilon}=F_{n_{\epsilon}}\) et nous avons
                \begin{equation}
                    F_{\epsilon}\subset \Omega\subset \Omega_{\epsilon}
                \end{equation}
                avec \( m(\Omega_{\epsilon}\setminus F_{\epsilon})\leq \epsilon\).

                L'ensemble \( \tribF\) contient les ouverts.

            \item[\( \tribF\) est une tribu]
                Il y a à vérifier les trois conditions de la définition \ref{DefjRsGSy}.
                \begin{subproof}
                \item[Les ensembles faciles]
                    Les ensembles \( X\) et \( \emptyset\) sont dans \( \tribF\) parce qu'ils sont ouverts et fermés.
                \item[Complémentaire]
                    Soit \( B\in \tribF\), soit \( \epsilon>0\) et les ensembles \( F_{\epsilon} \) et \( \Omega_{\epsilon}\) qui vont avec. Alors en passant au complémentaire nous avons
                    \begin{equation}
                        \Omega_{\epsilon}^c\subset B^c\subset F_{\epsilon}^c
                    \end{equation}
                    De plus
                    \begin{equation}
                        F_{\epsilon}^c\setminus \Omega_{\epsilon}^c=F_{\epsilon}^c\cap(\Omega_{\epsilon}^c)^c=F_{\epsilon}^c\cap \Omega_{\epsilon}=\Omega_{\epsilon}\setminus F_{\epsilon}.
                    \end{equation}
                    Par conséquent
                    \begin{equation}
                        m(F_{\epsilon}^c\setminus \Omega_{\epsilon}^c)=m(\Omega_{\epsilon}\setminus F_{\epsilon})\leq \epsilon.
                    \end{equation}
                    Cela montre que \( B^c\in \tribF\).
                \item[Union dénombrable]
                    Soient \( (B_n)\) une suite d'éléments de \( \tribF\) et \( \epsilon>0\). Pour chaque \( n\) nous choisissons un ouvert \( \Omega_n\) et un fermé \( F_n\) tels que \( F_n\subset  B_n\subset \Omega_n\) et 
                    \begin{equation}
                        m(\Omega_n\setminus F_n)\leq \frac{ \epsilon }{ 2^{n+2} }.
                    \end{equation}
                    Vu que \( \Omega_n\setminus B_n\subset \Omega_n\setminus F_n\) nous avons aussi
                    \begin{equation}
                        m(\Omega_n\setminus B_n)\leq m(\Omega_n\setminus F_n)\leq \frac{ \epsilon }{ 2^{n+2} }.
                    \end{equation}
                    Nous posons \( \Omega=\bigcup_{n\geq 1}\Omega_n\) (un ouvert) et \( B=\bigcup_{n\geq 1}B_n\) ainsi que \( A=\bigcup_{n\geq 1}F_n\) (qui n'est pas spécialement fermé).

                    Le but est de majorer \( m(\Omega\setminus F)\) où \( F\) est un fermé qui est encore à déterminer. Calculons déjà ceci :
                    \begin{subequations}
                        \begin{align}
                            \Omega\setminus B&=\bigcup_n\Omega_n\cap\big( \bigcup_kB_k \big)^c\\
                            &=\bigcup_n\Big( \Omega_n\cap\big( \bigcap_kB_k^c \big) \Big)\\
                            &\subset\bigcup_n\big( \Omega_n\cap B_n^c \big)\\
                            &=\bigcup_n(\Omega_n\setminus B_n)
                        \end{align}
                    \end{subequations}
                    où l'union n'est pas spécialement disjointe. Par conséquent,
                    \begin{equation}
                        m(\Omega\setminus B)\leq \sum_{n=1}^{\infty}m(\Omega_n\setminus B_n)\leq \sum_{n=1}^{\infty}\frac{ \epsilon }{ 2^{n+2} }=\frac{ \epsilon }{ 4 }.
                    \end{equation}
                    De la même façon nous avons
                    \begin{equation}
                        B\setminus A=\big( \bigcup_{n=1}^{\infty}B_n \big)\cap\big( \bigcup_{k=1}^{\infty}F_n \big)^c\subset \bigcup_{n=1}^{\infty}B_n\setminus F_n.
                    \end{equation}
                    Nous avons alors le inégalités de mesures
                    \begin{subequations}
                        \begin{align}
                            m(B\setminus A)&\leq \sum_{n=1}^{\infty}m(B_n\setminus F_n)\\
                            &\leq\sum_{n=1}^{\infty}m(\Omega_n\setminus F_n)\\
                            &\leq \frac{ \epsilon }{ 4 }.
                        \end{align}
                    \end{subequations}
                    C'est vraiment dommage que \( A\) ne soit pas en générale un fermé, sinon il répondrait à la question. Nous posons \( F'_1=F_1\) et \( F'_n=\bigcup_{k=1}^nF_k\). En tant qu'unions finies de fermés, les \( F'_n\) sont des fermés (lemme \ref{LemQYUJwPC}\ref{ItemKJYVooMBmMbG}). De plus la suite \( (F'_n)\)  est croissante et l'union est \( A\). Par le lemme \ref{LemAZGByEs}\ref{ItemJWUooRXNPci} nous avons
                    \begin{equation}
                        m(A)=m\big( \bigcup_nF'_n \big)=\lim_{n\to \infty} m(F'_n).
                    \end{equation}
                    Il existe donc \( n_{\epsilon}\) tel que 
                    \begin{equation}
                        m(A)-m(F'_{n})\leq \epsilon
                    \end{equation}
                    Nous posons \( F=F'_{n_{\epsilon}}\). Vu que \( F\subset A\) nous avons aussi \( m(A\setminus F)=m(A)-m(F)\leq \epsilon\). Et en plus \( F\subset A\subset B\subset \Omega\), ce qui donne bien la propriété voulue \( F\subset B\subset \Omega\). Il reste à nous assurer de \( m(\Omega\setminus F)\). Nous avons d'abord
                    \begin{equation}
                        m(B\setminus F)=m\big( (B\setminus A)\cup (A\setminus F) \big)=m(B\setminus A)+m(A\setminus F)\leq \frac{ 5\epsilon }{ 4 }.
                    \end{equation}
                    Et enfin :
                    \begin{equation}
                        m(\Omega\setminus F)=m\big( (\Omega\setminus B)\cup (B\setminus F) \big)=m(\Omega\setminus B)+m(B\setminus F)\leq \frac{ 6\epsilon }{ 4 }.
                    \end{equation}
                    Et donc à redéfinition près de \( \epsilon\) c'est d'accord. 
                    
                \end{subproof}

                Il est donc établi que \( \tribF\) est une tribu. Qui plus est, l'ensemble \( \tribF\) est une tribu incluse aux boréliens et contenant les ouverts. Ergo \( \tribF=\Borelien(X)\).

            \item[Régularité extérieure] 

                Soit \( B\) un borélien et \( \epsilon>0\). Alors il existe \( F_{\epsilon}\) fermé et \( \Omega_{\epsilon} \) ouvert tels que \( F_{\epsilon}\subset B\subset \Omega_{\epsilon}\) et \( m(\Omega_{\epsilon}\setminus F_{\epsilon})\leq \epsilon\). Vu que \( B\subset \Omega_{\epsilon}\) pour tout \( \epsilon\), nous avons aussi
                \begin{equation}
                    m(B)\leq \inf_{\epsilon}m(\Omega_{\epsilon}).
                \end{equation}
                Mais comme \( \mu(\Omega_{\epsilon})\geq m(B)\) pour tout \( \epsilon\), nous avons en réalité \( m(B)=\inf_{\epsilon}m(\Omega_{\epsilon})\).
                
                Soit maintenant un ouvert \( \Omega\) tel que \( B\subset \Omega\). Nous devons prouver l'existence d'un \( \epsilon>0\) tel que \( m(\Omega_{\epsilon})\leq m(\Omega)\). Cela permettra de conclure que l'infimum sur tous les ouverts contenant \( B\) est égal à l'infimum sur les ouverts de la forme \( \Omega_{\epsilon}\).

                Nous posons \( m(\Omega)=m(B)+\delta\) et avec \( \epsilon\leq \delta\) nous avons
                \begin{equation}
                    m(\Omega_{\epsilon}\setminus B)\leq m(\Omega_{\epsilon}\setminus F_{\epsilon})\leq \epsilon
                \end{equation}
                et donc aussi
                \begin{equation}
                    m(\Omega_{\epsilon})\leq m(B)+\epsilon\leq m(B)+\delta=m(\Omega).
                \end{equation}
            \item[Régularité intérieure]

                Elle se fait de même.
    \end{subproof}
\end{proof}

\begin{definition}      \label{DefFMTEooMjbWKK}
    Soit \( X\) un espace topologique et \( m\) une mesure positive sur \( \big( X,\Borelien(X) \big)\).
    \begin{enumerate}
        \item       \label{ItemTTPTooStDcpw}
            \( m\) est une \defe{mesure de Borel}{mesure!de Borel} si elle est finie sur tout compact.
        \item
            \( m\) est \defe{régulière extérieurement}{mesure!régulière!extérieure} si \( \forall B\in\Borelien(X)\), 
            \begin{equation}
                m(B)=\inf\{ m(\Omega)\tq\text{\( \Omega\) est ouvert et \( B\subset \Omega\)} \}
            \end{equation}
        \item
            \( m\) est \defe{régulière intérieurement}{mesure!régulière!intérieure} si \( \forall B\in\Borelien(X)\), 
            \begin{equation}
                m(B)=\sup\{ m(K)\tq\text{\(K\) est compact et \( K\subset B \)} \}
            \end{equation}
        \item
            \( m\) est une mesure \defe{régulière}{mesure!régulière} si elle est régulière dans les deux sens.
        \item
            \( m\) est une \defe{mesure de Radon}{mesure!de Radon} si elle est de Borel et régulière.
    \end{enumerate}
\end{definition}
\index{régularité!d'une mesure}

\begin{proposition}     \label{PropNCASooBnbFrc}
    Soit \( X\) un espace localement compact et dénombrable à l'infini\footnote{Définitions \ref{DefEIBYooAWoESf} et \ref{DefFCGBooLpnSAK}.} Alors toute mesure de Borel sur \( \big( X,\Borelien(X) \big)\) est de Radon.
\end{proposition}

\begin{proof}
    Nous avons une suite exhaustive\footnote{Définition \ref{DefTBJXooONOgxb}.} de compacts \( X_k\) tels que
    \begin{equation}
        X=\bigcup_{k\geq 1}X_k=\bigcup_{k\geq 1}\Int(X_k).
    \end{equation}
    \begin{subproof}
    \item[Régularité intérieure]
    Soit \( B\), un borélien de \( X\); nous avons \( B=\bigcup_{k\geq 1}(B\cap X_k)\) et comme cette union est croissante,
    \begin{equation}
        m(B)=\lim_{k\to \infty} m(B\cap X_k)
    \end{equation}
    par le lemme \ref{LemAZGByEs}\ref{ItemJWUooRXNPci}. Dans la suite, il va y avoir beaucoup de considérations sur les topologies induites. Nous nommons \( \tau_k\) la topologie de \( X_k\) induite depuis celle de \( X\). Il ne faudra pas confondre les expressions «un compact \emph{de} $X_k$»  et «un compact \emph{dans} \( X_k\)». La première parle d'un compact pour la topologie \( \tau_k\). La seconde parle d'un compact pour la topologie de \( X\), inclus à \( X_k\).
    
    
    Si \( a<m(B)\) alors il existe \( k\geq 1\) tel que \( a<m(B\cap X_k)\), c'est à dire
    \begin{equation}
        a<m(B\cap X_k)\leq m(B).
    \end{equation}
    Mais \( (X_k,m)\) est un espace mesuré borné parce que \( m\) est de Borel et \( X_k\) est compact. Par conséquent la (restriction de la) mesure \( m\) est régulière sur l'espace mesuré \( \big( X_k,\Borelien(X_k) \big)\) par le théorème \ref{ThoPKGEooVrpsGU}. De plus l'ensemble \( B\cap X_k\) est un borélien de \( (X_k,\tau_k)\) parce que 
    \begin{equation}
        B\cap X_k\in\Borelien(X)_{X_k}=\Borelien(X_k)
    \end{equation}
    où nous avons utilisé la propriété de compatibilité entre topologie induite et tribu des borélien du théorème \ref{ThoSVTHooChgvYa}. Il existe donc un fermé \( F_{\epsilon}\) de \( (X_k,\tau_k)\) tel que 
    \begin{subequations}
        \begin{numcases}{}
            F_{\epsilon}\subset B\cap X_k\\
            m(B\cap X_k)\leq m(F_{\epsilon})+\epsilon.
        \end{numcases}
    \end{subequations}
    En mettant bout à bout les inégalités nous avons trouvé
    \begin{equation}
        a<m(B\cap X_k)\leq m(F_{\epsilon})+\epsilon<m(F_{\epsilon}),
    \end{equation}
    et donc en particulier \( a<m(F_{\epsilon})\). L'ensemble \( F_{\epsilon}\) est en plus un compact de \( (X,\tau_X)\). En effet \( X_k\) étant fermé de \( (X,\tau_X)\), le lemme \ref{LemBWSUooCCGvax} nous dit que \( F_{\epsilon}\) est un fermé de \( (X,\tau_X)\). Mais \( X_k\) étant compact, \( F_{\epsilon}\) est un fermé inclus à un compact, il est donc compact (lemme \ref{LemnAeACf}).

    Pour tout \( a<m(B)\) nous avons trouvé un compact \( F_{\epsilon}\) inclus à \( B\) dont la mesure est plus grande que \( a\). Cela prouve la régularité intérieure de la mesure \( m\).

\item[Régularité extérieure]

    Soit un borélien \( B\) de \( X\). Si \( m(B)=\infty\) alors tous les ouverts contenant \( B\) ont mesure infinie et \( m(B)\) en est évidemment le supremum. Nous supposons donc que \( m(B)<\infty\). 

    Nous notons \( \tau_k\) la topologie induite de \( X\) sur \( \Int(X_k)\). Nous posons \( B_k=B\cap\Int(X_k)\). L'espace \( \big( \Int(X_k),m \big)\) est un espace mesuré borné et \( B_k\in \Borelien\Big( \Int(X_k) \Big)\). Il existe donc un ouvert \( \Omega_k\) de \( \big( \Int(X_k),\tau_k \big)\) tel que \( B_k\subset \Omega_k\) et 
    \begin{equation}
        m(\Omega_k\setminus B_k)\leq \frac{ \epsilon }{ 2^k }.
    \end{equation}
    De plus \( \Int(X_k)\) est un ouvert de \( (X,\tau_X)\), donc en réalité \( \Omega_k\) est un ouvert de \( X\). Nous posons
    \begin{equation}
        \Omega=\bigcup_{k=1}^{\infty}\Omega_k
    \end{equation}
    qui est encore un ouvert de \( (X,\tau_X)\). 
    
    Il est temps de voir que \( \Omega\) vérifie \( m(\Omega\setminus B)\leq \epsilon\). Pour cela,
    \begin{subequations}
        \begin{align}
            \Omega\setminus B=\big( \bigcup_k\Omega_k \big)\cap\big( \bigcup_lB_l \big)^c\\
            &=\big( \bigcup_k\Omega_k \big)\cap\big( \bigcap B_l^c \big)\\
            &\subset\bigcup_k(\Omega_k\cap B_k^c)\\
            &=\bigcup_k(\Omega_k\setminus B_k),
        \end{align}
    \end{subequations}
    ce qui donne au niveau des mesures :
    \begin{equation}
        m(\Omega\setminus B)\leq\sum_{k=1}^{\infty}m(\Omega_k\setminus B_k)\leq\sum_{k=1}^{\infty}\frac{ \epsilon }{ 2^k }=\epsilon.
    \end{equation}
    \end{subproof}
\end{proof}

\begin{remark}      \label{RemooOAGCooRHpjxd}
    Exprimé sur \( \eR^N\), la proposition \ref{PropNCASooBnbFrc} s'exprime en disant que toute mesure de Borel sur \( \eR^N\) est régulière. Typiquement, l'espace \( X\) dont il est question est un ouvert de \( \eR^N\).
\end{remark}

%---------------------------------------------------------------------------------------------------------------------------
\subsection{Théorème de récurrence}
%---------------------------------------------------------------------------------------------------------------------------

Soit \( X\) un espace mesurable, \( \mu\) une mesure finie sur \( X\) et \( \phi\colon X\to X\) une application mesurable préservant la mesure, c'est à dire que pour tout ensemble mesurable \( A\subset X\),
\begin{equation}
    \mu\big( \phi^{-1}(A) \big)=\mu(A).
\end{equation}
Si \( A\subset X\) est un ensemble mesurable, un point \( x\in A\) est dit \defe{récurrent}{récurrent!point d'un système dynamique} par rapport à \( A\) si et seulement si pour tout \( p\in \eN\), il existe \( k\geq p\) tel que \( \phi^k(x)\in A\).

\begin{theorem}[\wikipedia{fr}{Théorème_de_récurrence_de_Poincaré}{Théorème de récurrence de Poincaré}.]     \label{ThoYnLNEL}
    Si \( A\) est mesurable dans \( X\), alors presque tous les points de \( A\) sont récurrents par rapport à \( A\).
\end{theorem}

\begin{proof}
    Soit \( p\in \eN\) et l'ensemble
    \begin{equation}
        U_p=\bigcup_{k=p}^{\infty}\phi^{-k}(A)
    \end{equation}
    des points qui repasseront encore dans \( A\) après \( p\) itérations  de \( \phi\). C'est un ensemble mesurable en tant que union d'ensembles mesurables (pour rappel, les tribus sont stables par union dénombrable, comme demandé à la définition \ref{DefjRsGSy}), et nous avons donc
    \begin{equation}
        \mu(U_p)\leq \mu(X)<\infty.
    \end{equation}
    De plus \( U_p=\phi^{-p}(U_0)\), donc \( \mu(U_p)=\mu(U_0)\). Vu que \( U_p\subset U_p\), nous avons
    \begin{equation}
        \mu(U_0\setminus U_p)=0.
    \end{equation}
    Étant donné que \( A\subset U_0\) nous avons a fortiori que
    \begin{equation}
        \{ x\in A\tq x\notin U_p \}\subset U_0\setminus U_p,
    \end{equation}
    et donc
    \begin{equation}
        \mu\{ x\in A\tq x\notin U_p \}=0.
    \end{equation}
    Cela signifie exactement que l'ensemble des points \( x\) de \( A\) tels que aucun des \( \phi^k(x)\) avec \( k\geq p\) n'est dans \( A\) est de mesure nulle.
\end{proof}


%+++++++++++++++++++++++++++++++++++++++++++++++++++++++++++++++++++++++++++++++++++++++++++++++++++++++++++++++++++++++++++ 
\section{Mesurabilité des fonctions à valeurs réelles}
%+++++++++++++++++++++++++++++++++++++++++++++++++++++++++++++++++++++++++++++++++++++++++++++++++++++++++++++++++++++++++++

Nous allons parler de la mesurabilité de fonctions
\begin{equation}
    f\colon (S,\tribF)\to \big( \bar \eR,\Borelien(\bar \eR) \big)
\end{equation}
où \( \bar \eR=\eR\cup\{ \pm\infty \}\).

%--------------------------------------------------------------------------------------------------------------------------- 
\subsection{Quelques mots à propos de $\overline{ \eR }$}        % Pour quelque raisons, faire \bar \eR ne fonctionne pas dans le titre.
%---------------------------------------------------------------------------------------------------------------------------

\begin{normaltext}      \label{normooGAAJooUPCbzG}
Nous convenons que \( 0\times\pm\infty=0\) parce que nous voulons qu'une droite (qui est un rectangle dont une mesure est \( 0\) et l'autre \( \infty\)) soit de mesure nulle dans \( \eR^2\).

Les produits et sommes \( \pm\infty\pm\pm\infty\) et \( \pm\infty\times \pm\infty\) sont ceux que l'on croit. Sauf bien entendu \( +\infty-\infty\) et \( 1/0\) qui ne sont toujours pas définis.
\end{normaltext}

\begin{definition}[Topologie sur \( \bar\eR\)]
La topologie sur \(\bar \eR\) est celle sur \( \eR\) à laquelle nous ajoutons les voisinages de \( \pm\infty\) de la façon suivante. Une partie \( V\) de \( \bar \eR\) est un voisinage de \( +\infty\) si il existe \( m>0\) tel que \( \mathopen] m , +\infty \mathclose]\subset V\).
\end{definition}

\begin{lemma}       \label{LEMooBLOLooAdNViv}
    L'ensemble \( B\) est un borélien de \( \bar \eR\) si et seulement si il existe un borélien \( B_0\) de \( \eR\) tel que \( B\) soit \( B_0\) ou \( B_0\cup\{ -\infty \}\) ou \( B_0\cup\{ -\infty \}\) ou \( B_0\cup\{ +\infty,-\infty \}\).
\end{lemma}

\begin{proof}
    Vu que la topologie usuelle sur \( \eR\) est la topologie induite de celle sur \( \bar \eR\), la tribu induite l'est aussi par le théorème \ref{ThoJDOKooKaaiJh}. Donc si \( B\) est un borélien de \( \bar \eR\), l'ensemble \( B\cap \eR\) est un borélien de \( \eR\).
\end{proof}

\begin{lemma}[\cite{TribuLi}]       \label{LemooCRVJooQosHPq}
    Si \( \mS_0\) est l'ensemble des intervalles du type 
    \begin{equation}
        \begin{aligned}[]
        \mathopen] \alpha , \beta \mathclose[,&&\mathopen[ -\infty , \beta \mathclose[,&&\mathopen] \alpha , +\infty \mathclose]
        \end{aligned}
    \end{equation}
    avec \( -\infty<\alpha<\beta<+\infty\) alors \( \sigma(\mS_0)=\Borelien(\bar\eR)\).
\end{lemma}

\begin{proof}
Les intervalles \( \mathopen] \alpha , \beta \mathclose[\) engendrent la topologie de \( \eR\)\footnote{Parce toutes les boules sont des intervalles de ce type et que les boules forment une base de topologie, proposition \ref{PropNBSooraAFr}.}, donc \( \Borelien(\eR)\subset\sigma(\mS_0)\). De plus le lemme \ref{LemBWNlKfA} nous autorise à dire que 
    \begin{equation}
        \bigcap_{n\geq 1}\mathopen[ n , +\infty \mathclose]=\{ +\infty \}\in\sigma(\mS_0).
    \end{equation}
    Par conséquent tous les ensembles énumérés dans le lemme \ref{LEMooBLOLooAdNViv} font partie de \( \sigma(\mS_0)\). Cela implique que \( \Borelien(\bar\eR)\subset\sigma(\mS_0)\).

    Pour l'inclusion inverse, \( \sigma(\mS_0)\) est engendré par des parties qui font parie de \( \Borelien(\bar \eR)\), donc \( \sigma(\mS_0)\subset\Borelien(\bar \eR)\).
\end{proof}

Pour la suite nous utilisons la notation (pratique en probabilité)
\begin{equation}
    \{ f<a \}=\{ x\in S\tq f(x)<a \}.
\end{equation}

%--------------------------------------------------------------------------------------------------------------------------- 
\subsection{Limite supérieure et inférieure}
%---------------------------------------------------------------------------------------------------------------------------

\begin{definition}      \label{ooMVZAooVVCOnP}
    Soit \( (a_n)\) une suite dans \( \bar \eR\). Nous définissons la \defe{limite supérieure}{limite!supérieure} et la \defe{limite inférieure}{limite!inférieure} par
    \begin{equation}
        \limsup_{n\to\infty}a_n=\lim_{n\to \infty}\big( \sup_{k\geq n}a_k \big)
    \end{equation}
    et
    \begin{equation}
        \liminf_{n\to \infty}a_n=\lim_{n\to\infty}\big( \inf_{k\geq n}a_k \big).
    \end{equation}
\end{definition}
\nomenclature[Y]{\( \limsup a_n\)}{limite supérieure}
\nomenclature[Y]{\( \liminf a_n\)}{limite inférieure}

\begin{normaltext}      \label{ooEEQJooRMFzVR}
    En ce qui concerne les suites d'ensembles, utiles en théorie des probabilités, nous définissons de même. Si les \( A_n\) sont des parties de \( \Omega\), nous définissons la \defe{limite supérieure}{limite!supérieure} et la \defe{limite inférieure}{limite!inférieure} de la suite \( A_n\) par
\begin{equation}
    \limsup_{n\to\infty}A_n=\bigcap_{n\geq 1}\bigcup_{k\geq n}A_k
\end{equation}
et
\begin{equation}
    \liminf_{n\to\infty}A_n=\bigcup_{n\geq 1}\bigcap_{k\geq n}A_k
\end{equation}

Nous avons
\begin{equation}
    \limsup A_n=\{ \omega\in\Omega\tq \omega\in A_n\text{pour une infinité de \( n\)} \}.
\end{equation}
\end{normaltext}

\begin{lemma}     \label{ooAQTEooYDBovS}
    Nous avons les formules pratiques suivantes :
    \begin{subequations}
        \begin{align}
            \limsup a_n&=\inf_{n\geq 1}\big( \sup_{k\geq n}a_k \big)\\
            \liminf a_n&=\sup_{n\geq 1}\big( \inf_{k\geq n}a_k \big).
        \end{align}
    \end{subequations}
\end{lemma}

\begin{proof}
    La suite \( n\mapsto \sup_{k\geq n}a_k\) est une suite décroissante, donc la limite est l'infimum. Même argument pour l'autre.
\end{proof}

\begin{lemma}       \label{ooIQIKooXWwAmM}
    La suite \( (a_n)\) dans \( \eR\) converge si et seulement si
    \begin{equation}
        \limsup a_n=\liminf a_n.
    \end{equation}
    Dans ce cas, \( \lim a_n=\limsup a_n=\liminf a_n\).
\end{lemma}

\begin{proof}
    Nous commençons par supposer que \( \limsup a_n=\liminf a_n=l\), et nous prouvons que \( \lim a_n\) existe et vaut \( l\). Soit \( \epsilon>0\). Il existe \( N\) tel que si \( n\geq N\) nous avons
    \begin{equation}
        \big| \sup_{k\geq n}a_k-l \big|<\epsilon
    \end{equation}
    et
    \begin{equation}
        \big| \inf_{k\geq n}a_k-l \big|<\epsilon.
    \end{equation}
    Pour tout \( k\geq N\) nous avons alors \( a_k\leq l+\epsilon\) et \( a_k\geq l-\epsilon\). Cela donne \( a_n\in B(l,\epsilon)\), c'est à dire \( a_k\to l\) par la proposition \ref{PropLimiteSuiteNum}.

    Dans l'autre sens, nous supposons que \( \lim_n a_n=l\) et nous prouvons que les limites supérieures et inférieures sont toutes deux égales à \( l\). Soit \( \epsilon>0\) et \( N_{\epsilon}\) tel que \( | a_n-l |<\epsilon\) pour tout \( n\geq N_{\epsilon}\). Si \( n\geq N_{\epsilon}\) nous avons
    \begin{equation}
        \big| \sup_{k\geq n}a_k-l \big|\leq \epsilon
    \end{equation}
    et donc la limite de \( \sup_{k\geq n}a_k\) lorsque \( n\to \infty\) est bien \(l\).
\end{proof}

%--------------------------------------------------------------------------------------------------------------------------- 
\subsection{Fonctions à valeurs réelles sur un espace mesurable}
%---------------------------------------------------------------------------------------------------------------------------

\begin{theorem}     \label{THOooWHFLooKYGsOm}
    Soit un espace mesurable \( (S,\tribF)\) et une fonction \( f\colon S\to \bar \eR\). Les faits suivants sont équivalents.
    \begin{enumerate}
        \item\label{ITEMooHAMHooYLqUhVi}
            La fonction \( f\) est mesurable.
        \item\label{ITEMooHAMHooYLqUhVii}
            L'ensemble \( \{ f<a \}\) est dans \( \tribF\) pour tout \( a\in \eR\)
        \item\label{ITEMooHAMHooYLqUhViii}
            L'ensemble \( \{ f\leq a \}\) est dans \( \tribF\) pour tout \( a\in \eR\)
    \end{enumerate}
\end{theorem}

\begin{proof}
    Plusieurs implications à prouver.
    \begin{subproof}
        \item[\ref{ITEMooHAMHooYLqUhVi}\( \Rightarrow\)\ref{ITEMooHAMHooYLqUhVii}]
            Vu que \( f\) est mesurable et que \( \mathopen[ -\infty , a \mathclose[\in\Borelien(\bar\eR)\), nous avons \( f^{-1}\big( \mathopen[ -\infty , a \mathclose[ \big)\in\tribF\).
        \item[\ref{ITEMooHAMHooYLqUhVii}\( \Rightarrow\)\ref{ITEMooHAMHooYLqUhVi}]
            Nous posons \( \tribA=\{ \mathopen[ -\infty , a \mathclose[\tq a\in \eR \}\). 

                Nous avons \( \tribA\subset\mS_0\) (le \( \mS_0\) du lemme \ref{LemooCRVJooQosHPq}). Et de plus,
            \begin{equation}
            \mathopen] \alpha , \beta \mathclose[=\mathopen[ -\infty , \beta \mathclose[\setminus\mathopen[ -\infty , \alpha \mathclose]=\mathopen[ -\infty , \beta \mathclose[\setminus\bigcap_{n\geq 1}\mathopen[ -\infty , \alpha+\frac{1}{ n } \mathclose[.
            \end{equation}
        Donc \( \mathopen] \alpha , \beta \mathclose[\in\sigma(\tribA)\).

            Et aussi :
            \begin{equation}
                \mathopen] \alpha , +\infty \mathclose]=\bar\eR\setminus\mathopen[ -\infty , \alpha+\frac{1}{ n } \mathclose[,
            \end{equation}
        ce qui donne \( \mathopen] \alpha , +\infty \mathclose]\in \sigma(\tribA)\).

        Au final, \( \mS_0\subset\sigma(\tribA)\) et donc \( \sigma(\mS_0)\subset\sigma(\tribA)\). Le lemme \ref{LemooCRVJooQosHPq} nous dit que \( \sigma(\mS_0)=\Borelien(\bar \eR)\). Nous avons donc bien \( \sigma(\mS_0)=\sigma(\tribA)=\Borelien(\bar\eR)\).

        par ailleurs, nous savons que \( f^{-1}(\tribA)\subset\tribF\) parce que les éléments de \( \tribA\) sont de la forme \( \{ f<a \}\). Cela donne \( \sigma\big( f^{-1}(\tribA) \big)=\tribF\). Mais \( \sigma\big( f^{-1}(\tribA) \big)\) peut aussi s'exprimer par le lemme de transfert \ref{LemOQTBooWGYuDU} : \( \sigma\big( f^{-1}(\tribA) \big)=f^{-1}\big( \sigma(\tribA) \big)\). En combinant les deux,
        \begin{equation}
            f^{-1}\big( \sigma(\tribA) \big)=\tribF,
        \end{equation}
        et en remplaçant \( \sigma(\tribA)\) par \( \Borelien(\bar \eR)\) nous avons ce que nous voulions :
        \begin{equation}
            f^{-1}\big( \Borelien(\bar\eR) \big)\in\tribF,
        \end{equation}
        ce qui signifie que \( f\) est mesurable.
        \item[\ref{ITEMooHAMHooYLqUhViii}\( \Rightarrow\)\ref{ITEMooHAMHooYLqUhVii}]
            Nous avons
            \begin{equation}
                \{ f<a \}=\bigcup_{n\geq 1}\{ f\leq a-\frac{1}{ n } \}.
            \end{equation}
            donc cela est une union dénombrable d'éléments de \( \tribF\). Donc \( \{ f<a \}\) est dans \( \tribF\).
        \item[\ref{ITEMooHAMHooYLqUhVi}\( \Rightarrow\)\ref{ITEMooHAMHooYLqUhViii}]
            Nous avons
            \begin{equation}
                \{ f\leq a \}=\{ f<a \}\cup f^{-1}\big( \mathopen[ -\infty , a \mathclose] \big).
            \end{equation}
            Le premier ensemble est dans \( \tribF\) par \ref{ITEMooHAMHooYLqUhVii}. Ensuite \( \mathopen[ -\infty , a \mathclose]\) est un fermé de \( \bar \eR\) et donc un borélien de \( \bar \eR\). Son image réciproque est donc un élément de \( \tribF\) parce que \( f\) est mesurable. Au final nous avons bien \( \{ f\leq a \}\in\tribF\).
    \end{subproof}
\end{proof}

\begin{lemma}[\cite{NBoIEXO}]   \label{LemFOlheqw}
    Une fonction \( f\colon X\to \eR\) est mesurable si et seulement si \( f^{-1}(I)\) est mesurable pour tout \( I\) de la forme \( \mathopen] a , \infty \mathclose[\).
\end{lemma}

\begin{proof}
    Nous devons prouver que \( f^{-1}(A)\) est mesurable dans \( X\) pour tout borélien \( A\) de \( \eR\). Nous posons
    \begin{equation}
        S=\{ A\subset \eR\tq f^{-1}(A)\text{ est mesurable dans \( X\)} \}
    \end{equation}
    et nous prouvons que cela est une tribu. D'abord \( f^{-1}(\eR)=X\), et \( X\) est mesurable, donc \( \eR\in S\). Ensuite si \( A\in S\) alors \( f^{-1}(A^c)=f^{-1}(A)^c\). En tant que complémentaire d'un mesurable de \( X\), l'ensemble \( f^{-1}(A)^c\) est mesurable dans \( X\). Et enfin si \( A_n\in S \) alors \( f^{-1}(\bigcup_nA_n)=\bigcup_nf^{-1}(A_n)\) qui est encore mesurable dans \( X\) en tant qu'union de mesurables.

    Donc \( S\) est une tribu qui contient tous les ensembles de la forme \( \mathopen] a , \infty \mathclose]\). Le lemme \ref{LemZXnAbtl} conclu que \( S\) contient tous les boréliens de \( \eR\).
\end{proof}

\begin{lemma}[\cite{NBoIEXO}]   \label{LemIGKvbNR}
    Soit \( f_n\colon X\to \eR\) une suite de fonctions mesurables\footnote{Ici \( X\) est un espace mesuré et \( \eR\) est muni des boréliens.}. Alors \( \sup_n f_n\) est mesurable.
\end{lemma}

\begin{proof}
    Nous avons
    \begin{subequations}
        \begin{align}
            (\sup f_n)^{-1}\big( \mathopen] a , \infty \mathclose] \big)&=\{ x\in X\tq (\sup f_n)(x)>a \}\\
            &=\bigcup_n\{ x\in X\tq f_n(x)>a \}\\
            &=\bigcup_nf_n^{-1}\big( \mathopen] a , \infty \mathclose] \big).
        \end{align}
    \end{subequations}
    Étant donné que \( f_n\) est mesurable et que \( \mathopen] a , \infty \mathclose]\) est mesurable, chacun des \( f_n^{-1}\big( \mathopen] a , \infty \mathclose] \big) \) est mesurable dans \( X\). Nous sommes en présence d'une union dénombrable de mesurables, donc \( (\sup f_n)^{-1}\big( \mathopen] a , \infty \mathclose] \big)\) est mesurable.

    Le lemme \ref{LemFOlheqw} conclu que \( \sup f_n\) est mesurable.
\end{proof}

\begin{proposition}\label{PropFYPEOIJ}
    Si \( f_n\) est une suite de fonctions mesurables et positives, alors la fonction \( \sum_nf_n\) est mesurable.
\end{proposition}

\begin{proof}
    Nous considérons les fonctions \( s_k(x)=\sum_{n=0}^kf_n(x)\) qui vaut éventuellement \( \infty\) en certains points. Nous avons
    \begin{equation}
        \sum_nf_n(x)=\sup_ks_k(x),
    \end{equation}
    donc le lemme \ref{LemIGKvbNR} nous donne la mesurabilité de la somme de \( f_n\).
\end{proof}

\begin{definition}      \label{ooUDHFooJjKscR}
    Soit \( (S,\tribF)\) un espace mesurable. 
    Une \defe{partition mesurable dénombrable}{partition!dénombrable mesurable} de e \( S\) est une suite  \( (S_n)_{n\geq 1}\) de parties de \( S\) telles que
    \begin{enumerate}
        \item
            \( S_n\in\tribF\) pour tout \( n\),
        \item
            \( S_N\cap S_k=\emptyset\) si \( n\neq k\),
        \item
            \( S=\bigcup_{n\geq 1}S_n\).
    \end{enumerate}
\end{definition}

\begin{lemma}[Lemme de recollement]     \label{LEMooXAPQooPpZUmP}
    Soit \( (S_n)\) une partition mesurable dénombrable de l'espace mesurable $(S,\tribF)$. Soit \( (S',\tribF')\) un autre espace mesurable et des fonctions mesurables
    \begin{equation}
        f_n\colon (S_n,\tribF_{S_n})\to (S',\tribF')
    \end{equation}
    où \( \tribF_{S_n}\) est la tribu induite\footnote{Définition \ref{DefDHTTooWNoKDP}.}. Alors la fonction
    \begin{equation}
        \begin{aligned}
            f\colon (S,\tribF)&\to (S',\tribF') \\
            x&\mapsto \text{\( f_n(x)\) si \( x\in S_n\)} 
        \end{aligned}
    \end{equation}
    est mesurable.
\end{lemma}

\begin{proof}
    Soit \( A'\in\tribF'\); nous devons prouver que \( f^{-1}(A')\in \tribF\). Nous savons que 
    \begin{equation}        \label{EqooGKFFooEwTdtg}
        f^{-1}(A')=\bigcup_{n\geq 1}f_n^{-1}(A'),
    \end{equation}
    qui est une union dénombrable d'éléments \( f_n^{-1}(A')\in\tribF_{S_n}\). 

    Vu que \( S_n\in \tribF\) nous avons \( \tribF_{S_n}\subset\tribF\) parce qu'un élément de \( \tribF_{S_n}\) est de la forme \( S_n\cap B\) avec \( B\in\tribF\). Du coup pour chaque \( n\) nous avons
    \begin{equation}
        f_n^{-1}(A')\in\tribF_{S_n}\subset \tribF.
    \end{equation}
    Au final l'égalité \eqref{EqooGKFFooEwTdtg} écrit \( f^{-1}(A')\) comme une union d'éléments de \( \tribF\) et est donc un élément de \( \tribF\).
\end{proof}

\begin{proposition}     \label{PROPooODDVooEEmmTX}
    Soit \( (S,\tribF)\) un espace mesurable et des applications mesurables \( f,g\colon S\to \bar \eR\). Alors les fonctions suivantes sont mesurables :
    \begin{enumerate}
        \item
            \( \lambda f\) pour tout \( \lambda\in \eR\)
        \item
            \( f+g\) si elle existe.
        \item
            \( 1/f\) si elle existe.
        \item
            \( fg\).
    \end{enumerate}
\end{proposition}

\begin{proof}
    Commençons par clarifier « si elle existe». La fonction \( f+g\) n'existe pas au point \( x\in S\) si \( f(x)=+\infty\) et \( g(x)=-\infty\). La fonction \( 1/f\) n'existe pas au point \( x\in S\) si \( f(x)=0\). Voir le point \ref{normooGAAJooUPCbzG}.
    \begin{subproof}
    \item[La partie où \( f+g\) existe est mesurable]
        La partie de \( S\) sur laquelle \( f+g\) existe est 
        \begin{equation}
            \{ x\in S\tq \big( f(x),g(x) \big)\neq (+\infty,-\infty),\big( f(x),g(x) \big)\neq (-\infty,+\infty) \}.
        \end{equation}
        Nous avons
        \begin{equation}
            \{  (f,g)=(+\infty,-\infty) \}=\{ f=\infty \}\cap\{ g=-\infty \}
        \end{equation}
        qui est un ensemble mesurable parce que, par exemple,
        \begin{equation}
            \{ +\infty \}=\bigcap_{n\geq 1}\mathopen[ n , +\infty \mathclose].
        \end{equation}
        La cas \( (-\infty,+\infty)\) est identique, et au final la partie de \( S\) sur laquelle \( f+g\) n'existe pas est mesurable. Par complémentarité la partie sur laquelle \( f+g\) existe est également mesurable\footnote{Parfois on a envie de dire que l'affirmation «\( A\) est mesurable» ne passe pas le test de Popper.}.
    \item[Idem pour la partie sur laquelle \( 1/f\) existe]
        Idem.
    \item[Mesurabilité de \( \lambda f\)]
        Si \( \lambda=0\), nous avons une fonction constante dont la mesurabilité est évidente\footnote{Prenez quand même le temps d'y penser.}. Nous supposons \( \lambda>0\). Alors
        \begin{equation}
            \{ \lambda f<a \}=\{ f<a/\lambda \}\in \tribF.
        \end{equation}
        .Pour \( \lambda<0\) nous avons de la même manière
        \begin{equation}
            \{ \lambda f<a \}=\{ f>a/\lambda \}\in \tribF.
        \end{equation}
        Ce dernier point est suffisant pour que \( \lambda f\) soit mesurable par la théorème \ref{THOooWHFLooKYGsOm}\ref{ITEMooHAMHooYLqUhViii} et par complémentarité.
    \item[Mesurabilité de \( f+g\)]
        Soit \( a\in \eR\); le théorème \ref{THOooWHFLooKYGsOm} nous demande d'avoir envie de prouver que \(  \{ f+g <a\} \in \tribF \). Nous avons
        \begin{equation}
            f(x)+g(x)<a
        \end{equation}
        si et seulement si
        \begin{equation}
            f(x)<a-g(x)
        \end{equation}
        si et seulement si
        \begin{equation}
            \exists q\in \eQ\tq f(x)<q<a-g(x).
        \end{equation}
        Donc
        \begin{equation}
            \{ f+g<a \}=\bigcup_{q\in \eQ}\Big( \{ f<q \}\cap\{ g<a-r \} \Big),
        \end{equation}
        qui est une union dénombrable d'éléments de \( \tribF\). Donc \( \{ f+g<a \}\in \tribF\) et \( f+g\) est mesurable.

        Note qu'en toute rigueur il faudrait  «\( \cap\text{là où \( f+g\) est définie}\)» un peu partout, mais cela ne change rien parce que l'intersection de deux parties mesurables est mesurable.

    \item[Mesurabilité de \( 1/f\)]
        Soit \( a\in \eR\). Si \( a>0\) alors
        \begin{equation}
            \{ 1/f<a \}=\{ f<0 \}\cup\{ f>\frac{1}{ a } \}\in\tribF.
        \end{equation}
        et si \( a<0\) alors
        \begin{equation}
            \{ 1/f<a \}=\{ f<0 \}\cap\{ f>\frac{1}{ a } \}\in\tribF.
        \end{equation}
    \item[Mesurabilité de \( fg\)]
        Nous allons la prouver en plusieurs fois.
        \begin{subproof}
        \item[Si \( f\) est mesurable alors \( f^2\) est mesurable]
            Si \( a\leq 0\) alors \( \{ f^2<a \}=\emptyset\). Si \( a>0\) nous avons
            \begin{equation}
                 \{ f^2<a \}=\{ -\sqrt{a}<f<\sqrt{a} \}\in\tribF.
            \end{equation}
        \item[\( f\mtu_A\) est mesurable]
            Soit \( A\in \tribF\), et prouvons que \( f\mtu_A\) est mesurable. Par définition,
            \begin{equation}
                (f\mtu_A)(x)=\begin{cases}
                    f(x)    &   \text{si \( x\in A\)}\\
                    0    &    \text{si \( x\notin A\)}.
                \end{cases}
            \end{equation}
            Nous posons \begin{equation}
                \begin{aligned}
                    f_1\colon A^c&\to \bar \eR \\
                    x&\mapsto 0 
                \end{aligned}
            \end{equation}
            et
            \begin{equation}
                \begin{aligned}
                    f_2\colon A&\to \bar \eR \\
                    x&\mapsto f(x). 
                \end{aligned}
            \end{equation}
            Alors nous avons
            \begin{equation}
                (\mtu_Af)(x)=\begin{cases}
                    f_1(x)    &   \text{si \( x\in A^c\)}\\
                    f_2(x)    &    \text{si \( x\in  A\)}.
                \end{cases}
            \end{equation}
            Les ensembles \( A\) et $A^c$ forment une partition mesurable dénombrable de \( S\). La fonction \( f_1\) est mesurable; pour prouver que \( f_2\) est mesurable, nous l'écrivons \( f_2=f\circ j_A\) où \( j_A\colon A\to S\) est l'injection canonique. L'application
            \begin{equation}
                j_A\colon (A,\tribF_A)\to (S,\tribF)
            \end{equation}
            est mesurables parce que si \( B\in\tribF\) alors \( j_A^{-1}(B)=A\cap B\in\tribF_A\). D'autre part l'application
            \begin{equation}
                f\colon (S,\tribF)\to \big( \bar \eR,\Borelien(\bar \eR) \big)
            \end{equation}
            est mesurable par hypothèse. La composée \( f_2=f\circ j_A\) est alors mesurable par la proposition \ref{PROPooEFHKooARJBwW}. Le lemme de recollement \ref{LEMooXAPQooPpZUmP} nous donne alors la mesurabilité de \( f\mtu_A\).

        \item[Le produit \( fg\) est mesurable]
            Nous posons
            \begin{equation}
                F=\{ x\in S\tq | f(x) |<+\infty,| g(x) |<\infty \}.
            \end{equation}
            En tant qu'intersection de deux ensembles mesurables, \( F\) est mesurable. Par la partie précédente, les applications \( f_1=g\mtu_F\) et \( g_1=g\mtu_F\) sont mesurables. L'application \( f_1+g_1\colon S\to \eR\) est encore mesurable. Par conséquent l'application
            \begin{equation}
                f_1g_1=\frac{ 1 }{2}\big( (f_1+g_1)^2-f_1^2-g_1^2 \big)
            \end{equation}
            est mesurable.

            Voyons maintenant ce qui se passe en dehors de \( F\). Nous allons utiliser le lemme de recollement sur la fonction
            \begin{equation}
                (fg)(x)=\begin{cases}
                    (f_1f_2)(x)    &   \text{si \( x\in F\)}\\
                    -\infty    &    \text{si \( x\in\mU\)}\\
                    0    &    \text{si \( x\in \mV\)}\\
                    +\infty    &    \text{si \( x\in\mW\)}
                \end{cases}
            \end{equation}
            où \( F,\mU,\mV,\mW\) forment une partition mesurable dénombrable\footnote{Définition \ref{ooUDHFooJjKscR}.} de \( S\). Pour le sport nous montrons que \( \mU\) est mesurable :
            \begin{subequations}
                \begin{align}
                    \mU&=\big( \{ f=-\infty \}\cap\{ g>0 \} \big)\\
                    &\cup\big( \{ f=+\infty \}\cap\{ g<0 \} \big)\\
                    &\cup\big( \{ g=-\infty \}\cap\{ f>0 \} \big)\\
                    &\cup\big( \{ g=+\infty \}\cap\{ f<0 \}\big).
                \end{align}
            \end{subequations}
        \end{subproof}
    \end{subproof}
\end{proof}

\begin{proposition}     \label{ooABKWooPbfSOZ}
    Si \( f_n\colon S\to \bar \eR\) est une suite de fonctions mesurables, alors les fonctions \( \inf_n f_n\) et \( \sup_nf_n\) sont mesurables.
\end{proposition}

\begin{proof}
    Nous avons les découpages
    \begin{equation}
        \{ \inf_nf_n<a \}=\bigcup_n\{ f_n<a \}\in\tribF
    \end{equation}
    et
    \begin{equation}        \label{EQooNYKVooDOjOXM}
        \{ \sup_nf_n\leq a \}=\bigcap_n\{ f_n\leq a \}\in\tribF.
    \end{equation}
    Le théorème \ref{THOooWHFLooKYGsOm} permet de conclure.
\end{proof}
Note que pour \eqref{EQooNYKVooDOjOXM} nous ne pouvions pas utiliser les inégalités strictes parce que \( \{ \sup_nf_n<a \}\) n'est pas spécialement égal à \( \bigcap_n\{ f_n<a \}\).

\begin{normaltext}
    La proposition \ref{ooABKWooPbfSOZ} nous permet de définir les parties positives et négatives de \( f\) par \( f^+=\sup(f,0)\) et \( f^-=\sup(-f,0)\). Ce sont des applications mesurables. Nous avons les décompositions
    \begin{subequations}
        \begin{align}
            f=f^+-f^-\\
            | f |=f^++f^-.
        \end{align}
    \end{subequations}
\end{normaltext}

\begin{corollary}
    Si \( f\colon S\to \bar \eR\) est mesurable alors les applications \( f^+\), \( f^-\) et \( | f |\) sont mesurables en tant qu'applications \( S\to\bar \eR^+\).
\end{corollary}

\begin{proof}
    Nous faisons la preuve pour \( f^+\). Nous savons que \( f^+\colon S\to \bar \eR\) est mesurable par la proposition \ref{ooABKWooPbfSOZ}. Nous considérons l'injection canonique \( f\colon \bar \eR^+\to \bar \eR\) et 
    \begin{equation}
        \begin{aligned}
            f_1^+\colon S&\to \bar \eR^+ \\
            x&\mapsto f^+(x). 
        \end{aligned}
    \end{equation}
    Alors \( f_1^+=j\circ f^+\) est mesurable. Et c'est bien cela que nous voulions.

\end{proof}
Note : \( f^+\) et \( f_1^+\) sont exactement les mêmes fonctions. Elles ne diffèrent que par la tribu que nous considérons sur l'espace d'arrivée. Nous allons à partir de maintenant les noter toutes deux \( f^+\).

\begin{remark}
    L'application \( | f |\) peut être mesurable sans que \( f\) le soit. Soit en effet une partie \( A\notin \tribF\), et posons
    \begin{equation}
        f(x)=\begin{cases}
            1    &   \text{si \( x\in A\)}\\
            -1    &    \text{si \( x\in A^c\)}.
        \end{cases}
    \end{equation}
    Alors \( f^{-1}(\{ 1 \})=A\) n'est pas mesurable alors que \( | f |(x)=1\) pour tout \( x\).
\end{remark}

Il est temps d'aller relire les définitions \ref{ooMVZAooVVCOnP}.

\begin{proposition}     \label{PropooMFIBooJzaleK}
    Si les fonctions \( f_n\colon S\to \bar \eR\) sont mesurables alors les fonctions \( \limsup f_n\) et \( \liminf f_n\) sont mesurables.
\end{proposition}

\begin{proof}
    Par le lemme \ref{ooAQTEooYDBovS} nous écrivons \( \limsup_nf_n(x)=\inf_{n\geq 1}\sup_{k\geq n} f_k(x)\). Pour chaque \( k\) nous considérons la fonction \( g_k=\sup_{n\geq k}f_n\). Par la proposition \ref{ooABKWooPbfSOZ}, les fonctions \( g_k\) sont mesurables. En utilisant encore la même proposition, \( \inf_{n\geq 1}g_k\) est encore mesurable.
\end{proof}

\begin{proposition}[\cite{ooKKLCooZRxJnn}]      \label{PropooDXBGooSFqrai}
    Si \( f_n\colon S\to \bar \eR\) est une suite de fonctions mesurables telle dont la limite ponctuelle existe, alors la limite est mesurable.
\end{proposition}

\begin{proof}
    Si la limite existe, elle est égale à la limite supérieure par le lemme \ref{ooIQIKooXWwAmM}. Or la limite supérieure est mesurable par la proposition \ref{PropooMFIBooJzaleK}.
\end{proof}

%--------------------------------------------------------------------------------------------------------------------------- 
\subsection{Fonction étagée}
%---------------------------------------------------------------------------------------------------------------------------

\begin{definition}[\cite{ooARRSooBLWdam}]\label{DefBPCxdel}
    Soit \( (S,\tribF)\) un espace mesurable et une fonction \( f\colon S\to \big( \bar\eR,\Borelien(\bar\eR) \big)\). Il serait dommage de confondre les trois concepts suivants.
    \begin{itemize}
        \item
    Une \defe{fonction simple}{simple!fonction} est une fonction dont l'image est constituée d'un nombre fini de valeurs.
\item
    Une \defe{fonction étagée}{étagée!fonction} est une fonction simple qui est elle-même une fonction mesurable.
\item
    Une \defe{fonction en escalier}{escalier} est une fonction étagée dont les valeurs sont constantes sur des intervalles : ce sont donc des fonctions constantes par morceaux.
    \end{itemize}
\end{definition}

Dans les trois cas, la fonction \( f\) peut être écrite comme somme de fonctions caractéristiques :
\begin{equation}
    f(x)=\sum_{j=1}^p\alpha_j\mtu_{A_j}(x)
\end{equation}
où \( A_j=f^{-1}(\alpha_j)\). Ce qui change est la nature des \( A_j\).

\begin{itemize}
    \item Si \( f\) est  simple, les \( A_j\) sont quelconques.
    \item Si \( f\) est étagée, les \( A_i\) peuvent être choisis mesurables parce que \( \{\alpha_i \}\) est un borélien, ce qui fait que \( A_i=f^{-1}(\alpha_i)\) un choix mesurable.
    \item Si \( f\) est en escalier, les \( A_i\) sont des intervalles.
\end{itemize}

La \defe{forme canonique}{forme canonique!fonction simple} d'une fonction simple \( f\) est la suivante. Soit \( \{ \alpha_i \}_{i=1,\ldots, l}\) les valeurs distinctes prises par \( f\) et \( A_i=f^{-1}(\alpha_i)\). La forme canonique de \( f\) est alors
\begin{equation}
    f=\sum_{i=1}^l\alpha_i\mtu_{A_i}.
\end{equation}
Notons que nous avons \( S=\bigcup_iA_i\), et que cette union est disjointe dans le cas d'une représentation canonique.

\begin{probleme}
    Le lemme \ref{LemYFoWqmS} et le théorème \ref{THOooXHIVooKUddLi} disent la même chose alors que la preuve du théorème \ref{THOooXHIVooKUddLi} est beaucoup plus compliquée.La démonstration du lemme serait fausse ?

    M'est avis que ce que le théorème donne en plus est la convergence uniforme en cas de fonction bornée. La suite \eqref{EqooXQYIooSSJwtM} ne va pas converger uniformément.
\end{probleme}

\begin{lemma}[Limite croissante de fonctions étagées\cite{MonCerveau}]    \label{LemYFoWqmS}
    Soit \( f\colon (S,\tribF)\to \bar\eR\) une fonction positive mesurable. Il existe une suite \( f_n\colon S\to \eR\) de fonctions étagées positives telles que \( f_n\to f\) ponctuellement et \( f_n \leq f\).
\end{lemma}

\begin{proof}
    Nous considérons \( (q_n)\) une suite parcourant tous les rationnels positifs\footnote{Nous rappelons que \( \eQ\) est dénombrable et dense dans \( \eR\) par la proposition \ref{PropooUHNZooOUYIkn}.} avec \( q_0=0\) pour être sûr.
    Pour \( n\in \eN\) nous définissons la fonction
    \begin{equation}        \label{EqooXQYIooSSJwtM}
        f_n(x)= \max\{ q_i\tq i\leq n,\, q_i\leq f(x) \}.
    \end{equation}
    L'ensemble sur lequel le maximum est pris n'est pas vide parce que \( q_0=0\). La fonction \( f_n\) est simple parce qu'elle ne prend que \( n\) valeurs différentes. Nous avons aussi par construction que \(  f_n(x)\leq f(x) \). Nous avons aussi pour tout \( x\in S\) que \( f_n(x)\to f(x)\) parce que \( \eQ\) est dense dans \( \eR\).

    En ce qui concerne le fait que \( f_n\) est mesurable, nous notons \( \{ r_0,\ldots, r_{n} \}\) l'ensemble des \( \{ q_0,\ldots, q_n \}\) classés dans l'ordre croissant. Nous posons en plus \( r_{n+1}=+\infty\). Nous avons alors
    \begin{equation}
        f_n^{-1}(r_k)=\{ x\in S\tq f(x)\geq r_k,f(x)<r_{k+1} \}=\{ f\geq r_k \}\cap\{ f<r_{k+1} \}
    \end{equation}
    qui est un ensemble mesurable comme intersection deux ensembles mesurables par le théorème \ref{THOooWHFLooKYGsOm}.
\end{proof}

\begin{remark}
    Pour avoir \(  f_n <| f |\) nous pouvons poser
    \begin{equation}
        f_n(x)=\begin{cases}
            \max\{ q_i\tq i\leq n,\, q_i\leq f(x) \}    &   \text{si \( f(x)\geq 0\)}\\
            \min\{ q_i\tq i\leq n,\, q_i\geq f(x) \}    &    \text{si \( f(x)< 0\).}
        \end{cases}
    \end{equation}
\end{remark}

\begin{theorem}[Théorème fondamental d'approximation\cite{TribuLi}]\label{THOooXHIVooKUddLi}
    Pas à pas.
    \begin{enumerate}
        \item
    Soit une fonction mesurable \( f\colon S\to \mathopen[ 0 , +\infty \mathclose]\). Alors il existe une suite croissante de fonctions \( \varphi_n\colon S\to \mathopen[ 0 , +\infty \mathclose[\) étagées positives dont la limite ponctuelle est \( f\).
    \item
        Si de plus \( f\) est bornée, la convergence est uniforme.
    \item
        Idem pour \( f\) à valeurs dans \( \bar \eR\) ou \( \eC\).
    \end{enumerate}
\end{theorem}

\begin{proof}
    Nous découpons l'intervalle \( \mathopen[ 0 , n \mathclose]\) en plusieurs morceaux.
    \begin{equation}
        I_{n,k}=\begin{cases}
            \mathopen[ \frac{ k }{ 2^n } , \frac{ k+1 }{ 2^n } \mathclose[    &   \text{si \( 0\leq k\leq n2^n-1\)}\\
                \mathopen[ n , \infty \mathclose]    &    \text{si \( k=n2^n\)}.
        \end{cases}
    \end{equation}
    Nous posons \( S_{n,k}=f^{-1}(I_{n,k})\). Ce sont des ensembles mesurables parce que \( f\) est mesurable. Et de plus pour chaque \( n\), la suite \( (S_{n,k})_{k\geq 0}\) est une partition mesurable finie de \( S\). Nous posons
    \begin{equation}
        \varphi_n=\sum_{k=0}^{n2^n}\frac{ k }{ 2^n }\mtu_{S_{n,k}}.
    \end{equation}
    C'est à dire que sur chaque \( S_{n,k}\) nous approximons \( f\) par le bas. La fonction \( \varphi_n\) est étagée et positive : \( 0\leq \varphi_n(x)\leq f(x)\) par construction.
    \begin{subproof}
    \item[Croissance]
        Nous allons voir que \( \varphi_n\leq \varphi_{n+1}\). Soit \( k\neq n2^n\). Si \( x\in S_{n,k}\) alors \( \varphi_n(x)=\frac{ k }{ 2^n }\) et nous avons aussi la décomposition
        \begin{equation}
            S_{n,k}=S_{n+1,2k}\cup S_{n+,2k+1}.
        \end{equation}
        Si \( x\in S_{n+1,2k}\) alors \( \varphi_{n+1}(x)=\frac{ 2k }{ 2^{n+1} }=\frac{ k }{ 2^n }=\varphi_n(x)\). Et si \( x\in S_{n+1,2k+1}\) alors
        \begin{equation}
            \varphi_{n+1}(x)=\frac{ 2k+1 }{ 2^{n+1} }=\frac{ k+\frac{ 1 }{2} }{ 2^n }>\varphi_n(x).
        \end{equation}
        
        Il reste à traiter le cas \( x\in\{ f\geq n \}\). Dans ce cas nous avons \( \varphi_n(x)=n\). il y a encore deux cas à traiter : 
        \begin{equation}
            \{ f\geq n \}=\{ f\in\mathopen[ n , n+1 \mathclose[ \}\cup\{ f\in\mathopen[ n+1 , \infty \mathclose] \}.
        \end{equation}
        Pour plus de simplicité dans les notations, nous notons \( \bar n=n2^n\), c'est à dire que \( I_{n,\bar n}\) est le \( I_{n,k}\) avec le \( k\) le plus grand possible. Nous avons
        \begin{equation}
            I_{n,\bar n}=\mathopen[ n , n+1 \mathclose[\cup\mathopen[ n+1 , \infty \mathclose].
        \end{equation}
        Le premier élément se décompose en \( I_{n+1,k}\) avec \( k<n+1\) (nous préciserons plus tard exactement les valeurs de \( k\)) tandis que le second est \( \mathopen[ n+1 , \infty \mathclose]=I_{n+1,\overline{ n+1 }}\).

        Pour \( x\in S_{n+1,\overline{ n+1 }}\) nous avons
        \begin{equation}
            \varphi_{n+1}(x)=\frac{ (n+1)2^{n+1} }{ 2^{n+1} }=n+1>\varphi_n(x).
        \end{equation}
        Si au contraire \( f(x)\in\mathopen[ n , n+1 \mathclose[ \) nous devons précisément voir quels sont les \( k\) qui font en sorte que \( I_{n+1,k}\) recouvre \( \mathopen[ n , n+1 \mathclose[\). Le plus petit \( k\) est donné par \( \frac{ k }{ 2^{n+1} }=n\), c'est à dire \( k=n2^{n+1}\) et le plus grand \( k\) est donné par \( \frac{ k }{ 2^{n+1} }<n+1\), c'est à dire \( k=2^{n+1}(n+1)-1\). Donc si \( f(x)\in\mathopen[ n , n+1 \mathclose[\) alors \( x\in S_{n+1,k}\) avec
            \begin{equation}
                n2^{n+1}\leq k\leq (n+1)2^{n+1}-1
            \end{equation}
            Dans ce cas 
            \begin{equation}
                \varphi_{n+1}(x)=\frac{ k }{ 2^{n+1} }\geq \frac{ n2^{n+1} }{ 2^{n+1} }=n=\varphi_n(x).
            \end{equation}
    \item[Convergence ponctuelle]
        Si \( f(x)<\infty\) alors il existe\footnote{Le vrai snob citera ici le lemme \ref{LemooMWOUooVWgaEi}.} \( n_0\in\eN\) tel que \( f(x)<n_0\). Pour \( bn\geq n_0\) nous avons \( f(x)<n\) et donc \( \varphi_n(x)\) se calcule à partir d'un des intervalles de taille \( 1/2^n\) :
        \begin{equation}
            \varphi_n(x)=\frac{ k }{ 2^n }\leq f(x)<\frac{ k+1 }{ 2^n }.
        \end{equation}
        Donc 
        \begin{equation}
            | \varphi_n(x)-f(x) |\leq \frac{1}{ 2^n },
        \end{equation}
        ce qui signifie que \( \lim_{n\to \infty} \varphi_n(x)=f(x)\).

        Si \( f(x)=+\infty\) alors \( f(x)>n\) pour tout \( n\). Et alors \( \varphi_n(x)=n\) pour tout \( n\), ce qui donne bien \( \varphi_n(x)\to \infty\).
    \item[Convergence uniforme]
        Soit \( f\) bornée : \( 0\leq f(x)<M\) pour tout \( x\in S\). Soit aussi \( \epsilon>0\). Nous prenons \( n_0>M\) tel que \( \frac{1}{ 2^{n_0}<\epsilon }\). Alors pour tout \( n\geq n_0\) nous avons
        \begin{equation}
            0\leq f(x)-\varphi_n(x)\leq \frac{1}{ 2^n }\leq \frac{1}{ 2^{n_0} }\leq \epsilon.
        \end{equation}
        Note qu'il n'y a pas de valeurs absolues parce que nous savons déjà que la limite est croissante.
    \end{subproof}
\end{proof}

%--------------------------------------------------------------------------------------------------------------------------- 
\subsection{Fonctions réelle à variables réelles}
%---------------------------------------------------------------------------------------------------------------------------

Nous nous particularisons à présent au cas de fonctions
\begin{equation}
    f\colon \big( \eR,\Borelien(\eR) \big)\to \big( \bar\eR,\Borelien(\bar \eR) \big).
\end{equation}

\begin{remark}
    En théorie de l'intégration il se parle souvent de fonctions mesurables au sens de la tribu de Lebesgue : $f\colon \big( \eR,\Lebesgue(\eR) \big)\to \big( \bar\eR,\Borelien(\bar \eR) \big)$ où \( \Lebesgue(\eR)\) est la tribu de Lebesgue sur \( \eR\), c'est à dire la tribu complétée de celle des boréliens (définition \ref{DefooYZSQooSOcyYN}).

    Le fait est qu'en pratique, c'est déjà assez compliqué de construire des fonctions non mesurables au sens des boréliens; alors on va s'en contenter. Mais construire des fonctions non mesurables au sens de la tribu de Lebesgue, c'est encore plus compliqué.

    Nous allons donc nous contenter de donner des conditions assurant qu'une fonction \( f\colon  \big( \eR,\Borelien(\eR) \big)   \to (\bar \eR,\Borelien(\bar \eR))  \) soit mesurable. Ces fonctions seront a fortiori mesurables en les considérant comme fonctions \( f\colon   \big( \eR,\Lebesgue(\eR) \big)  \to  (\bar\eR,\Borelien(\bar \eR)) \).
\end{remark}

\begin{corollary}       \label{CorooJYDVooCrXVun}
    Si \( I\) est un intervalle de \( \eR\), alors toute application monotone \( f\colon I\to \eR\) est borélienne.
\end{corollary}

\begin{proof}
    Vu que \( f\) est monotone, l'ensemble \( \{ f<a \}\) est un intervalle. Or tous les intervalles sont boréliens, donc \( f\) est mesurable par le théorème \ref{THOooWHFLooKYGsOm}.
\end{proof}

\begin{definition}
Si \( I\) est un intervalle de \( \eR\), une fonction \( f\colon I\to \eR\) a une propriété (monotone, mesurable, continue, etc.) \defe{par morceaux}{morceau!fonction continue ou monotone}\index{monotone!par morceaux}\index{continue!par morceaux} si il existe une suite strictement croissante de points \( (x_I)_{i\in \eZ}\) dans \( I\) telle que \( f\) ait la propriété sur chacun des ouverts \( \mathopen] x_j ,x_{j+1} \mathclose[.\).
\end{definition}
Dans cette définition, les points sont labelles par \( \eZ\) et non par \( \eN\) parce que nous nous laissons la liberté d'avoir une infinité de points de chacun des deux côtés.

\begin{proposition}     \label{PropooLNBHooBHAWiD}
    Soit \( I\) un intervalle de \( \eR\) et une fonction \( f\colon I\to \eR\). Si \( f\) est continue ou monotone par morceaux sur \( I\) alors elle y est borélienne.
\end{proposition}

\begin{proof}
L'ensemble \( \{  \mathopen] x_j , x_{j+1} \mathclose[  \}_{j\in \eZ}\cup\{ x_i \}_{i\in \eZ}\) forme une partition mesurable dénombrable de \( I\) (les singletons sont des boréliens). À une belle redéfinition près de la numérotation (deux fois \( \eZ\) va dans \( \eN\)), nous les appelons \( (I_n)_{n\in \eN}\), et nous définissons les fonctions \( f_n\) comme étant les restrictions de \( f\) aux intervalles \( I_k\).

    Toute fonctions sur un singleton est mesurable. Toute fonctions continue sur un ouvert est mesurable (théorème \ref{ThoJDOKooKaaiJh}). Toute fonction monotone sur un ouvert est mesurable (corollaire \ref{CorooJYDVooCrXVun}).

    Le lemme de recollement \ref{LEMooXAPQooPpZUmP} donne alors la mesurabilité de \( f\).
\end{proof}

\begin{normaltext}
    Toutes les fonctions que nous pouvons écrire explicitement sont mesurables \ldots en tout cas toutes celles que l'on trouve en pratique. En effet nous avons déjà toutes les fonctions continues par morceaux via la proposition \ref{PropooLNBHooBHAWiD} et ensuite toutes les limites par la proposition \ref{PropooDXBGooSFqrai}. Cela donne les séries, les dérivées, les primitives, etc.
\end{normaltext}
