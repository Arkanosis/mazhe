% This is part of Mes notes de mathématique
% Copyright (c) 2011-2015
%   Laurent Claessens, Carlotta Donadello
% See the file fdl-1.3.txt for copying conditions.

%+++++++++++++++++++++++++++++++++++++++++++++++++++++++++++++++++++++++++++++++++++++++++++++++++++++++++++++++++++++++++++ 
\section{Fonctions mesurables}
%+++++++++++++++++++++++++++++++++++++++++++++++++++++++++++++++++++++++++++++++++++++++++++++++++++++++++++++++++++++++++++

%--------------------------------------------------------------------------------------------------------------------------- 
\subsection{Propriétés}
%---------------------------------------------------------------------------------------------------------------------------

\begin{definition}[Fonction mesurable] \label{DefQKjDSeC}
    Soit \( (E,\tribA)\) et \( (F,\tribF)\) deux espaces mesurés. Une fonction \( f\colon E\to F\) est \defe{mesurable}{mesurable!fonction} si pour tout \( \mO\in \tribF\), l'ensemble \( f^{-1}(\mO)\) est dans \( \tribA\).
\end{definition}

\begin{definition}[Fonction borélienne]     \label{DefHHIBooNrpQjs}
    Une application \( f\colon (\Omega,\tribA)\to (\eR^d,\Borelien(\eR^d))\) est \defe{borélienne}{borélienne!fonction}\index{fonction!borélienne} si elle est mesurable, c'est à dire si pour tout \( B\in\Borelien(\eR^d)\) nous avons \( f^{-1}(A)\in\tribA\).

    Si rien n'est précisé, une application entre deux espaces topologiques est borélienne lorsqu'elle est mesurable en considérant la tribu borélienne sur \emph{les deux} espace.
\end{definition}
Si \( \tribA\) est une tribu sur un ensemble \( E\), nous notons \( m(\tribA)\)\nomenclature[P]{\( m(\tribA)\)}{Ensemble des fonctions \( \tribA\)-mesurables} l'ensemble des fonctions qui sont \( \tribA\)-mesurables.

Le plus souvent lorsque nous parlerons de fonctions \( f\colon X\to Y\) où \( Y\) est un espace topologique, nous considérons la tribu borélienne sur \( Y\). Ce sera en particulier le cas dans la théorie de l'intégration.

\begin{remark}
    Lorsque nous considérons des fonctions à valeurs réelles \( f\colon X\to \eR\) nous utiliserons toujours la tribu borélienne sur \( \eR\). Pour \( X\), cela peut dépendre des contextes. En théorie de l'intégration, nous mettrons sur \( X\) la tribu des ensembles mesurables au sens de Lebesgue sur \( X\), \emph{tout en gardant celle des boréliens sur l'ensemble d'arrivée}.

    Pour toute la partie sur l'intégration, une fonction \( f\colon \eR^n\to \eR^m\) sera mesurable si pour tout borélien \( A\) de \( \eR^m\) l'ensemble \( f^{-1}(A)\) est Lebesgue-mesurable dans \( \eR^n\).

    Étant donné qu'il est franchement difficile de créer des ensembles non mesurables au sens de Lebesgue, il est franchement difficile de créer des fonction non mesurables à valeurs réelles. L'hypothèse de mesurabilité est donc toujours satisfaite dans les cas pratiques.
\end{remark}

\begin{proposition}     \label{PROPooEFHKooARJBwW}
    Soient \( (S_i,\tribF_i)\) (\( i=1,2,3\)) des espaces mesurables et des fonctions mesurables \( f\colon S_1\to S_2\) et \( g\colon S_2\to S_3\). Alors la fonction \( g\circ f\colon S_1\to S_3\) est mesurable.
\end{proposition}

\begin{proof}
    Soit \( B\in\tribF_3\). Alors
    \begin{equation}
        (g\circ f)^{-1}(B)=f^{-1}\big( g^{-1}(B) \big)\in f^{-1}(\tribF_2)\subset\tribF_1.
    \end{equation}
\end{proof}

%--------------------------------------------------------------------------------------------------------------------------- 
\subsection{D'une tribu à l'autre}
%---------------------------------------------------------------------------------------------------------------------------


\begin{lemma}[\cite{TribuLi}]       \label{LemooVDXJooZNYelH}
    Soit une application \( f\colon S_1\to S_2\) et une tribu \( \tribF_2\) sur \( S_2\). Alors \( f^{-1}(\tribF_2)\) est une tribu sur \( S_1\)
\end{lemma}

\begin{proof}
    Il faut prouver les trois propriétés de la définition \ref{DefjRsGSy} d'une tribu.
    \begin{enumerate}
        \item
            D'abord \( f\) est définit sur tout \( S_1\), donc \( f^{-1}(S_2)=S_1\) alors que \( S_2\in \tribF_2\).
        \item
            Soit \( A\in f^{-1}(\tribF_2)\), c'est à dire \( A=f^{-1}(B)\) pour un certain \( B\in \tribF_2\). En ce qui concerne le complémentaire :
            \begin{equation}
                A^c=f^{-1}(B)^c=S_1\setminus f^{-1}(B)=f^{-1}(S_2\setminus B)=f^{-1}(B^c).
            \end{equation}
        \item
            Si \( (A_i)_{i\in \eN}\) sont des éléments de \( f^{-1}(\tribF_2)\) avec \( A_i=f^{-1}(B_i)\) alors
            \begin{equation}
                \bigcup_iA_i=\bigcup_if^{-1}(B_i)=f^{-1}\big( \bigcup_iB_i \big).
            \end{equation}
            Ce qui est dans la dernière parenthèse est dans \( \tribF_2\) parce que cette dernière est une tribu.
    \end{enumerate}
\end{proof}

\begin{lemma}[\cite{TribuLi}]       \label{LemJYKBooBSXBXJ}
    Soit une application \( f\colon S_1\to S_2\) et \( \tribF\) une tribu de \( S_1\). Alors
    \begin{enumerate}
        \item
            L'ensemble
            \begin{equation}
                \tribF_f=\{  B\subset S_2\tq f^{-1}(B)\in \tribF  \}
            \end{equation}
            est une tribu sur \( S_2\).
        \item
            C'est la plus grande tribu de \( S_2\) pour laquelle \( f\) est mesurable.
    \end{enumerate}
\end{lemma}

\begin{proof}
    Encore les trois propriétés à vérifier.
    \begin{enumerate}
        \item
            \( S_2\in\tribF\), sont \( S_1=f^{-1}(S_2)\in \tribF_f\).
        \item
            Si \( A\in \tribF_f\) alors \( A=f^{-1}(B)\) pour un certain \( B\in \tribF\). Nous avons alors aussi \( B^c\in \tribF\) et donc 
            \begin{equation}
                f^{-1}(B^c)=f^{-1}(B)^c=A^c.
            \end{equation}
            Par conséquent \( A^c\) est dans \( \tribF_f\).
        \item
            Si \( (A_i)\) sont des éléments de \( \tribF_f\) avec \( A_i=f^{-1}(B_i)\) pour \( B_i\in \tribF\) alors \( \bigcup_iB_i\in\tribF\) et
            \begin{equation}
                f^{-1}\big( \bigcup_iB_i \big)=\bigcup_if^{-1}(B_i)\in\tribF_f.
            \end{equation}
    \end{enumerate}
    En ce qui concerne la maximalité, si \( R\subset S_2\) n'est pas dans \( \tribF_f\) alors \( f^{-1}(R)\) n'est pas dans \( \tribF\) et donc \( f\) ne serait pas mesurable.
\end{proof}

\begin{definition}[Tribu engendrée] \label{DefNOJWooLGKhmJ}
    Soit une application \( f\colon S_1\to S_2\) et \( \tribF\) une tribu de \( S_1\). Alors conformément au lemme \ref{LemJYKBooBSXBXJ} l'ensemble
            \begin{equation}
                \tribF_f=\{  B\subset S_2\tq f^{-1}(B)\in \tribF  \}
            \end{equation}
            est la \defe{tribu engendrée}{tribu!engendrée!par une application}.
\end{definition}

\begin{lemma}[Lemme de transfert]       \label{LemOQTBooWGYuDU}
    Soit \( f\colon S_1\to S_2\) une application et une classe \( \tribC\) de parties de \( S_2\). Alors
    \begin{equation}
        \sigma\big( f^{-1}(\tribC) \big)=f^{-1}\big( \sigma(\tribC) \big).
    \end{equation}
\end{lemma}
\index{lemme!de transfert}

\begin{proof}
    Vu que \( \sigma(\tribC)\) es tune tribu, dans \( S_2\) alors le lemme \ref{LemJYKBooBSXBXJ} dit que \( f^{-1}\big( \sigma(\tribC) \big)\) est une tribu qui contient en particulier \(  f^{-1}(\tribC) \). Nous en déduisons que \( \sigma\big( f^{-1}(\tribC) \big)\subset f^{-1}\big( \sigma(\tribC) \big)\).

    Réciproquement. Dans \( S_1\) nous avons la tribu \( \sigma\big( f^{-1}(\tribC) \big)\). Nous pouvons alors considérer la tribu
    \begin{equation}
        \tribF_f=\{ B\subset S_2\tq f^{-1}(B)\in\sigma\big( f^{-1}(\tribC) \big) \}.
    \end{equation}
    Montrons que \( \tribC\subset \tribF_f\). Lorsque \( B\in \tribC\) nous avons \( f^{-1}(B)\in f^{-1}(\tribC)\subset\sigma\big( f^{-1}(\tribC) \big)\). Du coup \( B\in \tribF_f\). Nous avons alors, en passant aux tribus engendrées :
    \begin{equation}
        \sigma(\tribC)\subset\sigma(\tribF_f)=\tribF_f.
    \end{equation}
    Si maintenant \( B\in\sigma(\tribC)\), nous avons \( f^{-1}(B)\in \sigma\big( f^{-1}(\tribC) \big)\), ce qui signifie que
    \begin{equation}
        f^{-1}\big( \sigma(\tribC) \big)\subset\sigma\big( f^{-1}(\tribC) \big).
    \end{equation}
\end{proof}

Le théorème suivant est important pour prouver qu'une application est mesurable. En effet, il permet de ne tester si une application est mesurable uniquement sur une partie génératrice de la tribu d'arrivé\footnote{Typiquement les ouverts pour les boréliens.}.
\begin{theorem}     \label{ThoECVAooDUxZrE}
    Soient des espaces mesurables \( ( S_1,\tribF_1 )\) et \( (S_2,\tribF_2)\) ainsi qu'une application \( f\colon S_1\to S_2\). Si il existe un ensemble de parties \( \tribC\) de \( S_2\) tel que
    \begin{itemize}
        \item \( \sigma(\tribC)=\tribF_2\)
        \item \( f^{-1}(B) \in \tribF_1 \) pour tout \( B\in \tribC\)
    \end{itemize}
    alors \( f\) est mesurable.
\end{theorem}

\begin{proof}
    Par hypothèse, \( \sigma(\tribC)=\tribF_2\) et \( f^{-1}(\tribC)\subset \tribF_1\) et nous pouvons utiliser le lemme de transfert \ref{LemOQTBooWGYuDU} :
    \begin{equation}
        \sigma\big( f^{-1}(\tribC) \big)=f^{-1}\big( \sigma(\tribC) \big)
    \end{equation}
    qui s'écrit ici
    \begin{equation}
        \sigma\big( f^{-1}(\tribC) \big)=f^{-1}(\tribF_2).
    \end{equation}
    Mais vu que \( f^{-1}(\tribC)\subset \tribF_1\), nous avons aussi \( \sigma\big( f^{-1}(\tribC) \big)\subset \tribF_1\), ce qui signifie que
    \begin{equation}
        f^{-1}(\tribF_2)\subset \tribF_1.
    \end{equation}
    Cela est exactement le fait que \( f\) soit mesurable.
\end{proof}

Le théorème suivant est très important parce qu'en pratique c'est souvent lui, en conjonction avec la proposition \ref{PropooLNBHooBHAWiD} qui permet de déduire qu'une fonction est borélienne.
\begin{theorem}[\cite{TribuLi}]     \label{ThoJDOKooKaaiJh}
    Soient \( X\) et \( Y\) deux espaces topologiques. Alors toute application continue \( f\colon X\to Y\) est borélienne\footnote{Définition \ref{DefHHIBooNrpQjs}.}.
\end{theorem}

\begin{proof}
    Pour vérifier que \( f\) est borélienne, nous devons prouver que \( f^{-1}(B)\) est borélien pour tout borélien \( B\) de \( Y\). Heureusement, le théorème \ref{ThoECVAooDUxZrE} nous permet de limiter la vérification aux \( B\) appartenant à une classe engendrant les boréliens de \( Y\).

    La classe en question est toute trouvée : ce sont les ouverts. Si \( \mO\) est un ouvert de \( Y\) alors \( f^{-1}(\mO)\) est un ouvert de \( X\) et donc un borélien de \( X\).
\end{proof}

Le théorème suivant donne une importante compatibilité entre l'induction de tribu et l'induction de topologie : la tribu induite à partir des boréliens sur un sous-espace topologique est la tribu des boréliens pour la topologie induite.
\begin{theorem}[\cite{TribuLi}]     \label{ThoSVTHooChgvYa}
    Soit \( X\), un espace topologique et \( Y\subset X\) une partie munie de la topologie induite. Alors
    \begin{equation}
        \Borelien(Y)=\Borelien(X)_Y
    \end{equation}
    où \( \Borelien(X)_Y\) est la tribu sur \( Y\) induite de \( \Borelien(X)\) par la définition \ref{DefDHTTooWNoKDP}.
\end{theorem}

\begin{proof}
    Nous notons \( \tau_X\) et \( \tau_Y\) les topologies de \( X\) et \( Y\). 
    \begin{subproof}
        \item[\( \Borelien(Y)\subset\Borelien(X)_Y\)]
            Si \( A\in \tau_Y\) alors \( A=Y\cap \Omega\) pour un \( \Omega\in \tau_X\). Mais vu que \(\Omega\) est un ouvert de \( X\), il est un borélien de \( X\), ce qui donne que \( Y\cap\Omega\) est un élément de \( \Borelien(X)_Y\). Cela prouve que \( \tau_Y\subset\Borelien(X)_Y\), c'est à dire que \( \Borelien(X)_Y\) est une tribu sur \( Y\) contenant les ouverts de \( Y\). Nous avons donc
            \begin{equation}
                \Borelien(X)\subset\Borelien(X)_Y.
            \end{equation}
        \item[Réciproquement]
            L'application \( \id\colon (Y,\tau_Y)\to (X,\tau_X)\) est continue parce que si \( \Omega\) est ouvert de \( X\) alors \( \id^{-1}(\Omega)=\Omega\cap Y\in \tau_Y\). Par conséquent l'identité est une application borélienne (théorème \ref{ThoJDOKooKaaiJh}), ce qui signifie que \( \id^{-1}\big( \Borelien(X) \big)\subset\Borelien(Y)\), ou encore que si \( B\in\Borelien(X)\), alors \( \id^{-1}(B)=B\cap Y\in\Borelien(Y)\). Cela signifie que 
            \begin{equation}
                \Borelien(X)_Y\subset \Borelien(Y).
            \end{equation}
    \end{subproof}
\end{proof}

\begin{corollary}       \label{CorooMJQYooFfwoTd}
    Si \( U\) est un borélien de l'espace topologique \( X\), alors les boréliens de \( U\) sont les boréliens de \( X\) inclus à \( U\) :
    \begin{equation}
        \Borelien(U)=\{ B\in\Borelien(X)\tq B\subset U \}.
    \end{equation}
\end{corollary}

\begin{proof}
    Si \( B'\in\Borelien(U)\), le théorème \ref{ThoSVTHooChgvYa} donne un borélien \( B\in\Borelien(X)\) tel que \( B'=B\cap U\). Mais \( U\) étant borélien de \( X\), l'intersection \( B\cap U\) est encore un borélien de \( X\).
\end{proof}
Ce corollaire s'applique en particulier lorsque \( U\) est un ouvert.

%--------------------------------------------------------------------------------------------------------------------------- 
\subsection{Mesure image}
%---------------------------------------------------------------------------------------------------------------------------

Le produit d'une mesure par une fonction est définit par la propriété \ref{PropooVXPMooGSkyBo}.

\begin{propositionDef}[Mesure image\cite{TribuLi}]     \label{PropJCJQooAdqrGA}
    Soient \( (S_1,\tribF_1)\) et \( (S_2,\tribF_2)\) des espaces mesurables. Soit \( \varphi\colon S_1\to S_2\) une application mesurable. Si \( m_1\) est une mesure positive sur \( S_1\) alors l'application définie par
    \begin{equation}
        m_2(A_2)=m_1\big( \varphi^{-1}(A_2) \big)
    \end{equation}
    est une mesure positive sur \( (S_2,\tribF_2)\).

    La mesure \( m_2\) ainsi définie est la \defe{mesure image}{mesure!image} de \( m_1\) par l'application \( \varphi\). Elle est notée \( \varphi(m_1)\).
\end{propositionDef}

\begin{proof}
    Il y a deux choses à vérifier pour avoir une mesure positive\footnote{Définition \ref{DefBTsgznn}}. D'abord pour l'ensemble vide :
    \begin{equation}
        m_2(\emptyset)=m_1\big( \varphi^{-1}(\emptyset) \big)=m_1(\emptyset)=0.
    \end{equation}
    Ensuite pour l'additivité. Soient \( A_n\) dans \( \tribF_2\) des parties deux à deux disjointes et telles que \( \bigcup_nA_n\in\tribF_2\). Alors nous avons
    \begin{subequations}
        \begin{align}
            m_2\big( \bigcup_nA_n \big)&=m_1\Big( \varphi^{-1}(\bigcup_nA_n) \Big)\\
            =&m_1\big( \bigcup_n\varphi^{-1}(A_n) \big)\\
            &=\sum_nm_1\big( \varphi(A_n) \big)\\
            &=\sum_nm_2(A_n).
        \end{align}
    \end{subequations}
\end{proof}

\begin{lemma}
    Soient deux espaces mesurables \( (S_1,\tribF_1)\) et \( (S_2,\tribF_2)\) ainsi que deux mesures \( \mu\) et \( \nu\) sur \( (S_1,\tribF_1)\). Si \( \varphi\colon S_1\to S_2\) est mesurable et si \( \mu\leq \nu\) alors \( \varphi(\mu)\leq \varphi(\nu)\).
\end{lemma}

\begin{proof}
    Soit \( B\) mesurable dans \( (S_2,\tribF_2)\) (c'est à dire \( B\in \tribF_2\)). Alors
    \begin{equation}
        \varphi(\mu)(B)=\mu\big( \varphi^{-1}(B) \big)\leq\nu\big( \varphi^{-1}(B) \big)=\varphi(\nu)(B).
    \end{equation}
\end{proof}

Il est naturel de se demander comment il faut intégrer par rapport à une mesure image. La réponse sera dans le théorème \ref{THOooVADUooLiRfGK}.

%--------------------------------------------------------------------------------------------------------------------------- 
\subsection{Fonction étagée}
%---------------------------------------------------------------------------------------------------------------------------

\begin{definition}[\cite{ooARRSooBLWdam}]\label{DefBPCxdel}
    Soit \( (S,\tribF)\) un espace mesurable et une fonction \( f\colon S\to \big( \bar\eR,\Borelien(\bar\eR) \big)\). Il serait dommage de confondre les trois concepts suivants.
    \begin{itemize}
        \item
    Une \defe{fonction simple}{simple!fonction} est une fonction dont l'image est constituée d'un nombre fini de valeurs.
\item
    Une \defe{fonction étagée}{étagée!fonction} est une fonction simple qui est elle-même une fonction mesurable.
\item
    Une \defe{fonction en escalier}{escalier} est une fonction étagée dont les valeurs sont constantes sur des intervalles : ce sont donc des fonctions constantes par morceaux.
    \end{itemize}
\end{definition}

Dans les trois cas, la fonction \( f\) peut être écrite comme somme de fonctions caractéristiques :
\begin{equation}
    f(x)=\sum_{j=1}^p\alpha_j\mtu_{A_j}(x)
\end{equation}
où \( A_j=f^{-1}(\alpha_j)\). Ce qui change est la nature des \( A_j\).

\begin{itemize}
    \item Si \( f\) est  simple, les \( A_j\) sont quelconques.
    \item Si \( f\) est étagée, les \( A_i\) peuvent être choisis mesurables parce que \( \{\alpha_i \}\) est un borélien, ce qui fait que \( A_i=f^{-1}(\alpha_i)\) un choix mesurable.
    \item Si \( f\) est en escalier, les \( A_i\) sont des intervalles.
\end{itemize}

La \defe{forme canonique}{forme canonique!fonction simple} d'une fonction simple \( f\) est la suivante. Soit \( \{ \alpha_i \}_{i=1,\ldots, l}\) les valeurs distinctes prises par \( f\) et \( A_i=f^{-1}(\alpha_i)\). La forme canonique de \( f\) est alors
\begin{equation}
    f=\sum_{i=1}^l\alpha_i\mtu_{A_i}.
\end{equation}
Notons que nous avons \( S=\bigcup_iA_i\), et que cette union est disjointe dans le cas d'une représentation canonique.

\begin{lemma}[Limite croissante de fonctions étagées]    \label{LemYFoWqmS}
    Soit \( f\colon (\Omega,\tribA)\to \eR\) une fonction mesurable. Il existe une suite \( f_n\colon \Omega\to \eR\) de fonctions étagées telles que \( f_n\to f\) ponctuellement et \( | f_n |<f\).
\end{lemma}

\begin{proof}
    Nous considérons \( (q_n)\) une suite parcourant tous les rationnels\footnote{Nous rappelons que \( \eQ\) est dénombrable et dense dans \( \eR\) par la proposition \ref{PropooUHNZooOUYIkn}.}.
    Pour \( n\in \eN\) nous définissons la fonction
    \begin{equation}
        f_n(\omega)=\begin{cases}
            \max\{ q_i\tq i\leq n,\, q_i\leq f(\omega) \}    &   \text{si \( f(\omega)\geq 0\)}\\
            \min\{ q_i\tq i\leq n,\, q_i\geq f(\omega) \}    &    \text{si \( f(\omega)< 0\).}
        \end{cases}
    \end{equation}
    La fonction \( f_n\) est simple parce qu'elle ne prend que \( n\) valeurs différentes. Nous avons aussi par construction que \( | f_n(\omega)|\leq |f(\omega) |\). Nous avons aussi pour tout \( \omega\in \Omega\) que \( f_n(\omega)\to f(\omega)\) parce que \( \eQ\) est dense dans \( \eR\).

    En ce qui concerne la mesurabilité de \( f_n\), les étages de \( f_n\) sont les ensembles de la forme \( \{ \omega\in \Omega\tq f(\omega)\in\mathopen[ a , b [ \}\) où \( a\) et \( b\) sont deux éléments de \( \{ q_1,\ldots, q_n \}\) qui sont consécutifs au sens de l'ordre dans \( \eQ\) (et non spécialement au sens de l'ordre d'apparition dans la suite), avec éventuellement \( b=\infty\) si \( a\) est le plus grand. Les ensembles \( \mathopen[ a , b [\) étant mesurables dans \( \eR\) et la fonction \( f\) étant mesurable par hypothèse, les ensembles \( f^{-1}\Big( \mathopen[ a , b [ \Big)\) sont mesurables dans \( (\Omega,\tribA)\).
\end{proof}

\begin{proposition}\label{PropWBavIf}
    Une fonction positive et mesurable sur \( \Omega\) est limite ponctuelle croissante de fonctions simples positives.
\end{proposition}

\begin{proof}
    Soit \( \{ q_i \}\) une énumération des rationnels positifs. Il suffit de poser
    \begin{equation}
        f_n(x)=\max\{ q_i\tq i\leq n,\text{ et }f(x)\geq q_i \}.
    \end{equation}
\end{proof}

\begin{theorem}\label{THOooXHIVooKUddLi}       
    Une fonction mesurable à valeurs dans \( \bar \eR^+\) est limite (ponctuelle) croissante de fonctions étagées positives.
\end{theorem}

%--------------------------------------------------------------------------------------------------------------------------- 
\subsection{Régularité d'une mesure}
%---------------------------------------------------------------------------------------------------------------------------

Certaines mesures ont de la compatibilité avec la topologie. Nous allons étudier ça.

\begin{theorem}[\cite{TribuLi}]     \label{ThoPKGEooVrpsGU}
    Soit \( X\) un espace métrique et \( m\) une mesure positive bornée sur \( \big(X,\Borelien(X)\big)\). Alors si \( B\) est un borélien,
    \begin{enumerate}
        \item
            Régularité extérieure : \( m(B)=\inf\{ m(\Omega)\text{où \( \Omega\) est un ouvert contenant \( B\)} \}\)
        \item
            Régularité intérieure : \( m(B)=\sup\{ m(F) \text{où \( F\) est un fermé, \( F\subset B\)} \}\).
    \end{enumerate}
\end{theorem}

\begin{proof}
    Soit \( \tribF\) l'ensemble des \( B\in\Borelien(X)\) tels que pour tout \( \epsilon>0\), il existe \( \Omega_{\epsilon}\) ouvert et \( F_{\epsilon}\) fermé tels que \( F_{\epsilon}\subset B\subset \Omega_{\epsilon}\) et \( m(\Omega_{\epsilon}\setminus F_{\epsilon})\leq \epsilon\). Nous allons montrer que cela est une tribu contenant les ouverts. Comme cela est inclus à la tribu borélienne, nous en déduirons que \( \tribF=\Borelien(X)\).
    \begin{subproof}
        \item[\( \tribF\) contient les ouverts]
            Soit \( \Omega\) un ouvert de \( X\). Alors \( \Omega^c\) est fermé et \( d(x,\Omega^c)=0\) si et seulement si \( x\in \Omega^c\) par la proposition \ref{PropGULUooNzqZKj}. Nous pouvons donc écrire
            \begin{equation}
                 \Omega^c=\bigcap_{n\geq 1}\{ x\in X\tq d(x,\Omega^c)<\frac{1}{ n } \}.
            \end{equation}
            En passant au complémentaire et en posant \( F_n=\{ x\in X\tq d(x,\Omega^c)\geq \frac{1}{ n } \}\) nous avons
            \begin{equation}
                \Omega=\bigcup_{n\geq 1}F_n.
            \end{equation}
            Chacun des \( F_n\) est fermé parce que \( F_n\) est l'image réciproque du fermé \( \mathopen[ \frac{1}{ n } , \infty \mathclose[\) par l'application \( x\mapsto d(x,\Omega^c)\) qui est continue. De plus les \( F_n\) forment une suite croissante, donc le lemme \ref{LemAZGByEs} nous assure que \( m(\Omega)=\lim_{n\to \infty}m(F_n)\). Et le lemme \ref{LemPMprYuC} que \( m(\Omega\setminus F_n)=m(\Omega)-m(F_n)\).
                
                Soit \( \epsilon>0\). Il existe alors \( n_{\epsilon}\geq 1\) tel que 
                \begin{equation}
                    m(\Omega\setminus F_n)=m(\Omega)-m(F_n)\leq \epsilon.
                \end{equation}
                Bref si \( \Omega\) est ouvert nous considérons \( \Omega_{\epsilon}=\Omega\) et \( F_{\epsilon}=F_{n_{\epsilon}}\) et nous avons
                \begin{equation}
                    F_{\epsilon}\subset \Omega\subset \Omega_{\epsilon}
                \end{equation}
                avec \( m(\Omega_{\epsilon}\setminus F_{\epsilon})\leq \epsilon\).

                L'ensemble \( \tribF\) contient les ouverts.

            \item[\( \tribF\) est une tribu]
                Il y a à vérifier les trois conditions de la définition \ref{DefjRsGSy}.
                \begin{subproof}
                \item[Les ensembles faciles]
                    Les ensembles \( X\) et \( \emptyset\) sont dans \( \tribF\) parce qu'ils sont ouverts et fermés.
                \item[Complémentaire]
                    Soit \( B\in \tribF\), soit \( \epsilon>0\) et les ensembles \( F_{\epsilon} \) et \( \Omega_{\epsilon}\) qui vont avec. Alors en passant au complémentaire nous avons
                    \begin{equation}
                        \Omega_{\epsilon}^c\subset B^c\subset F_{\epsilon}^c
                    \end{equation}
                    De plus
                    \begin{equation}
                        F_{\epsilon}^c\setminus \Omega_{\epsilon}^c=F_{\epsilon}^c\cap(\Omega_{\epsilon}^c)^c=F_{\epsilon}^c\cap \Omega_{\epsilon}=\Omega_{\epsilon}\setminus F_{\epsilon}.
                    \end{equation}
                    Par conséquent
                    \begin{equation}
                        m(F_{\epsilon}^c\setminus \Omega_{\epsilon}^c)=m(\Omega_{\epsilon}\setminus F_{\epsilon})\leq \epsilon.
                    \end{equation}
                    Cela montre que \( B^c\in \tribF\).
                \item[Union dénombrable]
                    Soient \( (B_n)\) une suite d'éléments de \( \tribF\) et \( \epsilon>0\). Pour chaque \( n\) nous choisissons un ouvert \( \Omega_n\) et un fermé \( F_n\) tels que \( F_n\subset  B_n\subset \Omega_n\) et 
                    \begin{equation}
                        m(\Omega_n\setminus F_n)\leq \frac{ \epsilon }{ 2^{n+2} }.
                    \end{equation}
                    Vu que \( \Omega_n\setminus B_n\subset \Omega_n\setminus F_n\) nous avons aussi
                    \begin{equation}
                        m(\Omega_n\setminus B_n)\leq m(\Omega_n\setminus F_n)\leq \frac{ \epsilon }{ 2^{n+2} }.
                    \end{equation}
                    Nous posons \( \Omega=\bigcup_{n\geq 1}\Omega_n\) (un ouvert) et \( B=\bigcup_{n\geq 1}B_n\) ainsi que \( A=\bigcup_{n\geq 1}F_n\) (qui n'est pas spécialement fermé).

                    Le but est de majorer \( m(\Omega\setminus F)\) où \( F\) est un fermé qui est encore à déterminer. Calculons déjà ceci :
                    \begin{subequations}
                        \begin{align}
                            \Omega\setminus B&=\bigcup_n\Omega_n\cap\big( \bigcup_kB_k \big)^c\\
                            &=\bigcup_n\Big( \Omega_n\cap\big( \bigcap_kB_k^c \big) \Big)\\
                            &\subset\bigcup_n\big( \Omega_n\cap B_n^c \big)\\
                            &=\bigcup_n(\Omega_n\setminus B_n)
                        \end{align}
                    \end{subequations}
                    où l'union n'est pas spécialement disjointe. Par conséquent,
                    \begin{equation}
                        m(\Omega\setminus B)\leq \sum_{n=1}^{\infty}m(\Omega_n\setminus B_n)\leq \sum_{n=1}^{\infty}\frac{ \epsilon }{ 2^{n+2} }=\frac{ \epsilon }{ 4 }.
                    \end{equation}
                    De la même façon nous avons
                    \begin{equation}
                        B\setminus A=\big( \bigcup_{n=1}^{\infty}B_n \big)\cap\big( \bigcup_{k=1}^{\infty}F_n \big)^c\subset \bigcup_{n=1}^{\infty}B_n\setminus F_n.
                    \end{equation}
                    Nous avons alors le inégalités de mesures
                    \begin{subequations}
                        \begin{align}
                            m(B\setminus A)&\leq \sum_{n=1}^{\infty}m(B_n\setminus F_n)\\
                            &\leq\sum_{n=1}^{\infty}m(\Omega_n\setminus F_n)\\
                            &\leq \frac{ \epsilon }{ 4 }.
                        \end{align}
                    \end{subequations}
                    C'est vraiment dommage que \( A\) ne soit pas en générale un fermé, sinon il répondrait à la question. Nous posons \( F'_1=F_1\) et \( F'_n=\bigcup_{k=1}^nF_k\). En tant qu'unions finies de fermés, les \( F'_n\) sont des fermés (lemme \ref{LemQYUJwPC}\ref{ItemKJYVooMBmMbG}). De plus la suite \( (F'_n)\)  est croissante et l'union est \( A\). Par le lemme \ref{LemAZGByEs}\ref{ItemJWUooRXNPci} nous avons
                    \begin{equation}
                        m(A)=m\big( \bigcup_nF'_n \big)=\lim_{n\to \infty} m(F'_n).
                    \end{equation}
                    Il existe donc \( n_{\epsilon}\) tel que 
                    \begin{equation}
                        m(A)-m(F'_{n})\leq \epsilon
                    \end{equation}
                    Nous posons \( F=F'_{n_{\epsilon}}\). Vu que \( F\subset A\) nous avons aussi \( m(A\setminus F)=m(A)-m(F)\leq \epsilon\). Et en plus \( F\subset A\subset B\subset \Omega\), ce qui donne bien la propriété voulue \( F\subset B\subset \Omega\). Il reste à nous assurer de \( m(\Omega\setminus F)\). Nous avons d'abord
                    \begin{equation}
                        m(B\setminus F)=m\big( (B\setminus A)\cup (A\setminus F) \big)=m(B\setminus A)+m(A\setminus F)\leq \frac{ 5\epsilon }{ 4 }.
                    \end{equation}
                    Et enfin :
                    \begin{equation}
                        m(\Omega\setminus F)=m\big( (\Omega\setminus B)\cup (B\setminus F) \big)=m(\Omega\setminus B)+m(B\setminus F)\leq \frac{ 6\epsilon }{ 4 }.
                    \end{equation}
                    Et donc à redéfinition près de \( \epsilon\) c'est d'accord. 
                    
                \end{subproof}

                Il est donc établi que \( \tribF\) est une tribu. Qui plus est, l'ensemble \( \tribF\) est une tribu incluse aux boréliens et contenant les ouverts. Ergo \( \tribF=\Borelien(X)\).

            \item[Régularité extérieure] 

                Soit \( B\) un borélien et \( \epsilon>0\). Alors il existe \( F_{\epsilon}\) fermé et \( \Omega_{\epsilon} \) ouvert tels que \( F_{\epsilon}\subset B\subset \Omega_{\epsilon}\) et \( m(\Omega_{\epsilon}\setminus F_{\epsilon})\leq \epsilon\). Vu que \( B\subset \Omega_{\epsilon}\) pour tout \( \epsilon\), nous avons aussi
                \begin{equation}
                    m(B)\leq \inf_{\epsilon}m(\Omega_{\epsilon}).
                \end{equation}
                Mais comme \( \mu(\Omega_{\epsilon})\geq m(B)\) pour tout \( \epsilon\), nous avons en réalité \( m(B)=\inf_{\epsilon}m(\Omega_{\epsilon})\).
                
                Soit maintenant un ouvert \( \Omega\) tel que \( B\subset \Omega\). Nous devons prouver l'existence d'un \( \epsilon>0\) tel que \( m(\Omega_{\epsilon})\leq m(\Omega)\). Cela permettra de conclure que l'infimum sur tous les ouverts contenant \( B\) est égal à l'infimum sur les ouverts de la forme \( \Omega_{\epsilon}\).

                Nous posons \( m(\Omega)=m(B)+\delta\) et avec \( \epsilon\leq \delta\) nous avons
                \begin{equation}
                    m(\Omega_{\epsilon}\setminus B)\leq m(\Omega_{\epsilon}\setminus F_{\epsilon})\leq \epsilon
                \end{equation}
                et donc aussi
                \begin{equation}
                    m(\Omega_{\epsilon})\leq m(B)+\epsilon\leq m(B)+\delta=m(\Omega).
                \end{equation}
            \item[Régularité intérieure]

                Elle se fait de même.
    \end{subproof}
\end{proof}

\begin{definition}      \label{DefFMTEooMjbWKK}
    Soit \( X\) un espace topologique et \( m\) une mesure positive sur \( \big( X,\Borelien(X) \big)\).
    \begin{enumerate}
        \item       \label{ItemTTPTooStDcpw}
            \( m\) est une \defe{mesure de Borel}{mesure!de Borel} si elle est finie sur tout compact.
        \item
            \( m\) est \defe{régulière extérieurement}{mesure!régulière!extérieure} si \( \forall B\in\Borelien(X)\), 
            \begin{equation}
                m(B)=\inf\{ m(\Omega)\tq\text{\( \Omega\) est ouvert et \( B\subset \Omega\)} \}
            \end{equation}
        \item
            \( m\) est \defe{régulière intérieurement}{mesure!régulière!intérieure} si \( \forall B\in\Borelien(X)\), 
            \begin{equation}
                m(B)=\sup\{ m(K)\tq\text{\(K\) est compact et \( K\subset B \)} \}
            \end{equation}
        \item
            \( m\) est une mesure \defe{régulière}{mesure!régulière} si elle est régulière dans les deux sens.
        \item
            \( m\) est une \defe{mesure de Radon}{mesure!de Radon} si elle est de Borel et régulière.
    \end{enumerate}
\end{definition}
\index{régularité!d'une mesure}

\begin{proposition}     \label{PropNCASooBnbFrc}
    Soit \( X\) un espace localement compact et dénombrable à l'infini\footnote{Définitions \ref{DefEIBYooAWoESf} et \ref{DefFCGBooLpnSAK}.} Alors toute mesure de Borel sur \( \big( X,\Borelien(X) \big)\) est de Radon.
\end{proposition}

\begin{proof}
    Nous avons une suite exhaustive\footnote{Définition \ref{DefTBJXooONOgxb}.} de compacts \( X_k\) tels que
    \begin{equation}
        X=\bigcup_{k\geq 1}X_k=\bigcup_{k\geq 1}\Int(X_k).
    \end{equation}
    \begin{subproof}
    \item[Régularité intérieure]
    Soit \( B\), un borélien de \( X\); nous avons \( B=\bigcup_{k\geq 1}(B\cap X_k)\) et comme cette union est croissante,
    \begin{equation}
        m(B)=\lim_{k\to \infty} m(B\cap X_k)
    \end{equation}
    par le lemme \ref{LemAZGByEs}\ref{ItemJWUooRXNPci}. Dans la suite, il va y avoir beaucoup de considérations sur les topologies induites. Nous nommons \( \tau_k\) la topologie de \( X_k\) induite depuis celle de \( X\). Il ne faudra pas confondre les expressions «un compact \emph{de} $X_k$»  et «un compact \emph{dans} \( X_k\)». La première parle d'un compact pour la topologie \( \tau_k\). La seconde parle d'un compact pour la topologie de \( X\), inclus à \( X_k\).
    
    
    Si \( a<m(B)\) alors il existe \( k\geq 1\) tel que \( a<m(B\cap X_k)\), c'est à dire
    \begin{equation}
        a<m(B\cap X_k)\leq m(B).
    \end{equation}
    Mais \( (X_k,m)\) est un espace mesuré borné parce que \( m\) est de Borel et \( X_k\) est compact. Par conséquent la (restriction de la) mesure \( m\) est régulière sur l'espace mesuré \( \big( X_k,\Borelien(X_k) \big)\) par le théorème \ref{ThoPKGEooVrpsGU}. De plus l'ensemble \( B\cap X_k\) est un borélien de \( (X_k,\tau_k)\) parce que 
    \begin{equation}
        B\cap X_k\in\Borelien(X)_{X_k}=\Borelien(X_k)
    \end{equation}
    où nous avons utilisé la propriété de compatibilité entre topologie induite et tribu des borélien du théorème \ref{ThoSVTHooChgvYa}. Il existe donc un fermé \( F_{\epsilon}\) de \( (X_k,\tau_k)\) tel que 
    \begin{subequations}
        \begin{numcases}{}
            F_{\epsilon}\subset B\cap X_k\\
            m(B\cap X_k)\leq m(F_{\epsilon})+\epsilon.
        \end{numcases}
    \end{subequations}
    En mettant bout à bout les inégalités nous avons trouvé
    \begin{equation}
        a<m(B\cap X_k)\leq m(F_{\epsilon})+\epsilon<m(F_{\epsilon}),
    \end{equation}
    et donc en particulier \( a<m(F_{\epsilon})\). L'ensemble \( F_{\epsilon}\) est en plus un compact de \( (X,\tau_X)\). En effet \( X_k\) étant fermé de \( (X,\tau_X)\), le lemme \ref{LemBWSUooCCGvax} nous dit que \( F_{\epsilon}\) est un fermé de \( (X,\tau_X)\). Mais \( X_k\) étant compact, \( F_{\epsilon}\) est un fermé inclus à un compact, il est donc compact (lemme \ref{LemnAeACf}).

    Pour tout \( a<m(B)\) nous avons trouvé un compact \( F_{\epsilon}\) inclus à \( B\) dont la mesure est plus grande que \( a\). Cela prouve la régularité intérieure de la mesure \( m\).

\item[Régularité extérieure]

    Soit un borélien \( B\) de \( X\). Si \( m(B)=\infty\) alors tous les ouverts contenant \( B\) ont mesure infinie et \( m(B)\) en est évidemment le supremum. Nous supposons donc que \( m(B)<\infty\). 

    Nous notons \( \tau_k\) la topologie induite de \( X\) sur \( \Int(X_k)\). Nous posons \( B_k=B\cap\Int(X_k)\). L'espace \( \big( \Int(X_k),m \big)\) est un espace mesuré borné et \( B_k\in \Borelien\Big( \Int(X_k) \Big)\). Il existe donc un ouvert \( \Omega_k\) de \( \big( \Int(X_k),\tau_k \big)\) tel que \( B_k\subset \Omega_k\) et 
    \begin{equation}
        m(\Omega_k\setminus B_k)\leq \frac{ \epsilon }{ 2^k }.
    \end{equation}
    De plus \( \Int(X_k)\) est un ouvert de \( (X,\tau_X)\), donc en réalité \( \Omega_k\) est un ouvert de \( X\). Nous posons
    \begin{equation}
        \Omega=\bigcup_{k=1}^{\infty}\Omega_k
    \end{equation}
    qui est encore un ouvert de \( (X,\tau_X)\). 
    
    Il est temps de voir que \( \Omega\) vérifie \( m(\Omega\setminus B)\leq \epsilon\). Pour cela,
    \begin{subequations}
        \begin{align}
            \Omega\setminus B=\big( \bigcup_k\Omega_k \big)\cap\big( \bigcup_lB_l \big)^c\\
            &=\big( \bigcup_k\Omega_k \big)\cap\big( \bigcap B_l^c \big)\\
            &\subset\bigcup_k(\Omega_k\cap B_k^c)\\
            &=\bigcup_k(\Omega_k\setminus B_k),
        \end{align}
    \end{subequations}
    ce qui donne au niveau des mesures :
    \begin{equation}
        m(\Omega\setminus B)\leq\sum_{k=1}^{\infty}m(\Omega_k\setminus B_k)\leq\sum_{k=1}^{\infty}\frac{ \epsilon }{ 2^k }=\epsilon.
    \end{equation}
    \end{subproof}
\end{proof}

\begin{remark}      \label{RemooOAGCooRHpjxd}
    Exprimé sur \( \eR^N\), la proposition \ref{PropNCASooBnbFrc} s'exprime en disant que toute mesure de Borel sur \( \eR^N\) est régulière. Typiquement, l'espace \( X\) dont il est question est un ouvert de \( \eR^N\).
\end{remark}

%---------------------------------------------------------------------------------------------------------------------------
\subsection{Théorème de récurrence}
%---------------------------------------------------------------------------------------------------------------------------

Soit \( X\) un espace mesurable, \( \mu\) une mesure finie sur \( X\) et \( \phi\colon X\to X\) une application mesurable préservant la mesure, c'est à dire que pour tout ensemble mesurable \( A\subset X\),
\begin{equation}
    \mu\big( \phi^{-1}(A) \big)=\mu(A).
\end{equation}
Si \( A\subset X\) est un ensemble mesurable, un point \( x\in A\) est dit \defe{récurrent}{récurrent!point d'un système dynamique} par rapport à \( A\) si et seulement si pour tout \( p\in \eN\), il existe \( k\geq p\) tel que \( \phi^k(x)\in A\).

\begin{theorem}[\wikipedia{fr}{Théorème_de_récurrence_de_Poincaré}{Théorème de récurrence de Poincaré}.]     \label{ThoYnLNEL}
    Si \( A\) est mesurable dans \( X\), alors presque tous les points de \( A\) sont récurrents par rapport à \( A\).
\end{theorem}

\begin{proof}
    Soit \( p\in \eN\) et l'ensemble
    \begin{equation}
        U_p=\bigcup_{k=p}^{\infty}\phi^{-k}(A)
    \end{equation}
    des points qui repasseront encore dans \( A\) après \( p\) itérations  de \( \phi\). C'est un ensemble mesurable en tant que union d'ensembles mesurables (pour rappel, les tribus sont stables par union dénombrable, comme demandé à la définition \ref{DefjRsGSy}), et nous avons donc
    \begin{equation}
        \mu(U_p)\leq \mu(X)<\infty.
    \end{equation}
    De plus \( U_p=\phi^{-p}(U_0)\), donc \( \mu(U_p)=\mu(U_0)\). Vu que \( U_p\subset U_p\), nous avons
    \begin{equation}
        \mu(U_0\setminus U_p)=0.
    \end{equation}
    Étant donné que \( A\subset U_0\) nous avons a fortiori que
    \begin{equation}
        \{ x\in A\tq x\notin U_p \}\subset U_0\setminus U_p,
    \end{equation}
    et donc
    \begin{equation}
        \mu\{ x\in A\tq x\notin U_p \}=0.
    \end{equation}
    Cela signifie exactement que l'ensemble des points \( x\) de \( A\) tels que aucun des \( \phi^k(x)\) avec \( k\geq p\) n'est dans \( A\) est de mesure nulle.
\end{proof}


%+++++++++++++++++++++++++++++++++++++++++++++++++++++++++++++++++++++++++++++++++++++++++++++++++++++++++++++++++++++++++++ 
\section{Mesurabilité des fonctions à valeurs réelles}
%+++++++++++++++++++++++++++++++++++++++++++++++++++++++++++++++++++++++++++++++++++++++++++++++++++++++++++++++++++++++++++

Nous allons parler de la mesurabilité de fonctions
\begin{equation}
    f\colon (S,\tribF)\to \big( \bar \eR,\Borelien(\bar \eR) \big)
\end{equation}
où \( \bar \eR=\eR\cup\{ \pm\infty \}\).

%--------------------------------------------------------------------------------------------------------------------------- 
\subsection{Quelque mots à propos de $\overline{ \eR }$}        % Pour quelque raisons, faire \bar \eR ne fonctionne pas dans le titre.
%---------------------------------------------------------------------------------------------------------------------------

\begin{normaltext}      \label{normooGAAJooUPCbzG}
Nous convenons que \( 0\times\pm\infty=0\) parce que nous voulons qu'une droite (qui est un rectangle dont une mesure est \( 0\) et l'autre \( \infty\)) soit de mesure nulle dans \( \eR^2\).

Les produits et sommes \( \pm\infty\pm\pm\infty\) et \( \pm\infty\times \pm\infty\) sont ceux que l'on croit. Sauf bien entendu \( +\infty-\infty\) et \( 1/0\) qui ne sont toujours pas définis.
\end{normaltext}

\begin{definition}[Topologie sur \( \bar\eR\)]
La topologie sur \(\bar \eR\) est celle sur \( \eR\) à laquelle nous ajoutons les voisinages de \( \pm\infty\) de la façon suivante. Une partie \( V\) de \( \bar \eR\) est un voisinage de \( +\infty\) si il existe \( m>0\) tel que \( \mathopen] m , +\infty \mathclose]\subset V\).
\end{definition}

\begin{lemma}       \label{LEMooBLOLooAdNViv}
    L'ensemble \( B\) est un borélien de \( \bar \eR\) si et seulement si il existe un borélien \( B_0\) de \( \eR\) tel que \( B\) soit \( B_0\) ou \( B_0\cup\{ -\infty \}\) ou \( B_0\cup\{ -\infty \}\) ou \( B_0\cup\{ +\infty,-\infty \}\).
\end{lemma}

\begin{proof}
    Vu que la topologie usuelle sur \( \eR\) est la topologie induite de celle sur \( \bar \eR\), la tribu induite l'est aussi par le théorème \ref{ThoJDOKooKaaiJh}. Donc si \( B\) est un borélien de \( \bar \eR\), l'ensemble \( B\cap \eR\) est un borélien de \( \eR\).
\end{proof}

\begin{lemma}[\cite{TribuLi}]       \label{LemooCRVJooQosHPq}
    Si \( \mS_0\) est l'ensemble des intervalles du type 
    \begin{equation}
        \begin{aligned}[]
        \mathopen] \alpha , \beta \mathclose[,&&\mathopen[ -\infty , \beta \mathclose[,&&\mathopen] \alpha , +\infty \mathclose]
        \end{aligned}
    \end{equation}
    avec \( -\infty<\alpha<\beta<+\infty\) alors \( \sigma(\mS_0)=\Borelien(\bar\eR)\).
\end{lemma}

\begin{proof}
Les intervalles \( \mathopen] \alpha , \beta \mathclose[\) engendrent la topologie de \( \eR\)\footnote{Parce toutes les boules sont des intervalles de ce type et que les boules forment une base de topologie, proposition \ref{PropNBSooraAFr}.}, donc \( \Borelien(\eR)\subset\sigma(\mS_0)\). De plus le lemme \ref{LemBWNlKfA} nous autorise à dire que 
    \begin{equation}
        \bigcap_{n\geq 1}\mathopen[ n , +\infty \mathclose]=\{ +\infty \}\in\sigma(\mS_0).
    \end{equation}
    Par conséquent tous les ensembles énumérés dans le lemme \ref{LEMooBLOLooAdNViv} font partie de \( \sigma(\mS_0)\). Cela implique que \( \Borelien(\bar\eR)\subset\sigma(\mS_0)\).

    Pour l'inclusion inverse, \( \sigma(\mS_0)\) est engendré par des parties qui font parie de \( \Borelien(\bar \eR)\), donc \( \sigma(\mS_0)\subset\Borelien(\bar \eR)\).
\end{proof}

Pour la suite nous utilisons la notation (pratique en probabilité)
\begin{equation}
    \{ f<a \}=\{ x\in S\tq f(x)<a \}.
\end{equation}

%--------------------------------------------------------------------------------------------------------------------------- 
\subsection{Fonctions sur un espace mesuré à valeurs réelles}
%---------------------------------------------------------------------------------------------------------------------------

\begin{theorem}     \label{THOooWHFLooKYGsOm}
    Soit un espace mesurable \( (S,\tribF)\) et une fonction \( f\colon S\to \bar \eR\). Les faits suivants sont équivalents.
    \begin{enumerate}
        \item\label{ITEMooHAMHooYLqUhVi}
            La fonction \( f\) est mesurable.
        \item\label{ITEMooHAMHooYLqUhVii}
            L'ensemble \( \{ f<a \}\) est dans \( \tribF\) pour tout \( a\in \eR\)
        \item\label{ITEMooHAMHooYLqUhViii}
            L'ensemble \( \{ f\leq a \}\) est dans \( \tribF\) pour tout \( a\in \eR\)
    \end{enumerate}
\end{theorem}

\begin{proof}
    Plusieurs implications à prouver.
    \begin{subproof}
        \item[\ref{ITEMooHAMHooYLqUhVi}\( \Rightarrow\)\ref{ITEMooHAMHooYLqUhVii}]
            Vu que \( f\) est mesurable et que \( \mathopen[ -\infty , a \mathclose[\in\Borelien(\bar\eR)\), nous avons \( f^{-1}\big( \mathopen[ -\infty , a \mathclose[ \big)\in\tribF\).
        \item[\ref{ITEMooHAMHooYLqUhVii}\( \Rightarrow\)\ref{ITEMooHAMHooYLqUhVi}]
            Nous posons \( \tribA=\{ \mathopen[ -\infty , a \mathclose[\tq a\in \eR \}\). 

                Nous avons \( \tribA\subset\mS_0\) (le \( \mS_0\) du lemme \ref{LemooCRVJooQosHPq}). Et de plus,
            \begin{equation}
            \mathopen] \alpha , \beta \mathclose[=\mathopen[ -\infty , \beta \mathclose[\setminus\mathopen[ -\infty , \alpha \mathclose]=\mathopen[ -\infty , \beta \mathclose[\setminus\bigcap_{n\geq 1}\mathopen[ -\infty , \alpha+\frac{1}{ n } \mathclose[.
            \end{equation}
        Donc \( \mathopen] \alpha , \beta \mathclose[\in\sigma(\tribA)\).

            Et aussi :
            \begin{equation}
                \mathopen] \alpha , +\infty \mathclose]=\bar\eR\setminus\mathopen[ -\infty , \alpha+\frac{1}{ n } \mathclose[,
            \end{equation}
        ce qui donne \( \mathopen] \alpha , +\infty \mathclose]\in \sigma(\tribA)\).

        Au final, \( \mS_0\subset\sigma(\tribA)\) et donc \( \sigma(\mS_0)\subset\sigma(\tribA)\). Le lemme \ref{LemooCRVJooQosHPq} nous dit que \( \sigma(\mS_0)=\Borelien(\bar \eR)\). Nous avons donc bien \( \sigma(\mS_0)=\sigma(\tribA)=\Borelien(\bar\eR)\).

        par ailleurs, nous savons que \( f^{-1}(\tribA)\subset\tribF\) parce que les éléments de \( \tribA\) sont de la forme \( \{ f<a \}\). Cela donne \( \sigma\big( f^{-1}(\tribA) \big)=\tribF\). Mais \( \sigma\big( f^{-1}(\tribA) \big)\) peut aussi s'exprimer par le lemme de transfert \ref{LemOQTBooWGYuDU} : \( \sigma\big( f^{-1}(\tribA) \big)=f^{-1}\big( \sigma(\tribA) \big)\). En combinant les deux,
        \begin{equation}
            f^{-1}\big( \sigma(\tribA) \big)=\tribF,
        \end{equation}
        et en remplaçant \( \sigma(\tribA)\) par \( \Borelien(\bar \eR)\) nous avons ce que nous voulions :
        \begin{equation}
            f^{-1}\big( \Borelien(\bar\eR) \big)\in\tribF,
        \end{equation}
        ce qui signifie que \( f\) est mesurable.
        \item[\ref{ITEMooHAMHooYLqUhViii}\( \Rightarrow\)\ref{ITEMooHAMHooYLqUhVii}]
            Nous avons
            \begin{equation}
                \{ f<a \}=\bigcup_{n\geq 1}\{ f\leq a-\frac{1}{ n } \}.
            \end{equation}
            donc cela est une union dénombrable d'éléments de \( \tribF\). Donc \( \{ f<a \}\) est dans \( \tribF\).
        \item[\ref{ITEMooHAMHooYLqUhVi}\( \Rightarrow\)\ref{ITEMooHAMHooYLqUhViii}]
            Nous avons
            \begin{equation}
                \{ f\leq a \}=\{ f<a \}\cup f^{-1}\big( \mathopen[ -\infty , a \mathclose] \big).
            \end{equation}
            Le premier ensemble est dans \( \tribF\) par \ref{ITEMooHAMHooYLqUhVii}. Ensuite \( \mathopen[ -\infty , a \mathclose]\) est un fermé de \( \bar \eR\) et donc un borélien de \( \bar \eR\). Son image réciproque est donc un élément de \( \tribF\) parce que \( f\) est mesurable. Au final nous avons bien \( \{ f\leq a \}\in\tribF\).
    \end{subproof}
\end{proof}

\begin{lemma}[\cite{NBoIEXO}]   \label{LemFOlheqw}
    Une fonction \( f\colon X\to \eR\) est mesurable si et seulement si \( f^{-1}(I)\) est mesurable pour tout \( I\) de la forme \( \mathopen] a , \infty \mathclose[\).
\end{lemma}

\begin{proof}
    Nous devons prouver que \( f^{-1}(A)\) est mesurable dans \( X\) pour tout borélien \( A\) de \( \eR\). Nous posons
    \begin{equation}
        S=\{ A\subset \eR\tq f^{-1}(A)\text{ est mesurable dans \( X\)} \}
    \end{equation}
    et nous prouvons que cela est une tribu. D'abord \( f^{-1}(\eR)=X\), et \( X\) est mesurable, donc \( \eR\in S\). Ensuite si \( A\in S\) alors \( f^{-1}(A^c)=f^{-1}(A)^c\). En tant que complémentaire d'un mesurable de \( X\), l'ensemble \( f^{-1}(A)^c\) est mesurable dans \( X\). Et enfin si \( A_n\in S \) alors \( f^{-1}(\bigcup_nA_n)=\bigcup_nf^{-1}(A_n)\) qui est encore mesurable dans \( X\) en tant qu'union de mesurables.

    Donc \( S\) est une tribu qui contient tous les ensembles de la forme \( \mathopen] a , \infty \mathclose]\). Le lemme \ref{LemZXnAbtl} conclu que \( S\) contient tous les boréliens de \( \eR\).
\end{proof}

\begin{lemma}[\cite{NBoIEXO}]   \label{LemIGKvbNR}
    Soit \( f_n\colon X\to \eR\) une suite de fonctions mesurables\footnote{Ici \( X\) est un espace mesuré et \( \eR\) est muni des boréliens.}. Alors \( \sup_n f_n\) est mesurable.
\end{lemma}

\begin{proof}
    Nous avons
    \begin{subequations}
        \begin{align}
            (\sup f_n)^{-1}\big( \mathopen] a , \infty \mathclose] \big)&=\{ x\in X\tq (\sup f_n)(x)>a \}\\
            &=\bigcup_n\{ x\in X\tq f_n(x)>a \}\\
            &=\bigcup_nf_n^{-1}\big( \mathopen] a , \infty \mathclose] \big).
        \end{align}
    \end{subequations}
    Étant donné que \( f_n\) est mesurable et que \( \mathopen] a , \infty \mathclose]\) est mesurable, chacun des \( f_n^{-1}\big( \mathopen] a , \infty \mathclose] \big) \) est mesurable dans \( X\). Nous sommes en présence d'une union dénombrable de mesurables, donc \( (\sup f_n)^{-1}\big( \mathopen] a , \infty \mathclose] \big)\) est mesurable.

    Le lemme \ref{LemFOlheqw} conclu que \( \sup f_n\) est mesurable.
\end{proof}

\begin{proposition}\label{PropFYPEOIJ}
    Si \( f_n\) est une suite de fonctions mesurables et positives, alors la fonction \( \sum_nf_n\) est mesurable.
\end{proposition}

\begin{proof}
    Nous considérons les fonctions \( s_k(x)=\sum_{n=0}^kf_n(x)\) qui vaut éventuellement \( \infty\) en certains points. Nous avons
    \begin{equation}
        \sum_nf_n(x)=\sup_ks_k(x),
    \end{equation}
    donc le lemme \ref{LemIGKvbNR} nous donne la mesurabilité de la somme de \( f_n\).
\end{proof}

\begin{definition}
    Soit \( (S,\tribF)\) un espace mesurable. 
    Une \defe{partition mesurable dénombrable}{partition!dénombrable mesurable} de e \( S\) est une suite  \( (S_n)_{n\geq 1}\) de parties de \( S\) telles que
    \begin{enumerate}
        \item
            \( S_n\in\tribF\) pour tout \( n\),
        \item
            \( S_N\cap S_k=\emptyset\) si \( n\neq k\),
        \item
            \( S=\bigcup_{n\geq 1}S_n\).
    \end{enumerate}
\end{definition}

\begin{lemma}[Lemme de recollement]     \label{LEMooXAPQooPpZUmP}
    Soit \( (S_n)\) une partition mesurable dénombrable de l'espace mesurable $(S,\tribF)$. Soit \( (S',\tribF')\) un autre espace mesurable et des fonctions mesurables
    \begin{equation}
        f_n\colon (S_n,\tribF_{S_n})\to (S',\tribF')
    \end{equation}
    où \( \tribF_{S_n}\) est la tribu induite\footnote{Définition \ref{DefDHTTooWNoKDP}.}. Alors la fonction
    \begin{equation}
        \begin{aligned}
            f\colon (S,\tribF)&\to (S',\tribF') \\
            x&\mapsto \text{\( f_n(x)\) si \( x\in S_n\)} 
        \end{aligned}
    \end{equation}
    est mesurable.
\end{lemma}

\begin{proof}
    Soit \( A'\in\tribF'\); nous devons prouver que \( f^{-1}(A')\in \tribF\). Nous savons que 
    \begin{equation}        \label{EqooGKFFooEwTdtg}
        f^{-1}(A')=\bigcup_{n\geq 1}f_n^{-1}(A'),
    \end{equation}
    qui est une union dénombrable d'éléments \( f_n^{-1}(A')\in\tribF_{S_n}\). 

    Vu que \( S_n\in \tribF\) nous avons \( \tribF_{S_n}\subset\tribF\) parce qu'un élément de \( \tribF_{S_n}\) est de la forme \( S_n\cap B\) avec \( B\in\tribF\). Du coup pour chaque \( n\) nous avons
    \begin{equation}
        f_n^{-1}(A')\in\tribF_{S_n}\subset \tribF.
    \end{equation}
    Au final l'égalité \eqref{EqooGKFFooEwTdtg} écrit \( f^{-1}(A')\) comme une union d'éléments de \( \tribF\) et est donc un élément de \( \tribF\).
\end{proof}

\begin{proposition}     \label{PROPooODDVooEEmmTX}
    Soit \( (S,\tribF)\) un espace mesurable et des applications mesurables \( f,g\colon S\to \bar \eR\). Alors les fonctions suivantes sont mesurables :
    \begin{enumerate}
        \item
            \( \lambda f\) pour tout \( \lambda\in \eR\)
        \item
            \( f+g\) si elle existe.
        \item
            \( 1/f\) si elle existe.
        \item
            \( fg\).
    \end{enumerate}
\end{proposition}

\begin{proof}
    Commençons par clarifier « si elle existe». La fonction \( f+g\) n'existe pas au point \( x\in S\) si \( f(x)=+\infty\) et \( g(x)=-\infty\). La fonction \( 1/f\) n'existe pas au point \( x\in S\) si \( f(x)=0\). Voir le point \ref{normooGAAJooUPCbzG}.
    \begin{subproof}
    \item[La partie où \( f+g\) existe est mesurable]
        La partie de \( S\) sur laquelle \( f+g\) existe est 
        \begin{equation}
            \{ x\in S\tq \big( f(x),g(x) \big)\neq (+\infty,-\infty),\big( f(x),g(x) \big)\neq (-\infty,+\infty) \}.
        \end{equation}
        Nous avons
        \begin{equation}
            \{  (f,g)=(+\infty,-\infty) \}=\{ f=\infty \}\cap\{ g=-\infty \}
        \end{equation}
        qui est un ensemble mesurable parce que, par exemple,
        \begin{equation}
            \{ +\infty \}=\bigcap_{n\geq 1}\mathopen[ n , +\infty \mathclose].
        \end{equation}
        La cas \( (-\infty,+\infty)\) est identique, et au final la partie de \( S\) sur laquelle \( f+g\) n'existe pas est mesurable. Par complémentarité la partie sur laquelle \( f+g\) existe est également mesurable\footnote{Parfois on a envie de dire que l'affirmation «\( A\) est mesurable» ne passe pas le test de Popper.}.
    \item[Idem pour la partie sur laquelle \( 1/f\) existe]
        Idem.
    \item[Mesurabilité de \( \lambda f\)]
        Si \( \lambda=0\), nous avons une fonction constante dont la mesurabilité est évidente\footnote{Prenez quand même le temps d'y penser.}. Nous supposons \( \lambda>0\). Alors
        \begin{equation}
            \{ \lambda f<a \}=\{ f<a/\lambda \}\in \tribF.
        \end{equation}
        .Pour \( \lambda<0\) nous avons de la même manière
        \begin{equation}
            \{ \lambda f<a \}=\{ f>a/\lambda \}\in \tribF.
        \end{equation}
        Ce dernier point est suffisant pour que \( \lambda f\) soit mesurable par la théorème \ref{THOooWHFLooKYGsOm}\ref{ITEMooHAMHooYLqUhViii} et par complémentarité.
    \item[Mesurabilité de \( f+g\)]
        Soit \( a\in \eR\); le théorème \ref{THOooWHFLooKYGsOm} nous demande d'avoir envie de prouver que \(  \{ f+g <a\} \in \tribF \). Nous avons
        \begin{equation}
            f(x)+g(x)<a
        \end{equation}
        si et seulement si
        \begin{equation}
            f(x)<a-g(x)
        \end{equation}
        si et seulement si
        \begin{equation}
            \exists q\in \eQ\tq f(x)<q<a-g(x).
        \end{equation}
        Donc
        \begin{equation}
            \{ f+g<a \}=\bigcup_{q\in \eQ}\Big( \{ f<q \}\cap\{ g<a-r \} \Big),
        \end{equation}
        qui est une union dénombrable d'éléments de \( \tribF\). Donc \( \{ f+g<a \}\in \tribF\) et \( f+g\) est mesurable.

        Note qu'en toute rigueur il faudrait  «\( \cap\text{là où \( f+g\) est définie}\)» un peu partout, mais cela ne change rien parce que l'intersection de deux parties mesurables est mesurable.

    \item[Mesurabilité de \( 1/f\)]
        Soit \( a\in \eR\). Si \( a>0\) alors
        \begin{equation}
            \{ 1/f<a \}=\{ f<0 \}\cup\{ f>\frac{1}{ a } \}\in\tribF.
        \end{equation}
        et si \( a<0\) alors
        \begin{equation}
            \{ 1/f<a \}=\{ f<0 \}\cap\{ f>\frac{1}{ a } \}\in\tribF.
        \end{equation}
    \item[Mesurabilité de \( fg\)]
        Nous allons la prouver en plusieurs fois.
        \begin{subproof}
        \item[Si \( f\) est mesurable alors \( f^2\) est mesurable]
            Si \( a\leq 0\) alors \( \{ f^2<a \}=\emptyset\). Si \( a>0\) nous avons
            \begin{equation}
                 \{ f^2<a \}=\{ -\sqrt{a}<f<\sqrt{a} \}\in\tribF.
            \end{equation}
        \item[\( f\mtu_A\) est mesurable]
            Soit \( A\in \tribF\), et prouvons que \( f\mtu_A\) est mesurable. Par définition,
            \begin{equation}
                (f\mtu_A)(x)=\begin{cases}
                    f(x)    &   \text{si \( x\in A\)}\\
                    0    &    \text{si \( x\notin A\)}.
                \end{cases}
            \end{equation}
            Nous posons \begin{equation}
                \begin{aligned}
                    f_1\colon A^c&\to \bar \eR \\
                    x&\mapsto 0 
                \end{aligned}
            \end{equation}
            et
            \begin{equation}
                \begin{aligned}
                    f_2\colon A&\to \bar \eR \\
                    x&\mapsto f(x). 
                \end{aligned}
            \end{equation}
            Alors nous avons
            \begin{equation}
                (\mtu_Af)(x)=\begin{cases}
                    f_1(x)    &   \text{si \( x\in A^c\)}\\
                    f_2(x)    &    \text{si \( x\in  A\)}.
                \end{cases}
            \end{equation}
            Les ensembles \( A\) et $A^c$ forment une partition mesurable dénombrable de \( S\). La fonction \( f_1\) est mesurable; pour prouver que \( f_2\) est mesurable, nous l'écrivons \( f_2=f\circ j_A\) où \( j_A\colon A\to S\) est l'injection canonique.
        \end{subproof}
        
    \end{subproof}
\end{proof}
<++>

%--------------------------------------------------------------------------------------------------------------------------- 
\subsection{Fonctions réelle à variables réelles}
%---------------------------------------------------------------------------------------------------------------------------

Nous nous particularisons à présent au cas de fonctions
\begin{equation}
    f\colon \big( \eR,\Borelien(\eR) \big)\to \big( \bar\eR,\Borelien(\bar \eR) \big).
\end{equation}

\begin{remark}
    En théorie de l'intégration il se parle souvent de fonctions mesurables au sens de la tribu de Lebesgue : $f\colon \big( \eR,\Lebesgue(\eR) \big)\to \big( \bar\eR,\Borelien(\bar \eR) \big)$ où \( \Lebesgue(\eR)\) est la tribu de Lebesgue sur \( \eR\), c'est à dire la tribu complétée de celle des boréliens (définition \ref{DefooYZSQooSOcyYN}).

    Le fait est qu'en pratique, c'est déjà assez compliqué de construire des fonctions non mesurables au sens des boréliens; alors on va s'en contenter. Mais construire des fonctions non mesurables au sens de la tribu de Lebesgue, c'est encore plus compliqué.

    Nous allons donc nous contenter de donner des conditions assurant qu'une fonction \( f\colon  \big( \eR,\Borelien(\eR) \big)   \to (\bar \eR,\Borelien(\bar \eR))  \) soit mesurable. Ces fonctions seront a fortiori mesurables en les considérant comme fonctions \( f\colon   \big( \eR,\Lebesgue(\eR) \big)  \to  (\bar\eR,\Borelien(\bar \eR)) \).
\end{remark}

\begin{corollary}       \label{CorooJYDVooCrXVun}
    Si \( I\) est un intervalle de \( \eR\), alors toute application monotone \( f\colon I\to \eR\) est borélienne.
\end{corollary}

\begin{proof}
    Vu que \( f\) est monotone, l'ensemble \( \{ f<a \}\) est un intervalle. Or tous les intervalles sont boréliens, donc \( f\) est mesurable par le théorème \ref{THOooWHFLooKYGsOm}.
\end{proof}

\begin{definition}
Si \( I\) est un intervalle de \( \eR\), une fonction \( f\colon I\to \eR\) a une propriété (monotone, mesurable, continue, etc.) \defe{par morceaux}{morceau!fonction continue ou monotone}\index{monotone!par morceaux}\index{continue!par morceaux} si il existe une suite strictement croissante de points \( (x_I)_{i\in \eZ}\) dans \( I\) telle que \( f\) ait la propriété sur chacun des ouverts \( \mathopen] x_j ,x_{j+1} \mathclose[.\).
\end{definition}
Dans cette définition, les points sont labelles par \( \eZ\) et non par \( \eN\) parce que nous nous laissons la liberté d'avoir une infinité de points de chacun des deux côtés.

\begin{proposition}     \label{PropooLNBHooBHAWiD}
    Soit \( I\) un intervalle de \( \eR\) et une fonction \( f\colon I\to \eR\). Si \( f\) est continue ou monotone par morceaux sur \( I\) alors elle y est borélienne.
\end{proposition}

\begin{proof}
L'ensemble \( \{  \mathopen] x_j , x_{j+1} \mathclose[  \}_{j\in \eZ}\cup\{ x_i \}_{i\in \eZ}\) forme une partition mesurable dénombrable de \( I\) (les singletons sont des boréliens). À une belle redéfinition près de la numérotation (deux fois \( \eZ\) va dans \( \eN\)), nous les appelons \( (I_n)_{n\in \eN}\), et nous définissons les fonctions \( f_n\) comme étant les restrictions de \( f\) aux intervalles \( I_k\).

    Toute fonctions sur un singleton est mesurable. Toute fonctions continue sur un ouvert est mesurable (théorème \ref{ThoJDOKooKaaiJh}). Toute fonction monotone sur un ouvert est mesurable (corollaire \ref{CorooJYDVooCrXVun}).

    Le lemme de recollement \ref{LEMooXAPQooPpZUmP} donne alors la mesurabilité de \( f\).
\end{proof}

\begin{normaltext}
    Toutes les fonctions que nous pouvons écrire explicitement sont mesurables \ldots en tout cas toutes celles que l'on trouve en pratique.
    % position 17349-27383
\end{normaltext}

%+++++++++++++++++++++++++++++++++++++++++++++++++++++++++++++++++++++++++++++++++++++++++++++++++++++++++++++++++++++++++++ 
\section{Intégrale par rapport à une mesure}
%+++++++++++++++++++++++++++++++++++++++++++++++++++++++++++++++++++++++++++++++++++++++++++++++++++++++++++++++++++++++++++

Nous avons besoin d'un peu de théorie de l'intégration parce que la définition de la mesure sur un espace mesurable\footnote{Théorème \ref{ThoWWAjXzi}.} produit passe par une intégrale.

Une mesure \( \mu\) sur un espace mesurable \( (\Omega,\tribA)\) permet de définir une fonctionnelle linéaire sur l'ensemble des fonctions mesurables \( \Omega\to \eR\). Cette fonctionnelle linéaire est l'intégrale que nous allons définir à présent.

\begin{definition}  \label{DefTVOooleEst}
    Soit \( (\Omega,\tribA,\mu)\) un espace mesuré. Si \( Y\in \tribA\) et si \( f\) est une fonction étagée\footnote{Définition \ref{DefBPCxdel}.}, si sa forme canonique est \( f=\sum_{i=1}^n\alpha_i\mtu_{A_i}\) alors nous définissons
    \begin{equation}
        \int_Yfd\mu=\sum_i\alpha_i\mu(Y\cap A_i).
    \end{equation}
    Pour une fonction \( \tribA\)-mesurable générale \( f\colon \Omega\to \mathopen[ 0 , \infty \mathclose]\) nous définissons l'intégrale de \( f\) sur \( Y\) par
    \begin{equation}        \label{EqDefintYfdmu}
        \int_Yfd\mu=\sup\Big\{ \int_Yhd\mu\,\text{où \( h\) est une fonction étagée telle que \( 0\leq h\leq f\)} \Big\}.
    \end{equation}

    Si $f$ est mesurable à valeurs dans \( \bar \eR\) ou \( \eC\), l'intégrale se définit séparément pour les parties positives, négatives, réelles et imaginaires.
\end{definition}

\begin{remark}
    Toute fonction mesurable à valeurs dans \( \bar \eR\) est intégrable (l'intégrale vaut éventuellement \( +\infty\)). Au moment où une fonction commence à prendre des valeurs positives et négatives, nous demandons à pouvoir intégrer séparément les parties positive et négative. C'est pour cela que nous disons qu'une fonction \( f\) à valeurs dans \( \eR\) est intégrable si \( | f |\) l'est.

    Cela est indépendant du fait que \( \int_0^{\infty}f\) en tant que limite de \( \int_0^{M}f\) peut très bien exister grâce à des compensations, alors que \( \int_{\mathopen[ 0 , \infty \mathclose[}| f |\) n'existe pas.
\end{remark}

\begin{lemma}       \label{LemooPJLNooVKrBhN}
    Si \( (\Omega,\tribA,\mu)\) est un espace mesuré et si \( B\in \tribA\) alors
    \begin{equation}
        \mu(B)=\int_B1d\mu=\int_{\Omega}\mtu_B.
    \end{equation}
\end{lemma}

\begin{proof}
    La fonction caractéristique d'une partie mesurable est une fonction étagée dont la forme canonique est \( \mtu_B=1\cdot \mtu_B+0\times \mtu_{B^c}\). Son intégrale est donc
    \begin{equation}
        \int\mtu_Bd\mu=1\times \mu(B)+0\times \mu(B^c)=\mu(B)
    \end{equation}
    parce que \( 0\times \mu(B^c)=0\), même si \( \mu(B^c)=\infty\), comme nous l'avons convenu en \ref{normooGAAJooUPCbzG}.
\end{proof}

\begin{definition}  \label{DefTCXooAstMYl}
    Une fonction \( f\colon \Omega\to \eR\) est dite \defe{intégrable}{intégrable} au sens de Lebesgue si \( \int_{\Omega}| f |<\infty\). Dans ce cas nous définissons
    \begin{equation}    \label{EqUHSooWfgUty}
        \int_{\Omega}f=\int_{\Omega}f^+-\int_{\Omega}f^-
    \end{equation}
    où \( f^+\) et \( f^-\) sont les parties positives et négatives de \( f\); les deux intégrales à droite dans \eqref{EqUHSooWfgUty} sont finies dès que \( f\) est intégrables.
\end{definition}
Nous verrons comment donner un sens à \( \int_{\Omega}f\) dans certains cas où \( f\) n'est pas intégrable sur \( \Omega\) dans la section \ref{SecGAVooBOQddU} sur les intégrales impropres.

Nous définissons aussi
\begin{equation}
    \mu(f)=\int_{\Omega}f
\end{equation}
si \( f\) est une fonction mesurable sur \( \Omega\).

\begin{remark}
    Dans \( \eR^d\), quasiment toutes les fonctions et ensembles sont mesurables. En effet la construction d'ensembles non mesurables demande obligatoirement l'utilisation de l'axiome du choix; de tels ensembles doivent être construits «exprès pour». Il y a très peu de chances pour que vous tombiez sur un ensemble non mesurable de \( \eR^d\) sans que vous ne vous en rendiez compte.
\end{remark}

\begin{remark}
    «Mesurable» ne signifie pas «intégrable». Par exemple la fonction 
    \begin{equation}
        \begin{aligned}
            f\colon \eR&\to \bar\eR \\
            \omega&\mapsto\begin{cases}
            \frac{1}{ \omega }    &   \text{si $ \omega\neq 0$}\\
            \infty    &    \text{$\omega=0$}.
            \end{cases}
        \end{aligned}
    \end{equation}
    est mesurable, mais non intégrable.
\end{remark}

%--------------------------------------------------------------------------------------------------------------------------- 
\subsection{Quelque propriétés}
%---------------------------------------------------------------------------------------------------------------------------

\begin{theorem}[\cite{MesureLebesgueLi}]        \label{ThoooCZCXooVvNcFD}
    Soient \( f,g\) des fonctions étagées positives sur \( (S,\tribF,\mu)\). Alors si \( \alpha\in\mathopen[ 0 , \infty \mathclose]\) nous avons
    \begin{enumerate}
        \item
            \begin{equation}
                \int_S(\alpha f)d\mu=\alpha\int_Sfd\mu.
            \end{equation}
        \item
            \begin{equation}
                \int_S(f+g)d\mu=\int_Sfd\mu+\int_Sgd\mu.
            \end{equation}
        \item\label{ITEMooOJRAooQkoQyD}
    Si \( a_k\in \eR^+\) et si les \( f_k\) sont étagées positives,
    \begin{equation}
        \int_S\left( \sum_{k=1}^na_kf_k \right)=\sum_{k=1}^na_k\left( \int_S f_kd\mu \right).
    \end{equation}
    \end{enumerate}
\end{theorem}

\begin{proof}
    En ce qui concerne le produit par un nombre, tout repose sur le fait que
    \begin{equation}
        (\alpha f)^{-1}(\alpha a_i)=f^{-1}(a_i), 
    \end{equation}
    ce qui fait que si la représentation canonique de \( f\) est \( f=\sum_ia_i\mtu_{A_i}\) alors la représentation canonique de \( \alpha f\) est \( \alpha f=\sum_i(\alpha a_i)\mtu_{A_i}\). Donc
    \begin{equation}
        \int_S\alpha fd\mu=\sum_i\alpha a_i\mu(A_i)=\alpha \sum_ia_i\mu(A_i)=\alpha\int_Sfd\mu.
    \end{equation}
    
    Pour la somme c'est plus lourd. Soient les formes canoniques
    \begin{subequations}
        \begin{align}
            f&=\sum_ia_i\mtu_{A_i}\\
            g&=\sum_jb_j\mtu_{B_i}.
        \end{align}
    \end{subequations}
    Vu que l'union des \( B_j\) est \( S\) nous avons l'union disjointe \( A_i=\bigcup_jA_i\cap B_j\) et donc \( \mu(A_i)=\sum_j\mu(A_i\cap B_j)\). Nous avons donc pour les intégrales :
    \begin{subequations}
        \begin{align}
            \int_Sfd\mu&=\sum_ia_i\sum_j\mu(A_i\cap B_j)\\
            \int_Sgd\mu&=\sum_ib_k\sum_l\mu(B_k\cap A_l).
        \end{align}
    \end{subequations}
    Pour la somme :
    \begin{equation}
        \int_Sfd\mu+\int_Sgd\mu=\sum_{k,l}(a_k+b_l)\mu(A_k\cap B_l).
    \end{equation}

    Nous devons maintenant évaluer \( \int_S(f+g)d\mu\). Pour cela nous remarquons que si \( c\in (f+g)(S)\) (l'ensemble des valeurs atteintes pas \( f+g\)), alors nous notons
    \begin{equation}
        I_c=\{ (k,l)\tq a_k+b_l=c \}
    \end{equation}
    et nous avons
    \begin{equation}
        \{ f+g=c \}=\bigcup_{(k,l)\in I_c}(A_k\cap B_l),
    \end{equation}
    et comme cette union est disjointe, nous pouvons faire la somme des mesures :
    \begin{equation}
        \mu(f+g=c)=\sum_{(k,l)\in I_c}\mu(A_k\cap B_l).
    \end{equation}
    Cela nous permet de faire le calcul suivant :
    \begin{subequations}
        \begin{align}
            \int_S(f+g)d\mu&=\sum_{c\in (f+g)(S)}c\mu(f+g=c)\\
            &=\sum_{c\in(f+g)(S)}c\sum_{(k,l)\in I_c}\mu(A_k\cap B_l)\\
            &=\sum_{c\in(f+g)(S)}\sum_{(k,l)\in I_c} (a_k+b_l) \mu(A_k\cap B_l)
        \end{align}
    \end{subequations}
    Dans cette double somme, tous les couples \( (k,l)\) sont tirés une et une seule fois parce qu'ils sont tous dans un et un seul des \( I_c\), donc
    \begin{subequations}
        \begin{align}
            \int_S(f+g)d\mu&= \sum_{c\in(f+g)(S)}\sum_{(k,l)\in I_c} (a_k+b_l) \mu(A_k\cap B_l)\\
            &=\sum_{(k,l)}(a_k+b_l)\mu(A_k\cap B_l)\\
            &=\int_Sfd\mu+\int_Sgd\mu.
        \end{align}
    \end{subequations}
\end{proof}

\begin{remark}
    Si \( f=\sum_ka_k\mtu_{A_k}\) n'est pas une décomposition canonique, il n'en reste pas moins que chacun des \( \mtu_{A_k}\) est la forme canonique de lui-même. Donc le théorème \ref{ThoooCZCXooVvNcFD} s'applique et nous avons quand même
    \begin{equation}
        \int_Sfd\mu=\sum_ka_k\mu(A_k).
    \end{equation}
\end{remark}

\begin{proposition}
    Si \( f,g\) sont des fonctions étagées et si \( A,B\subset S\) sont disjoints, alors
    \begin{enumerate}
        \item
            \( \int_Afd\mu=\int_Sf\mtu_Ad\mu\)
        \item
            \( \int_{A\cup B}fd\mu=\int_Afd\mu+\int_Bfd\mu\).
    \end{enumerate}
\end{proposition}

\begin{proof}
    Si \( f=\sum_ka_k\mtu_{B_k}\) alors d'une part
    \begin{equation}
        \int_Afd\mu=\sum_ka_k\mu(B_k\cap A)
    \end{equation}
    et d'autre part,
    \begin{equation}
        f\mtu_A)d=\sum_ka_k\mtu_A\mtu_{B_k}=\sum_ka_k\mtu_{B_k\cap A},
    \end{equation}
    ce qui donne
    \begin{equation}
        \int f\mtu_Ad\mu=\sum_ka_k\mu(B_k\cap A).
    \end{equation}

    En ce qui concerne la seconde égalité à prouver, tout repose sur le fait que \( \mtu_{A\cup B=\mtu_A+\mtu_B}\). Du coup nous avons, en utilisant le théorème \ref{ThoooCZCXooVvNcFD} :
    \begin{subequations}
        \begin{align}
            \int_{A\cup B}fd\mu&=\int_Sf\mtu_{A\cup B}d\mu\\
            &=\int_Sf(\mtu_A+\mtu_B)d\mu\\
            &=\int_Sf\mtu_A+\int_Sf\mtu_B\\
            &=\int_Af+\int_Bf.
        \end{align}
    \end{subequations}
\end{proof}

\begin{proposition}     \label{PropOPSCooVpzaBt}
    Si \( A,B\subset \Omega\) sont des ensembles disjoints et si \( f\) est intégrable sur \( A\cup B\) alors
    \begin{equation}
        \int_{A\cup B}f=\int_Af+\int_Bf.
    \end{equation}
\end{proposition}

Le lemme suivant nous aide à détecter des fonctions presque partout nulles.
\begin{lemma}   \label{Lemfobnwt}
    Soit \( f\) une fonction mesurable positive ou nulle telle que
    \begin{equation}
        \int_{\Omega}fd\mu=0.
    \end{equation}
    Alors \( f=0\) \( \mu\)-presque partout.
\end{lemma}

\begin{proof}
    L'ensemble des points \( x\in\Omega\) tels que \( f(x)\neq 0\) peut s'écrire comme une union dénombrable disjointe :
    \begin{equation}
        \{ x\in\Omega\tq f(x)\neq 0 \}=\bigcup_{i=0}^{\infty}E_i
    \end{equation}
    avec
    \begin{subequations}
        \begin{align}
            E_0&=\{ x\in\Omega\tq f(x)>1 \}\\
            E_i&=\{ x\in\Omega\tq \frac{1}{ i+1 }\leq f(x)<\frac{1}{ i } \}.
        \end{align}
    \end{subequations}
    Si un des ensembles \( E_i\) est de mesure non nulle, alors nous pouvons considérer la fonction simple \( h(x)=\frac{1}{ i+1 }\mtu_{E_i}\) dont l'intégrale sur \( \Omega\) est strictement positive. Par conséquent le supremum de la définition \eqref{EqDefintYfdmu} est strictement positif.

    Nous savons donc que \( \mu(E_i)=0\) pour tout \( i\). Étant donné que la mesure d'une union disjointe dénombrable est égale à la somme des mesures, nous avons
    \begin{equation}
        \mu\{ x\in\Omega\tq f(x)\neq 0 \}=0,
    \end{equation}
    ce qui signifie que \( f\) est nulle \( \mu\)-presque partout.
\end{proof}

\begin{corollary}   \label{CorjLYiSm}
    Soit \( f\) une fonction mesurable sur l'espace mesuré \( (\Omega,\tribA,\mu)\) telle que
    \begin{equation}
        \int_{\Omega}f\mtu_{f>0}d\mu=0.
    \end{equation}
    Alors \( f\leq 0\) presque partout.
\end{corollary}

\begin{proof}
    Nous avons l'égalité d'ensembles
    \begin{equation}
        \{ f\mtu_{f>0}\neq 0 \}=\{ \mtu_{f>0}\neq 0 \}.
    \end{equation}
    Mais lemme \ref{Lemfobnwt} implique que \( f\mtu_{f>0}\) est nulle presque partout, c'est à dire que la mesure de l'ensemble du membre de gauche est nulle par conséquent
    \begin{equation}
        \mu\{ \mtu_{f>0}\neq 0 \}=0.
    \end{equation}
    Cela signifie que la fonction \( f\) est presque partout négative ou nulle.
\end{proof}

\begin{lemma}   \label{LemPfHgal}
    Soit \( f\) une fonction telle que \( | f(x)|\leq g(x) \) pour tout \( x\in\Omega\). Si \( g\) est intégrable, alors \( f\) est intégrable.
\end{lemma}

\begin{proof}
    Nous décomposons \( f\) en parties positives et négatives :
    \begin{subequations}
        \begin{align}
            A_+&=\{ x\in\Omega\tq f(x)>0 \}\\
            A_-&=\{ x\in\Omega\tq f(x)<0 \}.
        \end{align}
    \end{subequations}
    Nous posons \( f_+(x)=f(x)\mtu_{A_+}\) et \( f_-(x)=f(x)\mtu_{A_-}\). Nous avons \( f=f_+-f_-\) et
    \begin{equation}
        \int_{\Omega}f=\int_{A_+}f+\int_{A_-}f
    \end{equation}
    parce que \( \Omega=A_+\cup A_-\cup\{ x\in\Omega\tq f(x)=0 \}\). Si \( \varphi\) est une fonction simple qui majore \( f_+\) nous avons
    \begin{equation}
        \varphi(x)=\sum_{k}a_k\mtu_{E_k}(x)\leq f(x)\mtu_{A_+}(x)\leq g(x).
    \end{equation}
    Par conséquent le supremum qui définit \( \int f_+\) est inférieur au supremum qui définit \( \int g\). La fonction \( f_+\) est donc intégrable. La même chose est valable pour la fonction \( f_-\).
\end{proof}

%--------------------------------------------------------------------------------------------------------------------------- 
\subsection{Permuter limite et intégrale}
%---------------------------------------------------------------------------------------------------------------------------

%--------------------------------------------------------------------------------------------------------------------------- 
\subsubsection{Convergence uniforme}
%---------------------------------------------------------------------------------------------------------------------------

\begin{proposition}[Permuter limite et intégrale]       \label{PropbhKnth}
    Soit \( f_n\to f\) uniformément sur un ensemble mesuré \( A\) de mesure finie. Alors si les fonctions \( f_n\) et \( f\) sont intégrables sur \( A\), nous avons
    \begin{equation}
        \lim_{n\to \infty} \int_A f_n=\int_A \lim_{n\to \infty} f_n.
    \end{equation}
\end{proposition}

\begin{proof}
    Notons \( f\) la limite de la suite \( (f_n)\). Pour tout \( n\) nous avons les majorations
    \begin{subequations}
        \begin{align}
            \left| \int_A f_n d\mu-\int_A fd\mu \right| &\leq \int_A| f_n-f |d\mu\\
            &\leq \int_A \| f_n-f \|_{\infty}d\mu\\
            &=\mu(A)\| f_n-f \|_{\infty}
        \end{align}
    \end{subequations}
    où \( \mu(A)\) est la mesure de \( A\). Le résultat découle maintenant du fait que \( \| f_n-f \|_{\infty}\to 0\).
\end{proof}
Il existe un résultat considérablement plus intéressant que cette proposition. En effet, l'intégrabilité de \( f\) n'est pas nécessaire. Cette hypothèse peut être remplacée soit par l'uniforme convergence de la suite (théorème \ref{ThoUnifCvIntRiem}), soit par le fait que les normes des \( f_n\) sont uniformément bornées (théorème de la convergence dominée de Lebesgue \ref{ThoConvDomLebVdhsTf}).

\begin{theorem}[\cite{BJblWiS}]			\label{ThoUnifCvIntRiem}
    La limite uniforme d'une suite de fonctions intégrables sur un borné est intégrable, et on peut permuter la limite et l'intégrale. 
    
    Plus précisément, soit \( A\) un ensemble de \( \mu\)-mesure finie et \( f_n\colon A\to \eR\) des fonctions intégrables sur \( A\). Si la limite \( f_n\to f\) est uniforme, alors \( f\) est intégrable sur \( A\) et nous pouvons inverser la limite et l'intégrale :
    \begin{equation}
        \lim_{n\to \infty} \int_A f_n=\int_A\lim_{n\to \infty} f_n.
    \end{equation}
\end{theorem}

\begin{proof}
    Soit \( \epsilon>0\) et \( n\) tel que \( \| f_n-f \|_{\infty}\leq \epsilon\) (ici la norme uniforme est prise sur \( A\)). Étant donné que \( f_n\) est intégrable sur \( A\), il existe une fonction simple \( \varphi_n\) qui minore \( f_n\) telle que
    \begin{equation}
        \left| \int_{A}\varphi_n-\int_A f_n \right| <\epsilon.
    \end{equation}
    La fonction \( \varphi_n+\epsilon\) est une fonction simple qui majore la fonction \( f\). Si \( \psi\) est une fonction simple qui minore \( f\), alors
    \begin{equation}
        \int_A\psi\leq\int_A\varphi_n+\epsilon\leq\int_A f_n+\epsilon\mu(A).
    \end{equation}
    Par conséquent le supremum qui définit \( \int_A f\) existe, ce qui montre que \( f\) est intégrable. Le fait qu'on puisse inverser la limite et l'intégrale est maintenant une conséquence de la proposition \ref{PropbhKnth}.
\end{proof}

\begin{remark}
    L'hypothèse sur le fait que \( A\) soit de mesure finie est importante. Il n'est pas vrai qu'une suite uniformément convergente de fonctions intégrables est intégrables. En effet nous avons par exemple la suite
    \begin{equation}
        f_n(x)=\begin{cases}
            1/x    &   \text{si \( x<n\)}\\
            0    &    \text{sinon}
        \end{cases}
    \end{equation}
    qui converge uniformément vers \( f(x)=1/x\) sur \( A=\mathopen[ 1 , \infty [\). Le limite n'est cependant guerre intégrable sur \( A\).
\end{remark}

%---------------------------------------------------------------------------------------------------------------------------
\subsubsection{Convergence monotone}
%---------------------------------------------------------------------------------------------------------------------------

\begin{theorem}[Théorème de la convergence monotone ou de Beppo-Levi\cite{mathmecaChoi}] \label{ThoRRDooFUvEAN}
    Soit un espace mesuré \( (\Omega,\tribA,\mu)\) et \( (f_n)\) une suite croissante de fonctions mesurables à valeurs dans \( \mathopen[ 0 , \infty \mathclose]\). Alors la limite ponctuelle \( \lim_{n\to \infty} f_n\) existe, est mesurable et
    \begin{equation}    \label{EqFHqCmLV}
        \lim_{n\to \infty} \int_{\Omega}f_nd\mu= \int_{\Omega}\lim_{n\to \infty} f_nd\mu,
    \end{equation}
    cette intégrable valant éventuellement \( \infty\).
\end{theorem}
\index{théorème!convergence!monotone}
\index{théorème!Beppo-Levi}
\index{permuter!limite et intégrale!convergence monotone}

\begin{proof}
    La limite ponctuelle de la suite est la fonction à valeurs dans \( \mathopen[ 0 , \infty \mathclose]\) donnée par
    \begin{equation}
        f(x)=\lim_{n\to \infty} f_n(x).
    \end{equation}
    Ces limites existent parce que pour chaque \( x\) la suite \( f_n(x)\) est une suite numérique croissante. Nous notons
    \begin{equation}
        I_0=\int_{\Omega}fd\mu.
    \end{equation}
    Nous posons par ailleurs
    \begin{equation}
        I_n=\int_{\Omega}f_n.
    \end{equation}
    Cela est une suite numérique croissante qui a par conséquent une limite que nous notons \( I=\lim_{n\to \infty} I_n\). Notre objectif est de montrer que \( I=I_0\). D'abord par croissance de la suite, pour tous $n$ nous avons \( I_n\leq I_0\), par conséquent \( I\leq I_0\).

    Nous prouvons maintenant l'inégalité dans l'autre sens en nous servant de la définition \eqref{EqDefintYfdmu}. Soit une fonction simple \( h\) telle que \( h\leq f\), et une constante \( 0<C<1\). Nous considérons les ensembles
    \begin{equation}
        E_n=\{ x\in\Omega\tq f_n(x)\geq Ch(x) \}.
    \end{equation}
    Ces ensembles vérifient les propriétés \( E_n\subset E_{n+1}\) et \( \bigcup_{n=1}^{\infty}E_n=\Omega\). Pour chaque \( n\) nous avons les inégalités
    \begin{equation}
        \int_{\Omega}f_n\geq\int_{E_n}f_n\geq C\int_{E_n}h.
    \end{equation}
    Si nous prenons la limite \( n\to\infty\) dans ces inégalités,
    \begin{equation}
        \lim_{n\to \infty} \int_{\Omega}f_n\geq C\lim_{n\to \infty} \int_{E_n}h=C\int_{\Omega}h.
    \end{equation}
    Par conséquent \( \lim_{n\to \infty} \int f_n\geq C\int_{\Omega}h\). Mais étant donné que cette inégalité est valable pour tout \( C\) entre \( 0\) et \( 1\), nous pouvons l'écrire sans le \( C\) :
    \begin{equation}        \label{EqzAKEaU}
        \lim_{n\to \infty} \int_{\Omega}f_n\geq \int_{\Omega}h.
    \end{equation}
    Par définition, l'intégrale de \( f\) est donné par le supremum des intégrales de \( h\) où \( h\) est une fonction simple dominée par \( f\). En prenant le supremum sur \( h\) dans l'équation \eqref{EqzAKEaU} nous avons
    \begin{equation}
        \lim_{n\to \infty} \int_{\Omega}f_n\geq\int_{\Omega}f,
    \end{equation}
    ce qu'il nous fallait.
\end{proof}

\begin{remark}
    La proposition \ref{PropWBavIf} ainsi que le lemme \ref{LemYFoWqmS} montrent qu'une fonction mesurable peut-être écrite comme limite croissante de fonctions simples. Cela permet de démontrer des théorèmes en commençant par prouver sur les fonctions simples et en utilisant Beppo-Levi pour généraliser.
\end{remark}

\begin{remark}
    Une des raisons de demander la positivité des fonctions \( f_n\) est de n'avoir pas d'ambiguïté à parler d'intégrales qui valent \( \infty\). Si par exemple nous prenons \( \Omega=\mathopen[ 0 , 1 \mathclose]\) et que nous considérons
    \begin{equation}
        f_n(x)=\begin{cases}
            0    &   \text{si \( x\leq \frac{1}{ n }\)}\\
            \frac{1}{ x }    &    \text{sinon}.
        \end{cases}
    \end{equation}
    Ce sont des fonctions intégrables, mais la limite étant la fonction \( 1/x\), l'égalité \eqref{EqFHqCmLV} est une égalité entre deux intégrales valant \( \infty\).
\end{remark}

\begin{corollary}[Inversion de somme et intégrales] \label{CorNKXwhdz}
    Si \( (u_n)\) est une suite de fonctions mesurables positives ou nulles, alors
    \begin{equation}
        \sum_{i=0}^{\infty}\int u_i=\int\sum_{i=0}^{\infty}u_i.
    \end{equation}
\end{corollary}
\index{permuter!somme et intégrale}

\begin{proof}
    Nous considérons la suite des sommes partielles de \( (u_n)\) : \( f_n(x)=\sum_{i=0}^nu_n(x)\). Le théorème de la convergence monotone (théorème \ref{ThoRRDooFUvEAN}) implique que
    \begin{equation}
        \lim_{n\to \infty} \int f_n=\int\lim_{n\to \infty} f_n.
    \end{equation}
    Nous remplaçons maintenant \( f_n\) par sa valeur en termes des \( u_i\) et dans le membre de gauche nous permutons l'intégrale avec la somme finie :
    \begin{equation}
        \lim_{n\to \infty} \sum_{i=0}^{\infty}\int u_n=\int\sum_{i=0}^{\infty}u_n,
    \end{equation}
    ce qu'il fallait démontrer.
\end{proof}

\begin{lemma}[Lemme de Fatou]\index{lemme!Fatou}\index{Fatou}   \label{LemFatouUOQqyk}
    Soit \( (\Omega,\tribA,\mu)\) un espace mesuré et \( f_n\colon \Omega\to \mathopen[ 0 , \infty \mathclose]  \) une suite de fonctions mesurables. Alors la fonction \( f(x)=\liminf f_n(x)\) est mesurable et
    \begin{equation}
        \int_{\Omega}\liminf f_nd\mu\leq\liminf\int_{\Omega}fd\mu.
    \end{equation}
\end{lemma}
%TODO : pour la mesurabilité, il faudra citer un théorème du genre de celui fait avec le sup.

\begin{proof}
    Nous posons 
    \begin{equation}
        g_n(x)=\inf_{i\geq n}f_i(x).
    \end{equation}
    Cela est une suite croissance de fonctions positives mesurables telles que, par définition, 
    \begin{equation}
        \lim_{n\to \infty}g_n(x)=\liminf f_n(x).
    \end{equation}
    Nous pouvons y appliquer le théorème de la convergence monotone,
    \begin{equation}
        \lim_{n\to \infty} \int g_n(x)=\int\liminf f_n(x).
    \end{equation}
    Par ailleurs, pour chaque \( i\geq n\) nous avons
    \begin{equation}
        \int g_n\leq \int f_i,
    \end{equation}
    en passant à l'infimum nous avons
    \begin{equation}
        \int g_n\leq \inf_{i\geq n}\int f_i,
    \end{equation}
    et en passant à la limite nous avons
    \begin{equation}
        \int\liminf f_n=\lim_{n\to \infty} \int g_n\leq \lim_{n\to \infty} \inf_{i\geq n}\int f_i=\liminf_{i\to\infty}\inf f_i.
    \end{equation}
\end{proof}

L'inégalité donnée dans ce lemme n'est en général pas une égalité, comme le montre l'exemple suivant :
\begin{equation}
    f_i=\begin{cases}
        \mtu_{\mathopen[ 0 , 1 \mathclose]}    &   \text{si \( i\) est pair}\\
        \mtu_{\mathopen[ 1 , 2 \mathclose]}    &    \text{si \( i\) est impair}.
    \end{cases}
\end{equation}
Nous avons évidemment \( g_n(x)=0\) tandis que \( \int_{\mathopen[ 0 , 2 \mathclose]}f_i=1\) pour tout \( i\).

%---------------------------------------------------------------------------------------------------------------------------
\subsubsection{Convergence dominée de Lebesgue}
%---------------------------------------------------------------------------------------------------------------------------

\begin{theorem}[Convergence dominée de Lebesgue]        \label{ThoConvDomLebVdhsTf}
    Soit \( (f_n)_{n\in\eN}\) une suite de fonctions intégrables sur \( (\Omega,\tribA,\mu)\) à valeurs dans \( \eC\) ou \( \eR\). Nous supposons que  \( f_n\to f\) simplement sur \( \Omega\) presque partout et qu'il existe une fonction intégrable \( g\) telle que
    \begin{equation}
        | f_n(x) |< g(x) 
    \end{equation}
    pour presque\footnote{Si il n'y avait pas le «presque» ici, ce théorème serait à peu près inutilisable en probabilité ou en théorie des espaces \( L^p\), comme dans la démonstration du théorème de Fischer-Riesz \ref{ThoGVmqOro} par exemple.} tout \( x\in\Omega\) et pour tout \( n\in \eN\). Alors
    \begin{enumerate}
        \item
            \( f\) est intégrable,
        \item
           $\lim_{n\to \infty} \int_{\Omega}f_n=\int_\Omega f$,
        \item
            $\lim_{n\to \infty} \int_{\Omega}| f_n-f |=0$.
    \end{enumerate}
\end{theorem}
\index{théorème!convergence!dominée de Lebesgue}
\index{dominée!convergence (Lebesgue)}
\index{permuter!limite et intégrale!convergence dominée}

\begin{proof}

    La fonction limite \( f\) est intégrable parce que \( | f |\leq g\) et \( g\) est intégrable (lemme \ref{LemPfHgal}). Par hypothèse nous avons
    \begin{equation}
        -g(x)\leq f_n(x)\leq g(x).
    \end{equation}
    En particulier la fonction \( g_n=f_n+g\) est positive et mesurable si bien que le lemme de Fatou (lemme \ref{LemFatouUOQqyk}) implique
    \begin{equation}
        \int_{\Omega}\liminf g_n\leq\liminf\int_{\Omega}g_n.
    \end{equation}
    Évidement nous avons \( \liminf g_n=f+g\), de telle sorte que
    \begin{equation}
        \int f+\int g\leq \liminf\int g_n=\liminf\int f_n+\int g,
    \end{equation}
    et le nombre \( \int g\) étant fini, nous pouvons le retrancher des deux côtés de l'inégalité :
    \begin{equation}
        \int f\leq\liminf\int f_n.
    \end{equation}
    Afin d'obtenir une minoration de \( \int f\) nous refaisons exactement le même raisonnement en utilisant la suite de fonctions \( k_n=-f_n\to k=-f\). Nous obtenons que
    \begin{equation}
        \int k\geq\liminf\int k_n=-\limsup\int f_n,
    \end{equation}
    et par conséquent
    \begin{equation}    \label{IneqsndMYTO}
        \liminf\int f_n\leq\int f\leq\limsup\int f_n.
    \end{equation}
    La limite supérieure étant plus grande ou égale à la limite inférieure, les trois quantités dans les inégalités \eqref{IneqsndMYTO} sont égales.

    Nous prouvons maintenant le troisième point. Soit la suite de fonctions
    \begin{equation}
        h_n(x)=| f_n(x)-f(x) |
    \end{equation}
    qui tend ponctuellement vers zéro. De plus
    \begin{equation}
    h_n(x)\leq | f_n(x) |+| f(x) |\leq 2g(x),
    \end{equation}
    ce qui prouve que les \( h_n\) majorés par une fonction intégrable. Donc
    \begin{equation}
        \lim_{n\to \infty} \int_{\Omega}| f_n-f |= \lim_{n\to \infty} \int_{\Omega}h_n(x)dx=\int_{\Omega}\lim_{n\to \infty} | f_n(x)-f(x) |=0
    \end{equation}
\end{proof}

\begin{remark}
    Lorsque nous travaillons sur des problèmes de probabilités, la fonction \( g\) peut être une constante parce que les constantes sont intégrables sur un espace de probabilité.
\end{remark}

\begin{corollary}       \label{CorCvAbsNormwEZdRc}
    Soit \( (a_i)_{i\in \eN}\) une suite numérique absolument convergente. Alors elle est convergente. Il en est de même pour les séries de fonctions si on considère la convergence ponctuelle.
\end{corollary}

\begin{proof}
    L'hypothèse est la convergence de l'intégrale \( \int_{\eN}| a_i |dm(i)\) où \( dm\) est la mesure de comptage. Étant donné que \( | a_i |\leq | a_i |\), la fonction \( a_i\) (fonction de \( i\)) peut jouer le rôle de \( g\) dans le théorème de la convergence dominée de Lebesgue (théorème \ref{ThoConvDomLebVdhsTf}).
\end{proof}
Nous utiliserons ce résultat pour montrer que la transformée de Fourier d'une fonction \( L^1(\eR^d)\) est continue (proposition \ref{PropJvNfj}).

\begin{proposition}[\cite{YHRSDGc}] \label{PropUXjnwLf}
    Approximation de fonctions mesurables par des fonctions étagées.
    \begin{enumerate}
        \item
            Une fonction mesurable et positive est limite (simple) d'une suite croissante de fonctions étagées, mesurables et positives.
        \item
            Si \( f\colon \eR^d\to \bar \eR\) est mesurable, alors elle est limite (simple) de fonctions étagées \( f_n\) telles que \( | f_n |\leq | f |\).
    \end{enumerate}
\end{proposition}
%TODO : la preuve est dans le document cité.

%--------------------------------------------------------------------------------------------------------------------------- 
\subsection{Produit d'une mesure par une fonction (mesure à densité)}
%---------------------------------------------------------------------------------------------------------------------------

\begin{propositionDef}[\cite{MonCerveau,ooGMNAooSLnIio}]\label{PropooVXPMooGSkyBo}
    Soit un espace mesuré \( (S,\tribF,\mu)\) et une fonction mesurable positive \( w\colon S\to \bar\eR^+\). Alors la formule
    \begin{equation}
        (w\cdot \mu)(A)=\int_Awd\mu
    \end{equation}
    pour tout \( A\in \tribF\) définit une mesure positive sir \( (S,\tribF)\) appelée \defe{produit}{produit!d'une mesure par une fonction} de la mesure \( \mu\) par la fonction \( w\). La fonction \( w\) est la \defe{densité}{densité!mesure} de la mesure \( w\cdot \mu\) par rapport à la mesure \( \mu\).
\end{propositionDef}

\begin{proof}
    D'abord \( (w\cdot \mu)(\emptyset)=0\) parce que 
    \begin{equation}
        (w\cdot \mu)\emptyset=\int_Sw\mtu_{\emptyset}d\mu=\int_S0d\mu=0\times \mu(S)=0
    \end{equation}
    où nous avons (éventuellement) utilisé deux fois la convention \( 0\times \infty=0\).


    Ensuite si les ensembles \( A_i\) sont des éléments deux à deux disjoints de \( \tribF\) alors nous avons \( \mtu_{\bigcup_{i=1}^{\infty}}=\sum_{i=1}^{\infty}\mtu_{A_i}\), et donc
    \begin{equation}
        (w\cdot \mu)(\bigcup_iA_i)=\int_Sw\mtu_{\bigcup_iA_i}d\mu=\int_S\big( \sum_{i=1}^{\infty}w\mtu_{A_i} \big)d\mu=\sum_i\int_Sw\mtu_{A_i}d\mu=\sum_i(w\cdot\mu)(A_i).
    \end{equation}
    Dans ce calcul nous avons utilisé le fait que \( f\) était positive pour justifier l'application du théorème de la convergence monotone \ref{ThoRRDooFUvEAN}.
\end{proof}

\begin{proposition}[\cite{ooGMNAooSLnIio}]  \label{PropooJMWAooDzfpmB}
    Soit une fonction mesurable \( w\colon (S,\tribF,\mu)\to \bar \eR^+\).
    \begin{enumerate}
        \item
            Si $f\colon S\to \bar\eR^+$ est mesurable, alors \( f\cdot(w\cdot \mu)=(fg)\cdot \mu\).
        \item
            Si \( f\colon S\to \bar \eR\) ou \( \eC\) est mesurable, elle est \( w\cdot\mu\)-intégrable si et seulement si \( fw\) est \( \mu\)-intégrable. Dans ce cas, nous avons encore \( f\cdot(w\cdot \mu)=(fg)\cdot\mu\).
    \end{enumerate}
    Attention : dans le cas où \( f\) est à valeurs dans \( \eC\), alors il faut que \( w\) soit à valeurs finies dans \( \eR\) parce que nous n'avons pas définit \( \infty\times z\) lorsque \( z\in \eC\).
\end{proposition}

\begin{proof}
    Nous commençons par prouver le résultat pour la fonction caractéristique de l'ensemble mesurable \( A\). Nous avons : $\mtu_A\cdot(w\cdot \mu)(B)=\int_B\mtu_Ad(w\cdot \mu)$. Mais par définition, l'intégrale d'une fonction indicatrice est la mesure de l'ensemble indiqué. En passant sur le fait que \( \mtu_A\mtu_B=\mtu_{A\cap B}\), 
    \begin{equation}
        \int_B\mtu_Ad(w\cdot \mu)=   (w\cdot\mu)(A\cap B)=\int_S\mtu_{A\cap B}wd\mu=\int_S\mtu_A\mtu_Bwd\mu=\int_B\mtu_Awd\mu=(\mtu_Aw)\cdot\mu(B).
    \end{equation}

    Supposons maintenant que \( f\) soit une fonction étagées qui s'écrit \( f=\sum_ka_k\mtu_{A_k}\) où les \( A_k\) sont des ensembles mesurables disjoints. Alors le calcul est le suivant, en utilisant le fait que sur \( A_k\), on a \( a_k=f(x)\) :
    \begin{subequations}
        \begin{align}
            f\cdot(g\cdot \mu)B&=\int_Bfd(g\cdot \mu)\\
            &=\sum_ka_k(g\cdot\mu)(A_k\cap B)\\
            &=\sum_ka_k\int_{A_k\cap B}gf\mu\\
            &=\int_{A_k\cap}f(x)g(x)d\mu(x)\\
            &=\sum_k(fg\cdot\mu)(A_k\cap B)\\
            &=(fg\cdot\mu)(B)
        \end{align}
    \end{subequations}
    parce que les \( A_k\cap B\) forment une partition de l'ensemble \( B\) (voir le point \ref{ItemQFjtOjXiii} de la définition \ref{DefBTsgznn}).

    Si \( f\colon S\to \bar\eR^+\) est mesurable, le théorème \ref{THOooXHIVooKUddLi} donne une suite croissante \( f_n\) de fonctions étagées positives convergeant (ponctuellement) vers \( f\). Vu que la fonction \( w\) est positive, nous avons aussi la limite positive et croissante \( wf_n\to wf\). Ainsi l'utilisation du théorème de la convergence monotone est justifié dans le calcul suivant :
    \begin{equation}
        \int_Sfd(w\cdot \mu)=\lim_{n\to \infty} \int_Sf_nd(w\cdot\mu)=\lim_{n\to \infty} \int_S(wf_n)d\mu=\int_Swfd\mu.
    \end{equation}
    
    Nous passons maintenant au cas général où \( f\) est une fonction à valeurs dans \( \bar\eR\) ou \( \eC\) (avec \( w\) finie dans ce dernier cas). Nous avons la chaîne d'équivalences 
    %\begin{itemize}{$\Leftrightarrow$}
    \begin{itemize}
            \renewcommand{\labelitemi}{$\Leftrightarrow$}
        \item \( f\) est \( (w\cdot\mu)\) intégrable
        \item \( | f |\) est \( (w\cdot\mu)\)-intégrable
        \item \( | f |w\) est \( \mu\)-intégrable
        \item \( | fw |\) est \( \mu\)-intégrable.
    \end{itemize}

    Si cela est le cas, la formule se démontre en se ramenant au cas déjà prouvé des fonctions positives en utilisant les \( (fw)^+=f^+w\), \( (fw)^-=f^-w\) etc.
\end{proof}

%+++++++++++++++++++++++++++++++++++++++++++++++++++++++++++++++++++++++++++++++++++++++++++++++++++++++++++++++++++++++++++ 
\section{Tribu produit, mesure produit}
%+++++++++++++++++++++++++++++++++++++++++++++++++++++++++++++++++++++++++++++++++++++++++++++++++++++++++++++++++++++++++++

%--------------------------------------------------------------------------------------------------------------------------- 
\subsection{Produit d'espaces mesurables}
%---------------------------------------------------------------------------------------------------------------------------

\begin{definition}      \label{DefTribProfGfYTuR}
    Si \( \tribA_1\) et \( \tribA_2\) sont deux tribus sur deux ensembles \( \Omega_1\) et \( \Omega_2\), nous définissons la \defe{tribu produit}{tribu!produit} \( \tribA_1\otimes\tribA_2\) comme étant la tribu engendrée par 
    \begin{equation}
        \{ X\times Y\tq X\in\tribA_1,Y\in\tribA_2 \}.
    \end{equation}
    Ces ensembles sont appelés \defe{rectangles}{rectangle!produit de tribus} de \( (\Omega_1,\tribA_1)\otimes (\Omega_2,\tribA_2)\).
\end{definition}

\begin{proposition}[\cite{KEQWooJsCGiw}]        \label{PropLJJWooKqWlTr}
    Soient deux espaces mesurables \( (S_1,\tribF_1)\) et \( (S_2,\tribF_2)\). Si \( \tribC_i\) est une classe de parties de \( S_i\) avec \( \tribF_i=\sigma(\tribC_i)\) et \( S_i\in\tribC_i\). Alors 
    \begin{equation}
        \tribF_1\otimes \tribF_2=\sigma(\tribC_1\times \tribC_2).
    \end{equation}
\end{proposition}

\begin{proof}
    Nous notons \( p_1\) et \( p_2\) les projections de \( S_1\times S_2\) vers \( S_1\) et \( S_2\). Nous commençons par prouver que
    \begin{equation}    \label{eqSGPBooLpQHfq}
        \tribF_1\otimes \tribF_2=\sigma\big( \p_1^{-1}(\tribF_1)\cup p_2^{-1}(\tribF_2) \big).
    \end{equation}
    En effet cette union est dans \( \tribF_1\otimes \tribF_2\) parce que ce sont tous des produits de la forme \( A_1\times S_2\) et \( S_1\times A_2\) où \( A_i\in \tribF_i\). Inversement, tous les produits de la forme \( A_1\times A_2\) sont dans la tribu engendrée par l'union parce que
    \begin{equation}
        A_1\cup A_2=(A_1\times S_2)\cap(S_1\times A_2).
    \end{equation}
    Par conséquent, la partie \( p_1^{-1}(\tribF_1)\cup p_2^{-1}(\tribF_2)\) engendre tous les produits qui \href{ https://fr.wikisource.org/wiki/Bible_Crampon_1923/Matthieu }{ engendrent } la tribu \( \tribF_1\otimes\tribF_2\). L'égalité \eqref{eqSGPBooLpQHfq} est donc correcte.
    
    Si \( C_1\in\tribC_1\) alors
    \begin{equation}
        p_1^{-1}(C_1)=C_1\times S_2\in\tribC_1\times \tribC_2
    \end{equation}
    et donc \( p_1^{-1}(\tribC_1)\subset \tribC_1\times \tribC_2\). En utilisant le lemme de transfert \ref{LemOQTBooWGYuDU} nous avons alors
    \begin{equation}        \label{EqDQLYooVOLqMZ}
        p_1^{-1}(\tribF_1)=p_1^{-1}\big( \sigma(\tribC_1) \big)=\sigma\big( p_1^{-1}\tribC_1 \big)\subset\sigma(\tribC_1\times \tribC_1)
    \end{equation}
    et au bout de la même façon,
    \begin{equation}        \label{EqMTRCooVHNTHJ}
        p_2^{-1}(\tribF_1)\subset\sigma(\tribC_1\times \tribC_2).
    \end{equation}

    Vu les relations \eqref{EqDQLYooVOLqMZ}, \eqref{EqMTRCooVHNTHJ} et \eqref{eqSGPBooLpQHfq} nous avons
    \begin{equation}
        \tribF_1\otimes\tribF_2=\sigma\big( \p_1^{-1}(\tribF_1)\cup p_2^{-1}(\tribF_2) \big)\subset\sigma(\tribC_1\times \tribC_2).
    \end{equation}

    Réciproquement, si \( C_1\in \tribC_1\) et \( C_2\in \tribC_2\) alors
    \begin{equation}
        C_1\times C_2=(C_1\times S_1)\cap(S_1\times C_2)=p_1^{-1}(C_1)\cap p_2^{-1}(C_2)\in\tribF_1\otimes\tribF_2.
    \end{equation}
\end{proof}

%--------------------------------------------------------------------------------------------------------------------------- 
\subsection{Le cas des boréliens}
%---------------------------------------------------------------------------------------------------------------------------

Si \( X_1\) et  \( X_2\) sont des espaces topologiques et si nous notons \( \mO_i\) l'ensemble de leurs ouverts, par définition \( \Borelien(X_i)=\sigma(\mO_i)\). De plus par la proposition \ref{PropLJJWooKqWlTr} nous savons que
\begin{equation}        \label{EqOHMSooRSLrDk}
    \sigma(\mO_1\times \mO_2)=\Borelien(X_1)\otimes \Borelien(X_2).
\end{equation}

\begin{lemma}       \label{LemDEDQooJyzXgC}
    Si \( (X_i,\mO_i)\) sont des espaces topologiques, alors
    \begin{equation}
        \Borelien(X_1)\otimes \Borelien(X_2)\subset \Borelien(X_1\times X_2)
    \end{equation}
\end{lemma}

\begin{proof}
    Si \( A_i\in \mO_i\) alors \( A_1\times A_2\) est un ouvert de \( X_1\times X_2\) (voir la définition \ref{DefIINHooAAjTdY}). Par conséquent, \( \mO_1\times \mO_2\) est contenu dans l'ensemble des ouverts de \( X_1\times X_2\) ou encore
    \begin{equation}
        \mO_1\times \mO_2\subset\Borelien(X_1\times X_2),
    \end{equation}
    et donc
    \begin{equation}
        \sigma(\mO_1\times \mO_2)\subset\sigma\big( \Borelien(X_1\times X_2) \big)
    \end{equation}
    finalement, par \eqref{EqOHMSooRSLrDk}
    \begin{equation}
        \Borelien(X_1)\otimes\Borelien(X_2)\subset\Borelien(X_1\times X_2).
    \end{equation}
\end{proof}

Il n'y a en général pas égalité, mais nous allons immédiatement voir que dans (presque) tous les cas raisonnables, les boréliens sur un produit sont le produit des boréliens.

\begin{proposition}[\cite{KEQWooJsCGiw}]        \label{PropNAAJooBPbjkX}
    Soient \( (X_1,d_1)\) et \( (X_2,d_2)\) des espace métriques séparables. Alors
    \begin{equation}
        \Borelien(X_1\times X_2)=\Borelien(X_1)\otimes \Borelien(X_2).
    \end{equation}
\end{proposition}

\begin{proof}
    Nous savons par le lemme \ref{LemDUJXooWsnmpL} que tout ouvert de \( X_1\times X_2\) est une réunion dénombrable d'éléments de \( \mO_1\times\mO_2\). Donc tout ouvert de \( X_1\times X_2\) est dans \( \Borelien(X_1)\otimes \Borelien(X_2)\). Par conséquent
    \begin{equation}
        \Borelien(X_1\times X_2)\subset \Borelien(X_1)\otimes \Borelien(X_2).
    \end{equation}
    L'inclusion inverse étant déjà acquise par le lemme \ref{LemDEDQooJyzXgC}, nous avons l'égalité.
\end{proof}

\begin{proposition}     \label{CorWOOOooHcoEEF}
    Les boréliens sur \( \eR^N\) sont ceux qu'on croit.
    \begin{enumerate}
        \item
            \( \Borelien(\eR^2)=\Borelien(\eR)\otimes \Borelien(\eR)\)
        \item
            \( \Borelien(\eR^{N+1})=\Borelien(\eR^N)\otimes \Borelien(\eR)\)
    \end{enumerate}
\end{proposition}

\begin{proof}
    Cela n'est rien d'autre que la proposition \ref{PropNAAJooBPbjkX}.
\end{proof}

\begin{proposition}
    Soit un espace mesurable \( (S,\tribF)\) et des applications \( f_k\colon S\to \eR\) (\( k=1,\ldots, N\)). Alors l'application
    \begin{equation}
        \begin{aligned}
            f\colon (S,\tribF)&\to (\eR^n,\Borelien(\eR^N)) \\
            x&\mapsto \big( f_1(x),\ldots, f_N(x) \big) 
        \end{aligned}
    \end{equation}
    est mesurable si et seulement si chacun des \( f_i\) est mesurable.
\end{proposition}

\begin{proof}
    Division en deux.
    \begin{subproof}
    \item[Condition nécessaire]
        Nous supposons que les \( f_i\) sont mesurables. Nous avons
        \begin{subequations}
            \begin{align}
            f^{-1}\big( \prod_{k=1}^N\mathopen] a_k , b_k \mathclose[ \big)&=\{ x\in S\tq f_1(x)\in\mathopen] a_1 , b_1 \mathclose[ ,\cdots f_N(x)\in\mathopen] a_N , b_N \mathclose[\}\\
            &=\bigcap_{k=1}^Nf_k^{-1}\big( \mathopen] a_k , b_k \mathclose[ \big).
            \end{align}
        \end{subequations}
        Cela est une intersection finie d'éléments de \( \tribF\) et est donc un élément de \( \tribF\). Mais les pavés ouverts engendrent \( \Borelien(\eR^N)\) parce qu'ils sont une base dénombrable de la topologie (proposition \ref{PROPooYEkvbWBz}). Le théorème \ref{ThoECVAooDUxZrE} nous assure alors que \( f\) est mesurable parce que l'image inverse d'une base de la tribu est mesurable.
    \item[Condition suffisante]
        Si \( f\) est mesurable alors en particulier
        \begin{equation}
            f_k^{-1}\big( \mathopen] a , b \mathclose[ \big)=f^{-1}\big( \eR\times\ldots\times \mathopen] a , b \mathclose[\times \eR\times\ldots\times \eR \big)\in\tribF.
        \end{equation}
        Pour cela nous avons utilisé la proposition \ref{CorWOOOooHcoEEF} qui nous indique que le produit dans la parenthèse est un borélien de \( \eR^N\) en tant que produit de boréliens de \( \eR\).

        Encore une fois \( f_k^{-1}\) tombe dans \( \tribF\) pour une base dénombrable de la topologie de \( \eR\) et est donc mesurable.
    \end{subproof}
\end{proof}

%--------------------------------------------------------------------------------------------------------------------------- 
\subsection{Produit de mesures}
%---------------------------------------------------------------------------------------------------------------------------

\begin{lemma}[Propriété des sections\cite{NBoIEXO}] \label{LemAQmWEmN}
    Soient \( \tribA_1\) et \( \tribA_2\) des tribus sur les ensembles \( \Omega_1\) et \( \Omega_2\). Si \( A\in\tribA_1\otimes\tribA_2\) alors pour tout \( x\in \Omega_1\) et \( y\in\Omega_2\), les ensembles
    \begin{subequations}    \label{subEqCTtPccK}
        \begin{align}
            A_1(y)=\{ x\in\Omega_1\tq (x,y)\in A \}\\
            A_2(x)=\{ y\in\Omega_2\tq (x,y)\in A \}
        \end{align}
    \end{subequations}
    sont mesurables.
\end{lemma}
\index{section!propriété des}

\begin{proof}
    Soit \( y\in\Omega_2\); nous allons prouver le résultat pour \( A_1(y)\). Pour cela nous notons 
    \begin{equation}
        S=\{ A\in \tribA_1\otimes\tribA_2\tq \forall y\in\Omega_2, A_1(y)\in\tribA_1 \},
    \end{equation}
    et nous allons noter que \( S\) est une tribu contenant les rectangles. Par conséquent, \( S\) sera égal à \( \tribA_1\otimes \tribA_2\).

    \begin{subproof}
        \item[Les rectangles]

            Considérons le rectangle \( A=X\times Y\) et si \( y\in \Omega_2\) alors
            \begin{equation}
                A_1(y)=\{ x\in \Omega_1\tq (x,y)\in X\times Y \}.  
            \end{equation}
            Donc soit \( y\in Y\) alors \( A_1(y)=X\in\tribA_1\), soit \( y\notin Y\) et alors \( A_1(y)=\emptyset\in\tribA_1\).

        \item[Tribu : ensemble complet]

            Nous avons \( \Omega_1\times \Omega_2\in S\) parce que c'est un rectangle.

        \item[Tribu : complémentaire] Soit \( A\in S\) et montrons que \( A^c\in S\). Nous avons d'abord
            \begin{equation}
                (A^c)_1(y)=\{ x\in \Omega_1\tq (x,y)\in A^c \}.
            \end{equation}
            D'autre part
            \begin{equation}
                A_1(y)^c=\{ x\in\Omega_1\tq (x,y)\notin A \}=\{ x\in \Omega_1\tq (x,y)\in A^c \}=(A^c)_1(y).
            \end{equation}
            Vu que \( \tribA_1\) est une tribu et que par hypothèse \( A_1(y)\in\tribA_1\), nous avons aussi \( A_1(y)^c\in S\), et donc \( (A^c)_1(y)\in \tribA_1\), ce qui prouve que \( A^c\in S\).

        \item[Tribu : union dénombrable] Soit une suite \( A_n\in S\). Nous avons
            \begin{equation}
                (\bigcup_nA_n)_1(y)=\{ x\in\Omega_1\tq (x,y)\in \bigcup_nA_n \}=\bigcup_n\{ x\in\Omega_1\tq (x,y)\in A_n \}=\bigcup_n(A_n)_1(y),
            \end{equation}
            et ce dernier ensemble est dans \( \tribA_1\) parce que c'est une union dénombrable d'éléments de \( \tribA_1\).
        
    \end{subproof}
    Nous avons donc prouvé que \( S\) est une tribu contenant les rectangles, donc \( S\) contient au moins \( \tribA_1\otimes \tribA_2\).
\end{proof}

\begin{corollary}
    Si \( f\colon \Omega_1\times \Omega_2\to \eR\) est une fonction mesurable\footnote{Définition \ref{DefQKjDSeC}.} sur \( X\times Y\) alors pour chaque \( y\) dans \( \Omega_2\), la fonction
    \begin{equation}
        \begin{aligned}
            f_y\colon X&\to \eR \\
            x&\mapsto f(x,y) 
        \end{aligned}
    \end{equation}
    est mesurable.
\end{corollary}

\begin{proof}
    Soit \( \mO\) un ensemble mesurable de \( \eR\) (i.e. un borélien), et \( y\in \Omega_2\). Nous avons
    \begin{equation}
        f_y^{-1}(\mO)=\{ x\in X\tq f(x,y)\in \mO \}=A_1(y)
    \end{equation}
    où
    \begin{equation}
        A=\{ (x,y)\in \Omega_1\times \Omega_2\tq f(x,y)\in \mO \}=f^{-1}(\mO).
    \end{equation}
    Ce dernier est mesurable parce que \( f\) l'est.
\end{proof}

\begin{theorem}[\cite{NBoIEXO}\footnote{Modèle non contractuel : des notations et la définition de \( \lambda\)-système peuvent varier entre la référence et le présent texte.}]    \label{ThoCCIsLhO}
    Soient \( (\Omega_i,\tribA_i,\mu_i)\) (\( i=1,2\)) deux espaces mesurés \( \sigma\)-finie. Soit \( A\in\tribA_1\otimes \tribA_2\). Alors les fonctions\footnote{Voir la notation du lemme \ref{subEqCTtPccK}.}
    \begin{subequations}
        \begin{align}
            x\mapsto\mu_2\big( A_2(x) \big)\\
            y\mapsto\mu_1\big( A_1(y) \big)
        \end{align}
    \end{subequations}
    sont mesurables et
    \begin{equation}    \label{EqRKXwsQJ}
        \int_{\Omega_1}\mu_2\big( A_2(x) \big)d\mu_1(x)=\int_{\Omega_2}\mu_2\big( A_1(y) \big)d\mu_2(y).
    \end{equation}
\end{theorem}

\begin{proof}
    Nous supposons d'abord que \( \mu_1\) et \( \mu_2\) sont finies et nous notons \( \tribD\) le sous-ensemble de \( \tribA_1\otimes \tribA_2\) sur lequel le théorème est correct. Nous allons commencer par prouver que \( \tribD\) est un \( \lambda\)-système.

    \begin{subproof}
        \item[\( \lambda\)-système : différence ensembliste]
            Soient \( A,B\in\tribD\) avec \( A\subset B\). Nous avons
            \begin{subequations}
                \begin{align}
                    (B\setminus A)_1(y)&=\{ x\in \Omega_1\tq(x,y)\in B\setminus A \}\\
                    &=\{ x\in \Omega_1\tq(x,y)\in B\}\setminus\{ x\in \Omega_1\tq(x,y)\in  A \}\\
                    &=B_1(y)\setminus A_1(y).
                \end{align}
            \end{subequations}
            Vu que \( A_1(y)\subset B_1(y)\) et que les mesure sont finies le lemme \ref{LemPMprYuC} nous donne
            \begin{equation}
                \mu_1\big( (B\setminus A)_1(y) \big)=\mu_1\big( B_1(y) \big)-\mu_1\big( A_1(y) \big),
            \end{equation}
            et similairement pour \( 1\leftrightarrow 2\). Les deux fonctions (de \( y\)) à droite étant mesurables, nous avons la mesurabilité de la fonction \( y\mapsto \mu_1\big( (B\setminus A)_1(y) \big)\).

            Prouvons la formule intégrale en nous rappelant que la formule \eqref{EqRKXwsQJ} est supposée correcte pour \( A\) et \( B\) séparément :
            \begin{subequations}
                \begin{align}
                    \int_{\Omega_2}\mu_1\big( (B\setminus A)_1(y) \big)d\mu_2(y)&=\int_{\Omega_2}\mu_1\big( B_1(y) \big)d\mu_2(y)-\int_{\Omega_2}\mu_1\big( A_1(y) \big)d\mu_2(y)\\
                    &=\int_{\Omega_1}\mu_2\big( B_2(x) \big)d\mu_1(x)-\int_{\Omega_1}\mu_2\big( A_2(x) \big)d\mu_1(x)\\
                    &=\int_{\Omega_1}\mu_2\big( (B\setminus A)_2(x) \big)d\mu_1(x).
                \end{align}
            \end{subequations}
            
    
        \item[\( \lambda\)-système : limite de suite croissante]

            Soit \( (A_n)\) une suite croissante dans \( \tribD\); nous posons \( B_n=A_n\setminus A_{n-1}\) et \( A_0=\emptyset\) de telle sorte à travailler avec une suite d'ensembles disjoints qui satisfait \( \bigcup_nA_n=\bigcup_nB_n\). Vu que la suite est croissante nous avons \( A_{n-1}\subset A_n\) et donc \( B_n\in\tribD\) par le point déjà fait sur la différence ensembliste. Nous avons :
            \begin{subequations}
                \begin{align}
                    \mu_1\big( (\bigcup_nB_n)_1(y) \big)&=\{ x\in \Omega_1\tq (x,y)\in\bigcup_nB_n \}\\
                    &=\bigcup_n\{ x\in\Omega_1\tq (x,y)\in B_n \}\\
                    &=\bigcup_n (B_n)_1(y).
                \end{align}
            \end{subequations}
            Par conséquent, par la propriété \ref{ItemQFjtOjXiii} d'une mesure nous avons
            \begin{equation}
                \mu_1\big( (\bigcup_nB_n)_1(y) \big)=\sum_n\mu_1\big( (B_n)_1(y) \big).
            \end{equation}
            En tant que somme de fonctions positives et mesurables, la fonction
            \begin{equation}
                y\mapsto\sum_n\mu_1\big( (B_n)_1(y) \big)
            \end{equation}
            est mesurable par la proposition \ref{PropFYPEOIJ}. Il faut encore vérifier la formule intégrale. Le gros du boulot est de permuter une somme et une intégrale par le corollaire \ref{CorNKXwhdz} :
            \begin{subequations}
                \begin{align}
                    \int_{\Omega_2}\sum_n\mu_1\big( (B_n)_1(y) \big)d\mu_2(y)&=\sum_n\int_{\Omega_2}\mu_1\big( (B_n)_1(y) \big)d\mu_2(y)\\
                    &=\sum_n\int_{\Omega_1}\mu_2\big( (B_n)_2(x) \big)d\mu_1(x)\\
                    &=\int_{\Omega_1}\sum_n\mu_2\big( (B_n)_2(x) \big)d\mu_1(x)\\
                    &=\int_{\Omega_1}\mu_2\big( (\bigcup_nB_n)_1(y) \big)d\mu_1(x).
                \end{align}
            \end{subequations}
    \end{subproof}
    Maintenant que \( \tribD\) est un $\lambda$-système contenant les rectangles, le lemme \ref{LemLUmopaZ} dit que la tribu engendrée par \( \tribD\) (c'est à dire \( \tribA_1\otimes \tribA_2\)) est le $\lambda$-système \( \tribD\) lui-même.

    La preuve est finie dans le cas de mesures finies. Nous commençons maintenant à prouver dans le cas où les mesures \( \mu_1\) et \( \mu_2\) sont seulement \( \sigma\)-finies. Nous considérons des suites croissantes \( \Omega_{i,n}\to\Omega_i\) d'ensembles mesurables et de mesure finie : \( \mu_i(\Omega_{i,n})<\infty\). D'abord remarquons que
    \begin{equation}\label{EqNFuBzBF}
        \mu_2\Big( (A\cap \Omega_{1,j}\times E_{2,j})_2(x) \Big)=\mu_2\Big( A_2(x)\cap \Omega_{2,j} \Big)\mtu_{\Omega_{1,j}}.
    \end{equation}
    En effet,
    \begin{subequations}
        \begin{align}
            \heartsuit&=(A\cap\Omega_{1,j}\times E_{2,j})_2(x)\\
            &=\{ y\in\Omega_2\tq (x,y)\in A\cap \Omega_{1,j}\times E_{2,j} \}\\
            &=\{ y\in \Omega_2\tq (x,y)\in A\times \Omega_{2,j} \}\cap\{ y\in\Omega_2\tq (x,y)\in \Omega_{1,j}\times \Omega_{2,j} \}.
        \end{align}
    \end{subequations}
    Si \( y\in \Omega_{1,j}\) alors \( \{ y\in \Omega_2\tq (x,y)\in \Omega_{1,j}\times \Omega_{2,j} \}=\Omega_{2,j}\) et dans ce cas
    \begin{equation}
        \heartsuit=\{ y\in \Omega_2\tq (x,y)\in A\times \Omega_{2,j} \}\cap \Omega_{2,j}=A_2(x)\cap E_{2,j}.
    \end{equation}
    Et inversement, si \( x\notin \Omega_{1,j}\) alors \( \heartsuit=\emptyset\). Dans les deux cas nous avons \eqref{EqNFuBzBF}.

    Les ensembles \( A\cap \Omega_{1,j}\times \Omega_{2,j}\) étant de mesure finie, nous pouvons leur appliquer la première partie :
    \begin{equation}
        \int_{\Omega_1}\mu_2\Big( (A\cap\Omega_{1,j}\times \Omega_{2,j})_2(x) \Big)d\mu_1(x)=\int_{\Omega_2}\mu_1\Big( (A\cap\Omega_{1,j}\times \Omega_{2,j})_1(y) \Big)d\mu_2(u),
    \end{equation}
    ou encore
    \begin{equation}
        \int_{\Omega_1}\mu_2\Big( A_2(x)\cap \Omega_{2,j} \Big)\mtu_{\Omega_{1,j}}(x)d\mu_1(x)=\int_{\Omega_2}\mu_1\Big( A_1(y)\cap \Omega_{1,j} \Big)\mtu_{\Omega_{2,j}}(y)d\mu_2(y).
    \end{equation}
    Ce que nous avons dans ces intégrales sont (par rapport à \( j\)) des suites croissantes de fonction positives; nous pouvons donc permuter une limite et une intégrale. En sachant que si \( k\to \infty\), alors
    \begin{subequations}
        \begin{align}
            \mtu_{1,j}(x)\to 1\\
            \mu_2\big( A_2(x)\cap \Omega_2,j \big)\to\mu_2\big( A_2(x) \big),
        \end{align}
    \end{subequations}
    nous trouvons le résultat demandé.
\end{proof}

\begin{theorem}[\cite{FubiniBMauray,MesIntProbb}]   \label{ThoWWAjXzi}
    Soient \( \mu_i\) des mesures $\sigma$-finies sur \( (\Omega_i,\tribA_i)\) (\( i=1,2\)). 
    \begin{enumerate}
        \item
            
    Il existe une et une seule mesure, notée \( \mu_1\otimes \mu_2\), sur \( (\Omega_1\times\Omega_2,\tribA_1\otimes\tribA_2)\) telle que
    \begin{equation}    \label{EqOIuWLQU}
        (\mu_1\otimes\mu_2)(A_1\times A_2)=\mu_1(A_1)\mu_2(A_2)
    \end{equation}
    pour tout \( A_1\in \tribA_1\) et \( A_2\in\tribA_2\). 
\item
    Cette mesure est donnée par la formule\footnote{Voir les notations du lemme \ref{LemAQmWEmN}.}
    \begin{equation}   \label{EqDFxuGtH}
        (\mu_1\otimes \mu_2)(A)=\int_{\Omega_1}\mu_2\big( A_2(x) \big)d\mu_1(x)=\int_{\Omega_2}\mu_1\big( A_1(y) \big)d\mu_2(y).
    \end{equation}
    Cette mesure est la \defe{mesure produit}{mesure!produit} de \( \mu_1\) par \( \mu_2\).
\item
    La mesure \( \mu_1\otimes \mu_2\) ainsi définie est \( \sigma\)-finie.
    \end{enumerate}
\end{theorem}
\index{mesure!produit}

\begin{proof}
    La partie «existence» sera divisée en deux parties : l'une pour prouver que les formules \eqref{EqDFxuGtH} donnent une mesure et une pour montrer que cette mesure vérifie la condition \eqref{EqOIuWLQU}.
    \begin{subproof}
    \item[Unicité]
        
    L'ensemble des rectangles de \( \Omega_1\times \Omega_2\) engendre la tribu \( \tribA_1\otimes\tribA_2\), est fermé par intersection et contient une suite croissante d'ensembles \( P_n\times R_n\) de mesure finie (\( \mu(P_n\times R_n)<\infty\)) telle que \( P_n\times R_n\to \Omega_1\times \Omega_2\). Cette suite est donné par le fait que \( \mu_1\) et \( \mu_2\) sont \( \sigma\)-finies. En effet si \( (X_n)\) et \( (Y_n)\) sont des recouvrements dénombrables de \( \Omega_1\) et \( \Omega_2\) par des ensembles de mesure finie, en posant \( P_n=\bigcup_{k=1}^nX_n\) et \( R_n=\bigcup_{k=1}^nY_n\) nous avons bien une suite croissante de rectangles qui tendent vers \( \Omega_1\times \Omega_2\). Avec ces rectangles en main, le théorème \ref{ThoJDYlsXu} donne l'unicité.

\item[Les formules définissent une mesure]
    Le théorème \ref{ThoCCIsLhO} dit que ces formules ont un sens et que l'égalité entre les deux intégrales est correcte. Nous prouvons à présent qu'elles déterminent effectivement une mesure sur \( (\Omega_1\times\Omega_2,\tribA_1\otimes \tribA_2)\).

    Pour tout \( A\in \tribA_1\otimes \tribA_2\), \( \mu(A)\geq 0\) parce que \( \mu\) est donnée par l'intégrale d'une fonction positive.

    En ce qui concerne la condition d'unions dénombrable disjointe, soient \( A^{i}\) des éléments disjoints de \( \tribA_1\otimes \tribA_2\); nous commençons par remarquer que
    \begin{subequations}
        \begin{align}
            \left( \bigcup_{i=1}^{\infty}A^{(i)} \right)_2(x)&=\{ y\in\Omega_2\tq (x,y)\in\bigcup_{i=1}^{\infty}A^{(i)} \}\\
            &=\bigcup_{i=1}^{\infty}\{ y\in\Omega_2\tq (x,y)\in A^{(i)} \}\\
            &=\bigcup_{i=1}^{\infty}A^{(i)}_2(x).
        \end{align}
    \end{subequations}
    Par conséquent,
    \begin{subequations}
        \begin{align}
            \mu\left( \bigcup_{i=1}^{\infty}A^{(i)} \right)&=\int_{\Omega_1}\mu_2\left(    \Big( \bigcup_{i=1}^{\infty}A^{(i)} \Big)_2(x)     \right)d\mu_1(x)\\
            &=\int_{\Omega_1}\sum_{i=1}^{\infty}\mu_2\big( A^{(i)}_2(x) \big)d\mu_1(x)\\
            &=\int_{\Omega_1}\lim_{n\to \infty} \sum_{i=1}^{n}\mu_2\big( A^{(i)}_2(x) \big)d\mu_1(x).
        \end{align}
    \end{subequations}
    où nous avons utilisé l'additivité de la mesure \( \mu_2\). À ce niveau, il serait commode de permuter la somme et l'intégrale. Pour ce faire nous considérons la suite (croissante) de fonctions
    \begin{equation}
        f_n(x)=\sum_{i=1}^n\mu_2\big( A_2^{(i)}(x) \big).
    \end{equation}
    Nous pouvons permuter la limite et l'intégrale grâce au théorème de la convergence monotone \ref{ThoRRDooFUvEAN}; ensuite la somme se permute avec l'intégrale en tant que somme finie :
    \begin{subequations}
        \begin{align}
            \mu\left( \bigcup_{i=1}^{\infty}A^{(i)} \right)&=\lim_{n\to \infty} \sum_{i=1}^n\int_{\Omega_1}\big( A_2^{(i)}(x) \big)d\mu_1(x)\\
            &=\lim_{n\to \infty} \sum_{i=1}^n\mu(A^{(i)})\\
            &=\sum_{i=1}^{\infty}\mu( A^{(i)} ).
        \end{align}
    \end{subequations}

\item[Elles vérifient la condition]
    Prouvons que les formules \eqref{EqDFxuGtH} se réduisent à \eqref{EqOIuWLQU} dans le cas des rectangles. Soit donc \( A=X_1\times X_2\) avec \( X_i\in\tribA_i\). Alors
    \begin{equation}
        A_1(y)=\{ x\in\Omega_1\tq (x,y)\in X_1\times X_2 \}
    \end{equation}
    et
    \begin{equation}
        \mu_1\big( A_1(y) \big)=\mtu_{X_2}(y)\mu_1(X_1),
    \end{equation}
    donc
    \begin{subequations}
        \begin{align}
            (\mu_1\otimes\mu_2)(A)&=\int_{\Omega_2}\mu_1\big( A_1(y) \big)d\mu_2(y)\\
            &=\int_{\Omega_2}\mu_1(X_1)\mtu_{X_2}(y)d\mu_2(y)\\
            &=\mu_1(X_1)\int_{\Omega_2}\mtu_{X_2}(y)d\mu_2(y)\\
            &=\mu_1(X_1)\mu_2(X_2).
        \end{align}
    \end{subequations}
    Pour cela nous avons utilisé le fait que l'intégrale de la fonction caractéristique d'un ensemble mesurable est la mesure de cet ensemble.
    \end{subproof}
\end{proof}

\begin{definition}[Produit d'espaces mesurés]  \label{DefUMlBCAO}
    Si \( (\Omega_i,\tribA_i,\mu_i)\) sont deux espaces mesurés, l'\defe{espace produit}{produit!espaces mesurés} est l'ensemble \( \Omega_1\times \Omega_2\) muni de la tribu produit \( \tribA_1\otimes \tribA_2\) de la définition \ref{DefTribProfGfYTuR} et de la mesure produit \( \mu_1\otimes \mu_2\) définie par le théorème \ref{ThoWWAjXzi}.
\end{definition}

\begin{remark}
    Il n'est pas garantit que la tribu \( \tribA_1\otimes\tribA_2\) soit la tribu la plus adaptée à l'ensemble \( S_1\times S_2\). Dans le cas de \( \eR^N\), il se fait que c'est le cas : en prenant des produits des boréliens sur \( \eR\) on obtient bien les boréliens sur \( \eR^N\), voir proposition \ref{CorWOOOooHcoEEF}.
\end{remark}
