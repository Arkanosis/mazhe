% This is part of Mes notes de mathématique
% Copyright (c) 2011-2015
%   Laurent Claessens, Carlotta Donadello
% See the file fdl-1.3.txt for copying conditions.

%+++++++++++++++++++++++++++++++++++++++++++++++++++++++++++++++++++++++++++++++++++++++++++++++++++++++++++++++++++++++++++ 
\section{Fonctions mesurables}
%+++++++++++++++++++++++++++++++++++++++++++++++++++++++++++++++++++++++++++++++++++++++++++++++++++++++++++++++++++++++++++

%--------------------------------------------------------------------------------------------------------------------------- 
\subsection{Propriétés}
%---------------------------------------------------------------------------------------------------------------------------

\begin{definition}[Fonction mesurable] \label{DefQKjDSeC}
    Soit \( (E,\tribA)\) et \( (F,\tribF)\) deux espaces mesurés. Une fonction \( f\colon E\to F\) est \defe{mesurable}{mesurable!fonction} si pour tout \( \mO\in \tribF\), l'ensemble \( f^{-1}(\mO)\) est dans \( \tribA\).
\end{definition}

\begin{definition}[Fonction borélienne]     \label{DefHHIBooNrpQjs}
    Une fonction à valeurs dans \( \eR^d\) est \defe{borélienne}{borélienne!fonction}\index{fonction!borélienne} si elle est mesurable pour la tribu des boréliens sur \( \eR^d\). Plus explicitement, \( f\colon (\Omega,\tribA)\to (\eR^d,\Borelien(\eR^d))\) est borélienne si pour tout \( \mO\in\Borelien\) nous avons \( f^{-1}(\mO)\in\tribA\).

    Plus généralement, une application entre deux espace topologiques est borélienne lorsqu'elle est mesurable quand on considère la tribu des boréliens sur les deux espaces.
\end{definition}
Si \( \tribA\) est une tribu sur un ensemble \( E\), nous notons \( m(\tribA)\)\nomenclature[P]{\( m(\tribA)\)}{Ensemble des fonctions \( \tribA\)-mesurables} l'ensemble des fonctions qui sont \( \tribA\)-mesurables.

Le plus souvent lorsque nous parlerons de fonctions \( f\colon X\to Y\) où \( Y\) est un espace topologique, nous considérons la tribu borélienne sur \( Y\). Ce sera en particulier le cas dans la théorie de l'intégration.

\begin{remark}
    Lorsque nous considérons des fonctions à valeurs réelles \( f\colon X\to \eR\) nous utiliserons toujours la tribu borélienne sur \( \eR\). Pour \( X\), cela peut dépendre des contextes. En théorie de l'intégration, nous mettrons sur \( X\) la tribu des ensembles mesurables au sens de Lebesgue sur \( X\), \emph{tout en gardant celle des boréliens sur l'ensemble d'arrivée}.

    Pour toute la partie sur l'intégration, une fonction \( f\colon \eR^n\to \eR^m\) sera mesurable si pour tout borélien \( A\) de \( \eR^m\) l'ensemble \( f^{-1}(A)\) est Lebesgue-mesurable dans \( \eR^n\).

    Étant donné qu'il est franchement difficile de créer des ensembles non mesurables au sens de Lebesgue, il est franchement difficile de créer des fonction non mesurables à valeurs réelles. L'hypothèse de mesurabilité est donc toujours satisfaite dans les cas pratiques.
\end{remark}

\begin{lemma}[\cite{NBoIEXO}]   \label{LemFOlheqw}
    Une fonction \( f\colon X\to \eR\) est mesurable si et seulement si \( f^{-1}(I)\) est mesurable pour tout \( I\) de la forme \( \mathopen] a , \infty \mathclose[\).
\end{lemma}

\begin{proof}
    Nous devons prouver que \( f^{-1}(A)\) est mesurable dans \( X\) pour tout borélien \( A\) de \( \eR\). Nous posons
    \begin{equation}
        S=\{ A\subset \eR\tq f^{-1}(A)\text{ est mesurable dans \( X\)} \}
    \end{equation}
    et nous prouvons que cela est une tribu. D'abord \( f^{-1}(\eR)=X\), et \( X\) est mesurable, donc \( \eR\in S\). Ensuite si \( A\in S\) alors \( f^{-1}(A^c)=f^{-1}(A)^c\). En tant que complémentaire d'un mesurable de \( X\), l'ensemble \( f^{-1}(A)^c\) est mesurable dans \( X\). Et enfin si \( A_n\in S \) alors \( f^{-1}(\bigcup_nA_n)=\bigcup_nf^{-1}(A_n)\) qui est encore mesurable dans \( X\) en tant qu'union de mesurables.

    Donc \( S\) est une tribu qui contient tous les ensembles de la forme \( \mathopen] a , \infty \mathclose]\). Le lemme \ref{LemZXnAbtl} conclu que \( S\) contient tous les boréliens de \( \eR\).
\end{proof}

\begin{lemma}[\cite{NBoIEXO}]   \label{LemIGKvbNR}
    Soit \( f_n\colon X\to \eR\) une suite de fonctions mesurables\footnote{Ici \( X\) est un espace mesuré et \( \eR\) est muni des boréliens.}. Alors \( \sup_n f_n\) est mesurable.
\end{lemma}

\begin{proof}
    Nous avons
    \begin{subequations}
        \begin{align}
            (\sup f_n)^{-1}\big( \mathopen] a , \infty \mathclose] \big)&=\{ x\in X\tq (\sup f_n)(x)>a \}\\
            &=\bigcup_n\{ x\in X\tq f_n(x)>a \}\\
            &=\bigcup_nf_n^{-1}\big( \mathopen] a , \infty \mathclose] \big).
        \end{align}
    \end{subequations}
    Étant donné que \( f_n\) est mesurable et que \( \mathopen] a , \infty \mathclose]\) est mesurable, chacun des \( f_n^{-1}\big( \mathopen] a , \infty \mathclose] \big) \) est mesurable dans \( X\). Nous sommes en présence d'une union dénombrable de mesurables, donc \( (\sup f_n)^{-1}\big( \mathopen] a , \infty \mathclose] \big)\) est mesurable.

    Le lemme \ref{LemFOlheqw} conclu que \( \sup f_n\) est mesurable.
\end{proof}

\begin{proposition}\label{PropFYPEOIJ}
    Si \( f_n\) est une suite de fonctions mesurables et positives, alors la fonction \( \sum_nf_n\) est mesurable.
\end{proposition}

\begin{proof}
    Nous considérons les fonctions \( s_k(x)=\sum_{n=0}^kf_n(x)\) qui vaut éventuellement \( \infty\) en certains points. Nous avons
    \begin{equation}
        \sum_nf_n(x)=\sup_ks_k(x),
    \end{equation}
    donc le lemme \ref{LemIGKvbNR} nous donne la mesurabilité de la somme de \( f_n\).
\end{proof}

%--------------------------------------------------------------------------------------------------------------------------- 
\subsection{D'une tribu à l'autre}
%---------------------------------------------------------------------------------------------------------------------------


\begin{lemma}[\cite{TribuLi}]
    Soit une application \( f\colon S_1\to S_2\) et une tribu \( \tribF_2\) sur \( S_2\). Alors \( f^{-1}(\tribF_2)\) est une tribu sur \( S_1\)
\end{lemma}

\begin{proof}
    Il faut prouver les trois propriétés de la définition \ref{DefjRsGSy} d'une tribu.
    \begin{enumerate}
        \item
            D'abord \( f\) est définit sur tout \( S_1\), donc \( f^{-1}(S_2)=S_1\) alors que \( S_2\in \tribF_2\).
        \item
            Soit \( A\in f^{-1}(\tribF_2)\), c'est à dire \( A=f^{-1}(B)\) pour un certain \( B\in \tribF_2\). En ce qui concerne le complémentaire :
            \begin{equation}
                A^c=f^{-1}(B)^c=S_1\setminus f^{-1}(B)=f^{-1}(S_2\setminus B)=f^{-1}(B^c).
            \end{equation}
        \item
            Si \( (A_i)_{i\in \eN}\) sont des éléments de \( f^{-1}(\tribF_2)\) avec \( A_i=f^{-1}(B_i)\) alors
            \begin{equation}
                \bigcup_iA_i=\bigcup_if^{-1}(B_i)=f^{-1}\big( \bigcup_iB_i \big).
            \end{equation}
            Ce qui est dans la dernière parenthèse est dans \( \tribF_2\) parce que cette dernière est une tribu.
    \end{enumerate}
\end{proof}

\begin{lemma}[\cite{TribuLi}]       \label{LemJYKBooBSXBXJ}
    Soit une application \( f\colon S_1\to S_2\) et \( \tribF\) une tribu de \( S_1\). Alors
    \begin{enumerate}
        \item
            L'ensemble
            \begin{equation}
                \tribF_f=\{  B\subset S_2\tq f^{-1}(B)\in \tribF  \}
            \end{equation}
            est une tribu sur \( S_2\).
        \item
            C'est la plus grande tribu de \( S_2\) pour laquelle \( f\) est mesurable.
    \end{enumerate}
\end{lemma}

\begin{proof}
    Encore les trois propriétés à vérifier.
    \begin{enumerate}
        \item
            \( S_2\in\tribF\), sont \( S_1=f^{-1}(S_2)\in \tribF_f\).
        \item
            Si \( A\in \tribF_f\) alors \( A=f^{-1}(B)\) pour un certain \( B\in \tribF\). Nous avons alors aussi \( B^c\in \tribF\) et donc 
            \begin{equation}
                f^{-1}(B^c)=f^{-1}(B)^c=A^c.
            \end{equation}
            Par conséquent \( A^c\) est dans \( \tribF_f\).
        \item
            Si \( (A_i)\) sont des éléments de \( \tribF_f\) avec \( A_i=f^{-1}(B_i)\) pour \( B_i\in \tribF\) alors \( \bigcup_iB_i\in\tribF\) et
            \begin{equation}
                f^{-1}\big( \bigcup_iB_i \big)=\bigcup_if^{-1}(B_i)\in\tribF_f.
            \end{equation}
    \end{enumerate}
    En ce qui concerne la maximalité, si \( R\subset S_2\) n'est pas dans \( \tribF_f\) alors \( f^{-1}(R)\) n'est pas dans \( \tribF\) et donc \( f\) ne serait pas mesurable.
\end{proof}

\begin{definition}[Tribu engendrée] \label{DefNOJWooLGKhmJ}
    Soit une application \( f\colon S_1\to S_2\) et \( \tribF\) une tribu de \( S_1\). Alors conformément au lemme \ref{LemJYKBooBSXBXJ} l'ensemble
            \begin{equation}
                \tribF_f=\{  B\subset S_2\tq f^{-1}(B)\in \tribF  \}
            \end{equation}
            est la \defe{tribu engendrée}{tribu!engendrée!par une application}.
\end{definition}

\begin{lemma}[Lemme de transfert]       \label{LemOQTBooWGYuDU}
    Soit \( f\colon S_1\to S_2\) une application et une classe \( \tribC\) de parties de \( S_2\). Alors
    \begin{equation}
        \sigma\big( f^{-1}(\tribC) \big)=f^{-1}\big( \sigma(\tribC) \big).
    \end{equation}
\end{lemma}
\index{lemme!de transfert}

\begin{proof}
    Vu que \( \sigma(\tribC)\) es tune tribu, dans \( S_2\) alors le lemme \ref{LemJYKBooBSXBXJ} dit que \( f^{-1}\big( \sigma(\tribC) \big)\) est une tribu qui contient en particulier \(  f^{-1}(\tribC) \). Nous en déduisons que \( \sigma\big( f^{-1}(\tribC) \big)\subset f^{-1}\big( \sigma(\tribC) \big)\).

    Réciproquement. Dans \( S_1\) nous avons la tribu \( \sigma\big( f^{-1}(\tribC) \big)\). Nous pouvons alors considérer la tribu
    \begin{equation}
        \tribF_f=\{ B\subset S_2\tq f^{-1}(B)\in\sigma\big( f^{-1}(\tribC) \big) \}.
    \end{equation}
    Montrons que \( \tribC\subset \tribF_f\). Lorsque \( B\in \tribC\) nous avons \( f^{-1}(B)\in f^{-1}(\tribC)\subset\sigma\big( f^{-1}(\tribC) \big)\). Du coup \( B\in \tribF_f\). Nous avons alors, en passant aux tribus engendrées :
    \begin{equation}
        \sigma(\tribC)\subset\sigma(\tribF_f)=\tribF_f.
    \end{equation}
    Si maintenant \( B\in\sigma(\tribC)\), nous avons \( f^{-1}(B)\in \sigma\big( f^{-1}(\tribC) \big)\), ce qui signifie que
    \begin{equation}
        f^{-1}\big( \sigma(\tribC) \big)\subset\sigma\big( f^{-1}(\tribC) \big).
    \end{equation}
\end{proof}

Le théorème suivant est important pour prouver qu'une application est mesurable. En effet, il permet de ne tester si une application est mesurable uniquement sur une partie génératrice de la tribu d'arrivé\footnote{Typiquement les ouverts pour les boréliens.}.
\begin{theorem}     \label{ThoECVAooDUxZrE}
    Soient des espaces mesurables \( ( S_1,\tribF_1 )\) et \( (S_2,\tribF_2)\) ainsi qu'une application \( f\colon S_1\to S_2\). Si il existe un ensemble de parties \( \tribC\) de \( S_2\) tel que
    \begin{itemize}
        \item \( \sigma(\tribC)=\tribF_2\)
        \item \( f^{-1}(B) \in \tribF_1 \) pour tout \( B\in \tribC\)
    \end{itemize}
    alors \( f\) est mesurable.
\end{theorem}

\begin{proof}
    Par hypothèse, \( \sigma(\tribC)=\tribF_2\) et \( f^{-1}(\tribC)\subset \tribF_1\) et nous pouvons utiliser le lemme de transfert \ref{LemOQTBooWGYuDU} :
    \begin{equation}
        \sigma\big( f^{-1}(\tribC) \big)=f^{-1}\big( \sigma(\tribC) \big)
    \end{equation}
    qui s'écrit ici
    \begin{equation}
        \sigma\big( f^{-1}(\tribC) \big)=f^{-1}(\tribF_2).
    \end{equation}
    Mais vu que \( f^{-1}(\tribC)\subset \tribF_1\), nous avons aussi \( \sigma\big( f^{-1}(\tribC) \big)\subset \tribF_1\), ce qui signifie que
    \begin{equation}
        f^{-1}(\tribF_2)\subset \tribF_1.
    \end{equation}
    Cela est exactement le fait que \( f\) soit mesurable.
\end{proof}

Le théorème suivant est très important parce qu'en pratique c'est souvent lui qui permet de déduire qu'une fonction est borélienne.
\begin{theorem}[\cite{TribuLi}]     \label{ThoJDOKooKaaiJh}
    Soient \( X\) et \( Y\) deux espaces topologiques. Alors toute application continue \( f\colon X\to Y\) est borélienne\footnote{Définition \ref{DefHHIBooNrpQjs}.}.
\end{theorem}

\begin{proof}
    Pour vérifier que \( f\) est borélienne, nous devons prouver que \( f^{-1}(B)\) est borélien pour tout borélien \( B\) de \( Y\). Heureusement, le théorème \ref{ThoECVAooDUxZrE} nous permet de limiter la vérification aux \( B\) appartenant à une classe engendrant les boréliens de \( Y\).

    La classe en question est toute trouvée : ce sont les ouverts. Si \( \mO\) est un ouvert de \( Y\) alors \( f^{-1}(\mO)\) est un ouvert de \( X\) et donc un borélien de \( X\).
\end{proof}

%--------------------------------------------------------------------------------------------------------------------------- 
\subsection{Mesure image}
%---------------------------------------------------------------------------------------------------------------------------

\begin{proposition}[Mesure image\cite{TribuLi}]
    Soient \( (S_1,\tribF_1)\) et \( (S_2,\tribF_2)\) des espaces mesurables. Soit \( \varphi\colon S_1\to S_2\) une application mesurable. Si \( m_1\) est une mesure positive sur \( S_1\) alors l'application définie par
    \begin{equation}
        m_2(A_2)=m_1\big( \varphi^{-1}(A_2) \big)
    \end{equation}
    est une mesure positive sur \( (S_2,\tribF_2)\).
\end{proposition}

\begin{definition}
    La mesure \( m_2\) ainsi définie est la \defe{mesure image}{mesure!image} de \( m_1\) par l'application \( \varphi\). Elle est notée \( \varphi(m_1)\).
\end{definition}

\begin{proof}
    Il y a deux choses à vérifier pour avoir une mesure positive\footnote{Définition \ref{DefBTsgznn}}.

    D'abord pour l'ensemble vide :
    \begin{equation}
        m_2(\emptyset)=m_1\big( \varphi^{-1}(\emptyset) \big)=m_1(\emptyset)=0.
    \end{equation}
    Ensuite pour l'additivité. Soient \( A_n\) dans \( \tribF_2\) des parties deux à deux disjointes et telles que \( \bigcup_nA_n\in\tribF_2\). Alors nous avons
    \begin{subequations}
        \begin{align}
            m_2\big( \bigcup_nA_n \big)&=m_1\Big( \varphi^{-1}(\bigcup_nA_n) \Big)\\
            =&m_1\big( \bigcup_n\varphi^{-1}(A_n) \big)\\
            &=\sum_nm_1\big( \varphi(A_n) \big)\\
            &=\sum_nm_2(A_n).
        \end{align}
    \end{subequations}
\end{proof}

%--------------------------------------------------------------------------------------------------------------------------- 
\subsection{Fonction simple}
%---------------------------------------------------------------------------------------------------------------------------

\begin{definition}  \label{DefBPCxdel}
Une fonction mesurable \( f\colon X\to \eR\) est \defe{simple}{simple!fonction}\index{fonction!simple} si son image est finie
\begin{equation}
    f(x)=\sum_{j=1}^p\alpha_j\mtu_{A_j}(x)
\end{equation}
où \( A_j=f^{-1}(\alpha_j)\). Notons que \( f\) étant mesurable, les ensembles \( A_j\) sont forcément mesurables.
\end{definition}

\begin{lemma}[Limite croissante de fonctions étagées mesurables]    \label{LemYFoWqmS}
    Soit \( f\colon (\Omega,\tribA)\to \eR\) une fonction mesurable. Il existe une suite \( f_n\colon \Omega\to \eR\) de fonctions simples telles que \( f_n\to f\) ponctuellement et \( | f_n |<f\).
\end{lemma}

\begin{proof}
    Nous considérons \( (q_n)\) une suite parcourant tous les rationnels\footnote{Nous rappelons que \( \eQ\) est dénombrable et dense dans \( \eR\).}.
    Pour \( n\in \eN\) nous définissons la fonction
    \begin{equation}
        f_n(\omega)=\begin{cases}
            \max\{ q_i\tq i\leq n,\, q_i\leq f(\omega) \}    &   \text{si \( f(\omega)\geq 0\)}\\
            \min\{ q_i\tq i\leq n,\, q_i\geq f(\omega) \}    &    \text{si \( f(\omega)< 0\).}
        \end{cases}
    \end{equation}
    La fonction \( f_n\) est simple parce qu'elle ne prend que \( n\) valeurs différentes. Nous avons aussi par construction que \( | f_n(\omega)|\leq |f(\omega) |\). Nous avons aussi pour tout \( \omega\in \Omega\) que \( f_n(\omega)\to f(\omega)\) parce que \( \eQ\) est dense dans \( \eR\).

    En ce qui concerne la mesurabilité de \( f_n\), les étages de \( f_n\) sont les ensembles de la forme \( \{ \omega\in \Omega\tq f(\omega)\in\mathopen[ a , b [ \}\) où \( a\) et \( b\) sont deux éléments de \( \{ q_1,\ldots, q_n \}\) qui sont consécutifs au sens de l'ordre dans \( \eQ\) (et non spécialement au sens de l'ordre d'apparition dans la suite), avec éventuellement \( b=\infty\) si \( a\) est le plus grand. Les ensembles \( \mathopen[ a , b [\) étant mesurables dans \( \eR\) et la fonction \( f\) étant mesurable par hypothèse, les ensembles \( f^{-1}\Big( \mathopen[ a , b [ \Big)\) sont mesurables dans \( (\Omega,\tribA)\).
\end{proof}

\begin{proposition}\label{PropWBavIf}
    Une fonction positive et mesurable sur \( \Omega\) est limite ponctuelle croissante de fonctions simples positives.
\end{proposition}

\begin{proof}
    Soit \( \{ q_i \}\) une énumération des rationnels positifs. Il suffit de poser
    \begin{equation}
        f_n(x)=\max\{ q_i\tq i\leq n,\text{ et }f(x)\geq q_i \}.
    \end{equation}
\end{proof}

%---------------------------------------------------------------------------------------------------------------------------
\subsection{Théorème de récurrence}
%---------------------------------------------------------------------------------------------------------------------------

Soit \( X\) un espace mesurable, \( \mu\) une mesure finie sur \( X\) et \( \phi\colon X\to X\) une application mesurable préservant la mesure, c'est à dire que pour tout ensemble mesurable \( A\subset X\),
\begin{equation}
    \mu\big( \phi^{-1}(A) \big)=\mu(A).
\end{equation}
Si \( A\subset X\) est un ensemble mesurable, un point \( x\in A\) est dit \defe{récurrent}{récurrent!point d'un système dynamique} par rapport à \( A\) si et seulement si pour tout \( p\in \eN\), il existe \( k\geq p\) tel que \( \phi^k(x)\in A\).

\begin{theorem}[\wikipedia{fr}{Théorème_de_récurrence_de_Poincaré}{Théorème de récurrence de Poincaré}.]     \label{ThoYnLNEL}
    Si \( A\) est mesurable dans \( X\), alors presque tous les points de \( A\) sont récurrents par rapport à \( A\).
\end{theorem}

\begin{proof}
    Soit \( p\in \eN\) et l'ensemble
    \begin{equation}
        U_p=\bigcup_{k=p}^{\infty}\phi^{-k}(A)
    \end{equation}
    des points qui repasseront encore dans \( A\) après \( p\) itérations  de \( \phi\). C'est un ensemble mesurable en tant que union d'ensembles mesurables (pour rappel, les tribus sont stables par union dénombrable, comme demandé à la définition \ref{DefjRsGSy}), et nous avons donc
    \begin{equation}
        \mu(U_p)\leq \mu(X)<\infty.
    \end{equation}
    De plus \( U_p=\phi^{-p}(U_0)\), donc \( \mu(U_p)=\mu(U_0)\). Vu que \( U_p\subset U_p\), nous avons
    \begin{equation}
        \mu(U_0\setminus U_p)=0.
    \end{equation}
    Étant donné que \( A\subset U_0\) nous avons a fortiori que
    \begin{equation}
        \{ x\in A\tq x\notin U_p \}\subset U_0\setminus U_p,
    \end{equation}
    et donc
    \begin{equation}
        \mu\{ x\in A\tq x\notin U_p \}=0.
    \end{equation}
    Cela signifie exactement que l'ensemble des points \( x\) de \( A\) tels que aucun des \( \phi^k(x)\) avec \( k\geq p\) n'est dans \( A\) est de mesure nulle.
\end{proof}

%+++++++++++++++++++++++++++++++++++++++++++++++++++++++++++++++++++++++++++++++++++++++++++++++++++++++++++++++++++++++++++ 
\section{Intégrale par rapport à une mesure}
%+++++++++++++++++++++++++++++++++++++++++++++++++++++++++++++++++++++++++++++++++++++++++++++++++++++++++++++++++++++++++++

Nous avons besoin d'un peu de théorie de l'intégration parce que la définition de la mesure sur un espace mesurable\footnote{Théorème \ref{ThoWWAjXzi}.} produit passe par une intégrale.

Une mesure \( \mu\) sur un espace mesurable \( (\Omega,\tribA)\) permet de définir une fonctionnelle linéaire sur l'ensemble des fonctions mesurables \( \Omega\to \eR\). Cette fonctionnelle linéaire est l'intégrale que nous allons définir à présent.

\begin{definition}  \label{DefTVOooleEst}
    Si \( Y\in \tribA\) et si \( f\) est une fonction simple nous définissons
    \begin{equation}
        \int_Yfd\mu=\sum_ia_i\mu(Y\cap E_i).
    \end{equation}
    Pour une fonction \( \mu\)-mesurable générale \( f\colon \Omega\to \mathopen[ 0 , \infty \mathclose]\) nous définissons l'intégrale de \( f\) sur \( Y\) par
    \begin{equation}        \label{EqDefintYfdmu}
        \int_Yfd\mu=\sup\Big\{ \int_Yhd\mu\,\text{où \( h\) est une fonction simple et mesurable telle que \( 0\leq h\leq f\)} \Big\}.
    \end{equation}
\end{definition}

\begin{definition}  \label{DefTCXooAstMYl}
    Une fonction \( f\colon \Omega\to \eR\) est dite \defe{intégrable}{intégrable} au sens de Lebesgue si \( \int_{\Omega}| f |<\infty\). Dans ce cas nous définissons
    \begin{equation}    \label{EqUHSooWfgUty}
        \int_{\Omega}f=\int_{\Omega}f^+-\int_{\Omega}f^-
    \end{equation}
    où \( f^+\) et \( f^-\) sont les parties positives et négatives de \( f\); les deux intégrales à droite dans \eqref{EqUHSooWfgUty} sont finies dès que \( f\) est intégrables.
\end{definition}
Nous verrons comment donner un sens à \( \int_{\Omega}f\) dans certains cas où \( f\) n'est pas intégrable sur \( \Omega\) dans la section \ref{SecGAVooBOQddU} sur les intégrales impropres.

Nous définissons aussi
\begin{equation}
    \mu(f)=\int_{\Omega}f
\end{equation}
si \( f\) est une fonction mesurable sur \( \Omega\).

\begin{remark}
    Dans \( \eR^d\), quasiment toutes les fonctions et ensembles sont mesurables. En effet la construction d'ensembles non mesurables demande obligatoirement l'utilisation de l'axiome du choix; de tels ensembles doivent être construits «exprès pour». Il y a très peu de chances pour que vous tombiez sur un ensemble non mesurable de \( \eR^d\) sans que vous ne vous en rendiez compte.
\end{remark}

\begin{remark}
    «Mesurable» ne signifie pas «intégrable». Par exemple la fonction 
    \begin{equation}
        \begin{aligned}
            f\colon \eR&\to \bar\eR \\
            \omega&\mapsto\begin{cases}
            \frac{1}{ \omega }    &   \text{si $ \omega\neq 0$}\\
            \infty    &    \text{$\omega=0$}.
            \end{cases}
        \end{aligned}
    \end{equation}
    est mesurable, mais non intégrable.
\end{remark}

\begin{proposition}     \label{PropOPSCooVpzaBt}
    Si \( A,B\subset \Omega\) sont des ensembles disjoints et si \( f\) est intégrable sur \( A\cup B\) alors
    \begin{equation}
        \int_{A\cup B}f=\int_Af+\int_Bf.
    \end{equation}
\end{proposition}

%--------------------------------------------------------------------------------------------------------------------------- 
\subsection{Quelque propriétés}
%---------------------------------------------------------------------------------------------------------------------------

Le lemme suivant nous aide à détecter des fonctions presque partout nulles.
\begin{lemma}   \label{Lemfobnwt}
    Soit \( f\) une fonction mesurable positive ou nulle telle que
    \begin{equation}
        \int_{\Omega}fd\mu=0.
    \end{equation}
    Alors \( f=0\) \( \mu\)-presque partout.
\end{lemma}

\begin{proof}
    L'ensemble des points \( x\in\Omega\) tels que \( f(x)\neq 0\) peut s'écrire comme une union dénombrable disjointe :
    \begin{equation}
        \{ x\in\Omega\tq f(x)\neq 0 \}=\bigcup_{i=0}^{\infty}E_i
    \end{equation}
    avec
    \begin{subequations}
        \begin{align}
            E_0&=\{ x\in\Omega\tq f(x)>1 \}\\
            E_i&=\{ x\in\Omega\tq \frac{1}{ i+1 }\leq f(x)<\frac{1}{ i } \}.
        \end{align}
    \end{subequations}
    Si un des ensembles \( E_i\) est de mesure non nulle, alors nous pouvons considérer la fonction simple \( h(x)=\frac{1}{ i+1 }\mtu_{E_i}\) dont l'intégrale sur \( \Omega\) est strictement positive. Par conséquent le supremum de la définition \eqref{EqDefintYfdmu} est strictement positif.

    Nous savons donc que \( \mu(E_i)=0\) pour tout \( i\). Étant donné que la mesure d'une union disjointe dénombrable est égale à la somme des mesures, nous avons
    \begin{equation}
        \mu\{ x\in\Omega\tq f(x)\neq 0 \}=0,
    \end{equation}
    ce qui signifie que \( f\) est nulle \( \mu\)-presque partout.
\end{proof}

\begin{corollary}   \label{CorjLYiSm}
    Soit \( f\) une fonction mesurable sur l'espace mesuré \( (\Omega,\tribA,\mu)\) telle que
    \begin{equation}
        \int_{\Omega}f\mtu_{f>0}d\mu=0.
    \end{equation}
    Alors \( f\leq 0\) presque partout.
\end{corollary}

\begin{proof}
    Nous avons l'égalité d'ensembles
    \begin{equation}
        \{ f\mtu_{f>0}\neq 0 \}=\{ \mtu_{f>0}\neq 0 \}.
    \end{equation}
    Mais lemme \ref{Lemfobnwt} implique que \( f\mtu_{f>0}\) est nulle presque partout, c'est à dire que la mesure de l'ensemble du membre de gauche est nulle par conséquent
    \begin{equation}
        \mu\{ \mtu_{f>0}\neq 0 \}=0.
    \end{equation}
    Cela signifie que la fonction \( f\) est presque partout négative ou nulle.
\end{proof}

\begin{lemma}   \label{LemPfHgal}
    Soit \( f\) une fonction telle que \( | f(x)|\leq g(x) \) pour tout \( x\in\Omega\). Si \( g\) est intégrable, alors \( f\) est intégrable.
\end{lemma}

\begin{proof}
    Nous décomposons \( f\) en parties positives et négatives :
    \begin{subequations}
        \begin{align}
            A_+&=\{ x\in\Omega\tq f(x)>0 \}\\
            A_-&=\{ x\in\Omega\tq f(x)<0 \}.
        \end{align}
    \end{subequations}
    Nous posons \( f_+(x)=f(x)\mtu_{A_+}\) et \( f_-(x)=f(x)\mtu_{A_-}\). Nous avons \( f=f_+-f_-\) et
    \begin{equation}
        \int_{\Omega}f=\int_{A_+}f+\int_{A_-}f
    \end{equation}
    parce que \( \Omega=A_+\cup A_-\cup\{ x\in\Omega\tq f(x)=0 \}\). Si \( \varphi\) est une fonction simple qui majore \( f_+\) nous avons
    \begin{equation}
        \varphi(x)=\sum_{k}a_k\mtu_{E_k}(x)\leq f(x)\mtu_{A_+}(x)\leq g(x).
    \end{equation}
    Par conséquent le supremum qui définit \( \int f_+\) est inférieur au supremum qui définit \( \int g\). La fonction \( f_+\) est donc intégrable. La même chose est valable pour la fonction \( f_-\).
\end{proof}

%+++++++++++++++++++++++++++++++++++++++++++++++++++++++++++++++++++++++++++++++++++++++++++++++++++++++++++++++++++++++++++ 
\section{Permuter limite et intégrale}
%+++++++++++++++++++++++++++++++++++++++++++++++++++++++++++++++++++++++++++++++++++++++++++++++++++++++++++++++++++++++++++

%--------------------------------------------------------------------------------------------------------------------------- 
\subsection{Convergence uniforme}
%---------------------------------------------------------------------------------------------------------------------------

\begin{proposition}[Permuter limite et intégrale]       \label{PropbhKnth}
    Soit \( f_n\to f\) uniformément sur un ensemble mesuré \( A\) de mesure finie. Alors si les fonctions \( f_n\) et \( f\) sont intégrables sur \( A\), nous avons
    \begin{equation}
        \lim_{n\to \infty} \int_A f_n=\int_A \lim_{n\to \infty} f_n.
    \end{equation}
\end{proposition}

\begin{proof}
    Notons \( f\) la limite de la suite \( (f_n)\). Pour tout \( n\) nous avons les majorations
    \begin{subequations}
        \begin{align}
            \left| \int_A f_n d\mu-\int_A fd\mu \right| &\leq \int_A| f_n-f |d\mu\\
            &\leq \int_A \| f_n-f \|_{\infty}d\mu\\
            &=\mu(A)\| f_n-f \|_{\infty}
        \end{align}
    \end{subequations}
    où \( \mu(A)\) est la mesure de \( A\). Le résultat découle maintenant du fait que \( \| f_n-f \|_{\infty}\to 0\).
\end{proof}
Il existe un résultat considérablement plus intéressant que cette proposition. En effet, l'intégrabilité de \( f\) n'est pas nécessaire. Cette hypothèse peut être remplacée soit par l'uniforme convergence de la suite (théorème \ref{ThoUnifCvIntRiem}), soit par le fait que les normes des \( f_n\) sont uniformément bornées (théorème de la convergence dominée de Lebesgue \ref{ThoConvDomLebVdhsTf}).

\begin{theorem}[\cite{BJblWiS}]			\label{ThoUnifCvIntRiem}
    La limite uniforme d'une suite de fonctions intégrables sur un borné est intégrable, et on peut permuter la limite et l'intégrale. 
    
    Plus précisément, soit \( A\) un ensemble de \( \mu\)-mesure finie et \( f_n\colon A\to \eR\) des fonctions intégrables sur \( A\). Si la limite \( f_n\to f\) est uniforme, alors \( f\) est intégrable sur \( A\) et nous pouvons inverser la limite et l'intégrale :
    \begin{equation}
        \lim_{n\to \infty} \int_A f_n=\int_A\lim_{n\to \infty} f_n.
    \end{equation}
\end{theorem}

\begin{proof}
    Soit \( \epsilon>0\) et \( n\) tel que \( \| f_n-f \|_{\infty}\leq \epsilon\) (ici la norme uniforme est prise sur \( A\)). Étant donné que \( f_n\) est intégrable sur \( A\), il existe une fonction simple \( \varphi_n\) qui minore \( f_n\) telle que
    \begin{equation}
        \left| \int_{A}\varphi_n-\int_A f_n \right| <\epsilon.
    \end{equation}
    La fonction \( \varphi_n+\epsilon\) est une fonction simple qui majore la fonction \( f\). Si \( \psi\) est une fonction simple qui minore \( f\), alors
    \begin{equation}
        \int_A\psi\leq\int_A\varphi_n+\epsilon\leq\int_A f_n+\epsilon\mu(A).
    \end{equation}
    Par conséquent le supremum qui définit \( \int_A f\) existe, ce qui montre que \( f\) est intégrable. Le fait qu'on puisse inverser la limite et l'intégrale est maintenant une conséquence de la proposition \ref{PropbhKnth}.
\end{proof}

\begin{remark}
    L'hypothèse sur le fait que \( A\) soit de mesure finie est importante. Il n'est pas vrai qu'une suite uniformément convergente de fonctions intégrables est intégrables. En effet nous avons par exemple la suite
    \begin{equation}
        f_n(x)=\begin{cases}
            1/x    &   \text{si \( x<n\)}\\
            0    &    \text{sinon}
        \end{cases}
    \end{equation}
    qui converge uniformément vers \( f(x)=1/x\) sur \( A=\mathopen[ 1 , \infty [\). Le limite n'est cependant guerre intégrable sur \( A\).
\end{remark}

%---------------------------------------------------------------------------------------------------------------------------
\subsection{Convergence monotone}
%---------------------------------------------------------------------------------------------------------------------------

\begin{theorem}[Théorème de la convergence monotone ou de Beppo-Levi\cite{mathmecaChoi}] \label{ThoRRDooFUvEAN}
    Soit un espace mesuré \( (\Omega,\tribA,\mu)\) et \( (f_n)\) une suite croissante de fonctions mesurables à valeurs dans \( \mathopen[ 0 , \infty \mathclose]\). Alors la limite ponctuelle \( \lim_{n\to \infty} f_n\) existe, est mesurable et
    \begin{equation}    \label{EqFHqCmLV}
        \lim_{n\to \infty} \int_{\Omega}f_nd\mu= \int_{\Omega}\lim_{n\to \infty} f_nd\mu,
    \end{equation}
    cette intégrable valant éventuellement \( \infty\).
\end{theorem}
\index{théorème!convergence!monotone}
\index{théorème!Beppo-Levi}
\index{permuter!limite et intégrale!convergence monotone}

\begin{proof}
    La limite ponctuelle de la suite est la fonction à valeurs dans \( \mathopen[ 0 , \infty \mathclose]\) donnée par
    \begin{equation}
        f(x)=\lim_{n\to \infty} f_n(x).
    \end{equation}
    Ces limites existent parce que pour chaque \( x\) la suite \( f_n(x)\) est une suite numérique croissante. Nous notons
    \begin{equation}
        I_0=\int_{\Omega}fd\mu.
    \end{equation}
    Nous posons par ailleurs
    \begin{equation}
        I_n=\int_{\Omega}f_n.
    \end{equation}
    Cela est une suite numérique croissante qui a par conséquent une limite que nous notons \( I=\lim_{n\to \infty} I_n\). Notre objectif est de montrer que \( I=I_0\). D'abord par croissance de la suite, pour tous $n$ nous avons \( I_n\leq I_0\), par conséquent \( I\leq I_0\).

    Nous prouvons maintenant l'inégalité dans l'autre sens en nous servant de la définition \eqref{EqDefintYfdmu}. Soit une fonction simple \( h\) telle que \( h\leq f\), et une constante \( 0<C<1\). Nous considérons les ensembles
    \begin{equation}
        E_n=\{ x\in\Omega\tq f_n(x)\geq Ch(x) \}.
    \end{equation}
    Ces ensembles vérifient les propriétés \( E_n\subset E_{n+1}\) et \( \bigcup_{n=1}^{\infty}E_n=\Omega\). Pour chaque \( n\) nous avons les inégalités
    \begin{equation}
        \int_{\Omega}f_n\geq\int_{E_n}f_n\geq C\int_{E_n}h.
    \end{equation}
    Si nous prenons la limite \( n\to\infty\) dans ces inégalités,
    \begin{equation}
        \lim_{n\to \infty} \int_{\Omega}f_n\geq C\lim_{n\to \infty} \int_{E_n}h=C\int_{\Omega}h.
    \end{equation}
    Par conséquent \( \lim_{n\to \infty} \int f_n\geq C\int_{\Omega}h\). Mais étant donné que cette inégalité est valable pour tout \( C\) entre \( 0\) et \( 1\), nous pouvons l'écrire sans le \( C\) :
    \begin{equation}        \label{EqzAKEaU}
        \lim_{n\to \infty} \int_{\Omega}f_n\geq \int_{\Omega}h.
    \end{equation}
    Par définition, l'intégrale de \( f\) est donné par le supremum des intégrales de \( h\) où \( h\) est une fonction simple dominée par \( f\). En prenant le supremum sur \( h\) dans l'équation \eqref{EqzAKEaU} nous avons
    \begin{equation}
        \lim_{n\to \infty} \int_{\Omega}f_n\geq\int_{\Omega}f,
    \end{equation}
    ce qu'il nous fallait.
\end{proof}

\begin{remark}
    La proposition \ref{PropWBavIf} ainsi que le lemme \ref{LemYFoWqmS} montrent qu'une fonction mesurable peut-être écrite comme limite croissante de fonctions simples. Cela permet de démontrer des théorèmes en commençant par prouver sur les fonctions simples et en utilisant Beppo-Levi pour généraliser.
\end{remark}

\begin{remark}
    Une des raisons de demander la positivité des fonctions \( f_n\) est de n'avoir pas d'ambiguïté à parler d'intégrales qui valent \( \infty\). Si par exemple nous prenons \( \Omega=\mathopen[ 0 , 1 \mathclose]\) et que nous considérons
    \begin{equation}
        f_n(x)=\begin{cases}
            0    &   \text{si \( x\leq \frac{1}{ n }\)}\\
            \frac{1}{ x }    &    \text{sinon}.
        \end{cases}
    \end{equation}
    Ce sont des fonctions intégrables, mais la limite étant la fonction \( 1/x\), l'égalité \eqref{EqFHqCmLV} est une égalité entre deux intégrales valant \( \infty\).
\end{remark}

\begin{corollary}[Inversion de somme et intégrales] \label{CorNKXwhdz}
    Si \( (u_n)\) est une suite de fonctions mesurables positives ou nulles, alors
    \begin{equation}
        \sum_{i=0}^{\infty}\int u_i=\int\sum_{i=0}^{\infty}u_i.
    \end{equation}
\end{corollary}
\index{permuter!somme et intégrale}

\begin{proof}
    Nous considérons la suite des sommes partielles de \( (u_n)\) : \( f_n(x)=\sum_{i=0}^nu_n(x)\). Le théorème de la convergence monotone (théorème \ref{ThoRRDooFUvEAN}) implique que
    \begin{equation}
        \lim_{n\to \infty} \int f_n=\int\lim_{n\to \infty} f_n.
    \end{equation}
    Nous remplaçons maintenant \( f_n\) par sa valeur en termes des \( u_i\) et dans le membre de gauche nous permutons l'intégrale avec la somme finie :
    \begin{equation}
        \lim_{n\to \infty} \sum_{i=0}^{\infty}\int u_n=\int\sum_{i=0}^{\infty}u_n,
    \end{equation}
    ce qu'il fallait démontrer.
\end{proof}

\begin{lemma}[Lemme de Fatou]\index{lemme!Fatou}\index{Fatou}   \label{LemFatouUOQqyk}
    Soit \( (\Omega,\tribA,\mu)\) un espace mesuré et \( f_n\colon \Omega\to \mathopen[ 0 , \infty \mathclose]  \) une suite de fonctions mesurables. Alors la fonction \( f(x)=\liminf f_n(x)\) est mesurable et
    \begin{equation}
        \int_{\Omega}\liminf f_nd\mu\leq\liminf\int_{\Omega}fd\mu.
    \end{equation}
\end{lemma}
%TODO : pour la mesurabilité, il faudra citer un théorème du genre de celui fait avec le sup.

\begin{proof}
    Nous posons 
    \begin{equation}
        g_n(x)=\inf_{i\geq n}f_i(x).
    \end{equation}
    Cela est une suite croissance de fonctions positives mesurables telles que, par définition, 
    \begin{equation}
        \lim_{n\to \infty}g_n(x)=\liminf f_n(x).
    \end{equation}
    Nous pouvons y appliquer le théorème de la convergence monotone,
    \begin{equation}
        \lim_{n\to \infty} \int g_n(x)=\int\liminf f_n(x).
    \end{equation}
    Par ailleurs, pour chaque \( i\geq n\) nous avons
    \begin{equation}
        \int g_n\leq \int f_i,
    \end{equation}
    en passant à l'infimum nous avons
    \begin{equation}
        \int g_n\leq \inf_{i\geq n}\int f_i,
    \end{equation}
    et en passant à la limite nous avons
    \begin{equation}
        \int\liminf f_n=\lim_{n\to \infty} \int g_n\leq \lim_{n\to \infty} \inf_{i\geq n}\int f_i=\liminf_{i\to\infty}\inf f_i.
    \end{equation}
\end{proof}

L'inégalité donnée dans ce lemme n'est en général pas une égalité, comme le montre l'exemple suivant :
\begin{equation}
    f_i=\begin{cases}
        \mtu_{\mathopen[ 0 , 1 \mathclose]}    &   \text{si \( i\) est pair}\\
        \mtu_{\mathopen[ 1 , 2 \mathclose]}    &    \text{si \( i\) est impair}.
    \end{cases}
\end{equation}
Nous avons évidemment \( g_n(x)=0\) tandis que \( \int_{\mathopen[ 0 , 2 \mathclose]}f_i=1\) pour tout \( i\).

%---------------------------------------------------------------------------------------------------------------------------
\subsection{Convergence dominée de Lebesgue}
%---------------------------------------------------------------------------------------------------------------------------

\begin{theorem}[Convergence dominée de Lebesgue]        \label{ThoConvDomLebVdhsTf}
    Soit \( (f_n)_{n\in\eN}\) une suite de fonctions intégrables sur \( (\Omega,\tribA,\mu)\) à valeurs dans \( \eC\) ou \( \eR\). Nous supposons que  \( f_n\to f\) simplement sur \( \Omega\) presque partout et qu'il existe une fonction intégrable \( g\) telle que
    \begin{equation}
        | f_n(x) |< g(x) 
    \end{equation}
    pour presque\footnote{Si il n'y avait pas le «presque» ici, ce théorème serait à peu près inutilisable en probabilité ou en théorie des espaces \( L^p\), comme dans la démonstration du théorème de Fischer-Riesz \ref{ThoGVmqOro} par exemple.} tout \( x\in\Omega\) et pour tout \( n\in \eN\). Alors
    \begin{enumerate}
        \item
            \( f\) est intégrable,
        \item
           $\lim_{n\to \infty} \int_{\Omega}f_n=\int_\Omega f$,
        \item
            $\lim_{n\to \infty} \int_{\Omega}| f_n-f |=0$.
    \end{enumerate}
\end{theorem}
\index{théorème!convergence!dominée de Lebesgue}
\index{dominée!convergence (Lebesgue)}
\index{permuter!limite et intégrale!convergence dominée}

\begin{proof}

    La fonction limite \( f\) est intégrable parce que \( | f |\leq g\) et \( g\) est intégrable (lemme \ref{LemPfHgal}). Par hypothèse nous avons
    \begin{equation}
        -g(x)\leq f_n(x)\leq g(x).
    \end{equation}
    En particulier la fonction \( g_n=f_n+g\) est positive et mesurable si bien que le lemme de Fatou (lemme \ref{LemFatouUOQqyk}) implique
    \begin{equation}
        \int_{\Omega}\liminf g_n\leq\liminf\int_{\Omega}g_n.
    \end{equation}
    Évidement nous avons \( \liminf g_n=f+g\), de telle sorte que
    \begin{equation}
        \int f+\int g\leq \liminf\int g_n=\liminf\int f_n+\int g,
    \end{equation}
    et le nombre \( \int g\) étant fini, nous pouvons le retrancher des deux côtés de l'inégalité :
    \begin{equation}
        \int f\leq\liminf\int f_n.
    \end{equation}
    Afin d'obtenir une minoration de \( \int f\) nous refaisons exactement le même raisonnement en utilisant la suite de fonctions \( k_n=-f_n\to k=-f\). Nous obtenons que
    \begin{equation}
        \int k\geq\liminf\int k_n=-\limsup\int f_n,
    \end{equation}
    et par conséquent
    \begin{equation}    \label{IneqsndMYTO}
        \liminf\int f_n\leq\int f\leq\limsup\int f_n.
    \end{equation}
    La limite supérieure étant plus grande ou égale à la limite inférieure, les trois quantités dans les inégalités \eqref{IneqsndMYTO} sont égales.

    Nous prouvons maintenant le troisième point. Soit la suite de fonctions
    \begin{equation}
        h_n(x)=| f_n(x)-f(x) |
    \end{equation}
    qui tend ponctuellement vers zéro. De plus
    \begin{equation}
    h_n(x)\leq | f_n(x) |+| f(x) |\leq 2g(x),
    \end{equation}
    ce qui prouve que les \( h_n\) majorés par une fonction intégrable. Donc
    \begin{equation}
        \lim_{n\to \infty} \int_{\Omega}| f_n-f |= \lim_{n\to \infty} \int_{\Omega}h_n(x)dx=\int_{\Omega}\lim_{n\to \infty} | f_n(x)-f(x) |=0
    \end{equation}
\end{proof}

\begin{remark}
    Lorsque nous travaillons sur des problèmes de probabilités, la fonction \( g\) peut être une constante parce que les constantes sont intégrables sur un espace de probabilité.
\end{remark}

\begin{corollary}       \label{CorCvAbsNormwEZdRc}
    Soit \( (a_i)_{i\in \eN}\) une suite numérique absolument convergente. Alors elle est convergente. Il en est de même pour les séries de fonctions si on considère la convergence ponctuelle.
\end{corollary}

\begin{proof}
    L'hypothèse est la convergence de l'intégrale \( \int_{\eN}| a_i |dm(i)\) où \( dm\) est la mesure de comptage. Étant donné que \( | a_i |\leq | a_i |\), la fonction \( a_i\) (fonction de \( i\)) peut jouer le rôle de \( g\) dans le théorème de la convergence dominée de Lebesgue (théorème \ref{ThoConvDomLebVdhsTf}).
\end{proof}
Nous utiliserons ce résultat pour montrer que la transformée de Fourier d'une fonction \( L^1(\eR^d)\) est continue (proposition \ref{PropJvNfj}).

\begin{proposition}[\cite{YHRSDGc}] \label{PropUXjnwLf}
    Approximation de fonctions mesurables par des fonctions étagées.
    \begin{enumerate}
        \item
            Une fonction mesurable et positive est limite (simple) d'une suite croissante de fonctions étagées, mesurables et positives.
        \item
            Si \( f\colon \eR^d\to \bar \eR\) est mesurable, alors elle est limite (simple) de fonctions étagées \( f_n\) telles que \( | f_n |\leq | f |\).
    \end{enumerate}
\end{proposition}
%TODO : la preuve est dans le document cité.

%+++++++++++++++++++++++++++++++++++++++++++++++++++++++++++++++++++++++++++++++++++++++++++++++++++++++++++++++++++++++++++ 
\section{Tribu produit, mesure produit}
%+++++++++++++++++++++++++++++++++++++++++++++++++++++++++++++++++++++++++++++++++++++++++++++++++++++++++++++++++++++++++++

\begin{definition}      \label{DefTribProfGfYTuR}
    Si \( \tribA_1\) et \( \tribA_2\) sont deux tribus sur deux ensembles \( \Omega_1\) et \( \Omega_2\), nous définissons la \defe{tribu produit}{tribu!produit} \( \tribA_1\otimes\tribA_2\) comme étant la tribu engendrée par 
    \begin{equation}
        \{ X\times Y\tq X\in\tribA_1,Y\in\tribA_2 \}.
    \end{equation}
    Ces ensembles sont appelés \defe{rectangles}{rectangle!produit de tribus} de \( (\Omega_1,\tribA_1)\otimes (\Omega_2,\tribA_2)\).
\end{definition}

\begin{proposition}[\cite{KEQWooJsCGiw}]
    Soient deux espaces mesurables \( (S_1,\tribF_1)\) et \( (S_2,\tribF_2)\). Si \( \tribC_i\) est une classe de parties de \( S_i\) avec \( \tribF_i=\sigma(\tribC_i)\) et \( S_i\in\tribC_i\). Alors 
    \begin{equation}
        \tribF_1\otimes \tribF_2=\sigma(\tribC_1\times \tribC_2).
    \end{equation}
\end{proposition}

\begin{proof}
    Nous notons \( p_1\) et \( p_2\) les projections de \( S_1\times S_2\) vers \( S_1\) et \( S_2\). Nous commençons par prouver que
    \begin{equation}    \label{eqSGPBooLpQHfq}
        \tribF_1\otimes \tribF_2=\sigma\big( \p_1^{-1}(\tribF_1)\cup p_2^{-1}(\tribF_2) \big).
    \end{equation}
    En effet cette union est dans \( \tribF_1\otimes \tribF_2\) parce que ce sont tous des produits de la forme \( A_1\times S_2\) et \( S_1\times A_2\) où \( A_i\in \tribF_i\). Inversement, tous les produits de la forme \( A_1\times A_2\) sont dans la tribu engendrée par l'union parce que
    \begin{equation}
        A_1\cup A_2=(A_1\times S_2)\cap(S_1\times A_2).
    \end{equation}
    Par conséquent, la partie \( p_1^{-1}(\tribF_1)\cup p_2^{-1}(\tribF_2)\) engendre tous les produits qui \href{ https://fr.wikisource.org/wiki/Bible_Crampon_1923/Matthieu }{ engendrent } la tribu \( \tribF_1\otimes\tribF_2\). L'égalité \eqref{eqSGPBooLpQHfq} est donc correcte.
    
    Si \( C_1\in\tribC_1\) alors
    \begin{equation}
        p_1^{-1}(C_1)=C_1\times S_2\in\tribC_1\times \tribC_2
    \end{equation}
    et donc \( p_1^{-1}(\tribC_1)\subset \tribC_1\times \tribC_2\). En utilisant le lemme de transfert \ref{LemOQTBooWGYuDU} nous avons alors
    \begin{equation}        \label{EqDQLYooVOLqMZ}
        p_1^{-1}(\tribF_1)=p_1^{-1}\big( \sigma(\tribC_1) \big)=\sigma\big( p_1^{-1}\tribC_1 \big)\subset\sigma(\tribC_1\times \tribC_1)
    \end{equation}
    et au bout de la même façon,
    \begin{equation}        \label{EqMTRCooVHNTHJ}
        p_2^{-1}(\tribF_1)\subset\sigma(\tribC_1\times \tribC_2).
    \end{equation}

    Vu les relations \eqref{EqDQLYooVOLqMZ}, \eqref{EqMTRCooVHNTHJ} et \eqref{eqSGPBooLpQHfq} nous avons
    \begin{equation}
        \tribF_1\otimes\tribF_2=\sigma\big( \p_1^{-1}(\tribF_1)\cup p_2^{-1}(\tribF_2) \big)\subset\sigma(\tribC_1\times \tribC_2).
    \end{equation}

    Réciproquement, si \( C_1\in \tribC_1\) et \( C_2\in \tribC_2\) alors
    \begin{equation}
        C_1\times C_2=(C_1\times S_1)\cap(S_1\times C_2)=p_1^{-1}(C_1)\cap p_2^{-1}(C_2)\in\tribF_1\otimes\tribF_2.
    \end{equation}
\end{proof}

\begin{lemma}[Propriété des sections\cite{NBoIEXO}] \label{LemAQmWEmN}
    Soient \( \tribA_1\) et \( \tribA_2\) des tribus sur les ensembles \( \Omega_1\) et \( \Omega_2\). Si \( A\in\tribA_1\otimes\tribA_2\) alors pour tout \( x\in \Omega_1\) et \( y\in\Omega_2\), les ensembles
    \begin{subequations}    \label{subEqCTtPccK}
        \begin{align}
            A_1(y)=\{ x\in\Omega_1\tq (x,y)\in A \}\\
            A_2(x)=\{ y\in\Omega_2\tq (x,y)\in A \}
        \end{align}
    \end{subequations}
    sont mesurables.
\end{lemma}
\index{section!propriété des}

\begin{proof}
    Soit \( y\in\Omega_2\); nous allons prouver le résultat pour \( A_1(y)\). Pour cela nous notons 
    \begin{equation}
        S=\{ A\in \tribA_1\otimes\tribA_2\tq \forall y\in\Omega_2, A_1(y)\in\tribA_1 \},
    \end{equation}
    et nous allons noter que \( S\) est une tribu contenant les rectangles. Par conséquent, \( S\) sera égal à \( \tribA_1\otimes \tribA_2\).

    \begin{subproof}
        \item[Les rectangles]

            Considérons le rectangle \( A=X\times Y\) et si \( y\in \Omega_2\) alors
            \begin{equation}
                A_1(y)=\{ x\in \Omega_1\tq (x,y)\in X\times Y \}.  
            \end{equation}
            Donc soit \( y\in Y\) alors \( A_1(y)=X\in\tribA_1\), soit \( y\notin Y\) et alors \( A_1(y)=\emptyset\in\tribA_1\).

        \item[Tribu : ensemble complet]

            Nous avons \( \Omega_1\times \Omega_2\in S\) parce que c'est un rectangle.

        \item[Tribu : complémentaire] Soit \( A\in S\) et montrons que \( A^c\in S\). Nous avons d'abord
            \begin{equation}
                (A^c)_1(y)=\{ x\in \Omega_1\tq (x,y)\in A^c \}.
            \end{equation}
            D'autre part
            \begin{equation}
                A_1(y)^c=\{ x\in\Omega_1\tq (x,y)\notin A \}=\{ x\in \Omega_1\tq (x,y)\in A^c \}=(A^c)_1(y).
            \end{equation}
            Vu que \( \tribA_1\) est une tribu et que par hypothèse \( A_1(y)\in\tribA_1\), nous avons aussi \( A_1(y)^c\in S\), et donc \( (A^c)_1(y)\in \tribA_1\), ce qui prouve que \( A^c\in S\).

        \item[Tribu : union dénombrable] Soit une suite \( A_n\in S\). Nous avons
            \begin{equation}
                (\bigcup_nA_n)_1(y)=\{ x\in\Omega_1\tq (x,y)\in \bigcup_nA_n \}=\bigcup_n\{ x\in\Omega_1\tq (x,y)\in A_n \}=\bigcup_n(A_n)_1(y),
            \end{equation}
            et ce dernier ensemble est dans \( \tribA_1\) parce que c'est une union dénombrable d'éléments de \( \tribA_1\).
        
    \end{subproof}
    Nous avons donc prouvé que \( S\) est une tribu contenant les rectangles, donc \( S\) contient au moins \( \tribA_1\otimes \tribA_2\).
\end{proof}

\begin{corollary}
    Si \( f\colon \Omega_1\times \Omega_2\to \eR\) est une fonction mesurable\footnote{Définition \ref{DefQKjDSeC}.} sur \( X\times Y\) alors pour chaque \( y\) dans \( \Omega_2\), la fonction
    \begin{equation}
        \begin{aligned}
            f_y\colon X&\to \eR \\
            x&\mapsto f(x,y) 
        \end{aligned}
    \end{equation}
    est mesurable.
\end{corollary}

\begin{proof}
    Soit \( \mO\) un ensemble mesurable de \( \eR\) (i.e. un borélien), et \( y\in \Omega_2\). Nous avons
    \begin{equation}
        f_y^{-1}(\mO)=\{ x\in X\tq f(x,y)\in \mO \}=A_1(y)
    \end{equation}
    où
    \begin{equation}
        A=\{ (x,y)\in \Omega_1\times \Omega_2\tq f(x,y)\in \mO \}=f^{-1}(\mO).
    \end{equation}
    Ce dernier est mesurable parce que \( f\) l'est.
\end{proof}

\begin{theorem}[\cite{NBoIEXO}\footnote{Modèle non contractuel : des notations et la définition de \( \lambda\)-système peuvent varier entre la référence et le présent texte.}]    \label{ThoCCIsLhO}
    Soient \( (\Omega_i,\tribA_i,\mu_i)\) (\( i=1,2\)) deux espaces mesurés \( \sigma\)-finie. Soit \( A\in\tribA_1\otimes \tribA_2\). Alors les fonctions\footnote{Voir la notation du lemme \ref{subEqCTtPccK}.}
    \begin{subequations}
        \begin{align}
            x\mapsto\mu_2\big( A_2(x) \big)\\
            y\mapsto\mu_1\big( A_1(y) \big)
        \end{align}
    \end{subequations}
    sont mesurables et
    \begin{equation}    \label{EqRKXwsQJ}
        \int_{\Omega_1}\mu_2\big( A_2(x) \big)d\mu_1(x)=\int_{\Omega_2}\mu_2\big( A_1(y) \big)d\mu_2(y).
    \end{equation}
\end{theorem}

\begin{proof}
    Nous supposons d'abord que \( \mu_1\) et \( \mu_2\) sont finies et nous notons \( \tribD\) le sous-ensemble de \( \tribA_1\otimes \tribA_2\) sur lequel le théorème est correct. Nous allons commencer par prouver que \( \tribD\) est un \( \lambda\)-système.

    \begin{subproof}
        \item[\( \lambda\)-système : différence ensembliste]
            Soient \( A,B\in\tribD\) avec \( A\subset B\). Nous avons
            \begin{subequations}
                \begin{align}
                    (B\setminus A)_1(y)&=\{ x\in \Omega_1\tq(x,y)\in B\setminus A \}\\
                    &=\{ x\in \Omega_1\tq(x,y)\in B\}\setminus\{ x\in \Omega_1\tq(x,y)\in  A \}\\
                    &=B_1(y)\setminus A_1(y).
                \end{align}
            \end{subequations}
            Vu que \( A_1(y)\subset B_1(y)\) et que les mesure sont finies le lemme \ref{LemPMprYuC} nous donne
            \begin{equation}
                \mu_1\big( (B\setminus A)_1(y) \big)=\mu_1\big( B_1(y) \big)-\mu_1\big( A_1(y) \big),
            \end{equation}
            et similairement pour \( 1\leftrightarrow 2\). Les deux fonctions (de \( y\)) à droite étant mesurables, nous avons la mesurabilité de la fonction \( y\mapsto \mu_1\big( (B\setminus A)_1(y) \big)\).

            Prouvons la formule intégrale en nous rappelant que la formule \eqref{EqRKXwsQJ} est supposée correcte pour \( A\) et \( B\) séparément :
            \begin{subequations}
                \begin{align}
                    \int_{\Omega_2}\mu_1\big( (B\setminus A)_1(y) \big)d\mu_2(y)&=\int_{\Omega_2}\mu_1\big( B_1(y) \big)d\mu_2(y)-\int_{\Omega_2}\mu_1\big( A_1(y) \big)d\mu_2(y)\\
                    &=\int_{\Omega_1}\mu_2\big( B_2(x) \big)d\mu_1(x)-\int_{\Omega_1}\mu_2\big( A_2(x) \big)d\mu_1(x)\\
                    &=\int_{\Omega_1}\mu_2\big( (B\setminus A)_2(x) \big)d\mu_1(x).
                \end{align}
            \end{subequations}
            
    
        \item[\( \lambda\)-système : limite de suite croissante]

            Soit \( (A_n)\) une suite croissante dans \( \tribD\); nous posons \( B_n=A_n\setminus A_{n-1}\) et \( A_0=\emptyset\) de telle sorte à travailler avec une suite d'ensembles disjoints qui satisfait \( \bigcup_nA_n=\bigcup_nB_n\). Vu que la suite est croissante nous avons \( A_{n-1}\subset A_n\) et donc \( B_n\in\tribD\) par le point déjà fait sur la différence ensembliste. Nous avons :
            \begin{subequations}
                \begin{align}
                    \mu_1\big( (\bigcup_nB_n)_1(y) \big)&=\{ x\in \Omega_1\tq (x,y)\in\bigcup_nB_n \}\\
                    &=\bigcup_n\{ x\in\Omega_1\tq (x,y)\in B_n \}\\
                    &=\bigcup_n (B_n)_1(y).
                \end{align}
            \end{subequations}
            Par conséquent, par la propriété \ref{ItemQFjtOjXiii} d'une mesure nous avons
            \begin{equation}
                \mu_1\big( (\bigcup_nB_n)_1(y) \big)=\sum_n\mu_1\big( (B_n)_1(y) \big).
            \end{equation}
            En tant que somme de fonctions positives et mesurables, la fonction
            \begin{equation}
                y\mapsto\sum_n\mu_1\big( (B_n)_1(y) \big)
            \end{equation}
            est mesurable par la proposition \ref{PropFYPEOIJ}. Il faut encore vérifier la formule intégrale. Le gros du boulot est de permuter une somme et une intégrale par le corollaire \ref{CorNKXwhdz} :
            \begin{subequations}
                \begin{align}
                    \int_{\Omega_2}\sum_n\mu_1\big( (B_n)_1(y) \big)d\mu_2(y)&=\sum_n\int_{\Omega_2}\mu_1\big( (B_n)_1(y) \big)d\mu_2(y)\\
                    &=\sum_n\int_{\Omega_1}\mu_2\big( (B_n)_2(x) \big)d\mu_1(x)\\
                    &=\int_{\Omega_1}\sum_n\mu_2\big( (B_n)_2(x) \big)d\mu_1(x)\\
                    &=\int_{\Omega_1}\mu_2\big( (\bigcup_nB_n)_1(y) \big)d\mu_1(x).
                \end{align}
            \end{subequations}
    \end{subproof}
    Maintenant que \( \tribD\) est un $\lambda$-système contenant les rectangles, le lemme \ref{LemLUmopaZ} dit que la tribu engendrée par \( \tribD\) (c'est à dire \( \tribA_1\otimes \tribA_2\)) est le $\lambda$-système \( \tribD\) lui-même.

    La preuve est finie dans le cas de mesures finies. Nous commençons maintenant à prouver dans le cas où les mesures \( \mu_1\) et \( \mu_2\) sont seulement \( \sigma\)-finies. Nous considérons des suites croissantes \( \Omega_{i,n}\to\Omega_i\) d'ensembles mesurables et de mesure finie : \( \mu_i(\Omega_{i,n})<\infty\). D'abord remarquons que
    \begin{equation}\label{EqNFuBzBF}
        \mu_2\Big( (A\cap \Omega_{1,j}\times E_{2,j})_2(x) \Big)=\mu_2\Big( A_2(x)\cap \Omega_{2,j} \Big)\mtu_{\Omega_{1,j}}.
    \end{equation}
    En effet,
    \begin{subequations}
        \begin{align}
            \heartsuit&=(A\cap\Omega_{1,j}\times E_{2,j})_2(x)\\
            &=\{ y\in\Omega_2\tq (x,y)\in A\cap \Omega_{1,j}\times E_{2,j} \}\\
            &=\{ y\in \Omega_2\tq (x,y)\in A\times \Omega_{2,j} \}\cap\{ y\in\Omega_2\tq (x,y)\in \Omega_{1,j}\times \Omega_{2,j} \}.
        \end{align}
    \end{subequations}
    Si \( y\in \Omega_{1,j}\) alors \( \{ y\in \Omega_2\tq (x,y)\in \Omega_{1,j}\times \Omega_{2,j} \}=\Omega_{2,j}\) et dans ce cas
    \begin{equation}
        \heartsuit=\{ y\in \Omega_2\tq (x,y)\in A\times \Omega_{2,j} \}\cap \Omega_{2,j}=A_2(x)\cap E_{2,j}.
    \end{equation}
    Et inversement, si \( x\notin \Omega_{1,j}\) alors \( \heartsuit=\emptyset\). Dans les deux cas nous avons \eqref{EqNFuBzBF}.

    Les ensembles \( A\cap \Omega_{1,j}\times \Omega_{2,j}\) étant de mesure finie, nous pouvons leur appliquer la première partie :
    \begin{equation}
        \int_{\Omega_1}\mu_2\Big( (A\cap\Omega_{1,j}\times \Omega_{2,j})_2(x) \Big)d\mu_1(x)=\int_{\Omega_2}\mu_1\Big( (A\cap\Omega_{1,j}\times \Omega_{2,j})_1(y) \Big)d\mu_2(u),
    \end{equation}
    ou encore
    \begin{equation}
        \int_{\Omega_1}\mu_2\Big( A_2(x)\cap \Omega_{2,j} \Big)\mtu_{\Omega_{1,j}}(x)d\mu_1(x)=\int_{\Omega_2}\mu_1\Big( A_1(y)\cap \Omega_{1,j} \Big)\mtu_{\Omega_{2,j}}(y)d\mu_2(y).
    \end{equation}
    Ce que nous avons dans ces intégrales sont (par rapport à \( j\)) des suites croissantes de fonction positives; nous pouvons donc permuter une limite et une intégrale. En sachant que si \( k\to \infty\), alors
    \begin{subequations}
        \begin{align}
            \mtu_{1,j}(x)\to 1\\
            \mu_2\big( A_2(x)\cap \Omega_2,j \big)\to\mu_2\big( A_2(x) \big),
        \end{align}
    \end{subequations}
    nous trouvons le résultat demandé.
\end{proof}

\begin{theorem}[\cite{FubiniBMauray,MesIntProbb}]   \label{ThoWWAjXzi}
    Soient \( \mu_i\) des mesures $\sigma$-finies sur \( (\Omega_i,\tribA_i)\) (\( i=1,2\)). 
    \begin{enumerate}
        \item
            
    Il existe une et une seule mesure, notée \( \mu_1\otimes \mu_2\), sur \( (\Omega_1\times\Omega_2,\tribA_1\otimes\tribA_2)\) telle que
    \begin{equation}    \label{EqOIuWLQU}
        (\mu_1\otimes\mu_2)(A_1\times A_2)=\mu_1(A_1)\mu_2(A_2)
    \end{equation}
    pour tout \( A_1\in \tribA_1\) et \( A_2\in\tribA_2\). 
\item
    Cette mesure est donnée par la formule\footnote{Voir les notations du lemme \ref{LemAQmWEmN}.}
    \begin{equation}   \label{EqDFxuGtH}
        (\mu_1\otimes \mu_2)(A)=\int_{\Omega_1}\mu_2\big( A_2(x) \big)d\mu_1(x)=\int_{\Omega_2}\mu_1\big( A_1(y) \big)d\mu_2(y).
    \end{equation}
    Cette mesure est la \defe{mesure produit}{mesure!produit} de \( \mu_1\) par \( \mu_2\).
\item
    La mesure \( \mu_1\otimes \mu_2\) ainsi définie est \( \sigma\)-finie.
    \end{enumerate}
\end{theorem}
\index{mesure!produit}

\begin{proof}
    La partie «existence» sera divisée en deux parties : l'une pour prouver que les formules \eqref{EqDFxuGtH} donnent une mesure et une pour montrer que cette mesure vérifie la condition \eqref{EqOIuWLQU}.
    \begin{subproof}
    \item[Unicité]
        
    L'ensemble des rectangles de \( \Omega_1\times \Omega_2\) engendre la tribu \( \tribA_1\otimes\tribA_2\), est fermé par intersection et contient une suite croissante d'ensembles \( P_n\times R_n\) de mesure finie (\( \mu(P_n\times R_n)<\infty\)) telle que \( P_n\times R_n\to \Omega_1\times \Omega_2\). Cette suite est donné par le fait que \( \mu_1\) et \( \mu_2\) sont \( \sigma\)-finies. En effet si \( (X_n)\) et \( (Y_n)\) sont des recouvrements dénombrables de \( \Omega_1\) et \( \Omega_2\) par des ensembles de mesure finie, en posant \( P_n=\bigcup_{k=1}^nX_n\) et \( R_n=\bigcup_{k=1}^nY_n\) nous avons bien une suite croissante de rectangles qui tendent vers \( \Omega_1\times \Omega_2\). Avec ces rectangles en main, le théorème \ref{ThoJDYlsXu} donne l'unicité.

\item[Les formules définissent une mesure]
    Le théorème \ref{ThoCCIsLhO} dit que ces formules ont un sens et que l'égalité entre les deux intégrales est correcte. Nous prouvons à présent qu'elles déterminent effectivement une mesure sur \( (\Omega_1\times\Omega_2,\tribA_1\otimes \tribA_2)\).

    Pour tout \( A\in \tribA_1\otimes \tribA_2\), \( \mu(A)\geq 0\) parce que \( \mu\) est donnée par l'intégrale d'une fonction positive.

    En ce qui concerne la condition d'unions dénombrable disjointe, soient \( A^{i}\) des éléments disjoints de \( \tribA_1\otimes \tribA_2\); nous commençons par remarquer que
    \begin{subequations}
        \begin{align}
            \left( \bigcup_{i=1}^{\infty}A^{(i)} \right)_2(x)&=\{ y\in\Omega_2\tq (x,y)\in\bigcup_{i=1}^{\infty}A^{(i)} \}\\
            &=\bigcup_{i=1}^{\infty}\{ y\in\Omega_2\tq (x,y)\in A^{(i)} \}\\
            &=\bigcup_{i=1}^{\infty}A^{(i)}_2(x).
        \end{align}
    \end{subequations}
    Par conséquent,
    \begin{subequations}
        \begin{align}
            \mu\left( \bigcup_{i=1}^{\infty}A^{(i)} \right)&=\int_{\Omega_1}\mu_2\left(    \Big( \bigcup_{i=1}^{\infty}A^{(i)} \Big)_2(x)     \right)d\mu_1(x)\\
            &=\int_{\Omega_1}\sum_{i=1}^{\infty}\mu_2\big( A^{(i)}_2(x) \big)d\mu_1(x)\\
            &=\int_{\Omega_1}\lim_{n\to \infty} \sum_{i=1}^{n}\mu_2\big( A^{(i)}_2(x) \big)d\mu_1(x).
        \end{align}
    \end{subequations}
    où nous avons utilisé l'additivité de la mesure \( \mu_2\). À ce niveau, il serait commode de permuter la somme et l'intégrale. Pour ce faire nous considérons la suite (croissante) de fonctions
    \begin{equation}
        f_n(x)=\sum_{i=1}^n\mu_2\big( A_2^{(i)}(x) \big).
    \end{equation}
    Nous pouvons permuter la limite et l'intégrale grâce au théorème de la convergence monotone \ref{ThoRRDooFUvEAN}; ensuite la somme se permute avec l'intégrale en tant que somme finie :
    \begin{subequations}
        \begin{align}
            \mu\left( \bigcup_{i=1}^{\infty}A^{(i)} \right)&=\lim_{n\to \infty} \sum_{i=1}^n\int_{\Omega_1}\big( A_2^{(i)}(x) \big)d\mu_1(x)\\
            &=\lim_{n\to \infty} \sum_{i=1}^n\mu(A^{(i)})\\
            &=\sum_{i=1}^{\infty}\mu( A^{(i)} ).
        \end{align}
    \end{subequations}

\item[Elles vérifient la condition]
    Prouvons que les formules \eqref{EqDFxuGtH} se réduisent à \eqref{EqOIuWLQU} dans le cas des rectangles. Soit donc \( A=X_1\times X_2\) avec \( X_i\in\tribA_i\). Alors
    \begin{equation}
        A_1(y)=\{ x\in\Omega_1\tq (x,y)\in X_1\times X_2 \}
    \end{equation}
    et
    \begin{equation}
        \mu_1\big( A_1(y) \big)=\mtu_{X_2}(y)\mu_1(X_1),
    \end{equation}
    donc
    \begin{subequations}
        \begin{align}
            (\mu_1\otimes\mu_2)(A)&=\int_{\Omega_2}\mu_1\big( A_1(y) \big)d\mu_2(y)\\
            &=\int_{\Omega_2}\mu_1(X_1)\mtu_{X_2}(y)d\mu_2(y)\\
            &=\mu_1(X_1)\int_{\Omega_2}\mtu_{X_2}(y)d\mu_2(y)\\
            &=\mu_1(X_1)\mu_2(X_2).
        \end{align}
    \end{subequations}
    Pour cela nous avons utilisé le fait que l'intégrale de la fonction caractéristique d'un ensemble mesurable est la mesure de cet ensemble.
    \end{subproof}
\end{proof}

\begin{definition}[Produit d'espaces mesurés]  \label{DefUMlBCAO}
    Si \( (\Omega_i,\tribA_i,\mu_i)\) sont deux espaces mesurés, l'\defe{espace produit}{produit!espaces mesurés} est l'ensemble \( \Omega_1\times \Omega_2\) muni de la tribu produit \( \tribA_1\otimes \tribA_2\) de la définition \ref{DefTribProfGfYTuR} et de la mesure produit \( \mu_1\otimes \mu_2\) définie par le théorème \ref{ThoWWAjXzi}.
\end{definition}

\begin{remark}
    Il n'est pas garantit que la tribu \( \tribA_1\otimes\tribA_2\) soit la tribu la plus adaptée à l'ensemble \( S_1\times S_2\). Dans le cas de \( \eR^N\), il se fait que c'est le cas : en prenant des produits des boréliens sur \( \eR\) on obtient bien les boréliens sur \( \eR^N\), voir proposition \ref{PropWOOOooHcoEEF}.
\end{remark}

%+++++++++++++++++++++++++++++++++++++++++++++++++++++++++++++++++++++++++++++++++++++++++++++++++++++++++++++++++++++++++++
\section{Mesure de Lebesgue sur \texorpdfstring{$ \eR$}{R}}
%+++++++++++++++++++++++++++++++++++++++++++++++++++++++++++++++++++++++++++++++++++++++++++++++++++++++++++++++++++++++++++
\label{SecZTFooXlkwk}

Nous notons \( \mS\) l'ensemble des intervalles\footnote{Définition \ref{DefEYAooMYYTz}.} de \( \eR\).

\begin{proposition}
    L'ensemble réunions finies d'éléments de \( \mS\) est une algèbre de parties de \( \eR\).
\end{proposition}
Cette algèbre de parties de $\eR$ est notée \( \tribA_{\mS}\).

\begin{proof}

    Nous devons vérifier la définition \ref{DefTCUoogGDud}. Les ensembles \( \eR\) et \( \emptyset\) sont des intervalles et font donc partie de \( \tribA_{\mS}\).
    
    Si \( A\in\tribA_{\mS}\) se décompose en union d'intervalles de la forme \( (a_k,b_k)\) avec \( k=1,\ldots, n\) (ici nous mettons des parenthèses au lieu de crochets parce qu'a priori nous ne savons pas). Alors
    \begin{equation}
        A^c=\bigcup_{k=0}^{k}(b_k,a_{k+1})
    \end{equation}
    où nous avons posé \( b_0=-\infty\) et \( a_{n+1}=+\infty\). Ici encore les parenthèses sont soit fermées soit ouvertes en fonction de ce qu'étaient celles dans la décomposition de \( A\). Quoi qu'il en soit, cette décomposition de \( A^c\) montre que \( A^c\in\tribA_{\mS}\).

    Enfin si \( A,B\in\tribA_{\mS}\) alors \( A\cup B\in\tribA_{\mS}\).
\end{proof}

\begin{lemma}
    Tout élément de \( \tribA_{\mS}\) admet une décomposition minimale unique en réunion finie d'intervalles. Cette décomposition est formée d'intervalles deux à deux disjoints.
\end{lemma}

\begin{proof}
    Nous allons montrer que si \( A\in\tribA_{\mS}\), alors la décomposition minimale consiste en les composantes connexes de \( A\). Pour cela nous rappelons que la proposition \ref{PropInterssiConn} dit qu'une partie de \( \eR\) est connexe si et seulement si elle est un intervalle. D'abord cela nous dit immédiatement que les composantes connexes de \( A\) forment une décomposition de \( A\) en intervalles. Nous devons prouver qu'elle est minimale.

    Soit \( \{ C_k \}_{k=1,\ldots, n}\) les composantes connexes de \( A\). Aucun connexe de \( \eR\) contenu dans \( A\) ne peut intersecter plus d'un des \( C_k\), et par conséquent nous ne pouvons pas décomposer \( A\) en moins de \( n\) intervalles. 
    
    Pour l'unicité, soit \( \{ I_k \}_{k=1,\ldots, n}\) un ensemble de \( n\) intervalles tels que \( \bigcup_{k=1}^nI_k=A\). Chacun des \( I_k\) intersecte un et un seul des \( C_k\). En effet si \( x\in I_k\cap C_i\) et \( y\in I_k\cap C_j\), alors \( \mathopen[ x , y \mathclose]\subset I_k\) parce que \( I_k\) est un intervalle. Mais \( C_i\) étant le plus grand connexe contenant \( x\), \( \mathopen[ x , y \mathclose]\subset C_i\) et de la même façon, \( \mathopen[ x , y \mathclose]\subset C_j\). Par conséquent \( C_i\) et \( C_j\) sont tous deux la composante connexe de \( x\) et \( y\). Nous en déduisons que \( C_i=C_j\), c'est à dire \( i=j\).

    Par ailleurs nous avons \( I_k\cap I_l=\emptyset\) dès que \( k\neq l\) parce que sinon l'ensemble \( I_k\cap I_l\) serait connexe et la décomposition des \( \{ I_k \}_{k=1,\ldots, n} \) ne serait pas minimale : en remplaçant \( I_k\) et \( I_l\) par \( I_k\cup I_l\) on aurait eu une décomposition contenant moins d'éléments. Donc à renumérotation près nous pouvons supposer que \( I_k\) intersecte \( C_l\) si et seulement si \( k=l\).

    Dans ce cas nous devons avoir \( I_k=C_k\), sinon les éléments de \( C_k\setminus I_k\) ne seraient pas dans \( \bigcup_{i=1}^nI_i\).
\end{proof}

\begin{definition}[longueur d'intervalle\cite{MesureLebesgueLi}]
    Si \( I\) est un intervalle d'extrémités \( a\) et \( b\) avec \( -\infty\leq a\leq b\leq +\infty\) alors nous définissons la \defe{longueur}{longueur!d'un intervalle}\index{intervalle!longueur} de \( I\) par
    \begin{equation}
        \ell(I)=\begin{cases}
            b-a    &   \text{si \( -\infty<a\leq b< +\infty\)}\\
            \infty    &    \text{si \( a\) ou \( b\) est infini}
        \end{cases}
    \end{equation}
    Si \( A\in\tribA_{\mS}\) et si sa décomposition minimale est \( A=\bigcup_{k=1}^nI_k\), alors on définit
    \begin{equation}
        \ell(A)=\sum_{k=1}^n\ell(I_k).
    \end{equation}
\end{definition}

Le lemme suivant nous indique que nous pouvons calculer la longueur d'un élément de \( \tribA_{\mS}\) sans savoir la décomposition minimale, pourvu que l'on connaisse une décomposition disjointe.
\begin{lemma}[\cite{MesureLebesgueLi}]\label{LemIUQooEzHun}
    Si
    \begin{equation}
        B=\bigcup_{r=1}^pJ_r
    \end{equation}
    est une décomposition de \( B\in\tribA_{\mS}\) en intervalles deux à deux disjoints alors
    \begin{equation}
        \ell(B)=\sum_{r=1}^p\ell(J_r).
    \end{equation}
\end{lemma}

\begin{proof}
    Nous prouvons dans un premier temps le résultat dans le cas où \( B=I\) est un intervalle. Soit \( I\) un intervalle et une décomposition en intervalles disjoints \( I=\bigcup_{r=1}^pJ_r\). Nous montrons qu'alors \( \ell(I)=\sum_{r=1}^p\ell(J_r)\). Nous verrons ensuite comment passer au cas où \( B\) est un élément générique de \( \tribA_{\mS}\).
    \begin{subproof}
    \item[Si \( B=I\) est un intervalle infini]

        Si \( I\) est infini alors un des \( J_r\) soit l'être et donc \( \sum_{r=1}^p\ell(J_r)=\infty=\ell(I)\).
    \item[Si \( B=I\) est un intervalle ininfini]

    Pour chaque \( r=1,\ldots, p\) nous notons \( a_r\) et \( b_r\) les extrémités de \( J_r\). Vu que les \( J_r\) sont connexes et disjoints, si \( a_k\leq a_l\) alors \( b_k\leq a_l\), sinon l'ensemble (non vide) \( \mathopen] a_l , b_k \mathclose[ \) serait dans l'intersection \( I_k\cap I_l\) qui, elle, est vide. Plus généralement, si \( x\in J_k\) et \( y\in J_l\) avec \( x<y\) alors pour tout \( x'\in J_k\) et tout \( y'\in J_l\) nous avons \( x'<y'\). Vu qu'il y a un nombre fini d'ensembles \( J_r\), nous pouvons les classer dans l'ordre croissant :
        \begin{equation}
            a_1\leq b_1\leq a_2\leq b_2\leq \ldots\leq b_{p-1}\leq a_p\leq b_p.
        \end{equation}
        Vu que les \( J_r\) sont disjoints et que leur union est connexe nous avons en réalité
        \begin{equation}
            a=a_1\leq b_1=a_2\leq b_2=a_3\leq\ldots\leq b_{p-1}= a_p\leq b_p,
        \end{equation}
        donc une somme télescopique donne
        \begin{equation}
            \ell(I)=b-a=\sum_{r=1}^p(b_r-a_r)=\sum_{r=1}^p\ell(J_r).
        \end{equation}

    \item[Si \( B\) n'est pas un intervalle]
        Soit \( \{ I_k \}_{k=1,\ldots, n}\) la décomposition minimale de \( B\). Alors
        \begin{equation}
            \spadesuit=\ell(B)=\sum_{k=1}^n\ell(I_k)=\sum_{k=1}^n\ell\big( \bigcup_{r=1}^p(I_k\cap J_r) \big).
        \end{equation}
        Mais \( I_k\) est un intervalle et s'écrit comme union disjointe \( I_k=\bigcup_{r=1}^p(I_k\cap J_r)\), donc par la première partie
        \begin{equation}
            \spadesuit=\sum_{k=1}^n\sum_{r=1}^p\ell(I_k\cap J_r)=\sum_{r=1}^p\sum_{k=1}^n\ell(I_k\cap J_r).
        \end{equation}
        Ici \( J_r\) est un intervalle qui se décompose en \( J_r=\bigcup_{k=1}^n(I_k\cap J_r)\), donc nous pouvons encore utiliser la première partie :
        \begin{equation}
            \spadesuit=\sum_{r=1}^p\ell(J_r),
        \end{equation}
        ce qu'il fallait.
    \end{subproof}
\end{proof}

\begin{lemma}   \label{LemPIOooRLkbo}
    Si \( A,B\in\tribA_{\mS}\) avec \( A\subset B\) alors \( \ell(A)\leq \ell(B)\).
\end{lemma}

\begin{proof}
    Nous avons évidemment \( B=A\cup B\setminus A\). Notons que \( B\setminus A\in\tribA_{\mS}\) par le lemme \ref{LemBFKootqXKl}. Si \( \{ I_k \}\) est une décomposition disjointe de \( A\) et \( \{ J_i \}\) une de \( B\setminus A\) alors \( \{ I_k \}\cup\{ J_i \}\) est une décomposition disjointe de \( A\cup B\setminus A\) et le lemme \ref{LemIUQooEzHun} nous dit que
    \begin{equation}
        \ell(B)=\ell(A\cup B\setminus A)=\ell(A)+\ell(B\setminus A).
    \end{equation}
    Par conséquent \( \ell(B)\geq \ell(A)\).
\end{proof}

\begin{lemma}   \label{LemUMVooZJgMu}
    Si \( I\) est un intervalle et si il se décompose en
    \begin{equation}
        I=\bigcup_{n\in \eN}I_n
    \end{equation}
    où les \( I_n\) sont des intervalles disjoints, alors
    \begin{equation}
        \ell(I)=\sum_{n=1}^{\infty}\ell(I_n).
    \end{equation}
\end{lemma}

\begin{proof}
    Nous allons encore diviser la preuve en deux parties suivant que \( I\) soit de longueur finie ou pas.   
    \begin{subproof}

        \item[Si \( I\) est de longueur finie]
        
            Soient \( a\) et \( b\) les extrémités de \( I\) : \( -\infty<a\leq b< +\infty\). Pour tout \( N\geq 1\) nous avons
            \begin{equation}
                \sum_{n=1}^N\ell(I_n)=\ell\big( \bigcup_{n=1}^nI_n \big)\leq \ell(I).
            \end{equation}
            La première égalité est le lemme dans le cas d'une union finie \ref{LemIUQooEzHun}. L'inégalité est le lemme \ref{LemPIOooRLkbo}. Cela étant vrai pour tout $N$, à la limite \( N\to\infty\) nous conservons l'inégalité :
            \begin{equation}
                \sum_{n=1}^{\infty}\ell(I_n)\leq \ell(I).
            \end{equation}
            Nous devons encore voir l'inégalité inverse. Pour cela nous supposons que \( a<b\). Sinon \( \ell(I)=0\) et tous les \( I_n\) doivent être vide sauf un qui contiendra seulement \( \{ a \}\) (si \( I\) le contient).

            Soit \( \epsilon>0\) avec \( \epsilon<b-a\) et l'intervalle
            \begin{equation}
                \mathopen[ a+\frac{ \epsilon }{ 4 } , b-\frac{ \epsilon }{ 4 } \mathclose]=\mathopen[ a' , b' \mathclose]\subset I.
            \end{equation}
            Si les \( a_n\) et le \( b_n\) sont le extrémités des \( I_n\) alors
            \begin{equation}
                \mathopen[ a' , b' \mathclose]\subset I=\bigcup_{n\geq 1}I_n\subset\bigcup_{n\geq 1}\mathopen] a_n-\frac{ \epsilon }{ 2^{n+2} } , b_n+\frac{ \epsilon }{ 2^{n+2} } \mathclose[=\bigcup_{n\geq 1}\mathopen] a'_n , b'_n \mathclose[
            \end{equation}
            où nous avons posé \( a'_n=a_n-\epsilon/2^{n+2}\) et \( b'_n=b_n+\epsilon/2^{n+2}\). Nous avons donc recouvert le compact\footnote{Lemme \ref{LemOACGWxV}.} \( \mathopen[ a' , b' \mathclose]\) par des ouverts. Nous pouvons donc en extraire un sous-recouvrement fini (c'est la définition de la compacité), c'est à dire une partie finie \( F\) de \( \eN\) telle que 
            \begin{equation}
                \mathopen[ a' , b' \mathclose]\subset \bigcup_{n\in F}\mathopen] a'_n , b'_n \mathclose[.
            \end{equation}
            Le lemme \ref{LemPIOooRLkbo} nous dit alors que
            \begin{equation}
                \heartsuit=b'-a'\leq \ell\big( \bigcup_{n\in F}\mathopen] a'_n , b'_n \mathclose[ \big)\leq \sum_{n\in F}(b'_n-a'_n).
            \end{equation}
            La seconde inégalité se prouve en recopiant\footnote{Nous ne pouvons pas invoquer directement le lemme \ref{LemZQUooMdCpq} parce que nous n'avons pas encore prouvé que \( \ell\) était une mesure sur \( (\eR,\tribA_{\mS})\).} la preuve de \ref{LemZQUooMdCpq}. Nous continuons le calcul :
            \begin{equation}
                \heartsuit\leq\sum_{n\in F}(b_n-a_n)+\sum_{n\in F}\frac{ \epsilon }{ 2^{n+1} }\leq \sum_{n\in F}(b_n-a_n)+\frac{ \epsilon }{2}.
            \end{equation}
            Mais \( b'-a'=(b-a)-\frac{ \epsilon }{2}\), donc
            \begin{equation}
                b-a-\frac{ \epsilon }{2}\leq \sum_{n\in F}(b_n-a_n)+\frac{ \epsilon }{2}.
            \end{equation}
            D'où nous déduisons que
            \begin{equation}
                \ell(I)=b-a\leq \sum_{n\in F}(b_n-a_n)+\epsilon\leq \sum_{n\in \eN}(b_n-a_n)+\epsilon=\sum_{n\in \eN}\ell(I_n)+\epsilon.
            \end{equation}
            Cela étant valable pour tout \( \epsilon\) nous déduisons que
            \begin{equation}
                \ell(I)\leq\sum_{n\in \eN}\ell(I_n).
            \end{equation}

        \item[Si \( I\) est de longueur infinie]

        Étant donné que \( I\) est un intervalle de longueur infinie, il contient au moins un ensemble du type \( \mathopen] -\infty , a \mathclose]\) ou \( \mathopen[ a , +\infty [\); donc  pour tout \( M>0\), il existe \( N\geq 1\) tel que
            \begin{equation}
                \ell\big( I\cap\mathopen[ -N , N \mathclose] \big)\geq M.
            \end{equation}
            Mais \( I\cap\mathopen[ -N , N \mathclose]\) est un intervalle et 
            \begin{equation}
                I\cap\mathopen[ -N , N \mathclose]=\bigcup_{n\in \eN}I_n\cap\mathopen[ -N , N \mathclose]
            \end{equation}
            qui est une union disjointe. Par conséquent,
            \begin{equation}
                M\leq \ell\big( I\cap\mathopen[ -N , N \mathclose] \big)=\sum_n\ell\big( I_n\cap\mathopen[ -N , N \mathclose] \big)\leq\sum_n\ell(I_n).
            \end{equation}
            Cela étant vrai pour tout \( M>0\), nous concluons que
            \begin{equation}
                \sum_{n\in \eN}\ell(I_n)=\infty.
            \end{equation}
    \end{subproof}
\end{proof}

\begin{remark}
    Pour la preuve de \ref{LemUMVooZJgMu} nous ne pouvons pas classer les \( I_n\) en ordre croissant comme nous l'avons fait dans la preuve de \ref{LemIUQooEzHun}. En effet si \( I=\mathopen[ 0 , 1 \mathclose]\) et que nous recouvrons \( \mathopen[ 0 , \frac{ 1 }{2} [\) et \( \mathopen] \frac{ 1 }{2} , 1 \mathclose]\) par une infinité d'intervalles chacun, nous ne pouvons plus les classer par ordre croissant.
\end{remark}

\begin{proposition}[\cite{MesureLebesgueLi}]     \label{PropULFoodgXrR}
    La fonction \( \ell\) ainsi définie est une mesure \( \sigma\)-finie sur l'algèbre de parties \( \tribA_{\mS}\).
\end{proposition}

\begin{proof}
    Le fait que \( \ell\) soit \( \sigma\)-finie provient par exemple du fait que \( \ell\big( \mathopen] -n , n \mathclose[ \big)=2n\) tandis que \( \bigcup_n\mathopen] -n , n \mathclose[=\eR\).

        Nous devons à présent prouver que \( \ell\) est additive. Soient \( (A_i)_{i\in \eN}\) des éléments disjoints de \( \tribA_{\mS}\), avec leurs décomposition minimales
            \begin{equation}
                A_i=\bigcup_{k=1}^nI^{(i)}_k.
            \end{equation}
            Pour chaque \( i\in \eN\), le lemme \ref{LemUMVooZJgMu} nous indique que
            \begin{equation}
                \ell(A_i)=\sum_{k\in \eN}\ell(I^{(i)}_k).
            \end{equation}
            L'ensemble \( \eN\times \eN\) est dénombrable et nous pouvons considérer la décomposition 
            \begin{equation}
                \bigcup_{i\in \eN}A_i=\bigcup_{(i,k)\in \eN\times \eN}I^{(i)}_k.
            \end{equation}
            Cette décomposition n'est pas spécialement minimale\footnote{\( A_1\) pourrait contenir \( \mathopen[ 0 , 1 \mathclose]\) et \( A_2\) contenir \( \mathopen] 1 , 2 \mathclose]\).} mais elle est disjointe.
            Le lemme \ref{LemUMVooZJgMu} donne
            \begin{equation}
                \ell(\bigcup_i A_i)=\sum_{(i,k)\in \eN\times \eN}\ell(I_k^{(i)})=\sum_{i\in \eN}\left( \sum_{k\in \eN}\ell(I^{(i)}_k)\right)=\sum_{i\in \eN}\ell(A_i).
            \end{equation}
            La décomposition de la somme sur \( \eN^2\) en deux sommes sur \( \eN\) est faite en vertu de la proposition \ref{PropVQCooYiWTs}.
            
\end{proof}

%--------------------------------------------------------------------------------------------------------------------------- 
\subsection{Mesure et tribu de Lebesgue}
%---------------------------------------------------------------------------------------------------------------------------

\begin{theorem} \label{ThoDESooEyDOe}
    Il existe une unique mesure \( \lambda\) sur \( \big( \eR,\Borelien(\eR) \big)\) telle que
    \begin{equation}
        \lambda\big( \mathopen] a , b \mathclose[ \big)=b-a
    \end{equation}
    pour tout \( a\leq b\) dans \( \eR\).
\end{theorem}

\begin{proof}
    
    L'existence provient du théorème de prolongement de Hahn \ref{ThoLCQoojiFfZ} : la mesure \( \ell\) sur \( (\tribA_{\mS})\) se prolonge à \( \sigma(\tribA_{\mS})=\Borelien(\eR)\).

    Nous ne pouvons pas prouver l'unicité en invoquant la partie unicité de Hahn (c'est tentant parce que \( \ell\) est \( \sigma\)-finie) parce que dans ce théorème nous ne fixons la valeur de \( \lambda\) que sur une toute petite partie de \( \tribA_{\mS}\). Nous allons cependant voir que cette petite partie suffit à garantir l'unicité.

    La classe 
    \begin{equation}
        \tribD=\{ \mathopen] a , b \mathclose[\tq -\infty<a\leq b< +\infty \}
    \end{equation}
    est stable par intersection finie et engendre la tribu borélienne. En effet \( \tribD\) contient toutes les boules et donc une base dénombrable de la topologie de \( \eR\) (proposition \ref{PropNBSooraAFr}). Donc tous les ouverts de \( \eR\) sont dans \( \sigma(\tribD)\) et \( \sigma(\tribD)=\Borelien(\eR)\). Nous pouvons donc dire grâce au théorème \ref{ThoJDYlsXu} qu'il y a unicité de la mesure sur \( \Borelien(\eR)\) lorsque les valeurs sur \( \tribD\) sont fixées.
\end{proof}

\begin{definition}
    La mesure de l'espace mesuré \( \big( \eR,\Borelien(\eR),\lambda \big)\) donné par le théorème \ref{ThoDESooEyDOe} est la \defe{mesure de Lebesgue}{mesure!de Lebesgue} sur \( \eR\).
\end{definition}

\begin{remark}
    La véritable mesure de Lebesgue munie de la tribu de Lebesgue est l'espace complété de \( \big( \eR,\Borelien(\eR),\lambda \big)\) par la proposition \ref{PropIIHooAIbfj}. Lorsque nous parlons de \( \eR\) sans précisions, les ensembles \defe{mesurables}{mesurable!ensemble} sont les éléments de cette tribu complétée. Ce sera le plus souvent le cas pour la théorie de l'intégration.

    Quoi qu'il en soit nous allons poursuivre en parlant de la mesure de Lebesgue sur les boréliens.
\end{remark}

\begin{remark}
    Il n'est pas évident que la tribu de Lebesgue soit plus grande que celle des boréliens, ni que la tribu des parties soit plus grande que celle de Lebesgue. Nous mentionnons cependant les faits suivants.
    %TODO : donner des exemples
    \begin{enumerate}
        \item
            Il existe des ensembles mesurables non-boréliens, et cela ne nécessite pas l'axiome du choix. Un argument classique de cardinalité est donné dans \cite{SFYoobgQUp}. La construction la plus explicite que j'aie trouvée est dans \cite{XSHoosgoQa}, mais ça a l'air de demander des connaissances précises sur les ordinaux.
        \item
            Vu que l'ensemble de Cantor \( C\) est mesurable de mesure nulle, tout sous-ensemble de Cantor est mesurable de mesure nulle parce que la tribu de Lebesgue est complète par définition. Le cardinal de \( \partP(C)\) est strictement supérieur à la puissance du continu, alors que le cardinal de l'ensemble des boréliens est au plus égal à la puissance du continu. Donc il existe des non boréliens contenus dans Cantor; de tels non boréliens sont alors mesurables au sens de Lebesgue.

            Pour savoir des choses sur la cardinalité de l'ensemble des parties, on peut aller voir dans \cite{KZIoofzFLV}.
        \item
            Si nous admettons l'axiome du choix alors il existe des ensembles non mesurables.
    \end{enumerate}
\end{remark}

%--------------------------------------------------------------------------------------------------------------------------- 
\subsection{Propriétés de la mesure de Lebesgue}
%---------------------------------------------------------------------------------------------------------------------------

\begin{proposition}
    Tout ensemble dénombrable de \( \eR\) est mesurable de mesure nulle.
\end{proposition}

\begin{proof}
    Un point de \( \eR\) est un intervalle de mesure nulle. Si \( D\) est dénombrable, il est union disjointes et dénombrable de points. Le lemme \ref{LemUMVooZJgMu} nous dit alors que sa mesure est \( \lambda(D)=\sum_{i=1}^{\infty}\lambda(\{ a_i \})=0\).
\end{proof}

\begin{remark}
    Il existe cependant des ensembles non dénombrables et tout de même de mesure nulle. Par exemple l'ensemble de Cantor (voir la proposition \ref{PropBEWooXZdKN}).
\end{remark}


\begin{proposition}
    La mesure de Lebesgue est invariante par translation, c'est à dire que si \( A\) est mesurable alors \( \lambda(A)=\lambda(A+\alpha)\) pour tout réel \( \alpha\).
\end{proposition}

\begin{proof}
    Nous commençons par les intervalles ouverts :
    \begin{equation}
    \lambda\big( \mathopen] a , b \mathclose[+\alpha \big)=\lambda\big( \mathopen] a+\alpha , b+\alpha \mathclose[ \big)=(b+\alpha)-(a+\alpha)=b-a=\lambda\big( \mathopen] a , b \mathclose[ \big).
    \end{equation}
    D'après ce qui est dit dans l'exemple \ref{ExDMPoohtNAj}, la mesure de Lebesgue sur les boréliens est invariante par translation.

    Si \( A\) est mesurable alors il existe un borélien \( B\) et un ensemble négligeable \( N\) tels que \( A=B\cup N\) par la caractérisation \ref{EqFJIoorxZNU} de la complétion. Alors \( A+\alpha=B+\alpha\cup N+\alpha\) et \( N+\alpha\) est encore un ensemble négligeable. Donc \( \lambda(A+\alpha)=\alpha(B+\alpha)=\lambda(B)\).
\end{proof}

Le mesure \( \ell\) définie sur l'algèbre de parties \( \tribA_{\mS}\) (voir proposition \ref{PropULFoodgXrR}). La proposition \ref{PropIUOoobjfIB} nous donne donc une mesure extérieure par
\begin{equation}    \label{EqJGXoogdKqb}
    \lambda^*(X)=\inf\{ \sum_n\ell(A_n);A_n\in\tribA_{\mS},X\subset\bigcup_nA_n \}.
\end{equation}

La proposition suivante montre que cette mesure extérieure peut être exprimée seulement avec des intervalles ouverts.
\begin{proposition} \label{PropTNOooDcfwn}
    Nous avons
    \begin{equation}
        \lambda^*(X)=\inf\{ \sum_{n\geq 1}\ell(I_n);\text{\( I_n\) sont des intervalles ouverts et }X\subset\bigcup_nI_n \}.
    \end{equation}
\end{proposition}

\begin{proof}
    Nous savons que dans la définition \eqref{EqJGXoogdKqb}, chacun des \( A_n\) est une réunion disjointe d'intervalles (pas spécialement ouverts) deux à deux disjoints; donc
    \begin{equation}
        \lambda^*(X)=\inf\{ \sum_n\ell(I_n);I_n\in\mS,X\subset\bigcup_nI_n \}.
    \end{equation}
    Soit \( \epsilon>0\). Si \( A\subset\bigcup_nI_n\), pour chaque \( n\geq 1\) nous considérons un intervalle ouvert \( J_n\) tel que \( I_n\subset J_n\) et \( \ell(I_n)+\frac{ \epsilon }{ 2^n }\leq \ell(J_n)\). Faisant cela pour chacun des découpages de \( X\) en intervalles nous trouvons
    \begin{equation}
        \lambda^*(X)\leq \inf\{ \sum_n\ell(J_n)\text{ \( J_n\) est ouvert et }X\subset\bigcup_nJ_n \}+\epsilon.
    \end{equation}
    Étant donné que \( \epsilon\) est arbitraire nous avons l'égalité.
\end{proof}

\begin{proposition}[\cite{MesureLebesgueLi}]    \label{PropMXIoojpKvd}
    Si \( X\subset \eR\) est tel que \( \lambda^*(X)<\infty\) alors
    \begin{enumerate}
        \item   \label{ItemGJUoozrDILi}
            Pour tout \( \epsilon>0\) il existe un ouvert \( \Omega_{\epsilon}\) tel que
            \begin{subequations}
                \begin{numcases}{}
                    X\subset\Omega_{\epsilon}\\
                    \lambda(\Omega_{\epsilon})\leq \lambda^*(X)+\epsilon.
                \end{numcases}
            \end{subequations}
        \item   \label{ItemGJUoozrDILii}
            Il existe une intersection dénombrable d'ouverts \( G\) telle que
            \begin{subequations}
                \begin{numcases}{}
                    X\subset G\\
                    \lambda(G)=\lambda^*(X).
                \end{numcases}
            \end{subequations}
    \end{enumerate}
\end{proposition}

\begin{proof}
    Pour \ref{ItemGJUoozrDILi}, la proposition \ref{PropTNOooDcfwn} nous a déjà dit que
    \begin{equation}
        \lambda^*(X)=\inf\{ \sum_n\ell(I_n)\text{ \( I_n\) est un intervalle ouvert}, X\subset\bigcup_nI_n \},
    \end{equation}
    donc si \( \epsilon>0\), il existe des intervalles ouverts \( I_n\) tels que 
    \begin{subequations}
        \begin{numcases}{}
            X\subset\bigcup_nI_n\\
            \sum_n\ell(I_n)\leq \lambda^*(X)+\epsilon.
        \end{numcases}
    \end{subequations}
    Si nous posons \( \Omega_{\epsilon}=\bigcup_nI_n\), alors nous avons bien
    \begin{subequations}
        \begin{numcases}{}
            X\subset\Omega_{\epsilon}\\
            \lambda(\Omega_{\epsilon})\leq\sum_n\ell(I_n)\leq \lambda^*(X)+\epsilon.
        \end{numcases}
    \end{subequations}

    En ce qui concerne \ref{ItemGJUoozrDILii}, pour chaque \( k\geq 1\) nous considérons l'ensemble \( \Omega_{1/k}\) obtenu comme précédemment avec \( \epsilon=1/k\) et nous posons \( G=\bigcap_{k\geq 1}\Omega_{1/k}\). Cela est une intersection dénombrable d'ouverts vérifiant \( X\subset G\) (parce que \( X\subset \Omega_{1/k}\) pour tout \( k\)) et donc \( \lambda^*(X)\leq\lambda^*(G)=\lambda(G)\). De plus pour tout \( k\) nous avons 
    \begin{equation}
        \lambda(G)\leq(\Omega_{1/k})\leq \lambda^*(X)+\frac{1}{ k }
    \end{equation}
    pour tout \( k\). En faisant \( k\to \infty\) nous avons
    \begin{equation}
        \lambda(G)\leq \lambda^*(X).
    \end{equation}
    Au final
    \begin{equation}
        \lambda(G)\leq \lambda^*(X)\leq \lambda(G),
    \end{equation}
    d'où l'égalité.
\end{proof}

\begin{corollary}
    Une partir \( N\subset \eR\) est négligeable\footnote{Définition \ref{DefAVDoomkuXi}.} si et seulement si \( \lambda^*(N)=0\).
\end{corollary}

\begin{proof}
    Nous savons que si \( N\) est négligeable il existe un borélien \( Y\) tel que \( N\subset Y\) avec \( \lambda(Y)=0\). Par conséquent\footnote{Au péril d'être lourd nous rappelons que \( \lambda^*\) est défini sur toutes les parties de \( \eR\).}
    \begin{equation}
        \lambda^*(N)\leq \lambda^*(Y)=\lambda(Y)=0.
    \end{equation}
    
    Pour l'implication inverse nous supposons que \( \lambda^*(N)=0\) et nous prenons l'ensemble \( G\) définit par la proposition \ref{PropMXIoojpKvd}\ref{ItemGJUoozrDILii} : c'est un borélien contenant \( N\) et tel que \( \lambda(G)=\lambda^*(N)=0\). L'ensemble \( N\) est donc négligeable.
\end{proof}

\begin{theorem}[Régularité extérieure de la mesure de Lebesgue] \label{ThoHFXooONFRN}
    Pour tout mesurable \( A\subset \eR\) nous avons
    \begin{equation}
        \lambda(A)=\inf\{ \lambda(\Omega);\text{\( \Omega\) ouvert contenant \( A\)} \}.
    \end{equation}
\end{theorem}
\index{régularité!extérieure de la mesure de Lebesgue}

\begin{proof}
    Nous commençons par le cas où \( B\) est un borélien.
    \begin{subproof}

        \item[Si \( B\) borélien, \( \lambda(B)<\infty\)]
        
        Soit \( \epsilon>0\); par la proposition \ref{PropMXIoojpKvd}\ref{ItemGJUoozrDILi} il existe un ouvert \( \Omega_{\epsilon}\) contenant \( B\) tel que \( \lambda(\Omega_{\epsilon})\leq \lambda^*(B)+\epsilon\). Vu qu'ici \( B\) est borélien, \( \lambda^*(B)=\lambda(B)\) et nous concluons que pour tout \( \epsilon\) il existe un ouvert \( \Omega_{\epsilon}\) tel que
        \begin{subequations}
            \begin{numcases}{}
                B\subset\Omega_{\epsilon}\\
                \lambda(\Omega_{\epsilon})\leq \lambda(B)+\epsilon,
            \end{numcases}
        \end{subequations}
        et donc
        \begin{equation}
            \lambda(B)=\inf\{ \lambda(\Omega);\text{ \( \Omega\) ouvert contenant \( B\) } \}.
        \end{equation}
        
        \item[Si \( B\) borélien, \( \lambda(B)=+\infty\)]

            Dans ce cas l'infimum est pris uniquement sur des ouverts \( \Omega\) tels que \( \lambda(\Omega)=\infty\).

        \item[Si \( A\) est mesurable non borélien]
    
    Nous passons maintenant au cas où \( A \) est mesurable sans être borélien. Il s'écrit donc \( A=B\cup N\) avec \( B\) borélien et \( N\) négligeable par la proposition \ref{thoCRMootPojn}, et par définition \( \lambda(A)=\lambda(B)\). Si \( Y\) est un borélien tel que \( N\subset Y\) et \( \lambda(Y)=0\) alors
    \begin{subequations}
        \begin{align}
            \lambda(A)=\lambda(B)&=\inf\{ \lambda(\Omega)\tq \text{ \( \Omega\) ouvert}, B\subset\Omega \}\label{subeqMTHoopkSKOi}\\
            &\leq\inf\{ \lambda(\Omega)\tq \text{ \( \Omega\) ouvert}, B\cup N\subset\Omega \}  \label{subeqMTHoopkSKOii}\\
            &\leq\inf_{\Omega',Y'}\{ \lambda(\Omega'\cup Y')\tq \text{ \( \Omega'\), \( Y'\) ouverts}, B\subset\Omega', Y\subset Y' \}\label{subeqMTHoopkSKOiii}\\
            &\leq\inf_{\Omega',Y'}\{ \lambda(\Omega')+\lambda(Y')\tq \text{ \( \Omega'\), \( Y'\) ouverts},  B\subset\Omega',Y\subset Y' \}\label{subeqMTHoopkSKOiv}\\
            &\leq\inf_{\Omega'}\{ \lambda(\Omega')\tq \text{ \( \Omega'\) ouvert},  B\subset\Omega \}\label{subeqMTHoopkSKOv}\\
            &=\lambda(B).
        \end{align}
    \end{subequations}
    Justifications :
    \begin{itemize}
        \item \eqref{subeqMTHoopkSKOi} Le cas borélien déjà fait.
        \item \eqref{subeqMTHoopkSKOii} Les ouverts \( \Omega\) tels que \( B\cup N\subset \Omega\) vérifient a fortiori \( B\subset \Omega\); nous avons donc agrandit l'ensemble sur lequel l'infimum est pris.
        \item \eqref{subeqMTHoopkSKOiii} Parmi les ouverts \( \Omega\) qui recouvrent \( B\cup N\), il y a ceux de la forme \( \Omega'\cup Y'\) où \( \Omega'\) recouvre \( B\) et \( Y'\) est un ouvert contenant \( Y\). Donc nous avons rétréci l'ensemble sur lequel l'infimum est pris et par conséquent agrandit l'infimum.
        \item \eqref{subeqMTHoopkSKOiv} Mesure d'une union majorée par la somme des mesures.
        \item \eqref{subeqMTHoopkSKOv} Vu que \( Y\) est borélien, \( \lambda(Y)=\inf_{\text{\( Y'\) ouvert}}\{ \lambda(Y')\tq Y\subset Y' \}=0\). Donc pour tout \( \Omega'\) et tout \( \epsilon>0\), nous pouvons trouver un \( Y'\) vérifiant les conditions tel que \( \lambda(\Omega')+\lambda(Y')\leq \lambda(\Omega')+\epsilon\).
    \end{itemize}
    Toutes les inégalités sont des égalités en en particulier \eqref{subeqMTHoopkSKOii} donne
    \begin{equation}
        \lambda(A)=\inf\{ \lambda(\Omega)\tq \text{ \( \Omega\) ouvert}, B\cup N\subset\Omega \},
    \end{equation}
    ce qu'il fallait.
    \end{subproof}
    
\end{proof}

\begin{proposition}[\cite{MesureLebesgueLi}]    \label{PropEZNoofLkVb}
    Si \( A\) est mesurable dans \( \eR\) et si \( \epsilon>0\) alors il existe un ouvert \( \Omega_{\epsilon}\) et un fermé \( F_{\epsilon}\) tels que
    \begin{subequations}    \label{subeqHNEooaNqDu}
        \begin{numcases}{}
            F_{\epsilon}\subset A\subset \Omega_{\epsilon}\\
            \lambda(\Omega_{\epsilon}\setminus F_{\epsilon})\leq \epsilon.
        \end{numcases}
    \end{subequations}
\end{proposition}

\begin{proof}
    Nous commençons par le cas d'un borélien \( B\).
    \begin{subproof}
        \item[Première étape]
        
            Montrons qu'il existe un ouvert \( U_{\epsilon}\) tel que
            \begin{subequations}
                \begin{numcases}{}
                    B\subset U_{\epsilon}\\
                    \lambda(U_{\epsilon}\setminus B)\leq \frac{ \epsilon }{2}.
                \end{numcases}
            \end{subequations}
            Si \( \lambda(B)<\infty\) alors le théorème \ref{ThoHFXooONFRN} nous donne un ouvert \( U_{\epsilon}\) tel que \( B\subset U_{\epsilon}\) et \( \lambda(U_{\epsilon})\leq \lambda(B)+\frac{ \epsilon }{2}\). Nous avons alors
            \begin{equation}
                \lambda(\Omega_{\epsilon}\setminus B)=\lambda(\Omega_{\epsilon})-\lambda(B)\leq \frac{ \epsilon }{2}.
            \end{equation}
            Si par contre \( \lambda(B)=\infty\), nous posons \( B_n=B\cap\mathopen[ -n , n \mathclose]\) et \( \epsilon_n=\epsilon/2^{n+1}\). Pour chaque \( n\) nous avons un ouvert \( \Omega_n\) tel que
            \begin{subequations}
                \begin{numcases}{}
                    B_n\subset \Omega_n\\
                    \lambda(\Omega_n\setminus B_n)\leq \frac{ \epsilon }{ 2^{n+1} }
                \end{numcases}
            \end{subequations}
            Par conséquent en posant \( \Omega=\bigcup_{n\geq 1}\Omega_n\) nous avons\footnote{Nous utilisons la petite relation ensembliste \( \big( \bigcup_nA_n \big)\setminus\big( \bigcup_nB_n \big)\subset \bigcup_n(A_n\setminus B_n)\).}
            \begin{subequations}
                \begin{numcases}{}
                    B\subset \Omega\\
                    \lambda(\Omega\setminus B)\leq \lambda\big( \bigcup_n(\Omega_n\setminus B_n) \big)\leq \sum_{n\geq 1}\lambda(\Omega_n\setminus B_n)=\frac{ \epsilon }{2}.
                \end{numcases}
            \end{subequations}
            La première étape est terminée.

        \item[Deuxième étape]

            Nous prouvons à présent qu'il existe un ouvert \( \Omega_{\epsilon}\) et un fermé \( F_{\epsilon}\) tels que
            \begin{subequations}
                \begin{numcases}{}
                    F_{\epsilon}\subset B\subset \Omega_{\epsilon}\\
                    \lambda(\Omega_{\epsilon}\setminus B)\leq \frac{ \epsilon }{2}\\
                    \lambda(B\setminus F_{\epsilon})\leq \frac{ \epsilon }{2}.
                \end{numcases}
            \end{subequations}
            L'ouvert \( \Omega_{\epsilon}\), nous l'avons déjà de l'étape précédente. Pour le fermé, nous appliquons la première étape au borélien \( B^c\); ce qui nous trouvons est un ouvert \( G_{\epsilon}\) tel que
            \begin{subequations}
                \begin{numcases}{}
                    B^c\subset G_{\epsilon}\\
                    \lambda(G_{\epsilon}\setminus B^c)\leq \frac{ \epsilon }{2}.
                \end{numcases}
            \end{subequations}
            En posant \( F_{\epsilon}=G_{\epsilon}^c\) nous avons un fermé tel que \( F_{\epsilon}\subset B\) et
            \begin{equation}
                \lambda(B\setminus F_{\epsilon})=\lambda(F_{\epsilon}^c\setminus B^c)=\lambda(G_{\epsilon}\setminus B^c)\leq \frac{ \epsilon }{2}.
            \end{equation}
            
        \item[Dernière étape]

            Les ensembles \( F_{\epsilon}\) et \( \Omega_{\epsilon}\) trouvés à la deuxième étape donnent bien les relations \eqref{subeqHNEooaNqDu}. En effet \( \Omega_{\epsilon}\setminus F_{\epsilon}=(\Omega_{\epsilon}\setminus B)\cup(B\setminus F_{\epsilon})\), donc
            \begin{equation}
                \lambda(\Omega_{\epsilon}\setminus F_{\epsilon})\leq \lambda(\Omega_{\epsilon}\setminus B)+\lambda(B\setminus F_{\epsilon})=\epsilon.
            \end{equation}
    \end{subproof}
    Nous passons au cas où \( A=B\cup N\) est mesurable. Nous commençons par prendre les \( \Omega_{\epsilon}\) et \( F_{\epsilon}\) qui correspondent à \( B\) :
    \begin{subequations}
        \begin{numcases}{}
            F_{\epsilon}\subset B\subset \Omega_{\epsilon}\\
            \lambda(\Omega_{\epsilon}\setminus F_{\epsilon})\leq \epsilon.
        \end{numcases}
    \end{subequations}
    Soit \( Y\) un borélien tel que \( N\subset Y\) et \( \lambda(Y)\) puis un ouvert \( Y'\) tel que \( \lambda(Y')\leq \epsilon\) et \( Y\subset Y'\). L'existence d'un tel \( Y'\) est assurée par la proposition \ref{ThoHFXooONFRN} appliquée à \( Y\). Nous vérifions que les ensembles \( F_{\epsilon}\) et \( \Omega_{\epsilon}\cup Y'\) fonctionnent. En effet \( \Omega_{\epsilon}\cup Y'\setminus F_{\epsilon}\subset (\Omega_{\epsilon}\setminus F_{\epsilon})\cup Y'\), donc
    \begin{subequations}
        \begin{numcases}{}
            F_{\epsilon}\subset B\cup N\subset \Omega_{\epsilon}\cup Y'\\
            \lambda\big( (\Omega_{\epsilon}\setminus F_{\epsilon}) \big)\leq \lambda(\Omega_{\epsilon}\setminus F_{\epsilon})+\lambda(Y')\leq 2\epsilon.
        \end{numcases}
    \end{subequations}
    Donc en réalité il faut choisir \( \Omega_{\epsilon/2}\), \( F_{\epsilon/2}\) et \( \lambda(Y')\leq \epsilon/2\).
\end{proof}

\begin{theorem}[Régularité intérieure de la mesure de Lebesgue]
    Si \( A\) est mesurable dans \( \eR\) alors
    \begin{equation}
        \lambda(A)=\sup\{ \lambda(K); \text{\( K\) compact contenu dans \( B\)} \}.
    \end{equation}
\end{theorem}
\index{régularité!intérieure de la mesure de Lebesgue}

\begin{proof}
    Par la proposition \ref{PropEZNoofLkVb} nous avons
    \begin{equation}    \label{EqTPEooUHTbH}
        \lambda(A)=\sup_{\text{\( F\) fermé dans \( A\)}}\lambda(F).
    \end{equation}
    Pour un tel \( F\) nous posons \( K_n=F\cap\mathopen[ -n , n \mathclose]\) qui est compact\footnote{parce que fermé et borné, théorème de Borel-Lebesgue \ref{ThoXTEooxFmdI}.} et contenu dans \( B\). De plus le lemme \ref{LemAZGByEs}\ref{ItemJWUooRXNPcii} nous dit que
    \begin{equation}
        \lambda(F)=\lim_{n\to \infty} \lambda(K_n)
    \end{equation}
    Donc tous les \( \lambda(F)\) peuvent être arbitrairement approchés par un \( \lambda(K)\) avec \( K\) compact dans \( A\), et le supremum \eqref{EqTPEooUHTbH} n'est pas affecté en nous restreignant à prendre des compacts contenus dans \( B\) :
    \begin{equation}    
        \lambda(A)=\sup_{\text{\( F\) fermé dans \( A\)}}\lambda(F)=\sup_{\text{\( K\) compact dans \( A\)}}\lambda(K).
    \end{equation}
\end{proof}

%+++++++++++++++++++++++++++++++++++++++++++++++++++++++++++++++++++++++++++++++++++++++++++++++++++++++++++++++++++++++++++ 
\section{Lebesgue sur \texorpdfstring{$ \eR^d$}{Rd}}
%+++++++++++++++++++++++++++++++++++++++++++++++++++++++++++++++++++++++++++++++++++++++++++++++++++++++++++++++++++++++++++

Quelque renvois :
\begin{itemize}
    \item Le produit de tribus est donné par la définition \ref{DefTribProfGfYTuR},     % Cette référence doit être vers le haut.
    \item le produit d'espaces mesurés est donné par la définition \ref{DefUMlBCAO}.     % Cette référence doit être vers le haut.
\end{itemize}

La mesure de Lebesgue sur \( \eR^d\), notée \( \lambda_N\) est d'abord la mesure définie sur \( \big( \eR^d,\Borelien(\eR^d) \big)\) comme le produit \( \lambda\otimes\ldots\otimes \lambda\). Puis la complétion.

\begin{proposition}[\cite{OYRmzAa}]     \label{PropTHDQooWMSbJe}
    Tout ouvert de \( \eR^n\) est une union dénombrable de rectangles presque disjoints\footnote{«presque» au sens où les intersections éventuelles sont de mesure de Lebesgue nulle.}.
\end{proposition}

\begin{proof}
    Soit \( G\) un ouvert de \( \eR^n\). Soit \( \{ Q_i^{1} \}_{i\in \eN}\) un découpage de \( \eR^n\) en cubes de côté \( 1\) et dont les sommets sont en les coordonnées entières. Ce sont des cubes presque disjoints. Nous considérons ensuite pour chaque \( k>1\) le découpage \( \{ Q_i^{(k)} \}_{i\in\eN}\) de \( \eR^n\) en cubes de côtés \( 2^{-k}\) qui consiste à découper en \( 2\) les côtés des cubes du découpage \( Q^{(k-1)}\). Ces cubes forment encore un découpage dénombrable de \( \eR^n\) en des cubes presque disjoints. Ensuite nous considérons \( \mE\) l'union de tous les \( Q_i^{(k)}\) contenus dans \( G\).

    Montrons que \( \mE=G\). D'abord \( \mE\subset G\) parce que \( \mE\) est une union d'ensembles contenus dans \( G\). Ensuite si \( x\in G\), il existe une boule de rayon \( r\) autour de \( x\) contenue dans \( G\); alors un des ensembles \( Q_i^{(k)}\) avec \( 2^{-j}<\frac{ r }{2}\) est contenue dans \( B(x,r)\) et donc dans \( \mE\).

    Bien entendu l'union qui donne \( \mE\) n'est pas satisfaisante par ce que les \( Q_i^{(k+1)}\) sont contenus dans les \( Q_i^{(k)}\); les intersections sont donc loin d'être de mesure nulle.

    Nous faisons ceci : 
    \begin{subequations}
        \begin{align}
            R^{(0)}&=\{ Q_i^{(1)} \text{contenu dans \( G\)} \}\\
            R^{(k+1)}&=\{ Q_i^{(k+1)}\text{contenus dans \( G\) et pas dans \( R^{(k)}\)} \}.
        \end{align}
    \end{subequations}
    En fin de compte l'union de tous les ensembles contenus dans les \( R^{(k)}\) forment encore \( \eR^n\), mais sont d'intersection presque vide.
\end{proof}

\begin{proposition}     \label{PropWOOOooHcoEEF}
    Les boréliens sur \( \eR^N\) sont ceux qu'on croit.
    \begin{enumerate}
        \item
            \( \Borelien(\eR^2)=\Borelien(\eR)\otimes \Borelien(\eR)\)
        \item
            \( \Borelien(\eR^{N+1})=\Borelien(\eR^N)\otimes \Borelien(\eR)\)
    \end{enumerate}
\end{proposition}

\begin{proof}
    % position 22462
\end{proof}
<++>

\begin{corollary}
    La mesure \( \lambda_N\) est l'unique mesure sur \(   (\eR^N,  \Borelien(\eR^N) )   \) à satisfaire 
    \begin{equation}
        \mu\big( \prod_{i=1}^N\mathopen[ a_i , b_i \mathclose] \big)=\prod_{i=1}^n| a_i-b_i |
    \end{equation}
\end{corollary}

\begin{proof}
    Par définition de la mesure produit, \( \lambda_N\) est l'unique mesure sur \(   (\eR^N,  \Borelien(\eR)\otimes\otimes\Borelien(\eR) )   \) à satisfaire la condition. La proposition \ref{PropWOOOooHcoEEF} conclut.
\end{proof}

%--------------------------------------------------------------------------------------------------------------------------- 
\subsection{Ensembles négligeables}
%---------------------------------------------------------------------------------------------------------------------------

\begin{lemma}[\cite{VSMEooLwNLHd}]
    L'image d'une partie négligeables de \( \eR^N\) par une application Lipschitz est négligeable.
\end{lemma}

\begin{proof}
    Soit \( N\) une partie négligeable de \( \eR^N\) et une application Lipschitz \( f\colon N\to \eR^N\).
\end{proof}
<++>

