%+++++++++++++++++++++++++++++++++++++++++++++++++++++++++++++++++++++++++++++++++++++++++++++++++++++++++++++++++++++++++++
\section{Avertissement}
%+++++++++++++++++++++++++++++++++++++++++++++++++++++++++++++++++++++++++++++++++++++++++++++++++++++++++++++++++++++++++++

Ceci sont des notes «prises au vol» de certains de mes cours pour l'agrégation. Aucune garantie. Merci de me signaler toute faute ou remarque. En particulier :
\begin{enumerate}
    \item
        l'énoncé et la démonstration de la proposition \ref{PropNsLqWb}.
\end{enumerate}

Une version avec ISBN de ce document est en préparation afin qu'il soit acceptable comme livre aux oraux de l'agrégation \ldots j'y travaille et on verra comment les choses vont évoluer.

%+++++++++++++++++++++++++++++++++++++++++++++++++++++++++++++++++++++++++++++++++++++++++++++++++++++++++++++++++++++++++++
\section{Auteurs et contributeurs}
%+++++++++++++++++++++++++++++++++++++++++++++++++++++++++++++++++++++++++++++++++++++++++++++++++++++++++++++++++++++++++++

Le principal auteur et metteur en \LaTeX\ de ce document est votre serviteur, Laurent Claessens.

D'autres ont participé.
\begin{enumerate}
    \item
        Nicolas Richard et Ivik Swan pour les parties des exercices et rappels de calcul différentiel et intégral (Université libre de Bruxelles, 2003-2004) qui leur reviennent.
    \item
        Pierre Bieliavsky pour les énoncés d'analyse numérique (MAT1151 à Louvain la Neuve 2009-2010)
    \item
        Jonathan Di Cosmo pour certaines corrections de MAT1151
    \item
        Le forum usenet de math, en particulier pour la construction des corps fini dans la fil «Vérifier qu'un polynôme est primitif» initié le 20 décembre 2011.
\end{enumerate}

%+++++++++++++++++++++++++++++++++++++++++++++++++++++++++++++++++++++++++++++++++++++++++++++++++++++++++++++++++++++++++++
					\section*{Ces notes sont les vôtres !}
%+++++++++++++++++++++++++++++++++++++++++++++++++++++++++++++++++++++++++++++++++++++++++++++++++++++++++++++++++++++++++


Il y a encore certainement des erreurs, des fautes de frappe et des choses pas claires. Je compte sur vous (oui : toi !) pour me signaler toute imperfection (y compris d'orthographe).

Plus vous signalez de fautes, meilleure sera la qualité du texte, et plus les étudiants de l'année prochaine vous seront reconnaissants.


%+++++++++++++++++++++++++++++++++++++++++++++++++++++++++++++++++++++++++++++++++++++++++++++++++++++++++++++++++++++++++++
\section{Instructions pour les examens et interrogations}
%+++++++++++++++++++++++++++++++++++++++++++++++++++++++++++++++++++++++++++++++++++++++++++++++++++++++++++++++++++++++++++

Ceci sont des conseils généraux que nous vous conseillons de suivre dans toutes les matières.
\begin{description}
    \item[numéroter] Numérotez clairement toutes les questions. Si votre réponse prend plus d'une page, écrivez «suite au verso», «suite à l'intercalaire \( n\)» etc. À l'endroit où la réponse continue, écrivez «question \( n\), suite».

    \item[vérifiez] Certaines erreurs sont faciles à détecter. Par exemple
        \begin{enumerate}
            \item
                les aires et volumes sont positifs;
            \item
                une intégrale \emph{définie} qui contient «\( dx\)» ne peut pas contenir de \( x\) dans la réponse;

            \item
                en physique et en chimie, les unités doivent être cohérentes : si la réponse est une énergie, vous devez avoir des joules (\unit{\square\metre\kilo\gram\per\square\second}).

        \end{enumerate}
    \item[votre nom] Écrivez votre nom et votre numéro de carte d'étudiant.

    \item[les faciles d'abord] Lisez d'abord toutes les questions avant de répondre. Commencez par les questions faciles.

    \item[justifier] Justifiez vos réponses. N'hésitez pas à écrire des phrases complètes : sujet, verbe, complément. N'abusez pas des symboles dont vous ignorez la signification :
        \begin{enumerate}
            \item
                «\( \Leftrightarrow\)» signifie «si et seulement si», et non «la suite de mon calcul»;
            \item

                «\( \nexists\)» signifie «il n'existe pas», et non «n'existe pas» ou «n'est pas défini».
        \end{enumerate}

    \item[ne pas passer en force] Si vous savez que votre réponse est fausse, mais vous ne savez pas la corriger, écrivez sur votre feuille «cette réponse est fausse pour telle raison, mais je ne sais pas comment corriger». Ne comptez pas sur une inattention du correcteur. En science, affirmer un fait que vous savez être faux s'appelle de la falsification; c'est déontologiquement inacceptable. De la même façon, si vous copiez sur votre voisin\footnote{Indépendamment que c'est sans doute interdit par le règlement; vérifiez avant.}, vous êtes priés de le citer : on ne s'approprie pas le travail d'autrui.

    \item[approximations numériques] Lorsque vous voulez écrire une approximation numérique, réfléchissez au sens de ce que vous allez écrire. En mathématique, ça n'a presque jamais de sens d'écrire une approximation parce que vous ne savez pas dans quel contexte votre calcul pourra être utilisé. Si vous laissez deux décimales à \( \pi\) pour calculer le volume d'eau dans votre piscine gonflable, ça fera l'affaire; si c'est pour calculer la masse du Higgs ou pour mettre un satellite autour de Mars, vous perdez plusieurs millions d'euros.

        En sciences naturelles (physique, chimie ou autres), vous pouvez donner des approximations numériques de façon circonstanciée. Demandez à votre prof de labo.

    \item[orthographe] Sans être obligatoire, ça ne fait jamais de mal. Surtout si le français est votre langue maternelle.
    \item[santé] Mangez des fruits et des légumes de saisons. Choisissez des producteurs locaux qui n'utilisent pas d'engrais synthétisés à base de pétrole. De toutes façons \href{http://www.energybulletin.net/node/51306}{vous n'avez pas le choix}.

\end{description}

%+++++++++++++++++++++++++++++++++++++++++++++++++++++++++++++++++++++++++++++++++++++++++++++++++++++++++++++++++++++++++++
\section{Propagande : utilisez un ordinateur !}
%+++++++++++++++++++++++++++++++++++++++++++++++++++++++++++++++++++++++++++++++++++++++++++++++++++++++++++++++++++++++++++

Si vous faites des exercices supplémentaires et que vous voulez des corrections, n'oubliez pas que vous avez un ordinateur à disposition. De nos jours, les ordinateurs sont capables de calculer à peu près tout ce qui se trouve dans vos cours de math\footnote{Avec une notable exception pour les limites à deux variables.}.

D'ailleurs, je te rappelle que nous sommes est déjà largement dans le vingt et unième siècle et que tu te destines à une carrière professionnelle dans laquelle tu auras des calculs à faire; si tu n'es pas encore capable d'utiliser un ordinateur pour faire ces calculs, il est temps de combler cette lacune.

%---------------------------------------------------------------------------------------------------------------------------
\subsection{Lancez-vous dans Sage}
%---------------------------------------------------------------------------------------------------------------------------

Le logiciel que je vous propose est \href{http://www.sagemath.org}{Sage}. Pour l'utiliser, il n'est même pas nécessaire de l'installer sur votre ordinateur~: il tourne en ligne, directement dans votre navigateur.

\begin{enumerate}

	\item
		Aller sur \href{http://www.sagenb.org}{http://www.sagenb.org}
	\item
		Créer un compte
	\item
		Créer des feuilles de calcul et amusez-vous !!

\end{enumerate}

Il y a beaucoup de \href{http://lmgtfy.com/?q=sage+documentation}{documentation} sur le \href{http://www.sagemath.org}{site officiel}\footnote{\href{http://www.sagemath.org}{http://www.sagemath.org}}.

Si vous comptez utiliser régulièrement ce logiciel, je vous recommande \emph{chaudement} de \href{http://mirror.switch.ch/mirror/sagemath/index.html}{l'installer} sur votre ordinateur. Ce logiciel étant distribué sous licence GPL, vous ne devez ni payer ni vous procurer de codes.

%---------------------------------------------------------------------------------------------------------------------------
\subsection{Exemples de ce que Sage peut faire pour vous}
%---------------------------------------------------------------------------------------------------------------------------

Voici une liste absolument pas exhaustive de ce que Sage peut faire pour vous, avec des exemples. 
\begin{enumerate}

	\item
		Calculer des limites de fonctions, voir l'exercice \ref{exoINGE11140028},

	\item
		D'autres limites et tracer des fonctions, voir l'exercice \ref{exoINGE11140031}.
	\item
		Calculer des dérivées, voir exercice \ref{exo0013}.
	\item
		Calculer des dérivées partielles de fonctions à plusieurs variables, voir exercice \ref{exoFoncDeuxVar0002}.
	\item
		Calculer des primitives, voir certains exercices \ref{exo0017}
	\item

		Résoudre des systèmes d'équations linéaires. Lire \href{http://www.sagemath.org/doc/constructions/linear_algebra.html#solving-systems-of-linear-equations}{la documentation} est ce qui fait la différence entre l'être humain et le non scientifique. Voir les exercices  \ref{exoINGE1121La0016} et \ref{exoINGE1121La0010}.

	\item
		Tout savoir d'une forme quadratique, voir exercice \ref{exoINGE1121La0018}.
	\item
		Calculer la matrice Hessienne de fonctions à deux variables, déterminer les points critiques, déterminer le genre de la matrice Hessienne aux points critiques et écrire extrema de la fonctions (sous réserve d'être capable de résoudre certaines équations), voir les exercices \ref{exoFoncDeuxVar0029} et \ref{exoFoncDeuxVar0028}.
	\item
		Lorsqu'il y a une infinité de solutions, Sage vous l'indique avec des paramètres (ne fonctionne hélas pas avec les fonctions trigonométriques), voir l'exercice \ref{exoDerrivePartielle-0007}.


	\item
		Calculer des dérivées symboliquement, voir exercice \ref{exoDerive-0002}.
	\item
		Calculer des approximations numériques comme dans l'exercice \ref{exoOutilsMath-0028}.
    \item
        Calculer dans un corps de polynômes modulo comme \( \eF_p[X]/P\) où \( P\) est un polynôme à coefficients dans \( \eF_p\). Voir l'exemple \ref{ExemWUdrcs}.
	\item
        Tracer des courbes en trois dimensions, voir exemple \ref{ExempleTroisDxxyy}. Notez que pour cela vous devez installer aussi le logiciel Jmol. Pour Ubuntu\footnote{Pour les autres, je ne sais pas, mais je laisserai jouer les adages (faux) «si tu es sous Linux c'est que tu es un pro» et «Windows c'est facile». Quant aux utilisateurs d'OS «alternatifs» comme Hurd ou BSD ben heu \ldots} c'est dans le paquet \info{icedtea6-plugin}.
\end{enumerate}

%+++++++++++++++++++++++++++++++++++++++++++++++++++++++++++++++++++++++++++++++++++++++++++++++++++++++++++++++++++++++++++
%\section{Propagande : n'utilisez pas votre calculatrice}
%+++++++++++++++++++++++++++++++++++++++++++++++++++++++++++++++++++++++++++++++++++++++++++++++++++++++++++++++++++++++++++

%D'abord, l'expérience montre que la majorité des fois qu'un étudiant sort sa calculatrice, c'est pour faire un calcul inutile, et le plus souvent la calculatrice fournit un résultat faux. Étant en 2011, vous ne devriez pas vous contenter de vos calculatrices qui coûtent un os, qui n'ont pas de puissance de calcul, qui ont une définition d'écran ridicule et en noir et blanc. Remarquez que votre GSM (et a forciori vos minis trucs qui se connectent a internet) sont considérablement plus puissants que ces vieilleries; ils ont un meilleur écran.
