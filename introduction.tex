%+++++++++++++++++++++++++++++++++++++++++++++++++++++++++++++++++++++++++++++++++++++++++++++++++++++++++++++++++++++++++++
\section*{Présentation de l'unité}
%+++++++++++++++++++++++++++++++++++++++++++++++++++++++++++++++++++++++++++++++++++++++++++++++++++++++++++++++++++++++++++

Le cours <<Géométrie analytique>> est en grande partie dédié à la généralisation à plusieurs variables des notions déjà vues dans le cas de fonctions numériques à variables réelles : limites, continuité, différentiabilité, intégration. Il est très important de réviser ce que vous avez étudié à ce sujet dans les cours précédents pour les fonctions numériques. Cela vous fournira une boussole pour suivre le déroulement de la théorie et vous permettra de comprendre au mieux les exemples dès le début. Pour cette raison, ces notes contiennent plusieurs annexes de rappel (les annexes ne font pas partie de la matière du cours). Cependant, il se trouve que la généralisation envisagée, qui nous est nécessaire pour l'étude de la physique et notamment pour la mécanique classique et l'électromagnétisme, ne soit pas tout a fait automatique. Des idées nouvelles et des techniques ingénieuses sont nécessaires et pendant plusieurs siècles les hommes de sciences de la plus haute valeur ont dû travailler durement pour donner à cette partie fondamentale des mathématiques son élégance actuelle. Vous n'êtes pas donc sensés tout comprendre dans un jour ou deux. Nous vous invitons à prendre votre temps pour lire les notes au moins deux fois et travailler à fond tous les exemples. Le mathématicien russe Vladimir.~I.~Arnol'd disait que pour un étudiant le contenu d'une théorie mathématique n'est jamais plus grand que l'ensemble des exemples qu'il a vraiment compris. Vous êtes aussi encouragés à nous contacter si vous avez des questions sur les notes ou sur les exercices.  

Une fois que vous avez bien lu la théorie et compris les exemples nous vous conseillons de faire le plus possible d'exercices. Attention : lire la théorie est un prérequis et non un complément !! 

Tous les exercices proposés sont résolus de façon détaillé, mais lire les solutions sans avoir d'abord essayé de les résoudre par par vous même est vivement déconseillé en tant qu'inutile et potentiellement nocif. Les solutions vous sont fournies comme référence pour vérifier votre travail et parce que parfois il est intéressant de faire quelques commentaires en plus ou de proposer des méthodes qu'on a pas pu présenter dans la théorie. 

La présentation proposée dans ces notes n'est forcement pas la plus abstraite ni la plus générale. En particulier, presque tous les exemples et les exercices sont sur les espaces $\mathbb{R}^2$ et $\mathbb{R}^3$.
 
\begin{description}
	\item[Le chapitre  \ref{ChapEspVectNorm}] présente les notions de distance, limite et continuité dans le cadre des espaces vectoriels normés quelconque de dimension finie. Nous abordons en même temps la topologie de ces espaces : boules, ouverts, fermés, adhérence, etc.
        \item[Le chapitre  \ref{Chap_SystcoordGA}] contient des rappels très importants sur les différentes systèmes de coordonnées. C'est un chapitre qui ne fait pas vraiment partie de la matière mais à qui il faut souvent faire référence.    
	\item[Le chapitre \ref{Chap_courbes}] est le seul chapitre de ce cours où nous sommes intéressés par des fonctions d'une seule variable réelle. En fait ici  nous étudions comment on décrit des courbes dans le plan ou dans l'espace : leur paramétrisation, le calcul de la longueur et comment introduire un repère orthogonal solidaire à la courbe. Ce chapitre contiendra bientôt une nouvelle partie sur les intégrales curvilignes (facultative).
	\item[Le chapitre \ref{ChapLimContinuite}] explique comment on généralise les notions de limite et de continuité pour fonction de plusieurs variables. Plusieurs méthodes de calcul sont aussi présentées. 
	\item[Le chapitre \ref{Chap_appl_lin}] recueillit des rappels sur les applications linéaires, la définition de la norme d'une application linéaire et une courte introduction aux applications multilinéaires. Dans ce chapitre, nous introduirons une norme (au sens du chapitre \ref{ChapEspVectNorm}) sur l'ensemble des applications linéaires. Ce sera l'occasion d'utiliser les techniques d'espaces vectoriels normés dans un cas différent du cas standard de l'espace $\eR^m$
	\item[Le chapitre \ref{ChapCalculDiff}] est dédié à la généralisation de la notion de dérivation et d'approximation linéaire.
	\item[Le chapitre \ref{ChapMultiples}] présente la théorie de l'intégration de Riemann et explique comment on peut calculer l'aire d'une portion du plan ou le volume d'un solide. 
        \item[Le chapitre \ref{AnalyseVectorielle}] est encore sous la forme d'ébauche. Il est censé contenir des compléments sur les opérateurs différentiels, la définition d'intégrale de surface et les théorèmes de Green, de Gauss-Ostrogradski et de Stokes.   

	\item[Le chapitre \ref{Chap_exercices}] contient les exercices. Ces derniers font partie de la matière (sauf ceux marqués du symbole \mortelexo), y compris les corrections.
	\item[Les appendices] Les appendices sont là en tant que compléments ou en tant que rappels. Bien qu'ils ne fassent pas partie de la matière proprement dite, vous n'êtes pas censé ignorer les rappels : ce qui a été vu l'année passée peut être utilisé (nous pensons surtout à l'analyse dans les réels et aux techniques d'intégration de fonctions à une variable).
	\item[Index, liste des notations] Si en lisant vous avez un doute sur la définition d'un mot ou sur une notation, allez voir dans l'index et dans la liste des notations.
\end{description}

Durant le premier semestre de l'année 2011/2012 nous avons utilisé ce poly pour le cours présentiel de la même unité. Nous avons remarquez plusieurs points faibles dans l'exposition et nous voudrions l'améliorer. Le gros du travail est fait et ce que vous avez dans le mains est la deuxième edition du poly. Cependant, une fois par mois environs, je mettrai en ligne sur moodle des nouvelles versions des chapitres que j'aurai modifié. Bien entendu, la copie que vous avez reçu est plus que suffisante pour bien réussir l'unité.       

%---------------------------------------------------------------------------------------------------------------------------------------
\subsection*{Organisation pratique}
%----------------------------------------------------------------------------------------------------------------------------------------
L'unité semestrielle <<Géométrie analytique>> se compose de 20 h de cours et 40 h de TD, correspondant à 6 crédits ECTS. 
Le cours, le travaux dirigés et les solutions détaillées des exercices proposés sont rassemblés dans cet volume. Vous recevrez (éventuellement dans des envois séparés) trois problèmes, qui seront corrigés. Si vous êtes  <<stagiaires>>  votre copie sera corrigée  par votre responsable de regroupement, si vous en avez  un. Si cela n'est pas le cas, ou vous n'êtes pas  <<stagiaires>> alors votre copie sera corrigé par Mlle Carlotta Donadello. Les solutions vous seront envoyés à votre adresse e-mail de l'université. Si vous n'avez pas d'accès à internet il faudra spécifier dans votre copie que vous souhaitez recevoir une copie de la correction par courrier. Les devoirs sont à renvoyer à l’adresse suivante : 

\begin{center}
  Université de Franche-Comté, CTU filière mathématique,\\
Bâtiment Bachelier, Domaine Universitaire de la Bouloie,\\
 Carlotta Donadello (Géométrie analytique),\\
 25030 BESANCON CEDEX.
\end{center}

Il est très important de considérer les trois devoirs comme des séances d'entraînement pour l'examen. Essayez d'abord de les faire dans le temps limite de trois heures et de ne pas utiliser vos notes ni la calculatrice. Évidemment vous pouvez y revenir ensuite si vous en avez le temps ou si une idée de génie vous frappe, mais le gros du travail devrait être fait en trois heures et en complète autonomie.  

L'examen final consistera en une épreuve écrite de trois heures. L’épreuve sera constituée de quelques exercices de difficulté différente et, éventuellement, d’une question de théorie.  Aucun document ni calculatrice n’est autorisé pendant l'épreuve.

Le matériel du cours pour l'année 2011/2012 est \textsc{essentiellement} contenu dans les chapitres  \ref{ChapEspVectNorm},  \ref{Chap_courbes}, \ref{ChapLimContinuite}, \ref{ChapCalculDiff}, \ref{ChapMultiples}. Voir les introductions des chapitres pour plus de précision. 

\vfill

La première version de ce texte e\'tait basée sur les notes manuscrites du cours présentiel donné par Mihai \textsc{Bostan} durant l'année 2009-2010. Les auteurs le remercient pour son travail et sa générosité. Plusieurs personnes ont participé à l'amélioration de ces notes. Sur tous, François \textsc{Lemeux} (qui était intervenant dans le TD du cours présentiel durant l'année 2011/2012) à donnée une contribution importante dans la révision des premiers chapitres et à proposée des nouveaux exercices sur les normes de matrices. Merci aussi à tous les étudiants qui ont pris le temps de nous signalé des coquilles et double merci à  ceux qui ont trouvé des fautes dans les exercices !!


