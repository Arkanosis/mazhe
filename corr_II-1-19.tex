% This is part of the Exercices et corrigés de CdI-2.
% Copyright (C) 2008, 2009
%   Laurent Claessens
% See the file fdl-1.3.txt for copying conditions.


\begin{corrige}{_II-1-19}

$y=(1+y'^2)^{1/2}$. Avant de passer en paramétrique, il faut vérifier que $y'$ n'est pas une constante. En effet, si tel était le cas, alors l'équation $y(\lambda)=(1+\lambda^2)^{1/2}$ ne serait valable que pour une seule valeur du paramètre ($\lambda =C$), ce qui est un peu pas trop bien pour une équation paramétrique.

Quoi qu'il en soit, la seule solution avec $y'=C$ est $y'=0$ et donc $y=1$.

Les autres solutions sont obtenues de façon paramétrique :
\begin{subequations}
\begin{numcases}{}
	y(\lambda)=(1+\lambda^2)^{1/3}\\
	t(\lambda)=t_0+\int_{\lambda_0}^{\lambda}\frac{ f'(\xi) }{ \xi }d\xi
\end{numcases}
\end{subequations}
où $f(\xi)=(1+\xi^2)^{1/2}$. Nous devons donc intégrer $1/\sqrt{1+\xi^2}$. Le formulaire donne
\begin{equation}
	\int\frac{ dv }{ \sqrt{v^2\pm a^2} }=\ln| v+\sqrt{v^2\pm a^2} |+C.
\end{equation}
\begin{lemma}
Nous avons la formule
\begin{equation}
	\arcsinh(a)=\ln(a+\sqrt{a^2+1})
\end{equation}
pour l'inverse du sinus hyperbolique.
\end{lemma}
\begin{proof}
Nous avons $x=\arcsinh(a)$, lorsque $e^x- e^{-x}=2a$. Posons $y=e^x$, et résolvons
\begin{equation}
	y+\frac{1}{ y }=2a
\end{equation}
par rapport à $y$. Il y a deux solutions : $y=a\pm\sqrt{a^2+1}$. La solution $a-\sqrt{a^2+1}$ est à rejeter parce que $y=e^x>0$. Donc il ne reste que
\begin{equation}
	x=\ln(a+\sqrt{a^2+1}).
\end{equation}
\end{proof}
Par conséquent, 
\begin{equation}
	t(\lambda)=t_0+\arcsinh(\lambda)-\arcsinh(\lambda_0).
\end{equation}
Cette relation peut être inversée pour connaître $\lambda$ en fonction de $t$ :
\begin{equation}
	\lambda(t)=\sinh(t-C)
\end{equation}
où nous avons, comme toujours, regroupé toutes les constantes dans une seule constante $C$. De cette façon, nous pouvons donner une forme explicite pour $y(t)$ :
\begin{equation}
	y(t)=y\big( \lambda(t) \big)=\big( 1+\lambda(t)^2 \big)^{1/2}=\cosh(t-C).
\end{equation}
Nous avons donc comme solution générale de l'équation différentielle : $y(t)=\cosh(t-C)$.

\begin{alternative}
	Nous pouvons résoudre $y=(1+y'^2)^{1/2}$ algébriquement par rapport à $y'(t)$ :
\begin{equation}
	y'(t)=\pm\sqrt{y^2-1},
\end{equation}
ce qui mène à
\begin{equation}
	\frac{ dy }{ \sqrt{y^2-1} }=\pm dt,
\end{equation}
et donc $y(t)=\pm\cosh(t-C)$. Le choix de $\pm$ se fixe en remontant à l'équation de départ : $y=(1+y'^2)^{1/2}$ demande $y>0$, et donc le choix de $y=\cosh(t+C)$. 
\end{alternative}

Nous pouvons à présent résoudre quelques problèmes de Cauchy.

\begin{enumerate}

\item 
$y(0)=\cosh(1)$. Nous avons $\cosh(C)=\cosh(1)$, donc $C=\pm 1$ (je rappelle que le cosinus hyperbolique est une fonction paire), et donc deux solutions :
\begin{equation}
	y(t)=\cosh(t\pm 1).
\end{equation}

\item
$y(0)=1$. Nous trouvons facilement $C=0$ et donc $y=\cosh(t)$. Ne pas oublier la solution $y=1$ que nous avons mentionné plus haut.

\item
$y(0)=0$, pas de solutions.

\end{enumerate}


\end{corrige}
