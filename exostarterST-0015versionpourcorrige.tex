% This is part of Analyse Starter CTU
% Copyright (c) 2014
%   Laurent Claessens,Carlotta Donadello
% See the file fdl-1.3.txt for copying conditions.

\begin{exercice}\label{exostarterST-0015versionpourcorrige}


On appelle \emph{sinus hyperbolique} $(\text{sinh})$ et \emph{cosinus hyperbolique} $(\text{cosh})$ les fonction définies sur $\mathbb{R}$ par 
\begin{equation}\label{defcoshetsinh}
  \text{sinh}(x) = \frac{e^x-e^{-x}}{2} ; \qquad  \text{cosh}(x) = \frac{e^x+e^{-x}}{2}. 
\end{equation}
\begin{enumerate}
\item[(4)] Démontrer les formules suivantes :
\begin{enumerate}
\item $\text{sinh} (x+y)=\text{sinh}(x) \text{cosh}(y)+\text{cosh}(x)\text{sinh}(y)$ ;
\item $\text{cosh} (x+y)=\text{cosh}(x) \text{cosh}(y)+\text{sinh}(x)\text{sinh}(y)$.
\end{enumerate}
\item[(5)] Donner des expressions de $\text{cosh}(2x)$ et $\text{sinh}(2x)$  en fonction de $\text{cosh}(x)$ et $\text{sinh}(x)$.
\item[(6)] Simplifier l'expression $f(x)=\cosh\Big(\ln(x+\sqrt{x^2-1})\Big)$ en utilisant la définition \eqref{defcoshetsinh}.
\end{enumerate}


\corrref{starterST-0015versionpourcorrige}
\end{exercice}
