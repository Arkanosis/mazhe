% This is part of (almost) Everything I know in mathematics
% Copyright (c) 2013-2014
%   Laurent Claessens
% See the file fdl-1.3.txt for copying conditions.

My main references for noncommutative geometry are \cite{Landi,ConnesNCG,ConnesMotives,itoNCG_Varilly}.

%%%%%%%%%%%%%%%%%%%%%%%%%%
%
   \section{Non commutative differential forms}
%
%%%%%%%%%%%%%%%%%%%%%%%%

\subsection{Universal differential forms}
%----------------------------------------

Let $\cA$ be an associative unital algebra on $\eC$. We are going to define step by step the \defe{universal algebra}{universal!algebra} of differential forms \nomenclature{$\Omega\cA$}{Universal algebra of $\cA$}
\[ 
  \Omega\cA=\bigoplus_p\Omega^p\cA.
\]
Existence the universal algebra will be proved by explicitly construction later. For unicity, we will prove an universality property.

First of all, $\Omega^0\cA=\cA$. Next, $\Omega^1\cA$ is the left $\cA$-module generated by the symbols $\delta a$ with $a\in\cA$ and relations
\begin{subequations}
\begin{align}
\delta(ab)&=(\delta a)b+a\delta b				\label{seq_deltaabi}\\  
\delta(\alpha a+\beta b)&=\alpha\delta a+\beta\delta b
\end{align}
\end{subequations}
for all $a$, $b\in\cA$ and $\alpha,\beta\in\eC$. A general element in $\Omega^1\cA$ is of the form $\sum_ia_i\delta b_i$, with $a_i$ and $b_i$ in $\cA$.

Equation \eqref{seq_deltaabi} with $a=1$ gives $(\delta 1)b=0$ for all $b\in\cA$, hence $\delta(1)=0$ and $\delta(\eC)=0$. So $\Omega^1\cA$ is a left $\cA$-module; we can give a structure of right $\cA$-module by defining
\[ 
  (\sum_ia_i\delta b_i)c=\sum_ia_i(\delta b_i)c,
\]
but equation \eqref{seq_deltaabi} gives $(\delta a)c=\delta(ac)-a\delta c$, therefore if $\omega=\sum_ia_i\delta b_i$,
\begin{equation}
\omega c=\sum_i a_i\delta(b_ic)-\sum a_ib_i\delta c.
\end{equation}
The rule \eqref{seq_deltaabi} is Leibnitz for the map $\delta\colon \cA\to \Omega^1\cA$, so we see $\delta$ as a derivation of $\cA$ with values in the bimodule $\Omega^1\cA$.


\subsubsection{Universal properties}
%///////////////////////////////////

The following proposition gives an universal property of $\Omega^1\cA$; in a certain sense, it is unique.

\begin{proposition}
Let $\modM$ be a $\cA$-bimodule and $\Delta\colon \cA\to \modM$ a derivation, i.e.
\[ 
  \Delta(ab)=(\Delta a)b+a\Delta b.
\]
There exists one and only one bimodule morphism $\rho_{\Delta}\colon \Omega^1\cA\to \modM$ such that $\Delta=\rho_{\Delta}\circ\delta$, i.e. the following diagram commutes:
\[ 
  \xymatrix{%
   \Omega^1\cA 		&	\\
   \cA \ar[r]_{\Delta}\ar[u]^{\delta}	&	\modM\ar@{.>}[ul]_{\rho_{\Delta}}
}
\]
\label{prop_modMununique}
\end{proposition}

\begin{proof}
First, remark that any bimodule morphism $\rho\colon \Omega^1\cA\to \modM$ makes $\rho\circ\delta$ a derivation with values in $\modM$. Indeed
\[ 
  (\rho\circ\delta)(ab)=\rho\big( (\delta a)b+a\delta b \big)
			=(\rho\circ\delta)(a)b+a(\rho\circ\delta)b
\]
Let us now prove the inverse: let $\Delta\colon \cA\to \modM$ be a derivation and $\rho_{\Delta}\colon \Omega^1\cA\to \modM$ be such that $\Delta=\rho_{\Delta}\circ\delta$. We want to prove unicity of this derivation. First, definition of $\Delta$ makes 
\begin{equation} \label{eq_rhpoDelta}
\rho_{\Delta}(\delta a)=\Delta a.
\end{equation}
This completely defines $\rho_{\Delta}$ from $\Delta$ because $\delta$ generates the whole $\Omega^1\cA$ as left $\cA$-module. Indeed the only way to extends $\rho_{\Delta}$ from \eqref{eq_rhpoDelta} as a morphism on $\Omega^1\cA$ is
\[ 
  \rho_{\Delta}\big( \sum_ia_i\delta b_i \big)=\sum_ia_i\Delta b_i.
\]
Now it is sufficient to prove that \eqref{eq_rhpoDelta} is a bimodule morphism:
\[ 
\begin{split}
\rho_{\Delta}\big( f(\sum a_i\delta b_i)g \big)&=\rho_{\Delta}\big( \sum_i fa_i[\delta(b_ig)-b_i\delta g] \big)\\
		&=\sum fa_i[\Delta(b_ig)-b_i\Delta g]\\
		&=\sum fa_i(\Delta b_i)g\\
		&=f\big( \sum ai\Delta b_i \big)g.
\end{split}  
\]

\end{proof}

The space $\Omega^p\cA$ is defined by
\[ 
  \Omega^p\cA:=\underbrace{\Omega^1\cA\ldots\Omega^1\cA}_{\text{$p$ times}}
\]
with multiplication rule
\begin{equation}
  (a_0\delta a_1)(b_0\delta b_1):=a_0(\delta a_1)b_0\delta b_1
		=a_0\delta(a_1b_0)\delta b_1-a_0a_1\delta b_0\delta b_1.
\end{equation}
This rule serves to show how to write elements of $\Omega^2\cA$ under the form $a\delta b_1\delta b_2$. Elements of $\Omega^p\cA$ are linear combination of elements of the form
\[ 
  \omega=a_0\delta a_1\delta a_2\ldots \delta a_p
\]
with $a_k\in\cA$, and the product $\Omega^p\cA\times\Omega^q\cA\to\Omega^{p+q}\cA$ is a juxtaposition and a rearrangement:
\[ 
\begin{split}
  (a_1\delta a_1\ldots\delta a_p)&(a_{p+1}\delta a_{p+2}\ldots\delta a_{p+q})\\
		&:=a_0\delta a_1\ldots (\delta a_p)a_{p+1}\delta a_{p+2}\ldots\delta a_{p+q}\\
	&=a_0\delta a_1\ldots \big[ \delta(a_pa_{p+1})-a_p(\delta a_{p+1}) \big]\delta a_{p+2}\ldots\delta a_{p+q}\\
	&=a_0\delta a_1\ldots\delta(a_pa_{p+1})\delta a_{p+2}\ldots\delta a_{p+q}\\
	&\quad -a_0\delta a_1\ldots\big[ \delta(a_{p-1}a_p)-a_{p-1}\delta a_p \big]\delta a_{p+1}\ldots\delta a_{p+q}.
\end{split}  
\]
With $p=q=3$ for example, we have
\[ 
\begin{split}
  (a_0\delta a_1\delta a_2)(a_3\delta a_4\delta a_5)&=a_0\delta a_1(\delta a_2)a_3\delta a_4\delta a_5\\
		&=a_0\delta a_1\big[ \delta(a_2 a_4)-a_2\delta a_3 \big]\delta a_4\delta a_5\\
		&=a_0\delta a_1\delta(a_2 a_3)\delta a_4\delta a_5\\
		&\quad -a_0\big[ \delta(a_1 a_2)-a_1\delta a_2 \big]\delta a_3\ldots\delta a_5\\
		&=a_0\delta a_1\delta(a_2a_3)\delta a_4\delta a_5\\
		&\quad-a_0\delta(a_1a_2)\delta a_3\ldots\delta a_5\\
		&\quad+a_0a_1\delta a_2\ldots\delta a_5,
\end{split}  
\]
and in general,
\begin{equation}    \label{Eq_decmProdConnForDelta}
\begin{split}
(a_0\delta a_1\ldots \delta a_p)&(a_{p+1}\delta a_{p+2}\ldots\delta a_{p+q})\\&=(-1)^pa_0a_1\delta a_2\ldots\delta a_{p+q}\\
	&\quad+\sum_{i=1}^p(-1)^{p-i}a_0\delta a_1\ldots\delta a_{i-1}\delta(a_ia_{i+1})\delta a_{i+2}\ldots\delta a_{p+q}.
\end{split}
\end{equation}
So $\Omega\cA$ is a left $\cA$-module. We turn it into $\cA$-bimodule by defining
\begin{equation}
\begin{split}
(a_0\delta a_1\ldots\delta a_p)b&=a_0\delta a_1\ldots(\delta a_p)b\\
		&=(-1)^pa_0a_1\delta a_2\ldots \delta a_p b\\
		&\quad+\sum_{i=1}^{p-1}(-1)^{p-i}a_0 \delta a_1\ldots \delta a_{i-1}\delta(a_ia_{i+1})\delta_{i+1}\ldots\delta a_p\delta b\\
		&\quad+a_0\delta a_1\ldots\delta a_{p-1}\delta(a_pb).
\end{split}
\end{equation}

Now we put a differential algebra structure on the $\cA$-bimodule $\Omega\cA$ by extending $\delta$ to
\begin{equation}
\begin{aligned}
 \delta\colon\Omega^p\cA&\to \Omega^{p+1}\cA \\ 
\delta(a_0\delta a_1\ldots\delta a_p)&:= \delta a_0\delta a_1\ldots \delta a_p. 
\end{aligned}
\end{equation}
One checks that
\[ 
  \delta^2=0
\]
and
\begin{equation}
  \delta(\omega_1\omega_2)=(\delta\omega_1)\omega_2+(-1)^p\omega_1\delta\omega_2
\end{equation}
for any $\omega_1\in\Omega^p\cA$ and $\omega_2\in\Omega\cA$.

If $\modE$ is a $\cA$-module, an element $\alpha\in\End_{\cA}(\modE)\otimes\Omega^1\cA$ acts on an element of $\modE\otimes_{\cA}\Omega^p\cA$ in the following way. If $\alpha=\sum_i\big( A_i\otimes_{\cA}a^i_0\delta a^i_1 \big)$, we define
\begin{equation}		\label{EqActallphaEOAp}
\alpha\big( \sum_j\xi_j\otimes_{\cA}\omega_j \big)=\sum_{ij}\big( A_i\xi_j\otimes_{\cA}a^i_0\delta a^i_1\omega_j \big)\in\modE\otimes_{\cA}\Omega^{p+1}\cA
\end{equation}
where $A_i\in\End_{\cA}(\modE)$, $a^i_j\in\cA$, $\xi_j\in\modE$ and $\omega_j\in\Omega^p\cA$. The element $\alpha$ acts in particular on $\modE$ \emph{via} the identification $\xi\in\modE\leftrightarrow\xi\otimes_{\cA}1\in\modE\otimes_{\cA}\Omega^0\cA$. In this case the result is denoted by $\alpha(\xi)$.

The following proposition gives the same type of result as proposition  \ref{prop_modMununique}.

\begin{proposition}
Let $(\Gamma,\Delta)$ a differential graded algebra and $\rho\colon \cA\to \Gamma^0$, a morphism of unital algebras.
There exists one and only one extension of $\rho$ into a differential graded algebra morphism $\tilde\rho\colon \Omega\cA\to \Gamma$ with $\tilde\rho\circ\delta=\Delta\circ\tilde\rho$:
\begin{equation}   \label{eq_diagrhotrho}
  \xymatrix{%
   \Gamma^p \ar[r]^{\Delta}\ar@{.>}[d]_{\tilde\rho}	&	\Gamma^{p+1}\ar[d]^{\tilde\rho}\\
   \Omega^p\cA \ar[r]_{\delta}				&	\Omega^{p+1}\cA
}
\end{equation}
\end{proposition}

\begin{proof}
The map $\rho\colon \cA\to \Gamma^0$ being given, we define
\begin{equation}
\begin{aligned}
 \tilde\rho\colon\Omega^p\cA&\to \Gamma^p \\ 
\tilde\rho(a_0\delta a_1\ldots\delta a_p)&:= \rho(a_0)\Delta(\rho (a_1))\ldots\Delta( \rho(a_p)). 
\end{aligned}
\end{equation}
It well defines $\tilde\rho$ since $\Omega^p\cA$ is generated by $a_0\delta a_1\ldots\delta a_p$. It sends products on products; for example with $p=2$,
\[ 
\begin{split}
  \tilde\rho\big[ (a_0\delta a_1)(b_0\delta b_1) \big]&=\rho(a_0)\Delta\rho(a_1b_0)\Delta\rho b_1\\
		&\quad-\rho(a_0a_1)\Delta\rho b_0\Delta\rho b_1\\
		&=\big[ \rho(a_0)\Delta\rho(a_1) \big]\big[ \rho(b_0)\Delta\rho b_1 \big].
\end{split}  
\]
Commutativity of diagram \eqref{eq_diagrhotrho} is as follows:
\[ 
\begin{split}
(\tilde\rho\circ\Delta)(a_0\delta a_1\ldots\delta a_p)&=\tilde\rho(\delta a_0\ldots\delta a_p)\\
		&=\Delta\rho a_0\ldots\Delta\rho a_p\\
		&=\Delta\big( \rho(a_0)\Delta\rho a_1\ldots\Delta\rho a_p \big)\\
		&=(\Delta\circ\tilde\rho)(a_0\delta a_1\ldots\delta a_p).
\end{split}  
\]

\end{proof}

\subsubsection{Cohomology}
%/////////////////////////

From definition,
\[ 
  \delta(a_0\delta a_1\ldots\delta a_p)=\delta a_0\ldots\delta a_p.
\]
If $\omega=a_0\delta a_1\ldots\delta a_p$, the only way to have $\delta\omega=0$ is $a_0=1$, i.e. $\omega=\delta(a_1\delta a_2\ldots\delta a_p)$. Then
\[ 
  H^p(\Omega\cA)=0
\]
when $p\neq 0$ because any closed form is exact. In the case $p=0$,
\[ 
  H^0(\Omega\cA)=\eC
\]
because $\delta(z)=0$ for all $z\in\eC$ while $z$ is not exact.

\subsubsection{Isomorphisms}
%//////////////////////////

\begin{lemma}
As bimodule, $\Omega^1(\cA)$ is isomorphic to $\ker m$ where $m\colon \cA\otimes_{\eC}\cA\to \cA$ is the multiplication map. The isomorphism is given by
\begin{equation}   \label{EqDEfphirov}
\begin{aligned}
 \varphi\colon \ker m&\to \Omega^1(\cA) \\ 
\sum_{j}a_i\otimes b_i&\mapsto \sum_{j}a_idb_i. 
\end{aligned}
\end{equation}

\end{lemma}

\begin{proof}
First remark that $\ker m$ is generated by elements of the form $1\otimes_{\eC}a-a\otimes_{\eC}1$. Indeed if $\sum_{j}a_jb_j=m\big( \sum_{i}a_j\otimes b_j \big)=0$, we have $\sum_{j}a_j\otimes_{\eC}b_j=\sum_{j}a_j(1\otimes_{\eC} b_j-b_j\otimes_{\eC}b_j)$.

Now consider the map 
\begin{equation}
\begin{aligned}
 \Delta\colon \cA&\to \ker m \\ 
\Delta a&= 1\otimes_{\eC}a-a\otimes_{\eC}1
\end{aligned}
\end{equation}
This map satisfies $\Delta(ab)=(\Delta a)b+a\Delta b$. Now we prove that the equation \eqref{EqDEfphirov} is surjective: $\sum_{j}a_jdb_j=\varphi\big( \sum_{j}a_j(1\otimes b_j-b_j\otimes 1) \big)$ where indeed $m\big( \sum_{j}a_j(1\otimes b_j-b_j\otimes 1) \big)=0$. 

The injectivity is evident because $\varphi\big( \sum_{j}a_j\otimes b_j \big)=\varphi\big( \sum_{j}a_j'\otimes b_j' \big)$ implies $\sum_{j}a_jdb_j=\sum_{j}a_j'db_j'$.

The fact that $\varphi$ provides an isomorphism of bimodule is
\[ 
	\varphi\big( c(a_i\otimes b_i)y \big)=\varphi\big( (xa_i)\otimes(b_i) \big)
		=xa_id(b_iy)
		=xa_i(db_i)y+x\underbrace{a_ib_i}_{=0}dy
		=x\varphi(a_i\otimes b_i)y.
\]
\end{proof}

\subsubsection{Involution}
%/////////////////////////

If the algebra $\cA$ has an involution (which is often the complex conjugation when $\cA$ is a function algebra), the universal algebra $\Omega\cA$ becomes an involutive algebra with the definition
\begin{align*}
(\delta a)^*&=-\delta(a^*),\\
(a_0\delta a_1\cdots\delta a_p)^*&=(\delta a_p)^*\cdots(\delta a_1)^*a_0^*.
\end{align*}


\subsection{Connes differential forms}
%+++++++++++++++++++++++++++++++++++

\begin{proposition}
The formula
\[ 
  \pi(a^0\delta a^1\cdots \delta a^{n})=a^0[D,a^1]\cdots[D,a^{n}]
\]
defines a $*$-representation of the reduced universal algebra on $\hH$.
\end{proposition}
\begin{proof}
No proof.
\end{proof}

\begin{proposition}
Let $J_0=\ker\pi$ and $J_0^{(k)}=\{ \omega\in\Omega^k(\cA)\tq \pi(\omega)=0 \}$. In this case, $J:=J_0+\delta J_0$ is a graded differential two-sided ideal of $\Omega^*(\cA)$.
\end{proposition}

\begin{proof}
The fact that $J$ is differential comes from the fact that $d^2=0$, but one has to remark that $J_0$ is not differential by itself because there exists some \defe{junk form}{junk form} $\omega$ such that $\pi(\omega)=0$ and $\pi(d\omega)\neq 0$. In order to see that $J$ is a left ideal, consider $\omega\in J^{(k)}$:
\[ 
  \omega=\omega_1+\delta\omega_2
\]
with $\omega_1\in J_0\cap\Omega^k$ and $\omega_2\in J_0\cap\Omega^{k-1}$. If $\omega'\in\Omega^l$, we have
\[ 
  \omega\omega'=\big( \omega_1\omega'+(-1)^k\omega_2\delta\omega' \big)+\delta(\omega_2\omega')\in J^{(k+l)}.
\]
The right side is proven in the same way.
\end{proof}

Since $J$ is an ideal, one can define
\[ 
  \Omega^*_D(\cA)=\Omega^*(\cA)/J.
\]
One can prove that for all $k\in\eN$, $\Omega^k_D(\cA)\simeq \pi(\Omega^k(\cA))/\pi\big( d(J_0\cap\Omega^{k-1}) \big)$. We consider the product 
\begin{equation}
	\langle T_1,\,T_2\rangle=\tr_{\omega}(T_2^*T_1| D |^-d)
\end{equation}
on $\pi(\Omega^k)$, and we define $\hH_k$ as the completion of $\pi(\Omega^k)$ for this product. This provides a Hilbert space in which $\pi\big( d(J_0\cap\Omega^{k-1}) \big)$ is a subspace. We denote by $P$ the projection parallel to this space.

\begin{proposition}
For all $\omega_i\in\pi\big( \Omega^k(\cA) \big)$,
\[ 
  \langle P\omega_1,\,\omega_2\rangle=\langle P\omega_1,\,P\omega_2\rangle.
\]

\end{proposition}
\begin{proof}
No proof.
\end{proof}
We denote by $\Lambda^k$ the completion of $\Omega^k_D$ with respect to this inner product. In fact, the proposition shows that the inner product can be used on $\pi(\Omega^k)$.



Let $(\cA,\hH,D)$ be a spectral triple, and $\Omega\cA$, the universal algebra of $\cA$. We consider the map
\begin{equation}	\label{EqDEfpirep}
\begin{aligned}
 \pi\colon\Omega\cA&\to \oB(\hH) \\ 
a_0\delta a_1\ldots\delta a_p&\mapsto a_0\circ[D,a_1]\circ\ldots\circ [D,a_p] 
\end{aligned}
\end{equation}
for $a_i\in\cA$. Since the operations $\delta$ and $D$ both are derivations, one can expect that $\pi$ will be a homomorphism.

\begin{proposition}
The operation $\pi$ is a homomorphism.
\end{proposition}
\begin{proof}
We will just check it in the case of $1$-forms. First, remark that
\[ 
 \begin{split}
(a_0\delta a_1)(b_0\delta b_1)&=a_0\big( \delta(a_1b_0)-a_1\delta b_0 \big)\delta b_1\\
		&=a_0\delta(a_1b_0)\delta b_1-a_0a_1\delta b_0\delta b_1.
\end{split} 
\]
Thus we have
\[ 
 \begin{split}
\pi\big( (a_0\delta a_1)(b_0\delta b_1) \big)&=a_0[D,a_1b_0][D,b_1]-a_0a_1[D,b_0][d,b_1]\\
		&=a_0[D,a_1]b_0[D,b_1]\\
  		&=\pi(a_0\delta a_1)\pi(b_0\delta b_1)
\end{split} 
\]

\end{proof}

\subsubsection{Junk forms}
%/////////////////////////

One cannot define $\pi(\Omega\cA)$ as differential forms because there exists some $\omega\in\Omega\cA$ such that $\pi(\omega)=0$ and $\pi(\delta\omega)\neq 0$. Such a form is said to be \defe{junk}{junk form}. We define
\begin{equation}  \label{eq_defJConnes}
  J_0^p=\{ \omega\in\Omega^p\cA\tq \pi(\omega)=0 \}
\end{equation}
 and $J=J_0+\delta J_0$. Then we define the \defe{Connes differential forms}{Connes differential form} as\nomenclature{$\Omega_D\cA$}{Connes differential forms}
\begin{equation}
\Omega_D\cA=\Omega\cA/J.
\end{equation}

\begin{proposition}
Let $J_0=\bigoplus_pJ_0^p$, the two-sided graded ideal of $\Omega\cA$ generated by \eqref{eq_defJConnes}.  The set $J=J_0/\delta J$ is a two-sided graded ideal of $\Omega\cA$. 
\end{proposition}

\begin{proof}
We begin by proving that $J$ is a differential algebra for the same $\delta$ as for $\cA$; the fact that $\delta^2=0$ is not a question. We have to prove that $\delta$ is internal in $J$. A general element of $J$ is $\omega=\alpha+\delta\beta$ with $\pi(\alpha)=\pi(\beta)=0$. In this case, $\delta\omega=\delta\alpha$ is of the same form. Now we want to prove that $J$ is a two-sided ideal. Let $\omega=\omega_1+\delta\omega_2\in J^p$ with $\omega_1\in J_0^p$ and $\omega_2\in J_0^{p-1}$ and consider $\eta\in\Omega^q\cA$. We have 
\[
  \omega\eta=\omega_1\eta+(\delta\omega_2)\eta
		=\omega_1\eta+\delta(\omega_2\eta)-(-1)^{p-1}\omega_2\delta\eta.
\]
The first term fulfil $\pi(\omega_1\eta)=\pi(\omega_1)\pi(\eta)=0$, the second one is the $\delta$ of something whose $\pi$ is zero  while the $\pi$ of the third one is zero. So $\omega\eta\in J$. In the same manner, we conclude that $\eta\omega\in J$ too.
\end{proof}

\begin{lemma}
We have
\begin{equation}
  \Omega\cA/J\simeq \pi(\Omega\cA)/\pi(\delta J_0).
\end{equation}
\label{lem_OCAisomppiOA}
\end{lemma}

\begin{proof}
On element of $\Omega\cA$ is of the form $[\omega]\sim [\omega+\omega_1+\delta\omega_2]$ with $\pi(\omega_1)=\pi(\omega_2)=0$. We define
\begin{equation}
\begin{aligned}
 \psi\colon \Omega\cA/J&\to \pi(\Omega\cA)/\pi(\delta J_0) \\ 
[\omega]&\mapsto [\pi(\omega)]_B
\end{aligned}
\end{equation}
where the class $[\cdots]_B$ is modulo $\pi(\delta J_0)$, in other words when $\pi(\omega)=0$, we have $[A]_B=[A+\pi(\delta\omega)]$.  The map $\psi$ is well defined because, when $\pi(\omega_1)=\pi(\omega_2)=0$,
\[ 
\psi[\omega+\omega_1+\delta\omega_2]=[\pi(\omega+\omega_1+\delta\omega_2)]_B
		=[\pi(\omega)]_B+[\pi(\delta\omega_2)]_B
		=[\pi(\omega)]
		=\psi[\omega].
\]

First suppose that $\psi[\omega]=0$, i.e $\psi[\omega]=[\pi(\delta\eta)]_B$ with $\pi(\eta)=0$. Then $\omega=\delta\eta+\sigma$ with $\sigma=0$. This is the definition of $[\omega]=0$. This proves that $\psi$ is injective. For surjectivity, consider $[\omega]_B\in\pi(\Omega\cA)/\pi(\delta J_0)$ and $\pi(\omega)$ a representative in $\pi(\Omega\cA)$. For this $\omega$, we have $\psi[\omega]=[\omega]_B$.
\end{proof}

\subsubsection{Grading the Connes differential forms}
%////////////////////////////////////////////////////

\begin{lemma}			\label{LemOmpmdDp}
The isomorphism of lemma \ref{lem_OCAisomppiOA} induces a grading 
\begin{equation} \label{eq_OpDAsimeqOpA}
  \Omega^p_D\cA\simeq \Omega^p\cA/J^p,
\end{equation}
and the differential
\begin{equation}
\begin{aligned}
 d\colon \Omega^p_D\cA&\to \Omega^{p+1}_D\cA \\ 
[\omega]&\mapsto [\delta\omega] 
\end{aligned}
\end{equation}
is well defined.
  \label{lem_isomgraODA}
\end{lemma}

\begin{proof}
 The well definiteness of the differential is because when $\pi(\omega_1)=\pi(\omega_2)$,
\[ 
d[\omega+\omega_1+\delta\omega_2]=[\delta\omega+\delta\omega_1+\delta^2\omega_2]
		=[\delta\omega].
\]
The isomorphism \eqref{eq_OpDAsimeqOpA} is given by   $\psi[\omega^p]=[\pi(\omega^p)+\pi(\delta\eta)]$ with $\pi(\eta)=0$ when $[\omega^p]\in\Omega^p\cA/J^p$, i.e. when $\omega^p\in\Omega^p\cA$.
\end{proof}

\subsubsection{\texorpdfstring{$0$}{t}-forms}
%///////////////////////

We have $J^0=J\cap\Omega^0\cA=J\cap\cA$ and $J^0=\{ a\in\cA\tq \pi(a)=0 \}$. As operators on $\hH$, $J^0=\{ 0 \}$. Therefore $\Omega^0_D\cA=\Omega^0\cA=\cA$.

Let us now briefly study the spaces of low degree forms.

\subsubsection{\texorpdfstring{$1$}{1}-forms}
%//////////////////////

From the isomorphism of lemma \ref{lem_isomgraODA}, we begin to study $\delta J_0^0$
\[ 
  J_0^0=\{ \omega\in\cA\tq\pi(\omega)=0 \}.
\]
The crucial point is that $\pi(a)=0$ implies $a=0$ when $a\in\cA$ (is is not true for any $\omega\in\Omega\cA$ !), so
\begin{equation}
\Omega^1\cA=\pi(\Omega^1\cA)
\end{equation}
and Connes $1$-forms are of the form
\[ 
  \omega_1=\sum_ja_0^j[D,a_1^j]
\]
with $a_i^j\in\cA$.

\subsubsection{Example on the canonical triple}
%///////////////////////////////////////////

%\begin{probleme}
%À ce propos, je te préviens que tu as un fichier nomé Dirac.dvi quelque part dans ta littérature qui explique la formule
%\[ 
%  [D,f]=c(df)
%\]
%quand $c$ est une action de Clifford sur un fibré vectoriel.
%\end{probleme}

Now we will use the $\gamma$ defined in subsection \ref{susec_Cliffmodule}. When we consider the canonical triple $(\cA,\hH,D)$ on a manifold $M$, $\cA\subset\Fun(M)$. We know that $\cA$ acts on $\hH$ by $(f\psi)(x)=f(x)\psi(x)$, so that $[D,f]\psi=(\gamma^{\mu}\partial_{\mu}f)\psi$. So we say that 
\[ 
  [D,f]=\gamma^{\mu}\partial_{\mu}f=\gamma(df).
\]
The Dirac operator act on functions as follows (see equation \eqref{eq_defDirac_f}):
\[ 
  (Df)(x)=g_{\alpha\beta}(x)\gamma^{\beta}_x(e_{\alpha}\cdot f),
\]
this definition is intended to get a Leibnitz rule for $D(f\psi)$. We have:
\[ 
[D,f]\psi(x)=D(f\psi)(x)-f(x)D\psi(x)\\
		=(Df)(x)\psi(x),
\]
so $[D,f]\psi=(Df)\psi$ and as operator on $\hH$, $[D,f]$ is the multiplicative operator by $Df$. When we consider a local orthonormal basis $e_{\alpha}$, we have
\[ 
(Df)(x)=g_{\alpha\beta}(x)\gamma_x^{\beta}(e_{\alpha}\cdot f)
		=\gamma^{\mu}\partial_{\mu}f(x).
\]
From all that we conclude that 
\[
  [D,f]=\gamma^{\mu}\partial_{\mu}f,
\]
and therefore that, on the canonical triple, 
\begin{equation}
\pi(\delta f)=[D,f]=\gamma^{\mu}\partial_{\mu}f.
\end{equation}
But $\gamma(df)=\partial_{\mu}f\gamma(dx^{\mu})=\gamma^{\mu}\partial_{\mu}f=\pi(\delta f)$. We will soon define the differential $d$ of $\Omega_D\cA$, so from now we denote by $d_M$ the usual differential of $M$ and we write $\pi(\delta f)=\gamma(d_Mf)$ and finally,
\begin{equation}
\pi(f_0\delta f_1\ldots \delta f_p)=f_0\gamma(d_Mf_1)\ldots\gamma(d_Mf_p).
\end{equation}
Note that $d_M\in\Gamma(M,\Cliff(M))$ and, since $\gamma$ is a morphism, 
\begin{equation} \label{EqpildotsdeltagamdM}
  \pi(f_0\delta f_1\ldots\delta f_p)=f_0\gamma(d_Mf_1\cdot\ldots\cdot d_Mf_p)
\end{equation}
where $\cdot$ denotes the Clifford product. 


\subsubsection{Differential \texorpdfstring{$0$}{0}-forms}
%-------------------------------------

\begin{lemma}
\[ 
\gamma(\Lambda^1(M))\simeq\Omega^1_D\cA
\]
\end{lemma}

\begin{proof}

A general $1$-form has the form $\sum_j f_0^jd_Mf_1^j$. Since $\gamma\colon \Gamma(M,\Cliff(M))\to \oB(\hH)$, we claim that the isomorphism is given by
\[ 
  \psi\big( f_0[D,f_1] \big)=\gamma\big( f_0d_Mf_1 \big).
\]
Surjectivity poses no problems because $f_0d_Mf_1$ is the general form of an element of $\Lambda^1(M)$. Now suppose that $\psi(f_0[D,f_1])=\gamma(f_0d_Mf_1)=0$. From linearity of $\gamma$,
\[ 
  f_0\gamma^{\mu}\partial_{\mu}f_1=0.
\]
At each point, either $f_0=0$ or $[D,f_1]=0$, so globally $f_0[D,f_1]=0$.
\end{proof}

\subsubsection{Differential \texorpdfstring{$1$}{1}-forms}
%-------------------------------------

Let $f\in\cA$ and 
  $\alpha=\frac{ 1 }{2}\big( f\delta f-(\delta f)f \big)$.
We have 
$(f\delta f)(x,y,z)=f(x)(\delta f)(y,z)
		=f(x)\big( f(y)-f(z) \big)$,
while
\[ 
  (\delta f)f(x,y,z)=\big( f(x)-f(y) \big)f(z).
\]

\begin{probleme}
	When we will speak about two points spaces, we will see that $(f\delta g)(x,y)=f(x)\big( g(y)-g(x) \big)$, and more or less the same for $(\delta f)g$. Thus what is done here is wrong and the correct result is
	\[ 
		2\alpha(x,y)=2f(x)f(y)-f(x)^{2}-f(y)^{2}.
	\]
	This does not change the conclusion, but it asks for a precise definition of $f\delta g$. Notice that, $f$ being a zero-form, and $\delta g$ a $1$-form, the product should be a $1$-form, and not a $2$-form.
\end{probleme}

This proves that $\alpha\neq 0$. The following computation uses the fact that the Leibnitz rule for $\delta$ is graded
\[ 
\delta\alpha=\frac{ 1 }{2}\big( \delta f\delta f+f\delta^2 f-(\delta^2 f)f+\delta f\delta f \big)
		=\delta f\delta f,
\]
so, on the one hand,
\[ 
\pi(\delta f)=\gamma^{\mu}\partial_{\mu}f\gamma^{\nu}\partial_{\nu}f
		=\frac{ 1 }{2}(\gamma^{\mu}\gamma^{\nu}+\gamma^{\nu}\gamma^{\mu})\partial_{\mu}f\partial_{\nu}f
		=-g^{\mu\nu}\partial_{\mu}f\partial_{\nu}f \mtu_{2^{[N/2]}}
\]
where $\mtu_{2^{[N/2]}}$ is the unit in the Clifford algebra of $\eR^n$. On the other hand,
\[ 
  \pi(\alpha)=\frac{ 1 }{2}\big( f\gamma^{\mu}\partial_{\mu}f-(\gamma^{\mu}\partial_{\mu}f)f \big)=0.
\]
This proves that $\alpha$ is a junk $1$-form.

\begin{probleme}
	It is also said that this is the general form of a junk, but I didn't succeed to prove it.
\end{probleme}

\subsubsection{Differential \texorpdfstring{$2$}{2}-forms}
%-----------------------------------

Let now take $f_{1}$, $f_{2}\in \cA$ and look at
\[ 
\begin{split}
  \gamma\big( d_{M}f_{1}\cdot d_{M}f_{2} \big)&=\gamma^{\mu}\gamma^{\nu}\partial_{\mu}f_{1j\partial_{\nu}}f_{2}\\
		&=\frac{1}{2}\big( \gamma^{\mu}\gamma^{\nu}+\gamma^{\mu}\gamma^{\nu}-\gamma^{\nu}\gamma^{\mu}+\gamma^{\nu}\gamma^{\mu} \big)\partial_{\mu}f_{1}\partial_{\nu}f_{2}\\
		&=\frac{1}{2}\big( \gamma^{\mu}\gamma^{\nu}+\gamma^{\nu}\gamma^{\mu} \big) \partial_{\mu}f_{1}\partial_{\nu}f_{2}
		 +\frac{1}{2}\big( \gamma^{\mu}\gamma^{\nu}-\gamma^{\nu}\gamma^{\mu} \big) \partial_{\mu}f_{1}\partial_{\nu}f_{2}.
\end{split}  
\]
The first term is
\[ 
  -g^{\mu\nu}\mtu\partial_{\mu}f_{1}\partial_{\nu}f_{2}=-g(dx^{\mu},dx^{\nu})\partial_{\mu}f_{1}\partial_{\nu}f_{2}\mtu=-g(d_{M}f_{1},d_{M}f_{2})\mtu.
\]
For the second term, first recall that
\[ 
  d_{M}f_{1}\wedge d_{M}f_{2}=d_{M}f_1\otimes d_{M}f_2-d_Mf_2\otimes d_Mf_1 
\]
where the $\otimes$ is, up to equivalence class, the product in Clifford. So the first term is
\[ 
  \frac{1}{2}\gamma( d_Mf_1\cdot d_Mf_2-d_Mf_2\cdot d_Mf_1 )=\gamma( d_Mf_1\wedge d_Mf_2 ).
\]
Finally we have
\begin{equation} \label{EqGamgGam}
\gamma(d_Mf_1\cdot d_Mf_2)=-g(d_Mf_1,d_Mf_2)\mtu+\gamma(d_Mf_1\wedge d_Mf_2).
\end{equation}
On the other hand, a general element of $\wedge^{2}(M)$ (the skew-symmetric differential $2$-forms on $M$) is
\[ 
  \sum_{j}^{}f_0^{j}\,d_Mf_1^{j}\wedge f_2^{j}
\]
with $f_0^{j}$, $f_1^{j}$, $f_2^{j}\in \cA$.

\begin{lemma}
\[ 
  \Omega_{D}^{2}\cA\simeq \gamma\big( \wedge^{2}(M) \big)
\]

\end{lemma}

\begin{proof}
We use the isomorphism
\[ 
  \Omega_D^2\cA\simeq\pi\big( \Omega^2\cA \big)/\pi\big( \delta(J_{0}\cap\Omega^1\cA) \big).
\]
where the elements of $J_{0}\cap\Omega^1\cA$ are of the form $\alpha_{f}=\frac{1}{2}\big( f\delta f-(\delta f)f \big)$. A general element of $\Omega_D^2\cA$ is a class of (sum of)
\[ 
  \pi(f_0\delta f_1\delta f_2),
\]
so from equation \eqref{EqpildotsdeltagamdM}, the idea is to define the candidate isomorphism $\psi$ by
\begin{equation}
\psi\Big( \big[ \gamma(f_0d_M f_1\cdot d_Mf_2) \big] \Big)=\gamma\big( d_Mf_1\wedge d_Mf_2 \big),
\end{equation}
and its linear extension. Let us compute $\psi[0]$ or $\psi[\pi\delta(\alpha_{f})]$. We have $\delta\alpha_{f}=\delta f\delta f$, so
\[ 
  \pi\delta(\alpha_{f})=\gamma(d_M f\cdot d_Mf).
\]
Thus
 \[ 
   \psi\big( [\delta\pi\alpha_{f}] \big)=\psi\Big( \big[ \gamma(d_Mf\cdot d_Mf) \big] \Big)
		=\gamma(d_Mf\wedge d_Mf)=0.
\]
We conclude that $\psi$ is well defined and injective. Surjectivity is clear.
\end{proof}

One can also prove the following generalization.

\begin{lemma}
\begin{equation}
\Omega^p_D\cA\simeq\wedge^{p}(M).
\end{equation}
\end{lemma}
\begin{proof}
No proof.
\end{proof}

\subsection{Example: two points space}		\label{SubSecTripleDeuxPoints}
%--------------------------------------

Let $Y=\{ 1,2 \}$, a space containing only two points. The space of continuous functions is $\cA=\eC\oplus\eC$ and a continuous function is of the form $f=(f_1,f_2)$ with $f_i=f(i)\in\eC$. We can build an even spectral triple of dimension zero $(\cA,\hH,D,\Gamma)$ as follows. Let $\hH_1$ and $\hH_2$ be two finite dimensional Hilbert space and $\hH=\hH_1\oplus\hH_2$. We define the action of $f\in\cA$ on $\psi\in\hH$ by
\[ 
  f\begin{pmatrix}
\psi_1\\\psi_2
\end{pmatrix}=
\begin{pmatrix}
f_1\psi_1\\f_2\psi_2
\end{pmatrix}
\]
if $\psi_i\in\hH_i$. This operator $f$ is clearly bounded on $\hH$. Let $M\colon \hH_1\to \hH_2$ be a linear map and define the operator $D$ as
\[ 
  D=\begin{pmatrix}
0& M^*\\
M&0
\end{pmatrix}.
\]
We want $[D,f]$ to be bounded, but
\[ 
  [D,f]\begin{pmatrix}
\psi_1\\\psi_2
\end{pmatrix}
=
D\begin{pmatrix}
f_1\psi_1\\f_2\psi_2
\end{pmatrix}-
\begin{pmatrix}
f_1(D\psi_1)\\
f_2(D\psi)_2
\end{pmatrix}.
\]
The component $(D\psi_1)_1$ does not affect the commutator; it is the reason why we had chosen an anti-diagonal operator $D$.

As parity map $\Gamma\colon \hH\to \hH$, we choose
\[ 
  \Gamma=\begin{pmatrix}
\id|_{\hH_1}\\ &-\id|_{\hH_2}
\end{pmatrix}.
\]
Now consider $f\in\cA$ and compute the commutator
\[ 
\begin{split}
  [D,f]\begin{pmatrix}
\psi_1\\\psi_2
\end{pmatrix}
&=
\begin{pmatrix}
f_2 M^*\psi_2\\f_1M\psi_1
\end{pmatrix}
-
\begin{pmatrix}
f_1M^*\psi_2\\f_2M\psi_1
\end{pmatrix}\\
&=(f_2-f_1)\begin{pmatrix}
M^*\psi_2\\-M\psi_1
\end{pmatrix}\\
&=(f_1-f_2)\begin{pmatrix}
0&-M^*\\M&0
\end{pmatrix}
\begin{pmatrix}
\psi_1\\\psi_2
\end{pmatrix}.
\end{split}  
\]
So 
\[ 
  \| [D,f] \|=| f_1-f_2 |\lambda
\]
where $\lambda$ is the larger eigenvalue of $\sqrt{MM^*}$. Hence the noncommutative distance between $1$ and $2$ is 
\[ 
  d(1,2)=\sup\{ | f_1-f_2 |\tq \| [D,f] \|\leq 1 \}=\frac{1}{ \lambda }.
\]
As real structure, one can take
\[ 
  J\begin{pmatrix}
\psi_1\\\psi_2
\end{pmatrix}=
\begin{pmatrix}
\overline{ \psi_2 }\\\overline{ \psi_1 }
\end{pmatrix}.
\]
Let $Y=\{ 1,2 \}$, its triple $(\cA,\hH,D)$ with $\cA=\eC\oplus\eC$. We are going to study $\Omega^1\cA$. We define $\delta f$ as being the map
\begin{equation}
	(\delta f)(x,y)=f(x)-f(y).
\end{equation}
The space $\Omega^1\cA$ is a left $\cA$-module by the definition
\begin{subequations}
\begin{equation}
(f\delta g)(x,y)=f(x)\delta g(x,y).          \label{SubEqfdeltagxya}  
\end{equation}
Now the Leibnitz rule imposes the following structure of right module:
\begin{equation}
	(\delta f)g(x,y)=(\delta f)(x,y)g(y).   \label{SubEqfdeltagxyb}
\end{equation}
\end{subequations}
Indeed $(\delta f)g=\delta(fg)-f\delta g$, so that
\begin{align*}
(\delta f)g(x,y)&=\delta(fg)(x,y)-f(x)(\delta g)(x,y)\\
		&=f(x)g(x)-f(y)g(y)-f(x)g(x)+f(x)g(y)\\
		&=\big( f(x)-f(y) \big)g(y)\\
		&=(\delta f)(x,y)g(y).
\end{align*}

\begin{probleme}
	I do not understand why things are like that, but if we look at an usual $1$-form on $\eR^N$, we need two vectors in order to get a number. For the $1$-form $\omega$ we need $x$ and $X$ in order to get $\omega_x(X)\in\eR$. As far as the multiplication by a function is concerned we write
\[ 
  (f\omega)_{x}(X)=f(x)\omega_{x}(X).
\]
Thus we have something like $(f\omega)(x,X)=f(x)\omega(x,X)$. This is more or less the philosophy of \eqref{SubEqfdeltagxya}. For \eqref{SubEqfdeltagxyb}, the fact that $g$ takes the $y$ instead of the $x$ is difficult to understand.
\end{probleme}
It gives 
\begin{subequations}
\begin{align}
  (f\delta g)(x,y)&=f(x)\big( g(y)-g(x) \big)\\
 (\delta f)g(x,y)&=\big( f(y)-f(x) \big)g(y),
\end{align}
\end{subequations}
and thus
\begin{align*}
\delta(fg)(x,y)&=(\delta f)g(x,y)+f(\delta g)(x,y)\\
		&=f(y)g(y)-f(x)g(x)\\
		&=(fg)(y)-(fg)(x),
\end{align*}
which is coherent.

The $1$-forms are functions of two variables which are zero on the diagonal. In the case of our two point space, they takes non zero values only at $(1,2)$ and $(2,1)$, so a basis of $\Omega^1\cA$ is given by $\omega$ and $\eta$ with 
\begin{align}
\omega(1,2)&=1 &\eta(1,2)&=0\\
\omega(2,1)&=0 &\eta(2,1)&=1.
\end{align}
Such a basis can be constructed by defining $e(0)=0$, $e(1)=1$ and considering $e\delta e$ and $(1-e)\delta(1-e)$.

\subsection{Example: manifold}
%---------------------------------

Let $M$ be a compact spin Riemannian manifold and $\cA$ the algebra of (continuous or more) functions on $M$. We also consider $D$, the Dirac operator on $\hH$, the space of the square integrable spinors over $M$. The algebra $\cA$ acts on $\hH$ by multiplication.

Let $C$ be the vector bundle over $M$ whose fibre are given by $C_x=\Cliff^{\eC}(T^*_xM)$. It is possible to define the notion of bounded measurable section of $C$. Let $\rho\colon M\to C$ one of them.

\begin{probleme}
	What is a bounded measurable section of $C$ ?
\end{probleme}

Since $f\in\cA$ is a function on $M$, the element $df$ is a section of $T^*M$ and can, up to the quotient \eqref{defI}, be seen as a section of $C$. When $df$ is seen in this way, it is denoted by $d_cf$ and we have
\[ 
  \pi(f^0df^1)=f^0[D,f^1]=i^{-1}\gamma(f^0d_cf^1).
\]

\begin{probleme}
It is true that
\[ 
  [D,f]=c(df)
\]
when $c$ is a Clifford action on a vector bundle. I should try to understand it better.
\end{probleme}


\section{Fredholm modules}
%-------------------------

Most of theory here and related topics is taken from \cite{ConnesNCG,Landi} an other source about Fredholm modules and spectral triples is \cite{Whittaker}. 

If $X$ and $Y$ are Banach spaces, an operator $T\colon X\to Y$ is a \defe{Fredholm operator}{Fredholm!operator} if there exists a bounded linear operator $S\colon Y\to X$ such that the operators
\begin{subequations}
	\begin{align}
		\id_X-ST\\
		\id_X-TS
	\end{align}
\end{subequations}
are compact.

%---------------------------------------------------------------------------------------------------------------------------
\subsection{Introductory example}
%---------------------------------------------------------------------------------------------------------------------------

Let $M$ be a compact manifold and $\cA=C(M)$ the $C^*$-algebra of continuous functions on $M$. We consider $E^{\pm}$, two Hermitian complex vector bundles on $M$ and an elliptic pseudo-differential operator of order $0$, $P\colon  C^{\infty}(M,E^+)\to  C^{\infty}(M,E^-)$. Such an operator can be extended to an operator 
\begin{equation}
	P\colon L^2(M,E^+)\to L^2(M,E^-)
\end{equation}
which has a \hyperlink{DefParametrix}{parametrix} $Q$. Consequently, $P$ is a Fredholm operator (in fact, \wikipedia{en}{Fredholm_operator}{wikipedia} says that all elliptic operators can be extended to Fredholm operator.)

The algebra $C(M)$ is naturally represented on $L^2(M,E^{\pm})$ by $\pi^{\pm}(f)(x)\xi=f(x)\xi(x)$ (pointwise multiplication of $\xi$ by $f$). Let us now consider the Hilbert space 
\begin{equation}
	\hH=\hH^+\oplus\hH^-=L^2(M,E^{+})\oplus L^2(M,E^{-})
\end{equation}
and its $\eZ_2$-graduation
\begin{equation}
	\gamma=\begin{pmatrix}
		1	&	0	\\ 
		0	&	-1	
	\end{pmatrix}.
\end{equation}
We represent $C(M)$ on $\hH$ by
\begin{equation}
	\pi(f)=\begin{pmatrix}
		\pi^+(f)	&	0	\\ 
		0	&	\pi^-(f)	
	\end{pmatrix},
\end{equation}
and we pose
\begin{equation}
	F=\begin{pmatrix}
		0	&	Q	\\ 
		P	&	0	
	\end{pmatrix}.
\end{equation}
In this case, the operators $[F,\pi(f)]$ and $F^2-\mtu$ are compact for every $f\in  C^{\infty}(M)$ because the operators
\begin{subequations}
	\begin{align}
		\pi^-P-P\pi^+\\
		\pi^+Q-Q\pi^-
	\end{align}
\end{subequations}
are compact\quext{I do not know why.}.

%---------------------------------------------------------------------------------------------------------------------------
\subsection{Definition}
%---------------------------------------------------------------------------------------------------------------------------

Let $\cA$ be an involutive algebra over $\eC$. A \defe{odd Fredholm module}{Fredholm!module!odd} over $\cA$ is 
 \begin{enumerate}
\item an involutive representation $\pi$ of $\cA$ on an Hilbert space $\hH$,
\item an operator $F\colon \hH\to \hH$ such which satisfies 
\begin{itemize}
\item $F=F^*$, $F^2=\mtu$,
\item $[F,\pi(a)]$ is a compact operator for each $a\in\cA$
\end{itemize}

\end{enumerate}

An \defe{even Fredholm modules}{Fredholm!module!even} is an odd Fredholm module with a $\eZ/2$ grading $\gamma\colon \hH\to \hH$ such that
\begin{itemize}
\item $\gamma=\gamma^*$, $\gamma^2=\mtu$,
\item $[\gamma,\pi(a)]=0$ for all $a\in\cA$,
\item $\gamma F=-F\gamma$.
\end{itemize}
We will almost always write $a\xi$ instead of $\pi(a)\xi$ when the underlying representation is clear. The Fredholm module $(\hH,F)$ is said to be \defe{$p$-summable}{$p$-summable Fredholm module} when $\forall a\in\cA$,
\[ 
  [F,a]\in\oL^p(\hH).
\]
One says that the Fredholm module $(\hH,F)$ is $\theta$-summable when $[F,a]\in J^{1/2}$ for all $a\in\cA$. The set $J^{1/2}$ is the two-sided ideal of compact operators $T$ such that
\[ 
  \mu_n(T)=O\big( (\ln n)^{-1/2} \big).
\]

\subsection{Cycle associated with Fredholm module}
%-----------------------------------------------

Let $(\hH,F)$ be a Fredholm module. We are going to build a cycle in the sense of section \ref{SecCyclicHomol} associated with $(\hH,F)$. For the graded algebra $\Omega=\oplus_k\Omega^k$, we begin by $\Omega^0=\cA$ and for $k>0$, we define $\Omega^k$ as the vector space spanned by operators of the form
\[ 
  a^0[F,a^1]\cdots[F,a^k]
\]
with $a^{j}\in\cA$. Except from $a^0$, this is a product of $k$ elements of $\oL^{n+1}$ which belongs to $\oL^{(n+1)/k}$ by equation \eqref{EqPropLLLsvn}. The fact that $\oL^q$ is an ideal makes that 
\[ 
  \Omega^k\subset\oL^{(n+1)/k}(\hH).
\]
The product in $\Omega$ is defined as the usual operator product.

\begin{lemma}
If $\omega\in\Omega^k$ and $\omega'\in\Omega^{k'}$, $\omega\omega'\in\Omega^{k+k'}$.
\end{lemma}

\begin{proof}
The fact that $[F,.]$ is a derivation on $\cA$ makes that 
\[ 
\begin{split}
  a^0[F,a^1]\cdots[F,a^{k}]a^{k+1}&=\sum_{j=1}^{k}(-1)^{k-j}a^0[F,a^1]\cdots[F,a^{j}a^{j+1}]\cdots[F,a^{k+1}]\\
					&\quad+(-1)^ka^0a^1[F,a^2]\cdots[F,a^{k+1}].
\end{split}
\]
This is the same computation as in equation \eqref{Eq_decmProdConnForDelta}. This is a sum of terms of the form $r^0[F,r^1]\cdots[F,r^k]$, thus the product $\omega\omega'$ reads
\[ 
  \underbrace{a^0[F,a^1]\cdots[F,a^{k}]b^0}_{\textrm{Sum of }a^0[F,r^1]\cdots[F,r^k]}[F,b^1]\cdots[F,b^k]
\]
which belongs to $\Omega^{k+k'}$.
\end{proof}
From here we have a graded algebra $\Omega^*$ with a product $\Omega^j\times\Omega^{k'}\to\Omega^{k+k'}$. As differential, we choose
\begin{equation}  \label{EqFreddDefbel}
d\omega=F\omega-(-1)^k\omega F=[F,a^0][F,a^1]\cdots[F,a^{k}].
\end{equation}
The second equality can be checked by virtue of $F[F,a]=-[F,a]F$. This differential is a \defe{graded differential}{graded differential}, i.e.
\begin{equation}
  d(\omega_1\omega_2)=(d\omega_1)\omega_2+(-1)^{k_1}\omega_1d\omega_2
\end{equation}
for all $\omega_1\in\Omega^{k_1}$. Indeed,
\[ 
\begin{split}
d(\omega_1\omega_2)&=F\omega_1\omega_2-(-1)^{k_1+k_2}\omega_1\omega_2 F\\
		&=F\omega_1\omega_2-(-1)^{k_1+k_2}\omega_1b^0[F,b^1]\cdots[F,b^{k_2}]F\\
		&=F\omega_1\omega_2-(-1)^{k_1}\omega_1\big( -[F,b^0]+Fb^0 \big)[F,b^1]\cdots[F,b^{k_2}]\\
		&=F\omega_1\omega_2+(-1)^{k_1}\omega_1d\omega_2-(-1)^{k_1}\omega_1 F\omega_2\\
		&=(d\omega_1)\omega_2+(-1)^{k_1}\omega_1 d\omega_2.
\end{split}  
\]
We also check that $d^2=0$ in the following way:
\begin{equation}
	\begin{aligned}[]
		d^2\omega&=d(F\omega-(-1)^k\omega F)\\
		&=F\big( F\omega-(-1)^k\omega F \big)+(-1)^{k}\big( F\omega-(-1)^k\omega F \big)\\
		&=\omega-(-1)^kF\omega F+(-1)^k F\omega F-\omega\\
		&=0
	\end{aligned}
\end{equation}
where we used the fact that $F^2=\mtu$.

The pair $(\Omega^*,d)$ is a graded differential algebra. We have to find a graded closed trace $\tr_s\colon \Omega^n\to \eC$. Let $T$ be an operator on $\hH$ such that $FT+TF\in\oL^1(\hH)$. We begin to define
\[ 
  \tr'(T)=\frac{ 1 }{2}\tr\big( F(FT+TF) \big).
\]
When $T\in\oL^1$, it makes sense to distribute the $F$ in the trace, in such a way that we obtain $\tr'(T)=\tr(T)$. In this case, we have
  $\tr'(T)=\frac{ 1 }{2}\tr\big( F(FT+TF) \big)=\frac{ 1 }{2}\tr(T+FTF)=\tr(T)$.
Now de define the trace $\tr_s\colon \Omega^n\to \eC$,
\[ 
  \tr_s\omega=
\begin{cases}
\tr'(\omega)&\text{if $n$ is odd},\\
\tr'(\gamma\omega)&\text{if $n$ is even}.
\end{cases}
\]
It makes sense because when $\omega\in\Omega^n$, it fulfills $F\omega+\omega F=d\omega\in\Omega^{n+1}\subset\oL^{(n+1)/(n+1)}=\oL^1$, so that we can use the usual trace. By the way, remark that the trace $\tr_s$ reads $\frac{ 1 }{2}\tr(Fd\omega)$ when $n$ is odd and $\frac{ 1 }{2}\tr(F\gamma d\omega)$ when $n$ is even.

\begin{proposition}
	The triple $(\Omega,d,\tr_s)$ is a $n$-dimensional cycle over  $\cA$ (see definition \ref{DefCycleCoh}).
\end{proposition}

\begin{proof}
Most of the work is already done; it just remains to prove that $\tr_s$ is a graded closed trace. First, we know that $d^2=0$, so the fact that $\tr_s$ only depends to $d\omega$ gives $\tr_s(d\omega)=0$. The form is thus closed.

Now we have to prove that it is a graded trace. If $\omega\in\Omega^k$ and $\omega'\in\Omega^{k'}$ with $k+k'=n$ (let us assume $n$ odd), we have
\[ 
  \tr_s(\omega\omega')=\frac{ 1 }{2}\tr\big( F(d\omega)\omega'+(-1)^{k}F\omega d\omega' \big)=\frac{ 1 }{2}\tr\big( (-1)^{k+1}(d\omega)F\omega'+(-1)^k(F\omega)d\omega' \big),
\]
but the usual trace has the property that, when $T_j\in\oL^j$ with $\frac{1}{ p_1 }+\frac{1}{ p_2 }=1$, $\tr(T_1T_2)=\tr(T_2T_1)$. So we have $\tr(F\omega d\omega')=\tr(d\omega'F\omega)$ because $F\omega\in\oL^{(n+1)/  k }$ and $d\omega'\in\Omega^{k'+1}\subset\oL^{(n+1)/(k'+1)}$. Thus we have
\[ 
  \tr_s(\omega\omega')=\frac{ 1 }{2}\big( (-1)^{k+1}d\omega F\omega'+(-1)^kd\omega' F\omega \big).
\]
This expression is symmetric or anti-symmetric with repsect to the inversion $\omega\leftrightarrow\omega'$ following that $(-1)^{kk'}$ equals $1$ or $-1$. We conclude that (at least when $n$ is odd) t
\[ 
  \tr_s(\omega\omega')=(-1)^{kk'}\tr_s(\omega'\omega).
\]
\end{proof}

The \defe{character}{character!of a cycle} is the cyclic cocycle
\begin{equation}
\tau_n(a^0,\cdots,a^{n})=
\begin{cases}
\tr'\big( a^0[F,a^1]\cdots[F,a^{n}] \big)&\text{if $n$ is odd},\\
\tr'\big( \gamma a^0[F,a^1]\cdots[F,a^{n}] \big)&\text{if $n$ is even}.
\end{cases}
\end{equation}

%+++++++++++++++++++++++++++++++++++++++++++++++++++++++++++++++++++++++++++++++++++++++++++++++++++++++++++++++++++++++++++
\section{Hochschild cohomology}
%+++++++++++++++++++++++++++++++++++++++++++++++++++++++++++++++++++++++++++++++++++++++++++++++++++++++++++++++++++++++++++

\begin{proposition}
Let $(\hH,F)$ be a Fredholm module $(n+1)$-summable over $\cA$. We suppose that this module has the same parity as $n$. Then the characters $\tau_{n+2q}$ satisfy
\begin{equation}
\tau_{m+2}=-\frac{ 2 }{ m+2 }S\tau_m\in HC^{m+2}(\cA)
\end{equation}
when $m=n+2q$, with $q\geq 0$.
\end{proposition}

\begin{proof}
In order to see that $\tau\in C^n_{\lambda}(\cA)$, we have to check that the equality $\tau_n(a^1,\cdots,a^{n},a^0)=(-1)^n\tau_n(a^0,\cdots,a^{n})$ holds for all $a^i\in\cA$. We have
\[ 
\tau_n(a^0,\cdots,a^{n})=\tr'(a^0da^1\cdots da^{n})=\tr'\Big( \big( d(a^0a^1)-da^0a^1 \big)da^2\cdots da^{n} \Big)
\]
in which the first term vanishes because $d^2=0$. We go on commuting $da^0$ and we finally get $(-1)^n\tau_n(a^1,\cdots,a^n,a^0)$.

\begin{probleme}
	This proof is not finished.
\end{probleme}

\end{proof}
%+++++++++++++++++++++++++++++++++++++++++++++++++++++++++++++++++++++++++++++++++++++++++++++++++++++++++++++++++++++++++++
\section{Fredholm module and conformal structure}
%+++++++++++++++++++++++++++++++++++++++++++++++++++++++++++++++++++++++++++++++++++++++++++++++++++++++++++++++++++++++++++

Let $V$ be a compact, oriented even dimensional manifold endowed with a \defe{conformal structure}{conformal structure}. That is an equivalence class of metrics where $g\sim h$ if and only if there exists a positive smooth function $\lambda$ such that $g=\lambda^2 h$.

Let $\hH_0=L^2\big( V,\Wedge^n_{\eC}(T^*V) \big)$ with the product \eqref{EqProdWedgeHOfge}. This becomes a $ C^{\infty}(V)$-module when we define
\begin{equation}
	(f\omega)(p)=f(p)\omega(p)
\end{equation}
for every $\omega\in\hH_0$, $f\in C^{\infty}(V)$ and $p\in V$. We can extend the graduation \eqref{EqGradWedge} to the Hilbert space $\hH_0$ by
\begin{equation}
	(\gamma\omega)(p)=\gamma\big( \omega(p) \big).
\end{equation}


Consider $\hH_0=L^2\big( V,\Wedge^n_{\eC}(T^*V) \big)$ be the space of square integrable sections of the bundle $\Wedge^n_{\eC}(T^*V)$ for the product

We can consider the complex space $\wedge^n_{\eC}E$ and the operator $\gamma\colon \Wedge^nE\to \Wedge^nE$,
\begin{equation}
	\gamma=(-1)^{n(n-1)/2}i^n *.
\end{equation}
This operator squares to $\mtu$, so that it creates a $\eZ/2$-graduation of $\Wedge^n_{\eC}E$.

We have 
\begin{equation}
	\frac{ 1+\gamma }{2}d(*\alpha^{n+1})=\frac{ 1 }{2}\Big( d(*\alpha^{n+1})+(-1)^s\delta\alpha^{n+1} \Big)
\end{equation}
where $s$ is a sign. We used the fact that $\delta=-*d*$. Thus the elements of $\hH_0$ of the form $\frac{ 1+\gamma }{2}d\alpha$ are orthogonal to the harmonic forms.


%+++++++++++++++++++++++++++++++++++++++++++++++++++++++++++++++++++++++++++++++++++++++++++++++++++++++++++++++++++++++++++
\section{Fredholm modules and $K$-cycles}
%+++++++++++++++++++++++++++++++++++++++++++++++++++++++++++++++++++++++++++++++++++++++++++++++++++++++++++++++++++++++++++

\begin{definition}
	A \defe{K-cycle}{K-cycle} $(\hH,D)$ over an involutive algebra $(\cA,*)$ is 
	\begin{enumerate}
		\item a $*$-representation of $\cA$ on $\hH$,
		\item a selfadjoint non bounded operator $D$ with compact resolvent and such that $[D,a]$ is bounded for each $a\in\cA$.
	\end{enumerate}
\end{definition}

\begin{remark}
	The condition ``compact resolvent'' means that the operators $(D-\lambda\mtu)^{-1}$ are compact for every $\lambda$ in $\eC\setminus\Spec(D)$ (lemma \ref{LemResLcmpResLLcmp}). In particular, the kernel of $D$ is finite dimensional (corollary \ref{CorRezcomkerfin}).
\end{remark}

From a K-cycle on $\cA$ we canonically build a Fredholm module $(\hH',F)$, the Fredholm module \defe{associated}{Fredholm!module!associated with a K-cycle} with the K-cycle $(\hH,D)$, in the following way.
\begin{enumerate}
	\item
		$\hH'=\hH\oplus\ker(D)=\ker(D)^{\perp}\oplus\ker(D)\oplus\ker(D)$;
	\item
		$a(\xi,\eta)=(a\xi,0)$ for every $\xi\in\hH$ and $\eta\in\ker(D)$;
	\item
		$F=\Sign(D)\oplus F_1$.
\end{enumerate}
The definition of $F$ deserves some comments. First, $\Sign(D)$ is the sign of $D$, that is the partial isometry in the polar decomposition $D=V| D |$ of $D$. That acts on $\ker(D)^{\perp}$. The operator $F_1$ is the operator which exchanges the two copies of $\ker(D)$. More explicitly, if $\xi\in\hH$ and $\eta\in\ker(D)$,
\begin{equation}
	F(\xi,\eta)=F(\xi_1+\xi_0,\eta)=(V\xi_1+\eta,\xi_0)
\end{equation}
where $\xi=\xi_1+\xi_0$ is the decomposition of $\xi$ with respect to $\hH=\ker(D)\oplus\ker(D)^{\perp}$. In order to prove that $F^2=\mtu$, we have to show that $V^2=\mtu$ and that $V\xi_1\in\ker(D)^{\perp}$.

Since the part $\Sign(D)$ only acts on $\ker(D)^{\perp}$, we can see this operator as in equation \eqref{EqPolarSSKerSign}, that is the sign of the operator $D$ restricted to the space $\ker(D)^{\perp}$. This is an operator on $\ker(D)^{\perp}$, so that $V\xi_1\in\ker(D)^{\perp}$ and $V^2=\id_{\ker(D)^{\perp}}$.

%+++++++++++++++++++++++++++++++++++++++++++++++++++++++++++++++++++++++++++++++++++++++++++++++++++++++++++++++++++++++++++
\section{Spectral triple}
%+++++++++++++++++++++++++++++++++++++++++++++++++++++++++++++++++++++++++++++++++++++++++++++++++++++++++++++++++++++++++++

\subsection{General spectral triple}
%-----------------------------------
A \defe{spectral triple}{spectral!triple} is a triple $(\cA,\hH,D)$ where
\begin{itemize}
\item $\hH$ is a Hilbert space,
\item $\cA$ is an involutive algebra of bounded operators on $\hH$ ,
\item $D$ is a self-adjoint ($D=D^*$) operator on $\hH$ such that
\begin{enumerate}
\item the resolvent $(D-\lambda)^{-1}$, $\lambda\notin\eR$ is a compact operator on $\hH$,
\item\label{item_DaDcirii} $[D,a]:=D\circ a-a\circ D$ is a bounded operator for all $a$.
\end{enumerate}
\end{itemize}
In general condition \ref{item_DaDcirii} can only be imposed on a dense subalgebra of $\cA$. 

\begin{definition}	\label{DefDimSpec}
The \defe{dimension spectrum}{dimension!of a spectral triple!spectrum} of the spectral triple $(\cA,\hH,D)$ is the set $\Pi$ of complex numbers $z$ such that $\real(z)\geq 0$ and $z$ is a singularity of the analytic function $\zeta_b(z)$ for $b\in\mB$ with positive real part. Here $\mB$ is the operator algebra generated by $\delta^k(a)$ and $\delta^k[D,a]$ with $a\in \cA$ and $\delta T=[| D |,T]$. When $b\in\mB$, the function $\zeta$ is given by
\[ 
  \zeta_b(z)=\tr(b| D |^{-z})
\]
which is well defined when $\real(z)>m$ where $m$ is the crude dimension of the triple.
\end{definition}

We say that the dimension of the triple is \defe{simple}{simple!dimension of a spectral triple}\index{dimension!of a spectral triple!simple} id the poles of the functions $\zeta_b$ are at most simple.

We say that the triple is \defe{even}{even spectral triple} if there exists an operator $\Gamma$ on $\hH$ such that
\begin{enumerate}
\item $\Gamma=\Gamma^*$,
\item $\Gamma^2=1$,
\item $[\Gamma,D]=0$ and $[\Gamma,a]=0$ for all $a\in\cA$.
\end{enumerate}
If the triple is not even, it is \defe{odd}{odd spectral triple}. The triple $(\cA,\hH,D)$ is of \defe{dimension}{dimension!of a spectral triple} $n>0$ if $| D |^{-1}$ is an infinitesimal of order $1/n$, in other words, if $| D |-n$ is infinitesimal of order $1$. A $n$-dimensional spectral triple is sometimes said to be $n$-summable. A \defe{real structure}{real!structure!on a spectral triple} on the spectral triple $(\cA,\hH,D)$ is an antilinear isometry $J\colon \hH\to \hH$ such that
\begin{align*}
J^2&=\epsilon(n)\mtu	&[a,b^0]&=0\\
JD&=\epsilon'(n)DJ	&\big[ [D,a],b^0 \big]&=0\\
J\Gamma&=(i)^n\Gamma J
\end{align*}
where $b^0=Jb^*J^*$ and $\Gamma$ is the $\eZ_2$ graduation if the triple is even; if the triple is odd, then the corresponding condition is removed. The functions $\epsilon$ and $\epsilon'$ are periodic with period $8$ and
\[ 
\begin{split}
\epsilon(n)	&=(1,1,-1,-1,-1,-1,1,1)\\
\epsilon'(n)	&=(1,-1,1,1,1,-1,1,1).
\end{split}  
\]

Notice that as direct consequence of the properties, we also have $\big[ [D,b^0] \big]=0$. When we consider a real spectral triple, we can endow $\hH$ with a bimodule structure over $\cA$ by
\[ 
  a\xi b=\pi(a)J\pi(b^*)J^*\xi.
\]
The left module structure is the usual one while the right is well defined because $J^*J=\cun$, so that
\begin{align*}
\xi(an)=Jb^*a^*J^*\xi=Jb^*J^*Ja^*J^*\xi=(\xi a)b.
\end{align*}

Two spectral triples $(\cA_i,\hH_i,\pi_i,D_i)$ are \defe{equivalent}{equivalence!of spectral triple} when there exists an unitary operator $U\colon \hH_1\to \hH_2$ such that $U\pi_1(a)U^*=\pi_2(a)$ for every $a\in\cA$ and $UD_1U^*=D_2$. If the triple is even or real, we ask moreover $U\Gamma_1U^*=\Gamma_2$ and $UJ+1U^*=J_2$.

\subsection{Commutative real triple}
%------------------------------------

When $\cA$ is commutative, the right action of $a$ is equivalent to the left action of $Ja^*J^*$ in the sense that
\[ 
  \xi(ab)=(Ja^*J^*)(Jb^*J^*)\xi
\]

\subsection{Analysis on a spectral triple}
%-----------------------------------------

The following is a direct computation.
\begin{lemma}
The operator $[D,\cdot\,]$ is a derivation of $\cA$.
\end{lemma}

\begin{lemma}
\begin{equation}
  [D,a]^*=-[D,a]
\end{equation}

\end{lemma}
\begin{proof}
It is nothing else than the fact that $D=D^*$:
\[ 
  [D,a]^*=(D\circ a)^*-(a\circ D)^*=a^*D-Da^*=[a,D]=-[D,a].
\]

\end{proof}

From definition of the spectral triple, the operator $(D-z\mtu)$ exists for all non real $z$, so the spectrum (definition \ref{def:spectre}) of $D$ is real:
\[ 
  \sigma(D)\subset \eR.
\]

\begin{probleme}
	The following statements need more theory about operators with compact resolvent.
\end{probleme}

The spectrum of $D$ is discrete and the elements $\{ \lambda_n \}$ are eigenvalues of finite multiplicity. Moreover characteristic values $\mu_n\big( (D-1)^{-1} \big)\to 0$ when $n\to\infty$ and so $| \lambda_n |=\mu_n(| D |)\to0$.

When $[D,a]$ is bounded we say that $a\in\cA$ is \defe{Lipschitz}{Lipschitz}. Let $\delta$ be the derivation on $\opB(\hH)$ defined (on a dense subspace) by
\[ 
  \delta(T)=[| D |,T].
\]
This generates a one parameter group of automorphism of $\opB(\hH)$ defined by
\begin{equation}
\alpha_s(T)= e^{is| D |}T e^{-is| D |}.
\end{equation}
We say that $a\in\cA$ is \defe{smooth}{smooth!in spectral triple} and we write $a\in C^{\infty}$ if the map 
\[ 
  s\to\alpha_s(a)
\]
is smooth. The element $a$ is of class $C^k$ when $s\to\alpha_s(a)$ is $C^k$.

\begin{proposition}
An element $a\in\cA$ is smooth if and only if $a\in\cap_{n\in\eN}\dom(\delta^n)$.
\end{proposition}

\begin{proof}
If $a$ is smooth, the existence of the derivative of $s\to  e^{is| D |}a e^{-is| D |}$ makes that $a\in\dom\delta$ because the derivative of this map is precisely $\delta$. A few computation shows that the second derivative of this map is $\delta^2$.

If, on the other hand, $a\in\cap_{n\in\eN}\dom(\delta^n)$, we have existence of all the derivatives of $s\to\alpha_s(a)$ and continuity is given by derivability
\begin{probleme}
	Is that justification correct ?
\end{probleme}

\end{proof}

\subsection{Spectral triple over a manifold}
%-------------------------------------------

Let $(M,g)$ be a Riemannian spin manifold of dimension $n$. The \defe{canonical triple}{canonic!spectral triple} on $M$ is 
\begin{enumerate}
\item $\cA= C^{\infty}(M)$,
\item $\hH=L^2(M,S)$, the bundle of square integrable spinors on $M$.
\item $D$ is the Dirac operator associated with the Levi-Civita connection of $g$.
\end{enumerate}
The rank of the spinor bundle is $2^{[n/2]}$ and the scalar product on $\hH$ is given by
\[ 
  (\psi,\phi)=\int \overline{ \psi(x) }\phi(x)\,d\mu(g)
\]
where the bar denotes the complex conjugation, $d\mu(g)$ is the measure associated with $g$ and $\overline{ \phi(x) }\phi(x)$ is the usual product in $\eC^{2[n/2]}$. The space $ C^{\infty}(M)$ is an algebra of operators on $\hH$ by simple multiplication:
\begin{equation}
(f\psi)(x):=f(x)p\psi(x)
\end{equation}
for all $f\in C^{\infty}(M)=\cA$ and $\psi\in\hH$.

Most of the interest in spectral triples over manifold comes from the following theorem.

\begin{theorem}
Let $(\cA,\hH,D)$ be the canonical triple on a manifold $M$. Then
\begin{enumerate}
\item $M$ is the structure space of the algebra $\overline{ \cA }=C(M)$, the norm closure of $\cA= C^{\infty}(M)$,
\item the geodesic distance between $p$, $q\in M$ is given by
\begin{equation} \label{eq_defdtriple}
  d(p,g)=\sup\{ | f(p)-f(q) |\tq f\in\cA\text{ and }\| [D,f] \|\leq 1 \},
\end{equation}
\item the Riemannian measure on $M$ is given by
\begin{equation}
\int_M f=c(n)\tr_{\omega}(f| D |^{-n})
\end{equation}
where $c(n)=2^{n-[n/2]-1}\pi^{n/2}n\Gamma(\frac{ n }{2})$.
\end{enumerate}

\end{theorem}

See \cite{Landi} for more complete proof and reference for even more complete proof.

\begin{proof}
The algebra $\overline{ \cA }$ is an unital commutative $C^*$-algebra. So Gelfand theorem \ref{thoGelfand} says that $\overline{ \cA }\simeq C\big( \Delta(\overline{ \cA }) \big)$, but by definition $\overline{\cA}=C(M)$, hence $M=\Delta(\overline{ \cA })$. This proves the first point.

For the second point, we use the form \eqref{eq_Dirac_deux} of Dirac operator, so
 \begin{equation}
\begin{split}
[D,f]\psi(x)&=D(f\psi)(x)-f(x)D\psi(x)\\
	&=\gamma^{\mu}(x)\big( (\partial_{\mu}f)\psi+f\partial_{\mu}\psi+f\omega_{\mu}^S\psi \big)\\
	&\quad -f(x)\gamma^{\mu}(x)\big( \partial_{\mu}\psi+\omega_{\mu}^S\psi \big)\\
	&=(\gamma^{\mu}\partial_{\mu}f)\psi.
\end{split}
\end{equation}
Hence, as multiplicative operator on $\hH$, we have $[D,f]=\gamma^{\mu}\partial_{\mu}f=\gamma(df)$ for all $f\in\cA$. The norm of this operator is
\[ 
  \| [D,f] \|=\sup | (\gamma^{\mu}\partial_{\mu}f)(\gamma^{\nu}\partial_{\nu}f)^* |^{\frac{ 1 }{2}}.
\]
One can prove (cf \cite{Landi} for a reference) that this is the Lipschitz norm of $f$:
\[ 
  \| f \|_{Lip}=\sup_{x\neq y}\frac{ | f(x)-f(y) | }{ d_{\gamma}(x,y) }
\]
where $d_{\gamma}$ is the usual geodesic distance that we want to prove to be equals to $d$. The condition $\| [D,f]\leq 1 \|$ in the definition  \eqref{eq_defdtriple} retrains us to only looks at $f$ such that 
\[ 
  \sup_{x\neq y}\frac{ | f(x)-f(y) | }{ d_{\gamma}(x,y) }\leq 1.
\]
If we fix $x=p$ and $y=q$, this condition becomes $| f(p)-f(q) |\leq d_{\gamma}(p,q)$; hence $d(p,q)\leq d_{\gamma}(p,q)$. 

We have to work out the inverse inequality. For, we fix a point $q$ and consider $f_{\gamma,q}=d_{\gamma}(x,q)$. This function fulfils $\| [D,f_{\gamma,q}] \| \leq 1$ and using this function as lower bound for the supremum which defines $d(p,q_0)$, we find
\[ 
  d(p,q)\geq | f_{\gamma,q}(p)-f_{\gamma,q}(q) |=d_{\gamma}(p,q)
\]
because $f_{\gamma,q}=d_{\gamma}(p,q)$ and $f_{\gamma,q}(q)=0$.

\end{proof}

\begin{probleme}
	The use of Gelfand's theorem at the beginning of the proof requires $M$ to be compact ?
\end{probleme}

\subsection{Distance over general triple}
%----------------------------------------

We will now show that formula \eqref{eq_defdtriple} generalises to a distance formula between states, see definition \ref{DefStateUnital}. The \defe{distance}{distance on spectral triple} is the distance on $\etS(\overline{ \cA })$ given by
\begin{equation}
d(\omega,\chi)=\sup_{a\in\cA}\{ | \omega(a)-\chi(a) |\tq \| [D,a] \|\leq 1 \}.
\end{equation}

When $(\cA,\hH,D)$ is a triple of dimension $n$, we define the \defe{integral}{integral!on a spectral triple} of $a\in \cA$ by
\begin{equation}
\int a:=\frac{1}{ V }\tr_{\omega}(a| D |^{-1})
\end{equation}
where $V$ is a constant defined by $\mu_j\leq V j^{-1}$ when $j\to\infty$. Here, $(\mu_j)$ is the sequence of characteristic values of $| D |^{-1}$. Why does $| D |^{-n}$ appears ? The operator $a$ is just bounded on $\hH$, hence the trace $\tr_{\omega}a$ makes no sense. The multiplication by $| D |^{-n}$ gives rise to an infinitesimal of order $1$ and Dixmier trace makes sense. On the other hand, the integral is normalised in the following sense:
\[ 
  \int\mtu=\frac{1}{ V }\tr_{\omega}| D |^{-n}
		=\frac{1}{ V }\lim_{N\to\infty}\sum_{j=1}^{N-1}\mu_j\big( | D |^{-n} \big)
		\leq\lim_{N\to\infty}\sum_{j=1}^{N-1}\frac{1}{ j }
		=1
\]
because $\frac{1}{ V }\mu_j(| D |^{-n})\leq j^{-1}$.

\begin{probleme}
	This only proves that $\int\mtu\leq1$, but the second line is probably
\[ 
  \frac{1}{ V }\lim_{\omega}\sum_{j=1}^{N-1}\mu_j(| D |^{-n}),
\]
and we should define $\lim_{\omega}$ in such a way that
\[ 
  \frac{1}{ V }\lim_{N\to\infty}\sum_{j=1}^{N-1}Vj^{-1}.
\]
After that, we still have to define $\lim_{\omega(V)}$ and prove that $\int a$ does not depend on its choice, on the choice of $V$ and of $\lim_{\omega(V)}$.
\end{probleme}

\subsection{Real triple}
%-----------------------

Let $(\cA,\hH,D)$ be a spectral triple of dimension $n$. A \defe{real structure}{real!structure!on a spectral triple} is an anti-linear isometry $J\colon \hH\to \hH$ such that
\begin{enumerate}
\item $J^2=\epsilon(n)\id$,
\item $JD=\epsilon'(n)DJ$,
\item $G\Gamma=i^n\Gamma J$ if $n$ is even with the grading $\Gamma$,
\item $[a,b^0]=0$,
\item $[ [D,a],b^0 ]=0$ where $b^0=Jb^*J^*$.
\end{enumerate}
The functions $\epsilon$ and $\epsilon'$ are defined modulo $8$: 
\[ 
\begin{split}
\epsilon(n)&=1,1,-1,-1,-1,-1,1,1\\
\epsilon'(n)&=1,-1,1,1,1,-1,1,1.
\end{split}  
\]

\subsection{Example: compact manifold}
%-------------------------------------

Let $M$ be a compact spin Riemannian manifold. We consider the triple $(\cA,\hH,D)$ where 
\begin{itemize}
\item $\hH$ is the Hilbert space of $L^2$ spinors on $M$,
\item $D$ is the Dirac operator on $\hH$,
\item $\cA$ is the abelian algebra of bounded measurable functions on $M$ with the multiplicative action on $\hH$.
\end{itemize}
Via the representation, space $C(M)$ of continuous functions on $M$ is seen as a subspace of the space of linear operators on $\hH$. Hence, up to a closure, we have $M=\Delta(\cA)$ and Gel'fand theorem \ref{thoGelfand} says that the compact topological space structure of $M$ is given by $\cA$. A point of $M$ is associated with an element of $\Delta(\cA)$, i.e. a homomorphism $\cA\to\eC$.

\begin{proposition}
Let $a\in\cA$. The operator $[D,a]$ is
\begin{enumerate}
\item defined on a dense subspace of $\hH$,
\item bounded if and only if $a$ is almost everywhere equals to a Lipschitz function.
\end{enumerate}

\end{proposition}
\begin{proof}
No proof.
\end{proof}

A function $f$ on the manifold $M$ is \defe{Lipschitz}{lipschitz!function} if for all $p$, $q\in M$,
\[ 
  | f(p)-f(q) |\leq C d(p,q)
\]
where $C$ is a constant and $d$ denotes the geodesic distance on $M$. Any Lipschitz function is continuous and the space of Lipschitz functions is dense in $C(M)$. Then $C(M)$ is the closure of $\cA$ in $\oL(\hH)$.
