% This is part of Outils mathématiques
% Copyright (c) 2012
%   Laurent Claessens
% See the file fdl-1.3.txt for copying conditions.

\begin{corrige}{OutilsMath-0150}

    Le volume considéré est le quart de paraboloïde posé sur la pointe. La base est le cercle de rayon \( \sqrt{2}\) situé à la hauteur \( 2\), et la pointe est sur l'origine. En coordonnées cylindriques nous avons \( \theta\colon 0\to \pi/2\) et \( z\colon 0\to 2\). Pour \( r\), il faut un peu réfléchir.
    
    L'équation \( x^2+y^2\leq z\) nous dit que \( r\leq \sqrt{z}\) et donc que
    \begin{equation}
        r\colon 0\to \sqrt{z}.
    \end{equation}
    La fonction à intégrer est \( x=r\cos(\theta)\), c'est à dire \( f(r,\theta,z)=r\cos(\theta)\). En n'oubliant pas le jacobien des coordonnées cylindriques (\( r\)), l'intégrale est
    \begin{equation}
        \int_0^{\pi/2}d\theta\int_0^2dz\int_0^{\sqrt{z}}r^2\cos(\theta)=\frac{ 8\sqrt{2} }{ 15 }.
    \end{equation}
    Notez que l'ordre d'intégration entre \( r\) et \( z\) est important.

\end{corrige}
