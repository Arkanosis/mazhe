% This is part of Exercices et corrigés de CdI-1
% Copyright (c) 2011
%   Laurent Claessens
% See the file fdl-1.3.txt for copying conditions.

\begin{corrige}{OutilsMath-0020}

	Par construction, la projection d'un point $P$ sur la droite de $X$ se fait le long de la droite perpendiculaire à $X$ qui passe par $P$. Le vecteur $Y-P$ sera donc perpendiculaire à $X$ et le produit scalaire sera nul. Faire un dessin sur lequel on visualise $P$, $X$ et $X-P$.

	Nous pouvons obtenir le résultat par calcul. Nous savons que
	\begin{equation}
		\| \pr_XY \|=\frac{ X\cdot Y }{ \| X \| }.
	\end{equation}
	Donc le vecteur $\pr_XY$ est le vecteur de norme $X\cdot Y/\| X \|$ parallèle à $X$. Par conséquent
	\begin{equation}
		\pr_XY=\frac{ X\cdot Y }{ \| X \| }\frac{ X }{ \| X \| }.
	\end{equation}
	Donc
	\begin{equation}
		\begin{aligned}[]
			(Y-\pr_XY)\cdot X=Y\cdot X-\frac{ X\cdot Y }{ \| X \|^2 }X\cdot X\\
			&=Y\cdot X-Y\cdot X\\
			&=0
		\end{aligned}
	\end{equation}

\end{corrige}
