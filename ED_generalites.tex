% This is part of Analyse Starter CTU
% Copyright (c) 2014
%   Laurent Claessens,Carlotta Donadello
% See the file fdl-1.3.txt for copying conditions.

%+++++++++++++++++++++++++++++++++++++++++++++++++++++++++++++++++++++++++++++++++++++++++++++++++++++++++++++++++++++++++++ 
\section{Équations différentielles du premier ordre}
%+++++++++++++++++++++++++++++++++++++++++++++++++++++++++++++++++++++++++++++++++++++++++++++++++++++++++++++++++++++++++++

\begin{definition}[Équation différentielle du premier ordre]
Une  \defe{équation différentielle du premier ordre}{équation différentielle!premier ordre} est une équation qui, sur un intervalle donné, \(I\), décrit la relation entre une variable réelle, notée \(x\) ou \(t\) dans \(I\), une fonction \(y \,:\,I\to\eR \), et la dérivée première de \(y\) qui on note \(y'\). 
\end{definition}
Souvent on écrit <<\(y'(x) = \text{une formule contenante }x \text{ et }y(x)\)>>, c'est \`a dire 
\begin{equation}\label{ed_generale}
  y'(x) = f(x,y(x)),\quad\text{pour }x\in I,
\end{equation}
où \(f\) est une fonction de deux variables réelles.  
\begin{remark}
  La théorie des fonctions de deux variables ne sera pas abordée dans ce cours, nous allons nous contenter de prendre \(f\) dans \eqref{ed_generale} comme une simple notation. 
\end{remark}
On peut presque toujours omettre d'écrire la dépendance de \(y\) en \(x\) et écrire simplement \eqref{ed_generale} sous la forme \(y' = f(x,y)\). 
\begin{definition}[Solution particulière d'une équation différentielle du premier ordre]
  Une \defe{solution particulière}{solution!particulière} de l'équation \eqref{ed_generale} sur l'intervalle \(I\) est une fonction \(z\,:\, I\to\eR\) telle que :
  \begin{enumerate}
  \item \(z\) est dérivable sur \(I\) ;
  \item \(z'(x) = f(x, z(x)), \) pour tout \(x\in I\). 
  \end{enumerate}
\end{definition}
\begin{definition}[Solution générale d'une équation différentielle du premier ordre]
  Résoudre une équation différentielle veut dire trouver l'ensemble qui contient toutes ses solutions particulières. Cet ensemble s'appelle \defe{solution générale}{solution!générale} de l'équation. 
\end{definition}
\begin{example}
    \begin{enumerate}
        \item
        Nous avons déjà rencontré des équations différentielles de la forme \(y'= f(x)\) dans le chapitre sur les intégrales \ref{chapint} (ici, \(f\) ne dépend que de \(x\)). Résoudre une telle équation revient à trouver l'ensemble des primitives de la fonction \(f\), qui est donc la solution générale de cette équation. Il y a donc une infinité de solutions particulières, déterminées par une constante additive. 

        Si \(f (x) = \sin(x)\) alors la solution générale sera \(\mathcal{Y} = \{-\cos(x) + C\, : \, C\in\eR\}\). 
        \item
        L'équation 
        \begin{equation}\label{equation_exponentielle}
          y'= y, \qquad x \in\eR,
        \end{equation}
        a peut-être été abordée dans votre cours de terminale lors de la définition de la fonction exponentielle. Sa solution générale est \(\mathcal{Y} = \{Ce^x\, : \, C\in\eR\}\). Ici aussi il y a une infinité de solutions particulières.  
    \end{enumerate}
\end{example}
\begin{remark}
  La solution générale d'une équation différentielle du premier ordre est une famille à un paramètre de fonctions.
\end{remark}
\begin{definition}[Équation differentielle du second ordre]
Une  \defe{équation différentielle du second ordre}{équation différentielle!second ordre} est une équation qui, sur un intervalle donne, \(I\), décrit la relation entre une variable réelle, notée \(x\) ou \(t\) dans \(I\), une fonction \(y \,:\,I\to\eR \), et les dérivées première et seconde de \(y\) qui on note \(y'\) et \(y''\) respectivement. 

On utilise la forme générale
\begin{equation}\label{ed_generale_second_ordre}
  y'' = f(x,y, y'),\quad\text{pour }x\in I.
\end{equation}
o\`u \(f\) est une fonction de trois variables réelles.
\end{definition}
On peut définir de manière analogue les équations différentielles d'ordre supérieur. Les définitions de solution particulière et de solution générale se généralisent aux équations différentielles d'ordre supérieur à un. 
\vspace{0.5cm}
\begin{definition}[Trajectoire]
  La trajectoire tracée par une solution particulière $y$ de l'équation \eqref{ed_generale} est le graphe de $y$ en tant que fonction de $x$.
\end{definition}
\begin{example}
  Nous allons regarder de plus près l'équation \eqref{equation_exponentielle}, $y'=y$, pour tout $x\in\eR$. Soient $y_1$ et $y_2$ deux solutions distinctes de cette équation. S'il existe un point $\bar x$ tel que $y_1(\bar x) = y_2 (\bar x)$ alors forcement $y_1(\bar x)/y_2 (\bar x)=1$. Or, la solution générale de l'équation est \(\mathcal{Y} = \{Ce^x\, : \, C\in\eR\}\), donc $y_i(x) = C_ie^x$, $i= 1,2$, o\`u les $C_i$ sont des constantes. Le rapport $y_1(\bar x)/y_2 (\bar x)$ vaut $C_1/C_2$ et par conséquent $C_1 = C_2$. Ce résultat contredit l'hypothèse que les deux solutions soient distinctes. On a donc montré que \emph{deux trajectoires distinctes de cette équations ne se croisent jamais}. 

\newcommand{\CaptionFigSBTooEasQsT}{Quelque trajectoires de l'équation \( y'=y\).}
\input{Fig_SBTooEasQsT.pstricks}

La figure \ref{LabelFigSBTooEasQsT} représente quelques trajectoires de l'équation. Si on les avait tracées toutes elles recouvriraient tout le plan $x$-$y$. Cela veut dire que \emph{par tout point $(x,y)$ passe une et une seule trajectoire de l'équation \eqref{equation_exponentielle}}. 

\end{example}
\begin{definition}[Condition initiale]
  Une \defe{condition initiale}{condition initiale} pour l'équation \eqref{ed_generale} sur l'intervalle \(I\) est un point \((\bar x, \bar y)\in I\times\eR\). 

On dit que la solution particulière \(z\) de \eqref{ed_generale} satisfait la condition initiale \((\bar x, \bar y)\in I\times\eR\) si \(z(\bar x) =\bar y\).
\end{definition}
\begin{definition}[Problème de Cauchy]
  L'association d'une équation différentielle et d'une condition initiale est appelée \defe{problème de Cauchy}{problème de Cauchy}
  \begin{equation}\label{plme_cauchy}
    \begin{cases}
      y'= f(x,y), \quad x\in I, \\
      y(\bar x) = \bar y.
    \end{cases}
  \end{equation}
\end{definition}
\begin{remark}
  Sous des conditions assez générales qui serons toujours vérifiées dans ce cours, tout problème de Cauchy admet une et une seule solution.  
\end{remark}
Pour passer de la solution générale d'une équation différentielle de premier ordre \`a une solution particulière il faut choisir une valeur du paramètre. Comme il y a un seul paramètre une seule condition (la trajectoire de la solution doit passer par un point fixe du plan) peut suffire. Pour une équation différentielle de second ordre comme \eqref{ed_generale_second_ordre}, nous aurons besoin de plus de conditions. Sans rentrer dans les détails, nous allons constater ce fait dans l'exemple suivant. 

\begin{example}
  La solution générale de l'équation 
    \begin{equation}\label{eq_expcompl}
      y'' = -y,
    \end{equation}
    est \(\mathcal{Y}= \{C_1\cos(x) + C_2\sin(x) \, :\, C_1,\, C_2\in\eR\}\). Remarquez que l'équation est du second ordre et que sa solution générale est une famille d'équations \`a deux paramètres réels. Ce sera toujours les cas pour les équations abordées dans la Section \ref{Secordredeux}. Pour déterminer une solution particulière de \eqref{eq_expcompl} il faut fixer les valeurs des deux paramètres et donc, en général, il sera nécessaire de donner deux conditions. 
\end{example}

    \begin{remark}
      Une condition comme \(y(0)=4\) nous dit que la constante $C_1 = 4$ mais elle ne nous permet pas de trouver $C_2$. Il y a donc une infinité de solutions de \eqref{eq_expcompl} qui satisfont \`a la condition \(y(0)=4\).
    \end{remark}

On peut fixer les deux condition de deux manières différentes.
\begin{enumerate}
\item{Problème  de Cauchy :} on fixe une terne de valeurs réels \(\bar x, \bar y, \bar y'\) et on cherche la solution telle que \(y(\bar x) = \bar y\), \(y'(\bar x) = \bar y'\).

  \begin{example}
    Les conditions \(y(0)=4\), \(y'(0)=15\) permettent de trouver la solution \(z(x) = 4\cos(x) + 15\sin(x)\).  
  \end{example}
\item{Problème aux bords :} on fixe deux point dans le plan $x$-$y$, \(A=(\bar x, \bar y\)) et \(B=(\tilde x, \tilde y)\), et on cherche la solution dont la trajectoire passe par $A$ et $B$, c'est \`a dire, on impose \(y(\bar x) = \bar y\), \(y(\tilde x) = \tilde y\). 

\begin{example}
    Les conditions \(y(0)=4\), \(y(\pi/2)=15\) permettent de trouver la solution \(z(x) = 4\cos(x) + 15\sin(x)\).  
  \end{example}
\end{enumerate}
