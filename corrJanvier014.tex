\begin{corrige}{Janvier014}


L'aire d'un carré de périmètre $a$ est $\frac{a^2}{16}$ ; l'aire d'un cercle de
circonférence $C$ est $\frac{C^2}{4\pi}$. On cherche à maximiser
\begin{equation*}
f(a,C) = \frac{a^2}{16} + \frac{C^2}{4\pi}
\end{equation*}
sous la contrainte $a + C = 10$.

On définit
\begin{equation*}
  g(a) = 16 \pi f(a,10-a) = \pi a^2 + 4(10-a)^2
\end{equation*}
qu'on cherche à maximiser. Dès lors on résout $g^\prime(a) = 0$ et on
trouve $a = \frac{40}{\pi+4}$, d'où $b = \frac{10\pi}{\pi+4}$. Il faut
donc couper à $a$ centimètres du bord (ou à $b$ centimètres).



\end{corrige}
% This is part of the Exercices et corrigés de mathématique générale.
% Copyright (C) 2009
%   Laurent Claessens
% See the file fdl-1.3.txt for copying conditions.
