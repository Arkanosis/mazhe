% This is part of the Exercices et corrigés de CdI-2.
% Copyright (C) 2008, 2009
%   Laurent Claessens
% See the file fdl-1.3.txt for copying conditions.


\begin{corrige}{_II-2-01}

\begin{enumerate}
\item 

Le système s'écrit sous la forme
\begin{equation}
	\begin{pmatrix}
	x'	\\ 
	y'	
\end{pmatrix}=
\begin{pmatrix}
	-1	&	-1	\\ 
	2	&	1	
\end{pmatrix}
\begin{pmatrix}
	x	\\ 
	y	
\end{pmatrix}.
\end{equation}
Les valeurs propres de la matrice sont données par l'annulation du déterminant
\begin{equation}
	\begin{vmatrix}
	-1-\lambda	&	-1	\\ 
	2	&	1-\lambda	
\end{vmatrix}=0.
\end{equation}
Les solutions sont $\lambda=\pm i$. Le vecteur propre pour la valeur propre $i$ est donné par
\begin{equation}
	v_1=\begin{pmatrix}
	1	\\ 
	-i-1	
\end{pmatrix}
\end{equation}
Le vecteur propre pour la valeur $-i$ est le complexe conjugué\footnote{Parce que si $Av=\lambda v$, alors $A^*v*=\lambda^*v^*$, tandis que $A^*=A$ et que les deux valeurs propres sont complexes conjuguées.} de $v_1$ :
\begin{equation}
	v_2=v_1^*=
\begin{pmatrix}
	1	\\ 
	i-1	
\end{pmatrix}.
\end{equation}
Les deux valeurs propres sont donc dans le cas \ref{ItemRapSystDc} de la recette de la page \pageref{ItemRapSystDc}. La solution du système s'écrit donc comme une combinaison linéaire
\begin{equation}
	\begin{pmatrix}
	x	\\ 
	y	
\end{pmatrix}=
C_1\begin{pmatrix}
	1	\\ 
	-i-1	
\end{pmatrix} e^{it}+
C_2\begin{pmatrix}
	1	\\ 
	i-1	
\end{pmatrix} e^{-it}
\end{equation}
où $C_1,C_2\in\eC$.

Cette solution est réelle si et seulement si $C_1=\bar C_2$. Si $C_1=a+bi$ et $C_2=a-bi$, alors les solutions prennent la forme
\begin{subequations}
\begin{numcases}{}
	x(t)=a\cos(t)+b\sin(t)\\
	y(t)=-(a+b)\cos(t)+(a-b)\sin(t).
\end{numcases}
\end{subequations}


\begin{alternative}
	Il y a moyen de résoudre cet exercice en calculant explicitement l'exponentielle de
\begin{equation}
	A=\begin{pmatrix}
	-1	&	-1	\\ 
	2	&	1	
\end{pmatrix}.
\end{equation}
D'abord, nous considérons $M_1$, la matrice qui agit comme l'identité sur le vecteur propre de $\lambda_1$ et qui donne zéro sur le vecteur propre de $\lambda_2$, et $M_2$, le contraire\footnote{Notez déjà que cela n'est possible que parce que nous avons une base de vecteurs propres.}. Dans ce cas, pour toute fonction analytique (c'est à dire pour toute fonction qui se définit par un développement en série de puissances), nous avons
\begin{equation}	\label{EqfAII201}
	f(A)=f(\lambda_1)M_1+f(\lambda_2)M_2.	
\end{equation}
En faisant $f(x)=x^0$, nous avons $f(A)=\mtu$ et $f(\lambda_1)=f(\lambda_2)=1$, donc
\begin{equation}		\label{EqMMII201a}
	\mtu=M_1+M_2.
\end{equation}
En prenant d'autre part $f(x)=x-i$, nous avons
\begin{equation}		\label{EqMMII201b}
	A-i\mtu=-2iM_2.
\end{equation}
En combinant \eqref{EqMMII201a} et \eqref{EqMMII201b}, nous trouvons
\begin{equation}
	\begin{aligned}[]
		M_1	&=\frac{ 1}{2}(\mtu-iA)\\
		M_2	&=\frac{ 1 }{2}(\mtu+iA).
	\end{aligned}
\end{equation}
Donc, en utilisant la formule \eqref{EqfAII201} avec $f=\exp$, nous avons
\begin{equation}		\label{EqExpMatrAII201}
	 e^{tA}= e^{it}\frac{ 1 }{2}(\mtu-i1)+ e^{-it}\frac{ 1 }{2}(\mtu-iA).
\end{equation}
Maintenant, la solution du système est donnée par
\begin{equation}
	\begin{pmatrix}
	x	\\ 
	y	
\end{pmatrix}=
\begin{pmatrix}
	\cos(t)-\sin(t)	&	-\sin(t)	\\ 
	2\sin(t)	&	\cos(t)+\sin(t)	
\end{pmatrix}\begin{pmatrix}
	x_0	\\ 
	y_0	
\end{pmatrix}.
\end{equation}
Tu sais comment vérifier que \eqref{EqExpMatrAII201} est bien l'exponentielle de $tA$ ? En utilisant la proposition 1 de la page II.40. Dérivons donc l'équation \eqref{EqExpMatrAII201} par rapport à $t$, et posons $t=0$.
\begin{equation}
	\frac{ d }{ dt }\big(  e^{it}\frac{ 1 }{2}(\mtu-iA)+ e^{-it}\frac{1}{ 2 }(\mtu+iA) \big)_{t=0}=\frac{ 1 }{2}(i\mtu+A-i\mtu+A)=A.
\end{equation}

\end{alternative}


\item
Dans ce système,
\begin{equation}
	A=\begin{pmatrix}
	-1	&	-1	\\ 
	1	&	-3	
\end{pmatrix},
\end{equation}
et le polynôme caractéristique est
\begin{equation}
	p(\lambda)=\lambda^2+4\lambda+4=0.
\end{equation}
Cela a une solution unique $\lambda=-2$ de multiplicité deux. Il n'y a hélas que un seul vecteur propre (à multiple près) :
\begin{equation}
	v=\begin{pmatrix}
	a	\\ 
	a	
\end{pmatrix}.
\end{equation}
Nous sommes donc dans le cas où il y a un seul vecteur pour une valeur propre multiple, c'est à dire le point \ref{ItemRapSystDe} de la page \pageref{ItemRapSystDe}. Nous cherchons donc des solutions sous la forme
\begin{equation}
	\begin{pmatrix}
	at+c	\\ 
	at+d	
\end{pmatrix} e^{-2t}
\end{equation}
où l'on a, conformément à la recette, prit le vecteur $\begin{pmatrix}
	a	\\ 
	a	
\end{pmatrix}$ comme coefficient de $ e^{\lambda t}$. Nous injections cette solution dans le système pour tenter de fixer $a,c$ et $d$. Après quelques calculs et après simplification par $ e^{-2t}$, nous tombons sur le système
\begin{subequations}
\begin{numcases}{}
a-c+d=0\\
a-c+d=0,
\end{numcases}
\end{subequations}
donc $a=c-d$. La solution du système est donc
\begin{equation}
	\begin{pmatrix}
	x	\\ 
	y	
\end{pmatrix}=\begin{pmatrix}
	(c-d)t+c	\\ 
	(c-d)t+d	
\end{pmatrix} e^{-2t}.
\end{equation}
Cette solution peut être rendue un peu plus jolie en changeant le nom des constantes :
\begin{equation}
	\begin{pmatrix}
	x	\\ 
	y	
\end{pmatrix}=
\begin{pmatrix}
	at+b	\\ 
	at+(b-a)	
\end{pmatrix} e^{-2t}.
\end{equation}
Cette solution est réelle lorsque $a,b\in\eR$.

\begin{alternative}
	Nous pouvons, ici aussi, résoudre le système en calculant l'exponentielle de la matrice. Le problème est cependant plus compliqué parce que nous avons un seul vecteur propre. Il n'est donc pas possible de définir les matrices $M_1$ et $M_2$ de l'équation \eqref{EqfAII201}.

	C'est pour parer à cette carence de vecteurs propres qu'on a inventé la \defe{forme de Jordan}{Jordan!forme}. Prenons une matrice $B$, et mettons la sous la forme $B=\lambda\mtu+N$, c'est à dire
\begin{equation}
	B=\begin{pmatrix}
 \lambda	&	1	&		&		\\ 
 	&	\lambda	&	1	&		\\ 
 	&		&	\lambda	&	1	\\ 
	&		&		&	\lambda	 
 \end{pmatrix},
\end{equation}
et 
\begin{equation}
	N=\begin{pmatrix}
 	&	1	&		&		\\ 
 	&		&	1	&		\\ 
 	&		&		&	1	\\ 
	&		&		&		 
 \end{pmatrix}.
\end{equation}
Il est vite vu qu'en calculant les puissances de $N$, la diagonale de $1$ se décale. Dans le cas de matrices $2\times 2$, nous avons $B=\begin{pmatrix}
	0	&	1	\\ 
	0	&	0	
\end{pmatrix}$, et $N^2=0$. Donc, $B^2=(\lambda\mtu+N)^2=\lambda^2\mtu+2\lambda N$. En calculant la puissance suivante de $B$, nous trouvons
\begin{equation}
	B^3=B^2(\lambda\mtu+N)=\lambda^2\mtu+3\lambda^2N,
\end{equation}
et aucun terme en $N^2$ pacque $N^2=0$. Par conséquent, lorsque $B$ est sous forme de Jordan,
\begin{equation}		\label{EqExpBJordanII201}
	f(B)=f(\lambda)\mtu+f'(\lambda)N.
\end{equation}

Dans le problème qui nous occupe, $A$ n'est pas sous forme de Jordan. Au lieu d'avoir \eqref{EqExpBJordanII201}, nous avons
\begin{equation}
	f(A)=f(\lambda)M_1+f'(\lambda)M_2
\end{equation}
où $M_1$ et $M_2$ sont des matrices à déterminer. Nous trouvons facilement que $M_1=\mtu$ et $M_2=A+2\mtu$.


\end{alternative}

\item
C'est un système $3\times 3$, avec pour matrice
\begin{equation}
	A=\begin{pmatrix}
  2	&	1	&	1\\ 
  1	&	2	&	1\\ 
 1	&	1	& 2	  
\end{pmatrix}.
\end{equation}
L'équation caractéristique est
\begin{equation}
	-\lambda^3+6\lambda^2-9\lambda+4=(\lambda-1)(-\lambda^2+5\lambda-4)=0.
\end{equation}
Les solutions sont $\lambda_1=4$ et $\lambda_2=1$. La seconde est de multiplicité deux. La valeur propre $4$ a pour vecteur propre $\begin{pmatrix}
	a	\\ 
	a	\\ 
	a	
\end{pmatrix}$, et la valeur double $1$ a deux vecteurs propres : $\begin{pmatrix}
	b	\\ 
	0	\\ 
	-b	
\end{pmatrix}$ et $\begin{pmatrix}
	0	\\ 
	c	\\ 
	-c	
\end{pmatrix}$. Nous sommes donc dans un cas où toutes les valeurs propres ont leur bon nombre de vecteurs propres, et la solution est simplement
\begin{equation}
	\begin{pmatrix}
	x	\\ 
	y	\\ 
	z	
\end{pmatrix}=\begin{pmatrix}
	a	\\ 
	a	\\ 
	a	
\end{pmatrix} e^{4t}
+\begin{pmatrix}
	b	\\ 
	0	\\ 
	b	
\end{pmatrix}e^t+\begin{pmatrix}
	0	\\ 
	c	\\ 
	-c	
\end{pmatrix}e^t.
\end{equation}

\begin{alternative}
	Il y a une autre façon de résoudre ce système. Remarquons en effet que $x'+y'+z'=4(x+y+z)$, donc
\begin{equation}
	x+y+z=K e^{4t}.
\end{equation}
Par conséquent, nous avons
\begin{equation}
	\begin{aligned}[]
		x'	&=2x+y+(-x-y+K e^{4t})\\
		y'	&=c+2y+(-x-y+K e^{4t}).
	\end{aligned}
\end{equation}
Nous en déduisons que
\begin{equation}
	\begin{aligned}[]
		x'	&=x+K e^{4t}\\
		y'	&=y+K e^{4t},
	\end{aligned}
\end{equation}
et donc
\begin{equation}
	\begin{aligned}[]
		x	&=Ce^t+\frac{ K }{ 3 } e^{4t}\\
		y	&=De^t+\frac{ K }{ 3 } e^{4t}.
	\end{aligned}
\end{equation}

\end{alternative}

\item
La matrice est
\begin{equation}
	A=\begin{pmatrix}
 0	&	0	&	1	&	0	\\ 
 2	&	0	&	0	&	1	\\ 
 -1	&	0	&	0	&	2	\\ 
0	&	0	&	0	&	0	 
 \end{pmatrix}.
\end{equation}
Vu qu'il y a une ligne de zéros, le déterminant de cette matrice est nul, et il y aura certainement $\lambda=0$ comme valeur propre. Le polynôme caractéristique est en effet $\lambda^4+\lambda^2$, et les valeurs propres sont $\lambda_1=i$, $\lambda_2=-i$ et $\lambda_3=\lambda_4=0$. Les vecteurs propres sont
\begin{equation}
	\begin{aligned}[]
		v_1&=\begin{pmatrix}
	1	\\ 
	-2i	\\ 
	i	\\ 
	0	
\end{pmatrix},&v_2&=\begin{pmatrix}
	1	\\ 
	2i	\\ 
	-i	\\ 
	0	
\end{pmatrix},&v_3&=\begin{pmatrix}
	0	\\ 
	a	\\ 
	0	\\ 
	0	
\end{pmatrix}.
	\end{aligned}
\end{equation}
Il n'y a que un seul vecteur propre pour la valeur propre zéro. Les deux valeurs propres \og comme il faut\fg{} fournissent les solutions
\begin{equation}
	\bar x=C_1\begin{pmatrix}
	1	\\ 
	-2i	\\ 
	i	\\ 
	0	
\end{pmatrix} e^{it}+
C_2\begin{pmatrix}
	1	\\ 
	2i	\\ 
	-i	\\ 
	0	
\end{pmatrix} e^{-it}.
\end{equation}
Les solutions correspondantes à la troisième valeur propre sont à chercher sous la forme $\bar x=(\bar at+\bar b) e^{i\lambda_3t}$ où $\bar a=\begin{pmatrix}
	0	\\ 
	a	\\ 
	0	\\ 
	0	
\end{pmatrix}$ et $\lambda_3=0$. Nous cherchons donc $\bar x$ sous la forme
\begin{equation}
	\bar x=\begin{pmatrix}
	0+b_1	\\ 
	a_2t+b_2	\\ 
	0+b_3	\\ 
	0+b_4	
\end{pmatrix}.
\end{equation}
Le système pour les constantes $a_2$ et $b_i$ est
\begin{subequations}
\begin{numcases}{}
0=b_3\\
a_2=2b_1+b_4\\
0=-b_1+2b_4\\
0=0.
\end{numcases}
\end{subequations}
Nous en déduisons donc la solution générale du système proposé :
\begin{equation}
	\bar x=C_1\begin{pmatrix}
	1	\\ 
	-2i	\\ 
	i	\\ 
	0	
\end{pmatrix} e^{it}+
C_2\begin{pmatrix}
	1	\\ 
	2i	\\ 
	-i	\\ 
	0	
\end{pmatrix} e^{-it}+
\begin{pmatrix}
	2a/5	\\ 
	at+b_2	\\ 
	0	\\ 
	a/5	
\end{pmatrix}
\end{equation}
avec $C_1$, $C_2$, $a$ et $b_2$ comme constantes.

\item
La matrice est
\begin{equation}
	A=\begin{pmatrix}
  1/2	&	-1/2	&	1/2\\ 
  0	&	1/2	&	-1/2\\ 
 1	&	0	& 0	  
\end{pmatrix},
\end{equation}
dont le polynôme caractéristique est $\det(A-\lambda\mtu)=-\lambda^3+\lambda^2+\frac{ \lambda }{ 4 }$. Les solutions sont $\lambda_1=0$, $\lambda_2=\frac{ 1+\sqrt{2} }{2}$ et $\lambda_3=\frac{ 1-\sqrt{2} }{2}$; les vecteur propres correspondants sont
\begin{equation}
	\begin{aligned}[]
		v_1&=\begin{pmatrix}
	0	\\ 
	1	\\ 
	1	
\end{pmatrix},&v_2&=\begin{pmatrix}
	-\frac{ \sqrt{2} }{ 2 }-1	\\ 
	1	\\ 
	-\sqrt{2}	
\end{pmatrix},&v_3&=\begin{pmatrix}
	\frac{ \sqrt{2} }{2}-1	\\ 
	1	\\ 
	\sqrt{2}	
\end{pmatrix}.
	\end{aligned}
\end{equation}
La solution du système s'écrit donc sous la forme
\begin{equation}
	\bar x=A
		\begin{pmatrix}
	0	\\ 
	1	\\ 
	1	
\end{pmatrix}+B\begin{pmatrix}
	-\frac{ \sqrt{2} }{ 2 }-1	\\ 
	1	\\ 
	-\sqrt{2}	
\end{pmatrix} e^{\lambda_2t}+C\begin{pmatrix}
	\frac{ \sqrt{2} }{2}-1	\\ 
	1	\\ 
	\sqrt{2}	
\end{pmatrix} e^{\lambda_3t}.
\end{equation}

\item
Pour un système non homogène, il y a la proposition 1 de la page II.35 qui dit qu'il faut trouver la solution générale du système homogène associé, et lui ajouter une solution particulière du système non homogène. Dans le cas présent, le système homogène a pour solutions
\begin{equation}
	\bar x_H=C_1\begin{pmatrix}
	1	\\ 
	-i	
\end{pmatrix} e^{(1+i)t}+C_2\begin{pmatrix}
	1	\\ 
	i	
\end{pmatrix} e^{1-i}t.
\end{equation}
Pour le système non homogène, nous essayons une solution de la forme
\begin{equation}
	\bar x_P=\begin{pmatrix}
	\alpha e^t+\beta e^{2t}	\\ 
	\gamma e^t+\delta e^{2t}	
\end{pmatrix},
\end{equation}
qu'on injecte dans l'équation
\begin{equation}
	\bar x'_P=\begin{pmatrix}
	1	&	-1	\\ 
	1	&	1	
\end{pmatrix}\bar x_P+\begin{pmatrix}
	e^t	\\ 
	 e^{2t}	
\end{pmatrix}.
\end{equation}
Après un certain nombre de calculs, nous trouvons
\begin{equation}
	\begin{aligned}[]
		\alpha&=0\\
		\beta&=-1/2\\
		\gamma&=1\\
		\delta&=1/2,
	\end{aligned}
\end{equation}
et donc la solution particulière est
\begin{equation}
	\bar x_P=\begin{pmatrix}
	-\frac{ 1 }{2} e^{2t}	\\ 
	e^t+\frac{ 1 }{2} e^{2t}.	
\end{pmatrix}.
\end{equation}

\end{enumerate}

\end{corrige}
