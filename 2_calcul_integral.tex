% This is part of Analyse Starter CTU
% Copyright (c) 2014
%   Laurent Claessens,Carlotta Donadello
% See the file fdl-1.3.txt for copying conditions.

%+++++++++++++++++++++++++++++++++++++++++++++++++++++++++++++++++++++++++++++++++++++++++++++++++++++++++++++++++++++++++++ 
\section{L'aire en dessous d'une courbe}
%+++++++++++++++++++++++++++++++++++++++++++++++++++++++++++++++++++++++++++++++++++++++++++++++++++++++++++++++++++++++++++

Soit $f$ une fonction à valeurs dans $\eR^+$.

Nous voudrions pouvoir calculer l'aire au-dessous du graphe de la fonction \( f\). Nous notons $S_f(x)$ l'aire là-dessous de la fonction $f$ entre l'abscisse $0$ et $x$, c'est à dire l'aire bleue de la figure \ref{LabelFigKKRooHseDzC}. 

\newcommand{\CaptionFigKKRooHseDzC}{L'aire en dessous d'une courbe. Le rectangle rouge d'aire $f(x)\Delta x$ approxime de combien la surface augmente lorsqu'on passe de $x$ à $x+\Delta x$.}
\input{Fig_KKRooHseDzC.pstricks}

Si la fonction $f$ est continue et que $\Delta x$ est assez petit, la fonction ne varie pas beaucoup entre $x$ et $x+\Delta x$. L'augmentation de surface entre $x$ et $x+\Delta x$ peut donc être approximée par le rectangle de surface $f(x)\Delta x$. Ce que nous avons donc, c'est que quand $\Delta x$ est très petit,
\begin{equation}
	S_f(x+\Delta x)-S_f(x)=f(x)\Delta x,
\end{equation}
ou encore
\begin{equation}
	f(x)=\frac{  S_f(x+\Delta x)-S_f(x)}{ \Delta x }.
\end{equation}
Nous formalisons la notion de «lorsque \( \Delta x\) est très petit» par une limite :
\begin{equation}
	f(x)=\lim_{\Delta x\to 0}\frac{  S_f(x+\Delta x)-S_f(x)}{ \Delta x }.
\end{equation}
Donc, la fonction $f$ est la dérivée de la fonction qui représente l'aire là-dessous de $f$. Calculer des surfaces revient donc au travail inverse de calculer des dérivées.

%+++++++++++++++++++++++++++++++++++++++++++++++++++++++++++++++++++++++++++++++++++++++++++++++++++++++++++++++++++++++++++ 
\section{Primitive et intégrale}
%+++++++++++++++++++++++++++++++++++++++++++++++++++++++++++++++++++++++++++++++++++++++++++++++++++++++++++++++++++++++++++

\begin{definition}[Primitive]
    Soit une fonction \( f\) définie sur un intervalle \( I\). Une \defe{primitive}{primitive} de \( f\) sur \( I\) est une fonction définie sur \( I\) dont la dérivée est \( f\).
\end{definition}

Les deux propriétés fondamentales sont les suivantes.
\begin{proposition}
    À propos de primitives.
    \begin{enumerate}
        \item
            Toute fonction continue sur un intervalle admet une primitive (sur cet intervalle).
        \item
            Deux primitives d'une même fonction sur le m\^eme intervalle ne diffèrent que d'une constante.
    \end{enumerate}
\end{proposition}

\begin{proof}
    La première partie est relativement compliquée et nous ne la traitons pas ici. En ce qui concerne la seconde partie, supposons que \( F_1\) et \( F_2\) soient deux primitives de la fonction \( f\) sur l'intervalle \( I\). Alors \( (F_1-F_2)'=f-f=0\). La fonction \( F_1-F_2\) est donc une fonction dont la dérivée est nulle; elle est donc constante. Nous avons donc \( F_1-F_2=c\) pour un certain \( c\in \eR\).
\end{proof}

Si \( f\) est une fonction définie sur un intervalle \( I\) et y admettant des primitives, nous notons
\begin{equation}
    \int f(x)dx
\end{equation}
l'ensemble des primitives de \( f\) sur \( I\) :
\begin{equation}
    \int f(x)dx=\left\{    F(x)+C\tq C\in \eR   \right\}
\end{equation}
où \( F\) est une quelconque primitive de \( f\).

\begin{example}
    Une primitive bien connue de \(  f\colon x\mapsto x^2 \) est la fonction \( F\colon x\to \frac{ x^3 }{ 3 }\). Nous écrivons donc
    \begin{equation}
        \int x^2dx=\frac{ x^3 }{ 3 }+C.
    \end{equation}
\end{example}

\begin{definition}[Intégrale]\label{defintegrale}
    Soit une fonction \( f\) continue sur un intervalle \( I\) et \( a\neq b\) deux nombres dans \( I\). Si \( F\) est une primitive de \( f\) sur \( I\) nous définissons l'\defe{intégrale}{intégrale} de \( a\) à \( b\) de la fonction \( f\) comme étant le nombre \( F(b)-F(a)\).
\end{definition}
En termes de notations, nous posons
\begin{equation}\label{Thfondcalc}
    \int_a^bf(t)dt=\Big[ F(t) \Big]_{t=a}^{t=b}=F(b)-F(a).
\end{equation}

\begin{remark}
  La valeur de l'intégrale ne dépend pas de la primitive qu'on choisi pour le calculer, car si $F_1$ et $F_2$ sont deux primitives de $f$ alors $F_1 = F_2 + C$ et $F_1(b)-F_1(a) = (F_2(b) + C)-(F_2(a)+C) = F_2(b)-F_2(a)$.
\end{remark}

\begin{remark}
  Si l'intervalle d'intégration est réduit à un seul point alors la valeur de l'intégrale est zéro, car $ \int_a^af(t)dt=F(a)-F(a) =0$.
\end{remark}

\begin{remark}
    Conformément à ce que nous montre la figure \ref{LabelFigKKRooHseDzC}, si une fonction continue est positive sur l'intervalle \( \mathopen[ a , b \mathclose]\), alors le nombre \( \int_a^bf(t)dt\) est l'aire de la portion de plan comprise entre les droites verticales \( x=a\), \( x=b\), la courbe représentant la fonction \( f\) et l'axe des abscisses.

    Si la fonction est négative : l'aire est comptée négativement.
\end{remark}

\begin{example} 
    Comme nous le voyons sur le dessin suivant,
    \begin{equation}
        \int_{-3\pi/2}^{3\pi/2}\sin(x)\,dx=0
    \end{equation}
    parce que les deux parties bleues s'annulent avec les deux parties rouges (qui sont comptées comme des aires négatives).
    \begin{center}
       \input{Fig_JSLooFJWXtB.pstricks}
    \end{center}
\end{example}

\begin{remark}
  Toute intégrale d'une fonction impaire sur un intervalle symétrique par rapport à l'origine est nulle. 
\end{remark}

\begin{proposition}[Intégrale et primitive] \label{PropZEJooEsnrgY}
    Soit \( f\) une fonction continue sur l'intervalle \( I\) et un élément \( a\in I\). Soit la fonction
    \begin{equation}
        \begin{aligned}
            F\colon I&\to \eR \\
            x&\mapsto \int_a^xf(t)dt. 
        \end{aligned}
    \end{equation}
    Alors \( F\) est l'unique primitive de \( f\) s'annulant en \( x=a\).
\end{proposition}


%+++++++++++++++++++++++++++++++++++++++++++++++++++++++++++++++++++++++++++++++++++++++++++++++++++++++++++++++++++++++++++ 
\section{Propriétés des intégrales}
%+++++++++++++++++++++++++++++++++++++++++++++++++++++++++++++++++++++++++++++++++++++++++++++++++++++++++++++++++++++++++++

\begin{proposition}[Relations de Chasles]
    Soit  \( f\) une fonction continue sur l'intervalle \( I\). Si \( a,b,c\in I\) nous avons
    \begin{equation}
        \int_a^cf(x)dx=\int_a^bf(x)dx+\int_b^cf(x)dx.
    \end{equation}
\end{proposition}
\index{relations!de Chasles}

Sur la figure \ref{LabelFigNWDooOObSHB}, la surface de \( a\) à \( c\) est évidemment égale à la somme des surfaces de \( a\) à \( b\) et de \( b\) à \( c\).
\newcommand{\CaptionFigNWDooOObSHB}{Illustration pour les relations de Chasles.}
\input{Fig_NWDooOObSHB.pstricks}

\begin{corollary}
  \begin{equation}
        \int_a^bf(x)dx=-\int_b^af(x)dx.
    \end{equation}
\end{corollary}

\begin{proposition}[Linéarité de l'intégrale]\label{lineariteintegrale}
    Si $f$ et $g$ sont deux fonction continues sur $I\subset\eR$, $a, \, b\in I$ et \( \lambda\in \eR\) nous avons
    \begin{equation}
        \int_a^b\big( f(x)+g(x) \big)dx=\int_a^bf(x)dx+\int_a^bg(x)dx,
    \end{equation}
    et
    \begin{equation}
        \int_a^b \lambda f(x)dx=\lambda\int_a^bf(x)dx.
    \end{equation}
\end{proposition}

\begin{proposition}[L'intégrale est monotone]   \label{PropCJIooHqECbq}
    Soient \( a,b\in I\) avec \( a<b\). Si \( f\geq g\) sur \( \mathopen[ a , b \mathclose]\) alors
    \begin{equation}
        \int_a^bf(x)dx\geq \int_a^bg(x)dx.
    \end{equation}
\end{proposition}

\begin{corollary}[Positivité] \label{PropHVWooBDRhCX}
    Si \( a<b\) et \( f\geq 0\) sur \( \mathopen[ a , b \mathclose]\) alors
    \begin{equation}
        \int_a^bf(x)dx\geq 0.
    \end{equation}
\end{corollary}

Ce résultat n'est qu'une application de la proposition \ref{PropCJIooHqECbq} car il consiste à prendre comme fonction $g$ la fonction nulle. 

%+++++++++++++++++++++++++++++++++++++++++++++++++++++++++++++++++++++++++++++++++++++++++++++++++++++++++++++++++++++++++++ 
\section{Techniques d'intégration}
%+++++++++++++++++++++++++++++++++++++++++++++++++++++++++++++++++++++++++++++++++++++++++++++++++++++++++++++++++++++++++++

Par la définition \ref{defintegrale} la calcul d'une intégrale consiste essentiellement à trouver une primitive de la fonction à intégrer.  Il est donc indispensable de bien connaître les dérivées des fonctions usuelles.

Voici un tableau des primitives à connaître.

\label{PageLCHooMbWjOj}
\begin{equation*}
    \begin{array}[]{|c||c|c|c|}
        \hline
        \text{Fonction}&\text{Primitive}&\text{Ensemble de définition}&\text{Remarques}\\
        f(x)&\int f(x)\, dx& \text{de } f&\\
        \hline\hline
        x^{\alpha}&\frac{ x^{\alpha+1} }{ \alpha+1 } + C& \text{dépend de $\alpha$\footnote{L'étude des différents cas est un exercice de révision très utile.}}&  \alpha\in \eR\setminus\{ -1 \}  \\
        \hline
        \frac{1}{ x }&\ln\big( | x | \big) + C&x\neq 0&\\
        \hline
        \frac{1}{ 1+x^2 }&\arctan(x) + C&\eR&\\
        \hline
        \frac{1}{ \sqrt{1-x^2} }&\arcsin(x) + C&\mathopen] -1 , 1 \mathclose[&\\
        \hline
        \frac{-1}{ \sqrt{1-x^2} }&\arccos(x) + C&\mathopen] -1 , 1 \mathclose[&\\
        \hline
        e^x&e^x + C&\eR&\\
        \hline
        \sin(x)&-\cos(x) + C&\eR&\\
        \hline
        \cos(x)&\sin(x) + C&\eR&\\
        \hline
    1+\tan^2(x)&\tan(x) + C&\text{in intervalle de la forme }\mathopen] -\frac{ \pi }{2} , \frac{ \pi }{2} \mathclose[+k\pi&\\
        \hline
    \end{array}
\end{equation*}



Notez que au signe près, les fonctions \( \arcsin \) et \( \arccos\) ont la même dérivée.

Si la fonction à intégrer est une combinaison linéaire de fonctions usuelles alors sa primitive peut \^etre calculée en utilisant la proposition \ref{lineariteintegrale}. Dans les sections suivantes on abordera deux autres cas où la fonction à intégrer peut s'écrire en termes de fonctions dont on connaît une primitive.

%--------------------------------------------------------------------------------------------------------------------------- 
\subsection{Intégration par parties}
%---------------------------------------------------------------------------------------------------------------------------

\begin{proposition}
    Si \( u\) et \( v\) sont deux fonctions dérivables de dérivées continues sur l'intervalle \( \mathopen[ a , b \mathclose]\) alors
    \begin{equation}
        \int_a^b u(x)v'(x)dx=\big[ u(x)v(x) \big]_a^b-\int_a^bu'(x)v(x)dx.
    \end{equation}
\end{proposition}

\begin{proof}
    Il s'agit d'utiliser à l'envers la formule de dérivation d'un produit :
    \begin{equation}
        uv'=(uv)'-u'v.
    \end{equation}
    Les fonctions à gauche et à droite étant égales, elles ont même intégrale sur \( \mathopen[ a , b \mathclose]\) et par linéarité, voir  proposition \ref{lineariteintegrale}, on a :
    \begin{equation}
        \int_a^b u(x)v'(x)dx=\int_a^b (uv)'(x)-\int_a^b u'(x)v(x)dx.
    \end{equation}
    La fonction \( uv\) est évidemment une primitive de \( (uv)'\), de telle sorte que l'on puisse un peu simplifier cette expression :
    \begin{equation}
        \int_a^b u(x)v'(x)dx= \Big[ u(x)v(x) \Big]_a^b -\int_a^b u'(x)v(x)dx,
    \end{equation}
    ce qu'il fallait démontrer.
\end{proof}

\begin{example} \label{ExWIEooVUgvSp}
    Un cas typique d'utilisation de l'intégrale par parties est le suivant. Soit à calculer
    \begin{equation}
       \int_0^{\pi}x\cos(x)dx.
    \end{equation}
    Nous devons écrire \( x\cos(x)\) comme un produit \( u(x)v'(x)\). Il y a (au moins) deux moyens de le faire :
    \begin{subequations}
        \begin{numcases}{}
            u=x\\
            v'=\cos(x).
        \end{numcases}
    \end{subequations}
    ou
    \begin{subequations}
        \begin{numcases}{}
            u=\cos(x)\\
            v'=x.
        \end{numcases}
    \end{subequations}
    Nous allons choisir le premier\footnote{Mais nous conseillons vivement au lecteur d'essayer le deuxième pour se rendre compte qu'il ne fonctionne pas.}. Nous avons donc
    \begin{equation}
        \begin{aligned}[]
            u&=x,&v'&=\cos(x)\\
            u'&=1&v&=\sin(x).
        \end{aligned}
    \end{equation}
    En utilisant la formule d'intégration par parties,
    \begin{equation}
        \int_0^{\pi}x\cos(x)dx=\Big[ x\sin(x) \Big]_0^{\pi}-\int_0^{\pi} 1\times \sin(x)dx=\pi\sin(\pi)-\Big[ -\cos(x) \Big]_0^{\pi}=-2.
    \end{equation}
\end{example}

Le plus souvent, pour alléger les notations, il est plus pratique d'utiliser l'intégration par parties pour déterminer une primitive. Nous utilisons pour cela la formule (sans doute plus simple à retenir)
\begin{equation}
    \int uv'=uv-\int u'v.
\end{equation}

\begin{example} \label{ExLTJooDZIYWP}
    Nous reprenons l'exemple \ref{ExWIEooVUgvSp} en déterminant cette fois une primitive de \( x\cos(x)\) :
    \begin{equation}\label{EqTQNooVTYkZX}
        \int x\cos(x)dx=x\sin(x)-\int \sin(x)dx=x\sin(x)+\cos(x) + C, \qquad C \in\eR.
    \end{equation}
    Nous retrouvons le résultat numérique de l'exemple précédent en ajoutant les extr\^emes d'intégration
    \begin{equation}
        \int_0^{\pi} x\cos(x)dx=\big[ x\sin(x)+\cos(x) \big]_0^{\pi}=-2.
    \end{equation}
\end{example}

\begin{remark}
    Lorsqu'on calcule des intégrales, il est bon de passer par la primitive (c'est à dire en suivant l'exemple \ref{ExLTJooDZIYWP} et non \ref{ExWIEooVUgvSp}) parce qu'il est alors facile de vérifier le résultat en calculant la dérivée de la primitive trouvée.
     
    Par exemple pour vérifier si \eqref{EqTQNooVTYkZX} est correct, il suffit de dériver \( x\sin(x)+\cos(x)\) :
    \begin{equation}
        \big( x\sin(x)+\cos(x) \big)'=\sin(x)+x\cos(x)-\sin(x)=x\cos(x).
    \end{equation}
    La fonction \( x\sin(x)+\cos(x)\) est donc bien une primitive de \( x\cos(x)\).
\end{remark}

Voici un exemple d'utilisation ingénieuse de l'intégration par parties.

\begin{example}\label{primln}
    Trouver la primitive de la fonction \( x\mapsto \ln(x)\). Pour calculer
    \begin{equation}
        \int\ln(x)dx
    \end{equation}
    nous écrivons \( \ln(x)=1\times \ln(x)\) et nous posons \( u'=1\) et \( v=\ln(x)\), c'est à dire
    \begin{equation}
        \begin{aligned}[]
            u'&=1&v=\ln(x)\\
            u&=x&v'=\frac{1}{ x }.
        \end{aligned}
    \end{equation}
    La formule d'intégration par parties donne donc 
    \begin{equation}
        \int \ln(x)=x\ln(x)-\int x\times \frac{1}{ x }=x\ln(x)-\int 1=x\ln(x)-x+C, \qquad C\in\eR.
    \end{equation}
    Il est facile de vérifier par un petit calcul que
    \begin{equation}
        \big( x\ln(x)-x \big)'=\ln(x).
    \end{equation}
\end{example}

%--------------------------------------------------------------------------------------------------------------------------- 
\subsection{Changement de variables -- pour trouver des primitives}
%---------------------------------------------------------------------------------------------------------------------------

De la même manière que l'utilisation «à l'envers» de la formule de dérivation du produit avait donné la méthode d'intégration par parties, nous allons voir que que l'utilisation «à l'envers» de la formule de dérivation d'une fonction composée donne lieu à la méthode d'intégration par changement de variables.
\begin{proposition}
    Soit \( I\) et \( J\) des intervalles de \( \eR\) et une fonction \( u\colon I\to J\) qui est dérivable de dérivée continue. Soit \( f\colon J\to \eR\) une fonction admettant une primitive \( F\). Alors la fonction
    \begin{equation}
        x\mapsto F\big( u(x) \big)
    \end{equation}
    est une primitive de
    \begin{equation}\label{changvar}
        f\big( u(x) \big)u'(x).
    \end{equation}
\end{proposition}

\begin{proof}
    Cela est une utilisation immédiate de la formule de dérivée des fonctions composées.
\end{proof}

\begin{example}
    Soit à calculer
    \begin{equation}
        \int x\sqrt{1-x^2}dx.
    \end{equation}
La fonction $g(x) = x\sqrt{1-x^2}$ est le produit de $x$ et de $\sqrt{1-x^2}$. On remarque que la dérivée de $1-x^2$ est $-2x$ : nous avons alors, à un facteur $-2$ près, une expression de la forme \eqref{changvar} où la racine carré joue le r\^ole de $f$, \( f(t)=\sqrt{t}\),   et $1-x^2$ le r\^ole de $u$.  Une primitive de la fonction \( f(t)=\sqrt{t}\) est $F(t) = 2t^{3/2}/3$. 
    
    Donc la fonction
      $  \frac{ 2u(x)^{3/2} }{ 3 }=\frac{ 2 }{ 3 }(1-x^2)^{3/2}$
    est primitive de
     $   -2x\sqrt{1-x^2} = -2 g(x)$.
    Autrement dit,
    \begin{equation}
        \int -2x\sqrt{1-x^2}\,dx=\frac{ 2 (1-x^2)^{3/2}}{ 3 } + C,
    \end{equation}
    et en divisant par \( -2\) nous trouvons la primitive demandée :
    \begin{equation}
        \int x\sqrt{1-x^2}\,dx=-\frac{ (1-x^2)^{3/2} }{ 3 } + C.
    \end{equation}
\end{example}

L'exemple suivant donne une façon plus économe de retenir la méthode du changement de variables.
\begin{example}\label{exempleprimitivechangvar}
    Soit à calculer
    \begin{equation}
        \int \cos(x) e^{\sin(x)}dx.
    \end{equation}
    Vu qu'il y a beaucoup de fonctions trigonométriques dans la fonction à intégrer, nous allons poser \( u(x)=\sin(x)\), et remplacer élément par élément tout ce qui contient du «$x$»  dans l'intégrale demandée par la quantité correspondante en termes de \( u\).

    La difficulté est de savoir ce que nous allons faire du «\( dx\)» dans l'intégrale. Ce \( dx \) marque une variation (infinitésimale) de \( x\). La formule des accroissements finis dit que si \( x\) augmente de la valeur \( dx\), alors \( u(x)\) augmente de $u'(x)dx$, c'est à dire que
    \begin{equation}
        du=\cos(x)dx.
    \end{equation}

    Nous avons donc les substitutions suivantes à faire :
    \begin{subequations}
        \begin{align}
            \sin(x)&=u\\
            du&=\cos(x)dx\\
            dx&=\frac{ du }{ \cos(x) }.
        \end{align}
    \end{subequations}
    La chose «magique» est que le \( \cos(x)\) se trouvant dans la fonction se simplifie avec le cosinus qui arrive lorsqu'on remplace \( dx\) par \( \frac{ du }{ \cos(x) }\). Les substitutions faites nous restons avec
    \begin{equation}
        \int\cos(x) e^{\sin(x)}dx=\int e^{u}du=e^u + C, \qquad \text{où } u= \sin(x).
    \end{equation}
   {\bf Attention : la réponse doit \^etre impérativement donnée en termes de \( x\) et non de \( u\).}

    Nous écrivons donc 
    \begin{equation}
        \int \cos(x) e^{\sin(x)}= e^{\sin(x)}+C.
    \end{equation}
\end{example}

%--------------------------------------------------------------------------------------------------------------------------- 
\subsection{Changement de variables -- pour calculer des intégrales}
%---------------------------------------------------------------------------------------------------------------------------

La définition \ref{defintegrale} fixe la relation entre la recherche des primitives de $f $ et la calcul de l'intégrale de $f$ sur l'intervalle d'extr\^emes $a$ et $b$. On a vu dans la section précédente comment utiliser le changement de variable pour trouver une primitive de $f$. Il faut maintenant comprendre comment appliquer ce qu'on a vu dans le calcul d'une intégrale. 

En effet nous avons le choix entre 
\begin{itemize}
\item trouver une primitive de $f$ comme dans la section précédente et appliquer ensuite la formule dans \ref{defintegrale} ; 
\item écrire une intégrale pour la nouvelle variable $u = u(x)$ sur l'intervalle entre $u(a)$ et $u(b)$.   
\end{itemize}

Nous allons voir ce deux méthodes dans des exemples. 

\begin{example}
    Soit à  calculer 
    \begin{equation}
        \int_{1/3}^{1/2}x\sqrt{1-x^2}dx.
    \end{equation}
   Les primitives $\int x\sqrt{1-x^2}dx$ ont été trouvé dans l'exemple \ref{exempleprimitivechangvar}. Une primitive est 
    \begin{equation}
        F(x)=\int x\sqrt{1-x^2}dx=-\frac{(1-x^2)^{3/2}}{ 3 }.
    \end{equation}
    Nous pouvons maintenant calculer l'intégrale de $x\sqrt{1-x^2}$ sur l'intervalle $[1/3, 1/2]$ par la définition
    \begin{equation}
        \int_{1/3}^{1/2}x\sqrt{1-x^2}dx=F\left(\frac{ 1 }{2}\right)-F\left(\frac{1}{ 3 }\right)=-\frac{ \sqrt{3} }{ 8 }+\frac{ 16\sqrt{2} }{ 81 }.
    \end{equation}
\end{example}
\begin{remark}
  Pour que le calcul d'intégrale donne quelque chose de sensé il faut absolument que la primitive soit écrite en tant que fonction de $x$ et non comme fonction de $u$. La méthode que nous allons voir dans l'exemple suivant réduit grandement la probabilité d'oublier ce détail, d'où le fait qu''elle soit de loin la plus utilisée. 
\end{remark}
\begin{example}
    Calculons à nouveau
    \begin{equation}
        \int_{1/3}^{1/2}x\sqrt{1-x^2}dx.
    \end{equation}
    Cette fois nous allons toucher à l'intervalle d'intégration en même temps que faire le changement de variables. Nous savons déjà les substitutions
    \begin{subequations}
        \begin{numcases}{}
            u=1-x^2\\
            du=-2xdx\\
            dx=\frac{ du }{ -2x }.
        \end{numcases}
    \end{subequations}
    En ce qui concerne les extr\^emes d'intégration, si \( x=1/3\) alors \( u=1-\frac{1}{ 9 }=\frac{ 8 }{ 9 }\) et si \( x=\frac{ 1 }{2}\) alors \( u=\frac{ 3 }{ 4 }\). Nous avons donc encore les substitutions suivantes  :
    \begin{subequations}
        \begin{numcases}{}
            x=1/3\to u=8/9\\
            x=1/2\to u=3/4
        \end{numcases}
    \end{subequations}
    Le calcul est alors
    \begin{equation}
        \int_{1/3}^{1/2}x\sqrt{1-x^2}dx=-\frac{ 1 }{2}\int_{8/9}^{3/4}\sqrt{u}du=-\frac{ 1 }{2}\left[  \frac{ u^{3/2} }{ 3/2 }    \right]_{8/9}^{3/4}=-\frac{ \sqrt{3} }{ 8 }+\frac{ 16\sqrt{2} }{ 81 }.
    \end{equation}
    Attention : la dernière égalité n'est pas immédiate; elle demande quelque calculs et une bonne utilisation des règles de puissances.
\end{example}
La deuxième méthode est plus utilisée et, avec un peu d'exercice, plus rapide à mettre en place que la première. 

\vspace{.5cm}

Jusqu'à présent nous avons utilisé des changement de variables dans lesquels nous exprimions \( u\) en termes de \( x\). Comme le montre l'exemple suivant, il est parfois fructueux d'utiliser le changement de variable dans le sens inverse : avec \( x\) exprimé en termes d'un paramètre.

\begin{example}\label{exemplepassagepolaires}
    À calculer :
    \begin{equation}
        \int_{1/2}^{\sqrt{3}/2}\sqrt{1-x^2}dx.
    \end{equation}
    Nous posons \( x=\sin(\theta)\) parce que nous savons que \( 1-\sin^2(\theta)=\cos^2(\theta)\); nous espérons que le changement de variables simplifie l'expression\footnote{Lorsqu'on fait un changement de variables, il s'agit toujours d'\emph{espérer} que l'expression se simplifie. Il n'y a pas moyen de savoir a priori si tel changement de variable va être utile. Il faut essayer.}. Les substitutions à faire dans l'intégrale sont :
    \begin{subequations}
        \begin{numcases}{}
            x=\sin(\theta)\\
            dx=\cos(\theta)d\theta,
        \end{numcases}
    \end{subequations}
    et en ce qui concerne les bornes, si \( x=1/2\) alors \( \sin(\theta)=\frac{ 1 }{2}\), c'est à dire \( \theta=\frac{ \pi }{ 6 }\). Si \( x=\sqrt{3}/2\) alors \( \theta=\frac{ \pi }{ 3 }\). Donc
    \begin{equation}
        \int_{1/2}^{\sqrt{3}/2}\sqrt{1-x^2}dx=\int_{\pi/6}^{\pi/3}\sqrt{1-\sin^2(\theta)}\cos(\theta)dt.
    \end{equation}
    Nous avons \( 1-\sin^2(\theta)=\cos^2(\theta)\) et vu que \( \theta\in\mathopen[ \frac{ \pi }{ 6 } , \frac{ \pi }{ 3 } \mathclose]\) nous avons toujours \( \cos(\theta)>0\), ce qui donne \( \sqrt{\cos^2(\theta)}=\cos(\theta)\). Nous devons donc calculer
    \begin{equation}
        \int_{\pi/6}^{\pi/3}\cos^2(\theta)d\theta.
    \end{equation}
    Pour celle-là, il faut utiliser une formule de trigonométrie\footnote{En fait, il y a moyen de terminer le calcul en intégrant deux fois par parties, mais c'est plus compliqué.} : 
    \begin{equation}
        \cos^2(\theta)=\frac{ 1+\cos(2\theta) }{ 2 }.
    \end{equation}
    Donc
    \begin{equation}
        \int_{\pi/6}^{\pi/3}\cos^2(\theta)d\theta=\int_{\pi/6}^{\pi/3}\frac{ 1+\cos(2\theta) }{2}d\theta=\left[ \frac{ \theta }{2}\right]_{\pi/6}^{\pi/3}+\int_{\pi/6}^{\pi/3}\frac{ \cos(2\theta) }{2}d\theta, 
    \end{equation}
    Pour calculer proprement la dernière intégrale nous effectuons un autre changement de variable (facile) en posant $t = 2\theta$, $dt = 2 d\theta$, $t(\pi/6) = \pi/3$ et $t(\pi/3) = 2\pi/3$, nous avons alors 
    \begin{equation}
        \int_{\pi/6}^{\pi/3}\cos^2(\theta)d\theta=\left[ \frac{ \theta }{2}\right]_{\pi/6}^{\pi/3}+\int_{\pi/3}^{2\pi/3}\frac{ \cos(t) }{4}dt  = \left[ \frac{ \theta }{2}\right]_{\pi/6}^{\pi/3}=\frac{ \pi }{ 6 }-\frac{ \pi }{ 12 }=\frac{ \pi }{ 12 }, 
    \end{equation}
    parce que \( \sin\big( \frac{ 2\pi }{ 3 } \big)=\sin\big( \frac{ \pi }{ 3 } \big)\). Au final,
    \begin{equation}
        \int_{1/2}^{\sqrt{3}/2}\sqrt{1-x^2}dx=\frac{ \pi }{ 12 }.
    \end{equation}
\end{example}

%--------------------------------------------------------------------------------------------------------------------------- 
\subsection{Intégrations des fractions rationnelles réduites}
%---------------------------------------------------------------------------------------------------------------------------

\begin{definition}
    Une \defe{fraction rationnelle}{fraction rationnelle} est un quotient de deux polynômes à coefficients réels ou complexes.
\end{definition}
Par exemple
\begin{equation}
    \frac{ x^5+7x^4-\frac{ x^3 }{2}+x }{ x^2-1 }
\end{equation}
est une fraction rationnelle.

Il sera expliqué dans le cours d'algèbre que toute fraction rationnelle peut être écrite sous forme d'une somme d'éléments simples, c'est à dire de fractions rationnelles d'un des deux types suivants :
\begin{subequations}
    \begin{align}
        \frac{ \alpha }{ (x-a)^m },&& \alpha,a\in \eR,m\in \eN  \label{CasMMIooZnZpUWi}\\
        \frac{ \alpha x+\beta }{ (x^2+ax+b)^m };&&\alpha, \beta ,a,b\in \eR,m\in \eN,a^2-4b<0. \label{CasMMIooZnZpUWii}
    \end{align}
\end{subequations}
Nous allons nous contenter de donner un exemple de chaque type.

\begin{enumerate}
    \item
        En ce qui concerne le cas \eqref{CasMMIooZnZpUWi} avec \( m=1\), nous avons par exemple
        \begin{equation}
            \int\frac{1}{ x-3 }dx=\ln\big( | x-3 | \big)+C .
        \end{equation}
        Si vous voulez en être tout à fait sûr, effectuez d'abord le changement de variables \( u=x-3\) qui donne \( dx=du\).
    \item 
        En ce qui concerne le cas \eqref{CasMMIooZnZpUWi} avec \( m\neq 1\), nous avons par exemple
        \begin{equation}
            \int\frac{1}{ (x-1)^4 }dx=-\frac{1}{ 3(x-1)^3 }+C.
        \end{equation}
        Encore une fois, pour s'en convaincre, utiliser le changement de variables \( u=x-1\), \( dx=du\) :
        \begin{equation}
            \int\frac{1}{ (x-1)^4 }dx=\int\frac{1}{ u^4 }du=\int u^{-4}du=-\frac{ u^{-3} }{ 3 }+C=-\frac{1}{ 3 }\frac{1}{ (x-1)^3 }+C.
        \end{equation}
    \item
        En ce qui concerne le cas \eqref{CasMMIooZnZpUWii} avec \( \alpha\neq 0\), nous avons par exemple
        \begin{equation}
            \int\frac{ x }{ x^2+4 }dx=\frac{ 1 }{2}\ln(x^2+4)+C.
        \end{equation}
        Pour ce faire, il faut faire le changement de variables \( u=x^2+4\), \( du=2xdx\), \( dx=\frac{ du }{ 2x }\) qui donne
        \begin{equation}
            \int \frac{ x }{ x^2+4 }dx=\frac{ 1 }{2}\int\frac{ du }{ u }=\frac{ 1 }{2}\ln(| u |)+C=\frac{ 1 }{2}\ln(| x^2+4 |)+C.
        \end{equation}
        Dans ce cas nous pouvons oublier d'écrire la valeur absolue dans le logarithme parce que de toutes façons, \( x^2+4\) est toujours positif.
    \item
        En ce qui concerne le cas \eqref{CasMMIooZnZpUWii} avec \( \alpha= 0\), nous avons par exemple
        \begin{equation}
            \int\frac{ dx }{ x^2+4 }=\frac{1}{ 4 }\int\frac{ dx }{ (\frac{ x }{2})^2+1 }=\frac{ 1 }{2}\arctan(\frac{ x }{2})+C.
        \end{equation}
        où nous avons utilisé la primitive \( \int \frac{dx}{ x^2+1 }dx=\arctan(x)\) du tableau de la page \pageref{PageLCHooMbWjOj}. Pour vous en convaincre vous pouvez faire la dernière étape avec le changement de variables \( u=x/2\), \( dx=2du\).
\end{enumerate}

%--------------------------------------------------------------------------------------------------------------------------- 
\subsection{Quelques formules à conna\"{i}tre}
%---------------------------------------------------------------------------------------------------------------------------

\begin{Aretenir}
  \begin{subequations}
    \begin{equation}
      \int \left(\alpha f(x) + \beta g(x)\right) \, dx = \alpha \int f(x) \, dx + \beta \int g(x) \, dx.
    \end{equation}
    \begin{equation}
      \int f(x) g'(x) \, dx = f(x)g(x) - \int f'(x) g(x) \, dx. 
    \end{equation}
    \begin{equation}
      \int f'(u(x))u'(x)\, dx = \int f(t)\, dt, \qquad \text{avec } t = u(x). 
    \end{equation}
    \begin{equation}
      \int \frac{f'(x)}{f(x)} \, dx = \log |f(x)| + C, \qquad \text{c'est un cas particulier de la formule précédente.}
    \end{equation}
  \end{subequations}
\end{Aretenir}
