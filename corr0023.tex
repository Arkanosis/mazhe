% This is part of Exercices et corrigés de CdI-1
% Copyright (c) 2011,2013
%   Laurent Claessens
% See the file fdl-1.3.txt for copying conditions.

\begin{corrige}{0023}

Nous allons faire un usage intensif (et sans justifications trop poussées) des choses dites autour de la règle de l'Hospital.

\begin{enumerate}

\item
 Nous commençons par mettre en évidence le plus haut degré de $x$ au numérateur et au dénominateur :
\begin{equation}
	\frac{ x+1 }{ x^2+2 }=\frac{ x\left( 1+\frac{1}{ x } \right) }{ x^2\left( 1+\frac{ 2 }{ x^2 } \right) }=\frac{ 1+\frac{1}{ x } }{ x\left( 1+\frac{ 2 }{ x^2 } \right) }.
\end{equation}
Fixons $\epsilon>0$, et considérons $X_1$ tel que $x>X_1$ implique $1+\frac{1}{ x }<1+\epsilon$. Nous pouvons aussi choisir $X_2$ tel que $x>X_2$ implique $x\left( 1+\frac{ 2 }{ x^2 } \right)$. En prenant le maximum de $X_1$ et $X_2$, nous trouvons
\begin{equation}
\frac{ 1+\frac{1}{ x } }{ x\left( 1+\frac{ 2 }{ x^2 } \right) }<\frac{ 1+\epsilon }{ M }.
\end{equation}
En prenant $M$ arbitrairement grand, nous pouvons rendre cette fraction arbitrairement petite. La limite cherchée est donc zéro.

\begin{alternative}
En utilisant la règle de l'Hospital, nous trouvons tout de suite
\begin{equation}
	\lim_{x\to\infty}\frac{ x+1 }{ x^2+2 }=\lim_{x\to\infty}\frac{ 1 }{ 2x }=0.
\end{equation}
\end{alternative}

\item
Le candidat limite n'est pas compliqué à deviner : c'est zéro parce que $\sin(x)$ reste borné tandis que le $x$ au dénominateur vient l'écraser. Nous avons très vite une majoration
\begin{equation}
	\left| \frac{ \sin(x)}{ x } \right| <\frac{1}{ x }\to 0.
\end{equation}
La limite est donc zéro. Notez que la règle de l'Hospital n'est pas valable parce que $\lim_{x\to\infty}\sin(x)$ n'existe pas.

\item
Ici par contre, la règle de l'Hospital fonctionne parce que $\lim_{x\to0}\sin(x)$ existe, et vaut zéro. Nous avons donc
\begin{equation}
	\lim_{x\to 0}\frac{ \sin(x) }{ x }=\lim_{x\to 0}\frac{ \cos(x) }{ 1 }=1.
\end{equation}

\item
Si $n$ est un entier positif, nous trouvons
\begin{equation}
	\lim_{x\to\infty}\frac{ x^n }{  e^{x} }=\lim_{x\to\infty}\frac{ nx^{n-1} }{ e^x }=\ldots=\lim_{x\to \infty}\frac{ n! }{ e^x }=0,
\end{equation}
en appliquant $n$ fois la règle de l'Hospital. Dans le cas où $n$ est un entier négatif, nous tombons sur un cas $0\cdot0=0$.

\item
Le changement de variable $u=x/a$ donne
\begin{equation}
	\left( 1+\frac{ a }{ x } \right)^x=\left( 1+\frac{1}{ u } \right)^{au}=\left[ \left( 1+\frac{1}{ u } \right)^u \right]^a.
\end{equation}
En utilisant le fait que $\lim_{t\to\infty}\left( 1+\frac{1}{ t } \right)^t=e$ (voir la remarque à la page 141), nous trouvons alors
\begin{equation}
	\lim_{x\to\infty}\left( 1+\frac{ a }{ x } \right)^x=e^a.
\end{equation}

\item
En mettant au même dénominateur, nous trouvons un cas $\frac{ 0 }{ 0 }$ qui peut se traiter en utilisant deux fois la règle de l'Hospital :
\begin{equation}
	\begin{aligned}[]
	\lim_{x\to\infty}\left( \frac{1}{ \sin(x) }-\frac{1}{ x } \right)&=\lim_{x\to 0}\left( \frac{ x-\sin(x) }{ x\sin(x) } \right)\\
				&=\lim_{x\to 0}\frac{ 1-\cos(x) }{ x\cos(x)+\sin(x) }\\
				&=\lim_{x\to 0}\frac{ \sin(x) }{ -x\sin(x)+\cos(x)+\cos(x) }\\
				&=\frac{ 0 }{ 2 }=0.
	\end{aligned}
\end{equation}

\item
La limite n'existe pas parce que $\forall X>0$, $\exists x_0,x_1>X$ tels que $\cos(2\pi x_0)=1$ et $\cos(2\pi x_1)=0$.

\item
Nous savons que $\sin(x)+2\in[1,3]$, donc
\begin{equation}
	\frac{ x }{ \sin(x)+2 }+\ln(x)\cos(x)>\frac{ x }{ 3 }+\ln(x)\cos(x)>\frac{ x }{ 3 }-\ln(x),
\end{equation}
qui est un cas $\infty-\infty$. Étudions la fonction
\begin{equation}		\label{Eq0023ffrac}
	f(x)=\frac{ x }{ 3 }-\ln(x).
\end{equation}
La dérivée de $f$ vaut $f'(x)=\frac{1}{ 3 }-\frac{1}{ x }>\frac{1}{ 4 }$. La fonction $f(x)$ majore donc une droite de coefficient directeur $\frac{ 1 }{ 4 }$ et tend donc vers l'infini.

\item
Nous majorons d'abord $|\sin(x)+2|$ par $3$, et puis nous appliquons la règle de l'Hospital :
\begin{equation}
	\lim_{x\to\infty}\left| \frac{ \ln(x)\big( \sin(x)+2 \big) }{ x } \right|<\lim_{x\to\infty}3\frac{ \ln(x) }{ x }=\lim_{x\to\infty}\frac{ 1/x }{ 1 }=0.
\end{equation}

\item
Pour ce dernier, nous utilisons le passage par $y= e^{\ln(y)}$ :
\begin{equation}
	x^{1/x}= e^{\ln(x^{1/x})}= e^{\frac{1}{ x }\ln(x)}.
\end{equation}
Mais nous savons déjà que $\lim_{x\to\infty}\frac{1}{ x }\ln(x)=0$, donc la limite cherchée est $ e^{0}=1$.

\end{enumerate}

\begin{alternative}

	\begin{enumerate}

		\item

		\item $\displaystyle{ \lim_{x\rightarrow \infty}\f{\sin(x)}{x}}$\hs

		Par la règle de l'étau: $ \f{-1}{x} \, \leq\, \f{sin(x)}{x} \,\leq \f{1}{x}$. Comme $\f{1}{x}$ tend vers $0$ quand $x$ tend vers $\infty$, on a que $ \lim_{x\rightarrow\infty}\f{\sin(x)}{x} \, =\,0$

		\item $ \dst{\lim_{x\rightarrow 0}\f{\sin(x)}{x}}$\hs

		Un cas $\f{0}{0}$. On peut donc appliquer la règle de l'Hospital:

		 \[\lim_{x\rightarrow 0}\f{\sin(x)}{x} \; =^{H}\; \lim_{x\rightarrow 0}\f{\cos(x)}{1}\;=\; \cos(0)\;=\;1\] o\`{u} dans la dernière limite on utilise le fait que la fonction $\cos(x)$ est continue en $0$.
		 
		 
		 \item $ \dst{\lim_{x\rightarrow \infty }\f{x^n}{e^x}}$\hs
		 
		Un cas $\f{\infty}{\infty}$. On peut donc appliquer la règle de l'Hospital:
	\begin{equation}
	\begin{aligned}[]
		\lim_{x\rightarrow \infty}\f{x^n}{e^x}&=^H\;  \lim_{x\rightarrow \infty}\frac{nx^{n-1}}{e^x}\\
			&=^H\;  \lim_{x\rightarrow \infty}\f{n(n-1)x^{n-2}}{e^x}\\
			&=^H\; \ldots\\
			&=^H\lim_{x\rightarrow \infty}\f{n!}{e^x}\;=\;0
	\end{aligned}
\end{equation}	 
		 où dans la dernière limite on utilise le fait que la fonction $e^x \rightarrow  \infty$ quand $x\rightarrow  \infty$, et $n!$ reste borné.
		 
		\item $\dst{ \lim_{x\rightarrow  \infty}(1+\f{a}{x})^x}$ \hs
		 
		Ici, on transforme: $ (1+\f{a}{x})^x = e^{x\ln(1+\f{a}{x})}$. Comme l'exponentielle est continue, on peut passer à la limite dans l'exponentielle, et l'exercice devient de calculer:
		 \[\lim_{x\rightarrow  \infty}x\ln(1+\f{a}{x}) = \lim_{x\rightarrow  \infty}\dfrac{\ln(1+\f{a}{x})}{\f{1}{x}}\]
		qui est un cas $\f{0}{0}$. On a donc, par l'Hospital:
		  \[\lim_{x\rightarrow  \infty}\dfrac{\ln(1+\f{a}{x})}{\f{1}{x}} \; =^H\;   \lim_{x\rightarrow  \infty}\dfrac{1/(1+\f{a}{x})\f{-a}{x^2}}{\f{-1}{x^2}} \; = \;      a\lim_{x\rightarrow  \infty}\dfrac{1}{(1+\f{a}{x})}   \;=\; a       \]
		Cet argument est valable quel que soit $a$, et le résultat final est donc  
		 
		 \[\dst{ \lim_{x\rightarrow  \infty}(1+\f{a}{x})^x} \;= \; e^a\] 
		 
		\item

		 \item $\dst{\lim_{x\rightarrow \infty} \cos(2\pi x)}$\hs
		 
		 On voit que cette fonction ne peut converger en l'infini. Pour le prouver, on peut par exemple prendre deux manières différentes d'aller vers l'infini qui donneront deux limites différentes de la fonction. 
		Si on prend $x_k=k$, et $y_k = \f{2k+1}{2}$, les deux suites tendent vers l'infini, mais  
		 $\dst{\lim_{k\rightarrow \infty} \cos(2\pi x_k)} = 1$ alors que $\dst{\lim_{k\rightarrow \infty} \cos(2\pi y_k)} =  -1$, ce qui prouve que la fonction ne converge pas quand $x$ tend vers l'infini.
		 
		\item 

		  \item $\dst{\lim_{x\rightarrow \infty} \f{\ln(x)(\sin(x)+2)}{x}}$\hs
		 
		 Par la règle de l'étau:\[\dst{\f{\ln(x)}{x} \leq \f{\ln(x)(\sin(x)+2)}{x} \leq 3\f{\ln(x)}{x} \, \forall x}\]
		 et, $\dst{\lim_{x\rightarrow \infty} \f{\ln(x)}{x} \; =^H \; \lim_{x\rightarrow \infty} \f{1/x}{1}  \;=\;0}$.
		 
	\end{enumerate}
\end{alternative}

\end{corrige}
