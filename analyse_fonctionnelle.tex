% This is part of Mes notes de mathématique
% Copyright (c) 2012-2013
%   Laurent Claessens
% See the file fdl-1.3.txt for copying conditions.


\begin{theorem}[Théorème d'isomorphisme de Banach]  \label{ThofQShsw}
    Une application linéaire continue et bijective entre deux espaces de Banach est un homéomorphisme.
\end{theorem}

%+++++++++++++++++++++++++++++++++++++++++++++++++++++++++++++++++++++++++++++++++++++++++++++++++++++++++++++++++++++++++++
\section{Espaces \texorpdfstring{$L^p$}{Lp}}
%+++++++++++++++++++++++++++++++++++++++++++++++++++++++++++++++++++++++++++++++++++++++++++++++++++++++++++++++++++++++++++

Pour la proposition suivante nous notons \( [f]\) la classe dans \( L^p\) de la fonction \( f\colon \Omega\to \eC\). Donc \( f\) représente une vraie fonction.
\begin{proposition}[\cite{bJOSNQ}]  \label{PropWoywYG}
    Soit \( 1\leq p\leq \infty\) et supposons que la suite \( [f_n]\) dans \( L^p(\Omega,\tribF,\mu)\) converge vers \( [f]\) au sens \( L^p\). Alors il existe une sous-suite \( (h_n)\) qui converge ponctuellement \( \mu\)-presque partout vers \( f\).
\end{proposition}
\index{complétude}
\index{espace!\( L^p\)}
\index{suite!de fonctions}
\index{limite!inversion}

\begin{proof}
    Si \( p=\infty\) nous sommes en train de parler de la convergence uniforme et il ne faut même pas prendre ni de sous-suite ni de «presque partout».

    Supposons que \( 1\leq p<\infty\). Nous considérons une sous-suite \( [h_n]\) de \( [f_n]\) telle que
    \begin{equation}
        \| [h_j]-[f] \|_p<2^{-j},
    \end{equation}
    puis nous posons \( u_k(x)=| h_k(x)-f(x) |^p\). Notons que ce \( u_k\) est une vraie fonction, pas une classe. Et en plus c'est une fonction positive. Nous avons
    \begin{equation}
        \int_{\Omega}u_kd\mu=\int_{\omega}| h_k(x)-f(x) |^pd\mu(x)=\| h_k-f \|_p^p\leq 2^{-kp}.
    \end{equation}
    Vu que \( u_k\) est une fonction positive la suite des sommes partielles de \( \sum_ku_k\) est croissante et vérifie donc le théorème de la convergence monotone \ref{ThoConvMonFtBoVh} :
    \begin{subequations}
        \begin{align}
            \int_{\Omega}\left( \sum_{k=0}^{\infty}u_k(x) \right)d\mu(x)&=\sum_{k=0}^{\infty}\int_{\Omega}u_k(x)d\mu(x)\\
            &\leq\sum_{k=0}^{\infty}2^{-kp}<\infty.
        \end{align}
    \end{subequations}
    Le fait que l'intégrale de la fonction \( \sum_ku_k\) est finie implique que cette fonction est finie \( \mu\)-presque partout. Donc le terme général tend vers zéro presque partout, c'est à dire
    \begin{equation}
        | h_k(x)-f(x) |^p\to 0.
    \end{equation}
    Cela signifie que \( h_k\to f\) presque partout ponctuellement.
\end{proof}

Est-ce qu'on peut faire mieux que la convergence ponctuelle presque partout d'une sous-suite ? En tout cas on ne peut pas espérer grand chose comme convergence pour la suite elle-même, comme le montre l'exemple suivant.

\begin{example}
    Nous allons montrer une suite de fonctions qui converge vers zéro dans \( L^p[0,1]\) (avec \( p<\infty\)) mais qui ne converge ponctuellement pour \emph{aucun} point. Cet exemple provient de \href{http://www.bibmath.net/dico/index.php?action=affiche&quoi=./b/bosseglissante.html}{bibmath.net}. 

    Nous construisons la suite de fonctions par paquets. Le premier paquet est formé de la fonction constante \( 1\).

    Le second paquet est formé de deux fonctions. La première est \( \mtu_{\mathopen[0 , 1/2 \mathclose]}\) et la seconde \( \mtu_{\mathopen[ 1/2 , 1 \mathclose]}\).

    Plus généralement le paquet numéro \( k\) est constitué des \( k\) fonctions \( \mtu_{\mathopen[ i/k , (i+1)/k \mathclose]}\) avec \( i=0,\ldots, k-1\).

    Vu que les fonctions du paquet numéro \( k\) ont pour norme \( \| f \|_p=\frac{1}{ k }\), nous avons évidemment \( f_n\to 0\) dans \( L^p\). Il est par contre visible que chaque paquet passe en revue tous les points de \( \mathopen[ 0 , 1 \mathclose]\). Donc pour tout \( x\) et pour tout \( N\), il existe (même une infinité) \( n>N\) tel que \( f_n(x)=1\). Il n'y a donc convergence ponctuelle nulle part.
\end{example}
