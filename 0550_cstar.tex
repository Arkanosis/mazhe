% This is part of (almost) Everything I know in mathematics
% Copyright (c) 2013-2014
%   Laurent Claessens
% See the file fdl-1.3.txt for copying conditions.

The main references for this chapter are \cite{Dixmier,Landsman}. In this chapter, all algebras are over $\eC$, or $\eR$ when it is mentioned. Definition and spectral properties of Banach algebras are given in chapter \ref{Sec_SpecBanach}.

\subsection*{Notations is Landsman and Dixmier}
%-----------------------------------------------

There are some differences in notations, definitions and conventions between the book of Dixmier and the lecture notes of Landsman. Here I summarize it. As much as possible, I try to follow Landsman.

In \lref{2.2.2}, the spectral radius is denoted by $r(x)$, while it is $\rho(x)$ in \dixref{B1}. In Landsman, $\rho(x)$ is the resolvent of $x$. \dixref{1.1.5, 1.1.6} define the spectrum separately for the algebras with and without unit. It notes it by $Sp\,x$ and $Sp'\,x$.

The set of the characters of $\cA$ is called the \defe{spectrum}{spectrum} of $\cA$ in Dixmier; it is the $\Delta(\cA)$ in Landsman. The Gelfand transform of $x\in\cA$ is the map $\dpt{\hx}{\Delta(\cA)}{\eC}$, $\hx(\omega):=\omega(x)$. If $\cA$ admits an unit, Dixmier defines
\[
   Sp_{\cA}x:=\{\hx(\omega):\omega\in\Delta(\cA)\}.
\]
Thanks to \leref{2.50}, the two systems of definitions are the same, but there is a problem on the word ``spectrum ''. Here I use it in the sense of Landsman, and I write the structure space by his notation $\Delta(\cA)$.

\section{Commutative Banach algebra}
%+++++++++++++++++++++++++++++++++++

We suppose the Banach algebra $\cA$ to be commutative.

\subsection{Structure space}
%----------------------------

\begin{definition}      \label{DefStructureSpaceDel}
    The \defe{structure space}{structure!space} $\Delta(\cA)$\nomenclature[C]{$\Delta(\cA)$}{structure space if the $C^*$-algebra $\cA$} of a commutative algebra is the set of the nonzero linear maps $\dpt{\omega}{\cA}{\eC}$ such that $\forall A$, $B\in\cA$,
\[
    \omega(AB)=\omega(A)\omega(B).
\]
We say that an element of this space is a \defe{character}{character!of an algebra}, or a \defe{multiplicative}{multiplicative} map of $\cA$.
\end{definition}
\lref{2.3.1}

\begin{proposition}
Let $\cA$ be an unital commutative Banach algebra. Then for any $\omega\in\Delta(\cA)$,
\begin{enumerate}

\item $\omega(\mtu)=1$.
\item the character $\omega$ is bounded (and then continuous from \ref{prop:cont_born}) with norm $\|\omega\|=1$ and for all $A\in\cA$,
\begin{equation} \label{eq:omAleqnA}
  \|\omega(A)\|\leq \|A\|.
\end{equation}
\end{enumerate}
\end{proposition}
\lref{2.3.2}


\begin{proof}
The first claim is obvious because $\omega(A)=\omega(\mtu A)=\omega(\mtu)\omega(A)$.  For the second one, we know from lemma \ref{lem:cv_Ak} that $(A-z)$ is invertible when $|z|>\|A\|$. By 
linearity,
\[
\omega(A-z)=\omega(A)-z\neq 0
\]
because $\omega$ in a homomorphism. Now remark that $A-z$ is invertible implies $|\omega(A)|\neq |z|$. Indeed let us suppose the opposite, then there exists a $\alpha\in\eR$ such that $\omega(A)=e^{i\alpha}z$, but $|e^{i\alpha}z|=|z|$. Conclusion: if $|z|>\|A\|$, then $|\omega(A)|\neq|z|$. This immediately yields $|\omega(A)|\leq\|A\|$.

From there, it is clear that $\|\omega\|=1$ because the norm is the supremum of $|\omega(A)|$ with $\|A\|=1$. Since $\omega(\mtu)=1$, $\|\omega\|\geq 1$, but what we just showed implies $\|\omega\|\leq 1$.

\end{proof}

\begin{theorem}
Let $\cA$ be an unital commutative Banach algebra. Then we have a bijection between $\Delta(\cA)$ and the set of maximal ideals in $\cA$. More precisely,

\begin{enumerate}
\item $\ker(\omega)$ is an ideal,                   \label{enuei}
\item $\omega_1=\omega_2$ if and only if $\ker\omega_1=\ker\omega_2$,   \label{enueii}  
\item each maximal ideal is the kernel of an element in $\Delta(\cA)$.  \label{enueiii}
\end{enumerate}\lref{2.3.3}\label{tho:ideal_kernel}
\end{theorem}


\begin{proof}
\ref{enuei} Since $\omega$ is continuous, the set $\ker(\omega)$ is closed. It is also clear that is $Z\in\ker(\omega)$, then $AZ\in\ker(\omega)$ for all $Z\in\cA$ because $\omega$ is multiplicative. Then $\ker(\omega)$ is an ideal. In order to see that it is a maximal ideal, remark that $\omega(X)=0$ is a linear equation which describe a vector subspace of $\cA$ of codimension\label{pg_codimun} $1$.

\ref{enueii} In any vector space, $\ker{\omega_1}=\ker{\omega_2}$ implies that $\omega_1$ and $\omega_2$ are multiples each others. In the case of $\Delta(\cA)$, this in turn implies the equality.

\ref{enueiii} Let $\cI\neq\cA$ be a maximal ideal and $B\neq 0$ outside $\cI$. Consider
\[
\cI_B=\{BA+J\tq A\in\cA\textrm{ and }J\in\cI\}.
\]
By construction it is a left-ideal and by commutativity of $\cA$, it is an ideal. We have $\cI\subsetneq\cI_B$. Since $\cI$ is maximal, the conclusion is $\cI_B=\cA$. In particular $\cun=BA+J$ for a suitable choice of $A\in\cA$ and $J\in\cI$. For these,
\[
  \tau(\cun)=\tau(BA)=\tau(B)\tau(A),
\]
but $B$ is arbitrary. Then any element of $\cA/\cI$ is invertible and the Gelfand-Mazur theorem (corollary \ref{cor:GelfandMazur}) concludes $\cA/\cI\simeq\eC$. Let $\dpt{\psi}{\cA/\cI}{\eC}$ be the isomorphism. We consider
        \begin{equation}
        \begin{aligned}
            \omega \colon \cA &\to \eC\
            A&\mapsto \psi(\tau(A)).
        \end{aligned}
    \end{equation}  
It is clearly linear (because $\psi$ and $\tau$ are) and $\omega(A)\omega(B)=\omega(AB)$. Furthermore $\omega(B)\neq 0$ and $\omega(\cun)=1$ are two good reasons to conclude that $\omega\neq 0$. Then $\omega\in\Delta(\cA)$. It remains to be proved that $\cI=\ker\omega$. First, $\cI=\ker\tau$, then $\cI\subseteq\ker\omega$. But when $B\notin\cI$, we have $\omega(B)\neq 0$, then $\cI=\ker\omega$. This finish the proof.
\end{proof}


\begin{theorem}[Banach-Alaoglu] 
If $X$ is a closed normed vector space, then the unit closed ball in the dual $X^*$ is compact for the $x^*$-topology.
 \label{tho:Banach_Alaoglu}
\end{theorem}

\begin{proposition}
When $\cA$ is an unital commutative Banach algebra, the space $\Delta(\cA)$ is compact and Hausdorff for the Gelfand topology.
\end{proposition} \label{prop:DcA_comp_Hauss}\lref{2.3.4}

\begin{proof}
We first prove that $\Delta(\cA)$ is closed by showing that it contains all limits of converging sequences\footnote{It is no related to \emph{complete} spaces in which any Cauchy sequence converge}. Let us take a sequence $\omega_n\to\omega$ with $\omega_n\in\Delta(\cA)$. We will show that $\omega\in\Delta(\cA)$:
\[ 
| \omega(AB)-\omega(A)\omega(B) | \leq| \omega(AB)-\omega_n(AB) |+| \omega_n(A)\omega_n(B)-\omega(A)\omega(B) |,
\]
but 
\[ 
 \begin{split}
\omega_n(A)\omega_n(B)-\omega(A)\omega(B)&=[\omega_n(A)-\omega(A)]\omega_n(B)+\omega(A)[\omega_n(B)-\omega(B)]\\
        &\leq | \omega_n(A)-\omega(A) |\| B \|+\| A \| \omega_n(B)-\omega(B) |
\end{split} 
\]
because $\omega_n(B)\leq \| B \|$. Taking the limit $n\to\infty$, we find
\[ 
 \begin{split}
| \omega(AB)-\omega(A)\omega(B) |&\leq| \omega(AB)-\omega_n(AB) |\\
        &\quad+| \omega_n(A)-\omega(A) |\| B \|\\
        &\quad+\| A \| |\omega_n(B)-\omega(B) |\to 0.
\end{split} 
\]
This proves that $\omega\in\Delta(\cA)$ and therefore that $\Delta(\cA)$ is closed. Since $\| \omega \|=1$ for all $\omega$, we have $\Delta(\cA)\subset\cA_1^*$, the unit ball in $\cA^*$. Theorem \ref{tho:Banach_Alaoglu} claims that $\cA_1^*$ is compact in the Gelfand topology. So $\Delta(\cA)$ is closed in a compact. This makes $\Delta(\cA)$ compact by lemma \ref{lem:ferme_compact}.

Now, we check that it is also Hausdorff. If $\omega\neq\eta\in\Delta(\cA)$, there exists a $A\in\cA$ such that $\omega(A)\neq\eta(A)$. We thus consider $\mO$ and $\mO'$, two disjoints open subsets of $\eC$ around $\omega(A)$ and $\eta(A)$ respectively. With these definition, it is easy to see that $\hat A^{-1}(\mO)$ and $\hat A^{-1}(\mO')$ are disjoints neighbourhoods of $\omega$ and~$\eta$.
\end{proof}
 

\subsection{Topology on \texorpdfstring{$\Delta(\cA)$}{DA}}\label{subsec:topo_Delta}
%----------------------------------------------
We begin to put the \defe{$w^*$-weak topology}{$w^*$-weak topology} on $\cA^*$ which defined by the convergence notion $\omega_n\to\omega$ if and only if $\omega_n(A)\to\omega(A)$ for all $A\in\cA$. 

The \defe{Gelfand topology}{Gelfand!topology} is the induced topology from $\cA^*$ on $\Delta(\cA)$. Let us define the \defe{Gelfand transform}{Gelfand!transform}
        \begin{equation}
        \begin{aligned}
            \hat A \colon \Delta(\cA) &\to \eC\\
            \hat A(\omega)&\mapsto \omega(A).
        \end{aligned}
    \end{equation}  
General theory of functional analysis shows that the $w^*$-weak topology is the weakest in which all linear functional are continuous, so a basis of this topology is given by sets of the form $\{f\in\cA^*\tq f(A)\in\mathcal{O}\}$ where $\mathcal{O}$ is an open in $\eC$ and $A\in\cA$.

A basis of the Gelfand topology is the intersection of these set with $\Delta(\cA)$:
\begin{equation}
  \hat A^{-1}(\mathcal{O})=\{\omega\in\Delta(\cA)\tq \omega(A)\in\mathcal{O}\}.
\end{equation}

%
% Ceci est une répétition de 2.3.4 qui est retapé un peu plus haut. Optimization quand tu nous tient !
%
%\begin{proposition}
%When $\cA$ is an unital commutative Banach algebra, $\Delta(\cA)$ is compact and Hausdorff in the Gelfand topology.
%\label{prop:DcA_comp_Hauss}\lref{2.3.4}
%\end{proposition}

%\begin{proof}
%First, we show that $\Delta(\cA)$ is closed by showing that the limits of converging sequences are in $\Delta(\cA)$. Let us consider $\omega_n\to\omega$ with $\omega_n\in\Delta(\cA)$ for any $n$. We have to see that $\omega\in\Delta(\cA)$.

%First remark that
%\begin{equation}
%\begin{split}
%|\omega(AB)-\omega(A)\omega(B)|&=|  \omega(AB)-\omega_n(AB)+\omega_n(A)\omega_n(B) -\omega(A)\omega(B) |\\
 %                              &\leq|\omega(AB)-\omega_n(AB)|
%                      +|\omega_n(A)\omega_n(B)-\omega_(A)\omega_(B)|\\
%               &\leq |\omega(AB)-\omega_n(AB)|\\
%               &\quad +|\omega_n(A)-\omega(A)|\|B\|\\
%               &\quad +\|A\||\omega_n(B)-\omega(B)|,
%\end{split}
%\end{equation}
%but the right hand side converges to zero when $n$ becomes large, so that $\omega\in\Delta(\cA)$.

%Since the norm of any element of $\Delta(\cA)$ is one, $\Delta(\cA)\subset\cA^*_1$, the unit ball in $\cA^*$: $\cA^*_1=\{\rho\in\cA^*=\|\rho\|=1\}$.

%\begin{theorem}[Banach-Alaoglu] 
%If $X$ is a closed normed vector space, then the unit closed ball in the dual $X^*$ is compact for the $x^*$-topology.
% \label{tho:Banach_Alaoglu}
%\end{theorem}

%This theorem makes $\cA^*_1$ compact. Thus $\Delta(\cA)$ is a closed subspace of a compact space. It is then compact by lemma \ref{lem:ferme_compact}.
%\end{proof}


\begin{lemma}
An element $A\in\cA$ is invertible if and only if $\omega(A)\neq0$ for all $\omega\in\Delta(\cA)$.
\end{lemma}


\begin{proof}
Let $A$ be an invertible element in $A$ and $\omega\in\Delta(\cA)$ such that $\omega(A)=0$. Then 
\[
  1=\omega(\cun)=\omega(A)\omega(A^{-1})=0.
\]

Let us take now a $A\notin G(\cA)$, then the ideal $\cI_A:=\{AB\tq B\in\cA\}$ don't contain $\cun$ and is not a proper ideal. From choice axiom, $\cI_A$ is contained in a maximal ideal $\cI$. From \ref{enueiii} of \ref{tho:ideal_kernel}, there exists a $\omega\in\Delta(\cA)$ whose kernel is $\cI$. In particular, $\omega(A)=0$.
\end{proof}

\begin{theorem}
Let $\cA$ be an unital commutative Banach algebra. Then
\begin{enumerate}
\item The Gelfand transform is a homomorphism $\cA\to C(\Delta(\cA))$. \label{enugi}
\item The image of $\cA$ under the Gelfand transform separates the points in $\Delta(\cA)$, see definition \ref{def:separe}. \label{enugii}
\item \label{enugiii} The spectrum of $A\in\cA$ is 
\[
   \sigma(A)=\sigma(\hat A)=\{\hat A(\omega):\omega\in\Delta(\cA)\}.
\]
\item \label{enugiv} The Gelfand transform is a \defe{contraction}{contraction}:   $\|\hat A\|_{\infty}\leq\|A\|$.
\end{enumerate}\label{tho:unital_comm}
\end{theorem}
 \lref{2.3.5} 
\begin{proof}
Item \ref{enugi} is easy: $\widehat{AB}(\omega)=\omega(AB)=\omega(A)\omega(B)=\hat A(\omega)\hB(\omega)$. Point \ref{enugii} is immediate too: let $\omega_1\neq\omega_2\in\Delta(\cA)$. We need a $A\in\cA$ such that $\hat A(\omega_1)\neq\hat A(\omega_2)$. But the definition of the inequality $\omega_1\neq\omega_2$ is the existence of a $A\in\cA$ such that $\omega_1(A)\neq\omega_2(A)$.

For \ref{enugiii}, recall that
\[
  \rho(A)=\{z\in\eC\tq(A-z)^{-1}\textrm{ exists}\}.
\]
From the lemma the existence of $(A-z)^{-1}$ makes that $\forall\omega\in\Delta(\cA)$, $\omega(A)\neq z$. So the complementary is
\begin{equation}
\begin{split}
 \sigma(A)&=\{z\in\eC\tq\exists\omega\in\Delta(\cA)\textrm{ such that }\omega(A)=z\}\\
          &=\{\omega(A)\tq\omega\in\Delta(\cA)\}\\
          &=\{\hat A(\omega)\tq\omega\in\Delta(\cA)\}.
\end{split}
\end{equation}

The fifth point comes from definition \ref{def:sup_norm} and the fact that, because of the third point,  $r(A)=sup\{\hat A(\omega):\omega\in\Delta(\cA)\}$.
Therefore
\[
\|\hat A\|_{\infty}=\sup_{\omega\in\Delta(\cA)}|\hat A(\omega)|=r(A)\leq\|A\|.
\]
\end{proof}

When a Banach algebra is non unital, one can extend it to $\cA_{\cun}$ and a character $\omega\in\Delta(\cA)$ can be extended too as $\tilde{\omega}\in\Delta(\cA_{\cun})$ by
\[
  \tilde{\omega}(A+\lambda\cun)=\omega(\cun)+\lambda.
\]
The fact that it is multiplicative is a simple computation.

\begin{theorem}
For every element $A$ of a commutative Banach algebra $\cA$, we have $\Spec(A)=\Spec(\hat A)$.
\end{theorem}

\begin{proof}
We want to prove that when $\lambda\in\Spec(A)$, there exists a $\varphi$ such that $\varphi(A)=\lambda$. The ideal generated by $(A-\lambda)$ is a proper ideal which is thus contained in a maximum ideal $M$ by Zorn's lemma. This maximal ideal is closed (if not, the closure would be bigger ideal). Consider an element $x$ in the quotient $\cA/M$. Since $\Spec(x)\neq\emptyset$, the element $(x-\lambda)$ is not invertible for some $\lambda$. That provides an isomorphism $\cA/M\simeq \eC$, and we define $\varphi$ as the composition of that isomorphism by the projection of $\cA$ into $\cA/M$. For this $\varphi$, we have $\varphi(A)=\lambda$.

\begin{probleme}
Faudrait creuser pourquoi on a un isomorphisme $\cA/M\simeq\eC$.
\end{probleme}
\end{proof}

\subsection{An example}
%----------------------

Let $\cA=L^1(\eR)$ with the norm
\[
  \|f\|_1=\int_{\eR}|f(x)|dx,
\]
and the convolution product
\[
   (f\star g)(x)=\int_{\eR}f(x-y)g(y)dy.
\]
We don't take care to analysis subtleties as completion and precise convergence of integrals. For example, we will use and abuse of Fubini's theorem and often say ``for all'' when ``for almost all'' should be preferable. From the fact that $|f(x-y)g(y)|\leq|f(x-y)||g(y)|$ and $\int_{\eR}f(x-y)dx=\int_{\eR}f(x)dx$, we find that 
\[
   \|f\star g\|_1\leq \|f\|_1\|g\|_1
\]
as needed to prove that $(L^1(\eR),\star)$ is a Banach algebra. This is a non unital Banach space because the unit should be the Dirac delta. From analysis, one knows that the dual space of $L^1(\eR)$ is $L^{\infty}(\eR)$ with, for $u\in L^{\infty}(\eR)$,
\[
  u(f)=\int_{\eR}f(x)u(x)dx.
\]
Since $\Delta(\cA)$ is a subset of $L^{\infty}(\eR)$, there exists, for each $\omega\in\Delta(L^1(\eR))$, a $\hat{\omega}$ such that $\omega(f)=\int_{\eR}f(x)\hat{\omega}(x)dx$. With an easy change of variable, the multiplicative condition $\omega(f\star g)=\omega(f)\omega(g)$ gives
\[ 
\int_{\eR^2}f(t)g(y)\hat{\omega}(t+y)dtdy=\int_{\eR^2}f(x)g(y)\hat{\omega}(x)\hat{\omega}(y)dxdy.
\]
We can conclude that $\hat{\omega}(x+y)=\hat{\omega}(x)\hat{\omega}(y)$, in such a manner that
\[ 
\hat{\omega}(x)=e^{ipx}
\]
for a certain $p\in\eC$. For $\hat{\omega}$ to belongs to $L^{\infty}(\eR)$, we must have $p\in\eR$. So we get a bijection $\Delta(\cA)\simeq\eR$. By this identification, we denote by $p$ the element of $\Delta(L^1(\eR))$ given by $\hat{\omega}(x)=e^{ipx}$. With theses notations, the Gelfand $\hat A(\omega)=\omega(A)$ transform reads
\begin{equation}
  \hat f(p)=\omega(f)=\int_{\eR}f(x)\hat{\omega}(x)
                     =\int_{\eR}f(x)e^{ipx}dx.
\end{equation}
This is nothing else than the Fourier transform ! We know that Fourier transform changes the convolution product into the pointwise usual product of functions:
\[ 
\widehat{f\star g}(p)=\hat f(p)\hat g(p)=(\hat f\hat g)(p).
\]
This express the fact that the Gelfand transform is a homomorphism between $\cA$ ---i.e. the product $\star$--- and $C(\Delta(\cA))$ ---i.e. the pointwise product. It is precisely the claim \ref{enugi} of theorem  \ref{tho:unital_comm}. 


\begin{theorem}
Let $\cA$ be a non unital commutative  Banach algebra. Then

\begin{enumerate}
\item \label{enuhi} The space $\Delta(\cA)$ is Hausdorff locally compact for the Gelfand topology,
\item \label{enuhii} $\Delta(\cA_{\cun})$ is the one point compactification of $\Delta(\cA)$,
\item \label{enuhiii} the Gelfand transformation is a homomorphism $\cA\to C_0(\Delta(\cA))$,
\item \label{enuhiv} the spectrum of $A\in\cA$ is 
\[ 
  \Spec(A)=\sigma(A)=\{0\}\cup\{\hat A(\omega)\tq\omega\in\Delta(\cA)\}.
\]
\item \label{enuhv} The image of $\cA$ by the Gelfand transform separates points in $\Delta(\cA)$,
\item \label{enuhvi} Gelfand transform is a contraction:
\[ 
\|\hat A\|_{\infty}\leq\|A\|.
\]

\end{enumerate}

\end{theorem}

For one point compactification issues, see section \ref{sec:compactific}.

\begin{proof}
\ref{enuhi} We add an unity to $\cA$ and we remark that
\[ 
\Delta(\cA_{\cun})=\Delta(\cA)\cup\infty
\]
where $\infty$ is defined by $\infty(A+\lambda\cun)=\lambda$. Indeed let $\psi\in\Delta(\cA)$ and let us ask ourself how to extend it to a multiplicative functional in $\varphi\in\Delta(\cA_{\cun})$. For, let $B\in\cA$ such that $\psi(A)\neq 0$ remark that multiplicative condition imposes $\varphi( (\lambda\cun)(B) )=\varphi(\lambda)\varphi(B)$ while the linearity gives $\varphi(\lambda B)=\lambda\varphi(B)$. Thus $\varphi(\lambda\cun)=\lambda$ and the unique possibility to extends $\psi$ is
\[ 
\varphi(A+\lambda\cun)=\varphi(A)+\lambda
\]
and we note $\infty$ the new functional
\[ 
\infty(A+\lambda\cun)=\lambda.
\]

Since $\cA_{\cun}$ is unital, the character space $\Delta(\cA_{\cun})$  is Hausdorff and compact for its Gelfand topology. As set
\[ 
  \Delta(\cA)=\Delta(\cA_{\cun})\setminus\{\infty\}.
\]
We should prove that the induced topology on $\Delta(\cA)$ from the Gelfand of $\Delta(\cA_{\cun})$ is precisely the own Gelfand topology of $\Delta(\cA)$. In this case, properties of compactification shall gives local compactness.

A basis of the topology of $\Delta(\cA_{\cun})$ is given by $\hat A^{-1}=\{\omega\in\Delta(\cA_{\cun})\tq\omega(A)\in\mO\}$. Then any open set of $\Delta(\cA)$ is open for the induced topology because
\[ 
\{\omega\in\Delta(\cA)\tq\omega(A)\in\mO\}=\{\eta\in\Delta(\cA_{\cun})\tq\eta(A)\in\mO\}\cap\Delta(\cA).
\]
For the converse, an open set for the induced topology is given by
\[
\begin{split}
&\{\omega\in\Delta(\cA_{\cun})\tq\exists A\in\cA,\lambda\in\eC:\omega(A+\lambda\cun)\in\mO\}\setminus\{\infty\}\\
&=\{\omega\in\Delta(\cA)\tq\exists A\in\cA,\lambda\in\eC:\omega(A)\in\mO-\lambda\}                             
\end{split}
\]
where $\mO-\lambda$ is as open as $\mO$. This proves \ref{enuhi} and \ref{enuhii}.

For \ref{enuhiii}, the point is not to prove that Gelfand transform is a homomorphism (that is trivial), but rather that it takes values in $C_0(\Delta(\cA))$.

The complementary of a compact set $K$ in $\Delta(\cA_{\cun})$ is an open set which contains $\infty$. Since $\hat A(\infty)=0$, the values of $\hat A$ in the complementary of $K$ are as small as we want when $K$ becomes larger and larger.

In order to prove \ref{enuhiv}, recall that, by definition, $\sigma_{\cA}(A)=\sigma_{\cA_{\cun}}(A)$. Then
\begin{equation}
  \sigma_{\cA_{\cun}}=\{\hat A(\omega)\tq\omega\in\Delta(\cA_{\cun})\}
                     =\{\hat A(\omega)\tq \omega\in\Delta(\cA)\}\cup\hat A(\infty)
                     =\{\hat A(\omega)\tq \omega\in\Delta(\cA)\}\cup\{0\}.
\end{equation}
Since \ref{enuhv} and \ref{enuhvi} are true for $\cA_{\cun}$, they are true for $\cA$.
 
\end{proof}

\section{Commutative \texorpdfstring{$C^*$}{C}-algebras}
%+++++++++++++++++++++++++++++++++++

A \defe{$C^*$-algebra}{c-star@$C^*$-algebra} is an involutive (complex) Banach algebra such that for all $A$, $B\in\cA$,
\begin{enumerate}
\item $\|AB\|\leq\|A\|\|B\|$,
\item $\|A^*A\|=\|A\|^2$.
\end{enumerate}
One immediately has $\|A\|^2=\|A^*A\|\leq\|A^*\|\|A\|$, then
\begin{equation}
\|A\|=\|A^*\|
\end{equation}
for any element $A$ in a $C^*$-algebra.

\begin{lemma}       \label{LemFiniCSestVNa}
    Every finite dimensional $C^*$-algebra is a von~Neumann algebra.
\end{lemma}

\begin{lemma}[Stone-Weierstrass theorem]
Let $X$ be a compact and Hausdorff space. Any $C^*$-subalgebra of $C(X)$ containing $1_X$ and separating points in $X$ is exactly $C(X)$ seen as $C^*$-algebra.
\end{lemma}\label{lem:Stone_W}
Here, $1_X$ denotes the constant function $1$ on $X$.

\begin{proposition}     \label{PropcomCstarDelCeqX}
Let $X$ be a compact Hausdorff space and see $C(X)$ as a commutative $C^*$-algebra. Then $\Delta(C(X))$ is homeomorphic to $X$.\label{prop:comHauffhomeo}
\end{proposition}
\lref{2.4.3}

\begin{proof}
For $x\in X$, one defines 
        \begin{equation}
        \begin{aligned}
            \omega_x \colon C(X) &\to \eC\\
            f&\mapsto f(x).
        \end{aligned}
    \end{equation}  
It is clearly non zero and multiplicative. Then $\omega_x\in\Delta(C(X))$. We denote by $ \dpt{E}{X}{\Delta(C(X))}$ the map which makes the correspondence between $x$ and $\omega_x$
\[ 
  E(x)f=f(x).
\]
Urysohn lemma \ref{lem:Urysohn} applied to the compact Hausdorff space $X=\Delta(C(X))$ makes that if $x\neq y$, then there exists a function $f\in C(X)$ such that $f(x)\neq f(y)$. This proves that $E$ is injective.

From theorem \ref{tho:ideal_kernel}, we know that 
\[ 
\cI_x=\ker\omega_x=\{f\in C(X)\tq f(x)=0\}
\]
is an ideal in $C(X)$. Suppose that $E$ is not surjective. Then there exists some $\omega\in\Delta(C(X)))$ which don't come from a $x\in X$; for such a $\omega$, we pose 
\[ 
\cI_{\omega}=\ker\omega=\{f\in C(X)\tq \omega(f)=0\}.
\]
 This $\cI_{\omega}$ can't contains any $\cI_x$ because they are maximal ideals. Then for all $x\in X$ , there exists a $f\in C(X)$ such that $f(x)=0$ with $f\notin\cI_{\omega}$. If $E$ is not surjective, then there exists a maximum ideal $\cI$, kernel of a character which is not in the image of $E$. In order this ideal to be included in none of the $\cI_x$, one needs that for all $x\in X$, there exists $f_x\in\cI$ such that $f_x(x)\neq 0$. Let $\mO_x$ be an open set on which $f_x\neq 0$. Since $X$ is compact, one can extract a finite subcovering $X=\cup_i\mO_{x_i}$. Now we build
\[ 
 g:=\sum_{i=1}^n|f_{x_i}|^2.
\]
This is a strictly positive function, then $1/g\in C(X)$, and then $\cI=C(X)$ and $\cI$ should be the kernel of a zero character. This is impossible, then $E$ is surjective and it is a bijection.

In order to prove that $E$ is an homeomorphism, we will use the lemmas \ref{lem:Hausweak} and \ref{lem:wiki}. Let $X_0$ be the space $X$ endowed with its initial topology and $X_G$ the same space with the topology induced from $E^{-1}$, i.e. that an open set in $X_G$ is always the image by $E^{-1}$ of an open set in $\Delta(C(X))$. From definition, $E$ is continuous for the topology $X_G$. We are going to prove that $X_0=X_G$. Definitions give for all $f\in C(X)$,
\[ 
 (\hat f\circ E)(x)=\hat f(\omega_x)=\omega_x(f)=f(x).
\]
But Gelfand topology is the weakest topology for which all $f$ are continuous. On the other hand, $f$ is continuous because it belongs to $C(X_0)$. Then the topology of $X_G$ is weaker than the one of $X_0$. Indeed, let $\mO$ be an open set for $X_G$ and let us prove that it contains an open set of $X_0$. From definition, $\mO=E^{-1}(\mO')$ for a certain open set $\mO'$ of $\Delta(C(X))$, i.e. $\mO'=\hat f^{-1}(A)$ for an open $A$ in $\eC$. The topology $X_G$ is the minimal one for which $E^{-1}\circ\hat f^{-1}(A)$ is open. But $E^{-1}\circ\hat j^{-1}=f^{-1}$, then $(E^{-1}\circ\hat f^{-1})(A)=f^{-1}(A)$ is open in $X_0$. 
\end{proof}


\begin{theorem}[Gelfand theorem]    \index{Gelfand!theorem}\index{theorem!Gelfand}

For any commutative unital $C^*$-algebra $\cA$, there exists an unique (up to isomorphism) compact and Hausdorff space $X$ such that  $\cA$ is isomorphic to $C(X)$.
\label{thoGelfand}
\end{theorem}

\begin{proof}
We immediately give the answer: $X=\Delta(\cA)$ and the isomorphism is
        \begin{equation}
        \begin{aligned}
            \varphi \colon \cA &\to C(\Delta(\cA))\\
            A&\mapsto \hat A.
        \end{aligned}
    \end{equation}  
We first have to prove that $\Delta(\cA)$ is compact and Hausdorff. Then it should be proved that $\varphi$ is an isometric $C^*$-algebra isomorphism and finally that this is the only possibility.

\subdem{The space $\Delta(\cA)$ is compact and Hausdorff}

Proposition \ref{prop:DcA_comp_Hauss} gives it.

\subdem{The map $\varphi$ takes values in $C(\Delta(\cA))$}

From discussion at top of subsection \ref{subsec:topo_Delta}, the functional $\hat A$ is continuous on $X=\Delta(\cA)$.

\subdem{The map $\varphi$ is a morphism}

Linearity of $\varphi$ is clear. Property $\varphi(AB)=\varphi(A)\varphi(B)$ comes from point \ref{enugi} of proposition \ref{prop:DcA_comp_Hauss}. So we are left to prove that $\varphi(A^*)=\varphi(A)^*$. It is sufficient to prove that, if $A=A^*$, then $\varphi(A)$ takes his values in $\eR$. So let $A\in\cA_{\eR}$ and write $\omega(A)=\alpha+i\beta$ with $\alpha,\beta\in\eR$. If we define $B=A-\alpha\cun$, then $\omega(B)=i\beta$ because $\omega(\cun)=1$. Furthermore $B=B^*$. Let $t\in\eR$; we have
\begin{equation}  \label{eq:rcinq}
|\omega(B+it\cun)|^2=|\omega(B)+it|^2
                    =\beta^2+2\beta t+t^2.
\end{equation}
Using formulas $|\omega(A)|\leq\|A\|$ and $\|AA^*\|=\|A\|^2$, we find
\begin{equation}
  |\omega(B+it\cun)|^2\leq\|B+it\cun\|^2
                      =\|B^2+t^2\|
                      \leq \|B\|^2+t^2.
\end{equation}
Then net result is that for all $t\in\eR$,   $\beta^2+2t\beta\leq \|B\|^2$. It is only possible when $\beta=0$. Then $\omega(A)\in\eR$ as soon as $A=A^*$.

\subdem{The map $\varphi$ is isometric}

Let us begin with $A=A^*$. So $\|A^2\|=\|A\|^2$ and $\|A^{2^m}\|=\|A\|^{2^m}$. Using proposition \ref{prop:An_usn}, we find
\[ 
  r(A)=\|A\|.
\]
On the other hand the definition of the supremum norm on the Hausdorff space $\Delta(\cA)$ reads
\begin{equation} \label{eq:AinfA }
  \|\hat A\|_{\infty}=\sup_{\omega\in\Delta(\cA)}|\hat A(\omega)|=r(A)=\|A\|.
\end{equation}
Then $\varphi$ is isometric when $A=A^*$. Now, $A^*A$ is selfadjoint and $\|A^*A\|=\|A\|^2$, then
\[ 
\|A\|^2=\|A^*A\|=\|\widehat{A^*A}\|_{\infty}=\|{\hat A}^*\hat A\|_{\infty}=\|\hat A\|^2_{\infty}.
\]

\subdem{The map $\varphi$ is injective}

If $\varphi(A)=\varphi(B)$, then $\varphi(A-B)=0$. The only way for $\varphi$ to be an isometry is $A-B=0$.

\subdem{The map $\varphi$ is surjective}

Since $\varphi$ is an isometry, it sends a closed set into a closed set, but $\cA$ is closed because it is a Banach space. Point \ref{enugii} of theorem \ref{tho:unital_comm} says that $\varphi(\cA)$ separates points in $\Delta(\cA)$ and we just proved the $\varphi$ preserves the adjoint, so $\varphi(\cA)$ is a $C^*$-subalgebra of $C(\Delta(\cA))$. Finally, it is clear that $\hat{\cun}=1_X$. Lemma \ref{lem:Stone_W} concludes $\varphi(\cA)=C(\Delta(\cA))$.

Now proposition \ref{prop:comHauffhomeo} makes $\varphi$ and homeomorphism between $\cA$ and $\Delta(\cA)$. So the topological structure of $\cA$ is encoded in the algebraic (Banach) structure of $C(\Delta(\cA))$. So if $C(Y)\simeq\cA\simeq C(X)$ as $C^*$-algebras, then $X\simeq Y$ as topological space. This proves the unicity part and concludes the Gelfand theorem.

\end{proof}

As far as notations are concerned, let us recall that the Gelfand transform is $A\mapsto\hat A$ with
\begin{equation}
\begin{aligned}
 \hat A\colon \Delta(\cA)&\to \eC \\ 
   \omega&\mapsto \omega(A). 
\end{aligned}
\end{equation}
One particular class of elements in $\Delta\big( C(X) \big)$ is the ones of the form $\omega_x$ for $x\in X$. These are defined by
\begin{equation}
\begin{aligned}
 \omega_x\colon C(X)&\to \eC \\ 
   f&\mapsto f(x). 
\end{aligned}
\end{equation}
The Gelfand theorem says that every element of $\Delta \big(C(X))$ reads $\omega_x$ for a certain $x\in X$.

\begin{lemma}
Let $\cA$ be a $C^*$-algebra, and $\oB(\cA)$, the set of bounded operators on $\cA$. Then

\begin{enumerate}
\item The map 
        \begin{equation}
        \begin{aligned}
            \rho \colon \cA &\to \oB(\cA)\\
            \rho(A)B&\mapsto AB
        \end{aligned}
    \end{equation}  
is a diffeomorphism between $\rho(\cA)$ and $\rho(\cA)\subset\oB(\cA)$.

\item If $\cA$ has no unit, one can define a norm on $\cA_{\cun}$ by
\begin{equation} \label{eq:normCAu}
\|A+\lambda\cun\|=\|\rho(A)+\lambda\cun\|
\end{equation}
where the right hand side norm is the one in $\oB(\cA)$, see \ref{def:normappl}. With the usual multiplication and the involution
\begin{equation}
  (A+\lambda\cun)^*=A^*+ \overline{\lambda} \cun,
\end{equation}
the set $\cA_{\cun}$ becomes an unital $C^*$-algebra.

\end{enumerate}
 \label{lem:unitariz_C}
\end{lemma}

\begin{proof}
Since $\cA$ is a $C^*$-algebra, $\|\rho(A)B\|\leq\| A \|\|B\|$, then for all $A\in \cA$, one has $\| \rho(A) \|\leq \| A \|$. On the other hand, we know that $\| A^*A \|=\| A \|^2$ and $\| A^* \|=\| A \|$, then
\[ 
\| A \|=\frac{\| AA^* \|}{\| A \|}=\left\|  \rho(A)\frac{A^*}{\| A \|}  \right\|\leq\| \rho(A) \|
\]
from definition of the sup norm. Then $\| \rho(A) \|=\| A \|$ and $\rho$ is an isometry and then is injective because it is linear. It is clearly a homomorphism too. The map $A+\lambda\cun\to\rho(A)+\lambda\cun$ is a $C^*$-algebra-morphism if we define\footnote{We know a definition of $*$ when we look at $\oB(H)$ where $H$ is a Hilbert space, but we are here with $\oB(\cA)$ where $\cA$ is no more than a Banach space; hence we do not have a definition of $*$.} $\rho(A)^*=\rho(A^*)$.  Since the sup norm fulfils condition \eqref{eq:normBanach}, the norm \eqref{eq:normCAu} fulfils the same. So $\cA_{\cun}$ becomes a Banach $*$-algebra and lemma \ref{lem:STARAlC} will help us to conclude that it is a $C^*$-algebra.

The formula $\| A \|^2-\varepsilon\leq\| Av \|^2$ holds for an operator $A$ on a general Banach algebra and an arbitrary vector $v$ with norm $1$. In our present case, if $\| B \|=1$,
\begin{equation}
\begin{split}
  \| \rho(A)+\lambda\cun \|^2-\varepsilon 
        &\leq \| (\rho(A)+\lambda\cun)B \|^2\\
        &    =\| AB+\lambda B \|^2\\
        &    =\| (AB+\lambda B)^*(AB+\lambda B) \|\\
        &    =\| \rho(B^*)\rho(A^*+\overline{\lambda}\cun)\rho(A+\lambda\cun)B \|\\
        &\leq \| \rho(B^*) \|\| (\rho(A)+\lambda\cun)^*(\rho(A)+\lambda\cun) \|\| B \|,
\end{split}
\end{equation}
but we also know that $\| \rho(B^*) \|=\| B^* \|=\| B \|=1$. Letting $\varepsilon\to 0$, we find $\| A \|^2\leq\| A^*A \|$ in the Banach $*$-algebra $\cA_{\cun}$.

\end{proof}

\label{pg:unit_nonunic} This lemma gives us an unitization of a $C^*$-algebra which is not the one previously given for a Banach algebra. This shows that unitization of Banach algebra is not unique. For a $C^*$-algebra, however, we have an unicity result:

\begin{proposition}
For every $C^*$-algebra without unit, there exists an unique unital $C^*$-algebra $\cA_{\cun}$ and an isometric morphism (hence injective) $\cA\to\cA_{\cun}$ such that $\cA/\cA_{\cun}=\eC$.
\label{prop_unitariz_csa}
\end{proposition}
\lref{1.4.6}

\begin{proposition}
If $\cA$ is a commutative $C^*$-algebra, any character is hermitian.
\end{proposition}

\begin{proof}
When $\chi$ is a character, $\chi(A)\in\sigma(A)$ for all $A\in\cA$ and when $A=A^*$, we have $\sigma(A)\subset\eR$. For any $A$, we have a decomposition $A=A_1+iA_2$ and
\[ 
  \chi(A^*)=\chi(A_1-iA_2)=\underbrace{\chi(A_1)}_{\in\eR}-i\underbrace{\chi(A_2)}_{\in\eR}=\overline{ \chi(A) }.
\]
\end{proof}

\section{Functional calculus in unital \texorpdfstring{$C^*$}{C}-algebras}
%+++++++++++++++++++++++++++++++++++++++++++++++++++++++++++++++++++++++++

From now, the $C^*$-algebra $\cA$ is no more assumed to be commutative, but it is unital. 

\begin{definition}      \label{DefElemNormal}
    An element $A$ in an involutive algebra is said \defe{normal}{normal!element of an involutive algebra} when $[A,A^*]=0$.
\end{definition}
This is a direct generalisation of the concept of normal operator in the Hilbert space setting (definition \ref{DefFQFKZbB}).

If $\cA$ is a $C^{*}$-algebra and $A$, $B\in\cA$ we denotes by $C^*(A_1,\ldots,A_n)$ the $C^{*}$-algebra generated by the $A_i$. This is the closure of every finite products of the form $Z_1\cdots Z_k$ where each $Z_j$ is one of the $A_i$.

For any $A$ in a $C^{*}$-algebra we know that $\|A^*A\|=\|A\|^2$.
%TODO : a proof of equation \eqref{eq:ray_norme}

If $A$ is normal, then $C^*(A,\cun)$ is commutative. Indeed any element of the form $A_A\ldots A_n$ with $A_i=A$ or $A^*$ can be written under the form $A\ldots AA^*\ldots A^*$.

\begin{proposition}
\begin{equation}\label{eq:ray_norme}
\|A\|=\sqrt{ r(A^*A) }
\end{equation}
\end{proposition}

\begin{proof}
Let $A$ be in $\cA$ and consider a $z\in\rho(A)$. By definition, $(A-z)^{-1}$ exists in $\cA$; since $\varphi$ is a morphism, $\varphi(A-z)$ is also invertible: it is clear that $\varphi( (A-z)^{-1} )$ is a two-sided inverse of $\varphi(A-z)$. Hence $\rho(A)\subseteq\rho(\varphi(A))$ and thus $\sigma(\varphi(A))\subseteq\sigma(A)$. Definition (\ref{def:spectre}) of the spectral radius makes $r(\varphi(A))\leq r(A)$ and equation \eqref{eq:ray_norme} gives the thesis.
\end{proof}


\begin{theorem}
 Consider an unital $C^{*}$-algebra $\cA$ and a $A\in\cA$ such that $A^*=A$. Then
\begin{enumerate}
\item The spectrum $\sigma_{\cA}(A)$ is the same as $\sigma_{C^*(A,\mtu)}(A)$, so that one can speak about $\sigma(A)$ without ambiguities.

\item $\sigma(A)\subset\eR$.

\item \label{enukiii} $\Delta(C^*(A,\mtu))$ is homeomorphic to $\sigma(A)$ and $C^*(A,\mtu)$ is isomorphic to $C(\sigma(A))$. Under this isomorphism, the Gelfand transformed $\dpt{\hat A}{\sigma(A)}{\eR}$ is the identity $\dpt{id_{\sigma(A)}}{t}{t}$.
\end{enumerate} \label{tho:l_2.5.1}
\end{theorem}\lref{2.5.1} 
\begin{proof}
We first consider a normal $B\in G(\cA)$, and the $C^{*}$-algebra $C^*(B,B^{-1},\mtu)$ generated by $B$, $B^{-1}$ and $\mtu$. Since $(B^{-1})^*=(B^*)^{-1}$ and $BB^*=N^*B$, $[B^{-1},{B^*}^{-1}]=0$.

Now, we are going to show that $[{B^*}^{-1},B]=0$. First remark that ${B^*}^{-1} B=(B^{-1} B^*)^{-1}$. We have to show that $B^{-1} B^*B{B^*}^{-1}=\mtu$ and
$B{B^*}^{-1} B^{-1} B^*=\mtu$. These two equalities comes from $[B,B^*]=0$ and $[B^{-1},{B^*}^{-1}]=0$. The same makes that $[B^*,B^{-1}]=0$.

The result is that $C^*(B,B^{-1},\mtu)$ is a commutative $C^{*}$-algebra So one can simply say that it is the closure of the polynomials in
$B$, $B^*$, $B^{-1}$, and ${B^*}^{-1}$.

By the Gelfand theorem, $C^*(B,B^{-1},\mtu)$ is then isomorphic to a $C(X)$ for some compact Hausdorff space $X$. Since $B$ is invertible and the Gelfand transform is an isomorphism, $\hB$ is invertible. Then $\forall\,x\in X$, $\hB(x)\neq 0$. Indeed, the $X$ is (up to an isomorphism) $\Delta(C^*(B,B^{-1},\mtu))$. If for an $\omega\in\Delta(C^*(B,B^{-1},\mtu))$, $\hB(\omega)$, then $\omega(B)=0$ and thus $\omega\equiv 0$. But in the definition of $\Delta(\cA)$, we have explicitly excluded the null form.

On the other hand let us consider $f\in C(X)$ everywhere non zero. Since (pointwise) $0<\|f\|^{-2}_{\infty}ff^*\leq 1$,
\begin{equation}\label{eq:ff}
   0\leq 1_X-\|f\|^{-2}_{\infty}ff^*<1.
\end{equation}
%
But \lref{2.2.4} if $\|A\|<1$, then
\[
   \sum_{k=0}^{n}A^k\to (\mtu-A)^{-1}.
\]
As far as $f$ is concerned for the sup norm, equation \eqref{eq:ff} makes $1_X-ff^*/\|f\|^2_{\infty}$ satisfy this convergence. Then
\[
   \left(
      \frac{ff^*}{\|f\|^2_{\infty}}
   \right)^{-1}
      =
   \sum_{k=0}^{\infty}
   \left(
       \mtu-\frac{ff^*}{\|f\|^2_{\infty}}
   \right)^k,
\]
so that
\begin{equation}
   \us{f}
      =
   \frac{f^*}{\|f\|^2_{\infty}}
   \sum_{k=0}^{\infty}
   \left(
       \mtu-\frac{ff^*}{\|f\|^2_{\infty}}
   \right)^k.
\end{equation}
This is true for any $f$ such that $f(x)\neq 0$ $\forall x\in X$; in particular, it is true for $\hB$. Thus $\hB^{-1}$ is a limit of polynomials in $\hB$ and $\hB^*$. By the inverse Gelfand transform (which is obviously an isomorphism), $B^{-1}$ is a limit of polynomials in $B$ and $B^*$. This is:
\begin{equation}
   C^*(B,B^{-1},\mtu)=C^*(B,\mtu).
\end{equation}

Now, we take our $A$ from the hypothesis: $A=A^*$. Clearly, $A$ is normal and $A-z$ too. If we take $z\in\rho(A)$, our work about $B$ applies to $A-z$. Then $(A-z)^{-1}$ can be written as polynomials in $(A-z)$ and $\mtu$. Thus $(A-z)^{-1}\in C^*(A-z,\mtu)$ and
$z\in\rho_{C^*(A-z,\mtu)}(A)$, but it is clear that $C^*(A-z,\mtu)=C^*(A,\mtu)$. Finally:
\[
   \rho_{\cA}(A)=\rho_{C^*(A,\mtu)}(A).
\]
The set $\sigma$ being nothing else than the complement of $\rho$, the first point of the theorem is finish.

In the course of the demonstration of the Gelfand theorem, we had shown that since $A=A^*$, $\forall\omega\in\Delta(\cA)$, $\hat A(\omega)\in\eR$. But \lref{2.3.5.3} makes
\[
   \sigma(A)=\{\hat A(\omega):\omega\in\Delta(\cA)\}.
\]
Then $\sigma(A)\subset\eR$.

The proof that $\hat A$ is a bijection and that it is continuous is well done in \lref{2.5.1}. Here we will just prove the continuity of $\hat A^{-1}$. From theorem \ref{tho:unital_comm} and what we just did, we know that
\[
   \sigma(A)=\{\hat A(\omega):\omega\in\Delta(\cA)\}\subset\eR,
\]
but $\hat A^{-1}(z)=\omega$ when $\hat A(\omega)=z$, or $\omega(A)=z$. Then $\hat A^{-1}$ is defined on $\sigma(A)$. So from now, one can only consider $z\in\sigma(A)$ and $\hat A^{-1}(z)(A)=z$. By induction, $\hat A^{-1}(z)(A^n)=z^n$. An element in $\Delta(C^*(A,\cun))$ is completely determined by its value on $A$. Then an open set therein has the general form
\[ 
  \mR=\{ \omega\in\Delta(C^*(A,\cun))\tq\hat A(\omega)\in\mO \}
\]
where $\mO$ is any open set in $\eC$. From definition, $\hat A(\mR)=\mO$. So $\hat A^{-1}$ is continuous.

\end{proof}


\begin{probleme}
    There is a notational clash: what is written $\sigma(A)$ is the spectrum of $A$. I want to write it $\Spec(A)$ instead.
\end{probleme}

The following proposition is the \defe{continuous functional calculus}{continuous!functional calculus!selfadjoint in $C^*$-algebra}. 
\begin{theorem}[Continuous functional calculus]     \label{ThoContFuncCalculus}
Let $A\in\cA$ be self-adjoint and $f\in C(\Spec(A))$. One can define a map $\tilde f\colon \cA\to \cA$ in such a way that when $f$ is a polynomial, $\tilde f=f$ and in other cases, it is the uniform approximation of $f$ by polynomials. This map $\tilde f$ which will be denoted by~$f$ fulfills
\begin{enumerate}
\item $\Spec(f(A))=f(\Spec(A))$,  \label{enuji}
\item $\|f(A)\|=\|f\|_{\infty}$.
\end{enumerate}\label{prop:cont_calc}
\end{theorem}

\begin{probleme}
    The proof has to be reordered.
\end{probleme}

\begin{proof}
We know from theorem \ref{tho:l_2.5.1} that $\Delta(C^*(A,\cun))$ is homeomorphic to $\sigma(A)$ and $C^*(A,\cun)$ to $C(\sigma(A))$.

The Gelfand theorem says that if one has a commutative unital $C^{*}$-algebra then  one has an unique (up to homeomorphism) $X$ such that $\cA$ is isomorphic to $C(X)$. Moreover, this isomorphism is an isometry\quext{Is is correct ?}. But we just showed that $C^*(A,\mtu)$ where isomorphic to $C(\sigma(A))$, then one has
\begin{equation}\label{eq:norm_vp_B}
  \|\varphi(B)\|=\|B\|,
\end{equation}
the first norm is taken in $C(\sigma(A))$ and the second one in $C^*(A,\mtu)$. But when $X$ is Hausdorff, we had adopted the $\|.\|_{\infty}$ norm, so that $\|\varphi(B)\|=\|f\|_{\infty}$ and equation \eqref{eq:norm_vp_B} reads:
\begin{equation}
\|f\|_{\infty}=\|f(A)\|.
\end{equation}

Now remark that $f(\sigma(A))$ is the set of values that $f$ takes on $\sigma(A)$, but we know\quext{Vas voir si on know \c{c}a vraiment} that
\[
\sigma(A)=\sigma(\hat A)=\{ \hat A(\omega):\omega\in\Delta(C^*(A,\mtu)) \}.
\]

It is now times to give a sense to $f(\hat{A})$.  Since $f$ is continuous on $\sigma(A)$, there exists a converging infinite sum such that $f(t)=\sum_{k=0}^{\infty}c_kt^k$ for any $t\in\sigma(A)$. In particular, $\forall\omega\in\Delta(C^*(A,\mtu))$, $\hat A(\omega)\in\sigma(A)$; thus 
   $\sum c_k[\hat A(\omega)]^k$ converges everywhere we want. This sum will be denoted by $f(\hat{A})(\omega)$: 
\begin{equation}
f(\hat A)(\omega)=\sum_{k=0}^{\infty}c_k[\hat A(\omega)]^k.
\end{equation}
In other words, $f(\hat A)(\omega)=f(\hat A(\omega))$. We have
\begin{equation}
   f(\sigma(A))=\{f(\hat A(\omega)):\omega\in\Delta( C^*(A,\mtu) )\}
               =\{ f(\hat A)(\omega):\omega\in\Delta( C^*(A,\mtu) ) \}
           =\sigma(f(\hat A)).
\end{equation}


It remains to be proved that $\sigma(f(\hat{A}))=\sigma(f(A))$. We already know that $\sigma(A)=\sigma(\hat{A})$, so we just have to prove that $f(\hat{A})=\widehat{ f(A) }$. On the one hand,
\[ 
 f(\hat{A})\omega=\sum c_k[\hat{A}(\omega)]^k
        =\sum c_k [\omega(A)]^k
        =f\big( \omega(A) \big),
\]
on the other hand,
\[ 
  \widehat{ f(A) }\omega=\omega\big( f(A) \big)=\omega\big[ \sum c_k A^k \big].
\]
On the other hand, one already know that $\sigma(A)=\sigma(A)$, thus we just have to see that 
$f(\hat A)=\widehat{f(A)}$ when $f\colon \sigma(A)\to \eC$ is continuous. The problem is a permutation of $\omega$ and a limit:
\[
 f(\hat A)\omega=\sum_{k=0}^{\infty} c_k\,\omega(A^k),\quad\widehat{f(A)}\omega=\omega\left(\sum_{k=0}^{\infty} c_kA^k\right).
\]
What theorem  \ref{tho:l_2.5.1} says is that $C^*(A,\cun)$ is isomorphic to $C(\sigma(A))$ with 
\[ 
  \sum c_kA^K\mapsto f(x)=\sum c_kx^k.
\]
 In this isomorphism, the map $\hat{A}\colon \sigma(A)\to \eR$ corresponds to the identity map. More precisely, the isomorphism $\varphi\colon C^*(A,\cun)\to C(\sigma(A))$ is the following:
\[ 
  \varphi(B)(t)=a+\sum c_kt^k
\]
when $B=a+\sum c_kA^k$. We know in general that
\[
\sigma(A)=\sigma(\hat{A})=\{ f\big( \hat{A}(\omega) \big)\tq\omega\in\Delta(\cA) \}.
\]
 In the present case, we are working with $\cA=C^*(A,\cun)$, therefore
\[ 
  f\big( \sigma(A) \big)=\{ f\big( \hat{A}(\omega) \big)\tq \omega\in\Delta\big( C^*(A,\cun) \big) \}.
\]
We have to prove that $f\big( \hat{A}(\omega) \big)=f(\hat{A})\omega$. Since $f$ is continuous on $\sigma(A)$, the sum $f(t)=\sum c_kt^k$ converges for all $t\in\sigma(A)$. In particular for $\hat{A}(\omega)\in\sigma(A)$, the sum $\sum_{k=0}^{\infty}\big[ \hat{A}(\omega) \big]^k$ converges.

It is now time to give a sense to $f(\hat{A})$. We know from theorem \ref{tho:l_2.5.1} that $\hat{A}\colon \Delta\big( C^*(A,\cun) \big) \to\sigma(A) $ is an isomorphism. As definition we set
 \begin{equation}
  f(\hat{A})(\omega)=\sum c_k\big[ \hat{A}(\omega) \big]^k
\end{equation}
everywhere it converges. But, since $\hat{A}(\omega)\in\sigma(A)$, it converges everywhere it is interesting for us.

By definition, $\sum_{k=0}^{\infty}=\lim_{n\to\infty}\sum_{k=0}^{n}$, but the proposition \ref{prop:continu_cv}, which gives link between convergence and continuity, assures us that one can permute the sum and $\omega$ because it is a continuous function on $C^*(A,\mtu)$ which is by definition the closure of all polynomials in $A$:
\begin{equation}
\begin{split}  
  \omega(\sum_k B^k)&=\omega(\lim_{n\to\infty}\sum_k^n B^k)
                    =\lim_{n\to\infty}\omega(\sum_k^n B^k)\\
            &=\lim_{n\to\infty}\sum_k^n\omega(B^k)
            =\sum_{k=0}^{\infty}\omega(B^k).
\end{split}         
\end{equation}


We now turn our attention to the second point: $\| f(A) \|_{C^*(A,\cun)}=\| f(A) \|_{\cA}$. It uses proposition 2.26, chapter 4 of \cite{LaHarpe}.

\end{proof}

\subsection{The isomorphism \texorpdfstring{$C^*(A,\mtu)\leftrightarrow C(\sigma(A))$}{AAm AsA}}
%--------------------------------------------------------------------

By definition an element $B\in C^*(A,\mtu)$ can be written as $B=f(A)$ where $f$ is a sum of $\mtu$, $A$, $A^2$, $A^*$, $(A^*)^2$,\ldots In the setting of continuous functional calculus, we suppose that $A$ is selfadjoint, i.e. $A=A^*$, so that $f(A)$ is polynomial (eventually infinite) in $A$ with an independent term $\mtu$. The isomorphism that we consider is 
\begin{equation}
\begin{aligned}
 \varphi\,:\,C^*(A,\mtu)&\to C(\sigma(A))\\
    \varphi(B)&=f\in C(\sigma(A))
\end{aligned}
\end{equation}
where $f$ is the ``definition'' function of $B$ in $C^*(A,\mtu)$. 

\begin{remark}      \label{RemExpansionSqrtConCal}
    The map $\varphi$ depends on $A$. It could be better written $\varphi_A$. As an example, if $A=A^*$, the element $A^{1/2}$ is computed as follows. First, we know the \wikipedia{en}{Taylor_series}{expansion}
    \begin{equation}        \label{EqExpanSqrtt}
        \sqrt{t}=\sum_ka_kt^k.
    \end{equation}
    Then we define $\sqrt{A}=\sum_k a_kA^k$ as element of $C^*(A,\mtu)$.
\end{remark}

\begin{probleme}
    An expansion \eqref{EqExpanSqrtt} is only possible when $t$ is close to $1$. Maybe the definition of $\sqrt{A}$ has to first look at $B=\lambda A$ with $\lambda$ such that the norm of $B$ is close to $1$. Then we write $\sqrt{A}=\frac{1}{ \sqrt{\lambda} }\sqrt{B}$. The square root of $\lambda$ is well defined as a square root in $\eR^+$.
\end{probleme}


In order to show that it is actually an isomorphism, we have to show the following points:
 \begin{enumerate}
     \item
          it is linear;
      \item
         bijective;
     \item
         $\varphi(CD)=\varphi(C)\varphi(D)$;
     \item
         $\varphi(B^*)=\varphi(B)^*$.
 \end{enumerate}
 Here are the proofs.
 \begin{enumerate}
     \item       
        The linearity is clear.
    \item       
        Suppose $\varphi(B)=\varphi(C)$. Definition of $\varphi$ gives $B=\varphi(B)(A)$ and $C=\varphi(C)(A)$. For the surjectivity, note that $C(\sigma(A))$ is given by continuous functions whose can be uniformly approximated by polynomials; then for each $f\in C( \sigma(A))$, there corresponds a $B=\varphi(A)\in C^*(A,\mtu)$.
    \item
        Consider $C=f(A)$, $D=g(A)$; thus $CD=(fg)(A)$ and $\varphi(CD)=fg=\varphi(C)\varphi(D)$. 
    \item
        The last point comes from the fact that $A=A^*$. Indeed, consider $B=f(A)=\sum_k c_kA^k$. Then
        \[
            B^*=\sum_k c_k^*(A^*)^k=\sum_k c_k^*A^k=f^*(A).
        \]
 \end{enumerate}

We have shown that $\varphi(B)=f$ when $B=f(A)$ is an isomorphism between $C^*(A,\mtu)$ and $C(\sigma(A))$ if $A$ is selfadjoint: $A=A^*$.

\begin{corollary}
For each $C^*$-algebra, there exists an unique unital $C^*$-algebra $\cA_{\cun}$ and an isometric morphism $\cA\to\cA_{\cun}$ such that $\cA_{\cun}/\cA\simeq\eC$.
\end{corollary}\label{cor_csa_unit}

\begin{proof}
We yet defined $\cA_{\cun}$ in lemma \ref{lem:unitariz_C} and we just prove that the norm was unique. Since all elements in $\cA_{\cun}$ are given under the form $A+\lambda\cun$ with $A\in\cA$, it is obvious that $\cA_{\cun}/\cA\simeq\eC$. The canonical injection $\varphi(A)=A$ is a morphism.
\end{proof}


\begin{lemma}
If $\dpt{\varphi}{\cA}{\cB}$ is a morphism of $C^*$-algebra and if $A=A^*$, then 
\[ 
  f(\varphi(A))=\varphi(f(A)).
\]
for all $f\in C(\sigma(A))$.
\end{lemma} \label{lem:fvpvpf}


\begin{proof}
Since $\sigma(\varphi(A))\subseteq\sigma(A)$, the function $f$ is well defined on $\varphi(A)$. If $f$ is a polynomial, the result comes from the fact that $\varphi(AB)=\varphi(A)\varphi(B)$. If $f$ is a general continuous function, it can be approximated by polynomials. Taking partial sums, $s_n=\sum_{k=1}^n\varphi(c_kA^k)$ and $v_n=\varphi(\sum_{k=1}c_kA^k)$, the linearity of \emph{finite sums} gives the result.
\end{proof}

Where in the proof did we use the assumptions ? The definition of $f(A)$ when $\dpt{f}{\eR}{\eR}$ was given in \ref{prop:cont_calc} in order to get formulas $\sigma\circ f=f\circ\sigma$ and $\| f(A) \|=\| f \|$.

\section{Positivity}
%+++++++++++++++++++

Let $\cA$ be a $C^*$-algebra. We say that $A\in\cA$ is \defe{positive}{positive!element!of a $C^*$-algebra} when 
\begin{enumerate}
\item  $A=A^*$
\item  $\Spec(A)\subset\eR^+$.
\end{enumerate}
In this case, we write $A\geq 0$ or $A\in\cA^+$,
\[ 
  \cA^+=\{ A\in\cA_{\eR}\tq\sigma(A)\subset\eR^+ \}.
\]
A set of particular importance is the set of selfadjoint elements:\nomenclature[C]{\(\cA_{\eR}\)}{The set of selfadjoint elements in \(\cA\)}
\begin{equation}
    \cA_{\eR}=\{ A\in\cA\tq A=A^* \}.
\end{equation}
These elements have real spectrum. We will see in theorem \ref{ThoElsPositifsBBstar} that the set of positive elements in $\cA$ is given by
\begin{equation}
    \cA^+=\{ A^2\tq A\in\cA_{\eR} \}=\{ B^*B\tq B\in\cA \}.
\end{equation}

\begin{lemma}
For all $A$ such that $A=A^*$, we have a decomposition
\[ 
  A=A_++A_-
\]
where $A_+,A_-\in\cA^+$ and $A_+A_-=0$. Furthermore
\[ 
  \| A_{\pm} \|\leq \| A \|.
\]
 \label{lem:AsAdecm}
\end{lemma}

\begin{proof}
We apply the continuous functional calculus with $f=\in_{\sigma(A)}=f_++f_-$ where
\begin{equation} \label{eq:rrdeux}
\begin{aligned}
  \id_{\sigma(A)(t)}&=\max{0,t}&&\textrm{because $\sigma(A)\subset\eR^+$}\\
  f_+(t)&=\max\{ 0,t \}\\
  f_-(t)&=\max\{ -t,0 \}.
\end{aligned}
\end{equation}
Recall that when $A=A^*$, the spectral radius is given by $r(A)=\| A \|$. Then $\| f_{\pm} \|_{\infty}\leq r(A)=\| A \|$.

Let us prove that $f_+(A)\in\cA^+$. From the continuous calculus and the fact that $f_+(A)^*=f_+(A)$, we find that $\sigma(f_+(A))\subset\eR^+$. Since $A\in\cA_{\eR}$, we know that $\sigma(A)\subset\eR$ and thus that $f_+(\sigma(A))\subset\eR^+$. From equation part \ref{enuji} of the continuous functional calculus, theorem \ref{prop:cont_calc}, we conclude that $\sigma(f_+(A))\subset\eR^+$ and then that $f_+(t)f_-(t)=0$.
\end{proof}

\begin{lemma} \label{lem:rtrois}
If $-C^*C\in\cA^+$ for $C\in\cA$, then $C=0$.
\end{lemma}

\begin{proof}
We can decompose $C=D+iE$ with $D$, $E\in\cA_{\eR}$; then 
\begin{equation}  \label{eq:rquare}
C^*C=2D^2+2E^2-CC^*.
\end{equation}
 If $z\neq 0$ and $AB-z$ is invertible, then $BA-z^{-1}\cun$ is invertible and $(BA-z)^{-1}=B(AB-z)^{-1}A-z^{-1}\cun$. Then $\sigma(AB)\cup\{ 0 \}=\sigma(BA)\cup\{ 0 \}$ and $\sigma(C^*C)\subset\eR^-$ imply $\sigma(-CC^*)\subset\eR^+$. Now all terms of the right hand side of \eqref{eq:rquare} are in $\cA^+$ and $C^*C\in\cA^+$. Since the assumption is $-C^*C\in\cA^+$, we conclude that $\sigma(C^*C)=0$ and $C=0$. 
\end{proof}


\begin{theorem}     \label{ThoElsPositifsBBstar}
The set of positive elements in $\cA$ is given by
\begin{equation}
\cA^+=\{ A^2\tq A\in\cA_{\eR} \}=\{ B^*B\tq B\in\cA \}
\end{equation}
when $\cA$ is an unital $C^*$-algebra.
\end{theorem}

\begin{proof}
If $A\in\cA^+$, one can define $\sqrt{A}\in\cA_{\eR}$ in the same way as in proposition \ref{prop:cont_calc} with $f=\sqrt{\cdot}$. With this definition we have $(\sqrt{A})^2=A$, so that $\cA^+\subset\{ A^2\tq A\in\cA_{\eR} \}$.

Using the linearity of the involution term by term in the formula $\sqrt{A}=\sum_k c_kA^k$ shows that $\sqrt{A}\in\cA_{\eR}$ when $A=A^*$. 

For the inverse inclusion, consider $A\in\cA_{\eR}$. Since $A=A^*$, we have $\sigma(A)\subset\eR$. Using formula $\sigma(f(A))=f(\sigma(A))$ with $f(t)=t^2$, we find $\sigma(A^2)=\sigma(A)^2\subset\eR^+$. The first equality is proved.
 
For the second equality, we begin by applying lemma \ref{lem:AsAdecm} to $B^*B$, let $B^*B=A_+-A_-$. From equations \eqref{eq:rrdeux} we see that $A_+-A_-=-A_-$. Then $(A_-)^3=-A_-(A_+-A_-)A_-=-A_-B^*BA_-=-(BA_-)^*BA_-$. Since $AA_-$ is positive, $\sigma(A_-)\subset\eR^+$. Using the continuous calculus with $f(t)=t^3$, it proves that $(A_-)^3\geq 0$ and thus that $-(BA_-)^*BA_-\geq 0$. Lemma \ref{lem:rtrois} shows that $BA_-=0$.

This proves that $(A_-)^3=0$. From the continuous functional calculus with $f(t)=t^{1/3}$, it proves that $A_-=0$ and then $B^*B=A_+\in\cA^+$.
\end{proof}
\begin{corollary}
When $A_1,A_2\in\cA_{\eR}$ and $B\in\cA$, if $A_1\leq A_2$, then $B^*A_1B\leq B^*A_2B$ .
\end{corollary}

\begin{proof}
The assumption is $A_2-A_1\geq 0$, but from theorem \ref{ThoElsPositifsBBstar}, there exists $A_3\in\cA$ such that $A_2-A_1=A^*_3A_3$. The same property shows that $(A_3B)^*A_3B\geq 0$. This gives the corollary.
\end{proof}


\begin{corollary}
For all $A$, $B\in\cA$, we have
\[ 
  B^*A^*AB\leq \| A \|^2B^*B.
\]
 \label{cor:BeAAeB}
\end{corollary}

\begin{proof}
The inequality $-\| A \|\cun\leq A\leq \| A \|\cun$ holds when $A=A^*$. Let us write it for $A^*A$ and recall that $\| A^*A \|=\| A \|^2$ in all $C^*$-algebra. Then $A^*A\leq \| A \|^2\cun$ and by applying the previous corollary, we find $B^*A^*AB\leq\| A \|^2B^*B$ 
\end{proof}



An element $A\in\cA$ is a \defe{projection}{projection!in a $C^*$-algebra } if $A=A^*$ and $A^2=A$. In the case of a $C^*$-algebra of linear operators acting on a vector space, if $x$ is an eigenvector of the projection $A$ with the eigenvalue $\lambda$, then $Ax=\lambda x$ and $A^2x=\lambda^2x=\lambda x$. Thus $1$ is the only eigenvalue of a projection (or zero, which is the kernel). In particular a projection is positive and reads\label{PgProjPositif} $A=B^*B$ for some $B\in \cA$ by theorem \ref{ThoElsPositifsBBstar}.

\begin{probleme}
    I think that the notation \(\cA_{\eR}\) stand for the elements with real spectrum. I have to check it and add to the notation index.
\end{probleme}

\begin{proposition}
An element $A\in\cA_{\eR}$ is positive if and only if the Gelfand transform $\hat A$ is pointwise positive in $C(\sigma(A))$.
\end{proposition}

\begin{proof}
\subdem{Necessary condition}
We know from theorem \ref{tho:unital_comm}, \ref{enugiii} that
\[ 
  \sigma(A)=\sigma(\hat A)=\{ \hat A(\omega)\tq\omega\in\Delta(\cA) \}, 
\]
but $\sigma(A)\subset\eR^+$ if $A$ is positive.

\subdem{Sufficient condition}
From hypothesis, $A\in\cA_{\eR}$ and $A^*=A$. We have to see that positivity of $\hat A$ implies $\sigma(A)\subset\eR^+$. From point \ref{enukiii} of theorem \ref{tho:l_2.5.1}, the function $\dpt{\hat A}{\sigma(A)}{\eR}$ is identity and positive, $\sigma(A)\subset\eR^+$.

\end{proof}

 
\begin{proposition}     \label{PropAplusConvexCone}
    The set $\cA^+$ of positive elements of the $C^*$-algebra $\cA$ is a convex cone (see definition \ref{DefConvexCone}).
\end{proposition}

Note that the $C^*$-algebra $\cA$ has to be commutative in order the Gelfand transform to be defined. It is supposed unital too. 

\begin{proof}

    We have to check the \(3\) points of definition \ref{DefConvexCone}.
    \begin{enumerate}
            \item

                We know that if $A=A^*$ and $f\in C(\sigma(A))$, the commutator $[\sigma,f]$ is zero; as a particular case $\sigma(tA)=t\sigma(A)$. Then for $t>0$, the element $tA$ is positive.

            \item

                The fact that $\sigma(A)\subset [0,r(A)]$ implies that for all $t\in\sigma(A)$ and for all $c\geq r(A)$,  $| c-t |\leq c$. Now we study the quantity
                \[ 
                  \sup_{t\in \sigma(A)}| c1_{\sigma(A)}-\hat A |.
                \]
                The function $1_{\sigma(A)}$ is $0$ or $1$ following the argument belongs to $\sigma(A)$ or not while $\hat A(t)=t$ in $\sigma(A)$. Then
                \[ 
                  \sup_{t\in\sigma(A)}| c1_{\sigma(A)}(t)-\hat A(t) |=\sup_{t\in\sigma(A)}| c-t |\leq c.
                \]
                This shows that 
                \begin{equation} \label{eq:cunhatA}
                    \| c1_{\sigma(A)}-\hat A \|_{\infty}\leq c
                \end{equation}
                for all $c>r(A)$ and then for all $c>\| A \|$. Since $\cA$ is commutative and $A=A^*$, we know that $\| \hat A \|_{\infty}=\| A \|$ from  \eqref{eq:AinfA }. Taking the inverse Gelfand transform of equation \eqref{eq:cunhatA}, we find
                \begin{equation} \label{eq:norcin}
                    \| c\cun-A \|\leq c
                \end{equation}
                for all $c\geq\| A \|$. Be careful on a point: the inverse Gelfand transform is not taken into $\cA$, but into $C^*(A,\cun)$ which is commutative and unital and then fulfills $\| \hat A \|_{\infty}=\| A \|$.

                We know that the norm of $f(A)$ in $\cA$ and in $C^*(A,\cun)$ are the same, namely equation \eqref{eq:norcin} is a relation for the norm of $c\cun-A$ in $C^*(A,\cun)$. Until now we had proved that if $\sigma(A)\subset\eR^+$, then $\| c\cun-A \|\leq c$ for all $c\geq\| A \|$. 

                Taking the inverse argument, we can say that if $\| c\cun-A \|\leq c$ for a certain $c\geq\| A \|$, then $\sigma(A)\subset\eR^+$. Indeed the Gelfand transform of the assumption gives $\| cA_{\sigma(A)}-\hat A \|_{\infty}\leq c$, i.e. $\sup_{t\in\sigma(A)}| c1_{\sigma(A)}A-\hat A |\leq c$. As $\hat A$ is identity on $\sigma(A)$, for all $t\in\sigma(A)$, we have $| c-t |\leq c$. This shows that $t>0$ for all $t\in\sigma(A)$. Thus $\sigma(A)\subset\eR^+$.

                Let us now take $A+B$ instead of $A$ and $c=\| A \|+\| B \|$. Remark that $c\geq \| A+B \|$. We have
                \[ 
                    \| c\cun-(A+B) \|\leq\| (\| A \|-A) \|+\| (\| B \|-B) \|
                \]
                where $\| A \|-A=r\cun-A$ with $r=\| A \|$. On the other hand, $\| r\cun-A \|\leq r$ for all $r\geq\| A \|$, then we can apply the first result to get 
                \[ 
                    \| c\cun-(A+B) \|\leq \| A \|+\| B \|=c
                \]
                with $c\geq \| A+B \|$. Then the inverse argument gives $\sigma(A+B)\subset\eR^+$ and $A+B\in\cA^+$.

            \item

                If $A\in\cA^+\cup (-\cA^+)$. Then $\sigma(A)\subset\eR^+$ and $\sigma(A)\subset\eR^-$; we conclude that $\sigma(A)=\{  0\}$. Since $\| A \|=r(A)$, this gives $\| A \|=0$.
                
        \end{enumerate}
\end{proof}


\begin{proposition}
Let $E$ be a real locally convex space and $C$ a closed convex cone with top on $0$\quext{Par top je veux dire le sommet du cône je ne sais pas comment dire en anglais.} and $x\in E$, $x\notin C$. Then there exists a continuous linear function $f\colon E\to \eR$ such that
\begin{itemize}
\item $f\geq 0$ on $C$,
\item $f(x)<0$.
\end{itemize}

\dixref{B.5}
\end{proposition}

\begin{proof}
It is possible to find a continuous linear form $f$ and a real $\alpha$ such that $f(y)\geq\alpha$ on $C$ and $f(x)<\alpha$ (see Urysohn lemma \ref{lem:Urysohn}). We have $0=f(0)\geq\alpha$, so $f(x)<0$. If $f(y)<0$ fora $y\in C$, we find $f(\lambda y)<\alpha$ for a large enough $\lambda$. This is absurd and we conclude that $f\geq0$ on $C$.
\end{proof}
