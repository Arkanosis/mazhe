\subsubsection{Light cone}
%------------------------

General theory about symmetric spaces (see lemma \ref{lem:AdkEcone} and \cite{kobayashi2}, pages 230--233) says that if $E$ is nilpotent in $\sQ$, then $\{\Ad(k)E\}_{k\in K_H}$ is the set of all the light-like vectors in $T_{[\mfo]}AdS_l\simeq\sQ$. Therefore
\begin{equation}
  \exp_{\mfo}( t\Ad(k)E )=\exp(t\Ad(k)E)\cdot\mfo.
\end{equation}
is the light-cone in $G/H$. Note that in this equation, the first $\exp$ is the
one defined from the $AdS_l$-connection
while the second is the exponential from a Lie algebra to the Lie group. It comes from the fact that in a symmetric space, $\exp_o v=e^z\cdot\mfo$.

So we begin by find a nilpotent element of $\sQ$. A general form is given by \eqref{EqGeneRedQ}. It is rather easy to see that the element
\begin{equation}
E_1=
\begin{pmatrix}
0&1&1&0\\
-1\\
1\\
0
\end{pmatrix}
\end{equation}
 fulfils $E_1^3=0$, in such a manner that Engel's theorem 
%\ref{lem:pre_Engel}
 makes it nilpotent. The future light-cone at $\mfo$ is given by
\begin{equation}  \label{eq:exprcone}
C_{\mfo}=\{  \pi\big(  e^{t\Ad(k)E_1}  \big)  \}_{t\in\eR^+,k\in K_H},
\end{equation}
and since $G$ acts on $AdS$ by isometries, the light-cone at $\pi(g)$ is given by
\begin{equation} \label{eq_defcone}
  C^+_{\pi(g)}=g\cdot C_{\mfo}=\{  \pi\big( g e^{t\Ad(k)E_1}  \big)  \}_{t\in\eR^+,k\in K_H},
\end{equation}
The denomination ``future'' refers to the fact that it only contains positive $t$. Past light cones correspond to negative $t$.
One can argue that this is not well defined because the product is taken at left while the quotient is taken at right. 

\begin{proposition}
The definition \eqref{eq_defcone} is correct because
\begin{equation}  \label{eq_statdefcone}
  \{ \Ad(hk)E_1 \}_{k\in K_H}=\{ \Ad(k)E_1 \}_{k\in K_H}
\end{equation}
for all $h\in H$. 
\end{proposition}

\begin{proof}
Let us decompose $h=a_hn_hk_h$; the part $k_h$ is just a redefinition of $k$ in equation \eqref{eq_statdefcone}, so we forget it. We begin by proving that \eqref{eq_statdefcone} holds whenever $\Ad(h)\in \SO(\sQ)$. Consider $\Ad(k')E_1=X\in\sQ$. If $\Ad(h)\in \SO(\sQ)$, then $\Ad(h^{-1})\in \SO(\sQ)$ too and we consider $Y=\Ad(h^{-1})X$ which is a vector of norm zero in $\sQ$. There exists $\bar k\in K_H$ such that $\Ad(\bar k)E_1X=Y$. Now,
\begin{equation}
\begin{split}
\Ad(h\bar kk')E_1&=\Ad(h\bar k)X\\
		&=\Ad(h)Y\\
		&=X.
\end{split}
\end{equation}
In order to prove that $\Ad(a_h)\in So(\sQ)$, we compute
\[ 
  \ad(J_1)\begin{pmatrix}
0	& z	& w_1	& w_2	& w3\\
-z\\
w_1\\
w_2\\
w_3
\end{pmatrix}
=\ad(J_1)(zq_0+w_iq_i).
\]
In the basis $\{ q_0,q_i \}$, we see that
\[ 
  \ad(J_1)=\begin{pmatrix}
0&0&-1&0\\
0\\-1\\0
\end{pmatrix}\in\mathfrak{so}(1,3),
\]
so $\Ad(J_1)\in \SO(\sQ)$. On the other hand, a general element of $\sN_{\sH}$ is
\[ 
  A=\begin{pmatrix}
\cdot\\
&\cdot& a&\cdot& v\\
&a&\cdot &-a&\cdot\\
&  \cdot& a&\cdot& v\\
&v&\cdot&-v&\cdot\\
\end{pmatrix},
\]
and simple computations shows that on $\sQ$,
\[ 
  \ad(A)=\begin{pmatrix}
\cdot &-a&\cdot&-v\\
-a&\cdot&-a&\cdot\\
\cdot&a&\cdot &v\\
-v&\cdot&v&\cdot
\end{pmatrix}\in\mathfrak{so}(1,3).
\]
\end{proof}

\begin{proof}[Alternative proof]
The Killing form of $\sG$ restricted to $\sQ$ is the metric on $\sQ$ (notice that $\sQ$ has no own Killing form for the simple reason that it is not a Lie algebra). From $\Ad$-invariance of Killing, we have in particular
\[ 
  B\big( \Ad(h)X,\Ad(h)Y \big)=B(X,Y)
\]
for all $h\in H$. The point is that reducibility makes $\Ad(h)X\in\sQ$ when $X\in\sQ$.
\end{proof}

We are now able to define the causality as follows.  A point $[g]\in AdS_l$ belongs to the \defe{interior region}{Interior region} if for all direction $k\in K_H$, the future light ray through $[g]$ intersects the singularity within a \emph{finite} time.  In other words, it is interior when the whole light cone ends up in the singularity.  A point is \defe{exterior}{Exterior region} when it is not interior.  A particularly important set of point is the \defe{event horizon}{Event horizon}, or simply \emph{horizon}, defined as the boundary of the interior. When a space contains a non trivial causal structure (i.e. when there exists a non empty horizon), we say that the definition of singularity gives rise to a \defe{black hole}{Black hole}.  

Now, if $uk$ is the decomposition of an element in $K=\SO(2)\times \SO(n)$ and $X=k'\cdot E_1$, then
\begin{equation}
\begin{split}
[uke^{tX}]&=[uke^{tX}k^{-1}]\\
          &=[u\AD(k)e^{tX}]\\
          &=[ue^{t\Ad(k)X}].
\end{split}
\end{equation}
So a product by an element of $\son$ is just a reparametrization of $K_H$. The conclusion is that we can put $g=e^{uR}$ with

\subsubsection{Causality}
%---------------------

We decree that the closed orbits are \emph{singular} ---we denote them by $\mS$--- and that $\pi(g)\in AdS$ is \defe{exterior}{Exterior!point} if there exists an open set $\mO$ of $K_H$ such that $\forall k\in\mO$ and $\forall t\in\eR^+$,
\[
  \pi\big( g e^{t\Ad(k)E_1}  \big)\cap\mS=\emptyset.
\]
Naturally, $\pi(g)$ is \defe{interior}{Interior!of a black hole} if it is not exterior. 

Now we compute $\Ad(k)E_1=e^{\ad(K)}E_1$ with a general $k\in \SO(n)$ in the sense of
\[
K=\begin{pmatrix}
0&0\\
0&0\\
&&\fbox{$\son$}
\end{pmatrix}.
\]
In particular, $K$ is antisymmetric and can be written as $K=a^{ij}(E_{ij}-E_{ji})$ with $i,j\geq 3$ and $a^{ij}=-a^{ji}$. In a first time we find
\[
[K,E_1]=(2a)^{i3}(E_{1i}+E_{i1}).
\]
By computing $\ad(K)^3E_1$, we see that
\[
\ad(K)^nE_1=\big((2a)^n\big)^{k3}(E_{k1}+E_{1k}),
\]
hence,
\begin{equation} \label{eq:Adkeu}
\begin{split}
\Ad(k)E_1=e^{\ad K}E_1&=E_1+\sum_{n\geq 1}\big(e^{2a}\big)^{k3}(E_{k1}+E_{1k})\\
	      &=E_1+\sum_{n=0}^{\infty}\big(e^{2a}\big)^{k3}(E_{k1}+E_{1k})-\delta^{j3}(E_{j1}+E_{1j})\\
	      &=q_0+w_1q_1+w_2q_2+w_3q_3.
\end{split}
\end{equation}

Since $a$ is skew-symmetric, it belongs to $\sod$ and $e^{2a}$ is a matrix of $\SO(2)$. Then $\sum_k(e^{2a})^{k3}=1$. Remark moreover that \emph{all} matrices of $\SO(2)$ can be written under the for $e^{2a}$ for a good choice of $a\in\sod$, then the light cone is given by \emph{all} the vectors of the form $(1,w_1,w_2,w_3)$ with $\|w\|^2=1$. If we consider the metric $diag(+---)$ on $\sQ$ with respect to the basis $\{q_i\}$, then
\[
  \|\Ad(k)E_1\|^2=0.
\]
As far as this should describe a light cone, it is a good point.

Let us point out the fact that only the first column of the ``direction''{} $k\in \SO(n)$ has an importance in causality questions. So the word ``directions''{} will often be used to refer to the vector $w$. It is not a particular feature of our explicit matrices choices. Indeed the element $k$ only appears in the combination $\Ad(k)E_1$ which is a light-like vector in $\sQ$, i.e. $\Ad(k)E=tq_0+\sum_i x_iq_i$ with $t^2-\sum_i x_i^2=0$ for any suitable choice of basis $\{q_i\}$ of $\sQ$. As far as causality is concerned, a rescaling $\Ad(k)E$ to $\lambda\Ad(k)E$ has no importance, so one can choice $t=1$ and keep with $\sum_i x_i=1$. Then it is a natural feature that the light-like rays are parametrized by an unital vector $w\in\eR^n$.

A time orientation on $\sQ$ is the choice of a vector $T$ such that $\scal{T}{T}>0$. When such a choice is made, a vector $v$ is \defe{future directed}{Future directed!vector} when $\scal{v}{T}>0$. In our case, the choice is completely intuitive: the vector $q_0$ defines the time orientation of $\sQ$ and $v=(v^0,v^1,v^2,v^3)$ is future directed if and only if $v^0>0$. So a light-like future directed vector is always --up to a positive multiple-- of the form $(1,\overline{v})$ with $\|\overline{v}\|=1$. For this reason, the set 
\[
  \{t\Ad(k)E_1\}_{%
\begin{subarray}{l}
t>0\\k\in \SO(3) 
\end{subarray}
}
\]
is exactly the set of all the light-like future-directed vectors of $\sQ$.

