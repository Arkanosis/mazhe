\begin{exercice}\label{exoCalculDifferentiel0018}

	Une application $f : U \to \eR $ de classe $C^2$ sur un ouvert $U$ de $\eR^3$ est dite \defe{harmonique}{harmonique!fonction} si et seulement si $\Delta f = 0$, où
	\begin{equation}
		\Delta f=\frac{ \partial^2f  }{ \partial x^2 }+\frac{ \partial^2f  }{ \partial y^2 }+\frac{ \partial^2f  }{ \partial z^2 }
	\end{equation}
 est le laplacien de $f$.
 \begin{enumerate}
 \item
  Montrer que si $f$ est harmonique sur $\eR^2$ et de classe $C^3$ alors $\frac{\partial f}{\partial x}$ et $y \frac{\partial f}{\partial x} - x \frac{\partial f}{\partial y}$ sont harmoniques. Note : deviner quelle est la formule du laplacien sur $\eR^2$.
 \item
	 Vérifier que l'application $f\colon\eR^3\to \eR$ définie par
	 \begin{equation}
 		f(x,y,z)  = \arctan \frac{y}{x} + \arctan \frac{z}{y} + \arctan \frac{x}{z}
	 \end{equation}
est harmonique sur $\eR^3$.
 \end{enumerate}
 
\corrref{CalculDifferentiel0018}
\end{exercice}
