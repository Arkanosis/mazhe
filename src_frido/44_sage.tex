% This is part of (almost) Everything I know in mathematics
% Copyright (C) 2009-2010,2016
%   Laurent Claessens
% See the file fdl-1.3.txt for copying conditions.

Ce chapitre est un foure-tout de choses que l'on peut faire avec Sage.


Certains exemples utilisent ce module qui sert à automatiser certaines tâches : \info{outilsINGE.py}
\lstinputlisting{outilsINGE.py}


\begin{example}     \label{ExBCRXooDVUdcf}
	Calculer la limite 
			\begin{equation}
				\lim_{x\to\infty}\frac{ \sin(x)\cos(x) }{ x }
			\end{equation}


            \begin{verbatim}
			var('x')
			f(x)=sin(x)*cos(x)/x
			limit(f(x),x=oo)
            \end{verbatim}
	La première ligne déclare que la lettre \texttt{x} désignera une variable. Pour la troisième ligne, notez que l'infini est écrit par deux petits \og o\fg.
\end{example}

\begin{example}     \label{ExCWDRooKxnjGL}
    Quelques limites et graphes avec Sage.

    \begin{enumerate}

		\item
			$\lim_{x\to 0} \frac{ \sin(\alpha x) }{ \sin(\beta x) }$.

			Pour effectuer cet exercice avec Sage, il faut taper les lignes suivantes~:

            
\begin{verbatim}
sage: var('x,a,b')
(x, a, b)
sage: f(x)=sin(a*x)/sin(b*x)
sage: limit( f(x),x=0  )
a/b
\end{verbatim}

			Noter qu'il faut déclarer les variables \texttt{x}, \texttt{a} et \texttt{b}.

		\item
			$\lim_{x\to \pm\infty} \frac{ \sqrt{x^2+1}-x }{ x-2 }$

            \begin{verbatim}
sage: f(x)=(sqrt(x**2+1))/(x-2)
sage: limit(f(x),x=oo)
1
sage: limit(f(x),x=-oo)
-1
            \end{verbatim}

			Noter la commande pour la racine carré~: \texttt{sqrt}. Étant donné que cette fonction diverge en $x=2$, si nous voulons la tracer, il faut procéder en deux fois :

            \begin{verbatim}
sage: plot(f,(-100,1.9))
Launched png viewer for Graphics object consisting of 1 graphics primitive
sage: plot(f,(2.1,100))
Launched png viewer for Graphics object consisting of 1 graphics primitive
            \end{verbatim}
			La première ligne trace de $-100$ à $1.9$ et la seconde de $2.1$ à $100$. Ces graphiques vous permettent déjà de voir les limites. Attention : ils ne sont pas des \emph{preuves} ! Mais ils sont de sérieux indices qui peuvent vous inspirer dans vos calculs.

	\end{enumerate}
\end{example}

\begin{example} \label{exJMGTooZcZYNy}
    

Calculer les dérivées partielles $\partial_xf$, $\partial_yf$, $\partial^2_xf$, $\partial^2_{xy}f$, $\partial^2_{yx}f$ et $\partial^2_yf$ des fonctions suivantes.
\begin{multicols}{2}
\begin{enumerate}
\item
$2x^3+3x^2y-2y^2$
\item
$\ln(xy^2)$
\item 
$\tan(x/y)$
\item 
$\frac{ xy^2 }{ x+y }$ 

\end{enumerate}	
\end{multicols}


Le script Sage suivant (\verb+exoDV002.sage+) résout l'exercice : 


\lstinputlisting{exoDV002.sage}

La sortie est :

\VerbatimInput[tabsize=3]{exoDV002.txt}

\end{example}

\begin{example}\label{exKGDIooVefujD}



Résoudre les systèmes suivants.
\begin{multicols}{2}
	\begin{enumerate}

			\item
                                        $
                                        \left\{
                                        \begin{array}{ll}
                                                        x_1 - 2x_2 + 3x_3 - 2x_4 = 0\\ 
                                        3x_1 - 7x_2 - 2x_3 + 4x_4 = 0\\ 
                                        4x_1 + 3x_2 + 5x_3 + 2x_4 = 0\\ 

                                        \end{array}
                                        \right.
                                        $



                                \item
                                        $
                                        \left\{
                                        \begin{array}{ll}
                                                        2x_1 + x_2 - 2x_3 + 3x_4 = 0\\ 
                                        3x_1 + 2x_2 - x_3 + 3x_4 = 4\\ 
                                        3x_1 + 3x_2 + 3x_3 - 3x_4 = 9\\ 

                                        \end{array}
                                        \right.
                                        $



                                \item
                                        $
                                        \left\{
                                        \begin{array}{ll}
                                                        x_1 + 2x_2 - 3x_3 = 0\\ 
                                        2x_1 + 5x_2 + 2x_3 = 0\\ 
                                        3x_1 - x_2 - 4x_3 = 0\\ 

                                        \end{array}
                                        \right.
                                        $



                                \item
                                        $
                                        \left\{
                                        \begin{array}{ll}
                                                        x_1 + 2x_2 - x_3 = 0\\ 
                                        2x_1 + 5x_2 + 2x_3 = 0\\ 
                                        x_1 + 4x_2 + 7x_3 = 0\\ 
                                        x_1 + 3x_2 + 3x_3 = 0\\ 

                                        \end{array}
                                        \right.
                                        $



                                \item
                                        $
                                        \left\{
                                        \begin{array}{ll}
                                                        x_1 + x_2 + x_3 + x_4 = 0\\ 
                                        x_1 + x_2 + x_3 - x_4 = 4\\ 
                                        x_1 + x_2 - x_3 + x_4 = -4\\ 
                                        x_1 - x_2 + x_3 + x_4 = 2\\ 

                                        \end{array}
                                        \right.
                                        $



                                \item
                                        $
                                        \left\{
                                        \begin{array}{ll}
                                                        x_1 + 3x_2 + 3x_3 = 1\\ 
                                        x_1 + 3x_2 + 4x_3 = 0\\ 
                                        x_1 + 4x_2 + 3x_3 = 3\\ 

                                        \end{array}
                                        \right.
                                        $



                                \item
                                        $
                                        \left\{
                                        \begin{array}{ll}
                                                        x_1 - 3x_2 + 2x_3 = -6\\ 
                                        -3x_1 + 3x_2 - x_3 = 17\\ 
                                        2x_1 - x_2 = 3\\ 

                                        \end{array}
                                        \right.
                                        $



                                \item
                                        $
                                        \left\{
                                        \begin{array}{ll}
                                                        x_1 - 2x_2 + 3x_3 - 2x_4 = 0\\ 
                                        3x_1 - 7x_2 - 2x_3 + 4x_4 = 0\\ 
                                        4x_1 + 3x_2 + 5x_3 + 2x_4 = 0\\ 

                                        \end{array}
                                        \right.
                                        $



                                \item
                                        $
                                        \left\{
                                        \begin{array}{ll}
                                                        2x_1 + x_2 - 2x_3 + 3x_4 = 0\\ 
                                        3x_1 + 2x_2 - x_3 + 3x_4 = 4\\ 
                                        3x_1 + 3x_2 + 3x_3 - 3x_4 = 9\\ 

                                        \end{array}
                                        \right.
                                        $



                                \item
                                        $
                                        \left\{
                                        \begin{array}{ll}
                                                        x_1 + 2x_2 - 3x_3 = 0\\ 
                                        2x_1 + 5x_2 + 2x_3 = 0\\ 
                                        3x_1 - x_2 - 4x_3 = 0\\ 

                                        \end{array}
                                        \right.
                                        $



                                \item
                                        $
                                        \left\{
                                        \begin{array}{ll}
                                                        x_1 + 2x_2 - x_3 = 0\\ 
                                        2x_1 + 5x_2 + 2x_3 = 0\\ 
                                        x_1 + 4x_2 + 7x_3 = 0\\ 
                                        x_1 + 3x_2 + 3x_3 = 0\\ 

                                        \end{array}
                                        \right.
                                        $



                                \item
                                        $
                                        \left\{
                                        \begin{array}{ll}
                                                        x_1 + x_2 + x_3 + x_4 = 0\\ 
                                        x_1 + x_2 + x_3 - x_4 = 4\\ 
                                        x_1 + x_2 - x_3 + x_4 = -4\\ 
                                        x_1 - x_2 + x_3 + x_4 = 2\\ 

                                        \end{array}
                                        \right.
                                        $



                                \item
                                        $
                                        \left\{
                                        \begin{array}{ll}
                                                        x_1 + 3x_2 + 3x_3 = 1\\ 
                                        x_1 + 3x_2 + 4x_3 = 0\\ 
                                        x_1 + 4x_2 + 3x_3 = 3\\ 

                                        \end{array}
                                        \right.
                                        $



                                \item
                                        $
                                        \left\{
                                        \begin{array}{ll}
                                                        x_1 - 3x_2 + 2x_3 = -6\\ 
                                        -3x_1 + 3x_2 - x_3 = 17\\ 
                                        2x_1 - x_2 = 3\\ 

                                        \end{array}
                                        \right.
                                        $

	\end{enumerate}


\end{multicols}


	Nous résolvons les systèmes en utilisant Sage avec le script suivant.

\lstinputlisting{exo11.sage}

Le résultat est le suivant :

\VerbatimInput[tabsize=3]{exo11.txt}

\end{example}

\begin{example}     \label{ExBGCEooPIQgGW}
    

	Pour chacun des systèmes suivants $A\cdot X=B$,
	\begin{enumerate}

		\item
			Résoudre le système par échelonnement,
		\item
			Calculer $A^{-1}$,
		\item
			Vérifier votre réponse en calculant $A^{-1}B$. Qu'êtes-vous censé obtenir ?

	\end{enumerate}

	Les énoncés sont
	\begin{enumerate}

		\item
			\begin{equation}
				\begin{aligned}[]
					A=\begin{pmatrix}
						2	&	1	&	-2	\\
						3	&	2	&	2	\\
						5	&	4	&	3
					\end{pmatrix},
					&&B=\begin{pmatrix}
						10	\\ 
						1	\\ 
						4	
					\end{pmatrix}
				\end{aligned}
			\end{equation}
	\end{enumerate}


	Nous utilisons Sage pour fournir la réponse. Le code suivant résout le système et donne l'inverse de la matrice :

\lstinputlisting{exo13.sage}

La sortie est ici :

\VerbatimInput[tabsize=3]{exo13.txt}



\end{example}

\begin{example}     \label{exBNGVooIvKfTT}
    

	 Sachant que $(-1,0,1,0)$ est un vecteur propre de la matrice
	\begin{equation}
		A=\begin{pmatrix}
			 2	&	1	&	-1	&	1	\\
			 1	&	0	&	1	&	1	\\
			 -1	&	1	&	2	&	1	\\ 
			 1	&	1	&	1	&	0	 
		 \end{pmatrix}
	\end{equation}
	\begin{enumerate}

		\item
			Diagonaliser $A$ au moyen d'une matrice orthogonale
		\item
			Écrire la forme quadratique $X^tAX$ sous forme d'une somme pondérée de carrés.
	\end{enumerate}



	Calculons $Av$ afin de savoir la valeur propre associée au vecteur donné :
	\begin{equation}
		\begin{pmatrix}
			 2	&	1	&	-1	&	1	\\
			 1	&	0	&	1	&	1	\\
			 -1	&	1	&	2	&	1	\\ 
			 1	&	1	&	1	&	0	 
		 \end{pmatrix}
		 \begin{pmatrix}
			 -1	\\ 
			 0	\\ 
			 1	\\ 
			 0	
		 \end{pmatrix}
		 =
		 \begin{pmatrix}
			 -3	\\ 
			 0	\\ 
			 3	\\ 
			 0	
		 \end{pmatrix}.
	\end{equation}
	La valeur propre est donc $3$. Nous savons donc que $(\lambda-3)$ pourra être factorisé dans le polynôme caractéristique.

	Pour le reste de l'exercice c'est standard et c'est résolu de la façon suivante :

	\lstinputlisting{exo65.sage}

	qui retourne

	\VerbatimInput[tabsize=3]{exo65.txt}

\end{example}

\begin{example}     \label{exZHGRooTQpVpq}
    

	 Rechercher les extrema des fonctions suivantes
	\begin{enumerate}

		\item
			$f(x,y)=2-\sqrt{x^2+y^2}$
		\item
			$f(x,y)=x^3+3xy^2-15x-12y$
		\item
			$f(x,y)=\frac{ x^3 }{ 3 }+\frac{ 4y^3 }{ 3 }-x^2-3x-4y-3$

	\end{enumerate}




	Les corrigés sont créés par le script Sage \verb+exo101.sage+

	\VerbatimInput[tabsize=3]{exo101.sage}

	Des réponses :
	
	\begin{enumerate}

		\item
			\VerbatimInput[tabsize=3]{exo101A.txt}

			Ici nous voyons que Sage a du mal à calculer la matrice Hessienne en $(0,0)$. En effet, nous tombons sur une division par zéro. Pour résoudre l'exercice, il faut se rendre compte que la fonction $(x,y)\mapsto\sqrt{x^2+y^2}$ est toujours positive et est nulle seulement au point $(0,0)$. Donc $f$ est toujours plus petite ou égale à deux tandis que $f(0,0)=2$. Le point est donc un maximum global.
		\item
			\VerbatimInput[tabsize=3]{exo101B.txt}

		\item
			\VerbatimInput[tabsize=3]{exo101C.txt}

	\end{enumerate}

\end{example}

\begin{example}     \label{exHWIHooOAvaDQ}
    


	Déterminer les valeurs extrêmes et les points de selle des fonctions suivantes.
	\begin{multicols}{2}
		\begin{enumerate}
			\item	%1
				$f(x,y)=x^2+4x+y^2-2y$.
			\item	%8
				$f(x,y)= e^{x^2+xy}$.
			\item	%17
				$f(x,y)=e^x\sin(y)$.
		\end{enumerate}
	\end{multicols}


	Certains corrigés de cet exercice ont étés réalisés par Sage. Le script utilisé est \verb+exo103.sage+

	\lstinputlisting{exo103.sage}

	Des réponses :

	\begin{enumerate}
		\item	%1
			\VerbatimInput[tabsize=3]{exo103A.txt}
		\item	%8
			\VerbatimInput[tabsize=3]{exo103H.txt}
		\item	%17
			\VerbatimInput[tabsize=3]{exo103Q.txt}

			Ici, Sage n'est pas capable de résoudre les équations qui annulent le jacobien. Les équations à résoudre sont pourtant faciles :
			\begin{subequations}
				\begin{numcases}{}
					e^{x}\cos(y)=0\\
					e^{x}\sin(y)=0
				\end{numcases}
			\end{subequations}
			Étant donné que l'exponentielle ne s'annule jamais, il faudrait avoir en même temps $\cos(y)=0$ et $\sin(y)=0$, ce qui est impossible. La fonction n'a donc aucun extrema local.

	\end{enumerate}

\end{example}


\begin{example}     \label{exEEHPooKDxLTJ}
    

	Considérons la fonction
	\begin{equation}
		f(x,y)=xy^2 e^{-(x^2+y^2)/4}.
	\end{equation}
	\begin{enumerate}

		\item
			Montrer qu'il y a une infinité de points critiques.
		\item
			Déterminer leur nature.

	\end{enumerate}

	Voici la fonction Sage qui fournit les informations :

	\lstinputlisting{exo104.sage}

	La sortie est

	\VerbatimInput[tabsize=3]{exo104.txt}

	Notez la présence de \verb+r1+ comme paramètres dans les solutions. Tous les points avec $y=0$ sont des points critiques. Cependant, Sage\footnote{ou, plus précisément, le programme que j'ai écrit avec Sage.} ne parvient pas à conclure la nature de ces points $(x,0)$.

	Notons que le nombre $f(x,y)$ a toujours le signe de $x$ parce que $y^2$ et l'exponentielle sont positives. Toujours ? En tout cas lorsque $x\neq 0$. Prenons un point $(a,0)$ avec $a>0$. Dans un voisinage de ce point, nous avons $f(x,y)>0$ parce que si $a>0$, alors $x>0$ dans un voisinage de $a$. Le point $(a,0)$ est un minimum local parce que $0=f(a,0)\leq f(x,y)$ pour tout $(x,y)$ dans un voisinage de $(a,0)$.

	De la même façon, les points $(a,0)$ avec $a<0$ sont des maxima locaux parce que dans un voisinage, la fonction est négative. 

	Le point $(0,0)$ n'est ni maximum ni minimum local. C'est un point de selle.

\end{example}

\begin{example}     \label{exRNZKooUIOfPU}

    Dériver les fonctions suivantes.
	\begin{enumerate}
		\item 
			$\sin\big( \ln(x) \big)$
		\item
			$\displaystyle \frac{\sin x}{x}$ ;
		\item
			$ e^{x^2}$
		\item
			$\cos(x)^{\sin(x)}$
	\end{enumerate}

Le programme suivant par Sage résout l'exercice:
\lstinputlisting{corrDerive_0002.sage}

Le résultat est :
\VerbatimInput[tabsize=3]{corrDerive_0002.txt}

\end{example}

\begin{example}     \label{exLFYFooNCXCJz}
    
	Donner une approximation de $\ln(1.0001)$.

	\begin{verbatim}
		----------------------------------------------------------------------
		| Sage Version 4.5.3, Release Date: 2010-09-04                       |
		| Type notebook() for the GUI, and license() for information.        |
		----------------------------------------------------------------------
		sage: numerical_approx(ln(1.0001))
		0.0000999950003332973
	\end{verbatim}

\end{example}
