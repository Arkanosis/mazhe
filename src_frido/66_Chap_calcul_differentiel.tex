% This is part of Mes notes de mathématique
% Copyright (c) 2006-2016
%   Laurent Claessens, Carlotta Donadello
% See the file fdl-1.3.txt for copying conditions.

%+++++++++++++++++++++++++++++++++++++++++++++++++++++++++++++++++++++++++++++++++++++++++++++++++++++++++++++++++++++++++++
\section{Compacité}
%+++++++++++++++++++++++++++++++++++++++++++++++++++++++++++++++++++++++++++++++++++++++++++++++++++++++++++++++++++++++++++
%http://fr.wikipedia.org/wiki/Espace_compact
%http://fr.wikipedia.org/wiki/Théorème_de_Heine-Borel
%http://fr.wikipedia.org/wiki/Émile_Borel
%http://fr.wikipedia.org/wiki/Henri_Léon_Lebesgue

Soit $E$, un sous ensemble de $\eR$. Nous pouvons considérer les ouverts suivants : 
\begin{equation}
    \mO_x=B(x,1)
\end{equation}
pour chaque $x\in E$. Évidement,
\begin{equation}
    E\subseteq \bigcup_{x\in E}\mO_x.
\end{equation}
Cette union est très souvent énorme, et même infinie. Elle contient de nombreuses redondances. Si par exemple $E=[-10,10]$, l'élément $3\in E$ est contenu dans $\mO_{3.5}$, $\mO_{2.7}$ et bien d'autres. Pire : même si on enlève par exemple $\mO_2$ de la liste des ouverts, l'union de ce qui reste continue à être tout $E$. La question est : \emph{est-ce qu'on peut en enlever suffisamment pour qu'il n'en reste qu'un nombre fini ?}
\begin{definition}
Soit $E$, un sous ensemble de $\eR$. Une collection d'ouverts $\mO_i$ est un \defe{recouvrement}{recouvrement} de $E$ si $E\subseteq \bigcup_{i}\mO_i$. Un sous ensemble $E$ de $\eR$ tel que de tout recouvrement par des ouverts, on peut extraire un sous-recouvrement fini est dit \defe{\href{http://fr.wikipedia.org/wiki/Espace_compact}{compact}}{compact}.
\end{definition}

\begin{proposition}
Les ensembles compacts sont fermés et bornés.
\end{proposition}

\begin{proof}
Prouvons d'abord qu'un ensemble compact est borné. Pour cela, supposons que $K$ est un compact non borné vers le haut\footnote{Nous laissons à titre d'exercice le cas où $K$ est borné par le haut et pas par le bas.}. Donc il existe une suite infinie de nombres strictement croissante $x_1<x_2<\ldots$ tels que $x_i\in K$. Prenons n'importe quel recouvrement ouvert de la partie de $K$ plus petite ou égale à $x_1$, et complétons ce recouvrement par les ouverts $\mO_i=]x_{i-1},x_i[$. Le tout forme bien un recouvrement de $K$ par des ouverts. 

Il n'y a cependant pas moyen d'en tirer un sous recouvrement fini parce que si on ne prends qu'un nombre fini parmi les $\mO_i$, on en aura fatalement un maximum, disons $\mO_k$. Dans ce cas, les points $x_{k+1}$, $x_{k+1}$,\ldots ne seront pas dans le choix fini d'ouverts.

Cela prouve que $K$ doit être borné.

Pour prouver que $K$ est fermé, nous allons prouver que le complémentaire est ouvert. Et pour cela, nous allons prouver que si le complémentaire n'est pas ouvert, alors nous pouvons construire un recouvrement de $K$ dont on ne peut pas extraire de sous recouvrement fini.

Si $\eR\setminus K$ n'est pas ouvert, il possède un point, disons $x$, tel que tout voisinage de $x$ intersecte $K$. Soit $B(x,\epsilon_1)$, un de ces voisinages, et prenons $k_1\in K\cap B(x,\epsilon_1)$. Ensuite, nous prenons $\epsilon_2$ tel que $k_1$ n'est pas dans $B(x,\epsilon_1)$, et nous choisissons $k_2\in K\cap B(x,\epsilon_2)$. De cette manière, nous construisons une suite de $k_i\in K$ tous différents et de plus en plus proches de $x$. Prenons un recouvrement quelconque par des ouverts de la partie de $K$ qui n'est pas dans $B(x,\epsilon_1)$. Les nombres $k_i$ ne sont pas dans ce recouvrement.

Nous ajoutons à ce recouvrement les ensembles $\mO=]k_i,k_{i+1}[$. Le tout forme un recouvrement (infini) par des ouverts dont il n'y a pas moyen de tirer un sous recouvrement fini, pour exactement la même raison que la première fois.
\end{proof}

Le résultat suivant le théorème de \href{http://fr.wikipedia.org/wiki/Théorème_de_Heine-Borel}{Borel-Lebesgue}, et la démonstration vient de wikipédia.
\begin{theorem}[\href{http://fr.wikipedia.org/wiki/Émile_Borel}{borel}-\href{http://fr.wikipedia.org/wiki/Henri_Léon_Lebesgue}{Lebesgue}]   \label{ThoBOrelLebesgue}
    Les intervalles de la forme $[a,b]$ sont compacts.
\end{theorem}

\begin{proof}
    Soit $\Omega$, un recouvrement du segment $[a,b]$ par des ouverts, c'est à dire que
    \begin{equation}
        [a,b]\subseteq\bigcup_{\mO\in\Omega}\mO.
    \end{equation}
    Nous notons par $M$ le sous-ensemble de $[a,b]$ des points $m$ tels que l'intervalle $[a,m]$ peut être recouvert par un sous-ensemble fini de $\Omega$. C'est à dire que $M$ est le sous ensemble de $[a,b]$ sur lequel le théorème est vrai. Le but est maintenant de prouver que $M=[a,b]$.
    \begin{description}
        \item[$M$ est non vide] En effet, $a\in M$ parce que il existe un ouvert $\mO\in\Omega$ tel que $a\in\mO$. Donc $\mO$ tout seul recouvre l'intervalle $[a,a]$. 
        \item[$M$ est un intervalle] Soient $m_1$, $m_2\in M$. Le but est de montrer que si $m'\in[m_1,m_2]$, alors $m'\in M$. Il y a un sous recouvrement fini de l'intervalle $[a,m_2]$ (par définition de $m_2\in M$). Ce sous recouvrement fini recouvre évidement aussi $[a,m']$ parce que $[a,m']\subseteq [a,m_2]$, donc $m'\in M$.
        \item[$M$ est une ensemble ouvert] Soit $m\in M$. Le but est de prouver qu'il y a un ouvert autour de $m$ qui est contenu dans $M$. Mettons que $\Omega'$ soit un sous recouvrement fini qui contienne l'intervalle $[a,m]$. Dans ce cas, on a un ouvert $\mO\in\Omega'$ tel que $m\in\mO$. Tous les points de $\mO$ sont dans $M$, vu qu'ils sont tous recouverts par $\Omega'$. Donc $\mO$ est un voisinage de $m$ contenu dans $M$.
        \item[$M$ est un ensemble fermé] $M$ est un intervalle qui commence en $a$, en contenant $a$, et qui finit on ne sait pas encore où. Il est donc soit de la forme $[a,m]$, soit de la forme $[a,m[$. Nous allons montrer que $M$ est de la première forme en démontrant que $M$ contient son supremum $s$. Ce supremum est un élément de $[a,b]$, et donc il est contenu dans un des ouverts de $\Omega$. Disons $s\in\mO_s$. Soit $c$, un élément de $\mO_s$ strictement plus petit que $c$; étant donné que $s$ est supremum de $M$, cet élément $c$ est dans $M$, et donc on a un sous recouvrement fini $\Omega'$ qui recouvre $[a,c]$. Maintenant, le sous recouvrement constitué de $\Omega'$ et de $\mO_s$ est fini et recouvre $[a,s]$.
    \end{description}
    Nous pouvons maintenant conclure : le seul intervalle non vide de $[a,b]$ qui soit à la fois ouvert et fermé est $[a,b]$ lui-même, ce qui prouve que $M=[a,b]$, et donc que $[a,b]$ est compact.
\end{proof}

Par le théorème des valeurs intermédiaires, l'image d'un intervalle par une fonction continue est un intervalle, et nous avons l'importante propriété suivante des fonctions continues sur un compact.

Le théorème suivant est un cas particulier du théorème \ref{ThoMKKooAbHaro}.
\begin{theorem}
    Si $f$ est une fonction continue sur l'intervalle compact $[a,b]$. Alors $f$ est bornée sur $[a,b]$ et elle atteint ses bornes.
\end{theorem}

\begin{proof}
    Étant donné que $[a,b]$ est un intervalle compact, son image est également un intervalle compact, et donc est de la forme $[m,M]$. Ceci découle du théorème \ref{ThoImCompCotComp} et le corollaire \ref{CorImInterInter}. Le maximum de $f$ sur $[a,b]$ est la borne $M$ qui est bien dans l'image (parce que $[m,M]$ est fermé). Idem pour le minimum $m$.
\end{proof}

%+++++++++++++++++++++++++++++++++++++++++++++++++++++++++++++++++++++++++++++++++++++++++++++++++++++++++++++++++++++++++++ 
\section{Dérivation et croissance}
%+++++++++++++++++++++++++++++++++++++++++++++++++++++++++++++++++++++++++++++++++++++++++++++++++++++++++++++++++++++++++++

Supposons une fonction dont la dérivée est positive. Étant donné que la courbe est \og collée \fg{} à ses tangentes, tant que les tangentes montent, la fonction monte. Or, une tangente qui monte correspond à une dérivée positive, parce que la dérivée est le coefficient directeur de la tangente.

Ce résultat très intuitif peut être prouvé rigoureusement. C'est la tache à laquelle nous allons nous atteler maintenant.

\begin{proposition} \label{PropGFkZMwD}
    Si $f$ et $f'$ sont des fonctions continues sur l'intervalle $[a,b]$ et si $f'(x)$ est strictement positive sur $[a,b]$, alors $f$ est croissante sur $[a,b]$.

    De la même manière, si $f'(x)$ est strictement négative sur $[a,b]$, alors $f$ est décroissante sur $[a,b]$.
\end{proposition}

\begin{proof}
    Nous n'allons prouver que la première partie. La seconde partie se prouve en considérant $-f$ et en invoquant alors la première\footnote{Méditer cela.}. Prenons $x_1$ et $x_2$ dans $[a,b]$ tels que $x_1<x_2$. Par hypothèse, pour tout $x$ dans $[x_1,x_2]$, nous avons
    \begin{equation}
        f'(x)=\lim_{\epsilon\to 0}\frac{ f(x+\epsilon)-f(x) }{\epsilon} >0.
    \end{equation}
    Maintenant, la proposition \ref{PropoLimPosFPos} dit que quand une limite est positive, alors la fonction dans la limite est positive sur un voisinage. En appliquant cette proposition à la fonction
    \begin{equation}
        r(\epsilon)=\frac{ f(x+\epsilon)-f(x) }{ \epsilon },
    \end{equation}
    dont la limite en zéro est positive, nous trouvons que $r(\epsilon)>0$ pour tout $\epsilon$ pas trop éloigné de zéro. En particulier, il existe un $\delta>0$ tel que $\epsilon<\delta$ implique $r(\epsilon)>0$; pour un tel $\epsilon$, nous avons donc
    \begin{equation}
        r(\epsilon)=\frac{ f(x+\epsilon)-f(x) }{ \epsilon }>0.
    \end{equation}
    Étant donné que $\epsilon>0$, nous avons que $f(x+\epsilon)-f(x)>0$, c'est à dire que $f$ est strictement croissante entre $x$ et $x+\delta$.

    Jusqu'ici, nous avons prouvé que la fonction $f$ était strictement croissante dans un voisinage autour de chaque point de $[a,b]$. Cela n'est cependant pas encore tout à fait suffisant pour conclure. Ce que nous voudrions faire, c'est de dire, c'est prendre un voisinage $]a,m_1[$ autour de $a$ sur lequel $f$ est croissante. Donc, $f(m_1)>f(a)$. Ensuite, on prend un voisinage $]m_1,m_2[$ de $m_1$ sur lequel $f$ est croissante. De ce fait, $f(m_2)>f(m_1)>f(a)$. Et ainsi de suite, nous voulons construire des $m_3$, $m_4$,\ldots jusqu'à arriver en $b$. Hélas, rien ne dit que ce processus va fonctionner. Il faut trouver une subtilité. Le problème est que les voisinages sur lesquels la fonction est croissante sont peut-être de plus en plus petit, de telle sorte à ce qu'il faille une infinité d'étapes avant d'arriver à bon port (en $b$).

    Heureusement, nous pouvons drastiquement réduire le nombre d'étapes en nous souvenant du théorème de Borel-Lebesgue (numéro \ref{ThoBOrelLebesgue}). Nous notons par $\mO_x$, un ouvert autour de $x$ tel que $f$ soit strictement croissante sur $\mO_x$. Un tel voisinage existe. Cela fait une infinité d'ouverts tels que
    \begin{equation}
        [a,b]\subseteq\bigcup_{x\in[a,b]}\mO_x.
    \end{equation}
    Ce que le théorème dit, c'est qu'on peut en choisir un nombre fini qui recouvre encore $[a,b]$. Soient $\{ \mO_{x_1},\ldots,\mO_{x_n} \}$, les heureux élus, que nous supposons prit dans l'ordre : $x_1<x_2<\ldots<x_n$. Nous avons
    \begin{equation}
        [a,b]\subseteq\bigcup_{i=1}^n\mO_i.
    \end{equation}
    Quitte à les rajouter à la collection, nous supposons que $x_1=a$ et que $x_n=b$. Maintenant nous allons choisir encore un sous ensemble de cette collection d'ouverts. On pose $\mA_1=\mO_{x_1}$. Nous savons que $\mA_1$ intersecte au moins un des autres $\mO_{x_i}$. Cette affirmation vient du fait que $[a,b]$ est connexe (proposition \ref{PropInterssiConn}), et que si $\mO_{x_1}$ n'intersectait personne, alors 
    \begin{equation}
        \begin{aligned}[]
            \mO_{x_1}&&\text{et}&&\bigcup_{i=2}^n\mO_{x_i}
        \end{aligned}
    \end{equation}
    forment une partition de $[a,b]$ en deux ouverts disjoints, ce qui n'est pas possible parce que $[a,b]$ est connexe. Nous nommons $\mA_2$, un des ouverts $\mO_{x_i}$ qui intersecte $\mA_1$. Disons que c'est $\mO_k$. Notons que $\mA_1\cup\mA_2$ est un intervalle sur lequel $f$ est strictement croissante. En effet, si $y_{12}$ est dans l'intersection, $f(a)<f(y_{12})$ parce que $f$ est strictement croissante sur $\mA_1$, et pour tout $x>y_{12}$ dans $\mA_2$, $f(x)>f(y_{12})$ parce que $f$ est strictement croissante dans $\mA_2$. 

    Maintenant, nous éliminons de la liste des $\mO_{x_i}$ tous ceux qui sont inclus à $\mA_1\cup\mA_2$. Dans ce qu'il reste, il y en a automatiquement un qui intersecte $\mA_1\cup\mA_2$, pour la même raison de connexité que celle invoquée plus haut. Nous appelons cet ouvert $\mA_3$, et pour la même raison qu'avant, $f$ est strictement croissante sur $\mA_1\cup\mA_2\cup\mA_3$.

    En recommençant suffisamment de fois, nous finissons par devoir prendre un des $\mO_{x_i}$ qui contient $b$, parce qu'au moins un des $\mO_{x_i}$ contient $b$. À ce moment, nous avons finit la démonstration.
\end{proof}

Il est intéressant de noter que ce théorème concerne la croissance d'une fonction sous l'hypothèse que la dérivée est positive. Il nous a fallu très peu de temps, en utilisant la positivité de la dérivée, pour conclure qu'autour de tout point, la fonction était strictement croissante. À partir de là, c'était pour ainsi dire gagné. Mais il a fallu un réel travail de topologie très fine\footnote{et je te rappelle que nous avons utilisé la proposition \ref{PropInterssiConn}, qui elle même était déjà un très gros boulot !} pour conclure. Étonnant qu'une telle quantité de topologie soit nécessaire pour démontrer un résultat essentiellement analytique dont l'hypothèse est qu'une limite est positive, n'est-ce pas ? 

Une petite facile, maintenant.
\begin{proposition}
    Si $f$ est croissante sur un intervalle, alors $f'\geq 0$ à l'intérieur cet intervalle, et si $f$ est décroissante sur l'intervalle, alors $f'\leq 0$ à l'intérieur de l'intervalle.
\end{proposition}

Note qu'ici, nous demandons juste la croissance de $f$, et non sa \emph{stricte} croissance.

\begin{proof}
    Soit $f$, une fonction croissante sur l'intervalle $I$, et $x$ un point intérieur de $I$. La dérivée de $f$ en $x$ vaut
    \begin{equation}
        f'(x)=\lim_{\epsilon\to 0}\frac{ f(x+\epsilon)-f(x) }{\epsilon},
    \end{equation}
    mais, comme $f$ est croissante sur $I$, nous avons toujours que $f(x+\epsilon)-f(x)\geq0$ quand $\epsilon>0$, et $f(x+\epsilon)-f(x)\leq0$ quand $\epsilon<0$, donc cette limite est une limite de nombre positifs ou nuls, qui est donc positive ou nulle. Cela prouve que $f'(x)\geq 0$.
\end{proof}

% http://fr.wikipedia.org/wiki/Théorème_de_Rolle
% http://gconnan.free.fr/les%20pdf/Deriv.pdf
Les deux prochains théorèmes sont très importants.
\begin{theorem}[\href{http://fr.wikipedia.org/wiki/Théorème_de_Rolle}{Théorème de Rolle}]       \label{ThoRolle}
    Soit $f$, une fonction continue sur $[a,b]$ et dérivable sur $]a,b[$. Si $f(a)=f(b)$, alors il existe un point $c\in]a,b[$ tel que $f'(c)=0$.
\end{theorem}
\index{théorème!Rolle}

\begin{proof}
    Étant donné que $[a,b]$ est un intervalle compact, l'image de $[a,b]$ par $f$ est un intervalle compact, soit $[m,M]$ (théorème \ref{ThoImCompCotComp}). Si $m=M$, alors le théorème est évident : c'est que la fonction est constante, et la dérivée est par conséquent nulle. Supposons que $M> f(a)$ (il se peut que $M=f(a)$, mais alors si $f$ n'est pas constante, il faut avoir $m<f(a)$ et le reste de la preuve peut être adaptée).

    Comme $M$ est dans l'image de $[a,b]$ par $f$, il existe $c\in ]a,b[$ tel que $f(c)=M$. Considérons maintenant la fonction
    \begin{equation}
        \tau(x) =\frac{ f(c+x)-f(c) }{ x }.
    \end{equation}
    Par définition, $\lim_{x\to 0}\tau(x)=f'(c)$. Par hypothèse, si $u<c$,
    \begin{equation}
        \tau(u-c) = \frac{ f(u)-f(c) }{ u-c }>0
    \end{equation}
    parce que $u-c<0$ et $f(u)-f(c)<0$. Par conséquent, $\lim_{x\to 0}\tau(x)\geq 0$. Nous avons aussi, pour $v>c$,
    \begin{equation}
        \tau(v-c) = \frac{ f(v)-f(c) }{ v-c }<0
    \end{equation}
    parce que $v-c>0$ et $f(v)-f(c)<0$. Par conséquent, $\lim_{x\to 0}\tau(x)\leq 0$. Mettant les deux ensemble, nous avons $f'(c)=\lim_{x\to 0}\tau(x)=0$, et $c$ est le point que nous cherchions.
\end{proof}

Sur wikipédia, deux démonstrations complètement différentes sont proposées, celle qui est présentée ici est adaptée de celle qui est proposée par le célèste mathémator de \href{http://gconnan.free.fr/les\%20pdf/Deriv.pdf}{Téhessin le Rézéen}.

Le théorème suivant est le théorème des \defe{accroissements finis}{théorème!accroissements finis!dans $\eR$}.

\begin{theorem}[Accroissements finis]       \label{ThoAccFinis}
    Soit $f$, une fonction continue sur $[a,b]$ et dérivable sur $]a,b[$. 
        \begin{enumerate}
            \item       \label{ITEMooFZONooXJqLyX}
               Il existe au moins un réel $c\in]a,b[$ tel que 
                   \begin{equation}
                   f(b)-f(a)=(b-a)f'(c) .
                   \end{equation}
                   Autrement dit, la tangente en \( c\) est parallèle à la corde entre \( a\) et \( b\).
               \item       \label{ITEMooXRQKooDBFpdQ}
               Nous avons la majoration
               \begin{equation}
                   \big| f(b)-f(a) \big|\leq \sup_{x\in\mathopen[ a , b \mathclose]}| f'(x) |  | b-a |.
               \end{equation}
        \end{enumerate}
\end{theorem}

\begin{proof}
    Considérons la fonction
    \begin{equation}
        \tau(x)=f(x)-\big( \frac{ f(b)-f(a) }{ b-a }x + f(a) - a\frac{ f(b)-f(a) }{ b-a } \big),
    \end{equation}
    c'est à dire la fonction qui donne la distance entre $f$ et le segment de droite qui lie $(a,f(a))$ à $(b,f(b))$. Par construction, $\tau(a)-\tau(b)=0$, donc le théorème de Rolle s'applique à $\tau$ pour laquelle il existe donc un $c\in]a,b[$ tel que $\tau'(c)=0$.

    En utilisant les règles de dérivation, nous trouvons que la dérivée de $\tau$ vaut
    \begin{equation}
        \tau'(x)= f'(x)-\frac{ f(b)-f(a) }{ b-a },
    \end{equation}
    donc dire que $\tau'(c)=0$ revient à dire que $f(b)-f(a)=(b-a)f'(c)$, ce qu'il fallait démontrer.

    La majoration est une conséquence immédiate, parce que le supremum de \( | f'(x) |\) est forcément plus grand que \( | f'(c) |\).
\end{proof}

\begin{corollary}
Soit $f$ une fonction dérivable sur $[a,b]$ telle que $f'(x) = 0$ pour tout $x \in [a,b]$. Alors $f$ est constante sur $[a,b]$.
\end{corollary}

\begin{proof}
    Si $f$ n'était pas constante sur $[a,b]$, il existerait un $x_1\in ]a,b[$ tel que $f(a)\neq f(x_1)$, et dans ce cas, il existerait un $c\in]a,x_1[$ tel que 
    \begin{equation}
        f'(c)=\frac{ f(x_1)-f(a) }{ x_1-a }\neq 0,
    \end{equation}
    ce qui contredirait les hypothèses.
\end{proof}

\begin{corollary}   \label{CorNErEgcQ}
    Soient $f$ et $g$, deux fonctions dérivables sur $[a,b]$ telles que
    \begin{equation}
        f'(x) = g'(x)
    \end{equation}
    pour tout $x \in [a,b]$. Alors existe un réel $C$ tel que $f (x) = g (x) + C$ pour tout $x\in [a,b]$.
\end{corollary}

\begin{proof}
    Considérons la fonction $h(x)=f(x)-g(x)$, dont la dérivée est, par hypothèse, nulle. L'annulation de la dérivée entraine par le corollaire \ref{CorNErEgcQ} que $h$ est  constante. Si $h(x)=C$, alors $f(x)=g(x)+C$, ce qu'il fallait prouver.
\end{proof}

\begin{definition}  \label{DefXVMVooWhsfuI}
    Soit \( I\) un intervalle ouvert de \( \eR\) et une fonction \( f\colon I\to \eR\). La fonction \( F\colon I\to \eR\) est une \defe{primitive}{primitive!fonction} de \( f\) si \( F\) est dérivable sur \( I\) et si \( F'(x)=f(x)\) pour tout \( x\) dans \( I\).
\end{definition}

Exprimé en termes des primitives, le corollaire \ref{CorNErEgcQ} signifie que
\begin{corollary}  \label{CorZeroCst}
    Si $F$ et $G$ sont deux primitives de la même fonction $f$ sur un intervalle, alors il existe une constante $C$ pour laquelle $F(x)=G(x)+C$.
\end{corollary}
Cela signifie qu'il n'y a, en réalité, pas des milliards de primitives différentes à une fonction. Il y en a essentiellement une seule, et puis les autres, ce sont juste les mêmes, mais décalées d'une constante.

\begin{remark}
    L'hypothèse de se limiter à un intervalle est importante parce que si on considère la fonction sur deux intervalles disjoints, nous pouvons choisir la constante indépendamment dans l'un et dans l'autre. Par exemple la fonction
    \begin{equation}
        F(x)=\begin{cases}
            \ln(x)+1    &   \text{si \( x>0\)}\\
            \ln(x)-7    &    \text{si \( x<0\)}
        \end{cases}
    \end{equation}
    est une primitive de \( \frac{1}{ x }\) sur l'ensemble \( \eR\setminus\{ 0 \}\).

    Certains ne s'en privent pas. Le logiciel \href{ http://sagemath.org }{ Sage } par exemple fait ceci :
    \begin{verbatim}
sage: f(x)=1/x
sage: F=f.integrate(x)
sage: A=F(x)-F(-x)
sage: A.full_simplify()
I*pi
    \end{verbatim}
    En réalité lorsque \( x>0\), Sage définit \( \ln(-x)=\ln(x)+i\pi\). Cela a une certaine logique parce que \( \ln(-1)=i\pi\) (du fait que \(  e^{i\pi}=-1\)), mais si on ne le sait pas, ça peut étonner.
\end{remark}

\begin{normaltext}
    Il existe plusieurs primitives à une fonction donnée. En physique, la constante arbitraire est souvent fixée par une condition initiale, comme nous le verrons dans la section \ref{SecMRUAsecondeGGdQoT}.
\end{normaltext}


%+++++++++++++++++++++++++++++++++++++++++++++++++++++++++++++++++++++++++++++++++++++++++++++++++++++++++++++++++++++++++++
\section{Dérivée directionnelle}
%+++++++++++++++++++++++++++++++++++++++++++++++++++++++++++++++++++++++++++++++++++++++++++++++++++++++++++++++++++++++++++

Nous sommes capables de dériver une fonction de deux variables $f(x,y)$ par rapport à $x$ et par rapport à $y$. C'est à dire que nous sommes capables de donner la variation de la fonction lorsqu'on bouge le long des axes horizontal et vertical. Il est évidemment souhaitable de parler de la variation de la fonction lorsqu'on se déplace le long d'autre droites.

Soit donc $u=\begin{pmatrix}
    u_1    \\ 
    u_2    
\end{pmatrix}$ un vecteur unitaire (c'est à dire $u_1^2+u_2^2=1$), et considérons la fonction de une variable
\begin{equation}
    \begin{aligned}
        \varphi\colon \eR&\to \eR \\
        t&\mapsto f(a+tu_1,b+tu_2). 
    \end{aligned}
\end{equation}
La fonction $\varphi$ n'est rien d'autre que la fonction $f$ vue le long de la droite de direction donnée par le vecteur $u$. Nous pouvons aussi l'écrire $\varphi(t)=f(p+tu)$.

Soit $f\colon \eR^2\to \eR$ une fonction de deux variables et soit $(a,b)\in\eR^2$. La façon la plus naturelle de définir une dérivée à deux variables est de considérer les \defe{dérivées partielles}{dérivée!partielle} définies par
\begin{equation}
    \begin{aligned}[]
        \frac{ \partial f }{ \partial x }(a,b)&=\lim_{x\to a} \frac{ f(x,b)-f(a,b) }{ x-a }\\
        \frac{ \partial f }{ \partial y }(a,b)&=\lim_{y\to b} \frac{ f(a,y)-f(a,b) }{y-b}.
    \end{aligned}
\end{equation}
Ces nombres représentent la façon dont le nombre $f(x,y)$ varie lorsque soit seul $x$ varie soit seul $y$ varie. Les dérivées partielles se calculent de la même façon que les dérivées normales. Pour calculer $\partial_xf$, on fait «comme si» $y$ était une constante, et pour calculer $\partial_yf$, on fait comme si $x$ était une constante.

%---------------------------------------------------------------------------------------------------------------------------
                    \subsection{Dérivée partielle et directionnelles}
%---------------------------------------------------------------------------------------------------------------------------

Soit une fonction $f:A\subset \mathbb{R}^n \rightarrow \mathbb{R}^m$. Si $n\neq 1$, la notion de \emph{dérivée} de la fonction $f$ n'a plus de sens puisqu'on ne peut plus parler de pente de \emph{la} tangente au graphe de $f$ en un point. On introduit alors quelques notions qui feront, en dimension quelconque, le même travail que la dérivée en dimension un : les dérivées directionnelles et la différentielle. Nous allons voir qu'en dimension un, la différentielle coïncide avec la dérivée.


\begin{definition} 
    Soit un point $a \in int\,A$ et un vecteur $u \in \mathbb{R}^n$ avec $\| u \| =1$. La dérivée de $f$ au point $a$ dans la direction $u$ est donnée par la limite suivante, si elle existe 
    \begin{equation}
        \frac{\partial f}{\partial u}(a) = \lim_{t\rightarrow 0}\frac{f(a+tu) - f(a)}{t}
    \end{equation}
\end{definition}

Géométriquement, il s'agit du taux de variation instantané de $f$ en $a$ dans la direction du vecteur $u$, c'est-à-dire de la pente de la tangente dans la direction du vecteur $u$ au graphe de $f$ au point $(a, f(a))$.

\begin{remark}
On peut reformuler la définition en écrivant $x = a + u$, on obtient~:
\begin{equation}
    \limite[u\neq 0]{u}{0} \frac{f(a+u)-f(a)-T(u)}{\norme{u}} = 0.
\end{equation}
\end{remark}

\begin{remark}
Pourquoi avons-nous posé la condition $\| u \|=1$ ? Le but de la dérivée directionnelle dans la direction $u$ est de savoir à quelle vitesse la fonction monte lorsque l'on se déplace en suivant la direction $u$. Cette information n'aura un caractère \og objectif\fg{} que si l'on avance à une vitesse donnée. En effet, si on se déplace deux fois plus vite, la fonction montera deux fois plus vite. Par convention, nous demandons donc d'avancer à vitesse $1$.
\end{remark}

\subsubsection*{Cas particulier où $n=2$:} $a = (a_1, a_2)$, $u =
(u_1,u_2)$ et
$$\frac{\partial f}{\partial u}(a_1, a_2) = \lim_{t\rightarrow
0}\frac{f(a_1+tu_1,a_2+tu_2) - f(a_1, a_2)}{t}$$

Un cas particulier des dérivées directionnelles est la dérivée partielle. Si nous considérons la base canonique $e_i$ de $\eR^n$, nous notons
\begin{equation}
    \frac{ \partial f }{ \partial x_i }=\frac{ \partial f }{ \partial e_i }.
\end{equation}
Dans le cas d'une fonction à deux variables, nous avons donc les deux dérivées partielles
\begin{equation}
    \begin{aligned}[]
        \frac{ \partial f }{ \partial x }(a)&&\text{et}&&\frac{ \partial f }{ \partial y }(a)
    \end{aligned}
\end{equation}
qui correspondent aux dérivées directionnelles dans les directions des axes. Ces deux nombres représentent de combien la fonction $f$ monte lorsqu'on part de $a$ en se déplaçant dans le sens des axes $X$ et $Y$.

%///////////////////////////////////////////////////////////////////////////////////////////////////////////////////////////
                    \subsubsection{Quelques propriétés et notations}
%///////////////////////////////////////////////////////////////////////////////////////////////////////////////////////////

\begin{enumerate}
\item
 $\forall \alpha \in \mathbb{R}$,
si $v = \alpha\,u$, alors $\frac{\partial f}{\partial v}(a) =
\alpha\,\frac{\partial f}{\partial u}(a)$.
\item Si on prend $u=e_j$ le $j$ème vecteur de la base canonique de
$\mathbb{R}^n$, alors
$$\frac{\partial f}{\partial e_j}(a) = \frac{\partial f}{\partial
x_j}(a)$$ c'est-à-dire que la dérivée de $f$ au point $a$ dans la
direction $e_j$ est la dérivée partielle de $f$ par rapport à sa
$j$ème variable.

\item 
Une fonction peut être dérivable dans certaines directions
mais pas dans d'autres (rappelez vous que si la limite à droite est
différente de la limite à gauche, la limite n'existe pas). 

\item
Même si une fonction est dérivable en un point dans toutes les
directions, on n'est pas sûr qu'elle soit continue en ce point. La
dérivabilité directionnelle n'est donc pas une notion suffisante
pour assurer la continuité. C'est pourquoi on introduit le concept
de \emph{différentiabilité}. 
\end{enumerate}

%++++++++++++++++++++++++++++++++++++++++++++++++++++++++++++++++++++++++++++++++++++++++
\section{Dérivée suivant un vecteur}		\label{SecDerDirect}
%++++++++++++++++++++++++++++++++++++++++++++++++++++++++++++++++++++++++++++++++++++++++
\begin{definition}
Soit $f$ une application de $U\subset\eR^m$ dans $\eR$, $a$ un point dans $U$ et $v$ un vecteur de $\eR^m$. On dit que $f$ admet une \defe{dérivée suivant le vecteur $v$ au point $a$}{dérivée!directionnelle} si la fonction $t\mapsto f(a+tv)$ admet une dérivée en $t=0$. La  dérivée de $f$ suivant le vecteur $v$ au point $a$ est alors cette dérivée, et $f$ est dite dérivable suivant $v$ en $a$,
\[
\partial_v f(a)=\lim_{
  \begin{subarray}{l}
    t\to 0\\ t\neq 0 
  \end{subarray}
 }\frac{f(a+tv)-f(a)}{t}.
\] 
\end{definition}

\begin{definition}
  La fonction $f:U\subset\eR^m\to \eR^n$ de composantes $(f_1,\ldots, f_n)$, est dite \defe{dérivable suivant $v$ au point $a$}{} si toute ses composante $f_i$, $i=1,\ldots, n$ sont dérivables suivant $v$ au point $a$. Dans ce cas, nous écrivons
  \begin{equation}
	\partial_v f(a)=\left(\partial_v f_1(a), \ldots, \partial_v f_n(a)\right)^T.
  \end{equation}
\end{definition}
On parle aussi souvent de dérivé \defe{dans la direction}{} du vecteur $v$. Une \defe{direction}{direction} dans $\eR^m$ est un vecteur de norme $1$. Tant que $u$ est un élément non nul de $\eR^m$, nous pouvons parler de la direction de $u$.

\begin{proposition}
Soit $u$ un vecteur de norme $1$ dans $\eR^m$ et soit $v=\lambda u$, avec $\lambda$ dans $\eR$. La fonction $f$ est dérivable suivant $v$ au point $a$ si et seulement si $f$ est dérivable suivant $u$ au point $a$, en outre  
\[
\partial_v f(a)=\lambda\partial_u f(a).
\]
\end{proposition}
\begin{proof}
  \begin{equation}
    \begin{aligned}
  \partial_v f(a)=&\lim_{\begin{subarray}{l}
     t\to 0\\ t\neq 0 
    \end{subarray}}\frac{f(a+tv)-f(a)}{t}=\lim_{\begin{subarray}{l}
     t\to 0\\ t\neq 0 
    \end{subarray}}\frac{f(a+t\lambda u)-f(a)}{t}=\\
&=\lambda \lim_{\begin{subarray}{l}
    t\to 0\\ t\neq 0 
  \end{subarray}}\frac{f(a+t\lambda u)-f(a)}{\lambda t}=\lambda \partial_u f(a).    
    \end{aligned}
  \end{equation}
\end{proof}
\begin{definition}
Soit $f$ une application de $U\subset\eR^m$ dans $\eR$. On appelle \defe{dérivées partielles de $f$ au point $a$}{dérivée!partielle} les dérivées de $f$ suivant les vecteurs de base $e_1,\ldots,e_m $ au point $a$, si elles existent.
\end{definition}
Si $m=2,3$ on peut utiliser la notation $f_x$, $\partial_x$  ou $\partial_1$ pour la dérivée partielle suivant $e_1$, $f_y$, $\partial_y$  ou $\partial_2$  pour la dérivée partielle suivant $e_2$ et $f_z$,  $\partial_z$  ou $\partial_3$  pour la dérivée partielle suivant $e_3$. En général, nous écrivons $\partial_i$ pour noter la la dérivée partielle suivant $e_i$.  

\begin{example}
Les dérivées partielles de la fonction $f(x,y)=xy^3+\sin y$ au point $(0,\pi)$ sont 
\[
\partial_xf(0,\pi)=\frac{ \partial f }{ \partial x }(0,\pi)=f_x(0,\pi)=\lim_{\begin{subarray}{l}
    t\to 0\\ t\neq 0 
  \end{subarray}} \frac{(t\pi^3+\sin \pi)-(\sin \pi)}{t}= \pi^3,
\] 
\[
\partial_yf(0,\pi)=\frac{ \partial f }{ \partial y }(0,\pi)=f_y(0,\pi)=\lim_{\begin{subarray}{l}
    t\to 0\\ t\neq 0 
  \end{subarray}} \frac{0(\pi+t)^3+\sin (t+\pi)-0\cdot \pi^3}{t}= \cos \pi=-1,
\]   
\end{example}
La fonction d'une seule variable qu'on obtient à partir de $f$ en fixant les $p-1$ variables  $x_1,\ldots, x_{i-1}, x_{i+1}, \ldots, x_p$ et qui associe à $x_i$ la valeur $f(x_1,\ldots, x_{i-1}, x_i, x_{i+1}, \ldots, x_p)$, est appelée $x_i$-ème \defe{section}{section} de $f$ en $x_1,\ldots, x_{i-1}, x_{i+1}, \ldots, x_p$. L'$i$-ème dérivée partielle de $f$ au point $a=(x_1,\ldots,x_m)$ est la dérivée de l'$i$-ème section de $f$ au point $x_i$. En pratique, pour calculer les dérivées partielles d'une fonction on fait une dérivation par rapport à la variable choisie en considérant les  autres variables comme des constantes.

\begin{example}
	Considérons la fonction $f(x,y)=2xy^2$. Lorsque nous calculons $\partial_xf(x,y)$, nous faisons comme si $y$ était constant. Nous avons donc $\partial_xf(x,y)=2y^2$. Par contre lors du calcul de $\partial_yf(x,y)$, nous prenons $x$ comme une constante. La dérivée de $y^2$ par rapport à $y$ est évidement $2y$, et par conséquent, $\partial_yf(x,y)=4xy$.
\end{example}

\begin{example}
  La fonction $f(x,y)=x^y$ est dérivable au point $(1,2)$ et on a
\[
\partial_x f(1,2)=(yx^{y-1})_{(x,y)=(1,2)}=2,
\]
\[
\partial_y f(1,2)=\partial_y\left(e^{y\ln x}\right)_{(x,y)=(1,2)}=\left(\ln x e^{y\ln x}\right)_{(x,y)=(1,2)}=\ln\big( 1- e^{2\ln(1)} \big)=0.
\]
\end{example}
\begin{definition}
  Soit $f$ une application de $U\subset\eR^m$ dans $\eR$ et $u$ un vecteur de $\eR^m$. La fonction $f$ est \defe{dérivable sur $U$ suivant le vecteur $u$}{}, si $f$ est dérivable  suivant le vecteur $u$ en tout point de $U$.
\end{definition} 

Pour les fonctions d'une seule variable la dérivabilité en un point $a$ implique la continuité en $a$. Cela n'est pas vrai pour les fonctions de plusieurs variables : il existe des fonction $f$  qui sont dérivables suivant tout vecteur au point $a$ sans pour autant être continue en $a$. 

  \begin{example}
    Considérons la fonction $f:\eR^2\to \eR$ 
    \begin{equation}
      f(x,y)=\left\{
      \begin{array}{ll}
        \frac{x^2y}{x^4+y^2} \qquad&\textrm{si } (x,y)\neq (0,0),\\
        0     & \textrm{sinon}.
      \end{array}
      \right.
    \end{equation}
Pour voir que $f$ n'est pas continue en $(0,0)$ il suffit de calculer la limite de $f$ restreinte à la parabole $y=x^2$
\[
\lim_{x\to 0} f(x,x^2)=\frac{1}{2} \neq 0.
\] 
Pourtant la fonction $f$ est dérivable en $(0,0)$ dans toutes les directions. En effet, soit $v=(v_1,v_2)$. Si $v_2\neq 0$, alors
\[
\partial_v f(a)=\lim_{\begin{subarray}{l}
			t\to 0\\ t\neq 0 
  		\end{subarray}}
  		\frac{t^3v_1^2v^2}{t^5 v_1^4+ t^3v_2^2}=\frac{v_1^2}{v_2},
\] 
tandis que si $v_2=0$, alors la valeur de $f(tv_1, 0)$  est $0$ pour tout $t$ et $v_1$, donc la dérivée partielle de $f$ par rapport à $x$ en l'origine existe et est nulle. 
\end{example}

\begin{example}
    Pour une fonction réelle à variable réelle, la dérivabilité entraine la continuité. Il n'en va pas de même pour les fonctions à plusieurs variables, comme le montre l'exemple suivant :
    \begin{equation}
        f(x,y)=\begin{cases}
            0    &   \text{si \( x=0\)}\\
            \frac{ y }{ x }\sqrt{x^2+y^2}    &    \text{sinon.}
        \end{cases}
    \end{equation}
    Nous avons tout de suite
    \begin{equation}
        \frac{ \partial f }{ \partial y }(0,0)=0.
    \end{equation}
    De plus si \( u_x\neq 0\) nous avons
    \begin{equation}
            \frac{ \partial f }{ \partial u }(0,0)=\frac{ u_y }{ u_x }\| u \|.
    \end{equation}
    Donc toutes les dérivées directionnelles de \( f\) en \( (0,0)\) existent alors que la fonction n'y est manifestement pas continue. En effet sous forme polaire,
    \begin{equation}
        f(r,\theta)=\frac{ r\sin(\theta) }{ \cos(\theta) },
    \end{equation}
    et quelle que soit la valeur de \( r\), en prenant \( \theta\) suffisamment proche de \( \pi/2\), la fraction peut être arbitrairement grande.

    Nous verrons par la proposition \ref{diff1} que la différentiabilité d'une fonction implique sa continuité.
\end{example}

\begin{theorem}[Accroissement finis pour les dérivées suivant un vecteur]\label{val_medio_1}		\index{théorème!accroissements finis!dérivée directionnelle}
    Soit $U$ un ouvert dans $\eR^m$ et soit $f:U\to\eR^n$ une fonction. Soient $a$ et $b$ deux points distincts dans $U$, tels que le segment\footnote{Définition \ref{DefLISOooDHLQrl}.} $[a,b]$ soit contenu dans $U$. Soit $u$ le vecteur 
	\[
		u=\frac{b-a}{\|b-a\|_m}.
	\] 
	Si $\partial_u f(x)$ existe pour tout $x$ dans $[a,b]$ on a
	\[
		\|f(b)-f(a)\|_n\leq \sup_{x\in[a,b]}\|\partial_uf(x)\|_n\|b-a\|_m.
	\]
\end{theorem}

\begin{proof}
	Nous considérons la fonction $g(t)=f\big( (1-t)a-tb \big)$. Elle décrit la droite entre $a$ et $b$ parce que $g(0)=a$ et $g(1)=b$. En ce qui concerne la dérivée,
	\begin{equation}
		\begin{aligned}[]
			g'(t)&=\lim_{h\to 0} \frac{ g(t+h)-g(t) }{ h }\\
			&=\lim_{h\to 0} \frac{ f\big( (1-t-h)a-(t+h)b \big) }{ h }\\
			&=\lim_{h\to 0} \frac{ f\big( a+(t+h)(b-a) \big)-f\big( a+t(b-a) \big) }{ h }\\
			&=\frac{ \partial f }{ \partial u }\big( a+t(b-a) \big)\| b-a \|.
		\end{aligned}
	\end{equation}
	Le dernier facteur $\| b-a \|$ apparaît pour la normalisation du vecteur $u$. En effet dans la limite, il apparaît $h(b-a)$, ce qui donnerait la dérivée le long de $b-a$, tandis que $u$ vaut $(b-a)/\| b-a \|$.

	Par le théorème des accroissements finis pour $g$, il existe $t_0\in\mathopen] 0 , 1 \mathclose[$ tel que
	\begin{equation}
		g(1)=g(0)+g'(t_0)(1-0).
	\end{equation}
	Donc
	\begin{equation}
		\| g(1)-g(0) \|\leq\sup_{t_0}\| g'(t_0) \|=\sum_{t_0\in\mathopen] 0 , 1 \mathclose[}\left\| \frac{ \partial f }{ \partial u }(a+t_0(b-a)) \right\|\| b-a \|.
	\end{equation}
	Mais lorsque $t_0$ parcours $\mathopen] 0 , 1 \mathclose[$, le point $a+t_0(b-a)$ parcours le segment $\mathopen] a , b \mathclose[$, d'où le résultat.
\end{proof}

\begin{corollary}
	Dans les mêmes hypothèses, si $n=1$, alors il existe $\bar x $ dans $]a,b[$ tel que
	\[
		f(b)-f(a)=\partial_uf(\bar x)\|b-a\|_m.
	\]    
\end{corollary}

\begin{definition}
    Le nombre
    \begin{equation}
        \lim_{t\to 0} \frac{ f\big( a+tu_1,b+tu_2 \big)-f(a,b) }{ t }
    \end{equation}
    est la \defe{dérivée directionnelle}{dérivée!directionnelle} de $f$ dans la direction de $u$ au point $(a,b)$. Il sera noté
    \begin{equation}
        \frac{ \partial f }{ \partial u }(a,b),
    \end{equation}
    ou plus simplement $\partial_uf(a,b)$.
\end{definition}

Lorsque $f$ est différentiable, la dérivée directionnelle est donnée par
\begin{equation}        \label{EqDerDirnablau}
    \frac{ \partial f }{ \partial u }(p)=\nabla f(p)\cdot u.
\end{equation}

%---------------------------------------------------------------------------------------------------------------------------
\subsection{Gradient : direction de plus grande pente}
%---------------------------------------------------------------------------------------------------------------------------

Étant donné que $u$ est de norme $1$, l'inégalité de Cauchy-Schwartz donne
\begin{equation}
    \big| \nabla f(a,b)\cdot \begin{pmatrix}
        u_1    \\ 
        u_2    
    \end{pmatrix}\big|\leq \| \nabla f(a,b) \|.
\end{equation}
Donc
\begin{equation}
    -\| \nabla f(p) \|\leq \nabla f(p)\cdot u\leq\| \nabla f(p) \|.
\end{equation}
La norme de la dérivée directionnelle (qui est la valeur absolue du nombre au centre) est donc «coincée» entre $-\| \nabla f(p) \|$ et $\| \nabla f(p) \|$. Prenons par exemple
\begin{equation}
    u=\frac{ \nabla f(p) }{ \| \nabla f(p) \| }.
\end{equation}
Dans ce cas, nous avons exactement
\begin{equation}
    \nabla f(p)\cdot u=\| \nabla f(p) \|,
\end{equation}
qui est la valeur maximale que la dérivée directionnelle peut prendre.

La direction du gradient est donc la direction suivant laquelle la dérivée directionnelle est la plus grande. Pour la même raison, la dérivée directionnelle est la plus petite dans le sens opposé au gradient.

En termes bien clairs : lorsqu'on veut aller le plus vite possible au ski, on prend la direction du gradient de la piste de ski. C'est dans cette direction que ça descend le plus vite. Dans quelle direction vont les débutants ? Ils vont perpendiculairement à la pente (ce qui ennuie tout le monde, mais c'est un autre problème). Les débutants vont donc dans la direction perpendiculaire au gradient. Prenons donc $u\perp \nabla f(p)$ et calculons la dérivée directionnelle de $f$ dans la direction $u$ en utilisant la formule \ref{EqDerDirnablau} :
\begin{equation}
    \frac{ \partial f }{ \partial u }(p)=\nabla f(p)\cdot u=0
\end{equation}
parce que nous avons choisi $u\perp \nabla f(p)$. Nous voyons donc que les débutants en ski ont eu la bonne intuition que la direction dans laquelle la piste ne descend pas, c'est la direction perpendiculaire au gradient.

C'est aussi pour cela que l'on a tendance à faire du zig-zag à vélo lorsqu'on monte une pente très forte et qu'on est fatigué. C'est toujours pour cela que les routes de montagne font de longs lacets. La montée est moins rude en suivant une direction proche d'être perpendiculaire au gradient !

\begin{theorem}
    Le gradient des fonction suit à peu près les mêmes règles que les dérivées. Soient $f$ et $g$ deux fonctions différentiables. Nous avons entre autres
    \begin{enumerate}
        \item
            $\nabla(f+g)=\nabla f+\nabla g$;
        \item
            $\nabla(fg)(a,b)=g(a,b)\nabla f(a,b)+f(a,b)\nabla g(a,b)$;
        \item
            Dès que $g(a,b)\neq 0$, nous avons
            \begin{equation}
                \nabla\frac{ f }{ g }=\frac{ g(a,b)\nabla f(a,b)-f(a,b)\nabla g(a,b) }{ g(a,b)^2 }.
            \end{equation}
    \end{enumerate}
\end{theorem}


%+++++++++++++++++++++++++++++++++++++++++++++++++++++++++++++++++++++++++++++++++++++++++++++++++++++++++++++++++++++++++++ 
\section{Différentielle}
%+++++++++++++++++++++++++++++++++++++++++++++++++++++++++++++++++++++++++++++++++++++++++++++++++++++++++++++++++++++++++++

Note : pour savoir des choses sur la différentielle de \( f\colon E\to F\) avec \( E\) et \( F\) de dimension infinie, il faut aller voir la section \ref{SecLStKEmc}. Ici nous ne parlerons que de dimension finie.

%--------------------------------------------------------------------------------------------------------------------------- 
\subsection{Exemples introductifs}
%---------------------------------------------------------------------------------------------------------------------------

La notion de dérivée est associée à la recherche de la droite tangente à une courbe. Reprenons rapidement le cheminement. La dérivée de $f\colon \eR\to \eR$ au point $a$ est un nombre $f'(a)$, qui définit donc une application linéaire dont le coefficients angulaire est $f'(a)$, et que nous notons $df_a$ :
\begin{equation}
    \begin{aligned}
        df_a\colon \eR&\to \eR \\
        u&\mapsto f'(a)u. 
    \end{aligned}
\end{equation}
La droite donnée par l'équation
\begin{equation}
    y(a+u)=f'(a)u
\end{equation}
est parallèle à la tangente en $a$. Pour trouver la tangente, il suffit de la décaler de la hauteur qu'il faut. L'équation de la droite tangente au graphe de $f$ au point $\big( a,f(a) \big)$ devient
\begin{equation}        \label{EqDiffRapTgDer}
    y(x)=f(a)+f'(a)(x-a)=f(a)+df_a(x-a).
\end{equation}
Nous nous proposons de généraliser cette formule au cas de la recherche du plan tangent à une surface.

\begin{example}
    Considérons $f(x,y)=x^2y+y^2 e^{x}$. Les dérivées partielles sont
    \begin{equation}
        \begin{aligned}[]
            \frac{ \partial f }{ \partial x }&=2xy+y^2e^x\\
            \frac{ \partial f }{ \partial y }&=x^2+2ye^x.
        \end{aligned}
    \end{equation}
\end{example}

Cet exemple était l'exemple facile où tout se passe bien.

\begin{example}
    Les choses sont moins simples lorsqu'on considère la fonction suivante :
    \begin{equation}
        f(x,y)=\begin{cases}
            \frac{ xy }{ x^2+y^2 }    &   \text{si $(x,y)\neq(0,0)$}\\
            0    &    \text{si $(x,y)=(0,0)$}.
        \end{cases}
    \end{equation}
    On voit que pour tout $x$ et tout $y$, nous avons $f(x,0)=f(0,y)=0$. Donc cette fonction est nulle sur les axes horizontaux et verticaux. Nous avons en particulier
    \begin{equation}
        \begin{aligned}[]
            \frac{ \partial f }{ \partial x }(0,0)&=0\\
            \frac{ \partial f }{ \partial y }(0,0)&=0.
        \end{aligned}
    \end{equation}
    Donc ces dérivées partielles existe.

    Il n'est par contre pas question de dire que cette fonction «va bien» autour du point $(0,0)$. En effet si nous regardons sa valeur sur la droite diagonale $y=x$, nous avons
    \begin{equation}
        f(x,x)=\frac{ x^2 }{ 2x^2 }=\frac{ 1 }{2}.
    \end{equation}
    Par conséquent si nous suivons la fonction le long de la droite $y=x$, la hauteur vaut $\frac{ 1 }{2}$ en permanence, sauf juste en $(0,0)$ où la fonction fait un grand plongeon !
    \begin{verbatim}
    sage: var('x,y')
    (x, y)
    sage: f(x,y)=(x*y)/(x**2+y**2)
    sage: plot3d(f,(x,-2,2),y(-2,2))
    \end{verbatim}

    D'ailleurs elle fait un plongeon le long de toutes les droites (sauf verticale et horizontale). En effet si nous regardons la fonction le long de la droite $y=mx$, nous avons
    \begin{equation}
        f(x,mx)=\frac{ mx^2 }{ x^2+m^2x^2 }=\frac{ m }{ 1+m^2 }.
    \end{equation}
    La fonction est donc \emph{constante} sur chacune de ces droites. Il n'est donc pas question de dire que cette fonction est «dérivable» en $(0,0)$, vu qu'elle fait des grands sauts dans presque toutes les directions.
\end{example}

Nous devons donc trouver mieux que les dérivées partielles pour étudier le comportement des fonctions un peu problématiques.

%--------------------------------------------------------------------------------------------------------------------------- 
\subsection{Définition de la différentielle}
%---------------------------------------------------------------------------------------------------------------------------

Nous nous souvenons de l'équation \eqref{EqCodeDerviffxam} qui nous dit que pour une fonction d'une variable la dérivabilité signifiait qu'il existait un nombre $\ell$ et une fonction $\alpha$ tels que
\begin{equation}
    f(x)=f(a)+\ell(x-a)+(x-a)\alpha(x-a)
\end{equation}
et $\lim_{t\to 0} \alpha(t)=0$. 

En nous inspirant de cela, nous posons la définition suivante.

\begin{definition}      \label{DefDifferentiellePta}
  Soit $U$ un ouvert dans $\eR^m$ et $a$ un point dans $U$. Soit $f$ une application de $U$ dans $\eR^n$. On dit que $f$ est \defe{différentiable au point $a$}{application!différentiable} s'il existe une application linéaire $T$ de $\eR^m$ dans $\eR^n$ qui satisfait
  \begin{equation}	\label{EqCritereDefDiff}
      \lim_{\substack{h\to 0\\h\in \eR^m}}\frac{f(a+h)-f(a)-T(h)}{\|h\|_m}=0.    
  \end{equation}
  Si une telle $T$ existe on l'appelle \defe{différentielle}{différentielle} de $f$ au point $a$, et on la note $df(a)$. 
\end{definition}
Note : $df_a$ est \emph{en soi} une application $df(a)\colon \eR^m\to \eR^n$. Nous notons $df_a(u)$\nomenclature{$df_a(u)$}{Application de la différentielle de $f$ sur le vecteur $u$} la valeur de $df_a$ sur le vecteur $u\in\eR^m$.

\begin{proposition}
    Si $f$ est différentiable au point $(a,b)$, alors elle y est continue, c'est à dire que
    \begin{equation}
        \lim_{(x,y)\to(a,b)}f(x,y)=f(a,b).
    \end{equation}
\end{proposition}

\begin{proof}
    Si nous considérons la différence entre $f(x,y)$ et $f(a,b)$, nous avons (en notations matricielle) :
    \begin{equation}
        | f(X)-f(P) |=| \ell\cdot(X-P)+\| X-P \|\alpha(\| X-P \|) |.
    \end{equation}
    Le membre de droite tend évidemment vers zéro lorsque $X$ tend vers $P$.
\end{proof}

Les propositions \ref{PropExistDiffUn} et \ref{PropExistDiffDeux} vont montrer qu'en étudiant bien les dérivées partielles, nous pouvons conclure à la différentiabilité d'une fonction.
Attention cependant, nous verrons dans l'exemple \ref{Exemple0046Diff} que l'existence des dérivées directionnelles partielles ne permettait pas de conclure à la différentiabilité. 

\begin{proposition} \label{PropExistDiffUn}
    Soit $f$ une fonction de $x$ et $y$ et un point $(a,b)\in\eR^2$. Si les nombres $\partial_xf(a,b)$ et $\partial_yf(a,b)$ existent et s'il existe une fonction $\alpha\colon \eR\to \eR$ telle que
    \begin{equation}        \label{eqCritDifffabsrt}
        \begin{aligned}[]
            f(x,y)=f(a,b)&+\frac{ \partial f }{ \partial x }(a,b)(x-a)+\frac{ \partial f }{ \partial y }(a,b)(y-b)\\
            &+\| (x,y)-(a,b) \| \alpha\Big( \| (x,y)-(a,b) \| \Big)
        \end{aligned}
    \end{equation}
    et
    \begin{equation}
        \lim_{t\to 0} \alpha(t)=0,
    \end{equation}
    alors $f$ est différentiable en $(a,b)$.
\end{proposition}
Dans cet énoncé nous avons écrit $d\big( (x,y),(a,b) \big)$ la distance entre $(x,y)$ et $(a,b)$, c'est à dire le nombre $\sqrt{(x-a)^2+(y-b)^2}$. Afin d'écrire l'équation \eqref{eqCritDifffabsrt} sous forme plus compacte, nous introduisons le vecteur
\begin{equation}
    \nabla f(a,b)=\begin{pmatrix}
        \frac{ \partial f }{ \partial x }(a,b)    \\ 
        \frac{ \partial f }{ \partial y }(a,b).    
    \end{pmatrix}
\end{equation}
L'équation \eqref{eqCritDifffabsrt} devient alors
\begin{equation}        \label{EqdiffComp}
    f(X)=f(P)+\nabla f(a,b)\cdot (X-P)+\| X-P \|\alpha\big( \| X-P \| \big).
\end{equation}
Le vecteur $\nabla f(a,b)$ est appelé le \defe{gradient}{gradient} de $f$ au point $(a,b)$.

\begin{proposition} \label{PropExistDiffDeux}
    Soit $f$ une fonction de deux variables admettant des dérivées partielles $\partial_xf(x,y)$ et $\partial_yf(x,y)$ qui sont elles-mêmes des fonctions continues de $x$ et $y$. Alors la fonction $f$ est différentiable partout.
\end{proposition}

\begin{proposition}
    Si $f$ est différentiable en $(a,b)$ alors pour tout vecteur \( u\), la fonction
    \begin{equation}
        \begin{aligned}
            \varphi\colon \eR&\to \eR \\
            t&\mapsto   f(a+tu_1,b+tu_2) 
        \end{aligned}
    \end{equation}
    est dérivable en $0$ et on a
    \begin{equation}
        \varphi'(0)=\nabla f(p)\cdot u
    \end{equation}
    où nous avons noté $p=(a,b)$.
\end{proposition}

\begin{proof}
    Récrivons la formule \eqref{EqdiffComp} sous la forme
    \begin{equation}
        f(x)=f(p)+\nabla f(p)\cdot (x-p)+\| x-p \|\alpha(\| x-p \|).
    \end{equation}
    Cela étant vrai pour tout $x$, nous l'écrivons en particulier pour $x=p+tu$ où $t$ est un réel et $u$ est le vecteur unitaire choisi. Nous avons donc
    \begin{equation}
        f(p+tu)=f(p)+t\nabla f(p)\cdot u+\| tu \|\alpha(\| tu \|).
    \end{equation}
    En utilisant le fait que $u$ est unitaire, $\| tu \|=| t |\| u \|=| t |$. La dérivée de $\varphi$ en $0$ est alors donnée par
    \begin{equation}
        \lim_{t\to 0} \frac{ f(p+tu)-f(p) }{ t }=\lim_{t\to 0} \nabla f(p)\cdot u+\alpha(| t |).    
    \end{equation}
    Lorsque nous prenons la limite, le membre de gauche devient $\varphi'(0)$ tandis que dans le membre de droite, le second terme disparaît. Nous avons finalement
    \begin{equation}
        \varphi'(0)=\nabla f(p)\cdot u
    \end{equation}
\end{proof}

Le théorème suivant reprend pas principales propriétés d'une fonction différentiable.
\begin{theorem}     \label{ThoRapPropDiffSi}
Si $f$ est différentiable en $a\in\eR^n$, alors
\begin{enumerate}
\item $f$ est continue en $a$.

\item  Toute les dérivées directionnelles $\partial_uf(a)$ existent et nous avons l'égalité
\begin{equation}        \label{EqDiffPartRap}
    \begin{aligned}
        df_a\colon \eR^n&\to \eR^m \\
        u&\mapsto df_a(u)=\frac{ \partial f }{ \partial u }(a)=\sum_i \frac{ \partial f }{ \partial x_i }u^i,
    \end{aligned}
\end{equation}
si les $u^i$ sont les composantes de $u$ dans la base canonique de $\eR^n$.

La différentielle de $f$ en $a$ envoie donc un vecteur $u$ sur la dérivée directionnelle de $f$ au point $a$ dans la direction $u$. 

\item\label{ItemThoDiffSiLin} L'application $df_a$ est une application linéaire.
\end{enumerate}
\end{theorem}
Le point \ref{ItemThoDiffSiLin} est évidement contenu dans la définition de la différentielle, mais c'est bien de la remettre en toute lettres. En regard avec la formule \eqref{EqDiffPartRap}, elle dit que $\partial_uf(a)$ est linéaire par rapport à $u$.


\begin{proposition}\label{diff1}
    Si $f$ est différentiable au point $a$ alors
    \begin{enumerate}
        \item
            elle est continue en \( a\),
        \item
            elle admet une dérivée dans toutes les directions de \( \eR^m\),
        \item
            si $T\in\aL(\eR^m,\eR^n)$ est la différentielle de $f$ au point $a$, alors
            \begin{equation}
                T(u)=df_a(u)=\partial_u f(a). 
            \end{equation}
    \end{enumerate}
\end{proposition}
\index{application!différentiable}

La dernière égalité sera de temps en temps utilisée sous la forme
\begin{equation}    \label{EqOWQSoMA}
    df_a(u)=\Dsdd{ f(a+tu) }{t}{0}.
\end{equation}

\begin{proof}
  La limite
\[
\lim_{h\to 0_m}\frac{\|f(a+h)-f(a)-T(h)\|_n}{\|h\|_m}=0,
\]
implique que
 \[
\lim_{h\to 0_m}\|f(a+h)-f(a)-T(h)\|_n=0.
\]
Comme $T$ est dans $\mathcal{L}(\eR^m,\eR^n)$, on a $\lim_{h\to 0}T(h)=0$, d'où la continuité de $f$ au point $a$.

Si $u$ est un vecteur non nul, la différentiabilité de $f$ au point $a$ implique
\[
\lim_{t\to 0}\frac{\|f(a+tu)-f(a)-T(tu)\|_n}{\|tu\|_m}=0,
\]
par la linéarité de $T$ et par l'égalité $\|tu\|_m=|t|\|u\|_m$ on obtient
\[
\lim_{t\to 0}\frac{f(a+tu)-f(a)}{|t|}= T(u).
\]
Donc $f$ est dérivable suivant le vecteur $u$ et $\partial_uf(a)=T(u)=df_a(u)$.
\end{proof}

Cette proposition est à ne pas confondre avec la proposition \ref{Diff_totale} qui dira que si les dérivées partielles \emph{sont continues} sur un voisinage de $a$, alors $f$ est différentiables en $a$.

Le lemme suivant regroupe quelques égalités avec lesquelles nous allons souvent travailler. Il explique comment sont liés les dérivées directionnelles, les dérivées partielles et la différentielle.
\begin{lemma}		\label{LemdfaSurLesPartielles}
	Si $f\colon \eR^m\to \eR^n$ est une fonction différentiable, alors
	\begin{equation}
        df_a(u)=\frac{ \partial f }{ \partial u }(a)=\Dsdd{ f(a+tu) }{t}{0}=\sum_{i=1}^mu_i\frac{ \partial f }{ \partial x_i }(a)=\nabla f(a)\cdot u
	\end{equation}
	pour tout vecteur $u\in\eR^m$
\end{lemma}

\begin{proof}
La première égalité est la proposition \ref{diff1}, et la seconde est seulement la définition de la dérivée directionnelle avec des notations un peu plus snob. En particulier nous avons
\begin{equation}
    df_a(e_i)=\frac{ \partial f }{ \partial x_i }(a).
\end{equation}
Pour le reste c'est la linéarité de la différentielle qui joue : le vecteur $u$ peut être écrit de façon unique comme combinaison linéaire des vecteurs de base 
\[
u=\sum_{i=1}^{m}u_i e_i, \qquad  u_i\in\eR,\, \forall i\in\{1,\ldots, m\}.
\]
Alors, la linéarité de $df_a$ nous donne
\begin{equation}
     df_a(u)= df_a\left(\sum_{i=1}^{m}u_i e_i\right)
=\sum_{i=1}^{m}u_i \left(df_ae_i\right)
=\sum_{i=1}^{m}u_i \frac{ \partial f }{ \partial x_i }(a).
 \end{equation}
Le lien avec le gradient est la définition du produit scalaire \eqref{DefYNWUFc}.
\end{proof}

%--------------------------------------------------------------------------------------------------------------------------- 
\subsection{Unicité de la différentielle}
%---------------------------------------------------------------------------------------------------------------------------

\begin{corollary}
	Soit $f$ une application de $U$ dans $\eR^n$ différentiable au point $a$ dans $U$. Alors l'application $df(a)$, différentielle de $f$ au point $a$, est unique, c'est à dire que si $T_1$ et $T_2$ sont deux applications vérifiant la condition \eqref{EqCritereDefDiff}, alors $T_1=T_2$.
\end{corollary}

\begin{proof}
    Pour tout vecteur $u$, la proposition \ref{diff1} implique que $T_1(u)=T_2(u)=\partial_uf(a)$.
\end{proof}

\begin{corollary}
Soit  $f:\eR^m\to \eR^n$ une fonction.  La dérivabilité de $f$ au point  $a$ suivant tout vecteur de $\eR^m$ est une condition nécessaire pour la différentiabilité de $f$ en $a$.
\end{corollary}

%--------------------------------------------------------------------------------------------------------------------------- 
\subsection{Cas particuliers}
%---------------------------------------------------------------------------------------------------------------------------

\begin{description}
    \item $n=1$:
$f: \mathbb{R}\rightarrow \mathbb{R}$ est dérivable en $a$ si et
seulement si $f$ est différentiable en $a$ et
$$df_a:\mathbb{R}\rightarrow \mathbb{R}: x \mapsto df_a(x) =
f'(a)\,.\,x$$ 
\item
    $n=2$: $f$ est différentiable en $a =(a_1, a_2)$
si et seulement si
$$\lim_{(v_1,v_2)\rightarrow (0,0)} \frac{f(a_1+v_1, a_2+v_2) - f(a_1,a_2) - [ \frac{\partial f}{\partial x}(a)\,v_1+
\frac{\partial f}{\partial y}(a)\,v_2]}{\sqrt{v_1^2+v_2^2}} = 0
$$\end{description}\vspace{0.3cm}


Parmi les vecteurs $u \in \mathbb{R}^n$, un vecteur d'origine $(a, f(a))$ se distingue des autres: le vecteur gradient de $f$ en $a$ donnant la direction de plus grande pente de $f$ en $a$.

%--------------------------------------------------------------------------------------------------------------------------- 
\subsection{Calcul de valeurs approchées}
%---------------------------------------------------------------------------------------------------------------------------

Si nous remplaçons les accroissements $x-a$ et $y-b$ par $h$ et $k$, le critère de différentiabilité s'écrit
\begin{equation}
    \begin{aligned}[]
        f(a+h,b+k)=f(a,b)+\frac{ \partial f }{ \partial x }(a,b)h&+\frac{ \partial f }{ \partial y }(a,b)k\\
        &+\sqrt{h^2+k^2}\alpha\big( \sqrt{h^2+k^2} \big).
    \end{aligned}
\end{equation}
Le dernier terme du membre de droite tend vers zéro à une vitesse double lorsque $h$ et $k$ tendent vers zéro : d'une part parce que $\sqrt{h^2+k^2}$ tend vers zéro et d'autre part parce que $\alpha\big( \sqrt{h^2+k^2} \big)$ tend vers zéro. Nous avons donc la «bonne» approximation
\begin{equation}        \label{EqFormApproxfxyab}
    f(x,y)\simeq f(a,b)+\frac{ \partial f }{ \partial x }(a,b)(x-a)+\frac{ \partial f }{ \partial y }(a,b)(y-b).
\end{equation}
lorsque $(x,y)$ n'est pas trop loin de $(a,b)$. Cette expression est évidemment une généralisation immédiate de l'équation \eqref{EqfxdxSimeqfxfpx}. Elle exprime que l'on peut obtenir des information sur la valeur d'une fonction en $(x,y)$ si on peut calculer la fonction et ses dérivées en un point $(a,b)$ non loin de $(x,y)$.

Cette formule peut aussi être vue sous la forme suivante, plus pratique dans certains calculs :
\begin{equation}        \label{EqFormApproxfxyabDF}
    f(a+\Delta x,b+\Delta y)\simeq f(a,b)+\Delta x\frac{ \partial f }{ \partial x }(a,b)+\Delta y\frac{ \partial f }{ \partial y }(a,b).
\end{equation}

\begin{example}
    Prenons la fonction $f(x,y)=\cos(x)\sin(y)$ et calculons une approximation de
    \begin{equation}
        f\big( \frac{ \pi }{ 3 }+0.01,\frac{ \pi }{ 2 }+0.03 \big).
    \end{equation}
    D'abord les dérivées partielles sont
    \begin{equation}
        \begin{aligned}[]
            \frac{ \partial f }{ \partial x }(x,y)=-\sin(x)\sin(y)\\
            \frac{ \partial f }{ \partial y }(x,y)=\cos(x)\cos(y).
        \end{aligned}
    \end{equation}
    Nous allons utiliser l'approximation
    \begin{equation}
        f\big( \frac{ \pi }{ 3 }+0.01,\frac{ \pi }{ 2 }+0.03 \big)\simeq f\big( \frac{ \pi }{ 3 },\frac{ \pi }{2} \big)+0.01\frac{ \partial f }{ \partial x }\big( \frac{ \pi }{ 3 },\frac{ \pi }{2} \big)+0.03\frac{ \partial f }{ \partial y }\big( \frac{ \pi }{ 3 },\frac{ \pi }{2} \big).
    \end{equation}
    Nous avons
    \begin{equation}
        \begin{aligned}[]
            \frac{ \partial f }{ \partial x }\big( \frac{ \pi }{ 3 },\frac{ \pi }{2} \big)&=-\sin\frac{ \pi }{ 3 }\sin\frac{ \pi }{ 2 }=-\frac{ \sqrt{3} }{2}\\
            \frac{ \partial f }{ \partial y }\big( \frac{ \pi }{ 3 },\frac{ \pi }{2} \big)&=\cos\frac{ \pi }{ 3 }\cos\frac{ \pi }{ 2 }=0.
        \end{aligned}
    \end{equation}
    Par conséquent
    \begin{equation}
        f\big( \frac{ \pi }{ 3 }+0.01,\frac{ \pi }{ 2 }+0.03 \big)\simeq \frac{ 1 }{2}-0.01\frac{ \sqrt{3} }{2}=\frac{ 1 }{2}-\frac{ \sqrt{3} }{ 200 }. 
    \end{equation}
    
    \begin{verbatim}
sage: var('x,y')
(x, y)
sage: f(x,y)=cos(x)*sin(y)
sage: a=f(pi/3+0.01,pi/2+0.03)
sage: numerical_approx(a)
0.491093815387986
sage: b=1/2-sqrt(3)/200
sage: numerical_approx(b)
0.491339745962156
sage: numerical_approx(a-b)
-0.000245930574169814
    \end{verbatim}
    Cela fait une erreur de l'ordre du dix millième. 
    
\end{example}

\begin{remark}
    Les esprits les plus critiques diront que cette vérification pas Sage n'en est pas une parce que Sage a certainement utilisé un algorithme d'approximation qui se base sur la même idée que ce que nous venons de faire, et que par conséquent le fait qu'il obtienne le même résultat que nous est un peu tautologique. 
    
    Ils n'auront pas tord. Cependant, le code source de Sage est disponible publiquement\footnote{Voir \url{http://www.sagemath.org}}; vous pouvez aller le lire et vérifier qu'il y a effectivement une \emph{preuve} que le résultat fourni par Sage possède une bonne dizaine de décimales correctes. 
    
    Cette disponibilité publique du code source est une des nombreuses différence fondamentale entre Sage et votre calculatrice\footnote{et les autres logiciels de type fenêtre, pomme ou feuille d'érable.}. Dois-je vous rappeler qu'un des principes fondamentaux de l'éthique scientifique est que les résultats et les méthodes utilisées doivent être absolument ouverts à la vérification et à la critique de tous ?
\end{remark}

\begin{equation}        \label{Eqdfpunfpdu}
    df_p(u)=\nabla f(p)\cdot u.
\end{equation}

\begin{definition}      \label{DefDifferentiablFnRn}
Soit un point $a \in int\,A$. La fonction $f$ est \defe{différentiable}{différentiable} au point $a$ s'il existe une application linéaire $df_a\colon \eR^n\to \eR^m$ telle que 
\begin{equation}        \label{EqDefDiffableT}
    \lim_{x\to a} \frac{f(x) - f(a) - df_a (x-a)}{\|x-a\|}=0.
\end{equation}
\end{definition}

Si $f$ est différentiable en $a$, l'application $df_a$ est appelée la différentielle de $f$ en $a$. Voyons comment cette application linéaire agit sur les vecteurs de $\mathbb{R}^n$.

La notion de dérivée partielle (ou de dérivée suivant un vecteur) pour une fonction de plusieurs variables n'est pas une  généralisation de la notion de dérivée en une variable d'espace. En fait, du point de vue géométrique, la dérivée de la fonction $g:\eR\to\eR$ au point $a$ est la pente de la ligne droite tangente au graphe de $g$ au point $(a, g(a))$. Cette ligne, d'équation $r(x)=g'(a)x+g(a)$, est la meilleure approximation affine du graphe de $g$ au point $a$, comme à la figure \ref{LabelFigTangentSegment}.
\newcommand{\CaptionFigTangentSegment}{Tangentes au graphe d'une fonction d'une variable}
\input{pictures_tex/Fig_TangentSegment.pstricks}

Le graphe d'une fonction $f$ de $\eR^2$ dans $\eR$ est une surface de deux paramètres dans $\eR^3$. Si l'approximation affine d'une telle surface au point $(x,y,f(x,y))$ existe, alors elle est un plan tangent. En dimension plus haute, le graphe de la fonction $f:\eR^m\to\eR$ est une surface de $m$ paramètres dans $\eR^{m+1}$ et son approximation affine (si elle existe) est un hyperplan de $\eR^m$. 

Nous allons voir que si $f$ prend ses valeurs dans $\eR^n$ l'approximation affine de $f$ au point $a$ est l'élément de $ f(a)+\mathcal{L}(\eR^m,\eR^n)$ qui ressemble le plus à $f$ au voisinage de $a$. Plus précisément, on utilise les définitions suivantes.         
\begin{definition}
  Soient $f$ et $g$ deux applications d'un ouvert $U$ de $\eR^m$ dans $\eR^n$. On dit que $g$ est \defe{tangente}{application!tangente} à $f$ au point $a\in U$ si $f(a)=g(a)$ et 
\[
\lim_{\begin{subarray}{l}
    x\to a\\ x\neq a
  \end{subarray}}\frac{\|f(x)-g(x)\|_n}{\|x-a\|_m}=0.
\]
\end{definition}
La relation de tangence est une relation d'équivalence. Nous sommes particulièrement intéressés par le cas où $f$ admet une application  affine tangente au point $a$. 


\newcommand{\CaptionFigDifferentielle}{Interprétation géométrique de la différentielle.}
\input{pictures_tex/Fig_Differentielle.pstricks}
En ce qui concerne l'interprétation géométrique, si nous regardons la figure \ref{LabelFigDifferentielle}, et d'ailleurs aussi en voyant la définition \ref{EqCritereDefDiff}, la fonction est différentiable et la différentielle est \( T\) s'il existe une fonction \( \alpha\) telle que
\begin{equation}
    f(a+u)-f(a)-T(u)=\alpha(u)
\end{equation}
où la fonction \( \alpha\) satisfait
\begin{equation}		\label{EqPresqueTa}
	\lim_{u\to 0} \frac{ \| \alpha(u)\| }{\| u \|}=0
\end{equation}
C'est cela qui fait écrire \( f(a+u)-f(a)-df_a(u)=o(\| u \|)\) à ceux qui n'ont pas peur de la notation \( o\).

La différentielle $df_a$ est donc la partie linéaire de l'application affine qui approxime au mieux la fonction $f$ autour du point $a$. La notion de différentielle est la vraie généralisation du concept de dérivée pour fonctions de plusieurs variables, en outre elle nous permet d'expliciter la relation qui associe au vecteur $u$ la dérivée $\partial_u f(a)$, pour $f$ et $a$ fixés.  

\begin{remark}
	Si on remplace les normes $\|\cdot\|_m$  et $\|\cdot\|_n$ par d'autres normes, l'existence et la valeur de la différentielle de $f$ au point $a$ ne sont pas remises en cause. En effet, soient  $\|\cdot\|_M$  une norme sur $\eR^m$ et $\|\cdot\|_N$ une norme sur $\eR^n$. Par le théorème \ref{ThoNormesEquiv}, ces normes sont équivalentes à $\| . \|_m$ et $\| . \|_m$ respectivement; il existe donc des constantes $k,\, K,\, l,\,L >0$ telles que  pour tout vecteur $u$ de $\eR^m$ et tout vecteur $v$ de $\eR^n$   
\[
k\|u\|_M\leq \|u\|_m\leq K\|u\|_M,
\]
\[
l\|v\|_N\leq \|v\|_n\leq L\|v\|_N.
\]
Les éléments de $\mathcal{L}(\eR^m, \eR^n)$ sont les mêmes et on a 
\begin{equation}
  \begin{aligned}
 & \frac{l}{K}  \frac{\|f(a+h)-f(a)-T(h)\|_N}{\|h\|_M}\leq \frac{\|f(a+h)-f(a)-T(h)\|_n}{\|h\|_m}\leq\\
&\leq\frac{L}{k} \frac{\|f(a+h)-f(a)-T(h)\|_N}{\|h\|_M}.
  \end{aligned}
\end{equation}
Il est donc possible, pour démontrer la différentiabilité ou pour calculer la différentielle, d'utiliser le critère \eqref{EqCritereDefDiff} avec une norme au choix. Parfois c'est utile.
\end{remark}

%---------------------------------------------------------------------------------------------------------------------------
                    \subsection{Prouver qu'un fonction n'est pas différentiable}
%---------------------------------------------------------------------------------------------------------------------------

Chacun des point du théorème \ref{ThoRapPropDiffSi} est en soi un critère pour montrer qu'une fonction n'est pas différentiable en un point.

%///////////////////////////////////////////////////////////////////////////////////////////////////////////////////////////
                    \subsubsection{Continuité}
%///////////////////////////////////////////////////////////////////////////////////////////////////////////////////////////


Le premier critère à vérifier est donc la continuité. Si une fonction n'est pas continue en un point, alors elle n'y sera pas différentiable. Pour rappel, la continuité en $a$ se teste en vérifiant si $\lim_{x\to a}f(x)=f(a)$.

%///////////////////////////////////////////////////////////////////////////////////////////////////////////////////////////
                    \subsubsection{Linéarité}
%///////////////////////////////////////////////////////////////////////////////////////////////////////////////////////////

Un second test est la linéarité de la dérivée directionnelle par rapport à la direction : l'application $u\mapsto\frac{ \partial f }{ \partial u }(a)$ doit être linéaire, sinon $df_a$ n'existe pas.

\begin{example}     \label{Exemple0046Diff}
Examinons la fonction
\begin{equation}
    \begin{aligned}
        f\colon \eR^2&\to \eR \\
        (x,y)&\mapsto \begin{cases}
    \frac{ xy^2 }{ x^2+y^4 }    &   \text{si $(x,y)\neq (0,0)$}\\
    0   &    \text{sinon}.
\end{cases}
    \end{aligned}
\end{equation}
Prenons $u=(u_1,u_2)$ et calculons la dérivée de $f$ dans la direction de $u$ au point~$(0,0)$ :
\begin{equation}
    \begin{aligned}[]
        \frac{ \partial f }{ \partial u }(0,0)  
            &=\lim_{t\to 0}\frac{ f(tu_1,tu_2)-f(0,0) }{ t }\\
            &=\lim_{t\to 0}\frac{1}{ t }\left( \frac{ tu_1t^2u_2 }{ t^2u_1^2+t^4u_2^4 } \right)\\
            &=\lim_{t\to 0}\left( \frac{ u_1u_2^2 }{ u_1^2+t^2u_2^4 } \right)\\
            &=\begin{cases}
    \frac{ u_2^2 }{ u_1 }   &   \text{si $u_1\neq 0$}\\
    0   &    \text{si $u_1=0$}.
\end{cases}
    \end{aligned}
\end{equation}
Cette application n'est pas linéaire par rapport à $u$. En effet, notons
\begin{equation}
    \begin{aligned}
        A\colon \eR^n&\to \eR \\
        u&\mapsto \frac{ \partial f }{ \partial u }(0,0), 
    \end{aligned}
\end{equation}
et vérifions que pour tout $u$ et $v$ dans $\eR^n$ et $\lambda\in\eR$, nous ayons $A(\lambda u)=\lambda A(u)$ et $A(u+v)=A(u)+A(v)$. Le premier fonctionne parce que
\begin{equation}
    A(\lambda u)=A(\lambda u_1,\lambda u_2)=\frac{ \lambda^2 u_2^2 }{ \lambda u_1 }=\lambda\frac{ u_2^2 }{ u_1 }=\lambda A(u).
\end{equation}
Mais nous avons par exemple
\begin{equation}
    A\big( (0,1)+(2,3) \big)=A(2,4)=\frac{ 16 }{ 2 }=8,
\end{equation}
tandis que
\begin{equation}
    A(0,1)+A(2,3)=0+\frac{ 9 }{ 2 }\neq 8.
\end{equation}
La fonction $f$ n'est donc pas différentiable en $(0,0)$, parce que la candidate différentielle, $df_{(0,0)}(u)=\frac{ \partial f }{ \partial u }(0,0)$, n'est même pas linéaire.

\end{example}

Voici une autre façon de traiter la fonction de l'exemple \ref{Exemple0046Diff}.

\begin{example} \label{ExeFHmCLII}
    La figure \ref{LabelFigFWJuNhU} représente le domaine d'une fonction $f\colon \eR^2\to \eR$, et sur chacune des parties, elle est définie différemment.
    \newcommand{\CaptionFigFWJuNhU}{La fonction de l'exemple \ref{ExeFHmCLII}.}
\input{pictures_tex/Fig_FWJuNhU.pstricks}

L'expression de $f$ est ici
\begin{equation}
  f(x,y) =
  \begin{cases}
    xy & \text{si $x < 0$ et $y > 0$}\\
    x-y & \text{si $x \geq 0$ et $y \geq 0$}\\
    x^2y & \text{si $x > 0$ et $y < 0$}\\
    x+y & \text{sinon.}
  \end{cases}
\end{equation}

On note que les deux axes forment une zone à problèmes. La zone hors
des axes est un ouvert sur lequel $f$ est différentiable car composée
de polynômes. Analysons chacun des points de la forme $(a,b)$ dans la
zone à problèmes (c'est-à-dire si $ab = 0$).

\subparagraph{Si $a = 0$ et $b > 0$} Un tel point $(0,b)$ est sur
l'axe verticale, dans la moitié supérieure. Pour calculer la limite de
$f$ en ce point, on peut restreindre notre étude au demi-plan ouvert
$y > 0$, ce qui revient à comparer la limite
\begin{equation*}
  \limite[y>0\\x\geq 0] {(x,y)} {(0,b)} f(x,y) =   \limite[y>0\\x\geq
  0] {(x,y)} {(0,b)} x-y = 0 - b = -b
\end{equation*}
avec la limite
\begin{equation*}
  \limite[y>0\\x<0] {(x,y)} {(0,b)} f(x,y) =   \limite[y>0\\x<0]
  {(x,y)} {(0,b)} xy = 0 b = 0
\end{equation*}
qui sont différentes puisque $b$ est supposé non-nul.

\conclusion $f$ n'est pas continue en un point du type $(0,b)$ avec $b
> 0$.

\subparagraph{Si $a = 0$ et $b < 0$} Un tel point $(0,b)$ est sur
l'axe verticale, dans la moitié inférieure. Pour calculer la limite de
$f$ en ce point, on peut restreindre notre étude au demi-plan ouvert
$y < 0$, ce qui revient à comparer la limite
\begin{equation*}
  \limite[y<0\\x\geq 0] {(x,y)} {(0,b)} f(x,y) =   \limite[y<0\\x\geq
  0] {(x,y)} {(0,b)} x^2 y = 0^2 b = 0
\end{equation*}
avec la limite
\begin{equation*}
  \limite[y<0\\x<0] {(x,y)} {(0,b)} f(x,y) =   \limite[y<0\\x<0]
  {(x,y)} {(0,b)} x+y = 0 + b = b
\end{equation*}
qui sont différentes puisque $b$ est supposé non-nul.

\conclusion $f$ n'est pas continue en un point du type $(0,b)$ avec $b
< 0$.

\subparagraph{Si $a > 0$ et $b = 0$} Un tel point $(a,0)$ est sur
l'axe horizontal, dans la moitié droite. Pour calculer la limite de
$f$ en ce point, on peut restreindre notre étude au demi-plan ouvert
$x > 0$, ce qui revient à comparer la limite
\begin{equation*}
  \limite[x>0\\y \geq 0] {(x,y)} {(a,0)} f(x,y) =   \limite[x>0\\y \geq
  0] {(x,y)} {(a,0)} x-y = a - 0 = a
\end{equation*}
avec la limite
\begin{equation*}
  \limite[x>0\\y < 0] {(x,y)} {(a,0)} f(x,y) =   \limite[x>0\\y < 0]
  {(x,y)} {(a,0)} x^2y = a^2 0 = 0
\end{equation*}
qui sont différentes puisque $a$ est supposé non-nul.

\conclusion $f$ n'est pas continue en un point du type $(a,0)$ avec $a
> 0$.

\subparagraph{Si $a < 0$ et $b = 0$} Un tel point $(a,0)$ est sur
l'axe horizontal, dans la moitié gauche. Pour calculer la limite de
$f$ en ce point, on peut restreindre notre étude au demi-plan ouvert
$x < 0$, ce qui revient à comparer la limite
\begin{equation*}
  \limite[x<0\\y> 0] {(x,y)} {(a,0)} f(x,y) =   \limite[x<0\\y>
  0] {(x,y)} {(a,0)} x y = a 0 = 0
\end{equation*}
avec la limite
\begin{equation*}
  \limite[x<0\\y\leq 0] {(x,y)} {(a,0)} f(x,y) =   \limite[x<0\\y\leq0]
  {(x,y)} {(a,0)} x+y = a + 0 = a
\end{equation*}
qui sont différentes puisque $a$ est supposé non-nul.

\conclusion $f$ n'est pas continue en un point du type $(a,0)$ avec $a
< 0$.

\subparagraph{Si $a = 0$ et $b = 0$} Le cas du point $(0,0)$ est
particulier, puisque il est adhérent aux quatre composantes du
domaine où la fonction est définie différemment. Pour étudier la
continuité, il faut donc étudier quatre limites. Ces limites ont déjà
été étudiées ci-dessus et valent toutes $0$, ce qui prouve la
continuité de $f$ en $(0,0)$.

En ce qui concerne la différentiabilité, on sait qu'il est nécessaire
que toutes les dérivées directionnelles existent. Calculons la dérivée
dans la direction $(0,1)$ (au point $(0,0)$)~:
\begin{equation*}
  \limite[t\neq0] t 0 \frac{f((0,0) + t(0,1)) - f(0,0)}{t} =%
  \limite[t\neq0] t 0 \frac{f(0,t)}{t} = \ldots
\end{equation*}
qu'on sépare en deux cas, car $f(0,t)$ possède une formule différente
si $t < 0$ ou si $t \geq 0$~:
\begin{equation*}
  \limite[t\neq0] t 0 \frac{f(0,t)}{t} = %
  \begin{arrowcases}
    \limite[t<0] t 0 \frac{f(0,t)}{t} = \limite[t<0] t 0 \frac{0+t}{t} = 1\\
    \limite[t\geq0] t 0 \frac{f(0,t)}{t} = \limite[t\geq0] t 0
    \frac{0-t}{t} = -1
  \end{arrowcases}
\end{equation*}
ce qui prouve que la limite n'existe pas, donc que la dérivée
directionnelle n'existe pas, et finalement que la fonction n'est pas
différentiable.

\conclusion La fonction donnée est continue hors des axes et au point
$(0,0)$, mais discontinue partout ailleurs sur les axes. Elle est
différentiable hors des axes, mais ne l'est pas sur les axes.

\end{example}

%///////////////////////////////////////////////////////////////////////////////////////////////////////////////////////////
                    \subsubsection{Cohérence des dérivées partielles et directionnelle}
%///////////////////////////////////////////////////////////////////////////////////////////////////////////////////////////

Dans la pratique, nous pouvons calculer $\partial_uf(a)$ pour une direction $u$ générale, et puis en déduire $\partial_xf$ et $\partial_yf$ comme cas particuliers en posant $u=(1,0)$ et $u=(0,1)$. Une chose incroyable, mais pourtant possible est qu'il peut arriver que
\begin{equation}
    \frac{ \partial f }{ \partial u }(a)\neq \sum_i\frac{ \partial f }{ \partial x_i }(a)u^i.
\end{equation}
Ceci se produit lorsque $f$ n'est pas différentiable en $a$. En voici un exemple.

%///////////////////////////////////////////////////////////////////////////////////////////////////////////////////////////
                    \subsubsection{Un candidat dans la définition (marche toujours)}
%///////////////////////////////////////////////////////////////////////////////////////////////////////////////////////////

Lorsqu'une fonction est donné, un candidat différentielle au point $(a_1,a_2)$ est souvent assez simple à trouver en un point :
\begin{equation}
    T(u_1,u_2)=\frac{ \partial f }{ \partial x }(a_1,a_2)u_1+\frac{ \partial f }{ \partial y }(a_1,a_2)u_2.
\end{equation}
L'application $T$ est la candidate différentielle en ce sens que si la différentielle existe, alors elle est égale à $T$. Ensuite, il faut vérifier si
\begin{equation}        \label{EqLimDefDiff}
    \lim_{(x,y)\to (a_1,a_2)} \frac{f(x,y) - f(a_1,a_2) - T\big( (x,y)-(a_1,a_2) \big)}{\| (x,y)-(a_1,a_2) \|}=0
\end{equation}
ou non. Si oui, alors la différentielle existe et $df_{(a,b)}(u)=T(u)$, sinon\footnote{y compris si la limite \eqref{EqLimDefDiff} n'existe même pas.}, la différentielle n'existe pas.

Attention : dans la ZAP, les dérivées partielles $\partial_xf$ et $\partial_yf$ ne peuvent en général pas être calculées en utilisant les règles de calcul (c'est bien pour ça que la ZAP est une zone à problèmes). Il faut d'office utiliser la définition
\begin{equation}
    \frac{ \partial f }{ \partial x }(a_1,a_2)=\lim_{t\to 0}\frac{ f(a_1+t,a_2)-f(a_1,a_2) }{ t },
\end{equation}
et la définition correspondante pour $\partial_yf$.

\subsubsection*{Conclusion}
Soient $f:A\subset \eR^n \rightarrow \eR^m$, et $a\in int\,A$. Si $f$ est différentiable en $a$, $$ (df_a (e_j))_i = d(f_i)_a(e_j) =\frac{\partial f_i}{\partial x_j}(a)= [Jac(f)_{|a}]_{ij}$$ et la matrice de l'application linéaire $df_a$ est la matrice jacobienne $m\times n$ de $f$ en $a$ notée $Jac(f)_{|a}$.
