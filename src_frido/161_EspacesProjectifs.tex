% This is part of Mes notes de mathématique
% Copyright (c) 2011-2016
%   Laurent Claessens
% See the file fdl-1.3.txt for copying conditions.

Sur les espaces projectifs : \cite{ProjRolland}.

\begin{definition}
    Soit \( E\) un espace vectoriel de dimension finie sur le corps commutatif \( \eK\). Nous définissons sur \( E\setminus\{ 0 \}\) la relation d'équivalence \( u\sim v\) si et seulement si \( u=\lambda v\) pour un certain \( \lambda\in\eK\). Cette relation est la relation de \defe{colinéarité}{colinéarité}. L'ensemble des classes d'équivalence de \( \sim\) est l'\defe{espace projectif}{espace!projectif}\index{projectif!espace} de \( E\) et sera noté \( P(E)\)\nomenclature[G]{\( P(E)\)}{l'espace projectif de $E$}.
\end{definition}

\begin{definition}  \label{DEFooTPPMooTDxNpg}
Si \( \dim E=2\), l'ensemble \( P(E)\) est la \defe{droite projective}{droite!projective}\index{projectif!droite}, et si \( \dim E=3\) nous parlons du \defe{plan projectif}{plan!projectif}\index{projectif!plan}.
\end{definition}

Étant donné que tous les \( \eK\)-espaces vectoriels de dimensions \( n+1\) sont isomorphes à \( \eK^{n+1}\), nous noterons \( P_n(\eK)\) ou \( P_n\) l'espace projectif \( P(\eK^{n+1})\). \label{PgNotimesjNtMoW}

\begin{example}
    Si \( n=1\) et \( \eK=\eR\), l'espace projectif est l'ensemble des droites vectorielles dans le plan usuel. Il y en a une pour chaque point du type \( (x,1)\) avec \( x\in\eR\) et ensuite une horizontale, passant par le point \( (1,0)\). Nous avons donc
    \begin{equation}
        P_1(\eR)=\{ (1,0) \}\cup\{ (x,1)\tq x\in \eR \}.
    \end{equation}
    Le point \( (1,0)\) est dit «point à l'infini».
\end{example}

%+++++++++++++++++++++++++++++++++++++++++++++++++++++++++++++++++++++++++++++++++++++++++++++++++++++++++++++++++++++++++++
\section{Sous espaces projectifs}
%+++++++++++++++++++++++++++++++++++++++++++++++++++++++++++++++++++++++++++++++++++++++++++++++++++++++++++++++++++++++++++

Un \defe{sous-espace projectif}{projectif!sous-espace} de \( P(E)\) est une partie de la forme \( P(F)\) où \( F\) est un sous-espace vectoriel de \( E\).

\begin{proposition}     \label{PropuqpWVx}
    Si \( F\) et \( G\) sont des sous-espaces vectoriels de \( E\), alors
    \begin{equation}
        P(F)\cap P(G)=P(F\cap G)
    \end{equation}
    et nous avons
    \begin{equation}        \label{EqNAdWfN}
        \dim P(F)+\dim P(G)=\dim P(F+G)+\dim P(F\cap G).
    \end{equation}
\end{proposition}

\begin{proof}
    Nous avons 
    \begin{equation}
        P(F)=\{ [v]\tq v\in F \}
    \end{equation}
    où les crochets signifient la classe par rapport à la relation de colinéarité. Nous avons alors
    \begin{equation}
        P(F)\cap P(G)=\{ [v]\tq v\in F\cap G \}=P(F\cap G).
    \end{equation}
    Cela prouve le premier point.

    En ce qui concerne l'équation \eqref{EqNAdWfN}, en considérant \( \dim P(E)=\dim E-1\) nous devons prouver l'égalité
    \begin{equation}
        \dim F+\dim G=\dim (F+G)+\dim(F\cap G)
    \end{equation}
    concernant les dimensions des espaces vectoriels usuelles. Si nous considérons une base de \( E\) telle que \( B_1=\{ e_1,\ldots, e_{k_1} \}\) est une base de \( F\cap G\), \( B_2=\{ e_{k_1+1},\ldots, e_{k_2} \}\) complète \( B_1\) en une base de \( F\) et \( B_3=\{ e_{k_2+1},\ldots, e_n \}\) complète \( B_1\cup B_2\) en une base de \( G\).

    Nous avons alors
    \begin{subequations}
        \begin{align}
            \dim F+\dim G&=2\Card(B_1)+\Card(B_2)+\Card(b_3)\\
            \dim(F+G)&=\Card(B_1)+\Card(b_2)+\Card(B_3)\\
            \dim(F\cap G)&=\Card(B_1).
        \end{align}
    \end{subequations}
    De là la relation \eqref{EqNAdWfN} se déduit immédiatement.    
\end{proof}

\begin{theorem}[incidence]\index{théorème!incidence}
    Soient \( F\) et \( F\) deux sous-espaces vectoriels de \( E\) tels que 
    \begin{equation}
        \dim P(F)+\dim P(G)\geq \dim P(E).
    \end{equation}
    Alors \( P(F)\cap P(G)\neq \emptyset\).
\end{theorem}

\begin{proof}
    En utilisant les hypothèses et la proposition \ref{PropuqpWVx} nous avons
    \begin{equation}
        \dim P(E)+\dim P(G)=\dim P(F+G)+\dim P(F\cap G)\geq \dim P(E).
    \end{equation}
    En passant aux espaces vectoriels correspondants,
    \begin{equation}
        \dim(F+G)+\dim(F\cap G)\geq \dim(E)+1.
    \end{equation}
    Mais nous avons aussi \( \dim(F+G)\leq \dim(E)\) et par conséquent \( \dim(F\cap G)\geq 1\). Au final, \( \dim P(F\cap G)\geq 0\). Cela prouve que \( P(F\cap G)\) contient au moins un élément (nous rappelons que lorsqu'un espace projectif contient un seul élément, sa dimension est zéro).
\end{proof}

\begin{example}
    Soient les plans \( \Pi_1\equiv x=0\) et \( \Pi_2\equiv y=0\). Nous avons
    \begin{subequations}
        \begin{align}
            P(\Pi_1)&=\{ [0,y,1] \}\cup\{ [0,1,0] \}\\
            P(\Pi_2)&=\{ [x,0,1] \}\cup\{ [1,0,0] \}
        \end{align}
    \end{subequations}
    où le crochet signifie la classe pour la colinéarité. Ces deux droites projectives ont comme point d'intersection le point \( [0,0,1]\).
\end{example}

\begin{definition}
    Un \defe{hyperplan projectif}{projectif!hyperplan} est un sous-espace projectif de \( P(E)\) de la forme \( P(V)\) où \( V\) est un hyperplan de \( E\).
\end{definition}

\begin{definition}      \label{DEFooBBMBooSVgTnn}
    Soit \( E\) un espace vectoriel de dimension au moins \( 3\). Nous disons que \( d\subset P(E)\) est une \defe{droite projective}{projectif!droite} de \( P(E)\) si \( d=P(D)\) pour une plan vectoriel \( D\subset E\).

    Nous disons que trois points de \( P(E)\) sont \defe{alignés}{alignement!dans un espace projectif} lorsqu'il existe une droite projective les contenant.
\end{definition}

\begin{normaltext}
    Dans la définition \ref{DEFooBBMBooSVgTnn} nous voyons \( P(D)\) comme inclus à \( P(E)\) dès que \( D\) est un sous-espace vectoriel de \( E\). Cela est possible parce que si la direction de \( v\in D\), c'est à dire la classe \( [v]\) est également une direction dans \( E\).
\end{normaltext}

Le lemme suivant peut paraître idiot, mais ce qui serait sûrement idiot est de l'utiliser sans s'en rendre compte.

\begin{lemma}
    Deux points dans \( P(E)\) sont toujours alignés.
\end{lemma}

\begin{proof}
    Soient deux points \( A,B\in P(E)\). Si \( A=\pi(a)\) et \( B=\pi(b)\) alors le plan \( D\) passant par \( a\), \( b\) et \( 0\) est vectoriel et \( P(D)\) contient \( A\) et \( B\).

    Note : si \( a\), \( b\) et \( 0\) sont trois points alignés, alors \( A=B\). Il suffit de prendre les points \( a\), \( c\) et \( 0\) où \( c\in E\) est un point quelconque non aligné avec \( 0\) et \( a\). Nous avons de toutes façons \( A=B=\pi(a)\).
\end{proof}

\begin{lemma}
    Trois points distincts \( A\), \( B\), \( C\) dans \( P(E)\) sont alignés si et seulement si il existe trois points non alignés \( a,b,c\in E\) tels que
    \begin{enumerate}
        \item
            le plan passant par \( a\), \( b\) et \( c\) est vectoriel (c'est à dire passe par \( 0\)),
        \item 
            \( A=\pi(a)\), \( B=\pi(b)\), et \( C=\pi(c)\).
    \end{enumerate}
\end{lemma}

\begin{proof}
    Deux implications à montrer.
    \begin{subproof}
        \item[Sens direct]
            Soient \( A\), \( B\), \( C\) distincts et alignés dans \( P(E)\). Alors il existe un plan vectoriel \( D\) tel que \( A,B,D\in P(D)\). 
            
            La condition \( A\in P(D)\) implique qu'il existe \( a\in D\) tel que \( A=\pi(A)\). Idem pour \( B\) et \( C\). Les points \( a\), \( b\) et \( c\) ainsi construits sont distincts parce que \( A\), \( B\) et \( C\) sont distincts. Si par malheur ces trois points étaient alignés, ce n'est pas grave : il suffit de remplacer \( a\) par \( \lambda a\) avec \( \lambda\neq 0\) pour qu'ils ne le soient plus (cette manipulation ne change pas le fait que le nouveau choix de point \( a\) reste dans \( D\) parce que \( D\) est vectoriel). Nous avons donc trois points non alignés \( a\), \( b\) et \( c\) tous contenus dans \( D\). Le plan \( D\) répond à la question.

        \item[Sens réciproque]

            Soient \( a\), \( b\) et \( c\) non alignés dans \( E\) tels que \( A=\pi(a)\), \( B=\pi(b)\) et \( C=\pi(c)\). Le plan \( D\) les contenant tous trois est vectoriel par hypothèse. Nous avons \( A,B,C\in P(D)\) et donc \( A,B\) et $C$ sont alignés dans \( P(E)\).
    \end{subproof}
\end{proof}

\begin{proposition}
    Soit \( H=P(V)\) un hyperplan projectif de \( P(E)\) et soit \( m\) hors de \( H\). Alors toute droite projective passant par \( m\) coupe \( H\) en un et un seul point.
\end{proposition}

\begin{proof}
    Si \( \dim E=n\) nous avons \( \dim V=n-1\). Soit \( d=P(D)\) une droite projective passant par \( m\), c'est à dire que \( D\) est de dimension \( 2\) dans \( E\). Si \( D\subset V\) alors \( m\in P(D)\subset P(V)\); or nous avons demandé que \( m\) soit hors de \( P(V)\). Par conséquent \( D\) n'est pas inclus à \( V\) et en particulier \( \dim(D+V)=\dim(E)\).

    Nous recopions la formule \eqref{EqNAdWfN} pour notre cas :
    \begin{equation}
        \underbrace{\dim d}_{=1}+\underbrace{\dim H}_{=n-2}=\underbrace{\dim P(D+V)}_{=n-1}+\dim P(D\cap V).
    \end{equation}
    Nous avons donc \( \dim P(D\cap V)=0\), ce qui signifie que l'ensemble \( P(D\cap V)=P(D)\cap P(V)=d\cap H\) contient un et un seul point.
\end{proof}

%+++++++++++++++++++++++++++++++++++++++++++++++++++++++++++++++++++++++++++++++++++++++++++++++++++++++++++++++++++++++++++
\section{Espace projectifs comme «complétés» d'espaces affines}
%+++++++++++++++++++++++++++++++++++++++++++++++++++++++++++++++++++++++++++++++++++++++++++++++++++++++++++++++++++++++++++

Soit \( E\) un espace vectoriel de dimension \( 2\) et \( P(E)\) la droite projective correspondante, et soit \( \{ e_1,e_2 \}\) une base de \( E\). Nous considérons la droite affine \( d\equiv y=1\). Nous avons la bijection
\begin{equation}        \label{EqvrfDLz}
    \begin{aligned}
        \phi\colon d\cup\{ \infty \}&\to P(E) \\
        (x,1)&\mapsto \text{la droite vectorielle passant par } (x,1) \\
        \infty&\mapsto \text{la droite vectorielle passant par } (1,0).
    \end{aligned}
\end{equation}

\begin{lemma}
    Si nous munissons l'ensemble \( d\cup\{ \infty \}\) de la topologie compactifiée d'Alexandroff, la bijection \eqref{EqvrfDLz} est un homéomorphisme.
\end{lemma}

Soient maintenant les plans affines dans l'espace vectoriel \( E\) de dimension \( 3\)
\begin{subequations}
    \begin{align}
        \Pi_1\equiv z&=0\\
        \Pi_2\equiv z&=1.
    \end{align}
\end{subequations}
Une droite (vectorielle) de \( E\) coupe \( \Pi_2\) en un et un seul point, sauf si elle est contenue dans \( \Pi_1\). Nous avons donc une bijection
\begin{equation}
    \begin{aligned}
        \phi\colon P(E)&\to \Pi_2\cup P(\Pi_1) \\
        d&\mapsto \begin{cases}
            \Pi_2\cap d    &   \text{si cette intersection est non vide}\\
            d    &    \text{sinon.}
        \end{cases}
    \end{aligned}
\end{equation}
La droite projective \( P(\Pi_1)\) est la droite à l'infini du plan projectif \( P(E)\). Nous voyons que le plan projectif \( P(E)\) peut être vu comme un plan affine \( (\Pi_2)\) «complété»  par une droite affine \( P(\Pi_1)\). Cette dernière droite est elle-même une droite affine complétée par un point à l'infini.

Nous pouvons généraliser cette démarche en considérant un espace affine \( \affE\) de direction \( E\) sur le corps \( \eK\). Nous construisons \( F=E\times \eK\) et nous considérons un repère affine sur \( F\) tel que \( E\equiv x_{n+1}=0\). Nous pouvons donc identifier \( \affE\) à l'hyperplan affine d'équation \( x_{n+1}=1\) dans \( F\).

Une droite vectorielle de \( F\) non contenue dans \( E\) coupe \( \affE\) en un unique point; nous avons donc une bijection
\begin{equation}
    \affE\cup P(E)\to P(F).
\end{equation}
Dans ce cadre, \( P(E)\) est l'hyperplan à l'infini et nous disons que \( P(E)\) est la \defe{complétion projective}{complétion!projective}\index{projectif!complétion} de \( \affE\).

\begin{example}
    Nous considérons les plans affines
    \begin{subequations}
        \begin{align}
            \Pi_1&\equiv z=0\\
            \Pi_2&\equiv z=1
        \end{align}
    \end{subequations}
    et nous avons la bijection
    \begin{equation}
        P(E)=\Pi_2\cup P(\Pi_1).
    \end{equation}
    Un plan affine \( D\) a deux possibilités : soit il coupe \( \Pi_2\) en une droite, soit il est égal à \( \Pi_1\). Si \( D\cap\Pi_2=d\) (\( d\) est une droite affine), alors nous avons
    \begin{equation}
        P(D)=d\cup\{ \infty_D \},
    \end{equation}
    ce qui justifie la terminologie comme quoi \( P(D)\) est une droite dans \( P(E)\).
\end{example}

Soit \( E\) un espace vectoriel de dimension \( 3\) et le plan projectif \( P(E)\). Nous avons deux types de droites projectives :
\begin{enumerate}
    \item
        D'abord nous avons la droite à l'infini, donnée\footnote{Dans notre représentation usuelle du plan projectif \( z=1\).} par \( P(z=0)\).
    \item
        Ensuite nous avons toutes les droites affines du plan \( z=1\). Chacune de ces droites est complétée par un point à l'infini. 
\end{enumerate}

\begin{example}     \label{ExempMyTmFp}
    Étudions un peu le second type de droites. D'abord si deux droites sont parallèles, leurs points à l'infini sont identiques. Prenons par exemple les droites \( d=\{ z=1,x=1 \}\) et \( d'=\{ z=1,x=2 \}\). Elles décrivent les directions des vecteurs
    \begin{equation}
        \begin{aligned}[]
            \begin{pmatrix}
                1    \\ 
                 y   \\ 
                1    
            \end{pmatrix}&&\text{et}&&
            \begin{pmatrix}
                2    \\ 
                y    \\ 
                1    
            \end{pmatrix}.
        \end{aligned}
    \end{equation}
    En normalisant, ce sont les vecteurs
    \begin{equation}
        \begin{aligned}[]
            \frac{1}{ \sqrt{2+y^2} }\begin{pmatrix}
                1    \\ 
                y    \\ 
                1    
            \end{pmatrix}&&\text{et}&&
            \frac{1}{ \sqrt{5+y^2} }\begin{pmatrix}
                2    \\ 
                y    \\ 
                1    
            \end{pmatrix},
        \end{aligned}
    \end{equation}
    et toutes deux tendent vers le vecteur \( (0,1,0)\) pour \( y\to\infty\).
\end{example}

\begin{lemma}
    Deux droites d'un plan projectif ont toujours une intersection.
\end{lemma}

\begin{proof}
    Si les deux droites sont des droites affines non parallèles, le résultat est évident. Si elles sont parallèles, alors l'intersection est donnée par le point à l'infini comme indiqué dans l'exemple \ref{ExempMyTmFp}.

    Supposons que \( d\) est la droite à l'infini tandis que \( d'\) est une droite affine. Dans notre représentation usuelle du plan affine, la droite à l'infini \( d\) a contient les vecteurs \( (1,y,0)\) et le point à l'infini \( (0,1,0)\). La droite affine \( d'\) a pour équation paramétriques
    \begin{subequations}
        \begin{numcases}{}
            x=at+c\\
            y=bt+d\\
            z=1.
        \end{numcases}
    \end{subequations}
    Les directions données par la droite \( d'\) sont donc
    \begin{equation}
        \frac{1}{ a^2t^2+b^2t^2+c^2+d^2}\begin{pmatrix}
            at+c    \\ 
            bt+d    \\ 
            1
        \end{pmatrix}
    \end{equation}
    Son point à l'infini est la direction du vecteur \( (a,b,0)\), qui est bien un point de la droite à l'infini (éventuellement son point à l'infini\footnote{D'accord, aller chercher le point à l'infini de la droite à l'infini, c'est chercher loin, mais n'empêche que ça existe.}).
\end{proof}

La plupart du temps nous considérons le plan projectif comme étant le plan affine \( z=1\) de l'espace affine de dimension \( 3\) complété par la droite affine \( x=1,z=0\), elle-même complétée par le point \( (0,1,0)\). Ce n'est évidemment pas la seule manière. Tout plan peut être considéré comme le plan à l'infini et pour une droite projective, tout point peut être considéré comme point à l'infini.

Sur la figure \ref{LabelFigChoixInfinissLabelSubFigChoixInfini0}, le point à l'infini est la direction \( (1,0)\) tandis que la direction \( (1,1)\) n'a rien de spécial. À l'inverse sur la figure \ref{LabelFigChoixInfinissLabelSubFigChoixInfini1}, la direction à l'infini est \( (1,1)\) tandis que la direction \( (1,0)\) est une direction usuelle.

%The result is on figure \ref{LabelFigChoixInfini}.
\newcommand{\CaptionFigChoixInfini}{Deux façons de voir la droite projective. Étant donné que les points de la droite projective doivent être interprétés comme des directions (des classes d'équivallence), en réalité les deux dessins représentent les mêmes ensembles.}
\input{pictures_tex/Fig_ChoixInfini.pstricks}

\begin{remark}
    Du point de vue de la topologie, si nous mettons celle de la compactification d'Alexandroff, tous les points de la droite projective sont équivalents.

    Du point de vue de la géométrie différentielle, c'est la même chose. En effet nous pouvons mettre sur la droite projective un système de deux cartes en pensant aux angles. La première sur \( \mathopen] -a , a \mathclose[\) avec par exemple \( a<\pi/4\). La seconde carte serait \( \mathopen] a/2 , \pi \mathclose[\). Dans ce cas la direction \( \theta=0\) semble jouer un rôle spécial, mais il n'en est rien.

    Nous pouvons également considérer les cartes \( \mathopen] \pi/4-a , \pi/4+a \mathclose[\) et \( \mathopen] \pi/4+a/2 , 5\pi/4 \mathclose[\). Dans ces cartes, c'est plutôt le point \( \theta=\pi/4\) qui semble différent (encore qu'il soit bien centré dans une carte).
\end{remark}


%+++++++++++++++++++++++++++++++++++++++++++++++++++++++++++++++++++++++++++++++++++++++++++++++++++++++++++++++++++++++++++
\section{Théorème de Pappus}
%+++++++++++++++++++++++++++++++++++++++++++++++++++++++++++++++++++++++++++++++++++++++++++++++++++++++++++++++++++++++++++

\begin{theorem}     \index{théorème!Pappus!affine}
    Soient deux droites \( d\) et \( d'\) dans un plan affine. Soient \( A,B,C\in d\) et \( A',B',C'\in d'\) tels que \( AB'\parallel BA'\) et \( BC'\parallel B'C\). Alors \( AC'\parallel A'C\).
\end{theorem}

\begin{proof}
    Si \( d\) et \( d'\) ne sont pas parallèles nous considérons \( o\), le point d'intersection. Les relations de parallélisme des hypothèses impliquent qu'il existe \( \lambda_1\) et \( \lambda_2\) tels que
    \begin{subequations}
        \begin{numcases}{}
            A=\lambda_1 B\\
            B'=\lambda_1 A'
        \end{numcases}
    \end{subequations}
    et
    \begin{subequations}
        \begin{numcases}{}
            B'=\lambda_2 C'\\
            C=\lambda_2 B.
        \end{numcases}
    \end{subequations}
    En substituant nous trouvons
    \begin{subequations}
        \begin{numcases}{}
            C=\frac{ \lambda_2 }{ \lambda_1 }A\\
            A'=\frac{ \lambda_2 }{ \lambda_1 }C',
        \end{numcases}
    \end{subequations}
    ce qui implique que \( A'C\parallel AC'\).

    Si les droites \( d\) et \( d'\) sont parallèles, alors nous avons les translations
    \begin{subequations}
        \begin{numcases}{}
            B=A+x\\
            A'=B'+x
        \end{numcases}
    \end{subequations}
    et
    \begin{subequations}
        \begin{numcases}{}
            B=C+y\\
            C'=B'+y,
        \end{numcases}
    \end{subequations}
    ce qui montre que
    \begin{subequations}
        \begin{numcases}{}
            C=A+x-y\\
            A'=C'+x-y,
        \end{numcases}
    \end{subequations}
    et donc que \( A'C\parallel AC'\).
\end{proof}

Le théorème suivant est une version projective.
\begin{theorem}     \index{théorème!Pappus!projectif}
    Soient \( d\) et \( d'\) deux droites projectives d'un plan projectif. Soient \( A,B,C\in d\) et \( A',B',C'\in d'\). Alors les points \( B'C\cap C'B\), \( C'A\cap A'C\) et \( A'B\cap B'A\) sont alignés.
\end{theorem}

\begin{proof}
    Soient \( E=BC'\cap C'B\) et \( E'=C'A\cap A'C\). Ces deux points existent parce que deux droites projectives distinctes ont toujours un unique point d'intersection. Nous allons prendre \( EE'\) comme droite à l'infini et prouver que le point \( A'B\cap B'A\) est dessus. Étant donné que le point d'intersection de \( B'C\) et \( C'B\) est à l'infini nous avons \( B'C\parallel C'B\) (cela est un exemple de la flexibilité de la notion de parallélisme en géométrie projective). De la même façon nous avons \( C'A\parallel A'C\).

    Par le théorème de Pappus affine nous avons alors \( A'B\parallel B'A\) et par conséquent le point d'intersection est sur la droite à l'infini, c'est à dire sur la droite \( EE'\).
\end{proof}

%+++++++++++++++++++++++++++++++++++++++++++++++++++++++++++++++++++++++++++++++++++++++++++++++++++++++++++++++++++++++++++
\section{Homographies}
%+++++++++++++++++++++++++++++++++++++++++++++++++++++++++++++++++++++++++++++++++++++++++++++++++++++++++++++++++++++++++++

%---------------------------------------------------------------------------------------------------------------------------
\subsection{Homographies}
%---------------------------------------------------------------------------------------------------------------------------

\begin{definition}      \label{DEFooKWSMooXvOeEP}
    Soient \( E\) et \( F\) deux espaces vectoriels avec leurs projections naturelles
    \begin{subequations}
        \begin{align}
            \pi_E\colon E\setminus\{ 0 \}&\to P(E)\\
            \pi_F\colon F\setminus\{ 0 \}&\to P(F).
        \end{align}
    \end{subequations}
    Une application \( g\colon P(E)\to P(F)\) est une \defe{homographie}{homographie} s'il existe un isomorphisme d'espaces vectoriels \( \bar g\colon E\to F\) tel que le diagramme
    \begin{equation}
        \xymatrix{%
        E\setminus\{ 0 \} \ar[r]^{\bar g}\ar[d]_{\pi_E}        &   F\setminus\{ 0 \}\ar[d]^{\pi_F}\\
           P(E) \ar[r]_{g}   &   P(F)
           }
    \end{equation}
    commute, c'est à dire s'il existe \( \bar g\colon E\to F\) telle que
    \begin{equation}        \label{EQooSEFWooRpjLxt}
        \pi_F\big( \bar g(v) \big)=g\big( \pi_E(v) \big)
    \end{equation}
    pour tout \( v\in E\).
\end{definition}

\begin{lemma}
    Si \( \bar g\colon E\to F\) est linéaire et si \( \ker\bar g=\{ 0 \}\) alors l'application \( g\) définie par
    \begin{equation}        \label{EqRlGIJW}
        g\big( \pi_E(v) \big)=\pi_F\big( \bar g(v) \big)
    \end{equation}
    est une homographie.
\end{lemma}

\begin{proof}
    Nous devons simplement vérifier que l'équation \eqref{EqRlGIJW} définit bien une application. Soient \( v,w\in E\) tels que \( \pi_Ev=\pi_Ew\); nous devons montrer que 
    \begin{equation}        \label{EqmoIUkH}
        \pi_F\bar gv=\pi_F\bar gw.
    \end{equation}
    L'équation \eqref{EqmoIUkH} sera vérifiée si et seulement s'il existe \( \lambda\in\eR\) tel que \( \bar gv=\lambda\bar gw\), c'est à dire si et seulement si \( \bar g(v-\lambda w)=0\). Étant donné que nous supposons que le noyau de \( \bar g\) est réduit à \( \{ 0 \}\), l'équation \eqref{EqmoIUkH} sera vérifiée si et seulement si \( v=\lambda w\), ce qui signifie exactement \( \pi_E(v)=\pi_E(w)\).
\end{proof}

La proposition suivante donne les premières propriétés des homographies.
\begin{proposition}     \label{PROPooGVYXooDIiIbW}
    Quelques propriétés des homographies.
    \begin{enumerate}
        \item       \label{ITEMooTIONooSKjfny}
            Une homographie est bijective.
        \item
            Si deux espaces projectifs sont homographes, alors ils ont même dimension.
        \item
            L'ensemble des homographies \( P(E)\to P(F)\) est un groupe (pour la composition).
        \item
            Une homographie conserve l'alignement des points.
    \end{enumerate}
\end{proposition}

\begin{proof}
    Nous considérons une homographie \( g\colon P(E)\to P(F)\), et \( \bar g\) l'isomorphisme d'espaces vectoriels correspondant.
    \begin{enumerate}
        \item
            Pour l'injectivité, si \( g\big( [v] \big)=g\big( [w] \big)\) alors en utilisant la définition d'une homographie, \( \pi_F\bar gv=\pi_F\bar gw\), ce qui implique que \( \bar gv=\lambda\bar gw\), et donc \( v=\lambda w\), ce qui signifie \( [v]=[w]\).

            Pour la surjectivité, un élément général de \( P(F)\) prend la forme \( \pi_F\bar gv\) pour un certain \( v\in E\). Nous avons \( g\big( \pi_Ev \big)=\pi_F\bar gv\). Par conséquent l'élément \( \pi_F\bar gv\) est bien dans l'image de \( g\).

        \item
            Une homographie \( P(E)\to P(F)\) n'existe que s'il existe un isomorphisme \( E\to F\). Les dimensions sont donc automatiquement égales.
        \item
            Il suffit de vérifier que l'application
            \begin{equation}
                \begin{aligned}
                    \varphi\colon P(E)&\to P(E) \\
                    \pi_F\bar gv&\mapsto \pi_Ev 
                \end{aligned}
            \end{equation}
            est bien définie et donne l'inverse de \( g\).
        \item
            Soient les points \( A,B,C\) alignés dans \( P(E)\); ils correspondent à des directions de \( E\) qui sont données par des vecteurs situés sur la même droite affine. Autrement dit, il existe trois points \( a,b,c\in E\) situés sur la même droite affine tels que \( A,B,C=\pi_E(a,b,c)\). Les images par \( g\) sont données par \( \pi_F\bar ga\), \( \pi_F\bar gb\), et \( \pi_F\bar gc\).

            Étant donné qu'un isomorphisme d'espaces vectoriels conserve l'alignement affin, les points \( \bar ga\), \( \bar gb\) et \( \bar gc\) sont alignés dans \( F\). Cela implique que les projections par \( \pi_F\) sont alignés dans \( P(F)\).
    \end{enumerate}
\end{proof}

%---------------------------------------------------------------------------------------------------------------------------
\subsection{Le groupe projectif}
%---------------------------------------------------------------------------------------------------------------------------

\begin{definition}
    Le groupe des homographies de l'espace \( P(E)\) est le \defe{groupe projectif}{groupe!projectif}\index{projectif!groupe}, noté \( \PGL(E)\).\nomenclature[G]{\( \PGL(E)\)}{groupe projectif}
\end{definition}

Nous avons une surjection naturelle
\begin{equation}        \label{EqpqNEfe}
    \begin{aligned}
         \GL(E)&\to \PGL(E) \\
        \bar g&\mapsto g 
    \end{aligned}
\end{equation}
qui s'avère être un morphisme de groupes.

\begin{proposition}
    Nous avons l'isomorphisme de groupes
    \begin{equation}
        \frac{ \GL(E) }{\{  \text{homothéties} \}}\simeq \PGL(E).
    \end{equation}
    
\end{proposition}

\begin{proof}
    Nous devons prouver que le noyau de l'application \eqref{EqpqNEfe} est constitué des homothéties. Considérons un automorphisme d'espace vectoriel \( f\colon E\to E\) dont l'homographie associée est l'identité, et prouvons que \( f\) est une homothétie. Nous avons le diagramme commutatif suivant :
    \begin{equation}
        \xymatrix{%
        E\setminus\{ 0 \} \ar[r]^{f}\ar[d]_{\pi_E}        &   E\setminus\{ 0 \}\ar[d]^{\pi_E}\\
           P(E) \ar[r]_{\id}   &   P(E).
           }
    \end{equation}
    Pour tout vecteur \( v\in E\) nous avons \( \pi_E(v)=\pi_E\big( f(v) \big)\). Cela implique qu'il existe \( \lambda\in\eR\) tel que \( f(v)=\lambda v\). Tous les vecteurs de \( E\) sont donc des vecteurs propres de \( f\). Cela n'est possible que si toutes les valeurs propres sont identiques, c'est à dire que \( f\) est une homothétie.
\end{proof}

%+++++++++++++++++++++++++++++++++++++++++++++++++++++++++++++++++++++++++++++++++++++++++++++++++++++++++++++++++++++++++++
\section{Coordonnées homogènes}
%+++++++++++++++++++++++++++++++++++++++++++++++++++++++++++++++++++++++++++++++++++++++++++++++++++++++++++++++++++++++++++

Soit \( E\) un espace vectoriel de dimension \( n+1\) et une base \( \{ e_0,\ldots, e_n \}\) de \( E\). Soit \( M\in P(E)\) et \( u\in E\) un élément engendrant \( M\). Au point \( M\) nous voudrions associer les coordonnées \( (x_0,\ldots, x_n)\) de \( u\) dans \( E\). Notons que toutes les coordonnées de \( u\) ne sont jamais nulles en même temps parce que \( u\) doit indiquer une direction. Nous savons par ailleurs que les coordonnées \( (x_0,\ldots, x_n)\) indiquent le même point de \( P(E)\) que les coordonnées \( (x'_0,\ldots, x'_n)\) si et seulement si \( x_i=\lambda x_i\).

\begin{definition}
    La classe d'équivalence de \( (x_0,\ldots, x_n)\) est la \defe{coordonnées homogène}{coordonnées!homogène} de \( M\). Nous la notons \( (x_0:\ldots :x_n)\).\nomenclature[G]{\( (x_0:\ldots:x_n)\)}{coordonnées homogènes dans un espace projectif}
\end{definition}


%---------------------------------------------------------------------------------------------------------------------------
\subsection{Dualité}
%---------------------------------------------------------------------------------------------------------------------------

Soit \( E\) un espace vectoriel de dimension \( n+1\). Une forme linéaire non nulle est un élément de \( E^*\), mais aussi un représentant d'un élément de \( P(E^*)\).

Le noyau d'une forme linéaire \( \omega\) est un hyperplan. Le noyau de la forme linéaire \( \lambda\omega\) étant le même hyperplan, l'hyperplan est donné par toute la classe de \( \omega\) dans \( P(E^*)\). Nous avons donc une bijection
\begin{equation}
    P(E^*)\leftrightarrow \{ \text{hyperplans vectoriels de } E \}.
\end{equation}

Soit \( E\) de dimension \( 3\) et une base \( \{ e_1,e_2,e_3 \}\). L'espace dual \( E^*\) possède la base duale \( \{ e_1^*,e_2^*,e_3^* \}\). À un élément \( m\in P(E^*)\) nous associons la droite
\begin{equation}
    H_m\{ (X:Y:T)\tq m(X,Y,T)=0 \}
\end{equation}
dans \( P(E)\). Si les coordonnées homogènes de \( m\) étaient \( (u:v:w)\) alors l'équation de la droite \( H_m\) est 
\begin{equation}    \label{Eqezgpmk}
    uX+vY+wT=0.
\end{equation}
En effet si \( \omega\in E^*\) est un représentant de \( m\) alors \( \omega=\lambda(ue_1^*+ve_2^*+we_3^*)\) et l'équation \eqref{Eqezgpmk} est indépendante de \( \lambda\) ainsi que du choix du représentant dans \( E\) du point \( (X:Y:T)\) dans \( P(E)\).

Si les points \( m_1\) et \( m_2\) sont distincts dans \( P(E^*)\), ils donnent deux droites \( m_1(X,Y,T)=0\) et \( m_2(X,Y,T)=0\). Les points de la droite qui joint \( m_1\) à \( m_2\) dans \( P(E^*)\) sont de la forme \( \lambda m_1+\mu m_2\) et ils sont associés à l'équation
\begin{equation}
    \lambda m_1(X,Y,T)+\mu m_2(X,Y,T)=0
\end{equation}
qui sont encore des droites dans \( P(E)\). Toutes ces droites passent par le point d'intersection des droits associées à \( m_1\) et \( m_2\). Nous avons donc
\begin{equation}
    \bigcap_{\lambda,\mu}H_{\lambda m_1+\mu m_2}=H_{m_1}\cap H_{m_2}.
\end{equation}

\begin{lemma}
    L'application
    \begin{equation}
        \begin{aligned}
            P(E^*)&\to \{ \text{droites dans } P(E) \} \\
            m&\mapsto H_m 
        \end{aligned}
    \end{equation}
    est une bijection.
\end{lemma}

\begin{proof}
    Une droite dans \( P(E)\) est donnée en coordonnées homogènes par une équation \( aX+bY+cT=0\). Cette droite est décrite par le point \( (a:b:c)\) dans \( P(E^*)\). Ce dernier correspond à la direction de la forme \( ae_1^*+be_2^*+ce_3^*\). Cela prouve que l'application est surjective.

    Pour l'injectivité, si \( m_1\neq m_2\) dans \( P(E^*)\), les formes \( \omega_1\) et \( \omega_2\) associées dans \( E^*\) ne sont pas multiples l'une de l'autre. Donc les équations
    \begin{equation}
        a_1X+b_1Y+z_1T=0
    \end{equation}
    et
    \begin{equation}
        a_2X+b_2Y+z_2T=0
    \end{equation}
    n'ont pas de solutions communes et décrivent donc des droites distinctes.
\end{proof}

\begin{lemma}   \label{LemjXywjH}
    Trois points distincts \( m_1\), \( m_2\) et \( m_3\) dans \( P(E^*)\) sont alignés si et seulement si les droites \( H_{m_1}\), \( H_{m_2}\) et \( H_{m_3}\) sont distinctes et concourantes.
\end{lemma}

\begin{proof}
    Supposons avoir trois points alignés, c'est à dire
    \begin{equation}    \label{EqXyfbmF}
        m_3=m_1+\mu(m_2-m_1).
    \end{equation}
    Soit \( X:Y:T\) le point d'intersection de \( H_{m_1}\) avec \( H_{m_2}\). Alors \( m_1(X,Y,T)=m_2(X,Y,T)=0\). En tenant compte de \eqref{EqXyfbmF} nous avons alors évidemment \( m_3(X,Y,T)=0\).

    Supposons maintenant que les trois droites \( H_{m_i}\) soient concourantes. Nous avons donc un point \( (X:Y:T)\) dans \( P(E)\) tel que \( m_i(X,Y,T)=0\). Si \( m_i\) est la classe de \( a_ie_1^*+b_ie_2^*+c_ie^*_3\) alors nous avons le système
    \begin{subequations}
        \begin{numcases}{}
            a_1X+b_1Y+c_1T=0\\
            a_2X+b_2Y+c_2T=0\\
            a_3X+b_3Y+c_3T=0.
        \end{numcases}
    \end{subequations}
    Afin que cela ait une solution non triviale nous devons avoir
    \begin{equation}
        \det\begin{pmatrix}
            a_1 &   b_1 &   c_1\\
            a_2 &   b_2 &   c_2\\
            a_3 &   b_3 &   c_3
        \end{pmatrix}\neq 0,
    \end{equation}
    c'est à dire que les points \( (a_i,b_i,c_i)\) soient alignés.
\end{proof}
 
En tenant compte de ce qui a été dit, une droite dans \( P(E^*)\) est constituée de points qui fournissent des droites concourantes dans \( P(E)\). Donc une droite de \( P(E^*)\) se caractérise par un point de \( P(E)\) (l'intersection) de la façon suivante. Un point \( M_d\in P(E)\) donne lieu à un \defe{faisceau de droites}{faisceau de droites} passant par \( M_d\). Chacune de ces droites donne lieu à un point de \( P(E^*)\) et tous ces points sont alignés. Nous avons ainsi construit la droite \( d\) dans \( P(E^*)\) correspondante au point \( M_d\) de \( P(E)\).

%---------------------------------------------------------------------------------------------------------------------------
\subsection{Polynômes}
%---------------------------------------------------------------------------------------------------------------------------

Soit l'espace projectif de dimension \( n\) avec ses coordonnées homogènes \( (X_0:\ldots :X_n)\). Nous considérons l'espace affine \( H\equiv X_n=1\) dans l'espace vectoriel \( E\) de dimension \( n+1\). Nous considérons pour \( H\) un repère affine ayant pour origine le point \( (0,\ldots, 0,1)\). Considérons un polynôme homogène \( P\) sur le corps \( \eK\). L'équation 
\begin{equation}
    P(X_0,\ldots, X_n)=0
\end{equation}
sur l'espace vectoriel \( E\) descend immédiatement à l'espace projectif : étant donné que \( P\) est homogène nous avons \( P(u)=0\) si et seulement si \( P(\lambda u)=0\).

Nous essayons de décrire l'ensemble \( A\) des points de \( P(E)\) satisfaisant \( P(X_0,\ldots, X_n)=0\). Nous savons que les éléments de \( P(E)\) ont chacun un représentant soit dans \( H\) soit sur la droite à l'infini. Ceux de \( A\) ayant un représentant dans \( H\) sont d'équation
\begin{equation}
    Q(x_0,\ldots, x_{n-1})=0
\end{equation}
où \( Q\) est le polynôme donné par \( Q(X_0,\ldots, x_{n-1})=P(x_0,\ldots, x_{n-1},1)\). Les points de \( A\) ayant un représentant sur la droite à l'infini s'obtiennent par l'équation
\begin{equation}
    R(x_0,\ldots, x_{n-1})=0
\end{equation}
où \( R\) est le polynôme donné par \( R(x_0,\ldots, x_{n-1})=P(x_0,\ldots, x_{n-1},0)\).

\begin{example}
    Nous considérons la conique projective
    \begin{equation}    \label{EqpLeQIN}
        X^2-XT-Y^2-T^2=0.
    \end{equation}
    Elle est décomposée en deux partie : une dans l'espace affine «normale» et une à l'infini. La première s'obtient en posant \( T=1\) dans \eqref{EqpLeQIN} :
    \begin{equation}    \label{EqdGHzqJ}
        x^2-x-y^2-1=0.
    \end{equation}
    L'autre est obtenue en posant \( T=0\) :
    \begin{equation}
        x^2-y^2=0.
    \end{equation}
    La partie à l'infini est donc composée de deux points : \( (1:1:0)\) et \( (1:-1:0)\).

    Le graphique de l'équation \eqref{EqdGHzqJ} est donné à la figure \ref{LabelFigProjPoly}. Nous y voyons que les asymptotes sont effectivement données par les directions \( (1,1)\) et \( (1,-1)\) dans le plan.
    \newcommand{\CaptionFigProjPoly}{Le graphique de \( x^2-x-y^2-1=0\).}
    \input{pictures_tex/Fig_ProjPoly.pstricks}
\end{example}

Nous pouvons tenter de faire l'exercice inverse : considérer une conique dans \( \eR^2\), la voir comme une partie d'une conique dans l'espace projectif et trouver les points à l'infini qui la complètent.

\begin{example}
    La droite projective usuelle est donnée par la droite affine \( y-1=0\). L'homogénéisation donne \( y-z=0\) et par conséquent la partie à l'infini est donnée par \( y=0\), c'est à dire la direction \( (1,0)\) comme il se doit.
\end{example}

\begin{example}
    Prenons la conique
    \begin{equation}
        x^2+xy+y^3-2=0.
    \end{equation}
    D'abord nous homogénéisons cette équation pour la voir dans \( \eR^3\) :
    \begin{equation}
        x^2z+xyz+y^3-2z^3=0.
    \end{equation}
    Les points à l'infini sont ceux qui correspondent à \( z=0\), c'est à dire la droite donnée en coordonnées homogènes par \( (1:0:0)\).
\end{example}

%---------------------------------------------------------------------------------------------------------------------------
\subsection{Repères projectifs}
%---------------------------------------------------------------------------------------------------------------------------

Si nous avons une base \( \{ e_i \}\) de \( \eR^n\) nous associons à \( M\in P(E)\) les coordonnées \( (X:Y:T)\). Mais si on prend la base \( \{ 2e_1,e_2,\ldots, e_n \}\), les coordonnées du même point deviennent \( (X/2:Y:T)\) alors que du point de vue de l'espace projectif, rien n'a été changé : la classe de \( e_1\) est la même que celle de \( 2e_1\). Les coordonnées homogènes ne sont donc pas intrinsèques.

\begin{definition}[\cite{BertrandProj}]
    Des éléments \( \{ P_i \}_{i\in I}\) sont \defe{projectivement independents}{indépendance!projective} si en choisissant \( v_i\in\pi^{-1}(P_i)\) nous obtenons des vecteurs \( \{ v_i \}_{i\in I}\) linéairement indépendants.
\end{definition}

\begin{definition}      \label{DEFooPZKFooDBXtEn}
    Soit \( E\) un espace vectoriel de dimension \( n+1\). Un \defe{repère projectif}{repère!projectif}\index{projectif!repère} de \( P(E)\) est la donnée de \( n+2\) points \( m_0,\ldots, m_{n+1}\) tels que
    \begin{enumerate}
        \item
            les vecteurs \( m_i\), \( i\neq 0\), sont les images d'une base \( \{ e_i \}\) de \( E\)
        \item
            \( m_0=\pi_E(e_1+e_2+\ldots +e_{n+1})\).
    \end{enumerate}
\end{definition}
Note que si \( m_k=\pi_E(v_k)\) (\( k=0,\ldots, n+1\)), alors tout choix de \( n+1\) vecteurs parmi les \( v_k\) est une base de \( E\).

\begin{example}
    Un repère projectif de l'espace \( P(\eR^3)\) est par exemple les éléments \( \{ m_i \}_{i=1,\ldots, 3}\) donnés par
    \begin{subequations}
        \begin{align}
            m_1=\pi(e_1)\\
            m_2=\pi(e_2)\\
            m_3=\pi(e_3)\\
            m_0=\pi(e_1+e_2+e_3).
        \end{align}
    \end{subequations}
\end{example}

\begin{normaltext}
Pourquoi voulons-nous des repères projectifs ? Pourquoi demander un quatrième élément alors que trois devraient suffire ? Le fait est que si \( E\) est de dimension \( 3\), nous voudrions pouvoir identifier \( E\) et \( P(\eR^3)\).

Plus précisément, si \( E\) est de dimension \( n+1\) et possède une base \( \{ f_i \}_{i=1,\ldots, n+1}\), il existe un unique isomorphisme d'espaces vectoriels \( E\to \eR^{n+1}\) qui envoie cette base sur la base canonique de \( \eR^{n+1}\). La base de \( E\) étant fixée, nous pouvons donner à un point de \( E\) les coordonnées de son image dans \( \eR^{n+1}\) par cet isomorphisme \emph{qui est unique}.

Dans le cas des espaces projectifs, nous voudrions avoir une unique homographie \( \phi\colon P(E)\to P(\eR^{n+1})\) qui permet de donner à un point \( A\in E\) les coordonnées de \( \pi^{-1}\big( \phi(A) \big)\). Bien entendu ce dernier n'est pas un élément bien défini de \( \eR^{n+1}\) parce qu'il y a toute une droite d'éléments de \( \eR^n\) qui se projettent sur \( \phi(A)\). 

L'idée d'imposer un point de plus est la bonne. Si nous imposons un point de plus, nous pouvons dire que les coordonnées de \( A\in P(E)\) sont celles dans \( \eR^{n+1}\) de l'élément de \( \pi^{-1}\big( \phi(A) \big)\) dont la dernière coordonnée est par exemple \( 1\).

Nous allons maintenant mettre ça en musique.
\end{normaltext}

D'abord nous donnons un exemple de non unicité.

\begin{example}
    Soit un espace vectoriel \( E\) de dimension \( 2\) et une base \(  \{ b_1,b_2 \}  \) de \( E\). Nous considérons également l'espace \( \eR^2\) muni de sa base canonique \( \{ e_1,e_2 \}\).

    Soit une homographie \( \phi\colon P(E)\to P(\eR^2)\) telle que 
    \begin{equation}        \label{EQooPMARooXGuKDD}
        \phi\big( \pi(b_i) \big)=\pi(e_i)
    \end{equation} 
    pour \( i=1,2\). Nous allons facilement construire une autre homographie qui vérifie les mêmes conditions.

    L'idée est la suivante. L'espace \( P(E)\) peut être vu comme la droite complétée \( \{ (  x,1   ) \}_{x\in \eR}\cup\{ (1,0) \} \) et l'espace \( P(\eR^2)\) également. Une homographie respectant \eqref{EQooPMARooXGuKDD} doit envoyer le \( (1,0)\) de \( E\) vers le \( (1,0)\) de \( \eR^2\) et le \( (1,1)\) de \( E\) vers le \( (1,1)\) de \( \eR^2\). Mais en ce qui concerne le reste de la droite, l'homographie peut la parcourir à la vitesse qu'elle veut.

    Il faut envoyer

    \begin{center}
   \input{pictures_tex/Fig_QSKDooujUbDCsu.pstricks}
   sur 
   \input{pictures_tex/Fig_TIMYoochXZZNGP.pstricks}
    \end{center}
    
    Soit donc une homographie \( \phi\colon P(E)\to P(\eR^2)\), et nous définissons
    \begin{equation}
        \begin{aligned}
            \phi'\colon P(E)&\to P(\eR^2) \\
            \pi(xb_1+yb2)& \phi\big( \pi(xb_1+\lambda yb_2) \big) 
        \end{aligned}
    \end{equation}
    pour un certain \( \lambda\neq 1\). En ce qui concerne le relèvement, l'application \( \bar\phi'\colon E\to \eR^2\) donnée par
    \begin{equation}
        \bar\phi'(xb_1+yb_2)=\bar\phi(xb_1+\lambda yb_2)
    \end{equation}
    est bien définie et vérifie
    \begin{equation}
        \pi_{\eR^2}\circ\bar\phi=\phi\circ\pi_E.
    \end{equation}
    Donc \( \phi'\) est une homographie. De plus 
    \begin{subequations}
        \begin{align}
            \phi'\big( \pi(b_1) \big)=\phi\big( \pi(b_1) \big)\\
            \phi'\big( \pi(b_2) \big)=\phi\big( \pi(\lambda b_2) \big)=\phi\big( \pi(b_2) \big)
        \end{align}
    \end{subequations}
    parce que \( \pi(\lambda b_2)=\pi(b_2)\).

    Nous n'avons donc pas l'unicité.
\end{example}

C'est pour rétablir cette unicité que nous demandons d'avoir un point de plus pour avoir un repère projectif. De cette façon nous aurons une unique homographie \( \phi\colon P(E)\to P(\eR^{n+1})\) vérifiant \( \phi\big( \pi_E(b_i) \big)=\pi_{\eR^{n+1}}(e_i)\) pour tout \( i=0,\ldots, n+1\).

\begin{lemma}
    Soit un espace vectoriel \( E\) de dimension \( n+1\) muni de deux bases \( \{ e_i \}_{i=1,\ldots, n+1}\) et \( \{ f_i \}_{i=1,\ldots, n+1}\). Soit un repère projectif \( \{ m_0,m_i \}_{i=1,\ldots, n+1}  \) de \( P(E)\).

    Si \( \pi(e_i)=\pi(f_i)=m_i\) pour tout \( i=1,\ldots, n+1\) et si 
    \begin{equation}
        \pi(e_1+\ldots +e_{n+1})=\pi(f_1+\ldots +f_{n+1})
    \end{equation}
    alors les deux bases sont proportionnelles : il existe \( \lambda\) tel que \( f_i=\lambda e_i\) pour \( i=1,\ldots, n+1\).
\end{lemma}

\begin{proof}
    Nous avons \( \pi(e_i)=\pi(f_i)\) pour tout \( i=1,\ldots, n+1\). Donc pour chaque $i=1,\ldots, n+1$ il existe \( \lambda_i\in \eK\) tel que \( e_i=\lambda f_i\). Nous devons voir que les \( \lambda_i\) sont en réalité tous égaux.

    Pour cela nous avons aussi l'égalité pour \( i=0\) :
    \begin{equation}
        \pi(e_1+\ldots +e_{n+1})=\pi(f_1+\ldots +f_{n+1}),
    \end{equation}
    ce qui donne un \( \mu\in \eK\) tel que $e_1+\ldots +e_{n+1}=\mu(f_1+\ldots +f_{n+1})$, c'est à dire
    \begin{equation}
        \lambda_1 f_1+\ldots +\lambda_{n+1}f_{n+1}=\mu f_1+\ldots +\mu f_{n+1}.
    \end{equation}
    Du fait que les \( f_i\) forment une base, cette égalité impose à tous les \( \lambda_i\) d'être égal à \( \mu\).
\end{proof}

\begin{theorem}[\cite{ooDTHEooBAnkGP}]     \label{THOooTXPVooJGigne}
    Soient \( P(E)\) et \( P(F)\) deux espaces projectifs de dimensions \( n\).
    \begin{enumerate}
        \item       \label{ITEMooRSIWooXbEnlT}
            Une homographie \( P(E)\to P(F)\) envoie un repère projectif sur un repère projectif.
        \item       \label{ITEMooQXQXooDyIsxsh}
            Si \( (m_0,\ldots, m_{n+1})\) est un repère projectif de \( P(E)\), si \( (m'_0,\ldots, m'_{n+1})\) est un repère projectif de \( P(F)\) alors il existe une unique homographie \( g\colon P(E)\to P(F)\) telle que \( g(m_i)=m'_i\) pour tout \( i=0,1,\ldots, n+1\)
    \end{enumerate}
\end{theorem}

\begin{proof}

    Un point à la fois.

    \begin{subproof}
        \item[\ref{ITEMooRSIWooXbEnlT}]
        

    Soit une homographie \( \phi\colon P(E)\to P(F)\) et un repère projectif \( \{ m_0,m_1,\ldots, m_{n+1} \}\) de \( P(E)\). Nous posons \( m'_i=\phi(m_i)\) pour tout \( i=0,\ldots, n+1\). Nous devons prouver que ces \( m'_i\) forment un repère projectif de \( P(F)\).

    D'abord pour \( i=1,\ldots, n+1\) nous avons \( m'_i=\phi\big( \pi_E(e_i) \big)=\pi_F\big( \bar \phi(e_i) \big)\), mais \( \{ \bar\phi(e_i) \}_{i=1,\ldots, n+1}\) est une base de \( F\) parce que \( \bar \phi\) est un isomorphisme d'espaces vectoriels. Donc oui : les \( m'_i\)  (\( i=1,\ldots, n+1\)) sont les projetés d'une base de \( F\).
    
    Nous posons au passage \( f_i=\bar\phi(e_i)\).  En ce qui concerne \( m_0\) nous savons que $m_0=\pi_E(e_1+\ldots +e_{n+1})$ et 
    \begin{equation}
            m'_0=\phi\big( \pi_E(e_1+\ldots +e_{n+1}) \big)
            =\pi_F\big( \bar\phi(e_1+\ldots +e_{n+1}) \big)
            =\pi_F(f_1+\ldots +f_{n+1}),
    \end{equation}
    ce qui termine de montrer que \( \{ m'_i \}_{i=0,\ldots, n+1}\) est un repère projectif de \( P(F)\).

        \item[\ref{ITEMooQXQXooDyIsxsh}]

            Soient un repère projectif \( (m_0,\ldots, m_{n+1})\) de \( P(E)\) et un repère projectif \( (m'_0,\ldots, m'_{n+1})\) de \( P(F)\). Nous choisissons des bases \( \{ e_i \}\) de \( E\) et \(  \{ f_i \}\) de \( F\) telles que
            \begin{subequations}
                \begin{align}
                    m_i&=\pi_E(e_i)\\
                    m'_i&=\pi_F(f_i)
                \end{align}
            \end{subequations}
            pour \( i=1,\ldots, n+1\) et
            \begin{subequations}
                \begin{align}
                    m_0&=\pi_E(e_1+\ldots +e_{n+1})\\
                    m'_0&=\pi_F(f_1+\ldots +f_{n+1}).
                \end{align}
            \end{subequations}
            Nous considérons un isomorphisme d'espace vectoriel \( \bar\phi\colon E\to F\) tel que \( \bar\phi(e_i)=f_i\) pour tout \( i\), et nous voulons définir \( \phi\colon P(E)\to P(F)\) par
            \begin{equation}        \label{EQooRMYIooKcPZwD}
                \phi\big( \pi_E(v) \big)=\pi_F\big( \bar\phi(v) \big).
            \end{equation}
            Cela est bien définit parce que si \( \pi_E(v)=\pi_E(w)\) alors \( w=\lambda v\) et
            \begin{equation}
                \pi_F\big( \bar\phi(\lambda v) \big)=\pi_F\big( \lambda\bar\phi(v) \big)=\pi_F\big( \bar\phi(v) \big).
            \end{equation}
            L'application définie par \eqref{EQooRMYIooKcPZwD} est une homographie qui envoie \( m_i\) sur \( m'_i\) pour tout \( i=0,\ldots, n+1\). Ceci prouve la partie «existence» du point \ref{ITEMooQXQXooDyIsxsh}.

            Pour l'unicité, soient des homographies
            \begin{subequations}
                \begin{align}
                    \phi_1\colon P(E)\to P(F)\\
                    \phi_2\colon P(E)\to P(F)
                \end{align}
            \end{subequations}
            telles que \( \phi_1(m_i)=\phi_2(m_i)\) pour tout \( i=0,\ldots, n+1\). Soit aussi une base \( \{ e_i \}_{i=1,\ldots, n+1}\) de \( E\) adaptée au repère projectif, c'est à dire \( m_i=\pi_E(e_i)\) pour \( i=1,\ldots, n+1\) et \( \pi_E(e_1+\ldots +e_{n+1})=m_0\). Nous considérons aussi les isomorphismes d'espaces vectoriels \( \bar\phi_1\) et \( \bar\phi_2\). Avec tout ce beau monde nous avons
            \begin{subequations}
                \begin{align}
                    \phi_1(m_i)&=\pi_E\big( \bar\phi_1(e_i) \big)\\
                    \phi_2(m_i)&=\pi_E\big( \bar\phi_2(e_i) \big).
                \end{align}
            \end{subequations}
            Mais nous savons que \( \phi_1(m_i)=\phi_2(m_i)\), donc nous savons que \( \pi_E\big( \bar\phi_1(e_i) \big)=\pi_E\big( \bar\phi_2(e_i) \big)\), ce qui nous fait conclure que
            \begin{equation}
                \bar\phi_1(e_i)=\lambda_i\bar\phi_2(e_i)
            \end{equation}
            pour certaines constantes \( \lambda_i\in \eK\). Le même raisonnement appliqué à \( m_0\) nous donne un \( \mu\in \eK\) tel que
            \begin{equation}
                \bar\phi_1(e_1)+\ldots +\bar\phi_1(e_{n+1})=\mu\big( \bar\phi_2(e_1)+\ldots +\bar\phi_2(e_{n+1}) \big).
            \end{equation}
            En mettant l'un dans l'autre :
            \begin{equation}
                \lambda_1\bar\phi_2(e_1)+\ldots +\lambda_{n+1}\bar\phi_2(e_{n+1})=\mu\big( \bar\phi_2(e_1)+\ldots +\bar\phi_2(e_{n+1}) \big).
            \end{equation}
            Sachant que \( \{ \bar\phi_2(e_i) \}_{i=1,\ldots, n+1}\) est une base de \( F\) et nous souvenant de l'unicité de la décomposition d'un élément dans une base\footnote{Proposition \ref{PROPooEIQIooXfWDDV}.}, nous en déduisons que tous les \( \lambda_i\) doivent être égaux à \( \mu\). Donc pour tout \( v\in E\) nous avons \( \bar\phi_1(v)=\lambda\bar\phi_2(v)\).

            Cela a pour conséquence que \( \phi_1=\phi_2\).
    \end{subproof}
\end{proof}

\begin{normaltext}
    Si nous avons une droite projective, trois points sont nécessaires pour créer un repère et donc pour construire une homographie de la droite sur elle-même. Soit \( E\) un espace vectoriel de dimension \( 2\) et \( P(E)\) la droite projective qui lui est associée. Soit une homographie \( f\colon P(E)\to P(E)\) et \( \bar f\colon E\to E\),l'isomorphisme d'espaces vectoriels associé (par \( f\circ\pi_E=\pi_E\circ \bar f\)). Si \( \{ e_1,e_2 \}\) est une base de $E$ alors l'application \( \bar f\) a une matrice
    \begin{equation}
        A=\begin{pmatrix}
            a_{11}    &   a_{12}    \\ 
            a_{21}    &   a_{22}    
        \end{pmatrix}\in \eM(2,\eK)
    \end{equation}
    avec \( \det A\neq 0\) parce que \( \bar f\) est un isomorphisme.

    La plupart des points de \( P(E)\) sont représentés par des points de la forme \( (z,1)\). Nous voudrions savoir quelle est la direction représentée par le point \( \bar f(z,1)\); c'est à dire que nous voudrions savoir \( f([z,1])\) sous la forme \( [z',1]\) (si possible). Nous avons
    \begin{equation}
        \bar f(z,1)=(a_{11}z+a_{12},a_{21}z+a_{22}).
    \end{equation}
    Nous posons \( \lambda=a_{21}z+a_{22}\) et nous avons
    \begin{equation}
        \bar f(z,1)=\lambda\left( \frac{ a_{11}z+a_{12} }{ \lambda },1 \right).
    \end{equation}
    Il y a plusieurs possibilités suivant les valeurs de \( \lambda\) et de \( z\).

    \begin{enumerate}
        \item
            Si \( \lambda=0\) c'est que nous avons \( \bar f(z,1)=(a_{11}z+a_{12},0)\). L'application \( f\) envoie donc le point \( (z:1)\) sur le point à l'infini.
        \item
            Si \( \lambda\neq \), alors \( f\) envoie le point \( (z:1)\) vers un autre point «normal».
        \item
            Si le point de départ est le point à l'infini alors \( \bar f(1,0)=(a_{11},a_{21})\). Cela peut être le point à l'infini ou non selon les valeurs des \( a_{ij}\).
    \end{enumerate}

    Dans tous les cas si nous posons
    \begin{subequations}
        \begin{numcases}{}
            \varphi_f(z)=\frac{ a_{11}z+a_{12} }{ a_{21}z+a_{22} }\\
            \varphi_f(\infty)=\frac{ a_{11} }{ a_{21} }
        \end{numcases}
    \end{subequations}
    alors nous avons
    \begin{equation}
        \bar f(z,1)=\big( \varphi_f(z),1 \big).
    \end{equation}
    Si nous prenons la convention que \( \frac{1}{ 0 }=\infty\) et que \( (\infty,0)\) est le point à l'infini, alors cette application \( \varphi_f\) donne bien toutes les valeurs de \( f\), y compris les cas à l'infini.
\end{normaltext}

%---------------------------------------------------------------------------------------------------------------------------
\subsection{Birapport}
%---------------------------------------------------------------------------------------------------------------------------

Pour rappel, une droite projective est l'espace projectif modelé sur un plan (définition \ref{DEFooBBMBooSVgTnn}). Nous avons la bijection
\begin{equation}        \label{EQooSIJDooTHPYMb}
    \begin{aligned}
        \varphi\colon P(\eK^2)&\to \eK\cup\{ \infty \} \\
        [k_1,k_2]&\mapsto \begin{cases}
            \frac{ k_1 }{ k_2 }    &   \text{si } k_2\neq 0\\
            \infty    &    \text{si } k_2=0.
        \end{cases}
    \end{aligned}
\end{equation}
Notons que nous utilisons ici le fait que \( \eK\) soit commutatif, sinon il aurait fallu choisir \( k_1k_2^{-1}\) ou \( k_2^{-1}k_1\) au lieu d'écrire gentiment \( k_1/k_2\). Pour simplifier les notations, nous nommons dans \( P(\eK^2)\) les points \( \infty=\varphi^{-1}(\infty)\), \( 0=\varphi^{-1}(0)\) et \( 1=\varphi^{-1}(1)\). Nous nous permettons de faire remarquer à l'honorable lecteur que les objets \( \infty\), \( 0\) et \( 1\) à gauche sont très différents de ceux à droite.

\begin{lemma}       \label{LEMooWESDooGwjKLr}
    Les points \( \{ \infty,0,1 \}\) forment un repère projectif de \( P(\eK^2)\).
\end{lemma}

\begin{proof}
    Un repère projectif est la définition \ref{DEFooPZKFooDBXtEn}. Nous avons
    \begin{subequations}
        \begin{align}
            \infty=\varphi^{-1}(\infty)=[1,0]=\pi_{K^2}\big( (1,0) \big)\\
            0=\varphi^{-1}(0)=[0,1]=\pi_{K^2}\big( (0,1) \big)\\
            1=\varphi^{-1}(1)=[1,1]=\pi_{K^2}\big( (1,1) \big)
        \end{align}
    \end{subequations}
    Les points \( (1,0)\) et \( (0,1)\) forment un base de \( \eK^2\) et nous avons bien \( (1,1)=(1,0)+(0,1)\). Donc le tout vérifie bien la définition d'un repère projectif.
\end{proof}

\begin{propositionDef}
    Soit une droite projective \( d\) et trois points distincts \( A\), \( B\) et \( C\) sur cette droite. Si \( X\) est un point de \( d\) alors nous nommons le \defe{birapport}{birapport} de \( X\) par rapport à \( A\), \( B\) et \( C\) l'élément
    \begin{equation}
        [A,B,C,X]=\phi(X)\in \eK\cup\{ \infty \}
    \end{equation}
    où \( \phi\colon d\to \eK\cup \{ \infty \} \) est l'unique homographie telle que
    \begin{subequations}
        \begin{align}
            \phi(A)&=\infty\\
            \phi(B)&=0\\
            \phi(C)&=1.
        \end{align}
    \end{subequations}
\end{propositionDef}

\begin{proof}
    L'objet de la preuve est de voir que la définition est sensée, c'est à dire qu'il existe bien une unique homographie vérifiant les conditions. Cela va se baser sur le théorème \ref{THOooTXPVooJGigne}\ref{ITEMooQXQXooDyIsxsh}. Notons que dans l'énoncé nous parlons de l'homographie \( \phi\colon d\to \hat \eK\). En réalité l'homographie est \( \phi\colon d\to P(\eK^2)\) et l'application dont la définition parle est la composée de ce \( \phi\) (qui est l'authentique homographie) avec la bijection \( \varphi\colon P(\eK^2)\to \hat \eK\) décrite en \eqref{EQooSIJDooTHPYMb}.

    Notons au passage que la définition du birapport dépend du choix de cette identification entre \( d=P(E)\) et \( \hat\eK=\eK\cup\{ \infty \}\).

    \begin{subproof}
        \item[\( A,B,C\) est un repère projectif de \( P(E)\)] 
            Nous avons \( d=P(E)\) où \( E\) est un espace vectoriel de dimension \( 2\) sur le corps \( \eK\). Nous avons \( a,b,c\in E\) tels que \( A=\pi(a)\), \( B=\pi(b)\) et \( C=\pi(c)\). Vu que \( A\neq B\), les vecteurs \( a\) et \( b\) ne sont pas proportionnels et la partie \( \{ a,b \}\) est libre dans \( E\). Autrement dit, c'est une base\footnote{Il convient de citer ici le théorème \ref{ThoMGQZooIgrXjy}\ref{ItemHIVAooPnTlsBi}.}. 
            
            Il existe donc \( \alpha,\beta\in \eK\) tels que \( c=\alpha a+\beta b\). De plus \( \alpha\) et \( \beta\) ne sont pas nuls parce que \( C\neq A\) et \( C\neq B\). En prenant \( a'=\alpha a\), \( b'=\beta b\) et \( c'=c\) nous avons : $A=\pi(a')$, $B=\pi(b')$, $C=\pi(c')$ en même temps que \( \{ a',b' \}\) est une base de \( E\) et \( c'=a'+b'\). Donc \( A,B,C\) est un repère projectif de \( d=P(E)\).

        \item[\( 0,1,\infty\) est un repère projectif de \( P(\eK^2)\)]
            Cela est le lemme \ref{LEMooWESDooGwjKLr}.
        \item[Conclusion]
            Le théorème \ref{THOooTXPVooJGigne}\ref{ITEMooQXQXooDyIsxsh} nous donne existence et unicité d'une homographie \( P(E)\to P(\eK^2) \) envoyant le premier repère sur le second.
    \end{subproof}
\end{proof}

\begin{lemma}[\cite{ooDTHEooBAnkGP}]        \label{LEMooCOFTooVGKdVO}
    Nous avons
    \begin{equation}
        [A,B,C,X]=\begin{cases}
            \infty    &   \text{si et seulement si } X=A\\
            0    &    \text{si et seulement si }X=B\\
            1    &    \text{si et seulement si }X=C.
        \end{cases}
    \end{equation}
\end{lemma}

\begin{proof}
    Vu que \( [A,B,C,X]=\phi(X)\), nous avons
    \begin{equation}
        [A,B,C,X]=\infty
    \end{equation}
    si et seulement si \( \phi(X)=\infty\). Mais \( \phi\) est une bijection vérifiant \( \phi(A)=\infty\). Donc la condition \( \phi(X)=\infty\) est équivalente à \( X=A\).

    Le même raisonnement tient pour les deux autres.
\end{proof}

\begin{proposition}
    Autres petites propriétés faciles \ldots soit une droite projective \( d=P(E)\) et trois points distincts \( A,B,C\in d\).
    \begin{enumerate}
        \item       \label{ITEMooOIPZooQFFYIn}
            Les points \( A\), \( B\), \( C\) et \( X\) sont distincts si et seulement si \( [A,B,C,X]\in \eK\setminus\{ 0,1 \}\).
        \item       \label{ITEMooBEBEooVfiJXY}
            Pour tout \( k\in \hat\eK\), il existe un unique \( X\in d\) tel que \( [A,B,C,X]=k\).
    \end{enumerate}
\end{proposition}

\begin{proof}
    Notons pour le point \ref{ITEMooOIPZooQFFYIn} que l'énoncé demande déjà que \( A\), \( B\) et \( C\) soient distincts. Sinon le birapport n'est pas définit.
    \begin{subproof}
        \item[\ref{ITEMooBEBEooVfiJXY}]

            Les points \( A\), \( B\) et \( C\) sont distincts par hypothèse. Vu le lemme \ref{LEMooCOFTooVGKdVO}, pour que \( X\) soit distincts de \( A\), \( B\) et \( C\) il faut et il suffit que le birapport ne soit ni \( \infty\) ni \( 1\) ni \( 0\). Donc \( \eK\setminus\{ 0,1 \}\).

        \item[\ref{ITEMooBEBEooVfiJXY}]

            Nous avons \( [A,B,C,X]=\phi(X)\) où \( \phi\colon P(E)\to P(\eK^2)\) est une homographie et donc une bijection par la proposition \ref{PROPooGVYXooDIiIbW}\ref{ITEMooTIONooSKjfny}. Donc oui, pour tout éléments de \( P(\eK^2)\) il existe un unique élément de \( P(E)\) dont le birapport par rapport à \( A\), \( B\) et \( C\) soit cet élément.

    \end{subproof}
    Notons encore une fois que nous avons identifié \( P(\eK^2)\) à \( \hat\eK\) par la bijection \eqref{EQooSIJDooTHPYMb}.
\end{proof}

\begin{theorem}
    Une bijection entre deux droites projectives est une homographie si et seulement si elle conserve le birapport.
\end{theorem}

\begin{proposition}
    Soient \( a,b,c,d\in \eK\cup\{ \infty \}\). Alors
    \begin{equation}        \label{EqIYLFEJ}
        [a,b,c,x]=\frac{ (x-b)/(x-a) }{ (c-b)/(c-a) }
    \end{equation}
    où par convention nous prenons \( 1/0=\infty\).
\end{proposition}

\begin{proof}
    Soit \( D\) une droite projective et les coordonnées \( \{ (1,z) \}\cup\{ \infty \}\) dessus. Nous notons \( \varphi\) la fonction du membre de droite de \eqref{EqIYLFEJ}. L'homographie
    \begin{equation}
        \begin{aligned}
            \colon D&\to \eK\cup\{ \infty \} \\
            z&\mapsto \varphi(z) 
        \end{aligned}
    \end{equation}
    envoie \( a\) sur \( \infty\), \( b\) sur \( 0\) et \( c\) sur \( 1\). C'est donc bien cette homographie que définit le birapport et \( x\mapsto[a,b,c,x]\).
\end{proof}

\begin{lemma}
    Soient \( a\), \( b\), \( c\) distincts sur la droite projective \( D=P(E)\). Soient \( x,y\in E\) tels que \( \pi_E(x)=a\), \( \pi_E(y)=b\), \( \pi_E(x+y)=c\). Alors
    \begin{equation}
        d=\pi_E(\lambda x+\mu y)
    \end{equation}
    si et seulement si
    \begin{equation}
        [a,b,c,d]=\pi_{\eK^2}(\lambda,\mu).
    \end{equation}
    
\end{lemma}

\begin{proof}
    Étant donné que \( a\) et \( b\) sont distincts, les vecteurs \( x\) et \( y\) forment une base de \( E\). Soit \( f\colon E\to \eK^2\) un isomorphisme qui envoie \( (x,y)\) sur \( e_1,e_2\) où \( e_i\) sont les vecteurs de base de \( \eK^2\). Ensuite nous considérons  \( g\colon P(E)\to P(\eK^2)\), l'homographie associée à \( f\). Par définition \( f\big( \pi_Ez \big)=\pi_{\eK^2}\big( f(z) \big)\). Par \( f\) nous avons
    \begin{equation}
        \begin{aligned}[]
            a&\mapsto \begin{pmatrix}
                1    \\ 
                0    
            \end{pmatrix}
            &b&\mapsto\begin{pmatrix}
                0    \\ 
                1    
            \end{pmatrix}.
        \end{aligned}
    \end{equation}
    Donc par \( g\) nous avons
    \begin{equation}
        \begin{aligned}[]
            a&\mapsto\infty&
            b\mapsto 0.
        \end{aligned}
    \end{equation}
    Nous avons aussi \( f(\lambda x+\mu y)=(\lambda,\mu)\) et 
    \begin{subequations}
        \begin{align}
            g(c)&=g\big( \pi_E(x+y) \big)\\
            &=\pi_Ff(x+y)\\
            &=\pi_F(f(x)+f(y))\\
            &=\pi_F\begin{pmatrix}
                1    \\ 
                1    
            \end{pmatrix}\\
            &=1.
        \end{align}
    \end{subequations}
    La dernière inégalité est le fait que la direction \( (1,1)\) dans \( \eR^2\) est représentée par le point \( x=1\) sur la droite \( y=1\) qui est notre «représentation» de la droite affine. L'application \( g\) a donc toutes les propriétés qu'il faut pour être l'application qui définit le birapport. Nous avons donc bien \( g(d)=[a,b,c,d]\).

    D'une part si \( d=\pi_E(\lambda x+\mu y)\) alors
    \begin{equation}
        g(d)=\pi_{\eK^2}f(\lambda x+\mu y)=\pi_{\eK^2}(\lambda,\mu).
    \end{equation}
    Dans l'autre sens si \( [a,b,c,d]=\pi_{\eK^2}(\lambda,\mu)\) alors supposons que \( g(d)=\pi_{\eK^2}(\lambda,\mu)\) avec \( d=\pi_E(v)\) alors
    \begin{equation}
        g\pi_Ev=\pi_{\eK^2}f(v),
    \end{equation}
    ce qui implique \( f(v)=\alpha(\lambda,\mu)\) pour un certain \( \alpha\in \eK\). Par conséquent \( v=\alpha(\lambda x+\mu y)\) et \( d=\pi_E(\lambda x+\mu y)\).
\end{proof}

%+++++++++++++++++++++++++++++++++++++++++++++++++++++++++++++++++++++++++++++++++++++++++++++++++++++++++++++++++++++++++++
\section{La sphère de Riemann \( P_1(\eC)\)}
%+++++++++++++++++++++++++++++++++++++++++++++++++++++++++++++++++++++++++++++++++++++++++++++++++++++++++++++++++++++++++++

\begin{definition}      \label{DEFooSZGNooTzFYbh}
    La \defe{sphère de Riemann}{sphère!de Riemann} est l'espace projectif modelé sur \( \eC^2\) : en vertu des notations données à la page \pageref{PgNotimesjNtMoW}, c'est\nomenclature[G]{\( P_1(\eC)\)}{sphère de Riemann}
    \begin{equation}
        P_1(\eC)=P(\eC^2).
    \end{equation}
\end{definition}
\ifbool{isEverything}{Nous parlerons aussi de sphère de Riemann en tant que compactification à un point de \( \eC\) dans la sous-section \ref{SEBSECooLJSEooNlyFYv}.}{}

L'ensemble \( P_1(\eC)\) est le quotient \( \eC^2\setminus\{ (0,0) \}/\sim\) où \( \sim\) est la relation d'équivalence de \( \eC\)-colinéarité dans \( \eC^2\).

\begin{lemma}       \label{LEMooKWZDooEIraSJ}
    L'application
    \begin{equation}        \label{EQooKJIZooZjhzuU}
        \begin{aligned}
            \varphi\colon P_1(\eC)&\to \eC\cup\{ \infty \} \\
            [z_1,z_2]&\mapsto \begin{cases}
                \frac{ z_1 }{ z_2 }    &   \text{si } z_2\neq 0\\
                \infty    &    \text{si } z_2=0
            \end{cases}
        \end{aligned}
    \end{equation}
    est une bijection qui respecte la conjugaisons complexe : \( \varphi\big( [z_1,z_2]^* \big)=\varphi\big( [z_1,z_2] \big)^*\).
\end{lemma}

\begin{proof}
    Notons d'abord que la définition a un sens parce que si un représentant que \( [z_1,z_2]\) est de la forme \( (z,0)\) alors ils sont tous de cette forme. L'affirmation «\( z_1\neq 0\) dans \( [z_1,z_2]\)» a donc un sens.
    \begin{subproof}
        \item[Injectif]
            Supposons \( \varphi\big( [z_1,z_2] \big)=\varphi\big( [t_1,t_2] \big)\). 
            
            Si les deux membres sont égaux à \( \infty\) alors nous avons \( z_2=t_2=0\), et alors avec \( \lambda=z_1/t_1\) nous avons \( (z_1,z_2)=\lambda (t_1,t_2)\), ce qui prouve que \( [z_1,z_2]=[t_1,t_2]\).

            Si les deux membres sont égaux à zéro alors \( z_1=t_1=0\) et le même raisonnement tient.

            Sinon nous avons \( z_1/z_2=t_1/t_2\) où tous les nombres sont non nuls. Cela donne 
            \begin{equation}
                z_2=\frac{ t_2 }{ t_1 }z_1,
            \end{equation}
            et donc
            \begin{equation}
                \frac{ t_1 }{ z_1 }(z_1,z_2)=(t_1,t_2),
            \end{equation}
            qui montre qu'au niveau des classes, \( [z_1,z_2]=[t_1,t_2]\).

        \item[Surjectif]

            Nous avons
            \begin{equation}
                \infty=\varphi\big( [1,0] \big)
            \end{equation}
            et si \( z\neq \infty\) nous avons \( z=\varphi\big( [z,1] \big)\).
    \end{subproof}
\end{proof}

%--------------------------------------------------------------------------------------------------------------------------- 
\subsection{Éléments de géométrie dans \( P_1(\eC)\)}
%---------------------------------------------------------------------------------------------------------------------------
\label{SUBSECooQPRLooAjMNqp}

Étant donné que nous sommes partis pour faire de la géométrie dans \( \eC\) et même dans \( \hat \eC=\eC\cup \{ \infty \}\), autant nous armer des équations de cercles et de droites dans \( \eC\), ainsi que de quelque notions adjacentes.

\begin{remark}
    La définition \ref{DEFooTPPMooTDxNpg} parle de plan et de droites projectives. Ici nous ne sommes pas dans ce cadre parce que nous travaillons sur \( P_1(\eC)\) où \( \eC\) n'est certainement pas un espace de dimension \( 3\). Les droites dont nous allons parler ne sont pas des droites projectives avec leur point à l'infini.
\end{remark}

%///////////////////////////////////////////////////////////////////////////////////////////////////////////////////////////
\subsubsection{Équation complexe d'une droite}
%///////////////////////////////////////////////////////////////////////////////////////////////////////////////////////////

L'équation d'une droite dans \( \eR^2\) est \( d\equiv ax+by=c\) avec \( a,b,c\in \eR\) et \( a,b\) non nuls en même temps. En posant \( z=x+iy\) nous voulons exprimer l'équation en termes de \( z\) au lieu de \( x\) et \( y\). Nous avons\cite{ooWNHWooGUnivi}
\begin{equation}
    \begin{aligned}[]
        x=\frac{ z+\bar z }{2},&&y&=\frac{ z-\bar z }{ 2i },
    \end{aligned}
\end{equation}
et nous pouvons écrire \( d\equiv a(z+\bar z)-ib(z-\bar z)=2c\), ou encore \( d\equiv (a-bi)z+(a+ib)\bar z=2c\). En posant \( \omega=a+bi\in \eC^*\) et \( k=2c\in \eR\) nous avons l'équation
\begin{equation}        \label{EQooPRCPooVvrHME}
    \bar \omega z+\omega z=k.
\end{equation}

\begin{definition}      \label{DEFooAQSMooWNOzAI}
    Une \defe{droite}{droite!dans la sphère de Riemann} est une partie de \( \hat\eC\) de la forme
    \begin{equation}
        d(\omega,k)=\{ z\in \eC\tq \bar\omega z+\omega\bar z=k \}\cup\{ \infty \}
    \end{equation}
    avec \( \omega\in \eC^*\) et \( k\in \eR\).
\end{definition}
Dans \( \hat \eC\), toutes les droites contiennent le point \( \infty\).

\begin{probleme}        \label{PROBooZHHTooIFNwxR}
    La proposition \ref{PROPooMIMRooTbQRVI} montre que toute inversion transforme un cercle-droite en un cercle-droite, nonobstant d'accepter de prolonger toute droite par \( \infty\).

    Est-ce que l'on peut dire que toutes les droites contiennent le point \( \infty\) ?

    En donnant \( \infty\) à toutes les droites et à aucun cercle, la proposition \ref{PROPooMIMRooTbQRVI} fonctionne partout en posant \( i_C(O)=\infty\) et \( i_C(\infty)=O\).

    De plus en pensant à la projection stéréographique, ce serait logique : quelle que soit la direction dans laquelle un point s'éloigne de \( z=0\), son image par l'inverse de la projection stéréographique s'approche du pôle nord.
\end{probleme}

%///////////////////////////////////////////////////////////////////////////////////////////////////////////////////////////
\subsubsection{Équation complexe d'un cercle}
%///////////////////////////////////////////////////////////////////////////////////////////////////////////////////////////

Un cercle de centre \( \omega\in \eC\) et de rayon \( r\) a pour équation \( | z-\omega |=r\), et nous avons les équivalences suivantes :
\begin{equation}
    | z-\omega |=r\Leftrightarrow | z-\omega |^2=r^2\Leftrightarrow (z-\omega)(\bar z-\bar \omega)=r^2\Leftrightarrow z\bar r-\bar \omega z-\omega \bar z=r^2-| \omega |^2.
\end{equation}
Donc un cercle de centre \( \omega\in \eC\) et de rayon \( r\in \eR\) a pour équation
\begin{equation}        \label{EQooDIFRooKRZZoi}
    z\bar z-\bar\omega z-\omega\bar z=r^2-| \omega |^2.
\end{equation}

\begin{definition}
    Un \defe{cercle}{cercle!dans la sphère de Riemann} dans \( \hat\eC\) est une partie de la forme
    \begin{equation}
        C(\omega,r)=\{ z\in \eC\tq z\bar z-\bar\omega z-\omega\bar z=r^2-| \omega |^2 \}
    \end{equation}
    avec \( \omega\in \eC\) et \( r\in \eR\).
\end{definition}
Dans la sphère de Riemann, aucun cercle ne contient le point \( \infty\).

\begin{example}
    Trouvons le centre et le rayon du cercle d'équation
    \begin{equation}
        \bar\omega z+\omega\bar z=kz\bar z   
    \end{equation}
    avec \( k\neq 0\). En divisant par \( k\) et en posant \( \sigma=\omega/k\) nous avons :
    \begin{equation}
        z\bar z-\bar\sigma z-\sigma\bar z=0.
    \end{equation}
    Cela est un cercle de centre \( \sigma\) et de rayon \( | \sigma |\). En effet si \( z\in \eC\) vérifie cette équation,
    \begin{equation}
        | z-\sigma |^2=(z-\sigma)(\bar z-\bar \sigma)=\underbrace{z\bar z-\bar \sigma z-\sigma \bar z}_{=0}+| \sigma |^2=| \sigma |^2,
    \end{equation}
    c'est à dire que tous les points de \( \eC\) qui vérifient l'équation donnée sont à la distance \( | \sigma |\) de \( \sigma\). En particulier \( z=0\) est sur le cercle.
\end{example}

%///////////////////////////////////////////////////////////////////////////////////////////////////////////////////////////
\subsubsection{Cercle-droite}
%///////////////////////////////////////////////////////////////////////////////////////////////////////////////////////////

Une chose de bien avec les équations complexes, c'est que nous pouvons écrire les droites et les cercles avec le même type d'équations.

\begin{lemmaDef}[\cite{ooWNHWooGUnivi}]     \label{LEMooHKHOooHpBuBZ}
    Un \defe{cercle-droite}{cercle-droite} est l'ensemble des points \( z\in \eC\) tels que
    \begin{equation}        \label{EQooUJAKooEVQNqa}
        az\bar z-\bar\omega z-\omega\bar z=k
    \end{equation}
    avec \( a,k\in \eR\) et \( \omega\in \eC\).
    \begin{enumerate}
        \item
            Si \( a=0\), cela est une droite;
        \item
            si \( a\neq 0\), cela est un cercle.
        \item
            Un cercle-droite peut être l'ensemble vide.
    \end{enumerate}
\end{lemmaDef}

\begin{proof}
    Si \( a=0\) alors nous tombons tout de suite sur l'équation \eqref{EQooPRCPooVvrHME}. Si \( a\neq 0\) alors nous pouvons diviser par \( a\), poser \( \sigma=\omega/a\) et \( l=k/a\) pour obtenir
    \begin{equation}        \label{EQooNBILooArHPCG}
        z\bar z-\bar\sigma-\sigma\bar z=l,
    \end{equation}
    qui est l'équation \eqref{EQooDIFRooKRZZoi} d'un cercle \ldots ou pas tout à fait. En effet, \eqref{EQooNBILooArHPCG} serait l'équation du cercle de centre \( \sigma\) et de rayon \( r\) donné par $l=r^2-| \sigma |^2$, c'est ) dire
    \begin{equation}
        r^2=l+| \sigma |^2,
    \end{equation}
    alors que rien n'assure que le nombre \( l+| \sigma |^2\) soit positif. Dans le cas où c'est positif, nous avons bien un cercle. Sinon c'est l'ensemble vide.
\end{proof}

\begin{remark}      \label{REMooBMAEooHDvNID}
    Lorsque nous parlons de cercle-droite, nous parlons de partie de \( \eC\) et non de \( \hat \eC\) parce que l'équation \eqref{EQooUJAKooEVQNqa} a du mal à traiter le cas \( z=\infty\). À cause du fait que nous avons décider de donner le point \( \infty\) à toutes les droites, la fusion des notions de droites et de cercles n'est pas totale; en tout cas pas en une seule équation.
\end{remark}

\begin{example}[\cite{ooWNHWooGUnivi}]      \label{EXooKFBIooOJKjGL}
    Soit le cercle de centre \( \omega=ir\) et de rayon \( r\). Quelle que soit la valeur de \( r>0\), ce cercle passe par le point \( 0\) et l'axe réel lui est tangent. L'équation de ce cercle est :
    \begin{equation}
        z\bar r+irz-ir\bar z=0.
    \end{equation}
    Vu que \( ir\neq 0\) nous pouvons diviser et obtenir
    \begin{equation}
        \frac{ z\bar z }{ ir }+z-\bar z=0.
    \end{equation}
    En faisant tendre \( r\) vers \( \infty\) nous obtenons \( z-\bar z=0\), c'est à dire l'équation de la droite réelle.

    Cela explique pourquoi il est souvent dit qu'une droite est un cercle dont le rayon est à l'infini.
\end{example}

\begin{normaltext}\label{NORMooCXVJooMTMqEU}
Notons que l'exemple \ref{EXooKFBIooOJKjGL} est générique : prenez une droite \( \ell\), un point \( P\) sur \( \ell\), et considérez un cercle dont le centre est situé sur la perpendiculaire à \( \ell\) passant par \( P\), et dont le rayon est tel que le cercle passe par \( P\). En prenant \( | \omega-P |\to \infty\), l'équation du cercle devient celle de la droite \( \ell\).

Cela est particulièrement pratique lorsque nous travaillons dans \( \hat \eC\) parce nous y avons une notion précise du point à l'infini. Notons que (peut-être contre intuitivement), il existe un seul point à l'infini dans \( \hat \eC\). Et ce point est le centre de tous les cercles que l'on veut transformer en droites. Cela pose évidemment la question de savoir comment on définit précisément un cercle dont le centre est réellement \( \infty\).
\end{normaltext}


\begin{center}
   \input{pictures_tex/Fig_PLTWoocPNeiZir.pstricks}
\end{center}

%///////////////////////////////////////////////////////////////////////////////////////////////////////////////////////////
\subsubsection{Rotation-homothétie}
%///////////////////////////////////////////////////////////////////////////////////////////////////////////////////////////

\begin{definition}
    Une \defe{rotation-homothétie}{rotation-homothétie} est une application \(  \hat \eC \to \hat \eC\) de la forme \( z\mapsto \lambda z\) avec \( \lambda\in \eC\).
\end{definition}
Le nom provient du fait que si \( \lambda\) est réel, alors \( z\mapsto \lambda z\) est une vraie homothétie, et si \( \lambda= e^{i\theta}\) alors \( z\mapsto  e^{i\theta}z\) est une vraie rotation. Pour une valeur \( \lambda\in \eC\) générique, l'application \( z\mapsto \lambda z\) est une composée des deux.

%///////////////////////////////////////////////////////////////////////////////////////////////////////////////////////////
\subsubsection{Application linéaire}
%///////////////////////////////////////////////////////////////////////////////////////////////////////////////////////////
\label{SSUBSooRBCWooSCIQEL}

Nous nous en voudrions de ne pas parler d'applications linéaires lorsque nous parlons de géométrie sur \( \hat \eC\). Soit \( \alpha\in \eC^*\) et \( \beta\in \eC\). Lorsque nous parlons de l'application linéaire
\begin{equation}
    \begin{aligned}
        f\colon \hat\eC&\to \hat\eC \\
        z&\mapsto \alpha z+\beta, 
    \end{aligned}
\end{equation}
nous entendons implicitement que \( f(\infty)=\infty\).

%///////////////////////////////////////////////////////////////////////////////////////////////////////////////////////////
\subsubsection{Inversion}
%///////////////////////////////////////////////////////////////////////////////////////////////////////////////////////////
\label{SSUBSooPOUNooTPilbE}

L'inversion d'un cercle de \( \eR^2\) est définie par la proposition \ref{PROPDEFooVLIWooQgpLQa}. De nombreuses propriétés y sont décrites, y compris son écriture complexe dans la proposition \ref{PROPooEWXNooNshvHq}. Tout cela était du temps de \( \eR^2\) ou de \( \eC\), mais maintenant nous sommes dans \( \hat \eC\) et nous voulons plus.

\begin{definition}[\cite{ooWNHWooGUnivi}]       \label{DEFooIUTZooWRaXts}
    Soit \( \omega\in \eC\) et \( R\in \eR^*\). L'\defe{inversion}{inversion dans $\eC\cup\{ \infty \}$} de centre \( \omega\) et de \defe{puissance}{puissance!d'une inversion} \( R^2\) est l'application 
    \begin{equation}
        \begin{aligned}
            i\colon \hat \eC&\to \hat \eC \\
            z&\mapsto \begin{cases}
                \dfrac{ R^2 }{ \bar z-\bar \omega }+\omega    &   \text{si } z\in \eC\setminus\{ \omega \}\\
                 \infty   &    \text{si } z=\omega\\
                 \omega   &    \text{si } z=\infty.
            \end{cases}
        \end{aligned}
    \end{equation}
\end{definition}
Notons que grâce aux conventions type \( 1/0=\infty\) et \( 1/\infty=0\), nous pouvons nous contenter de la première formule pour tout \( z\in \hat \eC\), et nous n'avons en réalité pas besoin de décrire \( i(\infty)\) et \( i(\omega)\) séparément.

\begin{example}
    L'inversion de cercle de centre \( 0\) et de rayon \( 1\) est l'application \( z\mapsto \frac{1}{ \bar z }\), que l'on prolonge avec \( 0\mapsto \infty\) et \( \infty\mapsto 0\).
\end{example}

%--------------------------------------------------------------------------------------------------------------------------- 
\subsection{Homographies}
%---------------------------------------------------------------------------------------------------------------------------

La notion d'homographie est la définition \ref{DEFooKWSMooXvOeEP}. Pour une homographie \( \phi\colon P_1(\eC)\to P_1(\eC)\) nous avons un isomorphisme d'espace vectoriel \( \bar\phi\colon \eC^2\to \eC^2\). Vue la bijection \eqref{EQooKJIZooZjhzuU}, nous voulons plutôt travailler avec \( \hat \eC=\eC\cup\{ \infty \}\) qui est un ensemble avec lequel nous sommes plus familier. Nous allons donc travailler avec
\begin{equation}
    \begin{aligned}
        \tilde \phi&\colon \hat \eC\to \hat \eC \\
        \tilde \phi&=\varphi\circ\phi\circ\varphi^{-1}.
    \end{aligned}
\end{equation}

\begin{proposition}
    Si \( \phi\colon P_1(\eC)\to P_1(\eC)\) est une homographie, alors il existe \( a,b,c,d\in \eC\) tels que \( ad-bc\neq 0\) et l'application \( \tilde \phi\colon \hat \eC\to \hat \eC\) associée est de la forme
    \begin{equation}
        \begin{aligned}[]
            \tilde \phi(\infty)=a/c\\
            \tilde \phi(z)=\frac{ az+b }{ cz+d }
        \end{aligned}
    \end{equation}
    où par convention nous posons \( z/0=\infty\) dès que \( z\neq 0\).

    Dans ce cas, nous avons aussi \( \tilde \phi(-d/c)=\infty\).
\end{proposition}

\begin{proof}
    La condition \eqref{EQooSEFWooRpjLxt} nous dit que
    \begin{equation}
        \big[ \bar \phi(z_1,z_2) \big]=\phi\big( [z_1,z_2] \big),
    \end{equation}
    et comme \( \bar \phi\) est un isomorphisme d'espace vectoriel nous avons \( a,b,c,d\in \eC\) vérifiant \( ad-cb\neq 0\) pour lesquels
    \begin{equation}
        \bar \phi(z_1,z_2)=\begin{pmatrix}
            az_1+bz_2    \\ 
            cz_1+dz_2    
        \end{pmatrix}.
    \end{equation}
    
    Soit \( z\in \eC\). Alors nous avons
    \begin{equation}        \label{EQooNWVZooUClOSd}
        \tilde \phi(z)=(\varphi\circ\phi)[z,1]=\varphi\Big( \big[ \bar\phi(z,1) \big] \Big)=\varphi\big( [az+b,cz+d] \big)=\frac{ az+b }{ cz+d }.
    \end{equation}
    Il est important de comprendre que cette formule fonctionne pour tout \( z\in \eC\). En effet nous pourrions avoir un doute sur \( z=-d/c\). D'abord si \( c=0\) alors \( d\neq 0\) et ce problème n'existe pas : le dénominateur est toujours non nul. Nous avons donc seulement un doute lorsque \( c\neq 0\). Dans ce cas,
    \begin{equation}
        \tilde \phi(-d/c)=\frac{ -\frac{ ad }{ c }+b }{ 0 }.
    \end{equation}
    Mais \( c\neq 0\), donc le numérateur est non nul. Or lorsque \( z\neq 0\) nous avons posé \( z/0=\infty\), donc dans notre cas,
    \begin{equation}
        \tilde \phi(-d/c)=\infty
    \end{equation}
    automatiquement, et cela est encodé dans la formule \eqref{EQooNWVZooUClOSd}.

    Il nous reste à déterminer \( \tilde \phi(\infty)\). Nous avons :
    \begin{equation}
        \tilde \phi(\infty)=(\varphi\circ\phi)\big( [1,0] \big)=\varphi\big[ \bar\phi(1,1) \big]=\varphi\big( [a,c] \big)=\begin{cases}
            a/c    &   \text{si } c\neq 0\\
            \infty    &    \text{si }c=0
        \end{cases}
    \end{equation}
    où la distinction entre les deux cas n'est pas fondamentale parce que si \( c=0\) alors \( a\neq 0\) et \( a/c=\infty\).
\end{proof}

\begin{remark}
    En prenant les conventions relativement claires \( \infty\times a=\infty\) (pour \( a\neq 0\)) et \( \infty\pm a=\infty\) (avec \( a\neq \infty\)), alors tout est dans la formule
    \begin{equation}
        \bar\phi(z)=\frac{ az+b }{ cz+d }
    \end{equation}
    avec \( ad-cb\neq 0\). Il n'y a pas besoin de traiter séparément le cas \( z=\infty\) ou \( z=-d/c\).
\end{remark}

\begin{proposition}[\cite{ooWNHWooGUnivi}]      \label{PROPooSQFOooRginjJ}
    L'application \( h\colon \hat\eC\to \hat \eC\) associée à une homographie est soit linéaire, soit de la forme \( h=l_1\circ \iota\circ l_2\) où \( l_i\) sont linéaires et \( \iota\) est l'application \( z\mapsto 1/z\).
\end{proposition}

\begin{proof}
    Commençons par une remarque : lorsque nous parlons d'une application linéaire, c'est au sens de la note \ref{SSUBSooRBCWooSCIQEL} qui explique qu'une application linéaire sur \( \eC\) est automatiquement prolongée à \( \hat \eC\) par \( f(\infty)=\infty\).

    Soit donc l'application 
    \begin{equation}
        h(z)=\frac{ az+b }{ cz+d }.
    \end{equation}
    Si \( c=0\), alors c'est une application linéaire et la preuve est terminée. Nous supposons que \( c\neq 0\). Nous posons
    \begin{equation}
        l_2(z)=cz+d,
    \end{equation}
    et ensuite \( l_1(z)=\alpha z+\beta\) avec \( \alpha\) et \( \beta\) à déterminer. Un peu de calcul :
    \begin{equation}
        (l_1\circ \iota\circ l_2)(z)=l_1\left( \frac{1}{ cz+d } \right)=\frac{ \alpha+\beta c z+\beta d }{ cz+d },
    \end{equation}
    et en imposant que cela soit égal à \( \frac{ az+b }{ cz+d }\) nous trouvons \( \beta=a/c\) et \( \alpha=b-ad/c\). Il est vite vérifier que ces choix donnent le bon résultat.
\end{proof}

\begin{normaltext}      \label{NORMooMMKOooQlzjqJ}
    Vu que les applications linéaires sont des composées d'une translation et d'une rotation-homothétie, et que l'application \( i\) est une composée d'une inversion \( z\mapsto 1/\bar z\) et d'une réflexion \( z\mapsto \bar z\), toutes les homographies sont des composées des éléments suivants :
    \begin{itemize}
        \item inversion\footnote{Oui, c'est l'inversion de la géométrie hyperbolique, voir \ref{SSUBSooPOUNooTPilbE}.} \( z\mapsto 1/\bar z\), prolongée par \( i(\infty)=0\) et \( i(0)=\infty\);
        \item réflexion \( z\mapsto \bar z\);
        \item translation \( z\mapsto z+\alpha\) avec \( \alpha\in \eC\);
        \item rotation-homothétie \( z\mapsto \lambda z\) avec \( \lambda\in \eC\).
    \end{itemize}
    
    Toutes ces opérations sont prolongées à \( \hat\eC\) par \( 1/\infty=0\), \( \lambda\cdot \infty=\omega\) (si \( \lambda\neq 0\)) et \( \infty+\lambda=\infty\). Nous ne définissons pas \( 0\cdot \infty\) et \( \infty-\infty\).
\end{normaltext}

\begin{normaltext}
    À partir de maintenant nous utiliserons le mot «homographie» pour désigner l'application \( \hat\eC\to\hat\eC\) qui décrit une homographie \( P_1(\eC)\to  P_1(\eC)\).
\end{normaltext}

Certes nous pouvons construire des homographies à partir d'ingrédients dont la conjugaison complexe. Il ne faudrait cependant pas déduire que cette conjugaison est une homographie.

\begin{lemma}       \label{LEMooGDDJooBpJlUf}
    La conjugaison complexe n'est pas une homographie.
\end{lemma}

\begin{proof}
    Si elle l'était nous aurions des nombres \( a,b,c,d\in \eC\) tels que \( ad-bc\neq 0\) et
    \begin{equation}        \label{EQooJMAZooVcJIAP}
        \frac{ az+b }{ cz+d }=\bar z
    \end{equation}
    pour tout \( z\in \eC\).

    En posant \( z=0\) nous avons déjà \( b/d=0\), c'est à dire \( b=0\). Avec \( z=1\) nous trouvons alors \( a/(c+d)=1\), c'est à dire
    \begin{equation}
        a=c+d.
    \end{equation}
    
    \begin{subproof}
        \item[Si \( ci+d\neq 0\)]

            Dans ce cas nous pouvons évaluer \eqref{EQooJMAZooVcJIAP} en \( z=i\) et avoir \( a=-ci+d\). Mais comme nous avions déjà \( a=c+d\) nous déduisons \( c=0\). Nous restons donc avec
            \begin{equation}
                \frac{ a }{ d }z=\bar z
            \end{equation}
            pour tout \( z\). En prenant \( z=1\) puis \( z=i\), il est vite remarqué que cela n'est pas possible.

        \item[Si \( ci+d =0\)]

            Nous rappelons que \( ad\neq 0\). Nous écrivons l'équation avec \( z=-i\) pour trouver
            \begin{equation}
                \frac{ -ai }{ -ci+d }=i,
            \end{equation}
            qui donne immédiatement \( a=ci-d\). Nous avons donc les trois équations
            \begin{subequations}
                \begin{numcases}{}
                    c=id\\
                    a=ci-d\\
                    a=c+d.
                \end{numcases}
            \end{subequations}
            Une tentative de résolution tombe rapidement sur une impossibilité (en substituant la première dans les deux autres et en comparant les deux valeurs de \( a\) par exemple).
    \end{subproof}
\end{proof}

\begin{example}[Inversion d'une droite passant par $z=0$]
    Soit la droite \( d\) dans \( \hat \eC\) donnée par l'équation
    \begin{equation}        \label{EQooKHLOooTuNxEw}
        i(z-\bar z)-(z+\bar z)=0.
    \end{equation}
    Nous rappelons qu'il y a toujours le point \( \infty\) sous-entendu lorsque nous donnons ce type d'équation dans \( \hat\eC\). Même si a strictement parler, \( z=\infty\) a du mal à vérifier la relation \eqref{EQooKHLOooTuNxEw}. Voir la définition \ref{DEFooAQSMooWNOzAI}.

    Nous voulons décrire \( \iota(d)\). Traitons \( z=0\) et \( z=\infty\) à part. Vu que \( \infty\in d\) nous avons \( 0\in\iota(d)\), et vu que \( 0\in d\) nous avons aussi \( \infty\in \iota(d)\).

    En ce qui concerne le reste, le points \( z\in \eC^*\) qui sont dans \( \iota(d)\), il faut chercher
    \begin{equation}
        i\big( \iota^{-1}(z)-\iota^{-1}(\bar z) \big)-\big( \iota^{-1}(z)+\iota^{-1}(\bar z) \big)=0.
    \end{equation}
    Nous savons que \( \iota\) est une involution : \( \iota^{-1}=\iota\) et voila :
    \begin{equation}
        i\left( \frac{1}{ \bar z }-\frac{1}{ z } \right)-\left( \frac{1}{ \bar z }+\frac{1}{ z} \right)=0.
    \end{equation}
    Étant donné que le cas de \( z=0\) est déjà traité nous pouvons multiplier l'équation par \( z\bar z\) et trouver
    \begin{equation}        \label{EQooMVQVooGpROOz}
        i(z-\bar z)-(z+\bar z)=0.
    \end{equation}
    Vu que \( \iota(d)\) contient les points vérifiant \eqref{EQooMVQVooGpROOz} ainsi que \( z=0\) et \( z=\infty\) nous avons trouvé que
    \begin{equation}
        \iota(d)=d.
    \end{equation}
    Cela est tout à fait dans la lignée de la proposition \ref{PROPooMIMRooTbQRVI}\ref{ITEMooNOXMooQYNPnv}.
\end{example}

\begin{example}[Inversion d'une droite ne passant pas par $ z=0$]
    Soit la droite 
    \begin{equation}
        d=\{ z\in \eC\tq z+\bar z-2i(z-\bar z)=3 \}\cup\{ \infty \}
    \end{equation}
    dans \( \hat\eC\). Vu que la restriction \( \iota\colon \eC^*\to \eC^*\) est une bijection (et même une involution) et que \( z=0\) n'est pas dans \( d\), nous pouvons écrire \( \iota(d)\) sous la forme
    \begin{equation}
        \iota(d)=\{   z\in\eC^*\tq \frac{1}{ \bar z }+\frac{1}{ z }-2i\big( \frac{1}{ \bar z }-\frac{1}{ z } \big)=3    \}\cup\{ 0 \}.
    \end{equation}
    En multipliant l'équation par \( z\bar z\) (permis parce que \( z=0\) n'est pas dedans),
    \begin{equation}
        \iota(d)=\{ z\in \eC^*\tq z+\bar z-2i(z-\bar z)=3z\bar z \}\cup\{ 0 \}.
    \end{equation}
    Bonne nouvelle : \( z=0\) respecte l'équation et l'écriture peut être simplifiée :
    \begin{equation}
        \iota(d)=\{ z\in \eC\tq z+\bar z-2i(z-\bar z)=3z\bar z \}.
    \end{equation}
    Cela est un cercle passant par \( z=0\).

    Tout droit dans la lignée de la proposition \ref{PROPooMIMRooTbQRVI}\ref{ITEMooRFPSooGdJdHD}.
\end{example}

\begin{proposition}     \label{PROPooYFJBooAWxFIs}
    Une homographie conserve l'ensemble des droites et cercles de $\hat\eC$.
\end{proposition}
Attention : cela ne veut pas dire qu'une homographie transforme une droite en une droite et un cercle en un cercle. Ça veut dire qu'une homographie transforme une droite en une droite ou un cercle et un cercle en une droite ou un cercle.

\begin{proof}
    Nous savons par la proposition \ref{PROPooMIMRooTbQRVI} et \ref{NORMooMMKOooQlzjqJ} que les homographies se décomposent en inversion, réflexion, translation et rotation-homothétie.

    À part pour les inversions, tout est clair comment ça fonctionne hein. 

    Nous pourrions seulement invoquer la proposition \ref{PROPooMIMRooTbQRVI} et faire quelque hum hum pour régler la question du point \( O\) qui devrait être envoyé sur le point \( \infty\).

    Pour être certain de ne pas nous planter, nous allons faire ça en détail et séparer les cercles, les droites, ceux qui passent par \( z=0\) et les autres. Dans chacun des cas nous allons voir que l'image par une inversion est un cercle ou une droit.

    \begin{subproof}
        \item[Cercle ne passant pas par \( 0\)]
            Nous considérons le cercle \( C(\omega,r)\) avec \( r^2\neq | \omega |^2\). Il ne contient ni \( \infty\) ni \( 0\) et nous avons alors
            \begin{equation}
                \iota\big( C(\omega,r) \big)=\{ z\in \eC^*\tq \frac{1}{ z\bar z }-\bar\omega\frac{1}{ \bar z }-\omega\frac{1}{ z }=r^2-| \omega |^2 \}.
            \end{equation}
            Vu que \( z\) n'est jamais nul nous pouvons multiplier l'équation par \( z\bar z\) :
            \begin{equation}
                \iota\big( C(\omega,r) \big)=\{ z\in \eC\tq \big( | \omega |^2-r^2 \big)z\bar z-\bar\omega z-\omega\bar z=-1 \}.
            \end{equation}
            Le coefficient \( | \omega |^2-r^2\) est non nul par hypothèse et cet ensemble est un cercle par le lemme \ref{LEMooHKHOooHpBuBZ}.

        \item[Cercle passant par \( 0\)]

            Le cercle passe par \( 0\), et donc son image par \( \infty\). Nous écrivons alors
            \begin{equation}
                \iota\big( C(\omega,| \omega |) \big)=\iota\big( C(\omega,| \omega |)\setminus\{ 0 \} \big)\cup\{ \infty \}.
            \end{equation}
            Nous avons 
            \begin{equation}
                C(\omega,| \omega |)\setminus \{ 0 \} =\{ z\in \eC^*\tq z\bar z-\bar \omega z-\omega\bar z=0 \},
            \end{equation}
            et un calcul usuel donne
            \begin{equation}
                \iota\big( C(\omega,| \omega |)\setminus \{ 0 \}  \big)=\{ z\in \eC^*\tq 1-\bar \omega z-\omega\bar z=0 \},
            \end{equation}
            et donc
            \begin{equation}
                \iota\big( C(\omega,| \omega |) \big)=d(\omega,1),
            \end{equation}
            en nous souvenant que le point \( \infty\) est contenu dans \( d(\omega,1)\).

        \item[Droite qui ne passe pas par \( 0\)]

            Une droite ne passant par par \( z=0\) est un ensemble de la forme
            \begin{equation}
                d(\omega,k)=\{ z\in \eC^*\tq \bar\omega z+\omega\bar z=k \}\cup \{ \infty \}
            \end{equation}
            avec \( k\neq 0\). En calculant \( \iota\) de cela nous avons :
            \begin{equation}
                \iota\big( d(\omega,k) \big)=\{ z\in \eC^*\tq \bar \omega z+\omega\bar z=kz\bar z \}\cup\{ 0 \}=  \{ z\in\tq \bar \omega z+\omega\bar z=kz\bar z \},
            \end{equation}
            qui est un cercle passant par l'origine.

        \item[Droite passant par \( 0\)]

            Soit la droite
            \begin{equation}
                d(\omega)=\{ z\in\eC\tq \bar \omega z+\omega\bar z=0 \}\cup\{ \infty \}.
            \end{equation}
            Vu que \( z=0\) et \( z=\infty\) sont sur cette droite, nous aurons aussi \( \{ 0,\infty \}\subset\iota\big( d(\omega) \big)\). La première chose à faire est de calculer
            \begin{equation}
                \iota\{ z\in \eC^*\tq \bar \omega z+\omega\bar z=0 \},
            \end{equation}
            ce qui donne la même équation : \( \bar\omega z+\omega\bar z=0\).

            Au final, \( \iota(d(\omega))=d(\omega)\).

    \end{subproof}
\end{proof}

%--------------------------------------------------------------------------------------------------------------------------- 
\subsection{Groupe circulaire}
%---------------------------------------------------------------------------------------------------------------------------

Nous avons vu que les homographies présent l'ensemble des cercles et droites. Nous pouvons nous demander quel est le groupe maximum préservant l'ensemble des cercles et droites.

\begin{definition}
    Le \defe{groupe circulaire}{groupe!circulaire} de \( \eC\) est le groupe de transformations de \( P_1(\eC)\) engendré par les homographies de \( P_1(\eC)\) et la conjugaison complexe.
\end{definition}

Plusieurs remarques à propos de cette définition.
\begin{enumerate}
    \item
        Le groupe circulaire de \( \eC\) est un groupe agissant sur \( P_1(\eC)\), et la conjugaison complexe, qui agit naturellement sur \( \hat\eC\), agit sur \( P_1(\eC)\) via la bijection \( \varphi\colon P_1(\eC)\to \hat \eC\) donnée par le lemme \ref{LEMooKWZDooEIraSJ}.
    \item
        Vu le lemme \ref{LEMooGDDJooBpJlUf}, la conjugaison complexe n'est pas une homographie. Donc cette définition n'est donc pas stupide : le groupe circulaire est strictement plus grand que le groupe des homographies.
    \item
        Le groupe circulaire agit sur \( \hat\eC\) par la bijection. Et c'est d'ailleurs dans ce sens que nous allons comprendre l'énoncé du théorème \ref{THOooKMKWooZPIDaK}.
    \item
        Vous vous souvenez de la définition d'un sous-groupe engendré ? C'est la définition \ref{DefooRDRXooEhVxxu}.
\end{enumerate}

\begin{theorem}[\cite{ooIPNKooPIzpxy,ooERDDooFeasGD}]       \label{THOooKMKWooZPIDaK}
    Le groupe circulaire de \( \eC\) est le groupe des bijections \( P_1(\eC)\to P_1(\eC)\) préservant l'ensemble des cercles-droites.
\end{theorem}

\begin{proof}
    Nous voyons immédiatement le groupe circulaire comme agissant sur \( \hat\eC\) à travers la bijection \( \varphi\colon P_1(\eC)\to \hat \eC\). Sinon, l'énoncé n'aurait que peu de sens.

    L'inclussions dans un sens est facile : les homographies conservent l'ensemble des cercles et droites par la proposition \ref{PROPooYFJBooAWxFIs}. Et la conjugaison complexe aussi.
\end{proof}
<++>

% TODO : Regarder si c'est possible de mettre les représentations de SL(2,C) et SO(1,3)

%---------------------------------------------------------------------------------------------------------------------------
\subsection{Action du groupe modulaire}
%---------------------------------------------------------------------------------------------------------------------------
\index{groupe!modulaire}

Le \defe{demi-plan de Poincaré}{Poincaré (demi-plan)} est l'ensemble
\begin{equation}
    P=\{ z\in \eC\tq \Im(z)>0 \}.
\end{equation}
Le \defe{groupe modulaire}{groupe!modulaire}\index{modulaire (groupe)} est le quotient de groupes
\begin{equation}
    \PSL(2,\eZ)=\frac{ \SL(2,\eZ) }{ \eZ_2 }.
\end{equation}
Ce sont donc les matrices au signe près de la forme
\begin{equation}
    \begin{pmatrix}
        a    &   b    \\ 
        c    &   d    
    \end{pmatrix}
\end{equation}
où \( a\), \( b\), \( c\) et \( d\) sont entiers tels que \( ad-cb=1\).

\begin{theorem}[\cite{NsoHIL}] \label{ThoItqXCm}
    Le groupe modulaire agit fidèlement (définition \ref{DefuyYJRh}) sur le demi-plan de Poincaré par
    \begin{equation}    \label{EqVXvwlB}
        \begin{pmatrix}
            a    &   b    \\ 
            c    &   d    
        \end{pmatrix}*z=\frac{ az+b }{ cz+d }.
    \end{equation}
    L'ensemble \(D= D_1\cup D_2\) avec
    \begin{subequations}
        \begin{align}
            D_1&=\{ z\in P\tq | z |>1,\,-\frac{ 1 }{2}\leq \Re(z)<\frac{ 1 }{2} \}\\
            D_2&=\{ z\in P\tq | z |=1,\,-\frac{ 1 }{2}\leq \Re(z)\leq0 \}
        \end{align}
    \end{subequations}
    est un domaine fondamental (définition \ref{DefcSuYxz}) de cette action.

    De plus si nous notons 
    \begin{equation}
        \begin{aligned}[]
            S&=\begin{pmatrix}
                0    &   -1    \\ 
                1    &   0    
            \end{pmatrix},&T&=\begin{pmatrix}
                1    &   1    \\ 
                0    &   1    
            \end{pmatrix},
        \end{aligned}
    \end{equation}
    alors pour tout \( z\in P\), il existe \( A\in \gr(S,T)\) telle que \( A*z\in D\).
    %TODO : faire un dessin
\end{theorem}
\index{groupe!action}
\index{groupe!partie génératrice}
\index{groupe!et géométrie}
\index{racine!de l'unité}
\index{matrice}
\index{homographie}
\index{géométrie!avec nombres complexes}
\index{géométrie!avec des groupes}

\begin{proof}

    Nous divisions la preuve en plusieurs étapes.
    \begin{subproof}
        \item[Bien définie]

            D'abord il faut remarquer que l'action \eqref{EqVXvwlB} est bien définie par rapport au quotient : \( A*z=(-A)*z\). La vérification est immédiate.

        \item[Interne]

            Montrons que si \( A\in \PSL(2,\eZ)\) et \( z\in P\) alors \( A*z\in P\). Nous avons
            \begin{equation}
                A*z=\frac{ az+b }{ cz+d }=\frac{ (az+b)(c\bar z+d) }{ | cz+d |^2 }=\frac{ a| z |c+azd+bc\bar z+bd }{ | cz+d |^2 },
            \end{equation}
            et donc en décomposant \( z=\Re(z)+i\Im(z)\),
            \begin{equation}
                \Im(A*z)=\Im\left( \frac{ azd+bc\bar z }{ | cz+d |^2 } \right)=\frac{ ad-bc }{ | cz+d |^2 }\Im(z)=\frac{ \Im(z) }{ | cz+d |^2 }
            \end{equation}
            où nous avons tenu compte de \( ad-bc=1\). Donc l'action respecte la (stricte) positivité de la partie imaginaire.

        \item[Action]

            Nous vérifions maintenant que la formule donne bien une action : \( A*(B*z)=(AB)*z\). Cela est un bon calcul :
            \begin{subequations}
                \begin{align}
                    A*(B*z)&=A*\left( \frac{ a'z+b }{ c'z+d' } \right)\\
                    &=\frac{ a\left( \frac{ a'z+b }{ c'z+d' } \right)+b }{ c\left( \frac{ a'z+b }{ c'z+d' } \right)+d }\\
                    &=\frac{ a(a'z+b')+b(c'z+d') }{ c(a'z+b')+d(c'z+d') }\\
                    &=\frac{ (aa'+bc')z+(ab'+bd') }{ (ca'+dc')z+(cb'+dd') }\\
                    &=\begin{pmatrix}
                        aa'+bc'    &   ab'+bd'    \\ 
                        a'c+dc'    &   cb'+dd'    
                    \end{pmatrix}*z\\
                    &=(AB)*z.
                \end{align}
            \end{subequations}
        \item[Fidèle]

            Soit \( A\in\PSL(2,\eZ)\) tel que pour tout \( z\in P\) nous ayons
            \begin{equation}
                \frac{ az+b }{ cz+d }=z.
            \end{equation}
            Alors nous avons
            \begin{equation}
                cz^2+(d-a)z+b=0.
            \end{equation}
            Cela est donc un polynôme en \( z\) qui s'annule sur un ouvert\footnote{On ne peut pas dire que \( b=0\) simplement en justifiant qu'on l'obtient en posant \( z=0\) parce que \( z=0\) n'est pas dans le demi-plan de Poincaré.} (le demi-plan de Poincaré). Il doit donc être identiquement nul, donc \( c=b=a-d=0\). Si vous n'y croyez pas, écrivez pour \( z=\epsilon i\) (avec \( \epsilon>0\)) :
            \begin{equation}
                -c\epsilon^2+\epsilon(d-a)i+b=0
            \end{equation}
            pour tout \( \epsilon\). Le fait d'avoir \( c\epsilon^2=b\) pour tout \( \epsilon\) implique que \( c=b=0\). Donc \( A\) est de la forme
            \begin{equation}
                A=\begin{pmatrix}
                    a    &   0    \\ 
                    0    &   d    
                \end{pmatrix},
            \end{equation}
            avec la contrainte supplémentaire que \( ad=1\), les nombres \( a\) et \( b\) étant entiers. Nous avons donc soit \( a=d=1\) soit \( a=d=-1\). Étant donné le quotient par \( \eZ_2\), ces deux possibilités donnent le même élément de \( \PSL(2,\eZ)\).
            

        \item[Les orbites intersectent \( D\)] 

            Soit \( z\in P\). Nous devons trouver \( A\in\PSL(2,\eZ)\) tel que \( A*z\in D\). Nous savons déjà que
            \begin{equation}
                \Im(A*z)=\frac{ \Im(z) }{ | cz+d |^2 }.
            \end{equation}
            Nous notons \( \mO_z\) l'orbite de \( z\) sous le groupe modulaire et nous posons
            \begin{equation}
                I_z=\{ \Im(u)\tq u\in \mO_z \}=\{ \Im(A*z)\tq A\in\PSL(2,\eZ) \},
            \end{equation}
            l'ensemble des parties imaginaires des éléments de l'orbite de \( z\). Nous allons montrer que cet ensemble est borné vers le haut en montrant que la quantité \( | cz+d |\) ne peut, ) \( z\) donné, prendre qu'un nombre fini de valeurs plus grandes que \( \Im(z)\)\footnote{Bien que cela ne soit pas indispensable pour la preuve, remarquons que \( I_z\) ne comprend qu'une quantité au plus dénombrable de valeurs. Le fait que, à \( z\) donné, la quantité \( | cz+d |^2\) puisse être rendue aussi grande que l'on veut est évident. Donc \( I_z\) est borné vers le bas par zéro (qui n'est pas atteint, mais qui est une valeur d'adhérence).}. Nous cherchons donc les couples \( (c,d)\in \eZ^2\) tels que \( | cz+d |<1\). 

            Nous avons \( \Im(cz+d)=c\Im(z)\), donc \( | cz+d |\geq |c \Im(z) |\), mais il n'y a qu'un nombre fini de \( c\in \eZ\) tels que \( | c\Im(z) |<1\). De la même façon, pour la partie réelle nous avons
            \begin{equation}
                \Re(cz+d)=c\Re(z)+d,
            \end{equation}
            et pour chaque \( c\),  il n'y a qu'un nombre fini de \( d\in \eZ\) qui laissent cette quantité plus petite que \( 1\) (en valeur absolue).

            Donc \( I_z\) possède un maximum. Soit \( A_1\in\PSL(2,\eZ)\) tel que \( \Im(A_1*z)=\max I_z\). Nous notons \( z_1=A_1*z\), et que nous n'avons a priori pas l'unicité. Nous allons maintenant agir sur \( z_1\) avec l'élément
            \begin{equation}
                T=\begin{pmatrix}
                    1    &   1    \\ 
                    0    &   1    
                \end{pmatrix}
            \end{equation}
            pour ramener \( z_1\) dans le domaine \( D\). Si \( u\in P\) nous avons \( T*u=u+1\) et donc
            \begin{equation}
                T^n*u=u+n.
            \end{equation}
            Vu que \( D\) est de largeur \( 1\), il existe un \( n\) (éventuellement négatif) tel que 
            \begin{equation}
                \Re(T^n*z_1)\in\mathopen[ -\frac{ 1 }{2} , \frac{ 1 }{2} [.
            \end{equation}
            Notons qu'ici le fait d'être ouvert d'un côté et fermé de l'autre joue de façon essentielle (pour l'unicité aussi). Nous notons \( z_2=T^n*z_1\).
            
            Supposons un instant que \( | z_2 |<1\). Nous considérons l'élément
            \begin{equation}
                S=\begin{pmatrix}
                    0    &   -1    \\ 
                    1    &   0    
                \end{pmatrix}
            \end{equation}
            qui fait
            \begin{equation}
                \Im(S*z)=\frac{ \Im z }{| z |^2}.
            \end{equation}
            Donc si \( | z_2 |<1\) alors \( \Im(S*z_2)>\Im(z_2)\), ce qui contredit la maximalité de \( \Im(z_2)\) dans \( I_z\). Nous en déduisons que \( | z_2 |\geq 1\). Nous en déduisons que \( | z_2 |\geq 1\).

            Si \( | z_2 |>1\), alors \( z_2\in D_1\) et c'est bon. Si \( | z_2 |=1\), alors il faut encore un peu travailler. Si \( z_2\pm 1\) est à l'intérieur du disque, alors en agissant avec \( T\) ou \( T^{-1}\) nous retrouvons la même contradiction que précédemment. En écrivant \( z_2= e^{i\theta}\), nous devons donc avoir \( 2\cos(\theta)\leq 1\) ou encore \( |\Re(z_2)|\leq \frac{ 1 }{2}\). Donc si \( \Re(z_2)\leq 0\) alors \( z_2\in D_2\).

            Le dernier cas à traiter est \( \Re(z_2)\in\mathopen] 0 , \frac{ 1 }{2} \mathclose]\), c'est à dire \( \theta\in \mathopen[ \frac{ \pi }{ 3 } , \frac{ \pi }{2} [\). Dans ce cas l'action avec \( S\) ramène l'angle dans la bonne zone parce que \( S*z=-\frac{1}{ z }\) et donc \( S*(\rho e^{-i\theta})=-\frac{1}{ \rho } e^{-i\theta}\).

            \item[Unicité]

                Nous voulons à présent montrer que si \( z\in D\), alors \( A*z\) n'est plus dans \( D\) (sauf si \( A=\pm\mtu\)). Nous supposons que \( z\in D\) et \( A\in \PSL(2,\eZ)\) soient tels que \( A*z\in D\), et nous prouvons qu'alors soit nous arrivons à une contradiction soit nous arrivons à \( A=\mtu\). Pour cela nous allons décomposer en de nombreux cas.

                \begin{enumerate}
                    \item
                        Nous commençons par \( \Im(A*z)\geq \Im(z)\). Dans ce cas nous avons \( | cz+d |\leq 1\) et en particulier \( | c | |\Im(z) |\leq 1\). Étant donné que le point de \( D\) qui a la partie imaginaire la plus petite est \( -\frac{ 1 }{2}+\frac{ 2 }{ \sqrt{3} }i\), nous trouvons \( | c |\leq 2/\sqrt{3}\). Vu que \( c\) doit être entier, nous avons trois cas : \( c=-1,0,1\).
                        \begin{enumerate}
                            \item
                                Soit \( c=0\). Alors \( A=\begin{pmatrix}
                                    a    &   b    \\ 
                                    0    &   d    
                                \end{pmatrix}\) et la condition de déterminant est \( ad=1\), ce qui signifie \( a=d=1\) (la possibilité \( a=b=-1\) est «éliminée» le quotient par \( \eZ_2\) définissant \( \PSL(2,\eZ)\)). La matrice \( A\) doit alors être de la forme
                                \begin{equation}
                                    A=\begin{pmatrix}
                                        1    &   b    \\ 
                                        0    &   1    
                                    \end{pmatrix}
                                \end{equation}
                                et \( A*z=z+b\). Si \( z\in D\), alors le seul \( z+b\) à être (peut-être) encore dans \( D\) est \( b=0\), mais alors \( A\) est l'identité.

                            \item
                                Soit \( c=1\). Alors la condition \( | cz+d |\leq 1\) nous donne trois possibilités\footnote{Je ne rigolais pas quand je disais qu'on allait avoir de nombreux cas.} : \( d=-1,0,1\).

                                \begin{enumerate}
                                    \item
                                        Si \( d=-1\), alors nous devons avoir \( | z-1 |\leq 1\). Il est instructif de faire un dessin, mais le point d'intersection entre les cercles \( | z |=1\) et \( | z-1 |=1\) est le point \( \frac{ 1 }{2}+\frac{ \sqrt{3} }{2}i\), qui n'est pas dans \( D\). Bref, il n'y a pas de points dans \( D\) vérifiant \( | z-1 |\leq 1\).

                                    \item
                                        Si \( d=1\), alors (et c'est maintenant que la dissymétrie de \( D\) intervient) nous avons le point
                                        \begin{equation}
                                            z=-\frac{ 1 }{2}+\frac{ \sqrt{3} }{2}i
                                        \end{equation}
                                        qui est dans \( D\) et qui vérifie \( | z+1 |\leq 1\). Voyons à quoi ressemble la matrice \( A\) dans ce cas. Son déterminant est \( a-b=1\). Nous écrivons donc
                                        \begin{equation}
                                            A=\begin{pmatrix}
                                                b+1    &   b    \\ 
                                                1    &   1    
                                            \end{pmatrix},
                                        \end{equation}
                                        et en tenant compte du fait que \( z\bar z=| z+1 |=1\), nous calculons
                                        \begin{subequations}
                                            \begin{align}
                                                A*z&=\frac{ (b+1)z+b }{ z+1 }\\
                                                &=\frac{ (bz+z+b)(\bar z+1) }{ | z+1 |^2 }\\
                                                &=z+b+1.
                                            \end{align}
                                        \end{subequations}
                                        La seule façon de ne pas quitter \( D\) est d'avoir \( b=-1\), mais alors nous avons
                                        \begin{equation}
                                            A=\begin{pmatrix}
                                                0    &   -1    \\ 
                                                1    &   1    
                                            \end{pmatrix}
                                        \end{equation}
                                        et \( A*z=z\). Donc au final \( z\) est quand même le seul de son orbite à être dans \( D\).

                                        Notons au passage cette très intéressante propriété du point
                                        \begin{equation}
                                            z_0=-\frac{ 1 }{2}+\frac{ \sqrt{3} }{2}i.
                                        \end{equation}
                                        C'est un point de qui vérifie \( z_0=A*z_0\) pour un élément non trivial \( A\) de \( \PSL(2,\eZ)\). L'existence d'un tel élément est ce qui va nous coûter un peu de sueur pour prouver que \( PSL(2,\eZ)\) est engendré par \( S\) et \( T\).

                                    \item
                                        Le cas \( d=0\) nous fait écrire \(1= \det A=-b\), donc \( b=-1\) et 
                                        \begin{equation}
                                            A=\begin{pmatrix}
                                                a    &   -1    \\ 
                                                1    &   0    
                                            \end{pmatrix}.
                                        \end{equation}
                                        Nous avons alors \( A*z=a-\frac{1}{ z }\). De plus la condition \( | z |\leq 1\) revient à \( | z=1 |\). Pour les nombres complexes de module \( 1\), l'opération \( z\to -1/z\) est la symétrie autour de l'axe des imaginaires purs. Le seul à ne pas sortir de \( D\) est le fameux \( z=-\frac{ 1 }{2}+\frac{ \sqrt{3} }{2}i\), qui revient sur lui-même avec \( a=-1\).
                                \end{enumerate}
                                
                        \end{enumerate}
                        
                        Nous passons à la possibilité \( c=-1\). Dans ce cas la matrice est de la forme
                        \begin{equation}
                            A=\begin{pmatrix}
                                a    &   b    \\ 
                                -1    &   d    
                            \end{pmatrix},
                        \end{equation}
                        et nous revenons au cas \( c=1\) en prenant $-A$ au lieu de \( A\).


                    \item
                        Nous passons au cas \( \Im(A*z)<\Im(z)\). Nous récrivons cette condition avec
                        \begin{equation}
                            \Im(A*z)<\Im\big( A^{-1}*(A*z) \big).
                        \end{equation}
                        Si nous supposons que \( z\) et \( A\) sont tels que \( z\) et \( A*z\) soient tous deux dans \( D\), alors \( z'=A*z\) est un élément de \( D\) tel que
                        \begin{equation}
                            \Im(z')<\Im(A^{-1}*z').
                        \end{equation}
                        Or nous avons vu qu'aucun élément de \( D\) vérifiant cette condition n'existait sans être trivial (celui qui ne bouge pas). Pour cela il suffit d'appliquer tout ce que nous venons de dire avec \( A^{-1}\) au lieu de \( A\).
                \end{enumerate}

            \item[Quelque conclusions]

                Après avoir passé tous les cas en revue, le fameux point \( z_0=-\frac{ 1 }{2}+\frac{ \sqrt{3} }{2}i\) est l'unique point de \( D\) à accepter une matrice non triviale \( A\in \PSL(2,\eZ)\) telle que \( z_0=A*z_0\).

                Nous remarquons aussi que tous les points de \( P\) sont ramenés dans \( D\) par une matrice obtenue comme produit de \( T\), \( S\), \( T^{-1}\) et \( S^{-1}\).

    \end{subproof}
\end{proof}

\begin{corollary}[\cite{SjxoHK}]    \label{CorJQwgNp}
    Les matrices \( S\) et \( T\) génèrent le groupe modulaire au sens où toute matrice de \( \PSL(2,\eZ)\) s'écrit comme
    \begin{equation}
        T^{m_1}S^{p_1}\ldots T^{m_k}S^{p_k}
    \end{equation}
    pour un certain \( k\) et des nombres \( m_i,p_i\in \eZ\). Autrement dit, \( \PSL(2,\eZ)=\gr(S,T)\).
\end{corollary}

\begin{proof}
    Soit \( z\), un point de \( D\) autre que \( z_0\). Alors si \( A\in \PSL(2,\eZ)\) est non trivial nous avons \( A*z\) hors de \( D\). Du coup, comme vu dans la démonstration du théorème \ref{ThoItqXCm}, il existe \( B\in \gr(S,T)\) tel que \( B*(A*z)\in D\). Vu que \( D\) ne contient qu'un seul point de chaque orbite, nous avons
    \begin{equation}
        B*A*z=z,
    \end{equation}
    et donc \( BA=\pm\mtu\), ce qui prouve que\footnote{Dans \( \PSL(2,\eZ)\), nous n'avons pas besoin de mettre \( \pm\) parce qu'il est compris dans la définition.} \( A=B^{-1}\), c'est à dire que \( A\in\gr(S,T)\).
\end{proof}

% TODO : Regarder si c'est possible de mettre les représentations de SL(2,C) et SO(1,3)
