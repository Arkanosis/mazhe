% This is part of Mes notes de mathématique
% Copyright (c) 2011-2016
%   Laurent Claessens
% See the file fdl-1.3.txt for copying conditions.

%+++++++++++++++++++++++++++++++++++++++++++++++++++++++++++++++++++++++++++++++++++++++++++++++++++++++++++++++++++++++++++
\section{Forme différentielle}
%+++++++++++++++++++++++++++++++++++++++++++++++++++++++++++++++++++++++++++++++++++++++++++++++++++++++++++++++++++++++++++
\label{SecFormDiffRappel}

Nous parlerons de formes différentielles exactes et fermées dans la section \ref{DefEFKQmPs}.

\begin{definition}
	Soit $D$, un ouvert dans $\eR^n$. Une $1$-\defe{forme différentielle}{forme!différentielle} $\omega$ sur $D$ est une application
	\begin{equation}
		\begin{aligned}
				\omega\colon D&\to (\eR^n)^* \\
				x&\mapsto \omega_x. 
			\end{aligned}
		\end{equation}
\end{definition}
Étant donné que $\{ dx_i \}$ est une base de $(\eR^n)^*$, pour chaque $x\in D$, il existe des uniques réels $a_i(x)$ tels que
\begin{equation}
	\omega_x=a_1(x)dx_1+\cdots+a_n(x)dx_n.
\end{equation}

\begin{lemma}
    Une $1$-forme différentielle est \defe{continue}{continue!forme différentielle} si les fonctions $a_i$ sont continues. La forme sera $C^k$ quand les $a_i$ seront $C^k$.
\end{lemma}

\begin{remark}
	L'ensemble des $1$-formes différentielles forment un espace vectoriel avec les définitions
	\begin{equation}
		\begin{aligned}[]
			(\lambda\omega)_x(v)&=\lambda\omega_x(v)\\
			(\omega+\mu)_x(v)&=\omega_x(v)+\mu_x(v).
		\end{aligned}
	\end{equation}
\end{remark}

Nous connaissons la \defe{base duale}{base!duale} de $(\eR^n)^*$, ce sont les formes $e^*_i$ définies par $e^*_i(e_j)=\delta_{ij}$. Dans le cadre du cours d'analyse, nous allons noter ces formes\footnote{Parce que ce sont les différentielles des fonctions de projection
\begin{equation}
	\begin{aligned}
			x_i\colon \eR^n&\to \eR \\
			x&\mapsto x_i 
		\end{aligned}
	\end{equation}
}
par $dx_i$ :
\begin{equation}
	\begin{aligned}[]
		e^*_1&=dx_1\colon v\mapsto v_1	\\
			&\vdots			\\
		e^*_n&=dx_n\colon v\mapsto v_n
	\end{aligned}
\end{equation}
Toute forme différentielle s'écrit
\begin{equation}
  \omega_x = \sum_{i=0}^n a_i(x) d x_i
\end{equation}
où $a_1,\ldots,a_n$ sont les composantes de $\omega$ dans la base usuelle, et sont des fonctions à valeurs réelles. Pour un vecteur $v = (v_1,\ldots,v_n)$ on a donc par définition de $d x_i$
\begin{equation}
  \omega_x v = \sum_{i=0}^n a_i(x) v_i.
\end{equation}
Ces fonctions $a_i$ peuvent être trouvées en appliquant $\omega$ aux éléments de la base canonique de $\eR^n$ :
\begin{equation}
	a_j(x)=\omega_x(e_j)
\end{equation}
parce que $\omega_x(e_j)=\sum_ia_i(x)dx_i(e_i)=\sum_ia_i(x)\delta_{ij}=a_j(x)$.




\begin{example}
    Un exemple type de forme différentielle est la différentielle d'une fonction $f\colon D\to \eR$. En effet, la différentielle d'une telle fonction est l'application linéaire
    \begin{equation}
        \begin{aligned}
            df\colon \eR^n&\to \eR \\
            v&\mapsto \frac{ \partial f }{ \partial x }v_x+\frac{ \partial f }{ \partial y }v_y. 
        \end{aligned}
    \end{equation}
\end{example}

Soit $D\subset\eR^n$. Par définition de la différentielle d'une $1$-forme, nous avons une formule de Leibnitz
\begin{equation}
    d(f\omega)=df\wedge\omega+fd\omega.
\end{equation}
En particulier,
\begin{equation}
    d(fdx)=df\wedge dx+f\underbrace{d(dx)}_{=0}=\frac{ \partial f }{ \partial x }dx\underbrace{dx\wedge dx}_{=0}+\frac{ \partial f }{ \partial y }dy\wedge dx. S
\end{equation}

Si $F\colon \eR^2\to \eR$ est une fonction $C^2$, sa différentielle est la forme
\begin{equation}
    dF=\frac{ \partial F }{ \partial x }dx+\frac{ \partial F }{ \partial y }dy.
\end{equation}
Si nous nommons $f$ et $g$ les fonctions $\partial_xF$ et $\partial_yF$, nous avons donc
\begin{equation}
    Df=fdx+gdy,
\end{equation}
qui vérifie
\begin{equation}
    \partial_yf=\partial_xg,
\end{equation}
parce que $\frac{ \partial f }{ \partial y }=\frac{ \partial^2F  }{ \partial x\partial y }=\frac{ \partial^2F  }{ \partial y\partial x }=\frac{ \partial g }{ \partial x }$. Ce que nous avons donc prouvé, c'est que 
\begin{lemma}
Si $fdx+gdy$ est la différentielle d'une fonction de classe $C^2$, alors $\partial_yf=\partial_xg$.
\end{lemma}

%///////////////////////////////////////////////////////////////////////////////////////////////////////////////////////////
\subsubsection{L'isomorphisme musical}
%///////////////////////////////////////////////////////////////////////////////////////////////////////////////////////////

Si $G$ est un champ de vecteur sur $\eR^n$, et si $x\in\eR^n$, nous pouvons définissons 
\begin{equation}		\label{EqDefBemol}
	\begin{aligned}[]
		G^{\flat}_x\colon \eR^n&\to \eR \\
			v&\mapsto \langle G(x), v\rangle 
	\end{aligned}
\end{equation}

Pour chaque $x$, l'application $G_x^{\flat}$ est une forme sur $\eR^n$, c'est à dire une application linéaire de $\eR^n$ vers $\eR$. Nous écrivons que
\begin{equation}
	G_x^{\flat}\in\big( \eR^n \big)^*.
\end{equation}

Nous pouvons ainsi déterminer le développement de $G^{\flat}$ dans la base des $dx_i$ en faisant le calcul
\begin{equation}
	G_x^{\flat}(e_i)=\langle G(x), e_i\rangle =G_i(x),
\end{equation}
donc les composantes de $G^{\flat}$ dans la base $dx_i$ sont exactement les composantes de $G$ dans la base $e_i$ :
\begin{equation}
	G^{\flat}_x=G_1(x)dx_1+\cdots+G_n(x)dx_n.
\end{equation}


La construction inverse existe également. Si $\omega$ est une $1$-forme différentielle, nous pouvons définir le champ de vecteur $\omega^{\sharp}$ par la formule (implicite)
\begin{equation}
	\omega_x(v)=\langle \omega^{\sharp}(x), v\rangle 
\end{equation}
pour tout $v\in\eR^n$. Par définition, $(\omega^{\sharp})^{\flat}=\omega$. 

\begin{lemma}
    En composantes nous avons :
	\begin{equation}
		\omega^{\sharp}(x)=\big( a_1(x),\ldots,a_n(x) \big).
	\end{equation}
	Si $G$ est un champ de vecteurs, alors $(G^{\flat})^{\sharp}=G$.
\end{lemma}

%+++++++++++++++++++++++++++++++++++++++++++++++++++++++++++++++++++++++++++++++++++++++++++++++++++++++++++++++++++++++++++
\section{Intégrale sur une variété}
%+++++++++++++++++++++++++++++++++++++++++++++++++++++++++++++++++++++++++++++++++++++++++++++++++++++++++++++++++++++++++++

%---------------------------------------------------------------------------------------------------------------------------
\subsection{Mesure sur une carte}
%---------------------------------------------------------------------------------------------------------------------------

Nous considérons dans cette section uniquement des variétés $M$ de dimension $2$ dans $\eR^3$.  Une particularité de $\eR^3$ (par rapport aux autres $\eR^n$) est qu'il existe le produit vectoriel. 

Si $v$, $w\in\eR^3$, alors le vecteur $v\times w$ est une vecteur normal au plan décrit par $v$ et $w$ qui jouit de l'importante propriété suivante :
\begin{equation}
	\text{aire du parallélogramme}=\| v\times w \|.
\end{equation}
L'aire du parallélogramme construit sur $v$ et $w$ est donnée par la norme du produit vectoriel. Afin de donner une mesure infinitésimale en un point $p\in M$, nous voudrions prendre deux vecteurs tangents à $M$ en $p$, et puis considérer la norme de leur produit vectoriel. Cette idée se heurte à la question du choix des vecteurs tangents à considérer.

Dans $\eR^2$, le choix est évident : nous choisissons $e_x$ et $e_y$, et nous avons $\|e_x\times e_y\|=1$. L'idée est donc de choisir une carte $F\colon W\to F(w)$ autour du point $p=F(w)$, et de choisir les vecteurs tangents qui correspondent à $e_x$ et $e_y$ via la carte, c'est à dire les vecteurs
\begin{equation}
	\begin{aligned}[]
		\frac{ \partial F }{ \partial x }(w),&&\text{et}&&\frac{ \partial F }{ \partial y }(w).
	\end{aligned}
\end{equation}
L'\defe{élément infinitésimal de surface}{element@élément de surface} sur $M$ au point $p=F(w)$ est alors défini par
\begin{equation}
	d\sigma_F=\|  \frac{ \partial F }{ \partial x }(w)\times\frac{ \partial F }{ \partial y }(w) \|dw,
\end{equation}
et si la partie $A\subset M$ est entièrement contenue dans $F(W)$, nous définissons la \defe{mesure}{mesure!dans une carte} de $A$ par
\begin{equation}		\label{EqDefMuDeuxDF}
	\mu_2(A)=\int_{F^{-1}(A)}d\sigma_F=\int_{F^{-1}(A)}\| \frac{ \partial F }{ \partial x }(w)\times\frac{ \partial F }{ \partial y }(w) \|dw.
\end{equation}
\begin{remark}
	Afin que cette définition ait un sens, nous devons prouver qu'elle ne dépend pas du choix de la carte $F$. En effet, les vecteurs $\partial_xF$ et $\partial_yF$ dépendent de la carte $F$, donc leur produit vectoriel (et sa norme) dépendent également de la carte $F$ choisie. Il faudrait donc un petit miracle pour que le nombre $\mu_2(A)$ donné par \eqref{EqDefMuDeuxDF} soit indépendant du choix de $F$.  Nous allons bientôt voir comme cas particulier du théorème \ref{ThoIntIndepF} que c'est en fait le cas. C'est à dire que si $F$ et $\tilde F$ sont deux cartes qui contiennent $A$, alors
	\begin{equation}
		\int_{F^{-1}(A)}d\sigma_F=\int_{\tilde F^{-1}(A)}d\sigma_{\tilde F}.
	\end{equation}
\end{remark}

%///////////////////////////////////////////////////////////////////////////////////////////////////////////////////////////
\subsubsection{Exemple : la mesure de la sphère}
%///////////////////////////////////////////////////////////////////////////////////////////////////////////////////////////

Nous nous proposons maintenant de calculer la surface de la sphère $S^2=x^2+y^2+z^2=R^2$. L'application $F\colon B( (0,0),R)\to R^3$ donnée par
\begin{equation}
	F(x,y)=\begin{pmatrix}
		x	\\ 
		y	\\ 
		\sqrt{R^2-x^2-y^2}	
	\end{pmatrix}
\end{equation}
est une carte pour une demi sphère. Ses dérivées partielles sont
\begin{equation}
	\begin{aligned}[]
		\frac{ \partial F }{ \partial x }&=\begin{pmatrix}
			1	\\ 
			0	\\ 
			-\frac{ x }{ \sqrt{R^2-x^2-y^2} }	
		\end{pmatrix},
		&\frac{ \partial F }{ \partial y }&=\begin{pmatrix}
			0	\\ 
			1	\\ 
			-\frac{ y }{ \sqrt{R^2-x^2-y^2} }	
		\end{pmatrix}.
	\end{aligned}
\end{equation}
Le produit vectoriel de ces deux vecteurs tangents donne
\begin{equation}
	\frac{ \partial F }{ \partial x }(x,y)\times\frac{ \partial F }{ \partial y }(x,y)=\frac{ x }{ \alpha }e_1+\frac{ y }{ \alpha }e_2+e_3
\end{equation}
où $\alpha=\sqrt{R^2-x^2-y^2}$. En calculant la norme, nous trouvons
\begin{equation}
	\| \frac{ \partial F }{ \partial x }(x,y)\times\frac{ \partial F }{ \partial y }(x,y)\| =\sqrt{  \frac{ R^2 }{ R^2-x^2-y^2 } },
\end{equation}
et en passant aux coordonnées polaires, nous écrivons l'intégrale \eqref{EqDefMuDeuxDF} sous la forme
\begin{equation}
	\int_B\| \partial_xF\times\partial_yF \|=\int_0^{2\pi}d\theta\int_0^R r\sqrt{  \frac{ R^2 }{ R^2-x^2-y^2 } }dr=2\pi R^2,
\end{equation}
qui est bien la mesure de la demi sphère.

%---------------------------------------------------------------------------------------------------------------------------
\subsection{Intégrale sur une carte}
%---------------------------------------------------------------------------------------------------------------------------

Nous pouvons maintenant définir l'intégrale d'une fonction sur une carte de la variété $M$.
\begin{definition}
	Soit $F\colon W\subset \eR^2\to \eR^3$, une carte pour une variété $M$. Soit $A$, une partie de $F(W)$ telle que $A=F(B)$ où $B\subset W$ est mesurable.  Soit encore $f\colon A\to \eR$, une fonction continue. L'\defe{intégrale}{intégrale!d'une fonction sur une carte} de $f$ sur $A$ est le nombre
	\begin{equation}	\label{EqDefIntDeuxDF}
		\int_Af=\int_Afd\sigma_F=\int_{F^{-1}(A)}(f\circ F)(w)\|  \frac{ \partial F }{ \partial x }(w)\times\frac{ \partial F }{ \partial y }(w) \| dw
	\end{equation}
\end{definition}

\begin{remark}
	L'intégrale \eqref{EqDefIntDeuxDF} n'est pas toujours bien définie. Étant donné que $F$ est $C^1$ et que $f$ est continue, l'intégrante est continue. L'intégrale sera donc bien définie par exemple lorsque $B$ est borné et si la fermeture $\bar A$ est un compact contenu dans $F(w)$.
\end{remark}

Le théorème suivant montre que le travail que nous avons fait jusqu'à présent ne dépend en fait pas du choix de carte $F$ effectué.

\begin{theorem}\label{ThoIntIndepF}
	Soient $F\colon W\to F(w)$ et $\tilde F\colon \tilde W\to \tilde F(\tilde W)$, deux cartes de la variété $M$. Soit une partie $A\subset F(W)\cap\tilde F(\tilde W)$ telle que $A=F(B)$ avec $B\subset W$ mesurable.  Alors $A=\tilde F(\tilde B)$ avec $\tilde B\subset\tilde W$ mesurable.

	Si $f$ est une fonction continue, et si $\int_Afd\sigma_F$ existe, alors $\int_Afd\sigma_{\tilde F}$ existe et
	\begin{equation}
		\int_Afd\sigma_F=\int_Afd\sigma_{\tilde F}.
	\end{equation}
\end{theorem}


%---------------------------------------------------------------------------------------------------------------------------
\subsection{Exemples}
%---------------------------------------------------------------------------------------------------------------------------

Intégrons la fonction $f(x,y,z)$ sur le carré $K=\mathopen] 0 , 1 \mathclose[\times \mathopen] 0 , 2 \mathclose[\times\{ 1 \}$. La première carte que nous pouvons utiliser est
\begin{equation}
	\begin{aligned}
		F\colon \mathopen] 0 , 1 \mathclose[\times\mathopen] 0 , 2 \mathclose[&\to K \\
		(x,y)&\mapsto (x,y,1). 
	\end{aligned}
\end{equation}
Nous trouvons aisément les vecteurs tangents qui forment l'élément de surface:
\begin{equation}
	\begin{aligned}[]
		\frac{ \partial F }{ \partial x }&=\begin{pmatrix}
			1	\\ 
			0	\\ 
			0	
		\end{pmatrix},
		&\frac{ \partial F }{ \partial y }&=\begin{pmatrix}
			0	\\ 
			1	\\ 
			0	
		\end{pmatrix},
	\end{aligned}
\end{equation}
donc $d\sigma_F=1\cdot dxdy$, et
\begin{equation}		\label{IntKSurcarrUn}
	\int_Kfd\sigma_F=\int_{\mathopen] 0 , 1 \mathclose[\times\mathopen] 0 , 2 \mathclose[}f(x,y,1)\cdot 1\cdot dxdy.
\end{equation}

Nous pouvons également utiliser la carte
\begin{equation}
	\begin{aligned}
		\tilde F\colon \mathopen] 0 , \frac{ 1 }{2} \mathclose[\times\mathopen] 0 , 6 \mathclose[&\to K \\
		(\tilde x,\tilde y)&\mapsto (2\tilde x,\frac{ \tilde y }{ 3 },1). 
	\end{aligned}
\end{equation}
Les vecteurs tangents sont maintenant
\begin{equation}
	\begin{aligned}[]
		\frac{ \partial \tilde F }{ \partial \tilde x }&=\begin{pmatrix}
			2	\\ 
			0	\\ 
			0	
		\end{pmatrix},
		&\frac{ \partial \tilde F }{ \partial \tilde y }&=\begin{pmatrix}
			0	\\ 
			1/3	\\ 
			0	
		\end{pmatrix},
	\end{aligned}
\end{equation}
et nous avons donc $d\sigma_{\tilde F}=\| \frac{ 2 }{ 3 }e_3 \|=\frac{ 2 }{ 3 }$. Cette fois, l'intégrale de $f$ sur $K$ s'écrit
\begin{equation}
	\int_Kfd\sigma_{\tilde F}=\int_{\mathopen] 0 , \frac{ 1 }{2} \mathclose[\times\mathopen] 0 , 6 \mathclose[}f\big( 2\tilde x,\frac{ \tilde y }{ 3 },1 \big)\cdot\frac{ 2 }{ 3 }\cdot d\tilde xs\tilde y.
\end{equation}
Conformément au théorème \ref{ThoIntIndepF}, cette dernière intégrale est égale à l'intégrale \eqref{IntKSurcarrUn} parce qu'il s'agit juste d'un changement de variable.

%---------------------------------------------------------------------------------------------------------------------------
\subsection{Orientation}
%---------------------------------------------------------------------------------------------------------------------------

Soient $F\colon W\to F(w)$ et $\tilde F\colon \tilde W\to \tilde F(\tilde W)$, deux cartes de la variété $M$. Nous pouvons considérer la fonction $h=\tilde F^{-1}\circ F$, définie uniquement sur l'intersection des cartes :
\begin{equation}
	h\colon F^{-1}\big( F(W)\cap\tilde F(\tilde W) \big)\to \tilde F^{-1}\big( F(W)\cap\tilde F(\tilde W) \big).
\end{equation}
Nous disons que $F$ et $\tilde F$ ont même \defe{orientation}{orientation} si
\begin{equation}
	J_h(w)>0
\end{equation}
pour tout $w\in  F^{-1}\big( F(W)\cap\tilde F(\tilde W) \big)$.

Considérons les deux carte suivantes pour le même carré:
\begin{equation}
	\begin{aligned}
		F\colon\mathopen] 0 , 1 \mathclose[\times \mathopen] 0 , 1 \mathclose[ &\to \eR^3 \\
		(x,y)&\mapsto (x,y,0) 
	\end{aligned}
\end{equation}
et
\begin{equation}
	\begin{aligned}
		\tilde F\colon\mathopen] 0 , \frac{ 1 }{2} \mathclose[\times\mathopen] 0 , \frac{1}{ 3 } \mathclose[ &\to \eR^3 \\
		(x,y)&\mapsto (2x,3y,0) 
	\end{aligned}
\end{equation}
Ici, $h(x,y)=\left( \frac{ x }{ 2 },\frac{ y }{ 3 } \right)$ et nous avons $J_h=\frac{1}{ 6 }>0$. Ces deux cartes ont même orientation. Notez que
\begin{equation}
	\frac{ \partial F }{ \partial x }\times\frac{ \partial F }{ \partial y }=e_3,
\end{equation}
tandis que
\begin{equation}
	\frac{ \partial \tilde F }{ \partial x }\times\frac{ \partial \tilde F }{ \partial y }=6e_3.
\end{equation}
Les vecteurs normaux à la paramétrisation pointent dans le même sens.

Si par contre nous prenons la paramétrisation
\begin{equation}
	\begin{aligned}
		G\colon \mathopen] 0,1 \mathclose[\times\mathopen] 0,1 ,  \mathclose[&\to \eR^2 \\
		(x,y)&\mapsto (x,(1-y),0), 
	\end{aligned}
\end{equation}
nous avons
\begin{equation}
	\frac{ \partial G }{ \partial x }\times\frac{ \partial G }{ \partial y }=-e_3,
\end{equation}
et si $g=G^{-1}\circ F$, alors $J_g=-1$.

L'orientation d'une carte montre donc si le vecteur normal à la surface pointe d'un côté ou de l'autre de la surface.

\begin{definition}
	Une variété $M$ est \defe{orientable}{orientable!variété} s'il existe un atlas de $M$ tel que deux cartes quelconques ont toujours même orientation. Une variété est \defe{orientée}{variété !orientée} lorsque qu'un tel choix d'atlas est fait.
\end{definition}

\begin{proposition}
	Soit $M$, une variété orientable et un atlas orienté $\{ F_i\colon W_i\to \eR^3 \}$. Alors le vecteur unitaire
	\begin{equation}
		\frac{   \frac{ \partial F }{ \partial x }(x,y)\times\frac{ \partial F }{ \partial y }(x,y)   }{ \| \frac{ \partial F }{ \partial x }(x,y)\times\frac{ \partial F }{ \partial y }(x,y)\| }
	\end{equation}
	ne dépend pas du choix de $F$ parmi les $F_i$.
\end{proposition}


\begin{proof}
	Considérons deux cartes $F_1$ et $F_2$, ainsi que l'application $h=F_2^{-1}\circ F_1$. Écrivons le vecteur $\partial_x F_1\times\partial_yF_1$ en utilisant $F_1=F_2\circ h$. D'abord, par la règle de dérivation de fonctions composées,
	\begin{equation}
		\frac{ \partial (F_2\circ h) }{ \partial x }=\frac{ \partial F_2 }{ \partial x }\frac{ \partial h_1 }{ \partial x }+\frac{ \partial F_2 }{ \partial y }\frac{ \partial h_2 }{ \partial x }.
	\end{equation}
	Après avoir fait le même calcul pour $\frac{ \partial (F_2\circ h) }{ \partial y }$, nous pouvons écrire
	\begin{equation}
		\partial_x(F_2\circ h)\times\partial_y(F_2\circ h)=(\partial_xh_1\partial_xF_2+\partial_xh_2\partial_yF_2)\times(\partial_yh_1\partial_xF_2+\partial_yh_2\partial_yF_2).
	\end{equation}
	Dans cette expression, les facteurs $\partial_ih_j$ sont des nombres, donc ils se factorisent dans les produits vectoriels. En tenant compte du fait que $\partial_xF_2\times\partial_xF_2=0$ et $\partial_yF_2\times\partial_yF_2=0$, ainsi que de l'antisymétrie du produit vectoriel, l'expression se réduit à
	\begin{equation}
		\left( \frac{ \partial F_2 }{ \partial x }\times\frac{ \partial F_2 }{ \partial y } \right)(\partial_xh_1\partial_yh_2-\partial_xh_2\partial_yh_2).
	\end{equation}
	Par conséquent,
	\begin{equation}
		\frac{ \partial F_1 }{ \partial x }\times\frac{ \partial F_1 }{ \partial y } =\frac{ \partial (F_2\circ h) }{ \partial x }\times\frac{ \partial (F_2\circ h) }{ \partial y } =\left( \frac{ \partial F_2 }{ \partial x }\times\frac{ \partial F_2 }{ \partial y } \right)\det J_h.
	\end{equation}
	Donc, tant que $J_h$ est positif, les vecteurs unitaires correspondants au membre de gauche et de droite sont égaux.
\end{proof}

\begin{corollary}
	Si nous avons choisit un atlas orienté pour la variété $M$, nous avons une fonction continue $G\colon M\to \eR^3$ telle que $\| G(p) \|=1$ pour tout $p\in M$. Cette fonction est donnée par
	\begin{equation}		\label{DefCarteGOritn}
		G(F(x,y))=\frac{   \frac{ \partial F }{ \partial x }(x,y)\times\frac{ \partial F }{ \partial y }(x,y)   }{ \| \frac{ \partial F }{ \partial x }(x,y)\times\frac{ \partial F }{ \partial y }(x,y)\| }
	\end{equation}
	sur l'image de la carte $F$.
\end{corollary}

\begin{proof}
	La fonction $G$ est construite indépendamment sur chaque carte $F(W)$ en utilisant la formule \eqref{DefCarteGOritn}. Cette fonction est une fonction bien définie sur tout $M$ parce que nous venons de démontrer que sur $F_1(W_1)\cap F_2(W_2)$, les fonctions construites à partir de $F_1$ et à partir de $F_2$ sont égales.
\end{proof}

Il est possible que prouver, bien que cela soit plus compliqué, que la réciproque est également vraie.
\begin{proposition}
	Une variété $M$ de dimension $2$ dans $\eR^3$ est orientable si et seulement s'il existe une fonction continue $G\colon M\to \eR^3$ telle que pour tout $p\in M$, le vecteur $G(p)$ soit de norme $1$ et normal à $M$ au point $p$.
\end{proposition}

%---------------------------------------------------------------------------------------------------------------------------
\subsection{Formes différentielles}
%---------------------------------------------------------------------------------------------------------------------------

Nous allons donner une toute petite introduction aux formes différentielles sur des variétés compactes.

\begin{lemma}[\cite{SpindelGeomDoff}]       \label{LemdwLGFG}
    Soit \( \omega\) une \( k\)-forme sur \( \eR^n\) et \( f\), une fonction \( C^{\infty}\) sur \( \eR^n\). Alors \( d(f^*\omega)=f^*d\omega\).
\end{lemma}

\begin{proof}
    Nous effectuons la preuve par récurrence sur le degré de la forme. Soit d'abord une \( 0\)-forme, c'est à dire une fonction \( g\colon \eR^n\to \eR\). Nous avons
    \begin{equation}
        d(d^*g)X=d(g\circ f)X=(dg\circ df)X=dg\big( df X \big)=(f^*dg)(X).
    \end{equation}
    
    Supposons maintenant que le résultat soit exact pour toute les \( p-1\) formes et montrons qu'il reste valable pour les \( p\)-formes. Par linéarité de la différentielle nous pouvons nous contenter de considérer la forme différentielle
    \begin{equation}
        \omega=g\,dx^1\wedge\ldots dx^p
    \end{equation}
    où \( g\) est une fonction \(  C^{\infty}\). Pour soulager les notations nous allons noter \( dx^I=dx^1\wedge\ldots dx^{p-1}\). Nous avons
    \begin{subequations}
        \begin{align}
            d(f^*\omega)&=d\big( f^*(gdx^I\wedge dx^p) \big)\\
            &=d\big( f^*(gdx^I)\wedge f^*dx^p \big)\\
            &=d\big( f^*(gdx^I)\big)\wedge f^*dx^p+(-1)^{p-1}f^*(gdx^I)\wedge(f^*dx^p)  \label{gnAnSt}\\
            &=f^*\big( d(gdx^I) \big)\wedge f^*dx^p      \label{xZrfjZ}\\
            &=f^*\big( d(gdx^I)\wedge dx^p \big)\\
            &=f^*d\omega        \label{loWUji}
        \end{align}
    \end{subequations}
    Justifications : \eqref{gnAnSt} est la formule de Leibnitz. \eqref{xZrfjZ} est parce que le second terme est nul : \( d(f^*dx^p)=f^*(d^2x^p)=0\). Nous avons utilisé l'hypothèse de récurrence et le fait que \( d^2=0\). L'étape \eqref{loWUji} est une utilisation à l'envers de la règle de Leibnitz en tenant compte que \( d^2x^p=0\).
\end{proof}

Soit \( M\) une variété de dimension \( n\) et \( \omega\) une \( n\)-forme différentielle
\begin{equation}
    \omega_p=f(p)dx_1\wedge\ldots\wedge dx_n.
\end{equation}
 Si \( (U,\varphi)\) est une carte (\( U\subset\eR^n\) et \( \varphi\colon U\to M\)) alors nous définissons
\begin{equation}
    \int_{\varphi(U)}\omega=\int_{U}f\big( \varphi(x) \big)dx_1\ldots dx_n.     
\end{equation}
Lorsque nous voulons intégrer sur une partie plus grande qu'une carte nous utilisons une partition de l'unité.
\begin{lemma}   \label{LemGPmRGZ}
    Soit \( \{ U_i \}\) un recouvrement de \( M\) par un nombre fini d'ouverts\footnote{Si \( M\) n'est pas compacte, alors il faut utiliser une version un peu plus élaborée du lemme\cite{SpindelGeomDoff}.}. Alors il existe une famille de fonctions \( f_i\in  C^{\infty}(M)\) telle que
    \begin{enumerate}
        \item
            \( \supp f_i\subset U_i\),
        \item
            pour tout \( i\), nous avons \( f_i\geq 0\),
        \item
            pour tout \( p\in M\) nous avons \( \sum_i f_i(p)=1\).
    \end{enumerate}
\end{lemma}
La famille \( (f_i)\) est une \defe{partition de l'unité}{partition!de l'unité} subordonnée au recouvrement \( \{ U_i \}\). 

\begin{definition}[Intégrale d'une forme sur une variété]       \label{DEFooOMQLooGiJWZS}
    Si \( \{ f_i \}\) est une partition de l'unité subordonnée à un atlas de \( M\) nous définissons
    \begin{equation}
        \int_M\omega=\sum_i\int_{U_i}f\omega.
    \end{equation}
    Il est possible de montrer que cette définition ne dépend pas du choix de la partition de l'unité.
\end{definition}

\begin{remark}
    Nous ne définissons pas d'intégrale de \( k\)-forme différentielle sur une variété de dimension \( n\neq k\). Le seul cas où cela se fait est le cas de \( 0\)-formes (les fonctions), mais cela n'est pas vraiment un cas particulier vu que les \( 0\)-formes sont associées aux \( n\)-formes de façon évidente.
\end{remark}

%---------------------------------------------------------------------------------------------------------------------------
\subsection{Intégrale d'une fonction sur une variété}
%---------------------------------------------------------------------------------------------------------------------------

Nous supposons à présent que $M$ est une variété compacte de dimension $2$ dans $\eR^3$. La compacité fait que $M$ possède un atlas contenant un nombre fini de cartes $F_i\colon W_i\to F_i(W_i)$. 

Si $A\subset M$ est tel que pour chaque $i$, $A\cap F_i(W_i)=F_i(V_i)$ pour une ensemble $V_i$ mesurable dans $\eR^2$, alors nous considérons
\begin{equation}
	A_1=A\cap F_1(W_2)=F_1(V_1).
\end{equation}
Ensuite, nous construisons $A_2$ en considérant $F_A(W_2)$ et en lui retranchant $A_1$ :
\begin{equation}
	A_2=\big( A\cap F_2(W_2) \big)\cap F_1(V_1).
\end{equation}
En continuant de la sorte, nous construisons la décomposition
\begin{equation}
	A=A_1\cup\ldots\cup A_p
\end{equation}
de $A$ en ouverts disjoints, chacun de ouverts $A_p$ étant compris dans une carte.

Il est possible de prouver que dans ce cas, la définition suivante a un sens et ne dépend pas du choix de l'atlas effectué.
\begin{definition}
	Si $f\colon A\to \eR$ est une fonction continue, alors l'\defe{intégrale}{intégrale!d'une fonction sur une variété} est le nombre
	\begin{equation}
		\int_Af=\sum_{i=1}^p\int_{A_i}fd\sigma_{F_i}.
	\end{equation}
\end{definition}

%+++++++++++++++++++++++++++++++++++++++++++++++++++++++++++++++++++++++++++++++++++++++++++++++++++++++++++++++++++++++++++ 
\section{Intégrales curvilignes}
%+++++++++++++++++++++++++++++++++++++++++++++++++++++++++++++++++++++++++++++++++++++++++++++++++++++++++++++++++++++++++++
\label{secintcurvi}

\subsection{Chemins de classe \texorpdfstring{$C^1$}{C1}}

\begin{definition}
    Soit $p, q\in \eR^n$. Un \defe{chemin}{chemin} $C^1$ par morceaux joignant $p$ à $q$ est une application continue
    \begin{equation}
      \gamma : [a,b] \to \eR^n \qquad \gamma(a) = p, \gamma(b) = q
    \end{equation}
    pour laquelle il existe une subdivision $a = t_0 < t_1 < \ldots < t_{r-1} < t_r = b$ telle que :
    \begin{enumerate}
    \item la restriction de $\gamma$ sur chaque ouvert $\mathopen]t_i, t_{i+1}\mathclose[$ est de classe $C^1$~;
    \item pour tout $0 \leq i \leq r$, $\gamma^\prime$ possède une limite à gauche (sauf pour $i = 0$) et une limite à droite (sauf pour $i = r$) en $t_i$.
    \end{enumerate}
    Le \defe{chemin $\gamma$ est (globalement) de classe $C^1$}{Chemin!classe $C^2$} si la subdivision peut être choisie de \og longueur\fg{} $r = 1$.
\end{definition}

\begin{remark}
	Si $a$ et $b$ sont des points de $\eR^n$, on peut créer le chemin particulier
  \begin{equation}
    \gamma : [0,1] \to \eR^n : t \mapsto (1-t)a + tb
  \end{equation}
  qui relie ces points par un segment de droite.
\end{remark}

\subsection{Intégrer une fonction}

\begin{definition}      \label{DEFooFAYUooCaUdyo}
    Soit $f : D \subset \eR^n \to \eR$ une fonction continue, et $\gamma : [a,b] \to D$ un chemin $C^1$. On définit \defe{l'intégrale de $f$ sur $\gamma$}{intégrale!sur un chemin} par
      \begin{equation}    \label{EqhJGRcb}
      \int_\gamma f d s = \int_\gamma f = \int_a^b f(\gamma(t)) \norme{\gamma^\prime(t)} d t.
    \end{equation}
\end{definition}

\begin{example}
    Soit l'hélice
    \begin{equation}
        \begin{aligned}
            \sigma\colon \mathopen[ 0 , 2\pi \mathclose]&\to \eR^3 \\
            t&\mapsto \begin{pmatrix}
                \cos(t)    \\ 
                \sin(t)    \\ 
                t    
            \end{pmatrix},
        \end{aligned}
    \end{equation}
    et la fonction $f(x,y,z)=x^2+y^2+z^2$. L'intégrale de $f$ sur $\sigma$ est
    \begin{equation}
        \begin{aligned}[]
            \int_{\sigma}f&=\int_0^{2\pi}(\cos^2t+\sin^2t+t^2)\| \sigma'(t) \|dt\\
            &=\int_0^{2\pi}(1+t^2)\sqrt{2}dt\\
            &=\sqrt{2}\left[ t+\frac{ t^3 }{ 3 } \right]_0^{2\pi}\\
            &=\sqrt{2}\left( 2\pi+\frac{ 8\pi^3 }{ 8 } \right).
        \end{aligned}
    \end{equation}
\end{example}

\begin{remark}
    Si $f=1$, alors nous tombons sur
    \begin{equation}
        \int_{\gamma}ds=\int_a^b\| \gamma'(t) \|dt,
    \end{equation}
    Nous verrons par le théorème \ref{ThoLongueurIntegrale} que cette dernière intégrale est la longueur de la courbe. Il est un fait général que l'intégrale de la fonction \( 1\) sur un ensemble en donne la «mesure».
    Cela est à mettre en rapport avec le lemme \ref{LemooPJLNooVKrBhN} en gardant en tête que \( \int_{\gamma}1\) n'est pas la mesure de l'image de \( \gamma\) dans \( \eR^2\).
\end{remark}

\begin{proposition}[Indépendence en la paramétrisation]
    La valeur de l'intégrale de $f$ sur $\gamma$ ne dépend pas du paramétrage (équivalent ou pas) choisi.
\end{proposition}

\begin{proof}
    Soit donc un chemin \( \gamma\colon \mathopen[ c , d \mathclose]\to \eR^3\) ainsi que $\varphi\colon \mathopen[ c , d \mathclose]\to \mathopen[ a , b \mathclose]$, une reparamétrisation de classe $C^1$, strictement monotone et le chemin \( \sigma\) définit par $\gamma(s)=\sigma\big( \varphi(s) \big)$ avec $s\in\mathopen[ c , d \mathclose]$. En supposant que $\varphi'(s)\geq 0$, nous avons
    \begin{equation}
        \begin{aligned}[]
            I=\int_{\gamma}f&=\int_c^df\big( \gamma(s) \big)\| \gamma'(s) \|ds\\
            &=\int_c^df\Big( \sigma\big( \varphi(s) \big) \Big)\| \sigma'\big( \varphi(s) \big) \| |\varphi'(s) |ds.
        \end{aligned}
    \end{equation}
    Pour cette intégrale, nous posons $t=\varphi(s)$, et par conséquent $dt=\varphi'(s)ds$. Étant donné que $\varphi'(s)\geq 0$, nous pouvons supprimer les valeurs absolues, et obtenir
    \begin{equation}
        \begin{aligned}[]
            I&=\int_{\varphi(c)}^{\varphi(d)}f\big( \sigma(t) \big)\| \sigma'(t) \|dt\\
            &=\int_a^bf\big( \sigma(t) \big)\| \sigma'(t) \|dt\\
            &=\int_{\sigma}f.
        \end{aligned}
    \end{equation}

    Essayez de faire le cas $\varphi'(s)\leq 0$. 
\end{proof}

\begin{remark}      \label{RemiqswPd}
    Attention : les intégrales sur des chemins dans \( \eC\) ne sont la même chose. En effet \( \eC\) doit être souvent plutôt traité comme \( \eR\) que comme \( \eR^2\). Si \( \gamma\) est un chemin dans \( \eC\), l'intégrale
    \begin{equation}
        \int_{\gamma}f
    \end{equation}
    doit être comprise comme une généralisation de \( \int_a^bf(x)dx\) et non comme l'intégrale sur un chemin. La différence est qu'en retournant les bornes d'une intégrale usuelle sur \( \eR\) on change le signe, alors qu'en retournant un chemin dans \( \eR^2\), on ne change pas. Bref, la définition est que si \( \gamma\colon \mathopen[ a , b \mathclose]\to \eC\) est un chemin, alors
    \begin{equation}
        \int_{\gamma}f=\int_{\gamma}f(z)dz=\int_a^bf\big( \gamma(t) \big)\gamma'(t)dt.
    \end{equation}
\end{remark}


Si on veut savoir la longueur d'une courbe donnée sous la forme d'une fonction $y=y(x)$, un chemin qui trace la courbe est évidement donné par
\begin{equation}
	\gamma(t)=(t,y(t)),
\end{equation}
et le vecteur tangent au chemin est $\gamma'(t)=(1,y'(t))$. Donc
\begin{equation}
	\| \gamma'(t) \|=\sqrt{1+y'(t)^2},
\end{equation}
et 
\begin{equation}			\label{EqLongFonction}
	L=\int_a^b\sqrt{1+y'(t)^2}.
\end{equation}


\subsection{Intégrer un champ de vecteurs}

\begin{definition}      \label{DEFooSHHFooVdsxMf}
    Un \defe{champ de vecteur}{champ!de vecteurs} est une application $G : \eR^n \to \eR^n$. On définit l'intégrale de $G$ sur un chemin $\gamma : [a,b] \to \eR^n$ par
    \begin{equation*}
      \int_\gamma G \pardef \int_a^b \scalprod {G(\gamma(t))}{\gamma^\prime(t)} d t.
    \end{equation*}
\end{definition}

\begin{remark}
  Cette définition ne dépend pas de la paramétrisation choisie, mais le signe change selon le sens du chemin.
\end{remark}



    Si $\sigma'(t)\neq 0$, nous pouvons considérer le vecteur unitaire tangent à la courbe :
    \begin{equation}
        T(t)=\frac{ \sigma'(t) }{ \| \sigma'(t) \| }.
    \end{equation}
    Si $F$ est un champ de vecteurs sur $\eR^3$, la circulation de $F$ le long de $\sigma$ sera donnée par 
    \begin{equation}
        \int_{\sigma}F\cdot ds=\int_a^b F\big( \sigma(t) \big)\cdot \sigma'(t)dt=\int_{a}^bF\big( \sigma(t) \big)\cdot\frac{ \sigma'(t) }{ \| \sigma'(t) \| }dt=\int_{\sigma} F\cdot T ds
    \end{equation}
    où dans la dernière expression, $F\cdot T$ est vu comme fonction $(x,y,z)\mapsto F(x,y,z)\cdot T(x,y,z)$. L'intégrale d'un champ de vecteurs sur une courbe n'est donc rien d'autre que l'intégrale de la composante tangentielle du champ de vecteurs.

%---------------------------------------------------------------------------------------------------------------------------
\subsection{Intégrer une forme différentielle sur un chemin}
%---------------------------------------------------------------------------------------------------------------------------

La formule d'intégration d'un champ de vecteur\footnote{Définition \ref{DEFooSHHFooVdsxMf}.},
\begin{equation}
	\int_{\gamma}G=\int_{[a,b]}\langle G (\gamma(t)), \gamma'(t)\rangle dt,
\end{equation}
contient quelque chose d'intéressant : la combinaison $\langle G( \gamma(t) ), \gamma'(t)\rangle$. Cette combinaison sert à transformer le vecteur tangent $\gamma'(t)$ en un nombre en utilisant le produit scalaire avec le vecteur $G( \gamma(t) )$.

Si $G$ est un champ de vecteur sur $\eR^n$, et si $x\in\eR^n$, nous pouvons utiliser l'isomorphisme musical (définition \ref{EqDefBemol}) 
\begin{equation}		
	\begin{aligned}[]
		G^{\flat}_x\colon \eR^n&\to \eR \\
			v&\mapsto \langle G(x), v\rangle 
	\end{aligned}
\end{equation}
pour écrire de façon plus compacte :
\begin{equation}
	\int_{\gamma}G=\int_{[a,b]} G^{\flat}_{\gamma(t)}\big( \gamma'(t)\big) dt.
\end{equation}


L'intégrale d'une forme différentielle sur un chemin est définie par
\begin{equation}    \label{EqEFIZyEe}
    \int_\gamma \omega = \int_a^b \omega_{\gamma(t)}\gamma^\prime(t) d t
\end{equation}

\begin{remark}
  Cette définition ne dépend pas de la paramétrisation choisie, mais
  le signe change selon le sens du chemin.
\end{remark}

%---------------------------------------------------------------------------------------------------------------------------
\subsection{Intégration d'une forme différentielle sur un chemin}
%---------------------------------------------------------------------------------------------------------------------------

Les formes intégrales que nous avons déjà vues sont celles de fonctions et de champs de vecteur sur des chemins. Si $\gamma\colon [a,b]\to \eR^n$ est le chemin, les formules sont
\begin{equation}
	\begin{aligned}[]
		\int_{\gamma}f&=\int_{[a,b]}f\big( \gamma(t) \big)\| \gamma'(t) \|dt\\
		\int_{\gamma}G&=\int_{[a,b]}\langle G\big( \gamma(t) \big), \gamma'(t)\rangle dt.
	\end{aligned}
\end{equation}
Dans les deux cas, le principe est que nous disposons de quelque chose (la fonction $f$ ou le vecteur $G$), et du vecteur tangent $\gamma'(t)$, et nous essayons d'en tirer un nombre que nous intégrons. Lorsque nous avons une $1$-forme, la façon de l'utiliser pour produire un nombre avec le vecteur tangent est évidement d'appliquer la $1$-forme au vecteur tangent. La définition suivante est donc naturelle.

\begin{definition}
	Soit $\gamma\colon [a,b]\to \eR^n$, un chemin de classe $C^1$ tel que son image est contenue dans le domaine $D$. Si $\omega$ es une $1$-forme différentielle sur $D$, nous définissons l'\defe{intégrale de $\omega$ le long de $\gamma$}{intégrale!d'une forme différentielle} le nombre
	\begin{equation}
		\begin{aligned}[]
			\int_{\gamma}\omega&=\int_a^b\omega_{\gamma(t)}\big( \gamma'(t) \big)dt\\
				&=\int_a^b\Big[ a_1\big( \gamma(t) \big)\gamma'_1(t)+\cdots +  a_n\big( \gamma(t) \big)\gamma'_n(t) \Big]dt.
		\end{aligned}
	\end{equation}
\end{definition}

Cette définition est une bonne définition parce que si on change la paramétrisation du chemin, on ne change pas la valeur de l'intégrale, c'est la proposition suivante.
\begin{proposition}
	Si $\gamma$ et $\beta$ sont des chemins équivalents, alors
	\begin{equation}
		\int_{\gamma}\omega=\int_{\beta}\omega,
	\end{equation}
	c'est à dire que l'intégrale est invariante sous les reparamétrisation du chemin.
\end{proposition}
\begin{proof}
	Deux chemins sont équivalents quand il existe un difféomorphisme $C^1$ $h\colon [a,b]\to [c,d]$ tel que $\gamma(t)=(\beta\circ h)(t)$. En remplaçant $\gamma$ par $(\beta\circ h)$ dans la définition de $\int_{\gamma}\omega$, nous trouvons
	\begin{equation}
		\int_a^b\omega_{\gamma(t)}\big( \gamma'(t) \big)dt=\int_a^b\omega_{(\beta\circ h)(t)}\big( (\beta\circ h)'(t) \big)dt.
	\end{equation}
	Un changement de variable $u=h(t)$ transforme cette dernière intégrale en $\int_{\beta}\omega$, ce qui prouve la proposition.
\end{proof}

\begin{remark}
	Si $\gamma$ est une somme de chemins, $\gamma=\gamma^{(1)}+\cdots+\gamma^{(n)}$, où chacun des $\gamma^{(i)}$ est un chemin, alors
	\begin{equation}
		\int_{\gamma}\omega=\sum_{i=1}^n\int_{\gamma_i}\omega
	\end{equation}
	parce que $\omega$ est linéaire.
\end{remark}

\begin{remark}
	Si $-\gamma$ est le chemin
	\begin{equation}
		\begin{aligned}
			- \gamma\colon [a,b]&\to \eR^n \\
			t&\mapsto \gamma\big( b-(t-a) \big),
		\end{aligned}
	\end{equation}
	alors
	\begin{equation}
		\int_{-\gamma}\omega=-\int_{\gamma}\omega,
	\end{equation}
	c'est à dire que si l'on parcours le chemin en sens inverse, alors on change le signe de l'intégrale.
\end{remark}

L'intégrale d'une forme différentielle sur un chemin est compatible avec l'intégrale déjà connue d'un champ de vecteur sur le chemin parce que si $G$ est un champ de vecteurs,
\begin{equation}
	\int_{\gamma}G^{\flat}=\int_{\gamma}G.
\end{equation}
En effet,
\begin{equation}
	\begin{aligned}[]
		\int_{\gamma G^{\flat}}	&=\int_a^b G_{\gamma(t)}^{\flat}(\gamma'(t))\\
					&=\int_a^b\big[ G_1( \gamma(t) )dx_1+\ldots G_n(\gamma(t))dx_n \big]\big( \gamma'_1(t),\ldots,\gamma'_n(t) \big)\\
					&=\int_{a}^b\langle G(\gamma(t)), \gamma'(t)\rangle \\
					&=\int_{\gamma}G.
	\end{aligned}
\end{equation}


\begin{proposition}
	Soit $\omega=df$, une $1$-forme exacte et continue sur le domaine $D$. Alors la valeur de $\int_{\gamma}df$ ne dépend que des valeurs de $f$ aux extrémités de $\gamma$.
\end{proposition}

\begin{proof}
	Nous avons
	\begin{equation}
		\begin{aligned}[]
			\int_{\gamma}\omega=\int_{\gamma}df&=\int_{a}^b\sum_{i=1}n\frac{ \partial f }{ \partial x_i }\big( \gamma(t) \big)\gamma'_i(t)dt\\
				&=\int_a^b\frac{ d }{ dt }\Big( (f\circ\gamma)(t) \Big)dt\\
				&=(f\circ\gamma)(b)-(f\circ\gamma(a)).
		\end{aligned}
	\end{equation}
\end{proof}

%---------------------------------------------------------------------------------------------------------------------------
\subsection{Interprétation physique : travail}
%---------------------------------------------------------------------------------------------------------------------------

\begin{definition}
	Une force $F\colon D\subset\eR^n\to \eR^n$ est
    \defe{\href{http://fr.wikipedia.org/wiki/Force_conservative}{conservative}}{Conservative}
    si elle dérive d'un potentiel, c'est à dire s'il existe une fonction $V\in C^1(D,\eR)$ telle que 
	\begin{equation}
		F(x)=(\nabla V)(x).
	\end{equation}
\end{definition}
Étant donné que $F$ est un champ de vecteurs, nous avons une forme différentielle associée $F^{\flat}$,
\begin{equation}
	F^{\flat}_x\colon x\mapsto \langle F(x), v\rangle .
\end{equation}

\begin{lemma}
	Le champ $F$ est conservatif si et seulement si la $1$-forme différentielle $F^{\flat}$ est exacte.
\end{lemma}

\begin{proof}
	Supposons que la force $F$ soit conservative, c'est à dire qu'il existe une fonction $V$ telle que $F=\nabla V$. Dans ce cas, il est facile de prouver que $F^{\flat}$ est exacte et est donnée par $F_x^{\flat}=dV(x)$. En effet,
	\begin{equation}
		\begin{aligned}[]
			F_x^{\flat}(v)	&=\langle F(x), v\rangle \\
					&=F_1(x)v_1+\cdots+F_n(x)v_n\\
					&=\frac{ \partial V }{ \partial x_1 }(x)v_1+\ldots\frac{ \partial V }{ \partial x_n }(x)v_n\\
					&=dV(x)v.
		\end{aligned}
	\end{equation}
	
	Pour le sens inverse, supposons que $F^{\flat}$ soit exacte. Dans ce cas, nous avons une fonction $V$ telle que $F^{\flat}=dV$. Il est facile de prouver qu'alors, $F=\nabla V$.
\end{proof}
En résumé, nous avons deux façons équivalentes d'exprimer que la force $F$ dérive du potentiel $V$ :  soit nous disons $F=\nabla V$, soit nous disons $F^{\flat}=dV$.

\begin{proposition}
	Si $F$ est une force conservative, alors le \href{http://fr.wikipedia.org/wiki/Travail_d'une_force}{travail} de $F$ lors d'un déplacement ne dépend pas du chemin suivit.
\end{proposition}

\begin{proof}
	Le travail d'une force le long d'un chemin n'est autre que l'intégrale de la force le long du chemin, et le calcul est facile :
	\begin{equation}
		W_{\gamma}(F)=\int_{\gamma}F=\int_{\gamma}dV=V\big( \gamma(b) \big)-V\big( \gamma(a) \big).
	\end{equation}
	Donc si $\beta$ est un autre chemin tel que $\beta(a)=\gamma(a)$ et $\beta(b)=\gamma(b)$, nous avons $W_{\beta}(F)=W_{\gamma}(F)$.
\end{proof}

%---------------------------------------------------------------------------------------------------------------------------
\subsection{Intégrer un champs de vecteurs sur un bord en $2D$}
%---------------------------------------------------------------------------------------------------------------------------

Si $D\subset\eR^2$ est tel que $\partial D$ est une variété de dimension $1$ et tel que $D$ accepte un champ de vecteur normal extérieur unitaire $\nu$. Si nous voulons définir 
\begin{equation}
	\int_{\partial D}G,
\end{equation}
le mieux est de prendre une paramétrisation $\gamma\colon \mathopen[ 0 , 1 \mathclose]\to \eR^2$ et de calculer
\begin{equation}
	\int_0^1 \langle G_{\gamma(t)}, \frac{ \dot\gamma(t) }{ \| \dot\gamma(t) \| }\rangle dt.
\end{equation}
Hélas, cette définition ne fonctionne pas parce que son signe dépend du sens de la paramétrisation $\gamma$. Si la paramétrisation tourne dans l'autre sens, il y a un signe de différence.

Nous allons définir
\begin{equation}		\label{EqIntVectbordDeux}
	\int_{\partial D}G=\int_0^1\langle G_{\gamma(t)}, T(t)\rangle dt
\end{equation}
où $T(t)=\dot\gamma(t)/\| \dot\gamma(t) \|$ et où $\gamma$ est choisit de telle façon que la rotation d'angle $\frac{ \pi }{ 2 }$ amène $\nu$ sur $T$. Cela fixe le choix de sens.

Ce choix de sens aura des répercussions dans l'application de la formule de Green et du théorème de Stokes.

%---------------------------------------------------------------------------------------------------------------------------
\subsection{Intégrer une forme différentielle sur un bord en $2D$}
%---------------------------------------------------------------------------------------------------------------------------

Nous n'allons pas chercher très loin :
\begin{equation}
	\int_{\partial D}\omega=\int_{\partial D}\omega^{\sharp},
\end{equation}
c'est à dire que l'intégrale de la forme différentielle est celle du champ de vecteur associé. Le membre de droite est définit par \eqref{EqIntVectbordDeux}, avec le choix d'orientation qui va avec.

%---------------------------------------------------------------------------------------------------------------------------
\subsection{Intégrer une forme différentielle sur un bord en $3D$}
%---------------------------------------------------------------------------------------------------------------------------

Nous allons maintenant intégrer une forme différentielle sur certains chemins fermés dans $\eR^3$. Soit $F(D)\subset\eR^3$, une variété de dimension $2$ dans $\eR^3$ où $F\colon D\subset\eR^2\to \eR^3$ est la carte. Nous supposons que $D$ vérifie les hypothèses de la formule de Green. Alors nous définissons
\begin{equation}		\label{EqDefIntTroisForBord}
	\int_{F(\partial D)}\omega = \int_{\partial D} F^*\omega
\end{equation}
où $F^*\omega$ est la forme différentielle définie sur $\partial D$ par $(F^*\omega)(v)=\omega\big( dF(v) \big)$.

Cette définition est très abstraite, mais nous n'allons, en pratique, jamais l'utiliser, grâce au théorème de Stokes.

%---------------------------------------------------------------------------------------------------------------------------
\subsection{Intégrer d'un champ de vecteurs sur un bord en $3D$}
%---------------------------------------------------------------------------------------------------------------------------

Encore une fois, nous n'allons pas chercher bien loin :
\begin{equation}
	\int_{F(\partial D)}G=\int_{F(\partial D)}G^{\flat}
\end{equation}
où $G^{\flat}$ est la forme différentielle associée au champ de vecteur. Le membre de droite est définit par l'équation \eqref{EqDefIntTroisForBord}.

%---------------------------------------------------------------------------------------------------------------------------
\subsection{Dérivées croisées et forme différentielle exacte}
%---------------------------------------------------------------------------------------------------------------------------

Nous considérons le problème suivant : trouver une fonction \( f\colon \eR^2\to \eR\) telle que
\begin{subequations}        \label{EqskfgfNr}
    \begin{numcases}{}
        \frac{ \partial f }{ \partial x }=a(x,y)\\
        \frac{ \partial f }{ \partial y }=b(x,y)
    \end{numcases}
\end{subequations}
où \( a\) et \( b\) sont des fonctions supposées suffisamment régulières. Nous savons que ce problème n'a pas de solutions lorsque
\begin{equation}
    \frac{ \partial a }{ \partial y }\neq\frac{ \partial b }{ \partial x }
\end{equation}
parce que cela impliquerait \( \partial^2_{xy}f\neq \partial^2_{yx}f\). Nous sommes en droit de nous demander si la condition
\begin{equation}
    \frac{ \partial a }{ \partial y }=\frac{ \partial b }{ \partial x }
\end{equation}
impliquerait qu'il existe une solution au problème \eqref{EqskfgfNr}. La réponse est oui, et nous allons brièvement la justifier. Pour plus de détails nous vous demandons de chercher un peu \href{http://www.bing.com/search?q=forme+diff\%C3\%A9rentielle+exacte+filetype\%3Apdf&form=QBRE&fit=all}{sur internet} les mots-clefs \emph{forme différentielles exacte}. Vous consulterez également avec profit \cite{DiffExact}.

\begin{proposition}
    Si \( a\) et \( b\) sont des fonctions qui satisfont à la condition
    \begin{equation}
        \frac{ \partial a }{ \partial y }=\frac{ \partial b }{ \partial x },
    \end{equation}
    alors la fonction
    \begin{equation}        \label{EqllhTaT}
        f(x,y)=\int_0^x a(t,0)dt+\int_0^yb(x,t)dt
    \end{equation}
    répond au problème
    \begin{subequations}     
        \begin{numcases}{}
            \frac{ \partial f }{ \partial x }=a(x,y)\\
            \frac{ \partial f }{ \partial y }=b(x,y)
        \end{numcases}
    \end{subequations}
\end{proposition}

La preuve qui suit n'en est pas complètement une parce qu'il manque des justification, notamment au moment de permuter la dérivée et l'intégrale.
\begin{proof}
    La clef de la preuve est le théorème fondamental de l'analyse :
    \begin{equation}
        \int_0^x \frac{ \partial f }{ \partial x }(t,y)dt=f(x,y)
    \end{equation}
    et son pendant par rapport à \( y\) :
    \begin{equation}
        \int_0^y \frac{ \partial f }{ \partial y }(x,t)dt=f(x,y).
    \end{equation}
    En appliquant ces version du théorème fondamental, nous obtenons immédiatement.
    \begin{equation}
        \frac{ \partial f }{ \partial y }=b(x,y).
    \end{equation}
    En ce qui concerne la dérivée par rapport à \( y\),
    \begin{subequations}
        \begin{align}
            \frac{ \partial f }{ \partial x }&=a(x,0)+\int_0^y\frac{ \partial b }{ \partial x }(x,t)dt\\
            &=a(x,0)+\int_0^y\frac{ \partial a }{ \partial y }(x,t)dt\\
            &=a(x,0)+[a(x,t)]_{t=0}^{t=y}\\
            &=a(x,y).
        \end{align}
    \end{subequations}
\end{proof}

En ce qui concerne l'unicité, supposons que \( f\) et \( g\) soient deux solutions au problème. L'équation
\begin{equation}
    \frac{ \partial f }{ \partial x }=a(x,y)=\frac{ \partial g }{ \partial x }
\end{equation}
implique que 
\begin{equation}
    f(x,y)=g(x,y)+C(y)
\end{equation}
où \( C\) est une fonction seulement de \( y\). L'autre équation implique
\begin{equation}
    f(x,y)=g(x,y)+D(x)
\end{equation}
où \( D\) est seulement une fonction de \( x\). En égalisant nous voyons que les fonctions \( C\) et \( D\) doivent être des constantes.

Par conséquent la fonction \( f\) est donnée à une constante près et en réalité la fonction \eqref{EqllhTaT} est suffisante pour répondre au problème de trouver toutes les fonctions dont les dérivées partielles sont données par les fonctions \( a\) et \( b\).

La fonction \( f\) ainsi créée est un \defe{potentiel}{potentiel} pour le champ de force
\begin{equation}
    F(x,y)=\begin{pmatrix}
        a(x,y)    \\ 
        b(x,y)  
    \end{pmatrix}.
\end{equation}
Notez que ce champ de vecteurs est le gradient de \( f\). La question initiale aurait donc pu être posée en les termes suivants : trouver une fonction \( f\) dont le gradient est donné par
\begin{equation}
    \nabla f=\begin{pmatrix}
        a(x,y)    \\ 
        b(x,y)    
    \end{pmatrix}.
\end{equation}
