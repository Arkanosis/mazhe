% This is part of Mes notes de mathématique
% Copyright (c) 2011-2015
%   Laurent Claessens
% See the file fdl-1.3.txt for copying conditions.

%+++++++++++++++++++++++++++++++++++++++++++++++++++++++++++++++++++++++++++++++++++++++++++++++++++++++++++++++++++++++++++
\section{Recherche d'extrema}
%+++++++++++++++++++++++++++++++++++++++++++++++++++++++++++++++++++++++++++++++++++++++++++++++++++++++++++++++++++++++++++

%---------------------------------------------------------------------------------------------------------------------------
\subsection{Extrema à une variable}
%---------------------------------------------------------------------------------------------------------------------------

\begin{definition}
Soit $f\colon A\subset \eR\to \eR$ et $a\in A$. Le point $a$ est un \defe{maximum local}{maximum!local} de $f$ s'il existe un voisinage $\mU$ de $a$ tel que $f(a)\geq f(x)$ pour tout $x\in\mU\cap A$. Le point $a$ est un \defe{maximum global}{maximum!global} si $f(a)\geq g(x)$ pour tout $x\in A$.
\end{definition}

La proposition basique à utiliser lors de la recherche d'extrema est la suivante :
\begin{proposition}
Soit $f\colon A\subset\eR\to \eR$ et $a\in\Int(A)$. Supposons que $f$ est dérivable en $a$. Si $a$ est un \href{http://fr.wikipedia.org/wiki/Extremum}{extremum} local, alors $f'(a)=0$.
\end{proposition}

La réciproque n'est pas vraie, comme le montre l'exemple de la fonction $x\mapsto x^3$ en $x=0$ : sa dérivée est nulle et pourtant $x=0$ n'est ni un maximum ni un minimum local. 

Cette proposition ne sert donc qu'à sélectionner des \emph{candidats} extremum. Afin de savoir si ces candidats sont des extrema, il y a la proposition suivante.
\begin{proposition}
Soit $f\colon I\subset \eR\to \eR$, une fonction de classe $C^k$ au voisinage d'un point $a\in\Int I$. Supposons que
\begin{equation}
    f'(a)=f''(a)=\ldots=f^{(k-1)}(a)=0,
\end{equation}
et que
\begin{equation}
    f^{(k)}(a)\neq 0.
\end{equation}
Dans ce cas,
\begin{enumerate}

\item
Si $k$ est pair, alors $a$ est un point d'extremum local de $f$, c'est un minimum si $f^{(k)}(a)>0$, et un maximum si $f^{(k)}(a)<0$,
\item
Si $k$ est impair, alors $a$ n'est pas un extremum local de $f$.

\end{enumerate}
\end{proposition}

Note : jusqu'à présent nous n'avons rien dit des extrema \emph{globaux} de $f$. Il n'y a pas grand chose à en dire. Si un point d'extremum global est situé dans l'intérieur du domaine de $f$, alors il sera extremum local (a fortiori). Ou alors, le maximum global peut être sur le bord du domaine. C'est ce qui arrive à des fonctions strictement croissantes sur un domaine compact.

Une seule certitude : si une fonction est continue sur un compact, elle possède une minimum et un maximum global par le théorème \ref{ThoMKKooAbHaro}.
 
%---------------------------------------------------------------------------------------------------------------------------
\subsection{Extrema libre}
%---------------------------------------------------------------------------------------------------------------------------

Un point $a$ à l'intérieur du domaine d'une fonction $f\colon A\subset\eR^n\to \eR$ est un \defe{point critique}{critique!point} de $f$ lorsque $df(a)=0$. Ces points sont analogues aux points où la dérivée d'une fonction sur $\eR$ s'annule. Les points critiques de $f$ sont dons les candidats à être des points d'extremum.

Pour rappel, dans le cas d'une fonction à deux variables, $d^2f(a)$ est la matrice (et donc l'application linéaire)
\begin{equation}
    d^2f(a)=\begin{pmatrix}
    \frac{ d^2f  }{ dx^2 }(a)   &   \frac{ d^2f  }{ dx\,dy }(a) \\ 
    \frac{ d^2f  }{ dy\,dx }(a)     &   \frac{ d^2f  }{ dy^2 }(a)
\end{pmatrix}.
\end{equation}
Dans le cas d'une fonction $C^2$, cette matrice est symétrique.

\begin{proposition} \label{PropUQRooPgJsuz}
    Soit $f\colon A\subset\eR^n\to \eR$ une fonction de classe $C^1$ au voisinage de $a\in\Int(A)$. Si \( a\) est un minimum local alors \( a\) est un point critique de \( f\).
\end{proposition}

\begin{proposition}     \label{PropoExtreRn}
    Soit $f\colon A\subset\eR^n\to \eR$ une fonction de classe $C^2$ au voisinage de $a\in\Int(A)$.
    \begin{enumerate}
        \item
            Si $a$ est un point critique de $f$, et si $d^2f(a)$ est \href{http://fr.wikipedia.org/wiki/Matrice_définie_positive}{définie positive\footnote{La fonction \( f\) est de classe \( C^2\), donc les dérivées croisées sont égales et \( d^2f\) est symétrique. La définition \ref{DefAWAooCMPuVM} s'applique donc.}}, alors $a$ est un minimum local strict de $f$,
        \item\label{ItemPropoExtreRn}
            Si $a$ est un minimum local, alors $d^2f(a)$ est définie positive.
    \end{enumerate}
\end{proposition}
\index{fonction!différentiable}
\index{extrema}
La seconde partie de l'énoncé est tout à fait comparable au fait bien connu que, pour une fonction $f\colon \eR\to \eR$, si le point $a$ est minimum local, alors $f'(a)=0$ et $f''(a)>0$.

La méthode pour chercher les extrema de $f$ est donc de suivre le points suivants :
\begin{enumerate}
    \item
        Trouver les candidats extrema en résolvant $\nabla f=(0,0)$,
    \item
        écrire $d^2f(a)$ pour chacun des candidats
    \item
        calculer les valeurs propres de $d^2f(a)$, déterminer si la matrice est définie positive ou négative,
    \item
        conclure.
\end{enumerate}

Une conséquence de la proposition \ref{PropcnJyXZ}\ref{ItemluuFPN}\footnote{La matrice $d^2f(a)$ est toujours symétrique quand $f$ est de classe $C^2$.} est que si \( \det M<0\), alors le point \( a\) n'est pas  un extrema dans le cas où $M=d^2f(a)$ par le point \ref{ItemPropoExtreRn} de la proposition \ref{PropoExtreRn}.

\begin{example}
    Soit la fonction \( f(x,y)=x^4+y^4-4xy\). C'est une fonction différentiable sans problèmes. D'abord sa différentielle est
    \begin{equation}
        df=\big(4x^3-4y;4y^3-4x),
    \end{equation}
    et la matrice des dérivées secondes est
    \begin{equation}
        M=d^2f(x,y)=\begin{pmatrix}
            12x^2    &   -4    \\ 
            -4    &   12y^2    
        \end{pmatrix}.
    \end{equation}
    Nous avons \( fd=0\) pour les trois points \( (0,0)\), \( (1,1)\) et \( -1,-1\).

    Pour le point \( (0,0)\) nous avons
    \begin{equation}
        M=\begin{pmatrix}
            0    &   -4    \\ 
            -4    &   0    
        \end{pmatrix},
    \end{equation}
    dont les valeurs propres sont \( 4\) et \( -4\). Elle n'est donc semi-définie ou définie rien du tout. Donc \( (0,0)\) n'est pas un extremum local.

    Au contraire pour les points \( (1,1)\) et \( (-1,-1)\) nous avons
    \begin{equation}
        M=\begin{pmatrix}
            12    &   -4    \\ 
            -4    &   12    
        \end{pmatrix},
    \end{equation}
    dont les valeurs propres sont \( 16\) et \( 8\). La matrice \( d^2f\) y est donc définie positive. Ces deux points sont donc extrema locaux.
\end{example}

%--------------------------------------------------------------------------------------------------------------------------- 
\subsection{Extrema liés}
%---------------------------------------------------------------------------------------------------------------------------

Soit $f$, une fonction sur $\eR^n$, et $M\subset \eR^n$ une variété de dimension $m$. Nous voulons savoir quelle sont les plus grandes et plus petites valeurs atteintes par $f$ sur $M$.

Pour ce faire, nous avons un théorème qui permet de trouver des extrema \emph{locaux} de $f$ sur la variété. Pour rappel, $a\in M$ est une \defe{extrema local de $f$ relativement}{extrema!local!relatif} à l'ensemble $M$ s'il existe une boule $B(a,\epsilon)$ telle que $f(a)\leq f(x)$ pour tout $x\in B(a,\epsilon)\cap M$.

\begin{theorem}[Extrema lié \cite{ytMOpe}] \label{ThoRGJosS}
    Soit \( A\), un ouvert de \( \eR^n\) et
    \begin{enumerate}
        \item
            une fonction (celle à minimiser) $f\in C^1(A,\eR)$,
        \item 
            des fonctions (les contraintes) $G_1,\ldots,G_r\in C^1(A,\eR)$,
        \item
            $M=\{ x\in A\tq G_i(x)=0\,\forall i\}$,
        \item
            un extrema local $a\in M$ de $f$ relativement à $M$.
    \end{enumerate}
    Supposons que les gradients $\nabla G_1(a)$, \ldots,$\nabla G_r(a)$ soient linéairement indépendants. Alors $a=(x_1,\ldots,x_n)$ est une solution de \( \nabla L(a)=0\) où
    \begin{equation}
        L(x_1,\ldots,x_n,\lambda_1,\ldots,\lambda_r)=f(x_1,\ldots,x_n)+\sum_{i=1}^r\lambda_iG_i(x_1,\ldots,x_n).
    \end{equation}
    Autrement dit, si \( a\) est un extrema lié, alors \( \nabla f(a)\) est une combinaisons des \( \nabla G_i(a)\), ou encore il existe des \( \lambda_i\) tels que
    \begin{equation}    \label{EqRDsSXyZ}
        df(a)=\sum_i\lambda_idG_i(a).
    \end{equation}
\end{theorem}
\index{théorème!inversion locale!utilisation}
\index{extrema!lié}
\index{théorème!extrema!lié}
\index{application!différentiable!extrema lié}
\index{variété}
\index{rang!différentielle}
\index{forme!linéaire!différentielle}
La fonction $L$ est le \defe{lagrangien}{lagrangien} du problème et les variables \( \lambda_i\) sont les \defe{multiplicateurs de Lagrange}{multiplicateur!de Lagrange}\index{Lagrange!multiplicateur}.

\begin{proof}
    Si \( r=n\) alors les vecteurs linéairement indépendantes \( \nabla G_i(a) \) forment une base de \( \eR^n\) et donc évidemment les \( \lambda_i\) existent. Nous supposons donc maintenant que \( r<n\). Nous notons \( (z_i)_{i=1\ldots n}\) les coordonnées sur \( \eR^n\).
    
    La matrice
    \begin{equation}
        \begin{pmatrix}
            \frac{ \partial G_1 }{ \partial z_1 }(a)    &   \cdots    &   \frac{ \partial G_1 }{ \partial z_n }(a)    \\
            \vdots    &   \ddots    &   \vdots    \\
            \frac{ \partial G_r }{ \partial z_1 }(a)    &   \cdots    &   \frac{ \partial G_r }{ \partial z_n }(a)
        \end{pmatrix}
    \end{equation}
    est de rang \( r\) parce que les lignes sont par hypothèses linéairement indépendantes. Nous nommons \( (y_i)_{i=1,\ldots, r}\) un choix de \( r\) parmi les \( (z_i)\) tels que
    \begin{equation}
        \det\begin{pmatrix}
            \frac{ \partial G_1 }{ \partial y_1 }    &   \ldots    &   \frac{ \partial G_1 }{ \partial y_r }    \\
            \vdots    &   \ddots    &   \vdots    \\
            \frac{ \partial G_r }{ \partial y_1 }    &   \ldots    &   \frac{ \partial G_r }{ \partial y_r }
        \end{pmatrix}\neq 0.
    \end{equation}
    Nous identifions \( \eR^n\) à \( \eR^s\times \eR^r\) dans lequel \( \eR^r\) est la partie générée par les \( (y_i)_{i=1,\ldots, r}\). Nous nommons \( (x_j)_{j=1,\ldots, s}\) les coordonnées sur \( \eR^s\). Autrement dit, les coordonnées sur \( \eR^n\) sont \( x_1,\ldots, x_s,y_1,\ldots, y_r\). Dans ces coordonnées, nous nommons \( a=(\alpha,\beta)\) avec \( \alpha\in \eR^s\) et \( \beta\in \eR^r\).

    Si nous notons \( G=(G_1,\ldots, G_r)\), le théorème de la fonction implicite (théorème \ref{ThoAcaWho})  nous dit qu'il existe un voisinage \( U'\) de \( \alpha\in \eR^n\), un voisinage \( V'\) de \( \beta\in \eR^r\) et une fonction \( \varphi\colon U'\to V'\) de classe \( C^1\) telle que si \( (x,y)\in U'\times V'\), alors
    \begin{equation}
        g(x,y)=0
    \end{equation}
    si et seulement si \( y=\varphi(x)\). Nous posons maintenant
    \begin{subequations}
        \begin{align}
            \psi(x)&=(x,\varphi(x))\\
            h(x)&=f\big( \psi(x) \big).
        \end{align}
    \end{subequations}
    Nous avons \( \psi(\alpha)=a\) et \( \psi(x)\in M\) pour tout \( x\in U'\). La fonction \( h\) a donc un extrema local en \( \alpha\) et donc les dérivées partielles de \( h\) y sont nulles. Cela signifie que
    \begin{equation}
        0=\frac{ \partial h }{ \partial x_i }(\alpha)=\sum_{j=1}^n\frac{ \partial f }{ \partial x_j }\frac{ \partial x_j }{ \partial x_i }+\sum_{k=1}^r\frac{ \partial f }{ \partial y_k }\frac{ \partial \varphi_k }{ \partial x_i },
    \end{equation}
    c'est à dire
    \begin{equation}
        \frac{ \partial f }{ \partial x_i }(\alpha)+\sum_{k=1}^r\frac{ \partial f }{ \partial y_k }(a)\frac{ \partial \varphi_k }{ \partial x_i }(\alpha)=0
    \end{equation}
    pour tout \( i=1,\ldots, s\). D'autre part pour tout $k$, la fonction \( l_k(x)=G_k\big( x,\varphi(x) \big)\) est constante et vaut zéro; ses dérivées partielles sont donc nulles :
    \begin{equation}
        \frac{ \partial l }{ \partial x_i }(\alpha)=\frac{ \partial G_k }{ \partial x_i }(\alpha)+\sum_{k=1}^r\frac{ \partial G_k }{ \partial y_k }(a)\frac{ \partial \varphi_k }{ \partial x_i }(\alpha)=0
    \end{equation}
    pour tout \( i=1,\ldots, s\) et \( k=1,\ldots, r\).
    
    Les \( s\) premières colonnes de la matrice
    \begin{equation}
        \begin{pmatrix}
            \frac{ \partial f }{ \partial x_1 }   &   \cdots    &   \frac{ \partial f }{ \partial x_s }    &   \frac{ \partial f }{ \partial y_1 }    &   \cdots    &   \frac{ \partial f }{ \partial y_r }\\  
            \frac{ \partial G_1 }{ \partial x_1 }    &   \cdots    &   \frac{ \partial G_1 }{ \partial x_s }    &   \frac{ \partial G_1 }{ \partial y_1 }    &   \cdots    &   \frac{ \partial G_1 }{ \partial y_r }\\
            \vdots    &   \vdots    &   \vdots    &   \vdots    &   \vdots    &   \vdots\\
            \frac{ \partial G_r }{ \partial x_1 }    &   \cdots    &   \frac{ \partial G_r }{ \partial x_s }    &   \frac{ \partial G_r }{ \partial y_1 }    &  \cdots   & \frac{ \partial G_r }{ \partial y_r }  
        \end{pmatrix}
    \end{equation}
    s'expriment en terme des \( r\) dernières. La matrice est donc au maximum de rang \( r\). Notons que la première ligne est \( \nabla f\) et les \( r\) suivantes sont les \( \nabla G_i\). Vu que ces lignes sont des vecteurs liés, il existe \( \mu_0,\ldots, \mu_r\) tels que
    \begin{equation}
        \mu_0\nabla f+\sum_{i=1}^r\mu_i\nabla G_i=0.
    \end{equation}
    Par hypothèse les \( \nabla G_i\) sont linéairement indépendants, ce qui nous dit que \( \mu_0\neq 0\). Donc nous avons ce qu'il nous faut :
    \begin{equation}
        \nabla f(a)=\sum_i\frac{ \mu_i }{ \mu_0 } \nabla G_i(a).
    \end{equation}

    Notons qu'au vu de l'expression \eqref{EqRDsSXyZ}, le fait que les formes \( \{ dG_i(a) \}_{1\leq i\leq r}\) forment une partie libre dans \( (\eR^n)^*\) implique que les \( \lambda_i\) sont uniques.
\end{proof}

La proposition suivante est la même que \ref{ThoRGJosS}.
\begin{proposition} \label{PropfPPUxh}
    Soit \( U\), un ouvert de \( \eR^n\) et des fonctions de classe \( C^1\) \( f,g_1,\ldots, g_r\colon U\to \eR\). Nous considérons
    \begin{equation}
        \Gamma=\{ x\in U\tq g_1(x)=\ldots=g_r(x)=0 \}.
    \end{equation}
    Soit \( a\) un extrémum de \( f|_{\Gamma}\). Supposons que les formes \( dg_1,\ldots, dg_r\) soient linéairement indépendantes en \( a\). Alors il existe \( \lambda_1,\ldots, \lambda_r\) dans \( \eR\) tel que
    \begin{equation}
        df_a=\sum_{i=1}^r\lambda_i(dg_i)_a.
    \end{equation}
\end{proposition}

En pratique les candidats extrema locaux sont tous les points où les gradients ne sont pas linéairement indépendants, plus tous les points donnés par l'équation $\nabla L=0$. Parmi ces candidats, il faut trouver lesquels sont maxima ou minima, locaux ou globaux.

L'existence d'extrema locaux se prouve généralement en invoquant de la compacité, et en invoquant le lemme suivant qui permet de réduire le problème à un compact.

\begin{lemma}       \label{LemmeMinSCimpliqueS}
    Soit $S$, une partie de $\eR^n$ et $C$, un ouvert de $\eR^n$. Si $a\in\Int S$ est un minimum local relatif à $S\cap C$, alors il est un minimum local par rapport à $S$.
\end{lemma}

\begin{proof}
    Nous avons que $\forall x\in B(a,\epsilon_1)\cap S\cap C$, $f(x)\geq f(x)$. Mais étant donné que $C$ est ouvert, et que $a\in C$, il existe un $\epsilon_2$ tel que $B(a,\epsilon_2)\subset C$. En prenant $\epsilon=\min\{ \epsilon_1,\epsilon_2 \}$, nous trouvons que $f(x)\geq f(a)$ pour tout $x\in B(a,\epsilon)\cap(S\cap C)=B(a,\epsilon)\cap S$.
\end{proof}

%--------------------------------------------------------------------------------------------------------------------------- 
\subsection{Algorithme du gradient à pas optimal}
%---------------------------------------------------------------------------------------------------------------------------

Une idée pour trouver un minimum à une fonction est de prendre un point \( p\) au hasard, calculer le gradient \(\nabla f(p) \) et suivre la direction \(-\nabla f(p)\) tant que ça descend. Une fois qu'on est «dans le creux», recalculer le gradient et continuer ainsi. Nous allons détailler cet algorithme dans un cas très particulier.

\begin{definition}  \label{DefQXPooYSygGP}
    Si \( X\) est un espace vectoriel normé et \( f\colon X\to \eR\cup\{ \pm\infty \}\) nous disons que \( f\) est \defe{coercive}{coercive} sur le domaine non borné \( P\) de \( X\) si pour tout \( M\in \eR\), l'ensemble
    \begin{equation}
        \{ x\in P\tq f(x)\leq M \}
    \end{equation}
    est borné.
\end{definition}
En langage imagé la coercivité de \( f\) s'exprime par la limite
\begin{equation}
    \lim_{\substack{\| x \|\to \infty\\x\in P}}f(x)=+\infty.
\end{equation}

\begin{proposition}
    Soit \( A\in S^{++}(n,\eR)\) et \( b\in \eR^n\). Nous considérons l'application
    \begin{equation}
        \begin{aligned}
            f\colon \eR^n&\to \eR \\
            x&\mapsto \frac{ 1 }{2}\langle Ax, x\rangle +\langle b, x\rangle . 
        \end{aligned}
    \end{equation}
    Il existe un unique \( \bar x\in \eR^n\) minimisant \( f\) et il vérifie \( A\bar x=-b\).
\end{proposition}

\begin{proof}
    La fonction est strictement convexe par la proposition \ref{PropHRLooTqIJPS} parce que elle est de classe \( C^2\) et vu que
    \begin{equation}
        f(x)=\frac{ 1 }{2}\sum_{kl}A_{kl}x_lx_k+\sum_kb_kx_k
    \end{equation}
    il est vite vu que \( \frac{ \partial^2f }{ \partial x_i\partial x_j }=A_{ij}\), c'est à dire que \( A\) est la matrice hessienne de \( f\). Cette matrice étant strictement définie positive par hypothèse, la fonction \( f\) est strictement convexe.

    Montrons à présent que \( f\) est coercive. Nous avons :
    \begin{subequations}
        \begin{align}
            | f(x) |&=\big| \frac{ 1 }{2}\langle Ax, x\rangle +\langle b, x\rangle  \big|\\
            &\geq\frac{ 1 }{2}| \langle Ax, x\rangle  |-| \langle b, x\rangle  |\\
            &\geq\frac{ 1 }{2}\lambda_{max}\| x \|^2-\| b \|\| x \|
        \end{align}
    \end{subequations}
    Pour la dernière ligne nous avons nommé \( \lambda_{max}\) la plus grande valeur propre de \( A\) et utilisé Cauchy-Schwarz pour le second terme. Nous avons donc bien \( | f(x) |\to \infty\) lorsque \( \| x \|\to\infty\) et la fonction \( f\) est coercive.

    Soit \( M\) une valeur atteinte par \( f\). L'ensemble
    \begin{equation}
        \{ x\in \eR^n\tq f(x)\leq M \}
    \end{equation}
    est fermé (parce que \( f\) est continue) et borné parce que \( f\) est coercive. Cela est donc compact\footnote{Théorème \ref{ThoXTEooxFmdI}} et \( f\) atteint un minimum qui sera forcément dedans. Cela est pour l'existence d'un minimum.

    Pour l'unicité du minimum nous invoquons la convexité : si \( \bar x_1\) et \( \bar x_2\) sont deux points réalisant le minimum de \( f\), alors
    \begin{equation}
        f\left( \frac{ \bar x_1+\bar x_2 }{2} \right)<\frac{ 1 }{2}f(\bar x_1)+\frac{ 1 }{2}f(\bar x_2)=f(\bar x_1),
    \end{equation}
    ce qui contredit la minimalité de \( f(\bar x_1)\).

    Nous devons maintenant prouver que \( \bar x\) vérifie l'équation \( A\bar x=-b\). Vu que \( \bar x\) est minimum local de \( f\) qui est une fonction de classe \( C^2\), le théorème des minima locaux \ref{PropUQRooPgJsuz} nous indique que \( \bar x\) est solution de \( \nabla f(x)=0\). Calculons un peu cela avec la formule
    \begin{equation}
        df_x(u)=\Dsdd{ f(x+tu) }{t}{0}=\frac{ 1 }{2}\big( \langle Ax, u\rangle +\langle Au, x\rangle  \big)+\langle b, u\rangle =\langle Ax, u\rangle +\langle b, u\rangle =\langle Ax+b, u\rangle .
    \end{equation}
    Donc demander \( df_x(u)=0\) pour tout \( u\) demande \( Ax+b=0\).
\end{proof}

\begin{proposition}[Gradient à pas optimal] \label{PropSOOooGoMOxG}
    Soit \( x_0\in \eR^n\). Nous définissons la suite \( (x_k)\) par
    \begin{equation}
        x_{k+1}=x_k+td_k
    \end{equation}
    où \( d_k=-\nabla f(x_k)\) et \( t_k\) est la valeur minimisant la fonction \( t\mapsto f(x_k+td_k)\) sur \( \eR\).

    Alors pour tout \( k\geq 0\) nous avons
    \begin{equation}
        \| x_k-\bar x \|\leq K \left( \frac{ c_2(A)-1 }{ c_2(A)+1 } \right)^k
    \end{equation}
    où \( c_2(A)=\frac{ \lambda_{max} }{ \lambda_{min} }\) est le rapport ente la plus grande et la plus petite valeur propre de la matrice \( A\).
\end{proposition}

\begin{proof}
    D'abord si \( \nabla f(x_k)=0\), c'est que \( x_k=\bar x\) et l'algorithme est fini. Nous allons donc supposer que \( \nabla f(x_k)\neq 0\) pour tout \( k\).
    \begin{subproof}
        \item[\( t_k\) est bien défini]

            Pour \( t\in \eR\) nous avons
            \begin{equation}    \label{EqKEHooYaazQi}
                f(x_k+td_k)=f(x_k)+\frac{ 1 }{2}t^2\langle Ad_k, d_k\rangle +t\langle \underbrace{Ax_k+b}_{=-d_k}, d_k\rangle=\frac{ 1 }{2}t^2\langle Ad_k, d_k\rangle -t_k\| d_k \|^2 +f(x_k).
            \end{equation}
            qui est un polynôme du second degré en \( t\). Le coefficient de \( t^2\) est \( \frac{ 1 }{2}\langle Ad_k, d_k\rangle >0\) parce que \( d_k\neq 0\) et \( A\) est strictement définie positive. Par conséquent la fonction \( t\mapsto f(x_k+td_k)\) admet bien un unique minimum. Nous pouvons même calculer \( t_k\) parce que l'on connaît pas cœur le sommet d'une parabole :
            \begin{equation}    \label{EqVWJooWmDSER}
                t_k=-\frac{ \langle Ax_k+b, d_k\rangle  }{ \langle Ad_k, d_k\rangle  }=\frac{ \| d_k \|^2 }{ \langle Ad_k, d_k\rangle  }
            \end{equation}
            parce que \( d_k=-\nabla f(x_k)=-(Ax_k+b)\).

        \item[La valeur de \( d_{k+1}\)]

            Par définition, \( d_{k+1}=-\nabla f(x_{k+1})=-(Ax_{k+1}+b)\). Mais \( x_{k+1}=x_k+t_kd_k\), donc
            \begin{equation}
                d_{k+1}=-Ax_k-t_kAd_k-b=d_k-t_kAd_k
            \end{equation}
            parce que \( -Ax_k-b=d_k\).
            
            Par ailleurs, \( \langle d_{k+1}, d_k\rangle =0\) parce que
            \begin{equation}
                \langle d_{k+1}, d_k\rangle =\langle d_k, d_k\rangle -t_k\langle d_k, Ad_k\rangle =\| d_k \|^2-\frac{ \| d_k \|^2 }{ \langle Ad_k, d_k\rangle  }\langle d_k, Ad_k\rangle =0
            \end{equation}
            où nous avons utilisé la valeur \eqref{EqVWJooWmDSER} de \( t_k\).
        
        \item[Calcul de \( f(x_{k+1})\)] 

            Nous repartons de \eqref{EqKEHooYaazQi} où nous substituons la valeur \eqref{EqVWJooWmDSER} de \( t_k\) :
            \begin{equation}
                f(x_{k+1})=f(x_k)+\frac{ 1 }{2}\frac{ \| d_k \|^4 }{ \langle Ad_k, d_k\rangle  }-\frac{ \| d_k \|^4 }{ \langle Ad_k, d_k\rangle  }=f(x_k)-\frac{ 1 }{2}\frac{ \| d_k \|^4 }{ \langle Ad_k, d_k\rangle  }.
            \end{equation}
            
        \item[Encore du calcul \ldots]

            Vu que le produit \( \langle Ad_k, d_k\rangle \) arrive tout le temps, nous allons étudier \( \langle A^{-1}d_k, d_k\rangle \). Le truc malin est d'essayer d'exprimer ça en termes de \( \bar x\) et \( \bar f=f(\bar x)\). Pour cela nous calculons \( f(\bar x)\) :
            \begin{equation}
                \bar f=f(\bar x)=f(-A^{-1} b)=-\frac{ 1 }{2}\langle b, A^{-1}b\rangle .
            \end{equation}
            Ayant cela en tête nous pouvons calculer :
            \begin{subequations}
                \begin{align}
                    \langle A^{-1}d_k, d_k\rangle &=\langle A^{-1}(Ax_k+b), Ax_k+b\rangle \\
                    &=\langle x_k, Ax_k\rangle +\langle A^{-1}b, Ax_k\rangle +\langle b, x_k\rangle+\underbrace{\langle A^{-1}b, b\rangle}_{-2\bar f} \\
                    &=\langle x_k, Ax_k\rangle +2\langle x_k, b\rangle  -2\bar f \label{subeqVIIooVzZlRc}\\
                    &=2\big( f(x_k)-\bar f \big)
                \end{align}
            \end{subequations}
            où nous avons utilisé le fait que \( \langle x, Ay\rangle =\langle Ax, y\rangle \) parce que \( A\) est symétrique.

        \item[Erreur sur la valeur du minimum]

            Nous voulons à présent estimer la différence \( f(x_{k+1})-\bar f\). Pour cela nous mettons en facteur \( f(x_k)-\bar f\) dans \( f(x_{k+1}-\bar f)\); et d'ailleurs c'est pour cela que nous avons calculé \( \langle A^{-1}d_k, d_k\rangle \) : parce que ça fait intervenir \( f(x_k)-\bar f\).
            \begin{subequations}
                \begin{align}
                    f(x_{k+1})-\bar f&=f(x_k)-\frac{ 1 }{2}\frac{ \| d_k \|^4 }{ \langle Ad_k, d_k\rangle  }-\bar f\\
                    &=\big( f(x_k)-\bar f \big)\left( 1-\frac{ 1 }{2}\frac{ \| d_k \|^{4} }{ \langle Ad_k, d_k\rangle \big( f(x_k)-\bar f \big) } \right)\\
                    &=\big( f(x_k)-\bar f \big)\left( 1-\frac{ \| d_k \|^{4} }{ \langle Ad_k, d_k\rangle \langle A^{-1}d_k, d_k\rangle  } \right).\label{subeqGFDooRAwAJk}
                \end{align}
            \end{subequations}
            Nous traitons le dénominateur à l'aide de l'inégalité de Kantorovitch \ref{PropMNUooFbYkug}. Nous avons
            \begin{equation}
                \frac{ \| d_k \|^4 }{ \langle Ad_k, d_k\rangle \langle A^{-1}d_k, d_k\rangle  }\geq \frac{ \| d_k \|^4 }{ \frac{1}{ 4 }\left( \sqrt{c_2(A)}+\frac{1}{ \sqrt{c_2(A)} } \right)^2\| d_k \|^4 }=\frac{ 4c_2(A) }{ (c_2(A)+1)^2 }.
            \end{equation}
            Mettre cela dans \eqref{subeqGFDooRAwAJk} est un calcul d'addition de fractions :
            \begin{equation}
                f(x_{k+1})-\bar f\leq \big( f(x_k)-\bar f \big)\left( \frac{ c_2(A)-1 }{ c_2(A)+1 } \right)^2.
            \end{equation}
            Par récurrence nous avons alors
            \begin{equation}    \label{eqANKooNPfCFj}
                f(x_k)-\bar f\leq \big( f(x_0)-\bar f \big)\left( \frac{ c_2(A)-1 }{ c_2(A)+1 } \right)^{2k}.
            \end{equation}
            Notons qu'il n'y a pas de valeurs absolues parce que \( \bar f\) étant le minimum de \( f\), les deux côtés de l'inégalité sont automatiquement positifs.

        \item[Erreur sur la position du minimum]

            Nous voulons à présent étudier la norme de \( x_k-\bar x\). Pour cela nous l'écrivons directement avec la définition de \( f\) en nous souvenant que \( b=-A\bar x\) :
            \begin{subequations}
                \begin{align}
                    f(x_k)-\bar f&=\frac{ 1 }{2}\langle Ax_k, x_k\rangle +\langle A\bar x, x_k\rangle +\frac{ 1 }{2}\langle A\bar x, \bar x\rangle +\langle A\bar x, \bar x\rangle \\
                    &=\frac{ 1 }{2}\langle Ax_k, x_k\rangle -\langle A\bar x, x_k\rangle +\frac{ 1 }{2}\langle A\bar x, \bar x\rangle \\
                    &=\frac{ 1 }{2}\langle Ax_k, x_k\rangle -\frac{ 1 }{2}\langle A\bar x, x_k\rangle-\frac{ 1 }{2}\langle A\bar x, x_k\rangle +\frac{ 1 }{2}\langle A\bar x, \bar x\rangle \\
                    &=\frac{ 1 }{2}\Big( \langle A(x_k-\bar x), x_k\rangle +\langle A\bar x, \bar x-x_k\rangle  \Big)\\
                    &=\frac{ 1 }{2}\Big( \langle A(x_k-\bar x), (x_k-\bar x)\rangle  \Big)
                \end{align}
            \end{subequations}
            où à la dernière ligne nous avons fait \( \langle A\bar x, \bar x-x_k\rangle =\langle \bar x, A(\bar x-x_k)\rangle \) en vertu de la symétrie de \( A\).

            Les produits de la forme \( \langle Ay, y\rangle \) sont majorés par \( \lambda_{min}\| y \|^2\) parce que \( \lambda_{min}\) est la plus grande valeur propre de \( A\). Dans notre cas,
            \begin{equation}    \label{EqVMRooUMXjig}
                f(x_k)-\bar f\geq \frac{ 1 }{2}\lambda_{min}\| x_k-\bar x \|^2
            \end{equation}
            
        \item[Conclusion]

            En combinant les inéquations \eqref{EqVMRooUMXjig} et \eqref{eqANKooNPfCFj} nous trouvons
            \begin{equation}
                \frac{ 1 }{2}\lambda_{min}\| x_k-\bar x \|^2\leq f(x_k)-\bar f\leq \big( f(x_0)-\bar f \big)\left( \frac{ c_2(A)-1 }{ c_2(A)+1 } \right)^{2k}, 
            \end{equation}
            c'est à dire
            \begin{equation}
                \| x_k-\bar x \|\leq \sqrt{\frac{ 2\big( f(x_0)-\bar f \big) }{ \lambda_{min} +1}}^{2k}.
            \end{equation}
    \end{subproof}
\end{proof}

Notons que lorsque \( c_2(A)\) est proche de \( 1\) la méthode converge rapidement. Par contre si \( c_2(A)\) est proche de zéro, la méthode converge lentement.

%+++++++++++++++++++++++++++++++++++++++++++++++++++++++++++++++++++++++++++++++++++++++++++++++++++++++++++++++++++++++++++ 
\section{Formes quadratiques, signature, et lemme de Morse}
%+++++++++++++++++++++++++++++++++++++++++++++++++++++++++++++++++++++++++++++++++++++++++++++++++++++++++++++++++++++++++++

Soit \( (E,\| . \|_E)\) un espace vectoriel réel normé de dimension finie \( n\). L'ensemble des formes quadratiques réelles\footnote{Définition \ref{DefBSIoouvuKR}.} sur \( E\) est vu comme l'ensemble des matrices symétriques \( S_n(\eR)\); il sera noté \( Q(E)\) et le sous-ensemble des formes quadratiques non dégénérées est \( S_n(\eR)\cap\GL(n,\eR)\) qui sera noté \( \Omega(E)\)\nomenclature[B]{\( \Omega(E)\)}{formes quadratiques non dégénérées}. Nous rappelons que la correspondance est donnée de la façon suivante. Si \( A\in S_n(\eR)\), la forme quadratique associée est \( q_A\) donnée par \( q_A(x)=x^tAx\).

Nous noterons encore \( Q^+(E)\)\nomenclature[B]{\( Q^+(E)\)}{formes quadratiques positives} les formes quadratiques positives sur \( E\) et \( Q^{++}(E)\)\nomenclature[B]{\( Q^{++}(E)\)}{formes quadratiques strictement définies positives} les formes quadratiques strictement définies positives sur \( E\).

Sur \( Q(E)\) nous mettons la norme
\begin{equation}
    N(q)=\sup_{\| x \|_E=1}| q(x) |,
\end{equation}
qui du point de vue de \( S_n(\eR)\) est
\begin{equation}    \label{EqDOgBNAg}
    N(A)=\sup_{\| x \|_E=1}| x^tAx |.
\end{equation}
Notons que à droite, c'est la valeur absolue usuelle sur \( \eR\).

Nous savons par le théorème de Sylvester (théorème \ref{ThoQFVsBCk}) que dans \( \eM(n,\eR)\), toute matrice symétrique de signature \( (p,q)\) est semblable à la matrice
\begin{equation}
    \mtu_{p,q}=\begin{pmatrix}
        \mtu_p    &       &       \\
        &   \mtu_{p}    &       \\
        &       &   0_{n-p-q}
    \end{pmatrix}.
\end{equation}
Donc deux matrices de \( S_n\) sont semblables si et seulement si elles ont la même signature (même si elles ne sont pas de rang maximum, cela soit dit au passage). Si nous notons \( S_n^{p,q}(\eR)\)\nomenclature[B]{\( S_n^{p,q}(\eR)\)}{matrices symétriques réelles de signature \( (p,q)\)} l'ensemble des matrices réelles symétriques de signature \( (p,q)\), alors
\begin{equation}
    S_n^{p,q}(\eR)=\{ P^tAP\tq P\in \GL(n,\eR) \}
\end{equation}
où \( A\) est une quelconque ce ces matrices.

Nous voudrions en savoir plus sur ces ensembles. En particulier nous aimerions savoir si la signature est une notion «stable» au sens où ces ensembles seraient ouverts dans \( S_n\). Pour cela nous considérons l'action de \( \GL(n,\eR)\) sur \( S_n\) définie par
\begin{equation}
    \begin{aligned}
        \alpha\colon \GL(n,\eR)\times S_n(\eR)&\to S_n(\eR) \\
        (P,A)&\mapsto P^tAP 
    \end{aligned}
\end{equation}
faite exprès pour que les orbites de cette action soient les ensembles \( S_n^{p,q}(\eR)\).

La proposition suivante montre que lorsque \( p+q=n\), c'est à dire lorsqu'on parle de matrices de rang maximum, les ensembles \( S_n^{p,q}(\eR)\) sont ouverts, c'est à dire que la signature d'une forme quadratique est une propriété «stable» par petite variations des éléments de matrice. Notons tout de suite que si le rang n'est pas maximum, le théorème de Sylvester dit qu'elle est semblable à une matrice diagonale avec des zéros sur la diagonale; en modifiant un peu ces zéros, on peut modifier évidemment la signature.
\begin{proposition}[\cite{KXjFWKA}] \label{PropNPbnsMd}
    Soit \( (E,\| . \|_{E})\) un espace vectoriel normé de dimension finie. Alors
    \begin{enumerate}
        \item
            les formes quadratiques non dégénérées forment un ouvert dans l'ensemble des formes quadratiques,
        \item
            les ensembles \( S_n^{p,q}(\eR)\) avec \( p+q=n\) sont ouverts dans \( S_n(\eR)\),
        \item   \label{ItemGOhRIiViii}
            les composantes connexes de \( \Omega(E)\) sont les \( S_n^{p,q}(\eR)\) avec \( p+q=n\),
        \item   \label{ItemGOhRIiViv}
            les \( S_n^{p,q}(\eR)\) non dégénérés sont connexes par arc.
    \end{enumerate}
\end{proposition}
\index{connexité!signature d'une forme quadratique}
\index{matrice!symétrique!réelle}
\index{forme!quadratique}

\begin{proof}
    Cette preuve est donnée du point de vue des matrices. La différence entre le point \ref{ItemGOhRIiViii} et \ref{ItemGOhRIiViv} est que dans le premier nous prouvons la connexité de \( S_n^{p,q}(\eR)\) à partir de la connexité de \( \GL^+(n,\eR)\), tandis que dans le second nous prouvons la connexité par arc de \( S_n^{p,q}(\eR)\) à partir de la connexité par arc de \( \GL^+(n,\eR)\). Bien entendu le second implique le premier.
    \begin{enumerate}
        \item
            Il s'agit simplement de remarquer que \( Q(E)=S_n(\eR)\), que \( \Omega(E)=S_n(\eR)\cap\GL(n,\eR)\) et que le déterminant est une fonction continue sur \( \eM(n,\eR)\).
        \item
            Soit \( A_0\in S_n^{p,q}(\eR)\). Le théorème de Sylvester \ref{ThoQFVsBCk} nous donne une matrice inversible \( P\) telle que \( P^tA_0P=\mtu_{p,q}\). Nous allons montrer qu'il existe un voisinage \( \mU\) de \( \mtu_{p,q}\) contenu dans \( S_n^{p,q}(\eR)\). À partir de là, l'ensemble \( (P^{-1})^t\mU P^{-1}\) sera un voisinage de \( A_0\) contenu dans \( S_n^{p,q}(\eR)\).

            Nous considérons les espaces vectoriels
            \begin{subequations}
                \begin{align}
                    F&=\Span\{ e_1,\ldots, e_p \}\\
                    G&=\Span\{ e_{p+1},\ldots, e_n \}
                \end{align}
            \end{subequations}
            La norme euclidienne \( \| . \|_p\) sur \( F\) est équivalente à la norme \( | . |_E\) par le théorème \ref{ThoNormesEquiv}. Donc il existe une constante \( k_1>0\) telle que pour tout \( x\in F\),
            \begin{equation}    \label{EqMViCjJJ}
                \| x \|_p\geq k_1\| x \|_E.
            \end{equation}
            De la même façon sur \( G\), il existe une constante \( k_2>0\) telle que
            \begin{equation}    \label{EqSFwOcDw}
                \| x \|_q\geq k_2\| x \|_E.
            \end{equation}
            Si nous posons \( k=\min\{ k_1^2,k_2^2 \}\), alors nous avons
            \begin{subequations}
                \begin{align}
                    \forall x\in F,\quad &\| x \|_p^2\geq k_1^2\| x \|_E^2\geq k\| x \|_E^2\\
                    \forall x\in G,\quad &\| x \|_q^2\geq k_2^2\| x \|_E^2\geq k\| x \|_E^2.
                \end{align}
            \end{subequations}
            
            Soit une matrice \( A\in S_n(\eR)\) telle que \( N(A-\mtu_{p,q})<k\), c'est à dire que \( A\) est dans un voisinage de \( \mtu_{p,q}\) pour la norme sur \( S_n(\eR)\) donné par \eqref{EqDOgBNAg}. Si \( x\) est non nul dans \( E\), nous avons
            \begin{equation}
                \big| x^t(A-\mtu_{p,q})x \big|\leq N(\mtu_{p,q}-A)\| x \|^2\leq k\| x \|^2.
            \end{equation}
            En déballant la valeur absolue, cela signifie que
            \begin{equation}
                -k\| x \|_E^2\leq x^t(A-\mtu_{p,q})x\leq k\| x \|^2.
            \end{equation}
            Si \( x\in F\), alors la première inéquation et \eqref{EqMViCjJJ} donnent
            \begin{equation}
                x^tAx\geq \| x \|_p^2-k\| x \|_E^2>0
            \end{equation}
            Si \( x\in G\), alors la seconde inéquation et \eqref{EqSFwOcDw} donnent
            \begin{equation}
                x^tAx\leq  k\| x \|_E^2-\| x \|_q^2<0.
            \end{equation}
            
            Nous avons donc montré que \( x\mapsto x^tAx\) est positive sur \( F\) et négative sur \( G\), ce qui prouve que \( A\) est bien de signature \( (p,q)\) et appartient donc à \( S_n^{p,q}(\eR)\). Autrement dit nous avons
            \begin{equation}
                B(\mtu_{p,q},k)\subset S_n^{p,q}(\eR).
            \end{equation}

        \item
            Cette partie de la preuve provient essentiellement de \cite{VKqpMYL}, et fonctionne pour tous les \( S_n^{p,q}(\eR)\), même pour ceux qui ne sont pas de rang maximum. 
            
            Soit \( A\in S_n^{p,q}(\eR)\). Nous savons que \( \GL(n,\eR)\) a deux composantes connexes (proposition \ref{PROPooBIYQooWLndSW}). Vu que l'application 
            \begin{equation}
                \begin{aligned}
                    \alpha\colon \GL(n,\eR)&\to S_n \\
                    P&\mapsto P^tAP 
                \end{aligned}
            \end{equation}
            est continue, l'image d'un connexe de \( \GL(n,\eR)\) par \( \alpha\) est connexe (proposition \ref{PropGWMVzqb}). En particulier, \( \alpha\big( \GL^{\pm}(n,\eR) \big)\) sont deux connexes et nous savons que \( S_n^{p,q}(\eR)\) a au plus ces deux composantes connexes. 

            Notre but est maintenant de trouver une intersection entre \( \alpha\big( \GL^+(n,\eR) \big)\) et \( \alpha\big( \GL^-(n,\eR) \big)\)\quext{À ce point, il me semble que \cite{VKqpMYL} fait erreur parce que la matrice \( -\mtu_n\) est de déterminant \( 1\) lorsque \( n\) est pair. L'argument donné ici provient de \cite{KXjFWKA}}. Soit par le théorème de Sylvester, soit par le théorème de diagonalisation des matrices symétriques réelles \ref{ThoeTMXla}, il existe une matrice \( P\in \GL(n,\eR)\) diagonalisant \( A\). En suivant la remarque \ref{RemGKDZfxu}, et en notant \( Q\) la matrice obtenue à partir de \( P\) en changeant le signe de sa première ligne, nous avons
            \begin{equation}
                \alpha(Q)=Q^tAQ=P^tAP=\alpha(P).
            \end{equation}
            Or si \( P\in \GL^+(n,\eR)\), alors \( Q\in \GL^-(n,\eR)\) et inversement. Donc nous avons trouvé une intersection entre \( \alpha\big( \GL^+(n,\eR) \big)\) et \( \alpha\big( \GL^-(n,\eR) \big)\).

        \item

            Soient \( A\) et \( B\) dans \( S_n^{p,q}(\eR)\cap\GL(n,\eR)\). Par le théorème de Sylvester, il existe \( P\) et \( Q\) dans \( \GL(n,\eR)\) telles que \( A=P^t\mtu_{p,q}P\) et \( B=Q^t\mtu_{p,q}Q\). Par la remarque \ref{RemGKDZfxu} nous pouvons choisir \( P\) et \( Q\) dans \( \GL^+(n,\eR)\). Ce dernier groupe étant connexe par arc, il existe un chemin
            \begin{equation}
                    \gamma\colon \mathopen[ 0 , 1 \mathclose]\to \GL^+(n,\eR) 
            \end{equation}
            tel que \( \gamma(0)=P\) et \( \gamma(1)=Q\). Alors le chemin
            \begin{equation}
                s\mapsto \gamma(s)^t\mtu_{p,q}\gamma(s)
            \end{equation}
            est un chemin continu dans \( S_n^{p,q}(\eR)\) joignant \( A\) à \( B\).
    \end{enumerate}
\end{proof}
% TODO : prouver la connexité par arc de GL^+(n,\eR) et mettre une référence ici.

Nous savons déjà de la proposition \ref{PropNPbnsMd} que les ensembles \( S_n^{p,q}(\eR)\) (pas spécialement de rang maximum) sont ouverts dans \( S_n(\eR)\). Le lemme suivant nous donne une précision à ce sujet, dans le cas des matrices de rang maximum, en disant que la matrice qui donne la similitude entre \( A_0\) et \( A\) est localement un \( C^1\)-difféomorphisme de \( A\).
\begin{lemma}   \label{LemWLCvLXe}
    Soit \( A_0\in \Omega(\eR^n)= S_n\cap\GL(n,\eR)\), une matrice symétrique inversible. Alors il existe un voisinage \( V\) de \( A_0\) dans \( S_n\) et une application \( \phi\colon V\to \GL(n,\eR)\) qui
    \begin{enumerate}
        \item
            est de classe \( C^1\),
        \item
            est telle que pour tout \( A\in V\), \( \varphi(A)^t A_0\phi(A)=A\).
    \end{enumerate}
\end{lemma}
\index{groupe!\( \GL(n,\eR)\)}
\index{forme!quadratique}
\index{matrice!symétrique}
\index{matrice!semblables}

\begin{proof}
    Nous considérons l'application
    \begin{equation}
        \begin{aligned}
            \varphi\colon \eM(n,\eR)&\to S_n \\
            M&\mapsto M^tA_0M. 
        \end{aligned}
    \end{equation}
    Étant donné que les composantes de \( \varphi(M)\) sont des polynômes en les entrées de \( M\), cette application est de classe \( C^1\) -- et même plus. Soit maintenant \( H\in \eM(n,\eR)\) et calculons \( d\varphi_{\mtu}(H)\) par la formule \eqref{EqOWQSoMA} :
    \begin{subequations}
        \begin{align}
            d\varphi_{\mtu}(H)&=\Dsdd{ \varphi(\mtu+tH) }{t}{0}\\
            &=\Dsdd{ (\mtu+tH^t)A_0(\mtu+tH) }{t}{0}\\
            &=\Dsdd{ A_0+tA_0H+tH^tA_0+t^2H^tA_0H }{t}{0}\\
            &=A_0H+H^tA_0.
        \end{align}
    \end{subequations}
    Donc
    \begin{equation}
        d\varphi_{\mtu}(H)=(A_0H)+(A_0H)^t.
    \end{equation}
    Par conséquent 
    \begin{equation}
        \ker(d\varphi_{\mtu})=\{ H\in \eM(n,\eR)\tq A_0H\text{ est antisymétrique} \},
    \end{equation}
    et si nous posons
    \begin{equation}
        F=\{ H\in \eM(n,\eR)\tq A_0H\text{ est symétrique} \}
    \end{equation}
    nous avons
    \begin{equation}
        \eM(n,\eR)=F\oplus\ker(d\varphi_{\mtu})
    \end{equation}
    parce que toute matrice peur être décomposée de façon unique en partir symétrique et antisymétrique. De plus l'application
    \begin{equation}    \label{EqGTBusDm}
        \begin{aligned}
            f\colon F&\to S_n \\
            H&\mapsto A_0H 
        \end{aligned}
    \end{equation}
    est une bijection linéaire. D'abord \( A_0H=0\) implique \( H=0\) parce que \( A_0\) est inversible, et ensuite si \( X\in S_n\), alors \( X=A_0A_0^{-1}X\), ce qui prouve que \( X\) est l'image par \( f\) de \( A_0^{-1}X\) et donc que \( f\) est surjective.

    Maintenant nous considérons la restriction \( \psi=\varphi_{|_F}\), \( \psi\colon F\to S_n\). Remarquons que \( \mtu\in F\) parce que \( A_0\in S_n\). L'application \( d\psi_{\mtu}\) est une bijection. En effet d'abord
    \begin{equation}
        d(\varphi_{|_F})_{\mtu}=(d\varphi_{\mtu})_{|_F},
    \end{equation}
    ce qui prouve que
    \begin{equation}
        \ker(d\psi_{\mtu})=\ker(d\varphi_{\mtu})\cap F=\{ 0 \},
    \end{equation}
    ce qui prouve que \( d\psi_{\mtu}\) est injective. Pour montrer que \( d\psi_{\mtu}\) est surjective, il suffit de mentionner le fait que \( \dim F=\dim S_n\) du fait que l'application \eqref{EqGTBusDm} est une bijection linéaire.

    Nous pouvons utiliser le théorème d'inversion locale (théorème \ref{ThoXWpzqCn}) et conclure qu'il existe un voisinage ouvert \( U\) de \( \mtu\) dans \( F\) tel que \( \psi\) soit un difféomorphisme \( C^1\) entre \( U\) et \( V=\psi(U)\). Vu que \( \GL(n,\eR)\) est ouvert dans \( \eM(n,\eR)\), nous pouvons prendre \( U\cap \GL(n,\eR)\) et donc supposer que \( U\subset \GL(n,\eR)\).

    Pour tout \( A\in V\), il existe une unique \( M\in U\) telle que \( \psi(M)=A\), c'est à dire telle que \( A=M^tA_0M\). Cette matrice \( M\) est \( \psi^{-1}(A)\) et est une matrice inversible. Bref, nous posons
    \begin{equation}
        \begin{aligned}
            \phi\colon V&\to \GL(n,\eR) \\
            A&\mapsto \psi^{-1}(A), 
        \end{aligned}
    \end{equation}
    et ce \( \phi\) est de classe \( C^1\) sur \( V\) parce que c'est ce que dit le théorème d'inversion locale. Cette application répond à la question parce que \( V\) est un voisinage de \( \varphi(\mtu)=A_0\) et pour tout \( A\in V\) nous avons
    \begin{equation}
        \phi(A)^tA_0\phi(A)=\varphi^{-1}(A)^tA_0\varphi^{-1}(A)=A.
    \end{equation}
\end{proof}

%--------------------------------------------------------------------------------------------------------------------------- 
\subsection{Lemme de Morse}
%---------------------------------------------------------------------------------------------------------------------------

\begin{lemma}[Lemme de Morse]     \label{LemNQAmCLo}
    Soit \( f\in C^3(\mU,\eR)\) où \( \mU\) est un ouvert de \( \eR^n\) contenant \( 0\). Nous supposons que \( df_0=0\) et que \( d^2f_0\) est non dégénérée\footnote{En tant qu'application bilinéaire.} et de signature \( (p,n-p)\). Alors il existe un \( C^1\)-difféomorphisme \( \varphi\) entre deux voisinages de \( 0\) dans \( \eR^n\) tel que
    \begin{enumerate}
        \item
            \( \varphi(0)=0\),
        \item
            si \( \varphi(x)=u\) alors
            \begin{equation}
                f(x)-f(0)=u_1^2+\cdots +u_p^2-u_{p+1}^2-\ldots-u_n^2.
            \end{equation}
    \end{enumerate}
    Une autre façon de dire est qu'il existe un \( C^1\)-difféomorphisme local \( \psi\) tel que
    \begin{equation}
        (f\circ\psi)(x)-f(0)=x_1^2+\cdots +x_p^2-x_{p+1}^2-\ldots-x_n^2.
    \end{equation}
\end{lemma}
\index{lemme!de Morse}
\index{développement!Taylor}
\index{application!différentiable}
\index{forme!quadratique}
\index{théorème!inversion locale!utilisation}
\index{action de groupe!sur des matrices}
\index{extremum}

\begin{proof}
    Nous allons noter \( Hf\) la matrice Hessienne de \( f\), c'est à dire \( Hf_a=d^2f_a\in\aL^{(2)}(\eR^n,\eR)\). Écrivons la formule de Taylor avec reste intégral (proposition \ref{PropAXaSClx} avec \( p=0\) et \( m=2\)) :
    \begin{equation}
        f(x)-f(0)=\underbrace{df_0(x)}_{=0}+\int_0^1(1-t)\underbrace{d^2f_{tx}(x,x)}_{x^t(Hf)_{tx}x=\langle Hf_{tx}x, x\rangle }dt=x^tQ(x)x
    \end{equation}
    avec
    \begin{equation}
        Q(x)=\int_0^1(1-t)(Hf)_{tx}dt
    \end{equation}
    qui est une intégrale dans \( \aL^{(2)}(\eR^n,\eR)\). Nous prouvons à présent que \( Q\) est de classe \( C^1\) en utilisant le résultat de différentiabilité sous l'intégrale \ref{PropAOZkDsh}. Pour cela nous passons aux composantes (de la matrice) et nous considérons
    \begin{equation}
        \begin{aligned}
            h_{kl}\colon U\times\mathopen[ 0 , 1 \mathclose]&\to \eR \\
            h_{kl}(x,t)&=(1-t)\frac{ \partial^2f  }{ \partial x_k\partial x_l }(tx).
        \end{aligned}
    \end{equation}
    Étant donné que \( f\) est de classe \( C^3\), la dérivée de \( h_{kl}\) par rapport à \( x_i\) ne pose pas de problèmes :
    \begin{equation}
        \frac{ \partial h_{kl} }{ \partial x_i }=t(t-1)\frac{ \partial^3f  }{ \partial x_i\partial x_k\partial x_l }(tx),
    \end{equation}
    qui est encore continue à la fois en \( t\) et en \( x\). La proposition \ref{PropAOZkDsh} nous montre à présent que
    \begin{equation}
        Q_{kl}(x)=\int_0^1(1-t)h_{kl}(tx)dt
    \end{equation}
    est une fonction \( C^1\). Étant donné que les composantes de \( Q\) sont \( C^1\), la fonction \( Q\) est également \( C^1\).

    Nous avons \( Q(0)=\frac{ 1 }{2}(Hf)_0\in S_n\cap \GL(n,\eR)\), d'abord parce que \( f\) est \( C^2\) (et donc la matrice hessienne est symétrique), ensuite par hypothèse \( d^2f_0\) est non dégénérée.
    %TODO : prouver que la matrice hessienne est symétrique lorsque f est C^2 (ou vérifier que c'est déjà fait), et référentier ici.

    À partir de là, le lemme \ref{LemWLCvLXe} donne un voisinage \( V\) de \( Q(0)\) dans \( S_n\) et une application \( \phi\) de classe \( C^1\)
    \begin{equation}
            \phi\colon V\to \GL(n,\eR) \\
    \end{equation}
    telle que pour tout \( A\in V\),
    \begin{equation}
        \phi(A)^tQ(0)\phi(A)=A.
    \end{equation}
    Si on pose \( M=\phi\circ Q\), et si \( x\) est dans un voisinage de zéro, \( Q\) étant continue nous avons \( Q(x)\in V\) et donc
    \begin{equation}
        Q(x)=M(x)^tQ(0)M(x).
    \end{equation}
    Notons que l'application \( \eM\colon \eR\to \GL(n,\eR)\) est de classe \( C^1\) parce que \( Q\) et \( \phi\) le sont.

    Nous avons
    \begin{equation}
        f(x)-f(0)=x^tQ(x)x=x^tM(x)^tQ(0)M(x)x=y(x)^tQ(0)y(x)
    \end{equation}
    où \( y(x)=M(x)x=(\phi\circ Q)(x)x\) est encore une fonction de classe \( C^1\) parce que la multiplication est une application \(  C^{\infty}\).

    D'un autre côté le théorème de Sylvester \ref{ThoQFVsBCk} nous donne une matrice inversible \( P\) telle que
    \begin{equation}
        Q(0)=P^t\begin{pmatrix}
            \mtu_p    &       \\ 
            &   -\mtu_{n-p}    
        \end{pmatrix}P.
    \end{equation}
    Et nous posons enfin \( u=\varphi(x)=Py(x)\) qui est toujours de classe \( C^1\) et qui donne
    \begin{subequations}
        \begin{align}
            f(x)-f(0)&=y^tQ(0)y\\
            &=y^tP^t\begin{pmatrix}
                \mtu    &       \\ 
                    &   -\mtu    
            \end{pmatrix}Py\\
            &=u^t\begin{pmatrix}
                \mtu    &       \\ 
                    &   -\mtu    
            \end{pmatrix}u\\
            &=u_1^2+\cdots +u_p^2-u_{p+1}^2-\ldots -u_n^2.
        \end{align}
    \end{subequations}
    
    Nous devons maintenant montrer que, quitte à réduire son domaine à un ouvert plus petit, \( \varphi\) est un \( C^1\)-difféomorphisme. Dans la chaine qui donne \( \varphi\), seule l'application 
    \begin{equation}
        \begin{aligned}
            g\colon U\subset \eR^n&\to \eR^n \\
            x&\mapsto M(x)x 
        \end{aligned}
    \end{equation}
    est sujette à caution. Nous allons appliquer le théorème d'inversion locale. Nous savons que \( g\) est de classe \( C^1\) et donc différentiable; calculons la différentielle en utilisant la formule \eqref{EqOWQSoMA} :
    \begin{equation}
        dg_0(x)=\Dsdd{ g(tx) }{t}{0}=\Dsdd{ tM(tx)x }{t}{0}=M(0)x.
    \end{equation}
    Note que nous avons utilisé la règle de Leibnitz pour la dérivée d'un produit, mais le second terme s'est annulé. Donc \( dg_0=M(0)\in \GL(n,\eR)\) et \( g\) est localement un \( C^1\)-difféomorphisme.

    Il suffit de restreindre \( \varphi\) au domaine sur lequel \( g\) est un \( C^1\)-difféomorphisme pour que \( \varphi\) devienne lui-même un \( C^1\)-difféomorphisme.

\end{proof}

\begin{definition}
    Un point \( a\) est un \defe{point critique}{point critique!définition} de la fonction différentiable \( f\) si \( df_a=0\).
\end{definition}

\begin{corollary}[\cite{XPautfO}]
    Les points critiques non dégénérés d'une fonction \( C^3\) sont isolés.
\end{corollary}

\begin{proof}
    Soit \( a\) un point critique non dégénéré. Par le lemme de Morse \ref{LemNQAmCLo}, il existe un \( C^1\)-difféomorphisme \( \psi\) et un entier \( p\) tel que
    \begin{equation}
        (f\circ \psi)(x)=x_1^2+\cdots +x_p^2-x_{p+1}^2-\ldots -x_n^2+f(a)
    \end{equation}
    sur un voisinage \( \mU\) de \( a\). Vue la formule générale \( df_x(u)=\nabla f(x)\cdot u\), si \( x\) est un point critique de \( f\), alors \( \nabla f(x)=0\). Dans notre cas, les points critiques de \( f\circ \psi\) dans \( \mU\) doivent vérifier \( x_i=0\) pour tout \( i\), et donc \( x=a\).

    Nous devons nous assurer que la fonction \( f\) elle-même n'a pas de points critiques dans \( \mU\). Pour cela nous utilisons la formule générale de dérivation de fonction composée :
    \begin{equation}
        \nabla(f\circ\psi)(x)=\sum_k \frac{ \partial f }{ \partial y_k }\big( g(x) \big)\nabla g_k(x).
    \end{equation}
    Si \( \psi(x)\) est une point critique de \( f\), alors le membre de droite est le vecteur nul parce que tous les \( \partial_kf\big( \psi(x) \big)\) sont nuls. Par conséquent le membre de gauche est également nul, et \( x\) est un point critique de \( f\circ\psi\). Or nous venons de voir que \( f\circ\psi\) n'a pas de points critiques dans \( \mU\).

    Donc \( f\) n'a pas de points critiques dans un voisinage d'un point critique non dégénéré.
\end{proof}

%+++++++++++++++++++++++++++++++++++++++++++++++++++++++++++++++++++++++++++++++++++++++++++++++++++++++++++++++++++++++++++
\section{Variétés}
%+++++++++++++++++++++++++++++++++++++++++++++++++++++++++++++++++++++++++++++++++++++++++++++++++++++++++++++++++++++++++++

\subsection{Introduction}
Soit $f : S^2 \to \eR$ une fonction définie sur la sphère usuelle
$S^2 \subset \eR^3$. Une question naturelle est d'estimer la
régularité de $f$ ; est-elle continue, dérivable, différentiable ? Il
n'existe pas de dérivée directionnelle étant donné que le quotient
différentiel
\begin{equation*}
  \frac{f(x + \epsilon u_1 ,y + \epsilon u_2) - f(x,y)}{\epsilon}
\end{equation*}
n'a pas de sens pour un point $(x + \epsilon u_1 ,y + \epsilon u_2)$
qui n'est pas --sauf valeurs particulières-- dans la surface. Pour la
même raison il n'est pas possible de parler de différentiabilité de
cette manière. Comment faire, sans devoir étendre le domaine de
définition de $f$ à un voisinage de la sphère ? Une solution possible
est de parler de la notion de variété.

Une variété est un objet qui ressemble, vu de près, à $\eR^m$ pour un
certain $m$. En d'autres termes, on imagine une variété comme un
recollement de morceaux de $\eR^m$ vivant dans un espace plus grand
$\eR^n$. Ces morceaux sont appelés des ouverts de carte, et
l'application qui exprime la ressemblance à $\eR^m$ est l'application
de carte.

%---------------------------------------------------------------------------------------------------------------------------
\subsection{Définition et propriétés}
%---------------------------------------------------------------------------------------------------------------------------

\begin{definition}
  Soit $\emptyset \neq M \subset \eR^n$, $1 \leq m < n$ et $k \geq
  1$. $M$ est une \Defn{variété de classe $C^k$ de dimension $m$} si
  pour tout $a \in  M$, il existe un voisinage ouvert $U$ de $a$
  dans $\eR^n$, et un ouvert $V$ de $\eR^m$ tel que $U \cap M$
  soit le graphe d'une fonction $f : V \subset \eR^m \to \eR^{n-m}$
  de classe $C^1$, c'est-à-dire qu'il existe un réagencement des
  coordonnées $(x_{i_1}, \ldots, x_{i_m}, x_{i_{m+1}}, \ldots,
  x_{i_n})$ avec
  \begin{equation*}
    M \cap U = \left\{ (x_1, \ldots, x_n) \in \eR^n \tq
%      \begin{array}{l} % deux conditions
      (x_{i_1}, \ldots, x_{i_m}) \in V \quad \left\{\begin{array}{c!{=}l} % 1: equations
        x_{i_{m+1}} & f_1(x_{i_1}, \ldots, x_{i_m})\\
        \vdots & \vdots \\
        x_{i_n} & f_{n-m}(x_{i_1}, \ldots, x_{i_m})
      \end{array}\right.
%    \end{array}
    \right\}
  \end{equation*}
  où $V$ est un voisinage ouvert de $(a_{i_1}, \ldots, a_{i_m}) \in \eR^m$.
\end{definition}

La littérature regorge de théorèmes qui proposent des conditions équivalentes à la définition d'une variété. Celle que nous allons le plus utiliser est la suivante% , de la page 268.
\begin{proposition}
    Soit $M\subset\eR^n$ et $1\leq m\leq n-1$. L'ensemble $M$ est une variété si et seulement si $\forall a\in M$, il existe un voisinage ouvert $\mU$ de $a$ dans $\eR^n$ et une application $F\colon W\subset\eR^m\to \eR^n$ où $W$ est un ouvert tels que
    \begin{enumerate}
        \item
            $F$ est un homéomorphisme de $W$ vers $M\cap\mU$,
        \item
            $F\in C^1(W,\eR^n)$,
        \item
            Le rang de $dF(w)\in L(\eR^m,\eR^n)$ est de rang maximum (c'est à dire $m$) en tout point $w\in W$.
    \end{enumerate}
\end{proposition}

Pour rappel, si $T\colon \eR^m\to \eR^n$ est une application linéaire, son rang\footnote{Définition \ref{DefALUAooSPcmyK}.} est la dimension de son image. Si $A$ est la matrice d'une application linéaire, alors le rang de cette application linéaire est égal à la taille de la plus grande matrice carré de déterminant non nul contenue dans $A$\footnote{Proposition \ref{PropEJBZooTNFPRj}}.

La condition de rang maximum sert à éviter le genre de cas de la figure \ref{LabelFigExempleNonRang} qui représente l'image de l'ouvert $\mathopen] -1 , 1 \mathclose[$ par l'application $F(t)=(t^2,t^3)$.
\newcommand{\CaptionFigExempleNonRang}{Quelque chose qui n'est pas de rang maximum et qui n'est pas une variété.}
\input{pictures_tex/Fig_ExempleNonRang.pstricks}
%\ref{LabelFigExempleNonRang} 
%\newcommand{\CaptionFigExempleNonRang}{Quelque chose qui n'est pas de rang maximum et qui n'est pas une variété.}
%\input{pictures_tex/Fig_ExempleNonRang.pstricks}
La différentielle a pour matrice
\begin{equation}
    dF(t)=(2t,3t^2).
\end{equation}
Le rang maximum est $1$, mais en $t=0$, la matrice vaut $(0,0)$ et son rang est zéro. Pour toute autre valeur de $t$, c'est bon.

Une autre caractérisation des variétés est donnée par la proposition suivante %(proposition 3, page 274).
\begin{proposition}     \label{PropCarVarZerFonc}
    Soit $M\in \eR^n$ et $1\leq m\leq n-1$. L'ensemble $M$ est une variété si et seulement si $\forall a\in M$, il existe un voisinage ouvert $\mU$ de $a$ dans $\eR^n$ tel et une application $G\in C^1(\mU,\eR^{n-m})$ tel que
    \begin{enumerate}

        \item
            le rang de $dG(a)\in L(\eR^n,\eR^{n-m})$ soit maximum (c'est à dire $n-m$) en tout $a\in M$,
        \item
            $M\cap\mU=\{ x\in\mU\tq G(x)=0 \}$.

    \end{enumerate}
\end{proposition}

%---------------------------------------------------------------------------------------------------------------------------
\subsection{Espace tangent}
%---------------------------------------------------------------------------------------------------------------------------

Soit $M$, une variété dans $\eR^n$, et considérons un chemin $\gamma\colon I\to \eR^n$ tel que $\gamma(t)\in M$ pour tout $t\in I$ et tel que $\gamma(0)=a$ et que $\gamma$ est dérivable en $0$. La \defe{tangente}{tangente à un chemin} au chemin $\gamma$ au point $a\in M$ est la droite
\begin{equation}
    s\mapsto a+s\gamma'(0).
\end{equation}
L'\defe{espace tangent}{espace!tangent} de $M$ au point $a$ est l'ensemble décrit par toutes les tangentes en $a$ pour tous les chemins $\gamma$ possibles.

\begin{proposition}         \label{PropDimEspTanVarConst}
    Une variété de dimension $m$ dans $\eR^n$ a un espace tangent de dimension $m$ en chacun de ses points.
\end{proposition}
