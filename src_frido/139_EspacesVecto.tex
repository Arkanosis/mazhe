% This is part of Mes notes de mathématique
% Copyright (c) 2011-2016
%   Laurent Claessens, Carlotta Donadello
% See the file fdl-1.3.txt for copying conditions.


%+++++++++++++++++++++++++++++++++++++++++++++++++++++++++++++++++++++++++++++++++++++++++++++++++++++++++++++++++++++++++++
\section{Polynôme d'endomorphismes}
%+++++++++++++++++++++++++++++++++++++++++++++++++++++++++++++++++++++++++++++++++++++++++++++++++++++++++++++++++++++++++++

Soit \( A\) un anneau commutatif et \( \eK\), un corps commutatif. L'injection canonique \( A\to A[X]\) se prolonge en une injection
\begin{equation}
   \eM(A)\to\eM\big( A[X] \big).
\end{equation}

%---------------------------------------------------------------------------------------------------------------------------
\subsection{Polynômes d'endomorphismes}
%---------------------------------------------------------------------------------------------------------------------------

Soit \( u\in\End(E)\) où \( E\) est un \( \eK\)-espace vectoriel. Nous considérons l'application
\begin{equation}    \label{EqOVKooeMJuv}
    \begin{aligned}
        \varphi_u\colon \eK[X]&\to \End(E) \\
        P&\mapsto P(u). 
    \end{aligned}
\end{equation}
L'image de \( \varphi_u\) est un sous-espace vectoriel. En effet si \( A=\varphi_u(P)\) et \( B=\varphi_u(Q)\), alors \( A+B=\varphi_u(P+Q)\) et \( \lambda A=(\lambda P)(u)\). En particulier c'est un espace fermé.

Soit \( u\) un endomorphisme d'un \( \eK\)-espace vectoriel \( E\) et \( P\), un polynôme. Nous disons que \( P\) est un polynôme \defe{annulateur}{polynôme!annulateur} de \( u\) si \( P(u)=0\) en tant que endomorphisme de \( E\).

\begin{lemma}       \label{LemQWvhYb}
    Si \( P\) et \( Q\) sont des polynômes dans \( \eK[X]\) et si \( u\) est un endomorphisme d'un \( \eK\)-espace vectoriel \( E\), nous avons
    \begin{equation}
        (PQ)(u)=P(u)\circ Q(u).
    \end{equation}
\end{lemma}

\begin{proof}
    Si \( P=\sum_i a_iX^i\) et \( Q=\sum_j b_jX^j\), alors le coefficient de \( X^k\) dans \( PQ\) est
    \begin{equation}        \label{EqCoefGPyVcv}
        \sum_la_lb_{k-l}.
    \end{equation}
    Par conséquent \( (PQ)(u)\) contient \( \sum_la_lb_{k-l}u^k\). Par ailleurs \( P(u)\circ Q(u)\) est donné par
    \begin{equation}
        \sum_ia_iu^i\left( \sum_jb_ju^j \right)(x)=\sum_{ij}a_ib_ju^{i+j}(x).
    \end{equation}
    Le coefficient du terme en \( u^k\) est bien le même que celui donné par \eqref{EqCoefGPyVcv}.
\end{proof}

\begin{theorem}[Décomposition des noyaux ou lemme des noyaux]       \label{ThoDecompNoyayzzMWod}
    Soit \( u\) un endomorphisme du \( \eK\)-espace vectoriel \( E\). Soit \( P\in\eK[X]\) un polynôme tel que \( P(u)=0\). Nous supposons que \( P\) s'écrive comme le produit \( P=P_1\ldots P_n\) de polynômes deux à deux étrangers\footnote{Définition \ref{DefDSFooZVbNAX}.}. Alors
    \begin{equation}
        E=\ker P_1(u)\oplus\ldots\oplus\ker P_n(u).
    \end{equation}
    De plus les projecteurs associés à cette décomposition sont des polynômes en \( u\).
\end{theorem}
\index{lemme!des noyaux}
Ce résultat est utilisé pour prouver que toute représentation est décomposable en représentations irréductibles, proposition \ref{PropHeyoAN} ainsi que pour le théorème \ref{ThoDigLEQEXR} qui dit que si le polynôme minimal d'un endomorphisme est scindé à racine simple alors il est diagonalisable.

\begin{proof}
    Nous posons 
    \begin{equation}
        Q_i=\prod_{j\neq i}P_i.
    \end{equation}
    Par le lemme \ref{LemuALZHn} ces polynômes sont étrangers entre eux et le théorème de Bézout (théorème \ref{ThoBezoutOuGmLB}) donne l'existence de polynômes \( R_i\) tels que
    \begin{equation}
        R_1Q_1+\cdots+R_nQ_n=1.
    \end{equation}
    Si nous appliquons cette égalité à \( u\) et ensuite à \( x\in E\) nous trouvons
    \begin{equation}        \label{EqqVcpUy}
        \sum_{i=1}^n(R_iQ_i)(u)(x)=x,
    \end{equation}
    et en particulier si nous posons \( E_i=\Image\big(P_iQ_i(u)\big)\) nous avons
    \begin{equation}
        E=\sum_{i=1}^nE_i.
    \end{equation}
    Cette dernière somme n'est éventuellement pas une somme directe. Si \( i\neq j\), alors \( Q_iQ_j\) est multiple de \( P\) et nous avons, en utilisant le lemme \ref{LemQWvhYb}, 
    \begin{equation}
        (R_iQ_i)(u)\circ (R_jQ_j)(u)=\big( R_iQ_iR_jQ_j \big)(u)=S_{ij}(u)\circ P(u)=0
    \end{equation}
    où \( S_{ij}\) est un polynôme. 

    Nous pouvons voir \( E\) comme un \( \eK\)-module et appliquer le théorème \ref{ThoProjModpAlsUR}. Les opérateurs \( R_iQ_i(u)\) ont l'identité comme somme et sont orthogonaux, et nous avons donc la décomposition en somme directe :
    \begin{equation}
        E=\bigoplus_{i=1}^nR_iQ_i(u)E.
    \end{equation}

    Afin de terminer la preuve, nous devons montrer que \( R_iQ_i(u)E=\ker P_i(u)\). D'abord nous avons
    \begin{equation}
        P_iR_iQ_i(u)=(R_iP)(u)=R_i(u)\circ P(u)=0,
    \end{equation}
    par conséquent \( \Image(R_iQ_i(u))\subset \ker P_i(u)\). Pour obtenir l'inclusion inverse, nous reprenons l'équation \eqref{EqqVcpUy} avec \( x\in\ker P_i(u)\). Elle se réduit à
    \begin{equation}
        (R_iQ_i)(u)x=x.
    \end{equation}
    Par conséquent \( x\in\Image\big( R_iQ_i(u) \big)\).
\end{proof}

\begin{corollary}   \label{CorKiSCkC}
    Soit \( E\), un \( \eK\)-espace vectoriel de dimension finie et \( f\), un endomorphisme semi-simple dont la décomposition du polynôme minimal \( \mu_f\) en facteurs irréductibles sur \( \eK[X]\) est \( \mu_f=M_1^{\alpha_1}\cdots M_r^{\alpha_r}\). Si \( F\) est un sous-espace stable par \( f\), alors
    \begin{equation}
        F=\bigoplus_{i=1}^r\ker M_i^{\alpha_i}(f)\cap F
    \end{equation}
\end{corollary}

\begin{proof}
    Nous posons \( E_i=\ker M_i^{\alpha_i}(f)\) et \( F_i=E_i\cap F\). Les polynômes \( M_i^{\alpha_i}\) sont deux à deux étrangers et \( \mu_f(f)=0\), donc le lemme des noyaux (\ref{ThoDecompNoyayzzMWod}) s'applique et
    \begin{equation}
        E=E_1\oplus\ldots\oplus E_r.
    \end{equation}
    Nous pouvons décomposer \( x\in F\) en termes de cette somme :
    \begin{equation}     \label{EqbBbrdi}
        x=x_1+\cdots +x_r
    \end{equation}
    avec \( x_i\in E_i\). Toujours selon le lemme des noyaux, les projections sur les espaces \( E_i\) sont des polynômes en \( f\). Par conséquent \( F\) est stable sous toutes ces projections \( \pr_i\colon E\to E_i\), et en appliquant \( \pr_i\) à \eqref{EqbBbrdi}, \( \pr_i(x)=x_i\). Vu que \( x\in F\), le membre de gauche est encore dans \( F\) et \( x_i\in E_i\cap F\). Nous avons donc
    \begin{equation}
        F\subset\bigoplus_{i=1}^rF_i.
    \end{equation}
    L'inclusion inverse est immédiate parce que \( F_i\subset F\) pour chaque \( i\).
\end{proof}

\begin{lemma}   \label{LemVISooHxMdbr}
    Si \( x\) est un vecteur propre de valeur propre \( \lambda\) pour l'endomorphisme \( u\) et si \( P\) est un polynôme, alors \( x\) est vecteur propre de \( u\) pour la valeur propre \( P(\lambda)\).
\end{lemma}

\begin{proof}
    C'est un simple calcul de \( P(u)x\) en ayant noté \( P(X)=\sum_{k=0}^nc_kX^n\) :
    \begin{equation}
        P(u)x=\sum_{k=0}^nc_ku^k(x)=\sum_{k=0}^nc_k\lambda^ku=P(\lambda)x.
    \end{equation}
\end{proof}

%--------------------------------------------------------------------------------------------------------------------------- 
\subsection{Calcul effectif de l'exponentielle d'une matrice}
%---------------------------------------------------------------------------------------------------------------------------

Nous reprenons l'exemple de \cite{MneimneReduct}. Soit \( A\) une matrice dont le polynôme minimum s'écrit
\begin{equation}
    P(X)=(X-1)^2(X-2).
\end{equation}
Par le théorème \ref{ThoDecompNoyayzzMWod} de décomposition des noyaux nous avons
\index{théorème!décomposition des noyaux!et exponentielle de matrice}
\begin{equation}
    E=\ker(A-1)^2\oplus\ker(A-2).
\end{equation}
En suivant les notations de ce théorème nous avons \( P_1(X)=(X-1)^2\), \( P_2(X)=X-2\) et
\begin{subequations}
    \begin{align}
        Q_1(X)&=X-2\\
        Q_2(X)&=(X-1)^2.
    \end{align}
\end{subequations}
Les polynômes \( R_i\) dont l'existence est assurée par le théorème de Bézout sont
\begin{equation}
    \begin{aligned}[]
        R_1(X)&=-X\\
        R_2(X)&=1.
    \end{aligned}
\end{equation}
Nous avons
\begin{equation}
    R_1Q_1+R_2Q_2=1.
\end{equation}
Le projecteur \( p_i\) sur \( \ker P_i\) est \( R_iQ_i\) :
\begin{equation}
    \begin{aligned}[]
        p_1&=-A(A-2)=\pr_{\ker(u-1)^2}\\
        p_2&=(A-1)^2=\pr_{\ker(u-2)}.
    \end{aligned}
\end{equation}
Passons maintenant au calcul de l'exponentielle. Nous avons évidemment
\begin{equation}
    e^A=e^Ap_1+e^Ap_2.
\end{equation}
Étant donné que \( p_1\) est le projecteur sur le noyau de \( (A-1)^2\), nous avons
\begin{equation}
    e^Ap_1=ee^{A-1}p_1=ep_1+e(u-1)1=ep_1=-Ae(A-2).
\end{equation}
En effet \( e^{A-1}p_1=\sum_{k=0}^{\infty}(A-1)^k\circ p_1\). De la même façon nous avons
\begin{equation}
    e^Ap_2=e^2e^{A-2}p_2=e^2p_2=e^2(A-1)^2.
\end{equation}
Au final,
\begin{equation}
    e^A=-Ae(A-2)+e^2(A-1)^2.
\end{equation}

%---------------------------------------------------------------------------------------------------------------------------
\subsection{Polynôme minimal et minimal ponctuel}
%---------------------------------------------------------------------------------------------------------------------------

\begin{lemmaDef}        \label{DefooOHUXooNkPWaB}
    Soit un endomorphisme \( f\colon E\to E\) d'un \( \eK\)-espace vectoriel de dimension finie. Il existe un unique polynôme annulateur normalisé de degré minimum.

    Il est nommé le \defe{polynôme minimal}{polynôme!minimal} de \( f\) et il est noté \( \mu_f\) ou simplement \( \mu\) lorsque la dépendance en \( f\) est claire.
\end{lemmaDef}

\begin{proof}
    Pour l'unicité, soient \( P\) et \( Q\) deux polynômes annulateur de \( f\) de même degré \( N\) et ayant tous deux \( 1\) comme coefficient de \( x^N\). Alors \( P-Q\) est de degré \( N-1\) tout en étant encore annulateur.

    Pour l'existence, les endomorphismes \( \id\), \( f\), \( f^2\), \ldots ne peuvent pas être tous linéairement indépendants parce que la dimension de \( \End(E)\) est finie. Il existe donc un nombre \( N\) et des coefficients \( a_k\) tels que \( \sum_{k=0}^Na_kf^k=0\). Le polynôme \( P(X)=\sum_{k=0}^Na_kX^k\) est donc annulateur de \( f\).

    Une autre façon de le dire est que l'application linéaire \( \varphi\colon \eK[X]\to \End(E)\) donnée par \( \varphi(P)=P(f)\) est un endomorphisme d'un espace vectoriel de dimension infinie vers un espace vectoriel de dimension finie. Il ne peut donc pas être injectif et possède donc un noyau non réduit à zéro.
\end{proof}

\begin{remark}
    La preuve donnée ci-dessus montre que \( \deg(\mu)\leq \dim(E)^2\). Comme conséquence du théorème de Caley-Hamilton \ref{ThoCalYWLbJQ} nous verrons qu'en réalité le degré du polynôme minimal est majoré par la dimension de l'espace.
\end{remark}

\begin{example}[Pas en dimension infinie]
    L'endomorphisme de dérivation
\end{example}


Dans la suite, l'endomorphisme \( f\) du \( \eK\)-espace vectoriel \( E\) de dimension \( n\) est fixé. Pour \( x\in E\) nous notons
\begin{equation}            \label{EqooOAYDooEpZELo}
    E_x=\{ P(f)x\tq P\in \eK[X] \}.
\end{equation}
Nous considérons le morphisme d'algèbres
\begin{equation}
    \begin{aligned}
        \varphi\colon \eK[X]&\to \End(E) \\
        P&\mapsto P(f) 
    \end{aligned}
\end{equation}
et si \( x\in E\) est donné nous considérons le morphisme de \( \eK\)-espaces vectoriels
\begin{equation}
    \begin{aligned}
        \varphi_x\colon \eK[X]&\to E \\
        P&\mapsto P(f)x. 
    \end{aligned}
\end{equation}
Les noyaux de ces applications sont des idéaux, entre autres par le lemme \ref{LemQWvhYb}. Ils ont donc un unique générateur unitaire (chacun) par le théorème \ref{ThoCCHkoU}. En termes de vocabulaire, l'ensemble
\begin{equation}
    \ker(\phi)=\{  Q\in\eK[X]\tq Q(f)=0  \}
\end{equation}
est l'\defe{idéal annulateur}{polynôme!annulateur} de \( f\) et un polynôme \( Q\) tel que \( Q(f)=0\) est une polynôme annulateur de \( f\).

\begin{definition}      \label{DEFooUICRooBGYhqQ}
    Le générateur unitaire de \( \ker(\varphi_x)\) est le \defe{polynôme minimal ponctuel}{polynôme!minimal!ponctuel} de \( f\) en \( x\). Il sera noté \( \mu_{f,x}\) ou \( \mu_x\) lorsque la dépendance en \( f\) est claire dans le contexte.
\end{definition}
Nous notons \( \mu\) le générateur unitaire du noyau de \( \varphi\) et \( \mu_x\) celui de \( \varphi_x\). Vu que \( \mu\in\ker(\varphi_x)\) pour tout \( x\) nous avons\( \mu_x\divides \mu\) pour tout \( x\).

\begin{example}[Pas en dimension infinie]       \label{ExooDTUJooIMqSKn}
    En dimension infinie, il n'y a pas toujours de polynôme annulateur. Si \( E\) est un espace vectoriel de dimension infine ayant une base dénombrable \( \{ e_i \}_{i\in \eN}\) alors l'opérateur donné par \( f(e_i)=e_{i+1}\) n'a pas de polynôme annulateur. Même pas ponctuel en quel que point que ce soir.

    De même l'opérateur donné par \( g(e_1)=0\) et \( g(e_i)=e_{i-1}\) si \( i\neq 1\) n'a pas de polynôme annulateur, mais il a un polynôme annulateur ponctuel évident en \( x=e_1\). L'exemple \ref{ExooLRHCooMYLQTU} donnera un habillage à peine subtil à cet exemple.
\end{example}

\begin{proposition}     \label{PropAnnncEcCxj}
    Si \( P\) est un polynôme tel que \( P(f)=0\), alors le polynôme minimal \( \mu_f\) divise \( P\). Autrement dit, le polynôme minimal engendre l'idéal des polynômes annulateurs.
\end{proposition}

\begin{proof}
    L'ensemble \( \ker(\varphi)=\{ Q\in \eK[X]\tq Q(u)=0 \} \) est un idéal par le lemme \ref{LemQWvhYb}. Le polynôme minimal de \( u\) est un élément de degré plus bas dans \( I\) et par conséquent \( I=(\mu_u)\) par le théorème \ref{ThoCCHkoU}. Nous concluons que \( \mu_u\) divise tous les éléments de \( I\).
\end{proof}

La proposition suivante permet de caractériser le polynôme minimal.
\begin{proposition}[\cite{ooEPEFooQiPESf}]      \label{PROPooVUJPooMzxzjE}
    Soit une application linéaire \( f\) sur un \( \eK\)-espace vectoriel. Il existe un unique polynôme unitaire\quext{À mon avis, «unitaire» manque dans \cite{ooEPEFooQiPESf}.} \( P\in \eK[X]\) tel que
    \begin{enumerate}
        \item
            \( P(f)=0\);
        \item
            l'application
            \begin{equation}        \label{EQooIBMDooVTaEhf}
                \begin{aligned}
                    \varphi\colon \frac{ \eK[X] }{ (P) }&\to \End(E) \\
                    \bar Q&\mapsto Q(f) 
                \end{aligned}
            \end{equation}
            est injective.
    \end{enumerate}
\end{proposition}

\begin{proof}
    En ce qui concerne l'existence, il existe le polynôme minimal de \( f\) qui satisfait les conditions. Pour l'unicité nous travaillons maintenant.

    Supposons que l'application \eqref{EQooIBMDooVTaEhf} soit injective. Alors pour tout \( Q\in \eK[X]\) tel que \( Q(f)=0\) nous avons \( \bar Q=0\), c'est à dire \( Q=PR\) pour un certain \( R\in \eK[X]\). Autrement dit : \( P\) est un générateur unitaire de l'idéal annulateur de \( f\). Le théorème \ref{ThoCCHkoU}\ref{ITEMooASHKooZqkiCH} nous dit alors que \( P=\mu\) parce que \( \mu\) est également générateur unitaire.
\end{proof}

\begin{lemma}[\cite{ooRJDSooXpVtMD}]\label{LemSYsJJj}
    Soit \( f\colon E\to E\) un endomorphisme de l'espace vectoriel \( E\). Il existe un élément \( x\in E\) tel que \( \mu_{f,x}=\mu_f\).
\end{lemma}

\begin{proof}
    Soit une décomposition en irréductibles du polynôme minimal \( \mu=P_1^{\alpha_1}\ldots P_r^{\alpha_r}\). Nous notons \( E_i=\ker\big( P_i^{\alpha_i}(f) \big)\). Les polynômes \( P_i\) sont étrangers deux à deux (un diviseur commun aurait a fortiori été un diviseur et aurait contredit l'irréductibilité). Le lemme des noyaux \ref{ThoDecompNoyayzzMWod} nous donne la somme directe
    \begin{equation}
        E=\bigoplus_{i=1}^r\ker\big( P_i^{\alpha_i}(f) \big).
    \end{equation}
    Si \( x_i\in E_i\) alors \( \mu_{x_i}\) est une puissance de \( P_i\). En effet \( \mu_{x_i}\divides \mu\) et est donc un produit des puissances des \( P_j\). Or si \( (QP_j)(f)x_i=0\) alors \( (P_jQ)(f)x_i=0\), ce qui donne \( Q(f)x_i\in E_j\cap E_i=\{ 0 \}\). Donc \( \mu_{x_i}\) n'est pas de la forme \( QP_j\) pour \( j\neq i\). Nous en déduisons que \( \mu_{x_i}\) est une puissance de \( P_i\) dès que \( x_i\in E_i\). Nous choisissons \( x_i\in E_i\) tel que \( \mu_{x_i}=P_i^{\alpha_i}\).

    Nous posons enfin \( a=x_1+\cdots +x_r\); par définition du polynôme annulateur \( \mu_a\), nous avons
    \begin{equation}        \label{EqooVIGGooSfuvwB}
        0=\mu_a(f)a=\mu_a(f)x_1+\cdots +\mu_a(f)x_r.
    \end{equation}
    Mais \( m_a(f)x_j\in E_i\), et la somme des \( E_j\) est directe, donc l'annulation de la somme \eqref{EqooVIGGooSfuvwB} implique l'annulation de chacun des termes : \( \mu_a(f)x_i=0\) pour tout \( i\). Cela prouve que \( \mu_{x_i}\divides \mu_a\). Mais comme les \( \mu_{x_i}\) sont premiers deux à deux (parce que ce sont les \( P_i^{\alpha_i}\)), nous avons que le produit divise encore \( \mu_a\) :
    \begin{equation}
        \prod_{i=1}^r\mu_{x_i}\divides \mu_a,
    \end{equation}
    c'est à dire \( \mu\divides \mu_a\). Comme nous avons aussi \( \mu_a\divides \mu\), nous déduisons \( \mu_a=\mu\).
\end{proof}

\begin{definition}[Matrices, endomorphismes et vecteurs cycliques]      \label{DEFooFEIFooNSGhQE}
    Une matrice est \defe{cyclique}{cyclique!matrice}\index{matrice!cyclique} si elle est semblable à une matrice compagnon. Un endomorphisme \( f\colon E\to E\) est \defe{cyclique}{cyclique!endomorphisme}\index{endomorphisme!cyclique} si il existe un vecteur \( x\in E\) tel que \( \{ f^k(x) \}_{k=0,\ldots, n-1} \) est une base de \( E\). Un vecteur ayant cette propriété est un \defe{vecteur cyclique}{vecteur!cyclique} pour \( f\).
\end{definition}

\begin{lemma}   \label{LemAGZNNa}
    Soit \( E\) un espace vectoriel de dimension finie et un endomorphisme cyclique\footnote{Voir la définition \ref{DEFooFEIFooNSGhQE}.} \( f\) de \( E\). Soit un vecteur cyclique \( v\) de \( f\), alors le polynôme minimal de \( f\) est égal au polynôme minimal de \( f\) au point \( v\) : \( \mu_{f}=\mu_{f,v}\).
\end{lemma}

\begin{proof}
    Montrons que \( \mu_{f,v}\) est un polynôme annulateur de \( f\), ce qui prouvera que \( \mu_f\) divise \( \mu_{f,v}\) par la proposition \ref{PropAnnncEcCxj}. Étant donné que \( v\) est cyclique, tout élément de \( E\) s'écrit sous la forme \( x=Q(f)v\). Prenons un polynôme \( P\) annulateur de \( f\) en \( v\) : \( P(f)v=0\). Nous montrons que \( P\) est alors un polynôme annulateur de \( f\). En effet, nous avons
    \begin{equation}
        P(f)x=\big( P(f)\circ Q(f) \big)v=\big( Q(f)\circ P(f) \big)v=0
    \end{equation}
    où nous avons utilisé le lemme \ref{LemQWvhYb}.
\end{proof}

\begin{lemma}[\cite{ooRJDSooXpVtMD}]
    Soit \( a\in E\) tel que \( \mu_a=\mu\). Alors \( E_a\) est un sous-espace stable pour \( f\) pour lequel il existe un supplémentaire stable.
\end{lemma}

\begin{proof}
    Soit \( l=\deg(\mu)=\deg(\mu_a)\). L'espace \( E_a\) étant engendré par les \( f^k(a)\) nous savons que \( e_1=a\), \( e_2=f(a)\),\ldots, \( e_l=k^{l-1}(a)\) forment une base de \( E_a\). Nous pouvons la compléter en une base \( \{ e_1,\ldots, e_n \}\) de \( E\). Et nous posons\footnote{ici, comme presque partout, \( e^*_{l}\) est le dual de \( e_l\), c'est à dire l'application linéaire sur \( E\) donnée par \( e^*_l(e_i)=\delta_{li}\). }
    \begin{subequations}
        \begin{align}
            G&=\{ x\in E\tq e^*_l\big( f^k(x) \big)=0\,\forall k\geq 0 \}\\
            &=\bigcap_{k\geq 0}\ker\{ e^*_l\circ f^k \}\\
            &=\bigcap_{k=0}^{l-1}\ker(  e^*_l\circ f^k ).
        \end{align}
    \end{subequations}
    La dernière égalité est due au fait que \( l\) soit le degré de \( \mu\). Du coup \( f^l\) est une combinaison linéaire des \( f^i\) avec \( i\leq l-1\).

    Nous avons \( f(G)\subset G\) et de plus \( E_a\cap G=\{ 0 \}\) parce qu'un élément de \( E_a\) est une combinaison linéaire d'éléments de la forme \( f^j(a)\) (\( j\leq l\)). Après application de \( f^{l-j}\), ces éléments obtiennent une composante \( f^l(a)=e_l\). De plus \( G\) est un sous-espace vectoriel du fait que \( e^*_l\circ f^i\) est une application linéaire. 
    
    Montrons enfin que \( \dim(G)=n-l\). Pour cela nous remarquons que \( G\) est une intersection d'hyperplans, et nous montrons que les équations définissant ces hyperplans sont linéairement indépendantes. Soit donc 
    \begin{equation}        \label{EqooOHESooRtBUfc}
        \sum_{j=0}^{l-1}\lambda_j\big( e^*_l\circ f^j \big)=0
    \end{equation}
    et montrons que \( \lambda_j=0\) pour tout $j$ est l'unique solution. Soit \( x\in E\) et appliquons l'opération \eqref{EqooOHESooRtBUfc} au vecteur \( f^i(x)\); le résultat est zéro :
    \begin{equation}
        0=\sum_{j=0}^{l-1}\lambda_j(e^*_l\circ f^i\circ f^j)=(e^*_l\circ f^i)P(u)
    \end{equation}
    où nous avons posé \( P(X)=\sum_{j=0}^{l-1}\lambda_jX^j\). Appliquons cela à \( a\) : pour tout \( i\) nous avons
    \begin{equation}
        (e^*_l\circ f^i)\big( P(f)a \big)=0.
    \end{equation}
    Mais par définition de \( E_a\), l'élément \(P(f)a \) est dans \( E_a\). Nous en déduisons que 
    \begin{equation}
        P(f)a\in G\cap E_a=\{ 0 \},
    \end{equation}
    c'est à dire que \( P\) est un polynôme annulateur de \( a\). Mais \( P\) est de degré \( l-1\) alors que le polynôme minimal de \( a\) est de degré \( l\). Par conséquent \( P=0\) et \( \lambda_j=0\) pour tout \( j\).
\end{proof}

\begin{definition}  \label{DEFooBOHVooSOopJN}
    Un endomorphisme d'un espace vectoriel est \defe{semi-simple}{semi-simple!endomorphisme} si tout sous-espace stable par \( u\) possède un supplémentaire stable.
\end{definition}

\begin{lemma}   \label{LemrFINYT}
    Si le polynôme minimal d'un endomorphisme est irréductible, alors il est semi-simple\footnote{Définition \ref{DEFooBOHVooSOopJN}.}.
\end{lemma}

\begin{proof}
    Soit \( f\), un endomorphisme dont le polynôme minimal est irréductible et \( F\), un sous-espace stable par \( f\). Nous devons en trouver un supplémentaire stable. Si \( F=E\), il n'y a pas de problèmes. Sinon nous considérons \( u_1\in E\setminus F\) et
    \begin{equation}
        E_{u_1}=\{ P(f)u_1\tq P\in \eK[X] \},
    \end{equation}
    qui est un espace stable par \( f\). 

    Montrons que \( E_{u_1}\cap F=\{ 0 \}\). Pour cela nous regardons l'idéal
    \begin{equation}
        I_{u_1}=\{ P\in \eK[X]\tq P(f)u_1=0 \}.
    \end{equation}
    Cela est un idéal non réduit à \( \{ 0 \}\) parce que le polynôme minimal de \( f\) par exemple est dans \( I_{u_1}\). Soit \( P_{u_1}\) un générateur unitaire de \( I_{u_1}\). Étant donné que \( \mu_f\in I_{u_1}\), nous avons que \( P_{u_1}\) divise \( \mu_f\) et donc \( P_{u_1}=\mu_f\) parce que \( \mu_f\) est irréductible par hypothèse.

    Soit \( y\in E_{u_1}\cap F\). Par définition il existe \( P\in\eK[X]\) tel que \( y=P(f)u_1\) et si \( y\neq 0\), ce la signifie que \( P\notin I_{u_1}\), c'est à dire que \( P_{u_1} \) ne divise pas \( P\). Étant donné que \( P_{u_1}\) est irréductible cela implique que \( P_{u_1}\) et \( P\) sont premiers entre eux (ils n'ont pas d'autre \( \pgcd\) que \( 1\)).

    Nous utilisons maintenant Bézout (théorème \ref{ThoBezoutOuGmLB}) qui nous donne \( A,B\in \eK[X]\) tels que 
    \begin{equation}
        AP+BP_{u_1}=1.
    \end{equation}
    Nous appliquons cette égalité à \( f\) et puis à \( u_1\):
    \begin{equation}
        u_1=A(f)\circ \underbrace{P(f)u_1}_{=y}+B(f)\circ \underbrace{P_{u_1}(u_1)}_{=0}=A(f)y.
    \end{equation}
    Mais \( y\in F\), donc \( A(f)y\in F\). Nous aurions donc \( u_1\in F\), ce qui est impossible par choix. Nous avons maintenant que l'espace \( E_{u_1}\oplus F\) est stable sous \( f\). Si cet espace est \( E\) alors nous arrêtons. Sinon nous reprenons le raisonnement avec \( E_{u_1}\oplus F\) en guise de \( F\) et en prenant \( u_2\in E\setminus(E_{u_1}\oplus F)\). Étant donné que \( E\) est de dimension finie, ce procédé s'arrête à un certain moment et nous aurons
    \begin{equation}
        E=F\oplus E_{u_1}\oplus\ldots\oplus E_{u_k}
    \end{equation}
    où chacun des \( E_{u_i}\) sont stables.
\end{proof}

\begin{theorem} \label{ThoFgsxCE}
    Un endomorphisme est semi-simple si et seulement si son polynôme minimal est produit de polynômes irréductibles distincts deux à deux.
\end{theorem}
\index{anneau!principal}

\begin{proof}

    Supposons que \( f\) soit semi-simple et que son polynôme minimal soit donné par \( \mu_f=M_1^{\alpha_1}\ldots M_r^{\alpha_r}\) où les \( M_i\) sont des polynômes irréductibles deux à deux distincts. Nous devons montrer que \( \alpha_i=1\) pour tout \( i\). Soit \( i\) tel que \( \alpha_i\geq 1\) et \( N\in \eK[X]\) tel que \( \mu_f=M^2N\) où l'on a noté \( M=M_i\). Nous étudions l'espace
    \begin{equation}
        F=\ker M(f)
    \end{equation}
    qui est stable par \( f\), et qui possède donc un supplémentaire \( S\) également stable par \( f\). Nous allons montrer que \( MN\) est un polynôme annulateur de \( f\).

    D'abord nous prenons \( x\in S\). Étant donné que \( F\) est le noyau de \( M(f)\),
    \begin{equation}
        M(f)\big( MN(f)x \big)=\mu_f(f)x=0,
    \end{equation}
    ce qui signifie que \( MN(f)x\in F\). Mais vu que \( S\) est stable par \( f\) nous avons aussi que \( MN(f)x\in S\). Finalement \( MN(f)x\in F\cap S=\{ 0 \}\). Autrement dit, \( MN(f)\) s'annule sur \( S\).

    Prenons maintenant \( y\in F\). Nous avons
    \begin{equation}
        MN(f)=N(f)\big( M(f)y \big)=0
    \end{equation}
    parce que \( y\in F=\ker M(f)\).

    Nous avons prouvé que \( MN(f)\) s'annule partout et donc que \( MN(f)\) est un polynôme annulateur de \( f\), ce qui contredit la minimalité de \( \mu_f=M^2N\).

    Nous passons au sens inverse. Soit \( m_f=M_1\ldots M_r\) une décomposition du polynôme minimal de l'endomorphisme \( f\) en irréductibles distincts deux à deux. Soit \( F\) un sous-espace vectoriel stable par \( f\). Nous notons
    \begin{equation}
        E_i=\ker(M_i(f))
    \end{equation}
    et \( f_i=f|_{E_i}\). Par le lemme \ref{CorKiSCkC} nous avons
    \begin{equation}
        F=\bigoplus_{i=1}^r(F\cap E_i).
    \end{equation}
    Les espaces \( E_i\) sont stables par \( f\) et étant donné que \( M_i\) est irréductible, il est le polynôme minimal de \( f_i\). En effet, \( M_i\) est annulateur de \( f_i\), ce qui montre que le minimal de \( f_i\) divise \( M_i\). Mais \( M_i\) étant irréductible, \( M_i\) est le polynôme minimal. Étant donné que \( \mu_{f_i}=M_i\), l'endomorphisme \( f_i\) est semi-simple par le lemme \ref{LemrFINYT}.

    L'espace \( F\cap E_i\) étant stable par l'endomorphisme semi-simple \( f_i\), il possède un supplémentaire stable que nous notons \( S_i\)~:
    \begin{equation}
        E_i=S_i\oplus(F\cap E_i).
    \end{equation}
    Étant donné que sur chaque \( S_i\) nous avons \( f|_{S_i}=f_i\), l'espace \( S=S_1\oplus\ldots\oplus S_r\) est stable par \( f\). Du coup nous avons
    \begin{subequations}
        \begin{align}
            E&=E_1\oplus\ldots\oplus E_r\\
            &=\big( S_1\oplus(F\cap E_1) \big)\oplus\ldots\oplus\big( S_r\oplus(F\cap E_r) \big)\\
            &=\big( \bigoplus_{i=1}^rS_i \big)\oplus\big( \bigoplus_{i=1}^rF\cap E_i \big)\\
            &=S\oplus F,
        \end{align}
    \end{subequations}
    ce qui montre que \( F\) a bien un supplémentaire stable par \( f\) et donc que \( f\) est semi-simple.
\end{proof}

\begin{example}[L'espace engendré par \( \mtu\), \( A\), \( A^2\),\ldots]
    Soit \( A\) une matrice, et 
    \begin{equation}
        V=\Span\{A^k\tq k\in \eN \}.
    \end{equation}
    Nous montrons que \( \dim(V)\) est le degré du polynôme minimal de \( A\).

    D'abord l'idéal annulateur de \( A\) est engendré par le polynôme minimal\footnote{Proposition \ref{PropAnnncEcCxj}.} que nous notons
        $\mu=\sum_{k=0}^pa_kX^k$.
    La partie \( \{ \mtu,\ldots, A^{p-1} \}\) est libre parce qu'une combinaison linéaire nulle de cela serait un polynôme annulateur en \( A\) de degré plus petit que \( p\). Donc \( \dim(V)\geq p\).

    La partie \( \{ \mtu,A,\ldots, A^p \}\) est liée à cause du polynôme minimal. Isoler \( A^p\) dans \( \mu(A)=0\) donne un polynôme \( f\) de degré \( p-1\) tel que \( A^p=f(A)\).

    Nous allons montrer à présent que la famille \( \{ \mtu,A,\ldots, A^{p-1} \}\) est génératrice (alors \( \dim(V)\leq p\)). Soit un entier \( q\geq p\)et de division euclidienne\footnote{Théorème \ref{ThoDivisEuclide}.} \( np+r=q\) avec \( r<p\). Nous avons \( A^q=A^{np}A^r\). D'une part
    \begin{equation}
        A^{np}=(A^p)^n=f(A)^n
    \end{equation}
    est de degré \( n(p-1)\). Par conséquent
    \begin{equation}
        A^q=f(A)^nA^r
    \end{equation}
    qui est de degré \( n(p-1)+r=q-n\). Autrement dit il existe un polynôme \( g_1\) de degré \( q-n\) tel que \( A^q=g_1(A)\). Si \( q-n>p-1\) alors nous pouvons recommencer et obtenir un polynôme \( g_2\) de degré strictement inférieur à celui de \( g_1\) tel que \( A^q=g_2(A)\). Au bout du compte, il existe un polynôme \( g\) de degré au maximum \( p-1\) tel que \( A^q=g(A)\). Cela prouve que la partie \( \{ \mtu,A,\ldots, A^{p-1} \}\) est génératrice de \( V\).

    La dimension de \( V\) est donc \( p\), le degré du polynôme minimal.
\end{example}

\begin{proposition}     \label{PropooCFZDooROVlaA}
    Soit \( f\) un endomorphisme d'un espace vectoriel de dimension finie. Nous avons l'isomorphisme d'espace vectoriel
    \begin{equation}
        \eK[f]\simeq\frac{ \eK[X] }{ (\mu_f) }
    \end{equation}
    La dimension en est \( \deg(\mu_f)\).
\end{proposition}

\begin{proof}
    Notons avant de commencer que \( (\mu)\) est l'idéal engendré par \( \mu\). Les classes dont il est question dans le quotient \( \eK[X]/(\mu)\) sont 
    \begin{equation}
        \bar P=\{ P+S\mu \}_{S\in \eK[X]}.
    \end{equation}
    Nous allons montrer que l'application suivante fournit l'isomorphisme : 
    \begin{equation}
        \begin{aligned}
            \psi\colon \frac{ \eK[X] }{ (\mu) }&\to \eK[f] \\
            \bar P&\mapsto P(f). 
        \end{aligned}
    \end{equation}
    \begin{subproof}
        \item[\( \psi\) est bien définie]
            Si \( Q\in \bar P\) alors \( Q=P+S\mu\) pour un certain \( S\in \eK[X]\). Du coup nous avons
            \begin{equation}
                \psi(\bar Q)=P(f)+(S\mu)(f).
            \end{equation}
            Mais \( \mu(f)=0\) donc le deuxième terme est nul. Donc \( \psi(\bar P)\) est bien définit.
        \item[Injectif]
            Si \( \psi(\bar P)=0\) nous avons \( P(f)=0\), ce qui signifie que \( P=S\mu\) pour un polynôme \( S\). Par conséquent \( P\in (\mu)\) et donc \( \bar P=0\).
        \item[Surjectif]
            Soit \( P\in \eK[X]\). L'élément \( P(f) \) de \( \eK[f]\) est dans l'image de \( \psi\) parce que c'est \( \psi(\bar P)\).
    \end{subproof}
    En ce qui concerne la dimension, le corollaire \ref{CorsLGiEN} en parle déjà : une base est donné par les projections de \( 1,X,\ldots, X^{\deg(\mu_a)-1}\).
\end{proof}

%--------------------------------------------------------------------------------------------------------------------------- 
\subsection{Polynôme caractéristique}
%---------------------------------------------------------------------------------------------------------------------------

\begin{definition}  \label{DefOWQooXbybYD}
    Soit un anneau commutatif \( A\). Si \( u\in\eM_n(A)\), nous définissons le \defe{polynôme caractéristique de \( u\)}{polynôme!caractéristique}\index{caractéristique!polynôme} :
    \begin{equation}    \label{Eqkxbdfu}
        \chi_u(X)=\det(X\mtu_n-u).
    \end{equation} 
    Nous définissons de même le polynôme caractéristique d'un 'endomorphisme \( u\colon E\to E\).
\end{definition}

\begin{lemma}       \label{LemooWCZMooZqyaHd}
    Le polynôme caractéristique \( \chi_u\) est unitaire et a pour degré la dimension de l'espace vectoriel \( E\)..
\end{lemma}

\begin{theorem}     \label{ThoNhbrUL}
    Soit \( E\) un \(\eK\)-espace vectoriel de dimension finie \( n\) et un endomorphisme \( u\in\End(E)\). Alors
    \begin{enumerate}
        \item
            Le polynôme caractéristique divise \( (\mu_u)^n\) dans \(\eK[X]\).
        \item
            Les polynômes caractéristiques et minimaux ont mêmes facteurs irréductibles dans \(\eK[X]\).
        \item
            Les polynômes caractéristiques et minimaux ont mêmes racines dans \(\eK[X]\).
        \item
            Le polynôme caractéristique est scindé si et seulement si le polynôme minimal est scindé.
    \end{enumerate}
\end{theorem}

\begin{definition}
    Si \( \lambda\in\eK\) est une racine de \( \chi_u\), l'ordre de l'annulation est la \defe{multiplicité algébrique}{multiplicité!valeur propre!algébrique} de la valeur propre \( \lambda\) de \( u\). À ne pas confondre avec la \defe{multiplicité géométrique}{multiplicité!valeur propre!géométrique} qui sera la dimension de l'espace propre.
\end{definition}

\begin{theorem} \label{ThoWDGooQUGSTL}
    Soit \( u\in\End(E)\) et \( \lambda\in\eK\). Les conditions suivantes sont équivalentes
    \begin{enumerate}
        \item\label{ItemeXHXhHi}
            \( \lambda\in\Spec(u)\)
        \item\label{ItemeXHXhHii}
            \( \chi_u(\lambda)=0\)
        \item\label{ItemeXHXhHiii}
            \( \mu_u(\lambda)=0\).
    \end{enumerate}
\end{theorem}

\begin{proof}
    \ref{ItemeXHXhHi} \( \Leftrightarrow\) \ref{ItemeXHXhHii}. Dire que \( \lambda\) est dans le spectre de \( u\) signifie que l'opérateur \( u-\lambda\mtu\) n'est pas inversible, ce qui est équivalent à dire que \( \det(u-\lambda\mtu)\) est nul par la proposition \ref{PropYQNMooZjlYlA}\ref{ItemUPLNooYZMRJy} ou encore que \( \lambda\) est une racine du polynôme caractéristique de \( u\). 

    \ref{ItemeXHXhHii} \( \Leftrightarrow\) \ref{ItemeXHXhHiii}. Cela est une application directe du théorème \ref{ThoNhbrUL} qui précise que le polynôme caractéristique a les mêmes racines dans \(\eK\) que le polynôme minimal.
\end{proof}


\begin{proposition}[\cite{RombaldiO}]\label{PropNrZGhT}
    Soit \( f\), un endomorphisme de \( E\) et \( x\in E\). Alors
    \begin{enumerate}
        \item
            L'espace \( E_{f,x}\) est stable par \( f\).
        \item\label{ItemfzKOCo}
            L'espace \( E_{f,x}\) est de dimension
            \begin{equation}
                p_{f,x}=\dim E_{f,x}=\deg(\mu_{f,x})
            \end{equation}
            où \( \mu_{f,x}\) est le générateur unitaire de \( I_{f,x}\).
        \item   \label{ItemKHNExH}
            Le polynôme caractéristique de \( f|_{E_{f,x}}\) est \( \mu_{f,x}\).
        \item   \label{ItemHMviZw}
            Nous avons
            \begin{equation}
                \chi_{f|_{E_{f,x}}}(f)x=\mu_{f,x}(f)x=0.
            \end{equation}
    \end{enumerate}
\end{proposition}

\begin{proof}
    Le fait que \( E_{f,x}\) soit stable par \( f\) est classique. Le point \ref{ItemHMviZw} est un une application du point \ref{ItemKHNExH}. Les deux gros morceaux sont donc les points \ref{ItemfzKOCo} et \ref{ItemKHNExH}.

    Étant donné que \( \mu_{f,x}\) est de degré minimal dans \( I_{f,x}\), l'ensemble
    \begin{equation}
        B=\{ f^k(x)\tq 0\leq k\leq p_{f,x}-1 \}
    \end{equation}
    est libre. En effet une combinaison nulle des vecteurs de \( B\) donnerait un polynôme en \( f\) de degré inférieur à \( p_{f,x}\) annulant \( x\). Nous écrivons
    \begin{equation}
        \mu_{f,x}(X)=X^{p_{f,x}}-\sum_{i=0}^{p_{f,x}-1}a_iX^k. 
    \end{equation}
    Étant donné que \( \mu_{f,x}(f)x=0\) et que la somme du membre de droite est dans \( \Span(B)\), nous avons \( f^{p_{f,x}}(x)\in\Span(B)\). Nous prouvons par récurrence que \( f^{p_{f,x}+k}(x)\in\Span(B)\). En effet en appliquant \( f^k\) à l'égalité
    \begin{equation}
        0=f^{p_{f,x}}(x)-\sum_{i=0}^{p_{f,x}-1}a_if^i(x)
    \end{equation}
    nous trouvons
    \begin{equation}
        f^{p_{f,x}+k}(x)=\sum_{i=0}^{p_{f,x}-1}a_if^{i+k}(x),
    \end{equation}
    alors que par hypothèse de récurrence le membre de droite est dans \( \Span(B)\). L'ensemble \( B\) est alors générateur de \( E_{f,x}\) et donc une base d'icelui. Nous avons donc bien \( \dim(E_{f,x})=p_{f,x}\).

    Nous montrons maintenant que \( \mu_{f,x}\) est annulateur de \( f\) au point \( x\). Nous savons que
    \begin{equation}
        \mu_{f,x}(f)x=0.
    \end{equation}
    En y appliquant \( f^k\) et en profitant de la commutativité des polynômes sur les endomorphismes (proposition \ref{LemQWvhYb}), nous avons
    \begin{equation}
        0=f^k\big( \mu_{f,x}(f)x \big)=\mu_{f,x}(f)f^k(x),
    \end{equation}
    de telle sorte que \( \mu_{f,x}(f)\) est nul sur \( B\) et donc est nul sur \( E_{f,x}\). Autrement dit,
    \begin{equation}
        \mu_{f,x}\big( f|_{E_{f,x}} \big)=0.
    \end{equation}
    Montrons que \( \mu_{f,x}\) est même minimal pour \( f|_{E_{f,x}}\). Sot \( Q\), un polynôme non nul de degré \( p_{f,x}-1\) annulant \( f|_{E_{f,x}}\). En particulier \( Q(f)x=0\), alors qu'une telle relation signifierait que \( B\) est un système lié, alors que nous avons montré que c'était un système libre. Nous concluons que \( \mu_{f,x}\) est le polynôme minimal de \( f|_{E_{f,x}}\).
\end{proof}

Cette histoire de densité permet de donner une démonstration alternative du théorème de Cayley-Hamilton.
\begin{theorem}[Cayley-Hamlilton]   \label{ThoCalYWLbJQ}
    Le polynôme caractéristique est un polynôme annulateur.
\end{theorem}
\index{théorème!Cayley-Hamilton}

Une démonstration plus simple via la densité des diagonalisables est donnée en théorème \ref{ThoHZTooWDjTYI}.
\begin{proof}
    Nous devons prouver que \( \chi_f(f)x=0\) pour tout \( x\in E\). Pour cela nous nous fixons un \( x\in E\), nous considérons l'espace \( E_{f,x}\) et \( \chi_{f,x}\), le polynôme caractéristique de \( f|_{E_{f,x}}\). Étant donné que \( E_{f,x}\) est stable par \( f\), le polynôme caractéristique de \( f|_{E_{j,x}}\) divise \( \chi_f\), c'est à dire qu'il existe un polynôme \( Q_x\) tel que
    \begin{equation}
        \chi_f=Q_x\chi_{f,x},
    \end{equation}
    et donc aussi
    \begin{equation}
        \chi_f(f)x=Q_x(f)\big( \chi_{f,x}(f)x \big)=0
    \end{equation}
    parce que la proposition \ref{PropNrZGhT} nous indique que \( \chi_{f,x}\) est un polynôme annulateur de \( f|_{E_{f,x}}\).
\end{proof}

\begin{corollary}
    Le degré du polynôme minimal est majoré par la dimension de l'espace.
\end{corollary}

\begin{proof}
    Le polynôme minimal engendre l'idéal des polynôme annulateurs (proposition \ref{PropAnnncEcCxj}), et divise donc le polynôme caractéristique. Or le degré du polynôme caractéristique est la dimension de l'espace par le lemme \ref{LemooWCZMooZqyaHd}.
\end{proof}

\begin{example}[Calcul de l'inverse d'un endomorphisme]
    Le polynôme de Cayley-Hamilton donne un moyen de calculer l'inverse d'un endomorphisme inversible pourvu que l'on sache son polynôme caractéristique. En effet, supposons que
    \begin{equation}
        \chi_f(X)=\sum_{k=0}^na_kX^k.
    \end{equation}
    Nous aurons alors
    \begin{equation}
        0=\chi_f(f)=\sum_{k=0}^na_kf^k.
    \end{equation}
    Nous appliquons \( f^{-1}\) à cette dernière égalité en sachant que \( f^{-1}(0)=0\) :
    \begin{equation}
        0=a_0f^{-1}+\sum_{k=1}^na_kf^{k-1},
    \end{equation}
    et donc
    \begin{equation}
        u^{-1}=-\frac{1}{ \det(f) }\sum_{k=1}^na_kf^{k-1}
    \end{equation}
    où nous avons utilisé le fait que \( a_0=\chi_f(0)=\det(f)\).
\end{example}

\begin{proposition}\label{PropooBYZCooBmYLSc}
    Si \( (X-z)^l\) (\( l\geq 1\)) est la plus grande puissance de \( (X-z)\) dans le polynôme caractéristique d'un endomorphisme \( u\) alors 
    \begin{equation}
        1\leq \dim(E_e)\leq l.
    \end{equation}
    C'est à dire que nous avons au moins un vecteur propre pour chaque racine du polynôme caractéristique.
\end{proposition}

\begin{proof}
    Si $(X-z)$ divise \( \chi_u\) alors en posant \( \chi_u=(X-z)P(X)\) nous avons
    \begin{equation}
        \det(u-X\mtu)=(X-z)P(X),
    \end{equation}
    ce qui, évalué en \( X=z\), donne \( \det(u-z\mtu)=0\). L'annulation du déterminant étant équivalente à l'existence d'un noyau non trivial, nous avons \( v\neq 0\) dans \( E\) tel que \( (u-z\mtu)v=0\). Cela donne \( u(v)=zv\) et donc que \( v\) est vecteur propre de \( u\) pour la valeur propre \( z\). Donc aussi \( \dim(E_z)\geq 1\).

    Si \( \dim(E_z)=k\) alors le théorème de la base incomplète \ref{ThonmnWKs} nous permet d'écrire une base de \( E\) dont les \( k\) premiers vecteurs forment une base de \( E_z\). Dans cette base, la matrice de \( u\) est de la forme
    \begin{equation}
        \begin{pmatrix}
             z   &       &       &   *    \\
                &   \ddots    &       &   \vdots    \\
                &       &   z    &   *    \\ 
                &       &       &   *     
         \end{pmatrix}
    \end{equation}
    où les étoiles représentent des blocs a priori non nuls. En tout cas il est vu sous cette forme que \( (X-z\mtu)^k\) divise \( \chi_u\).
\end{proof}

%+++++++++++++++++++++++++++++++++++++++++++++++++++++++++++++++++++++++++++++++++++++++++++++++++++++++++++++++++++++++++++ 
\section{Valeur propre et vecteur propre}
%+++++++++++++++++++++++++++++++++++++++++++++++++++++++++++++++++++++++++++++++++++++++++++++++++++++++++++++++++++++++++++

\begin{definition}      \label{DefooMMKZooVcskCc}
    Soit un \( \eK\)-espace vectoriel \( E\) et un endomorphisme \( A\colon V\to V\). Un \defe{vecteur propre}{vecteur!propre} de \( A\) est un vecteur \( v \neq 0\) tel que \( Av=\lambda v\) pour un certain \( \lambda\in \eK\). Dans ce cas, \( \lambda\) est la \defe{valeur propre}{valeur!propre} de \( v\).

    L'\defe{espace propre}{espace!propre} de \( A\) pour la valeur \( \lambda\)\footnote{Nous laissons au lecteur le soin de vérifier que c'est bien un sous-espace vectoriel de \( E\).} est l'ensemble des vecteurs propres de \( A\) pour la valeur propre \( \lambda\) et zéro.
\end{definition}
L'ensemble de valeurs propres de l'endomorphisme \( u\) est son \defe{spectre}{spectre!d'un endomorphisme} et est noté \( \Spec(u)\).

\begin{remark}
    Le nombre zéro peut être une valeur propre; c'est le vecteur zéro qui ne peut pas être vecteur propre. La matrice nulle est une matrice diagonalisable.
\end{remark}

\begin{lemma}       \label{LemjcztYH}
    Soit \( u\) un endomorphisme et \( E_{\lambda}(u)\)\nomenclature[A]{\( E_{\lambda}(u)\)}{Espace propre de \( u\)} ses espaces propres. La somme des \( V_{\lambda}\) est directe.
\end{lemma}

\begin{proof}
    Soit \( v_i\in V_{\lambda_i}\) un choix de vecteurs propres de \( u\). Si la somme n'est pas directe, nous pouvons considérer une combinaison linéaire des \( v_i\) qui soit nulle :
    \begin{equation}
        v_1+\cdots+v_p=0.
    \end{equation}
    Appliquons \( (A-\lambda_1\mtu)\) à cette égalité :
    \begin{equation}
        (\lambda_2-\lambda_1)v_1+\cdots+(\lambda_p-\lambda_1)v_p=0.
    \end{equation}
    En appliquant encore successivement les opérateurs \( (A-\lambda_i\mtu)\) nous réduisons le nombre de termes jusqu'à obtenir \( v_p=0\).
\end{proof}

\begin{example} \label{ExICOJcFp}
    Sur \( \eR^2\), nous considérons la matrice \( A=\begin{pmatrix}
        1    &   0    \\ 
        1    &   1    
    \end{pmatrix}\) qui a pour polynôme caractéristique le polynôme \( \chi_A=(X-1)^2\). Le nombre \( \lambda=1\) est une racine double de ce polynôme, et pourtant il n'y a qu'une seule dimension d'espace propre :
    \begin{equation}
        \begin{pmatrix}
            1    &   0    \\ 
            1    &   1    
        \end{pmatrix}\begin{pmatrix}
            x    \\ 
            y    
        \end{pmatrix}=\begin{pmatrix}
            x    \\ 
            y    
        \end{pmatrix}
    \end{equation}
    entraine \( x=0\).

    Ici la multiplicité algébrique est différente de la multiplicité géométrique.
\end{example}

\begin{proposition}[\cite{RombaldiO}]   \label{PropTVKbxU}
    Soit \( E\), un espace vectoriel sur un corps infini et \( (F_k)_{k=1,\ldots, r}\), des sous-espaces vectoriels propres\footnote{Définition \ref{DefooMMKZooVcskCc}.} de \( E\) tels que \( \bigcup_{i=1}^rF_i=E\). Alors \( E=F_k\) pour un certain \( k\).

    Autrement dit, l'union finie de sous-espaces propres ne peut être égal à l'espace complet.
\end{proposition}

La proposition suivante donne une utilisation amusante de la notion de polynôme caractéristique
\begin{proposition}[\cite{ooNGUJooPphdsT}]
    Soit un espace vectoriel \( V\) de dimension finie pour lequel il existe un endomorphisme \( f\colon V\to V\) tel que \( (f\circ f)(v)=-v\) pour tout \( v\in V\). Alors la dimension de \( V\) est paire.
\end{proposition}

\begin{proof}
    Cherchons les valeurs propres de \( f\) en résolvant l'équation \( f(v)=\lambda v\). Nous appliquons \( f\) à cette égalité :
    \begin{equation}
        -v=\lambda f(v)=\lambda^2v.
    \end{equation}
    Donc \( \lambda\) ne peut pas être réel. Nous avons montré que \( f\) n'a pas de valeurs propres réelles. Or le polynôme caractéristique de \( f\) est de degré égal à la dimension. Si la dimension est impaire, le polynôme caractéristique est de degré impair, et possède donc une racine réelle. Autrement dit, l'absence de racines réelles au polynôme caractéristique indique une dimension paire.
\end{proof}

Une autre preuve possible est d'utiliser le déterminant : si la dimension de \( V\) est \( n\) nous avons :
\begin{equation}
    \det(f^2)=\det(-\id)=(-1)^n.
\end{equation}
Donc \( (-1)^n\) est positif, ce qui montre que \( n\) est pair.


