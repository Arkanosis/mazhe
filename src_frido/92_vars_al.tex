% This is part of Agregation : modélisation
% Copyright (c) 2011-2015
%   Laurent Claessens
% See the file fdl-1.3.txt for copying conditions.

%---------------------------------------------------------------------------------------------------------------------------
\subsection{Fonction de répartition}
%---------------------------------------------------------------------------------------------------------------------------

Si \( X\) est une variable aléatoire réelle, nous définissons sa \defe{fonction de répartition}{fonction!de répartition} par
\begin{equation}
    \begin{aligned}
        F_X\colon \eR&\to \mathopen[ 0 , 1 \mathclose] \\
        F_X(x)&=P(X\leq x). 
    \end{aligned}
\end{equation}

%---------------------------------------------------------------------------------------------------------------------------
\subsection{Fonction caractéristique}
%---------------------------------------------------------------------------------------------------------------------------

La \defe{fonction caractéristique}{fonction!caractéristique!d'une variable aléatoire} de la variable aléatoire \( X\colon \Omega\to \eR\) est la fonction réelle définie par
\begin{equation}
    \Phi_X(t)=E( e^{itX}).
\end{equation}
Une autre façon d'écrire la définition est
\begin{equation}
    \Phi_X(t)=\int_{\eR} e^{itX}dP_X(x),
\end{equation}
ou encore, si \( X\) a une densité \( f_X\),
\begin{equation}        \label{EqFnCaractfncadens}
    \Phi_X(t)=\int_{\eR} e^{itx}f_X(x)dx
\end{equation}
Nous reconnaissons la transformée de Fourier :
\begin{equation}
    \Phi_X(t)=\hat f_X(-t/2\pi).
\end{equation}

La proposition suivante se déduit en utilisant le théorème de dérivation sous l'intégrale \ref{ThoDerSousIntegrale}.
\begin{proposition}     \label{PropDerFnCaract}
    Soit $X$ une variable aléatoire qui accepte un moment d'ordre \( r\geq 1\). Alors la fonction caractéristique \( \Phi_X\) est \( r\) fois continument dérivable et
    \begin{equation}
        \Phi_X^{(r)}(t)=E\big( (iX)^r e^{itX} \big).
    \end{equation}
\end{proposition}
\index{transformée!de Fourier}

\begin{proof}
    Nous étudions la fonction
    \begin{equation}
        \Phi(t)=\int_{\Omega} e^{itX(\omega)}dP(\omega).
    \end{equation}
    Nous considérons la fonction
    \begin{equation}
        \begin{aligned}
            f\colon \eR\times\Omega&\to \eR \\
            (t,\omega)&\mapsto  e^{itX(\omega)}. 
        \end{aligned}
    \end{equation}
    et nous regardons si ce contexte vérifie les hypothèses du théorème \ref{ThoDerSousIntegrale}.
    \begin{enumerate}
        \item
            Étant donné que \( X\) est mesurable, \( f\) sera mesurable.
        \item
            La fonction \( t\mapsto e^{itX(\omega)}\) est absolument continue pour chaque \( \omega\).
        \item
            Note : par rapport aux notations du théorème \ref{ThoDerSousIntegrale}, nous avons ici \( A=\eR\). Prenons donc un intervalle (compact) \( \mathopen[ a , b \mathclose]\subset\eR\) et calculons
            \begin{equation}        \label{EqfpfpttoiXieitXo}
                \frac{ \partial f }{ \partial t }(t,\omega)=iX(\omega) e^{itX},
            \end{equation}
            et
            \begin{equation}
                \int_a^b\int_{\Omega}\left| iX(\omega) e^{itX(\omega)} \right| d\omega\,dt=\int_a^b\int_{\Omega}| X(\omega) |d\omega\,dt.
            \end{equation}
            Par hypothèse \( X\) accepte un moment d'ordre \( 1\), de sorte que l'intégrale par rapport à \( \omega\) converge vers un nombre qui ne dépend pas de \( t\). L'intégrale sur \( t\) ne pose alors aucun problèmes.
    \end{enumerate}
    Par conséquent nous pouvons effectuer la première dérivation :
    \begin{equation}
        \Phi'(t)=\frac{ d\Phi }{ dt }(t)=\int_{\Omega}iX(\omega) e^{itX(\omega)}d\omega=E(iX e^{itX})
    \end{equation}
    et la fonction \( \Phi'\) est absolument continue. Ce dernier point est important parce que c'est lui qui permet de faire la récurrence et passer à l'ordre deux.

    Le résultat ressort alors en dérivant successivement l'expression \eqref{EqfpfpttoiXieitXo}.
\end{proof}

\begin{example}
    Sachant la fonction caractéristique de \( X\), nous pouvons calculer les moments. Par exemple
    \begin{equation}
        E(X^2)=\Phi''_X(0).
    \end{equation}
\end{example}

\begin{theorem}     \label{ThonMxtTy}
    Si \( \Phi_X=\Phi_Y\), alors \( P_X=P_Y\).
\end{theorem}
%TODO : trouver une preuve.
Notons que cela n'implique pas que \( X=Y\). En effet \( X\) et \( Y\) peuvent même être définis sur des espaces probabilisés différents.

Dans le cas d'une variable aléatoire vectorielle, nous définissons \( \Phi_X\colon \eR^d\to \eR\) par
\begin{equation}        \label{EqydvDxg}
    \Phi_X(v)=E\big(  e^{i\langle v, X\rangle } \big)
\end{equation}

%---------------------------------------------------------------------------------------------------------------------------
\subsection{Fonction génératrice des moments, transformée de Laplace}
%---------------------------------------------------------------------------------------------------------------------------

Soit \( X\) une variable aléatoire. Sa \defe{transformée de Laplace}{transformée!Laplace} ou \defe{fonction génératrice des moments}{moment!fonction génératrice}\index{fonction!génératrice} est la fonction
\begin{equation}
    M_X(t)=E( e^{tX})
\end{equation}
pour chaque \( t\) tel que cette espérance existe.

\begin{theorem}[\cite{LetacGen}]
    Soit \( X\) une variable aléatoire réelle et
    \begin{equation}
        I_X=\{ t\in \eR\tq \text{ $E( e^{tX})$ existe}\}.
    \end{equation}
    La fonction
    \begin{equation}
        \begin{aligned}
            M_X\colon I_X&\to \eR \\
            t&\mapsto E( e^{tX}) 
        \end{aligned}
    \end{equation}
    est la \defe{transformée de Laplace}{transformée!Laplace}\index{Laplace!transformée} de \( X\). 
    \begin{enumerate}
        \item
            \( I_X\) est un intervalle contenant \( 0\).
        \item
            Si \( I_X\) n'est pas réduit à \( \{ 0 \}\) alors \( M_X\) se développe en série entière
            \begin{equation}
                M_X(t)=\sum_{n=0}^{\infty}\frac{ E(X^n) }{ n! }t^n.
            \end{equation}
        \item
            Si \( X\) et \( Y\) sont des variables aléatoires indépendantes, alors \( I_{X+Y}=I_X\cap I_Y\) et 
            \begin{equation}
                M_{X+Y}=M_XM_Y
            \end{equation}
            sur \( I_{X+Y}\).
    \end{enumerate}
\end{theorem}
\index{transformée!Laplace}
\index{limite!permutation!utilisation}
\index{série!entière!utilisation}

\begin{proof}
    Le fait que \( 0\) soit dans \( I_X\) est évident : \( E(1)=1\). Pour montrer que \( I_X\) est un intervalle nous prenons \( z\in I_X\) et \( 0<s<z\) ou \( z<s<0\), puis nous montrons que \( s\in I_X\). Il faut remarquer que dans tous les cas,
    \begin{equation}
        e^{sX}\leq 1+ e^{zX}.
    \end{equation}
    En effet soit \( sX\) et \( zX\) sont tous deux à gauche de zéro et alors ils sont tous deux plus petit que \( 1\); soit ils sont tous deux à droite de \( 0\) et alors \( e^{zX}> e^{sX}\) par croissance de l'exponentielle. Nous avons donc dans tous les cas que
    \begin{equation}
        E( e^{sX})=\int_{\eR}f_X(x) e^{sX}dx\leq \int_{\eR}f_X(x)(1+ e^{zx})=1+E( e^{zX}).
    \end{equation}
    
    Soit maintenant \( a>0\) tel que \( \mathopen[ -a , a \mathclose]\in I_X\). Étant donné que \(  e^{a| X |}< e^{aX} e^{-aX}\), l'espérance \( E( e^{a| X |})\) existe toujours pour \( | t |\). Nous avons
    \begin{subequations}
        \begin{align}
            \left| M_X(t)-\sum_{n=0}^N\frac{ E(X^n) }{ n! }t^n \right| &=\left| E\Big(  e^{tX}-\sum_{n=0}^N\frac{ X^n }{ n! }t^n \Big) \right| \\
            &=\left| E\Big( \sum_{n=N+1}^{\infty}\frac{ X^n }{ n! }t^n \Big) \right| \\
            &\leq E\left( \sum_{n=N+1}^{\infty}\frac{ | tX |^n }{ n! } \right).
        \end{align}
    \end{subequations}
    Maintenant le but est de prendre la limite \( N\to\infty\) en inversant la limite et l'espérance par le théorème de la convergence dominée (\ref{ThoConvDomLebVdhsTf}). L'intégrale à traiter est
    \begin{equation}
        \lim_{N\to \infty} \int_{\Omega}\sum_{n=N+1}^{\infty}\frac{ | tX(\omega) |^n }{ n! }dP(\omega).
    \end{equation}
    L'intégrante est uniformément borné (en \( N\)) par \(  e^{tX(\omega)}\), qui est intégrable par hypothèse (choix de \( t\)). Du coup
    \begin{equation}
        \lim_{N\to \infty} \int_{\Omega}\sum_{n=N+1}^{\infty}\frac{ | tX(\omega) |^n }{ n! }dP(\omega)=E\left( \lim_{N\to \infty} \sum_{n=N+1}^{\infty}\frac{ | tX |^n }{ n! } \right)=0.
    \end{equation}
\end{proof}

%---------------------------------------------------------------------------------------------------------------------------
\subsection{Loi d'une variable aléatoire}
%---------------------------------------------------------------------------------------------------------------------------

La \defe{loi}{loi!d'une variable aléatoire} de la variable aléatoire \( X\), notée \( P_X\) est la mesure image de \( P\) par \( X\), c'est à dire
\begin{equation}
    P_X(B)=P(X\in B)
\end{equation}
pour tout borélien \( B\subset\eR^d\). Note :
\begin{equation}
    P(X\in B)=P\big( \{ \omega\in\Omega\tq X(\omega)\in B \} \big)=P\big( X^{-1}(B) \big).
\end{equation}
En particulier \( P_X\) est une mesure de probabilité sur \( \eR^d\) parce que 
\begin{equation}
    P_X(\eR^d)=P(\Omega)=1.
\end{equation}
Si \( Q\) est une mesure de probabilité sur \( \eR^d\), nous notons \( X\sim Q\) si \( P_X=Q\). Nous disons alors que «\( X\) suit la loi \( Q\)».

La proposition suivante permet de calculer en pratique les intégrales qui définissent par exemple l'espérance mathématique d'une variable aléatoire.
\begin{proposition}[Théorème de transfert\cite{ThjoTrnSaada}]\label{PropintdPintdPXeR}
    Si \( X\) est une variable aléatoire, alors
    \begin{equation}
        E(f\circ X)=\int_{\Omega}f\big( X(\omega) \big)dP(\omega)=\int_{\eR^d}f(x)dP_X(x)
    \end{equation}
    dès que \( f\colon \eR^d\to \bar\eR\) est telle qu'une des deux intégrales existe. En particulier, ça marche si \( f\) est borélienne.
\end{proposition}
\index{théorème!transfert}
% TODO : donner une preuve de ce théorème.


En utilisant cette proposition nous trouvons une formule pratique pour l'espérance d'une variable aléatoire réelle:
\begin{equation}
    E(X)=\int_{\Omega}X(\omega)dP(\omega)=\int_{\eR}xdP_X(x),
\end{equation}
en vertu de la proposition \ref{PropintdPintdPXeR} appliquée à la fonction \( f(x)=x\).

\begin{proposition}
    Une variable aléatoire réelle \( X\) est intégrable si et seulement si \( P(x=\pm\infty)=0\) et
    \begin{equation}
        \int_{\eR}| x |dP_X(x)<\infty.
    \end{equation}
\end{proposition}

Le lien entre la densité \( f_X\) de la variable aléatoire \( X\) et sa loi est
\begin{equation}
    P_X(A)=\int_Af_X(x)dx
\end{equation}
pour tout ensemble mesurable \( A\subset\eR\). Le lien entre la mesure de Lebesgue et celle de la loi de \( X\) est alors donné par
\begin{equation}
    dP_X(x)=f_X(x)dx.
\end{equation}
En particulier l'espérance de \( X\) peut être calculée à partir de sa densité via la formule
\begin{equation}        \label{EqEspDensform}
    E(X)=\int_{\eR}xdP_X(x)=\int_{\eR}x f_X(x)dx.
\end{equation}

%---------------------------------------------------------------------------------------------------------------------------
\subsection{Changement de variables}
%---------------------------------------------------------------------------------------------------------------------------

\begin{theorem}
    Soit \( \mO\), un ouvert de \( \eR^n\) et \( \mO'\) un ouvert de \( \eR^m\) ainsi qu'un difféomorphisme \( C^1\) \( \varphi\colon \mO\to \mO'\). Soit \( X\colon \Omega\to \eR^n\) une variable aléatoire prenant presque sûrement ses valeurs dans \( \mO\). Si nous supposons que \( X\) a la densité \( f_X\), alors la variable aléatoire \( Y=\varphi(X)\) accepte la densité \( f_Y\colon \mO'\to \eR\) donnée par
    \begin{equation}
        f_Y(v)=f_X\big( \varphi^{-1}(v) \big)| J_{\varphi^{-1}}(v) |.
    \end{equation}
\end{theorem}

\begin{proof}
    Nous devons vérifier la relation 
    \begin{equation}
        P(Y\in B)=\int_Bf_Y(v)dv
    \end{equation}
    pour tout borélien \( B\subset \mO'\). Nous avons
    \begin{subequations}
        \begin{align}
            P(Y\in B)&=\int_{\eR^m}\mtu_{B}(v)dP_Y(v)\\
            &=E(\mtu_B\circ Y)\\
            &=E\big( (\mtu_B\circ \varphi)\circ X\big)\\
            &=\int_{\eR^n}(\mtu_B\circ\varphi)(u)dP_X(u)\\
            &=\int_{\eR^n}(\mtu_B \circ\varphi)(u)f_X(u)du.
        \end{align}
    \end{subequations}
    À ce niveau, nous utilisons la formule de changement de variables du théorème \ref{THOooUMIWooZUtUSg}. Nous trouvons alors
    \begin{equation}
        P(Y\in B)=\int_{\eR^m}\mtu_B\big( \varphi^{-1}(v) \big)f_X\big( \varphi^{-1}(v) \big)| J_{\varphi^{-1}}(v) |dv.
    \end{equation}
\end{proof}

%+++++++++++++++++++++++++++++++++++++++++++++++++++++++++++++++++++++++++++++++++++++++++++++++++++++++++++++++++++++++++++
\section{Convergence}
%+++++++++++++++++++++++++++++++++++++++++++++++++++++++++++++++++++++++++++++++++++++++++++++++++++++++++++++++++++++++++++

Soient \( X_i\) des variables aléatoires réelles définies sur le même espace de probabilité \( (\Omega,\tribA,P)\). Nous disons que \( X_i\) converge \defe{presque sûrement}{convergence!presque sûrement} vers la variable aléatoire \( X\) et nous notons 
\begin{equation}
    X_n\stackrel{p.s.}{\longrightarrow}X
\end{equation}
si
\begin{equation}
    P\big( \{ \omega\in\Omega\tq\,X_n(\omega)\to X(\omega) \} \big)=1
\end{equation}
où la convergence \( X_n(\omega)\to X(\omega)\) est la convergence usuelle dans \( \eR\).

\begin{lemma}
    Nous avons \( X_n\stackrel{p.s.}{\longrightarrow}X\) si et seulement s'il existe un événement \( A\in\tribA\) tel que \( P(A)=1\) et tel que \( X_n(\omega)\to X(\omega)\) pour tout \( \omega\in A\).
\end{lemma}

\begin{lemma}
    Si \( X\) est une variable aléatoire à valeurs dans \( \eR\cup\{ \infty \}\), alors
    \begin{equation}
        X\wedge n\stackrel{p.s.}{\longrightarrow}X
    \end{equation}
\end{lemma}

\begin{proof}
    Si \( \omega\in\{ X=\infty \}\) alors \( (X\wedge n)(\omega)=n\) et d'accord. Si par contre \( \omega\in\{ X<\infty \}\) alors il existe \( N\) tel que si \( n\geq N\) alors \( n\geq T(\omega)\) et pour ces grandes valeurs de \( n\) nous avons \( (T\wedge n)(\omega)=T(\omega)\).
\end{proof}


Nous disons que les variables aléatoires réelles \( X_n\) convergent \defe{en probabilité}{convergence!en probabilité} vers la variable aléatoire \( X\) si pour tout \( \eta>0\), on a
\begin{equation}
    P\big( | X_n-X |\geq \eta \big)\to 0,
\end{equation}
et on note
\begin{equation}
    X_n\stackrel{P}{\longrightarrow}X.
\end{equation}

Nous disons que \( X_n\) converge vers \( X\) \defe{en loi}{convergence!en loi} vers la variable aléatoire \( X\) et nous notons
\begin{equation}
    X_n\stackrel{\hL}{\longrightarrow}X
\end{equation}
si pour toute fonction continue et bornée \( g\) nous avons
\begin{equation}
    E\big( g(X_n) \big)\to E\big( g(X) \big)=\int gdP_X.
\end{equation}

\begin{proposition}     \label{PrpopCaractCvL}
    Deux autres caractérisations de la convergence en loi.
    \begin{enumerate}
        \item
            Nous avons \( X_n\stackrel{\hL}{\longrightarrow}X\) si et seulement si
            \begin{equation}
                \Phi_{X_n}(v)\to\Phi_X(v)
            \end{equation}
            pour tout \( v\in\eR^d\). Ici \( \Phi_X\) est la fonction caractéristique de \( X\).
        \item
            Dans la définition de la convergence en loi nous pouvons indifféremment utiliser les fonctions continues et bornées, les fonctions continues à support compact ou les fonctions bornées uniformément continues.
    \end{enumerate}
\end{proposition}

\begin{proposition} \label{PropJFVJDuX}
    Les types de convergence sont reliées par les implications suivantes :
    \begin{equation}
        \text{presque sûre }\Rightarrow\text{ en probabilité }\Rightarrow\text{ en loi}.
    \end{equation}
\end{proposition}
La convergence en loi n'implique pas la convergence en probabilité, et par conséquent pas non plus la convergence presque certaine.
%TODO : trouver des contre-exemples.

Dans le cas particulier \( d=1\) nous avons quelques critères supplémentaires. 

\begin{proposition}     \label{PropoFnrepCvL}
    Supposons que les variables aléatoires \( X_n\) soient réelles, et notons \( F_n\) la fonction de répartition de \( X_n\). Si \( F_n(x)\to F(x)\) pour tout \( x\) dans l'ensemble des points de continuité de \( F\), alors \( X_n\stackrel{\hL}{\longrightarrow}X\).
\end{proposition}

\begin{proposition}
    Si les \( X_n\) sont des variables aléatoires réelles positives, et si \( X\) est une variable aléatoire positive, alors \( X_n\stackrel{\hL}{\longrightarrow}X\) si les transformées de Laplace des fonctions de répartition convergent ponctuellement, c'est à dire si
    \begin{equation}
        E\big(  e^{-\alpha X_n} \big)\to E\big(  e^{-\alpha X} \big)
    \end{equation}
    pour tout \( \alpha\geq 0\).
\end{proposition}

\begin{proposition}
    Si les \( X_n\) et \( X\) sont des variables aléatoires réelles discrètes à valeurs dans \( \{ x_0,x_1,\ldots \}\) alors \( X_n\stackrel{\hL}{\longrightarrow} X\) si et seulement si
    \begin{equation}
        P(X_n=x_k)\to P(X=x_k)
    \end{equation}
    pour tout \( k\in\eN\).
\end{proposition}

\begin{proposition}[\cite{CourgGudRennes}]     \label{PropXncvXFXcvFxt}
    Soient \( X_n\) et \( X\) des variables aléatoires réelles. Nous avons
    \begin{equation}
        X_n\stackrel{\hL}{\longrightarrow}X
    \end{equation}
    si et seulement si pour tout \( t\) où \( F_X\) est continue,
    \begin{equation}
        \lim_{n\to \infty} F_{X_n}(t)=F_X(t).
    \end{equation}
\end{proposition}

\begin{proposition}[\cite{CourgGudRennes}]     \label{PropCvLfcvPsicst}
    Soit \( X_n\) une suite de variables aléatoires \( \Omega\to \eR^d\) et \( a\in\eR^d\). Si \( X_n\stackrel{\hL}{\longrightarrow}a\), alors
    \begin{equation}
        X_n\stackrel{P}{\longrightarrow}a.
    \end{equation}
\end{proposition}

\begin{proof}
    Quitte à passer aux composantes, nous pouvons supposer que \( d=1\). Nous avons
    \begin{subequations}
        \begin{align}
           P\big( | X_n-a |>\eta \big)&=P(X_n-a>\eta)+P(-X_n-a>\eta)\\
           &=P(X_n>\eta+a)+1-P(X_n>a-\eta)  \label{EqPXngeqetaapumP}\\
           &=1-F_{X_n}(\eta+a)+F_{X_n}(a-\eta).
        \end{align}
    \end{subequations}
    Nous allons utiliser la proposition \ref{PropXncvXFXcvFxt}. La fonction de répartition de la variable aléatoire constante \( X=a\) est donnée par
    \begin{equation}
        F_a(t)=P(a<t)=\caract{\eR^+}(t-a).
    \end{equation}
    Par conséquent, la convergence en loi \( X_n\stackrel{\hL}{\longrightarrow}a\) nous montre que
    \begin{equation}
        F_{X_n}(t)\to \caract_{\eR^+}(t-a)
    \end{equation}
    pour tout \( t\neq a\) parce que \( t=0\) est un point de discontinuité de \( \caract_{\eR^+}\). Nous avons par conséquent
    \begin{equation}
        P\big( | X_n-a |>\eta \big)=1-\caract_{\eR^+}(\eta)+\caract_{\eR^+}(-\eta)=0
    \end{equation}
    parce que \( \eta>0\).
\end{proof}

% TpijUi
\begin{probleme}
    Dans \cite{CourgGudRennes}, l'équation \eqref{EqPXngeqetaapumP} arrive avec une inégalité. Pourquoi ?
\end{probleme}

Le lemme de Slutsky sera utilisé en combinaison avec la proposition \ref{PropcvLsousfonc} pour calculer des intervalles de confiance, voir par exemple ce qui se passe autour de l'équation \eqref{EqGJxxjqs}.
\begin{lemma}[Slutsky\cite{MPSmaitrise}]  \label{LemgXDlhs}
    Soient \( X_n\) et \( Y_n\) des suites de variables aléatoires réelles telles que
    \begin{equation}
        \begin{aligned}[]
            X_n&\stackrel{\hL}{\longrightarrow} X\\
            Y_n&\stackrel{P}{\longrightarrow}a\in\eR.
        \end{aligned}
    \end{equation}
    Alors \( (X_n,Y_n)\stackrel{\hL}{\longrightarrow}(X,a)\).
\end{lemma}
\index{lemme!de Slutsky}

\begin{proof}
    Étant donné que \( Y_n\stackrel{\hL}{\longrightarrow} a\), nous avons \( Y_n\stackrel{P}{\longrightarrow} a\) par la proposition \ref{PropCvLfcvPsicst}. Soit une fonction \(f\colon \eR^2\to \eR^2 \); nous devons prouver que
    \begin{equation}
        E\big( f(X_n,Y_n) \big)\to E\big( f(X,a) \big).
    \end{equation}
    Soit \( \epsilon>0\). Nous avons
    \begin{equation}    \label{EqEparXnYnfXa}
        E\big( \| f(X_n,Y_n)-f(X,a) \| \big)\leq E\big(  \| f(X_n,Y_n)-f(X_n,a) \|  \big)+E\big(   \| f(X_n,a)-f(X,a) \|  \big).
    \end{equation}
    La fonction \( g(t)=f(t,a)\) étant continue et bornée, la convergence en loi \( X_n\stackrel{\hL}{\longrightarrow}X\) donne
    \begin{equation}
        E\big( \| f(X_n,a)-f(X,a) \| \big)\to 0.
    \end{equation}
    Étudions à présent le premier terme du membre de droite de \eqref{EqEparXnYnfXa}. Pour tout \( \eta> 0\) et toute variables aléatoires \( Z\) et \( Z'\) nous avons
    \begin{equation}
        E(Z)=E(Z\caract_{| Z' |<\eta})+E(Z\caract_{| Z' |\geq \eta}).
    \end{equation}
    Nous décomposons donc le premier terme de \eqref{EqEparXnYnfXa} en
    \begin{equation}    \label{EqEXnADecomsecvolta}
        \begin{aligned}[]
            E\big( \| f(X_n,Y_n)-f(X_n,a) \| \big)&=E\big( \| f(X_n,Y_n)-f(X_n,a) \|\caract_{| Y_n-a |<\eta} \big)\\
            &\quad+E\big( \| f(X_n,Y_n)-f(X_n,a) \|\caract_{| Y_n-a |\geq\eta} \big).
        \end{aligned}
    \end{equation}
    Choisissons maintenant une valeur de \( \eta\) telle que
    \begin{equation}
        | (x,y)-(x',y') |<\eta\Rightarrow| f(x,y)-f(x',y') |\leq \epsilon.
    \end{equation}
    Un tel \( \eta\) existe par l'uniforme continuité de \( f\). Dans le premier terme, \( | Y_n-a |<\eta\), par conséquent
    \begin{equation}
        \| (X_n,Y_n)-(X_n,a) \|=| Y_n-a |<\eta
    \end{equation}
    et donc
    \begin{equation}
        \| f(X_n,Y_n)-f(X_n,a) \|\leq \epsilon.
    \end{equation}
    Le premier terme devient donc
    \begin{equation}
        E\big( \| f(X_n,Y_n)-f(X_n,a) \|\caract_{| Y_n-a |<\eta} \big)\leq \epsilon E(\caract_{| Y_n-a |<\eta})\leq \epsilon
    \end{equation}
    parce que \( E(\caract_A)=P(A)\leq 1\). Pour le second terme de \eqref{EqEXnADecomsecvolta} nous effectuons la majoration
    \begin{equation}
        \| f(X_n,Y_n)-f(X_n,a) \|\leq 2\| f \|_{\infty}
    \end{equation}
    tandis que la convergence \( Y_n\stackrel{P}{\longrightarrow} a\) entraine 
    \begin{equation}
        P\big( | Y_n-a |\geq \eta \big).
    \end{equation}
\end{proof}

\begin{proposition}[\cite{JPzkRmm}]     \label{PropcvLsousfonc}
    Soient \( X_i\) des variables aléatoires telles que
    \begin{equation}
        X_i\stackrel{\hL}{\longrightarrow} X
    \end{equation}
    et \( h\), une fonction mesurable sur l'espace d'arrivée de \( X_i\). Soit \( C\) l'ensemble des points de continuité de \( h\) au sens
    \begin{equation}
        C=\{ \omega\in\Omega\tq \text{\( h\) est continue en \( X_i(\omega)\)} \}.
    \end{equation}
    Alors si \( P(X\in C)=1\), nous avons 
    \begin{equation}
        h(X_i)\stackrel{\hL}{\longrightarrow}h(X).
    \end{equation}
\end{proposition}
%TODO : une preuve; elle a l'air d'être sur Wikipédia.

Une conséquence de cette proposition couplée au lemme de Slutsky est le résultats suivant, qui est donné sous le nom de \wikipedia{en}{Slutsky's_theorem}{théorème de Slutsky} sur wikipédia.
\begin{corollary}       \label{CorINgTPH}
    En reprenant les notations du lemme de Slutsky, si
    \begin{subequations}
        \begin{align}
            X_n&\stackrel{\hL}{\longrightarrow} X\\
            Y_n&\stackrel{P}{\longrightarrow} a,
        \end{align}
    \end{subequations}
    alors
    \begin{subequations}
        \begin{align}
            X_n+Y_n&\stackrel{\hL}{\longrightarrow}X+a\\
            X_nY_n&\stackrel{\hL}{\longrightarrow}aX\\
            Y_n^{-1}X_n&\stackrel{\hL}{\longrightarrow}a^{-1}X\\
        \end{align}
    \end{subequations}
    pourvu que \( a\) soit inversible.
\end{corollary}


\begin{lemma}[Borel-Cantelli]\index{lemme!de Borel-Cantelli}
    Soit \( (A_n)\) une suite d'événements (avec \( A_n\in\tribA\) pour tout \( n\)).
    \begin{enumerate}
        \item
            Si \( \sum_{n=0}^{\infty}P(A_n)\) converge, alors
            \begin{equation}
                P(A_n\,\text{i.s.})=0.
            \end{equation}
        \item
            Si la somme \( \sum_nP(A_n)\) diverge, et si de plus les \( A_i\) sont indépendants, alors
            \begin{equation}
                P(A_n\,\text{i.s.})=1.
            \end{equation}
    \end{enumerate}
\end{lemma}
La notation \( P(A_n\,\text{i.s.})\) signifie «infiniment souvent», c'est à dire
\begin{equation}
    P(A_n\,\text{i.s.})=P\big( \bigcap_{N\in\eN}\bigcup_{k\geq N}A_k \big)=P(\limsup A_n)
\end{equation}

Une façon de paraphraser le lemme de Borel-Cantelli est que nous avons l'alternative
\begin{equation}    \label{EqparaphrCantelli}
    P(\limsup A_n)=\begin{cases}
        0    &   \text{si $\sum_{n\geq 0}P(A_n)<\infty$}\\
        1    &    \text{sinon}.
    \end{cases}
\end{equation}

\begin{proposition}
    Soit \( X_n\), une suite de variables aléatoires et \( X\) une variable aléatoires. Si
    \begin{equation}
        \sum_nP\big( \| X_n-X \|>\eta \big)<\infty
    \end{equation}
    pour tout \( \epsilon\), alors \( X_n\stackrel{p.s.}{\longrightarrow}X\).
\end{proposition}

\begin{proof}
    Fixons \( \epsilon\) et considérons les événements \( A_{n}=\| X_n-X \|>\epsilon\). L'hypothèse dit que
    \begin{equation}
        \sum_nP(A_{n,\epsilon})<\infty
    \end{equation}
    et le lemme de Borel-Cantelli implique que
    \begin{equation}
        P(\limsup\| X_n-X \|>\epsilon)=0.
    \end{equation}
    Un élément \( \omega\) est dans \( \limsup A_n\) s'il est contenu dans tous les \( A_n\), par conséquent, pour chaque \( \epsilon\) nous avons l'inclusion
    \begin{equation}
        \{ \omega\in\Omega\tq X_n(\omega)\to X(\omega) \}\subset\complement\limsup A_n.
    \end{equation}
    Nous pouvons aller plus loin et écrire
    \begin{equation}        \label{EqProbomOmXnmXforalleps}
        \{ \omega\in\Omega\tq X_n(\omega)\to X(\omega) \}=\complement\{ \omega\in\Omega\tq \| X_n-X \|>\epsilon,\forall\epsilon>0 \}.
    \end{equation}
    Or la probabilité de l'ensemble
    \begin{equation}
        \{ \omega\in\Omega\tq\| X_n-X \|>\epsilon \}
    \end{equation}
    est \( 0\) pour chaque \( \epsilon\), et par conséquent la probabilité du membre de droite de \eqref{EqProbomOmXnmXforalleps} est \( 1\).
\end{proof}

\begin{example}
    Considérons une suite de \( 0\) et de \( 1\) dans laquelle le \( 1\) arrive avec une probabilité \( p\) et le \( 0\) avec une probabilité \( 1-p\). Une telle suite est modélisée par une suite de variables aléatoires de Bernoulli \( (X_n)_{n\in\eN}\) indépendantes de paramètre \( p\).

    Question : une telle suite contient elle une infinité de \( 1\) ? Considérons les événements indépendants \( A_n=\{ X_n=1 \}\). Nous avons
    \begin{equation}
        \sum_n P(A_n)=\sum_nP(X_n=1)=\sum_np=\infty.
    \end{equation}
    Par Borel-Cantelli et son expression \eqref{EqparaphrCantelli}, nous avons alors
    \begin{equation}
        P(\limsup A_n)=1.
    \end{equation}
    Donc une infinité d'événements \( A_n\) se produisent, et nous avons bien une infinité de \( 1\) dans la suite.

    Remarque : dans ce raisonnement nous pouvons considérer une probabilité non constante \( p_n\) tant que la série \( \sum_np_n\) diverge.
\end{example}

\begin{example}
    À propos de maximum. La fonction \( h\colon \eR^d\to \eR\) donnée par \( h(x_1,\ldots, x_d)=\max_i\{ x_i \}\) est une fonction continue. Nous voudrions prouver que si on a une famille (finie en \( i=1,\ldots, l\)) de suites (en \( n\)) variables aléatoires \( X^{(i)}_n\stackrel{p.s.}{\longrightarrow} a\) convergeant toutes vers la même limite \( a\), alors
    \begin{equation}
        M_n=\max_i\{ X^{(i)}_n \}\stackrel{p.s.}{\longrightarrow} a.
    \end{equation}
    D'abord si nous avons \( l\) suite numériques \( (x^{(i)}_n)\), alors la suite
    \begin{equation}
        M_n=\max_ix_n^{(i)}
    \end{equation}
    converge vers la même limite. En effet si \( \epsilon>0\) est donné, il suffit de prendre \( N_i\) l'entier tel que \( | x^{(i)}_n-a |\leq \epsilon\) pour tout \( n>N_i\). Et ensuite on prend \( N>\max\{ N_i \}\).

    Si maintenant au lieu de suites numériques, nous avons des variables aléatoires, le résultat reste valable. Nous cherchons à prouver que
    \begin{equation}\label{EqMZcnwTh}
        P\Big( \{ \omega\in \Omega\tq \max\{ X^{(i)}_n(\omega) \}\to a \} \Big)=1.
    \end{equation}
    Par ce que nous venons de dire sur les suites numériques, un élément \( \omega\) n'est
    pas dans cet ensemble seulement s'il y a un \( \infty\) pour lequel \( X^{(i)}_n\) ne converge pas vers \( a\). Or cela, pour chaque \( i\) est un événement de probabilité zéro.

    Les \( \omega\) qui ne fonctionneront pas dans l'équation \eqref{EqMZcnwTh} sont ceux de la réunion d'un ensemble fini d'ensembles de probabilité nulle. C'est donc de probabilité nulle.
\end{example}

%+++++++++++++++++++++++++++++++++++++++++++++++++++++++++++++++++++++++++++++++++++++++++++++++++++++++++++++++++++++++++++
\section{Loi des grands nombres, théorème central limite}
%+++++++++++++++++++++++++++++++++++++++++++++++++++++++++++++++++++++++++++++++++++++++++++++++++++++++++++++++++++++++++++

%---------------------------------------------------------------------------------------------------------------------------
\subsection{Loi des grands nombres}
%---------------------------------------------------------------------------------------------------------------------------

\begin{lemma}[Inégalité de Markov\cite{SGpYnKH}]
    Soit une variable aléatoire \( X\in L^p\) et \( \epsilon>0\). Nous avons
    \begin{equation}
        P(| X |\geq \epsilon)\leq \frac{1}{ \epsilon^r }E(| X |^r).
    \end{equation}
\end{lemma}
\index{Markov!inégalité}
\index{inégalité!Markov}

\begin{proof}
    Nous avons
    \begin{equation}
        E(| X |^r)\geq\int_{| X |\geq \epsilon}X(\omega)dP(\omega)\geq \epsilon^r\int_{| X |\geq\epsilon}dP=\epsilon^rP(| X |\geq\epsilon).
    \end{equation}
\end{proof}

\begin{corollary}[\cite{SGpYnKH}]   \label{CorEWhIsBB}
    Soit \( \phi\), une fonction croissante et positive ou nulle sur l'intervalle \( I\). Soit aussi une variable aléatoire \( Y\colon \Omega\to \eR\) telle que \( P(Y\in I)=1\). Alors pour tout \( b\in I\) tel que \( \phi(b)>0\) nous avons
    \begin{equation}
        P(Y\geq b)\leq \frac{ E\big[ \phi(Y) \big] }{ \phi(b) }.
    \end{equation}
\end{corollary}

\begin{theorem}[Loi forte des grands nombres]     \label{ThoefQyKZ}
    Soit \( (X_n)\) une suite de variables aléatoires réelles
    \begin{enumerate}
        \item
            indépendantes et identiquement distribuées,
        \item
            intégrables (c'est à dire dans \( L^1\)),
    \end{enumerate}
    alors
    \begin{equation}
        \frac{1}{ n }\sum_{i=1}^nX_i  \stackrel{p.s.}{\longrightarrow} E(X_1).
    \end{equation}
\end{theorem}
\index{loi!des grands nombres!forte}
Note : étant donné que les variables aléatoires sont identiquement distribuées, nous avons évidemment \( E(X_1)=E(X_2)=\ldots\)

\begin{probleme}
    Est-ce que les variables aléatoires doivent vraiment être réelles ?
\end{probleme}

\begin{corollary}
    Si les variables aléatoires réelles \( X_n\) sont
    \begin{enumerate}
        \item
            indépendantes et identiquement distribuées,
        \item
            dans \( L^2\)
    \end{enumerate}
    alors
    \begin{equation}
        \bar X_n\stackrel{P}{\longrightarrow}E(X_1).
    \end{equation}
\end{corollary}

\begin{proof}
    Nous voulons prouver que pour tout \( \eta>0\),
    \begin{equation}
        P\big( | \bar X_n-E(X_1) |>\eta \big)\to 0.
    \end{equation}
    Remarquons d'abord que les variables aléatoires \( X_n\) étant identiquement distribuées, \( E(\bar X_n)=E(X_1)\) parce que \( E(X_i)=E(X_1)\) pour tout \( i\). L'inégalité de Markov avec \( r=2\) nous donne
    \begin{equation}
        P\big( | \bar X_n-E(\bar X_n) |>\eta \big)\leq\frac{1}{ \eta^2 }E\big( | \bar X_n-E(\bar X_n) |^2 \big)
    \end{equation}
    où nous reconnaissons \( E\big( | \bar X_n-E(\bar X_n) |^2 \big)=\Var(\bar X_n)\). Par la proposition \ref{LemEXYEXEYindep} nous avons \( \Var(\bar X_n)=\Var(X_1)/n\) et par conséquent
    \begin{equation}
        P\big( | \bar X_n-E(\bar X_n) |>\frac{1}{ n\eta^2 }\Var(X_1),
    \end{equation}
    qui tend vers zéro lorsque \( n\to\infty\).
\end{proof}

\begin{proposition}
    Soient \( X_n\) des variables aléatoires indépendantes et identiquement distribuées avec \( X_n\geq 0\). Nous acceptons \( E(X_1)=\infty\), c'est à dire que nous relaxons la condition \( X_n\in L^1\) par rapport à la loi des grands nombres.

    Alors
    \begin{equation}
        \bar X_n\stackrel{p.s.}{\longrightarrow} E(X_1)\in\mathopen[ 0 , \infty \mathclose].
    \end{equation}
\end{proposition}

\begin{proof}
    Si \( E(X_1)<\infty\), nous sommes dans le cas de la loi des grands nombres. Pour chaque \( N\in\eN\) nous considérons la suite de variables aléatoires
    \begin{equation}
        X_n^{(N)}=\min(X_n,N).
    \end{equation}
    Nous avons évidement \( \bar X^{(N)}_n\leq \bar X_n\). Les variables aléatoires \( X^{(N)}_n\) étant bornées par \( N\), elles vérifient la loi des grands nombres pour chaque \( N\) séparément. Par conséquent nous avons pour chaque \( N\) la limite
    \begin{equation}        \label{EqbarXNbtoXnubus}
        \bar X^{(N)}_n\to E(X^{(N)}_1)
    \end{equation}
    Nous supposons que \( E(X_1)=\infty\), par conséquent  \( \lim_{N\to\infty}E(X_1^{(N)})=\infty\). Soit \( \eta>0\) et choisissons \( N\) de telle manière à avoir
    \begin{equation}
        E(X_1^{(N)})>\eta+1.
    \end{equation}
    La limite \eqref{EqbarXNbtoXnubus} nous permet de trouver \( n_0\) tel que pour tout \( n>n_0\) nous ayons \( \bar X^{(N)}_n>\eta\). Au final,
    \begin{equation}
        \eta<\bar X^{(N)}_n\leq \bar X_n,
    \end{equation}
    ce qui montre que \( \bar X_n\to\infty\).
\end{proof}

\begin{example}
    La loi des grands nombres justifie la pratique courante d'approximer une grandeur physique par la moyenne empirique d'un grand nombre de mesures.
\end{example}

%TODO : mettre cette référence vers wikisource en bibliographie.
\begin{example}
    Citons ici le dernier paragraphe de \emph{Le mystère de Marie Roget} par Edgar Allan Poe, traduit par Charles Baudelaire\footnote{Disponible sur \url{https://fr.wikisource.org/wiki/Le_Mystère_de_Marie_Roget}}.

    \begin{quote}
Rien, par exemple, n’est plus difficile que de convaincre le lecteur non spécialiste que, si un joueur de dés a amené les six deux fois coup sur coup, ce fait est une raison suffisante de parier gros que le troisième coup ne ramènera pas les six. Une opinion de ce genre est généralement rejetée tout d’abord par l’intelligence. On ne comprend pas comment les deux coups déjà joués, et qui sont maintenant complètement enfouis dans le Passé, peuvent avoir de l’influence sur le coup qui n’existe que dans le Futur. La chance pour amener les six semble être précisément ce qu’elle était à n’importe quel moment, c’est-à-dire soumise seulement à l’influence de tous les coups divers que peuvent amener les dés. Et c’est là une réflexion qui semble si parfaitement évidente, que tout effort pour la controverser est plus souvent accueilli par un sourire moqueur que par une condescendance attentive. 
    \end{quote}
    Dans le cours de la nouvelle, Edgar Poe cite et utilise la théorie des probabilités avec une justesse inaccoutumée dans la littérature. Mais dans ce paragraphe final, Poe montre de façon la plus formelle qu'il n'a \emph{rien} compris à la loi des grands nombres.
\end{example}

%---------------------------------------------------------------------------------------------------------------------------
\subsection{Théorème central limite}
%---------------------------------------------------------------------------------------------------------------------------

\begin{lemma}       \label{Lemexpznznsurnton}
    Soit \( z_n\to z\) une suite convergente dans \( \eC\). Alors
    \begin{equation}
        \left( 1+\frac{ z_n }{ n } \right)^n\to e^z.
    \end{equation}
\end{lemma}
%TODO : mettre ce résultat au bon endroit et voir si on en a une preuve.

\begin{theorem}     \label{ThoOWodAi}
    Si les variables aléatoires \( X_n\) sont
    \begin{enumerate}
        \item
            indépendantes et identiquement distribuées de loi parente \( X\),
        \item
            \( X_1\in L^2(\Omega,\tribA)\),
    \end{enumerate}
    alors nous notons \( \bar X_n=\frac{1}{ n }\sum_{i=1}^{n}X_i\), \( m=E(X_1)\) et \( \sigma^2=\Var(X_1)\) et nous avons
    \begin{equation}
        \frac{ \bar X_n-E(X) }{ \sqrt{\Var(\bar X_n)} }=\frac{ \bar X_n-m }{ \sigma/\sqrt{n} }\stackrel{\hL}{\longrightarrow}\dN(0,1).
    \end{equation}
\end{theorem}
\index{théorème!central limite}

\begin{proof}
    Nous allons écrire la démonstration dans le cas de variables aléatoires réelles. La proposition \ref{PrpopCaractCvL} dit que la suite \( X_n\) converge en loi vers \( X\) si et seulement si les fonctions caractéristiques convergent ponctuellement. Nous devons donc prouver, pour chaque\footnote{Chuck Norris peut \emph{vraiment} le faire pour \emph{chaque} $t\in\eR$} \( t\in\eR\), que
    \begin{equation}        \label{EqPhitophidNtznu}
        \Phi_{\frac{ S_n-nm }{ \sigma\sqrt{n} }}(t)\to\Phi_{\dN(0,1)}(t).
    \end{equation}
    Supposons dans un premier temps que \( E(X_i)=0\) et \( \sigma(X_i)=1\). Dans ce cas nous considérons la fonction 
    \begin{subequations}
        \begin{align}
            \Phi_{\frac{ S_n }{ \sqrt{n} }}&=E\big(  e^{i\frac{ t }{ \sqrt{n} }\sum_{k=1}^nX_k} \big)\\
            &=\prod_{k=1}^nE\big(  e^{i\frac{ t }{ \sqrt{n} }X_1} \big)\\
            &=\prod_{k=1}^n\Phi_{X_1}\left( \frac{ t }{ \sqrt{n} } \right)\\
            &=\Phi_{X_1}\left( \frac{ t }{ \sqrt{n} } \right)^n.
        \end{align}
    \end{subequations}
    Cette quantité est a priori complexe; nous ne pouvons donc pas immédiatement passer au logarithme. Nous pouvons par contre utiliser un développement en puissances de \( t\) en nous servant de la proposition \ref{PropDerFnCaract} et de l'hypothèse comme quoi \( X_1\in L^2\) :
    \begin{equation}
        \Phi_{X_1}(t)=\Phi_{X_1}(0)+\Phi_{X_1}'(0)t+\Phi_{X_1}''(0)\frac{ t^2 }{2}+\alpha(t)t^2
    \end{equation}
    où \( \alpha\) est une fonction qui a la propriété \( \lim_{x\to 0} \alpha(x)=0\).

    En utilisant les hypothèses et la formule de dérivation de la fonction caractéristique,
    \begin{subequations}
        \begin{align}
            \Phi_{X_1}(0)&=1\\
            \Phi'_{X_1}(0)&=E\big( (iX) \big)=0\\
            \Phi''_{X_1}(0)&=E(-X^2)=-\Var(X_1)=-1.
        \end{align}
    \end{subequations}
    Nous avons donc
    \begin{equation}
        \Phi_{X_1}\left( \frac{ t }{ \sqrt{n} } \right)=\underbrace{1-\frac{ 1 }{2}\frac{ t^2 }{ n }}_{\in\eR}+\underbrace{\frac{ t^2 }{ n }\alpha\left( \frac{ t }{ \sqrt{n} } \right)}_{\in\eC},
    \end{equation}
    de telle sorte que, en considérant une valeur fixée de \( t\),
    \begin{equation}        \label{EqPhifracfacbetanrigh}
        \Phi_{\frac{ S_n }{ \sqrt{n} }}(t)=\left( 1-\frac{ \frac{ t^2 }{ 2 }+\beta_n }{ n } \right)^n
    \end{equation}
    où \( \beta_n=t^2\alpha(t/\sqrt{n})\). Nous avons bien entendu \( \lim_{n\to \infty} \beta_n=0\).
    
    Nous pouvons appliquer le lemme \ref{Lemexpznznsurnton} pour obtenir la limite
    \begin{equation}        \label{EqlimninfySnsqrtntdsnd}
        \lim_{n\to \infty} \Phi_{\frac{ S_n }{ \sqrt{n} }}(t)= e^{-t^2/2}.
    \end{equation}
    La convergence \eqref{EqPhitophidNtznu} est par conséquent prouvée dans le cas où \( E(X_i)=0\) et $\Var(X_i)=1$.

    Considérons maintenant des variables aléatoires avec \( E(X_i)=m\) et \( \Var(X_i)=\sigma^2\). Elles peuvent être écrites sous la forme
    \begin{equation}
        X_i=\sigma X_i+m
    \end{equation}
    où \( X'_i\) est d'espérance nulle et de variance un. Nous avons alors
    \begin{equation}
        S_n=\sigma\sum_{i=1}^nX_i'+nm,
    \end{equation}
    et
    \begin{equation}
        \frac{ S_n-nm }{ \sigma\sqrt{n} }=\frac{ S'_n }{ \sqrt{n} }
    \end{equation}
    où \( S'_n=\sum_iX'_i\). L'étude de la variable aléatoire 
    \begin{equation}
        \frac{ S_n-nm }{ \sigma\sqrt{n} }
    \end{equation}
    revient donc à celle de \( S'_n/\sqrt{n}\) qui vient d'être effectuée.
\end{proof}

\begin{remark}  \label{RemRHFDooGbaPYu}

    Le théorème central limite s'applique quelle que soit la distribution des variables aléatoires \( X_i\) (dans les limites de hypothèses); en particulier il ne dit rien sur la moyenne des \( X_i\). Il dit seulement que l'écart de la moyenne «mesurée»  à la moyenne «théorique» est une variable aléatoire gaussienne si on a mesuré assez de fois.

    Autrement dit, si la durée d'attente à la poste est de \( 5\) minutes, et si j'y vais \( 2000\) fois, alors la probabilité que ma moyenne d'attente soit de \( 4\) minutes est la même que la probabilité qu'elle soit de \( 6\) minutes\footnote{Et en l'occurrence, cette probabilité est nulle parce qu'on est en train de parler de variable aléatoire continue, mais vous voyez l'idée.}.

    Voir la question \ref{ItemIQDVooAcFUuH}.
    
\end{remark}

    \begin{probleme}
        La remarque \ref{RemRHFDooGbaPYu} est une interprétation personnelle. J'aimerais avoir l'avis de quelqu'un de plus compétent.
    \end{probleme}


\begin{remark}
    %TODO : pour ceci, il faudra parler quelque part de la détermination du logarithme dans le plan complexe.
    Nous pouvons obtenir la limite \eqref{EqlimninfySnsqrtntdsnd} d'une façon alternative. Nous considérons la détermination du logarithme complexe sur \( \eC\setminus\eR^-\); cela est une fonction analytique vérifiant l'équation
    \begin{equation}
        e^{\ln(z)}=z
    \end{equation}
    pour tout \( z\in\eC\setminus\eR^-\) et le développement
    \begin{equation}
        \ln(1+z)=\sum_{n=1}^{\infty}(-1)^{n+1}\frac{ z^n }{ n }.
    \end{equation}
    En particulier, \( \ln(1+z)=z+z\alpha(z)\) où \( \lim_{z\to 0}\alpha(z)=0\). Nous reprenons à l'équation \eqref{EqPhifracfacbetanrigh} en fixant \( t\). Nous avons
    \begin{subequations}
        \begin{align}
            \Phi_{\frac{ S_n }{ \sqrt{n} }}(t)&=\exp\left[ \ln\big(\Phi_{\frac{ S_n }{ \sqrt{n} }}(t)\big) \right]\\
            &=\exp\left[ n\ln\left( 1+\frac{ -\frac{ t^2 }{2}-\beta_n }{ n } \right) \right]\\
            &=\exp\left[ -\frac{ t^2 }{2}-\beta_n+ \big( -\frac{ t^2 }{2}-\beta_n \big)\alpha\left( \frac{ -\frac{ t^2 }{2}-\beta_n }{ n } \right) \right].
        \end{align}
    \end{subequations}
    À la limite \( n\to 0\) nous tombons sur \(  e^{-t^2/2}\).
    
\end{remark}

\begin{remark}
    Étant donné que la variable aléatoire 
    \begin{equation}
        \frac{ S_n-nm }{ \sigma\sqrt{n} }
    \end{equation}
    converge en loi vers \( \dN(0,1)\), nous avons la convergence des fonctions de répartition partout où la fonction de répartition de la normale est continue (donc sur tout \( \eR\)). En particulier,
    \begin{equation}
        \left| P\left( \frac{ S_n-nm }{ \sigma\sqrt{n} }\leq x \right)-\int_{-\infty}^x\frac{1}{ \sqrt{2\pi} } e^{-y^2/2}dy \right| \to 0.
    \end{equation}
    Nous avons la borne de \defe{Berry-Esséen}{Berry-Esséen (borne)} qui donne une estimation de la vitesse de convergence: si \( X\in L^3\), alors  il existe une constante \( C\), indépendante de \( x\), des \( X_i\) et de \( n\) telle que
    \begin{equation}
        \left| P\left( \frac{ S_n-nm }{ \sigma\sqrt{n} }\leq x \right)-\int_{-\infty}^x\frac{1}{ \sqrt{2\pi} } e^{-y^2/2}dy \right| \leq\frac{ X\mu_3 }{ \sigma^3\sqrt{n} }
    \end{equation}
    où \( \mu_3=E\big( | X_1-m |^3 \big)\) est le moment d'ordre \( 3\) de \( X\). La chose à retenir est que la convergence est à la vitesse de \( 1/\sqrt{n}\).
\end{remark}

En dimension \( d>1\), nous avons encore un théorème central limite.
\begin{theorem}
    Si \( d>1\), et si nous avons des variables aléatoires \( X_n\) à valeurs dans \( \eR^d\) avec
    \begin{enumerate}
        \item
            les \( X_n\) sont indépendantes et identiquement distribuées
        \item
            les \( X_n\) sont dans \( L^2\).
    \end{enumerate}
    Alors nous notons \( X_1=(X_1^{(1)},\ldots,X_1^{(d)})\). Nous avons
    \begin{equation}
        \frac{ S_n-nm }{ \sqrt{n} }\stackrel{\hL}{\longrightarrow}\dN(0,\Sigma)
    \end{equation}
    où \( \Sigma\) est ma matrice de covariance du vecteur aléatoire \( X_1\) :
    \begin{equation}
        \Sigma=\big( \Cov(X_1^{(1),\ldots,X_1^{(d)}}) \big)_{i,j=1,\ldots,d}.
    \end{equation}
\end{theorem}

%---------------------------------------------------------------------------------------------------------------------------
\subsection{Marche aléatoire}
%---------------------------------------------------------------------------------------------------------------------------

Nous considérons un mobile qui se déplace sur l'axe \( \eZ\). À chaque pas de temps, nous supposons qu'il va faire un pas à gauche avec une probabilité \( p\) et un pas à droite avec une probabilité \( (1-p)\). Nous nous demandons quel est le mouvement du mobile sur le long terme.

La position \( S_n\) du mobile à l'instant \( n\) est donnée par
\begin{equation}
    S_n=\sum_{i=1}^nX_i
\end{equation}
où \( X_i\) est le pas effectué à l'instant \( i\). Ce sont des variables de Bernoulli indépendantes avec
\begin{equation}
    X_i\stackrel{\hL}{=}p\delta_{-1}+(1-p)\delta_1
\end{equation}
c'est à dire
\begin{subequations}
    \begin{align}
        P(X_i=-1)&=p\\
        P(X_i=1)&=1-p.
    \end{align}
\end{subequations}
Ces variables vérifient les hypothèses de la loi des grands nombres :
\begin{enumerate}
    \item
        elles sont indépendantes et identiquement distribuées,
    \item
        elles sont intégrables.
\end{enumerate}
Pour le second point, le calcul est
\begin{equation}
    \int_{\Omega}| X_i |dP=\int_{\eR}| x |dP_X=\int_{\eR}| x |(p\delta_{-1}+(1-p)\delta_{1})=| 1-p |+| p |=1.
\end{equation}
Nous avons par conséquent
\begin{equation}
    \bar X_n=\frac{1}{ n }\sum_{i=1}^nX_i\stackrel{p.s.}{\longrightarrow} E(X_1)=(1-2p)
\end{equation}
et
\begin{equation}
    \frac{ S_n }{ n }\to(1-2p).
\end{equation}
Si \( p\neq 1/2\) nous pouvons conclure que
\begin{enumerate}
    \item
        si \( p>1/2\), alors \( S_n\stackrel{p.s.}{\longrightarrow}-\infty\)
    \item
        si \( p<1/2\), alors \( S_n\stackrel{p.s.}{\longrightarrow}\infty\).
\end{enumerate}
De plus nous connaissons la vitesse de divergence : elle est linéaire. Le mobile suit essentiellement l'équation
\( p(n)=(1-2p)n\).

\begin{remark}
    Cela ne traite pas le cas \( p=1/2\). Dans ce cas, nous pouvons simplement dire que \( S_n=o(n)\).
\end{remark}
