% This is part of Mes notes de mathématique
% Copyright (c) 2011-2016
%   Laurent Claessens
% See the file fdl-1.3.txt for copying conditions.

%+++++++++++++++++++++++++++++++++++++++++++++++++++++++++++++++++++++++++++++++++++++++++++++++++++++++++++++++++++++++++++
\section{Le groupe symétrique}
%+++++++++++++++++++++++++++++++++++++++++++++++++++++++++++++++++++++++++++++++++++++++++++++++++++++++++++++++++++++++++++

\begin{definition}      \label{DEFooJNPIooMuzIXd}
    Le \defe{groupe symétrique}{groupe!symétrique} \( S_n\)\nomenclature[R]{\( S_n\)}{le groupe symétrique} est le groupe des permutations de l'ensemble \( \{ 1,\ldots,n \}\). C'est donc l'ensemble des bijections \( \{ 1,\ldots, n \}\to\{ 1,\ldots, n \}\).

    Plus généralement le groupe symétrique d'un ensemble est le groupe des bijections de cet ensemble sur lui-même.
\end{definition}

Nous disons qu'un élément \( s\in S_n\) \defe{inverse}{inversion!dans le groupe symétrique} les nombres \( i<j\) si \( s(i)>s(j)\). Soit \( N_s\) le nombre d'inversions que \( s\in S_n\) possède (c'est le nombre de couples \( (i,j)\) avec \( i<j\) tels que \( s(i)>s(j)\)). L'entier
\begin{equation}
    \epsilon(s)=(-1)^{N_s}
\end{equation}
est la \defe{signature}{signature!d'une permutation} de \( s\).

Un \wikipedia{fr}{Permutation}{élément du groupe symétrique} \( S_n\) peut être décomposé en produit de cycles de support disjoints de la façon suivante. D'abord écrire le cycle qui correspond à l'orbite de \( 1\). Ce sera le cycle
\begin{equation}
    (1,\sigma 1,\sigma^21,\ldots, \sigma^k1)
\end{equation}
avec \( \sigma^{k+1}1=1\). Ensuite nous recommençons avec le plus petit élément de \( \{ 1,\ldots, n \}\) à ne pas être dans ce cycle, et puis le suivant, etc. La \emph{structure} d'une telle décomposition est la donnée des nombres \( k_i\) donnant le nombre de cycles de longueur \( i\).

\begin{lemma}[\cite{Combes}]        \label{LemmvZFWP}
    Soit \( c=(i_1,\ldots, i_k)\in S_n\), un cycle de longueur \( k\) et \( s\in S_n\). Alors
    \begin{equation}
        csc^{-1}=\big( s(i_1),\ldots, s(i_k) \big).
    \end{equation}
    Tous les cycles de longueur \( k\) sont conjugués entre eux.
\end{lemma}

\begin{proposition}[Classes de conjugaison et structure en cycles\cite{UXMTXxl}] \label{PropEAHWXwe}
    Une classe de conjugaison dans \( S_n\) est formée des permutations ayant une décomposition en cycle disjoints de même structure. Autrement dit, deux permutations \( \sigma\) et \( \sigma'\) sont conjuguées si et seulement si le nombre \( k_i\) de cycles de longueur \( i\) dans \( \sigma\) est le même que le nombre \( k'_i\) de cycles de longueur \( i\) dans \( \sigma'\).
\end{proposition}

\begin{proof}
    Soit \( \sigma=c_1\ldots c_m\) la décomposition de \( \sigma\) en cycles de supports disjoints. Les \( c_i\) sont des cycles de supports disjoints. Si \( \tau\) est une permutation, alors
    \begin{equation}
        \sigma'=\tau\sigma\tau^{-1}=(\tau c_1\tau^{-1})\ldots (\tau c_m\tau^{-1}),
    \end{equation}
    mais \( \tau c_i\tau^{-1}\) est un cycle de même longueur que \( c\) parce que si \( \sigma=(a_1,\ldots, a_k)\), alors \( \tau c\tau^{-1}=\big( \tau(a_1),\ldots, \tau(a_k) \big)\). Notons encore que les cycles \( \tau c_i\tau^{-1}\) restent à support disjoints.

    Donc tous les éléments de la classe de conjugaison de \( \sigma\) sont des permutations de même structure de \( \sigma\).

    Réciproquement, si \( \sigma'=c'_1\ldots c'_m\) est une décomposition de \( \sigma'\) en cycles disjoints tels que la longueur de \( c_i\) est la même que la longueur de \( c'_i\), alors il suffit de prendre des permutations \( \tau_i\) telles que \( \tau_i c_i\tau_i^{-1}=c_i'\). Vu que les supports sont disjoints, la permutation \( \tau_1\ldots \tau_m\) conjugue \( \sigma\) et \( \sigma'\).
\end{proof}

\begin{example}
    Voyons les classes de conjugaison de \( S_3\). Étant donné que ce groupe agit par définition sur un ensemble à \( 3\) éléments, aucun élément de \( S_3\) ne possède un un cycle de plus de \( 3\) éléments. Il y a donc seulement des cycles de longueur deux ou trois (à part les triviaux). Aucun élément de \( S_3\) n'a une décomposition en cycles disjoints contenant deux cycles de deux ou un cycle de deux et un de trois.

    En résumé il y a trois classes de conjugaison dans \( S_3\). La première est celle contenant seulement l'identité. La seconde est celle contenant les cycles de longueur deux et la troisième contient les cycles de longueur \( 3\).

    Ce sont donc
    \begin{subequations}
        \begin{align}
            C_1&=\{ \id \}\\
            C_2&=\{ (1,2),(1,3),(2,3) \}\\
            C_3&=\{ (1,2,3),(2,1,3) \}.
        \end{align}
    \end{subequations}
\end{example}

\begin{example} \label{ExVYZPzub}
    Les classes de conjugaison de \( S_4\). Nous savons que les classes de conjugaison dans \( S_4\) sont caractérisées par la structure des décompositions en cycles (proposition \ref{PropEAHWXwe}). Le groupe symétrique \( S_4\) possède dont les classes de conjugaison suivantes.
\begin{enumerate}
    \item
        Le cycle vide qui représente la classe constituée de l'identité seule.
    \item
        Les permutations de type \( (a,b)\) qui sont au nombre de \( 6\).
    \item
        Les \( 3\)-cycles. Pour savoir \href{http://www.toujourspret.com/techniques/expression/chants/C/cantique_des_etoiles.php}{quel est leur nombre} nous commençons par remarquer qu'il y a \( 4\) façons de prendre \( 3\) nombres parmi \( 4\) et ensuite \( 2\) façons de les arranger. Il y a donc \( 8\) éléments dans cette classe de conjugaison.
    \item
        Les \( 4\)-cycles. Le premier est arbitraire (parce que c'est cyclique). Pour le second il y a \( 3\) possibilités, et deux possibilités pour le troisième; le quatrième est alors automatique. Cette classe de conjugaison contient donc \( 6\) éléments.
    \item
        Les doubles permutations, du type \( (a,b)(c,d)\). Il y a \( 6\) éléments dans cette classe parce qu'un élément est fixé par le choix de deux nombres parmi \( 4\) dans la première des deux permutations.
\end{enumerate}
\index{classe!conjugaison!dans \( S_4\)}
\end{example}

\begin{lemma}[\cite{PDFpersoWanadoo}]       \label{LemhxnkMf}
    Un \( k\)-cycle est une permutation impaire si \( k\) est pair et paire si \( k\) est impair.
\end{lemma}

\begin{proposition} \label{PropPWIJbu}
    Tout élément de \( S_n\) peut être écrit sous la forme d'un produit fini de transpositions.
\end{proposition}
Cette décomposition n'est pas à confondre avec celle en cycles de support disjoints. Par exemple \( (1,2,3)=(1,3)(1,2)\).

\begin{proposition}[\cite{Combes}]  \label{ProphIuJrC}
    Soit \( S_n\) le groupe symétrique.
    \begin{enumerate}
        \item       \label{ITEMooBQKUooFTkvSu}
            L'application \( \epsilon\colon S_n\to \{ 1,-1 \}\) est l'unique homomorphisme surjectif de \( S_n\) sur \( \{ -1,1 \}\).
        \item
            Si \( s=t_1\cdots t_k\) est le produit de \( k\) transpositions, alors \( \epsilon(s)=(-1)^k\).
    \end{enumerate}
\end{proposition}


\begin{proof}
    Soit \( s,t\in S_n\). Afin de montrer que \( \epsilon(st)=\epsilon(s)\epsilon(t)\), nous divisons les couples \( (i,j)\) tels que \( i\neq j\) en \( 4\) groupes suivant que \( t(i)\gtrless t(j)\) et \( s\big( t(i) \big)\gtrless s\big( t(j) \big)\). Nous notons \( N_1\), \( N_2\), \( N_3\) et \( N_4\) le nombre de couples dans chacun des quatre groupes :
    \begin{center}
    \begin{tabular}{c|c|c}
        $ (i,j)$&   \( s\big( t(i) \big)<s\big( t(j) \big)\)    &   \( s\big( t(i) \big)>s\big( t(j) \big)\)\\
        \hline
        \( t(i)<t(j)\)& \( N_1\)&\( N_2\)\\
        \hline
        \( t(i)>t(j)\)&\( N_3\)&\( N_4\)
    \end{tabular}
    \end{center}
    Nous avons immédiatement \( N_t=N_3+N_4\) et \( N_{st}=N_2+N_4\). Les éléments qui participent à \( N_s\) sont ceux où \( t(i)\) et \( t(j)\) sont dans l'ordre inverse de \( s\big( t(i) \big)\) et \( s\big( t(jj) \big)\) (parce que \( t\) est une bijection). Donc \( N_s=N_2+N_3\). Par conséquent nous avons
    \begin{equation}
        \epsilon(s)\epsilon(t)!(-1)^{N_2+N_3}(-1)^{N_3+N_4}=(-1)^{N_2+N_4}=(-1)^{N_{st}}=\epsilon(st).
    \end{equation}
    Nous avons prouvé que \( \epsilon\) est un homomorphisme. Pour montrer que \( \epsilon\) est surjectif sur \( \{ -1,1 \}\) nous devons trouver un élément \( t\in S_n\) tel que \( \epsilon(t)=-1\). Si \( t\) est la transposition \( 1\leftrightarrow 2\) alors le couple \( (1,2)\) est le seul à être inversé par \( t\) et nous avons \( \epsilon(t)=-1\).
    
    Avant de montrer l'unicité, nous montrons que si \( s=t_1\ldots t_k\) alors \( \epsilon(s)=(-1)^k\). Pour cela il faut montrer que \( \epsilon(t)=-1\) dès que \( t\) est une transposition. Soit \( t\), la transposition \( (i,j)\) et \( c=(i,i+1,\ldots, j-1)\) alors le lemme \ref{LemmvZFWP} dit que
    \begin{equation}
        t_{ij}=ct_{j-1,j}c^{-1}.
    \end{equation}
    La signature étant un homomorphisme,
    \begin{equation}
        \epsilon(t_{ij})=\epsilon(c)\epsilon(t_{j-1,j})\epsilon(c)^{-1}=\epsilon(t_{j-1,j})=1.
    \end{equation}
    
    Nous passons maintenant à la partir unicité de la proposition. Soit un homomorphisme surjectif \( \varphi\colon S_n\to \{ -1,1 \}\) et \( t\), une transposition telle que \( \varphi(t)=-1\) (qui existe parce que sinon \( \varphi\) ne serait pas surjectif\footnote{Nous utilisons ici le fait que tous les éléments de \( S_n\) sont des produits de transpositions, proposition \ref{PropPWIJbu}.}). Si \( t'\) est une autre transposition, il existe \( s\in S_n\) tel que \( t'=sts^{-1}\) (lemme \ref{LemmvZFWP}). Dans ce cas, \( \varphi(t')=\varphi(t)=-1\), et si \( s=t_1\ldots t_k\),
    \begin{equation}
         \varphi(s)=(-1)^k=\epsilon(s).
    \end{equation}
\end{proof}

Le groupe \( A_n\)\nomenclature[R]{\( A_n\)}{groupe alterné} des permutations paires est la \defe{groupe alterné}{alterné!groupe}\index{groupe!alterné}.

\begin{proposition}
    Le groupe alterné est un sous-groupe caractéristique de \( S_n\) d'indice \( 2\). C'est le seul sous-groupe d'indice \( 2\) dans \( S_n\).
\end{proposition}

\begin{proof}
    Soit \( \alpha\in \Aut(S_n)\). Étant donné que \( \alpha\circ\alpha\) est un homomorphisme surjectif sur \( \{ -1,1 \}\) nous avons \( \alpha\circ\alpha=\epsilon\), et donc \( \alpha(A_n)=A_n\). Par le premier théorème d'isomorphisme, il existe un isomorphisme
    \begin{equation}
        f\colon S_n/\ker\epsilon\to \Image(\epsilon).
    \end{equation}
    En égalisant le nombre d'éléments nous avons \( | S_n:\ker\epsilon |=| S_n:A_n |=2\).

    Nous prouvons maintenant l'unicité. Soit \( H\) un sous-groupe d'indice \( 2\) dans \( S_n\). Par le lemme \ref{LemSkIOOG}, \( H\) est distingué et nous pouvons considérer le groupe \( S_n/H\). Ce dernier ayant \( 2\) éléments, il est isomorphe à \( \{ -1,1 \}\). Soit \( \theta\) l'isomorphisme. Si nous notons \( \varphi\) le morphisme canonique \( \varphi\colon S_n\to S_n/H\) :
    \begin{equation}    \label{EqSZBPTH}
        \xymatrix{%
        S_n \ar[r]^{\varphi}        &   S_n/H\ar[r]^{\theta}&\{ -1,1 \}.
           }
    \end{equation}
    La composition \( \varphi\circ \theta\) est alors un homomorphisme surjectif de \( S_n\) sur \( \{ -1,1 \}\) et nous avons \( \varphi\circ\theta=\epsilon\) par la proposition \ref{ProphIuJrC}. L'enchaînement \eqref{EqSZBPTH} nous montre que \( H=\ker(\theta\circ\varphi)=\ker(\epsilon)=A_n\).
\end{proof}

\begin{lemma}   \label{LemiApyfp}   \index{groupe!dérivé!du groupe symétrique}
    Le groupe dérivé du groupe symétrique est le groupe alterné : \( D(S_n)=A_n\).
\end{lemma}

\begin{proof}
    Tout élément de \( D(S_n)\) s'écrit sous la forme \( ghg^{-1}h^{-1}\). Quel que soit le nombre de transpositions dans \( g\) et \( h\), le nombre de transpositions dans \( [g,h]\) est pair.
\end{proof}

\begin{proposition}[\cite{LoFdlw}]     \label{PropsHlmvv}
    Soit \( n\geq 3\). Les \( 3\)-cycles \( c_i=(1,2,i)\) avec \( i=3,\ldots, n\) engendrent le groupe alterné \( A_n\).
\end{proposition}

\begin{proof}
    Soit \( H\), le groupe engendré par les \( c_i\). D'abord nous avons 
    \begin{equation}
        c_i=(1,2,i)=(1,2)(2,i),
    \end{equation}
    de telle sorte que \( \epsilon(c_i)=1\). Par conséquent nous avons \( H\subset A_n\). Nos montrons par récurrence que \( A_n\subset H\).

    Pour \( n=3\) il suffit de vérifier que \( A_3=\{ \id,c_3,c_3^2 \}\). Supposons avoir obtenu le résultat pour \(A_{n-1}\), et prouvons le pour \( A_n\). Soit \( s\in A_n\).

    Si \( s(n)=n\), alors \( s\) se décompose de la même manière que sa restriction \( s'\) à \( \{ 1,\ldots, n-1 \}\). Par l'hypothèse de récurrence, cette restriction, appartenant à \( A_{n-1}\),  se décompose en produit des \( c_3,\ldots, c_{n-1}\) et de leurs inverses.

    Si \( s(n)=k\) alors nous considérons l'élément \( c^2_nc_ks\). Cet élément envoie \( n\) sur \( n\) et peut donc être décomposé avec les \( c_i\) (\( i=1,\ldots, n-1\)) en vertu du point précédent.
\end{proof}

\begin{proposition} \label{PropiodtBG}
    Lorsque \( n\geq 5\), tous les \( 3\)-cycles de \( A_n\) sont conjugués. Autrement dit, la classe de conjugaison d'un \( 3\)-cycle est l'ensemble des \( 3\)-cycles.
\end{proposition}

\begin{proof}
    Soient les \( 3\)-cycles \( \sigma=(i_1,i_2,i_3)\) et \( \varphi=(j_1,j_2,j_3)\). Nous considérons une bijection \( \alpha\) de \( \{ 1,\ldots, n \}\) telle que \( \alpha(i_s)=j_s\). Nous avons immédiatement que \( \alpha\in S_n\) et que \( \alpha\sigma\alpha^{-1}=\varphi\). Donc les \( 3\)-cycles sont conjugués dans \( S_n\). Il reste à prouver qu'ils le sont dans \( A_n\).

    Si \( \alpha\) est une permutation paire, la preuve est terminée. Si \( \alpha\) est impaire, alors nous devons un peu la modifier. Vu que \( n\geq 5\), nous pouvons prendre \( s\) et \( t\), des éléments distincts dans \( \{ 1,\ldots, n \}\setminus\{ j_1,j_2,j_3 \}\) et poser \( \tau=(st)\). Vu que la signature est un homomorphisme et que \( \tau\) et \( \alpha\) sont impairs, l'élément \( \tau\alpha\) est pair (lemme et proposition \ref{LemhxnkMf} et \ref{PropPWIJbu}) et est donc dans \( A_n\). Les supports de \( \tau\) et \( \varphi\) étant disjoints, ces derniers commutent et nous avons
    \begin{equation}
        (\tau\alpha)\sigma(\tau\sigma)^{-1}=\tau(\alpha\sigma\sigma^{-1})\tau^{-1}=\varphi.
    \end{equation}
    Donc \( \sigma\) et \( \varphi\) sont conjugués par \( \tau\alpha\) qui est dans \( A_n\).
\end{proof}

\begin{theorem}[\cite{PDFpersoWanadoo}] \label{ThoURfSUXP}
    Le groupe alterné \( A_n\) est simple pour \( n\geq 5\).
\end{theorem}
\index{sous-groupe!distingué!dans le groupe alterné}
\index{groupe!fini!alterné}
\index{groupe!partie génératrice}


\begin{proof}
    Soit \( N\), un sous-groupe normal de \( A_n\) non réduit à l'identité. Étant donné que les \( 3\)-cycles engendrent \( A_n\) (proposition \ref{PropsHlmvv}) et que tous les \( 3\)-cycles sont conjugués dans \( A_n\) (proposition \ref{PropiodtBG}), il suffit de montrer que \( N\) contient un \( 3\)-cycle. En effet si \( N\) contient un \( 3\)-cycle, le fait qu'il soit normal implique (par conjugaison) qu'il les contienne tous et donc qu'il contient une partie génératrice de \( A_n\).

    Soit donc \( \sigma\in N\) différent de l'identité. Nous prenons \( i\) dans le support de \( \sigma\) et \( j=\sigma(i)\). Nous choisissons ensuite \( k\in\{ 1,\ldots, n \}\setminus\{ i,j,\sigma^{-1}(i) \}\) et \( m=\sigma(k)\). Nous considérons la permutation \( \alpha=(ijk)\). Étant donné que \( N\) est normal l'élément
    \begin{equation}
        \theta=\alpha^{-1}\sigma\alpha\sigma^{-1}
    \end{equation}
    est dans \( N\). De plus en utilisant le lemme \ref{LemmvZFWP} et le fait que \( \alpha^{-1}=(ikj)\) nous avons
    \begin{equation}
        \theta=(ijk)(j\sigma(j)m).
    \end{equation}
    Cela n'est pas spécialement un \( 3\)-cycle, mais nous allons en construire un. Nous allons déterminer que \( \theta\) est soit un \( 5\)-cycle, soit un \( 3\)-cycle , soit un \( 2\times 2\)-cycle suivant les valeurs de \( \sigma(j)\) et \( m\). 

    Souvenons nous que \( j\neq\sigma(j)\) parce que \( i\) est dans le support de \( \sigma\); \( m\neq i\) parce que \( k\neq \sigma^{-1}(i)\); \( i\neq m\) parce que \( k\neq \sigma^{-1}(i)\). Les seules possibilités d'égalités sont \( i=\sigma(j)\), \( \sigma(j)=k\) et \( m=k\) (et les combinaisons, mais toutes ne sont pas possibles).
    
    Si \( i\), \( j\), \( k\), \( \sigma(j)\) et \( m\) sont cinq nombres différents, alors 
    \begin{equation}
        \theta=(i,j,\sigma(j),m,k)
    \end{equation}
    est un \( 5\)-cycle.

    Si \( i=\sigma(j)\), alors
    \begin{equation}
        \theta=(imk)
    \end{equation}
    qui est un \( 3\)-cycle. Notons que \( i\), \( m\) et \( k\) sont bien trois éléments différents.

    Si \( \sigma(j)=k\), alors
    \begin{equation}
        \theta=(ikm)
    \end{equation}
    qui est encore un \( 3\)-cycle.

    Si \( m=k\), nous avons
    \begin{equation}
        \theta=(ik)(j\sigma(j)).
    \end{equation}
    C'est a priori un \( 2\times 2\)-cycle. Mais si de plus \( i=\sigma(j)\), alors
    \begin{equation}
        \theta=(ijk)
    \end{equation}
    qui est un \( 3\)-cycle. Et si \( k=\sigma(j)\), alors
    \begin{equation}
        \theta=(ikj)
    \end{equation}
    qui est un autre \( 3\)-cycle.

    Bref nous avons montré que \( \theta\) est soit un \( 3\)-cycle, soit un \( 5\)-cycle, soit un \( 2\times 2\)-cycle. Si \( \theta\) est un \( 3\)-cycle, la preuve est terminée.

    Si \( \theta=(ab)(cd)\), alors on considère \( e\in \{ 1,\ldots, n \}\setminus\{ a,b,c,d \}\) et nous avons
    \begin{equation}
        \underbrace{(abe)^{-1}\theta(abe)}_{\in N}\theta^{-1}=(aeb)(ab)(cd)(abe)(an)(cd)=(abe)\in N.
    \end{equation}
    
    Si \( \theta\) est le \( 5\)-cycle \( (abcde)\), alors l'élément suivant est dans \( N\) :
    \begin{equation}
        (abc)^{-1}\theta(abc)\theta^{-1}=(acb)(abcde)(abc)(aedcb)=(acd).
    \end{equation}
    
    Dans tous les cas nous avons trouvé un \( 3\)-cycle dans \( N\) et nous avons par conséquent \( N=A_n\), ce qui fait que \( A_n\) ne contient pas de sous-groupes normaux non triviaux. Le groupe alterné \( A_n\) est donc simple.
\end{proof}

Nous en déduisons immédiatement que si \( n\geq 5\), le groupe dérivé de \( A_n\) est \( A_n\) parce que \( A_n\) ne contient pas d'autres sous-groupes non triviaux.\index{groupe!dérivé!du groupe alterné}

Le théorème suivant montre que tout groupe peut être vu, en agissant sur lui-même, comme une partie du groupe symétrique.
\begin{theorem}
    Un groupe \( G\) est isomorphe à un sous-groupe de son groupe symétrique \( S(G)\).
\end{theorem}

\begin{proof}
    Nous considérons \( \varphi\), la translation à gauche :
    \begin{equation}
        \begin{aligned}
            \varphi\colon G&\to S(G) \\
            g&\mapsto t_g 
        \end{aligned}
    \end{equation}
    où \( f_g(h)=gh\). Étant donné que
    \begin{equation}
        \varphi(gh)= ghx=g(t_hx)=t_g\circ t_h(x),
    \end{equation}
    l'application \( \varphi\) est un morphisme de groupes. Il est injectif parce que si \( gx=hx\) pour tout \( x\), en particulier pour \( x=e\) nous trouvons \( g=h\). 
    
    De la même manière, \( \varphi(g)x=\varphi(g)y\) implique \( x=y\). Cela montre que l'image est bien dans le groupe symétrique.

    L'ensemble \( \Image(\varphi)\) est donc un sous-groupe de \( S(G)\), et \( \varphi\) est un isomorphisme vers ce groupe.
\end{proof}

%+++++++++++++++++++++++++++++++++++++++++++++++++++++++++++++++++++++++++++++++++++++++++++++++++++++++++++++++++++++++++++
\section{Produit semi-direct de groupes}
%+++++++++++++++++++++++++++++++++++++++++++++++++++++++++++++++++++++++++++++++++++++++++++++++++++++++++++++++++++++++++++

Une \defe{suite exacte}{suite!exacte} est une suite d'applications comme suit :
\begin{equation}
    \xymatrix{%
    \cdots \ar[r]^{f_i}&A_i\ar[r]^{f_{i+1}}& A_{i+1}\ar[r]^{f_{i+2}}&\ldots
       }
\end{equation}
où pour chaque \( i\), les application \( f_i\) et \( f_{i+1}\) vérifient \( \ker(f_{i+1})=\Image(f_i)\). Lorsque les ensembles \( A_i\) sont des groupes, alors nous demandons de plus que les \( f_i\) soient de homomorphismes.

Très souvent nous sommes confrontés à des suites exactes de la forme
\begin{equation}
    \xymatrix{%
    1 \ar[r]& A\ar[r]^f&G\ar[r]^g&B\ar[r]&1
       }
\end{equation}
où \( G\), \( A\) et \( B\) sont des groupes, \( 1\) est l'identité. La première flèche est l'application \( \{ 1 \}\to A\) qui à \( 1\) fait correspondre \( 1\). La dernière est l'application \( B\to 1\) qui à tous les éléments de \( B\) fait correspondre \( 1\). Le noyau de \( f\) étant l'image de la première flèche (c'est à dire \( \{ 1 \}\)), l'application \( f\) est injective. L'image de \( g\) étant le noyau de la dernière flèche (c'est à dire \( B\) en entier), l'application \( g\) est surjective.

\begin{definition}     \label{DEFooKWEHooISNQzi}
    Soient \( N\) et \( H\) deux groupes et un morphisme de groupes \( \phi\colon H\to \Aut(N)\). Le \defe{produit semi-direct}{produit!semi-direct} de \( N\) et \( H\) relativement à \( \phi\), noté \( N\times_{\phi}H\)\nomenclature[R]{\( N\times_{\phi}H\)}{produit semi-direct} est l'ensemble \( N\times H\) muni de la loi (que l'on vérifiera être de groupe)
    \begin{equation}\label{EqDRgbBI}
        (n,h)\cdot (n',h')=(n\phi_h(n'),hh').
    \end{equation}
\end{definition}
Attention à l'ordre quelque peu contre intuitif. Lorsque nous notons \( N\times_{\phi}H\), c'est bien \( \phi\colon H\to \Aut(N)\), c'est à dire \( H\) qui agit sur \( N\) et non le contraire.

Lorsque \( N\) et \( H\) sont des sous-groupes d'un même groupe, le plus souvent \( \phi\) est l'action adjointe définie en \ref{DEFooCORTooEeOLPT}.

Le théorème suivant permet de reconnaître des produits semi-directs lorsqu'on en voit un.
\begin{theorem}[\cite{MathAgreg}]
    Soit une suite exacte de groupes
    \begin{equation}
    \xymatrix{%
    1 \ar[r]        & N\ar[r]^i&G\ar[r]^s&H\ar[r]&1
       }
    \end{equation}
    S'il existe un sous-groupe \( \tilde H\) de \( G\) à partir duquel \( s\) est un isomorphisme, alors
    \begin{equation}
        G\simeq i(N)\times_{\sigma}\tilde H
    \end{equation}
    où \( \sigma\) est l'action adjointe\footnote{Le fait que \( H\) agisse sur \( i(N)\) fait partie du théorème.} de \( \tilde H\) sur \( i(N)\).
\end{theorem}

\begin{proof}
    Nous posons \( \tilde N=i(N)\) et nous allons subdiviser la preuve en petits pas.

    \begin{enumerate}
        \item  \( \tilde N\) est normal dans \( G\). En effet étant donné que la suite est exacte nous avons \( \tilde N=\ker(s)\). Le noyau d'un morphisme est toujours un sous-groupe normal.

        \item \( \tilde N\cap\tilde H=\{ e \}\). L'application \( s\) étant un isomorphismes depuis $\tilde H$, il n'y a pas d'éléments de \( \tilde H\) dans \( \ker(s)\).
    
        \item\label{ItemzIaXGM} \( G=\tilde N\tilde H\). Nous considérons \( g\in G\) et \( h\in \tilde H\) tel que \( s(g)=s(h)\). L'existence d'un tel \( h\) est assurée par le fait que \( s\) est surjective depuis \( \tilde H\). Du coup nous avons \( e=s(gh^{-1})\), c'est à dire \( gh^{-1}\in \ker (s)=\tilde N\). Nous avons donc bien la décomposition \( g=(gh^{-1})h\), et donc \( G=\tilde N\tilde H\).

        \item\label{ItemUGFjle} L'écriture \( g=nh\) avec \( n\in \tilde N\) et \( h\in \tilde H\) est unique. Si \( nh=n'h'\), alors \( n=n'h'h^{-1}\), ce qui signifierait que \( h'h^{-1}\in\tilde N\). Mais étant donné que \( \tilde H\cap\tilde N=\{ e \}\), nous aurions \( h=h'\) et immédiatement après \( n=n'\).

        \item   \label{ItemUZlrKo}
            L'application
            \begin{equation}
                \begin{aligned}
                    \phi\colon G&\to \tilde N\times \tilde H \\
                    nh&\mapsto (n,h) 
                \end{aligned}
            \end{equation}
            est une bijection. C'est une conséquence des points \ref{ItemzIaXGM} et \ref{ItemUGFjle}.

        \item
            Si sur \( \tilde N\times \tilde H\) nous mettons le produit
            \begin{equation}
                (n,h)\cdot(n',h')=(n\sigma_hn',hh')
            \end{equation}
            où \( \sigma\) est l'action adjointe du groupe sur lui-même, c'est à dire \( \sigma_x(y)=xyx^{-1}\), alors \( \phi\) est un isomorphisme. Si \( g,g'\in G\) s'écrivent (de façon unique par le point \ref{ItemUZlrKo}) \( g=nh\) et \( g'=n'h'\) alors
            \begin{subequations}
                \begin{align}
                    \phi(nhn'h')&=\phi(n\underbrace{hn'h^{-1}}_{\in \tilde N}hh')\\
                    &=\phi\big( (nhn'h^{-1})(hh') \big)\\
                    &=(nhn'h^{-1},hh')\\
                    &=(n,h)\cdot(n',h')\\
                    &=\phi(nh)\phi(n'h').
                \end{align}
            \end{subequations}
    \end{enumerate}
\end{proof}

\begin{corollary}\label{CoroGohOZ}
    Soit \( G\) un groupe, et \( N,H\) des sous-groupes de \( G\) tels que
    \begin{enumerate}
        \item
            \( H\) normalise \( N\) (c'est à dire que \( hnh^{-1}\in N\) pour tout \( h\in H\) et \( n\in N\)\footnote{Ou encore que \( H\) agit sur \( N\) par automorphismes internes.}),
        \item
            \( H\cap N=\{ e \}\),
        \item
            \( HN=G\).
    \end{enumerate}
    Alors l'application
    \begin{equation}
        \begin{aligned}
            \psi\colon N\times_{\sigma}H&\to G \\
            (n,h)&\mapsto nh 
        \end{aligned}
    \end{equation}
    est un isomorphisme de groupes.
\end{corollary}
Dans les hypothèses, l'ordre entre \( N\) et \( H\) est important lorsqu'on dit que c'est \( N\) qui agit sur \( H\); mais l'hypothèse \( NH=G\) est équivalente à \( HN=G\) (passer à l'inverse pour s'en assurer).

Insistons encore un peu sur la notation : dans \( N\times_{\sigma}H\), c'est \( H\) qui agit sur \( N\) par \( \sigma\).

%+++++++++++++++++++++++++++++++++++++++++++++++++++++++++++++++++++++++++++++++++++++++++++++++++++++++++++++++++++++++++++ 
\section{Isométriques du cube}
%+++++++++++++++++++++++++++++++++++++++++++++++++++++++++++++++++++++++++++++++++++++++++++++++++++++++++++++++++++++++++++
\label{SecPVCmkxM}
\index{isométrie!espace euclidien!isométries du cube}
\index{groupe!et géométrie!isométries du cube}
Les isométries du cube proviennent de \cite{KXjFWKA}.

\begin{wrapfigure}{r}{5.0cm}
   \vspace{-0.5cm}        % à adapter.
   \centering
   \input{pictures_tex/Fig_MCKyvdk.pstricks}
\end{wrapfigure}
Nous considérons le cube centré en l'origine de \( \eR^3\) et \( G\), le groupe des isométries de \( \eR^3\) préservant ce cube. Nous notons aussi \( G^+\) le sous-groupes de \( G\) constitué des éléments de déterminant positif. Nous notons 
\begin{equation}
    \mD=\{ D_1,\ldots, D_4 \}
\end{equation}
l'ensemble des grandes diagonales, c'est à dire les segments \( [AG]\), \( [EC]\), \( [FD]\), et \( [BH]\). Nous savons que \( G\) préserve les longueurs et que ces segments sont les plus longs possibles à l'intérieur du cube. Donc \( G\) agit sur \( \mD\) parce qu'il ne peut transformer une grande diagonale qu'en une autre grande diagonale. Nous avons donc un morphisme de groupes
\begin{equation}
    \rho\colon G\to S_4.
\end{equation}
Nous montrons ce que morphisme est surjectif en montrant qu'il contient les transpositions. Le groupe \( G\) contient la symétrie axiale passant par le milieu \( M\) de \( [AE]\) et le milieu \( N\) de \( CG\). Il est facile de voir que cette symétrie permute \( [AG]\) avec \( [EC]\). De plus elle laisse \( [FD]\) inchangée. En effet, aussi incroyable que cela paraisse en regardant le dessin, nous avons \( FD\perp MN\), parce qu'en termes de vecteurs directeurs,
\begin{equation}
    \begin{aligned}[]
        \vect{ ON }&=\begin{pmatrix}
            1    \\ 
            -1    \\ 
            0    
        \end{pmatrix}&\vect{ OF }&=\begin{pmatrix}
            1    \\ 
            1    \\ 
            -1    
        \end{pmatrix}.
    \end{aligned}
\end{equation}

Étudions à présent le noyau \( \ker(\rho)\). Si \( f\in\ker(\rho)\) n'est pas l'identité, alors \( f(D_i)=D_i\) pour tout \( i\), mais au moins pour une des diagonales les sommets sont inversés. Quitte à renommer les sommets du cube nous supposons que la diagonale \( [AG]\) est retournée : \( f(A)=G\) et \( f(G)=A\). Regardons où peut partir \( B\) sous l'effet de \( f\). Étant donné que \( f\) préserve les diagonales, \( f(B)\in\{ B,C \}\), mais étant donné que \( f\) est une isométrie, \( d\big( f(B),f(G) \big)=d(B,G)\), et nous concluons que \( f(B)=H\). Donc la diagonale \( [BH]\) est retournée sous l'effet de \( f\). En raisonnant de même, nous voyons que \( f\) retourne toutes les diagonales. Donc les éléments non triviaux de \( \ker(\rho)\) retournent toutes les diagonales; il n'y en a donc qu'un seul et c'est la symétrie centrale :
\begin{equation}
    \ker(\rho)=\{ \id,s_0 \}.
\end{equation}
Le premier théorème d'isomorphisme \ref{ThoPremierthoisomo} nous permet d'écrire le quotient de groupes :
\begin{equation}
    \frac{ G }{ \{ \id,s_0 \} }\simeq S_4.
\end{equation}
Une classe d'équivalence modulo \( \ker(\rho)\) dans \( G\) est donc toujours de la forme \( \{ f,f\circ s_0 \}\). Et vu que \( s_0\) est de déterminant \( -1\), une classe contient toujours exactement un élément de déterminant \( 1\) et un de déterminant \( -1\).

D'autre part \( \ker(\rho)\) est normal dans \( G\) parce que en tant que matrice, \( s_0=-\mtu\), donc les problèmes de commutativité ne se posent pas. L'application
\begin{equation}
    \begin{aligned}
        \varphi\colon \frac{ G }{ \{ \id,s_0 \} }&\to G^+ \\
        [g]&\mapsto \begin{cases}
            g    &   \text{si } \det(g)>0\\
            g\circ s_0    &    \text{sinon}
        \end{cases}
    \end{aligned}
\end{equation}
est un isomorphisme de groupes. Et enfin nous pouvons écrire 
\begin{equation}
    G^+\simeq S_4.
\end{equation}

Nous allons maintenant utiliser le corollaire \ref{CoroGohOZ} pour montrer que \( G=G^+\times_{\sigma}\ker(\rho)\). Les conditions sont remplies :
\begin{itemize}
    \item \( \ker(\rho)\) normalise \( G^+\) parce que \( \ker(\rho)\) ne contient que \( \pm\mtu\).
    \item \( \ker(\rho)\cap G^+=\{ \id \}\).
    \item \( \ker(\rho)G^+=G\) parce que les classes d'équivalence de \( G\) modulo \( \ker(\rho)\) sont composées de \( \{ f,f\circ s_0 \}\).
\end{itemize}
Vu que \( G^+\simeq S_4\) et \( \ker(\rho)\simeq \eZ/2\eZ\) nous pouvons écrire de façon plus brillante que
\begin{equation}
    G\simeq S_4\times_{\sigma}\eZ/2\eZ.
\end{equation}
Mais étant donné que la conjugaison par \( s_0\) est triviale, le produit semi-direct est un produit direct :
\begin{equation}
    G\simeq S_4\times\eZ/2\eZ.
\end{equation}
Il est maintenant du meilleur goût de pouvoir identifier géométriquement ces éléments. Les éléments de \( \eZ/2\eZ=\{ \id,s_0 \}\) ne font pas de mystères. Dans \( S_4\) nous avons les classes de conjugaison des éléments \( \id\), \( (12)\), \( (123)\), \( (1234)\) et \( (12)(34)\) déterminées durant l'exemple \ref{ExVYZPzub}.
\begin{enumerate}
    \item
        L'élément \( (12)\) consiste à permuter deux diagonales et laisser les autres en place. Nous avons déjà vu que c'était une symétrie axiale passant par les milieux de deux côtés opposés. Cela fait \( 6\) axes d'ordre \( 2\).
    \item
        L'élément \( (123)\) fixe une des diagonales. C'est donc la symétrie axiale le long de la diagonale fixée. Par exemple la symétrie d'axe \( (AG)\) fait bouger le point \( B\) de la façon suivante :
        \begin{equation}
            B\to D\to E\to B.
        \end{equation}
        C'est une rotation est d'angle \( \frac{ 2\pi }{ 3 }\). Cela sont \( 8\) rotations d'ordre \( 3\).

        Notons à ce propos que la différence entre \( (234)\) et \( (243)\) est que la première fait une rotation d'angle \( 2\pi/3\) tandis que la seconde fait une rotation d'angle \( -2\pi/3\).

    \item
        L'élément \( (1234)\) ne maintient aucune des diagonales et est d'ordre \( 4\). C'est donc la rotation d'angle \( \pi/2\) ou \( -\pi/2\) autour de l'axe passant par les milieux de deux faces opposées. Il y en a \( 6\) comme ça (\( 3\) paires de faces puis pour chaque il y a \( \pi/2\) et \( -\pi/2\)), et ça tombe bien \( 6\) est justement la taille de la classe de conjugaison de \( (1234)\) dans \( S_4\).

    \item
        L'élément \( (12)(34)\) est le carré de la précédente\footnote{En fait c'est \( (13)(24)\), le carré de la précédente, mais c'est la même classe de conjugaison.}, c'est à dire les rotations d'angle \( \pi\) autour des mêmes axes. Cela fait \( 3\) éléments d'ordre \( 2\).
        
\end{enumerate}

%+++++++++++++++++++++++++++++++++++++++++++++++++++++++++++++++++++++++++++++++++++++++++++++++++++++++++++++++++++++++++++
\section{Groupe de torsion}
%+++++++++++++++++++++++++++++++++++++++++++++++++++++++++++++++++++++++++++++++++++++++++++++++++++++++++++++++++++++++++++

Soit \( G\) un groupe. Un élément \( g\in G\) est un \defe{élément de torsion}{element@élément!de torsion} s'il est d'ordre fini. La \defe{torsion}{torsion!d'un groupe} de \( G\) est l'ensemble de ses éléments de torsion. Nous disons qu'un groupe est un \defe{groupe de torsion}{groupe!de torsion} si tous ses éléments sont de torsion.

\begin{example}
    Le groupe additif \( \eQ/\eZ\) est un groupe de torsion parce que si \( [x]=[p/q]\), alors \( q[x]=[p]=[0]\).
\end{example}

\section{Famille presque nulle}
%+++++++++++++++++++++++++++++++++++++++++++++++++++++++++++++++++++++++++++++++++++++++++++++++++++++++++++++++++++++++++++

Soit \( (G,+)\) un groupe abélien et \( \mF=\{ g_i \}_{i\in I}\) une famille d'éléments de \( G\) indicés par un ensemble \( I\). Le \defe{support}{support!famille d'éléments} de \( \mF\) est l'ensemble \( \{ i\in I\tq g_i\neq 0 \}\). La famille est dite \defe{presque nulle}{presque!nulle} si le support est fini.

Nous disons que \( \mF\) est une \defe{suite}{suite} si \( I=\eN\).
