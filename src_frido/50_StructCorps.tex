% This is part of Mes notes de mathématique
% Copyright (c) 2011-2016
%   Laurent Claessens
% See the file fdl-1.3.txt for copying conditions.

%---------------------------------------------------------------------------------------------------------------------------
\subsection{Corps de décomposition}
%---------------------------------------------------------------------------------------------------------------------------

\begin{definition}
    Soit \( \eK\) un corps commutatif et \( F=(P_i)_{i\in I}\) une famille d'éléments non constants de \( \eK[X]\). Un \defe{corps de décomposition}{corps!de décomposition}\index{décomposition!corps} de \( F\) est une extension \( \eL\) de \( \eK\) telle que
    \begin{enumerate}
        \item
            les \( P_i\) sont scindés sur \( \eL\),
        \item
            \( \eL=\eK(R)\) où \( R=\bigcup_{i\in I}\{ x\in\eL\tq P_i(x)=0 \}\).
    \end{enumerate}
    C'est à dire que \( \eL\) étends \( \eK\) par toutes les racines de tous les polynômes de \( F\).
\end{definition}

L'unicité est due à la proposition suivante.
\begin{proposition}     \label{PropTMkfyM}
    Soit \( \eK\) un corps et \( P\in\eK[X]\). Soient \( \eL\) et \( \eF\) deux corps de décomposition de \( P\). Alors il existe un isomorphisme \( f\colon \eL\to \eF\) tel que \( f|_{\eK}=\id\).
\end{proposition}
Nous pouvons donc parler du corps de décomposition d'un polynôme.

Soit \( \eK\), un corps et \( \eL\), une extension de \( \eK\). Un élément \( a\in \eL\) est \defe{algébrique}{algébrique!nombre} sur \( \eK\) si il existe un polynôme \( P\in \eK[X]\) tel que \( P(a)=0\).

Une \defe{clôture algébrique}{clôture algébrique} du corps \( \eK\) est une extension algébriquement close de \( \eK\) dont tous les éléments sont algébriques sur \( \eK\).

\begin{remark}
    L'ensemble \( \eC\) n'est pas une clôture algébrique de \( \eQ\) parce qu'il existe des éléments de \( \eC\) qui ne sont pas des racines de polynômes à coefficients rationnels.
\end{remark}
L'existence d'une clôture algébrique pour tout corps est le théorème de Steinitz.
%TODO : à faire, le théorème de Steinitz.
% Lorsque ce sera faire, le référentier à la position EYRooJkxiFf

\begin{example}     \label{ExfUqQXQ}
    Soit \( p\) un nombre premier. Montrons que le polynôme 
    \begin{equation}
        Q(X)=X^p-X+1
    \end{equation}
    est irréductible dans \( \eF_p\). 

    Nous supposons qu'il n'est pas irréductible, c'est à dire que
    \begin{equation}
        Q(X)=R(X)S(X)
    \end{equation}
    avec \( R\) et \( S\), des polynômes de degrés \( \geq 1\) dans \( \eF_p[X]\)

    Soit \( \bar\eF_p\) une clôture algébrique de \( \eF_p\) et \( \alpha\in \bar \eF_p\) tel que \( R(\alpha)=0\). Pour tout \( a\in \eF_p\), nous avons
    \begin{subequations}
        \begin{align}
            Q(\alpha+a)&=(\alpha+a)^p-(\alpha+a)+1\\
            &=\alpha^p+a^p-\alpha-a+1\\
            &=\alpha^p-\alpha+1\\
            &=Q(\alpha)\\
            &=0
        \end{align}
    \end{subequations}
    où nous avons utilisé le fait que \( a^p=a\) et que \( \alpha\) était une racine de \( Q\). Ce que nous venons de prouver est que l'ensemble des racines de \( Q\) dans \( \bar\eF_p\) est donné par \( \{ \alpha+a\tq a\in \eF_p \}\).

    Les polynômes \( R\) et \( S\) sont donc formés de produits de termes \( X-(\alpha+a)\) avec \( a\in \eF_p\). L'un des deux --disons \( R\) pour fixer les idées-- doit bien en avoir plus que \( 1\). Nous avons alors
    \begin{equation}
        R(X)=\prod_{i=1}^{k}\big( X-(\alpha+a_i) \big)
    \end{equation}
    où les \( a_i\) sont les éléments de \( \eF_p\). En développant un peu,
    \begin{equation}
        R(X)=X^k-\sum_{i=1}^k(\alpha+a_i^{k-1})+\text{termes de degré plus bas en \( X\)}.
    \end{equation}
    Le coefficient devant \( X^{k-1}\) n'est autre que \( k\alpha+\sum_ia_i\). Étant donné que \( k\neq 0\) et que \( R\in \eF_p[X]\), nous devons avoir \( \alpha\in \eF_p\). Par conséquent nous avons \( \alpha^p=\alpha\) et une contradiction :
    \begin{equation}
        Q(\alpha)=\alpha^p-\alpha+1=1\neq 0.
    \end{equation}

    Le polynôme \( X^p-X+1\) est donc irréductible sur \( \eF_p\).
\end{example}

%---------------------------------------------------------------------------------------------------------------------------
\subsection{Clôture algébrique}
%---------------------------------------------------------------------------------------------------------------------------

\begin{theorem}     \label{THOooQFWWooMWXEhT}
    Tout corps \( \eK\) possède une clôture algébrique \( \Omega\). De plus si \( \eL\) est une extension de \( \eK\), alors \( \eL\) est \( \eK\)-isomorphe à un sous corps de \( \Omega\).
\end{theorem}
Les deux parties de ce théorème utilisent l'axiome du choix.

Notons en particulier que si \( \Omega'\) est une autre clôture algébrique de \( \eK\), alors \( \Omega\) et \( \Omega'\) sont des sous corps l'un de l'autre et sont donc \( \eK\)-isomorphes.

\begin{lemma}
    Les polynômes \( P,Q\in \eK[X]\) ne sont pas premiers entre eux si et seulement si ils ont une racine commune dans la clôture algébrique \( \Omega\) de \( \eK\).
\end{lemma}

\begin{proof}
    Soit \( A\) un polynôme non inversible divisant \( P\) et $Q$. Par définition de \( \Omega\), ce polynôme \( A\) a une racine dans \( \Omega\) qui est alors une racine commune à \( P\) et \( Q\) dans \( \Omega\).

    Pour le sens inverse, si \( \alpha\) est une racine commune de \( P\) et \( Q\), alors le polynôme \( X-\alpha\) divise \( P\) et \( Q\) et donc \( P\) et \( Q \) ne sont pas premiers entre eux.
\end{proof}


%---------------------------------------------------------------------------------------------------------------------------
\subsection{Extensions séparables}
%---------------------------------------------------------------------------------------------------------------------------

Notons que dans ce qui va suivre nous allons parler de \( \eK[X]\), l'ensemble des polynômes sur un corps. Cela ne s'applique donc pas à \( \eZ[X]\) par exemple.

Une des choses intéressantes avec les extensions séparables c'est qu'elles vérifient le théorème de l'élément primitif (\ref{ThoORxgBC}).

\begin{definition}
    Soit \( \eK\) un corps. Un polynôme \emph{irréductible} \( P\in \eK[X]\) est \defe{séparable}{séparable!polynôme irréductible}\index{polynôme!irréductible!séparable} sur $\eK$ si dans un corps de décomposition, ses racines sont distinctes.

    Si \( P\) est un polynôme non constant dont la décomposition en irréductibles est \( P=P_1\ldots P_r\), nous disons qu'il est \defe{séparable}{séparable!polynôme non constant}\index{polynôme!séparable} si tous les \( P_i\) le sont.
\end{definition}

La proposition suivante donne un sens à la définition de polynôme irréductible séparable.
\begin{proposition}
    Soit \( P\) irréductible dans \( \eK[X]\) ayant des racines distinctes dans le corps de décomposition \( \eL\). Si \( \eL'\) est un autre corps de décomposition pour \( P\), alors \( P\) a aussi ses racines distinctes dans \( \eL\).
\end{proposition}

\begin{proof}
    L'ingrédient est la proposition \ref{PropTMkfyM} qui donne l'unicité du corps de décomposition à \( \eK\)-isomorphisme près. Soit donc \( \psi\colon \eL\to \eL'\) un isomorphisme laissant invariant les éléments de \( \eK\). D'une part, étant donné que \( P\) est à coefficients dans \( \eK\), nous avons \( \psi(P)=P\). D'autre part dans \( \eL\) le polynôme \( P\) s'écrit
    \begin{equation}
        P=a(X-\alpha_1)\ldots (X-\alpha_n)
    \end{equation}
    avec \( a\in \eK\) et \( \alpha_i\in \eL\). Nous avons donc
    \begin{equation}
        P=\psi(P)=a(X-\psi(\alpha_1))\ldots (X-\psi(\alpha_n)).
    \end{equation}
    Donc les racines de \( P\) dans \( \eL'\) sont les éléments \( \psi(\alpha_i)\) qui sont distincts.
\end{proof}

\begin{example}
    Un polynôme peut être séparable sur un corps, mais non séparable sur un autre. Soit \( \eL=\eF_p(T)\) et \( \eK=\eF_p(T^p)\). Nous considérons le polynôme
    \begin{equation}
        P=X^p-T^p
    \end{equation}
    dans \( \eK[X]\). Par le morphisme de Frobenius nous avons 
    \begin{equation}
        P=(X-T)^p
    \end{equation}
    dans \( \eL[X]\). Le polynôme \( P\) est irréductible sur \( \eK[X]\) parce que ses diviseurs sont de la forme \( (X-T)^k\) qui contiennent \( T^k\) qui n'est pas dans \( \eK\) (sauf si \( k=n\) ou \( k=0\)).

    Ce polynôme n'est pas séparable sur \( \eK\) parce que dans le corps de décomposition \( \eL\), la racine \( T\) est multiple. Notons bien le raisonnement : \( P\) étant irréductible, pour savoir si il est séparable, on le regarde dans un corps de décomposition.

    Par contre si nous regardons \( P\) dans \( \eL[X]\) alors \( P\) n'est plus irréductible parce que ses facteurs irréductibles sont \( (X-T)\). N'étant pas irréductible, nous regardons les racines de \emph{ses facteurs irréductibles}. Or chacun des facteurs irréductibles étant \( X-T\), les racines sont simples.
\end{example}

\begin{example}
    Le polynôme \( (X-1)^3\) est séparable sur \( \eQ\) parce que ses facteurs irréductibles dans \( \eQ[X]\) sont \( X-1\) qui ont des racines simples.
\end{example}

\begin{example}
    Le polynôme \( (X^2+1)^2\) est séparable dans \( \eQ[X]\). En effet, il a pour facteurs irréductible le polynôme \( X^2+1\) dont les racines sont \( \pm i\) dans l'extension \( \eQ(i)\).
\end{example}

\begin{proposition}[\cite{vgQYwF}]  \label{PropolyeZff}
    Soit \( P\in \eK[X]\) un polynôme non constant. Les propriétés suivantes sont équivalentes.
    \begin{enumerate}
        \item\label{ItemdqPFUi}
            \( P\) a une racine multiple dans une extension de \( \eK\). C'est à dire qu'il existe une extension de \( \eK\) dans laquelle \( P\) a une racine multiple.
        \item\label{ItemdqPFUib}
            \( P\) a une racine multiple dans tout corps de décomposition .
        \item\label{ItemdqPFUii}
            \( P\) et \( P'\) ont une racine commune dans une extension de \( \eK\).
        \item\label{ItemdqPFUiii}
            le degré de \( \pgcd(P,P')\) est \( \geq 1\).
    \end{enumerate}
\end{proposition}
\index{corps!extension}

\begin{proof}
    \begin{subproof}
    \item[\ref{ItemdqPFUi}\( \Rightarrow\)\ref{ItemdqPFUib}] Soit \( a\), une racine multiple de \( P\) dans une extension \( \eL\) de \( \eK\), et \( \eE\), un corps de décomposition de \( P\). Alors nous voulons prouver que \( P\) ait une racine multiple dans \( \eE\).

        Nous pouvons voir \( P\in \eL[X]\), et construire une corps de décomposition \( \eE'\) qui est une extension de \( \eL\). Vu que \( \eE\) et \( \eE'\) sont deux corps de décomposition de \( P\) 
        % iDIUoR
        nous avons un isomorphisme \( \psi\colon \eE\to \eE'\). Si \( a\in \eE\) est une racine multiple de \( P\), alors \( \psi(a)\) est une racine multiple de \( P\) dans \( \eE'\) parce que
        \begin{equation}
            P\big( \psi(a) \big)=\psi\big( P(a) \big).
        \end{equation}
    \item[\ref{ItemdqPFUi}\( \Rightarrow\)\ref{ItemdqPFUii}] Soit \( \eL\) un corps de décomposition de \( P\) sur \( \eK\) et \( a\in \eL\), une racine multiple de \( P\). On a alors \( P=(X-a)^2Q\) avec \( Q\in \eL[X]\). En dérivant,
        \begin{equation}
            P'=2(X-a)Q+(X-a)^2Q',
        \end{equation}
        et donc \( a\) est également une racine de \( P'\).
    \item[\ref{ItemdqPFUii}\( \Rightarrow\)\ref{ItemdqPFUiii}] Soit \( D\) un \( \pgcd\) de \( P\) et \( P'\). D'après le théorème de Bézout il existe \( A,B\in \eK[X]\) tels que 
        \begin{equation}
            AP+BP'=D.
        \end{equation}
        Si \( a\) est une racine commune de \( P\) et \( P'\) dans une extension \( \eL\), alors c'est aussi une racine de \( D\) et donc \( \deg(D)\geq 1\).
    \item[\ref{ItemdqPFUiii}\(\Rightarrow\)\ref{ItemdqPFUi}] Si le degré de \( D\) est plus grand ou égal à \( 1\), alors nous considérons une racine \( a\) de \( D\) dans \( \eL\) (une extension de \( \eK\)). Étant donné que \( D\) divise \( P\) et \( P'\), l'élément \( a\) est une racine commune de \( P\) et \( P'\). Nous montrons maintenant que \( a\) est alors une racine multiple de \( P\). Vu que \( P(a)=0\) nous avons
        \begin{equation}
            P=(X-a)Q,
        \end{equation}
        et \( P'=Q+(X-a)Q'\). Mais alors \( P'(a)=Q(a)\) et donc \( Q(a)=0\) et donc \( a\) est une racine double de \( P\). Par conséquent \( a\) est une racine multiple de \( P\) dans \( \eK\).
    \end{subproof}
\end{proof}
Notons que si \( P\) est irréductible, cette proposition donne des conditions pour que \( P\) ne soit pas séparable.

\begin{proposition}
    Soit \( P\in \eK[X]\) irréductible. Le polynôme \( P\) est séparable si et seulement si \( P'\neq 0\).
\end{proposition}

\begin{proof}
    Soit \( D=\pgcd(P,P')\) et nous voudrions prouver que \( \deg(D)\geq 1\) si et seulement si \( P'=0\). Si \( P'=0\), alors \( \pgcd(P,P')=P\) est donc \( \deg'(D)\geq 1\).

    Dans l'autre sens, si \( P\) est irréductible, il est associé à \( D\) parce qu'il n'a pas d'autres diviseurs que lui-même (à part \( 1\)). Nous avons donc \( P=\lambda D\) avec \( \lambda\in \eK\) et donc \( \deg(P)\geq 1\). Cela prouve immédiatement que \( P'\neq 0\).
\end{proof}

\begin{corollary}   \label{CorUjfJSE}
    Si \( \eK\) est de caractéristique nulle, alors tout polynôme de \( \eK[X]\) est séparable.
\end{corollary}

\begin{proof}
    Il suffit de montrer que les irréductibles sont séparables. Soit \( P\) un polynôme irréductible et unitaire de degré \( d\). Le terme de plus haut degré de \( P'\) est alors \( dX^{d-1}\) qui est non nul parce que \( d\neq 0\) en caractéristique nulle. Donc \( P'\neq 0\) et donc \( P\) est séparable par la proposition \ref{PropolyeZff}.
\end{proof}

\begin{definition}
    Soit \( \eL\) une extension algébrique de \( \eK\).
    \begin{enumerate}
        \item
            On dit que l'élément \( a\in \eL\) est \defe{séparable}{séparable!élément d'une extension} sur \( \eK\) si son polynôme minimal dans \( \eK[X]\) est séparable sur \( \eK\).
        \item
            L'extension \( \eL\) est \defe{séparable}{séparable!extension de corps} si tous ses éléments sont séparables.
    \end{enumerate}
\end{definition}

\begin{proposition} \label{PropUmxJVw}
    Soit \( \eK\) un corps. Les conditions suivantes sont équivalentes :
    \begin{enumerate}
        \item
            toutes les extensions algébriques de \( \eK\) sont séparables;
        \item
            tout polynôme irréductible de \( \eK[X]\) est séparable.
    \end{enumerate}
    En particulier les extensions algébriques des corps de caractéristique nulle sont toutes séparables.
\end{proposition}

\begin{proof}
    Soit \( P\) un polynôme irréductible dans \( \eK[X]\). Soient \( a\) et \( b\) des racines de \( P\) dans un corps de décomposition \( \eL\). Par hypothèse, \( \eL\) est séparables, donc les polynômes minimaux \( \mu_a\) et \( \mu_b\) sont séparables dans \( \eK[X]\). Étant donné que \( P\) est irréductible, il est le seul à se diviser lui-même et donc \( \mu_a=\mu_b=P\). Donc \( P\) est séparable sur \( \eK\).

    Dans l'autre sens, soit \( \eL\) une extension algébrique de \( \eK\), soit \( a\in \eL\) et le polynôme minimal \( \mu_a\in \eK[X]\). Par définition il est irréductible et donc séparable (par hypothèse). Donc \( a\) est séparable et \( \eL\) est une extension séparable.

    La dernière phrase est une conséquence du corollaire \ref{CorUjfJSE}.
\end{proof}

\begin{example} \label{ExvQTyBl}
    Une des conséquences les plus intéressantes de la proposition \ref{PropUmxJVw} est que toutes les extensions algébriques de \( \eQ\) sont séparables.
\end{example}

\begin{theorem}[\cite{rqrNyg}]      \label{ThobkwCMm}
    Soit \( \eK\) un corps (pas spécialement fini). Tout sous-groupe fini de \( \eK^*\) est cyclique.
\end{theorem}

\begin{proof}
    Soit \( G\) un sous-groupe fini de \( \eK^*\) et \( \omega\) son exposant (qui est le PPCM des ordres des éléments de \( G\)). Étant donné que \( | G |\) est divisé par tous les ordres, il est divisé par le PPCM des ordres. Bref, nous avons
    \begin{equation}
        x^{\omega}=1
    \end{equation}
    pour tout \( x\in G\). Mais ce polynôme possède au plus \( \omega\) racines dans \( \eK\). Du coup \( | G |\leq \omega\). Et comme on avait déjà vu que \( \omega\divides | G |\), on a \( \omega=| G |\). Il suffit plus que trouver un élément d'ordre effectivement \( \omega\). Cela est fait par le lemme \ref{LemqAUBYn}.
\end{proof}

\begin{theorem}[Théorème de l'élément primitif\cite{rqrNyg}]   \label{ThoORxgBC}
    Toute extension de corps séparable finie admet un élément primitif.

    Plus explicitement, soient \( \alpha_1,\ldots, \alpha_n\) des éléments algébriques séparables sur \( \eK\); alors \( \eL=\eK(\alpha_1,\ldots, \alpha_n)\) admet un élément primitif.
\end{theorem}
\index{théorème!élément primitif}

\begin{proof}
    Si le corps \( \eK\) est fini, alors \( \eL\) est également fini. Donc \( \eL^*\) est cyclique par le théorème \ref{ThobkwCMm}. Si \( \theta\) est un générateur de \( \eL^*\), alors \( \eL=\eK(\theta)\).

    Passons au cas où \( \eK\) est infini. Il suffit d'examiner le cas \( n=2\); en effet pour \( n=1\) c'est trivial et si \( n>2\), alors
    \begin{equation}
        \eK(\alpha_1,\ldots, \alpha_n)=\eK(\alpha_1,\ldots, \alpha_{n-1})(\alpha_n),
    \end{equation}
    et donc si \( \eK(\alpha_1,\ldots, \alpha_{n-1})=\eK(\theta)\), nous avons
    \begin{equation}
        \eK(\alpha_1,\ldots, \alpha_n)=\eK(\theta,\alpha_n)
    \end{equation}
    et nous sommes réduit au cas \( n=2\) par récurrence. 

    Soit donc \( \eL=\eK(\alpha,\beta)\); soit \( P\) le polynôme minimal de \( \alpha\) sur \( \eK\) et \( Q\) celui de \( \beta\). Nous nommons \( \eE\), un corps de décomposition de \( PQ\). Nous avons \( \eL\subset \eE\). Vu que \( P\) et \( Q\) sont polynômes minimaux d'éléments qui sont par hypothèse séparables, les polynômes \( P\) et \( Q\) sont séparables. Donc dans \( \eE\) les racines de \( P\) sont distinctes parce que \( P\) est irréductible (et idem pour \( Q\)). Soient les racines
    \begin{equation}
        \alpha_1=\alpha,\alpha_2,\ldots, \alpha_r
    \end{equation}
    de \( P\) dans \( \eE\) et les racines
    \begin{equation}
        \beta_1=\beta,\beta_2,\ldots, \beta_s
    \end{equation}
    de \( Q\) dans \( \eE\). Ici \( r\) et \( s\) sont les degrés de \( P\) et \( Q\).

    Si \( s=1\) alors \( Q=X-\beta\) et donc \( \beta\in \eK\) (parce que \( Q\in \eK[X]\)). Du coup nous avons \( \eL=\eK(\alpha)\) et le théorème est démontré. Nous supposons donc maintenant que \( s\geq 2\).

    Pour chaque \( (i,j)\in\llbracket 1,r\rrbracket\times \llbracket 2,s\rrbracket\), l'équation \( \alpha_i+x\beta_k=\alpha_1+x\beta_1\) pour \( x\in \eK\) a au plus\footnote{La solution \eqref{EqWzUFHe} peut être dans \( \eL\) et non dans \( \eK\). L'équation peut donc très bien ne pas avoir de solutions \( x\in \eK\).} une solution donnée le cas échéant par
    \begin{equation}    \label{EqWzUFHe}
        x=(\alpha_i-\alpha_1)(\beta_1-\beta_k)^{-1}
    \end{equation}
    Notons que cela est de toutes façons dans \( \eL\) et qu'étant donné que \( \beta_1\neq \beta_k\), cette solution a un sens (ici on utilise l'hypothèse de séparabilité). Étant donné que \( \eK\) est infini nous pouvons donc trouver un \( c\in \eK\) qui ne résout aucune des équations \eqref{EqWzUFHe} :
    \begin{equation}
        \alpha_i+c\beta_k\neq \alpha_1+c\beta_1.
    \end{equation}
    Nous posons \( \theta=\alpha+c\beta\) et nous prétendons que \( \eL=\eK(\theta)\). Montrons que \( \beta=\eK(\theta)\). Soit dans \( \eK(\theta)[T]\) les polynômes \( Q(T)\) et \( S(T)=P(\theta-cT)\). Nous nommons \( R\) le PGCD de ces deux polynômes.
    
    D'une part, une racine de \( R\) doit être une racine de \( Q\), et donc être un des \( \beta_i\). D'autre part, le choix de \( c\) fait que \( \beta\) est une racine de \( R\) parce que
    \begin{equation}
        S(\beta)=P(\alpha+c\beta-c\beta)=P(\alpha)=0.
    \end{equation}
    Enfin si \( k\geq 2\), alors
    \begin{equation}
        S(\beta_k)=P\big(\alpha_1+c(\beta-\beta_k)\big)\neq 0
    \end{equation}
    parce que \( \alpha_1+c(\beta+\beta_k)\) n'est aucun des \( \alpha_i\). Nous concluons que \( \beta\) est l'unique racine de \( R\) et donc que 
    \begin{equation}
        R=X-\beta\in \eK(\theta)[T],
    \end{equation}
    et donc \( \beta\in \eK(\theta)\).

    De plus \( \alpha=\theta-c\beta\) est alors immédiatement dans \( \eK(\theta)\). À partir du moment où \( \alpha\) et \( \beta\) sont dans \( \eK(\theta)\), nous avons obtenu \( \eL=\eK(\alpha,\beta)=\eK(\theta)\).

\end{proof}

\begin{example}
    Le théorème de l'élément primitif \ref{ThoORxgBC} ne tient pas pour les corps non commutatifs. Par exemple si nous considérons pour \( \eK\) le corps des quaternions\index{quaternion} et comme \( G\) le groupe à \( 8\) éléments \( \{ \pm 1,\pm i,\pm j,\pm k \}\). Ce dernier groupe n'est pas cyclique alors qu'il est un groupe fini dans \( \eK^*\).
\end{example}

\begin{example}
    Il est aussi possible pour un groupe fini d'avoir \( \omega(G)=| G |\) sans pour autant que \( G\) soit cyclique. Par exemple pour \( G=S_3\), nous avons \( | S_3 |=6\) alors que les éléments de \( S_3\) sont soit d'ordre \( 2\) soit d'ordre \( 3\) et \( \omega(G)=\ppcm(2,3)=6\). Pourtant \( S_3\) n'est pas cyclique.
\end{example}


\begin{remark}
    Il est prouvé dans \cite{rqrNyg} que \( \eC\) est algébriquement clos à partir du théorème de l'élément primitif \ref{ThoORxgBC}, mais c'est encore un petit peu de travail.
\end{remark}

%---------------------------------------------------------------------------------------------------------------------------
\subsection{Idéal maximum}
%---------------------------------------------------------------------------------------------------------------------------

\begin{definition}
    Un nombre (dans \( \eC\)) est \defe{transcendant}{transcendant} si il n'est racine d'aucun polynôme non nul à coefficients entiers. Plus généralement si \( \eL\) est un extension du corps \( \eK\) alors si \( t\in \eL\) est une racine d'un polynôme dans \( \eK[X]\) nous disons que \( t\) est \defe{algébrique}{algébrique!par rapport à une extension de corps} sur \( \eK\); sinon nous disons que \( t\) est \defe{transcendant}{transcendant!par rapport à une extension de corps} sur \( \eK\).
\end{definition}

\begin{definition}  \label{DefWHDdTrC}
    Une \( \eK\)-algèbre est de \defe{type fini}{type!fini!en algèbre} si elle est le quotient de \( \eK[X_1,\ldots, X_n]\) par un idéal (pour un certain \( n\)).
\end{definition}

\begin{theorem}[\wikipedia{fr}{Idéal_maximal}{wikipédia}]\index{idéal!maximum}       \label{ThorqTTiJ}
    un idéal \( I\) d'un anneau commutatif \( \eA\) est maximal si et seulement si le quotient \( \eA/I\) est un corps.
\end{theorem}
%TODO : faire la démonstration

\begin{theorem}[\cite{OorXst}]      \label{ThonoZyKa}
    Soit \( \eK\) un corps et \( B\), une \( \eK\)-algèbre de type fini. Si \( B\) est un corps, alors c'est une extension algébrique finie de \( \eK\).
\end{theorem}
%TODO : faire la démonstration

\begin{theorem}[\cite{OorXst}]  \label{ThowgZYqx}
    Si \( \eK\) est un corps algébriquement clos, les idéaux maximaux de \( \eK[X_1,\ldots, X_n]\) sont de la forme
    \begin{equation}
        (X_1-a_1,\ldots, X_n-a_n)
    \end{equation}
    où les \( a_i\) sont des éléments de \( \eK\).
\end{theorem}

\begin{proof}
    Nous commençons par montrer que
    \begin{equation}
        J=(X_1-a_1,\ldots, X_n-a_n)
    \end{equation}
    est un idéal maximum. Pour cela nous considérons le morphisme surjectif d'anneaux
    \begin{equation}
        \begin{aligned}
            \phi\colon \eK[X_1,\ldots, X_n]&\to \eK \\
            P&\mapsto P(a_1,\ldots, a_n). 
        \end{aligned}
    \end{equation}
    Soit \( P\in\ker(\phi)\); nous écrivons la division euclidienne de \( P\) par \( X-a_1\) puis celle du reste par \( X-a_2\) et ainsi de suite :
    \begin{equation}    \label{EqDAkijH}
        P=(X-a_1)Q_1+\cdots +(X_n-a_n)Q_n+R
    \end{equation}
    où \( R\) doit être une constante parce que le premier reste est de degré zéro en \( X_1\), le second est de degré zéro en \( X_1\) et \( X_2\), etc. Afin d'identifier cette constante, nous appliquons l'égalité \eqref{EqDAkijH} à \( (a_1,\ldots, a_n)\) et en nous rappelant que \( P\in \ker(\phi)\) nous obtenons
    \begin{equation}
        0=P(a_1,\ldots, a_n)=R,
    \end{equation}
    donc \( R=0\) et \( P=(X_1-a_1)Q_1+\cdots +(X_n-a_n)Q_n\), c'est à dire \( P\in J\). Nous avons donc \( \ker(\phi)\subset J\). Par ailleurs \( J\subset \ker(\phi)\) est évident, donc \( J=\ker(\phi)\).

    Vu que \( J\) est le noyau de l'application \( \eK[X_1,\ldots, X_n]\to \eK\), nous avons 
    \begin{equation}
        \frac{ \eK[X_1,\ldots, X_n] }{ J }=\eK.
    \end{equation}
    Donc \( J\) est un idéal maximal parce que tout polynôme n'étant pas dans \( J\) doit avoir un terme indépendant non nul et donc être dans \( \eK\) vis à vis du quotient \( \eK[X_1,\ldots, X_n]/J\).

    Nous montrons maintenant l'implication inverse. Nous supposons que \( I\) est un idéal maximum et nous montrons qu'il doit être égal à \( J\) (pour un certain choix de \( a_1,\ldots, a_n\)).

    Le quotient
    \begin{equation}
        \frac{ \eK[X_1,\ldots, X_n] }{ I }
    \end{equation}
    est une \( \eK\)-algèbre de type fini (définition \ref{DefWHDdTrC}). De plus c'est un corps par le théorème \ref{ThorqTTiJ}. C'est donc une extension algébrique finie de \( \eK\) par le théorème \ref{ThonoZyKa}. Mais \( \eK\) étant algébriquement clos, il est sa propre et unique extension algébrique; nous en déduisons que
    \begin{equation}
        \frac{ \eK[X_1,\ldots, X_n] }{ I }=\eK.
    \end{equation}
    Donc pour tout \( 1\leq i\leq n\), il existe \( a_i\in \eK\) tel que \( X_i-a_i\in I\), sinon le monôme \( X_i\) ne se projetterait pas sur un élément dans \( \eK\) dans le quotient. Cela prouve que \( J\) est contenu dans \( I\); par maximalité nous avons donc \( I=J\).
\end{proof}

\begin{corollary}
    Soit \( \eK\) un corps algébriquement clos et \( I\), un idéal de \( \eK[X_1,\ldots, X_n]\). Si nous notons
    \begin{equation}
        V(I)=\{ x\in \eK^n\tq P(x_1,\ldots, x_n)=0 \}
    \end{equation}
    l'ensemble des racines communes à tous les éléments de \( I\), on a \( V(I)=\emptyset\) si et seulement si \( I=\eK[X_1,\ldots, X_n]\).
\end{corollary}

\begin{proof}
    Si \( I=\eK[X_1,\ldots, X_n]\) en particulier \( 1\in I\) et nous avons évidemment \( V(I)=\emptyset\). Le sens difficile est l'autre sens.

    Supposons que \( I\neq \eK[X_1,\ldots, X_n]\) et que \( K\) est un idéal maximum contenu dans \( I\). Nous savons déjà par le théorème \ref{ThowgZYqx} que \( K\) est de la forme \( K=(X_1-a_1,\ldots, X_n-a_n)\). Un élément de \( I\) est dans \( K\), donc si \( P\in I\) nous avons
    \begin{equation}
        P(a_1,\ldots, a_n)=0,
    \end{equation}
    c'est à dire que \( (a_1,\ldots, a_n)\in V(I)\) et donc que \( V(I)\neq 0\).
\end{proof}

%++++++++++++++++++++++++++++++++++++++++++++++++++++++++++++++++++++++++++++++++++++++++++++++++++++++++++++++++++++++++++++++++++++++++
\section{Espaces de polynômes}		\label{SecEspacePolynomes}
%++++++++++++++++++++++++++++++++++++++++++++++++++++++++++++++++++++++++++++++++++++++++++++++++++++++++++++++++++++++++++++++++++++++++
 
Dans cette section nous abandonnons pour quelques minutes l'espace $\eR^m$ et considérons plus attentivement l'espace des fonctions polynômiales $\mathcal{P}_{\eR}$ et de ses sous-espaces $\mathcal{P}_{\eR}^k$, pour $k$ dans $\eN_0$. 

Attention : les polynômes en soi sont définis par la définition \ref{DefRGOooGIVzkx}.

Pour chaque $k>0$ donné nous définissons
\begin{equation}
\mathcal{P}_\eR^k=\{p:\eR\to \eR\,|\, p : x\mapsto a_0+a_1 x +a_2 x^2 + \cdots+a_k x^k, \, a_i\in\eR,\,\forall i=0,\ldots,k\}.
\end{equation}   
Il est facile de se convaincre que la somme de deux polynômes de degré inférieur ou égal à $k$ est encore un polynôme de degré inférieur ou égal à $k$. En outre il est clair que la multiplication par un scalaire ne peut pas augmenter le degré d'un polynôme. L'ensemble $\mathcal{P}_\eR^k$ est donc un espace vectoriel muni des opérations héritées de $\mathcal{P}_{\eR}$. 

La base canonique de l'espace $\mathcal{P}_\eR^k$ est donné par les monômes $\mathcal{B}=\{x\mapsto x^j \,|\, j=0, \ldots, k\}$. Le fait que cela soit une base est vraiment facile à démontrer et est un exercice très utile si vous ne l'avez pas encore vu dans un cours précédent. 

Nous allons maintenant étudier trois application linéaires de $\mathcal{P}_\eR^k$ vers des autres espaces vectoriels
\begin{description}
  \item[L'isomorphisme canonique  $\phi:\mathcal{P}_\eR^k \to\eR^{k+1}$] Nous définissons $\phi$ par les relations suivantes
\[
\phi(x^j)=e_{j+1}, \qquad \forall j\in\{0,\dots, k\}. 
\]
Cela veut dire que pour tout $p$ dans $\mathcal{P}_\eR^k$, avec $p(x)=a_0+a_1 x +a_2 x^2 + \cdots+a_k x^K$, l'image de $p$ par $\phi$ est 
\[
\phi(p)=\phi\left(\sum_{j=0}^k a_j x^j\right)=\sum_{j=0}^k a_j e_{j+1}.
\]
\begin{example} Soit $k=5$ on a 
  \begin{equation}
    \phi(-8-7x-4x^2+4x^3+2x^5)=
  \begin{pmatrix}
    -8\\
    -7\\
    -4\\
    4\\
    0\\
    2
  \end{pmatrix}.
  \end{equation}
\end{example}

Cette application est clairement bijective et respecte les opérations d'espace vectoriel, donc elle est un isomorphisme d'espaces vectoriels. L'existence d'un isomorphisme entre $\mathcal{P}_\eR^k$  et $\eR^{k+1}$ est un cas particulier du théorème qui dit que  pour chaque $m$ dans $\eN_0$ fixée, tous les espaces vectoriels sur $\eR$ de dimension $m$ sont isomorphes à $\eR^m$. Vous connaissez peut être déjà ce théorème depuis votre cours d'algèbre linéaire.  
    \item[La dérivation $d: \mathcal{P}_\eR^k \to \mathcal{P}_\eR^{k-1}$] L'application de dérivation $d$ fait exactement ce qu'on s'attend d'elle 
\[
d(x^0)=d(1)=0, \qquad d(x^j)=j x^{j-1}, \quad \forall j\in\{1,\dots, k\}. 
\]
Cette application n'est pas injective, parce que l'image de $p$ ne dépend pas de la valeur de $a_0$, donc si deux polynômes sont les mêmes à une constante près ils auront la même image par $d$.

\begin{example} Soit $k=3$ on a 
  \begin{equation}
    d(-8-12x+4x^3)= -12 (1) + 4 (3x^2) = -12+12 x^2.
    \end{equation}

    Noter que $d(-30-12x+4x^3)=d(-8-12x+4x^3)$. Cela confirme, comme mentionné plus haut, que la dérivée n'est pas injective.
\end{example}
      \item[L'intégration $I: \mathcal{P}_\eR^k \to \mathcal{P}_\eR^{k+1}$] Nous pouvons définir une application que est <<à une constante prés>> l'application inverse de la dérivation
        \begin{equation}
          I(p)= \int_0^x p(t) \,dt.
        \end{equation}
Il faut comprendre que dans l'intégral la variable $t$ est simplement la variable d'intégration. La <<vraie>> variable de la fonction image de $p$ sera $x$ !
 
Comme d'habitude nous écrivons explicitement l'action de $I$ sur les éléments de la base canonique
\begin{equation}
    I(x^j)=\int_0^x t^k \,dt= \frac{x^{j+1}}{j+1}.
\end{equation}

\begin{example} 
   Soit $k=4$ on a 
  \begin{equation}
    I(6+2x+x^2+x^4)= 6x+x^2+\frac{x^3}{3}+\frac{x^5}{5}.
    \end{equation}
\end{example}

Remarquez que, étant donné que dans la définition de $I$ nous avons décidé d'intégrer entre zéro et $x$, tous les polynômes dans $\mathcal{P}_\eR^{k+1}$ qui sont l'image par $I$ d'un polynôme de $\mathcal{P}_\eR^{k}$ ont $a_0=0$. Cela veut dire que nous pouvons générer toute l'image de $I$ en utilisant un sous-ensemble de la base canonique de $\mathcal{P}_\eR^{k+1}$,  en particulier $\mathcal{B}_1=\{x\mapsto x^j \,|\, j=1, \ldots, k\}\subset \mathcal{B}$ nous suffira. Cela n'est guère surprenant, parce que l'image par une application linéaire d'un espace vectoriel de dimension finie ne peut pas être un espace de dimension supérieure. 
\end{description}

Les applications de dérivation et intégration correspondent évidemment à des application linéaires de $\mathcal{P}_\eR$ dans lui-même. 

L'espace de tous les polynômes étant de dimension infinie, il peut servir de contre exemple assez simple. Dans la sous-section \ref{SubSecPOlynomesCE}, nous verrons que toutes les normes ne sont pas équivalentes sur l'espace des polynômes.



%---------------------------------------------------------------------------------------------------------------------------
\subsection{Polynômes symétriques, alternés ou semi-symétriques}
%---------------------------------------------------------------------------------------------------------------------------
\cite{fJhCTE}.

Soit \( \eK\) un corps de caractéristique différente\footnote{Le truc de la caractéristique deux est que \( a=-a\) n'implique pas \( a=0\).} de \(2\). Le groupe \( S_n\) agit sur l'anneau \( \eK[T_1,\ldots, T_n]\) par
\begin{equation}
    (\sigma\cdot f)(T_1,\ldots, T_n)=f\big( T_{\sigma(1)},\ldots, T_{\sigma(n)} \big).
\end{equation}
On peut vérifier que c'est un action.

\begin{definition}
    Un polynôme \( Q\) en \( n\) indéterminées est 
    \begin{enumerate}
        \item
            \defe{symétrique}{polynôme!symétrique}\index{symétrique!polynôme} si \( Q=\sigma\cdot Q\) pour tout \( \sigma\in S_n\);
        \item
            \defe{alterné}{polynôme!alterné}\index{alterné!polynôme} si \( \sigma\cdot Q=\epsilon(\sigma)Q\) pour tout \( \sigma\in S_n\);
        \item
            \defe{semi-symétrique}{semi-symétrique!polynôme}\index{polynôme!semi-symétrique} si \( \sigma\cdot Q=Q\) pour tout \( \sigma\in A_n\)
    \end{enumerate}
\end{definition}
Le polynôme \( T_1+T_2\) est symétrique; le polynôme \( T_1+T_2^2\) ne l'est pas. 

%---------------------------------------------------------------------------------------------------------------------------
\subsection{Polynôme symétrique élémentaire}
%---------------------------------------------------------------------------------------------------------------------------

\begin{definition}  \label{DEFooTREUooZKoXeg}
    Le \( k\)ième \defe{polynôme symétrique élémentaire}{élémentaire!polynôme symétrique}\index{polynôme!symétrique!élémentaire} à \( n\) inconnues est le polynôme est
    \begin{equation}
        \sigma_k(T_1,\ldots, T_n)=\sum_{s\in F_k}\prod_{i=1}^kT_{s(i)}
    \end{equation}
    où \( F_k\) est l'ensemble des fonctions strictement croissantes \( \{ 1,2,\ldots, k \}\to\{ 1,2,\ldots, n \}\). 
\end{definition}

Une autre façon de décrire ces polynômes élémentaires est
\begin{equation}
    \sigma_k=\sum_{1\leq i_1<\ldots<i_k\leq n}X_{i_1}\ldots X_{i_k}.
\end{equation}
Par exemple
\begin{subequations}
    \begin{align}
        \sigma_1(T_1,\ldots, T_n)&=T_1+T_2+\cdots +T_n\\
        \sigma_2(T_1,\ldots, T_n)&=T_1T_2+\cdots +T_1T_n+T_2T_3+\cdots +T_2T_n+\cdots +T_{n-1}T_n\\
        \sigma_n(T_1,\ldots, T_n)&=T_1\ldots T_n.
    \end{align}
\end{subequations}
En particulier, \( \sigma_2(x,y,z)=xy+yz+xz\).

\begin{theorem}[\cite{PoloPolSym}]  \label{TholReBiw}
    Si \( Q\) est un polynôme symétrique en \( T_1,\ldots, T_n\), alors il existe un et un seul polynôme \( P\) en \( n\) indéterminées tel que
    \begin{equation}
        Q(T_1,\ldots, T_n)=P\big( \sigma_1(T_1,\ldots, T_n),\ldots, \sigma_n(T_1,\ldots, T_n) \big).
    \end{equation}
\end{theorem}
%TODO : la preuve de ce théorème

\begin{example}
    Nous voulons décomposer \( P(x,y,z)=x^3+y^3+z^3\) en polynômes symétriques élémentaires, c'est à dire en
    \begin{subequations}
        \begin{numcases}{}
            \sigma_1=x+y+z\\
            \sigma_2=xy+xz+yz\\
            \sigma_3=xyz.
        \end{numcases}
    \end{subequations}
    Étant donné que \( P\) est de degré \( 3\), les seules combinaisons des \( \sigma_i\) qui peuvent intervenir sont \( \sigma_1^3\), \( \sigma_1\sigma_2\) et \( \sigma_3\). Étant donné que dans \( P\) le coefficient de \( x^3\) est un, il est obligatoire d'avoir un coefficient \( 1\) devant \( \sigma_1^3\). Nous le calculons :
    \begin{verbatim}
----------------------------------------------------------------------
| Sage Version 4.8, Release Date: 2012-01-20                         |
| Type notebook() for the GUI, and license() for information.        |
----------------------------------------------------------------------
sage: var('x,y,z')
(x, y, z)
sage: P=x**3+y**3+z**3  
sage: S1=x+y+z    
sage: S2=x*y+x*z+y*z
sage: S3=x*y*z
sage: (S1**3).expand()
x^3 + 3*x^2*y + 3*x^2*z + 3*x*y^2 + 6*x*y*z + 3*x*z^2 + y^3 + 3*y^2*z + 3*y*z^2 + z^3
sage: (S1**3-P).expand()
3*x^2*y + 3*x^2*z + 3*x*y^2 + 6*x*y*z + 3*x*z^2 + 3*y^2*z + 3*y*z^2
x^3 + 3*x^2*y + 3*x^2*z + 3*x*y^2 + 6*x*y*z + 3*x*z^2 + y^3 + 3*y^2*z + 3*y*z^2 + z^3
    \end{verbatim}
    Dans la différence \( \sigma_1^3-P\) nous voyons que le terme en \( xyz\) est \( 6xyz\); par conséquent nous savons que le coefficient de \( \sigma_3\) sera \( -6\). Il nous reste :
    \begin{verbatim}
sage: (S1**3+6*S3-P).expand()
3*x^2*y + 3*x^2*z + 3*x*y^2 + 12*x*y*z + 3*x*z^2 + 3*y^2*z + 3*y*z^2    
    \end{verbatim}
    que nous identifions facilement avec \( 3\sigma_1\sigma_2\). Nous avons donc
    \begin{equation}
        P=\sigma_1^3-3\sigma_1\sigma_2+3\sigma_3.
    \end{equation}
\end{example}


\begin{lemma}[\cite{fJhCTE}]    \label{LemSoXCQH}
    Soit \( \eK\) une extension de degré \( \delta\) de \( \eQ\) et \( P\in \eK[T_1,\ldots, T_m]\). Alors il existe \( \bar P\in \eQ[T_1,\ldots, T_m]\) tel que
    \begin{enumerate}
        \item
            $\deg\bar P=\delta\deg(P)$
        \item
            pour tout \( (z_1,\ldots, z_m)\in \eC^m\) tel que \( P(z_1,\ldots, z_m)=0\), on a \( \bar P(z_1,\ldots, z_m)=0\).
    \end{enumerate}
\end{lemma}
\index{polynôme!symétrique}
\index{polynôme!racines}
\index{extension!de corps}
\index{corps!extension}

\begin{proof}
    En vertu de la proposition \ref{PropUmxJVw} et de l'exemple \ref{ExvQTyBl}, \( \eK\) est une extension séparable de \( \eQ\), et donc vérifie le théorème de l'élément primitif (\ref{ThoORxgBC}). Il existe \( \theta\in \eK\) tel que \( \eK=\eQ(\theta)\). Soit \( P_{\theta}\in\eQ[X]\) le polynôme minimal de \( \theta\). L'extension \( \eK\) étant de degré \( \delta\), et \( \theta\) étant un générateur, une base de \( \eK\) comme espace vectoriel sur \( \eQ\) est 
    \begin{equation}
        \{ 1,\theta,\ldots, \theta^{\delta-1} \}.
    \end{equation}
    Mais par ailleurs la proposition \ref{PropURZooVtwNXE}\ref{ItemJCMooDgEHajiv} nous indique qu'une base de \( \eQ(\theta)\) sur \( \eQ\) est donnée par
    \begin{equation}
        \{ 1,\theta,\ldots, \theta^{n-1} \}
    \end{equation}
    où \( n\) est le degré de \( P_{\theta}\). Donc \( P_{\theta}\) est de degré \( \delta\). Nous nommons \( \theta_1,\ldots, \theta_{\delta}\) les racines de \( P_{\theta}\) dans un corps de décomposition. Ici nous notons \( \theta=\theta_1\) et nous ne prétendons pas que \( \theta_k\in \eK\). Notons que ces \( \theta_i\) sont toutes des racines simples de \( P_{\theta}\), sinon nous aurions un facteur irréductible \( (X-\theta_k)^2\), et \( P_{\theta}\) ne serait pas irréductible sur \( \eQ\).

    Soit \( \sigma_k\) le morphisme canonique
    \begin{equation}
        \begin{aligned}
            \sigma_k\colon \eQ(\theta)&\to \eQ(\theta_k) \\
            \sum_i q_i\theta^i&\mapsto \sum_iq_i\theta_k^i 
        \end{aligned}
    \end{equation}
    Nous avons \( \sigma_1\colon \eK\to \eK\) qui est l'identité.

    Notons \( N\) le degré du polynôme \( P\in \eK[T_1,\ldots, T_m]\) dont il est question dans l'énoncé. Nous le décomposons alors en
    \begin{equation}
        P=\sum_{l=0}^N\sum_{i=1}^mc_{il}T_i^l
    \end{equation}
    avec \( c_{il}\in \eK\). Nous voyons \( c_{i,.}\) comme un élément de \( \eK^m\) et donc nous écrivons\footnote{Il me semble qu'il manque la somme sur \( i\) dans \cite{fJhCTE}.}
    \begin{equation}
        P=\sum_{l=0}^N\sum_{i=1}^m c_l(\theta)_iT_i^l
    \end{equation}
    où \( c_l\in \eQ[X]^m\). Nous pouvons choisir \( \deg(c_l)<\delta\) parce que les puissances plus grandes de \( \theta\) ne génèrent rien de nouveau.

    Nous posons aussi
    \begin{equation}
        P^{\sigma_k}=\sum_{l,i} c_l(\theta_k)_iT_i^l\in \eQ(\theta_k)[T_1,\ldots, T_m],
    \end{equation}
    et \( \bar P=PP^{\sigma_2}\ldots P^{\sigma_k}\). Le coefficient de \( T_i^l\) dans \( \bar P\) est
    \begin{equation}
        \bar c_l(\theta_1,\ldots, \theta_{\delta})_i=\sum_{l_1+\cdots +l_{\delta}=l}c_{l_1}(\theta_1)_i\ldots c_{l_{\delta}}(\theta_{\delta})_i.
    \end{equation}
    Ce dernier est un polynôme en les \( \theta_k\) à coefficients dans \( \eQ\). Qui plus est, c'est un polynôme symétrique. En effet un terme contenant \( \theta_k^a\theta_l^b\) provenant de \( c_{l_i}(\theta_k)c_{l_j}(\theta_l)\) a un terme correspondant \( \theta_k^b\theta_l^a\) provenant de \( c_{l_j}(\theta_k)c_{l_i}(\theta_l)\).

    C'est donc le moment d'utiliser le théorème \ref{TholReBiw} à propos des polynômes symétriques élémentaires qui nous dit que les coefficients de \( \bar P\) sont en réalité des polynômes en ceux de \( P_{\theta}\) qui sont dans \( \eQ\). Donc \( \bar P\in \eQ[T_1,\ldots, T_m]\). Par ailleurs nous avons que
    \begin{equation}
        \deg(\bar P)=\delta \deg(P)
    \end{equation}
    parce que \( \bar P\) est le produit de \( \delta\) «copies»  de \( P\). De plus \( P=P^{\sigma_1}\) divise \( \bar P \) donc on a bien que si \( P(z)=0\) alors \( \bar P(z)=0\). Le polynôme \( \bar P\) est celui que nous cherchions. 
\end{proof}

%--------------------------------------------------------------------------------------------------------------------------- 
\subsection{Relations coefficients racines}
%---------------------------------------------------------------------------------------------------------------------------

\begin{theorem}[Relations coeffitients-racines] \label{ThoOQRgjpl}
    Soit le polynôme \( P=a_nX^n+\cdots +a_1X+a_0\) et \( r_i\) ses \( n\) racines. Alors nous avons pour chaque \( 1\leq k\leq n\) la relation
    \begin{equation}
        \sigma_k(r_1,\ldots, r_n)=(-1)^k\frac{ a_{n-k} }{ a_n }
    \end{equation}
    où \( \sigma_k\) est le \( k\)\ieme polynôme symétrique définit en \ref{DEFooTREUooZKoXeg}.
\end{theorem}
\index{relations!coefficient-racines}
\index{polynôme!symétrique!élémentaire}

%TODO : citer Wikipédia pour l'exemple suivant.
%TODO : ici aussi il faudra faire référence au théorème sur le fait qu'un polynôme ait toutes ses racines dans \eC.

\begin{example} \label{ExHIfHhBr}
    Soit le polynôme
    \begin{equation}
        P(x)=x^3+2x^2+3x+4
    \end{equation}
    et ses racines que nous nommons \( a,b,c\). Nous voudrions calculer \( a^2+b^2+c^2\). D'abord nous décomposons \( Q(a,b,c)=a^2+b^2+c^2\) en polynômes symétriques élémentaires : \( Q(a,b,c)=\sigma_1(a,b,c)^2-2\sigma_2(a,b,c)\).

    Mais les relations coefficients-racines (théorème \ref{ThoOQRgjpl}) nous donnent \( \sigma_1(a,b,c)=-2\) et \( \sigma_2(a,b,c)=3\), donc
    \begin{equation}
        a^2+b^2+c^2=(-2)^2-2\cdot 3=-2.
    \end{equation}

    Cela nous assure déjà qu'au moins une des solutions n'est pas réelle.

    Nous pouvons en avoir une vérification directe en calculant explicitement les racines (ce qui est possible pour le degré \( 3\)) :
    \lstinputlisting{src_frido/VAYVmNRpolynomeSym.py}

    Notez qu'il faut un peu chipoter pour isoler les solutions depuis la réponse de la fonction \info{solve}.
\end{example}

En suivant le même cheminement que dans l'exemple, si \( P\) est un polynôme de degré \( n\) et si \( r_i\) sont ses racines, il est facile de calculer \( Q(r_1,\ldots, r_n)\) pour n'importe quel polynôme symétrique \( Q\)

\begin{proposition}[Annulation de fonctions polynômiales\cite{WARooZoFOBn}] \label{PropTETooGuBYQf}
    Soit \( \eK\) un corps et \( P\) un polynôme à \( n\) indéterminées. Nous supposons que \(P\) s'annule sur un ensemble de la forme \( A_1\times\cdots\times A_n\) avec \( \Card(A_j)>\deg_{X_j}(P)\) pour tout \( j\). Alors \( P=0\).

    De plus si \( P=0\) alors tous ses coefficients sont nuls\footnote{L'intérêt de cela est qu'un polynôme de \( \eZ[X_1,\ldots, X_n]\) peut s'évaluer sur un élément de n'importe quel corps; il restera le polynôme nul.}.
\end{proposition}

\begin{proof}
    Nous prouvons le résultat par récurrence sur le nombre \( n\) d'indéterminées. Si \( n=1\), cela est le théorème \ref{ThoLXTooNaUAKR}. Nous classons les monômes du polynôme \( P\) par ordre de puissance de \( X_n\) et nous le factorisons :
    \begin{equation}
        P=\sum_{i=1}^mP_iX_n^i
    \end{equation}
    avec \( P_i\in \eK[X_1,\ldots, X_{n-1}]\). Soit \( (a_1,\ldots, a_{n-1})\in A_1\times \ldots \times A_{n-1}\) et posons
    \begin{equation}
        Q(T)=P(a_1,\ldots, a_{n-1},T)= \sum_{i=1}^mP_i(a_1,\ldots, a_{n-1})T^i.
    \end{equation}
    Le polynôme \( Q\) s'annule sur \( A_n\) avec \( \deg(Q)=\deg_{X_n}(P)<\Card(A_n)\) et le théorème \ref{ThoLXTooNaUAKR} nous donne \( Q=0\). Or les coefficients des différentes puissances de \( T\) dans \( Q(T) \) sont les \( P_i(a_1,\ldots, a_{n-1})\); ils sont donc nuls.

    Nous avons montré que le polynôme \( P_i\) s'annule pour tout élément de \( A_1\times \ldots \times A_{n-1}\), mais nous avons
    \begin{equation}
        \deg_{X_j}(P_i)\leq \deg_{X_j}P<\Card(A_j),
    \end{equation}
    donc l'hypothèse de récurrence donne \( P_i=0\). Par suite, \( P=0\) également.
\end{proof}

\begin{example}\label{ExGRHooBNpjSP}
    Un polynôme à plusieurs variables peut s'annuler en une infinité de points sans être nul. Par exemple le polynôme \( X^2+Y^2-1\in\eR[X,Y]\) s'annule sur tout un cercle de \( \eR^2\) mais n'est pas nul, loin s'en faut.
\end{example}

%+++++++++++++++++++++++++++++++++++++++++++++++++++++++++++++++++++++++++++++++++++++++++++++++++++++++++++++++++++++++++++
\section{Polynômes cyclotomiques}
%+++++++++++++++++++++++++++++++++++++++++++++++++++++++++++++++++++++++++++++++++++++++++++++++++++++++++++++++++++++++++++

%---------------------------------------------------------------------------------------------------------------------------
\subsection{Définitions et propriétés}
%---------------------------------------------------------------------------------------------------------------------------

\begin{definition}  \label{DefXGHooRAXlpp}
    Le \defe{polynôme cyclotomique}{polynôme!cyclotomique} d'indice \( n\) est le polynôme
    \begin{equation}    \label{EqLjGYKK}
        \phi_n(X)=\prod_{z\in\Delta_n}(X-z)
    \end{equation}
    où \( \Delta_n\) est l'ensemble des racines primitives de l'unité de la définition \ref{DefLYGTooFPOYGZ} :
    \begin{equation}
        \Delta_n=\{  e^{2ik\pi/n}\tq 0\leq k\leq n-1\tq \pgcd(k,n)=1 \},
    \end{equation}
\end{definition}

Le polynôme \( \phi_n\) est un polynôme unitaire de degré \( \varphi(n)\) où \( \varphi\) est l'indicatrice d'Euler\footnote{Définie par l'équation \ref{EqEulerGqPsvi}.} \( \varphi(n)\). Nous avons par exemple
\begin{subequations}
    \begin{align}
        \Delta_1&=\{ 1 \}\\
        \Delta_2&=\{ -1 \}\\
        \Delta_3&=\{  e^{2\pi i/3}, e^{4\pi i/3} \}
    \end{align}
\end{subequations}
et les premiers polynômes cyclotomiques sont donnés par
\begin{subequations}
    \begin{align}
        \phi_1(X)&=X-1\\
        \phi_2(X)&=X+1\\
        \phi_3(X)&=X^2+X+1.
    \end{align}
\end{subequations}
Pour le dernier nous avons utilisé le fait que \(  e^{6\pi i/3}=1\) et \(  e^{4\pi i/3+ e^{2\pi i/3}}=-1\).

\begin{lemma}   \label{LemKYGBooAwpOHD}
    Le polynôme \( X^n-1\) se factorise des diverses manières suivantes :
    \begin{equation}
        X^n-1=\prod_{z\in \gU_n}(X-z)=\prod_{d\divides n}\prod_{z\in \Delta_d}(X-z)=\prod_{d\divides n}\phi_d(X)
    \end{equation}
\end{lemma}

\begin{proof}
    En ce qui concerne la première égalité, tous les éléments de \( \gU_n\) sont des racines simples de \( X^n-1\). Donc le théorème \ref{ThoSVZooMpNANi} dit qu'il existe un nombre \( k\) (polynôme de degré zéro) tel que \( X^n-1=k\prod_{z\in \gU_n}(X-z)\). Vu le coefficient du terme de plus haut degré, ce \( k\) ne peut être que \( 1\).

    Pour la suite nous utilisons l'union disjointe \( \gU_n=\bigcup_{d\divides n}\Delta_d\) du lemme \ref{LemKcpjee} et la définition \eqref{EqLjGYKK} des polynômes cyclotomiques.
\end{proof}

\begin{remark}
    Notons juste pour le plaisir que dans le produit \( \prod_{d\divides n}\prod_{z\in\Delta_d}\), il y a bien \( n\) termes parce que \( \Card(\Delta_d)=\varphi(d)\) et \( \sum_{d\divides n}\varphi(d)=n\) (définition \ref{DefLYGTooFPOYGZ} et lemme \ref{LemKcpjee}).
\end{remark}

\begin{proposition}
    Les polynômes cyclotomiques sont à coefficients entiers : \( \phi_n\in \eZ[X]\).
\end{proposition}

\begin{proof}
            Nous devons démontrer que les coefficients de \( \phi_n\) sont dans \( \eZ\) alors qu'ils sont a priori dans \( \eC\). Nous démontrons cela par récurrence. D'abord \( \phi_1(X)=X-1\), d'accord. Ensuite
            \begin{equation}
                X^{n+1}-1=\prod_{d\divides n+1}\phi_d(X)=\phi_{n+1}(X)\cdot\underbrace{\prod_{_{\substack{d\divides n+1\\d\leq n}}}\phi_d(X)}_{\in\eZ[X]\text{ par récurrence}}
            \end{equation}
            Le lemme \ref{LemzwkYdn} conclut que \( \phi_{n+1}\in \eZ[X]\). Nous avons vu \( \eZ\) comme sous anneau du corps \( \eC\).
\end{proof}

\begin{proposition}     \label{PropUImYnL}
    Soient \( 1\leq m\leq n\) deux entiers et
    \begin{equation}
        T(X)=\frac{ X^n-1 }{ X^m-1 }\in \eZ(X).
    \end{equation}
    Alors :
    \begin{enumerate}
        \item   \label{ItemhpDPKE}
            si \( m\divides n\) alors \( T\in \eZ[X]\),
        \item
            si \( m\divides n\) et si \( m<n\) alors \( \phi_n\) divise \( T\) dans \( \eZ[X]\).
    \end{enumerate}
\end{proposition}
\index{polynôme!cyclotomique!propriétés}
\index{racine!de l'unité!utilisation}

\begin{proof}
    Nous prouvons point par point.
    \begin{enumerate}
        \item
            Si \( m\) divise \( n\) alors les diviseurs de \( n\) sont l'union des diviseurs de \( m\) et des diviseurs de \( n\) qui ne divisent pas \( m\). Soit
            \begin{equation}
                Q=\{\text{diviseurs de \( n\) ne divisant pas \( m\)} \}.
            \end{equation}
            Nous avons alors
            \begin{equation}
                X^n-1=\prod_{d\divides n}\phi_d(X)=\prod_{d\divides m}\phi_d(X)\cdot\prod_{q\in Q}\phi_q(X)=(X^m-1)\cdot\prod_{q\in Q}\phi_q(X).
            \end{equation}
            Nous avons donc
            \begin{equation}
                T(X)=\frac{ X^n-1 }{ X^m-1 }=\prod_{q\in Q}\phi_q(X)\in \eZ[X].
            \end{equation}
            
        \item

            Nous venons de montrer que
            \begin{equation}
                T=\prod_{q\in Q}\phi_q\in \eZ[X].
            \end{equation}
            Étant donné que \( m<n\) nous avons \( n\in Q\) et donc
            \begin{equation}
                T=\phi_n\cdot\prod_{q\in Q\setminus\{ n \}}\phi_q.
            \end{equation}
            Par conséquent \( \phi_n\) divise \( T\) dans \( \eZ[X]\).
        \end{enumerate}
\end{proof}

\begin{corollary}   \label{CorTVUooErJiAC}
    Si \( p\) est premier alors le polynôme cyclotomique \( \phi_p\) a une bonne tête :
    \begin{equation}
        \phi_p(X)=1+X+\cdots +X^{p-1}.
    \end{equation}
\end{corollary}

\begin{proof}
    Nous utilisons la formule du lemme \ref{LemKYGBooAwpOHD} en remarquant que seuls \( p\) et \( 1\) divisent \( p\) :
    \begin{equation}
        X^p-1=\prod_{d\divides p}\phi_d(X)=\phi_1(X)\phi_p(X)=(X-1)\phi_p(X).
    \end{equation}
    Nous pouvons simplifier par \( X-1\) en utilisant la formule du lemme \ref{LemISPooHIKJBU}\ref{ItemLTBooAcyMtNii} :
    \begin{equation}
        1+X+\cdots +X^{p-1}=\phi_p(X)
    \end{equation}
\end{proof}

\begin{proposition}[Irréductibilité des polynômes cyclotomiques\cite{DVEIity}]      \label{PropoIeOVh}
    Les polynômes cyclotomiques sont irréductibles sur \( \eQ\).
\end{proposition}
\index{polynôme!cyclotomique!irréductibilité}
\index{Anneau!\( \eZ/n\eZ\)!polynôme cyclotomique}
\index{nombre premier!polynôme cyclotomique}
\index{racine!de l'unité}
\index{corps!de rupture!polynôme cyclotomique}

\begin{proof}
    Pour rappel, nous savons déjà que pour tout \( n\in\eN\), \( \phi_n\in \eZ[X]\). Vu que les racines de \( \phi_n\) sont les racines primitives de l'unité, nous devons montrer que toutes les racines primitives de l'unité ont même polynôme minimal (qui sera alors \( \phi_n\)); en effet vu que ces polynômes divisent \( \phi_n\), si ils sont distincts, la proposition \ref{PropyMTEbH} s'applique et le produit des polynômes minimaux diviserait \( \phi_n\). Dans le cas inverse, \( \phi_n\) est polynôme minimal des racines primitives de l'unité et est donc irréductible. Soit donc \( \xi\), une telle racine primitive. Une autre racine primitive est de la forme \( \xi^l\) où \( l\) est un nombre premier tel que \( \pgcd(l,n)=1\).

    Soient \( f\) et \( g\), les polynômes minimaux dans \( \eZ[X]\) de \( \xi\) et \( \xi^l\). Nous allons montrer que \( f=g\) et donc que \( f=g=\phi_n\). Supposons par l'absurde que \( f\neq g\). Dans ce cas ils seraient des facteurs irréductibles distincts de \( \phi_n\) et il existerait un polynôme \( h\) tel que \( \phi_n=fgh\). A priori, \( h\in \eQ[X]\) parce que nous sommes justement en train de prouver que \( \phi_n\) est irréductible dans \( \eQ[X]\). Quoi qu'il en soit, le lemme de Gauss \ref{LemEfdkZw} nous montre que \( h\in \eZ[X]\) parce que \( \phi_n\), \( f\) et \( g\) ont des coefficients entiers. Nous avons
    \begin{equation}
        f(\xi)=g(\xi^l)=0.
    \end{equation}
    Considérons le polynôme \( \psi(X)=g(X^l)\). Ce polynôme \( \psi\) est dans \( \eZ[X]\) et \( \psi\) est annulateur de \( \xi\), donc \( f\) divise \( \psi\) en tant que polynôme minimal de \( \xi\). Il y a un polynôme unitaire à coefficients entiers (lemme de Gauss forever) \( k\) tel que
    \begin{equation}
        \psi=fk
    \end{equation}
    Nous considérons maintenant les projections sur \( \eF_l[X]\) : étant donné que \( \phi_n=fgh\), nous savons que \( \bar f\bar g\) divise \( \bar\phi_n\). En même temps, \( \bar f\) divise \( \bar \psi\). En utilisant le morphisme de Frobenius (c'est ici que la projection sur \( \eF_l\) joue), nous avons aussi
    \begin{equation}
        \bar\psi(X)=\bar g(X^l)=\bar g(X)^l.
    \end{equation}
    Par conséquent dire que \( \bar f\) divise \( \bar\psi\) revient à dire que \( \bar f(X)\) divise \( \bar g(X)^l\). En particulier tous facteur irréductible de \( \bar f\) divise \( \bar g\). Un facteur irréductible de \( \bar f\) serait donc à la fois dans \( \bar f\) et dans \( \bar g\) et donc deux fois (au moins) dans \( \bar\phi_n\) parce que \( \bar f\bar g\) divise \( \phi_n\). Dans un corps de décomposition de ce facteur, \( \phi_n\) aurait une racine double, alors que ce n'est pas le cas. Contradiction. Nous concluons que \( f=g\).
\end{proof}

Le corollaire suivant va être utilisé pour déterminer les polygones constructibles à la règle et au compas, théorème de Gauss-Wantzel \ref{ThoTWAooEsLjJu}.
\begin{corollary}   \label{CorKRTooTJtyvP}
    Soit \( p\) un nombre premier et \( \alpha\) un entier non nul. Nous posons \( q=p^{\alpha}\). Alors le polynôme minimal de \(  e^{2 i\pi/q}\) sur \( \eQ\) est le polynôme cyclotomique \( \phi_q\).
\end{corollary}

\begin{proof}
    Le polynôme \( \phi_q\) est irréductible par la proposition \ref{PropoIeOVh}, il est unitaire par définition et contient le monôme \( X- e^{2i\pi/q}\), donc il est annulateur. Annulateur, irréductible et unitaire, le corollaire \ref{CorKZIooLPUjaf} en fait le polynôme minimal de \( \omega\).
\end{proof}

%TODO : comprendre pourquoi cette démonstration est plus compliquée.
%La preuve suivant est celle donnée par la taverne de l'irlandais et par Arnaud Girand. Je me demande pourquoi elle
%est plus compliquée que celle que je donne au dessus et qui vient de Wikipédia.
%\begin{proposition}[Irréductibilité des polynômes cyclotomiques\cite{KXjFWKA}] 
    %Les polynômes cyclotomiques sont irréductibles sur \( \eQ\).
%\end{proposition}
%
%\begin{proof}
    %L'anneau \( \eQ[X]\) est un anneau factoriel par le théorème \ref{ThoBUEDrJ}; il existe donc un unique \( r\)-uplet de polynômes irréductibles \( G_1,\ldots, G_r\in \eQ[X]\) tel que
    %\begin{equation}
        %\phi_n=\prod_{i=1}^rG_i.
    %\end{equation}
    %Pour chaque \( i\) nous considérons \( \alpha_i\in \eN^*\) tel que \( \alpha_iG_i\in \eZ[X]\) (par exemple \( \alpha_i\) est le $\ppcm$ des dénominateurs des coefficients de \( G_i\)). Nous avons alors
    %\begin{equation}
        %\big(  \prod_{i=1}^r\alpha_i \big)\phi_n=\prod_{i=1}^r\alpha_iG_i.
    %\end{equation}
    %D'autre part le polynôme \( \phi_n\) étant unitaire, son contenu est \( 1\) et donc
    %\begin{equation}
        %\prod_i\alpha_i=c\Big( \big( \prod_i\alpha_i \big)\phi_n \Big)=\prod_ic(\alpha_iG_i)
    %\end{equation}
    %par le lemme \ref{LemHULrVaF}. Nous posons à présent
    %\begin{equation}    \label{EqKNKxqXI}
        %F_i=\pm\frac{ \alpha_iG_i }{ c(\alpha_iG_i) }.
    %\end{equation}
    %Les polynômes \( F_i\) sont dans \( \eZ[X]\) parce que les \( \alpha_iG_i\) y sont. De plus nous avons
    %\begin{equation}
        %\prod_{i=1}^rF_i=\phi_n.
    %\end{equation}
    %Montrons que \( F_i\) est unitaire. Si \( b_i\) est le coefficient dominant de \( F_i\), alors nous avons \( \prod_ib_i=1\). Vu que les \( b_i\) sont dans \( \eZ\), il n'y a pas des tonnes de solutions : \( b_i= \pm 1\). Nous fixons donc le choix de \( \pm\) dans \eqref{EqKNKxqXI} de telle sorte à avoir \( b_i=1\).
%
    %Nous allons prouver maintenant que si \( k\) est premier avec \( n\) et si \( \xi\) est une racine de \( F_1\), alors \( F_1(\xi^k)=0\). Nous allons prouver cela par récurrence sur la longueur de la décomposition de \( k\) en nombres premiers. C'est à dire que nous posons \( k=p_1\ldots p_s\) (les \( p_i\) non spécialement distincts) et que nous faisons la récurrence sur \( s\).
%
    %Pour \( s=1\). Soit \( \xi\) une racine de \( F_1\) et \( p\) un nombre premier tel que \( p\notdivides n\). Vu que \( \xi\) est une racine de \( \phi_n\), c'est une racine \( n\)\ieme\ de l'unité. Vu que nous demandons \( \pgcd(n,p)=1\), le nombre \( \xi^p\) est également une racine primitive de l'unité, et il existe donc \( i\in\{ 1,\ldots, r \}\) tel que \( F_i(\xi^p)=0\). Nous considérons maintenant les polynômes \( F_1(X)\) et \( F_i(X^p)\), et nous montrons qu'ils ne sont pas premiers entre eux sur \( \eQ\). Si ils l'étaient, le sieur Bézout nous donnerait \( U,V\in \eQ[X]\) tels que
    %\begin{equation}
        %U(X)F_1(X)+V(X)F_i(X^p)=1.
    %\end{equation}
    %En évaluant cela en \( X=\xi\), nous obtenons \( 0=1\) qui est une contradiction. Mais étant donné que \( F_1\) est irréductible sur \( \eQ\), il doit être lui-même le facteur commun entre \( F_1\) et \( F_i\). Nous avons donc
    %\begin{equation}
        %F_1(X)\divides F_i(X^p)
    %\end{equation}
    %dans \( \eQ[X]\).
   % 
%\end{proof}
%

\begin{theorem} \label{ThojCJpFW}
    Soit \( P\in \eZ[X]\) un polynôme unitaire irréductible non constant tel que toutes les racines dans \( \eC\) soient de module \( \leq 1\). Alors \( P=X\) ou \( P\) est un polynôme cyclotomique.
\end{theorem}

\begin{proof}
    Nous supposons que \( X\neq 0\), et nous notons \( P=\sum_ia_iX^i\). Étant donné que \( P\) est irréductible et différent de \( X\), nous avons \( a_0\neq 0\) (sinon \( x=0\) serait une racine). Nous allons montrer que les racines de \( P\) sont toutes des racines \( N\)-ièmes de l'unité (avec le même \( N\) pour toutes).

    Soient \( \{ \xi_i \}_{i=1,\ldots, d}\) les racines de \( P\); on a
    \begin{equation}
        P=\prod_{i=1}^d(X-\xi_i)
    \end{equation}
    avec \( \prod_{i=1}^d\xi_i=a_0\). Par hypothèse, \( | \xi_i |\leq 1\) et donc \( 0<| a_0 |\leq 1\). Vu que \( P\in \eZ[X]\) nous avons donc \( a_0=1\) et donc \( | \xi_i |=1\) pour tout \( i\).

    Nous introduisons les polynômes
    \begin{equation}
        g_q(X)=\prod_{i=1}^d\big( X-(\xi_i)^q \big),
    \end{equation}
    et en particulier \( g_1=P\), et nous développons
    \begin{equation}
        g_q(X)=X^n+C_{1,q}X^{n-1}+\cdots +C_{n,q}
    \end{equation}
    où
    \begin{equation}
        C_{k,q}=(-1)^k\sum_{1\leq i_1<\ldots<i_k\leq d}(\xi_{i_1}\ldots \xi_{i_k})^q.
    \end{equation}
    Nous introduisons aussi les polynômes
    \begin{equation}
        F_{k,q}(X_1,\ldots, X_n)=(-1)^k\sum_{1\leq i_1<\ldots< i_k\leq d}(X_{i_1}\ldots X_{i_k})^q
    \end{equation}
    qui sont des polynômes symétriques. Ils vérifient deux propriétés. La première est que
    \begin{equation}
        C_{r,q}=F_{r,q}(\xi_1,\ldots, \xi_n),
    \end{equation}
    et la seconde est que les polynômes \( F_{r,1}\) sont les polynômes symétriques élémentaires à un coefficients près. Le théorème \ref{TholReBiw} nous donne alors des polynômes \( G_{k,q}\in \eZ[X_1,\ldots, X_n]\) tels que
    \begin{equation}
        F_{k,q}(X_1,\ldots, X_n)=G_{k,q}\big( F_{1,1}(X_1,\ldots, X_n),\ldots, F_{k,1}(X_1,\ldots, X_n) \big).
    \end{equation}
    Nous savons que
    \begin{equation}
        | C_{k,q} |\leq \sum_{1\leq i_1<\ldots<i_k<d}1={d\choose k}.
    \end{equation}
    Donc \( g_q\) fait partie de l'ensemble fini des polynômes dans \( \eZ[q]\) dont tous les coefficients sont bornée en valeur absolue par 
    \begin{equation}
        \max_{k=1,\ldots, d}{d\choose k}.
    \end{equation}
    Il existe un certain nombre d'ensembles \( \{ \xi_i \}\) qui sont racines de polynômes vérifiant les conditions du théorème. À chacun de ces ensembles est associé une suite de polynômes \( g_q\) et donc des coefficients \( C_{k,q}\). Ce que nous avons vu est que l'ensemble de tous les coefficients \( C_{k,q}\) possibles (pour un choix donné des \( \{ \xi_i \}\)) est fini, en particulier, vu que \( C_{1,q}=\sum_i\xi_i^q\), pour chaque \( k\), l'ensemble
    \begin{equation}
        \{ \xi_k^q\tq q\in \eN \}.
    \end{equation}
    Par le principe des tiroirs, il existe \( q_1\) et \( q_2\) tels que \( \xi_k^{q_1}=\xi_k^{q_2}\). Ici, \( q_1\) et \( q_2\) dépendent de \( k\) et nous notons \( N_k=q_1-q_2\); nous avons donc \( \xi_k^{N_k}=1\).

    En posant \( N=\ppcm(N_1,\ldots, N_d)\), nous avons
    \begin{equation}
        \xi_k^N=1
    \end{equation}
    pour tout \( k\).

    Mais \( P\) est irréductible dans \( \eZ[X]\); si il a \( \pm 1\) comme racines, alors c'est que \( P=X+1\) ou \( P=X-1\) et ce sont des polynômes cyclotomiques. Si \( P\) n'a pas \( \pm 1\) parmi ses racines, alors \( P\) n'a pas de racines dans \( \eQ\) parce que \( \pm 1\) sont les seules racines de \( X^N-1\) dans \( \eQ\).

    Par conséquent \( P\) est un facteur irréductible de \( X^N-1\) dans \( \eQ[X]\). Mais étant donné que
    \begin{equation}
        X^N-1=\prod_{d\divides N}\phi_d(X),
    \end{equation}
    les polynômes cyclotomiques sont les seuls facteurs irréductibles de \( X^N-1\). Donc \( P\) est un polynôme cyclotomique.
\end{proof}

%---------------------------------------------------------------------------------------------------------------------------
\subsection{Nombres premiers}
%---------------------------------------------------------------------------------------------------------------------------

\begin{lemma}[\cite{naKXuR}]    \label{LemiAqLEn}
    Soit \( n\geq 1\). Il existe un nombre premier \( p\) et un entier \( a\) tels que
    \begin{enumerate}
        \item
            \( p\) divise \( \phi_n(a)\),
        \item
            \( p\) ne divise aucun de \( \phi_d(a)\) avec \( d\divides n\) et \( d\neq n\).
    \end{enumerate}
    De tels \( p\) et \( a\) vérifient automatiquement
    \begin{enumerate}
        \item
            \( p\) divise \( a^n-1\),
        \item
            \( p\) ne divise aucun des \( a^d-1\) pour \( d\divides n\), \( d\neq n\).
    \end{enumerate}
\end{lemma}

\begin{proof}
    Nous posons
    \begin{equation}
        B(X)=\prod_{_{\substack{d\divides n\\d\neq n}}}\phi_d(X),
    \end{equation}
    et nous commençons par montrer que \( \phi_n\) est premier avec \( B\). Nous avons \( X^n-1=B\phi_n\), donc \( B\) et \( \phi_n\) n'ont pas de racines communes (même pas dans \( \eC\)) parce que ce serait une racine double de \( X^n-1\). Notons que par définition \ref{EqLjGYKK}, les polynômes cyclotomiques sont scindés (dans \( \eC\)), donc en particulier les polynômes \( \phi_n\) et \( B\) sont scindés et dons premiers entre eux, dans \( \eC\) et a fortiori dans \( \eQ\). Par Bézout (corollaire \ref{CorimHyXy}), il existe \( U,V\in\eQ[X]\) tels que
    \begin{equation}
        U\phi_n+VB=1.
    \end{equation}
    Si nous prenons \( a\in \eZ\) tel que \( U'=aU\) et \( V'=aV\) soient tous deux dans \( \eZ[X]\), alors nous avons
    \begin{equation}    \label{EqCpNMEi}
        U'\phi_n+V'B=a,
    \end{equation}
    égalité dans \( \eZ[X]\). Quitte à prendre un multiple assez grand de \( a\), nous pouvons choisir \( a\) de telle sorte que \( | \phi_n(a) |\geq 2\). Nous prenons alors un nombre premier \( p\) divisant \( \phi_n(a)\). 

    Montrons que le \( a\) et le \( p\) ainsi construis satisfont aux exigences.

    Vu que \( X^n-1=B\phi_n\), si \( p\) divise \( \phi_n(a)\), il divise automatiquement \( a^n-1\) et donc \( [a^n]_p=1\), ce qui signifie entre autres que \( a\) et \( p\) sont premiers entre eux. Évaluons l'équation \eqref{EqCpNMEi} en~\( a\) :
    \begin{equation}
        U'(a)\phi_n(a)+V'(a)B(a)=a.
    \end{equation}
    Le nombre \( p\) ne divisant pas \( a\), mais divisant \( \phi_n(a)\), il ne peux pas diviser \( B(a)\)\footnote{C'est pour pouvoir dire ça que l'on a choisit \( V'\in \eZ[X]\) de telle sorte que \( V'(a)\) soit dans \( \eZ\)}. Étant donné que \( p\) ne divise pas \( B(a)\), il ne divise aucun des \( \phi_d(a)\) avec \( d\divides n\) et \( d\neq n\).

    Nous passons maintenant à la seconde partie de la preuve. Nous supposons avoir \( a\) et \( p\) tels que \( p\) soit un nombre premier divisant \( \phi_n(a)\) et tels que \( p\) ne divise aucun des \( \phi_d(a)\) avec \( d\divides n\), \( d\neq n\). Le fait de diviser \( \phi_n(a)\) entraine le fait de diviser \( a^n-1\) parce que \( \phi_n\) est un des facteurs de \( X^n-1\). Soit maintenant \( d\neq n\) divisant \( n\); nous avons
    \begin{equation}    \label{EqwTWcCu}
        X^d-1=\prod_{d'\divides d}\phi_{d'},
    \end{equation}
    et cela est une partie du produit
    \begin{equation}
        \prod_{\substack{d\divides n\\d\neq n}}\phi_d.
    \end{equation}
    Vu que \( p\) ne divise aucun des \( \phi_d(a)\) de ce dernier produit, a fortiori, il ne divise pas le produit \ref{EqwTWcCu}, et donc pas \( a^d-1\).
\end{proof}

\begin{lemma}       \label{LemrZnmpG}
    Si \( n\geq 1\), alors il existe un nombre premier dans \( [1]_n\), c'est à dire un nombre premier de la forme \( 1+kn\) avec \( k\in \eN^*\). 
\end{lemma}

\begin{proof}
    Soit \( n\geq 1\) et les nombres \( p,a\) donnés par le lemme \ref{LemiAqLEn}. Vu que \( p\) divise \( \phi_n(a)\), \( p\) divise \( a^n-1\) et donc \( [a]_p\) a un ordre qui divise \( n\) dans \( (\eZ/p\eZ)^*\) parce que \( [a]_p^n=[1]_p\).

    Prenons \( d\neq n\) divisant \( n\). Nous savons que
    \begin{equation}
        a^d-1=\prod_{d'\divides d}\phi_{d'}(a).
    \end{equation}
    
    Par construction de \( a\) et \( p\), nous avons
    \begin{equation}
        [\phi_{d'}(a)]_p\neq 0
    \end{equation}
    Vu que \( \eZ/p\eZ\) est intègre, le produit est également non nul, c'est à dire
    \begin{equation}
        \big[ \prod_{d'\divides d}\phi_{d'}(a) \big]_p\neq 0,
    \end{equation}
    et donc \( [a]_p^a\neq 1\). Nous avons donc montré que si \( d\neq n\) divise \( n\), alors nous avons en même temps
    \begin{equation}
        [a]_p^n=1
    \end{equation}
    et
    \begin{equation}
        [a]_p^d\neq 1.
    \end{equation}
    Cela prouve que \( [a]_p\) est d'ordre exactement \( n\). Oui, mais l'ordre de \( [a]_p\) doit diviser l'ordre du groupe \( \eZ/p\eZ\) qui est \( p-1\), donc \( n\) divise \( p-1\) et nous écrivons \( p=kn+1\) avec \( k\) entier.
\end{proof}

\begin{theorem}[Forme faible du théorème de Dirichlet \cite{fJhCTE}]    \label{ThoxwTjcl}   
    Pour tout \( n\geq 1\), il existe une infinité de nombres premiers dans \( [1]_n\).
\end{theorem}
\index{nombre!premier}
\index{Dirichlet!théorème (sur les nombres premiers)}
\index{théorème!Dirichlet!forme faible}
\index{anneau!\( \eZ/n\eZ\)}
\index{racine!de l'unité}

\begin{proof}
    Le lemme \ref{LemrZnmpG} nous donne déjà l'existence de nombres premiers dans \( [1]_n\). Il faut maintenant voir qu'il y en a une infinité. Nous supposons qu'il y en ait seulement un nombre fini : \( p_1,\ldots, p_r\), et nous notons 
    \begin{equation}
        N=np_1\ldots p_r.
    \end{equation}
    Nous utilisons maintenant le lemme \ref{LemrZnmpG} avec ce \( N\), c'est à dire qu'on a un nombre premier de la forme
    \begin{equation}
        p=1+kN=1+knp_1\ldots p_r.
    \end{equation}
    Cela est un nombre premier plus grand que tous les \( p_i\) et de la forme \( 1+\lambda n\). Cela contredit l'exhaustivité de la liste \( p_1,\ldots, p_r\).
\end{proof}

%---------------------------------------------------------------------------------------------------------------------------
\subsection{Théorème de Wedderburn}
%---------------------------------------------------------------------------------------------------------------------------

\begin{theorem}[Théorème de Wedderburn\cite{SQxrsoL}]    \label{ThoMncIWA}
    Tout corps fini est commutatif.
\end{theorem}
\index{groupe!fini}
\index{théorème!Wedderburn}
\index{action!de groupe!Wedderburn}
\index{nombre!complexe!norme \( 1\)}
\index{groupe!fini!Wedderburn}
\index{corps!fini!Wedderburn}

\begin{proof}
    Soit \( \eK\) un corps fini et \( Z\), le centre de \( \eK\). Ce dernier est un corps fini et un sous corps de \( \eK\). Si \( q=\Card(Z)\) alors par le lemme \ref{LemobATFP} nous avons
    \begin{equation}
        \Card(\eK)=q^n
    \end{equation}
    pour un certain \( n\).

    Nous supposons maintenant que \( \eK\) est non commutatif. Dans ce cas \( Z\neq \eK\) et nous avons \( n\geq 2\). Nous considérons aussi
    \begin{equation}
        Z_x=\{ a\in \eK\tq ax=xa \}.
    \end{equation}
    Le centre \( Z\) est un sous corps de \( Z_x\), donc il existe \( d(x)\) tel  que
    \begin{equation}
        \Card(Z_x)=q^{d(x)}.
    \end{equation}
    De la même manière, \( Z_x\) est un sous corps de \( \eK\), donc il existe \( m(x)\) tel que
    \begin{equation}
        \Card(\eK)=\Card(Z_x)^{m(x)}.
    \end{equation}
    En mettant bout à bout nous avons
    \begin{equation}
        q^n=\Card(Z_x)^{m(x)}=q^{d(x)m(x)},
    \end{equation}
    et par conséquent \( n=d(x)m(x)\). Le point important à retenir est que \( d(x)\) divise \( n\) pour tout \( x\in \eK\).

    Nous considérons maintenant l'action adjointe du groupe \( \eK^*\) sur lui-même :
    \begin{equation}
        \varphi(k)x=kxk^{-1}.
    \end{equation}
    Nous notons \( \mO_x\) l'orbite de \( x\in \eK^*\) pour cette action, et \( \Stab(x)\) son stabilisateur. Nous avons
    \begin{equation}
        Z_y=\Stab(y)\cup\{ 0 \}
    \end{equation}
    parce que \( Z_y\) et \( \Stab(y)\) ont les mêmes définitions, sauf que \( \Stab(y)\) est dans \( \eK^*\) alors que \( Z_y\) est dans \( \eK\). Nous avons donc
    \begin{equation}
        \Card\big( \Stab(y) \big)=\Card(Z_y)-1=q^{d(y)}-1.
    \end{equation}
    Nous avons \( \Card(\mO_x)=1\) si et seulement si \( \mO_x=\{ x \}\) si et seulement si \( \Stab(x)=\eK^*\) si et seulement si \( z\in Z^*\). Soient \( z_0,\ldots, z_{q-1}\) les éléments de \( Z\) avec \( z_0=0\). Ce sont les éléments qui auront une orbite réduite à un point. Les orbites qui coupent \( Z^*\) sont
    \begin{equation}
        \{ z_1 \},\ldots, \{ z_{q-1} \}
    \end{equation}
    et il y en a \( q-1\). Soient \( \mO_{y_1},\ldots, \mO_{y_r}\), les autres orbites. Nous utilisons l'équation des classes \eqref{EqkgGmoq} :
    \begin{equation}
        \Card(\eK^*)=\Card(Z^*)+\sum_{i=1}^{r}\frac{ \Card(\eK^*) }{ \Card(\Stab(y_i)) },
    \end{equation}
    mais \( \Card(Z^*)=q-1\), \( \Card(\eK^*)=q^n-1\) et \( \Card\big( \Stab(y_i) \big)=q^{d(y_i)}-1\), donc
    \begin{equation}        \label{EqBPBDzE}
        q^n-1=(q-1)+\sum_{i=1}^{r}\frac{ q^n-1 }{ q^{d(y_i)}-1 }.
    \end{equation}
    Nous considérons la fraction rationnelle
    \begin{equation}        \label{EqATGciu}
        F(X)=(X^n-1)-\sum_{i=1}^{r}\frac{ X^n-1 }{ X^{d(y_i)}-1 }.
    \end{equation}
    Étant donné que \( d(y_i)\) divise \( n\), nous avons, contrairement aux apparences, que \( F\in \eZ[X]\) par la proposition \ref{PropUImYnL}\ref{ItemhpDPKE}.

    Nous pouvons exploiter un peu mieux la proposition \ref{PropUImYnL} en remarquant que \( d(y_i)<n\) parce que sinon \( \Card(Z_{y_i})=\Card(\eK)\), ce qui signifierait que \( y_i\in Z\), ce qui nous avions exclu. Par conséquent le polynôme cyclotomique \( \phi_n\) divise 
    \begin{equation}
        \frac{ X^n-1 }{ X^{d(y_i)}-1 }
    \end{equation}
    dans \( \eZ[X]\). Le polynôme cyclotomique \( \phi_n\) divise également \( X^n-1\) et par conséquent \( \phi_n\) divise \( F\). Il existe donc \( Q\in \eZ[X]\) tel que \( F=Q\phi_n\). En particulier en évaluant en \( q\) :
    \begin{equation}    \label{eqmoLdJy}
        F(q)=Q(q)\phi_n(q)=q-1.
    \end{equation}
    En effet nous avons \( F(q)=q-1\) par construction : comparer \eqref{EqBPBDzE} avec \eqref{EqATGciu}. Évidemment \( q\neq 1\) parce que si \( q=1\) alors \( \Card(\eK)=1\) et le théorème est trivial. Par ailleurs \( Q(q)\) est un entier (parce que \( Q\in \eZ[X]\) et \( q\in \eN\)) et \( Q(q)\neq 0\), parce qu'à droite de \eqref{eqmoLdJy} nous avons \( q-1\neq 0\). Nous avons donc \( | Q(q) |\geq 1\) et donc
    \begin{equation}
        | \phi_n(q) |\leq q-1.
    \end{equation}
    Par définition du polynôme cyclotomique nous avons
    \begin{equation}
        | \phi_n(q) |=\prod_{z\in\Delta_n}| q-z |.
    \end{equation}
    Étant donné que ce produit doit être inférieur à \( q-1\), au moins un des termes doit l'être : il existe \( z_0\in \Delta_n\) tel que \( | z_0-q |\leq q-1\). Étant donné que \( n\geq 2\) nous avons \( z_0\neq 1\).

    Mais d'autre part, comme indiqué sur la figure \ref{LabelFigtrigoWedd}, la distance entre \( z_0\) et \( q\) doit être strictement plus grande que \( q-1\) parce que \( q-1\) est le minimum de la distance entre le cercle trigonométrique et \( q\), et n'est atteint qu'en \( z=1\).
    \newcommand{\CaptionFigtrigoWedd}{Nous devons avoir \( | z_0-q |>q-1\).}
    \input{pictures_tex/Fig_trigoWedd.pstricks}

    Nous avons ainsi obtenu une contradiction, et nous concluons que le corps \( \eK\) est commutatif.
\end{proof}
