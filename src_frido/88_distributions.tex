% This is part of Mes notes de mathématique
% Copyright (c) 2011-2016
%   Laurent Claessens
% See the file fdl-1.3.txt for copying conditions.

Nous donnons ici une partie de la théorie sur les distributions. L'utilisation des distributions dans le cadre des équations différentielles est mise dans le chapitre sur les équations différentielles, section \ref{SecTNgeNms}.

%///////////////////////////////////////////////////////////////////////////////////////////////////////////////////////////
\subsubsection{Dérivée partielle au sens faible}
%///////////////////////////////////////////////////////////////////////////////////////////////////////////////////////////


\begin{definition}      \label{DEFooBRFCooPncSCE}
    Si \( i=1,\ldots, n\), la \defe{dérivée faible}{dérivée!faible} de \( v\) dans la direction \( e_i\) est l'application\footnote{En fait c'est une classe au sens de l'égalité presque partout.} notée \( \partial_iv\) définie par
    \begin{equation}        \label{EQooMRZUooFoqPqv}
        \langle \partial_iv, \phi\rangle =-\langle v, \partial_i\phi\rangle 
    \end{equation}
    pour tout \( \phi\in  C^{\infty}_c(\Omega)\).
\end{definition}

\begin{lemma}
    Si \( v\in L^2\) admet une dérivée faible, alors cette dernière est unique.
\end{lemma}

\begin{proof}
    Supposons \( f,g\) telles que \( \langle g, \phi\rangle \) et \( \langle f, \phi\rangle \) soient tous deux égaux à \( -\langle v, \partial_i\phi\rangle \). En particulier pour tout \( \phi\in  \swD(\Omega)\) nous avons \( \langle (f-g), \phi\rangle =0\). 

    Cela donne \( f-g=0\) par la proposition \ref{PropUKLZZZh}.
\end{proof}

%+++++++++++++++++++++++++++++++++++++++++++++++++++++++++++++++++++++++++++++++++++++++++++++++++++++++++++++++++++++++++++ 
\section{Topologie}
%+++++++++++++++++++++++++++++++++++++++++++++++++++++++++++++++++++++++++++++++++++++++++++++++++++++++++++++++++++++++++++

Soit \( \Omega\) un ouvert de \( \eR^d\). Le but de notre histoire est de définir une distribution comme étant un élément de l'espace dual (topologique, voir définition \ref{DefJPGSHpn}) de l'espace \( \swD(\Omega)\) des fonctions \( C^{\infty}\) à support compact dans \( \Omega\). Pour ce faire nous devons voir un peu de topologie sur différents espaces de fonctions. Notons que cet espace n'est pas réduit à la fonction nulle comme en témoigne l'exemple donné par l'équation \eqref{EqOBYNEMu}.

Pour chaque \( K\) compact dans \( \Omega\) et multiindice \( \alpha\in \eN^d\) nous considérons sur \(  C^{\infty}(\Omega)\) la semi-norme suivante :
\begin{equation}
    p_{K,m}(f)=\sum_{| \mu |\leq m}\| \partial^{\mu}f \|_{K,\infty}.
\end{equation}
%TODO : prouver que ce sont des semi-normes.
En particulier,
\begin{equation}
    p_{K,0}(f)=\sup_{x\in K}| f(x) |=\| f \|_{\infty,K}.
\end{equation}

\begin{definition}  \label{DefFGGCooTYgmYf}
Les topologies que nous allons considérer sont :
\begin{enumerate}
    \item
        Sur \(  C^{\infty}(\Omega)\), la topologie des semi-normes \( p_{K,m}\) (avec \( K\) et \( m\) comme paramètres).
    \item
        Sur \( \swD(K)\), la topologie des semi-normes \( p_{K,m}\) (avec seulement \( m\) comme paramètre).
    \item
        Sur \( \swD(\Omega)\), la topologie induite de \(  C^{\infty}(\Omega)\).
\end{enumerate}
\end{definition}
\index{topologie!sur $ C^{\infty}(\Omega)$ }
\index{topologie!sur $ \swD(K)$ }
\index{topologie!sur $ \swD(\Omega)$ }
Cela n'est pas très explicite, mais heureusement nous n'aurons souvent pas besoin de plus que de la notion de convergence dans \( \swD'(\Omega)\). Rappelons que la topologie d'un espace donne la notion de convergence par la définition \ref{DefXSnbhZX}.

\begin{lemma}[Convergence dans \( \swD(K)\)]    \label{LemXXwDjui}
    Si \( \alpha\) est un multiindice et si \( \varphi_n\stackrel{\swD(K)}{\longrightarrow}\varphi\), alors nous avons
    \begin{equation}
        \partial^{\alpha}\varphi_n\stackrel{unif}{\longrightarrow}\partial^{\alpha}\varphi.
    \end{equation}
\end{lemma}

\begin{proof}
    Quitte à considérer la suite \( \varphi_n-\varphi\) nous pouvons supposer \( \varphi_n\stackrel{\swD(K)}{\longrightarrow}0\). Nous avons
    \begin{equation}
        \| \partial^{\alpha}\varphi_n \|\leq \sum_{\mu\leq\alpha}\| \partial^{\mu}\varphi_n \|_{K,\infty}.
    \end{equation}
    Vu que le membre de droite tend vers zéro, nous avons 
    \begin{equation}
        \lim_{n\to \infty} \| \partial^{\alpha}\varphi_n \|_{K,\infty}\to 0,
    \end{equation}
    ce qui revient à dire que \( \partial^{\alpha}\varphi_n\) converge uniformément sur \( K\) vers \( \partial^{\alpha}\varphi\).
\end{proof}

\begin{lemma}   \label{LemWEGpemo}
    Si une fonction \( f\colon \swD(\Omega)\to \eR\) est continue sur chacun des \( \swD(K)\) pour tout \( K\) compact dans \( \Omega\) alors est continue sur \( \swD(\Omega)\).
\end{lemma}

\begin{proof}
    Soit \( I\) ouvert dans \( \eR\); nous devons trouver un ouvert \( \mO\) dans \(  C^{\infty}(\Omega)\) tel que \( f^{-1}(I)=\swD(\Omega)\cap\mO\). Vu que \( f\) est continue sur chacun des \( \swD(K)\) avec \( K\) compact dans \( \Omega\), pour tout tel compact nous avons un ouvert \( \mO_K\) dans \( \swD(K)\) tel que \( f^{-1}(I)\cap \swD(K)=\mO_K\). En tant qu'union d'ouverts\footnote{Voir définition \ref{DefTopologieGene}.}, l'ensemble
    \begin{equation}
        \mO=\bigcup_{\text{\( K\) compact de \( \Omega\)}}\mO_K
    \end{equation}
    est ouvert dans \(  C^{\infty}(\Omega)\). Si \( \phi\inf^{-1}(I)\), nous avons \( \phi\in\swD(K)\) pour un certain \( K\) compact de \( \Omega\), donc \( f^{-1}(I)\subset\mO\). A forciori nous avons \( f^{-1}(I)\subset\mO\cap\swD(\Omega)\).

    Dans l'autre sens, si \( \phi\in\mO\), alors \( \phi\) est dans un des \( \mO_K\) et donc dans \( f^{-1}(I)\). Nous avons donc bien \( f^{-1}(I)=\swD(\Omega)\cap \mO\).
\end{proof}

\begin{theorem}[Convergence dans \( \swD(\Omega)\)\cite{TQSWRiz}]       \label{ThoXYADBZr}
    Soit \( (\varphi_n)_{n\in \eN}\) une suite dans \( \swD(\Omega)\) et \( \varphi\in\swD(\Omega)\). Nous avons \( \varphi_n\stackrel{\swD(\Omega)}{\longrightarrow}\varphi\) si et seulement s'il existe \( K\) compact dans \( \Omega\) tel que \( \varphi_n\in\swD(K)\) pour tout \( n\) et \( \varphi_n\stackrel{\swD(K)}{\longrightarrow}\varphi\).
\end{theorem}

\begin{proof}
    Supposons que \( \varphi_n\stackrel{\swD(\Omega)}{\longrightarrow}\varphi\) et qu'il n'existe pas de compacts contenant tous les supports des \( \varphi_n\). Alors pour tout compact de \( \Omega\) il existe un \( n\) tel que le support de \( \varphi_n\) ne soit pas dans \( K\). Nous considérons une suite de compacts \( (K_i)\) tels que \( \Int(K_n)\subset K_{n+1}\) et \( \Omega=\bigcup_nK_n\). Une telle suite existe par le lemme \ref{LemGDeZlOo}. Ensuite nous construisons des sous-suites de la façon suivante. D'abord \( L_1=K_1\) et \( n_1\in \eN\) est choisit de telle sorte que \( \varphi_{n_1}\) ait un support non contenu dans \( L_1\). Ensuite \( L_i\) est un compact de la suite \( (K_n)\) choisit plus loin que \( L_{i-1}\) et tel que \( \varphi_{n_{i-1}}\in \swD(L_i)\). Le nombre \( n_{i}\) est alors choisit plus grand que \( n_{i-1}\) de telle sorte que \( \varphi_{n_i}\notin\swD(L_i)\). Ce faisant, en posant \( \phi_i=\varphi_{n_i}\) nous avons
    \begin{equation}
        \phi_i\in\swD(L_{i+1})\setminus\swD(L_i)
    \end{equation}
    et \( \Int(L_n)\subset L_{n+1}\) et \( \Omega=\bigcup_nL_n\). Étant donné que \( (\phi_i)\) et une sous-suite de \( (\varphi_i)\) nous avons encore \( \phi_i\stackrel{\swD(\Omega)}{\longrightarrow}\varphi\).

    Soit \( i\in \eN\). Nous allons utiliser la seconde forme géométrique du théorème de Hahn-Banach \ref{ThoACuKgtW} pour séparer les parties \( \{ \phi_i \}\) (compact) et \( \swD(L_i)\) (fermé) dans \( \swD(\Omega)\). Nous avons \( f_i\in \swD'(\Omega)\) telle que 
    \begin{subequations}
        \begin{numcases}{}
            f_i(\phi_i)>\alpha\\
            f\big( \swD(L_i) \big)<\alpha. 
        \end{numcases}
    \end{subequations}
    Nous redéfinissons immédiatement \( f_i\) de façon à avoir
    \begin{subequations}
        \begin{numcases}{}
            f_i(\phi_i)=0\\
            f\big( \swD(L_i) \big)<0.
        \end{numcases}
    \end{subequations}

    Nous introduisons la fonction définie sur \( \swD(\Omega)\) par
    \begin{equation}    \label{EqJCqeXti}
        p(\phi)=\sum_{i=1}^{\infty}i\frac{ f_i(\phi) }{ | f_i(\phi_i) | }.
    \end{equation}
    Si \( \phi\in L_k\), alors \( f_k(\phi)=0\) et même \( f_{l}(\phi)=0\) pour tout \( l\geq k\). Donc pour chaque \( k\), la somme définissant \( p\) est finie sur \( \swD(L_k)\). Nous en déduisons que \( p\) est continue sur chacun des \( \swD(L_k)\) et donc sur \( \swD(\Omega)\) par le lemme \ref{LemWEGpemo}.

    L'image de la suite convergente \( \phi_k\stackrel{\swD(\Omega)}{\longrightarrow}\varphi\) par \( p\) doit être bornée parce que \( p\) est continue. Mais dans la somme  \eqref{EqJCqeXti}, tous les termes sont positifs et en particulier le terme \( i=k\) vaut \( k\), donc \( p(\phi_k)\geq k\), ce qui contredit le fait que l'image de la suite soit bornée. Nous en déduisons donc l'existence d'un compact \( K\) tel que \( \varphi_n\in \swD(K)\) pour tout \( n\).

    Nous devons encore prouver que \( \varphi_n\stackrel{\swD(K)}{\to}\varphi\) pour ce choix de \( K\). Vu que \( \varphi_n\stackrel{\swD(\Omega)}{\longrightarrow}\varphi\), le lemme \ref{LemPESaiVw} nous dit que nous avons aussi \( \varphi_n\stackrel{ C^{\infty}(\Omega)}{\longrightarrow}\varphi\), ce qui signifie que pour tout \( K\) et \( m\) nous avons
    \begin{equation}
        p_{K,m}(\varphi_n-\varphi)\to 0.
    \end{equation}
    En particulier pour le \( K\) fixé plus haut nous avons \( p_m(\varphi_n-\varphi)\to 0\), c'est à dire que \( \varphi_n\stackrel{\swD(K)}{\longrightarrow}\varphi\).
    
\end{proof}

%\begin{definition}[Convergence dans \( \swD(\Omega)\)]
%    Nous avons \( f_n\stackrel{\swD'(\Omega)}{\longrightarrow}f\) si et seulement si \( f_n(\varphi)\to f(\varphi)\) pour tout \( \varphi\in\swD(\Omega)\).
%\end{definition}

\begin{proposition} \label{PropQAEVcTi}
    Si \( K\) est compact dans \( \Omega\), l'espace \( \swD(K)\) est métrique et complet.
\end{proposition}
\index{espace!complet!\( \swD(K)\)}

\begin{proof}

    Nous allons d'abord montrer que \( \swD(K)\) est complet. Ensuite nous allons montrer que sa topologie peut être donnée par une distance.

    \begin{subproof}
        \item[Complet]
            Nous considérons une suite de Cauchy \( (\varphi_n)\) dans \( \swD(K)\) au sens de la définition \ref{DefZSnlbPc}. Soient \( \epsilon>0\) et \( i\in \eN\); si \( k\) et \( l\) sont assez grands nous avons
            \begin{equation}
                \varphi_k-\varphi_l\in B_i(0,\epsilon).
            \end{equation}
            En particulier pour \( i=0\) nous avons l'inégalité
            \begin{equation}
                \| \varphi_k-\varphi_l \|_{\infty}\leq \epsilon,
            \end{equation}
            La suite \( (\varphi_n)\) est donc de Cauchy dans \( \big( C(K),\| . \|_{\infty} \big)\) et y converge donc par complétude, proposition \ref{PropSYMEZGU}. Il existe donc une fonction \( \varphi\in C(K)\) telle que
            \begin{equation}
                \varphi_n\stackrel{unif}{\longrightarrow}\varphi.
            \end{equation}
            Notre jeu à présent est de prouver que \( \varphi\in\swD(K)\), c'est à dire qu'elle est de classe \(  C^{\infty}\). 

            Soit un multiindice \( \alpha=\mu_1,\ldots, \mu_n,i\). Si \( k\) et \( l\) sont assez grands nous avons
            \begin{equation}
                \| \partial^{\alpha}(\varphi_k-\varphi_l) \|_{\infty}\leq \epsilon,
            \end{equation}
            c'est à dire que 
            \begin{equation}
                \| \partial_i(\partial^{\mu}\varphi_k)-\partial_i(\partial^{\mu}\varphi_l) \|_{\infty}\leq \epsilon.
            \end{equation}
            Si nous notons \( \psi_k=\partial^{\mu}\varphi_k\) cela signifie que \( (\partial_i\psi_n)\) est une suite de Cauchy dans \( \big( C(K),\| . \|_{\infty} \big)\). Elle y converge donc et il existe une fonction \( g_i\in C(K)\) telle que
            \begin{equation}
                \partial_i\psi_n\stackrel{unif}{\longrightarrow}g_i.
            \end{equation}
            Dans ce cas le théorème \ref{ThoSerUnifDerr} nous indique que \( \psi\) est de classe \( C^1\), c'est à dire que \( \varphi\in C^{n+1}(K)\).

        \item[Métrique]

            La proposition \ref{PropLOwUvCO} nous dit que la topologie donnée par l'écart 
            \begin{equation}
                d(\varphi_1,\varphi_2)=\sup_{k\geq 1}\min\{ \frac{1}{ k },p_{k-1}(\varphi_1-\varphi_2) \}
            \end{equation}
            est la même que celle de \( \swD(K)\). Il reste à montrer que cette formule est bien une distance au sens de la définition \ref{DefMVNVFsX}.
            \begin{enumerate}
                \item
                    Nous avons bien \( d(\varphi_1,\varphi_2)\geq 0\) parce que tous les éléments du supremum et du minimum sont positifs.
                \item
                    Si \( d(\varphi_1,\varphi_2)=0\) alors pour tout \( k\) nous devons avoir \( p_{k-1}(\varphi_1-\varphi_2)=0\); en particulier pour \( k=1\) cela donne \( \varphi_1=\varphi_2\).
                \item
                    Nous avons
                    \begin{equation}
                        p_k(\varphi_1-\varphi_2)=p_k\big( -(\varphi_2-\varphi_1) \big)=p_l(\varphi_2-\varphi_1)
                    \end{equation}
                    en utilisant la propriété \ref{ItemSHnimhDii} de la définition \ref{DefPNXlwmi} de semi-norme.
                \item
                    Nous avons
                    \begin{equation}
                        p_k(\varphi_1-\varphi_2)=p_k(\varphi_1-\varphi_3+\varphi_3-\varphi_2)\leq p_k(\varphi_1-\varphi_3)+p_k(\varphi_3-\varphi_2)
                    \end{equation}
                    en utilisant la propriété \ref{ItemSHnimhDiii} de la définition \ref{DefPNXlwmi}.
            \end{enumerate}
        \end{subproof}
\end{proof}
Notons que le proposition \ref{PropLOwUvCO} nous dit que \( \swD(K)\) est complet tout autant pour la topologie des semi-normes que pour celle de la distance que nous venons de décrire. Ces deux topologies sont les mêmes. Étant métrique et complet, l'espace \( \swD(\Omega)\) et donc de Baire par le théorème \ref{ThoBBIljNM}. Ce qui est bien avec ces deux topologies identiques c'est qu'on peut utiliser la propriété de Baire même en ne parlant que des semi-normes.

%+++++++++++++++++++++++++++++++++++++++++++++++++++++++++++++++++++++++++++++++++++++++++++++++++++++++++++++++++++++++++++ 
\section{Distributions}
%+++++++++++++++++++++++++++++++++++++++++++++++++++++++++++++++++++++++++++++++++++++++++++++++++++++++++++++++++++++++++++

Si \( \Omega\) est un ouvert de \( \eR^d\), alors l'ensemble \( \swD(\Omega)\) est contenu dans \(  C^{\infty}(\Omega)\). Nous allons commencer par définir une topologie sur \(  C^{\infty}(\Omega)\) et ensuite donner à \( \swD(\Omega)\) la topologie induite\footnote{Définition \ref{DefVLrgWDB}.}.

\begin{definition}[Distribution]    \label{DefPZDtWVP}
    Une \defe{distribution}{distribution} sur un ouvert \( \Omega\) de \( \eR^d\) est une forme linéaire continue sur \(\swD(\Omega)= C^{\infty}_c(\Omega)\)\nomenclature[Y]{\( \swD(\Omega)\)}{Les fonctions \( C^{\infty}\) à support compact sur \( \Omega\)}. C'est donc un élément de \( \swD'(\Omega)\).
\end{definition}

Le théorème suivant donne quelques façons de vérifier qu'une forme linéaire soit continue. En particulier il nous dit que pour prouver qu'une forme linéaire est une distribution il suffi de prouver la continuité séquentielle.
\begin{theorem}[\cite{TQSWRiz,RIOTOaj}] \label{ThoVDDBnVn}
    Soit \( T\) une forme linéaire sur \( \swD(\Omega)\). Nous avons équivalence entre les points suivants.
    \begin{enumerate}
        \item
            \( T\) est continue.
        \item   \label{ItemSPvoijoii}
            Pour tout compact \( K\subset \Omega\) il existe \( m\in \eN\) et \( C\geq 0\) tel que pour tout \( \varphi\in\swD(K)\) nous ayons
            \begin{equation}
                \big| T(\varphi) \big|\leq C p_{m,K}(\varphi).
            \end{equation}
        \item
            \( T\) est séquentiellement continue sur \( \swD(\Omega)\).
        \item
            \( T\) est séquentiellement continue en \( 0\).
        \item
            Pour tout compact \( K\subset \Omega\), la restriction de \( T\) à \( \swD(K)\) est continue.
    \end{enumerate}
\end{theorem}
Un certain nombre d'ouvrages comme \cite{WTefHHs} prennent le point \ref{ItemSPvoijoii} comme la définition d'une distribution.

\begin{definition}[Topologie sur \( \swD'(\Omega)\)]        \label{DefASmjVaT}
    Nous munissons l'espace \( \swD'(\Omega)\) de la \defe{topologie \( *\)-faible}{topologie!$*$-faible}, c'est à dire celle de la famille de semi-normes
    \begin{equation}
        \begin{aligned}
            p_{\varphi}\colon \swD'(\Omega)&\to \eR \\
            T&\mapsto \big| T(\varphi) \big|
        \end{aligned}
    \end{equation}
    avec \( \varphi\in\swD(\Omega)\).
\end{definition}
Oui, c'est bien une famille de semi-normes indicée par l'ensemble \( \swD(\Omega)\). Il n'y en a donc a priori pas du tout une quantité dénombrable.

\begin{proposition}[Convergence au sens des distributions]  \label{PropEUIsNhD}
    Nous avons \( T_n\stackrel{\swD'(\Omega)}{\longrightarrow}T\) si et seulement si \( T_n(\varphi)\to T(\varphi)\) pour tout \( \varphi\in\swD(\Omega)\).
\end{proposition}

\begin{proof}
    La convergence \( T_n\stackrel{\swD'(\Omega)}{\longrightarrow}T\) signifie que l'on ait \( p_{\varphi}(T_n-T)\to 0\) pour tout \( \varphi\in\swD(\Omega)\), ce qui en retour signifie que
    \begin{equation}
        \big| (T_n-T)(\varphi) \big|\to 0.
    \end{equation}
\end{proof}
Cette proposition suppose que l'on ait une distribution \( T\) qui vérifie \( T_n(\varphi)\to T(\varphi)\) et conclut qu'on a une convergence dans les distributions. Le théorème suivant est plus fort : il va seulement supposer que \( T_n(\varphi)\) converge dans \( \eC\) et va conclure que \( T\colon \varphi\mapsto \lim_{n\to \infty} T_n(\varphi)\) est une distribution.

\begin{theorem}[\cite{GQYneyj}]
    Soit \( (T_n)\) une suite dans \( \swD'(\Omega)\) et nous supposons que pour tout \( \varphi\in\swD(\Omega)\) la suite \( \big( T_n(\varphi) \big)\) converge dans \( \eC\). Alors il existe \( T\in\swD'(\Omega)\) telle que \( T_n\stackrel{\swD'(\Omega)}{\longrightarrow}T\).
\end{theorem}

\begin{proposition}\label{PROPooYAJSooMSwVOm}
    L'application
    \begin{equation}
        \begin{aligned}
            i\colon L^2(\Omega)&\to \swD'(\Omega) \\
            f&\mapsto T_f 
        \end{aligned}
    \end{equation}
    est une injection continue.
\end{proposition}

\begin{proof}
    Le fait que ce soit une injection est le fait que si \( T_f=T_g\) alors pour tout \( \phi\in \swD(\Omega)\) nous avons \( \langle f-g, \phi\rangle =0\), et cela implique que \( f-g\) est nulle presque partout en tant que fonction et est simplement nulle en tant que classe de fonction dans \( L^2\).

    En ce qui concerne la continuité, il suffit de la prouver en zéro (par linéarité). Soit donc \( f_n\stackrel{L^2}{\longrightarrow}0\) et montrons que \( T_{f_n}\stackrel{\swD'(\Omega)}{\longrightarrow} T_0\). Pour prouver cela, la proposition \ref{PropEUIsNhD} nous indique qu'il est suffisant de tester \( T_n(\phi)\to 0\) pour tout \( \phi\in\swD(\Omega)\).

    Notons que si \( \phi\in \swD\) a fortiori \( \phi\in L^2\). Nous avons
    \begin{equation}
        T_{f_n}(\phi)=\int_{\Omega}f_n\phi\leq \| f_n\phi \|_{L^1}\leq \| f_n \|_2\| \phi \|_2\to 0
    \end{equation}
    où nous avons utilisé l'inégalité de Hölder de la proposition \ref{ProptYqspT}.
\end{proof}

Cette proposition permet de donner un sens à des phrases du type «Soit une distribution \( T\). Si \( T\in L^2\), alors \ldots». Cela signifie qu'il existe \( u\in L^2\) tel que \( T=T_u\). Notons que dans ce cas, la distribution est définie sur \( L^2\) et non seulement sur \( \swD\).

%--------------------------------------------------------------------------------------------------------------------------- 
\subsection{Multiplication d'une distribution par une fonction}
%---------------------------------------------------------------------------------------------------------------------------

\begin{definition}  \label{DefZVRNooDXAoTU}
Si \( T\in\swD'(\Omega)\) et si \( f\in  C^{\infty}(\Omega)\) nous définissons la distribution \( fT\) par
\begin{equation}    \label{DefTDkrqkA}
    (fT)(\varphi)=T(f\varphi).
\end{equation}
Souvent écrit sous la forme plus compacte \( \langle fu, \phi\rangle =\langle u, f\phi\rangle \).
\end{definition}
\index{distribution!produit par une fonction}
\index{produit!distribution et fonction}
Cela a un sens parce que si \( \varphi\in\swD(\Omega)\) alors \( f\varphi\) est aussi dans \( \swD(\Omega)\). 

Cette définition est motivée par ce que l'on ferait pour une distribution à densité. Si \( T\) est une distribution de densité notée également \( T\), nous avons \( T(\phi)=\int T(x)\phi(x)\) et donc
\begin{equation}
    (fT)(\phi)=\int (fT)(x)\phi(x)=\int T(x)f(x)\phi(x)=\int T(x)(f\phi)(x)=T(f\phi).
\end{equation}

En ce qui concerne les distributions tempérées, nous pouvons définir le produit avec une fonction \( f\in\swS(\Omega)\) par la même formule : si \( f,\varphi\in\swS(\Omega)\) alors le produit \( f\varphi\) est encore Schwartz. Notons toutefois que nous ne pouvons pas définir \( fT\) dans \( \swS'(\Omega)\) si \( f\) est seulement dans \(  C^{\infty}(\Omega)\).
% TODO : il faudrait prouver cette dernière affirmation.

%--------------------------------------------------------------------------------------------------------------------------- 
\subsection{Dérivée de distribution}
%---------------------------------------------------------------------------------------------------------------------------

\begin{propositionDef} \label{PropKJLrfSX}
    Soit \( T\) une distribution sur \( \Omega\) et \( \alpha\in \eN^d\). Alors la formule
    \begin{equation}
        (\partial^{\alpha}T)(\varphi)=(-1)^{| \alpha |}T(\partial^{\alpha}\varphi)
    \end{equation}
    définit une distribution \( \partial^{\alpha}T\).

    Cette distribution \( \partial^{\alpha}T\) sera la \defe{dérivée distributionnelle}{dérivée!distributionnelle} de \( T\). Notons que le même résultat est encore valide pour des distributions tempérées, et la démonstration est la même.
\end{propositionDef}

\begin{proof}
    La forme linéaire \( \partial^{\alpha}T\) sera continue si elle est séquentiellement continue par le théorème \ref{ThoVDDBnVn}. Nous considérons donc une suite \( \varphi_n\stackrel{\swD(\Omega)}{\longrightarrow}\varphi\) et nous vérifions que
    \begin{equation}
        \lim_{n\to \infty} (\partial^{\alpha}T)(\varphi_n)=(\partial^{\alpha}T)(\varphi).
    \end{equation}
    D'abord \( T\) étant une distribution (et donc continue) nous pouvons la permuter avec la limite :
    \begin{equation}
        \lim_{n\to \infty} (\partial^{\alpha}T)(\varphi_n)=\lim_{n\to \infty} (-1)^{| \alpha |}T(\partial^{\alpha}\varphi_n)=(-1)^{| \alpha |}T\big( \lim_{n\to \infty} \partial^{\alpha}\varphi_n \big).
    \end{equation}
    Notons qu'à gauche la limite est une limite dans \( \eR\) tandis qu'à droite c'est une limite dans \( \swD(\Omega)\). Ensuite le lemme \ref{LemXXwDjui} nous dit que l'hypothèse \( \varphi_n\stackrel{\swD(\Omega)}{\longrightarrow}\varphi\) signifie en particulier que nous avons un compact \( K\subset\Omega\) contenant tous les supports des \( \varphi_n\) et que \( \partial^{\alpha}\varphi_n\) converge uniformément (sur \( K\) et donc sur \( \Omega\)) vers \( \partial^{\alpha}\varphi\). Donc
    \begin{equation}
        \lim_{n\to \infty} (\partial^{\alpha}T)(\varphi_n)=(-1)^{| \alpha |}T\big( \lim_{n\to \infty} \partial^{\alpha}\varphi_n \big)=(-1)^{| \alpha |}T\big( \partial^{\alpha}\varphi \big)=(\partial^{\alpha}T)(\varphi),
    \end{equation}
    ce qui est la relation demandée.
\end{proof}

Le lemme suivant montre une compatibilité entre la dérivée des distribution, la dérivée faible et l'injection de \( L^2\) dans l'espace des distributions.

\begin{lemma}       \label{LEMooQRUOooWVjCAV}
    Soit \( \Omega\) un ouvert bornée de \( \eR^n\) et \( f\in L^2(\Omega)\). Alors nous avons
    \begin{equation}
        \partial_i(T_f)=T_{\partial_if}
    \end{equation}
    où la dérivée à droite est la dérivée faible définie en \ref{DEFooBRFCooPncSCE}.
\end{lemma}

\begin{proof}
    En utilisant la définition de la dérivation de distribution, pour tout \( \phi\in \swD\) nous avons
    \begin{equation}
        \partial_i(T_f)\phi=-T_f(\partial_i\phi)=-\langle f, \partial_i\phi\rangle =\langle \partial_if, \phi\rangle =T_{\partial_if}(\phi).
    \end{equation}
    Nous avons utilisé la définition \eqref{EQooMRZUooFoqPqv} de la dérivée faible.
\end{proof}

%--------------------------------------------------------------------------------------------------------------------------- 
\subsection{Ordre et support d'une distribution}
%---------------------------------------------------------------------------------------------------------------------------

\begin{definition}[support d'une distribution\cite{OEVAuEz}]        \label{DefVILMooBIYerO}
    Soit \( T\) une distribution. Le \defe{support}{support!distribution} de \( T\) est le complémentaire de l'union des ouverts \( \mO\) tels que \( T(\varphi)=0\) pour tout \( \varphi\) à support dans \( \mO\).
\end{definition}
\index{support!distribution}
%TODO : dans le cas d'une distribution définie par une densité, il faut voir le lien entre le support de la fonction et le support de la distribution. (j'imagine qu'ils doivent être égaux).

\begin{definition}  \label{DefXAHIooFeiRMB}
    Si \( T\) est une distribution sur \( \Omega\), nous disons que \( T\) est d'\defe{ordre}{ordre!distribution} inférieur ou égal à \( p\in \eN\) si pour tout compact \( K\) de \( \Omega\), il existe \( C_K\in \eR\) tel que pour tout \( \varphi\in \swD(K)\),
    \begin{equation}
        \big| \langle T, \varphi\rangle  \big|\leq C_K\max_{| \alpha |\leq p}\| \partial^{\alpha}\varphi \|_{\infty}.
    \end{equation}
    Ici \( \alpha\) est un multiindice.

    La distribution \( T\) est d'ordre \( p\) si elle est d'ordre inférieur ou égal à \( p\) mais pas à \( p-1\).
\end{definition}

Pour la proposition suivante, on peut se remémorer la définition \ref{DefFGGCooTYgmYf} de la topologie sur \(  C^{\infty}(\Omega)\).
\begin{proposition}[\cite{RGWgOms}]
    Restriction entre \(  C^{\infty}\) et \( \swD\).
    \begin{enumerate}
        \item
            Si \( T\in C^{\infty}(\Omega)'\), alors la restriction de \( T\) à \( \swD(\Omega)\) est une distribution à support\footnote{Définition \ref{DefVILMooBIYerO}.} compact.
        \item
            Si \( T\) est une distribution à support compact alors elle se prolonge de façon unique en une forme linéaire continue sur \(  C^{\infty}(\Omega)\).
    \end{enumerate}
\end{proposition}

\begin{proposition}[\cite{TQSWRiz}]     \label{PropZLUEooHcVxQj}
    Une distribution à support compact est d'ordre fini.
\end{proposition}

\begin{lemma}[\cite{PAXrsMn}]   \label{LemYHRDooOdSnnK}
    Soit \( u\in\swD'(\eR)\) et \( \phi\in\swD(\eR)\) tels que \( \supp(u)\cap\supp(\phi)=\emptyset\). Alors \( \langle u, \phi\rangle =0\).
\end{lemma}

\begin{proof}
    Soit \( x\notin\supp(u)\). Alors il existe un voisinage \( V_x\) de \( x\) tel que \( \langle u, \psi\rangle =0\) pour tout \( \psi\in\swD(V_x)\). En particulier, si \( x\in\supp(\phi)\), alors \( x\) n'est pas dans le support de \( u\) et les ensembles \( \{ V_x\tq x\in\supp(\phi) \}\) recouvrent \( \supp(\phi)\). Cependant \( \phi\) est à support compact et nous pouvons extraire un sous-recouvrement fini de \( \supp(\phi)\) : il existe \( x_1,\ldots, x_p\) tels que
    \begin{equation}
        \supp(\phi)\subset\bigcup_{i=1}^pV_{x_i}.
    \end{equation}
    Nous prenons une partition de l'unité\footnote{Lemme \ref{LemGPmRGZ}.} subordonnée à ce recouvrement. C'est à dire des fonctions \( \chi_i\in\swD(V_{x_i})\) telles que pour tout \( x\in\supp(\phi)\), 
    \begin{equation}
        \sum_{i=1}^p\chi_i(x)=1.
    \end{equation}
    En particulier nous avons \( \sum_i\chi_i(x)\phi(x)=\phi(x)\), et donc
    \begin{equation}
        \langle u, \phi\rangle =\langle u, \sum \chi_i\phi\rangle =\sum\langle u, \chi_i\phi\rangle =0
    \end{equation}
    parce que \( \supp(\chi_i\phi)\subset V_{x_i}\).
\end{proof}

\begin{lemma}[\cite{PAXrsMn}]
    Si \( u\) est une distribution d'ordre fini \( N\) sur \( \eR\), si \( \supp(u)=\{ x_0 \}\) et si 
    \begin{equation}
        \phi(x_0)=\ldots=\phi^{(N)}(x_0)=0
    \end{equation}
    alors \( \langle u, \phi\rangle =0\).
\end{lemma}

\begin{proof}
    Les fonctions plateaux dont nous avons parlé dans la section \ref{subsecOSYAooXXCVjv} nous permettent de considérer une fonction \( \chi\in\swD(\eR)\) vérifiant
    \begin{equation}
        \chi(x)=\begin{cases}
            1    &   \text{si \( x\in\overline{ B(0,1) }\)}\\
            0    &    \text{si \( | x |>2\)}
        \end{cases}
    \end{equation}
    Ensuite nous posons \( \chi_n(x)=\chi\big( n(x-x_0) \big)\). Par conséquent \( \chi_n(x_0)=\chi(0)=1\) et même \( \chi_n(x_0+\epsilon)=\chi(\epsilon)=1\) tant que \( \epsilon\) est plus petit que disons \( \frac{ 1 }{2}\) pour être sur. Nous en déduisons que la fonction \( 1-\chi_n\) s'annule sur un voisinage de \( x_0\) et que donc \( x_0\) n'est pas dans le support de \( 1-\chi_n\). Donc \( \supp(1-\chi_n)\cap\supp(u)=\emptyset\) et le lemme \ref{LemYHRDooOdSnnK} est utilisable :
        $\langle u, (1-\chi_n\phi)\rangle =0$, ou encore :
        \begin{equation}
            \langle u, \phi\rangle =\langle u, \chi_n\phi\rangle 
        \end{equation}
        pour tout \( n\). Vu que le but est de prouver que \( \langle u, \phi\rangle =0\), nous allons prouver que 
        \begin{equation}
            | \langle u, \chi_n\phi\rangle  |\stackrel{n\to\infty}{\longrightarrow}0.
        \end{equation}
        Dans ce dessin nous posons     
        \begin{equation}
            \| f \|_n=\sup_{x\in\overline{ B(0,\frac{ 2 }{ n })}}\| f(x) \|
        \end{equation}
        et
        \begin{equation}
            \| f \|_{(p)}=\sup_{i\leq p}\| \partial^if \|_{\infty}.
        \end{equation}
        La distribution \( u\) est d'ordre fini \( N\), et nous en écrivons la définition \ref{DefXAHIooFeiRMB} en prenant \( \supp(\chi_n\phi)\) en tant que \( K\) :
        \begin{equation}
            \big| \langle u, \chi_n\phi\rangle  \big|\leq C\max_{k\leq N}\| \partial^k(\chi_n\phi) \|_{\infty}.
        \end{equation}
        En remplaçant le maximum par une somme de \( k=0\) à \( k=N\), nous majorons. De plus le support de \( \chi_n\) étant contenu dans \( B_n=B(x_0,2/n)\) nous ne changeons rient en utilisant \( \| . \|_n\) au lieu de \( \| . \|_{\infty}\). Donc
        \begin{equation}
            | \langle u, \chi_n\phi\rangle  |\leq C\sum_{k=0}^N\| \partial^k(\chi_n\phi) \|_n\leq C\sum_{k=0}^N\binom{ k }{ i }\| \partial^i\chi_n \|_n\| \partial^{k-i}\phi \|_n.
        \end{equation}
        Notons que la seconde inégalité est une inégalité du type \( \| fg \|\leq \| f \|\| g \|\). En dérivant un petit peu nous trouvons que
        \begin{equation}
            (\partial^i\chi_n)(x)=n^i(\partial^i\chi)\big( n(x-x_0) \big).
        \end{equation}
        Donc\quext{Dans \cite{PAXrsMn}, la dernière égalité vient avec une inégalité, et je comprends pas pourquoi.}
        \begin{equation}    \label{EqFKWTooFfZoSM}
            \| \partial^i\chi_n \|_n=\sup_{x\in B_n}n^i\big| (\partial^i\chi)\big( n(x-x_0) \big) \big|=n^i\sup_{y\in\mathopen[ -2 , 2 \mathclose]}\big| (\partial^i\chi)(y) \big|=n^i\| \partial^i\chi \|_{\infty}.
        \end{equation}
        Nous pouvons donc remplacer \( \| \partial^u\chi_n \|_n\) par \( n^i\| \partial^i\chi \|_{\infty}\).

        D'autre part nous voulons majorer \( \| \partial^{k-i}\phi \|_n\) par quelque chose ne dépendant ni de \( k\) ni de \( i\). Nous faisons le théorème des accroissements finis \ref{ThoNAKKght} : \( \| \partial^l\phi \|_n\leq \frac{ 2 }{ n }\| \partial^{l+1}\phi \|_n\). Ce \( n\) au dénominateur est salutaire parce que nous avions un \( n^i\) apparu à cause du remplacement \eqref{EqFKWTooFfZoSM}. Nous faisons donc \( i+1\) fois le théorème des accroissements finis :
        \begin{equation}
            \| \partial^{k-i} \phi\|_n\leq \left( \frac{ 2 }{ n } \right)^{i+1}\| \partial^{k+1}\phi \|_n.
        \end{equation}
        Toutes ces majorations donnent
        \begin{subequations}
            \begin{align}
            \big| \langle u, \chi_n\phi\rangle  \big|&\leq C\sum_{k=0}^N\sum_{i=0}^k\binom{ i }{ k }n^i\underbrace{\| \partial^i\chi \|_{\infty}}_{\leq \| \chi \|_{(N)}}\left( \frac{ 2 }{ n } \right)^{i+1}  \underbrace{\|  \partial^{k+1}\phi \|_n}_{\leq \| \phi \|_{(N+1)}}\\
            &\leq C\| \chi \|_{(N)}\| \phi \|_{(N+1)}\frac{1}{ n }\sum_{k=0}^N\sum_{i=0}^k\binom{ k }{ i }2^{i+1}\\
            &=\frac{ C' }{ n }
            \end{align}
        \end{subequations}
        où \( C'\) est une constante qui dépend de \( \chi\), de \( \phi\) et de \( N\), mais pas de \( n\). Vu que \( \frac{ C' }{ n }\to 0\) nous avons bien 
        \begin{equation}
            \langle u, \phi\chi_n\rangle=0,
        \end{equation}
        ce qu'il fallait démontrer.
\end{proof}


\begin{proposition}[\cite{PAXrsMn}]     \label{PropXXPLooSkgxOz}
    Soit \( u\in\swD(\eR)'\) avec \( \supp(u)=\{ x_0 \}\). Alors \( u=\sum_{i=0}^{N}a_i\partial^i\delta_{x_0}\) où \( N\) est l'ordre de \( u\).
\end{proposition}

\begin{proof}
    D'abord il faut préciser que l'ordre de \( u\) est fini parce que son support est compact (proposition \ref{PropZLUEooHcVxQj}); nous notons \( N\) cet ordre.

    Soit \( \phi\in\swD(\eR)\).  Nous considérons \( \chi\in\swD(\eR)\) telle que
    \begin{equation}
        \chi(x)=\begin{cases}
            1    &   \text{si \( x\in\overline{ B(x_0,1) }\)}\\
            0    &    \text{si \( | x-x_0 |>2\)}.
        \end{cases}
    \end{equation}
    Encore une fois, \( 1-\chi\) s'annule sur un voisinage autour de \( x_0\), ce qui fait que
    \begin{equation}
        \supp(u)\cap\supp\big( (1-\chi)\phi \big)=\emptyset,
    \end{equation}
    et donc \( \langle u, (1-\chi)\phi\rangle =0\). Au final,
    \begin{equation}
        \langle u, \phi\rangle =\langle u, \chi\phi\rangle.
    \end{equation}
    C'est le moment de poser
    \begin{equation}
        \psi(x)=\chi(x)\big[   \phi(x)-\sum_{k=1}^N\frac{1}{ k! }(\partial^k\phi)(x_0)(x-x_0)^k \big]
    \end{equation}
    La fonction \( \psi\) ayant un support disjoint de celui de \( u\), nous avons aussi \( \langle u, \psi\rangle =0\), ce qui donne
    \begin{equation}
        \langle u, \phi\rangle =\langle u, \chi\phi\rangle =\langle u, \chi\sum_{k=0}^N\frac{1}{ k! }(\partial^k\phi)(x_0)(x-x_0)^k\rangle .
    \end{equation}
    En posant \( a_k=\frac{1}{ k! }\langle u, x\mapsto \chi(x)(x-x_0)^k\rangle \) nous avons alors
    \begin{equation}
        \langle u, \phi\rangle =\sum_{k=0}^Na_k(\partial^k\phi)(x_0)=\sum_k(-1)^ka_k(\partial^k\delta_{x_0})(\phi).
    \end{equation}
\end{proof}

%+++++++++++++++++++++++++++++++++++++++++++++++++++++++++++++++++++++++++++++++++++++++++++++++++++++++++++++++++++++++++++ 
\section{Distributions tempérées}
%+++++++++++++++++++++++++++++++++++++++++++++++++++++++++++++++++++++++++++++++++++++++++++++++++++++++++++++++++++++++++++

L'espace de Schwartz\index{espace!de Schwartz}\footnote{Attention : ce Schwartz (avec un \emph{t}) est le Schwartz des distribution dont le prénom est Laurent. À ne pas confondre avec Schwarz (sans \emph{t}) dont le prénom est Cauchy.} \( \swS(\Omega)\) est défini dans la définition \ref{DefHHyQooK}; sa topologie y est discutée.
\begin{definition}
    Une \defe{distribution tempérée}{distribution!tempérée} est une forme linéaire continue sur \( \swS(\eR^d)\). L'ensemble des distributions tempérées est noté \( \swS'(\eR^d)\)\nomenclature[Y]{\( \swS'(\eR^d)\)}{espace des distributions tempérées}. Si \( T\) est une telle distribution, nous notons $\langle T, \varphi\rangle$ l'image de \( \varphi\) par \( T\).
\end{definition}

Si \( f\) est une fonction sur \( \eR^d\) telle que \( f\varphi\in L^1(\eR^d)\) pour tout \( \varphi\in \swS(\eR^d)\), alors nous définissons la distribution \( T_f\in\swS'(\eR^d)\) par
\begin{equation}
    \langle T_f, \varphi\rangle =\int_{\eR^d}f(x)\varphi(x)dx.
\end{equation}
Cette définition ne fonctionne pas pour toute les fonctions. Par exemple pour \( f(x)= e^{x^2}\), et \( \varphi(x)= e^{-x^2}\in\swS(\eR)\) nous avons \( f\varphi=1\) qui n'est pas du tout intégrable sur \( \eR\).

\begin{example}     \label{EXooUGOTooMDCFwD}
    La \defe{distribution de Dirac}{distribution!de Dirac} \( \delta\) est donnée par
    \begin{equation}
        \langle \delta, \varphi\rangle =\varphi(0).
    \end{equation}
    Montrons qu'elle est continue. Soit une suite \( \varphi_n\stackrel{\swS}{\to}0\). En particulier, \( p_{0,0}(\varphi_n)=\sup_x| \varphi_n(x) |\to 0\). Donc \( \varphi_n(0)\to 0\) comme il le faut.
\end{example}

\begin{example}
    La \defe{valeur principale}{valeur!principale (distribution)} de la fonction \( x\mapsto \frac{1}{ x }\) est la distribution
    \begin{equation}
        \begin{aligned}
            T\colon \swS(\eR)&\to \eR \\
            \varphi&\mapsto \lim_{\substack{\epsilon\to 0\\\epsilon>0}}\int_{| x |>\epsilon}\frac{ \varphi(x) }{ x }.
        \end{aligned}
    \end{equation}
    Montrons que cela définit bien une distribution tempérée.

    D'abord l'intégrale existe pour tout \( \epsilon\), par exemple parce que pour les grands \( | x |\) nous avons par exemple \( | \varphi(x)\leq x^3 |\) et donc \( \varphi(x)/x\leq 1/x^2\) dont l'intégrale converge. Nous devons maintenant regarder la limite.

    Nous considérons une suite \( \epsilon_n\to 0\) et la suite
    \begin{equation}
        a_n=\int_{| x |\geq \epsilon_n}\frac{ \varphi(x) }{ x }dx.
    \end{equation}
    Nous montrons que cette suite converge dans \( \eR\) en montrant qu'elle est de Cauchy. Pour cela nous travaillons un peu la forme de \( \varphi\) :
    \begin{equation}
        \varphi(x)=\varphi(0)+\int_0^x\varphi'(t)dt=\varphi(0)+\int_0^1x\varphi'(x\theta)d\theta.
    \end{equation}
    Ce qui est dans l'intégrale est borné par \( K=\| M_x\varphi' \|_{\infty}\) qui est parfaitement fini parce que \( \varphi\) est à décroissance rapide. Lorsque nous calculons \( | a_m-a_n |\), le terme \( \varphi(0)/x\) donne une intégrale nulle parce que le domaine d'intégration \( \epsilon_n\leq | x |\leq \epsilon_n\) est symétrique alors que la fonction \( 1/x\) est impaire.
    \begin{equation}
        | a_m-a_n |\leq \big| \int_{\epsilon_m<| x |<\epsilon_n}K \big|=2| \epsilon_n-\epsilon_m |K
    \end{equation}
    Tout cela nous dit que \( T\) est bien définie. Nous devons encore étudier sa continuité.

    Soit \( \chi\) une fonction dans \(  C^{\infty}_c(\eR)\) telle valant \( 1\) sur \( \mathopen[ -1 , 1 \mathclose]\), paire et à valeurs dans \( \mathopen[ 0 , 1 \mathclose]\).
    %TODO : il faudrait montrer qu'il existe des fonctions C infini à support compact qui ne sont pas nulles partout. C'est fait autour du lemme de Borel.
    Pour tout \( \epsilon>0\) nous avons \( \int_{| x |>\epsilon}\frac{ \chi(x) }{ x }dx=0\). 

    Nous avons aussi \( \varphi=\chi\varphi+(1-\chi)\varphi\), et donc
    \begin{subequations}
        \begin{align}
            \int_{| x |>\epsilon}\frac{ \varphi(x) }{ x }dx&=\int_{| \epsilon |>0}\chi(x)\frac{ \varphi(x)-\varphi(0) }{ x }dx+\int_{| \epsilon |>0}\big( 1-\chi(x) \big)\frac{ \varphi(x) }{ x }dx\\
            &=\int_{| \epsilon |>0}\chi(x)\int_0^1\underbrace{\varphi'(\theta x)}_{\leq \| \varphi' \|_{\infty}}d\theta+\int_{| x |\geq 1}\big( 1-\chi(x) \big)\frac{ \varphi(x) }{ x }dx\\
            &\leq\| \varphi' \|_{\infty}\int_{| x |\geq \epsilon}\chi(x)dx+\| \varphi \|_{L^1}\\
            &=C\| \varphi' \|_{\infty}+\| \varphi \|_{1}.
        \end{align}
    \end{subequations}
    Cela est valable pour toute fonction \( \varphi\in\swS(\eR)\). Mais nous savons que si \( \varphi_n\stackrel{\swS(\eR)}{\to}0\), alors \( \| \varphi_n \|_{\infty}\to 0\), \( \| \varphi'_n \|_{\infty}\to 0\) et \( \| \varphi_n \|_1\to 0\); donc si \( \varphi_n\stackrel{\swS(\eR)}{\to}0\), alors
    \begin{equation}
        T(\varphi_n)=\lim_{\substack{\epsilon\to 0\\\epsilon>0}}\int_{| x |>\epsilon}\frac{ \varphi(x) }{ x }\leq C\| \varphi_n' \|_{\infty}+\| \varphi_n \|_1\to 0.
    \end{equation}
\end{example}

%--------------------------------------------------------------------------------------------------------------------------- 
\subsection{Topologie}
%---------------------------------------------------------------------------------------------------------------------------

La topologie que nous mettons sur l'espace \( \swS'(\eR^d)\) est le même type que celle que nous mettons sur \( \swD'(\eR^d)\), c'est à dire celle des semi-normes \( p_{\varphi}(T)=| T(\varphi) |\). La définition \ref{DefASmjVaT} et la proposition \ref{PropEUIsNhD} restent.

\begin{proposition} \label{PropQAuJstI}
    Nous avons \( T_n\stackrel{\swS'(\eR^d)}{\longrightarrow}T\) si et seulement si pour tout \( \varphi\in\swS(\eR^d)\) nous avons \( T_n(\varphi)\to T(\varphi)\).
\end{proposition}

%--------------------------------------------------------------------------------------------------------------------------- 
\subsection{Distributions associées à des fonctions}
%---------------------------------------------------------------------------------------------------------------------------

Si \( f\in L^1_{loc}(\eR^d)\) alors nous lui associons une distribution \( T_f\in \swD'(\eR^d)\) définie par la formule
\begin{equation}
    T_f(\varphi)=\int_{\eR^d}f(x)\varphi(x)dx.
\end{equation}

\begin{proposition}
    L'application ainsi définie
    \begin{equation}
        \begin{aligned}
            L^1_{loc}(\eR^d)&\to \swD'(\eR^d) \\
            f&\mapsto T_f 
        \end{aligned}
    \end{equation}
    est injective.
\end{proposition}

\begin{proof}
    Si \( T_f=0\) alors pour tout \( \varphi\in \swD\) nous avons \( \int_{\eR^d}f\varphi=0\). En vertu de la proposition \ref{PropLGoLtcS} cela implique \( f=0\) presque partout.
\end{proof}

%--------------------------------------------------------------------------------------------------------------------------- 
\subsection{Composition avec une fonction}
%---------------------------------------------------------------------------------------------------------------------------

\begin{proposition}[\cite{GQYneyj}, page 113 et 32] \label{PropBQUOcyw}
    Soit \( T\in\swS'(\Omega)\) et \( f\in C^k(A\times \Omega)\) où \( A\) est ouvert dans \( \eR^d\). Nous posons
    \begin{equation}
        \begin{aligned}
            F\colon A&\to \eR \\
            \lambda&\mapsto T\big( f(\lambda,.) \big). 
        \end{aligned}
    \end{equation}
    Alors \( F\in C^k(A)\).
\end{proposition}
%TODO : une preuve.

%--------------------------------------------------------------------------------------------------------------------------- 
\subsection{Transformée de Fourier d'une distribution tempérée}
%---------------------------------------------------------------------------------------------------------------------------

\begin{definition}
    La \defe{transformée de Fourier}{transformée!Fourier!distribution tempérée} de la distribution tempérée \( T\in\swS'(\eR^d)\) est la distribution \( \hat T\) définie par
    \begin{equation}
        \hat T(\varphi)=T(\hat \varphi)
    \end{equation}
    pour tout \( \varphi\in\swS(\eR^d)\).
\end{definition}

\begin{lemma}
    Si \( f\in \swS(\eR^d)\), nous avons \( \hat T_f=T_{\hat f}\).
\end{lemma}

\begin{proof}
    En utilisant les définitions,
    \begin{equation}
        \hat T_f(\varphi)=T_f(\hat \varphi)=\int_{\eR^d}f(x)\hat \varphi(x)dx=\int_{\eR^d}f(x)\left[ \int_{\eR^d}\varphi(y) e^{-iyx}dy \right]dx
    \end{equation}
    où nous avons noté \( xy\) le produit scalaire \( x\cdot y\). Nous permutons les intégrales en utilisant le théorème de Fubini \ref{ThoFubinioYLtPI} avec la fonction
    \begin{equation}
        (x,y)\mapsto f(x)\varphi(y) e^{-ixy}
    \end{equation}
    qui est parfaitement dans \( L^1(\eR^d\times \eR^d)\). Nous écrivons alors
    \begin{equation}
        \hat T_f(\varphi)=\int_{\eR^d}\left[ \int_{\eR^d}f(x)\varphi(y) e^{-iyx}dx \right]dy=\int_{\eR^d}\varphi(y)\hat f(y)dy=T_{\hat t}(\varphi).
    \end{equation}
\end{proof}

%--------------------------------------------------------------------------------------------------------------------------- 
\subsection{Convolution d'une distribution par une fonction}
%---------------------------------------------------------------------------------------------------------------------------

Nous savons que si \( \psi\in\swS(\eR^d)\) et si \( x\in\eR^d\) alors la fonction \( y\mapsto\psi(x-y)\) est encore une fonction dans \( \swS(\eR^d)\). Donc si \( T\in\swS'(\eR^d)\) nous pouvons considérer la fonction \( T*\psi=\psi*T\) définie par
\begin{equation}        \label{EQooOUXKooGHDSzL}
    (T*\psi)(x)=T\big( y\mapsto\psi(x-y) \big).
\end{equation}
Notons que \( T*\psi\) est bien une fonction et non une distribution.

Le but de la définition est d'avoir
\begin{equation}
    T_f*\psi=f*\psi.
\end{equation}
En effet
\begin{equation}
    (T_f*\psi)(x)=T_f\big( y\mapsto \psi(x-y) \big)=\int_{\eR^d}f(y)\psi(x-y)dy=(f*\psi)(x).
\end{equation}

\begin{example}
    La distribution de Dirac est le neutre pour le produit de convolution. En effet
    \begin{equation}
        (\delta*\psi)(x)=\delta\big( y\mapsto\psi(x-y) \big)=\psi(x),
    \end{equation}
    c'est à dire \( \delta*\psi=\psi\).
\end{example}

\begin{proposition}[\cite{OEVAuEz}] \label{PropZMKYMKI}
    Si \( T\in\swS'(\eR^d)\) et \( \psi\in\swS(\eR^d)\), alors la distribution associée à la fonction \( T*\psi\) est tempérée.
\end{proposition}

\begin{proof}
    En agissant sur \( \varphi\in\swD(\eR^d)\) nous avons
    \begin{subequations}
        \begin{align}
            T_{T*\psi}(\varphi)&=\int_{\eR^d}T\big( y\mapsto (t_x\psi)(y) \big)\varphi(x)dx\\
            &=\int_{\eR^d}T\big( y\mapsto \varphi(x)\psi(x-y) \big)dx\\
            &=T\left( y\mapsto\int_{\eR^d}\varphi(x)\psi(x-y)dx \right)     \label{SubEqSVDsVTS}\\
            &=T\big(y\mapsto (\varphi*\check\psi)(y)\big)\\
            &=T(\varphi*\check\psi).
        \end{align}
    \end{subequations}

    Attention :
    \begin{probleme}
        Le passage à la ligne \eqref{SubEqSVDsVTS} n'est pas justifié.
    \end{probleme}
\end{proof}

%--------------------------------------------------------------------------------------------------------------------------- 
\subsection{Approximation de la distribution de Dirac}
%---------------------------------------------------------------------------------------------------------------------------


\begin{lemma}[\cite{MonCerveau}]        \label{LEMooHEEOooFtKgfz}
    Soient des fonction \( j_n\colon \eR\to \eR^+\) de classe \(  C^{\infty}\) telles que
    \begin{enumerate}
        \item
            Pour chaque \( n\), la fonction \( x\mapsto j_n\big( | x | \big)\) est strictement décroissante et converge ponctuellement vers zéro.
        \item
            Pour chaque \( x\), la suite \( n\mapsto j_n(x)\) est décroissante et converge vers \( 0\).
        \item       \label{ITEMooFQYXooEkUAIb}
            Pour tout \( M>0\), la suite \( j_n\) converge vers zéro uniformément sur \( B(0,M)^c\).
        \item       \label{ITEMooFYCRooFeRRjE}
            Pour tout \( \delta\) et \( \epsilon\), il existe un \( N\in \eN\) tel que \( | \int_{B(0,\delta)}j_n(x)dx-1 |\leq\epsilon\).
        \item
            Pour tout \( n\), nous avons \( \int_{\eR}j_n=1\).
    \end{enumerate}
    Alors si \( u\in\swS(\eR)\) nous avons
    \begin{equation}
        \lim_{n\to \infty} \int_{\eR}u(x)j_n(x)dx=u(0).
    \end{equation}
\end{lemma}

\begin{proof}
    Nous posons 
    \begin{subequations}
        \begin{align}
            I_n&=\int_{\eR}j_nu\\
            I_{\delta,n}&=\int_{B(0,\delta)}j_nu\\
            Z_{\delta,n}&=\int_{B(0,\delta)}u(0)j_n
        \end{align}
    \end{subequations}
    Nous allons progressivement montrer qu'en prenant \( \delta\) assez petit et \( n\) assez grand, les quantités \( | I_n-I_{\delta,n} |\), \( | I_{\delta,n}-Z_{\delta,n}  |\) et \( | Z_{\delta,n}-u(0) |\) peuvent être simultanément majorées par \( \epsilon\).

    Soit \( \delta>0\) et \( \epsilon>0\); vu que \( u\in\swS(\eR)\), il existe \( M\) tel que \( \int_{| x |>M}| u |<\epsilon\). Soit \( N_1\in \eN\) tel que pour tout \( n>N_1\) nous avons \( | j_n(x) |<1\) dès que \( | x |>M\) (hypothèse \ref{ITEMooFQYXooEkUAIb}). Alors
    \begin{equation}
        \int_{| x |>M}| j_n(x)u(x) |<\epsilon.
    \end{equation}
    De plus en posant \( s=\max\{ | u(x) |\tq \delta\leq | x |\leq M \}\) (qui existe parce que \( u\) est continue et prise sur un compact) nous pouvons considérer \( N_2\) tel que \( j_n(x)<\epsilon/s\) pour tout \( | x |>\delta\). 

    Avec \( n>\max\{ N_1,N_2 \}\) nous avons
    \begin{subequations}
        \begin{align}
            | \int_{B(0,\delta)}j_nu-\int_{\eR}j_nu |&=| \int_{B(0,\delta)^c}j_nu |\\
            &\leq \int_{\delta\leq| x |\leq M}| j_nu |+\int_{| x |\geq M}| j\nu |\\
            &\leq \epsilon(1+| M-\delta |).
        \end{align}
    \end{subequations}
    En redéfinissant le \( \epsilon\) nous avons donc montré que pour tout \( \epsilon\) et \( \delta\), il existe un \( N\in \eN\) tel que
    \begin{equation}
        | I_{\delta,n}-I_n |\leq \epsilon
    \end{equation}
    dès que \( n\geq N\).

    La fonction \( u\) est uniformément continue sur tout \( \overline{ B(0,\delta) }\), et nous pouvons donc choisir \( \delta\) tel que \( | u(0)-u(x) |\leq \epsilon\) pour tout \( x\in \overline{ B(0,\delta) }\). Pour ce \( \delta\), nous avons déjà trouvé un \( N\) tel que \( | I_{\delta,n}-I_n |\leq \epsilon\) dès que \( n>N\). Nous avons :
    \begin{subequations}
        \begin{align}
            | I_{\delta,n}-Z_{\delta,n} |&\leq \int_{B(0,\delta)}| u(x)-u(0) |j_n(x)dx\\
            &\leq \epsilon\int_{B(0,\delta)}j_n\\
            &\leq \epsilon.
        \end{align}
    \end{subequations}
    Nous avons donc prouvé que pour tout \( \epsilon>0\), il existe un \( \delta\) et un \( N\) tels que
    \begin{subequations}
        \begin{numcases}{}
            | I_{\delta,n}-I_n |\leq \epsilon\\
            | I_{\delta,n}-Z_{\delta,n} |\leq \epsilon
        \end{numcases}
    \end{subequations}
    dès que \( n\geq N\).

    Enfin nous avons
    \begin{equation}
        | z_{\delta,n}-u(0) |=u(0)\left( \int_{B(0,\delta)}j_n-1 \right),
    \end{equation}
    et par l'hypothèse \ref{ITEMooFYCRooFeRRjE} nous pouvons choisir \( n\) assez grand pour que la parenthèse soit plus petite que~\( \epsilon\). 

    Pour \( \epsilon\) donné, nous avons donc trouvé un \( \delta\) et un \( N\) tels que
    \begin{equation}
        | I_n-u(0) |\leq | I_n-I_{\delta,n} |+| I_{\delta,n}-Z_{\delta,n} |+| Z_{\delta,n}-u(0) |\leq 3\epsilon.
    \end{equation}
    En passant à la limite nous avons bien \( I_n\to u(0)\) dans \( \eR\).
\end{proof}

Il va sans dire que nous connaissons de telles fonctions. Nous en donnons une maintenant.

\begin{example}[\cite{ooKWQPooPhWNkI}]
    Nous introduisons la fonction \( f_{\epsilon}\) (\( \epsilon>0\)) donnée par
    \begin{equation}
        f_{\epsilon}(x)= e^{-\epsilon| x |}.
    \end{equation}
    Nous calculons la transformée de Fourier de \( f_{\epsilon}\) en divisant le domaine d'intégration :
    \begin{equation}
        \hat f_{\epsilon}(k)=\int_{\eR} e^{-ikx} e^{-\epsilon| x |}dx=\int_{-\infty}^0 e^{\epsilon x} e^{-ikx}dx+\int_{0}^{\infty} e^{-\epsilon x}\ e^{-ikx}dx
    \end{equation}
    En décomposant les parties imaginaires et réelles, et avec un peu de changement de variables, nous pouvons utiliser les intégrales \eqref{EQooNCVIooWqbbrH} et \eqref{EQooSAYUooSatbGc} pour obtenir
    \begin{equation}
        \hat f_{\epsilon}(k)=\frac{ 2\epsilon }{ k^2+\epsilon^2 }.
    \end{equation}
    Sachant que \( \arctg(x)\) est une primitive de \( \frac{1}{ x^2+1 }\) et avec encore un peu de changement de variables, nous avons\footnote{Et en écrivant correctement l'intégrale sur \( \eR\) comme une limite, etc.}
    \begin{equation}
        \int_{\eR}\hat f_{\epsilon}(k)dk=\int_{-\infty}^{\infty}\frac{ 2\epsilon }{ k^2+\epsilon^2 }=2[\arctg(x/\epsilon)]_{-\infty}^{\infty}=2\pi.
    \end{equation}
    Cela montre que si nous introduisons la fonction \( \delta_{\epsilon}\) donnée par
    \begin{equation}
        \delta_{\epsilon}(k)=\frac{1}{ \pi }\frac{ \epsilon }{ \epsilon^2+k^2 },
    \end{equation}
    alors nous avons une fonction qui tout en même temps ressemble à \( \hat f_{\epsilon}\) et vérifie
    \begin{equation}
        \int_{\eR}\delta_{\epsilon}(k)dk=1
    \end{equation}
    pour tout \( \epsilon\).

    Jusqu'ici nous avons montré que
    \begin{equation}        \label{EQooQOOOooBnfJNi}
        \int_{\eR} e^{-ikx} e^{-\epsilon| x |}dx=2\pi \delta_{\epsilon}(k).
    \end{equation}
    Pour chaque \( \epsilon>0\) nous avons \( \delta_{\epsilon}\in L^1(\eR)\).
\end{example}

\begin{proposition}[\cite{MonCerveau}]
    Soit \( g\in\swS(\eR)\). Alors nous avons
    \begin{equation}        \label{EQooBOJUooQGvMrk}
        \int_{\eR}\int_{\eR} g(x) e^{-ixy}dx\,dy=2\pi g(0).
    \end{equation}
\end{proposition}

\begin{proof}
    Soit \( u\in\swS(\eR)\); nous multiplions l'équation \eqref{EQooQOOOooBnfJNi} par \( u(k)\) et nous intégrons par rapport à \( k\) :
    \begin{equation}        \label{EQooTTQQooKxhxzl}
        \int_{\eR}u(k)\left[ \int_{\eR} e^{-ikx} e^{-\epsilon| x |}dx \right]dk=2\pi\int_{\eR}u(k)\delta_{\epsilon}(k)dk.
    \end{equation}
    Il s'agit de passer à la limite dans l'équation \eqref{EQooTTQQooKxhxzl}. Les intégrales à gauche peuvent effectuées séparément parce qu'elles respectent le théorème de Fubini. En effet soit la fonction
    \begin{equation}
        f(k,x)=u(k) e^{-ikx} e^{-\epsilon| x |}
    \end{equation}
    qui est dans \( L^1(\eR\times \eR)\) en vertu du critère du corollaire \ref{CorTKZKwP} et du fait que à la fois \( k\mapsto | u(k) | \) et \( x\mapsto  e^{-\epsilon| x |}\) sont dans \( L^1(\eR)\).

    Nous pouvons donc grouper et dégrouper les intégrales et en particulier les inverser. Si nous effectuons d'abord l'intégrale sur \( k\) nous trouvons
    \begin{equation}
        \int_{\eR}u(k)\left[ \int_{\eR} e^{-ikx} e^{-\epsilon| x |}dx \right]dk=\int_{\eR} e^{-\epsilon| x |}\int_{\eR}u(k) e^{-ikx}dk\,dx=\int_{\eR} e^{-\epsilon| x |}\hat u(x)dx.
    \end{equation}
    La fonction \( x\mapsto |  e^{-\epsilon| x |}\hat u(x) |\) est majorée (uniformément en \( \epsilon\)) par \( x\mapsto \hat u(x)\) qui est intégrable parce que la transformée de Fourier d'une fonction de \( \swS\) est dans \( \swS\) par la proposition \ref{PropKPsjyzT}. Le théorème de la convergence dominée de Lebesque \ref{ThoConvDomLebVdhsTf} nous permet de permuter la limite \( \epsilon\to 0 \) avec l'intégrale et obtenir
    \begin{equation}
        \lim_{\epsilon\to 0}\int_{\eR}u(k)\int_{\eR} e^{-ikx} e^{-\epsilon| x |}dx\,dk=\int_{\eR}\hat u(x)dx=\int_{\eR}\int_{\eR}u(k) e^{-ikx}dk\,dx.
    \end{equation}
    Notons qu'en passant à la limite nous avons perdu le droit de permuter les intégrales.

    Nous devons encore prouver que
    \begin{equation}
        \lim_{\epsilon\to 0}\int_{\eR}u(k)\delta_{\epsilon}(k)dk=u(0).
    \end{equation}
    Cela n'est rien d'autre que le lemme \ref{LEMooHEEOooFtKgfz} appliqué à la suite de fonctions \( j_n=\delta_{1/n}\).
\end{proof}

\begin{normaltext}
    Notons que les intégrales dans \eqref{EQooBOJUooQGvMrk} ne peuvent pas être permutées parce que \( \int_{\eR} e^{-ixy}dy\) n'existe pas. Il faut avouer que, malgré tous les conseils du type «attention : permuter des intégrales doit être fait avec prudence», ce n'est pas tous les jours que nous trouvons des intégrales qui ne peuvent pas être permutées, autrement que dans des exemples fait exprès.
\end{normaltext}

%--------------------------------------------------------------------------------------------------------------------------- 
\subsection{Peigne de Dirac}
%---------------------------------------------------------------------------------------------------------------------------

\begin{proposition}
    La formule
    \begin{equation}    \label{EqMEVmKvg}
        \Delta_a=\sum_{k\in \eZ}\delta_{ka}
    \end{equation}
    définit un élément de \( \swD'(\eR)\).
\end{proposition}
La forme linéaire \( \Delta_a\) est le \defe{peigne de Dirac}{peigne de Dirac} de pas \( a\).

\begin{proof}
    Nous utilisons le critère de continuité séquentielle en zéro du théorème \ref{ThoVDDBnVn}. Soit une suite \( \varphi_n\to 0\) dans \( \swD(\eR)\). Par le théorème \ref{ThoXYADBZr} il existe un compact \( K\) de \( \eR\) pour lequel \( \varphi_n\in\swD(K)\) pour tout \( n\) et \( \varphi_n\to0\) dans \( \swD(K)\). La somme \ref{EqMEVmKvg} est donc finie et nous pouvons la permuter avec une limite :
    \begin{equation}
        \lim_{n\to \infty} \Delta_a(\varphi_n)=\sum_{k\in\eZ}\lim_{n\to \infty} \varphi_n(ka).
    \end{equation}
    La limite \( \varphi_n\to 0\) dans \( \swD(K)\) signifie que nous avons convergence uniforme de la fonction et de toutes ses dérivées vers \( 0\). En particulier \( \| \varphi_n \|_{\infty}\to 0\); disons que la somme (qui est finie) fasse \( s\) termes :
    \begin{equation}
        \sum_{k\in \eZ}\varphi_n(ka)\leq s\| \varphi_n \|_{\infty}.
    \end{equation}
    Le terme de droite tend vers zéro lorsque \( n\) tend vers l'infini.
\end{proof}
Donc \( \Delta_a\) est bien une distribution au sens de la définition \ref{DefPZDtWVP}.

\begin{lemma}[\cite{CXCQJIt}]
    Le peigne de Dirac vérifie la relation
    \begin{equation}
        \Delta_a=\frac{1}{ a }\Delta_1\circ D_a
    \end{equation}
    où \( D_a\) est l'application \( D_a\colon \swD(\eR)\to \swD(\eR)\),
    \begin{equation}
        (D_af)(x)=af(ax).
    \end{equation}
\end{lemma}

\begin{proof}
    Pour \( \varphi\in\swD(\eR)\) nous avons
    \begin{equation}
        \Delta_a(\varphi)=\sum_{k\in \eZ}\varphi(ka)=\frac{1}{ a }\sum_{k\in \eZ}(D_a\varphi)(k)=\frac{1}{ a }\Delta_1(D_a\varphi).
    \end{equation}
\end{proof}

\begin{proposition}
    Le peigne de Dirac est une distribution tempérée.
\end{proposition}

Notez qu'il y a plus de fonctions dans \( \swS(\eR)\) que dans \( \swD(\eR)\); il est donc plus difficile de rentrer dans \( \swS'(\eR)\) que dans \( \swD'(\eR)\) : il est plus compliqué d'avoir existence de \( T(\varphi)\) pour tout \( \varphi\in\swS(\eR)\) que pour tout \( \varphi\in\swD(\eR)\).

\begin{proof}
    Soit \( \varphi\in\swS(\eR)\). Nous avons
    \begin{equation}
        |\Delta_a(\varphi)|=| \sum_k\varphi(ak) |=\left| \sum_k\frac{ (1+a^2k^2)\varphi(ak) }{ 1+a^2k^2 } \right| \leq_{x\in \eR}\big| (1+x^2)\varphi(x) \big|\sum_k\frac{1}{ 1+a^2k^2 }.
    \end{equation}
    La somme \( \sum_k\frac{1}{ 1+a^2k^2 }\) est une somme convergente, et et supremum est borné par la proposition \ref{PropCSmzwGv} en prenant \( Q(x)=1+x^2\). En effet sur \( \overline{ B(0,r) }\) la fonction \( x\mapsto (1+x^2)\varphi(x)\) est bornée par ce que c'est une fonction continue sur un compact, et à l'extérieur de \( B(0,r)\) cette fonction est alors bornée par \( 1\).
\end{proof}

Si aucune ambigüité n'est à craindre, nous noterons \( f\) la distribution \( T_f\).

\begin{example}
    La transformée de Fourier de la distribution de Dirac est la fonction constante : \( \hat \delta=1\). En effet si nous agissons sur une fonction test,
    \begin{equation}
        \hat \delta(\varphi)=\delta(\hat \varphi)=\hat \varphi(0)=\int_{\eR^d}\varphi(x)dx.
    \end{equation}
\end{example}

%+++++++++++++++++++++++++++++++++++++++++++++++++++++++++++++++++++++++++++++++++++++++++++++++++++++++++++++++++++++++++++ 
\section{L'espace \texorpdfstring{$  C^{\infty}(\eR,\swD'(\eR^d))$}{C(R,D')}}
%+++++++++++++++++++++++++++++++++++++++++++++++++++++++++++++++++++++++++++++++++++++++++++++++++++++++++++++++++++++++++++
\label{SecTEgDVWO}

D'abord parlons un peu de continuité en recopiant la proposition \ref{PropVKSNflB} dans notre contexte.
\begin{proposition}     \label{PropIPlKQBa}
    Soit \( I\) un intervalle ouvert de \( \eR\) et une fonction continue \( u\colon I\to \swD'(\eR^d)\). Alors
    \begin{enumerate}
        \item   \label{ItemYAhnNhBi}
            Pour tout \( \varphi\in\swD(\eR^d)\), l'application \( t\mapsto u_t(\varphi)\) est continue.
        \item
            Pour tout \( \varphi\in\swD(\eR^d)\), nous avons la limite
            \begin{equation}
                \lim_{t\to t_0} u_t(\varphi)=u_{t_0}(\varphi).
            \end{equation}
        \item
            Nous avons la limite dans \( \swD'(\eR^d)\)
            \begin{equation}
                \lim_{t\to t_0} u_t=u_{t_0}.
            \end{equation}
    \end{enumerate}
\end{proposition}
En ce qui concerne la définition de l'espace \( C^{\infty}(I,\swD'(\eR^d))\)\nomenclature[Y]{\(  C^{\infty}(I,\swD'(\eR^d))\)}{fonctions à valeurs dans les distributions}, c'est la définition \ref{DefDZsypWu}. Grâce au point \ref{ItemYAhnNhBi} de la proposition \ref{PropIPlKQBa}, nous retenons que la propriété fondamentale d'une application \( T\in C^k\big( I,\swD'(\Omega) \big)\) est que pour tout \( \varphi\in(\Omega)\), l'application
\begin{equation}
    \begin{aligned}
         I&\to \eC \\
        t&\mapsto T_t(\varphi) 
    \end{aligned}
\end{equation}
est de classe \( C^k\).

\begin{proposition} \label{PropOTlWzog}
    Soit \( (T_t)\in C^0\big( I,\swD'(\Omega) \big)\) et \( \psi\in\swD(I\times \Omega)\). Alors l'application 
    \begin{equation}
        t\mapsto T_t\big( \psi(t,.) \big)
    \end{equation}
    est continue sur \( I\).
\end{proposition}

\begin{proof}
    La fonction dont nous voulons prouver la continuité est une fonction \( \eR\to\eR\); il est donc loisible de se contenter de la continuité séquentielle. Soit \( t_0\in I\) et \( (t_j)\) une suite dans \( I\) convergeant vers \( t_0\). Nous posons \( U_j=T_{t_j}\) et \( \psi_j=\psi\big( t_j,. \big)\). Par hypothèse de continuité de \( (T_t)\) nous avons \( U_j\stackrel{\swD'(\Omega)}{\longrightarrow}T_{t_0}\). D'autre part le support de \( \psi\) étant compact nous avons \( \supp(\psi)\subset \mathopen[ c , d \mathclose]\times K\) où \( \mathopen[ c , d \mathclose]\subset I\) et \( K\) est compact dans \( \Omega\). Par conséquent nous avons aussi \( \supp(\psi_j)\subset K\).

    Affin d'alléger les notations notons \( \tilde \psi(x)=\psi(t_0,x)\). Pour tout multiindice \( \alpha\) et pour tout \( j\) nous avons
    \begin{equation}
        p_{\alpha}(\psi_i-\tilde \psi)=\Big|  \partial^{\alpha}\psi(t_j,x)-\partial^{\alpha}\psi(t_0,x)    \Big|\leq | t_j-t_0 |\sup_{\substack{t\in\mathopen[ c , d \mathclose]\\x\in K}}| \partial_t\partial^{\alpha}\psi(t,x) |\to 0.
    \end{equation}
    Nous avons donc la convergence
    \begin{equation}
        \psi_j\stackrel{\swD(K)}{\longrightarrow}\psi(t_0,.).
    \end{equation}
    
    Étant donné que \( U_j\stackrel{\swD'(\Omega)}{\longrightarrow}T_{t_0}\) et \( \psi_j\stackrel{\swD(\Omega)}{\longrightarrow}\tilde \psi\), le point \ref{ItemAEOtOMLiii} du corollaire \ref{CorPGwLluz} nous donne la convergence
    \begin{equation}
        U_j(\psi_j)\to T_{t_0}(\tilde \psi)
    \end{equation}
    dans \( \eC\). Cela est bien la continuité de la fonction \( t\mapsto T_t\big( \psi(t,.) \big)\).
\end{proof}


\begin{proposition}[\cite{GQYneyj}] \label{PropLKtBsVi}
    Soit \( (T_t)\in C^0\big( I,\swD'(\Omega) \big)\). Nous définissons l'application \( T\colon \swD(\Omega)\to \eC\) par la formule
    \begin{equation}
       T(\psi)=\int_I T_t\big( \psi(t,.) \big)\,dt
    \end{equation} 
    pour tout \( \psi\in\swD(I\times \Omega)\). Alors \( T\in \swD'(I\times \Omega)\).
\end{proposition}

\begin{proof}
    La proposition \ref{PropOTlWzog} nous indique que la fonction \( t\mapsto T_t\big( \psi(t,.) \big)\) est continue. Étant donné qu'elle est seulement non nulle sur un compact, l'intégrale
    \begin{equation}
        \int_IT_t\big( \psi(t,.) \big)dt
    \end{equation}
    a un sens et est finie. L'application \( T\colon \swD(I\times \Omega)\to \eC\) ainsi définie est linéaire. Il reste à voir qu'elle est continue. Pour cela nous allons utiliser le théorème \ref{ThoVDDBnVn}\ref{ItemSPvoijoii} qui nous dit que nous pouvons nous fixer un compact \( \mathopen[ c , d \mathclose]\times K\subset I\times\Omega\) et considérer \( \psi\in \swD\big( \mathopen[ c , d \mathclose]\times K \big)\).
    
    Soit, pour commencer, donnée une application \( \varphi\in\swD(K)\). L'application \( t\mapsto T_t(\varphi)\) est continue et non nulle sur le  et il existe donc \( C_{\varphi}>0\) tel que
    \begin{equation}
        | T_t(\varphi) |\leq C_{\varphi}
    \end{equation}
    pour tout \( t\in\mathopen[ c , d \mathclose]\). 
    
    Nous voulons utiliser le théorème de Banach-Steinhaus dans sa version \ref{ThoNBrmGIg} sur la famille d'applications paramétrée par \( u\in\mathopen[ c , d \mathclose]\) :
    \begin{equation}        \label{EqBEKoqMb}
        \begin{aligned}
            U_u\colon \swD\big( \mathopen[ c , d \mathclose]\times K \big)&\to \eR \\
            \psi&\mapsto T_u\big( \psi(u,.) \big). 
        \end{aligned}
    \end{equation}
    Commençons par prouver que cela est une application continue pour chaque \( u\). Ce sera le cas si la projection
    \begin{equation}
        \begin{aligned}
            \pr\colon \swD\big( \mathopen[ c , d \mathclose]\times K \big)&\to \swD(K) \\
            \psi&\mapsto \psi(u,.) 
        \end{aligned}
    \end{equation}
    est continue. Pour cela nous notons \( P_{kl}\) la semi-norme sur \( \swD\big( \mathopen[ c , d \mathclose]\times K \big)\) donnée par
    \begin{equation}
        P_{k,l}(\psi)=\sum_{n\leq k}\sum_{| \alpha |\leq l}\sup_{\substack{t\in\mathopen[ c , d \mathclose]\\x\in K}}\big| \partial_t^n\partial^{\alpha}\psi(t,x) \big|.
    \end{equation}
    Nous montrons que \( \pr\) est séquentiellement continue; étant donné que les topologies sur \( \swD(K)\) et \( \swD\big( \mathopen[ c , d \mathclose]\times K \big)\) sont données par des métriques (proposition \ref{PropQAEVcTi}), cela suffit pour assurer la continuité grâce à la proposition \ref{PropXIAQSXr}. Montrons que si \( \psi_n\stackrel{\swD\big( \mathopen[ c , d \mathclose]\times K \big)}{\longrightarrow}0\), alors \( \pr(\psi_n)\stackrel{\swD(K)}{\longrightarrow}0\). Pour cela nous remarquons que
    \begin{subequations}
        \begin{align}
            p_j\big( \pr(\psi) \big)&=\sum_{| \alpha |\leq j}\sup_{x\in K}| \partial^{\alpha}\psi(u,x) |\\
            &\leq \sum_{| \alpha |\leq j}\sup_{t\in\mathopen[ c , d \mathclose]}\sup_{x\in K}| \partial^{\alpha}\psi(t,x) |\\
            &=P_{0,j}(\psi).
        \end{align}
    \end{subequations}
    Par conséquent
    \begin{equation}
        p_j\big( \pr(\psi_n) \big)\leq P_{0,j}(\psi_n)\to 0
    \end{equation}
    où nous avons utilisé la proposition \ref{PropQPzGKVk}. Utilisant cette même proposition à l'envers, nous déduisons que \( \pr(\psi_n)\stackrel{\swD(K)}{\longrightarrow}0\). Les applications \( U_u\) sont donc continues; elles sont également bornées parce que si \( \psi\in\swD\big( \mathopen[ c , d \mathclose]\times K \big) \) nous avons
    \begin{equation}
        \sup_{u \in\mathopen[ c , d \mathclose]}\big| U_u(\psi) \big|=\sup_u \big| T_u\big( \psi(u,.) \big) \big|,    
    \end{equation}
    et la continuité déjà évoquée, sur le compact \( \mathopen[ c , d \mathclose]\), nous dit que cette quantité est finie. Le théorème de Banach-Steinhaus peut maintenant être appliqué et il existe \( C>0\) et \( k,l\in \eN\) tels que pour tout \( \psi\in\swD\big( \mathopen[ c , d \mathclose]\times K \big)\),
    \begin{equation}
        \big| U_u(\psi) \big|\leq C P_{k,l}(\psi)=C\sum_{| \alpha |\leq k}\sum_{n\leq l}\sup_{t,x}\big| \partial_t^n\partial^{\alpha}\psi(t,x) \big|\leq C\sum_{| \alpha |+n\leq k+l}\sup_{t,x}\big| \partial_t^n\partial^{\alpha}\psi(t,x) \big|.
    \end{equation}
    Quelque remarques
    \begin{itemize}
        \item Nous n'avons pas mis de maximum devant le supremum (alors que la conclusion \eqref{EqIFNGhtr} en demande) parce que dans le cas des semi-normes \( P_{kl}\), c'est toujours celle avec \( k\) et \( l\) le plus grand possible qui sont les plus grandes parce qu'elles sont des sommes emboitées les unes les autres.
        \item La fusion de deux sommes est bien une majoration parce qu'il y a plus de termes dans la seconde que dans la première.
        \item La quantité la plus à droite est (à part le \( C\)) ce que nous pouvons noter \( P_{k+l}(\psi)\) : c'est bien une des semi-normes associées à l'espace de dimension \( d+1\).
    \end{itemize}
    Nous majorons maintenant \( T(\psi)\) par
    \begin{equation}
            \big| T(\psi) \big|\leq \int_c^d\big| T_t\big( \psi(t,.) \big) \big|dt
            =\int_c^d\big|   U_t(\psi) \big|dt
            \leq C| d-c |P_{k+l}(\psi).
    \end{equation}
    Maintenant le théorème \ref{ThoVDDBnVn}\ref{ItemSPvoijoii} appliqué à l'ouvert \( I\times \Omega\) et avec \( \psi\) au lieu de \( \varphi\) nous informe que \( T\in\swD(I\times K)\). 
\end{proof}

%--------------------------------------------------------------------------------------------------------------------------- 
\subsection{Dérivation}
%---------------------------------------------------------------------------------------------------------------------------

Quelques propriétés de dérivation des fonctions \( I\to \swD(\Omega)\) seront directement énoncées et démontrées dans le cas des distributions tempérées. Les résultats \ref{PropGKoPbko} et \ref{PropUDkgksG} seront a fortiori valables si nous remplaçons \( \swS\) par \( \swD\).

%+++++++++++++++++++++++++++++++++++++++++++++++++++++++++++++++++++++++++++++++++++++++++++++++++++++++++++++++++++++++++++ 
\section{L'espace \texorpdfstring{$  C^{\infty}(\eR,\swS'(\eR^d))$}{C(R,S')}}
%+++++++++++++++++++++++++++++++++++++++++++++++++++++++++++++++++++++++++++++++++++++++++++++++++++++++++++++++++++++++++++

Dans cette section nous notons \( I\) un ouvert de \( \eR\) et \( \Omega\) un ouvert de \( \eR^d\); si \( \psi\) est une fonction sur \( I\times \Omega\) nous allons noter \( \psi_t\colon \Omega\to \eR\) la fonction \( \psi_t(x)=\psi(t,x)\). C'est une notation plus légère que \( \psi(t,.)\).

%--------------------------------------------------------------------------------------------------------------------------- 
\subsection{Propriétés générales}
%---------------------------------------------------------------------------------------------------------------------------

La définition de l'espace \(  C^{\infty}(I,\swS'(\Omega))\) est encore la définition \ref{DefDZsypWu} et les propriétés énoncées dans la proposition \ref{PropIPlKQBa} sont encore bonnes ici.

D'abord parlons un peu de continuité en recopiant la proposition \ref{PropVKSNflB} dans notre contexte.
\begin{proposition}     \label{PropBXFmvPs}
    Soit \( I\) un intervalle ouvert de \( \eR\) et une fonction continue \( u\colon I\to \swS'(\eR^d)\). Alors
    \begin{enumerate}
        \item   \label{ItemFTvVUEW}
            Pour tout \( \varphi\in\swS(\eR^d)\), l'application \( t\mapsto u_t(\varphi)\) est continue.
        \item
            Pour tout \( \varphi\in\swS(\eR^d)\), nous avons la limite
            \begin{equation}
                \lim_{t\to t_0} u_t(\varphi)=u_{t_0}(\varphi).
            \end{equation}
        \item
            Nous avons la limite dans \( \swS'(\eR^d)\)
            \begin{equation}
                \lim_{t\to t_0} u_t=u_{t_0}.
            \end{equation}
    \end{enumerate}
\end{proposition}

\begin{lemma}
    Nous avons \(  C^{\infty}(I,\swS'(\Omega))\subset C^{\infty}(I,\swD'(\Omega))\).
\end{lemma}

\begin{proof}
    Soit \( (T_t)\in C^{\infty}(I,\swS'(\Omega))\). Pour chaque \( t\) nous avons
    \begin{equation}
        T_t\in\swS'(\Omega)\subset\swD'(\Omega).
    \end{equation}
    Ensuite il suffit de dire que pour tout \( \varphi\in\swD(\Omega)\) la fonction
    \begin{equation}
        t\mapsto T_t(\varphi)
    \end{equation}
    est de classe \(  C^{\infty}\) parce que c'est le cas pour toute fonction dans \( \swS(\Omega)\). La proposition \ref{PropIPlKQBa} (en changeant \( \swD\) en \( \swS\)) conclut que \( (T_t)\in C^{\infty}(I,\swD'(\Omega))\).
\end{proof}

\begin{proposition} \label{PropIIAcyDq}
    L'espace \( \swS(\Omega)\) est complet et métrisable.
\end{proposition}
\index{espace!complet!\( \swS(\Omega)\)}

\begin{proof}
    En ce qui concerne le métrisable nous reprenons la formule de l'écart \eqref{EqAAghiUR}. Dans notre cas pour l'écrire explicitement il faudrait une énumération de \( \eN^2\) à partir de \( 1\) (et non de zéro). Cette formule donne bien une distance parce que si \( d(\varphi_1-\varphi_2)=0\) alors en particulier \( p_{00}(\varphi_1-\varphi_2)=\| \varphi_1-\varphi_2 \|_{\infty}=0\) et donc \( \varphi_1=\varphi_2\). 

    Nous montrons maintenant que \( \swS(\Omega)\) est complet en y considérant une suite de Cauchy \( (\varphi_n)\). Soit \( \epsilon>0\) et \( \alpha,\beta\in \eN\) ainsi que \( k,l\) assez grands pour que \( \varphi_k-\varphi_l\in B_{\alpha\beta}(0,\epsilon)\). En particulier pour \( \alpha=\beta=0\) nous avons \( \| \varphi_k-\varphi_l \|_{\infty}\leq \epsilon\), ce qui signifie que nous avons une suite vérifiant le critère de Cauchy uniforme \ref{PropNTEynwq}. Elle converge donc uniformément vers une certaine fonction \( \varphi\) que la proposition \ref{ThoUnigCvCont} nous assure être continue. Il existe donc \( \varphi\in C(\Omega)\) telle que
    \begin{equation}
        \varphi_k\stackrel{unif}{\longrightarrow}\varphi.
    \end{equation}
    Nous devons montrer que \( \varphi\in\swS(\Omega)\). Le fait que \( \varphi\) soit de classe \(  C^{\infty}\) s'obtient en utilisant les semi-normes \( p_{0,\alpha}(\varphi)=\| \partial^{\alpha}\varphi \|_{\infty}\) de la même façon que dans la preuve que \( \swD(\Omega)\) était complet (proposition \ref{PropQAEVcTi}). Nous obtenons en particulier que
    \begin{equation}    \label{EqSZyYkqk}
        \partial^{\alpha}\varphi_k\stackrel{unif}{\longrightarrow}\partial^{\alpha}\varphi
    \end{equation}
    pour tout multiindice \( \alpha\). Montrons encore que \( \varphi\) est à décroissance rapide : nous devons montrer que pour tout \( \alpha\) et \( \beta\) nous avons
    \begin{equation}
        p_{\alpha\beta}(\varphi)=\sup_{x\in \Omega}\big| x^{\beta}(\partial^{\alpha}\varphi)(x) \big|<\infty.
    \end{equation}
    Étant donné que \( (\varphi_n)\) est de Cauchy dans \( \swS(\Omega)\) nous avons (pour \( \epsilon\) fixé et \( k,l\) assez grands) :
    \begin{equation}
        \big| x^{\beta}(\partial^{\alpha}\varphi_k-\partial^{\alpha}\varphi_l)(x) \big|\leq \epsilon
    \end{equation}
    pour tout \( x\in\Omega\). En considérant \( l\) fixé et en prenant la limite \( k\to \infty\) et en utilisant la convergence uniforme \eqref{EqSZyYkqk} nous trouvons que
    \begin{equation}
        \big| x^{\beta}(\partial^{\alpha}\varphi-\partial^{\alpha}\varphi_l)(x) \big|\leq \epsilon
    \end{equation}
    Du coup nous pouvons faire la majoration
    \begin{equation}
        \sup_{x\in\Omega}\big| x^{\beta}(\partial^{\alpha\varphi})(x) \big|\leq\sup_x\big| x^{\beta}(\partial^{\alpha}\varphi-\partial^{\alpha}\varphi_l)(x) \big|+\sup_x\big| (\partial^{\alpha}\varphi_l)(x) \big|\leq\epsilon+p_{\alpha\beta}(\varphi_l)<\infty
    \end{equation}
    du fait que \( p_{\alpha\beta}(\varphi_l)<\infty\) parce que \( \varphi_l\in\swS(\Omega)\).

    Donc \( \varphi\in\swS(\Omega)\) et ce dernier est alors complet.
\end{proof}

\begin{proposition}
    Soit \( (T_t)\in C^0\big( I,\swS'(\Omega) \big)\) et \( \psi\in\swS(I\times \Omega)\). Alors la fonction
    \begin{equation}    \label{EqULcaYjm}
        t\mapsto T_t(\psi_t)
    \end{equation}
    est continue sur \( I\).
\end{proposition}

\begin{proof}
    Soit \( t_0\in I\) et une suite convergente vers \( t_0\) : \( t_j\to t_0\) dans \( \eR\). Vu que \( (T_t)\) est continue en \( t\), elle est en particulier séquentiellement continue et nous avons
    \begin{equation}
        T_{t_j}\stackrel{\swS'(\Omega)}{\longrightarrow}T_{t_0}.
    \end{equation}
    Montrons que nous avons aussi \( \psi_{t_j}\stackrel{\swS(\Omega)}{\longrightarrow}\psi_{t_0}\). Pour cela nous utilisons les semi-normes\footnote{Pas parce que nous en avons envie, mais bien parce qu'elles font partie de la définition de la convergence et de tous ces trucs.} \( p_{\alpha\beta}\) définies en \eqref{EqOWdChCu} :
    \begin{subequations}
        \begin{align}
            p_{\alpha\beta}(\psi_{t_j}-\psi_{t_0})&=\sum_{x\in \Omega}\Big| x^{\beta}\big( \partial^{\alpha}\psi(t_j,x)-\partial^{\alpha}\psi(t_0,x) \big)       \Big|\\
            &\leq\sup_{x\in\Omega}\Big|  x^{\beta}| t_0-t_j |\sup_{t\in \mathopen[ t_0 , t_j \mathclose]}\big| \partial_t\partial^{\alpha}\psi(t,x) \big|   \Big|\\
            &\leq| t_0-t_j |\sup_{x\in\Omega}\sup_{t\in I}\big| x^{\beta}\partial_t\partial^{\alpha}\psi(t,x) \big|\\
            &\leq| t_0-t_j |P_{(\alpha t),\beta}(\psi).
        \end{align}
    \end{subequations}
    Pour la première majoration nous avons utilisé le théorème des accroissements finie \ref{val_medio_2}. Pour la dernière ligne nous avons noté \( P_{\alpha\beta}\) les semi-normes de \( \swS(I\times \Omega)\) et \( (t\alpha)\) est le multiiindice qui commence par la variable \( t\) et qui continue par \( \alpha\). Étant donné que \( P_{(\alpha t)\beta}(\psi)<\infty\) nous avons bien
    \begin{equation}
        p_{\alpha\beta}(\psi_{t_j}-\psi_{t_0})\to 0
    \end{equation}
    et donc \( \psi_{t_j}\stackrel{\swS(\Omega)}{\longrightarrow}\psi_{t_0}\).

    Étant donné que \( \swS(\Omega)\) est métrisable et complet, le corollaire \ref{CorPGwLluz} nous dit que 
    \begin{equation}
        T_{t_j}(\psi_{t_j})\to T_{t_0}(\psi_{t_0}),
    \end{equation}
    ce qui est bien le critère de continuité séquentielle de la fonction \eqref{EqULcaYjm}.
\end{proof}

\begin{remark}  \label{RemZYVkHRT}
    La proposition \ref{PropLKtBsVi} nous dit, a fortiori, que si \( (T_t)\in C^{\infty}\big( I,\swS'(\Omega) \big)\) alors la formule
    \begin{equation}
        \tilde T(\psi)=\int_IT_t(\psi_t)
    \end{equation}
    donne un élément \( \tilde T\in\swD'(I\times \Omega)\). Au cas où aucune confusion n'est à craindre, nous pourrons noter également \( T\) l'élément de \( \swD'(I\times \Omega)\) déduit de \( T\in C^{\infty}\big( I,\swS'(\Omega) \big)\).
\end{remark}

Notons que ce \( T\) ne sera pas toujours une distribution tempérée comme le montre l'exemple suivant.

\begin{example}
    En posant \( T_t(\varphi)= e^{t^2}\varphi(0)\) avec \( I=\eR\), l'intégrale
    \begin{equation}
        T(\psi)=\int_{\eR}T_t(\psi_t)=\int_{\eR} e^{t^2}\psi(t,0)dt
    \end{equation}
    ne converge pas pour tout \( \psi\in\swS(\eR\times \Omega)\). En effet par rapport à \( t\), la fonction \( \psi(t,0)\) décroît rapidement mais pas spécialement assez rapidement pour compenser \( e^{t^2}\).
\end{example}

%--------------------------------------------------------------------------------------------------------------------------- 
\subsection{Dérivation}
%---------------------------------------------------------------------------------------------------------------------------

\begin{proposition}[\cite{GQYneyj}] \label{PropGKoPbko}
    Soit \( T\in C^k\big( I,\swS'(\Omega) \big)\) et \( 0\leq l\leq k\). Pour tout \( t_0\in I\) l'application
    \begin{equation}    \label{EqZMDeZco}
        \begin{aligned}
            T_{t_0}^{(l)}\colon \swS(\Omega)&\to \eC \\
            \varphi&\mapsto \left(\frac{ d^l  }{ dt }T_t(\varphi)\right)(t_0)
        \end{aligned}
    \end{equation}
    est bien définie, est une distribution et de plus   
    \begin{equation}
        t\mapsto T_t^{(l)}\in C^{k-l}\big( I,\swS'(\Omega) \big).
    \end{equation}
\end{proposition}
Attention que la formule \eqref{EqZMDeZco} est bonne si \( \varphi\in\swS(\Omega)\). Si par contre \( \psi\in\swS(I\times \Omega)\) et qu'on veut regarder \( u_t^{(1)}(\psi_t)\) alors il faut regarder la proposition \ref{PropUDkgksG} et utiliser la formule \eqref{EqSCNYYhE} dans laquelle se trouve \( u_t^{(1)}(\psi_t)\).

\begin{proof}
    Pour \( k=0\) nous avons \( T^{(0)}_t=T_t\) et c'est bon. Pour le cas \( k=1\) et \( l=0\) c'est encore \( T_t^{(0)}=T_t\) qui fonctionne.

    Le premier cas non trivial à traiter est \( k=1\) et \( l=1\). Nous considérons \( t_0\in I\); par définition de la dérivée, pour tout \( \varphi\in\swS(\Omega)\), nous avons (pour peu que les limites existent) :
    \begin{equation}    \label{EqCTuAfXe}
        T^{(1)}_{t_0}(\varphi)=  \Dsdd{ T_t(\varphi) }{t}{t_0}=\lim_{j\to \infty} \frac{ T_{t_0+\epsilon_j}(\varphi)-T_{t_0}(\varphi) }{ \epsilon_j }= \lim_{j\to \infty} U_j(\varphi)
    \end{equation}
    où
    \begin{equation}
        U_j=\frac{1}{ \epsilon_j }\big( T_{t_0+\epsilon_j}-T_{t_0} \big).
    \end{equation}
    et \( (\epsilon_j)\) est une suite de réels tendant vers zéro.

    Vu que \( (T_t)\in C^k(I,\swS'(\Omega))\), l'application \( t\mapsto T_t(\varphi)\) est de classe \( C^k\) et en particulier l'expression \eqref{EqCTuAfXe} a une limite lorsque \( j\to \infty\). Donc \( T^{(1)}_{t_0}(\varphi)\) est bien définie. Le point \ref{ItemAEOtOMLi} du corollaire \ref{CorPGwLluz} nous dit que \( \lim_{j\to \infty} U_j(\varphi)= T_{t_0}^{(1)}(\varphi)\) et \( T_{t_0}^{(1)}\) est une distribution (linéaire et continue).

    Nous devons encore voir que \( t\mapsto T^{(1)}_t\) est une application \( C^0\big( I,\swS'(\Omega) \big)\). Cela est une conséquence du fait que \( (T_t)\) soit de classe \( C^1\), ce qui se traduit par le fait que l'application
    \begin{equation}
        t\mapsto \frac{ d }{ dt }\Big( T_t(\varphi) \Big)
    \end{equation}
    est continue (définition de la dérivée et point \ref{ItemFTvVUEW} de la proposition \ref{PropBXFmvPs} appliquée à la dérivée).

    Les cas \( k\geq 1\) se traitent par récurrence.
\end{proof}

\begin{proposition}[\cite{GQYneyj}] \label{PropUDkgksG}
    Soit \( (T_t)\in C^1\big( I,\swS'(\Omega) \big)\) et \( \psi\in\swS(I\times \Omega)\). Alors la fonction
    \begin{equation}
        t\mapsto T_t\big( \psi(t,.) \big)
    \end{equation}
    est de classe \( C^1\) sur \( I\) et
    \begin{equation}    \label{EqSCNYYhE}
        \frac{ d }{ dt }\Big( T_t\big( \psi(t,.) \big) \Big)=T_t^{(1)}\big( \psi(t,.) \big)+T_t\left( \frac{ \partial \psi }{ \partial t }(t,.) \right)
    \end{equation}
\end{proposition}

\begin{proof}
    Soit \( t_0\in I\) et une suite réelle \( \epsilon_j\to 0\). Le membre de gauche de \eqref{EqSCNYYhE}, écrit en \( t_0\), donne
    \begin{equation}    \label{BJPHzwn}
        \spadesuit=\lim_{j\to \infty} \frac{ T_{t_0+\epsilon_j}\big( \psi(t_0+\epsilon_j,.) \big)-T_{t_0}\big( \psi(t_0,.) \big) }{ \epsilon_j }
    \end{equation}
    Afin d'alléger les notations nous allons écrire \( \psi_t=\psi(t,.)\). Dans le numérateur de \eqref{BJPHzwn} nous ajoutons et soustrayons la quantité \( T_{t_0+\epsilon_j}(\psi_{t_0})\) et nous découpons la limite en deux morceaux :
    \begin{equation}
        \spadesuit=\lim_{j\to \infty} \frac{ T_{t_0+\epsilon_j}(\psi_{t_0+\epsilon_j}-\psi_{t_0}) }{ \epsilon_j }+\lim_{j\to \infty} \frac{ (T_{t_0+\epsilon_j}-T_{t_0})(\psi_{t_0}) }{ \epsilon_j }
    \end{equation}
    Le second terme vaut 
    \begin{equation}
        \frac{ d }{ dt }\Big( T_t(\psi_{t_0}) \Big)_{t=t_0}=T_{t_0}^{(1)}(\psi_{t_0})
    \end{equation}
    par la proposition \ref{PropGKoPbko}. Occupons nous de l'autre morceau de \( \spadesuit\). Nous posons \( U_j=T_{t_0+\epsilon_j}\) et 
    \begin{equation}
        \varphi_j=\frac{1}{ \epsilon_j }(\psi_{t_0+\epsilon_j}-\psi_{t_0}).
    \end{equation}
    Nous voulons utiliser le corollaire \ref{CorPGwLluz}\ref{ItemAEOtOMLiii} pour obtenir
    \begin{equation}
        \lim_{j\to \infty} U_j(\varphi_j)=T_{t_0}\Big( \frac{ \partial \psi }{ \partial t }(t_0,.) \Big).
    \end{equation}
    D'une part \( (T_t)\) est de classe \(  C^{\infty}\) en \( t\) et nous avons donc la convergence \( U_j\stackrel{\swS'(\Omega)}{\longrightarrow}T_{t_0}\). Reste à prouver que 
    \begin{equation}
        \varphi_j\stackrel{\swS(\Omega)}{\longrightarrow}\frac{ \partial \psi }{ \partial t }(t_0,.).
    \end{equation}
    Cela en remarquant bien que la variable de dérivation n'est pas celle par rapport à laquelle nous voulons la convergence Schwartz\footnote{Je ne sais pas si je me suis bien fait comprendre là.}. Soient \( \alpha\) et \( \beta\) des naturels et calculons un peu :
    \begin{equation}    \label{EqEBUYDRA}
            p_{\alpha\beta}\big( \varphi_j-\frac{ \partial \psi }{ \partial t }(t_0,.) \big)=\sup_{x\in\Omega}\left| x^{\beta}\partial^{\alpha}\Big( \frac{1}{ \epsilon_j }\big(\psi(t_0+\epsilon_j,x)-\psi(t_0,x)\big) -\frac{ \partial \psi }{ \partial t }(t_0,x) \Big)\right| 
    \end{equation}
    Il est à présent l'heure d'utiliser un développement de Taylor avec le reste de la proposition \ref{PropResteTaylorc} :
    \begin{equation}
        \psi(t_0+\epsilon_j,x)=\psi(t_0,x)+\epsilon_j\frac{ \partial \psi }{ \partial t }(t_0,x)+\frac{ \epsilon_j^2 }{2}\frac{ \partial^2\psi  }{ \partial t^2 }(\bar t,x)
    \end{equation}
    pour un certain \( \bar t\in\mathopen[ t_0 , t_0+\epsilon_j \mathclose]\). En mettant ça dans le calcul \eqref{EqEBUYDRA} nous restons avec
    \begin{equation}
        p_{\alpha\beta}\big( \varphi_j-\frac{ \partial \psi }{ \partial t }(t_0,.) \big)=\sup_{x\in\Omega}\left| x^{\beta}\partial^{\alpha}\Big( \epsilon_j\frac{ \partial^2\psi }{ \partial t^2 }(\bar t,x) \Big) \right| \leq \epsilon_j P_{\alpha,2;\beta,0}(\psi)
    \end{equation}
    où \( P_{\alpha,k;\beta,l}\) sont les semi-normes de \( \swS(I\times \Omega)\) avec la notation plus ou moins évidente de prendre \( \alpha\) dérivations sur \( x\), \( k\) sur \( t\) puis de multiplier par \( x^{\beta}t^l\). Au final nous avons bien 
    \begin{equation}
        \lim_{j\to \infty} p_{\alpha\beta}\big( \varphi_j-\frac{ \partial \psi }{ \partial t }(t_0,.) \big)=0
    \end{equation}
    et donc la convergence \( \varphi_j\stackrel{\swS(\Omega)}{\longrightarrow}\frac{ \partial \psi }{ \partial t }(t_0,.)\).
\end{proof}

\begin{lemma}   \label{LemWRoRPIX}
    Soit \( (T_t)\in C^1\big( I,\swS'(\Omega) \big)\) alors si \( \TF\) dénote la transformée de Fourier nous avons
    \begin{equation}
        \TF\big( T_t^{(1)} \big)=(\TF T)_t^{(1)}
    \end{equation}
    où \( (\TF T)\) est la famille de distributions \( (\TF T)_t=\TF T_t\).
\end{lemma}

\begin{proof}
    Pour la preuve il suffit de tester l'égalité sur une fonction \( \varphi\in\swS(\Omega)\) :
    \begin{equation}
        (\TF T_t^{(1)})(\varphi)=T_t^{(1)}(\TF \varphi)=\frac{ d }{ dt }\Big( T_t(\TF \varphi) \Big)=\frac{ d }{ dt }\Big( (\TF T_t)(\varphi) \Big)=(\TF T)_t^{(1)}(\varphi).
    \end{equation}
\end{proof}

%+++++++++++++++++++++++++++++++++++++++++++++++++++++++++++++++++++++++++++++++++++++++++++++++++++++++++++++++++++++++++++ 
\section{Une équation de distribution}
%+++++++++++++++++++++++++++++++++++++++++++++++++++++++++++++++++++++++++++++++++++++++++++++++++++++++++++++++++++++++++++

Nous allons étudier l'équation
\begin{equation}    \label{EqLLTPooJHUVvU}
    (x-x_0)^{\alpha}u=0
\end{equation}
pour \( u\in\swD'(\eR)\) et \( \alpha\in \eN\) est donné fixé. Notons tout de suite que \eqref{EqLLTPooJHUVvU} est un petit abus de notation pour dire qu'en vertu de la définition \ref{DefZVRNooDXAoTU} du produit d'une distribution par une fonction, pour tout \( \phi\in\swD(\eR)\), nous avons \( u\Big( x\mapsto (x-x_0)\phi(x) \Big)=0\).

\begin{lemma}[\cite{PAXrsMn}]       \label{LemWIGKooQpGXoI}
    Soit \( \alpha\in \eN\). Une solution à l'équation
    \begin{equation}        \label{EqKVNEooJNwsPc}
        (x-x_0)^{\alpha}u=0
    \end{equation}
    est une distribution à support dans \( \{ x_0 \}\) et d'ordre fini.
\end{lemma}

\begin{proof}
    Nous commençons par prouver que \( u\) est une solution de \eqref{EqKVNEooJNwsPc} si et seulement si\footnote{En réalité nous n'aurons besoin que de la condition nécessaire, en particulier pour le théorème \ref{ThoRDUXooQBlLNb}.} \( \langle u, \phi\rangle =0\) pour tout \( \phi\in\swD\) telle que 
    \begin{equation}    \label{EqYLIPooYByzwC}
        \phi(x_0)=\ldots=\partial^{\alpha-1}\phi(x_0)=0.
    \end{equation}
    \begin{subproof}
    \item[Condition nécéssaire]
    Supposons que \( u\) soit une solution. Alors le corollaire \ref{CorQBXHooZVKeNG} du théorème de Hadamard donne \( \psi\in\swD(\eR)\) telle que
        $\phi(x)=(x-x_0)^{\alpha}\psi(x)$.
    Dans ce cas, si \( u\) est solution de \eqref{EqKVNEooJNwsPc}, alors
    \begin{equation}
        0=\langle (x-x_0)^{\alpha}u, \psi\rangle =\langle u, (x-x_0)^{\alpha}\psi(x)\rangle =\langle u, \phi\rangle .
    \end{equation}
    Nous avons vu que si \( u\) est solution, alors \( \langle u, \phi\rangle =0\) pour tout \( \phi\) satisfaisant la condition \eqref{EqYLIPooYByzwC}.

\item[Condition suffisante]
    Supposons maintenant l'inverse : \( u\) est une distribution s'annulant sur toute fonction \( \phi\in\swD'\) satisfaisant \eqref{EqYLIPooYByzwC}. Nous allons alors prouver que \( u\) est une solution. Soit donc \( \psi\in \swD\) et calculons
    \begin{equation}
        \langle (x-x_0)u, \psi\rangle =\langle u, (x-x_0)\psi\rangle =0
    \end{equation}
    parce que la fonction \( (x-x_0)\psi(x)\) vérifie la condition \eqref{EqYLIPooYByzwC}.
    \end{subproof}

    Nous passons maintenant au cœur de la preuve : nous supposons que \( u\) est une solution. Si le support de \( \phi\) est contenu dans \( \eR\setminus\{ x_0 \}\) alors \( \phi\) est nulle dans un voisinage de \( x_0\) (et donc \( \partial^k\phi=0\) pour tout \( k\)) et \( \langle u, \phi\rangle =0\). Autrement dit, pour  tout \( \phi\in\swD\big( \eR\setminus\{ x_0 \} \big)\) nous avons \( \langle u, \phi\rangle =0\), ce qui signifie que \( \supp(u)\cap\big( \eR\setminus\{ x_0 \} \big)=\emptyset\) ou encore que \( \supp(u)=\{ x_0 \}\).

    Maintenant que \( u\) a un support compact, la proposition \ref{PropZLUEooHcVxQj} nous indique qu'elle est d'ordre fini.
\end{proof}

\begin{theorem}[\cite{PAXrsMn}]     \label{ThoRDUXooQBlLNb}
    Soit \( \alpha\in \eN\) et l'équation
    \begin{equation}        \label{EqDONTooKPfDWU}
        (x-x_0)^{\alpha}u=0
    \end{equation}
    pour \( u\in\swD'(\eR)\). Les solutions sont les combinaisons linéaires des dérivées de \( \delta_{x_0}\) jusqu'à la \( \alpha\)\ieme exclue.
\end{theorem}

\begin{proof}
    D'abord montrons que les \( \partial^i\delta_{x_0}\) sont des solutions. Avec les définition \ref{DefZVRNooDXAoTU} et \ref{PropKJLrfSX} des dérivées de distributions et de leur produits avec des fonctions\footnote{Comme souvent, dans l'expression suivante, il y a un abus de notation parce que \( x\) est une variable muette : il faudrait écrire «\( x\mapsto\)» au début de la grade parenthèse.},
    \begin{equation}
        (x-x_0)^{\alpha}\partial^i\delta_{x_0}(\phi)=\delta_{x_0}\Big( \partial^i\big( (x-x_0)^{\alpha}\phi(x) \big) \Big)
    \end{equation}
    Si \( i<\alpha\) alors dans chaque terme de Leibnitz, il y aura un facteur \( (x-x_0)\), et la prise de \( \delta_{x_0}\) annulera. Si par contre \( i\geq \alpha\) alors il y aura le terme
    \begin{equation}
        \binom{ i }{ \alpha }\partial^{\alpha}\big( (x-x_0)^{\alpha} \big)\partial^{i-\alpha}=\binom{ i }{ \alpha }\alpha!(\partial^{i-\alpha}\phi)(x_0)
    \end{equation}
    qui est le seul terme contenant \( (\partial^{i-\alpha}\phi)(x_0)\). Il suffit alors de choisir \( \phi\in\swD(\eR) \) de sorte que
    \begin{equation}
        (\partial^k\phi)(x_0)=\begin{cases}
            0    &   \text{si \( k\neq i-\alpha\)}\\
            1    &    \text{si \( k=i-\alpha\)}
        \end{cases}
    \end{equation}
    et alors on est certain que le tout n'est pas nul, et donc que \( (x-x_0)^{\alpha}(\partial^i\delta_{x_0})\neq 0\).

    Jusqu'ici nous avons prouvé que \( \partial^i\delta_{x_0}\) est solution si et seulement si \( 0\leq i<\alpha\).

    Il faut encore prouver que les solutions sont toutes des combinaisons linéaires de dérivées de delta de Dirac centrées en \( x_0\). Pour cela nous invoquons d'abord le lemme \ref{LemWIGKooQpGXoI} qui nous assure que \( u\) est d'ordre fini et de support \( \{ x_0 \}\). Ensuite la proposition \ref{PropXXPLooSkgxOz} nous indique que \( u\) doit alors être une combinaisons linéaire de dérivées de Dirac.
\end{proof}
