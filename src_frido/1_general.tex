% This is part of Analyse Starter CTU
% Copyright (c) 2014,2016
%   Laurent Claessens,Carlotta Donadello
% See the file fdl-1.3.txt for copying conditions.

%+++++++++++++++++++++++++++++++++++++++++++++++++++++++++++++++++++++++++++++++++++++++++++++++++++++++++++++++++++++++++++ 
\section{Quelque rappels}
%+++++++++++++++++++++++++++++++++++++++++++++++++++++++++++++++++++++++++++++++++++++++++++++++++++++++++++++++++++++++++++

\begin{definition}[Intervalle]
    Une partie \( I\) de \( \eR\) est un \defe{intervalle}{intervalle} si pour tout \( a,b\in I\) nous avons \( t\in I\) dès que \( a\leq t\leq b\).

    Un intervalle est \defe{ouvert}{intervalle!ouvert} s'il est de la forme \( \mathopen] a , b \mathclose[\) avec éventuellement \( a=-\infty\) ou \( b=+\infty\). Un intervalle est \defe{fermé}{intervalle!fermé} s'il est de la forme \( \mathopen[ a , b \mathclose]\) ou \( \mathopen] -\infty , b \mathclose]\) ou \( \mathopen[ a , +\infty [\) avec \( a,b\in \eR\).
\end{definition}

\begin{remark}
  L'ensemble $\eR$ ne contient pas $-\infty$ et $-\infty$. L'intervalle $[-\infty, 5]$ par exeple, n'est pas une partie de $\eR$.
\end{remark}

\begin{example}
    \begin{enumerate}
        \item
        Les ensembles \( \mathopen] 3 , 7 \mathclose[\) et \( \mathopen] -\infty , \pi \mathclose[\) sont des intervalles ouverts.
        \item
            Les ensembles \( \mathopen[ 10 , 15 \mathclose]\) et \( \mathopen[ -1 , +\infty [\) sont des intervalles fermés.
        \item
        L'ensemble \( \mathopen] -4 , -2 \mathclose[\cup\mathopen] 2 , 9 \mathclose[\) n'est pas un intervalle (il y a un «trou» entre \(- 2\) et \( 2\)).
        \item
            L'ensemble \( \eR\) lui-même est un intervalle; par convention, il est à la fois ouvert et fermé.
    \end{enumerate}
Un intervalle peut n'être ni ouvert ni fermé; par exemple \( \mathopen] 4 , 8 \mathclose]\). Cet intervalle est «ouvert en \( 4\) et fermé en \( 8\)» .
\end{example}

\begin{definition}[Fonction, domaine, image, graphe]
  Soient $X$ et $Y$ deux ensembles. Une \defe{fonction}{fonction} $f$ définie sur $X$ et à valeurs dans $Y$ est une correspondence qui associe à chaque élément $x$ dans $X$ {\bf au plus} un élément $y$ dans $Y$. On écrit $y= f(x)$.
  \begin{itemize}
  \item La partie de $X$ qui contient tous les $x$ sur lesquels $f$ peut opérer est dite \defe{domaine}{domaine} de $f$. Le domaine de $f$ est indiqué par $\Dom f$.
  \item L'élément de $y\in Y$ associé par $f$ à un élément $x\in \Dom f$ (c'est à dire $f(x) = y$)  est appellé \defe{image}{image} de $x$ par $f$. L'\defe{image}{fonction!image} de la fonction $f$ est la partie de $Y$ qui contient les images de tous les éléments de $\Dom f$. L'image de $f$ est indiquée par $\Im f$.
  \item Le \defe{graphe}{graphe} de $f$ est l'ensemble de toutes les couples $(x, f(x))$ pour $x\in \Dom f$. Le graphe de $f$ est une partie de l'ensemble noté $X\times Y$ et il est indiqué par $\Graph f$. Dans ce cours $X = \eR$ et $Y = \eR$, donc le graphe de $f$ est contenu dans le plan cartésien. 
  \end{itemize}
\end{definition}

\begin{definition}[Fonction croissante, décroissante et monotone]
    Soit une fonction \( f\colon \eR\to \eR\) et un intervalle \( I\subset \eR\).
    \begin{enumerate}
        \item
            Le fonction \( f\) est \defe{croissante}{fonction!croissante} sur \( I\) si pour tout \( x<y\) dans \( I\) nous avons \( f(x)\leq f(y)\). Elle est \emph{strictement} croissante si \( f(x)<f(y)\) dès que \( x<y\).
        \item
            Le fonction \( f\) est \defe{décroissante}{fonction!décroissante} sur \( I\) si pour tout \( x<y\) dans \( I\) nous avons \( f(x)\geq f(y)\). Elle est \emph{strictement} décroissante si \( f(x)>f(y)\) dès que \( x<y\).
        \item
            La fonction \( f\) est dite \defe{monotone}{fonction!monotone} sur \( I\) si elle est soit croissante soit décroissante sur \( I\).
    \end{enumerate}
\end{definition}

\begin{example}
    La fonction \( x\mapsto x^2\) est décroissante sur l'intervalle \( \mathopen] -\infty , 0 \mathclose]\) et croissante sur l'intervalle \( \mathopen[ 0 , \infty \mathclose[\). Elle n'est par contre ni croissante ni décroissante sur l'intervalle \( \mathopen[ -4 , 3 \mathclose]\).
\end{example}



%+++++++++++++++++++++++++++++++++++++++++++++++++++++++++++++++++++++++++++++++++++++++++++++++++++++++++++++++++++++++++++ 
\section{Continuité et dérivabilité}
%+++++++++++++++++++++++++++++++++++++++++++++++++++++++++++++++++++++++++++++++++++++++++++++++++++++++++++++++++++++++++++
Dans cette section, nous désignerons par \( I\) un intervalle ouvert non vide contenu dans $\eR$.

\begin{definition}[Fonction continue]
    Une fonction \( f\colon I\to \eR\) est \defe{continue}{continue!fonction!en un point} au point \( x_0\in I\) si \( \lim_{x\to x_0} f(x)=f(x_0)\).

    La fonction est dite \defe{continue}{continue!fonction!sur un intervalle} sur l'intervalle \( I\) si elle est continue en tous les points de \( I\).
\end{definition}

\begin{theorem}[Théorème des valeurs intermédiaires]    \label{ThoLEPooJxGXSN}
    Si \( f\) est continue sur un intervalle \( I=\mathopen[ a , b \mathclose]\) avec \( f(a)\neq f(b)\) alors pour tout \( t\) entre \(f(a)\) et \(f(b)\), il existe \( x\in I\) tel que \( f(x)=t\).
\end{theorem}

Nous considérons la question suivante : étant donné une fonction \( f\) définie sur \( I\setminus\{ x_0 \}\), est-il possible de définir \( f\) en \( x_0\) de telles façon à ce qu'elle soit continue ?

\begin{example}
    La fonction
    \begin{equation}
        \begin{aligned}
            f\colon \eR\setminus\{ 0 \}&\to \eR \\
            x&\mapsto \frac{1}{ x } 
        \end{aligned}
    \end{equation}
    n'est pas définie pour \( x=0\) et il n'y a pas moyen de définir \( f(0)\) de telle sorte que \( f\) soit continue parce que \( \lim_{x\to 0} \frac{1}{ x }\) n'existe pas.
\end{example}

\begin{definition}[Prolongement par continuité]
    Soit \( f\colon I\setminus\{ x_0 \}\to \eR\) telle que \( \lim_{x\to x_{0}} f(x)=\ell\). La fonction
    \begin{equation}
        \begin{aligned}
            \tilde f\colon I&\to \eR \\
            \tilde f(x)&=\begin{cases}
                f(x)    &   \text{si } x\neq x_0\\
                \ell    &    \text{si } x=x_0
            \end{cases}
        \end{aligned}
    \end{equation}
    est une fonction continue sur \( I\) et est appelée le \defe{prolongement par continuité}{prolongement!par continuité} de \( f\) en \( x_0\).
\end{definition}

\begin{example}
    La fonction \( f(x)=x\ln(|x|)\) n'est pas définie en \( x=0\). Cependant
    \begin{equation}
        \lim_{x\to 0} x\ln(|x|)=0.
    \end{equation}
    Nous pouvons donc définir la fonction
    \begin{equation}
        \begin{aligned}
            \tilde f\colon \eR&\to \eR \\
            x&\mapsto \begin{cases}
                x\ln(| x |)    &   \text{si } x\neq 0\\
                0    &    \text{si } x=0.
            \end{cases}
        \end{aligned}
    \end{equation}
    Contrairement à la fonction initiale \( f\), cette fonction \( \tilde f\) est définie et continue en \( 0\). 

    Notez que sur le graphe de la fonction \( \tilde f\), la courbe est bien régulière en \( x=0\).
    \begin{center}
       \input{pictures_tex/Fig_XJMooCQTlNL.pstricks}
    \end{center}

\end{example}

\begin{example}
    La fonction
    \begin{equation}
        \begin{aligned}
            f\colon \eR\setminus\{ -3,2 \}&\to \eR \\
            x&\mapsto  \frac{ x^2+2x-3 }{ (x+3)(x-2) }
        \end{aligned}
    \end{equation}
    admet pour limite \( \lim_{x\to -3} f(x)=\frac{ 4 }{ 5 }\). Son prolongement par continuité en \( x=-3\) est donné par
    \begin{equation}
        \tilde f(x)=\frac{ x-1 }{ x-2 }.
    \end{equation}
    Notons que les fonctions \( f\) et \( \tilde f\) ne sont pas identiques : l'une est définie pour \( x=-3\) et l'autre pas. Lorsqu'on fait le calcul
    \begin{equation}
        \frac{ x^2+2x-3 }{ (x+3)(x-2) }=\frac{ (x-1)(x+3) }{ (x+3)(x-2) }=\frac{ x-1 }{ x-2 },
    \end{equation}
    la simplification n'est pas du tout un acte anodin. Le dernier signe «\( =\)» est discutable parce que les deux dernières expressions ne sont pas égales pour tout \( x\); elles ne sont égales «que» pour les \( x\) pour lesquels les deux expressions existent.
\end{example}

\begin{example} \label{ExQWHooGddTLE}
    La fonction 
    \begin{equation}
        f(x)=\frac{ \cos(x)-1 }{ x }
    \end{equation}
    n'est pas définie en \( x=0\), mais en la limite
    \begin{equation}
        \lim_{x\to 0} \frac{ \cos(x)-1 }{ x }
    \end{equation}
    nous reconnaissons la limite définissant la dérivée du cosinus en \( 0\), c'est à dire que
    \begin{equation}
        \lim_{x\to 0} \frac{ \cos(x)-1 }{ x }=\sin(0)=0.
    \end{equation}
    Nous avons donc le prolongement par continuité
    \begin{equation}
        \tilde f(x)=\begin{cases}
            \frac{ \cos(x)-1 }{ x }    &   \text{si } x\neq 0\\
            0    &    \text{sinon}.
        \end{cases}
    \end{equation}

    Encore une fois, le graphe de la fonction \(\tilde f\) ne présente aucune particularité autour de \( x=0\).
    \begin{center}
        \input{pictures_tex/Fig_RPNooQXxpZZ.pstricks}
    \end{center}
\end{example}

\begin{definition}[Fonction dérivable]\label{defderivable}
    Nous disons qu'une fonction \( f\) est \defe{dérivable}{dérivable} au point \( x_0\in I\) si la limite
    \begin{equation}
        \lim_{\epsilon\to 0}\frac{ f(x_0+\epsilon)-f(x_0) }{ \epsilon }
    \end{equation}
    existe.
\end{definition}

Si \( f\) est une fonction dérivable, rien n'empêche la fonction dérivée \( f'\) d'être elle-même dérivable. Dans ce cas nous notons \( f''\) ou \( f^{(2)}\) la dérivée de la fonction \( f'\). Cette fonction $f''$ est la \defe{dérivée seconde}{dérivée!seconde} de \( f\). Elle peut encore être dérivable; dans ce cas nous notons \( f^{(3)}\) sa dérivée, et ainsi de suite. Nous définissons \( f^{(n)}=(f^{(n-1)})'\) la dérivée \( n\)\ieme de \( f\). Nous posons évidemment $f^{(0)}=f$.

\begin{theorem}
  Toute fonction $f$ dérivable au point $x_0$ est continue au point $x_0$. 
\end{theorem}

\begin{remark}
  La réciproque du théorème précédent n'est pas vraie : il existent bien des fonctions qui sont continues à un point $x_0$ mais qui ne sont pas dérivables en $x_0$. La fonction valeur absolue, $x\mapsto |x|$, par exemple est continue sur tout $\eR$ mais elle n'est pas dérivable en $0$. 
\end{remark}

%--------------------------------------------------------------------------------------------------------------------------- 
\subsection{Quelques formules à connaître}
%---------------------------------------------------------------------------------------------------------------------------

\begin{Aretenir}\label{formulesderivation}
  \begin{subequations}
    \begin{equation}
      \left(\alpha f(x) + \beta g(x)\right)' = \alpha f'(x)  + \beta g'(x).
    \end{equation}
    \begin{equation}
       \left(f(x)g(x)\right)' =  f'(x) g(x) + f(x) g'(x). 
    \end{equation}
    \begin{equation}
      \left(f(u(x))\right)' =  f'(u(x))u'(x). 
    \end{equation}
    \begin{equation}
      \left(\frac{f(x)}{g(x)}\right)' = \frac{f'(x) g(x) - f(x) g'(x)}{(g(x))^2}.
    \end{equation}
  \end{subequations}
\end{Aretenir}

%+++++++++++++++++++++++++++++++++++++++++++++++++++++++++++++++++++++++++++++++++++++++++++++++++++++++++++++++++++++++++++ 
\section{Application réciproque}
%+++++++++++++++++++++++++++++++++++++++++++++++++++++++++++++++++++++++++++++++++++++++++++++++++++++++++++++++++++++++++++

%--------------------------------------------------------------------------------------------------------------------------- 
\subsection{Définitions}
%---------------------------------------------------------------------------------------------------------------------------

Les définitions d'injection, surjection, bijection et d'application réciproque sont les définitions \ref{DEFooBFCQooPyKvRK} et \ref{DEFooTRGYooRxORpY}.

\begin{example}     \label{EXooCWYHooLEciVj}
    \begin{enumerate}
        \item
            La fonction \( x\mapsto x^2\) n'est pas une bijection de \( \eR\) vers \( \eR\) parce qu'il n'existe aucun \( x\) tel que \( x^2=-1\).
        \item
            La fonction 
            \begin{equation}
                \begin{aligned}
                    f\colon \mathopen[ 0 , +\infty [&\to \mathopen[ 0 , +\infty [ \\
                    x&\mapsto x^2 
                \end{aligned}
            \end{equation}
            est une bijection. Notez que c'est la même fonction que celle de l'exemple précédent. Seul l'intervalle sur laquelle nous nous plaçons a changé.
        \item
            La fonction 
            \begin{equation}
                \begin{aligned}
                    \sin\colon \eR&\to \mathopen[ -1 , 1 \mathclose] \\
                    x&\mapsto \sin(x) 
                \end{aligned}
            \end{equation}
            n'est pas une bijection parce qu'il existe plus de un \( x\) tel que \( \sin(x)=1\). 
    \end{enumerate}
    En conclusion : il est très important de préciser les domaines des fonctions considérées.
\end{example}

\begin{remark}
    Dire que la fonction \( f\colon I\to J\) est bijective, c'est dire que l'équation \( f(x)=y\) d'inconnue \( x\) peut être résolue de façon univoque pour tout \( y\in J\).
\end{remark}

\begin{remark}
  Toute fonction strictement monotone sur un intervalle $I$ est injective. 
\end{remark}

\begin{proposition} \label{PropOARooUuCaYT}
    Une fonction monotone et surjective d'un intervalle $I$ sur un autre intervalle $J$ est continue sur $I$.
\end{proposition}

\begin{example}
    La fonction
    \begin{equation}
        \begin{aligned}
            f\colon \mathopen[ 2 , 3 \mathclose]&\to \mathopen[ 4 , 9 \mathclose] \\
            x&\mapsto x^2 
        \end{aligned}
    \end{equation}
    est une bijection. Sa réciproque est la fonction
    \begin{equation}
        \begin{aligned}
            f^{-1}\colon \mathopen[ 4 , 9 \mathclose]&\to \mathopen[ 2 , 3 \mathclose] \\
            x&\mapsto \sqrt{x}. 
        \end{aligned}
    \end{equation}
\end{example}

\begin{example}
    Trouvons la fonction réciproque de la fonction affine \( f\colon \eR\to \eR\), \( x\mapsto 3x-2\). Si \( y\in \eR\) le nombre \( f^{-1}(y)\) est la valeur de \( x\) pour laquelle \( f(x)=y\). Il s'agit donc de résoudre
    \begin{equation}
        3x-2=y
    \end{equation}
    par rapport à \( x\). La solution est \( x=\frac{ y+2 }{ 3 }\) et donc nous écrivons
    \begin{equation}
        f^{-1}(y)=\frac{ y+2 }{ 3 }.
    \end{equation}
    Notons que dans les calculs, il est plus simple d'écrire «\( y\)» que «\( x\)» la variable de la fonction réciproque. Il est néanmoins (très) recommandé de nommer «\( x\)» la variable dans la réponse finale. Dans notre cas nous concluons donc
    \begin{equation}
        f^{-1}(x)=\frac{ x+2 }{ 3 }.
    \end{equation}
\end{example}

%--------------------------------------------------------------------------------------------------------------------------- 
\subsection{Graphe de la fonction réciproque}
%---------------------------------------------------------------------------------------------------------------------------

Par définition le graphe de la fonction \( f\) est l'ensemble des points de la forme \( (x,y)\) vérifiant \( y=f(x)\). Affin de déterminer le graphe de la bijection réciproque nous pouvons faire le raisonnement suivant.

        Le point \( (x_0,y_0)\) est sur le graphe de \( f\)

\noindent\( \Leftrightarrow\)

        La relation \( f(x_0)=y_0\) est vérifiée

\noindent\( \Leftrightarrow\)

        La relation \( x_0=f^{-1}(y_0)\) est vérifiée

\noindent\( \Leftrightarrow\)

        Le point \( (y_0,x_0)\) est sur le graphe de \( f^{-1}\).

\begin{Aretenir}
    Dans un repère orthonormal, le graphe de le bijection réciproque est obtenu à parti du graphe de \( f\) en effectuant une symétrie par rapport à la droite d'équation \( y=x\).
\end{Aretenir}

Le dessin suivant montre le cas de la courbe de la fonction carré comparé à celle de la racine carré.
\begin{center}
   \input{pictures_tex/Fig_CELooGVvzMc.pstricks}
\end{center}

%--------------------------------------------------------------------------------------------------------------------------- 
\subsection{Théorème de la bijection}
%---------------------------------------------------------------------------------------------------------------------------

\begin{proposition}
    Soit une bijection \( f\colon I\to J\) et \( f^{-1}\colon J\to I\) sa réciproque. Alors pour tout \( x_0\in I\) nous avons
    \begin{equation}    \label{EqHQRooNmLYbF}
        f^{-1}\big( f(x_0) \big)=x_0
    \end{equation}
    et pour tout \( y_0\in J\) nous avons
    \begin{equation}    \label{EqIYTooQPvZDr}
        f\big( f^{-1}(y_0) \big)=y_0.
    \end{equation}
\end{proposition}

\begin{proof}
    Nous prouvons la relation \eqref{EqHQRooNmLYbF} et nous laissons \eqref{EqIYTooQPvZDr} comme exercice au lecteur.

    Soit \( x_0\in I\), et posons \( y_0=f(x_0)\). La définition de l'application réciproque est que pour \( y\in J\), \( f^{-1}(y)\) est l'unique élément \( x\) de \( I\) tel que \( f(x)=y\). Donc \( f^{-1}(y_0)\) est l'unique élément de \( I\) dont l'image est \( y_0\). C'est donc \( x_0\) et nous avons \( f^{-1}(y_0)=x_0\), c'est à dire
    \begin{equation}
        f^{-1}\big( f(x_0) \big)=x_0.
    \end{equation}
\end{proof}

\begin{theorem}[Théorème de la bijection] \label{ThoKBRooQKXThd}
    Soit $I$ un intervalle et $f$ une fonction continue et strictement monotone de $I$ dans \( \eR\). Nous avons alors :
    \begin{enumerate}
        \item
            $f(I)$ est un intervalle de \( \eR\) ;
        \item
            La fonction \( f\colon I\to f(I)\) est bijective
        \item
            La fonction \( f^{-1}\colon f(I)\to I\) est strictement monotone de même sens que $f$ ;
        \item \label{ItemEJZooKuFoeFiv}
            La fonction \( f\colon I\to f(I)\) est un homéomorphisme, c'est-à-dire que \( f^{-1}\colon f(I)\to I\) est continue.
    \end{enumerate}
\end{theorem}

\begin{proof}

    Prouvons les choses point par point. 

    \begin{enumerate}
    \item

        Supposons pour fixer les idées que \( f\) est monotone croissante\footnote{Traitez en tant que exercice le cas où \( f\) est décroissante.}.
        
        Soient \( a< b\) dans \( f(I)\). Par définition il existe \( x_1,x_2\in I\) tels que \( a=f(x_1)\) et \( b=f(x_2)\). La fonction \( f\) est continue sur l'intervalle \( \mathopen[ x_1 , x_2 \mathclose]\) et vérifie \( f(x_1)<f(x_2)\). Donc le théorème des valeurs intermédiaires \ref{ThoLEPooJxGXSN} nous dit que pour tout \( t\) dans \( \mathopen[ f(x_2) , f(x_2) \mathclose]\), il existe un \( x_0\in\mathopen[ x_1 , x_2 \mathclose]\) tel que \( f(x_0)=t\). Cela montre que toutes les valeurs intermédiaires entre \( a\) et \( b\) sont atteintes par \( f\) et donc que \( f(I)\) est un intervalle.

    \item

    Nous prouvons maintenant que \( f\) est bijective en prouvant séparément qu'elle est surjective et injective.

    \begin{subproof}

        \item[\( f\) est surjective]

            Une fonction est toujours surjective depuis un intervalle \( I\) vers l'ensemble \(\Im f \).

        \item[\( f\) est injective]
        
            Soit \( x\neq y\) dans \( I\); pour fixer les idées nous supposons que \( x<y\). La stricte monotonie de \( f\) implique que \( f(x)<f(y)\) ou que \( f(x)>f(y)\). Dans tous les cas \( f(x)\neq f(y)\).

    \end{subproof}
    La fonction \( f\) est donc bijective.

\item

    Comme d'accoutumée nous supposons que \( f\) est croissante. Soient \( y_1<y_2\) dans \( f(I)\); nous devons prouver que \( f^{-1}(y_1)\leq f^{-1}(y_2)\). Pour cela nous considérons les nombres \( x_1,x_2\in I\) tels que \( f(x_1)=y_1\) et \( f(x_2)=y_2\). Nous allons en prouver la contraposée en supposant que \( f^{-1}(y_1)>f^{-1}(y_2)\). En appliquant \( f\) (qui est croissante) à cette dernière inégalité il vient
    \begin{equation}
        f\big( f^{-1}(y_1) \big)\geq f\big( f^{-1}(y_2) \big),
    \end{equation}
    ce qui signifie
    \begin{equation}
        y_1\geq y_2
    \end{equation}
    par l'équation \eqref{EqIYTooQPvZDr}.

\item

    La fonction \( f^{-1}\colon f(I)\to I\) est une fonction monotone et surjective, donc continue par la proposition \ref{PropOARooUuCaYT}.
  

    \end{enumerate}
\end{proof}

\begin{proposition}[\cite{XGIooNMtKqx}] \label{PropMRBooXnnDLq}
    Soit \( f\colon I\to J=f(I)\) une fonction bijective, continue et dérivable\footnote{Pour rappel, une fonction dérivable est toujours continue; l'hypothèse de continuité n'est pas nécessaire}. Soit \( x_0\in I\) et \( y_0=f(x_0)\). Si \( f'(x_0)\neq 0\) alors la fonction réciproque \( f^{-1}\) est dérivable en \( y_0\) et sa dérivée est donnée par
    \begin{equation}    
        (f^{-1})'(y_0)=\frac{1}{ f'(x_0) }.
    \end{equation}
\end{proposition}
 
  \begin{Aretenir}
 Très souvent on préfère retenir la formule
    \begin{equation}\label{EqWWAooBRFNsv}
      (f^{-1})'(y_0) = \frac{1}{f'\left((f^{-1})(y_0)\right)}
    \end{equation}
  \end{Aretenir}


\begin{proof}
    Prouvons que \( f^{-1}\) est dérivable au point \( b=f(a)\in J\). Étant donné que \( f\) est dérivable en \( a\), nous avons
    \begin{equation}\label{EqJEWooSjQrfk}
        f'(a)=\lim_{x\to a} \frac{ f(x)-f(a) }{ x-a }.
    \end{equation}
    Par ailleurs, étant donnée la continuité de \( f^{-1}\) donnée par la proposition \ref{ThoKBRooQKXThd}\ref{ItemEJZooKuFoeFiv}, nous avons
    \begin{equation}
        \lim_{\epsilon\to 0} f^{-1}(b+\epsilon)=f^{-1}(b)=a.
    \end{equation}
    Nous pouvons donc remplacer dans \eqref{EqJEWooSjQrfk} tous les \( x\) par \( f^{-1}(b+\epsilon)\) et prendre la limite \( \epsilon\to 0\) au lieu de \( x\to a\) :
    \begin{equation}
        \begin{aligned}[]
            f'(a)&=\lim_{\epsilon\to 0}\frac{ f\big( f^{-1}(b+\epsilon) \big)-f(a) }{ f^{-1}(b+\epsilon)-a }\\
            &=\lim_{\epsilon\to 0}\frac{ b+\epsilon-f(a) }{ f^{-1}(b+\epsilon)-f^{-1}(b) }\\
            &=\lim_{\epsilon\to 0}\frac{ \epsilon }{ f^{-1}(b+\epsilon)-f^{-1}(b) }\\
            &=\frac{1}{ \lim_{\epsilon\to 0}\frac{ f^{-1}(b+\epsilon)-f^{-1}(b) }{ \epsilon } }\\
            &=\frac{1}{ (f^{-1})'(b) }.
        \end{aligned}
    \end{equation}
    Nous avons utilisé le fait que \( f(a)=b\) et \( a=f^{-1}(b)\).
\end{proof}

\begin{remark}
    Le formule \eqref{EqWWAooBRFNsv} est très simple à retenir : il suffit d'écrire
    \begin{equation}    
        f^{-1}\big( f(x) \big)=x
    \end{equation}
    puis de dériver les deux côtés par rapport à \( x\) en utilisant la règle de dérivation des fonctions composées :
    \begin{equation}
        (f^{-1})'\big( f(x) \big)f'(x)=1.
    \end{equation}
\end{remark}

\begin{example}[Exponentielle et logarithme]    \label{ExZLMooMzYqfK}
    Nous savons que la fonction
    \begin{equation}
        \begin{aligned}
        \exp\colon \eR&\to \mathopen] 0 , \infty \mathclose[ \\
            x&\mapsto e^x 
        \end{aligned}
    \end{equation}
    est croissante et dérivable. Elle est donc bijective, d'inverse continue et dérivable par le théorème \ref{ThoKBRooQKXThd} et la proposition \ref{PropMRBooXnnDLq}. Nous nommons \defe{logarithme}{logarithme} la fonction inverse de l'exponentielle :
    \begin{equation}
    \ln\colon \mathopen] 0 , \infty \mathclose[\to \eR.
    \end{equation}
    La proposition \ref{PropMRBooXnnDLq} nous enseigne que la fonction logarithme est croissante et que sa dérivée peut être calculée\footnote{Nous savons que \( \exp'(x)=\exp(x)\) : la dérivée de l'exponentielle est l'exponentielle elle-même.} : si \( y= e^{x}\) alors
    \begin{equation}
        \ln'(y)=\frac{1}{ \exp'(x) }=\frac{1}{ y }.
    \end{equation}
    Nous retrouvons ainsi la formule très connue comme quoi la dérivée du logarithme est l'inverse\footnote{Ou encore que le logarithme est une primitive de la fonction inverse.}.
\end{example}

%+++++++++++++++++++++++++++++++++++++++++++++++++++++++++++++++++++++++++++++++++++++++++++++++++++++++++++++++++++++++++++ 
\section{Rappels de trigonométrie}
%+++++++++++++++++++++++++++++++++++++++++++++++++++++++++++++++++++++++++++++++++++++++++++++++++++++++++++++++++++++++++++
\label{secHTVooJuBtam}

Dans ce cours tous les angles sont exprimés en radiants. 

\begin{definition}
  La fonction \defe{tangente}{tangente} est la fonction
  \begin{equation}
      \begin{aligned}
          \tan\colon \eR\setminus \eZ\pi&\to \eR \\
          x&\mapsto \frac{ \sin(x) }{ \cos(x) }. 
      \end{aligned}
  \end{equation}
\end{definition} 
  
\begin{proposition}
    Quelque propriétés de la fonction tangente.
    \begin{enumerate}
        \item
            L'image de la fonction tangente est $\eR$. 
        \item
            La fonction tangente est périodique, de période $\pi$.
        \item
            Son graphe est un réunion dénombrable de courbes disjointes, voir la figure \ref{LabelFigPVJooJDyNAg}. 
    \end{enumerate}
\end{proposition}

Voici un tableau qui rappelle les valeurs à retenir pour les fonctions sinus, cosinus et tangente.
\begin{equation*}
    \begin{array}[]{|c|c|c|c|}
      \hline
      x&\sin(x)&\cos(x)&\tan(x)\\
      \hline
      0&0&1&0\\
      \hline
      \pi/6&1/2&\sqrt{3}/2&\sqrt{3}/3\\
      \hline
      \pi/6&1/2&\sqrt{3}/2&\sqrt{3}/3\\
      \hline
      \pi/4&\sqrt{2}/2&\sqrt{2}/2&1\\
      \hline
      \pi/3&\sqrt{3}/2&1/2&\sqrt{3}\\
      \hline
      \pi/2&1&0&\text{N.D.}\\
      \hline
    \end{array}
\end{equation*}
où «N.D.»  signifie «non défini».

Rappelons le graphe de la fonction sinus :
\begin{center}
   \input{pictures_tex/Fig_TWHooJjXEtS.pstricks}
\end{center}
celui de la fonction cosinus :
\begin{center}
   \input{pictures_tex/Fig_JJAooWpimYW.pstricks}
\end{center}

The result is on figure \ref{LabelFigPVJooJDyNAg}. % From file PVJooJDyNAg
\newcommand{\CaptionFigPVJooJDyNAg}{<+Type your caption here+>}
\input{pictures_tex/Fig_PVJooJDyNAg.pstricks}

Nous allons donner une preuve géométrique de la limite remarquable (vue en terminale) 
\begin{equation}\label{sinsurx}
  \lim_{x\to 0} \frac{\sin(x)}{x} = 1.
\end{equation}
Cette preuve peut vous servir pour reviser la signification géométrique des fonction trigonométriques et leur propriétés de base. 
\begin{description}
  \item{Première étape : } On montre que 
    \begin{lemma}
      Pour toute valeur de $x\in \eR$ on a $|\sin(x)|\leq |x|$. 
    \end{lemma}
    \begin{itemize}
    \item Si $0\leq x\leq \pi/2$ alors le sinus de $x$ est la longueur du cathète verticale du triangle rectangle de sommets $O = (0,0)$, $A = (\cos(x), \sin(x))$ et $B = (\cos(x), 0)$. Le triangle de sommets $A$, $B$ et $C = (1, 0)$ est aussi rectangle et nous savons que chacun des cathètes ne peut pas \^etre plus long que l'hypoténuse. Donc $\sin(x)$ est inférieur à la longueur du segment $AC$. À son tour le segment $AC$ ne peut pas \^etre plus long que l'arc de cercle $\wideparen{A0C}$, car le chemin le plus court entre deux points du plan est toujours donné par un morceau de droite. La longueur de l'arc du cercle $\frown{AC}$ est \emph{par définition} la mesure en radiants de l'angle $\widehat{AOC}$, qui est $x$ et on a l'inégalité $\sin(x)\leq x$. 
    \item Si $-\pi/2\leq x\leq 0$ le m\^eme raisonnement que au point précedent permet de conclure que $\sin(x)\leq |x|$.
    \item Nous savons par ailleurs que la fonction sinus prend ses valeurs dans l'intervalle $[-1,1]$ et donc pour tout $x$ tel que $|x|\geq \pi/2 \equiv 1,57\ldots$ on a forcement $|\sin(x)|\leq |x|$.  
    \end{itemize}
  \item{Deuxième étape :} On commence par observer que la fonction $g(x)=\frac{\sin(x)}{x}$ est un rapport entre deux fonction impairers et est donc une fonction paire. Nous pouvons alors nous réduire à considèrer le cas où $x$ est positif. La première étape de cette preuve nous dit que $g(x)\leq 1$ pour tout $x\in\eR^{+,*}$. 

Nous voulons étudier le comportement de $g$ dans un voisinage de $0$. Nous pouvons alors supposer que $x$ soit inférieur à $\pi/2$. Soit $D = (1, \tan (x))$. On voit très bien dans le dessin que l'aire du triange de sommets $O$, $D$ et $C$ est supérieure à l'aire du secteur circulaire de sommets $O$, $A$ et $C$. Ces deux aires peuvent \^etre calculées très facilement et nous obtenons
\begin{equation*}
  \frac{\sin(x)}{2\cos(x)} \geq \frac{x}{2}.
\end{equation*}
À partir de cette dernière inégalité nous pouvons écrire 
\begin{equation*}
  1\geq \frac{\sin(x)}{x}\geq \cos(x).
\end{equation*}
En prenant la limite lorsque $x$ tend vers $0$ dans les trois membres de l'inégalité la règle de l'étau nous permet d'obtenir la limite remarquable  \eqref{sinsurx}. 
\end{description}

%+++++++++++++++++++++++++++++++++++++++++++++++++++++++++++++++++++++++++++++++++++++++++++++++++++++++++++++++++++++++++++ 
\section{Fonctions trigonométriques réciproques}
%+++++++++++++++++++++++++++++++++++++++++++++++++++++++++++++++++++++++++++++++++++++++++++++++++++++++++++++++++++++++++++

%--------------------------------------------------------------------------------------------------------------------------- 
\subsection{La fonction arc sinus}
%---------------------------------------------------------------------------------------------------------------------------

Nous voulons étudier la fonction
\begin{equation}
    \begin{aligned}
        \sin\colon \eR&\to \mathopen[ -1 , 1 \mathclose] \\
        x&\mapsto \sin(x) 
    \end{aligned}
\end{equation}
et sa réciproque éventuelle.

La fonction sinus est continue sur \( \eR\) mais n'est pas bijective : elle prend une infinité de fois chaque valeur de \( J=\mathopen[ -1 , 1 \mathclose]\). Pour définir une bijection réciproque de la fonction sinus en utilisant le théorème \ref{ThoKBRooQKXThd}, nous devons donc choisir un intervalle à partir duquel la fonction sinus est monotone. Nous choisissons l'intervalle
\begin{equation}
    I=\mathopen[ -\frac{ \pi }{ 2 } , \frac{ \pi }{2} \mathclose].
\end{equation}
La fonction
\begin{equation}
    \begin{aligned}
        \sin\colon \mathopen[ -\frac{ \pi }{2} , \frac{ \pi }{2} \mathclose]&\to \mathopen[ -1 , 1 \mathclose] \\
        x&\mapsto \sin(x) 
    \end{aligned}
\end{equation}
est une bijection croissante et continue. Nous avons donc le résultat suivant.
\begin{theorem}[Définition et propriétés de arc sinus]
    Nous nommons \defe{arc sinus}{arc sinus} la bijection inverse de la fonction \( \sin\colon I\to J\). La fonction
    \begin{equation}
        \begin{aligned}
            \arcsin\colon \mathopen[ -1 , 1 \mathclose]&\to \mathopen[ -\frac{ \pi }{2} , \frac{ \pi }{2} \mathclose] \\
            x&\mapsto \arcsin(x) 
        \end{aligned}
    \end{equation}
    ainsi définie est
    \begin{enumerate}
        \item
            continue et strictement croissante;
        \item
            impaire : pour tout \( x\in\mathopen[ -1 , 1 \mathclose]\) nous avons \( \arcsin(-x)=-\arcsin(x)\).
    \end{enumerate}
\end{theorem}

\begin{proof}
    Nous prouvons le fait que \( \arcsin\) est impaire. Un élément de l'ensemble de définition de \( \arcsin\) est de la forme \( y=\sin(x)\) avec \( x\in\mathopen[ -\pi/2 , \pi/2 \mathclose]\). La relation \eqref{EqHQRooNmLYbF} s'écrit dans notre cas
    \begin{equation}    \label{EqVUWooUwVxVp}
        x=\arcsin\big( \sin(x) \big).
    \end{equation}
    Nous écrivons d'une part cette équation avec \( -x\) au lieu de \( x\) :
    \begin{equation}    \label{EqRLYooIwOvSz}
        -x=\arcsin\big( \sin(-x) \big)=\arcsin\big( -\sin(x) \big)=\arcsin(-y);
    \end{equation}
    et d'autre part nous multiplions \eqref{EqVUWooUwVxVp} par \( -1\) :
    \begin{equation}    \label{EqTGIooDeRYyT}
        -x=-\arcsin\big( \sin(x) \big)=-\arcsin(y).
    \end{equation}
    En égalisant les valeurs \eqref{EqRLYooIwOvSz} et \eqref{EqTGIooDeRYyT} nous trouvons
    \begin{equation}
        \arcsin(-y)=-\arcsin(y),
    \end{equation}
    ce qui signifie que \( \arcsin\) est une fonction impaire.
\end{proof}
Notons que cette preuve repose sur le fait que tout élément de l'ensemble de définition de la fonction arc sinus peut être écrit sous la forme \( \sin(x)\) pour un certain \( x\).

Si \( x_0\in\mathopen[ -1 , 1 \mathclose]\) est donné, calculer \( \arcsin(x_0)\) revient à trouver un angle \( \theta_0\) dans \( \mathopen[ -\frac{ \pi }{2} , \frac{ \pi }{2} \mathclose]\) pour lequel \( \sin(\theta_0)=x_0\). Un tel angle sera forcément unique.

\begin{remark}
  La définition de arc sinus découle du choix de l'intervalle $I$, qui est une convention. Il aurait été possible de faire un choix différent : pourriez vous trouver la réciproque de la fonction sinus sur l'intervalle $[\pi/2, 3\pi/2]$ ? Le mieux est de l'écrire comme une translatée de arc sinus, en utilisant le fait que sinus est une fonction périodique. 
\end{remark}

\begin{example}
    Pour calculer \( \arcsin(1)\), il faut chercher un angle entre \( -\frac{ \pi }{2}\) et \( \frac{ \pi }{ 2 }\) ayant \( 1\) pour sinus : résoudre \( \sin(\theta)=1\). La solution est \( \theta=\frac{ \pi }{2}\) et nous avons donc \( \arcsin(1)=\frac{ \pi }{2}\).
\end{example}

À l'aide des valeurs remarquables de la fonction sinus nous obtenons le tableau suivant de valeurs remarquables pour l'arc sinus.
\begin{equation*}
    \begin{array}[]{|c|c|c|c|c|c|}
        \hline
        x&0&\frac{ 1 }{2}&\frac{ \sqrt{2} }{2}&\frac{ \sqrt{3} }{2}&1\\
          \hline
          \arcsin(x)&0&\frac{ \pi }{ 6 }&\frac{ \pi }{ 4 }&\frac{ \pi }{ 3 }&\frac{ \pi }{ 2 }\\ 
          \hline 
           \end{array}
\end{equation*}
Les autres valeurs remarquables peuvent être déduites du fait que l'arc sinus est une fonction impaire.

En ce qui concerne la dérivabilité de la fonction arc sinus, en application de la proposition \ref{PropMRBooXnnDLq} elle est dérivable en tout \( y=\sin(x)\) tel que \( \sin'(x)\neq 0\), c'est à dire tel que \( \cos(x)\neq 0\). Or \( \cos(x)=0\) pour \( x=\pm\frac{ \pi }{2}\), ce qui correspond à \( y=\sin(\pm\frac{ \pi }{2})=\pm 1\). La fonction arc sinus est donc dérivable sur \( \mathopen] -1 , 1 \mathclose[\). Nous avons donc la propriété suivante pour la dérivabilité.

\begin{proposition}
    La fonction arc sinus est continue sur \( \mathopen[ -1 , 1 \mathclose]\) et dérivable sur \( \mathopen] -1 , 1 \mathclose[\). Pour tout \( y\in\mathopen] -1 , 1 \mathclose[\), la dérivée est donnée par la formule \eqref{EqWWAooBRFNsv}, qui dans ce cas s'écrit
        \begin{equation}
            \arcsin'(y)=\frac{1}{ \cos\big( \arcsin(y) \big) }=\frac{1}{ \sqrt{1-y^2} }.
        \end{equation}
\end{proposition}
La dernière égalité viens du fait que si $x=\arcsin(y)$ alors $y = \sin(x)$ et $\cos(x)= \sqrt{1-\sin^2(x)} = \sqrt{1-y^2}$. 

Pour comprendre la dernière égalité, remarquer que dans le dessin suivant, \( \theta=\arcsin(y)\), donc $y = \sin(\theta)$, et \( x=\cos(\theta)\).
\begin{center}
    \input{pictures_tex/Fig_BIFooDsvVHb.pstricks}
\end{center}

Notons enfin que le graphe de la fonction arc sinus est donné à la figure \ref{LabelFigFGRooDhFkch}. % From file FGRooDhFkch
\newcommand{\CaptionFigFGRooDhFkch}{Le graphe de la fonction \( x\mapsto \arcsin(x)\)}
\input{pictures_tex/Fig_FGRooDhFkch.pstricks}

%--------------------------------------------------------------------------------------------------------------------------- 
\subsection{La fonction arc cosinus}
%---------------------------------------------------------------------------------------------------------------------------

Nous voulons étudier la fonction
\begin{equation}
    \begin{aligned}
        \cos\colon \eR&\to \mathopen[ -1 , 1 \mathclose] \\
        x&\mapsto \cos(x) 
    \end{aligned}
\end{equation}
et son éventuelle réciproque. Encore une fois il n'est pas possible d'en prendre la réciproque globale parce que ce n'est pas une bijection. Nous choisissons de considérer l'intervalle \( \mathopen[ 0 , \pi \mathclose]\) sur lequel la fonction cosinus est continue et strictement monotone décroissante.

Nous avons alors le résultat suivant :

\begin{proposition}
    La fonction
    \begin{equation}
        \begin{aligned}
            \cos\colon \mathopen[ 0 , \pi \mathclose]&\to \mathopen[ -1 , 1 \mathclose] \\
            x&\mapsto \cos(x) 
        \end{aligned}
    \end{equation}
    est une bijection continue strictement décroissante. Sa bijection réciproque, nommée \defe{arc cosinus}{arc cosinus}
    \begin{equation}
        \begin{aligned}
            \arccos\colon \mathopen[ -1 , 1 \mathclose]&\to \mathopen[ 0 , \pi \mathclose] \\
            x&\mapsto \arccos(x) 
        \end{aligned}
    \end{equation}
est continue, strictement décroissante et dérivable. Pour tout \( y\in\mathopen] -1 , 1 \mathclose[\), sa dérivée est donnée par
    \begin{equation}
        \arccos'(y)=\frac{1}{ -\sin\big( \arccos(y) \big) }=\frac{ -1 }{ \sqrt{1-y^2} }.
    \end{equation}
\end{proposition}

\begin{remark}
    Certes la fonction cosinus est paire (vue sur \( \eR\)), mais la fonction arc cosinus ne l'est pas car elle est une bijection entre \(\mathopen[ -1 , 1 \mathclose]\) et \(\mathopen[ 0 , \pi \mathclose]\).
\end{remark}

Pour \( y_0\in\mathopen[ -1 , 1 \mathclose]\), trouver la valeur de \( \arccos(y_0)\) revient à résoudre l'équation \( \cos(x_0)=y_0\). Cela nous permet de construire une tableau de valeurs :
\begin{equation*}
    \begin{array}[]{|c|c|c|c|c|c|c|c|c|c|}
        \hline
        x&-1&-\frac{ \sqrt{3} }{2}&-\frac{ \sqrt{2} }{2}&-\frac{ 1 }{2}&0&\frac{ 1 }{2}&\frac{ \sqrt{2} }{2}&\frac{ \sqrt{3} }{2}&1\\
          \hline
          \arccos(x)&\pi&\frac{ 5\pi }{ 6 }&\frac{ 3 }{ 4 }\pi&\frac{ 2 }{ 3 }\pi&\frac{ 1 }{2}\pi&\frac{ \pi }{ 3 }&\frac{1}{ 4 }\pi&\frac{1}{ 6 }\pi&0\\
          \hline 
           \end{array}
\end{equation*}

\begin{example}
    Cherchons \( \arccos(\frac{ 1 }{2})\). Il faut trouver un angle \( \theta\in\mathopen[ 0 , \pi \mathclose]\) tel que \( \cos(\theta)=\frac{ 1 }{2}\). La solution est \( \theta=\frac{ \pi }{ 3 }\). Donc \( \arccos(\frac{ 1 }{2})=\frac{ \pi }{ 3 }\).

    Il n'est cependant pas immédiat d'en déduire la valeur de \( \arccos(-\frac{ 1 }{2})\). En effet \( \theta=\arccos(-\frac{ 1 }{2})\) si et seulement si \( \cos(\theta)=-\frac{ 1 }{2}\) avec \( \theta\in\mathopen[ 0 , \pi \mathclose]\). La solution est \( \theta=\frac{ 2\pi }{ 3 }\).
\end{example}

En ce qui concerne la représentation graphique, il suffit de tracer la fonction cosinus entre \( 0\) et \( \pi\) puis de prendre le symétrique par rapport à la droite \( y=x\).

\begin{center}
    \input{pictures_tex/Fig_GMIooJvcCXg.pstricks}
\end{center}

%--------------------------------------------------------------------------------------------------------------------------- 
\subsection{La fonction arc tangente}
%---------------------------------------------------------------------------------------------------------------------------

La fonction tangente est donnée par la formule
\begin{equation}
    \tan(x)=\frac{ \sin(x) }{ \cos(x) }
\end{equation}
et n'est pas définie sur les points de la forme \( x=\frac{ \pi }{2}+k\pi\), \( k\in \eZ\). Afin de définir une bijection réciproque nous considérons l'intervalle \( \mathopen] -\frac{ \pi }{2} , \frac{ \pi }{2} \mathclose[\) (qui est ouvert, contrairement aux intervalles choisis pour arc cosinus et arc sinus). Le résultat est le suivant.

\begin{theorem}
    La fonction
    \begin{equation}
        \begin{aligned}
        \tan\colon \left] -\frac{ \pi }{2} , \frac{ \pi }{2} \right[&\to \eR \\
            x&\mapsto \tan(x) 
        \end{aligned}
    \end{equation}
    est une bijection strictement croissante.

    La bijection réciproque 
    \begin{equation}
        \begin{aligned}
        \arctan\colon \eR&\to \left] -\frac{ \pi }{2} , \frac{ \pi }{2} \right[ \\
            x&\mapsto \arctan(x) 
        \end{aligned}
    \end{equation}
    nommée \defe{arc tangente}{arc tangente} est
    \begin{enumerate}
        \item
            impaire et strictement croissante sur \( \eR\).
        \item
            dérivable sur \( \eR\) de dérivée
            \begin{equation}
                \arctan'(x)=\frac{1}{ 1+\tan^2\big( \arctan(x) \big) }=\frac{1}{ 1+x^2 }.
            \end{equation}
    \end{enumerate}
\end{theorem}
Note : la dernière ligne n'a rien de mystérieux : \( \tan\big( \arctan(x) \big)=x\) et donc \( \tan^2\big( \arctan(x) \big)=x^2\).    

En ce qui concerne la dérivabilité nous savons que
\begin{equation}
    \tan'(x)=1+\tan^2(x) ,
\end{equation}
qui ne s'annule pour aucune valeur de \( x\); c'est pour cela que \( \arctan\) est dérivable sur tout \( \eR\).

Le nombre \( \arctan(x_0)\) se calcule en cherchant l'angle \( \theta\in\mathopen[ -\frac{ \pi }{2} , \frac{ \pi }{2} \mathclose]\) dont la tangente vaut \( x_0\). Nous obtenons le tableau de valeurs suivant :
\begin{equation*}
    \begin{array}[]{|c|c|c|c|c|}
        \hline
        x&0&\frac{1}{ \sqrt{3} }&1&\sqrt{3}\\
        \hline
        \arctan(x)&0&\frac{ \pi }{ 6 }&\frac{ \pi }{ 4 }&\frac{ \pi }{ 3 }\\
        \hline
    \end{array}
\end{equation*}

En ce qui concerne la représentation graphique de la fonction \( x\mapsto\arctan(x)\), elle s'obtient «en retournant» la partie entre \( -\frac{ \pi }{2}\) et \( \frac{ \pi }{ 2 }\) du graphique de la fonction tangente (voir les rappels \ref{secHTVooJuBtam}).
\begin{center}
   \input{pictures_tex/Fig_UQZooGFLNEq.pstricks}
\end{center}

%+++++++++++++++++++++++++++++++++++++++++++++++++++++++++++++++++++++++++++++++++++++++++++++++++++++++++++++++++++++++++++ 
\section{Trigonométrie hyperbolique}
%+++++++++++++++++++++++++++++++++++++++++++++++++++++++++++++++++++++++++++++++++++++++++++++++++++++++++++++++++++++++++++

\begin{definition}
    Les fonction \defe{sinus hyperbolique}{sinus!hyperbolique} et \defe{cosinus hyperbolique}{cosinus!hyperbolique} sont les fonctions définies sur $\eR$ par les formules suivantes :
    \begin{subequations}
        \begin{align}
            \cosh(x)&=\frac{  e^{x}+ e^{-x} }{2}\\
            \sinh(x)&=\frac{  e^{x}- e^{-x} }{2}
        \end{align}
    \end{subequations}
\end{definition}

Leurs principales propriétés sont :
\begin{enumerate}
    \item
        \( \cosh^2(x)-\sinh^2(x)=1\)
    \item
        \( \cosh'(x)=\sinh(x)\) 
    \item
        \( \sinh'(x)=\cosh\).
\end{enumerate}

Les représentations graphiques sont ceci :
\begin{center}
   \input{pictures_tex/Fig_UNVooMsXxHa.pstricks}
\end{center}

La \defe{tangente hyperbolique}{tangente hyperbolique} est donnée par le quotient
\begin{equation}
    \tanh(x)=\frac{ \sinh(x) }{ \cosh(x) }.
\end{equation}

%Les fonction réciproques de $\sinh$, $\cosh$ et $\tanh$ sont traitées dans les exercices.
