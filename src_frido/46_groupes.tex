% This is part of Mes notes de mathématique
% Copyright (c) 2011-2016
%   Laurent Claessens
% See the file fdl-1.3.txt for copying conditions.

%+++++++++++++++++++++++++++++++++++++++++++++++++++++++++++++++++++++++++++++++++++++++++++++++++++++++++++++++++++++++++++
\section{Groupes}
%+++++++++++++++++++++++++++++++++++++++++++++++++++++++++++++++++++++++++++++++++++++++++++++++++++++++++++++++++++++++++++

\begin{definition}[\cite{Kropholler}]
    Soit \( G\) un groupe. Le \defe{centralisateur}{centralisateur} de \( H\subset G\) est 
    \begin{equation}
        \mZ_G(H)=\{ g\in G\tq hg=gh\,\forall h\in h\}
    \end{equation}
    Si \( H\) est un sous-groupe, son \defe{normalisateur}{normalisateur} est
    \begin{equation}
        N_G(H)=\{ g\in G\tq gH=Hg \}.
    \end{equation}
\end{definition}

Le \defe{centre}{centre!d'un groupe} du groupe \( G\) est l'ensemble
\begin{equation}
    Z_G=\{ z\in G\tq gz=zg\forall g\in G \}.
\end{equation}

\begin{definition}
    Un sous-groupe \( N\) de \( G\) est \defe{normal}{normal!sous-groupe} ou \defe{distingué}{distingué} si pour tout \( g\in G\) et pour tout \( n\in N\), \( gng^{-1}\in N\). Autrement dit lorsque \( gNg^{-1}\subset N\). 

    Lorsque \( N\) est normal dans \( G\) il est parfois noté \( N\normal G\)\nomenclature[]{\(N \normal G\)}{Le sous-groupe \( N\) est normal dans \( G\)}.

    Un sous-groupe \( H\) de \( G\) est un sous-groupe \defe{caractéristique}{sous-groupe!caractéristique}\index{caractéristique!sous-groupe} si \( \alpha(H)=H\) pour tout \( \alpha\in \Aut(G)\).
\end{definition}

\begin{definition}
    Soit \( G\), un groupe et \( A\) une partie de \( G\). Nous notons \( \gr(A)\)\nomenclature[R]{\( \gr\)}{groupe engendré} l'intersection de tous les sous-groupes de \( G\) contenant \( A\). C'est le plus petit (pour l'inclusion) groupe de \( G\) contenant \( A\). Si \( \gr(A)=G\), alors nous disons que \( A\) est une \defe{partie génératrice}{partie génératrice} le groupe \( G\).

    Un groupe est \defe{monogène}{monogène} s'il a une partie génératrice réduite à un seul élément.
\end{definition}

\begin{definition}[Groupe cyclique]     \label{DefHFJWooFxkzCF}
    Un élément \( a\in G\) est un \defe{générateur}{générateur} de \( G\) si tous les éléments de \( G\) s'écrivent sous la forme \( a^n\) pour un certain \( n\in\eZ\). Un groupe fini et monogène est dit \defe{cyclique}{cyclique!groupe}.
\end{definition}

\begin{definition}[Sous-groupe engendré]        \label{DefooRDRXooEhVxxu}
    Soit \( A\) une partie du groupe \( G\). Le sous-groupe \defe{engendré}{sous-groupe!engendré}\index{engendré!sous-groupe} est l'intersection de tous les sous-groupes de \( G\) contenant \( A\).
\end{definition}

\begin{lemma}   \label{LemFUIZooBZTCiy}
    Si \( A\) est une partie du groupe \( G\), alors le sous-groupe engendré\footnote{Définition \ref{DefooRDRXooEhVxxu}.} par \( A\) est l'ensemble de tous les produits finis d'éléments de \( A\) et de \( A^{-1}\) (l'identité est le produit à zéro éléments).
\end{lemma}

\begin{proof}
    Nous nommons \( K\) le groupe engendré par \( A\) et par \( H\) l'ensemble de tous les produits finis d'éléments de \( A\) et de \( A^{-1}\). Cela est un sous-groupe de \( G\) contenant \( A\). Donc \( K\subset H\) parce que \( K\) est une intersection dont un des éléments est \( H\).

    Par ailleurs tout groupe contenant \( A\) doit contenir les inverses et les produits finis, donc \( H\subset K\). 

    Au final, \( H=K\), ce qu'il fallait.
\end{proof}

\begin{example}
    Soit \( G\) le groupe des rotations d'angle \( 2k\pi/10\). La rotation d'angle \( \pi/2\) n'est pas génératrice parce qu'elle n'engendre que \( \pi/2\), \( \pi\),\( 3\pi/2\) et l'identité. La rotation d'angle \( 2\pi/10\) par contre est génératrice.
\end{example}

%+++++++++++++++++++++++++++++++++++++++++++++++++++++++++++++++++++++++++++++++++++++++++++++++++++++++++++++++++++++++++++ 
\section{Sous groupe normal}
%+++++++++++++++++++++++++++++++++++++++++++++++++++++++++++++++++++++++++++++++++++++++++++++++++++++++++++++++++++++++++++

\begin{proposition}
    Soit \( N\) un sous-groupe de \( G\). Les propriétés suivantes sont équivalentes :
    \begin{enumerate}
        \item
            \( gNg^{-1}\subseteq N\) pour tout \( g\in G\),
        \item
            \( gNg^{-1}= N\) pour tout \( g\in G\),
        \item
            \( gN=Ng\) pour tout \( g\in G\),
        \item
            \( N\) est une union de classes de conjugaison de \( G\),
        \item
            \( N\) est normal.
    \end{enumerate}
\end{proposition}

\begin{definition}
    Soit \( g\in G\) et \( n\in \eZ\). Nous définissons \( g^n\) par
    \begin{enumerate}
        \item
            \( g^0=e\) et \( g^n=gg^{n-1}\) si \( n\) est positif.
        \item
            si \( n<0\), nous posons \( g^n=(g^{-1})^{-n}\).
    \end{enumerate}
\end{definition}

\begin{definition}
    Si \( G\) est un groupe, l'\defe{ordre}{ordre!d'un groupe} est la cardinalité de \( G\) et est noté \( | G |\). L'\defe{ordre}{ordre!élément} d'un élément \( g\) de \( G\) est le naturel
    \begin{equation}
        \min\{ n\in\eN\tq g^n=e \}.
    \end{equation}
    Si le minimum n'existe pas, nous disons que l'ordre de \( g\) est infini.
\end{definition}
Nous verrons que le corollaire \ref{CorpZItFX} au théorème de Lagrange dira que l'ordre d'un élément divise l'ordre du groupe.

\begin{lemma}[\cite{PDFpersoWanadoo}]\label{LemHUkMxp}
    Si \( H\) et \( K\) sont normaux dans le groupe \( G\) et si \( H\cap K=\{ e \}\) alors \( HK\simeq H\times K\).
\end{lemma}

\begin{definition}  \label{DefvtSAyb}
    L'\defe{exposant}{exposant!d'un groupe} du groupe \( G\) est le plus petit entier non
    nul \( n\) tel que \( g^n=e\) pour tout \( g\in G\). S'il n'existe pas, nous disons que l'exposant du groupe est infini.
\end{definition}
Si l'ordre de tous les éléments acceptent un majorant commun, alors l'exposant du groupe est le plus petit commun multiple des ordres des éléments. En particulier pour un groupe fini, l'exposant est le $\ppcm$ des ordres des éléments du groupe.

Le théorème de Burnside \ref{ThooJLTit} nous donnera un bon paquet d'exemples de groupes d'exposant fini dans \( \GL(n,\eC)\).

\begin{proposition} \label{PropSRMJooIDPBoW}
    Soit \( H\) un sous-groupe normal de \( G\) et \( \psi\colon G\to K\) un homomorphisme. 
    \begin{enumerate}
        \item
            \( \psi(H)\) est normal dans \( \psi(G)\)
        \item
            Si \( G/H\) est abélien alors \( \psi(G)/\psi(H)\) est abélien.
    \end{enumerate}
\end{proposition}

\begin{proof}
    Soient \( h\in H\) et \( g\in G\). Alors \( \psi(g)\psi(h)\psi(g)^{-1}=\psi(ghg^{-1})\in\psi(H)\). Donc \( \psi(H)\) est normal dans \( \psi(G)\).

    Pour la seconde partie nous notons \( [\ldots]\) les classes par rapport à \( \psi(H)\) et \( \overline{ \vphantom{g}\ldots }\) celles par rapport à \( H\). Nous avons
    \begin{subequations}
        \begin{align}
            [\psi(g_1)][\psi(g_2)]&=\big[ \psi(g_1)\psi(g_2) \big]\\
            &=\big[ \psi(g_1g_2) \big]\\
            &=\{ \psi(g_1g_2)\psi(h)\tq h\in H \}\\
            &=\{ \psi(g_1g_2h)\tq h\in H \}\\
            &=\psi\Big(  \{ g_1g_2h\tq h\in H \}  \Big) \\
            &=\psi\big( \overline{ g_1g_2 } \big)\\
            &=\psi(\overline{ g_2g_1 })\\
            &=\text{refaire à l'envers}\\
            &=[\psi(g_2)][\psi(g_1)].
        \end{align}
    \end{subequations}
    Par conséquent \( \psi(G)/\psi(H)\) est abélien.
\end{proof}

%+++++++++++++++++++++++++++++++++++++++++++++++++++++++++++++++++++++++++++++++++++++++++++++++++++++++++++++++++++++++++++ 
\section{Groupe dérivé}
%+++++++++++++++++++++++++++++++++++++++++++++++++++++++++++++++++++++++++++++++++++++++++++++++++++++++++++++++++++++++++++

\begin{definition}\label{DefVUFBooNQjEdn}
    Si \( G\) est un groupe et si \( g,h\in G\), nous notons \( [g,h]=ghg^{-1}h^{-1}\)\nomenclature[R]{\( [g,h]\)}{commutateur dans un groupe} le \defe{commutateur}{commutateur!dans un groupe} de \( g\) et \( h\). Le \defe{groupe dérivé}{dérivé!groupe}\index{groupe!dérivé} de \( G\) est le sous-groupe note \( D(G)\)\nomenclature[R]{\( D(G)\)}{groupe dérivé} ou \( [G,G]\)\nomenclature[R]{\( [G,G]\)}{groupe dérivé} engendré par les commutateurs.
\end{definition}
Autrement dit, \( D(G)\) est l'intersection de tous les sous-groupes de \( G\) contenant tous les commutateurs. Intersection non vide parce que \( G\) lui-même en fait partie.

En vertu du lemme \ref{LemFUIZooBZTCiy}, le groupe dérivé de \( G\) est l'ensemble des produits finis de commutateurs. C'est à dire que si \( S_m\) est l'ensemble des produits de \( m\) commutateurs, alors
\begin{equation}
    D(G)=\bigcup_{m=1}^{\infty}S_m.
\end{equation}

\begin{lemma}   \label{LemMMOCooDJJJhy}
    Le groupe dérivé est un sous-groupe caractéristique, et un sous-groupe normal.
\end{lemma}

\begin{proof}
    Il est évident que si \( \alpha\in\Aut(G)\) alors 
    \begin{equation}
        \alpha\big( [g,h] \big)=\big[ \alpha(g),\alpha(h) \big],
    \end{equation}
    c'est à dire que \( D(G)\) est un sous-groupe caractéristique. En particulier si \( c\) est un commutateur, alors \( xcx^{-1}\) en est encore un, ce qui montre que \( D(G)\) est normal dans \( G\). Plus spécifiquement,
    \begin{equation}
        x(ghg^{-1}h^{-1})x^{-1}=(xgx^{-1})(xhx^{-1})(xg^{-1}x^{-1})(xh^{-1}x^{-1})=(xgx^{-1})(xhx^{-1})(xgx^{-1})^{-1}(xhx^{-1})^{-1}.
    \end{equation}
\end{proof}

\begin{proposition}\label{PropAPRGooHBkELf}
    Le groupe quotient \( G/D(G)\) est abélien.
\end{proposition}

\begin{proof}
    En ce qui concerne le fait que \( G/D(G)\) soit abélien, nous savons que pour tout \( g,h\in G\) nous avons \( h^{-1}g^{-1}hg\in D(G)\) et donc
    \begin{equation}
        [g][h]=[gh]=[ghh^{-1}g^{-1}hg]=[hg]=[h][g].
    \end{equation}
\end{proof}

Le groupe quotient \( G/D(G)\) est appelé l'\defe{abélianisé}{abélianisé} de \( G\) et est parfois noté \( G^{ab}\)\nomenclature[R]{\( G^{ab}\)}{groupe abélianisé de \( G\)}.

Si \( f\colon G\to A\) est un homomorphisme entre le groupe \( G\) et un groupe abélien \( A\), alors \( f\big( D(G) \big)=\{ 0 \}\). Du coup \( f\) passe au quotient de \( G\) par \( D(G)\), et il existe une unique application \( \bar f\colon G/D(G)\to A\) telle que \( f=\bar f\circ \pi\) où \( \pi\colon G\to G/D(G)\) est la projection canonique.

%+++++++++++++++++++++++++++++++++++++++++++++++++++++++++++++++++++++++++++++++++++++++++++++++++++++++++++++++++++++++++++
\section{Théorèmes d'isomorphismes}
%+++++++++++++++++++++++++++++++++++++++++++++++++++++++++++++++++++++++++++++++++++++++++++++++++++++++++++++++++++++++++++

Si \( G\) est un groupe et si \( N\) est un sous-groupe normal, alors l'ensemble \( G/N\) a une structure de groupe et la projection canonique \( \pi\colon G\to G/N\) est un homomorphisme surjectif de noyau~\( N\).

\begin{theorem}[premier théorème d'isomorphisme]        \label{ThoPremierthoisomo}
    Soit \( \theta\colon G\to H\) un homomorphisme de groupe. Alors
    \begin{enumerate}
        \item
            \( \Kernel\theta\) est normal dans \( G\),
        \item
            \( \Image \theta\) est un sous-groupe de \( H\)
        \item   \label{ItemWLCLdk}
            nous avons un isomorphisme naturel
            \begin{equation}
                G/\Kernel\theta\simeq \Image\theta
            \end{equation}
    \end{enumerate}
\end{theorem}
\index{théorème!isomorphisme!premier}
%TODO : une preuve des théorèmes d'isomorphismes.

\begin{proof}
    \begin{enumerate}
        \item
        \item
        \item
            Si \( [g]\) représente la classe de \( g\) dans \( G/\Kernel\theta\), l'isomorphisme est donné par \( \varphi[g]=\theta(g)\).
    \end{enumerate}
\end{proof}


\begin{theorem}[Deuxième théorème d'isomorphisme]
    Soient \( H\) et \( N\) deux sous-groupes de \( G\) et supposons que \( N\) soit normal. Alors
    \begin{enumerate}
        \item
            \( NH=HN\) est un sous-groupe
        \item
            Le groupe \( N\) est normal dans \( NH\).
        \item
            Le groupe \( N\cap H\) est normal dans \( H\).

        \item\label{ItemjRPajc}
            nous avons l'isomorphisme
            \begin{equation}
                \frac{ HN }{ N }\simeq\frac{ H }{ H\cap N }.
            \end{equation}
        \item   \label{ItembgDQEN}
            L'isomorphisme du point \ref{ItemjRPajc} est encore valable si \( N\) n'est pas normal mais si seulement \( H\) normalise \( N\), c'est à dire si \( hNh^{-1}\in N\) pour tout \( h\in H\).
    \end{enumerate}
\end{theorem}
\index{théorème!isomorphisme!second}
%TODO : trouver une démonstration du dernier point.

\begin{proof}
    \begin{enumerate}
        \item
        \item
        \item
        \item
            Il faut d'abord remarquer que \( H\) et \( N\) étant des groupes et le produit \( NH\) étant un groupe, nous avons \( NH=HN\). Soit le morphisme injectif
            \begin{equation}
                \begin{aligned}
                    j\colon H&\to HN \\
                    h&\mapsto h
                \end{aligned}
            \end{equation}
            et la surjection canonique
            \begin{equation}
                \sigma\colon HN\to HN/N 
            \end{equation}
            Nous considérons ensuite l'application composée
            \begin{equation}
                \begin{aligned}
                    f\colon H&\to HN/N \\
                    h&\mapsto hN. 
                \end{aligned}
            \end{equation}
            L'application \( f\) est surjective parce que l'élément \( hnN\in HN/N\) est l'image de \( h\), étant donné que \( hnN=hN\).

            Le noyau de \( f\) est \( \Kernel f=H\cap N\). En effet si \( a\in H\cap N\), nous avons \( f(a)=\sigma(a)\in K\). Par conséquent \( H\cap N\subset \Kernel f\). D'autre part si \( h\in H\) vérifie \( h\in\Kernel f\), alors \( f(h)=hN=N\), ce qui est uniquement possible si \( h\in N\).

            Le premier théorème d'isomorphisme implique alors que \( H/\Kernel f\simeq \Image f\), c'est à dire
            \begin{equation}
                H/N\cap H\simeq HN/N.
            \end{equation}
    \end{enumerate}
\end{proof}

\begin{theorem}[Troisième théorème d'isomorphisme]  \label{ThoezgBep}
    Soient \( N\) et \( M\) deux sous-groupes normaux de \( G\) avec \( M\subset N\). Alors \( N/M\) est normal dans \( G/M\) et
    \begin{equation}
        (G/M)/(N/M)\simeq G/N.
    \end{equation}
\end{theorem}
\index{théorème!isomorphisme!troisième}

\begin{proof}
    Afin de montrer que \( N/M\) est normal dans \( G/M\), nous considérons \( g\in G\), \( nM\in N/M\) et nous calculons
    \begin{equation}
        gnMg^{-1}=gng^{-1}\underbrace{gMg^{-1}}_{=M}=\underbrace{gng^{-1}}_{\in N}M\in N/M.
    \end{equation}

    Pour prouver l'isomorphisme nous considérons le morphisme
    \begin{equation}
        \begin{aligned}
            \varphi\colon G/M&\to G/N \\
            gM&\mapsto gN. 
        \end{aligned}
    \end{equation}
    C'est surjectif et le noyau est \( N/M\) parce que \( \varphi(gM)=N\) uniquement si \( g\in N\). Nous pouvons appliquer le premier théorème d'isomorphisme à \( \varphi\) en écrivant
    \begin{equation}
        (G/M)/\Kernel \varphi\simeq\Image \varphi,
    \end{equation}
    c'est à dire
    \begin{equation}
        (G/M)/(N/M)\simeq G/N.
    \end{equation}
\end{proof}

%+++++++++++++++++++++++++++++++++++++++++++++++++++++++++++++++++++++++++++++++++++++++++++++++++++++++++++++++++++++++++++
\section{Le groupe et anneau des entiers}
%+++++++++++++++++++++++++++++++++++++++++++++++++++++++++++++++++++++++++++++++++++++++++++++++++++++++++++++++++++++++++++

Certes \( \eZ\) est un groupe pour l'addition, mais c'est également un anneau\footnote{Définition \ref{DefHXJUooKoovob}.} parce que nous avons les deux opérations d'addition et de multiplication. Nous n'allons pas nous priver de cette belle structure juste parce que le titre du chapitre est «groupes».

\begin{proposition} \label{PropSsgpZestnZ}
    Une partie \( H\) du groupe \( (\eZ,+)\) est un sous-groupe si et seulement s'il existe \( n\in\eN\) tel que \( H=n\eZ\).
\end{proposition}

\begin{proof}
    Soit \( H\neq\{ 0 \}\) un sous-groupe de \( \eZ\). L'ensemble \( H\cap\eN^*\) contient un élément minimum que nous notons \( n\). Nous avons certainement \( n\eZ\subset H\) parce que \( H\) est un groupe (donc \( n+n\) et \( -n\) sont dans \( H\) dès que \( n\) est dans \( H\)). Nous devons prouver que \( H\subset n\eZ\).

    Si \( x\in H\), il existe \( q\in\eZ\) et \( r\in\eN_{n-1}\) tels que \( x=nq+r\). 

    \begin{probleme}
        Justification par la division euclidienne à venir.
    \end{probleme}

    Nous savons déjà que \( nq\in H\), donc \( r\in H\) parce que \( x\) est également dans \( H\). Mais nous avions décidé que \( n\) serait le plus petit naturel de \( H\cap\eN^*\). Par conséquent \( r=0\) et \( x=nq\in n\eZ\).
\end{proof}


Notons que si un sous-groupe \( H\) de \( \eZ\) est donné, alors le nombre \( n\) tel que \( H=n\eZ\) est unique. En effet si \( n\eZ=m\eZ\) nous avons que \( n\) divise \( m\) (parce que \( m\in m\eZ\subset n\eZ\)) et que \( m\) divise \( n\) parce que \( n\in m\eZ\). Par conséquent \( n=m\).

\begin{lemma}   \label{LemZhxMit}
    Soient \( G\) et \( H\) deux groupes monogènes de même ordre. Soient \( g\) un générateur de \( G\) et \( h\), un générateur de \( H\). Il existe un isomorphisme de \( G\) sur \( H\) qui envoie \( g\) sur \( h\).
\end{lemma}

%---------------------------------------------------------------------------------------------------------------------------
\subsection{Division euclidienne}
%---------------------------------------------------------------------------------------------------------------------------

\begin{theorem}[Division euclidienne]     \label{ThoDivisEuclide}
    Soient \( a\in\eZ\) et \( b\in\eN^*\). Il existe un unique couple \( (q,r)\in\eZ\times\eN\) tel que
    \begin{equation}
        a=bq+r
    \end{equation}
    avec \( 0\leq r<b\).
\end{theorem}

L'opération \( (a,b)\mapsto(a,r)\) donnée par le théorème \ref{ThoDivisEuclide} est la \defe{division euclidienne}{division!euclidienne}. Le nombre \( q\) est le \defe{quotient}{quotient} et \( r\) est le \defe{reste}{reste} de la division de \( a\) par \( b\).

% TODO : À propos de restes, il n'est peut-être pas mal de parler d'algorithme de calcul de la date de pâques.
% L'algorithme de Gauss, Meeus utilise des arrondis.
% http://fr.wikipedia.org/wiki/Calcul_de_la_date_de_Pâques


%---------------------------------------------------------------------------------------------------------------------------
\subsection{PGCD, PPCM et Bézout}
%---------------------------------------------------------------------------------------------------------------------------

Pour des versions plus générales, nous avons Bézout dans un anneau principal au théorème \ref{CorimHyXy} et les définitions de PGCD et PPCM dans un anneau intègre à la définition \ref{DefrYwbct}. Voir également le théorème de Bézout pour les polynômes, théorème \ref{ThoBezoutOuGmLB}.

Soient \( p,q\in\eZ^*\). Les ensembles \( p\eZ\cap q\eZ\) et \( p\eZ+q\eZ\) sont des sous-groupes de \( \eZ\). Par conséquent il existe des entiers \( \ppcm(p,q)\) et \( \pgcd(p,q)\) tels que
\begin{subequations}
    \begin{align}
        p\eZ\cap q\eZ&=\ppcm(p,q)\eZ\\
        p\eZ + q\eZ&=\pgcd(p,q)\eZ.
    \end{align}
\end{subequations}

\begin{definition}  \label{DefZHRXooNeWIcB}
    Si \( \pgcd(p,q)=1\), nous disons que \( p\) et \( q\) sont \defe{premiers entre eux}{nombre!premier!deux nombres entre eux}. Si nous avons un ensemble d'entiers \( a_i\), nous disons qu'ils sont premiers \defe{dans leur ensemble}{nombre!premier!dans leur ensemble} si \( 1\) est le PGCD de tous les \( a_i\) ensemble.
\end{definition}

Les nombres \( 2\), \( 4\) et \( 7\) ne sont pas premiers deux à deux (à cause de \( 2\) et \( 4\)), mais ils sont premiers dans leur ensemble parce qu'il n'y a pas de diviseurs communs à tout le monde.

\begin{theorem}[Théorème de Bézout\footnote{Il y a une super application ici : \url{https://perso.univ-rennes1.fr/matthieu.romagny/agreg/dvt/mauvais_prix.pdf}.}\cite{LSAmvR}] \label{ThoBuNjam}   
    Deux entiers non nuls \( a,b\in\eZ^*\) sont premiers entre eux si et seulement s'il existe \( u,v\in\eZ\) tels que 
    \begin{equation}
        au+bv=1
    \end{equation}
\end{theorem}
\index{Bézout!nombres entiers}

\begin{proof}
    Soit \( d=\pgcd(a,b)\) et des nombres \( u,v\) tels que \( au+bv=1\). Le PGCD \( d\) divise à la fois \( a\) et \( b\), et donc divise \( au+bv\). Nous en déduisons que \( d\) divise \( 1\) et est par conséquent égal à \( 1\).

    Nous supposons maintenant que \( \pgcd(a,b)=1\) et nous considérons l'ensemble
    \begin{equation}
        E=\{ au+bv\tq u,v\in \eZ \}\cap \eN^*.
    \end{equation}
    C'est à dire l'ensemble des nombres strictement positifs pouvant s'écrire sous la forme \( au+bv\). Cet ensemble est non vide parce qu'il contient par exemple soit \( a\) soit \( -a\). Soit \( m\) le plus petit élément de \( E\) et écrivons
    \begin{equation}    \label{EqMBsfrP}
        m=au_1+bv_1.
    \end{equation}
    Écrivons la division euclidienne de \( a\) par \( m\), c'est à dire
    \begin{equation}
        a=mq+r
    \end{equation}
    avec \( 0\leq r<m\). En remplaçant \( m\) par sa valeur \eqref{EqMBsfrP}, \( a=(au_1+bv_1)q+r\) et 
    \begin{equation}
        r=a(1-u_1q)-bv_1q,
    \end{equation}
    c'est à dire que \( r\in \eZ a+\eZ b\) en même temps que \( 0\leq r<m\), donc \( r=0\) par minimalité de \( m\). Donc \( a\) est divisible par \( m\). 

    De la même façon nous prouvons que \( b\) est divisible par \( m\). Vu que \( m\) divise à la fois \( a\) et \( b\) nous avons \( m=1\).
\end{proof}

\begin{corollary}       \label{CorgEMtLj}
    Soient \( p\) et \( q\) deux entiers premiers entre eux. Nous avons
    \begin{equation}
        p\eZ+q\eZ=\eZ.
    \end{equation}
\end{corollary}

Notons que l'application \( p\eZ+q\eZ\) vers \( \eZ\) n'est évidemment pas injective.

\begin{proof}
    Soit \( x\in \eZ\). Le théorème de Bézout nous donne \( k\) et \( l\) tels que \( kp+lq=1\). Du coup, \( (xk)p+(xl)q=x\).
\end{proof}

La proposition suivante établit que si \( x\) est assez grand, alors il peut même être écrit comme une combinaison de \( p\) et \( q\) à coefficients positifs. Elle sera utilisée pour démontrer que les états apériodiques d'une chaîne de Markov peuvent être atteins à tout moments (assez grand), voir la définition \ref{DefCxvOaT} et ce qui suit.
\begin{proposition}     \label{PropLAbRSE}
    Soient \( a\) et \( b\) deux éléments de \( \eN\) premiers entre eux. Il existe \( N>0\) tel que tout \( x>N\) appartient à \( a\eN+b\eN\).
\end{proposition}

\begin{proof}
    Soient \( a\) et \( b\), premiers entre eux, et \( x\in \eN\). Soit \( p\) positif tel que
    \begin{subequations}
        \begin{numcases}{}
            pa+pb\geq x\\
            p(a+b)-x\leq a+b.
        \end{numcases}
    \end{subequations}
    C'est à dire que \( p(a+b)\) est le multiple supérieur de \( a+b\) le plus proche de \( a+b\). Nous avons alors
    \begin{equation}
        p(a+b)-ra-sb=x
    \end{equation}
    où \( r\) et \( s\) sont donnés par le théorème de Bézout. Il s'agit maintenant de savoir si nous pouvons être assuré d'avoir \( p>r\) et \( q>s\) dès que \( x\) est assez grand. Pour cela nous considérons les nombres \( r_i\) et \( s_i\) définis par
    \begin{equation}
        r_ia+s_ib=i
    \end{equation}
    pour \( i=1,\ldots, a+b\). Nous posons \( r^*=\max\{ r_i \}\), \( s^*=\max\{ s_i   \}\), et \( p^*=\max\{ r^*,s^* \}\). Maintenant si \( x>p^*(a+b)\), alors
    \begin{equation}
        x=p(a+b)-r_ka-s_kb
    \end{equation}
    où \( k=x-p(a+b)\).
\end{proof}


%\begin{proof}
    %Soit \( x\in \eN\) et \( k_1,l_1\in \eN\) les plus petits entiers tels que \( k_1p\geq x/2\) et \( l_1q\geq x/2\). Nous avons alors
    %\begin{equation}
        %x\leq k_1p+l_1q<x+(p+q).
    %\end{equation}
    %Nous posons \( \delta=k_1p+l_1q-x\).
   % 
    %Soient des entiers \( a_i,b_i\) tels que \( a_ip+b_iq=i\). Nous notons
    %\begin{subequations}
        %\begin{align}
            %A=\max\{ a_i\tq i=1,\ldots, k+p \}\\
            %B=\max\{ b_i\tq i=1,\ldots, k+p \}
        %\end{align}
    %\end{subequations}
    %Notons que \( A\) et \( B\) sont donnés uniquement en termes de \( p\) et \( q\). Ils ne sont en aucun cas dépendants de \( x\).
   % 
    %Nous avons
    %\begin{equation}
        %x=k_1p+lq-\delta=(k_1-a_{\delta})p+(l_1+b_{\delta})q
    %\end{equation}
    %avec \( k_1-a_{\delta}\geq k_1-A\) et \( l_1-b_{\delta}\geq l_1-B\). Si \( x\) est suffisamment grand pour avoir \( k_1>A\) et \( l_1>B\), alors la décomposition souhaitée est trouvée.  
%
    %Une borne pour \( x\) est donnée par 
    %\begin{equation}    \label{EqjQpURG}
        %x>\max\{ 2pA,2qB \}.
    %\end{equation}
%\end{proof}

\begin{example}
    Écrivons \( 1000=a\cdot 7+b\cdot 5\) avec \( a,b\in \eN\). D'abord \( 72\cdot 7=504\) et \( 100\cdot 5=500\). Nous avons donc 
    \begin{equation}
        1004=72\cdot 7+100\cdot 5.
    \end{equation}
    Ensuite \( 4=25-21=-3\cdot 7+5\cdot 5\). Au final,
    \begin{equation}
        1000=75\cdot 7+95\cdot 5.
    \end{equation}
\end{example}

\begin{proposition}
    soient \( a,b\in\eZ^*\). Si
    \begin{equation}
        \begin{aligned}[]
            a&=\epsilon\prod_{\text{\( p\) premiers}}p^{\mu(p)}&b&=\epsilon'\prod_{\text{\( p\) premier}}p^{\nu(p)},
        \end{aligned}
    \end{equation}
    alors
    \begin{subequations}
        \begin{align}
            \pgcd(a,b)&=\prod p^{\min\{ \mu(p),\nu(p) \}}\\
            \ppcm(a,b)&=\prod p^{\max\{ \mu(p),\nu(p) \}}
        \end{align}
    \end{subequations}    
\end{proposition}

Cette proposition implique que \( m\leq p^n\) et \( \pgcd(m,p^n)\neq 1\) si et seulement si \( m=qp\) avec \( q\leq p^{n-1}\).

%---------------------------------------------------------------------------------------------------------------------------
\subsection{Calcul effectif du PGCD et de Bézout}
%---------------------------------------------------------------------------------------------------------------------------
\label{subSecIpmnhO}

Source : \cite{BezoutCos}.

Soient \( A\) et \( B\), deux entier disons positifs. Nous allons voir maintenant l'algorithme de \defe{Euclide étendu}{Euclide!algorithme étendu} qui est capable, pour \( A\) et \( B\) donnés, de calculer le \( \pgcd(A,B)\) et un couple de Bézout \( (u,v)\) tel que \( uA+vB=\pgcd(A,B)\). Ce calcul est indispensable si on veut implémenter RSA (\ref{SecEVaFYi}).

Cela se base sur le lemme suivant.

\begin{lemma}       \label{LemiVqita}
    Soient \( A,B\in \eN\) et des nombres \( q\) et \( r\) tels que \( A=qB+r\) avec \( r<B\). Alors \( \pgcd(A,B)=\pgcd(r,B)\).
\end{lemma}

\begin{proof}
    Il suffit de voir que les diviseurs communs de \( A\) et \( B\) sont diviseurs de \( r\) et que les diviseurs communs de \( r\) et \( B\) divisent \( A\).

    Si \( s\) divise \( A\) et \( B\), alors dans l'équation \( \frac{ A }{ s }=\frac{ qB }{ s }+\frac{ r }{ s }\), les termes \( A/s\) et \( qB/s\) sont entiers, donc \( r/s\) doit aussi être entier.

    Inversement, si \( s\) divise \( r\) et \( B\), alors il divise \( qB+r\) et donc \( A\).
\end{proof}

\begin{remark}
    Ce lemme est surtout intéressant lorsque \( A=qB+r\) est la division euclidienne de \( A\) par \( B\).
\end{remark}

%///////////////////////////////////////////////////////////////////////////////////////////////////////////////////////////
\subsubsection{L'idée}
%///////////////////////////////////////////////////////////////////////////////////////////////////////////////////////////

L'algorithme pour calculer \( \pgcd(A,B)\) consiste à écrire la division euclidienne de \( A\) par \( B\) puis celle de \( B\) par \( r\) :
\begin{subequations}
    \begin{align}
        A&=qB+r&&r<B\\
        B&=q'r+r'&&r'<r
    \end{align}
\end{subequations}
et donc \( \pgcd(A,B)=\pgcd(B,r)=\pgcd(r,r')\). Étant donné que les inégalités \( r<B\) et \( r'<r\) sont strictes, en continuant ainsi nous finissons sur zéro, c'est à dire
\begin{equation}
    r_{n-1}=q_nr_n,
\end{equation}
et à ce moment nous avons \( \pgcd(A,B)=\pgcd(r_{n-1},r_n)=r_n\).

%///////////////////////////////////////////////////////////////////////////////////////////////////////////////////////////
\subsubsection{Pour le PGCD}
%///////////////////////////////////////////////////////////////////////////////////////////////////////////////////////////
\index{pgcd!calcul effectif}

Écrivons cela en détail (parce que Bézout, ça va être le même chose en cinq fois plus compliqué). On pose
\begin{subequations}
    \begin{align}
        r_0=A\\
        r_1=B.
    \end{align}
\end{subequations}
Ensuite on écrit la division euclidienne \( A=q_1B+r_2\), c'est à dire \( r_0=q_1r_1+r_2\). Cela donne \( r_2\) et \( q_1\) en termes de \( r_0\) et \( r_1\) :
\begin{equation}
    r_2=r_0-q_1r_1.
\end{equation}
Ensuite, sachant \( r_2\) nous pouvons continuer :
\begin{equation}
    r_1=q_2r_2+r_3
\end{equation}
donne \( q_2\) et \( r_3=r_1-q_2r_2\). On continue avec \( r_2=q_3r_3+r_4\). Tout cela pour poser la suite
\begin{equation}
    \begin{aligned}[]
        r_0&=A\\
        r_1&=B\\
        r_k&=q_{k+1}r_{k+1}+r_{r+2}
    \end{aligned}
\end{equation}
où la troisième ligne est la définition de \( r_{k+2}\) et de \( q_{k+1}\) en fonction de \( r_k\) et \( r_{k+1}\). La suite \( (r_k)\) ainsi construite est strictement décroissante et à chaque étape le lemme \ref{LemiVqita} et le principe de l'algorithme d'Euclide nous donnent
\begin{subequations}
    \begin{numcases}{}
        \pgcd(r_k,r_{k+1})=\pgcd(r_{k+1},r_{k+2})=\pgcd(A,B)\\
        0\leq r_{k+1}<r_k.
    \end{numcases}
\end{subequations}
La suite étant strictement décroissante, nous prenons \( r_n\), le dernier non nul : \( r_{n+1}=0\). Dans ce cas la dernière équation sera
\begin{equation}
    r_{n-1}=q_nr_n
\end{equation}
avec \( \pgcd(A,B)=\pgcd(r_n,r_{n-1})=r_n\). 

\begin{example}
    Calculons le PGCD de \( 18\) et \( 231\). Pour cela nous écrivons les divisions euclidiennes en chaîne :
    \begin{subequations}
        \begin{align}
            231&=18\cdot 12+15\\
            18&=1\cdot 15 + 3\\
            15&=5\cdot 5+0. 
        \end{align}
    \end{subequations}
    Donc le PGCD est \( 3\).
\end{example}

%///////////////////////////////////////////////////////////////////////////////////////////////////////////////////////////
\subsubsection{Bézout : calcul effectif}
%///////////////////////////////////////////////////////////////////////////////////////////////////////////////////////////
\index{Bézout!calcul effectif}

La difficulté est de construire la suite qui donne Bézout. Elle va être construite à l'envers. Nous supposons déjà connaître la liste complète des \( r_k\) jusqu'à \( r_n=\pgcd(A,B)\), ainsi que la liste complète des divisions euclidiennes
\begin{equation}
    r_k=q_{k+1}r_{k+1}+r_{k+2}.
\end{equation}

Nous voulons trouver les couples \( (u_k,v_k)\) de telle façon à avoir à chaque étape
\begin{equation}
    r_n=u_kr_k+v_kr_{k-1}.
\end{equation}
Notons que c'est à chaque fois \( r_n\) que nous construisons. La première équation de type Bézout à résoudre est 
\begin{equation}
    r_n=u_nr_n+v_nr_{n-1},
\end{equation}
sachant que \( r_{n-1}=q_nr_n\). On pose \( v_n=0\) et \( u_n=1\) et c'est bon. Pour la récurrence, nous égalisons les deux expressions pour \( r_n\) :
\begin{equation}
    r_n=u_kr_k+v_kr_{k-1}=u_{k-1}r_{k-1}+v_{k-1}r_{k-2}
\end{equation}
dans laquelle nous substituons \( r_{k-2}=q_{k-1}r_{k-1}+r_k\) et nous égalisons les coefficients de \( r_k\) et \( r_{k-1}\) :
\begin{equation}
    u_kr_k+v_kr_{k-1}=u_{k-1}r_{k-1}+v_{k-1}(q_{k-1}r_{k-1}+r_k),
\end{equation}
cela donne
\begin{subequations}
    \begin{numcases}{}
        v_{k-1}=u_k\\
        u_{k-1}=v_k-v_{k-1}q_{k-1}.
    \end{numcases}
\end{subequations}
Dès que \( u_k\) et \( v_k\) ainsi que \( q_{k-1}\) sont connus, on peut calculer \( u_{k-1}\) et \( v_{k-1}\). 

La dernière équation, celle avec \( k=1\), est
\begin{equation}
    r_n=u_1r_1+v_1r_0,
\end{equation}
c'est à dire
\begin{equation}        \label{EqNDMLooDvaiAc}
    \pgcd(A,B)=u_1B+v_1A.
\end{equation}
Nous avons ainsi résolu Bézout.

\begin{lemma}[Lemme de Gauss]    \label{LemPRuUrsD}
    Soient \( a,b,c\in \eZ\) tels que \( a\) divise \( bc\). Si \( a\) est premier avec \( c\), alors \( a\) divise \( b\).
\end{lemma}
\index{lemme!de Gauss!pour des entiers}

\begin{proof}
    Vu que \( a\) est premier avec \( c\), nous avons \( \pgcd(a,c)=1\) et Bézout (\ref{ThoBuNjam}) nous donne donc \( s,t\in \eZ\) tels que \( sa+tc=1\). En multipliant par \( b\), nous avons $sab+tbc=b$. Mais les deux termes du membre de gauche sont multiples de \( a\) parce que \( a\) divise \( bc\). Par conséquent \( b\) est somme de deux multiples de \( a\) et donc est multiple de \( a\).
\end{proof}

\begin{lemma}[Lemme d'Euclide\cite{BTDWooZCyXfb}]       \label{LemAXINooOeuMJZ}
    Si un nombre premier $p$ divise le produit de deux nombres entiers $b$ et $c$, alors $p$ divise $b$ ou $c$.
\end{lemma}
\index{Euclide!lemme}

\begin{proof}
    Vu que \( p\) est premier, s'il ne divise pas \( a\) c'est que \( \pgcd(a,p)=1\). Dans ce cas le lemme de Gauss \ref{LemPRuUrsD} implique que \( p\) divise \( b\).
\end{proof}
\index{lemme!d'Euclide}

%--------------------------------------------------------------------------------------------------------------------------- 
\subsection{Écriture des fractions}
%---------------------------------------------------------------------------------------------------------------------------

\begin{theorem}     \label{THOooWYQVooRBaAAM}
    Tout élément de \( \eQ^+\) s'écrit de façon unique comme quotient de deux entiers premiers entre eux.
\end{theorem}

\begin{proof}
    En deux parties
    \begin{subproof}
        \item[Unicité]
            Supposons avoir \( \frac{ a }{ b }=\frac{ c }{ d }\) avec \( \pgcd(a,b)=\pgcd(d,d)=1\). Nous avons 
            \begin{equation}
                ad=bc
            \end{equation}
            donc
            \begin{enumerate}
                \item
                    \( a\) divise \( bc\) mais est premier avec \( b\) donc \( a\) divise \( c\) par le lemme de Gauss \ref{LemPRuUrsD}.
                \item
                    \( c\) divise \( ad\) mais est premier avec \( d\) donc \( c\) divise \( a\) par le lemme de Gauss \ref{LemPRuUrsD}.
            \end{enumerate}
            En conclusion \( a\) divise \( c\) et \( c\) divise \( a\), ergo \( a=c\). L'égalité \( b=d\) est alors immédiate.
        \item[Existence]
            Soit le quotient \( \frac{ a }{ b }\). Nous avons
            \begin{equation}
                \frac{ a }{ b }=\frac{ a/\pgcd(a,b) }{ b/\pgcd(a,b) },
            \end{equation}
            qui est encore un quotient d'entiers parce que \( \pgcd(a,b)\) divise aussi bien \( a\) que \( b\). Il faut montrer que les nombres \( a/\pgcd(a,b)\) et \( b/\pgcd(a,b)\) sont premiers entre eux. Pour cela nous supposons que \( k\) est un diviseur commun. En particulier, les nombres \( a/k\pgcd(a,b)\) et \( b/k\pgcd(a,b)\) sont des entiers, ce qui fait que \( k\pgcd(a,b)\) est un diviseur commun de \( a\) et \( b\). Étant donné que \( \pgcd(a,b)\) est le plus grand tel diviseur, nous devons avoir \( k\pgcd(a,b)=\pgcd(a,b)\) c'est à dire que \( k=1\). Donc les nombres \( a/\pgcd(a,b)\) et \( b/\pgcd(a,b)\) sont premiers entre eux.
    \end{subproof}
\end{proof}

\begin{proposition}     \label{PROPooRZDDooLJabov}
    Les entiers \( p\) et \( q\) sont premiers entre eux si et seulement si \( p^2\) et \( q^2\) sont premiers entre eux.
\end{proposition}

\begin{proof}
    Si \( p^2\) et \( q^2\) sont premiers entre eux, par le théorème de Bézout \ref{ThoBuNjam} il exist e\( a,b\in \eZ\) tels que
    \begin{equation}
        ap^2+bq^2=1
    \end{equation}
    Dans ce cas, \( (ap)p+(bq)q=1\), ce qui montre (par encore Bézout, mais l'autre sens) que \( p\) et \( q\) sont premiers entre eux.

    Réciproquement, supposons que \( p\) et \( q\) ne sont pas premiers entre eux. Alors \( \pgcd(p,q)=k\neq 1\). L'entier \( k\) divise \( p\) et donc \( p^2\); et l'entier \( k\) divise \( q\) et donc \( q^2\). Au final, \( k\) divise \( p^2\) et \( q^2\), ce qui montre que \( p^2\) et \( q^2\) ne sont pas premiers entre eux.
\end{proof}

Une des conséquences de ces résultats sera le fait que \( \sqrt{n}\) est irrationnelle dès que \( n\) n'est pas un carré parfait, théorème \ref{THOooYXJIooWcbnbm}.

%--------------------------------------------------------------------------------------------------------------------------- 
\subsection{Équation diophantienne linéaire à deux inconnues}
%---------------------------------------------------------------------------------------------------------------------------
\label{subsecZVKNooXNjPSf}

\index{équation!diophantienne}


Soient \( a\), \( b\) et \( c\) des entiers naturels donnés. Nous considérons l'équation
\begin{equation}        \label{EqTOVSooJbxlIq}
    ax+by=c
\end{equation}
à résoudre\cite{PAYUooYVuNAB} pour \( (x,y)\in \eN^2\).

Si \( a\) ou \( b\) est nul, c'est facile; nous supposons donc que \( a\) et \( b\) sont tout deux non nuls. Nous commençons par simplifier l'équation en cherchant les diviseurs communs. Soit \( d=\pgcd(a,b)\) et notons \( a=da'\), \( b=db'\). Nous avons déjà l'équation
\begin{equation}
    d(a'x+b'y)=c,
\end{equation}
et donc si \( c\) n'est pas un multiple de \( d\), il n'y a pas de solutions\footnote{Exemple : \( 8x+2y=9\). Le membre de gauche est certainement un nombre pair et il n'y a donc pas de solutions.}. Si par contre \( c\) est un multiple de \( d\) alors nous écrivons \( c=c'd\) et l'équation devient
\begin{equation}
    a'x+b'y=c'
\end{equation}
C'est maintenant que nous utilisons le théorème de Bézout \ref{ThoBuNjam} : vu que \( a'\) et \( b'\) sont premiers entre eux, nous avons la relation  \( a'u+b'v=1\) pour certains\footnote{Nous avons décrit un algorithme pour les trouver en démontrant l'équation \ref{EqNDMLooDvaiAc}.} nombres entiers \( u\) et \( v\). Nous récrivons notre équation sous la forme \( a'x+b'y=c'(a'u+b'v)\) et rassemblons les termes :
\begin{equation}
    a'(x-c'u)=b'(c'v-y).
\end{equation}
C'est à dire que si \( (x,y)\) est une solution, alors \( a'\) divise \( b'(c'v-y)\). Mais comme \( a'\) et \( b'\) sont premiers entre eux, le nombre \( a'\) doit forcément\footnote{C'est Gauss \ref{LemPRuUrsD} qui le dit, et vous savez que lorsque Gauss dit, c'est \emph{forcément} vrai.} diviser \( c'v-y\). Disons \( c'v-y=ka'\). Alors \( a'(x-c'u)=b'ka'\) et donc
\begin{equation}
    x=b'k+c'u.
\end{equation}
Nous trouvons alors une expression pour \( y\) en injectant cela dans  \( a'x+b'y=c'\) et en utilisant Bézout : \( a'c'u=(1-b'v)c'\). Au final nous avons prouvé que toutes les solutions sont de la forme
\begin{subequations}            \label{EqYCQVooZzHuRq}
    \begin{numcases}{}
        x=b'k+c'u\\
        y=vc'-a'k
    \end{numcases}
\end{subequations}
avec \( k\in\eZ\). Si nous voulons réellement seulement des solutions dans \( \eN\) et non dans \( \eZ\), il faut seulement un peu restreindre les valeurs de \( k\). Il en reste un nombre fini parce que l'équation pour \( x\) borne \( k\) vers le bas tandis que celle pour \( y\) borne \( k\) vers le haut.

Par ailleurs, il est très vite vérifié que tous les couples \( (x,y)\) de la forme \eqref{EqYCQVooZzHuRq} sont solutions.

\begin{example}
    Résoudre l'équation \( 2x+6y=52\).

    Nous pouvons factoriser \( 2\) dans le membre de gauche et simplifier alors toute l'équation par \( 2\) :
    \begin{equation}
        x+3y=26.
    \end{equation}
    Nous cherchons une relation de Bézout pour \( u+3v=1\). Ce n'est heureusement pas très compliqué : \( u=-5\), \( v=2\). Nous pouvons alors écrire
    \begin{equation}
        x+3y=26\times (-5+3\times 2),
    \end{equation}
    et donc \( x+5\times 26=3(y-26\times 6)\), et en posant \( k=y-26\times 6\) nous avons
    \begin{equation}
        x=3k-130.
    \end{equation}
    En injectant nous trouvons l'équation \( 3k-130+3y=26\) et donc
    \begin{equation}
        y=52-k.
    \end{equation}
\end{example}

%---------------------------------------------------------------------------------------------------------------------------
\subsection{Quotients}
%---------------------------------------------------------------------------------------------------------------------------

Nous écrivons \( a=b\mod p\) essentiellement s'il existe \( k\in \eZ\) tel que \( b+kp=a\). Plus généralement nous notons \( [a]_p=\{ a+kp|k\in \eZ \}\)\nomenclature[R]{\( [a]_p\)}{ensemble des \( a+kp\)} et l'écriture «\( a=n\mod p\)» peut tout autant signifier \( a=[b]_p\) que \( a\in [b]_p\). La différence entre les deux est subtile mais peut de temps en temps valoir son pesant d'or.

\begin{proposition}
    Soit \( n\in\eN\). Le groupe \( \eZ/n\eZ\) est monogène. Si \( n\neq 0\), le groupe \( \eZ/n\eZ\) est cyclique d'ordre \( n\).
\end{proposition}

\begin{proof}
    Nous considérons la surjection canonique \( \mu\colon \eZ\to \eZ/n\eZ\). Si \( a\in\eZ\), alors \( \mu(a)=a\mu(1)\). Par conséquent \( \eZ/n\eZ=\gr\big( \mu(1) \big)\) parce que tout groupe contenant \( \mu(1)\) contient tous les multiples de \( \mu(1)\), et par conséquent contient \( \mu(\eZ)=\eZ/n\eZ\).

    Soit \( x\in\eZ/n\eZ\) et considérons \( x_0\), le plus petit naturel représentant \( x\). Nous notons \( x=[x_0]\). Le théorème de la division euclidienne \ref{ThoDivisEuclide} donne l'existence de \( q\) et \( r\) avec \( 0\leq r<n\) et \( q\geq 0\) tels que
    \begin{equation}
        x_0=nq+r.
    \end{equation}
    Nous avons \( [x_0]=[r]=\mu(r)\) parce que \( x_0-r\) est un multiple de \( n\). Nous avons donc \( [x_0]\in\mu(\eN_{n-1})\). Par conséquent
    \begin{equation}
        \eZ/n\eZ=\mu(\eZ)=\mu(\eN_{n-1}).
    \end{equation}
    La restriction \( \mu\colon \eN_{n-1}\to \eZ/n\eZ\) est donc surjective. Montrons qu'elle est également injective. Si \( \mu(x_0)=\mu(x_1)\), alors \( x_1=x_0+kn\). Si nous supposons que \( x_1>x_0\), alors \( k>0\) et si \( x_0\in\eN_{n-1}\), alors \( x_1>n-1\).

    L'ordre de \( \eZ/n\eZ\) est donc le même que le cardinal de \( \eN_{n-1}\), c'est à dire \( n\). Le groupe \( \eZ/n\eZ\) est donc fini, d'ordre \( n\) et monogène parce que \( \eZ/n\eZ=\gr(\mu(1))\). Il est donc cyclique.
\end{proof}

\begin{lemma}[\cite{KXjFWKA}]
    Soit \( q\in \eN\) avec \( q\geq 2\). Soient \( n,d\in \eN\) tels que \( q^d-1\divides q^n-1\). Alors \( d\divides n\).
\end{lemma}

\begin{proof}
    Par la divisions euclidienne \ref{ThoDivisEuclide} il existe \( a,b\in \eZ\) tels que \( n=ad+b\) avec \( 0\leq b<d\). En remarquant que \( q^d\in[1]_{q^d-1}\) nous avons
    \begin{equation}
        q^n=(q^d)^aq^b\in[1]_{q^d-1}q^b=[q^b]_{q^d-1}.
    \end{equation}
    Pour cela nous avons utilisé d'abord le fait que si \( a\in [z]_k\), alors \( a^n\in[z^n]_k\) et ensuite le fait que \( [1]_kx=[x]_k\). D'autre part l'hypothèse \( q^d-1\divides q^n-1\) implique 
    \begin{equation}
        q^n\in[1]_{q^d-1}.
    \end{equation}
    Par conséquent le nombre \( q^n\) est à la fois dans \( [q^b]_{q^d-1}\) et dans \( [1]_{q^d-1}\). Cela implique que
    \begin{equation}
        [1]_{q^d-1}=[q^b]_{q^d-1},
    \end{equation}
    ou encore que \( q^b\in[1]_{q^d-1}\) ou encore que \( q^d-1\divides q^b-1\).

    Étant donné que \( b<d\) et que \( q\geq 2\), nous avons que \( q^b-1<q^d-1\); donc pour que \( q^d-1\) divise \( q^b-1\), il faut que \( q^b-1\) soit zéro, c'est à dire \( b=0\). 

    Mais dire \( b=0\) revient à dire que \( d\divides n\), ce qu'il fallait démontrer.
\end{proof}

%+++++++++++++++++++++++++++++++++++++++++++++++++++++++++++++++++++++++++++++++++++++++++++++++++++++++++++++++++++++++++++
\section{Indice d'un sous-groupe et ordre des éléments}
%+++++++++++++++++++++++++++++++++++++++++++++++++++++++++++++++++++++++++++++++++++++++++++++++++++++++++++++++++++++++++++

Soit \( G\) un groupe fini et \( H\), un sous-groupe. L'\defe{indice}{indice} de \( H\) dans \( G\) est le nombre \( | G |/| H |\), souvent noté \( | G:H |\). Le théorème de Lagrange dira en particulier que l'indice est toujours un nombre entier. C'est à ne pas confondre avec le degré d'une extension de corps (définition \ref{DefUYiyieu}).


\begin{theorem}[Théorème de Lagrange]\index{théorème!Lagrange}      \label{ThoLagrange}
    Soit \( H\) un sous-groupe du groupe fini \( G\).  Alors
    \begin{enumerate}
        \item
    L'ordre de \( H\) divise l'ordre de \( G\).
\item 
    Les trois nombres suivants sont égaux :
    \begin{itemize}
        \item
            le nombre de classes de \( H\) à gauche,
        \item
            le nombre de classes de \( H\) à droite,
        \item
            l'indice de \( H\) dans \( G\).
    \end{itemize}
    \end{enumerate}
    En particulier si \( H\) est distingué dans \( G\) nous avons
    \begin{equation}
        | G/H |=\frac{ | G | }{ | H | }.
    \end{equation}
\end{theorem}

\begin{proof}
    Nous commençons par montrer que les classes de \( H\) ont toutes les même nombre d'éléments que \( H\). En effet pour chaque \( g\in G\) nous avons la bijection
    \begin{equation}
        \begin{aligned}
            \varphi\colon H&\to gH \\
            h&\mapsto gh. 
        \end{aligned}
    \end{equation}
    L'injectivité de \( \varphi\) est le fait que \( gh=gh'\) implique \( h=h'\). La surjectivité est par définition de la classe. 

    Les classes à gauche formant une partition de \( G\), le cardinal de \( G\) est le produit de la taille des classes par le nombre de classes :
    \begin{equation}
        | G |=| H |\cdot\text{nombre de classes}.
    \end{equation}
    En particulier nous voyons que \( | H |\) divise \( | G |\).

    La dernière formule exprime simplement que \( G/H\) est par définition le nombre de classes de \( H\) à gauche (ou à droite) dans \( G\).
\end{proof}

\begin{corollary}       \label{CorpZItFX}
    L'ordre d'un élément d'un groupe fini divise l'ordre du groupe. En particulier dans un groupe d'ordre \( n\) tous les éléments vérifient \( q^n=e\).
\end{corollary}

\begin{proof}
    Soit \( G\) un groupe fini et considérons le sous-groupe
    \begin{equation}
        H=\{ g^k\tq k\in\eN \}.
    \end{equation}
    Par le théorème de Lagrange, l'ordre de \( H\) divise \( | G |\), mais l'ordre de \( H\) est le plus petit \( k\) tel que \( g^k=e\), c'est à dire l'ordre de \( g\).
\end{proof}

Le lemme suivant indique que sous hypothèse de commutativité, l'ordre d'un élément est une notion multiplicative.
\begin{lemma}[\cite{rqrNyg}]    \label{LemyETtdy}
    Soit \( G\) un groupe et \( a,b\in G\) tels que \( ab=ba\) d'ordres respectivement \( r\) et \( s\), deux nombres premiers entre eux. Alors l'élément \( ab\) est d'ordre \( rs\).
\end{lemma}

\begin{proof}
    Étant donné que \( (ab)^{rs}=a^{rs}b^{rs}=1\), l'ordre de \( ab\) divise \( rs\). Et vu que \( r\) et \( s\) sont premiers entre eux, l'ordre de \( ab\) s'écrit sous la forme \( r_1s_1\) avec \( r_1\divides r\) et \( s_1\divides s\). Nous avons
    \begin{equation}
        a^{r_1s_1}b^{r_1s_1}=(ab)^{r_1s_1}=1,
    \end{equation}
    que nous élevons à la puissance \( r_2\) où \( r_2\) est définit en posant \(r=r_1r_2\) :
    \begin{equation}
        a^{rs_1}b^{rs_1}=1.
    \end{equation}
    Et comme \( a^{rs_1}=1\), nous concluons que \( b^{rs_1}=1\). Donc \( s\divides rs_1\). Par le lemme de Gauss (\ref{LemPRuUrsD}), nous avons en fait \( s\divides s_1\). Vu qu'on a aussi \( s_1\divides s\), nous avons \( s=s_1\).

    Le même type d'argument donne \( r=r_1\), et finalement l'ordre de \( ab\) est \( r_1s_1=rs\).
\end{proof}

\begin{lemma}       \label{LemqAUBYn}
    L'ensemble des ordres d'un groupe \emph{commutatif} est stable par PPCM.

    Autrement dit, si \( x\in G\) est d'ordre \( r\) et si \( y\in G\) est d'ordre \( s\), alors il existe un élément d'ordre \( \ppcm(r,s)\).
\end{lemma}

\begin{proof}
    Soit \( m=\ppcm(r,s)\). Afin d'écrire \( m\) sous une forme pratique, nous considérons les décompositions en facteurs premiers de \( r\) et \( s\) :
    \begin{subequations}
        \begin{align}
            r&=\prod_{i=1}^kp_i^{\alpha_i}\\
            s&=\prod_{i=1}^kp_i^{\beta_i}
        \end{align}
    \end{subequations}
    où \( \{ p_i \}_{i=1\ldots k}\) est l'ensemble des nombres premiers arrivant dans les décompositions de \( r\) et de \( s\). Si nous posons
    \begin{subequations}
        \begin{align}
            r'&=\prod_{\substack{i=1\\\alpha_1>\beta_i}}^kp_i^{\alpha_i}\\
            s'&=\prod_{\substack{i=1\\a_i\leq \beta_i}}^kp_i^{\beta_i},
        \end{align}
    \end{subequations}
    alors \( \ppcm(r,s)=r's'\) et \( r'\) et \( s'\) sont premiers entre eux. L'élément \( x^{r/r'}\) est d'ordre \( r'\) et l'élément \( y^{s/s'}\) est d'ordre \( s'\). Maintenant nous utilisons le fait que \( G\) soit commutatif et le lemme \ref{LemyETtdy} pour conclure que l'ordre de \( x^{r/r'}y^{s/s'}\) est \( r's'=m\).
\end{proof}

\begin{lemma}[\cite{Combes}]    \label{LemSkIOOG}
    Un sous-groupe d'indice \( 2\) est un sous-groupe normal.
\end{lemma}

\begin{lemma}[\cite{NielsBMorph}]\label{PropubeiGX}
    Soit \( H\), un sous-groupe normal d'indice \( m\) de \( G\). Alors pour tout \( a\in G\) nous avons \( a^m\in H\).
\end{lemma}

\begin{proof}
    Par définition de l'indice, le groupe \( G/H\) est d'ordre \( m\). Donc si \( [a]\in G/H\), nous avons \( [a]^m=[e]\), ce qui signifie \( [a^m]=[e]\), ou encore \( a^m\in H\).
\end{proof}

\begin{proposition}[\cite{NielsBMorph}]
    Soit un groupe fini \( G\) et \( H\), un sous-groupe normal d'ordre \( n\) et d'indice \( m\) avec \( m\) et \( n\) premiers entre eux. Alors \( H\) est l'unique sous-groupe de \( G\) à être d'ordre \( n\).
\end{proposition}
Notons que cette proposition ne dit pas qu'il existe un sous-groupe d'ordre \( n\) et d'indice \( m\). Il dit juste que s'il y en a un et s'il est normal, alors il n'y en a pas d'autres.

\begin{proof}
    Soit \( H'\) un sous-groupe d'ordre \( n\). Si \( h\in H'\) alors \( h^n=1\) et \( h^m\in H\) (lemme \ref{PropubeiGX}). Étant donné que \( m\) et \( n\) sont premiers entre eux, il existe \( a,b\in \eZ\) tels que (Bézout, théorème \ref{ThoBuNjam})
    \begin{equation}
        am+bn=1.
    \end{equation}
    Du coup \( h=h^1=(h^m)^a(h^n)^b\). En tenant compte du fait que \( a^n=1\) et \( h^m\in H\), nous avons \( h\in H\). Ce que nous venons de prouver est que \( H'\subset H\) et donc que \( H=H'\) parce que \( | H' |=| H |=| G |/m\).
\end{proof}

%+++++++++++++++++++++++++++++++++++++++++++++++++++++++++++++++++++++++++++++++++++++++++++++++++++++++++++++++++++++++++++ 
\section{Suite de composition}
%+++++++++++++++++++++++++++++++++++++++++++++++++++++++++++++++++++++++++++++++++++++++++++++++++++++++++++++++++++++++++++
\index{sous-groupe!normal}\index{groupe!quotient}\index{quotient!de groupe}

%TODO : citer la page de la wikiversité sur Jordan-Hölder.
%TODO : donner la définition d'un raffinement de suite de composition.

Sources : \cite{NjCCfW,jxWKGB}.

\begin{definition}  \label{DefJWZSooNcntfK}
Soit \( G\) un groupe. Une \defe{suite de composition}{composition!suite de}\index{suite!de composition} pour \( G\) est une suite finie de sous-groupes \( (G_i)_{i=0,\ldots, n}\) telle que
\begin{equation}
    \{ e \}=G_n\subseteq G_{n-1}\subseteq\ldots\subseteq G_1\subseteq G_0=G
\end{equation}
et telle que \( G_{i+1}\) est normal\footnote{Nous rappelons au cas où que «normal» signifie «distingué».} dans \( G_i\). Les groupes \( G_i/G_{i+1}\) sont les \defe{quotients}{quotient!dans une suite de composition} de la suite de composition.

    Une suite de \defe{Jordan-Hölder}{suite!de Jordan-Hölder}\index{Jordan-Hölder} est une suite de composition dont tous les quotients sont simples.
\end{definition}
L'objet de nos prochaines pérégrinations mathématiques est de montrer que tout groupe fini admet une suite de Jordan-Hölder (théorème \ref{ThoLgxWIC}).

\begin{lemma}[du papillon\cite{NjCCfW}]\label{LemsKpXCG}
    Soit \( G\) un groupe et des sous-groupes \( A\) et \( B\). Soit \( A'\) normal dans \( A\) et \( B'\) normal dans \( B\). Alors
    \begin{enumerate}
        \item
            \( A'(A\cap B')\) est normal dans \( A'(A\cap B)\)
        \item
            \( (A'\cap B)B'\) est normal dans \( (A\cap B)B'\)
        \item
            Nous avons les isomorphismes de groupes
            \begin{equation}
                \frac{ A'(A\cap B) }{ A'(A\cap B') }\simeq\frac{ (A\cap B)B' }{ (A'\cap B)B' }\simeq\frac{ B'(A\cap B) }{ B'(A'\cap B) }.
            \end{equation}
    \end{enumerate}
\end{lemma}

\begin{proof}
    Nous n'allons pas démontrer chacun des points; nous renvoyons au fameux «preuve très similaire dans les autres cas» pour plus de justifications.

    Commençons par montrer que \( A'(A\cap B')\) est un groupe. Si \( a,b\in A'\) et \( x,y\in A\cap B'\),
    \begin{equation}
        axby=xx^{-1}axbx^{-1}xy
    \end{equation}
    En utilisant la normalité, \( x^{-1}ax\in A'\), donc \( xx^{-1}axbx^{-1}\in A'\) et donc le tout est dans \( A'(A\cap B')\). L'ensemble \( A'(A\cap B')\) est également stable pour l'inverse parce que
    \begin{equation}
        x^{-1}a^{-1}=\underbrace{x^{-1}a^{-1}x}_{\in A'}x^{-1}.
    \end{equation}
    
    Nous montrons maintenant que \( A'(A\cap B')\) est normal dans \( A'(A\cap B)\). Soient \( a,b\in A'\), \( x\in A\cap B'\) et \( f\in A\cap B\). Alors
    \begin{subequations}
        \begin{align}
        (bf)^{-1}(ax)(bf)&=(bf)^{-1}(a\underbrace{xbx^{-1}}_{=c\in A'}xf)\\
        &=f^{-1}b^{-1}acxf\\
        &=f^{-1}b^{-1}acf\underbrace{f^{-1}xf}_{=y\in A\cap B'}\\
        &=\underbrace{f^{-1}b^{-1}acf}_{\in A'}y\\
        &\in A'(A\cap B').
        \end{align}
    \end{subequations}
    
    Pour prouver l'isomorphisme
    \begin{equation}
        \frac{ A'(A\cap B) }{ A'(A\cap B') }=\frac{ (A\cap B)B' }{ (A'\cap B)B' },
    \end{equation}
    nous allons utiliser le deuxième théorème d'isomorphisme (\ref{ThoezgBep}\ref{ItembgDQEN}). Que nous appliquons à \( H=A\cap B\) et \( N=A'(A\cap B')\). La vérification que \( H\) normalise \( N\) est usuelle. Nous commençons par écrire
    \begin{equation}    \label{EqkphNsE}
        \frac{ A'(A\cap B')(A\cap B) }{ A'(A\cap B') }\simeq\frac{ A\cap B }{ A\cap B\cap A'(A\cap B') }.
    \end{equation}
    Pour simplifier un peu cette expression nous prouvons d'abord que
    \begin{equation}    \label{EqkhsyNh}
        (A\cap B)\cap A'(A\cap B')=(A'\cap B)(A\cap B').
    \end{equation}
    L'inclusion \( \supset\) est facile. Pour l'autre sens, étant donné que \( A'(A\cap B')\subset A\) nous avons
    \begin{equation}
        A\cap B\cap A'(A\cap B)=B\cap A'(A\cap B).
    \end{equation}
    Un élément de \( B\cap A'(A\cap B)\) est un élément de \(   B\) qui s'écrit sous la forme \( s=ax\) avec \( a\in A'\) et \( x\in A\cap B'\). Nous avons alors \( a=sx^{-1}\) avec \( s\in B\) et \( x^{-1} \in B'\). Par conséquent \( a\in B\) et donc \( a\in A'\cap B\). Nous avons donc
    \begin{equation}
        (A\cap B)\cap A'(A\cap B')=B\cap A'(A\cap B)\subset (A'\cap B)(A\cap B'),
    \end{equation}
    et donc l'égalité \eqref{EqkhsyNh}. Toujours dans l'idée de simplifier \eqref{EqkphNsE} nous remarquons que \( A\cap B'\) est un sous-ensemble de \( A\cap B'\), donc \( A'(A\cap B')(A\cap B)=A'(A\cap B)\). Il reste donc
    \begin{equation}
        \frac{ A'(A\cap B) }{ A'(A\cap B') }=\frac{ A\cap B }{ (A'\cap B)(A\cap B') }.
    \end{equation}
    Étant donné que les hypothèses sur \( A\) et \( B\) sont symétriques, le membre de droite peut aussi s'écrire en inversant \( A\) et \( B\). Nous en sommes à
    \begin{equation}
        \frac{ B'(A\cap B) }{ B'(A'\cap B) }=\frac{ A'(A\cap B) }{ A'(A\cap B') }.
    \end{equation}
    Nous devons encore justifier \( B'(A\cap B)=(A\cap B)B'\) et \( B'(A'\cap B)=(A'\cap B)B'\). Faisons le premier et laissons le second \href{http://abstrusegoose.com/395}{au lecteur}.
    Si \( b\in B'\) et \( x\in A\cap B\), alors
    \begin{equation}
        bx=x\underbrace{x^{-1}bx}_{\in B'}\in (A\cap B)B'.
    \end{equation}
\end{proof}

\begin{proposition}
    Si \( G\) est un groupe fini et que \( (G_i)\) est une suite de composition pour \( G\), alors l'ordre de \( G\) est le produit des ordres de ses quotients.
\end{proposition}

\begin{proof}
    Étant donné que \( G_{i+1}\) est toujours normal dans \( G_i\), le théorème de Lagrange (\ref{ThoLagrange}) s'applique et nous avons à chaque pas de la suite de composition nous avons
    \begin{equation}
        | \frac{ G_i }{ G_{i+1} } |=\frac{ | G_i | }{ | G_{i+1} | } 
    \end{equation}
    et il suffit d'écrire \( | G |\) de façon télescopique :
    \begin{equation}
        | G |=\prod_{0\leq i\leq n-1}\frac{ | G_i | }{ | G_{i+1} | }
    \end{equation}
\end{proof}

Nous disons que les deux suites de composition \( (G_i)_{0\leq i\leq r}\) et \( (G_j)_{0\leq j\leq s}\) sont \defe{équivalentes}{equivalence@équivalence!suite de composition} si \( r=s\) et s'il existe une permutation \( \sigma\in S_{r-1}\) telle que
\begin{equation}
    \frac{ G_i }{ G_{i+1} }\simeq\frac{ H_{\sigma(i)} }{ H_{\sigma(i)+1} }.
\end{equation}

\begin{proposition}[Schreider]\index{lemme!de Schreider}
    Deux suites de composition d'un même groupe admettent des raffinements équivalents.
\end{proposition}

\begin{proof}
    Soient les suites de composition
    \begin{subequations}
        \begin{align}
            \{ e \}=G_m\subseteq\ldots\subseteq G_1\subseteq G_0=G\\
            \{ e \}=H_m\subseteq\ldots\subseteq H_1\subseteq H_0=G
        \end{align}
    \end{subequations}
    Nous raffinons la suite \( (G_i)\) en remplaçant \( G_{i+1}\subseteq G_i\) par
    \begin{equation}
        G_{i+1}=G_{i+1}(G_i\cap H_n)\subset G_{i+1}(G_i\cap H_{n-1})\subseteq\ldots\subseteq G_{i+1}(G_i\cap H_0)=G_i,
    \end{equation}
    et de même pour \( (H_j)\). Le groupe \( G_{i+1}(G_i\cap H_k)\) est normal dans \( G_{i+1}(G_i\cap H_{k-1})\) parce que \( G_{i+1}\) étant normal dans \( G_i\) et \( H_k\) dans \( H_{k-1}\), le lemme \ref{LemsKpXCG} s'applique. Nous avons donc bien défini un raffinement.

    Nous devons maintenant prouver que les deux raffinements ainsi construits sont des suites de composition équivalentes. D'abord elles ont la même longueur \( mn\) parce que chacun des \( m\) éléments de la suite \( (G_i)\) a été remplacé par \( n\) éléments et inversement, chacun de \( n\) éléments de la suite \( (H_j)\) a été remplacé par \( m\) éléments.

    Par ailleurs, les quotients du raffinement de \( (G_i)\) sont de la forme
    \begin{equation}    \label{EqPAYTCB}
        \frac{ G_{i+1}(G_i \cap H_k) }{ G_{i+1}(G_i\cap H_{k+1}) }\simeq \frac{ H_{k+1}(H_k\cap G_i) }{ H_{k+1}(H_k\cap G_{i+1}) }
    \end{equation}
    en vertu du lemme du papillon (\ref{LemsKpXCG}). Le membre de droite de \eqref{EqPAYTCB} est un des quotients du raffinement de \( (H_j)\).
\end{proof}

\begin{lemma}[Schreider strictement décroissant]    \label{LemBSicRJ}
    Soient \( \Sigma_1\) et \( \Sigma_2\), deux suites de composition strictement décroissantes du groupe \( G\). Alors elles admettent des raffinements équivalents strictement décroissants.
\end{lemma}

\begin{proof}
    Par hypothèse, \( \Sigma_1\) et \( \Sigma_2\) n'ont pas de répétitions. Soient \( \Sigma''_1\) et \( \Sigma''_2\), des raffinements équivalents donnés par le lemme de Schreider. Étant donné que ce sont des suites de composition équivalentes, elles ont le même nombre de quotients réduits à \( \{ e \}\), c'est à dire le même nombre de répétitions.

    Les suites \( \Sigma'_1\) et \( \Sigma'_2\) obtenues en retirant les répétitions de \( \Sigma''_1\) et \( \Sigma''_2\) sont des raffinements équivalents de \( \Sigma_1\) et \( \Sigma_2\) et strictement décroissants.
\end{proof}

\begin{theorem}[Jordan-Hölder]\label{ThoLgxWIC}
    Tout groupe fini admet une suite de Jordan-Hölder.

    Deux suites de Jordan-Hölder sont équivalentes.
\end{theorem}
% TODO : trouver une preuve du fait que tout groupe fini admet une suite de Jordan-Hölder.

\begin{proof}
    Nous ne prouvons que le second point.

    Par définition, une suite de Jordan-Hölder n'a pas de raffinement strictement décroissant (à part elle-même) parce que \( G_{i+1}\) est normal maximum dans \( G_i\). Si \( \Sigma_1\) et \( \Sigma_2\) sont des suites de Jordan-Hölder nous pouvons considérer les raffinements équivalents strictement décroissants \( \Sigma'_1\) et \( \Sigma'_2\) du lemme de Schreider \ref{LemBSicRJ}. Nous avons \( \Sigma'_1\sim\Sigma'_2\), mais par ce que nous venons de dire à propos de la maximalité, \( \Sigma'_1=\Sigma_1\) et \( \Sigma'_2=\Sigma_2\). D'où le résultat.
\end{proof}

%+++++++++++++++++++++++++++++++++++++++++++++++++++++++++++++++++++++++++++++++++++++++++++++++++++++++++++++++++++++++++++ 
\section{Groupes résolubles}
%+++++++++++++++++++++++++++++++++++++++++++++++++++++++++++++++++++++++++++++++++++++++++++++++++++++++++++++++++++++++++++

\begin{definition}  \label{DefOSYNooTROIKs}
    Le groupe \( G\) est \defe{résoluble}{groupe!résoluble} s'il existe une suite finie de sous-groupes \( G_i\)
    \begin{equation}
        \{ e \}=G_n\subset G_{n-1}\subset\ldots\subset G_1\subset G_0=G
    \end{equation}
    avec \( G_i\) normal dans \( G_{i+1}\) et \( G_i/G_{i+1}\) abélien.
\end{definition}
Il s'agit d'un groupe qui admet une suite de composition\footnote{Voir définition \ref{DefJWZSooNcntfK}.} dont les quotients sont abéliens.

\begin{lemma}[\cite{HQRooKGAfpu}]   \label{LemOARMooYhYmbH}
    Soit \( G\) un groupe et \( H\) un sous-groupe normal. Le groupe \( G/H\) est abélien si et seulement si \( D(G)\subset H\).
\end{lemma}

\begin{proof}
    Les proposition suivantes sont équivalentes :
    \begin{itemize}
        \item Le groupe \( G/H\) est abélien
        \item pour tout \( x,y\in G\), \( \bar x\bar y=\bar y\bar x\)
        \item $\bar x\bar y\bar x^{-1}\bar y^{-1}=\bar e$
        \item \( \overline{ xyx^{-1}y^{-1} }=\bar e\)
        \item \( [x,y]\in H\)
        \item \( D(G)\subset H\).
    \end{itemize}
\end{proof}

\begin{proposition}[\cite{HQRooKGAfpu}] \label{PropRWYZooTarnmm}
    Un groupe est résoluble si et seulement si sa suite dérivée termine sur \( \{ e \}\).
\end{proposition}

\begin{proof}
    Grâce au lemme \ref{LemMMOCooDJJJhy} et à la proposition \ref{PropAPRGooHBkELf}, si la suite dérivée termine sur \( \{ e \}\) alors la suite dérivé est une suite qui répond aux conditions de la définition \ref{DefOSYNooTROIKs} de groupe résoluble.

    Il faut donc encore montrer le sens direct. Nous supposons que \( G\) est un groupe résoluble et nous étudions sa suite dérivée. Nous avons une suite
    \begin{equation}
        \{ e \}=G_n\subset G_{n-1}\subset\ldots\subset G_1\subset G_0=G
    \end{equation}
    avec \( G_i/G_{i+1}\) abélien et \( G_{i+1}\) normal dans \( G_i\). Nous allons prouver par récurrence que \( D^i(G)\subset G_i\).
    
    Pour \( i=0\) nous avons bien \( G\subset G_0\). Notre hypothèse de récurrence est : 
    \begin{equation}    \label{EqEAQEooEaeIEo}
        D^i(G)\subset G_i
    \end{equation}
    Par le lemme \ref{LemOARMooYhYmbH} nous avons aussi 
    \begin{equation}    \label{EqEDJXooLOLQcr}
        D(G_i)\subset G_{i+1}.
    \end{equation}
    En dérivant \eqref{EqEAQEooEaeIEo} et en tenant compte de \eqref{EqEDJXooLOLQcr}, \( D^{i+1}(G)\subset D(G_i)\subset G_{i+1}\). Donc par récurrence nous avons bien \( D^k(G)\subset G_k\) pour tout \( k\). Mais \( G_r=\{ e \}\) pour un certain \( r\), donc pour ce \( r\) nous avons \( D^r(G)=\{ e \}\), ce qu'il fallait.
\end{proof}

\begin{proposition} \label{PropBNEZooJMDFIB}
    Soient des groupes \( G\) et \( H\). Nous supposons que \( G\) est résoluble et nous considérons un homomorphisme \( \psi\colon G\to H\). Alors \( \psi(G)\) est résoluble.
\end{proposition}

\begin{proof}
    Vu que \( G\) est résoluble, il existe une suite de sous-groupes \( G_i\) tels que
    \begin{equation}
        \{ e \}=G_n\subset G_{n-1}\subset\ldots\subset G_1\subset G_0=G
    \end{equation}
    avec \( G_i\) normal dans \( G_{i+1}\) et \( G_i/G_{i+1}\) abélien. Nous posons \( \psi(G)_i=\psi(G_i)\) et nous avons \( \psi(G)_n=\psi\big( \{ e \} \big)=\{ e \}\) ainsi que \( \psi(G)_0=\psi(G)\); donc
    \begin{equation}
        \{ e \}=\psi(G)_n\subset \psi(G)_{n-1}\subset\ldots\subset \psi(G)_1\subset \psi(G)_0=\psi(G).
    \end{equation}

    Les faits que \( \psi(G)_i\) soit normal dans \( \psi(G)_{i+1}\) et que \( \psi(G)_i/\psi(G)_{i+1}\) soit abélien est directement la proposition \ref{PropSRMJooIDPBoW}.

\end{proof}


%+++++++++++++++++++++++++++++++++++++++++++++++++++++++++++++++++++++++++++++++++++++++++++++++++++++++++++++++++++++++++++
\section{Action de groupes}
%+++++++++++++++++++++++++++++++++++++++++++++++++++++++++++++++++++++++++++++++++++++++++++++++++++++++++++++++++++++++++++

Si \( G\) agit sur un ensemble \( E\), nous notons \( G\cdot x\) l'orbite de \( x\in E\) sous l'action de $G$. Nous notons \( G_x\)\nomenclature[R]{\( G_x\)}{stabilisateur de \( x\)} ou \( \Stab(x)\) le stabilisateur de \( x\) :
\begin{equation}
    G_x=\Stab(x)=\{ g\in G\tq g\cdot x=x \}.
\end{equation}
Pour \( g\in G\), nous notons aussi \( \Fix(g)\) le \defe{fixateur}{fixateur} de \( g\) :
\begin{equation}
    \Fix(g)=\{ x\in E\tq g\cdot x=x \}.
\end{equation}

\begin{definition}  \label{DefuyYJRh}
    L'action de \( G\) sur \( E\) est \defe{fidèle}{fidèle (action)}\index{action!fidèle} si l'identité est le seul élément de \( G\) à fixer tous les points de \( E\), c'est à dire si \( gx=x\,\forall x\in E\Rightarrow g=e\).
\end{definition}

Un exemple d'action fidèle tout à fait non trivial sera donné avec l'action du groupe modulaire sur le plan de Poincaré dans le théorème \ref{ThoItqXCm}.

Le groupe \( G\) agit toujours sur lui même à gauche et à droite. L'action à gauche est \( g\cdot h=gh\); celle à droite est \( g\cdot h=hg^{-1}\). 

\begin{definition}      \label{DEFooCORTooEeOLPT}
    L'action \defe{adjointe}{action!adjointe} définie par \( g\cdot h=ghg^{-1}\) est une manière pour un groupe d'agir sur lui-même par automorphismes. Cela est souvent noté \( \AD(g)h=ghh^{-1}\).
\end{definition}
En effet pour tout \( g\in G\), l'application \( \AD(g)\colon G\to G\) est un automorphisme de \( G\).

Si \( H\) est un sous-groupe de  \( G\), nous notons \( G/H\) le quotient de $G$ par la relation \( g\sim gh\) pour tout \( h\in H\). Lorsque la distinction est importante, nous noterons \( (G/H)_g\)\nomenclature[R]{$(G/H)_g$}{classes à gauche} pour les classes à gauche et \( (G/H)_d\) pour les classes à droite.

Nous avons une relation d'équivalence à gauche et une à droite. D'abord
\begin{equation}
    x\sim_g y\Leftrightarrow xh=y
\end{equation}
pour un certain \( h\in H\). Ensuite
\begin{equation}
    x\sim_d y\Leftrightarrow hx=y
\end{equation}
pour un certain \( h\in H\). 

Le lemme suivant est une généralisation du théorème de Lagrange \ref{ThoLagrange}.

\begin{lemma}
    L'ensemble \( (G/H)_g\) est fini si et seulement si l'ensemble \( (G/H)_d\) est fini. s'il en est ainsi, alors \( (G/H)_g\) et \( (G/H)_d\) ont même cardinal qui vaut l'indice de \( H\) dans \( G\).
\end{lemma}

\begin{proof}
    L'application
    \begin{equation}
        \begin{aligned}
            f\colon (G/H)_g&\to (G/H)_d \\
            [x]_g&\mapsto [x^{-1}]_d 
        \end{aligned}
    \end{equation}
    est une bijection bien définie. En effet si \( x\sim_g y\), nous avons \( h\in H\) tel que \( y^{-1}h=x^{-1}\), c'est à dire que \( x^{-1}\sim_d y^{-1}\) et \( f\) est bien définie. Le fait que \( f\) soit surjective est évident. Pour l'injectivité, soit
    \begin{equation}
        f([x]_g)=f([y]_h).
    \end{equation}
    Alors \( x^{-1}\sim_d y^{-1}\), ce qui implique l'existence de \( h\in H\) tel que \( hx^{-1}=y^{-1}\), ou encore que \( xh^{-1}=y\), ce qui signifie que \( x\sim_gy\).

    Pour l'énoncé à propos de l'indice, nous procédons en plusieurs étapes simples.
    \begin{enumerate}
        \item
            Les classes (les éléments de \( (G/H)_g\)) formes une partition de $G$.
        \item
            Toutes les classes ont le même nombre d'éléments par la bijection 
            \begin{equation}
                \begin{aligned}
                    f\colon [x]_g&\to [y]_g \\
                    xh&\mapsto yh. 
                \end{aligned}
            \end{equation}
        \item
            Le nombre d'éléments dans une classe est égal à \( | H |\) par la bijection
            \begin{equation}
                \begin{aligned}
                    g\colon [x]_g&\to H \\
                    xh&\mapsto h. 
                \end{aligned}
            \end{equation}
    \end{enumerate}
    Par conséquent
    \begin{equation}
        | G |=| H |\cdot \text{nombre de classes}=| H |\cdot\text{cardinal de $(G/H)_g$},
    \end{equation}
    et nous avons bien 
    \begin{equation}
        \text{cardinal de $(G/H)_g$}=\frac{ | G | }{ | H | }=| G:H |.
    \end{equation}
\end{proof}

\begin{proposition}[Orbite-stabilisateur\cite{Combes}]\index{équation!orbite-stabilisateur}     \label{Propszymlr}
    Soit \( G\) un groupe agissant sur un ensemble \( E\) et \( x\in E\).
    \begin{enumerate}
        \item
            Les ensembles \( G\cdot x\) et \( G/G_x\) sont équipotents.
        \item
            L'orbite de \( G_x\) est finie si et seulement si \( G_x\) est d'indice fini dans \( G\). Dans ce cas nous avons 
            \begin{equation}        \label{EqnLCHCE}
                \Card(G\cdot x)=| G:G_x |.
            \end{equation}
            Une autre façon d'écrire la même formule :
            \begin{equation}        \label{EqCewSXT}
                | G |=| \Stab(x) | |\mO_x |.
            \end{equation}
    \end{enumerate}
\end{proposition}
C'est la formule \eqref{EqnLCHCE} qui est nommée \wikipedia{fr}{Action_de_groupe_(mathématiques)\#Formule_des_classes.2C_formule_de_Burnside}{formule des classes} sur wikipédia.

\begin{proof}
    \begin{enumerate}
        \item
    Soit l'application
    \begin{equation}
        \begin{aligned}
            \psi\colon G\cdot x&\to G/G_x \\
            a\cdot x&\mapsto [a]. 
        \end{aligned}
    \end{equation}
    Cette application est bien définie parce que si \( a\cdot x=b\cdot x\), alors il existe \( h\in G_x\) tel que \( b=ah\), et par conséquent \( [a]=[b]\). Cette application est une bijection et par conséquent \( G\cdot x\) est équipotent à \( G/G_x\).
    \item
        Soit \( y\in \mO_x\) et \( A_y=\{ g\in G\tq g\cdot x=y \}\). L'ensemble \( A_y\) est une classe à gauche de \( \Stab(x)\), par conséquent \( | A_y |=|\Stab(x)|\) pour tout \( y\in\mO_x\). Les \( A_y\) pour différents \( y\) sont disjoints et nous avons de plus
        \begin{equation}
            \bigcup_{y\in\mO_x}A_y=G.
        \end{equation}
        Les ensemble \( A_y\) divisent donc \( G\) en \( | \mO_x |\) paquets de \( | \Stab(x) |\) éléments. D'où la formule \eqref{EqCewSXT}.
        
    \end{enumerate}
\end{proof}

\begin{corollary}
    Soit \( C_g\) la classe de conjugaison de l'élément  \( g\) du groupe fini \( G\). Alors
    \begin{equation}
        \Card(C_g)=| G:Z_g |
    \end{equation}
\end{corollary}

\begin{proof}
    Cela est une application directe de la proposition \ref{Propszymlr} dans le cas de l'action adjointe de \( G\) sur lui-même.
\end{proof}

\begin{lemma}
    Soit \( G\) un groupe agissant sur l'ensemble \( E\). On définit \( x\sim x'\) si et seulement s'il existe \( g\in G\) tel que \( g\cdot x=x'\). Alors
    \begin{enumerate}
        \item
            la relation \( \sim\) est une relation d'équivalence.
        \item
            la classe \( [x]\) est l'orbite \( \mO_x\) de \( x\) sous \( G\).
    \end{enumerate}
\end{lemma}

\begin{corollary}[Équation des orbites]\index{équation!des orbites} \label{CorARFVMP}
    Soit \( G\) un groupe agissant sur l'ensemble \( E\) et \( \mO_1,\ldots, \mO_k  \) la liste des orbites (distinctes). Alors
    \begin{enumerate}
        \item
            \( E=\bigcup_i\mO_1\), l'union est disjointe,
        \item
            \( \Card(E)=\sum_i\Card(\mO_i)\).
    \end{enumerate}
\end{corollary}

\begin{definition}  \label{DefcSuYxz}
    Soit \( G\) un groupe agissant sur l'ensemble \( E\). Un \defe{domaine fondamental}{domaine!fondamental d'une action}\index{fondamental!domaine d'une action}\index{action!domaine fondamental} ou une \defe{transversale}{transversale} est une partie de \( E\) contenant un et un seul élément de chaque orbite.
\end{definition}
Autrement dit, les images des éléments d'un domaine fondamental forment une partition de l'ensemble :
\begin{equation}
    E=\bigsqcup_{g\in G}f(F),
\end{equation}
union disjointe, c'est à dire que si \( g\neq g'\), alors \( g(F)\cap g'(F)=\emptyset\).

\begin{proposition}[Équation des classes\cite{FabricegPSFinis}]     \label{PropUyLPdp}
    Soit \( G\), un groupe fini opérant sur un ensemble \( E\). Si \( E''\) est un ensemble contenant exactement un élément de chaque orbite dans \( E\setminus\Stab_G(E)\), alors
    \begin{equation}        \label{EqobuzfK}
        | G |=| \Stab_G(E) |+\sum_{x\in E''}\frac{ | G | }{ | \Stab_G(x) | }.
    \end{equation}
    Si de plus \( G\) est un $p$-groupe, alors 
    \begin{equation}    \label{EqbzLEVJ}
        | E |=| \Stab_G(E) |\mod p.
    \end{equation}
\end{proposition}


\begin{proof}
    Par le corollaire \ref{CorARFVMP}, nous avons \( | G |=\sum_{x\in E'}| \mO_x |\) où \( E'\) est une transversale.  En séparant la somme entre les orbites à un élément et les autres,
    \begin{equation}    \label{EqeggkBs}
        | G |=\Card(\Stab_G(E))+\sum_{x\in E''}\frac{ | G | }{ | \Stab_G(x) | }
    \end{equation}  \label{EqDgYbhm}
    où nous avons utilisé le fait que \( | G |=| \Stab_G(x) | |\mO_x |\).

    Si \( G\) est un \( p\)-groupe alors si \( x\in E''\), \( \Stab_G(x)\) est un sous-groupe propre de \( G\) et donc \( | \Stab_G(x) |\) est un diviseur propre de \( | G |\). Du coup la fraction \( | G |/|\Stab_G(x)|\) est une puissance non nulle de \( p\) et l'équation \eqref{EqobuzfK} devient immédiatement \eqref{EqbzLEVJ}.
\end{proof}
 

\begin{corollary}[Équation des classes]\index{équation!des classes}
    Soit \( G\), un groupe et \( C_1\),\ldots, \( C_l\) la liste de ses classes de conjugaison contenant plus de un éléments. Alors
    \begin{equation}        \label{EqkgGmoq}
        \Card(G)=\Card\big( Z(G) \big)+\sum_i| G:Z_{g_i} |=\Card\big( Z(G) \big)+\sum_i\frac{ \Card(G) }{ \Card\big( \Stab(g_i) \big) }
    \end{equation}
    si \( g_i\in C_i\).
\end{corollary}

\begin{proof}
    Étant donné que les classes de conjugaison sont disjointes, le cardinal de \( G\) est bien la somme des cardinaux de ses classes. Les classes ne contenant que un seul élément sont celles des éléments de \( Z(G)\). En ce qui concerne les autres orbites, \( \Card(C_{g_i})=| G:Z_{g_i} |\) par le théorème orbite-stabilisateur (proposition \ref{Propszymlr}).
\end{proof}

\begin{theorem}[\wikipedia{fr}{Action_de_groupe_(mathématiques)}{Formule de Burnside}]      \label{THOooEFDMooDfosOw}
    Si \( G\) est un groupe fini agissant sur l'ensemble fini \( E\) et si \( \Omega\) est l'ensemble des orbites, alors
    \begin{equation}    \label{EqTUsblv}
        \Card(\Omega)=\frac{1}{ | G | }\sum_{g\in G}\Card\big( \Fix(g) \big).
    \end{equation}
\end{theorem}
\index{Burnisde!formule}
\index{formule!Burnside}

\begin{proof}
    Nous considérons l'ensemble 
    \begin{equation}
        A=\{ (g,x)\in G\times E\tq gx=x \},
    \end{equation}
    et nous en calculons le cardinal de deux façons. D'abord
    \begin{subequations}
        \begin{align}
            \Card(A)&=\sum_{x\in E}\Card\{ g\in g\tq gx=x \}\\
            &=\sum_{x\in E}\Card(\Stab(x))\\
            &=\sum_{\omega\in \Omega}\sum_{x\in \omega}\Card(\Stab(x))\\
            &=\sum_{x\in \omega}\frac{ | G | }{ \Card(\omega) }     \label{EqyVtkyf}\\
            &=| G |.
        \end{align}
    \end{subequations}
    Pour obtenir \eqref{EqyVtkyf} nous avons utilisé l'équation des classes \eqref{EqCewSXT}. L'autre façon de calculer \( \Card(A)\) est de regrouper ainsi :
    \begin{equation}
        \Card(A)=\sum_{g\in G}\Card\{ x\in E\tq gx=x \}=\sum_{g\in G}\Card(\Fix(g)).
    \end{equation}
    En égalisant les deux expressions de \( \Card(A)\) nous trouvons
    \begin{equation}
        | G |\Card(\Omega)=\sum_{g\in G}\Card(\Fix(g)).
    \end{equation}
\end{proof}

Liens internes vers des applications :
\begin{itemize}
    \item Le jeu de la roulette et l'affaire du collier, \ref{pTqJLY} et \ref{siOQlG}.
\end{itemize}

\begin{proposition}
    Soit \( G\) un groupe et \( H\), un sous-groupe du centre de \( G\).
    \begin{enumerate}
        \item
            \( H\) est normal dans \( G\).
        \item
            Si \( G/H\) est monogène, alors \( G\) est abélien.
        \item
            Si \( G\) est fini de centre \( Z\), alors \( | G:H |\) n'est pas premier.
    \end{enumerate}
\end{proposition}

\begin{theorem}
    Soit \( G\) un groupe cyclique d'ordre \( n\).
    \begin{enumerate}
        \item
            Tout sous-groupe de \( G\) est cyclique.
        \item 
            Pour chaque diviseur \( d\) de \( n\), il existe un unique sous-groupe \( H_d\) de \( G\) d'ordre \( d\).
    \end{enumerate}
    Si \( a\) est un générateur de \( G\), alors \( H_d\) peut être décrit des façons suivantes :
    \begin{equation}
        H_d=\{ x\in G\tq x^d=e \}=\{ x\in G\tq\exists y\in G\tq y^{n/d}=x \}=\langle a^{n/d}\rangle.
    \end{equation}
\end{theorem}

\begin{example}     \label{ExemMaKdwt}
    Si \( E\) est un espace vectoriel alors \( (E,+)\) est un groupe commutatif. L'inverse de \( x\) est \( -x\).
\end{example}

\begin{definition}
    Soit \( G\) un groupe agissant sur un ensemble \( E\). Nous disons que l'action est \defe{transitive}{transitive}\index{action!transitive} si elle possède une seule orbite. L'action est \defe{libre}{libre!action}\index{action!libre} si \( g\cdot x=g'\cdot x\) implique \( g=g'\).
\end{definition}
