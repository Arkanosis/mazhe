% This is part of Mes notes de mathématique
% Copyright (c) 2011-2012
%   Laurent Claessens
% See the file fdl-1.3.txt for copying conditions.

Deux textes assez complets : \cite{MesIntProbb,MathAgreg}.

%+++++++++++++++++++++++++++++++++++++++++++++++++++++++++++++++++++++++++++++++++++++++++++++++++++++++++++++++++++++++++++
					\section{Binôme de Newton}
%+++++++++++++++++++++++++++++++++++++++++++++++++++++++++++++++++++++++++++++++++++++++++++++++++++++++++++++++++++++++++++

\begin{proposition}     \label{PropBinomFExOiL}
Pour tout $x$, $y\in\eR$ et $n\in\eN$, nous avons
\begin{equation}		\label{EqNewtonB}
	(x+y)^n=\sum_{k=0}^n{n\choose k}x^{n-k}y^k
\end{equation}
où
\begin{equation}
	{n\choose k}=\frac{ n! }{ k!(n-k)! }
\end{equation}
sont les \defe{coefficients binomiaux}{Coefficients binomiaux}.
\end{proposition}

La preuve qui suit provient de \href{http://fr.wikipedia.org/wiki/Formule_du_binôme_de_Newton}{wikipédia}.
\begin{proof}
La preuve se fait par récurrence. La vérification pour $n=0$ et $n=1$ sont faciles. Supposons que la formule \eqref{EqNewtonB} soit vraie pour $n$, et prouvons la pour $n+1$. Nous avons
\begin{equation}		\label{EqBinTrav}
	\begin{aligned}[]
		(x+y)^{n+1}	&=(x+y)\cdot  \sum_{k=0}^n{n\choose k}x^{n-k}y^k\\
				&= \sum_{k=0}^n{n\choose k}x^{n-k+1}y^k+\sum_{k=0}^n{n\choose k}x^{n-k}y^{k+1}\\
				&=x^{n+1}+ \sum_{k=1}^n{n\choose k}x^{n-k+1}y^k+\sum_{k=0}^{n-1}{n\choose k}x^{n-k}y^{k+1}+y^{n+1}.
	\end{aligned}
\end{equation}
La seconde grande somme peut être transformée en posant $i=k+1$ :
\begin{equation}
	\sum_{k=0}^{n-1}{n\choose k}x^{n-k}y^{k+1}	=\sum_{i=1}^n{n\choose i-1}x^{n-(i-1)}y^{i-1+1},
\end{equation}
dans lequel nous pouvons immédiatement renommer $i$ par $k$. En remplaçant dans la dernière expression de \eqref{EqBinTrav}, nous trouvons
\begin{equation}
	(x+y)^{n+1}=x^{n+1}+y^{n+1}+\sum_{k=1}^n\left[ {n\choose k}+{n\choose k-1} \right]x^{n-k+1}y^k.
\end{equation}
La thèse découle maintenant de la formule
\begin{equation}
	{n\choose k}+{n\choose k-1}={n+1\choose k}
\end{equation}
qui est vraie parce que
\begin{equation}
	\frac{ n! }{ k!(n-k)! }+\frac{ n! }{ (k-1)(n-k+1)! }=\frac{ n!(n-k+1)+n!k }{ k!(n-k+1)! }=\frac{ n!(n+1) }{  k!(n-k+1)!  },
\end{equation}
par simple mise au même dénominateur.
\end{proof}


 \section{Les nombres complexes}
 \subsection{Définitions de base}
 Un nombre complexe s'écrit sous la forme $z = a + b i$, où $a$ et $b$
 sont des nombres réels appelés (et notés) respectivement partie réelle
 ($a = \Re(z)$) et partie imaginaire ($b = \Im(z)$) de $z$. L'ensemble
 des nombres de cette forme s'appelle l'ensemble des nombres complexes
 ; cet ensemble porte une structure de corps et est noté $\CC$. Le
 nombre complexe $i = 0 + 1 i$ est un nombre imaginaire qui a la
 particularité que $i^2 = -1$.

 Deux nombres complexes $a + bi$ et $c + di$ sont égaux si et seulement
 si $a = c$ et $b = d$, c'est-à-dire leurs parties réelles sont égales,
 et leurs parties imaginaires sont égales.

 Un nombre complexe étant représenté par deux nombres, on peut le
 représenter dans un plan appelé « plan de Gauss ». La plupart des
 opérations sur les nombres complexes ont leur interprétation
 géométrique dans ce plan.

 Pour $z = a + bi$ un nombre complexe, on note $\bar z = a - bi$ le
 \Defn{complexe conjugué} de $z$. Dans le plan de Gauss, il s'agit du
 symétrique de $z$ par rapport à la droite réelle (généralement
 dessinée horizontalement).

 On définit le module du complexe $z$ par $\module z = \sqrt{z\bar z} =
 \sqrt{a^2 + b^2}$. Dans le plan de Gauss, il s'agit de la distance
 entre $0$ et $z$.

 \begin{proposition}
Pour tout $z = a+bi$ et $z^\prime$ nombres complexes, on a
   \begin{enumerate}
   \item $z \bar z = a^2 + b^2$;
   \item $\bar{\bar{z}} = z$;
   \item $\module z = \module {\bar z}$;
   \item $\module{zz^\prime} = \module z \module{z^\prime}$;
   \item $\module{z+z^\prime} \leq \module z + \module{z^\prime}$.
   \end{enumerate}
 \end{proposition}

 \subsection{Forme polaire ou trigonométrique}
 Dans le plan de Gauss, le module d'un complexe $z$ représente la
 distance entre $0$ et $z$. On appelle \Defn{argument} de $z$ (noté
 $\arg z$) l'angle (déterminé à $2\pi$ près) entre le demi-axe des
 réels positifs et la demi-droite qui part de $0$ et passe par $z$. Le
 module et l'argument d'un complexe permettent de déterminer
 univoquement ce complexe puisqu'on a la formule
 \[z = a + bi = \module z \left( \cos(\arg(z)) + i \sin(\arg(z))
 \right)\]

 L'argument de $z$ se détermine via les formules
 \[\frac a {\module z} = \cos(\arg(z)) \quad \frac b {\module z} =
 \sin(\arg(z))\] ou encore par la formule
 \[\frac b a = \tan(\arg(z)) \quad \text{en vérifiant le
   quadrant.}\]%
 La vérification du quadrant vient de ce que la tangente ne détermine
 l'angle qu'à $\pi$ près.

%+++++++++++++++++++++++++++++++++++++++++++++++++++++++++++++++++++++++++++++++++++++++++++++++++++++++++++++++++++++++++++
\section{Suites et séries de nombres}
%+++++++++++++++++++++++++++++++++++++++++++++++++++++++++++++++++++++++++++++++++++++++++++++++++++++++++++++++++++++++++++
%\label{secseries}

%---------------------------------------------------------------------------------------------------------------------------
\subsection{Moyenne de Cesaro}
%---------------------------------------------------------------------------------------------------------------------------

Si \( (a_n)_{n\in \eN} \) est une suite dans \( \eR\) ou \( \eC\), alors sa \defe{moyenne de Cesaro}{moyenne!de Cesaro}\index{Cesaro!moyenn} est la limite (si elle existe) de la suite
\begin{equation}
    c_n=\frac{1}{ n }\sum_{k=1}^na_k.
\end{equation}
En un mot, c'est la limite des moyennes partielles.

\begin{lemma}       \label{LemyGjMqM}
    Si la suite \( (a_n)\) converge vers la limite \( \ell\) alors la suite admet une moyenne de Cesaro qui vaudra \( \ell\).
\end{lemma}

\begin{proof}
    Soit \( \epsilon>0\) et \( N\in \eN\) tel que \( | a_n-\ell |<\epsilon\) pour tout \( n>N\). En remarquant que
    \begin{equation}
        \frac{1}{ n }\sum_{k=1}^nk-\ell=\frac{1}{ n }\sum_{k=1}^n(a_k-\ell),
    \end{equation}
    nous avons
    \begin{subequations}
        \begin{align}
            | \frac{1}{ n }\sum_{k=1}^na_k-\ell |&\leq| \frac{1}{ n }\sum_{k=1}^N| a_k-\ell | |+\big| \frac{1}{ n }\sum_{k=N+1}^n\underbrace{| a_k-\ell |}_{\leq \epsilon} \big|\\
            &\leq \epsilon+\frac{ n-N-1 }{ n }\epsilon\\
            &\leq 2\epsilon.
        \end{align}
    \end{subequations}
    Dans ce calcul nous avons redéfinit \( N\) de telle sorte que le premier terme soit inférieur à \( \epsilon\).
\end{proof}


\subsection{Rappels et définitions}

La notion de série formalise le concept de somme infinie. L'absence de certaines propriétés de ces objets (problèmes de commutativité et même d'associativité) incitent à la prudence et montrent à quel point une définition précise est importante.

Soit ${(a_k)}_{k \geq k_0}$ une suite réelle. La \defe{série}{série} de terme général $(a_k)${série}, notée
\begin{math}
  \sum_{i=k_0}^\infty a_i,
\end{math}
est la suite ${(s_n)}_{n \geq k_0}$ dont les termes sont donnés par
\begin{equation*}
  s_n \pardef \sum_{i=k_0}^k a_i
\end{equation*}
et sont appelés les \defe{sommes partielles}{somme!partielle} de la série.

La série $\sum_{i=k_0}^\infty a_i$ \defe{converge}{série!convergence} si la suite $(s_n)$
converge vers un réel $s$. Sa limite est appelée la \defe{somme de la
série}{série!somme} et on note
\begin{equation}		\label{EqDefSommeLim}
  \sum_{i=k_0}^\infty a_i = \lim_{n\to\infty}\sum_{i=k_0}^na_i.
\end{equation}
Si la série ne converge pas, elle \defe{diverge}{série!divergence} et peut alors avoir
une limite infinie (uniquement si le terme général est réel, on note
alors $+\infty$ ou $-\infty$ sa limite) ou pas de limite.  La série
$\sum_{i=k_0}^\infty a_i$ \defe{converge absolument}{convergence!absolue} si la série
$\sum_{i=k_0}^\infty \abs{a_i}$ converge.

\begin{proposition}\label{propnseries_propdebase}
Les principales propriétés de la somme définie par la limite \eqref{EqDefSommeLim} sont
  \begin{enumerate}
  \item Si une série converge absolument, alors elle converge
    (simplement).
  \item Si la série est à termes positifs --c'est-à-dire pour tout
    indice $k$, $a_k \in \eR$ et $a_k \geq 0$-- il n'y a aucune
    différence entre convergence absolue et convergence simple.
  \item\label{point3-seriepropdebase} Si une série converge, son terme général doit tendre vers $0$.
\item 
Si la série converge alors la somme est associative
\item
Si la série converge absolument, alors la somme est commutative.
  \end{enumerate}
\end{proposition}

\begin{remark}Vue comme somme infinie, l'associativité et la
  commutativité dans une série sont perdues. Néanmoins, il subsiste
  que
  \begin{enumerate}
  \item si la série converge, on peut regrouper ses termes sans
    modifier la convergence ni la somme (associativité),
  \item si la série converge absolument, on peut modifier l'ordre des
    termes sans modifier la convergence ni la somme (commutativité).
  \end{enumerate}
\end{remark}

\begin{example}\label{exemplesseries}
\begin{enumerate}

\item
    La \defe{série harmonique}{série!harmonique} est
\begin{equation}
\sum_{i=1}^\infty \frac1i
\end{equation}
et diverge (possède une limite $+\infty$).

\item
    La \defe{série géométrique}{série!géométrique} de raison $q \in \eC$ est
\begin{equation}
\sum_{i=0}^\infty q^i.
\end{equation}
Étudions la somme partielle \( S_N=1+q+q^2+\ldots +q^{n}\). Nous avons évidemment $S_N-zS_N=1-q^{N+1}$ et donc
\begin{equation}
    S_N=\sum_{n=0}^Nq^n=\frac{ 1-q^{N+1} }{ 1-q }.
\end{equation}
La limite \( \lim_{N\to \infty} S_N\) existe si et seulement si \( | q |\leq 1\) et dans ce cas nous avons
\begin{equation}
    \sum_{n=1}^{\infty}z^n=\frac{ 1 }{ 1-z }.
\end{equation}
La convergence est absolue.

\item
Pour $\alpha \in \RR$, la série de Riemann (ou Dirichlet)
\begin{equation}		\label{EqSerRiem}
\sum_{i=1}^\infty \frac1{i^\alpha}
\end{equation}
converge (absolument, puisque réelle et positive) si et seulement
si $\alpha > 1$, et diverge sinon.
\end{enumerate}
\end{example}

\subsection{Critères de convergence absolue}

  Étant donné le terme général d'une série, il est souvent --dans les cas qui nous intéressent-- difficile de déterminer la somme de la série. L'exemple de la série géométrique est particulier, puisqu'on connait une formule pour chaque somme partielle, mais pour l'exemple des séries de Riemann il n'y a aucune formule simple pour un $\alpha$ général. D'où l'intérêt d'avoir des critères de convergence ne nécessitant aucune connaissance de l'éventuelle limite de la série.

\subsubsection{Critère de comparaison} 

\begin{lemma}[Critère de comparaison]   \label{LemgHWyfG}
Soient $\sum_i a_i$ et $\sum_j
b_j$ deux séries à termes positifs vérifiant
\begin{equation*}
  0 \leq a_i \leq b_i
\end{equation*}
alors
\begin{enumerate}
\item si $\sum_i a_i$ diverge, alors $\sum_j b_j$ diverge,
\item si $\sum_j b_j$ converge, alors $\sum_i a_i$ converge
  (absolument).
  \end{enumerate}
\end{lemma}

\subsubsection{Critère d'équivalence}
%\label{PgCritEquiv}

\begin{proposition}[\cite{TrenchRealAnalisys}]
 Soient $\sum_i a_i$ et $\sum_j b_j$ deux séries à termes positifs. Supposons l'existence de la limite (éventuellement infinie) suivante
\begin{equation}
  \limite i \infty \frac{a_i}{b_i} = \alpha \in \RR \text{ ou $\alpha =
    \infty$.}
\end{equation}
Dans ce cas, nous avons
\begin{enumerate}
\item si $\alpha \neq 0$ et $\alpha\neq \infty$, alors
  \begin{equation}
    \sum_i a_i \text{~converge} \ssi \sum_j b_j\text{~converge,}
  \end{equation}
\item si $\alpha = 0$ et $\sum_j b_j$ converge, alors $\sum_i a_i$
  converge (absolument),
\item si $\alpha = +\infty$ et $\sum_j b_j$ diverge, alors $\sum_i
  a_i$ diverge.
\end{enumerate}
\end{proposition}

\begin{proof}
\begin{enumerate}
	\item
		Le fait que la suite $a_n/b_n$ converge vers $\alpha$ signifie que tant sa limite supérieure que sa limite inférieure convergent vers $\alpha$. En particulier la suite $\frac{ a_n }{ b_n }$ est bornée vers le haut et vers le bas. À partir d'un certain rang $N$, il existe $M$ tel que 
		\begin{equation}
			\frac{ a_n }{ b_n }<M
		\end{equation}
		et il existe $m$ tel que
		\begin{equation}
			\frac{ a_n }{ b_n }>m.
		\end{equation}
		Nous avons donc $a_n<Mb_n$ et $a_n>mb_n$. La série de $(a_n)$ converge donc si et seulement si la série de $(b_n)$ converge.
	\item
		Si $\alpha=0$, cela signifie que pour tout $\epsilon$, il existe un rang tel que $\frac{ a_n }{ b_n }<\epsilon$, et donc tel que $a_n<\epsilon b_k$. La suite de $(a_i)$ converge donc dès que la suite de $(b_i)$ converge.
	\item
		Pour tout $M$, il existe un rang dans la suite à partir duquel on a $\frac{ a_i }{ b_i }>M$, et donc $a_k>Mb_k$. Si la série de $(b_k)$ diverge, la série de $(a_k)$ doit également diverger.
\end{enumerate}
\end{proof}

\subsubsection{Critère du quotient}

\begin{proposition}[\cite{KeislerElemCalculus}]
    Soit $\sum_i a_i$ une série. Supposons l'existence de la limite (éventuellement infinie) suivante
    \begin{equation}
      \limite i \infty \abs{\frac{a_{i+1}}{a_i}} = L\in \RR \text{ ou $L =
        \infty$.}
    \end{equation}
    Alors
    \begin{enumerate}
    \item si $L < 1$, la série converge absolument,
    \item si $L > 1$, la série diverge,
    \item si $L = 1$ le critère échoue : il existe des exemple de convergence et des exemples de divergence.
    \end{enumerate}
\end{proposition}

\begin{proof}
\begin{enumerate}
	\item
		Soit $b$ tel que $L<b<1$. À partir d'un certain rang $K$, on a $\left| \frac{ a_{i+1} }{ a_i } \right| <b$. En particulier,
		\begin{equation}
			| a_{K+1} |<b| a_K |,
		\end{equation}
		et pour $a_{K+2}$ nous avons
		\begin{equation}
			| a_{K+2} |<b| a_{K+1} |<b^2| a_K |.
		\end{equation}
		Au final,
		\begin{equation}
			| a_{K+n} |<b^n| a_K |.
		\end{equation}
		Étant donné que la série $\sum_{n\geq K}b^n$ converge (parce que $b<1$), la queue de suite $\sum_{i\geq K}a_i$ converge, et par conséquent la suite au complet converge.
	\item
		Si $L>1$, on a
		\begin{equation}
			| a_K |<| a_{K+1} |<| a_{K+2} |<\ldots
		\end{equation}
		Il est donc impossible que la suite $(a_i)$ converge vers zéro. La série ne peut donc pas converger.
	\item
		Par exemple la suite harmonique $a_n=\frac{1}{ n }$ vérifie $L=1$, mais la série ne converge pas. Par contre, la suite $a_n=\frac{ 1 }{ n^2 }$ vérifie aussi le critère avec $L=1$ tandis que la série $\sum_n\frac{1}{ n^2 }$ converge.
\end{enumerate}
\end{proof}

\subsubsection{Critère de la racine}

\begin{proposition}[\cite{TrenchRealAnalisys}]
    Soit $\sum_i a_i$ une série, et considérons
    \begin{equation*}
      \limsup_{i \rightarrow \infty} \sqrt[i]{\abs{a_i}} = L \in \RR
      \text{ ou $L =
        \infty$.}
    \end{equation*}
    Alors
    \begin{enumerate}
    \item si $L < 1$, la série converge absolument,
    \item si $L> 1$, la série diverge,
    \item si $L = 1$ le critère échoue.
    \end{enumerate}
\end{proposition}

\begin{proof}
    \begin{enumerate}
        \item
            Si $L<1$, il existe un $r\in \mathopen] 0 , 1 \mathclose[$ tel que $| a_n |^{1/n}<r$ pour les grands $n$. Dans ce cas, $| a_n |<r^{n}$, et la série converge absolument parce que la série $\sum_nr^n$ converge du fait que $r<1$.
        \item
            Si $L>1$, il existe un $r>1$ tel que $| a_n |^{1/n}>r>1$. Cela fait que $| a_n |$ prend des valeurs plus grandes que $n$ pour une infinité de termes. Le terme général $a_n$ ne peut donc pas être une suite convergente. Par conséquent la suite diverge au sens où elle ne converge pas.

    \end{enumerate}
\end{proof}

%---------------------------------------------------------------------------------------------------------------------------
\subsection{Critères de convergence simple}
%---------------------------------------------------------------------------------------------------------------------------

Les critères de comparaison, d'équivalence, du quotient et de la racine sont des critères de convergence absolue. Pour conclure à une convergence simple qui n'est pas une convergence absolue, le critère d'Abel sera notre outil principal.  

\subsubsection{Critère d'Abel}

\begin{proposition}[Critère d'Abel]
    Soit la série $\sum_i c_iz_i$ avec
    \begin{enumerate}
        \item $(c_i)$ est une suite réelle décroissante qui tend vers zéro,
        \item $(z_i)$ est une suite dans $\eC$ dont la suite des sommes partielles est bornée dans $\eC$, c'est à dire qu'il existe un $M>0$ tel que pour tout $n$,
        \begin{equation}
            \left| \sum_{i=1}^nz_i \right| \leq M.
        \end{equation}
        Alors la série $\sum_ic_iz_i$ est convergente.
    \end{enumerate}
\end{proposition}
Remarquons que ce critère ne donne pas de convergence absolue.

\begin{corollary}[Critère des séries alternées]\index{critère!série alternée}       \label{CoreMjIfw}
    Si \( (a_n)\) est une suite décroissante à limite nulle, alors la série
  \begin{equation}
    \sum_{n=0}^\infty {(-1)}^n a_n
  \end{equation}
  converge simplement.
\end{corollary}



\section{Continuité et dérivabilité}
\label{seccontetderiv}

On considère dans la suite une fonction $f : A \to \eR$, où $a \in A \subset \eR$ ; cependant, les notions de continuité et de dérivabilité se généralisent immédiatement au cas de fonctions à valeurs vectorielles ; la notion de continuité se généralise au cas des fonctions à plusieurs variables (la notion de dérivabilité est remplacée par celle de différentiabilité dans ce cadre).

\begin{definition}
    La fonction $f$ est \Defn{continue en $a$} si
  \begin{equation*}
    \limite x a f(x)
  \end{equation*}
  existe (et vaut alors $f(a)$).
\end{definition}

\begin{definition}
    La fonction $f$ est \Defn{dérivable en $a$} si $a \in
  \operatorname{int} A$ et si
  \begin{equation*}
    \lim_{\substack{x\rightarrow a\\x\neq a}} \frac{f(x)-f(a)}{x-a}
  \end{equation*}
  existe. On note alors cette quantité $f^\prime(a)$, c'est le nombre
  dérivé de $f$ en $a$. La \Defn{fonction dérivée} de $f$ est
  \begin{equation*}
    f^\prime : A^\prime \to \eR : a \mapsto f^\prime(a)
  \end{equation*}
  définie sur l'ensemble noté $A^\prime$ des points $a$ où $f$ est
  dérivable.
\end{definition}

\begin{definition}
  Une fonction est \Defn{continue} (resp. \Defn{dérivable}) si elle est continue (resp. dérivable) en tout point $a \in A$ de son domaine.
\end{definition}

\begin{example}
      Montrons que la fonction $f : \eR \to \eR : x\mapsto x$ est continue et dérivable. Exceptionnellement (bien qu'on sache que la dérivabilité implique la continuité), montrons ces deux assertions séparément.
      \begin{description}
      \item[Continuité] Pour prouver la continuité au point $a \in \eR$ nous devons montrer que
     \begin{equation}
       \limite x a x = a
     \end{equation}
     c'est-à-dire
     \begin{equation}
       \forall \epsilon > 0, \exists \delta > 0 :  \forall x \in \eR \abs{x-a} <
       \delta \Rightarrow \abs{x-a} < \epsilon
     \end{equation}
     ce qui est clair en prenant $\delta = \epsilon$.

      \item[Dérivabilité] Soit $a \in \eR$. Calculons la limite du quotient différentiel
        \begin{equation}
          \limite[x\neq a]{x}{a} \frac{x-a}{x-a} = \limite[x\neq a]x a 1 = 1
        \end{equation}
        ce qui prouve que $f$ est dérivable et que sa dérivée vaut $1$ en
        tout point $a$ de $\eR$.
      \end{description}

     On a donc montré que la fonction $x \mapsto x$ est continue, dérivable, et que sa dérivée vaut $1$ en tout point $a$ de son domaine.

\end{example}

%+++++++++++++++++++++++++++++++++++++++++++++++++++++++++++++++++++++++++++++++++++++++++++++++++++++++++++++++++++++++++++
\section{Espace des fonctions continues}
%+++++++++++++++++++++++++++++++++++++++++++++++++++++++++++++++++++++++++++++++++++++++++++++++++++++++++++++++++++++++++++

\begin{definition}
    Soit \( I\), un intervalle de \( \eR\). L'\defe{oscillation}{oscillation!d'une fonction} sur \( I\) est le nombre
    \begin{equation}
        \omega_f(I)=\sup_{x\in I}f(x)-\inf_{x\in I}f(x).
    \end{equation}
\end{definition}
    Pour chaque \( x\) fixé, la fonction
    \begin{equation}
        x\mapsto \omega_f\big( B(x,\delta) \big)
    \end{equation}
    est une fonction positive, croissante et a donc une limite (pour \( \delta\to 0\)). Nous notons \( \omega_f(x)\) cette limite qui est l'\defe{oscillation}{oscillation!d'une fonction en un point} de \( f\) en ce point. Une propriété immédiate est que \( f\) est continue en \( x_0\) si et seulement si \( \omega_f(x_0)=0\).

    \begin{lemma}       \label{LemuaPbtQ}
    L'ensemble des points de discontinuité d'une fonction \( f\colon \eR\to \eR\) est une réunion dénombrable de fermés.
\end{lemma}

\begin{proof}
    Soit \( D\) l'ensemble des points de discontinuité de \( f\). Nous avons
    \begin{equation}
        D=\bigcup_{n=1}^{\infty}\{ x\tq \omega_f(x)\geq \frac{1}{ n } \}.
    \end{equation}
    Il nous suffit donc de montrer que pour tout \( \epsilon\), l'ensemble
    \begin{equation}
        \{ x\tq \omega_f(x)<\epsilon \}
    \end{equation}
    est ouvert. Soit en effet \( x_0\) dans cet ensemble. Il existe \( \delta\) tel que \( \omega_f\big( B(x_0,\delta) \big)<\epsilon\). Si \( x\in B(x_0,\delta)\), alors si on choisit \( \delta'\) tel que \( B(x,\delta')\subset B(x_0,\delta)\), nous avons \( \omega_f\big( B(x,\delta') \big)<\epsilon\), ce qui justifie que \( \omega_f(x)<\epsilon\) et donc que \( x\) est également dans l'ensemble considéré.
\end{proof}

\begin{theorem}
    L'ensemble des points de discontinuité d'une limite simple de fonctions continues est de première catégorie.
\end{theorem}

\begin{proof}
    Soit \( (f_n)\) une suite de fonctions convergent simplement vers \( f\). Nous devons écrire l'ensemble des points de discontinuité de \( f\) comme une union dénombrable d'ensembles tels que sur tout intervalle \( I\), aucun de ces ensembles n'est dense. Nous savons déjà par le lemme \ref{LemuaPbtQ} que l'ensemble des points de discontinuité  de \( f\) est donné par
    \begin{equation}
        D=\bigcup_{n=1}^{\infty}\{ x\tq \omega_f(x)\geq \frac{1}{  n } \}.
    \end{equation}
    Nous essayons donc de prouver que pour tout \( \epsilon\), l'ensemble 
    \begin{equation}
        F=\{ x\tq \omega_f(x)\geq \epsilon \}
    \end{equation}
    est nulle part dense. Soit
    \begin{equation}
        E_n=\bigcap_{i,j>n}\{ x\tq | f_i(x)-f_j(x) |<\epsilon \}.
    \end{equation}
    Nous montrons que cet ensemble est fermée en étudiant le complémentaire. Soit \( x\notin E_n\); alors il existe un couple \( (i,j)\) tel que
    \begin{equation}
        | f_i(x)-f_j(x) |>\epsilon.
    \end{equation}
    Par continuité, cette inégalité reste valide dans un voisinage de \( x\). Donc il existe un voisinage de \( x\) contenu dans \( \complement E_n\) et \( E_n\) est donc fermé.

    De plus nous avons \( E_n\subset E_{n+1}\) et \( \bigcup_nE_n=\eR\). Ce dernier point est dû au fait que pour tout \( x\), il existe \( N\) tel que \( i,j>N\) implique \( | f_i(x)-f_j(x) |\leq \epsilon\). Cela est l'expression du fait que la suite \( \big( f_n(x) \big)_{n\in \eN}\) est de Cauchy.

    Soit \( I\), un intervalle fermé de \( \eR\). Nous voulons trouver un intervalle \( J\subset I\) sur lequel \( f\) est continue. Nous écrivons \( I\) sous la forme 
    \begin{equation}
        I=\bigcup_{n=1}^{\infty}(E_n\cap I).
    \end{equation}
    Tous les ensembles \( J_n=E_n\cap I\) ne peuvent être nulle part dense en même temps (à cause du théorème de Baire \ref{ThoQGalIO}). Il existe donc un \( n\) tel que \( J_n\) contienne un ouvert \( J\). Le but est de montrer que \( f\) est continue sur \( J\). Pour ce faire, nous n'allons pas simplement majorer \( | f(x)-f(x_0) |\) par \( \epsilon\) lorsque \( | x-x_0 |\) est petit. Ce que nous allons faire est majorer l'oscillation de \( f\) sur \( B(x_0,\delta)\) lorsque \( \delta\) est petit. Pour cela nous prenons \( x_0\) et \( x\) dans \( J\) et nous écrivons
    \begin{equation}
        | f(x)-f(x_0) |\leq | f(x)-f_n(x) |+| f_n(x)-f_n(x_0) |.
    \end{equation}
    À ce niveau nous rappelons que \( n\) est fixé par le choix de \( J\), dans lequel \( \epsilon\) est déjà inclus. Nous choisissons évidemment \( | x-x_0 |\leq \delta\) de telle sorte que le second terme soit plus petit que \( \epsilon\) en vertu de la continuité de \( f_n\). Pour le premier terme, pour tout \( i,j\geq n\) nous avons
    \begin{equation}
        | f_i(x)-f_j(x) |<\epsilon.
    \end{equation}
    Si nous posons \( j=n\) et \( i\to\infty\), en tenant compte du fait que \( f_i\to f\) simplement,
    \begin{equation}
        | f(x)-f_n(x) |\leq \epsilon.
    \end{equation}
    Nous avons donc obtenu \( | f(x)-f_n(x_0) |\leq 2\epsilon\). Cela signifie que dans un voisinage de rayon \( \delta\) autour de \( x_0\), les valeurs extrêmes prises par \( f(x) \) sont \( f_n(x_0)\pm 4\epsilon\). Nous avons donc prouvé que pour tout \( \epsilon\), il existe \( \delta\) tel que
    \begin{equation}
        \omega_f\big( \mathopen[ x_0-\delta , x_0+\delta \mathclose] \big)\leq 4\epsilon.
    \end{equation}
    De là nous concluons que
    \begin{equation}
        \lim_{\delta\to 0}\omega_f\big( \mathopen[ x_0-\delta , x_0+\delta \mathclose] \big)=0,
    \end{equation}
    ce qui signifie que \( f\) est continue en \( x_0\).
\end{proof}


%+++++++++++++++++++++++++++++++++++++++++++++++++++++++++++++++++++++++++++++++++++++++++++++++++++++++++++++++++++++++++++
\section{Limites à plusieurs variables}
%+++++++++++++++++++++++++++++++++++++++++++++++++++++++++++++++++++++++++++++++++++++++++++++++++++++++++++++++++++++++++++
\label{SecLimVarsPlus}

Prenons une fonction $f\colon \eR^n\to \eR$. Nous disons que
\begin{equation}
	\lim_{x\to x_0}f(x)=l\in\eR
\end{equation}
lorsque $\forall \epsilon>0$, $\exists\delta$ tel que $\| x-x_0 \|\leq\delta$ implique $| f(x)-l |\leq \epsilon$. 

Remarquez qu'ici, $x\in\eR^n$, et sachez distinguer $\| . \|$, la norme dans $\eR^n$ de $| . |$ qui est la valeur absolue dans $\eR$. Une autre façon d'exprimer cette définition est que l'ensemble des valeurs atteintes par $f$ dans une boule de rayon $\delta$ autour de $x_0$ n'est pas très loin de $l$. Nous définissons donc
\begin{equation}
	E_{\delta}=\{ f(x)\tq x\in B(x_0,\delta) \}.
\end{equation}
Notez que si $f$ n'est pas définie en $x_0$, il n'y a pas de valeurs correspondantes au centre de la boule dans $E_{\delta}$. Ceci est évidement la situation générique lorsqu'il y a une indétermination à lever dans le calcul de la limite. Nous avons alors que
\begin{equation}
	\lim_{x\to x_0}f(x)=l
\end{equation}
lorsque $\forall\epsilon>0$, $\exists\delta$ tel que 
\begin{equation}		\label{Eqvmoinsrapplimdeux}
	\sup\{ | v-l |\tq v\in E_{\delta} \}\leq\epsilon.
\end{equation}
Une façon classique de montrer qu'une limite n'existe pas, est de prouver que, pour tout $\delta$, l'ensemble $E_{\delta}$ contient deux valeurs constantes. Si par exemple $0\in E_{\delta}$ et $1\in E_{\delta}$ pour tout $\delta$, alors aucune valeur de $l$ (même pas $l=\pm\infty$) ne peut satisfaire à la condition \eqref{Eqvmoinsrapplimdeux} pour toute valeur de $\epsilon$.

Nous laissons à la sagacité de l'étudiant le soin d'adapter tout ceci pour le cas $\lim_{x\to x_0}f(x)=\pm\infty$.

La proposition suivante semble évidente, mais nous sera tellement
utile qu'il est préférable de l'expliciter~:
\begin{proposition}
Soit $f : D \to \eR$ une fonction dont le domaine
  s'écrit comme une réunion \emph{finie}
  \begin{equation*}
    D = \bigcup_{i=1}^k A_i
  \end{equation*}  
  où $k$ est un entier. Soit $a \in \adh D$ tel que $a \in \adh A_i$
  pour tout $i \leq k$, et soit $b \in \eR$. Alors, la limite
  \begin{equation*}
    \limite x a f(x)
  \end{equation*}
  existe et vaut $b$ si et seulement si chacune des limites
  \begin{equation*}
    \limite[x \in A_i] x a f(x)
  \end{equation*}
  existe et vaut $b$.
\end{proposition}

\begin{proof}On sait déjà que si la limite de $f : D \to \eR$
  existe, alors toute restriction à $A_i$ admet la même limite. Il
  suffit donc de prouver la réciproque.

  Par hypothèse, pour tout $i = 1 \ldots k$, nous savons que
  \begin{equation*}
    \forall \epsilon > 0\, \exists \delta_i > 0 \tq (x \in A_i)
    \text{ et }
    (\norme{x-a} < \delta_i) \Rightarrow \norme{f(x) - b} < \epsilon
  \end{equation*}

  Si $\epsilon$ est fixé, posons $\delta = \min_i\{\delta_i\}$. Nous
  savons alors que
  \begin{enumerate}
  \item pour tout $x \in D$, il existe $i$ tel que $x \in A_i$, et
  \item si $x$ vérifie $\norme{x-a} < \delta$, alors pour tout $i$,
    $\norme{x-a} < \delta_i$ par définition de $\delta$.
  \end{enumerate}
  
  On en déduit que si $x \in D$ et $\norme{x-a} < \delta$, alors il
  existe $i$ tel que $x \in A_i$ et $\norme{x-a} < \delta_i$, ce qui
  implique $\abs{f(x) - b} < \epsilon$ et prouve la continuité.
\end{proof}

\begin{example}
  \begin{enumerate}
  \item Pour qu'une fonction $f : \eR \to \eR$ admette une limite en
    $a \in \eR$, il faut et il suffit qu'elle y admette une limite à
    droite et une limite à gauche qui soient égales.

  \item Une suite $(x_k)$ admet une limite si et seulement si les
    sous suites $(x_{2k})$ et $(x_{2k+1})$ convergent vers la même
    limite. Ceci n'est pas une application directe de la proposition,
    mais la teneur est la même.
  \end{enumerate}
\end{example}

%+++++++++++++++++++++++++++++++++++++++++++++++++++++++++++++++++++++++++++++++++++++++++++++++++++++++++++++++++++++++++++
					\section{Maximisation sans contraintes}
%+++++++++++++++++++++++++++++++++++++++++++++++++++++++++++++++++++++++++++++++++++++++++++++++++++++++++++++++++++++++++++

%---------------------------------------------------------------------------------------------------------------------------
					\subsection{Maximisation à une variable}
%---------------------------------------------------------------------------------------------------------------------------

\begin{definition}
Soit $f\colon A\subset \eR\to \eR$ et $a\in A$. Le point $a$ est un \defe{maximum local}{maximum!local} de $f$ si il existe un voisinage $\mU$ de $a$ tel que $f(a)\geq f(x)$ pour tout $x\in\mU\cap A$. Le point $a$ est un \defe{maximum global}{maximum!global} si $f(a)\geq g(x)$ pour tout $x\in A$.
\end{definition}

La proposition basique à utiliser lors de la recherche d'extrema est la suivante :
\begin{proposition}
Soit $f\colon A\subset\eR\to \eR$ et $a\in\Int(A)$. Supposons que $f$ est dérivable en $a$. Si $a$ est un \href{http://fr.wikipedia.org/wiki/Extremum}{extremum} local, alors $f'(a)=0$.
\end{proposition}

La réciproque n'est pas vraie, comme le montre l'exemple de la fonction $x\mapsto x^3$ en $x=0$ : sa dérivée est nulle et pourtant $x=0$ n'est ni un maximum ni un minimum local. 

Cette proposition ne sert donc qu'à sélectionner des \emph{candidats} extremum. Afin de savoir si ces candidats sont des extrema, il y a la proposition suivante.
\begin{proposition}
Soit $f\colon I\subset \eR\to \eR$, une fonction de classe $C^k$ au voisinage d'un point $a\in\Int I$. Supposons que
\begin{equation}
	f'(a)=f''(a)=\ldots=f^{(k-1)}(a)=0,
\end{equation}
et que
\begin{equation}
	f^{(k)}(a)\neq 0.
\end{equation}
Dans ce cas,
\begin{enumerate}

\item
Si $k$ est pair, alors $a$ est un point d'extremum local de $f$, c'est un minimum si $f^{(k)}(a)>0$, et un maximum si $f^{(k)}(a)<0$,
\item
Si $k$ est impair, alors $a$ n'est pas un extremum local de $f$.

\end{enumerate}
\end{proposition}

Note : jusqu'à présent nous n'avons rien dit des extrema \emph{globaux} de $f$. Il n'y a pas grand chose à en dire. Si un point d'extremum global est situé dans l'intérieur du domaine de $f$, alors il sera extremum local (a fortiori). Ou alors, le maximum global peut être sur le bord du domaine. C'est ce qui arrive à des fonctions strictement croissantes sur un domaine compact.

Une seule certitude : si une fonction est continue sur un compact, elle possède une minimum et un maximum global.

%---------------------------------------------------------------------------------------------------------------------------
					\subsection{Maximisation à plusieurs variables}
%---------------------------------------------------------------------------------------------------------------------------

%---------------------------------------------------------------------------------------------------------------------------
					\subsection{Quelque mots à propos de matrices}
%---------------------------------------------------------------------------------------------------------------------------

Les notions qui suivent seront vues au cours de géométrie lorsqu'il sera temps. 

Si $g$ est une application bilinéaire sur $\eR^2$, nous disons qu'elle est
\begin{enumerate}

\item
\defe{Définie positive}{Application!définie positive} si $g(u,u)\geq 0$ pour tout $u\in\eR^2$ et $g(u,u)=0$ si et seulement si $u=0$.

\item
\defe{semi-définie positive}{Application!semi-définie positive} si $g(u,u)\geq 0$ pour tout $u\in\eR^2$. 

\end{enumerate}

Une matrice $M$ est définie positive si $v^tMv>0$ pour tout $v\neq 0$, en particulier si $v$ est un vecteur propre de valeur propre $\lambda$ (c'est à dire si $Mv=\lambda v$), alors $\lambda v^tv>0$, et donc $\lambda>0$. Donc $M$ sera définie positive si toutes ses valeurs propres sont positives.

\begin{proposition}
	Soit $M$, une matrice $2\times 2$ symétrique\footnote{la matrice $d^2f(a)$ est toujours symétrique quand $f$ est de classe $C^2$.}. Nous avons
	\begin{enumerate}
		\item
		$\det M>0$ et $\tr(M)>0$ implique $M$ définie positive,
		\item
		$\det M>0$ et $\tr(M)<0$ implique $M$ définie négative,
		\item
		$\det M<0$ implique ni semi définie positive, ni définie négative (et donc pas un extrema dans le cas où $M=d^2f(a)$ par le point \ref{ItemPropoExtreRn} de la proposition \ref{PropoExtreRn}),
		\item
		$\det M=0$ implique $M$ semi-définie positive ou semi-définie négative.
	\end{enumerate}
\end{proposition}
 

%---------------------------------------------------------------------------------------------------------------------------
					\subsection{Les théorèmes}
%---------------------------------------------------------------------------------------------------------------------------



Un point $a$ à l'intérieur du domaine d'une fonction $f\colon A\subset\eR^n\to \eR$ est une \defe{point critique}{Critique!point} de $f$ lorsque $df(a)=0$. Ces points sont analogues aux points où la dérivée d'une fonction sur $\eR$ s'annule. Les points critiques de $f$ sont dons les candidats à être des points d'extremum.

\begin{proposition}		\label{PropoExtreRn}
Soit $f\colon A\subset\eR^n\to \eR$ une fonction de classe $C^2$ au voisinage de $a\in\Int(A)$.
\begin{enumerate}

\item
Si $a$ est un point critique de $f$, et si $d^2f(a)$ est \href{http://fr.wikipedia.org/wiki/Matrice_définie_positive}{définie positive}, alors $a$ est un minimum local strict de $f$,
\item		\label{ItemPropoExtreRn}
Si $a$ est un minimum local, alors $a$ est un point critique et $d^2f(a)$ est semi-définie positive.

\end{enumerate}
\end{proposition}

La seconde partie de l'énoncé est tout à fait comparable au fait bien connu que, pour une fonction $f\colon \eR\to \eR$, si le point $a$ est minimum local, alors $f'(a)=0$ et $f''(a)\geq 0$. Notez le fait que l'inégalité n'est pas stricte, ce qui correspond à $d^2f(a)$ \emph{semi}-definie positive.

Pour rappel, dans le cas d'une fonction à deux variables, $d^2f(a)$ est la matrice (et donc l'application linéaire)
\begin{equation}
	d^2f(a)=\begin{pmatrix}
	\frac{ d^2f  }{ dx^2 }(a)	&	\frac{ d^2f  }{ dx\,dy }(a)	\\ 
	\frac{ d^2f  }{ dy\,dx }(a) 	&	\frac{ d^2f  }{ dy^2 }(a)
\end{pmatrix}.
\end{equation}
Dans le cas d'une fonction $C2$, cette matrice est symétrique.

La méthode pour chercher les extrema de $f$ est donc de suivre le points suivants :
\begin{enumerate}
\item
Trouver les candidats extrema en résolvant $\nabla f=(0,0)$,
\item
écrire $d^2f(a)$ pour chacun des candidats
\item
calculer les valeurs propres de $d^2f(a)$, déterminer si la matrice est définie positive ou négative,
\item
conclure.
\end{enumerate}

%+++++++++++++++++++++++++++++++++++++++++++++++++++++++++++++++++++++++++++++++++++++++++++++++++++++++++++++++++++++++++++
					\section{Développements de Taylor et Maclaurin}
%+++++++++++++++++++++++++++++++++++++++++++++++++++++++++++++++++++++++++++++++++++++++++++++++++++++++++++++++++++++++++++

%---------------------------------------------------------------------------------------------------------------------------
\subsection{Fonctions «petit o» }
%---------------------------------------------------------------------------------------------------------------------------

Nous voulons formaliser l'idée d'une fonction qui tend vers zéro \og plus vite\fg{} qu'une autre. Nous disons que $f\in o\big(\varphi(x)\big)$ si
\begin{equation}
	\lim_{x\to 0} \frac{ f(x) }{ \varphi(x) }=0.
\end{equation}
En particulier, nous disons que $f\in o(x)$ lorsque $\lim_{x\to 0} f(x)/x=0$.

%---------------------------------------------------------------------------------------------------------------------------
\subsection{Formule et reste}
%---------------------------------------------------------------------------------------------------------------------------

\begin{proposition}		\label{PropDevTaylorPol}
    Soient $f\colon I\subset\eR\to \eR$ et $a\in\Int(I)$. Soit un entier $k\geq 1$. Si $f$ est $k$ fois dérivable en $a$, alors il existe un et un seul polynôme $P$ de degré $\leq k$ tel que
    \begin{equation}
        f(x)-P(x-a)\in o\big( | x-a |^k \big)
    \end{equation}
    lorsque $x\to a$, $x\neq a$. Ce polynôme  est donné par
    \begin{equation}
        P(h)=f(a)+f'(a)h+\frac{ f''(a) }{ 2! }h^2+\ldots+\frac{ f^{(k)}(a) }{ k! }h^k.
    \end{equation}
    Notons encore deux façons alternatives d'écrire le résultat. Si \( f\in C^k\) il existe une fonction \( \alpha\) telle que \( \lim_{t\to 0} \alpha(t)=0\) et
    \begin{equation}
        f(x)=\sum_{n=0}^k\frac{ f^{(n)}(a) }{ n! }(x-a)^n+(x-a)^n\alpha(x-a).
    \end{equation}
    Si \( f\in C^{k+1}\) alors
    \begin{equation}        \label{EquQtpoN}
        f(x)=\sum_{n=0}^k\frac{ f^{(n)}(a) }{ n! }(x-a)^n+(x-a)^{n+1}\xi(x-a)
    \end{equation}
    où \( \xi\) est une fonction telle que \( \xi(t)\) tend vers une constante lorsque \( t\to 0\).
\end{proposition}

La proposition suivant donne une intéressante façon de trouver le reste d'un développement de Taylor.
\begin{proposition}		\label{PropResteTaylorc}
Soient $I$, un intervalle dans $\eR$ et $f\colon I\to \eR$ une fonction de classe $C^k$ sur $I$ telle que $f^{(k+1)}$ existe sur $I$. Soient $a\in\Int(I)$ et $x\in I$. Alors il existe $c$ strictement compris entre $x$ et $a$ tel que 
\begin{equation}
	R_{f,a,k}(x)=\frac{ f^{(k+1)}(c) }{ (k+1)! }(x-a)^{k+1}.
\end{equation}
\end{proposition}

%---------------------------------------------------------------------------------------------------------------------------
					\subsection{Exemple : un calcul heuristique de limite}
%---------------------------------------------------------------------------------------------------------------------------
\label{SubSecCalcLimHeuris}

Au cours de la résolution de l'exercice \ref{exoEqsDiff0002}\ref{ItemfEqsDiff00002}, nous devons calculer la limite suivante :
\begin{equation}
	\lim_{x\to 0} \frac{  e^{-2\cos(x)+2}\sin(x) }{ \sqrt{ e^{2\cos(x)+2}}-1 }.
\end{equation}
La stratégie que nous allons suivre pour calculer cette limite est de développer certaines parties de l'expression en série de Taylor, afin de simplifier l'expression. La première chose à faire est de remplacer $ e^{y(x)}$ par $1+y(x)$ lorsque $y(x)\to 0$. La limite devient
\begin{equation}
	\lim_{x\to 0} \frac{ \big( -2\cos(x)+3 \big)\sin(x) }{ \sqrt{-2\cos(x)+2} }.
\end{equation}
Nous allons maintenant remplacer $\cos(x)$ par $1$ au numérateur et par $1-x^2/2$ au dénominateur. Pourquoi ? Parce que le cosinus du dénominateur est dans une racine, donc nous nous attendons à ce que le terme de degré deux du cosinus donne un degré un en dehors de la racine, alors que du degré un est exactement ce que nous avons au numérateur : le développement du sinus commence par $x$.

Nous calculons donc
\begin{equation}
	\begin{aligned}[]
		\lim_{x\to 0} \frac{ \sin(x) }{ \sqrt{-2\left( 1-\frac{ x^2 }{ 2 } \right)+2} }=\lim_{x\to 0} \frac{ \sin(x) }{ x }=1.
	\end{aligned}
\end{equation}
Tout ceci n'est évidement pas très rigoureux, mais en principe vous avez tous les éléments en main pour justifier les étapes.

%+++++++++++++++++++++++++++++++++++++++++++++++++++++++++++++++++++++++++++++++++++++++++++++++++++++++++++++++++++++++++++
\section{Théorie de la mesure}
%+++++++++++++++++++++++++++++++++++++++++++++++++++++++++++++++++++++++++++++++++++++++++++++++++++++++++++++++++++++++++++

Pour la théorie de la mesure, voir entre autres \cite{FubiniBMauray,ProbaDanielLi}.

%---------------------------------------------------------------------------------------------------------------------------
\subsection{Espaces mesurables et mesurés}
%---------------------------------------------------------------------------------------------------------------------------

\begin{definition}
    Si \( \Omega\) est un ensemble, un ensemble \( \tribA\) de sous-ensembles de \( \Omega\) est une \defe{tribu}{tribu} si
    \begin{enumerate}
        \item
            \( \Omega\in\tribA\);
        \item
            \( \complement A\in A\) pour tout \( A\in\tribA\);
        \item
            si \( (A_i)_{i\in I}\) est un ensemble au plus dénombrable d'éléments de \( \tribA\), alors \( \sup_{n\geq 1}A_n=\bigcup_{i\in I}A_i\in\tribA\).
    \end{enumerate}
    Le couple \( (\Omega,\tribA)\) est alors un \defe{espace mesuré}{espace!mesuré}.
\end{definition}

\begin{lemma}
    Une tribu est stable par intersections au plus dénombrables.
\end{lemma}

\begin{proof}
    Soit \( (A_i)_{i\in I}\) une famille au plus dénombrable d'éléments de la tribu \( \tribA\). Nous devons prouver que \( \bigcap_{i\in I}A_i\) est également un élément de \( \tribA\). Pour cela nous passons au complémentaire :
    \begin{equation}
        \complement\left( \bigcap_{i\in I}A_i \right)=\bigcup_{i\in I}\complement A_i.
    \end{equation}
    La définition d'une tribu implique que le membre de droite est un élément de la tribu. Par stabilité d'une tribu par complémentaire, l'ensemble \( \bigcap_{i\in I}A_i\) est également un élément de la tribu.
\end{proof}

La tribu que nous utiliserons toujours dans \( \eR^d\) est la tribu des \defe{boréliens}{boréliens}, notée \( \Borelien(\eR^d)\), qui est la tribu engendrée par les ouverts de \( \eR^d\). Une fonction \( f\colon (\Omega,\tribA)\to (\eR^d,\Borelien(\eR^d))\) est \defe{borélienne}{borélienne} si pour tout \( \mO\in\Borelien\), \( f^{-1}(\mO)\in\tribA\).

\begin{definition}
    Une \defe{\wikipedia{en}{Measure_space}{mesure}}{mesure} sur l'espace mesurable \( (\Omega,\tribA)\) est une application \( \mu\colon \tribA\to \eR\cup\{ \infty \}\) telle que
    \begin{enumerate}
        \item
            \( \mu(A)\geq 0\) pour tout \( A\in\tribA\);
        \item
            \( \mu(\emptyset)=0\);
        \item
            \( \mu\left( \bigcup_{i=0}^{\infty}A_i\right)=\sum_{i=0}^{\infty}\mu(A_i)\) si les \( A_i\) sont des éléments de \( \tribA\) deux à deux disjoints.
    \end{enumerate}
    Une mesure est \defe{\( \sigma\)-finie}{mesure!$\sigma$-finie} si il existe une suite croissante (pour l'inclusion) d'éléments \( (E_n)_{n\in\eN}\) de la tribu, tous de mesure finie et tels que \( \Omega=\bigcup_{n\in \eN}E_n\). Si la mesure est $\sigma$-finie, nous disons que l'espace \( (\Omega,\tribA,\mu)\) est un espace mesuré $\sigma$-fini.
\end{definition}

\begin{example}
    La mesure de Lebesgue sur \( \eR^n\) est \( \sigma\)-finie parce que les boules de rayon \( n\) forment un ensemble dénombrable d'ensembles de mesures finies dont l'union est évidemment tout \( \eR^n\).

    L'intervalle \( I=\mathopen[ 0 , 1 \mathclose]\) muni de la tribu de toutes ses parties et de la mesure de comptage n'est pas un espace mesuré \( \sigma\)-fini.
\end{example}

\begin{definition}
    Une application entre espace mesurés
    \begin{equation}
        f\colon (\Omega,\tribA)\to (\Omega',\tribA')
    \end{equation}
    est \defe{mesurable}{mesurable!application} si pour tout \( B\in\tribA'\), l'ensemble \( f^{-1}(B)\) est dans \( \tribA\).
\end{definition}

Si \( \mu\) est une mesure sur \( \eR^d\), une fonction \( f\colon \eR^d\to \eR\) est une \defe{densité}{densité d'une mesure} si pour tout \( A\subset\eR^d\) nous avons
\begin{equation}
    \mu(A)=\int_Af(x)dx
\end{equation}
où \( dx\) est la mesure de Lebesgue.

%---------------------------------------------------------------------------------------------------------------------------
\subsection{Mesure produit}
%---------------------------------------------------------------------------------------------------------------------------

\begin{definition}      \label{DefTribProfGfYTuR}
    Si \( \tribA\) et \( \tribB\) sont deux tribus sur deux ensembles \( \Omega_1\) et \( \Omega_2\), nous définissons la \defe{tribu produit}{tribu!produit} \( \tribA\otimes\tribB\) comme étant la tribu engendrée par 
    \begin{equation}
        \{ A\times B\tq A\in\tribA,B\in\tribB \}.
    \end{equation}
\end{definition}

\begin{theorem}\index{mesure!produit}
    Soient \( \mu_i\) des mesures $\sigma$-finies sur \( (\Omega_i,\tribA_i)\) (\( i=1,2\)). Il existe une et une seule mesure, notée \( \mu_1\otimes \mu_2\), sur \( (\Omega_1\times\Omega_2,\tribA_1\otimes\tribA_2)\) telle que
    \begin{equation}
        (\mu_1\otimes\mu_2)(A_1\times A_2)=\mu_1(A_1)\mu_2(A_2)
    \end{equation}
    pour tout \( A_1\in \tribA_1\) et \( A_2\in\tribA_2\).
\end{theorem}
Une preuve peut être trouvée dans \cite{FubiniBMauray}.

%---------------------------------------------------------------------------------------------------------------------------
\subsection{Intégrale par rapport à une mesure}
%---------------------------------------------------------------------------------------------------------------------------

Une fonction \( f\colon (\Omega,\tribA)\to (\Omega',\tribA')\) est \defe{mesurable}{mesurable!fonction} si 
\begin{equation}
    f^{-1}(E)\in\tribA
\end{equation}
pour tout \( E\in\tribA'\).


Une mesure \( \mu\) sur un espace mesurable \( (\Omega,\tribA)\) permet de définir une fonctionnelle linéaire sur l'ensemble des fonctions mesurables \( \Omega\to \eR\). Cette fonctionnelle linéaire est l'intégrale que nous allons définir à présent.

D'abord nous considérons les fonction \defe{simples}{simple!fonction}\index{fonction!simple}, c'est à dire les fonctions de la forme
\begin{equation}
    f=\sum_{i=1}^Na_i\caract_{E_i}
\end{equation}
où \( a_i\in\eR\) tandis que les \( E_i\) sont des ensembles \( \mu\)-mesurables. Si \( Y\in \tribA\) nous définissons
\begin{equation}
    \int_Yfd\mu=\sum_ia_i\mu(Y\cap E_i).
\end{equation}
Pour une fonction \( \mu\)-mesurable générale \( f\colon \Omega\to \mathopen[ 0 , \infty \mathclose]\) nous définissons l'intégrale de \( f\) sur \( Y\) par
\begin{equation}        \label{EqDefintYfdmu}
    \int_Yfd\mu=\sup\Big\{ \int_Yhd\mu\,\text{où \( h\) est une fonction simple et mesurable telle que \( 0\leq h\leq f\)} \Big\}.
\end{equation}
Maintenant nous définissons
\begin{equation}
    \mu(f)=\int_{\Omega}f
\end{equation}
si \( f\) est une fonction mesurable sur \( \Omega\).

\begin{remark}
    Dans \( \eR^d\), quasiment toutes les fonctions et ensembles sont mesurables. En effet la construction d'ensembles non mesurables demande obligatoirement l'utilisation de l'axiome du choix; de tels ensembles doivent être construits «exprès pour». Il y a très peu de chances pour que vous tombiez sur un ensemble non mesurable de \( \eR^d\) sans que vous ne vous en rendiez compte.

    Par exemple la variable aléatoire 
    \begin{equation}
        X(\omega)=\begin{cases}
            \frac{1}{ \omega }    &   \text{si $ \omega\neq 0$}\\
            \infty    &    \text{$\omega=0$}.
        \end{cases}
    \end{equation}
    est mesurable, mais non intégrable.
\end{remark}

\begin{lemma}   \label{Lemfobnwt}
    Soit \( f\) une fonction mesurable positive ou nulle telle que
    \begin{equation}
        \int_{\Omega}fd\mu=0.
    \end{equation}
    Alors \( f=0\) \( \mu\)-presque partout.
\end{lemma}

\begin{proof}
    L'ensemble des points \( x\in\Omega\) tels que \( f(x)\neq 0\) peut s'écrire comme une union dénombrable disjointe :
    \begin{equation}
        \{ x\in\Omega\tq f(x)\neq 0 \}=\bigcup_{i=0}^{\infty}E_i
    \end{equation}
    avec
    \begin{subequations}
        \begin{align}
            E_0&=\{ x\in\Omega\tq f(x)>1 \}\\
            E_i&=\{ x\in\Omega\tq \frac{1}{ i+1 }\leq f(x)<\frac{1}{ i } \}.
        \end{align}
    \end{subequations}
    Si un des ensembles \( E_i\) est de mesure non nulle, alors nous pouvons considérer la fonction simple \( h(x)=\frac{1}{ i+1 }\mtu_{E_i}\) dont l'intégrale sur \( \Omega\) est strictement positive. Par conséquent le supremum de la définition \eqref{EqDefintYfdmu} est strictement positif.

    Nous savons donc que \( \mu(E_i)=0\) pour tout \( i\). Étant donné que la mesure d'une union disjointe dénombrable est égale à la somme des mesures, nous avons
    \begin{equation}
        \mu\{ x\in\Omega\tq f(x)\neq 0 \}=0,
    \end{equation}
    ce qui signifie que \( f\) est nulle \( \mu\)-presque partout.
\end{proof}

\begin{corollary}   \label{CorjLYiSm}
    Soit \( f\) une fonction mesurable sur l'espace mesuré \( (\Omega,\tribA,\mu)\) telle que
    \begin{equation}
        \int_{\Omega}f\mtu_{f>0}d\mu=0.
    \end{equation}
    Alors \( f\leq 0\) presque partout.
\end{corollary}

\begin{proof}
    Nous avons l'égalité d'ensembles
    \begin{equation}
        \{ f\mtu_{f>0}\neq 0 \}=\{ \mtu_{f>0}\neq 0 \}.
    \end{equation}
    Mais lemme \ref{Lemfobnwt} implique que \( f\mtu_{f>0}\) est nulle presque partout, c'est à dire que la mesure de l'ensemble du membre de gauche est nulle par conséquent
    \begin{equation}
        \mu\{ \mtu_{f>0}\neq 0 \}=0.
    \end{equation}
    Cela signifie que la fonction \( f\) est presque partout négative ou nulle.
\end{proof}

\begin{lemma}   \label{LemPfHgal}
    Soit \( f\) une fonction telle que \( | f(x)|\leq g(x) \) pour tout \( x\in\Omega\). Si \( g\) est intégrable, alors \( f\) est intégrable.
\end{lemma}

\begin{proof}
    Nous décomposons \( f\) en parties positives et négatives :
    \begin{subequations}
        \begin{align}
            A_+&=\{ x\in\Omega\tq f(x)>0 \}\\
            A_-&=\{ x\in\Omega\tq f(x)<0 \}.
        \end{align}
    \end{subequations}
    Nous posons \( f_+(x)=f(x)\mtu_{A_+}\) et \( f_-(x)=f(x)\mtu_{A_-}\). Nous avons \( f=f_+-f_-\) et
    \begin{equation}
        \int_{\Omega}f=\int_{A_+}f+\int_{A_-}f
    \end{equation}
    parce que \( \Omega=A_+\cup A_-\cup\{ x\in\Omega\tq f(x)=0 \}\). Si \( \varphi\) est une fonction simple qui majore \( f_+\) nous avons
    \begin{equation}
        \varphi(x)=\sum_{k}a_k\mtu_{E_k}(x)\leq f(x)\mtu_{A_+}(x)\leq g(x).
    \end{equation}
    Par conséquent le supremum qui définit \( \int f_+\) est inférieur au supremum qui définit \( \int g\). La fonction \( f_+\) est donc intégrable. La même chose est valable pour la fonction \( f_-\).
\end{proof}

\begin{proposition} \label{PropWBavIf}
    Soit \( f\) une fonction positive \( \tribA\)-mesurable et bornée. Alors \( f\) est limite ponctuelle croissante de fonction simples.
\end{proposition}

\begin{proof}
    Soit \( \sigma_n=\{ a_0=0,\ldots, a_{r_n}=n \}\) une subdivision de \( \mathopen[ 0 , n \mathclose]\) en intervalles de taille plus petites que \( 1/n\) choisis de sorte que \( \sigma_{n-1}\subset\sigma_{n}\), et
    \begin{equation}
        f_n(x)=\begin{cases}
            0    &   \text{si \( f(x)>n\)}\\
            a_i    &    \text{sinon}
        \end{cases}
    \end{equation}
    où \( a_i\) est le plus grand élément de \( \sigma_n\) inférieur à \( f(x)\). La fonction \( f_n\) est simple et nous avons pour tout \( x\)
    \begin{equation}
        \lim_{n\to \infty} f_n(x)=f(x)
    \end{equation}
\end{proof}

%---------------------------------------------------------------------------------------------------------------------------
\subsection{Mesure dominée}
%---------------------------------------------------------------------------------------------------------------------------

Soient \( \mu\) et \( \nu\) deux mesures sur le même espace \( \Omega\) et la même tribu \( \tribA\). Nous disons que la mesure \( \mu\) est \defe{dominée}{dominée!mesure}\cite{PersoFeng} par \( \nu\) si pour tout ensemble mesurable \( A\), \( \nu(A)=0\) implique \( \mu(A)=0\).

La mesure \( \mu\) est \defe{portée}{portée!mesure} par l'ensemble \( E\in\tribA\) si pour tout \( A\in\tribA\), 
\begin{equation}
    \mu(A)=\mu(A\cap E).
\end{equation}

Nous écrivons que \( \mu\perp\nu\)\nomenclature[Y]{\( \mu\perp\nu\)}{mesures perpendiculaires} si il existe un ensemble \( E\in\tribA\) tel que \( \mu\) soit porté par \( E\) et \( \nu\) soit porté par \( \complement E\).

\begin{theorem}[Radon-Nikodym]\index{Radon-Nikodym}
    Soient \( \mu\) et \( m\) deux mesures \( \sigma\)-finies sur un espace métrisable \( (\Omega,\tribA)\).
    \begin{enumerate}
        \item
            Il existe un unique couple de mesures \( \mu_1\) et \( \mu_2\) telles que
            \begin{enumerate}
                \item
                    \( \mu=\mu_1+\mu_2\)
                \item
                    \( \mu_1\) est dominé par \( m\)
                \item
                    \( \mu_2\perp m\).
            \end{enumerate}
            Dans ce cas, les mesures \( \mu_1\) et \( \mu_2\) sont positives et \( \sigma\)-finies.
        \item
            À égalité \(  m\)-presque partout près, il existe une unique fonction mesurable positive \( f\) telle que pour tout mesurable \( A\),
            \begin{equation}
                \mu_1(A)=\int_Ad\mu_1=\int_{\Omega}\mtu_Afd m.
            \end{equation}
        \item
            À égalité \( m\)-presque partout près, il existe une unique fonction positive mesurable \( h\) telle que \( \mu_1=hm\).
    \end{enumerate}
\end{theorem}
Une démonstration est donné dans \cite{NikoLi}.

\begin{corollary}   \label{CorZDkhwS}
    Si \( \mu\) es une mesure \( \sigma\)-finie dominée par la mesure \( \sigma\)-finie \( m\), alors \( \mu\) possède une unique fonction de densité.
\end{corollary}

\begin{corollary}       \label{CorDomDens}
    Soient \( \mu\) et \( m\), deux mesures positives \( \sigma\)-finies sur \( (\Omega,\tribA)\). Alors \( m\) domine \( \mu\) si et seulement si \( \mu\) possède une densité par rapport à \( m\).
\end{corollary}
 
\begin{proof}
    Si \( \mu\) est dominée par \( m\), alors la décomposition \( \mu=\mu+0\) satisfait le théorème de Radon-Nikodym. Par conséquent il existe une fonction \( f\) telle que
    \begin{equation}
        \mu(A)=\int_Afdm.
    \end{equation}
    Cette fonction est alors une densité pour \( \mu\) par rapport à \( m\).

    Pour la réciproque, nous supposons que \( \mu\) a une densité \( f\) par rapport à \( m\), et que \( A\) est une ensemble de \( m\)-mesure nulle :
    \begin{equation}
        m(A)=\int_{\Omega}\mtu_Adm=0.
    \end{equation}
    Cela signifie que la fonction \( \mtu_A\) est \( m\)-presque partout nulle. La fonction produit \( \mtu_Af\) est également nulle \( m\)-presque partout, et par conséquent
    \begin{equation}
        \mu(A)=\int_{\Omega}\mtu_Afdm=0.
    \end{equation}
\end{proof}

\begin{probleme}
    Est-ce que la démonstration de cela ne demande pas la convergence monotone d'une façon ou d'une autre ?
\end{probleme}

%---------------------------------------------------------------------------------------------------------------------------
\subsection{Changement de variables dans une intégrale}
%---------------------------------------------------------------------------------------------------------------------------

\begin{theorem} \label{ThomFeRCi}
    Soit \( \mO\) un ouvert de \( \eR^n\) et \( \mO'\) un ouvert de \( \eR^m\). Soit \( \varphi\colon \mO\to \mO'\) un difféomorphisme \( C^1\). Si \( f\colon \mO\to \eR\) est une fonction mesurable, positive et intégrable, alors
    \begin{equation}
        \int_{\mO}f(u)du=\int_{\mO'}f\big( \varphi^{-1}(v) \big)| J_{\varphi^{-1}}(v) |dv.
    \end{equation}
\end{theorem}

%+++++++++++++++++++++++++++++++++++++++++++++++++++++++++++++++++++++++++++++++++++++++++++++++++++++++++++++++++++++++++++
\section{Mesure de Lebesgue}
%+++++++++++++++++++++++++++++++++++++++++++++++++++++++++++++++++++++++++++++++++++++++++++++++++++++++++++++++++++++++++++

Nous construisons à présent la mesure de Lebesgue sur \( \eR^n\). Un \defe{pavé}{pavé} dans \( \eR^n\) est un ensemble de la forme 
\begin{equation}
    B=\prod_{i=1}^n\mathopen[ a_i , b_i \mathclose];
\end{equation}
le volume d'un tel pavé est défini par \( \Vol(B)=\prod_i(b_i-a_i)\). Soit maintenant \( A\subset \eR^n\). La \defe{mesure externe}{mesure!externe} de \( A\) est le nombre
\begin{equation}
    m^*(A)=\inf\{ \sum_{B\in\mF}\Vol(B)\text{ où \( \mF\) est un ensemble dénombrable de pavés dont l'union recouvre \( A\).} \}
\end{equation}
Nous disons que \( A\) est \defe{mesurable}{mesurable!Lebesgue} au sens de Lebesgue si pour tout ensemble \( S\subset \eR^n\) nous avons l'égalité
\begin{equation}
    m^*(S)=m^*(A\cap S)+m^*(S\setminus A).
\end{equation}
Dans ce cas nous disons que la mesure de Lebesgue de \( A\) est \( m(A)=m^*(A)\).

\begin{lemma}\label{LemTHBSEs}
    Si \( f\) est une fonction sur \( \mathopen[ a , \infty [\), alors nous avons la formule
    \begin{equation}
        \lim_{b\to \infty}\int_a^bf(x)dx=\int_a^{\infty}f(x)dx
    \end{equation}
    au sens où si un des deux membres existe, alors l'autre existe et est égal.
\end{lemma}

\begin{proof}
    Supposons que le membre de gauche existe. Cela signifie que la fonction
    \begin{equation}
        \psi(x)=\int_a^xf
    \end{equation}
    est bornée. Soit \( M\), un majorant. Pour toute fonction simple \( \varphi\) dominant \( f\), on a que \( \int\varphi\leq M\), donc l'ensemble sur lequel on prend le supremum pour calculer \( \int_a^{\infty}f\) est majoré par \( M\) et possède donc un supremum. Nous avons donc
    \begin{equation}
        \int_a^{\infty}f\leq\lim_{b\to\infty}\int_a^bf.
    \end{equation}
\end{proof}

%+++++++++++++++++++++++++++++++++++++++++++++++++++++++++++++++++++++++++++++++++++++++++++++++++++++++++++++++++++++++++++
\section{Mesure de Haar}
%+++++++++++++++++++++++++++++++++++++++++++++++++++++++++++++++++++++++++++++++++++++++++++++++++++++++++++++++++++++++++++

\begin{definition}
    Soit \( G\) un groupe topologique. Une \defe{mesure de Haar}{mesure!de Haar} sur \( G\) est une mesure \( \mu\) telle que 
    \begin{enumerate}
        \item
            \( \mu(gA)=\mu(A)\) pour tout mesurable \( A\) et tout \( g\in G\),
        \item
            \( \mu(K)<\infty\) pour tout compact \( K\subset G\).
    \end{enumerate}
    Si de plus le groupe \( G\) lui-même est compact nous demandons que la mesure soit normalisée : \( \mu(G)=1\).
\end{definition}
L'existence d'une mesure de Haar sur un groupe compact sera une conséquence du théorème de point fixe de Markov-Katutani \ref{ThoeJCdMP}.

