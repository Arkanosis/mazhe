Ce chapitre provient en grande partie de \cite{ProbaDanielLi}.

\begin{definition}
    Si \( \Omega\) est un ensemble, un ensemble \( \mA\) de sous-ensembles de \( \Omega\) est une \defe{tribu}{tribu} si
    \begin{enumerate}
        \item
            \( \Omega\in\mA\);
        \item
            \( A\cup B\in\mA\) pour tout \( A,B\in\mA\), ce qui signifie que toutes les unions finies d'éléments de \( \mA\) sont dans \( \mA\);
        \item
            \( \complement A\in A\) pour tout \( A\in\mA\);
        \item
            si \( A_n\) est une suite dénombrable d'éléments de \( \mA\), alors \( \sup_{n\geq 1}A_n\in\mA\).
    \end{enumerate}
    Le couple \( (\Omega,\mA)\) est alors un \defe{espace mesuré}{espace!mesuré}.
\end{definition}

Une \defe{\wikipedia{en}{Measure_space}{mesure}}{mesure} sur l'espace mesurable \( (\Omega,\mA)\) est une application \( \mu\colon \mA\to \eR\cup\{ \infty \}\) telle que
\begin{enumerate}
    \item
        \( \mu(A)\geq 0\) pour tout \( A\in\mA\);
    \item
        \( \mu(\emptyset)=0\);
    \item
        \( \mu\left( \bigcup_{i=0}^{\infty}A_i\right)=\sum_{i=0}^{\infty}\mu(A_i)\) si les \( A_i\) sont des éléments de \( \mA\) deux à deux disjoints.
\end{enumerate}

Une \defe{mesure de probabilité}{mesure!probabilité} sur un espace mesuré \( (\Omega,\mA)\) est une mesure positive telle que \( P(\Omega)=1\). Dans ce cas, le triple \( (\Omega,\mA,P)\) est un \defe{espace de probabilité}{espace!de probabilité}.

Un point \( \omega\in\Omega\) est une \defe{observation}{observation}, une partie mesurable \( A\in\mA\) est un \defe{événement}{événement}. L'ensemble \( A\cup B\) représente l'événement \( A\) ou \( B\) tandis que l'ensemble \( A\cap B\) représente l'événement \( A\) et \( B\).



