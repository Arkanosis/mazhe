% This is part of Mes notes de mathématique
% Copyright (c) 2011-2013
%   Laurent Claessens
% See the file fdl-1.3.txt for copying conditions.

Deux textes assez complets : \cite{MesIntProbb,MathAgreg}.

 \section{Les nombres complexes}
 \subsection{Définitions de base}
 Un nombre complexe s'écrit sous la forme $z = a + b i$, où $a$ et $b$
 sont des nombres réels appelés (et notés) respectivement partie réelle
 ($a = \Re(z)$) et partie imaginaire ($b = \Im(z)$) de $z$. L'ensemble
 des nombres de cette forme s'appelle l'ensemble des nombres complexes
 ; cet ensemble porte une structure de corps et est noté $\eC$. Le
 nombre complexe $i = 0 + 1 i$ est un nombre imaginaire qui a la
 particularité que $i^2 = -1$.

 Deux nombres complexes $a + bi$ et $c + di$ sont égaux si et seulement
 si $a = c$ et $b = d$, c'est-à-dire leurs parties réelles sont égales,
 et leurs parties imaginaires sont égales.

 Un nombre complexe étant représenté par deux nombres, on peut le
 représenter dans un plan appelé « plan de Gauss ». La plupart des
 opérations sur les nombres complexes ont leur interprétation
 géométrique dans ce plan.

 Pour $z = a + bi$ un nombre complexe, on note $\bar z = a - bi$ le
 \Defn{complexe conjugué} de $z$. Dans le plan de Gauss, il s'agit du
 symétrique de $z$ par rapport à la droite réelle (généralement
 dessinée horizontalement).

 On définit le module du complexe $z$ par $\module z = \sqrt{z\bar z} =
 \sqrt{a^2 + b^2}$. Dans le plan de Gauss, il s'agit de la distance
 entre $0$ et $z$.

 \begin{proposition}
Pour tout $z = a+bi$ et $z^\prime$ nombres complexes, on a
   \begin{enumerate}
   \item $z \bar z = a^2 + b^2$;
   \item $\bar{\bar{z}} = z$;
   \item $\module z = \module {\bar z}$;
   \item $\module{zz^\prime} = \module z \module{z^\prime}$;
   \item $\module{z+z^\prime} \leq \module z + \module{z^\prime}$.
   \end{enumerate}
 \end{proposition}

 \subsection{Forme polaire ou trigonométrique}
 Dans le plan de Gauss, le module d'un complexe $z$ représente la
 distance entre $0$ et $z$. On appelle \Defn{argument} de $z$ (noté
 $\arg z$) l'angle (déterminé à $2\pi$ près) entre le demi-axe des
 réels positifs et la demi-droite qui part de $0$ et passe par $z$. Le
 module et l'argument d'un complexe permettent de déterminer
 univoquement ce complexe puisqu'on a la formule
 \[z = a + bi = \module z \left( \cos(\arg(z)) + i \sin(\arg(z))
 \right)\]

 L'argument de $z$ se détermine via les formules
 \[\frac a {\module z} = \cos(\arg(z)) \quad \frac b {\module z} =
 \sin(\arg(z))\] ou encore par la formule
 \[\frac b a = \tan(\arg(z)) \quad \text{en vérifiant le
   quadrant.}\]%
 La vérification du quadrant vient de ce que la tangente ne détermine
 l'angle qu'à $\pi$ près.

%+++++++++++++++++++++++++++++++++++++++++++++++++++++++++++++++++++++++++++++++++++++++++++++++++++++++++++++++++++++++++++
                    \section{Maximisation sans contraintes}
%+++++++++++++++++++++++++++++++++++++++++++++++++++++++++++++++++++++++++++++++++++++++++++++++++++++++++++++++++++++++++++

%---------------------------------------------------------------------------------------------------------------------------
                    \subsection{Maximisation à une variable}
%---------------------------------------------------------------------------------------------------------------------------

\begin{definition}
Soit $f\colon A\subset \eR\to \eR$ et $a\in A$. Le point $a$ est un \defe{maximum local}{maximum!local} de $f$ si il existe un voisinage $\mU$ de $a$ tel que $f(a)\geq f(x)$ pour tout $x\in\mU\cap A$. Le point $a$ est un \defe{maximum global}{maximum!global} si $f(a)\geq g(x)$ pour tout $x\in A$.
\end{definition}

La proposition basique à utiliser lors de la recherche d'extrema est la suivante :
\begin{proposition}
Soit $f\colon A\subset\eR\to \eR$ et $a\in\Int(A)$. Supposons que $f$ est dérivable en $a$. Si $a$ est un \href{http://fr.wikipedia.org/wiki/Extremum}{extremum} local, alors $f'(a)=0$.
\end{proposition}

La réciproque n'est pas vraie, comme le montre l'exemple de la fonction $x\mapsto x^3$ en $x=0$ : sa dérivée est nulle et pourtant $x=0$ n'est ni un maximum ni un minimum local. 

Cette proposition ne sert donc qu'à sélectionner des \emph{candidats} extremum. Afin de savoir si ces candidats sont des extrema, il y a la proposition suivante.
\begin{proposition}
Soit $f\colon I\subset \eR\to \eR$, une fonction de classe $C^k$ au voisinage d'un point $a\in\Int I$. Supposons que
\begin{equation}
    f'(a)=f''(a)=\ldots=f^{(k-1)}(a)=0,
\end{equation}
et que
\begin{equation}
    f^{(k)}(a)\neq 0.
\end{equation}
Dans ce cas,
\begin{enumerate}

\item
Si $k$ est pair, alors $a$ est un point d'extremum local de $f$, c'est un minimum si $f^{(k)}(a)>0$, et un maximum si $f^{(k)}(a)<0$,
\item
Si $k$ est impair, alors $a$ n'est pas un extremum local de $f$.

\end{enumerate}
\end{proposition}

Note : jusqu'à présent nous n'avons rien dit des extrema \emph{globaux} de $f$. Il n'y a pas grand chose à en dire. Si un point d'extremum global est situé dans l'intérieur du domaine de $f$, alors il sera extremum local (a fortiori). Ou alors, le maximum global peut être sur le bord du domaine. C'est ce qui arrive à des fonctions strictement croissantes sur un domaine compact.

Une seule certitude : si une fonction est continue sur un compact, elle possède une minimum et un maximum global.

%---------------------------------------------------------------------------------------------------------------------------
                    \subsection{Quelque mots à propos de matrices}
%---------------------------------------------------------------------------------------------------------------------------

Les notions qui suivent seront vues au cours de géométrie lorsqu'il sera temps. 

Si $g$ est une application bilinéaire sur $\eR^2$, nous disons qu'elle est
\begin{enumerate}

\item
\defe{définie positive}{application!définie positive} si $g(u,u)\geq 0$ pour tout $u\in\eR^2$ et $g(u,u)=0$ si et seulement si $u=0$.

\item
\defe{semi-définie positive}{application!semi-définie positive} si $g(u,u)\geq 0$ pour tout $u\in\eR^2$. 

\end{enumerate}

Une matrice $M$ est définie positive si $v^tMv>0$ pour tout $v\neq 0$, en particulier si $v$ est un vecteur propre de valeur propre $\lambda$ (c'est à dire si $Mv=\lambda v$), alors $\lambda v^tv>0$, et donc $\lambda>0$. Donc $M$ sera définie positive si toutes ses valeurs propres sont positives.

\begin{proposition}     \label{PropcnJyXZ}
    Soit $M$, une matrice $2\times 2$ symétrique\footnote{la matrice $d^2f(a)$ est toujours symétrique quand $f$ est de classe $C^2$.}. Nous avons
    \begin{enumerate}
        \item
        $\det M>0$ et $\tr(M)>0$ implique $M$ définie positive,
        \item
        $\det M>0$ et $\tr(M)<0$ implique $M$ définie négative,
    \item   \label{ItemluuFPN}
        $\det M<0$ implique ni semi définie positive, ni définie négative 
        \item
        $\det M=0$ implique $M$ semi-définie positive ou semi-définie négative.
    \end{enumerate}
\end{proposition}


 
%---------------------------------------------------------------------------------------------------------------------------
                    \subsection{Les théorèmes}
%---------------------------------------------------------------------------------------------------------------------------

Un point $a$ à l'intérieur du domaine d'une fonction $f\colon A\subset\eR^n\to \eR$ est un \defe{point critique}{critique!point} de $f$ lorsque $df(a)=0$. Ces points sont analogues aux points où la dérivée d'une fonction sur $\eR$ s'annule. Les points critiques de $f$ sont dons les candidats à être des points d'extremum.

Pour rappel, dans le cas d'une fonction à deux variables, $d^2f(a)$ est la matrice (et donc l'application linéaire)
\begin{equation}
    d^2f(a)=\begin{pmatrix}
    \frac{ d^2f  }{ dx^2 }(a)   &   \frac{ d^2f  }{ dx\,dy }(a) \\ 
    \frac{ d^2f  }{ dy\,dx }(a)     &   \frac{ d^2f  }{ dy^2 }(a)
\end{pmatrix}.
\end{equation}
Dans le cas d'une fonction $C^2$, cette matrice est symétrique.

\begin{proposition}     \label{PropoExtreRn}
    Soit $f\colon A\subset\eR^n\to \eR$ une fonction de classe $C^2$ au voisinage de $a\in\Int(A)$.
    \begin{enumerate}
        \item
            Si $a$ est un point critique de $f$, et si $d^2f(a)$ est \href{http://fr.wikipedia.org/wiki/Matrice_définie_positive}{définie positive}, alors $a$ est un minimum local strict de $f$,
        \item\label{ItemPropoExtreRn}
            Si $a$ est un minimum local, alors $a$ est un point critique et $d^2f(a)$ est définie positive.
    \end{enumerate}
\end{proposition}
\index{fonction!différentiable}
\index{extrema}
La seconde partie de l'énoncé est tout à fait comparable au fait bien connu que, pour une fonction $f\colon \eR\to \eR$, si le point $a$ est minimum local, alors $f'(a)=0$ et $f''(a)>0$.

La méthode pour chercher les extrema de $f$ est donc de suivre le points suivants :
\begin{enumerate}
    \item
        Trouver les candidats extrema en résolvant $\nabla f=(0,0)$,
    \item
        écrire $d^2f(a)$ pour chacun des candidats
    \item
        calculer les valeurs propres de $d^2f(a)$, déterminer si la matrice est définie positive ou négative,
    \item
        conclure.
\end{enumerate}

Une conséquence du point \ref{ItemluuFPN} de la proposition \ref{PropcnJyXZ} est que si \( \det M<0\), alors le point \( a\) n'est pas  un extrema dans le cas où $M=d^2f(a)$ par le point \ref{ItemPropoExtreRn} de la proposition \ref{PropoExtreRn}.

\begin{example}
    Soit la fonction \( f(x,y)=x^4+y^4-4xy\). C'est une fonction différentiable sans problèmes. D'abord sa différentielle est
    \begin{equation}
        df=|big(4x^3-4y;4y^3-4x),
    \end{equation}
    et la matrice des dérivées secondes est
    \begin{equation}
        M=d^2f(x,y)=\begin{pmatrix}
            12x^2    &   -4    \\ 
            -4    &   12y^2    
        \end{pmatrix}.
    \end{equation}
    Nous avons \( fd=0\) pour les trois points \( (0,0)\), \( (1,1)\) et \( -1,-1\).

    Pour le point \( (0,0)\) nous avons
    \begin{equation}
        M=\begin{pmatrix}
            0    &   -4    \\ 
            -4    &   0    
        \end{pmatrix},
    \end{equation}
    dont les valeurs propres sont \( 4\) et \( -4\). Elle n'est donc semi-définie ou définie rien du tout. Donc \( (0,0)\) n'est pas un extremum local.

    Au contraire pour les points \( (1,1)\) et \( (-1,-1)\) nous avons
    \begin{equation}
        M=\begin{pmatrix}
            12    &   -4    \\ 
            -4    &   12    
        \end{pmatrix},
    \end{equation}
    dont les valeurs propres sont \( 16\) et \( 8\). La matrice \( d^2f\) y est donc définie positive. Ces deux points sont donc extrema locaux.
\end{example}
