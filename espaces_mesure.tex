% This is part of Agregation : modélisation
% Copyright (c) 2011
%   Laurent Claessens
% See the file fdl-1.3.txt for copying conditions.

Ce chapitre provient en grande partie de \cite{ProbaDanielLi}.

%+++++++++++++++++++++++++++++++++++++++++++++++++++++++++++++++++++++++++++++++++++++++++++++++++++++++++++++++++++++++++++
\section{Espaces mesurés}
%+++++++++++++++++++++++++++++++++++++++++++++++++++++++++++++++++++++++++++++++++++++++++++++++++++++++++++++++++++++++++++

\begin{definition}
    Si \( \Omega\) est un ensemble, un ensemble \( \tribA\) de sous-ensembles de \( \Omega\) est une \defe{tribu}{tribu} si
    \begin{enumerate}
        \item
            \( \Omega\in\tribA\);
        \item
            \( A\cup B\in\tribA\) pour tout \( A,B\in\tribA\), ce qui signifie que toutes les unions finies d'éléments de \( \tribA\) sont dans \( \tribA\);
        \item
            \( \complement A\in A\) pour tout \( A\in\tribA\);
        \item
            si \( A_n\) est une suite dénombrable d'éléments de \( \tribA\), alors \( \sup_{n\geq 1}A_n\in\tribA\).
    \end{enumerate}
    Le couple \( (\Omega,\tribA)\) est alors un \defe{espace mesuré}{espace!mesuré}.
\end{definition}

La tribu que nous utiliserons toujours dans \( \eR^d\) est la tribu des \defe{boréliens}{boréliens}, notée \( \Borelien(\eR^d)\), qui est la tribu engendrée par les ouverts de \( \eR^d\). Une fonction \( f\colon (\Omega,\tribA)\to (\eR^d,\Borelien(\eR^d))\) est \defe{borélienne}{borélienne} si pour tout \( \mO\in\Borelien\), \( f^{-1}(\mO)\in\tribA\).

\begin{definition}
    Une \defe{\wikipedia{en}{Measure_space}{mesure}}{mesure} sur l'espace mesurable \( (\Omega,\tribA)\) est une application \( \mu\colon \tribA\to \eR\cup\{ \infty \}\) telle que
    \begin{enumerate}
        \item
            \( \mu(A)\geq 0\) pour tout \( A\in\tribA\);
        \item
            \( \mu(\emptyset)=0\);
        \item
            \( \mu\left( \bigcup_{i=0}^{\infty}A_i\right)=\sum_{i=0}^{\infty}\mu(A_i)\) si les \( A_i\) sont des éléments de \( \tribA\) deux à deux disjoints.
    \end{enumerate}
\end{definition}

Si \( \mu\) est une mesure sur \( \eR^d\), une fonction \( f\colon \eR^d\to \eR\) est une \defe{densité}{densité d'une mesure} si pour tout \( A\subset\eR^d\) nous avons
\begin{equation}
    \mu(A)=\int_Af(x)dx
\end{equation}
où \( dx\) est la mesure de Lebesgue.

%---------------------------------------------------------------------------------------------------------------------------
\subsection{Intégrale par rapport à une mesure}
%---------------------------------------------------------------------------------------------------------------------------

Une fonction \( f\colon (\Omega,\tribA)\to (\Omega',\tribA')\) est \defe{mesurable}{mesurable!fonction} si 
\begin{equation}
    f^{-1}(E)\in\tribA
\end{equation}
pour tout \( E\in\tribA'\).


Une mesure \( \mu\) sur un espace mesurable \( (\Omega,\tribA)\) permet de définir une fonctionnelle linéaire sur l'ensemble des fonctions mesurables \( \Omega\to \eR\). Cette fonctionnelle linéaire est l'intégrale que nous allons définir à présent.

D'abord nous considérons les fonction \defe{simples}{simple!fonction}\index{fonction!simple}, c'est à dire les fonctions de la forme
\begin{equation}
    f=\sum_{i=1}^Na_i\caract_{E_i}
\end{equation}
où \( a_i\in\eR\) tandis que les \( E_i\) sont des ensembles \( \mu\)-mesurables. Si \( Y\in \tribA\) nous définissons
\begin{equation}
    \int_Yfd\mu=\sum_ia_i\mu(Y\cap E_i).
\end{equation}
Pour une fonction \( \mu\)-mesurable générale \( f\colon \Omega\to \mathopen[ 0 , \infty \mathclose]\) nous définissons l'intégrale de \( f\) sur \( Y\) par
\begin{equation}        \label{EqDefintYfdmu}
    \int_Yfd\mu=\sup\Big\{ \int_Yhd\mu\,\text{où \( h\) est une fonction simple et mesurable telle que \( 0\leq h\leq f\)} \Big\}.
\end{equation}
Maintenant nous définissons
\begin{equation}
    \mu(f)=\int_{\Omega}f
\end{equation}
si \( f\) est une fonction mesurable sur \( \Omega\).

\begin{remark}
    Dans \( \eR^d\), quasiment toutes les fonctions et ensembles sont mesurables. En effet la construction d'ensembles non mesurables demande obligatoirement l'utilisation de l'axiome du choix; de tels ensembles doivent être construits «exprès pour». Il y a très peu de chances pour que vous tombiez sur un ensemble non mesurable de \( \eR^d\) sans que vous ne vous en rendiez compte.

    Par exemple la variable aléatoire 
    \begin{equation}
        X(\omega)=\begin{cases}
            \frac{1}{ \omega }    &   \text{si $ \omega\neq 0$}\\
            \infty    &    \text{$\omega=0$}.
        \end{cases}
    \end{equation}
    est mesurable, mais non intégrable.
\end{remark}



%+++++++++++++++++++++++++++++++++++++++++++++++++++++++++++++++++++++++++++++++++++++++++++++++++++++++++++++++++++++++++++
\section{Espace de probabilité}
%+++++++++++++++++++++++++++++++++++++++++++++++++++++++++++++++++++++++++++++++++++++++++++++++++++++++++++++++++++++++++++

Une \defe{mesure de probabilité}{mesure!probabilité} sur un espace mesuré \( (\Omega,\tribA)\) est une mesure positive telle que \( P(\Omega)=1\). Dans ce cas, le triple \( (\Omega,\tribA,P)\) est un \defe{espace de probabilité}{espace!de probabilité}.

Un point \( \omega\in\Omega\) est une \defe{observation}{observation}, une partie mesurable \( A\in\tribA\) est un \defe{événement}{événement}. L'ensemble \( A\cup B\) représente l'événement \( A\) ou \( B\) tandis que l'ensemble \( A\cap B\) représente l'événement \( A\) et \( B\).


Si les \( A_n\) sont des événements, nous définissons la \defe{limite supérieur}{limite!supérieure} et la \defe{limite inférieure}{limite!inférieure} de la suite \( A_n\) par
\begin{equation}
    \limsup_{n\to\infty}A_n=\bigcap_{n\geq 1}\bigcup_{k\geq n}A_k
\end{equation}
et
\begin{equation}
    \liminf_{n\to\infty}A_n=\bigcup_{n\geq 1}\bigcap_{k\geq n}A_k
\end{equation}
Si \( \omega\in\liminf A_n\), alors \( \omega\) réalise tous les \( A_n\) sauf un nombre fini.

Nous avons
\begin{equation}
    \limsup A_n=\{ \omega\in\Omega\tq \omega\in A_n\text{pour une infinité de \( n\)} \}.
\end{equation}

\begin{theorem}[Borel-Cantelli]\index{théorème!Borel-Cantelli}
    Si
    \begin{equation}
        \sum_{n=1}^{\infty}P(A_n)<\infty
    \end{equation}
    alors \( P(\limsup A_n)=0\).
\end{theorem}

\begin{proof}
    La condition \( \sum_{n\geq 1}P(A_n)<\infty\) signifie que la fonction
    \begin{equation}
        \varphi=\sum_{n\geq 1}\caract_{A_n}
    \end{equation}
    est \( P\)-intégrable. Par conséquent, elle est finie presque partout (au sens de \( P\)), c'est à dire
    \begin{equation}
        P(\varphi=\infty)=0.
    \end{equation}
    Les points \( \omega\) sur lesquels \( \varphi(\omega)=\infty\) sont ceux tels que
    \begin{equation}
        \sum_{n\geq 1}\caract_{A_n}(\omega)=\infty,
    \end{equation}
    c'est à dire les \( \omega\) qui appartiennent à une infinité d'ensembles \( A_n\), ou encore les \( \omega\in\limsup A_n\). Nous avons donc montré que
    \begin{equation}
        \{ \omega\tq \varphi(\omega)=\infty \}=\{ \omega\in\Omega\tq \omega\in A_n\text{pour une infinité de \( n\)} \}=\limsup A_n.
    \end{equation}
    Or l'hypothèse signifie que la probabilité du membre de gauche est nulle.
\end{proof}

\begin{corollary}
    Si \( \sum_{n=1}^{\infty}P(\complement A_n)<\infty\), alors presque surement tous les \( B_n\) sont réalisés à l'exception d'un nombre fini.
\end{corollary}

%+++++++++++++++++++++++++++++++++++++++++++++++++++++++++++++++++++++++++++++++++++++++++++++++++++++++++++++++++++++++++++
\section{Variables aléatoires}
%+++++++++++++++++++++++++++++++++++++++++++++++++++++++++++++++++++++++++++++++++++++++++++++++++++++++++++++++++++++++++++

\begin{definition}
    Une \defe{variable aléatoire}{variable aléatoire} est une application mesurable
    \begin{equation}
        X\colon (\Omega,\tribA)\to (\eR^d,\Borelien(\eR^d)).
    \end{equation}
\end{definition}
Nous convenons que \( \eR^1=\bar\eR\), c'est à dire que dans le cas où la variable aléatoire \( X\) est réelle, nous acceptons les valeurs \( \pm\infty\).

On dit que la variable aléatoire \( X\) a un \defe{moment d'ordre \( p\)}{moment} si \( X\in L^p(\Omega,\tribA,P)\) (\( 1\leq p<\infty\)). Si \( X\) est \defe{intégrable}{variable aléatoire!intégrable} (c'est à dire si \( X\in L^1\)), alors nous définissons
\begin{equation}
    E(X)=\int_{\Omega}XdP\in\eR^d.
\end{equation}
Si \( E(X)=0\) nous disons que la variable aléatoire est \defe{centrée}{variable aléatoire!centrée}. La variable aléatoire \( X-E(X)\) est la variable aléatoire centrée associée à \( X\).

%---------------------------------------------------------------------------------------------------------------------------
\subsection{Variance}
%---------------------------------------------------------------------------------------------------------------------------

Si \( X\in L^2(\Omega,\tribA,P)\) alors nous définissons la \defe{variance}{variance} de \( X\) par
\begin{equation}
    \Var(X)=E\big( [X-E(X)]^2 \big).
\end{equation}

\begin{proposition}
    La variance de la variable aléatoire \( X\) peut être exprimée par la formule
    \begin{equation}
        \Var(X)=E(X^2)-[E(X)]^2
    \end{equation}
    où \( X^2=X\cdot X\) et \( E(X)^2=\) sont des produits scalaires dans \( \eR^d\).
\end{proposition}

\begin{proof}
    De façon explicite, nous avons
    \begin{equation}
        E\big( [X-E(X)]^2 \big)=\int_{\Omega}\big( X(\omega)-E(X) \big)\cdot\big( X(\omega)-E(X) \big)dP(\omega)
    \end{equation}
    où \( E(X)\in\eR^d\) est une constante. En développant le produit scalaire nous avons
    \begin{subequations}
        \begin{align}
            E\big( [X-E(X)]^2 \big)&=E\big( X^2-2X\cdot E(X)+E(X)^2 \big)\\
            &=E(X^2)-2E(X)^2+E(X)^2\\
            &=E(X^2)-E(X)^2.
        \end{align}
    \end{subequations}
\end{proof}

Si \( a\in\eR\) nous avons
\begin{equation}
    \Var(ax)=a^2\Var(X)
\end{equation}
et nous définissons l'\defe{écart-type}{écart-type} de \( X\) par
\begin{equation}
    \sigma_X=\sqrt{\Var(X)}.
\end{equation}
En d'autres termes,
\begin{equation}
    \sigma_X=\| X-E(X) \|_{L^2}.
\end{equation}
On définit encore la \defe{moyenne quadratique}{moyenne!quadratique} de \( X\) par
\begin{equation}
    \| X \|_{L^2}=\big[ E(X^2) \big]^{1/2}.
\end{equation}

Si \( X\) est une variable aléatoire réelle, nous définissons sa \defe{fonction de répartition}{fonction!de répartition} par
\begin{equation}
    \begin{aligned}
        F_X\colon \eR&\to \mathopen[ 0 , 1 \mathclose] \\
        F_X(x)&=P(X\leq x). 
    \end{aligned}
\end{equation}

%---------------------------------------------------------------------------------------------------------------------------
\subsection{Loi d'une variable aléatoire}
%---------------------------------------------------------------------------------------------------------------------------

La \defe{loi}{loi d'une variable aléatoire} de la variable aléatoire \( X\), notée \( P_X\) est la mesure image de \( P\) par \( X\), c'est à dire
\begin{equation}
    P_X(B)=P(X\in B)
\end{equation}
pour tout borélien \( B\subset\eR^d\). Note :
\begin{equation}
    P(X\in B)=P\big( \{ \omega\in\Omega\tq X(\omega)\in B \} \big)=P\big( X^{-1}(B) \big).
\end{equation}
En particulier \( P_X\) est une mesure de probabilité sur \( \eR^d\) parce que 
\begin{equation}
    P_X(\eR^d)=P(\Omega)=1.
\end{equation}
Si \( Q\) est une mesure de probabilité sur \( \eR^d\), nous notons \( X\sim Q\) si \( P_X=Q\). Nous disons alors que «\( X\) suit la loi \( Q\)».

La proposition suivante permet de calculer en pratique les intégrales qui définissent par exemple l'espérance mathématique d'une variable aléatoire.
\begin{proposition}
    Si \( f\colon \eR^d\to \bar\eR\) est borélienne et si \( X\) est une variable aléatoire, alors
    \begin{equation}
        \int_{\eR^d}f(x)dP_x(x)=\int_{\Omega}f\big( X(\omega) \big)dP(\omega)=E(f\circ X).
    \end{equation}
\end{proposition}

\begin{proof}
    No proof.
\end{proof}

En utilisant cette proposition nous trouvons une formule pratique pour l'espérance d'une variable aléatoire réelle:
\begin{equation}
    E(X)=\int_{\Omega}X(\omega)dP(\omega)=\int_{\eR}xdP_X(x),
\end{equation}
en vertu de la proposition appliquée à la fonction \( f(x)=x\).

\begin{proposition}
    Une variable aléatoire réelle \( X\) est intégrable si et seulement si \( P(x=\pm\infty)=0\) et
    \begin{equation}
        \int_{\eR}| x |dP_X(x)<\infty.
    \end{equation}
\end{proposition}

%+++++++++++++++++++++++++++++++++++++++++++++++++++++++++++++++++++++++++++++++++++++++++++++++++++++++++++++++++++++++++++
\section{Lois usuelles}
%+++++++++++++++++++++++++++++++++++++++++++++++++++++++++++++++++++++++++++++++++++++++++++++++++++++++++++++++++++++++++++

%---------------------------------------------------------------------------------------------------------------------------
\subsection{Loi de Bernouilli}
%---------------------------------------------------------------------------------------------------------------------------

Une variable aléatoire réelle est de \defe{Berouilli}{Bernouilli} de paramètre \( p\) (\( 0<p<1\)) si
\begin{equation}
    X\colon \Omega\to \eR
\end{equation}
avec \( P(x=1)=p\) et \( P(X=0)=1-p\). En tant que mesure sur \( \eR\), nous avons
\begin{equation}
    P_X=p\delta_1+(1-p)\delta_0.
\end{equation}
Une fonction \( h\) qui réalise le supremum de la formule \eqref{EqDefintYfdmu} est par exemple une fonction en escalier qui vaut en \( x\) le plus petit entier plus grand ou égal à \( x\). L'espérance d'une loi de Bernouilli est alors
\begin{equation}
    E(x)=p.
\end{equation}
Étant donné que la variable aléatoire \( X\) prend seulement les valeurs \( 0\) et \( 1\), nous avons pour tout ensemble mesurable \( B\)
\begin{equation}
    P_{X^2}(B)=P(X^2\in B)=P(X\in B),
\end{equation}
et par conséquent \( P_{X^2}=P_X\) et \( E(X^2)=E(X)\). Nous trouvons donc la variance
\begin{equation}
    \Var(X)=E(X^2)-E(X)^2=p-p^2=p(1-p).
\end{equation}

%---------------------------------------------------------------------------------------------------------------------------
\subsection{Loi normale}
%---------------------------------------------------------------------------------------------------------------------------




