% This is part of Mes notes de mathématique
% Copyright (c) 2011-2012
%   Laurent Claessens
% See the file fdl-1.3.txt for copying conditions.

Deux textes assez complets : \cite{MesIntProbb,MathAgreg}.



 \section{Les nombres complexes}
 \subsection{Définitions de base}
 Un nombre complexe s'écrit sous la forme $z = a + b i$, où $a$ et $b$
 sont des nombres réels appelés (et notés) respectivement partie réelle
 ($a = \Re(z)$) et partie imaginaire ($b = \Im(z)$) de $z$. L'ensemble
 des nombres de cette forme s'appelle l'ensemble des nombres complexes
 ; cet ensemble porte une structure de corps et est noté $\CC$. Le
 nombre complexe $i = 0 + 1 i$ est un nombre imaginaire qui a la
 particularité que $i^2 = -1$.

 Deux nombres complexes $a + bi$ et $c + di$ sont égaux si et seulement
 si $a = c$ et $b = d$, c'est-à-dire leurs parties réelles sont égales,
 et leurs parties imaginaires sont égales.

 Un nombre complexe étant représenté par deux nombres, on peut le
 représenter dans un plan appelé « plan de Gauss ». La plupart des
 opérations sur les nombres complexes ont leur interprétation
 géométrique dans ce plan.

 Pour $z = a + bi$ un nombre complexe, on note $\bar z = a - bi$ le
 \Defn{complexe conjugué} de $z$. Dans le plan de Gauss, il s'agit du
 symétrique de $z$ par rapport à la droite réelle (généralement
 dessinée horizontalement).

 On définit le module du complexe $z$ par $\module z = \sqrt{z\bar z} =
 \sqrt{a^2 + b^2}$. Dans le plan de Gauss, il s'agit de la distance
 entre $0$ et $z$.

 \begin{proposition}
Pour tout $z = a+bi$ et $z^\prime$ nombres complexes, on a
   \begin{enumerate}
   \item $z \bar z = a^2 + b^2$;
   \item $\bar{\bar{z}} = z$;
   \item $\module z = \module {\bar z}$;
   \item $\module{zz^\prime} = \module z \module{z^\prime}$;
   \item $\module{z+z^\prime} \leq \module z + \module{z^\prime}$.
   \end{enumerate}
 \end{proposition}

 \subsection{Forme polaire ou trigonométrique}
 Dans le plan de Gauss, le module d'un complexe $z$ représente la
 distance entre $0$ et $z$. On appelle \Defn{argument} de $z$ (noté
 $\arg z$) l'angle (déterminé à $2\pi$ près) entre le demi-axe des
 réels positifs et la demi-droite qui part de $0$ et passe par $z$. Le
 module et l'argument d'un complexe permettent de déterminer
 univoquement ce complexe puisqu'on a la formule
 \[z = a + bi = \module z \left( \cos(\arg(z)) + i \sin(\arg(z))
 \right)\]

 L'argument de $z$ se détermine via les formules
 \[\frac a {\module z} = \cos(\arg(z)) \quad \frac b {\module z} =
 \sin(\arg(z))\] ou encore par la formule
 \[\frac b a = \tan(\arg(z)) \quad \text{en vérifiant le
   quadrant.}\]%
 La vérification du quadrant vient de ce que la tangente ne détermine
 l'angle qu'à $\pi$ près.

%+++++++++++++++++++++++++++++++++++++++++++++++++++++++++++++++++++++++++++++++++++++++++++++++++++++++++++++++++++++++++++
\section{Suites et séries de nombres}
%+++++++++++++++++++++++++++++++++++++++++++++++++++++++++++++++++++++++++++++++++++++++++++++++++++++++++++++++++++++++++++
%\label{secseries}

%---------------------------------------------------------------------------------------------------------------------------
\subsection{Moyenne de Cesaro}
%---------------------------------------------------------------------------------------------------------------------------

Si \( (a_n)_{n\in \eN} \) est une suite dans \( \eR\) ou \( \eC\), alors sa \defe{moyenne de Cesaro}{moyenne!de Cesaro}\index{Cesaro!moyenn} est la limite (si elle existe) de la suite
\begin{equation}
    c_n=\frac{1}{ n }\sum_{k=1}^na_k.
\end{equation}
En un mot, c'est la limite des moyennes partielles.

\begin{lemma}       \label{LemyGjMqM}
    Si la suite \( (a_n)\) converge vers la limite \( \ell\) alors la suite admet une moyenne de Cesaro qui vaudra \( \ell\).
\end{lemma}

\begin{proof}
    Soit \( \epsilon>0\) et \( N\in \eN\) tel que \( | a_n-\ell |<\epsilon\) pour tout \( n>N\). En remarquant que
    \begin{equation}
        \frac{1}{ n }\sum_{k=1}^nk-\ell=\frac{1}{ n }\sum_{k=1}^n(a_k-\ell),
    \end{equation}
    nous avons
    \begin{subequations}
        \begin{align}
            | \frac{1}{ n }\sum_{k=1}^na_k-\ell |&\leq| \frac{1}{ n }\sum_{k=1}^N| a_k-\ell | |+\big| \frac{1}{ n }\sum_{k=N+1}^n\underbrace{| a_k-\ell |}_{\leq \epsilon} \big|\\
            &\leq \epsilon+\frac{ n-N-1 }{ n }\epsilon\\
            &\leq 2\epsilon.
        \end{align}
    \end{subequations}
    Dans ce calcul nous avons redéfinit \( N\) de telle sorte que le premier terme soit inférieur à \( \epsilon\).
\end{proof}


\subsection{Rappels et définitions}

La notion de série formalise le concept de somme infinie. L'absence de certaines propriétés de ces objets (problèmes de commutativité et même d'associativité) incitent à la prudence et montrent à quel point une définition précise est importante.

Soit ${(a_k)}_{k \geq k_0}$ une suite réelle. La \defe{série}{série} de terme général $(a_k)${série}, notée
\begin{math}
  \sum_{i=k_0}^\infty a_i,
\end{math}
est la suite ${(s_n)}_{n \geq k_0}$ dont les termes sont donnés par
\begin{equation*}
  s_n \pardef \sum_{i=k_0}^k a_i
\end{equation*}
et sont appelés les \defe{sommes partielles}{somme!partielle} de la série.

La série $\sum_{i=k_0}^\infty a_i$ \defe{converge}{série!convergence} si la suite $(s_n)$
converge vers un réel $s$. Sa limite est appelée la \defe{somme de la
série}{série!somme} et on note
\begin{equation}        \label{EqDefSommeLim}
  \sum_{i=k_0}^\infty a_i = \lim_{n\to\infty}\sum_{i=k_0}^na_i.
\end{equation}
Si la série ne converge pas, elle \defe{diverge}{série!divergence} et peut alors avoir
une limite infinie (uniquement si le terme général est réel, on note
alors $+\infty$ ou $-\infty$ sa limite) ou pas de limite.  La série
$\sum_{i=k_0}^\infty a_i$ \defe{converge absolument}{convergence!absolue} si la série
$\sum_{i=k_0}^\infty \abs{a_i}$ converge.

\begin{proposition}\label{propnseries_propdebase}
Les principales propriétés de la somme définie par la limite \eqref{EqDefSommeLim} sont
  \begin{enumerate}
  \item Si une série converge absolument, alors elle converge
    (simplement).
  \item Si la série est à termes positifs --c'est-à-dire pour tout
    indice $k$, $a_k \in \eR$ et $a_k \geq 0$-- il n'y a aucune
    différence entre convergence absolue et convergence simple.
  \item\label{point3-seriepropdebase} Si une série converge, son terme général doit tendre vers $0$.
\item 
Si la série converge alors la somme est associative
\item
Si la série converge absolument, alors la somme est commutative.
  \end{enumerate}
\end{proposition}

\begin{remark}Vue comme somme infinie, l'associativité et la
  commutativité dans une série sont perdues. Néanmoins, il subsiste
  que
  \begin{enumerate}
  \item si la série converge, on peut regrouper ses termes sans
    modifier la convergence ni la somme (associativité),
  \item si la série converge absolument, on peut modifier l'ordre des
    termes sans modifier la convergence ni la somme (commutativité).
  \end{enumerate}
\end{remark}

\begin{example}\label{exemplesseries}
\begin{enumerate}

\item
    La \defe{série harmonique}{série!harmonique} est
\begin{equation}
\sum_{i=1}^\infty \frac1i
\end{equation}
et diverge (possède une limite $+\infty$).

\item
    La \defe{série géométrique}{série!géométrique} de raison $q \in \eC$ est
\begin{equation}
\sum_{i=0}^\infty q^i.
\end{equation}
Étudions la somme partielle \( S_N=1+q+q^2+\ldots +q^{n}\). Nous avons évidemment $S_N-zS_N=1-q^{N+1}$ et donc
\begin{equation}
    S_N=\sum_{n=0}^Nq^n=\frac{ 1-q^{N+1} }{ 1-q }.
\end{equation}
La limite \( \lim_{N\to \infty} S_N\) existe si et seulement si \( | q |\leq 1\) et dans ce cas nous avons
\begin{equation}
    \sum_{n=1}^{\infty}z^n=\frac{ 1 }{ 1-z }.
\end{equation}
La convergence est absolue.

\item
Pour $\alpha \in \RR$, la série de Riemann (ou Dirichlet)
\begin{equation}        \label{EqSerRiem}
\sum_{i=1}^\infty \frac1{i^\alpha}
\end{equation}
converge (absolument, puisque réelle et positive) si et seulement
si $\alpha > 1$, et diverge sinon.
\end{enumerate}
\end{example}

\subsection{Critères de convergence absolue}

  Étant donné le terme général d'une série, il est souvent --dans les cas qui nous intéressent-- difficile de déterminer la somme de la série. L'exemple de la série géométrique est particulier, puisqu'on connait une formule pour chaque somme partielle, mais pour l'exemple des séries de Riemann il n'y a aucune formule simple pour un $\alpha$ général. D'où l'intérêt d'avoir des critères de convergence ne nécessitant aucune connaissance de l'éventuelle limite de la série.

\subsubsection{Critère de comparaison} 

\begin{lemma}[Critère de comparaison]   \label{LemgHWyfG}
Soient $\sum_i a_i$ et $\sum_j
b_j$ deux séries à termes positifs vérifiant
\begin{equation*}
  0 \leq a_i \leq b_i
\end{equation*}
alors
\begin{enumerate}
\item si $\sum_i a_i$ diverge, alors $\sum_j b_j$ diverge,
\item si $\sum_j b_j$ converge, alors $\sum_i a_i$ converge
  (absolument).
  \end{enumerate}
\end{lemma}

\subsubsection{Critère d'équivalence}
%\label{PgCritEquiv}

\begin{proposition}[\cite{TrenchRealAnalisys}]
 Soient $\sum_i a_i$ et $\sum_j b_j$ deux séries à termes positifs. Supposons l'existence de la limite (éventuellement infinie) suivante
\begin{equation}
  \limite i \infty \frac{a_i}{b_i} = \alpha \in \RR \text{ ou $\alpha =
    \infty$.}
\end{equation}
Dans ce cas, nous avons
\begin{enumerate}
\item si $\alpha \neq 0$ et $\alpha\neq \infty$, alors
  \begin{equation}
    \sum_i a_i \text{~converge} \ssi \sum_j b_j\text{~converge,}
  \end{equation}
\item si $\alpha = 0$ et $\sum_j b_j$ converge, alors $\sum_i a_i$
  converge (absolument),
\item si $\alpha = +\infty$ et $\sum_j b_j$ diverge, alors $\sum_i
  a_i$ diverge.
\end{enumerate}
\end{proposition}

\begin{proof}
\begin{enumerate}
    \item
        Le fait que la suite $a_n/b_n$ converge vers $\alpha$ signifie que tant sa limite supérieure que sa limite inférieure convergent vers $\alpha$. En particulier la suite $\frac{ a_n }{ b_n }$ est bornée vers le haut et vers le bas. À partir d'un certain rang $N$, il existe $M$ tel que 
        \begin{equation}
            \frac{ a_n }{ b_n }<M
        \end{equation}
        et il existe $m$ tel que
        \begin{equation}
            \frac{ a_n }{ b_n }>m.
        \end{equation}
        Nous avons donc $a_n<Mb_n$ et $a_n>mb_n$. La série de $(a_n)$ converge donc si et seulement si la série de $(b_n)$ converge.
    \item
        Si $\alpha=0$, cela signifie que pour tout $\epsilon$, il existe un rang tel que $\frac{ a_n }{ b_n }<\epsilon$, et donc tel que $a_n<\epsilon b_k$. La suite de $(a_i)$ converge donc dès que la suite de $(b_i)$ converge.
    \item
        Pour tout $M$, il existe un rang dans la suite à partir duquel on a $\frac{ a_i }{ b_i }>M$, et donc $a_k>Mb_k$. Si la série de $(b_k)$ diverge, la série de $(a_k)$ doit également diverger.
\end{enumerate}
\end{proof}

\subsubsection{Critère du quotient}

\begin{proposition}[\cite{KeislerElemCalculus}]
    Soit $\sum_i a_i$ une série. Supposons l'existence de la limite (éventuellement infinie) suivante
    \begin{equation}
      \limite i \infty \abs{\frac{a_{i+1}}{a_i}} = L\in \RR \text{ ou $L =
        \infty$.}
    \end{equation}
    Alors
    \begin{enumerate}
    \item si $L < 1$, la série converge absolument,
    \item si $L > 1$, la série diverge,
    \item si $L = 1$ le critère échoue : il existe des exemple de convergence et des exemples de divergence.
    \end{enumerate}
\end{proposition}

\begin{proof}
\begin{enumerate}
    \item
        Soit $b$ tel que $L<b<1$. À partir d'un certain rang $K$, on a $\left| \frac{ a_{i+1} }{ a_i } \right| <b$. En particulier,
        \begin{equation}
            | a_{K+1} |<b| a_K |,
        \end{equation}
        et pour $a_{K+2}$ nous avons
        \begin{equation}
            | a_{K+2} |<b| a_{K+1} |<b^2| a_K |.
        \end{equation}
        Au final,
        \begin{equation}
            | a_{K+n} |<b^n| a_K |.
        \end{equation}
        Étant donné que la série $\sum_{n\geq K}b^n$ converge (parce que $b<1$), la queue de suite $\sum_{i\geq K}a_i$ converge, et par conséquent la suite au complet converge.
    \item
        Si $L>1$, on a
        \begin{equation}
            | a_K |<| a_{K+1} |<| a_{K+2} |<\ldots
        \end{equation}
        Il est donc impossible que la suite $(a_i)$ converge vers zéro. La série ne peut donc pas converger.
    \item
        Par exemple la suite harmonique $a_n=\frac{1}{ n }$ vérifie $L=1$, mais la série ne converge pas. Par contre, la suite $a_n=\frac{ 1 }{ n^2 }$ vérifie aussi le critère avec $L=1$ tandis que la série $\sum_n\frac{1}{ n^2 }$ converge.
\end{enumerate}
\end{proof}

\subsubsection{Critère de la racine}

\begin{proposition}[\cite{TrenchRealAnalisys}]
    Soit $\sum_i a_i$ une série, et considérons
    \begin{equation*}
      \limsup_{i \rightarrow \infty} \sqrt[i]{\abs{a_i}} = L \in \RR
      \text{ ou $L =
        \infty$.}
    \end{equation*}
    Alors
    \begin{enumerate}
    \item si $L < 1$, la série converge absolument,
    \item si $L> 1$, la série diverge,
    \item si $L = 1$ le critère échoue.
    \end{enumerate}
\end{proposition}

\begin{proof}
    \begin{enumerate}
        \item
            Si $L<1$, il existe un $r\in \mathopen] 0 , 1 \mathclose[$ tel que $| a_n |^{1/n}<r$ pour les grands $n$. Dans ce cas, $| a_n |<r^{n}$, et la série converge absolument parce que la série $\sum_nr^n$ converge du fait que $r<1$.
        \item
            Si $L>1$, il existe un $r>1$ tel que $| a_n |^{1/n}>r>1$. Cela fait que $| a_n |$ prend des valeurs plus grandes que $n$ pour une infinité de termes. Le terme général $a_n$ ne peut donc pas être une suite convergente. Par conséquent la suite diverge au sens où elle ne converge pas.

    \end{enumerate}
\end{proof}

%---------------------------------------------------------------------------------------------------------------------------
\subsection{Critères de convergence simple}
%---------------------------------------------------------------------------------------------------------------------------

Les critères de comparaison, d'équivalence, du quotient et de la racine sont des critères de convergence absolue. Pour conclure à une convergence simple qui n'est pas une convergence absolue, le critère d'Abel sera notre outil principal.  

\subsubsection{Critère d'Abel}

\begin{proposition}[Critère d'Abel]
    Soit la série $\sum_i c_iz_i$ avec
    \begin{enumerate}
        \item $(c_i)$ est une suite réelle décroissante qui tend vers zéro,
        \item $(z_i)$ est une suite dans $\eC$ dont la suite des sommes partielles est bornée dans $\eC$, c'est à dire qu'il existe un $M>0$ tel que pour tout $n$,
        \begin{equation}
            \left| \sum_{i=1}^nz_i \right| \leq M.
        \end{equation}
        Alors la série $\sum_ic_iz_i$ est convergente.
    \end{enumerate}
\end{proposition}
Remarquons que ce critère ne donne pas de convergence absolue.

\begin{corollary}[Critère des séries alternées]\index{critère!série alternée}       \label{CoreMjIfw}
    Si \( (a_n)\) est une suite décroissante à limite nulle, alors la série
  \begin{equation}
    \sum_{n=0}^\infty {(-1)}^n a_n
  \end{equation}
  converge simplement.
\end{corollary}

%---------------------------------------------------------------------------------------------------------------------------
\subsection{Théorème taubérien de Hardi-Littlewood}
%---------------------------------------------------------------------------------------------------------------------------

Un théorème \defe{taubérien}{taubérien}\index{théorème!taubérien} est un théorème qui compare les modes de convergence d'une série.

\begin{lemma}
    Si \( f\) et \( g\) sont des fonctions continues, alors \( s(x)=\max\{ f(x),g(x) \}\) est également une fonction continue.
\end{lemma}

\begin{proof}
    Soit \( x_0\) et prouvons que \( s\) est continue en \( x_0\). Si \( f(x_0)\neq g(x_0)\) (supposons \( f(x_0)>g(x_0)\) pour fixer les idées), alors nous avons un voisinage de \( x_0\) sur lequel \( f>g\) et alors \( s=f\) sur ce voisinage et la continuité provient de celle de \( f\).

    Si au contraire \( f(x_0)=g(x_0)=s(x_0)\) alors si \( (a_n)\) est une suite tendant vers \( x_0\), nous prenons \( N\) tel que \( \big| f(a_n)-f(x_0) \big|\leq \epsilon\) pour tout \( n>N\) et \( M\) tel que \( \big| g(a_n)-g(x_0) \big|\leq \epsilon\) pour tout \( n> M\). Alors pour tout \( n>\max\{ N,M \}\) nous avons
    \begin{equation}
        \big| s(a_n)-s(x_0) \big|\leq \epsilon,
    \end{equation}
    d'où la continuité de \( s\) en \( x_0\).
\end{proof}

La proposition suivante dit que si une fonction connaît un saut, alors on peut le lisser par une fonction continue.
\begin{proposition} \label{PropTIeYVw}
    Soit \( f\) continue sur \( \mathopen[ a , x_0 [\) et sur \( \mathopen[ x_0 , b \mathclose]\) avec \( f(x_0^-)<f(x_0)\). En particulier nous supposons que \( f(x^-)\) existe et est finie. Alors pour tout \( \epsilon>0\), il existe une fonction continue \( s\) telle que sur \( \mathopen[ a , b \mathclose]\) on ait \( s\leq f\) et
    \begin{equation}
        \int_a^bs(x)-f(x)\,dx\leq \epsilon.
    \end{equation}
\end{proposition}

\begin{proof}
    Nous notons \( A\) la taille du saut :
    \begin{equation}
        A=f(x_0)-f(x_0^-).
    \end{equation}
    Quitte à changer \( a\) et \( b\), nous pouvons supposer que
    \begin{equation}
        f(x)<f(x_0)+\frac{ A }{ 3 }
    \end{equation}
    pour \( x\in \mathopen[ a , x_0 [\) et 
    \begin{equation}
        f>f(x_0)+\frac{ 2A }{ 3 }
    \end{equation}
    pour \( x\in \mathopen[ x_0 , b \mathclose]\). C'est le théorème des valeurs intermédiaires qui nous permet de faire ce choix.

    Soit \( m(x)\) la droite qui joint le point \( \big( x_0-\epsilon, f(x_0-\epsilon) \big)\) au point \( \big( x_0,f(x_0^+) \big)\). Nous posons
    \begin{equation}
        s(x)=\begin{cases}
            f(x)    &   \text{si \( x<x_0-\epsilon\)}\\
            \max\{ m(x),f(x) \}    &   \text{si \( x_0-\epsilon\leq x\leq x_0\)}\\
            f(x)    &    \text{si $x>x_0$}.
        \end{cases}
    \end{equation}
    En vertu des différents choix effectués, c'est une fonction continue. En effet
    \begin{equation}
        s(x_0-\epsilon)=\max\{ f(x_0-\epsilon),f(x_0,\epsilon) \}=f(x_0-\epsilon)
    \end{equation}
    et 
    \begin{equation}
        s(x_0)=\max\{ m(x_0),f(x_0^+) \}=f(x_0^+)
    \end{equation}
    parce que \( m(x_0)=f(x_0^+)\). En ce qui concerne l'intégrale, si nous posons
    \begin{equation}
        M=\sup_{x,y\in \mathopen[ a , b \mathclose]}| f(x)-f(y) |,
    \end{equation}
    nous avons
    \begin{equation}
        \int_a^bs-f=\int_{x_0-\epsilon}^{x_0}s-f\leq \epsilon M.
    \end{equation}
\end{proof}

\begin{lemma}\label{LemauxrKN}
    Pour tout polynôme \( P\), nous avons la formule
    \begin{equation}
        \lim_{x\to 1^-} (1-x)\sum_{n=0}^{\infty}x^nP(x^n)=\int_0^1P(x)dx.
    \end{equation}
\end{lemma}

\begin{proof}
    D'abord pour \( P=1\), la formule se réduit à la série harmonique connue. Ensuite nous prouvons la formule pour le polynôme \( P=X^k\) et la linéarité fera le reste pour les autres polynômes. Nous avons
    \begin{equation}
        (1-x)\sum_nx^nx^{kn}=(1-x)\sum_n(x^{1+k})^n=\frac{ 1-x }{ 1-x^{1+k} }=\frac{1}{ 1+x+\ldots+x^k }.
    \end{equation}
    Donc
    \begin{equation}
        \lim_{x\to 1^-} (1-x)\sum_nx^nP(x^n)=\frac{1}{ 1+k }.
    \end{equation}
    Par ailleurs, c'est vite vu que
    \begin{equation}
        \int_0^1 x^kdx=\frac{1}{ k+1 }.
    \end{equation}
\end{proof}

\begin{theorem}[Hardy-Littlewood\cite{ytMOpe}]\index{théorème Hardy-Littlewood}\index{Hardy-Littlewood (théorème)}      \label{ThoPdDxgP}
    Soit \( (a_n)\) une suite réelle telle que
    \begin{enumerate}
        \item
            \( \frac{ a_n }{ n }\) tends vers une constante,
        \item
            \( F(x)=\sum_{n=0}^{\infty}a_nx^n\) a un rayon de convergence \( \geq 1\),
        \item
            \( \lim_{x\to 1^-} F(x)=l\).
    \end{enumerate}
    Alors \( \sum_{n=0}^{\infty}a_n=l\).
\end{theorem}

\begin{proof}
    Quitte à prendre la suite \( b_0=a_0-l\) et \( b_n=a_n\), on peut supposer \( l=0\).

    Soit \( \Gamma\) l'ensemble des fonctions
    \begin{equation}
         \gamma\colon \mathopen[ 0 , 1 \mathclose]\to \eR 
    \end{equation}
    telles que 
    \begin{enumerate}
        \item
            $\sum_{n=0}^{\infty}a_n\gamma(x^n)$ converge pour \( 0\leq x<1\),
        \item
            \( \lim_{x\to 1^-} \sum_{n\geq 0}a_n\gamma(^n)=0\).
    \end{enumerate}
    Ce \( \Gamma\) est un espace vectoriel.
    \begin{subproof}
    \item[Les polynômes sont dans \( \Gamma\)]
        Soit \( \gamma(t)=t^s\). Pour \( 0\leq x<1\) nous avons
        \begin{equation}
            \sum_{n=0}^{\infty}a_n\gamma(x^n)=\sum_{n=0}^{\infty}a_nx^{ns}<\sum_{n=0}^{\infty}a_nx^n.
        \end{equation}
        Donc la condition de convergence est vérifiée. En ce qui concerne la limite,
        \begin{equation}
            \lim_{x\to 1^-} \sum_{n=0}^{\infty}a_nx^{ns}=\lim_{x\to 1^-} F(x^s)=0
        \end{equation}
        parce que par hypothèse, \( \lim_{x\to 1^-} F(x)=0\).

    \item[Définition de la fonction qui va donner la réponse]
        Nous considérons la fonction \( g=\mtu_{\mathopen[ \frac{ 1 }{2} , 1 \mathclose]}\), c'est à dire
        \begin{equation}
            g(t)=\begin{cases}
                0    &   \text{si \( 0\leq t<1/2\)}\\
                1    &    \text{si \( 1/2\leq t\leq 1\)}.
            \end{cases}
        \end{equation}
        Nous montrons que si \( g\in \gamma\), alors le théorème est terminé. Si \( 0\leq x\leq 1\), on a \( 0\leq x^n<1/2\) dès que
        \begin{equation}
            n>-\frac{ \ln(2) }{ \ln(x) }
        \end{equation}
        avec une note comme quoi \( \ln(x)<0\), donc la fraction est positive. Nous désignons par \( N_x\) la partie entière de ce \( n\) adapté à \( x\). L'idée est que la fonction  \( g(x^n)\) est la fonction indicatrice de \(0 \leq n\leq N_x\), et donc
        \begin{equation}
            \sum_{n\geq 0}a_ng(x^n)=\sum_{n=0}^{N_x}a_n.
        \end{equation}
        Mais si \( x\to 1^-\), alors \( N_x\to \infty\), donc
        \begin{equation}
            \lim_{N\to \infty} \sum_{n=0}^Na_n=\lim_{x\to 1^-} \sum_{n=0}^{N_x}a_n=\lim_{x\to 1^-} \sum_{n\in \eN}a_ng(x^n),
        \end{equation}
        et cela fait zéro si \( g\in \Gamma\).
        
    \item[Approximation de \( g\) par des polynômes]

        Nous considérons la fonction
        \begin{equation}
            h(t)=\frac{ g(t)-t }{ t(1-1) }=\begin{cases}
                \frac{1}{ t-1 }    &   \text{si \( t\in \mathopen[ 0 , 1/2 [\)}\\
                \frac{1}{ t }    &    \text{si \( t\in \mathopen[ 1/2 , 1 \mathclose]\)}.
            \end{cases}
        \end{equation}
        La seconde égalité est au sens du prolongement par continuité. La fonction \( h\) est une fonction non continue qui fait un saut de \( -2\) à \( 2\) en \( x=1/2\). En vertu de la proposition \ref{PropTIeYVw} (un peu adaptée), nous pouvons considérer deux fonctions continues \( s_1\) et \( s_2\) telles que
        \begin{equation}
            s_1\leq h\leq s_2
        \end{equation}
        et
        \begin{equation}
            \int_{0}^1s_2-s_1\leq \epsilon.
        \end{equation}
        Notons que l'inégalité \( s_1\leq s_2\) doit être stricte sur au moins un petit intervalle autour de \( x=1/2\). Soient \( P_1\) et \( P_2\), deux polynômes tels que \( \| P_1-s_1 \|_{\infty}\leq \epsilon\) et \( \| P_2-s_2 \|_{\infty}\leq \epsilon\) (ici la norme supremum est prise sur \( \mathopen[ 0 , 1 \mathclose]\)). C'est le théorème de Stone-Weierstrass (\ref{ThoGddfas}) qui nous permet de le faire.

        Nous posons aussi\footnote{À ce niveau, je crois qu'il y a une faute de frappe dans \cite{ytMOpe}.}
        \begin{subequations}
            \begin{align}
                Q_1=P_1+\epsilon\\
                Q_2=P_2-\epsilon.
            \end{align}
        \end{subequations}
        Nous avons
        \begin{equation}
            \int_0^1Q_1-Q_2\leq\int_0^1 Q_1-P_1+P_1-P_2+P_2-Q_2.
        \end{equation}
        Pour majorer cela, d'abord \( Q_1-P_1=P_2-Q2=\epsilon\), ensuite,
        \begin{equation}
            P_1-P_2=P_1-s_1+s_1-s_2+s_2-P_2
        \end{equation}
        dans lequel nous avons \( P_1-s_1\leq \epsilon\), \( s_2-P_2\leq \epsilon\) et \( \int_0^1s_1-s_2\leq\epsilon\). Au final, nous posons \( q=Q_2-Q_1\) et nous avons
        \begin{equation}
            \int_0^1q\leq 5\epsilon.
        \end{equation}
        Enfin nous posons aussi
        \begin{equation}
            R_i(x)=x+x(1-x)Q_i.
        \end{equation}
        Ces polynômes vérifient \( R_i(0)=0\), \( R_i(1)=1\) et
        \begin{equation}
            R_1\leq g\leq R_2
        \end{equation}
        parce que
        \begin{equation}
            Q_1\leq P_1\leq h\leq  P_2\leq Q_2
        \end{equation}
        et
        \begin{equation}
            t+t(1-t)Q_1\leq \underbrace{t+t(1-t)h(t)}_{g(t)}\leq t+t(1-t)Q_2.
        \end{equation}
        
    \item[Preuve que \( g\) est dans \( \Gamma\)]

        D'abord si \( 0\leq x<1\), \( x^N<\frac{ 1 }{2}\) pour un certain \( N\), et alors \( g(x^N)=0\). Du coup la série
        \begin{equation}
            \sum_{n=0}^{\infty}a_ng(x^n)=\sum_{n=0}^{N}a_n
        \end{equation}
        est une somme finie qui converge donc.

        D'autre part nous prenons \( M\) tel que \( | a_n |<\frac{ M }{ n }\) pour tout \( n\). Nous majorons \( \sum_{n \in \eN}a_ng(x^n)\) en utilisant \( R_1\). Mais vu que \( R_1\) est un polynôme, nous pouvons dire que \( | \sum_{n=0}^{\infty}a_nR_1(x^n) |\leq \epsilon\) en prenant \( x\in\mathopen[ \lambda , 1 [\) et \( \lambda\) assez grand. Nous avons :
        \begin{subequations}
            \begin{align}
                \left| \sum_{n=0}^{\infty}a_ng(x^n) \right| &\leq\left| \sum_{n=0}^{\infty}a_ng(x^n)-\sum_{n=0}^{\infty}a_nR_1(x^n) \right| +\underbrace{\left| \sum_{n=0}^{\infty}a_nR_1(x^n) \right|}_{\leq \epsilon} \\
                &\leq \epsilon+\sum_{n=0}^{\infty}| a_n |(g-R_1)(x^n)\\
                &\leq \epsilon+\sum_{n=0}^{\infty}| a_n |(R_2-R_1)(x^n)\\
                &\leq \epsilon+M\sum_{n=0}^{\infty}\frac{ x^n(1-x^n) }{ n }(Q_2-Q_1)(x^n)   &R_2-R_1=x(1-x)(Q_2-Q_1)\\
                &=\epsilon+M\sum_{n=0}^{\infty}\frac{ x^n(1-x^n) }{ n }q(x^n)\\
                &\leq \epsilon+M(1-x)\sum_nx^nq(x^n)   \label{subeqtZXDvu} 
            \end{align}
        \end{subequations}
        où la ligne \eqref{subeqtZXDvu} provient d'une majoration sauvage de \( 1/n\) par \( 1\) et de \( 1-x^n\) par \( 1-x\). Par le lemme \ref{LemauxrKN}, nous avons alors
        \begin{equation}
            \lim_{x\to 1^-} | \sum_na_ng(x^n) |\leq \epsilon+M\int_0^1q\leq 6\epsilon.
        \end{equation}
    \end{subproof}
\end{proof}


\section{Continuité et dérivabilité}
\label{seccontetderiv}

On considère dans la suite une fonction $f : A \to \eR$, où $a \in A \subset \eR$ ; cependant, les notions de continuité et de dérivabilité se généralisent immédiatement au cas de fonctions à valeurs vectorielles ; la notion de continuité se généralise au cas des fonctions à plusieurs variables (la notion de dérivabilité est remplacée par celle de différentiabilité dans ce cadre).

\begin{definition}
    La fonction $f$ est \Defn{continue en $a$} si
  \begin{equation*}
    \limite x a f(x)
  \end{equation*}
  existe (et vaut alors $f(a)$).
\end{definition}

\begin{definition}
    La fonction $f$ est \Defn{dérivable en $a$} si $a \in
  \operatorname{int} A$ et si
  \begin{equation*}
    \lim_{\substack{x\rightarrow a\\x\neq a}} \frac{f(x)-f(a)}{x-a}
  \end{equation*}
  existe. On note alors cette quantité $f^\prime(a)$, c'est le nombre
  dérivé de $f$ en $a$. La \Defn{fonction dérivée} de $f$ est
  \begin{equation*}
    f^\prime : A^\prime \to \eR : a \mapsto f^\prime(a)
  \end{equation*}
  définie sur l'ensemble noté $A^\prime$ des points $a$ où $f$ est
  dérivable.
\end{definition}

\begin{definition}
  Une fonction est \Defn{continue} (resp. \Defn{dérivable}) si elle est continue (resp. dérivable) en tout point $a \in A$ de son domaine.
\end{definition}

\begin{example}
      Montrons que la fonction $f : \eR \to \eR : x\mapsto x$ est continue et dérivable. Exceptionnellement (bien qu'on sache que la dérivabilité implique la continuité), montrons ces deux assertions séparément.
      \begin{description}
      \item[Continuité] Pour prouver la continuité au point $a \in \eR$ nous devons montrer que
     \begin{equation}
       \limite x a x = a
     \end{equation}
     c'est-à-dire
     \begin{equation}
       \forall \epsilon > 0, \exists \delta > 0 :  \forall x \in \eR \abs{x-a} <
       \delta \Rightarrow \abs{x-a} < \epsilon
     \end{equation}
     ce qui est clair en prenant $\delta = \epsilon$.

      \item[Dérivabilité] Soit $a \in \eR$. Calculons la limite du quotient différentiel
        \begin{equation}
          \limite[x\neq a]{x}{a} \frac{x-a}{x-a} = \limite[x\neq a]x a 1 = 1
        \end{equation}
        ce qui prouve que $f$ est dérivable et que sa dérivée vaut $1$ en
        tout point $a$ de $\eR$.
      \end{description}

     On a donc montré que la fonction $x \mapsto x$ est continue, dérivable, et que sa dérivée vaut $1$ en tout point $a$ de son domaine.

\end{example}

%+++++++++++++++++++++++++++++++++++++++++++++++++++++++++++++++++++++++++++++++++++++++++++++++++++++++++++++++++++++++++++
\section{Espace des fonctions continues}
%+++++++++++++++++++++++++++++++++++++++++++++++++++++++++++++++++++++++++++++++++++++++++++++++++++++++++++++++++++++++++++

\begin{definition}
    Soit \( I\), un intervalle de \( \eR\). L'\defe{oscillation}{oscillation!d'une fonction} sur \( I\) est le nombre
    \begin{equation}
        \omega_f(I)=\sup_{x\in I}f(x)-\inf_{x\in I}f(x).
    \end{equation}
\end{definition}
    Pour chaque \( x\) fixé, la fonction
    \begin{equation}
        x\mapsto \omega_f\big( B(x,\delta) \big)
    \end{equation}
    est une fonction positive, croissante et a donc une limite (pour \( \delta\to 0\)). Nous notons \( \omega_f(x)\) cette limite qui est l'\defe{oscillation}{oscillation!d'une fonction en un point} de \( f\) en ce point. Une propriété immédiate est que \( f\) est continue en \( x_0\) si et seulement si \( \omega_f(x_0)=0\).

    \begin{lemma}       \label{LemuaPbtQ}
    L'ensemble des points de discontinuité d'une fonction \( f\colon \eR\to \eR\) est une réunion dénombrable de fermés.
\end{lemma}

\begin{proof}
    Soit \( D\) l'ensemble des points de discontinuité de \( f\). Nous avons
    \begin{equation}
        D=\bigcup_{n=1}^{\infty}\{ x\tq \omega_f(x)\geq \frac{1}{ n } \}.
    \end{equation}
    Il nous suffit donc de montrer que pour tout \( \epsilon\), l'ensemble
    \begin{equation}
        \{ x\tq \omega_f(x)<\epsilon \}
    \end{equation}
    est ouvert. Soit en effet \( x_0\) dans cet ensemble. Il existe \( \delta\) tel que \( \omega_f\big( B(x_0,\delta) \big)<\epsilon\). Si \( x\in B(x_0,\delta)\), alors si on choisit \( \delta'\) tel que \( B(x,\delta')\subset B(x_0,\delta)\), nous avons \( \omega_f\big( B(x,\delta') \big)<\epsilon\), ce qui justifie que \( \omega_f(x)<\epsilon\) et donc que \( x\) est également dans l'ensemble considéré.
\end{proof}

\begin{theorem}
    L'ensemble des points de discontinuité d'une limite simple de fonctions continues est de première catégorie.
\end{theorem}

\begin{proof}
    Soit \( (f_n)\) une suite de fonctions convergent simplement vers \( f\). Nous devons écrire l'ensemble des points de discontinuité de \( f\) comme une union dénombrable d'ensembles tels que sur tout intervalle \( I\), aucun de ces ensembles n'est dense. Nous savons déjà par le lemme \ref{LemuaPbtQ} que l'ensemble des points de discontinuité  de \( f\) est donné par
    \begin{equation}
        D=\bigcup_{n=1}^{\infty}\{ x\tq \omega_f(x)\geq \frac{1}{  n } \}.
    \end{equation}
    Nous essayons donc de prouver que pour tout \( \epsilon\), l'ensemble 
    \begin{equation}
        F=\{ x\tq \omega_f(x)\geq \epsilon \}
    \end{equation}
    est nulle part dense. Soit
    \begin{equation}
        E_n=\bigcap_{i,j>n}\{ x\tq | f_i(x)-f_j(x) |<\epsilon \}.
    \end{equation}
    Nous montrons que cet ensemble est fermée en étudiant le complémentaire. Soit \( x\notin E_n\); alors il existe un couple \( (i,j)\) tel que
    \begin{equation}
        | f_i(x)-f_j(x) |>\epsilon.
    \end{equation}
    Par continuité, cette inégalité reste valide dans un voisinage de \( x\). Donc il existe un voisinage de \( x\) contenu dans \( \complement E_n\) et \( E_n\) est donc fermé.

    De plus nous avons \( E_n\subset E_{n+1}\) et \( \bigcup_nE_n=\eR\). Ce dernier point est dû au fait que pour tout \( x\), il existe \( N\) tel que \( i,j>N\) implique \( | f_i(x)-f_j(x) |\leq \epsilon\). Cela est l'expression du fait que la suite \( \big( f_n(x) \big)_{n\in \eN}\) est de Cauchy.

    Soit \( I\), un intervalle fermé de \( \eR\). Nous voulons trouver un intervalle \( J\subset I\) sur lequel \( f\) est continue. Nous écrivons \( I\) sous la forme 
    \begin{equation}
        I=\bigcup_{n=1}^{\infty}(E_n\cap I).
    \end{equation}
    Tous les ensembles \( J_n=E_n\cap I\) ne peuvent être nulle part dense en même temps (à cause du théorème de Baire \ref{ThoQGalIO}). Il existe donc un \( n\) tel que \( J_n\) contienne un ouvert \( J\). Le but est de montrer que \( f\) est continue sur \( J\). Pour ce faire, nous n'allons pas simplement majorer \( | f(x)-f(x_0) |\) par \( \epsilon\) lorsque \( | x-x_0 |\) est petit. Ce que nous allons faire est majorer l'oscillation de \( f\) sur \( B(x_0,\delta)\) lorsque \( \delta\) est petit. Pour cela nous prenons \( x_0\) et \( x\) dans \( J\) et nous écrivons
    \begin{equation}
        | f(x)-f(x_0) |\leq | f(x)-f_n(x) |+| f_n(x)-f_n(x_0) |.
    \end{equation}
    À ce niveau nous rappelons que \( n\) est fixé par le choix de \( J\), dans lequel \( \epsilon\) est déjà inclus. Nous choisissons évidemment \( | x-x_0 |\leq \delta\) de telle sorte que le second terme soit plus petit que \( \epsilon\) en vertu de la continuité de \( f_n\). Pour le premier terme, pour tout \( i,j\geq n\) nous avons
    \begin{equation}
        | f_i(x)-f_j(x) |<\epsilon.
    \end{equation}
    Si nous posons \( j=n\) et \( i\to\infty\), en tenant compte du fait que \( f_i\to f\) simplement,
    \begin{equation}
        | f(x)-f_n(x) |\leq \epsilon.
    \end{equation}
    Nous avons donc obtenu \( | f(x)-f_n(x_0) |\leq 2\epsilon\). Cela signifie que dans un voisinage de rayon \( \delta\) autour de \( x_0\), les valeurs extrêmes prises par \( f(x) \) sont \( f_n(x_0)\pm 4\epsilon\). Nous avons donc prouvé que pour tout \( \epsilon\), il existe \( \delta\) tel que
    \begin{equation}
        \omega_f\big( \mathopen[ x_0-\delta , x_0+\delta \mathclose] \big)\leq 4\epsilon.
    \end{equation}
    De là nous concluons que
    \begin{equation}
        \lim_{\delta\to 0}\omega_f\big( \mathopen[ x_0-\delta , x_0+\delta \mathclose] \big)=0,
    \end{equation}
    ce qui signifie que \( f\) est continue en \( x_0\).
\end{proof}


%+++++++++++++++++++++++++++++++++++++++++++++++++++++++++++++++++++++++++++++++++++++++++++++++++++++++++++++++++++++++++++
\section{Limites à plusieurs variables}
%+++++++++++++++++++++++++++++++++++++++++++++++++++++++++++++++++++++++++++++++++++++++++++++++++++++++++++++++++++++++++++
\label{SecLimVarsPlus}

Prenons une fonction $f\colon \eR^n\to \eR$. Nous disons que
\begin{equation}
    \lim_{x\to x_0}f(x)=l\in\eR
\end{equation}
lorsque $\forall \epsilon>0$, $\exists\delta$ tel que $\| x-x_0 \|\leq\delta$ implique $| f(x)-l |\leq \epsilon$. 

Remarquez qu'ici, $x\in\eR^n$, et sachez distinguer $\| . \|$, la norme dans $\eR^n$ de $| . |$ qui est la valeur absolue dans $\eR$. Une autre façon d'exprimer cette définition est que l'ensemble des valeurs atteintes par $f$ dans une boule de rayon $\delta$ autour de $x_0$ n'est pas très loin de $l$. Nous définissons donc
\begin{equation}
    E_{\delta}=\{ f(x)\tq x\in B(x_0,\delta) \}.
\end{equation}
Notez que si $f$ n'est pas définie en $x_0$, il n'y a pas de valeurs correspondantes au centre de la boule dans $E_{\delta}$. Ceci est évidement la situation générique lorsqu'il y a une indétermination à lever dans le calcul de la limite. Nous avons alors que
\begin{equation}
    \lim_{x\to x_0}f(x)=l
\end{equation}
lorsque $\forall\epsilon>0$, $\exists\delta$ tel que 
\begin{equation}        \label{Eqvmoinsrapplimdeux}
    \sup\{ | v-l |\tq v\in E_{\delta} \}\leq\epsilon.
\end{equation}
Une façon classique de montrer qu'une limite n'existe pas, est de prouver que, pour tout $\delta$, l'ensemble $E_{\delta}$ contient deux valeurs constantes. Si par exemple $0\in E_{\delta}$ et $1\in E_{\delta}$ pour tout $\delta$, alors aucune valeur de $l$ (même pas $l=\pm\infty$) ne peut satisfaire à la condition \eqref{Eqvmoinsrapplimdeux} pour toute valeur de $\epsilon$.

Nous laissons à la sagacité de l'étudiant le soin d'adapter tout ceci pour le cas $\lim_{x\to x_0}f(x)=\pm\infty$.

La proposition suivante semble évidente, mais nous sera tellement
utile qu'il est préférable de l'expliciter~:
\begin{proposition}
Soit $f : D \to \eR$ une fonction dont le domaine
  s'écrit comme une réunion \emph{finie}
  \begin{equation*}
    D = \bigcup_{i=1}^k A_i
  \end{equation*}  
  où $k$ est un entier. Soit $a \in \adh D$ tel que $a \in \adh A_i$
  pour tout $i \leq k$, et soit $b \in \eR$. Alors, la limite
  \begin{equation*}
    \limite x a f(x)
  \end{equation*}
  existe et vaut $b$ si et seulement si chacune des limites
  \begin{equation*}
    \limite[x \in A_i] x a f(x)
  \end{equation*}
  existe et vaut $b$.
\end{proposition}

\begin{proof}On sait déjà que si la limite de $f : D \to \eR$
  existe, alors toute restriction à $A_i$ admet la même limite. Il
  suffit donc de prouver la réciproque.

  Par hypothèse, pour tout $i = 1 \ldots k$, nous savons que
  \begin{equation*}
    \forall \epsilon > 0\, \exists \delta_i > 0 \tq (x \in A_i)
    \text{ et }
    (\norme{x-a} < \delta_i) \Rightarrow \norme{f(x) - b} < \epsilon
  \end{equation*}

  Si $\epsilon$ est fixé, posons $\delta = \min_i\{\delta_i\}$. Nous
  savons alors que
  \begin{enumerate}
  \item pour tout $x \in D$, il existe $i$ tel que $x \in A_i$, et
  \item si $x$ vérifie $\norme{x-a} < \delta$, alors pour tout $i$,
    $\norme{x-a} < \delta_i$ par définition de $\delta$.
  \end{enumerate}
  
  On en déduit que si $x \in D$ et $\norme{x-a} < \delta$, alors il
  existe $i$ tel que $x \in A_i$ et $\norme{x-a} < \delta_i$, ce qui
  implique $\abs{f(x) - b} < \epsilon$ et prouve la continuité.
\end{proof}

\begin{example}
  \begin{enumerate}
  \item Pour qu'une fonction $f : \eR \to \eR$ admette une limite en
    $a \in \eR$, il faut et il suffit qu'elle y admette une limite à
    droite et une limite à gauche qui soient égales.

  \item Une suite $(x_k)$ admet une limite si et seulement si les
    sous suites $(x_{2k})$ et $(x_{2k+1})$ convergent vers la même
    limite. Ceci n'est pas une application directe de la proposition,
    mais la teneur est la même.
  \end{enumerate}
\end{example}

%+++++++++++++++++++++++++++++++++++++++++++++++++++++++++++++++++++++++++++++++++++++++++++++++++++++++++++++++++++++++++++
                    \section{Différentiabilité}
%+++++++++++++++++++++++++++++++++++++++++++++++++++++++++++++++++++++++++++++++++++++++++++++++++++++++++++++++++++++++++++

%---------------------------------------------------------------------------------------------------------------------------
                    \subsection{Le pourquoi et le comment de la dérivée}
%---------------------------------------------------------------------------------------------------------------------------

La notion de dérivée est associée à la recherche de la droite tangente à une courbe. Reprenons rapidement le cheminement. La dérivée de $f\colon \eR\to \eR$ au point $a$ est un nombre $f'(a)$, qui définit donc une application linéaire dont le coefficients angulaire est $f'(a)$, et que nous notons $df_a$ :
\begin{equation}
    \begin{aligned}
        df_a\colon \eR&\to \eR \\
        u&\mapsto f'(a)u. 
    \end{aligned}
\end{equation}
La droite donnée par l'équation
\begin{equation}
    y(a+u)=f'(a)u
\end{equation}
est parallèle à la tangente en $a$. Pour trouver la tangente, il suffit de la décaler de la hauteur qu'il faut. L'équation de la droite tangente au graphe de $f$ au point $\big( a,f(a) \big)$ devient
\begin{equation}        \label{EqDiffRapTgDer}
    y(x)=f(a)+f'(a)(x-a)=f(a)+df_a(x-a).
\end{equation}
Nous nous proposons de généraliser cette formule au cas de la recherche du plan tangent à une surface.
 
%---------------------------------------------------------------------------------------------------------------------------
                    \subsection{Dérivée partielle et directionnelles}
%---------------------------------------------------------------------------------------------------------------------------

Soit une fonction $f:A\subset \mathbb{R}^n \rightarrow \mathbb{R}^m$. Si $n\neq 1$, la notion de \emph{dérivée} de la fonction $f$ n'a plus de sens puisqu'on ne peut plus parler de pente de \emph{la} tangente au graphe de $f$ en un point. On introduit alors quelque notions qui feront, en dimension quelconque, le même travail que la dérivée en dimension un : les dérivées directionnelles et la différentielle. Nous allons voir qu'en dimension un, la différentielle coïncide avec la dérivée.


\begin{definition} 
    Soit un point $a \in int\,A$ et un vecteur $u \in \mathbb{R}^n$ avec $\| u \| =1$. La dérivée de $f$ au point $a$ dans la direction $u$ est donnée par la limite suivante, si elle existe 
    \begin{equation}
        \frac{\partial f}{\partial u}(a) = \lim_{t\rightarrow 0}\frac{f(a+tu) - f(a)}{t}
    \end{equation}
\end{definition}

Géométriquement, il s'agit du taux de variation instantané de $f$ en $a$ dans la direction du vecteur $u$, c'est-à-dire de la pente de la tangente dans la direction du vecteur $u$ au graphe de $f$ au point $(a, f(a))$.

\begin{remark}
On peut reformuler la définition en écrivant $x = a + u$, on obtient~:
\begin{equation}
    \limite[u\neq 0]{u}{0} \frac{f(a+u)-f(a)-T(u)}{\norme{u}} = 0.
\end{equation}
\end{remark}

\begin{remark}
Pourquoi avons-nous posé la condition $\| u \|=1$ ? Le but de la dérivée directionnelle dans la direction $u$ est de savoir à quelle vitesse la fonction monte lorsque l'on se déplace en suivant la direction $u$. Cette information n'aura un caractère \og objectif\fg{} que si l'on avance à une vitesse donnée. En effet, si on se déplace deux fois plus vite, la fonction montera deux fois plus vite. Par convention, nous demandons donc d'avancer à vitesse $1$.
\end{remark}

\subsubsection*{Cas particulier où $n=2$:} $a = (a_1, a_2)$, $u =
(u_1,u_2)$ et
$$\frac{\partial f}{\partial u}(a_1, a_2) = \lim_{t\rightarrow
0}\frac{f(a_1+tu_1,a_2+tu_2) - f(a_1, a_2)}{t}$$

Un cas particulier des dérivées directionnelles est la dérivée partielle. Si nous considérons la base canonique $e_i$ de $\eR^n$, nous notons
\begin{equation}
    \frac{ \partial f }{ \partial x_i }=\frac{ \partial f }{ \partial e_i }.
\end{equation}
Dans le cas d'une fonction à deux variables, nous avons donc les deux dérivées partielles
\begin{equation}
    \begin{aligned}[]
        \frac{ \partial f }{ \partial x }(a)&&\text{et}&&\frac{ \partial f }{ \partial y }(a)
    \end{aligned}
\end{equation}
qui correspondent aux dérivées directionnelles dans les directions des axes. Ces deux nombres représentent de combien la fonction $f$ monte lorsqu'on part de $a$ en se déplaçant dans le sens des axes $X$ et $Y$.

%///////////////////////////////////////////////////////////////////////////////////////////////////////////////////////////
                    \subsubsection{Quelque propriétés et notations}
%///////////////////////////////////////////////////////////////////////////////////////////////////////////////////////////

\begin{enumerate}
\item
 $\forall \alpha \in \mathbb{R}$,
si $v = \alpha\,u$, alors $\frac{\partial f}{\partial v}(a) =
\alpha\,\frac{\partial f}{\partial u}(a)$.
\item Si on prend $u=e_j$ le $j$ème vecteur de la base canonique de
$\mathbb{R}^n$, alors
$$\frac{\partial f}{\partial e_j}(a) = \frac{\partial f}{\partial
x_j}(a)$$ c'est-à-dire que la dérivée de $f$ au point $a$ dans la
direction $e_j$ est la dérivée partielle de $f$ par rapport à sa
$j$ème variable.

\item 
Une fonction peut être dérivable dans certaines directions
mais pas dans d'autres (rappelez vous que si la limite à droite est
différente de la limite à gauche, la limite n'existe pas). 

\item
Même si une fonction est dérivable en un point dans toutes les
directions, on n'est pas sûr qu'elle soit continue en ce point. La
dérivabilité directionnelle n'est donc pas une notion suffisante
pour assurer la continuité. C'est pourquoi on introduit le concept
de \emph{différentiabilité}. 
\end{enumerate}

%---------------------------------------------------------------------------------------------------------------------------
                    \subsection{Différentielle}
%---------------------------------------------------------------------------------------------------------------------------

\begin{definition}      \label{DefDifferentiablFnRn}
Soit un point $a \in int\,A$. La fonction $f$ est \defe{différentiable}{différentiable} au point $a$ si il existe une application linéaire $df_a\colon \eR^n\to \eR^m$ telle que 
\begin{equation}        \label{EqDefDiffableT}
    \lim_{x\to a} \frac{f(x) - f(a) - df_a (x-a)}{\|x-a\|}=0.
\end{equation}
\end{definition}

Si $f$ est différentiable en $a$, l'application $df_a$ est appelée la différentielle de $f$ en $a$. Voyons comment cette application linéaire agit sur les vecteurs de $\mathbb{R}^n$.

Le théorème suivant reprend pas principales propriétés d'une fonction différentiable.
\begin{theorem}     \label{ThoRapPropDiffSi}
Si $f$ est différentiable en $a\in\eR^n$, alors
\begin{enumerate}
\item $f$ est continue en $a$.

\item  Toute les dérivées directionnelles $\partial_uf(a)$ existent et nous avons l'égalité
\begin{equation}        \label{EqDiffPartRap}
    \begin{aligned}
        df_a\colon \eR^n&\to \eR^m \\
        u&\mapsto df_a(u)=\frac{ \partial f }{ \partial u }(a)=\sum_i \frac{ \partial f }{ \partial x_i }u^i,
    \end{aligned}
\end{equation}
si les $u^i$ sont les composantes de $u$ dans la base canonique de $\eR^n$.

La différentielle de $f$ en $a$ envoie donc un vecteur $u$ sur la dérivée directionnelle de $f$ au point $a$ dans la direction $u$. 

\item\label{ItemThoDiffSiLin} L'application $df_a$ est une application linéaire.
\end{enumerate}
\end{theorem}
Le point \ref{ItemThoDiffSiLin} est évidement contenu dans la définition de la différentielle, mais c'est bien de la remettre en toute lettre. En regard avec la formule \eqref{EqDiffPartRap}, elle dit que $\partial_uf(a)$ est linéaire par rapport à $u$.

\subsubsection*{Cas particuliers} \begin{description} \item $n=1$:
$f: \mathbb{R}\rightarrow \mathbb{R}$ est dérivable en $a$ si et
seulement si $f$ est différentiable en $a$ et
$$df_a:\mathbb{R}\rightarrow \mathbb{R}: x \mapsto df_a(x) =
f'(a)\,.\,x$$ \item $n=2$: $f$ est différentiable en $a =(a_1, a_2)$
si et seulement si
$$\lim_{(v_1,v_2)\rightarrow (0,0)} \frac{f(a_1+v_1, a_2+v_2) - f(a_1,a_2) - [ \frac{\partial f}{\partial x}(a)\,v_1+
\frac{\partial f}{\partial y}(a)\,v_2]}{\sqrt{v_1^2+v_2^2}} = 0
$$\end{description}\vspace{0.3cm}


Parmi les vecteurs $u \in \mathbb{R}^n$, un vecteur d'origine $(a,
f(a))$ se distingue des autres: le vecteur gradient de $f$ en $a$
donnant la direction de plus grande pente de $f$ en
$a$.\vspace{0.3cm}


\begin{definition}
La courbe de niveau de $f$ associée à a est donnée par
$$ S_a = f^{-1}\,(f(a)) = \{(x_1, \ldots, x_n)\in \mathbb{R}^n : f(x_1, \ldots,
x_n)=f(a) \}$$
\end{definition}

%---------------------------------------------------------------------------------------------------------------------------
\subsection{Règles de calculs}
%---------------------------------------------------------------------------------------------------------------------------

\begin{proposition}[Règles de calculs] Soient $f$ et $g$ des fonctions
  différentiables en $g(a)$ et $a$ respectivement, alors la composée
  $f\circ g$ est différentiable en $a$ et
  \begin{equation*}
    d (f\circ g)_a = d f_{g(a)} \circ d g_a
  \end{equation*}
  et de plus les jacobiennes correspondantes vérifient
  \begin{equation*}
    \Jac (f\circ g)_a = \Jac f_{g(a)} \Jac g_a
  \end{equation*}
  où le membre de droite est le produit (non-commutatif !) des deux matrices.
\end{proposition}

\begin{corollary}[Chain rule] Si $f : \eR^p \to \eR$ et $g : \eR \to
  \eR^p$, alors
  \begin{equation*}
    (f\circ g)^\prime(t) = \sum_{i=1}^p \pder f {x_i}(g(t)) g_i^\prime(t).
  \end{equation*}
\end{corollary}

\begin{remark}
  \begin{enumerate}
  \item Si $p = 1$, on retrouve la règle usuelle de dérivation de
    fonctions composées.

  \item 
      Si $g$ est à plusieurs variables, cette règle permet de déterminer les dérivées partielles de $f \circ g$, puisqu'une dérivée partielle peut être vue comme dérivée usuelle par rapport à une seule variable (voir remarque page \pageref{deriveepartielles}).

  \item Si $f$ est à valeurs vectorielles, cette formule permet de
    retrouver la jacobienne de $f \circ g$ puisqu'il suffit de traiter
    chaque composante de $f$ séparément.
  \end{enumerate}
\end{remark}


%---------------------------------------------------------------------------------------------------------------------------
                    \subsection{Gradient et recherche du plan tangent}
%---------------------------------------------------------------------------------------------------------------------------

Nous avons maintenant en main les concepts utiles pour trouver l'équation du plan tangent à une surface.

De la même manière que la tangente à une courbe était la droite de coefficient angulaire donné par la dérivée, maintenant, le plan tangent à une surface est le plan dont les vecteurs directeurs sont les dérivées partielles :

La généralisation de l'équation \eqref{EqDiffRapTgDer} est 
\begin{equation}        \label{EqDefPlanTag}
    T_a(x)=f(a)+\sum_i\frac{ \partial f }{ \partial x_i }(a)(x-a)^i
\end{equation}

Nous introduisons aussi souvent l'opérateur différentiel abstrait \defe{nabla}{nabla}, noté $\nabla$ et qui est donné par le vecteur
\begin{equation}
    \nabla=\left( \frac{ \partial  }{ \partial x_1 },\ldots,\frac{ \partial  }{ \partial x_n } \right).
\end{equation}
Les égalités suivantes sont juste des notations, sommes toutes logiques, liées à $\nabla$ :
\begin{equation}
    \nabla f=\left( \frac{ \partial f }{ \partial x_1 },\ldots,\frac{ \partial f }{ \partial x_n } \right),
\end{equation}
et
\begin{equation}        \label{EqDefGradient}
    \nabla f(a) = \left(\frac{\partial f}{\partial x_1}(a), \frac{\partial f}{\partial x_2}(a), \ldots, \frac{\partial f}{\partial x_n}(a)\right).
\end{equation}
Ce dernier est un élément de $\eR^n$ : chaque entrée est un nombre réel.

\begin{definition} 
Le vecteur gradient de $f$ au point $a$ est le vecteur donné par la formule \eqref{EqDefGradient}.
\end{definition}
La notation $\nabla$ permet d'écrire la différentielle sous forme un peu plus compacte. En effet, la formule \eqref{EqDiffPartRap} peut être notée
\begin{equation}
    df_a(u)=\scal{\nabla f(a)}{u}.
\end{equation}

En utilisant ce produit scalaire, l'équation \eqref{EqDefPlanTag} peut se récrire
\begin{equation}
    T_a(x)=f(a)+\sum_i\frac{ \partial f }{ \partial x_i }(a)(x-a)^i=f(a)+\scal{\nabla f(a)}{x-a}.
\end{equation}

Affin d'éviter les confusions, il est parfois souhaitable de bien mettre les parenthèses et noter $(\nabla f)(a)$ au lieu de $\nabla f(a)$.

\begin{proposition}
$\nabla f(a)\,\bot \,S_a$
\end{proposition}


\begin{equation}        \label{EqPlanTgSansNabla}
    z=f(a)+\sum_i\frac{ \partial f }{ \partial f }(a)(x-a)^i.
\end{equation}

\subsubsection*{Cas particulier où $n=2$:} 
Le plan $T_a$ avec $a=(a_1,a_2)$ a pour équation dans $\eR^3$:
\begin{equation}        \label{EqPlanTgEnDimDeux}
    z = f(a_1,a_2) + \frac{\partial f}{\partial x}(a_1,a_2)\,(x-a_1)+ \frac{\partial f}{\partial y}(a_1,a_2)\,(y-a_2).
\end{equation}

\begin{definition}
  Soit $f : \eR^n \to\eR$ une fonction différentiable en un point
  $a$. Le \emph{plan tangent} au graphe de $f$ en $(a,f(a))$ est
  l'ensemble des points
  \begin{equation*}
    \begin{split}
      T_af &= \{ (x,z) \in \eR^n \times \eR \tq z = f(a) + d f_a (x-a)\}\\
      &= \{ (x,z) \in \eR^n \times \eR \tq z = f(a) + \scalprod{\nabla f(a)}{x-a}\}
    \end{split}
  \end{equation*}
\end{definition}

%---------------------------------------------------------------------------------------------------------------------------
                    \subsection{Différentielle comme élément de l'espace dual}
%---------------------------------------------------------------------------------------------------------------------------


Si nous considérons la base canonique $\{ e_i \}_{i=1,\ldots,n}$ de $\eR^n$. À partir d'elle, nous considérons la \defe{base duale}{Base duale}. En termes pratiques, nous définissons $dx_i$ comme la forme sur $\eR^n$ qui à un vecteur $u$ fait correspondre sa composante $i$ :
\begin{equation}
    dx_i\begin{pmatrix}
    u^1 \\ 
    \vdots  \\ 
    u^n 
\end{pmatrix}=u^i.
\end{equation}
En termes savants, $dx_i$ est le dual de $e_i$. Si tu ne l'as pas encore compris, Jean Doyen va te le faire comprendre !


Maintenant, dans la formule \eqref{EqDiffPartRap}, nous pouvons remplacer $u^i$ par $dx_i(u)$, et écrire
\begin{equation}
    df_a(u)=\sum_i\frac{ \partial f }{ \partial x_i }(a)u^i=\sum_i\frac{ \partial f }{ \partial x_i }(a)dx_i(u).
\end{equation}
Ce qui arrive tout à droite est explicitement vu comme une forme sur $\eR$, dont les composantes dans la base duale sont les dérivées partielles de $f$ au point $a$, agissant sur $u$. En faisant un pas en arrière, nous omettons le $u$, et nous écrivons
\begin{equation}
    df_a=\sum_{i=1}^n\frac{ \partial f }{ \partial x_i }(a)dx^i
\end{equation}

Cette notation $dx_i$ pour la forme duale de $e_i$ est en réalité parfaitement logique parce que $dx^i$ est la différentielle de la projection
\begin{equation}
    \begin{aligned}
        x^i\colon \eR^n&\to \eR \\
        (x^1,\ldots,x^n)&\mapsto x^i. 
    \end{aligned}
\end{equation}
Je te laisse un peu méditer sur cette différentielle de la projection. L'important est que tu aies compris cela d'ici la fin de ta deuxième année.


%---------------------------------------------------------------------------------------------------------------------------
                    \subsection{Prouver qu'un fonction n'est pas différentiable}
%---------------------------------------------------------------------------------------------------------------------------

Chacun des point du théorème \ref{ThoRapPropDiffSi} est en soi un critère pour montrer qu'une fonction n'est pas différentiable en un point.

%///////////////////////////////////////////////////////////////////////////////////////////////////////////////////////////
                    \subsubsection{Continuité}
%///////////////////////////////////////////////////////////////////////////////////////////////////////////////////////////


Le premier critère à vérifier est donc la continuité. Si une fonction n'est pas continue en un point, alors elle n'y sera pas différentiable. Pour rappel, la continuité en $a$ se teste en vérifiant si $\lim_{x\to a}f(x)=f(a)$.

%///////////////////////////////////////////////////////////////////////////////////////////////////////////////////////////
                    \subsubsection{Linéarité}
%///////////////////////////////////////////////////////////////////////////////////////////////////////////////////////////

Un second test est la linéarité de la dérivée directionnelle par rapport à la direction : l'application $u\mapsto\frac{ \partial f }{ \partial u }(a)$ doit être linéaire, sinon $df_a$ n'existe pas.

\begin{example}     \label{Exemple0046Diff}
Examinons la fonction
\begin{equation}
    \begin{aligned}
        f\colon \eR^2&\to \eR \\
        (x,y)&\mapsto \begin{cases}
    \frac{ xy^2 }{ x^2+y^4 }    &   \text{si $(x,y)\neq (0,0)$}\\
    0   &    \text{sinon}.
\end{cases}
    \end{aligned}
\end{equation}
Prenons $u=(u_1,u_2)$ et calculons la dérivée de $f$ dans la direction de $u$ au point~$(0,0)$ :
\begin{equation}
    \begin{aligned}[]
        \frac{ \partial f }{ \partial u }(0,0)  
            &=\lim_{t\to 0}\frac{ f(tu_1,tu_2)-f(0,0) }{ t }\\
            &=\lim_{t\to 0}\frac{1}{ t }\left( \frac{ tu_1t^2u_2 }{ t^2u_1^2+t^4u_2^4 } \right)\\
            &=\lim_{t\to 0}\left( \frac{ u_1u_2^2 }{ u_1^2+t^2u_2^4 } \right)\\
            &=\begin{cases}
    \frac{ u_2^2 }{ u_1 }   &   \text{si $u_1\neq 0$}\\
    0   &    \text{si $u_1=0$}.
\end{cases}
    \end{aligned}
\end{equation}
Cette application n'est pas linéaire par rapport à $u$. En effet, notons
\begin{equation}
    \begin{aligned}
        A\colon \eR^n&\to \eR \\
        u&\mapsto \frac{ \partial f }{ \partial u }(0,0), 
    \end{aligned}
\end{equation}
et vérifions que pour tout $u$ et $v$ dans $\eR^n$ et $\lambda\in\eR$, nous ayons $A(\lambda u)=\lambda A(u)$ et $A(u+v)=A(u)+A(v)$. Le premier fonctionne parce que
\begin{equation}
    A(\lambda u)=A(\lambda u_1,\lambda u_2)=\frac{ \lambda^2 u_2^2 }{ \lambda u_1 }=\lambda\frac{ u_2^2 }{ u_1 }=\lambda A(u).
\end{equation}
Mais nous avons par exemple
\begin{equation}
    A\big( (0,1)+(2,3) \big)=A(2,4)=\frac{ 16 }{ 2 }=8,
\end{equation}
tandis que
\begin{equation}
    A(0,1)+A(2,3)=0+\frac{ 9 }{ 2 }\neq 8.
\end{equation}
La fonction $f$ n'est donc pas différentiable en $(0,0)$, parce que la candidate différentielle, $df_{(0,0)}(u)=\frac{ \partial f }{ \partial u }(0,0)$, n'est même pas linéaire.

Nous verrons, dans l'exercice \ref{exo0046}, une autre manière de traiter cette fonction.
\end{example}

%///////////////////////////////////////////////////////////////////////////////////////////////////////////////////////////
                    \subsubsection{Cohérence des dérivées partielles et directionnelle}
%///////////////////////////////////////////////////////////////////////////////////////////////////////////////////////////

Dans la pratique, nous pouvons calculer $\partial_uf(a)$ pour une direction $u$ générale, et puis en déduire $\partial_xf$ et $\partial_yf$ comme cas particuliers en posant $u=(1,0)$ et $u=(0,1)$. Une chose incroyable, mais pourtant possible est qu'il peut arriver que
\begin{equation}
    \frac{ \partial f }{ \partial u }(a)\neq \sum_i\frac{ \partial f }{ \partial x_i }(a)u^i.
\end{equation}
Ceci se produit lorsque $f$ n'est pas différentiable en $a$. En voici un exemple.

%///////////////////////////////////////////////////////////////////////////////////////////////////////////////////////////
                    \subsubsection{Un candidat dans la définition (marche toujours)}
%///////////////////////////////////////////////////////////////////////////////////////////////////////////////////////////

Lorsqu'une fonction est donné, un candidat différentielle au point $(a_1,a_2)$ est souvent assez simple à trouver en un point :
\begin{equation}
    T(u_1,u_2)=\frac{ \partial f }{ \partial x }(a_1,a_2)u_1+\frac{ \partial f }{ \partial y }(a_1,a_2)u_2.
\end{equation}
L'application $T$ est la candidate différentielle en ce sens que si la différentielle existe, alors elle est égale à $T$. Ensuite, il faut vérifier si
\begin{equation}        \label{EqLimDefDiff}
    \lim_{(x,y)\to (a_1,a_2)} \frac{f(x,y) - f(a_1,a_2) - T\big( (x,y)-(a_1,a_2) \big)}{\| (x,y)-(a_1,a_2) \|}=0
\end{equation}
ou non. Si oui, alors la différentielle existe et $df_{(a,b)}(u)=T(u)$, sinon\footnote{y compris si la limite \eqref{EqLimDefDiff} n'existe même pas.}, la différentielle n'existe pas.

Attention : dans la ZAP, les dérivées partielles $\partial_xf$ et $\partial_yf$ ne peuvent en général pas être calculées en utilisant les règles de calcul (c'est bien pour ça que la ZAP est une zone à problèmes). Il faut d'office utiliser la définition
\begin{equation}
    \frac{ \partial f }{ \partial x }(a_1,a_2)=\lim_{t\to 0}\frac{ f(a_1+t,a_2)-f(a_1,a_2) }{ t },
\end{equation}
et la définition correspondante pour $\partial_yf$.


\subsubsection*{Conclusion}
Soient $f:A\subset \eR^n \rightarrow \eR^m$, et $a\in int\,A$. Si $f$ est différentiable en $a$, $$ (df_a (e_j))_i = d(f_i)_a(e_j) =\frac{\partial f_i}{\partial x_j}(a)= [Jac(f)_{|a}]_{ij}$$ et la matrice de l'application linéaire $df_a$ est la matrice jacobienne $m\times n$ de $f$ en $a$ notée $Jac(f)_{|a}$.


%---------------------------------------------------------------------------------------------------------------------------
                    \subsection{Calcul de différentielles}
%---------------------------------------------------------------------------------------------------------------------------


\begin{remark}      \label{deriveepartielles}
  En pratique, ayant une formule pour la fonction $f$, on dérive --grâce aux règles usuelles de dérivation-- par rapport à la variable $x_i$ en considérant que les autres ($x_j$ avec $j \neq i$) sont des constantes.
\end{remark}

\begin{example}Pour $f(x,y) = xy + x^2$, les dérivées partielles
  s'écrivent
  \begin{equation*}
    \frac{\partial f}{\partial x} = y + 2x \quad\text{et}\quad \frac{\partial f}{\partial y} = x
  \end{equation*}
\end{example}


Des \emph{règles de calcul} sont d'application. En particulier, quand
ces opérations existent, les sommes, différences, produits, quotients
et compositions d'applications différentiables sont différentiables.

Toute application linéaire est différentiable, et sa différentielle en
tout point est égale à l'application elle-même. En particulier, les
\Defn{projections canoniques}, c'est-à-dire les applications du type
$(x,y,z) \mapsto y$, sont linéaires donc différentiables.

\begin{example}
Les cas suivants sont faciles :
  \begin{enumerate}
  \item En combinant les projections canoniques avec les règles de
    calculs, on obtient que toute fonction polynômiale à $n$ variables
    est différentiable comme application de $\eR^n$ dans $\eR$.

  \item Toute fonction rationnelle, du type $f(x) \pardef
    \frac{P(x)}{Q(x)}$ où $P$ et $Q$ sont des polynômes, est
    différentiable en tout point $a$ tel que $Q(a) \neq 0$.

  \item Pour une fonction d'une variable $f : D \subset \eR \to
    \eR$, le caractère différentiable et le caractère dérivable
    coïncident. De plus, on a
    \begin{equation*}
      d f_a(u) = f'(a) u.
    \end{equation*}
  \end{enumerate}
\end{example}

%---------------------------------------------------------------------------------------------------------------------------
                    \subsection{Notes idéologiques sur le concept de plan tangent}
%---------------------------------------------------------------------------------------------------------------------------
\label{ssecConceptPlanTag}

Notons $G$, le graphe d'une fonction $f$, c'est à dire
\begin{equation}
    G=\{ (x,y,z)\in\eR^3\tq z=f(x,y) \}.
\end{equation}
Première affirmation : si $\gamma\colon \eR\to G$ est une courbe telle que $\gamma(0)=\big( a,f(a) \big)$, alors $\gamma'(0)\in\eR^n$ est dans le plan tangent à $G$ au point $\big( a,f(a) \big)$.

Plus fort : tous les éléments du plan tangent sont de cette forme.

Le plan tangent à $G$ en un point $x\in G$ est donc constitué des vecteurs vitesse de tous les chemins qui passent par $x$.

Prenons maintenant $S$, une courbe de niveau de $G$, c'est à dire
\begin{equation}
    S=\{ (x,y)\in\eR^2\tq f(x,y)=C \}.
\end{equation}
Si nous prenons un chemin dans $G$ qui est, de plus, contraint à $S$, c'est à dire tel que $\gamma(t)\in S$, alors $\gamma'(0)$ sera tangent à $G$ (ça, on le savait déjà), mais en plus, $\gamma'(0)$ sera tangent à $S$, ce qui est logique.

La morale est que si vous prenez un chemin qui se ballade dans n'importe quoi, alors la dérivée du chemin sera un vecteur tangent à ce n'importe quoi.

En outre, si $\gamma(t)\in S$ et $\gamma(0)=a$, alors
\begin{equation}
    \scal{\nabla f(a)}{\gamma'(0)}=0,
\end{equation}
c'est à dire que le vecteur tangent à la courbe de niveau est perpendiculaire au gradient. Cela est intuitivement logique parce que la tangente à la courbe de niveau correspond à la direction de \emph{moins} grande pente.

%+++++++++++++++++++++++++++++++++++++++++++++++++++++++++++++++++++++++++++++++++++++++++++++++++++++++++++++++++++++++++++
\section{Jacobienne}
%+++++++++++++++++++++++++++++++++++++++++++++++++++++++++++++++++++++++++++++++++++++++++++++++++++++++++++++++++++++++++++

\subsection{Rappels et définitions}

Dans cette section nous considérons des fonctions $f : D \to \eR^m$
où $D \subset \eR^n$, et un point $a \in \interieur D$ où $f$ est
différentiable.
\begin{remark}
  La définition de continuité (resp. différentiabilité) pour une
  fonction à valeurs vectorielles est celle introduite précédemment,
  et on remarque que pour avoir la continuité
  (resp. différentiabilité) de $f$ en un point, il faut et il suffit
  de chacune des composantes de $f = (f_1,\ldots, f_m)$, vues
  séparément comme fonctions à $n$ variables et à valeurs réelles,
  soit continue (resp. différentiable) en ce point.
\end{remark}

\begin{definition}La \Defn{jacobienne} de $f$ en $a$ est la matrice
  \begin{equation*}
    (\Jac f)_a \begin{pmatrix}
      \pder {f_1} {x_1}(a) & \ldots &\pder {f_1} {x_n}(a)\\
      \vdots& & \vdots\\
      \pder {f_m} {x_1}(a) & \ldots &\pder {f_m} {x_n}(a)
    \end{pmatrix}
  \end{equation*}
  composée de l'ensemble des dérivées partielles de $f$.

  Si $m = 1$, cette matrice ne contient qu'une ligne ; c'est donc un
  vecteur appelé le \Defn{gradient de $f$ en $a$} et noté $\nabla f(a)$.
\end{definition}

\begin{remark}
  \begin{enumerate}
  \item Si la fonction est supposée différentiable, calculer la
    jacobienne revient à connaître la différentielle. En effet, par
    linéarité de la différentielle et par définition des dérivées
    partielles, nous avons
    \begin{equation*}
      d f_a (u) =%
      \begin{pmatrix}
        \pder {f_1} {x_1}(a) & \ldots &\pder {f_1} {x_n}(a)\\
        \vdots& & \vdots\\
        \pder {f_m} {x_1}(a) & \ldots &\pder {f_m} {x_n}(a)
      \end{pmatrix}
      \begin{pmatrix}u_1\\\vdots\\u_n\end{pmatrix}
    \end{equation*}
    où $u = (u_1, \ldots, u_n)$ et où le membre de droite est un
    produit matriciel

  \item Remarquons que la jacobienne peut exister en un point donné
    sans que la fonction soit différentiable en ce point !
  \end{enumerate}
\end{remark}

%+++++++++++++++++++++++++++++++++++++++++++++++++++++++++++++++++++++++++++++++++++++++++++++++++++++++++++++++++++++++++++
					\section{Théorème de la fonction implicite}
%+++++++++++++++++++++++++++++++++++++++++++++++++++++++++++++++++++++++++++++++++++++++++++++++++++++++++++++++++++++++++++

%---------------------------------------------------------------------------------------------------------------------------
\subsection{Mise en situation}
%---------------------------------------------------------------------------------------------------------------------------

Dans un certain nombre de situation, il n'est pas possible de trouver des solutions explicites aux équations qui apparaissent. Néanmoins, l'existence «théorique» d'une telle solution est souvent déjà suffisante. C'est l'objet du théorème de la fonction implicite.

Prenons par exemple la fonction sur $\eR^2$ donnée par 
\begin{equation}
	F(x,y)=x^2+y^2-1.
\end{equation}
Nous pouvons bien entendu regarder l'ensemble des points donnés par $F(x,y)=0$. C'est le cercle dessiné à la figure \ref{LabelFigCercleImplicite}.
\newcommand{\CaptionFigCercleImplicite}{Un cercle pour montrer l'intérêt de la fonction implicite. Si on donne \( x\), nous ne pouvons pas savoir si nous parlons de \( P\) ou de \( P'\).}
\input{Fig_CercleImplicite.pstricks}

%\ref{LabelFigCercleImplicite}.
%\newcommand{\CaptionFigCercleImplicite}{Un cercle pour montrer l'intérêt de la fonction implicite.}
%\input{Fig_CercleImplicite.pstricks}

Nous ne pouvons pas donner le cercle sous la forme $y=y(x)$ à cause du $\pm$ qui arrive quand on prend la racine carrée. Mais si on se donne le point $P$, nous pouvons dire que \emph{autour de $P$}, le cercle est la fonction
\begin{equation}
	y(x)=\sqrt{1-x^2}.
\end{equation}
Tandis que autour du point $P'$, le cercle est la fonction
\begin{equation}
	y(x)=-\sqrt{1-x^2}.
\end{equation}
Autour de ces deux point, donc, le cercle est donné par une fonction. Il n'est par contre pas possible de donner le cercle autour du point $Q$ sous la forme d'une fonction.

Ce que nous voulons faire, en général, est de voir si l'ensemble des points tels que
\begin{equation}
	F(x_1,\ldots,x_n,y)=0
\end{equation}
peut être donné par une fonction $y=y(x_1,\ldots,x_n)$. En d'autre termes, est-ce qu'il existe une fonction $y(x_1,\ldots,x_n)$ telle que
\begin{equation}
	F\big( x_1,\ldots,x_n,y(x_1,\ldots,x_n)\big)=0.
\end{equation}



\subsection{Définitions et rappels}
Soit
\begin{equation}
    \begin{aligned}
        F\colon D\subset \eR^n\times \eR^m&\to \eR^m \\
        (x,y)&\mapsto \big( F_1(x,y),\ldots, F_m(x,y) \big) 
    \end{aligned}
\end{equation}
avec $x = (x_1,\ldots, x_n)$ et $y = (y_1,\ldots,y_m)$.

Pour chaque $x$ fixé, on s'intéresse aux solutions du système de $m$
 équations $F(x,y) = 0$ pour les inconnues $y$ ; en particulier, on
 voudrait pouvoir écrire $y = \varphi(x)$ vérifiant $F(x,\varphi(x)) = 0$.

Pour $(x,y) \in \interieur D$, la matrice
\begin{equation}
    \frac{ \partial (F_1,\ldots, F_m) }{ \partial (y_1,\ldots, y_m) }=
\begin{pmatrix}
\pder {F_1}{y_1}(x,y)& \ldots& \pder {F_1}{y_m}(x,y)\\
\vdots& \ddots & \vdots\\
\pder {F_m}{y_1}(x,y)& \ldots& \pder {F_m}{y_m}(x,y)\\
\end{pmatrix}
\end{equation}
est la \defe{matrice jacobienne}{jacobienne!matrice} de $F$ par rapport à $y$ au point
$(x,y)$; son déterminant est appelé le \defe{jacobien}{jacobien} de $F$ par
rapport à $y$.

\begin{theorem}[théorème de la fonction implicite] \index{théorème!fonction implicite} \label{ThoAcaWho}
    Soit une fonction \( F\colon \eR^n\times \eR^m\to \eR^m\) de classe \( C^k\) et \( (\alpha,\beta)\in \eR^n\times \eR^m\) tels que
    \begin{enumerate}
        \item
            \( f(\alpha,\beta)=0\),
        \item
            $\det\frac{ \partial (F_1,\ldots, F_m) }{ \partial (y_1,\ldots, y_m) }\neq 0$.
    \end{enumerate}
    Alors il existe un voisinage ouvert \( V\) de \( \alpha\) dans \( \eR^n\), un voisinage ouvert \( W\) de \( \beta\) dans \( \eR^m\) et une application \( \varphi\colon V\to W\) de classe \( C^k\)  telle que pour \( (x,y)\in V\times W\) on ait
    \begin{equation}
        F(x,y)=0
    \end{equation}
    si et seulement si \( y=\varphi(x)\).
\end{theorem}
	
Le théorème de la fonction implicite a pour objet de donner l'existence de la fonction $\varphi$. Maintenant nous pouvons dire beaucoup de choses sur les dérivées de $\varphi$ en considérant la fonction
\begin{equation}
	x\mapsto F\big( x,\varphi(x) \big).
\end{equation}
Par définition de $\varphi$, cette fonction est toujours nulle. En particulier, nous pouvons dériver l'équation
\begin{equation}
	F\big( x,\varphi(x) \big)=0,
\end{equation}
et nous trouvons plein de choses.

%---------------------------------------------------------------------------------------------------------------------------
\subsection{Exemple}
%---------------------------------------------------------------------------------------------------------------------------

Prenons par exemple la fonction
\begin{equation}
	F\big( (x,y),z \big)=ze^z-x-y,
\end{equation}
et demandons nous ce que nous pouvons dire sur la fonction $z(x,y)$ telle que
\begin{equation}
	F\big( x,y,z(x,y) \big)=0,
\end{equation}
c'est à dire telle que
\begin{equation}		\label{EqDefZImplExemple}
	z(x,y) e^{z(x,y)}-x-y=0.
\end{equation}
pour tout $x$ et $y\in\eR$. Nous pouvons facilement trouver $z(0,0)$ parce que
\begin{equation}
	z(0,0) e^{z(0,0)}=0,
\end{equation}
donc $z(0,0)=0$.

Nous pouvons dire des choses sur les dérivées de $z(x,y)$. Voyons par exemple $(\partial_xz)(x,y)$. Pour trouver cette dérivée, nous dérivons la relation \eqref{EqDefZImplExemple} par rapport à $x$. Ce que nous trouvons est
\begin{equation}
	(\partial_xz)e^z+ze^z(\partial_xz)-1=0.
\end{equation}
Cette équation peut être résolue par rapport à $\partial_xz$~:
\begin{equation}
	\frac{ \partial z }{ \partial x }(x,y)=\frac{1}{ e^z(1+z) }.
\end{equation}
Remarquez que cette équation ne donne pas tout à fait la dérivée de $z$ en fonction de $x$ et $y$, parce que $z$ apparaît dans l'expression, alors que $z$ est justement la fonction inconnue. En général, c'est la vie, nous ne pouvons pas faire mieux.

Dans certains cas, on peut aller plus loin. Par exemple, nous pouvons calculer cette dérivée au point $(x,y)=(0,0)$ parce que $z(0,0)$ est connu :
\begin{equation}
	\frac{ \partial z }{ \partial x }(0,0)=1.
\end{equation}
Cela est pratique pour calculer, par exemple, le développement en Taylor de $z$ autour de $(0,0)$.


%+++++++++++++++++++++++++++++++++++++++++++++++++++++++++++++++++++++++++++++++++++++++++++++++++++++++++++++++++++++++++++
                    \section{Maximisation sans contraintes}
%+++++++++++++++++++++++++++++++++++++++++++++++++++++++++++++++++++++++++++++++++++++++++++++++++++++++++++++++++++++++++++

%---------------------------------------------------------------------------------------------------------------------------
                    \subsection{Maximisation à une variable}
%---------------------------------------------------------------------------------------------------------------------------

\begin{definition}
Soit $f\colon A\subset \eR\to \eR$ et $a\in A$. Le point $a$ est un \defe{maximum local}{maximum!local} de $f$ si il existe un voisinage $\mU$ de $a$ tel que $f(a)\geq f(x)$ pour tout $x\in\mU\cap A$. Le point $a$ est un \defe{maximum global}{maximum!global} si $f(a)\geq g(x)$ pour tout $x\in A$.
\end{definition}

La proposition basique à utiliser lors de la recherche d'extrema est la suivante :
\begin{proposition}
Soit $f\colon A\subset\eR\to \eR$ et $a\in\Int(A)$. Supposons que $f$ est dérivable en $a$. Si $a$ est un \href{http://fr.wikipedia.org/wiki/Extremum}{extremum} local, alors $f'(a)=0$.
\end{proposition}

La réciproque n'est pas vraie, comme le montre l'exemple de la fonction $x\mapsto x^3$ en $x=0$ : sa dérivée est nulle et pourtant $x=0$ n'est ni un maximum ni un minimum local. 

Cette proposition ne sert donc qu'à sélectionner des \emph{candidats} extremum. Afin de savoir si ces candidats sont des extrema, il y a la proposition suivante.
\begin{proposition}
Soit $f\colon I\subset \eR\to \eR$, une fonction de classe $C^k$ au voisinage d'un point $a\in\Int I$. Supposons que
\begin{equation}
    f'(a)=f''(a)=\ldots=f^{(k-1)}(a)=0,
\end{equation}
et que
\begin{equation}
    f^{(k)}(a)\neq 0.
\end{equation}
Dans ce cas,
\begin{enumerate}

\item
Si $k$ est pair, alors $a$ est un point d'extremum local de $f$, c'est un minimum si $f^{(k)}(a)>0$, et un maximum si $f^{(k)}(a)<0$,
\item
Si $k$ est impair, alors $a$ n'est pas un extremum local de $f$.

\end{enumerate}
\end{proposition}

Note : jusqu'à présent nous n'avons rien dit des extrema \emph{globaux} de $f$. Il n'y a pas grand chose à en dire. Si un point d'extremum global est situé dans l'intérieur du domaine de $f$, alors il sera extremum local (a fortiori). Ou alors, le maximum global peut être sur le bord du domaine. C'est ce qui arrive à des fonctions strictement croissantes sur un domaine compact.

Une seule certitude : si une fonction est continue sur un compact, elle possède une minimum et un maximum global.

%---------------------------------------------------------------------------------------------------------------------------
                    \subsection{Maximisation à plusieurs variables}
%---------------------------------------------------------------------------------------------------------------------------

%---------------------------------------------------------------------------------------------------------------------------
                    \subsection{Quelque mots à propos de matrices}
%---------------------------------------------------------------------------------------------------------------------------

Les notions qui suivent seront vues au cours de géométrie lorsqu'il sera temps. 

Si $g$ est une application bilinéaire sur $\eR^2$, nous disons qu'elle est
\begin{enumerate}

\item
\defe{Définie positive}{Application!définie positive} si $g(u,u)\geq 0$ pour tout $u\in\eR^2$ et $g(u,u)=0$ si et seulement si $u=0$.

\item
\defe{semi-définie positive}{Application!semi-définie positive} si $g(u,u)\geq 0$ pour tout $u\in\eR^2$. 

\end{enumerate}

Une matrice $M$ est définie positive si $v^tMv>0$ pour tout $v\neq 0$, en particulier si $v$ est un vecteur propre de valeur propre $\lambda$ (c'est à dire si $Mv=\lambda v$), alors $\lambda v^tv>0$, et donc $\lambda>0$. Donc $M$ sera définie positive si toutes ses valeurs propres sont positives.

\begin{proposition}
    Soit $M$, une matrice $2\times 2$ symétrique\footnote{la matrice $d^2f(a)$ est toujours symétrique quand $f$ est de classe $C^2$.}. Nous avons
    \begin{enumerate}
        \item
        $\det M>0$ et $\tr(M)>0$ implique $M$ définie positive,
        \item
        $\det M>0$ et $\tr(M)<0$ implique $M$ définie négative,
        \item
        $\det M<0$ implique ni semi définie positive, ni définie négative (et donc pas un extrema dans le cas où $M=d^2f(a)$ par le point \ref{ItemPropoExtreRn} de la proposition \ref{PropoExtreRn}),
        \item
        $\det M=0$ implique $M$ semi-définie positive ou semi-définie négative.
    \end{enumerate}
\end{proposition}
 

%---------------------------------------------------------------------------------------------------------------------------
                    \subsection{Les théorèmes}
%---------------------------------------------------------------------------------------------------------------------------



Un point $a$ à l'intérieur du domaine d'une fonction $f\colon A\subset\eR^n\to \eR$ est une \defe{point critique}{Critique!point} de $f$ lorsque $df(a)=0$. Ces points sont analogues aux points où la dérivée d'une fonction sur $\eR$ s'annule. Les points critiques de $f$ sont dons les candidats à être des points d'extremum.

\begin{proposition}     \label{PropoExtreRn}
Soit $f\colon A\subset\eR^n\to \eR$ une fonction de classe $C^2$ au voisinage de $a\in\Int(A)$.
\begin{enumerate}

\item
Si $a$ est un point critique de $f$, et si $d^2f(a)$ est \href{http://fr.wikipedia.org/wiki/Matrice_définie_positive}{définie positive}, alors $a$ est un minimum local strict de $f$,
\item       \label{ItemPropoExtreRn}
Si $a$ est un minimum local, alors $a$ est un point critique et $d^2f(a)$ est semi-définie positive.

\end{enumerate}
\end{proposition}

La seconde partie de l'énoncé est tout à fait comparable au fait bien connu que, pour une fonction $f\colon \eR\to \eR$, si le point $a$ est minimum local, alors $f'(a)=0$ et $f''(a)\geq 0$. Notez le fait que l'inégalité n'est pas stricte, ce qui correspond à $d^2f(a)$ \emph{semi}-definie positive.

Pour rappel, dans le cas d'une fonction à deux variables, $d^2f(a)$ est la matrice (et donc l'application linéaire)
\begin{equation}
    d^2f(a)=\begin{pmatrix}
    \frac{ d^2f  }{ dx^2 }(a)   &   \frac{ d^2f  }{ dx\,dy }(a) \\ 
    \frac{ d^2f  }{ dy\,dx }(a)     &   \frac{ d^2f  }{ dy^2 }(a)
\end{pmatrix}.
\end{equation}
Dans le cas d'une fonction $C^2$, cette matrice est symétrique.

La méthode pour chercher les extrema de $f$ est donc de suivre le points suivants :
\begin{enumerate}
\item
Trouver les candidats extrema en résolvant $\nabla f=(0,0)$,
\item
écrire $d^2f(a)$ pour chacun des candidats
\item
calculer les valeurs propres de $d^2f(a)$, déterminer si la matrice est définie positive ou négative,
\item
conclure.
\end{enumerate}

%+++++++++++++++++++++++++++++++++++++++++++++++++++++++++++++++++++++++++++++++++++++++++++++++++++++++++++++++++++++++++++
\section{Extrema liés}
%+++++++++++++++++++++++++++++++++++++++++++++++++++++++++++++++++++++++++++++++++++++++++++++++++++++++++++++++++++++++++++

Soit $f$, une fonction sur $\eR^n$, et $M\subset \eR^n$ une variété de dimension $m$. Nous voulons savoir quelle sont les plus grandes et plus petites valeurs atteintes par $f$ sur $M$.

Pour ce faire, nous avons un théorème qui permet de trouver des extrema \emph{locaux} de $f$ sur la variété. Pour rappel, $a\in M$ est une \defe{extrema local de $f$ relativement}{Extrema local!relatif} à l'ensemble $M$ si il existe une boule $B(a,\epsilon)$ telle que $f(a)\leq f(x)$ pour tout $x\in B(a,\epsilon)\cap M$.

\begin{theorem}[Extrema lié] \label{ThoRGJosS}
	Soit \( A\), un ouvert de \( \eR^n\) et
	\begin{enumerate}
		\item
			une fonction (celle à minimiser) $f\in C^1(A,\eR)$,
		\item 
			des fonctions (les contraintes) $G_1,\ldots,G_r\in C^1(A,\eR)$,
		\item
			$M=\{ x\in A\tq G_i(x)=0\,\forall i\}$,
		\item
			un extrema local $a\in M$ de $f$ relativement à $M$.
	\end{enumerate}
	Supposons que les gradients $\nabla G_1(a)$, \ldots,$\nabla G_r(a)$ soient linéairement indépendants. Alors $a=(x_1,\ldots,x_n)$ est une solution de \( \nabla L(a)=0\) où
	\begin{equation}
		L(x_1,\ldots,x_n,\lambda_1,\ldots,\lambda_r)=f(x_1,\ldots,x_n)+\sum_{i=1}^r\lambda_iG_i(x_1,\ldots,x_n).
	\end{equation}
    Autrement dit, si \( a\) est un extrema lié, alors \( \nabla f(a)\) est une combinaisons des \( \nabla G_i(a)\).
\end{theorem}
La fonction $L$ est le \defe{lagrangien}{lagrangien} du problème et les variables \( \lambda_i\) sont les \defe{multiplicateurs de Lagrange}{multiplicateur!de Lagrange}.

La preuve provient de \cite{ytMOpe} dans lequel il est énoncé sous la forme \ref{PropfPPUxh}\footnote{Il y a donc adaptation des notations et donc sûrement des fautes de frappe; soyez prudent.}.
\begin{proof}
    Si \( r=n\) alors les vecteurs linéairement indépendantes \( \nabla G_i(a) \) forment une base de \( \eR^n\) et donc évidemment les \( \lambda_i\) existent. Nous supposons donc maintenant que \( r<n\). Nous notons \( (z_i)_{i=1\ldots n}\) les coordonnées sur \( \eR^n\).
    
    La matrice
    \begin{equation}
        \begin{pmatrix}
            \frac{ \partial G_1 }{ \partial z_1 }(a)    &   \cdots    &   \frac{ \partial G_1 }{ \partial z_n }(a)    \\
            \vdots    &   \ddots    &   \vdots    \\
            \frac{ \partial G_r }{ \partial z_1 }(a)    &   \cdots    &   \frac{ \partial G_r }{ \partial z_n }(a)
        \end{pmatrix}
    \end{equation}
    est de rang \( r\) parce que les lignes sont par hypothèses linéairement indépendantes. Nous nommons \( (y_i)_{i=1,\ldots, r}\) un choix de \( r\) parmi les \( (z_i)\) tels que
    \begin{equation}
        \det\begin{pmatrix}
            \frac{ \partial G_1 }{ \partial y_1 }    &   \ldots    &   \frac{ \partial G_1 }{ \partial y_r }    \\
            \vdots    &   \ddots    &   \vdots    \\
            \frac{ \partial G_r }{ \partial y_1 }    &   \ldots    &   \frac{ \partial G_r }{ \partial y_r }
        \end{pmatrix}\neq 0.
    \end{equation}
    Nous identifions \( \eR^n\) à \( \eR^s\times \eR^r\) dans lequel \( \eR^r\) est la partie générée par les \( (y_i)_{i=1,\ldots, r}\). Nous nommons \( (x_j)_{j=1,\ldots, s}\) les coordonnées sur \( \eR^s\). Autrement dit, les coordonnées sur \( \eR^n\) sont \( x_1,\ldots, x_s,y_1,\ldots, y_r\). Dans ces coordonnées, nous nommons \( a=(\alpha,\beta)\) avec \( \alpha\in \eR^s\) et \( \beta\in \eR^r\).

    Si nous notons \( G=(G_1,\ldots, G_r)\), le théorème de la fonction implicite (théorème \ref{ThoAcaWho})  nous dit qu'il existe un voisinage \( U'\) de \( \alpha\in \eR^n\), un voisinage \( V'\) de \( \beta\in \eR^r\) et une fonction \( \varphi\colon U'\to V'\) de classe \( C^1\) telle que si \( (x,y)\in U'\times V'\), alors
    \begin{equation}
        g(x,y)=0
    \end{equation}
    si et seulement si \( y=\varphi(x)\). Nous posons maintenant
    \begin{subequations}
        \begin{align}
            \psi(x)&=(x,\varphi(x))\\
            h(x)&=f\big( \psi(x) \big).
        \end{align}
    \end{subequations}
    Nous avons \( \psi(\alpha)=a\) et \( \psi(x)\in M\) pour tout \( x\in U'\). La fonction \( h\) a donc un extrema local en \( \alpha\) et donc les dérivées partielles de \( h\) y sont nulles. Cela signifie que
    \begin{equation}
        0=\frac{ \partial h }{ \partial x_i }(\alpha)=\sum_{j=1}^n\frac{ \partial f }{ \partial x_j }\frac{ \partial x_j }{ \partial x_i }+\sum_{k=1}^r\frac{ \partial f }{ \partial y_k }\frac{ \partial \varphi_k }{ \partial x_i },
    \end{equation}
    c'est à dire
    \begin{equation}
        \frac{ \partial f }{ \partial x_i }(\alpha)+\sum_{k=1}^r\frac{ \partial f }{ \partial y_k }(a)\frac{ \partial \varphi_k }{ \partial x_i }(\alpha)=0
    \end{equation}
    pour tout \( i=1,\ldots, s\). D'autre part pour tout $k$, la fonction \( l_k(x)=G_k\big( x,\varphi(x) \big)\) est constante et vaut zéro; ses dérivées partielles sont donc nulles :
    \begin{equation}
        \frac{ \partial l }{ \partial x_i }(\alpha)=\frac{ \partial G_k }{ \partial x_i }(\alpha)+\sum_{k=1}^r\frac{ \partial G_k }{ \partial y_k }(a)\frac{ \partial \varphi_k }{ \partial x_i }(\alpha)=0
    \end{equation}
    pour tout \( i=1,\ldots, s\) et \( k=1,\ldots, r\).
    
    Les \( s\) premières colonnes de la matrice
    \begin{equation}
        \begin{pmatrix}
            \frac{ \partial f }{ \partial x_1 }   &   \cdots    &   \frac{ \partial f }{ \partial x_s }    &   \frac{ \partial f }{ \partial y_1 }    &   \cdots    &   \frac{ \partial f }{ \partial y_r }\\  
            \frac{ \partial G_1 }{ \partial x_1 }    &   \cdots    &   \frac{ \partial G_1 }{ \partial x_s }    &   \frac{ \partial G_1 }{ \partial y_1 }    &   \cdots    &   \frac{ \partial G_1 }{ \partial y_r }\\
            \vdots    &   \vdots    &   \vdots    &   \vdots    &   \vdots    &   \vdots\\
            \frac{ \partial G_r }{ \partial x_1 }    &   \cdots    &   \frac{ \partial G_r }{ \partial x_s }    &   \frac{ \partial G_r }{ \partial y_1 }    &  \cdots   & \frac{ \partial G_r }{ \partial y_r }  
        \end{pmatrix}
    \end{equation}
    s'expriment en terme des \( r\) dernières. La matrice est donc au maximum de rang \( r\). Notons que la première ligne est \( \nabla f\) et les \( r\) suivantes sont les \( \nabla G_i\). Vu que ces lignes sont des vecteurs liés, il existe \( \mu_0,\ldots, \mu_r\) tels que
    \begin{equation}
        \mu_0\nabla f+\sum_{i=1}^r\mu_i\nabla G_i=0.
    \end{equation}
    Par hypothèse les \( \nabla G_i\) sont linéairement indépendants, ce qui nous dit que \( \mu_0\neq 0\). Donc nous avons ce qu'il nous faut :
    \begin{equation}
        \nabla f(a)=\sum_i\frac{ \mu_i }{ \mu_O } \nabla G_i(a).
    \end{equation}
\end{proof}

La proposition suivante est la même que \ref{ThoRGJosS}.
\begin{proposition} \label{PropfPPUxh}
    Soit \( U\), un ouvert de \( \eR^n\) et des fonctions de classe \( C^1\) \( f,g_1,\ldots, g_r\colon U\to \eR\). Nous considérons
    \begin{equation}
        \Gamma=\{ x\in U\tq g_1(x)=\ldots=g_r(x)=0 \}.
    \end{equation}
    Soit \( a\) un extrémum de \( f|_{\Gamma}\). Supposons que les formes \( dg_1,\ldots, dg_r\) soient linéairement indépendantes en \( a\). Alors il existe \( \lambda_1,\ldots, \lambda_r\) dans \( \eR\) tel que
    \begin{equation}
        df_a=\sum_{i=1}^r\lambda_i(dg_i)_a.
    \end{equation}
\end{proposition}



\href{http://www.sagenb.org/home/pub/353/}{En pratique}, les candidats extrema locaux sont tous les points où les gradients ne sont pas linéairement indépendants, plus tous les points donnés par l'équation $\nabla L=0$. Parmi ces candidats, il faut trouver lesquels sont maxima ou minima, locaux ou globaux.

L'existence d'extrema locaux se prouve généralement en invoquant de la compacité, et en invoquant le lemme suivant qui permet de réduire le problème à un compact.

\begin{lemma}		\label{LemmeMinSCimpliqueS}
	Soit $S$, un ensemble dans $\eR^n$ et $C$, un ouvert de $\eR^n$. Si $a\in\Int S$ est un minimum local relatif à $S\cap C$, alors il est un minimum local par rapport à $S$.
\end{lemma}

\begin{proof}
	Nous avons que $\forall x\in B(a,\epsilon_1)\cap S\cap C$, $f(x)\geq f(x)$. Mais étant donné que $C$ est ouvert, et que $a\in C$, il existe un $\epsilon_2$ tel que $B(a,\epsilon_2)\subset C$. En prenant $\epsilon=\min\{ \epsilon_1,\epsilon_2 \}$, nous trouvons que $f(x)\geq f(a)$ pour tout $x\in B(a,\epsilon)\cap(S\cap C)=B(a,\epsilon)\cap S$.
\end{proof}



%+++++++++++++++++++++++++++++++++++++++++++++++++++++++++++++++++++++++++++++++++++++++++++++++++++++++++++++++++++++++++++
                    \section{Développements de Taylor et Maclaurin}
%+++++++++++++++++++++++++++++++++++++++++++++++++++++++++++++++++++++++++++++++++++++++++++++++++++++++++++++++++++++++++++

%---------------------------------------------------------------------------------------------------------------------------
\subsection{Fonctions «petit o» }
%---------------------------------------------------------------------------------------------------------------------------

Nous voulons formaliser l'idée d'une fonction qui tend vers zéro \og plus vite\fg{} qu'une autre. Nous disons que $f\in o\big(\varphi(x)\big)$ si
\begin{equation}
    \lim_{x\to 0} \frac{ f(x) }{ \varphi(x) }=0.
\end{equation}
En particulier, nous disons que $f\in o(x)$ lorsque $\lim_{x\to 0} f(x)/x=0$.

%---------------------------------------------------------------------------------------------------------------------------
\subsection{Formule et reste}
%---------------------------------------------------------------------------------------------------------------------------

\begin{proposition}     \label{PropDevTaylorPol}
    Soient $f\colon I\subset\eR\to \eR$ et $a\in\Int(I)$. Soit un entier $k\geq 1$. Si $f$ est $k$ fois dérivable en $a$, alors il existe un et un seul polynôme $P$ de degré $\leq k$ tel que
    \begin{equation}
        f(x)-P(x-a)\in o\big( | x-a |^k \big)
    \end{equation}
    lorsque $x\to a$, $x\neq a$. Ce polynôme  est donné par
    \begin{equation}
        P(h)=f(a)+f'(a)h+\frac{ f''(a) }{ 2! }h^2+\ldots+\frac{ f^{(k)}(a) }{ k! }h^k.
    \end{equation}
    Notons encore deux façons alternatives d'écrire le résultat. Si \( f\in C^k\) il existe une fonction \( \alpha\) telle que \( \lim_{t\to 0} \alpha(t)=0\) et
    \begin{equation}
        f(x)=\sum_{n=0}^k\frac{ f^{(n)}(a) }{ n! }(x-a)^n+(x-a)^n\alpha(x-a).
    \end{equation}
    Si \( f\in C^{k+1}\) alors
    \begin{equation}        \label{EquQtpoN}
        f(x)=\sum_{n=0}^k\frac{ f^{(n)}(a) }{ n! }(x-a)^n+(x-a)^{n+1}\xi(x-a)
    \end{equation}
    où \( \xi\) est une fonction telle que \( \xi(t)\) tend vers une constante lorsque \( t\to 0\).
\end{proposition}

La proposition suivant donne une intéressante façon de trouver le reste d'un développement de Taylor.
\begin{proposition}     \label{PropResteTaylorc}
Soient $I$, un intervalle dans $\eR$ et $f\colon I\to \eR$ une fonction de classe $C^k$ sur $I$ telle que $f^{(k+1)}$ existe sur $I$. Soient $a\in\Int(I)$ et $x\in I$. Alors il existe $c$ strictement compris entre $x$ et $a$ tel que 
\begin{equation}
    R_{f,a,k}(x)=\frac{ f^{(k+1)}(c) }{ (k+1)! }(x-a)^{k+1}.
\end{equation}
\end{proposition}

%---------------------------------------------------------------------------------------------------------------------------
                    \subsection{Exemple : un calcul heuristique de limite}
%---------------------------------------------------------------------------------------------------------------------------
\label{SubSecCalcLimHeuris}

Au cours de la résolution de l'exercice \ref{exoEqsDiff0002}\ref{ItemfEqsDiff00002}, nous devons calculer la limite suivante :
\begin{equation}
    \lim_{x\to 0} \frac{  e^{-2\cos(x)+2}\sin(x) }{ \sqrt{ e^{2\cos(x)+2}}-1 }.
\end{equation}
La stratégie que nous allons suivre pour calculer cette limite est de développer certaines parties de l'expression en série de Taylor, afin de simplifier l'expression. La première chose à faire est de remplacer $ e^{y(x)}$ par $1+y(x)$ lorsque $y(x)\to 0$. La limite devient
\begin{equation}
    \lim_{x\to 0} \frac{ \big( -2\cos(x)+3 \big)\sin(x) }{ \sqrt{-2\cos(x)+2} }.
\end{equation}
Nous allons maintenant remplacer $\cos(x)$ par $1$ au numérateur et par $1-x^2/2$ au dénominateur. Pourquoi ? Parce que le cosinus du dénominateur est dans une racine, donc nous nous attendons à ce que le terme de degré deux du cosinus donne un degré un en dehors de la racine, alors que du degré un est exactement ce que nous avons au numérateur : le développement du sinus commence par $x$.

Nous calculons donc
\begin{equation}
    \begin{aligned}[]
        \lim_{x\to 0} \frac{ \sin(x) }{ \sqrt{-2\left( 1-\frac{ x^2 }{ 2 } \right)+2} }=\lim_{x\to 0} \frac{ \sin(x) }{ x }=1.
    \end{aligned}
\end{equation}
Tout ceci n'est évidement pas très rigoureux, mais en principe vous avez tous les éléments en main pour justifier les étapes.

%+++++++++++++++++++++++++++++++++++++++++++++++++++++++++++++++++++++++++++++++++++++++++++++++++++++++++++++++++++++++++++
\section{Théorie de la mesure}
%+++++++++++++++++++++++++++++++++++++++++++++++++++++++++++++++++++++++++++++++++++++++++++++++++++++++++++++++++++++++++++

Pour la théorie de la mesure, voir entre autres \cite{FubiniBMauray,ProbaDanielLi}.

%---------------------------------------------------------------------------------------------------------------------------
\subsection{Espaces mesurables et mesurés}
%---------------------------------------------------------------------------------------------------------------------------

\begin{definition}
    Si \( \Omega\) est un ensemble, un ensemble \( \tribA\) de sous-ensembles de \( \Omega\) est une \defe{tribu}{tribu} si
    \begin{enumerate}
        \item
            \( \Omega\in\tribA\);
        \item
            \( \complement A\in A\) pour tout \( A\in\tribA\);
        \item
            si \( (A_i)_{i\in I}\) est un ensemble au plus dénombrable d'éléments de \( \tribA\), alors \( \sup_{n\geq 1}A_n=\bigcup_{i\in I}A_i\in\tribA\).
    \end{enumerate}
    Le couple \( (\Omega,\tribA)\) est alors un \defe{espace mesuré}{espace!mesuré}.
\end{definition}

\begin{lemma}
    Une tribu est stable par intersections au plus dénombrables.
\end{lemma}

\begin{proof}
    Soit \( (A_i)_{i\in I}\) une famille au plus dénombrable d'éléments de la tribu \( \tribA\). Nous devons prouver que \( \bigcap_{i\in I}A_i\) est également un élément de \( \tribA\). Pour cela nous passons au complémentaire :
    \begin{equation}
        \complement\left( \bigcap_{i\in I}A_i \right)=\bigcup_{i\in I}\complement A_i.
    \end{equation}
    La définition d'une tribu implique que le membre de droite est un élément de la tribu. Par stabilité d'une tribu par complémentaire, l'ensemble \( \bigcap_{i\in I}A_i\) est également un élément de la tribu.
\end{proof}

La tribu que nous utiliserons toujours dans \( \eR^d\) est la tribu des \defe{boréliens}{boréliens}, notée \( \Borelien(\eR^d)\), qui est la tribu engendrée par les ouverts de \( \eR^d\). Une fonction \( f\colon (\Omega,\tribA)\to (\eR^d,\Borelien(\eR^d))\) est \defe{borélienne}{borélienne} si pour tout \( \mO\in\Borelien\), \( f^{-1}(\mO)\in\tribA\).

\begin{definition}
    Une \defe{\wikipedia{en}{Measure_space}{mesure}}{mesure} sur l'espace mesurable \( (\Omega,\tribA)\) est une application \( \mu\colon \tribA\to \eR\cup\{ \infty \}\) telle que
    \begin{enumerate}
        \item
            \( \mu(A)\geq 0\) pour tout \( A\in\tribA\);
        \item
            \( \mu(\emptyset)=0\);
        \item
            \( \mu\left( \bigcup_{i=0}^{\infty}A_i\right)=\sum_{i=0}^{\infty}\mu(A_i)\) si les \( A_i\) sont des éléments de \( \tribA\) deux à deux disjoints.
    \end{enumerate}
    Une mesure est \defe{\( \sigma\)-finie}{mesure!$\sigma$-finie} si il existe une suite croissante (pour l'inclusion) d'éléments \( (E_n)_{n\in\eN}\) de la tribu, tous de mesure finie et tels que \( \Omega=\bigcup_{n\in \eN}E_n\). Si la mesure est $\sigma$-finie, nous disons que l'espace \( (\Omega,\tribA,\mu)\) est un espace mesuré $\sigma$-fini.
\end{definition}

\begin{example}
    La mesure de Lebesgue sur \( \eR^n\) est \( \sigma\)-finie parce que les boules de rayon \( n\) forment un ensemble dénombrable d'ensembles de mesures finies dont l'union est évidemment tout \( \eR^n\).

    L'intervalle \( I=\mathopen[ 0 , 1 \mathclose]\) muni de la tribu de toutes ses parties et de la mesure de comptage n'est pas un espace mesuré \( \sigma\)-fini.
\end{example}

\begin{definition}
    Une application entre espace mesurés
    \begin{equation}
        f\colon (\Omega,\tribA)\to (\Omega',\tribA')
    \end{equation}
    est \defe{mesurable}{mesurable!application} si pour tout \( B\in\tribA'\), l'ensemble \( f^{-1}(B)\) est dans \( \tribA\).
\end{definition}

Si \( \mu\) est une mesure sur \( \eR^d\), une fonction \( f\colon \eR^d\to \eR\) est une \defe{densité}{densité d'une mesure} si pour tout \( A\subset\eR^d\) nous avons
\begin{equation}
    \mu(A)=\int_Af(x)dx
\end{equation}
où \( dx\) est la mesure de Lebesgue.

%---------------------------------------------------------------------------------------------------------------------------
\subsection{Mesure produit}
%---------------------------------------------------------------------------------------------------------------------------

\begin{definition}      \label{DefTribProfGfYTuR}
    Si \( \tribA\) et \( \tribB\) sont deux tribus sur deux ensembles \( \Omega_1\) et \( \Omega_2\), nous définissons la \defe{tribu produit}{tribu!produit} \( \tribA\otimes\tribB\) comme étant la tribu engendrée par 
    \begin{equation}
        \{ A\times B\tq A\in\tribA,B\in\tribB \}.
    \end{equation}
\end{definition}

\begin{theorem}\index{mesure!produit}
    Soient \( \mu_i\) des mesures $\sigma$-finies sur \( (\Omega_i,\tribA_i)\) (\( i=1,2\)). Il existe une et une seule mesure, notée \( \mu_1\otimes \mu_2\), sur \( (\Omega_1\times\Omega_2,\tribA_1\otimes\tribA_2)\) telle que
    \begin{equation}
        (\mu_1\otimes\mu_2)(A_1\times A_2)=\mu_1(A_1)\mu_2(A_2)
    \end{equation}
    pour tout \( A_1\in \tribA_1\) et \( A_2\in\tribA_2\).
\end{theorem}
Une preuve peut être trouvée dans \cite{FubiniBMauray}.

%---------------------------------------------------------------------------------------------------------------------------
\subsection{Intégrale par rapport à une mesure}
%---------------------------------------------------------------------------------------------------------------------------

Une fonction \( f\colon (\Omega,\tribA)\to (\Omega',\tribA')\) est \defe{mesurable}{mesurable!fonction} si 
\begin{equation}
    f^{-1}(E)\in\tribA
\end{equation}
pour tout \( E\in\tribA'\).


Une mesure \( \mu\) sur un espace mesurable \( (\Omega,\tribA)\) permet de définir une fonctionnelle linéaire sur l'ensemble des fonctions mesurables \( \Omega\to \eR\). Cette fonctionnelle linéaire est l'intégrale que nous allons définir à présent.

D'abord nous considérons les fonction \defe{simples}{simple!fonction}\index{fonction!simple}, c'est à dire les fonctions de la forme
\begin{equation}
    f=\sum_{i=1}^Na_i\caract_{E_i}
\end{equation}
où \( a_i\in\eR\) tandis que les \( E_i\) sont des ensembles \( \mu\)-mesurables. Si \( Y\in \tribA\) nous définissons
\begin{equation}
    \int_Yfd\mu=\sum_ia_i\mu(Y\cap E_i).
\end{equation}
Pour une fonction \( \mu\)-mesurable générale \( f\colon \Omega\to \mathopen[ 0 , \infty \mathclose]\) nous définissons l'intégrale de \( f\) sur \( Y\) par
\begin{equation}        \label{EqDefintYfdmu}
    \int_Yfd\mu=\sup\Big\{ \int_Yhd\mu\,\text{où \( h\) est une fonction simple et mesurable telle que \( 0\leq h\leq f\)} \Big\}.
\end{equation}
Maintenant nous définissons
\begin{equation}
    \mu(f)=\int_{\Omega}f
\end{equation}
si \( f\) est une fonction mesurable sur \( \Omega\).

\begin{remark}
    Dans \( \eR^d\), quasiment toutes les fonctions et ensembles sont mesurables. En effet la construction d'ensembles non mesurables demande obligatoirement l'utilisation de l'axiome du choix; de tels ensembles doivent être construits «exprès pour». Il y a très peu de chances pour que vous tombiez sur un ensemble non mesurable de \( \eR^d\) sans que vous ne vous en rendiez compte.

    Par exemple la variable aléatoire 
    \begin{equation}
        X(\omega)=\begin{cases}
            \frac{1}{ \omega }    &   \text{si $ \omega\neq 0$}\\
            \infty    &    \text{$\omega=0$}.
        \end{cases}
    \end{equation}
    est mesurable, mais non intégrable.
\end{remark}

\begin{lemma}   \label{Lemfobnwt}
    Soit \( f\) une fonction mesurable positive ou nulle telle que
    \begin{equation}
        \int_{\Omega}fd\mu=0.
    \end{equation}
    Alors \( f=0\) \( \mu\)-presque partout.
\end{lemma}

\begin{proof}
    L'ensemble des points \( x\in\Omega\) tels que \( f(x)\neq 0\) peut s'écrire comme une union dénombrable disjointe :
    \begin{equation}
        \{ x\in\Omega\tq f(x)\neq 0 \}=\bigcup_{i=0}^{\infty}E_i
    \end{equation}
    avec
    \begin{subequations}
        \begin{align}
            E_0&=\{ x\in\Omega\tq f(x)>1 \}\\
            E_i&=\{ x\in\Omega\tq \frac{1}{ i+1 }\leq f(x)<\frac{1}{ i } \}.
        \end{align}
    \end{subequations}
    Si un des ensembles \( E_i\) est de mesure non nulle, alors nous pouvons considérer la fonction simple \( h(x)=\frac{1}{ i+1 }\mtu_{E_i}\) dont l'intégrale sur \( \Omega\) est strictement positive. Par conséquent le supremum de la définition \eqref{EqDefintYfdmu} est strictement positif.

    Nous savons donc que \( \mu(E_i)=0\) pour tout \( i\). Étant donné que la mesure d'une union disjointe dénombrable est égale à la somme des mesures, nous avons
    \begin{equation}
        \mu\{ x\in\Omega\tq f(x)\neq 0 \}=0,
    \end{equation}
    ce qui signifie que \( f\) est nulle \( \mu\)-presque partout.
\end{proof}

\begin{corollary}   \label{CorjLYiSm}
    Soit \( f\) une fonction mesurable sur l'espace mesuré \( (\Omega,\tribA,\mu)\) telle que
    \begin{equation}
        \int_{\Omega}f\mtu_{f>0}d\mu=0.
    \end{equation}
    Alors \( f\leq 0\) presque partout.
\end{corollary}

\begin{proof}
    Nous avons l'égalité d'ensembles
    \begin{equation}
        \{ f\mtu_{f>0}\neq 0 \}=\{ \mtu_{f>0}\neq 0 \}.
    \end{equation}
    Mais lemme \ref{Lemfobnwt} implique que \( f\mtu_{f>0}\) est nulle presque partout, c'est à dire que la mesure de l'ensemble du membre de gauche est nulle par conséquent
    \begin{equation}
        \mu\{ \mtu_{f>0}\neq 0 \}=0.
    \end{equation}
    Cela signifie que la fonction \( f\) est presque partout négative ou nulle.
\end{proof}

\begin{lemma}   \label{LemPfHgal}
    Soit \( f\) une fonction telle que \( | f(x)|\leq g(x) \) pour tout \( x\in\Omega\). Si \( g\) est intégrable, alors \( f\) est intégrable.
\end{lemma}

\begin{proof}
    Nous décomposons \( f\) en parties positives et négatives :
    \begin{subequations}
        \begin{align}
            A_+&=\{ x\in\Omega\tq f(x)>0 \}\\
            A_-&=\{ x\in\Omega\tq f(x)<0 \}.
        \end{align}
    \end{subequations}
    Nous posons \( f_+(x)=f(x)\mtu_{A_+}\) et \( f_-(x)=f(x)\mtu_{A_-}\). Nous avons \( f=f_+-f_-\) et
    \begin{equation}
        \int_{\Omega}f=\int_{A_+}f+\int_{A_-}f
    \end{equation}
    parce que \( \Omega=A_+\cup A_-\cup\{ x\in\Omega\tq f(x)=0 \}\). Si \( \varphi\) est une fonction simple qui majore \( f_+\) nous avons
    \begin{equation}
        \varphi(x)=\sum_{k}a_k\mtu_{E_k}(x)\leq f(x)\mtu_{A_+}(x)\leq g(x).
    \end{equation}
    Par conséquent le supremum qui définit \( \int f_+\) est inférieur au supremum qui définit \( \int g\). La fonction \( f_+\) est donc intégrable. La même chose est valable pour la fonction \( f_-\).
\end{proof}

\begin{proposition} \label{PropWBavIf}
    Soit \( f\) une fonction positive \( \tribA\)-mesurable et bornée. Alors \( f\) est limite ponctuelle croissante de fonction simples.
\end{proposition}

\begin{proof}
    Soit \( \sigma_n=\{ a_0=0,\ldots, a_{r_n}=n \}\) une subdivision de \( \mathopen[ 0 , n \mathclose]\) en intervalles de taille plus petites que \( 1/n\) choisis de sorte que \( \sigma_{n-1}\subset\sigma_{n}\), et
    \begin{equation}
        f_n(x)=\begin{cases}
            0    &   \text{si \( f(x)>n\)}\\
            a_i    &    \text{sinon}
        \end{cases}
    \end{equation}
    où \( a_i\) est le plus grand élément de \( \sigma_n\) inférieur à \( f(x)\). La fonction \( f_n\) est simple et nous avons pour tout \( x\)
    \begin{equation}
        \lim_{n\to \infty} f_n(x)=f(x)
    \end{equation}
\end{proof}

%---------------------------------------------------------------------------------------------------------------------------
\subsection{Mesure dominée}
%---------------------------------------------------------------------------------------------------------------------------

Soient \( \mu\) et \( \nu\) deux mesures sur le même espace \( \Omega\) et la même tribu \( \tribA\). Nous disons que la mesure \( \mu\) est \defe{dominée}{dominée!mesure}\cite{PersoFeng} par \( \nu\) si pour tout ensemble mesurable \( A\), \( \nu(A)=0\) implique \( \mu(A)=0\).

La mesure \( \mu\) est \defe{portée}{portée!mesure} par l'ensemble \( E\in\tribA\) si pour tout \( A\in\tribA\), 
\begin{equation}
    \mu(A)=\mu(A\cap E).
\end{equation}

Nous écrivons que \( \mu\perp\nu\)\nomenclature[Y]{\( \mu\perp\nu\)}{mesures perpendiculaires} si il existe un ensemble \( E\in\tribA\) tel que \( \mu\) soit porté par \( E\) et \( \nu\) soit porté par \( \complement E\).

\begin{theorem}[Radon-Nikodym]\index{Radon-Nikodym}
    Soient \( \mu\) et \( m\) deux mesures \( \sigma\)-finies sur un espace métrisable \( (\Omega,\tribA)\).
    \begin{enumerate}
        \item
            Il existe un unique couple de mesures \( \mu_1\) et \( \mu_2\) telles que
            \begin{enumerate}
                \item
                    \( \mu=\mu_1+\mu_2\)
                \item
                    \( \mu_1\) est dominé par \( m\)
                \item
                    \( \mu_2\perp m\).
            \end{enumerate}
            Dans ce cas, les mesures \( \mu_1\) et \( \mu_2\) sont positives et \( \sigma\)-finies.
        \item
            À égalité \(  m\)-presque partout près, il existe une unique fonction mesurable positive \( f\) telle que pour tout mesurable \( A\),
            \begin{equation}
                \mu_1(A)=\int_Ad\mu_1=\int_{\Omega}\mtu_Afd m.
            \end{equation}
        \item
            À égalité \( m\)-presque partout près, il existe une unique fonction positive mesurable \( h\) telle que \( \mu_1=hm\).
    \end{enumerate}
\end{theorem}
Une démonstration est donné dans \cite{NikoLi}.

\begin{corollary}   \label{CorZDkhwS}
    Si \( \mu\) es une mesure \( \sigma\)-finie dominée par la mesure \( \sigma\)-finie \( m\), alors \( \mu\) possède une unique fonction de densité.
\end{corollary}

\begin{corollary}       \label{CorDomDens}
    Soient \( \mu\) et \( m\), deux mesures positives \( \sigma\)-finies sur \( (\Omega,\tribA)\). Alors \( m\) domine \( \mu\) si et seulement si \( \mu\) possède une densité par rapport à \( m\).
\end{corollary}
 
\begin{proof}
    Si \( \mu\) est dominée par \( m\), alors la décomposition \( \mu=\mu+0\) satisfait le théorème de Radon-Nikodym. Par conséquent il existe une fonction \( f\) telle que
    \begin{equation}
        \mu(A)=\int_Afdm.
    \end{equation}
    Cette fonction est alors une densité pour \( \mu\) par rapport à \( m\).

    Pour la réciproque, nous supposons que \( \mu\) a une densité \( f\) par rapport à \( m\), et que \( A\) est une ensemble de \( m\)-mesure nulle :
    \begin{equation}
        m(A)=\int_{\Omega}\mtu_Adm=0.
    \end{equation}
    Cela signifie que la fonction \( \mtu_A\) est \( m\)-presque partout nulle. La fonction produit \( \mtu_Af\) est également nulle \( m\)-presque partout, et par conséquent
    \begin{equation}
        \mu(A)=\int_{\Omega}\mtu_Afdm=0.
    \end{equation}
\end{proof}

\begin{probleme}
    Est-ce que la démonstration de cela ne demande pas la convergence monotone d'une façon ou d'une autre ?
\end{probleme}

%---------------------------------------------------------------------------------------------------------------------------
\subsection{Changement de variables dans une intégrale}
%---------------------------------------------------------------------------------------------------------------------------

\begin{theorem} \label{ThomFeRCi}
    Soit \( \mO\) un ouvert de \( \eR^n\) et \( \mO'\) un ouvert de \( \eR^m\). Soit \( \varphi\colon \mO\to \mO'\) un difféomorphisme \( C^1\). Si \( f\colon \mO\to \eR\) est une fonction mesurable, positive et intégrable, alors
    \begin{equation}
        \int_{\mO}f(u)du=\int_{\mO'}f\big( \varphi^{-1}(v) \big)| J_{\varphi^{-1}}(v) |dv.
    \end{equation}
\end{theorem}

%+++++++++++++++++++++++++++++++++++++++++++++++++++++++++++++++++++++++++++++++++++++++++++++++++++++++++++++++++++++++++++
\section{Mesure de Haar}
%+++++++++++++++++++++++++++++++++++++++++++++++++++++++++++++++++++++++++++++++++++++++++++++++++++++++++++++++++++++++++++

\begin{definition}
    Soit \( G\) un groupe topologique. Une \defe{mesure de Haar}{mesure!de Haar} sur \( G\) est une mesure \( \mu\) telle que 
    \begin{enumerate}
        \item
            \( \mu(gA)=\mu(A)\) pour tout mesurable \( A\) et tout \( g\in G\),
        \item
            \( \mu(K)<\infty\) pour tout compact \( K\subset G\).
    \end{enumerate}
    Si de plus le groupe \( G\) lui-même est compact nous demandons que la mesure soit normalisée : \( \mu(G)=1\).
\end{definition}
L'existence d'une mesure de Haar sur un groupe compact sera une conséquence du théorème de point fixe de Markov-Katutani \ref{ThoeJCdMP}.


%+++++++++++++++++++++++++++++++++++++++++++++++++++++++++++++++++++++++++++++++++++++++++++++++++++++++++++++++++++++++++++
\section{Variétés et extrema liés}
%+++++++++++++++++++++++++++++++++++++++++++++++++++++++++++++++++++++++++++++++++++++++++++++++++++++++++++++++++++++++++++

\subsection{Introduction}
Soit $f : S^2 \to \eR$ une fonction définie sur la sphère usuelle
$S^2 \subset \eR^3$. Une question naturelle est d'estimer la
régularité de $f$ ; est-elle continue, dérivable, différentiable ? Il
n'existe pas de dérivée directionnelle étant donné que le quotient
différentiel
\begin{equation*}
  \frac{f(x + \epsilon u_1 ,y + \epsilon u_2) - f(x,y)}{\epsilon}
\end{equation*}
n'a pas de sens pour un point $(x + \epsilon u_1 ,y + \epsilon u_2)$
qui n'est pas --sauf valeurs particulières-- dans la surface. Pour la
même raison il n'est pas possible de parler de différentiabilité de
cette manière. Comment faire, sans devoir étendre le domaine de
définition de $f$ à un voisinage de la sphère ? Une solution possible
est de parler de la notion de variété.

Une variété est un objet qui ressemble, vu de près, à $\eR^m$ pour un
certain $m$. En d'autres termes, on imagine une variété comme un
recollement de morceaux de $\eR^m$ vivant dans un espace plus grand
$\eR^n$. Ces morceaux sont appelés des ouverts de carte, et
l'application qui exprime la ressemblance à $\eR^m$ est l'application
de carte.

%---------------------------------------------------------------------------------------------------------------------------
\subsection{Définition et propriétés}
%---------------------------------------------------------------------------------------------------------------------------

\begin{definition}
  Soit $\emptyset \neq M \subset \eR^n$, $1 \leq m < n$ et $k \geq
  1$. $M$ est une \Defn{variété de classe $C^k$ de dimension $m$} si
  pour tout $a \in \var M$, il existe un voisinage ouvert $U$ de $a$
  dans $\eR^n$, et un ouvert $V$ de $\eR^m$ tel que $U \cap \var M$
  soit le graphe d'une fonction $f : V \subset \eR^m \to \eR^{n-m}$
  de classe $C^1$, c'est-à-dire qu'il existe un réagencement des
  coordonnées $(x_{i_1}, \ldots, x_{i_m}, x_{i_{m+1}}, \ldots,
  x_{i_n})$ avec
  \begin{equation*}
    M \cap U = \left\{ (x_1, \ldots, x_n) \in \eR^n \tq
%      \begin{array}{l} % deux conditions
      (x_{i_1}, \ldots, x_{i_m}) \in V \quad \left\{\begin{array}{c!{=}l} % 1: equations
        x_{i_{m+1}} & f_1(x_{i_1}, \ldots, x_{i_m})\\
        \vdots & \vdots \\
        x_{i_n} & f_{n-m}(x_{i_1}, \ldots, x_{i_m})
      \end{array}\right.
%    \end{array}
    \right\}
  \end{equation*}
  où $V$ est un voisinage ouvert de $(a_{i_1}, \ldots, a_{i_m}) \in \eR^m$.
\end{definition}

La littérature regorge de théorèmes qui proposent des conditions équivalentes à la définition d'une variété. Celle que nous allons le plus utiliser est la suivante% , de la page 268.
\begin{proposition}
	Soit $M\subset\eR^n$ et $1\leq m\leq n-1$. L'ensemble $M$ est une variété si et seulement si $\forall a\in M$, il existe un voisinage ouvert $\mU$ de $a$ dans $\eR^n$ et une application $F\colon W\subset\eR^m\to \eR^n$ où $W$ est un ouvert tels que
	\begin{enumerate}
		\item
			$F$ est un homéomorphisme de $W$ vers $M\cap\mU$,
		\item
			$F\in C^1(W,\eR^n)$,
		\item
			Le rang de $dF(w)\in L(\eR^m,\eR^n)$ est de rang maximum (c'est à dire $m$) en tout point $w\in W$.
	\end{enumerate}
\end{proposition}
Pour rappel, si $T\colon \eR^m\to \eR^n$ est une application linéaire, son \defe{rang}{rang} est la dimension de son image. On peut prouver que si $A$ est la matrice d'une application linéaire, alors le rang de cette application linéaire est égal à la taille de la plus grande matrice carré de déterminant non nul contenue dans $A$.

La condition de rang maximum sert à éviter le genre de cas de la figure \ref{LabelFigExempleNonRang} qui représente l'image de l'ouvert $\mathopen] -1 , 1 \mathclose[$ par l'application $F(t)=(t^2,t^3)$.
\newcommand{\CaptionFigExempleNonRang}{Quelque chose qui n'est pas de rang maximum et qui n'est pas une variété.}
\input{Fig_ExempleNonRang.pstricks}
%\ref{LabelFigExempleNonRang} 
%\newcommand{\CaptionFigExempleNonRang}{Quelque chose qui n'est pas de rang maximum et qui n'est pas une variété.}
%\input{Fig_ExempleNonRang.pstricks}
La différentielle a pour matrice
\begin{equation}
	dF(t)=(2t,3t^2).
\end{equation}
Le rang maximum est $1$, mais en $t=0$, la matrice vaut $(0,0)$ et son rang est zéro. Pour toute autre valeur de $t$, c'est bon.

Une autre caractérisation des variétés est donnée par la proposition suivante %(proposition 3, page 274).
\begin{proposition}		\label{PropCarVarZerFonc}
	Soit $M\in \eR^n$ et $1\leq m\leq n-1$. L'ensemble $M$ est une variété si et seulement si $\forall a\in M$, il existe un voisinage ouvert $\mU$ de $a$ dans $\eR^n$ tel et une application $G\in C^1(\mU,\eR^{n-m})$ tel que
	\begin{enumerate}

		\item
			le rang de $dG(a)\in L(\eR^n,\eR^{n-m})$ soit maximum (c'est à dire $n-m$) en tout $a\in M$,
		\item
			$M\cap\mU=\{ x\in\mU\tq G(x)=0 \}$.

	\end{enumerate}
\end{proposition}

%---------------------------------------------------------------------------------------------------------------------------
\subsection{Espace tangent}
%---------------------------------------------------------------------------------------------------------------------------

Soit $M$, une variété dans $\eR^n$, et considérons un chemin $\gamma\colon I\to \eR^n$ tel que $\gamma(t)\in M$ pour tout $t\in I$ et tel que $\gamma(0)=a$ et que $\gamma$ est dérivable en $0$. La \defe{tangente}{tangente à un chemin} au chemin $\gamma$ au point $a\in M$ est la droite
\begin{equation}
	s\mapsto a+s\gamma'(0).
\end{equation}
L'\defe{espace tangent}{Espace tangent} de $M$ au point $a$ est l'ensemble décrit par toutes les tangentes en $a$ pour tous les chemins $\gamma$ possibles.

\begin{proposition}			\label{PropDimEspTanVarConst}
	Une variété de dimension $m$ dans $\eR^n$ a un espace tangent de dimension $m$ en chacun de ses points.
\end{proposition}


