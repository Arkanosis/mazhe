% This is part of Agregation : modélisation
% Copyright (c) 2011
%   Laurent Claessens
% See the file fdl-1.3.txt for copying conditions.

Ce chapitre provient en grande partie de \cite{ProbaDanielLi}.

%---------------------------------------------------------------------------------------------------------------------------
\subsection{Espaces mesurables et mesurés}
%---------------------------------------------------------------------------------------------------------------------------

\begin{definition}
    Si \( \Omega\) est un ensemble, un ensemble \( \tribA\) de sous-ensembles de \( \Omega\) est une \defe{tribu}{tribu} si
    \begin{enumerate}
        \item
            \( \Omega\in\tribA\);
        \item
            \( A\cup B\in\tribA\) pour tout \( A,B\in\tribA\), ce qui signifie que toutes les unions finies d'éléments de \( \tribA\) sont dans \( \tribA\);
        \item
            \( \complement A\in A\) pour tout \( A\in\tribA\);
        \item
            si \( A_n\) est une suite dénombrable d'éléments de \( \tribA\), alors \( \sup_{n\geq 1}A_n\in\tribA\).
    \end{enumerate}
    Le couple \( (\Omega,\tribA)\) est alors un \defe{espace mesuré}{espace!mesuré}.
\end{definition}

La tribu que nous utiliserons toujours dans \( \eR^d\) est la tribu des \defe{boréliens}{boréliens}, notée \( \Borelien(\eR^d)\), qui est la tribu engendrée par les ouverts de \( \eR^d\). Une fonction \( f\colon (\Omega,\tribA)\to (\eR^d,\Borelien(\eR^d))\) est \defe{borélienne}{borélienne} si pour tout \( \mO\in\Borelien\), \( f^{-1}(\mO)\in\tribA\).

\begin{definition}
    Une \defe{\wikipedia{en}{Measure_space}{mesure}}{mesure} sur l'espace mesurable \( (\Omega,\tribA)\) est une application \( \mu\colon \tribA\to \eR\cup\{ \infty \}\) telle que
    \begin{enumerate}
        \item
            \( \mu(A)\geq 0\) pour tout \( A\in\tribA\);
        \item
            \( \mu(\emptyset)=0\);
        \item
            \( \mu\left( \bigcup_{i=0}^{\infty}A_i\right)=\sum_{i=0}^{\infty}\mu(A_i)\) si les \( A_i\) sont des éléments de \( \tribA\) deux à deux disjoints.
    \end{enumerate}
\end{definition}

Si \( \mu\) est une mesure sur \( \eR^d\), une fonction \( f\colon \eR^d\to \eR\) est une \defe{densité}{densité d'une mesure} si pour tout \( A\subset\eR^d\) nous avons
\begin{equation}
    \mu(A)=\int_Af(x)dx
\end{equation}
où \( dx\) est la mesure de Lebesgue.

%---------------------------------------------------------------------------------------------------------------------------
\subsection{Intégrale par rapport à une mesure}
%---------------------------------------------------------------------------------------------------------------------------

Une fonction \( f\colon (\Omega,\tribA)\to (\Omega',\tribA')\) est \defe{mesurable}{mesurable!fonction} si 
\begin{equation}
    f^{-1}(E)\in\tribA
\end{equation}
pour tout \( E\in\tribA'\).


Une mesure \( \mu\) sur un espace mesurable \( (\Omega,\tribA)\) permet de définir une fonctionnelle linéaire sur l'ensemble des fonctions mesurables \( \Omega\to \eR\). Cette fonctionnelle linéaire est l'intégrale que nous allons définir à présent.

D'abord nous considérons les fonction \defe{simples}{simple!fonction}\index{fonction!simple}, c'est à dire les fonctions de la forme
\begin{equation}
    f=\sum_{i=1}^Na_i\caract_{E_i}
\end{equation}
où \( a_i\in\eR\) tandis que les \( E_i\) sont des ensembles \( \mu\)-mesurables. Si \( Y\in \tribA\) nous définissons
\begin{equation}
    \int_Yfd\mu=\sum_ia_i\mu(Y\cap E_i).
\end{equation}
Pour une fonction \( \mu\)-mesurable générale \( f\colon \Omega\to \mathopen[ 0 , \infty \mathclose]\) nous définissons l'intégrale de \( f\) sur \( Y\) par
\begin{equation}        \label{EqDefintYfdmu}
    \int_Yfd\mu=\sup\Big\{ \int_Yhd\mu\,\text{où \( h\) est une fonction simple et mesurable telle que \( 0\leq h\leq f\)} \Big\}.
\end{equation}
Maintenant nous définissons
\begin{equation}
    \mu(f)=\int_{\Omega}f
\end{equation}
si \( f\) est une fonction mesurable sur \( \Omega\).

\begin{remark}
    Dans \( \eR^d\), quasiment toutes les fonctions et ensembles sont mesurables. En effet la construction d'ensembles non mesurables demande obligatoirement l'utilisation de l'axiome du choix; de tels ensembles doivent être construits «exprès pour». Il y a très peu de chances pour que vous tombiez sur un ensemble non mesurable de \( \eR^d\) sans que vous ne vous en rendiez compte.

    Par exemple la variable aléatoire 
    \begin{equation}
        X(\omega)=\begin{cases}
            \frac{1}{ \omega }    &   \text{si $ \omega\neq 0$}\\
            \infty    &    \text{$\omega=0$}.
        \end{cases}
    \end{equation}
    est mesurable, mais non intégrable.
\end{remark}

%---------------------------------------------------------------------------------------------------------------------------
\subsection{Dérivation sous le signe intégral}
%---------------------------------------------------------------------------------------------------------------------------

Une question classique est la dérivation par rapport à \( x\) d'une fonction du type
\begin{equation}
    F(x)=\int f(x,t)dt.
\end{equation}
Il existe divers théorèmes qui répondent à ces questions. Dans notre cadre nous utiliserons le suivant.
\begin{theorem}     \label{ThoDerSousIntegrale}
    Soit \( A\) un ouvert de \( \eR\) et \( \Omega\), un espace mesuré. Soit une fonction \( f\colon A\times \Omega\to \eR\) qui satisfait
    \begin{enumerate}
        \item
            La fonction \( f\) est mesurable en tant que fonction \( A\times\Omega\to \eR\). Pour chaque \( x\in A\), la fonction \( f(x,\cdot)\) est intégrable sur \( \Omega\).
        \item
            Pour presque tout \( \omega\in\Omega\), la fonction \( f(x,\omega)\) est une fonction absolument continue de \( x\).
        \item
            La fonction \( \frac{ \partial f }{ \partial x }\) est localement intégrable, c'est à dire que pour tout \( \mathopen[ a , b \mathclose]\subset A\),
            \begin{equation}
                \int_a^b\int_{\Omega}\left| \frac{ \partial f }{ \partial x }(x,\omega) \right| d\omega\,dx<\infty.
            \end{equation}
    \end{enumerate}
    Alors la fonction de \( x\)
    \begin{equation}
        F(x)=\int_{\Omega}f(x,\omega)d\omega
    \end{equation}
    est absolument continue et pour presque tout \( x\in A\), la dérivée est donné par
    \begin{equation}
        \frac{ d }{ dx }\int_{\Omega}f(x,\omega)d\omega=\int_{\Omega}\frac{ \partial f }{ \partial x }(x,\omega)d\omega.
    \end{equation}
\end{theorem}

%+++++++++++++++++++++++++++++++++++++++++++++++++++++++++++++++++++++++++++++++++++++++++++++++++++++++++++++++++++++++++++
\section{Espace de probabilité}
%+++++++++++++++++++++++++++++++++++++++++++++++++++++++++++++++++++++++++++++++++++++++++++++++++++++++++++++++++++++++++++

Une \defe{mesure de probabilité}{mesure!probabilité} sur un espace mesuré \( (\Omega,\tribA)\) est une mesure positive telle que \( P(\Omega)=1\). Dans ce cas, le triple \( (\Omega,\tribA,P)\) est un \defe{espace de probabilité}{espace!de probabilité}.

Un point \( \omega\in\Omega\) est une \defe{observation}{observation}, une partie mesurable \( A\in\tribA\) est un \defe{événement}{événement}. L'ensemble \( A\cup B\) représente l'événement \( A\) ou \( B\) tandis que l'ensemble \( A\cap B\) représente l'événement \( A\) et \( B\).


Si les \( A_n\) sont des événements, nous définissons la \defe{limite supérieur}{limite!supérieure} et la \defe{limite inférieure}{limite!inférieure} de la suite \( A_n\) par
\begin{equation}
    \limsup_{n\to\infty}A_n=\bigcap_{n\geq 1}\bigcup_{k\geq n}A_k
\end{equation}
et
\begin{equation}
    \liminf_{n\to\infty}A_n=\bigcup_{n\geq 1}\bigcap_{k\geq n}A_k
\end{equation}
Si \( \omega\in\liminf A_n\), alors \( \omega\) réalise tous les \( A_n\) sauf un nombre fini.

Nous avons
\begin{equation}
    \limsup A_n=\{ \omega\in\Omega\tq \omega\in A_n\text{pour une infinité de \( n\)} \}.
\end{equation}

\begin{theorem}[Borel-Cantelli]\index{théorème!Borel-Cantelli}
    Si
    \begin{equation}
        \sum_{n=1}^{\infty}P(A_n)<\infty
    \end{equation}
    alors \( P(\limsup A_n)=0\).
\end{theorem}

\begin{proof}
    La condition \( \sum_{n\geq 1}P(A_n)<\infty\) signifie que la fonction
    \begin{equation}
        \varphi=\sum_{n\geq 1}\caract_{A_n}
    \end{equation}
    est \( P\)-intégrable. Par conséquent, elle est finie presque partout (au sens de \( P\)), c'est à dire
    \begin{equation}
        P(\varphi=\infty)=0.
    \end{equation}
    Les points \( \omega\) sur lesquels \( \varphi(\omega)=\infty\) sont ceux tels que
    \begin{equation}
        \sum_{n\geq 1}\caract_{A_n}(\omega)=\infty,
    \end{equation}
    c'est à dire les \( \omega\) qui appartiennent à une infinité d'ensembles \( A_n\), ou encore les \( \omega\in\limsup A_n\). Nous avons donc montré que
    \begin{equation}
        \{ \omega\tq \varphi(\omega)=\infty \}=\{ \omega\in\Omega\tq \omega\in A_n\text{pour une infinité de \( n\)} \}=\limsup A_n.
    \end{equation}
    Or l'hypothèse signifie que la probabilité du membre de gauche est nulle.
\end{proof}

\begin{corollary}
    Si \( \sum_{n=1}^{\infty}P(\complement A_n)<\infty\), alors presque surement tous les \( B_n\) sont réalisés à l'exception d'un nombre fini.
\end{corollary}

%+++++++++++++++++++++++++++++++++++++++++++++++++++++++++++++++++++++++++++++++++++++++++++++++++++++++++++++++++++++++++++
\section{Variables aléatoires}
%+++++++++++++++++++++++++++++++++++++++++++++++++++++++++++++++++++++++++++++++++++++++++++++++++++++++++++++++++++++++++++

\begin{definition}
    Une \defe{variable aléatoire}{variable aléatoire} est une application mesurable
    \begin{equation}
        X\colon (\Omega,\tribA)\to (\eR^d,\Borelien(\eR^d)).
    \end{equation}
\end{definition}
Nous convenons que \( \eR^1=\bar\eR\), c'est à dire que dans le cas où la variable aléatoire \( X\) est réelle, nous acceptons les valeurs \( \pm\infty\).

On dit que la variable aléatoire \( X\) a un \defe{moment d'ordre \( p\)}{moment} si \( X\in L^p(\Omega,\tribA,P)\) (\( 1\leq p<\infty\)). Si \( X\) est \defe{intégrable}{variable aléatoire!intégrable} (c'est à dire si \( X\in L^1\)), alors nous définissons l'\defe{espérance}{espérance} de \( X\) par
\begin{equation}
    E(X)=\int_{\Omega}XdP\in\eR^d.
\end{equation}
Si \( E(X)=0\) nous disons que la variable aléatoire est \defe{centrée}{variable aléatoire!centrée}. La variable aléatoire \( X-E(X)\) est la variable aléatoire centrée associée à \( X\).

\begin{definition}      \label{DefAbsoluCont}
    Une fonction \( F\colon \eR\to \eR\) est \defe{absolument continue}{absolument continue} sur \( \mathopen[ a , b \mathclose]\) si il existe une fonction \( f\) sur \( \mathopen[ a , b \mathclose]\) telle que
    \begin{equation}
        F(x)=\int_a^xf(t)dt
    \end{equation}
    pour tout \( x\in\mathopen[ a , b \mathclose]\).

    Une variable aléatoire réelle \( X\) est \defe{absolument continue}{variable aléatoire!absolument continue} si il existe une fonction positive et intégrable \( f\colon \eR\to \eR\) telle que pour tout intervalle \( I\subset\eR\),
    \begin{equation}
        P(X\in I)=\int_If(t)dt.
    \end{equation}
    Nous disons alors que \( f\) est la \defe{densité}{densité} de \( X\).
\end{definition}



La définition suivante vient de l'instructive motivation de \cite{CourgGudRennes}.
\begin{definition}
    Les variables aléatoires \( X\) sur \( (\Omega,\tribA)\) et \( X'\) sur \( (\Omega',\tribA')\) sont \defe{indépendantes}{indépendance!variables aléatoires} si \( \forall B\in\tribA\) et \( \forall B'\in\tribA'\) nous avons
    \begin{equation}
        P(X\in B,X'\in B')=P(X\in B)P(X'\in B').
    \end{equation}
\end{definition}


%---------------------------------------------------------------------------------------------------------------------------
\subsection{Variance}
%---------------------------------------------------------------------------------------------------------------------------

Si \( X\in L^2(\Omega,\tribA,P)\) alors nous définissons la \defe{variance}{variance} de \( X\) par
\begin{equation}
    \Var(X)=E\big( [X-E(X)]^2 \big).
\end{equation}

\begin{proposition}     \label{PrropVarAlterfrom}
    La variance de la variable aléatoire \( X\) peut être exprimée par la formule
    \begin{equation}
        \Var(X)=E(X^2)-[E(X)]^2
    \end{equation}
    où \( X^2=X\cdot X\) et \( E(X)^2=\) sont des produits scalaires dans \( \eR^d\).
\end{proposition}

\begin{proof}
    De façon explicite, nous avons
    \begin{equation}
        E\big( [X-E(X)]^2 \big)=\int_{\Omega}\big( X(\omega)-E(X) \big)\cdot\big( X(\omega)-E(X) \big)dP(\omega)
    \end{equation}
    où \( E(X)\in\eR^d\) est une constante. En développant le produit scalaire nous avons
    \begin{subequations}
        \begin{align}
            E\big( [X-E(X)]^2 \big)&=E\big( X^2-2X\cdot E(X)+E(X)^2 \big)\\
            &=E(X^2)-2E(X)^2+E(X)^2\\
            &=E(X^2)-E(X)^2.
        \end{align}
    \end{subequations}
\end{proof}


Nous définissons l'\defe{écart-type}{écart-type} de \( X\) par
\begin{equation}
    \sigma_X=\sqrt{\Var(X)}.
\end{equation}
En d'autres termes,
\begin{equation}
    \sigma_X=\| X-E(X) \|_{L^2}.
\end{equation}
On définit encore la \defe{moyenne quadratique}{moyenne!quadratique} de \( X\) par
\begin{equation}
    \| X \|_{L^2}=\big[ E(X^2) \big]^{1/2}.
\end{equation}

Si \( X\) est une variable aléatoire réelle, nous définissons sa \defe{fonction de répartition}{fonction!de répartition} par
\begin{equation}
    \begin{aligned}
        F_X\colon \eR&\to \mathopen[ 0 , 1 \mathclose] \\
        F_X(x)&=P(X\leq x). 
    \end{aligned}
\end{equation}


La \defe{fonction caractéristique}{fonction!caractéristique} de la variable aléatoire \( X\colon \Omega\to \eR\) est la fonction réelle définie par
\begin{equation}
    \Phi_X(t)=E( e^{itX}),
\end{equation}
ou encore, si \( X\) a une densité \( f_X\),
\begin{equation}
    \Phi_X(t)=\int_{\eR} e^{itX}dP_X(x)=\int_{\eR} e^{itx}f_X(x)dx
\end{equation}
Nous reconnaissons la transformée de Fourier :
\begin{equation}
    \Phi_X(t)=(TF\,f_X)(-t/2\pi).
\end{equation}

La proposition suivante se déduit en utilisant le théorème de dérivation sous l'intégrale \ref{ThoDerSousIntegrale}.
\begin{proposition}     \label{PropDerFnCaract}
    Soit $X$ une variable aléatoire qui accepte un moment d'ordre \( r\geq 1\). Alors la fonction de répartition \( \Phi_X\) est \( r\) fois continument dérivable et
    \begin{equation}
        \Phi_X^{(r)}(t)=E\big( (iX)^r e^{itX} \big).
    \end{equation}
\end{proposition}

\begin{proof}
    Nous étudions la fonction
    \begin{equation}
        \Phi(t)=\int_{\Omega} e^{itX(\omega)}dP(\omega).
    \end{equation}
    Nous considérons la fonction
    \begin{equation}
        \begin{aligned}
            f\colon \eR\times\Omega&\to \eR \\
            (t,\omega)&\mapsto  e^{itX(\omega)}. 
        \end{aligned}
    \end{equation}
    et nous regardons si ce contexte vérifie les hypothèses du théorème \ref{ThoDerSousIntegrale}.
    \begin{enumerate}
        \item
            Étant donné que \( X\) est mesurable, \( f\) sera mesurable.
        \item
            La fonction \( t\mapsto e^{itX(\omega)}\) est absolument continue pour chaque \( \omega\).
        \item
            Note : par rapport aux notations du théorème \ref{ThoDerSousIntegrale}, nous avons ici \( A=\eR\). Prenons donc un intervalle (compact) \( \mathopen[ a , b \mathclose]\subset\eR\) et calculons
            \begin{equation}        \label{EqfpfpttoiXieitXo}
                \frac{ \partial f }{ \partial t }(t,\omega)=iX(\omega) e^{itX},
            \end{equation}
            et
            \begin{equation}
                \int_a^b\int_{\Omega}\left| iX(\omega) e^{itX(\omega)} \right| d\omega\,dt=\int_a^b\int_{\Omega}| X(\omega) |d\omega\,dt.
            \end{equation}
            Par hypothèse \( X\) accepte un moment d'ordre \( 1\), de sorte que l'intégrale par rapport à \( \omega\) converge vers un nombre qui ne dépend pas de \( t\). L'intégrale sur \( t\) ne pose alors aucun problèmes.
    \end{enumerate}
    Par conséquent nous pouvons effectuer la première dérivation :
    \begin{equation}
        \Phi'(t)=\frac{ d\Phi }{ dt }(t)=\int_{\Omega}iX(\omega) e^{itX(\omega)}d\omega=E(iX e^{itX})
    \end{equation}
    et la fonction \( \Phi'\) est absolument continue. Ce dernier point est important parce que c'est lui qui permet de faire la récurrence et passer à l'ordre deux.

    Le résultat ressort alors en dérivant successivement l'expression \eqref{EqfpfpttoiXieitXo}.
\end{proof}


\begin{theorem}
    Si \( \Phi_X=\Phi_Y\), alors \( P_X=P_Y\).
\end{theorem}
\begin{proof}
    No proof.
\end{proof}
Notons que cela n'implique pas que \( X=Y\). En effet \( X\) et \( Y\) peuvent même être définis sur des espaces probabilisés différents.

Dans le cas d'une variable aléatoire vectorielle, nous définissons \( \Phi_X\colon \eR^d\to \eR\) par
\begin{equation}
    \Phi_X(v)=E\big(  e^{i\langle v, X\rangle } \big)
\end{equation}
%---------------------------------------------------------------------------------------------------------------------------
\subsection{Loi d'une variable aléatoire}
%---------------------------------------------------------------------------------------------------------------------------

La \defe{loi}{loi d'une variable aléatoire} de la variable aléatoire \( X\), notée \( P_X\) est la mesure image de \( P\) par \( X\), c'est à dire
\begin{equation}
    P_X(B)=P(X\in B)
\end{equation}
pour tout borélien \( B\subset\eR^d\). Note :
\begin{equation}
    P(X\in B)=P\big( \{ \omega\in\Omega\tq X(\omega)\in B \} \big)=P\big( X^{-1}(B) \big).
\end{equation}
En particulier \( P_X\) est une mesure de probabilité sur \( \eR^d\) parce que 
\begin{equation}
    P_X(\eR^d)=P(\Omega)=1.
\end{equation}
Si \( Q\) est une mesure de probabilité sur \( \eR^d\), nous notons \( X\sim Q\) si \( P_X=Q\). Nous disons alors que «\( X\) suit la loi \( Q\)».

La proposition suivante permet de calculer en pratique les intégrales qui définissent par exemple l'espérance mathématique d'une variable aléatoire.
\begin{proposition}     \label{PropintdPintdPXeR}
    Si \( f\colon \eR^d\to \bar\eR\) est borélienne et si \( X\) est une variable aléatoire, alors
    \begin{equation}
        \int_{\eR^d}f(x)dP_X(x)=\int_{\Omega}f\big( X(\omega) \big)dP(\omega)=E(f\circ X).
    \end{equation}
\end{proposition}

\begin{proof}
    No proof.
\end{proof}

En utilisant cette proposition nous trouvons une formule pratique pour l'espérance d'une variable aléatoire réelle:
\begin{equation}
    E(X)=\int_{\Omega}X(\omega)dP(\omega)=\int_{\eR}xdP_X(x),
\end{equation}
en vertu de la proposition \ref{PropintdPintdPXeR} appliquée à la fonction \( f(x)=x\).

\begin{proposition}
    Une variable aléatoire réelle \( X\) est intégrable si et seulement si \( P(x=\pm\infty)=0\) et
    \begin{equation}
        \int_{\eR}| x |dP_X(x)<\infty.
    \end{equation}
\end{proposition}

Le lien entre la densité \( f_X\) de la variable aléatoire \( X\) et sa loi est
\begin{equation}
    P_X(A)=\int_Af_X(x)dx
\end{equation}
pour tout ensemble mesurable \( A\subset\eR\). Le lien entre la mesure de Lebesgue et celle de la loi de \( X\) est alors donné par
\begin{equation}
    dP_X(x)=f_X(x)dx.
\end{equation}
En particulier l'espérance de \( X\) peut être calculée à partir de sa densité via la formule
\begin{equation}        \label{EqEspDensform}
    E(X)=\int_{\eR}xdP_X(x)=\int_{\eR}x f_X(x)dx.
\end{equation}

%---------------------------------------------------------------------------------------------------------------------------
\subsection{Lois conjointes et indépendance}
%---------------------------------------------------------------------------------------------------------------------------

\begin{definition}
    Deux événements \( A\) et \( B\) sont dits \defe{indépendants}{indépendance} si
    \begin{equation}
        P(A\cap B)=P(A)P(B).
    \end{equation}
\end{definition}
Si nous considérons \( n\) variables aléatoires réelles \( X_1,\ldots,X_n\colon\Omega\to\eR\), la loi du \( n\)-uplet \( X=(X_1,\ldots,X_n)\) est une variable aléatoire \( X\colon \Omega\to \eR^n\) appelée la \defe{loi conjointe}{loi!conjointe} des lois \( X_i\). Dans ce cas, les variables aléatoires \( X_i\) elles-mêmes sont dites lois \defe{marginales}{loi!marginale} de \( X\).

\begin{proposition}     \label{PropPXXXPXPXPX}
    Les variables aléatoires \( \{ X_i \}\) sont indépendantes si et seulement si
    \begin{equation}
        P_{(X_1,\ldots,X_n)}=P_{X_1}\otimes\ldots\otimes P_{X_n}.
    \end{equation}
\end{proposition}

\begin{definition}      \label{DefFonrepConj}
    Soient \( \{ X_i \}_{1\leq i\leq n}\) des variables aléatoires réelles (pas spécialement indépendantes). La \defe{densité conjointe}{densité!conjointe} de \( X_1\),\ldots,\( X_n\) est la fonction \( f\colon \eR^n\to \eR\) qui satisfait
    \begin{enumerate}
        \item
            \( f(x_1,\ldots,x_n)\geq 0\) pour tout \( (x_1,\ldots,x_n)\in\eR^n\),
        \item
            \( \int_{\eR^n}f=1\),
        \item       \label{ItemDefFonrepConjiii}<++>
            pour tout \( A_i\subset\eR \) nous avons
            \begin{equation}
                P(\bigcap_{i=1}^n X_i\in A_i)=\int_{\prod_i A_i}f(x_1,\ldots,x_n)dx_1\ldots dx_n.
            \end{equation}
    \end{enumerate}
\end{definition}

\begin{proposition}     \label{PropDensiteConjIndep}<++>
    Si les variables aléatoires \( X_1\),\ldots \( X_n\) sont indépendantes et ont des densités \( f_{X_1}\),\ldots,\( f_{X_n}\), alors la variable aléatoire conjointe \( X=(X_1,\ldots,X_n)\) a pour densité conjointe la fonction
    \begin{equation}
        f_X(x_1,\ldots,x_n)=f_{X_1}(x_1)\ldots f_{X_n}(x_n).
    \end{equation}
\end{proposition}

\begin{proof}
    En partant de la définition de l'indépendance et de la fonction de densité conjointe, ainsi qu'en utilisant le théorème de Fubini,
    \begin{equation}
        \begin{aligned}[]
            \int_{A_1\times \ldots\times A_n}f_X(x_1,\ldots,x_n)dx_1\ldots dx_n&=
            P(X_1\in A_1,\ldots,X_n\in A_n)\\
            &=P(X_1\in A_1)\ldots P(X_n\in A_n)\\
            &=\left( \int_{A_1}f_{X_1}(x_1)dx_1 \right)\ldots\left( \int_{A_n}f_{X_n}(x_n)dx_n \right)\\
            &=\int_{A_1\times\ldots\times A_n}f_{X_1}(x_1)\ldots f_{X_n}(x_n)dx_1\ldots dx_n.
        \end{aligned}
    \end{equation}
    La fonction \( (x_1,\ldots,x_n)\mapsto f_{X_1}(x_1)\ldots f_{X_n}(x_n)\) vérifie donc la condition \ref{ItemDefFonrepConjiii} de la définition \ref{DefFonrepConj}. La vérification des autres conditions est immédiate.
\end{proof}


La proposition suivante\cite{ProbaDanielLi} provient du fait que la mesure d'une loi conjointe est le produit des mesures lorsque les variables aléatoires sont indépendantes (proposition \ref{PropPXXXPXPXPX}).
\begin{proposition}
    Si les variables aléatoires réelles \( X_1\),\ldots,\( X_n\) sont intégrables et indépendantes, alors leur produit est intégrable et l'espérance du produit est égal au produit des espérances :
    \begin{equation}
        E(X_1\cdots X_n)=E(X_1)\ldots E(X_n).
    \end{equation}
\end{proposition}

%///////////////////////////////////////////////////////////////////////////////////////////////////////////////////////////
\subsubsection{Somme de variables aléatoires indépendantes}
%///////////////////////////////////////////////////////////////////////////////////////////////////////////////////////////
\label{subsubsecscnvommevariablsindep}

Le cas plus général de variable aléatoires non indépendantes peut être trouvé dans \cite{Marazzi}.

Soient \( X\) et \( Y\), deux variables aléatoires réelles indépendantes. Nous voudrions étudier la loi de la variable aléatoire \( S=X+Y\). Nous commençons par calculer la fonction de répartition en utilisant le résultat de la proposition \ref{PropDensiteConjIndep} :
\begin{subequations}
    \begin{align}
        F_{X+Y}(z)=P(X+Y\leq z)&=\int_{x+y\leq z}f_{X,Y}(x,y)dx\,dy\\
        &=\int_{-\infty}^{\infty}dx\int_{-\infty}^{z-x}dyf_X(x)f_Y(y)\\
        &=\int_{\eR}\left( \int_{-\infty}^{z-x}f_Y(y)dy \right)f_X(x)dx\\
        &=\int_{\eR}F_Y(z-x)f_X(x)dx.
    \end{align}
\end{subequations}
Pour calculer la fonction de densité de \( S\), nous dérivons la fonction de répartition :
\begin{subequations}
    \begin{align}
        f_{X+Y}(z)&=\frac{ d F_{X+Y} }{ d z }(z)\\
        &=\int_{\eR}f_Y(z-x)f_X(x)dx,
    \end{align}
\end{subequations}
ce qui nous amène à dire que la densité de la somme est le produit de convolution\index{convolution} des densités :
\begin{equation}
    f_{X+Y}(x)=\int_{\eR}f_Y(x-t)f_X(t)dt,
\end{equation}
ou encore \( f_{X+Y}=f_X\star f_Y\).

Notez que nous avons passé sous le silence la difficulté d'inverser la dérivée et l'intégrale. Un exemple sera donné au point \ref{subsecPoissonetexpo}.


%+++++++++++++++++++++++++++++++++++++++++++++++++++++++++++++++++++++++++++++++++++++++++++++++++++++++++++++++++++++++++++
\section{Lois usuelles}
%+++++++++++++++++++++++++++++++++++++++++++++++++++++++++++++++++++++++++++++++++++++++++++++++++++++++++++++++++++++++++++

%---------------------------------------------------------------------------------------------------------------------------
\subsection{Loi de Bernoulli}
%---------------------------------------------------------------------------------------------------------------------------

Une variable aléatoire réelle est de \defe{Bernoulli}{Bernoulli} de paramètre \( p\) (\( 0<p<1\)) si
\begin{equation}
    X\colon \Omega\to \eR
\end{equation}
avec \( P(x=1)=p\) et \( P(X=0)=1-p\). En tant que mesure sur \( \eR\), nous avons
\begin{equation}
    P_X=p\delta_1+(1-p)\delta_0.
\end{equation}
Une fonction \( h\) qui réalise le supremum de la formule \eqref{EqDefintYfdmu} est par exemple une fonction en escalier qui vaut en \( x\) le plus petit entier plus grand ou égal à \( x\). L'espérance d'une loi de Bernoulli est alors
\begin{equation}
    E(x)=p.
\end{equation}
Étant donné que la variable aléatoire \( X\) prend seulement les valeurs \( 0\) et \( 1\), nous avons pour tout ensemble mesurable \( B\)
\begin{equation}
    P_{X^2}(B)=P(X^2\in B)=P(X\in B),
\end{equation}
et par conséquent \( P_{X^2}=P_X\) et \( E(X^2)=E(X)\). Nous trouvons donc la variance
\begin{equation}
    \Var(X)=E(X^2)-E(X)^2=p-p^2=p(1-p).
\end{equation}

%---------------------------------------------------------------------------------------------------------------------------
\subsection{Loi normale}
%---------------------------------------------------------------------------------------------------------------------------

La loi normale de paramètres \( m\) et \( \sigma>0\), notée \( \dN(m,\sigma^2)\) est la loi donnée par la densité
\begin{equation}
    \gamma_{m,\sigma^2}(x)=\frac{1}{ \sigma\sqrt{2\pi} }\exp\left[ -\frac{ 1 }{2}\left( \frac{ x-m }{ \sigma } \right)^2 \right].
\end{equation}

\begin{proposition}
    Si la variable aléatoire réelle \( X\) suit une loi normale \( \dN(m,\sigma^2)\), alors nous avons \( E(X)=m\) et \( \Var(X)=\sigma^2\).
\end{proposition}

\begin{proof}
    L'espérance d'une variable aléatoire se calcule à partir de la formule \eqref{EqEspDensform}:
    \begin{subequations}
        \begin{align}
            E(X)&=\frac{1}{ \sigma\sqrt{2\pi} }\int_{\eR}x\exp\left[ -\frac{1}{ 2 }\left( \frac{ x-m }{ \sigma } \right)^2 \right]dx\\
            &=\frac{1}{ \sigma\sqrt{2\pi} }\sigma\int_{\eR}(\sigma u+m) e^{-u^2/2}
        \end{align}
    \end{subequations}
    où nous avons effectué le changement de variable \( u=(x-m)/\sigma\). Nous utilisons ensuite l'intégrale remarquable
    \begin{equation}
        \int_{\eR} e^{-u^2/2}du=\sqrt{2\pi}.
    \end{equation}

    En ce qui concerne la variance, nous avons le même genre de calculs.
\end{proof}

La \defe{loi normale réduite}{normale!loi réduite} est la densité
\begin{equation}
    \gamma(x)=\gamma_{0,1}(x)=\frac{1}{ \sqrt{2\pi} } e^{-x^2/2}.
\end{equation}
La variable aléatoire \( X\) suit la loi \( \dN(m,\sigma^2)\) si et seulement si la variable aléatoire $Z=\frac{ X-m }{ \sigma }$ suit la loi normale réduite \( \dN(0,1)\).

%+++++++++++++++++++++++++++++++++++++++++++++++++++++++++++++++++++++++++++++++++++++++++++++++++++++++++++++++++++++++++++
\section{Événements}
%+++++++++++++++++++++++++++++++++++++++++++++++++++++++++++++++++++++++++++++++++++++++++++++++++++++++++++++++++++++++++++

\begin{lemma}       \label{LemEXYEXEYindep}\label{PropVarPropnnlin}
    Si \( X\) est une variable aléatoire,
    \begin{enumerate}
        \item
            $\Var(ax)=a^2\Var(X)$ pour tout \( a\in\eR\);
        \item
            Si de plus \( Y\) est une variable aléatoire indépendante de \( X\), alors $\Var(X+Y)=\Var(X)+\Var(Y)$.
    \end{enumerate}
\end{lemma}

\begin{proof}
    Nous avons
    \begin{subequations}
        \begin{align}
            \Var(X+Y)&=E(X^2+Y^2+2XY)-\big( E(X)+E(Y) \big)^2\\
            &=E(X^2)+E(Y^2)+2E(XY)-E(X)^2-E(Y)^2E(X)E(Y).
        \end{align}
    \end{subequations}
    Étant donné que \( X\) et \( Y\) sont indépendantes nous avons \( E(XY)=E(X)E(Y)\).
\end{proof}


Si les \( X_1,\ldots,X_n\) sont des variables aléatoires on considère la \defe{moyenne empirique}{moyenne!empirique}
\begin{equation}
    \bar X_n=\frac{ X_1+\ldots+X_n }{ n }.
\end{equation}

%+++++++++++++++++++++++++++++++++++++++++++++++++++++++++++++++++++++++++++++++++++++++++++++++++++++++++++++++++++++++++++
\section{Convergence}
%+++++++++++++++++++++++++++++++++++++++++++++++++++++++++++++++++++++++++++++++++++++++++++++++++++++++++++++++++++++++++++

Soient \( X_i\) des variables aléatoires réelles définies sur le même espace de probabilité \( (\Omega,\tribA,P)\). Nous disons que \( X_i\) converge \defe{presque surement}{convergence!presque surement} vers la variable aléatoire \( X\) et nous notons 
\begin{equation}
    X_n\stackrel{p.s.}{\longrightarrow}X
\end{equation}
si
\begin{equation}
    P\big( \{ \omega\in\Omega\tq\,X_n(\omega)\to X(\omega) \} \big)=1
\end{equation}
où la convergence \( X_n(\omega)\to X(\omega)\) est la convergence usuelle dans \( \eR\).

\begin{lemma}
    Nous avons \( X_n\stackrel{p.s.}{\longrightarrow}X\) si et seulement si il existe un événement \( A\in\tribA\) tel que \( P(A)=1\) et tel que \( X_n(\omega)\to X(\omega)\) pour tout \( \omega\in A\).
\end{lemma}

Nous disons que les variables aléatoires réelles \( X_n\) convergent \defe{en probabilité}{convergence!en probabilité} vers la variable aléatoire \( X\) si pour tout \( \eta>0\), on a
\begin{equation}
    P\big( | X_n-X |\geq \eta \big)\to 0,
\end{equation}
et on note
\begin{equation}
    X_n\stackrel{P}{\longrightarrow}X.
\end{equation}

Nous disons que \( X_n\) converge vers \( X\) \defe{en loi}{convergence!en loi} vers la variable aléatoire \( X\) et nous notons
\begin{equation}
    X_n\stackrel{\hL}{\longrightarrow}X
\end{equation}
si pour toute fonction continue et bornée \( g\) nous avons
\begin{equation}
    E\big( g(X_n) \big)\to E\big( g(X) \big)=\int gdP_X.
\end{equation}


\begin{proposition}     \label{PrpopCaractCvL}
    Deux autres caractérisations de la convergence en loi.
    \begin{enumerate}
        \item
            Nous avons \( X_n\stackrel{\hL}{\longrightarrow}X\) si et seulement si
            \begin{equation}
                \Phi_{X_n}(v)\to\Phi_X(v)
            \end{equation}
            pour tout \( v\in\eR^d\). Ici \( \Phi_X\) est la fonction caractéristique de \( X\).
        \item
            Dans la définition de la convergence en loi nous pouvons indifféremment utiliser les fonctions continues et bornées, les fonctions continues à support compact ou les fonctions bornées uniformément continues.
    \end{enumerate}
\end{proposition}

\begin{proposition}
    La convergence presque sure entraine à la fois la convergence en probabilité et celle en loi. La convergence en probabilité entraine la convergence en loi.
\end{proposition}
La convergence en loi n'implique pas la convergence en probabilité, et par conséquent pas non plus la convergence presque certaine.

Dans le cas particulier \( d=1\) nous avons quelque critères supplémentaires. 

\begin{proposition}
    Supposons que les variables aléatoires \( X_n\) soient réelles, et notons \( F_n\) la fonction de répartition de \( X_n\). Si \( F_n(x)\to F(x)\) pour tout \( x\) dans l'ensemble des points de continuité de \( F\), alors \( X_n\stackrel{\hL}{\longrightarrow}X\).
\end{proposition}

\begin{proposition}
    Si les \( X_n\) sont des variables aléatoires réelles positives, et si \( X\) est une variable aléatoire positive, alors \( X_n\stackrel{\hL}{\longrightarrow}X\) si les transformées de Laplace des fonctions de répartition convergent ponctuellement, c'est à dire si
    \begin{equation}
        E\big(  e^{-\alpha X_n} \big)\to E\big(  e^{-\alpha X} \big)
    \end{equation}
    pour tout \( \alpha\geq 0\).
\end{proposition}

\begin{proposition}
    Si les \( X_n\) et \( X\) sont des variables aléatoires réelles discrètes à valeurs dans \( \{ x_0,x_1,\ldots \}\) alors \( X_n\stackrel{\hL}{\longrightarrow} X\) si et seulement si
    \begin{equation}
        P(X_n=x_k)\to P(X=x_k)
    \end{equation}
    pour tout \( k\in\eN\).
\end{proposition}

\begin{proposition}     \label{PropXncvXFXcvFxt}
    Soient \( X_n\) et \( X\) des variables aléatoires réelles. Nous avons
    \begin{equation}
        X_n\stackrel{\hL}{\longrightarrow}X
    \end{equation}
    si et seulement si pour tout \( t\) où \( F_X\) est continue,
    \begin{equation}
        \lim_{n\to \infty} F_{X_n}(t)=F_X(t).
    \end{equation}
\end{proposition}
Pour une preuve, voir \cite{CourgGudRennes}.

\begin{proposition}     \label{PropCvLfcvPsicst}
    Soit \( X_n\) une suite de variables aléatoires \( \Omega\to \eR^d\) et \( a\in\eR^d\). Si \( X_n\stackrel{\hL}{\longrightarrow}\), alors
    \begin{equation}
        X_n\stackrel{P}{\longrightarrow}a.
    \end{equation}
\end{proposition}

\begin{proof}
    Cette preuve provient de \cite{CourgGudRennes}. Quitte à passer aux composantes, nous pouvons supposer que \( d=1\). Nous avons
    \begin{subequations}
        \begin{align}
           P\big( | X_n-a |>\eta \big)&=P(X_n-a>\eta)+P(-X_n-a>\eta)\\
           &=P(X_n>\eta+a)+1-P(X_n>a-\eta)  \label{EqPXngeqetaapumP}\\
           &=1-F_{X_n}(\eta+a)+F_{X_n}(a-\eta).
        \end{align}
    \end{subequations}
    Nous allons utiliser la proposition \ref{PropXncvXFXcvFxt}. La fonction de répartition de la variable aléatoire constante \( X=a\) est donnée par
    \begin{equation}
        F_a(t)=P(a<t)=\caract{\eR^+}(t-a).
    \end{equation}
    Par conséquent, la convergence en loi \( X_n\stackrel{\hL}{\longrightarrow}a\) nous montre que
    \begin{equation}
        F_{X_n}(t)\to \caract_{\eR^+}(t-a)
    \end{equation}
    pour tout \( t\neq a\) parce que \( t=0\) est un point de discontinuité de \( \caract_{\eR^+}\). Nous avons par conséquent
    \begin{equation}
        P\big( | X_n-a |>\eta \big)=1-\caract_{\eR^+}(\eta)+\caract_{\eR^+}(-\eta)=0
    \end{equation}
    parce que \( \eta>0\).
\end{proof}

\begin{probleme}
    Dans \cite{CourgGudRennes}, l'équation \eqref{EqPXngeqetaapumP} arrive avec une inégalité. Pourquoi ?
\end{probleme}

\begin{lemma}[Slutsky]\index{lemme!de Slutsky}
    Soient \( X_n\) et \( Y_n\) des suites de variables aléatoires réelles telles que
    \begin{equation}
        \begin{aligned}[]
            X_n&\stackrel{\hL}{\longrightarrow} X&\text{et}&&Y_n&\stackrel{\hL}{\longrightarrow}a\in\eR.
        \end{aligned}
    \end{equation}
    Alors \( (X_n,Y_n)\stackrel{\hL}{\longrightarrow}(X,a)\).
\end{lemma}

\begin{proof}
    La preuve qui suit provient de \cite{MPSmaitrise}. Étant donné que \( Y_n\stackrel{\hL}{\longrightarrow} a\), nous avons \( Y_n\stackrel{P}{\longrightarrow} a\) par la proposition \ref{PropCvLfcvPsicst}. Soit une fonction \(f\colon \eR^2\to \eR^2 \); nous devons prouver que
    \begin{equation}
        E\big( f(X_n,Y_n) \big)\to E\big( f(X,a) \big).
    \end{equation}
    Soit \( \epsilon>0\). Nous avons
    \begin{equation}    \label{EqEparXnYnfXa}
        E\big( \| f(X_n,Y_n)-f(X,a) \| \big)\leq E\big(  \| f(X_n,Y_n)-f(X_n,a) \|  \big)+E\big(   \| f(X_n,a)-f(X,a) \|  \big).
    \end{equation}
    La fonction\( g(t)=f(t,a)\) étant continue et bornée, la convergence en loi \( X_n\stackrel{\hL}{\longrightarrow}X\) donne
    \begin{equation}
        E\big( \| f(X_n,a)-f(X,a) \| \big)\to 0.
    \end{equation}
    Étudions à présent le premier terme du membre de droite de \eqref{EqEparXnYnfXa}. Pour tout \( \eta> 0\) et toute variables aléatoires \( Z\) et \( Z'\) nous avons
    \begin{equation}
        E(Z)=E(Z\caract_{| Z' |<\eta})+E(Z\caract_{| Z' |\geq \eta}).
    \end{equation}
    Nous décomposons donc le premier terme de \eqref{EqEparXnYnfXa} en
    \begin{equation}    \label{EqEXnADecomsecvolta}
        \begin{aligned}[]
            E\big( \| f(X_n,Y_n)-f(X_n,a) \| \big)&=E\big( \| f(X_n,Y_n)-f(X_n,a) \|\caract_{| Y_n-a |<\eta} \big)\\
            &\quad+E\big( \| f(X_n,Y_n)-f(X_n,a) \|\caract_{| Y_n-a |\geq\eta} \big).
        \end{aligned}
    \end{equation}
    Choisissons maintenant une valeur de \( \eta\) telle que
    \begin{equation}
        | (x,y)-(x',y') |<\eta\Rightarrow| f(x,y)-f(x',y') |\leq \epsilon.
    \end{equation}
    Un tel \( \eta\) existe par l'uniforme continuité de \( f\). Dans le premier terme, \( | Y_n-a |<\eta\), par conséquent
    \begin{equation}
        \| (X_n,Y_n)-(X_n,a) \|=| Y_n-a |<\eta
    \end{equation}
    et donc
    \begin{equation}
        \| f(X_n,Y_n)-f(X_n,a) \|\leq \epsilon.
    \end{equation}
    Le premier terme devient donc
    \begin{equation}
        E\big( \| f(X_n,Y_n)-f(X_n,a) \|\caract_{| Y_n-a |<\eta} \big)\leq \epsilon E(\caract_{| Y_n-a |<\eta})\leq \epsilon
    \end{equation}
    parce que \( E(\caract_A)=P(A)\leq 1\). Pour le second terme de \eqref{EqEXnADecomsecvolta} nous effectuons la majoration
    \begin{equation}
        \| f(X_n,Y_n)-f(X_n,a) \|\leq 2\| f \|_{\infty}
    \end{equation}
    tandis que la convergence \( Y_n\stackrel{P}{\longrightarrow} a\) entraine 
    \begin{equation}
        P\big( | Y_n-a |\geq \eta \big).
    \end{equation}
\end{proof}

\begin{lemma}[Borel-Cantelli]\index{lemme!de Borel-Cantelli}
    Soit \( (A_n)\) une suite d'événements (avec \( A_n\in\tribA\) pour tout \( n\)).
    \begin{enumerate}
        \item
            Si \( \sum_{n=0}^{\infty}P(A_n)\) converge, alors
            \begin{equation}
                P(A_n\,\text{i.s.})=0.
            \end{equation}
        \item
            Si la somme \( \sum_nP(A_n)\) diverge, et si de plus les \( A_i\) sont indépendants, alors
            \begin{equation}
                P(A_n\,\text{i.s.})=1.
            \end{equation}
    \end{enumerate}
\end{lemma}
La notation \( P(A_n\,\text{i.s.})\) signifie «infiniment souvent», c'est à dire
\begin{equation}
    P(A_n\,\text{i.s.})=P\big( \bigcap_{N\in\eN}\bigcup_{k\geq N}A_k \big)=P(\limsup A_n)
\end{equation}

Une façon de paraphraser le lemme de Borel-Cantelli est que nous avons l'alternative
\begin{equation}    \label{EqparaphrCantelli}
    P(\limsup A_n)=\begin{cases}
        0    &   \text{si $\sum_{n\geq 0}P(A_n)<\infty$}\\
        1    &    \text{sinon}.
    \end{cases}
\end{equation}

\begin{proposition}
    Soit \( X_n\), une suite de variables aléatoires et \( X\) une variable aléatoires. Si
    \begin{equation}
        \sum_nP\big( \| X_n-X \|>\eta \big)<\infty
    \end{equation}
    pour tout \( \epsilon\), alors \( X_n\stackrel{p.s.}{\longrightarrow}X\).
\end{proposition}

\begin{proof}
    Fixons \( \epsilon\) et considérons les événements \( A_{n}=\| X_n-X \|>\epsilon\). L'hypothèse dit que
    \begin{equation}
        \sum_nP(A_{n,\epsilon})<\infty
    \end{equation}
    et le lemme de Borel-Cantelli implique que
    \begin{equation}
        P(\limsup\| X_n-X \|>\epsilon)=0.
    \end{equation}
    Un élément \( \omega\) est dans \( \limsup A_n\) si il est contenu dans tous les \( A_n\), par conséquent, pour chaque \( \epsilon\) nous avons l'inclusion
    \begin{equation}
        \{ \omega\in\Omega\tq X_n(\omega)\to X(\omega) \}\subset\complement\limsup A_n.
    \end{equation}
    Nous pouvons aller plus loin et écrire
    \begin{equation}        \label{EqProbomOmXnmXforalleps}
        \{ \omega\in\Omega\tq X_n(\omega)\to X(\omega) \}=\complement\{ \omega\in\Omega\tq \| X_n-X \|>\epsilon,\forall\epsilon>0 \}.
    \end{equation}
    Or la probabilité de l'ensemble
    \begin{equation}
        \{ \omega\in\Omega\tq\| X_n-X \|>\epsilon \}
    \end{equation}
    est \( 0\) pour chaque \( \epsilon\), et par conséquent la probabilité du membre de droite de \eqref{EqProbomOmXnmXforalleps} est \( 1\).
\end{proof}

\begin{example}
    Considérons une suite de \( 0\) et de \( 1\) dans laquelle le \( 1\) arrive avec une probabilité \( p\) et le \( 0\) avec une probabilité \( 1-p\). Une telle suite est modélisée par une suite de variables aléatoires de Bernoulli \( (X_n)_{n\in\eN}\) indépendantes de paramètre \( p\).

    Question : une telle suite contient elle une infinité de \( 1\) ? Considérons les événements indépendants \( A_n=\{ X_n=1 \}\). Nous avons
    \begin{equation}
        \sum_n P(A_n)=\sum_nP(X_n=1)=\sum_np=\infty.
    \end{equation}
    Par Borel-Cantelli et son expression \eqref{EqparaphrCantelli}, nous avons alors
    \begin{equation}
        P(\limsup A_n)=1.
    \end{equation}
    Donc une infinité d'événements \( A_n\) se produisent, et nous avons bien une infinité de \( 1\) dans la suite.

    Remarque : dans ce raisonnement nous pouvons considérer une probabilité non constante \( p_n\) tant que la série \( \sum_np_n\) diverge.
\end{example}

%+++++++++++++++++++++++++++++++++++++++++++++++++++++++++++++++++++++++++++++++++++++++++++++++++++++++++++++++++++++++++++
\section{Loi des grands nombres, théorème central limite}
%+++++++++++++++++++++++++++++++++++++++++++++++++++++++++++++++++++++++++++++++++++++++++++++++++++++++++++++++++++++++++++

\begin{lemma}[Inégalité de Markov]\index{Markov (inégalité)}
    Soit une variable aléatoire \( X\in L^p\) et \( \epsilon>0\). Nous avons
    \begin{equation}
        P(| X |\geq \epsilon)\leq \frac{1}{ \epsilon^r }E(| X |^r).
    \end{equation}
\end{lemma}

\begin{proof}
    Nous avons
    \begin{equation}
        E(| X |^r)\geq\int_{| X |\geq \epsilon}X(\omega)dP(\omega)\geq \epsilon^r\int_{| X |\geq\epsilon}dP=\epsilon^rP(| Y |\geq\epsilon).
    \end{equation}
\end{proof}

\begin{theorem}[Loi forte des grands nombres]\index{loi!des grands nombres (forte)}
    Soit \( (X_n)\) une suite de variables aléatoires réelles
    \begin{enumerate}
        \item
            indépendantes et identiquement distribuées,
        \item
            intégrables (c'est à dire dans \( L^1\)),
    \end{enumerate}
    et \( \bar X_n=\frac{1}{ n }\sum_{i=1}^nX_i\). Alors
    \begin{equation}
        \bar X_n\stackrel{p.s.}{\longrightarrow} E(X_1).
    \end{equation}
\end{theorem}
Note : étant donné que les variables aléatoires sont identiquement distribuées, nous avons évidemment \( E(X_1)=E(X_2)=\ldots\)

\begin{probleme}
    Est-ce que les variables aléatoires doivent vraiment être réelles ?
\end{probleme}

\begin{corollary}
    Si les variables aléatoires réelles \( X_n\) sont
    \begin{enumerate}
        \item
            indépendantes et identiquement distribuées,
        \item
            dans \( L^2\)
    \end{enumerate}
    alors
    \begin{equation}
        \bar X_n\stackrel{P}{\longrightarrow}E(X_1).
    \end{equation}
\end{corollary}

\begin{proof}
    Nous voulons prouver que pour tout \( \eta>0\),
    \begin{equation}
        P\big( | \bar X_n-E(X_1) |>\eta \big)\to 0.
    \end{equation}
    Remarquons d'abord que les variables aléatoires \( X_n\) étant identiquement distribuées, \( E(\bar X_n)=E(X_1)\) parce que \( E(X_i)=E(X_1)\) pour tout \( i\). L'inégalité de Markov avec \( r=2\) nous donne
    \begin{equation}
        P\big( | \bar X_n-E(\bar X_n) |>\eta \big)\leq\frac{1}{ \eta^2 }E\big( | \bar X_n-E(\bar X_n) |^2 \big)
    \end{equation}
    où nous reconnaissons \( E\big( | \bar X_n-E(\bar X_n) |^2 \big)=\Var(\bar X_n)\). Par la proposition \ref{LemEXYEXEYindep} nous avons \( \Var(\bar X_n)=\Var(X_1)/n\) et par conséquent
    \begin{equation}
        P\big( | \bar X_n-E(\bar X_n) |>\frac{1}{ n\eta^2 }\Var(X_1),
    \end{equation}
    qui tend vers zéro lorsque \( n\to\infty\).
\end{proof}

\begin{proposition}
    Soient \( X_n\) des variables aléatoires indépendantes et identiquement distribuées avec \( X_n\geq 0\). Nous acceptons \( E(X_1)=\infty\), c'est à dire que nous relaxons la condition \( X_n\in L^1\) par rapport à la loi des grands nombres.

    Alors
    \begin{equation}
        \bar X_n\stackrel{p.s.}{\longrightarrow} E(X_1)\in\mathopen[ 0 , \infty \mathclose].
    \end{equation}
\end{proposition}

\begin{proof}
    Si \( E(X_1)<\infty\), nous sommes dans le cas de la loi des grands nombres. Pour chaque \( N\in\eN\) nous considérons la suite de variables aléatoires
    \begin{equation}
        X_n^{(N)}=\min(X_n,N).
    \end{equation}
    Nous avons évidement \( \bar X^{(N)}_n\leq \bar X_n\). Les variables aléatoires \( X^{(N)}_n\) étant bornées par \( N\), elles vérifient la loi des grands nombres pour chaque \( N\) séparément. Par conséquent nous avons pour chaque \( N\) la limite
    \begin{equation}        \label{EqbarXNbtoXnubus}
        \bar X^{(N)}_n\to E(X^{(N)}_1)
    \end{equation}
    Nous supposons que \( E(X_1)=\infty\), par conséquent  \( \lim_{N\to\infty}E(X_1^{(N)})=\infty\). Soit \( \eta>0\) et choisissons \( N\) de telle manière à avoir
    \begin{equation}
        E(X_1^{(N)})>\eta+1.
    \end{equation}
    La limite \eqref{EqbarXNbtoXnubus} nous permet de trouver \( n_0\) tel que pour tout \( n>n_0\) nous ayons \( \bar X^{(N)}_n>\eta\). Au final,
    \begin{equation}
        \eta<\bar X^{(N)}_n\leq \bar X_n,
    \end{equation}
    ce qui montre que \( \bar X_n\to\infty\).
\end{proof}

\begin{example}
    La loi des grands nombres justifie la pratique courante d'approximer une grandeur physique par la moyenne empirique d'un grand nombre de mesures.
\end{example}

%---------------------------------------------------------------------------------------------------------------------------
\subsection{Marche aléatoire}
%---------------------------------------------------------------------------------------------------------------------------

Nous considérons un mobile qui se déplace sur l'axe \( \eZ\). À chaque pas de temps, nous supposons qu'il va faire un pas à gauche avec une probabilité \( p\) et un pas à droite avec une probabilité \( (1-p)\). Nous nous demandons quel est le mouvement du mobile sur le long terme.

La position \( S_n\) du mobile à l'instant \( n\) est donnée par
\begin{equation}
    S_n=\sum_{i=1}^nX_i
\end{equation}
où \( X_i\) est le pas effectué à l'instant \( i\). Ce sont des variables de Bernoulli indépendantes avec
\begin{equation}
    X_i\stackrel{\hL}{=}p\delta_{-1}+(1-p)\delta_1
\end{equation}
c'est à dire
\begin{subequations}
    \begin{align}
        P(X_i=-1)&=p\\
        P(X_i=1)&=1-p.
    \end{align}
\end{subequations}
Ces variables vérifient les hypothèses de la loi des grands nombres :
\begin{enumerate}
    \item
        elles sont indépendantes et identiquement distribuées,
    \item
        elles sont intégrables.
\end{enumerate}
Pour le second point, le calcul est
\begin{equation}
    \int_{\Omega}| X_i |dP=\int_{\eR}| x |dP_X=\int_{\eR}| x |(p\delta_{-1}+(1-p)\delta_{1})=| 1-p |+| p |=1.
\end{equation}
Nous avons par conséquent
\begin{equation}
    \bar X_n=\frac{1}{ n }\sum_{i=1}^nX_i\stackrel{p.s.}{\longrightarrow} E(X_1)=(1-2p)
\end{equation}
et
\begin{equation}
    \frac{ S_n }{ n }\to(1-2p).
\end{equation}
Si \( p\neq 1/2\) nous pouvons conclure que
\begin{enumerate}
    \item
        si \( p>1/2\), alors \( S_n\stackrel{p.s.}{\longrightarrow}-\infty\)
    \item
        si \( p<1/2\), alors \( S_n\stackrel{p.s.}{\longrightarrow}\infty\).
\end{enumerate}
De plus nous connaissons la vitesse de divergence : elle est linéaire. Le mobile suit essentiellement l'équation
\( p(n)=(1-2p)n\).

\begin{remark}
    Cela ne traite pas le cas \( p=1/2\). Dans ce cas, nous pouvons simplement dire que \( S_n=o(n)\).
\end{remark}

%---------------------------------------------------------------------------------------------------------------------------
\subsection{Théorème central limite}
%---------------------------------------------------------------------------------------------------------------------------

\begin{lemma}       \label{Lemexpznznsurnton}
    Soit \( z_n\to z\) une suite convergente dans \( \eC\). Alors
    \begin{equation}
        \left( 1+\frac{ z_n }{ n } \right)^n\to e^z.
    \end{equation}
\end{lemma}

\begin{theorem}
    Si les variables aléatoires \( X_n\) sont
    \begin{enumerate}
        \item
            indépendantes et identiquement distribuées,
        \item
            \( X_1\in L^2(\Omega,\tribA)\),
    \end{enumerate}
    alors nous notons \( S_n=\sum_{i=1}^{n}X_i\), \( m=E(X_1)\) et \( \sigma^2=\Var(X_1)\) et nous avons
    \begin{equation}
        \frac{ S_n-nm }{ \sqrt{\sigma^2n} }\stackrel{\hL}{\longrightarrow}\dN(0,1).
    \end{equation}
    
\end{theorem}

\begin{proof}
    Nous allons écrire la démonstration dans le cas de variables aléatoires réelles. La proposition \ref{PrpopCaractCvL} dit que la suite \( X_n\) converge en loi vers \( X\) si et seulement si les fonctions caractéristiques convergent ponctuellement. Nous devons donc prouver, pour chaque\footnote{Chuck Norris peut \emph{vraiment} le faire pour \emph{chaque} $t\in\eR$} \( t\in\eR\) que
    \begin{equation}        \label{EqPhitophidNtznu}
        \Phi_{\frac{ S_n-nm }{ \sigma\sqrt{n} }}(t)\to\Phi_{\dN(0,1)}(t).
    \end{equation}
    Supposons dans un premier temps que \( E(X_i)=0\) et \( \sigma(X_i)=1\). Dans ce cas nous considérons la fonction 
    \begin{subequations}
        \begin{align}
            \Phi_{\frac{ S_n }{ \sqrt{n} }}&=E\big(  e^{i\frac{ t }{ \sqrt{n} }\sum_{k=1}^nX_k} \big)\\
            &=\prod_{k=1}^nE\big(  e^{i\frac{ t }{ \sqrt{n} }X_1} \big)\\
            &=\prod_{k=1}^n\Phi_{X_1}\left( \frac{ t }{ \sqrt{n} } \right)\\
            &=\Phi_{X_1}\left( \frac{ t }{ \sqrt{n} } \right)^n.
        \end{align}
    \end{subequations}
    Cette quantité est a priori complexe; nous ne pouvons donc pas immédiatement passer au logarithme. Nous pouvons par contre utiliser un développement en puissances de \( t\) en nous servant de la proposition \ref{PropDerFnCaract} et de l'hypothèse comme quoi \( X_1\in L^2\) :
    \begin{equation}
        \Phi_{X_1}(t)=\Phi_{X_1}(0)+\Phi_{X_1}'(0)t+\Phi_{X_1}''(0)\frac{ t^2 }{2}+\alpha(t)t^2
    \end{equation}
    où \( \alpha\) est une fonction qui a la propriété \( \lim_{x\to 0} \alpha(x)=0\).

    En utilisant les hypothèses et la formule de dérivation de la fonction caractéristique,
    \begin{subequations}
        \begin{align}
            \Phi_{X_1}(0)&=1\\
            \Phi'_{X_1}(0)&=E\big( (iX) \big)=0\\
            \Phi''_{X_1}(0)&=E(-X^2)=-\Var(X_1)=-1.
        \end{align}
    \end{subequations}
    Nous avons donc
    \begin{equation}
        \Phi_{X_1}\left( \frac{ t }{ \sqrt{n} } \right)=\underbrace{1-\frac{ 1 }{2}\frac{ t^2 }{ n }}_{\in\eR}+\underbrace{\frac{ t^2 }{ n }\alpha\left( \frac{ t }{ \sqrt{n} } \right)}_{\in\eC},
    \end{equation}
    de telle sorte que, en considérant une valeur fixée de \( t\),
    \begin{equation}        \label{EqPhifracfacbetanrigh}
        \Phi_{\frac{ S_n }{ \sqrt{n} }}(t)=\left( 1-\frac{ \frac{ t^2 }{ 2 }+\beta_n }{ n } \right)^n
    \end{equation}
    où \( \beta_n=t^2\alpha(t/\sqrt{n})\). Nous avons bien entendu \( \lim_{n\to \infty} \beta_n=0\).
    
    Nous pouvons appliquer le lemme \ref{Lemexpznznsurnton} pour obtenir la limite
    \begin{equation}        \label{EqlimninfySnsqrtntdsnd}
        \lim_{n\to \infty} \Phi_{\frac{ S_n }{ \sqrt{n} }}(t)= e^{-t^2/2}.
    \end{equation}
    La convergence \eqref{EqPhitophidNtznu} est par conséquent prouvée dans le cas où \( E(X_i)=0\) et $\Var(X_i)=1$.

    Considérons maintenant des variables aléatoires avec \( E(X_i)=m\) et \( \Var(X_i)=\sigma^2\). Elles peuvent être écrites sous la forme
    \begin{equation}
        X_i=\sigma X_i+m
    \end{equation}
    où \( X'_i\) est d'espérance nulle et de variance un. Nous avons alors
    \begin{equation}
        S_n=\sigma\sum_{i=1}^nX_i'+nm,
    \end{equation}
    et
    \begin{equation}
        \frac{ S_n-nm }{ \sigma\sqrt{n} }=\frac{ S'_n }{ \sqrt{n} }
    \end{equation}
    où \( S'_n=\sum_iX'_i\). L'étude de la variable aléatoire 
    \begin{equation}
        \frac{ S_n-nm }{ \sigma\sqrt{n} }
    \end{equation}
    revient donc à celle de \( S'_n/\sqrt{n}\) qui vient d'être effectuée.
\end{proof}

\begin{remark}
    Nous pouvons obtenir la limite \eqref{EqlimninfySnsqrtntdsnd} d'une façon alternative. Nous considérons la détermination du logarithme complexe sur \( \eC\setminus\eR^-\); cela est une fonction analytique vérifiant l'équation
    \begin{equation}
        e^{\ln(z)}=z
    \end{equation}
    pour tout \( z\in\eC\setminus\eR^-\) et le développement
    \begin{equation}
        \ln(1+z)=\sum_{n=1}^{\infty}(-1)^{n+1}\frac{ z^n }{ n }.
    \end{equation}
    En particulier, \( \ln(1+z)=z+z\alpha(z)\) où \( \lim_{z\to 0}\alpha(z)=0\). Nous reprenons à l'équation \eqref{EqPhifracfacbetanrigh} en fixant \( t\). Nous avons
    \begin{subequations}
        \begin{align}
            \Phi_{\frac{ S_n }{ \sqrt{n} }}(t)&=\exp\left[ \ln\big(\Phi_{\frac{ S_n }{ \sqrt{n} }}(t)\big) \right]\\
            &=\exp\left[ n\ln\left( 1+\frac{ -\frac{ t^2 }{2}-\beta_n }{ n } \right) \right]\\
            &=\exp\left[ -\frac{ t^2 }{2}-\beta_n+ \big( -\frac{ t^2 }{2}-\beta_n \big)\alpha\left( \frac{ -\frac{ t^2 }{2}-\beta_n }{ n } \right) \right].
        \end{align}
    \end{subequations}
    À la limite \( n\to 0\) nous tombons sur \(  e^{-t^2/2}\).
    
\end{remark}

\begin{remark}
    Étant donné que la variable aléatoire 
    \begin{equation}
        \frac{ S_n-nm }{ \sigma\sqrt{n} }
    \end{equation}
    converge en loi vers \( \dN(0,1)\), nous avons la convergence des fonctions de répartition partout où la fonction de répartition de la normale est continue (donc sur tout \( \eR\)). En particulier,
    \begin{equation}
        \left| P\left( \frac{ S_n-nm }{ \sigma\sqrt{n} }\leq x \right)-\int_{-\infty}^x\frac{1}{ \sqrt{2\pi} } e^{-y^2/2}dy \right| \to 0.
    \end{equation}
    Nous avons la borne de \defe{Berry-Esséen}{Berry-Esséen (borne)} qui donne une estimation de la vitesse de convergence: si \( X\in L^3\), alors  il existe une constante \( C\), indépendante de \( x\), des \( X_i\) et de \( n\) telle que
    \begin{equation}
        \left| P\left( \frac{ S_n-nm }{ \sigma\sqrt{n} }\leq x \right)-\int_{-\infty}^x\frac{1}{ \sqrt{2\pi} } e^{-y^2/2}dy \right| \leq\frac{ X\mu_3 }{ \sigma^3\sqrt{n} }
    \end{equation}
    où \( \mu_3=E\big( | X_1-m |^3 \big)\) est le moment d'ordre \( 3\) de \( X\). La chose à retenir est que la convergence est à la vitesse de \( 1/\sqrt{n}\).
\end{remark}

En dimension \( d>1\), nous avons encore un théorème central limite.
\begin{theorem}
    Si \( d>1\), et si nous avons des variables aléatoires \( X_n\) à valeurs dans \( \eR^d\) avec
    \begin{enumerate}
        \item
            les \( X_n\) sont indépendantes et identiquement distribuées
        \item
            les \( X_n\) sont dans \( L^2\).
    \end{enumerate}
    Alors nous notons \( X_1=(X_1^{(1)},\ldots,X_1^{(d)})\). Nous avons
    \begin{equation}
        \frac{ S_n-nm }{ \sqrt{n} }\stackrel{\hL}{\longrightarrow}\dN(0,\Sigma)
    \end{equation}
    où \( \Sigma\) est ma matrice de covariance du vecteur aléatoire \( X_1\) :
    \begin{equation}
        \Sigma=\big( \Cov(X_1^{(1),\ldots,X_1^{(d)}}) \big)_{i,j=1,\ldots,d}.
    \end{equation}
    
\end{theorem}



