% This is part of Mes notes de mathématique
% Copyright (c) 2011-2012
%   Laurent Claessens
% See the file fdl-1.3.txt for copying conditions.

Deux textes assez complets : \cite{MesIntProbb,MathAgreg}.



 \section{Les nombres complexes}
 \subsection{Définitions de base}
 Un nombre complexe s'écrit sous la forme $z = a + b i$, où $a$ et $b$
 sont des nombres réels appelés (et notés) respectivement partie réelle
 ($a = \Re(z)$) et partie imaginaire ($b = \Im(z)$) de $z$. L'ensemble
 des nombres de cette forme s'appelle l'ensemble des nombres complexes
 ; cet ensemble porte une structure de corps et est noté $\eC$. Le
 nombre complexe $i = 0 + 1 i$ est un nombre imaginaire qui a la
 particularité que $i^2 = -1$.

 Deux nombres complexes $a + bi$ et $c + di$ sont égaux si et seulement
 si $a = c$ et $b = d$, c'est-à-dire leurs parties réelles sont égales,
 et leurs parties imaginaires sont égales.

 Un nombre complexe étant représenté par deux nombres, on peut le
 représenter dans un plan appelé « plan de Gauss ». La plupart des
 opérations sur les nombres complexes ont leur interprétation
 géométrique dans ce plan.

 Pour $z = a + bi$ un nombre complexe, on note $\bar z = a - bi$ le
 \Defn{complexe conjugué} de $z$. Dans le plan de Gauss, il s'agit du
 symétrique de $z$ par rapport à la droite réelle (généralement
 dessinée horizontalement).

 On définit le module du complexe $z$ par $\module z = \sqrt{z\bar z} =
 \sqrt{a^2 + b^2}$. Dans le plan de Gauss, il s'agit de la distance
 entre $0$ et $z$.

 \begin{proposition}
Pour tout $z = a+bi$ et $z^\prime$ nombres complexes, on a
   \begin{enumerate}
   \item $z \bar z = a^2 + b^2$;
   \item $\bar{\bar{z}} = z$;
   \item $\module z = \module {\bar z}$;
   \item $\module{zz^\prime} = \module z \module{z^\prime}$;
   \item $\module{z+z^\prime} \leq \module z + \module{z^\prime}$.
   \end{enumerate}
 \end{proposition}

 \subsection{Forme polaire ou trigonométrique}
 Dans le plan de Gauss, le module d'un complexe $z$ représente la
 distance entre $0$ et $z$. On appelle \Defn{argument} de $z$ (noté
 $\arg z$) l'angle (déterminé à $2\pi$ près) entre le demi-axe des
 réels positifs et la demi-droite qui part de $0$ et passe par $z$. Le
 module et l'argument d'un complexe permettent de déterminer
 univoquement ce complexe puisqu'on a la formule
 \[z = a + bi = \module z \left( \cos(\arg(z)) + i \sin(\arg(z))
 \right)\]

 L'argument de $z$ se détermine via les formules
 \[\frac a {\module z} = \cos(\arg(z)) \quad \frac b {\module z} =
 \sin(\arg(z))\] ou encore par la formule
 \[\frac b a = \tan(\arg(z)) \quad \text{en vérifiant le
   quadrant.}\]%
 La vérification du quadrant vient de ce que la tangente ne détermine
 l'angle qu'à $\pi$ près.

	


%+++++++++++++++++++++++++++++++++++++++++++++++++++++++++++++++++++++++++++++++++++++++++++++++++++++++++++++++++++++++++++
					\section{Théorème de la fonction implicite}
%+++++++++++++++++++++++++++++++++++++++++++++++++++++++++++++++++++++++++++++++++++++++++++++++++++++++++++++++++++++++++++

%---------------------------------------------------------------------------------------------------------------------------
\subsection{Mise en situation}
%---------------------------------------------------------------------------------------------------------------------------

Dans un certain nombre de situation, il n'est pas possible de trouver des solutions explicites aux équations qui apparaissent. Néanmoins, l'existence «théorique» d'une telle solution est souvent déjà suffisante. C'est l'objet du théorème de la fonction implicite.

Prenons par exemple la fonction sur $\eR^2$ donnée par 
\begin{equation}
	F(x,y)=x^2+y^2-1.
\end{equation}
Nous pouvons bien entendu regarder l'ensemble des points donnés par $F(x,y)=0$. C'est le cercle dessiné à la figure \ref{LabelFigCercleImplicite}.
\newcommand{\CaptionFigCercleImplicite}{Un cercle pour montrer l'intérêt de la fonction implicite. Si on donne \( x\), nous ne pouvons pas savoir si nous parlons de \( P\) ou de \( P'\).}
\input{Fig_CercleImplicite.pstricks}

%\ref{LabelFigCercleImplicite}.
%\newcommand{\CaptionFigCercleImplicite}{Un cercle pour montrer l'intérêt de la fonction implicite.}
%\input{Fig_CercleImplicite.pstricks}

Nous ne pouvons pas donner le cercle sous la forme $y=y(x)$ à cause du $\pm$ qui arrive quand on prend la racine carrée. Mais si on se donne le point $P$, nous pouvons dire que \emph{autour de $P$}, le cercle est la fonction
\begin{equation}
	y(x)=\sqrt{1-x^2}.
\end{equation}
Tandis que autour du point $P'$, le cercle est la fonction
\begin{equation}
	y(x)=-\sqrt{1-x^2}.
\end{equation}
Autour de ces deux point, donc, le cercle est donné par une fonction. Il n'est par contre pas possible de donner le cercle autour du point $Q$ sous la forme d'une fonction.

Ce que nous voulons faire, en général, est de voir si l'ensemble des points tels que
\begin{equation}
	F(x_1,\ldots,x_n,y)=0
\end{equation}
peut être donné par une fonction $y=y(x_1,\ldots,x_n)$. En d'autre termes, est-ce qu'il existe une fonction $y(x_1,\ldots,x_n)$ telle que
\begin{equation}
	F\big( x_1,\ldots,x_n,y(x_1,\ldots,x_n)\big)=0.
\end{equation}



\subsection{Définitions et rappels}
Soit
\begin{equation}
    \begin{aligned}
        F\colon D\subset \eR^n\times \eR^m&\to \eR^m \\
        (x,y)&\mapsto \big( F_1(x,y),\ldots, F_m(x,y) \big) 
    \end{aligned}
\end{equation}
avec $x = (x_1,\ldots, x_n)$ et $y = (y_1,\ldots,y_m)$.

Pour chaque $x$ fixé, on s'intéresse aux solutions du système de $m$
 équations $F(x,y) = 0$ pour les inconnues $y$ ; en particulier, on
 voudrait pouvoir écrire $y = \varphi(x)$ vérifiant $F(x,\varphi(x)) = 0$.

Pour $(x,y) \in \interieur D$, la matrice
\begin{equation}
    \frac{ \partial (F_1,\ldots, F_m) }{ \partial (y_1,\ldots, y_m) }=
\begin{pmatrix}
\pder {F_1}{y_1}(x,y)& \ldots& \pder {F_1}{y_m}(x,y)\\
\vdots& \ddots & \vdots\\
\pder {F_m}{y_1}(x,y)& \ldots& \pder {F_m}{y_m}(x,y)\\
\end{pmatrix}
\end{equation}
est la \defe{matrice jacobienne}{jacobienne!matrice} de $F$ par rapport à $y$ au point
$(x,y)$; son déterminant est appelé le \defe{jacobien}{jacobien} de $F$ par
rapport à $y$.

\begin{theorem}[théorème de la fonction implicite] \index{théorème!fonction implicite} \label{ThoAcaWho}
    Soit une fonction \( F\colon \eR^n\times \eR^m\to \eR^m\) de classe \( C^k\) et \( (\alpha,\beta)\in \eR^n\times \eR^m\) tels que
    \begin{enumerate}
        \item
            \( f(\alpha,\beta)=0\),
        \item
            $\det\frac{ \partial (F_1,\ldots, F_m) }{ \partial (y_1,\ldots, y_m) }\neq 0$.
    \end{enumerate}
    Alors il existe un voisinage ouvert \( V\) de \( \alpha\) dans \( \eR^n\), un voisinage ouvert \( W\) de \( \beta\) dans \( \eR^m\) et une application \( \varphi\colon V\to W\) de classe \( C^k\)  telle que pour \( (x,y)\in V\times W\) on ait
    \begin{equation}
        F(x,y)=0
    \end{equation}
    si et seulement si \( y=\varphi(x)\).
\end{theorem}
	
Le théorème de la fonction implicite a pour objet de donner l'existence de la fonction $\varphi$. Maintenant nous pouvons dire beaucoup de choses sur les dérivées de $\varphi$ en considérant la fonction
\begin{equation}
	x\mapsto F\big( x,\varphi(x) \big).
\end{equation}
Par définition de $\varphi$, cette fonction est toujours nulle. En particulier, nous pouvons dériver l'équation
\begin{equation}
	F\big( x,\varphi(x) \big)=0,
\end{equation}
et nous trouvons plein de choses.

%---------------------------------------------------------------------------------------------------------------------------
\subsection{Exemple}
%---------------------------------------------------------------------------------------------------------------------------

Prenons par exemple la fonction
\begin{equation}
	F\big( (x,y),z \big)=ze^z-x-y,
\end{equation}
et demandons nous ce que nous pouvons dire sur la fonction $z(x,y)$ telle que
\begin{equation}
	F\big( x,y,z(x,y) \big)=0,
\end{equation}
c'est à dire telle que
\begin{equation}		\label{EqDefZImplExemple}
	z(x,y) e^{z(x,y)}-x-y=0.
\end{equation}
pour tout $x$ et $y\in\eR$. Nous pouvons facilement trouver $z(0,0)$ parce que
\begin{equation}
	z(0,0) e^{z(0,0)}=0,
\end{equation}
donc $z(0,0)=0$.

Nous pouvons dire des choses sur les dérivées de $z(x,y)$. Voyons par exemple $(\partial_xz)(x,y)$. Pour trouver cette dérivée, nous dérivons la relation \eqref{EqDefZImplExemple} par rapport à $x$. Ce que nous trouvons est
\begin{equation}
	(\partial_xz)e^z+ze^z(\partial_xz)-1=0.
\end{equation}
Cette équation peut être résolue par rapport à $\partial_xz$~:
\begin{equation}
	\frac{ \partial z }{ \partial x }(x,y)=\frac{1}{ e^z(1+z) }.
\end{equation}
Remarquez que cette équation ne donne pas tout à fait la dérivée de $z$ en fonction de $x$ et $y$, parce que $z$ apparaît dans l'expression, alors que $z$ est justement la fonction inconnue. En général, c'est la vie, nous ne pouvons pas faire mieux.

Dans certains cas, on peut aller plus loin. Par exemple, nous pouvons calculer cette dérivée au point $(x,y)=(0,0)$ parce que $z(0,0)$ est connu :
\begin{equation}
	\frac{ \partial z }{ \partial x }(0,0)=1.
\end{equation}
Cela est pratique pour calculer, par exemple, le développement en Taylor de $z$ autour de $(0,0)$.


%+++++++++++++++++++++++++++++++++++++++++++++++++++++++++++++++++++++++++++++++++++++++++++++++++++++++++++++++++++++++++++
                    \section{Maximisation sans contraintes}
%+++++++++++++++++++++++++++++++++++++++++++++++++++++++++++++++++++++++++++++++++++++++++++++++++++++++++++++++++++++++++++

%---------------------------------------------------------------------------------------------------------------------------
                    \subsection{Maximisation à une variable}
%---------------------------------------------------------------------------------------------------------------------------

\begin{definition}
Soit $f\colon A\subset \eR\to \eR$ et $a\in A$. Le point $a$ est un \defe{maximum local}{maximum!local} de $f$ si il existe un voisinage $\mU$ de $a$ tel que $f(a)\geq f(x)$ pour tout $x\in\mU\cap A$. Le point $a$ est un \defe{maximum global}{maximum!global} si $f(a)\geq g(x)$ pour tout $x\in A$.
\end{definition}

La proposition basique à utiliser lors de la recherche d'extrema est la suivante :
\begin{proposition}
Soit $f\colon A\subset\eR\to \eR$ et $a\in\Int(A)$. Supposons que $f$ est dérivable en $a$. Si $a$ est un \href{http://fr.wikipedia.org/wiki/Extremum}{extremum} local, alors $f'(a)=0$.
\end{proposition}

La réciproque n'est pas vraie, comme le montre l'exemple de la fonction $x\mapsto x^3$ en $x=0$ : sa dérivée est nulle et pourtant $x=0$ n'est ni un maximum ni un minimum local. 

Cette proposition ne sert donc qu'à sélectionner des \emph{candidats} extremum. Afin de savoir si ces candidats sont des extrema, il y a la proposition suivante.
\begin{proposition}
Soit $f\colon I\subset \eR\to \eR$, une fonction de classe $C^k$ au voisinage d'un point $a\in\Int I$. Supposons que
\begin{equation}
    f'(a)=f''(a)=\ldots=f^{(k-1)}(a)=0,
\end{equation}
et que
\begin{equation}
    f^{(k)}(a)\neq 0.
\end{equation}
Dans ce cas,
\begin{enumerate}

\item
Si $k$ est pair, alors $a$ est un point d'extremum local de $f$, c'est un minimum si $f^{(k)}(a)>0$, et un maximum si $f^{(k)}(a)<0$,
\item
Si $k$ est impair, alors $a$ n'est pas un extremum local de $f$.

\end{enumerate}
\end{proposition}

Note : jusqu'à présent nous n'avons rien dit des extrema \emph{globaux} de $f$. Il n'y a pas grand chose à en dire. Si un point d'extremum global est situé dans l'intérieur du domaine de $f$, alors il sera extremum local (a fortiori). Ou alors, le maximum global peut être sur le bord du domaine. C'est ce qui arrive à des fonctions strictement croissantes sur un domaine compact.

Une seule certitude : si une fonction est continue sur un compact, elle possède une minimum et un maximum global.

%---------------------------------------------------------------------------------------------------------------------------
                    \subsection{Maximisation à plusieurs variables}
%---------------------------------------------------------------------------------------------------------------------------

%---------------------------------------------------------------------------------------------------------------------------
                    \subsection{Quelque mots à propos de matrices}
%---------------------------------------------------------------------------------------------------------------------------

Les notions qui suivent seront vues au cours de géométrie lorsqu'il sera temps. 

Si $g$ est une application bilinéaire sur $\eR^2$, nous disons qu'elle est
\begin{enumerate}

\item
\defe{Définie positive}{Application!définie positive} si $g(u,u)\geq 0$ pour tout $u\in\eR^2$ et $g(u,u)=0$ si et seulement si $u=0$.

\item
\defe{semi-définie positive}{Application!semi-définie positive} si $g(u,u)\geq 0$ pour tout $u\in\eR^2$. 

\end{enumerate}

Une matrice $M$ est définie positive si $v^tMv>0$ pour tout $v\neq 0$, en particulier si $v$ est un vecteur propre de valeur propre $\lambda$ (c'est à dire si $Mv=\lambda v$), alors $\lambda v^tv>0$, et donc $\lambda>0$. Donc $M$ sera définie positive si toutes ses valeurs propres sont positives.

\begin{proposition}     \label{PropcnJyXZ}
    Soit $M$, une matrice $2\times 2$ symétrique\footnote{la matrice $d^2f(a)$ est toujours symétrique quand $f$ est de classe $C^2$.}. Nous avons
    \begin{enumerate}
        \item
        $\det M>0$ et $\tr(M)>0$ implique $M$ définie positive,
        \item
        $\det M>0$ et $\tr(M)<0$ implique $M$ définie négative,
    \item   \label{ItemluuFPN}
        $\det M<0$ implique ni semi définie positive, ni définie négative 
        \item
        $\det M=0$ implique $M$ semi-définie positive ou semi-définie négative.
    \end{enumerate}
\end{proposition}


 
%---------------------------------------------------------------------------------------------------------------------------
                    \subsection{Les théorèmes}
%---------------------------------------------------------------------------------------------------------------------------

Un point $a$ à l'intérieur du domaine d'une fonction $f\colon A\subset\eR^n\to \eR$ est une \defe{point critique}{Critique!point} de $f$ lorsque $df(a)=0$. Ces points sont analogues aux points où la dérivée d'une fonction sur $\eR$ s'annule. Les points critiques de $f$ sont dons les candidats à être des points d'extremum.

\begin{proposition}     \label{PropoExtreRn}
Soit $f\colon A\subset\eR^n\to \eR$ une fonction de classe $C^2$ au voisinage de $a\in\Int(A)$.
\begin{enumerate}

\item
Si $a$ est un point critique de $f$, et si $d^2f(a)$ est \href{http://fr.wikipedia.org/wiki/Matrice_définie_positive}{définie positive}, alors $a$ est un minimum local strict de $f$,
\item       \label{ItemPropoExtreRn}
Si $a$ est un minimum local, alors $a$ est un point critique et $d^2f(a)$ est semi-définie positive.

\end{enumerate}
\end{proposition}

La seconde partie de l'énoncé est tout à fait comparable au fait bien connu que, pour une fonction $f\colon \eR\to \eR$, si le point $a$ est minimum local, alors $f'(a)=0$ et $f''(a)\geq 0$. Notez le fait que l'inégalité n'est pas stricte, ce qui correspond à $d^2f(a)$ \emph{semi}-definie positive.

Pour rappel, dans le cas d'une fonction à deux variables, $d^2f(a)$ est la matrice (et donc l'application linéaire)
\begin{equation}
    d^2f(a)=\begin{pmatrix}
    \frac{ d^2f  }{ dx^2 }(a)   &   \frac{ d^2f  }{ dx\,dy }(a) \\ 
    \frac{ d^2f  }{ dy\,dx }(a)     &   \frac{ d^2f  }{ dy^2 }(a)
\end{pmatrix}.
\end{equation}
Dans le cas d'une fonction $C^2$, cette matrice est symétrique.

La méthode pour chercher les extrema de $f$ est donc de suivre le points suivants :
\begin{enumerate}
\item
Trouver les candidats extrema en résolvant $\nabla f=(0,0)$,
\item
écrire $d^2f(a)$ pour chacun des candidats
\item
calculer les valeurs propres de $d^2f(a)$, déterminer si la matrice est définie positive ou négative,
\item
conclure.
\end{enumerate}

Une conséquence du point \ref{ItemluuFPN} de la proposition \ref{PropcnJyXZ} est que si \( \det M<0\), alors le point \( a\) n'est pas  un extrema dans le cas où $M=d^2f(a)$ par le point \ref{ItemPropoExtreRn} de la proposition \ref{PropoExtreRn}.

%+++++++++++++++++++++++++++++++++++++++++++++++++++++++++++++++++++++++++++++++++++++++++++++++++++++++++++++++++++++++++++
\section{Variétés et extrema liés}
%+++++++++++++++++++++++++++++++++++++++++++++++++++++++++++++++++++++++++++++++++++++++++++++++++++++++++++++++++++++++++++

\subsection{Introduction}
Soit $f : S^2 \to \eR$ une fonction définie sur la sphère usuelle
$S^2 \subset \eR^3$. Une question naturelle est d'estimer la
régularité de $f$ ; est-elle continue, dérivable, différentiable ? Il
n'existe pas de dérivée directionnelle étant donné que le quotient
différentiel
\begin{equation*}
  \frac{f(x + \epsilon u_1 ,y + \epsilon u_2) - f(x,y)}{\epsilon}
\end{equation*}
n'a pas de sens pour un point $(x + \epsilon u_1 ,y + \epsilon u_2)$
qui n'est pas --sauf valeurs particulières-- dans la surface. Pour la
même raison il n'est pas possible de parler de différentiabilité de
cette manière. Comment faire, sans devoir étendre le domaine de
définition de $f$ à un voisinage de la sphère ? Une solution possible
est de parler de la notion de variété.

Une variété est un objet qui ressemble, vu de près, à $\eR^m$ pour un
certain $m$. En d'autres termes, on imagine une variété comme un
recollement de morceaux de $\eR^m$ vivant dans un espace plus grand
$\eR^n$. Ces morceaux sont appelés des ouverts de carte, et
l'application qui exprime la ressemblance à $\eR^m$ est l'application
de carte.

%---------------------------------------------------------------------------------------------------------------------------
\subsection{Définition et propriétés}
%---------------------------------------------------------------------------------------------------------------------------

\begin{definition}
  Soit $\emptyset \neq M \subset \eR^n$, $1 \leq m < n$ et $k \geq
  1$. $M$ est une \Defn{variété de classe $C^k$ de dimension $m$} si
  pour tout $a \in \var M$, il existe un voisinage ouvert $U$ de $a$
  dans $\eR^n$, et un ouvert $V$ de $\eR^m$ tel que $U \cap \var M$
  soit le graphe d'une fonction $f : V \subset \eR^m \to \eR^{n-m}$
  de classe $C^1$, c'est-à-dire qu'il existe un réagencement des
  coordonnées $(x_{i_1}, \ldots, x_{i_m}, x_{i_{m+1}}, \ldots,
  x_{i_n})$ avec
  \begin{equation*}
    M \cap U = \left\{ (x_1, \ldots, x_n) \in \eR^n \tq
%      \begin{array}{l} % deux conditions
      (x_{i_1}, \ldots, x_{i_m}) \in V \quad \left\{\begin{array}{c!{=}l} % 1: equations
        x_{i_{m+1}} & f_1(x_{i_1}, \ldots, x_{i_m})\\
        \vdots & \vdots \\
        x_{i_n} & f_{n-m}(x_{i_1}, \ldots, x_{i_m})
      \end{array}\right.
%    \end{array}
    \right\}
  \end{equation*}
  où $V$ est un voisinage ouvert de $(a_{i_1}, \ldots, a_{i_m}) \in \eR^m$.
\end{definition}

La littérature regorge de théorèmes qui proposent des conditions équivalentes à la définition d'une variété. Celle que nous allons le plus utiliser est la suivante% , de la page 268.
\begin{proposition}
	Soit $M\subset\eR^n$ et $1\leq m\leq n-1$. L'ensemble $M$ est une variété si et seulement si $\forall a\in M$, il existe un voisinage ouvert $\mU$ de $a$ dans $\eR^n$ et une application $F\colon W\subset\eR^m\to \eR^n$ où $W$ est un ouvert tels que
	\begin{enumerate}
		\item
			$F$ est un homéomorphisme de $W$ vers $M\cap\mU$,
		\item
			$F\in C^1(W,\eR^n)$,
		\item
			Le rang de $dF(w)\in L(\eR^m,\eR^n)$ est de rang maximum (c'est à dire $m$) en tout point $w\in W$.
	\end{enumerate}
\end{proposition}
Pour rappel, si $T\colon \eR^m\to \eR^n$ est une application linéaire, son \defe{rang}{rang} est la dimension de son image. On peut prouver que si $A$ est la matrice d'une application linéaire, alors le rang de cette application linéaire est égal à la taille de la plus grande matrice carré de déterminant non nul contenue dans $A$.

La condition de rang maximum sert à éviter le genre de cas de la figure \ref{LabelFigExempleNonRang} qui représente l'image de l'ouvert $\mathopen] -1 , 1 \mathclose[$ par l'application $F(t)=(t^2,t^3)$.
\newcommand{\CaptionFigExempleNonRang}{Quelque chose qui n'est pas de rang maximum et qui n'est pas une variété.}
\input{Fig_ExempleNonRang.pstricks}
%\ref{LabelFigExempleNonRang} 
%\newcommand{\CaptionFigExempleNonRang}{Quelque chose qui n'est pas de rang maximum et qui n'est pas une variété.}
%\input{Fig_ExempleNonRang.pstricks}
La différentielle a pour matrice
\begin{equation}
	dF(t)=(2t,3t^2).
\end{equation}
Le rang maximum est $1$, mais en $t=0$, la matrice vaut $(0,0)$ et son rang est zéro. Pour toute autre valeur de $t$, c'est bon.

Une autre caractérisation des variétés est donnée par la proposition suivante %(proposition 3, page 274).
\begin{proposition}		\label{PropCarVarZerFonc}
	Soit $M\in \eR^n$ et $1\leq m\leq n-1$. L'ensemble $M$ est une variété si et seulement si $\forall a\in M$, il existe un voisinage ouvert $\mU$ de $a$ dans $\eR^n$ tel et une application $G\in C^1(\mU,\eR^{n-m})$ tel que
	\begin{enumerate}

		\item
			le rang de $dG(a)\in L(\eR^n,\eR^{n-m})$ soit maximum (c'est à dire $n-m$) en tout $a\in M$,
		\item
			$M\cap\mU=\{ x\in\mU\tq G(x)=0 \}$.

	\end{enumerate}
\end{proposition}

%---------------------------------------------------------------------------------------------------------------------------
\subsection{Espace tangent}
%---------------------------------------------------------------------------------------------------------------------------

Soit $M$, une variété dans $\eR^n$, et considérons un chemin $\gamma\colon I\to \eR^n$ tel que $\gamma(t)\in M$ pour tout $t\in I$ et tel que $\gamma(0)=a$ et que $\gamma$ est dérivable en $0$. La \defe{tangente}{tangente à un chemin} au chemin $\gamma$ au point $a\in M$ est la droite
\begin{equation}
	s\mapsto a+s\gamma'(0).
\end{equation}
L'\defe{espace tangent}{Espace tangent} de $M$ au point $a$ est l'ensemble décrit par toutes les tangentes en $a$ pour tous les chemins $\gamma$ possibles.

\begin{proposition}			\label{PropDimEspTanVarConst}
	Une variété de dimension $m$ dans $\eR^n$ a un espace tangent de dimension $m$ en chacun de ses points.
\end{proposition}


%+++++++++++++++++++++++++++++++++++++++++++++++++++++++++++++++++++++++++++++++++++++++++++++++++++++++++++++++++++++++++++
\section{Extrema liés}
%+++++++++++++++++++++++++++++++++++++++++++++++++++++++++++++++++++++++++++++++++++++++++++++++++++++++++++++++++++++++++++

Soit $f$, une fonction sur $\eR^n$, et $M\subset \eR^n$ une variété de dimension $m$. Nous voulons savoir quelle sont les plus grandes et plus petites valeurs atteintes par $f$ sur $M$.

Pour ce faire, nous avons un théorème qui permet de trouver des extrema \emph{locaux} de $f$ sur la variété. Pour rappel, $a\in M$ est une \defe{extrema local de $f$ relativement}{Extrema local!relatif} à l'ensemble $M$ si il existe une boule $B(a,\epsilon)$ telle que $f(a)\leq f(x)$ pour tout $x\in B(a,\epsilon)\cap M$.

\begin{theorem}[Extrema lié] \label{ThoRGJosS}
	Soit \( A\), un ouvert de \( \eR^n\) et
	\begin{enumerate}
		\item
			une fonction (celle à minimiser) $f\in C^1(A,\eR)$,
		\item 
			des fonctions (les contraintes) $G_1,\ldots,G_r\in C^1(A,\eR)$,
		\item
			$M=\{ x\in A\tq G_i(x)=0\,\forall i\}$,
		\item
			un extrema local $a\in M$ de $f$ relativement à $M$.
	\end{enumerate}
	Supposons que les gradients $\nabla G_1(a)$, \ldots,$\nabla G_r(a)$ soient linéairement indépendants. Alors $a=(x_1,\ldots,x_n)$ est une solution de \( \nabla L(a)=0\) où
	\begin{equation}
		L(x_1,\ldots,x_n,\lambda_1,\ldots,\lambda_r)=f(x_1,\ldots,x_n)+\sum_{i=1}^r\lambda_iG_i(x_1,\ldots,x_n).
	\end{equation}
    Autrement dit, si \( a\) est un extrema lié, alors \( \nabla f(a)\) est une combinaisons des \( \nabla G_i(a)\).
\end{theorem}
La fonction $L$ est le \defe{lagrangien}{lagrangien} du problème et les variables \( \lambda_i\) sont les \defe{multiplicateurs de Lagrange}{multiplicateur!de Lagrange}.

La preuve provient de \cite{ytMOpe} dans lequel il est énoncé sous la forme \ref{PropfPPUxh}\footnote{Il y a donc adaptation des notations et donc sûrement des fautes de frappe; soyez prudent.}.
\begin{proof}
    Si \( r=n\) alors les vecteurs linéairement indépendantes \( \nabla G_i(a) \) forment une base de \( \eR^n\) et donc évidemment les \( \lambda_i\) existent. Nous supposons donc maintenant que \( r<n\). Nous notons \( (z_i)_{i=1\ldots n}\) les coordonnées sur \( \eR^n\).
    
    La matrice
    \begin{equation}
        \begin{pmatrix}
            \frac{ \partial G_1 }{ \partial z_1 }(a)    &   \cdots    &   \frac{ \partial G_1 }{ \partial z_n }(a)    \\
            \vdots    &   \ddots    &   \vdots    \\
            \frac{ \partial G_r }{ \partial z_1 }(a)    &   \cdots    &   \frac{ \partial G_r }{ \partial z_n }(a)
        \end{pmatrix}
    \end{equation}
    est de rang \( r\) parce que les lignes sont par hypothèses linéairement indépendantes. Nous nommons \( (y_i)_{i=1,\ldots, r}\) un choix de \( r\) parmi les \( (z_i)\) tels que
    \begin{equation}
        \det\begin{pmatrix}
            \frac{ \partial G_1 }{ \partial y_1 }    &   \ldots    &   \frac{ \partial G_1 }{ \partial y_r }    \\
            \vdots    &   \ddots    &   \vdots    \\
            \frac{ \partial G_r }{ \partial y_1 }    &   \ldots    &   \frac{ \partial G_r }{ \partial y_r }
        \end{pmatrix}\neq 0.
    \end{equation}
    Nous identifions \( \eR^n\) à \( \eR^s\times \eR^r\) dans lequel \( \eR^r\) est la partie générée par les \( (y_i)_{i=1,\ldots, r}\). Nous nommons \( (x_j)_{j=1,\ldots, s}\) les coordonnées sur \( \eR^s\). Autrement dit, les coordonnées sur \( \eR^n\) sont \( x_1,\ldots, x_s,y_1,\ldots, y_r\). Dans ces coordonnées, nous nommons \( a=(\alpha,\beta)\) avec \( \alpha\in \eR^s\) et \( \beta\in \eR^r\).

    Si nous notons \( G=(G_1,\ldots, G_r)\), le théorème de la fonction implicite (théorème \ref{ThoAcaWho})  nous dit qu'il existe un voisinage \( U'\) de \( \alpha\in \eR^n\), un voisinage \( V'\) de \( \beta\in \eR^r\) et une fonction \( \varphi\colon U'\to V'\) de classe \( C^1\) telle que si \( (x,y)\in U'\times V'\), alors
    \begin{equation}
        g(x,y)=0
    \end{equation}
    si et seulement si \( y=\varphi(x)\). Nous posons maintenant
    \begin{subequations}
        \begin{align}
            \psi(x)&=(x,\varphi(x))\\
            h(x)&=f\big( \psi(x) \big).
        \end{align}
    \end{subequations}
    Nous avons \( \psi(\alpha)=a\) et \( \psi(x)\in M\) pour tout \( x\in U'\). La fonction \( h\) a donc un extrema local en \( \alpha\) et donc les dérivées partielles de \( h\) y sont nulles. Cela signifie que
    \begin{equation}
        0=\frac{ \partial h }{ \partial x_i }(\alpha)=\sum_{j=1}^n\frac{ \partial f }{ \partial x_j }\frac{ \partial x_j }{ \partial x_i }+\sum_{k=1}^r\frac{ \partial f }{ \partial y_k }\frac{ \partial \varphi_k }{ \partial x_i },
    \end{equation}
    c'est à dire
    \begin{equation}
        \frac{ \partial f }{ \partial x_i }(\alpha)+\sum_{k=1}^r\frac{ \partial f }{ \partial y_k }(a)\frac{ \partial \varphi_k }{ \partial x_i }(\alpha)=0
    \end{equation}
    pour tout \( i=1,\ldots, s\). D'autre part pour tout $k$, la fonction \( l_k(x)=G_k\big( x,\varphi(x) \big)\) est constante et vaut zéro; ses dérivées partielles sont donc nulles :
    \begin{equation}
        \frac{ \partial l }{ \partial x_i }(\alpha)=\frac{ \partial G_k }{ \partial x_i }(\alpha)+\sum_{k=1}^r\frac{ \partial G_k }{ \partial y_k }(a)\frac{ \partial \varphi_k }{ \partial x_i }(\alpha)=0
    \end{equation}
    pour tout \( i=1,\ldots, s\) et \( k=1,\ldots, r\).
    
    Les \( s\) premières colonnes de la matrice
    \begin{equation}
        \begin{pmatrix}
            \frac{ \partial f }{ \partial x_1 }   &   \cdots    &   \frac{ \partial f }{ \partial x_s }    &   \frac{ \partial f }{ \partial y_1 }    &   \cdots    &   \frac{ \partial f }{ \partial y_r }\\  
            \frac{ \partial G_1 }{ \partial x_1 }    &   \cdots    &   \frac{ \partial G_1 }{ \partial x_s }    &   \frac{ \partial G_1 }{ \partial y_1 }    &   \cdots    &   \frac{ \partial G_1 }{ \partial y_r }\\
            \vdots    &   \vdots    &   \vdots    &   \vdots    &   \vdots    &   \vdots\\
            \frac{ \partial G_r }{ \partial x_1 }    &   \cdots    &   \frac{ \partial G_r }{ \partial x_s }    &   \frac{ \partial G_r }{ \partial y_1 }    &  \cdots   & \frac{ \partial G_r }{ \partial y_r }  
        \end{pmatrix}
    \end{equation}
    s'expriment en terme des \( r\) dernières. La matrice est donc au maximum de rang \( r\). Notons que la première ligne est \( \nabla f\) et les \( r\) suivantes sont les \( \nabla G_i\). Vu que ces lignes sont des vecteurs liés, il existe \( \mu_0,\ldots, \mu_r\) tels que
    \begin{equation}
        \mu_0\nabla f+\sum_{i=1}^r\mu_i\nabla G_i=0.
    \end{equation}
    Par hypothèse les \( \nabla G_i\) sont linéairement indépendants, ce qui nous dit que \( \mu_0\neq 0\). Donc nous avons ce qu'il nous faut :
    \begin{equation}
        \nabla f(a)=\sum_i\frac{ \mu_i }{ \mu_O } \nabla G_i(a).
    \end{equation}
\end{proof}

La proposition suivante est la même que \ref{ThoRGJosS}.
\begin{proposition} \label{PropfPPUxh}
    Soit \( U\), un ouvert de \( \eR^n\) et des fonctions de classe \( C^1\) \( f,g_1,\ldots, g_r\colon U\to \eR\). Nous considérons
    \begin{equation}
        \Gamma=\{ x\in U\tq g_1(x)=\ldots=g_r(x)=0 \}.
    \end{equation}
    Soit \( a\) un extrémum de \( f|_{\Gamma}\). Supposons que les formes \( dg_1,\ldots, dg_r\) soient linéairement indépendantes en \( a\). Alors il existe \( \lambda_1,\ldots, \lambda_r\) dans \( \eR\) tel que
    \begin{equation}
        df_a=\sum_{i=1}^r\lambda_i(dg_i)_a.
    \end{equation}
\end{proposition}

\href{http://www.sagenb.org/home/pub/353/}{En pratique}, les candidats extrema locaux sont tous les points où les gradients ne sont pas linéairement indépendants, plus tous les points donnés par l'équation $\nabla L=0$. Parmi ces candidats, il faut trouver lesquels sont maxima ou minima, locaux ou globaux.

L'existence d'extrema locaux se prouve généralement en invoquant de la compacité, et en invoquant le lemme suivant qui permet de réduire le problème à un compact.

\begin{lemma}		\label{LemmeMinSCimpliqueS}
	Soit $S$, un ensemble dans $\eR^n$ et $C$, un ouvert de $\eR^n$. Si $a\in\Int S$ est un minimum local relatif à $S\cap C$, alors il est un minimum local par rapport à $S$.
\end{lemma}

\begin{proof}
	Nous avons que $\forall x\in B(a,\epsilon_1)\cap S\cap C$, $f(x)\geq f(x)$. Mais étant donné que $C$ est ouvert, et que $a\in C$, il existe un $\epsilon_2$ tel que $B(a,\epsilon_2)\subset C$. En prenant $\epsilon=\min\{ \epsilon_1,\epsilon_2 \}$, nous trouvons que $f(x)\geq f(a)$ pour tout $x\in B(a,\epsilon)\cap(S\cap C)=B(a,\epsilon)\cap S$.
\end{proof}
