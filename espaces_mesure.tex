% This is part of Mes notes de mathématique
% Copyright (c) 2011-2012
%   Laurent Claessens
% See the file fdl-1.3.txt for copying conditions.


\section{Séries}
\label{secseries}

\subsection{Rappels et définitions}
La notion de série formalise le concept de somme infinie. L'absence de
certaines propriétés de ces objets (problèmes de commutativité et même
d'associativité) incitent à la prudence et montrent à quel point une
définition précise est importante.

Soit ${(a_k)}_{k \geq k_0}$ une suite réelle. La \defe{série} de terme
général $(a_k)${série}, notée
\begin{math}
  \sum_{i=k_0}^\infty a_i,
\end{math}
est la suite ${(s_n)}_{n \geq k_0}$ dont les termes sont donnés par
\begin{equation*}
  s_n \pardef \sum_{i=k_0}^k a_i
\end{equation*}
et sont appelés les \defe{sommes partielles}{somme!partielle} de la série.

La série $\sum_{i=k_0}^\infty a_i$ \defe{converge}{série!convergence} si la suite $(s_n)$
converge vers un réel $s$. Sa limite est appelée la \defe{somme de la
série}{série!somme} et on note
\begin{equation}		\label{EqDefSommeLim}
  \sum_{i=k_0}^\infty a_i = \lim_{n\to\infty}\sum_{i=k_0}^na_i.
\end{equation}
Si la série ne converge pas, elle \defe{diverge}{série!divergence} et peut alors avoir
une limite infinie (uniquement si le terme général est réel, on note
alors $+\infty$ ou $-\infty$ sa limite) ou pas de limite.  La série
$\sum_{i=k_0}^\infty a_i$ \defe{converge absolument}{convergence!absolue} si la série
$\sum_{i=k_0}^\infty \abs{a_i}$ converge.

\begin{proposition}\label{propnseries_propdebase}
Les principales propriétés de la somme définie par la limite \eqref{EqDefSommeLim} sont
  \begin{enumerate}
  \item Si une série converge absolument, alors elle converge
    (simplement).
  \item Si la série est à termes positifs --c'est-à-dire pour tout
    indice $k$, $a_k \in \eR$ et $a_k \geq 0$-- il n'y a aucune
    différence entre convergence absolue et convergence simple.
  \item\label{point3-seriepropdebase} Si une série converge, son terme général doit tendre vers $0$.
\item 
Si la série converge alors la somme est associative
\item
Si la série converge absolument, alors la somme est commutative.
  \end{enumerate}
\end{proposition}

\begin{remark}Vue comme somme infinie, l'associativité et la
  commutativité dans une série sont perdues. Néanmoins, il subsiste
  que
  \begin{enumerate}
  \item si la série converge, on peut regrouper ses termes sans
    modifier la convergence ni la somme (associativité),
  \item si la série converge absolument, on peut modifier l'ordre des
    termes sans modifier la convergence ni la somme (commutativité).
  \end{enumerate}
\end{remark}

\begin{example}\label{exemplesseries}
\begin{enumerate}

\item
    La \defe{série harmonique}{série!harmonique} est
\begin{equation}
\sum_{i=1}^\infty \frac1i
\end{equation}
et diverge (possède une limite $+\infty$).

\item
    La \defe{série géométrique}{série!géométrique} de raison $q \in \eC$ est
\begin{equation}
\sum_{i=0}^\infty q^i.
\end{equation}
Étudions la somme partielle \( S_N=1+q+q^2+\ldots +q^{n}\). Nous avons évidemment $S_N-zS_N=1-q^{N+1}$ et donc
\begin{equation}
    S_N=\sum_{n=0}^Nq^n=\frac{ 1-q^{N+1} }{ 1-q }.
\end{equation}
La limite \( \lim_{N\to \infty} S_N\) existe si et seulement si \( | q |\leq 1\) et dans ce cas nous avons
\begin{equation}
    \sum_{n=1}^{\infty}z^n=\frac{ 1 }{ 1-z }.
\end{equation}
La convergence est absolue.

\item
Pour $\alpha \in \RR$, la série de Riemann (ou Dirichlet)
\begin{equation}		\label{EqSerRiem}
\sum_{i=1}^\infty \frac1{i^\alpha}
\end{equation}
converge (absolument, puisque réelle et positive) si et seulement
si $\alpha > 1$, et diverge sinon.
\end{enumerate}
\end{example}

\subsection{Critères de convergence absolue}

  Étant donné le terme général d'une série, il est souvent --dans les cas qui nous intéressent-- difficile de déterminer la somme de la série. L'exemple de la série géométrique est particulier, puisqu'on connait une formule pour chaque somme partielle, mais pour l'exemple des séries de Riemann il n'y a aucune formule simple pour un $\alpha$ général. D'où l'intérêt d'avoir des critères de convergence ne nécessitant aucune connaissance de l'éventuelle limite de la série.

\subsubsection{Critère de comparaison} 

Soient $\sum_i a_i$ et $\sum_j
b_j$ deux séries à termes positifs vérifiant
\begin{equation*}
  0 \leq a_i \leq b_i
\end{equation*}
alors
\begin{enumerate}
\item si $\sum_i a_i$ diverge, alors $\sum_j b_j$ diverge,
\item si $\sum_j b_j$ converge, alors $\sum_i a_i$ converge
  (absolument).
\end{enumerate}

\subsubsection{Critère d'équivalence}
\label{PgCritEquiv}

\begin{proposition}[\cite{TrenchRealAnalisys}]
 Soient $\sum_i a_i$ et $\sum_j b_j$ deux séries à termes positifs. Supposons l'existence de la limite (éventuellement infinie) suivante
\begin{equation}
  \limite i \infty \frac{a_i}{b_i} = \alpha \in \RR \text{ ou $\alpha =
    \infty$.}
\end{equation}
Dans ce cas, nous avons
\begin{enumerate}
\item si $\alpha \neq 0$ et $\alpha\neq \infty$, alors
  \begin{equation}
    \sum_i a_i \text{~converge} \ssi \sum_j b_j\text{~converge,}
  \end{equation}
\item si $\alpha = 0$ et $\sum_j b_j$ converge, alors $\sum_i a_i$
  converge (absolument),
\item si $\alpha = +\infty$ et $\sum_j b_j$ diverge, alors $\sum_i
  a_i$ diverge.
\end{enumerate}
\end{proposition}

\begin{proof}
\begin{enumerate}
	\item
		Le fait que la suite $a_n/b_n$ converge vers $\alpha$ signifie que tant sa limite supérieure que sa limite inférieure convergent vers $\alpha$. En particulier la suite $\frac{ a_n }{ b_n }$ est bornée vers le haut et vers le bas. À partir d'un certain rang $N$, il existe $M$ tel que 
		\begin{equation}
			\frac{ a_n }{ b_n }<M
		\end{equation}
		et il existe $m$ tel que
		\begin{equation}
			\frac{ a_n }{ b_n }>m.
		\end{equation}
		Nous avons donc $a_n<Mb_n$ et $a_n>mb_n$. La série de $(a_n)$ converge donc si et seulement si la série de $(b_n)$ converge.
	\item
		Si $\alpha=0$, cela signifie que pour tout $\epsilon$, il existe un rang tel que $\frac{ a_n }{ b_n }<\epsilon$, et donc tel que $a_n<\epsilon b_k$. La suite de $(a_i)$ converge donc dès que la suite de $(b_i)$ converge.
	\item
		Pour tout $M$, il existe un rang dans la suite à partir duquel on a $\frac{ a_i }{ b_i }>M$, et donc $a_k>Mb_k$. Si la série de $(b_k)$ diverge, la série de $(a_k)$ doit également diverger.
\end{enumerate}
\end{proof}

\subsubsection{Critère du quotient}

\begin{proposition}[\cite{KeislerElemCalculus}]
    Soit $\sum_i a_i$ une série. Supposons l'existence de la limite (éventuellement infinie) suivante
    \begin{equation}
      \limite i \infty \abs{\frac{a_{i+1}}{a_i}} = L\in \RR \text{ ou $L =
        \infty$.}
    \end{equation}
    Alors
    \begin{enumerate}
    \item si $L < 1$, la série converge absolument,
    \item si $L > 1$, la série diverge,
    \item si $L = 1$ le critère échoue : il existe des exemple de convergence et des exemples de divergence.
    \end{enumerate}
\end{proposition}

\begin{proof}
\begin{enumerate}
	\item
		Soit $b$ tel que $L<b<1$. À partir d'un certain rang $K$, on a $\left| \frac{ a_{i+1} }{ a_i } \right| <b$. En particulier,
		\begin{equation}
			| a_{K+1} |<b| a_K |,
		\end{equation}
		et pour $a_{K+2}$ nous avons
		\begin{equation}
			| a_{K+2} |<b| a_{K+1} |<b^2| a_K |.
		\end{equation}
		Au final,
		\begin{equation}
			| a_{K+n} |<b^n| a_K |.
		\end{equation}
		Étant donné que la série $\sum_{n\geq K}b^n$ converge (parce que $b<1$), la queue de suite $\sum_{i\geq K}a_i$ converge, et par conséquent la suite au complet converge.
	\item
		Si $L>1$, on a
		\begin{equation}
			| a_K |<| a_{K+1} |<| a_{K+2} |<\ldots
		\end{equation}
		Il est donc impossible que la suite $(a_i)$ converge vers zéro. La série ne peut donc pas converger.
	\item
		Par exemple la suite harmonique $a_n=\frac{1}{ n }$ vérifie $L=1$, mais la série ne converge pas. Par contre, la suite $a_n=\frac{ 1 }{ n^2 }$ vérifie aussi le critère avec $L=1$ tandis que la série $\sum_n\frac{1}{ n^2 }$ converge.
\end{enumerate}
\end{proof}

\subsubsection{Critère de la racine}

\begin{proposition}[\cite{TrenchRealAnalisys}]
    Soit $\sum_i a_i$ une série, et considérons
    \begin{equation*}
      \limsup_{i \rightarrow \infty} \sqrt[i]{\abs{a_i}} = L \in \RR
      \text{ ou $L =
        \infty$.}
    \end{equation*}
    Alors
    \begin{enumerate}
    \item si $L < 1$, la série converge absolument,
    \item si $L> 1$, la série diverge,
    \item si $L = 1$ le critère échoue.
    \end{enumerate}
\end{proposition}

\begin{proof}
    \begin{enumerate}
        \item
            Si $L<1$, il existe un $r\in \mathopen] 0 , 1 \mathclose[$ tel que $| a_n |^{1/n}<r$ pour les grands $n$. Dans ce cas, $| a_n |<r^{n}$, et la série converge absolument parce que la série $\sum_nr^n$ converge du fait que $r<1$.
        \item
            Si $L>1$, il existe un $r>1$ tel que $| a_n |^{1/n}>r>1$. Cela fait que $| a_n |$ prend des valeurs plus grandes que $n$ pour une infinité de termes. Le terme général $a_n$ ne peut donc pas être une suite convergente. Par conséquent la suite diverge au sens où elle ne converge pas.

    \end{enumerate}
\end{proof}

%---------------------------------------------------------------------------------------------------------------------------
\subsection{Critères de convergence simple}
%---------------------------------------------------------------------------------------------------------------------------

Les critères de comparaison, d'équivalence, du quotient et de la racine sont des critères de convergence absolue. Pour conclure à une convergence simple qui n'est pas une convergence absolue, le critère d'Abel sera notre outil principal.  

\subsubsection{Critère d'Abel}

\begin{proposition}[Critère d'Abel]
    Soit la série $\sum_i c_iz_i$ avec
    \begin{enumerate}
        \item $(c_i)$ est une suite réelle décroissante qui tend vers zéro,
        \item $(z_i)$ est une suite dans $\eC$ dont la suite des sommes partielles est bornée dans $\eC$, c'est à dire qu'il existe un $M>0$ tel que pour tout $n$,
        \begin{equation}
            \left| \sum_{i=1}^nz_i \right| \leq M.
        \end{equation}
        Alors la série $\sum_ic_iz_i$ est convergente.
    \end{enumerate}
\end{proposition}
Remarquons que ce critère ne donne pas de convergence absolue.

\begin{corollary}[Critère des séries alternées]\index{critère!série alternée}       \label{CoreMjIfw}
    Si \( (a_n)\) est une suite décroissante à limite nulle, alors la série
  \begin{equation}
    \sum_{n=0}^\infty {(-1)}^n a_n
  \end{equation}
  converge simplement.
\end{corollary}

%+++++++++++++++++++++++++++++++++++++++++++++++++++++++++++++++++++++++++++++++++++++++++++++++++++++++++++++++++++++++++++
					\section{Développements de Taylor et Maclaurin}
%+++++++++++++++++++++++++++++++++++++++++++++++++++++++++++++++++++++++++++++++++++++++++++++++++++++++++++++++++++++++++++

%---------------------------------------------------------------------------------------------------------------------------
\subsection{Fonctions «petit o» }
%---------------------------------------------------------------------------------------------------------------------------

Nous voulons formaliser l'idée d'une fonction qui tend vers zéro \og plus vite\fg{} qu'une autre. Nous disons que $f\in o\big(\varphi(x)\big)$ si
\begin{equation}
	\lim_{x\to 0} \frac{ f(x) }{ \varphi(x) }=0.
\end{equation}
En particulier, nous disons que $f\in o(x)$ lorsque $\lim_{x\to 0} f(x)/x=0$.

%---------------------------------------------------------------------------------------------------------------------------
\subsection{Formule et reste}
%---------------------------------------------------------------------------------------------------------------------------

\begin{proposition}		\label{PropDevTaylorPol}
    Soient $f\colon I\subset\eR\to \eR$ et $a\in\Int(I)$. Soit un entier $k\geq 1$. Si $f$ est $k$ fois dérivable en $a$, alors il existe un et un seul polynôme $P$ de degré $\leq k$ tel que
    \begin{equation}
        f(x)-P(x-a)\in o\big( | x-a |^k \big)
    \end{equation}
    lorsque $x\to a$, $x\neq a$. Ce polynôme  est donné par
    \begin{equation}
        P(h)=f(a)+f'(a)h+\frac{ f''(a) }{ 2! }h^2+\ldots+\frac{ f^{(k)}(a) }{ k! }h^k.
    \end{equation}
    Notons encore deux façons alternatives d'écrire le résultat. Si \( f\in C^k\) il existe une fonction \( \alpha\) telle que \( \lim_{t\to 0} \alpha(t)=0\) et
    \begin{equation}
        f(x)=\sum_{n=0}^k\frac{ f^{(n)}(a) }{ n! }(x-a)^n+(x-a)^n\alpha(x-a).
    \end{equation}
    Si \( f\in C^{k+1}\) alors
    \begin{equation}        \label{EquQtpoN}
        f(x)=\sum_{n=0}^k\frac{ f^{(n)}(a) }{ n! }(x-a)^n+(x-a)^{n+1}\xi(x-a)
    \end{equation}
    où \( \xi\) est une fonction telle que \( \xi(t)\) tend vers une constante lorsque \( t\to 0\).
\end{proposition}

La proposition suivant donne une intéressante façon de trouver le reste d'un développement de Taylor.
\begin{proposition}		\label{PropResteTaylorc}
Soient $I$, un intervalle dans $\eR$ et $f\colon I\to \eR$ une fonction de classe $C^k$ sur $I$ telle que $f^{(k+1)}$ existe sur $I$. Soient $a\in\Int(I)$ et $x\in I$. Alors il existe $c$ strictement compris entre $x$ et $a$ tel que 
\begin{equation}
	R_{f,a,k}(x)=\frac{ f^{(k+1)}(c) }{ (k+1)! }(x-a)^{k+1}.
\end{equation}
\end{proposition}

%---------------------------------------------------------------------------------------------------------------------------
					\subsection{Exemple : un calcul heuristique de limite}
%---------------------------------------------------------------------------------------------------------------------------
\label{SubSecCalcLimHeuris}

Au cours de la résolution de l'exercice \ref{exoEqsDiff0002}\ref{ItemfEqsDiff00002}, nous devons calculer la limite suivante :
\begin{equation}
	\lim_{x\to 0} \frac{  e^{-2\cos(x)+2}\sin(x) }{ \sqrt{ e^{2\cos(x)+2}}-1 }.
\end{equation}
La stratégie que nous allons suivre pour calculer cette limite est de développer certaines parties de l'expression en série de Taylor, afin de simplifier l'expression. La première chose à faire est de remplacer $ e^{y(x)}$ par $1+y(x)$ lorsque $y(x)\to 0$. La limite devient
\begin{equation}
	\lim_{x\to 0} \frac{ \big( -2\cos(x)+3 \big)\sin(x) }{ \sqrt{-2\cos(x)+2} }.
\end{equation}
Nous allons maintenant remplacer $\cos(x)$ par $1$ au numérateur et par $1-x^2/2$ au dénominateur. Pourquoi ? Parce que le cosinus du dénominateur est dans une racine, donc nous nous attendons à ce que le terme de degré deux du cosinus donne un degré un en dehors de la racine, alors que du degré un est exactement ce que nous avons au numérateur : le développement du sinus commence par $x$.

Nous calculons donc
\begin{equation}
	\begin{aligned}[]
		\lim_{x\to 0} \frac{ \sin(x) }{ \sqrt{-2\left( 1-\frac{ x^2 }{ 2 } \right)+2} }=\lim_{x\to 0} \frac{ \sin(x) }{ x }=1.
	\end{aligned}
\end{equation}
Tout ceci n'est évidement pas très rigoureux, mais en principe vous avez tous les éléments en main pour justifier les étapes.

%+++++++++++++++++++++++++++++++++++++++++++++++++++++++++++++++++++++++++++++++++++++++++++++++++++++++++++++++++++++++++++
\section{Théorie de la mesure}
%+++++++++++++++++++++++++++++++++++++++++++++++++++++++++++++++++++++++++++++++++++++++++++++++++++++++++++++++++++++++++++

Pour la théorie de la mesure, voir entre autres \cite{FubiniBMauray,ProbaDanielLi}.

%---------------------------------------------------------------------------------------------------------------------------
\subsection{Espaces mesurables et mesurés}
%---------------------------------------------------------------------------------------------------------------------------

\begin{definition}
    Si \( \Omega\) est un ensemble, un ensemble \( \tribA\) de sous-ensembles de \( \Omega\) est une \defe{tribu}{tribu} si
    \begin{enumerate}
        \item
            \( \Omega\in\tribA\);
        \item
            \( \complement A\in A\) pour tout \( A\in\tribA\);
        \item
            si \( (A_i)_{i\in I}\) est un ensemble au plus dénombrable d'éléments de \( \tribA\), alors \( \sup_{n\geq 1}A_n=\bigcup_{i\in I}A_i\in\tribA\).
    \end{enumerate}
    Le couple \( (\Omega,\tribA)\) est alors un \defe{espace mesuré}{espace!mesuré}.
\end{definition}

\begin{lemma}
    Une tribu est stable par intersections au plus dénombrables.
\end{lemma}

\begin{proof}
    Soit \( (A_i)_{i\in I}\) une famille au plus dénombrable d'éléments de la tribu \( \tribA\). Nous devons prouver que \( \bigcap_{i\in I}A_i\) est également un élément de \( \tribA\). Pour cela nous passons au complémentaire :
    \begin{equation}
        \complement\left( \bigcap_{i\in I}A_i \right)=\bigcup_{i\in I}\complement A_i.
    \end{equation}
    La définition d'une tribu implique que le membre de droite est un élément de la tribu. Par stabilité d'une tribu par complémentaire, l'ensemble \( \bigcap_{i\in I}A_i\) est également un élément de la tribu.
\end{proof}

La tribu que nous utiliserons toujours dans \( \eR^d\) est la tribu des \defe{boréliens}{boréliens}, notée \( \Borelien(\eR^d)\), qui est la tribu engendrée par les ouverts de \( \eR^d\). Une fonction \( f\colon (\Omega,\tribA)\to (\eR^d,\Borelien(\eR^d))\) est \defe{borélienne}{borélienne} si pour tout \( \mO\in\Borelien\), \( f^{-1}(\mO)\in\tribA\).

\begin{definition}
    Une \defe{\wikipedia{en}{Measure_space}{mesure}}{mesure} sur l'espace mesurable \( (\Omega,\tribA)\) est une application \( \mu\colon \tribA\to \eR\cup\{ \infty \}\) telle que
    \begin{enumerate}
        \item
            \( \mu(A)\geq 0\) pour tout \( A\in\tribA\);
        \item
            \( \mu(\emptyset)=0\);
        \item
            \( \mu\left( \bigcup_{i=0}^{\infty}A_i\right)=\sum_{i=0}^{\infty}\mu(A_i)\) si les \( A_i\) sont des éléments de \( \tribA\) deux à deux disjoints.
    \end{enumerate}
    Une mesure est \defe{\( \sigma\)-finie}{mesure!$\sigma$-finie} si il existe une suite croissante (pour l'inclusion) d'éléments \( (E_n)_{n\in\eN}\) de la tribu, tous de mesure finie et tels que \( \Omega=\bigcup_{n\in \eN}E_n\). Si la mesure est $\sigma$-finie, nous disons que l'espace \( (\Omega,\tribA,\mu)\) est un espace mesuré $\sigma$-fini.
\end{definition}

\begin{definition}
    Une application entre espace mesurés
    \begin{equation}
        f\colon (\Omega,\tribA)\to (\Omega',\tribA')
    \end{equation}
    est \defe{mesurable}{mesurable!application} si pour tout \( B\in\tribA'\), l'ensemble \( f^{-1}(B)\) est dans \( \tribA\).
\end{definition}

Si \( \mu\) est une mesure sur \( \eR^d\), une fonction \( f\colon \eR^d\to \eR\) est une \defe{densité}{densité d'une mesure} si pour tout \( A\subset\eR^d\) nous avons
\begin{equation}
    \mu(A)=\int_Af(x)dx
\end{equation}
où \( dx\) est la mesure de Lebesgue.

%---------------------------------------------------------------------------------------------------------------------------
\subsection{Mesure produit}
%---------------------------------------------------------------------------------------------------------------------------

\begin{definition}      \label{DefTribProfGfYTuR}
    Si \( \tribA\) et \( \tribB\) sont deux tribus sur deux ensembles \( \Omega_1\) et \( \Omega_2\), nous définissons la \defe{tribu produit}{tribu!produit} \( \tribA\otimes\tribB\) comme étant la tribu engendrée par 
    \begin{equation}
        \{ A\times B\tq A\in\tribA,B\in\tribB \}.
    \end{equation}
\end{definition}

\begin{theorem}\index{mesure!produit}
    Soient \( \mu_i\) des mesures $\sigma$-finies sur \( (\Omega_i,\tribA_i)\) (\( i=1,2\)). Il existe une et une seule mesure, notée \( \mu_1\otimes \mu_2\), sur \( (\Omega_1\times\Omega_2,\tribA_1\otimes\tribA_2)\) telle que
    \begin{equation}
        (\mu_1\otimes\mu_2)(A_1\times A_2)=\mu_1(A_1)\mu_2(A_2)
    \end{equation}
    pour tout \( A_1\in \tribA_1\) et \( A_2\in\tribA_2\).
\end{theorem}
Une preuve peut être trouvée dans \cite{FubiniBMauray}.

%---------------------------------------------------------------------------------------------------------------------------
\subsection{Intégrale par rapport à une mesure}
%---------------------------------------------------------------------------------------------------------------------------

Une fonction \( f\colon (\Omega,\tribA)\to (\Omega',\tribA')\) est \defe{mesurable}{mesurable!fonction} si 
\begin{equation}
    f^{-1}(E)\in\tribA
\end{equation}
pour tout \( E\in\tribA'\).


Une mesure \( \mu\) sur un espace mesurable \( (\Omega,\tribA)\) permet de définir une fonctionnelle linéaire sur l'ensemble des fonctions mesurables \( \Omega\to \eR\). Cette fonctionnelle linéaire est l'intégrale que nous allons définir à présent.

D'abord nous considérons les fonction \defe{simples}{simple!fonction}\index{fonction!simple}, c'est à dire les fonctions de la forme
\begin{equation}
    f=\sum_{i=1}^Na_i\caract_{E_i}
\end{equation}
où \( a_i\in\eR\) tandis que les \( E_i\) sont des ensembles \( \mu\)-mesurables. Si \( Y\in \tribA\) nous définissons
\begin{equation}
    \int_Yfd\mu=\sum_ia_i\mu(Y\cap E_i).
\end{equation}
Pour une fonction \( \mu\)-mesurable générale \( f\colon \Omega\to \mathopen[ 0 , \infty \mathclose]\) nous définissons l'intégrale de \( f\) sur \( Y\) par
\begin{equation}        \label{EqDefintYfdmu}
    \int_Yfd\mu=\sup\Big\{ \int_Yhd\mu\,\text{où \( h\) est une fonction simple et mesurable telle que \( 0\leq h\leq f\)} \Big\}.
\end{equation}
Maintenant nous définissons
\begin{equation}
    \mu(f)=\int_{\Omega}f
\end{equation}
si \( f\) est une fonction mesurable sur \( \Omega\).

\begin{remark}
    Dans \( \eR^d\), quasiment toutes les fonctions et ensembles sont mesurables. En effet la construction d'ensembles non mesurables demande obligatoirement l'utilisation de l'axiome du choix; de tels ensembles doivent être construits «exprès pour». Il y a très peu de chances pour que vous tombiez sur un ensemble non mesurable de \( \eR^d\) sans que vous ne vous en rendiez compte.

    Par exemple la variable aléatoire 
    \begin{equation}
        X(\omega)=\begin{cases}
            \frac{1}{ \omega }    &   \text{si $ \omega\neq 0$}\\
            \infty    &    \text{$\omega=0$}.
        \end{cases}
    \end{equation}
    est mesurable, mais non intégrable.
\end{remark}

\begin{lemma}   \label{Lemfobnwt}
    Soit \( f\) une fonction mesurable positive ou nulle telle que
    \begin{equation}
        \int_{\Omega}fd\mu=0.
    \end{equation}
    Alors \( f=0\) \( \mu\)-presque partout.
\end{lemma}

\begin{proof}
    L'ensemble des points \( x\in\Omega\) tels que \( f(x)\neq 0\) peut s'écrire comme une union dénombrable disjointe :
    \begin{equation}
        \{ x\in\Omega\tq f(x)\neq 0 \}=\bigcup_{i=0}^{\infty}E_i
    \end{equation}
    avec
    \begin{subequations}
        \begin{align}
            E_0&=\{ x\in\Omega\tq f(x)>1 \}\\
            E_i&=\{ x\in\Omega\tq \frac{1}{ i+1 }\leq f(x)<\frac{1}{ i } \}.
        \end{align}
    \end{subequations}
    Si un des ensembles \( E_i\) est de mesure non nulle, alors nous pouvons considérer la fonction simple \( h(x)=\frac{1}{ i+1 }\mtu_{E_i}\) dont l'intégrale sur \( \Omega\) est strictement positive. Par conséquent le supremum de la définition \eqref{EqDefintYfdmu} est strictement positif.

    Nous savons donc que \( \mu(E_i)=0\) pour tout \( i\). Étant donné que la mesure d'une union disjointe dénombrable est égale à la somme des mesures, nous avons
    \begin{equation}
        \mu\{ x\in\Omega\tq f(x)\neq 0 \}=0,
    \end{equation}
    ce qui signifie que \( f\) est nulle \( \mu\)-presque partout.
\end{proof}

\begin{corollary}   \label{CorjLYiSm}
    Soit \( f\) une fonction mesurable sur l'espace mesuré \( (\Omega,\tribA,\mu)\) telle que
    \begin{equation}
        \int_{\Omega}f\mtu_{f>0}d\mu=0.
    \end{equation}
    Alors \( f\leq 0\) presque partout.
\end{corollary}

\begin{proof}
    Nous avons l'égalité d'ensembles
    \begin{equation}
        \{ f\mtu_{f>0}\neq 0 \}=\{ \mtu_{f>0}\neq 0 \}.
    \end{equation}
    Mais lemme \ref{Lemfobnwt} implique que \( f\mtu_{f>0}\) est nulle presque partout, c'est à dire que la mesure de l'ensemble du membre de gauche est nulle par conséquent
    \begin{equation}
        \mu\{ \mtu_{f>0}\neq 0 \}=0.
    \end{equation}
    Cela signifie que la fonction \( f\) est presque partout négative ou nulle.
\end{proof}

\begin{lemma}   \label{LemPfHgal}
    Soit \( f\) une fonction telle que \( | f(x)|\leq g(x) \) pour tout \( x\in\Omega\). Si \( g\) est intégrable, alors \( f\) est intégrable.
\end{lemma}

\begin{proof}
    Nous décomposons \( f\) en parties positives et négatives :
    \begin{subequations}
        \begin{align}
            A_+&=\{ x\in\Omega\tq f(x)>0 \}\\
            A_-&=\{ x\in\Omega\tq f(x)<0 \}.
        \end{align}
    \end{subequations}
    Nous posons \( f_+(x)=f(x)\mtu_{A_+}\) et \( f_-(x)=f(x)\mtu_{A_-}\). Nous avons \( f=f_+-f_-\) et
    \begin{equation}
        \int_{\Omega}f=\int_{A_+}f+\int_{A_-}f
    \end{equation}
    parce que \( \Omega=A_+\cup A_-\cup\{ x\in\Omega\tq f(x)=0 \}\). Si \( \varphi\) est une fonction simple qui majore \( f_+\) nous avons
    \begin{equation}
        \varphi(x)=\sum_{k}a_k\mtu_{E_k}(x)\leq f(x)\mtu_{A_+}(x)\leq g(x).
    \end{equation}
    Par conséquent le supremum qui définit \( \int f_+\) est inférieur au supremum qui définit \( \int g\). La fonction \( f_+\) est donc intégrable. La même chose est valable pour la fonction \( f_-\).
\end{proof}

\begin{proposition} \label{PropWBavIf}
    Soit \( f\) une fonction positive \( \tribA\)-mesurable et bornée. Alors \( f\) est limite ponctuelle croissante de fonction simples.
\end{proposition}

\begin{proof}
    Soit \( \sigma_n=\{ a_0=0,\ldots, a_{r_n}=n \}\) une subdivision de \( \mathopen[ 0 , n \mathclose]\) en intervalles de taille plus petites que \( 1/n\) choisis de sorte que \( \sigma_{n-1}\subset\sigma_{n}\), et
    \begin{equation}
        f_n(x)=\begin{cases}
            0    &   \text{si \( f(x)>n\)}\\
            a_i    &    \text{sinon}
        \end{cases}
    \end{equation}
    où \( a_i\) est le plus grand élément de \( \sigma_n\) inférieur à \( f(x)\). La fonction \( f_n\) est simple et nous avons pour tout \( x\)
    \begin{equation}
        \lim_{n\to \infty} f_n(x)=f(x)
    \end{equation}
\end{proof}

%---------------------------------------------------------------------------------------------------------------------------
\subsection{Mesure dominée}
%---------------------------------------------------------------------------------------------------------------------------

Soient \( \mu\) et \( \nu\) deux mesures sur le même espace \( \Omega\) et la même tribu \( \tribA\). Nous disons que la mesure \( \mu\) est \defe{dominée}{dominée!mesure}\cite{PersoFeng} par \( \nu\) si pour tout ensemble mesurable \( A\), \( \nu(A)=0\) implique \( \mu(A)=0\).

La mesure \( \mu\) est \defe{portée}{portée!mesure} par l'ensemble \( E\in\tribA\) si pour tout \( A\in\tribA\), 
\begin{equation}
    \mu(A)=\mu(A\cap E).
\end{equation}

Nous écrivons que \( \mu\perp\nu\)\nomenclature[Y]{\( \mu\perp\nu\)}{mesures perpendiculaires} si il existe un ensemble \( E\in\tribA\) tel que \( \mu\) soit porté par \( E\) et \( \nu\) soit porté par \( \complement E\).

\begin{theorem}[Radon-Nikodym]\index{Radon-Nikodym}
    Soient \( \mu\) et \( m\) deux mesures \( \sigma\)-finies sur un espace métrisable \( (\Omega,\tribA)\).
    \begin{enumerate}
        \item
            Il existe un unique couple de mesures \( \mu_1\) et \( \mu_2\) telles que
            \begin{enumerate}
                \item
                    \( \mu=\mu_1+\mu_2\)
                \item
                    \( \mu_1\) est dominé par \( m\)
                \item
                    \( \mu_2\perp m\).
            \end{enumerate}
            Dans ce cas, les mesures \( \mu_1\) et \( \mu_2\) sont positives et \( \sigma\)-finies.
        \item
            À égalité \(  m\)-presque partout près, il existe une unique fonction mesurable positive \( f\) telle que pour tout mesurable \( A\),
            \begin{equation}
                \mu_1(A)=\int_Ad\mu_1=\int_{\Omega}\mtu_Afd m.
            \end{equation}
        \item
            À égalité \( m\)-presque partout près, il existe une unique fonction positive mesurable \( h\) telle que \( \mu_1=hm\).
    \end{enumerate}
\end{theorem}
Une démonstration est donné dans \cite{NikoLi}.

\begin{corollary}   \label{CorZDkhwS}
    Si \( \mu\) es une mesure \( \sigma\)-finie dominée par la mesure \( \sigma\)-finie \( m\), alors \( \mu\) possède une unique fonction de densité.
\end{corollary}

\begin{corollary}       \label{CorDomDens}
    Soient \( \mu\) et \( m\), deux mesures positives \( \sigma\)-finies sur \( (\Omega,\tribA)\). Alors \( m\) domine \( \mu\) si et seulement si \( \mu\) possède une densité par rapport à \( m\).
\end{corollary}
 
\begin{proof}
    Si \( \mu\) est dominée par \( m\), alors la décomposition \( \mu=\mu+0\) satisfait le théorème de Radon-Nikodym. Par conséquent il existe une fonction \( f\) telle que
    \begin{equation}
        \mu(A)=\int_Afdm.
    \end{equation}
    Cette fonction est alors une densité pour \( \mu\) par rapport à \( m\).

    Pour la réciproque, nous supposons que \( \mu\) a une densité \( f\) par rapport à \( m\), et que \( A\) est une ensemble de \( m\)-mesure nulle :
    \begin{equation}
        m(A)=\int_{\Omega}\mtu_Adm=0.
    \end{equation}
    Cela signifie que la fonction \( \mtu_A\) est \( m\)-presque partout nulle. La fonction produit \( \mtu_Af\) est également nulle \( m\)-presque partout, et par conséquent
    \begin{equation}
        \mu(A)=\int_{\Omega}\mtu_Afdm=0.
    \end{equation}
\end{proof}

\begin{probleme}
    Est-ce que la démonstration de cela ne demande pas la convergence monotone d'une façon ou d'une autre ?
\end{probleme}

%+++++++++++++++++++++++++++++++++++++++++++++++++++++++++++++++++++++++++++++++++++++++++++++++++++++++++++++++++++++++++++
\section{Mesure de Lebesgue}
%+++++++++++++++++++++++++++++++++++++++++++++++++++++++++++++++++++++++++++++++++++++++++++++++++++++++++++++++++++++++++++

Nous construisons à présent la mesure de Lebesgue sur \( \eR^n\). Un \defe{pavé}{pavé} dans \( \eR^n\) est un ensemble de la forme 
\begin{equation}
    B=\prod_{i=1}^n\mathopen[ a_i , b_i \mathclose];
\end{equation}
le volume d'un tel pavé est défini par \( \Vol(B)=\prod_i(b_i-a_i)\). Soit maintenant \( A\subset \eR^n\). La \defe{mesure externe}{mesure!externe} de \( A\) est le nombre
\begin{equation}
    m^*(A)=\inf\{ \sum_{B\in\mF}\Vol(B)\text{ où \( \mF\) est un ensemble dénombrable de pavés dont l'union recouvre \( A\).} \}
\end{equation}
Nous disons que \( A\) est \defe{mesurable}{mesurable!Lebesgue} au sens de Lebesgue si pour tout ensemble \( S\subset \eR^n\) nous avons l'égalité
\begin{equation}
    m^*(S)=m^*(A\cap S)+m^*(S\setminus A).
\end{equation}
Dans ce cas nous disons que la mesure de Lebesgue de \( A\) est \( m(A)=m^*(A)\).

%+++++++++++++++++++++++++++++++++++++++++++++++++++++++++++++++++++++++++++++++++++++++++++++++++++++++++++++++++++++++++++
\section{Mesure de Haar}
%+++++++++++++++++++++++++++++++++++++++++++++++++++++++++++++++++++++++++++++++++++++++++++++++++++++++++++++++++++++++++++

\begin{definition}
    Soit \( G\) un groupe topologique. Une \defe{mesure de Haar}{mesure!de Haar} sur \( G\) est une mesure \( \mu\) telle que 
    \begin{enumerate}
        \item
            \( \mu(gA)=\mu(A)\) pour tout mesurable \( A\) et tout \( g\in G\),
        \item
            \( \mu(K)<\infty\) pour tout compact \( K\subset G\).
    \end{enumerate}
    Si de plus le groupe \( G\) lui-même est compact nous demandons que la mesure soit normalisée : \( \mu(G)=1\).
\end{definition}
L'existence d'une mesure de Haar sur un groupe compact sera une conséquence du théorème de point fixe de Markov-Katutani \ref{ThoeJCdMP}.

%+++++++++++++++++++++++++++++++++++++++++++++++++++++++++++++++++++++++++++++++++++++++++++++++++++++++++++++++++++++++++++
\section{Suite et séries de fonctions}
%+++++++++++++++++++++++++++++++++++++++++++++++++++++++++++++++++++++++++++++++++++++++++++++++++++++++++++++++++++++++++++

Source : \cite{TrenchRealAnalisys}.

%---------------------------------------------------------------------------------------------------------------------------
\subsection{Convergence de suites de fonctions}
%---------------------------------------------------------------------------------------------------------------------------

Nous considérons un espace normé \( (\Omega,\| . \|)\). Nous disons qu'une suite de fonctions \( f_n\) \defe{converge}{convergence!en norme} vers \( f\) pour la norme \( \| . \|\) si \( \forall \epsilon>0\), \( \exists N\) tel que \( n\geq N\) implique \( \| f_n-f \|<\epsilon\).

Dans le cas particulier de la norme 
\begin{equation}
    \| f \|_{\infty}=\sup_{x\in\Omega}| f(x) |,
\end{equation}
nous parlons que \defe{convergence uniforme}{convergence!uniforme}.

\begin{theorem}[Critère de Cauchy]  \label{ThoCauchyZelUF}
    Une suite de fonctions  \( (f_n)_{n\in\eN}\) sur \( \Omega\) converge en norme sur \( \Omega\) si et seulement si \( \forall\epsilon>0\), \( \exists N\) tel que
    \begin{equation}
        \| f_n-f_m \|<\epsilon
    \end{equation}
    pour \( n,m>N\).
\end{theorem}

\begin{corollary}       \label{CorCauchyCkXnvY}
    La série \( \sum f_n\) converge en norme sur \( \Omega\) si et seulement si \( \exists N\) tel que
    \begin{equation}
        \| f_n+\ldots+f_m \|\leq \epsilon
    \end{equation}
    pour tout \( n,m>N\).
\end{corollary}

\begin{proof}
    L'hypothèse montre que la suite des sommes partielles de la série \( \sum f_n\) vérifie le critère de Cauchy du théorème \ref{ThoCauchyZelUF}.
\end{proof}

\begin{definition}
    Nous disons qu'un sous ensemble \( A\) de \( \Omega\) est \defe{complet}{complet} si toute suite de Cauchy d'éléments de \( A\) converge vers un élément de \( A\).
\end{definition}

\begin{theorem}[Weierstrass]
    La série de fonctions \( \sum_{n=1}^{\infty}f_n\) converge en norme si \( \| f_n \|\leq M_n\) avec \( \sum_nM_n<\infty\).
\end{theorem}

\begin{proof}
    Si \( (S_n)_{n\in\eN}\) est la suite des sommes partielles de \( \sum M_n\), alors le critère de Cauchy pour la convergence de suites numériques dit que si \( m\) et \( n\) sont assez grands, \( S_m-S_{n-1}<\epsilon\), c'est à dire
    \begin{equation}
        M_n+\ldots+M_n<\epsilon.
    \end{equation}
    Par conséquent nous avons
    \begin{equation}
        \| f_n+\ldots+f_m \|\leq\| f_n \|+\ldots\| f_m \|\leq \epsilon.
    \end{equation}
    Le corollaire \ref{CorCauchyCkXnvY} montre alors la convergence de la série.
\end{proof}

En corollaire, si \( \sum_n\| f_n \|\) converge, alors \( \sum_nf_n\) converge.

\begin{remark}
    Il n'y a pas de critère correspondant pour les suites. Il n'est pas vrai que si \( \lim_{n\to \infty}\| f_n \| \) existe, alors \( \lim_{n\to \infty} f_n\) existe, comme le montre l'exemple
    \begin{equation}
        f_n(x)=\begin{cases}
            1    &   \text{si \( x\in\mathopen[ 0 , 1 \mathclose]\) et \( n\) est pair}\\
            1    &    \text{si \( x\in\mathopen[ 1 , 2 \mathclose]\) et \( n\) est impair}\\
             0   &    \text{sinon.}
        \end{cases}
    \end{equation}
\end{remark}

%---------------------------------------------------------------------------------------------------------------------------
\subsection{Convergence monotone}
%---------------------------------------------------------------------------------------------------------------------------

Source : \cite{mathmecaChoi}.

\begin{theorem}[Théorème de la convergence monotone ou de Beppo-Levi] \label{ThoConvMonFtBoVh}\index{théorème!convergence monotone}\index{théorème!Beppo-Levi}
    Soit un espace mesuré \( (\Omega,\tribA,\mu)\) et \( (f_n)\) une suite croissante de fonctions mesurables à valeurs dans \( \mathopen[ 0 , \infty \mathclose]\). Alors la limite ponctuelle \( \lim_{n\to \infty} f_n\) existe, est mesurable et
    \begin{equation}
        \lim_{n\to \infty} \int_{\Omega}f_nd\mu= \int_{\Omega}\lim_{n\to \infty} f_nd\mu.
    \end{equation}
\end{theorem}

\begin{proof}
    La limite ponctuelle de la suite est la fonction à valeurs dans \( \mathopen[ 0 , \infty \mathclose]\) donnée par
    \begin{equation}
        f(x)=\lim_{n\to \infty} f_n(x).
    \end{equation}
    Ces limites existent parce que pour chaque \( x\) la suite \( f_n(x)\) est une suite numérique croissante. Nous notons
    \begin{equation}
        I_0=\int_{\Omega}fd\mu.
    \end{equation}
    Nous posons par ailleurs
    \begin{equation}
        I_n=\int_{\Omega}f_n.
    \end{equation}
    Cela est une suite numérique croissante qui a par conséquent une limite que nous notons \( I=\lim_{n\to \infty} I_n\). Notre objectif est de montrer que \( I=I_0\). D'abord par croissance de la suite, pour tous $n$ nous avons \( I_n\leq I_0\), par conséquent \( I\leq I_0\).

    Nous prouvons maintenant l'inégalité dans l'autre sens en nous servant de la définition \eqref{EqDefintYfdmu}. Soit une fonction simple \( h\) telle que \( h\leq f\), et une constante \( 0<C<1\). Nous considérons les ensembles
    \begin{equation}
        E_n=\{ x\in\Omega\tq f_n(x)\geq Ch(x) \}.
    \end{equation}
    Ces ensembles vérifient les propriétés \( E_n\subset E_{n+1}\) et \( \bigcup_{n=1}^{\infty}E_n=\Omega\). Pour chaque \( n\) nous avons les inégalités
    \begin{equation}
        \int_{\Omega}f_n\geq\int_{E_n}f_n\geq C\int_{E_n}h.
    \end{equation}
    Si nous prenons la limite \( n\to\infty\) dans ces inégalités,
    \begin{equation}
        \lim_{n\to \infty} \int_{\Omega}f_n\geq C\lim_{n\to \infty} \int_{E_n}h=C\int_{\Omega}h.
    \end{equation}
    Par conséquent \( \lim_{n\to \infty} \int f_n\geq C\int_{\Omega}h\). Mais étant donné que cette inégalité est valable pour tout \( C\) entre \( 0\) et \( 1\), nous pouvons l'écrire sans le \( C\) :
    \begin{equation}        \label{EqzAKEaU}
        \lim_{n\to \infty} \int_{\Omega}f_n\geq \int_{\Omega}h.
    \end{equation}
    Par définition, l'intégrale de \( f\) est donné par le supremum des intégrales de \( h\) où \( h\) est une fonction simple dominée par \( f\). En prenant le supremum sur \( h\) dans l'équation \eqref{EqzAKEaU} nous avons
    \begin{equation}
        \lim_{n\to \infty} \int_{\Omega}f_n\geq\int_{\Omega}f,
    \end{equation}
    ce qu'il nous fallait.
\end{proof}

\begin{corollary}[Inversion de somme et intégrales]
    Si \( (u_n)\) est une suite de fonctions mesurables positives ou nulles, alors
    \begin{equation}
        \sum_{i=0}^{\infty}\int u_i=\int\sum_{i=0}^{\infty}u_i.
    \end{equation}
\end{corollary}

\begin{proof}
    Nous considérons la suite des sommes partielles de \( (u_n)\) : \( f_n(x)=\sum_{i=0}^nu_n(x)\). Le théorème de la convergence monotone (théorème \ref{ThoConvMonFtBoVh}) implique que
    \begin{equation}
        \lim_{n\to \infty} \int f_n=\int\lim_{n\to \infty} f_n.
    \end{equation}
    Nous remplaçons maintenant \( f_n\) par sa valeur en termes des \( u_i\) et dans le membre de gauche nous permutons l'intégrale avec la somme finie :
    \begin{equation}
        \lim_{n\to \infty} \sum_{i=0}^{\infty}\int u_n=\int\sum_{i=0}^{\infty}u_n,
    \end{equation}
    ce qu'il fallait démontrer.
\end{proof}

\begin{lemma}[Lemme de Fatou]\index{lemme!Fatou}\index{Fatou}   \label{LemFatouUOQqyk}
    Soit \( (\Omega,\tribA,\mu)\) un espace mesuré et \( f_n\colon \Omega\to \mathopen[ 0 , \infty \mathclose]  \) une suite de fonctions mesurables. Alors la fonction \( f(x)=\liminf f_n(x)\) est mesurable et
    \begin{equation}
        \int_{\Omega}\liminf f_nd\mu\leq\liminf\int_{\Omega}fd\mu.
    \end{equation}
\end{lemma}

\begin{proof}
    Nous posons 
    \begin{equation}
        g_n(x)=\inf_{i\geq n}f_i(x).
    \end{equation}
    Cela est une suite croissance de fonctions positives mesurables telles que, par définition, 
    \begin{equation}
        \lim_{n\to \infty}g_n(x)=\liminf f_n(x).
    \end{equation}
    Nous pouvons y appliquer le théorème de la convergence monotone,
    \begin{equation}
        \lim_{n\to \infty} \int g_n(x)=\int\liminf f_n(x).
    \end{equation}
    Par ailleurs, pour chaque \( i\geq n\) nous avons
    \begin{equation}
        \int g_n\leq \int f_i,
    \end{equation}
    en passant à l'infimum nous avons
    \begin{equation}
        \int g_n\leq \inf_{i\geq n}\int f_i,
    \end{equation}
    et en passant à la limite nous avons
    \begin{equation}
        \int\liminf f_n=\lim_{n\to \infty} \int g_n\leq \lim_{n\to \infty} \inf_{i\geq n}\int f_i=\liminf_{i\to\infty}\inf f_i.
    \end{equation}
\end{proof}

L'inégalité donnée dans ce lemme n'est en général pas une égalité, comme le montre l'exemple suivant :
\begin{equation}
    f_i=\begin{cases}
        \mtu_{\mathopen[ 0 , 1 \mathclose]}    &   \text{si \( i\) est pair}\\
        \mtu_{\mathopen[ 1 , 2 \mathclose]}    &    \text{si \( i\) est impair}.
    \end{cases}
\end{equation}
Nous avons évidemment \( g_n(x)=0\) tandis que \( \int_{\mathopen[ 0 , 2 \mathclose]}f_i=1\) pour tout \( i\).

%---------------------------------------------------------------------------------------------------------------------------
\subsection{Convergence dominée de Lebesgue}
%---------------------------------------------------------------------------------------------------------------------------

\begin{theorem}[Convergence dominée de Lebesgue]\index{théorème!convergence dominée de Lebesgue}        \label{ThoConvDomLebVdhsTf}
    Soit \( (f_n)_{n\in\eN}\) une suite de fonctions intégrables sur \( (\Omega,\tribA,\mu)\) à valeurs dans \( \eC\) ou \( \eR\). Nous supposons que  \( f_n\to f\) simplement sur \( \Omega\) presque partout et qu'il existe une fonction intégrable \( g\) telle que
    \begin{equation}
        | f_n(x) |< g(x) 
    \end{equation}
    pour tout \( x\in\Omega\) et pour tout \( n\in \eN\). Alors
    \begin{enumerate}
        \item
            \( f\) est intégrable,
        \item
           $\lim_{n\to \infty} \int_{\Omega}f_n=\int_\Omega f$,
        \item
            $\lim_{n\to \infty} \int_{\Omega}| f_n-f |=0$.
    \end{enumerate}
\end{theorem}

\begin{proof}

    La fonction limite \( f\) est intégrable parce que \( | f |\leq g\) et \( g\) est intégrable (lemme \ref{LemPfHgal}). Par hypothèse nous avons
    \begin{equation}
        -g(x)\leq f_n(x)\leq g(x).
    \end{equation}
    En particulier la fonction \( g_n=f_n+g\) est positive et mesurable si bien que le lemme de Fatou (lemme \ref{LemFatouUOQqyk}) implique
    \begin{equation}
        \int_{\Omega}\liminf g_n\leq\liminf\int_{\Omega}g_n.
    \end{equation}
    Évidement nous avons \( \liminf g_n=f+g\), de telle sorte que
    \begin{equation}
        \int f+\int g\leq \liminf\int g_n=\liminf\int f_n+\int g,
    \end{equation}
    et le nombre \( \int g\) étant fini, nous pouvons le retrancher des deux côtés de l'inégalité :
    \begin{equation}
        \int f\leq\liminf\int f_n.
    \end{equation}
    Afin d'obtenir une minoration de \( \int f\) nous refaisons exactement le même raisonnement en utilisant la suite de fonctions \( k_n=-f_n\to k=-f\). Nous obtenons que
    \begin{equation}
        \int k\geq\liminf\int k_n=-\limsup\int f_n,
    \end{equation}
    et par conséquent
    \begin{equation}    \label{IneqsndMYTO}
        \liminf\int f_n\leq\int f\leq\limsup\int f_n.
    \end{equation}
    La limite supérieure étant plus grande ou égale à la limite inférieure, les trois quantités dans les inégalités \eqref{IneqsndMYTO} sont égales.
\end{proof}

\begin{corollary}       \label{CorCvAbsNormwEZdRc}
    Soit \( (a_i)_{i\in \eN}\) une suite numérique absolument convergente. Alors elle est convergente. Il en est de même pour les séries de fonctions si on considère la convergence ponctuelle.
\end{corollary}

\begin{proof}
    L'hypothèse est la convergence de l'intégrale \( \int_{\eN}| a_i |dm(i)\) où \( dm\) est la mesure de comptage. Étant donné que \( | a_i |\leq | a_i |\), la fonction \( a_i\) (fonction de \( i\)) peut jouer le rôle de \( g\) dans le théorème de la convergence dominée de Lebesgue (théorème \ref{ThoConvDomLebVdhsTf}).
\end{proof}

%---------------------------------------------------------------------------------------------------------------------------
\subsection{Dérivation, intégration}
%---------------------------------------------------------------------------------------------------------------------------

\begin{theorem}[\cite{TrenchRealAnalisys}]      \label{ThoCciOlZ}
    Supposons que \( \sum_{n=0}^{\infty}f_n\) converge uniformément vers \( F\) sur \( \mathopen[ a , b \mathclose]\). Si \( F\) et \( f_n\) sont des fonctions intégrables sur \( \mathopen[ a , b \mathclose]\) alors
    \begin{equation}
        \int_a^bF(x)dx=\sum_{n=0}^{\infty}\int_a^bf_n(x)dx.
    \end{equation}
\end{theorem}

\begin{theorem} \label{ThoCSGaPY}
    Soit \( f_n\) des fonctions \( C^1\mathopen[ a , b \mathclose]\) telles que
    \begin{enumerate}
        \item
            la série \( \sum_n f_n(x_0)\) converge pour un certain \( x_0\in\mathopen[ a , b \mathclose]\),
        \item
            la série des dérivées \( \sum_n f'_n\) converge uniformément sur \( \mathopen[ a , b \mathclose]\).
    \end{enumerate}
    Alors la série \( \sum_n f_n\) converge vers une fonction \( F\) et
    \begin{enumerate}
        \item
            La convergence est uniforme sur \( \mathopen[ a , b \mathclose]\).
        \item
            La fonction \( F\) est dérivable
        \item
            \( F'(x)=\sum_nf'_n(x)\).
    \end{enumerate}
\end{theorem}

%+++++++++++++++++++++++++++++++++++++++++++++++++++++++++++++++++++++++++++++++++++++++++++++++++++++++++++++++++++++++++++
\section{Séries entières}
%+++++++++++++++++++++++++++++++++++++++++++++++++++++++++++++++++++++++++++++++++++++++++++++++++++++++++++++++++++++++++++

Source : \cite{RomainBoilEnt}. Dans cette section nous allons parler de séries complexes autant que de séries réelles. L'étude des propriétés à proprement parler complexes des séries entières (holomorphie) sera effectuée dans le chapitre dédié, à la section \ref{SecoLNvnO}.

Nous rappelons qu'une série de nombres \( \sum_{n=0}^{\infty}a_n\) converge \defe{absolument}{convergence!absolue} si la série
\begin{equation}
    \sum_{n=0}^{\infty}| a_n |
\end{equation}
converge. Cette définition s'étend immédiatement aux séries dans n'importe quel espace normé.

La convergence est \defe{normale}{convergence!normale} si la suite de fonctions \( f_N(z)=\sum_{n=0}^N a_nz^n\) converge uniformément (c'est à dire pour la norme supremum).

\begin{definition}
    Une \defe{série entière}{série!entière} est une somme de la forme
    \begin{equation}
        \sum_{n=0}^{\infty}a_nz^n
    \end{equation}
    avec \( a_n,z\in\eC\).    
\end{definition}
Une série entière peut définir une fonction
\begin{equation}
    f(z)=\sum_na_nz^n.
\end{equation}
Le but de cette section est d'étudier des conditions sur la suite \( (a_n)\) qui assurent la continuité de \( f\) ou la possibilité de dériver ou intégrer la série terme à terme.

%---------------------------------------------------------------------------------------------------------------------------
					\subsection{Série de puissances}
%---------------------------------------------------------------------------------------------------------------------------

Une \defe{série de puissance}{Série!de puissance} est une série de la forme
\begin{equation}		\label{eqseriepuissance}
	\sum_{k=0}^{\infty}c_k(z-z_0)^k
\end{equation}
où $z_0\in \eC$ est fixé, $(c_k)$ est une suite complexe fixée, et $z$ est un paramètre complexe. Nous disons que cette série est \emph{centrée} en $z_0$.

\begin{theorem}		\label{ThoSerPuissRap}
Considérons la série de puissances donnée par le terme général $c_k(z-z_0)^k$. Si nous posons
\begin{equation}		\label{EqRayCOnvSer}
	\alpha=\frac{1}{ R } =\limsup\sqrt[k]{| c_k |}
\end{equation}
alors la série converge absolument si $| z-z_0 |<R$ et diverge si $| z-z_0 |>R$.

De plus, ce rayon de convergence peut être calculé par la formule alternative 
\begin{equation}		\label{EqAlphaSerPuissAtern}
	\alpha=\frac{1}{ R }=\limite k \infty \abs{\frac{c_{k+1}}{c_k}}
\end{equation}
lorsque $c_k$ est non nul à partir d'un certain $k$.
\end{theorem}
Le disque $| z-z_0 |\leq R$ est le \defe{disque de convergence}{Disque de convergence} de la série \eqref{eqseriepuissance}. Notez que ce théorème ne dit rien pour les points tels que $| z-z_0 |=R$. Il faut traiter ces points au cas par cas. Et le pire, c'est qu'une série donnée peut converger pour certain des points sur le bord du disque, et diverger en d'autres.  Il y a un dessin à la figure \ref{LabelFigDisqueConv}.
\newcommand{\CaptionFigDisqueConv}{À l'intérieur du disque de convergence, la convergence est absolue. En dehors, la série diverge. Sur le cercle proprement dit, tout peut arriver.}
\input{Fig_DisqueConv.pstricks}

Le disque de centre $z_0$ et de rayon $R$ est appelé \Defn{disque de convergence}. Pour un complexe $z$ sur le bord de ce disque, c'est-à-dire tel que $\abs{z-z_0} = R$, le comportement peut-être très varié (convergence absolue, convergence simple ou divergence) et n'est éventuellement pas le même sur tout le bord.

L'étude de ce qu'il se passe sur le bord du disque de convergence commence par y étudier la convergence absolue, c'est à dire étudier la série
\begin{equation}
	\sum_k| c_k(z-z_0)^k |=\sum_k| c_k |R^k
\end{equation}
parce que sur le bord, $| z-z_0 |=R$. L'étude du terme général $| c_k |R^k$ a deux utilités :
\begin{enumerate}
\item Si la somme $\sum_{k}| c_k |R^k$ converge, alors la série converge uniformément sur le bord,
\item si la suite $| c_k |R^k$ ne tend pas vers zéro, alors la série ne converge même pas simplement sur le bord.
\end{enumerate}

%---------------------------------------------------------------------------------------------------------------------------
\subsection{Convergence normale}
%---------------------------------------------------------------------------------------------------------------------------

Une série de fonctions \( \sum_{n\in \eN}u_n \) converge \defe{normalement}{convergence!normale} si la série de nombre \( \sum_n\| u_n \|_{\infty}\) converge.

\begin{lemma}
    Soient des fonctions \( u_n\colon \Omega\to \eC\). Si il existe une suite réelle positive \( (a_n)_{n\in \eN}\) telle que
    \begin{enumerate}
        \item
            pour tout \( z\in \Omega\) et pour tout \( n\in \eN\) nous avons \( | u_n(z) |\leq a_n\) (c'est à dire \( a_n\geq \| u_n \|_{\infty}\)),
        \item
            la somme \( \sum_{n}a_n\) converge,
    \end{enumerate}
    alors la série de fonctions \( \sum_{n=0}^{\infty}u_n\) converge normalement.
\end{lemma}

\begin{proof}
    Découle du lemme de comparaison.
\end{proof}

\begin{proposition}
    Soit \( (u_n)\) une suite de fonctions continues \( u_n\colon \Omega\subset\eC\to \eC\). Si la série \( \sum_nu_n\) converge normalement alors la somme est continue.
\end{proposition}

\begin{proof}
    Nous posons \( u(z)=\lim_{n\to \infty} u_n(z)\), et nous vérifions que la fonction ainsi définie sur \( \Omega\) est continue. Soit \( z\in \Omega\) et prouvons la continuité de \( u\) au point \( z\). Pour tout \( z'\) dans un voisinage de \( z\) nous avons 
    \begin{subequations}
        \begin{align}
            \big| u(z)-u(z') \big|&=\left| \sum_{n=0}^{N}u_n(z)-\sum_{n=0}^{N}u_n(z')+\sum_{n=N+1}^{\infty}u_n(z)-\sum_{n=N+1}^{\infty}u_n(z') \right| \\
            &\leq \left| \sum_{n=0}^N u_n(z)-\sum_{n=0}^Nu_n(z') \right| +\sum_{n=N+1}^{\infty}| u_n(z) |+\sum_{n=N+1}^{\infty}| u_n(z') |.
        \end{align}
    \end{subequations}
    Étant donné que les sommes partielles sont continues, en prenant \( N\) suffisamment grand, le premier terme peut être rendu arbitrairement petit. Si \( N\) est suffisamment grand, le second terme est également petit. Par contre, cet argument ne tient pas pour le troisième terme parce que nous souhaitons une majoration pour tout \( z'\) dans une boule autour de \( z\). Nous devons donc écrire
    \begin{equation}
        \sum_{n=N}^{\infty}| u_(z) |\leq \sum_{n=N+1}^{\infty}\| u_n \|_{\infty}.
    \end{equation}
    Ce dernier est arbitrairement petit lorsque \( N\) est grand. Notons que nous avons utilisé l'hypothèse de convergence normale.
\end{proof}

\begin{lemma}[Critère d'Abel]\index{critère!Abel}
    Soit \( (a_n)\) une suite dans \( \eC\) et \( r>0\). Si la suite \( (a_nr^n)\) est bornée alors pour tout \( z\in B(0,r)\) la série \( \sum a_nz^n\) converge absolument.
\end{lemma}

\begin{proof}
    Soit \( M\in \eR\) tel que \( | a_n |r^n\leq M\) pour tout \( n\). Alors nous avons
    \begin{equation}
        | a_nz^n |=| a_n |r^n\big( \frac{ | z | }{ r } \big)^n\leq M\left( \frac{ | z | }{ r } \right)^n
    \end{equation}
    Si \( | z |<r\) alors nous tombons sur la série géométrique qui converge. Par le critère de comparaison la série \( \sum_{n=0}^{\infty}| a_nz^n |\) converge.
\end{proof}

\begin{definition}
    Soit \( \sum_{n\in \eN}a_nz^n\) une série entière. Le \defe{rayon de convergence}{rayon!de convergence} de cette série est le nombre
    \begin{equation}
        R=\sup\{ r\in \eR^+\tq \text{la suite \((a_nr^n)\) est bornée} \}\in\mathopen[ 0 , \infty \mathclose].
    \end{equation}
\end{definition}
Le rayon de convergence d'une série ne dépend que des réels \( | a_n |\), même si à la base \( a_n\in \eC\).

\begin{theorem}
    Soit \( R>0\) le rayon de convergence de la somme \( \sum_na_nz^n\) et \( z\in \eC\).
    \begin{enumerate}
        \item
            Si \( | z |<R\) alors la série converge absolument.
        \item
            Si \( R<\infty\) et si \( | z |>R\) alors la suite \( (a_nz^n)\) n'est pas bornée et la série diverge.
    \end{enumerate}
\end{theorem}

\begin{proof}
    \begin{enumerate}
        \item
            Étant donné que \( | z |<R\), il existe \( r>0\) tel que \( | z |<r<R\). On a que \( (a_nr^n)\) est borné (parce que \( R\) est le supremum) et donc \( (a_n| z_n |)\) est bornée. Le critère d'Abel conclu.
        \item
            Par hypothèse la suite \( (a_n| z |^n)\) n'est pas bornée. La suite \( (a_nz^n)\) n'est donc pas bornée non plus et la série ne peut pas converger.
    \end{enumerate}
\end{proof}

Si les suites \( a_n\) et \( b_n\) sont équivalentes, alors les séries correspondantes auront le même rayon de convergence. Cela ne signifie pas que sur le bord du disque de convergence, elles aient même comportement. Par exemple nous avons
\begin{equation}
    \frac{1}{ \sqrt{n} }\sim \frac{1}{ \sqrt{n} }+\frac{ (-1)^n }{ n }.
\end{equation}
En même temps, en \( z=-1\) la série 
\begin{equation}
    \sum_{n\geq 1}\frac{ z^n }{ \sqrt{n} }
\end{equation}
converge par le critère des séries alternées (corollaire \ref{CoreMjIfw}). Par contre la série
\begin{equation}
    \sum_{n\geq 1}\left( \frac{1}{ \sqrt{n} }+\frac{ (-1)^n }{ n } \right)z^n
\end{equation}
ne converge pas pour \( z=-1\).

\begin{example}
    Soit \( \alpha\in \eR\) et considérons la série \( \sum_{n\geq 1}a_nz^n\) où \( a_n\) est la \( n\)-ième décimale de \( \alpha\). Si \( \alpha\) est un nombre décimal limité, la suite \( (a_n)\) est finie et le rayon de convergence est infini. Sinon, pour tout \( N\) il existe un \( n>N\) tel que \( a_n\neq 0\) et la suite \( (a_n)\) ne tend pas vers zéro. Par conséquent la série
    \begin{equation}
        \sum_{n}a_nz^n
    \end{equation}
    diverge pour \( z=1\) et le rayon de convergence satisfait \( R\leq 1\). Nous avons aussi \( | a_n |\leq 9\), de telle manière à ce que la série soit bornée et par conséquent majorée en module par \( 9z^n\), ce qui signifie que \( R\geq 1\). 

    Nous déduisons alors \( R=1\).
\end{example}

%---------------------------------------------------------------------------------------------------------------------------
\subsection{Propriétés de la somme}
%---------------------------------------------------------------------------------------------------------------------------

\begin{theorem}     \label{ThokPTXYC}
    Soient \( \sum_na_nz^n\) et \( \sum b_nz^n\) deux séries de rayon de convergences respectivement \( R_a\) et \( R_b\).
    \begin{enumerate}
        \item   \label{IteWlajij}
            Si \( R_s\) est le rayon de convergence de \( \sum_n(a_n+b_n)z^n\), nous avons
            \begin{equation}
                R_s\geq \min\{ R_a,R_b \}
            \end{equation}
            et nous avons l'égalité si pour tout \( |z |\leq\min\{ R_a,R_b \}\), \( \sum (a_n+b_n)z^n=\sum_n a_nz^n+\sum_nb_nz^n\).
        \item
            Si \( \lambda\neq 0\) la série \( \sum_n(\lambda a_n)z^n\) a le même rayon de convergence que la série \( \sum_na_nz^n\) et si \( | z |<R_a\) nous avons
            \begin{equation}
                \sum_{n=0}^{\infty}(\lambda a_n)z^n=\lambda\sum_{n=0}^{\infty}a_nz^n.
            \end{equation}
        \item
            Le \defe{produit de Cauchy}{Cauchy!produit}\index{produit!de Cauchy} des deux séries est donné par
            \begin{equation}
                \sum_{n=0}^{\infty}\left( \sum_{i+j=n}a_ib_j \right)z^n.
            \end{equation}
            Si \( R_p\) est le rayon de convergence de ce produit nous avons
            \begin{equation}
                R_p\geq \min\{ R_a,R_b \}
            \end{equation}
            et si \( | z |<\min\{ R_a,R_b \}\) alors
            \begin{equation}
                \sum_{n=0}^{\infty}\left( \sum_{i+j=n}a_ib_j \right)z^n=\left( \sum_{n=0}^{\infty}a_nz^n \right)\left( \sum_{n=0}^{\infty}b_nz^n \right).
            \end{equation}
            
    \end{enumerate}
    
\end{theorem}

\begin{proof}
    Nous prouvons la partie sur le produit de Cauchy. En utilisant la propriété du produit de la somme par un scalaire nous avons
    \begin{subequations}
        \begin{align}
            \left( \sum_{n=0}^{\infty}a_nz^n \right)\left( \sum_{m=0}^{\infty}b_mz^m \right)&=\sum_{n=0}^{\infty}\left( \sum_{m=0}^{\infty}b_ma_nz^{m+n} \right)\\
            &=\lim_{N\to \infty} \lim_{M\to \infty} \sum_{n=0}^N\sum_{m=0}^Mb_ma_nz^{m+n}\\
            &=\lim_{N\to \infty} \lim_{M\to \infty} \sum_{k=0}^{N+M}\sum_{i+k=k}b_ia_jz^k\\
            &=\lim_{N\to \infty} \sum_{k=0}^{\infty}\sum_{i+k=k}b_ia_jz^k\\
            &=\sum_{k=0}^{\infty}\sum_{i+j=k}b_ia_jz^k.
        \end{align}
    \end{subequations}
\end{proof}

\begin{example}
    Montrons un produit de Cauchy dont le rayon de convergence est strictement plus grand que le minimum. D'abord nous considérons
    \begin{equation}
        A=1-z,
    \end{equation}
    c'est à dire \( a_0=1\), \( a_1=-1\), \( a_{n\geq 2}=0\) avec \( R_a=\infty\). Ensuite nous considérons
    \begin{equation}
        B=\sum_nz^n,
    \end{equation}
    c'est à dire \( B=(1-z)^{-1}\) et \( R_b=1\). Le produit de Cauchy de ces deux séries valant \( 1\), le rayon de convergence est infini.
\end{example}

\Exo{reserve0005}

\begin{theorem}
    Une série entière converge normalement sur tout disque fermé inclus au disque de convergence.
\end{theorem}

\begin{proof}
    Toute boule fermée inclue à \( B(0,R)\) est inclue à la boule \( \overline{ B(0,r) }\) pour un certain \( r<R\). Nous nous concentrons donc sur une telle boule fermée.

    Pour chaque \( n\) nous posons \( u_n(z)=a_nz^n\) que nous voyons comme une fonction sur \( \overline{ B(0,r) }\). Pour tout \( n\in \eN\) et tout \( z\in\overline{ B(0,r) }\) nous avons 
    \begin{equation}
        \| u_n \|_{\infty}\leq| a_nz^n |\leq | a_n |r^n.
    \end{equation}
    Étant donné que \( r<R\) la série \( \sum_n | a_n |r^n\) converge et la série \( \sum_n\| u_n \|\) est convergente. La série \( \sum_na_nz^n\) est alors normalement convergente.
\end{proof}

\begin{example}
    Encore une fois nous n'avons pas d'informations sur le comportement au bord. Par exemple la série \( \sum_nz^n\) a pour rayon de convergence \( R=1\), mais \( \sup_{z\in B(0,1)}| z^n |=1\) de telle façon à ce que nous n'avons pas de convergence normale sur la boule fermée.
\end{example}
La convergence normale n'est donc pas de mise sur tout l'intérieur du disque de convergence. La continuité, par contre est effective sur la boule. En effet si \( z_0\in B(0,R)\) alors il existe un rayon \( 0<r<R\) tel que \( B(z_0,r)\subset B(0,R)\). Sur \( B(z_0,r)\) nous avons convergence normale et donc continuité en \( z_0\).

La différence est que la continuité est une propriété locale tandis que la convergence normale est une propriété globale.

\begin{proposition}
    Soit \( f(z)=\sum_na_nz^n\) avec un rayon de convergence \( R\). Si \( \sum | a_n |R^n\) converge alors
    \begin{enumerate}
        \item
            la série \( \sum_na_nz^n\) converge normalement sur \( \overline{ B(0,R) }\),
        \item
            \( f\) est continue sur \( \overline{ B(0,R) }\).
    \end{enumerate}
\end{proposition}

\begin{proof}
    La conclusion est claire dans l'intérieur du disque de convergence. En ce qui concerne le bord, chacune des sommes partielles est une fonction continue. De plus nous avons \( \| u_n \|\leq | a_n |R^n\), dont la série converge. Par conséquent nous avons convergence normale sur le disque fermé.
\end{proof}

Le théorème suivant permet de donner, dans le cas de fonctions réelle, des informations sur la convergence en une des deux extrémités de l'intervalle de convergence.
\begin{theorem}[Convergene radiale de Abel]\index{Abel!convergence radiale} \label{ThoLUXVjs}
    Soit \( f(x)=\sum_na_nx^n\) une série réelle de rayon de convergence \( 0<R<\infty\).
    \begin{enumerate}
        \item
            Si \( \sum a_nR^n\) converge, alors \( f\) est continue sur \( \mathopen[ 0 , R \mathclose]\).
        \item
            Si \( \sum_na_n(-R)^n\) converge, alors \( f\) est continue sur \( \mathopen[ -R , 0 \mathclose]\).
    \end{enumerate}
\end{theorem}

\Exo{reserve0006}

Le résultat suivant permet d'identifier deux séries complexes lorsque leurs valeurs sur \( \eR\) sont identiques.
\begin{proposition}
    Soient les séries \( f(z)=\sum a_nz^n\) et \( g(z)=\sum b_n z^n\) convergentes dans \( B(0,R)\). Si \( f(x)=g(x)\) pour \( x\in \mathopen[ 0 , R [\) alors \( a_n=b_n\).
\end{proposition}

\begin{proof}
    Soit \( n_0\) le plus petit entier tel que \( a_{n_0}\neq b_{n_0}\). Pour tout \( z\in B(0,R)\) nous avons
    \begin{equation}
        f(z)-g(z)=\sum_{n=n_0}^{\infty}(a_n-b_n)z^n=z^{n_0}\varphi(z)
    \end{equation}
    où
    \begin{equation}
        \varphi(z)=\sum_{n\geq 0}(a_{n+n_0}-b_{n+n_0})z^n.
    \end{equation}
    Par le théorème \ref{ThokPTXYC}\ref{IteWlajij} le rayon de convergence de \( \varphi\) est plus grand que \( R\) et la fonction \( \varphi\) est continue en \( 0\). Étant donné que \( \varphi(0)=a_{n_0}-b_{n_0}\neq 0\) et que \( \varphi\) est continue nous avons un \( \rho\) tel que \( \varphi\neq 0\) sur \( B(0,\rho)\). Or cela n'est pas possible parce que au moins sur la partie réelle de cette dernière boule, \( \varphi\) doit être nulle.
\end{proof}

\begin{lemma}       \label{LemFVMaSD}
    Soit une série entière \( \sum a_nz^n\) de rayon de convergence \( R\). Les séries
    \begin{equation}
        \sum \frac{ a_n }{ n+1 }z^{n+1}
    \end{equation}
    et
    \begin{equation}
        \sum_{n\geq 1}na_nz^{n-1}
    \end{equation}
    ont même rayon de convergence \( R\).
\end{lemma}

Notons toutefois que nonobstant ce lemme, les séries dont il est question peuvent se comporter différemment sur le bord du disque de convergence. En effet la série
\begin{equation}
    \sum \frac{1}{ n }z^n
\end{equation}
diverge pour \( z=1\) alors que 
\begin{equation}
    \sum\frac{1}{ n(n+1) }z^{n+1}
\end{equation}
converge pour \( z=1\).


%---------------------------------------------------------------------------------------------------------------------------
\subsection{Dérivation, intégration}
%---------------------------------------------------------------------------------------------------------------------------

Les théorèmes de dérivation et d'intégration de séries de fonctions (théorèmes \ref{ThoCciOlZ} et \ref{ThoCSGaPY}) fonctionnent bien dans le cas des séries entières.

\begin{proposition}
    Soit la série entière
    \begin{equation}
        f(x)=\sum a_nx^n
    \end{equation}
    de rayon de convergence \( R\). Pour tout segment \( \mathopen[ a , b \mathclose]\subset\mathopen] -R , R \mathclose[\) nous pouvons intégrer terme à terme :
    \begin{equation}
        \int_a^bf(t)dt=\sum_{n=0}^{\infty}a_n\int_a^bt^ndt.
    \end{equation}
\end{proposition}

\begin{proof}
    Ceci est un cas particulier du théorème général \ref{ThoCciOlZ}. Notons que par le lemme \ref{LemFVMaSD}, la série entière qui intègre la série de \( f\) terme à terme a le même rayon de convergence que celui de \( f\).
\end{proof}

\begin{proposition}
    Soit la série entière
    \begin{equation}
        f(x)=\sum_{n=0}^{\infty}a_n x^n
    \end{equation}
    de rayon de convergence \( R\). Alors la fonction \( f\) est \( C^1\) sur \( \mathopen] -R , R \mathclose[\) et se dérive terme à terme :
    \begin{equation}
        f'(x)=\sum_{n=1}^{\infty}na_nx^{n-1}
    \end{equation}
    pour tout \( x\in\mathopen] -R , R \mathclose[\).
\end{proposition}

\begin{proof}
    Nous savons que la série \( \sum_{n=1}^{\infty}na_nx^{n-1}\) a le même rayon de convergente que celui de la série \( f\). En particulier cette série des dérivées converge normalement sur tout compact dans \( \mathopen] -R , R \mathclose[\) et la somme est continue. Le théorème \ref{ThoCSGaPY} conclu.
\end{proof}

\begin{example}
    Montrons que la fonction
    \begin{equation}
        \begin{aligned}
            f\colon \eR_+\setminus\{ 0,1 \}&\to \eR \\
            x&\mapsto \frac{ \ln(x) }{ x-1 } 
        \end{aligned}
    \end{equation}
    admet un prolongement \( C^{\infty}\) sur \( \eR_+\setminus\{ 0 \}\).

    Nous allons étudier la fonction
    \begin{equation}
        f(x)=\frac{ \ln(1+x) }{ x }
    \end{equation}
    autour de \( x=0\). Le logarithme ne pose pas de problèmes à développer dans un voisinage :
    \begin{subequations}
        \begin{align}
            f(x)&=\frac{1}{ x }\sum_{n=1}^{\infty}\frac{ (-1)^{n+1} }{ n }x^n\\
            &=\sum_{n=1}^{\infty}\frac{ (-1)^{n+1} }{ n }x^{n-1}\\
            &=\sum_{n=0}^{\infty}\frac{ (-1)^k }{ k+1 }x^k.
        \end{align}
    \end{subequations}
    Cette série a un rayon de convergence égal à \( 1\), et donc définit sans problèmes une fonction \( C^{\infty}\) dans un voisinage de \( x=0\). Notons que par convention \( x^0=1\) même si \( x=0\).
\end{example}

%---------------------------------------------------------------------------------------------------------------------------
\subsection{Développement en série et Taylor}
%---------------------------------------------------------------------------------------------------------------------------

Soit une fonction \( f\colon \eC\to \eC\) et \( z_0\in \eC\). Nous disons que \( f\) est \defe{développable en série entière}{développable!en série entière} dans un voisinage de \( z_0\) si il existe une série \( \sum_n a_nz^n\) de rayon de convergence \( R>0\) et \( r\leq R\) tel que
\begin{equation}
    f(z)=\sum_{n=0}^{\infty}a_n(z-z_0)^n
\end{equation}
pour tout \( z\in B(z_0,r)\).

\begin{proposition}
    Si \( V\) est un ouvert dans \( \eC\) alors l'ensemble des fonctions \( V\to \eC\) développables en série entière forme une \( \eC\)-algèbre.
\end{proposition}

\begin{proof}
    Les séries entières passent aux sommes et aux produits en gardant des rayons de convergence non nuls.
\end{proof}

\begin{proposition}
    Si \( f\) est développable en série entière à l'origine alors elle est \( C^{\infty}\) sur un voisinage de l'origine et le développement est celui de \defe{Taylor}{Taylor!série entière} :
    \begin{equation}
        f(x)=\sum_{n=0}^{\infty}\frac{ f^{(n)}(0) }{ n! }x^n.
    \end{equation}
    <++>
\end{proposition}
<++>

%---------------------------------------------------------------------------------------------------------------------------
\subsection{Resommer une série}
%---------------------------------------------------------------------------------------------------------------------------

Pour des techniques de calculs de sommes, voir \cite{DAnSerEntiere}.

\begin{verbatim}
----------------------------------------------------------------------
| Sage Version 4.7.1, Release Date: 2011-08-11                       |
| Type notebook() for the GUI, and license() for information.        |
----------------------------------------------------------------------
sage: n=var('n')
sage: S(x)=sum(  [ n**3*x**n for n in range(0,30)  ]   )
sage: f(x)=-1/(1-x)+7/((x-1)**2)+12/((x-1)**3)+6/( (x-1)**4  )
sage: S(0.1)
0.214906264288980
sage: f(0.1)
0.214906264288981
sage: f.plot(-0.5,0.5)+S.plot(-0.5,0.5)
\end{verbatim}
Les deux courbes se confondent.
% qPPKLJ


%+++++++++++++++++++++++++++++++++++++++++++++++++++++++++++++++++++++++++++++++++++++++++++++++++++++++++++++++++++++++++++
\section{Dérivation sous le signe intégral}
%+++++++++++++++++++++++++++++++++++++++++++++++++++++++++++++++++++++++++++++++++++++++++++++++++++++++++++++++++++++++++++

Une question classique est la dérivation par rapport à \( x\) d'une fonction du type
\begin{equation}
    F(x)=\int f(x,t)dt.
\end{equation}
Il existe divers théorèmes qui répondent à ces questions. Dans notre cadre nous utiliserons le suivant.
\begin{theorem}     \label{ThoDerSousIntegrale}
    Soit \( A\) un ouvert de \( \eR\) et \( \Omega\), un espace mesuré. Soit une fonction \( f\colon A\times \Omega\to \eR\) qui satisfait
    \begin{enumerate}
        \item
            La fonction \( f\) est mesurable en tant que fonction \( A\times\Omega\to \eR\). Pour chaque \( x\in A\), la fonction \( f(x,\cdot)\) est intégrable sur \( \Omega\).
        \item
            Pour presque tout \( \omega\in\Omega\), la fonction \( f(x,\omega)\) est une fonction absolument continue de \( x\).
        \item
            La fonction \( \frac{ \partial f }{ \partial x }\) est localement intégrable, c'est à dire que pour tout \( \mathopen[ a , b \mathclose]\subset A\),
            \begin{equation}
                \int_a^b\int_{\Omega}\left| \frac{ \partial f }{ \partial x }(x,\omega) \right| d\omega\,dx<\infty.
            \end{equation}
    \end{enumerate}
    Alors la fonction de \( x\)
    \begin{equation}
        F(x)=\int_{\Omega}f(x,\omega)d\omega
    \end{equation}
    est absolument continue et pour presque tout \( x\in A\), la dérivée est donné par
    \begin{equation}
        \frac{ d }{ dx }\int_{\Omega}f(x,\omega)d\omega=\int_{\Omega}\frac{ \partial f }{ \partial x }(x,\omega)d\omega.
    \end{equation}
\end{theorem}

\begin{proposition}[Innégalité de Hölder]
    Soit \( \Omega\) un espace mesuré et \( 1\leq p\), \( q\leq\infty\) satisfaisant \( \frac{1}{ p }+\frac{1}{ q }=1\). Soient \( f\in L^p(\Omega)\), \( g\in L^q(\Omega)\). Alors le produit \( fg\) est dans \( L^1(\Omega)\) et nous avons
    \begin{equation}
        \| fg \|_1\leq \| f \|_p\| g \|_q.
    \end{equation}
\end{proposition}

\begin{remark}      \label{RemNormuptNird}
    Dans le cas d'un espace de probabilité, la fonction constante \( g=1\) appartient à \( L^p(\Omega)\). En prenant \( p=q=2\) nous obtenons
    \begin{equation}
        \| f \|_1\leq\| f \|_2.
    \end{equation}
\end{remark}

%+++++++++++++++++++++++++++++++++++++++++++++++++++++++++++++++++++++++++++++++++++++++++++++++++++++++++++++++++++++++++++
\section{Théorème de Fubini}
%+++++++++++++++++++++++++++++++++++++++++++++++++++++++++++++++++++++++++++++++++++++++++++++++++++++++++++++++++++++++++++

\begin{theorem}[Fubini-Tonelli]\index{Funibi-Tonelli}\index{théorème!Fubini-Tonelli}
    Soient \( (\Omega_i,\tribA_i,\mu_i)\) des espaces mesurés $\sigma$-finis. Si
    \begin{equation}
        f\colon \Omega_1\times\Omega_2\to \mathopen[ 0 , \infty \mathclose]
    \end{equation}
    est mesurable pour \( \mu_1\otimes \mu_2\), alors
    \begin{enumerate}
        \item
            Les fonctions
            \begin{subequations}
                \begin{align}
                    x&\mapsto\int_{\Omega_2}f(x,y)d\mu_2(y)\\
                    y&\mapsto\int_{\Omega_2}f(x,y)d\mu_1(x)
                \end{align}
            \end{subequations}
            sont mesurables
        \item
            Nous avons la formule pratique
            \begin{equation}
                \int_{\Omega_1\times\Omega_2}fd(\mu_1\otimes\mu_2)=\int_{\Omega_1}\left( \int_{\Omega_2}f(x,y)d\mu_2(y) \right)d\mu_1(x)
                =\int_{\Omega_2}\left( \int_{\Omega_1}f(x,y)d\mu_1(x) \right)d\mu_2(y).
            \end{equation}
    \end{enumerate}
\end{theorem}

\begin{theorem}
    Soient \( \mu_i\) des mesures \( \sigma\)-finies sur les espaces mesurables \( (\Omega_i,\tribA_i)\) ($i=1,2$). Nous considérons une fonction \( f\colon \Omega_1\otimes\Omega_2\to \eR,\eC\) qui soit mesurable pour la tribu \( \tribA_1\otimes \tribA_2\) et intégrable pour la mesure \( \mu_1\otimes \mu_2\). Alors
    \begin{enumerate}
        \item
            La fonction \( x\mapsto f(x,y)\) est \( \mu_1\)-intégrable pour presque tout \( y\) (par rapport à \( \mu_2\)).
        \item
            La fonction
            \begin{equation}
                y\mapsto\int_{\Omega_1}f(x,y)d\mu_1(x)
            \end{equation}
            est \( \mu_2\)-intégrable. 
        \item
            Nous avons la formule de Fubini
            \begin{equation}
                \int_{\Omega_1\times\Omega_2}fd(\mu_1\otimes\mu_2)=\int_{\Omega_2}\left( \int_{\Omega_1}f(x,y)d\mu_1(x)\right)d\mu_2(y).
            \end{equation}
    \end{enumerate}
\end{theorem}

\begin{example}
    Nous montrons que le théorème ne tient pas si une des deux mesures n'est pas \( \sigma\)-finie. Soit \( I=\mathopen[ 0 , 1 \mathclose]\). Nous considérons l'espace mesuré
    \begin{equation}
        (I,\Borelien(I),\lambda)
    \end{equation}
    où \( \Borelien(I)\) est la tribu des boréliens sur \( I\) et \( \lambda\) est la mesure de Lebesgue (qui est $\sigma$-finie). D'autre part nous considérons l'espace mesuré
    \begin{equation}
        (I,\partP(I),m)
    \end{equation}
    où \( \partP(I)\) est l'ensemble des parties de \( I\) et \( m\) est la mesure de comptage. Cette dernière n'est pas $\sigma$-finie parce que les seuls ensembles de mesure finie pour la mesure de comptage sont des ensembles finis, or une union dénombrable d'ensemble finis ne peut pas recouvrir l'intervalle \( I\).

    Nous allons montrer que dans ce cadre, l'intégrale de la fonction indicatrice de la diagonale sur \( I^2\) ne vérifie pas le théorème de Fubini. Étant donné que \( \Borelien(I)\subset\partP(I)\) nous avons
    \begin{equation}
        \Borelien(I^2)\subset\Borelien(I)\otimes\partP(I).
    \end{equation}
    Soit \( \Delta=\{ (x,x)\tq x\in I \}\). La fonction
    \begin{equation}
        \begin{aligned}
            g\colon I^2&\to \eR \\
            (x,y)&\mapsto x-y 
        \end{aligned}
    \end{equation}
    est continue et \( \Delta=g^{-1}(\{ 0 \})\) est donc fermé dans \( I^2\). L'ensemble \( \Delta\) est donc un borélien de \( I^2\) et par conséquent un élément de la tribu \( \Borelien(I)\otimes\partP(I)\). La fonction indicatrice \( \mtu_{\Delta}\) est alors mesurable pour l'espace mesuré
    \begin{equation}
        (I\times I,\Borelien(I)\otimes\partP(I),\lambda\otimes m).
    \end{equation}
    Pour \( x\) fixé nous avons
    \begin{equation}
        \mtu_{\Delta}(x,y)=\begin{cases}
            1    &   \text{si \( y= x\)}\\
            1    &    \text{si \( y\neq x\)}
        \end{cases}=\mtu_{\{ x \}}(y),
    \end{equation}
    et donc
    \begin{subequations}
        \begin{align}
            A_1&=\int_I\left( \int_I\mtu_{\Delta}(x,y)dm(y) \right)d\lambda(x)\\
            &=\int_I\left( \int_I\mtu_{\{ x \}}(y)dm(y) \right)d\lambda(x)\\
            &=\int_I\Big( m(\{ x \}) \Big)d\lambda(x)\\
            &=\int_I 1d\lambda(x)\\
            &=1.
        \end{align}
    \end{subequations}
    Par contre le support de \( \mtu_{\Delta}\) étant de mesure nulle pour la mesure de Lebesgue, nous avons
    \begin{equation}
        \int_I\mtu_{\Delta}(x,y)d\lambda(x)=0
    \end{equation}
    et par conséquent
    \begin{equation}
        A_2=\int_I\left( \int_I\mtu_{\Delta}(x,y)d\lambda(x) \right)dm(y)=0.
    \end{equation}
    Nous voyons donc que le théorème de Fubini ne s'applique pas.
\end{example}

\begin{example}
    Le théorème de Fubini est utilisé dans le calcul de l'intégrale gaussienne
    \begin{equation}
        G=\int_{\eR} e^{-x^2}dx.
    \end{equation}
    Par symétrie nous pouvons nous contenter de calculer
    \begin{equation}
        G_+=\int_0^{\infty} e^{-x^2}dx.
    \end{equation}
    L'astuce est de passer par l'intermédiaire
    \begin{subequations}
        \begin{align}
            H&=\int_{\eR^+\times\eR^+} e^{-(x^2+y^2)}dxdy       \label{EqIntFausasub}\\
            &=\int_{\eR^+}\left( \int_{\eR^+} e^{-x^2} e^{-y^2}dx \right)dy\\
            &=\left( \int_I e^{-x^2} dx\right)^2\\
            &=G_+^2
        \end{align}
    \end{subequations}
    L'intégrale \eqref{EqIntFausasub} se calcule en passant aux coordonnées polaires et le résultat est \( H=\frac{ \pi }{ 4 }\). Nous avons alors \( G=\frac{ \sqrt{\pi} }{ 2 }\) et
    \begin{equation}
        \int_{\eR} e^{-x^2}=\sqrt{\pi}.
    \end{equation}
\end{example}

\begin{example} \label{ExempInversSumIntFub}
    Le théorème de Fubini-Tonelli nous permet également d'inverser des sommes et des séries. En effet une somme n'est rien d'autre qu'une intégrale pour la mesure de comptage :
    \begin{equation}
        \sum_{n=0}^{\infty}a_n=\int_{\eN}a_ndm(n).
    \end{equation}
    Considérons une suite de fonctions \( f_n\colon \eR^d\to \eR\) \emph{positives}, la quantité
    \begin{equation}    \label{EqAcalculParFubIntSum}
        I=\sum_{n=0}^{\infty}\int_{\eR^n}f_n(x)dx
    \end{equation}
    et les espaces mesurés \( (\eN,\partP(\eN),m)\), \( (\eR^n,\Borelien(\eR^n),\lambda)\) où \( \lambda\) est la mesure de Lebesgue. En écrivant la formule \eqref{EqAcalculParFubIntSum}, nous supposons que pour chaque \( n\), la fonction \( f_n\) est intégrable sur \( \eR^d\) et que le résultat soit sommable. Nous pouvons la récrire sous la forme
    \begin{equation}
        \int_{\eN}\left( \int_{\eR^n}f(n,x)dx \right)dm(n)
    \end{equation}
    avec la notation évidente \( f(n,x)=f_n(x)\). Prouvons que la fonction \( f\colon \eN\times\eR^d\to \eR\) ainsi définie est une fonction mesurable pour l'espace mesuré
    \begin{equation}
        \big( \eN\times\eR^d,\partP(\eN)\otimes\Borelien(\eR^d),m\otimes\lambda \big).
    \end{equation}
    Si \( A\subset\eR\), nous avons
    \begin{equation}
        f^{-1}(A)=\bigcup_{n\in\eN}\{ n \}\times f_n^{-1}(A).
    \end{equation}
    Chacun des ensembles dans l'union appartient à la tribu \( \partP(\eN)\times\Borelien(\eR^d)\) tandis que les tribus sont stables sous les unions dénombrables. La fonction \( f\) est donc mesurable. La fonction \( f\) est donc mesurable. Comme nous avons supposé que \( f\) était positive, le théorème de Fubini-Tonelli s'applique et nous avons
    \begin{equation}
        I=\int_{\eR^d}\left( \int_{\eN}f(n,x)dm(n) \right)dx=\int_{\eR^d}\sum_{n\in \eN}f_n(x)dx.
    \end{equation}
\end{example}

%+++++++++++++++++++++++++++++++++++++++++++++++++++++++++++++++++++++++++++++++++++++++++++++++++++++++++++++++++++++++++++
\section{Théorème de Stone-Weierstrass}
%+++++++++++++++++++++++++++++++++++++++++++++++++++++++++++++++++++++++++++++++++++++++++++++++++++++++++++++++++++++++++++

\begin{definition}
    Nous disons qu'une algèbre \( A\) de fonctions sur un espace \( X\) \defe{sépare les points}{sépare!les points} de \( X\) si pour tout \( x_1\neq x_2\) il existe \( g\in A\) telle que \( g(x_1)\neq g(x_2)\).
\end{definition}


\begin{theorem}[Stone-Weierstrass]\index{théorème!Stone-Weierstrass}\label{ThoWmAzSMF}
    Soit \( X\), un espace compact et Hausdorff et \( A\) une sous algèbre de \( C(X,\eR)\) contenant une fonction constante non nulle. Alors \( A\) est dense dans \( C(X,\eR)\) si et seulement si \( A\) sépare les points de \(X\).
\end{theorem}

%+++++++++++++++++++++++++++++++++++++++++++++++++++++++++++++++++++++++++++++++++++++++++++++++++++++++++++++++++++++++++++
\section{Théorème du point fixe de Picard}
%+++++++++++++++++++++++++++++++++++++++++++++++++++++++++++++++++++++++++++++++++++++++++++++++++++++++++++++++++++++++++++

\begin{definition}
    Une application \( f\colon (X,\| . \|_X)\to (Y,\| . \|_Y)\) entre deux espaces métriques est \defe{contractante}{contractante} si elle est \( k\)-\defe{Lipschitz}{Lipschitz} pour un certain \( 0\leq k<1\), c'est à dire si pour tout \( x,y\in X\) nous avons
    \begin{equation}
        \| f(x)-f(y) \|_Y\leq k\| x-y \|_{X}.
    \end{equation}
\end{definition}

\begin{theorem}[Picard]     \label{ThoEPVkCL}\index{théorème!Picard}
    Soit \( (X,d)\) un espace métrique complet et \( f\colon X\to X\) une application contractante. Alors \( f\) admet un unique point fixe. De plus pour tout \( x_0\in X\), l'unique point fixe de \( f\) est donné par la limite de la suite définie par
    \begin{equation}
        x_{n+1}=f(x_n).
    \end{equation}
\end{theorem}

\begin{remark}
    Nous verrons dans la sous section \ref{subseciKpMFx} que la suite convergente donnée par le théorème de Picard est résistante aux erreurs. En effet si à chaque étape du calcul nous commettons une erreur (pas trop grande), alors la suite obtenue converge également vers le point fixe cherché. Cette propriété se transmet automatiquement aux théorèmes de Cauchy-Lipschitz et à la méthode de Newton.
\end{remark}

\begin{remark}  \label{remIOHUJm}
    Si \( f\) elle-même n'est pas contractante, mais si \( f^p\) est contractante pour un certain \( p\in \eN\) alors la conclusion du théorème de Picard reste valide et \( f\) a le même unique point fixe que \( f^p\). En effet nommons \( x\) le point fixe de \( f\) : \( f^p(x)=x\). Nous avons alors
    \begin{equation}
        f^p\big( f(x) \big)=f\big( f^p(x) \big)=f(x),
    \end{equation}
    ce qui prouve que \( f(x)\) est un point fixe de \( f^p\). Par unicité nous avons alors \( f(x)=x\), c'est à dire que \( x\) est également un point fixe de \( f\).

    Cette remarque est le sujet d'une partie de l'exercice \ref{exoTP20090002}
\end{remark}

Si la fonction n'est pas Lipschitz mais presque, nous avons une variante.
\begin{proposition}
    Soit \( E\) un ensemble compact\footnote{Notez cette hypothèse plus forte} et si \( f\colon E\to E\) est une fonction telle que
    \begin{equation}        \label{EqLJRVvN}
        \| f(x)-f(y) \|< \| x-y \|
    \end{equation}
    pour tout \( x\neq y\) dans \( E\) alors \( f\) possède un unique point fixe.
\end{proposition}

\begin{proof}
    La suite \( x_{n+1}=f(x_n)\) possède une sous suite convergente. La limite de cette sous suite est un point fixe de \( f\) parce que \( f\) est continue. L'unicité est due à l'aspect strict de l'inégalité \eqref{EqLJRVvN}.
\end{proof}

%---------------------------------------------------------------------------------------------------------------------------
\subsection{Théorème de Cauchy-Lipschitz}
%---------------------------------------------------------------------------------------------------------------------------

\begin{definition}
    Une fonction 
    \begin{equation}
        \begin{aligned}
            f\colon \eR^n\times R^m&\to \eR^p \\
            (t,y)&\mapsto f(t,y) 
        \end{aligned}
    \end{equation}
    est \defe{localement Lipschitz}{Lipschitz!localement} en \( y\) au point \( (t_0,y_0)\) si il existe des voisinages \( V\) de \( t_0\) et \( W\) de \( y_0\) et un nombre \( k>0\) tels que pour tout \( (t,y)\in V\times W\) on ait
    \begin{equation}
        \big\| f(t_0,y_0)-f(t,y) \big\|\leq k\| y-y_0 \|.
    \end{equation}
    La fonction est localement Lipschitz sur un ouvert \( U\) de \( \eR^n\times \eR^m\) si elle est localement Lipschitz en chaque point de \( U\).
\end{definition}

\begin{lemma}       \label{LemdLKKnd}
    Soient \( A\) et \( B\) deux espaces compact. L'ensemble des fonctions continues de \( A\) vers \( B\) muni de la norme uniforme est complet.
\end{lemma}

\begin{proof}
    Soit \( (f_k)\) une suite de Cauchy de fonctions dans \( C(A,B)\). Pour chaque \( x\in A \) nous avons
    \begin{equation}
        \| f_k(x)-f_l(x) \|_B\leq \| f_k-f_l \|_{\infty},
    \end{equation}
    de telle sorte que la suite \( (f_k(x))\) est de Cauchy dans \( B\) et converge donc vers un élément de \( B\). La suite de Cauchy \( (f_k)\) converge donc vers une fonction \( f\colon A\to B\). Nous devons encore voir que cette fonction est continue; ce sera l'uniformité de la norme qui donnera la continuité. En effet soit \( x_n\to x\) une suite dans \( A\) convergent vers \( x\in A\). Pour chaque \( k\in \eN\) nous avons
    \begin{equation}
        \| f(x_n)-f(x) \|\leq \| f(x_n)-f_k(x_n) \|  +\| f_k(x_n)-f_k(x) \|+\| f_k(x)-f(x) \|.
    \end{equation}
    En prenant \( k\) et \( n\) assez grands, cette expression peut être rendue aussi petite que l'on veut. La suite \( f(x_n)\) est donc convergente vers \( f(x)\) et la fonction \( f\) est continue.
\end{proof}

%\newpage

%\url{http://student.ulb.ac.be/~lclaesse/mes_notes.pdf}

\begin{theorem}[Cauchy-Lipschitz\cite{SandrineCL}]\index{théorème!Cauchy-Lipschitz}\label{ThokUUlgU}
    Nous considérons l'équation différentielle
    \begin{subequations}        \label{XtiXON}
        \begin{numcases}{}
            y'=f(t,y)\\
            y(t_0)=y_0
        \end{numcases}
    \end{subequations}
    avec \( f\colon U\to \eR^n\) où \( U\) est un ouvert de \( \eR\times \eR^n\). Nous supposons que \( f\) est continue sur \( U\) et localement Lipschitz\footnote{Nous ne supposons pas que \( f\) soit une contraction.} par rapport à \( y\). Alors le système \eqref{XtiXON} admet une unique solution maximale. Cette solution est \( C^1\). 
\end{theorem}

\begin{remark}
    L'écriture «\( y'=f(t,y)\)» est un abus de notation pour demander que pour chaque \( t\) nous demandons \( y'(t)=f(t,y(t))\).
\end{remark}

\begin{proof}
    Si \( y\) est une solution de l'équation différentielle considérée, elle vérifie
    \begin{equation}        \label{EqPGLwcL}
        y(t)=y_0+\int_{t_0}^tf\big( u,y(u) \big)du.
    \end{equation}
    Ceci nous incite à considérer l'opérateur \( \Phi\colon \mF\to \mF\) défini par
    \begin{equation}
        \Phi(y)(t)=y_0+\int_{t_0}^tf\big( u,y(u) \big)du.
    \end{equation}
    Précisons l'espace fonctionnel \( \mF\) adéquat. Soient \( V\) et \( W\) les voisinages de \( t_0\) et \( y_0\) sur lesquels \( f\) est localement Lipschitz. Nous considérons les quantités suivantes :
    \begin{enumerate}
        \item
            \( M=\sup_{V\times W}f\) ;
        \item
            \( r>0\) tel que \( \overline{ B(y_0,r) }\subset V\)
        \item
            \( T>0\) tel que \( \overline{ B(t_0,T) }\subset W\) et \( T<r/M\).
    \end{enumerate}
    Nous considérons alors \( \mF\), l'ensemble des fonctions continues \( \overline{ B(t_0,T) }\to \overline{ B(y_0,r) }\) muni de la norme uniforme. Par le lemme \ref{LemdLKKnd} l'espace \( \mF\) est complet.

    Le fait que \( \Phi(y)\) soit continue lorsque \( y\) est continue est une propriété de l'intégration et du fait que \( f\) soit continue en ses deux variables. Prouvons que \( \Phi(y)(t)\in\overline{ B(y_0,r) }\). Pour cela, notons que
    \begin{equation}
        | \Phi(y)(t)-y_0 |\leq \int_{t_0}^t |f\big( u,y(u) \big)|du\leq | t-t_0 |\| f \|_{\infty}.
    \end{equation}
    Étant donné que \( t\in\overline{ B(t_0,T) }\) nous avons \( | t-t_0 |\leq r/M\) et donc \( | \Phi(y)(t)-y_0 |\leq r\).

    L'équation \eqref{EqPGLwcL} signifie que \( y\) est un point fixe de \( \Phi\). L'espace \( \mF\) étant complet le théorème de point fixe de Picard (théorème \ref{ThoEPVkCL}) s'applique. Nous allons montrer qu'il existe un \( p\in\eN\) tel que \( \Phi^p\) soit contractante. Par conséquent \( \Phi^p\) aura un unique point fixe qui sera également unique point fixe de \( \Phi\) par la remarque \ref{remIOHUJm}.
    
    Prouvons donc que \( \Phi^p\) est contractante pour un certain \( p\). Pour cela nous commençons par montrer la formule suivante par récurrence :
    \begin{equation}        \label{EqRAdKxT}
        \big\| \Phi^p(x)(t)-\Phi^p(y)(t) \big\|\leq \frac{ k^p| t-t_0 |^p }{ p! }\| x-y \|_{\infty}
    \end{equation}
    pour tout \( x,y\in\mF\), et pour tout \( t\in\overline{ B(t_0,T) }\). Pour \( p=0\) la formule \eqref{EqRAdKxT} est vérifiée parce que \( \| x-y \|_{\infty}\) est le supremum de \( \| x(t)-y(t) \|\) pour \( t\in\overline{ B(t_0,T) }\). Supposons que la formule soit vraie pour \( p\) et calculons pour \( p+1\). Pour tout \( t\in\overline{ B(t_0,T) }\) nous avons
    \begin{subequations}
        \begin{align}
            \big\| \Phi^{p+1}(x)(t)-\Phi^{p+1}(y)(t) \big\|&\leq \left| \int_{t_0}^t\big\| f\big( u,\Phi^p(x)(u) \big)-f\big( u,\Phi^p(y)(u) \big) \big\|du \right| \\
            &\leq \left| \int_{t_0}^tk\| \Phi^p(x)(u)-\Phi^p(y)(u) \|du \right|    \label{subIKYixF}\\
            &\leq \left| \int_{t_0}^tk\frac{ k^p| t-t_0 | }{ p! }\| x-y \|_{\infty} \right| \label{subxkNjiV} \\
            &=\frac{ k^{p+1}| t-t_0 |^{p+1} }{ (p+1)! }\| x-y \|_{\infty}.
        \end{align}
    \end{subequations}
    Justifications :
    \begin{itemize}
        \item \eqref{subIKYixF} parce que \( f\) est Lipschitz.
        \item \eqref{subxkNjiV} par hypothèse de récurrence.
    \end{itemize}
    La formule \eqref{EqRAdKxT} est maintenant établie. Nous pouvons maintenant montrer que \( \Phi^p\) est une contraction pour un certain \( p\). Pour tout \( t\in \overline{ B(t_0,T) }\) nous avons
    \begin{subequations}
        \begin{align}
        \big\| \Phi^p(x)-\Phi^p(y) \big\|_{\infty}&\leq \| \Phi^p(x)(t)-\Phi^p(y)(t) \|\\
        &\leq \frac{ k^p }{ t! }| t-t_0 |^p\| x-y \|_{\infty}\\
        &\leq \frac{ k^pT^p }{ p! }\| x-y \|_{\infty}
        \end{align}
    \end{subequations}
    où nous avons utilisé le fait que \( | t-t_0 |^p<T^p\). Le membre de droite tend vers zéro lorsque \( p\to\infty\) parce que \( k<1\) et \( T^p/p!\to 0\). Nous concluons donc que \( \Phi^p\) est une contraction pour un certain \( p\).

    L'unique point fixe de \( \Phi\) est alors l'unique solution continue de l'équation différentielle \eqref{XtiXON}. Par ailleurs l'équation elle-même \( y'=f(t,y)\) demande implicitement que \( y\) soit dérivable et donc continue. Nous concluons que l'unique point fixe de \( \Phi\) est l'unique solution de l'équation différentielle donnée. Cette dernière est automatiquement \( C^1\) parce que si \( y\) est continue alors \( u\mapsto f(u,y(u))\) est continue, c'est à dire que \( y'\) est continue.


    Nous passons maintenant à la partie «prolongement maximum» du théorème. Soient \( x_1\) et \( x_2\) deux solutions maximales du problème \eqref{XtiXON} sur des intervalles \( I_1\) et \( I_2\) respectivement. Les intervalles \( I_1\) et \( I_2\) contiennent \( \overline{ B(t_0,r) }\) sur lequel \( x_1=x_2\) par unicité.
    
    
    Nous allons maintenant montrer que pour tout \( t\geq t_0\) pour lequel \( x_1\) ou \( x_2\) est défini, \( x_1(t)\) et \( x_2(t)\) sont définis et sont égaux. Le raisonnement sur \( t\leq t_0\) est similaire.
    
    Supposons que l'ensemble des \( t\geq t_0\) tels que \( x_1=x_2\) soit ouvert à droite, c'est à dire soit de la forme \( \mathopen[ t_0 ,b [\). Dans ce cas, soit \( x_1\) soit \( x_2\) (soit les deux) cesse d'exister en \( b\). En effet si nous avions les fonctions \( x_i\) sur \(\mathopen[ t_0 , b+\epsilon [\) alors l'équation \( x_1=x_2\) définirait un fermé dans \( \mathopen[ t_0 , b+\epsilon [\). Supposons pour fixer les idées que \( x_1\) cesse d'exister : le domaine de \( x_1\) (parmi les \( t\geq 0\)) est \( \mathopen[ t_0 , b [\) et sur ce domaine nous avons \( x_1=x_2\). Dans ce cas \( x_1\) pourrait être prolongé en \( x_2\) au-delà de \( b\). Si \( x_1\) et \( x_2\) s'arrêtent d'exister en même temps en \( b\), alors nous avons bien \( x_1=x_2\).

    Nous devons donc traiter le cas où \( x_1=x_2\) sur \( \mathopen[ t_0 , b \mathclose]\) alors que \( x_1\) et \( x_2\) existent sur \( \mathopen[ t_0 , b+\epsilon [\) pour un certain \( \epsilon\).

    Nous pouvons appliquer le théorème d'existence locale au problème
    \begin{subequations}
        \begin{numcases}{}
            y'=f(t,y)\\
            y(b)=x_1(b).
        \end{numcases}
    \end{subequations}
    Il existe un voisinage de \( b\) sur lequel la solution est unique. Sur ce voisinage nous devons donc avoir \( x_1=x_2\), ce qui contredit le fait que \( x_1\neq x_2\) en dehors de \( \mathopen[ t_0 , b \mathclose]\).
\end{proof}

%\newpage

%---------------------------------------------------------------------------------------------------------------------------
\subsection{Équation de Fredholm}
%---------------------------------------------------------------------------------------------------------------------------

\begin{theorem}[Équation de Fredholm]\index{Fredholm!équation}\index{équation!Fredholm}     \label{ThoagJPZJ}
    Soit \( K\colon \mathopen[ a , b \mathclose]\times \mathopen[ a , b \mathclose]\to \eR\) et \( \varphi\colon \mathopen[ a , b \mathclose]\to \eR\), deux fonctions continues. Alors si \( \lambda\) est suffisamment petit, l'équation
    \begin{equation}
        f(x)=\lambda\int_a^bK(x,y)f(y)dy+\varphi(x)
    \end{equation}
    admet une unique solution qui sera de plus continue sur \( \mathopen[ a , b \mathclose]\).
\end{theorem}

\begin{proof}
    Nous considérons l'ensemble \( \mF\) des fonctions continues \( \mathopen[ a , b \mathclose]\to\mathopen[ a , b \mathclose]\) muni de la norme uniforme. Le lemme \ref{LemdLKKnd} implique que \( \mF\) est complet. Nous considérons l'application \( \Phi\colon \mF\to \mF\) donnée par
    \begin{equation}
        \Phi(f)(x)=\lambda\int_a^bK(x,y)f(y)dy+\varphi(x). 
    \end{equation}
    Nous montrons que \( \Phi^p\) est une application contractante pour un certain \( p\). Pour tout \( x\in \mathopen[ a , b \mathclose]\) nous avons
    \begin{subequations}
        \begin{align}
            \| \Phi(f)-\Phi(g) \|_{\infty}&\leq \| \Phi(f)(x)-\Phi(g)(x) \|\\
            &=| \lambda |\Big\| \int_a^bK(x,y)\big( f(y)-g(y) \big)dy  \Big\|\\
            &\leq | \lambda |\| K \|_{\infty}| b-a |\| f-g \|_{\infty}
        \end{align}
    \end{subequations}
    Si \( \lambda\) est assez petit, et si \( p\) est assez grand, l'application \( \Phi^p\) est donc une contraction. Elle possède donc un unique point fixe par le théorème de Picard \ref{ThoEPVkCL}.
\end{proof}

%---------------------------------------------------------------------------------------------------------------------------
\subsection{Théorème d'inversion locale}
%---------------------------------------------------------------------------------------------------------------------------

Le théorème suivant est une conséquence du théorème de point fixe de Picard \ref{ThoEPVkCL}.
\begin{theorem}[Inversion locale]
    Soit \( a\in U\) avec \( U\), un ouvert de \( \eR^n\) et \( f\colon U\to \eR^n\), une application \( C^1\) telle que \( df_a\) soit inversible. Alors il existe un voisinage \( V\) de \( a\) et un voisinage \( W\) de \( f(a)\) tels que \( f\colon V\to W\) soit un homéomorphisme.
\end{theorem}

\begin{example}
    Est-ce que l'équation \( e^{y}+xy=0\) définit au moins localement une fonction \( y(x)\) ? Nous considérons la fonction
    \begin{equation}
        f(x,y)=\begin{pmatrix}
            x    \\ 
            e^{y}+xy    
        \end{pmatrix}
    \end{equation}
    La différentielle de cette application est
    \begin{subequations}
        \begin{align}
            df_{(0,0)}(u)&=\frac{ d }{ dt }\Big[ f(tu_1,tu_2) \Big]_{t=0}\\
            &=\frac{ d }{ dt }\begin{pmatrix}
                tu_1    \\ 
                e^{tu_2}+t^2u_1u_2    
            \end{pmatrix}_{t=0}\\
            &=\begin{pmatrix}
                u_1    \\ 
                u_2    
            \end{pmatrix}.
        \end{align}
    \end{subequations}
    L'application \( f\) définit donc un difféomorphisme local autour des points \( (x_0,y_0)\) et \( f(x_0,y_0)\). Soit \( (u,0)\) un point dans le voisinage de \( f(x_0,y_0)\). Alors il existe un unique \( (x,y)\) tel que
    \begin{equation}
        f(x,y)=\begin{pmatrix}
               x \\ 
            e^y+xy    
        \end{pmatrix}=
        \begin{pmatrix}
            u    \\ 
                0
        \end{pmatrix}.
    \end{equation}
    Nous avons automatiquement \( x=u\) et \( e^y+xy=0\). Notons toutefois que pour que ce procédé donne effectivement une fonction implicite \( y(x)\) nous devons avoir des points de la forme \( (u,0)\) dans le voisinage de \( f(x_0,y_0)\).
\end{example}

%+++++++++++++++++++++++++++++++++++++++++++++++++++++++++++++++++++++++++++++++++++++++++++++++++++++++++++++++++++++++++++
					\section{Théorème de la fonction implicite}
%+++++++++++++++++++++++++++++++++++++++++++++++++++++++++++++++++++++++++++++++++++++++++++++++++++++++++++++++++++++++++++

%---------------------------------------------------------------------------------------------------------------------------
\subsection{Mise en situation}
%---------------------------------------------------------------------------------------------------------------------------

Dans un certain nombre de situation, il n'est pas possible de trouver des solutions explicites aux équations qui apparaissent. Néanmoins, l'existence «théorique» d'une telle solution est souvent déjà suffisante. C'est l'objet du théorème de la fonction implicite.

Prenons par exemple la fonction sur $\eR^2$ donnée par 
\begin{equation}
	F(x,y)=x^2+y^2-1.
\end{equation}
Nous pouvons bien entendu regarder l'ensemble des points donnés par $F(x,y)=0$. C'est le cercle dessiné à la figure \ref{LabelFigCercleImplicite}.
\newcommand{\CaptionFigCercleImplicite}{Un cercle pour montrer l'intérêt de la fonction implicite.}
\input{Fig_CercleImplicite.pstricks}
Nous ne pouvons pas donner le cercle sous la forme $y=y(x)$ à cause du $\pm$ qui arrive quand on prend la racine carrée. Mais si on se donne le point $P$, nous pouvons dire que \emph{autour de $P$}, le cercle est la fonction
\begin{equation}
	y(x)=\sqrt{1-x^2}.
\end{equation}
Tandis que autour du point $P'$, le cercle est la fonction
\begin{equation}
	y(x)=-\sqrt{1-x^2}.
\end{equation}
Autour de ces deux point, donc, le cercle est donné par une fonction. Il n'est par contre pas possible de donner le cercle autour du point $Q$ sous la forme d'une fonction.

Ce que nous voulons faire, en général, est de voir si l'ensemble des points tels que
\begin{equation}
	F(x_1,\ldots,x_n,y)=0
\end{equation}
peut être donné par une fonction $y=y(x_1,\ldots,x_n)$. En d'autre termes, est-ce qu'il existe une fonction $y(x_1,\ldots,x_n)$ telle que
\begin{equation}
	F\big( x_1,\ldots,x_n,y(x_1,\ldots,x_n)\big)=0.
\end{equation}



\subsection{Définitions et rappels}
Soit
\begin{equation*}
  F : D \subset (\RR^n \times \RR^m) \to \RR^m : (x,y) \mapsto F(x,y) =
  (F_1(x,y),\ldots,F_m(x,y))
\end{equation*}
avec $x = (x_1,\ldots, x_n)$ et $y = (y_1,\ldots,y_m)$.
% Pour chaque $x$ fixé, on s'intéresse aux solutions du système de $m$
% équations $F(x,y) = 0$ pour les inconnues $y$ ; en particulier, on
% voudrait pouvoir écrire $y = \varphi(x)$ vérifiant $F(x,\varphi(x))
% = 0$.

Pour $(x,y) \in \interieur D$, la matrice
\begin{equation*}
\begin{pmatrix}
\pder {F_1}{y_1}(x,y)& \ldots& \pder {F_1}{y_m}(x,y)\\
\vdots& \ddots & \vdots\\
\pder {F_m}{y_1}(x,y)& \ldots& \pder {F_m}{y_m}(x,y)\\
\end{pmatrix}
\end{equation*}
est la \defe{matrice jacobienne}{jacobienne!matrice} de $F$ par rapport à $y$ (au point
$(x,y)$; son déterminant est appelé le \defe{jacobien}{jacobien} de F par
    rapport à y et se note
  $\pder{(F_1,\ldots,F_m)}{(y_1,\ldots,y_m)}(x,y)$.


\begin{theorem}
	Soit $(\bar x,\bar y)$ tel que $F(\bar x,\bar y) = 0$ et
  $\pder{(F_1,\ldots,F_m)}{(y_1,\ldots,y_m)}(\bar x,\bar y) \neq
  0$. Alors il existe un voisinage $U$ de $x$ dans $\RR^n$, un
  voisinage $V$ de $y$ dans $\RR^m$ et une unique application $\varphi
  : U \to V$ tels que
  \begin{enumerate}
  \item $\varphi(\bar x) = \bar y$ ; 
  \item $F(x,\varphi(x)) = 0$ pour tout $x \in U$.
  \end{enumerate}
\end{theorem}
	
Le théorème de la fonction implicite a pour objet de donner l'existence de la fonction $\varphi$. Maintenant nous pouvons dire beaucoup de choses sur les dérivées de $\varphi$ en considérant la fonction
\begin{equation}
	x\mapsto F\big( x,\varphi(x) \big).
\end{equation}
Par définition de $\varphi$, cette fonction est toujours nulle. En particulier, nous pouvons dériver l'équation
\begin{equation}
	F\big( x,\varphi(x) \big)=0,
\end{equation}
et nous trouvons plein de choses.

%---------------------------------------------------------------------------------------------------------------------------
\subsection{Exemple}
%---------------------------------------------------------------------------------------------------------------------------

Prenons par exemple la fonction
\begin{equation}
	F\big( (x,y),z \big)=ze^z-x-y,
\end{equation}
et demandons nous ce que nous pouvons dire sur la fonction $z(x,y)$ telle que
\begin{equation}
	F\big( x,y,z(x,y) \big)=0,
\end{equation}
c'est à dire telle que
\begin{equation}		\label{EqDefZImplExemple}
	z(x,y) e^{z(x,y)}-x-y=0.
\end{equation}
pour tout $x$ et $y\in\eR$. Nous pouvons facilement trouver $z(0,0)$ parce que
\begin{equation}
	z(0,0) e^{z(0,0)}=0,
\end{equation}
donc $z(0,0)=0$.

Nous pouvons dire des choses sur les dérivées de $z(x,y)$. Voyons par exemple $(\partial_xz)(x,y)$. Pour trouver cette dérivée, nous dérivons la relation \eqref{EqDefZImplExemple} par rapport à $x$. Ce que nous trouvons est
\begin{equation}
	(\partial_xz)e^z+ze^z(\partial_xz)-1=0.
\end{equation}
Cette équation peut être résolue par rapport à $\partial_xz$~:
\begin{equation}
	\frac{ \partial z }{ \partial x }(x,y)=\frac{1}{ e^z(1+z) }.
\end{equation}
Remarquez que cette équation ne donne pas tout à fait la dérivée de $z$ en fonction de $x$ et $y$, parce que $z$ apparaît dans l'expression, alors que $z$ est justement la fonction inconnue. En général, c'est la vie, nous ne pouvons pas faire mieux.

Dans certains cas, on peut aller plus loin. Par exemple, nous pouvons calculer cette dérivée au point $(x,y)=(0,0)$ parce que $z(0,0)$ est connu :
\begin{equation}
	\frac{ \partial z }{ \partial x }(0,0)=1.
\end{equation}
Cela est pratique pour calculer, par exemple, le développement en Taylor de $z$ autour de $(0,0)$.

%+++++++++++++++++++++++++++++++++++++++++++++++++++++++++++++++++++++++++++++++++++++++++++++++++++++++++++++++++++++++++++
\section{Théorèmes de Brouwer et Schauder}
%+++++++++++++++++++++++++++++++++++++++++++++++++++++++++++++++++++++++++++++++++++++++++++++++++++++++++++++++++++++++++++

\begin{proposition}
    Soit \( f\colon \mathopen[ 0 , 1 \mathclose]\to \mathopen[ 0 , 1 \mathclose]\) une fonction continue. Alors \( f\) accepte un point fixe.
\end{proposition}

\begin{proof}
    En effet si nous considérons \( g(x)=f(x)-x\) alors nous avons \( g(0)=f(0)\geq 0\) et \( g(1)=f(1)-1\leq 0\). Si \( g(0)\) ou \( g(1)\) est nul, la proposition est démontrée; nous supposons donc que \( g(0)>0\) et \( g(1)<0\). La proposition découle à présent du théorème des valeurs intermédiaires.
\end{proof}

%---------------------------------------------------------------------------------------------------------------------------
\subsection{Formes différentielles}
%---------------------------------------------------------------------------------------------------------------------------

Nous allons donner une toute petite introduction aux formes différentielles sur des variétés compactes.

\begin{lemma}[\cite{SpindelGeomDoff}]       \label{LemdwLGFG}
    Soit \( \omega\) une \( k\)-forme sur \( \eR^n\) et \( f\), une fonction \( C^{\infty}\) sur \( \eR^n\). Alors \( d(f^*\omega)=f^*d\omega\).
\end{lemma}

\begin{proof}
    Nous effectuons la preuve par récurrence sur le degré de la forme. Soit d'abord une \( 0\)-forme, c'est à dire une fonction \( g\colon \eR^n\to \eR\). Nous avons
    \begin{equation}
        d(d^*g)X=d(g\circ f)X=(dg\circ df)X=dg\big( df X \big)=(f^*dg)(X).
    \end{equation}
    
    Supposons maintenant que le résultat soit exact pour toute les \( p-1\) formes et montrons qu'il reste valable pour les \( p\)-formes. Par linéarité de la différentielle nous pouvons nous contenter de considérer la forme différentielle
    \begin{equation}
        \omega=g\,dx^1\wedge\ldots dx^p
    \end{equation}
    où \( g\) est une fonction \(  C^{\infty}\). Pour soulager les notations nous allons noter \( dx^I=dx^1\wedge\ldots dx^{p-1}\). Nous avons
    \begin{subequations}
        \begin{align}
            d(f^*\omega)&=d\big( f^*(gdx^I\wedge dx^p) \big)\\
            &=d\big( f^*(gdx^I)\wedge f^*dx^p \big)\\
            &=d\big( f^*(gdx^I)\big)\wedge f^*dx^p+(-1)^{p-1}f^*(gdx^I)\wedge(f^*dx^p)  \label{gnAnSt}\\
            &=f^*\big( d(gdx^I) \big)\wedge f^*dx^p      \label{xZrfjZ}\\
            &=f^*\big( d(gdx^I)\wedge dx^p \big)\\
            &=f^*d\omega        \label{loWUji}
        \end{align}
    \end{subequations}
    Justifications : \eqref{gnAnSt} est la formule de Leibnitz. \eqref{xZrfjZ} est parce que le second terme est nul : \( d(f^*dx^p)=f^*(d^2x^p)=0\). Nous avons utilisé l'hypothèse de récurrence et le fait que \( d^2=0\). L'étape \eqref{loWUji} est une utilisation à l'envers de la règle de Leibnitz en tenant compte que \( d^2x^p=0\).
\end{proof}

Soit \( M\) une variété de dimension \( n\) et \( \omega\) une \( n\)-forme différentielle
\begin{equation}
    \omega_p=f(p)dx_1\wedge\ldots\wedge dx_n.
\end{equation}
 Si \( (U,\varphi)\) est une carte (\( U\subset\eR^n\) et \( \varphi\colon U\to M\)) alors nous définissons
\begin{equation}
    \int_{\varphi(U)}\omega=\int_{U}f\big( \varphi(x) \big)dx_1\ldots dx_n.     
\end{equation}
Lorsque nous voulons intégrer sur une partie plus grande qu'une carte nous utilisons une partition de l'unité.
\begin{lemma}   \label{LemGPmRGZ}
    Soit \( \{ U_i \}\) un recouvrement de \( M\) par un nombre fini d'ouverts\footnote{Si \( M\) n'est pas compacte, alors il faut utiliser une version un peu plus élaborée du lemme\cite{SpindelGeomDoff}.}. Alors il existe une famille de fonctions \( f_i\in  C^{\infty}(M)\) telle que
    \begin{enumerate}
        \item
            \( \supp f_i\subset U_i\),
        \item
            pour tout \( i\), nous avons \( f_i\geq 0\),
        \item
            pour tout \( p\in M\) nous avons \( \sum_i f_i(p)=1\).
    \end{enumerate}
\end{lemma}
La famille \( (f_i)\) est une \defe{partition de l'unité}{partition!de l'unité} subordonnée au recouvrement \( \{ U_i \}\). Si \( \{ f_i \}\) est une partition de l'unité subordonnée à un atlas de \( M\) nous définissons
\begin{equation}
    \int_M\omega=\sum_i\int_{U_i}f\omega.
\end{equation}
Il est possible de montrer que cette définition ne dépend pas du choix de la partition de l'unité.

\begin{remark}
    Nous ne définissons pas d'intégrale de \( k\)-forme différentielle sur une variété de dimension \( n\neq k\). Le seul cas où cela se fait est le cas de \( 0\)-formes (les fonctions), mais cela n'est pas vraiment un cas particulier vu que les \( 0\)-formes sont associées aux \( n\)-formes de façon évidente.
\end{remark}

%---------------------------------------------------------------------------------------------------------------------------
\subsection{Théorème de Brouwer}
%---------------------------------------------------------------------------------------------------------------------------

Nous commençons par énoncer et démontrer le théorème de Brouwer dans le cas des fonctions \(  C^{\infty}\) en utilisant le théorème de Stockes.
\begin{proposition}     \label{PropDRpYwv}
    Soit \( B\) la boule fermée de centre \( 0\) et de rayon \( 1\) de \( \eR^n\) et \( f\colon B\to B\) une fonction \(  C^{\infty}\). Alors \( f\) admet un point fixe.
\end{proposition}

\begin{proof}
    Supposons que \( f\) ne possède pas de points fixes. Alors pour tout \( x\in B\) nous considérons la ligne droite partant de \( x\) dans la direction de \( f(x)\) (cette droite existe parce que \( x\) et \( f(x)\) sont supposés distincts). Cette ligne intersecte \( \partial B\) en un point que nous appelons \( F(x)\). La fonction \( F\) ainsi définie vérifie deux propriétés :
    \begin{enumerate}
        \item
            elle est \(  C^{\infty}\) parce que \( f\) l'est;
        \item
            elle est l'identité sur \( \partial B\).
    \end{enumerate}
    La suite de la preuve consiste à montrer qu'une telle rétraction sur \( B\) ne peut pas exister\footnote{Notons qu'il n'existe pas non plus de rétractions continues sur \( B\), mais pour le montrer il faut utiliser d'autres méthodes que Stockes, ou alors présenter les choses dans un autre ordre.}.

    Nous considérons une forme de volume \( \omega\) sur \( \partial B\) : l'intégrale de \( \omega\) sur \( \partial B\) est la surface de \( \partial B\) qui est non nulle. Nous avons alors
    \begin{equation}
        0<\int_{\partial B}\omega
        =\int_{\partial B}F^*\omega
        =\int_Bd(F^*\omega)
        =\int_Bd^*(d\omega)
        =0
    \end{equation}
    Justifications :
    \begin{itemize}
        \item 
            L'intégrale \( \int_{\partial B}\omega\) est la surface de \( \partial B\) et est donc non nulle.
        \item
            La fonction \( F\) est l'identité sur \( \partial B\). Nous avons donc \( \omega=F^*\omega\).
        \item
            Théorème de Stockes.
        \item
            Le lemme \ref{LemdwLGFG}.
        \item
            La forme \( \omega\) est de volume, par conséquent de degré maximum et \( d\omega=0\).
    \end{itemize}
\end{proof}

Un des points délicats est de se ramener au cas de fonctions \( C^{\infty}\). Pour la régularisation par convolution, voir \cite{AllardBrouwer}; pour celle utilisant le théorème de Weierstrass, voir \cite{KuttlerTopInAl}.
\begin{theorem}[Brouwer]\index{théorème!Brouwer}\label{ThoRGjGdO}
    Soit \( B\) la boule fermée de centre \( 0\) et de rayon \( 1\) de \( \eR^n\) et \( f\colon B\to B\) une fonction continue. Alors \( f\) admet un point fixe.
\end{theorem}

\begin{proof}
    Nous commençons par définir une suite de fonctions
    \begin{equation}
        f_k(x)=\frac{ f(x) }{ 1+\frac{1}{ k } }.
    \end{equation}
    Nous avons \( \| f_k-f \|_{\infty}\leq \frac{1}{ 1+k }\) où la norme est la norme uniforme sur \( B\). Par le théorème de Weierstrass \ref{ThoWmAzSMF} il existe une suite de fonctions \(  C^{\infty}\) \( g_k\) telles que
    \begin{equation}
        \|  g_k-f_k\|_{\infty}\leq\frac{1}{ 1+k }.
    \end{equation}
    Vérifions que cette fonction \( g_k\) soit bien une fonction qui prend ses valeurs dans \( B\) :
    \begin{subequations}
        \begin{align}
            \| g_k(x) \|&\leq \| g_k(x)-f_k(x) \|+\| f_k(x) \|\\
            &\leq \frac{1}{ 1+k }+\frac{ \| f(x) \| }{ 1+\frac{1}{ k } }\\
            &\leq \frac{1}{ 1+k}+\frac{1}{ 1+\frac{1}{ k } }\\
            &=1.
        \end{align}
    \end{subequations}
    Par la version \(  C^{\infty}\) du théorème (proposition \ref{PropDRpYwv}), \( g_k\) admet un point fixe que l'on nomme \( x_k\).

    Étant donné que \( x_k\) est dans le compact \( B\), quitte à prendre une sous suite nous supposons que la suite \( (x_k)\) converge vers un élément \( x\in B\). Nous montrons maintenant que \( x\) est un point fixe de \( f\) :
    \begin{subequations}
        \begin{align}
            \| f(x)-x \|&=\| f(x)-g_k(x)+g_k(x)-x_k+x_k-x \|\\
            &\leq \| f(x)-g_k(x) \| +\underbrace{\| g_k(x)-x_k \|}_{=0}+\| x_k-x \|\\
            &\leq \frac{1}{ 1+k }+\| x_k-x \|.
        \end{align}
    \end{subequations}
    En prenant le limite \( k\to\infty\) le membre de droite tend vers zéro et nous obtenons \( f(x)=x\).
\end{proof}

%---------------------------------------------------------------------------------------------------------------------------
\subsection{Théorème de Schauder et équations différentielles}
%---------------------------------------------------------------------------------------------------------------------------

Une conséquence du théorème de Brouwer est le théorème de Schauder qui est valide en dimension infinie.


%\newpage


\begin{theorem}[Théorème de Schauder\cite{LeDretSc}]\index{théorème!Schauder}       \label{ThovHJXIU}
    Soit \( E\), un espace vectoriel normé, \( K\) un convexe compact de \( E\) et \( f\colon K\to K\) une fonction continue. Alors \( f\) admet un point fixe.
\end{theorem}

\begin{proof}
    Étant donné que \( f\colon K\to K\) est continue, elle y est uniformément continue. Si nous choisissons \( \epsilon\) alors il existe \( \delta>0\) tel que 
    \begin{equation}
        \| f(x)-f(y) \|\leq \epsilon
    \end{equation}
    dès que \( \| x-y \|\leq \delta\). La compacité de \( K\) permet de choisir un recouvrement fini par des ouverts de la forme
    \begin{equation}    \label{EqKNPUVR}
        K\subset \bigcup_{1\leq i\leq p}B(x_j,\delta)
    \end{equation}
    où \( \{ x_1,\ldots, x_p \}\subset K\). Nous considérons maintenant \( L=\Span\{ f(x_j)\tq 1\leq j\leq p \}\) et
    \begin{equation}
        K^*=K\cap L.
    \end{equation}
    Le fait que \( K\) et \( L\) soient convexes implique que \( K^*\) est convexe. L'ensemble \( K^*\) est également compact parce qu'il s'agit d'une partie fermée de \( K\) qui est compact (lemme \ref{LemnAeACf}). Notons en particulier que \( K^*\) est contenu dans un espace vectoriel de dimension finie, ce qui n'est pas le cas de \( K\).

    Nous allons à présent construire une sorte de partition de l'unité subordonnée au recouvrement \eqref{EqKNPUVR} sur \( K\) (voir le lemme \ref{LemGPmRGZ}). Nous commençons par définir
    \begin{equation}
        \psi_j(x)=\begin{cases}
            0    &   \text{si \( \| x-x_j \|\geq \delta\)}\\
            1-\frac{ \| x-x_j \| }{ \delta }    &    \text{sinon}.
        \end{cases}
    \end{equation}
    pour chaque \( 1\leq j\leq p\). Notons que \( \psi_j\) est une fonction positive, nulle en-dehors de \( B(x_j,\delta)\). En particulier la fonction suivante est bien définie :
    \begin{equation}
        \varphi_j(x)=\frac{ \psi_j(x) }{ \sum_{k=1}^p\psi_k(x) }
    \end{equation}
    et nous avons \( \sum_{j=1}^p\varphi_j(x)=1\). Les fonctions \( \varphi_j\) sont continues sur \( K\) et nous définissons finalement
    \begin{equation}
        g(x)=\sum_{j=1}^p\varphi_j(x)f(x_j).
    \end{equation}
    Pour chaque \( x\in K\), l'élément \( g(x)\) est une combinaison des éléments \( f(x_j)\in K^*\). Étant donné que \( K^*\) est convexe et que la somme des coefficients \( \varphi_j(x)\) vaut un, nous avons que \( g\) prend ses valeurs dans \( K^*\) par la proposition \ref{PropPoNpPz}.

    Nous considérons seulement la restriction \( g\colon K^*\to K^*\) qui est continue sur un compact contenu dans un espace vectoriel de dimension finie. Le théorème de Brouwer nous enseigne alors que \( g\) a un point fixe (proposition \ref{ThoRGjGdO}). Nous nommons \( y\) ce point fixe. Notons que \( y\) est fonction du \( \epsilon\) choisit au début de la construction, via le \( \delta\) qui avait conditionné la partition de l'unité.

    Nous avons
    \begin{subequations}        \label{EqoXuTzE}
        \begin{align}
            f(y)-y&=f(y)-g(y)\\
            &=\sum_{j=1}^p\varphi_j(y)f(y)-\sum_{j=1}^p\varphi_j(y)f(x_j)\\
            &=\sum_{j=1}^p\varphi(j)(y)\big( f(y)-f(x_j) \big).
        \end{align}
    \end{subequations}
    Par construction, \( \varphi_j(y)\neq 0\) seulement si \( \| y-x_j \|\leq \delta\) et par conséquent seulement si \( \| f(y)-f(x_j) \|\leq \epsilon\). D'autre par nous avons \( \varphi_j(y)\geq 0\); en prenant la norme de \eqref{EqoXuTzE} nous trouvons
    \begin{equation}
        \| f(y)-y \|\leq \sum_{j=1}^p\| \varphi_j(y)\big( f(y)-f(x_j) \big) \|\leq \sum_{j=1}^p\varphi_j(y)\epsilon=\epsilon.
    \end{equation}
    Nous nous souvenons maintenant que \( y\) était fonction de \( \epsilon\). Soit \( y_m\) le \( y\) qui correspond à \( \epsilon=2^{-m}\). Nous avons alors
    \begin{equation}
        \| f(y_m)-y_m \|\leq 2^{-m}.
    \end{equation}
    L'élément \( y_m\) est dans \( K^*\) qui est compact, donc quitte à choisir une sous suite nous pouvons supposer que \( y_m\) est une suite qui converge vers \( y^*\in K\)\footnote{Notons que même dans la sous suite nous avons \( \| f(y_m)-y_m \|\leq 2^{-m}\), avec le même «\( m\)» des deux côtés de l'inégalité.}. Nous avons les majorations
    \begin{equation}
        \| f(y^*)-y^* \|\leq \| f(y^*)-f(y_m) \|+\| f(y_m)-y_m \|+\| y_m-y^* \|.
    \end{equation}
    Si \( m\) est assez grand, les trois termes du membre de droite peuvent être rendus arbitrairement petits, d'où nous concluons que
    \begin{equation}
        f(y^*)=y^*
    \end{equation}
    et donc que \( f\) possède un point fixe.
\end{proof}

%\newpage

Ce théorème permet de démontrer une version du théorème de Cauchy-Lipschitz (théorème \ref{ThokUUlgU}) sans la condition Lipschitz, mais alors sans unicité de la solution. Notons que de ce point de vue nous sommes dans la même situation que la différence entre le théorème de Brouwer et celui de Picard : hors hypothèse de type «contraction», point d'unicité.

\begin{theorem}[Cauchy-Arzela\cite{ClemKetl}]\index{théorème!Cauchy-Arzela}
    Nous considérons le système d'équation différentielles
    \begin{subequations}        \label{EqTXlJdH}
        \begin{numcases}{}
            y'=f(t,y)\\
            y(t_0)=y_0.
        \end{numcases}
    \end{subequations}
    avec \( f\colon U\to \eR^n\), continue où \( U\) est ouvert dans \( \eR\times \eR^n\). Alors il existe un voisinage fermé \( V\) de \( t_0\) sur lequel une solution \( C^1\) du problème \eqref{EqTXlJdH} existe.
\end{theorem}

\begin{proof}[Idée de la démonstration]
    Nous considérons \( M=\| f \|_{\infty}\) et \( K\), l'ensemble des fonctions \( M\)-Lipschitz sur \( U\). Nous prouvons que \( (K,\| . \|_{\infty})\) est compact. Ensuite nous considérons l'application
    \begin{equation}
        \begin{aligned}
            \Phi\colon K&\to K \\
            \Phi(f)(t)&=x_0+\int_{t_0}^tf\big( u,f(u) \big)du. 
        \end{aligned}
    \end{equation}
    Après avoir prouvé que \( \Phi\) était continue, nous concluons qu'elle a un point fixe par le théorème de Schauder \ref{ThovHJXIU}.
\end{proof}


%+++++++++++++++++++++++++++++++++++++++++++++++++++++++++++++++++++++++++++++++++++++++++++++++++++++++++++++++++++++++++++
\section{Théorème de Markov-Kakutani et mesure de Haar}
%+++++++++++++++++++++++++++++++++++++++++++++++++++++++++++++++++++++++++++++++++++++++++++++++++++++++++++++++++++++++++++

\begin{theorem}[Markov-Katutani\cite{BeaakPtFix}]\index{théorème!Markov-Takutani}   \label{ThoeJCdMP}
    Soit \( E\) un espace vectoriel normé et \( K\), une partie non vide, convexe, fermée et bornée de \( E'\). Soit \( T\colon K\to K\) une application continue. Alors \( T\) a un point fixe.
\end{theorem}

\begin{proof}
    Nous considérons un point \( x_0\in K\) et la suite
    \begin{equation}
        x_n=\frac{1}{ n+1 }\sum_{i=0}^n T^ix_0.
    \end{equation}
    La somme des coefficients devant les \( T^i(x_0)\) étant \( 1\), la convexité de \( K\) montre que \( x_n\in K\). Nous considérons l'ensemble
    \begin{equation}
        C=\bigcap_{n\in \eN}\overline{ \{ x_m\tq m\geq n \} }.
    \end{equation}
    Le lemme \ref{LemooynkH} indique que \( C\) n'est pas vide, et de plus il existe une sous suite de \( (x_n)\) qui converge vers un élément \( x\in C\). Nous avons
    \begin{equation}
        \lim_{n\to \infty} x_{\sigma(n)}(v)=x(v)
    \end{equation}
    pour tout \( v\in E\). Montrons que \( x\) est un point fixe de \( T\). Nous avons
    \begin{subequations}
        \begin{align}
            \| (Tx_{\sigma(k)}-x_{\sigma(k)})v \|&=\Big\| T\frac{1}{ 1+\sigma(k) }\sum_{i=0}^{\sigma(k)}T^ix_0(v)-\frac{1}{ 1+\sigma(k) }\sum_{i=0}^{\sigma(k)}T^ix_0(v) \Big\|\\
            &=\Big\| \frac{1}{ 1+\sigma(k) }\sum_{i=0}^{\sigma(k)}T^{i+1}x_0(v)-T^ix_0(v) \Big\|\\
            &=\frac{1}{ 1+\sigma(k) }\big\| T^{\sigma(k)+1}x_0(v)-x_0(v) \big\|\\
            &\leq\frac{ 2M }{ \sigma(k)+1 }
        \end{align}
    \end{subequations}
    où \( M=\sum_{y\in K}\| y(v) \|<\infty\) parce que \( K\) est borné. En prenant \( k\to\infty\) nous trouvons
    \begin{equation}
        \lim_{k\to \infty} \big( Tx_{\sigma(k)}-x_{\sigma(k)} \big)v=0,
    \end{equation}
    ce qui signifie que \( Tx=x\) parce que \( T\) est continue.
\end{proof}

Le théorème suivant est une conséquence du théorème de Markov-Katutani.
\begin{theorem}\index{mesure!Haar}
    Si \( G\) est un groupe topologique compact possédant une base dénombrable de topologie alors \( G\) accepte une unique mesure de Haar normalisée. De plus elle est unimodulaire :
    \begin{equation}
        \mu(Ag)=\mu(gA)=\mu(A)
    \end{equation}
    pour tout mesurables \( A\subset G\) et tout élément \( g\in G\).
\end{theorem}

%+++++++++++++++++++++++++++++++++++++++++++++++++++++++++++++++++++++++++++++++++++++++++++++++++++++++++++++++++++++++++++
\section{Méthode de Newton}
%+++++++++++++++++++++++++++++++++++++++++++++++++++++++++++++++++++++++++++++++++++++++++++++++++++++++++++++++++++++++++++

L'objectif de la méthode de Newton est d'évaluer une racine \( a\) de l'équation \( f(x)=0\) lorsque nous avons déjà une approximation \( x_0\) de \( a\).

%---------------------------------------------------------------------------------------------------------------------------
\subsection{Points fixes attractifs et répulsifs}
%---------------------------------------------------------------------------------------------------------------------------

Source : \cite{DemaillyNum}.

Soit \( I\) un intervalle fermé de \( \eR\) et \( \varphi\colon I\to I\) une application \( C^1\). Soit \( a\) un point fixe de \( \varphi\). Nous disons que \( a\) est \defe{attractif}{point fixe!attractif}\index{attractif!point fixe} si il existe un voisinage \( V\) de \( a\) tel que pour tout \( x_0\in V\) la suite \( x_{n+1}=\varphi(x_n)\) converge vers \( a\). Le point \( a\) sera dit \defe{répulsif}{répulsif!point fixe} si il existe un voisinage \( V\) de \( a\) tel que pour tout \( x_0\in V\) la suite \( x_{n+1}=\varphi(x_n)\) diverge.

\begin{lemma}
    Si \( | \varphi'(a) |<1\) alors \( a\) est attractif et la convergence est au moins exponentielle.

    Si \( | \varphi'(a) |>1\) alors \( a\) est répulsif et la divergence est au moins exponentielle.
\end{lemma}

\begin{proof}
    Si \( | \varphi'(a)<1 |\) alors il existe \( k\) tel que \( | \varphi'(a) |<k<1\) et par continuité il existe un voisinage \( V\) de \( a\) dans lequel \( | \varphi'(x) |<k\) pour tout \( x\in V\). En utilisant le théorème des accroissements finis nous avons
    \begin{equation}
        | x_n-a |=\big| f(x_{n-1}-a) \big|\leq k| x_{n-1}-a |
    \end{equation}
    et par récurrence
    \begin{equation}
        | x_n-a |\leq k^n| x_0-a |.
    \end{equation}

    Le cas \( | \varphi'(a)>1 |\) se traite de façon similaire.
\end{proof}

\begin{remark}
    Dans le cas \(| \varphi'(a) |=1\), nous ne pouvons rien conclure. Si \( \varphi(x)=\sin(x)\) nous avons \( \sin(x)<x\) et le point \( a=0\) est attractif. A contrario, si \( \varphi(x)=\sinh(x)\) nous avons \( |\sinh(x)|>|x|\) et le point \( a=0\) est répulsif.
\end{remark}

%---------------------------------------------------------------------------------------------------------------------------
\subsection{Méthode de Newton}
%---------------------------------------------------------------------------------------------------------------------------

\begin{lemma}       \label{LemXdObnV}
    Soient \( A\) et \( B\) deux matrices inversibles telles que la matrice \( (A+\epsilon B)\) soit inversible pour tout \( \epsilon\) assez petit. Alors il existe une matrice \( X(\epsilon)\) telle que
    \begin{equation}
        (A+\epsilon B)^{-1}=(A^{-1}+\epsilon X)
    \end{equation}
    et telle que \( \lim_{\epsilon\to 0}X(\epsilon)=-A^{-1} BA^{-1}\).
\end{lemma}

\begin{proof}
    Le candidat matrice \( X\) est relativement simple à trouver en écrivant
    \begin{equation}
        (A+\epsilon B)(A^{-1}+\epsilon X)=\mtu+\epsilon AX+\epsilon BA^{-1}+\epsilon^2BX.
    \end{equation}
    En imposant que cela soit \( \mtu\), nous trouvons
    \begin{equation}        \label{EqahWJUU}
        X(\epsilon)=-(A+\epsilon B)^{-1} BA^{-1}.
    \end{equation}
    La matrice \( X(\epsilon)\) donnant un inverse à droite de \( (A+\epsilon B)\), son déterminant est non nul et \( X\) est inversible. Par conséquent elle est également inversible au sens usuel. Le calcul de la limite est direct :
    \begin{equation}
        \lim_{\epsilon\to 0}-(A+\epsilon B)^{-1} BA^{-1}=A^{-1} BA^{-1}
    \end{equation}
    parce que l'inverse est une fonction continue sur \( \eM(n,\eR)\).

    Nous pouvons vérifier explicitement que \eqref{EqahWJUU} donne également un inverse à droite en calculant :
    \begin{subequations}
        \begin{align}
            (A^{-1}+\epsilon X)(A+\epsilon B)&=\big( A^{-1}-\epsilon(A+\epsilon B)^{-1} BZ^{-1} \big)(A+\epsilon B)\\
            &=1+\epsilon A^{-1} B-\epsilon(A+\epsilon B)^{-1} B-\epsilon^2(A+\epsilon B)^{-1} BA^{-1} B.
        \end{align}
    \end{subequations}
    Nous devons donc vérifier que
    \begin{equation}
        A^{-1} B-(A+\epsilon B)^{-1}B-\epsilon(A+\epsilon B)^{-1} BA^{-1} B=0.
    \end{equation}
    Pour le vérifier, il suffit de mettre \( B\) en facteur à droite et d'introduire \( AA^{-1}\):
    \begin{subequations}
        \begin{align}
            A^{-1} B-(A+\epsilon B)^{-1}B&-\epsilon(A+\epsilon B)^{-1} BA^{-1} B\\
            &=\left[ A^{-1}-\Big( (A+\epsilon B)^{-1}(A+\epsilon BA^{-1}) \Big) \right]AA^{-1} B\\
            &=\left[ 1-(A+\epsilon B)^{-1}(A+\epsilon B) \right]A^{-1}B\\
            &=0.
        \end{align}
    \end{subequations}
\end{proof}

\begin{remark}
    Un calcul naïf nous permet de trouver le même résultat de façon plus heuristique. En effet un développement usuel (dans \( \eR\)) est
    \begin{equation}
        \frac{1}{ a+\epsilon b }=\frac{1}{ a }-\frac{ \epsilon b }{ a^2 }+\ldots
    \end{equation}
    Si nous récrivons cela avec des matrices, nous écrivons (attention : passage heuristique!) :
    \begin{equation}
        (A+\epsilon B)^{-1}=A^{-1}-\epsilon A^{-1} BA^{-1}+\ldots
    \end{equation}
    Notons le choix de généraliser \( b/a^2\) par \( a^{-1} ba^{-1}\). Dans les réels les deux écritures sont équivalentes, mais pas dans les matrices.

    Étudions si \( A^{-1}-\epsilon A^{-1}BA^{-1}\) est bien un inverse à \( \epsilon^2\) près de \( (A+\epsilon B)\) :
    \begin{equation}
        (A+\epsilon B)(A^{-1}+\epsilon A^{-1} BA^{-1})=1-\epsilon BA^{-1}+\epsilon BA^{-1}-\epsilon^2BA^{-1}BA^{-1}=1-\epsilon^2BA^{-1} BA^{-1}.
    \end{equation}
    Par conséquent, à des termes en \( \epsilon^2\) près la matrice \( A^{-1}-\epsilon A^{-1}BA^{-1}\) est bien un inverse de \( A+\epsilon B\).
\end{remark}

%\newpage

\begin{theorem}[Méthode de Newton\cite{ChambertNewton}]\index{Newton!méthode}\index{méthode!Newton}
    Soit \( f\colon \eR^n\to \eR^n\) une application de classe \( C^2\) et un point \( a\in \eR^n\) tel que \( f(a)=0\). Nous supposons que \( df_a\) est inversible.

    Alors il existe un voisinage \( V\) de \( a\) tel que pour tout \( x_0\in V\) la suite définie par récurrence
    \begin{equation}
        x_{n+1}=x_n-(df_{x_n})^{-1}\big( f(x_n) \big)
    \end{equation}
    converge vers \( a\). De plus la vitesse est quadratique au sens où il existe \( C>1\) tel que 
    \begin{equation}        \label{EqtkiDXt}
        \| x_n-a \|\leq C^{-1-2^n}.
    \end{equation}
\end{theorem}

\begin{proof}
    Étant donné que \( df_a\) est inversible et que \( df\) est continue, l'application \( df_x\) est inversible\footnote{Nous pouvons voir \( df\) comme l'application qui à \( x\) fait correspondre la matrice \( df_x\in\eM(n,\eR)\). Cette application étant continue et la non inversibilité d'une matrice étant donnée par l'annulation du déterminant, les matrices inversibles forment un ouvert dans l'ensemble des matrices.} pour tout \( x\) dans un voisinage de \( a\). Nous prenons \( r>0\) tel que \( df_x\) soit inversible pour tout \( x\in B(a,r)\).

    Nous considérons la fonction 
    \begin{equation}
        \begin{aligned}
                F\colon B(a,r)&\to \eR^n \\
                x&\mapsto x-(df_x)^{-1}\big( f(x) \big). 
            \end{aligned}
        \end{equation}
        Cela est une application \( C^1\). La clef est de montrer que l'application de \( F\) à un point \( a+h\) rapproche de \( a\) pourvu que \( h\) soit assez petit. Nous avons la formule suivante :
        \begin{equation}        \label{EqyDLQeE}
            F(a+h)-F(a)=h-\big( df_{a+h} \big)^{-1}\big( f(a+h) \big).
        \end{equation}
        Nous allons maintenant utiliser un développement de Taylor par rapport à \( h\) en suivant la formule \eqref{EquQtpoN}. Nous avons
        \begin{equation}
            f(a+h)=f(a)+df_a(h)+\| h \|^2\xi(h)
        \end{equation}
        où \( \xi\colon \eR^n\to \eR^n\) est une fonction qui tend vers une constante lorsque \( h\to 0\). Nous avons aussi
        \begin{equation}
            df_{a+h}=df_a+\| h \|\tau(h)
        \end{equation}
        où \( \tau\colon \eR^n\to \eM(n,\eR)\) est une application qui tend vers une constante lorsque \( h\to 0\). En ce qui concerne l'inverse nous utilisons le lemme\footnote{Pour l'inversibilité de \( \| h \|\tau(h)\), notons que \( df_a\) est inversible et que par hypothèse la somme \( df_a+\| h \|\tau(h)\) est inversible.} \ref{LemXdObnV} :
        \begin{equation}
            \big( df_a+\| h \|\tau(h) \big)^{-1}=(df_a)^{-1}+\| h \|A(h)
        \end{equation}
        où \( A\) est une autre matrice fonction de \(h\) qui tend vers une constante lorsque \( h\) tend vers zéro. En substituant le tout dans \eqref{EqyDLQeE} nous trouvons
        \begin{equation}
            F(a+h)-F(a)=\| h \|^2(df_a)^{-1}\xi(h)+\| h \|\big( A(h)\circ df_a \big)(h)+\| h \|^3A(h)\xi(h).
        \end{equation}
        En ce qui concerne la norme nous utilisons le fait que si \( T\) est un opérateur, \( \| Tx \|\leq \| T \|\| x \|\). Nous trouvons
        \begin{subequations}
            \begin{align}
                \| F(a+h)-F(a) \|&\leq \| h \|^2\| (df_a)^{-1} \|\| \xi(h) \|+\| h \|^2\| A(h)\circ df_a \|+\| h \|^3\| A(h) \|\| \xi(h) \|\\
                &=\| h \|^2\alpha(h)
            \end{align}
        \end{subequations}
    pour une certaine fonction \( \alpha\colon \eR^n\to \eR\) qui tend vers une constante lorsque \( h\to 0\). 

    En posant \( C=\lim_{h\to 0}\alpha(h) \) nous avons la majoration
    \begin{equation}        \label{EqSYiuYF}
        \| F(x)-a \|\leq C\| x-a \|^2.
    \end{equation}
    Nous pouvons également supposer que \( C>1\). Affin de prouver la vitesse de convergence \eqref{EqtkiDXt}, nous allons encore redéfinir \( r\) en demandant \( r<1/C^2\). De cette manière nous avons
    \begin{equation}
        \| x_0-a \|\leq \frac{1}{ C^2 }
    \end{equation}
    et
    \begin{equation}
        \| x_{n+1}-a \|=\| F(x_n)-a \|\leq C\| x_n-a \|^2\leq C\big( C^{-1-2^n} \big)^2=C^{-1-2^{n+1}}.
    \end{equation}

\end{proof}

\begin{remark}
    La valeur de la constate \( C\) a été fixée par l'équation \eqref{EqSYiuYF}. Certes nous pouvons toujours choisir \( C\) plus grand affin d'augmenter la vitesse de convergence, mais le point de départ \( x_0\) devant être dans une boule de taille \( 1/C^2\) autour de \( a\), demander \( C \) plus grand revient à demander un point de départ plus précis.
\end{remark}

%\newpage

%---------------------------------------------------------------------------------------------------------------------------
\subsection{Auto correction d'erreurs}
%---------------------------------------------------------------------------------------------------------------------------
\label{subseciKpMFx}

Source : \cite{NourdinAnal}\footnote{Il me semble qu'à la page 100 de ce livre, l'hypothèse H1 qui est prouvée ne prouve pas Hn dans le cas \( n=1\).}.

Nous commençons par donner une précision au théorème de Picard.
\begin{proposition}
    Soit \( (X,d)\) un espace métrique complet et \( f\colon X\to X\) une application contractante. Alors \( f\) admet un unique point fixe. De plus pour tout \( x_0\in X\), l'unique point fixe de \( f\) est donné par la limite de la suite définie par
    \begin{equation}
        x_{n+1}=f(x_n).
    \end{equation}
    De plus nous estimons l'erreur par
    \begin{equation}    \label{EqKErdim}
        \| x_n-x \|\leq \frac{ k^n }{ 1-k }\| x_n-x_{n-1} \|\leq \frac{ k^n }{ 1-k }\| x_1-x_0 \|.
    \end{equation}
\end{proposition}
La première inégalité donne une estimation de l'erreur calculable en cours de processus; la seconde donne une estimation de l'erreur calculable avant de commencer.

\begin{proof}
    Nous devons seulement prouver les inégalités \eqref{EqKErdim}. En utilisant une somme télescopique nous avons
    \begin{subequations}
        \begin{align}
            \| x_{n+p}-x_n \|&\leq \| x_{n+p}-x_{n+p-1} \|+\ldots +\| x_{n+1}-x_n \|\\
            &\leq k^p\| x_n-x_{n-1} \|+k^{p-1}\| x_n-x_{n-1} \|+\ldots +k\| x_n-x_{n-1} \|\\
            &=k(1+\ldots +k^{p-1})\| x_n-x_{n-1} \\
            &\leq \frac{ k }{ 1-k }\| x_n-x_{n-1} \|.
        \end{align}
    \end{subequations}
    En prenant la limite \( p\to \infty\) nous trouvons
    \begin{equation}        \label{EqlUMVGW}
        \| x-x_n \|\leq \frac{ k }{ 1-k }\| x_n-x_{n-1} \|\leq \frac{ k }{ 1-k }\| x_1-x_0 \|.
    \end{equation}
\end{proof}

Nous pouvons maintenant établir le résultat d'auto correction de la suite de Picard.
\begin{proposition}
    Soit \( E\) un espace métrique complet, \( a\in E\), \( K\) la boule fermée de rayon \( r\) centrée en \( a\) et une application contractante \( f\colon K\to K\) de constante Lipschitz \( k\). Nous définissons les suites \( (u_n)\) et \( (v_n)\) de la façon suivante :
    \begin{enumerate}
        \item
            \( u_0=v_0\in B(a,r)\).
        \item
            \( u_{n+1}=f(v_n)\), \( v_n\in B(u_n,\epsilon)\).
    \end{enumerate}
    Alors \( f\) a un point fixe \( \xi\in B(a,r)\) et 
    \begin{equation}
        \| v_n-\xi \|\leq \frac{ k^n }{ 1-k }\| u_1-u_0 \|+\frac{ \epsilon }{ 1-k }.
    \end{equation}
\end{proposition}

\begin{proof}
    Le fait que \( f\) ait un point fixe est simplement une application du théorème de Picard. Nous allons montrer la relation par récurrence. Tout d'abord pour \( n=1\) nous avons
    \begin{equation}
        \| v_1-\xi \|\leq\| v_1-u_1 \|+\| u_1-\xi \|\leq \epsilon+\frac{ k }{ 1-k }\| u_1-u_0 \|
    \end{equation}
    où nous avons utilisé l'estimation \eqref{EqlUMVGW}. Nous pouvons maintenant faire la récurrence :
    \begin{subequations}
        \begin{align}
            \| v_{n+1}-\xi \|&\leq \| v_{n+1}-u_{n+1} \|+\| u_{n+1}-\xi \|\\
            &\leq \epsilon+k\| v_n-\xi \|\\
            &\leq \epsilon+k\left( \frac{ k^n }{ 1-k }\| u_1-u_0 \|+\frac{ \epsilon }{ 1-k } \right)\\
            &=\frac{ \epsilon }{ 1-k }+\frac{ k^{n+1} }{ 1-k }\| u_1-u_0 \|.
        \end{align}
    \end{subequations}
\end{proof}
