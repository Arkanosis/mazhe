% This is part of Analyse Starter CTU
% Copyright (c) 2014
%   Laurent Claessens,Carlotta Donadello
% See the file fdl-1.3.txt for copying conditions.

\begin{corrige}{autoanalyseCTU-51devoir}

 
 
   \begin{enumerate}
   \item On calcule d'abord  
\[
\ln(x+1)-\sin(x) =x -\frac{x^2}{2} + \frac{x^3}{3} -\frac{x^4}{4} - x+\frac{x^3}{6} + x^4\alpha(x)= -\frac{x^2}{2}+  x^2\alpha(x),
\]
ce qui nous dit que au voisinage de zéro la fonction $\displaystyle \dfrac{\ln(1+x)-\sin x}{x}$ a le m\^eme comportement que $-x/2$. La valeur de la limite est donc zéro. 

   \end{enumerate}



\end{corrige}   
