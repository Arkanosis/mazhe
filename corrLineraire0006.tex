% This is part of the Exercices et corrigés de mathématique générale.
% Copyright (C) 2009
%   Laurent Claessens
% See the file fdl-1.3.txt for copying conditions.
\begin{corrige}{Lineraire0006}

	En général, nous avons
	\begin{equation}
		(A+B)^2=(A+B)(A+B)=A^2+AB+BA+B^2.
	\end{equation}
	Il n'est pas vrai de dire $AB=BA$, parce que ce n'est pas toujours le cas avec des matrices. Ici, nous voulons que $(A+B)^2=A^2+B^2$, c'est à dire $AB+BA=0$. Calculons les produits $AB$ et $BA$ avec les matrices proposées :
	\begin{equation}
		AB=\begin{pmatrix}
			1	&	-1	\\ 
			2	&	-1	
		\end{pmatrix}
		\begin{pmatrix}
			1	&	1	\\ 
			4	&	-1	
		\end{pmatrix}=
		\begin{pmatrix}
			-3	&	2	\\ 
			-2	&	3	
		\end{pmatrix},
	\end{equation}
	tandis que
	\begin{equation}
		BA=\begin{pmatrix}
			3	&	-2	\\ 
			2	&	-3	
		\end{pmatrix}.
	\end{equation}
	Nous avons donc bien $AB+BA=0$ dans ce cas ci.


\end{corrige}
