% This is part of Exercices et corrigés de CdI-1
% Copyright (c) 2011
%   Laurent Claessens
% See the file fdl-1.3.txt for copying conditions.

\begin{corrige}{0096}

Nous cherchons deux points $x_1$ et $x_2$ tels que
\begin{equation}
	\frac{ f(x_2)-f(x_1) }{ x_2-x_1 }=-\lambda f(a).
\end{equation}
Étant donné que $g(-1)=g(1)=0$, le théorème des accroissements finis dit qu'il y a un point $a$ dans $\mathopen]-1,1\mathclose[$ tel que $g'(a)=0$. Mais
\begin{equation}
	g'(x)=\lambda e^{\lambda x}f(x)+ e^{\lambda x}f'(x),
\end{equation}
donc la condition $g'(a)=0$ dit que $f'(a)=-\lambda f(a)$.

\end{corrige}
