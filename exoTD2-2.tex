% This is part of Exercices de mathématique pour SVT
% Copyright (c) 2010
%   Laurent Claessens et Carlotta Donadello
% See the file fdl-1.3.txt for copying conditions.

\begin{exercice}[\minsyndical]\label{exoTD2_2}

Déterminer dans chaque cas le domaine de définition, la périodicité et/ou les symétries éventuelles et les limites aux extrêmes du domaine. Calculer ensuite la dérivée de la fonction, la où elle est définie.
\begin{enumerate}
\item 
  $\displaystyle f_1(x)=\ln (x^4)$ ;
  \item 
    $\displaystyle f_2(x)=\ln^4 (x)$ ;
    \item
      $\displaystyle f_3(x)=\ln (4x)$ ;
      \item
        $\displaystyle f_4(x)=(x^2-2x)e^x$ ;
        \item 
          $\displaystyle f_5(x)=\sqrt{x^2-2}$ ;
          \item
            $\displaystyle f_6(x)=\frac{\sin x +\cos x}{\sin x -\cos x}$ ;
            \item
              $\displaystyle f_7(x)=e^{x^3/2}$ ;
              \item 
                $\displaystyle f_8(x)=\sqrt{e^{x^3}} $ ;
                \item 
                  $\displaystyle f_9(x)=e^{\sqrt{3+\cos x}}$ ;
                  \item
                    $\displaystyle f_{10}(x)=\frac{x^2+1}{(x+1)^3}$ ;
\end{enumerate}

%\corrref{TD2_2}
  
\end{exercice}
