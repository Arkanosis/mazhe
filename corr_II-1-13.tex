% This is part of the Exercices et corrigés de CdI-2.
% Copyright (C) 2008, 2009
%   Laurent Claessens
% See the file fdl-1.3.txt for copying conditions.


\begin{corrige}{_II-1-13}

\begin{enumerate}

\item $y'=(t+2y+1)/(2t-3)$.
Le déterminant
\begin{equation}
	\begin{vmatrix}
	1	&	2	\\ 
	2	&	0	
\end{vmatrix}=-4\neq 0.
\end{equation}
Nous devons donc résoudre le système
\begin{subequations}
\begin{numcases}{}
	\bar t+2\bar y+1	&=0\\
	2\bar t-3		&=0.
\end{numcases}
\end{subequations}
La solution est $\bar t=3/2$ et $\bar y=-5/4$. Nous posons donc $\tau=t-\frac{ 3 }{ 2 }$, et $u(\tau)=y(t)+\frac{ 5 }{ 4 }$. Essayons d'effectuer ce changement de variable de façon la plus explicite qui soit. Nous avons
\begin{equation}
	\frac{ du }{ d\tau }(\tau)=\frac{ d }{ d\tau }\big( (y\circ t)(\tau) \big)=y'\big( t(\tau) \big)\cdot\frac{ dt }{ d\tau }=y'\big( t(\tau) \big)
\end{equation}
où $y'$ est la dérivée de $y$. Nous pouvons écrire l'équation différentielle qui nous intéresse au point $t(\tau)$ :
\begin{equation}		\label{EqII113AvecTauHom}
	u'(\tau)=y'\big( t(\tau) \big)=\frac{ t(\tau)+2y\big( t(\tau) \big)+1 }{ 2t(\tau)-3 }=\frac{ \tau+2u(\tau) }{ 2\tau }.
\end{equation}
C'est cette équation que nous allons résoudre pour $u(\tau)$. C'est une équation homogène, donc nous posons $z(\tau)=u(\tau)/\tau$, et nous avons $u'(\tau)=\tau^2z'(\tau)+z(\tau)$, et l'équation \ref{EqII113AvecTauHom} devient
\begin{equation}
	z'(\tau)=\frac{1}{ 2\tau },
\end{equation}
ce qui donne $z=\frac{ 1 }{2}\ln(k\tau)$, et donc
\begin{equation}
	u(\tau)=\frac{ 1 }{2}\tau\ln(k\tau).
\end{equation}
Maintenant nous pouvons en tirer $y(t)$ en utilisant le fait que 
\begin{equation}
	u(t-\frac{ 3 }{ 2 })=y(t)+\frac{ 5 }{ 4 },
\end{equation}
parce que $y(t)+\frac{ 5 }{ 4 }=u(\tau)$.

\item $y'=\frac{ t+2y+1 }{ 2t+4y+3 }$.
Ici, le déterminant est nul, et nous écrivons l'équation en faisant apparaître le numérateur au dénominateur (opération possible parce que déterminant est nul) :
\begin{equation}
	y'(t)=\frac{ 2+2t+1 }{ 2(t+2t+1)+1 },
\end{equation}
et nous effectuons le changement de variable $u(t)=t+2y(t)+1$ qui donne immédiatement $y'=(u'-1)/2$. L'équation se met alors sous la forme
\begin{equation}
	u'=\frac{ 4u+1 }{ 2u+1 },
\end{equation}
qui est une équation à variables séparées. Pour la résoudre, il faut trouver une primitive de $(2u+1)/(4u+1)$, et la réponse est
\begin{equation}
	t+\frac{1}{ 2 }\left( \frac{1}{ 4 }\ln(4u+1)+u \right)=C.
\end{equation}
C'est une formule implicite pour $u(t)$, et donc pour $y(t)$.


\item $y'=(2t+3y+5)^2$.
En posant simplement $u=2t+3y+5$, nous trouvons 
\begin{equation}
	\frac{ u' }{ 3 }-\frac{ 2 }{ 3 }=u^2,
\end{equation}
qui est une équation à variables séparées pour laquelle il faut une primitive de $1/(3u^2+2)$ par rapport à $u$. La réponse est
\begin{equation}
	\frac{1}{ \sqrt{6} }\arctg\left( \frac{ 3u }{ \sqrt{6} } \right)=t+C.
\end{equation}



\end{enumerate}

\end{corrige}
