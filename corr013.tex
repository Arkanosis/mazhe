\begin{corrige}{013}
Let us parametrize the torus by two angles $\theta$ (radius $R$) and $\varphi$ (radius $r$). We suppose that $\theta$ descibes a circle in the plane $xz$ (or a parallel plane when $\varphi$ is non zero) while the plane of the rotation $\varphi$ depends of $\theta$.

A simple rotation (an isometry of $\eR^3$) allows us to only consider points with $\theta=0$. So we expect that the result will not depend of $\theta$. Let us explicitly give the parametrization $\dpt{ f }{ \eR^2 }{ T^2 }$. When $\varphi_0$ is fixed, the curve $f(\theta,\varphi_0)$ is a circle of radius $R+r\cos\varphi_0$ in a plane parallel to $xz$ with center at $(0,r\sin\varphi_0,0)$. Thus we have
\begin{equation}
f(\theta,\varphi)=\Big( (R+r\cos\varphi)\sin\theta,r\sin\varphi,(R+r\cos\varphi)\cos\theta \Big).
\end{equation}
We find
\begin{subequations}
\begin{align}
\partial_{\varphi}|_{(\theta_0,\varphi_0)}&\Dsdd{ f(\theta_0,\varphi+t) }{t}{0}  =-r\Big( \sin_0\varphi\sin\theta_0,-\cos\varphi_0,\sin\varphi_0\cos\theta_0 \Big)\\
	\partial_{\theta}|{(\theta_0,\varphi_0)}&= \Dsdd{ f(\theta_0+t,\varphi_0) }{t}{0}  =(R+r\cos\varphi_0)\Big( \cos\theta_0,0,-\sin\theta_0 \Big).
\end{align}
\end{subequations}
Given under a matrix form, the result is
\[ 
  g=
\begin{pmatrix}
  r^2&0\\0&(R+r\cos\varphi)^2
\end{pmatrix}.
\]
Can you geometrically understand the dependance in $r$, $R$ and $\varphi$ ?

Note that the norm of an angular coordinate is the radius of the corresponding circle.

\end{corrige}
