This chapter actually don't deal with \emph{quantum} field theory in the sense that our wave functions aren't operators which acting on a Fock space. So this is just relativistic field theory. 

\section{Connections}
%++++++++++++++++++++

\subsection{Gauge potentials}
%----------------------------

Let us consider a \defe{section}{section} $\salpha$ of $P$ over $\mU_{\alpha}$. It is a map $\dpt{\salpha}{\mU_{\alpha}}{P}$ such that $\pi\circ\salpha=\id$. A \defe{connection}{connection} on $P$ is a $1$-form $\dpt{\omega}{T_pP}{\yG}\in\Omega^1(P)$ which satisfies the following two conditions:
\begin{subequations}
\begin{align}
   \omega_p(Y^*_p)&=Y,   \label{conn_1}\\
   \omega(dR_g\xi)&=g^{-1}\omega(\xi)g.\label{conn_2}
\end{align}
\end{subequations}
The \defe{gauge potential}{gauge!principal potential} of $\omega$ with respect of the local section\label{PgLocSecConn} $\salpha$ is  the $1$-form on $\mU_{\alpha}$ given by
\begin{equation}
          A_{\alpha}(x)(v)=(\salpha^*\omega)_x(v).
\end{equation}
We will not always explicitly write the dependence of $A_{\alpha}$ in $x$.\nomenclature{$A_{\alpha}$}{Gauge potentials} Now we consider another section $\dpt{\sbeta}{\mU_{\beta}}{P}$ which is related on $\mU_{\alpha}\cap\mU_{\beta}$ to $\salpha$ by $\sbeta(x)=\salpha(x)\cdot g_{\alpha\beta}(x)$ for a well defined map $\dpt{g_{\alpha\beta}}{\mU_{\alpha}\cap\mU_{\beta}}{G}$.

\begin{proposition}
The gauge potentials $A_{\alpha}$ and $A_{\beta}$ are related by
\begin{equation}\label{trans_A}
                A_{\beta}=g^{-1} A_{\alpha} g-g^{-1} dg.
\end{equation}
\label{prop:trans_A}
\end{proposition}

\begin{proof}
By definition, for $v\in T_x\mU_{\alpha}$,
\[
   A_{\beta}(v)=(\sbeta^*\omega)_x(v)=
         \omega_{\salpha(x)\cdot g_{\alpha\beta}(x)}\big((d\sbeta)_x(v)\big).
\]
We begin by computing $d\sbeta(v)$. Let us take a path $v(t)$ in $\mU_{\alpha}$ such that $v(0)=x$ and $v'(0)=v$. We have :
\begin{equation}\label{eq:1407r1}
\begin{split}
   (d\sbeta)_x(v)&=\dsdd{\sbeta(v(t))}{t}{0}\\
                 &=\dsdd{\salpha(v(t))\cdot\gab(v(t))}{t}{0}\\
		 &=\Dsdd{\salpha(v(t))\cdot\gab(x)}{t}{0}
		    +\Dsdd{\salpha(x)\cdot\gab(v(t))}{t}{0}\\
		 &=dR_{\gab(x)}d\salpha(v)+\Dsdd{\salpha(x)\cdot\gab(x)e^{-ts}}{t}{0}\\
		 &=dR_{\gab(x)}d\salpha(v)+s^*_{\salpha(x)\cdot\gab(x)}
\end{split}
\end{equation}
where $s$ is defined by the requirement\label{pg:justif_s} that $\gab(x)^{-1}\gab(v(t))$ can be replaced in the derivative by $e^{-ts}$, so that we can replace $\gab(v(t))$ by $\gab(x)e^{-ts}$. As far as the derivatives are concerned, $e^{-ts}=\gab(x)^{-1}\gab(v(t))$, then
\[
     s=-\dsdd{\gab(x)^{-1}\gab(v(t))}{t}{0}=-\gab(x)^{-1} d\gab(v),
\]
this equality being a notation. Now, properties \eqref{conn_1} and \eqref{conn_2} make that
\[
   A_{\beta}(v)=\gab(x)^{-1}\omega_{\salpha(x)}(d\salpha(v))\gab(x)+s.
\]
The thesis is just the same, with ``reduced'' notations (see section \ref{subsec:digress}).
.
\end{proof}
An explicit form for this transformation law is :
\begin{equation}
    A_{\beta}(v)=\Dsdd{g^{-1} e^{tA_{\alpha}(v)}g}{t}{0}-\Dsdd{g^{-1}\gab(v(t))}{t}{0},
\end{equation}
where $g:=\gab(x)$.

\subsection{Covariant derivative}
%-------------------------------

When we have a connection on a principal bundle, we can define a covariant derivative\index{covariant!derivative} on any associated bundle. Let us quickly review it. An associated bundle is the semi-product $E=P\times_{\rho} V$ where $V$ is a vector space on which acts the representation $\rho$ of $G$. We denote the canonical projection by $\dpt{\pi_p}{E}{M}$. The classes are taken with respect to the equivalence relation $(p,v)\sim(p\cdot g,\rho(g^{-1})v)$.

A \defe{section}{section!of associated bundle} of $E$ is a map $\dpt{\psi}{M}{E}$ such that $\pi\circ\psi=\id$. We denote by $\Gamma(E)$ the set of all the sections of $E$. A section of $E$ defines (and is defined by) an equivariant function\index{equivariant!function} $\dpt{\hpsi}{P}{V}$ such that
\begin{subequations}
\begin{align}
  \psi(\pi(\xi))&=[\xi,\hpsi(\xi)],\\
  \hpsi(\xi\cdot g)&=\rho(g^{-1})\hpsi(\xi).
\end{align}
\end{subequations}
For a section $\psi\in\Gamma(E)$, we define $\dpt{\psi\bsa}{\mU_{\alpha}}{V}$ by
 \[
 \psi\bsa(x)=\hpsi(\sigma(x)).
 \]
We saw in \eqref{eq:nabla_coord} that a covariant derivative on $E$ is given by
\begin{equation}\label{3008e1}
  (D_X\psi)\bsa(x)=X_x\psi\bsa-\rho_*\Big((\salpha^*\omega)_x(X_x)\Big)\psi\bsa(x).
\end{equation}
Since $(d\psi)(X)=X(\psi)$, we can rewrite this formula in a simpler manner by forgetting the index $\alpha$ and the mention of $X$ :
\[
    D\psi=d\psi-(\rho_*A_{\alpha})\psi.
\]
If we note $(\rho_*A_{\alpha})\psi$ by $A_{\alpha}\psi$, we have
\begin{equation}
        D\psi=d\psi-A\psi.
\end{equation}
One has to understand that equation as a ``notational trick''\ for \eqref{3008e1}. By the way, remark that $(\rho_*A_{\alpha})$ is the only ``reasonable'' way for $A$ to act on $\psi$.

\section{Gauge transformation}
%++++++++++++++++++++++++++++

A \defe{gauge transformation}{gauge!transformation} of a $G$-principal bundle is a diffeomorphism $\dpt{\varphi}{P}{P}$ which satisfies
\begin{subequations}
\begin{align}
   \pi\circ\varphi&=\pi,\\
   \varphi(\xi\cdot g)&=\varphi(\xi)\cdot g.
\end{align}
\end{subequations}
In local coordinates, it can be expressed in terms of a function $\dpt{\tilde{\varphi}_{\alpha}}{\mU_{\alpha}}{G}$ by the requirement that
\begin{equation}\label{def_vpt}
    \varphi(\salpha(x))=\salpha(x)\cdot\tilde{\varphi}_{\alpha}(x).
\end{equation}

We have shown in proposition \ref{prop:vp_conn} that, if $\omega$ is a connection $1$-form on $P$, the form $\varphi\cdot\omega:=\varphi^*\omega$ is still a connection $1$-form on $P$. Of course, with the same section $\salpha$ than before, we can define a gauge potential $(\varphi\cdot A)_{\alpha}$ for this new connection. We will see how it is related to $A_{\alpha}$. The reader can guess the result (it will be the same as proposition \ref{prop:trans_A}). We show it.

\begin{proposition}		\label{Proptr_de_A}
\begin{equation}\label{tr_de_A}
     (\varphi\cdot A)=\tilde{\varphi}^{-1} A\tilde{\varphi}-\tilde{\varphi}^{-1} d\tilde{\varphi}.
\end{equation}
\end{proposition}
\begin{proof}
Let us consider $x\in\mU_{\alpha}$, and $v\in T_x\mU_{\alpha}$, the vector which is tangent to the curve $v(t)\in\mU_{\alpha}$. We compute
\[
    \salpha^*(\varphi^*\omega)_x(v)=\omega_{(\varphi\circ\salpha)(x)}((d\varphi\circ d\salpha)(v)),
\]
but equation \eqref{def_vpt} makes
\begin{equation}
\begin{split}
   (d\varphi\circ d\salpha)(v)&=\dsdd{\varphi(\salpha(v(t)))}{t}{0}\\
                          &=\dsdd{\salpha(v(t))\cdot\tilde{\varphi}_{\alpha}(v(t))}{t}{0}.
\end{split}
\end{equation}
Now, we are in the same situation as in equation \eqref{eq:1407r1}.
\end{proof}

If $\dpt{\psi}{M}{E}$ is a section of $E$, the gauge transformation $\dpt{\varphi}{P}{P}$ acts on $\psi$ by
\begin{equation}
   \widehat{\varphi\cdot\psi}(\xi)=\hpsi(\varphi^{-1}(\xi)).
\end{equation}
On the other hand, $\varphi$ acts on the covariant derivative (and the potential) :
$\varphi\cdot D$ is the covariant derivative  of the connection $\varphi\cdot\omega$. Of course, we define
\begin{equation}
    (\varphi\cdot D)\psi=d\psi-(\varphi\cdot A)\psi.
\end{equation}

\begin{lemma}
If $\dpt{\varphi}{P}{P}$ is a gauge transformation, then
\begin{enumerate}
\item $\varphi^{-1}$ is also a gauge transformation and
             $(\widetilde{\varphi^{-1}})_{\alpha}(x)=\tilde{\varphi}_{\alpha}(x)^{-1}$, \label{lem:i}
\item $(\varphi\cdot\psi)\bsa(x)=\rho(\tilde{\varphi}_x^{-1})\psi\bsa(x)$.\label{lem:ii}
\end{enumerate}
\label{lem:prop_gauge}
\end{lemma}

\begin{proof}
The first part is clear while the second is a computation :
\begin{equation}
    (\varphi\cdot\psi)\bsa=\widehat{\varphi\cdot\psi}(\salpha(x))
                         =\hpsi(\varphi^{-1}(\salpha(x)))
			 =\hpsi(\salpha(x)\cdot\tilde{\varphi}_{\alpha}(x)^{-1})
			 =\rho(\tilde{\varphi}_{\alpha}(x))\psi\bsa(x).
\end{equation}
\end{proof}

Now, we will proof the main theorem: the one which explains why the covariant derivative is ``covariant''.

\begin{theorem}
The covariant derivative $D$ fulfils a ``covariant'' transformation rule under gauge transformations:
\begin{equation}\label{eq:covariance_math}
      (\varphi\cdot D)(\varphi^{-1}\cdot \psi)=\varphi^{-1}(D\psi).
\end{equation}
\label{th:covariance}
\end{theorem}
\begin{remark}
Let us use more intuitive notations: we write \eqref{tr_de_A} under the form $A'=g^{-1} Ag-g^{-1} dg$. If we have two sections  $\psi$ and $\psi'$, they are necessarily related by a gauge transformation: $\psi'=g^{-1}\psi$. Then, the theorem tells us that the equation $D\psi=d\psi-A\psi$ becomes $D'\psi'=g^{-1} D\psi$ ``under a gauge transformation''. This is: $D\psi$ transforms under a gauge transformation as $d\psi$ transforms under a constant linear transformation. This is the reason why $D$ is a \emph{covariant} derivative. The physicist way to write \eqref{eq:covariance_math} is
\begin{equation}\label{eq:covariance_phys}
    D'\psi'=g^{-1} D\psi
\end{equation}
\label{rem:intuitif}
\end{remark}

\begin{proof}[Proof of theorem \ref{th:covariance}]
First, we look at $(\varphi\cdot A)\psi_{\alpha}$. Using all the notational tricks used to give a sens to $A\psi$, we write :
\[
   [(\varphi\cdot A)_X\psi]\bsa(x)=(\varphi\cdot A)_X\psi\bsa(x)=\rho_*(\varphi\cdot A(X))\psi\bsa(x).
\]
But we know that $\varphi\cdot A=\tilde{\varphi}^{-1} A\tilde{\varphi}-\tilde{\varphi}^{-1} d\tilde{\varphi}$, then
\begin{equation}\label{eq:en_deux}
\begin{split}
  (\varphi\cdot A)_X\psi\bsa(x)
  &=\rho_*(\tilde{\varphi}^{-1} A(X)\tilde{\varphi})\psi\bsa(x)\\
  &\quad-\rho_*(\tilde{\varphi}^{-1} d\tilde{\varphi}(X))\psi\bsa(x)\\
  &=\Dsdd{ \rho(\tilde{\varphi}^{-1} e^{tA(X)}\tilde{\varphi})\psi\bsa(x)}{t}{0}\\
 &\quad-\Dsdd{ \rho(\tilde{\varphi}^{-1}\tilde{\varphi}(X_t))\psi\bsa(x) }{t}{0}
\end{split}
\end{equation}
Now, we have to write this equation with $\varphi^{-1}\cdot\psi$ instead of $\psi$. Using lemma \ref{lem:prop_gauge}, we find :
\begin{equation}
\begin{split}
   (\varphi\cdot A)_X(\varphi^{-1}\cdot\psi)\bsa(x)
   &=\Dsdd{ \rho(\tilde{\varphi}^{-1} e^{tA(X)}\tilde{\varphi}\tilde{\varphi}^{-1})\psi\bsa(x)}{t}{0}\\
   &\quad-\Dsdd{ \rho(\tilde{\varphi}^{-1}\tilde{\varphi}(X_t)\tilde{\varphi}^{-1})\psi\bsa(x) }{t}{0}
\end{split}
\end{equation}
After simplification, the first term is a term of the thesis: $\tilde{\varphi}(x)^{-1}(A\psi)_{\alpha}(x)$ and we let the second one as it is. Now, we turn our attention to the second term of \eqref{eq:covariance_math}; the same argument gives:
\begin{equation}
\begin{split}
  d(\varphi^{-1}\psi\bsa)_xX
  &=\Dsdd{(\varphi^{-1}\cdot\psi)\bsa(X_t)}{t}{0}\\
  &=\Dsdd{\rho(\tilde{\varphi}(X_t)^{-1})\psi\bsa(X_t)}{t}{0}\\
  &=\Dsdd{\rho(\tilde{\varphi}(X_t)^{-1})\psi\bsa(x)}{t}{0}
  +\Dsdd{\rho(\tilde{\varphi}^{-1})\psi\bsa(X_t)}{t}{0}.
\end{split}
\end{equation}
The second term is $\tilde{\varphi}^{-1} d\psi_{\alpha}(X)$. In definitive, we need to prove that the two exceeding terms cancel each other:
\begin{equation}\label{eq:le_zero}
  \Dsdd{\rho(\tilde{\varphi}^{-1}\tilde{\varphi}(X_t)\tilde{\varphi}^{-1})\psi\bsa(x)}{t}{0}
  +\Dsdd{\rho(\tilde{\varphi}(X_t)^{-1})\psi\bsa(x)}{t}{0}
\end{equation}
must be zero.

One can find a $g(t)\in G$ such that $\tilde{\varphi}(X_t)=\tilde{\varphi} g(t)$, $g(0)=e$. Then, what we have in the $\rho$ of these two terms is respectively $g(t)\tilde{\varphi}^{-1}$ and $g(t)^{-1}\tilde{\varphi}^{-1}$. As far as the derivative are concerned, $g(t)$ can be written as $e^{tZ}$ for a certain $Z\in\yG$. So we see that $g(t)^{-1}=e^{-tZ}$ and the derivative will come with the right sign to makes the sum zero.
\end{proof}

\begin{remark}
If we naively make the computation with the notations of remark~\ref{rem:intuitif}, we replace $\psi'=g^{-1}\psi$ and $A'=g^{-1} Ag-g^{-1} dg$ in
\[
  D'\psi'=d\psi'-A'\psi',
\]
using some intuitive ``Leibnitz formulas'', we find :
$D'\psi'=dg^{-1}\psi+g^{-1} d\psi+g^{-1} A\psi+g^{-1} dg g^{-1}\psi$. It is exactly $g^{-1} d\psi+g^{-1} A\psi$ with two additional terms: $dg^{-1}\psi$ and $g^{-1} dg g^{-1}\psi$. One sees that these are precisely the two terms of the expression \eqref{eq:le_zero}. We will give a sens to this ``naive''\ computation in section~\ref{subsec:digress}.
\end{remark}
\section{A bite of physics}
%++++++++++++++++++++++++++

\subsection{Example: electromagnetism}\index{electromagnetism}
%-------------------------------------

Let us consider the electromagnetism as the simplest example of a gauge invariant physical theory. We first discuss the theory of free electromagnetic field (this is: without taking into account the interactions with particles) from Maxwell's\index{Maxwell} equations, see \cite{Schomblond_em,llf}. The electric field $\bE$ and the magnetic field $\bB$ are subject to following relations:
\begin{subequations}
\begin{align}
\nabla\cdot{\bE}&=\rho,    \label{M1}\\
\nabla\cdot{\bB}&=0,                     \label{M2}  \\
\nabla\times{\bE}+\partial_t{\bB}&=0,         \label{M3}\\
\nabla\times{\bB}-\partial_t{\bE}&={\mathbf{j}}. \label{M4}
\end{align}
\end{subequations}
Comparing \eqref{M1} and \eqref{M2}, we see that Maxwell's theory does not incorporate magnetic monopoles.
Suppose that we can use the Poincaré lemma. Equation \eqref{M2} gives a vector field $\bA$ such that $\bB=\nabla\times\bA$, so that \eqref{M3} becomes $\nabla\times(\bE+\partial_t\bA)=0$ which gives a scalar field $\phi$ such that $-\nabla\cdot\phi=\bE+\partial_t\bA$.

Now the equations \eqref{M1}--\eqref{M4} are equations for the potentials $\bA$ and $\phi$, and we find back the ``physical''\ field by
\begin{subequations}\label{BE_de_A}
\begin{align}
    \bB&=\nabla\times\bA,\\
    \bE&=-\nabla\phi-\partial_t\bA.
\end{align}
\end{subequations}

One can easily see that there are several choice of potentials\index{potential} which describe the same electromagnetic field. Indeed, if
\begin{subequations}\label{jauge_A}
\begin{align}
    \bA'&=\bA+\nabla\lambda,\\
    \phi'&=\phi-\partial_t\lambda,
\end{align}
\end{subequations}
the electromagnetic field given (via \eqref{BE_de_A}) by $\{\phi',\bA'\}$ is the same as the one given by $\{\phi,\bA\}$

The Maxwell's equations can be written in a more ``covariant''\ way by defining
\begin{equation}\label{def_F}
F=\begin{pmatrix}
0 & -E_x/c & -E_y/c & -E_z/c \\
\cdot & 0 & -B_z & \cdot \\
\cdot & \cdot & 0 & -B_x \\
\cdot & -B_y & \cdot & 0
\end{pmatrix},
\end{equation} $F^{\mu\nu}=-F^{\nu\mu}$ and\nomenclature{$F_{\mu\nu}$}{Electromagnetic field strength}
\[
 J=\begin{pmatrix}
c\rho & j_x & j_y & j_z
\end{pmatrix}.
\]\nomenclature{$J_{\mu}$}{Electromagnetic $4$-current}
We also define $\star F^{\alpha\beta}=\frac{1}{2} e^{\alpha\beta\lambda\mu}F_{\lambda\mu}$. With all that, Maxwell's equations read:
\begin{equation}
\begin{split}
\partial_{\mu} F^{\mu\nu}&=\mu_0J^{\nu},\\
\partial_{\alpha}\star F^{\alpha\beta}&=0.
\end{split}
\end{equation}
If we define
\begin{equation}\label{def_A}
  A=\begin{pmatrix}
\frac{\displaystyle\phi}{\displaystyle c} & -A_x & -A_y & -A_z
\end{pmatrix},
\end{equation}
%
the physical fields are given by
\[
  F_{\mu\nu}=\partial_{\mu}A_{\nu}-\partial_{\nu}A_{\mu}.
\]
The \defe{gauge invariance}{gauge!invariance!of electromagnetism} of this theory is the fact that
\begin{subequations}
\begin{equation}
  F'_{\mu\nu}=\partial_{\mu}A'_{\nu}-\partial_{\nu}A'_{\mu}=F_{\mu\nu}
\end{equation}
when
\begin{equation}
A'_{\mu}(x)=A_{\mu}(x)+\partial_{\mu}f(x)
\end{equation}
\end{subequations}
for any scalar \emph{function} $f$ (to be compared with \eqref{jauge_A}).

This is: in the picture of the world in which we see the $A$ as fundamental field of physics, several (as much as you have functions in $C^{\infty}(\eR^4)$) fields $A$, $A'$,\ldots\ describe the \emph{same} physical situation because the fields $\bE$ and $\bB$ which acts on the particle are the same for $A$ and $A'$.

Now, we turn our attentions to the interacting field theory of electromagnetism. As far as we know, the electron makes interactions with the electromagnetic field via a term $\overline{ \psi } A_{\mu}\psi$ in the Lagrangian. The free Lagrangian for an electron is
%
%\begin{center}
%\begin{fmffile}{monfey}
%\begin{fmfgraph*}(40,25)
%\fmfleft{i1}
%\fmfright{o1,o2}
%\fmf{fermion}{i1,v1}
%\fmf{photon}{v1,o1}
%%\fmf{fermion}{v1,o2}
%\fmflabel{$e^-$}{i1}
%\fmflabel{$e^-$}{o2}
%\fmflabel{$A_{\mu}$}{o1}
%\end{fmfgraph*}
%\end{fmffile}
%\end{center}
%
\begin{equation}\label{eq:freeL}
\mL=\overline{ \psi }(\gamma^{\mu}\partial_{\mu}+m)\psi.
\end{equation}
%
 The easiest way to include a $\overline{ \psi } A\psi$ term is to change $\partial_{\mu}$ to $\partial_{\mu}+A_{\mu}$. But we want to preserve the powerful gauge invariance of classical electrodynamics, then we want the new Lagrangian to keep unchanged if we do
 \begin{equation} \label{eq_jaugeA_em}
 A_{\mu}\rightarrow A_{\mu}'=A_{\mu}-i\partial_{\mu}\phi.
 \end{equation}
  In order to achieve it, we remark that the $\psi$ must be transformed \textit{simultaneously} into
\begin{equation}\label{eq:jaugepsi_em}
 \psi'(x)=e^{i\phi(x)}\psi(x).
\end{equation}

The conclusion is that if one want to write down a Lagrangian for QED\index{QED}, one must find a Lagrangian which remains unchanged under certain transformation $A\rightarrow A'$ and $\psi\rightarrow\psi'$. In other words the set $\{\psi,A\}$ of fields which describe the world of an electron in an electromagnetic field is not well defined from data of the physical situation alone: it is defined up to a certain invariance which is naturally called a \defe{gauge invariance}{gauge!invariance}.

\begin{remark}
In the physics books, the matter is presented in a slightly different way. We observe that the Lagrangian \eqref{eq:freeL} is invariant under
\begin{equation}\label{eq:globale}
\psi(x)\rightarrow\psi'(x)=e^{i\alpha}\psi(x)
\end{equation}
%
for any \emph{constant} $\alpha$. One can see that the associated conserved current (Noether) is closely related to the electric current. The idea (of Yang-Mills) is to develop this symmetry. Since the symmetry \eqref{eq:globale} depends only on a constant, we say it a \defe{global}{global symmetry} symmetry; we will simultaneously add a new field $A_{\mu}$ and upgrade \eqref{eq:globale} to a \defe{local}{local symmetry (physics)} symmetry:
 \begin{equation}\label{eq:locate}
\psi(x)\rightarrow\psi'(x)=e^{i\phi(x)}\psi(x).
\end{equation}
Then, we deduce the transformation law of $A_{\mu}$.
 \end{remark}

Because of the form of \eqref{eq:jaugepsi_em}, we say that the electromagnetism is a $U(1)$-gauge theory. The fact that this is an abelian group have a deep physical meaning and many consequences.

\subsection{Little more general, slightly more formal}
%--------------------------------------------

The aim of this text is to interpret the field $A$ as a gauge potential for a connection. But equation \eqref{eq_jaugeA_em} is not exactly the expected one which is \eqref{tr_de_A}. The point is that equation \eqref{eq_jaugeA_em} concerns a theory in which the gauge transformation of the field was a simple multiplication by a scalar field, so that simplifications as $e^{-i\phi(x)}A_{\mu}(x)e^{i\phi(x)}=A_{\mu}(x)$ are allowed.

Now, we consider a vector space $V$, a manifold $M$ and a function $\dpt{\psi}{M}{V}$ which ``equation of motion''\ is
\[
   L^i(\partial_i+m_i)\psi=0
\]
Where we imply an unit matrix behind $\partial$ and $m$; the indices $i,j$ are the (local) coordinates in $M$ and $a,b$, the coordinates in $V$. Let $G$ be a matrix group which acts on $V$. If $\psi$ is a solution, $\Lambda^{-1}\psi$ is also a solution as far as $\Lambda$ is a constant --does not depend on $x\in M$-- matrix of $G$. In other words, $L^i(\partial_i+m_i)\psi_a=0$ for all $a$ implies $L^i(\partial_i+m_i)((\Lambda^{-1})^b_a\psi_b)=0$.

The function, $\psi'(x)=\Lambda(x)^{-1}\psi(x)$ is no more a solution. If we want it to be solution of the same equation as $\psi$, we have to change the equation and consider
\[
   L^i(\partial_i+A_i+m_i)\psi=0.
\]
This equation is preserved under the \emph{simultaneous} change
\begin{equation}\label{eq:jaugeG}
    \left\{\begin{aligned}
	   \psi_a'&=(\Lambda^{-1})^b_a\psi_b\\
           (A_i')^a_b&=(\Lambda^{-1})^c_b (A_i)^d_c(\Lambda^a_d)-(\partial_i\Lambda^{-1})^d_b\Lambda^a_d.
          \end{aligned}\right.
\end{equation}
The second line show that the formalism in which $A$ is a connection is the good one to write down covariant equations. This has to be compared with \eqref{trans_A}. Logically, a theory which includes an invariance under transformations as \eqref{eq:jaugeG} is called a $G$-gauge theory.

\subsection{A ``final'' formalism}
%---------------------------------------

Now, we work with fields which are sections of some fiber bundle build over $M$, the physical space. More precisely, let $G$ be a matrix group. 

\begin{probleme}
	For sure, it also works for a much lager class of groups. Which one ?
\end{probleme}


We search for a theory which is ``locally invariant under $G$''. In order to achieve it, we consider a $G$-principal bundle $P$ over $M$ and the associated bundle $E=P\times_{\rho}V$ for a certain vector space $V$, and a representation $\rho$ of $G$ on $V$. Typically, $V$ is $\eC$ or the vector space on which the spinor representation acts.

The physical fields are sections $\dpt{\psi}{M}{E}$. If we choose some reference sections $\dpt{\sigma_{\alpha}}{M}{P}$, they can be expressed by $\psi\bsa(x)=\hpsi(\salpha(x))$. We translate the idea of a local invariance under $G$ by requiring an invariance under
\[
     \psi'\bsa(x)=\rho(g(x))\psi\bsa(x)
\]
for every $\dpt{g}{M}{G}$. By \ref{lem:ii} of lemma \ref{lem:prop_gauge}, we see that $\psi'\bsa(x)=(\varphi^{-1}\cdot\psi)\bsa(x)$, where $\dpt{\varphi}{P}{P}$ is the gauge transformation given by
\[
   \varphi(\salpha(x))=\salpha(x)\cdot g(x).
\]

We want $\psi$ and $\psi'$ to ``describe the same physics''. From a mathematical point of view, we want $\psi$ and $\psi'$ to \emph{satisfy the same equation}. It is clear that equation $d\psi=0$ will not work.

The trick is to consider any connection $\omega$ on $P$ and the gauge potential $A$ of $\omega$. In this case the equation
\begin{equation}\label{eq:Dpsi}
    (d-A)\psi=0\qquad\textrm{ or }\qquad D\psi=0
\end{equation}
is preserved under 
\[ 
 \begin{split}
A&\rightarrow\varphi\cdot A,\\
 \psi&\rightarrow\varphi^{-1}\cdot\psi.
\end{split} 
\]
 Theorem \ref{th:covariance} powa !
        
In this sense, we say that equation \eqref{eq:Dpsi} is gauge invariant, and is thus taken by physicists to build some theories when they need a ``local $G$-covariance''. This gives rise to the famous Yang-Mills theories.

In this picture the matter field $\psi$ and the bosonic field $A$ are both defined from a $U(1)$-principal bundle. When physicists say 
\begin{quote}
	$\psi$ transforms as ``blahblah'' under a $U(1)$ transformation,
\end{quote}
they mean that $\psi$ is a section of an $U(1)$-associated bundle; 
when they say
\begin{quote}
	$A$ transforms as ``blahblah'' under a $U(1)$ transformation,
\end{quote}
they mean that $A$ is the gauge potential of a connection on a $U(1)$-principal bundle. In each case, the ``blahblah'' denotes an irreducible\footnote{Irreducibility is for elementary particles} representation of $U(1)$.

\begin{remark}
The mathematics of equation \eqref{eq:Dpsi} only requires a $\yG$-valued connection on $P$. There are several physical constraints on the choice of the connection. These give rise to interaction terms between the gauge bosons. We will not discuss it at all. This a matter of books about quantum field theories.

The most used Yang-Mills groups in physics are $U(1)$ for the QED, $SU(2)$ for the weak interactions and $SU(3)$ for chromodynamic.
\end{remark}
\section{Curvature}
%+++++++++++++

\subsection{Intuitive setting}
%-------------------------------

From the $\yG$-valued connection $1$-form $\omega$ on $P$, we may define its \defe{curvature $2$-form}{curvature} :
\begin{equation}
     \Omega=d\omega+\omega\wedge\omega.
\end{equation}
As before, we can see $\Omega$ as a $2$-form on $M$ instead of $P$. For this, we just need some sections $\dpt{\salpha}{\mU_{\alpha}}{P}$ and define
\begin{equation}
        F_{\alpha}=\salpha^*\Omega.
\end{equation}
This $F$ is called the \defe{Yang-Mills field strength}{Yang-Mills!field strength}. The question is now to see how does it transform under a change of chart ? What is $F_{\beta}=\sbeta^*\Omega$ in terms of~$F_{\alpha}$ ?

\begin{theorem}
\begin{equation}
     F_{\beta}=g^{-1} F_{\alpha} g.
\end{equation}
\label{tho:trans_F}
\end{theorem}

\begin{proof}[Naive proof]
Let us accept $F_{\beta}=dA_{\beta}+A_{\beta}\wedge A_{\beta}$. With proposition \ref{prop:trans_A}, we can perform a simple computation with all the intuitive ``Leibnitz rules''\ :
\[
   dA_{\beta}=-g^{-1} dg\, g^{-1}\wedge A_{\alpha} g+g^{-1} dA_{\alpha} g+g^{-1} A_{\alpha}\wedge dg-g^{-1} dg\,g^{-1}\wedge dg,
\]
and
\[
  A_{\beta}\wedge A_{\beta}=g^{-1} A_{\alpha} g\wedge g^{-1} A_{\alpha} g+g^{-1} A_{\alpha} g\wedge g^{-1} dg+g^{-1} dg\wedge g^{-1} A_{\alpha} g+g^{-1} dg\wedge g^{-1} dg.
\]
The sum is obviously the announced result.
\end{proof}
This proof seems too beautiful to be false\footnote{More precisely, it is as beautiful as we want it to be true.}. We will now try to give a sense to this computation.
A complete proof of the theorem is reported until page \pageref{preuve_trans_F}.

First, note that we can't try to find a relation like $d(g\omega)=dg\wedge\omega+g\,d\omega$. Pose $A_x=g(x)\omega_x$:
\[
  A_x(v)=\dsdd{g(x)e^{t\omega_x(v)}}{t}{0}.
\]
Using
\[
   (d\alpha)(v,w)=v(\alpha(w))-w(\alpha(v))-\alpha([v,w]),
\]
we are led to write
\begin{equation}
     w(A(v))=d(A(v))w
            =\dsdd{A_{w_u}(v)}{u}{0}
	    =\dsdd{ \Dsdd{g(w_u)e^{t\omega_{w_u}(v)}}{t}{0} }{u}{0}.
\end{equation}
But at $t=u=0$, the expression in the bracket is $g(x)$, and not $e$. Then the whole expression is not an element of $\yG$. In other words, the problem is that for $\dpt{g}{M}{G}$, we have $\dpt{dg_x}{T_xM}{T_{g(x)}G\neq T_eG}$.

Now, remark that in our matter, the problem will not arise because in the expressions $A_{\beta}=g^{-1} A_{\alpha} g+g^{-1} dg$, each term has a $g$ and a $g^{-1}$.

\begin{lemma}
\begin{equation}
   d(g^{-1})_x(v)=-g(x)^{-1} dg(v)g(x)^{-1}.
\end{equation}
\label{lem:dgemu}
\end{lemma}

\begin{proof}
Let $v_t$ be a path which defines the vector $v$, and define $Y\in\yG$ such that as far as the derivative are concerned, we have $g(v_t)=g(x)e^{tY}$. Then,
\[
      g(g^{-1})(v)=\Dsdd{g(v_t)^{-1}}{t}{0}=\Dsdd{e^{-tY}g(x)^{-1}}{t}{0}.
\]
But on the other hand,
\[
  g^{-1} dg(v)g^{-1}=\Dsdd{g(x)^{-1} g(v_t)g(x)^{-1}}{t}{0}=\Dsdd{e^{tY}g(x)^{-1}}{t}{0},
\]
thus $d(g^{-1})_x(v)=-g(x)^{-1} dg(v)g(x)^{-1}$, as we want.
\end{proof}
\subsection{A digression:  \texorpdfstring{$T_Y\yG$}{TYG} and \texorpdfstring{$\yG$}{G}}\label{subsec:digress}
%+++++++++++++++++++++++++++++++++++++++++++

We define two product: $G\times\yG\to TG$ and $\yG\times\yG\to\yG$. If $g\in G$ and $X\in\yG$, we put
\begin{subequations}
\begin{equation} \label{eq_gXdefa}
   gX=\dsdd{ge^{tX}}{t}{0},
\end{equation}
and if $X$, $Y\in\yG$,
\begin{equation}\label{eq:yGyGb}
   XY=\DDsdd{e^{tX}e^{uY}}{t}{0}{u}{0}.
\end{equation}
\end{subequations}
We naturally define the product of a $\yG$-valued $1$-form $A$ by an element $g\in G$ by $(gA)v=gA(v)$.

 Note that $gX$ does not belong to $\yG$ but to $T_{g}G$. Fortunately, in the expressions which we will meet, there will  always be a $g^{-1}$ to save the situation.

  Let us now see a great consequence of the second definition.
\begin{proposition}
The formula
\begin{equation}
   XY-YX=[X,Y].
\end{equation}
links the formal product inside the Lie algebra and the Lie bracket.
\label{prop:XY_YX}
\end{proposition}

In order to get a real proof (given from page \pageref{pg_demXY_YX}) of this, we have to give some precisions about derivatives as \eqref{eq:yGyGb}. We consider the expression
\[
  \frac{d}{du}\left( \left.\frac{d}{dt} c_u(t)\right|_{t=0}\right)_{u=0},
\]
which will be more simply written as :
\begin{equation}\label{eq:2307e1}
\DDsdd{ c_u(t) }{u}{0}{t}{0}
\end{equation}
with $c_u(t)\in G$ for all $u,t$; $c_u(0)=e$ for all $u$ and $c_0'(0)=Y\in\yG$ where the prime stands for the derivative with respect of $t$. So $\dsdd{c_u(t)}{t}{0}\in\yG$ for each $u$ and \eqref{eq:2307e1} belongs to $T_Y\yG$. But we know that $\yG$ is a vector space, then $T_Y\yG\simeq\yG$, the isomorphism being given by the following idea: if $\{\partial_i\}$ is a basis of $\yG$ and $\{e_i\}$ the corresponding basis of $T_Y\yG$, we define the action of $A^ie_i\in T_Y\yG$ on $\dpt{f}{G}{\eR}$ by $(A^ie_i)f:=A^i\partial_if$.

\begin{lemma}
Seen as an equality in $\yG$, for $\dpt{f}{G}{\eR}$ we have :
\begin{equation}
   \DDsdd{c_u(t)}{u}{0}{t}{0}f=\DDsdd{f(c_u(t))}{u}{0}{t}{0}.
\end{equation}
\end{lemma}

\begin{proof}
Let us consider $X_u=X_u^i\partial_i=c_u'(0)$ and $X_0=Y$. We naturally have
\begin{align}
   X_uf&=\dsdd{f(c_u(t))}{t}{0},&\text{ and } &&\dsdd{X_u}{u}{0}\in T_Y\yG.
\end{align}
Now, we consider a function $\dpt{h}{\yG}{\eR}$ and compute :
\[
  \Dsdd{X_u}{u}{0}h=\Dsdd{h(X_u)}{u}{0}
                   =\dsdd{ h(\Dsdd{c_u(t)}{t}{0}) }{u}{0}.
\]		   
If $\{\partial_i\}$ is a basis of $\yG$ and $\{e_i\}$, the corresponding one of $T_Y\yG$, thus
\begin{equation}
                   \Dsdd{X_u}{u}{0}h=\left.\dsd{h}{e_i}\right|_Y\DDsdd{c^i_u(t)}{u}{0}{t}{0}.
\end{equation}
So, we can write
\[
   \Dsdd{X_u}{u}{0}=\DDsdd{c^i_u(t)}{u}{0}{t}{0}\left.\dsd{}{e_i}\right|_Y,
\]
and, as element of $\yG$, we consider
\[
  \Dsdd{X_u}{u}{0}=\DDsdd{ c^i_u(t) }{u}{0}{t}{0}\partial_i|_e.
\]
Now, we can compute the action of $\dsdd{X_u}{u}{0}$ on a function $\dpt{f}{G}{\eR}$ as
\begin{equation}
\begin{split}
\Dsdd{X_u}{u}{0}f&=\DDsdd{c^i_u(t)}{u}{0}{t}{0}\left.\dsd{f}{x^i}\right|_e\\
                 &=\Dsdd{ \left.\dsd{f}{x^i}\right|_e\dsdd{c^i_u(t)}{t}{0}  }{u}{0}\\
		 &=\Dsdd{ \dsdd{f(c_u(t))}{t}{0} }{u}{0}.
\end{split}
\end{equation}
\begin{probleme}
Je ne sais pas pourquoi tout d'un coup la dernière équation était commentée, et donc la phrase n'était pas finie.
\end{probleme}


\end{proof}


\begin{proof}[Proof of proposition \ref{prop:XY_YX}] \label{pg_demXY_YX}
From this, we can precise our definition of $XY$ when $X$, $Y\in\yG$. The product $XY$ acts on $\dpt{f}{G}{\eR}$ by
\[
  (XY)f=\DDsdd{f(e^{tX}e^{uY})}{t}{0}{u}{0}.
\]
We can get a more geometric interpretation of this. We know how to build a left invariant vector field $\tilde Y$ from any $Y\in\yG$ : for each $g\in G$ we just have to define
\[
  \tY_g(f)=\Dsdd{f(gY(s))}{s}{0}.
\]
%
First remark: $\tY_g$ is precisely the object that previously named ``$gY$''. In order to construct the basis blocks of the formula $XY-YX=[X,Y]$, we turn our attention to $\tX_e\tY$. It is clear that $\tY(f)$ is a function from $G$ to $\eR$, so we can apply $\tX_e$ on it. If $X_t$ is a path which gives the vector $\tX_e$ (for example: $X_t=e^{tX}$), we have
\begin{equation}
  \tX_e(\tY(f))=\Dsdd{\tY(f)_{X_t}}{t}{0}\\
               =\DDsdd{f(X_tY(u))}{u}{0}{t}{0}\\
	       =\DDsdd{f(e^{tX}e^{uY})}{u}{0}{t}{0}.
\end{equation}
Thus we have: $XY=\tX_e\tY$, but it is clear that $[\tX,\tY]_e=\tX_e\tY-\tY_e\tX$. The claim reads now: $[\tX,\tY]_e=[X,Y]$. We can actually take it as de \emph{definition} of $[X,Y]$. It is done in \cite{Helgason}. The link with the definition in terms of successive derivations of $\AD_g(x)=gxg^{-1}$ is not trivial but it can be done. 
\end{proof}

Now, we can give a powerful definition of the wedge for two $\yG$-valued $1$-forms. If $A$, $B\in\Omega^1(M,\yG)$ and $v$, $w\in\cvec(M)$, we define
\begin{equation}
  (A\wedge B)(v,w)=A(v)B(w)-A(w)B(v).
\end{equation}
For $A^2$, we find back the usual definition :
\[
  (A\wedge A)(v,w)=A(v)A(w)-A(w)A(v)=[A(v),A(w)].
\]
%
One can see that for any section $\dpt{\salpha}{\mU_{\alpha}}{P}$, we have
\begin{equation}\label{eq:1907r2}
   \salpha^*(A\wedge A)=(\salpha^*A)\wedge(\salpha^*A).
\end{equation}


\begin{lemma}
If $A$ and $B$ are two $\yG$-valued $1$-forms, one can make  ``simplifications'' as
\begin{equation}
 (Ag)\wedge(g^{-1} B)=A\wedge B.
\end{equation}
\label{lem:simplif}
\end{lemma}

\begin{proof}
We just have to prove that for $A$, $B\in\yG$, $(Ag)(g^{-1} B)=AB$ with definitions \eqref{eq_gXdefa} and \eqref{eq:yGyGb}. Remark that $Ag=\Dsdd{e^{sA}g}{s}{0}$, so
\[
  e^{tAg}=\exp(t\dsdd{e^{sA}g}{s}{0})=\exp(\dsdd{e^{stA}g}{s}{0})=e^{tA}g.
\]
Therefore
\[
  (Ag)(g^{-1} B)=\DDsdd{  e^{tAg}e^{ug^{-1} B}  }{t}{0}{u}{0}=\DDsdd{  e^{tA}gg^{-1} e^{uB}  }{t}{0}{u}{0}=AB.
\]
\end{proof}

\begin{lemma}
\begin{equation}
    F_{\beta}=dA_{\beta}+A_{\beta}^2.
\end{equation}
\end{lemma}

\begin{proof}
This is  a direct consequence of \eqref{eq:1907r2} and $[\sbeta^*,d]=0$.
\end{proof}
Now, we can prove the theorem.

\begin{proof}[Ultimate proof of theorem \ref{tho:trans_F}]\label{preuve_trans_F}
First we compute $d(g^{-1} A_{\alpha} g)$. In order to do this, remark that the $1$-form $g^{-1} A_{\alpha} g$ is explicitly given on $v\in\cvec(M)$ by
\[
   (g^{-1} A_{\alpha} g)(v)_x=\Dsdd{g(x)^{-1} e^{tA(v)_x}g(x)}{t}{0}.
\]
For all $x\in M$, this expression is an element of $\yG$; then we can say that $(g^{-1} A_{\alpha} g)(v)$ is a map $\dpt{(g^{-1} A_{\alpha} g)(v)}{M}{\yG}$. So it is unambiguous to write $w((g^{-1} A_{\alpha} g)(v))\in\yG$ for $w\in T_xM$.

We will use the formula
\[
   d(g^{-1} A_{\alpha} g)(v,w)=v(g^{-1} A_{\alpha} g)(w)-w(g^{-1} A_{\alpha} g)(v)-(g^{-1} A_{\alpha} g)([v,w]).
\]
As $w((g^{-1} A_{\alpha} g)(v))=d((g^{-1} A_{\alpha} g)(v))w$, we have
\begin{equation}
\begin{split}
    w((g^{-1} A_{\alpha} g)(v))&=\dsdd{(g^{-1} A_{\alpha} g)(v)_{w_u}}{u}{0}\\
                &=\dsdd{ \Dsdd{g(w_u)^{-1} e^{tA(v)_{w_u}}g(w_u)}{t}{0}  }{u}{0}\\
		&=\dsdd{  \Dsdd{g(w_u)^{-1}}{u}{0} e^{tA(v)_x}g(x)  }{t}{0}\\
		&\quad+\dsdd{ g(x)^{-1} \Dsdd{e^{tA(v)_{w_u}}}{u}{0} g(x)  }{t}{0}\\
		&\quad+\dsdd{  g(x)^{-1} e^{tA(v)_x} \Dsdd{g(w_u)}{u}{0}  }{t}{0}\\
		&=d(g^{-1})(w)A(v)_xg(x)\\
		&\quad+g(x)^{-1} w_x(A(v))g(x)\\
		&\quad+g(x)^{-1} A(v)_x dg(w).
\end{split}
\end{equation}
On the other hand, one easily finds that
\[
     (g^{-1} A_{\alpha} g)([v,w])=g(x)^{-1} A([v,w])g(x).
\]
 Using lemma \ref{lem:dgemu}, we have
\begin{equation}
\begin{split}
   d(g^{-1} A_{\alpha} g)_x(v,w)&=-g(x)^{-1} dg(v)g(x)^{-1} A(w)_xg(x)+g(x)^{-1} v(A(w))g(x)\\&\quad+g(x)^{-1} A(w)_xdg(v)_x\\
                   &\quad+g(x)^{-1} dg(w)_xg(x)^{-1} A(v)_xg(x)-g(x)^{-1} w(A(v))g(x)\\&\quad-g(x)^{-1} A(v)_xdg(w)\\
		   &\quad-g(x)^{-1} A([v,w])g(x).
\end{split}
\end{equation}
We can regroup the terms two by two in order to form $dA_{\alpha}$ and some wedge; with simpler notations, we can write :
\begin{equation}\label{eq:dA_1}
  d(g^{-1} A_{\alpha} g)=-(g^{-1} dg\,g)\wedge(A_{\alpha} g)-(g^{-1} A)\wedge dg+(g^{-1} dA g).
\end{equation}
We compute $d(g^{-1} dg)$ in the same way; the result is
\[
   (g^{-1} dg)(v)_x=\Dsdd{g(x)^{-1} g(v_t)}{t}{0}\in\yG.
\]
For $v$, $w\in\cvec(M)$, we have :
\begin{equation}
\begin{split}
   w\big((g^{-1} dg)(v)\big)&=\dsdd{ (g^{-1} dg)(v)_{w_u} }{u}{0}\\
                   &=\DDsdd{  g(w_u)^{-1} g(v_{w_u}(t))  }{u}{0}{t}{0}\\
		   &=\DDsdd{  g(w_u)^{-1} g(v_t)  }{t}{0}{u}{0}
		      +\DDsdd{  g(x)^{-1} g(w_u(t))  }{t}{0}{u}{0}\\
		   &=d(g^{-1})(w)dg(v)+\Dsdd{ g(x)^{-1} dg(v_{w_u}) }{u}{0}\\
		   &=-g^{-1} dg(w)g^{-1} dg(v)+g(x)^{-1} w(dg(v))
\end{split}
\end{equation}
where $w_u$ is a path such that $w'_0=w_x$ and $v_{w_u}(t)$ is, with respect of $t$, a path which gives the vector $v_{w_u}$. On the another hand, we have
\[
   (g^{-1} dg)([v,w])=g^{-1} dg([v,w]).
\]

Remark that the term $g(x)^{-1} w(dg(v))$ of  $w((g^{-1} dg)(v))$ together with the same with $v\leftrightarrow w$ and $(g^{-1} dg)([v,w])$ which comes from  $(g^{-1} dg)([v,w])$ will give $g(x)^{-1}(d^2g)(v,w)=0$ when we will compute $d(g^{-1} dg)$.
Finally,
\begin{equation}\label{eq:dA_2}
   d(g^{-1} dg)=-(g^{-1} dg\,g^{-1}\wedge dg).
\end{equation}
The equations \eqref{eq:dA_1} and \eqref{eq:dA_2} allow us to write :
\begin{equation}\label{eq:dA}
\begin{split}
    (dA_{\beta})&=d(g^{-1} A_{\alpha} g)+d(g^{-1} dg)\\
              &=-(g^{-1} dg\,g^{-1}) \wedge(A_{\alpha} g)-(g^{-1} A_{\alpha})\wedge dg\\
	      &\quad+(g^{-1} dA_{\alpha} g)-(g^{-1} dg\,g^{-1})\wedge dg.
\end{split}
\end{equation}
Notice that the term $(g^{-1} dA_{\alpha} g)$ corresponds to the first one in $F_{\beta}=g^{-1}(dA_{\beta}+A_{\beta}\wedge A_{\beta})g$.

For anyone who had understood the whole computations up to here, it is clear that
\begin{equation}
\begin{split}
     [A_{\beta}(v),A_{\beta}(w)]&=\DDsdd{ e^{tA_{\beta}(v)}e^{tA_{\beta}(w)} }{t}{0}{u}{0}\\
                            &\quad-\DDsdd{ e^{tA_{\beta}(w)}e^{tA_{\beta}(v)} }{t}{0}{u}{0}\,,
\end{split}
\end{equation}
so that
\begin{equation}
\begin{split}
  A_{\beta}\wedge A_{\beta}&=g^{-1} A_{\alpha} g\wedge g^{-1} A_{\alpha} g
                         +g^{-1} A_{\alpha} g\wedge g^{-1} dg\\
		       &\quad+g^{-1} dg\wedge g^{-1} A_{\alpha} g
		       +g^{-1} dg\wedge g^{-1} dg.
\end{split}
\end{equation}
Lemma \ref{lem:simplif} allows us to write it under the form
\begin{equation}\label{eq:AA}
\begin{split}
  A_{\beta}\wedge A_{\beta}&=g^{-1} A_{\alpha} g\wedge g^{-1} A_{\alpha} g
                         +g^{-1} A_{\alpha} g\wedge g^{-1} dg\\
		       &\quad+g^{-1} dg\wedge g^{-1} A_{\alpha} g
		       +g^{-1} dg\wedge g^{-1} dg.
\end{split}
\end{equation}
Here the term $(g^{-1} A_{\alpha}\wedge A_{\alpha} g)$ corresponds to the second one in $F_{\beta}=g^{-1}(dA_{\beta}+A_{\beta}\wedge A_{\beta})g$. The sum of \eqref{eq:dA} and \eqref{eq:AA} is
\[
    F_{\beta}=g^{-1} F_{\alpha} g.
\]
\end{proof}
\subsection{The electromagnetic field \texorpdfstring{$F$}{F}}
%--------------------------------------
Now, we are able to interpret the field $F$ introduced in equation \eqref{def_F}.  We follow \cite{Preparation}. From now, we use the usual Minkowski metric $g=diag(-,+,+,+)$.
From the vector given by \eqref{def_A}, we define a (local) potential $1$-form
\[
   A=A_{\mu}dx^{\mu}=-\phi dt+A_x dx+A_y dy+A_z dz.
\]
The field strength\index{Yang-Mills!field strength} is $F=dA$. We easily find that
\begin{equation}
\begin{split}
   F&=(dt\wedge dx)(\partial_x\phi+\partial_t A_x)+\ldots\\
    &\quad+(dx\wedge dy)(-\partial_z A_x+\partial_x A_y)+\ldots
\end{split}
\end{equation}
But the fields $\bB$ and $\bE$ are defined from $\bA$ and $\phi$ by \eqref{BE_de_A}, so
\begin{equation}
\begin{split}
  F&=-E_x(dt\wedge dx)-E_y(dt\wedge dy)-E_z(dt\wedge dz)\\
   &\quad+B_x(dy\wedge dz)+B_y(dz\wedge dx)+B_z(dx\wedge dy).
\end{split}
\end{equation}

We naturally have $dF=d^2A=0$. But conversely, $dF=0$ ensures the existence of a $1$-form $A$ such that $F=dA$. If we define\footnote{\emph{i.e.} we consider $F$ as the main physical field while $\bE$ and $\bB$ are ``derived'' fields.} $\bB=\nabla\times\bA$ and $\bE=-\nabla\phi-\partial_t\bA$, equations \eqref{M2} and \eqref{M3} are obviously satisfied. So in the connection formalism, the equations ``without sources''\ are written by
\begin{equation}\label{M23}
 dF=0.
\end{equation}
In order to write the two others, we introduce the current $1$-form\index{current!$1$-form} :
\[
  j=j_{\mu}dx^{\mu}=-\rho dt+j_x dx+j_y dy+j_z dz.
\]
%
One sees that
\begin{equation}
\begin{split}
  \delta F:=\star d\star F&=-dt(\nabla\cdot\bE)\\
                &\quad+ dx(-\partial_t \bE_x+ (\nabla\times\bB)_x )\\
		&\quad+ dy(-\partial_t \bE_y+ (\nabla\times\bB)_y )\\
		&\quad+ dz(-\partial_t \bE_z+ (\nabla\times\bB)_z ),
\end{split}
\end{equation}
so that equation $\delta F=j$ gives equations \eqref{M1} and \eqref{M4}. Now, the complete set of Maxwell's\index{Maxwell} equations is :
\begin{subequations}
\begin{align}
   d F&= 0\label{SM1}\\
   \delta F &=j\label{SM2}
\end{align}
\end{subequations}
with
\begin{subequations}
\begin{align}
j&=-\rho dt+j_x dx+j_y dy+j_z dz,\\
  \bB&=\nabla\times\bA\\
  \bE&=-\nabla\phi-\partial_t\bA
\end{align}
\end{subequations}
where $A$ is a $1$-form such that $F=dA$ whose existence is given by \eqref{SM1}.
\section{Inclusion of the Lorentz group}\label{subsec:incl_Lorentz}
%+++++++++++++++++++++++++++++++++++++++

Up to now we had seen how to express the \emph{gauge} invariance of a physical theory. In particle physics, a really funny field theory must be invariant under the Lorentz group; it is rather clear that, from the bundle point of view, this feature will be implemented by a Lorentz-principal bundle and some associated bundles. A spinor will be a section of an associated bundle for spin one half representation of the Lorentz group on $\eC^4$. In order to describe non-zero spin particle interacting with an electromagnetic field (represented by a connection on a $U(1)$-principal bundle), we have to build a correct $\SLdc\times U(1)$-principal bundle. We are going to use the ideas of \ref{subsec:sym_nature}.

A \defe{space-time}{space-time} is a differentiable \defe{pseudo-Riemannian}{pseudo-Riemannian} $4$-dimensional manifold. The pseudo-Riemannian structure is a $2$-form $g\in\Omega^2(M)$ for which we can find at each point $x\in M$ a basis $b=({\bf b}_0,\ldots,{\bf b}_3)$ which fulfils
\[
  g_x({\bf b}_i,{\bf b}_j)=\eta_{ij}.
\]
When we use an adapted coordinates, the metric reads $g=\eta_{ij}dx^i\otimes dx^j$.

One says that $M$ is \defe{time orientable}{time!orientable} if one can find a vector field $T\in\cvec(M)$ such that $g_x(T_x,T_x)>0$ for all $x\in M$. A \defe{time orientation}{time!orientation} is a choice of such a vector field. A vector $v\in T_xM$ is \defe{future directed}{future!directed vector} if $g_x(T_x,v)>0$.

The Lorentz group $L$ acts on the orthogonal basis of each $T_xM$, but you may note that $L$ don't act on $M$; it's just when the metric is flat that one can identify the whole manifold with a tangent space and consider that $L$ is the space-times isometry group. In the case of a curved metric, the Lorentz group have to be introduced pointwise and the building of a frame bundle is natural.

Now, we are mainly interested in the frame related each other by a transformation of $L_+^{\uparrow}$. An arising question is to know if one can make a choice of some basis of each $T_xM$ in such a manner that 

\begin{enumerate}
\item pointwise, the chosen frames are related by a transformation of $L_+^{\uparrow}$,
\item the choice is globally well defined.
\end{enumerate}
The first point is trivial to fulfil from the definition of a space-time. For the second, it turns out that a good choice can be performed if and only if there exists a vector field $V\in\cvec(M)$ such that $g_x(V_x,V_x)>0$ for all $x\in M$. We suppose that it is the case\footnote{That condition is rather restrictive because we cannot, for example, find an everywhere non zero vector field on the sphere $S^n$ with $n$ even.}.

So our first principal bundle attempt to describe the space-time symmetry is the $L_+^{\uparrow}$-principal bundle of orthonormal oriented frame on $M$ :
 \begin{equation}\label{bun:LpF}
\xymatrix{
    L_+^{\uparrow}  \ar@{~>}[r] & L(M) \ar[d]^{p_L} \\
    &M
  }
\end{equation}
The notion of ``\defe{relativistic invariance}{relativistic invariance}'' has to be understood in the sense of associated bundle to this one. The next step is to recall ourself (see subsection \ref{subsec:sym_nature}) that the physical fields doesn't transform under representation of the group $L_+^{\uparrow}$ but rather under representations of $\SLdc$. So we build a $\SLdc$-principal bundle
 \[
\xymatrix{
    \SLdc  \ar@{~>}[r] & S(M) \ar[d]^{p_S} \\
    &M
  }
\]
In order this bundle to ``fit''{} as close as possible the bundle \eqref{bun:LpF}, we impose the existence of a map $\dpt{\lambda}{S(M)}{L(M)}$ such that

\begin{enumerate}
\item $p_B(\lambda(\xi))=p_S(\xi)$ for all $\xi\in S(M)$ and
\item $\lambda(\xi\cdot g)=\lambda(\xi)\cdot\mSpin(g)$ for all $g\in\SLdc$.
\end{enumerate}
You can recognize the definition of a \defe{spin structure}{spin!structure}\label{pg_spinenphyz}. Notice that the existence of a spin structure on a given manifold is a non trivial issue.

Now a physical field is given by a section of the associated bundle $E=S(M)\times_{\rho} V$ where $\rho$ is a representations of $\SLdc$ on $V$. For an electron, it is $V=\eC^4$ and $\rho=D^{(1/2,0)}\oplus D^{(0,1/2)}$. That describes a \emph{free} electron is the sense that it doesn't interacts with a gauge field. So in order to write down the formalism in which lives a non zero spin particle, we have to build a $U(1)\times\SLdc$-principal bundle. For this, we follow the procedure given in section \ref{sec:produit_bundle}

\section{Interactions}
%++++++++++++++++++

\subsection{Spin zero}
%--------------------

The general framework is the following :
\[
 \xymatrix{
    U(1)  \ar@{~>}[r] & P \ar[d]_{\displaystyle \pi} && E=P\times_{\rho} V \\
                      & M&\mU_{\alpha}\ar@{^{(}->}[l]  \ar[ur]_{\displaystyle\phi} \ar[lu]_{\displaystyle\sigma_{\alpha}}
  }
\]
a $U(1)$-principal bundle over a manifold $M$ (as far as topological subtleties are concerned, we suppose $M=\eR^4$) and a section $\phi$ of an associated bundle for a representation $\rho$ of $U(1)$ on $V$. We consider $M$ with the Lorentzian metric but, since we are intended to treat with scalar (spin zero) fields, we still don't include the Lorentz (or $\SLdc$) group in the picture. We also consider local sections $\dpt{\sigma_{\alpha}}{\mU_{\alpha}}{P}$, a connection $\omega$ on $P$ and $\Omega$ its curvature. We define $A_{\alpha}=\sigma_{\alpha}^*\omega$.

Now we particularize ourself to the target space $V=\eC$ on which we put the scalar product
\begin{equation}		\label{EqProdScalVeCU}
	\scal{z_1}{z_2}=\frac{1}{2}( z_1\overline{ z }_2+z_2\overline{ z }_1  ),
\end{equation}
and the representation $\dpt{\rho_n}{U(1)}{GL(\eC)}$,
\[
  \rho_n(g)z=g\cdot z=g^nz
\]
where we identify $U(1)$ to the unit circle in $\eC$ in order to compute the product. A property of the product \eqref{EqProdScalVeCU} is to make $\rho_n$ an isometry: for all $g\in U(1)$, $z_1,z_2\in\eC$,
\[
  \scal{\rho_n(g)z_1}{\rho_n(g)z_2}=\scal{z_1}{z_2}.
\]
Our first aim is to write the covariant derivative of $\phi$ with respect to the connection $\omega$. For this we work on the section $\phi$ under the form $\dpt{\phi_{(\alpha)}}{M}{V}$ and we use formula \eqref{3008e1} :
\begin{equation}
  (D_X\phi)\bsa(x)=X_x\phi\bsa-\rho_*\big( (\sigma^*_{\alpha}\omega)_xX_x \big)\phi\bsa(x).
\end{equation}
Let us study this formula. We know that $(\sigma_{\alpha}^*\omega)_x=A_{\alpha}(x)\,:\,T_x\mU_{\alpha}\stackrel{\sigma}{\to}T_{\sigma_{\alpha}(x)}P\stackrel{\omega}{\to}u(1)$. Thus $A_{\alpha}(x)X_x$ is given by a path in $U(1)$; it is this path which is taken by $\rho_*$. Therefore (we forget some dependences in $x$)
\begin{equation}
\begin{split}
  \rho_*\big( A_{\alpha}(x)X_x \big)\phi\bsa(x)&=\Dsdd{ \rho_n\big( (A_{\alpha} X)(t)  \big)\phi\bsa(x) }{t}{0}\\
                                                &=\Dsdd{ (A_{\alpha} X)(t)^n }{t}{0}\phi\bsa(x)\\
						&=n\Dsdd{(A_{\alpha} X)(t)}{t}{0}\phi\bsa(x)\\
						&=nA_{\alpha}(X)\phi\bsa(x).
\end{split}						
\end{equation}
Thus the covariant derivative is given by
\begin{equation}
  (D_X\phi)\bsa(x)=X_x\phi\bsa-nA_{\alpha}(x)(X_x)\phi\bsa(x).
\end{equation}

One can guess an electromagnetic coupling for a particle of electric charge~$n$. If this reveals to be physically relevant, it shows that the ``electromagnetic identity card'' of a particle is given by a representation of $U(1)$. This has to be seen in relation to the discussion on page \pageref{pg:phyz_reprez} where the ``type of particle'' was closely related to representations of the Lorentz group. It is a remarkable piece of quantum field theory: the properties of a particle are encoded in representations of some symmetry groups.

Now we are going to prove that $\|D\phi\|^2$ is a gauche invariant quantity. The first step is to give a sense to this norm. We consider $X_i$ ($i=0,1,2,3$), an orthonormal basis of $T_xM$ and we naturally denote $D_i=D_{X_i}$, $\partial_i=X_i$ and $A_{\alpha i}=A_{\alpha}(\partial_i)$. Remark that 
\begin{equation}
   A_{\alpha}(x)X_x=(\sigma_{\alpha}^*)_xX_x
                 =\omega( d\sigma_{\alpha} X_x )
                 =\omega\Dsdd{ \sigma_{\alpha}(X(t)) }{t}{0}\in u(1),
\end{equation}
so this is given by a path in $U(1)$ which can be taken by $\rho$. Let $c(t)$ be this path, then
\[
   A_{\alpha}\phi\bsa(x)=\Dsdd{ e^{ic(t)}\phi\bsa(x) }{t}{0},
\]
so that under the conjugation, $\overline{A_{\alpha}\phi\bsa(x)}=-A_{\alpha}\overline{ \phi }\bsa(x)$. Now our definition of $\|D\phi\|^2$ is a composition of the norm on $V$ and the one on $T_xM$ :
\begin{equation}
  \|D\phi\|^2=\eta^{ij}\scal{ D_i\phi\bsa }{D_j\phi\bsa}
\end{equation}
Using the notation in which the upper indices are contractions with $\eta^{ij}$, we have
\[
\|D\phi\|^2=\Big(  (\partial_i\phi\bsa)(x)-nA_{\alpha i}\phi\bsa(x)   \Big)
            \Big(  (\partial^i\overline{ \phi }\bsa)(x)+nA_{\alpha}^i\overline{ \phi }\bsa(x)   \Big).
\]

\subsubsection{Gauge transformation law}
%////////////////////////////////////////

A gauge transformation $\varphi$ is given by an equivariant function $\dpt{\tilde{\varphi}_{\alpha}}{\mU_{\alpha}}{U(1)}$ which can be written under the form
\[
   \tilde{\varphi}_{\alpha}(x)=e^{i\Lambda(x)}
\]
for a certain function $\dpt{\Lambda}{\mU_{\alpha}}{\eR}$. From the general formula \ref{lem:ii} of lemma \ref{lem:prop_gauge},
\begin{equation}
  (\varphi\cdot\phi)\bsa(x)=\rho_n(e^{-i\Lambda(x)})\phi\bsa(x)
                          =e^{-ni\Lambda(x)}\phi\bsa(x).
\end{equation}
The transformation of the gauche field $A$ is given by equation \eqref{tr_de_A}. Let us see the meaning of the term $d\tilde{\varphi}$. For $v\in T_x\mU_{\alpha}$,
\begin{equation}
  (d\tilde{\varphi}_{\alpha})_xv=\Dsdd{\tilde{\varphi}_{\alpha}(v(t))}{t}{0}
                   =\Dsdd{ e^{i\Lambda(v(t))} }{t}{0}
		   =i\Dsdd{\Lambda(v(t))}{t}{0}e^{i\Lambda(v(0))}
		   =i(d\Lambda)_xve^{i\Lambda(x)}.
\end{equation}
Thus $\tilde{\varphi}_{\alpha}^{-1}(x)(d\tilde{\varphi}_{\alpha})_x=i(d\Lambda)_x$. Since $U(1)$ is abelian, $\tilde{\varphi}^{-1} A\tilde{\varphi}=A$. Finally,
\begin{equation}
  (\varphi\cdot A)_{\alpha}(x)=A_{\alpha}(x)+i(d\Lambda)_x.
\end{equation}
Now we are able to prove the invariance of $\|D\phi\|^2$. First,
\begin{equation}
\begin{split}
  (\varphi\cdot A)_{i\alpha}(x)=(\varphi\cdot A)_{\alpha}(\partial_i)
                           =A_{i_{\alpha}}(x)+i(\partial_i\Lambda)(x);
\end{split}
\end{equation}
second,
\begin{equation}
\partial_i\left(  e^{-ni\Lambda(x)}\phi\bsa(x)  \right)=-ni(\partial_i\Lambda)(x)\phi\bsa(x)+e^{-in\Lambda(x)}(\partial_i\phi\bsa)(x).
\end{equation}
With these two results, 
\begin{equation}
\partial_i(\varphi\cdot\phi)\bsa(x)+n(\varphi\cdot A)_{\alpha i}(\varphi\cdot\phi)\bsa(x)=e^{-in\Lambda(x)}( nA_{\alpha i}(x)+\partial_i\phi\bsa(x) ).
\end{equation}

The Yang-Mills \defe{field strength}{field!strength} is given by $F\bsa=\sigma^*_{\alpha}\Omega$ (cf. page \pageref{pg:curv_princ}). Since $U(1)$ is abelian, $dF\bsa=0$, so that the second pair of Maxwell's equations is complete without any Lagrangian assumptions.

 The full Yang-Mills action is written as
\[
  S(\omega,\phi)=\int_M\left[  -\frac{1}{4}F\bsa_{ij}F\bsa^{ij}+\frac{1}{2}\|D\phi\|^2+\frac{1}{2} m\phi\bsa\overline{\phi\bsa}  \right].
\]
The Euler-Lagrange equations are
\begin{subequations}
\begin{align}
(\partial_i-inA_{\alpha i})(\partial^i-inA_{\alpha}^i)\phi_{\alpha}+m^2\phi_{\alpha}&=0\\
        \partial_iF\bsa^{ij}&=0.
\end{align}
\end{subequations}
So the Yang-Mills Lagrangian only gives the first pair of Maxwell's equations while the second one is given by the geometric nature of fields.

As explained in \cite{AlexModaveII}, the topology of the physical space has deep implications on the physics of Yang-Mills equations. The absence of magnetic monopoles for example is ultimately linked to the (simple) connectedness of $\eR^4$. When one consider the $U(1)$ Yang-Mills on a sphere, some topological charges appear and magnetic monopoles naturally arise.

\subsection{Non zero spin formalism}
%++++++++++++++++++++++++++++++++

The formalism for a non zero spin particle in an electromagnetic field is described in section \ref{sec:produit_bundle}. We consider the spinor bundle
\[
\xymatrix{
    \SLdc \ar@{~>}[r] & S(M)\ar[d]^{p_S}\\ &M 
  }
\]
with the spinor connection on $S(M)$, and $\rho_1$, a representation of $\SLdc$ on $V$. For an electron, it is $V=\eC^4$ and $\rho_1=D^{(1/2,0)}\oplus D^{(0,1/2)}$, so for $g_1\in\SLdc$,
\begin{equation}
  \rho_1(g_1)\begin{pmatrix}
  z_1\\\vdots\\z_4
             \end{pmatrix}
=
\begin{pmatrix}
  g_1\\\\
&(\overline{g_1}^t)^{-1}
\end{pmatrix}
\begin{pmatrix}
  z_1\\\vdots\\z_4
             \end{pmatrix}.
\end{equation}
On the other hand, we consider the principal bundle
\[
\xymatrix{
    U(1) \ar@{~>}[r] & P\ar[d]^{p_U}\\ &M 
  }
\]
with a connection $\omega_2$ which describes the electromagnetic field. As representation $\dpt{\rho_2}{U(1)}{GL(\eC^4)}$ we choose the multiplication coordinate by coordinate :
\begin{equation}
\rho_2(g_2)\begin{pmatrix}
  z_1\\\vdots\\z_4
             \end{pmatrix}
=
\begin{pmatrix}
  g_2z_1\\\vdots\\g_2z_4
             \end{pmatrix}.
\end{equation}
The physical picture of the electron is now the principal bundle
\[
\xymatrix{
    \SLdc\times U(1) \ar@{~>}[r] & S(M)\circ P\ar[d]^{p}\\ &M,
  }
\]
and the field is a section of the associated bundle $(S(M)\circ P)\times_{\rho}\eC^4$.
