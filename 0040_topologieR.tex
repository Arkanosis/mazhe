% This is part of Mes notes de mathématique
% Copyright (c) 2012-2013
%   Laurent Claessens, Carlotta Donadello
% See the file fdl-1.3.txt for copying conditions.

Dans ce chapitre nous allons parler de topologie sur \( \eR\) puis sur \( \eR^n\). Presque tous les résultats se généralisent aux espaces vectoriels normés sur \( \eR\).

%+++++++++++++++++++++++++++++++++++++++++++++++++++++++++++++++++++++++++++++++++++++++++++++++++++++++++++++++++++++++++++ 
\section{Un peu de topologie sur $\eR$}
%+++++++++++++++++++++++++++++++++++++++++++++++++++++++++++++++++++++++++++++++++++++++++++++++++++++++++++++++++++++++++++

Une construction des réels via les coupures de Dedekind est donnée dans \cite{PaulinTopGmVegN}.

Afin de pouvoir étudier la topologie des espaces métriques, il faut savoir quelque propriétés des réels parce que nous allons étudier la fonction distance qui est une fonction continue à valeurs dans les réels.

%--------------------------------------------------------------------------------------------------------------------------- 
\subsection{Suites numériques}
%---------------------------------------------------------------------------------------------------------------------------

Une \defe{suite numérique}{suite numérique} est une application $x\colon \eN\to \eR$. Une telle application sera notée $(x_n)$. L'élément numéro $k$ de la suite sera noté $x_k$.

\begin{definition}[Limite d'une suite numérique]	\label{DefLimiteSuiteNum}
	Nous disons que la suite $(x_n)$ est une suite \defe{convergente}{convergence!suite numérique} si il existe un réel $\ell$ tel que
	\begin{equation}		\label{EqDefLimSuite}
		\forall \varepsilon>0,\,\exists N\in\eN\tq\forall n\geq N,\,| x_n-\ell |<\varepsilon.
	\end{equation}
	Dans ce cas, le nombre $\ell$ est nommé \defe{limite}{limite!suite numérique} de la suite $(x_n)$. Nous dirons aussi souvent que la suite \defe{converge}{} vers le nombre $\ell$.
\end{definition}
	Une façon équivalente d'exprimer le critère \eqref{EqDefLimSuite} est de dire que pour tout $\varepsilon$ positif, il existe un rang $N\in\eR$ tel que l'intervalle $\mathopen[ \ell-\varepsilon , \ell+\varepsilon \mathclose]$ contient tous les termes $x_n$ au-delà de $N$.

Il est à noter que le rang $N$ dont il est question dans la définition de suite convergente dépend de~$\varepsilon$.

\begin{example}
	Quelque suites usuelles.
	\begin{enumerate}
		\item
			La suite $x_n=\frac{1}{ n }$ converge vers $0$.
		\item
			La suite $x_n=(-1)^n$ ne converge pas.
	\end{enumerate}
\end{example}

Une suite est dite \defe{contenue}{} dans un ensemble $A$ si $x_n\in A$ pour tout $n$. Une suite est \defe{bornée supérieurement}{bornée!suite} si il existe un $M$ tel que $x_n\leq M$ pour tout $n$. De la même manière, la suite est bornée inférieurement si il existe un $m$ tel que $x_n\geq m$ pour tout $n$.

Le lemme suivant est souvent utilisé pour prouver qu'une suite est convergente.
\begin{lemma}		\label{LemSuiteCrBorncv}
	Une suite croissante et bornée supérieurement converge. Une suite décroissante bornée inférieurement est convergente.
\end{lemma}

%--------------------------------------------------------------------------------------------------------------------------- 
\subsection{Critère de Cauchy}
%---------------------------------------------------------------------------------------------------------------------------

\begin{definition}
    Une suite \( (x_n)\) dans \( \eR\) est \defe{de Cauchy}{suite!de Cauchy}\index{Cauchy!suite} si pour tout \( \epsilon>0\), il existe \( N\in \eN\) tel que \( m,n>N\) implique \( | x_n-x_m |<\epsilon\).
\end{definition}

Toute suite de Cauchy est évidemment bornée et donc continue dans un compact.

\begin{theorem} \label{ThoTFGioqS}
    Une suite dans \( \eR\) est convergente si et seulement si elle est de Cauchy.
\end{theorem}
%TODO : la démonstration de la complétude de R.
Même si l'énoncé semble très raisonnable, la démonstration n'est pas simple parce que c'est la complétude de \( \eR\) qui est en jeu.

%--------------------------------------------------------------------------------------------------------------------------- 
\subsection{Maximum, supremum et compagnie}
%---------------------------------------------------------------------------------------------------------------------------

Ce n'est un secret pour personne que $\eR$ est un \href{http://fr.wikipedia.org/wiki/Relation_d'ordre}{ensemble ordonné} : il y a des éléments plus grands que d'autres, et mieux : à chaque fois que je prends deux éléments différents dans $\eR$, il y en a un des deux qui est plus grand que l'autre. Il n'y a pas d'\emph{ex æquo} dans $\eR$.

  Si je regarde l'intervalle $I=[0,1]$, je peux même dire que $10$ est plus grand que tous les éléments de $I$. Nous disons que $10$ est un \emph{majorant} de $[0,1]$. La définition est la suivante.
\begin{definition}
Lorsque $A$ est un sous-ensemble de $\eR$, on dit que $s$ est un \defe{majorant}{majorant} de $A$ si $s$ est plus grand que tous les éléments de $A$. En d'autres termes, si
\[
  \forall x\in A,\,s\geq x.
\]
\end{definition}
Je me permet d'insister sur le fait que l'inégalité n'est pas stricte. Ainsi, $1$ est un majorant de $[0,1]$. Dès qu'un ensemble a un majorant, il en a plein. Si $s$ majore l'ensemble $A$, alors évident $s+1$, $s+4$, $s+\pi^2$ majorent également $A$.

\begin{example}
Une petite galerie d'exemples de majorants.
\begin{itemize}
\item L'intervalle fermé $[4,8]$ admet entre autres $8$ et $130$ comme majorants,
\item l'intervalle ouvert $]4,8[$ admet également $8$ et $130$ comme majorants,
\item $7$ n'est pas un majorant de $[1,5]\cup]8,32]$,
\item $10/10$ majore les notes qu'on peut obtenir à un devoir.
\item l'intervalle $[4,\infty[$ n'a pas de majorants.
\end{itemize}
\end{example}

\begin{definition}
Le \defe{supremum}{supremum} d'un ensemble est le plus petit majorant. En d'autres terme, $s$ est un supremum de $A$ si tout nombre plus petit que $s$ ne majore pas $A$, ou encore,
\[
  \forall x<s,\exists y\in A\text{ tel que } y>x.
\]
Nous disons que $M$ est un \defe{maximum}{maximum} de $A$ si $M$ est un supremum \emph{et} $M\in A$.
\end{definition}
Quand $s$ est un supremum de $A$, ça veut dire que le moindre pas vers la gauche que l'on fait à partir de $s$ (c'est à dire le moindre $\epsilon$), et on tombe dans $A$, ou tout au moins, il existe des éléments de $A$ qui sont plus grand que $s-\epsilon$.

\subsubsection{\ldots et quelque exemples}
%//////////////////////

En matières de notations, le maximum de l'ensemble $A$ est noté $\max A$, le supremum est noté $\sup A$. Le minimum et l'infimum sont notés $\min A$ et $\inf A$.

\begin{example}
Exemples de différence entre majorant, supremum et maximum.
\begin{itemize}
\item Le nombre $10$ est un supremum, majorant et maximum de l'intervalle fermé $[0,10]$,
\item Le nombre $10$ est un majorant et un supremum, mais pas un maximum de l'intervalle ouvert $]0,10[$,
\item Le nombre $136$ est un majorant, mais ni un maximum ni un supremum de l'intervalle $[0,10]$.
\end{itemize}
\end{example}

En utilisant les notations concises, ces différents cas s'écrivent ainsi :
\begin{align*}
10&=\max[0,10]=\sup[0,10]	& 10&=\sup[0,10[
\end{align*}


\begin{example}
Si on dit que un pont s'effondre à partir d'une charge de $10$ tonnes, alors $10$ tonnes est un \emph{supremum} des charges que le pont peut supporter : si on met $9,999999$ tonnes dessus, il tient encore le coup, mais si on ajoute un gramme, alors il s'effondre (on sort de l'ensemble des charges acceptables).
\end{example}

\begin{example}
Si on dit qu'un pont résiste jusqu'à $10$ tonnes, alors $10$ tonnes est un \emph{maximum} de la charge acceptable. Sur ce pont-ci, on peut ajouter le dernier gramme. Mais à partir de là, le moindre truc qu'on ajoute, il s'effondre.
\end{example}

Maintenant il est important de se rendre compte d'une chose : un ensemble ne peut avoir qu'un seul maximum et supremum. Jusqu'à présent nous avons toujours dit \emph{un} supremum. À partir de maintenant nous pouvons dire \emph{le} supremum. La preuve de cela est assez simple.
\begin{proposition}
Si $A$ est un sous-ensemble de $\eR$ admettant un supremum, alors il n'a qu'un seul supremum; et si il accepte un maximum, il n'en accepte un seul, et le maximum est égal au supremum.
\end{proposition}

\begin{proof}
Commençons par l'affirmation concernant le supremum. Supposons que $x$ et $y$ soient tous les deux suprema différents de $A$. Étant donné que $x\neq y$, nous pouvons supposer que $x<y$, et donc, par définition du fait que $y$ est un supremum, il existe un élément de $A$ qui est plus grand que $x$. Cela contredit le fait que $x$ soit supremum. En conclusion, il ne peut pas y avoir deux suprema différents pour un même ensemble.

Étant donné qu'un maximum est un supremum, il ne peut pas y avoir deux maxima différents vu qu'il ne peut pas y avoir deux suprema différents.
\end{proof}


Lorsque vous lisez que la charge maximale d'un camion est de \unit{2.5}{\ton}, est-ce que cela veut dire que vous pouvez y mettre \unit{2.5}{\ton}, mais qui si un oiseau se pose dessus, le camion s'effondre ? Ou bien est-ce que cela signifie qu'à \unit{2.5}{\ton} le camion s'écroule, mais que toute charge inférieure est valable ?

C'est à cette rude question que nous allons nous attaquer maintenant.

\begin{definition}
Soit une partie $A$ de $\eR$. Nous disons qu'un nombre $M$ est un \defe{majorant}{majorant} de $A$ si $M$ est plus grand ou égal que tous les éléments de $A$, c'est à dire si
\begin{equation}
	\forall a\in A,\, M\geq a.
\end{equation}
Un \defe{minorant}{minorant} de $A$ est un nombre $m$ tel que 
\begin{equation}
	\forall a\in A,\, m\leq a.
\end{equation}
\end{definition}

\begin{definition}		\label{DefSupeA}
Soit $A$ une partie majorée de $\eR$. Le \defe{supremum}{supremum} de $A$ est le plus petit des majorants, c'est à dire le nombre $M$ tel que
\begin{enumerate}
	\item
		$M\geq x$ pour tout $x\in A$,
	\item
		pour tout $\varepsilon$, le nombre $M-\varepsilon$ n'est pas un majorant de $a$, c'est à dire qu'il existe un élément $x\in A$ tel que $x>M-\varepsilon$.
\end{enumerate}
Nous notons $\sup A$ le supremum de $A$.

De la même façon, \defe{l'infimum}{infimum} de $A$, noté $\inf A$, est le plus grand de ses minorants. 
\end{definition}
Par convention, si la partie n'est pas bornée vers le haut, nous dirons que son supremum n'existe pas, ou bien qu'il est égal à $+\infty$, suivant les contextes. Pour votre culture générale, sachez toutefois que $\infty\notin\eR$.

La définition est justifiée par le lemme \ref{LemInfUnique} et la proposition \ref{PropBorneSupInf}. Le premier montre que si $A$ possède un infimum, alors il est unique, tandis que le second montre que toute partie majorée de $\eR$ accepter un supremum, et que toute partie minorée accepte un infimum.
\begin{lemma}		\label{LemInfUnique}
	Soit $A$ une partie de $\eR$. Supposons que $m_1$ et $m_2$ soient deux nombres qui vérifient les propriétés de l'infimum de $A$. Alors $m_1=m_2$.
\end{lemma}

\begin{proof}
	Si $_1\neq m_2$, nous pouvons supposer $m_2>m_1$. Dans ce cas, étant donné que $m_1$ est un infimum, $m_2$ ne peut pas minorer $A$, et donc ne peut pas être un infimum.
\end{proof}

\begin{proposition}		\label{PropBorneSupInf}
	Tout sous-ensemble de $\eR$ borné vers le bas possède un infimum; tout sous-ensemble de $\eR$ borné vers le haut possède un supremum.
\end{proposition}

La preuve qui suit est proche de celle donnée par Wikipédia  dans l'article \wikipedia{en}{http://en.wikipedia.org/wiki/Least_upper_bound_principle}{Least uppert bound principle}.

\begin{proof}
	Soit $A$, une partie de $\eR$. Nous allons trouver son infimum en suivant une méthode de dichotomie. Pour cela nous allons construire trois suites en même temps de la façon suivante. D'abord nous choisissons un point $x_0$ de $A$ et un point $x_1$ qui minore $A$ (qui existe par hypothèse) :
	\begin{equation}
		\begin{aligned}[]
			x_0&\text{ est un élément de $A$},\\
			x_1&\text{ est un minorant de $A$},\\
			a_0&=x_0\\
			b_0&=x_1\\
			b_1&=x_1.
		\end{aligned}
	\end{equation}
	Ensuite, nous faisons la récurrence suivante :
	\begin{equation}
		\begin{aligned}[]
			x_{n+1}&=\frac{ a_n+b_n }{2},\\
			a_{n+1}&=\begin{cases}
				a_{n}	&	\text{si $x_{n+1}$ minore $A$}\\
				x_{n+1}	&	 \text{sinon},
			\end{cases}\\
			b_{n+1}&=\begin{cases}
				x_{n+1}	&	\text{si $x_{n+1}$ minore $A$}\\
				b_n	&	 \text{sinon}.
			\end{cases}
		\end{aligned}
	\end{equation}
    Nous allons montrer que \( a_n\) et \( (b_n)\) sont des suites convergentes de même limite et que cette limite est l'infimum de \( A\).

	Soit $n\in\eN$; il y a deux possibilités. Soit $a_n=a_{n-1}$ et $b_n=x_n$, soit $a_n=x_n$ et $b_n=b_{n-1}$. Supposons que nous soyons dans le premier cas (le second se traite de façon similaire). Alors nous avons
	\begin{equation}
		\begin{aligned}[]
			| a_n-b_n |&=| a_{n-1}-x_n |\\
			&=\left| a_{n-1}-\frac{ a_{n-1}+b_{n-1} }{2} \right| \\
			&=\frac{ 1 }{2}| a_{n-1}-b_{n-1} |,
		\end{aligned}
	\end{equation}
	ce qui prouve que $| a_n-b_n |\to 0$. Nous montrons maintenant que la suite \( (a_n)\) est de Cauchy. En effet nous avons
    \begin{equation}
        | a_n-a_{n-1} |=\begin{cases}
          0\\
          \left| \frac{ a_n -b_n}{ 2} \right|   
      \end{cases}\leq \frac{1}{ 2n }.
    \end{equation}
    Il en est de même pour la suite \( (b_n)\). Ce sont deux suites de Cauchy (donc convergentes par le théorème \ref{ThoTFGioqS}) qui convergent vers la même limite. Soit \( \ell\) cette limite.
    
	Le nombre $\ell$ minore $A$. En effet si $a\in A$ est plus petit que $\ell$, les éléments $b_n$ tels que $| b_n-\ell |<| a-\ell |$ ne peuvent pas minorer $A$. D'autre part, pour tout $\epsilon$, le nombre $\ell+\epsilon$ ne peut pas minorer $A$. En effet, $\ell$ est la limite de la suite décroissante $(a_n)$, donc il existe $a_n$ entre $\ell$ et $\ell+\epsilon$. Mais $a_n$ ne minore pas $A$, donc $\ell+\epsilon$ ne minore pas non plus $A$.

	Nous avons prouvé que toute partie minorée de $\eR$ possède un infimum. La preuve que toute partie majorée possède un supremum se fait de la même façon.
	
\end{proof}


\begin{definition}
	Si le supremum d'un ensemble appartient à l'ensemble, nous l'appelons \defe{maximum}{maximum}. De la même façon si l'infimum d'un ensemble appartient à l'ensemble, nous disons que c'est le \defe{minimum}{minimum}.
\end{definition}

\begin{example}
	Pour les intervalles, ces notions sont simples : les bornes de l'intervalle sont les supremum et infimum, et ce sont des minima et maxima si l'intervalle est fermé. Le nombre $53$ est un majorant.
	\begin{enumerate}
		\item
			$A=\mathopen[ 1 , 2 \mathclose]$. Tous les nombres plus petits ou égaux à $1$ sont minorants, $1$ est infimum et minimum. Le nombre $2$ est un majorant, le maximum et le supremum.
		\item
			$B=\mathopen] 3 , \pi \mathclose[$. Le nombre $\pi$ est le supremum et est un majorant, mais n'est pas le maximum (parce que $\pi\notin B$). L'ensemble $B$ n'a pas de maximum. Bien entendu, $-1000$ est un minorant.
	\end{enumerate}
\end{example}

Il existe évidement de nombreux exemples plus vicieux.

\begin{example}
	Prenons $E=\{ \frac{1}{ n }\tq n\in\eN_0 \}$, dont les premiers points sont indiqués sur la figure \ref{LabelFigSuiteUnSurn}. Cet ensemble est constitué des nombres $1$, $\frac{ 1 }{2}$, $\frac{1}{ 3 }$, \ldots Le plus grand d'entre eux est $1$ parce que tous les nombres de la forme $\frac{1}{ n }$ avec $n\geq 1$ sont plus petits ou égaux à $1$. Le nombre $1$ est donc maximum de $E$.

	L'ensemble $E$ n'a par contre pas de minimum parce que tout élément de $E$ s'écrit $\frac{1}{ n }$ pour un certain $n$ et est plus grand que $\frac{1}{ n+1 }$ qui est également dans $E$.

	Prouvons que zéro est l'infimum de $E$. D'abord, tous les éléments de $E$ sont strictement positifs, donc zéro est certainement un minorant de $E$. Ensuite, nous savons que pour tout $\varepsilon>0$, il existe un $n$ tel que $\frac{1}{ n }$ est plus petit que $\varepsilon$. L'ensemble $E$ possède donc un élément plus petit que $0+\varepsilon$, et zéro est bien l'infimum.
\end{example}

\newcommand{\CaptionFigSuiteUnSurn}{Les premiers points du type $x_n=1/n$.}
\input{Fig_SuiteUnSurn.pstricks}

L'exemple suivant est une source classique d'erreurs en ce qui concerne l'infimum. Il sera à relire après avoir vu la définition de limite (définition \ref{DefLimiteSuiteNum}).

\begin{example}
	Les premiers points de l'ensemble $F=\{ \frac{ (-1)^n }{ n }\tq n\in\eN_0 \}$ sont représentés à la figure \ref{LabelFigSuiteInverseAlterne}. Bien que (comme nous le verrons plus tard) la limite de la suite $x_n=(-1)^n/n$ soit zéro, il n'est pas correct de dire que zéro est l'infimum de l'ensemble $F$. Le dessin, au contraire, montre bien que $-1$ est le minium (aucun point est plus bas que $-1$), tandis que le maximum est $1/2$.

	Nous reviendrons avec cet exemple dans la suite. Pour l'instant, ayez bien en tête que zéro n'est rien de spécial pour l'ensemble $F$ en ce qui concerne les notions de maximum, minimum et compagnie.
\end{example}
\newcommand{\CaptionFigSuiteInverseAlterne}{Les quelque premiers points du type $(-1)^n/n$.}
\input{Fig_SuiteInverseAlterne.pstricks}

%--------------------------------------------------------------------------------------------------------------------------- 
\subsection{Intervalles et connexité}
%---------------------------------------------------------------------------------------------------------------------------

Lorsque $x\in E$, nous disons qu'un \defe{voisinage}{voisinage} de $x$ est n'importe quel sous-ensemble de $E$ qui contient une boule ouverte centrée en $x$. Nous disons qu'un ensemble est \defe{ouvert}{ouvert} si il contient un voisinage de chacun de ses points. Par convention, nous disons que l'ensemble vide est ouvert.

\begin{definition}
L'ensemble des boules ouvertes d'un espace métrique forment la \defe{topologie}{topologie!métrique} de l'espace.
\end{definition}

Nous allons dire qu'une partie $A$ d'un espace métrique est \defe{bornée}{bornée} si il existe une boule\footnote{À titre d'exercice, je te laisse te convaincre que l'on peut dire boule \emph{ouverte} ou \emph{fermée} au choix sans changer la définition.} qui contient $A$.

\begin{lemma}  \label{LemSupOuvPas}
Le supremum d'un ensemble ouvert n'est pas dans l'ensemble (et n'est donc pas un maximum).
\end{lemma}

\begin{proof}
Soit $\mO$, un ensemble ouvert et $s$, son supremum. Si $s$ était dans $\mO$, on aurait un voisinage $B=B(s,r)$ de $s$ contenu dans $\mO$. Le point $s+r/2$ est alors à la fois dans $\mO$ et plus grand que $s$, ce qui contredit le fait que $s$ soit un supremum de $\mO$.
\end{proof}

Par le même genre de raisonnements, on montre que l'union et l'intersection de deux ouverts sont encore des ouverts.

\begin{remark}
L'intersection d'une \emph{infinité} d'ouverts n'est pas spécialement un ouvert comme le montre l'exemple suivant :
\[ 
  \mO_i=]1,2+\frac{ 1 }{ i }[.
\]
Tous les ensembles $\mO_i$ contiennent le point $2$ qui est donc dans l'intersection. Mais quel que soit le $\epsilon>0$ que l'on choisisse, le point $2+\epsilon$ n'est pas dans $\mO_{(1/\epsilon)+1}$. Donc aucun point au-delà de $2$ n'est dans l'intersection, ce qui prouve que $2$ ne possède pas de voisinages contenus dans $\cap_{i=1}^{\infty}\mO_i$.
\end{remark}

\begin{proposition}
Prouver que, quels que soient les ensembles $A$ et $B$ dans $\eR$, nous avons
\[ 
  \sup(A\cap B)\leq\sup A\leq\sup(A\cup B).
\]
\end{proposition}


%--------------------------------------------------------------------------------------------------------------------------- 
\subsection{Connexité et intervalles}
%---------------------------------------------------------------------------------------------------------------------------

Nous allons déterminer tous les sous-ensembles connexes de $\eR$. Pour cela nous avons besoin d'une définition précise de ce que l'on appelle un \emph{intervalle} dans~$\eR$.
\begin{definition}
    Un \defe{intervalle}{intervalle} est une partie de $\eR$ telle que tout élément compris entre deux éléments de la partie soit dedans. En formule, la partie $I$ de $\eR$ est un intervalle si
    \[
      \forall a,b\in I,(a\leq x\leq b)\Rightarrow x\in I.
    \]
\end{definition}
Cette définition englobe tous les exemples connus d'intervalles ouverts, fermés avec ou sans infini : $[a,b]$, $[a,b[$, $]-\infty,a]$, \ldots

Une des nombreuses propositions qui vont servir à prouver le théorème des \href{http://fr.wikipedia.org/wiki/Théorème_des_valeurs_intermédiaires}{valeurs intermédiaires} (théorème numéro \ref{ThoValInter}) est la suivante.
\begin{proposition} \label{PropInterssiConn}
    Une partie de $\eR$ est connexe si et seulement si c'est un intervalle.
\end{proposition}
\index{connexité!et intervalles}

\begin{proof}
    La preuve est en deux partie. D'abord nous démontrons que si un sous-ensemble de $\eR$ est connexe, alors c'est un intervalle; et ensuite nous démontrons que tout intervalle est connexe.

    Affin de prouver qu'un ensemble connexe est toujours un intervalle, nous allons prouver que si un ensemble n'est pas un intervalle, alors il n'est pas connexe. Prenons $A$, une partie de $\eR$ qui n'est pas un intervalle. Il existe donc $a$, $b\in A$ et un $x_0$ entre $a$ et $b$ qui n'est pas dans $A$. Comme le but est de prouver que $A$ n'est pas connexe, il faut couper $A$ en deux ouverts disjoints. L'élément $x_0$ qui n'est pas dans $A$ est le bon candidat pour effectuer cette coupure. Prenons $M$, un majorant de $A$ et $m$, un minorant de $A$, et définissons 
    \begin{align*}
        \mO_1&=]m,x_0[\\
        \mO_2&=]x_0,M[.
    \end{align*}
    Si $A$ n'a pas de minorant, nous remplaçons la définition de $\mO_1$ par $]-\infty,x_0[$, et si $A$ n'a pas de majorant, nous remplaçons la définition de $\mO_2$ par $]x_0,\infty[$. Dans tous les cas, ce sont deux ensembles ouverts dont l'union recouvre tout $A$. En effet, $\mO_1\cup \mO_2$ contient tous les nombres entre un minorant de $A$ et un majorant sauf $x_0$, mais on sait que $x_0$ n'est pas dans $A$. Cela prouve que $A$ n'est pas connexe.

    Jusqu'à présent nous avons prouvé que si un ensemble n'est pas un intervalle, alors il ne peut pas être connexe. Pour remettre les choses à l'endroit, prenons un ensemble connexe, et demandons-nous si il peut être autre chose qu'un intervalle ? La réponse est \emph{non} parce que si il était autre chose, il ne serait pas connexe.

    Prouvons à présent que tout intervalle est connexe. Pour cela, nous refaisons le coup de \href{http://fr.wikipedia.org/wiki/Contraposée}{la contraposée}. Nous allons donc prendre une partie $A$ de $\eR$, supposer qu'elle n'est pas connexe et puis prouver qu'elle n'est alors pas un intervalle. Nous avons deux ouverts disjoints $\mO_1$ et $\mO_2$ tels que $A\subset \mO_1\cup \mO_2$. Prenons $a\in A_1$ et $b\in A_2$. Pour fixer les idées, on suppose que $a<b$. Maintenant, le jeu est de montrer qu'il existe une point $x_0$ entre $a$ et $b$ qui ne soit pas dans $A$ (cela montrerait que $A$ n'est pas un intervalle). Nous allons prouver que c'est le cas du point
    \[ 
      x_0=\sup\{ x\in\mO_1\tq x<b \}.
    \]
    Étant donné que l'ensemble $\mA=\{ x\in\mO_1\tq x<b \}$ est ouvert\footnote{C'est l'intersection entre l'ouvert $\mO_1$ et l'ouvert $\{x\tq x<b \}$.}, le point $x_0$ n'est pas dans l'ensemble par le lemme \ref{LemSupOuvPas}. Nous avons donc
    \begin{itemize}
        \item soit $x_0$ n'est pas dans $\mO_1$,
        \item soit $x_0\leq b$,
        \item soit les deux en même temps.
    \end{itemize}
    Nous allons montrer qu'un tel $x_0$ ne peut pas être dans $A$. D'abord, remarquons que $\sup\mA\leq\sup\mO$ parce que $\mA$ est une intersection de $\mO$ avec quelque chose. Ensuite, il n'est pas possible que $x_0$ soit dans $\mO_2$ parce que tout élément de $\mO_2$ possède un voisinage contenu dans $\mO_2$. Un point de $\mO_2$ est donc toujours strictement plus grand que le supremum de $\mO_1$.

    Maintenant, remarque que si $x_0\leq b$, alors $x_0=b$, sinon $b$ serait un majorant de $\mA$ plus petit que $x_0$, ce qui n'est pas possible vu que $x_0$ est le supremum de $\mA$ et donc le plus petit majorant. Oui mais si $x_0=b$, c'est que $x_0\in\mO_2$, ce qu'on vient de montrer être impossible. Nous voila déjà débarrassé des deuxièmes et troisièmes possibilités. 

    Si la première possibilité est vraie, alors $x_0$ n'est pas dans $A$ parce qu'on a aussi prouvé que $x_0\notin\mO_2$. Or n'être ni dans $\mO_1$ ni dans $\mO_2$ implique de ne pas être dans $A$. Ce point $x_0=\sup\mA$ est donc hors de $A$.

    Oui, mais comme $a\in\mA$, on a obligatoirement que $x_0\geq a$. Mais par construction, on a aussi que $x_0\leq b$ (ici, l'inégalité est même stricte, mais ce n'est pas important). Donc
    \[ 
      a\leq x_0\leq b
    \]
    avec $a$, $b\in A$, et $x_0\notin A$. Cela finit de prouver que $A$ n'est pas un intervalle.
\end{proof}

%+++++++++++++++++++++++++++++++++++++++++++++++++++++++++++++++++++++++++++++++++++++++++++++++++++++++++++++++++++++++++++
\section{Ensembles nulle part denses}
%+++++++++++++++++++++++++++++++++++++++++++++++++++++++++++++++++++++++++++++++++++++++++++++++++++++++++++++++++++++++++++

Nous allons nous limite au cas de \( \eR\), mais je crois que ça se généralise sans trop de peine aux espaces en tout cas métriques.

\begin{definition}
    Un ensemble est dit \defe{nulle part dense}{nulle part dense}\index{dense!nulle part} si il n'est dense dans aucun intervalle.

    Un ensemble dans \( \eR\) est de \defe{première catégorie}{catégorie!ensemble de première} ou \defe{maigre}{maigre (ensemble)} si il est une union dénombrable d'ensembles nulle part dense (c'est à dire d'ensembles denses sur aucun intervalle).
\end{definition}

\begin{theorem}[Baire\cite{BaireZied}]  \index{Baire!théorème}\index{théorème!Baire}    \label{ThoQGalIO}
    Une réunion dénombrable d'ensembles nulle part denses est d'intérieur vide.
\end{theorem}

\begin{proof}
    Soit \( a\in S\) et \( \epsilon>0\). Nous allons trouver un élément dans \( B(a,\epsilon)\) qui n'est pas dans \( S\). Nous commençons par choisir \( x_1\in B(a,\epsilon)\) et \( r_1<\frac{ \epsilon }{2}\) tel que
    \begin{equation}
        B(x_1,r_1)\cap A_1=\emptyset.
    \end{equation}
    Ensuite nous choisissons \( x_2\in B(x_1,r_1)\) et \( r_2<\epsilon/4\) tel que \( B(x_2,r_2)\subset B(x_1,r_1)\) et \( B(x_2,r_2)\cap A_2=\emptyset\). Notons que \( B(x_2,r_2)\cap A_1=\emptyset\) aussi, par construction.

    Par récurrence nous construisons une suite d'éléments \( x_n\) et de rayons \( r_n<\epsilon/2^n\) tels que
    \begin{enumerate}
        \item
            \( B(x_n,r_n)\cap A_j=\emptyset\) pour tout \( j\leq n\),
        \item
            \( \overline{ B(x_n,r_n) }\subset B(x_{n-1},r_{r-1})\).
    \end{enumerate}
    Cette suite étant de Cauchy (parce que contenue dans des intervalles emboités de rayon décroissant vers zéro), elle converge\footnote{Par le théorème \ref{ThoTFGioqS}} donc vers un point qui en particulier appartient à \( B(a,\epsilon)\). Mais la limite n'est dans aucun des \( A_n\) et donc pas dans \( S\).
\end{proof}

%%%%%%%%%%%%%%%%%%%%%%%%%%
%
   \section{Topologie dans \texorpdfstring{$\eR^n$}{Rn}}
%
%%%%%%%%%%%%%%%%%%%%%%%%

Dans cette section, nous travaillons dans l'espace $\eR^n$ pour un certain naturel $n$. Nous y définissons la notion d'ouvert et de fermé, qui sont la base de la topologie générale. Notons que ces définitions n'ont de sens que relativement à l'espace ambiant, aussi un ouvert de $\eR$ ne sera en général pas un ouvert de $\eR^2$~: d'une part, il n'y a pas d'inclusion canonique de $\eR$ dans $\eR^2$ (les ouverts du second ne sont même pas des sous-ensembles du premier) et, d'autre part, les définitions se basent sur la notion de boule de $\eR^n$ qui dépend évidemment de la valeur de $n$ (une boule dans $\eR$ est un intervalle, dans $\eR^2$ c'est un disque, etc.)

%---------------------------------------------------------------------------------------------------------------------------
					\subsection{Ouverts et fermés}
%---------------------------------------------------------------------------------------------------------------------------

\begin{definition}
	La \defe{boule ouverte}{boule!ouverte} de centre $x_0 \in \eR^n$ et de rayon $r \in
	\eR^+$ est définie par
	\begin{equation}
		B(x_0,r) = \{ x \in \eR^n \tq \norme{x - x_0} < r \},
	\end{equation}
	tandis que la \defe{boule fermée}{boule!fermée} de centre $x_0$ et de rayon $r$ est
	\begin{equation}
		\bar B(x_0,r) = \{ x \in \eR^n \tq \norme{x - x_0} \leq r \};
	\end{equation}
	la différence est que l'inégalité dans la première est stricte.
\end{definition}

%---------------------------------------------------------------------------------------------------------------------------
					\subsection{Intérieur, adhérence et frontière}
%---------------------------------------------------------------------------------------------------------------------------

\begin{definition}
  Soit $A \subset \eR^n$ et $x \in \eR^n$. Le point $x$ est \defe{intérieur}{intérieur} à $A$ si il existe une boule autour de $x$ complètement contenue dans $A$. L'ensemble des points intérieurs à $A$ est noté $\interieur A$ ou $\mathring A$, de sorte qu'on a précisément
  \begin{equation*}
    x \in \interieur A \iffdefn  \exists \epsilon > 0 \tq
    B(x,\epsilon) \subset A.
  \end{equation*}
\end{definition}


\begin{definition}
Le point $x$ est dans l'\defe{adhérence}{adhérence} de $A$ si toute boule autour de $x$ intersecte $A$. L'ensemble de ces points est noté $\adh A$ ou $\bar A$, et on a donc de manière plus précise
\begin{equation}
	x \in \adh A \iffdefn \forall \epsilon > 0, B(x,\epsilon) \cap A \neq \emptyset
\end{equation}
\end{definition}

\begin{proposition}
Pour $A \subset \eR^n$, nous avons
\begin{equation*}
	\interieur A \subseteq A  \subseteq \adh A
\end{equation*}
\end{proposition}

\begin{definition}
  La \defe{frontière}{frontière} ou le \defe{bord}{bord} de $A$ est défini par $\partial A = \adh A \setminus \interieur A$. L'ensemble $A$ est un \defe{ouvert}{ouvert} si $A = \interieur A$, et c'est un \defe{fermé}{fermé} si $A = \adh A$.
\end{definition}

On vérifiera que les notations et les dénominations sont cohérentes en
prouvant la proposition suivante.
\begin{proposition}Pour $\epsilon > 0$,
  \begin{enumerate}
  \item l'adhérence de $B(x,\epsilon)$ est $\bar B(x,\epsilon)$,
  \item l'intérieur de $\bar B(x,\epsilon)$ est $B(x,\epsilon)$,
  \item la boule ouverte $B(x,\epsilon)$ est un ouvert,
  \item la boule fermée $\bar B(x,\epsilon)$ est un fermé.
  \end{enumerate}
\end{proposition}

Nous avons également les liens suivants entre intérieur, adhérence,
ouvert, fermé et passage au complémentaire (noté ${}^c$)~:
\begin{proposition}
Si $A \subset \eR^n$ et $A^c = \eR^n\setminus A$, nous
  avons
  \begin{enumerate}
  \item $(\interieur A)^c = \adh (A^c)$ et $(\adh A)^c = \interieur
    (A^c)$,
  \item $A$ est ouvert si et seulement si $A^c$ est fermé,
  \item $\interieur A$ est le plus grand ouvert contenu dans $A$,
  \item $\adh A$ est le plus petit fermé contenant $A$,
    % \item
  \end{enumerate}
\end{proposition}

\begin{proposition}		\label{PropCvRpComposante}
	Une suite $(x_n)$ dans $\eR^m$ est convergente dans $\eR^m$ si et seulement si les suites de chaque composantes sont convergentes dans $\eR$. Dans ce cas nous avons
	 \begin{equation}
		 \lim x_n=\Big( \lim(x_n)_1,\lim (x_n)_2,\ldots,\lim (x_n)_m \Big)
	 \end{equation}
	 où $(x_n)_k$ dénote la $k$-ième composante de $(x_n)$.
\end{proposition}

\begin{example}
	La suite $x_n=\big( \frac{1}{ n },1-\frac{1}{ n } \big)$ converge vers $(0,1)$ dans $\eR^2$. En effet, en utilisant la proposition \ref{PropCvRpComposante}, nous devons calculer séparément les limites
	\begin{equation}
		\begin{aligned}[]
			\lim\frac{1}{ n }&=0\\
			\lim\big( 1-\frac{1}{ n } \big)&=1.
		\end{aligned}
	\end{equation}
\end{example}

\begin{example}
	Étant donné que la suite $(-1)^n$ n'est pas convergente, la suite $x_n=\big( (-1)^n,\frac{1}{ n } \big)$ n'est pas convergente dans $\eR^2$.
\end{example}

%+++++++++++++++++++++++++++++++++++++++++++++++++++++++++++++++++++++++++++++++++++++++++++++++++++++++++++++++++++++++++++
\section{Point d'accumulation, point isolé}
%+++++++++++++++++++++++++++++++++++++++++++++++++++++++++++++++++++++++++++++++++++++++++++++++++++++++++++++++++++++++++++

Soit $D\subset\eR$. Un point $a\in D$ est \defe{isolé}{isolé!élément de $\eR$} dans $D$ (relativement à $\eR$) si il existe $\varepsilon>0$ tel que 
\begin{equation}
	\mathopen[ a-\varepsilon , a+\varepsilon \mathclose]\cap D=\{ a \}.
\end{equation}
Autrement dit, il existe un intervalle autour de $a$ dans lequel $a$ est le seul élément de $D$.

Un point $a\in \eR$ est un \defe{point d'accumulation}{accumulation!dans $\eR$} de $D$ si pour tout $\varepsilon>0$, 
\begin{equation}
	\Big( \mathopen[ a-\varepsilon , a+\varepsilon \mathclose]\setminus\{ a \} \Big)\cap D\neq\emptyset.
\end{equation}
Autrement dit, quel que soit l'intervalle autour de  $a$ que l'on considère, le point $a$ n'est pas tout seul dans $D$.

\begin{example}
	Prenons $D=\mathopen[ 0 , 1 [\cup\mathopen] 2 , 3 \mathclose]$. Cet ensemble n'a pas de points isolés, et l'ensemble de ses points d'accumulation est $\mathopen[ 0 , 1 \mathclose]\cup\mathopen[ 2,3  \mathclose]$.

	Notez que les points $1$ et $2$ sont des points d'accumulation de $D$ qui ne font pas partie de $D$. Il est possible d'être un point d'accumulation de $D$ sans être dans $D$, mais pour être un point isolé dans $D$, il faut être dans $D$.
\end{example}

\begin{example}
	Soit $D=\{ \frac{1}{ n }\}_{n\in\eN}$. Tous les points de cet ensemble sont des points isolés (vérifier !).  Aucun point de $D$ n'est point d'accumulation. Cependant $0$ est un point d'accumulation.
\end{example}

% TODO: retrouver où je comptais mettre cette référence.
\cite{GGIibHE}

%+++++++++++++++++++++++++++++++++++++++++++++++++++++++++++++++++++++++++++++++++++++++++++++++++++++++++++++++++++++++++++
\section{Limites de suites}
%+++++++++++++++++++++++++++++++++++++++++++++++++++++++++++++++++++++++++++++++++++++++++++++++++++++++++++++++++++++++++++

\begin{definition}[Limite d'une suite dans $\eR^m$]
	Une suite de points $(x_n)$ dans $\eR^m$ est dite \defe{convergente}{convergence!suite dans $\eR^m$} si il existe un élément $\ell\in\eR^m$ tel que
	\begin{equation}	\label{EqCondLimSuite}
		\forall\varepsilon>0,\,\exists N\in \eN\tq\,\forall n\geq N,\,\| x_n-\ell \|<\varepsilon.
	\end{equation}
	Dans ce cas, nous disons que $\ell$ est la \defe{limite}{limite!suite dans $\eR^m$} de la suite $(x_n)$ et nous écrivons $\lim x_n=\ell$ ou plus simplement $x_n\to \ell$.
\end{definition}
Notez aussi la similarité avec la définition \ref{DefLimiteSuiteNum}.

\begin{remark}
	Nous n'écrivons pas «$\lim_{n\to\infty}x_n$» parce que, lorsqu'on parle de suites, la limite est \emph{toujours} lorsque $n$ tend vers l'infini. Il n'y a aucun intérêt à chercher par exemple $\lim_{n\to 4}x_n$ parce que cela vaudrait $x_4$ et rien d'autre.

	Ceci est une différence importante avec les limites de fonctions.
\end{remark}

\begin{lemma}[Unicité de la limite]
	Il ne peut pas y avoir deux nombres différents qui satisfont à la condition \eqref{EqCondLimSuite}. En d'autres termes, si $\ell$ et $\ell'$ sont deux limites de la suite $(x_n)$, alors $\ell=\ell'$.
\end{lemma}

\begin{proof}
	Soit $\varepsilon>0$. Nous considérons $N$ tel que
	\begin{equation}
		\| x_n-\ell \|<\varepsilon
	\end{equation}
	pour tout $n\geq N$, et $N'>0$ tel que 
	\begin{equation}
		\| x_n-\ell' \|<\epsilon
	\end{equation}
	pour tout $n>N'$. Maintenant, nous prenons $n$ plus grand que $N$ et $N'$ de telle façon à ce que $x_n$ vérifie les deux inéquations en même temps. Alors
	\begin{equation}
		\| \ell-\ell' \|=\| \ell-x_n+x_n-\ell' \|\leq\| \ell-x_n \|+\| x_n-\ell' \|<2\varepsilon.
	\end{equation}
	Cela prouve que $\| \ell-\ell' \|=0$.
\end{proof}

%---------------------------------------------------------------------------------------------------------------------------
					\subsection{Bornés et compacts}
%---------------------------------------------------------------------------------------------------------------------------


\begin{definition}
  Un sous ensemble $A \subset \eR^n$ est \defe{borné}{borné} si il existe une boule de $\eR^n$ contenant $A$.
\end{definition}

\begin{proposition}
  Toute réunion finie d'ensembles bornés est un ensemble borné. Toute
  partie d'un ensemble borné est un ensemble borné.
\end{proposition}


% TODO: regarder ceci à propos des compacts.
% En particulier, si on recouvre $A$ par l'ensemble des boules
% $B(x,1)$ où $x$ parcourt $A$ (de sorte que tout point de $A$ est
% dans \og sa\fg{} boule, et donc la réunion des boules recouvre bien
% $A$), on doit pouvoir en tirer un recouvrement fini, c'est-à-dire
% des boules $B(x_1,1), B(x_2,1), \ldots, B(x_k,1)$ (avec $k$ un
% naturel) dont la réunion contient $A$.

\begin{proposition}
Une partie de $\eR^n$ est compacte si et seulement si elle est fermée et bornée.
\end{proposition}

%---------------------------------------------------------------------------------------------------------------------------
					\subsection{Connexité}
%---------------------------------------------------------------------------------------------------------------------------

\begin{definition}
  Le sous ensemble $A \subset \eR^n$ est \defe{connexe par arcs}{connexe!par arc} si pour tout $x, y \in
  A$, il existe un chemin\footnote{Attention : ici quand on dit \emph{chemin}, on demande que l'application soit continue. Dans de nombreux cours de géométrie différentielle, on demande $ C^{\infty}$. Il faut s'adapter au contexte.} contenu dans $A$ les reliant, c'est-à-dire
  une application continue
  \begin{equation*}
    \gamma : [0,1] \to \eR^n \tq \gamma(0) = x~\text{et}~\gamma(1) = y
  \end{equation*}
  avec $\gamma(t) \in A$ pour tout $t\in [0,1]$.
\end{definition}

\begin{theorem}
	Toute suite réelle contenue dans un compact admet une sous-suite convergente.
\end{theorem}

\begin{proof}
	Soit $(x_n)$ une suite contenue dans la partie bornée $A\subset\eR$. Nous disons qu'un élément $x_k$ de la suite est \emph{maximal} si il est plus grand ou égal que tous les suivants : $x_k\geq x_{k'}$ dès que $k'\geq k$.

	Si la suite a un nombre infini d'éléments maximaux, alors la suite des éléments maximaux est décroissante. Si nous n'avons qu'un nombre fini de points maximaux, alors la suite de départ est croissante à partir du dernier point maximal.

	Dans les deux cas nous avons trouvé une sous-suite des $x_n$ qui est monotone (décroissante ou croissante selon le cas), et donc convergente parce que contenue dans un borné (lemme \ref{LemSuiteCrBorncv}).
\end{proof}

%TODO : le théorème sur l'équivalence des normes sur les espaces vectoriels normés devrait être énoncé comme le fait que si N1 et N2 sont deux normes sur V, alors 
%       nous avons un isomorphisme d'espace topologique (V,N1) ~ (V,N2). L'isomorphisme étant donné par l'identité.

\begin{theorem}[Théorème de Bolzano-Weierstrass]		\label{ThoBolzanoWeierstrassRn}\label{PropSuiteCompactSScv}
	Toute suite contenue dans un compact de \( \eR^m\) admet une sous-suite convergente.
\end{theorem}
Ce théorème sera démontré dans un espace topologique général par le théorème \ref{ThoBWFTXAZNH}. Nous donnons ici une preuve plus directe valable dans un espace vectoriel normé.

\begin{proof}
    Soit $(x_n)$ une suite contenue dans une partie bornée de $\eR^m$. Considérons $(a_n)$, la suite réelle des premières composantes des éléments de $(x_n)$ : pour chaque $n\in\eN$, le nombre $a_n$ est la première composante de $x_n$. Étant donné que la suite $(x_n)$ est bornée, il existe un $M$ tel que $\| x_n \|<M$. La croissance de la fonction racine carré donne
	\begin{equation}
        | a_n |\leq\| x_n \|\leq M.
	\end{equation}
	La suite $(a_n)$ est donc une suite réelle bornée et donc contient une sous-suite convergente. Soit $a_{I_1}$ une sous-suite convergente de $(a_n)$. Nous considérons maintenant $x_{I_1}$, c'est à dire la suite de départ dont on a enlevé tous les éléments qu'il faut pour qu'elle converge en ce qui concerne la première composante.

	Si nous considérons la suite $b_{I_1}$ des \emph{secondes} composantes de $x_{I_1}$, nous en extrayons, de la même façon que précédemment, une sous-suite convergente, c'est à dire que nous avons un $I_2\subset I_1$ tel que $b_{I_2}$ est convergent. Notons que $a_{I_2}$ est une sous-suite de la (sous) suite convergente $x_{I_1}$, et donc $a_{I_2}$ est encore convergente.

	En continuant ainsi, nous construisons une sous-sous-sous-suite $x_{I_3}$ telle que la suite des \emph{troisième} composantes est convergente. Lorsque nous avons effectué cette procédure $m$ fois, la suite $x_{I_m}$ est une suite dont toutes les composantes convergent, et donc est une suite convergente par la proposition \ref{PropCvRpComposante}.
	
	Le tableau suivant donne un petit schéma de la façon dont nous procédons. Les $\bullet$ sont les éléments de la suite que nous gardons, et les $\times$ sont ceux que nous «jetons».
	\begin{equation}
		\begin{array}{lccccccccccc}
			x_{\eN}	&	\bullet&\bullet&\bullet&\bullet&\bullet&\bullet&\bullet&\bullet&\bullet&\bullet&\ldots\\
			x_{I_1}	&	\times&\bullet&\bullet&\times&\bullet&\times&\times&\bullet&\bullet&\bullet&\ldots\\
			x_{I_2}	&	\times&\bullet&\times&\times&\bullet&\times&\times&\bullet&\bullet&\times&\ldots\\
			\vdots\\
			x_{I_m}	&	\times&\times&\times&\times&\bullet&\times&\times&\times&\bullet&\times&\ldots
		\end{array}
	\end{equation}
	La première ligne, $x_{\eN}$, est la suite de départ.
\end{proof}

%+++++++++++++++++++++++++++++++++++++++++++++++++++++++++++++++++++++++++++++++++++++++++++++++++++++++++++++++++++++++++++
					\section{Uniforme continuité}
%+++++++++++++++++++++++++++++++++++++++++++++++++++++++++++++++++++++++++++++++++++++++++++++++++++++++++++++++++++++++++++

\begin{proposition}	\label{PropoInvCompCont}
Soit $f\colon A\subset\eR^n\to B\subset\eR^m$ une bijection continue. Si $A$ est compact, alors $f^{-1}\colon B\to A$ est continue.
\end{proposition}

\begin{proposition}		\label{PropIntContMOnIvCont}
Soient $I$ un intervalle dans $\eR$ et $f\colon I\to \eR$ une fonction continue strictement monotone. Alors la fonction réciproque $f^{-1}\colon f(I)\to \eR$ est continue sur l'intervalle $f(I)$.
\end{proposition}

