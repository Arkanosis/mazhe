% This is part of the Exercices et corrigés de CdI-2.
% Copyright (C) 2008, 2009, 2012
%   Laurent Claessens
% See the file fdl-1.3.txt for copying conditions.

\begin{corrige}{CourbesSurfaces0017}


Étant donné que nous ne définissons $F(x,y)$ que pour des $(x,y)\in D$, la fonction $t\mapsto f(tx,ty)$ est $C^1$ sur tout le compact $[0,1]$ et aucune divergence de l'intégrale n'est à craindre. Nous sommes donc dans le cadre de la proposition \ref{PropDerrSSIntegraleDSD}, et nous pouvons dériver sous le signe intégral.

Nous calculons, en utilisant la règle de dérivation de fonctions composées
\begin{equation}		\label{EqI33dsdsFlolo}
	\begin{aligned}[]
		\frac{ \partial F }{ \partial x }(x,t)	&=\int_0^1\left[   f\frac{ \partial f }{ \partial x }(tx,ty)x+f(tx,ty)+t\frac{ \partial g }{ \partial x }(tx,ty)y  \right]dt\\
		&=\int_0^1\left[ t\Big( x\frac{ \partial f }{ \partial x }(tx,ty)+y\frac{ \partial f }{ \partial y }(tx,ty) \Big)+f(tx,ty) \right]dt
	\end{aligned}
\end{equation}
où nous avons utilisé l'hypothèse $\partial_yf=\partial_xg$. Ce qui se trouve dans la parenthèse n'est autre que $\partial_t\big( f(tx,ty) \big)$, plus précisément, si nous posons $\mF(x,y,t)=f(tx,ty)$, nous avons
\begin{equation}
	\frac{ \partial \mF }{ \partial t }(x,y,t)= x\frac{ \partial f }{ \partial x }(tx,ty)+y\frac{ \partial f }{ \partial y }(tx,ty).
\end{equation}
En recopiant le résultat \eqref{EqI33dsdsFlolo} en termes de $\mF$, nous avons
\begin{equation}
	\begin{aligned}[]
		\frac{ \partial F }{ \partial x }(x,t)	&=\int_0^1\left( t\frac{ \partial \mF }{ \partial t }(x,y,t)+\mF(x,y,t) \right)dt\\
		&=\int_0^1\partial_t\big( t\mF(x,y,t) \big)dt\\
		&=\big[ f\mF(x,y,t) \big]_0^1\\
		&=\mF(x,y,1)\\
		&=f(x,y).
	\end{aligned}
\end{equation}
Le résultat correspondant pour $\frac{ \partial F }{ \partial y }(x,y)=g(x,y)$ s'obtient de la même manière.

\begin{alternative}
Nous posons $u=tx$ et $v=ty$, ainsi que $\mF(x,y,t)=f(u,v)$ et $\mG(x,y,t)=g(u,v)$. Avec cette notation, nous avons $F(x,y)=\int_0^1\big( x\mF(x,y,t)+y\mG(x,y,t) \big)dt$, et
\begin{equation}
	\begin{aligned}[]
		\frac{ \partial \mF }{ \partial x }&=\frac{ \partial f }{ \partial u }\frac{ \partial u }{ \partial x }+\frac{ \partial f }{ \partial v }\frac{ \partial v }{ \partial x }=t\frac{ \partial f }{ \partial u },\\
		\frac{ \partial \mG }{ \partial x }&=t\frac{ \partial g }{ \partial u }.
	\end{aligned}
\end{equation}
Ainsi,
\begin{equation}
	\begin{aligned}[]
		\frac{ \partial F }{ \partial x }	&=\int_0^1\left( x\frac{ \partial \mF }{ \partial x }+\mF+y\frac{ \partial G }{ \partial x } \right)dt\\
							&=\int_0^1\left( xt\frac{ \partial f }{ \partial u } +\mF+yt\frac{ \partial g }{ \partial u } \right)dt\\
							&=\int_0^1\left[  t\left( x\frac{ \partial f }{ \partial u }+y\frac{ \partial f }{ \partial v } \right)+\mF  \right]dt.
	\end{aligned}
\end{equation}
où nous avons utilisé le fait que, par hypothèse, $\frac{ \partial g }{ \partial u }=\frac{ \partial f }{ \partial v }$. Nous calculons par ailleurs que
\begin{equation}
	\frac{ \partial F }{ \partial t }=\frac{ \partial f }{ \partial u }\frac{ \partial u }{ \partial t }+\frac{ \partial f }{ \partial v }\frac{ \partial v }{ \partial t }=x\frac{ \partial f }{ \partial u }+y\frac{ \partial f }{ \partial v }.
\end{equation}
Donc, nous avons
\begin{equation}
	\frac{ \partial F }{ \partial x }=\int_0^1\left( t\frac{ \partial \mF }{ \partial t }+\mF \right)dt=\int_0^1\frac{ \partial  }{ \partial t }(t\mF)dt.
\end{equation}
Par conséquent,
\begin{equation}
	\frac{ \partial F }{ \partial x }=[t\mF]_0^1=\mF(x,y,1)=f(x,y).
\end{equation}
Le même genre de calculs fournit $\frac{ \partial F }{ \partial y }=g(x,y)$.
\end{alternative}

\end{corrige}
