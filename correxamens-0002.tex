% This is part of Mes notes de mathématique
% Copyright (c) 2012
%   Laurent Claessens
% See the file fdl-1.3.txt for copying conditions.


\begin{corrige}{examens-0002}

    En ce qui concerne le domaine, il faut que le contenu du logarithme soit \emph{strictement} positif. Nous demandons donc
    \begin{equation}
        xyz>-1.
    \end{equation}

    Le calcul des dérivée est standard. 
    \begin{verbatim}
----------------------------------------------------------------------
| Sage Version 4.8, Release Date: 2012-01-20                         |
| Type notebook() for the GUI, and license() for information.        |
----------------------------------------------------------------------
sage: var('x,y,z')
(x, y, z)
sage: f(x,y,z)=y*z*ln(1+x*y*z)+2*y*z
sage: f.diff(x)
(x, y, z) |--> y^2*z^2/(x*y*z + 1)
sage: f.diff(y)
(x, y, z) |--> x*y*z^2/(x*y*z + 1) + z*log(x*y*z + 1) + 2*z
sage: f.diff(z)
(x, y, z) |--> x*y^2*z/(x*y*z + 1) + y*log(x*y*z + 1) + 2*y
    \end{verbatim}

Pour savoir leur valeur au point demandé :

    \begin{verbatim}
sage: f.diff(x)(1,1,2)         
4/3
sage: f.diff(y)(1,1,2)
2*log(3) + 16/3
sage: f.diff(z)(1,1,2)
log(3) + 8/3
    \end{verbatim}
    

    Le gradient vaut donc
    \begin{equation}
        \nabla f(1,1,2)=\begin{pmatrix}
            4/3\\ 
            2\ln(3)+16/3    \\ 
            \ln(3)+8/3    
        \end{pmatrix}.
    \end{equation}
    Notez que Sage écrit \info{log} pour le logarithme en base \( e\).

    La différentielle, appliquée au vecteur \( u=(u_1,u_2,u_3)\), est alors
    \begin{equation}
        df_{(1,1,2)}(u)=\frac{ 4 }{ 3 }u_1+\big( 2\ln(3)+\frac{ 16 }{ 3 } \big)u_2+\big( \ln(3)+\frac{ 8 }{ 3 } \big)u_3.
    \end{equation}
    Nous avons appliqué la formule \eqref{Eqsuitedfnfdsdfu}.
    
\end{corrige}
