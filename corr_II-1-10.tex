% This is part of the Exercices et corrigés de CdI-2.
% Copyright (C) 2008, 2009
%   Laurent Claessens
% See the file fdl-1.3.txt for copying conditions.


\begin{corrige}{_II-1-10}

\begin{enumerate}

\item 
$y'=y/(y-t)$.

Nous posons, conformément à \eqref{EqDiffHomoPoser} $y=tz$, donc $y'=z+tz'$, et nous trouvons
\begin{equation}
	tz'=\frac{ z-z^2+z }{ z-1 }=\frac{ z(z-2) }{ 1-z }.
\end{equation}

Si $z(z-2)=0$, alors nous avons les solutions particulières $z(t)=0$ et $z(t)=2$ qui correspondent aux solutions $y(t)=0$ et $y(t)=2t$. Si $z(z-2)\neq 0$, alors nous pouvons faire la manipulation suivante :
\begin{equation}
	\frac{ z'(1-z) }{ z(z-2) }=\frac{1}{ t }.
\end{equation}
Remarquez que le numérateur est à peu près la dérivée du numérateur. En posant $u=z(z-2)$, nous récrivons l'équation sous la forme plus simple
\begin{equation}
	\frac{ -u'/2 }{ u }=\frac{1}{ t }.
\end{equation}
En écrivant $u'=du/dt$ et en faisant passer le $dt$ de l'autre côté, nous avons
\begin{equation}
	\frac{ du }{ u }=-2\frac{ dt }{ t } 
\end{equation}
que nous intégrons des deux côtés :
\begin{equation}
	\ln(u)=\ln(Kt^{-2}),
\end{equation}
ce qui amène au final
\begin{equation}
	y(t)=t\big( 1\pm\sqrt{1+Kt^{-2}} \big).
\end{equation}

\item
$y'=y/(t-2(ty)^{1/2})$.

Nous posons $y=tz$, et après quelques manipulations algébriques nous remettons tout sous la forme
\begin{equation}		\label{EqII110EqPourz}
	\frac{ z'(1-2z^{1/2}) }{ 2z^{3/2} }=\frac{1}{ t }.
\end{equation}
Nous utilisons l'astuce de la section \ref{SecFairedzdt}, en commençant par écrire
\begin{equation}
	\frac{ 1-2z^{1/2} }{ 2z^{3/2} }=\frac{ dt }{ t },
\end{equation}
dont nous intégrons les deux membres :
\begin{equation}
	-\ln(z)-\frac{1}{ \sqrt{z} }=\ln| t |+C,
\end{equation}
où nous avons mit toutes les constantes dans $C$.
Ensuite, nous remettons l'ancienne variable :
\begin{equation}
	\ln| t |=-\ln| \frac{ y }{ t } |-\frac{1}{ \sqrt{y/t} }+C,
\end{equation}
ou encore
\begin{equation}
	\ln| y |+\sqrt{\frac{ t }{ y }}+C=0.
\end{equation}
Cela est une équation implicite pour $y(t)$.

\end{enumerate}


\end{corrige}
