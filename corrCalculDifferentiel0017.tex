\begin{corrige}{CalculDifferentiel0017}

	Il faut d'abord bien comprendre ce que signifie $d^2(f)$ parce que déjà $df$ est une application linéaire. En ce qui concerne les espaces, nous avons
	\begin{equation}
		\begin{aligned}[]
			f&\colon \eR^N\to \eR\\
			df&\colon \eR^N\to \aL(\eR^N,\eR)\\
			d(df)&\colon \eR^n\to \aL\big( \eR^N,\aL(\eR^N,\eR) \big).
		\end{aligned}
	\end{equation}
	En général, nous avons toujours, si $f\colon A\to B$, alors $df\colon A\to \aL(A,B)$. Nous avons juste appliqué ce principe à $f$ puis à $df$ lui-même.

	La proposition \ref{PropAnnulationEtConstance} dit que lorsque $df$ est nulle, $f$ est constante. En extrapolant un peu ce résultat, nous acceptons que si $d(df)$ est nulle, alors $df$ est constante en tant que application de $\eR^N$ dans $\aL(\eR^N,\eR)$. Il existe donc $T\in\aL(\eR^N,\eR)$ tel que
	\begin{equation}		\label{CD17dfxT}
		T=df(x)
	\end{equation}
	pour tout $x$. Si nous considérons la base canonique $\{ e_i \}$, nous pouvons expliciter \eqref{CD17dfxT} en l'appliquant le vecteurs $e_k$ :
	\begin{equation}
		T(e_k)=df(x).e_k=\sum_i\frac{ \partial f }{ \partial x_i }(x)\delta_{ik}=\frac{ \partial f }{ \partial x_k }(x)
	\end{equation}
	parce que $(e_k)_i=\delta_{ik}$. Nous notons $a_k=T(e_k)$; c'est un réel bien défini. Maintenant nous avons l'équation
	\begin{equation}
		\frac{ \partial f }{ \partial x_k }(x)=a_k
	\end{equation}
	pour tout $x\in\eR^N$. Si nous l'intégrons par rapport à $x_k$, nous trouvons
	\begin{equation}		\label{EqCD17fxxka}
		f(x)=x_ka_k+\varphi(x_1,\ldots,\hat x_k,\ldots,x_N)
	\end{equation}
	où $\varphi$ est la «constante» d'intégration. Elle dépend de tous les $x_i$ sauf de $x_k$ (c'est le sens du chapeau que nous avons mis sur $x_k$). L'équation \eqref{EqCD17fxxka} est valable pour chaque $k$. Nous avons donc en réalité $N$ équations. En dimension deux, nous aurions les équations
	\begin{subequations}
		\begin{align}
			f(x_1,x_2)&=x_1a_1+\varphi_1(x_2)\\
			f(x_1,x_2)&=x_2a_2+\varphi_2(x_1).
		\end{align}
	\end{subequations}
	La seule façon de choisir le fonction $\varphi_1$ et $\varphi_2$ de telle façon à avoir les deux égalités en même temps est de prendre $\varphi_1(x_2)=a_2x_2+C$ et $\varphi_2(x_1)=a_1x_1+C$. La même chose se passe en dimension plus grande : la seule façon de choisir les fonctions $\varphi_k$ de façon à satisfaire toutes les équations \eqref{EqCD17fxxka} en même temps est de prendre
	\begin{equation}
		\varphi_k(x_1,\ldots,\hat x_k,\cdots,x_N)=a_1x_1+\ldots+a_{k-1}x_{k-1}+a_{k+1}x_{k+1}+\ldots+a_Nx_N+C,
	\end{equation}
	et donc nous devons avoir,
	\begin{equation}
		f(x)=a_1x_1+\ldots+a_Nx_N+C,
	\end{equation}
	c'est à dire exactement $f(x)=a\cdot x+C$ où $C$ est une constante.

\end{corrige}
