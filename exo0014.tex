% This is part of Exercices et corrigés de CdI-1
% Copyright (c) 2011
%   Laurent Claessens
% See the file fdl-1.3.txt for copying conditions.

\begin{exercice}\label{exo0014}

Prenons deux suites $\{a_n\}$ et  $\{b_n\}$ qui tendent toutes les deux vers l'infini (resp. 0). On dira que la suite $\{a_n\}$ converge plus vite (resp. plus lentement) que la suite $\{b_n\}$ si $\dst{\lim_{n\rightarrow \infty}\f{a_n}{b_n} = \infty}$, aussi vite si $\dst{\lim_{n\rightarrow \infty}\f{a_n}{b_n} }$ existe et est finie, et plus lentement (resp. plus vite)  si $\dst{\lim_{n\rightarrow \infty}\f{a_n}{b_n} = 0}$.
\begin{enumerate}
	\item Montrer qu'il existe deux suites qui tendent vers $\infty$ (ou 0) mais qui n'ont pas la même  vitesse d'approche.
	\item Montrer que pour toute suite qui tend vers  $\infty$ (ou 0), il existe une suite qui tend vers  $\infty$ (ou 0) plus vite.
	\item Donner une suite non exponentielle qui tend vers l'infini plus vite que la suite $x_k = e^k$.
\end{enumerate}

\corrref{0014}
\end{exercice}
