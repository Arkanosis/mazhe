% This is part of Exercices et corrigés de CdI-1
% Copyright (c) 2011
%   Laurent Claessens
% See the file fdl-1.3.txt for copying conditions.

\begin{corrige}{Variete0001}

	\begin{enumerate}

		\item
			Il s'agit du cercle représenté sur la figure \ref{LabelFigCercleVarCarte}.
			\newcommand{\CaptionFigCercleVarCarte}{On ne peut pas trouver un atlas d'une seule carte pour le cerlce.}
			\input{Fig_CercleVarCarte.pstricks}
			Si nous prenons un point $(x,y)$ différent du pôle sud $S$, alors nous avons une carte autour du point donnée par
			\begin{equation}
				\begin{aligned}
					F_1\colon \mathopen] -\frac{ \pi }{2} , \frac{ 3\pi }{ 2 } \mathclose[&\to \eR^2 \\
					\theta&\mapsto \big( \cos(\theta),\sin(\theta) \big).
				\end{aligned}
			\end{equation}
			La différentielle de cette application est
			\begin{equation}
				dF_1(\theta)=\big( -\sin(\theta),\cos(\theta) \big),
			\end{equation}
			qui est toujours de rang maximum (c'est à dire $1$) parce qu'il n'arrive jamais que $dF_1(\theta)=(0,0)$.

			Autour du point $S$, nous pouvons prendre la carte $F_2\colon \mathopen] -\pi,0 ,  \mathclose[\to \eR^2$ définie de la même manière.
			
			Les cartes $F_1$ et $F_2$ forment un atlas pour le cercle (de nombreux autres choix sont évidement possibles). Notons toutefois qu'il n'est pas possible de trouver une seule carte qui paramétrise tout le cercle parce que le cercle est fermé alors que les cartes doivent partir d'ensembles ouverts.

		\item
		\item
		\item
			Il est apparent que cela ne va pas être une variété à cause des coins. Étudions donc ce qu'il s'y passe en regardant la figure \ref{LabelFigCoinPasVar}
			\newcommand{\CaptionFigCoinPasVar}{Ceci n'est pas une variété. Même pas le dessin d'une variété.}
			\input{Fig_CoinPasVar.pstricks}
			Supposons avoir une carte
			\begin{equation}
				t\mapsto F\big( x(t),y(t) \big).
			\end{equation}
			Il y a un seul $t_0$ tel que $F(t_0)=N$. Si $t_1$ est tel que $f(t_1)$ est à gauche de $N$, nous avons
			\begin{equation}
				\frac{ x(t_1) }{ y(t_1) }=-1,
			\end{equation}
			et si $t_2$ est tel que $F(t_2)$ est à droite de $N$, alors
			\begin{equation}
				\frac{ x(t_2) }{ y(t_2) }=1.
			\end{equation}
			Il n'y a donc que (au maximum) un seul point ($t_0$) où le rapport $x(t)/y(t)\neq\pm 1$. Il n'est donc pas possible d'avoir $x(t)$ et $y(t)$ continues parce que $y(t)$ ne s'annulant pas, si $x(t)$ et $y(t)$ étaient continues, alors le rapport serait continu.

		\item
		\item
			L'ensemble $x^2+y^2=1$ dans $\eR^3$ est un cylindre dont l'équation en coordonnées cylindrique est $r=1$. Pour avoir un atlas, il faut donc simplement prendre le même atlas que le cercle de rayon $1$, et de le \og multiplier\fg{} par $\eR$.

		\item
			L'ensemble donné par l'équation $x^2+y^2=z^2$ est un cône. Nous avons donc des doutes quant au fait que ce serait une variété, à cause du sommet.
			Nous allons montrer que ce n'est pas une variété parce que nous pouvons trouver trois vecteurs tangents linéairement indépendants en $(0,0,0)$, alors qu'une variété de dimension deux ne peut pas en avoir plus que deux (proposition \ref{PropDimEspTanVarConst}). Nous considérons les trois chemins
			\begin{equation}
				\begin{aligned}[]
					\gamma_1(t)&=(-t,0,t)\\
					\gamma_2(t)&=(t,0,t)\\
					\gamma_3(t)&=(0,t,t).
				\end{aligned}
			\end{equation}
			Il n'est pas très compliqué de prouver que ces trois chemins sont contenus dans l'ensemble. Les vecteurs tangents correspondants sont
			\begin{equation}
				\begin{aligned}[]
					\gamma'_1(0)&=(-1,0,1)\\
					\gamma'_2(0)&=(1,0,1)\\
					\gamma'_2(0)&=(0,1,1).
				\end{aligned}
			\end{equation}
			Ces trois vecteurs sont linéairement indépendants et génèrent tout $\eR^3$.

		\item
			Cet ensemble est la sphère. Pour créer un atlas, il ne faut pas oublier de mettre deux cartes par angles. Il est facile de trouver une carte, centrée en le pôle nord, qui ne contient ni le pôle nord ni le pôle sud en donnant un angle et une distance. Ensuite, il suffit de prendre une seconde carte centrée au pôle ouest et qui ne contient ni le pôle est ni le pôle ouest (mais bien les deux autres pôles).

		\item
			Nous allons utiliser la caractérisation de la proposition \ref{PropCarVarZerFonc} avec la fonction
			\begin{equation}
				G(x,y,z)=\big( (x-1)^2+y^2 \big)\big( (x+1)^2+y^2 \big)-z=0.
			\end{equation}
			Il est certain que $G=0$ est exactement l'ensemble considéré. Il faut encore vérifier que la fonction $G$ satisfait aux hypothèses de la proposition, c'est à dire que la différentielle $dG$ est de rang maximum. Ici, le rang maximum est $1$. Nous avons
			\begin{equation}
				dG(x,y,z)=
				\begin{pmatrix}
					4x(y^2+x^2-1)	\\ 
					4y(y^2+x^2+1)	\\ 
					-1	
				\end{pmatrix}.
			\end{equation}
			Grâce au $-1$, c'est toujours de rang $1$. Donc oui, l'ensemble considéré est une variété.

		\item
			Cette fois, le $-1$ n'apparaît pas, donc nous ne pouvons pas utiliser la même fonction pour prouver que $G=0$ est une variété.

			Autour de $x=0$, nous pouvons résoudre par rapport à $y$ :
			\begin{equation}
				y=\pm\sqrt{  \sqrt{4x^2+1}-x^2-1   }.
			\end{equation}

			Ces deux courbes sont représentées à la figure \ref{LabelFigExoVarj}.	
			\newcommand{\CaptionFigExoVarj}{Ce à quoi ça ressemble. En bleu la partie sans le signe, et en rouge avec le signe moins.}
			\input{Fig_ExoVarj.pstricks}

			En prenant un chemin bleu à gauche et rouge à droite, puis un autre chemin rouge à gauche et bleu à droite, on a deux chemins qui ont des vecteurs tangents linéairement indépendants. Hélas, dans $\eR^2$, seules des variétés de dimension $1$ sont possibles. Donc l'ensemble proposé n'est pas une variété.


	\end{enumerate}

\end{corrige}
