% This is part of Exercices et corrigés de CdI-1
% Copyright (c) 2011
%   Laurent Claessens
% See the file fdl-1.3.txt for copying conditions.

\begin{corrige}{OptimSS0001}

\begin{enumerate}

\item
La fonction $e^x\cos(x)$ n'a pas d'extrema globaux, mais la dérivée
\begin{equation}
	f'(x)=e^x\big( \cos(x)-\sin(x) \big)
\end{equation}
s'annule en $a_i=\frac{ \pi }{ 4 }+2k\pi$ et $b_i=\frac{ 5\pi }{ 4 }+2k\pi$. Étant donné que $f''(x)=-2e^x\sin(x)$, les premiers points $a_i$ sont des maxima locaux, et les points $b_i$ sont des minima locaux.

\item
La dérivée de $f$ est $f'(x)=\frac{ 1-\ln(x) }{ x^2 }$ et s'annule en $x=e$, qui et un point de maximum global.
\item
\item
\item
La dérivée de $f$ s'annule en $t=\pm 1$, où la dérivée seconde vaut respectivement $6$ et $-6$, ce qui fait que $1$ est minimum local, et $-1$ est maximum local. Ils ne sont cependant pas extrema globaux.

\item
Idem que le point précédent, mais $-3$ est minimum global et $3$ est maximum global.

\item
\item
Le dénominateur ne s'annule jamais, donc c'est une fonction qui s'étudie de façon usuelle pour $x>1$ et pour $x<-1$ séparément (à cause de la valeur absolue). Il faut cependant étudier le point $x=-1$ de façon séparée, parce que la fonction n'y est pas dérivable.

Nous avons $f(-1)=0$, alors que la fonction est toujours strictement positive ailleurs. Nous concluons que $x=-1$ est minimum global.
\end{enumerate}

\end{corrige}
