% This is part of Exercices et corrigés de CdI-1
% Copyright (c) 2011
%   Laurent Claessens
% See the file fdl-1.3.txt for copying conditions.

\begin{corrige}{IntMult0002}

\begin{enumerate}

\item
Nous utilisons le changement de variable
\begin{equation}
	\left\{
\begin{array}{ll}
x=(u+v)/2\\
y=(u-v)/2
\end{array}
\right.
\end{equation}
dont le jacobien est donné par
\begin{equation}
\begin{pmatrix}
	1/2	&	1/2	\\ 
	1/2	&	-1/2	
\end{pmatrix}=-\frac{ 1 }{2}.
\end{equation}
Nous devons donc calculer
\begin{equation}
	\int_Ef(x,y)=\int_E(y-x)=\int_{g^{-1}(E)}-v\cdot\frac{ 1 }{2}.
\end{equation}
Ne pas oublier que le jacobien doit être pris en valeur absolue. Le domaine d'intégration est donné par $-1<v<3$ et $1<u<5$. Donc l'intégrale à calculer est
\begin{equation}
	I=-\int_{-1}^3dv\int_1^5du\frac{ v }{ 2 }=-8.
\end{equation}

\item
En passant aux coordonnées polaires, nous devons simplement calculer l'intégrale
\begin{equation}
	\int_0^{2\pi}d\theta\int_0^R e^{aR}\cdot r\,dr=2\pi\left[ \frac{  e^{ar^2} }{ 2a } \right]_0^R=\frac{ \pi }{ a }( e^{aR^2}-1).
\end{equation}

\item
Nous passons en coordonnées cylindriques, dans lesquelles le domaine d'intégration s'écrit $r^2+z^2<1$, donc nous devons calculer
\begin{equation}
	\int_0^{2\pi}d\theta\int_0^1dr\int_{-\sqrt{1-r^2}}^{\sqrt{1-r^2}}z^2r\,dr=\frac{ 4\pi }{ 15 }.
\end{equation}
Il est également possible de faire cet exercice en coordonnées sphériques.

\item
La fonction $f(x,y,z)=\sqrt{x^2+y^2+z^2}$ et le domaine $x^2+y^2+z^2<1$ se prêtent parfaitement bien aux coordonnées sphériques. Il faut donc intégrer
\begin{equation}
	I=\int_0^{2\pi}d\theta\int_0^{\pi}d\varphi\int_0^1r\cdot r^2\sin\varphi\,dr=\pi.
\end{equation}

\item
Nous calculons en coordonnées cylindriques. Le domaine est $r^2<z^2$ et $0<z<1$, donc
\begin{equation}
	I=\int_0^1dz\int_0^{2\pi}d\theta\int_0^z zr\,dr=\frac{ \pi }{ 4 }.
\end{equation}


\end{enumerate}


\end{corrige}
