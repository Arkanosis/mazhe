% This is part of Exercices et corrigés de CdI-1
% Copyright (c) 2011
%   Laurent Claessens
% See the file fdl-1.3.txt for copying conditions.

\begin{corrige}{IntMult0003}

Nous intégrons en polaires. L'angle $\theta$ va de $0$ à $\pi/4$ ($x=y$), et pour chacun de ces angles, le rayon va de $0$ à $\cos(\theta)$. L'intégrale est donc
\begin{equation}
	I=\int_0^{\pi/4}d\theta\int_0^{\cos(\theta)}r\,dr=\frac{ \pi+2 }{ 16 }
\end{equation}
où nous avons utilisé la primitive $\int\cos(\theta)^2d\theta=\frac{ \sin(2\theta)+2\theta }{ 4 }$.

\end{corrige}
