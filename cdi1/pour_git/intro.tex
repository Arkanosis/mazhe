% This is part of Exercices et corrigés de CdI-1
% Copyright (c) 2011
%   Laurent Claessens
% See the file fdl-1.3.txt for copying conditions.

\chapter*{Introduction}
\addcontentsline{toc}{chapter}{Introduction}

%+++++++++++++++++++++++++++++++++++++++++++++++++++++++++++++++++++++++++++++++++++++++++++++++++++++++++++++++++++++++++++
\section*{Historique}
%+++++++++++++++++++++++++++++++++++++++++++++++++++++++++++++++++++++++++++++++++++++++++++++++++++++++++++++++++++++++++++

Ces notes sont parties des exercices résolus et tapés par Yvik Swan à Bruxelles il y a bien longtemps\footnote{Une partie des sources \LaTeX\ lui sont dues.}. Avec l'aide de Nicolas Richard, toujours à Bruxelles, j'en ai fait des exercices et corrigés presque complets pour le cours de calcul différentiel et intégrale de première année en mathématique et en physique.

Vu que le cours a été remanié durant l'année 2009-2010, il n'y a plus de raisons de tenir ces notes en l'état, d'où l'idée d'en faire des notes pour le cours «outils math» de l'université de Franche-Comté. Le texte dédié à «outils math» est largement inspiré des notes manuscrites de 

% TODO: ajouter le nom.

%+++++++++++++++++++++++++++++++++++++++++++++++++++++++++++++++++++++++++++++++++++++++++++++++++++++++++++++++++++++++++++
					\section*{Ces notes sont les vôtres !}
%+++++++++++++++++++++++++++++++++++++++++++++++++++++++++++++++++++++++++++++++++++++++++++++++++++++++++++++++++++++++++


Il y a encore certainement des erreurs, des fautes de frappe et des choses pas claires. Je compte sur vous (oui : toi !) pour me signaler toute imperfection (y compris d'orthographe).

Plus vous signalez de fautes, meilleure sera la qualité du texte, et plus les étudiants de l'année prochaine vous seront reconnaissants.

%+++++++++++++++++++++++++++++++++++++++++++++++++++++++++++++++++++++++++++++++++++++++++++++++++++++++++++++++++++++++++++
					\section*{Autres notes}
%+++++++++++++++++++++++++++++++++++++++++++++++++++++++++++++++++++++++++++++++++++++++++++++++++++++++++++++++++++++++++++

Quelque sites sur lesquelles il y a des choses bonnes à lire pour tout de suite ou pour plus tard.
\begin{enumerate}
	\item
		\url{http://student.ulb.ac.be/~lclaesse/} C'est mon site. Il contient d'autres cours et exercices de math et de physique.
	\item
		\url{http://student.ulb.ac.be/~lclaesse/physique-math.pdf} Un certain nombre de pré requis qui auraient pu ou dû être vus en secondaire sont disponibles.
	\item
		\url{http://student.ulb.ac.be/~lclaesse/geog.pdf} Si vous avez besoin d'exercices de drill sur les limites, dérivées et primitives.
	\item
		\url{http://cel.archives-ouvertes.fr/} Offre de nombreux cours sur différents sujets de mathématique, physique et autres sciences. Un site à retenir; vous pourriez en avoir besoin dans les années à venir.

\end{enumerate}
