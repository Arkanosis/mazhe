% This is part of Exercices et corrigés de CdI-1
% Copyright (c) 2011
%   Laurent Claessens
% See the file fdl-1.3.txt for copying conditions.

\begin{exercice}\label{exoEqsDiff0012}

On cherche à modéliser l'évolution dans le temps de la quantité d'une substance (par exemple la pénicilline) contenue dans le sang lors d'une injection en continu. La substance est injectée à raison de $I$ grammes par minute. Par ailleurs la substance est éliminée du sang à une vitesse proportionnelle à la quantité de substance présente au temps $t$.

\begin{enumerate}
\item Justifier que la quantité de substance $y$ présente dans le sang au temps $t$ satisfait l'équation différentielle \[y'=-ky+I,\] pour une certaine constante $k>0$.
\item Chercher une expression de $y$ en fonction de $t$, sachant qu'au temps $t=0$ (début de l'injection), on a $y=0$. Calculer la limite de $y$ quand $t$ tend vers l'infini. Interprétation.
\end{enumerate}

\end{exercice}
