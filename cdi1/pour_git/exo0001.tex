% This is part of Exercices et corrigés de CdI-1
% Copyright (c) 2011
%   Laurent Claessens
% See the file fdl-1.3.txt for copying conditions.

\begin{exercice}\label{exo0001}

Déterminez, s'ils existent, les supremum, infimum, maximum et minimum dans $\R$ des sous ensembles de $\R$ ci-dessous.
\begin{multicols}{2}
\begin{enumerate}
\item $]10,36] $
\item $[10,36[ $
\item $]5,+\infty[ $
\item $]8,9[ $
\item $\{ \frac{1}{2}, \frac{-2}{3}, \frac{3}{4}, \frac{-4}{5},  \frac{5}{6}, ... \} $
\item $\{ \frac{1}{2}, \frac{1}{3}, \frac{1}{4}, \frac{1}{5},  \frac{1}{6}, ... \} $
\item\label{itemexo1g} $\{ \frac{1}{k} \mid k \in \N_0 \} $
\item\label{itemexo1h} $\{ \frac{(-1)^k k}{k+1} \mid k \in \N_0 \} $
\item $\{ \sin(\frac{\pi k}{100}) \mid k \in \Z \} $
\item $\{ \sin(\frac{\pi k}{3}) \mid k \in \Z \} $
\item\label{itemexo1k}  $\{ x^2 \mid x\in ]-1,\frac{1}{2} [ \} $
\item $\Image([x \rightarrow e^x]) := \{ e^x \mid x\in \R \}$
\item $\Image(e^x cos(x)) := \{ e^x cos(x) \mid x\in \R \} $
\item $\Image(\varphi(x)) := \{ \varphi(x) \mid x \in [-1,1] \} $
\item $\Image( |\varphi(x)| ) := \{  |\varphi(x)| \mid x \in [-1,1] \} $
\end{enumerate}
\end{multicols}
où
\[
\varphi= \left\{ \begin{array}{ll} \sqrt{1-x^2} & x \leq 0 \\ x-2 & x>0 \\ \end{array} \right.
\]

\corrref{0001}
\end{exercice}
