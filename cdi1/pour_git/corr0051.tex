% This is part of Exercices et corrigés de CdI-1
% Copyright (c) 2011
%   Laurent Claessens
% See the file fdl-1.3.txt for copying conditions.

\begin{corrige}{0051}

\begin{enumerate}

\item 
Le dénominateur ne s'annule qu'au point $(0,2)$. Étudions ce qu'il se passe ailleurs. Les dérivées partielles par rapport à $x$ et $y$ ne sont pas compliquées à calculer (quoiqu'un peu long) :
\begin{equation}
	\frac{ \partial f_1 }{ \partial x }(0,0)=0,
\end{equation}
et
\begin{equation}
	\frac{ \partial f_1 }{ \partial y }(0,0)=-1,
\end{equation}
donc si la différentielle existe (et elle existe),
\begin{equation}
	d{f_1}_{(0,0)}=-dy.
\end{equation}

\item
$f_2(x,y)=(x+1)^{x+y}$.
Nous avons 
\begin{equation}
	\begin{aligned}[]
		\frac{ \partial f_2 }{ \partial x }(x,y)	&=(x+1)^{x+y}\left( \frac{ x+y }{ x+1 }+\ln(x+1) \right),\\
		\frac{ \partial f_2 }{ \partial y }(x,y)	&=(x+1)^{x+1}\ln(x+1),
	\end{aligned}
\end{equation}
donc
\begin{equation}
	d{f_2}_{(0,0)}=0.
\end{equation}

\end{enumerate}


\end{corrige}
