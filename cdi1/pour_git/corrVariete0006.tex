% This is part of Exercices et corrigés de CdI-1
% Copyright (c) 2011
%   Laurent Claessens
% See the file fdl-1.3.txt for copying conditions.

\begin{corrige}{Variete0006}

	\begin{enumerate}
		\item
			Il s'agit d'appliquer la formule \eqref{EqLongFonction}. Il faut donc intégrer
			\begin{equation}
				L=\int_0^{1/2}\sqrt{1+\left( \frac{ 2x }{ 1-x^2 } \right)^2}=\int_0^{1/2}\frac{ x^2+1 }{ x^1-1 }.
			\end{equation}
			La division euclidienne de $x^2+1$ par $x^2-1$ donne $1$ comme quotient et $2$ comme reste, c'est à dire que $x^2+1=(x^2-1)+2$. Cela suggère de découper la fraction comme
			\begin{equation}
				\frac{ x^2-1 }{ x^2-1 }+\frac{ 2 }{ x^2-1 }=1+\frac{ 2 }{ x^2-1 }.
			\end{equation}
			Ces deux fractions sont dans le formulaire. Après calcul, la réponse est $\ln(\frac{ 3 }{ 4 })-\frac{ 11 }{ 6 }$.

		\item
			\begin{equation}
				L=\int_0^5\sqrt{1+\frac{ 9 }{ 4 }x}=\frac{ 335 }{ 27 }.
			\end{equation}
			Notez que pour effectuer l'intégrale, le changement de variable $u=1+9x/4$ est conseillé.

		\item
			L'intégrale sur laquelle on tombe est $\int\frac{1}{ \cos(x) }dx$. Il faut savoir\footnote{oui oui, c'est une notation qui s'utilise de temps en temps.} que $1/sin(x)=\cosec(x)$. Cette intégrale est donc dans le formulaire.
			

	\end{enumerate}
	

\end{corrige}
