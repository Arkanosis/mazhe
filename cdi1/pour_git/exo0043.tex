% This is part of Exercices et corrigés de CdI-1
% Copyright (c) 2011
%   Laurent Claessens
% See the file fdl-1.3.txt for copying conditions.

\begin{exercice}\label{exo0043}

Supposons que $a_1, a_2, a_3$ et $a_4$ soient des fonctions $C^\infty$ de $\eR^2$ dans $\eR$. Pour les fonctions suivantes, représentez dans le plan leur domaine de définition(s) et écrivez les limites qu'il faudra étudier pour déterminer leur continuité sur $\eR^2$ :

\begin{multicols}{2}
\begin{enumerate}
	\item
	\begin{equation}
		f(x,y)=
		\begin{cases}
			a_1	&	\text{si $x>0$}\\
			a_2	&	 \text{si $x\leq 0$}
		\end{cases}
	\end{equation}


	\item
	\begin{equation}
		f(x,y)=
		\begin{cases}
			a_1	&	\text{si $x>0$ et $y>0$}\\
			a_2	&	 \text{sinon}
		\end{cases}
	\end{equation}
	\item
	\begin{equation}
		f(x,y)=
		\begin{cases}
			a_1	&	\text{si $x=y$}\\
			a_2	&	 \text{sinon}
		\end{cases}
	\end{equation}
	\item
	\begin{equation}
		f(x,y)=
		\begin{cases}
			a_1	&	\text{si $x>e^y$}\\
			a_2	&	 \text{sinon}
		\end{cases}
	\end{equation}
	\item
	\begin{equation}
		f(x,y)=
		\begin{cases}
			a_1	&	\text{si $x>0$, $y>0$}\\
			a_2	&	 \text{si $x\geq 0$, $y\leq 0$}\\
			a_3	&	 \text{si $x<0$, $y<0$}\\
			a_4	&	 \text{sinon}
		\end{cases}
	\end{equation}
	\item
	\begin{equation}
		f(x,y)=
		\begin{cases}
			a_1	&	\text{si $xy>0$}\\
			a_2	&	 \text{si $xy<0$}\\
			a_3	&	 \text{sinon}
		\end{cases}
	\end{equation}
\end{enumerate}
\end{multicols}
Note : ici et dans ces exercices, lorsque nous écrivons \og $a_1$\fg, nous sous-entendons bien entendu \og $a_1(x,y)$\fg. Le but de cet exercice est de permettre aux étudiants de cerner rapidement quelles sont les zones à problèmes.

\corrref{0043}
\end{exercice}
