% This is part of Exercices et corrigés de CdI-1
% Copyright (c) 2011
%   Laurent Claessens
% See the file fdl-1.3.txt for copying conditions.

\begin{exercice}\label{exo0059}

Considérons la fonction:\[f(x,y)=\left\{\begin{array}{cl}
                        \dfrac{\tan(\pi(15x^5+x^2+(y-2)^2))}{\pi(x^2+(y-2)^2)} & {\rm quand \ le \ d\acute{e}nominateur \ est \ non \ nul}  \\
                        1             						    & {\rm ailleurs}
                        \end{array}\right.\]	
 
 \begin{enumerate}
 \item  Cette fonction est-elle différentiable en $(0,0)$? Si oui, calculez  $\nabla f(0,0)$ et $df_{(0,0)}(2,3)$. 
 
 \item Écrivez l'équation du plan tangent au graphe de $f$ en $(0,0,f(0,0))$.  
 \item Soit $g:\Rn^2\lra\Rn^2:(u,v)\lra(\ln(\sin^2(u)),uv-\pi)$. Calculez $g(\f{\pi}{2}, 2)$. Calculez $d(f\circ g)_{(\f{\pi}{2},2)}$ à partir de $df$ et $dg$. 
 
 \item Étudiez la continuité et la différentiabilité de $f$. 
\end{enumerate}

\corrref{0059}
\end{exercice}
