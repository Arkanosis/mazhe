% This is part of Exercices et corrigés de CdI-1
% Copyright (c) 2011
%   Laurent Claessens
% See the file fdl-1.3.txt for copying conditions.

\begin{exercice}\label{exo0013}

Vrai ou faux (démontrez ou donnez un contre exemple):
\begin{enumerate}
	\item\label{ItemAExo0013} Si $u_n$ est telle que $u_{2n}\rightarrow  0$ et $u_{2n+1}\rightarrow  0$,  alors $u_n\rightarrow  0$.
	\item Si $u_n$ est telle que $u_{2n}\rightarrow  0$ et $u_{3n}\rightarrow  0$ et $u_{4n}\rightarrow  0$, et $\ldots$,  alors $u_n\rightarrow  0$.
	\item $\exists$ une suite qui prend une infinité de fois les valeurs 1,2.
	\item $\exists$ une suite qui prend une infinité de fois les valeurs 1,2,3.
	\item\label{ItemEnumiE13} $\exists$ une suite qui prend une infinité de fois toutes les valeurs entières.
	\item Toute suite périodique et convergente est constante.
	\item Si  $\| x_k\| \to \| x \|$, alors $x_k\to x$.
	\item Toute suite monotone qui a une sous-suite bornée converge.
	\item Il existe une suite telle que $\lim_{n\rightarrow \infty}u_n = \infty$ mais qui n'est jamais monotone.
\end{enumerate}
\corrref{0013}
\end{exercice}
