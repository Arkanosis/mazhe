% This is part of Exercices et corrigés de CdI-1
% Copyright (c) 2011
%   Laurent Claessens
% See the file fdl-1.3.txt for copying conditions.

\begin{exercice}\label{exo0017}

Déterminez le supremum (sauf pour l'exercice \ref{ItemGExo0017}), l'infimum (sauf pour l'exercice \ref{ItemGExo0017}), la limite, les limites supérieure et inférieure (s'ils existent) de chacune des suites ci-dessous.
\begin{enumerate}
	\item $(\frac{1}{2}, \frac{-2}{3}, \frac{3}{4}, \frac{-4}{5}, \frac{5}{6}, \ldots) $
	\item $(0,1,2,1,2,3,1,2,3,4,1,2,3,4,5, \ldots) $
	\item $1 + \frac{(-1)^k}{k} $
	\item $\frac{i^k}{k}$
	\item $\cos(\frac{\pi}{4} + k \frac{\pi}{2})$
	\item $(1 + \frac{25}{k^2})$
	\item\label{ItemGExo0017} $\frac{\sin(k)}{k}$
	\item $ \lfloor \frac{15}{7+k} \rfloor k^2 + (1 - \lfloor \frac{15}{ 7+k} \rfloor ) \frac{1}{k}$
\end{enumerate}
où $\lfloor r \rfloor$ désigne le plus grand entier inférieur à $r$. Ainsi $\lfloor 2,5 \rfloor$ = 2 et $\lfloor -2,5 \rfloor$ = -3.

\corrref{0017}
\end{exercice}
