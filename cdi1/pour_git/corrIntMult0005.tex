% This is part of Exercices et corrigés de CdI-1
% Copyright (c) 2011
%   Laurent Claessens
% See the file fdl-1.3.txt for copying conditions.

\begin{corrige}{IntMult0005}

\begin{enumerate}

\item
Le domaine
\begin{equation}
	\frac{ x^2 }{ a^2 }+\frac{ y^2 }{ b^2 }+\frac{ z^2 }{ c^2 }
\end{equation}
se prête bien au changement de variables
\begin{equation}
	\left\{
\begin{array}{ll}
x=au\\
y=bv\\
z=cs
\end{array}
\right.,
\end{equation}
dont le jacobien est
\begin{equation}
	J=\det\begin{pmatrix}
  a	&	0	&	0\\ 
  0	&	b	&	0\\ 
 0	&	0	& c	  
\end{pmatrix}=abc.
\end{equation}
Nous devons donc intégrer
\begin{equation}
	\int_{u^2+v^2+s^2<1}abc\,du\,dv\,ds=\frac{ 4\pi abc }{ 3 }.
\end{equation}

\item
Nous refaisons le même changement de variable :
\begin{equation}
	\left\{
\begin{array}{ll}
x=3u\\
y=3v\\
z=2s
\end{array}
\right.,
\end{equation}
dont le jacobien vaut $12$. Le domaine devient $u^2+v^2+s^2<1$, limité par $0<s<\frac{1}{ 2 }$. Le volume cherché est donc $12$ fois le volume du morceau de sphère contenu entre les hauteurs $0$ et $\frac{1}{ 2 }$.

Il faut donc intégrer
\begin{equation}
	\int_0^{1/2}S(z)dz
\end{equation}
où $S(z)=\pi r(z)^2$ est la surface de la section de la sphère à la hauteur $z$. Le rayon $r(z)$ est donné par Pythagore : $z^2+r(z)^2=1$. Nous avons donc
\begin{equation}
	V=12\int_0^{1/2}\pi(1-z^2)dz=\frac{ 11\pi }{ 2 }.
\end{equation}

\item
Intersection de deux cylindres. Ici, si nous passons en coordonnées cylindriques, nous pouvons simplifier l'expression d'un des deux cylindres, mais nous compliquerions l'expression de l'autre. Nous restons donc en cartésiennes. Les contraintes $x^2+y^2<1$ et $y^2+z^2<1$ font que le plus simple est de laisser $y$ varier de $-1$ à $1$ et de faire $z$ et $x$ s'adapter :
\begin{equation}
	V=\int_{-1}^1\Big( \int_{-\sqrt{1-y^2}}^{\sqrt{1-y^2}}\big( \int_{-\sqrt{1-y^2}}^{\sqrt{1-y^2}}dx\big)dz\Big)dy.
\end{equation}
L'avantage de cet ordre d'intégration est que les intégrales sur $x$ et sur $z$ donnent le même résultat : $2\sqrt{1-y^2}$. Il reste donc à calculer
\begin{equation}
	\int_{-1}^1 4(1-y^2)dy=\frac{ 16 }{ 3 }.
\end{equation}

\end{enumerate}

\end{corrige}
