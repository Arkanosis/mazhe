% This is part of Exercices et corrigés de CdI-1
% Copyright (c) 2011
%   Laurent Claessens
% See the file fdl-1.3.txt for copying conditions.

\begin{corrige}{continueSup0005}


\begin{enumerate}
\item Il suffit de prouver qu'elle est continue en $0$.
\item% $\dst{\lim_{x\rightarrow 0}e^x=1}$?\\
\underline{Affirmation} $\forall  x \in [0,1], \forall n$,  \[1-nx\leq(1-x)^n\leq 1 - \f{nx}{1+(n-1)x}\]

\underline{en effet}: 
\begin{enumerate}
\item $1-nx\leq(1-x)^n$, \ $\forall  x \in [0,1], \forall n$, par induction  directe sur $n$.
\item $(1-x)^n\leq 1 - \f{nx}{1+(n-1)x}$, \ $\forall  x \in [0,1], \forall n$.\\
Réécrivons la autrement: 
\[ \ba{c}  (1-x)^n \leq  \f{1-x}{1+(n-1)x} \\  \iff \\ (1+(n-1)x)(1-x)^n\leq 1-x\ea\]
Et, à nouveau par induction directe sur $n$.

\end{enumerate}

\item $\dst{\lim_{x\rightarrow 0}e^x=1}?$\\ Nous allons nous restreindre, sans perte de généralité (pourquoi?),  à regarder les $x\in [0,1]$. Il suffit pour cela de montrer que $\lim_{x\rightarrow 0}e^{-x}=1$ (pourquoi?).
Nous savons que, par définition,  \[e^{-x} = \lim_{n\rightarrow 	\infty}(1-\f{x}{n})^n.\] 
En utilisant l'affirmation au point (ii),  $\forall n, \ \forall x\in [0,1]$
\[ 1-x \  \leq \ (1-\f{x}{n})^n \ \leq \  1-\f{x}{1+(1-\f{1}{n})x}. \]
 et donc en passant à la limite,
\[1-x \leq e^{-x} \leq 1-\f{x}{1+x}\]
ce qui par la règle de l'étau, suffit à prouver le résultat. 
\item $e^x$ est dérivable une infinité de fois et chacune de ses dérivées est continue et égale à $e^x$.

Ceci se prouve  facilement en utilisant les mêmes inégalités qu'au point précédent.
\end{enumerate}

\end{corrige}
