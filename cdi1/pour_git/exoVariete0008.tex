% This is part of Exercices et corrigés de CdI-1
% Copyright (c) 2011
%   Laurent Claessens
% See the file fdl-1.3.txt for copying conditions.

\begin{exercice}\label{exoVariete0008}

Calculez $\int_\gamma 2xy dx+x^2dy$ le long des différents chemins suivants joignant l'origine $(0,0)$ au point $a=(2,1)$
\begin{enumerate}
\item la droite $y=\frac{x}{2}$
\item la parabole $y=\frac{x^2}{4}$
\item la parabole $y^2=\frac{x}{2}$
\item la ligne brisée $Oba$ où $b=(0,1)$
\end{enumerate}
Donnez une interprétation de ces résultats. Pour ce faire, sachez qu'en physique, il y a un adage qui dit que \emph{le travail d'une force conservative ne dépend pas du chemin suivit}. En mathématique, il y a un théorème qui dit la même chose. Citer ce théorème et prouver que la forme $\omega=2xydx+x^2dy$ vérifie les hypothèses de ce théorème.

\corrref{Variete0008}
\end{exercice}
