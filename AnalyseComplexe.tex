% This is part of Mes notes de mathématique
% Copyright (c) 2012
%   Laurent Claessens
% See the file fdl-1.3.txt for copying conditions.

%+++++++++++++++++++++++++++++++++++++++++++++++++++++++++++++++++++++++++++++++++++++++++++++++++++++++++++++++++++++++++++
\section{Fonctions holomorphes}
%+++++++++++++++++++++++++++++++++++++++++++++++++++++++++++++++++++++++++++++++++++++++++++++++++++++++++++++++++++++++++++

Dans cette partie, nous désignons par \( \Omega\) un ouvert de \( \eC\). Une fonction \( f\colon \Omega\to \eC\) est $C$-dérivable si la limite
\begin{equation}
    \lim_{h\to 0} \frac{ f(a+h)-f(a) }{ h }
\end{equation}
existe. Dans ce cas, cette limite est la dérivée de \( f\).

\begin{definition}
    Soit \( \Omega\) un ouvert dans \( \eC\). Une fonction \( \Omega\colon \Omega\to \eC\) est \defe{holomorphe}{holomorphe}\index{fonction!holomorphe} si elle est \( C^1\) et \( \eC\)-dérivable sur \( \Omega\). 
\end{definition}

\begin{proposition}
    Une fonction \( f\colon\Omega \to \eC\) est $C$-dérivable en \( a\in\Omega\) si et seulement si elle est différentiable en \( a\) et la différentielle \( df_a\) est une similitude.
\end{proposition}

\begin{proposition}
    Une fonction \( f\colon \Omega\to \eC\) est $C$-dérivable en \( a\in\Omega\) si et seulement si elle est différentiable en \( a\) et si \( df_a\) est une similitude.
\end{proposition}

\begin{theorem}
    Si \( f\in C^1(\Omega)\) alors nous avons équivalence des faits suivants :
    \begin{enumerate}
        \item
            \( f\) est holomorphe sur \( \Omega\),
        \item
            \( f\) vérifie \( \partial_{\bar z}f=0\).
    \end{enumerate}
\end{theorem}

\begin{definition}
    Une fonction \( f\colon \Omega\to \eC\) est \( \eC\)-analytique sur \( \Omega\) si pour tout \( z_0\in \Omega\) il existe une suite \( (c_n)\) dans \( \eC\) et \( r>0\) tels que
    \begin{equation}
        f(z)=\sum_{n=0}^{\infty}c_n(z-z_0)^n
    \end{equation}
    pour tout \( z\in B(z_0,r)\).
\end{definition}

\begin{proposition}
    Une application \( f\colon \Omega\to \eC\) est $C$-dérivable sur \( \Omega\) si et seulement si elle est différentiable et
    \begin{subequations}        \label{EqmblExI}
        \begin{numcases}{}
            \frac{ \partial u }{ \partial x }=\frac{ \partial v }{ \partial y }\\
            \frac{ \partial u }{ \partial y }=-\frac{ \partial v }{ \partial x }
        \end{numcases}
    \end{subequations}
    où \( f(x+iy)=u(x,y)+iv(x,y)\).
\end{proposition}
Les équations \eqref{EqmblExI} sont les équations de \defe{Cauchy-Riemann}{Cauchy-Riemann}.

\begin{proof}
    La différentielle de \( f\colon \eR^2\to \eR^2\) est donnée par la matrice
    \begin{equation}        \label{EQwtagsz}
        T=\begin{pmatrix}
            \partial_xu(a)    &   \partial_yu(a)    \\ 
            \partial_xv(a)    &   \partial_yv(a)    
        \end{pmatrix}.
    \end{equation}
    Cette matrice est une similitude si et seulement si les équations de Cauchy-Riemann sont satisfaites. En effet si \( 1=\begin{pmatrix}
        1    \\ 
        0    
    \end{pmatrix}\) et \( i=\begin{pmatrix}
        0    \\ 
        1    
    \end{pmatrix}\), la matrice \( T\) est une similitude (écrivons \( \alpha+i\beta\) son coefficient) si
    \begin{subequations}
        \begin{numcases}{}
            T(1)=\alpha+i\beta\\
            T(i)=-\beta+i\alpha,
        \end{numcases}
    \end{subequations}
    c'est à dire
    \begin{equation}
        T=\begin{pmatrix}
            \alpha    &   -\beta    \\ 
           \beta    &   \alpha    
        \end{pmatrix}.
    \end{equation}
    Identifier cette matrice à \eqref{EQwtagsz} fournit le résultat annoncé.
\end{proof}

\begin{proposition}
    Une fonction \( f\colon \Omega\to \eC\) est $C$-dérivable si et seulement si elle est différentiable et \( df_a\) est une similitude.
\end{proposition}

%+++++++++++++++++++++++++++++++++++++++++++++++++++++++++++++++++++++++++++++++++++++++++++++++++++++++++++++++++++++++++++
\section{Séries entières}
%+++++++++++++++++++++++++++++++++++++++++++++++++++++++++++++++++++++++++++++++++++++++++++++++++++++++++++++++++++++++++++
\label{SecoLNvnO}

\begin{proposition}     \label{PropRZCKeO}
    Si \( f(z)=\sum_na_nz^n\) a pour rayon de convergence \( R\), alors \( f\) est $C$-dérivable et nous pouvons dériver terme à terme dans la boule ouverte \( B(0,R)\).
\end{proposition}

\begin{proof}
    Cela est exactement la proposition \ref{ProptzOIuG}.
\end{proof}

\begin{definition}
    Une fonction \( f\colon \Omega\to \eC\) est \( \eC\)-analytique si pour tout \( z_0\in\Omega\), il existe une suite \( c_n\) et \( r>0\) tels que
    \begin{equation}
        f(z)=\sum_n c_n(z-z_0)^n
    \end{equation}
    pour tout \( z\in B(z_0,r)\).
\end{definition}

\begin{proposition}
    Une fonction analytique est holomorphe.
\end{proposition}

\begin{definition}
    Une \defe{mesure de Radon}{mesure!de Radon} sur un compact \(  K\) de \( \eC\) est une forme linéaire continue sur \( C(K)\). Si \( \mu\) est une mesure de Radon, on définit la \defe{transformée de Cauchy}{transformée!de Cauchy} de \( \mu\) par 
    \begin{equation}
        \begin{aligned}
            \hat \mu\colon \eC\setminus K&\to \eC \\
            z&\mapsto -\frac{1}{ \pi }\mu\left( \frac{1}{ \xi-z } \right). 
        \end{aligned}
    \end{equation}
\end{definition}

\begin{theorem}     \label{ThoJVNTzn}
    Si \( \mu\) est une mesure de Radon sur \( K\) alors \( \hat \mu\) est infiniment \( \eC\)-dérivable sur \( \Omega=\eC\setminus K\) et nous avons
    \begin{equation}
        \hat\mu^{(n)}(z)=-\frac{ n! }{ \pi }\mu\left( \frac{1}{ (\xi-z)^{n+1} } \right).
    \end{equation}
\end{theorem}

\begin{lemma}
    Si \( f\) est holomorphe sur \( \Omega\) et si \( B\) est une boule fermée dans \( \Omega\) alors pour tout \( z\in \Int(B)\) nous avons
    \begin{equation}
        f^{(n)}(z)=\frac{ n! }{ 2i\pi }\int_{\partial B}\frac{ f(\xi) }{ (\xi-z)^{n+1} }d\xi.
    \end{equation}
\end{lemma}

\begin{proof}
    Appliquer le théorème \ref{ThoJVNTzn} à la mesure de Radon
    \begin{equation}
        \mu(\phi)=\int_{\partial B}\phi(\xi)d\xi.
    \end{equation}
\end{proof}

\begin{lemma}
    Si \( f\) est holomorphe sur \( \Omega\) et si \( B\) est une boule fermée dans \( \Omega\) alors pour tout \( z\) dans l'intérieur de \( B\) nous avons
    \begin{equation}
        f^{(n)}(z)=\frac{ n! }{ 2i\pi }\int_{\partial B}\frac{ f(\xi) }{ (\xi-z)^{n+1} }d\xi.
    \end{equation}
    
\end{lemma}

\begin{proof}
    
\end{proof}
<++>

\begin{theorem}
    Si \( f\) est une fonction holomorphe sur le disque ouvert \( B(z_0,R)\) alors
    \begin{equation}
        f(z)=\sum_{n=0}^{\infty}\frac{ f^{(n)}(z_0) }{ n! }(z-z_0)^n
    \end{equation}
    et cette série converge uniformément sur tout compact.
\end{theorem}

\begin{proof}
    Sans perte de généralité nous supposons que \( z_0=0\). La formule de Cauchy fournit
    \begin{equation}
        f(z)=\frac{1}{ 2\pi i }\int_{\partial B}\frac{ f(\xi) }{ \xi-z }d\xi=\frac{1}{ 2\pi i }\int_{\partial B}\frac{ f(\xi) }{ 1-(z/\xi) }\frac{ d\xi }{ \xi }.
    \end{equation}
    Nous utilisons la série géométrique
    \begin{equation}
        \frac{1}{ 1-(z/\xi) }=\sum_{n=0}^{\infty}\left( \frac{ z }{ \xi } \right)^n,
    \end{equation}
    nous avons
    \begin{subequations}        \label{EqXSgZGw}
        \begin{align}
            f(z)&=\frac{1}{ 2\pi i }\sum_{n=0}^{\infty}\int_{\partial B}\frac{ z^nf(\xi) }{ \xi^{n+1} }\\
            &=\sum_{n=0}^{\infty}\left( \frac{1}{ 2\pi i }\int_{\partial B}\frac{ f(\xi) }{ \xi^{n+1} } \right)z^n.
        \end{align}
    \end{subequations}
    Nous devons maintenant montrer que ce qui se trouve dans la grande parenthèse vaut \( f^{(n)}(0)/n!\). Nous utilisons le théorème de Radon \ref{ThoJVNTzn} à la mesure
    \begin{equation}
        \mu(\phi)=\int_{\partial B}\phi(\xi)d\xi.
    \end{equation}
    La transformée de Cauchy est
    \begin{equation}        \label{EqTzkmeL}
        \hat \mu(z)=-\frac{1}{ \pi }\mu\left( \frac{1}{ \xi-z } \right)=-\frac{1}{ \pi }\int_{\partial B}\frac{1}{ \xi-z }d\xi,
    \end{equation}
    et le théorème assure que
    \begin{equation}
        \hat\mu^{(n)}(z)=-\frac{ n! }{ \pi }\mu\left( \frac{1}{ (\xi-z)^{n+1} } \right)=-\frac{ n! }{ \pi }\int_{\partial B}\frac{ 1 }{ (\xi-z)^{n+1} }d\xi.
    \end{equation}
    En comparant la formule \eqref{EqTzkmeL} avec la formule de Cauchy nous voyons que \( \hat\mu(z)=-2i f(z)\). Par conséquent
    \begin{equation}
        f^{(n)}(z)=-\frac{1}{ 2i }\hat\mu^{(n)}(z)=\frac{ n! }{ 2\pi i }\int_{\partial B}\frac{1}{ (\xi-z)^{n+1} }d\xi,
    \end{equation}
    et
    \begin{equation}
        f^{(n)}(0)=\frac{ n! }{ 2\pi i }\int_{\partial B}\frac{1}{ \xi^{n+1} }d\xi.
    \end{equation}
\end{proof}
% TODO : justifier la permutation entre la somme et l'intégrale.

%---------------------------------------------------------------------------------------------------------------------------
\subsection{Fonction exponentielle}
%---------------------------------------------------------------------------------------------------------------------------

Soit \( z=x+iy\in \eC\). Nous définissons l'\defe{exponentielle}{exponentielle!complexe} de \( z\) par
\begin{equation}
    e^{z}=e^x(\cos y+i\sin y).
\end{equation}

\begin{theorem}
    La fonction exponentielle vérifie les propriétés suivantes.
    \begin{enumerate}
        \item
            \( \exp\) est holomorphe.
        \item
            \( (e^z)'=e^z\).
        \item
            L'exponentielle est développable en série entière,
            \begin{equation}
                e^z=\sum_{n=0}^{\infty}\frac{ z^n }{ n! }
            \end{equation}
            et la série converge normalement sur tout compact de \( \eC\).
    \end{enumerate}
\end{theorem}

\begin{proof}
    En tant que application \( E\colon \eR^2\to \eC\), la fonction
    \begin{equation}
        E(x,y)=e^x(\cos y+i\sin y)
    \end{equation}
    est \( C^{\infty}\). De plus nous avons
    \begin{subequations}
        \begin{align}
            \frac{ \partial E }{ \partial x }(x,y)= e^{x+iy}=E(x,y)\\
            \frac{ \partial E }{ \partial y }(x,y)=iE(x,y),
        \end{align}
    \end{subequations}
    et par conséquent la fonction \( E\) vérifie les équations de Cauchy-Riemann.


    Si \( r\) est fixé, par le critère d'Abel appliqué à la suite \(r/n!\) nous savons que la série \( \sum z^n/n!\) converge normalement sur le compact \( B(0,r)\).
\end{proof}



