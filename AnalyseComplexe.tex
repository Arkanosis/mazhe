% This is part of Mes notes de mathématique
% Copyright (c) 2012
%   Laurent Claessens
% See the file fdl-1.3.txt for copying conditions.

%+++++++++++++++++++++++++++++++++++++++++++++++++++++++++++++++++++++++++++++++++++++++++++++++++++++++++++++++++++++++++++
\section{Fonctions holomorphes}
%+++++++++++++++++++++++++++++++++++++++++++++++++++++++++++++++++++++++++++++++++++++++++++++++++++++++++++++++++++++++++++

\begin{definition}
    Soit \( \Omega\) un ouvert dans \( \eC\). Une fonction \( \Omega\colon \Omega\to \eC\) est \defe{holomorphe}{holomorphe}\index{fonction!holomorphe} si elle est \( C^1\) et \( \eC\)-dérivable sur \( \Omega\). 
\end{definition}

\begin{proposition}
    Une fonction \( f\colon\Omega \to \eC\) est $C$-dérivable en \( a\in\Omega\) si et seulement si elle est différentiable en \( a\) et la différentielle \( df_a\) est une similitude.
\end{proposition}

\begin{theorem}
    Si \( f\in C^1(\Omega)\) alors nous avons équivalence des faits suivants :
    \begin{enumerate}
        \item
            \( f\) est holomorphe sur \( \Omega\),
        \item
            \( f\) vérifie \( \partial_{\bar z}f=0\).
    \end{enumerate}
\end{theorem}

\begin{definition}
    Une fonction \( f\colon \Omega\to \eC\) est \( \eC\)-analytique sur \( \Omega\) si pour tout \( z_0\in \Omega\) il existe une suite \( (c_n)\) dans \( \eC\) et \( r>0\) tels que
    \begin{equation}
        f(z)=\sum_{n=0}^{\infty}c_n(z-z_0)^n
    \end{equation}
    pour tout \( z\in B(z_0,r)\).
\end{definition}

%+++++++++++++++++++++++++++++++++++++++++++++++++++++++++++++++++++++++++++++++++++++++++++++++++++++++++++++++++++++++++++
\section{Séries entières}
%+++++++++++++++++++++++++++++++++++++++++++++++++++++++++++++++++++++++++++++++++++++++++++++++++++++++++++++++++++++++++++

Source : \cite{RomainBoilEnt}.

Nous rappelons qu'une série de nombres \( \sum_{n=0}^{\infty}a_n\) converge \defe{absolument}{convergence!absolue} si la série
\begin{equation}
    \sum_{n=0}^{\infty}| a_n |
\end{equation}
converge. Cette définition s'étend immédiatement aux séries dans n'importe quel espace normé.

La convergence est \defe{normale}{convergence!normale} si la suite de fonctions \( f_N(z)=\sum_{n=0}^N a_nz^n\) converge uniformément (c'est à dire pour la norme supremum).

\begin{definition}
    Une \defe{série entière}{série!entière} est une somme de la forme
    \begin{equation}
        \sum_{n=0}^{\infty}a_nz^n
    \end{equation}
    avec \( a_n,z\in\eC\).    
\end{definition}
Une série entière peut définir une fonction
\begin{equation}
    f(z)=\sum_na_nz^n.
\end{equation}
Le but de cette section est d'étudier des conditions sur la suite \( (a_n)\) qui assurent la continuité de \( f\) ou la possibilité de dériver ou intégrer la série terme à terme.

%---------------------------------------------------------------------------------------------------------------------------
					\subsection{Série de puissances}
%---------------------------------------------------------------------------------------------------------------------------

Une \defe{série de puissance}{Série!de puissance} est une série de la forme
\begin{equation}		\label{eqseriepuissance}
	\sum_{k=0}^{\infty}c_k(z-z_0)^k
\end{equation}
où $z_0\in \eC$ est fixé, $(c_k)$ est une suite complexe fixée, et $z$ est un paramètre complexe. Nous disons que cette série est \emph{centrée} en $z_0$.

\begin{theorem}		\label{ThoSerPuissRap}
Considérons la série de puissances donnée par le terme général $c_k(z-z_0)^k$. Si nous posons
\begin{equation}		\label{EqRayCOnvSer}
	\alpha=\frac{1}{ R } =\limsup\sqrt[k]{| c_k |}
\end{equation}
alors la série converge absolument si $| z-z_0 |<R$ et diverge si $| z-z_0 |>R$.

De plus, ce rayon de convergence peut être calculé par la formule alternative 
\begin{equation}		\label{EqAlphaSerPuissAtern}
	\alpha=\frac{1}{ R }=\limite k \infty \abs{\frac{c_{k+1}}{c_k}}
\end{equation}
lorsque $c_k$ est non nul à partir d'un certain $k$.
\end{theorem}
Le disque $| z-z_0 |\leq R$ est le \defe{disque de convergence}{Disque de convergence} de la série \eqref{eqseriepuissance}. Notez que ce théorème ne dit rien pour les points tels que $| z-z_0 |=R$. Il faut traiter ces points au cas par cas. Et le pire, c'est qu'une série donnée peut converger pour certain des points sur le bord du disque, et diverger en d'autres.  Il y a un dessin à la figure \ref{LabelFigDisqueConv}.
\newcommand{\CaptionFigDisqueConv}{À l'intérieur du disque de convergence, la convergence est absolue. En dehors, la série diverge. Sur le cercle proprement dit, tout peut arriver.}
\input{Fig_DisqueConv.pstricks}

Le disque de centre $z_0$ et de rayon $R$ est appelé \Defn{disque de convergence}. Pour un complexe $z$ sur le bord de ce disque, c'est-à-dire tel que $\abs{z-z_0} = R$, le comportement peut-être très varié (convergence absolue, convergence simple ou divergence) et n'est éventuellement pas le même sur tout le bord.

L'étude de ce qu'il se passe sur le bord du disque de convergence commence par y étudier la convergence absolue, c'est à dire étudier la série
\begin{equation}
	\sum_k| c_k(z-z_0)^k |=\sum_k| c_k |R^k
\end{equation}
parce que sur le bord, $| z-z_0 |=R$. L'étude du terme général $| c_k |R^k$ a deux utilités :
\begin{enumerate}
\item Si la somme $\sum_{k}| c_k |R^k$ converge, alors la série converge uniformément sur le bord,
\item si la suite $| c_k |R^k$ ne tend pas vers zéro, alors la série ne converge même pas simplement sur le bord.
\end{enumerate}

%---------------------------------------------------------------------------------------------------------------------------
\subsection{Convergence normale}
%---------------------------------------------------------------------------------------------------------------------------

Une série de fonctions \( \sum_{n\in \eN}u_n \) converge \defe{normalement}{convergence!normale} si la série de nombre \( \sum_n\| u_n \|_{\infty}\) converge.

\begin{lemma}
    Soient des fonctions \( u_n\colon \Omega\to \eC\). Si il existe une suite réelle positive \( (a_n)_{n\in \eN}\) telle que
    \begin{enumerate}
        \item
            pour tout \( z\in \Omega\) et pour tout \( n\in \eN\) nous avons \( | u_n(z) |\leq a_n\) (c'est à dire \( a_n\geq \| u_n \|_{\infty}\)),
        \item
            la somme \( \sum_{n}a_n\) converge,
    \end{enumerate}
    alors la série de fonctions \( \sum_{n=0}^{\infty}u_n\) converge normalement.
\end{lemma}

\begin{proof}
    Découle du lemme de comparaison.
\end{proof}

\begin{proposition}
    Soit \( (u_n)\) une suite de fonctions continues \( u_n\colon \Omega\subset\eC\to \eC\). Si la série \( \sum_nu_n\) converge normalement alors la somme est continue.
\end{proposition}

\begin{proof}
    Nous posons \( u(z)=\lim_{n\to \infty} u_n(z)\), et nous vérifions que la fonction ainsi définie sur \( \Omega\) est continue. Nous avons
    \begin{subequations}
        \begin{align}
            \big| u(z)-u(z') \big|&=\left| \sum_{n=0}^{N}u_n(z)-\sum_{n=0}^{N}u_n(z')+\sum_{n=N+1}^{\infty}u_n(z)-\sum_{n=N+1}^{\infty}u_n(z') \right| \\
            &\leq \left| \sum_{n=0}^N u_n(z)-\sum_{n=0}^Nu_n(z') \right| +\sum_{n=N+1}^{\infty}| u_n(z) |+\sum_{n=N+1}^{\infty}| u_n(z') |.
        \end{align}
    \end{subequations}
    <++>
\end{proof}
<++>
