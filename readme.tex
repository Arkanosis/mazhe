\documentclass[a4paper,12pt]{article}

\usepackage[utf8]{inputenc}
\usepackage[T1]{fontenc}

\usepackage{color}
\usepackage{listingsutf8}


\definecolor{dkgreen}{rgb}{0,0.4,0}
\definecolor{gray}{rgb}{0.5,0.5,0.5}
\definecolor{mauve}{rgb}{0.58,0,0.82}

% Thi 'lstset' is from Lilian Besson 
\lstset{ %
  inputencoding=utf8/latin1,
  backgroundcolor=\color{white},  % choose the background color; you must add \usepackage{color} or \usepackage{xcolor}
  basicstyle=\ttfamily, % \texttt\small,              % the size of the fonts that are used for the code, FIXME \ttfamily
  breakatwhitespace=false,        % sets if automatic breaks should only happen at whitespace
  breaklines=true,                % sets automatic line breaking
  captionpos=b,                   % sets the caption-position to bottom
  commentstyle=\small\color{dkgreen},   % comment style
%  deletekeywords={...},          % if you want to delete keywords from the given language
%  escapeinside={\%*}{*)},        % if you want to add LaTeX within your code
  frame=single,                   % adds a frame around the code
  keywordstyle=\small\color{blue},      % keyword style
  language=python,                % the language of the code
  fontadjust=false,
  % if you want to add more keywords to the set
%  morekeywords={define,domain,objects,init,goal,problem,action,parameters,precondition,effect,types,requirements,strips,typing},
  numbers=left,                   % where to put the line-numbers; possible values are (none, left, right)
  numbersep=5pt,                  % how far the line-numbers are from the code
  numberstyle=\tiny\color{gray},  % the style that is used for the line-numbers
  rulecolor=\color{black},        % if not set, the frame-color may be changed on line-breaks within not-black text (e.g. comments (green here))
  showspaces=false,               % show spaces everywhere adding particular underscores; it overrides 'showstringspaces'
  showstringspaces=false,         % underline spaces within strings only
  showtabs=false,                 % show tabs within strings adding particular underscores
  stepnumber=1,                   % the step between two line-numbers. If it's 1, each line will be numbered
  stringstyle=\small\color{mauve},      % string literal style
  tabsize=2,                      % sets default tabsize to 2 spaces
  prebreak = \raisebox{0ex}[0ex][0ex]{\ensuremath{\hookleftarrow}}, % pour la fin des lignes.
  aboveskip={1.5\baselineskip},
  %title=\lstname                  % show the filename of files included with \lstinputlisting; also try caption instead of title
%  title=\tiny{File \textcolor{blue}{\url{\lstname}}}          % show the filename of files included with \lstinputlisting; also try caption instead of title
  %% FIXME title !
}

\usepackage{textcomp}
\usepackage{lmodern}
\usepackage[a4paper,margin=2cm]{geometry} 
\usepackage{hyperref}                      
\usepackage[english,frenchb]{babel}

\newcommand{\info}[1]{\texttt{#1}}

\begin{document}

\selectlanguage{english}
\title{How to compile and work with \info{mazhe.tex} ?}
\author{Laurent Claessens}
\maketitle

\tableofcontents

\selectlanguage{frenchb}

Tout commence par télécharger les sources à l'adresse
\begin{center}
    \url{https://github.com/LaurentClaessens/mazhe}
\end{center}

Pour information le numéro du commit qui correspond à ce qui a été soumis au jury pour approbation en septembre 2015 est \info{c038d9eee475b3a1d6e5414a698d50007dc69af0}.

%+++++++++++++++++++++++++++++++++++++++++++++++++++++++++++++++++++++++++++++++++++++++++++++++++++++++++++++++++++++++++++ 
\section{Pour l'agrégation}
%+++++++++++++++++++++++++++++++++++++++++++++++++++++++++++++++++++++++++++++++++++++++++++++++++++++++++++++++++++++++++++

Le plus facile est de télécharger la paquet \info{exocorr.sty} (voir plus bas) puis de compiler avec :
\begin{quote}
    \info{pdflatex -synctex=1 -shell-escape Inter\_frido-mazhe\_pytex.tex  }
\end{quote}
et quelques fois \info{bibtex} et \info{makeindex}.

Ce faisant vous n'aurez cependant pas les moyens de changer l'ordre des chapitres\footnote{Ou en tout cas pas de me soumettre vos changements parce que c'est dans \info{mazhe.tex} que je travaille.} parce que ce fichier est créé automatiquement par quelques scripts de précompilation. En travaillant de la sorte vous aurez la possibilité d'effectuer des modifications dans les fichiers inclus.

Vu que de mon côté ce fichier est recréé automatiquement à chaque compilation, un \info{git pull} écrasera les modifications que vous y auriez apporté. Si vous voulez savoir la raison de ce fait, faites
\begin{quote}
    \info{pdflatex mazhe.tex}
\end{quote}
Voyant le résultat vous comprendrez pourquoi compiler le document destiné à l'agrégation demande un peu de travail.

%--------------------------------------------------------------------------------------------------------------------------- 
\subsection{Ce qu'il faut télécharger}
%---------------------------------------------------------------------------------------------------------------------------

Pour avoir une maîtrise plus fine de la compilation, vous devrez télécharger un certain nombre de choses.
\begin{description}
    \item[Le paquet \info{exocorr}]
        Vous devez récupérer le paquet \info{exocorr} à l'adresse
        \begin{quote}
            \url{https://github.com/LaurentClaessens/exocorr}
        \end{quote}
        Seul le fichier \info{.sty} vous intéresse a priori. Mettez-le là où vous mettez vos paquets \LaTeX.
    \item[Le module \info{LaTeXparser}]
        Vous le téléchargez à l'adresse
        \begin{quote}
            \url{https://github.com/LaurentClaessens/LaTeXparser}
        \end{quote}
        et vous le mettez quelque part là où Python pourra le retrouver.
    \item[Le script \info{pytex}]
        Il est à l'adresse
        \begin{quote}
            \url{https://github.com/LaurentClaessens/pytex}
        \end{quote}
\end{description}

Note. Il est conseillé de mettre tous ces fichiers dans des répertoires séparés, obéissant à une certaine logique : les paquets \LaTeX\ avec les autres paquets \LaTeX, les modules python avec les autres modules python. Cela surtout si vous comptez compiler souvent. Si votre but est seulement de compiler pour tester, vous pouvez tout aussi bien mettre \info{pytex.py} et \info{exocorr.sty} dans le répertoire courant.



%--------------------------------------------------------------------------------------------------------------------------- 
\subsection{Compiler tout le document}
%---------------------------------------------------------------------------------------------------------------------------

Lorsque tout est téléchargé et correctement configuré (\info{LaTeXparser} doit être trouvable par python et \info{pytex} trouvable par votre terminal), vous compilez le Frido avec
\begin{quote}
    \begin{verbatim}
    pytex lst_frido.py --lotex
    \end{verbatim}
\end{quote}
Le script s'occupe d'extraire de \info{mazhe.tex} les choses nécessaires au Frido, crée un fichier intermédiaire et le compile. Des passes de \info{bibtex} et \info{makeindex} sont également automatiquement effectuées.

Les \info{ref} et \info{eqref} ne correspondant à aucun \info{label} sont indiqués. Il ne devrait y en avoir aucun.

La compilation produit deux fichiers \info{pdf}. Le premier est \info{Inter\_frido-mazhe\_pytex.pdf} qui est créé par \LaTeX\ lui-même durant la compilation. Le second est \info{0-lefrido.pdf} qui en est une simple copie effectuée après la compilation. Vous devriez ouvrir \info{0-lefrido.pdf} de façon à éviter que votre lecteur de \info{pdf} ne se mette en mode «rafraichissement» durant toute la durée de la compilation.

%--------------------------------------------------------------------------------------------------------------------------- 
\subsection{Compiler seulement une partie}
%---------------------------------------------------------------------------------------------------------------------------

Lisez le fichier \info{lst\_exemple.py} :

\lstinputlisting{lst_exemple.py}

A priori la seule chose qui vous intéresse est la liste des \texttt{ok\_filename}. Je crois qu'elle est assez auto-explicative. Le fichier principal \info{mazhe.tex} contient une série de \info{input}. Seuls ceux de la liste seront effectués.

La ligne \info{new\_output\_filename} donne le nom du fichier de sortie. En l'omettant, ce sera un nom compliqué du type \info{0-Inter\_\ldots}. Pour compiler :
\begin{quote}
    \begin{verbatim}
    pytex lst_exemple.py --lotex
    \end{verbatim}
\end{quote}
Le \verb+--lotex+ est seulement là pour dire qu'il faut faire tourner la compilation autant de fois qu'il faudra pour que les références soient justes\footnote{Pour que la bibliographie soit correcte, il faut encore compiler une fois ou deux.}. 

Après compilation, une liste des références incorrectes est donnée. Bien entendu si vous ne compilez qu'une partie, il risque d'y avoir beaucoup de références \emph{manquantes}, mais il ne devrait n'y avoir aucune références \emph{duplicate} !


%+++++++++++++++++++++++++++++++++++++++++++++++++++++++++++++++++++++++++++++++++++++++++++++++++++++++++++++++++++++++++++ 
\section{Les politiques éditoriales}
%+++++++++++++++++++++++++++++++++++++++++++++++++++++++++++++++++++++++++++++++++++++++++++++++++++++++++++++++++++++++++++

Certaines parties de ce texte ne respectent pas les politiques éditoriales. Ce sont des erreurs de jeunesse, et j'en suis le premier triste.

%--------------------------------------------------------------------------------------------------------------------------- 
\subsection{Licence libre}
%---------------------------------------------------------------------------------------------------------------------------

Je crois que c'est clair.

%--------------------------------------------------------------------------------------------------------------------------- 
\subsection{pdflatex}
%---------------------------------------------------------------------------------------------------------------------------

Tout est compilable avec pdf\LaTeX. Pas de pstricks, de psfrag ou de ps<quoiquecesoit>.

%--------------------------------------------------------------------------------------------------------------------------- 
\subsection{utf8}
%---------------------------------------------------------------------------------------------------------------------------

Je crois que c'est clair.

%--------------------------------------------------------------------------------------------------------------------------- 
\subsection{Notations}
%---------------------------------------------------------------------------------------------------------------------------

On essaie d'être cohérent dans les notations et les conventions. Pour la transformée de Fourier par exemple, je crois que la définition du produit scalaire dans \( L^2\), des coefficients de Fourier, de la transformation et de la transformation inverse sont cohérents. Cela demande, lorsqu'on suit un livre qui ne suit pas les conventions utilisées ici, de convertir parfois massivement.

%--------------------------------------------------------------------------------------------------------------------------- 
\subsection{De la bibliographie}
%---------------------------------------------------------------------------------------------------------------------------

On évite d'écrire en haut de chapitre «les références pour ce chapitre sont \ldots». Il est mieux d'écrire au niveau des théorèmes entre parenthèse, les références. De préférence en ligne.

%--------------------------------------------------------------------------------------------------------------------------- 
\subsection{Faire des références à tout}
%---------------------------------------------------------------------------------------------------------------------------

Lorsqu'un utilise le théorème des accroissements finis, il ne faut pas écrire «d'après le théorème des accroissements finis, blablabla». Il faut écrire un \verb+\ref{}+ explicite vers le résultat. Cela alourdit un peu le texte, mais lorsqu'on joue avec un texte de plus de 1000 pages, il est parfois vraiment laborieux de trouver le résultat qu'on cherche (surtout si il existe plusieurs version d'un résultat et que l'on veut faire référence à une version particulière).

%--------------------------------------------------------------------------------------------------------------------------- 
\subsection{Pas de références vers le futur}
%---------------------------------------------------------------------------------------------------------------------------

Dans le Frido, \emph{aucune} preuve ne peut faire une référence vers un résultat prouvé plus bas. On n'utilise pas le théorème 10 dans la démonstration du théorème 7. Cela est une contrainte forte sur le découpage en chapitre et sur l'ordre de présentation des matières.

Il est bien entendu accepté et même encouragé de mettre des notes du type «Nous verrons plus loin un théorème qui \ldots». Tant que ce théorème n'est pas \emph{utilisé}, ça va. Il y a dans le texte plusieurs listes de résultats «analogues». Par exemple au début du chapitre sur les séries de Fourier, une liste de références vers les différents théorèmes qui disent que la série de Fourier de \( f\) converge vers \( f\).  Je crois que ces listes sont bien et qu'il faut en faire plein.

En faisant
\begin{quote}
    \begin{verbatim}
    pytex lst_frido.py --verif
    \end{verbatim}
\end{quote}
vous aurez une liste des références vers le bas. Cette liste doit être vide ! Ce programme cherche tous les \verb+\ref{}+ et \verb+\eqref{}+ ainsi que les \verb+\label{}+ correspondants et vous prévient lorsque le \verb+\label{}+ est après le \verb+\ref{}+.

Si vous pensez qu'une référence pointée doit être accepté (par exemple parce c'est dans une des listes de résultats), alors vous ajoutez son hash dans la liste du fichier \info{commons.py}. Si il s'agit vraiment d'une référence vers un résultat que vous utilisez, alors vous devez déplacer des choses. Soit votre résultat vers le bas, soit celui que vous utilisez vers le haut. Parfois cela demande de déplacer ou redécouper des chapitres entiers\ldots\ Si il n'y a vraiment pas moyen, bravo, vous venez de prouver que la mathématique est logiquement inconsistante.

%+++++++++++++++++++++++++++++++++++++++++++++++++++++++++++++++++++++++++++++++++++++++++++++++++++++++++++++++++++++++++++ 
\section{Vérifier si vous n'avez pas fait de bêtises}
%+++++++++++++++++++++++++++++++++++++++++++++++++++++++++++++++++++++++++++++++++++++++++++++++++++++++++++++++++++++++++++

Lorsqu'on fait de lourdes modifications (déplacement de parties, fusion de théorèmes, etc) il est toujours possible de faire des bêtises d'au moins deux types : créer des références vers le futur et supprimer des parties (genre couper-coller en oubliant le coller). Pour s'en prémunir, il est bon de lancer les compilations suivantes :
\begin{verbatim}
pytex lst_frido.py --lotex
pytex lst_everything.py --lotex
pytex lst_everything.py --verif
pytex lst_mesnotes.py --lotex
\end{verbatim}
Aucun ne devrait provoquer d'erreurs.

Les \verb+--lotex+ font une compilation complète et servent donc à vérifier si tous les \info{label} sont corrects. En particulier \info{lst\_mesnotes.py} est une version du Frido contenant une liste de développements. Si il passe sans erreurs, c'est que vous n'avez certainement pas supprimé de parties importantes.

\end{document}
