\begin{corrige}{IntegralesMultiples0007}

	\begin{enumerate}
		\item
			Ceci est une ellipse de demi grands axes $a$ et $b$. Les coordonnées qui s'invitent d'elles mêmes sont
			\begin{subequations}
				\begin{numcases}{}
					x=ar\cos(\theta)\\
					y=br\sin(\theta),
				\end{numcases}
			\end{subequations}
			dont le jacobien est $abr$.
			\begin{equation}
				A=\int_0^1\int_0^{2\pi}abr\,d\theta dr=ab\pi.
			\end{equation}
			Notez que si $a=b=R$, c'est le cercle de rayon $R$ et nous avons la formule habituelle $\pi R^2$ pour l'aire.

		\item
			Ici c'est la même chose, mais avec trois variables. On prend les coordonnées sphériques modifiées
			\begin{subequations}
				\begin{numcases}{}
					x=a\rho\cos(\theta)\sin(\phi)\\
					y=b\rho\sin(\theta)\sin(\phi)\\
					z=c\rho\cos(\phi),
				\end{numcases}
			\end{subequations}
			dont le jacobien vaut $abc\rho^2\sin(\phi)$. L'intégrale à calculer est donc
			\begin{equation}
				V=abc\int_0^1\int_{-\pi}^{\pi}\int_0^{\pi}\rho^2\sin(\phi)\,d\phi d\theta d\rho=\frac{ 4\pi }{ 3 }abc.
			\end{equation}
			Si $a=b=c=R$, nous avons la sphère de rayon $R$, et nous retrouvons la formule classique du volume de la sphère $V=\frac{ 4 }{ 3 }\pi R^3$.

		\item
			L'ensemble proposé est un cylindre plein de rayon $R$ et de hauteur $h$. Les coordonnées cylindriques sont là pour ça :
			\begin{equation}
				V=\int_0^R\int_0^h\int_0^{2\pi} rd\theta dzdr=\pi R^2 h.
			\end{equation}
		\item
			Il s'agit du volume contenu en dessous du plan $x+y+z-1=0$ dans le premier octant. En fait, il s'agit de la même pyramide qu'on a déjà rencontrée dans l'exercice \ref{exoIntegralesMultiples0005}. L'intégrale à faire est
			\begin{equation}
				V=\int_0^1\int_0^{1-x}\int_0^{1-x-y}dzdydx=\frac{1}{ 6 }.
			\end{equation}
		\item
			Ceci est un cône posé sur sa pointe et de hauteur $h$. En effet, passons aux coordonnées cylindriques, nous avons $r^2<z^2/h^2$, et donc $r<z/h$ parce que $r$ est toujours positif. À chaque hauteur $z$, nous avons donc un disque de rayon $z/h$. Pour calculer le volume, c'est l'intégrale
			\begin{equation}
				V=\int_0^h\int_0^{z/h}\int_0^{2\pi} r\,d\theta drdz=\frac{ \pi h }{ 3 }.
			\end{equation}
			
	\end{enumerate}
	
\end{corrige}
