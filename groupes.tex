
%+++++++++++++++++++++++++++++++++++++++++++++++++++++++++++++++++++++++++++++++++++++++++++++++++++++++++++++++++++++++++++
\section{Relations d'équivalences}
%+++++++++++++++++++++++++++++++++++++++++++++++++++++++++++++++++++++++++++++++++++++++++++++++++++++++++++++++++++++++++++

Soit \( f\) une application entre deux ensembles \( E\) et \( F\). Nous définissons une relation d'équivalence sur \( E\) par
\begin{equation}
    x\sim y\Leftrightarrow f(x)=f(y).
\end{equation}
Nous notons par \( \pi\colon E\to E/\sim\) la projection canonique. L'application
\begin{equation}
    \begin{aligned}
        g\colon E/\sim&\to F \\
        [x]&\mapsto f(x) 
    \end{aligned}
\end{equation}
est bien définie et injective. Elle n'est pas surjective tant que \( f\) ne l'est pas. La \defe{décomposition canonique}{canonique!décomposition}\index{décomposition!canonique} de \( f\) est 
\begin{equation}
    f=g\circ\pi.
\end{equation}

%+++++++++++++++++++++++++++++++++++++++++++++++++++++++++++++++++++++++++++++++++++++++++++++++++++++++++++++++++++++++++++
\section{Groupes}
%+++++++++++++++++++++++++++++++++++++++++++++++++++++++++++++++++++++++++++++++++++++++++++++++++++++++++++++++++++++++++++

Nous allons suivre dans un premier temps \cite{Kropholler}.

\begin{definition}
    Soit \( G\) un groupe. Le \defe{centralisateur}{centralisateur} de \( H\subset G\) est 
    \begin{equation}
        \mZ_G(H)=\{ g\in G\tq hg=gh\,\forall h\in h\}
    \end{equation}
    Si \( H\) est un sous groupe, son \defe{centralisateur}{centralisateur} est
    \begin{equation}
        N_G(H)=\{ g\in G\tq gH=Hg \}.
    \end{equation}
\end{definition}

\begin{definition}
    Un sous groupe \( N\) de \( G\) est \defe{normal}{normal!sous groupe} ou \defe{distingué}{distingué} si pour tout \( g\in G\) et pour tout \( n\in N\), \( gng^{-1}\in N\). Autrement dit lorsque \( gNg^{-1}\subset N\). Lorsque \( N\) est normal dans \( G\) nous noterons \( N\normal G\)\nomenclature[]{\(N \normal G\)}{\( N\) est normal dans \( G\)}.
\end{definition}

\begin{proposition}
    Soit \( N\) un sous groupe de \( G\). Les propriétés suivantes sont équivalentes :
    \begin{enumerate}
        \item
            \( gNg^{-1}\subseteq N\) pour tout \( g\in G\),
        \item
            \( gNg^{-1}= N\) pour tout \( g\in G\),
        \item
            \( gN=Ng\) pour tout \( g\in G\),
        \item
            \( N\) est une union de classes de conjugaison de \( G\),
        \item
            \( N\) est normal.
    \end{enumerate}
\end{proposition}

\begin{definition}
    Soit \( g\in G\) et \( n\in \eZ\). Nous définissons \( g^n\) par
    \begin{enumerate}
        \item
            \( g^0=e\) et \( g^n=gg^{n-1}\) si \( n\) est positif.
        \item
            si \( n<0\), nous posons \( g^n=(g^{-1})^{-n}\).
    \end{enumerate}
\end{definition}

\begin{definition}
    Si \( G\) est un groupe, l'\defe{ordre}{ordre!d'un groupe} est la cardinalité de \( G\) et est noté \( | G |\). L'\defe{ordre}{ordre!élément} d'un élément \( g\) de \( G\) est le naturel
    \begin{equation}
        \min\{ n\in\eN\tq g^n=e \}.
    \end{equation}
    Si le minimum n'existe pas, nous disons que l'ordre de \( g\) est infini.
\end{definition}

%+++++++++++++++++++++++++++++++++++++++++++++++++++++++++++++++++++++++++++++++++++++++++++++++++++++++++++++++++++++++++++
\section{Théorèmes d'isomorphismes}
%+++++++++++++++++++++++++++++++++++++++++++++++++++++++++++++++++++++++++++++++++++++++++++++++++++++++++++++++++++++++++++

Si \( G\) est un groupe et si \( N\) est un sous groupe normal, alors l'ensemble \( G/N\) a une structure de groupe et la projection canonique \( \pi\colon G\to G/N\) est un homomorphisme surjectif de noyau~\( N\).

\begin{theorem}[premier théorème d'isomorphisme]        \label{ThoPremierthoisomo}
    Soit \( \theta\colon G\to H\) un homomorphisme de groupe. Alors
    \begin{enumerate}
        \item
            \( \Kernel\theta\) est normal dans \( G\),
        \item
            \( \Image \theta\) est un sous groupe de \( H\)
        \item
            nous avons un isomorphisme naturel
            \begin{equation}
                G/\Kernel\theta\simeq \Image\theta
            \end{equation}
    \end{enumerate}
\end{theorem}

\begin{proof}
    \begin{enumerate}
        \item
        \item
        \item
            Si \( [g]\) représente la classe de \( g\) dans \( G/\Kernel\theta\), l'isomorphisme est donné par \( \varphi[g]=\theta(g)\).
    \end{enumerate}
\end{proof}


\begin{theorem}[Deuxième théorème d'isomorphisme]
    Soient \( H\) et \( N\) deux sous groupes de \( G\) et supposons que \( N\) soit normal. Alors
    \begin{enumerate}
        \item
            \( NH=HN\) est un sous groupe
        \item
            \( N\normal NH\)
        \item
            \( N\cap H\normal H\)
        \item
            nous avons un isomorphisme
            \begin{equation}
                NH/N\simeq H/H\cap N.
            \end{equation}
    \end{enumerate}
\end{theorem}

\begin{proof}
    \begin{enumerate}
        \item
        \item
        \item
        \item
            Il faut d'abord remarquer que \( H\) et \( N\) étant des groupes et le produit \( NH\) étant un groupe, nous avons \( NH=HN\). Soit le morphisme injectif
            \begin{equation}
                \begin{aligned}
                    j\colon H&\to HN \\
                    h&\mapsto h
                \end{aligned}
            \end{equation}
            et la surjection canonique
            \begin{equation}
                \sigma\colon HN\to HN/N 
            \end{equation}
            Nous considérons ensuite l'application composée
            \begin{equation}
                \begin{aligned}
                    f\colon H&\to HN/N \\
                    h&\mapsto hN. 
                \end{aligned}
            \end{equation}
            L'application \( f\) est surjective parce que l'élément \( hnN\in HN/N\) est l'image de \( h\), étant donné que \( hnN=hN\).

            Le noyau de \( f\) est \( \Kernel f=H\cap N\). En effet si \( a\in H\cap N\), nous avons \( f(a)=\sigma(a)\in K\). Par conséquent \( H\cap N\subset \Kernel f\). D'autre part si \( h\in H\) vérifie \( h\in\Kernel f\), alors \( f(h)=hN=N\), ce qui est uniquement possible si \( h\in N\).

            Le premier théorème d'isomorphisme implique alors que \( H/\Kernel f\simeq \Image f\), c'est à dire
            \begin{equation}
                H/N\cap H\simeq HN/N.
            \end{equation}
    \end{enumerate}
\end{proof}

\begin{theorem}[Troisième théorème d'isomorphisme]
    Soient \( N\) et \( M\) deux sous groupes normaux de \( G\) avec \( M\subset N\). Alors \( N/M\) est normal dans \( G/M\) et
    \begin{equation}
        (G/M)/(N/M)\simeq G/N.
    \end{equation}
\end{theorem}

\begin{proof}
    Afin de montrer que \( N/M\) est normal dans \( G/M\), nous considérons \( g\in G\), \( nM\in N/M\) et nous calculons
    \begin{equation}
        gnMg^{-1}=gng^{-1}\underbrace{gMg^{-1}}_{=M}=\underbrace{gng^{-1}}_{\in N}M\in N/M.
    \end{equation}

    Pour prouver l'isomorphisme nous considérons le morphisme
    \begin{equation}
        \begin{aligned}
            \varphi\colon G/M&\to G/N \\
            gM&\mapsto gN. 
        \end{aligned}
    \end{equation}
    C'est surjectif et le noyau est \( N/M\) parce que \( \varphi(gM)=N\) uniquement si \( g\in N\). Nous pouvons appliquer le premier théorème d'isomorphisme à \( \varphi\) en écrivant
    \begin{equation}
        (G/M)/\Kernel \varphi\simeq\Image \varphi,
    \end{equation}
    c'est à dire
    \begin{equation}
        (G/M)/(N/M)\simeq G/N.
    \end{equation}
\end{proof}

%+++++++++++++++++++++++++++++++++++++++++++++++++++++++++++++++++++++++++++++++++++++++++++++++++++++++++++++++++++++++++++
\section{Indice d'un sous groupe}
%+++++++++++++++++++++++++++++++++++++++++++++++++++++++++++++++++++++++++++++++++++++++++++++++++++++++++++++++++++++++++++

Soit \( G\) un groupe fini et \( H\), un sous groupe. L'\defe{indice}{indice} de \( H\) dans \( G\) est le nombre \( | G |/| H |\), souvent noté \( | G:H |\). Le théorème de Lagrange dira en particulier que l'indice est toujours un nombre entier.

\begin{theorem}[Théorème de Lagrange]\index{théorème!Lagrange}      \label{ThoLagrange}
    Soit \( H\) un sous groupe du groupe fini \( G\). Alors \( | H |\) divise \( | G |\) et les trois nombres suivants sont égaux :
    \begin{enumerate}
        \item
            le nombre de classes de \( H\) à gauche,
        \item
            le nombre de classes de \( H\) à droite,
        \item
            l'indice de \( H\) dans \( G\).
    \end{enumerate}
\end{theorem}

\begin{proof}
    Nous commençons par montrer que les classes de \( H\) ont toutes les même nombre d'éléments que \( H\). En effet pour chaque \( g\in G\) nous avons la bijection
    \begin{equation}
        \begin{aligned}
            \varphi\colon H&\to gH \\
            h&\mapsto gh. 
        \end{aligned}
    \end{equation}
    L'injectivité de \( \varphi\) est le fait que \( gh=gh'\) implique \( h=h'\). La surjectivité est par définition de la classe. 

    Les classes à gauche formant une partition de \( G\), le cardinal de \( G\) est le produit de la taille des classes par le nombre de classes :
    \begin{equation}
        | G |=| H |\cdot\text{nombre de classes}.
    \end{equation}
    En particulier nous voyons que \( | H |\) divise \( | G |\).
\end{proof}

%+++++++++++++++++++++++++++++++++++++++++++++++++++++++++++++++++++++++++++++++++++++++++++++++++++++++++++++++++++++++++++
\section{Action de groupes}
%+++++++++++++++++++++++++++++++++++++++++++++++++++++++++++++++++++++++++++++++++++++++++++++++++++++++++++++++++++++++++++

Si \( G\) agit sur un ensemble \( E\), nous notons \( G\cdot x\) l'orbite de \( x\in E\) sous l'action de $G$. Nous notons \( G_x\) le stabilisateur de \( x\) :
\begin{equation}
    G_x=\{ g\in G\tq g\cdot x=x \}.
\end{equation}

Le groupe \( G\) agit toujours sur lui même à gauche et à droite. L'action à gauche est \( g\cdot h=gh\); celle à droite est \( g\cdot h=hg^{-1}\). Il existe aussi l'action \defe{adjointe}{action!adjointe} définie par \( g\cdot h=ghg^{-1}\).

Si \( H\) est un sous groupe de  \( G\), nous notons \( G/H\) le quotient de $G$ par la relation \( g\sim gh\) pour tout \( h\in H\). Lorsque la distinction est importante, nous noterons \( (G/H)_g\)\nomenclature[A]{$(G/H)_g$}{classes à gauche} pour les classes à gauche et \( (G/H)_d\) pour les classes à droite.

Nous avons une relation d'équivalence à gauche et une à droite. D'abord
\begin{equation}
    x\sim_g y\Leftrightarrow xh=y
\end{equation}
pour un certain \( h\in H\). Ensuite
\begin{equation}
    x\sim_d y\Leftrightarrow hx=y
\end{equation}
pour un certain \( h\in H\). 

Le lemme suivant est une généralisation du théorème de Lagrange \ref{ThoLagrange}.

\begin{lemma}
    L'ensemble \( (G/H)_g\) est fini si et seulement si l'ensemble \( (G/H)_d\) est fini. Si il en est ainsi, alors \( (G/H)_g\) et \( (G/H)_d\) ont même cardinal qui vaut l'indice de \( H\) dans \( H\).
\end{lemma}

\begin{proof}
    L'application
    \begin{equation}
        \begin{aligned}
            f\colon (G/H)_g&\to (G/H)_d \\
            [x]_g&\mapsto [x^{-1}]_d 
        \end{aligned}
    \end{equation}
    est une bijection bien définie. En effet si \( x\sim_g y\), nous avons \( h\in H\) tel que \( y^{-1}h=x^{-1}\), c'est à dire que \( x^{-1}\sim_d y^{-1}\) et \( f\) est bien définie. Le fait que \( f\) soit surjective est évident. Pour l'injectivité, soit
    \begin{equation}
        f([x]_g)=f([y]_h).
    \end{equation}
    Alors \( x^{-1}\sim_d y^{-1}\), ce qui implique l'existence de \( h\in H\) tel que \( hx^{-1}=y^{-1}\), ou encore que \( xh^{-1}=y\), ce qui signifie que \( x\sim_gy\).

    Pour l'énoncé à propos de l'indice, nous procédons en plusieurs étapes simples.
    \begin{enumerate}
        \item
            Les classes (les éléments de \( (G/H)_g\)) formes une partition de $G$.
        \item
            Toutes les classes ont le même nombre d'éléments par la bijection 
            \begin{equation}
                \begin{aligned}
                    f\colon [x]_g&\to [y]_g \\
                    xh&\mapsto yh. 
                \end{aligned}
            \end{equation}
        \item
            Le nombre d'éléments dans une classe est égal à \( | H |\) par la bijection
            \begin{equation}
                \begin{aligned}
                    g\colon [x]_g&\to H \\
                    xh&\mapsto h. 
                \end{aligned}
            \end{equation}
    \end{enumerate}
    Par conséquent
    \begin{equation}
        | G |=| H |\cdot \text{nombre de classes}=| H |\cdot\text{cardinal de $(G/H)_g$},
    \end{equation}
    et nous avons bien 
    \begin{equation}
        \text{cardinal de $(G/H)_g$}=\frac{ | G | }{ | H | }=| G:H |.
    \end{equation}
\end{proof}

\begin{proposition}
    Soit \( G\) un groupe agissant sur un ensemble \( E\) et \( x\in E\).
    \begin{enumerate}
        \item
            Les ensembles \( G\cdot x\) et \( G/G_x\) sont équipotents.
        \item
            L'orbite de \( G_x\) est finie si et seulement si \( G_x\) est d'indice fini dans \( G\). Dans ce cas nous avons 
            \begin{equation}
                \Card(G\cdot x)=| G:G_x |.
            \end{equation}
    \end{enumerate}
\end{proposition}

\begin{proof}
    Soit l'application
    \begin{equation}
        \begin{aligned}
            \psi\colon G\cdot x&\to G/G_x \\
            a\cdot x&\mapsto [a]. 
        \end{aligned}
    \end{equation}
    Cette application est bien définie parce que si \( a\cdot x=b\cdot x\), alors il existe \( h\in G_x\) tel que \( b=ah\), et par conséquent \( [a]=[b]\). Cette application est une bijection et par conséquent \( G\cdot x\) est équipotent à \( G/G_x\).
\end{proof}


%+++++++++++++++++++++++++++++++++++++++++++++++++++++++++++++++++++++++++++++++++++++++++++++++++++++++++++++++++++++++++++
\section{Le groupe des entiers}
%+++++++++++++++++++++++++++++++++++++++++++++++++++++++++++++++++++++++++++++++++++++++++++++++++++++++++++++++++++++++++++

\begin{definition}
    Soit \( G\), un groupe et \( A\) une partie de \( G\). Nous notons \( \gr(A)\)\nomenclature[A]{\( \gr\)}{groupe engendré} l'intersection de tous les sous groupes de \( G\) contenant \( A\). C'est le plus petit (pour l'inclusion) groupe de \( G\) contenant \( A\). Si \( \gr(A)=G\), alors nous disons que \( A\) est une \defe{partie génératrice}{partie génératrice} le groupe \( G\).

    Un groupe est \defe{monogène}{monogène} si il a une partie génératrice réduite à un seul élément.

    Un élément \( a\in G\) est un \defe{générateur}{générateur} de \( G\) si tous les éléments de \( G\) s'écrivent sous la forme \( a^n\) pour un certain \( n\in\eZ\). Un groupe fini et monogène est dit \defe{cyclique}{cyclique}.
\end{definition}

\begin{example}
    Soit \( G\) le groupe des rotations d'angle \( 2k\pi/10\). La rotation d'angle \( \pi/2\) n'est pas génératrice parce qu'elle n'engendre que \( \pi/2\), \( \pi\),\( 3\pi/2\) et l'identité. La rotation d'angle \( 2\pi/10\) par contre est génératrice.
\end{example}


Nous notons \( \eN_{k}=\{ 0,1,\ldots, k \}\subset\eN\)\nomenclature[A]{\( \eN_k\)}{Un sous ensemble des naturels}.
 
\begin{proposition} \label{PropSsgpZestnZ}
    Une partie \( H\) du groupe \( \eZ\) est un sous groupe si et seulement si il existe \( n\in\eN\) tel que \( H=n\eZ\).
\end{proposition}

\begin{proof}
    Soit \( H\neq\{ 0 \}\) un sous groupe de \( \eZ\). L'ensemble \( H\cap\eN^*\) contient un élément minimum que nous notons \( n\). Nous avons certainement \( n\eZ\subset H\) parce que \( H\) est un groupe (donc \( n+n\) et \( -n\) sont dans \( H\) dès que \( n\) est dans \( H\)). Nous devons prouver que \( H\subset n\eZ\).

    Si \( x\in H\), il existe \( q\in\eZ\) et \( r\in\eN_{n-1}\) tels que \( x=nq+r\).

    \begin{probleme}
        Justification par la division euclidienne à venir.
    \end{probleme}
    Nous savons déjà que \( nq\in H\), donc \( r\in H\) parce que \( x\) est également dans \( H\). Mais nous avions décidé que \( n\) serait le plus petit naturel de \( H\cap\eN^*\). Par conséquent \( r=0\) et \( x=nq\in n\eZ\).

\end{proof}

Notons que si un sous groupe \( H\) de \( \eZ\) est donné, alors le nombre \( n\) tel que \( H=n\eZ\) est unique. En effet si \( n\eZ=m\eZ\) nous avons que \( n\) divise \( m\) (parce que \( m\in m\eZ\subset n\eZ\)) et que \( m\) divise \( n\) parce que \( n\in m\eZ\). Par conséquent \( n=m\).

%---------------------------------------------------------------------------------------------------------------------------
\subsection{Division euclidienne}
%---------------------------------------------------------------------------------------------------------------------------

\begin{theorem}[Division euclidienne]     \label{ThoDivisEuclide}
    Soient \( a\in\eZ\) et \( b\in\eN^*\). Il existe un unique couple \( (q,r)\in\eZ\times\eN\) tel que
    \begin{equation}
        a=bq+r
    \end{equation}
    avec \( 0\leq r<b\).
\end{theorem}

L'opération \( (a,b)\mapsto(a,r)\) donnée par le théorème \ref{ThoDivisEuclide} est la \defe{division euclidienne}{division!euclidienne}. Le nombre \( q\) est le \defe{quotient}{quotient} et \( r\) est le \defe{reste}{reste} de la division de \( a\) par \( b\).

%---------------------------------------------------------------------------------------------------------------------------
\subsection{pgcd et ppcm}
%---------------------------------------------------------------------------------------------------------------------------

Soient \( p,q\in\eZ^*\). Les ensembles \( p\eZ\cap q\eZ\) et \( p\eZ+q\eZ\) sont des sous groupes de \( \eZ\). Par conséquent il existe des entiers \( \ppcm(p,q)\) et \( \pgcd(p,q)\) tels que
\begin{subequations}
    \begin{align}
        p\eZ\cap q\eZ&=\ppcm(p,q)\eZ\\
        p\eZ + q\eZ&=\pgcd(p,q)\eZ.
    \end{align}
\end{subequations}

Si \( \pgcd(p,q)=1\), nous disons que \( p\) et \( q\) sont \defe{premiers entre eux}{premiers!deux nombres}.

\begin{theorem}[Théorème de Bezout]
    Deux entiers non nuls \( p,q\in\eZ^*\) sont premiers entre eux si et seulement si il existe \( m,n\in\eZ\) tels que 
    \begin{equation}
        pm+qn=1.
    \end{equation}
\end{theorem}

\begin{proposition}
    soient \( a,b\in\eZ^*\). Si
    \begin{equation}
        \begin{aligned}[]
            a&=\epsilon\prod_{\text{\( p\) premiers}}p^{\mu(p)}&b&=\epsilon'\prod_{\text{\( p\) premier}}p^{\nu(p)},
        \end{aligned}
    \end{equation}
    alors
    \begin{subequations}
        \begin{align}
            \pgcd(a,b)&=\prod p^{\min\{ \mu(p),\nu(p) \}}\\
            \ppcm(a,b)&=\prod p^{\max\{ \mu(p),\nu(p) \}}
        \end{align}
    \end{subequations}    
\end{proposition}

Cette proposition implique que \( m\leq p^n\) et \( \pgcd(m,p^n)\) si et seulement si \( m=qp\) avec \( q\leq p^{n-1}\).

%---------------------------------------------------------------------------------------------------------------------------
\subsection{Quotients}
%---------------------------------------------------------------------------------------------------------------------------

Nous notons \( \eZ_n=\eZ/n\eZ\)\nomenclature[A]{\( \eZ_n\)}{groupe des entiers modulo \( n\)} le groupe des entiers modulo \( n\).

\begin{proposition}
    Soit \( n\in\eN\). Le groupe \( \eZ_n\) est monogène. Si \( n\neq 0\), le groupe \( \eZ_n\) est cyclique d'ordre \( n\).
\end{proposition}

\begin{proof}
    Nous considérons la surjection canonique \( \mu\colon \eZ\to \eZ_n\). Si \( a\in\eZ\), alors \( \mu(a)=a\mu(1)\). Par conséquent \( \eZ_n=\gr\big( \mu(1) \big)\) parce que tout groupe contenant \( \mu(1)\) contient tous les multiples de \( \mu(1)\), et par conséquent contient \( \mu(\eZ)=\eZ_n\).

    Soit \( x\in\eZ_n\) et considérons \( x_0\), le plus petit naturel représentant \( x\). Nous notons \( x=[x_0]\). Le théorème de la division euclidienne \ref{ThoDivisEuclide} donne l'existence de \( q\) et \( r\) avec \( 0\leq r<n\) et \( q\geq 0\) tels que
    \begin{equation}
        x_0=nq+r.
    \end{equation}
    Nous avons \( [x_0]=[r]=\mu(r)\) parce que \( x_0-r\) est un multiple de \( n\). Nous avons donc \( [x_0]\in\mu(\eN_{n-1})\). Par conséquent
    \begin{equation}
        \eZ_n=\mu(\eZ)=\mu(\eN_{n-1}).
    \end{equation}
    La restriction \( \mu\colon \eN_{n-1}\to \eZ_n\) est donc surjective. Montrons qu'elle est également injective. Si \( \mu(x_0)=\mu(x_1)\), alors \( x_1=x_0+kn\). Si nous supposons que \( x_1>x_0\), alors \( k>0\) et si \( x_0\in\eN_{n-1}\), alors \( x_1>n-1\).

    L'ordre de \( \eZ_n\) est donc le même que le cardinal de \( \eN_{n-1}\), c'est à dire \( n\). Le groupe \( \eZ_n\) est donc fini, d'ordre \( n\) et monogène (parce que \( \eZ_n=\gr(\mu(1))\)). Il est donc cyclique.
\end{proof}
 
Pour \( n\in\eN^+\) nous introduisons l'ensemble\nomenclature[A]{\( P_n\)}{les nombres premiers avec \( n\)}
\begin{equation}    \label{EqDefPnEntierldeost}
    P_n=\{ x\in\{ 1,\ldots,n \}\tq\pgcd(x,n)=1 \}.
\end{equation}
C'est l'ensemble des entiers inférieurs à \( n\), premiers avec \( n\). La fonction \( \varphi\) donnée par
\begin{equation}
    \varphi(n)=\Card(P_n)
\end{equation}
est l'\defe{indicatrice d'Euler}{indicatrice d'Euler}\index{Euler!indicatrice}. Nous avons par exemple
\begin{subequations}
    \begin{align}
        P_8&=\{ 1,3,5,7 \}\\
        P_7&=\{ 1,2,3,4,5,6 \}.
    \end{align}
\end{subequations}
Si \( p\) est un nombre premier, alors \( \varphi(p)=p-1\).

\begin{lemma}
    Si \( p\) est un nombre premier, alors \( \varphi(p^n)=p^n-p^{n-1}\).
\end{lemma}

\begin{proof}
    Les éléments de \( \{ 1,\ldots,p^n \}\) qui ont un \( \pgcd\) différent de \( 1\) avec \( p^n\) sont des nombres qui s'écrivent sous la forme \( qp\) avec \( q\leq p^{n-1}\). Il y a évidemment \( p^{n-1}\) tels nombres.

    Par conséquent le cardinal de \( P_{p^n}\) est \( \varphi(p^{n})=p^n-p^{n-1}\).
\end{proof}

\begin{proposition}     \label{PropZnmuphiGensn}
    Soit \( n\in\eN^*\) et \( \mu\colon \eZ\to \eZ_n\) la projection canonique. Les générateurs de \( \eZ_n\) sont les \( \mu(x)\) avec \( x\in P_n\). En particulier \( \eZ_n\) est un groupe contenant \( \varphi(n)\) générateurs.
\end{proposition}

\begin{proof}
    La classe de \( 1\) dans \( \eZ_n\) est \( \mu(1)\), par conséquent nous avons \( \gr\big( \mu(1) \big)=\eZ_n\). L'élément \( \mu(x)\) sera générateur si et seulement si il génère \( \mu(1)\), c'est à dire si il existe \( p\) tel que \( p\mu(x)=\mu(1)\). Cette dernière égalité étant une égalité de classes dans \( \eZ_n\), elle sera vraie si et seulement si il existe \( q\) tel que
    \begin{equation}
        px+qn=1.
    \end{equation}
    Cela signifie entre autres que \( x\eZ+n\eZ=\eZ\), c'est à dire que \( \pgcd(x,n)=1\) et que \( x\in P_n\).
\end{proof}


\begin{theorem}
    Un groupe monogène est abélien. Plus précisément,
    \begin{enumerate}
        \item
            un groupe monogène infini est isomorphe à \( \eZ\),
        \item
            un groupe monogène fini est isomorphe à \( \eZ_n\) pour un certain \( n\).
    \end{enumerate}
    Un groupe monogène d'ordre \( n\) possède \( \varphi(n)\) générateurs.
\end{theorem}

\begin{proof}
    Le groupe est abélien parce que $g=a^n$, \( g'=a^{n'}\) implique \( gg'=q^{n+n'}=g'g\). Nous considérons un générateur \( a\) de \( G\) (qui existe parce que $G$ est monogène) et le morphisme surjectif
    \begin{equation}
        \begin{aligned}
            f\colon \eZ&\to G \\
            p&\mapsto a^p. 
        \end{aligned}
    \end{equation}
    Si \( G\) est infini, alors \( f\) est injective parce que si \( a^n=a^{n'}\), alors \( a^{n-n'}=e\), ce qui rendrait \( G\) cyclique et par conséquent non infini. Nous concluons que si \( G\) est infini, alors \( f\) est une bijection et donc un isomorphisme \( \eZ\simeq G\).

    Si \( G\) est fini, alors \( f\) n'est pas injective et a un noyau \( \ker f\). Étant donné que \( \ker f\) est un sous groupe de \( G\), il existe un (unique) \( n\) tel que \( \ker f=n\eZ\) et le premier théorème d'isomorphisme (théorème \ref{ThoPremierthoisomo}) nous indique que
    \begin{equation}
        \eZ/\ker f=\eZ/n\eZ=\Image f=G.
    \end{equation}
    Dans ce cas, le fait qu'un groupe monogène d'ordre \( n\) possède \( \varphi(n)\) générateurs est le contenu de la proposition \ref{PropZnmuphiGensn}.
\end{proof}

%+++++++++++++++++++++++++++++++++++++++++++++++++++++++++++++++++++++++++++++++++++++++++++++++++++++++++++++++++++++++++++
\section{Théorèmes de Sylow}
%+++++++++++++++++++++++++++++++++++++++++++++++++++++++++++++++++++++++++++++++++++++++++++++++++++++++++++++++++++++++++++

\begin{lemma}
    Soient \( H\) et \( K\) des sous groupes finis de \( G\). Alors
    \begin{equation}
        \Card(HK)=\frac{ | H |\cdot | K | }{ | H\cap K | }.
    \end{equation}
\end{lemma} 
Attention : dans ce lemme, l'ensemble \( HK\) n'est pas spécialement un groupe. Ce serait le cas si \( H\) normaliserait \( K\), c'est à dire si nous avions \( hkh^{-1}\in K<,\forall h,k\in H\times K\).

\begin{theorem}[Théorème de Cauchy]\index{Cauchy!théorème}\index{Théorème!Cauchy}       \label{ThoCauchyGpFini}
    Soit \( G\) un groupe fini et \( p\) un nombre premier divisant \( | G |\). Alors 
    \begin{enumerate}
        \item
            \( G\) contient un élément d'ordre \( p\).
        \item
            Si \( G\) est un \( p\)-groupe, il existe un élément central d'ordre \( p\) dans \( G\).
    \end{enumerate}
\end{theorem}

\begin{lemma}
    Soit \( G\) un groupe fini et \( P\), \( Q\) des \( p\)-sous groupes. Nous supposons que \( Q\) normalise \( P\). Alors \( PQ\) est un \( p\)-sous groupe de \( G\).
\end{lemma}


\begin{definition}
    Soit \( p\) un nombre premier. Un \( p\)-groupe est un groupe dont tous les éléments sont d'ordre \( p^m\) pour un certain \( m\) (dépendant de l'élément).

    Soit \( G\) un groupe fini de cardinal \( N\) et \( p\), un diviseur premier de $| G |$. Un \( p\)-Sylow dans \( G\) est une \( p\)-sous groupe d'ordre \( p^n\) où \( p^n\) est la plus grande puissance de \( p\) divisant \( | G |\).
\end{definition}

Si \( S\) est un \( p\)-Sylow, alors \( p\) ne divise pas le nombre \( | G:S |=| G |/| S |\).

\begin{proposition}
    Soit \( p\) un nombre premier. Un groupe fini \( G\) est un $p$-groupe si et seulement l'ordre de \( G\) est \( p^n\) pour un certain \( n\).
\end{proposition}

\begin{proof}
    Supposons que \( G\) est un $p$-groupe. Soit \( q\) un nombre premier divisant \( | G |\). Par le théorème de Cauchy (\ref{ThoCauchyGpFini}), le groupe \( G\) contient un élément d'ordre \( q\), soit \( g\) un tel élément. Étant donné que \( G\) est un $p$-groupe, \( g^{p^n}=g^q=e\) pour un certain \( n\). Donc $q=p^n$ et \( q=p\) parce que \( q\) est premier. Nous venons de prouver que \( p\) est le seul nombre premier qui divise \( | G |\). L'ordre de \( G\) est par conséquent une puissance de \( p\).

    Nous nous intéressons maintenant à l'implication inverse. Nous supposons que \( | G |=p^n\) pour un certain entier \( n\geq 0\). Soit \( g\in G\); nous notons \( r\) l'ordre de \( G\). Le sous groupe \( \gr(g)\) est d'ordre \( r\), donc \( r\) divise \( | G |\) (par le théorème \ref{ThoLagrange} de Lagrange). Le nombre \( r\) est alors une puissance de \( p\).
\end{proof}


\begin{theorem}[Théorème de Sylow]
    Soit \( G\) un groupe fini de cardinal \( N\) et \( p\), un diviseur premier de \( N\). Alors
    \begin{enumerate}
        \item
            \( G\) possède des \( p\)-Sylow.
        \item
            Tout \( p\)-sous groupe de \( G\) est contenu dans un \( p\)-Sylow.
        \item
            Les \( p\)-Sylow de \( G\) sont conjugués.
        \item
            Le nombre de \( p\)-Sylow de \( G\) est congru à \( 1\) modulo \( p\).
    \end{enumerate}
\end{theorem}

\begin{proof}
    \begin{enumerate}
        \item
            Nous faisons la récurrence sur l'ordre \( N\) de \( G\). Pour \( N=1\), le seul \( p\)-Sylow est le groupe entier. Si \( N>1\), nous commençons par supposer que \( G\) contient un sous groupe propre \( H\) tel que \( \pgcd(| G:H |,p)=1\).

            Si \( G\) contient un sous groupe propre \( H\) tel que \( \pgcd(| G:H |,p)=1\), alors \( p\) divise \( | G:H |\). En effet \( p\) divise \( N\) et \( | G:H |=| G |/| H |\). Si il n'y a pas de \( p\) dans la décomposition de \( | H |\), alors \( p\) divise encore \( | G |/| H |\). Étant donné que \( p\) est un diviseur premier de \( | H |\), le groupe \( H\) contient des \( p\)-Sylow par hypothèse de récurrence. Montrons que si \( S\) est un \( p\)-Sylow de \( H\), alors \( S\) est également un $p$-Sylow de \( G\).

            Soit \( S\), un $p$-Sylow de \( H\). Nous avons \( | S |=p^n\) où \( n\) est la plus grande puissance de \( p\) divisant \( H\). Par hypothèse, il n'y a pas de \( p\) dans la décomposition de \( | G:H |\); par conséquent \( p^n\) est également la plus grande puissance de \( p\) qui divise \( | G |\) et \( S\) est alors un $p$-Sylow de \( G\).

            Nous supposons maintenant que \( G\) ne possède pas de sous groupe \( H\) tels que \( \pgcd(| G:H |,p)=1\).

    \end{enumerate}
\end{proof}

