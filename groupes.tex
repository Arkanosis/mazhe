Nous allons suivre dans un premier temps \cite{Kropholler}.


\begin{definition}
    Soit \( G\) un groupe. Le \defe{centralisateur}{centralisateur} de \( H\subset G\) est 
    \begin{equation}
        \mZ_G(H)=\{ g\in G\tq hg=gh\,\forall h\in h\}
    \end{equation}
    Si \( H\) est un sous groupe, son \defe{centralisateur}{centralisateur} est
    \begin{equation}
        N_G(H)=\{ g\in G\tq gH=Hg \}.
    \end{equation}
\end{definition}

\begin{definition}
    Un sous groupe \( N\) de \( G\) est \defe{normal}{normal!sous groupe} si pour tout \( g\in G\) et pour tout \( n\in N\), \( gng^{-1}\in N\). Autrement dit lorsque \( gNg^{-1}\subset N\). Lorsque \( N\) est normal dans \( G\) nous noterons \( N\normal G\)\nomenclature[]{\(N \normal G\)}{\( N\) est normal dans \( G\)}.
\end{definition}

\begin{proposition}
    Soit \( N\) un sous groupe de \( G\). Les propriétés suivantes sont équivalentes :
    \begin{enumerate}
        \item
            \( gNg^{-1}\subseteq N\) pour tout \( g\in G\),
        \item
            \( gNg^{-1}= N\) pour tout \( g\in G\),
        \item
            \( gN=Ng\) pour tout \( g\in G\),
        \item
            \( N\) est une union de classes de conjugaison de \( G\),
        \item
            \( N\) est normal.
    \end{enumerate}
\end{proposition}


