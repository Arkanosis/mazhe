Nous allons suivre dans un premier temps \cite{Kropholler}.


\begin{definition}
    Soit \( G\) un groupe. Le \defe{centralisateur}{centralisateur} de \( H\subset G\) est 
    \begin{equation}
        \mZ_G(H)=\{ g\in G\tq hg=gh\,\forall h\in h\}
    \end{equation}
    Si \( H\) est un sous groupe, son \defe{centralisateur}{centralisateur} est
    \begin{equation}
        N_G(H)=\{ g\in G\tq gH=Hg \}.
    \end{equation}
\end{definition}

\begin{definition}
    Un sous groupe \( N\) de \( G\) est \defe{normal}{normal!sous groupe} si pour tout \( g\in G\) et pour tout \( n\in N\), \( gng^{-1}\in N\). Autrement dit lorsque \( gNg^{-1}\subset N\). Lorsque \( N\) est normal dans \( G\) nous noterons \( N\normal G\)\nomenclature[]{\(N \normal G\)}{\( N\) est normal dans \( G\)}.
\end{definition}

\begin{proposition}
    Soit \( N\) un sous groupe de \( G\). Les propriétés suivantes sont équivalentes :
    \begin{enumerate}
        \item
            \( gNg^{-1}\subseteq N\) pour tout \( g\in G\),
        \item
            \( gNg^{-1}= N\) pour tout \( g\in G\),
        \item
            \( gN=Ng\) pour tout \( g\in G\),
        \item
            \( N\) est une union de classes de conjugaison de \( G\),
        \item
            \( N\) est normal.
    \end{enumerate}
\end{proposition}

\begin{definition}
    Soit \( g\in G\) et \( n\in \eZ\). Nous définissons \( g^n\) par
    \begin{enumerate}
        \item
            \( g^0=e\) et \( g^n=gg^{n-1}\) si \( n\) est positif.
        \item
            si \( n<0\), nous posons \( g^n=(g^{-1})^{-n}\).
    \end{enumerate}
\end{definition}

\begin{definition}
    Si \( G\) est un groupe, l'\defe{ordre}{ordre!d'un groupe} est la cardinalité de \( G\) et est noté \( | G |\). L'\defe{ordre}{ordre!élément} d'un élément \( g\) de \( G\) est le naturel
    \begin{equation}
        \min\{ n\in\eN\tq g^n=e \}.
    \end{equation}
    Si le minimum n'existe pas, nous disons que l'ordre de \( g\) est infini.
\end{definition}

%+++++++++++++++++++++++++++++++++++++++++++++++++++++++++++++++++++++++++++++++++++++++++++++++++++++++++++++++++++++++++++
\section{Théorèmes d'isomorphismes}
%+++++++++++++++++++++++++++++++++++++++++++++++++++++++++++++++++++++++++++++++++++++++++++++++++++++++++++++++++++++++++++

Si \( G\) est un groupe et si \( N\) est un sous groupe normal, alors l'ensemble \( G/N\) a une structure de groupe et la projection canonique \( \pi\colon G\to G/N\) est un homomorphisme surjectif de noyau~\( N\).

\begin{theorem}[premier théorème d'isomorphisme]
    Soit \( \theta\colon G\to H\) un homomorphisme de groupe. Alors
    \begin{enumerate}
        \item
            \( \Kernel\theta\) est normal dans \( G\),
        \item
            \( \Image \theta\) est un sous groupe de \( H\)
        \item
            nous avons un isomorphisme naturel
            \begin{equation}
                G/\Kernel\theta\simeq \Image\theta
            \end{equation}
    \end{enumerate}
\end{theorem}

\begin{proof}
    \begin{enumerate}
        \item
        \item
        \item
            Si \( [g]\) représente la classe de \( g\) dans \( G/\Kernel\theta\), l'isomorphisme est donné par \( \varphi[g]=\theta(g)\).
    \end{enumerate}
\end{proof}


\begin{theorem}[Deuxième théorème d'isomorphisme]
    Soient \( H\) et \( N\) deux sous groupes de \( G\) et supposons que \( N\) soit normal. Alors
    \begin{enumerate}
        \item
            \( NH=HN\) est un sous groupe
        \item
            \( N\normal NH\)
        \item
            \( N\cap H\normal H\)
        \item
            nous avons un isomorphisme
            \begin{equation}
                NH/N\simeq H/H\cap N.
            \end{equation}
    \end{enumerate}
\end{theorem}

\begin{proof}
    \begin{enumerate}
        \item
        \item
        \item
        \item
            Il faut d'abord remarquer que \( H\) et \( N\) étant des groupes et le produit \( NH\) étant un groupe, nous avons \( NH=HN\). Soit le morphisme injectif
            \begin{equation}
                \begin{aligned}
                    j\colon H&\to HN \\
                    h&\mapsto h
                \end{aligned}
            \end{equation}
            et la surjection canonique
            \begin{equation}
                \sigma\colon HN\to HN/N 
            \end{equation}
            Nous considérons ensuite l'application composée
            \begin{equation}
                \begin{aligned}
                    f\colon H&\to HN/N \\
                    h&\mapsto hN. 
                \end{aligned}
            \end{equation}
            L'application \( f\) est surjective parce que l'élément \( hnN\in HN/N\) est l'image de \( h\), étant donné que \( hnN=hN\).

            Le noyau de \( f\) est \( \Kernel f=H\cap N\). En effet si \( a\in H\cap N\), nous avons \( f(a)=\sigma(a)\in K\). Par conséquent \( H\cap N\subset \Kernel f\). D'autre part si \( h\in H\) vérifie \( h\in\Kernel f\), alors \( f(h)=hN=N\), ce qui est uniquement possible si \( h\in N\).

            Le premier théorème d'isomorphisme implique alors que \( H/\Kernel f\simeq \Image f\), c'est à dire
            \begin{equation}
                H/N\cap H\simeq HN/N.
            \end{equation}
    \end{enumerate}
\end{proof}

\begin{theorem}[Troisième théorème d'isomorphisme]
    Soient \( N\) et \( M\) deux sous groupes normaux de \( G\) avec \( M\subset N\). Alors \( N/M\) est normal dans \( G/M\) et
    \begin{equation}
        (G/M)/(N/M)\simeq G/N.
    \end{equation}
\end{theorem}

\begin{proof}
    Afin de montrer que \( N/M\) est normal dans \( G/M\), nous considérons \( g\in G\), \( nM\in N/M\) et nous calculons
    \begin{equation}
        gnMg^{-1}=gng^{-1}\underbrace{gMg^{-1}}_{=M}=\underbrace{gng^{-1}}_{\in N}M\in N/M.
    \end{equation}

    Pour prouver l'isomorphisme nous considérons le morphisme
    \begin{equation}
        \begin{aligned}
            \varphi\colon G/M&\to G/N \\
            gM&\mapsto gN. 
        \end{aligned}
    \end{equation}
    C'est surjectif et le noyau est \( N/M\) parce que \( \varphi(gM)=N\) uniquement si \( g\in N\). Nous pouvons appliquer le premier théorème d'isomorphisme à \( \varphi\) en écrivant
    \begin{equation}
        (G/M)/\Kernel \varphi\simeq\Image \varphi,
    \end{equation}
    c'est à dire
    \begin{equation}
        (G/M)/(N/M)\simeq G/N.
    \end{equation}
\end{proof}

%+++++++++++++++++++++++++++++++++++++++++++++++++++++++++++++++++++++++++++++++++++++++++++++++++++++++++++++++++++++++++++
\section{Indice d'un sous groupe}
%+++++++++++++++++++++++++++++++++++++++++++++++++++++++++++++++++++++++++++++++++++++++++++++++++++++++++++++++++++++++++++

Soit \( G\) un groupe fini et \( H\), un sous groupe. L'\defe{indice}{indice} de \( H\) dans \( G\) est le nombre \( | G |/| H |\), souvent noté \( | G:H |\). Le théorème de Lagrange dira en particulier que l'indice est toujours un nombre entier.

\begin{theorem}[Théorème de Lagrange]\index{théorème!Lagrange}
    Soit \( H\) un sous groupe du groupe fini \( G\). Alors \( | H |\) divise \( | G |\) et les trois nombres suivants sont égaux :
    \begin{enumerate}
        \item
            le nombre de classes de \( H\) à gauche,
        \item
            le nombre de classes de \( H\) à droite,
        \item
            l'indice de \( H\) dans \( G\).
    \end{enumerate}
\end{theorem}

\begin{proof}
    Nous commençons par montrer que les classes de \( H\) ont toutes les même nombre d'éléments que \( H\). En effet pour chaque \( g\in G\) nous avons la bijection
    \begin{equation}
        \begin{aligned}
            \varphi\colon H&\to gH \\
            h&\mapsto gh. 
        \end{aligned}
    \end{equation}
    L'injectivité de \( \varphi\) est le fait que \( gh=gh'\) implique \( h=h'\). La surjectivité est par définition de la classe. 

    Les classes à gauche formant une partition de \( G\), le cardinal de \( G\) est le produit de la taille des classes par le nombre de classes :
    \begin{equation}
        | G |=| H |\cdot\text{nombre de classes}.
    \end{equation}
    En particulier nous voyons que \( | H |\) divise \( | G |\).
\end{proof}
