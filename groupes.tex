% This is part of Mes notes de mathématique
% Copyright (c) 2011-2012
%   Laurent Claessens
% See the file fdl-1.3.txt for copying conditions.

% TODO : à certains endroits j'utilise la macro \Gl. Est-ce que ça ne devrait pas être \GL ?

\nomenclature{$\eN_0$}{les naturels non nuls : $\eN_0=\eN\setminus\{ 0 \}$}
%+++++++++++++++++++++++++++++++++++++++++++++++++++++++++++++++++++++++++++++++++++++++++++++++++++++++++++++++++++++++++++
\section{Complémentaire}
%+++++++++++++++++++++++++++++++++++++++++++++++++++++++++++++++++++++++++++++++++++++++++++++++++++++++++++++++++++++++++++
\label{AppComplement}

Soit $E$, un ensemble et $A$, une partie de $E$ (c'est à dire un sous-ensemble de $E$). Nous désignons par $\complement A$\nomenclature[T]{$\complement A$}{Le complémentaire de l'ensemble $A$} désigne le \defe{complémentaire}{complémentaire} de l'ensemble $A$ dans $E$. Il s'agit de l'ensemble des points de $E$ qui ne font pas partie de $A$ :
\begin{equation}
	\complement A=E\setminus A=\{ x\in E\tq x\notin A \}.
\end{equation}

\begin{lemma}		\label{LemPropsComplement}
	Quelque propriétés à propos des complémentaires. Si $E$ est un ensemble et si $A$ et $B$ sont des sous-ensembles de $E$, nous avons
	\begin{enumerate}
		\item
			$\complement \complement A =A $, en d'autres termes, $E\setminus(E\setminus A)=A$,
		\item
			$\complement(A\cap B)=\complement A\cup\complement B$,
		\item
			$\complement(A\cup B)=\complement A\cap\complement B$,
		\item	\label{ItemLemPropComplementiii}
			$A\setminus B=A\cap\complement B$.
	\end{enumerate}
\end{lemma}

%+++++++++++++++++++++++++++++++++++++++++++++++++++++++++++++++++++++++++++++++++++++++++++++++++++++++++++++++++++++++++++
\section{Relations d'équivalence}
%+++++++++++++++++++++++++++++++++++++++++++++++++++++++++++++++++++++++++++++++++++++++++++++++++++++++++++++++++++++++++++
\label{appEquivalence}

\begin{definition}  \label{DefHoJzMp}
Si $E$ est un ensemble, une \defe{relation d'équivalence}{equivalence@équivalence!relation} sur $E$ est une relation $\sim$ telle qui est à la fois
\begin{description}
	\item[réflexive] $x\sim x$ pour tout $x\in E$,
	\item[symétrique] $x\sim y$ si et seulement si $y\sim x$;
	\item[transitive] si $x\sim y$ et $y\sim z$, alors $x\sim z$.
\end{description}
\end{definition}

Par exemple, sur l'ensemble de tous les polygones du plan, la relation «a le même nombre de côté» est une relation d'équivalence. Plus précisément, si $P$ et $Q$ sont deux polygones, nous disons que $P\sim Q$ si et seulement si $P$ et $Q$ ont le même nombre de côté. Cela est une relation d'équivalence :
\begin{itemize}
	\item 
		un polygone $P$ a toujours le même nombre de côtés que lui-même : $P\sim P$;
	\item
		si $P$ a le même nombre de côtés que $Q$ ($P\sim Q$), alors $Q$ a le même nombre de côtés que $P$ ($Q\sim P$);
	\item
		si $P$ a le même nombre de côtés que $Q$ ($P\sim Q$) et que $Q$ a le même nombre de côtés que $R$ ($Q\sim R$), alors $P$ a le même nombre de côtés que $R$ ($P\sim R$).
\end{itemize}

Soit \( f\) une application entre deux ensembles \( E\) et \( F\). Nous définissons une relation d'équivalence sur \( E\) par
\begin{equation}
    x\sim y\Leftrightarrow f(x)=f(y).
\end{equation}
Nous notons par \( \pi\colon E\to E/\sim\) la projection canonique. L'application
\begin{equation}
    \begin{aligned}
        g\colon E/\sim&\to F \\
        [x]&\mapsto f(x) 
    \end{aligned}
\end{equation}
est bien définie et injective. Elle n'est pas surjective tant que \( f\) ne l'est pas. La \defe{décomposition canonique}{canonique!décomposition}\index{décomposition!canonique} de \( f\) est 
\begin{equation}
    f=g\circ\pi.
\end{equation}

%+++++++++++++++++++++++++++++++++++++++++++++++++++++++++++++++++++++++++++++++++++++++++++++++++++++++++++++++++++++++++++
\section{Groupes}
%+++++++++++++++++++++++++++++++++++++++++++++++++++++++++++++++++++++++++++++++++++++++++++++++++++++++++++++++++++++++++++

Nous allons suivre dans un premier temps \cite{Kropholler}.

\begin{definition}
    Soit \( G\) un groupe. Le \defe{centralisateur}{centralisateur} de \( H\subset G\) est 
    \begin{equation}
        \mZ_G(H)=\{ g\in G\tq hg=gh\,\forall h\in h\}
    \end{equation}
    Si \( H\) est un sous-groupe, son \defe{normalisateur}{normalisateur} est
    \begin{equation}
        N_G(H)=\{ g\in G\tq gH=Hg \}.
    \end{equation}
\end{definition}

Le \defe{centre}{centre!d'un groupe} du groupe \( G\) est l'ensemble
\begin{equation}
    Z_G=\{ z\in G\tq gz=zg\forall g\in G \}.
\end{equation}

\begin{definition}
    Un sous-groupe \( N\) de \( G\) est \defe{normal}{normal!sous-groupe} ou \defe{distingué}{distingué} si pour tout \( g\in G\) et pour tout \( n\in N\), \( gng^{-1}\in N\). Autrement dit lorsque \( gNg^{-1}\subset N\). Lorsque \( N\) est normal dans \( G\) nous noterons \( N\normal G\)\nomenclature[]{\(N \normal G\)}{Le sous-groupe \( N\) est normal dans \( G\)}.

    Un sous-groupe \( H\) de \( G\) est un sous-groupe \defe{caractéristique}{sous-groupe!caractéristique}\index{caractéristique!sous-groupe} si \( \alpha(H)=H\) pour tout \( \alpha\in \Aut(G)\).
\end{definition}

%---------------------------------------------------------------------------------------------------------------------------
\subsection{Sous groupe normal}
%---------------------------------------------------------------------------------------------------------------------------

\begin{proposition}
    Soit \( N\) un sous-groupe de \( G\). Les propriétés suivantes sont équivalentes :
    \begin{enumerate}
        \item
            \( gNg^{-1}\subseteq N\) pour tout \( g\in G\),
        \item
            \( gNg^{-1}= N\) pour tout \( g\in G\),
        \item
            \( gN=Ng\) pour tout \( g\in G\),
        \item
            \( N\) est une union de classes de conjugaison de \( G\),
        \item
            \( N\) est normal.
    \end{enumerate}
\end{proposition}

\begin{definition}
    Soit \( g\in G\) et \( n\in \eZ\). Nous définissons \( g^n\) par
    \begin{enumerate}
        \item
            \( g^0=e\) et \( g^n=gg^{n-1}\) si \( n\) est positif.
        \item
            si \( n<0\), nous posons \( g^n=(g^{-1})^{-n}\).
    \end{enumerate}
\end{definition}


\begin{definition}
    Si \( G\) est un groupe, l'\defe{ordre}{ordre!d'un groupe} est la cardinalité de \( G\) et est noté \( | G |\). L'\defe{ordre}{ordre!élément} d'un élément \( g\) de \( G\) est le naturel
    \begin{equation}
        \min\{ n\in\eN\tq g^n=e \}.
    \end{equation}
    Si le minimum n'existe pas, nous disons que l'ordre de \( g\) est infini.
\end{definition}
Nous verrons que le corollaire \ref{CorpZItFX} au théorème de Lagrange dira que l'ordre d'un élément divise l'ordre du groupe.

\begin{lemma}[\cite{PDFpersoWanadoo}]\label{LemHUkMxp}
    Si \( H\) et \( K\) sont normaux dans le groupe \( G\) et si \( H\cap K=\{ e \}\) alors \( HK\simeq H\times K\).
\end{lemma}

\begin{definition}  \label{DefvtSAyb}
    L'\defe{exposant}{exposant!d'un groupe} du groupe \( G\) est le plus petit entier non nul \( n\) tel que \( g^n=e\) pour tout \( g\in G\). Si il n'existe pas, nous disons que l'exposant du groupe est infini.
\end{definition}
Si l'ordre de tous les éléments acceptent un majorant commun, alors l'exposant du groupe est le plus petit commun multiple des ordres des éléments. En particulier pour un groupe fini, l'exposant est le $\ppcm$ des ordres des éléments du groupe.

Le théorème de Burnside \ref{ThooJLTit} nous donnera un bon paquet d'exemples de groupes d'exposant fini dans \( \GL(n,\eC)\).

%---------------------------------------------------------------------------------------------------------------------------
\subsection{Groupe dérivé}
%---------------------------------------------------------------------------------------------------------------------------

Si \( G\) est un groupe et si \( g,h\in G\), nous notons \( [g,h]=ghg^{-1}h^{-1}\)\nomenclature[R]{\( [g,h]\)}{commutateur dans un groupe} le \defe{commutateur}{commutateur!dans un groupe} de \( g\) et \( h\). Le \defe{groupe dérivé}{dérivé!groupe}\index{groupe!dérivé} de \( G\) est le sous-groupe note \( D(G)\)\nomenclature[R]{\( D(G)\)}{groupe dérivé} ou \( [G,G]\)\nomenclature[R]{\( [G,G]\)}{groupe dérivé} engendré par les commutateurs.

\begin{proposition}
    Le groupe dérivé est un sous-groupe caractéristique, et est en particulier un sous-groupe normal.

    Le groupe quotient \( G/D(G)\) est abélien.
\end{proposition}

\begin{proof}
    Si \( \alpha\in \Aut(G)\), alors
    \begin{equation}
        \alpha\big( [g,h] \big)=\big[ \alpha(g),\alpha(h) \big],
    \end{equation}
    En particulier le sous-groupe \( D(G)\) est normal dans \( G\).

    En ce qui concerne le fait que \( G/D(G)\) soit abélien, nous savons que pour tout \( g,h\in G\) nous avons \( h^{-1}g^{-1}hg\in D(G)\) et donc
    \begin{equation}
        [g][h]=[gh]=[ghh^{-1}g^{-1}hg]=[hg]=[h][g].
    \end{equation}
\end{proof}

Le groupe quotient \( G/D(G)\) est appelé l'\defe{abélianisé}{abélianisé} de \( G\) et est parfois noté \( G^{ab}\)\nomenclature[R]{\( G^{ab}\)}{groupe abélianisé de \( G\)}.


Si \( f\colon G\to A\) est un homomorphisme entre le groupe \( G\) et un groupe abélien \( A\), alors \( f\big( D(G) \big)=\{ 0 \}\). Du coup \( f\) passe au quotient de \( G\) par \( D(G)\), et il existe une unique application \( \bar f\colon G/D(G)\to A\) telle que \( f=\bar f\circ \pi\) où \( \pi\colon G\to G/D(G)\) est la projection canonique.

%+++++++++++++++++++++++++++++++++++++++++++++++++++++++++++++++++++++++++++++++++++++++++++++++++++++++++++++++++++++++++++
\section{Théorèmes d'isomorphismes}
%+++++++++++++++++++++++++++++++++++++++++++++++++++++++++++++++++++++++++++++++++++++++++++++++++++++++++++++++++++++++++++

Si \( G\) est un groupe et si \( N\) est un sous-groupe normal, alors l'ensemble \( G/N\) a une structure de groupe et la projection canonique \( \pi\colon G\to G/N\) est un homomorphisme surjectif de noyau~\( N\).

\begin{theorem}[premier théorème d'isomorphisme]        \label{ThoPremierthoisomo}
    Soit \( \theta\colon G\to H\) un homomorphisme de groupe. Alors
    \begin{enumerate}
        \item
            \( \Kernel\theta\) est normal dans \( G\),
        \item
            \( \Image \theta\) est un sous-groupe de \( H\)
        \item   \label{ItemWLCLdk}
            nous avons un isomorphisme naturel
            \begin{equation}
                G/\Kernel\theta\simeq \Image\theta
            \end{equation}
    \end{enumerate}
\end{theorem}

\begin{proof}
    \begin{enumerate}
        \item
        \item
        \item
            Si \( [g]\) représente la classe de \( g\) dans \( G/\Kernel\theta\), l'isomorphisme est donné par \( \varphi[g]=\theta(g)\).
    \end{enumerate}
\end{proof}


\begin{theorem}[Deuxième théorème d'isomorphisme]
    Soient \( H\) et \( N\) deux sous-groupes de \( G\) et supposons que \( N\) soit normal. Alors
    \begin{enumerate}
        \item
            \( NH=HN\) est un sous-groupe
        \item
            \( N\normal NH\)
        \item
            \( N\cap H\normal H\)
        \item\label{ItemjRPajc}
            nous avons l'isomorphisme
            \begin{equation}
                \frac{ HN }{ N }\simeq\frac{ H }{ H\cap N }.
            \end{equation}
        \item   \label{ItembgDQEN}
            L'isomorphisme du point \ref{ItemjRPajc} est encore valable si \( N\) n'est pas normal mais si seulement \( H\) normalise \( N\), c'est à dire si \( hNh^{-1}\in N\) pour tout \( h\in H\).
    \end{enumerate}
\end{theorem}
%TODO : trouver une démonstration du dernier point.

\begin{proof}
    \begin{enumerate}
        \item
        \item
        \item
        \item
            Il faut d'abord remarquer que \( H\) et \( N\) étant des groupes et le produit \( NH\) étant un groupe, nous avons \( NH=HN\). Soit le morphisme injectif
            \begin{equation}
                \begin{aligned}
                    j\colon H&\to HN \\
                    h&\mapsto h
                \end{aligned}
            \end{equation}
            et la surjection canonique
            \begin{equation}
                \sigma\colon HN\to HN/N 
            \end{equation}
            Nous considérons ensuite l'application composée
            \begin{equation}
                \begin{aligned}
                    f\colon H&\to HN/N \\
                    h&\mapsto hN. 
                \end{aligned}
            \end{equation}
            L'application \( f\) est surjective parce que l'élément \( hnN\in HN/N\) est l'image de \( h\), étant donné que \( hnN=hN\).

            Le noyau de \( f\) est \( \Kernel f=H\cap N\). En effet si \( a\in H\cap N\), nous avons \( f(a)=\sigma(a)\in K\). Par conséquent \( H\cap N\subset \Kernel f\). D'autre part si \( h\in H\) vérifie \( h\in\Kernel f\), alors \( f(h)=hN=N\), ce qui est uniquement possible si \( h\in N\).

            Le premier théorème d'isomorphisme implique alors que \( H/\Kernel f\simeq \Image f\), c'est à dire
            \begin{equation}
                H/N\cap H\simeq HN/N.
            \end{equation}
    \end{enumerate}
\end{proof}

\begin{theorem}[Troisième théorème d'isomorphisme]  \label{ThoezgBep}
    Soient \( N\) et \( M\) deux sous-groupes normaux de \( G\) avec \( M\subset N\). Alors \( N/M\) est normal dans \( G/M\) et
    \begin{equation}
        (G/M)/(N/M)\simeq G/N.
    \end{equation}
\end{theorem}

\begin{proof}
    Afin de montrer que \( N/M\) est normal dans \( G/M\), nous considérons \( g\in G\), \( nM\in N/M\) et nous calculons
    \begin{equation}
        gnMg^{-1}=gng^{-1}\underbrace{gMg^{-1}}_{=M}=\underbrace{gng^{-1}}_{\in N}M\in N/M.
    \end{equation}

    Pour prouver l'isomorphisme nous considérons le morphisme
    \begin{equation}
        \begin{aligned}
            \varphi\colon G/M&\to G/N \\
            gM&\mapsto gN. 
        \end{aligned}
    \end{equation}
    C'est surjectif et le noyau est \( N/M\) parce que \( \varphi(gM)=N\) uniquement si \( g\in N\). Nous pouvons appliquer le premier théorème d'isomorphisme à \( \varphi\) en écrivant
    \begin{equation}
        (G/M)/\Kernel \varphi\simeq\Image \varphi,
    \end{equation}
    c'est à dire
    \begin{equation}
        (G/M)/(N/M)\simeq G/N.
    \end{equation}
\end{proof}

%+++++++++++++++++++++++++++++++++++++++++++++++++++++++++++++++++++++++++++++++++++++++++++++++++++++++++++++++++++++++++++
\section{Le groupe et anneau des entiers}
%+++++++++++++++++++++++++++++++++++++++++++++++++++++++++++++++++++++++++++++++++++++++++++++++++++++++++++++++++++++++++++

Certes \( \eZ\) est un groupe pour l'addition, mais c'est également un anneau parce que nous avons les deux opérations d'addition et de multiplication. Nous n'allons pas nous priver de cette belle structure juste parce que le titre du chapitre est «groupes».

\begin{definition}
    Soit \( G\), un groupe et \( A\) une partie de \( G\). Nous notons \( \gr(A)\)\nomenclature[R]{\( \gr\)}{groupe engendré} l'intersection de tous les sous-groupes de \( G\) contenant \( A\). C'est le plus petit (pour l'inclusion) groupe de \( G\) contenant \( A\). Si \( \gr(A)=G\), alors nous disons que \( A\) est une \defe{partie génératrice}{partie génératrice} le groupe \( G\).

    Un groupe est \defe{monogène}{monogène} si il a une partie génératrice réduite à un seul élément.

    Un élément \( a\in G\) est un \defe{générateur}{générateur} de \( G\) si tous les éléments de \( G\) s'écrivent sous la forme \( a^n\) pour un certain \( n\in\eZ\). Un groupe fini et monogène est dit \defe{cyclique}{cyclique!groupe}.
\end{definition}

\begin{example}
    Soit \( G\) le groupe des rotations d'angle \( 2k\pi/10\). La rotation d'angle \( \pi/2\) n'est pas génératrice parce qu'elle n'engendre que \( \pi/2\), \( \pi\),\( 3\pi/2\) et l'identité. La rotation d'angle \( 2\pi/10\) par contre est génératrice.
\end{example}


Nous notons \( \eN_{k}=\{ 0,1,\ldots, k \}\subset\eN\)\nomenclature[R]{\( \eN_k\)}{Un sous ensemble des naturels}.
 
\begin{proposition} \label{PropSsgpZestnZ}
    Une partie \( H\) du groupe \( \eZ\) est un sous-groupe si et seulement si il existe \( n\in\eN\) tel que \( H=n\eZ\).
\end{proposition}

\begin{proof}
    Soit \( H\neq\{ 0 \}\) un sous-groupe de \( \eZ\). L'ensemble \( H\cap\eN^*\) contient un élément minimum que nous notons \( n\). Nous avons certainement \( n\eZ\subset H\) parce que \( H\) est un groupe (donc \( n+n\) et \( -n\) sont dans \( H\) dès que \( n\) est dans \( H\)). Nous devons prouver que \( H\subset n\eZ\).

    Si \( x\in H\), il existe \( q\in\eZ\) et \( r\in\eN_{n-1}\) tels que \( x=nq+r\).

    \begin{probleme}
        Justification par la division euclidienne à venir.
    \end{probleme}
    Nous savons déjà que \( nq\in H\), donc \( r\in H\) parce que \( x\) est également dans \( H\). Mais nous avions décidé que \( n\) serait le plus petit naturel de \( H\cap\eN^*\). Par conséquent \( r=0\) et \( x=nq\in n\eZ\).

\end{proof}

Notons que si un sous-groupe \( H\) de \( \eZ\) est donné, alors le nombre \( n\) tel que \( H=n\eZ\) est unique. En effet si \( n\eZ=m\eZ\) nous avons que \( n\) divise \( m\) (parce que \( m\in m\eZ\subset n\eZ\)) et que \( m\) divise \( n\) parce que \( n\in m\eZ\). Par conséquent \( n=m\).

\begin{lemma}   \label{LemZhxMit}
    Soient \( G\) et \( H\) deux groupes monogènes de même ordre. Soient \( g\) un générateur de \( G\) et \( h\), un générateur de \( H\). Il existe un isomorphisme de \( G\) sur \( H\) qui envoie \( g\) sur \( h\).
\end{lemma}

%---------------------------------------------------------------------------------------------------------------------------
\subsection{Division euclidienne}
%---------------------------------------------------------------------------------------------------------------------------

\begin{theorem}[Division euclidienne]     \label{ThoDivisEuclide}
    Soient \( a\in\eZ\) et \( b\in\eN^*\). Il existe un unique couple \( (q,r)\in\eZ\times\eN\) tel que
    \begin{equation}
        a=bq+r
    \end{equation}
    avec \( 0\leq r<b\).
\end{theorem}

L'opération \( (a,b)\mapsto(a,r)\) donnée par le théorème \ref{ThoDivisEuclide} est la \defe{division euclidienne}{division!euclidienne}. Le nombre \( q\) est le \defe{quotient}{quotient} et \( r\) est le \defe{reste}{reste} de la division de \( a\) par \( b\).

%---------------------------------------------------------------------------------------------------------------------------
\subsection{PGCD, PPCM et Bézout}
%---------------------------------------------------------------------------------------------------------------------------

Pour des versions plus générales, nous avons Bézout dans un anneau principal au théorème \ref{CorimHyXy} et les définitions de PGCD et PPCM dans un anneau intègre à la définition \ref{DefrYwbct}. Voir également le théorème de Bézout pour les polynômes, théorème \ref{ThoBezoutOuGmLB}.

Soient \( p,q\in\eZ^*\). Les ensembles \( p\eZ\cap q\eZ\) et \( p\eZ+q\eZ\) sont des sous-groupes de \( \eZ\). Par conséquent il existe des entiers \( \ppcm(p,q)\) et \( \pgcd(p,q)\) tels que
\begin{subequations}
    \begin{align}
        p\eZ\cap q\eZ&=\ppcm(p,q)\eZ\\
        p\eZ + q\eZ&=\pgcd(p,q)\eZ.
    \end{align}
\end{subequations}

Si \( \pgcd(p,q)=1\), nous disons que \( p\) et \( q\) sont \defe{premiers entre eux}{premier!deux nombres entre eux}.

\begin{theorem}[Théorème de Bézout\cite{LSAmvR}] \label{ThoBuNjam}   \index{Bézout!nombres entiers}
    Deux entiers non nuls \( a,b\in\eZ^*\) sont premiers entre eux si et seulement si il existe \( u,v\in\eZ\) tels que 
    \begin{equation}
        au+bv=1
    \end{equation}
\end{theorem}

\begin{proof}
    Soit \( d=\pgcd(a,b)\) et des nombres \( u,v\) tels que \( au+bv=1\). Le PGCD \( d\) divise à la fois \( a\) et \( b\), et donc divise \( au+bv\). Nous en déduisons que \( d\) divise \( 1\) et est par conséquent égal à \( 1\).

    Nous supposons maintenant que \( \pgcd(a,b)=1\) et nous considérons l'ensemble
    \begin{equation}
        E=\{ au+bv\tq u,v\in \eZ \}\cap \eN^*.
    \end{equation}
    C'est à dire l'ensemble des nombres strictement positifs pouvant s'écrire sous la forme \( au+bv\). Cet ensemble est non vide parce qu'il contient par exemple soit \( a\) soit \( -a\). Soit \( m\) le plus petit élément de \( E\) et écrivons
    \begin{equation}    \label{EqMBsfrP}
        m=au_1+bv_1.
    \end{equation}
    Écrivons la division euclidienne de \( a\) par \( m\), c'est à dire
    \begin{equation}
        a=mq+r
    \end{equation}
    avec \( 0\leq r<m\). En remplaçant \( m\) par sa valeur \eqref{EqMBsfrP}, \( a=(au_1+bv_1)q+r\) et 
    \begin{equation}
        r=a(1-u_1q)-bv_1q,
    \end{equation}
    c'est à dire que \( r\in \eZ a+\eZ b\) en même temps que \( 0\leq r<m\), donc \( r=0\) par minimalité de \( m\). Donc \( a\) est divisible par \( m\). 

    De la même façon nous prouvons que \( b\) est divisible par \( m\). Vu que \( m\) divise à la fois \( a\) et \( b\) nous avons \( m=1\).
\end{proof}

Nous utiliserons le théorème de Bézout en parlant des racines de l'unité et des générateurs du groupe unitaire dans le lemme \ref{LemcFTNMa}. Au passage nous y parlerons de solfège.

\begin{corollary}       \label{CorgEMtLj}
    Soient \( p\) et \( q\) deux entiers premiers entre eux. Nous avons
    \begin{equation}
        p\eZ+q\eZ=\eZ.
    \end{equation}
\end{corollary}

Notons que l'application \( p\eZ+q\eZ\) vers \( \eZ\) n'est évidemment pas injective.

\begin{proof}
    Soit \( x\in \eZ\). Le théorème de Bézout nous donne \( k\) et \( l\) tels que \( kp+lq=1\). Du coup, \( (xk)p+(xl)q=x\).
\end{proof}

La proposition suivante établit que si \( x\) est assez grand, alors il peut même être écrit comme une combinaison de \( p\) et \( q\) à coefficients positifs. Elle sera utilisée pour démontrer que les états apériodiques d'une chaîne de Markov peuvent être atteins à tout moments (assez grand), voir la définition \ref{DefCxvOaT} et ce qui suit.
\begin{proposition}     \label{PropLAbRSE}
    Soient \( a\) et \( b\) deux éléments de \( \eN\) premiers entre eux. Il existe \( N>0\) tel que tout \( x>N\) appartient à \( a\eN+b\eN\).
\end{proposition}

\begin{proof}
    Soient \( a\) et \( b\), premiers entre eux, et \( x\in \eN\). Soit \( p\) positif tel que
    \begin{subequations}
        \begin{numcases}{}
            pa+pb\geq x\\
            p(a+b)-x\leq a+b.
        \end{numcases}
    \end{subequations}
    C'est à dire que \( p(a+b)\) est le multiple supérieur de \( a+b\) le plus proche de \( a+b\). Nous avons alors
    \begin{equation}
        p(a+b)-ra-sb=x
    \end{equation}
    où \( r\) et \( s\) sont donnés par le théorème de Bézout. Il s'agit maintenant de savoir si nous pouvons être assuré d'avoir \( p>r\) et \( q>s\) dès que \( x\) est assez grand. Pour cela nous considérons les nombres \( r_i\) et \( s_i\) définis par
    \begin{equation}
        r_ia+s_ib=i
    \end{equation}
    pour \( i=1,\ldots, a+b\). Nous posons \( r^*=\max\{ r_i \}\), \( s^*=\max\{ s_i   \}\), et \( p^*=\max\{ r^*,s^* \}\). Maintenant si \( x>p^*(a+b)\), alors
    \begin{equation}
        x=p(a+b)-r_ka-s_kb
    \end{equation}
    où \( k=x-p(a+b)\).
\end{proof}


%\begin{proof}
    %Soit \( x\in \eN\) et \( k_1,l_1\in \eN\) les plus petits entiers tels que \( k_1p\geq x/2\) et \( l_1q\geq x/2\). Nous avons alors
    %\begin{equation}
        %x\leq k_1p+l_1q<x+(p+q).
    %\end{equation}
    %Nous posons \( \delta=k_1p+l_1q-x\).
   % 
    %Soient des entiers \( a_i,b_i\) tels que \( a_ip+b_iq=i\). Nous notons
    %\begin{subequations}
        %\begin{align}
            %A=\max\{ a_i\tq i=1,\ldots, k+p \}\\
            %B=\max\{ b_i\tq i=1,\ldots, k+p \}
        %\end{align}
    %\end{subequations}
    %Notons que \( A\) et \( B\) sont donnés uniquement en termes de \( p\) et \( q\). Ils ne sont en aucun cas dépendants de \( x\).
   % 
    %Nous avons
    %\begin{equation}
        %x=k_1p+lq-\delta=(k_1-a_{\delta})p+(l_1+b_{\delta})q
    %\end{equation}
    %avec \( k_1-a_{\delta}\geq k_1-A\) et \( l_1-b_{\delta}\geq l_1-B\). Si \( x\) est suffisamment grand pour avoir \( k_1>A\) et \( l_1>B\), alors la décomposition souhaitée est trouvée.  
%
    %Une borne pour \( x\) est donnée par 
    %\begin{equation}    \label{EqjQpURG}
        %x>\max\{ 2pA,2qB \}.
    %\end{equation}
%\end{proof}

\begin{example}
    Écrivons \( 1000=a\cdot 7+b\cdot 5\) avec \( a,b\in \eN\). D'abord \( 72\cdot 7=504\) et \( 100\cdot 5=500\). Nous avons donc 
    \begin{equation}
        1004=72\cdot 7+100\cdot 5.
    \end{equation}
    Ensuite \( 4=25-21=-3\cdot 7+5\cdot 5\). Au final,
    \begin{equation}
        1000=75\cdot 7+95\cdot 5.
    \end{equation}
\end{example}

\begin{proposition}
    soient \( a,b\in\eZ^*\). Si
    \begin{equation}
        \begin{aligned}[]
            a&=\epsilon\prod_{\text{\( p\) premiers}}p^{\mu(p)}&b&=\epsilon'\prod_{\text{\( p\) premier}}p^{\nu(p)},
        \end{aligned}
    \end{equation}
    alors
    \begin{subequations}
        \begin{align}
            \pgcd(a,b)&=\prod p^{\min\{ \mu(p),\nu(p) \}}\\
            \ppcm(a,b)&=\prod p^{\max\{ \mu(p),\nu(p) \}}
        \end{align}
    \end{subequations}    
\end{proposition}

Cette proposition implique que \( m\leq p^n\) et \( \pgcd(m,p^n)\) si et seulement si \( m=qp\) avec \( q\leq p^{n-1}\).

\begin{lemma}[\href{http://ljk.imag.fr/membres/Bernard.Ycart/mel/ar/node6.html}{lemme de Gauss}]  \index{lemme!Gauss!entiers}\index{Gauss!lemme!entiers}      \label{LemSdnZNX}
    Soient \( a,b,c\in \eN^*\) tels que \( a\) divise \( bc\). Si \( a\) est premier avec \( c\), alors \( a\) divise \( b\).
\end{lemma}
Il y aura aussi un lemme de Gauss à propos de polynômes (lemme \ref{LemEfdkZw}).

\begin{proof}
    Vu que \( a\) est premier avec \( c\), nous avons \( \pgcd(a,c)=1\) et Bézout (\ref{ThoBuNjam}) nous donne donc \( s,t\in \eZ\) tels que \( sa+tc=1\). En multipliant par \( b\),
    \begin{equation}
        sab+tbc=b.
    \end{equation}
    Mais les deux termes du membre de gauche sont multiples de \( a\) parce que \( a\) divise \( bc\). Par conséquent \( b\) est somme de deux multiples de \( a\) et donc est multiple de \( a\).
\end{proof}

%---------------------------------------------------------------------------------------------------------------------------
\subsection{Calcul effectif du PPCM et de Bézout}
%---------------------------------------------------------------------------------------------------------------------------
\label{subSecIpmnhO}

Source : \cite{BezoutCos}.

Soient \( A\) et \( B\), deux entier disons positifs. Nous allons voir maintenant l'algorithme de \defe{Euclide étendu}{Euclide!algorithme étendu} qui est capable, pour \( A\) et \( B\) donnés, de calculer le \( \pgcd(A,B)\) et un couple de Bézout \( (u,v)\) tel que \( uA+vB=\pgcd(A,B)\). Ce calcul est indispensable si on veut implémenter RSA (\ref{subSecEVaFYi}).

Cela se base sur le lemme suivant.

\begin{lemma}       \label{LemiVqita}
    Soient \( A,B\in \eN\) et leur division euclidienne \( A=qB+r\) avec \( r<B\). Alors \( \pgcd(A,B)=\pgcd(r,B)\).
\end{lemma}

\begin{proof}
    Il suffit de voir que les diviseurs communs de \( A\) et \( B\) sont diviseurs de \( r\) et que les diviseurs communs de \( r\) et \( B\) divisent \( A\).

    Si \( s\) divise \( A\) et \( B\), alors dans l'équation \( \frac{ A }{ s }=\frac{ qB }{ s }+\frac{ r }{ s }\), les termes \( A/s\) et \( qB/s\) sont entiers, donc \( r/s\) doit aussi être entier.

    Inversement, si \( s\) divise \( r\) et \( B\), alors il divise \( qB+r\) et donc \( A\).
\end{proof}

L'algorithme pour calculer \( \pgcd(A,B)\) consiste à écrire la division euclidienne de \( A\) par \( B\) puis celle de \( B\) par \( r\) :
\begin{subequations}
    \begin{align}
        A&=qB+r&&r<B\\
        B&=q'r+r'&&r'<r
    \end{align}
\end{subequations}
et donc \( \pgcd(A,B)=\pgcd(B,r)=\pgcd(r,r')\). Étant donné que les inégalités \( r<B\) et \( r'<r\) sont strictes, en continuant ainsi nous finissons sur zéro, c'est à dire
\begin{equation}
    r_{n-1}=q_nr_n,
\end{equation}
et à ce moment nous avons \( \pgcd(A,B)=\pgcd(r_{n-1},r_n)=r_n\).

Écrivons cela en détail (parce que Bézout, ça va être le même chose en cinq fois plus compliqué). On pose
\begin{subequations}
    \begin{align}
        r_0=A\\
        r_1=B.
    \end{align}
\end{subequations}
Ensuite on écrit la division euclidienne \( A=q_1B+r_2\), c'est à dire \( r_0=q_1r_1+r_2\). Cela donne \( r_2\) et \( q_1\) en termes de \( r_0\) et \( r_1\) :
\begin{equation}
    r_2=r_0-q_1r_1.
\end{equation}
Ensuite, sachant \( r_2\) nous pouvons continuer :
\begin{equation}
    r_1=q_2r_2+r_3
\end{equation}
donne \( q_2\) et \( r_3=r_1-q_2r_2\). On continue avec \( r_2=q_3r_3+r_4\). La suite \( (r_k)\) ainsi construite est strictement décroissante et à chaque étape nous avons
\begin{equation}
    \pgcd(r_k,r_{k+1})=\pgcd(r_{k+1},r_{k+2})=\pgcd(A,B).
\end{equation}

Lançons nous maintenant dans Bézout. Nous posons \( r_0=A\), \( r_1=B\) puis nous définissons les suites nombres \( r_k\) et \( q_k\) par la relation de récurrence
\begin{equation}
    r_{k-1}=q_kr_k+r_{k+1}.
\end{equation}
À chaque étape le lemme \ref{LemiVqita} et le principe de l'algorithme d'Euclide nous donne
\begin{subequations}
    \begin{numcases}{}
        \pgcd(r_k,r_{k-1})=\pgcd(r_k,r_{k+1})=\pgcd(A,B)\\
        0\leq r_{k+1}<r_k.
    \end{numcases}
\end{subequations}
La suite étant strictement décroissante, nous prenons \( r_n\), le dernier non nul : \( r_{n+1}=0\). Dans ce cas la dernière équation sera
\begin{equation}
    r_{n-1}=q_nr_n
\end{equation}
avec \( \pgcd(A,B)=\pgcd(r_n,r_{n-1})=r_n\). La difficulté est de construire la suite qui donne Bézout. Elle va être construite à l'envers. Nous voulons trouver les couples \( (u_k,v_k)\) de telle façon à avoir à chaque étape
\begin{equation}
    r_n=u_kr_k+v_kr_{k-1}.
\end{equation}
Notons que c'est à chaque fois \( r_n\) que nous construisons. La première équation de type Bézout à résoudre est 
\begin{equation}
    r_n=u_nr_n+v_nr_{n-1},
\end{equation}
sachant que \( r_{n-1}=q_nr_n\). On pose \( v_n=0\) et \( u_n=1\) et c'est bon. Pour la récurrence, nous égalisons les deux expressions pour \( r_n\) :
\begin{equation}
    r_n=u_kr_k+v_kr_{k-1}=u_{k-1}r_{k-1}=v_{k-1}r_{k-2}
\end{equation}
dans laquelle nous substituons \( r_{k-2}=q_{k-1}r_{k-1}+r_k\) et nous égalisons les coefficients de \( r_k\) et \( r_{k-1}\) :
\begin{equation}
    u_kr_k+v_kr_{k-1}=u_{k-1}r_{k-1}+v_{k-1}(q_{k-1}r_{k-1}+r_k),
\end{equation}
cela donne
\begin{subequations}
    \begin{numcases}{}
        v_{k-1}=u_k\\
        u_{k-1}=v_k-v_{k-1}q_{k-1}.
    \end{numcases}
\end{subequations}
Dès que \( u_k\) et \( v_k\) ainsi que \( q_{k-1}\) sont connus, on peut calculer \( u_{k-1}\) et \( v_{k-1}\). 

La dernière équation, celle avec \( k=1\), est
\begin{equation}
    r_n=u_1r_1+v_1r_0,
\end{equation}
c'est à dire
\begin{equation}
    \pgcd(A,B)=u_1B+v_1A.
\end{equation}
Nous avons ainsi résolu Bézout.

%---------------------------------------------------------------------------------------------------------------------------
\subsection{Quotients}
%---------------------------------------------------------------------------------------------------------------------------

Nous notons \( \eZ_n=\eZ/n\eZ\)\nomenclature[R]{\( \eZ_n\)}{groupe des entiers modulo \( n\)} le groupe des entiers modulo \( n\).

Nous écrivons \( a=b\mod p\) essentiellement si il existe \( k\in \eZ\) tel que \( b+kp=a\). Plus généralement nous notons \( [a]_p=\{ a+kp|k\in \eZ \}\)\nomenclature[R]{\( [a]_p\)}{ensemble des \( a+kp\)} et l'écriture «\( a=n\mod p\)» peut tout autant signifier \( a=[b]_p\) que \( a\in [b]_p\). La différence entre les deux est subtile mais peut de temps en temps valoir son pesant d'or.


\begin{proposition}
    Soit \( n\in\eN\). Le groupe \( \eZ_n\) est monogène. Si \( n\neq 0\), le groupe \( \eZ_n\) est cyclique d'ordre \( n\).
\end{proposition}

\begin{proof}
    Nous considérons la surjection canonique \( \mu\colon \eZ\to \eZ_n\). Si \( a\in\eZ\), alors \( \mu(a)=a\mu(1)\). Par conséquent \( \eZ_n=\gr\big( \mu(1) \big)\) parce que tout groupe contenant \( \mu(1)\) contient tous les multiples de \( \mu(1)\), et par conséquent contient \( \mu(\eZ)=\eZ_n\).

    Soit \( x\in\eZ_n\) et considérons \( x_0\), le plus petit naturel représentant \( x\). Nous notons \( x=[x_0]\). Le théorème de la division euclidienne \ref{ThoDivisEuclide} donne l'existence de \( q\) et \( r\) avec \( 0\leq r<n\) et \( q\geq 0\) tels que
    \begin{equation}
        x_0=nq+r.
    \end{equation}
    Nous avons \( [x_0]=[r]=\mu(r)\) parce que \( x_0-r\) est un multiple de \( n\). Nous avons donc \( [x_0]\in\mu(\eN_{n-1})\). Par conséquent
    \begin{equation}
        \eZ_n=\mu(\eZ)=\mu(\eN_{n-1}).
    \end{equation}
    La restriction \( \mu\colon \eN_{n-1}\to \eZ_n\) est donc surjective. Montrons qu'elle est également injective. Si \( \mu(x_0)=\mu(x_1)\), alors \( x_1=x_0+kn\). Si nous supposons que \( x_1>x_0\), alors \( k>0\) et si \( x_0\in\eN_{n-1}\), alors \( x_1>n-1\).

    L'ordre de \( \eZ_n\) est donc le même que le cardinal de \( \eN_{n-1}\), c'est à dire \( n\). Le groupe \( \eZ_n\) est donc fini, d'ordre \( n\) et monogène (parce que \( \eZ_n=\gr(\mu(1))\)). Il est donc cyclique.
\end{proof}

%+++++++++++++++++++++++++++++++++++++++++++++++++++++++++++++++++++++++++++++++++++++++++++++++++++++++++++++++++++++++++++
\section{Indice d'un sous-groupe et ordre des éléments}
%+++++++++++++++++++++++++++++++++++++++++++++++++++++++++++++++++++++++++++++++++++++++++++++++++++++++++++++++++++++++++++

Soit \( G\) un groupe fini et \( H\), un sous-groupe. L'\defe{indice}{indice} de \( H\) dans \( G\) est le nombre \( | G |/| H |\), souvent noté \( | G:H |\). Le théorème de Lagrange dira en particulier que l'indice est toujours un nombre entier.

\begin{theorem}[Théorème de Lagrange]\index{théorème!Lagrange}      \label{ThoLagrange}
    Soit \( H\) un sous-groupe du groupe fini \( G\).  Alors
    \begin{enumerate}
        \item
    L'ordre de \( H\) divise l'ordre de \( G\).
\item 
    Les trois nombres suivants sont égaux :
    \begin{itemize}
        \item
            le nombre de classes de \( H\) à gauche,
        \item
            le nombre de classes de \( H\) à droite,
        \item
            l'indice de \( H\) dans \( G\).
    \end{itemize}
    \end{enumerate}
    En particulier si \( H\) est distingué dans \( G\) nous avons
    \begin{equation}
        | G/H |=\frac{ | G | }{ | H | }.
    \end{equation}
\end{theorem}

\begin{proof}
    Nous commençons par montrer que les classes de \( H\) ont toutes les même nombre d'éléments que \( H\). En effet pour chaque \( g\in G\) nous avons la bijection
    \begin{equation}
        \begin{aligned}
            \varphi\colon H&\to gH \\
            h&\mapsto gh. 
        \end{aligned}
    \end{equation}
    L'injectivité de \( \varphi\) est le fait que \( gh=gh'\) implique \( h=h'\). La surjectivité est par définition de la classe. 

    Les classes à gauche formant une partition de \( G\), le cardinal de \( G\) est le produit de la taille des classes par le nombre de classes :
    \begin{equation}
        | G |=| H |\cdot\text{nombre de classes}.
    \end{equation}
    En particulier nous voyons que \( | H |\) divise \( | G |\).

    La dernière formule exprime simplement que \( G/H\) est par définition le nombre de classes de \( H\) à gauche (ou à droite) dans \( G\).
\end{proof}

\begin{corollary}       \label{CorpZItFX}
    L'ordre d'un élément d'un groupe fini divise l'ordre du groupe. En particulier dans un groupe d'ordre \( n\) tous les éléments vérifient \( q^n=e\).
\end{corollary}

\begin{proof}
    Soit \( G\) un groupe fini et considérons le sous-groupe
    \begin{equation}
        H=\{ g^k\tq k\in\eN \}.
    \end{equation}
    Par le théorème de Lagrange, l'ordre de \( H\) divise \( | G |\), mais l'ordre de \( H\) est le plus petit \( k\) tel que \( g^k=e\), c'est à dire l'ordre de \( g\).
\end{proof}

Le lemme suivant indique que sous hypothèse de commutativité, l'ordre d'un élément est une notion multiplicative.
\begin{lemma}[\cite{rqrNyg}]    \label{LemyETtdy}
    Soit \( G\) un groupe et \( a,b\in G\) tels que \( ab=ba\) d'ordres respectivement \( r\) et \( s\), deux nombres premiers entre eux. Alors l'élément \( ab\) est d'ordre \( rs\).
\end{lemma}

\begin{proof}
    Étant donné que \( (ab)^{rs}=a^{rs}b^{rs}=1\), l'ordre de \( ab\) divise \( rs\). Et vu que \( r\) et \( s\) sont premiers entre eux, l'ordre de \( ab\) s'écrit sous la forme \( r_1s_1\) avec \( r_1\divides r\) et \( s_1\divides s\). Nous avons
    \begin{equation}
        a^{r_1s_1}b^{r_1s_1}=(ab)^{r_1s_1}=1,
    \end{equation}
    que nous élevons à la puissance \( r_2\) où \( r_2\) est définit en posant \(r=r_1r_2\) :
    \begin{equation}
        a^{rs_1}b^{rs_1}=1.
    \end{equation}
    Et comme \( a^{rs_1}=1\), nous concluons que \( b^{rs_1}=1\). Donc \( s\divides rs_1\). Par le lemme de Gauss (\ref{LemSdnZNX}), nous avons en fait \( s\divides s_1\). Vu qu'on a aussi \( s_1\divides s\), nous avons \( s=s_1\).

    Le même type d'argument donne \( r=r_1\), et finalement l'ordre de \( ab\) est \( r_1s_1=rs\).
\end{proof}

\begin{lemma}       \label{LemqAUBYn}
    L'ensemble des ordres d'un groupe \emph{commutatif} est stable par PPCM.

    Autrement dit, si \( x\in G\) est d'ordre \( r\) et si \( y\in G\) est d'ordre \( s\), alors il existe un élément d'ordre \( \ppcm(r,s)\).
\end{lemma}

\begin{proof}
    Soit \( m=\ppcm(r,s)\). Afin d'écrire \( m\) sous une forme pratique, nous considérons les décompositions en facteurs premiers de \( r\) et \( s\) :
    \begin{subequations}
        \begin{align}
            r&=\prod_{i=1}^kp_i^{\alpha_i}\\
            s&=\prod_{i=1}^kp_i^{\beta_i}
        \end{align}
    \end{subequations}
    où \( \{ p_i \}_{i=1\ldots k}\) est l'ensemble des nombres premiers arrivant dans les décompositions de \( r\) et de \( s\). Si nous posons
    \begin{subequations}
        \begin{align}
            r'&=\prod_{\substack{i=1\\\alpha_1>\beta_i}}^kp_i^{\alpha_i}\\
            s'&=\prod_{\substack{i=1\\a_i\leq \beta_i}}^kp_i^{\beta_i},
        \end{align}
    \end{subequations}
    alors \( \ppcm(r,s)=r's'\) et \( r'\) et \( s'\) sont premiers entre eux. L'élément \( x^{r/r'}\) est d'ordre \( r'\) et l'élément \( y^{s/s'}\) est d'ordre \( s'\). Maintenant nous utilisons le fait que \( G\) soit commutatif et le lemme \ref{LemyETtdy} pour conclure que l'ordre de \( x^{r/r'}y^{s/s'}\) est \( r's'=m\).
\end{proof}

\begin{lemma}[\cite{Combes}]    \label{LemSkIOOG}
    Un sous-groupe d'indice \( 2\) est un sous-groupe normal.
\end{lemma}

\begin{lemma}[\cite{NielsBMorph}]\label{PropubeiGX}
    Soit \( H\), un sous-groupe normal d'indice \( m\) de \( G\). Alors pour tout \( a\in G\) nous avons \( a^m\in H\).
\end{lemma}

\begin{proof}
    Par définition de l'indice, le groupe \( G/H\) est d'ordre \( m\), de telle façon à ce que si \( [a]\in G/H\), nous avons \( [a]^m=[e]\), ce qui signifie \( [a^m]=[e]\), ou encore \( a^m\in H\).
\end{proof}

\begin{proposition}[\cite{NielsBMorph}]
    Soit un groupe fini \( G\) et \( H\), un sous-groupe normal d'ordre \( n\) et d'indice \( m\) avec \( m\) et \( n\) premiers entre eux. Alors \( H\) est l'unique sous-groupe de \( G\) à être d'ordre \( n\).
\end{proposition}
Notons que cette proposition ne dit pas qu'il existe un sous-groupe d'ordre \( n\) et d'indice \( m\). Il dit juste que si il y en a un et si il est normal, alors il n'y en a pas d'autres.

\begin{proof}
    Soit \( H'\) un sous-groupe d'ordre \( n\). Si \( h\in H'\) alors \( h^n=1\) et \( h^m\in H\) (lemme \ref{PropubeiGX}). Étant donné que \( m\) et \( n\) sont premiers entre eux, il existe \( a,b\in \eZ\) tels que (Bézout, théorème \ref{ThoBuNjam})
    \begin{equation}
        am+bn=1.
    \end{equation}
    Du coup \( h=h^1=(h^m)^a(h^n)^b\). En tenant compte du fait que \( a^n=1\) et \( h^m\in H\), nous avons \( h\in H\). Ce que nous venons de prouver est que \( H'\subset H\) et donc que \( H=H'\) parce que \( | H' |=| H |=| G |/m\).
\end{proof}

%---------------------------------------------------------------------------------------------------------------------------
\subsection{Suite de composition}
%---------------------------------------------------------------------------------------------------------------------------

%TODO : citer la page de la wikiversité sur Jordan-Hölder.
%TODO : donner la définition d'un raffinement de suite de composition.

Sources : \cite{NjCCfW,jxWKGB}.

\begin{definition}
Soit \( G\) un groupe. Une \defe{suite de composition}{composition!suite de}\index{suite!de composition} pour \( G\) est une suite finie de sous-groupes \( (G_i)_{i=0,\ldots, n}\) telle que
\begin{equation}
    \{ e \}=G_n\subseteq G_{n-1}\subseteq\ldots\subseteq G_1\subseteq G_0=G
\end{equation}
et telle que \( G_{i+1}\) est normal\footnote{Nous rappelons au cas où que «normal» signifie «distingué».} dans \( G_i\). Les groupes \( G_i/G_{i+1}\) sont les \defe{quotients}{quotient!dans une suite de composition} de la suite de composition.

    Une suite de \defe{Jordan-Hölder}{suite!de Jordan-Hölder}\index{Jordan-Hölder} est une suite de composition dont tous les quotients sont simples.
\end{definition}
L'objet de nos prochaines pérégrinations mathématiques est de montrer que tout groupe fini admet une suite de Jordan-Hölder (théorème \ref{ThoLgxWIC}).

\begin{lemma}[du papillon\cite{NjCCfW}]\label{LemsKpXCG}
    Soit \( G\) un groupe et des sous-groupes \( A\) et \( B\). Soit \( A'\) normal dans \( A\) et \( B'\) normal dans \( B\). Alors
    \begin{enumerate}
        \item
            \( A'(A\cap B')\) est normal dans \( A'(A\cap B)\)
        \item
            \( (A'\cap B)B'\) est normal dans \( (A\cap B)B'\)
        \item
            Nous avons les isomorphismes de groupes
            \begin{equation}
                \frac{ A'(A\cap B) }{ A'(A\cap B') }\simeq\frac{ (A\cap B)B' }{ (A'\cap B)B' }\simeq\frac{ B'(A\cap B) }{ B'(A'\cap B) }.
            \end{equation}
    \end{enumerate}
\end{lemma}

\begin{proof}
    Nous n'allons pas démontrer chacun des points; nous renvoyons au fameux «preuve très similaire dans les autres cas» pour plus de justifications.

    Commençons par montrer que \( A'(A\cap B')\) est un groupe. Si \( a,b\in A'\) et \( x,y\in A\cap B'\),
    \begin{equation}
        axby=xx^{-1}axbx^{-1}xy
    \end{equation}
    En utilisant la normalité, \( x^{-1}ax\in A'\), donc \( xx^{-1}axbx^{-1}\in A'\) et donc le tout est dans \( A'(A\cap B')\). L'ensemble \( A'(A\cap B')\) est également stable pour l'inverse parce que
    \begin{equation}
        x^{-1}a^{-1}=\underbrace{x^{-1}a^{-1}x}_{\in A'}x^{-1}.
    \end{equation}
    
    Nous montrons maintenant que \( A'(A\cap B')\) est normal dans \( A'(A\cap B)\). Soient \( a,b\in A'\), \( x\in A\cap B'\) et \( f\in A\cap B\). Alors
    \begin{subequations}
        \begin{align}
        (bf)^{-1}(ax)(bf)&=(bf)^{-1}(a\underbrace{xbx^{-1}}_{=c\in A'}xf)\\
        &=f^{-1}b^{-1}acxf\\
        &=f^{-1}b^{-1}acf\underbrace{f^{-1}xf}_{=y\in A\cap B'}\\
        &=\underbrace{f^{-1}b^{-1}acf}_{\in A'}y\\
        &\in A'(A\cap B').
        \end{align}
    \end{subequations}
    
    Pour prouver l'isomorphisme
    \begin{equation}
        \frac{ A'(A\cap B) }{ A'(A\cap B') }=\frac{ (A\cap B)B' }{ (A'\cap B)B' },
    \end{equation}
    nous allons utiliser le deuxième théorème d'isomorphisme (\ref{ThoezgBep}\ref{ItembgDQEN}). Que nous appliquons à \( H=A\cap B\) et \( N=A'(A\cap B')\). La vérification que \( H\) normalise \( N\) est usuelle. Nous commençons par écrire
    \begin{equation}    \label{EqkphNsE}
        \frac{ A'(A\cap B')(A\cap B) }{ A'(A\cap B') }\simeq\frac{ A\cap B }{ A\cap B\cap A'(A\cap B') }.
    \end{equation}
    Pour simplifier un peu cette expression nous prouvons d'abord que
    \begin{equation}    \label{EqkhsyNh}
        (A\cap B)\cap A'(A\cap B')=(A'\cap B)(A\cap B').
    \end{equation}
    L'inclusion \( \supset\) est facile. Pour l'autre sens, étant donné que \( A'(A\cap B')\subset A\) nous avons
    \begin{equation}
        A\cap B\cap A'(A\cap B)=B\cap A'(A\cap B).
    \end{equation}
    Un élément de \( B\cap A'(A\cap B)\) est un élément de \(   B\) qui s'écrit sous la forme \( s=ax\) avec \( a\in A'\) et \( x\in A\cap B'\). Nous avons alors \( a=sx^{-1}\) avec \( s\in B\) et \( x^{-1} \in B'\). Par conséquent \( a\in B\) et donc \( a\in A'\cap B\). Nous avons donc
    \begin{equation}
        (A\cap B)\cap A'(A\cap B')=B\cap A'(A\cap B)\subset (A'\cap B)(A\cap B'),
    \end{equation}
    et donc l'égalité \eqref{EqkhsyNh}. Toujours dans l'idée de simplifier \eqref{EqkphNsE} nous remarquons que \( A\cap B'\) est un sous-ensemble de \( A\ca B'\), donc \( A'(A\cap B')(A\cap B)=A'(A\cap B)\). Il reste donc
    \begin{equation}
        \frac{ A'(A\cap B) }{ A'(A\cap B') }=\frac{ A\cap B }{ (A'\cap B)(A\cap B') }.
    \end{equation}
    Étant donné que les hypothèses sur \( A\) et \( B\) sont symétriques, le membre de droite peut aussi s'écrire en inversant \( A\) et \( B\). Nous en sommes à
    \begin{equation}
        \frac{ B'(A\cap B) }{ B'(A'\cap B) }=\frac{ A'(A\cap B) }{ A'(A\cap B') }.
    \end{equation}
    Nous devons encore justifier \( B'(A\cap B)=(A\cap B)B'\) et \( B'(A'\cap B)=(A'\cap B)B'\). Faisons le premier et laissons le second \href{http://abstrusegoose.com/395}{au lecteur}.
    Si \( b\in B'\) et \( x\in A\cap B\), alors
    \begin{equation}
        bx=x\underbrace{x^{-1}bx}_{\in B'}\in (A\cap B)B'.
    \end{equation}
\end{proof}

\begin{proposition}
    Si \( G\) est un groupe fini et que \( (G_i)\) est une suite de composition pour \( G\), alors l'ordre de \( G\) est le produit des ordres de ses quotients.
\end{proposition}

\begin{proof}
    Étant donné que \( G_{i+1}\) est toujours normal dans \( G_i\), le théorème de Lagrange (\ref{ThoLagrange}) s'applique et nous avons à chaque pas de la suite de composition nous avons
    \begin{equation}
        | \frac{ G_i }{ G_{i+1} } |=\frac{ | G_i | }{ | G_{i+1} | } 
    \end{equation}
    et il suffit d'écrire \( | G |\) de façon télescopique :
    \begin{equation}
        | G |=\prod_{0\leq i\leq n-1}\frac{ | G_i | }{ | G_{i+1} | }
    \end{equation}
\end{proof}

Nous disons que les deux suites de composition \( (G_i)_{0\leq i\leq r}\) et \( (G_j)_{0\leq j\leq s}\) sont \defe{équivalentes}{équivalence!suite de composition} si \( r=s\) et si il existe une permutation \( \sigma\in S_{r-1}\) telle que
\begin{equation}
    \frac{ G_i }{ G_{i+1} }\simeq\frac{ H_{\sigma(i)} }{ H_{\sigma(i)+1} }.
\end{equation}

\begin{proposition}[Schreider]\index{lemme!de Schreider}
    Deux suites de composition d'un même groupe admettent des raffinements équivalents.
\end{proposition}

\begin{proof}
    Soient les suites de composition
    \begin{subequations}
        \begin{align}
            \{ e \}=G_m\subseteq\ldots\subseteq G_1\subseteq G_0=G\\
            \{ e \}=H_m\subseteq\ldots\subseteq H_1\subseteq H_0=G
        \end{align}
    \end{subequations}
    Nous raffinons la suite \( (G_i)\) en remplaçant \( G_{i+1}\subseteq G_i\) par
    \begin{equation}
        G_{i+1}=G_{i+1}(G_i\cap H_n)\subset G_{i+1}(G_i\cap H_{n-1})\subseteq\ldots\subseteq G_{i+1}(G_i\cap H_0)=G_i,
    \end{equation}
    et de même pour \( (H_j)\). Le groupe \( G_{i+1}(G_i\cap H_k)\) est normal dans \( G_{i+1}(G_i\cap H_{k-1})\) parce que \( G_{i+1}\) étant normal dans \( G_i\) et \( H_k\) dans \( H_{k-1}\), le lemme \ref{LemsKpXCG} s'applique. Nous avons donc bien défini un raffinement.

    Nous devons maintenant prouver que les deux raffinements ainsi construits sont des suites de composition équivalentes. D'abord elles ont la même longueur \( mn\) parce que chacun des \( m\) éléments de la suite \( (G_i)\) a été remplacé par \( n\) éléments et inversement, chacun de \( n\) éléments de la suite \( (H_j)\) a été remplacé par \( m\) éléments.

    Par ailleurs, les quotients du raffinement de \( (G_i)\) sont de la forme
    \begin{equation}    \label{EqPAYTCB}
        \frac{ G_{i+1}(G_i \cap H_k) }{ G_{i+1}(G_i\cap H_{k+1}) }\simeq \frac{ H_{k+1}(H_k\cap G_i) }{ H_{k+1}(H_k\cap G_{i+1}) }
    \end{equation}
    en vertu du lemme du papillon (\ref{LemsKpXCG}). Le membre de droite de \eqref{EqPAYTCB} est un des quotients du raffinement de \( (H_j)\).
\end{proof}

\begin{lemma}[Schreider strictement décroissant]    \label{LemBSicRJ}
    Soient \( \Sigma_1\) et \( \Sigma_2\), deux suites de composition strictement décroissantes du groupe \( G\). Alors elles admettent des raffinements équivalents strictement décroissants.
\end{lemma}

\begin{proof}
    Par hypothèse, \( \Sigma_1\) et \( \Sigma_2\) n'ont pas de répétitions. Soient \( \Sigma''_1\) et \( \Sigma''_2\), des raffinements équivalents donnés par le lemme de Schreider. Étant donné que ce sont des suites de composition équivalentes, elles ont le même nombre de quotients réduits à \( \{ e \}\), c'est à dire le même nombre de répétitions.

    Les suites \( \Sigma'_1\) et \( \Sigma'_2\) obtenues en retirant les répétitions de \( \Sigma''_1\) et \( \Sigma''_2\) sont des raffinements équivalents de \( \Sigma_1\) et \( \Sigma_2\) et strictement décroissants.
\end{proof}

\begin{theorem}[Jordan-Hölder]\label{ThoLgxWIC}
    Tout groupe fini admet une suite de Jordan-Hölder.

    Deux suites de Jordan-Hölder sont équivalentes.
\end{theorem}
% TODO : trouver une preuve du fait que tout groupe fini admet une suite de Jordan-Hölder.

\begin{proof}
    Nous ne prouvons que le second point.

    Par définition, une suite de Jordan-Hölder n'a pas de raffinement strictement décroissant (à part elle-même) parce que \( G_{i+1}\) est normal maximum dans \( G_i\). Si \( \Sigma_1\) et \( \Sigma_2\) sont des suites de Jordan-Hölder nous pouvons considérer les raffinements équivalents strictement décroissants \( \Sigma'_1\) et \( \Sigma'_2\) du lemme de Schreider \ref{LemBSicRJ}. Nous avons \( \Sigma'_1\sim\Sigma'_2\), mais par ce que nous venons de dire à propos de la maximalité, \( \Sigma'_1=\Sigma_1\) et \( \Sigma'_2=\Sigma_2\). D'où le résultat.
\end{proof}

%+++++++++++++++++++++++++++++++++++++++++++++++++++++++++++++++++++++++++++++++++++++++++++++++++++++++++++++++++++++++++++
\section{Action de groupes}
%+++++++++++++++++++++++++++++++++++++++++++++++++++++++++++++++++++++++++++++++++++++++++++++++++++++++++++++++++++++++++++

Si \( G\) agit sur un ensemble \( E\), nous notons \( G\cdot x\) l'orbite de \( x\in E\) sous l'action de $G$. Nous notons \( G_x\)\nomenclature[R]{\( G_x\)}{stabilisateur de \( x\)} ou \( \Stab(x)\) le stabilisateur de \( x\) :
\begin{equation}
    G_x=\Stab(x)=\{ g\in G\tq g\cdot x=x \}.
\end{equation}
Pour \( g\in G\), nous notons aussi \( \Fix(g)\) le \defe{fixateur}{fixateur} de \( g\) :
\begin{equation}
    \Fix(g)=\{ x\in E\tq g\cdot x=x \}.
\end{equation}

\begin{definition}  \label{DefuyYJRh}
    L'action de \( G\) sur \( E\) est \defe{fidèle}{fidèle (action)}\index{action!fidèle} si l'identité est le seul élément de \( G\) à fixer tous les points de \( E\), c'est à dire si \( gx=x\,\forall x\in E\Rightarrow g=e\).
\end{definition}

Un exemple d'action fidèle tout à fait non trivial sera donné avec l'action du groupe modulaire sur le plan de Poincaré dans le théorème \ref{ThoItqXCm}.

Le groupe \( G\) agit toujours sur lui même à gauche et à droite. L'action à gauche est \( g\cdot h=gh\); celle à droite est \( g\cdot h=hg^{-1}\). Il existe aussi l'action \defe{adjointe}{action!adjointe} définie par \( g\cdot h=ghg^{-1}\).

Si \( H\) est un sous-groupe de  \( G\), nous notons \( G/H\) le quotient de $G$ par la relation \( g\sim gh\) pour tout \( h\in H\). Lorsque la distinction est importante, nous noterons \( (G/H)_g\)\nomenclature[R]{$(G/H)_g$}{classes à gauche} pour les classes à gauche et \( (G/H)_d\) pour les classes à droite.

Nous avons une relation d'équivalence à gauche et une à droite. D'abord
\begin{equation}
    x\sim_g y\Leftrightarrow xh=y
\end{equation}
pour un certain \( h\in H\). Ensuite
\begin{equation}
    x\sim_d y\Leftrightarrow hx=y
\end{equation}
pour un certain \( h\in H\). 

Le lemme suivant est une généralisation du théorème de Lagrange \ref{ThoLagrange}.

\begin{lemma}
    L'ensemble \( (G/H)_g\) est fini si et seulement si l'ensemble \( (G/H)_d\) est fini. Si il en est ainsi, alors \( (G/H)_g\) et \( (G/H)_d\) ont même cardinal qui vaut l'indice de \( H\) dans \( G\).
\end{lemma}

\begin{proof}
    L'application
    \begin{equation}
        \begin{aligned}
            f\colon (G/H)_g&\to (G/H)_d \\
            [x]_g&\mapsto [x^{-1}]_d 
        \end{aligned}
    \end{equation}
    est une bijection bien définie. En effet si \( x\sim_g y\), nous avons \( h\in H\) tel que \( y^{-1}h=x^{-1}\), c'est à dire que \( x^{-1}\sim_d y^{-1}\) et \( f\) est bien définie. Le fait que \( f\) soit surjective est évident. Pour l'injectivité, soit
    \begin{equation}
        f([x]_g)=f([y]_h).
    \end{equation}
    Alors \( x^{-1}\sim_d y^{-1}\), ce qui implique l'existence de \( h\in H\) tel que \( hx^{-1}=y^{-1}\), ou encore que \( xh^{-1}=y\), ce qui signifie que \( x\sim_gy\).

    Pour l'énoncé à propos de l'indice, nous procédons en plusieurs étapes simples.
    \begin{enumerate}
        \item
            Les classes (les éléments de \( (G/H)_g\)) formes une partition de $G$.
        \item
            Toutes les classes ont le même nombre d'éléments par la bijection 
            \begin{equation}
                \begin{aligned}
                    f\colon [x]_g&\to [y]_g \\
                    xh&\mapsto yh. 
                \end{aligned}
            \end{equation}
        \item
            Le nombre d'éléments dans une classe est égal à \( | H |\) par la bijection
            \begin{equation}
                \begin{aligned}
                    g\colon [x]_g&\to H \\
                    xh&\mapsto h. 
                \end{aligned}
            \end{equation}
    \end{enumerate}
    Par conséquent
    \begin{equation}
        | G |=| H |\cdot \text{nombre de classes}=| H |\cdot\text{cardinal de $(G/H)_g$},
    \end{equation}
    et nous avons bien 
    \begin{equation}
        \text{cardinal de $(G/H)_g$}=\frac{ | G | }{ | H | }=| G:H |.
    \end{equation}
\end{proof}

\begin{proposition}[Orbite-stabilisateur\cite{Combes}]\index{équation!orbite-stabilisateur}     \label{Propszymlr}
    Soit \( G\) un groupe agissant sur un ensemble \( E\) et \( x\in E\).
    \begin{enumerate}
        \item
            Les ensembles \( G\cdot x\) et \( G/G_x\) sont équipotents.
        \item
            L'orbite de \( G_x\) est finie si et seulement si \( G_x\) est d'indice fini dans \( G\). Dans ce cas nous avons 
            \begin{equation}        \label{EqnLCHCE}
                \Card(G\cdot x)=| G:G_x |.
            \end{equation}
            Une autre façon d'écrire la même formule :
            \begin{equation}        \label{EqCewSXT}
                | G |=| \Stab(x) | |\mO_x |.
            \end{equation}
    \end{enumerate}
\end{proposition}
C'est la formule \eqref{EqnLCHCE} qui est nommée \wikipedia{fr}{Action_de_groupe_(mathématiques)\#Formule_des_classes.2C_formule_de_Burnside}{formule des classes} sur wikipédia.

\begin{proof}
    \begin{enumerate}
        \item
    Soit l'application
    \begin{equation}
        \begin{aligned}
            \psi\colon G\cdot x&\to G/G_x \\
            a\cdot x&\mapsto [a]. 
        \end{aligned}
    \end{equation}
    Cette application est bien définie parce que si \( a\cdot x=b\cdot x\), alors il existe \( h\in G_x\) tel que \( b=ah\), et par conséquent \( [a]=[b]\). Cette application est une bijection et par conséquent \( G\cdot x\) est équipotent à \( G/G_x\).
    \item
        Soit \( y\in \mO_x\) et \( A_y=\{ g\in G\tq g\cdot x=y \}\). L'ensemble \( A_y\) est une classe à gauche de \( \Stab(x)\), par conséquent \( | A_y |=|\Stab(x)|\) pour tout \( y\in\mO_x\). Les \( A_y\) pour différents \( y\) sont disjoints et nous avons de plus
        \begin{equation}
            \bigcup_{y\in\mO_x}A_y=G.
        \end{equation}
        Les ensemble \( A_y\) divisent donc \( G\) en \( | \mO_x |\) paquets de \( | \Stab(x) |\) éléments. D'où la formule \eqref{EqCewSXT}.
        
    \end{enumerate}
\end{proof}

\begin{corollary}
    Soit \( C_g\) la classe de conjugaison de l'élément  \( g\) du groupe fini \( G\). Alors
    \begin{equation}
        \Card(C_g)=| G:Z_g |
    \end{equation}
\end{corollary}

\begin{proof}
    Cela est une application directe de la proposition \ref{Propszymlr} dans le cas de l'action adjointe de \( G\) sur lui-même.
\end{proof}

\begin{lemma}
    Soit \( G\) un groupe agissant sur l'ensemble \( E\). On définit \( x\sim x'\) si et seulement si il existe \( g\in G\) tel que \( g\cdot x=x'\). Alors
    \begin{enumerate}
        \item
            la relation \( \sim\) est une relation d'équivalence.
        \item
            la classe \( [x]\) est l'orbite \( \mO_x\) de \( x\) sous \( G\).
    \end{enumerate}
\end{lemma}

\begin{corollary}[Équation des orbites]\index{équation!des orbites} \label{CorARFVMP}
    Soit \( G\) un groupe agissant sur l'ensemble \( E\) et \( \mO_1,\ldots, \mO_k  \) la liste des orbites (distinctes). Alors
    \begin{enumerate}
        \item
            \( E=\bigcup_i\mO_1\), l'union est disjointe,
        \item
            \( \Card(E)=\sum_i\Card(\mO_i)\).
    \end{enumerate}
\end{corollary}

\begin{definition}  \label{DefcSuYxz}
    Soit \( G\) un groupe agissant sur l'ensemble \( E\). Un \defe{domaine fondamental}{domaine!fondamental d'une action}\index{fondamental!domaine d'une action}\index{action!domaine fondamental} ou une \defe{transversale}{transversale} est une partie de \( E\) contenant un et un seul élément de chaque orbite.
\end{definition}
Autrement dit, les images des éléments d'un domaine fondamental forment une partition de l'ensemble :
\begin{equation}
    E=\bigsqcup_{g\in G}f(F),
\end{equation}
union disjointe, c'est à dire que si \( g\neq g'\), alors \( g(F)\cap g'(F)=\emptyset\).

\begin{proposition}[Équation des classes\cite{FabricegPSFinis}]     \label{PropUyLPdp}
    Soit \( G\), un groupe fini opérant sur un ensemble \( E\). Si \( E''\) est un ensemble contenant exactement un élément de chaque orbite dans \( E\setminus\Stab_G(E)\), alors
    \begin{equation}        \label{EqobuzfK}
        | G |=| \Stab_G(E) |+\sum_{x\in E''}\frac{ | G | }{ | \Stab_G(x) | }.
    \end{equation}
    Si de plus \( G\) est un $p$-groupe, alors 
    \begin{equation}    \label{EqbzLEVJ}
        | E |=| \Stab_G(E) |\mod p.
    \end{equation}
\end{proposition}


\begin{proof}
    Par le corollaire \ref{CorARFVMP}, nous avons \( | G |=\sum_{x\in E'}| \mO_x |\) où \( E'\) est une transversale.  En séparant la somme entre les orbites à un élément et les autres,
    \begin{equation}    \label{EqeggkBs}
        | G |=\Card(\Stab_G(E))+\sum_{x\in E''}\frac{ | G | }{ | \Stab_G(x) | }
    \end{equation}  \label{EqDgYbhm}
    où nous avons utilisé le fait que \( | G |=| \Stab_G(x) | |\mO_x |\).

    Si \( G\) est un \( p\)-groupe alors si \( x\in E''\), \( \Stab_G(x)\) est un sous-groupe propre de \( G\) et donc \( | \Stab_G(x) |\) est un diviseur propre de \( | G |\). Du coup la fraction \( | G |/|\Stab_G(x)|\) est une puissance non nulle de \( p\) et l'équation \eqref{EqobuzfK} devient immédiatement \eqref{EqbzLEVJ}.
\end{proof}
 

\begin{corollary}[Équation des classes]\index{équation!des classes}
    Soit \( G\), un groupe et \( C_1\),\ldots, \( C_l\) la liste de ses classes de conjugaison contenant plus de un éléments. Alors
    \begin{equation}        \label{EqkgGmoq}
        \Card(G)=\Card\big( Z(G) \big)+\sum_i| G:Z_{g_i} |=\Card\big( Z(G) \big)+\sum_i\frac{ \Card(G) }{ \Card\big( \Stab(g_i) \big) }
    \end{equation}
    si \( g_i\in C_i\).
\end{corollary}

\begin{proof}
    Étant donné que les classes de conjugaison sont disjointes, le cardinal de \( G\) est bien la somme des cardinaux de ses classes. Les classes ne contenant que un seul élément sont celles des éléments de \( Z(G)\). En ce qui concerne les autres orbites, \( \Card(C_{g_i})=| G:Z_{g_i} |\) par le théorème orbite-stabilisateur (proposition \ref{Propszymlr}).
\end{proof}

\begin{theorem}[\wikipedia{fr}{Action_de_groupe_(mathématiques)}{formule de Burnside}]  \index{Burnisde!formule}\index{formule!Burnside}
    Si \( G\) est un groupe fini agissant sur l'ensemble fini \( E\) et si \( \Omega\) est l'ensemble des orbites, alors
    \begin{equation}    \label{EqTUsblv}
        \Card(\Omega)=\frac{1}{ | G | }\sum_{g\in G}\Card\big( \Fix(g) \big).
    \end{equation}
\end{theorem}

\begin{proof}
    Nous considérons l'ensemble 
    \begin{equation}
        A=\{ (g,x)\in G\times E\tq gx=x \},
    \end{equation}
    et nous en calculons le cardinal de deux façons. D'abord
    \begin{subequations}
        \begin{align}
            \Card(A)&=\sum_{x\in E}\Card\{ g\in g\tq gx=x \}\\
            &=\sum_{x\in E}\Card(\Stab(x))\\
            &=\sum_{\omega\in \Omega}\sum_{x\in \omega}\Card(\Stab(x))\\
            &=\sum_{x\in \omega}\frac{ | G | }{ \Card(\omega) }     \label{EqyVtkyf}\\
            &=| G |.
        \end{align}
    \end{subequations}
    Pour obtenir \eqref{EqyVtkyf} nous avons utilisé l'équation des classes \eqref{EqCewSXT}. L'autre façon de calculer \( \Card(A)\) est de regrouper ainsi :
    \begin{equation}
        \Card(A)=\sum_{g\in G}\Card\{ x\in E\tq gx=x \}=\sum_{g\in G}\Card(\Fix(g)).
    \end{equation}
    En égalisant les deux expressions de \( \Card(A)\) nous trouvons
    \begin{equation}
        | G |\Card(\Omega)=\sum_{g\in G}\Card(\Fix(g)).
    \end{equation}
\end{proof}

\begin{proposition}
    Soit \( G\) un groupe et \( H\), un sous-groupe du centre de \( G\).
    \begin{enumerate}
        \item
            \( H\) est normal dans \( G\).
        \item
            Si \( G/H\) est monogène, alors \( G\) est abélien.
        \item
            Si \( G\) est fini de centre \( Z\), alors \( | G:H |\) n'est pas premier.
    \end{enumerate}
\end{proposition}

\begin{theorem}
    Soit \( G\) un groupe cyclique d'ordre \( n\).
    \begin{enumerate}
        \item
            Tout sous-groupe de \( G\) est cyclique.
        \item 
            Pour chaque diviseur \( d\) de \( n\), il existe un unique sous-groupe \( H_d\) de \( G\) d'ordre \( d\).
    \end{enumerate}
    Si \( a\) est un générateur de \( G\), alors \( H_d\) peut être décrit des façons suivantes :
    \begin{equation}
        H_d=\{ x\in G\tq x^d=e \}=\{ x\in G\tq\exists y\in G\tq y^{n/d}=x \}=\langle a^{n/d}\rangle.
    \end{equation}
\end{theorem}

\begin{example}     \label{ExemMaKdwt}
    Si \( E\) est un espace vectoriel alors \( (E,+)\) est un groupe commutatif. L'inverse de \( x\) est \( -x\).
\end{example}

\begin{definition}
    Soit \( G\) un groupe agissant sur un ensemble \( E\). Nous disons que l'action est \defe{transitive}{transitive}\index{action!transitive} si elle possède une seule orbite. L'action est \defe{libre}{libre!action}\index{action!libre} si \( g\cdot x=g'\cdot x\) implique \( g=g'\).
\end{definition}

%+++++++++++++++++++++++++++++++++++++++++++++++++++++++++++++++++++++++++++++++++++++++++++++++++++++++++++++++++++++++++++
\section{Le groupe symétrique}
%+++++++++++++++++++++++++++++++++++++++++++++++++++++++++++++++++++++++++++++++++++++++++++++++++++++++++++++++++++++++++++
Une source intéressante d'informations est \cite{LoFdlw}.

Le \defe{groupe symétrique}{groupe!symétrique} \( S_n\)\nomenclature[R]{\( S_n\)}{le groupe symétrique} est le groupe des permutations de l'ensemble \( \{ 1,\ldots,n \}\). C'est donc l'ensemble des bijections \( \{ 1,\ldots, n \}\to\{ 1,\ldots, n \}\).

Nous disons qu'un élément \( s\in S_n\) \defe{inverse}{inversion!dans le groupe symétrique} les nombres \( i<j\) si \( s(i)>s(j)\). Soit \( N_s\) le nombre d'inversions que \( s\in S_n\) possède (c'est le nombre de couples \( (i,j)\) avec \( i<j\) tels que \( s(i)>s(j)\)). L'entier
\begin{equation}
    \epsilon(s)=(-1)^{N_s}
\end{equation}
est la \defe{signature}{signature} de \( s\).

Un \wikipedia{fr}{Permutation}{élément du groupe symétrique} \( S_n\) peut être décomposé en produit de cycles de support disjoints de la façon suivante. D'abord écrire le cycle qui correspond à l'orbite de \( 1\). Ce sera le cycle
\begin{equation}
    (1,\sigma 1,\sigma^21,\ldots, \sigma^k1)
\end{equation}
avec \( \sigma^{k+1}1=1\). Ensuite nous recommençons avec le plus petit élément de \( \{ 1,\ldots, n \}\) à ne pas être dans ce cycle, et puis le suivant, etc. La \emph{structure} d'une telle décomposition est la donnée des nombres \( k_i\) donnant le nombre de cycles de longueur \( i\).

\begin{lemma}[\cite{Combes}]        \label{LemmvZFWP}
    Soit \( c=(i_1,\ldots, i_k)\in S_n\), un cycle de longueur \( k\) et \( s\in S_n\). Alors
    \begin{equation}
        csc^{-1}=\big( s(i_1),\ldots, s(i_k) \big).
    \end{equation}
    Tous les cycles de longueur \( k\) sont conjugués entre eux.
\end{lemma}

\begin{proposition}[\wikipedia{fr}{Permutation}{wikipédia}]
    Une classe de conjugaison dans \( S_n\) est formée des permutations ayant une décomposition en cycle disjoints de même structure. Autrement dit, deux permutations \( \sigma\) et \( \sigma'\) sont conjuguées si et seulement si le nombre \( k_i\) de cycles de longueur \( i\) dans \( \sigma\) est le même que le nombre \( k'_i\) de cycles de longueur \( i\) dans \( \sigma'\).
\end{proposition}

\begin{proof}
    Soit \( \sigma=c_1\ldots c_m\) la décomposition de \( \sigma\) en cycles de supports disjoints. Les \( c_i\) sont des cycles de supports disjoints. Si \( \tau\) est une permutation, alors
    \begin{equation}
        \sigma'=\tau\sigma\tau^{-1}=(\tau c_1\tau^{-1})\ldots (\tau c_m\tau^{-1}),
    \end{equation}
    mais \( \tau c_i\tau^{-1}\) est un cycle de même longueur que \( c\) parce que si \( \sigma=(a_1,\ldots, a_k)\), alors \( \tau c\tau^{-1}=\big( \tau(a_1),\ldots, \tau(a_k) \big)\). Notons encore que les cycles \( \tau c_i\tau^{-1}\) restent à support disjoints.

    Donc tous les éléments de la classe de conjugaison de \( \sigma\) sont des permutations de même structure de \( \sigma\).

    Réciproquement, si \( \sigma'=c'_1\ldots c'_m\) est une décomposition de \( \sigma'\) en cycles disjoints tels que la longueur de \( c_i\) est la même que la longueur de \( c'_i\), alors il suffit de prendre des permutations \( \tau_i\) telles que \( \tau_i c_i\tau_i^{-1}=c_i'\). Vu que les supports sont disjoints, la permutation \( \tau_1\ldots \tau_m\) conjugue \( \sigma\) et \( \sigma'\).
\end{proof}

\begin{example}
    Voyons les classes de conjugaison de \( S_3\). Étant donné que ce groupe agit par définition sur un ensemble à \( 3\) éléments, aucun élément de \( S_3\) ne possède un un cycle de plus de \( 3\) éléments. Il y a donc seulement des cycles de longueur deux ou trois (à part les triviaux). Aucun élément de \( S_3\) n'a une décomposition en cycles disjoints contenant deux cycles de deux ou un cycle de deux et un de trois.

    En résumé il y a trois classes de conjugaison dans \( S_3\). La première est celle contenant seulement l'identité. La seconde est celle contenant les cycles de longueur deux et la troisième contient les cycles de longueur \( 3\).

    Ce sont donc
    \begin{subequations}
        \begin{align}
            C_1&=\{ \id \}\\
            C_2&=\{ (1,2),(1,3),(2,3) \}\\
            C_3&=\{ (1,2,3),(2,1,3) \}.
        \end{align}
    \end{subequations}
\end{example}

\begin{lemma}[\cite{PDFpersoWanadoo}]       \label{LemhxnkMf}
    Un \( k\)-cycle est une permutation impaire si \( k\) est pair et paire si \( k\) est impair.
\end{lemma}

\begin{proposition} \label{PropPWIJbu}
    Tout élément de \( S_n\) peut être écrit sous la forme d'un produit fini de transpositions.
\end{proposition}
Cette décomposition n'est pas à confondre avec celle en cycles de support disjoints. Par exemple \( (1,2,3)=(1,3)(1,2)\).

\begin{proposition}[\cite{Combes}]  \label{ProphIuJrC}
    Soit \( S_n\) le groupe symétrique.
    \begin{enumerate}
        \item
            L'application \( \epsilon\colon S_n\to \{ 1,-1 \}\) est l'unique homomorphisme surjectif de \( S_n\) sur \( \{ -1,1 \}\).
        \item
            Si \( s=t_1\cdots t_k\) est le produit de \( k\) transpositions, alors \( \epsilon(s)=(-1)^k\).
    \end{enumerate}
\end{proposition}


\begin{proof}
    Soit \( s,t\in S_n\). Afin de montrer que \( \epsilon(st)=\epsilon(s)\epsilon(t)\), nous divisons les couples \( (i,j)\) tels que \( i\neq j\) en \( 4\) groupes suivant que \( t(i)\gtrless t(j)\) et \( s\big( t(i) \big)\gtrless s\big( t(j) \big)\). Nous notons \( N_1\), \( N_2\), \( N_3\) et \( N_4\) le nombre de couples dans chacun des quatre groupes :
    \begin{center}
    \begin{tabular}{c|c|c}
        $ (i,j)$&   \( s\big( t(i) \big)<s\big( t(j) \big)\)    &   \( s\big( t(i) \big)>s\big( t(j) \big)\)\\
        \hline
        \( t(i)<t(j)\)& \( N_1\)&\( N_2\)\\
        \hline
        \( t(i)>t(j)\)&\( N_3\)&\( N_4\)
    \end{tabular}
    \end{center}
    Nous avons immédiatement \( N_t=N_3+N_4\) et \( N_{st}=N_2+N_4\). Les éléments qui participent à \( N_s\) sont ceux où \( t(i)\) et \( t(j)\) sont dans l'ordre inverse de \( s\big( t(i) \big)\) et \( s\big( t(jj) \big)\) (parce que \( t\) est une bijection). Donc \( N_s=N_2+N_3\). Par conséquent nous avons
    \begin{equation}
        \epsilon(s)\epsilon(t)!(-1)^{N_2+N_3}(-1)^{N_3+N_4}=(-1)^{N_2+N_4}=(-1)^{N_{st}}=\epsilon(st).
    \end{equation}
    Nous avons prouvé que \( \epsilon\) est un homomorphisme. Pour montrer que \( \epsilon\) est surjectif sur \( \{ -1,1 \}\) nous devons trouver un élément \( t\in S_n\) tel que \( \epsilon(t)=-1\). Si \( t\) est la transposition \( 1\leftrightarrow 2\) alors le couple \( (1,2)\) est le seul à être inversé par \( t\) et nous avons \( \epsilon(t)=-1\).
    
    Avant de montrer l'unicité, nous montrons que si \( s=t_1\ldots t_k\) alors \( \epsilon(s)=(-1)^k\). Pour cela il faut montrer que \( \epsilon(t)=-1\) dès que \( t\) est une transposition. Soit \( t\), la transposition \( (i,j)\) et \( c=(i,i+1,\ldots, j-1)\) alors le lemme \ref{LemmvZFWP} dit que
    \begin{equation}
        t_{ij}=ct_{j-1,j}c^{-1}.
    \end{equation}
    La signature étant un homomorphisme,
    \begin{equation}
        \epsilon(t_{ij})=\epsilon(c)\epsilon(t_{j-1,j})\epsilon(c)^{-1}=\epsilon(t_{j-1,j})=1.
    \end{equation}
    
    Nous passons maintenant à la partir unicité de la proposition. Soit un homomorphisme surjectif \( \varphi\colon S_n\to \{ -1,1 \}\) et \( t\), une transposition telle que \( \varphi(t)=-1\) (qui existe parce que sinon \( \varphi\) ne serait pas surjectif\footnote{Nous utilisons ici le fait que tous les éléments de \( S_n\) sont des produits de transpositions, proposition \ref{PropPWIJbu}.}). Si \( t'\) est une autre transposition, il existe \( s\in S_n\) tel que \( t'=sts^{-1}\) (lemme \ref{LemmvZFWP}). Dans ce cas, \( \varphi(t')=\varphi(t)=-1\), et si \( s=t_1\ldots t_k\),
    \begin{equation}
         \varphi(s)=(-1)^k=\epsilon(s).
    \end{equation}
\end{proof}

Le groupe \( A_n\)\nomenclature[R]{\( A_n\)}{groupe alterné} des permutations paires est la \defe{groupe alterné}{alterné!groupe}\index{groupe!alterné}.

\begin{proposition}
    Le groupe alterné est un sous-groupe caractéristique de \( S_n\) d'indice \( 2\). C'est le seul sous-groupe d'indice \( 2\) dans \( S_n\).
\end{proposition}

\begin{proof}
    Soit \( \alpha\in \Aut(S_n)\). Étant donné que \( \alpha\circ\alpha\) est un homomorphisme surjectif sur \( \{ -1,1 \}\) nous avons \( \alpha\circ\alpha=\epsilon\), et donc \( \alpha(A_n)=A_n\). Par le premier théorème d'isomorphisme, il existe un isomorphisme
    \begin{equation}
        f\colon S_n/\ker\epsilon\to \Image(\epsilon).
    \end{equation}
    En égalisant le nombre d'éléments nous avons \( | S_n:\ker\epsilon |=| S_n:A_n |=2\).

    Nous prouvons maintenant l'unicité. Soit \( H\) un sous-groupe d'indice \( 2\) dans \( S_n\). Par le lemme \ref{LemSkIOOG}, \( H\) est distingué et nous pouvons considérer le groupe \( S_n/H\). Ce dernier ayant \( 2\) éléments, il est isomorphe à \( \{ -1,1 \}\). Soit \( \theta\) l'isomorphisme. Si nous notons \( \varphi\) le morphisme canonique \( \varphi\colon S_n\to S_n/H\) :
    \begin{equation}    \label{EqSZBPTH}
        \xymatrix{%
        S_n \ar[r]^{\varphi}        &   S_n/H\ar[r]^{\theta}&\{ -1,1 \}.
           }
    \end{equation}
    La composition \( \varphi\circ \theta\) est alors un homomorphisme surjectif de \( S_n\) sur \( \{ -1,1 \}\) et nous avons \( \varphi\circ\theta=\epsilon\) par la proposition \ref{ProphIuJrC}. L'enchaînement \eqref{EqSZBPTH} nous montre que \( H=\ker(\theta\circ\varphi)=\ker(\epsilon)=A_n\).
\end{proof}

\begin{lemma}   \label{LemiApyfp}   \index{groupe!dérivé!du groupe symétrique}
    Le groupe dérivé du groupe symétrique est le groupe alterné : \( D(S_n)=A_n\).
\end{lemma}

\begin{proof}
    Tout élément de \( D(S_n)\) s'écrit sous la forme \( ghg^{-1}h^{-1}\). Quel que soit le nombre de transpositions dans \( g\) et \( h\), le nombre de transpositions dans \( [g,h]\) est pair.
\end{proof}

\begin{proposition}     \label{PropsHlmvv}
    Soit \( n\geq 3\). Les \( 3\)-cycles \( c_i=(1,2,i)\) avec \( i=3,\ldots, n\) engendrent le groupe alterné \( A_n\).
\end{proposition}

\begin{proof}
    Soit \( H\), le groupe engendré par les \( c_i\). D'abord nous avons 
    \begin{equation}
        c_i=(1,2,i)=(1,2)(2,i),
    \end{equation}
    de telle sorte que \( \epsilon(c_i)=1\). Par conséquent nous avons \( H\subset A_n\). Nos montrons par récurrence que \( A_n\subset H\).

    Pour \( n=3\) il suffit de vérifier que \( A_3=\{ \id,c_3,c_3^2 \}\). Supposons avoir obtenu le résultat pour \(A_{n-1}\), et prouvons le pour \( A_n\). Soit \( s\in A_n\).

    Si \( s(n)=n\), alors \( s\) se décompose de la même manière que sa restriction \( s'\) à \( \{ 1,\ldots, n-1 \}\). Par l'hypothèse de récurrence, cette restriction, appartenant à \( A_{n-1}\),  se décompose en produit des \( c_3,\ldots, c_{n-1}\) et de leurs inverses.

    Si \( s(n)=k\) alors nous considérons l'élément \( c^2_nc_ks\). Cet élément envoie \( n\) sur \( n\) et peut donc être décomposé avec les \( c_i\) (\( i=1,\ldots, n-1\)) en vertu du point précédent.
\end{proof}

\begin{proposition} \label{PropiodtBG}
    Lorsque \( n\geq 5\), tous les \( 3\)-cycles de \( A_n\) sont conjugués. Autrement dit, la classe de conjugaison d'un \( 3\)-cycle est l'ensemble des \( 3\)-cycles.
\end{proposition}

\begin{proof}
    Soient les \( 3\)-cycles \( \sigma=(i_1,i_2,i_3)\) et \( \varphi=(j_1,j_2,j_3)\). Nous considérons une bijection \( \alpha\) de \( \{ 1,\ldots, n \}\) telle que \( \alpha(i_s)=j_s\). Nous avons immédiatement que \( \alpha\in S_n\) et que \( \alpha\sigma\alpha^{-1}=\varphi\). Donc les \( 3\)-cycles sont conjugués dans \( S_n\). Il reste à prouver qu'ils le sont dans \( A_n\).

    Si \( \alpha\) est une permutation paire, la preuve est terminée. Si \( \alpha\) est impaire, alors nous devons un peu la modifier. Vu que \( n\geq 5\), nous pouvons prendre \( s\) et \( t\), des éléments distincts dans \( \{ 1,\ldots, n \}\setminus\{ j_1,j_2,j_3 \}\) et poser \( \tau=(st)\). Vu que la signature est un homomorphisme et que \( \tau\) et \( \alpha\) sont impairs, l'élément \( \tau\alpha\) est pair (lemme et proposition \ref{LemhxnkMf} et \ref{PropPWIJbu}) et est donc dans \( A_n\). Les supports de \( \tau\) et \( \varphi\) étant disjoints, ces derniers commutent et nous avons
    \begin{equation}
        (\tau\alpha)\sigma(\tau\sigma)^{-1}=\tau(\alpha\sigma\sigma^{-1})\tau^{-1}=\varphi.
    \end{equation}
    Donc \( \sigma\) et \( \varphi\) sont conjugués par \( \tau\alpha\) qui est dans \( A_n\).
\end{proof}

\begin{theorem}[\cite{PDFpersoWanadoo}]
    Le groupe alterné \( A_n\) est simple pour \( n\geq 5\).
\end{theorem}

\begin{proof}
    Soit \( N\), un sous-groupe normal de \( A_n\) non réduit à l'identité. Étant donné que les \( 3\)-cycles engendrent \( A_n\) (proposition \ref{PropsHlmvv}) et que tous les \( 3\)-cycles sont conjugués dans \( A_n\) (proposition \ref{PropiodtBG}), il suffit de montrer que \( N\) contient un \( 3\)-cycle. En effet si \( N\) contient un \( 3\)-cycle, le fait qu'il soit normal implique (par conjugaison) qu'il les contienne tous et donc qu'il contient une partie génératrice de \( A_n\).

    Soit donc \( \sigma\in N\) différent de l'identité. Nous prenons \( i\) dans le support de \( \sigma\) et \( j=\sigma(i)\). Nous choisissons ensuite \( k\in\{ 1,\ldots, n \}\setminus\{ i,j,\sigma^{-1}(i) \}\) et \( m=\sigma(k)\). Nous considérons la permutation \( \alpha=(ijk)\). Étant donné que \( N\) est normal l'élément
    \begin{equation}
        \theta=\alpha^{-1}\sigma\alpha\sigma^{-1}
    \end{equation}
    est dans \( N\). De plus en utilisant le lemme \ref{LemmvZFWP} et le fait que \( \alpha^{-1}=(ikj)\) nous avons
    \begin{equation}
        \theta=(ijk)(j\sigma(j)m).
    \end{equation}
    Cela n'est pas spécialement un \( 3\)-cycle, mais nous allons en construire un. Nous allons déterminer que \( \theta\) est soit un \( 5\)-cycle, soit un \( 3\)-cycle , soit un \( 2\times 2\)-cycle suivant les valeurs de \( \sigma(j)\) et \( m\). 

    Souvenons nous que \( j\neq\sigma(j)\) parce que \( i\) est dans le support de \( \sigma\); \( m\neq i\) parce que \( k\neq \sigma^{-1}(i)\); \( i\neq m\) parce que \( k\neq \sigma^{-1}(i)\). Les seules possibilités d'égalités sont \( i=\sigma(j)\), \( \sigma(j)=k\) et \( m=k\) (et les combinaisons, mais toutes ne sont pas possibles).
    
    Si \( i\), \( j\), \( k\), \( \sigma(j)\) et \( m\) sont cinq nombres différents, alors 
    \begin{equation}
        \theta=(i,j,\sigma(j),m,k)
    \end{equation}
    est un \( 5\)-cycle.

    Si \( i=\sigma(j)\), alors
    \begin{equation}
        \theta=(imk)
    \end{equation}
    qui est un \( 3\)-cycle. Notons que \( i\), \( m\) et \( k\) sont bien trois éléments différents.

    Si \( \sigma(j)=k\), alors
    \begin{equation}
        \theta=(ikm)
    \end{equation}
    qui est encore un \( 3\)-cycle.

    Si \( m=k\), nous avons
    \begin{equation}
        \theta=(ik)(j\sigma(j)).
    \end{equation}
    C'est a priori un \( 2\times 2\)-cycle. Mais si de plus \( i=\sigma(j)\), alors
    \begin{equation}
        \theta=(ijk)
    \end{equation}
    qui est un \( 3\)-cycle. Et si \( k=\sigma(j)\), alors
    \begin{equation}
        \theta=(ikj)
    \end{equation}
    qui est un autre \( 3\)-cycle.

    Bref nous avons montré que \( \theta\) est soit un \( 3\)-cycle, soit un \( 5\)-cycle, soit un \( 2\times 2\)-cycle. Si \( \theta\) est un \( 3\)-cycle, la preuve est terminée.

    Si \( \theta=(ab)(cd)\), alors on considère \( e\in \{ 1,\ldots, n \}\setminus\{ a,b,c,d \}\) et nous avons
    \begin{equation}
        \underbrace{(abe)^{-1}\theta(abe)}_{\in N}\theta^{-1}=(aeb)(ab)(cd)(abe)(an)(cd)=(abe)\in N.
    \end{equation}
    
    Si \( \theta\) est le \( 5\)-cycle \( (abcde)\), alors l'élément suivant est dans \( N\) :
    \begin{equation}
        (abc)^{-1}\theta(abc)\theta^{-1}=(acb)(abcde)(abc)(aedcb)=(acd).
    \end{equation}
    
    Dans tous les cas nous avons trouvé un \( 3\)-cycle dans \( N\) et nous avons par conséquent \( N=A_n\), ce qui fait que \( A_n\) ne contient pas de sous-groupes normaux non triviaux. Le groupe alterné \( A_n\) est donc simple.
\end{proof}

Nous en déduisons immédiatement que si \( n\geq 5\), le groupe dérivé de \( A_n\) est \( A_n\) parce que \( A_n\) ne contient pas d'autres sous-groupes non triviaux.\index{groupe!dérivé!du groupe alterné}

Le théorème suivant montre que tout groupe peut être vu, en agissant sur lui-même, comme une partie du groupe symétrique.
\begin{theorem}
    Un groupe \( G\) est isomorphe à un sous-groupe de son groupe symétrique \( S(G)\).
\end{theorem}

\begin{proof}
    Nous considérons \( \varphi\), la translation à gauche :
    \begin{equation}
        \begin{aligned}
            \varphi\colon G&\to S(G) \\
            g&\mapsto t_g 
        \end{aligned}
    \end{equation}
    où \( f_g(h)=gh\). Étant donné que
    \begin{equation}
        \varphi(gh)= ghx=g(t_hx)=t_g\circ t_h(x),
    \end{equation}
    l'application \( \varphi\) est un morphisme de groupes. Il est injectif parce que si \( gx=hx\) pour tout \( x\), en particulier pour \( x=e\) nous trouvons \( g=h\). 
    
    De la même manière, \( \varphi(g)x=\varphi(g)y\) implique \( x=y\). Cela montre que l'image est bien dans le groupe symétrique.

    L'ensemble \( \Image(\varphi)\) est donc un sous-groupe de \( S(G)\), et \( \varphi\) est un isomorphisme vers ce groupe.
\end{proof}

%+++++++++++++++++++++++++++++++++++++++++++++++++++++++++++++++++++++++++++++++++++++++++++++++++++++++++++++++++++++++++++
\section{Théorèmes de Sylow}
%+++++++++++++++++++++++++++++++++++++++++++++++++++++++++++++++++++++++++++++++++++++++++++++++++++++++++++++++++++++++++++

\begin{lemma}
    Soient \( H\) et \( K\) des sous-groupes finis de \( G\). Alors
    \begin{equation}
        \Card(HK)=\frac{ | H |\cdot | K | }{ | H\cap K | }.
    \end{equation}
\end{lemma} 
Attention : dans ce lemme, l'ensemble \( HK\) n'est pas spécialement un groupe. Ce serait le cas si \( H\) normaliserait \( K\), c'est à dire si nous avions \( hkh^{-1}\in K<,\forall h,k\in H\times K\).

\begin{theorem}[Théorème de Cauchy]\index{Cauchy!théorème}\index{théorème!Cauchy}       \label{ThoCauchyGpFini}
    Soit \( G\) un groupe fini et \( p\) un nombre premier divisant \( | G |\). Alors 
    \begin{enumerate}
        \item
            \( G\) contient un élément d'ordre \( p\).  
        \item
            Si \( G\) est un \( p\)-groupe, il existe un élément central d'ordre \( p\) dans \( G\).
    \end{enumerate}
\end{theorem}
Une preuve du premier point est sur \wikipedia{fr}{Théorème_de_Cauchy_(groupes)}{wikipedia}.

\begin{lemma}[Théorème de Cayley]    \label{ThoIfdlEB}   \index{Cayley!théorème}
    Si \( G\) est un groupe d'ordre \( n\) alors il est isomorphe à un sous-groupe du groupe symétrique \( S_n\).
\end{lemma}

\begin{proof}
    L'action à gauche de \( G\) sur lui-même
    \begin{equation}
        \begin{aligned}
            \varphi\colon G&\to S_n \\
            \varphi(x)g&\mapsto xg 
        \end{aligned}
    \end{equation}
    est une permutation des éléments de \( G\). Cela donne un morphisme injectif parce que si \( \varphi(x)=\varphi(y)\) nous avons \( xg=yg\) pour tout \( g\) et en particulier pour \( g=e\) nous trouvons \( x=y\).
\end{proof}

\begin{lemma}       \label{LemaQxjcm}
    Soit \( p\) un diviseur premier de \( n\). Alors le groupe symétrique \( S_n\) se plonge dans \( \GL_n(\eF_p)\).
\end{lemma}

\begin{proof}
    Soit \( \{ e_i \}\) la base canonique de \( \eF_p\). Nous avons le morphisme injectif $\varphi\colon S_n\to \Gl(n,\eF)$ donné par \( \varphi(\sigma)e_i=e_{\sigma(i)}\).
\end{proof}
 
\begin{remark}  \label{RemFzxxst}
    En mettant bout à bout les lemmes \ref{ThoIfdlEB} et \ref{LemaQxjcm}, nous trouvons que si \( p\) est un diviseur premier de \( | G |\), alors \( G\) peut être vu comme un sous-groupe de \( \Gl(n,\eF_p)\).
\end{remark}

\begin{lemma}
    Soit \( G\) un groupe fini et \( P\), \( Q\) des \( p\)-sous-groupes. Nous supposons que \( Q\) normalise \( P\). Alors \( PQ\) est un \( p\)-sous-groupe de \( G\).
\end{lemma}


\begin{definition}
    Soit \( p\) un nombre premier. Un \defe{$p$-groupe}{$p$-groupe}\index{groupe!$p$-groupe} est un groupe dont tous les éléments sont d'ordre \( p^m\) pour un certain \( m\) (dépendant de l'élément).

    Soit \( G\) un groupe fini et \( p\), un diviseur premier de $| G |$. Un \defe{\(p\)-Sylow}{$p$-Sylow}\index{Sylow!$ p$-Sylow} dans \( G\) est un \( p\)-sous-groupe d'ordre \( p^n\) où \( p^n\) est la plus grande puissance de \( p\) divisant \( | G |\).
\end{definition}
Notons que si \( p\) est un nombre premier, alors tout groupe d'ordre \( p^m\) est un \( p\)-groupe.

Si \( S\) est un \( p\)-Sylow, alors \( p\) ne divise pas le nombre \( | G:S |=| G |/| S |\).

\begin{proposition}     \label{Propvocmon}
    Soit le corps fini \( \eF_p=\eZ/p\eZ\) (\( p\) premier). Soit \( T\) se sous-ensemble de \( \GL_n(\eF_p)\) formé des matrices triangulaires supérieures de rang \( n\) et dont les éléments diagonaux sont \( 1\). Alors \( T\) est un \( p\)-Sylow de \( \GL_n(\eF_p)\).
\end{proposition}

\begin{proof}
    Nous commençons par étudier le cardinal de \( \GL_n(\eF_p)\). Pour la première colonne, la seule contrainte à vérifier est qu'elle ne soit pas nulle. Il y a donc \( p^n-1\) possibilités. Pour la seconde, il faut ne pas être multiple de la première. Il y a donc \( p^n-p\) possibilités (parce qu'il y a \( p\) multiples possibles de la premières colonne). Pour la \( k\)-ième colonne, il faut éviter toutes les combinaisons linéaires des \( (k-1)\) premières colonnes. Il y a \( p^{k-1}\) telles combinaisons et donc \( p^n-p^{k-1}\) possibilités pour la \( k\)-ième colonne. Nous avons donc
    \begin{subequations}
        \begin{align}
            \Card\big( \GL(n,\eF_{p}) \big)&=(p^n-1)(p^n-p)\ldots(p^n-p^{n-1})\\
            &=p\cdot p^2\cdots p^{n-1}(p^n-1)(p^{n-1}-1)\ldots (p-1)\\
            &=p^{\frac{ n(n-1) }{2}}m
        \end{align}
    \end{subequations}
    où \( m\) est un entier qui ne divise pas \( p\).

    En ce qui concerne le cardinal de \( T\), le calcul est plus simple : pour la première ligne nous avons \( p^{n-1}\) choix (parce qu'il y a un \( 1\) qui est imposé sur la diagonale), pour la seconde \( p^{n-2}\), etc. En tout nous avons alors
    \begin{equation}
        | T |=p^{\frac{ n(n-1) }{2}},
    \end{equation}
    et \( T\) est un \( p\)-Sylow de \( \GL_n(\eF_p)\).
\end{proof}


\begin{proposition}
    Soit \( p\) un nombre premier. Un groupe fini \( G\) est un $p$-groupe si et seulement l'ordre de \( G\) est \( p^n\) pour un certain \( n\).
\end{proposition}

\begin{proof}
    Supposons que \( G\) est un $p$-groupe. Soit \( q\) un nombre premier divisant \( | G |\). Par le théorème de Cauchy (\ref{ThoCauchyGpFini}), le groupe \( G\) contient un élément d'ordre \( q\), soit \( g\) un tel élément. Étant donné que \( G\) est un $p$-groupe, \( g^{p^n}=g^q=e\) pour un certain \( n\). Donc $q=p^n$ et \( q=p\) parce que \( q\) est premier. Nous venons de prouver que \( p\) est le seul nombre premier qui divise \( | G |\). L'ordre de \( G\) est par conséquent une puissance de \( p\).

    Nous nous intéressons maintenant à l'implication inverse. Nous supposons que \( | G |=p^n\) pour un certain entier \( n\geq 0\). Soit \( g\in G\); nous notons \( r\) l'ordre de \( G\). Le sous-groupe \( \gr(g)\) est d'ordre \( r\), donc \( r\) divise \( | G |\) (par le théorème \ref{ThoLagrange} de Lagrange). Le nombre \( r\) est alors une puissance de \( p\).
\end{proof}

\begin{lemma}       \label{LemwDYQMg}
    Soit \( G\), un groupe fini de cardinal \( | G |=n\) et \( p\), un diviseur premier de \( n\). Nous notons \( n=p^m\cdot r\) où \( p\) ne divise pas \( r\). Soit \( H\) un sous-groupe de \( G\) et \( S\), un \( p\)-Sylow de \( G\). Alors il existe \( g\in G\) tel que 
    \begin{equation}
        gSg^{-1}\cap H
    \end{equation}
    soit un \( p\)-Sylow de \( H\).
\end{lemma}

\begin{proof}
    Nous considérons l'ensemble \( G/S\) sur lequel \( H\) agit. Si \( a\in G\), le stabilisateur de \( [a]\) dans \( G/S\) est
    \begin{subequations}
        \begin{align}
            \Stab\big( [a] \big)&=\{ h\in H\tq [ha]=[a] \}\\
            &=\{ h\in H\tq a^{-1}ha\in S\}\\
            &=aSa^{-1}\cap H.
        \end{align}
    \end{subequations}
    Nous cherchons \( a\in G\) tel que l'entier
    \begin{equation}        \label{EqZpUbWx}
        \frac{ \Card(H) }{ \Card\big( aSa^{-1}\cap H \big) }
    \end{equation}
    soit premier avec \( p\). En effet, dans ce cas le groupe \( \Stab([a])\) est un $p$-Sylow de \( H\) parce que \( | H:aSa^{-1}\cap H |\) ne divise pas \( p\). La formule des orbites (équation \eqref{EqCewSXT}) nous dit que
    \begin{equation}
        \frac{ | H | }{ | aSa^{-1}\cap H | }=\Card\big( \mO_{[a]} \big).
    \end{equation}
    Supposons que toutes les orbites aient un cardinal divisible par \( p\). Étant donné que \( G/S\) est une réunion disjointe de ses orbites, nous aurions
    \begin{equation}
        p\divides \Card(G/S)=\frac{ | G | }{ | S | }
    \end{equation}
    alors que \( S\) étant un $p$-Sylow, \( p\) ne peut pas diviser \( | G |/| S |\). Toutes les orbites n'ont donc pas un cardinal divisible par \( p\), et il existe un \( a\in G\) tel que \eqref{EqZpUbWx} soit vérifiée.
\end{proof}


\begin{theorem}[Théorème de Sylow]  \label{ThoUkPDXf}
    Soit \( G\) un groupe fini et \( p\), un diviseur premier de \( | G |\). Alors
    \begin{enumerate}
        \item
            \( G\) possède des \( p\)-Sylow.
        \item
            Tout \( p\)-sous-groupe de \( G\) est contenu dans un \( p\)-Sylow.
        \item   \label{ItemMzNRVf}
            Les \( p\)-Sylow de \( G\) sont conjugués.
        \item   \label{ItemkYbdzZ}
            Si \( n_p\) est le nombre de $p$-Sylow de \( G\), alors \( n_p\) divise \( | G |\) et \( n_p\equiv 1\mod p\).
    \end{enumerate}
\end{theorem}

\begin{proof}

    % Il y a ici un début d'une preuve qui fonctionne d'une autre manière. Dans cette démonstration, N est l'ordre de G.

    %\begin{enumerate}
        %\item
         %   Nous faisons la récurrence sur l'ordre \( N\) de \( G\). Pour \( N=1\), le seul \( p\)-Sylow est le groupe entier. Si \( N>1\), nous commençons par supposer que \( G\) contient un sous-groupe propre \( H\) tel que \( \pgcd(| G:H |,p)=1\).
%
 %           Si \( G\) contient un sous-groupe propre \( H\) tel que \( \pgcd(| G:H |,p)=1\), alors \( p\) divise \( | G:H |\). En effet \( p\) divise \( N\) et \( | G:H |=| G |/| H |\). Si il n'y a pas de \( p\) dans la décomposition de \( | H |\), alors \( p\) divise encore \( | G |/| H |\). Étant donné que \( p\) est un diviseur premier de \( | H |\), le groupe \( H\) contient des \( p\)-Sylow par hypothèse de récurrence. Montrons que si \( S\) est un \( p\)-Sylow de \( H\), alors \( S\) est également un $p$-Sylow de \( G\).
%
 %           Soit \( S\), un $p$-Sylow de \( H\). Nous avons \( | S |=p^n\) où \( n\) est la plus grande puissance de \( p\) divisant \( H\). Par hypothèse, il n'y a pas de \( p\) dans la décomposition de \( | G:H |\); par conséquent \( p^n\) est également la plus grande puissance de \( p\) qui divise \( | G |\) et \( S\) est alors un $p$-Sylow de \( G\).
%
 %           Nous supposons maintenant que \( G\) ne possède pas de sous-groupe \( H\) tels que \( \pgcd(| G:H |,p)=1\).
%
 %   \end{enumerate}

    \begin{enumerate}
        \item
            
            Nous savons de la remarque \ref{RemFzxxst} que \( G\) est un sous-groupe de \( \GL_n(\eF_p)\) et que ce dernier a un $p$-Sylow par la proposition \ref{Propvocmon}. Par conséquent \( G\) possède un $p$-Sylow par le lemme \ref{LemwDYQMg}.

        \item

            Soit \( H\) un \( p\)-sous-groupe de \( G\) et \( S\), un $p$-Sylow de \( G\) (qui existe par le point précédent). Par le lemme \ref{LemwDYQMg} il existe \( a\in G\) tel que \( aSa^{-1}\cap H\) soit un $p$-Sylow de \( H\). Mais \( H\) est un \(p\)-groupe et un $p$-Sylow dans un \( p\)-groupe est automatiquement le groupe entier. Par conséquent,
            \begin{equation}
                H=aSa^{-1}\cap H
            \end{equation}
            et \( H\subset aSa^{-1}\), ce qui signifie que \( H\) est inclus à un $p$-Sylow.

        \item

            Soit \( H\) un $p$-Sylow. Nous venons de voir que si \( S\) est un $p$-Sylow quelconque, alors \( H\) est inclus au $p$-Sylow \( aSa^{-1}\) pour un certain \( a\in G\). Donc \( H\) est un $p$-Sylow inclus dans le $p$-Sylow \( aSa^{-1}\), donc \( H=aSa^{-1}\).

        \item

            Le fait que \( n_p\) divise \( n\) est parce que tous les $p$-Sylow ont le même nombre d'éléments (ils sont conjugués) et sont deux à deux disjoints. Donc ils forment une partition de \( G\) et \( | G |=n_p| S |\) si \( S\) est un $p$-Sylow quelconque.
            
            Montrons maintenant que \( n_p\) est congru à un modulo \( p\). Soit \( E\) l'ensemble des $p$-Sylow de \( G\). Le groupe \( G\) agit sur \( E\) par conjugaison. Soit \( S\) un $p$-Sylow et considérons l'ensemble
            \begin{equation}
                E_S=\{ T\in E\tq s\cdot T=T\forall s\in S \}.
            \end{equation}
            où l'action est celle par conjugaison. C'est l'ensemble des points fixes de \( E\) sous l'action de \( S\). L'ensemble \( E\) est la réunion des orbites sous \( S\) et chacune de ces orbites a un cardinal qui divise \( | S |=p^m\). Par conséquent \( | \mO_T |\) vaut \( 1\) lorsque \( T\in E_S\) et est un multiple de \( p\) sinon. Nous avons donc
            \begin{equation}
                | E |\equiv | E_S |\mod p.
            \end{equation}
            Nous voulons obtenir \( | E_S |=1\). Évidemment \( S\in E_S\) parce que si \( s\in S\) alors \( sSs^{-1}=S\). Nous voudrions montrer que \( S\) est le seul élément de \( E_S\). Soit \( T\in E_S\), c'est à dire que \( T\) est un $p$-Sylow de \( G\) tel que
            \begin{equation}
                sTs^{-1}=T
            \end{equation}
            pour tout \( s\in S\). Soit \( N\) le groupe engendré par \( S\) et \( T\). Montrons que \( T\) est normal dans \( N\). Un élément \( g\) dans \( N\) s'écrit
            \begin{equation}
                g=s_1t_1\cdots s_rt_r
            \end{equation}
            avec \( s_i\in S\) et \( t_i\in T\). Si \( t\in T\), en utilisant le fait que \( T\) est un groupe et le fait que \( S\) le normalise, nous avons
            \begin{equation}
                gtg^{-1}=s_1t_1\ldots s_rt_rtt_r^{-1}s_r^{-1}\ldots t_1^{-1}s_r^{-1}\in T.
            \end{equation}
            Donc \( T\) est un sous-groupe normal de \( N\). Mais \( S\) et \( T\) sont conjugués dans \( N\) (parce que ils sont des $p$-Sylow de \( N\)), donc il existe un élément \( a\in N\) tel que \( aTa^{-1}=S\). Mais étant donné que \( T\) est normal,
            \begin{equation}
                S=aTa^{-1}=T.
            \end{equation}
            Ceci achève la démonstration des théorèmes de Sylow.

    \end{enumerate}
\end{proof}

\begin{proposition}
    Si \( S\) est un \( p\)-Sylow dans le groupe \( G\) alors pour tout \( g\in G\), l'ensemble \( gSg^{-1}\) est encore un \( p\)-groupe.    
\end{proposition}

\begin{proof}
    Si les éléments de \( S\) sont d'ordre \( p^n\), alors nous avons
    \begin{equation}
        (gsg^{-1})^q=gs^qg^{-1}=e.
    \end{equation}
    Pour avoir \( gs^qg^{-1}=e\), il faut et suffit que \( gs^q=g\), alors \( s^q=e\), c'est à dire \( q=p^n\). Donc \( gSg^{-1}\) est encore un \( p\)-Sylow.
\end{proof}

Les deux résultats \ref{Lemcmbzum} et \ref{PropyfhTmf} proviennent de la \wikiversity{fr}{Groupe_(mathématiques)/Exercice/Premiers_résultats_sur_les_groupes_simples}{wikiversité}.
\begin{lemma}\label{Lemcmbzum}
    Soit \( G\), un groupe fini et \( p\), un nombre premier. Si \( H\) et \( K\) sont des groupes distincts d'ordre \( p\), alors \( H\cap K=\{ e \}\).
\end{lemma}

\begin{proof}
    L'ensemble \( H\cap K\) est un sous-groupe de \( H\). Par conséquent son ordre divise celui de \( H\) qui est un nombre premier. Par conséquent soit \( | H\cap K |=1\), soit \( | H\cap K |=| H |\). Dans le second cas nous aurions \( H=K\), alors que nous avons supposé que \( H\) et \( K\) étaient distincts.
\end{proof}

\begin{proposition} \label{PropyfhTmf}
    Soit \( G\) un groupe fini et \( n\) le nombre de sous-groupes d'ordre \( p\) dans \( G\). Alors le nombre d'éléments d'ordre \( p\) dans \( G\) vaut \( n(p-1)\).
\end{proposition}

\begin{proof}
    Si \( g\) est un élément d'ordre \( p\) dans \( G\), le groupe \( H\) engendré par \( g\) est d'ordre \( p\). Réciproquement si \( H\) est un groupe d'ordre \( p\), tous les éléments de \( H\setminus\{ e \}\) sont d'ordre \( p\) (parce que l'ordre d'un élément divise l'ordre du groupe). Donc l'ensemble des éléments d'ordre \( p\) dans \( G\) est la réunion des ensembles \( H\setminus\{ e \}\) où \( H\) parcours les sous-groupes d'ordre \( p\) dans \( G\). Chacun de ces ensembles possède \( p-1\) éléments et le lemme \ref{Lemcmbzum} nous assure qu'ils sont disjoints. Par conséquent nous avons \( n(p-1)\) éléments d'ordre \( p\) dans \( G\).
\end{proof}

\begin{corollary}
    Un groupe d'ordre premier est cyclique.
\end{corollary}

\begin{proof}
    Soit \( p\) l'ordre de \( G\). Le nombre de sous-groupes d'ordre \( p\) est \( n=1\) (et c'est \( G\) lui-même). La proposition \ref{PropyfhTmf} nous dit alors que le nombre d'éléments d'ordre \( p\) dans \( G\) est \( p-1\). Donc tout élément est générateur.
\end{proof}

\begin{lemma}
    Le groupe \( A_6\) n'accepte pas de sous-groupes normaux d'ordre \( 60\).
\end{lemma}

\begin{proof}
    Soit \( G\) normal dans \( A_6\), et \( a\), un élément d'ordre \( 5\) dans \( G\) (qui existe parce que \( 5\) divise \( 60\)). Soit aussi un élément \( b\) d'ordre \( 5\) dans \( A_6\). Les groupes \( \gr(a)\) et \( \gr(b)\) sont deux \( 5\)-Sylow dans \( A_6\). En effet, \( 5\) un nombre premier et est la plus grande puissance de \( 5\) dans la décomposition de \( 60\); donc \( \gr(a)\) est un \( 5\)-Sylow dans \( G\). D'autre part, l'ordre de \( A_6\) (qui est \( \frac{ 1 }{2}6!\)) ne possède également que \( 5\) à la puissance \( 1\) dans sa décomposition.

    En vertu du théorème de Sylow \ref{ThoUkPDXf}\ref{ItemMzNRVf}, les \( 5\)-Sylow \( \gr(a)\) et \( \gr(b)\) sont conjugués et il existe \( \tau\in A_6\) tel que \( b=\tau a\tau^{-1}\). Mais \( G\) étant normal dans \( A_6\), l'élément \( \tau a\tau^{-1}\) est encore dans \( G\), de telle sorte que \( b\in G\). Du coup \( G\) doit contenir tous les éléments d'ordre \( 5\) de \( A_6\).

    Les éléments d'ordre $5$ de \( A_6\) doivent fixer un des points de \( \{ 1,2,3,4,5,6 \}\) puis permuter les autres de façon à n'avoir qu'un seul cycle. Un cycle correspond à écrire les nombres \( 1,2,3,4,5\) dans un certain ordre. Ce faisant, le premier n'a pas d'importance parce qu'on considère la permutation cyclique, par exemple \( (3,5,2,1,4)\) est la même chose que \( (5,2,1,4,3)\). Le nombre de cycles sur \( \{ 1,2,3,4,5 \}\) est donc de \( 4!\), et par conséquent le nombre d'éléments d'ordre \( 5\) dans \( A_6\) est \( 6\cdot 4!=144\).

    Le groupe \( G\) doit contenir au moins \( 144\) éléments alors que par hypothèse il en contient \( 60\); contradiction.
    
\end{proof}

\begin{proposition}[\cite{Exo7Sylow}]
    Tout groupe simple d'ordre \( 60\) est isomorphe au groupe alterné \( A_5\).
\end{proposition}
Une autre preuve de ce résultat peut être trouvée sur la \wikiversity{fr}{Groupe_(mathématiques)/Exercice/Premiers_résultats_sur_les_groupes_simples}{wikiversité}. 

\begin{proof}
    Nous avons la décomposition en nombres premiers \( 60=2^2\cdot 3\cdot 5\). Déterminons pour commencer le nombre \( n_5\) de \( 5\)-Sylow dans \( G\). Le théorème de Sylow \ref{ThoUkPDXf}\ref{ItemkYbdzZ} nous renseigne que \( n_5\) doit diviser \( 60\) et doit être égal à \( 1\mod 5\). Les deux seules possibilités sont \( n_5=1\) et \( n_5=6\). Étant donné que tous les \( p\)-Sylow sont conjugués, si \( n_5=1\) alors le \( 5\)-Sylow serait un sous-groupe invariant à l'intérieur de $G$, ce qui est impossible vu que \( G\) est simple. Donc \( n_5=6\).

    Par le point \ref{ItemMzNRVf} du théorème de Sylow, le groupe \( G\) agit transitivement sur l'ensemble des \( 5\)-Sylow par l'action adjointe :
    \begin{equation}
        g\cdot S=gSg^{-1}.
    \end{equation}
    Cela donne donc un morphisme \( \theta\colon G\to S_6\). Le noyau de \( \theta\) est un sous-groupe normal. En effet si \( k\in \ker\theta\) et si \( g\in G\) nous avons
    \begin{subequations}
        \begin{align}
            (gkg^{-1})\cdot S&=gkg^{-1} Ggk^{-1}g^{-1}\\
            &=gkTk^{-1}g^{-1}\\
            &=gTg^{-1}\\
            &=S
        \end{align}
    \end{subequations}
    où \( T\) est le Sylow \( T=g^{-1}Sg\). Étant donné que \( k\in \ker\theta\) nous avons utilisé \( kTk^{-1}=aT\). Au final \( gkg^{-1}\cdot S=S\), ce qui prouve que \( gkg^{-1} \in\ker\theta\).

    Étant donné que \( \ker\theta\) est normal dans \( G\), soit est soit réduit à \( \{ e \}\) soit il vaut \( G\). La seconde possibilité est exclue parce qu'elle reviendrait à dire que \( G\) agit trivialement, ce qui n'est pas correct étant donné qu'il agit transitivement. Nous en déduisons que \( \ker\theta=\{ e \}\), que \( \theta\) est injective et que \( G\) est isomorphe à un sous-groupe de \( S_6\).

    Par ailleurs le groupe dérivé de \( G\) est un sous-groupe normal (et non réduit à l'identité parce que \( G\) est non commutatif). Donc \( D(G)=G\). Étant donné que \( G\subset S_6\), nous avons
    \begin{equation}
        G=D(G)\subset D(S_6)=A_6
    \end{equation}
    parce que le groupe dérivé du groupe symétrique est le groupe alterné (lemme \ref{LemiApyfp}).

    L'ensemble \( \theta^{-1}(A_6)\) est distingué dans \( G\). En effet si \( \sigma\in A_6\) et si \( g\in G\) nous avons
    \begin{equation}
        \theta\big( g\theta^{-1}(\sigma)g^{-1} \big)=\theta(g)\sigma \theta(g)^{-1}\in A_6.
    \end{equation}
    Nous en déduisons que \( \theta^{-1}(A_6)\) est soit \( G\) entier soit réduit à \( \{ e \}\). Si \( \theta^{-1}(A_6)=\{ e \}\), alors pour tout \( g\in G\) nous aurions \( g^2=e\) parce que \( \theta(g^2)\in A_6\). L'ordre de \( G\) étant \( 60\), il n'est pas possible que tous ses éléments soient d'ordre \( 2\). Nous en déduisons que \( \theta(G)\subset A_6\).

    Nous nommons \( H=\theta(G)\) et nous considérons l'ensemble \( X=A_6/H\) où les classes sont prises à gauche, c'est à dire 
    \begin{equation}
        [\sigma]=\{ h\sigma\tq h\in H \}.
    \end{equation}
    Évidemment \( A_6\) agit sur \( X\) de façon naturelle. Au niveau de la cardinalité,
    \begin{equation}
        \Card(X)=\frac{ | A_6 | }{ | G | }=\frac{ 360 }{ 60 }=6.
    \end{equation}
    Le groupe \( A_6\) agit sur \( X\) qui a \( 6\) éléments. Nous avons donc une application \( \varphi\colon A_6\to A_6\). Encore une fois, la simplicité de \( A_6\) montre que \( \varphi(A_6)=A_6\).

    Nous étudions maintenant \( \varphi(H)\) agissant sur \( X\). Un élément \( x\in A_6\) fixe la classe de l'unité \( [e]\) si et seulement si \( x\in H\) et par conséquent \( \varphi(H)\) est la fixateur de \( [e]\) dans \( X\). À la renumérotation près, nous pouvons identifier \( \varphi(H)\) au sous-groupe de \( A_6\) agissant sur \( \{ 1,\ldots, 6 \}\) et fixant \( 6\). Nous avons alors \( \varphi(H)=S_5\cap A_6=A_5\). Nous venons de prouver que \( \varphi\) fournit un isomorphisme entre \( A_5\) et \( H\). Étant donné que \( H\) était isomorphe à \( G\), nous concluons que \( G\) est isomorphe à \( A_6\).
\end{proof}

 

%+++++++++++++++++++++++++++++++++++++++++++++++++++++++++++++++++++++++++++++++++++++++++++++++++++++++++++++++++++++++++++
\section{Produits semi-directs}
%+++++++++++++++++++++++++++++++++++++++++++++++++++++++++++++++++++++++++++++++++++++++++++++++++++++++++++++++++++++++++++

%---------------------------------------------------------------------------------------------------------------------------
\subsection{Généralités}
%---------------------------------------------------------------------------------------------------------------------------

Une \defe{suite exacte}{suite!exacte} est une suite d'applications comme suit :
\begin{equation}
    \xymatrix{%
    \cdots \ar[r]^{f_i}&A_i\ar[r]^{f_{i+1}}& A_{i+1}\ar[r]^{f_{i+2}}&\ldots
       }
\end{equation}
où pour chaque \( i\), les application \( f_i\) et \( f_{i+1}\) vérifient \( \ker(f_{i+1})=\Image(f_i)\). Lorsque les ensembles \( A_i\) sont des groupes, alors nous demandons de plus que les \( f_i\) soient de homomorphismes.

Très souvent nous sommes confrontés à des suites exactes de la forme
\begin{equation}
    \xymatrix{%
    1 \ar[r]& A\ar[r]^f&G\ar[r]^g&B\ar[r]&1
       }
\end{equation}
où \( G\), \( A\) et \( B\) sont des groupes, \( 1\) est l'identité. La première flèche est l'application \( \{ 1 \}\to A\) qui à \( 1\) fait correspondre \( 1\). La dernière est l'application \( B\to 1\) qui à tous les éléments de \( B\) fait correspondre \( 1\). Le noyau de \( f\) étant l'image de la première flèche (c'est à dire \( \{ 1 \}\)), l'application \( f\) est injective. L'image de \( g\) étant le noyau de la dernière flèche (c'est à dire \( B\) en entier), l'application \( g\) est surjective.

\begin{definition}
    Soient \( N\) et \( H\) deux groupes et un morphisme de groupes \( \phi\colon H\to \Aut(N)\). Le \defe{produit semi-direct}{produit!semi-direct} de \( N\) et \( H\) relativement à \( \phi\), noté \( N\times_{\phi}H\)\nomenclature[R]{\( N\times_{\phi}H\)}{produit semi-direct} est l'ensemble \( N\times H\) muni de la loi (que l'on vérifiera être de groupe)
    \begin{equation}\label{EqDRgbBI}
        (n,h)\cdot (n',h')=(n\phi_h(n'),hh').
    \end{equation}
\end{definition}

Le théorème suivant permet de reconnaître des produits semi-directs lorsqu'on en voit un.
\begin{theorem}[\cite{MathAgreg}]
    Soit une suite exacte de groupes
    \begin{equation}
    \xymatrix{%
    1 \ar[r]        & N\ar[r]^i&G\ar[r]^s&H\ar[r]&1
       }
    \end{equation}
    Si il existe un sous-groupe \( \tilde H\) de \( G\) à partir duquel \( s\) est un isomorphisme, alors
    \begin{equation}
        G\simeq i(N)\times_{\sigma}\tilde H
    \end{equation}
    où \( \sigma\) est l'action adjointe\footnote{Le fait que \( H\) agisse sur \( i(N)\) fait partie du théorème.} de \( \tilde H\) sur \( i(N)\).
\end{theorem}

\begin{proof}
    Nous posons \( \tilde N=i(N)\) et nous allons subdiviser la preuve en petits pas.

    \begin{enumerate}
        \item  \( \tilde N\) est normal dans \( G\). En effet étant donné que la suite est exacte nous avons \( \tilde N=\ker(s)\). Le noyau d'un morphisme est toujours un sous-groupe normal.

        \item \( \tilde N\cap\tilde H=\{ e \}\). L'application \( s\) étant un isomorphismes depuis $\tilde H$, il n'y a pas d'éléments de \( \tilde H\) dans \( \ker(s)\).
    
        \item\label{ItemzIaXGM} \( G=\tilde N\tilde H\). Nous considérons \( g\in G\) et \( h\in \tilde H\) tel que \( s(g)=s(h)\). L'existence d'un tel \( h\) est assurée par le fait que \( s\) est surjective depuis \( \tilde H\). Du coup nous avons \( e=s(gh^{-1})\), c'est à dire \( gh^{-1}\in \ker (s)=\tilde N\). Nous avons donc bien la décomposition \( g=(gh^{-1})h\), et donc \( G=\tilde N\tilde H\).

        \item\label{ItemUGFjle} L'écriture \( g=nh\) avec \( n\in \tilde N\) et \( h\in \tilde H\) est unique. Si \( nh=n'h'\), alors \( n=n'h'h^{-1}\), ce qui signifierait que \( h'h^{-1}\in\tilde N\). Mais étant donné que \( \tilde H\cap\tilde N=\{ e \}\), nous aurions \( h=h'\) et immédiatement après \( n=n'\).

        \item   \label{ItemUZlrKo}
            L'application
            \begin{equation}
                \begin{aligned}
                    \phi\colon G&\to \tilde N\times \tilde H \\
                    nh&\mapsto (n,h) 
                \end{aligned}
            \end{equation}
            est une bijection. C'est une conséquence des points \ref{ItemzIaXGM} et \ref{ItemUGFjle}.

        \item
            Si sur \( \tilde N\times \tilde H\) nous mettons le produit
            \begin{equation}
                (n,h)\cdot(n',h')=(n\sigma_hn',hh')
            \end{equation}
            où \( \sigma\) est l'action adjointe du groupe sur lui-même, c'est à dire \( \sigma_x(y)=xyx^{-1}\), alors \( \phi\) est un isomorphisme. Si \( g,g'\in G\) s'écrivent (de façon unique par le point \ref{ItemUZlrKo}) \( g=nh\) et \( g'=n'h'\) alors
            \begin{subequations}
                \begin{align}
                    \phi(nhn'h')&=\phi(n\underbrace{hn'h^{-1}}_{\in \tilde N}hh')\\
                    &=\phi\big( (nhn'h^{-1})(hh') \big)\\
                    &=(nhn'h^{-1},hh')\\
                    &=(n,h)\cdot(n',h')\\
                    &=\phi(nh)\phi(n'h').
                \end{align}
            \end{subequations}
    \end{enumerate}
\end{proof}

\begin{corollary}\label{CoroGohOZ}
    Soit \( G\) un groupe, et \( N,H\) des sous-groupes de \( G\) tels que
    \begin{enumerate}
        \item
            \( H\) normalise \( N\) (c'est à dire que \( H\) agit sur \( N\) par automorphismes internes),
        \item
            \( H\cap N=\{ e \}\),

        \item
            \( NH=G\).
    \end{enumerate}
    Alors l'application
    \begin{equation}
        \begin{aligned}
            \psi\colon N\times_{\sigma}H&\to G \\
            (n,h)&\mapsto nh 
        \end{aligned}
    \end{equation}
    est un isomorphisme de groupes.
\end{corollary}

%---------------------------------------------------------------------------------------------------------------------------
\subsection{Groupe diédral}
%---------------------------------------------------------------------------------------------------------------------------
\label{subsecHibJId}

Le \defe{groupe diédral}{groupe!diédral} \( D_n\)\nomenclature[R]{\( D_n\)}{groupe diédral} est le groupe des isométries de \( \eR^2\) laissant invariant un polygone régulier à \( n\) côtés. Il peut être vu comme le stabilisateur de l'ensemble
\begin{equation}
    \{  e^{2ik\pi/n},k=0,\ldots, n-1 \}
\end{equation}
dans le groupe des isométries affines de \( \eC^*\).
% TODO : les racines de l'unité forment un polygone régulier.

\begin{proposition}[\cite{tzHydF}]
    Le groupe \( D_n\) contient un sous groupe cyclique d'ordre \( 2\) et un sous groupe cyclique d'ordre \( n\).
\end{proposition}

\begin{proof}
    Si \( s\) est la conjugaison complexe, alors \( s\) agit sur les racines de l'unité et est d'ordre \( 2\). Donc \( \gr(s)subset D_n\).

    De la même façon, la rotations d'angle \(2\pi/n\), que l'on note \( r\), agit sur les racines de l'unité et engendre un le groupe d'ordre \( n\) des rotations d'angle \(2 k\pi/n\).
\end{proof}

\begin{proposition}[\cite{tzHydF}]
    Nous avons \( (sr)^2=\id\).
\end{proposition}

\begin{proof}
    Si \( z^n=1\), alors
    \begin{equation}
        (srsr)z=srs e^{2 i\pi/n}z=sr\big( e^{-2\pi i/n\bar z}\big)=s\bar z=z.
    \end{equation}
\end{proof}

Afin d'aller plus loin dans l'exploration du groupe diédral, en particulier pour déterminer ses générateurs (proposition \ref{PropLDIPoZ}), nous devons déterminer quelles sont les isométries du plan. Cela sera fait plus tard.

%+++++++++++++++++++++++++++++++++++++++++++++++++++++++++++++++++++++++++++++++++++++++++++++++++++++++++++++++++++++++++++
\section{Un peu de classification}
%+++++++++++++++++++++++++++++++++++++++++++++++++++++++++++++++++++++++++++++++++++++++++++++++++++++++++++++++++++++++++++

%---------------------------------------------------------------------------------------------------------------------------
\subsection{Automorphismes du groupe $\eZ/n\eZ$}
%---------------------------------------------------------------------------------------------------------------------------

Notons que \( \eZ_n=\eZ/n\eZ=\eF_n\) est un groupe pour l'addition tandis que \( \eZ_n^*\) est un groupe pour la multiplication. Il ne peut donc pas y avoir d'équivoque.

\begin{theorem}[\href{https://fr.wikiversity.org/wiki/Groupe\_\%28math\%C3\%A9matiques\%29/Automorphismes\_d'un\_groupe\_cyclique}{Wikiversité}%
    ]   \label{ThoozyeSn}
    Pour chaque \( x\in \eZ_n^*\) nous considérons l'application
    \begin{equation}
        \begin{aligned}
            \sigma_x\colon \eZ_n&\to \eZ_n \\
            y&\mapsto xy. 
        \end{aligned}
    \end{equation}
    L'application
    \begin{equation}
        \sigma\colon (\eZ_n^*,\cdot)\to \Aut\big( (\eZ_n,+) \big)
    \end{equation}
    ainsi définie est un isomorphisme de groupes.
\end{theorem}
L'énoncé de ce théorème s'écrit souvent rapidement par 
\begin{equation}
    \Aut(\eZ/n\eZ)=(\eZ/n\eZ)^*,
\end{equation}
mais il faut bien garder à l'esprit qu'à gauche on considère le groupe additif et à droite celui multiplicatif.

\begin{proof}
    Nous notons \( [x]\) la classe de \( x\) dans \( \eZ/n\eZ\). Nous avons \( \eZ_n=[1]\). Soit \( f\) un automorphisme de \( (\eZ_n,+)\); pour tout \( r\in \eZ\) nous avons
    \begin{equation}
        f([r])=f(r[1])=rf([1])=[r]f([1]).
    \end{equation}
    En particulier, vu que \( f\) est surjective, il existe un \( r\) tel que \( f([r])=[1]\). Pour un tel \( r\) nous avons \( [1]=[r]f([1])\), c'est à dire que nous avons montré que \( f([1])\) est inversible dans \( (\eZ_n^*,\cdot)\). Nous montrons à présent que\footnote{Le \( \sigma\) donné ici est l'inverse de celui donné dans l'énoncé. Cela ne change évidemment rien à la validité de l'énoncé et de la preuve.}
    \begin{equation}
        \begin{aligned}
            \sigma\colon \Aut( (\eZ_n,+))&\to (\eZ_n^*,\cdot) \\
            f&\mapsto f([1]) 
        \end{aligned}
    \end{equation}
    est un isomorphisme.

    Nous commençons par la surjectivité. Soit \( [a]\in \eZ_n^*\). Les élément \( [a]\) et \( [1]\) étant tous deux des générateurs de \( (\eZ_n,+)\), il existe un automorphisme de \( \eZ_n\) qui envoie \( [1]\) sur \( [a]\) par le lemme \ref{LemZhxMit}. Cela prouve la surjectivité de \( \sigma\).

    En ce qui concerne l'injectivité, considérons \( f_1\) et \( f_2\) sont de automorphismes de \( (\eZ_n,+)\) tels que \( f_1([1])=f_2([1])\). Les automorphismes \( f_1\) et \( f_2\) prennent la même valeur sur un générateur et donc sur tout le groupe. Donc \( f_1=f_2\).

    Enfin nous prouvons que \( \sigma\) est un morphisme, c'est à dire que \( \sigma(f\circ g)=\sigma(f)\sigma(g)\). Nous avons
    \begin{subequations}
        \begin{align}
            f\big( g([1]) \big)&=f\big( g([1])[1j] \big)=g([1])f([1])=\sigma(f)\sigma(g).
        \end{align}
    \end{subequations}
\end{proof}

Ce dernier résultat s'étend aux groupes cycliques.
\begin{proposition}
    Si \( G\) est un groupe cyclique d'ordre \( n\), alors
    \begin{equation}
        \Aut(G)=(\eZ/n\eZ)^*.
    \end{equation}
\end{proposition}

\begin{corollary}       \label{CorwgmoTK}
    Si \( p\) divise \( q-1\) alors \( \Aut(\eF_q)\) possède un unique sous-groupe d'ordre \( p\).
\end{corollary}

\begin{proof}
    Si \( a\) est un générateur de \( \eF_q^*\) alors le groupe
    \begin{equation}    \label{EqAdGiil}
        \gr\left( a^{\frac{ q-1 }{ p }} \right)
    \end{equation}
    est un sous-groupe d'ordre \( p\). En ce qui concerne l'unicité, soit \( S\) un sous-groupe d'ordre \( p\). Il est donc d'indice \( (q-1)/p\) dans \( \eF_q^*\) et le lemme \ref{PropubeiGX} nous enseigne que le groupe donné en \eqref{EqAdGiil} est contenu dans \( S\). Il est donc égal à \( S\) parce qu'il a l'ordre de \( S\). Le fait que \( S\) soit normal est dû au fait que \( \eF_q^*\) est abélien.
\end{proof}

%---------------------------------------------------------------------------------------------------------------------------
\subsection{Groupes abéliens finis}
%---------------------------------------------------------------------------------------------------------------------------
 
Source : \cite{FabricegPSFinis}.

Nous rappelons que l'exposant\index{exposant} d'un groupe fini est le \( \ppcm\) des ordres de ses éléments. Dans le cas des groupes abéliens finis, l'exposant joue un rôle important du fait qu'il existe un élément dont l'ordre est l'exposant. Cela est le théorème suivant.

\begin{theorem}[Exposant dans un groupe abélien fini]
    Un groupe abélien fini contient un élément dont l'ordre est l'exposant du groupe.
\end{theorem}

\begin{proof}
    Soit \( G\) un groupe abélien fini et \( x\in G\), un élément d'ordre maximum \( m\). Nous montrons par l'absurde que l'ordre de tous les éléments de \( G\) divise \( m\). Soit donc \( y\in G\), un élément dont l'ordre ne divise pas \( m\); nous notons $q$ son ordre. Vu que \( q\) ne divise pas \( m\), le nombre \( q\) possède au moins un facteur premier plus de fois que \( m\) : soit \( p\) premier tel que la décomposition de \( q\) contienne \( p^{\beta}\) et celle de \( m\) contienne \( p^{\alpha}\) avec \( \beta>\alpha\). Autrement dit,
    \begin{subequations}
        \begin{align}
            m=p^{\alpha}m'\\
            q=p^{\beta}q'
        \end{align}
    \end{subequations}
    où \( m'\) et \( q'\) ne contiennent plus le facteur \( p\). L'élément \( x\) étant d'ordre \( m\), l'élément \( x^{p^{\alpha}}\) est d'ordre \( m'\). De la même manière, l'élément \( y^{q'}\) est d'ordre \( p^{\beta}\). Étant donné que \( p^{\beta}\) et \( m'\) sont premiers entre eux, l'élément  \( x^{p^{\alpha}}y^{q'}\) est d'ordre \( p^{\alpha}m'>m\). D'où une contradiction avec le fait que \( x\) était d'ordre maximal.

    Par conséquent l'ordre de tous les éléments de $G$ divise celui de \( x\) qui est alors le \( \ppcm\) des ordres de tous les éléments de \( G\), c'est à dire l'exposant de \( G\).
\end{proof}

\begin{proposition} \label{PropfPRVxi}
    Soit \( G\) un groupe abélien fini et \( x\in G\), un élément d'ordre maximum. Alors
    \begin{enumerate}
        \item
            Il existe un morphisme \( \varphi\colon G\to \gr(x)\) tel que \( \varphi(x)=x\).
        \item   \label{ItemKRYwjU}
            Il existe un sous-groupe \( K\) de \( G\) tel que \( G=\gr(x)\oplus K\).
    \end{enumerate}
\end{proposition}

\begin{proof}
    Nous notons \( a\) l'ordre de \( x\) qui est également l'exposant du groupe \( G\).

    Nous allons prouver la première partie par récurrence sur l'ordre du groupe. Si \( G=\gr(x)\), alors c'est évident. Soit \( H\) un sous-groupe propre de \( G\) contenant \( x\) et tel que le problème soit déjà résolu pour \( H\) : il existe un morphisme \( \varphi\colon H\to \gr(x)\) tel que \( \varphi(x)=x\). Soit \( y\in G\setminus H\), d'ordre \( b\). Nous allons trouver un morphisme $\hat\varphi\colon \gr(H,y)\to \gr(x) $ telle que \( \hat\varphi(x)=x\).

    Pour cela nous commençons par construire les applications suivantes :
    \begin{equation}
        \begin{aligned}
            \tilde \varphi\colon \eZ/b\eZ\times H&\to \gr(x) \\
            (\bar k,h)&\mapsto x^{kl}\varphi(h) 
        \end{aligned}
    \end{equation}
    où \( l\) est encore à déterminer, et
    \begin{equation}
        \begin{aligned}
            p\colon \eZ/b\eZ\times H&\to \gr(y,H) \\
            (\bar k,h)&\mapsto y^kh. 
        \end{aligned}
    \end{equation}
    Pour que \( \tilde \varphi\) soit bien définie, il faut que \( a\) divise \( bl\). L'application \( p\) est bien définie parce que \( \bar k\) est pris dans \( \eZ/b\eZ\) et que \( b\) est l'ordre de \( y\).

    Nous allons construire le morphisme \( \hat \varphi\) en considérant le diagramme 
    \begin{equation}
    \xymatrix{%
    \ker(p) \ar@{^{(}->}[r]        &   \eZ/b\eZ\times H\ar[d]_{\tilde \varphi}\ar[r]^p&\gr(y,H)\ar[ld]^{\hat \varphi}\\
          &   \gr(x)
       }
    \end{equation}

    que l'on voudra être commutatif. Vu que \( p\) est surjective, les théorèmes d'isomorphismes nous disent que
    \begin{equation}
        \gr(y,H)\simeq\frac{ \eZ/b\eZ\times H }{ \ker p }.
    \end{equation}
    Si \( [\bar k,h]\) est la classe de \( (\bar k,h)\) modulo \( \ker(p)\) alors nous voudrions définir \( \hat \varphi\) par
    \begin{equation}        \label{EqeesVxc}
        \hat\varphi\big( [\bar k,h] \big)=\tilde \varphi(\bar k,h).
    \end{equation}
    Pour que cela soit bien définit, il faut que si \( (\bar r,z)\in \ker p\),
    \begin{equation}
        \hat\varphi\big( [\bar k\bar r,hz] \big)=\hat\varphi\big( [\bar k,h] \big),
    \end{equation}
    c'est à dire que \( \tilde \varphi(\bar r,z)=e\). Du coup la définition \eqref{EqeesVxc} n'est bonne que si et seulement si
    \begin{equation}
        \ker(p)\subset\ker(\tilde\varphi ).
    \end{equation}
    Nous pouvons obtenir cela en choisissant bien \( l\).

    Déterminons d'abord le noyau de \( p\). Pour cela nous considérons un nombre \( \beta\) divisant \( b\) tel que \( \gr(y)\cap H=\gr(y^{\beta})\). Nous aurons \( p(\bar k,h)=e\) si et seulement si \( y^h=e\). En particulier \( h=y^{-k}\in\gr(y)\cap H=\gr(y^{\beta})\). Si \( h=(y^{\beta})^m=y^{m\beta}\), alors \( k=-m\beta\) et nous avons
    \begin{equation}
        \ker(p)=\{ (-m\beta,y^{m\beta})\tq m\in \eZ \}.
    \end{equation}
    En plus court : \( \ker(p)=\gr(\beta,y^{-\beta})\). Nous devons donc fixer \( l\) de telle sorte que \( \tilde \varphi(\beta,y^{-\beta})=e\). Étant donné que \( \varphi\) prend ses valeurs dans \( \gr(x)\), il existe un entier \( \alpha\) tel que \( \varphi(y^{-\beta})=x^{\alpha}\); en utilisant cet \( \alpha\), nous écrivons
    \begin{equation}
        \tilde \varphi(\beta,y^{-\beta})=x^{\beta l}\varphi(y^{-\beta})=x^{\beta l+\alpha}.
    \end{equation}
    Par conséquent nous choisissons \( l=-\alpha/\beta\). Nous devons maintenant vérifier que ce choix est légitime, c'est à dire que \( a\) divise \( bl\) et que \( \alpha/\beta\) est un entier.

    Étant donné que \( y\) est d'ordre \( b\),
    \begin{equation}
        e=\varphi(y^b)=\varphi(y^{-\beta b/\beta})=\varphi(y^{-\beta})^{b/\beta}=x^{b\beta/\alpha}.
    \end{equation}
    Par conséquent \( a\) divise \( \frac{ b\alpha }{ \beta }=-bl\).

    Pour voir que \( l\) est entier, nous nous rappelons que \( a\) est l'exposant de \( G\) (parce que \( x\) est d'ordre maximum) et que par conséquent \( b\) divise \( a\). Mais \( a\) divise \( \alpha\frac{ b }{ \beta }\). Donc \( \alpha/\beta\) est entier.

    Nous passons maintenant à la seconde partie de la preuve. Nous considérons un morphisme \( \varphi\colon G\to \gr(x)\) tel que \( \varphi(x)=x\). La première partie nous en assure l'existence. Nous montrons que 
    \begin{equation}
        \begin{aligned}
            \psi\colon G&\to \gr(x)\oplus \ker(\varphi) \\
            g&\mapsto \big( \varphi(g),g\varphi(g)^{-1} \big) 
        \end{aligned}
    \end{equation}
    est un isomorphisme. D'abord \( g\varphi(g)^{-1}\) est dans le noyau de \( \varphi\) parce que \( \varphi(g)^{-1}\) étant dans \( \gr(x)\), et \( \varphi\) étant un morphisme,
    \begin{equation}
        \varphi\big( g\varphi(g)^{-1} \big)=\varphi(g)\varphi(g)^{-1}=e.
    \end{equation}
    L'application \( \psi\) est un morphisme parce que, en utilisant le fait que \( G\) est abélien,
    \begin{subequations}
        \begin{align}
            \psi(g_1g_2)&=\big( \varphi(g_1g_2),g_1g_2\varphi(g_1g_2)^{-1} \big)\\
            &=\big( \varphi(g_1)\varphi(g_2),g_1\varphi(g_1)^{-1}g_2\varphi(g_2)^{-1} \big)\\
            &=\psi(g_1)\psi(g_2).
        \end{align}
    \end{subequations}
    L'application \( \psi\) est injective parce que si \( \psi(g)=(e,e)\) alors \( \varphi(g)=e\) et \( g\varphi(g)^{-1}=e\), ce qui implique \( g=e\).

    Enfin \( \psi\) est surjective parce qu'elle est injective et que les ensembles de départ et d'arrivée ont même cardinal. En effet par le premier théorème d'isomorphisme (théorème \ref{ThoPremierthoisomo}) appliqué à \( \varphi\) nous avons
    \begin{equation}
        | G |=| \gr(x) |\cdot | \ker(\varphi) |.
    \end{equation}
\end{proof}

\begin{theorem} \label{ThoRJWVJd}
    Tout groupe abélien fini (non trivial) se décompose en
    \begin{equation}
        G\simeq \eZ/d_1\eZ\oplus\ldots\oplus \eZ/d_r\eZ
    \end{equation}
    avec \( d_1\geq 1\) et \( d_i\) divise \( d_{i+1}\) pour tout \( i=1,\ldots, r-1\).

    De plus la liste \( (d_1,\ldots, d_r)\) vérifiant ces propriétés est unique.
\end{theorem}

\begin{proof}
    Soit \( x_1\) un élément d'ordre maximal dans \( G\). Soit \( n_1\) son ordre et
    \begin{equation}
        H_1=\gr(x_1)=\eF_{n_1}.
    \end{equation}
    D'après la proposition \ref{PropfPRVxi}\ref{ItemKRYwjU}, il existe un supplémentaire \( K_1\) tel que \( G=\eF_{n_1}\oplus K_1\). Si \( K_1=\{ 1 \}\) on s'arrête et on garde \( G=\eF_{n_1}\). Sinon on continue de la sorte en prenant \( x_2\) d'ordre maximal dans \( K_1\) etc.

    Nous devons maintenant prouver l'unicité de cette décomposition. Soit
    \begin{equation}
        G=\eF_{d_1}\oplus\ldots\oplus \eF_{d_r}=\eF_{s_1}\oplus\ldots\oplus \eF_{s_q}.
    \end{equation}
    L'exposant de \( G\) est \( d_r\) et \( s_q\). Donc \( d_r=s_q\). Les complémentaires étant égaux nous avons
    \begin{equation}
        \eF_{d_1}\oplus\ldots\oplus \eF_{d_{r-1}}=\eF_{s_1}\oplus\ldots\oplus \eF_{s_{q-1}}.
    \end{equation}
    En continuant nous trouvons \( r=q\) et \( d_i=s_i\).
\end{proof}
 
%---------------------------------------------------------------------------------------------------------------------------
\subsection{Groupes d'ordre $pq$}
%---------------------------------------------------------------------------------------------------------------------------

Soit \( G\) un groupe d'ordre \( pq\) où \( p\) et \( q\) sont des nombres premiers distincts. Nous supposons que \( p<q\). Montrons que \( G\) ne possède qu'un seul \( q\)-Sylow. Soit \( n_q\) le nombre de \( q\)-Sylow; par les théorèmes de Sylow nous avons
\begin{equation}
    n_q=1\mod q
\end{equation}
et \( n_q\) divise \( | G |=pq\). Donc \( n_q\) vaut \( p\), \( q\) ou \( 1\). Avoir \( n_q=p\) n'est pas possible parce que \( n_q=1\mod q\) et \( p<q\). Avoir \( n_q=q\) n'est pas possible non plus, pour la même raison. Donc \( n_q=1\). Notons \( H\) cet unique \( q\)-Sylow de \( G\).

Notons que cet unique \( q\)-Sylow est un sous-groupe normal dans \( G\) qui n'est égal ni à \( \{ 1 \}\) ni à \( \{ G \}\) parce que
\begin{equation}
    1<p=| H |<pq=| G |.
\end{equation}
Par conséquent \( G\) n'est pas simple.

\begin{theorem} \label{ThoLnTMBy}
    Soit \( G\) un groupe d'ordre \( pq\) où \( q>p\) sont des nombres premiers distincts\footnote{Le cas \( p=q\) sera traité par la proposition \ref{PropssttFK}.}. 

    Si \( q\neq 1\mod p\) alors \( G\simeq \eZ/pq\eZ\).

    Si \( q=1\mod p\), alors soit \( G\) est abélien; dans ce cas \( G\simeq \eZ/pq\eZ\), soit \( G\) n'est pas abélien et
    \begin{equation}    \label{EqNuuTRE}
        G\simeq \eZ/q\eZ\times_{\varphi}\eZ/p\eZ
    \end{equation}
    où \( \varphi(\bar 1)\) est d'ordre \( p\) dans \( \Aut(\eZ/q\eZ)\).

    De plus tous les produits semi-directs non triviaux de la forme \eqref{EqNuuTRE} sont isomorphes entre eux, c'est à dire que si \( \eZ/q\eZ\times_{\varphi}\eZ/p\eZ\) et \( \eZ/q\eZ\times_{\varphi'}\eZ/p\eZ\) sont d'ordre \( pq\), alors ils sont isomorphes.

    En particulier si \( p\) et \( q\) sont premiers entre eux, le produit est direct.
\end{theorem}

\begin{proof}
    Soient \( H\), un \( q\)-Sylow et \( K\), un \( p\)-Sylow de \( G\). Ils existent parce que \( p\) et \( q\) sont des diviseurs premiers de \( | G |\) (théorème de Sylow \ref{ThoUkPDXf}). Si \( n_q\) est le nombre de \( q\)-Sylow dans \( G\) alors \( n_q\) divise \( | G |\) et \( n_q=1\mod q\). Donc d'abord \( n_q\) vaut \( 1\), \( p\) ou \( q\). Ensuite \( n_q=q\) est exclu par la condition \( n_q=1\mod q\); la possibilité \( n_q=p\) est également impossible parce que \( p=1\mod q\) est impossible avec \( p<q\). Donc \( n_q=1\) et \( H\) est normal dans \( G\).

    L'ensemble \( H\cap H\) est un sous-groupe à la fois de \( H\) et de \( K\), ce qui entraine que (théorème de Lagrange \ref{ThoLagrange}) \( | H\cap K |\) divise à la fois \( p\) et \( q\). Nous en déduisons que \( | H\cap K |=1\) et donc que \( H\cap K=\{ e \}\).

    Étant donné que \( H\) est normal, l'ensemble \( HK\) est un sous-groupe de \( G\). De plus l'application
    \begin{equation}
        \begin{aligned}
            \psi\colon H\times K&\to HK \\
            (h,k)&\mapsto hk 
        \end{aligned}
    \end{equation}
    est un bijection. Nous ne devons vérifier seulement l'injectivité. Supposons que \( hk=h'k'\). Alors \( e=h^{-1}h'k'k^{-1}\), et donc
    \begin{equation}
        h^{-1} h'=(k'k^{-1})^{-1}\in H\cap K=\{ e \}.
    \end{equation}
    Par conséquent \( | pq |=| H\times K |=| HK |\), et \( HK=G\). Le corollaire \ref{CoroGohOZ} nous indique que
    \begin{equation}    \label{EqGjQjFN}
        G=H\times{\varphi}K
    \end{equation}
    où \( \varphi\) est l'action adjointe. Nous devons maintenant identifier cette action. En d'autres termes, nous savons que \( H=\eZ/q\eZ\) et \( K=\eZ/p\eZ\) et que \( \varphi\colon \eZ/p\eZ\to \Aut(\eZ/q\eZ)\) est un morphisme. Nous devons déterminer les possibilités pour \( \varphi\).

    Soit \( n_p\) le nombre de \( p\)-Sylow de \( G\). Comme précédemment, \( n_p\) vaut \( 1\), \( p\) ou \( q\) et la possibilité \( n_p=p\) est exclue. Donc \( n_p\) est \( 1\) ou \( q\).

    Supposons \( q\neq 1\mod p\), c'est à dire \( q\notin [1]_p\). Dans ce cas \( n_p=q\) est impossible parce que \( n_p\in [1]_p\). Donc \( n_p=1\) et \( K\) est également normal dans \( G\). Du coupe le produit semi-direct \eqref{EqGjQjFN} est en réalité un produit direct (\( \varphi\) est triviale) et nous avons
    \begin{equation}
        G=\eZ/q\eZ\times \eZ/p\eZ=\eZ/pq\eZ.
    \end{equation}
    
    Supposons à présent\footnote{Note : il existe des nombres premiers \( p\) et \( q\) tels que \( q=1\mod p\). Par exemple \( 7=1\mod 3\).} que \( q=1\mod p\). Cette fois \( n_p=1\) et \( n_p=q\) sont tous deux possibles. Ce que nous savons est que \( \varphi(\eZ/p\eZ)\) est un sous-groupe de \( \Aut(\eZ/q\eZ)\). Par le premier théorème d'isomorphisme \ref{ThoPremierthoisomo}, nous avons
    \begin{equation}
        | \varphi(\eZ/p\eZ) |=\frac{ | \eZ/p\eZ | }{ | \ker\varphi | },
    \end{equation}
    ce qui signifie que \( | \varphi(\eZ/p\eZ) |\) divise \( | \eZ/p\eZ |=p\). Par conséquent, \( | \varphi(\eZ/p\eZ) |\) est égal à \( 1\) ou \( p\). Si c'est \( 1\), alors l'action est triviale et le produit est direct.

    Nous supposons que \( | \varphi(\eZ/p\eZ) |=p\). Le corollaire \ref{CorwgmoTK} nous indique que \( \Aut(\eZ/q\eZ)\) possède un unique sous-groupe d'ordre \( p\) que nous notons \( \Gamma\); c'est à dire que \( \Gamma=\Image(\varphi)\). Vu que \( \varphi\colon \eZ/p\eZ\to \Aut(\eZ/q\eZ)\) est un morphisme, \( \Gamma\) est généré par \( \varphi(\bar 1)\) qui est alors un élément d'ordre \( p\), comme annoncé.

    Nous nous attaquons maintenant à l'unicité. Soient \( \varphi\) et \( \varphi'\) deux morphismes non triviaux \( \eZ/p\eZ\to \Aut(\eZ/q\eZ)\). Étant donné que \( \Aut(\eZ/q\eZ)\) ne possède qu'un seul sous-groupe d'ordre \( p\), nous savons que \( \Image(\varphi)=\Image(\varphi')=\Gamma\). Nous pouvons donc parler de \( \varphi'^{-1}\) en tant qu'application de \( \eZ/p\eZ\) dans \( \Gamma\). Nous montrons que
    \begin{equation}
        \begin{aligned}
            f\colon \eZ/q\eZ\times_{\varphi}\eZ/p\eZ&\to \eZ/q\eZ\times_{\varphi'}\eZ/p\eZ \\
            (h,k)&\mapsto (h,\alpha(k)) 
        \end{aligned}
    \end{equation}
    où \( \alpha=\varphi'^{-1}\circ\varphi\) est un isomorphisme de groupes. Le calcul est immédiat :
    \begin{subequations}
        \begin{align}
            f(h_1,k_1)f(h_2mk_2)&=\big( h_1,\alpha(k_1) \big)(h_2,\alpha(k_2))\\
            &=\big( h_1\varphi'(\alpha(k_1))h_2m\alpha(k_1k_2) \big)\\
            &=f\big( h_1\varphi(k_1)h_2,k_1k_2 \big)\\
            &=f\big( (h_1,k_1),(h_2,k_2) \big).
        \end{align}
    \end{subequations}
    Par conséquent \( \eZ/q\eZ\times_{\varphi}\eZ/p\eZ\simeq \eZ/q\eZ\times_{\varphi'} \eZ/p\eZ\).
\end{proof}

\begin{proposition}[\cite{PDFpersoWanadoo}]
    Soit \( G\) un groupe fini d'ordre \( pq\) où \( p\) et \( q\) sont deux nombres premiers distincts vérifiant
    \begin{subequations}
        \begin{numcases}{}
            p\neq 1\mod q\\
            q\neq 1\mod p.
        \end{numcases}
    \end{subequations}
    Alors \( G\) est cyclique, abélien et 
    \begin{equation}
        G\simeq \eZ/p\eZ\times \eZ/q\eZ.
    \end{equation}
\end{proposition}

\begin{proof}
    Soient \( n_p\) et \( n_q\) les nombres de \( p\)-Sylow et \( q\)-Sylow. Par le théorème de Sylow \ref{ThoUkPDXf}, \( n_p\) divise \( pq\) et \( n_p=1\mod p\). Le second point empêche \( n_p\) de diviser \( p\). Par conséquent \( n_p\) divise \( q\) et donc \( n_p\) vaut \( 1\) ou \( q\). La possibilité \( n_p=q\) est exclue par l'hypothèse \( q\neq 1\mod p\). Donc \( n_p=1\), et de la même façon nous obtenons \( n_q=1\).

    Soient \( S\) l'unique \( p\)-Sylow et \( T\), l'unique \( q\)-Sylow. Pour les mêmes raisons que celles exposée plus haut, ce sont deux sous-groupes normaux dans \( G\). Étant donné que \( S\) est d'ordre \( p^n\) pour un certain \( n\) et que l'ordre de \( S\) doit diviser celui de \( G\), nous avons \( |S|=p\). De la même façon, \( | T |=q\). Par conséquent \( S\) est un groupe cyclique d'ordre \( p\) et nous considérons \( x\), un de ses générateurs. De la même façon soit \( y\), un générateur de \( T\).

    Nous montrons maintenant que \( x\) et \( y\) commutent, puis que \( xy\) engendre \( G\). Nous savons que \( S\cap T\) est un sous-groupe à la fois de \( S\) et de \( T\), de telle façon que \( | S\cap T |\) divise à la fois \( | S |=p\) et \( | T |=q\). Nous avons donc \( | S\cap T |=1\) et donc \( S\cap T\) se réduit au neutre. Par ailleurs, \( S\) et \( T\) sont normaux, donc
    \begin{subequations}
        \begin{align}
            (xyx^{-1})y^{-1}\in T\\
            x(yx^{-1})y^{-1})\in S,
        \end{align}
    \end{subequations}
    donc \( xyx^{-1}y^{-1}=e\), ce qui montre que \( xy=yx\). 

    Montrons que \( xy\) engendre \( G\). Soit \( m>0\) tel que \( (xy)^m=e\). Pour ce \( m\) nous avons \( x^m=y^{-m}\) et \( y^{-m}=x^m\), ce qui signifie que \( x^m\) et \( y^m\) appartiennent à \( S\cat T\) et donc \( x^m=y^m=e\). Les nombres \( p\) et \( q\) divisent donc tous deux \( m\); par conséquent \( \ppcm(p,q)=pq\) divise \( m\). Nous en concluons que \( xy\) est d'ordre \( pq\) (il ne peut pas être plus) et qu'il est alors générateur.

    Pour la suite nous allons d'abord prouver que \( G=ST\) puis que \( G\simeq S\times T\). Nous savons déjà que \( | S\cap T |=1\), ce qui nous amène à dire que \( | ST |=| S | |T |\). En effet si \( s,s'\in S\) et \( t,t'\in t\) et si \( st=s't'\), alors \( t=s^{-1}s't'\), ce qui voudrait dire que \( s^{-1}s'\in T\) et donc que \( s^{-1}s'=e\). Au final nous avons
    \begin{equation}
        | ST |=| S | |T |=pq=| G |.
    \end{equation}
    Par conséquent \( G=ST\). En nous rappelant du fait que \( S\cap T=\{ e \}\) et que \( S\) et \( T\) sont normaux, le lemme \ref{LemHUkMxp} nous dit que \( G\simeq S\times T\). Le groupe \( S\) étant cyclique d'ordre \( p\) nous avons \( S=\eZ/p\eZ\) et pour \( T\), nous avons la même chose : \( T=\eZ/q\eZ\). Nous concluons que
    \begin{equation}
        G\simeq \eZ/p\eZ\times \eZ/q\eZ.
    \end{equation}
\end{proof}
 


\begin{theorem}[Théorème de Burnside\cite{FabricegPSFinis}] \label{ThoImkljy}
    Le centre d'un \( p\)-groupe non trivial est non trivial.
\end{theorem}

\begin{proof}
    Soit \( G\) un $p$-groupe non trivial. Nous considérons l'action adjointe \( G\) sur lui-même. Les points fixes de cette action sont les éléments du centre :
    \begin{equation}
        \mZ_G=\{ z\in G\tq \sigma_x(z)=z\forall x\in G \}=\Stab_G(G).
    \end{equation}
    Nous utilisons l'équation aux classes \eqref{PropUyLPdp} pour dire que \( | G |=| \mZ_G |\mod p\). Mais \( | \mZ_G |\) n'est pas vide parce qu'il contient l'identité. Donc \( | \mZ_G |\) est au moins d'ordre \( p\).
\end{proof}

\begin{proposition} \label{PropssttFK}
    Si \( p\) est un nombre premier, tout groupe d'ordre \( p\) ou \( p^2\) est abélien.
\end{proposition}
Rappel : un groupe d'ordre \( p\) ou \( p^2\) est automatiquement un $p$-groupe.

\begin{proof}
    Si \( | G |=p\), alors le théorème de Cauchy \ref{ThoCauchyGpFini} nous donne l'existence d'un élément d'ordre \( p\). Cet élément est alors automatiquement générateur, \( G\) est cyclique et donc abélien.

    Si par contre \( G\) est d'ordre \( p^2\), alors les choses se compliquent (un peu). D'après le théorème de Burnside \ref{ThoImkljy}, le centre \( \mZ\) n'est pas trivial; il est alors d'ordre \( p\) ou \( p^2\). Supposons qu'il soit d'ordre \( p\) et prenons \( x\in G\setminus\mZ\). Alors le stabilisateur de \( x\) pour l'action adjointe contient au moins \( \mZ\) et \( x\), c'est à dire que \( |\Stab_G(x)|\geq p+1\). Étant donné que \( \Stab_G(x)\) est un sous-groupe, son ordre est automatiquement \( 1\), \( p\) ou \( p^2\). En l'occurrence, il doit être \( p^2\) (parce que plus grand que \( p\)), et donc \( x\) doit être central, ce qui est une contradiction.
\end{proof}

%+++++++++++++++++++++++++++++++++++++++++++++++++++++++++++++++++++++++++++++++++++++++++++++++++++++++++++++++++++++++++++
\section{Fonction indicatrice d'Euler}
%+++++++++++++++++++++++++++++++++++++++++++++++++++++++++++++++++++++++++++++++++++++++++++++++++++++++++++++++++++++++++++

%---------------------------------------------------------------------------------------------------------------------------
\subsection{Introduction par les nombres}
%---------------------------------------------------------------------------------------------------------------------------

Pour \( n\in\eN^+\) nous introduisons l'ensemble\nomenclature[R]{\( P_n\)}{les nombres premiers avec \( n\)}
\begin{equation}    \label{EqDefPnEntierldeost}
    P_n=\{ x\in\{ 1,\ldots,n \}\tq\pgcd(x,n)=1 \}.
\end{equation}
C'est l'ensemble des entiers inférieurs à \( n\), premiers avec \( n\). La fonction \( \varphi\) donnée par
\begin{equation}    \label{EqEulerGqPsvi}
    \varphi(n)=\Card(P_n)
\end{equation}
est l'\defe{indicatrice d'Euler}{indicatrice d'Euler}\index{Euler!indicatrice}. Nous avons par exemple
\begin{subequations}
    \begin{align}
        P_1&=\{ 1 \}\\
        P_2&=\{ 1 \}\\
        P_3&=\{ 1,2 \}\\
        P_4&=\{ 1,3 \}\\
        P_7&=\{ 1,2,3,4,5,6 \}\\
        P_8&=\{ 1,3,5,7 \}\\
    \end{align}
\end{subequations}
Si \( p\) est un nombre premier, alors \( \varphi(p)=p-1\).

\begin{proposition}
    Pour tout entier \( r\), nous avons la formule
    \begin{equation}
        r=\sum_{d\divides r}\varphi(d)
    \end{equation}
    où la somme s'étend à tous les diviseurs de \( r\).
\end{proposition}
Une autre preuve sera donné à l'occasion du lemme \ref{LemKcpjee}.

\begin{proof}
    Pour chaque diviseur \( d\) de \( r\) nous considérons l'ensemble
    \begin{equation}
        \Phi_r(d)=\{ n\in \eN\tq \pgcd(r,n)=\frac{ r }{ d } \}.
    \end{equation}
    Étant donné que tous les entiers entre \( 0\) et \( r\) ont un pgcd avec \( r\) qui est automatiquement un quotient de \( r\) nous avons
    \begin{equation}
        \{ 0,\ldots, r \}=\bigcup_{d\divides r}\Phi_r(d)
    \end{equation}
    où l'union est disjointe. Par ailleurs nous savons que si \( \pgcd(a,b)=1\), alors \( \pgcd(ka,kb)=k\). Donc si \( n\in \Delta(d)\), alors \( n\cdot \frac{ r }{ d }\) appartient à \( \Phi_r(d)\). En d'autres termes, \( a\mapsto \frac{ r }{ d }a\) est une bijection entre \( \Delta(d)\) et \( \Phi_r(d)\).

    Nous avons donc \( \Card(\Phi_r(d))=\Card(\Delta(d))=\varphi(d)\) et finalement
    \begin{equation}
        \Card\{ 1,\ldots, r \}=\sum_{d\divides r}\Card(\Phi_r(d))=\sum_{d\divides r}\varphi(d).
    \end{equation}
\end{proof}

%---------------------------------------------------------------------------------------------------------------------------
\subsection{Introduction par les racines de l'unité}
%---------------------------------------------------------------------------------------------------------------------------
\label{SubSechZeTuL}

Une racine \( n\)ième de l'unité est une racine du polynôme \( X^n-1\). Dans \( \eC\) nous avons au maximum \( n\) telles racines, et il est facile de voir qu'il y en a effectivement \( n\) distinctes données par les éléments du groupe multiplicatif
\begin{equation}
    \gU_n=\{ \xi^k\tq k=0,\ldots, n-1 \}
\end{equation}
avec \( \xi= e^{2i\pi/n}\). Nous voyons que \( \gU_n\) est un groupe cyclique d'ordre \( n\) généré par \( \xi\). Il est en particulier isomorphe à \( \eZ/n\eZ\). Le lemme suivant donne les autres générateurs.

\begin{lemma}   \label{LemcFTNMa}
    Le nombre \( \xi^a\) est un générateur de \( \gU_n\) si et seulement si \( \pgcd(a,n)=1\).
\end{lemma}

\begin{proof}
    Si \( \pgcd(a,n)=1\) alors le théorème de Bézout \ref{ThoBuNjam} nous fournit des entiers \( u\) et \( v\) tels que \( ua+vn=1\). Alors nous avons
    \begin{equation}
        e^{2i\pi /n}= e^{2(ua+vn)i\pi/n}=( e^{2ai\pi/n})^u,
    \end{equation}
    ce qui signifie que \( \xi\) est dans le groupe engendré par \( \xi^a\), et par conséquent tout \( \gU_n\) est engendré.

    Pour l'implication inverse, nous utilisons Bézout dans le sens inverse. Soit \( \xi^a\) un générateur de \( \gU_n\). Alors il existe \( u\) tel que \( (\xi^a)^u=\xi\), donc \( \xi^{au-1}=1\), c'est à dire qu'il existe \( v\) tel que \( au-1=vn\). Cette dernière égalité implique que \( \pgcd(a,n)=1\).
\end{proof}

\begin{remark}
Une conséquence tout à fait extraordinaire de ce lemme est que \( 7\) est générateur de \( \eZ/12\eZ\) (parce que \( \pgcd(7,12)=1\)). Or en solfège\index{solfège}, une quinte fait \( 7\) demi-tons, et une gamme en fait 12. Le cycle des quintes est donc générateur de la gamme\cite{YDXsAM}. Cela est un fait connu des pianistes\footnote{Même ceux qui ignorent le théorème de Bézout.} depuis des siècles.
\end{remark}

Les générateurs de \( \gU_n\) sont les \defe{racines primitives}{racine!primitive de l'unité}\footnote{parce qu'en prenant les puissances successives de l'une d'entre elles, nous retrouvons toutes les racines de l'unité, voir aussi la définition \ref{DefnPNCFO}.} de l'unité dans \( \eC\). Nous nommons \( \Delta_n\) leur ensemble :
\begin{equation}
    \Delta_n=\{  e^{2ki\pi/n}\tq 0\leq k\leq n-1,\pgcd(k,n)=1 \}.
\end{equation}
Par définition nous avons
\begin{equation}
    \Card(\Delta_n)=\varphi(n)
\end{equation}
où \( \varphi\) est la fonction d'Euler définie par \eqref{EqEulerGqPsvi}. Nous avons par exemple
\begin{subequations}
    \begin{align}
        \Delta_1&=\{ 1 \}\\
        \Delta_2&=\{  e^{\pi i} \}\\
        \Delta_4&=\{  e^{\pi i/2}, e^{3\pi i/2} \}.
    \end{align}
\end{subequations}
Notons que \( 1\in \Delta_d\) seulement avec \( d=1\).

\begin{lemma}       \label{LemKcpjee}
    Nous avons
    \begin{equation}        \label{EqpZuIyL}
        \gU_n=\bigcup_{d\divides n}\Delta_d
    \end{equation}
    et l'union est disjointe. Nous avons aussi la formule
    \begin{equation}        \label{EqTPHqgJ}
        n=\sum_{d\divides n}\varphi(d).
    \end{equation}
\end{lemma}

\begin{proof}
    À l'application \( x\mapsto  e^{2i\pi x}\) près, nous pouvons considérer
    \begin{equation}
        \Delta_d=\{ \frac{ k }{ d }\tq k=0,\ldots, d-1, \pgcd(k,d)=1 \},
    \end{equation}
    c'est à dire l'ensemble des fractions irréductibles dont le dénominateur est \( d\). L'union des \( \Delta_d\) sera donc disjointe.
    
    Toujours à l'application \( x\mapsto  e^{2i\pi x}\) près, le groupe \( \gU_n\) est donné par
    \begin{equation}
        \gU_n=\{ \frac{ k }{ n }\tq k=0,\ldots, n-1 \}.
    \end{equation}
    L'égalité \eqref{EqpZuIyL} revient maintenant à dire que toute fraction de la forme \( \frac{ k }{ n }\) s'écrit de façon irréductible avec un dénominateur qui divise \( n\).

    La relation \eqref{EqTPHqgJ} consiste à prendre le cardinal des deux côtés de \eqref{EqpZuIyL}. Nous avons \( \Card(\gU_n)=n\) et l'union étant disjointe, à droite nous avons la somme des cardinaux.
\end{proof}

%///////////////////////////////////////////////////////////////////////////////////////////////////////////////////////////
\subsubsection{Propriétés}
%///////////////////////////////////////////////////////////////////////////////////////////////////////////////////////////

\begin{lemma}
    Si \( p\) est un nombre premier, alors \( \varphi(p^n)=p^n-p^{n-1}\).
\end{lemma}

\begin{proof}
    Les éléments de \( \{ 1,\ldots,p^n \}\) qui ont un \( \pgcd\) différent de \( 1\) avec \( p^n\) sont des nombres qui s'écrivent sous la forme \( qp\) avec \( q\leq p^{n-1}\). Il y a évidemment \( p^{n-1}\) tels nombres.

    Par conséquent le cardinal de \( P_{p^n}\) est \( \varphi(p^{n})=p^n-p^{n-1}\).
\end{proof}

\begin{proposition}     \label{PropZnmuphiGensn}
    Soit \( n\in\eN^*\) et le groupe (additif) \( \eZ/n\eZ\). L'élément \( [x]_n\) est un générateur de \( \eZ/n\eZ\) si et seulement si \( x\in P_n\). En particulier \( \eZ/n\eZ\) est un groupe contenant \( \varphi(n)\) générateurs.
\end{proposition}

\begin{proof}
    Nous avons \( \gr\big( [1]_n \big)=\eZ/n\eZ\). L'élément \( [x]_n\) sera générateur si et seulement si il génère \( [1]_n \), c'est à dire si il existe \( p\) tel que \( p[x]_n=[1]_n\). Cette dernière égalité étant une égalité de classes dans \( \eZ_n\), elle sera vraie si et seulement si il existe \( q\) tel que
    \begin{equation}
        px+qn=1.
    \end{equation}
    Cela signifie entre autres que \( x\eZ+n\eZ=\eZ\), c'est à dire que \( \pgcd(x,n)=1\) et que \( x\in P_n\).
\end{proof}

\begin{corollary}       \label{CorlvTmsf}
    L'indicatrice d'Euler est multiplicative : si \( p\) est premier avec \( q\), alors \( \varphi(pq)=\varphi(p)\varphi(q)\). De plus si \( p\) et \( q\) sont premiers entre eux,
    \begin{equation}
        \varphi(pq)=(p-1)(q-1).
    \end{equation}
\end{corollary}

\begin{proof}
    Nous savons que si \( p\) et \( q\) sont premiers entre eux, alors le théorème \ref{ThoLnTMBy} nous donne l'isomorphisme de groupe
    \begin{equation}
        (\eZ/pq\eZ,+)\simeq(\eZ/p\eZ,+)\times(\eZ/q\eZ,+).
    \end{equation}
    Un élément \( (x,y)\) est générateur du produit si et seulement si \( x\) est générateur de \( \eZ/p\eZ\) et \( y\) est générateur de \( \eZ/q\eZ\). Par la proposition \ref{PropZnmuphiGensn}, il y a \( \varphi(p)\varphi(q)\) tels éléments. Par ailleurs le nombre de générateurs de \( \eZ/pq\eZ\) est \( \varphi(pq)\), d'où l'égalité.

    Si \( p\) est premier, nous avons \( \varphi(p)=p-1\) parce que tous les entiers de \( \{ 1,\ldots, p-1 \}\) sont premiers avec \( p\).
\end{proof}

Dans \( \eN\), il y a assez bien de nombres premiers. Nous allons voir maintenant que la somme des inverses des nombres premiers diverge. Pour comparaison, la somme des inverses des carrés converge. Il y a donc plus de nombres premiers que de carrés.
\begin{lemma}   \label{LemheKdsa}
    Un entier \( n\geq 1\) se décompose de façon unique en produit de la forme \( n=qm^2\) où \( q\) est un entier sans facteurs carrés et \( m\), un entier.
\end{lemma}

\begin{proof}
    Pour \( n=1\), c'est évident. Nous supposons \( n\geq 2\).

    En ce qui concerne l'existence, nous décomposons \( n\) en facteurs premiers et nous séparons les puissances paires des puissances impaires :
    \begin{subequations}
        \begin{align}
            n&=\prod_{i=1}^rp_p^{2\alpha_i}\prod_{j=1}^sq_{j}^{2\beta_j+1}\\
            &=\underbrace{\left( \prod_{i=1}^rp_i^{2\alpha_i}\prod_{j=1}^sq^{2\beta_j} \right)}_{m^2}\underbrace{\prod_{j=1}^sq_j}_{q}.
        \end{align}
    \end{subequations}
    
    Nous passons à l'unicité. Supposons que \( n=q_1m_1^2=q_2m_2^2\) avec \( q_1\) et \( q_2\) sans facteurs carrés (dans leur décomposition en facteurs premiers). Soit \( d=\pgcd(m_1,m_2)\) et \( k_1\), \( k_2\) définis par \( m_1=dk_1\), \( m_2=dk_2\). Par construction, \( \pgcd(k_1,k_2)=1\). Étant donné que
    \begin{equation}        \label{EqWPOtto}
        n=q_1d^2k_1^2=q_2d^2k_2^2,
    \end{equation}
    nous avons \( q_1k_1^2=q_2k_2^2\) et donc \( k_1^2\) divise \( q_2k_2^2\). Mais \( k_1\) et \( k_2\) n'ont pas de facteurs premiers en commun, donc \( k_1^2\) divise \( q_2\), ce qui n'est possible que si \( k_1=1\) (parce que \( k_1^2\) n'a que des facteurs premiers alors que \( q_2\) n'en a pas). Dans ce cas, \( d=m_1\) et \( m_1\) divise \( m_2\). Si \( m_2=lm_1\) alors l'équation \eqref{EqWPOtto} se réduit à  \( n=q_1m_1^2=q_2l^2m_1^2\) et donc
    \begin{equation}
        q_1=q_2l^2,
    \end{equation}
    ce qui signifie \( l=1\) et donc \( m_1=m_2\).

\end{proof}

\begin{theorem} \label{ThonfVruT}
    Soit \( P\), l'ensemble des nombres premiers. Alors la somme \( \sum_{p\in P}\frac{1}{ p }\) diverge et plus précisément,
    \begin{equation}
        \sum_{\substack{p\leq x\\p\in P}}\frac{1}{ p }\geq \ln(\ln(x))-\ln(2).
    \end{equation}
\end{theorem}

\begin{proof}
    Nous posons
    \begin{equation}
        S_x=\{ \text{\( q\leq x\) avec \( q\) sans facteurs carrés} \}
    \end{equation}
    et
    \begin{equation}
        P_x=\{ p\in P\tq p\leq x \}.
    \end{equation}
    Si
    \begin{equation}
        K_x=\{ \text{\( (q,m)\) tels que \( q\) n'a pas de facteurs carrés et \( qm^2\leq x\)} \},
    \end{equation}
    alors nous avons
    \begin{equation}
        K_x=\bigcup_{q\in S_x}\bigcup_{m\leq \sqrt{x/q}}(q,m).
    \end{equation}
    Par définition et par le lemme \ref{LemheKdsa} nous avons aussi
    \begin{equation}
        \{ n\leq x \}=\{ qm^2\tq (q,m)\in K_x \}.
    \end{equation}
    Tout cela pour décomposer la somme
    \begin{equation}        \label{EqpoJpuC}
        \sum_{n\leq x}\frac{1}{ n }=\sum_{q\in S_x}\sum_{m\leq\sqrt{x/q}}\frac{1}{ m^2 }\leq \sum_{q\in S_x}\frac{1}{ q }\underbrace{\sum_{m\geq 1}\frac{1}{ m^2 }}_{=C}.
    \end{equation}
    Nous avons aussi
    \begin{subequations}
        \begin{align}
            \prod_{p\in P_x}\left( 1+\frac{1}{ p } \right)&=1+\sum_{p\in P_x}\frac{1}{ p }+\sum_{\substack{p,q\in P_x\\p<q}}\frac{1}{ pq }+\sum_{\substack{p,q,r\in P_x\\p<q<r}}\frac{1}{ pqr }+\ldots\\
            &\geq 1+\sum_{p\in P_x}\frac{1}{ p }+\sum_{\substack{p,q\in P_x\\pq\leq x}}\frac{1}{ pq }+\sum_{\substack{p,q,r\in P_x\\pqr\leq x}}\frac{1}{ pqr }+\ldots
        \end{align}
    \end{subequations}
    Les sommes sont finies. Les sommes s'étendent sur toutes les façons de prendre des produits de nombres premiers distincts de telle sorte de conserver un produit plus petit que \( x\); c'est à dire que les sommes se résument en une somme sur les éléments de \( S_x\) :
    \begin{equation}        \label{EqooilOz}
        \exp\left( \sum_{p\in P_x}\frac{1}{ p } \right)\geq\prod_{p\in P_x}\left( 1+\frac{1}{ p } \right)\geq \sum_{q\in S_x}\frac{1}{ q }.
    \end{equation}
    La première inégalité est simplement le fait que \( 1+u\leq e^u\) si \( u\geq 0\). Nous prolongeons maintenant les inégalités
    \begin{equation}
        \ln(x)\leq \sum_{n\leq x}\int_{n}^{n+1}\frac{dt}{ t }\leq \sum_{n\geq x}\frac{1}{ n }
    \end{equation}
    avec les inégalités \eqref{EqpoJpuC} et \eqref{EqooilOz} :
    \begin{equation}
        \ln(x)\leq \sum_{n\geq}\frac{1}{ n }\leq C\sum_{q\in S_x}\frac{1}{ q }\leq C\leq \exp\left( \sum_{p\in P_x}\frac{1}{ p } \right).
    \end{equation}
    En passant au logarithme,
    \begin{equation}
        \ln\big( \ln(x) \big)\leq\ln(C)+\sum_{p\in P_x}\frac{1}{ p }.
    \end{equation}
    Ceci montre la divergence de la série de droite. Nous cherchons maintenant une borne pour \( C\). Pour cela nous écrivons
    \begin{subequations}
        \begin{align}
            \sum_{n=1}^N\frac{1}{ n^2 }&\leq 1+\sum_{n=2}\frac{1}{ n(n-1) }\\
            &=1+\sum_{n=2}^N\left( \frac{1}{ n-1 }-\frac{1}{ n } \right)\\
            &=1+1-\frac{1}{ N }\\
            &\leq 2.
        \end{align}
    \end{subequations}
    Donc \( C\leq 2\).
\end{proof}

%---------------------------------------------------------------------------------------------------------------------------
\subsection{Groupe monogène}
%---------------------------------------------------------------------------------------------------------------------------

\begin{theorem}
    Un groupe monogène est abélien. Plus précisément,
    \begin{enumerate}
        \item
            un groupe monogène infini est isomorphe à \( \eZ\),
        \item
            un groupe monogène fini est isomorphe à \( \eZ_n\) pour un certain \( n\).
    \end{enumerate}
    Un groupe monogène d'ordre \( n\) possède \( \varphi(n)\) générateurs.
\end{theorem}

\begin{proof}
    Le groupe est abélien parce que $g=a^n$, \( g'=a^{n'}\) implique \( gg'=q^{n+n'}=g'g\). Nous considérons un générateur \( a\) de \( G\) (qui existe parce que $G$ est monogène) et le morphisme surjectif
    \begin{equation}
        \begin{aligned}
            f\colon \eZ&\to G \\
            p&\mapsto a^p. 
        \end{aligned}
    \end{equation}
    Si \( G\) est infini, alors \( f\) est injective parce que si \( a^n=a^{n'}\), alors \( a^{n-n'}=e\), ce qui rendrait \( G\) cyclique et par conséquent non infini. Nous concluons que si \( G\) est infini, alors \( f\) est une bijection et donc un isomorphisme \( \eZ\simeq G\).

    Si \( G\) est fini, alors \( f\) n'est pas injective et a un noyau \( \ker f\). Étant donné que \( \ker f\) est un sous-groupe de \( G\), il existe un (unique) \( n\) tel que \( \ker f=n\eZ\) et le premier théorème d'isomorphisme (théorème \ref{ThoPremierthoisomo}) nous indique que
    \begin{equation}
        \eZ/\ker f=\eZ/n\eZ=\Image f=G.
    \end{equation}
    Dans ce cas, le fait qu'un groupe monogène d'ordre \( n\) possède \( \varphi(n)\) générateurs est le contenu de la proposition \ref{PropZnmuphiGensn}.
\end{proof}

%+++++++++++++++++++++++++++++++++++++++++++++++++++++++++++++++++++++++++++++++++++++++++++++++++++++++++++++++++++++++++++
\section{Action de groupe et connexité}
%+++++++++++++++++++++++++++++++++++++++++++++++++++++++++++++++++++++++++++++++++++++++++++++++++++++++++++++++++++++++++++

Sources : \cite{MneimneLie} et \wikipedia{fr}{Matrice_normale}{wikipédia}.

\begin{theorem}     \label{ThojrLKZk}
    Soit \( G\) un groupe topologique localement compact et dénombrable à l'infini\footnote{Cela signifie qu'il est une réunion dénombrable de compacts} agissant continument et transitivement sur un espace topologique localement compact \( E\). Alors l'application
    \begin{equation}
        \begin{aligned}
            \varphi\colon G/G_x&\to E \\
            [g]&\mapsto g\cdot x 
        \end{aligned}
    \end{equation}
    est un homéomorphisme.
\end{theorem}

\begin{lemma}       \label{LemkLRAet}
    Si \( G\) et \( H\) sont des groupes topologiques tels que $G/H$ et \( H\) sont connexes, alors \( G\) est connexe.
\end{lemma}

\begin{proof}
    Soit \( f\colon G\to \{ 0,1 \}\) une fonction continue. Considérons l'application
    \begin{equation}
        \begin{aligned}
            \tilde f\colon G/H&\to \{ 0,1 \} \\
            [g]&\mapsto f(g). 
        \end{aligned}
    \end{equation}
    D'abord nous montrons qu'elle est bien définie. En effet si \( h\in H\) nous aurions \( \tilde f([gh])=f(gh)\), mais étant donné que \( H\) est connexe, l'ensemble \( gH\) est également connexe, de telle façon à ce que la fonction continue \( f\) soit constante sur \( gH\). Nous avons donc \( f(gh)=f(g)\).

    Étant donné que \( G/H\) est également connexe, la fonction \( \tilde f\) doit être constante. Si \( g_1\) et \( g_2\) sont deux éléments du groupe, nous avons \( f(g_1)=\tilde f([g_1])=\tilde f([g_2])=f(g_2)\). Nous en déduisons que \( f\) est constante et que \( G\) est connexe.
\end{proof}

\begin{theorem}
    Le groupe \( \SO(n)\) est connexe, le groupe \( \gO(n)\) a deux composantes connexes.
\end{theorem}

\begin{proof}
    La seconde assertion découle de la première parce que les matrices de déterminant \( 1\) et celles de déterminant \( -1\) ne peuvent pas être reliées par un chemin continu tandis que l'application
    \begin{equation}
        M\mapsto \begin{pmatrix}
            -1    &       &       \\
                &   1    &       \\
                &       &   1
        \end{pmatrix}M
    \end{equation}
    est un homéomorphisme entre les matrices de déterminant \( 1\) et celles de déterminants \( -1\). Montrons donc que \( G=\SO(n)\) est connexe par arcs pour \( n\geq 2\) en procédant par récurrence sur la dimension.
    
    Nous acceptons le résultat pour $G=\SO(2)$. Notons que nous en avons besoin pour prouver que la sphère \( S^{n-1}\) est connexe.
    
    Le groupe \( \SO(n)\) agit, par définition, de façon transitive sur la sphère \( S^{n-1}\). Soit \( a\in S^{n-1}\), nous avons
    \begin{subequations}
        \begin{align}
            G\cdot a&=S^{n-1}\\
            G_a&\simeq \SO(n-1)
        \end{align}
    \end{subequations}
    où \( G_a\) est le fixateur de \( a\) dans \( G\). Pour montrer le second point, nous considérons \( \{ e_i \}\), la base canonique de \( \eR^n\) et \( M\in G\) telle que \( Ma=e_1\). Le fixateur de \( e_1\) est évidemment isomorphe à \( \SO(n-1)\) parce qu'il est constitué des matrices de la forme
    \begin{equation}
        \begin{pmatrix}
             1   &   0    &   \ldots    &   0    \\
             0   &   a_{11}    &   \ldots    &   a_{1,n-1}    \\
             \vdots   &   \vdots    &   \ddots    &   \vdots    \\ 
             0   &   a_{n-1,1}    &   \ldots    &   a_{n-1,n-1}     
         \end{pmatrix}
    \end{equation}
    où \( (a_{ij})\in \SO(n-1)\). L'application 
    \begin{equation}
        \begin{aligned}
            \alpha\colon G_{e_1} &\to G_{a} \\
            A&\mapsto M^{-1}A M
        \end{aligned}
    \end{equation}
    est un isomorphisme entre \( G_a\) et \( \SO(n-1)\). Le théorème \ref{ThojrLKZk} nous montre alors que, en tant qu'espaces topologiques,
    \begin{equation}
        G/G_a=S^{n-1}.
    \end{equation}
    L'hypothèse de récurrence montre que \( G_a=\SO(n-1)\) est connexe tandis que nous savons que \( S^{n-1}\) est connexe. Le lemme \ref{LemkLRAet} conclu que \( G=\SO(n)\) est connexe.
\end{proof}

\begin{lemma}       \label{LemIbrsFT}
    Une bijection continue entre un espace compact et un espace séparé est un homéomorphisme.
\end{lemma}

\begin{proposition}
    Les groupes \( \gU(n)\) et \( \SU(n)\) sont connexes.
\end{proposition}

\begin{proof}
    Soit \( G(n)\) le groupe \( \SU(n)\) ou \( \gU(n)\). Ce groupe opère transitivement sur la sphère complexe
    \begin{equation}
        S_{\eC}^{n-1}=\{ z\in \eC^n\tq \langle z, z\rangle=\sum_k| z_k |^2 =1 \}.
    \end{equation}
    Cet ensemble est le même que \( S^{2n-1}\) parce que \( |z_k|=x_k^2+y_k^2\). Nous avons une bijection continue entre \( S^{n-1}\) et \( S^{n-1}_{\eC}\) et donc un homéomorphisme (lemme \ref{LemIbrsFT}). Soit \( a\in S^{n-1}_{\eC}\), nous avons
    \begin{subequations}
        \begin{align}
            G\cdot a&=S^{n-1}_{\eC}\\
            G_a&\simeq G(n-1).
        \end{align}
    \end{subequations}
    La seconde ligne est un isomorphisme de groupe et un homéomorphisme. Il est donné de la façon suivante. D'abord le fixateur de \( e_1\) dans \( G(n)\) est donné par les matrices de la forme
    \begin{equation}
        \begin{pmatrix}
             1   &   0    &   \ldots    &   0    \\
             0   &   a_{11}    &   \ldots    &   a_{1,n-1}    \\
             \vdots   &   \vdots    &   \ddots    &   \vdots    \\ 
             0   &   a_{n-1,1}    &   \ldots    &   a_{n-1,n-1}     
         \end{pmatrix}
    \end{equation}
    où \( (a_{ij})\in G(n-1)\). Par ailleurs si \( M\) est une matrice de \( G(n)\) telle que \( Ma=e_1\), nous avons l'homéomorphisme
  
    \begin{equation}
        \begin{aligned}
            \alpha\colon G_{e_1}&\to G_a \\
            A&\mapsto M^{-1} AM. 
        \end{aligned}
    \end{equation}
    Encore une fois, cela est un homéomorphisme par le lemme \ref{LemIbrsFT}. Par composition nous avons \( G_a\simeq G(n-1)\) et un homéomorphisme
    \begin{equation}
        G(n)/G_a=S^{n-1}_{\eC}.
    \end{equation}
    Le groupe \( G_a\) et l'ensemble \( S^{n-1}_{\eC}\) étant connexes, le groupe \( G(n)\) est connexe par le lemme \ref{LemkLRAet}.
\end{proof}

%+++++++++++++++++++++++++++++++++++++++++++++++++++++++++++++++++++++++++++++++++++++++++++++++++++++++++++++++++++++++++++
\section{Groupe de torsion}
%+++++++++++++++++++++++++++++++++++++++++++++++++++++++++++++++++++++++++++++++++++++++++++++++++++++++++++++++++++++++++++

Soit \( G\) un groupe. Un élément \( g\in G\) est un \defe{élément de torsion}{élément!de torsion} si il est d'ordre fini. La \defe{torsion}{torsion!d'un groupe} de \( G\) est l'ensemble de ses éléments de torsion. Nous disons qu'un groupe est un \defe{groupe de torsion}{groupe!de torsion} si tous ses éléments sont de torsion.

\begin{example}
    Le groupe additif \( \eQ/\eZ\) est un groupe de torsion parce que si \( [x]=[p/q]\), alors \( q[x]=[p]=[0]\).
\end{example}

\section{Famille presque nulle}
%+++++++++++++++++++++++++++++++++++++++++++++++++++++++++++++++++++++++++++++++++++++++++++++++++++++++++++++++++++++++++++

Soit \( (G,+)\) un groupe abélien et \( \mF=\{ g_i \}_{i\in I}\) une famille d'éléments de \( G\) indicés par un ensemble \( I\). Le \defe{support}{support!famille d'éléments} de \( \mF\) est l'ensemble \( \{ i\in I\tq g_i\neq 0 \}\). La famille est dite \defe{presque nulle}{presque nulle} si le support est fini.

Nous disons que \( \mF\) est une \defe{suite}{suite} si \( I=\eN\).
