% This is part of Mes notes de mathématique
% Copyright (c) 2012-2013
%   Laurent Claessens, Carlotta Donadello
% See the file fdl-1.3.txt for copying conditions.

%+++++++++++++++++++++++++++++++++++++++++++++++++++++++++++++++++++++++++++++++++++++++++++++++++++++++++++++++++++++++++++
					\section{Topologie en général}
%+++++++++++++++++++++++++++++++++++++++++++++++++++++++++++++++++++++++++++++++++++++++++++++++++++++++++++++++++++++++++++

\begin{definition}		\label{DefTopologieGene}
Soit $E$, un ensemble et $\mT$, une partie de l'ensemble de ses parties qui vérifie les propriétés suivantes
\begin{enumerate}

\item
les ensembles $\emptyset$ et $E$ sont dans $\mT$,

\item
    Une union quelconque d'éléments de \( \mT\) est dans \( \mT\).
\item
    Une intersection \emph{finie} d'éléments de \( \mT\) est dans \( \mT\).

\end{enumerate}
Un tel choix $\mT$ de sous-ensembles de $E$ est une  \defe{\href{http://fr.wikipedia.org/wiki/Espace_topologique}{topologie}}{topologie} sur $E$, et les éléments de $\mT$ sont appelés des \defe{ouverts}{ouvert}. Nous disons que un sous ensemble $A$ de $E$ est \defe{fermé}{fermé} si son complémentaire, $A^c$ est ouvert.
\end{definition}

\begin{lemma}   \label{LemQYUJwPC}
    Une intersection quelconque de fermés est fermée.
\end{lemma}

\begin{proof}
    Soit \( \{ F_i \}_{i\in I} \) est un ensemble de fermés; nous avons
    \begin{equation}
        \left( \bigcap_{i\in I}F_i \right)^c=\cup_{i\in I}F_i^c.
    \end{equation}
    L'union de droite est un ouvert (union d'ouvert), et donc le terme de gauche est le complémentaire d'un ouvert. Donc fermé.
\end{proof}

\begin{definition}
    Un \defe{homéomorphisme}{homéomorphisme} est une application bijective continue entre deux espaces topologiques dont la réciproque est continue. Deux espaces topologiques homéomorphes sont dits \defe{isomorphes}{isomorphisme!d'espaces topologiques}.
\end{definition}
%TODO : il faut mettre ici la définition de la continuité pour un espace topologique.

\begin{definition}
    Si deux points distincts ont toujours deux voisinages distincts, nous disons que l'espace est \defe{séparé}{séparé} ou \defe{Hausdorff}{Hausdorff}.
\end{definition}

Dès que nous avons une topologie, nous avons une notion de convergence de suite : nous disons qu'une suite $x_n$ d'éléments de $E$ \defe{converge}{convergence!en topologie} vers l'élément $x$ de $E$ si pour tout ouvert $\mO$ contenant $x$, il existe un $K$ tel que $k>K$ implique $x_k\in\mO$. Cette définition est exactement celle donnée pour la convergence de suites dans $\eR^n$, à part que nous avons remplacé le mot \og boule\fg{} par \og ouvert\fg.

Dans un espace topologique, nous avons une caractérisation très importante des ouverts.
\begin{theorem}		\label{ThoPartieOUvpartouv}
    Une partie $A$ de $E$ est ouverte si et seulement si pour tout $x\in A$, il existe un ouvert autour de $x$ contenu dans $A$.
\end{theorem}

\begin{proof}
Le sens direct est évident : $A$ lui-même est un ouvert autour de $x\in A$, qui est inclus à $A$.

Pour le sens inverse, pour chaque $x\in A$, nous considérons l'ensemble $\mO_x\subset A$, un ouvert autour de $x$. Nous avons que
\begin{equation}	\label{EqAUniondesOx}
	A=\bigcup_{x\in A}\mO_x.
\end{equation}
En effet $A\subset\bigcup_{x\in A}\mO_x$ parce que tous les éléments de $A$ sont dans un des $\mO_x$, par construction. D'autre part, $\bigcup_{x\in A}\mO_x\subset A$ parce que chacun des $\mO_x$ est compris dans $A$.

L'union du membre de droite de \eqref{EqAUniondesOx} est une union d'ouverts et est donc un ouvert. Cela prouve que $A$ est un ouvert.

\end{proof}
Une utilisation typique de ce théorème est faite dans le lemme \ref{LemMESSExh}.

\begin{definition}  \label{DefIQbkyDP}
    Une suite \( (x_n)\) dans un espace topologique \( X\) \defe{converge}{convergence!dans un espace topologique} vers \( x\in X\) si pour tout ouvert \( \mO\) autour de \( x\), il existe \( N\in \eN\) tel que \( n\geq N\) implique \( x_n\in \mO\).
\end{definition}

%+++++++++++++++++++++++++++++++++++++++++++++++++++++++++++++++++++++++++++++++++++++++++++++++++++++++++++++++++++++++++++ 
\section{Fonctions continues}
%+++++++++++++++++++++++++++++++++++++++++++++++++++++++++++++++++++++++++++++++++++++++++++++++++++++++++++++++++++++++++++

La définition suivante est \emph{la} définition de la continuité dans tous les cas.
\begin{definition}  \label{DeOLNtrxB}
    Si \( X\) et \( Y\) sont des espaces topologiques, une application \( f\colon X\to Y\) est \defe{continue}{continue!fonction entre espaces topologiques} si pour tout ouvert \( \mO\) dans \( Y\), l'ensemble \( f^{-1}(\mO)\) est ouvert dans \( X\).
\end{definition}

%+++++++++++++++++++++++++++++++++++++++++++++++++++++++++++++++++++++++++++++++++++++++++++++++++++++++++++++++++++++++++++
\section{Topologie induite}
%+++++++++++++++++++++++++++++++++++++++++++++++++++++++++++++++++++++++++++++++++++++++++++++++++++++++++++++++++++++++++++

\begin{definition}  \label{DefVLrgWDB}
Soit \( X\) un espace topologique et \( A\subset X\). L'ensemble \( A\) devient un espace topologique en lui-même par la \defe{topologie induite}{topologie!induite} de \( X\). Un ouvert de \( A\) est un ensemble de la forme \( A\cap\mO\) où \( \mO\) est un ouvert de \( X\).
\end{definition}

\begin{lemma}       \label{LemkUYkQt}
    Si \( B\subset A\) alors la fermeture de \( B\) pour la topologie de \( A\) (induite de \( X\)) que nous noterons \( \tilde B\) est 
    \begin{equation}
        \tilde B=\bar B\cap A
    \end{equation}
    où \( \bar B\) est la fermeture de \( B\) pour la topologie de \( X\).
\end{lemma}

\begin{proof}
    Si \( a\in \bar B\cap A\), un ouvert de \( A\) autour de \( a\) est un ensemble de la forme \( \mO\cap A\) où \( \mO\) est un ouvert de \( X\). Vu que \( a\in\bar B\), l'ensemble \( \mO\) intersecte \( B\) et donc \( (\mO\cap A)\cap B\neq \emptyset\). Donc \( a\) est bien dans l'adhérence de \( B\) au sens de la topologie de \( A\).

    Pour l'inclusion inverse, soit \( a\in \tilde  B\), et montrons que \( a\in \bar B\cap A\). Par définition \( a\in A\), parce que \( \tilde B\) est une fermeture dans l'espace topologique \( A\). Il faut donc seulement montrer que \( a\in\bar B\). Soit donc \( \mO\) un ouvert de \( X\) contenant \( a\); par hypothèse \( \mO\cap A\) intersecte \( B\) (parce que tout ouvert de \( A\) contenant \( a\) intersecte \( B\)). Donc \( \mO\) intersecte \( B\). Cela signifie que tout ouvert (de \( X\)) contenant \( a\) intersecte \( B\), ou encore que \( a\in \bar B\).
\end{proof}

\begin{example} \label{ExloeyoR}
    Si \( A\) est un ouvert de \( X\), on pourrait croire que la topologie induite n'a rien de spécial. Il est vrai que \( B\) sera ouvert dans \( A\) si et seulement si il est ouvert dans \( X\), mais des choses se passent quand même. Prenons \( X=\eR\) et \( A=\mathopen] 0 , 1 \mathclose[\). Si \( B=\mathopen] \frac{ 1 }{2} , 1 \mathclose[ \), alors la fermeture de \( B\) dans \( A\) sera \( \tilde B=\mathopen[ \frac{ 1 }{2} , 1 [\) et non \( \mathopen[ \frac{ 1 }{2} , 1 \mathclose]\) comme on l'aurait dans \( \eR\).
\end{example}

\begin{lemma}   \label{LemPESaiVw}
    Soit \( A\subset X\) muni de la topologie induite de \( X\) et \( (x_n)\) une suite dans \( A\). Si \( x_n\stackrel{A}{\longrightarrow}x\), alors \( x_n\stackrel{X}{\longrightarrow}x\). 
\end{lemma}

\begin{proof}
    Soit \( \mO\) un ouvert autour de \( x\) dans \( X\). Alors \( A\cap\mO\) est un ouvert autour de \( x\) dans \( A\) et il existe \( N\in \eN\) tel que si \( n\geq N\), alors \( x_n\in A\cap\mO\subset\mO\).
\end{proof}

%+++++++++++++++++++++++++++++++++++++++++++++++++++++++++++++++++++++++++++++++++++++++++++++++++++++++++++++++++++++++++++ 
\section{Compacité}
%+++++++++++++++++++++++++++++++++++++++++++++++++++++++++++++++++++++++++++++++++++++++++++++++++++++++++++++++++++++++++++

\begin{definition}
  Une partie $A$ d'un espace topologique est \defe{compacte}{compact} si il vérifie la propriété de Borel-Lebesgue : pour tout recouvrement de $A$ par des ouverts (c'est-à-dire une collection d'ouverts dont la réunion contient $A$) on peut tirer un recouvrement fini.
\end{definition}
\begin{remark}
    Certaines sources (dont \wikipedia{fr}{Compacité_(mathématiques)}{wikipédia}) disent que pour être compact il faut aussi être \defe{séparable}{séparable} (c'est à dire que deux points distincts ont des voisinages distincts). Pour ces sources, un espace qui ne vérifie que la propriété de Borel-Lebesgue est alors dit \defe{quasi-compact}{quasi-compact}\index{compact!quasi}.
\end{remark}

\begin{definition}
    Une famille \( \mA\) de parties de \( X\) a la \defe{propriété d'intersection finie non vide}{propriété d'intersection non vide} si tout sous-ensemble fini de \( \mA\) a une intersection non vide.
\end{definition}

\begin{proposition}\label{PropXKUMiCj}
    Soit \( X\) un espace topologique et \( K\subset X\). Les propriétés suivantes sont équivalentes :
    \begin{enumerate}
        \item\label{ItemXYmGHFai}
            \( K\) est compact.
        \item\label{ItemXYmGHFaii}
            Si \( \{ F_i \}\) est une famille de fermés telle que \( K\bigcap_{i\in I}F_i=\emptyset\), alors il existe un sous-ensemble fini \( A\) de \( I\) tel que \( K\bigcap_{i\in A}F_i=\emptyset\).
        \item\label{ItemXYmGHFaiii}
            Si \( \{ F_i \}_{i\in I}\) est une famille de fermés telle que \( K\bigcap_{i\in A}F_i\neq\emptyset\) pour tout choix de \( A\) fini dans \( I\), alors l'intersection complète est non vide : \( K\bigcap_{i\in I}F_i\neq\emptyset\).
        \item\label{ItemXYmGHFaiv}
            Toute famille ayant la propriété d'intersection finie non vide a une intersection non vide.
    \end{enumerate}
\end{proposition}

\begin{proof}
    Les propriétés \ref{ItemXYmGHFaiii} et \ref{ItemXYmGHFaii} sont équivalentes par contraposition. De plus le point \ref{ItemXYmGHFaiv} est une simple reformulation en français de la propriété \ref{ItemXYmGHFaiii}.

    Prouvons \ref{ItemXYmGHFai} \( \Rightarrow\) \ref{ItemXYmGHFaii}. Soit \( \{ F_i \}_{i\in I}\) une famille de fermés tels que \( K\bigcap_{i\in I}F_i=\emptyset\). Les complémentaires \( \mO_i\) de \( F_i\) dans \( X\) recouvrent \( K\) et donc on peut en extraire un sous-recouvrement fini :
    \begin{equation}
        K\subset\bigcup_{i\in A}\mO_i
    \end{equation}
    pour un certain sous-ensemble fini \( A\) de \( I\). Pour ce même choix \( A\), nous avons alors aussi
    \begin{equation}
        K\bigcap_{i\in A}F_i=\emptyset.
    \end{equation}

    L'implication \ref{ItemXYmGHFaii} \( \Rightarrow\) \ref{ItemXYmGHFai} est la même histoire.
\end{proof}

\begin{theorem}     \label{ThoImCompCotComp}
L'image d'un compact par une fonction continue est un compact
\end{theorem}

\begin{proof}
    Soit $K\subset X$, un ensemble compact, et regardons $f(K)$; en particulier, nous considérons $\Omega$, un recouvrement de $f(K)$ par des ouverts. Nous avons que
    \begin{equation}
        f(K)\subseteq\bigcup_{\mO\in\Omega}\mO.
    \end{equation}
    Par construction, nous avons aussi
    \begin{equation}
        K\subseteq\bigcup_{\mO\in\Omega}f^{-1}(\mO),
    \end{equation}
    en effet, si $x\in K$, alors $f(x)$ est dans un des ouverts de $\Omega$, disons $f(x)\in \mO_0$, et évidemment, $x\in f^{-1}(\mO)$.  Les $f^{-1}(\mO)$ recouvrent le compact $K$, et donc on peut en choisir un sous-recouvrement fini, c'est à dire un choix de $\{ f^{-1}(\mO_1),\ldots,f^{-1}(\mO_n) \}$ tels que
    \begin{equation}
        K\subseteq \bigcup_{i=1}^nf^{-1}(\mO_i).
    \end{equation}
    Dans ce cas, nous avons que
    \begin{equation}
        f(K)\subseteq\bigcup_{i=1}^n\mO_i,
    \end{equation}
    ce qui prouve la compacité de $f(K)$.
\end{proof}

\begin{theorem}[Tykhonov]\index{théorème!Tykhonov}\label{ThoFWXsQOZ}
    Un produit quelconque d'espaces métriques non vides est compact si et seulement si chacun de ses facteurs est compact.
\end{theorem}
Nous n'allons donner la preuve que dans le cas d'un produit dénombrable, dans le théorème \ref{ThoCDhbZbf}.

%+++++++++++++++++++++++++++++++++++++++++++++++++++++++++++++++++++++++++++++++++++++++++++++++++++++++++++++++++++++++++++ 
\section{Connexité}
%+++++++++++++++++++++++++++++++++++++++++++++++++++++++++++++++++++++++++++++++++++++++++++++++++++++++++++++++++++++++++++

Dès qu'un ensemble est muni d'une métrique, nous pouvons définir les boules ouvertes, les voisinages et les sous-ensembles ouverts. Dès que l'on a identifié les sous-ensemble ouverts de $E$, nous disons que $E$ devient un \defe{espace topologique}{espace!topologique}. Nous allons maintenant un pas plus loin.

\begin{definition}
 Lorsque $E$ est un espace topologique, nous disons qu'un sous-ensemble $A$ est \defe{non connexe}{connexité!définition} quand on peut trouver des ouverts $O_1$ et $O_2$ disjoints tels que
\begin{equation}    \label{EqDefnnCon}
  A=(A\cap O_1)\cup (A\cap O_2),
\end{equation}
et tels que $A\cap O_1\neq\emptyset$, et $A\cap O_2\neq\emptyset$.
Si un sous-ensemble n'est pas non-connexe, alors on dit qu'il est connexe.
\end{definition}
Une autre façon d'exprimer la condition \eqref{EqDefnnCon} est de dire que $A$ n'est pas connexe quand il est contenu dans la réunion de deux ouverts disjoints qui intersectent tous les deux $A$.

\begin{proposition} \label{PropHSjJcIr}
    Soit \( X\) un espace topologique. Les conditions suivantes sont équivalentes.
    \begin{enumerate}
        \item
            L'espace \( X\) est connexe.
        \item
            Si \( X=A\sqcup B\) avec \( A\) et \( B\) fermés dans \( X\), alors \( A=\emptyset\) ou \( B=\emptyset\).
        \item
            Si \( A\subset X\) avec \( A\) ouvert et fermé en même temps, alors \( A=\emptyset\) ou \( A=X\).
        \item
            Toute application continue \( X\to \eZ\) est constante.
    \end{enumerate}
\end{proposition}
%TODO : une preuve.

\begin{proposition}\label{PropGWMVzqb}
    L'image d'un ensemble connexe par une fonction continue est connexe.
\end{proposition}

\begin{proof}
    Soit \( f\colon X\to Y\) une application continue entre deux espaces topologiques, et \( E\) une partie connexe de \( X\). Nous devons montrer que \( f(E)\) est connexe dans \( Y\).

    Par l'absurde nous considérons \( A\) et \( B\), deux ouverts de \( Y\) disjoints recouvrant \( f(E)\). Étant donné que \( f\) est continue, les ensembles \( f^{-1}(A)\) et \( f^{-1}(B)\) sont ouverts dans \( X\). De plus ces deux ensembles recouvrent \( E\).

    Si \( x\) est un élément de \( f^{-1}(A)\cap f^{-1}(B)\), alors \( f(x)\in A\cap B\), ce qui est impossible parce que nous avons supposé que \( A\) et \( B\) étaient disjoints. Par conséquent \( f^{-1}(A)\) et \( f^{-1}(B)\) sont deux ouverts disjoints recouvrant \( E\). Contradiction avec la connexité de \( E\). Nous concluons que \( f(E)\) est connexe.
\end{proof}
Une application de ce théorème sera le théorème de valeurs intermédiaires \ref{ThoValInter}.

\begin{example}
    L'application
    \begin{equation}
        \begin{aligned}
            f\colon \mathopen] 0 , 2\pi \mathclose[&\to S^1 \\
                x&\mapsto \begin{pmatrix}
                    \cos(x)    \\ 
                    \sin(x)    
                \end{pmatrix}
        \end{aligned}
    \end{equation}
    est un homéomorphisme. Vu que \( \mathopen] 0 , 2\pi \mathclose[\) est connexe (proposition \ref{PropInterssiConn}), nous en concluons que le cercle privé d'un point est connexe.
\end{example}

\begin{example}
    Les espaces topologiques \( \eR\) et \( \eR^2\) ne sont pas homéomorphes.
\end{example}

\begin{proof}
    Soit \( f\colon \eR\to \eR^2\) un homéomorphisme. Nous posons \( E=f\big( \eR\setminus\{ 0 \} \big)\) et \( z_0=f(0)\). Vu que \( f\) est bijective nous avons
    \begin{equation}
        E=\eR^2\setminus\{ z_0 \},
    \end{equation}
    qui est connexe.

    Vu que \( E\) est connexe et que \( f^{-1}\) est continue, la proposition \ref{PropGWMVzqb} nous dit que \( f^{-1}(E)\) est connexe. Mais par définition, \( f^{-1}(E)=\eR\setminus\{ 0 \}\) qui n'est pas connexe.
\end{proof}


\begin{proposition}
    Si \( A\subset X\) est connexe et si \( A\subset B\subset \bar A\), alors \( B\) est connexe.
\end{proposition}
%TODO : une preuve.

\begin{proposition} \label{PropIWIDzzH}
    Stabilité de la connexité par union.
    \begin{enumerate}
        \item
    Une union quelconque ce connexes ayant une intersection non vide est connexe.
\item
    Si \( A_1,\ldots, A_n\) sont des connexes de \( X\) avec \( A_i\cap A_{i+1}\neq \emptyset\), alors l'union \( \bigcup_{n=1}^nA_i\) est connexe.
    \end{enumerate}
\end{proposition}

\begin{proof}
    \begin{enumerate}
        \item
    Soient \( \{ C_i \}_{i\in I}\) un ensemble de connexes et un point \( p\) dans l'intersection : \( p\in\bigcap_{i\in I}C_i\). Supposons que l'union ne soit pas connexe. Alors nous considérons \( A\) et \( B\), deux ouverts disjoints recouvrant tous les \( C_i\) et ayant chacun une intersection non vide avec l'union.

    Supposons pour fixer les idées que \( p\in A\) et prenons \( x\in B\cap\bigcup_{i\in I}C_i\). Il existe un \( j\in I\) tel que \( x\in C_j\). Avec tout cela nous avons
    \begin{enumerate}
        \item
            \( C_j\subset A\cup B\),
        \item
            \( C_j\cap A\neq \emptyset\) parce que \( p\) est dans l'intersection,
        \item
            \( C_j\cap B\neq\emptyset\) parce que \( x\) est dans cette intersection.
    \end{enumerate}
    Cela contredit le fait que \( C_j\) soit connexe.

\item

    Pour la seconde partie nous procédons de proche en proche. D'abord \( A_1\cup A_2\) est connexe par la première partie, ensuite \( (A_1\cup A_2)\cup A_3\) est connexe parce que les connexes \( A_1\cap A_2\) et \( A_3\) ont un point d'intersection par hypothèse, et ainsi de suite.
    \end{enumerate}
\end{proof}

%--------------------------------------------------------------------------------------------------------------------------- 
\subsection{Connexité de quelque groupes}
%---------------------------------------------------------------------------------------------------------------------------

\begin{proposition} \label{PropIFabDZz}
    Nous avons quelque résultats de connexité de groupes.
    \begin{enumerate}
        \item
            Le groupe \( \GL(n,\eC)\) est connexe.
        \item
            Le groupe \( \GL(n,\eR)\) n'est pas connexe, mais les groupes \( \GL^+(n,\eR)\) et \( \GL^-(n,\eR)\) le sont.
            % La proposition \ref{PropYGBEECo} en parle aussi.
        \item
            Le groupe \( \gO(n)\) n'est pas connexe.
        \item
            Les groupes \( \SU(n)\) et \( \SO(n)\) sont connexes.
    \end{enumerate}
\end{proposition}
\index{connexité!de groupes connus}
%TODO : il faut des preuves de tout ça, et certainement déplacer certains.

%+++++++++++++++++++++++++++++++++++++++++++++++++++++++++++++++++++++++++++++++++++++++++++++++++++++++++++++++++++++++++++
\section{Action de groupe et connexité}
%+++++++++++++++++++++++++++++++++++++++++++++++++++++++++++++++++++++++++++++++++++++++++++++++++++++++++++++++++++++++++++

Sources : \cite{MneimneLie} et \wikipedia{fr}{Matrice_normale}{wikipédia}.

\begin{theorem}     \label{ThojrLKZk}
    Soit \( G\) un groupe topologique localement compact et dénombrable à l'infini\footnote{Cela signifie qu'il est une réunion dénombrable de compacts} agissant continument et transitivement sur un espace topologique localement compact \( E\). Alors l'application
    \begin{equation}
        \begin{aligned}
            \varphi\colon G/G_x&\to E \\
            [g]&\mapsto g\cdot x 
        \end{aligned}
    \end{equation}
    est un homéomorphisme.
\end{theorem}

\begin{lemma}       \label{LemkLRAet}
    Si \( G\) et \( H\) sont des groupes topologiques tels que $G/H$ et \( H\) sont connexes, alors \( G\) est connexe.
\end{lemma}

\begin{proof}
    Soit \( f\colon G\to \{ 0,1 \}\) une fonction continue. Considérons l'application
    \begin{equation}
        \begin{aligned}
            \tilde f\colon G/H&\to \{ 0,1 \} \\
            [g]&\mapsto f(g). 
        \end{aligned}
    \end{equation}
    D'abord nous montrons qu'elle est bien définie. En effet si \( h\in H\) nous aurions \( \tilde f([gh])=f(gh)\), mais étant donné que \( H\) est connexe, l'ensemble \( gH\) est également connexe, de telle façon à ce que la fonction continue \( f\) soit constante sur \( gH\). Nous avons donc \( f(gh)=f(g)\).

    Étant donné que \( G/H\) est également connexe, la fonction \( \tilde f\) doit être constante. Si \( g_1\) et \( g_2\) sont deux éléments du groupe, nous avons \( f(g_1)=\tilde f([g_1])=\tilde f([g_2])=f(g_2)\). Nous en déduisons que \( f\) est constante et que \( G\) est connexe.
\end{proof}

\begin{theorem}
    Le groupe \( \SO(n)\) est connexe, le groupe \( \gO(n)\) a deux composantes connexes.
\end{theorem}

\begin{proof}
    La seconde assertion découle de la première parce que les matrices de déterminant \( 1\) et celles de déterminant \( -1\) ne peuvent pas être reliées par un chemin continu tandis que l'application
    \begin{equation}
        M\mapsto \begin{pmatrix}
            -1    &       &       \\
                &   1    &       \\
                &       &   1
        \end{pmatrix}M
    \end{equation}
    est un homéomorphisme entre les matrices de déterminant \( 1\) et celles de déterminants \( -1\). Montrons donc que \( G=\SO(n)\) est connexe par arcs pour \( n\geq 2\) en procédant par récurrence sur la dimension.
    
    Nous acceptons le résultat pour $G=\SO(2)$. Notons que nous en avons besoin pour prouver que la sphère \( S^{n-1}\) est connexe.
    
    Le groupe \( \SO(n)\) agit, par définition, de façon transitive sur la sphère \( S^{n-1}\). Soit \( a\in S^{n-1}\), nous avons
    \begin{subequations}
        \begin{align}
            G\cdot a&=S^{n-1}\\
            G_a&\simeq \SO(n-1)
        \end{align}
    \end{subequations}
    où \( G_a\) est le fixateur de \( a\) dans \( G\). Pour montrer le second point, nous considérons \( \{ e_i \}\), la base canonique de \( \eR^n\) et \( M\in G\) telle que \( Ma=e_1\). Le fixateur de \( e_1\) est évidemment isomorphe à \( \SO(n-1)\) parce qu'il est constitué des matrices de la forme
    \begin{equation}
        \begin{pmatrix}
             1   &   0    &   \ldots    &   0    \\
             0   &   a_{11}    &   \ldots    &   a_{1,n-1}    \\
             \vdots   &   \vdots    &   \ddots    &   \vdots    \\ 
             0   &   a_{n-1,1}    &   \ldots    &   a_{n-1,n-1}     
         \end{pmatrix}
    \end{equation}
    où \( (a_{ij})\in \SO(n-1)\). L'application 
    \begin{equation}
        \begin{aligned}
            \alpha\colon G_{e_1} &\to G_{a} \\
            A&\mapsto M^{-1}A M
        \end{aligned}
    \end{equation}
    est un isomorphisme entre \( G_a\) et \( \SO(n-1)\). Le théorème \ref{ThojrLKZk} nous montre alors que, en tant qu'espaces topologiques,
    \begin{equation}
        G/G_a=S^{n-1}.
    \end{equation}
    L'hypothèse de récurrence montre que \( G_a=\SO(n-1)\) est connexe tandis que nous savons que \( S^{n-1}\) est connexe. Le lemme \ref{LemkLRAet} conclu que \( G=\SO(n)\) est connexe.
\end{proof}

\begin{lemma}       \label{LemIbrsFT}
    Une bijection continue entre un espace compact et un espace séparé est un homéomorphisme.
\end{lemma}

\begin{proposition}
    Les groupes \( \gU(n)\) et \( \SU(n)\) sont connexes.
\end{proposition}

\begin{proof}
    Soit \( G(n)\) le groupe \( \SU(n)\) ou \( \gU(n)\). Ce groupe opère transitivement sur la sphère complexe
    \begin{equation}
        S_{\eC}^{n-1}=\{ z\in \eC^n\tq \langle z, z\rangle=\sum_k| z_k |^2 =1 \}.
    \end{equation}
    Cet ensemble est le même que \( S^{2n-1}\) parce que \( |z_k|=x_k^2+y_k^2\). Nous avons une bijection continue entre \( S^{n-1}\) et \( S^{n-1}_{\eC}\) et donc un homéomorphisme (lemme \ref{LemIbrsFT}). Soit \( a\in S^{n-1}_{\eC}\), nous avons
    \begin{subequations}
        \begin{align}
            G\cdot a&=S^{n-1}_{\eC}\\
            G_a&\simeq G(n-1).
        \end{align}
    \end{subequations}
    La seconde ligne est un isomorphisme de groupe et un homéomorphisme. Il est donné de la façon suivante. D'abord le fixateur de \( e_1\) dans \( G(n)\) est donné par les matrices de la forme
    \begin{equation}
        \begin{pmatrix}
             1   &   0    &   \ldots    &   0    \\
             0   &   a_{11}    &   \ldots    &   a_{1,n-1}    \\
             \vdots   &   \vdots    &   \ddots    &   \vdots    \\ 
             0   &   a_{n-1,1}    &   \ldots    &   a_{n-1,n-1}     
         \end{pmatrix}
    \end{equation}
    où \( (a_{ij})\in G(n-1)\). Par ailleurs si \( M\) est une matrice de \( G(n)\) telle que \( Ma=e_1\), nous avons l'homéomorphisme
  
    \begin{equation}
        \begin{aligned}
            \alpha\colon G_{e_1}&\to G_a \\
            A&\mapsto M^{-1} AM. 
        \end{aligned}
    \end{equation}
    Encore une fois, cela est un homéomorphisme par le lemme \ref{LemIbrsFT}. Par composition nous avons \( G_a\simeq G(n-1)\) et un homéomorphisme
    \begin{equation}
        G(n)/G_a=S^{n-1}_{\eC}.
    \end{equation}
    Le groupe \( G_a\) et l'ensemble \( S^{n-1}_{\eC}\) étant connexes, le groupe \( G(n)\) est connexe par le lemme \ref{LemkLRAet}.
\end{proof}

%+++++++++++++++++++++++++++++++++++++++++++++++++++++++++++++++++++++++++++++++++++++++++++++++++++++++++++++++++++++++++++ 
\section{Espaces métriques}
%+++++++++++++++++++++++++++++++++++++++++++++++++++++++++++++++++++++++++++++++++++++++++++++++++++++++++++++++++++++++++++

Si $E$ est un ensemble, une \defe{distance}{distance} sur $E$ est une application $d\colon E\times E\to \eR$ telle que pour tout $x,y\in E$,
\begin{enumerate}

\item
$d(x,y)\geq 0$

\item
$d(x,y)=0$ si et seulement si $x=y$,

\item
$d(x,y)=d(y,x)$

\item
$d(x,y)\leq d(x,z)+d(z,y)$.

\end{enumerate}
La dernière condition est l'\defe{inégalité triangulaire}{inégalité!triangulaire}. Le couple $(E,d)$ d'un ensemble et d'une métrique est un \defe{espace métrique}{espace!métrique}.

La définition suivante donne une topologie sur les espaces métriques en partant des boules et en reprenant mot à mot la définition \ref{DefUOyCQtW} :
\begin{definition}      \label{DefHTGYFpT}
    Dès que l'ensemble $E$ est muni d'une distance, nous définissons une topologie en disant que les boules
    \begin{equation}
        B(x,r)=\{ y\in E\tq d(x,y)<r \}
    \end{equation}
    sont ouvertes. Une partie \( \mO\) de \( E\) est alors \defe{ouverte}{ouvert!dans un espace métrique} si pour tout point \( x\) de \( \mO\), il existe une boule ouverte centrée en \( x\) et contenue dans \( \mO\).
\end{definition}

La propriété suivante donne une caractérisation importante de la continuité dans le cas des espaces métriques.



\begin{definition}
Si $E$ est un ensemble quelconque, nous disons qu'une \defe{distance}{distance} sur $E$ est une fonction $d\colon E\times E\to \eR^+$ telle que
\begin{description}
\item[Symétrie] $d(x,y)=d(y,x)$,
\item[Séparation] $d(x,y)=0$ ssi $x=y$. Insistons sur le fait que dans tous les cas, nous devons avoir $d(x,y)\geq 0$,
\item[Inégalité triangulaire] $d(x,z)\leq d(x,y)+d(y,z)$
\end{description}
pour tout $x$, $y$, $z\in E$. Un ensemble muni d'une loi de distance s'appelle un \href{http://fr.wikipedia.org/wiki/Espace_métrique}{espace métrique}.
\end{definition}

\begin{example}
Le premier exemple d'espace métrique que nous connaissons est $\eR$ muni de la distance usuelle ente deux nombres :
\begin{equation}
d(x,y)=| y-x |.
\end{equation}
Je me permets de faire remarquer la valeur absolue.
\end{example}

\begin{exercice}
Que penser de la formule $d(x,y)=y-x$ pour définir une distance sur $\eR$ ?
\end{exercice}

À partir de là, nous définissons la notion de \defe{boule ouverte}{boule!ouverte} sur l'ensemble $E$ centrée au point $x$ et de rayon $r>0$ comme
\[
  B(x,r)=\{ y\in\eR\tq d(x,y)< r \}.
\]
La \defe{boule fermée}{boule!fermée} centrée en $x$ et de rayon $r>0$ est définie par
\[
  \bar B(x,r)=\{ y\in\eR\tq d(x,y)\leq r \}.
\]
La différence est que dans la première l'inégalité est stricte.

\begin{theorem}     \label{ThoBoulOuvVois}
Une boule ouverte contient une boule ouverte autour de chacun de ses points.
\end{theorem}

\begin{proof}
Prenons $y\in B(x,r)$, et prouvons que la boule $B(y,r-d(x,y))$ est contenue dans $B(x,r)$. Première chose : $r-d(x,y)>0$ parce que $y$ est dans la boule ouverté centrée en $x$ et de rayon $r$. Pour prouver que  $B(y,r-d(x,y))\subset B(x,r)$, prenons un point dans le premier ensemble et montrons qu'il est dans le second ensemble.

Soit donc $z\in B\big(y,r-d(x,y)\big)$ et testons $d(x,z)$ que nous voudrions être plus petit que~$r$. Et, miracle, il l'est parce que
\begin{align*}
  d(x,z)    &\leq d(x,y)+d(y,z)&\text{inégalité triangulaire}\\
        &<d(x,y)+\big(r-d(x,y)\big)&\text{$z\in B\big(y,r-d(x,y)\big)$}\\
        &=r.
\end{align*}
Remarquez que la première inégalité n'est pas stricte, tandis que la seconde est stricte. Nous avons donc bien $d(x,z)<r$ (strictement) comme le demandé pour que $z$ soit dans la boule \emph{ouverte} de centre $x$ et de rayon $r$.
\end{proof}

\begin{proposition}
    Soient \( E\) et \( F\), deux espaces métriques et une fonction \( f\colon E\to F\). Nous avons équivalence entre
    \begin{enumerate}
        \item\label{ItemCBUoRWJi}
            \( f\) est continue en \( a\),
        \item\label{ItemCBUoRWJii}
            Pour tout voisinage ouvert \( W\) de \( f(a)\), il existe un voisinage ouvert \( V\) de \( a\) tel que \( f(V)\subset W\).
        \item\label{ItemCBUoRWJiii}
            Pour toute boule \( W'=B\big( f(a),\epsilon \big)\), il existe une boule \( V'=B(a,\delta)\) telle que \( f(V)\subset W\).
        \item\label{ItemCBUoRWJiv}
            $\forall \epsilon>0,\,\exists \delta>0\,\tq f\big( B(a,\delta) \big)\subset B\big( f(a),\epsilon \big)$.
    \end{enumerate}
\end{proposition}

\begin{proof}
    L'équivalence \ref{ItemCBUoRWJi} \( \Leftrightarrow\) \ref{ItemCBUoRWJii} est la définition \ref{DeOLNtrxB}. L'équivalence \ref{ItemCBUoRWJiii} \( \Leftrightarrow\) \ref{ItemCBUoRWJiv} est une simple paraphrase.

    Montrons \ref{ItemCBUoRWJii} \( \Rightarrow\) \ref{ItemCBUoRWJiii}. Si \( W'=B\big( f(a),\delta \big)\), nous avons un voisinage \( V\) de \( a\) tel que \( f(V)\subset W\). L'ensemble \( V\) contenant une boule autour de chacun de ses points\footnote{Cela est la définition \ref{DefHTGYFpT} des ouverts dans un espace métrique, à ne pas confondre avec le théorème \ref{ThoPartieOUvpartouv}.}, il en contient un autour de \( a\) : \( V'=B(a,\delta)\subset V\). A fortiori nous avons \( f(V')\subset W\).

    Montrons \ref{ItemCBUoRWJiii} \( \Rightarrow\) \ref{ItemCBUoRWJii}. Si \( W\) est un ouvert autour de \( f(a)\), il contient une boule autour de \( f(a)\) : \( B\big( f(a),\epsilon \big)\subset W\). Il existe donc une boule \( V'=B(a,\delta)\) telle que \( f(V')\subset B\big( f(a),\epsilon \big)\subset W\).
\end{proof}

\begin{proposition}[Caractérisation séquentielle de la continuité]		\label{PropFnContParSuite}
	Soient \( X\) et \( Y\) des espaces métriques. Une fonction $f\colon X\to Y$ est continue en $a\in X$ si et seulement si pour toute suite $(x_n)$ dans $X\setminus\{ a \}$ convergente vers $a$, nous avons $\lim f(x_n)=f(a)$.
\end{proposition}
%TODO : une preuve.

Les espaces métriques ont une propriété importante que la \wikipedia{fr}{Espace_séquentiel}{fermeture séquentielle} est équivalente à la fermeture.
\begin{proposition}[Caractérisation séquentielle d'un fermé]    \label{PropLFBXIjt}
    Soit \( X\) un espace métrique et \( F\subset X\). L'ensemble \( F\) est fermé si et seulement si toute suite contenue dans \( F\) et convergeant dans \( X\) converge vers un élément de \( F\).
\end{proposition}
\index{fermeture!séquentielle}

\begin{proof}
    Si \( F\) est fermé alors \( X\setminus F\) est ouvert. Soit une suite \( x_n\stackrel{X}{\longrightarrow}x\) contenue dans \( F\). Si \( A\) est un ouvert autour de \( x\) contenu dans \( X\setminus F\) (existence par le théorème \ref{ThoBoulOuvVois}), alors la suite ne peut pas entrer dans \( A\) et ne peut donc pas converger vers \( x\).

    Dans l'autre sens maintenant. Supposons que \( X\setminus F\) ne soit pas ouvert. Alors il existe \( x\in X\setminus F\) pour lequel tout voisinage intersecte \( F\). En prenant \( x_k\in B(x,\frac{1}{ k })\), nous construisons une suite contenue dans \( F\) qui converge vers \( x\).
\end{proof}

\begin{proposition}\label{PropLYMgVMJ}
    Une isométrie entre deux espaces métriques est continue.
\end{proposition}

\begin{proof}
    Soient \( f\colon X\to Y\), une application isométrique et \( \mO\) un ouvert de \( Y\). Soit \( a\in f^{-1}(\mO)\); si \( d(a,b)<r\), alors \( d\big( f(a),f(b) \big)<r\) et donc \( b\in f^{-1}\big( B(f(a),r) \big)\). Donc autour de chaque point de \( f^{-1}(\mO)\) nous pouvons trouver une boule ouverte contenue dans \( f^{-1}(\mO)\), ce qui prouve que \( f^{-1}(\mO)\) est ouvert.
\end{proof}

\begin{theorem}[Théorème \wikipedia{fr}{Théorème_des_fermés_emboités}{des fermés emboîtés}\cite{OIywOjl}]
    Soit \( (E,d)\) un espace métrique. Il est complet si et seulement si toute suite décroissante de fermés non vides dont le diamètre tend vers zéro a une intersection qui se réduit à un seul point.
\end{theorem}

\begin{proof}
    \begin{subproof}
    \item[Condition suffisante]

        Soit \( \{ F_n \}_{n\in \eN}\) une telle suite de fermés emboités. Si nous choisissons des points \( x_n\in F_n\), nous obtenons une suite \( (x_n)\) de Cauchy et qui est par conséquent convergente vu que l'espace est par hypothèse complet. De plus, pour chaque \( N\geq n\), la queue de suite \( (x_n)_{n\geq N}\) est contenue dans \( F_N\) et donc converge vers un élément de \( F_N\) (parce que ce dernier est fermé). Donc la limite de \( (x_n)\) est dans \( \bigcap_{n\in \eN}F_n\).

        De plus cette intersection a diamètre nul parce que le diamètre de \( \bigcap_{n\in \eN}F_n\) est majoré par tous les diamètres des \( F_n\), lesquels sont arbitrairement petits par hypothèse. Donc l'intersection est réduite a un point.

    \item[Condition nécessaire]

        Soit une suite de Cauchy \( (x_n)\). Nous considérons les ensembles
        \begin{equation}
            F_n=\overline{ \{ x_i\tq i\geq n \} }.
        \end{equation}
        Le fait que la suite soit de Cauchy implique que \( \diam(F_n)\to 0\). Par hypothèse, nous avons alors
        \begin{equation}
            \bigcap_{n\in \eN}F_n=\{ a \}.
        \end{equation}
        Pour s'assurer que \( a\) est bien la limite de \( (x_n)\), il suffit de remarquer que
        \begin{equation}
            d(x_n,a)\leq \diam F_n\to 0.
        \end{equation}
    \end{subproof}
\end{proof}

%--------------------------------------------------------------------------------------------------------------------------- 
\subsection{Compacité}
%---------------------------------------------------------------------------------------------------------------------------

\begin{lemma}[de Lebesgue\cite{JBRzHwn}]    \label{LemQFXOWyx}
    Soit \( (X,d)\) un espace métrique tel que toute suite ait une sous-suite convergente à l'intérieur de l'espace. Si \( \{ V_i \}\) est un recouvrement par des ouverts de \( X\), alors il existe \( \epsilon\) tel que pour tout \( x\in X\), nous ayons \( B(x,\epsilon)\subset V_i\) pour un certain \( i\).
\end{lemma}

\begin{proof}
    Par l'absurde, nous supposons que pour tout \( n\), il existe un \( x_n\in X\) tel que la boule \( B(x_n,\frac{1}{ n })\) n'est contenue dans aucun des \( V_i\). Ce des \( x_n\) nous extrayons une sous-suite convergente (que nous nommons encore \( (x_n)\)) et nous posons \( x_n\to x\). Pour \( n\) assez grand (\( \frac{1}{ n }<\epsilon\)) nous avons \( x_n\in B(x,\epsilon)\), donc tous les \( x_n\) suivants sont dans le \( V_i\) qui contient \( x\).
\end{proof}

\begin{lemma}[\cite{JBRzHwn}]   \label{LemMGQqgDG}
    Soit \( (X,d)\) un espace métrique tel que toute suite possède une sous-suite convergente. Pour tout \( \epsilon>0\), il existe un ensemble fini \( \{ x_i \}_{i\in I}\) tel que les boules \( B(x_i,\epsilon)\) recouvrent \( X\).
\end{lemma}

\begin{proof}
    Soit par l'absurde un \( \epsilon>0\) contredisant le lemme. Il n'existe pas d'ensemble finis autour des points duquel les boules de taille \( \epsilon\) recouvrent \( X\).

    Nous construisons par récurrence une suite ne possédant pas de sous-suites convergente. Le premier terme, \( x_0\) est pris arbitrairement dans \( X\). Ensuite si nous en avons \( N\) termes, nous savons que les boules de rayon \( \epsilon\) et centrées en les points \( \{ x_i \}_{i=1,\ldots, N}\) ne recouvrent pas \( X\). Donc nous prenons \( x_{N+1}\) hors de l'union de ces boules.

    Ainsi nous avons une suite \( (x_n)\) dont tous les termes sont à distance plus grande que \( \epsilon\) les uns des autres. Une telle suite ne peut pas contenir de sous-suite convergente. Contradiction.
\end{proof}

\begin{theorem}[Bolzano-Weierstrass\cite{JBRzHwn}]\label{ThoBWFTXAZNH}
    Un espace métrique est compact si et seulement si toute suite admet une sous-suite qui converge à l'intérieur de l'espace.
\end{theorem}
\index{théorème!Bolzano-Weierstrass}
\index{compacité}
Une version dédiée à \( \eR^n\) sera démontrée dans le théorème \ref{ThoBolzanoWeierstrassRn}.

\begin{proof}
   Soit \( X\) un espace métrique compact et \( (x_n)\) une suite dans \( X\). Nous considérons la suite de fermés emboités
   \begin{equation}
       X_n=\overline{ \{ x_k\tq k>n \} }.
   \end{equation}
   Ce sont des fermés ayant la propriété d'intersection finie non vide, et donc la proposition \ref{PropXKUMiCj} nous dit qu'ils ont une intersection non vide. Un élément de cette intersection est automatiquement un point d'accumulation de la suite.

   Nous passons à l'autre sens. Nous supposons que toute suite dans \( X\) contient une sous-suite convergente, et nous considérons \( \{ V_i \}_{i\in I}\), un recouvrement de \( X\) par des ouverts. Par le lemme \ref{LemQFXOWyx}, nous considérons un \( \epsilon\) tel que pour tout \( x\), il existe un \( i\in I\) avec \( B(x,\epsilon)\subset V_i\). Par le lemme \ref{LemMGQqgDG}, nous considérons un ensemble fini \( \{ y_i \}_{i\in A}\) tel que le boules \( B(y_i,\epsilon)\) recouvrent \( X\).

   Par construction, chacune de ces boules \( B(y_i,\epsilon)\) est contenue dans un des ouverts \( V_i\). Nous sélectionnons donc parmi les \( V_i\) le nombre fini qu'il faut pour recouvrir les \( B(y_i,\epsilon)\) et donc pour recouvrir \( X\).
\end{proof}

Le théorème de Bolzano–Weierstrass \ref{ThoBWFTXAZNH} a l'importante conséquence suivante.
\begin{theorem}[Weierstrass]		\label{ThoWeirstrassRn}
	Une fonction continue à valeurs réelles définie sur un compact est bornée et atteint ses bornes.
\end{theorem}

\begin{proof}
	Soit $K$ une partie compacte et $f\colon K\to \eR$ une fonction continue. Nous désignons par $A$ l'ensemble des valeurs prises par $f$ sur $K$ :
	\begin{equation}
		A=f(K)=\{ f(x)\tq x\in K \}.
	\end{equation}
	Nous considérons le supremum $M=\sup A=\sup_{x\in K}f(x)$ avec la convention comme quoi si $A$ n'est pas borné supérieurement, nous posons $M=\infty$ (voir définition \ref{DefSupeA}).

	Nous allons maintenant construire une suite $(x_n)$ de deux façons différentes suivant que $M=\infty$ ou non.
	\begin{enumerate}
		\item
			Si $M=\infty$, nous choisissons, pour chaque $n\in\eN$, un $x_n\in K$ tel que $f(x_n)>n$. Cela est certainement possible parce que si $A$ n'est pas borné, nous pouvons y trouver des nombres aussi grands que nous voulons.
		\item
			Si $M<\infty$, nous savons que pour tout $\varepsilon$, il existe un $y\in A$ tel que $y>M-\varepsilon$. Pour chaque $n$, nous choisissons donc $x_n\in K$ tel que $f(x_n)>M-\frac{1}{ n }$.
	\end{enumerate}
    Quel que soit le cas dans lequel nous sommes, la suite $(x_n)$ est une suite dans $K$ qui est compact, et donc nous pouvons en extraire une sous-suite convergente à l'intérieur de \( K\) par le théorème de Bolzano-Weierstrass\ref{ThoBWFTXAZNH}. Afin d'alléger la notation, nous allons noter $(x_n)$ la sous-suite convergente. Nous avons donc 
	\begin{equation}
		x_n\to x\in K.
	\end{equation}
	Par la proposition \ref{PropFnContParSuite}, nous avons que $f$ prend en \( x\) la valeur
	\begin{equation}
		f(x)=\lim_{n\to \infty} f(x_n).
	\end{equation}
	Donc $f(x)<\infty$. Évidement, si nous avions été dans le cas où $M=\infty$, la suite $x_n$ aurait été choisie pour avoir $f(x_n)>n$ et donc il n'aurait pas été possible d'avoir $\lim_{n\to \infty} f(x_n)<\infty$. Nous en concluons que $M<\infty$, et donc que $f$ est bornée sur $K$.

	Afin de prouver que $f$ atteint sa borne, c'est à dire que $M\in A$, nous considérons les inégalités
	\begin{equation}
		M-\frac{1}{ n }<f(x_n)\leq M.
	\end{equation}
	En passant à la limite $n\to \infty$, ces inégalités deviennent
	\begin{equation}
		M\leq f(x)\leq M,
	\end{equation}
	et donc $f(x)=M$, ce qui prouve que $f$ atteint sa borne $M$ au point $x\in K$.
\end{proof}

\begin{lemma}   \label{LemnAeACf}
    Si \( K\) est compact et si \( F\) est fermé dans \( K\), alors \( F\) est compact.
\end{lemma}

\begin{proof}
    Nous allons utiliser la caractérisation de la compacité en termes de suites donnée par le théorème de Bolzano-Weierstrass \ref{ThoBWFTXAZNH}. Soit \( (x_n)\) une suite dans \( F\); par la compacité de \( K\) nous pouvons considérer une sous suite \( (y_n)\) qui converge dans \( K\) (proposition \ref{ThoBWFTXAZNH}). Étant donné que \( (y_n)\) est une suite convergente contenue dans \( F\) et étant donné que \( F\) est fermé, la limite est dans \( F\), ce qui prouve que \( (x_n)\) possède une sous suite convergente dans $F$ et par conséquent que \( F\) est compact.
\end{proof}

\begin{lemma}       \label{LemooynkH}
    Soit \( A_n\) une suite décroissante de fermés dans un compact \( K\). Alors
    \begin{equation}
        C=\bigcap_{n\in \eN}A_n
    \end{equation}
    est non vide. 
\end{lemma}

\begin{proof}
    Soit \( (x_n)\) une suite dans \( K\) telle que \( x_n\in A_n\). La suite étant contenue dans \( A_1\), et \( A_1\) étant compact (lemme \ref{LemnAeACf}), elle possède une sous suite \( (y_n=x_{\sigma_1(n)})\) convergente dont la limite est dans \( A_1\) par le théorème de Bolzano-Weierstrass \ref{ThoBWFTXAZNH}. Une queue de la suite \( y_n\) est dans \( A_2\) et nous considérons donc une sous suite convergente dans \( A_2\) donnée par
    \begin{equation}
        z_n=y_{\sigma_2(n)}=x_{\sigma_1\sigma_2(n)}.
    \end{equation}
    En continuant ainsi nous construisons une suite convergente dans \( A_k\). Nous considérons enfin la suite
    \begin{equation}
        y_n=x_{\sigma_1\ldots \sigma_n(n)}.
    \end{equation}
    Pour tout \( k\), une queue de cette suite est une sous suite de \( x_{\sigma_1\ldots \sigma_k(n)}\) et par conséquent cette suite converge dans \( A_k\). La limite de cette suite est donc dans l'intersection demandée.
\end{proof}

\begin{remark}
    Cette propriété est fausse pour les ouverts. Par exemple
    \begin{equation}
        \bigcap_{n>1}\mathopen] 0 , \frac{1}{ n } \mathclose[=\emptyset.
    \end{equation}
\end{remark}

\begin{lemma}   \label{LemKIcAbic}
    Si \( K\) est un compact et \( F\) un fermé disjoint de \( K\), alors \( d(K,F)>0\).
\end{lemma}

\begin{proof}
    Le fonction 
    \begin{equation}
        \begin{aligned}
             K&\to \eR \\
            x&\mapsto d(x,F) 
        \end{aligned}
    \end{equation}
    est une fonction continue sur \( K\), et donc atteint son minimum par le théorème de Weierstrass \ref{ThoWeirstrassRn}. Soit \( x_0\in K\) un point de \( K\) qui réalise ce minimum. Si \( d(x_0,F)=0\), alors on aurait une suite \( (x_n)\) dans \( F\) qui convergerait vers \( x_0\), mais \( F\) étant fermé cela signifierait que \( x_0\) serait dans \( F\), ce qui contredirait l'hypothèse que \( F\) et \( K\) sont disjoints.
\end{proof}

\begin{proposition}[\cite{JBRzHwn}]
    une isométrie d'un espace métrique compact sur lui-même est une bijection.
\end{proposition}

\begin{proof}
    Soit \( X\) un espace métrique compact et \( f\colon X\to X\) une isométrie. Le fait que \( f\) soit injective est obligatoire (sinon il y a des images dont la distance est nulle). Il faut montrer que \( f\) est surjective.

    Soit \( x\in X\) hors de \( f(X)\). Le lemme \ref{LemKIcAbic} appliqué au fermé \( \{ x \}\) et au compact \( f(K)\) donne un \( r>0\) tel que
    \begin{equation}
        d\big( x,f(K)\big)>r.
    \end{equation}
    Soit la suite \( u_n=f^n(x)\); c'est une suite dans \( K\) et possède donc une sous-suite convergente (Bolzano-Weierstrass\ref{ThoBWFTXAZNH}) que l'on nomme \( (y_n)\). Vu que \( f\) est une isométrie,
    \begin{equation}
        d(y_{n},y_{n+1})=d(x,y_m)>r
    \end{equation}
    pour un certain \( m\leq n+1\). Ce la signifie que pour tout \( n\), nous avons \( d(y_n,y_{n+1})>r\), ce qui contredit le fait que la suite \( (y_n)\) converge.
\end{proof}

\begin{proposition} \label{PropLHWACDU}
    Soit \( (X,d)\) un espace métrique compact et \( (u_n)\) une suite de \( X\) telle que
    \begin{equation}
        \lim_{n\to \infty} d(u_n,u_{n+1})=0.
    \end{equation}
    Alors l'ensemble des points d'accumulation de \( (u_n)\) est connexe.
\end{proposition}
\index{connexité!points d'accumulation}
\index{compacité}

\begin{proof}
    Nous notons \( \Gamma\) l'ensemble des points d'accumulation de la suite.
    \begin{subproof}
    \item[\( \Gamma\) est compact]
        Nous notons \( A_p=\{ u_n\tq n\geq p \}\) et nous avons
        \begin{equation}
            \Gamma=\bigcap_{p\in \eN}\overline{ A_p }
        \end{equation}
        parce que si \( x\in\Gamma\), alors pour tout \( n\), il existe \( m>n\) tel que \( x_m\in B(x,\epsilon)\), et donc tel que \( x\in B(x_m,\epsilon)\). Donc pour tout \( \epsilon\) et pour tout \( p\), l'intersection \( B(x,\epsilon)\cap A_p\) est non vide.

        En tant qu'intersection de fermés, \( \Gamma\) est fermé (lemme \ref{LemQYUJwPC}). En tant que fermé dans un compact, \( \Gamma\) est compact (lemme \ref{LemnAeACf}).

    \item[Recouvrement par deux compacts]

        Supposons que \( \Gamma\) ne soit\quext{est-ce qu'il faut vraiment un subjonctif ici ?} pas connexe. Nous pouvons alors considérer \( S\) et \( O\), deux ouverts disjoints recouvrant \( \Gamma\) et intersectant tout deux \( \Gamma\). Nous posons alors
        \begin{subequations}
            \begin{align}
                A&=S\cap\Gamma\\
                B&=O\cap\Gamma,
            \end{align}
        \end{subequations}
        et nous avons évidemment \( \Gamma=A\cup B\). Montrons que \( A\) est fermé (\( B\) le sera aussi par le même raisonnement). Soit une suite d'éléments de \( S\cap \Gamma\) convergent dans \( X\). Alors la limite est dans \( \bar\Gamma=\Gamma\) et donc elle est donc \( O\) ou \( S\), mais elle est certainement dans \( \bar S\). Cependant \( \bar S\) n'intersecte pas \( O\). En effet si \( x\in \bar S\cap O\), alors tout voisinage de \( x\) intersecterait \( S\), mais il y a des voisinages de \( x\) étant inclus dans \( O\) parce que \( O\) est ouvert; cela donnerait une intersection entre \( O\) et \( S\), ce qui est impossible. Donc la limite n'est pas dans \( O\) et donc elle est dans \( S\). Au final la limite est dans \( S\cap \Gamma\), ce qui prouve son caractère fermé. 

        Comme d'habitude, \( \Gamma\cap S\) est compact parce que fermé dans un compact.

    \item[Décomposition en trois morceaux]

        Vu que \( A\) et \( B\) sont des compacts disjoints, nous avons \( d(A,B)=\alpha>0\) pour un certain \( \alpha\) par le lemme \ref{LemKIcAbic}. Nous notons
        \begin{subequations}
            \begin{align}
                A'&=\{x\in X\tq d(x,A)<\frac{ \alpha }{ 3 }\}\\
                B'&=\{x\in X\tq d(x,B)<\frac{ \alpha }{ 3 }\}
            \end{align}
        \end{subequations}
        Nous avons \( A'=\bigcup_{x\in A}B(x,\frac{ \alpha }{ 3 })\) et donc en tant qu'union d'ouverts, \( A'\) est ouvert (définition de la topologie). Même chose pour \( B'\).

        Enfin nous notons 
        \begin{equation}
            K=X\setminus(A'\cup B')
        \end{equation}
        qui est fermé en tant que complémentaire d'ouvert, et donc compact. Étant donné que \( A\subset A'\) et \( B\subset B' \), nous avons \( K\cap \Gamma=\emptyset\).

        L'idée est maintenant de montrer que \( K\) contient un point d'accumulation de \( (u_n)\).

    \item[Sous-suites de \( (u_n)\)]

        L'hypothèse sur la suite \( (u_n)\) nous indique qu'il existe un \( N_0\) tel que \( \forall n\geq N_0\),
        \begin{equation}    \label{EqIHioHjW}
            d(u_{n},u_{n+1})<\frac{ \alpha }{ 3 }.
        \end{equation}
        Soit \( N>N_0 \) et \( x_0\in A\). Étant donné que \( x_0\) est point d'accumulation de la suite, il existe \( n_1>N\) tel que \( d(x_0,u_{n_1})<\frac{ \alpha }{ 3 }\). Même chose dans \( B\) : nous prenons \( y_0\in B\) et un naturel \( n_2>n_1\) tel que \( d(y_0,u_{n_2})<\frac{ \alpha }{ 3 }\). Nous avons \( u_{n_1}\in A'\) et \( u_{n_2}\in B'\).

        Soit \( n_0\) le plus petit naturel supérieur à \( n_1\) tel que \( u_{n_0}\notin A'\). Cela existe parce que \( u_{n_2}\in B'\) et \( B'\cap A'=\emptyset\), mais \( n_0\) n'est pas \( n_2\) lui-même parce que \( d(A',B')\geq \frac{ \alpha }{ 3 }\) alors que nous considérons \( n_0,n_1,n_2>N_0\) et donc pour tous les \( i\) entre \( n_1\) et \( n_2\) (compris), \( d(u_i,u_{i+1})<\frac{ \alpha }{ 3 }\). Notons qu'ici le strict dans la condition \eqref{EqIHioHjW} est important. Nous avons donc \(N_0<n_1<n_0<n_2\).

        Nous allons maintenant montrer que \( u_{n_0}\) est dans \( K\). C'est fait pour : il est loin en même temps de \( A'\) et de \( B'\). En utilisant l'inégalité triangulaire à l'envers, nous avons
        \begin{equation}
            \begin{aligned}[]
            d(u_{n_0},B)&\geq d(u_{n_0-1},B)-d(u_{n_0-1},u_{n0})\\
            &\geq d(A,B)-d(u_{n_0-1},A)-d(u_{n_0-1},u_{n_0})\\
            &\geq \alpha-\frac{ \alpha }{ 3 }-\frac{ \alpha }{ 3 }\\
            &=\frac{ \alpha }{ 3 }.
            \end{aligned}
        \end{equation}
        Pour la dernière inégalité nous avons utilisé le fait que \( u_{n_0-1}\) n'est pas dans \( A'\). Bref, nous avons montré que \( u_{n_0}\) n'est pas dans \( B'\) (dans la définition de ce dernier nous avons bien une inégalité stricte). Vu que par définition \( u_{n_0}\) n'est pas non plus dans \( A'\), nous avons \( u_{n_0}\in K\).

        Nous avons montré jusqu'à présent que pour tout \( N\geq N_0\), il existe un \( n_0\geq N\) tel que \( u_{n_0}\in K\). Cela nous construit donc une sous-suite \( (v_n)\) de \( (u_n)\) contenue dans \( K\). En tant que suite dans le compact \( K\), la suite \( (v_n)\) admet un point d'accumulation dans \( K\). Ce point est également point d'accumulation de la suite \( (u_n)\) complète, ce qui donne un point d'accumulation de \( (u_n)\) dans \( K\) et donc une contradiction.

    \end{subproof}
    Nous concluons que \( \Gamma\) est connexe.
\end{proof}

Encore une petite conséquence sans ambition du théorème de Bolzano-Weierstrass.
\begin{proposition}\label{PropHNylIAW}
    Si \( (x_n)\) est une suite dans un compact telle que toute sous-suite convergente ait le même point \( x\) comme limite. Alors la suite entière converge vers \( x\).
\end{proposition}

\begin{proof}
    Supposons que ce ne soit pas le cas. Alors il existe un \( \epsilon\) tel que pour tout \( N>0\), il existe \( n>N\) avec \( d(x_n,x)>\epsilon\). Cela nous donne une sous-suite de \( (x_n)\) composée d'éléments tous à une distance de \( x\) supérieure à \( \epsilon\). Nous la nommons \( (y_n)\); c'est une suite dans un compact qui admet donc une sous-suite convergente (et une telle sous-suite est une sous-suite de \( (x_n)\)) dont la limite devrait être \( x\), mais c'est impossible par construction.
\end{proof}

\begin{lemma}       \label{LemGDeZlOo}
    Soi \( \Omega\) un ouvert dans un espace métrique \( E\). Il existe une suite \( (K_n)\) de compacts tels que
    \begin{enumerate}
        \item
            \( K_n\subset \Omega\)
        \item
            \( \bigcup_{n=0}^{\infty}K_n=\Omega\)
        \item
            \( K_n\subset\Int(K_{n+1})\).
    \end{enumerate}
    Une telle suite de compacts vérifie alors
    \begin{enumerate}
        \item
            Il existe \( \delta_n\) tel que pour tout \( z\in K_n\), \( B(z,\delta_n)\subset K_{n+1}\).
        \item
            Tout compact de \( \Omega\) est inclus à \( \Int(K_n)\) pour un certain \( n\).            
    \end{enumerate}
\end{lemma}

\begin{proof}
    Nous considérons les ensembles
    \begin{equation}
        V_n=\{ z\in E\tq | z | \}\cup\bigcup_{a\notin\Omega}B(a,\frac{1}{ n }),
    \end{equation}
    et nous définissons \( K_n=\complement V_n\). Vérifions que ces ensembles vérifient tout ce qu'il faut.
    \begin{enumerate}
        \item
            Si \( a\notin\Omega\) alors \( a\) est dans tous les \( V_n\) et donc dans aucun des \( K_n\); nous avons donc bien \( K_n\subset\Omega\).
        \item
            Si \( z\in \Omega\) alors nous prenons \( n_1>| z |\) puis \( n_2\) tel que \( B(z,\frac{1}{ n_2 })\subset \Omega\). Alors \( z\in K_n\) avec \( n>\max(n_1,n_2)\).
        \item
            Une chose à comprendre est que si \( z\in K_n\), alors \( d(z,\complement \Omega)\geq \frac{1}{ n }\). Du coup si nous prenons \( \delta\) tel que
            \begin{equation}
                \frac{1}{ n+1 }<\delta<\frac{1}{ n }
            \end{equation}
            alors \( B(z,\delta)\subset K_{n+1}\).
        \item
            Enfin, les \( K_n\) sont tous compacts. En effet ils sont bornés parce que \( K_n\subset B(0,n)\) et ensuite \( K_n\) est fermé en tant que complémentaire d'un ouvert (\( V_n\) est ouvert en tant qu'union d'ouverts).
    \end{enumerate}

    Nous passons maintenant aux propriétés, qui sont indépendantes de la façon dont nous avons construit les \( K_n\) vérifiant les conditions.
    \begin{enumerate}
        \item
            
            Nous pouvons considérer la fonction \( K_n\to \eR\) donnée par \( z\mapsto d(z,\complement K_{n+1})\). Vu que \( K_n\subset\Int(K_{n+1})\), c'est une fonction (continue sur le compact \( K_n\)) prenant des valeurs strictement positives. Elle a donc un minimum strictement positif. Si \( \delta_n\) est plus petit que ce minimum nous avons \( B(z,\delta_n)\subset K_{n+1}\) pour tout \( z\in K_n\).

        \item
    
            D'abord nous avons \( \Omega=\bigcup_{n=0}^{\infty}\Int(K_n)\). En effet nous avons
            \begin{equation}
                \Omega=\bigcup_{n=0}^{\infty}K_n\subset\bigcup_{n=0}^{\infty}\Int(K_{n+1})\subset\bigcup_{n=0}^{\infty}\Int(K_n).
            \end{equation}
            L'inclusion dans l'autre sens est facile.

            Soit \( K\) compact dans \( \Omega\). Vu que \( \Omega\) est l'union des \( \Int(K_n)\), nous avons
            \begin{equation}
                K\subset\bigcup_{n=0}^{\infty}\Int(K_n).
            \end{equation}
            Cela donne à \( K\) un recouvrement par des ouverts dont nous pouvons extraire un sous-recouvrement fini par compacité. Les \( K_n\) étant croissants, du recouvrement fini, il suffit de prendre le plus grand (disons \( K_m\)) et nous avons \( K\subset\Int(K_m)\).

    \end{enumerate}
\end{proof}
Notons qu'avec la suite de \( K_n\) telle que construite, le dernier point est réglé en prenant
\begin{equation}
    \frac{1}{ n+1 }<\delta_n<\frac{1}{ n }.
\end{equation}

%--------------------------------------------------------------------------------------------------------------------------- 
\subsection{Ensembles enchaînés}
%---------------------------------------------------------------------------------------------------------------------------

Soit \( (x,d)\) un espace métrique.
\begin{definition}
    Une \defe{\( \epsilon\)-chaîne}{chaîne} joignant les points \( a\) et \( b\) de \( X\) est une suite finie \( (u_0,\ldots, u_n)\) dans \( X\) telle que \( u_0=a\), \( u_n=b\) et pour tout \( 0\leq i\leq n-1\) nous avons \( d(u_n,u_{n+1})\leq \epsilon\).

    Une partie \( A\) de \( X\) est \defe{bien enchaînée}{bien!enchaîné} si pour tout \( \epsilon>0\) et pour tout \( a,b\in A\), il existe une \( \epsilon\)-chaîne joignant \( a\) et \( b\) dans $A$.
\end{definition}
Les rationnels dans \( \eR\) sont bien enchaînés.

\begin{proposition}
    Un espace connexe est bien enchaîné.
\end{proposition}
%TODO: une preuve.

\begin{proposition}
    La fermeture d'un ensemble bien enchaîné dans un espace métrique compact \( (X,d)\) est connexe.
\end{proposition}
\index{connexité}
\index{compacité}

\begin{proof}
    Soit \( A\subset X\) un ensemble bien enchaîné, et soient \( a,b\in \bar A\). Nous construisons une suite \( (u_k)\) dans \( A\) de la façon suivante. Pour chaque \( n>0\) nous prenons \( a'\in B(a,\frac{1}{ n })\cap A\) et \( b'\in B(b,\frac{1}{ n })\cap A\). Ensuite nous considérons une \( \frac{1}{ n }\)-chaîne \( \{ v_i^{(n)} \}_{i\in I_n}\) dans \( A\) entre \( a'\) et \( b'\). Ici l'ensemble \( I_n\) est fini. La suite \( (u_k)\) est simplement construite en mettant bout à bout les éléments \( v_i^{(n)}\).
 
    La suite ainsi construite est une suite dans \( A\) admettant \( a\) et \( b\) comme points d'accumulation (les autres points d'accumulation sont également dans \( \bar A\)) et telle que \( \lim_{k\to \infty} d(u_k,u_{k+1})=0\). Par conséquent la proposition \ref{PropLHWACDU} nous dit que l'ensemble des points d'accumulation de \( (u_k)\) est connexe dans \( X\). Nous le notons \( C_{a,b}\).

    Si nous fixons \( a\in \bar A\), alors nous avons
    \begin{equation}
        \bigcup_{x\in \bar A}C_{a,x}=\bar A.
    \end{equation}
    Vu que le membre de gauche est une union de connexes, c'est un connexe par la proposition \ref{PropIWIDzzH}.
\end{proof}
En particulier, un espace métrique compact est connexe si et seulement si il est bien enchaîné.

%--------------------------------------------------------------------------------------------------------------------------- 
\subsection{Produit dénombrables d'espaces métriques}
%---------------------------------------------------------------------------------------------------------------------------

\begin{definition}
    Soient \( (E_n,d_n)\) des espaces métriques. Sur l'ensemble produit \( E=\prod_{i=1}^{\infty}E_i\) nous définissons la métrique
    \begin{equation}
        d(x,y)=\sum_{i=1}^{\infty}\frac{1}{ 2^i }d'_n(x_i,y_i)
    \end{equation}
    où \( d'_i=\min(d_i,1)\).
\end{definition}
On peut montrer que ce \( d\) est bien une distance et que \( (E,d)\) devient un espace métrique.
%TODO: le faire.

\begin{theorem}\label{ThoCDhbZbf}
    Un produit dénombrable d'espaces métriques non vides est compact si et seulement si chacun de ses facteurs est compact.
\end{theorem}
Note : ce résultat est encore valable pour un produit quelconque, c'est le théorème de Tykhonov \ref{ThoFWXsQOZ}.

%+++++++++++++++++++++++++++++++++++++++++++++++++++++++++++++++++++++++++++++++++++++++++++++++++++++++++++++++++++++++++++ 
\section{Semi-normes et espaces métrisables}
%+++++++++++++++++++++++++++++++++++++++++++++++++++++++++++++++++++++++++++++++++++++++++++++++++++++++++++++++++++++++++++

\begin{definition}  \label{DefPNXlwmi}
    Si \( E\) est un espace vectoriel, une \defe{semi-norme}{semi-norme} sur \( E\) est une application \( p\colon E\to \eR\) telle que
    \begin{enumerate}
        \item
            \( p(x)\geq 0\),
        \item
            \( p(\lambda x)=| \lambda |p(x)\)
        \item
            \( p(x+y)\leq p(x)+p(y)\).
    \end{enumerate}
\end{definition}

Soit une famille \( (p_i)_{i\in I}\) de semi-normes sur \( E\). Nous construisons alors une topologie sur \( E\) de la façon suivante\cite{SOdaAdx}. Pour tout \( J\) fini dans \( I\) nous définissons les boules ouvertes
\begin{equation}
    B_J(x,r)=\{ y\in E\tq p_j(y-x)<r\,\forall j\in J \}.
\end{equation}

La topologie sur \( E\) donnée par la famille de semi-norme est celle engendrée par les boules ainsi définies.
\index{topologie!et semi-norme}

\begin{proposition} \label{PropQPzGKVk}
    Une suite \( (x_n)\) dans \( E\) converge vers \( x\) au sens de la topologie des semi-normes si et seulement si pour tout \( i\in I\),
    \begin{equation}
        p_i(x-x_n)\to 0.
    \end{equation}
\end{proposition}

\begin{proof}
    Si la suite \( (x_n)\) converge vers \( x\), alors pour tout ouvert \( \mO\) autour de \( x\), il existe un \( N\) tel que si \( n\geq N\), alors \( x_n\in\mO\). En particulier pour tout \( j\) et pour tout \( \epsilon>0\), il doit exister un \( n\geq N_j\) tel que \( x_n\in B_j(x,\epsilon)\).

    Voyons l'implication inverse. Soit \( \epsilon>0\). Pour tout \( i\in I\), il existe un \( N_i\) tel que \( n\geq N_i\) implique \( p_i(x-x_n)\leq \epsilon\). Si \( \mO\) est un ouvert, il doit contenir une boule du type \( B_J(x,r)\) pour un certain ensemble fini \( J\subset I\).

    En prenant \( N=\max\{ N_j\tq j\in J \}\), nous avons \( p_j(x-x_n)\leq \epsilon\) pour tout \( j\) et donc \( x_n\in B_J(x,r)\).
\end{proof}

\begin{definition}
    Si \( X\) est un espace topologique, une fonction \( f\colon X\to \eR\) est \defe{séquentiellement continue}{continuité!séquentielle} si pour toute suite convergente \( x_n\to x\) dans \( X\) nous avons \( f(x_n)\to f(x)\) dans \( \eR\).
\end{definition}

%TODO : il y a un contre-exemple à faire à la page http://www.les-mathematiques.net/phorum/read.php?14,787368,787582

Une fonction continue est séquentiellement continue. Dans les espaces métriques la réciproque est également vraie\footnote{Proposition \ref{PropFnContParSuite}.} et la continuité est équivalente à la continuité séquentielle. Cela n'est cependant pas vrai pour n'importe quel espace topologique.

\begin{definition}
    Un espace topologique est \defe{métrisable}{espace!topologique!métrisable} si il est homéomorphe à un espace métrique.
\end{definition}

\begin{proposition} \label{PropXIAQSXr}
    Soit \( E\) et \( Y\), deux espaces métriques. Soit \( f\colon E\to Y\) une application séquentiellement continue. Alors \( f\) est continue.
\end{proposition}

\begin{proof}
    Soit \( \mO\) un ouvert de \( Y\); nous allons voir que le complémentaire de \( f^{-1}(\mO)\) est fermé dans \( E\). Pour cela nous considérons une suite convergente \( x_k\stackrel{E}{\longrightarrow} x\) avec \( x_k\in\complement f^{-1}(\mO)\) pour tout \( k\). Nous allons montrer que \( x\in \complement f^{-1}(\mO)\) et la caractérisation séquentielle\footnote{Proposition \ref{PropLFBXIjt}.} de la fermeture conclura que \( \complement f^{-1}(\mO)\) est fermé.

    Pour tout \( k\), nous avons \( f(x_k)\in\complement \mO\), mais \( \mO\) est ouvert et \( f(x_k)\stackrel{Y}{\longrightarrow}f(x)\) parce que \( f\) est séquentiellement continue. Par conséquent \( f(x)\in\complement \mO\) et \( x\in\complement f^{-1}(\mO)\).
\end{proof}

\begin{proposition}
    Si \( X\) est un espace topologique dont la topologie est donnée par une famille dénombrable de semi-normes, alors il est métrisable.
\end{proposition}
%TODO : une preuve

\begin{proposition}
    Une fonction séquentiellement continue sur un espace métrisable et à valeurs dans un espace métrique est continue.
\end{proposition}

\begin{proof}
    Soit \( E\) un espace métrique et un homéomorphisme \( \phi\colon X\to (E,d)\). Nous supposons que \( f\colon X\to Y\) est séquentiellement continue. Nous considérons l'application \( \tilde f=f\circ\phi^{-1}\), c'est à dire
    \begin{equation}
        \begin{aligned}
            \tilde f\colon E&\to Y \\
            a&\mapsto f\big( \phi^{-1}(a) \big). 
        \end{aligned}
    \end{equation}
    L'application \( \phi^{-1}\) est continue et donc séquentiellement continue. De plus \( \tilde f\) est séquentiellement continue. En effet si \( a_k\stackrel{E}{\longrightarrow}a\), alors
    \begin{equation}
        \tilde f(a_k)=f\big( \phi^{-1}(a_k) \big),
    \end{equation}
    mais \( \phi^{-1}\) est séquentiellement continue, donc \( \phi^{-1}(a_k)\stackrel{X}{\longrightarrow}\phi^{-1}(a)\), ce qui signifie que \( \phi^{-1}(a_k)\) est une suite convergente dans \( X\) et donc
    \begin{equation}
        \lim_{k\to \infty} \tilde f(a_k)=\lim_{k\to \infty} f\big( \phi^{-1}(a_k) \big)=f\big( \phi^{-1}(a) \big)=\tilde f(a).
    \end{equation}
    L'application \( \tilde f\) est donc séquentiellement continue. Mais étant donné que \( \tilde f\) est définie sur un espace métrique (\( E\)) et à valeurs dans un métrique, elle est continue par la proposition \ref{PropXIAQSXr}. L'application \( f=\tilde f\circ\phi\) est donc continue en tant que composée d'applications continues.
\end{proof}

%---------------------------------------------------------------------------------------------------------------------------
\subsection{Espaces d'opérateurs}
%---------------------------------------------------------------------------------------------------------------------------

Soit \( E\), un espace vectoriel normé. Une topologie possible\footnote{C'est, dans l'idée, celle qui sera choisie pour les espaces de distributions, voir la définition \ref{DefASmjVaT}.} sur l'espace des opérateurs \( \aL(E,E)\) est la \defe{topologie \( *\)-faible}{topologie!$*$-faible} qui est la topologie des semi-normes
\begin{equation}
    p_v(T)=\| T(v) \|_E.
\end{equation}
C'est une famille de semi-normes indicées par les éléments de \( E\). La proposition suivante indique qu'elle est un peu la topologie de la convergence ponctuelle.

\begin{proposition}
    Soit une suite \( (T_n)\) dans \( \aL(E,E)\) et \( T\in \aL(E,E)\). Nous avons \( T_n\stackrel{*}{\longrightarrow}T\) si et seulement si \( T_n(v)\stackrel{E}{\longrightarrow}T(v)\) pour tout \( v\in E\).
\end{proposition}

\begin{proof}
    Nous avons équivalence entre les lignes suivantes :
    \begin{subequations}
        \begin{align}
            T_n\stackrel{*}{\longrightarrow}T\\
            p_v(T_n-T)\to 0\,\forall v\in E &&\text{proposition \ref{PropQPzGKVk}}\\
            \| T_n(v)-T(v) \|_E\to 0\,\forall v\in E\\
            T_n(v)\stackrel{E}{\longrightarrow}T(v).
        \end{align}
    \end{subequations}
\end{proof}

%+++++++++++++++++++++++++++++++++++++++++++++++++++++++++++++++++++++++++++++++++++++++++++++++++++++++++++++++++++++++++++ 
\section{Espaces de Baire}
%+++++++++++++++++++++++++++++++++++++++++++++++++++++++++++++++++++++++++++++++++++++++++++++++++++++++++++++++++++++++++++
\label{SecBDlaUrz}

\begin{definition}
    Un \defe{espace de Baire}{espace!de Baire}\index{Baire!espace} est un espace topologique dans lequel toute intersection dénombrable d'ouverts denses est dense.
\end{definition}

\begin{theorem}[Théorème de Baire\cite{SIdTHwW}]    \label{ThoBBIljNM}
    Les espaces suivants ont de Baire :
    \begin{enumerate}
        \item
            les espaces topologiques localement compacts,
        \item
            les espaces métriques complets (donc ceux de Banach en particulier),
        \item
            tout ouvert d'un espace de Baire.
    \end{enumerate}
\end{theorem}
\index{théorème!de Baire}
\index{Baire!théorème}
%TODO : une preuve c'est sans doute bien, et ça a l'air d'être pas trop dur et donné sur Wikipédia.
