% This is part of Mes notes de mathématique
% Copyright (c) 2012-2014
%   Laurent Claessens, Carlotta Donadello
% See the file fdl-1.3.txt for copying conditions.

%+++++++++++++++++++++++++++++++++++++++++++++++++++++++++++++++++++++++++++++++++++++++++++++++++++++++++++++++++++++++++++
					\section{Topologie en général}
%+++++++++++++++++++++++++++++++++++++++++++++++++++++++++++++++++++++++++++++++++++++++++++++++++++++++++++++++++++++++++++

\begin{definition}		\label{DefTopologieGene}
Soit $E$, un ensemble et $\mT$, une partie de l'ensemble de ses parties qui vérifie les propriétés suivantes
\begin{enumerate}

\item
les ensembles $\emptyset$ et $E$ sont dans $\mT$,

\item
    Une union quelconque\footnote{Par «quelconque» nous entendons vraiment quelconque : c'est à dire indicée par un ensemble qui peut autant être \( \eN\) que \( \eR\) qu'un pt1 d'ensemble réellement violent.} d'éléments de \( \mT\) est dans \( \mT\).
\item
    Une intersection \emph{finie} d'éléments de \( \mT\) est dans \( \mT\).

\end{enumerate}
Un tel choix $\mT$ de sous-ensembles de $E$ est une  \defe{\href{http://fr.wikipedia.org/wiki/Espace_topologique}{topologie}}{topologie} sur $E$, et les éléments de $\mT$ sont appelés des \defe{ouverts}{ouvert}. Nous disons que un sous ensemble $A$ de $E$ est \defe{fermé}{fermé} si son complémentaire, $A^c$ est ouvert.
\end{definition}

\begin{lemma}   \label{LemQYUJwPC}
    Une intersection quelconque de fermés est fermée.
\end{lemma}

\begin{proof}
    Soit \( \{ F_i \}_{i\in I} \) est un ensemble de fermés; nous avons
    \begin{equation}
        \left( \bigcap_{i\in I}F_i \right)^c=\bigcup_{i\in I}F_i^c.
    \end{equation}
    L'union de droite est un ouvert (union d'ouvert), et donc le terme de gauche est le complémentaire d'un ouvert. Donc fermé.
\end{proof}

Dans un espace topologique, nous avons une caractérisation très importante des ouverts.
\begin{theorem}		\label{ThoPartieOUvpartouv}
    Une partie d'un espace topologique est ouverte si et seulement si elle contient un voisinage ouvert de chacun de ses éléments.
\end{theorem}

\begin{proof} 
    Soit \( X\) un espace topologique et \( A\subset X\). Le sens direct est évident : $A$ lui-même est un ouvert autour de $x\in A$, qui est inclus à $A$.

Pour le sens inverse, nous supposons que \( A\) contienne un ouvert atour de chacun de ses points. Pour chaque $x\in A$, nous considérons l'ensemble $\mO_x\subset A$, un ouvert autour de $x$. Nous avons que
\begin{equation}	\label{EqAUniondesOx}
	A=\bigcup_{x\in A}\mO_x.
\end{equation}
En effet $A\subset\bigcup_{x\in A}\mO_x$ parce que tous les éléments de $A$ sont dans un des $\mO_x$, par construction. D'autre part, $\bigcup_{x\in A}\mO_x\subset A$ parce que chacun des $\mO_x$ est compris dans $A$.

L'union du membre de droite de \eqref{EqAUniondesOx} est une union d'ouverts et est donc un ouvert. Cela prouve que $A$ est un ouvert.

\end{proof}
Une utilisation typique de ce théorème est faite dans le lemme \ref{LemMESSExh}.

\begin{definition}[Base de topologie]   \label{DefQELfbBEyiB}
    Une famille \( \mB\) d'ouverts de \( X\) est une \defe{base de la topologie}{base!de topologie} de \( X\) si pour tout \( x\in X\) et pour tout voisinage \( V\) de \( x\), il existe \( A\in \mB\) tel que \( x\in A\subset V\).
\end{definition}

\begin{proposition} \label{PropMMKBjgY}
    Si \( \mB\) est une base de la topologie de \( X\) alors tout ouvert de \( X\) est une union d'éléments de \( \mB\).
\end{proposition}

\begin{proof}
    Soit \( \mO\) un ouvert de \( X\); pour chaque \( x\in\mO\) nous considérons un ouvert \( U(x)\) tel que \( x\in U(x)\subset \mO\) (possible par le théorème \ref{ThoPartieOUvpartouv}). Nous prenons alors \( B(x)\in\mB\) tel que 
    \begin{equation}
        x\in B(x)\subset U(x)\subset \mO.
    \end{equation}
    Alors nous avons \( \mO=\bigcup_{x\in \mO}B(x)\).
\end{proof}
Notons toutefois que nous sommes loin d'avoir une union dénombrable en général.

%+++++++++++++++++++++++++++++++++++++++++++++++++++++++++++++++++++++++++++++++++++++++++++++++++++++++++++++++++++++++++++ 
\section{Séparabilité}
%+++++++++++++++++++++++++++++++++++++++++++++++++++++++++++++++++++++++++++++++++++++++++++++++++++++++++++++++++++++++++++

\begin{definition}[Espace séparé]  \label{DefYFmfjjm}
    Si deux points distincts ont toujours deux voisinages distincts, nous disons que l'espace est \defe{séparé}{séparé} ou \defe{Hausdorff}{Hausdorff}.
\end{definition}
La notion d'espace séparé est particulièrement importante parce qu'elle assure l'unicité de la limite d'une fonction en un point par la proposition \ref{PropFObayrf}.

\begin{definition}
    Un espace topologique est \defe{séparable}{séparable!espace topologique} si il possède une partie dénombrable dense.
\end{definition}
À ne pas confondre avec :
\begin{definition}  \label{DefWEOTrVl}
    Un espace topologique est \defe{séparé}{séparé!espace topologique} ou \defe{Hausdorff}{Hausdorff} si deux points distincts possèdent des voisinages disjoints.
\end{definition}

%+++++++++++++++++++++++++++++++++++++++++++++++++++++++++++++++++++++++++++++++++++++++++++++++++++++++++++++++++++++++++++ 
\section{Limite et continuité}
%+++++++++++++++++++++++++++++++++++++++++++++++++++++++++++++++++++++++++++++++++++++++++++++++++++++++++++++++++++++++++++

\begin{definition}
    Un \defe{homéomorphisme}{homéomorphisme} est une application bijective continue entre deux espaces topologiques dont la réciproque est continue. Deux espaces topologiques homéomorphes sont dits \defe{isomorphes}{isomorphisme!d'espaces topologiques}.
\end{definition}

Dès que nous avons une topologie nous avons une notion de convergence.
\begin{definition}[Convergence de suite] \label{DefXSnbhZX}
    Une suite $(x_n)$ d'éléments de $E$ \defe{converge}{convergence!de suite} vers l'élément $y$ de $E$ si pour tout ouvert $\mO$ contenant $y$, il existe un $K$ tel que $k>K$ implique $x_k\in\mO$. 
\end{definition}

\begin{definition}[Limite d'une fonction]\label{DefYNVoWBx}
    Soient \( X\) et \( Y\) des espaces topologiques, \( A\subset X\) et \( a\in\bar A\). L'élément \( y\in Y\) est une \defe{limite}{limite!d'une fonction} de \( f\colon A\to Y\) en \( a\) si pour tout voisinage \( V\) de \( y\), il existe un voisinage \( W\) de \( a\) dans \( X\) tel que 
    \begin{equation}
        f(W\cap A)\subset V.
    \end{equation}
    Si il y a unicité de l'élément vers lequel \( f\) peut converger, alors nous disons que cet élément est la \defe{limite}{limite!fonction} de \( f\) et nous notons
    \begin{equation}
        \lim_{x\to a} f(x)=y.
    \end{equation}
\end{definition}
La proposition \ref{PropRBCiHbz} nous dira que l'unicité est de mise dans le cas des espaces duaux pour la topologie \( *\)-faible. La proposition \ref{PropFObayrf} nous dira qu'il y a unicité dès que l'espace d'arrivée est séparé.

\begin{example}
    Oui, il y a moyen de converger vers plusieurs points distincts si l'espace n'est pas super cool. Nous pouvons par exemple\footnote{Cet exemple provient de Wikipédia\cite{EJVQuas}. Sa présence ici mot pour mot justifie que, pour des raisons de licence, non, il n'y aura pas de versions non libres de ce livre. Ni aujourd'hui ni jamais.} considérer la droite réelle munie de sa topologie usuelle et y ajouter un point $0'$ (qui clone le réel $0$) dont les voisinages sont les voisinages de $0$ dans lesquels nous remplaçons $0$ par $0'$. Dans cet espace, la suite $(1/n)$ converge à la fois vers $0$ et $0'$.
\end{example}

\begin{proposition}[Unicité de la limite pour un espace séparé]\label{PropFObayrf}
    Soient \( X\) un espace topologique, \( A\) une partie de \( X\) et \( Y\) un espace topologique séparé\footnote{Définition \ref{DefYFmfjjm}.}. Nous considérons une fonction \( f\colon A\to Y\). Si \( a\in\bar A\), alors \( f\) admet au plus une limite en \( a\).
\end{proposition}

\begin{proof}
    Soient \( y\) et \( y'\) des limites de \( f\) en \( a\), ainsi que des voisinages \( V\) et \( V'\) de \( y\) et \( y'\). Nous prenons également les voisinages \( W\) et \( W'\) correspondants :
    \begin{subequations}
        \begin{numcases}{}
            f(W\cap A)\subset V\\
            f(W'\cap A)\subset V'.
        \end{numcases}
    \end{subequations}
    Quitte à prendre des sous-ensembles nous pouvons supposer que \( W\) et \( W'\) sont ouverts. L'ensemble \( W\cap W'\) est un ouvert contenant \( a\) et intersecte donc \( A\). L'ensemble \( (W\cap W')\cap A\) est non vide et
    \begin{subequations}
        \begin{numcases}{}
            f(W\cap W'\cap A)\subset f(W\cap A)\subset V\\
            f(W\cap W'\cap A)\subset f(W'\cap A)\subset V'.
        \end{numcases}
    \end{subequations}
    Donc les ensembles \( V\) et \( V'\) ont une intersection. Au final nous avons prouvé que deux voisinages de \( y\) et \( y'\) ont forcément une intersection; étant donné que \( Y\) est séparé, nous devons avoir \( y=y'\).
\end{proof}

La définition suivante est \emph{la} définition de la continuité dans tous les cas.
\begin{definition}[Fonction continue]\label{DefOLNtrxB}
    Une fonction \( f\colon X\to Y\) est \defe{continue au point}{continue!fonction!en un point} \( a\in X\) si \( f(a)\) est une limite\footnote{Définition \ref{DefYNVoWBx}.} de \( f\) en \( a\).

    Une fonction \( f\colon X\to Y\) est \defe{continue}{continue!fonction entre espaces topologiques} si pour tout ouvert \( \mO\) de \( Y\), l'ensemble \( f^{-1}(\mO)\) est ouvert dans \( X\).
\end{definition}
La proposition \ref{PropQZRNpMn} donnera des détails sur ce qu'il se passe lorsque l'espace est métrique.

\begin{theorem} \label{ThoESCaraB}
    Une fonction \( f\colon X\to Y\) est une fonction continue si et seulement si elle est continue en chacun des points de \( X\).
\end{theorem}

\begin{proof}
    \begin{subproof}
    \item[Sens direct]
        Nous supposons que \( f\) est une fonction continue. Soit \( a\in X\) et \( V\), un voisinage de \( f(a)\). Nous considérons \( \mO\), un voisinage ouvert de \( f(a)\) contenu dans \( V\); l'ensemble \( f^{-1}(\mO)\) est alors un ouvert contenant \( a\), et l'image de \( f^{-1}(\mO)\) par \( f\) est bien entendu contenue dans \( V\).

    \item[Sens inverse]

        Soit \( \mO\) un ouvert de \( Y\). Pour prouver que \( f^{-1}(\mO)\) est un ouvert de\( X\), nous allons considérer un élément \( a\in f^{-1}(\mO)\) et montrer qu'il existe un voisinage ouvert de \( a\) contenu dans \( f^{-1}(\mO)\); le théorème \ref{ThoPartieOUvpartouv} nous assurera alors que \( f^{-1}(\mO)\) est ouvert.

        L'ensemble \( \mO\) est un voisinage ouvert de \( f(a)\) parce que \( a\) a été choisit dans \( f^{-1}(\mO)\). Donc la continuité de \( f\) en \( a\)\footnote{Définition \ref{DefOLNtrxB}.} nous assure qu'il existe un voisinage \( W\) de \( a\) tel que \( f(W)\subset\mO\). En prenant un ouvert contenant \( a\) à l'intérieur de \( W\) nous avons un voisinage ouvert de \( a\) contenu dans \( f^{-1}(\mO)\).
    \end{subproof}
\end{proof}

\begin{definition}[Suite de Cauchy\cite{TQSWRiz}]   \label{DefZSnlbPc}
    Soit \( E\) un espace vectoriel topologique. Une suite \( (x_k)\) dans \( E\) est une \defe{suite de Cauchy}{suite!de Cauchy} si pour tout voisinage \( \mU\) de \( 0\) il existe \( N\in \eN\) tel que \( x_k-x_l\in\mU\) pour tout \( k,l\geq N\).
\end{definition}

\begin{definition}[Espace complet]
    Nous disons qu'un sous ensemble \( A\) d'un espace topologique est \defe{complet}{complet} si toute suite de Cauchy d'éléments de \( A\) converge vers un élément de \( A\).
\end{definition}

%+++++++++++++++++++++++++++++++++++++++++++++++++++++++++++++++++++++++++++++++++++++++++++++++++++++++++++++++++++++++++++
\section{Topologie induite}
%+++++++++++++++++++++++++++++++++++++++++++++++++++++++++++++++++++++++++++++++++++++++++++++++++++++++++++++++++++++++++++

\begin{definition}  \label{DefVLrgWDB}
Soit \( X\) un espace topologique et \( A\subset X\). L'ensemble \( A\) devient un espace topologique en lui-même par la \defe{topologie induite}{topologie!induite} de \( X\). Un ouvert de \( A\) est un ensemble de la forme \( A\cap\mO\) où \( \mO\) est un ouvert de \( X\).
\end{definition}

\begin{lemma}       \label{LemkUYkQt}
    Si \( B\subset A\) alors la fermeture de \( B\) pour la topologie de \( A\) (induite de \( X\)) que nous noterons \( \tilde B\) est 
    \begin{equation}
        \tilde B=\bar B\cap A
    \end{equation}
    où \( \bar B\) est la fermeture de \( B\) pour la topologie de \( X\).
\end{lemma}

\begin{proof}
    Si \( a\in \bar B\cap A\), un ouvert de \( A\) autour de \( a\) est un ensemble de la forme \( \mO\cap A\) où \( \mO\) est un ouvert de \( X\). Vu que \( a\in\bar B\), l'ensemble \( \mO\) intersecte \( B\) et donc \( (\mO\cap A)\cap B\neq \emptyset\). Donc \( a\) est bien dans l'adhérence de \( B\) au sens de la topologie de \( A\).

    Pour l'inclusion inverse, soit \( a\in \tilde  B\), et montrons que \( a\in \bar B\cap A\). Par définition \( a\in A\), parce que \( \tilde B\) est une fermeture dans l'espace topologique \( A\). Il faut donc seulement montrer que \( a\in\bar B\). Soit donc \( \mO\) un ouvert de \( X\) contenant \( a\); par hypothèse \( \mO\cap A\) intersecte \( B\) (parce que tout ouvert de \( A\) contenant \( a\) intersecte \( B\)). Donc \( \mO\) intersecte \( B\). Cela signifie que tout ouvert (de \( X\)) contenant \( a\) intersecte \( B\), ou encore que \( a\in \bar B\).
\end{proof}

\begin{example} \label{ExloeyoR}
    Si \( A\) est un ouvert de \( X\), on pourrait croire que la topologie induite n'a rien de spécial. Il est vrai que \( B\) sera ouvert dans \( A\) si et seulement si il est ouvert dans \( X\), mais des choses se passent quand même. Prenons \( X=\eR\) et \( A=\mathopen] 0 , 1 \mathclose[\). Si \( B=\mathopen] \frac{ 1 }{2} , 1 \mathclose[ \), alors la fermeture de \( B\) dans \( A\) sera \( \tilde B=\mathopen[ \frac{ 1 }{2} , 1 [\) et non \( \mathopen[ \frac{ 1 }{2} , 1 \mathclose]\) comme on l'aurait dans \( \eR\).
\end{example}

Prendre la topologie induite de \( \eR\) vers un fermé de \( \eR\) donne des boules un peu spéciales comme le montre l'exemple suivant.

\begin{example}  \label{ExKYZwYxn}
    Quid de la boule ouverte \( B(1,\epsilon)\) dans le compact \( \mathopen[ 0 , 1 \mathclose]\) ? Par définition c'est
    \begin{equation}
        B(1,\epsilon)=\{ x\in\mathopen[ 0 , 1 \mathclose]\tq | x-1 |<\epsilon \}=\mathopen] 1-\epsilon , 1 \mathclose].
    \end{equation}
    Oui, cela est \emph{ouvert} dans \( \mathopen[ 0 , 1 \mathclose]\). C'est d'ailleurs un des ouverts de la topologie induite de \( \eR\) sur \( \mathopen[ 0 , 1 \mathclose]\).

    Donc pour la topologie de \( \mathopen[ 0 , 1 \mathclose]\), toutes les boules ouvertes \( B(x,\delta)\) avec \( x\in\mathopen[ 0 , 1 \mathclose]\) sont incluses à \( \mathopen[ 0 , 1 \mathclose]\).
\end{example}


\begin{lemma}   \label{LemPESaiVw}
    Soit \( A\subset X\) muni de la topologie induite de \( X\) et \( (x_n)\) une suite dans \( A\). Si \( x_n\stackrel{A}{\longrightarrow}x\), alors \( x_n\stackrel{X}{\longrightarrow}x\). 
\end{lemma}

\begin{proof}
    Soit \( \mO\) un ouvert autour de \( x\) dans \( X\). Alors \( A\cap\mO\) est un ouvert autour de \( x\) dans \( A\) et il existe \( N\in \eN\) tel que si \( n\geq N\), alors \( x_n\in A\cap\mO\subset\mO\).
\end{proof}

%+++++++++++++++++++++++++++++++++++++++++++++++++++++++++++++++++++++++++++++++++++++++++++++++++++++++++++++++++++++++++++ 
\section{Compacité}
%+++++++++++++++++++++++++++++++++++++++++++++++++++++++++++++++++++++++++++++++++++++++++++++++++++++++++++++++++++++++++++

\begin{definition}  \label{DefJJVsEqs}
  Une partie $A$ d'un espace topologique est \defe{compacte}{compact} si il vérifie la propriété de Borel-Lebesgue : pour tout recouvrement de $A$ par des ouverts (c'est-à-dire une collection d'ouverts dont la réunion contient $A$) on peut tirer un recouvrement fini.
\end{definition}
\begin{remark}
    Certaines sources (dont \wikipedia{fr}{Compacité_(mathématiques)}{wikipédia}) disent que pour être compact il faut aussi être séparé\footnote{Définition \ref{DefWEOTrVl}.}. Pour ces sources, un espace qui ne vérifie que la propriété de Borel-Lebesgue est alors dit \defe{quasi-compact}{quasi-compact}\index{compact!quasi}.
\end{remark}

\begin{definition}
    Une famille \( \mA\) de parties de \( X\) a la \defe{propriété d'intersection finie non vide}{propriété d'intersection non vide} si tout sous-ensemble fini de \( \mA\) a une intersection non vide.
\end{definition}

\begin{proposition}\label{PropXKUMiCj}
    Soit \( X\) un espace topologique et \( K\subset X\). Les propriétés suivantes sont équivalentes :
    \begin{enumerate}
        \item\label{ItemXYmGHFai}
            \( K\) est compact.
        \item\label{ItemXYmGHFaii}
            Si \( \{ F_i \}\) est une famille de fermés telle que \( K\bigcap_{i\in I}F_i=\emptyset\), alors il existe un sous-ensemble fini \( A\) de \( I\) tel que \( K\bigcap_{i\in A}F_i=\emptyset\).
        \item\label{ItemXYmGHFaiii}
            Si \( \{ F_i \}_{i\in I}\) est une famille de fermés telle que \( K\bigcap_{i\in A}F_i\neq\emptyset\) pour tout choix de \( A\) fini dans \( I\), alors l'intersection complète est non vide : \( K\bigcap_{i\in I}F_i\neq\emptyset\).
        \item\label{ItemXYmGHFaiv}
            Toute famille ayant la propriété d'intersection finie non vide a une intersection non vide.
    \end{enumerate}
\end{proposition}

\begin{proof}
    Les propriétés \ref{ItemXYmGHFaiii} et \ref{ItemXYmGHFaii} sont équivalentes par contraposition. De plus le point \ref{ItemXYmGHFaiv} est une simple reformulation en français de la propriété \ref{ItemXYmGHFaiii}.

    Prouvons \ref{ItemXYmGHFai} \( \Rightarrow\) \ref{ItemXYmGHFaii}. Soit \( \{ F_i \}_{i\in I}\) une famille de fermés tels que \( K\bigcap_{i\in I}F_i=\emptyset\). Les complémentaires \( \mO_i\) de \( F_i\) dans \( X\) recouvrent \( K\) et donc on peut en extraire un sous-recouvrement fini :
    \begin{equation}
        K\subset\bigcup_{i\in A}\mO_i
    \end{equation}
    pour un certain sous-ensemble fini \( A\) de \( I\). Pour ce même choix \( A\), nous avons alors aussi
    \begin{equation}
        K\bigcap_{i\in A}F_i=\emptyset.
    \end{equation}

    L'implication \ref{ItemXYmGHFaii} \( \Rightarrow\) \ref{ItemXYmGHFai} est la même histoire.
\end{proof}

\begin{theorem}     \label{ThoImCompCotComp}
L'image d'un compact par une fonction continue est un compact
\end{theorem}

\begin{proof}
    Soit $K\subset X$, un ensemble compact, et regardons $f(K)$; en particulier, nous considérons $\Omega$, un recouvrement de $f(K)$ par des ouverts. Nous avons que
    \begin{equation}
        f(K)\subseteq\bigcup_{\mO\in\Omega}\mO.
    \end{equation}
    Par construction, nous avons aussi
    \begin{equation}
        K\subseteq\bigcup_{\mO\in\Omega}f^{-1}(\mO),
    \end{equation}
    en effet, si $x\in K$, alors $f(x)$ est dans un des ouverts de $\Omega$, disons $f(x)\in \mO_0$, et évidemment, $x\in f^{-1}(\mO)$.  Les $f^{-1}(\mO)$ recouvrent le compact $K$, et donc on peut en choisir un sous-recouvrement fini, c'est à dire un choix de $\{ f^{-1}(\mO_1),\ldots,f^{-1}(\mO_n) \}$ tels que
    \begin{equation}
        K\subseteq \bigcup_{i=1}^nf^{-1}(\mO_i).
    \end{equation}
    Dans ce cas, nous avons que
    \begin{equation}
        f(K)\subseteq\bigcup_{i=1}^n\mO_i,
    \end{equation}
    ce qui prouve la compacité de $f(K)$.
\end{proof}

\begin{theorem}[Tykhonov]\index{théorème!Tykhonov}\label{ThoFWXsQOZ}
    Un produit quelconque d'espaces métriques non vides est compact si et seulement si chacun de ses facteurs est compact.
\end{theorem}
Nous n'allons donner la preuve que dans le cas d'un produit fini dans le théorème \ref{PropIYmxXuu}.

\begin{definition}
  Un sous ensemble $A \subset \eR^n$ est \defe{borné}{borné} si il existe une boule de $\eR^n$ contenant $A$.
\end{definition}

\begin{proposition}
  Toute réunion finie d'ensembles bornés est un ensemble borné. Toute partie d'un ensemble borné est un ensemble borné.
\end{proposition}


% TODO: regarder ceci à propos des compacts.
% En particulier, si on recouvre $A$ par l'ensemble des boules
% $B(x,1)$ où $x$ parcourt $A$ (de sorte que tout point de $A$ est
% dans \og sa\fg{} boule, et donc la réunion des boules recouvre bien
% $A$), on doit pouvoir en tirer un recouvrement fini, c'est-à-dire
% des boules $B(x_1,1), B(x_2,1), \ldots, B(x_k,1)$ (avec $k$ un
% naturel) dont la réunion contient $A$.

% Il me semble que c'est le coup qu'il ne faut vérifier le sous-recouvrement que pour des recouvrements composés d'ouverts issus d'une base donnée de la topologie.

%+++++++++++++++++++++++++++++++++++++++++++++++++++++++++++++++++++++++++++++++++++++++++++++++++++++++++++++++++++++++++++ 
\section{Connexité}
%+++++++++++++++++++++++++++++++++++++++++++++++++++++++++++++++++++++++++++++++++++++++++++++++++++++++++++++++++++++++++++

Dès qu'un ensemble est muni d'une métrique, nous pouvons définir les boules ouvertes, les voisinages et les sous-ensembles ouverts. Dès que l'on a identifié les sous-ensemble ouverts de $E$, nous disons que $E$ devient un \defe{espace topologique}{espace!topologique}. Nous allons maintenant un pas plus loin.

\begin{definition}
     Lorsque $E$ est un espace topologique, nous disons qu'un sous-ensemble $A$ est \defe{non connexe}{connexité!définition} quand on peut trouver des ouverts $O_1$ et $O_2$ disjoints tels que
    \begin{equation}    \label{EqDefnnCon}
        A=(A\cap O_1)\cup (A\cap O_2),
    \end{equation}
    et tels que $A\cap O_1\neq\emptyset$, et $A\cap O_2\neq\emptyset$. Si un sous-ensemble n'est pas non-connexe, alors on dit qu'il est connexe.
\end{definition}
Une autre façon d'exprimer la condition \eqref{EqDefnnCon} est de dire que $A$ n'est pas connexe quand il est contenu dans la réunion de deux ouverts disjoints qui intersectent tous les deux $A$.

\begin{proposition} \label{PropHSjJcIr}
    Soit \( X\) un espace topologique. Les conditions suivantes sont équivalentes.
    \begin{enumerate}
        \item
            L'espace \( X\) est connexe.
        \item
            Si \( X=A\sqcup B\) avec \( A\) et \( B\) fermés dans \( X\), alors \( A=\emptyset\) ou \( B=\emptyset\).
        \item
            Si \( A\subset X\) avec \( A\) ouvert et fermé en même temps, alors \( A=\emptyset\) ou \( A=X\).
        \item
            Toute application continue \( X\to \eZ\) est constante.
    \end{enumerate}
\end{proposition}
%TODO : une preuve.

\begin{proposition}\label{PropGWMVzqb}
    L'image d'un ensemble connexe par une fonction continue est connexe.
\end{proposition}

\begin{proof}
    Soit \( f\colon X\to Y\) une application continue entre deux espaces topologiques, et \( E\) une partie connexe de \( X\). Nous devons montrer que \( f(E)\) est connexe dans \( Y\).

    Par l'absurde nous considérons \( A\) et \( B\), deux ouverts de \( Y\) disjoints recouvrant \( f(E)\). Étant donné que \( f\) est continue, les ensembles \( f^{-1}(A)\) et \( f^{-1}(B)\) sont ouverts dans \( X\). De plus ces deux ensembles recouvrent \( E\).

    Si \( x\) est un élément de \( f^{-1}(A)\cap f^{-1}(B)\), alors \( f(x)\in A\cap B\), ce qui est impossible parce que nous avons supposé que \( A\) et \( B\) étaient disjoints. Par conséquent \( f^{-1}(A)\) et \( f^{-1}(B)\) sont deux ouverts disjoints recouvrant \( E\). Contradiction avec la connexité de \( E\). Nous concluons que \( f(E)\) est connexe.
\end{proof}
Une application de ce théorème sera le théorème de valeurs intermédiaires \ref{ThoValInter}.

\begin{example}
    Les espaces topologiques \( \eR\) et \( \eR^2\) ne sont pas homéomorphes.
\end{example}

\begin{proof}
    Soit \( f\colon \eR\to \eR^2\) un homéomorphisme. Nous posons \( E=f\big( \eR\setminus\{ 0 \} \big)\) et \( z_0=f(0)\). Vu que \( f\) est bijective nous avons
    \begin{equation}
        E=\eR^2\setminus\{ z_0 \},
    \end{equation}
    qui est connexe.

    Vu que \( E\) est connexe et que \( f^{-1}\) est continue, la proposition \ref{PropGWMVzqb} nous dit que \( f^{-1}(E)\) est connexe. Mais par définition, \( f^{-1}(E)=\eR\setminus\{ 0 \}\) qui n'est pas connexe.
\end{proof}


\begin{proposition}
    Si \( A\subset X\) est connexe et si \( A\subset B\subset \bar A\), alors \( B\) est connexe.
\end{proposition}
%TODO : une preuve.

\begin{proposition} \label{PropIWIDzzH}
    Stabilité de la connexité par union.
    \begin{enumerate}
        \item
    Une union quelconque ce connexes ayant une intersection non vide est connexe.
\item
    Si \( A_1,\ldots, A_n\) sont des connexes de \( X\) avec \( A_i\cap A_{i+1}\neq \emptyset\), alors l'union \( \bigcup_{n=1}^nA_i\) est connexe.
    \end{enumerate}
\end{proposition}

\begin{proof}
    \begin{enumerate}
        \item
    Soient \( \{ C_i \}_{i\in I}\) un ensemble de connexes et un point \( p\) dans l'intersection : \( p\in\bigcap_{i\in I}C_i\). Supposons que l'union ne soit pas connexe. Alors nous considérons \( A\) et \( B\), deux ouverts disjoints recouvrant tous les \( C_i\) et ayant chacun une intersection non vide avec l'union.

    Supposons pour fixer les idées que \( p\in A\) et prenons \( x\in B\cap\bigcup_{i\in I}C_i\). Il existe un \( j\in I\) tel que \( x\in C_j\). Avec tout cela nous avons
    \begin{enumerate}
        \item
            \( C_j\subset A\cup B\),
        \item
            \( C_j\cap A\neq \emptyset\) parce que \( p\) est dans l'intersection,
        \item
            \( C_j\cap B\neq\emptyset\) parce que \( x\) est dans cette intersection.
    \end{enumerate}
    Cela contredit le fait que \( C_j\) soit connexe.

\item

    Pour la seconde partie nous procédons de proche en proche. D'abord \( A_1\cup A_2\) est connexe par la première partie, ensuite \( (A_1\cup A_2)\cup A_3\) est connexe parce que les connexes \( A_1\cap A_2\) et \( A_3\) ont un point d'intersection par hypothèse, et ainsi de suite.
    \end{enumerate}
\end{proof}

%--------------------------------------------------------------------------------------------------------------------------- 
\subsection{Connexité de quelque groupes}
%---------------------------------------------------------------------------------------------------------------------------

\begin{proposition} \label{PropIFabDZz}
    Nous avons quelque résultats de connexité de groupes.
    \begin{enumerate}
        \item
            Le groupe \( \GL(n,\eC)\) est connexe.
        \item
            Le groupe \( \GL(n,\eR)\) n'est pas connexe, mais les groupes \( \GL^+(n,\eR)\) et \( \GL^-(n,\eR)\) le sont.
            % La proposition \ref{PropYGBEECo} en parle aussi.
        \item
            Le groupe \( \gO(n)\) n'est pas connexe.
        \item
            Les groupes \( \SU(n)\) et \( \SO(n)\) sont connexes.
    \end{enumerate}
\end{proposition}
\index{connexité!de groupes connus}
%TODO : il faut des preuves de tout ça, et certainement déplacer certains.

%+++++++++++++++++++++++++++++++++++++++++++++++++++++++++++++++++++++++++++++++++++++++++++++++++++++++++++++++++++++++++++
\section{Action de groupe et connexité}
%+++++++++++++++++++++++++++++++++++++++++++++++++++++++++++++++++++++++++++++++++++++++++++++++++++++++++++++++++++++++++++

Sources : \cite{MneimneLie} et \wikipedia{fr}{Matrice_normale}{wikipédia}.

\begin{theorem}     \label{ThojrLKZk}
    Soit \( G\) un groupe topologique localement compact et dénombrable à l'infini\footnote{Cela signifie qu'il est une réunion dénombrable de compacts} agissant continument et transitivement sur un espace topologique localement compact \( E\). Alors l'application
    \begin{equation}
        \begin{aligned}
            \varphi\colon G/G_x&\to E \\
            [g]&\mapsto g\cdot x 
        \end{aligned}
    \end{equation}
    est un homéomorphisme.
\end{theorem}

\begin{lemma}       \label{LemkLRAet}
    Si \( G\) et \( H\) sont des groupes topologiques tels que $G/H$ et \( H\) sont connexes, alors \( G\) est connexe.
\end{lemma}

\begin{proof}
    Soit \( f\colon G\to \{ 0,1 \}\) une fonction continue. Considérons l'application
    \begin{equation}
        \begin{aligned}
            \tilde f\colon G/H&\to \{ 0,1 \} \\
            [g]&\mapsto f(g). 
        \end{aligned}
    \end{equation}
    D'abord nous montrons qu'elle est bien définie. En effet si \( h\in H\) nous aurions \( \tilde f([gh])=f(gh)\), mais étant donné que \( H\) est connexe, l'ensemble \( gH\) est également connexe, de telle façon à ce que la fonction continue \( f\) soit constante sur \( gH\). Nous avons donc \( f(gh)=f(g)\).

    Étant donné que \( G/H\) est également connexe, la fonction \( \tilde f\) doit être constante. Si \( g_1\) et \( g_2\) sont deux éléments du groupe, nous avons \( f(g_1)=\tilde f([g_1])=\tilde f([g_2])=f(g_2)\). Nous en déduisons que \( f\) est constante et que \( G\) est connexe.
\end{proof}

\begin{theorem}
    Le groupe \( \SO(n)\) est connexe, le groupe \( \gO(n)\) a deux composantes connexes.
\end{theorem}

\begin{proof}
    La seconde assertion découle de la première parce que les matrices de déterminant \( 1\) et celles de déterminant \( -1\) ne peuvent pas être reliées par un chemin continu tandis que l'application
    \begin{equation}
        M\mapsto \begin{pmatrix}
            -1    &       &       \\
                &   1    &       \\
                &       &   1
        \end{pmatrix}M
    \end{equation}
    est un homéomorphisme entre les matrices de déterminant \( 1\) et celles de déterminants \( -1\). Montrons donc que \( G=\SO(n)\) est connexe par arcs pour \( n\geq 2\) en procédant par récurrence sur la dimension.
    
    Nous acceptons le résultat pour $G=\SO(2)$. Notons que nous en avons besoin pour prouver que la sphère \( S^{n-1}\) est connexe.
    
    Le groupe \( \SO(n)\) agit, par définition, de façon transitive sur la sphère \( S^{n-1}\). Soit \( a\in S^{n-1}\), nous avons
    \begin{subequations}
        \begin{align}
            G\cdot a&=S^{n-1}\\
            G_a&\simeq \SO(n-1)
        \end{align}
    \end{subequations}
    où \( G_a\) est le fixateur de \( a\) dans \( G\). Pour montrer le second point, nous considérons \( \{ e_i \}\), la base canonique de \( \eR^n\) et \( M\in G\) telle que \( Ma=e_1\). Le fixateur de \( e_1\) est évidemment isomorphe à \( \SO(n-1)\) parce qu'il est constitué des matrices de la forme
    \begin{equation}
        \begin{pmatrix}
             1   &   0    &   \ldots    &   0    \\
             0   &   a_{11}    &   \ldots    &   a_{1,n-1}    \\
             \vdots   &   \vdots    &   \ddots    &   \vdots    \\ 
             0   &   a_{n-1,1}    &   \ldots    &   a_{n-1,n-1}     
         \end{pmatrix}
    \end{equation}
    où \( (a_{ij})\in \SO(n-1)\). L'application 
    \begin{equation}
        \begin{aligned}
            \alpha\colon G_{e_1} &\to G_{a} \\
            A&\mapsto M^{-1}A M
        \end{aligned}
    \end{equation}
    est un isomorphisme entre \( G_a\) et \( \SO(n-1)\). Le théorème \ref{ThojrLKZk} nous montre alors que, en tant qu'espaces topologiques,
    \begin{equation}
        G/G_a=S^{n-1}.
    \end{equation}
    L'hypothèse de récurrence montre que \( G_a=\SO(n-1)\) est connexe tandis que nous savons que \( S^{n-1}\) est connexe. Le lemme \ref{LemkLRAet} conclut que \( G=\SO(n)\) est connexe.
\end{proof}

\begin{lemma}       \label{LemIbrsFT}
    Une bijection continue entre un espace compact et un espace séparé est un homéomorphisme.
\end{lemma}

\begin{proposition}
    Les groupes \( \gU(n)\) et \( \SU(n)\) sont connexes.
\end{proposition}

\begin{proof}
    Soit \( G(n)\) le groupe \( \SU(n)\) ou \( \gU(n)\). Ce groupe opère transitivement sur la sphère complexe
    \begin{equation}
        S_{\eC}^{n-1}=\{ z\in \eC^n\tq \langle z, z\rangle=\sum_k| z_k |^2 =1 \}.
    \end{equation}
    Cet ensemble est le même que \( S^{2n-1}\) parce que \( |z_k|=x_k^2+y_k^2\). Nous avons une bijection continue entre \( S^{n-1}\) et \( S^{n-1}_{\eC}\) et donc un homéomorphisme (lemme \ref{LemIbrsFT}). Soit \( a\in S^{n-1}_{\eC}\), nous avons
    \begin{subequations}
        \begin{align}
            G\cdot a&=S^{n-1}_{\eC}\\
            G_a&\simeq G(n-1).
        \end{align}
    \end{subequations}
    La seconde ligne est un isomorphisme de groupe et un homéomorphisme. Il est donné de la façon suivante. D'abord le fixateur de \( e_1\) dans \( G(n)\) est donné par les matrices de la forme
    \begin{equation}
        \begin{pmatrix}
             1   &   0    &   \ldots    &   0    \\
             0   &   a_{11}    &   \ldots    &   a_{1,n-1}    \\
             \vdots   &   \vdots    &   \ddots    &   \vdots    \\ 
             0   &   a_{n-1,1}    &   \ldots    &   a_{n-1,n-1}     
         \end{pmatrix}
    \end{equation}
    où \( (a_{ij})\in G(n-1)\). Par ailleurs si \( M\) est une matrice de \( G(n)\) telle que \( Ma=e_1\), nous avons l'homéomorphisme
  
    \begin{equation}
        \begin{aligned}
            \alpha\colon G_{e_1}&\to G_a \\
            A&\mapsto M^{-1} AM. 
        \end{aligned}
    \end{equation}
    Encore une fois, cela est un homéomorphisme par le lemme \ref{LemIbrsFT}. Par composition nous avons \( G_a\simeq G(n-1)\) et un homéomorphisme
    \begin{equation}
        G(n)/G_a=S^{n-1}_{\eC}.
    \end{equation}
    Le groupe \( G_a\) et l'ensemble \( S^{n-1}_{\eC}\) étant connexes, le groupe \( G(n)\) est connexe par le lemme \ref{LemkLRAet}.
\end{proof}
