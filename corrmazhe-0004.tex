% This is part of (almost) Everything I know in mathematics
% Copyright (c) 2016
%   Laurent Claessens
% See the file fdl-1.3.txt for copying conditions.

\begin{corrige}{mazhe-0004}

    Note : si c'est pour chercher à la main des approximation pour démarrer, il est évidemment préférable de dessiner \( f_1(x)=2x^2-4x+2\) et \( f_2(x)=- e^{-x}\) séparément.

    Quoi qu'il en soit, voici un graphique :

\begin{center}
   \input{Fig_OQTEoodIwAPfZE.pstricks}
\end{center}

    Nous voyons trois racines : \( \alpha_1\in\mathopen[ 0,0.5 ,  \mathclose]\), \( \alpha_2\in\mathopen[ 1, 1.5 \mathclose]\) et \( \alpha_0\in\mathopen[ -4 , -3.5 \mathclose]\).

    La plus grande solution est \( \alpha_2\). Nous pouvons déjà remplir le tableau des précisions :

    \begin{equation*}
        \begin{array}[]{|c|c|c|c|c|c|}
            \hline
            n&a_n&b_n&x_n&g(x_n)&| b_n-a_n |\\
            \hline\hline
            <++>&<++>&<++>&<++>&<++>&0.5\\
            \hline
            <++>&<++>&<++>&<++>&<++>&0.25\\
            \hline
            <++>&<++>&<++>&<++>&<++>&0.125\\
            \hline
        \end{array}
    \end{equation*}

    Et nous calculons les valeurs de \( f\) aux points d'extrémité de l'intervalle. Note que seul le signe nous importe :
    \begin{subequations}
        \begin{align}
            f(1)\simeq -0.368\\
            f(1.5)\simeq 0.278.
        \end{align}
    \end{subequations}
    Voici donc le tableau avec le signe de \( f\) indiqué :

    \begin{equation*}
        \begin{array}[]{|c|c|c|c|c|c|}
            \hline
            n&a_n&b_n&x_n&g(x_n)&| b_n-a_n |\\
            \hline\hline
            0&1 (+)&1.5 (-)&<++>&-0.162\times 10^{0}&0.5\\
            \hline
            1&<++>&<++>&<++>&<++>&0.25\\
            \hline
            2&<++>&<++>&<++>&<++>&0.125\\
            \hline
        \end{array}
    \end{equation*}
    Puis :
    \begin{equation*}
        \begin{array}[]{|c|c|c|c|c|c|}
            \hline
            n&a_n&b_n&x_n&g(x_n)&| b_n-a_n |\\
            \hline\hline
            0&1 (+)&1.5 (-)&1.25 (-)&-0.162\times 10^{0}&0.5\\
            \hline
            1&<++>&<++>&<++>&<++>&0.25\\
            \hline
            2&<++>&<++>&<++>&<++>&0.125\\
            \hline
        \end{array}
    \end{equation*}
    Et enfin :
    \begin{equation*}
        \begin{array}[]{|c|c|c|c|c|c|}
            \hline
            n&a_n&b_n&x_n&g(x_n)&| b_n-a_n |\\
            \hline\hline
            0&1 (+)&1.5 (-)&1.25 (-)&-0.162\times 10^{0}&0.5\\
            \hline
            1&1.25 (-)&1.5(+)&1.375(+)&+0.284\times 10^{-1}&0.25\\
            \hline
            2&1.25(-)&1.375(+)&1.3125(-)&)0.738\times 10^{-1}&0.125\\
            \hline
        \end{array}
    \end{equation*}
    Note que les \( f(x_n)\) restent toujours du même ordre de grandeur. Si un moment on voit un \( 1.6\times 10^{7}\), c'est qu'une erreur a été commise.

    En ce qui concerne le calcul de l'erreur relative, la première chose à faire est de vérifier que le \( \alpha\) proposé est dans l'intervalle qui nous reste. Sinon c'est qu'une erreur a été commise.

    De plus notre approximation est \( x_n=1.3125\), dont déjà deux chiffres sont corrects. En deux itérations de bisection en partant de \( 0.5\), nous ne pouvons pas nous attendre à mieux.

\end{corrige}
