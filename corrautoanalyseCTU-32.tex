% This is part of Analyse Starter CTU
% Copyright (c) 2014
%   Laurent Claessens,Carlotta Donadello
% See the file fdl-1.3.txt for copying conditions.

\begin{corrige}{autoanalyseCTU-32}

\begin{enumerate}
\item On commence par trouver la solution générale de $(E_{1})$. Comme l'équation est à variables séparables nous allons intégrer de deux c\^otés par rapport à la variable $x$. 
\[
\int (1+y(x))y'(x)\, dx = \int 4x^3\, dx.
\]
Nous faisons le changement de variable $y = y(x) $ dans le membre de gauche pour obtenir une expression facile à intégrer 
\[
\int (1+y)\, dy = \int 4x^3\, dx.
\]
et finalement nous avons 
\[
y+ \frac{y^2}{2} = x^4 + C.
\]
Il n'est pas possible de trouver une forme explicite pour $y$ sans prendre en compte la condition initiale, car la fonction $y\mapsto  \frac{y^2}{2}+ y $ n'admet pas de réciproque. Il vaut mieux, dans ce cas, dire que la solution générale de l'équation est l'ensemble de toutes les  fonctions $y$ qui satisfont $y+ \frac{y^2}{2} = x^4 + C$ pour un quelque $C$ dans $\eR$. 

Pour déterminer la solution particulière $f$ nous devons remplacer $x$ par $5$ et $y$ par $14$ dans $y+ \frac{y^2}{2} = x^4 + C$. Cela nous permet de fixer une valeur de $C$ : $14+ \frac{(14)^2}{2}  = (5)^4 + C$ implique $ C=-513$. 

La forme explicite de $f$ est donnée par la formule suivante :
\[
f(x) = \frac{-2+ 2\sqrt{1 + 2 (x^4 - 513)}}{2},
\] 
où le fait de conna\^{i}tre la valeur de $f$ en $x=5$ nous a permis de choisir entre $\displaystyle \frac{-2+ 2\sqrt{1 + 2 (x^4 - 513)}}{2}$ et $\displaystyle \frac{-2- 2\sqrt{1 + 2 (x^4 - 513)}}{2}$.
  
\item On commence par traiter le cas où $y = 0$ : on aurait alors que $y' = 0$ et la seule solution possible de l'équation différentielle serait la fonction qui vaut toujours zéro. Cette fonction ne satisfait pas la condition initiale et par conséquent nous pouvons supposer que $g(x)\neq 0$ pour tout $x$. Cela nous permet de diviser par $y^2$ les deux membres de l'équation différentielle et de séparer ainsi les variables\[
\frac{y'}{y^2} = x. 
\] 
En intégrant des deux c\^otés nous obtenons la solution générale  $\frac{1}{y} =- \left(\dfrac{x^2}{2} + C\right)$ avec $C\in\eR$, qu'on peut expliciter par $y = \frac{-2}{x^2 + C}$. La valeur de $C$ qui correspond à $g$ est facile à trouver : $ -\frac{1}{2} =  \frac{-2}{1 + C}$ implique $C=3$, c'est à dire $g(x) = \frac{-2}{x^2 + 3}$.
\item[(3)]
  \begin{enumerate}
  \item En intégrant des deux c\^otés de l'équation nous obtenons  $\displaystyle \frac{y^3}{3}=\frac{x^3}{3}+C$, donc la forme explicite de la la solution générale de cette équation est $\displaystyle y=\left(x^3+C\right)^{1/3}$.
  \item La solution particulière qui satisfait la condition $\phi_1(0)=0$ est $\phi_1(x)=x$ car la constante $C$ doit \^etre nulle et $\left(x^3\right)^{1/3}=x$.
La solution particulière qui satisfait la condition $\phi_2(0)=1$ est $\phi_2(x)=\left(x^3+1\right)^{1/3}$, celle qui satisfait la condition $\phi_3(0)=-1$ est $\phi_3(x)=\left(x^3-1\right)^{1/3}$. 
  \end{enumerate}
\end{enumerate}


\end{corrige}   
