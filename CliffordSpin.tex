Bibliography for Clifford algebras, spin group and related topics are \cite{memP,Michelson,Witkowski,mellor,ResEtaDiracType}. More algebraic point of view  can be found in \cite{Fult,Chevalley}. More details about ``square rooting'' second order differential operators are in \cite{Bronn}. For physical concerns, the reader should refer to \cite{Weinberg,Peskin,schwabl}. 

\section{Invitation : Clifford algebra in quantum field theory}\index{quantum!field theory}
%++++++++++++++++++++++++++++++++++++++++++++++++

\label{Secqft}
\subsection{Schrödinger, Klein-Gordon and Dirac}
%----------------------------------

The origin of the Klein-Gordon equation is almost the same as the one of the Schrödinger: one replace physical functions by operators. For a free particle, the correspondence are
\begin{align*}
 \textrm{energy}&& E&\rightarrow i\hbar\dsd{}{t},\\
 \textrm{momentum}&& \overline{p}&\rightarrow -i\hbar\overline{ \nabla }.
\end{align*}
The Schrödinger equation\index{equation!Schrödinger} (which is the non relativistic quantum wave equation) comes from replacement in the non non relativistic expression of the Hamiltonian
\[
E=\frac{\overline{p}^2}{2m}\longrightarrow\left(\partial_t-\frac{i\hbar}{2m}\nabla^2\right)\psi=0,
\]
while the Klein-Gordon\index{equation!Klein-Gordon} one (which is the relativistic quantum wave equation) comes from the relativistic corresponding equation:
\[
E^2=\overline{p}^2c^2+m^2c^4\longrightarrow\left(\partial\hmu\partial_{\mu}+(\frac{mc}{\hbar})^2\right)\psi=0.
\]

This is a second order differential equation; there are however no ``law of nature''{} which forbid a first order equation. We try
\[
 i\hbar\dsd{\psi}{t}
 =\left(\frac{\hbar c}{i}\alpha^k\partial_k+\beta mc^2\right)\psi\equiv \hat{H}\psi.
\]

There are some physical constraints on the coefficients $\alpha^k$ and $\beta$. We will study one of them: we want the components of $\psi$ to satisfy the Klein-Gordon equation, so that the plane waves fulfill the fundamental relation $E^2=p^2c^2+m^2c^4$.

In order to see the implications of this constraint on the coefficients, we apply two times the operator $\hat{H}$, and we compare the result with the Klein-Gordon equation. We find:
\begin{subequations}
\begin{align}
 \alpha^i\alpha^j+\alpha^j\alpha^i&=2\delta^{ij}\mtu,\\
 \alpha^i\beta+\beta\alpha^i&=0,\\
 (\alpha^i)^2=\beta^2&=\mtu.
\end{align}
\end{subequations}
%
If we define $\gamma^0=\beta$ and $\gamma^i=\beta\alpha^i$, we find that the matrices $\gamma^{\mu}$ have to give a representation of the Clifford algebra\footnote{Don't be afraid with the extra minus sign: the quantum field theory is most written with the metric $(+,-,-,-)$ instead of $(-,+,+,+)$.}:
\begin{equation}\label{cliffphys}
	\gamma\hmu\gamma\hnu+\gamma\hnu\gamma\hmu=2\eta^{\mu\nu}\mtu.
\end{equation}
The Dirac equation reads
\[
\left(-i\gamma\hmu\partial_{\mu}+\frac{mc}{\hbar}\right)\psi=0.
\]
If we want to perform some computation with the quantum field theory, we need an explicit form for the $\gamma$'s; that's the reason why we study representations of the Clifford algebra. The \defe{Dirac operator}{dirac!operator} $\Dir$ is the operator which lies in the Dirac equation:
\begin{equation}
 \label{dirflat}\Dir=\sum_{\mu=0}^3\gamma\hmu\dsd{}{x\hmu}.
\end{equation}

\subsection{Lorentz algebra}
%---------------------------

There is an other physical reason (which is in fact the same, but differently presented) justifying the study of the Clifford algebra. The quantum field theory need representation of the Lorentz algebra\footnote{When one think to real infinitesimal rotation matrices, the presence of $i$ seems not natural, but one redefines $J\to iJ$ for formalism reasons.}\index{lorentz!algebra}
\[
 [J^{\mu\nu},J^{\rho\sigma}]=i(\eta^{\nu\rho}J^{\mu\sigma}-\eta^{\mu\rho}J^{\nu\sigma}
 -\eta^{\nu\sigma}J^{\mu\rho}+\eta^{\mu\sigma}J^{\nu\rho}).
\]
A proof of these relations is given in lemma \ref{LemCommsopqAlg}. Dirac had a trick to find such $J$ matrices from a representation of the Clifford algebra. If we have $n\times n$ matrices $\gamma_{\mu}$ such that
\[
	\gamma^{\mu}\gamma^{\nu}+\gamma^{\nu}\gamma^{\mu}=2\eta^{\mu\nu}\mtu_{n\times n},
\]
a $n$-dimensional representation of the Lorenz algebra is obtained by
\[
	S^{\mu\nu}=\frac{i}{4}\left[\gamma^{\mu},\gamma^{\nu}\right].
\]

\begin{lemma}		\label{LemCommsopqAlg}
The matrices of $\so(p,q)$ satisfy the definition relation
\begin{equation}
	M^t\eta+\eta M=0,
\end{equation}
and if $M^{ab}$ is the ``rotation'' in the place of directions $a$ and $b$ (i.e. a trigonometric or an hyperbolic rotation following that $a$ and $b$ are of the same type or not), then the action on $\eR^{(p,q)}$ is given by $(x')^{\mu}=(M^{ab})^{\mu}_{\nu}x^{\nu}$ with
\begin{equation}
	(M^{ab})^{\mu}_{\nu}=\eta^{a\mu}\delta^b_{\nu}-\eta^{b\mu}\delta^a_{\nu}.
\end{equation}
The commutation relations are given by
\begin{equation}
	[M^{ab},M^{cd}]=-\eta^{ac}M^{bd}+\eta^{ad}M^{bc}+\eta^{bc}M^{ad}-\eta^{bd}M^{ac}.
\end{equation}
Notice that $M^{ab}=-M^{ba}$. 
\end{lemma}
See section 12.5 of \cite{Schomblond_em}. By a simple redefinition $J=iM$, one obtains 
\begin{equation}			\label{EqJJietaJcomm}
	[J,J]=i\eta J
\end{equation}
instead of $[M,M]=\eta M$, and the matrices $J$ are Hermitian. Here $\eta$ is the matrix $\eta=diag(\underbrace{+,\ldots,+}_{\text{$p$ times}},\underbrace{-,\ldots,-}_{\text{$q$ times}})$. As convention, we say that a direction corresponding to a \emph{positive} entry in the metric is a \emph{time} direction, while the spatial directions are negative. That corresponds to the convention of page \pageref{PgDefsGenre} to say that a \emph{time-like} vector has positive norm.

\section{Clifford algebra}
%++++++++++++++++++++++++++

\subsection{Definition and universal problem}
%------------------------------------------------------


\begin{definition}
Let $V$ be a (finite dimensional) vector space and $q$, a bilinear quadratic form over $V$. The \defe{Clifford algebra}{Clifford!algebra}\index{algebra!Clifford} $\Cliff(V,q)$ is the unital associative algebra generated by $V$ subject to the relation
\begin{equation}\label{501r1}
       v\cdot v=q(v)
\end{equation}
for all $v$ in $\Cliff(V,q)$. Here the dot denotes the algebra product and $q(v)$ means $q(v,v)$.
\end{definition}
Theorem \ref{tho_Cliffunif} proves unicity of such an algebra, so that it makes sense.

\begin{remark}
The relation \eqref{501r1} is no more a restriction for the elements in $\Cliff(V,q)$ than a restriction on the choice of the algebra product.
\end{remark}


\begin{theorem}\index{Clifford!algebra!Universal property of}
Let $E$ be an unital associative algebra and $\dpt{j}{V}{E}$ a linear map such that
\begin{equation}
    j(v)\cdot j(v)=q(v)1.        \label{102r1}
\end{equation}
Then we have an unique extension of $j$ to a homomorphism $\dpt{\tilde\jmath}{\Cliff(V,q)}{E}$. Moreover, $\Cliff(V,q)$ is the unique associative algebra which have this property for all such~$E$.
\[
\xymatrix{
    \Cliff(V,q) \ar@{^{(}->}[d]_{\displaystyle i} \ar[rd]^{\displaystyle\tilde{j}} &  \\
    V \ar[r]_{\displaystyle j} & D
  }
\]
\label{tho_Cliffunif}
\end{theorem}
This theorem can be seen as a definition of $\Cliff(V,q)$.

\begin{proof}
The proof shall belongs two parts: the first one will show how to extend $j$ and why it is unique, and the second one will prove the unicity of $\Cliff(V,q)$.

We begin by define the extension of $j$. First note that any linear map $\dpt{f}{V}{E}$ can be extended to an algebra homomorphism $\dpt{\overline{f}}{T(V)}{E}$ in only one way. Indeed, the homomorphism condition require that $\overline{f}(v\otimes w)=f(v)\cdot f(w)$.  The whole map $\overline{f}$ is then well defined by the data of $f$ alone.

As far as the map $j$ is concerned, we have the relation \eqref{102r1} which says that $\overline{j}(\mI)=0$. Indeed,
\begin{equation}
 \ovj(v\otimes v-q(v)\cdot(1))=\ovj(v)\cdot\ovj(v)-q(v)\ovj(1)
                              =j(v)\cdot j(v)-q(v)1
                              =0.
\end{equation}
Thus $\dpt{\ovj}{T(V)}{E}$ is a class map for $\mI$, and we can descent $\ovj$ from $T(V)$ to $\Cliff(V,q)$, We define
$\dpt{\tilde\jmath}{\Cliff(V,q)}{E}$ by
\begin{equation}
         \tilde\jmath[x]=\ovj(x)
\end{equation}
where $[x]$ is the class of $x$. That's for the existence part.

The unicity is clear: $f_1=f_2$ on $V$ implies that $\overline{f_1}=\overline{f_2}$ on $T(V)$. Thus $\tilde{f_1}=\tilde{f_2}$ on $\Cliff(V,q)$.

We turn now our attention to the unicity of $\Cliff(C,q)$. Let $D$ be an unital associative algebra such that
\begin{enumerate}
\item $V\subset D$,
\item For any unital associative algebra $E$ and for any $\dpt{f}{D}{E}$ such that $f(v)\cdot f(v)=-q(v)1$, there exists only one homomorphic map $\dpt{\tilde{f}}{D}{E}$ which extend $f$.
\end{enumerate}
We should find a homomorphic map $\dpt{\tilde{k}}{D}{\Cliff(V,q)}$. Let $i$ be the canonical injection $\dpt{i}{V}{D}$. Clearly, we have a homomorphism $V\rightarrow i(V)$. Now, as a space $E$, we can take $\Cliff(V,q)$; $i$ can be seen as a linear map $\dpt{i}{V}{\Cliff(V,q)}$ such that $i(v)\cdot i(v)=q(v)1$. The assumptions say that $i$ can be extended (in only one way) to a homomorphic map $\dpt{\tilde{i}}{D}{\Cliff(V,q)}$.

The Clifford algebra is thus unique up to a homomorphism.

\end{proof}

What we proved is the following: if for any $E$ and for any $\dpt{j}{V}{E}$ such that $j(v)\cdot j(v)=q(v)1$, there exist an unique $\dpt{\tilde{j}}{D}{E}$ which extend $j$, then $D=\Cliff(V,q)$ up to a homomorphism. One ays that $\Cliff(V,q)$ solve an \defe{universal problem}{universal!problem}.

An explicit construction of $\Cliff(V,q)$ can be achieved in the following way. We consider the tensor algebra $T(V)=\bigoplus_{n\geq 0}\left(\otimes^nV\right)=\eC\oplus V\oplus(V\otimes V)\oplus\ldots$ over $V$ the two-sided ideal $\mI$ generated by elements of the form $v\otimes v-q(v)1$. The  \defe{Clifford algebra}{Clifford!algebra}\index{algebra!Clifford} for $(V,q)$ is given by\nomenclature[G]{$\Cliff(p,q)$}{Clifford algebra of $\eR^{1,3}$}
\begin{equation}	\label{defI}
	\Cliff(p,q):=T(V)/\mI
\end{equation}
in which product of $\Cliff(V,q)$ is naturally defined by $[a]\otimes[b]=[a\otimes b]$ if $[a]$ is the class of $a\in T(V)$.

Let us now fix some notations more adapted to what we want to do. Let $V=\eR^{p,q}$ the vector space $\eR^{p+q}$ endowed with a diagonal metric which contains $p$ plus sign and $q$ minus signs. For $v$, $w\in V$, the inner product with respect to the metric $\eta$ of $v$ by $w$ will be denoted by $\eta(v,w)$.  The norm on $V$ will be defined by $\|v\|^2=-\eta(v,v)$. It is neither positive defined, nor negative defined. The explanation of the minus sign will come soon. The Clifford algebra is the quotient $\Cliff(p,q):=T(V)/\mI$ of the tensor algebra by the two-sided ideal $\mI$ generated by elements of the form
\[
	(v\otimes w)\oplus (w\otimes v)\oplus 2\eta(v,w)1
\]
 for $v,w$ in $V$. Depending on the context, we will often use the notations $\Cliff(\eta)$ or $\Cliff(V)$ or $\Cliff(p,q)$. The algebra product is $[x]\cdot[y]=[x\otimes y]$, $x$, $y\in T(V)$.  As long as $z\in V\subset\Cliff(p,q)$, the expression $\eta(z,z)$ is meaningful. The definition of $\Cliff$ is such that $z\cdot z=-\eta(z,z)$. This leads to the somewhat surprising formula  $z^2=\|z\|^2=-\eta(z,z)$.

\subsection{First representation}
%----------------------------------

Let $(V,g)$ be a metric vector space and $\Cliff(V,g)$ its Clifford algebra. For each $v\in V$, we define the two following elements of $\End_{\eR}(\Wedge V)$:
\begin{subequations}
\begin{align}
\epsilon(v)\big( u_1\wedge\cdots\wedge u_k \big)&=v\wedge u_1\wedge\cdots\wedge u_k\\
\iota(v)\big( u_1\wedge\cdots\wedge u_k \big)&=\sum_{j=1}^k(-1)^{j-1}g(u,u_j)u_1\wedge\cdots\wedge\hat u_j\wedge\cdots\wedge u_j.
\end{align}
\end{subequations}
One has $\epsilon(v)^2=0$ and $\iota(v)^2=0$ because $v\wedge v=0$. In order to understand the latter, we wonder what are the terms with $g(v,u_i)g(v,u_j)$ are in
\[ 
  \iota(v)^2\big( u_1\wedge\cdots\wedge u_k \big)=\sum_{l=1}^k(-1)^{j-1}g(v,u_j)\sum_{l=1}^{k-1}(-1)^{l-1}g(v,u_l)u_1\wedge\hat u_l\wedge\hat u_j\wedge\cdots\wedge u_k.
\]
Let's suppose $i<j$. The first term comes when the first $\iota(v)$ acts on $u_j$, its sign is given by $(-1)^{j-1}(-1)^{i-1}$. The second term has the same $(-1)^{i-1}$, but in this term, $u_j$ is on the position $j-1$ because $u_i$ has disappeared.

Now we use $c(v)=\epsilon(v)+\iota(v)$ which fulfils for all $u$, $v\in V$:
\[ 
\begin{split}
c(v)^2&=g(v,v,)1\\
c(u)v(v)+c(v)c(u)&=2g(u,v)1.
\end{split}  
\]
Therefore $c$ can be extended to a representation $c\colon \Cliff(V,g)\to \End(\Wedge V)$. If $\{ e_0,\cdots e_n \}$ is an orthonormal basis of $V$ (i.e. $g(e_i,e_j)=\eta_{ij}$); in this case the $c(e_j)$ are anticommuting and a basis of $\Cliff(V,g)$ is given by 
\begin{equation}
 \{ c(e_{k_1})\cdots c(e_{k_r})\tq 1\leq k_1<\cdots<k_r\leq n \}.
\end{equation}

\subsection{Some consequences of the universal property}
%---------------------------------------------------

The map $-\id|_V$ extends to $\alpha\in\Aut\big( \Cliff(V) \big)$,
\[ 
  \alpha(v_1\cdots v_r)=(-1)^rv_1\cdots v_r
\]
($v_i\in V$) and provides a graduation
\[ 
  \Cliff(V)=\Cliff^0(V)\oplus\Cliff^1(V).
\]
The map $\tau\colon \Cliff(V)\to \Cliff(V)$ extends $\id|_V$ to an anti-homomorphism:
\begin{equation}
\tau(v_1\cdots v_r)=v_r\cdots v_1.
\end{equation}

The \defe{complexification}{complexification!of Clifford algebra} of $\Cliff(V,g)$ is 
\[ 
  \CCliff(V,g):=\Cliff(V,g)\otimes_{\eR}\eC\simeq\Cliff(V^{\eC},g^{\eC}),
\]
the isomorphism being a $\eC$-algebra isomorphism. The $\eR$-linear operator $v\mapsto \overline{ v }$ in $V^{\eC}$ of complex conjugation extends to a $\eR$-linear automorphism $a\mapsto \overline{ a }$. We define the \defe{adjoint}{adjoint!in Clifford algebra} by
\begin{equation}
  a^*=\tau(\overline{ a })
\end{equation}

\subsection{Trace}
%---------------------

\begin{theorem}
There exists one an only one trace $\tr\colon \CCliff(V)\to \eC$ such that
\begin{enumerate}
\item $\tr(1)=1$,
\item $\tr(a)=0$ when $a$ is odd.
\end{enumerate}

\end{theorem}

\begin{proof}

Let $\{ e_1,\cdots,e_n \}$ be an orthonormal basis of $(V,g)$ and $a\in\CCliff(V)$. When one decomposes $a$ into the basis of $e_i$, one finds a lot of terms of each order. Since $\tr$ is a trace, when the $k_i$ are all different,
\[ 
\tr(e_{k_1}\cdots e_{k_{2r}})	=\tr(-e_{k_2}\cdots e_{k_{2r}} e_{k_1}
				=\tr(-e_{k_1}\cdots e_{k_{2r}})\\
\]
So the trace of any even element is zero. We decompose $a$ into
\[ 
  a=\sum_K a_K\prod_{i\in K}e_i
\]
where the sum is taken on the subsets of $\{ 1,\ldots, n \}$. A trace which fulfils the conditions must vanishes on even (but non zero) elements as well as on odd elements, so the only possible form is
\[ 
  \tr a=a_{\emptyset}.
\]
Notice that in order to get this precise form, we used $\tr(1)=1$ and linearity. This proves unicity and existence. Now we have to prove that this is a good definition in the sense that an other choice of basis gives the same result. So we take a new orthonormal basis
\[ 
  e'_j=\sum_{k=1}^nH_{jk}e_k
\]
with $H^tH=\mtu_{n\times n}$. Now we have
\[ 
  a=\sum_{K}^{}a_K\prod_{i\in K}e_i=\sum_{K}^{}a'_K\prod_{i\in K}e'_i,
\]
and we will prove that $a_{\emptyset}=a'_{\emptyset}$. Let's compute a lot:
\[ 
\begin{split}
   e_i'e_j'&=\sum_{k}^{}\sum_{l}^{}H_{ik}H_{jl}e_ke_l\\
		&=\sum_{k=l}^{}H_{ik}H_{jl}e_{k}e_{l}+\sum_{k\neq l}^{}H_{ik}H_{jl}e_{k}e_{l}\\
		&=\sum_{k}^{}H_{ik}H_{jk}1+\sum_{k\neq l}^{}H_{ik}H_{jl}e_{k}e_{l}\\
		&=(HH^t)_{ij}1+\sum_{k\neq l}^{}H_{ik}H_{jl}e_{k}e_{l}.
\end{split}  
\]
The sense of this formula is that when $i\neq j$, the product $e'_{i}e'_{j}$ has no term of order zero. In other terms, as long as we only have terms of order zero, one and two, a change $e\to e'$ does not change the term of order zero. We are now going to an induction proof: we want to prove that $e'_{j_{1}}\ldots e'_{j_{2r}}e'_{l}e'_{k}$ has no scalar term assuming that no even combination has scalar terms up to $2(r-1)$. It reads
\[ 
  \sum_{\text{$K$ even}}^{}a_{K}\prod_{i\in K} e_{i}e'_{l}e'_{k},
\]
therefore we just have to look at terms of the form
\[ 
  e_{j_{1}}\ldots e_{j_{2r}}\Big( (HH)^t_{kl}1-\sum_{i\neq j}^{}C_{kl}^{ij}e_{i}e_{j} \Big)
\]
where the $e_{j_{l}}$ are all different. The first term cannot produce a scalar term. In order to find a scalar term in $e'_{j_{1}}\ldots e'_{j_{2r}}e_{k}e_{l}$, we begin to look at terms whose decomposition of $e'_{j_{1}}\ldots e'_{j_{2r}}$ ends by $e_{l}e_{k}$, i.e.
\[ 
  H_{j_{2r-2}l}H_{j_{2r-1}k}e'_{j_{1}}\ldots e'_{2r-3}e_{l}e_{k}e_{k}e_{l}.
\]
The induction assumption says that there are no scalar term in $e'_{2r-3}e_{l}e_{k}e_{k}e_{l}$.

\end{proof}
One can prove that $\CCliff(C)$ is a Hilbert space with the scalar product 
\begin{equation}
\langle a |b\rangle=\tr(a^*b).
\end{equation}


Let $v\in V$ with $g(v,v)=1$ (thus in $\Cliff(V)$, we have $v^2=1$); since $v=\overline{ v }$, we have
\[ 
  a^*v=vv^*=v^2=1.
\]

\begin{lemma}
The maps $a\mapsto ua$ and $a\mapsto au$ are unitary if and only if $uu^*=u^*u=1$.
\end{lemma}

\begin{proof}
We pick $\lambda\in U(1)$ and $w=\lambda v\in V^{\eC}$ which fulfils $w^*w=1$. This is the most general element such that $ww^*=w^*w=1$. Now for an arbitrary $a$, $b\in\CCliff(V)$, we compute the two followings:
\[ 
  \langle wa|wb\rangle=\tr\big( (wa)^*wb \big)
		=\tr\big( a^*w^*wb \big)
		=\tr(a^*b)
		=\langle a|b\rangle,
\]
and
\[ 
  \langle aw|bw\rangle=\tr\big( w^*a^*bw \big)
		=\tr(ww^*a^*b)
		=\tr(a^*b)
		=\langle a|b\rangle.
\]
This proves that $a\mapsto wa$ and $a\mapsto aw$ are two unitary operators on the Hilbert space $\CCliff(V)$.

For the converse, we impose for all $a$, $b\in\CCliff(V)$:
\[ 
  \langle ua|ub\rangle=\tr(ba^*u^*u)\stackrel{!}{=}\tr(ba^*).
\]
In particular with $a^*b=1$, $\tr(u^*u)=\tr(1)=1$, thus the scalar part of $u^*u$ is $1$. So we write $u^*u=1+f$ where $f$ is non scalar, and for any $x\in\CCliff(V)$ , we have
\[
   \tr(x)=\tr(xu^*u)=\tr(x)+\tr(xf).
\]
We conclude that $\tr(xf)=0$, and therefore that $f=0$.
\end{proof}


\section{Spinor representation}
%+++++++++++++++++++++++++++++

For the spinor representation, we restrict ourself to the even case $p+q=2n$.

The aim of this subsection is to find some faithful representations of the complex Clifford algebra $\Cliff^{\eC}(p,q)$. In order to achieve this, we first consider $V^{\eC}$, the complex vector space of $V$ with an orthonormal basis $\{ e_1,\cdots,e_{p-1},e_p,\cdots,e_q  \}$. The metric is $\eta(e_k,e_k)=1$ and $\eta(e_{p+k},e_{p+k})=-1$ for $k=0,\cdots,p-1$. We use the following basis:
\begin{align}
f_k&=\frac{1}{2}(e_k+e_{p+k}),& g_k&=\frac{1}{2}(e_k-e_{p+k}),\\
 f_{p+s}&=\frac{1}{2}(e_{2p+2s}+ie_{2p+2s+1}),& g_{p+s}&=\frac{1}{2}(e_{2p+2s}-ie_{2p+2s})
\end{align}
for $k=0,\cdots,p-1$.
We note that $\{f_0,g_0\}$ spans a $\eC^2$-space which is $\eta$-orthogonal to the one which is spanned by $\{f_1,g_1\}$. The following two  spaces will prove to be useful:
\begin{subequations}
\begin{align}
  W           &=\Span_{\eC}\{f_0,f_1\}\simeq\eC^2,\\
 \underline{W}&=\Span_{\eC}\{g_0,g_1\}\simeq\eC^2.
\end{align}
\end{subequations}
\nomenclature{$W,\underline{W}$}{Totally isotropic subspace}
It is easy to compute the various products; among others we find
\begin{equation}
 \eta(f_0,f_0)=0,\quad
 \eta(f_1,f_0)=0,\quad
  \eta(f_1,f_1)=0;
\end{equation}
so that for any $w\in W$, we have $\scal{w}{w}=0$; for this reason, we say that $W$ is a \defe{completely isotropic}{isotropic!subspace!completely} subspace of $(V^{\eC},\eta^{\eC})$. The space $\underline{W}$ has the same property.

\begin{proposition}
We have
\begin{equation}
  \underline{W}\simeq W^*,
\end{equation}
where $W^*$ is the dual space of $W$. By $\simeq$ we mean that there exists a linear bijective map $\dpt{\psi}{\underline{W}}{W^*}$.
\end{proposition}
\begin{proof}
For each $\uw\in\uW$, we define $\dpt{\psi(\uw)}{W}{\eC}$ by
\[
   \psi(\uw)(w)=\eta(w,\uw).
\]
We first show that the map $\psi$ is injective. Let $\uw\in\uW$ be so that $\psi(\uw)=0$. Thus for all $v\in W$, we have
\begin{eqnarray}
   \label{3101r1}\psi(\uw)v=\eta(\uw,v)=0.
\end{eqnarray}
By decomposing $\uw=ag_0+bg_1$ and taking successively $v=f_0$ and $v=f_1$, we see that $a=b=0$.

The next step is to see that the map $\psi$ is surjective. We know that $dim_{\eC}\uW=dim_{\eC}W^*=2$ and that $\psi(g_0)\neq 0$. Let's prove that $\{\psi(g_0),\psi(g_1)\}$ is a basis of $W^*$. It is clear by linearity that $\{\psi(ag_0):a\in\eC\}=\Span\{\psi(g_0)\}$. The fact that $\psi$ is injective  imposes that $\psi(g_1)$ doesn't belong to $\Span\{\psi(g_0)\}$. So $\{\psi(g_0),\psi(g_1)\}$ is a two-dimensional free subset of $W^*$, and therefore is a basis of $W^*$.
\end{proof}

We turn our attention to the exterior algebra $\Lambda W=\eC\oplus W\oplus(W\wedge W)\oplus\cdots\oplus\wedge^{p+q}W$\nomenclature{$\Lambda W$}{Space of spinor representation} of $W$. 
\nomenclature{$\dpt{\tilde\rho}{(\eR^{1+3})^{\eC}}{\End(\Lambda W )}$}{Spinor representation}

\begin{definition}
\index{endomorphism!of $\Lambda W$}
We define the homomorphism $\dpt{\tilde\rho}{V^{\eC}}{\End(\Lambda W)}$ by 
\begin{equation}
\begin{split}
 \tilde\rho(f_i)\alpha&=f_i\wedge\alpha,\\
 \tilde\rho(g_i)\alpha&=-\iota(g_i)\alpha
\end{split}
\end{equation}
($v\in V^{\eC}$, $\alpha\in\Lambda W$) where $\iota$ denotes the interior product defined in page \pageref{pg_DefProdExt}.\
\label{defrt}
\end{definition}
  More explicitly, for all $z\in\eC$ and for all $w,w'\in W$, we have
\begin{subequations}
\begin{align}
 \tilde\rho(f_i)z&=zf_i,&\tilde\rho(g_i)z&=0,\\
 \tilde\rho(f_i)w&=f_i\wedge w,&\tilde\rho(g_i)w&=-\eta(g_i,w)1,\\
 \tilde\rho(f_i)(w\wedge w')&=0,&\tilde\rho(g_i)(w\wedge w')&=-\eta(g_i,w)w'+\eta(g_i,w')w.
\end{align}
\end{subequations}
We will see that, \emph{via} some manipulations, $\tilde\rho$ provides a faithful representation of the Clifford algebra, the \defe{spinor representation}{spinor!representation}.
\index{representation!of Clifford algebra}

\begin{remark}
By ``endomorphism of $\Lambda W$'', we mean an endomorphism for the \emph{linear} structure of $\Lambda W$. We obviously not have $\tilde\rho(x)(\alpha\wedge\beta)=\tilde\rho(x)\alpha\wedge\tilde\rho(x)\beta$. 
\end{remark}

\begin{proposition}
The map $\tilde\rho$ is injective.
\end{proposition}

\begin{proof}
We have to show that $\tilde\rho(v)=0$ ($v$ in $V^{\eC}$) implies $v=0$. Any $v\in V^{\eC}$ can be written as
$v=a^if_i+b^ig_i$ with a sum over $i$. We first have that
\[
 \tilde\rho(a^if_i+b^ig_i)z=za^if_i=0,
\]
 but the $f_i$ are independents and then $a^i=0$. We can also write
\[
 \tilde\rho(b^0g_0+b^1g_1)f_1=-b^0\eta(g_0,f_1)-b^1\eta(g_1,f_1)=-\frac{b^1}{2}=0,
\]
 then $b^1=0$. The same with $f_0$ proves that $b^0=0$.
 \end{proof}

The homomorphism $\tilde\rho$ extends to the whole the tensor algebra of $V^{\eC}$ by the following definitions:
\begin{subequations}
\begin{align}
 \tilde\rho(1)              &=\id_{\Lambda W},\\
 \tilde\rho(e_k)            &=\tilde\rho(e_k),\\
 \tilde\rho(e_{k_1}\otimes\ldots\otimes e_{k_r})&=
                      \tilde\rho(e_{k_1})\circ\ldots\circ\tilde\rho(e_{k_r}).\label{eq:3101r2}
\end{align}
\end{subequations}
So we get $\dpt{\tilde\rho}{T(V^{\eC})}{\End(\Lambda W)}$.  The following proposition will allow us to descent $\tilde\rho$ to a representation of the Clifford algebra.

\begin{proposition}
The homomorphism $\tilde\rho$ maps $\mI$ to $0$: $\tilde\rho(\mI)=0$.
\end{proposition}

\begin{probleme}
This proposition is wrong: there is a double covering.

Moreover, there is a sign problem in the proof: the sign in the first lines is not the one used in the definition of the Clifford algebra.
\end{probleme}


\begin{proof}
We have to check the following:
\[\tilde\rho(v\otimes w\oplus w\otimes v-2\eta(v,w)1)=0\]
for any choice of
 $v,w$ in $\{e_0,e_1,e_2,e_3\}$.
  Here we will just check it explicitly for $v=e_0$ and $w=e_1$. The computation uses the definition \eqref{eq:3101r2}:
\begin{equation}
\begin{split}
\tilde\rho(e_0\otimes e_1\oplus e_1\otimes
             e_0-2\eta(e_0,e_1)&=\tilde\rho(e_0)\circ\tilde\rho(e_1)+\tilde\rho(e_1)\circ\tilde\rho(e_0)\\
                               &=2\left[\tilde\rho(f_0)^2-\tilde\rho(g_0)^2\right].
\end{split}
\end{equation}
It is easy to see that $\tilde\rho(f_0)^2=0$:
\begin{equation}
 \tilde\rho(f_0)^2\left[z\oplus w\oplus w_1\wedge w_2 \right]=\tilde\rho(f_0)[zf_0\oplus f_0\wedge w]
                                                   =zf_0\wedge f_0,
							=0.
\end{equation}
 The proof that $\tilde\rho(g_0)^2=0$ is almost the same:
\[ 
 \tilde\rho(g_0)^2\left[z\oplus w\oplus w_1\wedge w_2 \right]
 =\tilde\rho(g_0)[-\eta(g_0,w)1\oplus-\eta(g_0,w_1)w_2\oplus\eta(g_0,w_2)w_1].
\]

\end{proof}

We can now see $\tilde\rho$ as a map $\dpt{\tilde\rho}{\Cliff^{\eC}(p,q)}{\End(\Lambda W )}$. By construction, it is a homomorphism and, thus, is a representation of $\Cliff^{\eC}(p,q)$ on $\Lambda W$. For compactness, we use the notation \index{dirac!matrices}\nomenclature{$\gamma_i$}{Abstract definition of Dirac matrices}
\begin{equation}
 \label{defgamma}\gamma_a:=\sqrt{2}\tilde\rho(e_a). 
\end{equation}

\begin{lemma}
The $\gamma$'s operators satisfy the following relation:
\begin{equation}\label{3101r3}
  \gamma_a\gamma_b+\gamma_b\gamma_a=-2\eta_{ab}\mtu.
\end{equation}
\label{3101l1}
\end{lemma}

\begin{proof}
We have to check this equality on any element of $\Lambda W$. If we choose
$w_1=af_0+bf_1$ and $w_2=a'f_0+b'f_1$,
 we find $w_1\wedge w_2=(ab'-ba')f_0\wedge f_1$.

For example, we will explicitly check \eqref{3101r3} with $a=b=0$, i.e. $\tilde\rho(e_0)\circ\tilde\rho(e_0)=\frac{1}{2}\id$, which proves that $\gamma_0\circ\gamma_0=\id$.
 \begin{equation}
\begin{split}
   \tilde\rho(e_0)^2[z\oplus w\oplus(ab'-ba') f_0\wedge f_1]&=\tilde\rho(f_0+g_0)^2[z\oplus w\oplus(ab'-ba') f_0\wedge f_1]\\ 
                                              &=\tilde\rho(f_0+g_0)\Big[zf_0\oplus f_0\wedge w\oplus-\eta(g_0,w)1\\
                                                            &\qquad-(ab'-ba')\eta(g_0,f_0)f_1\\
                                                             &\qquad+(ab'-ba')\eta(g_0,f_1)f_0\Big]\\
                                              &=\frac{1}{2}(z\oplus w\oplus(ab'-ba') f_0\wedge f_1).
\end{split}				      
\end{equation}
\end{proof}

\begin{lemma}
For any sequence $i_0,\ldots i_3$ of $0$ and $1$ (with at least one of them equals to $1$), we have
\begin{equation}
 \tr(\gamma_0^{i_0}\cdots \gamma^{i_{2n-1}}_{2n-1})=0.
\end{equation}
We take the convention that $\gamma_a^0=\mtu$.
\label{3101l2}
\end{lemma}

\begin{proof}
 If the number of nonzero $i_k$ is even (say $2m$), we have:
\[
	\tr(\gamma_{a_1}\ldots\gamma_{a_{2m}})=\tr(\gamma_{a_{2n}}\gamma_{a_1}\ldots\gamma_{a_{2m-1}})
\] 
because the trace is invariant under cyclic permutations. But we can also permute $\gamma_{a_{2m}}$ with the $2m-1$ other $\gamma$'s.  $\tr(\gamma_{a_1}\ldots\gamma_{a_{2m}})=(-1)^{2n-1}\tr(\gamma_{a_{2m}}\gamma_{a_1}\ldots\gamma_{a_{2m-1}})$ because each permutation gives an extra minus sign (\hbox{lemma \ref{3101l1}}). Then the trace is zero.

If the number of nonzero $i_k$ is odd (say $2m-1$). Let $i_a=0$ (we restrict ourself to the even dimensional case). We have $\tr(A)=-\eta_{aa}\tr(A\gamma_a\gamma_a)$. Using once again the cyclic invariance of the trace, $\tr(\gamma_{a_1}\ldots\gamma_{a_{2m-1}}\gamma_a\gamma_a)=\tr(\gamma_a\gamma_{a_1}\ldots\gamma_{a_{2m-1}}\gamma_a)$. But, if we permute the \emph{first} $\gamma_a$ with the $2m-1$ first $\gamma$'s, we find \hbox{$\tr(\gamma_{a_1}\ldots\gamma_{a_{2m-1}}\gamma_a\gamma_a)=-\tr(\gamma_a\gamma_{a_1}\ldots\gamma_{a_{2m-1}}\gamma_a)$ }, and the trace is zero again.
\end{proof}

\begin{proposition}
The subset
\[
	\left\{\mtu,\dga{a}{b}\,(a<b),\tga{a}{b}{c}\,(a<b<c), \cdots, \gamma_{0}\cdots\gamma_{2n} \right\}
\]
 is free in $\End(\Lambda W)$.
\end{proposition}

\begin{proof}
We consider a general linear combination of these operators:
\[
 E=\lambda\mtu+\sum_a\lambda_a\gamma_a+\sum_{a<b}\lambda_{ab}\dga{a}{b}+\ldots+
 \sum_{a<b<c<d}\lambda_{abcd}\qga{a}{b}{c}{d}.
\]
The claim is that if $E=0$, then all the coefficients $\lambda_{(\ldots)}$ must be zero. First note that $Tr(E)=0=\lambda$ by lemma \ref{3101l2}. It is also clear that $Tr(\gamma_iE)=0=\lambda_i$. In order to see that $\lambda_{ij}=0$, we compute $Tr(\gamma_j\gamma_iE)=0=\lambda_{ij}$. And so on.
\end{proof}

How many operators does we have in this free system ? Any operators in this system can be written as $\gamma_{0}^{i_{0}},\cdots\gamma^{i_{2n-1}}_{2n-1}$ with $i_k$ equal to zero or one. Thus we have $2^{2n}$ operators. On the other hand, we know that $dim_{\eC}\Lambda W=2p+2$, and then that $dim_{\eC}\End(\Lambda W)=4^2=16$. The result is that $\{ \gamma_{0}^{i_{0}},\cdots\gamma^{i_{2n-1}}_{2n-1}  \tq i_k=0\,or\,1\}$ is a basis of $\End(\Lambda W)$. In other words (if we suppose a suitable ordering), the image by $\tilde\rho$ of 
\[ 
B=\{1,e_a,e_a\otimes e_b,e_a\otimes e_b\otimes e_c,e_a\otimes e_b\otimes e_c\otimes e_d\}
\]
 is a basis of $\End(\Lambda W)$.

If $B$ is a basis of $C^{\eC}_{(p,q)}$, then $\tilde\rho$ is bijective and thus isomorphic.  Therefore, we expect $\dpt{\tilde\rho}{C^{\eC}_{(p,q)}}{\End(\Lambda W)}$ to be a faithful representation\index{representation!of Clifford algebra}. It is not difficult to see that $B$ is indeed a basis thanks to the equivalence relation.

\subsection{Explicit representation}
%------------------------------

First, we choose a basis for $\Lambda W$:
\begin{eqnarray}\label{102r2} 1=\left(\begin{matrix}
1 \\
0 \\
0 \\
0
\end{matrix}\right),\quad
f_0=\left(\begin{matrix}
0 \\
1 \\
0 \\
0
\end{matrix}\right),\quad
f_1=\left(\begin{matrix}
0 \\
0 \\
1 \\
0
\end{matrix}\right),\quad
f_0\wedge f_1=\left(\begin{matrix}
0 \\
0 \\
0 \\
1
\end{matrix}\right).
\end{eqnarray}
 Here is the explicit computation for the matrix $\gamma_0$ in this basis. First remark that $\tilde\rho(e_0)1=f_0$, $\tilde\rho(e_0)f_0=\frac{1}{2}$, $\tilde\rho(e_0)f_1=f_0\wedge f_1$, $\tilde\rho(e_0)(f_0\wedge f_1)=\frac{1}{2} f_1$. Then
\begin{equation}
\begin{split}
 \gamma_0\left(\begin{matrix}1 \\0 \\0 \\0\end{matrix}\right)=\sqrt{2}\left(\begin{matrix}0 \\1 \\0  \\0\end{matrix}\right),\quad
 \gamma_0\left(\begin{matrix}0 \\1 \\0 \\0\end{matrix}\right)=\sqrt{2}\left(\begin{matrix}\frac{1}{2} \\0 \\0 \\0\end{matrix}\right),\\
 \gamma_0\left(\begin{matrix}0 \\0 \\1 \\0\end{matrix}\right)=\sqrt{2}\left(\begin{matrix}0 \\0 \\0 \\1\end{matrix}\right),\quad
 \gamma_0\left(\begin{matrix}1 \\0 \\0 \\0\end{matrix}\right)=\sqrt{2}\left(\begin{matrix}0 \\0 \\\frac{1}{2} \\0\end{matrix}\right).
\end{split}
\end{equation}
This allows us to write down $\gamma_0$; the same computation gives the other matrices.\index{dirac!matrices}\nomenclature{$\gamma_i$}{Explicit form of gamma matrices}
\begin{equation}
\begin{split}
\gamma_0=\sqrt{2}\begin{pmatrix}
0 & \frac{1}{2} & 0 & 0 \\
1 & 0 & 0 & 0 \\
0 & 0 & 0 & \frac{1}{2} \\
0 & 0 & 1 & 0
\end{pmatrix}, \qquad
\gamma_1=\sqrt{2}\left(\begin{matrix}
0 & -\frac{1}{2} & 0 & 0 \\
1 & 0 & 0 & 0 \\
0 & 0 & 0 & -\frac{1}{2} \\
0 & 0 & 1 & 0
\end{matrix}\right),\\
\gamma_2=\sqrt{2}\begin{pmatrix}
0 & 0 & -\frac{1}{2} & 0 \\
0 & 0 & 0 & \frac{1}{2} \\
1 & 0 & 0 & 0 \\
0 & -1 & 0 & 0
\end{pmatrix},\qquad
\gamma_3=\sqrt{2}\begin{pmatrix}
0 & 0 & -\frac{i}{2} & 0 \\
0 & 0 & 0 & \frac{i}{2} \\
-i & 0 & 0 & 0 \\
0 & i & 0 & 0
\end{pmatrix}.
\end{split}
\end{equation}
It is easy to check that these matrices satisfies \eqref{3101r3}. 

Notice that, up to a suitable change of basis in $\Lambda W $, these are the usual Dirac matrices\index{dirac!matrices}. Indeed we actually solved the physical problem to find a representation of the algebra \eqref{cliffphys}.  We understand by the way why do physicists work with $4$-components spinors: the $\gamma$'s are operators on the four-dimensional space $\Lambda W$; hence the Dirac operator will naturally acts on four-components objects.

The main result of this section is an explicit faithful representation of $\CCliff(p,q)$. This allows us to write a \defe{Dirac operator}{dirac!operator!on $\protect\eR^{1,3}$} which solve (see the invitation \ref{Secqft} and \cite{Bronn}) the problem  to find a ``square root'' of the d'Alembert operator: the differential operator $\Dir=\gamma\hmu\partial_{\mu}$ satisfies $\Dir^2=\Box$.

\subsection{A remark}  % C'est ce qui arrive quand je ne sais pas quel titre donner.
%---------------------

 Let us compare the two faithful representations
\[ 
\begin{split}
  c\colon \Cliff(V)&\to \End_{\eR}(\wedge V)\\
\tilde\rho\colon \CCliff&\to  \End_{\eR}(\wedge W).
\end{split}  
\]
They obviously comes from the same ideas. One common point is that 
\[ 
  c(e_1)(e_1\wedge e_2)=2\tilde\rho (e_1)(e_1\wedge e_2)=e_2,
\]
but they are different:
\[ 
\begin{split}
  \tilde\rho(e_3)(e_0\wedge e_2)&=0\\
c(e_3)(e_0\wedge e_2)&=e_3\wedge e_0\wedge e_1.
\end{split}  
\]

\subsection{General two dimensional Clifford algebra}
%---------------------------------------------------

The Clifford algebra for the metric
\[ 
  g=\begin{pmatrix}
\alpha&\delta\\\delta&\beta
\end{pmatrix}
\]
is realised by matrices
\[ 
  \gamma_1=
\epsilon\begin{pmatrix}
\sqrt{\alpha}\\ & -\sqrt{\alpha}
\end{pmatrix},\quad
\gamma_2=
\epsilon\begin{pmatrix}
\delta/\sqrt{\alpha}& \beta-\delta^2/| \alpha |\\
1		& -\delta/\sqrt{\alpha}	
\end{pmatrix}
\]
where $\epsilon=\pm 1$ is chosen in such a way that $\epsilon| \alpha |=\alpha$. 
 
\section{Spin group}
%+++++++++++++++++++

We will not immediately go on with Dirac operators on Riemannian manifolds because we still have to build some theory about the Clifford algebra itself. In particular, we have to define the spin group which will play a central role in the definition of the Dirac operator. Almost all --and (too ?) much more-- the concepts we will introduce in this section can be found in \cite{Chevalley}; a more physical oriented but useful approach can be found in \cite{Preparation}.

Let define the map $\dpt{\chi}{\Gamma(p,q)}{GL(\eR^{1,3})}$ by 
\begin{equation}
                \chi(x)y=\alpha(x)\cdot y\cdot x^{-1}.
\end{equation}
\nomenclature{$\chi$}{A representation of $\Gamma(p,q)$}\nomenclature[G]{$\Gamma(p,q)$}{Clifford group}
Let
\[
 \Gamma(p,q)=\{x\in\Cliff(p,q)\tq\textrm{$x$ is invertible and }  \chi(x)y  \in V\textrm{ for all $y\in V$}\}.
\]
It should be remarked that this definition comes back to the real Clifford algebra. The Clifford algebra product gives this subset a group structure which is called the \defe{Clifford group}{Clifford!group}. Any $x\in V$ is invertible since $x\cdot x=-\eta(x,x)1$, the inverse of $x$ is given by $x^{-1}=x/\|x\|^2$.

\index{Clifford!algebra!grading of}
The subset $\Cliff(p,q)^+$ (resp. $\Cliff(p,q)^-$) of $\Cliff(p,q)$ is the image of even (resp. odd) tensors of $T(V)$ by the canonical projection $T(V)\to\Cliff(p,q)$. With these definitions, we have a natural grading of $\Cliff$:
\begin{equation}
 \label{directC}\Cliff(p,q)=\Cliff(p,q)^+\oplus\Cliff(p,q)^-,
 \end{equation}
and the subgroups
\begin{align}
\label{defgplus}
\Gamma(p,q)^+&=\Gamma(p,q)\cap\Cliff(p,q)^+,&\Gamma(p,q)^-&=\Gamma(p,q)\cap\Cliff(p,q)^-.
\end{align}\nomenclature[G]{$\Cliff(p,q)^{\pm}$}{Grading of Clifford algebra}

For $x_1,\ldots,x_n\in V$, we have $\tau(x_1\cdots x_n)=x_n\cdots x_1$. \nomenclature[G]{$\Spin(p,q)$}{Spin group of $\eR^{1,3}$} The \defe{spin group}{spin!group!on $\protect\eR^{1,3}$} is
\begin{equation}   \label{defSpinun}
 \Spin(p,q)=\{x\in\Gamma(p,q)^+\vert\tau(x)=x^{-1}\}
\end{equation}
while the \defe{spin norm}{spin!norm}\nomenclature{$\dpt{N}{\Gamma(p,q)}{\Gamma(p,q)}$}{Spin norm} is the map $\dpt{N}{\Gamma(p,q)}{\Gamma(p,q)}$ defined by
\[
 N(x)=x\tau(\alpha(x)).
\]
We will see in proposition \ref{proppourN} that $N$  actually takes its values in $\eR$ and is therefore a homomorphism $N\colon \Gamma(p,q)\to \eR$

\begin{remark}
The elements of $\Spin(p,q)$ are spin-normed at $1$. Indeed, take a $s$ in $\Spin(p,q)$. We have $N(s)=s\cdot \tau(s)=1$ because $\alpha(s)=s$ and $\tau(s)=s^{-1}$. In particular $\Spin(p,q)\cap\eR=\eZ_{2}$.
\label{rem:spin_norm_u}
\end{remark}

\subsection{Studying the group structure}
%--------------------------------

\begin{proposition}
The set $\Gamma(p,q)$ admits a Lie group structure.
\end{proposition}
\begin{proof}

During this proof, $\mu$ denotes the Clifford multiplication: $\mu(x,y)=x\cdot y$. We know that $\Cliff^{\eC}(p,q)$ is isomorphic to $\End(\Lambda W)$ in which the multiplication is a continuous map. Thus $\mu$ is continuous on $\Cliff^{\eC}(p,q)$. But $\Cliff(p,q)$ is a closed subset of $\Cliff^{\eC}(p,q)$, so $\mu$ is a continuous map in $\Cliff(p,q)$. This proves that  $\chi$ seen as a map from $\Gamma(p,q)\times V$ to $V$ is a continuous map.

The space $V$ is closed in $\Cliff(p,q)$, thus $\sigma^{-1}(V)$ is also closed. But $\sigma^{-1}(V)=\Gamma(p,q)\times\Cliff(p,q)$. So $\Gamma(p,q)$ is closed in $\Cliff(p,q)$.

Now the result is just a consequence of theorems \ref{Helgason2.3} and \ref{Helgason4.2}. Indeed, let us study the subset $\mI$ which appears in the definitions of the Clifford algebra. It makes no difficult to convince ourself that it is a closed subgroup of $T(V)$. The theorem \ref{Helgason4.2} thus makes $\Cliff(p,q)=T(V)/\mI$ a Lie group. But we just say that $\Gamma(p,q)$ is closed in $\Cliff(p,q)$, and the fact that $\Gamma(p,q)$ is a subgroup of $\Cliff(p,q)$ is clear. By theorem \ref{Helgason2.3} we conclude that there exists a Lie group structure on $\Gamma(p,q)$.
\end{proof}

\begin{lemma}
The map $\chi$ is a homomorphism, in other words $\chi$ is a representation of $\Gamma(p,q)$.
\index{representation!of $\Gamma(p,q)$}
\end{lemma}

\begin{proof}
The following computation uses the fact that $\alpha$ is a homomorphism:
\[
\begin{split}
\chi(a\cdot b)y&=\alpha(a\cdot b)\cdot y\cdot (a\cdot b)^{-1}
               =\alpha(a)\cdot\alpha(b)y\cdot b^{-1}\cdot a^{-1}\\
               &=\alpha(a)\cdot\chi(b)y\cdot a^{-1}
               =\chi(a)\chi(b)y.
\end{split}
\]
\end{proof}
Let $y\in\Gamma(p,q)^-$ and $v\in V$. Where is $y\cdot v$ ? First note that $(y\cdot v)^{-1}=v^{-1}\cdot y^{-1}$, so that
\begin{equation}
\begin{split}
  \alpha(y\cdot v)\cdot w\cdot(y\cdot v)^{-1}&=-\alpha(y)\cdot v\cdot w\cdot v^{-1}\cdot y^{-1}\\
                                            &=-\alpha(y)\big( 2\eta(v,w)-w\cdot v \big)\cdot v^{-1}\cdot y^{-1}\\
					    &=-2\eta(v,w)\alpha(y)\cdot v^{-1}\cdot y+\alpha(y)\cdot w\cdot y^{-1}
\end{split}
\end{equation}
which belongs to $V$ because $y\in\Gamma(p,q)$. This reasoning shows that (apart for $0$), $y\cdot v\in\Gamma(p,q)^+$ if and only if $y\in\Gamma(p,q)^-$.

\begin{lemma}
If $x\in V$ is non-isotropic (i.e. $\eta(x,x)\neq 0$), the automorphism $\chi(x)$  is the orthogonal symmetry with respect to $x^{\perp}$.
\end{lemma}

We recall that\nomenclature{$x^{\perp}$}{Space orthogonal to $x$}
\[ 
  x^{\perp}=\{ y\in V\tq\eta(x,y)=0  \}.
\]
We will denote by $\sigma^x$ the orthogonal symmetry with respect to $x^{\perp}$.

\begin{proof}
When the operator $\sigma^x$ acts on $y$, it just change the sign of the ``$x$-part''\ of $y$. So we can write $\sigma^x y=y-2\eta(x,y) 1_x$, where $1_x:=x/\|x\|$. It should be checked if
$\chi(x)y=\alpha(x)\cdot y\cdot x^{-1}$ is equal to $y-2\eta(x,y) 1_x$ or not. We know that $x\cdot x=\eta(x,x)1=-\|x\|$. It follows that
\[
  x\cdot y+y\cdot x=2\eta(x,y)\frac{x\cdot x}{\|x\|}.
  \]
If we multiply this at right by $x^{-1}$, using the fact that $\alpha(x)=-x$, we find
\[
-\alpha(x)\cdot y\cdot x^{-1}=-y+2\eta(x,y) 1_x,
\]
which is precisely the identity we wanted to check.
\end{proof}

The following result will help us to identify subgroups of Clifford group as isometry groups.
\begin{theorem}[Cartan-Dieudonné theorem]
\index{Cartan-Dieudonné theorem}\index{theorem!Cartan-Dieudonné}
Each $\sigma$ in $O(1,3)$ can be written as
\hbox{$\sigma=\tau_1\circ\ldots\circ\tau_m$}, where the $\tau$'s are orthogonal symmetries with respect to hyperplanes which are orthogonal to non-isotropic vectors.
\label{CartanDieu}
\end{theorem}

\begin{proposition}
\[
              \chi(\Gamma(p,q))=O(p,q).
\]
\label{prop1001t1}
\end{proposition}

\begin{proof}
In order to show that $\chi(\Gamma(p,q))\subset O(p,q)$ take $z\in V$ and $x\in\Gamma(p,q)$. Since $\alpha(x)\cdot z\cdot x^{-1}$ lies in $V$, we can write:
\[
\alpha(x)\cdot z\cdot x^{-1}=-\alpha\left(\alpha(x)\cdot z\cdot x^{-1}\right)
=-x\cdot\alpha(z)\cdot\alpha(x^{-1})=x\cdot z\cdot\alpha(x^{-1}).
\]
In order to see that $\chi(x)\in O(p,q)$, we have to prove that $\left\|\chi(x)y\right\|_{(p,q)}^2=\|y\|_{(p,q)}^2$. This is achieved by the following computation:
\begin{equation}
\begin{split}
 \left\|\chi(x)y\right\|_{(p,q)}^2&=-\left(\alpha(x)\cdot y\cdot x^{-1}\right)^2
                                  =\left(\alpha(x)\cdot y\cdot x^{-1}\right)\left(x\cdot y\cdot\alpha(x^{-1})\right)\\
                                  &=-\alpha(x)\cdot y^2\cdot\alpha(x^{-1})
                                  =\|y\|^2_{(p,q)}.
\end{split}
\end{equation}
The last step is simply the fact that $y^2\in\eR$ and therefore commutes with anything. We now know that $\chi(x)\in O(p,q)$ for all $x\in\Gamma(p,q)$. Thus $\chi(\Gamma(p,q))\subset O(p,q)$.

For the second part, let $\sigma$ be in $O(p,q)$. The Cartan-Dieudonné theorem\index{Cartan-Dieudonné theorem}\index{theorem!Cartan-Dieudonné}(theorem \ref{CartanDieu}) says that $\sigma=\sigma^{x_1}\circ\ldots\circ\sigma^{x_r}$ for some $x_1,\ldots, x_r$ in $V$. Thus $\sigma=\chi(x_1\cdots x_r)$, \hbox{and $O(p,q)\subset\chi(\Gamma(p,q))$}.
\end{proof}

\begin{proposition}
\begin{equation}
      \ker\chi=\eR\invtible
\end{equation}
where the right hand side is the set of invertible elements of $\eR$.
\label{prop1001p1}
\end{proposition}
\nomenclature{$\cA\invtible$}{The set of invertible elements of the algebra $\cA$; for example $\eR\invtible=\eR\setminus\{ 0 \}$}

\begin{proof}
Before beginning the proof, we want to insist on the fact that $x\in \ker\chi$ does not mean that $\chi(x)y=0$ for all $y$ in $V$. The ``zero''\ of an algebra is the element $e$ which satisfies $e\cdot y=y\cdot e=y$ for all $y$ in the algebra. In other words, $x$ is in the kernel of $\chi$ if and only if $\chi(x)=\id$.

First we show that $\eR_0\subset \ker\chi$. If $x\in \eR$, then $\alpha(x)=x$. Therefore, when $x\neq 0$,
\[
\chi(x)y=\alpha(x)\cdot y\cdot x^{-1}=y,
\]
because the algebra product $\cdot$ between an element of $\Cliff(p,q)$ and a real is commutative. Note that this does not work with $x=0$.

We are now going to show that $\ker\chi\subset\eR$. Let $z\in\ker\chi$. We decompose (definitions \eqref{defgplus}) it into his odd and even part: $z=z^++z^-$, with $z^{\pm}\in\Gamma(p,q)^{\pm}$. These two can be written as $z^+=e_{j_1}\cdots e_{j_{2r}}$ and $z^-=e_{i_1}\cdots e_{i_{2r-1}}$ with no two $i_k$ or $j_k$ equals. This is almost the general form of elements in even and odd part of $\Gamma(p,q)$: the only other possibility is $z$ in $\eR$. Obviously $\alpha(z^{\pm})=\pm z^{\pm}$. Multiplying the condition $\chi(z)y=y$ at right by $(z^++z^-)$, we find \[(z^+-z^-)y=y(z^++z^-).\] Thanks to equation \eqref{directC}, we can split this condition into even and odd parts:
\begin{align}
 z^+y&=yz^+,
 &z^-y&=-yz^-.
\end{align}
The first equation with $y=e_{j_1}$ gives $e_{j_1}\cdots e_{j_{2r}}\cdot e_{j_1}=e_{j_1}e_{j_1}\cdots e_{j_{2r}}$. In the left hand side, permute the last $e_{j_1}$ from last to second position. So we find the right hand side, with an extra minus sign. This means that $z^+=0$. In the same way, the second equation gives $z^-=0$. We are left with the last possibility: $z\in\eR$.
\end{proof}

\begin{corollary}
For any $s\in\Gamma(p,q)$, there exists some non-isotropic vectors $x_1,\ldots,x_r$, and $c\in\eR$ such that $s=cx_1\cdots x_r$.
\label{602c1}
\end{corollary}

\begin{proof}
Let us take a $s\in\Gamma(p,q)$; we just saw (theorem \ref{prop1001t1}) that $\chi(s)$ is an element of $O(p,q)$. It can be written $\chi(s)=\sigma_1\circ\ldots\circ\sigma_m$. But we had shown that $\sigma_i=\chi(x_i)$ for any $x_i$ normal to the hyperplane defining $\sigma_i$. We thus have
\[
     \chi(s)=\chi(x_1\cdots x_m),
\] 
where $s$ belongs to $\Gamma(p,q)$ and is therefore invertible. This leads us to write $\id=\chi(s^{-1}\cdot x_1\cdots x_m)$. But the kernel of $\chi$ is $\eR$ (proposition \ref{prop1001p1}); so one can find a $r\in\eR$ such that $s^{-1}\cdot x_1\cdots x_m=r$. The claim follows.
\end{proof}

\begin{lemma}
If $v\in V$,
\begin{equation}
                 \det\chi(v)=-1.
\end{equation}
\end{lemma}

\begin{proof}
We already know that $det\chi(v)=\pm 1$. To check that the right sign is plus, take the following basis of $V$: $\{v,v_i^{\perp}\}$ where $\{v_i^{\perp}\}$ is a basis of $v^{\perp}$. Calculating the action of $\chi(v)$ on this basis, we find:
\begin{equation}
\begin{split}
 \chi(v)v&=-v\cdot v\cdot v^{-1}=-v,\\
 \chi(v)v_i^{\perp}&=-v\cdot v_i^{\perp}\cdot v^{-1}
                   =v_i^{\perp}\cdot v\cdot v^{-1}
                   =v_i^{\perp}.
\end{split}
\end{equation}
In this computation\nomenclature{$\sQ$}{A subgroup of $\sG$}, we used the relation $v\cdot w=-w\cdot v-2\brak{v}{w}$ which is true for all $v$, $w$ in $V$. The action of $\chi(v)$ on this basis is thus to let unchanged three vectors and to change the sign of the fourth. This proves the claim.
\end{proof}

\begin{theorem}
\begin{equation}
                   \chi(\Gamma(p,q)^+)=\SO(p,q).
\end{equation}
\label{2102p1}
\end{theorem}

\begin{proof}
From corollary \ref{602c1}, and definition \ref{defgplus}, an element $s\in\Gamma(p,q)^+$ reads $s=cv_1\cdots v_{2r}$. Thus
\begin{equation}
 \det\chi(s)=\det\chi(v_1\cdots v_{2r})
            =\det\left[\chi(v_1)\ldots\chi(v_{2r})\right].
\end{equation}
 But we know that, for all $v_i$ in $V$, $det\chi(v_i)=-1$. So $\det\chi(s)=1$ and $\chi(\Gamma(p,q)^+)\subseteq \SO(p,q)$. As set, 
\[
  \Gamma(p,q)=\Gamma(p,q)^+\cup\Gamma(p,q)^-,
\]
but the lemma shows that $\det\chi(\Gamma(p,q)^-)=-1$ so, from theorem \ref{prop1001t1}, $\chi(\Gamma(p,q)^+)$ must be the whole $\SO(p,q)$.
\end{proof}

\begin{proposition}
The map $N$ takes values in $\eR$ and the formula
\begin{equation}
             N(x\cdot y)=N(x)N(y),
\end{equation}
holds for all $x$, $y\in\Gamma(p,q)$.
\label{proppourN}
\end{proposition}

\begin{proof}
We write as usual $x\in\Gamma(p,q)$ as $x=cv_1\cdots v_r$. So,
\begin{equation}
 N(x)=cv_1\cdots v_r\tau(\alpha(cv_1\cdots v_r))
     =\me{r}c^2v_1\cdots v_r\cdot v_r\cdots v_1.
\end{equation}
The first equality comes from the fact that $\alpha(cv_1\cdots v_r)=\me{r}cv_1\cdots v_r$. Now $N(x)\in\eR$ because $v_i\cdot v_i=-\brak{v_i}{v_i}\in\eR$ for all $i$. Hence the following hold:
\begin{equation}
\begin{split}
 N(x\cdot y)&=v\cdot y\cdot\tau(\alpha(v\cdot y))\\
            &=v\cdot y\cdot\tau(\alpha(y))\cdot\tau(\alpha(v))\\
            &=v\cdot N(y)\tau(\alpha(v))\\
            &=N(y)N(x).
\end{split}
\end{equation}
This is the claim.
\end{proof}


\begin{theorem}
We have the following isomorphism of groups
\[ 
  \Spin(p,q)=\SO_0(p,q).
\]
provided by the map $\chi$.
\end{theorem}

\begin{probleme}
	This result is wrong because of a double covering issue. The real proposition is the next one. I should try to merge the proofs.
\end{probleme}

\begin{proof}
Let $\{ e_1,\cdots,e_p,f_1,\cdots,f_p \}$ be a basis of $\eR^{p+q}$ where the $e_i$'s are time-like and the $f_j$'s are space-like.
Following the discussion at page \pageref{PgDisGeoConnSO}, we have
\[ 
  \SO(p,q)=\SO_0(p,q)\cup\xi \SO_0(p,q)
\]
where $\xi$ is defined as follows: $\xi e_1=-e_1$, $\xi f_1=-f_1$ and $\xi e_k=e_k$, $\xi f_k=f_k$ for $k\neq 1$. This element can be implemented as $\xi=\chi(g)$ for $g=e_1f_1$. It is easy to see that $g^{-1}=-f_1e_1$ and that $\tau(g)=f_1e_1$, so that $g\notin\Spin(p,q)$.

Is it possible to find another $h\in\Gamma(p,q)$ such that $\chi(h)=\xi$ ? If $\chi(a)=\chi(b)$ for $a$, $b\in\Gamma(p,q)$, then $a=rb$ for a certain $r\in\eR$. So we find that $h=g^{-1}/r$ is the general form of an element in $\Gamma(p,q)$ such that $\chi(h)=\xi$. This is an element of $\Spin(p,q)$ if and only if $\tau(h)=h^{-1}$, or $-e_1f_1/r=re_1f_1$ which has no solutions. We conclude that no element of $\Spin(p,q)$ is send on $\xi$ by $\chi$. So
\[ 
  \chi\big( \Spin(p,q) \big)\subset SO_0(p,q).
\]

\begin{probleme}
	I still have to prove the surjectivity of $\chi$ from $\Spin(p,q)$ to $\SO(p,q)$.
\end{probleme}

\end{proof}
\begin{theorem}
\begin{equation}	\label{EqchiSpinSO}	
             \chi(\Spin(p,q))=\SO_0(p,q)
\end{equation} 
where the index $0$ means the identity component.
\end{theorem}

\begin{proof}
Proposition \ref{prop1001p1}, theorem \ref{2102p1} and remark \ref{rem:spin_norm_u} show that the map $\dpt{\chi}{\Spin(p,q)}{\SO(p,q)}$ is a homomorphism with $\mathbb{Z}_2$ as kernel. We begin to prove that $\dpt{\chi}{\Spin(p,q)}{\SO_0(p,q)}$ is surjective. On the one hand, elements of $\Spin(p,q)$ satisfy one more condition than the ones of $\Gamma(p,q)^+$. Thus the algebra $\Spin(p,q)$ has codimension one in $\Gamma(p,q)^+$.

On the other hand, we know that $\SO(p,q)=\SO_0(p,q)\cup h\SO_0(p,q)$ where $h$ is the matrix such that $he_i=-e_i$ for $i=0,\ldots,3$. Since $\Spin(p,q)$ has codimension one in $\Gamma(p,q)^+$, there is at most one more generator in $\chi(\Gamma(p,q)^+)$ than in $\chi(\Spin(p,q))$ (because $\chi$ is a homomorphism). In order to prove this theorem, we just need to show that elements of $\chi(\Gamma(p,q)^+)$ which do not belong to $\chi(\Spin(p,q))$ is $h$.

Is is no difficult to see that $\chi(e_0\cdot e_1\cdot e_2\cdot e_3)e_i=-e_i$ for $i=0\ldots 3$: just write
$\chi(e_0\cdot e_1\cdot e_2\cdot e_3)e_i=e_0\cdot e_1\cdot e_2\cdot e_3\cdot e_i\cdot e_3^{-1}\cdot e_2^{-1}\cdot e_1^{-1}\cdot e_0^{-1}$ and use the commutation relations. An easy computation gives
$N(e_0\cdot e_1\cdot e_2\cdot e_3)=-1$; then this is not in $\Spin(p,q)$ by remark \ref{rem:spin_norm_u}.
\end{proof}

We write it by the exact sequence
\begin{equation}
 \xymatrix{
    \eZ_2  \ar@{^{(}->}[r] & \Sppq \ar[r]^{\chi} & \SO_0(p,q)
  } 
\end{equation}
we say that the group $\Spin(p,q)$ is a \defe{double covering}{double covering!of $\SO_{0}(p,q)$} of $\SO_0(p,q)$.

\begin{lemma}
If $\dpt{\pi}{\tX}{X}$ is a covering which satisfies

\begin{enumerate}
\item $X$ is path connected,
\item $\forall x\in X$, $\tX_x:=\pi^{-1}(x)$ is path connected in $\tX$ \emph{i.e.} for all $a$, $b\in \tX$, there exist a path in $\tX$ which joins $a$ and $b$,
\end{enumerate}
then $\tX$ is path connected.
\label{lem_cov_path_con}
\end{lemma}
\begin{proof}
If $\tx$ and $\ty$ are in $\tX$, we can suppose that $\pi(\tx)\neq\pi(\ty)$ (because if $\pi(\tx)=\pi(\ty)$, the second assumption gives the thesis). We define $x$ and $y$ as their projections: $x=\pi(\tx)$ and $y=\pi(\ty)$. Let $\gamma$ be a path such that $\gamma(0)=x$ and $\gamma(1)=y$, and $\tgamma$ be the lift of $\gamma$ in $\tX$ which contains $\tx$: $\tgamma(0)=\tx$ and $\pi(\tgamma(1))=\gamma(1)=y$. Then $\tgamma(1)$ lies in $\tX_y$. Therefore, we can consider $\gamma'$ which joins $\tgamma(1)$ and $\ty$.

So, $\gamma'\circ\tgamma$ is a path which contains $\tx$ and $\ty$.
\end{proof}


\begin{proposition}
 The group $\Spin(p,q)$ is connected.
\end{proposition}

\begin{proof}
We will prove that the covering $\dpt{\chi}{\Spin(p,q)}{\SO_0(p,q)}$ fulfils lemma \ref{lem_cov_path_con}. We just have to show that $\Spin(p,q)$ fulfills the second assumption of the lemma. First note that $\chi(\tx)=\chi(\ty)$ implies $\chi(\tx\ty^{-1})=e$, and then $\tx=\pm\ty$ because of proposition \ref{prop1001p1}. Since the other case is trivial, we can suppose $\tx=-\ty$.

It remains to prove that for every $g\in\Spin(p,q)$, there is a path in $\Spin(p,q)$ which joins $g$ and $-g$. The answer is given by the path $t\mapsto \gamma(t)g$ where
\[
\gamma(t)=\exp(te_1\cdot e_2)=\cos(t)(-1)+\sin(t)e_1\cdot e_2
\]
which satisfies $\gamma(0)=1$ and $\gamma(\pi)=-1$. 
\end{proof}

\begin{proposition}

The homomorphism $\tilde\rho$ restricts to a homomorphism $\tilde\rho\colon \Spin(p,q)\to \GL(\Lambda^+W)$.
\end{proposition}

\begin{proof}
An element in $\Spin(p,q)$ reads $s=cv_1\cdots v_{2r}$ and its image by $\tilde\rho$ is
\[ 
  \tilde\rho(s)=c\tilde\rho(v_1)\circ \cdots \circ\tilde\rho(v_{2r}).
\]
When one applies $\tilde\rho(v_1)$ to an element $\alpha\in\Lambda^kW$, one obtains a linear combination of an element of $\Lambda^{k-1}W$ and one of $\Lambda^{k+1}W$. The element $\tilde\rho(s)$ being an even composition of such maps, its transforms an element of $\Lambda^+W$ into an element of $\Lambda^+W$. 
\end{proof}

Notice that an element of $V$ ---no $V^{\eC}$--- is represented on $\Lambda^+W$ by complex matrices. This is not a problem. In the case of $\eR^{1,3}$, we have $\dim\Lambda^+W=2$ and thus 
\[ 
  \tilde\rho\big( \Spin(1,3) \big)\subset \GL(2,\eC).
\]
The following is the lemma 8.5 (page 57) of \cite{Michelson}.

\begin{lemma}
Let $\rho\colon \Cl(p,q)\to \Hom_{\eC}(E,E)$ be a representation of the Clifford algebra on a vector space $E$. If $p+q\geq 2$, then for all $s\in\Spin(p-1,q)\subset \Cl(p,q) $,
\[ 
  \det{}_{\eC}\big( \rho(s) \big)=\pm 1.
\]

\end{lemma}
\begin{proof}
No proof.
\end{proof}

\begin{theorem}
The representation $\tilde\rho$ provides a group isomorphism 
\[ 
  \Spin(1,3)\simeq \SL(2,\eC)
\]

\end{theorem}

\begin{proof}
In the case $p=2$, $q=3$, the lemma assures us that for each $s$ in the spin group, $\det\tilde\rho(s)=1$. Since $\Spin(1,3)$ is connected and the determinant function is continuous, we deduce that $\det\tilde\rho(s)\equiv 1$. This proves that $\tilde\rho\big( \Spin(1,3) \big)\subset \SL(2,\eC)$. The proposition \ref{PropUssGpGenere} thus implies that
\[ 
  \tilde\rho\big( \Spin(1,3) \big)=\SL(2,\eC),
\]
 but from $\Cl(1,3)$, the representation $\tilde\rho$ is yet injective. \emph{A forciori}, the representation $\tilde\rho$ is injective from $\Spin(1,3)$. This finishes the proof.
\end{proof}

\subsection{Redefinition of \texorpdfstring{$\Spin(V)$}{Spin(V)}}
%----------------------------------------------------------------

As it, this new definition only holds when $g$ is positive defined.

\begin{probleme}
	When we work with a signature $(p,q)$, maybe we only get the connected part. To be checked.
\end{probleme}

Let us take $v$, $x\in V$ with $g(v,v)=1$. We have 
\[ 
  -vxv^{-1}=-vxv=-2g(x,v)v+xv^2
		=x-2g(x,v)v\in V.
\]
The effect was to reverse the $v$ component of $x$; the map $x\mapsto -vxv^{-1}$ is $\sigma^v$. Now, when $\lambda\in U(1)$ and $w=\lambda v$, we also have that $x\mapsto -wxw^{-1}$ is $\sigma^v$. Now we look at $\chi(a)\colon x\mapsto \alpha(a)xa^{-1}$ with $a=w_{1}\ldots w_{r}$, a product of unitary vectors in $V^{\eC}$. Explicitly,
\[ 
  \chi(a)x=(-1)^{r} w_{1}\ldots w_{r}xw_{r}^{-1}\ldots w_{1}^{-1},
\]
a composition of reflexions in $V$. When $r$ is even, it is a rotation. We conclude that when $a$ is an even product of unitary vectors in $V^{\eC}$, then $\chi(a)\in \SO(V)$. Theorem \ref{CartanDieu} states that any rotation of $V$ is a composition of reflexions. So we define\nomenclature[G]{$\Spin^{c}(V)$}{A group related to $\Spin$}
\begin{equation}
\Spin^{c}(V)=\{ w_{1}\ldots w_{2k}\tq w_{j}\in V^{\eC},\,w_{j}^*w_{j}=1 \}\subset \CCliff^{0}(V),
\end{equation}
and $\chi\colon \Spin^{c}(V)\to \SO(V)$ is a surjective group homomorphism. The inverse in $\Spin^{c}(V)$ is given by
\[ 
  (w_{1}\ldots w_{2k})^{-1}=w_{2k}^*\ldots w_{1}^*=\overline{ w_{2k} }\ldots\overline{ w_{1} }.
\]
In the real case, proposition \ref{prop1001p1} says that $\ker\chi=\eR\invtible$. In the complex case we get  $\ker\chi=\eC\invtible$ and, when we look at $\ker\chi|_{\Spin^{c}(V)}$, we find
\begin{equation}
\ker\chi=U(1).
\end{equation}
Then we find the short exact sequence 
\begin{equation}
\xymatrix{%
   1 \ar[r]^-{\id}&U(1) \ar[r]^-{\id}&\Spin^{c}(V) \ar[r]^-{\chi}&\SO(V)\ar[r]^-{\id}&1.
}
\end{equation}
Let $u=w_{1}\ldots w_{2k}\in\Spin^{c}(V)$ with $w_{j}=\lambda_{j}v_{j}$ and $\lambda_{j}\in V$, so $\tau(u)=w_{2k}\ldots w_{1}$ and
\[ 
  \tau(u)u=w_{2k}\ldots w_{1}w_{1}\ldots w_{2k}
		=\lambda_{1}^{2}\ldots \lambda_{2k}^{2}\in U(1).
\]
This proves that $\tau(u)u$ is central in $\Spin^{c}(V)$. We define the homomorphism
\begin{equation}
\begin{aligned}
\nu \colon \Spin^{c}(V)&\to U(1) \\ 
u&\mapsto \tau(u)u. 
\end{aligned}
\end{equation}
This is a homomorphism because
\[ 
\begin{split}
  \nu(u_{1}u_{2})&=\tau(u_{1}u_{2})u_{1}u_{2}
		=\tau(u_{2})\underbrace{\tau(u_{1})u_{1}}_{\text{central}}u_{2}
		=\tau(u_{2})u_{2}\tau(u_{1})u_{1}\\
		&=\nu(u_{2})\nu(u_{1})
		=\nu(u_{1})\nu(u_{2}).
\end{split}  
\]
The map $\nu$ naturally restricts to $U(1)$ as
\[ 
  \nu(\lambda)=\lambda^{2}.
\]
The combined map $(\chi,\nu)\colon \Spin^{c}(V)\to \SO(V)\times U(1)$ has kernel $\{ \pm 1 \}$. We define\nomenclature[G]{$\Spin(V)$}{The spin group}
\begin{equation}  \label{eq_defSpindeux}
\Spin(V)=\ker\nu|_{\Spin^{c}(V)}.
\end{equation}

\begin{lemma}
This group is the same as the one defined in equation \eqref{defSpinun}. 
\end{lemma}

\begin{proof}
Let $u\in\Spin(V)$ (in the sense of equation \eqref{eq_defSpindeux}). The fact for $u$ to belongs to $\Spin(V)$ implies the two following:
\begin{enumerate}
\item $u\in\Spin^{c}(V)\Rightarrow u^*u=1$,
\item $u\in\ker\nu\Rightarrow \tau(u)u=1$.
\end{enumerate}
The second point says that $u^{-1}=\tau(u)$, which is a first good point to fit the first definition of $\Spin(V)$. Now we have to prove that $u\in\Gamma^{+}(V)$: $u$ must be invertible and $\chi(u)x$ must belongs to $V$ for all $x\in V$. These two points are contained in the definition of $\Spin^{c}(V)$.
\end{proof}
Let us see in the new definition how is $\chi\colon \Spin(V)\to \SO(V)$. On $\Spin^{c}(V)$, we have $\ker\chi=U(1)$, but on $\Spin(V)$ we require moreover $\tau(u)u=1$, thus an element of $\ker\chi$ in $\Spin(V)$ fulfils $\tau(\lambda)\lambda=1$, so that $\lambda=\{ \pm1 \}$. We conclude that $\ker\chi|_{\Spin(V)}=\{ \pm 1 \}$, and then that $\Spin(V)$ is a double covering of $\SO(V)$.\index{double covering!of $\SO(V)$}


\subsection{A few about Lie algebra}
%----------------------------------

\nomenclature[G]{$\spin(p,q)$}{Lie algebra of the group $\Spin(p,q)$}
\begin{proposition}
We have an isomorphism
\[ 
                    \spin(p,q)\simeq\so(p,q)
\]
between the Lie algebras of $\Spin(p,q)$ and $\SO(p,q)$.
\label{prop:spin_so}   
\end{proposition}

\begin{proof}
Using the second part of lemma \ref{Helgason5.1}, with the map $\dpt{\chi}{\Spin(p,q)}{\SO(p,q)}$, we find that $d\chi_e(\spin(p,q))=\so(p,q)$. Then we know (lemma \ref{1203r1}) that 
\[
	\so(p,q)=\spin(p,q)/\ker\,d\chi_e.
\]
On the other hand, the first part of the same lemma gives us that $\chi^{-1}(e)$ is a Lie subgroup of $\Spin(p,q)$ whose Lie algebra is $\ker\,d\chi_e$. But $\chi^{-1}(e)=\eZ_2$, so $\ker\,d\chi_e=\{0\}$.
\end{proof}

Let us now shortly speak about the Lie algebra of $\Gamma(p,q)^+$. A basis of $\Cliff(p,q)^+$ is \[\{1,\gamma_0\cdot\gamma_1,\gamma_0\cdot\gamma_1 ,\gamma_0\cdot\gamma_3
,\gamma_0\cdot\gamma_1\cdot\gamma_2\cdot\gamma_3  \}.\] Thanks to the anticommutation relations, we don't need $\gamma_1\cdot\gamma_2$ in the basis.

Remember that $\Gamma(p,q)^+$ is the set of the $x\in\Cliff^+(p,q)$ such that $x\cdot v\cdot\alpha(x^{-1})$ lies in $V$ for all $v\in V$. Let $x(t)$ be a path in $\Gamma(p,q)^+$ such that $x(0)=e$ and $\dot{x}(0)=X$. Differentiating the definition relation, we find
 \[
 \dot{x}\cdot v\cdot\alpha(x^{-1})|_0+x\cdot v\cdot(-)\alpha(\dot{x})|_0=X\cdot v-v\cdot X,
 \]
 therefore\nomenclature[G]{$\Lie{\Gamma(p,q)^+}$}{Algèbre de $\Gamma(p,q)^+$}
\[
  \Lie{\Gamma(p,q)^+}=\left\{X\in\Cliff^+(p,q)\textrm{ such that } X\cdot v-v\cdot X\in V,\,\forall v\in V\right\}.
\]

It is clear that $\eC$ is a subset of $\Lie{\Gamma(p,q)^+}$, and that $V$ is not. The following computation shows that $V\cdot V$ is a subset $\Lie{\Gamma(p,q)^+}$:
\[
         a\cdot b\cdot v-v\cdot a\cdot b=2\eta(v,a)b-2\eta(v,b)a.
\]
 We can also check that $V\cdot V\cdot V\cdot V\cap\Lie{\Gamma(p,q)^+}=\emptyset$. A basis of $\Lie{\Gamma(p,q)^+}$ is
\[
	\{ 1,e_{\alpha}\cdot e_{\beta}\tq \alpha<\beta \}
\]

 We know that $\ker[\dpt{\chi}{\Gamma(p,q)^+}{\SO(p,q)}]=\eR_0$. So the kernel of the restriction of $d\chi_e$ to $\Lie{\Gamma(p,q)^+}$ is the Lie algebra of $\eR_0$ (see lemma \ref{Helgason5.1}), which is $\eR$. Therefore, a basis of $\spin(p,q)$ is 
\[
	\{e_{\alpha}\cdot e_{\beta}\tq \alpha<\beta\}.
\]

\subsection{Grading \texorpdfstring{$\Lambda W$}{LW}}
%-------------------------------

We already know that $\Lambda W=\eC\oplus W\oplus\Lambda^2W$. This space can be written as \[\Lambda W =\Lambda W^+\oplus\Lambda W^-,\] with $\Lambda W^+=W$ and $\Lambda W^-=\eC\oplus\Lambda^2W$. The interest of such a decomposition lies in the definition of an action of $\Cliff^+(p,q)$ on $\Lambda W $. This action will be defined by $\dpt{\bullet}{\Cliff^+(p,q)\times\Lambda W }{\Lambda W }$, 
  \[
 x\bullet\alpha=\tilde\rho(x)\alpha
 \]
for any $x$ in $\Cliff^+(p,q)$ and any $\alpha$ in $\Lambda W $ (see definition \ref{defrt}).

\begin{proposition}
This action preserves the grading of $\Lambda W $:
\begin{equation}
\begin{split}
 \Cliff^+(p,q)\bullet\Lambda W^+&=\Lambda W^+\\
 \Cliff^+(p,q)\bullet\Lambda W^-&=\Lambda W^-.
\end{split}
\end{equation}

\end{proposition}
\begin{proof}
For $x\in\eC$, theses equalities are obvious. We have to check it for $x=e_i\cdot e_j$. Here, we will just check that $(e_1\cdot e_0)\bullet(v\wedge w)\in\Lambda W^+$. This follows from a simple computation:
\begin{equation}
\begin{split}
\tilde\rho(e_1)\tilde\rho(f_0+g_0)(v\wedge w)&=
                         \tilde\rho(f_1+g_1)\left[-\eta(g_0,v)w+\eta(g_0,w)v\right]\\
                    &=-\eta(g_0,v)f_1\wedge w+\eta(g_0,w)f_1\wedge v\\
                    &\quad+\eta(g_0,v)\eta(g_1,w)-\eta(g_0,w)\eta(g_1,v).
\end{split}
\end{equation}
\end{proof}

Since $\Spin(p,q)$ is a subgroup of $\Cliff^+(p,q)$, we can construct two new representation of $\Spin(p,q)$. These are $\dpt{\rho^{\pm}}{\Spin(p,q)\times\Lambda W ^{\pm}}{\Lambda W ^{\pm}}$,
\begin{equation}
\begin{split}
 \rho^-(s)w^-&=\tilde\rho(s)w^-,\\
 \rho^+(s)w^+&=\tilde\rho(s)w^+,
\end{split}
\end{equation}
for $w^{\pm}$ in $\Lambda W ^{\pm}$. This is no more than the fact that $\tilde\rho$ is reducible and that two invariant subspaces are $\Lambda W^+$ and $\Lambda W^-$.
\subsection{Clifford algebra for \texorpdfstring{$V=\eR^2$}{V=R2}}\label{cliffR2}
%----------------------------------------------

\subsubsection{General definitions}
%/////////////////////////////////

The whole construction can also be applied to $V=\eR^2$ with the Euclidean metric. This is our business now. We take the complex vector space $V^{\eC}$ and an orthonormal basis $\{e_1,e_2\}$. As before, we define
 \[
f_1=\frac{1}{2}(e_1+ie_2),\qquad g_1=\frac{1}{2}(e_1-ie_2).
\]
There are no difficulties to see that $Span(f_1)$ is a completely isotropic subspace\index{isotropic!subspace!in $\eR^{2}$} of $V^{\eC}$. Thus we define $W=\eC f_1$, $\Lambda W =\eC\oplus W$, $\Lambda W^+=\eC$, and $\Lambda W^-=W$\nomenclature{$\Lambda W^{\pm}$}{Spinor space}. The homomorphism $\dpt{\tilde\rho}{V^{\eC}}{\End(\Lambda W )}$\nomenclature{$\dpt{\tilde\rho}{(\eR^2)^{\eC}}{\End(\Lambda W )}$}{Spinor representation} in $\Lambda W $ is defined by
\begin{equation}
\begin{split}
 \tilde\rho(f_1)\alpha&=f_1\wedge\alpha,\\
 \tilde\rho(g_1)\alpha&=-i(g_1)\alpha,
\end{split}
\end{equation}
where $\alpha$ is any element of $\Lambda W $. In the basis $1=\begin{pmatrix}
1 \\
0
\end{pmatrix} $ and $f_1=\begin{pmatrix}
0 \\
1
\end{pmatrix} $, we easily find that
\[
 \tilde\rho(e_1)=\begin{pmatrix}
 0 & -\frac{1}{2} \\
 1 & 0
 \end{pmatrix},\quad\tilde\rho(e_2)=\begin{pmatrix}
 0 & -\frac{i}{2} \\
 -i & 0
 \end{pmatrix}.\]
For $c\in\eR$ we	 also have $\tilde\rho(c)f_1=cf_1$ and $\tilde\rho(c)1=c$, thus we assign the matrix $\begin{pmatrix}
c & 0 \\
0 & c
\end{pmatrix}$ to the number $c$.

As before, we define $\gamma_i=\sqrt{2}\tilde\rho(e_i)$. We immediately have $\gamma_1\gamma_2+\gamma_2\gamma_1=0$ and $\gamma_i\gamma_i=-2\mtu$, so that the $\gamma$'s satisfy the Clifford algebra for the euclidian metric.

For notational conveniences, it proves useful to make a change of basis so that we get
\begin{equation}\label{gammaR2}
\gamma_1=\begin{pmatrix}
0 & -1 \\
1 & 0
\end{pmatrix},\quad\gamma_2=-\begin{pmatrix}
0 & i \\
i & 0
\end{pmatrix}.
\end{equation}

The algebra $\Cliff(2)$\nomenclature[G]{$\Cliff(2)$}{Clifford algebra of $\eR^2$} is isomorphic to the algebra which is generated by direct sum $\Cliff(2)\simeq\eR\oplus\gamma_1\oplus\gamma_2\oplus\eR\gamma_1\gamma_2$. A general element of $\Cliff(2)$ can be written as $x\gamma_1+y\gamma_2+x'\eR+y'\gamma_1\gamma_2$. In the representation of $\tilde\rho$, a general element of $\Cliff(2)$ is therefore
\[\begin{pmatrix}
x'+iy' & x+iy \\
-x+iy & x'-iy'
\end{pmatrix},\] so that we can write the Clifford algebra of $\eR^2$ as\index{algebra!Clifford}
\[
\Cliff(2)=\left\{\begin{pmatrix}
 \alpha & \beta \\
 -\obeta & \oalpha
 \end{pmatrix}\,:\,\alpha,\beta\in\eC\right\}.
\]
The following four matrices provide a basis:
\begin{align}\label{pauli} 
1&=\begin{pmatrix}
1 & 0 \\
0 & 1
\end{pmatrix}, &i&=\begin{pmatrix}
-i & 0 \\
0 & i
\end{pmatrix},&j&=\begin{pmatrix}
0 & i \\
i & 0
\end{pmatrix},&k&=\begin{pmatrix}
0 & 1 \\
-1 & 0
\end{pmatrix}.
\end{align}
We can check that these matrices satisfies the quaternionic algebra\index{quaternion!algebra}\index{algebra!quaternion} :
\begin{equation}
\begin{split}
i^2&=j^2=k^2=-1\\
ij &=-ji=k,\\
jk &=-kj=i,\\
ki &=-ik=j.
\end{split}
\end{equation}
The algebra $\Cliff(2)=\eH$\nomenclature{$\eH$}{quaternionic algebra} is represented by $\tilde\rho$ on $\eC^2$ by the \defe{Pauli matrices}{pauli matrices} $1,i,j,k$ which are given by \eqref{pauli}.

\subsubsection{The maps \texorpdfstring{$\alpha$}{a} and \texorpdfstring{$\tau$}{t}}
%///////////////////////////////////////////////

What are the matrices which represent $V$ ? These are $\tilde\rho(e_1)$ and $\tilde\rho(e_2)$. Thus we can write $V=\Span_{\eR}\{\gamma_1,\gamma_2\}=\Span_{\eR}\{j,k\}$, or
\[
 V=\left\{\begin{pmatrix}
 0 & \xi \\
 -\oxi & 0
 \end{pmatrix}\,:\,\xi\in\eC\right\}.
\]

As before, $\alpha$ is the unique homomorphic extension to $\Cliff(2)$ of $-\id$ on $V$. From the definitions, we get $\alpha(j)=-j$, $\alpha(k)=-k$.
The extension present no difficult. For example: $\alpha(i)=\alpha(jk)=\alpha(j)\alpha(k)=jk=i$, but $\alpha(jk)=\alpha(i)$; then $\alpha(i)=i$. The same gives $\alpha(1)=1$.

The case of $\tau$ is treated in similar way. We find: $\tau(j)=j$, $\tau(k)=k$, $\tau(i)=-i$, $\tau(1)=1$.

Now, we can find the group $\gud$. The condition for $x\in\Cliff(2)$ to be in $\gud$ is $\alpha(x)yx^{-1}$ to belongs to $V$ for all $y\in V$. We put
\[ x=\begin{pmatrix}
\alpha & \beta \\
-\obeta & \oalpha
\end{pmatrix},\qquad\alpha(x)=\begin{pmatrix}
\alpha & -\beta \\
\obeta & \oalpha
\end{pmatrix}.\]
A typical $y$ in $V$ is
\[
 y=\begin{pmatrix}
 0 & \eta \\
 -\oeta & 0
 \end{pmatrix}.
\]
A few computation gives:
\[
 \alpha(x)yx^{-1}=\us{|\alpha|^2+|\beta|^2}\begin{pmatrix}
 \alpha\eta\obeta+\beta\oeta\oalpha & \alpha\alpha\eta-\beta\beta\oeta \\
 \obeta\obeta\eta-\oalpha\oalpha\eta & \eta\alpha\obeta+\oalpha\oeta\beta
 \end{pmatrix}.
\]
If we impose it to be of the form $\begin{pmatrix}
0 & \xi \\
-\oxi & 0
\end{pmatrix} $ for all $\eta\in\eC$, we get, for all $\eta\in\eC$, 
 $\Reel(\oalpha\beta\oeta)=0$, which implies $\oalpha\beta=0$. So we conclude:
\[
 \gud=\left\{\begin{pmatrix}
 \alpha & 0 \\
 0 & \oalpha
 \end{pmatrix}, \begin{pmatrix}
 0 & \beta \\
 -\obeta & 0
 \end{pmatrix}\,:\,\alpha,\beta\in\eC\textrm{ not both equals zero}\right\}.
\]
Be careful on a point: $\gud$ is the \emph{multiplicative} group generated by these two matrices, not the additive one.

\subsubsection{The spin group}
%////////////////////////////

It present no difficult to find that
\begin{equation}
 \gud^+=\left\{\begin{pmatrix}
 \alpha & 0 \\
 0 & \oalpha
 \end{pmatrix}\,:\,\alpha\neq 0\right\}.
\end{equation}
The \defe{spin group}{spin!group!on $\protect\eR^2$} is made of elements of $\gud^+$ which satisfy $\tau(x)=x^{-1}$. We know that
$\tau\begin{pmatrix}
\alpha & 0 \\
0 & \oalpha
\end{pmatrix} =\begin{pmatrix}
\oalpha & 0 \\
0 & \alpha
\end{pmatrix}$ and that $\begin{pmatrix}
\alpha & 0 \\
0 & \oalpha
\end{pmatrix}^{-1} =\us{\displaystyle\alpha\oalpha}\begin{pmatrix}
\oalpha & 0 \\
0 & \alpha
\end{pmatrix}$. Thus the condition \hbox{$\tau(x)=x^{-1}$} becomes $|\alpha|^2=1$. The first conclusion is that
\begin{equation}
                    \Spin(2)=U(1).
\end{equation}
A typical $s$ in $\Spin(2)$ is
\[s=e^{i\theta}=\begin{pmatrix}
e^{i\theta} & 0 \\
0 & e^{-i\theta}
\end{pmatrix}.\]

The next point is to see the action of $\Spin(2)$ on $V$.\index{action!of $\Spin(2)$ on $\eR^2$} The action of $s\in\Spin(2)$ on a vector $v\in V$ is still defined by $s\bullet v=\chi(s)v=\alpha(s)\cdot v\cdot s^{-1}$. More explicitly:
\begin{equation}
 \chi(s)v=\begin{pmatrix}
 e^{i\theta} & 0 \\
 0 & e^{-i\theta}
 \end{pmatrix} \begin{pmatrix}
 0 & z \\
 -\overline{z} & 0
 \end{pmatrix} \begin{pmatrix}
 e^{-i\theta} & 0 \\
 0 & e^{i\theta}
 \end{pmatrix}=\begin{pmatrix}
 0 & e^{2i\theta}z  \\
 -e^{-2i\theta}\overline{z} & 0
 \end{pmatrix},
\end{equation}
where the  matrix $\begin{pmatrix}
0 & z \\
\overline{z} & 0
\end{pmatrix} $ denotes the representation of the vector $v$ of $V$. This equality can be written $e^{i\theta}\cdot v=e^{2i\theta}v$. If we note $v=v_1+iv_2=\begin{pmatrix}
v_1 \\
v_2
\end{pmatrix} $, we get
\[ e^{2i\theta}\bullet v=\begin{pmatrix}
\cos 2\theta & -\sin 2\theta \\
\sin 2\theta & \cos 2\theta
\end{pmatrix}\begin{pmatrix}
v_1 \\
v_2
\end{pmatrix}. \]
Therefore, we can write
\[\chi(e^{i\theta})=\begin{pmatrix}
\cos 2\theta & -\sin 2\theta \\
\sin 2\theta & \cos 2\theta
\end{pmatrix}.\]

So $\chi$ projects $U(1)$ into $\SO(2)$ with a kernel $\eZ_2$, for this reason, we say that $U(1)$ is a \defe{double covering}{double covering!of $\SO(2)$} of $\SO(2)$. We note it
\begin{equation}
            \eZ_2\rightarrow U(1)\stackrel{\chi}{\rightarrow}\SO(2).
\end{equation}

\section{Clifford modules}  \label{susec_Cliffmodule}\index{Clifford!module}
%---------------------------

References: \cite{ResEtaDiracType,mellor}.

Let $M$ be a manifold. We denote by $\CCliff(M)$ the bundle whose fibre at $x\in M$ is the complex Clifford algebra of the metric $g_x$ : $\CCliff(M)_x=\CCliff(g_x)$. We define the important map
\begin{equation}
\begin{aligned}
 \gamma\colon \Gamma(M,\CCliff(M))&\to \oB(\hH) \\ 
\gamma(dx^{\mu})&\mapsto \gamma^{\mu}(x)  
\end{aligned}
\end{equation}
which can be extended to the whole Clifford algebra.

Let $V$ be a vector space endowed with a bilinear symmetric form. We consider $\Cliff(V)$, the corresponding Clifford algebra. A \defe{Clifford module}{Clifford!module} is a real vector space $E$ with a $\eZ_2$-graduation and a morphism 
\[ 
  \rho_E\colon \Cliff(V)\to \End(E)
\]
of $\eZ_2$-graded vector spaces. It is defined by a linear map $\rho_E\colon V\to \End(V)$ such that
\begin{equation}
\rho_E(v)\rho_E(w)+\rho_E(w)\rho_E(v)=B(v,w)\id
\end{equation}
for every $v$, $w\in E$. The element $\rho_E(x)v$ will often be denoted by $x\cdot v$ and the operation $\rho_E$ is the \defe{Clifford multiplication}{Clifford!multiplication}. The \defe{dual module}{dual module} $E^*$ is defined by $\rho_{E^*}(x)=\rho_E(x^t)^*$, i.e.
\begin{equation}
\langle \rho_{E^*}(x)\psi,v \rangle =(-1)^{| \psi | |x |}\langle \psi, \rho_E\big( \tau(x) \big)v\rangle 
\end{equation}
for every $\psi\in E^*$ and $v\in E$. Here

Let $\cA$ be a $\eZ_2$-graded subalgebra of $\Cliff(V)$ and $E_1$, a $\cA$-module. Then the space\nomenclature{$\Ind_{\cA}^{\Cliff(V)}(E_1)$}{Induced Clifford module}
\[ 
  E=\Ind_{\cA}^{\Cliff(V)}(E_1)=\Cliff(V)\otimes_{\cA}E_1
\]
has a structure of Clifford module, the \defe{induced module}{induced!Clifford module}. The tensor product $\otimes_{\cA}$ is the usual one modulo the subspace spanned by elements of the form 
\[ 
  x\otimes a\cdot y-xa\otimes y
\]
for every $x$, $a\in\Cliff(V)$ and $y\in E_1$. In a similar way, if $E$ is a complex vector space we have a notion of $\CCliff(V)$-module. 

Let $x\in\Cliff(V)$ be such that $x^2=1$. In that case the Clifford multiplication $\rho_E(x)$ decomposes $E$ in eigenspaces
\[ 
  E^{\pm}=\frac{ 1 }{2}\big( 1\pm\rho_E(x) \big)E.
\]

If $V$ is a $n$-dimensional vector space with an oriented orthonormal basis $\{ e_1,\ldots, e_n \}$, the algebra $\Cliff(V)$ has a \defe{volume element}{volume!element} $\omega=e_1e_2\ldots e_n$ which does not depend on the choice of the basis. The volume element squares to
\begin{equation}
\omega^2=(-1)^{n(n+1)/2}.
\end{equation}
In the complex case, we consider the complex vector space $V^{\eC}$ and the complex Clifford algebra $\CCliff(V)=\Cliff(V)\otimes_{\eR}\eC$, and the volume element is defined as
\begin{equation}
\omega_{\eC}=i^{[ (n+1)/2 ]}\omega.
\end{equation}
where $[x]$ is denotes the integer part of $x$. Performing a separate computation for $n$ even or odd, it is easy to see that in both case,
\begin{equation}
\omega_{\eC}^2=1.
\end{equation}
So in the complex case we always have an element in $\Cliff(V)$ which squares to $1$, and a $\CCliff(V)$-module $W$ always accepts a decomposition as $W^{\pm}=\frac{ 1 }{2}(1+\omega_{\eC})W$.

One says that a representation\index{representation!of Clifford algebra} $\rho$ of $\Cliff(V)$ on $W$ is \defe{reducible}{reducible!representation of Clifford} if there exists a splitting $W=W_1\oplus W_2$ such that $\rho(\Cliff(V))W_i\subset W_i$. If the representation is not reducible, it is said to be irreducible. Two representations $\rho_j\colon \Cliff(V)\to \End(W_j)$ are \defe{equivalent}{equivalence!representation of Clifford} if there exists a linear isomorphism $F\colon W_1\to W_2$ such that $F\circ\rho_1(x)\circ F^{-1}=\rho_2(x)$ for every $x\in\Cliff(V)$.

\begin{proposition}
The real Clifford algebra has
\[ 
 \begin{cases}
2&\text{if $n+1=0\mod 4$}\\
1&\text{otherwise}
\end{cases} 
\]
inequivalent irreducible representations. The complex Clifford algebra $\CCliff(V)$ has
\[ 
 \begin{cases}
2&\text{if $n$ is odd}\\
1&\text{if $n$ is even}
\end{cases} 
\]
inequivalent irreducible representations. 
\end{proposition}
\begin{proof}
No proof.
\end{proof}


If $M$ is a manifold, we denote by $\Cliff(M)=\Cliff(TM)$ the bundle whose fiber at $x$ is the Clifford algebras of $T_xM$. We consider an orthonormal basis $\{ e_i \}$ and if $\Sigma$ is a multi-index $\{ 1\leq\sigma_1,\ldots,\leq\sigma_t\leq m \}$, we pose $e_{\Sigma}=e_{\sigma_1}\ldots e_{\sigma_t}\in\Cliff(M)$. By convention, $e_{\emptyset}=1$. Since the elements $e_i$ are ordered, they provide an orientation:
\begin{equation}
d\vol=e_1\wedge\ldots\wedge e_m\in\Wedge^m(M).
\end{equation}
Since the map $e_{\sigma_1}\wedge\ldots\wedge e_{\sigma_t}\mapsto e_{\sigma_1\ldots e_{\sigma_t}}$ is an isomorphism between $\Cliff(M)$ and $\Wedge(M)$, we say that $d\vol\in\Cliff(M)$. Now we define 
\[ 
  \kappa=i^{-[(m+1)/2]}d\vol,
\]
which is nothing else that the volume form normalised in such a way that $\kappa^2=1$. If $m$ is even, it anti-commutes with $TM$, and if $m$ is odd, it commutes with $TM$.

Let $V$ be a $m$-dimensional real vector space, and $\CCliff(V)$, the corresponding complex Clifford algebra.
\begin{lemma}
Every $\CCliff(V)$-module accepts an unique decomposition as sum of irreducible representations as follows
\begin{enumerate}
\item if $m=2n$, there exists one and only one irreducible $\CCliff(V)$-module $\Delta$ and $\dim(\Delta)=2n$,
\item if $m=2n+1$, we have two inequivalent irreducible modules $\Delta_{\pm}$ with $\gamma(\kappa)=\pm 1$ on $\Delta_{\pm}$ and $\dim(\Delta_{\pm})=2^n$.
\end{enumerate}
\end{lemma}
\begin{proof}
No proof.
\end{proof}

Let $V$ be a vector bundle over $M$. A structure of $\Cliff(M)$-module over $V$ is a morphism of unital algebra $\gamma\colon \Cliff(M)\to \End(V)$. When one has a basis $\{ e_i \}$ of $V$, we pose $\gamma_i=\gamma(e_i)$. The following lemma is the lemma 1.2 of \cite{ResEtaDiracType}. 
\begin{lemma}			\label{LemGammaBaseConstant}
Let $V$ be a $\Cliff(V)$-module and $\{ e_i \}$, an orthonormal basis for $TM$ on a contractible open set $V$. Then there exists a local frame for $V$ such that the matrices $\gamma(e_i)$ are constant.
\end{lemma}
We also define $\gamma^i=\gamma(dx^i)=g^{ij}\gamma_j$. One easily proves that
\begin{equation}
\gamma^i\gamma^j+\gamma^j\gamma^i=-2g^{ij}
\end{equation}
where $(g^{ij})$ is the inverse matrix of $(g_{ij})$. If the endomorphisms $\gamma_i$ are constant in the basis $\{ e_i \}$, then the endomorphisms $\gamma^i$ are constant in the basis $\{ f_i=g_{ki}e_k \}$.



\section{Spin structure}	\label{sec:spin_str}
%++++++++++++++++++++++++

We consider a (pseudo-)Riemannian manifold $(M,g)$ with metric signature $(p,q)$, and $\SO(M)$, its frame bundle; it admits a $\SO(p,q)$-principal fibre bundle structure which is well defined by the metric $g$ (see \ref{subsubsecframebundle}).

\begin{definition}
We say that $(M,g)$ is a  \defe{spin manifold}{spin!manifold} if there exists a $\Sppq$-principal bundle $P$ over $M$ and a principal bundle homomorphism $\dpt{\varphi}{P}{\SO(M)}$ which induced covering for the structure groups is $\chi$, i.e.
$\varphi(\xi\cdot s)=\varphi(\xi)\cdot\chi(s)$. A choice of $P$ and $\varphi$ is a \defe{spin structure}{spin!structure} on $M$.
\label{defvarspin}
\end{definition}
\[
\xymatrix{ \Sppq \ar@{~>}[r]	& P \ar[rr]^-{\displaystyle\varphi} \ar[rd]_{\displaystyle\pi} && \SO(M) \ar[ld]^{\displaystyle p}&\SO(p,q) \ar@{~>}[l]  \\& & M }
\]
The wavy arrows mean ``structural group of''.

\begin{remark}
When we will use the concept of spin structure in the physical oriented chapters, we will naturally use $\SLdc$ as group instead of $\Sppq$. The isomorphism $\SLdc\simeq\Sput$ is proved in \cite{Michelson}. A physical motivation of such a structure is given at page \pageref{pg_spinenphyz}.
\end{remark}

\subsection{Example: spin structure on the sphere \texorpdfstring{$S^2$}{S2}}
%----------------------------------------------------------------

It is no difficult to see that $\SO(S^2)\simeq \SO(3)$. Indeed, each element of $\SO(S^2)$ is described by three orthonormal vectors: one which point to an element $x$ of $S^2$ and two which gives a basis of $T_xS^2$. The action $\SO(3)\times S^2\to S^2$ is transitive, and the stabilizer of any element is $\SO(2)$.

We define $\dpt{\alpha}{\SO(3)/\SO(2)}{S^2}$ by $\alpha(g\SO(2))=g$. Proposition \ref{propHelgason4.3} shows that $\alpha$ is a diffeomorphism. Then
\[
                S^2=\frac{\SO(3)}{\SO(2)}.
\]

On the other hand,  we know that
\begin{eqnarray}\label{explsu2} T_eSU(2)=su(2)=\left\{\begin{pmatrix}
ix & \xi \\
-\oxi & -ix
\end{pmatrix}\,:\,\xi\in\eC,x\in\eR\right\}.
\end{eqnarray}
 It is a classical result that $\mathfrak{su}(2)\simeq\eR^3$ not only as set but also as metric space with the identification
\[
\langle X,Y\rangle=-\frac{1}{2}\tr(XY),
\]
 for all $X$, $Y\in su(2)$. As we are in matrix groups, we know (see \cite{Lie} to get more details) that $Ad_xY=xYx^{-1}$. In our case, this gives the formula
\[
                \langle Ad(g)X,Ad(g)Y\rangle=\langle X,Y\rangle.
\]
We can now state the result for $S^2$.

 \begin{proposition}
The manifold $S^2$ with the usual metric induced from $\eR^3$ admits the following spin structure:
\begin{eqnarray}\label{spins2}
\xymatrix{ \Spin(2)\ar@{~>}[r] &SU(2) \ar[rr]^{\displaystyle \varphi=Ad} \ar[rd]_{\displaystyle U(1)}^{\displaystyle\pi} && \SO(3) \ar[ld]^{\displaystyle \SO(2)}_{\displaystyle p} \\& & S^2 },
\end{eqnarray} where the arrow
$\xymatrix{X \ar[r]^{f}_G & Y }$ means that $G$ is the kernel of the map $\dpt{f}{X}{Y}$.
\end{proposition}
\begin{proof}
 First, let us precise the concept of frame bundle for $S^2$, and how it is well described by $\SO(3)$. Let $\{e_1,e_2,e_3\}$ be the canonical basis of $\eR^3$. To $A\in \SO(3)$, we make correspond the basis $\{Ae_2,Ae_3\}$ at the point $Ae_1$ of $S^2$. The projection $\dpt{p}{\SO(3)}{S^2}$ is then defined by $p(A)=Ae_1$. It is clear that we will  define the map $\dpt{\pi}{SU(2)}{S^2}$ in the same way: $\pi(U)=p(Ad(U))$.

For the rest of the demonstration, we will use the ``$su(2)$ description''\ of $\eR^3$ given by \eqref{explsu2} with $\xi=y+iz$.

Now, let us show that $\dpt{\pi}{SU(2)}{S^2}$ is a $\Spin(2)$-principal bundle. Since we had already shown that $\Spin(2)\simeq U(1)$, we define the right action of $\Spin(2)$ on $SU(2)$ by right multiplication: $U\cdot s=Us$ with $s=\begin{pmatrix}
e^{i\theta} & 0 \\
0 & e^{-i\theta}
\end{pmatrix}$. It is clear that $\pi(Us)=\pi(U)$:
\begin{equation}
 Ad(Us)e_1=(Us)\begin{pmatrix}
 1 \\
 0 \\
 0
 \end{pmatrix}s^{-1} U^{-1}=Us
 \begin{pmatrix}
 i&0\\
 0&-i
 \end{pmatrix}s^{-1} U^{-1},
\end{equation}
because $\begin{pmatrix}
i&0\\
0&-i
\end{pmatrix}$ is the vector $e_1$ in the ``$su(2)$ description''\ of $\eR^3$.

In order for $\dpt{\pi}{SU(2)}{S^2}$ to be a $\Spin(2)$-principal bundle, we still need to show that for all $x\in S^2$,
\[
   \pi^{-1}(x)=\left\{\xi\cdot g\tq g\in\Spin(2)\,\forall\xi\in\pi^{-1}(x)\right\}.
\]
Take $A$, $B\in\pi^{-1}(x)$, i.e. $Ae_1=Be_1=x$. We need to find a $s\in\Spin(2)$ such that
\begin{eqnarray}
 \label{1603r3} A=B\cdot s.
\end{eqnarray}
The matrices $A$ and $B$ are such that
\begin{eqnarray}\label{1603r1}
 B^{-1} A\begin{pmatrix}
 i&0\\
 0&-i
         \end{pmatrix}A^{-1} B=\begin{pmatrix}
 i&0\\
 0&-i
         \end{pmatrix}.
\end{eqnarray}
This implies that $B^{-1} A\in\Spin(2)$. As $Ad$ is surjective from $SU(2)$ into $\SO(3)$, a general $C$ in $\SO(3)$ which acts on $e_1$ can be written $Ue_1U^{-1}$ for $U\in SU(2)$ such that $Ad(U)=C$. The condition \eqref{1603r1} becomes
\[
\begin{pmatrix}
\alpha&\beta\\
-\obeta&\oalpha
\end{pmatrix}
\begin{pmatrix}
i&0\\
0&-i
\end{pmatrix}
\begin{pmatrix}
\oalpha&-\beta\\
\obeta&\alpha
\end{pmatrix}=
\begin{pmatrix}
i&0\\
0&-i
\end{pmatrix},
\]
which implies $\alpha=e^{i\theta}$, $\beta=0$. Then $B^{-1} A$ belongs to $\Spin(2)$, and $s=B^{-1} A$ fulfills the condition \eqref{1603r3}.

What about the induced covering for the structural groups ? The structural group of $\dpt{\pi}{SU(2)}{S^2}$ is $\Spin(2)$, while the one of $\dpt{p}{\SO(3)}{S^2}$ is $\SO(2)$. Indeed, for each $x\in S^2$, $\SO(2)$ acts on $T_xS^2$, leaving $x$ unchanged. We have the following associations:
\[
         U\in SU(2)\stackrel{\varphi}{\longrightarrow}A\in \SO(3),
\]
the matrix $A$ being defined by $A\cdot X=UXU^{-1}$. For $s\in\Spin(2)$ we of course also have
\[
         Us\in SU(2)\stackrel{\varphi}{\longrightarrow}As\in \SO(3),
\]
with $As\cdot X=UsXs^{-1} U^{-1}$. As we act by $\Spin(2)$ on $SU(2)$, in the fibres of $\SO(3)$, the action of $\Spin(2)$ is --via $\varphi$-- the composition with $X\to sXs^{-1}$. But this is exactly $\chi(s)X$ because $\alpha(s)=s$, since $s\in\Spin(2)$.
\end{proof}

\subsection{Spinor bundle}
%--------------------------

Let us take once again the spin structure on the (pseudo-)Riemannian manifold $(M,g)$:
\[
  \xymatrix{ \Sppq \ar@{~>}[r]& P \ar[rr]^-{\displaystyle\varphi}
   \ar[rd]_{\displaystyle\pi} && \SO(M) \ar[ld]^{\displaystyle p}&\SO(p,q) \ar@{~>}[l]
   \\& &   M }
\]
with $\varphi(\xi\cdot g)=\varphi(\xi)\cdot\chi(g)$.

Let us define $S=\Lambda W $, and $\mS=P\times_{\rho}S$. Take $\dpt{\rho}{\Sppq\times\mS}{\mS}$, $\rho(g,s)=\tilde\rho(g)s$, where $\tilde\rho$ is the spinor representation of $\Sppq$ on $S$. We also have
$\dpt{\chi}{\Sppq}{\SO_0(p,q)}$, $\chi(g)v=\alpha(g)\cdot v\cdot g^{-1}$, with $\alpha(g)=g$ for $g\in\Sppq$.

The \defe{spinor bundle}{spinor!bundle} is the associated bundle
\begin{equation}
                   \mS=P\times_{\rho}S\to M
\end{equation}
A \defe{spinor field}{spinor!field} is an element of $\Gamma(\mS)$, the space of section of the spinor bundle.

On $\SO(M)$, we look at a connection $1$-form $\alpha\in\Omega^1(\SO(M),so(\eR^m))$,
and, if $T(M)$ is the tensor bundle over $M$, we define a covariant derivative $\dpt{\nabla^{\alpha}}{\cvec(M)\times T(M)}{T(M)}$ by
 \[
             \widehat{\nabla^{\alpha}_X s}=\overline{X}\hat{s},
\]
 for any $s\in T(M)$. See theorem \ref{tho_nablaE}, and the fact that $T(M)$ can be see as an associated bundle; it is explicitly done for $\cvec(M)$ at page \pageref{equivvec}.

As seen in point \ref{subsubsec_levi}, an automatic property of this connection is $\nabla^{\alpha} g=0$ if $g$ is the metric of $M$. The \defe{Levi-Civita connection}{connection!Levi-Civita}\index{connection!Levi-Civita} is the unique\footnote{We will not prove unicity.} such connection which is torsion-free: $T^{\nabla^{\alpha}}=0$.


\begin{proposition}
The $1$-form $\talpha=\varphi^*\alpha\in\Omega^1(P,so(\eR^{m}))$ defines a connection on $P$. See definition \ref{defconnform} and theorem \ref{tho_nablaE}.
\end{proposition}

\begin{proof}
Let us denote by $R_g$ the right action of $g\in\Sppq$ on $P$ (\emph{id est} $R_g\xi=\xi\cdot g$), and by $R_u^{\SO(M)}$ the right action of $u\in\Sopq$ on $\SO(M)$.
We  have to check the usual two conditions of a connection.

\subdem{First condition}
The first one is:
\[
            (R_g^*\talpha)_{\xi}(\Sigma)=Ad(g^{-1})(\talpha_{\xi}(\Sigma)),
\]
for all $\xi\in P$, and $\Sigma\in T\bxi P$. In order to check this, we first remark that $\varphi\circ R_g=R_{\chi(g)}^{\SO(M)}\circ\varphi$. Indeed, for all $\xi\in P$, definition \ref{defvarspin} gives us $\varphi(R_g\xi)=\varphi(\xi\cdot g)=\varphi(\xi)\cdot\chi(g)$.  With this, we can make the following computation:
\begin{equation}\label{1603r4}
\begin{aligned}
R_g^*\talpha&=R_g^*\varphi^*\alpha=(\varphi\circ   R_g)^*\alpha	=(R_{\chi(g)}^{\SO(M)}\circ\varphi)^*\alpha\\
            &=\varphi^*R_{\chi(g)}^{\SO(M)*}\alpha=\varphi^*(Ad(\chi(g)^{-1})\circ\alpha).
\end{aligned}
\end{equation}
The last equality comes from the fact that $\alpha$ is a connection $1$-form. As we are in matrix groups, we have $Ad(g)x=gxg^{-1}$, so
\begin{equation}
   [Ad(\chi(g))x]v=[\chi(g) x \chi(g)^{-1}]v
                  =\chi(g)[xg^{-1} vg]
                  =gxg^{-1}.
\end{equation}
In the first line, the product is the usual matrix product which can be seen as operator composition.

But $(Ad(g)x)v=gxg^{-1} v$. Then $Ad(g)=Ad(\chi(g))$, if we identify $\sppq\simeq\sopq$ by proposition \ref{prop:spin_so}. Moreover, the action of $Ad$ is linear, so it commutes with $\varphi^*$. With these remarks, we can continue the computation \eqref{1603r4}:
\begin{equation}
 \varphi^*(Ad(\chi(g)^{-1})\circ\alpha)=\varphi^*(Ad(g^{-1})\circ\alpha)
                                  =Ad(g^{-1})\circ\varphi^*\alpha
                                  =Ad(g^{-1})\circ\talpha.
\end{equation}
This proves the first condition.

\subdem{Second condition}
The second one is $\talpha(A^*\bxi)=-A$ with the definition \eqref{defastar}. This is also a computation. First remark
\[
 \talpha\bxi(A\bxi^*)=(\varphi^*\alpha)\bxi(A^*\bxi)=\alpha_{\varphi(\xi)}(\varphi_{*\xi}A^*\bxi).
\]
 We compute $\varphi_{*\xi}A^*$ with lemma \ref{lemsur5d}:
\begin{equation}
\begin{split}
 \varphi_{*\xi}A^*&=\dsdd{\varphi(\xi\cdot\exp -tA)}{t}{0} =\dsdd{(R_{\chi(\exp -tA)}^{\SO(M)}\circ\varphi)(\xi)}{t}{0}\\
              &=\dsdd{\varphi(\xi)\cdot\chi(\exp -tA)}{t}{0}=\dsdd{\varphi(\xi)\cdot\exp(-td\chi_eA)}{t}{0}=(d\chi_eA)^*_{\varphi(\xi)}.
\end{split}
\end{equation}
But $d\chi_e=\id_{so(p,q)}$, thus $\varphi_{*\xi}A^*=A^*_{\varphi(\xi)}$. The whole makes that:
\[
\talpha\bxi(A^*\bxi)=\alpha_{\varphi(\xi)}(\varphi_{*\xi}A^*\bxi)=\alpha_{\varphi(\xi)}(A^*_{\varphi(\xi)})=-A.
\] 
This completes the proof.
\end{proof}

\begin{definition}
This connection $1$-form on $P$ is called the \defe{spinor connection}{spinor!connection}. It gives us a covariant derivative on any associated bundle and in particular on the spinor bundle, $\dpt{\tnab}{\cvec(M)\times\Gamma(\mS)}{\Gamma(\mS)}$.
 \label{spinconn}
\end{definition}
\nomenclature[D]{$\dpt{\tnab}{\cvec(M)\times\Gamma(\mS)}{\Gamma(\mS)}$}{Covariant derivative for the spinor connection}

\begin{proposition}
If $X$, $Y\in\cvec(M)$ are such that $X_x=Y_x$, then for all $s\in\Gamma(\mS)$,
\[
              (\tnab_Xs)(x)=(\tnab_Ys)(x).
\]
 \label{2303p1}
\end{proposition}
\begin{proof}
We just have to show that for all vector field $Z$ such that $Z_x=0$, $(\tnab_Zs)(x)=0$. Such a $Z$ can be written as $Z=fZ'$ for a function $f$ on $M$ which satisfies $f(x)=0$. We have:
\[
\tnab_Zs=\tnab_{fZ'}s=f\tnab_{Z'}s,
\]
which is obviously zero at $x$.
\end{proof}

Let $x\in M$ and $\{e_{\alpha\,x}\}$ be an orthonormal basis of $T_xM$. We can extend it to $\{e_{\alpha}\}$, a local basis field around $x$ such that $e_{\alpha}$ is a section of the frame bundle (in other words, we ask the extension to be smooth). The claim of proposition \ref{2303p1} is that $\tnab_{e_{\alpha}}(x)$ is an element of $\mS_x$ which doesn't depend on the extension.

\section[Dirac operator]{Dirac operator\protect\quad{\Huge\Smiley}}		\label{applgamma}
%---------------------------------------------------------------------

\subsection{Preliminary definition}
%----------------------------------

Let $M$ be a $m$-dimensional (pseudo)Riemannian manifold with its spin structure 
\[
\xymatrix{ \Sppq \ar@{~>}[r]& P \ar[rr]^{\displaystyle\varphi} \ar[rd]_{\displaystyle\pi} && \SO(M) \ar[ld]^{\displaystyle p}&\SO(p,q) \ar@{~>}[l]  \\& & M }
\]
where $\varphi$  satisfies $\varphi(\xi\cdot g)=\varphi(\xi)\cdot\chi(g)$.

Recall that for any vector space, one can see $\End{V}=V^*\otimes V$ with the definition $(v^*\otimes v)w=(v^*w)v$. This allows us to define an action of $\Sppq$ on $\End{S}$ by defining an action of $\Sppq$ on $S$ and $S^*$ separately. We know the action 
\begin{equation}
\begin{aligned}
 \Spin(p,q)\times S&\to S \\ 
(g,v)&\mapsto \tilde\rho(g)v,
\end{aligned}
\end{equation}
and as action on $S^*$, we take the dual one
\begin{equation}
\begin{aligned}
 \Spin(p,q)\times S^*&\to S^* \\ 
 g\cdot\alpha&= \alpha\circ\tilde\rho(g^{-1}) 
\end{aligned}
\end{equation}
for all $g\in\Spin(p,q)$ and $\alpha\in S^*$.

Now we can make the following computation with $g\in\Sppq$, $\alpha\in S^*$ and $v\in S$, using the fact that $\tilde\rho$ is linear:
\begin{equation}
\begin{split}
[g\cdot(\alpha\otimes v)]w&=[(\alpha\circ\tilde\rho(g^{-1}))w]\tilde\rho(g)v\\
                          &=\tilde\rho\left([(\alpha\circ\tilde\rho(g^{-1}))w]g\right)v\\
                          &=\big[\tilde\rho(g)\circ(\alpha\otimes v)\circ\tilde\rho(g^{-1})\big]w.
\end{split}
\end{equation}
Then we write the action of $\Sppq$ on $\End{S}$\index{action!of $\Sppq$ on $\End{S}$} by ($A\in\End S$)
\begin{equation}
     g\cdot A=\tilde\rho(g)\circ A\circ\tilde\rho(g^{-1}).                          \label{actspin}
\end{equation}
Notice that this definition is the one required in condition \eqref{equivA}.

The tangent bundle $T_xM$ is given with a metric $g_x$. As usual, we build $S_x=\Lambda W _x$, a completely isotropic subspace of $T_xM$ with respect to the metric $g_x$, and a representation
\[ 
\tilde \rho_x\colon T_xM\to \End(\Lambda W_x)
\]
The first step in the definition of $\gamma(X)$ is to build $\dpt{\ha_X}{P}{\End(\Lambda W )}$ setting\footnote{See subsection \ref{equivvec} for the definition of $\hX$.} $\ha_X(p)=\tilde\rho(\hX_{\varphi(p)})$.

\begin{lemma}
The function $\hat a$ is equivariant, i.e. it satisfies
\begin{equation}
     \ha_X(p\cdot g)=g^{-1}\cdot\ha_X(p)                             \label{equivaX}
\end{equation}
for all $g\in\Sppq$.
\end{lemma}

\begin{proof}
It is no more than a simple computation using the equivariance of $\hX$. Indeed:
\begin{equation}
\begin{split}
 \ha_X(p\cdot g)&=\tilde\rho(\hX_{\varphi(p\cdot g)})=\tilde\rho(\hX_{\varphi(p)\chi(g)})=\tilde\rho(\chi(g^{-1})\cdot\hX_{\varphi(p)})\\
		&=\tilde\rho(g^{-1}\cdot\hX_{\varphi(p)}\cdot g)=\tilde\rho(g^{-1})\circ\tilde\rho(\hX_{\varphi(p)})\circ\tilde\rho(g)\\
                &=g^{-1}\cdot\ha_X(p).
\end{split}
\end{equation}
In the fourth line, the dots mean the Clifford product, and the last equality comes from the definition of the action \eqref{actspin} of $\Sppq$ on $\End{S}$.
\end{proof}

From the discussion of section \ref{sec_fnequiv}, the function $\dpt{\ha_X}{P}{\End{S}}$ defines a section $\dpt{a_X}{M}{\End{\mS}}$. We define $\dpt{\gamma}{\cvec(M)}{\End{\Gamma(\mS)}}$ by
\nomenclature[D]{$\dpt{\gamma}{\cvec(M)}{\End{\Gamma(\mS)}}$}{A key ingredient for Dirac operator}
\begin{equation}		\label{EqDefgammax}
                       \gamma(X)=a_X.
\end{equation}
We immediately have
\[
                     \widehat{\gamma(X)}(p)=\tilde\rho(\hX_{\varphi(p)})
\]
for any $p\in P$. If we define
\begin{equation}\label{3103r1}
  \widehat{\gamma\cdot a_X}(p)=\widehat{\gamma(X)}(p),
\end{equation}
the map $\gamma$ can be seen as an action on the section of $\mS$. Indeed, $\widehat{\gamma\cdot s}_X$ is an equivariant function:
\begin{equation}
\begin{split}
 \hat{\gamma}(p\cdot g)(\ha_X(p\cdot g))
                  &=\rho(g)^{-1}\hat{\gamma}(p)\rho(g)\rho(g^{-1})\ha_X(p)\\
                  &=\rho(g)^{-1}\hat{\gamma}(p)\ha_X(p)\\
                  &=\rho(g^{-1})\widehat{\gamma\cdot a_X}(p),
\end{split}
\end{equation}
 so that
\[
    \widehat{\gamma\cdot a_X}(p)=\rho(g^{-1})\widehat{\gamma\cdot a_X}(p).
\]

The map $\dpt{\widehat{\gamma\cdot a_X}}{P}{\End{\Lambda W }}$ defined by \eqref{3103r1} is equivariant, and thus defines a section
$\gamma\cdot a_X\in\Gamma(\mS)$, as seen in the section \ref{sec_fnequiv}.

\subsection{Definition of Dirac}
%-------------------------------

If we consider a basis $\{e_{\alpha}\}$ of $TM$, \emph{i.e.} $m$ sections $\dpt{e_{\alpha}}{M}{TM}$ such that for all $x$ in $M$, the set $\{e_{\alpha x}\}$ is a basis of $T_xM$, we note $\gamma^{\alpha}:=\gamma(e_{\alpha})\in\End(\mS)$.

\begin{remark}  
This is not always globally possible. The example of the sphere is given in subsection \ref{subsec_DimofModule}. 
\label{rem_secnoglobal}
\end{remark}

For any $s\in\Gamma(\mS)$, we consider the local\footnote{Extensions of $e_{\alpha}$ do not always globally exist, see remark \ref{rem_secnoglobal}.} section $\psi$ of $\mS$ given by
\[
    \psi(x)=
   \sum_{\alpha\beta}g_x(e_{\alpha},e_{\beta})\gamma_x\hbeta(\tnab_{e_{\alpha}}s)(x).
\]

For each $x\in M$, take a $A_x$ in\footnote{By $\SO(g_x)$, we mean the set of all the matrix $A$ such that $A^tg_xA=g$; $A_x$ is an isometry of $(T_xM,g_x)$. In other words, we consider $A$ as a section of what we could call the ``isometry bundle''.} $\SO(g_x)$, and consider the new basis $e'_{\alpha}=A_{\alpha}^{\phantom{\alpha}\beta}e_{\beta}$. As $A$ is an isometry, $g_x(e'_{\alpha},e'_{\beta})=g_x(e_{\alpha},e_{\beta})$; and since $\tilde\rho$ is linear, $\gamma_x'^{\alpha}=\tilde\rho_x(e'_{\alpha x})=A_{\alpha}^{\phantom{\alpha}\beta}\tilde\rho(e_{\beta x})=A_{\alpha}^{\phantom{\alpha}\beta}\gamma_x\hbeta$. In the new basis, the section reads:
\begin{equation}
\begin{split}
   \psi(x)&=\sum_{\alpha\beta\eta\sigma}g_x(e_{\alpha},e_{\beta})
                A_{\beta}^{\phantom{\beta}\sigma}\gamma_x^{\sigma}
                (\tnab_{A_{\alpha}^{\phantom{\alpha}\eta}e_{\eta}}s)(x)\\
          &=\sum_{\alpha\beta\eta\sigma}(A^t)\heta_{\phantom{\eta}\alpha}
                  g_{\alpha\beta}(x)A_{\beta}^{\phantom{\beta}\sigma}
                  \gamma_x^{\sigma}(\tnab_{e_{\eta}}s)(x)\\
          &=\sum_{\eta\sigma}g_x(e_{\eta},e_{\sigma})\gamma_x^{\sigma}(\tnab_{e_{\eta}}s)(x),
\end{split}
\end{equation}
where we used the fact that $A^tgA=g$ and that all the $A_{\alpha}^{\phantom{\alpha}\beta}$ are $C^{\infty}$ functions on $M$, so that
$\tnab_{A_{\alpha}^{\phantom{\alpha}\beta}X}=A_{\alpha}^{\phantom{\alpha}\beta}\tnab_X$.  This shows that $\psi(x)$ doesn't depend on the choice of the basis, so it defines a section from the data of $s$ alone.


The \defe{Dirac operator}{dirac!operator!on $(M,g)$, a spin manifold}\nomenclature[D]{$\Dir$}{Dirac operator} $\Dir\colon \Gamma(\mS)\to \Gamma(\mS)$ acting on a spinor field is defined by 
\begin{equation}\label{dirac}
(\Dir s)(x)=g_x(e_{\alpha},e_{\beta})\gamma_x\hbeta(\tnab_{e_{\alpha}}s)(x).
\end{equation}

\begin{proposition}
If the field of basis $e_{\alpha}\in\cvec(M)$ is everywhere an orthonormal basis, the Dirac operator reads
\begin{equation}
(\Dir s)(x)=g_{\alpha\beta}\gamma^{\alpha}(\tilde\nabla_{e_{\beta}}s)(x)
\end{equation}
where $\gamma^{\alpha}$ is a constant numeric matrix acting on $\Lambda W$.
\end{proposition}

\begin{proof}
The building of the Dirac operator begins by considering the vector space $T_xM$ endowed with the metric $g_x$; then the spinor representation $\tilde\rho_x\colon T_xM\to \End(\Lambda W_x)$ where $\Lambda W_x$ is build from isotropic vectors of $T_xM$ is defined. If the vector fields $e_{\alpha}\in\cvec(M)$ are everywhere orthonormal for the metric $g$, then we have the matricial equality
\begin{equation}
	\tilde\rho_x\big( (e_{\alpha})_x \big)_{ij}=\tilde\rho(v_{\alpha})_{ij}
\end{equation}
where the left hand side describe the matrix component of a linear operator acting on $\Lambda W_x$ while in the right hand side we have the matrix component of a linear operator acting on $\Lambda W$ and $v_{\alpha}$ is a basis on $\eR^n$ with respect to which the metric is the same as the metric $g_x$ in the basis $(e_{\alpha})_x$. Let $\hat{\psi}\colon P\to \Lambda W$ be an equivariant function; from definition \eqref{EqDefgammax} of $\gamma$ we have
\[ 
  \big( \gamma(e_{\alpha}\hat{\psi}) \big)(\xi)=(a_{\alpha}\hat{\psi})(\xi)
\]
where $a_{\alpha}(\xi)=\tilde\rho\Big( \tilde{e}_{\alpha}\big( \phi(\xi) \big) \Big)$. In this expression, $\tilde{e}_{\alpha}$ is the equivariant function associated with the vector field $e_{\alpha}\in\cvec(M)$. It is defined in subsection \ref{equivvec} as
\begin{equation}
\begin{aligned}
 \tilde{e}_{\alpha}\colon \SO(M)&\to \eR^m \\ 
b&\mapsto b^{-1}\big( (e_{\alpha})_{\pi(b)} \big). 
\end{aligned}
\end{equation}
So we have $\hat a_{\alpha}\colon P\to \End(\Lambda W)$ defined by
\[ 
  \hat a_{\alpha}(\xi)=\tilde\rho\big( \varphi(\xi)^{-1}e_{\alpha}(x) \big)
\]
with $x=\pi(\xi)$. Now if $\xi$ is any element of $\pi^{-1}(x)$, we have
\begin{align*}
\big( \gamma(e_{\alpha})\psi \big)(x)&=(a_{\alpha}\psi)(X)=\big[ \xi,\hat a_{\alpha}(\xi)\hat\psi(\xi) \big]
		=\big[ \xi,\tilde\rho\big( \varphi(\xi)^{-1}e_{\alpha}(x) \big)\hat{\psi}(\xi) \big].
\end{align*}
There exists a $g\in\Spin(p,q)$ such that $\varphi(\xi\cdot g)=\mtu$; taking this element and using equivariance of the latter expression,
\begin{align}
  \big( \gamma(e_{\alpha})\psi \big)(x)=\big[ \xi\cdot g,\tilde\rho\big( e_{\alpha}(x) \big)\hat{\psi}(\xi\cdot g) \big]
		=\big[ \xi\cdot g,\gamma^{\alpha}\hat{\psi}(\xi) \big]
		=[\xi,\gamma^{\alpha}\hat{\psi}(\xi)].
\end{align}
What we proved is that $\big( \gamma e_{\alpha}\psi \big)(x)=\gamma^{\alpha}\psi(x)$ is the sense that
\begin{equation}
	\widehat{\gamma(e_{\alpha})\psi}=\gamma^{\alpha}\hat{\psi}.
\end{equation}
Hence the Dirac operator reads
\[ 
  (\Dir s)(x)=g_{\alpha\beta}\gamma^{\alpha}\big( \tilde\nabla_{e_{\beta}}s \big)(x)
\]
in the sense that
\begin{equation}
\widehat{\Dir s}=g_{\alpha\beta}\gamma^{\alpha}\widehat{  \tilde\nabla_{e_{\beta}}s }.
\end{equation}

\end{proof}


An often more convenient way to write the Dirac operator is to consider an orthonormal basis (so that the metric $g$ and the matrices $\gamma$ are constant) and to consider the equivariant functions:
\[ 
  \widehat{\Dir\psi}=g_{\alpha\beta}\gamma^{\alpha}\widehat{\nabla_{e_{\alpha}}\psi}.
\]
This formulation is typically used when one search for Dirac operator on Lie groups. In this case, we choose left invariant vector fields generated by an orthonormal basis of the Lie algebra. The resulting field of basis is everywhere Killing-orthonormal.

Acting on a function $\dpt{f}{M}{\eR}$, it is defined by $\dpt{\Dir}{C^{\infty}(M)}{C^{\infty}(M)}$\index{dirac!operator!on functions},
\begin{equation} \label{eq_defDirac_f}
(\Dir f)(x)=g_x(e_{\alpha},e_{\beta})\gamma\hbeta_x(e_{\alpha x}\cdot f).
\end{equation}
With these definitions, one has
\[(\Dir(fs))(x)=(f\Dir s)(x)+(\Dir f)(x).\] Indeed,
\begin{equation}
\begin{split}
   (\Dir(fs))(x)&=g_{\alpha\beta}\gamma_x\hbeta(\tnab_{e_{\alpha}}fs)(s)\\
                &=g_{\alpha\beta}\Big((e_{\alpha}\cdot f)s(x)+f(x)(\tnab_{e_{\alpha}}s)(x)\Big)\\
                &=f(x)(\Dir s)(x)+g_{\alpha\beta}\gamma_x\hbeta(e_{\alpha x}\cdot f)\\
                &=(f\Dir s)(x)+(\Dir f)(x).
\end{split}
\end{equation}
With that definition, the Dirac operator becomes a derivation of the spinor bundle.


\section[Dirac operator on  \texorpdfstring{$\eR^2$}{R2}]{Example: Dirac operator on \texorpdfstring{$\eR^2$}{R2} with the euclidian metric}\label{Pg_exempleRdeux}\index{dirac!operator!on $\eR^2$}
%---------------------------------------------------

%\subsubsection{Example: tangent bundle}
%--------------------------

Since the frame bundle $B(M)$ is a principal bundle (see subsection \ref{subsec_frbundle}), one can consider some associated bundles on it. We are now going to see that the one given by the definition representation $\dpt{\rho}{GL(n,\eR)}{GL(n,\eR)}$ on $\eR^n$ is the tangent bundle. So we study $B(M)\times_{\rho} \eR^n$. By choosing a basis on each point of $M$, we identify each $T_xM$ to $\eR^n$. An element of $B(M)\times \eR^n$ is a pair $(b,v)$ with $b=(\overline{b}_1,\ldots,\overline{b}_n)$ and $v=(v^1,\ldots,v^n)$. We can identify $v$ to the element of $T_xM$ given by $v=v^i\overline{b}_i$.

In order to build the associated bundle, we make the identifications
\[
  (b,v)\cdot g\sim(b\cdot g,g^{-1} v).
\]
Here, by $gv$ we mean the vector whose components are given by $(gv)^i=v^j\bghd{g}{j}{i}$. The tangent vector given by $(b\cdot g,g^{-1} v)$ is $(g^{-1} v)^i(b\cdot g)_i=v^j\bghd{(g^{-1})}{j}{i}\bghd{g}{i}{k}\overline{b}_k=v^k\overline{b}_k$ So the identification map $\dpt{\psi}{B(M)\times_{\rho}\eR^n}{TM}$ given by
\[
  \psi([b,v])=v^i\overline{b}_i
\]
is well defined.

\index{spin!structure!on $\eR^2$}
The following step is to consider the following spin structure:
 \[\xymatrix{
    \Spin(2)  \ar@{~>}[r]&  \eR^2\times \SO(2) \ar[r]^-{\displaystyle\varphi} & \SO(\eR^2)  & \SO(2) \ar@{~>}[l].
  }\]

We have to define the two actions and $\varphi$. One of the main result of section~\ref{cliffR2} is that $\dpt{\chi}{\Spin(2)=U(1)}{\SO(2)}$ is surjective. So, we can define the action of $\Spin(2)$ on $P$ by
\[(x,b)\cdot s=(x,\chi(s)^{-1} b).\]

On the other hand, an element $A$ in $\SO(\eR^2)$ can be written as $A=\baz{a}{x}$ where $e_i$ is the canonical basis of $T_x\eR^2$, and $a$ is a matrix of $\SO(2)$. See subsection \ref{subsec_frbundle}. For $g\in \SO(2)$, we define
\begin{eqnarray}
 \label{r1504d2}A\cdot g=\{g^{-1} ae_i\}_x.
\end{eqnarray}
and  $\dpt{\varphi}{\eR^2\times \SO(2)}{\SO(\eR^2)}$ by
\[
\varphi(x,b)=\{be_i\}_x.
\]
The following shows that these definitions give a spin structure:
\begin{equation}
   \varphi((x,b)\cdot s)=\varphi(x,\chi(s)^{-1} b)
                    =\{\chi(s)^{-1} be_i\}_x
                    =\{be_i\}_x\cdot\chi(s)
                    =\varphi(x,b)\cdot\chi(s).
\end{equation}


\subsection{Connection on \texorpdfstring{$\SO(\eR^2)$}{SO(R2)}}\index{connection!on $\SO(\eR^2)$}
%///////////////////////////////////////

We are searching for a torsion-free connection on the simplest metric space: the euclidian $\eR^2$. Thus we will try the simplest choice of horizontal space: we want an horizontal vector to be tangent to a curve of the form $X(t)=\baz{b}{x(t)}$. For this reason, we want to define the connection $1$-form by $\omega(X)=b'(0)$. For technical reasons which will soon be apparent, we will not exactly proceed in this manner. For $X(t)=\baz{b}{x(t)}$, we define
\begin{equation}
                       \omega(X)=-(b(t)b(0)^{-1})'(0).
\end{equation}
We of course have $\omega(X)=0$ if and only if $b'(0)=0$: this choice of $\omega$ follows our first idea. In order for $\omega$ to be a connection form, we have to verify the two conditions of definition \ref{defconnform}.

\begin{proposition}
The $1$-form defined by
\[
              \omega(X)=-(b(t)b(0)^{-1})'(0)
\]
for $X=\displaystyle\dsdd{\baz{b(t)}{x(t)}}{t}{0}$ is a connection $1$-form.
\end{proposition}

\begin{proof}
Let $A\in \SO(2)$. If $u=\baz{b}{x}$, equation \eqref{r1504d2} gives:
\[
   A^*_u=\dsdd{\baz{e^{-tA}b}{x}}{t}{0},
\]
 so that $\omega(A^*_u)=-(e^{-tA}bb^{-1})'(0)=A$. This checks the first condition. For the second, one remarks that the path in $\SO(\eR^2)$ which defines the vector $R_{g*}X$ is $(R_{g*}X)(t)=\baz{g^{-1} b(t)}{x}$. It follows that
\begin{equation}
\begin{split}
\omega(R_{g*}X)&=-(g^{-1} b(t)b(0)^{-1} g)'(0)\\
               &=-\left(\AD_{g^{-1}}(b(t)b(0)^{-1})\right)'(0)\\
               &=-Ad_{g^{-1}}(b(t)b(0)^{-1})'(0)\\
               &=Ad_{g^{-1}}\omega(X).
\end{split}
\end{equation}

\end{proof}

\begin{proposition}
The covariant derivative induced on $M$ by this connection is
\begin{equation}\label{derrcovexplicite}
                 \nabla_XY=X(Y).
\end{equation}
\end{proposition}

\begin{proof}
In this demonstration, we will use the equivariant functions defined in \ref{equivvec}. In order to compute $(\nabla_XY)_x$, we have to use the definition of theorem \ref{tho_nablaE}. We first have to compute the horizontal lift of $X$. It is no difficult to see that $\oX_{\baz{b}{x}}$ is given by the path
\[\oX(t)=\baz{b}{X(t)}\]
if the vector field $X$ is given by the path $X(t)$ in $M$. Indeed, it is trivial that $\omega(\oX)=0$, and
\[d\pi_*\oX=\dsdd{\pi\baz{b}{X(t)}}{t}{0}=\dsdd{X(t)}{t}{0}=X.\]

Now, we compute $(\oX\hs)(b)$ for $b=\{Se_i\}_x$. We begin using the basic definitions and notations:
\[
(\oX\hs)(b)=\oX_b\hs=\dsdd{\hs(\oX_b(t))}{t}{0}=\dsdd{\hs(\baz{S}{X(t)})}{t}{0}.
\]
We can rewrite it with $\hY$ instead of $\hs$. By construction (see \eqref{r1404e1}), if $b=\baz{S}{x}$, $\hY(b)=S^{-1}(Y_x)$. Thus
\[
(\oX\hY)(b)=\dsdd{S^{-1}(Y_{X(t)})}{t}{0},
\]
where, if $\{\oui\}$ is a basis of $\eR^m$, then $S$ is 
\begin{equation}
\begin{aligned}
 S\colon\eR^m&\to T_{X(t)}M \\ 
 v^i\oui &\mapsto S^i_jv^j(\partial_j)_{X(t)} 
\end{aligned}
\end{equation}
So if we write $Y_x=Y^i(x)\partial_i$, we have
\[
S^{-1}(Y_{X(t)})=(S^{-1})^i_jY^j(X(t))\oui
\]
and
\[
\dsdd{S^{-1}(Y_{X(t)})}{t}{0}=(S^{-1})^i_j\dsdd{Y^j(X(t))}{t}{0}\oui=(S^{-1})^i_jX(Y^j)\oui.
\]
Since $b$ is an isomorphism, we can apply $b$ on both side of $\hX(b)=b^{-1}(X_x)$, and take $\nabla_XY$ instead of $X$:
\begin{equation}
 (\nabla_XY)(x)=b\big((S^{-1})^i_jX(Y^j)\oui\big)
               =S^k_i(S^{-1})^i_jX(Y^j)(\partial_k)_x
               =X(Y^j)(\partial_j)_x
               =X(Y)_x.
\end{equation}
\end{proof}

From this and definition \ref{deftorsion}, we immediately conclude that our connection is torsion-free. In a certain manner, one can say that our covariant derivative is the usual one.

\subsection{Construction of \texorpdfstring{$\gamma$}{g}}
%//////////////////////////////////////

Now, we construct the map $\gamma$ of subsection \ref{applgamma}. The first step is to define $\dpt{\ha_X}{P}{\End{(\Lambda W )}}$ by
\[
\ha_X(p)=\tilde\rho(\hX_{\varphi(p)}).
\]
Here, $\Lambda W $ is the completely isotropic subspace of $(\eR^2)^{\eC}$ with euclidian metric; thus we can use the result of section \ref{cliffR2}. In particular, we know the representation $\tilde\rho$.

To see it more explicitly, we need the expression of $\hX$. It is given in subsection \ref{equivvec}: if $b$ is the basis $\baz{b}{x}$, $\hY(b)=b^{-1}(Y_x)$. As $\varphi(b,x)=\baz{b}{x}$, we have
\[
\ha_X(b,x)=\tilde\rho(b^{-1}(X_x)).
\]

The subsection \ref{equivendo} explains how to explicitly get $\gamma(X)$ with the definition $\gamma(X)=a_X$. If $\psi$ is a section of $\mS$ and $\psi(x)=[\xi,v]$, the general definition gives us $(a_X\psi)(x)=[\xi,\ha_X(\xi)v]$ and in our particular case, if $\xi=(b,x)$, we get:
\begin{eqnarray}
 \label{gammaX}(\gamma(X)\psi)(x)=[\xi,\tilde\rho(b^{-1}(X_x))v].
\end{eqnarray}

\subsection{Covariant derivative on \texorpdfstring{$\protect\Gamma(\mS)$}{S}}
%///////////////////////////////////////////

Remember the spin structure of $\SO(\eR^2)$: $\varphi(x,S)=\{Se_i\}_x$. We now construct the connection on $P=\eR^2\times \SO(2)$. It is defined by the $1$-form $\tomega=\varphi^*\omega$. If $v$ is a vector of $P$, it is described by a path $v(t)=(x(t),b(t))$, then the path of $d\varphi(v)$ is $\{b(t)e_i\}_{x(t)}$ and $\tomega(v)=\omega(d\varphi(v))=-(b(t)b(0)^{-1})'(0)$.

The next step defining the Dirac operator is to find out an explicit form for the map $\dpt{\tnab}{\cvec(M)\times\Gamma(\mS)}{\Gamma(\mS)}$. A section $s\in\Gamma(\mS)$ is a map $\dpt{s}{M}{\mS=(\eR^2\times \SO(2))\times_{\rho}\Lambda W }$; it is defined by an equivariant function $\dpt{\hs}{P}{\Lambda W }$. In order to find the value of $(\tnab_Xs)(x)$ for $X\in\cvec(M)$, we use the definition
\[
 \widehat{\tnab_Xs}(\xi)=\oX\bxi(\hs)
\]
where $\oX$ is the horizontal lift in the sense of $\tomega$. For the same reason as in the proof of proposition \ref{derrcovexplicite}, $\oX_{(b,x)}$ is given by the path $\oX(t)=(b,X(t))$ where $X(t)$ is the path which defines $X$. So we have
\[
 \widehat{\tnab_Xs}(\xi)=\oX_{(b,x)}(\hs)=\dsdd{\hs(b,X(t))}{t}{0}.
\]
Remark that $\Lambda W $ is a vector space; so for every $\alpha\in\Lambda W $, the identification $T_{\alpha}\Lambda W =\Lambda W $ is correct.

Our first form of $\tnab$ is
\[
(\tnab_Xs)(x)=\Big[\xi,\dsdd{\hs(b,X(t))}{t}{0}\Big],
\]
but we can modify this in order to get simpler expressions. Remark that we have an equivalence class, so that we can always choose the element of the class such that $\xi=(\mtu,x)$. We define $\dpt{\os}{\eR^2}{\Lambda W }$, $\os(v)=\hs(\mtu,v)$. Our second and final form for $\tnab$ is:
\begin{subequations}
 \begin{align}
 (\tnab_Xs)(x)&=\Big[(\mtu,x),\dsdd{\os(X(t))}{t}{0}\Big]\\\label{nabs}
              &=[(\mtu,x),X(\os)],
\end{align}
\end{subequations}
where $X(\os)$ is well defined because $\os$ is a map from $\eR^2$ into a vector space (namely: $\Lambda W $).

\subsection{Dirac operator on the euclidian \texorpdfstring{$\eR^2$}{R2}}
%///////////////////////////////////////////////////
\index{dirac!operator!on $\eR^2$}

We continue to write explicitly the definition \eqref{dirac}. Putting together \eqref{gammaX} and \eqref{nabs}, one finds
\begin{equation}
 \gamma^{\alpha}_x(\tnab_{e_{\beta}}s)(x)	=\gamma(e_{\alpha x})[\xi,e_{\beta}(\os)]
                                     		=[\xi,\tilde\rho(b^{-1}(e_{\alpha x}))e_{\beta}(\os)].
\end{equation}
Here, $e_{\beta}=\partial_{\beta}$ and $b=\mtu$, then
\[
 \gamma^{\alpha}_x(\tnab_{e_{\beta}}s)(x)=[(\mtu,x),\tilde\rho(e_{\alpha})\partial_{\beta}\os].
\]
Now, the Dirac operator reads
\[
 (\Dir s)(x)=[(\mtu,x),\gamma^{\alpha}\partial_{\alpha}\os].
\]

We can obtain a more compact expression by defining ``$Ys$''\ and ``$As$'' when $s\in\Gamma(\mS)$, $Y\in\cvec(\eR^2)$ and $A\in\End{\Lambda W }$. The definitions are
\begin{align*}
(Ys)(x)&=[(\mtu,x),(Y\os)(x)],\\
(As)(x)&=[(\mtu,x),A\os(x)].
\end{align*}
With these conventions, one writes:
\[
(\Dir s)(x)=\gamma^{\alpha}(\partial_{\alpha} s)(x).
\]
This justifies the expression \eqref{dirflat}: $\Dir=\gamma^{\alpha}\partial_{\alpha}$ on flat spaces. With a good choice of basis of $\Lambda W $, the matrices $\gamma^{\alpha}$ are given by \eqref{gammaR2}, and
\[
\gamma^{\alpha}\partial_{\alpha}=
\begin{pmatrix}
0 & -1 \\
1 & 0
\end{pmatrix}\partial_x-
\begin{pmatrix}
0 & i \\
i & 0
\end{pmatrix}\partial_y.
\] 
If we identify $\eR^2$ with $\eC$ we have the following definitions:
\[
\partial_z=\frac{1}{2}(\partial_x-i\partial_y),\qquad\partial_{\overline{z}}=\frac{1}{2}(\partial_x+i\partial_y),\]
so that
\[\Dir=\begin{pmatrix}
0 & -\partial_{\overline{z}} \\
\partial_z & 0
\end{pmatrix}.
\]
\section{Clifford algebras and Morita equivalence}
%++++++++++++++++++++++++++++++++++++++++++++++++

Let $\cA$ be an algebra. An algebra $\cB$ is said to be \defe{Morita equivalent}{Morita equivalence}\label{PgMoritaEq} to $\cA$ if $\cB=\End_{\cA}(\modE)$ for some finite projective module $\modE$ over $\cA$. The algebra $\cA$ is Morita equivalent to itself taking the trivial module $\modE=\cA$.

We consider a manifold $M$ of dimension $n=2m$.

\begin{probleme}
	The two following statements are imprecise.
\end{probleme}

\begin{proposition}
A module which implement a Morita equivalence between two $C^*$-algebras is finite projective.
\end{proposition}

\begin{theorem}[Serre-Swan]
If one of the two Morita equivalent is the continuous function space over a manifold $\cA=C(M)$, then the module which gives the Morita equivalence is the section of continuous sections of a vector bundle over $M$, $\modE=\Gamma(E)$.
\end{theorem}
Furthermore, if $\cA=C(M)$ and $\cB=\Gamma(\Cl(M))$, we have $\End E\simeq \Cl(M)$ as isomorphism of vector bundle. Since $\Cl M$ is of rank $2^n$, $\End E$ has same rank and $E_x$ has dimension $\sqrt{2^n}=2^m$. So it is possible to choose the Clifford action in such a way that $\Gamma(E)$ is an irreducible Clifford module.

We often look at an anti-linear map $J\colon \Gamma(E)\to \Gamma(E)$ such that for all $\psi\in\Gamma(E)$
\begin{enumerate}
\item $J(\psi f)=(J\psi)\overline{ f }$ for all $f\in C(M)$,
\item $J(a\psi)=\epsilon(a)a J\psi$ for all $a\in\Gamma^{\infty}(\Cl M)$.
\end{enumerate}
How to define $a\psi$ ? We consider $\cA=C(M)$, $\cB=\Gamma(\Cl M)$ and we define $\Gamma(E)$ is such a way that it implements a Morita equivalence between $\cA$ and $\cB$; hence $\Gamma(E)$ is a $C(M)$-module. From dimensional considerations, we can define on $\Gamma(E)$ a Clifford module structure, i.e. a $C(M)$-linear
\begin{equation}
  c\colon \Gamma(\Cl M)\to \End(\Gamma E),
\end{equation}
hence $a\psi$ makes sense for any $a\in\Gamma^{\infty}(\Cl M)$ and $\psi\in\Gamma(E)$ with definition
\begin{equation}
 (a\psi)(x)=\big( c(a)\psi \big)(x)
		=c(a(x))\psi(x)
\end{equation}

\begin{theorem}
Let $(M,S,J)$ be a spin manifold of dimension $n$. There exists an unique connection 
\[ 
  \nabla^S\colon \Gamma^{\infty}(S)\to \Gamma^{\infty}(S)\otimes\Omega^1(S)
\]
such that
\begin{enumerate}
\item $\scalp{ \nabla^S\psi }{ \phi }+\scalp{ \psi }{ \nabla^S\phi }=d\scalp{ \psi }{ \phi }$,
\item $[\nabla^S,J]=0$,
\item $\nabla^S\big( c(a)\psi \big)=c\big( \nabla a \big)\psi+c(a)\nabla^S\psi$ for all $a\in\Cl(M)$ and $\psi\in\Gamma^{\infty}(S)$.
\end{enumerate}
In the latter, the action of $\Gamma^{\infty}(\Cl M)$ on $\Gamma^{\infty}(S)$ is induced from the action $c\colon \Cl(T^*_xM)\to \End S$. The $\nabla$ which acts on $a$ is the connection extended to $\Gamma^{\infty}(\Cl M)$ by virtue of Leibnitz rule $\nabla(uv)=\nabla(u)v+u\nabla(c)$.

\end{theorem}

\begin{proof}
No proof
\end{proof}


In this setting, we define 
\begin{equation}
\begin{aligned}
 \hat c\colon\Gamma^{\infty}(S)\otimes\Gamma^{\infty}(\Cl M)&\to \Gamma^{\infty}(S) \\ 
 \psi\otimes a&\mapsto c(a)\psi. 
\end{aligned}
\end{equation}
Then we define the \defe{Dirac operator}{dirac!operator} $\Dir\colon \Gamma^{\infty}(S)\to \Gamma^{\infty}(S)$,
\begin{equation}
  \Dir=-i(\hat c\circ\nabla^S).
\end{equation}

\subsection{Example: quantum field theory}
%-----------------------------------------

Let us show how does this operator gives back the usual Dirac operator of quantum field theory. Let $M$ be a manifold and with two local basis $\{ \partial_u \}$ and $\{ \partial_{\alpha} \}$ of $T_xM$. The first one is the ``natural'' basis: $g(\partial_u,\partial_v)=g_{uv}$ has no particular properties while the second one is orthonormal $g(\partial_{\alpha},\partial_{\beta})=\delta_{\alpha\beta}$. The first dual basis is defined by $dx^{\alpha}\partial_{\beta}=\delta^{\alpha}_{\beta}$. 

We write $\partial_{\alpha}=e_{\alpha}^u\partial_u$ and for the dual basis, $dx^{\alpha}=e^{\alpha}_u\,dx^u$. In order these definition to be coherent, we impose $dx^{\alpha}\partial_{\beta}=\delta^{\alpha}_{\beta}$ :
\begin{equation}
  dx^{\alpha}\partial_{\beta}=e_u^{\alpha}dx^u\big( e^v_{\beta}\partial_v \big)
		=e^{\alpha}_ue^v_{\beta}\delta^u_v
		=e^{\alpha}_ue^u_{\beta}.
\end{equation}
We conclude that the \defe{vielbein}{vielbein} $(e^{\alpha}_u)$ is the inverse of $(e^v_{\beta})$: $e^{\alpha}_ue^u_{\beta}=\delta^{\alpha}_{\beta}$. The vielbein are eventually complexes.



\subsection{An other definition of the Dirac operator}
%-----------------------------------------------------

Let us consider an orthonormal basis $\{ e_a \}$ of $M$, i.e. on each $x\in M$, 
\[ 
  g_x(e_a(x),e_b(x))=\eta_{ab}.
\]
This basis is related to a ``natural'' basis $\{ \partial_{\mu} \}$ by
\begin{equation} 
  e_a=e_a^{\mu}\partial_{\mu}
\end{equation}
where  $e_a^{\mu}$ is called \defe{vielbein}{vielbein} (here, they are more precisely $n$-beins). As far as metric is concerned we have
\begin{subequations}
\begin{align}
	g^{\mu\nu}&=e_a^{\mu}e_b^{\nu}\eta_{ab}\\
	\eta_{ab}&=e_a^{\mu}e_b^{\nu}g_{\mu\nu}.
\end{align}
\end{subequations}
If $\nabla$ is the covariant derivative associated with $g$, we define the coefficients $\omega_{\mu a}^b$ by
\begin{equation}
\nabla_{\mu}e_a=\omega_{\mu a}^be_b.
\end{equation}
On the other hand, $\nabla$ is related to the Christoffel symbols by
\begin{equation}
\nabla_{\mu}\partial_{\nu}=\Gamma_{\mu\nu}^{\sigma}\partial_{\sigma}.
\end{equation}
Let $\Cl(M)$ be the Clifford module whose fibre is the Clifford complex algebra $\Cl(T^*_xM)^{\eC}$. We consider $\Gamma(\Cl(M))$, the module of corresponding sections. It gives an algebra morphism
\begin{equation}
\begin{aligned}
 \gamma\colon\Gamma(\Cl(M))&\to \opB(\hH) \\ 
dx^{\mu}&\mapsto \gamma^{\mu}(x)=\gamma^ae_a^{\mu} 
\end{aligned}
\end{equation}
which can be extended to the whole Clifford algebra. One can choose matrices $\gamma^{\mu}(x)$ and $\gamma^a$ to be hermitian; they satisfy
\begin{subequations}
\begin{align}
\gamma^{\mu}(x)\gamma^{\nu}(x)+\gamma^{\nu}(x)\gamma^{\mu}(x)&=-2g(dx^{\mu},dx^{\nu})=-2g^{\mu\nu}\\
\gamma^a\gamma^b+\gamma^b\gamma^a&=-2\eta^{ab}.
\end{align}
\end{subequations}
All this allow us to lift the Levi-Civita connection from the tangent bundle to the spinor bundle by defining
\begin{equation}
\nabla_{\mu}^S=\partial_{\mu}+\omega^S_{\mu}=\partial_{\mu}+\frac{ 1 }{2}\omega_{\mu ab}\gamma^a\gamma^b.
\end{equation}
The \defe{Dirac operator}{operator!Dirac} is then given by
\[ 
  \Dir=\gamma\circ\nabla
\]
and can locally be written under the form
\begin{equation}  \label{eq_Dirac_deux}
\Dir=\gamma^{\mu}(x)(\partial_{\mu}+\omega_{\mu}^S)
	=\gamma^ae_a^{\mu}(\partial_{\mu}+\omega^S_{\mu}).
\end{equation}
