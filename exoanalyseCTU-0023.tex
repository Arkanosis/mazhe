% This is part of Analyse Starter CTU
% Copyright (c) 2014
%   Laurent Claessens,Carlotta Donadello
% See the file fdl-1.3.txt for copying conditions.

\begin{exercice}\label{exoanalyseCTU-0023}

\begin{enumerate}
\item Donner un exemple de fonction bijective de $I$ dans $J$ pour :

    \begin{enumerate}
        \item
            
 $I=[0,1]$, $J=[1,2]$\hspace*{1.5cm} 
 \item
 $I=]0,1[$, $J=\eR$\hspace*{1.5cm} 
 \item
 $I=\eR^+$, $J=[0,1[$.
    \end{enumerate}
\item Donner un exemple d'intervalles $I$ et $J$ pour que $f$ soit  bijective de $I$ dans $J$ lorsque : 

    \begin{enumerate}
        \item
 $f(x)=x^2$\hspace*{1.5cm} 
 \item
 $f(x)=\ln(x^2)$\hspace*{1.5cm} 
    \end{enumerate}
\end{enumerate}

\corrref{analyseCTU-0023}
\end{exercice}
