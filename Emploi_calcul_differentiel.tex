Nous avons déjà étudié en détail le concept de limite pour des fonctions de plusieurs variables. La pas suivant est l'étude de la dérivabilité. Nous commencerons par la notion de dérivée directionnelle (section \ref{SecDerDirect}) avant d'étudier la différentielle d'une fonction à la section \ref{SecDifferentielle}. 

La notion de différentiabilité sera centrale pour tout le chapitre. Nous allons étudier en détail durant la section \ref{SecDifferentielle} ce que nous pouvons dire de la «dérivabilité» d'une fonction différentiable. Le lemme \ref{LemdfaSurLesPartielles} donnera le lien entre la différentielle, les dérivées partielles et le gradient d'une fonction différentiable.

Nous introduirons le gradient $\nabla f$ d'une fonction, et nous expliquerons que ce dernier donne la direction de plus grande pente. En physique, le gradient sera utilisé pour trouver la force qui s'exerce sur un mobile qui se meut dans un champ de potentiel.

La section \ref{SecPlanTangent} redeviendra plus «graphique» en introduisant une belle méthode pour trouver le plan tangent au graphe d'une fonction. Ici encore le gradient jouera un rôle central.

Avec la section \ref{SecPropDiffs}, nous retournons aux techniques de calcul. Nous y verrons les formules nécessaires pour calculer, en pratique, la différentielle d'une fonction. En particulier, nous verrons comment on trouve la différentielle d'un produit de fonction, ou de composées de fonctions.

Le théorème de accroissements finis sera vu dans la section \ref{SecThoAccrsFinis}. Ce théorème généralise les théorèmes de Rolle et des accroissements finis pour les fonctions à plusieurs variables.

Enfin nous clôturerons le chapitre avec quelque remarques sur les différentielles d'ordre supérieures à la section \ref{SecDiffOrdSup}. Nous introduisons la matrice hessienne qui joue le rôle de dérivée seconde pour la recherche des extrema d'une fonction. Nous n'aurons hélas pas le temps d'aborder cette problématique.

La section \ref{SecDerDirFnComp} ne fait pas partie de la matière, à part le théorème \ref{ThoDerDirFnComp} qu'il faut pouvoir utiliser dans les exercices.

