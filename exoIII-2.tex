% This is part of the Exercices et corrigés de mathématique générale.
% Copyright (C) 2009
%   Laurent Claessens
% See the file fdl-1.3.txt for copying conditions.
\begin{exercice}\label{exo2}

Une population de bactéries croît selon la loi $p(t)=a 2^{bt}$. Au temps $t=2$, on mesure $p(2)=640$, et au temps $t=3$, on mesure $p(3)=5120$.
\begin{enumerate}

\item
 Calculer $a$ et $b$. 

\item
En combien de temps la population est-elle doublée ?

\end{enumerate}

\corrref{2}
\end{exercice}
