% This is part of (almost) Everything I know in mathematics
% Copyright (c) 2013-2014
%   Laurent Claessens
% See the file fdl-1.3.txt for copying conditions.

\section{Statement of some results}
%+++++++++++++++++++++++++++++++++

We provide here some results that are at undergrad level. Some of them are contained in the part «Pour l'agrégation» in French.

\begin{lemma}[Schur's lemma] \label{lem:Schur}
If $\dpt{\phi}{\lG}{\gl(V)}$ is irreducible, then the only endomorphism of $V$ which commutes with all $\phi(\lG)$ are multiples of identity.
\end{lemma}
%TODO : look the link with the Schur'lemma given in Agregation.

\begin{theorem}[Cayley-Hamilton]\index{Cayley-Hamilton theorem} \label{ThoCayleyHamilton}
    A square matrix on \( \eR\) or \( \eC\) satisfies its own characteristic equation. If \( A\in\eM_n(\eK)\), we consider the polynomial \( p(\lambda)=\det(A-\lambda\mtu)\) in \( \lambda\). Thus \( p(A)=0\).
\end{theorem}
For a proof see \wikipedia{en}{Cayley–Hamilton_theorem}{Wikipedia}.

\begin{proposition}[\cite{SerreMatrices}]     \label{PropMtrDiagablaUnit}
    If \( M\) is a complex \( n\times n\) matrix, then there exists an unitary matrix \( U\) such that \( U^*MU\) is upper triangular.
\end{proposition}

\begin{definition}
    A \defe{positive defined}{positive!defined matrix} matrix is a matrix $B$ such that
    \begin{equation}
        \sum_{ij}B_{ij}\overline{ x_i }x_j
    \end{equation}
    is real and positive for every complex vector $x$.
\end{definition}

\begin{proposition}
    A positive defined matrix is Hermitian.
\end{proposition}

\begin{proof}
    We define the Hermitian matrices $M=(B+B^*)/2$ and $N=(B-B^*)/2i$, so $B=M+iN$ and
    \begin{equation}
        \bar x Bx=\bar x M x+i\bar x Nx.
    \end{equation}
    The matrices $M$ and $B$ being Hermitian, the numbers $\bar xMx$ and $\bar xNx$ are real. If $\bar xBx$ has to be real, we need $\bar xNx=0$ for every $x$. This shows that $N=0$, so that $B=N$.
\end{proof}

\begin{theorem}
Let $G$ be a Lie group and $H$ a subgroup (with no special other structures) of $G$. If $H$ is a closed subset of $G$ then there exists an unique analytic structure on $H$ such that $H$ is a topological Lie subgroup of $G$.
\label{Helgason2.3}
\end{theorem}
This comes from \cite{Helgason}, chapter 2, theorem 2.3.

\begin{lemma}
Let $G$ be a connected Lie group with Lie algebra $\mG$ and let $\varphi$ be an analytic homomorphism of $G$ into a Lie group $X$ with Lie algebra $\mathcal{X}$. Then

\begin{enumerate}
\item The kernel $\varphi^{-1}(e)$ is a topological Lie subgroup of $G$. Its Lie algebra is the kernel of $d\varphi_e$.
\item The image $\varphi(G)$ is a Lie subgroup of $\mathcal{X}$ with Lie algebra $d\varphi(\mG)\subset\mathcal{X}$.
\end{enumerate}
\label{Helgason5.1}
\end{lemma}
This comes from \cite{Helgason}, chapter 2, lemma 5.1.

\begin{lemma}
Let $G$ and $H$ be two Lie group, whose Lie algebra are $\mG$ and $\mH$. If $\dpt{\theta}{G}{H}$ is a surjective map, then we have $\mH\simeq\mG/Ker\,d\theta_e$.
\label{1203r1}
\end{lemma}

\begin{theorem} \label{1503t1}
Let us consider $\dpt{Ad}{SU(2)}{GL(3)}$, $Ad(U)X=UXU^{-1}$. We have the following properties:

\begin{enumerate}
\item $Ad$ is a linear homomorphism,
\item it takes his values in $\SO(3)$; then we can write $\dpt{Ad}{SU(2)}{\SO(3)}$,
\item it is surjective,
\item $Ker\,Ad=\eZ_2$,
\item all these properties show that \[\SO(3)=\frac{SU(2)}{\eZ_2}.\]
\end{enumerate}
\end{theorem}

\begin{definition}
If $(a_k)$ is a sequence in $\eR$, its \defe{upper limit}{upper limit} is the real number
\[
  \lim\sup_{n\to\infty}a_n=\lim_{l\to\infty}\sup\{a_k:k\geq l\}.
\]
\end{definition}

\begin{lemma}
If $\omega$ is a $k$-form (not specially a symplectic one), and $\nabla$ a torsion free connection, one has
\begin{equation}\label{eq:d_omega_nabla}
  (d\omega)(X_0,\ldots,X_k)=\sum_{i=0}^k (-1)^i(\nabla_{X_i}\omega)(X_0,\ldots,\hX_i,\ldots X_k).
\end{equation}
\end{lemma}

\begin{remark}
The link between $d$ and $\nabla$ comes from the fact that in the left hand side of \eqref{eq:d_omega_nabla} appears some commutators $[X_i,X_j]$, but since the connection is torsion-free,
\[
  [X_i,X_j]=\nabla_{X_i}X_j-\nabla_{X_j}X_i
\]
\end{remark}
The main consequence of this lemma is that $\nabla\omega=0$ implies $d\omega=0$. 

\begin{proposition} \label{prop:fdefint}
    Consider a function $\dpt{f}{X\times E}{\overline{\eR}}$ and $z_0\in E$ such that
    \begin{itemize}
        \item for all $z\in E$, the function $x\to(x,z)$ is integrable,
        \item for (almost) all $x\in X$, the function $z\to f(x,z)$ is continuous at $z_0$,
        \item there exists a function $g\geq 0$ such that for all $z\in E$, $| f(x,z) |\leq g(x)$ almost everywhere in $X$.
    \end{itemize}
    Then the function $\dpt{h}{E}{\eR}$ defined by $h(z)=\int_Xf(x,z)$ is continuous at $z_0$.
\end{proposition}
This proposition is (up to ``almost'') the theorem \ref{ThoKnuSNd}.

% \begin{definition}

% If $(a_k)$ is a sequence in $\eR$, its \defe{upper limit}{upper limit} is the real number
% \[
%   \lim\sup_{n\to\infty}a_n=\lim_{l\to\infty}\sup\{a_k:k\geq l\}.
% \]
% \end{definition}
% 
% \begin{proposition}\label{prop:cv_lim_sup}
% If $(a_k)$ is a sequence in $\eR$ such that there exists a $a\in\eR$ for which for any $k\in\eN$,
% \[
% \lim\sup_{n\to\infty}a_n\leq a\leq a_k
% \]
% then $(a_k)$ admits a limit and $\lim_{n\to\infty}a_k=a$.
% \end{proposition}
% 
% \begin{lemma}
% If $\omega$ is a $k$-form (not specially a symplectic one), and $\nabla$ a torsion free connection, one has
% 
% \begin{equation}\label{eq:d_omega_nabla}
%   (d\omega)(X_0,\ldots,X_k)=\sum_{i=0}^k (-1)^i(\nabla_{X_i}\omega)(X_0,\ldots,\hX_i,\ldots X_k).
% \end{equation}
% \end{lemma}
% 
% \begin{remark}
% The link between $d$ and $\nabla$ comes from the fact that in the left hand side of \eqref{eq:d_omega_nabla} appears some commutators $[X_i,X_j]$, but since the connection is torsion-free,
% \[
%   [X_i,X_j]=\nabla_{X_i}X_j-\nabla_{X_j}X_i
% \]
% \end{remark}
% The main consequence of this lemma is that $\nabla\omega=0$ implies $d\omega=0$. 
