% This is part of Exercices et corrigés de CdI-1
% Copyright (c) 2011
%   Laurent Claessens
% See the file fdl-1.3.txt for copying conditions.

\begin{exercice}\label{exo0047}


Prouvez que la fonction suivante $f$ est continue en $(0,0)$, que
toutes les dérivées directionnelles en $(0,0)$ de cette fonction
existent mais qu'elle n'est pas différentiable en $(0,0)$ car
l'application qui envoie un vecteur $u$ sur la dérivée de la fonction
$f$ dans la direction $u$ en $(0,0)$ n'est pas linéaire.
\[
f:(x,y) \rightarrow \left\{ \begin{array}{ll}
x & xy>0 \\
y & xy  \leq 0
\end{array} \right.
\]


\corrref{0047}
\end{exercice}
