% This is part of Exercices et corrigés de CdI-1
% Copyright (c) 2011
%   Laurent Claessens
% See the file fdl-1.3.txt for copying conditions.

\begin{corrige}{EqsDiff0007}

\begin{enumerate}

\item
$y''-2y'= e^{t}\sin(t)$. Un système fondamental pour l'équation homogène est donné par
\begin{equation}
	\left\{
\begin{array}{ll}
y_1=1\\
y_2= e^{2t}.
\end{array}
\right.
\end{equation}
Étant donné que le second membre contient $e^t$, il est naturel de mettre $e^t$ dans un essai de solution particulière. Ensuite, la présence de $\sin(t)$ nous incite à mettre $A\cos(t)+B\sin(t)$. Donc nous essayons
\begin{equation}
		y_P(t)= e^{t}\big( A\cos(t)+B\sin(t) \big).
\end{equation}
Un petit calcul de dérivation avec cette fonction montre que
\begin{equation}
	\begin{aligned}[]
		y''_P-2y'_P&- e^{t}\sin(t)\\
			&=e^t\big( \sin(t)(-2B-2(A-B)-1)+\cos(t)(2A-2(B+A)) \big),
	\end{aligned}
\end{equation}
que nous devons égaler à zéro pour trouver $A$ et $B$. Le système à résoudre est
\begin{equation}
	\left\{
\begin{array}{ll}
-2B-2(A-B)-1=0\\
2A-2(B+A)=0,
\end{array}
\right.
\end{equation}
et la solution est $A=-\frac{ 1 }{2}$, $B=0$. Nous avons donc pour solution particulière
\begin{equation}
	y_P(t)=-\frac{ 1 }{2}e^t\sin(t).
\end{equation}

\item
$y''+y=-2\sin(t)+4t\cos(t)$. L'équation caractéristique est $x^2+1=0$, et donc $x=\pm i$. Un système fondamental de solutions réelles est donné par
\begin{equation}
	\left\{
\begin{array}{ll}
y_1=\cos(t)\\
y_2=\sin(t).
\end{array}
\right.
\end{equation}
Nous notons $P$ l'opérateur $f\mapsto P(x)$ défini par
\begin{equation}
	P(f)(x)=f''(x)+f(x)+2\sin(x)-4x\cos(x).
\end{equation}
La fonction $y_P(x)$ est solution particulière de l'équation si $P(y_P)=0$. Le second membre contenant du sinus et du cosinus, nous allons essayer une solution qui sera une combinaison de sinus et cosinus. Au niveau des coefficients, nous allons mettre des polynômes. Nous essayons
\begin{equation}
	y_P(x)=(ax^3+bx^2+cx+d)\cos(x)+(a'x^3+b'x^2+c'x+d)\sin(x).
\end{equation}
Les termes $d$ et $d'$ correspondent à $\cos(x)$ et $\sin(x)$, qui sont solutions de l'homogène. Ils ne vont donc pas contribuer à obtenir le second membre; nous pouvons donc directement les oublier et essayer
\begin{equation}
	y_P(x)=(ax^3+bx^2+cx)\cos(x)+(a'x^3+b'x^2+c'x)\sin(x).
\end{equation}
Le calcul montre que
\begin{equation}
	\begin{aligned}[]
		P\big( y_P(x) \big)&=\big( -6ax^2+(6a'-4b)x-2c+2b'+2 \big)\sin(x)\\
				&\quad+\big( 6a'x^2+(4b'+6a-4)x+2c'+2b \big)\cos(x).
	\end{aligned}
\end{equation}
Afin que cela soit zéro, il faut $a=a'=0$, $b=0$, $b'=1$, $c'=0$ et $b'=1$. Une solution particulière de l'équation non homogène est donc donnée par
\begin{equation}
	y_P(x)=2x\cos(x)+x^2\sin(x).
\end{equation}

\item
$y''-y=e^t+2$. Un système fondamental pour l'homogène est
\begin{equation}
	\left\{
\begin{array}{ll}
y_1=e^x\\
y_2= e^{-x}.
\end{array}
\right.
\end{equation}
Étant donné que $e^x$ est déjà une solution de l'homogène, nous ne mettons pas $e^x$ dans un essai de solution particulière, mais nous essayons plutôt $axe^x$, donc nous essayons
\begin{equation}
	y_P(x)=axe^x-2.
\end{equation}
Le $-2$ est là pour obtenir le $2$ du membre de droite. En remettant dans l'équation, nous trouvons $(2a-1)e^x$, qui s'annule pour $a=1/2$, donc
\begin{equation}
	y_P(x)=\frac{ x }{ 2 }e^x-2,
\end{equation}
et la solution générale de l'équation recherchée est
\begin{equation}
	y(x)=Ae^x+Be^x+\frac{ x }{ 2 }e^x-2.
\end{equation}
En ce qui concerne le problème de Cauchy, nous devons calculer la dérivée :
\begin{equation}
	y'(3)=-B e^{-3}+A e^{3}+\frac{ 3 }{ 2 }e^3+\frac{ 1 }{2}e^3,
\end{equation}
et nous voyons que les conditions imposent 
\begin{equation}
	\begin{aligned}[]
		A&=-\frac{ e^{-3}(7e^3-6) }{ 4 },&B&=\frac{ e^6+2e^3 }{ 4 }.
	\end{aligned}
\end{equation}

\item
$y''-2y'+3y=t^3+\sin(x)$. Un système fondamental de solutions réelles est
\begin{equation}
	\left\{
\begin{array}{ll}
y_1(x)=e^x\cos(\sqrt{2}x)\\
y_2(x)=e^x\sin(\sqrt{2}x)\\
\end{array}
\right.
\end{equation}
Nous allons chercher une solution particulière en deux parties. Une pour obtenir $x^3$ et une pour obtenir $\sin(x)$. Afin d'obtenir $x^3$, nous essayons un polynôme :
\begin{equation}
	\begin{aligned}[]
		y_{P1}=ax^3+bx^2+cx+d\\
		y'_{P1}=3ax^2+2bx+c\\
		y''_{P1}=6ax+2b
	\end{aligned}
\end{equation}
Afin que $y_{P1}''-2y'_{P1}+3y_{P1}=x^3$, nous devons choisir
\begin{equation}
	\begin{aligned}[]
		a&=\frac{1}{ 3 },&b&=\frac{ 2 }{ 3 },&c&=\frac{ 2 }{ 9 },&d&=-\frac{ 8 }{ 27 }.
	\end{aligned}
\end{equation}
Nous cherchons le second morceau de telle façon à avoir $y_{P2}''-2y'_{P2}+3y_{P2}=\sin(x)$. Pour cela, nous regardons $y_{P2}(x)=A\cos(x)+B\sin(x)$ et il est vite vu que $A=B=\frac{1}{ 4 }$. La solution particulière ainsi construite est
\begin{equation}
	y_P(x)=\frac{ x^3 }{ 3 }+\frac{ 2x^2 }{ 3 }+\frac{ 2x }{ 9 }-\frac{ 8 }{ 27 }+\frac{1}{ 4 }\cos(x)+\frac{1}{ 4 }\sin(x).
\end{equation}

\end{enumerate}

\end{corrige}
