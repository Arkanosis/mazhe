% This is part of Analyse Starter CTU
% Copyright (c) 2014
%   Laurent Claessens,Carlotta Donadello
% See the file fdl-1.3.txt for copying conditions.

%+++++++++++++++++++++++++++++++++++++++++++++++++++++++++++++++++++++++++++++++++++++++++++++++++++++++++++++++++++++++++++ 
\section{Équations différentielles du premier ordre à variables séparables}
%+++++++++++++++++++++++++++++++++++++++++++++++++++++++++++++++++++++++++++++++++++++++++++++++++++++++++++++++++++++++++++

Pour certaines équations différentielles la recherche d'une solution particulière se réduit à une recherche de primitive moyennant un changement de variables. 
\begin{definition}[Équation différentielle du premier ordre \`a variables separables]
Une  \defe{équation différentielle du premier ordre à variables séparables}{équation différentielle!variables séparables} est une équation qui, pour tout les \(x\) dans un intervalle donné, \(I\), peut se mettre sous la forme
\begin{equation}\label{eq_var_sep}
  f(y)y' = g(x),
\end{equation}
o\`u \(f\) et \(g\) sont deux fonction de \(\eR\) dans \(\eR\).
\end{definition}
Nous pouvons intégrer les deux cotes de l'égalité par rapport à \(x\) et obtenir 
\[
  \int f(y(x))y'(x)\, dx = G(x)+C,
\]
o\`u $G$ est une primitive de $g$ et $C$ une constante réelle. Il est facile \`a ce point d'effectuer une changement de variable dans le membre de gauche de l'équation en posant (sans surprise) \(y= y(x)\) et donc \(y'(x)\,dx = dy\). 
\[
  \int f(y(x))y'(x)\, dx =  \int f(y)\, dy  = F(y(x)) + C ,
\]
o\`u $F$ est une primitive de $f$ et $C$ une constante réelle. En somme nous avons
\[
  F(y(x)) = G(x) + C ,
\]
et, si $F$ admet une fonction réciproque, alors 
\begin{equation}
  y(x) = F^{-1} (G(x)+C).
\end{equation}
\begin{remark}
  L'expression de $F^{-1} $ peut être difficile à calculer. Il sera alors préférable de garder $y$ dans la forme implicite. 
\end{remark}

\begin{example}
  L'équation
  \begin{equation}\label{ex_un_var_sep}
    3y^2 y' = x, \qquad\text{pour tout } x\in\eR,
  \end{equation}
est une équation à variables séparables. Pour reprendre les notations du début du chapitre, ici \(f(y) = 3y^2\) et \(g(x) = x\). En intégrant de deux cotes on trouve 
\[
y^3 = \frac{x^2}{2} + C .
\]
La fonction $F(y) = y^3$ est une bijection de $\eR$ dans $\eR$, donc nous pouvons écrire la solution générale de l'équation \eqref{ex_un_var_sep} dans la forme 
\[
\mathcal{Y} =\left\{ \left(\frac{x^2}{2} + C \right)^{1/3} \:\text{tel que } C\in\eR\right\}. 
\]
\end{example}

\begin{example}
  En intégrant de deux cotes l'équation à variables séparables
  \begin{equation}
    2y y' = x, \text{pour tout } x\in\eR,
  \end{equation}
  on trouve 
  \[
  y^2 = \frac{x^2}{2} + C .
  \]
La fonction $F(y) = y^2$ est \emph{n'est pas inversible} sur tout $\eR$, et on sait que \(\sqrt{y^2} = |y|\).  Au moment de rendre $y$ explicite on doit choisir entre  
\[
y = \left(\frac{x^2}{2} + C \right)^{1/2}\qquad \text{ou}\qquad y = -\left(\frac{x^2}{2} + C \right)^{1/2} .
\]
Ce choix se fait suivant la condition initiale, si elle est donnée. S'il n'y a pas de condition initiale nous pouvons écrire que la solution générale est l'ensemble
\[
\mathcal{Y} =\left\{ y \,:\, \eR \to \eR\:\text{tels que }  y^2 = \frac{x^2}{2} + C \:\text{et } C\in\eR\right\}. 
\]
\end{example}

\begin{example}
  On considère le problème de Cauchy
  \begin{equation}
    \begin{cases}
      e^y y' = \frac{1}{x+3}, \quad x\in ]-\infty, -3[,\\
      y(-4) = 0.
    \end{cases}
  \end{equation}
En intégrant des deux cotes nous trouvons
\[
e^y = \ln(|x+3|) +C.
\]
Nous pouvons alors imposer la condition initiale et obtenir $e^{0} =\ln(|-4+3|) +C $, c'est \`a dire $C = 1- \ln(1) = 1$.
\begin{remark}
  L'énoncé du problème de Cauchy dit que $x$ peut varier dans \(]-\infty, -3[\), mais nous voyons maintenant que la solution n'est pas définie sur toute la demi-droite, parce que $e^y$ est toujours positif  et $\ln(|x+3|) +1$ est positif seulement pour $x < -(1/e + 3)\approx -3,3679$. 
\end{remark}
Donc la solution du problème de Cauchy est \(y(x) = \ln(|x+3|) +1\) pour tout $x\in ]-\infty, -(1/e +3)[$.
\end{example}

\begin{example}\label{exemple_eq_hom}
  \textbf{Attention, cet exemple est le plus important de la section !}

On considère l'équation à variables séparables 
\begin{equation}    \label{EqYNXooFzYeZS}
  y' = \sin(x) y , \qquad x\in\eR.
\end{equation}
Dans ce cas, pour pouvoir écrire l'équation dans la forme \eqref{eq_var_sep} il faut pouvoir multiplier les deux c\^otés par $1/y$. Il faut donc éliminer tout de suite le cas o\`u $y = 0$. 

Si $y= 0$ alors $y' =0$ et on a une solution constante (on dit souvent : une solution stationnaire) de l'équation. Par ailleurs les trajectoires des solutions ne peuvent pas se croiser; donc si \( y_G\) est une solution non nulle de l'équation \eqref{EqYNXooFzYeZS} alors \( y_G(x)\neq 0\) pour tout \( x\)\footnote{Ça vaut la peine de prendre un peu de temps pour bien comprendre cela.}. Il n'y a donc aucun danger à diviser par \( y\) dans la recherche d'une solution non identiquement nulle.

Supposons maintenant que $y\neq 0$ et écrivons $y'/y = \sin(x)$. En intégrant des deux cotes on trouve 
\[
  \ln(|y|) =- \cos(x) +C, 
\]
d'où
\[
|y| = e^{- \cos(x) +C}= e^{C}e^{- \cos(x)}.
\]
Si on avait impose une condition initiale alors on pourrait déterminer une solution particulière de l'équation en choisissant une valeur de la constante $C$. Nous pouvons observer cependant que la fonction exponentielle est bijective de $\eR$ dans $\eR^{+,\star}$ et par conséquent il n'y a pas de perte de généralité en disant que la solution générale de l'équation est 
\begin{equation*}
  \mathcal{Y} = \left\{ y \,:\, |y| = Ke^{- \cos(x)}, \:\text{pour } K\in\eR^{+,\star}\right\}\cup\{y\equiv 0\}.
\end{equation*}
Il n'empêche qu'il serait plus élégant d'écrire la solution générale de l'équation sous une forme plus explicite, sans valeur absolue. Nous pouvons le faire en nous nous rappelant que 
\begin{equation*}
 |x| =  \begin{cases}
    x & \quad\text{si } x \geq 0 ,\\
    -x & \quad\text{si } x <0 ,
  \end{cases}
\end{equation*}
Il suffit alors d'autoriser $K$ dans $\eR^{\star}$ pour éliminer la valeur absolue. 

Pour écrire la solution générale de façon encore plus compacte nous observons que si $K=0$ alors $y \equiv 0$, c'est \`a dire, on retrouve la solution constante nulle. 

Finalement, la solution générale de cette équation sera toujours écrite sous la forme suivante
\begin{equation}
  \mathcal{Y} = \left\{ y = Ke^{- \cos(x)}, \:\text{pour } K\in\eR\right\}.
\end{equation}
\end{example}

