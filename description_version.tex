
\thispagestyle{empty}

Ce document existe en plusieurs versions.
\begin{description}

    \item[La version courante] 

        Vous trouverez une version dédiée à l'agrégation régulièrement mise à jour à l'adresse suivante :
        \begin{center}
            \url{http://laurent.claessens-donadello.eu/pdf/lefrido.pdf}
        \end{center}

    \item[Préliminaire de ce qui sera commercialisé]

        Le jury d'agrégation veut uniquement des ressources commercialisées ? Eh bien, le Frido sera prochainement commercialisé en deux volumes sur \href{http://www.thebookedition.com/fr/}{thebookedition.com}
        \begin{center}
            \url{http://laurent.claessens-donadello.eu/pdf/préliminaire-thebookedition_vol1.pdf}\\
            \url{http://laurent.claessens-donadello.eu/pdf/préliminaire-thebookedition_vol2.pdf}
        \end{center}
        
    \item[Pour les étudiants]

        Un texte ne contenant que (et tout) ce qui est destiné aux étudiants que j'ai eu (biologistes, agronomes, physiciens, etc) :
        \begin{center}
        \url{http://laurent.claessens-donadello.eu/pdf/enseignement.pdf}
        \end{center}

    \item[La version la plus complète]

        Une version plus complète, comprenant à la fois les deux autres et de la mathématique de niveau recherche : 
        \begin{center}
        \url{http://laurent.claessens-donadello.eu/pdf/mazhe.pdf}
        \end{center}

    \item[Tout ce qu'il faut savoir pour recompiler soi-même]
        Pour savoir comment recompiler ce document à l'identique, il faut lire
        \begin{center}
            \url{http://laurent.claessens-donadello.eu/pdf/readme.pdf}
        \end{center}

\end{description}


\vfill

\LogoEtLicence
\clearpage
