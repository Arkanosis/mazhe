% This is part of Exercices et corrigés de CdI-1
% Copyright (c) 2011
%   Laurent Claessens
% See the file fdl-1.3.txt for copying conditions.

\begin{corrige}{0041}

\begin{enumerate}

\item 
$f(x)=\sqrt{x^2+y^2}$. Les courbes de niveau $f(x)=C$ sont des cercles (sauf $f(x)=0$ qui se réduit à un point). Les sections horizontales étant des cercles, et le rayon de ces cercles augmentant linéairement, le graphe est une cône. Nous pouvons nous en convaincre en vérifiant par exemple que la droite $t\mapsto(t,0,t)$ est bien entièrement contenue dans $z=f(x,y)$. 

Afin de déterminer la différentielle, nous calculons les dérivées partielles
\begin{equation}		\label{EqDerrPaert0041x}
	\frac{ \partial f }{ \partial x }=\frac{1}{ 2 }(x^2+y^2)^{-1/2}\cdot 2x=\frac{ x }{ \sqrt{x^2+y^2} },
\end{equation}
et
\begin{equation}		\label{EqDerrPart0041y}
	\frac{ \partial f }{ \partial y }=\frac{ y }{ \sqrt{x^2+y^2} }.
\end{equation}

Pour le plan tangent, nous essayons d'utiliser la formule \eqref{EqPlanTgSansNabla} ou bien \eqref{EqPlanTgEnDimDeux}. Pour cela, nous devons trouver les dérivées partielles en zéro. 

En vertu de la remarque \ref{Rem0041DiffExCon} ci après, il ne suffit pas de calculer les limites de \ref{EqDerrPaert0041x} et de \ref{EqDerrPart0041y} pour trouver la différentielle de $f$ en $(0,0)$. Il n'est par contre pas très compliqué de remarquer que les dérivées partielles n'existent pas en $(0,0)$, par exemple parce que
\begin{equation}
	\lim_{t\to 0}\frac{ f(t,0)-f(0,0) }{ t }
\end{equation}
n'existe pas pour cause de limite différente pour $t>0$ et $t<0$. Il n'y a donc pas de plan tangent.  Ceci est conforme à l'intuition : il n'y a pas de plan tangent à un cône en son sommet.

Nous pouvons faire une petite vérification du fait que le graphe est bien un cône : la droite reliant $(0,0,0)$ à $(x,y,\sqrt{x^2+y^2})$ est entièrement contenue dans le graphe de $f$. En effet si nous posons
\begin{equation}
	\gamma(t)=(tx,ty,t\sqrt{x^2+y^2}),
\end{equation}
pour tout $t$, nous avons $\gamma_z(t)=f\big( \gamma_x(t)^2+\gamma_y(t)^2 \big)$.

\begin{remark}	\label{Rem0041DiffExCon}
Il est vite vu que la formule \eqref{EqDerrPaert0041x} n'a pas de limite pour $(x,y)\to(0,0)$. Ceci \emph{ne prouve pas} que la différentielle de $f$ n'existe pas en $(0,0)$. L'existence de la différentielle implique la continuité de la fonction, et non de la différentielle elle-même. En effet, une différentielle peut exister en un point sans qu'elle soit la limite de la différentielle aux autres points. Nous avons vu par exemple, dans l'exercice \ref{exo0035}\ref{Item0035d}, un exemple de fonction dérivable\footnote{Pour rappel, en dimension un, la dérivée est \emph{exactement} la notion de différentielle.} en $0$, mais dont la dérivée n'est pas continue en zéro.
\end{remark}


\item
$f(x,y)=\sqrt{1-x^2-y^2}$. Les courbes de niveau $f(x,y)=C$ n'existent que pour $C\leq 1$, et ce sont des cercles
\begin{equation}
	x^2+y^2=1-C.
\end{equation}
Cette fois, le graphe est une coupole ce sphère. Nous allons en effet vérifier que l'arc de cercle centré en $(0,0,0)$ joignant se sommet $(0,0,1)$ à $(1,0,0)$ dans le plan $y=0$ est entièrement contenu dans le graphe de $f$. La symétrie de $f$ sous les rotations dans le plan $x-y$ fait le reste. L'arc de cercle en question est le chemin
\begin{equation}
	\gamma(t)=\big( 1-t,0,\sqrt{1-(1-t)^2} \big).
\end{equation}
Chaque point de ce chemin vérifie bien la relation
\begin{equation}
	f\big( \gamma_x(t),\gamma_y(t) \big)=\gamma_z(t).
\end{equation}

Le plan tangent à la coupole de sphère en $(0,0,1)$ est évidement horizontal. Nous nous attendons donc à trouver que la différentielle de $f$ en $(0,0)$ est nulle. Simple calcul :
\begin{equation}
	\frac{ \partial f }{ \partial x }(x,y)=\frac{ 1 }{2}\frac{ -2x }{ \sqrt{1-x^2-y^2} },
\end{equation}
et 
\begin{equation}
	\frac{ \partial f }{ \partial y }(x,y)=\frac{ 1 }{2}\frac{ -2y }{ \sqrt{1-x^2-y^2} }.
\end{equation}
Évaluées en $(0,0)$, ce deux dérivées partielles sont nulles. Donc \emph{si la différentielle existe} en $(0,0)$, elle sera nulle (voir l'expression \eqref{EqDiffPartRap}). Afin de voir qu'elle existe, il faut juste montrer que $df_{(0,0)}(x,y)=0$ fonctionne dans la définition \ref{DefDifferentiablFnRn}.

\item
$f(x,y)=(x^2+y^2)^2-8xy$. La courbe de niveau zéro, en coordonnées polaire est donnée par
\begin{equation}
	r=2\sqrt{\sin(2\theta)}.
\end{equation}
Les dérivées partielles sont données par
\begin{equation}
	\begin{aligned}[]
		\frac{ \partial f }{ \partial x }(x,y)	&=4(x^2+y^2)x-8y\\
		\frac{ \partial f }{ \partial y }(x,y)	&=4(x^2+y^2)y-8x
	\end{aligned}
\end{equation}

\end{enumerate}

\end{corrige}
