% This is part of the Exercices et corrigés de mathématique générale.
% Copyright (C) 2009
%   Laurent Claessens
% See the file fdl-1.3.txt for copying conditions.
\begin{corrige}{Lineraire0033}

	Si $v$ est une vecteur propre de valeur propre $\lambda$, c'est que l'équation suivante est vérifiée :
	\begin{equation}
		Av=\lambda v.
	\end{equation}
	\begin{enumerate}

		\item
			Calculons $(\alpha A)v=\alpha(Av)=\alpha\lambda v$, donc $v$ est vecteur propre de $\alpha A$ pour la valeur $\alpha\lambda$.

		\item
			Nous avons
			\begin{equation}
				A^2v=AAv=A\lambda v=\lambda Av=\lambda\lambda v=\lambda^2v.
			\end{equation}
			Le vecteur $v$ est donc vecteur propre de $A^2$, de valeur propre $\lambda$.

		\item Essayons de voir si nous avons une égalité $A^{-1}v=\mu v$ pour un certain $\mu$. En appliquant $A$ des deux côtés, nous avons $v=\mu Av=\mu\lambda v$, donc nous avons $\mu=\frac{1}{ \lambda }$.

	\end{enumerate}
	Noter que $\lambda=0$ n'est pas possible parce qu'on a supposé que $A$ est inversible.

\end{corrige}
