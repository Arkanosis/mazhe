% This is part of the Exercices et corrigés de mathématique générale.
% Copyright (C) 2009
%   Laurent Claessens
% See the file fdl-1.3.txt for copying conditions.
\begin{corrige}{TP50003}

	Une bonne idée est de lire le point \ref{SubSecMtrSym} avant de se lancer dans cet exercice. 
	\begin{enumerate}

		\item
			Nous savons que le vecteur 
			\begin{equation}
				f_1=\begin{pmatrix}
					1	\\ 
					1	\\ 
					1	
				\end{pmatrix}
			\end{equation}
			est sur la droite, et donc ne va pas bouger. Ce vecteur est donc un vecteur propre de valeur propre $1$. Les vecteurs perpendiculaires à la droite vont changer de signe. Il n'est pas très difficile d'en trouver deux linéairement indépendants. Un premier qui est perpendiculaire à $f_1$ est par exemple
			\begin{equation}
				f_2=\begin{pmatrix}
					1	\\ 
					0	\\ 
					-1	
				\end{pmatrix}.
			\end{equation}
			Pour vérifier qu'il est perpendiculaire, penser au produit scalaire. Pour en trouver $f_3$ qui est à la fois perpendiculaire à la droite et à $f_2$, prenons le produit vectoriel :
			\begin{equation}
				f_3=f_1\times f_2=\begin{vmatrix}
					e_1	&	e_2	&	e_3	\\
					1	&	1	&	1	\\
					1	&	0	&	-1
				\end{vmatrix}=-e_1+2e_2-e3=\begin{pmatrix}
					-1	\\ 
					2	\\ 
					-1	
				\end{pmatrix}.
			\end{equation}
			Par définition, si $A$ est la rotation demandée, nous avons
			\begin{equation}
				\begin{aligned}[]
					Af_1&=f_1\\
					Af_2&=-f_2\\
					Af_3&=-f_3\\
				\end{aligned}
			\end{equation}
			Ce sont donc tous les trois des vecteurs propres, de valeurs propres respectivement $1$, $-1$ et $-1$.

			Au cas où vous voudriez savoir la matrice de la rotation, il faut d'abord exprimer $e_1$, $e_2$ et $e_3$ en termes de $f_1$, $f_2$ et $f_3$ :
			\begin{equation}
				\begin{aligned}[]
					e_1&=\frac{1}{ 3 }f_1+\frac{1}{ 2 }f_2-\frac{1}{ 6 }f_3\\
					e_2&=\frac{1}{ 3 }f_1+\frac{1}{ 3 }f_3\\
					e_3&=\frac{1}{ 3 }f_1-\frac{1}{ 2 }f_2-\frac{1}{ 6 }f_3
				\end{aligned}
			\end{equation}
			Maintenant, trouver $Ae_1$, $Ae_2$ et $Ae_3$ revient à changer quelque signes dans ces équation, et nous trouvons
			\begin{equation}
				A=\frac{1}{ 3 }\begin{pmatrix}
					-1	&	2	&	2	\\
					2	&	-1	&	2	\\
					2	&	2	&	-1
				\end{pmatrix}.
			\end{equation}
			Vous pouvez même vous amuser à trouver les valeurs propres et vecteurs propres de cette matrice (vous les savez déjà !).

		\item
			La rotation de $90$ degrés autour de $Ox$ laisse $e_x$ invariant, puis fait $e_y\to e_z$ et $e_z\to -e_y$ (ne pas oublier ce signe, faire un dessin !). Nous avons donc
			\begin{equation}
				\begin{aligned}[]
					A\begin{pmatrix}
						1	\\ 
						0	\\ 
						0	
					\end{pmatrix}&=\begin{pmatrix}
						1	\\ 
						0	\\ 
						0	
					\end{pmatrix},\\
					A\begin{pmatrix}
						0	\\ 
						1	\\ 
						0	
					\end{pmatrix}&=\begin{pmatrix}
						0	\\ 
						0	\\ 
						1	
					\end{pmatrix},\\
					A\begin{pmatrix}
						0	\\ 
						0	\\ 
						1	
					\end{pmatrix}&=\begin{pmatrix}
						0	\\ 
						-1	\\ 
						0	
					\end{pmatrix},
				\end{aligned}
			\end{equation}
			et donc la matrice est
			\begin{equation}
				A=\begin{pmatrix}
					1	&	0	&	0	\\
					0	&	0	&	-1	\\
					0	&	1	&	0
				\end{pmatrix}.
			\end{equation}
			C'est certain que $e_1$ est vecteur propre de valeur propre $1$. Pour les autres, voyons l'équation caractéristique :
			\begin{equation}
				\begin{vmatrix}
					1-\lambda	&	0	&	0	\\
					0	&	-\lambda	&	-1	\\
					0	&	1	&	-\lambda
				\end{vmatrix}=(1-\lambda)(\lambda^2+1)=0.
			\end{equation}
			Seul $\lambda=1$ est solution, et comme cette racine est une racine simple, il n'y a que un seul vecteur propre correspondant. Nous concluons que $e_1$ (et ses multiples) est l'unique vecteur propre.
			Nous pouvons cependant le vérifier en résolvant l'équation
			\begin{equation}
				\begin{pmatrix}
					1	&	0	&	0	\\
					0	&	0	&	-1	\\
					0	&	1	&	0
				\end{pmatrix}\begin{pmatrix}
					x	\\ 
					y	\\ 
					z	
				\end{pmatrix}=\begin{pmatrix}
					x	\\ 
					y	\\ 
					z	
				\end{pmatrix}.
			\end{equation}
			Le système correspondant est
			\begin{equation}
				\begin{cases}
					x=x\\
					-z=y\\
					y=z.
				\end{cases}
			\end{equation}
			Les deux dernières équations imposent $y=z=0$, tandis que la première laisse $x$ libre. Nous avons donc bien que les multiples de $e_1$.


		\item
			Trouvons d'abord deux vecteurs linéairement indépendants dans le plan $x+y+z=0$. Prenons par exemple
			\begin{equation}
				f_1=\begin{pmatrix}
					1	\\ 
					-1	\\ 
					0	
				\end{pmatrix}
			\end{equation}
			et
			\begin{equation}
				f_2=\begin{pmatrix}
					1	\\ 
					0	\\ 
					-1	
				\end{pmatrix}.
			\end{equation}
			Noter qu'ils ne sont pas perpendiculaires, mais ce n'est pas important. L'important est que ces deux vecteurs soient linéairement indépendants et qu'ils restent inchangés par la symétrie considérée. Ils sont donc vecteurs propres de valeur propre $1$.
			Pour trouver un vecteur perpendiculaire au plan (et qui sera donc vecteur propre de valeur propre $-1$), nous prenons le produit vectoriel :
			\begin{equation}
				f_3=f_1\times f_2=\begin{vmatrix}
					e_1	&	e_2	&	e_3	\\
					1	&	-1	&	0	\\
					1	&	0	&	-1
				\end{vmatrix}=e_1+e_2+e_3=\begin{pmatrix}
					1	\\ 
					1	\\ 
					1	
				\end{pmatrix}.
			\end{equation}
			Cela est un troisième vecteur propre. Sa valeur propre est $-1$.

		\item
			La symétrie autour de l'axe $Oz$ fait en sorte que $e_z$ ne change pas, et que les deux autres changent de signe. Ensuite, la dilatation de facteur $5$ multiplie tout par $5$. Nous avons donc au final
			\begin{equation}
				\begin{aligned}[]
					\begin{pmatrix}
						0	\\ 
						0	\\ 
						1	
					\end{pmatrix}&\to\begin{pmatrix}
						0	\\ 
						0	\\ 
						5	
					\end{pmatrix}\\
					\begin{pmatrix}
						0	\\ 
						1	\\ 
						0	
					\end{pmatrix}&\to
					\begin{pmatrix}
						O	\\ 
						-5	\\ 
						0	
					\end{pmatrix}\\
					\begin{pmatrix}
						1	\\ 
						0	\\ 
						0	
					\end{pmatrix}&\to\begin{pmatrix}
						-5	\\ 
						0	\\ 
						0	
					\end{pmatrix}.
				\end{aligned}
			\end{equation}
			Ce sont les trois vecteur propres, de valeurs propres $5$, $-5$ et $-5$.

	\end{enumerate}
	

\end{corrige}
