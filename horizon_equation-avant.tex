%+++++++++++++++++++++++++++++++++++++++++++++++++++++++++++++++++++++++++++++++++++++++++++++++++++++++++++++++++++++++++++
					\section{Characterisation as orbit of group (by the equation)}
%+++++++++++++++++++++++++++++++++++++++++++++++++++++++++++++++++++++++++++++++++++++++++++++++++++++++++++++++++++++++++++
\label{SecHOrOrbEquation}

This section proves that, if we embed $AdS_3$ in $AdS_4$, one can express the horizon in $AdS_4$ as the result of the action of a one dimensional group on the horizon of $AdS_3$ (seen in $AdS_4$), theorem \ref{ThoEqHorQCoore}.

%---------------------------------------------------------------------------------------------------------------------------
					\subsection{The old three dimensional case}
%---------------------------------------------------------------------------------------------------------------------------

As mentioned in \cite{Keio}, the singularity of the three dimensional black hole in $AdS_3$ (seen as the group $\SL(2,\eR)$) accepts a nice description as lateral classes of $AN$ and $A\bar N$. That description is recalled in the proposition \ref{PropLatClassANSLdeuxR}. We want here to provide a similar description for the dimensional generalization $AdS_l=\SO(2,l-1)/\SO(1,l-1)$. 

Let us first make a simple remark. A lateral class in the description of proposition \ref{PropLatClassANSLdeuxR} is not guaranteed to be a lateral class in the description $AdS=G/H$. Moreover the ``$AN$'' of equation  \eqref{EqHorClassLatdeux} is not the ``$AN$'' of $\SO(2,2)$, but the one of $\SL(2,\eR)$. The results from the description $AdS_3=\SL(2,\eR)$ cannot be that simply translated into results in the description of $AdS_3=\SO(2,2)/\SO(1,2)$.

Let us begin by finding a group description of the horizon in $AdS_3$ in the description $AdS_3=\SO(2,2)/SO(1,2)$. The matricial expression of $ANJ$ in $AdS_3=\SL(2,\eR)$ is
\begin{equation}		\label{EqProSLJANexp}
\begin{pmatrix}
	e^a	&	le^a	\\ 
	0	&	 e^{-a}	
\end{pmatrix}
\begin{pmatrix}
	0	&	1	\\ 
	-1	&	0	
\end{pmatrix}
=
\begin{pmatrix}
	-le^a	&	e^a	\\ 
	- e^{-a}	&	0	
\end{pmatrix}
\end{equation}
The part of the hyperboloid described by these matrices is obtained by equating \eqref{EqProSLJANexp} with the matrix
\begin{equation}		\label{EqIdentMatriSLAdS}
	\begin{pmatrix}
	u+x	&	y+t	\\ 
	y-t	&	u-x	
\end{pmatrix}.
\end{equation}
The result is the vectors of the form
\begin{equation}		\label{EqVectoPotementSingAN}
\psi\big( Z(G)ANJ \big)\leadsto
	\begin{pmatrix}
	u	\\ 
	t	\\ 
	x	\\ 
	y	
\end{pmatrix}=
\pm
\begin{pmatrix}
	-\frac{ 1 }{2}e^al	\\ 
	\cosh(a)	\\ 
	-\frac{ 1 }{2}e^al	\\ 
	\sinh(a)	
\end{pmatrix}
=
\pm
\begin{pmatrix}
	\alpha	\\ 
	\cosh(a)	\\ 
	\alpha	\\ 
	\sinh(a)	
\end{pmatrix}
=\pm r_{AN}
\end{equation}
with $\alpha$, $a\in\eR$. This is a (almost\footnote{We did not compute the $A\bar N$ part of the horizon in $\SL(2,\eR)$.}) general vector of $AdS_3$ with $u^2-x^2=0$, which is coherent with the description \eqref{BTZSingHor}.

The same computation, using \eqref{EqGeneANbarSLdeuxR}, shows that the other part of the horizon in $AdS_3$ is given by
\begin{equation}		\label{EqVectoPotementSingANbar}
\psi\big( Z(G)A\bar NJ\big)
=
\pm\psi
\begin{pmatrix}
	0	&	e^a	\\ 
	- e^{-a}	&	l e^{-a}	
\end{pmatrix}
\leadsto
\begin{pmatrix}
	u	\\ 
	t	\\ 
	x	\\ 
	y	
\end{pmatrix}=
\pm
\begin{pmatrix}
	\frac{1}{ 2 } e^{-a}l	\\ 
	\cosh(a)	\\ 
	-\frac{1}{ 2 } e^{-a}l	\\ 
	\sinh(a)	
\end{pmatrix}
=
\pm
\begin{pmatrix}
	\alpha	\\ 
	\cosh(a)	\\ 
	-\alpha	\\ 
	\sinh(a)	
\end{pmatrix}
=\pm
r_{A\bar N}
\end{equation}
where $a$ and $\alpha$ are running over $\eR$.

From the equations \eqref{EqVectoPotementSingAN} and \eqref{EqVectoPotementSingANbar}, we are able to express the horizon in $AdS_3$ as union of lateral classes of the element
\begin{equation}
	b=\begin{pmatrix}
		0	\\ 
		1	\\ 
		0	\\ 
		0	
	\end{pmatrix}.
\end{equation}
It is, indeed, easy to see that $G_{ X_{(-1,1)},J_1}\cdot b =G_{ X_{(1,1)},J_1}\cdot b$ and $G_{ J_1,X_{(1,-1)},J_1}\cdot b=G_{ J_1,X_{(-1,-1)} ,J_1}\cdot b$. We can express the horizon $\hH_3$ in the following way :
\begin{equation}
	\begin{aligned}[]
		\hH_3	&=\pm G_{ X_{(-1,1)},J_1}\cdot b\cup \pm G_{ X_{(1,-1)},J_1}\cdot b  \\
			&=\pm G_{ \{J_1,X_{(1,1)}\}}\cdot b\cup \pm G_{ \{J_1,X_{(-1,-1)}\}}\cdot b,
	\end{aligned}
\end{equation}
and the two other combinations. Here, $G_{X,Y}$ is the group generated by $X$ and $Y$.

%---------------------------------------------------------------------------------------------------------------------------
\subsection{Characterization by induction on the dimension}
%---------------------------------------------------------------------------------------------------------------------------

From a computational point of view, it reveals to be more or less impossible to directly check that \eqref{EqVectoPotementSingAN} belongs to the singularity using the method of equation \eqref{EqhohnCondHOrExpl}, not even in dimension $4$. Here is the strategy to compute the horizon in higher dimension:
\begin{enumerate}
\item
The map $\psi\colon \SL(2,\eR)\to AdS_3$ given by \eqref{EqIdentMatriSLAdS} is an isometry which maps the singularity into the singularity. Thus it has to map the horizon to the horizon. If $\hH_{\SL(2,\eR)}$ denotes the horizon in $\SL(2,\eR)$, then the set $\psi\big( \hH_{\SL(2,\eR)} \big)$ is the horizon in $AdS_3=\SO(2,2)/SO(2,1)$.

\item
We consider the inclusion $\iota\colon \SO(2,n)\to \SO(2,n+1)$ given by $g\mapsto\begin{pmatrix}
	g	&	0	\\ 
	0	&	1	
\end{pmatrix}$ and its differential $d\iota\colon \so(2,n)\to \so(2,n+1)$, $X\mapsto\begin{pmatrix}
	X	&	0	\\ 
	0	&	0	
\end{pmatrix}$. Now, we are going to build the horizons of $AdS_l$ by induction over $l$, starting on $l=3$.
\end{enumerate}

We denote by $\hH_l$ and $\hS_l$ the horizon and the singularity in $AdS_l$. The structure of the algebras (equations \eqref{EqLeANEnDimAlg} and \eqref{EqTableSOIwa}) show immediately that
\begin{equation}
	(\sA\oplus\sN)_{\so(2,n+1)}=\Span\left\{   d\iota(\sA\oplus\sN)_{\so(2,n)},V_{n+2},W_{n+2}  \right\},
\end{equation}
so that the structure of one dimension is defined from the structure of the previous one by adding the two new vectors $V$ and $W$. The same holds for $\sA\oplus\bar\sN$.


Now, the work is to find what is \emph{added} to the horizon when one passes from one dimension to the higher one. From that point of view, the matrix $V_i$ has a wonderful property: $ e^{V}$ does not change the $t$ and $y$ component of the vector on which it acts. Thus we have the following.
\begin{lemma}		\label{LemHorpigeVDdeux}
We have
\begin{equation}
	\pi(g e^{-s\Ad(k)E_1})\in\hS
\end{equation}
if and only if
\begin{equation}
	 \pi(e^{V}g e^{-s\Ad(k)E_1})\in\hS.
\end{equation}
\end{lemma}

\begin{proof}
The exponential of the matrix $V_5$ is given in equation \eqref{EqExpDeV}. The second and fourth column being the identity, $e^V1_t=1_y$ and $e^V1_y=1_y$. Thus the characterisation $t^2-y^2=0$ of the singularity is satisfied for one point $x\in AdS$ if and only if it is satisfied by the point $e^Vx$.
\end{proof}

Lemma \ref{LemHorpigeVDdeux} still holds if one replace $V$ by $X$.
	
We consider the following points in the horizon:
\begin{equation}		\label{EqPartewWrAN}
	\begin{aligned}[]
	r(a,\alpha,w)&= e^{wW}r_{AN}=
\frac{ 1 }{2}
\begin{pmatrix}
	2\alpha	\\ 
	e^{-a}w^2+2\cosh(a)\\ 
	2\alpha	\\ 
	e^{-a}w^2+2\sinh(a)	\\ 
	2 e^{-a}w	
\end{pmatrix},\\
	\bar r(a,\alpha,w)&= e^{wW}r_{A\bar N}=
\frac{ 1 }{2}
\begin{pmatrix}
	2\alpha	\\ 
	 e^{-a}w^2+2\cosh(a) \\ 
	-2\alpha	\\ 
	e^{-a}w^2+2\sinh(a)	\\ 
	 2e^{-a}w	
\end{pmatrix}.
	\end{aligned}
\end{equation}

The tangent vectors of that surface are given by
\begin{equation}
	\begin{aligned}[]
		(\partial_ar)(a,\alpha,w)&=
\begin{pmatrix}
	0	\\ 
	\frac{ - e^{-a}w^2+2\sinh(a) }{2}	\\ 
	0	\\ 
	\frac{ - e^{-a}w^2+2\cosh(a) }{2}	\\ 
	- e^{-a}w	
\end{pmatrix}
,&
		(\partial_{\alpha}r)(a,\alpha,w)&=
\begin{pmatrix}
	1	\\ 
	0\\ 
	1	\\ 
	0\\ 
	0	
\end{pmatrix}
,&
		(\partial_wr)(a,\alpha,w)&=
\begin{pmatrix}
	0	\\ 
	 e^{-a}w\\ 
	0	\\ 
	 e^{-a}w\\ 
	e^{-a}
\end{pmatrix}
	\end{aligned}.
\end{equation}
Notice that these three vectors are nowhere vanishing. It is immediate that the vector $\partial_{\alpha}r$ is linearly independent of $\partial_{a}r$ and of $\partial_wr$. It is also immediately apparent that $\partial_ar=-w\partial_wr$ is the worse possible situation. It is, however, not possible because it would imply that 
\begin{equation}
	\begin{aligned}[]
		-w^2 e^{-a}&=\frac{ - e^{-a}w^2+2\sinh(a) }{2}&\text{and}&&-w^2 e^{-a}&=\frac{ - e^{-a}w^2+2\cosh(a) }{2},
	\end{aligned}
\end{equation}
which is only possible when $\cosh(a)=\sinh(a)$, in other words : never. Thus, the part of $AdS_4$ described by \eqref{EqPartewWrAN} has dimension $3$.


\begin{proposition}
We have
\begin{equation}
	 G_W\cdot\iota(\hH_3)=
	\{ r(a,\alpha,w)\cup\bar r(a,\alpha,w) \}_{a,\alpha,w\in\eR}.	
\end{equation}
\end{proposition}

\begin{proof}
The facts that $G_W\cdot\iota(\hH_3)=\{ r(a,\alpha,w)\cup\bar r(a,\alpha,w) \}_{a,\alpha,w\in\eR}$ and that all the elements of that set are subject to $u^2-x^2=0$ are by construction.

We still have to prove that $\{ u^2-x^2=0\}\subseteq G_W\cdot\iota(\hH_3)$.

Let $v=(y,t,x,y,z)$ be a vector which satisfies $u^2-x^2=0$. Following the signs of $u$ and $t$, we are searching $v$ under the form $\pm r(\alpha,a,w)$ or $\pm \bar r(\alpha,a,w)$. In any case, the value of $u$ and $x$ fix $\alpha$ and we are left with the condition
\begin{equation}
\pm\frac{ 1 }{2}
	\begin{pmatrix}
	e^{-a}w^2+2\cosh(a)	\\ 
	e^{-a}w^2+2\sinh(a)	\\ 
	2 e^{-a}w	
\end{pmatrix}
=
\begin{pmatrix}
	t	\\ 
	y	\\ 
	z	
\end{pmatrix}
\end{equation}
with $t^2-y^2-z^2=1$.  If $t-y>0$, we choose the sign $+$ and the value of $t-y$ fixes $a$ because $t-y= e^{-a}$. In that situation, $w$ is given by $w= e^{a}\big(2y-2\sinh(a)\big)$. If $t-y<0$, we have $t-y=- e^{-a}$ and the same argument holds.

\end{proof}

In the sequel, we will use the following notations :
\begin{equation}
	\begin{aligned}[]
		G_W&=\{  e^{wW}\tq w\in\eR \}\\
		G_V&=\{  e^{\alpha V}\tq \alpha\in\eR \}\\
		G_X&=\{  e^{\beta X}\tq \beta\in\eR \}\\
		G_Y&=\{  e^{y Y}\tq y\in\eR \}
	\end{aligned}
\end{equation}
These are one parameter subgroups of $SO(2,3)$.

\begin{proposition}		\label{PropInclusionsTroisQuatreWVXY}
If $v\in AdS_4$ satisfies $u-t\neq 0$, then $v= e^{wW}v'$ for a certain $v'\in AdS_3$. In other words,
\begin{equation}
	\{ y-t\neq 0 \}_4\subset G_W\cdot\iota(AdS_3).
\end{equation}
In particular, every points outside the singularity $\hS_4$ are obtained by action of $G_W$ on a point of $AdS_3$. We also have
\begin{equation}
	\begin{aligned}[]
		\{ x-u\neq 0 \}_4&\subseteq G_V\cdot\iota(AdS_3)\\
		\{ x+u\neq 0 \}_4&\subseteq G_X\cdot\iota(AdS_3).
	\end{aligned}
\end{equation}
\end{proposition}

\begin{proof}
We have
\begin{equation}
	 e^{wW}
\begin{pmatrix}
	u'	\\ 
	t'	\\ 
	x'	\\ 
	y'	\\ 
	0	
\end{pmatrix}=
\begin{pmatrix}
	u'	\\ 
	\left( 1+\frac{ w^2 }{2} \right)t'-\frac{ w^2 }{2}y'	\\ 
	x'	\\ 
	\frac{ w^2 }{2}t'+\left( 1-\frac{ w^2 }{2} \right)y'	\\ 
	w(t'-y')	
\end{pmatrix}
=
\begin{pmatrix}
	u	\\ 
	t	\\ 
	x	\\ 
	y	\\ 
	z	
\end{pmatrix}
\end{equation}
when
\begin{equation}
	\begin{aligned}[]
		u'&=u,& t'&=\frac{ z^2+2ty-2t^2 }{ 2(y-t) },&x'&=x,&y'&=\frac{ z^2-2ty+2y^2 }{ 2(y-t) },&w&=-\frac{ z }{ y-t }.
	\end{aligned}
\end{equation}
In the same way, the equation
\begin{equation}
	 e^{\alpha V}\begin{pmatrix}
	u'	\\ 
	t'	\\ 
	x'	\\ 
	y'	\\ 
	0	
\end{pmatrix}=
	 \begin{pmatrix}
	u	\\ 
	t\\ 
	x	\\ 
	y\\ 
	z	
\end{pmatrix}
\end{equation}
is solved by
\begin{equation}
	\begin{aligned}[]
		u'&=\frac{ z^2+2ux-2u^2 }{ 2(x-u) },&x'&=\frac{ z^2-2ux+2x^2 }{ 2(x-u) },&\alpha&=-\frac{ z }{ x-u }.
	\end{aligned}
\end{equation}
Thus, $\{ x-u\neq 0 \}_4\subseteq G_V\cdot\iota(AdS_3)$. And, finally, the equation
\begin{equation}
	 e^{\beta X}\begin{pmatrix}
	u'	\\ 
	t'	\\ 
	x'	\\ 
	y'	\\ 
	0	
\end{pmatrix}=
	 \begin{pmatrix}
	u	\\ 
	t	\\ 
	x	\\ 
	y	\\ 
	z	
\end{pmatrix}
\end{equation}
is solved by
\begin{equation}
	\begin{aligned}[]
		u'&=\frac{ z^2-2ux-2u^2 }{ 2(x+u) },&x'&=\frac{ z^2+2ux+2x^2 }{ 2(x+u) },&\beta&=-\frac{ z }{ x+u }.
	\end{aligned}
\end{equation}
Thus, $\{ x+u\neq 0 \}_4\subseteq G_X\cdot\iota(AdS_3)$. 

\end{proof}

One interest of that proposition resides in the fact that every element of $AdS_4$ outside the singularity is the image of an element of $AdS_3$ by $G_W$.


\begin{proposition}		\label{PropSingQTiV}
We have 
\begin{equation}
	\hS_4=G_V\cdot\iota(\hS_3)
\end{equation}
where $G_V=\{  e^{vV}\tq v\in\eR \}$ is the group generated by $V$.
\end{proposition}

\begin{proof}
A point of $\iota(\hS_3)$ is of the form
$	\begin{pmatrix}
	u	\\ 
	\alpha	\\ 
	x	\\ 
	\epsilon\alpha	\\ 
	0	
\end{pmatrix}
$, while an element of $\hS_4$ is of the form
$
	\begin{pmatrix}
	u'	\\ 
	\alpha	\\ 
	x'	\\ 
	\epsilon\alpha	\\ 
	z'	
\end{pmatrix}
$ where $\epsilon=\pm 1$. So we have to solve the equation
\begin{equation}
	 e^{vV}
\begin{pmatrix}
	u	\\ 
	\alpha	\\ 
	x	\\ 
	\epsilon\alpha	\\ 
	0	
\end{pmatrix}=
\begin{pmatrix}
	u\left( \frac{ v^2 }{ 2 }+1 \right)+\frac{ v^2 }{2}x	\\ 
	\alpha	\\ 
	\left( 1-\frac{ v^2 }{2} \right)x+\frac{ v^2 }{2}u	\\ 
	\epsilon\alpha	\\ 
	v(u-x)	
\end{pmatrix}
\stackrel{!}{=}
\begin{pmatrix}
	u'	\\ 
	\alpha	\\ 
	x'	\\ 
	\epsilon\alpha	\\ 
	z'	
\end{pmatrix}
\end{equation}
with respect to $v$, $u$ and $x$. A solution is given by
\begin{equation}
	\begin{aligned}[]
		u&=\frac{ z'^2+2u'x'-2u'^2 }{ 2(x'-u') },&x&=\frac{ z'^2-2u'x'+2x'^2 }{ 2(x'-u') },&v&=-\frac{ z' }{ x'-u' }.
	\end{aligned}
\end{equation}
The condition $u'^2-x'^2-z'^2=1$ imposes $x'\neq u'$, so that that solution always makes sense: a point of $\hS_4$ is always obtained as the result of the action of an element of $G_V$ on an element of~$\hS_3$.

Since the operator $ e^{vV}$ does not touch the variables $t$ and $y$, it is obvious that $G_V\cdot \hS_3\subseteq\hS_4$.
\end{proof}

\begin{lemma}		\label{LemTNTroisIneq}
In $AdS_3$, the black hole is given by $u^2-x^2>0$
\end{lemma}

\begin{proof}
The black hole is the set of point from which every light ray intersect the singularity. The boundary of that set is given by the horizon (this is the definition of the horizon), and we already proved that $\hH_3\equiv u^2-x^2=0$. Thus the black hole is $u^2-x^2>0$, or $u^2-x^2<0$. Since the singularity (which is part of the black hole) is given by $t^2-y^2=0$, the singularity satisfies $u^2-x^2=1$, and is thus in the part $u^2-x^2>0$.
\end{proof}


Let $TN[g]$ be the subset of $\{ \Ad(k)E_1 \}_{k\in \SO(3)}$ of elements for which there exists a $s>0$ such that
\begin{equation}
	\pi(g e^{s\Ad(k)E_1})\in\hS.
\end{equation}
In other words, $TN[g]$ is the set of directions along which $[g]$ falls in the singularity. If the complementary $TN[g]^c$ has a non empty interior, the by continuity, the complementary $TN[g']$ will have an interior as well for every $[g']$ close enough from $[g]$. In that case, $[g]$ does not belongs to the horizon. So a point belongs to the horizon when the set of safe direction has no interior.


\begin{lemma}
We have
\begin{equation}
	G_V\cdot\iota(\hH_3)\equiv u^2-x^2-z^2=0,
\end{equation}
so that it is the good candidate to be the horizon.
\end{lemma}

\begin{proof}
An element of $\iota(\hH_3)$ has the form
$r=\begin{pmatrix}
	u'	\\ 
	t'	\\ 
	x'	\\ 
	\pm\sqrt{t'^2-1}	\\ 
	0	
\end{pmatrix}$,
so that we have to solve the equation
\begin{equation}
	 e^{vV}r=\begin{pmatrix}
	\left( \frac{ v^2 }{ 2 }+1 \right)u'-\frac{ v^2 }{ 2 }x'	\\ 
	t'	\\ 
	\left( 1-\frac{ v^2 }{ 2 } \right)x'+\frac{ v^2 }{ 2 }u'	\\ 
	\pm\sqrt{t'^2-1}	\\ 
	v(u'-x')	
\end{pmatrix}
=
\begin{pmatrix}
	u	\\ 
	t	\\ 
	x	\\ 
	\pm\sqrt{t^2-1}	\\ 
	z	
\end{pmatrix}.
\end{equation}
The solution is
\begin{equation}
	\begin{aligned}[]
		u'&=\frac{ z^2+2ux-2u^2 }{ 2(x-u) },&x'&=\frac{ z^2-2ux+2x^2 }{ 2(x-u) },&v&=-\frac{ z }{ x-u }.
	\end{aligned}
\end{equation}
Since $u^2-x^2-z^2=1$, we have $x-u\neq 0$, so that these solutions always make sense.
\end{proof}

\begin{lemma}		\label{LemPasseTroisQuatreOuvert}
Let $v\in AdS_4$ and $g$ a representative of the class of $v$ in $\SO(2,3)/\SO(1,3)$. If the set
\begin{equation}
	\big\{ \begin{pmatrix}
	w_1	\\ 
	w_2	
\end{pmatrix}\in S^1\tq \pi(g\begin{pmatrix}
	1	\\ 
	-s	\\ 
	s\bar w	\\ 
	0	
\end{pmatrix})\cap\hS_4=\emptyset\text{ with $s>0$} \big\}
\end{equation}
has an interior (in $S^1$), then the set
\begin{equation}
	\big\{ \begin{pmatrix}
	w_1	\\ 
	w_2	\\
	w_3
\end{pmatrix}\in S^2\tq \pi(g\begin{pmatrix}
	1	\\ 
	-s	\\ 
	s\bar w
\end{pmatrix})\cap\hS_4=\emptyset\text{ with $s>0$} \big\}
\end{equation}
has an interior in $S^2$.
\end{lemma}

\begin{proof}
The matrix $g$ in $\SO(2,3)$ representing the point $v$ has the form
\begin{equation}
	g=\begin{pmatrix}
 u	&	.	&	.	&	.	&	.\\ 
 t	&	a	&	b	&	c	&	d\\ 
 x	&	.	&	.	&	.	&	.\\ 
 y	&	a'	&	b'	&	c'	&	d'\\ 
z	&	.	&	.	&	.	&	. 
 \end{pmatrix}
\end{equation}
where the numbers $a,b,c,d,a',b',c',d'$ are not uniquely determined. From the assumption, we suppose that the direction
\begin{equation}
	 e^{\Ad(k)E_1}=\begin{pmatrix}
	1	\\ 
	-1	\\ 
	w_1	\\ 
	w_2	\\ 
	0	
\end{pmatrix}
\end{equation}
escapes the singularity, when one starts from the point $[g]$, in other words, the path
\begin{equation}
	\pi(g e^{s\Ad(k)E_1})= 
	g=\begin{pmatrix}
 u	&	.	&	.	&	.	&	.\\ 
 t	&	a	&	b	&	c	&	d\\ 
 x	&	.	&	.	&	.	&	.\\ 
 y	&	a'	&	b'	&	c'	&	d'\\ 
z	&	.	&	.	&	.	&	. 
 \end{pmatrix}
\begin{pmatrix}
	1	\\ 
	-s	\\ 
	sw_1	\\ 
	sw_2	\\ 
	0	
\end{pmatrix}
\end{equation}
does not intersects $\hS$. Moreover, if one replaces $(w_1,w_2)$ by an other direction $(w_1',w_2')$ close to $(w_1,w_2)$, the new path does not intersect the singularity neither.

Let
\begin{equation}
	\begin{aligned}[]
		T(w_1,w_2)&=t+s(bw_1+cw_2-a)\\
		Y(w_1,w_2)&=y+s(b'w_1+c'w_2-a')\\
		A_+(w_1,w_2)&=(b+b')w_1+(c+c')w_2-(a+a')\\
		A_-(w_1,w_2)&=(b-b')w_1+(c-c')w_2-(a-a').
	\end{aligned}
\end{equation}
We also denote by $\sigma_{\pm}$ the sign of $t\pm y$.

A simple computation shows that $T+Y=0$ when
\begin{equation}
	s=s_+=-\frac{ t+y }{ A_+(w_1,w_2) },
\end{equation}
and $T-Y=0$ when
\begin{equation}
	s=s_-=-\frac{ t-y }{ A_-(w_1,w_2) },
\end{equation}
The assumption is that the direction $(w_1,w_2,0)$ (and an open set in $S^1$ with respect to $(w_1,w_2)$) escapes the singularity, so that for every $(w_1',w_2')$ in a neighborhood of $(w_1,w_2)$, we have
\begin{equation}
	\begin{aligned}[]
		\sigma_{\pm}A_{\pm}(w_1',w_2')\geq 0,
	\end{aligned}
\end{equation}
which assures that the values of $s$ which annihilate $T+Y$ and $T-Y$ are negative or non existing. Let us consider, in that neighbourhood, a direction $(w_1,w_2)$ such that $\sigma_{\pm}A_{\pm}(w_1',w_2')>0$. Notice that, by continuity, there exists a neighbourhood of $(w_1,w_2)$ in $S^1$ which escapes the singularity.

We are now studying what happens when one looks at a neighbourhood of $(w_1,w_2,0)$ in $S^3$. The path is replaced by
\begin{equation}
	\pi(g e^{s\Ad(k)E_1})= 
	\begin{pmatrix}
 u	&	.	&	.	&	.	&	.\\ 
 t	&	a	&	b	&	c	&	d\\ 
 x	&	.	&	.	&	.	&	.\\ 
 y	&	a'	&	b'	&	c'	&	d'\\ 
z	&	.	&	.	&	.	&	. 
 \end{pmatrix}
\begin{pmatrix}
	1	\\ 
	-s	\\ 
	s(w_1+\epsilon_1)	\\ 
	s(w_2+\epsilon_2)	\\ 
	\epsilon_3	
\end{pmatrix},
\end{equation}
and we consider
\begin{equation}
	\begin{aligned}[]
		T(w_1,w_2,\bar\epsilon)&=t+s\big( b(w_1+\epsilon_1)+c(w_2+\epsilon_2)+d\epsilon_3-a \big)\\
		Y(w_1,w_2,\bar\epsilon)&=y+s\big( b'(w_1+\epsilon_1)+c'(w_2+\epsilon_2)+d'\epsilon_3-a' \big)
	\end{aligned}
\end{equation}
where $\bar\epsilon$ stands for $\epsilon_1$, $\epsilon_2$ and $\epsilon_3$. The same computations as before shows that $T+Y=0$ when
\begin{equation}
	s=s_+=-\frac{ t+y }{ A_+(w_1,w_2)+(b+b')\epsilon_1+(c+c')\epsilon_2+(d+d')\epsilon_3 },
\end{equation}
Since $\sigma_+A(w_1,w_2)>0$, there exists a $\delta$ such that $s_+$ stays negative for every choice of $\bar\epsilon<\delta$. The same holds with $T-Y$ which is zero when
\begin{equation}
	s=s_-=-\frac{ t-y }{ A_-(w_1,w_2)+(b-b')\epsilon_1+(c-c')\epsilon_2 +(d-d')\epsilon_3 }.
\end{equation}
Since $\sigma_-A_-(w_1,w_2)>0$, one can find a $\delta>0$ such that $\bar\epsilon<\delta$ implies that this fraction remains negative.

Thus, there exists a neighbourhood of $(w_1,w_2,0)$ of directions in $S^2$ which escape the singularity.
\end{proof}


\begin{lemma}		\label{LemHfHfdedans}
If  $\hF_l$ denotes the free zone of $AdS_l$ (the points from which there exists one geodesic escaping the singularity), 
\begin{equation}
	\iota\big( \Int(\hF_3) \big)\subseteq \Int\big( \hF_4 \big)
\end{equation}
where $\Int$ stands for the interior. In order words, 
\begin{equation}
	\Adh(BH_4)\cap\iota(AdS_3)\subset\iota\big( \Adh(BH_3) \big).
\end{equation}
\end{lemma}

\begin{proof}

Let $v=\iota(v')\notin\iota\big( \Adh(BH_3) \big)$, we also consider $g'$ a representative of $v'$ and $g=\iota(g')$, which is a representative of $v$. Here, $BH_4$ is the set of points in $AdS_4$ from which all light like geodesics intersect the singularity, and $\Adh$ stands for the closure. The element $v'$ is in the interior of the free zone: there exists an open set of directions which do not intersect the singularity of $AdS_3$. In other words, the set
\begin{equation}
	\begin{pmatrix}
	w_1	\\ 
	w_2	
\end{pmatrix}\in S^1\tq
\pi g'\begin{pmatrix}
	1	\\ 
	-s	\\ 
	sw_1	\\ 
	sw_2	
\end{pmatrix}\cap\hS_3 =\emptyset
\end{equation}
contains an open set of $S^1$. Thus, for the element $g=\iota(g')$, there is an open subset of
\begin{equation}		\label{EqOUvertDirUunun}
	\begin{pmatrix}
	w_1	\\ 
	w_2	
\end{pmatrix}\in S^1\tq
\pi g\begin{pmatrix}
	1	\\ 
	-s	\\ 
	s\bar w	\\ 
	0	
\end{pmatrix}\cap\hS_3 =\emptyset.
\end{equation}
For each of these $\bar w$, the $z$-component of $g\begin{pmatrix}
	1	\\ 
	-s	\\ 
	s\bar w	\\ 
	0	
\end{pmatrix}$ is obviously zero because $g=\iota(g')$ has the form
\begin{equation}
	g=\begin{pmatrix}
 .	&	.	&	.	&	.	&	0\\ 
 .	&	.	&	.	&	.	&	0\\ 
 .	&	.	&	.	&	.	&	0\\ 
 .	&	.	&	.	&	.	&	0\\ 
0	&	0	&	0	&	0	&	1 
 \end{pmatrix}.
\end{equation}
Thus, in fact, the open set of \eqref{EqOUvertDirUunun} which does not intersect the singularity $\hS_3$ does not intersects the singularity $\hS_4$ neither, and the set
\begin{equation}
	\begin{pmatrix}
	w_1	\\ 
	w_2	
\end{pmatrix}\in S^1\tq
\pi g\begin{pmatrix}
	1	\\ 
	-s	\\ 
	s\bar w	\\ 
	0	
\end{pmatrix}\cap\hS_4 =\emptyset
\end{equation}
has an interior.

By lemma \ref{LemPasseTroisQuatreOuvert}, there exists an open subset of
\begin{equation}
	\begin{pmatrix}
	w_1	\\ 
	w_2	\\ 
	w_3	
\end{pmatrix}\in S^2\tq
\pi g\begin{pmatrix}
	1	\\ 
	-s	\\ 
	s\bar w	
\end{pmatrix}\bar\hS_4=\emptyset,
\end{equation}
from which we conclude that $v=\iota(v')$ is in the interior of the free part of $AdS_4$.

\end{proof}


\begin{proposition}		\label{PropFqTroisFt}
We have $\hF_4\cap\iota(AdS_3)\subset \iota(\hF_3)$.
\end{proposition}

\begin{proof}
The fact that $v\in\iota(AdS_3)$ means that $v=\iota(v')$ with $v'$ under the form $v'=\pi(g')$ where $g'\in \SO(2,2)$. By construction, $\iota(g')$ has the form
\begin{equation}		\label{EqRepresSOiotag}
	\iota(g')=
\begin{pmatrix}
 u	&	.	&	.	&	.	&	0\\ 
 t	&	a	&	b	&	c	&	0\\ 
 x	&	.	&	.	&	.	&	0\\ 
 y	&	a'	&	b'	&	c'	&	0\\ 
0	&	0	&	0	&	0	&	1 
 \end{pmatrix}
\end{equation}
The assumption is that, for every representative $g'$ of $v'$, there exists a direction $(w_1,w_2,w_3)\in S^2$ such that
\begin{equation}		\label{EqGedgpudt}
	\pi\big(   \iota(g')\begin{pmatrix}
	1	\\ 
	-s	\\ 
	sw_1	\\ 
	sw_2	\\ 
	sw_3	
\end{pmatrix}  \big)
\end{equation}
does not intersects the singularity. The values of $s$ that annihilate $t^2-y^2$ in the geodesic \eqref{EqGedgpudt} are
\begin{equation}
	\begin{aligned}[]
		s_+	&=-\frac{ t+y }{ -(a+a')+(b+b')w_1+(c+c')w_2 }\\
		s_-	&=-\frac{ t-y }{ -(a-a')+(b-b')w_1+(c-c')w_2 },
	\end{aligned}
\end{equation}
and these two values are either negative either non existing (vanishing denominator).

The work is now to find a direction $(w'_1,w'_2)\in S^1$ such that the geodesic
\begin{equation}
	\pi\big( g'\begin{pmatrix}
	1	\\ 
	-s	\\ 
	sw'_1	\\ 
	sw'_2	
\end{pmatrix} \big)
\end{equation}
does not intersect the singularity. The values of $s$ for which that geodesics intersects the singularity are
\begin{equation}
	\begin{aligned}[]
		s'_+	&=-\frac{ t+y }{ -(a+a')+(b+b')w'_1+(c+c')w'_2 }\\
		s'_-	&=-\frac{ t-y }{ -(a-a')+(b-b')w'_1+(c-c')w'_2 }.
	\end{aligned}
\end{equation}
If $w_3=0$, the proposition is true because one can choose $(w'_1,w'_2)=(w_1,w_2)$. If $w_3\neq 0$, the vector $(w_1,w_2)$ does not belong to $S^1$.

Let us consider the following two cases.
\begin{enumerate}
\item
there exists a representative \eqref{EqRepresSOiotag} with $a=a'=0$,
\item
there exists a representative \eqref{EqRepresSOiotag} with $c=c'=0$.
\end{enumerate}
In the first case, we have 
\begin{equation}		\label{EqDenoAAnnulerspm}
	s'_{\pm}=-\frac{ t\pm y }{ (b\pm b')w'_1+(c\pm c')w'_2 },
\end{equation}
and we can choose $(w'_1,w'_2)=N(w_1,w_2)$ with $N\in\eR$ fixed in such a way that $(w'_1,w'_2)\in S^1$. Thus we have $s'_{\pm}=\frac{1}{ N }s_{\pm}$ and it is sufficient to choose $N>0$ in order to leave the denominators of \eqref{EqDenoAAnnulerspm} of the right sign or zero.

In the second case, we have
\begin{equation}
	s'_{\pm}=-\frac{ t\pm y }{ -(a\pm a')+(b\pm b')w'_1 },
\end{equation}
thus one has to choose $w'_1=w_1$ and $w'_2=\sqrt{1-w_1^2}$.

Let us now discuss the values of $u$, $t$, $x$ and $y$ for which the first or the second cases are enforced. In order to be in the first case, we need to build a matrix of $\SO(2,2)$ of the form
\begin{equation}
	g'=\begin{pmatrix}
 u	&	\alpha	&	.	&	.	\\ 
 t	&	0	&	.	&	.	\\ 
 x	&	\beta	&	.	&	.	\\ 
 y	&	0	&	.	&	.	 
 \end{pmatrix}.
\end{equation}
That requires $\alpha^2-\beta^2=1$ and $u\alpha-x\beta=0$, while, for the second case, we need to build a matrix of $\SO(2,2)$ of the form
\begin{equation}
	g'=\begin{pmatrix}
 u	&	.	&	\alpha	&	.	\\ 
 t	&	.	&	0	&	.	\\ 
 x	&	.	&	\beta	&	.	\\ 
 y	&	.	&	0	&	.		 
 \end{pmatrix}.
\end{equation}
That requires $\alpha^2-\beta^2=-1$ and $u\alpha-x\beta=0$. In both cases, we have $\beta=\frac{ u }{ x }\alpha$ and $\alpha^2-\beta^2=\alpha^2\left( 1-\frac{ u^2 }{ x^2 } \right)$. If $| u |>| x |$, then we can solve the first case, and if $| u |<| x |$, then we can solve the second case. 

The last possible situation is $u=\pm x$. A point of $AdS_3$ in that situation belongs to the horizon, while one knows that point of horizon do have some directions which escape the singularity by corollary \ref{CorTNFermeHorEchape}. Notice that in the latter situation, we do not use the assumption that $\iota(v')$ is free in $AdS_4$.
\end{proof}

\begin{corollary}		\label{CorHansDansH}
We have $\iota(BH_3)\subset BH_4$ and $\iota(\hH_3)\subset\hH_4$.
\end{corollary}

\begin{proof}

Let $\bar v\in BH_3$. If $\iota(\bar v)\notin BH_4$, then $\iota(\bar v)\in\hF_4\cap\iota(AdS_3)\subset \iota(\hF_3)$, which contradicts the fact that $\bar v\in BH_3$.

If $v\in\hH_3$, then $\iota(v)\in\hF_4$. The assumption on $v$ makes that in every neighbourhood of $v$, there exists a $\bar v\in BH_3$. The point $\iota(\bar v)$ belongs to $BH_4$ by the first part. Thus in every neighbourhood of $\iota(v)$, there exists a $\iota(\bar v)\in BH_4$, which shows that $\iota(v)\in \hH_4$.
\end{proof}

\begin{lemma}
If $[g]=\begin{pmatrix}
	u	\\ 
	t	\\ 
	x	\\ 
	y	\\ 
	z	
\end{pmatrix}\in AdS_4$ with $u$ and $x$ not both vanishing, then 
\begin{equation}
	[g]\in G_V\cdot\iota(AdS_3)\cup G_X\cdot \iota(AdS_3).
\end{equation}
Notice that the union is not disjoint.
\end{lemma}

\begin{proof}
The proof is a simple computation. Following proposition \ref{PropInclusionsTroisQuatreWVXY}, we have $\{ x-u\neq 0 \}_4\subseteq G_V\cdot\iota(AdS_3)$ and $\{ x+u\neq 0 \}_4\subseteq G_X\cdot\iota(AdS_3)$.

So the only part of $AdS_4$ which is not included in $G_V\cdot \iota(AdS_3)\cup G_X\cdot\iota(AdS_3)$ is the part where $x+u=x-u=0$.
\end{proof}

Now, we want to study the horizon, that means the boundary of $BH_4$. If $v\in\partial\big(\Adh(BH_4) \big)$, there exists, in any neighbourhood of $v$, an element $\bar v$ and a direction following which the geodesic from $\bar v$ escapes the singularity.


\begin{lemma}		\label{LemHiotaiotaH}
We have $\hH_4\cap\iota(AdS_3)\subset\iota(\hH_3)$.
\end{lemma}

\begin{proof}
Let $v\in \hH_4\cap \iota(AdS_3)$; in particular $v\in\hF_4\cap\iota(AdS_3)$ and proposition \ref{PropFqTroisFt} shows $v=\iota(v')$ with $v'\in \hF_3$. If $v'\notin\hH_3$, then $v'\in\Int(\hF_3)$, which implies that $\iota(v')\in\Int(\hF_4)$ by lemma \ref{LemHfHfdedans}. The latter contradicts the fact that $v\in\hH_4$. 
\end{proof}

Up to now, we studied the way $AdS_3$ embed in $AdS_4$. In particular, we proved that the horizon of $AdS_3$ is included in the horizon of $AdS_4$. We can propagate the results by $G_V$ and $G_X$ because, given a $v\in AdS_3$, the existence of a $\alpha$ such that $ e^{\alpha V}v\in\iota(AdS_3)$ or $ e^{\alpha X}v\in\iota(AdS_3)$ is related to the fact that $u^2-x^2\neq 0$, while that condition holds in a neighbourhood of $v$. 

\begin{lemma}		\label{LemPresqueHOrQadp}
Let $v\in\hH_4$ such that $u$ and $x$ are not both vanishing. In that case, $v\in G_V\cdot \iota(\hH_3)\cup G_X\cdot\iota(\hH_3)$.
\end{lemma}

\begin{proof}
The assumption on $u$ and $x$ make that $v\in G_V\cdot(AdS_3)\cup G_X\iota(AdS_3)$. In order to fix ideas, let us suppose that $v= e^{\alpha V}\iota(v')$ with $v'\in AdS_3$. Since the set of directions $(w_1,w_2,w_3)\in S^2$ which save the points $v$, $ e^{\alpha V}v$ and $ e^{\beta X}v$ are the same, the assumption that $v\in\hH_4$ implies that $\iota(v')\in \hH_4$, which in turn proves that $v'\in \hH_3$ by lemma \ref{LemHiotaiotaH}. Thus $v\in G_V\cdot\iota(\hH_3)$.

The same being true with $X$ instead of $V$, the lemma is proved.
\end{proof}

\begin{lemma}		\label{LemPasLEsDerniersAQ}
The points of $AdS_4$ of the form $v=\begin{pmatrix}
	0	\\ 
	t	\\ 
	0	\\ 
	y	\\ 
	z	
\end{pmatrix}$ do not belong to the horizon.
\end{lemma}

\begin{proof}

Since the horizon is $A$-invariant, we can reduce the lemma to the case of any element of the form $ e^{\eta J_1}v$. We have
\begin{equation}
	 e^{\eta J_1}
\begin{pmatrix}
	0	\\ 
	t	\\ 
	0	\\ 
	y	\\ 
	z	
\end{pmatrix}=
\begin{pmatrix}
 1	&	0		&	0	&	0		&	0\\ 
 0	&	\cosh(\eta)	&	0	&	\sinh(\eta)	&	0\\ 
 0	&	0		&	1	&	0		&	0\\ 
 0	&	\sinh(\eta)	&	0	&	\cosh(\eta)	&	0\\ 
 0	&	0		&	0	&	0		&	1 
 \end{pmatrix}
\begin{pmatrix}
	0	\\ 
	t	\\ 
	0	\\ 
	y	\\ 
	z	
\end{pmatrix}=
\begin{pmatrix}
	0				\\ 
	\cosh(\eta)t+\sinh(\eta)y	\\ 
	0				\\ 
	\sinh(\eta)t+\cosh(\eta)y	\\ 
	z
\end{pmatrix}
\end{equation}
We annihilate the $y$ component by choosing $\eta=\ln\left( \frac{ t-y }{ t+y } \right)$. Notice that $t^2-y^2>0$, thus we have $| t |>| y |$ and the expression in the logarithm is always positive.

A representative of $(0,t,0,0,z)$ in $\SO(2,2)$ is easy to find, and the geodesic in the direction $\bar w\in S^2$ is given by
\begin{equation}
	\begin{pmatrix}
 0	&	1	&	0	&	0	&	0\\ 
 t	&	0	&	0	&	0	&	-z\\ 
 0	&	0	&	1	&	0	&	0\\ 
 0	&	0	&	0	&	1	&	0\\ 
z	&	0	&	0	&	0	&	-t 
 \end{pmatrix}
\begin{pmatrix}
	1	\\ 
	-s	\\ 
	sw_1	\\ 
	sw_2	\\ 
	sw_3	
\end{pmatrix}=
\begin{pmatrix}
	.	\\ 
	t-szw_3	\\ 
	.	\\ 
	sw_2	\\ 
	.	
\end{pmatrix}.
\end{equation}
It belongs to the singularity when $s$ takes one of the values
\begin{equation}
	s_{\pm}=\frac{ t }{ w_3z\pm w_2 }.
\end{equation}
As long as $|w_2|<|w_3z|$, the two values $s_{\pm}$ have the same sign, which can be decided by making $w_3$ positive or negative. That provides an open set in $S^2$ of directions which escape the singularity, so that $v\notin\hH_4$.
\end{proof}

%klklklmkmlkklmmlkkmlkmùlmklkll

\begin{theorem}		\label{ThoEqHorQCoore}
The horizon of $AdS_4$ is given by
\begin{equation}		\label{EqHorQ}
	\hH_4=G_V\cdot \iota(\hH_3)\cup G_X\cdot\iota(\hH_3),
\end{equation}
i.e. an union of lateral classes of the horizon of $AdS_3$ by one dimensional subgroups of $N$ and $\bar N$. 
\end{theorem}

\begin{proof}
Lemma \ref{LemPresqueHOrQadp} proves that all points of the horizon with $u$ and $x$ not both vanishing are of the form \eqref{EqHorQ}, while lemma \ref{LemPasLEsDerniersAQ} shows that, in fact, there are no such points in $\hH_4$. This proves the inclusion in the direct sense.

For the reverse sense, corollary \ref{CorHansDansH} shows that $\iota(\hH_3)\subset\hH_4$. Now, because $ e^{\alpha V}$ and $ e^{\beta X}$ do not change the $t$ and $y$ component of the vector on which they acts, we have $G_V\cdot\hH_4\subset\hH_4$ and $G_X\cdot\hH_4\subset\hH_4$.
\end{proof}


