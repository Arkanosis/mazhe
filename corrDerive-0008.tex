% This is part of Outils mathématiques
% Copyright (c) 2012
%   Laurent Claessens
% See the file fdl-1.3.txt for copying conditions.

\begin{corrige}{Derive-0008}

    Les dérivées partielles premières sont un calcul usuel :
    \begin{verbatim}
sage: f(x,y,z)=x**2*y*z+2*y**2*sin(x*y*z)                                                                                                                   
sage: f.diff(x)
(x, y, z) |--> 2*y^3*z*cos(x*y*z) + 2*x*y*z
sage: f.diff(y)
(x, y, z) |--> 2*x*y^2*z*cos(x*y*z) + x^2*z + 4*y*sin(x*y*z)
sage: f.diff(z)
(x, y, z) |--> 2*x*y^3*cos(x*y*z) + x^2*y
    \end{verbatim}
    C'est à dire
    \begin{subequations}
        \begin{align}
            \frac{ \partial f }{ \partial x }&=2xyz+2y^3z\cos(xyz)\\
            \frac{ \partial f }{ \partial y }&=x^2z+4y\sin(xyz)+2xy^2z\cos(xyz)  \\
            \frac{ \partial f }{ \partial z }&=2xy^3\cos(xyz)+x^2y.
        \end{align}
    \end{subequations}
    Le rotationnel de \( F=\nabla f\) est nul parce c'est la dérivée d'un potentiel (même pas besoin de faire des calculs).

    En ce qui concerne la circulation, nous utilisons le fait que nous connaissons le potentiel :
    \begin{equation}
        \int_{\sigma}\nabla f=f(1,1,\frac{ \pi }{2})-f(1,2,\pi)=-\frac{ 3\pi }{2}+2.
    \end{equation}

\end{corrige}
