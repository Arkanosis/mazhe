% This is part of the Exercices et corrigés de mathématique générale.
% Copyright (C) 2009-2010
%   Laurent Claessens
% See the file fdl-1.3.txt for copying conditions.


\begin{corrige}{Mars20100001}

	Si on veut un \emph{rang} égal à deux, il faut que la matrice \emph{contienne} une sous matrice $2\times 2$ de déterminant non nul (le plus simple est la matrice identité). Le plus simple pour obtenir un \emph{déterminant} nul est d'avoir une ligne de zéros.

	On peut donc considérer cette matrice :
	\begin{equation}
		\begin{pmatrix}
			1	&	0	&	0	\\
			0	&	1	&	0	\\
			0	&	0	&	0
		\end{pmatrix}.
	\end{equation}
	Il y a énormément d'autres exemples possibles. En tout cas, il faut une matrice carré (pour que le concept de déterminant ait un sens) de taille au moins égale à trois.

\end{corrige}
