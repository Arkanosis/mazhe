\begin{exercice}\label{exoSC_serie1-0005}

	On considère la combustion du propane
	% Merci à qui peut mettre cette équation un peu plus belle avec un paquet spécial de chimie.
	\begin{equation}
		C_3H_5+5O_2+(\text{air en excès})\to 3CO_2+4H_20+(\text{air en excès})
	\end{equation}
	en présence d'un excès d'air de $25\%$, ce qui signifie que l'air fournit est égal à $125\%$ de ce qui est requis pour une combustion complète. On demande de calculer le nombre de moles d'air nécessaires à l'entrée pour $100$ moles de gaz sortant (celui-ci étant composé de $CO_2$, de $H_20$, de $O_2$ et de $N_2$). Pour répondre à cette question, on notera
	\begin{itemize}

		\item
			$P$ le nombre de moles de propane entrant;
		\item
			$A$ le nombre de moles d'air entrant;
		\item
			$C$ le nombre de moles de $CO_2$ sortant;
		\item
			$W$ le nombre de moles de $H_2O$ sortant;
		\item
			$N$ le nombre de moles de $N_2$ sortant;
		\item
			$X$ le nombre de moles $O_2$ sortant;

	\end{itemize}
	toutes ces quantités sont pour $100$ moles de gaz sortant.
	\begin{enumerate}

		\item
			Montrer que ces quantités sont liées par les équations suivantes:
			\begin{subequations}
				\begin{numcases}{}
					3P=C\\
					4P=W\\
					0.21A=C+\frac{ W }{2}+X\\
					0/79A=N\\
					0.21 A=(1.25)(5P)\
					C+W+N+X=100
				\end{numcases}
			\end{subequations}
			(on considère que l'air entrant est composé de $21\%$ de $O_2$ et de $79\%$ de $N_2$).

		\item
			Résoudre ce système et déterminer en particulier $A$.

	\end{enumerate}

\corrref{SC_serie1-0005}
\end{exercice}
