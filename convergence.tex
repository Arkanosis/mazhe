% This is part of the Exercices et corrigés de CdI-2.
% Copyright (C) 2008, 2009, 2012
%   Laurent Claessens
% See the file fdl-1.3.txt for copying conditions.


%+++++++++++++++++++++++++++++++++++++++++++++++++++++++++++++++++++++++++++++++++++++++++++++++++++++++++++++++++++++++++++
					\section[Intégrales de fonctions et domaines non bornées]{Intégrales de fonctions non bornées sur des domaines non bornés}
%+++++++++++++++++++++++++++++++++++++++++++++++++++++++++++++++++++++++++++++++++++++++++++++++++++++++++++++++++++++++++++

%---------------------------------------------------------------------------------------------------------------------------
					\subsection[Fonctions et ensembles non bornés]{Intégrales de fonctions non bornées sur des ensembles non bornés}
%---------------------------------------------------------------------------------------------------------------------------

Soit $f\colon \eR^n\to \overline{ \eR }$, une fonction positive. On dit qu'elle est \defe{intégrable}{intégrable (fonction)} sur $E\subset\eR^n$ si
\begin{enumerate}
    \item $\forall r>0$, la fonction $f_r(x)=f(x)\mtu_{f<r}$ est intégrable sur $E_r$;
\item la limite $\lim_{r\to\infty}\int_{E_r}f_r$ est finie.
\end{enumerate}
Dans ce cas, on pose 
\begin{equation}
	\int_Ef=\lim_{r\to\infty}\int_{E_r}f_r.
\end{equation}

\begin{theorem}[Page I.38]		\label{ThoFnTestIntnnBorn}
Soit $E$ mesurable dans $\eR^n$ et $f\colon E\to \overline{ \eR }$. Si $f$ est mesurable et si il existe $g\colon E\to \overline{ \eR }$ intégrable sur $E$ telle que $| f(x) |\leq g(x)$ pour tout $x\in E$, alors $f$ est intégrable sur~$E$.

Réciproquement, si $f$ est intégrable sur $E$, alors $f$ est mesurable.
\end{theorem}

%---------------------------------------------------------------------------------------------------------------------------
					\subsection{Passage à la limite sous le signe intégral}
%---------------------------------------------------------------------------------------------------------------------------

Un autre résultat très important pour l'étude de l'intégrabilité est le théorème de la \defe{convergence dominée de Lebesgue}{}:
\begin{theorem}
	Soit $E\subset \eR^n$ un ensemble mesurable et $\{ f_k \}$, une suite de fonctions intégrables sur $E$ qui converge simplement vers une fonction $f\colon E\to \overline{ \eR }$. Supposons qu'il existe une fonction $g$ intégrable sur $E$ telle que pour tout $k$,
\begin{equation}
	| f(x) |\leq g(x)
\end{equation}
pour tout $x\in E$. Alors $f$ est intégrable sur $E$ et 
\begin{equation}
	\int_Ef=\lim_{k\to\infty}\int_Ef_k.
\end{equation}
\end{theorem}

%---------------------------------------------------------------------------------------------------------------------------
					\subsection{Théorème de Fubini et changement de variables}
%---------------------------------------------------------------------------------------------------------------------------

\begin{theorem}[Fubini]\index{théorème!Fubini}\index{Fubini!théorème}		\label{ThoFubini}
Soit $(x,t)\mapsto f(x,y)\in\bar \eR$ une fonction intégrable sur $B_n\times B_m\subset\eR^{n+m}$ où $B_n$ et $B_m$ sont des ensembles mesurables de $\eR^n$ et $\eR^m$. Alors :
\begin{enumerate}
\item pour tout $x\in B_n$, sauf éventuellement en les points d'un ensemble $G\subset B_n$ de mesure nulle, la fonction $y\in B_m\mapsto f(x,y)\in\bar\eR$ est intégrable sur $B_m$
\item
la fonction
\begin{equation}
	x\in B_n\setminus G\mapsto\int_{B_m}f(x,y)dy\in\eR
\end{equation}
est intégrable sur $B_n\setminus G$

\item 
On a
\begin{equation}
	\int_{B_n\times B_m}f(x,y)dxdy=\int_{B_n}\left( \int_{B_m}f(x,y)dy \right)dx.
\end{equation}

\end{enumerate}
\end{theorem}

Notons en particulier que si $f(x,y)=\varphi(x)\phi(y)$, alors $\int_{B_m}\varphi(y)dy$ est une constante qui peut sortir de l'intégrale sur $B_n$, et donc
\begin{equation}		\label{EqFubiniFactori}
	\int_{B_n\times B_m}\varphi(x)\phi(y)dxdy=\int_{B_n}\varphi(x)dx\int_{B_m}\phi(y)dy.
\end{equation}

%---------------------------------------------------------------------------------------------------------------------------
					\subsection{Intégrale en dimension un}
%---------------------------------------------------------------------------------------------------------------------------

\begin{proposition}[Critère de comparaison]
Soit $f$ mesurable sur $]a,\infty[$ et bornée sur tout $]a,b]$, et supposons qu'il existe un $X_0\geq a$, tel que sur $]X_0,\infty[$,
\begin{equation}
	| f(x) |\leq g(x)
\end{equation}
où $g(x)$ est intégrable. Alors $f(x)$ est intégrable sur $]a,\infty[$.
\end{proposition}

\begin{corollary}[Critère d'équivalence]
Soient $f$ et $g$ des fonctions mesurables et positives ou nulles sur $]a,\infty[$, bornées sur tout $]a,b]$, telles que 
\begin{equation}
	\lim_{x\to\infty}\frac{ f(x) }{ g(x) }=L
\end{equation}
existe dans $\bar\eR$.
\begin{enumerate}
\item Si $L\neq\infty$ et $\int_{a}^{\infty}g(x)$ existe, alors $\int_a^{\infty}f(x)dx$ existe,
\item Si $L\neq 0$ et si $\int_a^{\infty}f(x)dx$ existe, alors $\int_a^{\infty}g(x)dx$ existe,
\end{enumerate}
\end{corollary}

\begin{corollary}[Critère des fonctions test]			\label{CorCritFonsTest}
Soit $f(x)$ une fonction mesurable et positive ou nulle sur $]a,\infty[$ et bornée pour tout $]a,b]$. Nous posons
\begin{equation}
	L(\alpha)=\lim_{x\to\infty}x^{\alpha}f(x),
\end{equation}
et nous supposons qu'elle existe.
\begin{enumerate}
\item Si il existe $\alpha>1$ tel que $L(\alpha)\neq\infty$, alors $\int_a^{\infty}f(x)dx$ existe,
\item Si il existe $\alpha\leq1$ et $L(\alpha)\neq 0$, alors $\int_a^{\infty}f(x)dx$ n'existe pas.
\end{enumerate}
\end{corollary}

\begin{corollary}		\label{CorAlphaLCasInteabf}
	Soit $f\colon ]a,b]\to \eR$ une fonction mesurable, positive ou nulle, et bornée sur $[a+\epsilon,b]$ $\forall\epsilon>0$. Si $\lim_{x\to a}(x-a)^{\alpha}f(x)=L$ existe, alors
	\begin{enumerate}
		\item Si $\alpha<1$ et $L\neq\infty$, alors $\int_a^bf(x)dx$ existe,
		\item Si $\alpha\geq 1$ et $L\neq 0$, alors $\int_a^bf(x)dx$ n'existe pas.
	\end{enumerate}
\end{corollary}

%---------------------------------------------------------------------------------------------------------------------------
					\subsection{Intégrales convergentes}
%---------------------------------------------------------------------------------------------------------------------------

\begin{definition}

Soit $f$, une fonction mesurable sur $[a,\infty[$, bornée sur tout intervalle $[a,b]$. On dit que l'intégrale
\begin{equation}
	\int_a^{\infty}f(x)dx
\end{equation}
\defe{converge}{intégrale!convergente} si la limite
\begin{equation}		\label{EqDEfConvergeZeroInftX}
	\lim_{X\to\infty}\int_a^{X}f
\end{equation}
existe et est finie.
\end{definition}


%+++++++++++++++++++++++++++++++++++++++++++++++++++++++++++++++++++++++++++++++++++++++++++++++++++++++++++++++++++++++++++
					\section{Fonctions définies par des intégrales et régularisation}
%+++++++++++++++++++++++++++++++++++++++++++++++++++++++++++++++++++++++++++++++++++++++++++++++++++++++++++++++++++++++++++
%+++++++++++++++++++++++++++++++++++++++++++++++++++++++++++++++++++++++++++++++++++++++++++++++++++++++++++++++++++++++++++
%					\section[Fonctions définies par des intégrales]{Fonctions définies par des intégrales et régularisation}

%+++++++++++++++++++++++++++++++++++++++++++++++++++++++++++++++++++++++++++++++++++++++++++++++++++++++++++++++++++++++++++

Supposons $A\subset\eR^m$  et $B\subset\eR^n$ compact. Nous considérons $f(x,t)\colon A\times B\to \eR$ une fonction bornée sur $A\times B$ et intégrable par rapport à $t$ pour tout $x\in A$. Soit $F\colon A\to \eR$ définie par
\begin{equation}
	F(x)=\int_Bf(x,t)dt.
\end{equation}

%---------------------------------------------------------------------------------------------------------------------------
					\subsection{Fonction définies par une intégrale sur un compact}
%---------------------------------------------------------------------------------------------------------------------------

\begin{proposition}		\label{PropDerrSSIntegraleDSD}
Supposons $A\subset\eR^m$ ouvert et $B\subset\eR^n$ compact. Si pour un $i\in\{ i,\ldots,n \}$, la dérivée partielle $\frac{ \partial f }{ \partial x_i }$ existe dans $A\times B$ et est continue, alors $\frac{ \partial F }{ \partial x_i }$ existe dans $A$, est continue et
\begin{equation}
	\frac{ \partial F }{ \partial x_i }=\int_B\frac{ \partial f }{ \partial x_i }(x,t)dt,
\end{equation}
l'égalité signifie que l'on peut \og dériver sous le signe intégral\fg.
\end{proposition}

%---------------------------------------------------------------------------------------------------------------------------
					\subsection{Intégrale sur un segment variable}
%---------------------------------------------------------------------------------------------------------------------------


\begin{proposition}		\label{PropDerrFnAvecBornesFonctions}
Soit $f(x,t)$ une fonction continue sur $[\alpha,\beta]\times[a,b]$, telle que $\frac{ \partial f }{ \partial x }$ existe et soit continue sur $]\alpha,\beta[\times[a,b]$. Soient $\varphi(x)$ et $\psi(x)$, des fonctions continues de $[\alpha,\beta]$ dans $\eR$ et admettant une dérivée continue sur $]\alpha,\beta [$. Alors la fonction
\begin{equation}
	F(x)=\int_{\varphi(x)}^{\psi(x)}f(x,t)dt
\end{equation}
admet une dérivée continue sur $]\alpha,\beta[$ et
\begin{equation}	\label{EqFormDerrFnAvecBorneNInt}
	\frac{ dF }{ dx }=\int_{\varphi(x)}^{\psi(x)}\frac{ \partial f }{ \partial x }(x,t)dt+f\big( x,\psi(x) \big)\cdot\frac{ d\psi }{ dx }- f\big( x,\varphi(x) \big)\cdot\frac{ d\varphi }{ dx }.
\end{equation}
\end{proposition}

%---------------------------------------------------------------------------------------------------------------------------
					\subsection{Intégrales convergentes}
%---------------------------------------------------------------------------------------------------------------------------

\begin{theorem}		\label{ThoInDerrtCvUnifFContinue}
	Uniforme convergence, continuité et dérivation.
	\begin{enumerate}
		\item 
		Soit $f(x,t)$ continue sur $[\alpha,\beta]\times[a,\alpha[$ et $F(x)=\int_a^{\infty}f(x,t)dt$; cette intégrale étant supposée uniformément convergente. Alors $F(x)$ est continue.
		\item
		Supposons $f$ continue et sa dérivée partielle $\frac{ \partial f }{ \partial x }$ continue sur $[\alpha,\beta]\times[a,\alpha[$. Supposons que $F(x)=\int_a^{\infty}f(x,t)dt$ converge et que $\int_a^{\infty}\frac{ \partial f }{ \partial x }dt$ converge uniformément. Alors $F$ est $C^1$ sur $[\alpha,\beta]$ et 
		\begin{equation}
			\frac{ dF }{ dx }=\int_a^{\infty}\frac{ \partial f }{ \partial x }dt.
		\end{equation}
	\end{enumerate}
\end{theorem}
%---------------------------------------------------------------------------------------------------------------------------
					\subsection{Critères de convergence uniforme}
%---------------------------------------------------------------------------------------------------------------------------


Affin de tester l'uniforme convergence d'une intégrale, nous avons le \defe{critère de Weierstrass}{Critère!Weierstrass}:
\begin{theorem}		\label{ThoCritWeiIntUnifCv}
Soit $f(x,t)\colon [\alpha,\beta]\times[a,\infty[ \to \eR$, une fonction dont la restriction à toute demi-droite $x=cst$ est mesurable. Si $| f(x,t) |< \varphi(t)$ et $\int_a^{\infty}\varphi(t)dt$ existe, alors l'intégrale
\begin{equation}
	\int_0^{\infty}f(x,t)dt
\end{equation}
est uniformément convergente.
\end{theorem}


Le théorème suivant est le \defe{critère d'Abel}{Critère!Abel pour intégrales} :
\begin{theorem}		\label{ThoAbelIntUnif}
	Supposons que $f(x,t)=\varphi(x,t)\psi(x,t)$ où $\varphi$ et $\psi$ sont bornée et intégrables en $t$ au sens de Riemann sur tout compact $[a,b]$, $b\geq a$. Supposons que :
	\begin{enumerate}
		\item $| \int_a^{T}\varphi(x,t)dt |\leq M$ où $M$ est indépendant de $T$ et de $x$,
		\item $\psi(x,t)\geq 0$,
		\item pour tout $x\in[\alpha,\beta]$, $\psi(x,t)$ est une fonction décroissante de $t$,
		\item les fonctions $x\mapsto \psi(x,t)$ convergent uniformément vers $0$ lorsque $t\to\infty$.
	\end{enumerate}
	Alors l'intégrale
	\begin{equation}
		\int_a^{\infty}f(x,t)dt
	\end{equation}
	est uniformément convergente.
\end{theorem}
