% This is part of (almost) Everything I know in mathematics
% Copyright (c) 2015
%   Laurent Claessens
% See the file fdl-1.3.txt for copying conditions.

%+++++++++++++++++++++++++++++++++++++++++++++++++++++++++++++++++++++++++++++++++++++++++++++++++++++++++++++++++++++++++++ 
\section{The conformal condition}
%+++++++++++++++++++++++++++++++++++++++++++++++++++++++++++++++++++++++++++++++++++++++++++++++++++++++++++++++++++++++++++

%--------------------------------------------------------------------------------------------------------------------------- 
\subsection{Preliminary discussion}
%---------------------------------------------------------------------------------------------------------------------------

If one looks at \cite{ooIYOHooMRMfXl,ooDPRUooOFPyPH}, one sees that a conformal transformation is a transformation of a (pseudo)riemannian manifold that leaves the metric unchanged up to a positive scalar function : \( g'_{\mu\nu}(x')=\Omega(x)g_{\mu\nu}(x)\). Thus we are looking for the maps \( \phi\colon M\to M\) that realise that condition.

Since our objective is to do differential geometry, we cannot follow the computations in \cite{ooIYOHooMRMfXl,ooDPRUooOFPyPH} because there are too much «infinitesimal» in that and we don't understand anything\footnote{No pun intended : these books are very okay, but as far as we are interested in bundles, there are some works to do.}.

In order to determine the Poincaré group that leaves the metric invariant, we were fortunate because of theorem \ref{ThoDsFErq} that ensured linearity of the map \( \phi\). Our search for the Poincaré group\footnote{By the way given by theorem \ref{THOooQJSRooMrqQct}.} was thus simplified by two circumstances :
\begin{itemize}
    \item \( \phi\) and \( d\phi\) are the same.
    \item the condition \( g_{\mu\nu}=g'_{\mu\nu}\) does not involves a specific point, so that we had not to ask ourself questions about the «base point» of the vectors.
\end{itemize}
Here we are in a more complicated situation. 

\begin{normaltext}
Here is our first try. Let \( (M,g)\) be a (pseudo)riemannian manifold. A conformal map will be \( \phi\colon M\to M\) such that
\begin{equation}        \label{EQooCCMMooXVbTAd}
    g_{\phi(x)}\big( d\phi_xv,d\phi_xw \big)=\Omega(x)g_x(v,w)
\end{equation}
for a function \( \Omega\in C^{\infty}(M)\). The condition \eqref{EQooCCMMooXVbTAd} has to hold for every \( x\in M\) and \( v,w\in T_xM\).
\end{normaltext}
    

\begin{normaltext}
Well. This is not the correct definition. Let us be clear : the set of maps satisfying \eqref{EQooCCMMooXVbTAd} is for sure interesting, but this is not the what we call the conformal group.

The reason is that we want to encode a deforming material which respects the angles. A vector at point \( A\) is an arrow joining point \( A\) to a point \( B\). The image of the vector \( \vect{ AB }\) has to be \( \vect{ \phi(A)\phi(B) }\) when the material is deformed. Thus vectors have to be transported by \( \phi\) instead of \( d\phi\). We could speak about affine spaces as described around definition \ref{DEFooQELZooEXvxgw}, but instead we will describe our subject with a vector bundle.
\end{normaltext}

%--------------------------------------------------------------------------------------------------------------------------- 
\subsection{The setting}
%---------------------------------------------------------------------------------------------------------------------------

Let \( V\) be a finite dimensional vector space and \( E=V\times V\) be the trivial vector bundle with fibre \( V\). We denote \( V_x=\{ (x,v)\tq v\in V \}\) and for each \( x\in V\) we have a non-degenerate bilinear form \( g_x\colon V\to V\). We define
\begin{equation}
    g\big( (x,v),(x,w) \big)=g_x(v,w)
\end{equation}
and we will often directly write \( g_x(v,w)\) or \( v\cdot w\) when \( v,w\) belong to \(V_x\) instead of \( V\).

A map \( \phi\colon V\to V\) also acts on the vector bundle as
\begin{equation}
    \phi(x,v)=\big( \phi(x),\phi(x+v)-\phi(x) \big).
\end{equation}
This way to act translates the fact that for a vector, we displace the ending point as well as the starting point with \( \phi\). This is not the same as displacing the vector by \( d\phi_x\).

\begin{definition}      \label{DEFooVKNBooFBWQQM}
    A \defe{conformal map}{conformal map} is a \(  C^{\infty}\) map \( \phi\colon V\to V\) for which there exists a function \( \Omega\in C^{\infty}(V)\) satisfying
    \begin{equation}        \label{EQooOZDUooCDaIrh}
        v\cdot w=\Omega(x) \phi(v)\cdot \phi(w)
    \end{equation}
    for every \( x\in V\) and every \( v,w\in V_x\).
\end{definition}
With no abuse of notations, the condition \eqref{EQooOZDUooCDaIrh} reads, for \( v,w\in V\) :
\begin{equation}\label{EQooFZUFooTGWpBn}
    g_x(v,w)=\Omega(x)g_{\phi(x)}\big(  \phi(x+v)-\phi(x),\phi(x+w)-\phi(x)  \big).
\end{equation}

Moreover we consider the case in which the metric is flat an \( g_x=\eta\) for every \( x\).

\begin{normaltext}
    We are going not to determine all the conformal transformations\quext{Can you answer the question \url{http://math.stackexchange.com/questions/1549670/rigorous-definition-of-a-generator-for-a-transformation-group} ?}, but the (connected to the identity) Lie group of conformal transformations. That is : we are searching for a Lie group \( G\) of diffeomorphisms \( V\to V\) satisfying the condition \eqref{EQooFZUFooTGWpBn}.
\end{normaltext}

Since \( G\) is a Lie group, each element can be written under the form \( g= e^{X}\) for some \( X\in\lG\) (here \( \lG\) is the Lie algebra, that is \( T_eG\)). Following the definition \ref{DEFooUYOZooWdcClz} we define \( X\colon V\to V\) by
\begin{equation}
    X(v)=\Dsdd{  \exp(-tX)(v) }{t}{0}.
\end{equation}
The exponential here is the one from the Lie algebra to the Lie group. 


