% This is part of Exercices et corrigés de CdI-1
% Copyright (c) 2011,2014,2016
%   Laurent Claessens
% See the file fdl-1.3.txt for copying conditions.

\begin{exercice}\label{exo0060}

Soit $F\colon \eR^2\to \eR^2$,  $(x,t) \mapsto F(x,t)$ une fonction de classe $C^2$. On dit que $F$ satisfait l'équation des cordes vibrantes (ou équation des ondes), si 
\begin{equation}		\label{EqONde0060}
	\frac{\partial^2 F}{\partial t^2 } = c^2 \frac{\partial^2 F}{\partial x^2}
\end{equation}
 où $c$ représente la vitesse de propagation de l'onde.

\begin{enumerate}
\item
Prouvez que si $f$ et $g$ sont deux solutions de l'équation et $a$ et $b$ deux réels alors $af+bg$ est  encore une solution de l'équation.  (l'ensemble des solutions de l'équation est un sous vectoriel du vectoriel des fonctions $C^2$ de $\eR^2$ vers $\eR$)
\item

Vérifiez que si $\phi$ et $\psi$ sont deux fonctions de classe $C^2$ de
$\eR$ dans $\eR$ alors
\[
F: \eR^2 \rightarrow \eR: (x,t) \rightarrow \phi(x+ct) + \psi(x-ct)
\]
est une solution de l'équation.
\item
Soit le changement de variable
\[
\left\{ \begin{array}{l} \xi = x + c t \\ \eta = x-ct \end{array} \right.
\]
Écrivez l'équation dans ces nouvelles variables et prouvez que les solutions exhibées au point précédent sont les seules possibles.

\item
Si vous êtes étudiant en physique, vous devriez être capable d'expliquer pourquoi le paramètre $c$ de l'équation d'onde est la vitesse de l'onde. Si vous êtes étudiant en mathématique, cela ne vous dispense moralement pas de vous poser la question. Remarquez l'analogie entre l'équation d'onde \eqref{EqONde0060} et l'équation 
\begin{equation}
	\Box \psi(\bar x,t)=0
\end{equation}
où $\Box =\frac{1}{ c^2 }\frac{ \partial^2 }{ \partial t^2 }-\Delta$. Cette dernière équation est l'équation (10.1.5) de la page 101 du cours d'électromagnétisme de deuxième année\cite{Schomblond_em}.

\end{enumerate}


\corrref{0060}
\end{exercice}
