% This is part of the Exercices et corrigés de mathématique générale.
% Copyright (C) 2009-2011
%   Laurent Claessens
% See the file fdl-1.3.txt for copying conditions.


%+++++++++++++++++++++++++++++++++++++++++++++++++++++++++++++++++++++++++++++++++++++++++++++++++++++++++++++++++++++++++++
\section{Techniques d'intégration}
%+++++++++++++++++++++++++++++++++++++++++++++++++++++++++++++++++++++++++++++++++++++++++++++++++++++++++++++++++++++++++++

%---------------------------------------------------------------------------------------------------------------------------
\subsection{Reformer un carré au dénominateur}
%---------------------------------------------------------------------------------------------------------------------------
\label{subsecCarreDenoPar}

Lorsqu'on a un second degré au dénominateur, le bon plan est de reformer un carré parfait. Par exemple : 
\begin{equation}
	x^2+2x+2=(x+1)^2+1.
\end{equation}
Ensuite, le changement de variable $t=x+1$ est pratique parce que cela donne $t^2+1$ au dénominateur.

Cherchons
\begin{equation}
	I=\int \frac{ 1-x }{ x^2+2x+2 }dx=\int\frac{ 1-x }{ (x+1)^2+1 }dx=\int\frac{ 1-(t-1) }{ t^2+1 }
\end{equation}
où nous avons fait le changement de variable $t=x+1$, $dt=dx$. L'intégrale se coupe maintenant en deux parties :
\begin{equation}
	I=\int\frac{ -t }{ t^2+1 }+\int \frac{ 2 }{ t^2+1 }.
\end{equation}
La seconde est dans les formulaires et vaut 
\begin{equation}
	2\arctan(t)=2\arctan(x+1),
\end{equation}
tandis que la seconde est presque de la forme $f'/f$ :
\begin{equation}
	\int\frac{ t }{ t^2+1 }=\frac{ 1 }{2}\int \frac{ 2t }{ t^2+1 }=\frac{ 1 }{2}\ln(t^1+1)=\frac{ 1 }{2}\ln(u^2+2u+2).
\end{equation}

%+++++++++++++++++++++++++++++++++++++++++++++++++++++++++++++++++++++++++++++++++++++++++++++++++++++++++++++++++++++++++++
					\section{Primitives et surfaces}
%+++++++++++++++++++++++++++++++++++++++++++++++++++++++++++++++++++++++++++++++++++++++++++++++++++++++++++++++++++++++++++

Soit $f\colon \eR\to \eR$, une fonction continue, et $x\in\eR$. Pour chaque $x\in\eR$, nous pouvons considérer le nombre $F(x)$ défini par
\begin{equation}
	F(x)=\int_a^x f(t)dt.
\end{equation}
\newcommand{\CaptionFigSurfacePrimitive}{La primitive décrit la surface}
\input{Fig_SurfacePrimitive.pstricks}
La fonction $F$ ainsi définie a deux importantes propriétés :
\begin{enumerate}

\item
C'est une primitive de $f$,
\item
Elle donne la surface en dessous de $f$ entre les points $a$ et $x$, voir la figure \ref{LabelFigSurfacePrimitive}.

\end{enumerate}

Notons que tant que $f$ est positive, la surface est croissante.


La manière de calculer la surface comprise entre deux fonctions est dessinée à la figure \ref{LabelFigSurfaceEntreCourbes}.
\newcommand{\CaptionFigSurfaceEntreCourbes}{Le calcul de la surface comprise entre deux fonctions.}
\input{Fig_SurfaceEntreCourbes.pstricks}
La surface entre les deux fonctions $y_1(x)$ et $y_2(x)$ se calcule comme suit.
\begin{enumerate}

\item
On calcule les intersections entre $y1$ et $y_2$. Notons $a$ et $b$ les ordonnées obtenues.
\item
La surface demandée est la différence entre la surface sous la fonction $y_1$ (la plus grande) et la surface sous la fonction $y_2$ (la plus petite), donc
\begin{equation}
	S=\int_{a}^by_1-\int_a^by_1.
\end{equation}

\end{enumerate}

%---------------------------------------------------------------------------------------------------------------------------
					\subsection{Longueur d'arc de courbe}
%---------------------------------------------------------------------------------------------------------------------------

La longueur de l'arc de courbe de la fonction $y=f(x)$ entre les abscisses $x_0$ et $x_1$ est donné par la formule
\begin{equation}		\label{EqLongArcCourbe}
	l(x_0,x_1)=\int_{x_0}^{x_1}\sqrt{1+y'(t)^2}dt.
\end{equation}

Lorsque la courbe est donnée sous forme paramétrique
\begin{subequations}
\begin{numcases}{}
	x=x(t)\\
	y=y(t),
\end{numcases}
\end{subequations}
alors la formule devient
\begin{equation}		\label{EqLongArcParam}
	l(t_1,t_2)=\int_{t_1}^{t_2}\sqrt{\dot x(t)^2+\dot y(t)^2}dt,
\end{equation}
où $\dot x(t)=x'(t)$.

%---------------------------------------------------------------------------------------------------------------------------
					\subsection{Aire de révolution}
%---------------------------------------------------------------------------------------------------------------------------

Pour savoir l'aire engendrée par la ligne $y=f(x)$ entre $a$ et $b$ autour de l'axe $Ox$, on utilise la formule
\begin{equation}
	S=2\pi\int_a^b\sqrt{1+f'(x)^2}f(x)dx.
\end{equation}


%+++++++++++++++++++++++++++++++++++++++++++++++++++++++++++++++++++++++++++++++++++++++++++++++++++++++++++++++++++++++++++
					\section{Équations différentielles}
%+++++++++++++++++++++++++++++++++++++++++++++++++++++++++++++++++++++++++++++++++++++++++++++++++++++++++++++++++++++++++++

%---------------------------------------------------------------------------------------------------------------------------
					\subsection{Équations à variables séparées}
%---------------------------------------------------------------------------------------------------------------------------

Ce sont les équations pour lesquelles on peut mettre tous les $y$ d'un côté. Elles se présentent sous la forme
\begin{equation}
	y'=u(x)f(y).
\end{equation}
On peut évidement mettre tous les $y$ et $y'$ d'un côté :
\begin{equation}
	\frac{ y' }{ f(y) }=u(x).
\end{equation}
Une fois que cela est fait, on écrit $y'=\frac{ dy }{ dx }$, et on envoie le $dx$ du côté des $x$ :
\begin{equation}
	\frac{ dy }{ f(y) }=u(x)dx.
\end{equation}
Maintenant il suffit de prendre l'intégrale des deux côtés : comme la position des $dx$ et $dy$ l'indiquent, il faut intégrer par rapport à $y$ d'un côté et par rapport à $dx$ de l'autre côté.

L'intégrale à gauche est facile : c'est $\ln(y)$. À droite, par contre, ça dépend tout à fait de $u$.

%---------------------------------------------------------------------------------------------------------------------------
					\subsection{Équations homogènes}
%---------------------------------------------------------------------------------------------------------------------------

Une équation différentielle homogène se présente sous la forme
\begin{equation}
	y'=\frac{ \text{degré $n$ en $x,y$} }{  \text{degré $n$ en $x,y$}  },
\end{equation}
avec pas de $y'$ à droite : juste du $y$ et du $x$.

Pour traiter une équation différentielle homogène, le bon plan est de changer de fonction inconnue et poser
\begin{equation}		\label{EqSubstHomouyp}
	u(x)=\frac{ y(x) }{ x },
\end{equation}
ce qui fait $y=ux$ et $y'(x)=u(x)+xu'(x)$, à replacer dans l'équation de départ.


%---------------------------------------------------------------------------------------------------------------------------
					\subsection{Équations linéaires}
%---------------------------------------------------------------------------------------------------------------------------

Tant qu'il n'y a pas de second membre, c'est facile. Prenons l'exemple suivant :
\begin{equation}
	y'+2xy=0.
\end{equation}
Nous mettons tous les $x$ d'un côté et tous les $y$ et $y'$ de l'autre :
\begin{equation}
	\frac{ y' }{ y }=-2x,
\end{equation}
et puis on intègre sans oublier la constante d'intégration :
\begin{equation}
	\ln(y)=-x^2+C,
\end{equation}
et donc $y(x)=K e^{-x^2}$.

Lorsqu'il y a un second membre, il y a une astuce. Prenons par exemple
\begin{equation}		\label{EqDiffExLin}
	y'+2xy=4x.
\end{equation}
L'astuce est de commencer par résoudre l'équation sans le second membre (l'équation homogène associée). Nous notons $y_H$ la solution. Ici, la réponse est
\begin{equation}
	y_H(x)=K e^{-x^2}.
\end{equation}
Ensuite le truc est d'essayer de trouver la solution de l'équation \eqref{EqDiffExLin} sous la forme
\begin{equation}		\label{EqEssaiLin}
	y(x)=K(x) e^{x^2}.
\end{equation}
L'idée est de prendre la même que la solution de l'équation homogène (sans second membre), mais en disant que $K$ est une fonction. Afin de trouver la fonction $K$ qui donne la solution, il suffit de remettre l'essai \eqref{EqEssaiLin} dans l'équation \eqref{EqDiffExLin} :
\begin{equation}
	\underbrace{K' e^{-x^2}-2xK e^{-x^2}}_{y'(x)}+\underbrace{2xK e^{-x^2}}_{2xy(x)}=4x
\end{equation}
Les deux termes avec $K$ se simplifient et il reste
\begin{equation}
	K'(x)=4x e^{x^2},
\end{equation}
ce qui signifie $K(x)=2 e^{x^2+C}$. Nous avons donc déterminé la fonction qui fait fonctionner l'essai, et la solution à l'équation est
\begin{equation}
	y(x)=\big( 2 e^{x^2}+C \big) e^{-x^2}=2+C e^{-x^2}.
\end{equation}

%+++++++++++++++++++++++++++++++++++++++++++++++++++++++++++++++++++++++++++++++++++++++++++++++++++++++++++++++++++++++++++
\section{Équations différentielles du second ordre}
%+++++++++++++++++++++++++++++++++++++++++++++++++++++++++++++++++++++++++++++++++++++++++++++++++++++++++++++++++++++++++++

%---------------------------------------------------------------------------------------------------------------------------
\subsection{Avec second membre}
%---------------------------------------------------------------------------------------------------------------------------

Une équation différentielle du second ordre avec un second membre se présente sous la forme
\begin{equation}
	ay''(x)+by'(x)+cy(x)=v(x)
\end{equation}
où $v(x)$ est une fonction donnée. Le truc est de commencer par résoudre l'équation différentielle sans second membre, c'est à dire trouver la fonction $y_H(x)$ telle que
\begin{equation}
	ay''_H(x)+by_H'(x)+cy_H(x)=0.
\end{equation}
Cela se fait en utilisant la méthode du polynôme caractéristique.

Ensuite, il faut trouver une solution particulière $y_P(x)$ de l'équation avec le second membre. Une seule. Pour y parvenir, il faut du doigté et un peu de technique. Il faut faire des essais en fonction de ce à quoi ressemble le $v(t)$ :
\begin{enumerate}

	\item
		Si $v(x)$ est un polynôme, alors il faut essayer un polynôme,

	\item
		Si $v(x)=\cos(\omega x)$ ou bien $v(x)=\sin(\omega x)$, alors essayer $y_P(x)=A\cos(x)+B\sin(\omega x)$,

	\item
		Si $v(x)= e^{\omega x}$, alors essayer $y_P(x)=A e^{\omega x}$.

\end{enumerate}


\section{Fonctions réelles de deux variables réelles}

Une \textbf{fonction réelle de 2 variables réelles} est une fonction $f : A \subset \eR^2 \to \eR : (x,y) \mapsto z = f(x,y)$.

Le \textbf{graphe de $f$}, noté $\Graphe f$, est un sous-ensemble de $\eR^3$:\[\Graphe f = \{(x,y,z) \in \eR^3 \mid (x,y) \in A \text{ et } z = f(x,y)\}\]

Les \textbf{courbes de niveau} de la fonction $f$ sont obtenues en posant $f(x,y)=\lambda$.

%---------------------------------------------------------------------------------------------------------------------------
\subsection{Limites de fonctions à deux variables}
%---------------------------------------------------------------------------------------------------------------------------

À peu près tout ce qu'une personne de ton âge peut savoir sur les limites à deux variables se trouve dans la référence \cite{ExoCdI1}. Ici nous n'allons pas entrer dans tous les détails, mais simplement mentionner les quelque techniques les plus courantes. 

\begin{theorem}		\label{ThoLimiteCompose}
	Soient deux fonctions $f\colon \eR^n\to \eR^p$ et $g\colon \eR^p\to \eR^q$. Si $a$ est un point adhérent au domaine de $g\circ f$ et si
	\begin{equation}
		\begin{aligned}[]
			\lim_{x\to a}f(x)&=b\\
			\lim_{y\to b}g(y)&=c,
		\end{aligned}
	\end{equation}
	alors 
	\begin{equation}
		\lim_{x\to a}(g\circ f)(x)=c.
	\end{equation}
\end{theorem}



Les techniques usuelles sont
\begin{enumerate}

	\item
		La règle de l'étau. Cette technique demande un peu plus d'imagination parce qu'il faut penser à un «truc» différent pour chaque exercice. En revanche, la justification est facile : il y a un théorème qui dit que ça marche.

	\item
		Lorsqu'on applique la règle de l'étau, penser à
		\begin{equation}
			| x |=\sqrt{x^2}\leq\sqrt{x^2+y^2}.
		\end{equation}
		Cela permet de majorer le numérateur. Attention : ce genre de majoration ne fonctionnent qu'au numérateur : agrandir le dénominateur ferait diminuer la fraction.

	\item
		Il n'est pas vrai que
		\begin{equation}
			| x |=\sqrt{x^2}\leq\sqrt{x^4}\leq\sqrt{x^4+2y^4}.
		\end{equation}
		En effet, si $x$ est petit, alors $x^2>x^4$, et non le contraire.

\end{enumerate}

Une technique très efficace pour les limites $(x,y)\to (0,0)$ est le passage aux coordonnées polaires. Il s'agit de poser
\begin{subequations}
	\begin{numcases}{}
		x=r\cos(\theta)\\
		y=r\sin(\theta)
	\end{numcases}
\end{subequations}
et puis de faire la limite $r\to 0$.

Si la limite obtenue {\bf ne dépend pas de $\theta$}, alors c'est la limite cherchée. Des exemples sont donnés dans les corrections de l'exercice \ref{exoFoncDeuxVar0010}.

%---------------------------------------------------------------------------------------------------------------------------
\subsection{Dérivées partielles}
%---------------------------------------------------------------------------------------------------------------------------


La \defe{dérivée partielle}{Dérivée partielle} par rapport à $x$ au point $(x,y)$ est notée
\begin{equation}
	\frac{\partial f}{\partial x}(x,y) 
\end{equation}
et se calcule en dérivant $f$ par rapport  à $x$ en considérant que $y$ est constante.

De la même manière, la dérivée partielle par rapport à $y$ au point $(x,y)$ est notée
\begin{equation}
	\frac{\partial f}{\partial y}(x,y) 
\end{equation}
et se calcule en dérivant $f$ par rapport  à $y$ en considérant que $x$ est constante.



Pour les dérivées partielles secondes,
\begin{itemize}
\item $f''_{xx} (x,y) = (f'_x)'_x = \frac{\partial^2 f}{\partial x^2}(x,y) = \frac{\partial}{\partial x}(\frac{\partial f}{\partial x})$.
\item $f''_{yy} (x,y) = (f'_y)'_y = \frac{\partial^2 f}{\partial y^2}(x,y) = \frac{\partial}{\partial y}(\frac{\partial f}{\partial y})$.
\item $f''_{xy} (x,y) = (f'_x)'_y  = (f'_y)'_x = f''_{yx} (x,y) \text{ ou } \frac{\partial^2 f}{\partial x \partial y}(x,y) = \frac{\partial}{\partial x}(\frac{\partial f}{\partial y})  = \frac{\partial}{\partial y}(\frac{\partial f}{\partial x}) =\frac{\partial^2 f}{\partial y \partial x}(x,y)$.
\end{itemize}

%---------------------------------------------------------------------------------------------------------------------------
\subsection{Différentielle et accroissement}
%---------------------------------------------------------------------------------------------------------------------------

La \defe{différentielle totale}{Différentielle!totale} de $f$ au point $(a,b)$ est donnée, quand elle existe (!), par la formule
\begin{equation}
	df(a,b) = \frac{\partial f}{\partial x}(a,b)dx + \frac{\partial f}{\partial y}(a,b) dy.
\end{equation}

De la même façon que la formule des accroissements finis disait que $f(x+a)\simeq f(x)+af'(x)$, en deux dimensions nous avons que l'\defe{accroissement}{Accroissement} approximatif de $f$ au point $(a,b)$ pour des accroissements $\Delta x$ et $\Delta y$ est 
\begin{equation}
	f(x+\Delta x,y+\Delta y)=f(x,y)+\Delta x\frac{ \partial f }{ \partial x }(x,y)+\Delta y\frac{ \partial f }{ \partial y }(x,y).
\end{equation}


Le \defe{plan tangent}{Plan tangent} au graphe de $f$ au point $\big(a,b,f(a,b)\big)$ est 
\begin{equation}
	T_{(a,b)}(x,y) = f(a,b) + \frac{\partial f}{\partial x}(a,b) (x-a) + \frac{\partial f}{\partial y}(a,b) (y-b)
\end{equation}
Essayez d'écrire l'équation de la droite tangente au graphe de $f(x)$ au point $x=a$ en terme de la dérivée de $f$, et comparez votre résultat à cette formule.

Un des principaux théorèmes pour tester la différentiabilité d'une fonction est le suivant.

\begin{theorem}		\label{ThoProuverDiffable}
	Soit une fonction $f\colon \eR^m\to \eR^p$. Si les dérivées partielles existent dans un voisinage de $a$ et donc continues en $a$, alors $f$ est différentiable en $a$.
\end{theorem}
Le plus souvent, nous prouvons qu'une fonction est différentiable en calculant les dérivées partielles et en montrant qu'elles sont continues.

%---------------------------------------------------------------------------------------------------------------------------
\subsection{Recherche d'extrema locaux}
%---------------------------------------------------------------------------------------------------------------------------

(ULB : Théorème 13.8.3 p. 168)
\begin{enumerate}
\item Rechercher les points critiques, càd les $(x,y)$ tels que
\[\begin{cases} \frac{\partial f}{\partial x}(x,y) = 0 \\ \frac{\partial f}{\partial y}(x,y) = 0 \end{cases} \]
En effet, si $(x_0,y_0)$ est un extrémum local de $f$, alors $\frac{\partial f}{\partial x}(x_0,y_0) = 0 = \frac{\partial f}{\partial y}(x_0,y_0)$.
\item Déterminer la nature des points critiques: «test» des dérivées secondes:
\[\text{On pose }H(x_0,y_0) = \frac{\partial^2 f}{\partial x^2}(x_0,y_0)\frac{\partial f^2}{\partial y^2}(x_0,y_0) - \left(\frac{\partial^2 f}{\partial x\partial y}(x_0,y_0)\right)^2\]
\begin{enumerate}
\item Si $H(x_0,y_0) > 0$ et $\frac{\partial^2 f}{\partial x^2}(x_0,y_0) > 0 \Longrightarrow (x_0,y_0)$ est un minimum local de $f$.
\item Si $H(x_0,y_0) > 0$ et $\frac{\partial^2 f}{\partial x^2}(x_0,y_0) < 0 \Longrightarrow (x_0,y_0)$ est un maximum local de $f$.
\item Si $H(x_0,y_0) < 0 \Longrightarrow f$ a un point de selle en $(x_0,y_0)$.
\item Si $H(x_0,y_0) = 0 \Longrightarrow$ on ne peut rien conclure.
\end{enumerate}
\end{enumerate}

\textbf{Dérivation implicite:} Soit $F(x,f(x)) = 0$ la représentation implicite d'une fonction $y=f(x)$ alors \[y' = f'(x) = - \frac{F'_x}{F'_y}.\]

%+++++++++++++++++++++++++++++++++++++++++++++++++++++++++++++++++++++++++++++++++++++++++++++++++++++++++++++++++++++++++++
\section{Méthode de Gauss pour résoudre des systèmes d'équations linéaires}
%+++++++++++++++++++++++++++++++++++++++++++++++++++++++++++++++++++++++++++++++++++++++++++++++++++++++++++++++++++++++++++


Pour résoudre un système d'équations linéaires, on procède comme suit:
\begin{enumerate}
\item Écrire le système sous forme matricielle. \[\text{p.ex. } \begin{cases} 2x+3y &= 5 \\ x+2y &= 4 \end{cases} \Leftrightarrow \left(\begin{array}{cc|c} 2 & 3 & 5 \\ 1 & 2 & 4 \end{array}\right) \]
\item Se ramener à une matrice avec un maximum de $0$ dans la partie de gauche en utilisant les transformations admissibles:
\begin{enumerate}
\item Remplacer une ligne par elle-même + un multiple d'une autre;
\[\text{p.ex. } \left(\begin{array}{cc|c} 2 & 3 & 5 \\ 1 & 2 & 4 \end{array}\right)  \stackrel{L_1  - 2. L_2 \mapsto L_1'}{\Longrightarrow} \left(\begin{array}{cc|c} 0 & -1 & -3 \\ 1 & 2 & 4 \end{array}\right) \]
\item Remplacer une ligne par un multiple d'elle-même;
\[\text{p.ex. } \left(\begin{array}{cc|c} 0 & -1 & -3 \\ 1 & 2 & 4 \end{array}\right)  \stackrel{-L_1  \mapsto L_1'}{\Longrightarrow} \left(\begin{array}{cc|c} 0 & 1 & 3 \\ 1 & 2 & 4 \end{array}\right) \]
\item Permuter des lignes.
\[\text{p.ex. } \left(\begin{array}{cc|c} 0 & 1 & 3 \\ 1 & 0 & -2 \end{array}\right)  \stackrel{L_1  \mapsto L_2' \text{ et } L_2  \mapsto L_1'}{\Longrightarrow} \left(\begin{array}{cc|c} 1 & 0 & -2 \\ 0 & 1 & 3  \end{array}\right) \]
\end{enumerate}
\item Retransformer la matrice obtenue en système d'équations.
\[\text{p.ex. }  \left(\begin{array}{cc|c} 1 & 0 & -2 \\ 0 & 1 & 3  \end{array}\right) \Leftrightarrow \begin{cases} x &= -2 \\ y &= 3 \end{cases}  \]
\end{enumerate}

\textbf{Remarques :} 
\begin{itemize}
\item Si on obtient une ligne de zéros, on peut l'enlever:
\[\text{p.ex. }  \left(\begin{array}{ccc|c} 3 & 4 & -2 & 2 \\ 4 & -1 & 3 & 0 \\ 0 & 0 & 0 & 0 \end{array}\right) \Leftrightarrow  \left(\begin{array}{ccc|c} 3 & 4 & -2 & 2 \\ 4 & -1 & 3 & 0 \end{array}\right) \]
\item Si on obtient une ligne de zéros suivie d'un nombre non-nul, le système d'équations n'a pas de solution:
\[\text{p.ex. }  \left(\begin{array}{ccc|c} 3 & 4 & -2 & 2 \\ 4 & -1 & 3 & 0 \\ 0 & 0 & 0 & 7 \end{array}\right) \Leftrightarrow  \begin{cases} \cdots \\ \cdots \\ 0x + 0y + 0z = 7 \end{cases} \Rightarrow \textbf{Impossible} \]
\item Si on moins d'équations que d'inconnues, alors il y a une infinité de solutions qui dépendent d'un ou plusieurs paramètres:
\[\text{p.ex. }  \left(\begin{array}{ccc|c} 1 & 0 & -2 & 2 \\ 0 & 1 & 3 & 0 \end{array}\right) \Leftrightarrow  \begin{cases} x - 2z = 2 \\ y + 3z = 0 \end{cases} \Leftrightarrow  \begin{cases} x = 2 + 2\lambda \\ y = -3\lambda \\ z = \lambda \end{cases} \]
\end{itemize}

%+++++++++++++++++++++++++++++++++++++++++++++++++++++++++++++++++++++++++++++++++++++++++++++++++++++++++++++++++++++++++++
\section{Matrices, applications linéaires et directions conservées}
%+++++++++++++++++++++++++++++++++++++++++++++++++++++++++++++++++++++++++++++++++++++++++++++++++++++++++++++++++++++++++++

Nous savons qu'une application \emph{linéaire} $A\colon \eR^3\to \eR^3$ est complètement définie par la donnée de son action sur les trois vecteurs de base, c'est à dire par la donnée de
\begin{equation}
	\begin{aligned}[]
		Ae_1,&&Ae_2&&\text{et}&&Ae_3.
	\end{aligned}
\end{equation}
Nous allons former la matrice de $A$ en mettant simplement les vecteurs $Ae_1$, $Ae_2$ et $Ae_3$ en colonne. Donc la matrice
\begin{equation}		\label{EqExempleALin}
	A=\begin{pmatrix}
		3	&	0	&	0	\\
		0	&	1	&	0	\\
		0	&	1	&	0
	\end{pmatrix}
\end{equation}
signifie que l'application linéaire $A$ envoie le vecteur $e_1$ sur $\begin{pmatrix}
	3	\\ 
	0	\\ 
	0	
\end{pmatrix}$, le vecteur $e_2$ sur $\begin{pmatrix}
	0	\\ 
	0	\\ 
	1	
\end{pmatrix}$ et le vecteur $e_3$ sur $\begin{pmatrix}
	0	\\ 
	1	\\ 
	0	
\end{pmatrix}$.
Pour savoir comment $A$ agit sur n'importe quel vecteur, on applique la règle de produit vecteur$\times$matrice :
\begin{equation}
	\begin{pmatrix}
		1	&	2	&	3	\\
		4	&	5	&	6	\\
		7	&	8	&	9
	\end{pmatrix}\begin{pmatrix}
		x	\\ 
		y	\\ 
		z	
	\end{pmatrix}=
	\begin{pmatrix}
		x+2y+3z	\\ 
		4x+5y+6z	\\ 
		7x+8y+9z	
	\end{pmatrix}.
\end{equation}

Une chose intéressante est de savoir quelles sont les directions invariantes de la transformation linéaire. Par exemple, on peut lire sur la matrice \eqref{EqExempleALin} que la direction $\begin{pmatrix}
	1	\\ 
	0	\\ 
	0	
\end{pmatrix}$ est invariante : elle est simplement multipliée par $3$. Dans cette direction, la transformation est juste une dilatation. Affin de savoir si $v$ est un vecteur d'une direction conservée, il faut voir si il existe un nombre $\lambda$ tel que $Av=\lambda v$, c'est à dire voir si $v$ est simplement dilaté.

L'équation $Av=\lambda v$ se récrit $(A-\lambda\mtu)v=0$, c'est à dire qu'il faut résoudre l'équation
\begin{equation}
	(A-\lambda\mtu)\begin{pmatrix}
		x	\\ 
		y	\\ 
		z	
	\end{pmatrix}=
	\begin{pmatrix}
		0	\\ 
		0	\\ 
		0	
	\end{pmatrix}.
\end{equation}
Nous savons qu'une telle équation ne peut avoir de solutions que si $\det(A-\lambda\mtu)=0$. La première étape est donc de trouver les $\lambda$ qui vérifient cette condition.


%---------------------------------------------------------------------------------------------------------------------------
\subsection{Comment trouver la matrice d'une symétrie donnée ?}
%---------------------------------------------------------------------------------------------------------------------------
\label{SubSecMtrSym}

Ceci est une FAQ (Faut Avoir Quompri). 

%///////////////////////////////////////////////////////////////////////////////////////////////////////////////////////////
\subsubsection{Symétrie par rapport à un plan}
%///////////////////////////////////////////////////////////////////////////////////////////////////////////////////////////

Comment trouver par exemple la matrice $A$ qui donne la symétrie autour du plan $z=0$ ? La définition d'une telle symétrie est que les vecteurs du plan $z=0$ ne bougent pas, tandis que les vecteurs perpendiculaires changent de signe. Ces informations vont permettre de trouver comment $A$ agit sur une base de $\eR^3$. En effet :
\begin{enumerate}

	\item
		Le vecteur $\begin{pmatrix}
			1	\\ 
			0	\\ 
			0	
		\end{pmatrix}$ est dans le plan $z=0$, donc il ne bouge pas,

	\item
		le vecteur $\begin{pmatrix}
			0	\\ 
			1	\\ 
			0	
		\end{pmatrix}$ est également dans le plan, donc il ne bouge pas non plus,

	\item
		et le vecteur $\begin{pmatrix}
			0	\\ 
			0	\\ 
			1	
		\end{pmatrix}$ est perpendiculaire au plan $z=0$, donc il va changer de signe.

\end{enumerate}
Cela nous donne directement les valeurs de $A$ sur la base canonique et nous permet d'écrire 
\begin{equation}
	A=\begin{pmatrix}
		1	&	0	&	0	\\
		0	&	1	&	0	\\
		0	&	0	&	-1
	\end{pmatrix}.
\end{equation}
Pour écrire cela, nous avons juste mit en colonne les images des vecteurs de base. Les deux premiers n'ont pas changé et le troisième a changé.

Et si maintenant on donne un plan moins facile que $z=0$ ? Le principe reste le même : il faudra trouver deux vecteurs qui sont dans le plan (et dire qu'ils ne bougent pas), et puis un vecteur qui est perpendiculaire au plan\footnote{Pour le trouver, penser au produit vectoriel.}, et dire qu'il change de signe.

Voyons ce qu'il en est pour le plan $x=-z$. Il faut trouver deux vecteurs linéairement indépendants dans ce plan. Prenons par exemple
\begin{equation}		\label{EqffudE}
	\begin{aligned}[]
		f_1&=\begin{pmatrix}
			0	\\ 
			1	\\ 
			0	
		\end{pmatrix},&f_2&=\begin{pmatrix}
			1	\\ 
			0	\\ 
			-1	
		\end{pmatrix}.
	\end{aligned}
\end{equation}
Nous avons 
\begin{equation}
	\begin{aligned}[]
		Af_1&=f_1\\
		Af_2&=f_2.
	\end{aligned}
\end{equation}
Afin de trouver un vecteur perpendiculaire au plan, calculons le produit vectoriel :
\begin{equation}
	f_3=f_1\times f_2=\begin{vmatrix}
		e_1	&	e_2	&	e_3	\\
		0	&	1	&	0	\\
		1	&	0	&	-1
	\end{vmatrix}=-e_1-e_3=\begin{pmatrix}
		-1	\\ 
		0	\\ 
		-1	
	\end{pmatrix}.
\end{equation}
Nous avons 
\begin{equation}
	Af_3=-f_3.
\end{equation}
Afin de trouver la matrice $A$, il faut trouver $Ae_1$, $Ae_2$ et $Ae_3$. Pour ce faire, il faut d'abord écrire $\{ e_1,e_2,e_3 \}$ en fonction de $\{ f_1,f_2,f_3 \}$. La première des équation \eqref{EqffudE} dit que 
\begin{equation}
	f_1=e_2.
\end{equation}
Ensuite, nous avons
\begin{equation}
	\begin{aligned}[]
		f_2&=e_1-e_3\\
		f_3&=-e_1-e_3.
	\end{aligned}
\end{equation}
La somme de ces deux équations donne $-2e_3=f_2+f_3$, c'est à dire
\begin{equation}
	e_3=-\frac{ f_2+f_3 }{ 2 }
\end{equation}
Et enfin, nous avons
\begin{equation}
	e_1=\frac{ f_2-f_3 }{ 2 }.
\end{equation}

Maintenant nous pouvons calculer les images de $e_1$, $e_2$ et $e_3$ en faisant
\begin{equation}
	\begin{aligned}[]
		Ae_1&=\frac{ Af_2-Af_3 }{ 2 }=\frac{1 }{2}\begin{pmatrix}
			0	\\ 
			0	\\ 
			-2	
		\end{pmatrix}=\begin{pmatrix}
			0	\\ 
			0	\\ 
			-1	
		\end{pmatrix},\\
		Ae_2&=Af_1=f_1=\begin{pmatrix}
			0	\\ 
			1	\\ 
			0	
		\end{pmatrix},\\
		Ae_3&=-\frac{ f_2-f_3 }{ 2 }=-\frac{ 1 }{2}\begin{pmatrix}
			2	\\ 
			0	\\ 
			0	
		\end{pmatrix}=\begin{pmatrix}
			-1	\\ 
			0	\\ 
			0	
		\end{pmatrix}.
	\end{aligned}
\end{equation}
La matrice $A$ s'écrit maintenant en mettant les trois images trouvées en colonnes :
\begin{equation}
	A=\begin{pmatrix}
		0	&	0	&	-1	\\
		0	&	1	&	0	\\
		-1	&	0	&	0
	\end{pmatrix}.
\end{equation}

%///////////////////////////////////////////////////////////////////////////////////////////////////////////////////////////
\subsubsection{Symétrie par rapport à une droite}
%///////////////////////////////////////////////////////////////////////////////////////////////////////////////////////////

Le principe est exactement le même : il faut trouver trois vecteurs $f_1$, $f_2$ et $f_3$ sur lesquels on connaît l'action de la symétrie. Ensuite il faudra exprimer $e_1$, $e_2$ et $e_3$ en termes de $f_1$, $f_2$ et $f_3$.

Le seul problème est de trouver les trois vecteurs $f_i$. Le premier est tout trouvé : c'est n'importe quel vecteur sur la droite. Pour les deux autres, il faut un peu ruser parce qu'il faut impérativement qu'ils soient perpendiculaire à la droite. Pour trouver $f_2$, on peut écrire
\begin{equation}
	f_2=\begin{pmatrix}
		1	\\ 
		0	\\ 
		x	
	\end{pmatrix},
\end{equation}
et puis fixer le $x$ pour que le produit scalaire de $f_2$ avec $f_1$ soit nul. Si il n'y a pas moyen (genre si $f_1$ a sa troisième composante nulle), essayer avec $\begin{pmatrix}
	x	\\ 
	1	\\ 
	0	
\end{pmatrix}$. Une fois que $f_2$ est trouvé (il y a des milliards de choix possibles), trouver $f_3$ est super facile : prendre le produit vectoriel entre $f_1$ et $f_2$.

%///////////////////////////////////////////////////////////////////////////////////////////////////////////////////////////
\subsubsection{En résumé}
%///////////////////////////////////////////////////////////////////////////////////////////////////////////////////////////


La marche à suivre est

\begin{enumerate}

	\item
		Trouver trois vecteurs $f_1$, $f_2$ et $f_3$ sur lesquels on connaît l'action de la symétrie. Typiquement : des vecteurs qui sont sur l'axe ou le plan de symétrie, et puis des perpendiculaires. Pour la perpendiculaire, penser au produit scalaire et au produit vectoriel.

	\item
		Exprimer la base canonique $e_1$, $e_2$ et $e_3$ en termes de $f_1$, $f_2$, $f_3$.

	\item
		Trouver $Ae_1$, $Ae_2$ et $Ae_3$ en utilisant leur expression en termes des $f_i$, et le fait que l'on connaisse l'action de $A$ sur les $f_i$.

	\item
		La matrice s'obtient en mettant les images des $e_i$ en colonnes.

\end{enumerate}

%+++++++++++++++++++++++++++++++++++++++++++++++++++++++++++++++++++++++++++++++++++++++++++++++++++++++++++++++++++++++++++
\section{Orthogonalité}
%+++++++++++++++++++++++++++++++++++++++++++++++++++++++++++++++++++++++++++++++++++++++++++++++++++++++++++++++++++++++++++

\begin{proposition}			\label{PropVectsOrthLibres}
	si $v_1,\cdots,v_k$ sont des vecteurs non nuls, orthogonaux deux à deux, alors ces vecteurs forment une famille libre.
\end{proposition}

