% This is part of the Exercices et corrigés de mathématique générale.
% Copyright (C) 2009-2011,2014
%   Laurent Claessens
% See the file fdl-1.3.txt for copying conditions.
Lorsque nous demandons d'étudier une fonction, nous demandons les éléments suivants : domaine de définition, croissance, extrema, points d'inflexion, asymptote et dessiner le graphe.


\Exo{III-1}
\Exo{III-2}
\Exo{III-3}
\Exo{III-4}
\Exo{III-5}
\Exo{TP40001}
\Exo{TP40002}
\Exo{TP40003}
\Exo{TP40004}
\Exo{TP40005}
\Exo{TP50001}
\Exo{TP50002}
\Exo{TP50003}
\Exo{TP50004}

%---------------------------------------------------------------------------------------------------------------------------
\subsection{Quelque fautes usuelles}
%---------------------------------------------------------------------------------------------------------------------------

Pour l'exercice \ref{exoTP40001}, les fautes les plus souvent commises sont
\begin{enumerate}

	\item
		$f'= e^{2x}$ implique $f=\frac{1}{ 2 } e^{x}$. Cela n'est pas vrai. La dérivée de $ e^{2x}$ est $2 e^{2x}$. Le $2$ reste dans l'exponentielle.

	\item
		Lorsqu'on intègre par partie, il faut aussi mettre les bornes pour le morceau qui n'est pas dans la nouvelle intégrale :
		\begin{equation}
			\int_a^b fg'=[fg]_a^b-\int_a^bf'g.
		\end{equation}
\end{enumerate}

Pour l'exercice \ref{exoTP40002}, les fautes les plus souvent commises sont
\begin{enumerate}

	\item
		Lorsqu'on a trouvé la solution générale $y_k(x)$ qui dépend du paramètre $k$ (ou $C$), il faut encore trouver la valeur du paramètre $k$ telle que $y_k(\pi)=0$.

\end{enumerate}

Pour l'exercice \ref{exoTP40003}, les fautes les plus souvent commises sont
\begin{enumerate}

	\item
		Ne pas oublier que $e^0=1$.
\end{enumerate}


