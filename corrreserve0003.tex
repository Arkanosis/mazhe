% This is part of Exercices et corrigés de CdI-1
% Copyright (c) 2011
%   Laurent Claessens
% See the file fdl-1.3.txt for copying conditions.

% This is part of the Exercices et corrigés de CdI-1
% Copyright (C) 2010
%   Laurent Claessens
% See the file fdl-1.3.txt for copying conditions.

\begin{corrige}{reserve0003}

    \begin{enumerate}
        \item
            Si \( a=0\), alors \( x=0\) est la seule solution. Si \( a\neq 0 \) alors \( a\) est une puissance de \( \omega\); nous posons \( a=\omega^l\). Nous cherchons \( x\) sous la forme \( x=\omega^k\). L'équation à résoudre pour \( k\) est
            \begin{equation}
                \omega^{5k}=\omega^l
            \end{equation}
            où \( l\) est donné. Cette équation revient à 
            \begin{equation}
                5k=l\mod 15.
            \end{equation}
            Si \( l\) n'est pas un multiple de \( 5\), alors il n'y a pas de solutions. Il n'y a des solutions uniquement pour \( l=0,5,10\) et elles sont :
            \begin{equation}
                k=\begin{cases}
                    3,6,9,12     &   \text{si \( l\)=0}\\
                    1     &   \text{si \( l\)=5}\\
                    2     &   \text{si \( l\)=10}
                \end{cases}
            \end{equation}
        \item
            Nous cherchons \( \gamma\) sous la forme \( \gamma=\omega^k\). Parmi les nombreuses contraintes liées à l'énoncé nous devons avoir
            \begin{equation}
                \gamma^5=1,\gamma,\gamma^2,\gamma^4,\gamma^8.
            \end{equation}
            Les possibilités \( \gamma^5=\gamma,\gamma^2,\gamma^4,\gamma^5\) ne sont pas bonnes parce qu'elles impliqueraient que \( B_{\gamma}\) n'est pas une base. Reste à explorer \( \gamma^5=1\).

            Étant donné le premier point nous restons avec les possibilités
            \begin{equation}
                \gamma=1,\omega^3,\omega^6,\omega^9,\omega^{12}.
            \end{equation}
            Évidemment \( \gamma=1\) ne produit pas une base. Avec \( \gamma=\omega^3\) nous trouvons
            \begin{equation}
                B_{\gamma}=\{ \omega^3,\omega^6,\omega^{12},\omega^{24} \}=\{ \omega^3,\omega^6,\omega^{12},\omega^9 \}
            \end{equation}
            où nous avons utilisé le fait que \( \omega^k=\omega^{k\mod 15}\). En utilisant le fait que \( \omega^4=\omega^3+1\) nous trouvons
            \begin{subequations}
                \begin{align}
                    \omega^5&=\omega^3+\omega+1\\
                    \omega^6&=\omega^3+\omega^2+\omega+1\\
                    \omega^9&=\omega^2+1\\
                    \omega^{12}&=\omega+1.
                \end{align}
            \end{subequations}
            L'ensemble \( B_{\gamma}\) est alors formé des éléments
            \begin{subequations}
                \begin{align}
                    f_1&=\omega^3\\
                    f_2&=\omega^3+\omega^2+\omega+1\\
                    f_3&=\omega+1\\
                    f_4&=\omega^2+1.
                \end{align}
            \end{subequations}
            Il est assez simple de vérifier que cela est une base en remarquant que \( f_1+f_2+f_2+f_4=1\).

            Les possibilités \( \gamma=\omega^6,\omega^9,\omega^{12}\) produisent les mêmes ensemble \( B_{\gamma}\).
    \end{enumerate}

\end{corrige}
