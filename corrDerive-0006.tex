% This is part of Outils mathématiques
% Copyright (c) 2011,2014
%   Laurent Claessens
% See the file fdl-1.3.txt for copying conditions.

\begin{corrige}{Derive-0006}

    La croissance de la fonction revient à la positivité de la dérivée. La dérivée de $f$ est donnée par
    \begin{equation}
        f'(x)=1-\frac{ a }{ x^2 },
    \end{equation}
    et donc
    \begin{equation}
        f'\left( \frac{1}{ 10 } \right)=1-100a.
    \end{equation}
    Nous devons donc résoudre l'inéquation
    \begin{equation}
        1-100a>0
    \end{equation}
    par rapport à $a$. La solution est 
    \begin{equation}
        a<\frac{1}{ 100 }
    \end{equation}
    
    Comment faire avec Sage ?
    \begin{verbatim}
----------------------------------------------------------------------
| Sage Version 4.6.1, Release Date: 2011-01-11                       |
| Type notebook() for the GUI, and license() for information.        |
----------------------------------------------------------------------
sage: var('x,a')
(x, a)
sage: f=x+a/x
sage: derive=f.diff(x)
sage: derive(1/10)
/home/moky/Sage/local/lib/python2.6/site-packages/IPython/iplib.py:2073: 
DeprecationWarning: Substitution using function-call syntax and unnamed
arguments is deprecated and will be removed from a future release
of Sage; you can use named arguments instead,like EXPR(x=..., y=...)
exec code_obj in self.user_global_ns, self.user_ns -1/10/x^2 + 1
sage: derive(x=1/10)
-100*a + 1
sage: solve(derive(x=1/10)>0,a)
[[a < (1/100)]]
    \end{verbatim}
    Notes :
    \begin{enumerate}
        \item
            Il faut dire que $a$ va être une variable. C'est pour cela qu'on a commencé par \info{var('x,a')}. Étant donné que Sage comprend tout seul que \info{x} est une variable, on aurait pu ne pas déclarer \info{x}.
        \item
            Lorsqu'on veut dériver, il faut dire par rapport à quelle variable on veut dériver. Exemple :
            \begin{verbatim}
sage: f
x + a/x
sage: f.diff(a)
1/x
            \end{verbatim}
            C'est pour cela qu'il faut taper \info{f.diff(x)}.
        \item
            Le long message d'erreur \info{DeprecationWarning} est dû à une faute de ma part. En effet la fonction \info{derive} est une fonction de deux variables : \info{x} et $\info{a}$. C'est pour cela que je dois préciser que je veux l'évaluer avec \info{x=1/10} et en laissant \info{a} comme variable. Exemples:
            \begin{verbatim}
sage: derive
-a/x^2 + 1
sage: derive(a=5)
-5/x^2 + 1
sage: derive(a=x)
-1/x + 1
            \end{verbatim}
        \item
            Enfin, vous noterez la fonction \info{solve} qui sert à résoudre des équations et inéquations.

    \end{enumerate}

\end{corrige}
