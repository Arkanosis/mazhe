% This is part of Exercices et corrigés de CdI-1
% Copyright (c) 2011
%   Laurent Claessens
% See the file fdl-1.3.txt for copying conditions.

%+++++++++++++++++++++++++++++++++++++++++++++++++++++++++++++++++++++++++++++++++++++++++++++++++++++++++++++++++++++++++++
					\section{Quelque corrections}
%+++++++++++++++++++++++++++++++++++++++++++++++++++++++++++++++++++++++++++++++++++++++++++++++++++++++++++++++++++++++++++



\noindent 31.
\begin{enumerate}
\item $\dst{df_{(1,1)}}$ et $\dst{dg_{(\sqrt2,\frac{\pi}{4})}}$\\
\[\ba{l}\dst{\dfdx(x,y) = \f{1}{y}\ln(\f{x}{y})e^{\f{x}{y}}+\f{1}{x}e^{\f{x}{y}}}\\
            \dst{\dfdx(1,1)=e}\\
            \dst{\dfdy(x,y)=-\f{x}{y^2}\ln(\f{x}{y})e^{\f{x}{y}}-xe^{\f{x}{y}}}\\
            \dst{\dfdx(1,1)=e}\ea\]
            
 \noindent Par les règles de calcul, $f$ est différentiable en $(1,1)$. la différentielle $df_{(1,1)}$ est donc représentée dans les bases canoniques de $\eR^2$ et $\eR$ par la matrice jacobienne (ici gradient):\[df_{(1,1)}=(e \;\; -e)\]
 
\[\ba{lclllllcl}\dst{\dgudr(r,\theta)} &=&\cos(\theta)& & & & \dst{\dgudth(r,\theta)}   & =&-r\sin(\theta)\\
            \dst{\dgudr(\sqrt2, \f{\pi}{4})}&=&\f{\sqrt2}{2}& & &&\dst{\dgudth(\sqrt2, \f{\pi}{4})}& =&-1 \\
            \dst{\dgddr(r,\theta)} &=&\sin(\theta)&  && &\dst{\dgddth(r,\theta)}  &=&r\cos(\theta) \\
            \dst{\dgddr(\sqrt2, \f{\pi}{4})}&=&-\f{\sqrt2}{2}&& & &\dst{\dgddth(\sqrt2, \f{\pi}{4})}& = &1\ea\]

La fonction $g$ est également différentiable en $(\sqrt2, \f{\pi}{4})$ et sa matrice Jacobienne est:
\[dg_{(\sqrt2, \f{\pi}{4})}=\left(\ba{cc} \f{\sqrt2}{2} & -1\\
							\f{\sqrt2}{2}&1\ea\right)\]	


\item $\dst{\tilde{f} \;=\;e^{\cos(\theta)}\ln(\cos(\theta))}$.
\item On voit d'abord que $g(\sqrt2, \f{\pi}{4})\;=\;(1,1)$. Donc
\[\ba{cccc} d\tilde{f}_{(\sqrt2, \f{\pi}{4})} & = & df_{g(\sqrt2, \f{\pi}{4})}\circ dg_{(\sqrt2, \f{\pi}{4})}\\
							    & =& df_{(1,1)}\circ dg_{(\sqrt2, \f{\pi}{4})} \ea\]
et  la matrice jacobienne de la différentielle de la composée est donc:\[d\tilde{f}_{(\sqrt2, \f{\pi}{4})}=(e\;\;-e)\left(\ba{cc} \f{\sqrt2}{2} & -1\\
							\f{\sqrt2}{2}&1\ea\right)=(0\;\;-2e)\]

							    		

\end{enumerate}


\noindent 32.
\begin{enumerate}
\item $\dst{\dgdu = e^v\dfdx(\star,\star)+2uv\dfdy(\star,\star)}$
\item $\dst{\dgdv = ue^v\dfdx(\star,\star)+(1+u^2)\dfdy(\star,\star)}$
\end{enumerate}
où $\dst{(\star,\star) = (ue^v,v(1+u^2))}$.

\vspace{1cm}

\noindent 33. \\

\noindent Soit $\dst{g:\eR^2\rightarrow \eR:(x,y)\rightarrow  f(x^2-y^2)}$. Dérivées partielles de:\[(x,y)\rightarrow  y(\partial_xg)(x,y)+x(\partial_yg)(x,y)?\]
Nommons cette fonction $h$. 
\begin{enumerate}
\item $\dst{\partial_xg(x,y) = 2xf'(x^2-y^2)}$
\item$\dst{\partial_yg(x,y) = -2yf'(x^2-y^2)}$
\end{enumerate}
et donc $\dst{h(x,y) = 0 \hs \forall (x,y)\in \eR^2}$.

\vspace{1cm}


\noindent 34. \\

\noindent $\dst{h(t)=f(t,g(t^2))}$.\\

\begin{enumerate}
\item $\dst{h'(t)=\dfdx(\star,\star)+\dfdy(\star,\star)2tg'(t^2)}$
\item $ \ba{rl} h''(t)=     &  \dst{ \ddfdx(\star,\star)+4tg'(t^2)\ddfdxy(\star,\star)+4t^2(g'(t^2))^2\ddfdy(\star,\star) }\\     		
				    & \dst{+[2g'(t^2)+4t^2g''(t^2)]\dfdy(\star,\star)}\ea$

\end{enumerate}
où $(\star,\star) = (t,g(t^2))$.

\vspace{1cm}

\noindent 35.
\[h:\eR^2\rightarrow \eR:(u,v)\rightarrow  f(g(ue^v),g(v)(1+u^2))^{g(u+v)}\]

\noindent Comme toujours il vaut mieux faire ce genre d'exercices prudemment. Renommons donc les diverses composantes de cette fonction.\\

\noindent Soit $l(u,v)=f(g(ue^v),g(v)(1+u^2))$. On a alors:
\begin{enumerate}
\item $\dst{\dldu(u,v) = \dfdx(\star,\star)g'(ue^v)e^v + \dfdy(\star,\star)g(v)2u}$
\item $\dst{\dldv(u,v) = \dfdx(\star,\star)g'(ue^v)ue^v+\dfdy(\star,\star)g'(v)(1+u^2)}$
\end{enumerate}
o\`{u} $(\star,\star)=(g(ue^v),g(v)(1+u^2))$.\\

\noindent Alors $\dst{h(u,v)=l(u,v)^{g(u+v)} = e^{g(u+v)\ln(l(u,v))}}$ qui est facile à dériver:

\begin{enumerate}
\item $\dst{\dhdu = [g'(u+v)\ln(l(u,v))+\f{g(u+v)}{l(u,v)}\dldu(u,v)] l(u,v)^{g(u+v)}}$
\item $\dst{\dhdv = [g'(u+v)\ln(l(u,v))+\f{g(u+v)}{l(u,v)}\dldv(u,v)] l(u,v)^{g(u+v)}}$
\end{enumerate}



\noindent 26.
\begin{enumerate}
\item $(x,y)\rightarrow  3x^2+x^3y+x$.\\
Combinaison linéaire de fonctions continues et différentiables sur $\eR^2$ (Exercice: prouver rigoureusement que les polyn\^{o}mes sont bien des fonctions continues et différentiables sur $\eR^2$).

\item  $\dst{x\rightarrow \left\{\ba{c}e \; {\rm si}\;xy\neq0\\
              			        e^{x+y}\;{\rm sinon} \ea\right.}$\\
N.B.: Il est toujours utile de se représenter le domaine de chacune des fonctions. 

\noindent La première remarque est que cette fonction est clairement continue et différentiable en tout point hors de $\{xy=0\}$ (fonction constante). Sur $\{xy=0\}$?
\begin{enumerate}
\item Continuité:\\
Prenons un point dans $\{xy=0\}$, par exemple le point $(a,0)$ (Remarquez que le cas $(0,b)$ est réglé par symétrie). Pour voir si la fonction est continue en ce point il faut voir si \[\dst{\lim_{(x,y)\rightarrow (0,0)}f(x,y)=f(0,0)=e^a}.\] Si on prend deux manières différentes d'aller vers $(a,0)$ ($y=0$ puis $x=a$) on voit que si $a \neq1$ la fonction ne peut pas \^{e}tre continue. Et en $(1,0)$? Si on $(x,y)\rightarrow (1,0)$ avec d'abord $y=0$ puis $y\neq0$ on aura regardétoutes les manières de tendre vers $(1,0)$. Or dans les deux cas les limites valent $e = f(1,0)$, ce qui prouve que la fonction est continue en $(1,0)$ (et $(0,1)$ par symétrie).

\item Différentiabilité:\\
Comme la fonction est discontinue en tout point $(a,0)$ et $(0,b)$ avec $a\neq1$ et $b\neq1$ elle est aussi non différentiable en chacun de ces points. Il reste donc les points $(1,0)$ et $(0,1)$. Comme toujours, nous regardons d'abord les dérivées directionnelles en $(1,0)$:
\[\dfdu(1,0) \;=\;\lim_{t\rightarrow 0}\f{f(1+tu_1,tu_2)-e}{t}\]
Il y a deux possibilités: $u_2=0$ et donc $u=(\pm1.0)$ ou$u_2\neq0$ (pourquoi ne regarde-t-on que ces deux cas?).

\begin{enumerate}
\item si $u\neq(\pm1,0)$.\\
$\dst{\dfdu(1,0) \;=\;\lim_{t\rightarrow 0}\f{e-e}{t}\;=\;0}$.
\item si $u=(\pm1,0)$, i.e. si $u=(1,0) = e_1$\\
$\dst{\dfdu(1,0) = \dfdx(1,0)=\lim_{t\rightarrow 0}\f{f(1+t,0)-e}{t}=\lim_{t\rightarrow 0}\f{e^{1+t}-e}{t} =^H0}$.
\end{enumerate}
\end{enumerate}
\underline{Conclusion}:\\
Si $f$ était différentiable en $(1,0)$, on aurait que sa différentielle prendrait la forme suivante:
\[\ba{cc} df_{(1,0)}u& = \dfdx(1,0)u_1+\dfdy(1,0)u_2\\
			      & = eu_1\;\;\forall u\in\eR^2 \ea \]
Sa différentielle satisferait également \ac:
\[	df_{(1,0)}u = \dfdu(1,0) = 0 \;\; \forall u \neq (\pm1,0) \in \eR^2\]
Les deux propriétés étant contradictoires, la fonction $f$ ne peut \^{e}tre différentiable en $(1,0)$ (ni en $(0,1)$ par symétrie). 		      
\item		  $\dst{x\rightarrow \left\{\ba{c}\f{x}{y} \; {\rm si}\;y\neq0\\
              			       0\;{\rm sinon} \ea\right.}$\\
			       
Continue et différentiable sur $\eR-\{y=0\}$. Sur l'axe $y=0$ elle n'est pas continue.	       
			       
\item		  $\dst{x\rightarrow \left\{\ba{l}x+ay \; {\rm si}\;x>0\\
              			                       x\;{\rm sinon} \ea\right.}$\\
Si $a=0$ fonction continue et différentiable sur $\eR^2$. Si $a\neq0$, fonction continue et différentiable partout en dehors de l'axe $x=0$. Sur cet axe, elle est discontinue en tout point sauf en $(0,0)$ o\`{u} elle est continue. Mais elle n'est pas différentiable en $(0,0)$ car toutes ses dérivées directionnelles  n'y sont pas définies.
	      
\item     $\dst{x\rightarrow \left\{\ba{c}\f{xy^5}{x^6+y^6} \; {\rm si}\;x\neq y\\
              			       0\;{\rm sinon} \ea\right.}$\\
Fonction continue et différentiable partout en dehors de la droite $x=y$.  La fonction est discontinue en chacun des points de cette droite.

\end{enumerate}

30.
\begin{enumerate}
\item $(u,v)\rightarrow  u^3+12u^2v-5v^3$\\
\begin{enumerate}
\item $\dst{\dfdu = 3u^2+24uv}$
\item $\dst{\dfdv = 12u^2-15v^2}$
\end{enumerate}
\item $(u,v)\rightarrow  f(u^2)\ln(v)$\\
\begin{enumerate}
\item $\dst{\dfdu = 2uf'(u^2)\ln(v)}$
\item $\dst{\dfdv = \f{f(u^2)}{v}}$
\end{enumerate}
\item $(x,y)\rightarrow \tan(x+y^2)$\\
\begin{enumerate}
\item $\dst{\dfdx =\f{1}{cos^2(x+y^2)}}$
\item $\dst{\dfdv = \f{2y}{cos^2(x+y^2)}}$
\end{enumerate}
\item $(r,\theta)\rightarrow  r^\theta$
\begin{enumerate}
\item $\dst{\dfdr =\theta r^{\theta-1}}$
\item $\dst{\dfdth =\ln(r)r^\theta}$
\end{enumerate}
\item $(x,y)\rightarrow (x+3)e^x$
\begin{enumerate}
\item $\dst{\dfdx =e^x(x+4)}$
\item $\dst{\dfdy =0}$
\end{enumerate}
\item $(u,v)\rightarrow  \ln(f(uv)) $\\

\begin{enumerate}
\item $\dst{\dfdu = \f{vf'(uv)}{f(uv)}}$
\item $\dst{\dfdv = \f{uf'(uv)}{f(uv)}}$
\end{enumerate}\pagebreak
\end{enumerate}

\noindent 32.
\begin{enumerate}
\item $\dst{\dgdu = e^v\dfdx(\star,\star)+2uv\dfdy(\star,\star)}$
\item $\dst{\dgdv = ue^v\dfdx(\star,\star)+(1+u^2)\dfdy(\star,\star)}$
\end{enumerate}
o\`{u} $\dst{\dfdx(\star,\star) = \dfdx(ue^v,v(1+u^2))}$ et $\dst{\dfdy(\star,\star) = \dfdy(ue^v,v(1+u^2))}$.

\noindent 34. $\dst{h(t)=f(t,g(t^2))}$.\\

\begin{enumerate}
\item $\dst{h'(t)=\dfdx(\star,\star)+\dfdy(\star,\star)2tg'(t^2)}$
\item $ \ba{rl} h''(t)=     &  \dst{ \ddfdx(\star,\star)+4tg'(t^2)\ddfdxy(\star,\star)+4t^2(g'(t^2))^2\ddfdy(\star,\star) }\\     		
				    & \dst{+[2g'(t^2)+4t^2g''(t^2)]\dfdy(\star,\star)}\ea$

\end{enumerate}
où $(\star,\star) = (t,g(t^2))$.




 \section{Intégration}
 \subsection{Série A}
 Exercice 11
 \begin{enumerate}
   \exr $\int \frac{x^3+3x+1}{x} d x = \frac{x^3}3 + 3x + \ln(x)$%
   \exr $\int x^2d x = \frac{x^3}3$%
   \exr $\int 3(x^2+1)^2 d x = \int 3 x^4 + 6 x^2 + 3 d x = \frac 35
   x^5 + 2 x^3 + 3x$%
   \exr $\int (3x^2 - 6x)^3 (x-1) d x = \frac1{12} (3x^2 - 6x)^4$
 \end{enumerate}

 Exercice 12
 \begin{enumerate}
   \exr $\int \sin^2(x^2+1) \cos(x^2+1) x d x = \frac16
   \sin(x^2+1)^3$%
   \exr $\int \tan(x) d x = -\ln\abs{\cos(x)}$%
   \exr $\int \frac{1}{(2+\sqrt{x})\sqrt x} d x= 2 \ln(2+\sqrt{x})$%
   \exr $\int \frac{\ln(x)}{x(1- \ln^2(x)} d x = \frac12
   \ln\abs{1-\ln^2(x)}$%
 \end{enumerate}



   Travaux perso 2 ---------------

   1. Soit deux réels $x$ et $y$ vérifiant $0 < x < y$. On veut montrer
   que pour tout naturel $k \geq 2$, on a
   \[0 < \sqrt[k]{y} - \sqrt[k]{x} < \sqrt[k]{y-x}.\]

   La première inégalité vient de l'inégalité $x < y$ élevée à la
   puissance $\frac1k$.

   On peut ré-écrire la deuxième, sachant que $x > 0$, en divisant par
   $\sqrt[k]{x}$ pour obtenir
   \[\sqrt[k]{\frac yx} - 1 - \sqrt[k]{\frac yx-1} < 0 \quad \text{
     avec $\frac xy > 1$}\] ce qui s'écrit encore $f(t) < 0$ en posant
   $f(t) \pardef \sqrt[k]t - \sqrt[k]{t-1} - 1$. On peut alors étudier
   la fonction $f$. Étant donné que $f(1) = 0$, il suffirait que $f$
   soit strictement décroissante sur $]1;\infty[$ pour qu'on ait
   l'inégalité voulue, à savoir $f(t) < 0$ dès que $t > 1$.

   Pour le montrer, on voit que
   \[f^\prime(t) = \frac 1k \left(t^{\frac{1-k}k} -
     ({t-1})^{\frac{1-k}k}\right)\] d'où on tire les équivalences
   suivantes
   \begin{align}
     & & f^\prime(t) < 0\\
     &\ssi& t^{\frac{1-k}k} < ({t-1})^{\frac{1-k}k}\\
     &\ssi& t^{1-k} < ({t-1})^{1-k}\\
     &\ssi& t > t-1\\
     &\ssi& 0 > -1
   \end{align}
   où la dernière inégalité est manifestement vraie, ce qui prouve la
   première inégalité et achève l'exercice.

   2.


 \paragraph{Exercice 1}
 \begin{enumerate}
 \item Par exemple, $B(x,r)$ avec $x \in \R^n$ et $r > 0$.

 \item On utilise la densité de $\Q$ dans $\R$ pour voir que $B(q,r)$
   ($q \in \Q^n$ et $r > 0$) est également une base.

   On observe ensuite que seuls les $r$ \og petits\fg{} sont utiles,
   donc on se restreint aux boules de la forme $B(q,1/n)$ ($q \in
   \Q^n$ et $n \in \N_0$). Cet ensemble de boules est une base
   dénombrable\marginpar{Pourquoi ?} de la topologie usuelle sur
   $\R^n$.
 \end{enumerate}

 \paragraph{Exercice 2}
 \emph{Principe.} L'idée est de considérer une propriété topologique
 (invariante par homéomorphisme) et de voir qu'elle est vérifiée par
 les ouverts de $\R^2$ mais pas ceux du cône.

 \begin{lem}Si $V$ est un voisinage de $0$ sur le cône $C$, alors
   $V\setminus\{0\}$ n'est pas connexe, donc n'est pas connexe par
   arc.\end{lem}
 \begin{proof}Le cône $C$ est la réunion de $C^+ = C \cap
   \left(\R^2\times \R_0^+\right)$ et $C^- = C \cap \left(\R^2\times
     \R_0^-\right)$ car le seul point à cote nulle est la singularité
   $0$. Dès lors, $V$ s'écrit comme l'union disjointe de $V\cap C^+$
   et $V\cap C^-$, qui sont non-vides. Donc $V$ n'est pas
   connexe.\end{proof}

On procède en deux étapes, en montrant d'abord qu'il
   existe des points en \og dessous\fg{} et au \og dessus\fg{} de
   $0$, puis en essyant de les relier.
     Comme $V$ est un voisinage de $0$, il existe un ouvert $U$ du
     cône centré en $0$ inclu à $V$. Donc par définition de la
     topologie induite, et puisque les boules forment une base de la
     topologie de $\R^3$, il existe une boule $B$ centrée en $0$ dont
     $U$ est la trace sur $C$, telle que $0 \in (B \cap C) \subset
     V$. On choisit $p = (p_x,p_y,p_z) \in (B \cap C)$, et en
     considérant $p^\prime = (p_x, p_y, -p_z)$ on a ainsi trouvé deux
     points qui vérifient $p_z > 0$ et $p^\prime_z < 0$ (au besoin,
     on les échange).

   \begin{enumerate}
   \item Supposons que $V\setminus\{0\}$ soit connexe par arc. Donc
     il existe un chemin
     \[\gamma : [0;1] \to V\setminus\{0\} : t \mapsto
     (\gamma_x(t),\gamma_y(t),\gamma_z(t))\] qui relie $p$ à
     $p^\prime$ et qui vérifie $\gamma_z(0) = p_z > 0$ et
     $\gamma_z(1) = -p_z < 0$. Or $\gamma_z(t)$ est une fonction
     continue (car $\gamma$ est continu), donc par le théorème des
     valeurs intermédiaires, il existe $\bar t$ qui vérifie
     $\gamma_z(\bar t) = 0$. Or le seul point de $C$ dont la cote
     (coordonnée en $z$) soit nulle est le sommet $0$ qui n'est pas
     dans $V\setminus\{0\}$, d'où la contradiction.
   \end{enumerate}

 \begin{rem}Soient deux espaces topologiques $E$ et $F$, et $f :
   E\to F$ un homéomorphisme. Pour toute partie $A$ de $E$,
   l'espace $E\setminus A$ est homéomorphe au sous-espace $F\setminus
   f(A)$ via la restriction $f_{\vert E\setminus A}$.\end{rem}

 \begin{lem}Soient deux espaces topologiques $E$ et $F$, et $f :
   E\to F$ un homéomorphisme. $E$ est connexe par arc si et
   seulement si $F$ l'est.\end{lem}
 \begin{proof}On montre en réalité que l'image d'un connexe par arc
   par une application continue est un connexe par arc, ce qui
   implique chaque sens de l'équivalence de l'énoncé.

   Soient $p$ et $q$ des points de $F$. Il existe un chemin reliant
   un antécédant de $p$ et un antécédant de $q$ (dans $E$). L'image
   de ce chemin est un chemin reliant $p$ et $q$ (dans $F$) puisque
   composé d'applications continues.
 \end{proof}

 \begin{lem}Une sphère de $\R^n$ est connexe par arc si $n >
   1$\end{lem}
 \begin{proof}On voit qu'un cercle est connexe par arc car on a une
   paramétrisation en sinus et cosinus. Pour une sphère $S$ de centre
   $a$ en dimension $n > 2$, on se donne $p$ et $q$ sur $S$ et on
   définit $P$ le plan affin passant par $a$, $p$ et $q$. Alors $P
   \cap S$ est un cercle, donc on peut relier $p$ à $q$ par un chemin
   dans cette intersection.

   Pour voir sur une formule que $P \cap S$ est un cercle, on peut
   écrire $x - a = \lambda(a-p) + \mu(a-q)$ l'équation (en $x$) du
   plan $P$, et $\module{x-a}^2 = R^2$ l'équation (en $x$) de la
   sphère. En injectant, on obtient une équation du second degré en
   $\lambda,\mu$ qui se révèle être l'équation d'un cercle à une
   transformation affine près.
 \end{proof}

 \begin{lem}Un ouvert connexe par arc dans $\R^n$ ($n \geq 2$) reste
   connexe par arc même si on lui enlève un point.\end{lem}
 \begin{proof}
   En effet, soit $U$ un tel ouvert connexe par arc, et $p$ un point
   de $U$. Soient $x$ et $y$ sur $U\setminus\{p\}$. Il existe un
   chemin $\gamma$ de $x$ à $y$. Si le chemin ne passe pas par $p$,
   c'est gagné. Si il passe par $p$, on choisit une boule $B$ fermée
   (de rayon non-nul) centrée en $p$ qui ne contient ni $x$ ni
   $y$. On note
   \[E = \gamma^{-1}(B) \subset [0;1]\] c'est un ensemble compact
   (fermé, par continuité de $\gamma$, et borné) dont on regarde le
   maximum $\bar t$ et le minimum $\underline t$.

   Il reste enfin à définir un chemin entre $p$ et $q$ par morceaux
   \begin{enumerate}
   \item Les points $p$ et $\gamma(\underline t)$ sont reliés par
     $\gamma$,
   \item Par connexité par arc, il existe un chemin sur la sphère qui
     relie $\gamma(\underline t)$ à $\gamma(\bar t)$,
   \item et enfin $\gamma(\bar t)$ et $q$ sont reliés via $\gamma$;
   \end{enumerate}
   ce qui achève la construction d'un chemin continu entre $p$ et
   $q$.
 \end{proof}
 Pour conclure l'exercice, par l'absurde, on prend un voisinage
 connexe et ouvert $V$ de $0$ dans le cône, homéomorphe à un ouvert
 connexe $U$ de $\R^2$. Or $V\setminus\{0\}$ n'est pas connexe par
 arc, alors que l'ouvert dont on retire un point reste connexe par
 arc. C'est impossible, donc l'homéomorphisme n'existe pas, et le
 cône n'est pas une variété de dimension $2$.
