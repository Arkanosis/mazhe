% This is part of Mes notes de mathématique
% Copyright (c) 2011-2012
%   Laurent Claessens
% See the file fdl-1.3.txt for copying conditions.

\documentclass[a4paper,12pt]{book}
%\documentclass[a4paper,12pt,draft]{book}

\usepackage{etex}

\usepackage{ifthen}
\usepackage{pdfsync}

\usepackage{latexsym}
\usepackage{amsfonts}
\usepackage{amsmath}
\usepackage{amsthm}
\usepackage{amssymb}
\usepackage{bbm}
\usepackage{mathrsfs}           
\usepackage{mathabx}           % Pour \divides

\usepackage{pstricks,pst-eucl,pstricks-add,calc,pst-math,catchfile}   % Les dépendances de phystricks.
\usepackage{graphicx}                   % Pour l'inclusion d'image en pfd.

\newcommand{\EpsOrPdfincludegraphics}[2][]{%
        \ifpdf
            \includegraphics[#1]{#2.png}
        \else
            \includegraphics[#1]{#2.eps}
        \fi
        }

\usepackage{subfigure}

\usepackage{fancyvrb}
\usepackage{stmaryrd}       % Pour le \obslash
\usepackage{xstring}        % Utilisé pour les références vers wikipédia
\usepackage{cases}
\usepackage{lscape}         % pour l'environnement landscape, utilisé dans la correction corr0076.tex
\usepackage{multicol}
\usepackage{import}         % Pour le hack qui sert à inclure GeomAnal

% TODO : n'en utiliser qu'un
\usepackage[normalem]{ulem}		% Pour le barré, commande \sout
\usepackage{soul}		% Pour le barré, commande \st

\usepackage[all]{xy}

\let\second\undefined      % le paquet amthabx définit \second
\let\degree\undefined       % le paquet amthabx définit \degree
\usepackage[cdot,thinqspace,amssymb]{SIunits} 
 % L'option amssymb sert à éviter un conflit avec la commande \square de amssymb. Note qu'elle n'est plus accessible. Si tu en as besoin, faudra RTFM
%ftp://ftp.belnet.be/packages/ctan/macros/latex/contrib/SIunits/SIunits.pdf

\usepackage[nottoc]{tocbibind}

%
%\newcounter{VerifForwardRef}
%\setcounter{VerifForwardRef}{0}
%\ifthenelse{\value{VerifForwardRef}=0}{
    %\usepackage[ps2pdf]{hyperref}                  %Doit êre appelé en dernier.
    %\hypersetup{
    %colorlinks=true,
    %linkcolor=blue,
    %urlcolor=green,     % couleur des url
    %filecolor=magenta   % couleur des textes qui sont des liens
    %}
    %% https://en.wikibooks.org/wiki/LaTeX/Hyperlinks
    %\usepackage[numbers,sort&compress]{natbib}
    %\usepackage{hypernat}
%}
%
%\ifthenelse{\value{VerifForwardRef}=1}{
    %\let\Oldref\ref
    %\renewcommand{\ref}[1]{%
    %\ifthenelse{\value{page}<\pageref{#1}}{ {\huge ATTENTION} }{}%
    %\Oldref{#1}%
    %}
    %\newcommand{\url}[1]{}
    %\newcommand{\href}[2]{}
    %\renewcommand{\caption}[1]{}
    %\newcommand{\texorpdfstring}[2]{#1}
%}{}


%%%%%%%%%%%%%%%%%%%%%%%%%%
%
%   Trucs mathématiques
%
%%%%%%%%%%%%%%%%%%%%%%%%

% ENSEMBLES DE NOMBRES
\newcommand{\eA}{\mathbbm{A}}
\newcommand{\eC}{\mathbbm{C}}
\newcommand{\eD}{\mathbbm{D}}
\newcommand{\eE}{\mathbbm{E}}
\newcommand{\eF}{\mathbbm{F}}
\newcommand{\eG}{\mathbbm{G}}
\newcommand{\eH}{\mathbbm{H}}
\newcommand{\eK}{\mathbbm{K}}
\newcommand{\eL}{\mathbbm{L}}
\newcommand{\eM}{\mathbbm{M}}
\newcommand{\eN}{\mathbbm{N}}
\newcommand{\eP}{\mathbbm{P}}
\newcommand{\eQ}{\mathbbm{Q}}
\newcommand{\eR}{\mathbbm{R}}
\newcommand{\eZ}{\mathbbm{Z}}

% ENSEMBLES de fonctions
\newcommand{\aL}{\mathcal{L}}       % Les applications linéaires
\newcommand{\aC}{\mathcal{C}}       % Les fonctions C^1, C^2 etc

% AUTRES
\newcommand{\sdS}{\mathcal{S}}      % L'ensemble des subdivisions d'un intervalle.



\newcommand{\mF}{\mathcal{F}}
\newcommand{\mG}{\mathcal{G}}
\newcommand{\mI}{\mathcal{I}}
\newcommand{\mL}{\mathcal{L}}
\newcommand{\mS}{\mathcal{S}}   % Utilisé pour l'espace des fonctions Schwartz
\newcommand{\mZ}{\mathcal{Z}}


\newcommand{\mtu}{\mathbbm{1}}              % La matrice unité
\newcommand{\caract}{\mathbbm{1}}    % Characteristic function of a set

\DeclareMathOperator{\val}{val}     % valuation d'un polynôme


%\newcommand{\efrac}[2]{\frac{ \displaystyle #1 }{\displaystyle #2 }}
%%%%%%%%%%%%%%%%%%%%%%%%%%
%
%   Numérotations en tout genre
%
%%%%%%%%%%%%%%%%%%%%%%%%

\setcounter{tocdepth}{2}        % Profondeur de la table des matières
\setcounter{secnumdepth}{2}     % Profondeur dans le texte

%%%%%%%%%%%%%%%%%%%%%%%%%%
%
%   Les lignes magiques pour le texte en français.
%
%%%%%%%%%%%%%%%%%%%%%%%%

\usepackage[utf8]{inputenc}
\usepackage[T1]{fontenc}
\usepackage{textcomp}
\usepackage{lmodern}
\usepackage[a4paper,margin=2cm]{geometry} 
\usepackage[english,frenchb]{babel}


%\usepackage[ps2pdf]{hyperref}                  %Doit êre appelé en dernier.
\usepackage{hyperref}                  %Doit êre appelé en dernier.
\hypersetup{
colorlinks=true,
linkcolor=blue,
urlcolor=magenta,     % couleur des url
filecolor=magenta   % couleur des textes qui sont des liens
}

\usepackage[fr]{exocorr}
%%%%%%%%%%%%%%%%%%%%%%%%%%
%
%   Les théorèmes et choses attenantes
%
%%%%%%%%%%%%%%%%%%%%%%%%


\newcounter{numtho}
\newcounter{numprob}

\makeatletter
\@addtoreset{numtho}{chapter}
\@addtoreset{CountExercice}{chapter}
\makeatother

\newlength{\EnvSpace}
\setlength{\EnvSpace}{9pt}      % C'est la distance que je veux mettre avant et après les théorèmes, remarques, \ldots

\newtheoremstyle{MyTheorems}%
        {\EnvSpace}{\EnvSpace}%
        {\itshape}%
        {}%
        {\bfseries}{.}%
        {\newline}%
        {}%
\newtheoremstyle{MyExamples}%
        {\EnvSpace}{\EnvSpace}%
        {}%
        {}%
        {\bfseries}{.}%
        {\newline}%
        {}%
\newtheoremstyle{MyRemarks}%
        {\EnvSpace}{\EnvSpace}%
        {}%
        {}%
        {\bfseries}{.}%
        {\newline}%
        {}%

%\theoremstyle{MyExamples}   %\newtheorem{exemple}[numtho]{Exemple}      % Pour unification, ne plus utiliser
%                            \newtheorem{example}[numtho]{Exemple}
\newcounter{CounterExample}
\renewcommand{\theCounterExample}{\thechapter.\arabic{CounterExample}}

\newenvironment{example}{\vspace{\EnvSpace}\refstepcounter{numtho}\noindent{\bf Exemple \thenumtho}\newline}{\phantom{a}\hfill $\triangle$\vspace{\EnvSpace}}

\theoremstyle{MyRemarks}    \newtheorem{remark}[numtho]{Remarque}

                \newtheorem{amusement}[numtho]{Amusement}
                \newtheorem{erreur}[numtho]{Error}
                \newtheorem{probleme}[numprob]{\fbox{\bf Problèmes et choses à faire}}

\theoremstyle{MyTheorems}
            \newtheorem{lemma}[numtho]{Lemme}
            \newtheorem{corollary}[numtho]{Corollaire}
            \newtheorem{theorem}[numtho]{Théorème}      
            \newtheorem{definition}[numtho]{Définition}      
            \newtheorem{proposition}[numtho]{Proposition}      

            \newtheorem{exo}[CountExercice]{Exercice}       % C'est provisoire, pour Chafaï

\renewcommand{\thenumtho}{\thechapter.\arabic{numtho}}
% La numérotation des équations change dans les corrigés
\renewcommand{\theequation}{\thechapter.\arabic{equation}}
\renewcommand{\theCountExercice}{\arabic{CountExercice}}       % Ce compteur est défini dans SystemeCorr.sty
\newcommand{\defe}[2]{\textbf{#1}\index{#2}}

\renewcommand{\labelenumi}{\theenumi}
\renewcommand{\theenumi}{(\arabic{enumi})}


%%%%%%%%%%%%%%%%%%%%%%%%%%
%
%   Les macros qui font des choses
%
%%%%%%%%%%%%%%%%%%%%%%%%

\newcommand{\mA}{\mathcal{A}}
\newcommand{\mO}{\mathcal{O}}
\newcommand{\mR}{\mathcal{R}}
\newcommand{\mT}{\mathcal{T}}
\newcommand{\mU}{\mathcal{U}}

\newcommand{\scal}[2]{ \langle #1,#2\rangle }

\newcommand{\tq}{\text{ tel que }}
\newcommand{\tqs}{\text{ tels que }}
\newcommand{\quext}[1]{ \footnote{\textsf{#1}}  }
\newcommand{\info}[1]{\texttt{#1}}

\newcommand{\normal}{\lhd}
\newcommand{\swS}{\mathscr{S}}          % L'ensemble des fonctions Schwartz

\newcommand{\Borelien}{\mathcal{B}\text{or}}       % Les boréliens
\newcommand{\tribA}{\mathcal{A}}            % Une tribu A
\newcommand{\tribB}{\mathcal{B}}            
\newcommand{\tribF}{\mathcal{F}}            % Une tribu F

\newcommand{\affE}{\mathcal{E}}            % Un espace affine E
\newcommand{\affF}{\mathcal{F}}            
\newcommand{\affG}{\mathcal{G}}            

\newcommand{\statS}{\mathcal{S}}            % Un modèle statistique
\newcommand{\partP}{\mathcal{P}}            % L'ensemble des parties d'un ensemble

\newcommand{\polyP}{\mathcal{P}}            % L'ensemble des polynômes

\newcommand{\dB}{\mathscr{B}}       % la distribution de Bernoulli
\newcommand{\dE}{\mathscr{E}}       % la distribution exponentielle
\newcommand{\dG}{\mathscr{G}}       % la distribution géométrique.
\newcommand{\dM}{\mathscr{M}}       % la distribution multinomiale
\newcommand{\dN}{\mathscr{N}}       % la distribution normale.
\newcommand{\dP}{\mathscr{P}}       % la distribution de Poisson.
\newcommand{\dT}{\mathscr{T}}       % la distribution de Student
\newcommand{\dU}{\mathscr{U}}       % la distribution uniforme

\newcommand{\hL}{\mathscr{L}}       
\newcommand{\cL}{\hL}           % Pour la partie Chafai

\newcommand{\modE}{\mathcal{E}}         % Le E des modules
\newcommand{\hH}{\mathscr{H}}           % Le H des espaces de Hilbert

%%%%%%%%%%%%%%%%%%%%%%%%%%
%
%   Bibliographie, index et liste des notations
%
%%%%%%%%%%%%%%%%%%%%%%%%

\usepackage{makeidx}
\usepackage[nottoc]{tocbibind}      % Le paquetage qui fait en sorte que la biblio soit inclue correctement dans la table des matières.
\usepackage[refpage]{nomencl}
\renewcommand{\nomname}{Liste des notations}
%
%   Comment introduire des éléments dans l'index des notations.
%
% La liste des tags à mettre pour bien classer mes notations est :
% T     pour la topologie et théorie des ensembles
%
% La syntaxe est facile, par exemple 
%       $\SL(2,\eR)$\nomenclature[G]{$\SL(2,\eR)$}{Le groupe de matrices deux par deux de déterminant 1.}
\renewcommand{\nomgroup}[1]{%
    \ifthenelse{\equal{#1}{A}}{\item[\textbf{Algèbre}]}{}%
    \ifthenelse{\equal{#1}{G}}{\item[\textbf{Géométrie}]}{}%
    \ifthenelse{\equal{#1}{R}}{\item[\textbf{Théorie des groupes}]}{}%
    \ifthenelse{\equal{#1}{P}}{\item[\textbf{Probabilités et statistique}]}{}%
    \ifthenelse{\equal{#1}{Y}}{\item[\textbf{Analyse}]}{}%
    \ifthenelse{\equal{#1}{M}}{\item[\textbf{Chaînes de Markov}]}{}%
}

%%%%%%%%%%%%%%%%%%%%%%%%%%
%
%   DeclareMathOperator
%
%%%%%%%%%%%%%%%%%%%%%%%%

\DeclareMathOperator{\signe}{sgn}
\DeclareMathOperator{\Vol}{Vol}
\DeclareMathOperator{\Int}{Int}     % Intérieur d'un ensemble.
\DeclareMathOperator{\Ind}{Ind}     % l'indice d'un chemin en analyse complexe
\DeclareMathOperator{\Diam}{Diam}   
\DeclareMathOperator{\id}{Id}   
\DeclareMathOperator{\Graph}{Graph} 
\DeclareMathOperator{\pr}{\texttt{proj}}
\DeclareMathOperator{\dom}{dom}

\DeclareMathOperator{\Graphe}{Gr}
\DeclareMathOperator{\Spec}{Spec}   % spectre d'un opérateur
\DeclareMathOperator{\arctg}{arctg}
\DeclareMathOperator{\cotg}{cotg}
\DeclareMathOperator{\cosec}{cosec}
\DeclareMathOperator{\arcsinh}{arcsinh}

\DeclareMathOperator{\GL}{GL}   % le groupe linéaire
\DeclareMathOperator{\PGL}{PGL}   % le groupe projectif
\DeclareMathOperator{\SO}{SO}           
\DeclareMathOperator{\SL}{SL}           
\DeclareMathOperator{\PSL}{PSL}   % Le groupe modulaire SL(2,Z)/Z2
\DeclareMathOperator{\gO}{O}           
\DeclareMathOperator{\SU}{SU}           
\DeclareMathOperator{\gU}{U}           

\DeclareMathOperator{\Reel}{Re}        % La partie réelle d'un nombre complexe

\DeclareMathOperator{\Image}{Image}        % ... avec \Image qui donne l'image d'une fonction ou d'un opérateur.
\DeclareMathOperator{\rang}{rg}   
\DeclareMathOperator{\Kernel}{Ker}
\DeclareMathOperator{\Domaine}{Dom}
\DeclareMathOperator{\Span}{Span}
\DeclareMathOperator{\Hom}{Hom}
\DeclareMathOperator{\End}{End}     % L'ensemble des endomorphismes
\DeclareMathOperator{\tr}{Tr}       % la trace
\DeclareMathOperator{\Majorant}{Maj}
\DeclareMathOperator{\codim}{codim} % pour la codimension.
\DeclareMathOperator{\diam}{diam} % le diamètre d'un ensemble.

\DeclareMathOperator{\Var}{Var}     % Variance d'une variable aléatoire.
\DeclareMathOperator{\Fun}{\texttt{Fun}}     % Ensemble des applications d'un ensemble vers l'autre.
\DeclareMathOperator{\Cov}{Cov}     % la covariance.
\DeclareMathOperator{\gr}{gr}     % le groupe engendré
\DeclareMathOperator{\pgcd}{pgcd}     
\DeclareMathOperator{\ppcm}{ppcm}     
\DeclareMathOperator{\Frob}{Frob}     
\DeclareMathOperator{\Card}{Card}       % Le cardinal d'un ensemble.
\DeclareMathOperator{\Stab}{Stab}       % Le stabilisateur d'un point sous l'action d'un groupe.

\DeclareMathOperator{\Frac}{Frac}       % le corps des fractions d'un anneau
\DeclareMathOperator{\Aff}{Aff}         %  l'espace affine engendré

\newenvironment{subproof}{\begin{description}}{\end{description}}

%%%%%%%%%%%%% TRUCS DE YVIK POUR FAIRE FONCTIONNER CdI1 %%%%%%%%%%%%%%%%%%%%%%
%

%\newcommand{\proofend}{\hspace*{\fill} $\Box$\\}
%\newcommand{\diam}{\hspace*{\fill} $\Diamond$\\}
%\def\s{\smallskip}
%\def\m{\medskip}
%\def\my{\bf}
\newcommand{\eps}{\varepsilon}
\newcommand{\Ker}{\operatorname{Ker}}
\newcommand{\IM}{\operatorname {Im}}
\newcommand{\cat}{\operatorname{cat}}
\newcommand{\crit}{\operatorname{crit}}
\newcommand{\Crit}{\operatorname{Crit}}
\newcommand{\Rest}{\operatorname{Rest}}
\newcommand{\grad}{\operatorname{grad}}
\newcommand{\sgrad}{\operatorname{sgrad}}
\newcommand{\Fix}{\operatorname{Fix}}
\newcommand{\pt}{\operatorname{pt}}
\newcommand{\cl}{\operatorname{cl}}
\newcommand{\B}{\operatorname {B}}
\newcommand{\C}{\operatorname {C}}
%\newcommand{\S}{\operatorname {S}}
\newcommand{\Gr}{\operatorname {Gr\;\!}}
%\def\dim{\operatorname {dim}}
\newcommand{\inj}{\operatorname {inj}}
%\newcommand{\Vol}{\operatorname {Vol}\:\!}
%\newcommand{\Int}{\operatorname {Int}\:\!}
\newcommand{\dist}{\operatorname {dist}}
%\def\inter{\operatorname {int}}
\newcommand{\ext}{\operatorname {ext}}
%\newcommand{\diameter}{\operatorname {diam}\:\!}
\newcommand{\Emb}{\operatorname {Emb}}
\newcommand{\can}{\operatorname {can}}
\newcommand{\euler}{\mbox{\rm e}}
\newcommand{\sii}{\mbox{\rm \scriptsize i}}
\newcommand{\VB}{\mbox{V}_{\!\!B}}   
\newcommand{\VC}{\mbox{V}_{\!\!C}}   
\newcommand{\VS}{\mbox{V}_{\!\!S}}   
\newcommand{\f}{\frac}
\newcommand{\ga}{\alpha}
\newcommand{\gb}{\beta}
%\newcommand{\gg}{\gamma}
\newcommand{\gd}{\delta}
\newcommand{\gve}{\varepsilon}
\newcommand{\gf}{\varphi}
\newcommand{\gk}{\kappa}
\newcommand{\gkk}{\varkappa}
\newcommand{\gl}{\lambda}
\newcommand{\go}{\omega}
\newcommand{\gs}{\sigma}
\newcommand{\gt}{\vartheta}
\newcommand{\gy}{\upsilon}
\newcommand{\gv}{\varrho}
\newcommand{\gz}{\zeta}
\newcommand{\gD}{\Delta}
\newcommand{\gF}{\Phi}
\newcommand{\gG}{\Gamma}
\newcommand{\gL}{\Lambda}
%\newcommand{\gO}{\Omega}
\newcommand{\gS}{\Sigma}

%\long\def\forget#1\forgotten{} %
%\def\end{center}{{\mathfrak C}}
%\def\ea{{\mathfrak A}}

\newcommand{\ca}{{\mathcal A}}
\newcommand{\cb}{{\mathcal B}}
\newcommand{\cc}{{\mathcal C}}
\newcommand{\cd}{{\mathcal D}}
\newcommand{\ce}{{\mathcal E}}
\newcommand{\cf}{{\mathcal F}}
\newcommand{\cg}{{\mathcal G}}
\newcommand{\ch}{{\mathcal H}}
\newcommand{\cj}{{\mathcal J}}
\newcommand{\ck}{{\mathcal K}}
\newcommand{\cn}{{\mathcal N}}
\newcommand{\co}{{\mathcal O}}
\newcommand{\cp}{{\mathcal P}}
\newcommand{\cq}{{\mathcal Q}}
\newcommand{\cs}{{\mathcal S}}
\newcommand{\ct}{{\mathcal T}}
\newcommand{\cu}{{\mathcal U}}
\newcommand{\cv}{{\mathcal V}}
\newcommand{\cw}{{\mathcal W}}
\newcommand{\eb}{{\mathfrak B}}
\newcommand{\ed}{{\mathfrak D}}
\newcommand{\ee}{{\mathfrak E}}
\newcommand{\ef}{{\mathfrak F}}
\newcommand{\eg}{{\mathfrak G}}
\newcommand{\ej}{{\mathfrak J}}
\newcommand{\eh}{{\mathfrak H}}
\newcommand{\en}{{\mathfrak N}}
\newcommand{\eo}{{\mathfrak O}}
\newcommand{\ep}{{\mathfrak P}}
\newcommand{\eq}{{\mathfrak Q}}
\newcommand{\es}{{\mathfrak S}}
\newcommand{\et}{{\mathfrak T}}
\newcommand{\eu}{{\mathfrak U}}
\newcommand{\ev}{{\mathfrak V}}
\newcommand{\ew}{{\mathfrak W}}



%\def\NN{\mathbbm{N}}
%\def\QQ{\mathbbm{Q}}
%\def\RR{\mathbbm{R}}
%\def\SS{\mathbbm{S}}
%\def\11{\mathbbm{1}}
%\def\ZZ{\mathbbm{Z}}
%\def\TT{\mathbbm{T}}
\newcommand{\RR}{\eR}
\newcommand{\DD}{\mathbbm{D}}
\newcommand{\HH}{\mathbbm{H}}
\newcommand{\II}{\mathbbm{I}}
\newcommand{\N}{\mathbbm{N}}
\newcommand{\PP}{\mathbbm{P}}
\newcommand{\Q}{\mathbbm{Q}}
\newcommand{\RRR}{\mathbbm{R}_+}
\newcommand{\Z}{\mathbbm{Z}}
\newcommand{\RP}{{\RR\PP}} 
%\newcommand{\CP}{{\CC\PP}} 
\newcommand{\pp}{\partial}
\newcommand{\ww}{\wedge}
%\newcommand{\dc}{d^\CC}
\newcommand{\sym}{Sp(n;\RR)}
\newcommand{\ha}{\hookrightarrow}
\newcommand{\Ra}{\Rightarrow}
\newcommand{\Lra}{\Leftrightarrow} 

%\def\ni{\noindent}
%\def\b{\bigskip}
%\def\m{\medskip}
%\def\im{\mbox{Im}\,}

\newcommand{\de}{\stackrel{\mbox{\scriptsize{def}}}{=}}
%\newcommand{\id}{\mbox{id}}

%\def\sq{\square}
%\def\tr{\triangle}
%\def\trd{\bigtriangledown}
%\def\proof{\noindent {\it Proof. \;}}


%	La num\'erotation des exercices


\newcounter{exoNico}
\setcounter{exoNico}{1}
\newcommand{\exerNico}{\stepcounter{exoNico}{\bf Exercice }\arabic{exoNico}. }


%++++++++++ACCENTS++++++++++++++++++
\newcommand{\e}{\'{e}}
%\newcommand{\esp}{\'{e }}
%\newcommand{\eg}{\`{e}}
\newcommand{\ac}{\`{a} }
%\newcommand{\meme}{m\^{e}me }
\newcommand{\ou}{o\`{u} }

%+++++++++++NEWCOMMANDS+++++++++++
\newcommand{\dst}{\displaystyle}
\newcommand{\ba}{\begin{array}}
%\newcommand{\ea}{\end{array}}
%++++++++++FORMULAS+++++++++++++
\newcommand{\hs}{\hspace{0.3cm}}
%\newcommand{\eps}{\epsilon}
%\newcommand{\f}{\frac}
\newcommand{\arcth}{{\rm arctanh}}
\newcommand{\arcsh}{{\rm arcsinh}}
\newcommand{\arcch}{{\rm arccosh}}
\newcommand{\csec}{{\rm cosec}}
\newcommand{\cotan}{{\rm cotg}}
\newcommand{\cis}{(\cos+i\sin)( }
%\newcommand{\ra}{\rightarrow}
\newcommand{\lra}{\longrightarrow}
\newcommand{\ceil}{\rm plafond(}
\newcommand{\dfdu}{\frac{\partial f}{\partial u}}
\newcommand{\dfdw}{\frac{\partial f}{\partial w}}
\newcommand{\dfdx}{\frac{\partial f}{\partial x}}
\newcommand{\dfdy}{\frac{\partial f}{\partial y}}
\newcommand{\dudx}{\frac{\partial u}{\partial x}}
\newcommand{\dvdx}{\frac{\partial v}{\partial x}}
\newcommand{\dUdx}{\dfrac{\partial U}{\partial x}}
\newcommand{\dVdx}{\dfrac{\partial V}{\partial x}}
\newcommand{\dhdx}{\frac{\partial h}{\partial x}}
\newcommand{\dhdy}{\frac{\partial h}{\partial y}}
\newcommand{\dgdu}{\frac{\partial g}{\partial u}}
\newcommand{\dgdv}{\frac{\partial g}{\partial v}}
\newcommand{\dgudu}{\frac{\partial g_1}{\partial u}}
\newcommand{\dgudv}{\frac{\partial g_1}{\partial v}}
\newcommand{\dgddu}{\frac{\partial g_2}{\partial u}}
\newcommand{\dgddv}{\frac{\partial g_2}{\partial v}}
\newcommand{\dhdu}{\frac{\partial h}{\partial u}}
\newcommand{\dhdv}{\frac{\partial h}{\partial v}}
\newcommand{\dldu}{\frac{\partial l}{\partial u}}
\newcommand{\dldv}{\frac{\partial l}{\partial v}}
\newcommand{\dgudr}{\frac{\partial g_1}{\partial r}}
\newcommand{\dgudth}{\frac{\partial g_1}{\partial \theta}}
\newcommand{\dgddr}{\frac{\partial g_2}{\partial r}}
\newcommand{\dgddth}{\frac{\partial g_2}{\partial \theta}}
\newcommand{\dfdv}{\frac{\partial f}{\partial v}}
\newcommand{\dfdr}{\frac{\partial f}{\partial r}}

\newcommand{\dfdth}{\frac{\partial f}{\partial \theta}}
\newcommand{\ddfdx}{\frac{\partial^2 f}{\partial x^2}}
\newcommand{\ddfdy}{\frac{\partial^2 f}{\partial y^2}}
\newcommand{\ddfdxy}{\frac{\partial^2 f}{\partial y\partial x}}
\newcommand{\ddfdt}{\frac{\partial^2 f}{\partial^2 t}}

\newcommand{\ud}{\underline}

 % *** Blackboard math symbols ***
 %\newcommand{\N}{\mathbb{N}}
 %\newcommand{\Z}{\mathbb{Z}}
 %\newcommand{\Q}{\mathbb{R}}
 %\newcommand{\K}{\mathbb{K}}
 %\newcommand{\R}{\mathbb{R}}
 %\newcommand{\C}{\mathbb{C}}
 %\newcommand{\F}{\mathbb{F}}
 %\newcommand{\J}{\mathbb{J}}
\newcommand{\Qn}{\mathbb{Q}}

\newcommand{\Rn}{\eR} 
\newcommand{\Nn}{\eN}


\newtheorem{theo}{Th{\'e}or{\`e}me}[section]
\newtheorem{defn}{D{\'e}finition}
\newtheorem{prop}{Proposition}     % redef encore dans Chafaï
%\newtheorem{rem}{Remarque}[section]
\newtheorem{lem}{Lemme}[section]
\newcommand{\R}{\mathbb{R}}
\newcommand{\dem}{\textbf{D{\'e}monstration.}}
\newcommand{\vc}[1]{\boldsymbol{#1}}
\newcommand{\p}{\textrm{P}}
%\newcommand{\e}{\textrm{E}}
\newcommand{\mbt}{arbre binaire markovien}
\newcommand{\mbts}{arbres binaires markoviens}

\newcommand{\ea}{\end{array}}


%%%%%%%%%%%%%%%%%%%%%%%%%%%%%%%%%%%%%%
%
% les petis yeux 
%
%%%%%%%%%%%%%%%%%%%%%%%%%%%%%%%%%%%%%%%%%%%%%

\newcommand{\coolexo}{$\circledast\circledast$}
\newcommand{\boringexo}{$\circleddash\circleddash$}
\newcommand{\minsyndical}{$\odot\odot$}
\newcommand{\mortelexo}{$\obslash\oslash$}


%%%%%%%%%%%%%% FIN TRUCS DE YVIK %%%%%%%%%%%%%%%%%%%%%%

%%%%%%%%%%%%%% TRUCS DE PIERRE %%%%%%%%%%%%%%%%%%%%%%


% Le paquet array est là pour faire fonctionner l'environement arrowcases dans les trucs de Pierre.
\usepackage{array}

%\documentclass[11pt,a4paper,openany]{book}
%\usepackage[ansinew]{inputenc}
%\usepackage{pstricks, pst-node, array, ifpdf, comment, pst-plot}
%\usepackage[marginparwidth=2cm]{geometry}
%\usepackage[dvips,colorlinks]{hyperref}
%\usepackage[frenchb]{entetes}

%\usepackage{bigcenter}
%%%% debut macro %%%%
%%% ----------debut de bigcenter.sty--------------

%%% nouvel environnement bigcenter
%%% pour centrer sur toute la page (sans overfull)
%\makeatletter
%\newskip\@bigflushglue \@bigflushglue = -100pt plus 1fil

%\def\bigcenter{\trivlist \bigcentering\item\relax}
%\def\bigcentering{\let\\\@centercr\rightskip\@bigflushglue%
%\def\endbigcenter{\endtrivlist}

%\leftskip\@bigflushglue
%\parindent\z@\parfillskip\z@skip}
%\makeatother

%%% ----------fin de bigcenter.sty--------------
%%%% fin macro %%%%

%\input{mfpic}

% À régler par l'utilisateur
\newlength{\arrowsep}\setlength{\arrowsep}{3pt}
\newlength{\arrowlength}\setlength{\arrowlength}{1cm}

% Cet environnement est sympa, mais il dépend trop de ps; en tout cas il ne passe pas dans pdflatex
% 15 mars 1012
%\newenvironment{arrowcases}	{%
%			\pnode(\arrowsep,0.5ex){A}%
%			\hspace{\arrowlength}%
%			\begin{array}{>{\displaystyle\pnode(-\arrowsep,0.5ex){B}}l<{\ncline{A}{B}}@{}}
%				}
%			{
%			  \end{array}
%			}

\newenvironment{arrowcases}%
{\begin{cases}}
{\end{cases}}



\makeatletter %% \limite[condition]x x_0
\newcommand*{\limite}[3][\@empty]{\lim_{\substack{#2\rightarrow#3\\#1}}}
\makeatother

% \newenvironment{split+justif}{%
% \begin{split}%
% \let\ampori&
% \def&#1&#2\\{}
% }{%

% \end{split}}%

%\def\ncov{\tilde\nabla} % Nouvelle dérivée covariante
\newcommand*\sev{<} % 

%\newcommand{\hgot}{\mathfrak{h}} % h gothique (ss algebre de Lie)
%\def\var#1{{\mathbf #1}} % \var <-> Une variété
%\def\pardef{\stackrel{def}{=}} % = par définition.
%\newcommand{\bbar#1}{\bar{\bar{#1}}}
%\def\cov{\nabla} % Derivee covariante / connexion
%\newcommand{\gl}{\mathfrak{gl}} % algèbre linéaire
%\def\doubleprime{{\prime\prime}} % Isomorphique 
%\def\scal(#1,#2){\langle #1,#2\rangle}
%\def\agit(#1,#2){\langle #1,#2^\vee\rangle}
%\let\phiori\phi
%\let\phi\varphi
\let\ssi\iff
%\def\iddc{\mathcal I}
\newcommand*{\ideal}[1]{\{#1\}}
\newcommand*{\fleche}[1]{\stackrel{#1}\longrightarrow}

%\newcounter{exercice}
%\setcounter{exercice}{0}
\setcounter{CountExercice}{0}

% \newenvironment{exo}[1][\relax]{%
% \stepcounter{exercice}%
% \par\medskip%
% #1{\textbf{Exercice}~\arabic{exercice}.}\quad}%
% {\par}

% \newenvironment{rep}{\hspace{1em}\par\textbf{Solution
%     proposée.\quad}}{\par\noindent\hrulefill\par}

%\date{}
\newcommand{\Acplx}{A_\cdot}
\newcommand{\Bcplx}{B_\cdot}
\newcommand{\toisom}{\fleche\simeq}
\newcommand{\D}{\partial}
%\newcommand{\cat}[1]{{\bf #1}}
%\newcommand{\donc}{\Rightarrow}
%\newcommand{\im}{\text{im}}
%\newcommand{\coker}{\text{coker}}
\newcommand{\lied}{\mathcal L}
\newcommand*{\nom}[1]{\textsc{#1}}
\makeatletter
\newcommand*{\attention}[1]{\@latex@warning{#1}{!\small\bf #1!}\marginpar{Warning}}
\makeatother
\newcommand*{\inner}{\imath}
\newcommand*{\newexo}{}
\newcommand*{\principe}{}
\newcommand*{\etape}{}
\newcommand*{\preuve}{}
\newcommand*{\exr}{\item}
%\def\prim#1\expandafter\d#2 {\int #1\d#2}

\newcommand*{\crochets}[1]{\Bigl[ #1 \Bigr]}
\newcommand*{\llbrack}[1]{\left\lbrack #1 \right\lbrack}
\newcommand*{\rlbrack}[1]{\left\rbrack #1 \right\lbrack}
\newcommand*{\lrbrack}[1]{\left\lbrack #1 \right\rbrack}
\newcommand*{\rrbrack}[1]{\left\rbrack #1 \right\rbrack}
\newcommand*{\vecteur}[1]{\mathbf{#1}}


%% Maths : Les ensembles
\newcommand*{\ens}[1]{\mathbb{#1}} % Ensemble de nombres
\newcommand*{\var}[1]{\mathbf{#1}} % Variété
\newcommand*{\alg}[1]{\mathcal{#1}} % Algèbre
%\newcommand*{\RR}{\ens R}%
\newcommand*{\TT}{\ens T}% Tore !
%\expandafter\show\csname SS \endcsname
%\renewcommand*{\SS}{\var S}% 
%\newcommand*{\CC}{\ens C}%
\newcommand*{\ZZ}{\ens Z}%
\newcommand*{\QQ}{\ens Q}%
\newcommand*{\NN}{\ens N}%
\newcommand{\schwartz}{\mathcal S} % Espace de Schwartz
\newcommand*{\topologie}{\mathscr{T}}
\newcommand*{\Topologie}{\textcursive{T}}
\newcommand{\LL}{\text{\textup{L}}} %% Espace de Lebesgue droit
\newcommand{\Ll}{\mathcal{L}} %% Lebesgue ronde
\newcommand{\fronde}{\mathcal{F}} %% Transformée de Fourier.
\newcommand{\sigmaalgebre}[1]{\mathcal{#1}} %% Une sigma algèbre...
\DeclareMathOperator{\SymMatrix}{Sym}
\DeclareMathOperator{\ASymMatrix}{ASym}
\newcommand{\Sym}{\SymMatrix}
\newcommand{\ASym}{\ASymMatrix}
\newcommand{\transpose}[1]{{\vphantom{#1}}^{\mathit t}{\/#1}}
\newcommand*{\Sp}{\textup{Sp}}
\newcommand*{\Gl}{\textup{GL}}
%\renewcommand*{\sp}{\textup{sp}}
\newcommand*{\dprime}{{\prime\prime}}
%\show\span
%\newcommand*{\Span}[1]{\mathopen> #1 \mathclose<}

%% Maths : Symboles divers
%\newcommand{\pp}{\text{\textup{~p.p.}}} %% Presque partout
%\PackageWarning{entetes}{Redefining command \d}
%\renewcommand{\d}{\mbox{$\,$\textrm{d}}}
\newcommand{\surj}{\vers}
\newcommand{\isom}{\simeq}
\newcommand*{\Tau}{\alg T}
\newcommand{\cdv}{\mathfrak{X}} % Champs de vecteurs


%% Maths : Constructions

%\let\Exp\exp
%\renewcommand{\exp}[1]{e^{#1}} % On préfère e^{} que exp{}

%\renewcommand{\exp}[1]{e^{#1}} % On préfère e^{} que exp{}
%\renewcommand{\vec}[1]{\mathbf{#1}} % Désigner un vecteur
\newcommand{\set}[1]{\left\{#1\right\}} % Un ensemble { }
\newcommand*{\abs}[1]{\left\vert#1\right\vert} % Valeur absolue.
\newcommand*{\module}[1]{\left\vert#1\right\vert} % Valeur absolue.
\newcommand*{\norme}[1]{\left\Vert#1\right\Vert} % norme
\newcommand*{\ordre}[1]{\left\vert#1\right\vert} % L'ordre d'un élément.
%\def\scal(#1,#2){% Produit scalaire.
%  \PackageWarning{entetes}{Obsolete command \string\scal}%
%  \scalprod{#1}{#2}%
%}
\newcommand*{\scalprod}[2]{\left\langle #1,#2\right\rangle}
\let\dual\ast

\newcommand*{\pardef}{\stackrel{\text{def}}{=}} % Par définition.
\newcommand*{\iffdefn}{\stackrel{\text{def}}{\iff}} % Par définition.
%\newcommand*{\telque}{\mbox{~\entetes@name@telque~}} % tel que, dans un ensemble.
\newcommand*{\Defn}[1]{\emph{#1}} %
\newcommand*{\tensor}{\otimes}
\newcommand*{\pder}[2]{\frac{\partial #1}{\partial #2}}

%{{{ Fraction in-line plus jolie
% \DeclareRobustCommand\sfrac[1]{\@ifnextchar/{\@sfrac{#1}}%
%                                             {\@sfrac{#1}/}}
% \def\@sfrac#1/#2{\leavevmode\kern.1em\raise.5ex
%          \hbox{$\m@th{\fontsize\sf@size\z@
%                            \selectfont#1}$}\kern-.1em
%          /\kern-.15em\lower.25ex
%           \hbox{$\m@th{\fontsize\sf@size\z@
%                             \selectfont#2}$}}
%}}} 

\DeclareRobustCommand{\sfrac}[3][\mathrm]{\hspace{0.1em}%
  \raisebox{0.4ex}{$#1{\scriptstyle
#2}$}\hspace{-0.1em}/\hspace{-0.07em}%
  \mbox{$#1{\scriptstyle #3}$}}



%% Maths : Opérateurs
%\DeclareMathOperator{\tr}{Tr}
%\DeclareMathOperator{\pr}{\texttt{pr}}
\DeclareMathOperator{\supp}{supp}
\DeclareMathOperator{\adh}{adh}
\DeclareMathOperator{\interior}{int}
\DeclareMathOperator{\im}{Im}
\DeclareMathOperator{\Id}{Id}
\DeclareMathOperator{\Aut}{Aut}
\DeclareMathOperator{\Iso}{Iso}
\DeclareMathOperator{\Jac}{Jac} % jacobienne
\DeclareMathOperator{\coker}{coker}
\DeclareMathOperator{\interieur}{int}
\DeclareMathOperator{\Tor}{Tor}
\DeclareMathOperator{\divg}{div}
\DeclareMathOperator{\rot}{rot}
%\DeclareMathOperator{\cosec}{cosec}


%% Pour obtenir le \Sha cyrillique...
% \RequirePackage[OT2,T1]{fontenc}
% \DeclareSymbolFont{cyrletters}{OT2}{wncyr}{m}{n}
% \DeclareMathSymbol{\Sha}{\mathalpha}{cyrletters}{"58}


% \newcounter{@institute}
% \let\authorori\author
% \def\@institute{}\def\@auteurs{}
% %\newcommand{\institute}[2]{\refstepcounter{@institute}\label{#1}\def\@institute{\@institute\small
% %#2}\set@authors}
% \newcommand{\institute}[2]{\refstepcounter{@institute}\label{#1}%
%   \let\maketitleori\maketitle%
%   \renewcommand\maketitle{\footnote{#2}\maketitleori}%
% }%
% \renewcommand{\author}[1]{\def\@auteurs{#1}\set@authors}
% \def\the@institute{${}^{(\roman{@institute})}$}
% \newcommand{\inst}[1]{\ref{#1}}

% \newcommand{\set@authors}{\authorori{\@auteurs}}% \\ \@institute}}


% \newcounter{@institute}
% \newcommand{\institute}[1]{
%   \let\labelori\label
%   \renewcommand{\label}[1]{%
%     \refstepcounter{@institute}\labelori{##1}
%     \begin{tabular}{cc}%format ?!
%     \begin{minipage}[t]
      


%   }

% }%

\newcommand*{\conclusion}{\emph{Conclusion~:~}}
\newcommand{\hint}{\par\emph{Aide~:~}\hspace{1em}}
\newcommand{\rappel}{\par\emph{Rappels~:~}\hspace{1em}}

%\newcounter{enumarray} 
%\newenvironment{enumarray}[1]{% Merci Ulrike Fischer
% \setcounter{enumarray}{0}%
% \begin{array}{% motif
%     >{% Au début de chaque ligne
%       \stepcounter{enumarray}%
%       (\alph{enumarray})%
%       \hspace{2em}
%     }% 
%     #1%
%   }% fin motif
% }{%
% \end{array}%
%}
\newenvironment{displayinline}{% displaystyle + inline.
  $\displaystyle%
}{%
  $%
}

\newcommand{\telque}{\vert\,}
\newcommand{\donc}{\Rightarrow}
 
%\newcounter{coroNico}
%\newcommand{\corrNico}{\refstepcounter{coroNico}\paragraph{Correction \arabic{coroNico}.}}


%

% Les commandes suivantes sont pour Chafaï:

%%% Macros Dj. 

%%% Pour les notes
\newcommand{\NB}[1]{{\large\textbf{*** #1 ***}}}

%%% Pour passer en mode mathématiques automatiquement 
\newcommand{\EM}{\ensuremath}

%%% Pour théorèmes etc...  Nécessite le paquetage amsthm ou bien classe amsart
%\newcommand{\THMEN}{%
% \theoremstyle{plain}%
% \newtheorem{thm}{Theorem}[section]%
% \newtheorem{cor}[thm]{Corollary}%
% \newtheorem{prop}[thm]{Proposition}%
% \newtheorem{lem}[thm]{Lemma}%
% \theoremstyle{definition}%
% \newtheorem{defi}[thm]{Definition}%
% \theoremstyle{remark}%
% \newtheorem{rem}[thm]{Remark}%
% \newtheorem{xpl}[thm]{Example}%
% \newtheorem{exe}[thm]{Exercise}%
% \newtheorem{hyp}[thm]{Hypothesis}%
%}

%\theoremstyle{plain}%
%\newtheorem{thm}{Théorème}[section]%
%\newtheorem{cor}[thm]{Corollaire}%
%\newtheorem{prop}[thm]{Proposition}%
%\newtheorem{lem}[thm]{Lemme}%
%\theoremstyle{definition}%
%\newtheorem{defi}[thm]{Définition}%
%\theoremstyle{remark}%
%%\newtheorem{rem}[thm]{Remarque}%
%\newtheorem{xpl}[thm]{Exemple}%
%%\newtheorem{exo}[thm]{Exercice}%
%\newtheorem{hyp}[thm]{Hypothèse}%
\newtheorem{eur}[numtho]{Heuristique}%
%\newtheorem{pro}[thm]{Problème}%

%%% Titre des *sections en + petit
\makeatletter
\newcommand{\SMALLSECS}{%
%\renewcommand{\section}{\@startsection%
 %{section}%                           % name
 %{1}%                                 % level
 %{0em}%                               % indent
 %{\baselineskip}%                     % beforeskip
 %{0.5\baselineskip}%                  % afterskip
 %{\normalfont\large\bfseries}}%       % style
%\renewcommand{\subsection}{\@startsection%
 %{subsection}%                        % name
 %{2}%                                 % level
 %{0em}%                               % indent
 %{\baselineskip}%                     % beforeskip
 %{0.25\baselineskip}%                 % afterskip
 %{\normalfont\bfseries}}%             % style
}
\makeatother

%%% Limites
\newcommand{\limLeb}[2]
{\underset{#1\to+\infty}{\overset{\bL^{#2}}{\longrightarrow}}}
\newcommand{\limE}[1]
{\underset{#1\to+\infty}{\overset{\text{étr.}}{\longrightarrow}}}
\newcommand{\limP}[1]
{\underset{#1\to+\infty}{\overset{\dP}{\longrightarrow}}}
\newcommand{\limL}[1]
{\underset{#1\to+\infty}{\overset{\cL}{\longrightarrow}}}
\newcommand{\limn}[1]
{\underset{#1\to+\infty}{\longrightarrow}}
\newcommand{\mylim}[2]
{\underset{#1\to#2}{\longrightarrow}}
\newcommand{\limPS}[1]
{\underset{#1\to+\infty}{\overset{\text{p.s}}{\longrightarrow}}}

%%% Heure 
\makeatletter
\providecommand{\timenow}{\@tempcnta\time
\@tempcntb\@tempcnta
\divide\@tempcntb60
\ifnum10>\@tempcntb0\fi\number\@tempcntb
\multiply\@tempcntb60
\advance\@tempcnta-\@tempcntb
:\ifnum10>\@tempcnta0\fi\number\@tempcnta}
\makeatother

% Pour le draft-stamping dans les bas de page
%\makeatletter
%\newcommand{\versiondetravail}{%
% \renewcommand{\@evenfoot}{%
% \hfil{\tiny\texttt{%
%   Version préliminaire, compilée le \today{} à \timenow.}\hfill}}%
% \renewcommand{\@oddfoot}{\@evenfoot}%
%}
%\makeatother

%%% Doubles lettres 
\newcommand{\dA}{\EM{\mathbb{A}}}
%\newcommand{\dB}{\EM{\mathbb{B}}}
\newcommand{\dC}{\EM{\mathbb{C}}}
\newcommand{\dD}{\EM{\mathbb{D}}}
%\newcommand{\dE}{\EM{\mathbb{E}}}
\newcommand{\dF}{\EM{\mathbb{F}}}
%\newcommand{\dG}{\EM{\mathbb{G}}}
\newcommand{\dH}{\EM{\mathbb{H}}}
\newcommand{\dI}{\EM{\mathbb{I}}}
\newcommand{\dJ}{\EM{\mathbb{J}}}
\newcommand{\dK}{\EM{\mathbb{K}}}
\newcommand{\dL}{\EM{\mathbb{L}}}
%\newcommand{\dM}{\EM{\mathbb{M}}}
%\newcommand{\dN}{\EM{\mathbb{N}}}
\newcommand{\dO}{\EM{\mathbb{O}}}
%\newcommand{\dP}{\EM{\mathbb{P}}}
\newcommand{\dQ}{\EM{\mathbb{Q}}}
\newcommand{\dR}{\EM{\mathbb{R}}}
\newcommand{\dS}{\EM{\mathbb{S}}}
%\newcommand{\dT}{\EM{\mathbb{T}}}
%\newcommand{\dU}{\EM{\mathbb{U}}}
\newcommand{\dV}{\EM{\mathbb{V}}}
\newcommand{\dW}{\EM{\mathbb{W}}}
\newcommand{\dX}{\EM{\mathbb{X}}}
\newcommand{\dY}{\EM{\mathbb{Y}}}
\newcommand{\dZ}{\EM{\mathbb{Z}}}

%%% Lettres droites
\newcommand{\rA}{\EM{\mathrm{A}}}
\newcommand{\rB}{\EM{\mathrm{B}}}
\newcommand{\rC}{\EM{\mathrm{C}}}
\newcommand{\rD}{\EM{\mathrm{D}}}
\newcommand{\rE}{\EM{\mathrm{E}}}
\newcommand{\rF}{\EM{\mathrm{F}}}
\newcommand{\rG}{\EM{\mathrm{G}}}
\newcommand{\rH}{\EM{\mathrm{H}}}
\newcommand{\rI}{\EM{\mathrm{I}}}
\newcommand{\rJ}{\EM{\mathrm{J}}}
\newcommand{\rK}{\EM{\mathrm{K}}}
\newcommand{\rL}{\EM{\mathrm{L}}}
\newcommand{\rM}{\EM{\mathrm{M}}}
\newcommand{\rN}{\EM{\mathrm{N}}}
\newcommand{\rO}{\EM{\mathrm{O}}}
\newcommand{\rP}{\EM{\mathrm{P}}}
\newcommand{\rQ}{\EM{\mathrm{Q}}}
\newcommand{\rR}{\EM{\mathrm{R}}}
\newcommand{\rS}{\EM{\mathrm{S}}}
\newcommand{\rT}{\EM{\mathrm{T}}}
\newcommand{\rU}{\EM{\mathrm{U}}}
\newcommand{\rV}{\EM{\mathrm{V}}}
\newcommand{\rW}{\EM{\mathrm{W}}}
\newcommand{\rX}{\EM{\mathrm{X}}}
\newcommand{\rY}{\EM{\mathrm{Y}}}
\newcommand{\rZ}{\EM{\mathrm{Z}}}

% Lettres caligraphiques
\newcommand{\cA}{\EM{\mathcal{A}}}
\newcommand{\cB}{\EM{\mathcal{B}}}
\newcommand{\cC}{\EM{\mathcal{C}}}
\newcommand{\cD}{\EM{\mathcal{D}}}
\newcommand{\cE}{\EM{\mathcal{E}}}
\newcommand{\cF}{\EM{\mathcal{F}}}
\newcommand{\cG}{\EM{\mathcal{G}}}
\newcommand{\cH}{\EM{\mathcal{H}}}
\newcommand{\cI}{\EM{\mathcal{I}}}
\newcommand{\cJ}{\EM{\mathcal{J}}}
\newcommand{\cK}{\EM{\mathcal{K}}}
%\newcommand{\cL}{\EM{\mathcal{L}}}
\newcommand{\cM}{\EM{\mathcal{M}}}
\newcommand{\cN}{\EM{\mathcal{N}}}
\newcommand{\cO}{\EM{\mathcal{O}}}
\newcommand{\cP}{\EM{\mathcal{P}}}
\newcommand{\cQ}{\EM{\mathcal{Q}}}
\newcommand{\cR}{\EM{\mathcal{R}}}
\newcommand{\cS}{\EM{\mathcal{S}}}
\newcommand{\cT}{\EM{\mathcal{T}}}
\newcommand{\cU}{\EM{\mathcal{U}}}
\newcommand{\cV}{\EM{\mathcal{V}}}
\newcommand{\cW}{\EM{\mathcal{W}}}
\newcommand{\cX}{\EM{\mathcal{X}}}
\newcommand{\cY}{\EM{\mathcal{Y}}}
\newcommand{\cZ}{\EM{\mathcal{Z}}}

%%% Lettres grasses 
\newcommand{\bA}{\EM{\mathbf{A}}}
\newcommand{\bB}{\EM{\mathbf{B}}}
\newcommand{\bC}{\EM{\mathbf{C}}}
\newcommand{\bD}{\EM{\mathbf{D}}}
\newcommand{\bE}{\EM{\mathbf{E}}}
\newcommand{\bF}{\EM{\mathbf{F}}}
\newcommand{\bG}{\EM{\mathbf{G}}}
\newcommand{\bH}{\EM{\mathbf{H}}}
\newcommand{\bI}{\EM{\mathbf{I}}}
\newcommand{\bJ}{\EM{\mathbf{J}}}
\newcommand{\bK}{\EM{\mathbf{K}}}
\newcommand{\bL}{\EM{\mathbf{L}}}
\newcommand{\bM}{\EM{\mathbf{M}}}
\newcommand{\bN}{\EM{\mathbf{N}}}
\newcommand{\bO}{\EM{\mathbf{O}}}
\newcommand{\bP}{\EM{\mathbf{P}}}
\newcommand{\bQ}{\EM{\mathbf{Q}}}
\newcommand{\bR}{\EM{\mathbf{R}}}
\newcommand{\bS}{\EM{\mathbf{S}}}
\newcommand{\bT}{\EM{\mathbf{T}}}
\newcommand{\bU}{\EM{\mathbf{U}}}
\newcommand{\bV}{\EM{\mathbf{V}}}
\newcommand{\bW}{\EM{\mathbf{W}}}
\newcommand{\bX}{\EM{\mathbf{X}}}
\newcommand{\bY}{\EM{\mathbf{Y}}}
\newcommand{\bZ}{\EM{\mathbf{Z}}}

%%% Quelques lettres grecques et symboles
\newcommand{\al}{\alpha}
\newcommand{\be}{\beta}
\newcommand{\De}{\Delta}
%\newcommand{\de}{\delta}
%\newcommand{\ga}{\gamma}
\newcommand{\Ga}{\Gamma}
\newcommand{\g}{\gamma}
\newcommand{\gn}{\g_n}
%\newcommand{\gt}[1]{\g^{\otimes #1}}
\newcommand{\la}{\lambda}
\newcommand{\La}{\Lambda}
\newcommand{\lan}{\la_n}
\newcommand{\na}{\nabla}
\newcommand{\Om}{\Omega}
\newcommand{\om}{\omega}
\newcommand{\ph}{\Phi}
\newcommand{\Si}{\Sigma}
\newcommand{\si}{\sigma}
\newcommand{\Te}{\Theta}
\newcommand{\te}{\theta}
\newcommand{\ta}{\tau}
\newcommand{\veps}{\varepsilon}
\newcommand{\vphi}{\varphi}
\newcommand{\bul}{\EM{\bullet}}

%%% Prototype de fonction avec dimensionnement
%\newcommand{\p}[4]{{#3}\!\left#1{#4}\right#2} 

%%% Normes et assimilées 
\newcommand{\ABS}[1]{\EM{{\left| #1 \right|}}} % |1|
\newcommand{\BRA}[1]{\EM{{\left\{#1\right\}}}} % {1}
\newcommand{\DP}[1]{\EM{{\left<#1\right>}}} % <1>
\newcommand{\NRM}[1]{\EM{{\left\| #1\right\|}}} % ||1||
\newcommand{\NI}[1]{\EM{{\left\| #1\right\|}_\infty}} % norme infinie
\newcommand{\OSC}[1]{\EM{{\p(){\mathrm{osc}}{#1}}}} % oscillation
\newcommand{\PAR}[1]{\EM{{\left(#1\right)}}} % (1)
\newcommand{\BPAR}[1]{\EM{{\biggl(#1\biggr)}}} % (1)
\newcommand{\BABS}[1]{\EM{{\biggl|#1\biggr|}}} % (1)
\newcommand{\pd}{\EM{\partial}} % dérivée partielle
\newcommand{\PD}[2]{\EM{{\frac{\partial #1}{\partial #2}}}} % dérivée partielle en fraction
\newcommand{\SBRA}[1]{\EM{{\left[#1\right]}}} % [1]
\newcommand{\VT}[1]{\EM{{\| #1\|}_{\mbox{{\scriptsize VT}}}}} % variation totale
\newcommand{\LIP}[1]{\EM{\|#1\|_{\mathrm{Lip}}}} % norme de lipschitz

%%% Fonctions et fonctionnelles 
\newcommand{\sentf}{\bH}
\newcommand{\sent}[1]{\p(){\sentf}{#1}}
\newcommand{\bentf}[1]{\bH_{#1}}
\newcommand{\bent}[2]{\p(){\bentf{#1}}{#2}}
\newcommand{\eentf}{\bN}
\newcommand{\eent}[1]{\p(){\eentf}{#1}}
\newcommand{\entf}[1]{\mathbf{Ent}_{#1}}
\newcommand{\ent}[2]{\p(){\entf{#1}}{#2}}
\newcommand{\ientf}[1]{\bH^{#1}}
\newcommand{\ient}[2]{\p(){\ientf{#1}}{#2}}
\newcommand{\rentf}{\mathbf{Ent}}
\newcommand{\rent}[2]{\p(){\rentf}{#1\,\vert\,#2}}
\newcommand{\entr}{\mathbf{Ent}_r}
\newcommand{\enef}[1]{\boldsymbol{\mathcal{E}}_{#1}}
\newcommand{\ene}[2]{\p(){\enef{#1}}{#2}}
\newcommand{\imutf}{\bI}
\newcommand{\imut}[1]{\p(){\imutf}{#1}}
\newcommand{\fcrf}{\dI}
\newcommand{\fcr}[1]{\p(){\fcrf}{#1}}
\newcommand{\fishf}{\bJ}
\newcommand{\fish}[1]{\p(){\fishf}{#1}}
\newcommand{\fishmf}{\dJ}
\newcommand{\fishm}[1]{\p(){\fishmf}{#1}}
\newcommand{\moyf}[1]{\bE_{#1}}
\newcommand{\moyp}[2]{{\moyf{#1}}{#2}}
\newcommand{\bmoy}[2]{\moyf{#1}\!\biggl(#2\biggr)}
\newcommand{\corrf}[1]{\mathbf{Cor}_{#1}}
\newcommand{\corr}[3]{\p(){\corrf{#1}}{#2,#3}}
\newcommand{\covf}[1]{\mathbf{Cov}_{#1}}
\newcommand{\cov}[3]{\p(){\covf{#1}}{#2,#3}}
\newcommand{\bcov}[3]{\covf{#1}\!\biggl(#2,#3\biggr)}
%\newcommand{\suppf}{\mathrm{supp}}
%\newcommand{\supp}[1]{\suppf\PAR{#1}}
\newcommand{\varf}[1]{\mathbf{Var}_{#1}}
%\newcommand{\var}[2]{\p(){\varf{#1}}{#2}}
\newcommand{\varp}[2]{{\varf{#1}}{#2}}
\newcommand{\bvar}[2]{\varf{#1}\!\biggl(#2\biggr)}
\newcommand{\Kf}{\bK}
\newcommand{\K}[1]{\p(){\Kf}{#1}}
\newcommand{\vrs}[1]{\mathbf{L}_{#1}}

%%% Ensembles, espaces de fonctions... 
%\newcommand{\C}[1]{\p(){\cC}{#1}}
\newcommand{\Cb}[1]{\p(){\cC_b}{#1}}
\newcommand{\Cc}[1]{\p(){\cC_c}{#1}}
\newcommand{\Cn}[2]{\p(){\cC^{#1}}{#2}}
\newcommand{\Ci}[1]{\Cn{\infty}{#1}}
\newcommand{\Cic}[1]{\p(){\cC_c^\infty}{#1}}
\newcommand{\Cnc}[2]{\p(){\cC_c^{#1}}{#2}}
\newcommand{\Cnb}[2]{\p(){\cC_b^{#1}}{#2}}
\newcommand{\Cib}[1]{\p(){\cC_b^\infty}{#1}}
\newcommand{\leb}[2]{\p(){\bL^{#1}}{#2}}
\newcommand{\lebb}[1]{\bL^{#1}}

%%% Determinant, trace...
\newcommand{\Tr}{\mathrm{Tr\,}}
\newcommand{\Det}[1]{\mathrm{Det}\,}
\newcommand{\TR}[1]{\p(){\mathrm{Tr}}{#1}}
\newcommand{\DIAG}[1]{\p(){\mathrm{Diag}}{#1}}
\newcommand{\DET}[1]{\p(){\mathrm{Det}}{#1}}
\newcommand{\SIG}[1]{\p(){\mathrm{Sign}}{#1}}
\newcommand{\ID}{\mathbf{Id}}
%\newcommand{\Id}{\mathrm{Id}}
\newcommand{\Mo}{\mathbf{1}}
\newcommand{\Vo}{\mathrm{1}}

%%% Semi-groupes, générateurs, carré-du-champ... 
\newcommand{\CD}{CD(\rho ,\infty)}
\newcommand{\GA}{\boldsymbol{\Gamma}}
\newcommand{\GD}{\GA_{\!\!{\mathbf 2}}}
\newcommand{\GI}{\bL}
\newcommand{\GR}{\nabla}
\newcommand{\GIV}{\EM{\overrightarrow{\GI}}}
\newcommand{\GIB}{\EM{\overline{\GI}}}
\newcommand{\LA}{\boldsymbol{\Delta}}
\newcommand{\ROT}{\mathbf{rot}\,}
\newcommand{\DIV}{\mathbf{div}\,}
\newcommand{\PT}[1][t]{\mathbf{P}_{\!#1}}
\newcommand{\SGf}[1]{{\mathbf P}_{#1}}
\newcommand{\SG}[2]{\p(){\SGf{\!#1}}{#2}}
\newcommand{\SGQf}[1]{{\mathbf Q}_{#1}}
\newcommand{\SGQ}[2]{\p(){\SGQf{#1}}{#2}}
\newcommand{\isopf}{\mathbf{I}}
\newcommand{\isop}[1]{\p(){\isopf}{#1}}
\newcommand{\disop}[1]{\p(){\isopf'}{#1}}
\newcommand{\ddisop}[1]{\p(){\isopf''}{#1}}
\newcommand{\HESS}[1]{\GR^2\!{#1}}
\newcommand{\Hess}[1]{{\p(){\mathrm{Hess}}{#1}}}
\newcommand{\HAM}{\bH}
\newcommand{\POT}{\bV}
\newcommand{\PF}{\bZ}
\newcommand{\DOM}{\EM{\cD_2(\GI)}}
\newcommand{\laV}{\EM{\overrightarrow{\la}}}
\newcommand{\DOML}{\EM{\cD_2\PAR{\GI}}}
\newcommand{\DOMB}{\EM{\cD_2(\GIB)}}
\newcommand{\DOMV}{\EM{\cD_2(\GIV)}}
\newcommand{\GAV}{\EM{\overrightarrow{\GA}}}
%\newcommand{\D}{\mathbf{D}}

%%% Topo
\newcommand{\inter}[1]{{\overset{\circ}{#1}}} % interior
\newcommand{\adher}[1]{{\overline{#1}}} % adherence
\newcommand{\ADH}[1]{\mathbf{adh}(#1)}
\newcommand{\INT}[1]{\mathbf{int}(#1)}

%%% Racourcis vers des noms propres 
\newcommand{\bern}{\textsc{Bernoulli}}
\newcommand{\bob}{\textsc{Bobkov}}
\newcommand{\boc}{\textsc{Bochner}}
%\newcommand{\cs}{\textsc{Cauchy}-\textsc{Schwarz}}
\newcommand{\fub}{\textsc{Fubini}}
\newcommand{\gaus}{\textsc{Gauss}}
\newcommand{\hol}{\textsc{Hölder}}
\newcommand{\jen}{\textsc{Jensen}}
\newcommand{\lap}{\textsc{Laplace}}
\newcommand{\ls}{\sobo{} logarithmique}
\newcommand{\mink}{\textsc{Minkowski}}
\newcommand{\mrkv}{\textsc{Markov}}
%\newcommand{\ou}{\textsc{Ornstein-Uhlenbeck}}
\newcommand{\poin}{\textsc{Poincaré}}
\newcommand{\pois}{\textsc{Poisson}}
\newcommand{\ric}{courbure de \textsc{Ricci}}
\newcommand{\shan}{\textsc{Shannon}}
\newcommand{\sobo}{\textsc{Sobolev}}

%%% Divers
\newcommand{\SSK}[1]{\substack{#1}}
\renewcommand{\leq}{\leqslant}
\renewcommand{\geq}{\geqslant}
\newcommand{\bs}{\EM{\backslash}} 
\newcommand{\Inf}{\boldsymbol{\inf}}
\newcommand{\Sup}{\boldsymbol{\sup}}
\newcommand{\ds}[1]{\EM{\displaystyle{#1}}}
%\newcommand{\eg}{\overset{\Delta}{=}}
\newcommand{\fdefeq}{\overset{\text{déf.}}{=}}
\newcommand{\edefeq}{\overset{\text{def.}}{=}}
\newcommand{\defeq}{:=}
\newcommand{\equ}{\; \Leftrightarrow \;}
\newcommand{\ex}{\exists \,}
\newcommand{\imp}{\Rightarrow  \;}
%\newcommand{\ssi}{{\it ssi}}
\newcommand{\tout}{\forall \,}
%\newcommand{\tq}{\,|\,}
%\newcommand{\1}{\hbox{1}\!\!\hbox{I}} %Bug avec paquet xy
\newcommand{\DSFRAC}[2]{\EM{\frac{\ds{#1}}{\ds{#2}}}}
\newcommand{\SR}[2]{\EM{\strackrel{#1}{#2}}}

%Ajout à Babel
\def\Ieme{\up{\lowercase{ième}}\xspace}

% Here are two LaTeX macros for the word "C++". They prevent line breaks
% between the C and "++", and the first packs the two "+"s close to each
% other but the second does not. Try them both and see which one you
% like best.
%\newcommand{\CC}{C\nolinebreak\hspace{-.05em}\raisebox{.4ex}{\tiny\bf +}%
%\nolinebreak\hspace{-.10em}\raisebox{.4ex}{\tiny\bf +}}
%\def\CC{{C\nolinebreak[4]\hspace{-.05em}\raisebox{.4ex}{\tiny\bf ++}}}
% Here are two more LaTeX macros for the word "C++". They allow line
% breaks between the C and "++", which may not be desirable, but they're
% included here just in case.
%\def\CC{C\raise.22ex\hbox{{\footnotesize +}}\raise.22ex\hbox{\footnotesize +}}
%\def\CC{{C\hspace{-.05em}\raisebox{.4ex}{\tiny\bf ++}}}


\usepackage{amsmath,amsfonts,amssymb,amsthm}
%\usepackage[a4paper,portrait]{geometry}
%\geometry{left=2.5cm,right=2.5cm}
\usepackage{graphicx}
\usepackage{rotating}
\usepackage{multicol}
\usepackage{moreverb}
%\usepackage[latin1]{inputenc}
\usepackage{xspace}
%\usepackage[francais]{babel} 
\usepackage{color} 
\usepackage{listingsutf8}

\lstset{extendedchars=true,%
       basicstyle=\small,%
       commentstyle=\ttfamily\color[rgb]{0,0,0.5},%
       stringstyle=\color[rgb]{0,0.5,0},%
       frame=tblr,frameround=tttt,%
        extendedchars=\true,       %   Ces deux lignes sont les miennes pour faire fonctionner le UTF8
       inputencoding=utf8/latin1   %
       %labelstyle=\tiny, labelstep=1,labelsep=10pt
       }

%\ifx\pdfoutput\undefined
% % no problemo.
%\else % compatibilité listings/hyperref.
% \makeatletter
% \providecommand*{\toclevel@lstlisting}{0}
% \makeatother
%\fi

%\input{macros}
%\THMFR
%
%\newcommand{\MFILE}[1]{
%\begin{lstinputlisting}[language=Matlab]{#1.m}%
%\label{co:#1}% 
%\end{lstinputlisting}}
%
\newcommand{\OC}{Octave}
%\newcommand{\SL}{Scilab}
%\newcommand{\MP}{Maple}
\newcommand{\ML}{Matlab}
%\newcommand{\MU}{Mupad}
\newcommand{\SB}{Stixbox}
%
\newcommand{\FIG}[4]{%  fname scale floatparams caption
 \begin{figure}[#3]%
  \begin{center}%
  \includegraphics[scale=#2]{#1}%
  \caption{#4}%
  \label{fi:#1}%
  \end{center}%
 \end{figure}
}
%
%% Parce que j'ai l'impression que ces figures font mourir la conversion en pdf
%\renewcommand{\FIG}[4]{}
