Nous donnons ici quelque idées de développements associés aux leçons. Parfois, il est bon d'ajouter quelque lemmes au développement proposé, si il est trop court.


\paragraph{104 - Groupes finis. Exemples et applications.}
\paragraph{105 - Groupe des permutations d’un ensemble fini. Applications.}
\paragraph{106 - Groupe linéaire d’un espace vectoriel de dimension finie $E$ , sous-groupes de $\GL(E)$. Applications.}
\paragraph{108 - Exemples de parties génératrices d’un groupe. Applications.}
\paragraph{109 - Anneaux $\eZ/n\eZ$. Applications.}

\paragraph{110 - Nombres premiers. Applications.}
\begin{itemize}
    \item Structure des groupes d'ordre \( pq\), théorème \ref{ThoLnTMBy}.
    \item Divergence de la somme des inverses des nombres premiers, théorème \ref{ThonfVruT}.
\end{itemize}

\paragraph{111 - Anneaux principaux. Applications}
\begin{itemize}
    \item Théorème des deux carrés \ref{ThospaAEI}.
    \item Polynôme minimal d'endomorphisme semi-simple, théorème \ref{ThoFgsxCE}.
\end{itemize}

\paragraph{112 - Corps finis. Applications.}
\begin{itemize}
    \item Théorème de Chevalley-Warning \ref{ThoLTcYKk}.
    \item Loi de réciprocité quadratique \ref{ThoMiEiUm}.
\end{itemize}

\paragraph{116 - Polynômes irréductibles à une indéterminée. Corps de rupture. Exemples et applications.}
\paragraph{119 - Exemples d’actions de groupes sur les espaces de matrices.}
\paragraph{120 - Dimension d’un espace vectoriel (on se limitera au cas de la dimension finie). Rang. Exemples et applications.}
\paragraph{123 - Déterminant. Exemples et applications.}
\paragraph{124 - Polynômes d’endomorphisme en dimension finie. Réduction d’un endomorphisme en dimension finie. Applications.}
\paragraph{128 - Endomorphismes trigonalisables. Endomorphismes nilpotents.}
\paragraph{131 - Formes quadratiques sur un espace vectoriel de dimension finie. Orthogonalité, isotropie. Applications.}
\paragraph{132 - Formes linéaires et hyperplans en dimension finie. Exemples et applications.}
\paragraph{133 - Endomorphismes remarquables d’un espace vectoriel euclidien (de dimension finie).}
\paragraph{137 - Barycentres dans un espace affine réel de dimension finie ; convexité. Applications.}
\paragraph{139 - Applications des nombres complexes à la géométrie.}
\paragraph{140 - Systèmes d’équations linéaires. Systèmes échelonnés. Résolution. Exemples et applications.}
\paragraph{141 - Utilisation des groupes en géométrie.}
\paragraph{145 - Méthodes combinatoires, problèmes de dénombrement.}
\paragraph{203 - Utilisation de la notion de compacité.}
\paragraph{206 - Théorèmes de point fixe. Exemples et applications.}
\paragraph{208 - Espaces vectoriels normés, applications linéaires continues. Exemples.}
\paragraph{214 - Théorème d’inversion locale, théorème des fonctions implicites. Exemples et applications.}
\paragraph{215 - Applications différentiables définies sur un ouvert de $\eR^n$ . Exemples et applications.}
\paragraph{218 - Applications des formules de TAYLOR.}
\paragraph{219 - Problèmes d’extremums.}
\paragraph{220 - Équations différentielles $X' = f (t , X )$. Exemples d’études qualitatives des solutions.}
\paragraph{221 - Équations différentielles linéaires. Systèmes d’équations différentielles linéaires. Exemples et applications.}
\paragraph{224 - Comportement asymptotique de suites numériques. Rapidité de convergence. Exemples.}
\begin{itemize}
    \item Divergence de la somme des inverses des nombres premiers, théorème \ref{ThonfVruT}.
\end{itemize}
\paragraph{226 - Comportement d’une suite réelle ou vectorielle définie par une itération \( u_{n+1}=f(u_n)\). Exemples.}
\paragraph{229 - Fonctions monotones. Fonctions convexes. Exemples et applications.}
\paragraph{230 - Séries de nombres réels ou complexes. Comportement des restes ou des sommes partielles des séries numériques. Exemples.}
\begin{itemize}
    \item Divergence de la somme des inverses des nombres premiers, théorème \ref{ThonfVruT}.
\end{itemize}
\paragraph{232 - Méthodes d’approximation des solutions d’une équation $F (X ) = 0$. Exemples.}
\paragraph{236 - Illustrer par des exemples quelques méthodes de calcul d’intégrales de fonctions d’une ou plusieurs variables réelles.}
\paragraph{239 - Fonctions définies par une intégrale dépendant d’un paramètre. Exemples et applications.}
\paragraph{240 - Transformation de FOURIER. Applications.}
\paragraph{243 - Convergence des séries entières, propriétés de la somme. Exemples et applications.}
\paragraph{246 - Séries de FOURIER. Exemples et applications.}
\paragraph{252 - Loi binomiale. Loi de POISSON. Applications.}


