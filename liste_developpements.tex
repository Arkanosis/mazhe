% This is part of Mes notes de mathématique
% Copyright (c) 2012-2013
%   Laurent Claessens
% See the file fdl-1.3.txt for copying conditions.

Nous donnons ici quelque idées de développements associés aux leçons. Parfois, il est bon d'ajouter quelque lemmes au développement proposé, si il est trop court. Si l'un ou l'autre ne vous semble pas adapté à l'énoncé de la leçon, faites le moi savoir.

%+++++++++++++++++++++++++++++++++++++++++++++++++++++++++++++++++++++++++++++++++++++++++++++++++++++++++++++++++++++++++++
\section{Algèbre et géométrie}
%+++++++++++++++++++++++++++++++++++++++++++++++++++++++++++++++++++++++++++++++++++++++++++++++++++++++++++++++++++++++++++

%---------------------------------------------------------------------------------------------------------------------------------------------
\paragraph{101 - Groupe opérant sur un ensemble. Exemples et applications.}
\begin{itemize}
    \item Action du groupe modulaire sur le demi-plan de Poincaré, théorème \ref{ThoItqXCm}.
    \item Polynômes semi-symétriques, proposition \ref{PropUDqXax}.
    \item Lemme de Morse, lemme \ref{LemNQAmCLo}.
    \item Générateurs du groupe diédral, proposition \ref{PropLDIPoZ}.
    \item Sous-groupes compacts de \( \GL(n,\eR)\), lemme \ref{LemOCtdiaE} ou proposition \ref{PropQZkeHeG}.
\end{itemize}
%---------------------------------------------------------------------------------------------------------------------------------------------
\paragraph{103 - Exemples et applications des notions de sous-groupe distingué et de groupe quotient.}
\begin{itemize}
    \item Suites de décomposition et théorème de Jordan-Hölder \ref{ThoLgxWIC}.
    \item Groupes d'ordre \( pq\), théorème \ref{ThoLnTMBy}.
    \item Le groupe alterné est simple, théorème \ref{ThoURfSUXP}.
\end{itemize}

%---------------------------------------------------------------------------------------------------------------------------------------------
\paragraph{104 - Groupes finis. Exemples et applications.}
\begin{itemize}
    \item RSA, section \ref{subSecEVaFYi}.
    \item Théorème de Wedderburn \ref{ThoMncIWA}.
    \item Théorème de Sylow \ref{ThoUkPDXf}. Tout le théorème, c'est un peu long. On peut se contenter de la partie qui dit que \( G\) contient un \( p\)-Sylow.
    \item Coloriage de roulette (\ref{pTqJLY}) et composition de colliers (\ref{siOQlG}).
    \item Suites de décomposition et théorème de Jordan-Hölder \ref{ThoLgxWIC}.
    \item Théorème de Burnside sur les sous groupes d'exposant fini de \( \GL(n,\eC)\), théorème \ref{ThooJLTit}.
    \item \( (\eZ/p\eZ)^*\simeq \eZ/(p-1)\eZ\), corollaire \ref{CorpRUndR}.
    \item Groupes d'ordre \( pq\), théorème \ref{ThoLnTMBy}.
    \item Générateurs du groupe diédral, proposition \ref{PropLDIPoZ}.
    \item Le groupe alterné est simple, théorème \ref{ThoURfSUXP}.
\end{itemize}

%---------------------------------------------------------------------------------------------------------------------------------------------
\paragraph{105 - Groupe des permutations d’un ensemble fini. Applications.}
\begin{itemize}
    \item RSA, section \ref{subSecEVaFYi}. Parce que RSA est une permutation de \( \eF_n\).
    \item Coloriage de roulette (\ref{pTqJLY}) et composition de colliers (\ref{siOQlG}).
    \item Forme alternées de degré maximum, proposition \ref{ProprbjihK}.
    \item Décomposition de Bruhat, théorème \ref{ThoizlYJO}.
    \item Polynômes semi-symétriques, proposition \ref{PropUDqXax}.
    \item Table des caractères du groupe diédral, section \ref{SecWMzheKf}.
\end{itemize}

%---------------------------------------------------------------------------------------------------------------------------------------------
\paragraph{106 - Groupe linéaire d’un espace vectoriel de dimension finie $E$ , sous-groupes de $\GL(E)$. Applications.}
%\index{groupe!linéaire}
\begin{itemize}
    \item Théorème de Burnside sur les sous groupes d'exposant fini de \( \GL(n,\eC)\), théorème \ref{ThooJLTit}.
    \item Décomposition de Bruhat, théorème \ref{ThoizlYJO}.
    \item Le lemme au lemme de Morse, lemme \ref{LemWLCvLXe}.
    \item Décomposition polaire \ref{ThoLHebUAU}.
    \item Enveloppe convexe du groupe orthogonal \ref{ThoVBzqUpy}.
    \item Sous-groupes compacts de \( \GL(n,\eR)\), lemme \ref{LemOCtdiaE} ou proposition \ref{PropQZkeHeG}.
    \item Théorème de Von Neumann \ref{ThoOBriEoe}.
\end{itemize}
%---------------------------------------------------------------------------------------------------------------------------------------------
\paragraph{107 - Représentations et caractères d’un groupe fini sur un \( \eC\)-espace vectoriel.}
\begin{itemize}
    \item Table des caractères du groupe diédral, section \ref{SecWMzheKf}.
\end{itemize}
%---------------------------------------------------------------------------------------------------------------------------------------------
\paragraph{108 - Exemples de parties génératrices d’un groupe. Applications.}
\begin{itemize}
    \item RSA, section \ref{subSecEVaFYi}. Assez indirect : la système RSA se base sur la formule \( \varphi(pq)=(p-1)(q-1)\), laquelle se base sur l'isomorphisme \( \eZ/p\eZ\times \eZ/q\eZ\simeq \eZ/pq\eZ\) et leurs générateurs.
    \item Action du groupe modulaire sur le demi-plan de Poincaré, théorème \ref{ThoItqXCm}, parce que c'est avec lui qu'on montre les générateurs du groupe modulaire dans le corollaire \ref{CorJQwgNp}.
    \item Générateurs du groupe diédral, proposition \ref{PropLDIPoZ}
    \item Table des caractères du groupe diédral, section \ref{SecWMzheKf}.
    \item Le groupe alterné est simple, théorème \ref{ThoURfSUXP}.
\end{itemize}

%---------------------------------------------------------------------------------------------------------------------------------------------
\paragraph{109 - Anneaux $\eZ/n\eZ$. Applications.}
\begin{itemize}
    \item RSA, section \ref{subSecEVaFYi}.
    \item Forme faible du théorème de Dirichlet (avec ses deux lemmes) \ref{ThoxwTjcl}.
    \item \( (\eZ/p\eZ)^*\simeq \eZ/(p-1)\eZ\), corollaire \ref{CorpRUndR}.
    \item Groupes d'ordre \( pq\), théorème \ref{ThoLnTMBy}.
    \item Irréductibilité des polynômes cyclotomiques, proposition \ref{PropoIeOVh}.
\end{itemize}

%---------------------------------------------------------------------------------------------------------------------------------------------
\paragraph{110 - Nombres premiers. Applications.}
\begin{itemize}
    \item Structure des groupes d'ordre \( pq\), théorème \ref{ThoLnTMBy}.
    \item Divergence de la somme des inverses des nombres premiers, théorème \ref{ThonfVruT}.
    \item RSA, section \ref{subSecEVaFYi}. Ce développement est redondant avec la structure des groupes d'ordre \( pq\).
    \item Forme faible du théorème de Dirichlet (avec ses deux lemmes) \ref{ThoxwTjcl}.
    \item \( (\eZ/p\eZ)^*\simeq \eZ/(p-1)\eZ\), corollaire \ref{CorpRUndR}, peut-être redondant avec les groupes d'ordre \( pq\).
    \item Irréductibilité des polynômes cyclotomiques, proposition \ref{PropoIeOVh}.
    \item Théorème des deux carrés, théorème \ref{ThospaAEI}.
\end{itemize}

%---------------------------------------------------------------------------------------------------------------------------------------------
\paragraph{111 - Anneaux principaux. Applications}
\begin{itemize}
    \item Polynôme minimal d'endomorphisme semi-simple, théorème \ref{ThoFgsxCE}.
    \item Théorème de Bézout, corollaire \ref{CorimHyXy}.
    \item Théorème des deux carrés, théorème \ref{ThospaAEI}.
\end{itemize}
%---------------------------------------------------------------------------------------------------------------------------------------------
\paragraph{112 - Corps finis. Applications.}
\begin{itemize}
    \item Théorème de Chevalley-Warning \ref{ThoLTcYKk}.
    \item Loi de réciprocité quadratique \ref{ThoMiEiUm}.
    \item \( (\eZ/p\eZ)^*\simeq \eZ/(p-1)\eZ\), corollaire \ref{CorpRUndR}.
    \item Polynômes irréductibles sur \( \eF_q\).
\end{itemize}
%---------------------------------------------------------------------------------------------------------------------------------------------
\paragraph{113 - Groupe des nombres complexes de module 1. Sous-groupes des racines de l’unité. Applications.}
\begin{itemize}
    \item Théorème de Burnside sur les sous groupes d'exposant fini de \( \GL(n,\eC)\), théorème \ref{ThooJLTit}.
    \item Forme faible du théorème de Dirichlet (avec ses deux lemmes) \ref{ThoxwTjcl} (parce qu'on parle de polynômes cyclotomiques qui sont basés sur les racines de l'unité).
    \item Action du groupe modulaire sur le demi-plan de Poincaré, théorème \ref{ThoItqXCm}, parce qu'on y utilise un peu les propriétés des nombres dy type \( | z |=1\).
    \item Générateurs du groupe diédral, proposition \ref{PropLDIPoZ}.
    \item Irréductibilité des polynômes cyclotomiques, proposition \ref{PropoIeOVh}.
\end{itemize}
%---------------------------------------------------------------------------------------------------------------------------------------------
\paragraph{114 - Anneau de séries formelles. Applications.}
\begin{itemize}
    \item Nombres de Bell, théorème \ref{ThoYFAzwSg}.
    \item Partitions d'un entier en parts fixes, proposition \ref{PropWUFpuBR}.
\end{itemize}
%---------------------------------------------------------------------------------------------------------------------------------------------
\paragraph{115 - Corps des fractions rationnelles à une indéterminée sur un corps commutatif. Applications.}
\begin{itemize}
    \item Théorème de Rothstein-Trager \ref{ThoXJFatfu}.
    \item Partitions d'un entier en parts fixes, proposition \ref{PropWUFpuBR}.
\end{itemize}
%---------------------------------------------------------------------------------------------------------------------------------------------
\paragraph{116 - Polynômes irréductibles à une indéterminée. Corps de rupture. Exemples et applications.}
\begin{itemize}
    \item Irréductibilité des polynômes cyclotomiques, proposition \ref{PropoIeOVh}.
    \item Polynômes irréductibles sur \( \eF_q\).
\end{itemize}
%---------------------------------------------------------------------------------------------------------------------------------------------
\paragraph{117 - Algèbre des polynômes à \( n\) indéterminées (\( n\geq 2\)). Polynômes symétriques. Applications.}
\begin{itemize}
    \item À propos d'extensions de \( \eQ\), le lemme \ref{LemSoXCQH}.
    \item Polynômes semi-symétriques, proposition \ref{PropUDqXax}.
    \item Théorème de Chevalley-Warning \ref{ThoLTcYKk}.
    \item Théorème de Kronecker \ref{ThoOWMNAVp}.
\end{itemize}
%---------------------------------------------------------------------------------------------------------------------------------------------
\paragraph{118 - Exemples d’utilisation de la notion de dimension d’un espace vectoriel.}
%\index{espace!vectoriel!dimension}
\begin{itemize}
    \item Forme alternées de degré maximum, proposition \ref{ProprbjihK}, parce que c'est ce théorème qui donne l'unicité du déterminant du fait que l'espace est de dimension un.
    \item Théorème de la dimension \ref{ThonmnWKs}, bien que ce soit plutôt dans la définition de la dimension que dans l'utilisation.
    \item Théorème de Carathéodory \ref{ThoJLDjXLe}.
\end{itemize}
%---------------------------------------------------------------------------------------------------------------------------------------------
\paragraph{119 - Exemples d’actions de groupes sur les espaces de matrices.}
\begin{itemize}
    \item Action du groupe modulaire sur le demi-plan de Poincaré, théorème \ref{ThoItqXCm}.
    \item Lemme de Morse, lemme \ref{LemNQAmCLo}.
    \item Sous-groupes compacts de \( \GL(n,\eR)\), lemme \ref{LemOCtdiaE} ou proposition \ref{PropQZkeHeG}.
\end{itemize}
%---------------------------------------------------------------------------------------------------------------------------------------------
\paragraph{120 - Dimension d’un espace vectoriel (on se limitera au cas de la dimension finie). Rang. Exemples et applications.}
%\index{rang}
\begin{itemize}
    \item Forme alternées de degré maximum, proposition \ref{ProprbjihK}, parce que c'est ce théorème qui donne l'unicité du déterminant du fait que l'espace est de dimension un.
    \item Théorème de la dimension \ref{ThonmnWKs}.
        %Ici on peut mettre le théorème de Sylvester.
    \item Extrema liés, théorème \ref{ThoRGJosS}.
    \item Théorème \ref{ThoeTMXla} sur la diagonalisation de matrices symétriques.
\end{itemize}
%---------------------------------------------------------------------------------------------------------------------------------------------
\paragraph{121 - Matrices équivalentes. Matrices semblables. Applications.}
\begin{itemize}
    \item Racine carré d'une matrice hermitienne positive, proposition \ref{PropVZvCWn}.
    \item Sous-groupes compacts de \( \GL(n,\eR)\), lemme \ref{LemOCtdiaE} ou proposition \ref{PropQZkeHeG}.
        %Ici on peut mettre le théorème de Sylvester.
\end{itemize}
%---------------------------------------------------------------------------------------------------------------------------------------------
\paragraph{122 - Opérations élémentaires sur les lignes et les colonnes d’une matrice. Exemples et applications.}
%\index{matrice!lignes et colonnes}
\begin{itemize}
    \item Décomposition de Bruhat, théorème \ref{ThoizlYJO}.
\end{itemize}
%---------------------------------------------------------------------------------------------------------------------------------------------
\paragraph{123 - Déterminant. Exemples et applications.}
\begin{itemize}
    \item Forme alternées de degré maximum, proposition \ref{ProprbjihK}, parce que c'est ce théorème qui donne l'unicité du déterminant du fait que l'espace est de dimension un.
    \item Théorème de Rothstein-Trager \ref{ThoXJFatfu} parce que le résultant est est un.
\end{itemize}
%---------------------------------------------------------------------------------------------------------------------------------------------
\paragraph{124 - Polynômes d’endomorphisme en dimension finie. Réduction d’un endomorphisme en dimension finie. Applications.}
\begin{itemize}
    \item Racine carré d'une matrice hermitienne positive, proposition \ref{PropVZvCWn}.
    \item Théorème de Burnside sur les sous groupes d'exposant fini de \( \GL(n,\eC)\), théorème \ref{ThooJLTit}.
    \item Décomposition de Dunford, théorème \ref{ThoRURcpW}. 
\end{itemize}
%---------------------------------------------------------------------------------------------------------------------------------------------
\paragraph{125 - Sous-espaces stables d’un endomorphisme ou d'une famille d’un espace vectoriel de dimension finie. Applications.}
%\index{endomorphisme!sous-espace stable}
\begin{itemize}
    \item Équation de Hill \( y''+qy=0\), proposition \ref{PropGJCZcjR}.
    \item Décomposition de Dunford, théorème \ref{ThoRURcpW}. 
\end{itemize}
%---------------------------------------------------------------------------------------------------------------------------------------------
\paragraph{126 - Endomorphismes diagonalisables en dimension finie.}
\begin{itemize}
    \item Théorème de Burnside sur les sous groupes d'exposant fini de \( \GL(n,\eC)\), théorème \ref{ThooJLTit}.
    \item Racine carré d'une matrice hermitienne positive, proposition \ref{PropVZvCWn}, parce qu'un utilise le résultat de diagonalisation simultanée.
    \item Équation de Hill \( y''+qy=0\), proposition \ref{PropGJCZcjR}.
    \item Décomposition de Dunford, théorème \ref{ThoRURcpW}. 
\end{itemize}
%---------------------------------------------------------------------------------------------------------------------------------------------
\paragraph{127 - Exponentielle de matrices. Applications.}
\begin{itemize}
    \item Décomposition de Dunford, théorème \ref{ThoRURcpW}.
    \item Théorème de Von Neumann \ref{ThoOBriEoe}.
\end{itemize}
%---------------------------------------------------------------------------------------------------------------------------------------------
\paragraph{128 - Endomorphismes trigonalisables. Endomorphismes nilpotents.}
\begin{itemize}
    \item Théorème de Burnside sur les sous groupes d'exposant fini de \( \GL(n,\eC)\), théorème \ref{ThooJLTit}.
    \item Décomposition de Dunford, théorème \ref{ThoRURcpW}. 
\end{itemize}
%---------------------------------------------------------------------------------------------------------------------------------------------
\paragraph{129 - Algèbre des polynômes d’un endomorphisme en dimension finie. Applications.}
\begin{itemize}
    \item Racine carré d'une matrice hermitienne positive, proposition \ref{PropVZvCWn}.
\end{itemize}
%---------------------------------------------------------------------------------------------------------------------------------------------
\paragraph{130 - Matrices symétriques réelles, matrices hermitiennes.}
\begin{itemize}
    \item Racine carré d'une matrice hermitienne positive, proposition \ref{PropVZvCWn}.
    \item Le lemme au lemme de Morse, lemme \ref{LemWLCvLXe}.
    \item Connexité des formes quadratiques de signature donnée, proposition \ref{PropNPbnsMd}.
    \item Théorème \ref{ThoeTMXla} sur la diagonalisation de matrices symétriques.
\end{itemize}
%---------------------------------------------------------------------------------------------------------------------------------------------
\paragraph{131 - Formes quadratiques sur un espace vectoriel de dimension finie. Orthogonalité, isotropie. Applications.}
\begin{itemize}
    \item Le lemme au lemme de Morse, lemme \ref{LemWLCvLXe}, voir le lemme de Morse lui-même \ref{LemNQAmCLo}.
    \item Connexité des formes quadratiques de signature donnée, proposition \ref{PropNPbnsMd}.
    \item Sous-groupes compacts de \( \GL(n,\eR)\), lemme \ref{LemOCtdiaE} ou proposition \ref{PropQZkeHeG}.
\end{itemize}
%---------------------------------------------------------------------------------------------------------------------------------------------
\paragraph{132 - Formes linéaires et hyperplans en dimension finie. Exemples et applications.}
\begin{itemize}
    \item Extrema liés, théorème \ref{ThoRGJosS}.
\end{itemize}
%---------------------------------------------------------------------------------------------------------------------------------------------
\paragraph{133 - Endomorphismes remarquables d’un espace vectoriel euclidien (de dimension finie).}
%\index{endomorphisme!décomposition!polaire}
\begin{itemize}
    \item Décomposition polaire \ref{ThoLHebUAU}.
    \item Sous-groupes compacts de \( \GL(n,\eR)\), lemme \ref{LemOCtdiaE} ou proposition \ref{PropQZkeHeG}.
    \item Théorème \ref{ThoeTMXla} sur la diagonalisation de matrices symétriques.
\end{itemize}
%---------------------------------------------------------------------------------------------------------------------------------------------
\paragraph{135 - Isométries d’un espace affine euclidien de dimension finie. Forme réduite. Applications en dimensions $2$ et $3$.}
\begin{itemize}
    \item Points extrémaux de la boule unité dans \( \aL(E)\), théorème \ref{ThoBALmoQw}.
\end{itemize}
\paragraph{136 - Coniques. Applications.}
%---------------------------------------------------------------------------------------------------------------------------------------------
\paragraph{137 - Barycentres dans un espace affine réel de dimension finie ; convexité. Applications.}
\begin{itemize}
    \item Théorème de Carathéodory \ref{ThoJLDjXLe}.
    \item Points extrémaux de la boule unité dans \( \aL(E)\), théorème \ref{ThoBALmoQw}.
        % Il faudrait peut-être un autre développement ici.
\end{itemize}
%---------------------------------------------------------------------------------------------------------------------------------------------
\paragraph{138 - Homographies de la droite projective complexe. Applications.}
\begin{itemize}
    \item Action du groupe modulaire sur le demi-plan de Poincaré, théorème \ref{ThoItqXCm}. Parce que l'action est avec des homographies.
\end{itemize}
%---------------------------------------------------------------------------------------------------------------------------------------------
\paragraph{139 - Applications des nombres complexes à la géométrie.}
\begin{itemize}
    \item Action du groupe modulaire sur le demi-plan de Poincaré, théorème \ref{ThoItqXCm}.
    \item Générateurs du groupe diédral, proposition \ref{PropLDIPoZ}
\end{itemize}
%---------------------------------------------------------------------------------------------------------------------------------------------
\paragraph{140 - Systèmes d’équations linéaires. Systèmes échelonnés. Résolution. Exemples et applications.}
%---------------------------------------------------------------------------------------------------------------------------------------------
\paragraph{141 - Utilisation des groupes en géométrie.}
\begin{itemize}
    \item Coloriage de roulette (\ref{pTqJLY}) et composition de colliers (\ref{siOQlG}).
    \item Forme alternées de degré maximum, proposition \ref{ProprbjihK}, parce que c'est ce théorème qui donne l'unicité du déterminant du fait que l'espace est de dimension un.
    \item Action du groupe modulaire sur le demi-plan de Poincaré, théorème \ref{ThoItqXCm}.
    \item Générateurs du groupe diédral, proposition \ref{PropLDIPoZ}
\end{itemize}
%---------------------------------------------------------------------------------------------------------------------------------------------
\paragraph{144 - Problèmes d’angles et de distances en dimension $2$ ou $3$.}
%---------------------------------------------------------------------------------------------------------------------------------------------
\paragraph{145 - Méthodes combinatoires, problèmes de dénombrement.}
\begin{itemize}
    \item Coloriage de roulette (\ref{pTqJLY}) et composition de colliers (\ref{siOQlG}).
    \item Nombres de Bell, théorème \ref{ThoYFAzwSg}.
\end{itemize}
%---------------------------------------------------------------------------------------------------------------------------------------------
\paragraph{146 - Résultant. Applications.}
\begin{itemize}
    \item Théorème de Rothstein-Trager \ref{ThoXJFatfu}.
    \item Théorème de Kronecker \ref{ThoOWMNAVp}.
\end{itemize}
%---------------------------------------------------------------------------------------------------------------------------------------------
\paragraph{148 - Formes quadratiques réelles. Exemples et applications.}
\begin{itemize}
    \item Le lemme au lemme de Morse, lemme \ref{LemWLCvLXe}.
    \item Connexité des formes quadratiques de signature donnée, proposition \ref{PropNPbnsMd}.
    \item Sous-groupes compacts de \( \GL(n,\eR)\), lemme \ref{LemOCtdiaE} ou proposition \ref{PropQZkeHeG}.
% On pourra mettre le théorème de Sylvester.
\end{itemize}
%---------------------------------------------------------------------------------------------------------------------------------------------
\paragraph{149 - Représentations de groupes finis de petit cardinal.}
\begin{itemize}
    \item Table des caractères du groupe diédral, section \ref{SecWMzheKf}.
\end{itemize}
%---------------------------------------------------------------------------------------------------------------------------------------------
\paragraph{150 - Racines d’un polynômes. Fonctions symétriques élémentaires. Localisation des racines dans les cas réel et complexe.}
\begin{itemize}
    \item À propos d'extensions de \( \eQ\), le lemme \ref{LemSoXCQH}.
\end{itemize}
%---------------------------------------------------------------------------------------------------------------------------------------------
\paragraph{151 - Extensions de corps. Exemples et applications.}
\begin{itemize}
    \item Polynômes séparables, proposition \ref{PropolyeZff}.
    \item Lien entre les racines (multiples) de \( P\) et \( P'\), proposition \ref{PropolyeZff}.
    \item Théorème de l'élément primitif \ref{ThoORxgBC}.
    \item À propos d'extensions de \( \eQ\), le lemme \ref{LemSoXCQH}.
    \item Polynômes irréductibles sur \( \eF_q\).
\end{itemize}
%---------------------------------------------------------------------------------------------------------------------------------------------
\paragraph{Exemples de décompositions remarquables dans le groupe linéaire. Applications}
\begin{itemize}
    \item Décomposition polaire \ref{ThoLHebUAU}.
    \item Décomposition de Dunford, théorème \ref{ThoRURcpW}. 
\end{itemize}

%+++++++++++++++++++++++++++++++++++++++++++++++++++++++++++++++++++++++++++++++++++++++++++++++++++++++++++++++++++++++++++
\section{Analyse}
%+++++++++++++++++++++++++++++++++++++++++++++++++++++++++++++++++++++++++++++++++++++++++++++++++++++++++++++++++++++++++++

%---------------------------------------------------------------------------------------------------------------------------------------------
\paragraph{201 - Espaces de fonctions : exemples et applications.}
\begin{itemize}
    \item Théorème de Fischer-Riesz \ref{ThoGVmqOro}.
    \item Espace de Sobolev \( H^1(I)\), théorème \ref{ThoESIyxfU}.
\end{itemize}
%---------------------------------------------------------------------------------------------------------------------------------------------
\paragraph{202 - Exemples de parties denses et applications.}
\begin{itemize}
    \item Prolongement de fonction définie sur une partie dense, théorème \ref{ThoPVFQMi}
    \item Complétion d'un espace métrique, théorème \ref{ThoKHTQJXZ}.
    \item Points extrémaux de la boule unité dans \( \aL(E)\), théorème \ref{ThoBALmoQw}.
    \item Critère de Weyl, proposition \ref{PropDMvPDc}.
\end{itemize}
%---------------------------------------------------------------------------------------------------------------------------------------------
\paragraph{203 - Utilisation de la notion de compacité.}
\begin{itemize}
    \item Le théorème de Weierstrass sur la limite uniforme de fonctions holomorphes, théorème \ref{ThoArYtQO}.
    \item Suite telle que \( \lim_{k\to \infty} d(u_{k+1},u_k)=0\), théorème \ref{PropLHWACDU}.
    \item Sous-groupes compacts de \( \GL(n,\eR)\), lemme \ref{LemOCtdiaE} ou proposition \ref{PropQZkeHeG}.
\end{itemize}
%---------------------------------------------------------------------------------------------------------------------------------------------
\paragraph{204 - Connexité. Exemples et applications.}
\begin{itemize}
    \item Théorème de Runge \ref{ThoMvMCci}.
    \item Suite telle que \( \lim_{k\to \infty} d(u_{k+1},u_k)=0\), théorème \ref{PropLHWACDU}.
\end{itemize}
%---------------------------------------------------------------------------------------------------------------------------------------------
\paragraph{205 - Espaces complets. Exemples et applications.}
\begin{itemize}
    \item La proposition \ref{PropWoywYG} qui donne des indications sur la notion de classes dans \( L^p\).
    \item Prolongement de fonction définie sur une partie dense, théorème \ref{ThoPVFQMi}
    \item Complétion d'un espace métrique, théorème \ref{ThoKHTQJXZ}.
    \item Théorème de Fischer-Riesz \ref{ThoGVmqOro}.
\end{itemize}
%---------------------------------------------------------------------------------------------------------------------------------------------
\paragraph{206 - Théorèmes de point fixe. Exemples et applications.}
\begin{itemize}
    \item Processus de Galton-Watson, théorème \ref{ThoJZnAOA}.
    \item Théorème d'inversion local, théorème \ref{ThoXWpzqCn}.
\end{itemize}
%---------------------------------------------------------------------------------------------------------------------------------------------
\paragraph{207 - Prolongement de fonctions. Exemples et applications.}
\begin{itemize}
    \item Prolongement de fonction définie sur une partie dense, théorème \ref{ThoPVFQMi}
    \item Lemme de Borel \ref{LemRENlIEL}.
\end{itemize}
%---------------------------------------------------------------------------------------------------------------------------------------------
\paragraph{208 - Espaces vectoriels normés, applications linéaires continues. Exemples.}
\begin{itemize}
    \item Théorème de Fischer-Riesz \ref{ThoGVmqOro}.
\end{itemize}
%---------------------------------------------------------------------------------------------------------------------------------------------
\paragraph{213 - Espaces de Hilbert . Bases hilbertiennes. Exemples et applications.}
\begin{itemize}
    \item Espace de Sobolev \( H^1(I)\), théorème \ref{ThoESIyxfU}.
    \item Inégalité isopérimétrique, théorème \ref{ThoIXyctPo}.
\end{itemize}
%---------------------------------------------------------------------------------------------------------------------------------------------
\paragraph{214 - Théorème d’inversion locale, théorème des fonctions implicites. Exemples et applications.}
\begin{itemize}
    \item Extrema liés, théorème \ref{ThoRGJosS}.
    \item Théorème d'inversion local, théorème \ref{ThoXWpzqCn}.
    \item Lemme de Morse, lemme \ref{LemNQAmCLo}.
    \item Théorème de Von Neumann \ref{ThoOBriEoe}.
\end{itemize}
%---------------------------------------------------------------------------------------------------------------------------------------------
\paragraph{215 - Applications différentiables définies sur un ouvert de $\eR^n$ . Exemples et applications.}
\begin{itemize}
    \item Extrema liés, théorème \ref{ThoRGJosS}.
    \item Théorème d'inversion local, théorème \ref{ThoXWpzqCn}.
    \item Lemme de Morse, lemme \ref{LemNQAmCLo}.
\end{itemize}
%---------------------------------------------------------------------------------------------------------------------------------------------
\paragraph{216 - Étude métrique des courbes. Exemples.}
\begin{itemize}
    \item Inégalité isopérimétrique, théorème \ref{ThoIXyctPo}.
\end{itemize}
%---------------------------------------------------------------------------------------------------------------------------------------------
\paragraph{217 - Sous-variétés de \( \eR^n\). Exemples.}
\begin{itemize}
    \item Extrema liés, théorème \ref{ThoRGJosS}.
    \item Théorème de Von Neumann \ref{ThoOBriEoe}.
\end{itemize}
%---------------------------------------------------------------------------------------------------------------------------------------------
\paragraph{218 - Applications des formules de Taylor.}
\begin{itemize}
    \item Méthode de Newton, théorème \ref{ThoHGpGwXk}
    \item Lemme de Morse, lemme \ref{LemNQAmCLo}.
\end{itemize}
%---------------------------------------------------------------------------------------------------------------------------------------------
\paragraph{219 - Problèmes d’extrema.}
\begin{itemize}
    \item Extrema liés, théorème \ref{ThoRGJosS}.
    \item Lemme de Morse, lemme \ref{LemNQAmCLo}.
\end{itemize}
%---------------------------------------------------------------------------------------------------------------------------------------------
\paragraph{220 - Équations différentielles $X' = f (t , X )$. Exemples d’études qualitatives des solutions.}
%\index{équation!différentielle!étude qualitative}
\begin{itemize}
    \item Équation de Hill \( y''+qy=0\), proposition \ref{PropGJCZcjR}.
\end{itemize}
%---------------------------------------------------------------------------------------------------------------------------------------------
\paragraph{221 - Équations différentielles linéaires. Systèmes d’équations différentielles linéaires. Exemples et applications.}
\begin{itemize}
    \item Équation de Hill \( y''+qy=0\), proposition \ref{PropGJCZcjR}.
\end{itemize}
%---------------------------------------------------------------------------------------------------------------------------------------------
\paragraph{222 - Exemples d’équations différentielles. Solutions exactes ou approchées.}
\begin{itemize}
    \item Équation \( y''+qy=0\), \ref{subsecSyTwyM}.
\end{itemize}
%---------------------------------------------------------------------------------------------------------------------------------------------
\paragraph{223 - Convergence des suites numériques. Exemples et applications.}
\begin{itemize}
    \item Calcul d'intégrale par suite équirépartie \ref{PropDMvPDc}.
    \item Théorème taubérien de Hardy-Littlewood \ref{ThoPdDxgP}.
    \item Méthode de Newton, théorème \ref{ThoHGpGwXk}
\end{itemize}
%---------------------------------------------------------------------------------------------------------------------------------------------
\paragraph{224 - Comportement asymptotique de suites numériques. Rapidité de convergence. Exemples.}
\begin{itemize}
    \item Divergence de la somme des inverses des nombres premiers, théorème \ref{ThonfVruT}.
    \item Formule sommatoire de Poisson, proposition \ref{ProprPbkoQ}, grâce à l'exemple \ref{ExDLjesf}.
    \item Méthode de Newton, théorème \ref{ThoHGpGwXk}
    \item Estimation des grands écarts, théorème \ref{ThoYYaBXkU}.
\end{itemize}
%---------------------------------------------------------------------------------------------------------------------------------------------
\paragraph{225 - Étude locale de surfaces. Exemples.}
\begin{itemize}
    \item Lemme de Morse, lemme \ref{LemNQAmCLo}.
\end{itemize}
%---------------------------------------------------------------------------------------------------------------------------------------------
\paragraph{226 - Comportement d’une suite réelle ou vectorielle définie par une itération \( u_{n+1}=f(u_n)\). Exemples.}
%---------------------------------------------------------------------------------------------------------------------------------------------
\begin{itemize}
    \item Processus de Galton-Watson, section \ref{SecBPmrPdtGalton}.
    \item Méthode de Newton, théorème \ref{ThoHGpGwXk}
\end{itemize}
\paragraph{227 - Exemples de développements asymptotiques.}
%---------------------------------------------------------------------------------------------------------------------------------------------
\paragraph{228 - Continuité et dérivabilité des fonctions réelles d’une variable réelle. Exemples et contre-exemples.}
\begin{itemize}
    \item Les théorèmes sur les fonctions définies par des intégrales, section \ref{SecCHwnBDj}.
    \item Lemme de Borel \ref{LemRENlIEL}.
\end{itemize}
%---------------------------------------------------------------------------------------------------------------------------------------------
\paragraph{229 - Fonctions monotones. Fonctions convexes. Exemples et applications.}
\begin{itemize}
    \item La proposition \ref{PropMYskGa} donne un résultat sur \( y''+qy=0\) à partir d'une hypothèse de croissance.
    %\item Méthode de Newton, théorème \ref{ThoHGpGwXk}      Arnaud Girand la mets, mais je ne vois pas pourquoi.
\end{itemize}
%---------------------------------------------------------------------------------------------------------------------------------------------
\paragraph{230 - Séries de nombres réels ou complexes. Comportement des restes ou des sommes partielles des séries numériques. Exemples.}
%\index{série!numérique}
\begin{itemize}
    \item Divergence de la somme des inverses des nombres premiers, théorème \ref{ThonfVruT}.
    \item Formule sommatoire de Poisson, proposition \ref{ProprPbkoQ}.
    \item Théorème taubérien de Hardy-Littlewood \ref{ThoPdDxgP}.
    \item Nombres de Bell, théorème \ref{ThoYFAzwSg}.
    \item Partitions d'un entier en parts fixes, proposition \ref{PropWUFpuBR}.
\end{itemize}
%---------------------------------------------------------------------------------------------------------------------------------------------
\paragraph{231 - Illustrer par des exemples et des contre-exemples la théorie des séries numériques.}
%---------------------------------------------------------------------------------------------------------------------------------------------
\paragraph{232 - Méthodes d'approximation des solutions d’une équation $F (X ) = 0$. Exemples.}
\begin{itemize}
    \item Méthode de Newton, théorème \ref{ThoHGpGwXk}
\end{itemize}
%---------------------------------------------------------------------------------------------------------------------------------------------
\paragraph{234 - Espaces \( L^p\), \( 1\leq p\leq\infty\)}
\begin{itemize}
    \item La proposition \ref{PropWoywYG} qui donne des indications sur la notion de classes dans \( L^p\).
    \item Théorème de Fischer-Riesz \ref{ThoGVmqOro}.
    \item Espace de Sobolev \( H^1(I)\), théorème \ref{ThoESIyxfU}.
\end{itemize}
%---------------------------------------------------------------------------------------------------------------------------------------------
\paragraph{235 - Suites et séries de fonctions intégrables. Exemples et applications.}
\begin{itemize}
    \item La proposition \ref{PropWoywYG} qui donne des indications sur la notion de classes dans \( L^p\).
    \item Le théorème de Weierstrass sur la limite uniforme de fonctions holomorphes, théorème \ref{ThoArYtQO}.
    \item Les théorèmes sur les fonctions définies par des intégrales, section \ref{SecCHwnBDj}.
    \item Théorème de Fischer-Riesz \ref{ThoGVmqOro}.
\end{itemize}
%---------------------------------------------------------------------------------------------------------------------------------------------
\paragraph{236 - Illustrer par des exemples quelques méthodes de calcul d’intégrales de fonctions d’une ou plusieurs variables réelles.}
\begin{itemize}
    \item Calcul d'intégrale par suite équirépartie \ref{PropDMvPDc}.
    \item Théorème de Rothstein-Trager \ref{ThoXJFatfu}.
\end{itemize}
%---------------------------------------------------------------------------------------------------------------------------------------------
\paragraph{238 - Méthodes de calcul approché d’intégrales et de solutions d’équations différentielles.}
\begin{itemize}
    \item Calcul d'intégrale par suite équirépartie \ref{PropDMvPDc}.
\end{itemize}
%---------------------------------------------------------------------------------------------------------------------------------------------
\paragraph{239 - Fonctions définies par une intégrale dépendant d’un paramètre. Exemples et applications.}
\begin{itemize}
    \item Le théorème de Weierstrass sur la limite uniforme de fonctions holomorphes, théorème \ref{ThoArYtQO}.
    \item Les théorèmes sur les fonctions définies par des intégrales, section \ref{SecCHwnBDj}.
    \item Lemme de Morse, lemme \ref{LemNQAmCLo}.
\end{itemize}
%---------------------------------------------------------------------------------------------------------------------------------------------
\paragraph{240 - Transformation de Fourier. Applications.}
\begin{itemize}
    \item Formule sommatoire de Poisson, proposition \ref{ProprPbkoQ}.
\end{itemize}

%---------------------------------------------------------------------------------------------------------------------------------------------
\paragraph{241 - Suites et séries de fonctions. Exemples et contre-exemples.}
\begin{itemize}
    \item Formule sommatoire de Poisson, proposition \ref{ProprPbkoQ}.
    \item Théorème taubérien de Hardy-Littlewood \ref{ThoPdDxgP}.
    \item Le théorème de Weierstrass sur la limite uniforme de fonctions holomorphes, théorème \ref{ThoArYtQO}.
    \item La proposition \ref{PropWoywYG} qui donne des indications sur la notion de classes dans \( L^p\).
\end{itemize}
%---------------------------------------------------------------------------------------------------------------------------------------------
\paragraph{242 - Utilisation en probabilités du produit de convolution et de la transformation de Fourier ou de Laplace.}
\begin{itemize}
    \item Processus de Galton-Watson, lemme \ref{LemezrOiI} et théorème \ref{ThoJZnAOA}.
    \item Fonction caractéristique \ref{PropDerFnCaract}.
    \item Théorème central limite \ref{ThoOWodAi}.
\end{itemize}

%---------------------------------------------------------------------------------------------------------------------------------------------
\paragraph{243 - Convergence des séries entières, propriétés de la somme. Exemples et applications.}
\begin{itemize}
    \item Processus de Galton-Watson, théorème \ref{ThoJZnAOA}.
    \item Formule sommatoire de Poisson, proposition \ref{ProprPbkoQ}.
    \item Nombres de Bell, théorème \ref{ThoYFAzwSg}.
\end{itemize}
%---------------------------------------------------------------------------------------------------------------------------------------------
\paragraph{245 - Fonctions holomorphes et méromorphes sur un ouvert de \( \eC\). Exemples et applications.}
\begin{itemize}
    \item Le théorème de Weierstrass sur la limite uniforme de fonctions holomorphes, théorème \ref{ThoArYtQO}.
\end{itemize}
%---------------------------------------------------------------------------------------------------------------------------------------------
\paragraph{246 - Séries de Fourier. Exemples et applications.}
\begin{itemize}
    \item Formule sommatoire de Poisson, proposition \ref{ProprPbkoQ}.
    \item Inégalité isopérimétrique, théorème \ref{ThoIXyctPo}.
\end{itemize}
%---------------------------------------------------------------------------------------------------------------------------------------------
\paragraph{247 - Exemples de problèmes d’interversion de limites.}
%\index{limite!inversion}
\begin{itemize}
    \item Théorème taubérien de Hardy-Littlewood \ref{ThoPdDxgP} parce que l'énoncé revient à dire que \( \lim_{x\to 1^-} \sum_{n\in \eN}a_nx^n=\sum_{n\in \eN}a_n\).
    \item Le théorème de Weierstrass sur la limite uniforme de fonctions holomorphes, théorème \ref{ThoArYtQO}.
    \item La proposition \ref{PropWoywYG} qui donne des indications sur la notion de classes dans \( L^p\). Ça utilise la convergence monotone pour  pour permuter une somme et une intégrale.
    \item Les théorèmes sur les fonctions définies par des intégrales, section \ref{SecCHwnBDj}.
    \item Nombres de Bell, théorème \ref{ThoYFAzwSg}.
\end{itemize}
%---------------------------------------------------------------------------------------------------------------------------------------------
\paragraph{248 - Approximation des fonctions numériques par des fonctions polynomiales. Exemples.}
\begin{itemize}
    \item Théorème taubérien de Hardy-Littlewood \ref{ThoPdDxgP}.
    \item Théorème de Runge \ref{ThoMvMCci}.
\end{itemize}
%---------------------------------------------------------------------------------------------------------------------------------------------
\paragraph{249 - Suites de variables de Bernoulli indépendantes.}
\begin{itemize}
    \item Processus de Galton-Watson, section \ref{SecBPmrPdtGalton}.
    \item Estimation des grands écarts, théorème \ref{ThoYYaBXkU}.
\end{itemize}
%---------------------------------------------------------------------------------------------------------------------------------------------
\paragraph{250 - Loi des grands nombres. Théorème de la limite centrale. Applications.}
\begin{itemize}
    \item Presque tous les nombres sont normaux, proposition \ref{PropEEOXLae}.
    \item Estimation des grands écarts, théorème \ref{ThoYYaBXkU}.
\end{itemize}
%---------------------------------------------------------------------------------------------------------------------------------------------
\paragraph{251 - Indépendance d’événements et de variables aléatoires. Exemples.}
\begin{itemize}
    \item Presque tous les nombres sont normaux, proposition \ref{PropEEOXLae}.
    \item Estimation des grands écarts, théorème \ref{ThoYYaBXkU}.
\end{itemize}
%---------------------------------------------------------------------------------------------------------------------------------------------
\paragraph{252 - Loi binomiale. Loi de Poisson. Applications.}
\begin{itemize}
    \item Estimation des grands écarts, théorème \ref{ThoYYaBXkU}.
\end{itemize}
%---------------------------------------------------------------------------------------------------------------------------------------------
\paragraph{253 - Utilisation de la notion de convexité en analyse.}
    %\item Méthode de Newton, théorème \ref{ThoHGpGwXk}     Je ne vois pas trop pourquoi \ldots
%---------------------------------------------------------------------------------------------------------------------------------------------
\paragraph{254 - Espaces de Schwartz et distributions tempérées.}
\begin{itemize}
    \item Formule sommatoire de Poisson, proposition \ref{ProprPbkoQ}.
\end{itemize}
%---------------------------------------------------------------------------------------------------------------------------------------------
\paragraph{255 - Dérivation au sens des distributions. Exemples et applications.}
\begin{itemize}
    \item Espace de Sobolev \( H^1(I)\), théorème \ref{ThoESIyxfU}.
\end{itemize}
%---------------------------------------------------------------------------------------------------------------------------------------------
\paragraph{256 - Transformation de Fourier dans \( \swS(\eR^d)\) et \( \swS'(\eR^d)\).}
\begin{itemize}
    \item Formule sommatoire de Poisson, proposition \ref{ProprPbkoQ}.
\end{itemize}
