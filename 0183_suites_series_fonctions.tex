% This is part of Mes notes de mathématique
% Copyright (c) 2011-2014
%   Laurent Claessens
% See the file fdl-1.3.txt for copying conditions.

%+++++++++++++++++++++++++++++++++++++++++++++++++++++++++++++++++++++++++++++++++++++++++++++++++++++++++++++++++++++++++++
\section{Intégrales de surface}
%+++++++++++++++++++++++++++++++++++++++++++++++++++++++++++++++++++++++++++++++++++++++++++++++++++++++++++++++++++++++++++

%---------------------------------------------------------------------------------------------------------------------------
\subsection{Intégrale d'une fonction}
%---------------------------------------------------------------------------------------------------------------------------
\label{secintsurfaciques}
Soit $M$ une variété de dimension $n$ dans $\eR^m$. Soit $F : W \subset \eR^n \to M$ une paramétrisation d'un ouvert relatif de $M$.  

Si $f$ est une fonction définie sur un sous-ensemble $A \subset F(W)$ tel que $F^{-1}(A)$ est mesurable, l'\Defn{intégrale de $f$ sur $A$} est définie par
\begin{equation*}
  \int_A f = \int_{F^{-1}(A)} f(F(w)) \sqrt{\det(\transpose{J_F(w)} {J_F(w)})} dw
\end{equation*}
où l'intégrale est l'intégration usuelle (de Lebesgue) sur $F^{-1}(A) \subset \eR^n$. On écrit parfois cette intégrale $\int_{F^{-1}(A)} f(F(w)) d\sigma$ où
\begin{equation*}
  d\sigma = \sqrt{\det(\transpose{J_F(w)} {J_F(w)})} dw
\end{equation*}
est l'\Defn{élément infinitésimal de volume} de la variété. 

Si $m = 3$ et $n = 2$, l'élément infinitésimal de volume vaut
\begin{equation*}
  d \sigma = \norme{\pder F {w_1} \wedge \pder F {w_2}} dw
\end{equation*}
où $\wedge$ représente le produit vectoriel dans $\eR^3$, et $(w_1,w_2)$ sont les coordonnées sur $W \subset \eR^2$. Dans la suite, nous ne regarderons plus que ce cas.

\subsection{Intégrale d'un champ de vecteurs}
Dans l'intégration curviligne, on a noté que si l'intégrale d'une fonction ne dépendait pas de l'orientation du chemin, l'intégrale d'un champ de vecteurs ou d'une forme différentielle en dépendait. Ce problème d'orientation apparait également dans l'intégration sur des surfaces de l'espace.

%% Page 530, exemple 4
Une \Defn{orientation} sur une surface $S \subset \eR^3$ est le choix
d'un champ de vecteurs continu $\nu : S \to \eR^3$ dont la norme en
tout point de $S$ vaut $1$. On remarque qu'ayant fait un tel choix
d'orientation $\nu(x)$ en un point $x$, le seul autre choix possible
en $x$ est $-\nu(x)$.
%% Page 
Si $S$ est le bord d'un ouvert $D \subset \eR^3$, l'\Defn{orientation
  induite par $D$ sur $S$} est, si elle existe, l'orientation qui
pointe hors de $D$ en tout point de $S$. Plus précisément, il faut que
pour tout $x \in D$ il existe $\epsilon > 0$ vérifiant, pour tout $0 <
t < \epsilon$, la relation $t \nu(x) \notin D$. Dans ce cas, le champ
de vecteurs $\nu$ est appelé le \Defn{vecteur normal unitaire
  extérieur} à $D$ et il est forcément unique.

Soit $G$ un champ de vecteurs défini sur une surface orientée par un
champ $\nu$. L'intégrale de $G$ sur $S$, aussi appelée le \Defn{flux
de $G$ à travers $S$}, est
\begin{equation}\label{eqflux-star}
  \iint_S G \cdot d S \pardef \iint_S \scalprod{G}{\nu} d \sigma.
\end{equation}
Si on suppose que la surface est paramétrisée par une application
\begin{equation*}
  F : W \subset \eR^2 \to \eR^3 : (u,v) \mapsto (F_1(u,v),F_2(u,v),F_3(u,v))
\end{equation*}
alors un vecteur unitaire $\nu$ peut s'écrire sous la forme
\begin{equation*}
  \nu = \frac{\pder F u \wedge \pder F v}{\norme{\pder F u \wedge \pder F v}}
\end{equation*}
et grâce à cette paramétrisation l'intégrale \eqref{eqflux-star}
devient
\begin{equation*}
  \iint_S G \cdot d S = \iint_W \scalprod{G(F(u,v))}{\pder F u \wedge \pder F v} d u
  d v.
\end{equation*}
où on utilise l'expression de $d \sigma$ obtenue précédemment dans le
cas qui nous intéresse (surface dans l'espace).


%+++++++++++++++++++++++++++++++++++++++++++++++++++++++++++++++++++++++++++++++++++++++++++++++++++++++++++++++++++++++++++
\section{Divergence, Green, Stokes}
%+++++++++++++++++++++++++++++++++++++++++++++++++++++++++++++++++++++++++++++++++++++++++++++++++++++++++++++++++++++++++++

Le théorème de Stokes (et ses variations) peut se voir comme une généralisation du théorème fondamental du calcul différentiel et intégral qui stipule que
\begin{equation*}
	\int_a^b f^\prime(x) d x = f(b) - f(a)
\end{equation*}
c'est-à-dire qui relie l'intégrale de $f^\prime$ sur $I = [a,b]$ aux valeurs de $f$ sur le bord $\partial I = \{a,b\}$. Le signe $-$ qui apparait vient de l'orientation ; celle-ci requiert de la prudence dans l'utilisation des théorèmes.

Voici, pour votre culture générale, un énoncé général :
\begin{theorem} \label{ThoATsPuzF}
	Si $M$ est une variété orientable de dimension $n$ avec un bord noté $\partial  M$, alors pour toute forme différentielle $\omega$ de degré $n-1$ on a 
	\begin{equation*}
		\int_{ M} d \omega = \int_{\partial  M} \omega.
	\end{equation*}
	où $d \omega$ désigne la différentielle extérieure de $\omega$.
\end{theorem}
Attention : la différentielle extérieure n'est pas la différentielle usuelle. Certes dans le cas d'une \( 0\)-forme (c'est à dire d'une fonction), les deux notions coïncident, mais ça ne va pas plus loin. La différentielle extérieure vérifie \( d^2\omega=0\) pour tout \( \omega\), y compris pour les fonctions : si \( \omega=df\) alors \( d\omega=0\).

%TODO : donner la définition et quelque exemples de différentielle extérieure.

Nous allons maintenant voir quelque cas particuliers. 

%--------------------------------------------------------------------------------------------------------------------------- 
\subsection{Intégrales curviligne}
%---------------------------------------------------------------------------------------------------------------------------

Une des nombreuses formes du théorème de Stokes (théorème \ref{ThoATsPuzF}) est que si la forme différentielle \( \omega\) est exacte alors son intégrale est facile.
\begin{theorem} \label{ThoUJMhFwU}
    Si \( \gamma\) est une chemin de classe \( C^1\) dans un ouvert \( \Omega\) et si \( \omega\) est la forme différentielle exacte \( \omega=df\), alors
    \begin{equation}
        \int_{\gamma}df=f\big( \gamma(1) \big)-f\big( \gamma(0) \big).
    \end{equation}
\end{theorem}
Cela est également une extension du théorème fondamental du calcul différentiel.
%TODO : Ce serait vachement mieux de donner une preuve indépendante de cela.


\subsection{Théorème de la divergence}

Si nous considérons une surface dans $\eR^n$ et un champ de vecteurs, il est bon de se demander quelle \og quantité de vecteurs\fg{} traverse la surface. Soit $D$, un ouvert borné de $\eR^n$ telle que $\partial D$ soit une variété de dimension $n-1$, et $G$, un champ de vecteurs défini sur $\bar D$. Afin de compter combien de $G$ traverse $\partial D$, il faudra faire en sorte de ne considérer que la composante de $G$ normale à $\partial D$ : pas question d'intégrer par exemple la norme de $G$ sur $\partial D$.

Comme nous le savons, la composante du vecteur $v$ dans la direction $w$ est le produit scalaire $v\cdot 1_w$ où $1_w$ est le vecteur de norme $1$ dans la direction $w$. Nous allons donc introduire le concept de vecteur normal extérieur. Soit $x\in\partial D$ et $\nu\in\eR^n$, nous disons que $\nu$ est un \defe{vecteur normal extérieur}{normal extérieur!vecteur} de $\partial D$ si
\begin{enumerate}

	\item
		$\langle \nu, v\rangle =0$ pour tout vecteur tangent $v$ à $\partial D$ au point $x$. Pour rappel, $\partial D$ étant une variété de dimension $n-1$, il y a $n-1$ tels vecteurs $v$ linéairement indépendants.
	
	\item
		Il existe un $\delta>0$ tel que $\forall t\in\mathopen] 0 , \delta \mathclose[$, nous avons $c+t\nu\notin \bar D$ et $x-t\nu\in D$.
 
\end{enumerate}

Nous pouvons maintenant définir le concept de flux. Soit $D\subset \eR^n$ tel que $\partial D$ soit une variété de dimension $n-1$ qui admette un vecteur normal extérieur $\nu(x)$ en chaque point. Soit aussi $G\colon \bar D\to \eR^n$, un champ de vecteur de classe $C^1$. Le \defe{flux}{flux!d'un champ de vecteur} de $G$ au travers de $\partial D$ est le nombre
\begin{equation}
	\int_{\partial D}\langle G(x), \nu(x)\rangle d\sigma(x).
\end{equation}

Cette intégrale est en général très compliquée à calculer parce qu'il faut trouver le champ de vecteur normal, puis une paramétrisation de la surface $\partial D$ et ensuite appliquer la méthode décrite au point \ref{secintsurfaciques}. 

Heureusement, il y a un théorème qui nous permet de calculer plus facilement : sans devoir trouver de vecteurs normaux.

Il n'est pas plus contraignant d'énoncer ce théorème dans le cadre d'une hypersurface de $\eR^n$, ce que nous faisons donc~:
\begin{theorem}[Formule de la divergence]
	Soit $D$ un ouvert borné de $\eR^n$ dont le bord est \og assez régulier par morceaux\fg{}, c'est-à-dire~:
	\begin{equation}
		\partial D = A_1 \cup \ldots A_p \cup N
	\end{equation} 
	où
	\begin{enumerate}
		\item $A_1, \ldots, A_p, N$ sont deux à deux disjoints,
		\item pour tout $i \leq p$, $A_i$ est un ouvert relatif d'une certaine variété $M_i$ de dimension $(n-1)$
		\item $\bar A_i \subset M_i$
		\item $N$ est un compact contenu dans une réunion finie de variétés de dimensions $(n-2)$.
	\end{enumerate}
	Supposons également qu'en chaque point de $A_1 \cup \ldots \cup A_p$ il existe un vecteur normal extérieur $\nu$.
	
	Si $G$ est un champ de vecteurs de classe $C^1$ sur $\bar D$ alors
	\begin{equation}
		\int_D \nabla\cdot G = \sum_{i=1}^p \int_{A_i} \scalprod{G}{\nu}.
	\end{equation}
	L'intégrale du membre de gauche est l'intégrale sur un ouvert d'une simple fonction.
\end{theorem}

\subsection{Formule de Green}

Pour rappel, une chemin $\gamma\colon \mathopen[ 0 , 1 \mathclose]\to \eR^n$ est \defe{régulier}{régulier!chemin} si il est $C^1$ et si $\gamma(t)\neq 0$ pour tout $r$. Le chemin est de \defe{Jordan}{Jordan!chemin} si $\gamma(1)=\gamma(0)$ et si $\gamma\colon \mathopen[ a , b [\to \eR^n$ est injective.

La formule de Green est un cas particulier du théorème de la divergence dans
le cas $n = 2$, légèrement reformulé~:
\begin{theorem}
	Soit $D \subset \eR^2$ ouvert borné tel que son bord est est la réunion finie d'un certain nombre de chemins de classe $C^1$ de Jordan réguliers.  Supposons qu'en chaque point de son bord, $D$ possède un vecteur normal unitaire extérieur $\nu$. Soient $P$ et $Q$ deux fonctions réelles de classe $C^1$ sur $\bar D$. Alors
    \begin{equation}  \label{EqYLblSqV}
    \iint_D (\partial_xQ - \partial_yP)dx\,dy = \oint_{\partial D}
    Pd x + Q d y
  \end{equation}
  où chaque chemin $\gamma$ formant le bord de $D$ est orienté de
  sorte que $T \nu = \frac{\dot\gamma}{\norme{\dot\gamma}}$ où $T$
  représente la rotation d'angle $+\frac\pi2$.
\end{theorem}

Justifions le fait que cela soit un cas particulier de la formule de Stokes du théorème \ref{ThoATsPuzF}. Nous considérons la forme différentielle
\begin{equation}
    \omega=Pdx+Qdy,
\end{equation}
et sa différentielle
\begin{subequations}
    \begin{align}
    d\omega&=\sum_id\omega_i\wedge dx_i\\
    &=\left( \frac{ \partial P }{ \partial x }dx+\frac{ \partial P }{ \partial y }dy \right)\wedge dx+\left( \frac{ \partial Q }{ \partial x }dx+\frac{ \partial Q }{ \partial y }dy \right)\wedge dy\\
    &=\left( \frac{ \partial Q }{ \partial x }-\frac{ \partial P }{ \partial y } \right)dx\wedge dy.
    \end{align}
\end{subequations}

Intégrons cette forme \( d\omega\) sur le domaine ouvert \( D\) que nous paramétrons de façon triviale par
\begin{equation}
    \begin{aligned}
        \varphi\colon D&\to \eR^2 \\
        (u,v)&\mapsto (u,v).
    \end{aligned}
\end{equation}
Ce que nous avons est
\begin{equation}\label{EqKYjFEGF}
    \iint_D d\omega=\iint_D d\omega_{(u,v)}\left( \frac{ \partial \varphi }{ \partial u },\frac{ \partial \varphi }{ \partial v } \right)dudv
\end{equation}
Nous avons aussi \( T_u=\frac{ \partial \varphi }{ \partial u }=\begin{pmatrix}
    1    \\ 
    0    
\end{pmatrix}\) et \(T_v= \frac{ \partial \varphi }{ \partial v }=\begin{pmatrix}
    0    \\ 
      1  
\end{pmatrix}\) et donc 
\begin{equation}
    (dx\wedge dy)(T_u,T_v)=dx(T_u)dy(T_v)-dx(T_v)dy(T_u)=1-0=1.
\end{equation}
L'intégrale \eqref{EqKYjFEGF} se développe donc en
\begin{equation}
    \iint_Dd\omega=\iint_D\left( \frac{ \partial Q }{ \partial x }(u,v)-\frac{ \partial P }{ \partial y }(u,v) \right)(dx\wedge dy)(T_u,T_v)dudv=\iint_D\left( \frac{ \partial Q }{ \partial x }-\frac{ \partial P }{ \partial y } \right)dudv.
\end{equation}
Par conséquent la formule de Stokes nous donne la formule \eqref{EqYLblSqV}.

La formule de Green nous permet de calculer l'aire de la surface délimitée par une courbe fermée en termes de l'intégrale d'une forme bien choisie le long du contour. Pour cela nous prenons la forme
\begin{equation}    \label{EqZNXYMQb}
    \omega=-\frac{ y }{2}dx+\frac{ x }{2}dy,
\end{equation}
de telle sorte que \( \partial_xQ-\partial_yQ=1\) et que
\begin{equation}
    \iint_Dd\omega=\iint_Dddudv=S,
\end{equation}
et au final l'aire est donnée par
\begin{equation}
    S=\int_{\partial D}\left( -\frac{ y }{2}dx+\frac{ x }{2}dy \right).
\end{equation}

Lorsque le bord de \( D\) est paramétré par
\begin{equation}
    \begin{aligned}
        \gamma\colon \mathopen[ a , b \mathclose]&\to \eR^2 \\
        u&\mapsto \begin{pmatrix}
            x(u)    \\ 
            y(u)    
        \end{pmatrix},
    \end{aligned}
\end{equation}
nous avons
\begin{equation}
    (Pdx+Qdy)\gamma'(u)=Px'+Qy',
\end{equation}
et alors
\begin{equation}
    \int_{\partial D}Pdx+Qdy=\int_a^b P\big( x(u),y(u) \big)x'(u)+Q\big( x(u),y(u) \big)y'(u)du.
\end{equation}
En ce qui concerne l'aire de la surface, nous prenons les \( P\) et \( Q\) de la forme \ref{EqZNXYMQb} :
\begin{equation}    \label{EqAJGrtOk}
    S=\frac{ 1 }{2}\int_a^b\Big( -y(u)x'(u)+x(u)y'(u) \Big)du.
\end{equation}

\subsection{Formule de Stokes}
\label{secstokesusuel}

La formule de Stokes est la version classique, qui permet d'exprimer la circulation d'un champ de vecteur le long d'une courbe de $\eR^3$ comme le flux de son rotationnel à travers n'importe quel surface dont le bord est la courbe. La version présentée ici suppose que la surface peut se paramétrer en un seul morceau~:
\begin{theorem}
  Soit $F : W\subset \eR^2 \to \eR^3$ une paramétrisation (carte) d'une surface dans $\eR^3$, supposée de classe $C^2$. Soit $D$ un ouvert de $\eR^2$ vérifiant les hypothèses de la formule de Green, et tel que $\bar D \subset W$. Soit $G$ un champ de vecteurs de classe $C^1$ défini sur $F(\bar D)$, et soit $N$ le champ normal unitaire donné par la paramétrisation
  \begin{equation}		
	N = \frac{\pder F u \wedge \pder F v}{\norme{\pder F u \wedge \pder F v}}
\end{equation}
  alors
  \begin{equation}\label{EqStokesTho}
    \iint_{F(D)} \scalprod{\rot G}{N} d\sigma_F = \int_{F(\partial D)} G
  \end{equation}
  où les chemins formant le bord $\partial D$ sont orientés comme dans le théorème de Green.
\end{theorem}
Notons, juste pour avoir une bonne nouvelle de temps en temps, que 
\begin{equation}
	d\sigma_F=\left\| \frac{ \partial F }{ \partial u }\times\frac{ \partial F }{ \partial v }  \right\|dudv,
\end{equation}
mais cette norme apparaît exactement au dénominateur de $N$. Il ne faut donc pas la calculer parce qu'elle se simplifie.

Sous forme un peu plus physicienne\footnote{et surtout plus explicite.}, la formule \eqref{EqStokesTho} s'écrit
\begin{equation}
	\int_{F(D)}\langle \nabla\times G, N(x)\rangle\, d\sigma_F(x)=\int_{F(\gamma)}\langle G, T\rangle\, ds
\end{equation}
où $T$ est le vecteur unitaire tangent à $F(\gamma)$.

%///////////////////////////////////////////////////////////////////////////////////////////////////////////////////////////
\subsubsection{Quelle est la bonne orientation ?}
%///////////////////////////////////////////////////////////////////////////////////////////////////////////////////////////

Le signe du vecteur normal $N$ dépend du choix de l'ordre des coordonnées dans la carte. Supposons que je veuille paramétrer la surface $x^2+y^2=1$, $z=1$. Nous prenons naturellement comme carte le cercle $C$ de rayon $1$ dans $\eR^2$ et la carte
\begin{equation}
	F(r,\theta)=\begin{pmatrix}
		r\cos\theta	\\ 
		r\sin\theta	\\ 
		1	
	\end{pmatrix}.
\end{equation}
Mais nous aurions aussi pu mettre les coordonnées $r$ et $\theta$ dans l'autre ordre :
\begin{equation}
	\tilde F(\theta,r)=\begin{pmatrix}
		r\cos\theta	\\ 
		r\sin\theta	\\ 
		1	
	\end{pmatrix}.
\end{equation}
Les vecteurs normaux ne sont pas les même : la carte $F$ donnera $\partial_rF\times\partial_{\theta}F$, tandis que l'autre donnera $\partial_{\theta}\tilde F\times\partial_r\tilde F$. Le signe change !

Il faut savoir laquelle choisir. Le cercle $C\subset \eR^2$ a une orientation donnée par le théorème de Green. Nous choisissons l'ordre des coordonnées pour que $1_{\theta}$ et $1_{r}$ soient dans la même orientation que les vecteurs $\nu$ et $T$ tels que donnés par le théorème de Green, et tels que dessinés sur la figure \ref{LabelFigCercleTnu}.
\newcommand{\CaptionFigCercleTnu}{L'orientation sur le cercle. Si nous les prenons dans l'ordre, les vecteurs $(1_r,1_{\theta})$ ont la même orientation que celle donnée par les vecteurs $(\nu,T)$ donnés par la convention de Green.}
\input{Fig_CercleTnu.pstricks}

%\ref{LabelFigCercleTnu}.
%\newcommand{\CaptionFigCercleTnu}{L'orientation sur le cercle. Si nous les prenons dans l'ordre, les vecteurs $(1_r,1_{\theta})$ ont la même orientation que celle donnée par les vecteurs $(\nu,T)$ donnés par la convention de Green.}
%\input{Fig_CercleTnu.pstricks}

Plus généralement, nous choisissons l'ordre des coordonnées $u$ et $v$ pour que la base $(1_u,1_v)$ ait la même orientation que $(\nu,T)$ où $T$ a le sens convenu dans le théorème de Green.

%+++++++++++++++++++++++++++++++++++++++++++++++++++++++++++++++++++++++++++++++++++++++++++++++++++++++++++++++++++++++++++
\section{Suites de fonctions}
%+++++++++++++++++++++++++++++++++++++++++++++++++++++++++++++++++++++++++++++++++++++++++++++++++++++++++++++++++++++++++++

\begin{definition}[\cite{TrenchRealAnalisys}]
    Nous disons qu'une suite de fonctions \( (f_n)\) définies sur un ensemble \( A\) \defe{converge uniformément}{convergence!uniforme} vers une fonction \( f\) si
    \begin{equation}
        \lim_{n\to \infty} \| f_n-f \|_A=0
    \end{equation}
    où \( \| g \|_A=\sup_{x\in A}\| g(x) \|\).
\end{definition}

%--------------------------------------------------------------------------------------------------------------------------- 
\subsection{Convergence uniforme}
%---------------------------------------------------------------------------------------------------------------------------

\begin{proposition}[Critère de Cauchy uniforme\cite{LCbyNWQ}]   \label{PropNTEynwq}
    Soit \( X\) un espace topologique et \( (Y,d)\) un espace topologique complet. La suite de fonction \( f_n\colon X\to Y\) converge uniformément sur \( A\) si et seulement si pour tout \( \epsilon>0\) il existe \( N\in \eN\) tel que si \( k,l>N\) alors
    \begin{equation}
        d\big( f_k(x),f_l(x) \big)\leq \epsilon
    \end{equation}
    pour tout \( x\in X\).
\end{proposition}
\index{Cauchy!critère!uniforme}
\index{critère!Cauchy!uniforme}
Grosso modo, cela dit que si qu'une suite de Cauchy pour la norme uniforme est une suite uniformément convergente. Le fait que la suite converge fait partie du résultat et n'est pas une hypothèse. Ce critère sera utilisé pour montrer que \( \big( C(K),\| . \|_{\infty} \big)\) est complet, proposition \ref{PropSYMEZGU}. 

\begin{proof}
    Si \( f_n\stackrel{unif}{\longrightarrow}f\) alors le critère est satisfait; c'est dans l'autre sens que la preuve est intéressante.

    Soit donc une suite de fonctions satisfaisant au critère et montrons qu'elle converge uniformément. Pour tout \( x\in X\) la suite \( n\mapsto f_n(x)\) est de Cauchy dans l'espace complet \( Y\); nous avons donc convergence ponctuelle \( f_n\to f\). Nous devons prouver que cette convergence est uniforme. Soit \( \epsilon>0\) et \( N\in \eN\) tel que si \( k,l>N\) alors
    \begin{equation}
        d\big( f_k(x),f_l(x) \big)\leq \epsilon
    \end{equation}
    pour tout \( x\in X\). Si nous nous fixons un tel \( k\) et un \( x\in A\) nous considérons l'inégalité
    \begin{equation}
        d\big( f_k(x),f_l(x) \big)\leq \epsilon
    \end{equation}
    qui est vraie pour tout \( l\). En passant à la limite \( l\to\infty\) (limite qui commute avec la fonction distance par définition de la topologie) nous avons
    \begin{equation}
        d\big( f_k(x),f(x) \big)\leq \epsilon.
    \end{equation}
    Cette inégalité étant valable pour tout \( x\in X\), ce la signifie que \( f_n\stackrel{unif}{\longrightarrow}f\).
\end{proof}

\begin{theorem}[Limite uniforme de fonctions continues]			\label{ThoUnigCvCont}
    Soit \( A\), un ensemble mesuré et \( f_n\colon A\to \eR^n\), une suite de fonctions continues convergeant uniformément vers \( f\). Si les fonctions \( f_n\) sont toutes continues en \( x_0\in A\), alors \( f\) est continue en \( x_0\).
\end{theorem}

\begin{proof}
    Soit \( \epsilon>0\). Si \( x\in A\) nous avons, pour tout \( n\), la majoration
    \begin{subequations}
        \begin{align}
            \| f(x)-f(x_0) \|&\leq \| f(x)-f_n(x) \|+\| f_n(x)-f_n(x_0) \|+\| f_n(x_0)-f(x_0) \|\\
            &\leq\| f_n(x)-f_n(x_0) \|+2\| f_n-f \|_{\infty}.
        \end{align}
    \end{subequations}
    Grâce à l'uniforme convergence, nous considérons \(N\in \eN\) tel que \( \| f_n-f \|\leq \epsilon\) pour tout \( n\geq N\). Pour de tels \( n\), nous avons
    \begin{equation}
        \| f(x)-f(x_0) \|\leq 2\epsilon\| f_n-f \|+\| f_n(x)-f_n(x_0) \|.
    \end{equation}
    La continuité de \( f_n\) nous fournit un \( \delta>0\) tel que \( \| f_n(x_0)-f_n(x) \|<\epsilon\) dès que \( \| x-x_0 \|<\delta\). Pour ce \( \delta\), nous avons alors \( \| f(x)-f(x_0) \|<\epsilon\).
\end{proof}

\begin{theorem}[Théorème de Dini\cite{JIFGuct}] \label{ThoUFPLEZh}
    Soit \( D\) un espace métrique compact et une suite de fonctions \( f_n\in C(D,\eR)\) telle que
    \begin{enumerate}
        \item
            \( f_n\to g\) ponctuellement,
        \item
            \( g\in C(D,\eR)\),
        \item
            la suite \( (f_n)\) est croissante, c'est à dire que pour tout \( x\in D\) et pour tout \( n\geq 0\) nous avons \( f_{n+1}(x)\geq f_n(x)\).
    \end{enumerate}
    Alors la convergence est uniforme.
\end{theorem}
\index{convergence!uniforme!théorème de Dini}
\index{compacité!théorème de Dini}
\index{théorème!Dini}

\begin{proof}
    Soit \( x\in D\) et \( \epsilon>0\). Il existe \( N(x)\in \eN\) tel que
    \begin{equation}
        g(x)-\epsilon\leq f_{N(x)}\leq g(x).
    \end{equation}
    De plus \( g\) et \( f_{N(x)}\) sont des fonctions continues, donc il existe \( \eta(x)\) tel que si \( y\in B\big( x,\eta(x) \big)\) alors
    \begin{subequations}
        \begin{align}
            g(y)&\in B\big( g(x),\epsilon \big) \label{subEqXKjgKgv}\\
            f_{N(x)}(y)&\in B\big( f_{N(x)}(x),\epsilon \big)   \label{subEqHTiYZLd}.
        \end{align}
    \end{subequations}
    Si \( n\geq N(x)\) et si \( y\in B(x,\eta(x))\) alors nous avons les majorations
    \begin{equation}
            g(y)\geq f_n(y)
            \geq f_{N(x)}(y)
            \geq f_{N(x)}(x)-\epsilon
            \geq g(x)-2\epsilon
            \geq g(y)-3\epsilon.
    \end{equation}
    Justifications :
    \begin{multicols}{2}
        \begin{enumerate}
            \item
                Les deux première inégalités sont la croissance de la suite.
            \item
                La suivante est \eqref{subEqHTiYZLd}.
            \item
                Ensuite il y a le choix de \( N(x)\).
            \item
                Et enfin il y a \eqref{subEqXKjgKgv}.
        \end{enumerate}
    \end{multicols}
    Nous retenons que si \( x\in D\) et si \( n\geq N(x)\) alors
    \begin{equation}    \label{EqJCMktdj}
        g(y)\geq f_n(y)\geq g(y)-3\epsilon
    \end{equation}
    pour tout \( y\in B(x,\eta(x))\).

    Nous utilisons maintenant la compacité de \( D\). Pour chaque \( x\in D\) nous pouvons considérer la boule ouverte \( B\big( x,\eta(x) \big)\); ces boules recouvrent \( D\). Nous en extrayons un sous-recouvrement fini, c'est à dire un ensemble fini d'éléments \( x_1\),\ldots, \( x_K\) tels que
    \begin{equation}
        D=\bigcup_{k=1}^K B\big(x_k,\eta(x_k)\big).
    \end{equation}
    Si à ce moment vous ne comprenez pas pourquoi c'est une égalité au lieu d'une inclusion, il faut lire l'exemple \ref{ExKYZwYxn}. Considérons 
    \begin{equation}
        n\geq N=\max\{ N(x_1),\ldots, N(x_K) \}.
    \end{equation}
    Pour tout \( y\in D\) il existe \( k\in\{ 1,\ldots, K \}\) tel que \( y\in B\big( x_k,\eta(x_k) \big)\), et vu que \( n\geq N(x_k)\) nous reprenons la majoration \eqref{EqJCMktdj} :
    \begin{equation}
        g(y)\geq f_n(y)\geq g(y)-3\epsilon.
    \end{equation}
    Pour le \( n\) choisi nous avons ces inégalités pour tout \( y\in D\), c'est à dire que nous avons \( \| f_n-g \|\leq 3\epsilon\) et donc la convergence uniforme.
\end{proof}

%--------------------------------------------------------------------------------------------------------------------------- 
\subsection{Permuter avec une intégrale}
%---------------------------------------------------------------------------------------------------------------------------

\begin{proposition}[Permuter limite et intégrale]       \label{PropbhKnth}
    Soit \( f_n\to f\) uniformément sur un ensemble mesuré \( A\) de mesure finie. Alors si les fonctions \( f_n\) et \( f\) sont intégrables sur \( A\), nous avons
    \begin{equation}
        \lim_{n\to \infty} \int_A f_n=\int_A \lim_{n\to \infty} f_n.
    \end{equation}
\end{proposition}

\begin{proof}
    Notons \( f\) la limite de la suite \( (f_n)\). Pour tout \( n\) nous avons les majorations
    \begin{subequations}
        \begin{align}
            \left| \int_A f_n d\mu-\int_A fd\mu \right| &\leq \int_A| f_n-f |d\mu\\
            &\leq \int_A \| f_n-f \|_{\infty}d\mu\\
            &=\mu(A)\| f_n-f \|_{\infty}
        \end{align}
    \end{subequations}
    où \( \mu(A)\) est la mesure de \( A\). Le résultat découle maintenant du fait que \( \| f_n-f \|_{\infty}\to 0\).
\end{proof}
Il existe un résultat considérablement plus intéressant que cette proposition. En effet, l'intégrabilité de \( f\) n'est pas nécessaire. Cette hypothèse peut être remplacée soit par l'uniforme convergence de la suite (théorème \ref{ThoUnifCvIntRiem}), soit par le fait que les normes des \( f_n\) sont uniformément bornées (théorème de la convergence dominée de Lebesgue \ref{ThoConvDomLebVdhsTf}).

\begin{theorem}[\cite{BJblWiS}]			\label{ThoUnifCvIntRiem}
    La limite uniforme d'une suite de fonctions intégrables sur un borné est intégrable, et on peut permuter la limite et l'intégrale. 
    
    Plus précisément, soit \( A\) un ensemble de \( \mu\)-mesure finie et \( f_n\colon A\to \eR\) des fonctions intégrables sur \( A\). Si la limite \( f_n\to f\) est uniforme, alors \( f\) est intégrable sur \( A\) et nous pouvons inverser la limite et l'intégrale :
    \begin{equation}
        \lim_{n\to \infty} \int_A f_n=\int_A\lim_{n\to \infty} f_n.
    \end{equation}
\end{theorem}

\begin{proof}
    Soit \( \epsilon>0\) et \( n\) tel que \( \| f_n-f \|_{\infty}\leq \epsilon\) (ici la norme uniforme est prise sur \( A\)). Étant donné que \( f_n\) est intégrable sur \( A\), il existe une fonction simple \( \varphi_n\) qui minore \( f_n\) telle que
    \begin{equation}
        \left| \int_{A}\varphi_n-\int_A f_n \right| <\epsilon.
    \end{equation}
    La fonction \( \varphi_n+\epsilon\) est une fonction simple qui majore la fonction \( f\). Si \( \psi\) est une fonction simple qui minore \( f\), alors
    \begin{equation}
        \int_A\psi\leq\int_A\varphi_n+\epsilon\leq\int_A f_n+\epsilon\mu(A).
    \end{equation}
    Par conséquent le supremum qui définit \( \int_A f\) existe, ce qui montre que \( f\) est intégrable. Le fait qu'on puisse inverser la limite et l'intégrale est maintenant une conséquence de la proposition \ref{PropbhKnth}.
\end{proof}

\begin{remark}
    L'hypothèse sur le fait que \( A\) est de mesure finie est importante. Il n'est pas vrai qu'une suite uniformément convergente de fonctions intégrables est intégrables. En effet nous avons par exemple la suite
    \begin{equation}
        f_n(x)=\begin{cases}
            1/x    &   \text{si \( x<n\)}\\
            0    &    \text{sinon}
        \end{cases}
    \end{equation}
    qui converge uniformément vers \( f(x)=1/x\) sur \( A=\mathopen[ 1 , \infty [\). Le limite n'est cependant pas intégrable sur \( A\).
\end{remark}

%--------------------------------------------------------------------------------------------------------------------------- 
\subsection{Permuter avec les dérivées partielles}
%---------------------------------------------------------------------------------------------------------------------------

\begin{theorem}		\label{ThoSerUnifDerr}
	Soit $U\subset\eR^n$ ouvert, $f_k\colon U\to \eR$ et $f_k$ de classe $C^1$. Supposons que $f_k$ converge simplement vers $f$ et que $\partial_if_k$ converge uniformément sur tout compact  vers une fonction $g_i$ pour $i=1,\ldots,n$. Alors $f$ est de classe $C^1$ et $\partial_if=g_i$. De plus, $f_k$ converge vers $f$ uniformément.
\end{theorem}
\index{permutation!dérivée et limite}
%TODO : une preuve.


%---------------------------------------------------------------------------------------------------------------------------
\subsection{Convergence de suites de fonctions}
%---------------------------------------------------------------------------------------------------------------------------

Nous considérons un espace normé \( (\Omega,\| . \|)\). Nous disons qu'une suite de fonctions \( f_n\) \defe{converge}{convergence!en norme} vers \( f\) pour la norme \( \| . \|\) si \( \forall \epsilon>0\), \( \exists N\) tel que \( n\geq N\) implique \( \| f_n-f \|<\epsilon\).

Dans le cas particulier de la norme 
\begin{equation}
    \| f \|_{\infty}=\sup_{x\in\Omega}| f(x) |,
\end{equation}
nous parlons que \defe{convergence uniforme}{convergence!uniforme!suite de fonctions}.

\begin{theorem}[Critère de Cauchy]  \label{ThoCauchyZelUF}
    Une suite de fonctions  \( (f_n)_{n\in\eN}\) sur \( \Omega\) converge en norme sur \( \Omega\) si et seulement si \( \forall\epsilon>0\), \( \exists N\) tel que
    \begin{equation}
        \| f_n-f_m \|<\epsilon
    \end{equation}
    pour \( n,m>N\).
\end{theorem}

\begin{corollary}       \label{CorCauchyCkXnvY}
    La série \( \sum f_n\) converge en norme sur \( \Omega\) si et seulement si \( \exists N\) tel que
    \begin{equation}
        \| f_n+\ldots+f_m \|\leq \epsilon
    \end{equation}
    pour tout \( n,m>N\).
\end{corollary}

\begin{proof}
    L'hypothèse montre que la suite des sommes partielles de la série \( \sum f_n\) vérifie le critère de Cauchy du théorème \ref{ThoCauchyZelUF}.
\end{proof}

%---------------------------------------------------------------------------------------------------------------------------
\subsection{Convergence monotone}
%---------------------------------------------------------------------------------------------------------------------------

\begin{theorem}[Théorème de la convergence monotone ou de Beppo-Levi\cite{mathmecaChoi}] \label{ThoConvMonFtBoVh}
    Soit un espace mesuré \( (\Omega,\tribA,\mu)\) et \( (f_n)\) une suite croissante de fonctions mesurables à valeurs dans \( \mathopen[ 0 , \infty \mathclose]\). Alors la limite ponctuelle \( \lim_{n\to \infty} f_n\) existe, est mesurable et
    \begin{equation}    \label{EqFHqCmLV}
        \lim_{n\to \infty} \int_{\Omega}f_nd\mu= \int_{\Omega}\lim_{n\to \infty} f_nd\mu,
    \end{equation}
    cette intégrable valant éventuellement \( \infty\).
\end{theorem}
\index{théorème!convergence!monotone}
\index{théorème!Beppo-Levi}

\begin{proof}
    La limite ponctuelle de la suite est la fonction à valeurs dans \( \mathopen[ 0 , \infty \mathclose]\) donnée par
    \begin{equation}
        f(x)=\lim_{n\to \infty} f_n(x).
    \end{equation}
    Ces limites existent parce que pour chaque \( x\) la suite \( f_n(x)\) est une suite numérique croissante. Nous notons
    \begin{equation}
        I_0=\int_{\Omega}fd\mu.
    \end{equation}
    Nous posons par ailleurs
    \begin{equation}
        I_n=\int_{\Omega}f_n.
    \end{equation}
    Cela est une suite numérique croissante qui a par conséquent une limite que nous notons \( I=\lim_{n\to \infty} I_n\). Notre objectif est de montrer que \( I=I_0\). D'abord par croissance de la suite, pour tous $n$ nous avons \( I_n\leq I_0\), par conséquent \( I\leq I_0\).

    Nous prouvons maintenant l'inégalité dans l'autre sens en nous servant de la définition \eqref{EqDefintYfdmu}. Soit une fonction simple \( h\) telle que \( h\leq f\), et une constante \( 0<C<1\). Nous considérons les ensembles
    \begin{equation}
        E_n=\{ x\in\Omega\tq f_n(x)\geq Ch(x) \}.
    \end{equation}
    Ces ensembles vérifient les propriétés \( E_n\subset E_{n+1}\) et \( \bigcup_{n=1}^{\infty}E_n=\Omega\). Pour chaque \( n\) nous avons les inégalités
    \begin{equation}
        \int_{\Omega}f_n\geq\int_{E_n}f_n\geq C\int_{E_n}h.
    \end{equation}
    Si nous prenons la limite \( n\to\infty\) dans ces inégalités,
    \begin{equation}
        \lim_{n\to \infty} \int_{\Omega}f_n\geq C\lim_{n\to \infty} \int_{E_n}h=C\int_{\Omega}h.
    \end{equation}
    Par conséquent \( \lim_{n\to \infty} \int f_n\geq C\int_{\Omega}h\). Mais étant donné que cette inégalité est valable pour tout \( C\) entre \( 0\) et \( 1\), nous pouvons l'écrire sans le \( C\) :
    \begin{equation}        \label{EqzAKEaU}
        \lim_{n\to \infty} \int_{\Omega}f_n\geq \int_{\Omega}h.
    \end{equation}
    Par définition, l'intégrale de \( f\) est donné par le supremum des intégrales de \( h\) où \( h\) est une fonction simple dominée par \( f\). En prenant le supremum sur \( h\) dans l'équation \eqref{EqzAKEaU} nous avons
    \begin{equation}
        \lim_{n\to \infty} \int_{\Omega}f_n\geq\int_{\Omega}f,
    \end{equation}
    ce qu'il nous fallait.
\end{proof}

\begin{remark}
    La proposition \ref{PropWBavIf} ainsi que le lemme \ref{LemYFoWqmS} montrent qu'une fonction mesurable peut-être écrite comme limite croissante de fonctions simples. Cela permet de démontrer des théorèmes en commençant par prouver sur les fonctions simples et en utilisant Beppo-Levi pour généraliser.
\end{remark}

\begin{remark}
    Une des raisons de demander la positivité des fonctions \( f_n\) est de n'avoir pas d'ambiguïté à parler d'intégrales qui valent \( \infty\). Si par exemple nous prenons \( \Omega=\mathopen[ 0 , 1 \mathclose]\) et que nous considérons
    \begin{equation}
        f_n(x)=\begin{cases}
            0    &   \text{si \( x\leq \frac{1}{ n }\)}\\
            \frac{1}{ x }    &    \text{sinon}.
        \end{cases}
    \end{equation}
    Ce sont des fonctions intégrables, mais la limite étant la fonction \( 1/x\), l'égalité \eqref{EqFHqCmLV} est une égalité entre deux intégrales valant \( \infty\).
\end{remark}

\begin{corollary}[Inversion de somme et intégrales] \label{CorNKXwhdz}
    Si \( (u_n)\) est une suite de fonctions mesurables positives ou nulles, alors
    \begin{equation}
        \sum_{i=0}^{\infty}\int u_i=\int\sum_{i=0}^{\infty}u_i.
    \end{equation}
\end{corollary}

\begin{proof}
    Nous considérons la suite des sommes partielles de \( (u_n)\) : \( f_n(x)=\sum_{i=0}^nu_n(x)\). Le théorème de la convergence monotone (théorème \ref{ThoConvMonFtBoVh}) implique que
    \begin{equation}
        \lim_{n\to \infty} \int f_n=\int\lim_{n\to \infty} f_n.
    \end{equation}
    Nous remplaçons maintenant \( f_n\) par sa valeur en termes des \( u_i\) et dans le membre de gauche nous permutons l'intégrale avec la somme finie :
    \begin{equation}
        \lim_{n\to \infty} \sum_{i=0}^{\infty}\int u_n=\int\sum_{i=0}^{\infty}u_n,
    \end{equation}
    ce qu'il fallait démontrer.
\end{proof}

\begin{lemma}[Lemme de Fatou]\index{lemme!Fatou}\index{Fatou}   \label{LemFatouUOQqyk}
    Soit \( (\Omega,\tribA,\mu)\) un espace mesuré et \( f_n\colon \Omega\to \mathopen[ 0 , \infty \mathclose]  \) une suite de fonctions mesurables. Alors la fonction \( f(x)=\liminf f_n(x)\) est mesurable et
    \begin{equation}
        \int_{\Omega}\liminf f_nd\mu\leq\liminf\int_{\Omega}fd\mu.
    \end{equation}
\end{lemma}
%TODO : pour la mesurabilité, il faudra citer un théorème du genre de celui fait avec le sup.

\begin{proof}
    Nous posons 
    \begin{equation}
        g_n(x)=\inf_{i\geq n}f_i(x).
    \end{equation}
    Cela est une suite croissance de fonctions positives mesurables telles que, par définition, 
    \begin{equation}
        \lim_{n\to \infty}g_n(x)=\liminf f_n(x).
    \end{equation}
    Nous pouvons y appliquer le théorème de la convergence monotone,
    \begin{equation}
        \lim_{n\to \infty} \int g_n(x)=\int\liminf f_n(x).
    \end{equation}
    Par ailleurs, pour chaque \( i\geq n\) nous avons
    \begin{equation}
        \int g_n\leq \int f_i,
    \end{equation}
    en passant à l'infimum nous avons
    \begin{equation}
        \int g_n\leq \inf_{i\geq n}\int f_i,
    \end{equation}
    et en passant à la limite nous avons
    \begin{equation}
        \int\liminf f_n=\lim_{n\to \infty} \int g_n\leq \lim_{n\to \infty} \inf_{i\geq n}\int f_i=\liminf_{i\to\infty}\inf f_i.
    \end{equation}
\end{proof}

L'inégalité donnée dans ce lemme n'est en général pas une égalité, comme le montre l'exemple suivant :
\begin{equation}
    f_i=\begin{cases}
        \mtu_{\mathopen[ 0 , 1 \mathclose]}    &   \text{si \( i\) est pair}\\
        \mtu_{\mathopen[ 1 , 2 \mathclose]}    &    \text{si \( i\) est impair}.
    \end{cases}
\end{equation}
Nous avons évidemment \( g_n(x)=0\) tandis que \( \int_{\mathopen[ 0 , 2 \mathclose]}f_i=1\) pour tout \( i\).

%---------------------------------------------------------------------------------------------------------------------------
\subsection{Convergence dominée de Lebesgue}
%---------------------------------------------------------------------------------------------------------------------------

\begin{theorem}[Convergence dominée de Lebesgue]        \label{ThoConvDomLebVdhsTf}
    Soit \( (f_n)_{n\in\eN}\) une suite de fonctions intégrables sur \( (\Omega,\tribA,\mu)\) à valeurs dans \( \eC\) ou \( \eR\). Nous supposons que  \( f_n\to f\) simplement sur \( \Omega\) presque partout et qu'il existe une fonction intégrable \( g\) telle que
    \begin{equation}
        | f_n(x) |< g(x) 
    \end{equation}
    pour presque\footnote{Si il n'y avait pas le «presque» ici, ce théorème serait à peu près inutilisable en probabilité ou en théorie des espaces \( L^p\), comme dans la démonstration du théorème de Fischer-Riesz \ref{ThoGVmqOro} par exemple.} tout \( x\in\Omega\) et pour tout \( n\in \eN\). Alors
    \begin{enumerate}
        \item
            \( f\) est intégrable,
        \item
           $\lim_{n\to \infty} \int_{\Omega}f_n=\int_\Omega f$,
        \item
            $\lim_{n\to \infty} \int_{\Omega}| f_n-f |=0$.
    \end{enumerate}
\end{theorem}
\index{théorème!convergence!dominée de Lebesgue}
\index{dominée!convergence (Lebesgue)}

\begin{proof}

    La fonction limite \( f\) est intégrable parce que \( | f |\leq g\) et \( g\) est intégrable (lemme \ref{LemPfHgal}). Par hypothèse nous avons
    \begin{equation}
        -g(x)\leq f_n(x)\leq g(x).
    \end{equation}
    En particulier la fonction \( g_n=f_n+g\) est positive et mesurable si bien que le lemme de Fatou (lemme \ref{LemFatouUOQqyk}) implique
    \begin{equation}
        \int_{\Omega}\liminf g_n\leq\liminf\int_{\Omega}g_n.
    \end{equation}
    Évidement nous avons \( \liminf g_n=f+g\), de telle sorte que
    \begin{equation}
        \int f+\int g\leq \liminf\int g_n=\liminf\int f_n+\int g,
    \end{equation}
    et le nombre \( \int g\) étant fini, nous pouvons le retrancher des deux côtés de l'inégalité :
    \begin{equation}
        \int f\leq\liminf\int f_n.
    \end{equation}
    Afin d'obtenir une minoration de \( \int f\) nous refaisons exactement le même raisonnement en utilisant la suite de fonctions \( k_n=-f_n\to k=-f\). Nous obtenons que
    \begin{equation}
        \int k\geq\liminf\int k_n=-\limsup\int f_n,
    \end{equation}
    et par conséquent
    \begin{equation}    \label{IneqsndMYTO}
        \liminf\int f_n\leq\int f\leq\limsup\int f_n.
    \end{equation}
    La limite supérieure étant plus grande ou égale à la limite inférieure, les trois quantités dans les inégalités \eqref{IneqsndMYTO} sont égales.

    Nous prouvons maintenant le troisième point. Soit la suite de fonctions
    \begin{equation}
        h_n(x)=| f_n(x)-f(x) |
    \end{equation}
    qui tend ponctuellement vers zéro. De plus
    \begin{equation}
    h_n(x)\leq | f_n(x) |+| f(x) |\leq 2g(x),
    \end{equation}
    ce qui prouve que les \( h_n\) majorés par une fonction intégrable. Donc
    \begin{equation}
        \lim_{n\to \infty} \int_{\Omega}| f_n-f |= \lim_{n\to \infty} \int_{\Omega}h_n(x)dx=\int_{\Omega}\lim_{n\to \infty} | f_n(x)-f(x) |=0
    \end{equation}
\end{proof}

\begin{remark}
    Lorsque nous travaillons sur des problèmes de probabilités, la fonction \( g\) peut être une constante parce que les constantes sont intégrables sur un espace de probabilité.
\end{remark}

\begin{corollary}       \label{CorCvAbsNormwEZdRc}
    Soit \( (a_i)_{i\in \eN}\) une suite numérique absolument convergente. Alors elle est convergente. Il en est de même pour les séries de fonctions si on considère la convergence ponctuelle.
\end{corollary}

\begin{proof}
    L'hypothèse est la convergence de l'intégrale \( \int_{\eN}| a_i |dm(i)\) où \( dm\) est la mesure de comptage. Étant donné que \( | a_i |\leq | a_i |\), la fonction \( a_i\) (fonction de \( i\)) peut jouer le rôle de \( g\) dans le théorème de la convergence dominée de Lebesgue (théorème \ref{ThoConvDomLebVdhsTf}).
\end{proof}
Nous utiliseront ce résultat pour montrer que la transformée de Fourier d'une fonction \( L^1(\eR^d)\) est continue (proposition \ref{PropJvNfj}).

\begin{proposition}[\cite{YHRSDGc}] \label{PropUXjnwLf}
    \begin{enumerate}
        \item
            Une fonction mesurable et positive est limite (simple) d'une suite croissante de fonctions étagées, mesurables et positives.
        \item
            Si \( f\colon \eR^d\to \bar \eR\) est mesurable, alors elle est limite (simple) de fonctions étagées \( f_n\) telles que \( | f_n |\leq | f |\).
    \end{enumerate}
\end{proposition}
%TODO : la preuve est dans le document cité.

%+++++++++++++++++++++++++++++++++++++++++++++++++++++++++++++++++++++++++++++++++++++++++++++++++++++++++++++++++++++++++++ 
\section{Séries de fonctions}
%+++++++++++++++++++++++++++++++++++++++++++++++++++++++++++++++++++++++++++++++++++++++++++++++++++++++++++++++++++++++++++

Les séries de fonctions sont des cas particuliers de suites, étant donné que, par définition,
\begin{equation}
    \sum_{n=1}^{\infty}f_n=\lim_{N\to \infty} \sum_{n=1}^{N}f_n.
\end{equation}

\begin{definition}  \label{DefQDrDqek}
    Une série de nombres \( \sum_{n=0}^{\infty}a_n\) converge \defe{absolument}{convergence!absolue} si la série $\sum_{n=0}^{\infty}| a_n |$ converge. Cette définition s'étend immédiatement aux séries dans n'importe quel espace normé.

    Une série de fonctions \( \sum_{n\in \eN}u_n \) converge \defe{normalement}{convergence!normale} si la série de nombre \( \sum_n\| u_n \|_{\infty}\) converge.
\end{definition}

La convergence normale est à ne pas confondre avec la convergence uniforme. La somme \( \sum_nf_n\) \defe{converge uniformément}{convergence!uniforme!série de fonctions} vers la fonction \( F\) si la suite des sommes partielles converge uniformément, c'est à dire si 
\begin{equation}
    \lim_{N\to \infty} \| \sum_{n=1}^Nf_n-F \|_{\infty}=0.
\end{equation}

\begin{lemma}
    Soient des fonctions \( u_n\colon \Omega\to \eC\). Si il existe une suite réelle positive \( (a_n)_{n\in \eN}\) telle que
    \begin{enumerate}
        \item
            pour tout \( z\in \Omega\) et pour tout \( n\in \eN\) nous avons \( | u_n(z) |\leq a_n\) (c'est à dire \( a_n\geq \| u_n \|_{\infty}\)),
        \item
            la somme \( \sum_{n}a_n\) converge,
    \end{enumerate}
    alors la série de fonctions \( \sum_{n=0}^{\infty}u_n\) converge normalement.
\end{lemma}

\begin{proof}
    Découle du lemme de comparaison \ref{LemgHWyfG}.
\end{proof}

\begin{theorem}				\label{ThoSerCritAbel}
	Soit $\sum_{k=1}^{\infty}g_k(x)$, une série de fonctions complexes où $g_k(x)=\varphi_k(x)\psi_k(x)$. Supposons que
	\begin{enumerate}

		\item
			$\varphi_k\colon A\to \eC$ et $| \sum_{k=1}^K\varphi_k(x) |\leq M$ où $M$ est indépendant de $x$ et $K$,
		\item
			$\psi_k\colon A\to \eR$ avec $\psi_k(x)\geq 0$ et pour tout $x$ dans $A$, $\psi_{k+1}(x)\leq \psi_k(x)$, et enfin supposons que $\psi_k(x)$ converge uniformément vers $0$.

	\end{enumerate}
	Alors $\sum_{k=1}^{\infty}g_k$ est uniformément convergente.
\end{theorem}

\begin{theorem}		\label{ThoAbelSeriePuiss}
	Si la série de puissances (réelle) converge en $x=x_0+R$, alors elle converge uniformément sur $\mathopen[ x_0-R+\epsilon , x_0+R \mathclose]$ ($\epsilon>0$) vers une fonction continue.
\end{theorem}


\begin{proposition}     \label{PropUEMoNF}
    Soit \( (u_n)\) une suite de fonctions continues \( u_n\colon \Omega\subset\eC\to \eC\). Si la série \( \sum_nu_n\) converge normalement alors la somme est continue.
\end{proposition}

\begin{proof}
    Nous posons \( u(z)=\lim_{N\to \infty} \sum_{n=0}^N u_n(z)\), et nous vérifions que la fonction ainsi définie sur \( \Omega\) est continue. Soit \( z\in \Omega\) et prouvons la continuité de \( u\) au point \( z\). Pour tout \( z'\) dans un voisinage de \( z\) nous avons 
    \begin{subequations}
        \begin{align}
            \big| u(z)-u(z') \big|&=\left| \sum_{n=0}^{N}u_n(z)-\sum_{n=0}^{N}u_n(z')+\sum_{n=N+1}^{\infty}u_n(z)-\sum_{n=N+1}^{\infty}u_n(z') \right| \\
            &\leq \left| \sum_{n=0}^N u_n(z)-\sum_{n=0}^Nu_n(z') \right| +\sum_{n=N+1}^{\infty}| u_n(z) |+\sum_{n=N+1}^{\infty}| u_n(z') |.
        \end{align}
    \end{subequations}
    Étant donné que les sommes partielles sont continues, en prenant \( N\) suffisamment grand, le premier terme peut être rendu arbitrairement petit. Si \( N\) est suffisamment grand, le second terme est également petit. Par contre, cet argument ne tient pas pour le troisième terme parce que nous souhaitons une majoration pour tout \( z'\) dans une boule autour de \( z\). Nous devons donc écrire
    \begin{equation}
        \sum_{n=N}^{\infty}| u_n(z) |\leq \sum_{n=N+1}^{\infty}\| u_n \|_{\infty}.
    \end{equation}
    Ce dernier est arbitrairement petit lorsque \( N\) est grand. Notons que nous avons utilisé l'hypothèse de convergence normale.
\end{proof}

La même propriété, avec la même démonstration, tient dans le cas d'espaces vectoriels normée.
\begin{proposition} \label{PropOMBbwst}
    Soient \( E\) et \( F\), deux espaces vectoriels normés, \( \Omega\) une partie ouverte de \( E\) et une suite de fonctions \( u_n\colon \Omega\to F\) convergeant normalement sur \( \Omega\), c'est à dire que \( \sum_n\| u_n \|_{\infty}\) converge, la norme \( \| . \|_{\infty} \) devant être comprise comme la norme supremum sur \( \Omega\). Alors la fonction \( u=\sum_nu_n\) est continue sur \( \Omega\).
\end{proposition}

\begin{proof}
    Soit \( x,x'\in \Omega\) en supposant que \( \| x-x' \|\) est petit. Soit encore \( \epsilon>0\). Nous allons montrer la continuité en \( x\). Pour cela nous savons que pour tout \( N\) l'inégalité suivante est correcte :
    \begin{equation}
        \| u(x)-u(x') \|\leq \left\|  \sum_{n=0}^Nu_n(x)-\sum_{n=0}^{N}u_n(x') \right\|+\sum_{n=N+1}^{\infty}\| u_n(x) \|+\sum_{n=N+1}^{\infty}\| u_n(x') \|.
    \end{equation}
    Les deux derniers termes sont majorés par \( \sum_{n=N+1}^{\infty}\| u_n \|_{\infty}\) qui, par hypothèse, peut être rendu aussi petit que souhaité en choisissant \( N\) assez grand. Nous choisissons donc un \( N\) tel que ces deux termes soient plus petits que \( \epsilon\). Ce \( N\) étant fixé, la fonction \( \sum_{n=0}^{N}u_n\) est continue et nous pouvons choisir \( x'\) assez proche de \( x\) pour que le premier terme soit majoré par \( \epsilon\).
\end{proof}

\begin{theorem}			\label{ThoSerUnifCont}
	Si les $g_k$ sont continues et si $\sum g_k$ converge uniformément, alors $\sum g_k$ est continue.
\end{theorem}

\begin{theorem}[Critère de Weierstrass]\index{critère!Weierstrass!série de fonctions}		\label{ThoCritWeierstrass}
	Soit une suite de fonctions $f_k\colon A\to \eC$ telles que $| f_k(x) |\leq M_k\in\eR$, $\forall x\in A$. Si $\sum_{k=1}^{\infty}M_k$ converge, alors $\sum_{k=1}^{\infty}f_k$ converge absolument et uniformément.
\end{theorem}

\begin{proof}
    La convergence normale est facile : l'hypothèse dit que \( \| f_k \|_{\infty}\leq M_k\), et donc que
    \begin{equation}
        \sum_{k=1}^{\infty}\| f_k \|_{\infty}\leq \sum_kM_k<\infty.
    \end{equation}
    
    La convergence uniforme est à peine plus subtile. Nous nommons \( F\) la fonction somme. Pour tout \( x\) et pour tout \( N\), nous avons
    \begin{subequations}
        \begin{align}
            \left\| \sum_{n=1}^Nf_n(x)-F(x) \right\|&=\| \sum_{n=N}^{\infty}f_n(x) \|\\
            &\leq\sum_{n=N}^{\infty}\| f_k(x) \|\\
            &\leq \sum_{n=N}^{\infty}\| f_n \|_{\infty}.
        \end{align}
    \end{subequations}
    La convergence normale étant assurée, la série \( \sum_{n_1}^{\infty}\| f_n \|_{\infty}\) est finie, ce qui implique que la queue de somme \( \sum_{n=N}^{\infty}\| f_n \|_{\infty}\) tend vers zéro lorsque \( N\to \infty\). Pour tout \( \epsilon\), il existe donc un \( N\) (non dépendant de \( x\)) tel que
    \begin{equation}
        \| \sum_{n=1}^Nf_n(x)-F(x) \|\leq \epsilon.
    \end{equation}
    En prenant le supremum sur \( x\in A\) nous trouvons la convergence uniforme.
\end{proof}

\begin{remark}
    Il n'y a pas de critère correspondant pour les suites. Il n'est pas vrai que si \( \lim_{n\to \infty}\| f_n \| \) existe, alors \( \lim_{n\to \infty} f_n\) existe, comme le montre l'exemple
    \begin{equation}
        f_n(x)=\begin{cases}
            1    &   \text{si \( x\in\mathopen[ 0 , 1 \mathclose]\) et \( n\) est pair}\\
            1    &    \text{si \( x\in\mathopen[ 1 , 2 \mathclose]\) et \( n\) est impair}\\
             0   &    \text{sinon.}
        \end{cases}
    \end{equation}
\end{remark}

\begin{theorem}      \label{ThoCciOlZ}
    La somme uniforme de fonctions intégrables sur un ensemble de mesure fini est intégrable et on peut permuter la somme et l'intégrale.

    En d'autres termes, supposons que \( \sum_{n=0}^{\infty}f_n\) converge uniformément vers \( F\) sur \( A\) avec \( \mu(A)<\infty\). Si \( F\) et \( f_n\) sont des fonctions intégrables sur \( A\) alors
    \begin{equation}
        \int_AF(x)d\mu(x)=\sum_{n=0}^{\infty}\int_Af_n(x)d\mu(x).
    \end{equation}
\end{theorem}

\begin{proof}
    Ce théorème est une conséquence du théorème \ref{ThoUnifCvIntRiem}. En effet nous définissons la suite des sommes partielles
    \begin{equation}
        F_N=\sum_{n=0}^Nf_n.
    \end{equation}
    La limite \( \lim_{N\to \infty} F_N=F\) est uniforme. Par conséquent la fonction \( F\) est intégrable et
    \begin{equation}
        \int_A F=\lim_{N\to \infty} \int_AF_N=\lim_{N\to \infty} \int_A\sum_{n=0}^Nf_n=\lim_{N\to \infty} \sum_{n=0}^N\int_Af_n=\sum_{n=0}^{\infty}\int_Af_n.
    \end{equation}
    La première égalité est le théorème \ref{ThoUnifCvIntRiem}, les autres sont de simples manipulations rhétoriques.
\end{proof}


Le théorème suivant est une paraphrase du théorème de la convergence dominée de Lebesgue (\ref{ThoConvDomLebVdhsTf}).
\begin{theorem}     \label{ThoockMHn}
    Soient des fonctions \( (f_n)_{n\in \eN}\) telles que \( \sum_{n=0}^Nf_n\) soit intégrable sur \( (\Omega,\tribA,\mu)\) pour chaque \( N\). Nous supposons que la somme converge simplement vers
    \begin{equation}
        f(x)=\sum_{n=0}^{\infty}f_n(x)
    \end{equation}
    et qu'il existe une fonction \( g\) telle que
    \begin{equation}
        \left| \sum_{n=0}^Nf_n \right| <g
    \end{equation}
    pour tout \( N\in \eN\). Alors
    \begin{enumerate}
        \item
            \( \sum_{n=0}^{\infty}f_n\) est intégrable,
        \item
            on peut permuter somme et intégrale :
            \begin{equation}
                \lim_{N\to \infty} \int_{\Omega}\sum_{n=0}^Nf_nd\mu=\int_{\Omega}\sum_{n=0}^{\infty}f_n,
            \end{equation}
        \item
            \begin{equation}
                \lim_{N\to \infty} \int_{\Omega}\left| \sum_{n=0}^Nf_n-\sum_{n=0}^{\infty}f_n \right| =\lim_{N\to \infty} \int_{\Omega}\left| \sum_{n=N}^{\infty}f_n \right| =0.
            \end{equation}
    \end{enumerate}
\end{theorem}


\begin{theorem} \label{ThoCSGaPY}
    Soit \( f_n\) des fonctions \( C^1\mathopen[ a , b \mathclose]\) telles que
    \begin{enumerate}
        \item
            la série \( \sum_n f_n(x_0)\) converge pour un certain \( x_0\in\mathopen[ a , b \mathclose]\),
        \item
            la série des dérivées \( \sum_n f'_n\) converge uniformément sur \( \mathopen[ a , b \mathclose]\).
    \end{enumerate}
    Alors la série \( \sum_n f_n\) converge vers une fonction \( F\) et
    \begin{enumerate}
        \item
            La convergence est uniforme sur \( \mathopen[ a , b \mathclose]\).
        \item
            La fonction \( F\) est dérivable
        \item
            \( F'(x)=\sum_nf'_n(x)\).
    \end{enumerate}
\end{theorem}

\begin{lemma}
    Soient \( E\) et \( F\) deux espaces vectoriels normés. Si la suite \( (T_n))\) converge vers \( T\) dans \( \aL(E,F)\), alors pour tout \( v\in E\) nous avons
    \begin{equation}
        \left( \sum_{n=0}^{\infty}T_n \right)(v)=\sum_{n=0}^{\infty}T_n(v).
    \end{equation}
\end{lemma}

\begin{theorem}[\cite{DHdwZRZ}] \label{ThoLDpRmXQ}
    Soit \( E\) et \( F\), deux espaces vectoriels normés, \( \Omega\) un ouvert connexe par arcs de \( E\). Soit \( (u_n)\) une suite de fonctions \( u_n\colon \Omega\to F\) telle que
    \begin{enumerate}
        \item
            pour tout \( n\), la fonction \( u_n\) est de classe \( C^1\) sur \( \Omega\),
        \item
            la série \( \sum_nu_n\) converge simplement sur \( \Omega\),
        \item
            la série des différentielles \( \sum_n(du_n)\) converge normalement sur tout compact de \( \Omega\).
    \end{enumerate}
    Alors la somme \( u=\sum_nu_n\) est de classe \( C^1\) sur \( \Omega\) et sa différentielle est donnée par
    \begin{equation}
        du=\sum_{n=0}^{\infty}du_n.
    \end{equation}
\end{theorem}

\begin{proof}
    Pour chaque \( n\), la fonction \( du_n\colon \Omega\to \aL(E,F)\) est une fonction continue parce que \( u_n\) est de classe \( C^1\). La série convergeant normalement, la fonction \( \sum_{n=0}^{\infty}du_n\) est également continue par la proposition \ref{PropOMBbwst}. La difficulté de ce théorème est donc de prouver que cela est bien la différentielle de la fonction \( \sum_nu_n\).

    Soit \( a,x\in \Omega\) et \( \gamma\colon \mathopen[ 0 , 1 \mathclose]\to \Omega\) un chemin joignant \( a\) à \( x\). Nous considérons ce chemin en coordonnées normales et nous notons \( l\) sa longueur. Par définition \ref{EqEFIZyEe},
    \begin{equation}
        \clubsuit=\int_{\gamma}\sum_{n=0}^{\infty}du_n=\int_0^l\sum_n(du_n)_{\gamma(t)}\big( \gamma'(t) \big)dt
    \end{equation}
    Si nous notons \( f_n(t)=(du_n)_{\gamma(t)}\big( \gamma'(t) \big)\), sachant que la paramétrisation est normale (\( \| \gamma'(t) \|=1\)) nous avons\footnote{Histoire de ne pas s'embrouiller, il faut se rendre compte que \( \| du_n \|_{\infty}=\sup_{x\in \Omega}\| (du_n)_x \|\).}
    \begin{equation}
        \| f_n(t) \|\leq \|   (du_n)_{\gamma(t)}  \|\leq \| du_n \|_{\infty}.
    \end{equation}
    Or la série des \( \| du_n \|_{\infty}\) converge par hypothèse. L'intervalle \( \mathopen[ 0 , l \mathclose]\) étant compact, les fonctions \( f_n\) sont uniformément (en \( n\)) bornées par le nombre \( \sum_n\| du_n \|_{\infty}\) qui est intégrable sur \( \mathopen[ 0 , 1 \mathclose]\). Par la convergence dominée (théorème \ref{ThoConvDomLebVdhsTf}) nous permutons la somme et l'intégrale :
    \begin{equation}
        \clubsuit=\sum_{n=0}^{\infty}\int_0^l(du_n)_{\gamma(t)}\big( \gamma'(t) \big)dt=\sum_{n=0}^{\infty}u_n(x)-\sum_{n=0}^{\infty}u_n(a)=u(x)-u(a)
    \end{equation}
    où nous avons utilisé le théorème \ref{ThoUJMhFwU}. Jusqu'à présent nous avons montré que
    \begin{equation}
        u(x)=u(a)+\int_{\gamma}\sum_{n=0}^{\infty}du_n=u(a)+\int_0^l\sum_{n=0}^{\infty}(du_n)_{\gamma(t)}\big( \gamma'(t) \big)dt.
    \end{equation}
    Nous allons utiliser cela pour calculer \( du_x(v)\) selon la bonne vieille formule
    \begin{equation}
        du_x(v)=\Dsdd{ u(x+sv) }{s}{0}.
    \end{equation}
    Cela sera fait en considérant à nouveau un chemin \( \gamma_s \) joignant \( a\) à \( x+sv\) en paramétrisation normale; nous notons \( l_s\) sa longueur. Dans le calcul suivant, nous inversons la somme et l'intégrale de la même façon qu'avant. En piste maestro
    \begin{subequations}
        \begin{align}
            du_x(v)&=\frac{ d  }{ d s }\left.\int_0^{l_s}\sum_{n=0}^{\infty}(du_n)_{\gamma_s(t)}\big( \gamma'_s(t) \big)dt\right|_{s=0}\\
            &=\frac{ d  }{ d s }\left.\sum_{n=0}^{\infty}\int_{\gamma_s}du_n\right|_{s=0}\\
            &=\frac{ d  }{ d s }\sum_{n=0}^{\infty}\Big[ u_n\big( \gamma_s(l_s)\big)-u_n\big( \gamma_s(0) \big)  \Big]_{s=0}\\
            &=d\left( \sum_{n=0}^{\infty}u_n \right)_x(v).
        \end{align}
    \end{subequations}
\end{proof}

%---------------------------------------------------------------------------------------------------------------------------
\subsection{Théorème de Stone-Weierstrass}
%---------------------------------------------------------------------------------------------------------------------------

Comme presque tous les théorèmes importants, le théorème de Stone-Weierstrass possède de nombreuses formulations à divers degrés de généralité.

Le lemme suivant est une cas particulier du théorème \ref{ThoGddfas}, mais nous en donnons une démonstration indépendante afin d'isoler la preuve de la généralisation \ref{ThoWmAzSMF}. Une version pour les polynômes trigonométrique sera donnée dans le lemme \ref{LemXGYaRlC}.

\begin{lemma}       \label{LemYdYLXb}
    Il existe une suite de polynômes sur \( \mathopen[ 0 , 1 \mathclose]\) convergent uniformément vers la fonction racine carré.
\end{lemma}

\begin{proof}
    Nous donnons cette suite par récurrence :
    \begin{subequations}
        \begin{align}
            P_0(t)&=0\\
            P_{n+1}(t)&=P_n(t)+\frac{ 1 }{2}\big( t-P_n(t)^2 \big).
        \end{align}
    \end{subequations}
    Nous commençons par montrer que pour tout \( t\in \mathopen[ 0 , 1 \mathclose]\), \( P_n(t)\in\mathopen[ 0 , \sqrt{t} \mathclose]\). Pour \( P_0\), c'est évident. Ensuite nous avons
    \begin{subequations}
        \begin{align}
            P_{n+1}(t)-\sqrt{t}&=P_n(t)-\sqrt{t}+\frac{ 1 }{2}(t-P_n(t)^2)\\
            &=\big( P_n(t)-\sqrt{t} \big)\left( 1-\frac{ 1 }{2}\frac{ t-P_n(t)^2 }{ P_n(t)-\sqrt{t} } \right)\\
            &=\big( P_n(t)-\sqrt{t} \big)\left( 1-\frac{ \sqrt{t}+P_n(t) }{2} \right)\\
            &\leq 0
        \end{align}
    \end{subequations}
    parce que \( \sqrt{t} \leq 1\) et \( P_n(t)\leq 1\) par hypothèse de récurrence.

    Nous savons au passage que \( P_n(t)\) est une suite réelle croissante parce que \( t-P_n(t)^2\geq t-(\sqrt{t})^2=0\). La suite \( P_n(t)\) est donc croissante et majorée par \( \sqrt{t}\); elle converge donc. Les candidats limites sont déterminés par l'équation
    \begin{equation}
        \ell=\ell+\frac{ 1 }{2}(t-\ell^2),
    \end{equation}
    dont les solutions sont \( \ell=\pm\sqrt{t}\). La suite étant positive, nous avons une convergence ponctuelle de \( P_n\) vers la racine carré. Cette suite étant une suite croissante de fonctions continues sur un compact, convergeant ponctuellement vers une fonction continue, la convergence est uniforme par le théorème de Dini \ref{ThoUFPLEZh}.
\end{proof}

\begin{lemma}           \label{LemUuxcqY}
    Soit \( K\), un compact de \( \eR\) et \( f_n\) une suite de fonctions sur \( K\) convergeant uniformément vers \( f\). Soit \( g\colon X\to K\) une fonction depuis un espace topologique \( K\). Alors \( f_n\circ g\) converge uniformément vers \( f\circ g\).
\end{lemma}

\begin{proof}
    En effet, pour tout \( x\in X\) nous avons
    \begin{equation}
        \| (f_n\circ g)-(f\circ g) \|_{\infty}=\sup_{x\in X} \| f_n\big( g(x) \big)-f\big( g(x) \big) \|\leq \| f_n-f \|_{\infty}.
    \end{equation}
    Par conséquent, si \( \epsilon\>0\) est donné, il suffit de choisir \( n\) de telle sorte à avoir \( \| f_n-f \|_{\infty}<\epsilon\) et nous avons \( \| (f_n\circ g)-(f\circ g) \|_{\infty}\leq \epsilon\).
\end{proof}

\begin{definition}
    Nous disons qu'une algèbre \( A\) de fonctions sur un espace \( X\) \defe{sépare les points}{sépare!les points} de \( X\) si pour tout \( x_1\neq x_2\) il existe \( g\in A\) telle que \( g(x_1)\neq g(x_2)\).
\end{definition}

Nous pouvons maintenant énoncer et démontrer une forme nettement plus générale du théorème de Stone-Weierstrass.
\begin{theorem}[Stone-Weierstrass\cite{MGecheleSW}] \label{ThoWmAzSMF}
    Soit \( X\), un espace compact et Hausdorff et \( A\) une sous algèbre de \( C(X,\eR)\) contenant une fonction constante non nulle. Alors \( A\) est dense dans \( \Big( C(X,\eR),\| . \|_{\infty}\Big)\) si et seulement si \( A\) sépare les points de \(X\).

    Nous pouvons remplacer \( \eR\) par \( \eC\) si de plus l'algèbre \( A\) est auto-adjointe : \( g\in A\) implique \( \bar g\in A\).
\end{theorem}
\index{théorème!Stone-Weierstrass}

\begin{proof}
    Nous allons écrire la démonstration en plusieurs étapes (dont la première est le lemme \ref{LemYdYLXb}).

    \begin{description}
        \item[Première étape] Pour tout \( x\neq y\in X\) et pour tout \( \alpha,\beta\in \eR\), il existe une fonction \( f\in A\) telle que \( f(x)=\alpha\) et \( f(y)=\beta\). 

            En effet, vu que \( A\) sépare les points nous pouvons considérer une fonction \( g\in A\) telle que \( g(x)\neq g(y)\) et ensuite poser
            \begin{equation}
                f(z)=\alpha+\frac{ \alpha-\beta }{ g(y)-g(x) }\big( g(z)-g(x) \big).
            \end{equation}
            Les constantes faisant partie de \( A\), cette fonction \( f\) est encore dans \( A\).

        \item[Seconde étape] Pour tout \( n\)-uples de fonctions \( f_1,\ldots, f_n\) dans \( \bar A\), les fonctions \( \min(f_1,\ldots, f_n)\) et \( \max(f_1,\ldots, f_n)\) sont dans \( \bar A\).

            Nous le démontrons pour \( n=2\); le reste allant évidemment par récurrence. Soient \( f,g\in \bar A\). Étant donné que
            \begin{subequations}
                \begin{align}
                    \max(f,g)&=\frac{ f+g }{2}+\frac{ | f-g | }{2}\\
                    \min(f,g)&=\frac{ f+g }{2}-\frac{ | f-g | }{2},
                \end{align}
            \end{subequations}
            if suffit de montrer que si \( f\in\bar A\) alors \( | f |\in \bar A\). Si \( f\) est nulle, c'est évident; supposons que \( f\neq 0\) et posons \( M=\| f \|_{\infty}\neq 0\). Pour tout \( x\in X\) nous avons
            \begin{equation}
                \frac{ f(x)^2 }{ M^2 }\in \mathopen[ 0 , 1 \mathclose].
            \end{equation}
            Nous considérons alors la suite
            \begin{equation}
                h_n=P_n\circ\frac{ f^2 }{ M^2 }
            \end{equation}
            où \( P_n\) est une suite de polynômes convergent uniformément vers la racine carré (voir lemme \ref{LemYdYLXb}). Le lemme \ref{LemUuxcqY} nous assure que \( h_n\) converge uniformément vers \( \frac{ | f | }{ M }\) dans \( C(X,\eR)\). Étant donné que \( \bar A\) est également une algèbre, \( h_n\) est dans \( \bar A\) pour tout \( n\) et la limite s'y trouve également (pour rappel, la fermeture \( \bar A\) est celle de la topologie de la convergence uniforme).

        \item[Troisième étape] Soit \( \epsilon>0\), \( f\in C(X,\eR)\) et \( x\in X\). Il existe une fonction \( g_x\in \bar A\) telle que 
            \begin{subequations}
                \begin{numcases}{}
                    g_x(x)=f(x)\\
                    g_x(y)\leq f(y)+\epsilon
                \end{numcases}
            \end{subequations}
            pour tout \( y\in X\).

            Soit \( z\in X\setminus\{ x \}\) et une fonction \( h_z\) telle que \( h_z(x)=f(x)\) et \( h_z(z)=f(z)\). Une telle fonction existe par une des étapes précédentes. Étant donné que \( f\) et \( h_z\) sont continues, il existe un voisinage ouvert \( V_z\) de \( z\) sur lequel
            \begin{equation}
                h_z(y)\leq f(y)+\epsilon
            \end{equation}
            pour tout \( y\in V_z\). Nous pouvons sélectionner un nombre fini de points \( z_1,\ldots, z_n\) tels que les ouverts \( V_{z_1},\ldots, V_{z_n}\) recouvrent \( X\) (parce que \( X\) est compact, de tout recouvrement par des ouverts, nous extrayons un sous recouvrement fini.). Nous posons 
            \begin{equation}
                g_x=\min(h_{z_1},\ldots, h_{z_n})\in \bar A.
            \end{equation}
            Si \( y\in X\), nous sélectionnons le \( i\) tel que \( h_{z_i}(y)\leq f(y)+\epsilon\) et nous avons
            \begin{equation}
                g_x(y)\leq h_{z_i}(y)\leq f(y)+\epsilon.
            \end{equation}
            
        \item[Étape \wikipedia{fr}{Final_Doom}{finale}] Soit \( \epsilon>0\) et \( f\in C(X,\eR)\). Pour chaque \( x\in X\) nous considérons une fonction \( g_x\in \bar A\) telle que
            \begin{subequations}
                \begin{numcases}{}
                    g_x(x)=f(x)\\
                    g_x(y)\leq f(y)+\epsilon
                \end{numcases}
            \end{subequations}
            pour tout \( y\in X\). Les fonctions \( f\) et \( g_x\) sont continues, donc il existe un voisinage ouvert \( W_x\) de \( x\) sur lequel
            \begin{equation}
                g_x(y)\geq f(y)-\epsilon.
            \end{equation}
            De ces \( W_x\) nous extrayons un sous recouvrement fini de \( X\) : \( W_{x_1},\ldots, W_{x_m}\) et nous posons
            \begin{equation}
                \varphi=\max(g_{x_1},\ldots, g_{x_n})\in \bar A.
            \end{equation}
            Si \( y\in X\), il existe un \( i\) tel que 
            \begin{equation}
                \varphi(y)\geq g_{x_i}(y)\geq f(y)-\epsilon.
            \end{equation}
            La première inégalité est le fait que \( \varphi\) est le maximum des \( g_{x_k}\), et la seconde est le choix de \( i\). Donc pour tout \( y\in X\) nous avons
            \begin{equation}        \label{EqJMxHaF}
                f(y)-\epsilon\leq \varphi(y)\leq f(y)+\epsilon.
            \end{equation}
            La première inégalité est ce que l'on vient de faire. La seconde est le fait que pour tout \( i\) nous ayons \( g_{x_i}(y)\leq f(y)+\epsilon\); le fait que \( \varphi\) soit le maximum sur les \( i\) ne change pas l'inégalité.

            Le fait que les inégalités \eqref{EqJMxHaF} soient vraies pour tout \( y\in X\) signifie que \( \| \varphi-f \|_{\infty}\leq \epsilon\), et donc que \( f\in \bar{\bar A}=\bar A\).
    \end{description}

    Tout cela prouve que \( C(X,\eR)\subset \bar A\). L'inclusion inverse est le fait que \( C(X,\eR)\) est fermé pour la norme \( \| . \|_{\infty}\), étant donné qu'une limite uniforme de fonctions continues est continue.

\end{proof}

Le théorème suivant est un des énoncés les plus classiques de Stone-Weierstrass. Il découle évidement du théorème général \ref{ThoWmAzSMF} (encore qu'il faut alors bien comprendre qu'il faut traiter la fonction \( x\mapsto \sqrt{x}\) séparément). Il en existe cependant une preuve indépendante.
%TODO : trouver cette preuve indépendante.
\begin{theorem}     \label{ThoGddfas}   \index{théorème!Stone-Weierstrass}
    Soit \( f\), une fonction continue de l'intervalle compact \( \mathopen[ a , b \mathclose]\) à valeurs dans \( \eR\). Alors pour tout \( \epsilon>0\), il existe un polynôme \( P\) tel que \( \| P-f \|_{\infty}<\epsilon\).

    Autrement dit, les polynômes sont denses dans \( C\mathopen[ a , b \mathclose]\) pour la norme uniforme.
\end{theorem}

\begin{corollary}   \label{CorRSczQD}
    Si \( X\subset \eR\) est compact et de mesure finie\footnote{Dans \( \eR\) cette hypothèse est évidemment superflue par rapport à l'hypothèse de compacité; mais ça suggère des généralisations \ldots}, alors l'ensemble des polynômes est denses dans \( \big( C(X,\eR),\| . \|_2 \big)\).
\end{corollary}

\begin{proof}
    Si \( f\) est une fonction dans \( C(X,\eR)\) et si \( \epsilon\geq 0\) est donné alors nous pouvons considérer un polynôme \( P\) tel que \( \| f-P \|_{\infty}\leq \epsilon\). Dans ce cas nous avons
    \begin{equation}
        \| f-P \|_2^2=\int_X| f(x)-P(x) |^2dx\leq \int_X\epsilon^2dx=\epsilon^2\mu(X)
    \end{equation}
    où \( \mu(X)\) est la mesure de \( X\) (finie par hypothèse).
\end{proof}

%---------------------------------------------------------------------------------------------------------------------------
\subsection{Théorème taubérien de Hardi-Littlewood}
%---------------------------------------------------------------------------------------------------------------------------

Un théorème \defe{taubérien}{taubérien}\index{théorème!taubérien} est un théorème qui compare les modes de convergence d'une série.

\begin{lemma}
    Si \( f\) et \( g\) sont des fonctions continues, alors \( s(x)=\max\{ f(x),g(x) \}\) est également une fonction continue.
\end{lemma}

\begin{proof}
    Soit \( x_0\) et prouvons que \( s\) est continue en \( x_0\). Si \( f(x_0)\neq g(x_0)\) (supposons \( f(x_0)>g(x_0)\) pour fixer les idées), alors nous avons un voisinage de \( x_0\) sur lequel \( f>g\) et alors \( s=f\) sur ce voisinage et la continuité provient de celle de \( f\).

    Si au contraire \( f(x_0)=g(x_0)=s(x_0)\) alors si \( (a_n)\) est une suite tendant vers \( x_0\), nous prenons \( N\) tel que \( \big| f(a_n)-f(x_0) \big|\leq \epsilon\) pour tout \( n>N\) et \( M\) tel que \( \big| g(a_n)-g(x_0) \big|\leq \epsilon\) pour tout \( n> M\). Alors pour tout \( n>\max\{ N,M \}\) nous avons
    \begin{equation}
        \big| s(a_n)-s(x_0) \big|\leq \epsilon,
    \end{equation}
    d'où la continuité de \( s\) en \( x_0\).
\end{proof}

La proposition suivante dit que si une fonction connaît un saut, alors on peut le lisser par une fonction continue.
\begin{proposition} \label{PropTIeYVw}
    Soit \( f\) continue sur \( \mathopen[ a , x_0 [\) et sur \( \mathopen[ x_0 , b \mathclose]\) avec \( f(x_0^-)<f(x_0)\). En particulier nous supposons que \( f(x^-)\) existe et est finie. Alors pour tout \( \epsilon>0\), il existe une fonction continue \( s\) telle que sur \( \mathopen[ a , b \mathclose]\) on ait \( s\leq f\) et
    \begin{equation}
        \int_a^bs(x)-f(x)\,dx\leq \epsilon.
    \end{equation}
\end{proposition}

\begin{proof}
    Nous notons \( A\) la taille du saut :
    \begin{equation}
        A=f(x_0)-f(x_0^-).
    \end{equation}
    Quitte à changer \( a\) et \( b\), nous pouvons supposer que
    \begin{equation}
        f(x)<f(x_0)+\frac{ A }{ 3 }
    \end{equation}
    pour \( x\in \mathopen[ a , x_0 [\) et 
    \begin{equation}
        f>f(x_0)+\frac{ 2A }{ 3 }
    \end{equation}
    pour \( x\in \mathopen[ x_0 , b \mathclose]\). C'est le théorème des valeurs intermédiaires qui nous permet de faire ce choix.

    Soit \( m(x)\) la droite qui joint le point \( \big( x_0-\epsilon, f(x_0-\epsilon) \big)\) au point \( \big( x_0,f(x_0^+) \big)\). Nous posons
    \begin{equation}
        s(x)=\begin{cases}
            f(x)    &   \text{si \( x<x_0-\epsilon\)}\\
            \max\{ m(x),f(x) \}    &   \text{si \( x_0-\epsilon\leq x\leq x_0\)}\\
            f(x)    &    \text{si $x>x_0$}.
        \end{cases}
    \end{equation}
    En vertu des différents choix effectués, c'est une fonction continue. En effet
    \begin{equation}
        s(x_0-\epsilon)=\max\{ f(x_0-\epsilon),f(x_0,\epsilon) \}=f(x_0-\epsilon)
    \end{equation}
    et 
    \begin{equation}
        s(x_0)=\max\{ m(x_0),f(x_0^+) \}=f(x_0^+)
    \end{equation}
    parce que \( m(x_0)=f(x_0^+)\). En ce qui concerne l'intégrale, si nous posons
    \begin{equation}
        M=\sup_{x,y\in \mathopen[ a , b \mathclose]}| f(x)-f(y) |,
    \end{equation}
    nous avons
    \begin{equation}
        \int_a^bs-f=\int_{x_0-\epsilon}^{x_0}s-f\leq \epsilon M.
    \end{equation}
\end{proof}

\begin{lemma}\label{LemauxrKN}
    Pour tout polynôme \( P\), nous avons la formule
    \begin{equation}
        \lim_{x\to 1^-} (1-x)\sum_{n=0}^{\infty}x^nP(x^n)=\int_0^1P(x)dx.
    \end{equation}
\end{lemma}

\begin{proof}
    D'abord pour \( P=1\), la formule se réduit à la série harmonique connue. Ensuite nous prouvons la formule pour le polynôme \( P=X^k\) et la linéarité fera le reste pour les autres polynômes. Nous avons
    \begin{equation}
        (1-x)\sum_nx^nx^{kn}=(1-x)\sum_n(x^{1+k})^n=\frac{ 1-x }{ 1-x^{1+k} }=\frac{1}{ 1+x+\ldots+x^k }.
    \end{equation}
    Donc
    \begin{equation}
        \lim_{x\to 1^-} (1-x)\sum_nx^nP(x^n)=\frac{1}{ 1+k }.
    \end{equation}
    Par ailleurs, c'est vite vu que
    \begin{equation}
        \int_0^1 x^kdx=\frac{1}{ k+1 }.
    \end{equation}
\end{proof}

\begin{theorem}[Hardy-Littlewood\cite{ytMOpe}]\index{théorème!Hardy-Littlewood}\index{Hardy-Littlewood (théorème)}      \label{ThoPdDxgP}
    Soit \( (a_n)\) une suite réelle telle que
    \begin{enumerate}
        \item
            \( \frac{ a_n }{ n }\) tends vers une constante,
        \item
            \( F(x)=\sum_{n=0}^{\infty}a_nx^n\) a un rayon de convergence \( \geq 1\),
        \item
            \( \lim_{x\to 1^-} F(x)=l\).
    \end{enumerate}
    Alors \( \sum_{n=0}^{\infty}a_n=l\).
\end{theorem}
\index{convergence!suite numérique}
\index{série!nombres}
\index{série!fonctions}
\index{limite!inversion}
\index{approximation!par polynômes}

\begin{proof}
    Quitte à prendre la suite \( b_0=a_0-l\) et \( b_n=a_n\), on peut supposer \( l=0\).

    Soit \( \Gamma\) l'ensemble des fonctions
    \begin{equation}
         \gamma\colon \mathopen[ 0 , 1 \mathclose]\to \eR 
    \end{equation}
    telles que 
    \begin{enumerate}
        \item
            $\sum_{n=0}^{\infty}a_n\gamma(x^n)$ converge pour \( 0\leq x<1\),
        \item
            \( \lim_{x\to 1^-} \sum_{n\geq 0}a_n\gamma(^n)=0\).
    \end{enumerate}
    Ce \( \Gamma\) est un espace vectoriel.
    \begin{subproof}
    \item[Les polynômes sont dans \( \Gamma\)]
        Soit \( \gamma(t)=t^s\). Pour \( 0\leq x<1\) nous avons
        \begin{equation}
            \sum_{n=0}^{\infty}a_n\gamma(x^n)=\sum_{n=0}^{\infty}a_nx^{ns}<\sum_{n=0}^{\infty}a_nx^n.
        \end{equation}
        Donc la condition de convergence est vérifiée. En ce qui concerne la limite,
        \begin{equation}
            \lim_{x\to 1^-} \sum_{n=0}^{\infty}a_nx^{ns}=\lim_{x\to 1^-} F(x^s)=0
        \end{equation}
        parce que par hypothèse, \( \lim_{x\to 1^-} F(x)=0\).

    \item[Définition de la fonction qui va donner la réponse]
        Nous considérons la fonction \( g=\mtu_{\mathopen[ \frac{ 1 }{2} , 1 \mathclose]}\), c'est à dire
        \begin{equation}
            g(t)=\begin{cases}
                0    &   \text{si \( 0\leq t<1/2\)}\\
                1    &    \text{si \( 1/2\leq t\leq 1\)}.
            \end{cases}
        \end{equation}
        Nous montrons que si \( g\in \gamma\), alors le théorème est terminé. Si \( 0\leq x\leq 1\), on a \( 0\leq x^n<1/2\) dès que
        \begin{equation}
            n>-\frac{ \ln(2) }{ \ln(x) }
        \end{equation}
        avec une note comme quoi \( \ln(x)<0\), donc la fraction est positive. Nous désignons par \( N_x\) la partie entière de ce \( n\) adapté à \( x\). L'idée est que la fonction  \( g(x^n)\) est la fonction indicatrice de \(0 \leq n\leq N_x\), et donc
        \begin{equation}
            \sum_{n\geq 0}a_ng(x^n)=\sum_{n=0}^{N_x}a_n.
        \end{equation}
        Mais si \( x\to 1^-\), alors \( N_x\to \infty\), donc
        \begin{equation}
            \lim_{N\to \infty} \sum_{n=0}^Na_n=\lim_{x\to 1^-} \sum_{n=0}^{N_x}a_n=\lim_{x\to 1^-} \sum_{n\in \eN}a_ng(x^n),
        \end{equation}
        et cela fait zéro si \( g\in \Gamma\).
        
    \item[Approximation de \( g\) par des polynômes]

        Nous considérons la fonction
        \begin{equation}
            h(t)=\frac{ g(t)-t }{ t(1-1) }=\begin{cases}
                \frac{1}{ t-1 }    &   \text{si \( t\in \mathopen[ 0 , 1/2 [\)}\\
                \frac{1}{ t }    &    \text{si \( t\in \mathopen[ 1/2 , 1 \mathclose]\)}.
            \end{cases}
        \end{equation}
        La seconde égalité est au sens du prolongement par continuité. La fonction \( h\) est une fonction non continue qui fait un saut de \( -2\) à \( 2\) en \( x=1/2\). En vertu de la proposition \ref{PropTIeYVw} (un peu adaptée), nous pouvons considérer deux fonctions continues \( s_1\) et \( s_2\) telles que
        \begin{equation}
            s_1\leq h\leq s_2
        \end{equation}
        et
        \begin{equation}
            \int_{0}^1s_2-s_1\leq \epsilon.
        \end{equation}
        Notons que l'inégalité \( s_1\leq s_2\) doit être stricte sur au moins un petit intervalle autour de \( x=1/2\). Soient \( P_1\) et \( P_2\), deux polynômes tels que \( \| P_1-s_1 \|_{\infty}\leq \epsilon\) et \( \| P_2-s_2 \|_{\infty}\leq \epsilon\) (ici la norme supremum est prise sur \( \mathopen[ 0 , 1 \mathclose]\)). C'est le théorème de Stone-Weierstrass (\ref{ThoGddfas}) qui nous permet de le faire.

        Nous posons aussi\footnote{À ce niveau, je crois qu'il y a une faute de frappe dans \cite{ytMOpe}.}
        \begin{subequations}
            \begin{align}
                Q_1=P_1+\epsilon\\
                Q_2=P_2-\epsilon.
            \end{align}
        \end{subequations}
        Nous avons
        \begin{equation}
            \int_0^1Q_1-Q_2\leq\int_0^1 Q_1-P_1+P_1-P_2+P_2-Q_2.
        \end{equation}
        Pour majorer cela, d'abord \( Q_1-P_1=P_2-Q2=\epsilon\), ensuite,
        \begin{equation}
            P_1-P_2=P_1-s_1+s_1-s_2+s_2-P_2
        \end{equation}
        dans lequel nous avons \( P_1-s_1\leq \epsilon\), \( s_2-P_2\leq \epsilon\) et \( \int_0^1s_1-s_2\leq\epsilon\). Au final, nous posons \( q=Q_2-Q_1\) et nous avons
        \begin{equation}
            \int_0^1q\leq 5\epsilon.
        \end{equation}
        Enfin nous posons aussi
        \begin{equation}
            R_i(x)=x+x(1-x)Q_i.
        \end{equation}
        Ces polynômes vérifient \( R_i(0)=0\), \( R_i(1)=1\) et
        \begin{equation}
            R_1\leq g\leq R_2
        \end{equation}
        parce que
        \begin{equation}
            Q_1\leq P_1\leq h\leq  P_2\leq Q_2
        \end{equation}
        et
        \begin{equation}
            t+t(1-t)Q_1\leq \underbrace{t+t(1-t)h(t)}_{g(t)}\leq t+t(1-t)Q_2.
        \end{equation}
        
    \item[Preuve que \( g\) est dans \( \Gamma\)]

        D'abord si \( 0\leq x<1\), \( x^N<\frac{ 1 }{2}\) pour un certain \( N\), et alors \( g(x^N)=0\). Du coup la série
        \begin{equation}
            \sum_{n=0}^{\infty}a_ng(x^n)=\sum_{n=0}^{N}a_n
        \end{equation}
        est une somme finie qui converge donc.

        D'autre part nous prenons \( M\) tel que \( | a_n |<\frac{ M }{ n }\) pour tout \( n\). Nous majorons \( \sum_{n \in \eN}a_ng(x^n)\) en utilisant \( R_1\). Mais vu que \( R_1\) est un polynôme, nous pouvons dire que \( | \sum_{n=0}^{\infty}a_nR_1(x^n) |\leq \epsilon\) en prenant \( x\in\mathopen[ \lambda , 1 [\) et \( \lambda\) assez grand. Nous avons :
        \begin{subequations}
            \begin{align}
                \left| \sum_{n=0}^{\infty}a_ng(x^n) \right| &\leq\left| \sum_{n=0}^{\infty}a_ng(x^n)-\sum_{n=0}^{\infty}a_nR_1(x^n) \right| +\underbrace{\left| \sum_{n=0}^{\infty}a_nR_1(x^n) \right|}_{\leq \epsilon} \\
                &\leq \epsilon+\sum_{n=0}^{\infty}| a_n |(g-R_1)(x^n)\\
                &\leq \epsilon+\sum_{n=0}^{\infty}| a_n |(R_2-R_1)(x^n)\\
                &\leq \epsilon+M\sum_{n=0}^{\infty}\frac{ x^n(1-x^n) }{ n }(Q_2-Q_1)(x^n)   &R_2-R_1=x(1-x)(Q_2-Q_1)\\
                &=\epsilon+M\sum_{n=0}^{\infty}\frac{ x^n(1-x^n) }{ n }q(x^n)\\
                &\leq \epsilon+M(1-x)\sum_nx^nq(x^n)   \label{subeqtZXDvu} 
            \end{align}
        \end{subequations}
        où la ligne \eqref{subeqtZXDvu} provient d'une majoration sauvage de \( 1/n\) par \( 1\) et de \( 1-x^n\) par \( 1-x\). Par le lemme \ref{LemauxrKN}, nous avons alors
        \begin{equation}
            \lim_{x\to 1^-} | \sum_na_ng(x^n) |\leq \epsilon+M\int_0^1q\leq 6\epsilon.
        \end{equation}
    \end{subproof}
\end{proof}

%---------------------------------------------------------------------------------------------------------------------------
\subsection{Théorème de Müntz}
%---------------------------------------------------------------------------------------------------------------------------

\begin{theorem}[Théorème de Müntz\cite{jqZSyG,oYGash}]  \label{ThoAEYDdHp}
    Soit \( C=C_0\big( \mathopen[ 0 , 1 \mathclose] \big)\), l'espace des fonctions continues sur \( \mathopen[ 0 , 1 \mathclose]\) muni de la norme \( \| . \|_{\infty}\) ou \( \| . \|_2\) et une suite \( (\alpha_n)\) strictement croissante de nombres positifs. Nous notons \( \phi_{\lambda}\) la fonction \( x\mapsto x^{\lambda}\).

    Alors l'espace \( \Span\{ \phi_{\alpha_n} \}\) est dense dans \( C\) si et seulement si \( \sum_{n=1}^{\infty}\frac{1}{ \alpha_n }\) diverge.
\end{theorem}

Nous prouvons le théorème pour la norme \( \| . \|_2\).
\begin{proof}
    Soit \( m\in \eR^+\); nous notons \( \Delta_N(m)\) la distance entre \( \phi_m\) et \( \Span\{ \phi_{\alpha_1},\ldots, \phi_{\alpha_N} \}\). Cette distance peut être évaluée avec le déterminant de Gram\index{déterminant!Gram} (proposition \ref{PropMsZhIK})
    \begin{equation}
        \Delta_N(m)^2=\frac{ G(\phi_m,\phi_{\alpha_1},\ldots, \phi_{\alpha_N}) }{ G(\phi_{\alpha_1},\ldots, \phi_{\alpha_N}) }.
    \end{equation}
    Pour calculer cela nous avons besoin des produits scalaires\footnote{C'est ici qu'on se particularise à la norme \( \| . \|_2\).}
    \begin{equation}
        \langle \phi_a, \phi_b\rangle =\int_0^1 x^{a+b}dx=\frac{1}{ a+b+1 }.
    \end{equation}
    Donc nous avons à calculer le déterminant
    \begin{equation}
        G(\phi_m,\phi_{\alpha_1},\ldots, \phi_{\alpha_N})=\det\begin{pmatrix}
            \frac{1}{ 2m+1 }   &   \frac{1}{ m+\alpha_1+1 }    &   \cdots    &   \frac{1}{ m+\alpha_N+1 }    \\
            \frac{1}{ m+\alpha_1+1 }   &   \frac{1}{ 2\alpha_1+1 }    &   \cdots    &   \frac{1}{ \alpha_1+\alpha_N+1 }    \\
             \vdots   &   \vdots    &   \ddots    &   \vdots    \\ 
             \frac{1}{ m+\alpha_N+1 }   &   \frac{1}{ \alpha_1+\alpha_N+1 }    &   \cdots    &   \frac{1}{ 2\alpha_N+1 }     
         \end{pmatrix}
    \end{equation}
    dans lequel nous reconnaissons un déterminant de Cauchy (proposition \ref{ProptoDYKA})\index{déterminant!Cauchy} en posant, dans \( \frac{1}{ \alpha_i+\alpha_j+1 }\), \( a_i=\alpha_i\) et \( b_j=\alpha_j+1\). Au passage nous nommons \( \alpha_0=m\) pour se simplifier les notations. Avec ces conventions, étant donné que \( b_j-b_i=a_j-a_i\), les facteurs des deux produits
    \begin{equation}
        \prod_{i<j}(a_j-a_i)\prod_{i<j}(b_j-b_i)
    \end{equation}
    sont les mêmes et donc le numérateur de \( G(\phi_m,\phi_{\alpha_1},\ldots, \phi_{\alpha_N})\) est donné par
    \begin{equation}
        \prod_{i<j}(\alpha_i-\alpha_j)^2\prod_i(\alpha_i-m)^2.
    \end{equation}
    En ce qui concerne le dénominateur, il faut prendre tous les couples \( (i,j)\) avec \( i\) et \( j\) éventuellement égaux à zéro. Nous décomposant cela en trois paquets. Le premier est \( (0,0)\); le second est \( (0,i)\) (chaque couple arrive en fait deux fois parce qu'il y a aussi \( (i,0)\)); et le troisième sont les \( i,j\) tous deux différents de zéro :
    \begin{equation}
        (2m+1)\prod_{ij}(\alpha_i+\alpha_j+1)\prod_i(\alpha_i+m+1)^2.
    \end{equation}
    Notons que dans le produit central, le carré est contenu dans le fait qu'on écrit \( \prod_{ij}\) et non \( \prod_{i<j}\). Nous avons donc
    \begin{equation}
        G(\phi_m,\phi_{\alpha_1},\ldots, \phi_{\alpha_N})=\frac{ \prod_{i<j}(\alpha_i-\alpha_j)^2\prod_i(\alpha_i-m)^2 }{ (2m+1)\prod_{ij}(\alpha_i+\alpha_j+1)\prod_i(\alpha_i+m+1)^2 }.
    \end{equation}
    
    Le calcul de \( G(\phi_{\alpha_1},\ldots, \phi_{\alpha_N})\) est plus simple\footnote{Je crois qu'il y a une faute de frappe dans le dénominateur de \cite{jqZSyG}.} :
    \begin{equation}
        G(\phi_{\alpha_1},\ldots, \phi_{\alpha_N})=\frac{ \prod_{i<j}(\alpha_i-\alpha_j)^2 }{ \prod_{ij}(\alpha_i+\alpha_j+1) }.    
    \end{equation}
    En divisant l'un par l'autre il ne reste que les facteurs comprenant \( m\) et en prenant la racine carré,
    \begin{equation}    \label{EqANiuNB}
        \Delta_N(m)=\frac{1}{ \sqrt{2m+1} }\prod_{i=1}^N\left| \frac{ \alpha_i-m }{ \alpha_i+m+1 } \right| .
    \end{equation}
    
    Nous passons maintenant à la preuve proprement dite. Supposons que \( V=\Span\{ \phi_{\alpha_i},i\in \eN \}\) est dense; alors nous avons en particulier que \( \phi_m\) peut être arbitrairement approché par les \( \phi_{\alpha_i}\), c'est à dire que
    \begin{equation}
        \lim_{N\to \infty} \Delta_N(m)=0
    \end{equation}
    Nous posons 
    \begin{equation}
        u_n=\ln\left( \frac{ \alpha_n-m }{ \alpha_n+m+1 } \right)
    \end{equation}
    et nous prouvons que la série \( \sum_nu_n\) diverge. En effet nous nous souvenons de la formule \( \ln(ab)=\ln(a)+\ln(b)\), de telle sorte que la \( N\)ième somme partielle de \( \sum_nu_n\) est
    \begin{equation}
        \ln\left( \frac{ \alpha_1-m }{ \alpha_1+m+1 }\cdot\ldots\cdot \frac{ \alpha_N-m }{ \alpha_N+m+1 } \right)=\ln\left( \sqrt{2m+1}\Delta_N(m) \right),
    \end{equation}
    qui tends vers \( -\infty\) lorsque \( N\to \infty\).

    Si la suite \( (\alpha_n)\) est majorée et plus généralement si nous n'avons pas \( \alpha_n\to \infty\), alors évidemment la série \( \sum_n\frac{1}{ \alpha_n }\) diverge. Nous supposons donc que \( \lim_{n\to \infty} \alpha_n=\infty\). Nous avons aussi\footnote{Je crois qu'il y a une faute de signe dans la dernière expression de \cite{oYGash}.}
    \begin{equation}
        u_n=\ln\left( \frac{ \alpha_n-m }{ \alpha_n+m+1 } \right)=\ln\left( 1-\frac{ 2m+1 }{ \alpha_n+m+1 } \right)\sim-\frac{ 2m+1 }{ \alpha_n }.
    \end{equation}
    Une justification est donné à l'équation \eqref{EqGICpOX}. Ce que nous avons surtout est
    \begin{equation}
        \sum_n u_n\sim -(2m+1)\sum_n\frac{1}{ \alpha_n }.
    \end{equation}
    Étant donné que la série de gauche diverge, celle de droite diverge\footnote{Nous utilisons le fait que si \( u_n\sum v_n\) en tant que suites et si \( \sum_nu_n\) diverge, alors \( \sum_nv_n\) diverge.}.

    Nous faisons maintenant le sens opposé : nous supposons que la série \( \sum_n1/\alpha_n\) diverge et nous nous posons
    \begin{equation}
        V=\Span\{ \phi_{\alpha_n}\tq n\in \eN \}.
    \end{equation}
    Si \( \alpha_n\to \infty\), alors il suffit de prouver que \( \phi_m\in \bar V\) pour tout \( m\) parce qu'un corollaire du théorème de Stone-Weierstrass \ref{CorRSczQD} montre que \( \Span\{ \phi_k\tq k\in \eN \}\) est dense dans \( C\) pour la norme \( \| . \|_2\). Nous avons :
    \begin{equation}
        u_n\sim\frac{ 2m+1 }{ \alpha }\to 0
    \end{equation}
    et alors \( \Delta_N(m)\to 0\). Dans ce cas nous avons immédiatement \( \phi_m\in \bar V\).

    Si par contre \( \alpha_n\) ne tend pas vers l'infini, nous repartons de l'expression \eqref{EqANiuNB}, nous posons \( \alpha=\sup_i\alpha_i\) et nous calculons :
    \begin{subequations}
        \begin{align}
            \sqrt{2m+1}\Delta_N(m)&=\prod_{i=1}^N\frac{ | \alpha_i-m | }{ \alpha_i+m+1 }\\
            &\leq \prod_{i=1}^N\frac{ \alpha_i+m }{ \alpha_i+m+1 }\\
            &=\prod_{i=1}^N\left( 1-\frac{ 1 }{ \alpha_i+m+1 } \right)\\
            &\leq \prod_{i=1}^N\left( 1-\frac{1}{ \alpha+m+1 } \right)\\
            &=\left( 1-\frac{1}{ \alpha+m+1 } \right)^N.
        \end{align}
    \end{subequations}
    Cette dernière expression tend vers \( 0\) lorsque \( N\to \infty\).
\end{proof}

\begin{example}
    Nous savons depuis le théorème \ref{ThonfVruT} que la somme des inverses des nombres premiers diverge.
\end{example}
