The main references for this chapter are \cite{Dixmier,Landsman}. In this chapter, all algebras are over $\eC$, or $\eR$ when it is mentioned. Definition and spectral properties of Banach algebras are given in chapter \ref{Sec_SpecBanach}.

\subsection*{Notations is Landsman and Dixmier}
%-----------------------------------------------

There are some differences in notations, definitions and conventions between the book of Dixmier and the lecture notes of Landsman. Here I summarize it. As much as possible, I try to follow Landsman.

In \lref{2.2.2}, the spectral radius is denoted by $r(x)$, while it is $\rho(x)$ in \dixref{B1}. In Landsman, $\rho(x)$ is the resolvent of $x$. \dixref{1.1.5, 1.1.6} define the spectrum separately for the algebras with and without unit. It notes it by $Sp\,x$ and $Sp'\,x$.

The set of the characters of $\cA$ is called the \defe{spectrum}{spectrum} of $\cA$ in Dixmier; it is the $\Delta(\cA)$ in Landsman. The Gelfand transform of $x\in\cA$ is the map $\dpt{\hx}{\Delta(\cA)}{\eC}$, $\hx(\omega):=\omega(x)$. If $\cA$ admits an unit, Dixmier defines
\[
   Sp_{\cA}x:=\{\hx(\omega):\omega\in\Delta(\cA)\}.
\]
Thanks to \leref{2.50}, the two systems of definitions are the same, but there is a problem on the word ``spectrum ''. Here I use it in the sense of Landsman, and I write the structure space by his notation $\Delta(\cA)$.

%+++++++++++++++++++++++++++++++++++++++++++++++++++++++++++++++++++++++++++++++++++++++++++++++++++++++++++++++++++++++++++
\section{Ideal in Banach algebras}

\begin{definition}
An \defe{ideal}{ideal in Banach algebra} in a Banach algebra $\cA$ is a closed vector subspace $\cI\subseteq\cA$ such that $\forall I\in\cI$, $\forall A\in\cA$, $AI\in\cI$ and $IA\in\cI$.

A \defe{left ideal}{left!ideal} has just $AI\in\cI$.

A  \defe{maximum ideal}{maximum ideal} is an ideal $\cI\neq\cA$ for which there exists no strict intermediary ideal between $\cI$ and $\cA$, i.e. no ideal $\tilde{\cI}\neq\cI$  with $\tilde{\cI}\neq\cA$ and $\cI\subset\tilde{\cI}$.
\end{definition}

An ideal is a Banach algebra, and the only ideal in $\cA$ which contains an invertible element is $\cA$ itself.

\begin{proposition} 
If $\cI$ is an ideal in a Banach algebra $\cA$, then the quotient $\cA/\cI$ becomes a Banach algebra with the norm
\begin{equation}  \label{eq:norm_ideal}
\|\tau(A)\|=\inf_{J\in\cI}\|A+J\|
\end{equation}
and the multiplication rule
\begin{equation}   \label{eq:prod_ideal}
\tau(A)\tau(B)=\tau(A)\tau(B).
\end{equation}
 \label{prop:ideal_Banach}
\end{proposition}

\begin{proof}
The fact that it is a Banach space is non trivial and proved in \cite{thomaslassen}. We begin to prove that \eqref{eq:prod_ideal} is a well defined product:
\begin{equation}
   \tau(A+J_1)\tau(A+J_2)=\tau(AB+AJ_2+J_1B+J_1J_2)
                         =\tau(AB)
\end{equation}
because $AJ_2+J_1B+J_1J_2\in\cI$.

Next we have to prove that $\|\tau(A)\tau(B)\|\leq\|\tau(A)\|\|\tau(B)\|$. Remark that
\begin{equation} \label{eq:tauAA}
\|\tau(A)\|\leq\|A\|
\end{equation}
because 
\[
  \|\tau(A)\|=\inf\|A+J\|
             \leq\inf(\|A\|+\|J\|)
             =\|A\|+\inf\|J\|
             =\|A\|.
\]
Then $\forall\varepsilon>0$, there exists a $J\in\cI$ such that 
\begin{equation} \label{eq:tauAeps}
\|\tau(A)\|+\varepsilon\geq\|A+J\|
\end{equation}
It also gives $\|\tau(A)\|=\|\tau(A+J)\|\leq\|A+J\|$. Take $A$, $B\in\cA$ and a $\varepsilon$ for which \eqref{eq:tauAeps} holds for both $A$ and $B$. Then
\begin{equation}
\begin{split}
\|\tau(A)\tau(B)\|&=\|\tau\big( (A+J_1)(A+J_2)\big)\|\\
                  &\leq\|(A+J_A)(B+J_2)\|\\
                  &\leq\|A+J_1\|\|B+J_2\|\\
                  &\leq(\|\tau(A)\|+\varepsilon)(\|\tau(B)\|+\varepsilon),
\end{split}
\end{equation}
which gives the result as $\varepsilon\to 0$.

\end{proof}

If $\cA$ is unital, then $\cA/\cI$ is unital and his unit is given by $\tau(\cun)$. Let us prove that $\|\tau(\cun)\|=1$. From equation \eqref{eq:tauAA}, $\|\tau(\cun)\|\leq\|\cun\|=1$. On the other hand, from equation $\|\tau(A)\tau(B)\|\leq\|\tau(A)\|\|\tau(B)\|$ with $B=\cun$, we find $1\leq\|\tau(\cun)\|$. 

\begin{corollary}
Let $\cA$ and $\cB$ be $C^*$-algebras, $\varphi\colon \cA\to \cB$ a morphism and $\cI$ the kernel of $\varphi$. We consider the canonical decomposition of $\varphi$ into
\begin{equation}
\xymatrix{%
   \cA \ar[r]^{\tau}    & \cA/\cI\ar[r]^{\psi}  & \varphi(\cA)\ar[r]    & \cB.
}
\end{equation}
Then $\cI$ is closed in $\cA$, $\varphi(\cA)$ is closed in $\cB$ and $\psi$ is an isometric isomorphism.
\end{corollary}

\begin{proof}
A morphism of $C^*$-algebras is always continuous, then $\cI=\ker\varphi=\varphi^{-1}(\{ 0 \})$ is closed because $\{ 0 \}$ is closed. The map $\psi\colon \cA/\cI\to \varphi(\cA)$ is injective because when $\psi([A])=\psi([B])$, we have $\varphi(A)=\varphi(B)$ and $A=B+I$ for a certain $I\in\cI$, hence $[A]=[B]$. Proposition \ref{prop:vp_geq} gives $\| \varphi(A) \|\geq\| A \|$ while proposition \ref{prop:vp_leq} gives $\| \varphi(A) \|\leq \| A \|$. So $\varphi$ is isometric.

Hence $\varphi(\cA)$ is complete and therefore closed in $\cB$.
\end{proof}


\section{Commutative Banach algebra}
%+++++++++++++++++++++++++++++++++++

We suppose the Banach algebra $\cA$ to be commutative.

\subsection{Structure space}
%----------------------------

\begin{definition}      \label{DefStructureSpaceDel}
    The \defe{structure space}{structure!space} $\Delta(\cA)$\nomenclature[C]{$\Delta(\cA)$}{structure space if the $C^*$-algebra $\cA$} of a commutative algebra is the set of the nonzero linear maps $\dpt{\omega}{\cA}{\eC}$ such that $\forall A$, $B\in\cA$,
\[
    \omega(AB)=\omega(A)\omega(B).
\]
We say that an element of this space is a \defe{character}{character!of an algebra}, or a \defe{multiplicative}{multiplicative} map of $\cA$.
\end{definition}
\lref{2.3.1}

\begin{proposition}
Let $\cA$ be an unital commutative Banach algebra. Then for any $\omega\in\Delta(\cA)$,
\begin{enumerate}

\item $\omega(\mtu)=1$.
\item the character $\omega$ is bounded (and then continuous from \ref{prop:cont_born}) with norm $\|\omega\|=1$ and for all $A\in\cA$,
\begin{equation} \label{eq:omAleqnA}
  \|\omega(A)\|\leq \|A\|.
\end{equation}
\end{enumerate}
\end{proposition}
\lref{2.3.2}


\begin{proof}
The first claim is obvious because $\omega(A)=\omega(\mtu A)=\omega(\mtu)\omega(A)$.  For the second one, we know from lemma \ref{lem:cv_Ak} that $(A-z)$ is invertible when $|z|>\|A\|$. By 
linearity,
\[
\omega(A-z)=\omega(A)-z\neq 0
\]
because $\omega$ in a homomorphism. Now remark that $A-z$ is invertible implies $|\omega(A)|\neq |z|$. Indeed let us suppose the opposite, then there exists a $\alpha\in\eR$ such that $\omega(A)=e^{i\alpha}z$, but $|e^{i\alpha}z|=|z|$. Conclusion: if $|z|>\|A\|$, then $|\omega(A)|\neq|z|$. This immediately yields $|\omega(A)|\leq\|A\|$.

From there, it is clear that $\|\omega\|=1$ because the norm is the supremum of $|\omega(A)|$ with $\|A\|=1$. Since $\omega(\mtu)=1$, $\|\omega\|\geq 1$, but what we just showed implies $\|\omega\|\leq 1$.

\end{proof}

\begin{theorem}
Let $\cA$ be an unital commutative Banach algebra. Then we have a bijection between $\Delta(\cA)$ and the set of maximal ideals in $\cA$. More precisely,

\begin{enumerate}
\item $\ker(\omega)$ is an ideal,                   \label{enuei}
\item $\omega_1=\omega_2$ if and only if $\ker\omega_1=\ker\omega_2$,   \label{enueii}  
\item each maximal ideal is the kernel of an element in $\Delta(\cA)$.  \label{enueiii}
\end{enumerate}\lref{2.3.3}\label{tho:ideal_kernel}
\end{theorem}


\begin{proof}
\ref{enuei} Since $\omega$ is continuous, the set $\ker(\omega)$ is closed. It is also clear that is $Z\in\ker(\omega)$, then $AZ\in\ker(\omega)$ for all $Z\in\cA$ because $\omega$ is multiplicative. Then $\ker(\omega)$ is an ideal. In order to see that it is a maximal ideal, remark that $\omega(X)=0$ is a linear equation which describe a vector subspace of $\cA$ of codimension\label{pg_codimun} $1$.

\ref{enueii} In any vector space, $\ker{\omega_1}=\ker{\omega_2}$ implies that $\omega_1$ and $\omega_2$ are multiples each others. In the case of $\Delta(\cA)$, this in turn implies the equality.

\ref{enueiii} Let $\cI\neq\cA$ be a maximal ideal and $B\neq 0$ outside $\cI$. Consider
\[
\cI_B=\{BA+J\tq A\in\cA\textrm{ and }J\in\cI\}.
\]
By construction it is a left-ideal and by commutativity of $\cA$, it is an ideal. We have $\cI\subsetneq\cI_B$. Since $\cI$ is maximal, the conclusion is $\cI_B=\cA$. In particular $\cun=BA+J$ for a suitable choice of $A\in\cA$ and $J\in\cI$. For these,
\[
  \tau(\cun)=\tau(BA)=\tau(B)\tau(A),
\]
but $B$ is arbitrary. Then any element of $\cA/\cI$ is invertible and the Gelfand-Mazur theorem (corollary \ref{cor:GelfandMazur}) concludes $\cA/\cI\simeq\eC$. Let $\dpt{\psi}{\cA/\cI}{\eC}$ be the isomorphism. We consider
        \begin{equation}
        \begin{aligned}
            \omega \colon \cA &\to \eC\
            A&\mapsto \psi(\tau(A)).
        \end{aligned}
    \end{equation}  
It is clearly linear (because $\psi$ and $\tau$ are) and $\omega(A)\omega(B)=\omega(AB)$. Furthermore $\omega(B)\neq 0$ and $\omega(\cun)=1$ are two good reasons to conclude that $\omega\neq 0$. Then $\omega\in\Delta(\cA)$. It remains to be proved that $\cI=\ker\omega$. First, $\cI=\ker\tau$, then $\cI\subseteq\ker\omega$. But when $B\notin\cI$, we have $\omega(B)\neq 0$, then $\cI=\ker\omega$. This finish the proof.
\end{proof}


\begin{theorem}[Banach-Alaoglu] 
If $X$ is a closed normed vector space, then the unit closed ball in the dual $X^*$ is compact for the $x^*$-topology.
 \label{tho:Banach_Alaoglu}
\end{theorem}

\begin{proposition}
When $\cA$ is an unital commutative Banach algebra, the space $\Delta(\cA)$ is compact and Hausdorff for the Gelfand topology.
\end{proposition} \label{prop:DcA_comp_Hauss}\lref{2.3.4}

\begin{proof}
We first prove that $\Delta(\cA)$ is closed by showing that it contains all limits of converging sequences\footnote{It is no related to \emph{complete} spaces in which any Cauchy sequence converge}. Let us take a sequence $\omega_n\to\omega$ with $\omega_n\in\Delta(\cA)$. We will show that $\omega\in\Delta(\cA)$:
\[ 
| \omega(AB)-\omega(A)\omega(B) | \leq| \omega(AB)-\omega_n(AB) |+| \omega_n(A)\omega_n(B)-\omega(A)\omega(B) |,
\]
but 
\[ 
 \begin{split}
\omega_n(A)\omega_n(B)-\omega(A)\omega(B)&=[\omega_n(A)-\omega(A)]\omega_n(B)+\omega(A)[\omega_n(B)-\omega(B)]\\
        &\leq | \omega_n(A)-\omega(A) |\| B \|+\| A \| \omega_n(B)-\omega(B) |
\end{split} 
\]
because $\omega_n(B)\leq \| B \|$. Taking the limit $n\to\infty$, we find
\[ 
 \begin{split}
| \omega(AB)-\omega(A)\omega(B) |&\leq| \omega(AB)-\omega_n(AB) |\\
        &\quad+| \omega_n(A)-\omega(A) |\| B \|\\
        &\quad+\| A \| |\omega_n(B)-\omega(B) |\to 0.
\end{split} 
\]
This proves that $\omega\in\Delta(\cA)$ and therefore that $\Delta(\cA)$ is closed. Since $\| \omega \|=1$ for all $\omega$, we have $\Delta(\cA)\subset\cA_1^*$, the unit ball in $\cA^*$. Theorem \ref{tho:Banach_Alaoglu} claims that $\cA_1^*$ is compact in the Gelfand topology. So $\Delta(\cA)$ is closed in a compact. This makes $\Delta(\cA)$ compact by lemma \ref{lem:ferme_compact}.

Now, we check that it is also Hausdorff. If $\omega\neq\eta\in\Delta(\cA)$, there exists a $A\in\cA$ such that $\omega(A)\neq\eta(A)$. We thus consider $\mO$ and $\mO'$, two disjoints open subsets of $\eC$ around $\omega(A)$ and $\eta(A)$ respectively. With these definition, it is easy to see that $\hat A^{-1}(\mO)$ and $\hat A^{-1}(\mO')$ are disjoints neighbourhoods of $\omega$ and~$\eta$.
\end{proof}
 

\subsection{Topology on \texorpdfstring{$\Delta(\cA)$}{DA}}\label{subsec:topo_Delta}
%----------------------------------------------
We begin to put the \defe{$w^*$-weak topology}{$w^*$-weak topology} on $\cA^*$ which defined by the convergence notion $\omega_n\to\omega$ if and only if $\omega_n(A)\to\omega(A)$ for all $A\in\cA$. 

The \defe{Gelfand topology}{Gelfand!topology} is the induced topology from $\cA^*$ on $\Delta(\cA)$. Let us define the \defe{Gelfand transform}{Gelfand!transform}
        \begin{equation}
        \begin{aligned}
            \hat A \colon \Delta(\cA) &\to \eC\\
            \hat A(\omega)&\mapsto \omega(A).
        \end{aligned}
    \end{equation}  
General theory of functional analysis shows that the $w^*$-weak topology is the weakest in which all linear functional are continuous, so a basis of this topology is given by sets of the form $\{f\in\cA^*\tq f(A)\in\mathcal{O}\}$ where $\mathcal{O}$ is an open in $\eC$ and $A\in\cA$.

A basis of the Gelfand topology is the intersection of these set with $\Delta(\cA)$:
\begin{equation}
  \hat A^{-1}(\mathcal{O})=\{\omega\in\Delta(\cA)\tq \omega(A)\in\mathcal{O}\}.
\end{equation}

%
% Ceci est une répétition de 2.3.4 qui est retapé un peu plus haut. Optimization quand tu nous tient !
%
%\begin{proposition}
%When $\cA$ is an unital commutative Banach algebra, $\Delta(\cA)$ is compact and Hausdorff in the Gelfand topology.
%\label{prop:DcA_comp_Hauss}\lref{2.3.4}
%\end{proposition}

%\begin{proof}
%First, we show that $\Delta(\cA)$ is closed by showing that the limits of converging sequences are in $\Delta(\cA)$. Let us consider $\omega_n\to\omega$ with $\omega_n\in\Delta(\cA)$ for any $n$. We have to see that $\omega\in\Delta(\cA)$.

%First remark that
%\begin{equation}
%\begin{split}
%|\omega(AB)-\omega(A)\omega(B)|&=|  \omega(AB)-\omega_n(AB)+\omega_n(A)\omega_n(B) -\omega(A)\omega(B) |\\
 %                              &\leq|\omega(AB)-\omega_n(AB)|
%                      +|\omega_n(A)\omega_n(B)-\omega_(A)\omega_(B)|\\
%               &\leq |\omega(AB)-\omega_n(AB)|\\
%               &\quad +|\omega_n(A)-\omega(A)|\|B\|\\
%               &\quad +\|A\||\omega_n(B)-\omega(B)|,
%\end{split}
%\end{equation}
%but the right hand side converges to zero when $n$ becomes large, so that $\omega\in\Delta(\cA)$.

%Since the norm of any element of $\Delta(\cA)$ is one, $\Delta(\cA)\subset\cA^*_1$, the unit ball in $\cA^*$: $\cA^*_1=\{\rho\in\cA^*=\|\rho\|=1\}$.

%\begin{theorem}[Banach-Alaoglu] 
%If $X$ is a closed normed vector space, then the unit closed ball in the dual $X^*$ is compact for the $x^*$-topology.
% \label{tho:Banach_Alaoglu}
%\end{theorem}

%This theorem makes $\cA^*_1$ compact. Thus $\Delta(\cA)$ is a closed subspace of a compact space. It is then compact by lemma \ref{lem:ferme_compact}.
%\end{proof}


\begin{lemma}
An element $A\in\cA$ is invertible if and only if $\omega(A)\neq0$ for all $\omega\in\Delta(\cA)$.
\end{lemma}


\begin{proof}
Let $A$ be an invertible element in $A$ and $\omega\in\Delta(\cA)$ such that $\omega(A)=0$. Then 
\[
  1=\omega(\cun)=\omega(A)\omega(A^{-1})=0.
\]

Let us take now a $A\notin G(\cA)$, then the ideal $\cI_A:=\{AB\tq B\in\cA\}$ don't contain $\cun$ and is not a proper ideal. From choice axiom, $\cI_A$ is contained in a maximal ideal $\cI$. From \ref{enueiii} of \ref{tho:ideal_kernel}, there exists a $\omega\in\Delta(\cA)$ whose kernel is $\cI$. In particular, $\omega(A)=0$.
\end{proof}

\begin{theorem}
Let $\cA$ be an unital commutative Banach algebra. Then
\begin{enumerate}
\item The Gelfand transform is a homomorphism $\cA\to C(\Delta(\cA))$. \label{enugi}
\item The image of $\cA$ under the Gelfand transform separates the points in $\Delta(\cA)$, see definition \ref{def:separe}. \label{enugii}
\item \label{enugiii} The spectrum of $A\in\cA$ is 
\[
   \sigma(A)=\sigma(\hat A)=\{\hat A(\omega):\omega\in\Delta(\cA)\}.
\]
\item \label{enugiv} The Gelfand transform is a \defe{contraction}{contraction}:   $\|\hat A\|_{\infty}\leq\|A\|$.
\end{enumerate}\label{tho:unital_comm}
\end{theorem}
 \lref{2.3.5} 
\begin{proof}
Item \ref{enugi} is easy: $\widehat{AB}(\omega)=\omega(AB)=\omega(A)\omega(B)=\hat A(\omega)\hB(\omega)$. Point \ref{enugii} is immediate too: let $\omega_1\neq\omega_2\in\Delta(\cA)$. We need a $A\in\cA$ such that $\hat A(\omega_1)\neq\hat A(\omega_2)$. But the definition of the inequality $\omega_1\neq\omega_2$ is the existence of a $A\in\cA$ such that $\omega_1(A)\neq\omega_2(A)$.

For \ref{enugiii}, recall that
\[
  \rho(A)=\{z\in\eC\tq(A-z)^{-1}\textrm{ exists}\}.
\]
From the lemma the existence of $(A-z)^{-1}$ makes that $\forall\omega\in\Delta(\cA)$, $\omega(A)\neq z$. So the complementary is
\begin{equation}
\begin{split}
 \sigma(A)&=\{z\in\eC\tq\exists\omega\in\Delta(\cA)\textrm{ such that }\omega(A)=z\}\\
          &=\{\omega(A)\tq\omega\in\Delta(\cA)\}\\
          &=\{\hat A(\omega)\tq\omega\in\Delta(\cA)\}.
\end{split}
\end{equation}

The fifth point comes from definition \ref{def:sup_norm} and the fact that, because of the third point,  $r(A)=sup\{\hat A(\omega):\omega\in\Delta(\cA)\}$.
Therefore
\[
\|\hat A\|_{\infty}=\sup_{\omega\in\Delta(\cA)}|\hat A(\omega)|=r(A)\leq\|A\|.
\]
\end{proof}

When a Banach algebra is non unital, one can extend it to $\cA_{\cun}$ and a character $\omega\in\Delta(\cA)$ can be extended too as $\tilde{\omega}\in\Delta(\cA_{\cun})$ by
\[
  \tilde{\omega}(A+\lambda\cun)=\omega(\cun)+\lambda.
\]
The fact that it is multiplicative is a simple computation.

\begin{theorem}
For every element $A$ of a commutative Banach algebra $\cA$, we have $\Spec(A)=\Spec(\hat A)$.
\end{theorem}

\begin{proof}
We want to prove that when $\lambda\in\Spec(A)$, there exists a $\varphi$ such that $\varphi(A)=\lambda$. The ideal generated by $(A-\lambda)$ is a proper ideal which is thus contained in a maximum ideal $M$ by Zorn's lemma. This maximal ideal is closed (if not, the closure would be bigger ideal). Consider an element $x$ in the quotient $\cA/M$. Since $\Spec(x)\neq\emptyset$, the element $(x-\lambda)$ is not invertible for some $\lambda$. That provides an isomorphism $\cA/M\simeq \eC$, and we define $\varphi$ as the composition of that isomorphism by the projection of $\cA$ into $\cA/M$. For this $\varphi$, we have $\varphi(A)=\lambda$.

\begin{probleme}
Faudrait creuser pourquoi on a un isomorphisme $\cA/M\simeq\eC$.
\end{probleme}
\end{proof}

\subsection{An example}
%----------------------

Let $\cA=L^1(\eR)$ with the norm
\[
  \|f\|_1=\int_{\eR}|f(x)|dx,
\]
and the convolution product
\[
   (f\star g)(x)=\int_{\eR}f(x-y)g(y)dy.
\]
We don't take care to analysis subtleties as completion and precise convergence of integrals. For example, we will use and abuse of Fubini's theorem and often say ``for all'' when ``for almost all'' should be preferable. From the fact that $|f(x-y)g(y)|\leq|f(x-y)||g(y)|$ and $\int_{\eR}f(x-y)dx=\int_{\eR}f(x)dx$, we find that 
\[
   \|f\star g\|_1\leq \|f\|_1\|g\|_1
\]
as needed to prove that $(L^1(\eR),\star)$ is a Banach algebra. This is a non unital Banach space because the unit should be the Dirac delta. From analysis, one knows that the dual space of $L^1(\eR)$ is $L^{\infty}(\eR)$ with, for $u\in L^{\infty}(\eR)$,
\[
  u(f)=\int_{\eR}f(x)u(x)dx.
\]
Since $\Delta(\cA)$ is a subset of $L^{\infty}(\eR)$, there exists, for each $\omega\in\Delta(L^1(\eR))$, a $\hat{\omega}$ such that $\omega(f)=\int_{\eR}f(x)\hat{\omega}(x)dx$. With an easy change of variable, the multiplicative condition $\omega(f\star g)=\omega(f)\omega(g)$ gives
\[ 
\int_{\eR^2}f(t)g(y)\hat{\omega}(t+y)dtdy=\int_{\eR^2}f(x)g(y)\hat{\omega}(x)\hat{\omega}(y)dxdy.
\]
We can conclude that $\hat{\omega}(x+y)=\hat{\omega}(x)\hat{\omega}(y)$, in such a manner that
\[ 
\hat{\omega}(x)=e^{ipx}
\]
for a certain $p\in\eC$. For $\hat{\omega}$ to belongs to $L^{\infty}(\eR)$, we must have $p\in\eR$. So we get a bijection $\Delta(\cA)\simeq\eR$. By this identification, we denote by $p$ the element of $\Delta(L^1(\eR))$ given by $\hat{\omega}(x)=e^{ipx}$. With theses notations, the Gelfand $\hat A(\omega)=\omega(A)$ transform reads
\begin{equation}
  \hat f(p)=\omega(f)=\int_{\eR}f(x)\hat{\omega}(x)
                     =\int_{\eR}f(x)e^{ipx}dx.
\end{equation}
This is nothing else than the Fourier transform ! We know that Fourier transform changes the convolution product into the pointwise usual product of functions:
\[ 
\widehat{f\star g}(p)=\hat f(p)\hat g(p)=(\hat f\hat g)(p).
\]
This express the fact that the Gelfand transform is a homomorphism between $\cA$ ---i.e. the product $\star$--- and $C(\Delta(\cA))$ ---i.e. the pointwise product. It is precisely the claim \ref{enugi} of theorem  \ref{tho:unital_comm}. 


\begin{theorem}
Let $\cA$ be a non unital commutative  Banach algebra. Then

\begin{enumerate}
\item \label{enuhi} The space $\Delta(\cA)$ is Hausdorff locally compact for the Gelfand topology,
\item \label{enuhii} $\Delta(\cA_{\cun})$ is the one point compactification of $\Delta(\cA)$,
\item \label{enuhiii} the Gelfand transformation is a homomorphism $\cA\to C_0(\Delta(\cA))$,
\item \label{enuhiv} the spectrum of $A\in\cA$ is 
\[ 
  \Spec(A)=\sigma(A)=\{0\}\cup\{\hat A(\omega)\tq\omega\in\Delta(\cA)\}.
\]
\item \label{enuhv} The image of $\cA$ by the Gelfand transform separates points in $\Delta(\cA)$,
\item \label{enuhvi} Gelfand transform is a contraction:
\[ 
\|\hat A\|_{\infty}\leq\|A\|.
\]

\end{enumerate}

\end{theorem}

For one point compactification issues, see section \ref{sec:compactific}.

\begin{proof}
\ref{enuhi} We add an unity to $\cA$ and we remark that
\[ 
\Delta(\cA_{\cun})=\Delta(\cA)\cup\infty
\]
where $\infty$ is defined by $\infty(A+\lambda\cun)=\lambda$. Indeed let $\psi\in\Delta(\cA)$ and let us ask ourself how to extend it to a multiplicative functional in $\varphi\in\Delta(\cA_{\cun})$. For, let $B\in\cA$ such that $\psi(A)\neq 0$ remark that multiplicative condition imposes $\varphi( (\lambda\cun)(B) )=\varphi(\lambda)\varphi(B)$ while the linearity gives $\varphi(\lambda B)=\lambda\varphi(B)$. Thus $\varphi(\lambda\cun)=\lambda$ and the unique possibility to extends $\psi$ is
\[ 
\varphi(A+\lambda\cun)=\varphi(A)+\lambda
\]
and we note $\infty$ the new functional
\[ 
\infty(A+\lambda\cun)=\lambda.
\]

Since $\cA_{\cun}$ is unital, the character space $\Delta(\cA_{\cun})$  is Hausdorff and compact for its Gelfand topology. As set
\[ 
  \Delta(\cA)=\Delta(\cA_{\cun})\setminus\{\infty\}.
\]
We should prove that the induced topology on $\Delta(\cA)$ from the Gelfand of $\Delta(\cA_{\cun})$ is precisely the own Gelfand topology of $\Delta(\cA)$. In this case, properties of compactification shall gives local compactness.

A basis of the topology of $\Delta(\cA_{\cun})$ is given by $\hat A^{-1}=\{\omega\in\Delta(\cA_{\cun})\tq\omega(A)\in\mO\}$. Then any open set of $\Delta(\cA)$ is open for the induced topology because
\[ 
\{\omega\in\Delta(\cA)\tq\omega(A)\in\mO\}=\{\eta\in\Delta(\cA_{\cun})\tq\eta(A)\in\mO\}\cap\Delta(\cA).
\]
For the converse, an open set for the induced topology is given by
\[
\begin{split}
&\{\omega\in\Delta(\cA_{\cun})\tq\exists A\in\cA,\lambda\in\eC:\omega(A+\lambda\cun)\in\mO\}\setminus\{\infty\}\\
&=\{\omega\in\Delta(\cA)\tq\exists A\in\cA,\lambda\in\eC:\omega(A)\in\mO-\lambda\}                             
\end{split}
\]
where $\mO-\lambda$ is as open as $\mO$. This proves \ref{enuhi} and \ref{enuhii}.

For \ref{enuhiii}, the point is not to prove that Gelfand transform is a homomorphism (that is trivial), but rather that it takes values in $C_0(\Delta(\cA))$.

The complementary of a compact set $K$ in $\Delta(\cA_{\cun})$ is an open set which contains $\infty$. Since $\hat A(\infty)=0$, the values of $\hat A$ in the complementary of $K$ are as small as we want when $K$ becomes larger and larger.

In order to prove \ref{enuhiv}, recall that, by definition, $\sigma_{\cA}(A)=\sigma_{\cA_{\cun}}(A)$. Then
\begin{equation}
  \sigma_{\cA_{\cun}}=\{\hat A(\omega)\tq\omega\in\Delta(\cA_{\cun})\}
                     =\{\hat A(\omega)\tq \omega\in\Delta(\cA)\}\cup\hat A(\infty)
                     =\{\hat A(\omega)\tq \omega\in\Delta(\cA)\}\cup\{0\}.
\end{equation}
Since \ref{enuhv} and \ref{enuhvi} are true for $\cA_{\cun}$, they are true for $\cA$.
 
\end{proof}

\section{Commutative \texorpdfstring{$C^*$}{C}-algebras}
%+++++++++++++++++++++++++++++++++++

A \defe{$C^*$-algebra}{c-star@$C^*$-algebra} is an involutive (complex) Banach algebra such that for all $A$, $B\in\cA$,
\begin{enumerate}
\item $\|AB\|\leq\|A\|\|B\|$,
\item $\|A^*A\|=\|A\|^2$.
\end{enumerate}
One immediately has $\|A\|^2=\|A^*A\|\leq\|A^*\|\|A\|$, then
\begin{equation}
\|A\|=\|A^*\|
\end{equation}
for any element $A$ in a $C^*$-algebra.

\begin{lemma}       \label{LemFiniCSestVNa}
    Every finite dimensional $C^*$-algebra is a von~Neumann algebra.
\end{lemma}

\begin{lemma}[Stone-Weierstrass theorem]
Let $X$ be a compact and Hausdorff space. Any $C^*$-subalgebra of $C(X)$ containing $1_X$ and separating points in $X$ is exactly $C(X)$ seen as $C^*$-algebra.
\end{lemma}\label{lem:Stone_W}
Here, $1_X$ denotes the constant function $1$ on $X$.

\begin{proposition}     \label{PropcomCstarDelCeqX}
Let $X$ be a compact Hausdorff space and see $C(X)$ as a commutative $C^*$-algebra. Then $\Delta(C(X))$ is homeomorphic to $X$.\label{prop:comHauffhomeo}
\end{proposition}
\lref{2.4.3}

\begin{proof}
For $x\in X$, one defines 
        \begin{equation}
        \begin{aligned}
            \omega_x \colon C(X) &\to \eC\\
            f&\mapsto f(x).
        \end{aligned}
    \end{equation}  
It is clearly non zero and multiplicative. Then $\omega_x\in\Delta(C(X))$. We denote by $ \dpt{E}{X}{\Delta(C(X))}$ the map which makes the correspondence between $x$ and $\omega_x$
\[ 
  E(x)f=f(x).
\]
Urysohn lemma \ref{lem:Urysohn} applied to the compact Hausdorff space $X=\Delta(C(X))$ makes that if $x\neq y$, then there exists a function $f\in C(X)$ such that $f(x)\neq f(y)$. This proves that $E$ is injective.

From theorem \ref{tho:ideal_kernel}, we know that 
\[ 
\cI_x=\ker\omega_x=\{f\in C(X)\tq f(x)=0\}
\]
is an ideal in $C(X)$. Suppose that $E$ is not surjective. Then there exists some $\omega\in\Delta(C(X)))$ which don't come from a $x\in X$; for such a $\omega$, we pose 
\[ 
\cI_{\omega}=\ker\omega=\{f\in C(X)\tq \omega(f)=0\}.
\]
 This $\cI_{\omega}$ can't contains any $\cI_x$ because they are maximal ideals. Then for all $x\in X$ , there exists a $f\in C(X)$ such that $f(x)=0$ with $f\notin\cI_{\omega}$. If $E$ is not surjective, then there exists a maximum ideal $\cI$, kernel of a character which is not in the image of $E$. In order this ideal to be included in none of the $\cI_x$, one needs that for all $x\in X$, there exists $f_x\in\cI$ such that $f_x(x)\neq 0$. Let $\mO_x$ be an open set on which $f_x\neq 0$. Since $X$ is compact, one can extract a finite subcovering $X=\cup_i\mO_{x_i}$. Now we build
\[ 
 g:=\sum_{i=1}^n|f_{x_i}|^2.
\]
This is a strictly positive function, then $1/g\in C(X)$, and then $\cI=C(X)$ and $\cI$ should be the kernel of a zero character. This is impossible, then $E$ is surjective and it is a bijection.

In order to prove that $E$ is an homeomorphism, we will use the lemmas \ref{lem:Hausweak} and \ref{lem:wiki}. Let $X_0$ be the space $X$ endowed with its initial topology and $X_G$ the same space with the topology induced from $E^{-1}$, i.e. that an open set in $X_G$ is always the image by $E^{-1}$ of an open set in $\Delta(C(X))$. From definition, $E$ is continuous for the topology $X_G$. We are going to prove that $X_0=X_G$. Definitions give for all $f\in C(X)$,
\[ 
 (\hat f\circ E)(x)=\hat f(\omega_x)=\omega_x(f)=f(x).
\]
But Gelfand topology is the weakest topology for which all $f$ are continuous. On the other hand, $f$ is continuous because it belongs to $C(X_0)$. Then the topology of $X_G$ is weaker than the one of $X_0$. Indeed, let $\mO$ be an open set for $X_G$ and let us prove that it contains an open set of $X_0$. From definition, $\mO=E^{-1}(\mO')$ for a certain open set $\mO'$ of $\Delta(C(X))$, i.e. $\mO'=\hat f^{-1}(A)$ for an open $A$ in $\eC$. The topology $X_G$ is the minimal one for which $E^{-1}\circ\hat f^{-1}(A)$ is open. But $E^{-1}\circ\hat j^{-1}=f^{-1}$, then $(E^{-1}\circ\hat f^{-1})(A)=f^{-1}(A)$ is open in $X_0$. 
\end{proof}


\begin{theorem}[Gelfand theorem]    \index{Gelfand!theorem}\index{theorem!Gelfand}

For any commutative unital $C^*$-algebra $\cA$, there exists an unique (up to isomorphism) compact and Hausdorff space $X$ such that  $\cA$ is isomorphic to $C(X)$.
\label{thoGelfand}
\end{theorem}

\begin{proof}
We immediately give the answer: $X=\Delta(\cA)$ and the isomorphism is
        \begin{equation}
        \begin{aligned}
            \varphi \colon \cA &\to C(\Delta(\cA))\\
            A&\mapsto \hat A.
        \end{aligned}
    \end{equation}  
We first have to prove that $\Delta(\cA)$ is compact and Hausdorff. Then it should be proved that $\varphi$ is an isometric $C^*$-algebra isomorphism and finally that this is the only possibility.

\subdem{The space $\Delta(\cA)$ is compact and Hausdorff}

Proposition \ref{prop:DcA_comp_Hauss} gives it.

\subdem{The map $\varphi$ takes values in $C(\Delta(\cA))$}

From discussion at top of subsection \ref{subsec:topo_Delta}, the functional $\hat A$ is continuous on $X=\Delta(\cA)$.

\subdem{The map $\varphi$ is a morphism}

Linearity of $\varphi$ is clear. Property $\varphi(AB)=\varphi(A)\varphi(B)$ comes from point \ref{enugi} of proposition \ref{prop:DcA_comp_Hauss}. So we are left to prove that $\varphi(A^*)=\varphi(A)^*$. It is sufficient to prove that, if $A=A^*$, then $\varphi(A)$ takes his values in $\eR$. So let $A\in\cA_{\eR}$ and write $\omega(A)=\alpha+i\beta$ with $\alpha,\beta\in\eR$. If we define $B=A-\alpha\cun$, then $\omega(B)=i\beta$ because $\omega(\cun)=1$. Furthermore $B=B^*$. Let $t\in\eR$; we have
\begin{equation}  \label{eq:rcinq}
|\omega(B+it\cun)|^2=|\omega(B)+it|^2
                    =\beta^2+2\beta t+t^2.
\end{equation}
Using formulas $|\omega(A)|\leq\|A\|$ and $\|AA^*\|=\|A\|^2$, we find
\begin{equation}
  |\omega(B+it\cun)|^2\leq\|B+it\cun\|^2
                      =\|B^2+t^2\|
                      \leq \|B\|^2+t^2.
\end{equation}
Then net result is that for all $t\in\eR$,   $\beta^2+2t\beta\leq \|B\|^2$. It is only possible when $\beta=0$. Then $\omega(A)\in\eR$ as soon as $A=A^*$.

\subdem{The map $\varphi$ is isometric}

Let us begin with $A=A^*$. So $\|A^2\|=\|A\|^2$ and $\|A^{2^m}\|=\|A\|^{2^m}$. Using proposition \ref{prop:An_usn}, we find
\[ 
  r(A)=\|A\|.
\]
On the other hand the definition of the supremum norm on the Hausdorff space $\Delta(\cA)$ reads
\begin{equation} \label{eq:AinfA }
  \|\hat A\|_{\infty}=\sup_{\omega\in\Delta(\cA)}|\hat A(\omega)|=r(A)=\|A\|.
\end{equation}
Then $\varphi$ is isometric when $A=A^*$. Now, $A^*A$ is selfadjoint and $\|A^*A\|=\|A\|^2$, then
\[ 
\|A\|^2=\|A^*A\|=\|\widehat{A^*A}\|_{\infty}=\|{\hat A}^*\hat A\|_{\infty}=\|\hat A\|^2_{\infty}.
\]

\subdem{The map $\varphi$ is injective}

If $\varphi(A)=\varphi(B)$, then $\varphi(A-B)=0$. The only way for $\varphi$ to be an isometry is $A-B=0$.

\subdem{The map $\varphi$ is surjective}

Since $\varphi$ is an isometry, it sends a closed set into a closed set, but $\cA$ is closed because it is a Banach space. Point \ref{enugii} of theorem \ref{tho:unital_comm} says that $\varphi(\cA)$ separates points in $\Delta(\cA)$ and we just proved the $\varphi$ preserves the adjoint, so $\varphi(\cA)$ is a $C^*$-subalgebra of $C(\Delta(\cA))$. Finally, it is clear that $\hat{\cun}=1_X$. Lemma \ref{lem:Stone_W} concludes $\varphi(\cA)=C(\Delta(\cA))$.

Now proposition \ref{prop:comHauffhomeo} makes $\varphi$ and homeomorphism between $\cA$ and $\Delta(\cA)$. So the topological structure of $\cA$ is encoded in the algebraic (Banach) structure of $C(\Delta(\cA))$. So if $C(Y)\simeq\cA\simeq C(X)$ as $C^*$-algebras, then $X\simeq Y$ as topological space. This proves the unicity part and concludes the Gelfand theorem.

\end{proof}

As far as notations are concerned, let us recall that the Gelfand transform is $A\mapsto\hat A$ with
\begin{equation}
\begin{aligned}
 \hat A\colon \Delta(\cA)&\to \eC \\ 
   \omega&\mapsto \omega(A). 
\end{aligned}
\end{equation}
One particular class of elements in $\Delta\big( C(X) \big)$ is the ones of the form $\omega_x$ for $x\in X$. These are defined by
\begin{equation}
\begin{aligned}
 \omega_x\colon C(X)&\to \eC \\ 
   f&\mapsto f(x). 
\end{aligned}
\end{equation}
The Gelfand theorem says that any element of $\Delta \big(C(X))$ reads $\omega_x$ for a certain $x\in X$.

\begin{lemma}
Let $\cA$ be a $C^*$-algebra, and $\oB(\cA)$, the set of bounded operators on $\cA$. Then

\begin{enumerate}
\item The map 
        \begin{equation}
        \begin{aligned}
            \rho \colon \cA &\to \oB(\cA)\\
            \rho(A)B&\mapsto AB
        \end{aligned}
    \end{equation}  
is a diffeomorphism between $\rho(\cA)$ and $\rho(\cA)\subset\oB(\cA)$.

\item If $\cA$ has no unit, one can define a norm on $\cA_{\cun}$ by
\begin{equation} \label{eq:normCAu}
\|A+\lambda\cun\|=\|\rho(A)+\lambda\cun\|
\end{equation}
where the right hand side norm is the one in $\oB(\cA)$, see \ref{def:normappl}. With the usual multiplication and the involution
\begin{equation}
  (A+\lambda\cun)^*=A^*+ \overline{\lambda} \cun,
\end{equation}
the set $\cA_{\cun}$ becomes an unital $C^*$-algebra.

\end{enumerate}
 \label{lem:unitariz_C}
\end{lemma}

\begin{proof}
Since $\cA$ is a $C^*$-algebra, $\|\rho(A)B\|\leq\| A \|\|B\|$, then for all $A\in \cA$, one has $\| \rho(A) \|\leq \| A \|$. On the other hand, we know that $\| A^*A \|=\| A \|^2$ and $\| A^* \|=\| A \|$, then
\[ 
\| A \|=\frac{\| AA^* \|}{\| A \|}=\left\|  \rho(A)\frac{A^*}{\| A \|}  \right\|\leq\| \rho(A) \|
\]
from definition of the sup norm. Then $\| \rho(A) \|=\| A \|$ and $\rho$ is an isometry and then is injective because it is linear. It is clearly a homomorphism too. The map $A+\lambda\cun\to\rho(A)+\lambda\cun$ is a $C^*$-algebra-morphism if we define\footnote{We know a definition of $*$ when we look at $\oB(H)$ where $H$ is a Hilbert space, but we are here with $\oB(\cA)$ where $\cA$ is no more than a Banach space; hence we do not have a definition of $*$.} $\rho(A)^*=\rho(A^*)$.  Since the sup norm fulfils condition \eqref{eq:normBanach}, the norm \eqref{eq:normCAu} fulfils the same. So $\cA_{\cun}$ becomes a Banach $*$-algebra and lemma \ref{lem:STARAlC} will help us to conclude that it is a $C^*$-algebra.

The formula $\| A \|^2-\varepsilon\leq\| Av \|^2$ holds for an operator $A$ on a general Banach algebra and an arbitrary vector $v$ with norm $1$. In our present case, if $\| B \|=1$,
\begin{equation}
\begin{split}
  \| \rho(A)+\lambda\cun \|^2-\varepsilon 
        &\leq \| (\rho(A)+\lambda\cun)B \|^2\\
        &    =\| AB+\lambda B \|^2\\
        &    =\| (AB+\lambda B)^*(AB+\lambda B) \|\\
        &    =\| \rho(B^*)\rho(A^*+\overline{\lambda}\cun)\rho(A+\lambda\cun)B \|\\
        &\leq \| \rho(B^*) \|\| (\rho(A)+\lambda\cun)^*(\rho(A)+\lambda\cun) \|\| B \|,
\end{split}
\end{equation}
but we also know that $\| \rho(B^*) \|=\| B^* \|=\| B \|=1$. Letting $\varepsilon\to 0$, we find $\| A \|^2\leq\| A^*A \|$ in the Banach $*$-algebra $\cA_{\cun}$.

\end{proof}

\label{pg:unit_nonunic} This lemma gives us an unitization of a $C^*$-algebra which is not the one previously given for a Banach algebra. This shows that unitization of Banach algebra is not unique. For a $C^*$-algebra, however, we have an unicity result:

\begin{proposition}
For every $C^*$-algebra without unit, there exists an unique unital $C^*$-algebra $\cA_{\cun}$ and an isometric morphism (hence injective) $\cA\to\cA_{\cun}$ such that $\cA/\cA_{\cun}=\eC$.
\label{prop_unitariz_csa}
\end{proposition}
\lref{1.4.6}

\begin{proposition}
If $\cA$ is a commutative $C^*$-algebra, any character is hermitian.
\end{proposition}

\begin{proof}
When $\chi$ is a character, $\chi(A)\in\sigma(A)$ for all $A\in\cA$ and when $A=A^*$, we have $\sigma(A)\subset\eR$. For any $A$, we have a decomposition $A=A_1+iA_2$ and
\[ 
  \chi(A^*)=\chi(A_1-iA_2)=\underbrace{\chi(A_1)}_{\in\eR}-i\underbrace{\chi(A_2)}_{\in\eR}=\overline{ \chi(A) }.
\]

\end{proof}


\section{Functional calculus in unital \texorpdfstring{$C^*$}{C}-algebras}
%+++++++++++++++++++++++++++++++++++++++++++++++++++++++++++++++++++++++++

From now, the $C^*$-algebra $\cA$ is no more assumed to be commutative, but it is unital. 

\begin{definition}      \label{DefElemNormal}
An element $A$ in an involutive algebra is said \defe{normal}{normal!element of an involutive algebra} when $[A,A^*]=0$.
\end{definition}

If $\cA$ is a $C^{*}$-algebra and $A$, $B\in\cA$ we denotes by $C^*(A_1,\ldots,A_n)$ the $C^{*}$-algebra generated by the $A_i$. This is the closure of every finite products of the form $Z_1\cdots Z_k$ where each $Z_j$ is one of the $A_i$.

For any $A$ in a $C^{*}$-algebra we know that $\|A^*A\|=\|A\|^2$.\quext{Il faut encore donner une démonstration de l'équation \eqref{eq:ray_norme}}

If $A$ is normal, then $C^*(A,\cun)$ is commutative. Indeed any element of the form $A_A\ldots A_n$ with $A_i=A$ or $A^*$ can be written under the form $A\ldots AA^*\ldots A^*$.

\begin{proposition}
\begin{equation}\label{eq:ray_norme}
\|A\|=\sqrt{ r(A^*A) }
\end{equation}
\end{proposition}

\begin{proof}
Let $A$ be in $\cA$ and consider a $z\in\rho(A)$. By definition, $(A-z)^{-1}$ exists in $\cA$; since $\varphi$ is a morphism, $\varphi(A-z)$ is also invertible: it is clear that $\varphi( (A-z)^{-1} )$ is a two-sided inverse of $\varphi(A-z)$. Hence $\rho(A)\subseteq\rho(\varphi(A))$ and thus $\sigma(\varphi(A))\subseteq\sigma(A)$. Definition (\ref{def:spectre}) of the spectral radius makes $r(\varphi(A))\leq r(A)$ and equation \eqref{eq:ray_norme} gives the thesis.
\end{proof}


\begin{theorem}
 Consider an unital $C^{*}$-algebra $\cA$ and a $A\in\cA$ such that $A^*=A$. Then
\begin{enumerate}
\item The spectrum $\sigma_{\cA}(A)$ is the same as $\sigma_{C^*(A,\mtu)}(A)$, so that one can speak about $\sigma(A)$ without ambiguities.

\item $\sigma(A)\subset\eR$.

\item \label{enukiii} $\Delta(C^*(A,\mtu))$ is homeomorphic to $\sigma(A)$ and $C^*(A,\mtu)$ is isomorphic to $C(\sigma(A))$. Under this isomorphism, the Gelfand transformed $\dpt{\hat A}{\sigma(A)}{\eR}$ is the identity $\dpt{id_{\sigma(A)}}{t}{t}$.
\end{enumerate} \label{tho:l_2.5.1}
\end{theorem}\lref{2.5.1} 
\begin{proof}
We first consider a normal $B\in G(\cA)$, and the $C^{*}$-algebra $C^*(B,B^{-1},\mtu)$ generated by $B$, $B^{-1}$ and $\mtu$. Since $(B^{-1})^*=(B^*)^{-1}$ and $BB^*=N^*B$, $[B^{-1},{B^*}^{-1}]=0$.

Now, we are going to show that $[{B^*}^{-1},B]=0$. First remark that ${B^*}^{-1} B=(B^{-1} B^*)^{-1}$. We have to show that $B^{-1} B^*B{B^*}^{-1}=\mtu$ and
$B{B^*}^{-1} B^{-1} B^*=\mtu$. These two equalities comes from $[B,B^*]=0$ and $[B^{-1},{B^*}^{-1}]=0$. The same makes that $[B^*,B^{-1}]=0$.

The result is that $C^*(B,B^{-1},\mtu)$ is a commutative $C^{*}$-algebra So one can simply say that it is the closure of the polynomials in
$B$, $B^*$, $B^{-1}$, and ${B^*}^{-1}$.

By the Gelfand theorem, $C^*(B,B^{-1},\mtu)$ is then isomorphic to a $C(X)$ for some compact Hausdorff space $X$. Since $B$ is invertible and the Gelfand transform is an isomorphism, $\hB$ is invertible. Then $\forall\,x\in X$, $\hB(x)\neq 0$. Indeed, the $X$ is (up to an isomorphism) $\Delta(C^*(B,B^{-1},\mtu))$. If for an $\omega\in\Delta(C^*(B,B^{-1},\mtu))$, $\hB(\omega)$, then $\omega(B)=0$ and thus $\omega\equiv 0$. But in the definition of $\Delta(\cA)$, we have explicitly excluded the null form.

On the other hand let us consider $f\in C(X)$ everywhere non zero. Since (pointwise) $0<\|f\|^{-2}_{\infty}ff^*\leq 1$,
\begin{equation}\label{eq:ff}
   0\leq 1_X-\|f\|^{-2}_{\infty}ff^*<1.
\end{equation}
%
But \lref{2.2.4} if $\|A\|<1$, then
\[
   \sum_{k=0}^{n}A^k\to (\mtu-A)^{-1}.
\]
As far as $f$ is concerned for the sup norm, equation \eqref{eq:ff} makes $1_X-ff^*/\|f\|^2_{\infty}$ satisfy this convergence. Then
\[
   \left(
      \frac{ff^*}{\|f\|^2_{\infty}}
   \right)^{-1}
      =
   \sum_{k=0}^{\infty}
   \left(
       \mtu-\frac{ff^*}{\|f\|^2_{\infty}}
   \right)^k,
\]
so that
\begin{equation}
   \us{f}
      =
   \frac{f^*}{\|f\|^2_{\infty}}
   \sum_{k=0}^{\infty}
   \left(
       \mtu-\frac{ff^*}{\|f\|^2_{\infty}}
   \right)^k.
\end{equation}
This is true for any $f$ such that $f(x)\neq 0$ $\forall x\in X$; in particular, it is true for $\hB$. Thus $\hB^{-1}$ is a limit of polynomials in $\hB$ and $\hB^*$. By the inverse Gelfand transform (which is obviously an isomorphism), $B^{-1}$ is a limit of polynomials in $B$ and $B^*$. This is:
\begin{equation}
   C^*(B,B^{-1},\mtu)=C^*(B,\mtu).
\end{equation}

Now, we take our $A$ from the hypothesis: $A=A^*$. Clearly, $A$ is normal and $A-z$ too. If we take $z\in\rho(A)$, our work about $B$ applies to $A-z$. Then $(A-z)^{-1}$ can be written as polynomials in $(A-z)$ and $\mtu$. Thus $(A-z)^{-1}\in C^*(A-z,\mtu)$ and
$z\in\rho_{C^*(A-z,\mtu)}(A)$, but it is clear that $C^*(A-z,\mtu)=C^*(A,\mtu)$. Finally:
\[
   \rho_{\cA}(A)=\rho_{C^*(A,\mtu)}(A).
\]
The set $\sigma$ being nothing else than the complement of $\rho$, the first point of the theorem is finish.

In the course of the demonstration of the Gelfand theorem, we had shown that since $A=A^*$, $\forall\omega\in\Delta(\cA)$, $\hat A(\omega)\in\eR$. But \lref{2.3.5.3} makes
\[
   \sigma(A)=\{\hat A(\omega):\omega\in\Delta(\cA)\}.
\]
Then $\sigma(A)\subset\eR$.

The proof that $\hat A$ is a bijection and that it is continuous is well done in \lref{2.5.1}. Here we will just prove the continuity of $\hat A^{-1}$. From theorem \ref{tho:unital_comm} and what we just did, we know that
\[
   \sigma(A)=\{\hat A(\omega):\omega\in\Delta(\cA)\}\subset\eR,
\]
but $\hat A^{-1}(z)=\omega$ when $\hat A(\omega)=z$, or $\omega(A)=z$. Then $\hat A^{-1}$ is defined on $\sigma(A)$. So from now, one can only consider $z\in\sigma(A)$ and $\hat A^{-1}(z)(A)=z$. By induction, $\hat A^{-1}(z)(A^n)=z^n$. An element in $\Delta(C^*(A,\cun))$ is completely determined by its value on $A$. Then an open set therein has the general form
\[ 
  \mR=\{ \omega\in\Delta(C^*(A,\cun))\tq\hat A(\omega)\in\mO \}
\]
where $\mO$ is any open set in $\eC$. From definition, $\hat A(\mR)=\mO$. So $\hat A^{-1}$ is continuous.

\end{proof}


\begin{probleme}
    There is a notational clash: what is written $\sigma(A)$ is the spectrum of $A$. I want to write it $\Spec(A)$ instead.
\end{probleme}

The following proposition is the \defe{continuous functional calculus}{continuous!functional calculus!selfadjoint in $C^*$-algebra}. 
\begin{theorem}[Continuous functional calculus]     \label{ThoContFuncCalculus}
Let $A\in\cA$ be self-adjoint and $f\in C(\Spec(A))$. One can define a map $\tilde f\colon \cA\to \cA$ in such a way that when $f$ is a polynomial, $\tilde f=f$ and in other cases, it is the uniform approximation of $f$ by polynomials. This map $\tilde f$ which will be denoted by~$f$ fulfills
\begin{enumerate}
\item $\Spec(f(A))=f(\Spec(A))$,  \label{enuji}
\item $\|f(A)\|=\|f\|_{\infty}$.
\end{enumerate}\label{prop:cont_calc}
\end{theorem}

\begin{probleme}
    The proof has to be reordered.
\end{probleme}

\begin{proof}
We know from theorem \ref{tho:l_2.5.1} that $\Delta(C^*(A,\cun))$ is homeomorphic to $\sigma(A)$ and $C^*(A,\cun)$ to $C(\sigma(A))$.

The Gelfand theorem says that if one has a commutative unital $C^{*}$-algebra then  one has an unique (up to homeomorphism) $X$ such that $\cA$ is isomorphic to $C(X)$. Moreover, this isomorphism is an isometry\quext{Is is correct ?}. But we just showed that $C^*(A,\mtu)$ where isomorphic to $C(\sigma(A))$, then one has
\begin{equation}\label{eq:norm_vp_B}
  \|\varphi(B)\|=\|B\|,
\end{equation}
the first norm is taken in $C(\sigma(A))$ and the second one in $C^*(A,\mtu)$. But when $X$ is Hausdorff, we had adopted the $\|.\|_{\infty}$ norm, so that $\|\varphi(B)\|=\|f\|_{\infty}$ and equation \eqref{eq:norm_vp_B} reads:
\begin{equation}
\|f\|_{\infty}=\|f(A)\|.
\end{equation}

Now remark that $f(\sigma(A))$ is the set of values that $f$ takes on $\sigma(A)$, but we know\quext{Vas voir si on know \c{c}a vraiment} that
\[
\sigma(A)=\sigma(\hat A)=\{ \hat A(\omega):\omega\in\Delta(C^*(A,\mtu)) \}.
\]

It is now times to give a sense to $f(\hat{A})$.  Since $f$ is continuous on $\sigma(A)$, there exists a converging infinite sum such that $f(t)=\sum_{k=0}^{\infty}c_kt^k$ for any $t\in\sigma(A)$. In particular, $\forall\omega\in\Delta(C^*(A,\mtu))$, $\hat A(\omega)\in\sigma(A)$; thus 
   $\sum c_k[\hat A(\omega)]^k$ converges everywhere we want. This sum will be denoted by $f(\hat{A})(\omega)$: 
\begin{equation}
f(\hat A)(\omega)=\sum_{k=0}^{\infty}c_k[\hat A(\omega)]^k.
\end{equation}
In other words, $f(\hat A)(\omega)=f(\hat A(\omega))$. We have
\begin{equation}
   f(\sigma(A))=\{f(\hat A(\omega)):\omega\in\Delta( C^*(A,\mtu) )\}
               =\{ f(\hat A)(\omega):\omega\in\Delta( C^*(A,\mtu) ) \}
           =\sigma(f(\hat A)).
\end{equation}


It remains to be proved that $\sigma(f(\hat{A}))=\sigma(f(A))$. We already know that $\sigma(A)=\sigma(\hat{A})$, so we just have to prove that $f(\hat{A})=\widehat{ f(A) }$. On the one hand,
\[ 
 f(\hat{A})\omega=\sum c_k[\hat{A}(\omega)]^k
        =\sum c_k [\omega(A)]^k
        =f\big( \omega(A) \big),
\]
on the other hand,
\[ 
  \widehat{ f(A) }\omega=\omega\big( f(A) \big)=\omega\big[ \sum c_k A^k \big].
\]
On the other hand, one already know that $\sigma(A)=\sigma(A)$, thus we just have to see that 
$f(\hat A)=\widehat{f(A)}$ when $f\colon \sigma(A)\to \eC$ is continuous. The problem is a permutation of $\omega$ and a limit:
\[
 f(\hat A)\omega=\sum_{k=0}^{\infty} c_k\,\omega(A^k),\quad\widehat{f(A)}\omega=\omega\left(\sum_{k=0}^{\infty} c_kA^k\right).
\]
What theorem  \ref{tho:l_2.5.1} says is that $C^*(A,\cun)$ is isomorphic to $C(\sigma(A))$ with 
\[ 
  \sum c_kA^K\mapsto f(x)=\sum c_kx^k.
\]
 In this isomorphism, the map $\hat{A}\colon \sigma(A)\to \eR$ corresponds to the identity map. More precisely, the isomorphism $\varphi\colon C^*(A,\cun)\to C(\sigma(A))$ is the following:
\[ 
  \varphi(B)(t)=a+\sum c_kt^k
\]
when $B=a+\sum c_kA^k$. We know in general that
\[
\sigma(A)=\sigma(\hat{A})=\{ f\big( \hat{A}(\omega) \big)\tq\omega\in\Delta(\cA) \}.
\]
 In the present case, we are working with $\cA=C^*(A,\cun)$, therefore
\[ 
  f\big( \sigma(A) \big)=\{ f\big( \hat{A}(\omega) \big)\tq \omega\in\Delta\big( C^*(A,\cun) \big) \}.
\]
We have to prove that $f\big( \hat{A}(\omega) \big)=f(\hat{A})\omega$. Since $f$ is continuous on $\sigma(A)$, the sum $f(t)=\sum c_kt^k$ converges for all $t\in\sigma(A)$. In particular for $\hat{A}(\omega)\in\sigma(A)$, the sum $\sum_{k=0}^{\infty}\big[ \hat{A}(\omega) \big]^k$ converges.

It is now time to give a sense to $f(\hat{A})$. We know from theorem \ref{tho:l_2.5.1} that $\hat{A}\colon \Delta\big( C^*(A,\cun) \big) \to\sigma(A) $ is an isomorphism. As definition we set
 \begin{equation}
  f(\hat{A})(\omega)=\sum c_k\big[ \hat{A}(\omega) \big]^k
\end{equation}
everywhere it converges. But, since $\hat{A}(\omega)\in\sigma(A)$, it converges everywhere it is interesting for us.

By definition, $\sum_{k=0}^{\infty}=\lim_{n\to\infty}\sum_{k=0}^{n}$, but the proposition \ref{prop:continu_cv}, which gives link between convergence and continuity, assures us that one can permute the sum and $\omega$ because it is a continuous function on $C^*(A,\mtu)$ which is by definition the closure of all polynomials in $A$:
\begin{equation}
\begin{split}  
  \omega(\sum_k B^k)&=\omega(\lim_{n\to\infty}\sum_k^n B^k)
                    =\lim_{n\to\infty}\omega(\sum_k^n B^k)\\
            &=\lim_{n\to\infty}\sum_k^n\omega(B^k)
            =\sum_{k=0}^{\infty}\omega(B^k).
\end{split}         
\end{equation}


We now turn our attention to the second point: $\| f(A) \|_{C^*(A,\cun)}=\| f(A) \|_{\cA}$. It uses proposition 2.26, chapter 4 of \cite{LaHarpe}.

\end{proof}

\subsection{The isomorphism \texorpdfstring{$C^*(A,\mtu)\leftrightarrow C(\sigma(A))$}{AAm AsA}}
%--------------------------------------------------------------------

By definition an element $B\in C^*(A,\mtu)$ can be written as $B=f(A)$ where $f$ is a sum of $\mtu$, $A$, $A^2$, $A^*$, $(A^*)^2$,\ldots In the setting of continuous functional calculus, we suppose that $A$ is selfadjoint, i.e. $A=A^*$, so that $f(A)$ is polynomial (eventually infinite) in $A$ with an independent term $\mtu$. The isomorphism that we consider is 
\begin{equation}
\begin{aligned}
 \varphi\,:\,C^*(A,\mtu)&\to C(\sigma(A))\\
    \varphi(B)&=f\in C(\sigma(A))
\end{aligned}
\end{equation}
where $f$ is the ``definition'' function of $B$ in $C^*(A,\mtu)$. 

\begin{remark}      \label{RemExpansionSqrtConCal}
    The map $\varphi$ depends on $A$. It could be better written $\varphi_A$. As an example, if $A=A^*$, the element $A^{1/2}$ is computed as follows. First, we know the \wikipedia{en}{Taylor_series}{expansion}
    \begin{equation}        \label{EqExpanSqrtt}
        \sqrt{t}=\sum_ka_kt^k.
    \end{equation}
    Then we define $\sqrt{A}=\sum_k a_kA^k$ as element of $C^*(A,\mtu)$.
\end{remark}

\begin{probleme}
    An expansion \eqref{EqExpanSqrtt} is only possible when $t$ is close to $1$. Maybe the definition of $\sqrt{A}$ has to first look at $B=\lambda A$ with $\lambda$ such that the norm of $B$ is close to $1$. Then we write $\sqrt{A}=\frac{1}{ \sqrt{\lambda} }\sqrt{B}$. The square root of $\lambda$ is well defined as a square root in $\eR^+$.
\end{probleme}


In order to show that it is actually an isomorphism, we have to show the following points:
 \begin{enumerate}
     \item
          it is linear;
      \item
         bijective;
     \item
         $\varphi(CD)=\varphi(C)\varphi(D)$;
     \item
         $\varphi(B^*)=\varphi(B)^*$.
 \end{enumerate}
 Here are the proofs.
 \begin{enumerate}
     \item       
        The linearity is clear.
    \item       
        Suppose $\varphi(B)=\varphi(C)$. Definition of $\varphi$ gives $B=\varphi(B)(A)$ and $C=\varphi(C)(A)$. For the surjectivity, note that $C(\sigma(A))$ is given by continuous functions whose can be uniformly approximated by polynomials; then for each $f\in C( \sigma(A))$, there corresponds a $B=\varphi(A)\in C^*(A,\mtu)$.
    \item
        Consider $C=f(A)$, $D=g(A)$; thus $CD=(fg)(A)$ and $\varphi(CD)=fg=\varphi(C)\varphi(D)$. 
    \item
        The last point comes from the fact that $A=A^*$. Indeed, consider $B=f(A)=\sum_k c_kA^k$. Then
        \[
            B^*=\sum_k c_k^*(A^*)^k=\sum_k c_k^*A^k=f^*(A).
        \]
 \end{enumerate}

We have shown that $\varphi(B)=f$ when $B=f(A)$ is an isomorphism between $C^*(A,\mtu)$ and $C(\sigma(A))$ if $A$ is selfadjoint: $A=A^*$.

%VNVNVNVNVN
% La proposition \label{prop:unicitenormcsa} venait ici.

\begin{corollary}
For each $C^*$-algebra, there exists an unique unital $C^*$-algebra $\cA_{\cun}$ and an isometric morphism $\cA\to\cA_{\cun}$ such that $\cA_{\cun}/\cA\simeq\eC$.
\end{corollary}\label{cor_csa_unit}

\begin{proof}
We yet defined $\cA_{\cun}$ in lemma \ref{lem:unitariz_C} and we just prove that the norm was unique. Since all elements in $\cA_{\cun}$ are given under the form $A+\lambda\cun$ with $A\in\cA$, it is obvious that $\cA_{\cun}/\cA\simeq\eC$. The canonical injection $\varphi(A)=A$ is a morphism.
\end{proof}

\begin{proposition}
Let $\cA$ be an involutive Banach algebra, $\cB$ a $C^{*}$-algebra and $\dpt{\pi}{\cA}{\cB}$ a morphism. Then $\forall A\in\cA$,
\begin{equation}  \label{eq_morleqpi}
  \|\pi(A)\|\leq\|A\|.
\end{equation}
Hence $\varphi$ is continuous by lemma \ref{lem:lin_vec_cont}. 
 \label{prop:vp_leq}\dixref{1.3.7}
\end{proposition}

\begin{remark}
Proposition \ref{prop:vp_leq} together with proposition \ref{prop:vp_geq} makes that any $C^{*}$-algebra isomorphism is isometric. The lemma \lref{2.7.6} gives a slightly stronger result.
\end{remark}

\begin{proof}
If $B\in\cB$ is hermitian, $\|B^2\|=\|B^*B\|=\|B\|^2$. An induction shows that $\|B^{2n}\|^{-2n}=\|B\|$. With $n\to\infty$, the left hand side goes to the spectral radius (\ref{prop:An_usn}). Then 
\begin{equation}
r(B)=\|B\|
\end{equation}
when $B$ is hermitian.

Let us now take $A\in\cA$. We have $\sigma_{\cB}(\pi(A))\subset \sigma_{\cA}(A)$ because 
$(\pi(A)-\lambda\mtu)v=\mtu$ for a $v\neq 0$ in $\cB$ implies $(A-\lambda\mtu)\pi^{-1}(v)=\mtu$ with $\pi^{-1}(v)\neq 0$ (because $\pi(\mtu)=\mtu$).

Remark that $(A-\lambda\mtu)(a)=0$ with $a\neq 0$

\end{proof}


\begin{lemma}
If $\dpt{\varphi}{\cA}{\cB}$ is a morphism of $C^*$-algebra and if $A=A^*$, then 
\[ 
  f(\varphi(A))=\varphi(f(A)).
\]
for all $f\in C(\sigma(A))$.
\end{lemma} \label{lem:fvpvpf}


\begin{proof}
Since $\sigma(\varphi(A))\subseteq\sigma(A)$, the function $f$ is well defined on $\varphi(A)$. If $f$ is a polynomial, the result comes from the fact that $\varphi(AB)=\varphi(A)\varphi(B)$. If $f$ is a general continuous function, it can be approximated by polynomials. Taking partial sums, $s_n=\sum_{k=1}^n\varphi(c_kA^k)$ and $v_n=\varphi(\sum_{k=1}c_kA^k)$, the linearity of \emph{finite sums} gives the result.
\end{proof}

Where in the proof did we use the assumptions ? The definition of $f(A)$ when $\dpt{f}{\eR}{\eR}$ was given in \ref{prop:cont_calc} in order to get formulas $\sigma\circ f=f\circ\sigma$ and $\| f(A) \|=\| f \|$.

\section{Positivity}
%+++++++++++++++++++

Let $\cA$ be a $C^*$-algebra. We say that $A\in\cA$ is \defe{positive}{positive!element!of a $C^*$-algebra} when 
\begin{enumerate}
\item  $A=A^*$
\item  $\Spec(A)\subset\eR^+$.
\end{enumerate}
In this case, we write $A\geq 0$ or $A\in\cA^+$,
\[ 
  \cA^+=\{ A\in\cA_{\eR}\tq\sigma(A)\subset\eR^+ \}.
\]
A set of particular importance is the set of selfadjoint elements:\nomenclature[C]{\(\cA_{\eR}\)}{The set of selfadjoint elements in \(\cA\)}
\begin{equation}
    \cA_{\eR}=\{ A\in\cA\tq A=A^* \}.
\end{equation}
These elements have real spectrum. We will see in theorem \ref{ThoElsPositifsBBstar} that the set of positive elements in $\cA$ is given by
\begin{equation}
    \cA^+=\{ A^2\tq A\in\cA_{\eR} \}=\{ B^*B\tq B\in\cA \}.
\end{equation}

An element $A\in\cA$ is a \defe{projector}{projector}, or a \defe{projection}{projection!in a $C^*$-algebra } if $A=A^*$ and $A^2=A$. In the case of a $C^*$-algebra of linear operators acting on a vector space, if $x$ is an eigenvector of the projector $A$ with the eigenvalue $\lambda$, then $Ax=\lambda x$ and $A^2x=\lambda^2x=\lambda x$. Thus $1$ is the only eigenvalue of a projector (or zero, which is the kernel). In particular a projection is positive and reads\label{PgProjPositif} $A=B^*B$ for some $B\in \cA$ by theorem \ref{ThoElsPositifsBBstar}.

\begin{probleme}
    I think that the notation \(\cA_{\eR}\) stand for the elements with real spectrum. I have to check it and add to the notation index.
\end{probleme}

\begin{proposition}
An element $A\in\cA_{\eR}$ is positive if and only if the Gelfand transform $\hat A$ is pointwise positive in $C(\sigma(A))$.
\end{proposition}

\begin{proof}
\subdem{Necessary condition}
We know from theorem \ref{tho:unital_comm}, \ref{enugiii} that
\[ 
  \sigma(A)=\sigma(\hat A)=\{ \hat A(\omega)\tq\omega\in\Delta(\cA) \}, 
\]
but $\sigma(A)\subset\eR^+$ if $A$ is positive.

\subdem{Sufficient condition}
From hypothesis, $A\in\cA_{\eR}$ and $A^*=A$. We have to see that positivity of $\hat A$ implies $\sigma(A)\subset\eR^+$. From point \ref{enukiii} of theorem \ref{tho:l_2.5.1}, the function $\dpt{\hat A}{\sigma(A)}{\eR}$ is identity and positive, $\sigma(A)\subset\eR^+$.

\end{proof}

 
\begin{proposition}     \label{PropAplusConvexCone}
    The set $\cA^+$ of positive elements of the $C^*$-algebra $\cA$ is a convex cone (see definition \ref{DefConvexCone}).
\end{proposition}

Note that the $C^*$-algebra $\cA$ has to be commutative in order the Gelfand transform to be defined. It is supposed unital too. 

\begin{proof}

    We have to check the \(3\) points of definition \ref{DefConvexCone}.
    \begin{enumerate}
            \item

                We know that if $A=A^*$ and $f\in C(\sigma(A))$, the commutator $[\sigma,f]$ is zero; as a particular case $\sigma(tA)=t\sigma(A)$. Then for $t>0$, the element $tA$ is positive.

            \item

                The fact that $\sigma(A)\subset [0,r(A)]$ implies that for all $t\in\sigma(A)$ and for all $c\geq r(A)$,  $| c-t |\leq c$. Now we study the quantity
                \[ 
                  \sup_{t\in \sigma(A)}| c1_{\sigma(A)}-\hat A |.
                \]
                The function $1_{\sigma(A)}$ is $0$ or $1$ following the argument belongs to $\sigma(A)$ or not while $\hat A(t)=t$ in $\sigma(A)$. Then
                \[ 
                  \sup_{t\in\sigma(A)}| c1_{\sigma(A)}(t)-\hat A(t) |=\sup_{t\in\sigma(A)}| c-t |\leq c.
                \]
                This shows that 
                \begin{equation} \label{eq:cunhatA}
                    \| c1_{\sigma(A)}-\hat A \|_{\infty}\leq c
                \end{equation}
                for all $c>r(A)$ and then for all $c>\| A \|$. Since $\cA$ is commutative and $A=A^*$, we know that $\| \hat A \|_{\infty}=\| A \|$ from  \eqref{eq:AinfA }. Taking the inverse Gelfand transform of equation \eqref{eq:cunhatA}, we find
                \begin{equation} \label{eq:norcin}
                    \| c\cun-A \|\leq c
                \end{equation}
                for all $c\geq\| A \|$. Be careful on a point: the inverse Gelfand transform is not taken into $\cA$, but into $C^*(A,\cun)$ which is commutative and unital and then fulfills $\| \hat A \|_{\infty}=\| A \|$.

                We know that the norm of $f(A)$ in $\cA$ and in $C^*(A,\cun)$ are the same, namely equation \eqref{eq:norcin} is a relation for the norm of $c\cun-A$ in $C^*(A,\cun)$. Until now we had proved that if $\sigma(A)\subset\eR^+$, then $\| c\cun-A \|\leq c$ for all $c\geq\| A \|$. 

                Taking the inverse argument, we can say that if $\| c\cun-A \|\leq c$ for a certain $c\geq\| A \|$, then $\sigma(A)\subset\eR^+$. Indeed the Gelfand transform of the assumption gives $\| cA_{\sigma(A)}-\hat A \|_{\infty}\leq c$, i.e. $\sup_{t\in\sigma(A)}| c1_{\sigma(A)}A-\hat A |\leq c$. As $\hat A$ is identity on $\sigma(A)$, for all $t\in\sigma(A)$, we have $| c-t |\leq c$. This shows that $t>0$ for all $t\in\sigma(A)$. Thus $\sigma(A)\subset\eR^+$.

                Let us now take $A+B$ instead of $A$ and $c=\| A \|+\| B \|$. Remark that $c\geq \| A+B \|$. We have
                \[ 
                    \| c\cun-(A+B) \|\leq\| (\| A \|-A) \|+\| (\| B \|-B) \|
                \]
                where $\| A \|-A=r\cun-A$ with $r=\| A \|$. On the other hand, $\| r\cun-A \|\leq r$ for all $r\geq\| A \|$, then we can apply the first result to get 
                \[ 
                    \| c\cun-(A+B) \|\leq \| A \|+\| B \|=c
                \]
                with $c\geq \| A+B \|$. Then the inverse argument gives $\sigma(A+B)\subset\eR^+$ and $A+B\in\cA^+$.

            \item

                If $A\in\cA^+\cup (-\cA^+)$. Then $\sigma(A)\subset\eR^+$ and $\sigma(A)\subset\eR^-$; we conclude that $\sigma(A)=\{  0\}$. Since $\| A \|=r(A)$, this gives $\| A \|=0$.
                
        \end{enumerate}
\end{proof}


\begin{proposition}
Let $E$ be a real locally convex space and $C$ a closed convex cone with top on $0$\quext{Par top je veux dire le sommet du cône je ne sais pas comment dire en anglais.} and $x\in E$, $x\notin C$. Then there exists a continuous linear function $f\colon E\to \eR$ such that
\begin{itemize}
\item $f\geq 0$ on $C$,
\item $f(x)<0$.
\end{itemize}

\dixref{B.5}
\end{proposition}

\begin{proof}
It is possible to find a continuous linear form $f$ and a real $\alpha$ such that $f(y)\geq\alpha$ on $C$ and $f(x)<\alpha$ (see Urysohn lemma \ref{lem:Urysohn}). We have $0=f(0)\geq\alpha$, so $f(x)<0$. If $f(y)<0$ fora $y\in C$, we find $f(\lambda y)<\alpha$ for a large enough $\lambda$. This is absurd and we conclude that $f\geq0$ on $C$.
\end{proof}

\section{Ordering relation}
%+++++++++++++++++++++++++++++

We know from proposition \ref{PropAplusConvexCone} that the set \(\cA^+\) of positive elements in the $C^*$-algebra \(\cA\) is a convex cone in \(\cA\). We saw in subsection\ref{SubsecPosiCconePartOrder} that in a real vector space a convex cone is equivalent to a notion of positivity and to a partial ordering. Here we consider the real vector space \(\cA_{\eR}\) and we say that $A\leq B$ if\nomenclature[C]{$\leq$}{Ordering relation in $C^*$-algebra} when $B-A\in\cA^+$. Thus we write
\begin{equation}
    \cA^+=\{ A\in\cA_{\eR}\tq  A\geq 0\}.
\end{equation}


\begin{proposition}     \label{PropAAsmAuAAu}
If $A=A^*$, then
\[ 
  -\| A \|\cun\leq A\leq\| A \|\cun.
\]
\end{proposition}

\begin{proof}
Since $A=A^*$ and $\| A \|\in\eR^+$, we know that $\| A \|\cun-A\in\cA_{\eR}$. We have to show that $\| A \|\cun-1$ is positive. In other words we have to show that its Gelfand transform is pointwise positive in $C(\sigma(A))$. We consider $\| A \|\cun-A$ as an element of $C^*(A,\cun)$ and the Gelfand transform gives
\[ 
  \| A \|1_{\sigma(A)}-\hat A.
\]
We will explain just after the proof why we write $1_{\sigma(A)}$ instead of $1$. If one apply this on an element of $\sigma(A)$, one has to remember that $\hat A(t)=t$, so $\hat A(t)$ is at most $r(A)$ because $\sigma(A)\subset\eR^+$. Since $r(A)\leq\| A \|$, one sees that
\begin{equation}
    \big( \| A \|1_{\sigma(A)}-\hat A \big)(t)=\| A \|-t,
\end{equation}
but $t\in\sigma(A)$ implies $t\leq r(A)$. Then the latter expression is positive and $\| A \|\cun-A\in\cA_{\eR}^+$.

\end{proof}

Let us now see why $\hat \cun=1_{\sigma(A)}$ in $C^*(A,\cun)$. First, we consider $\cun\in C^*(A,\cun)$; from properties of characters, $\hat\cun(\omega)=\omega(\cun)=1$. So considering $\cun\in C^*(A,\cun)$, the character $\hat\cun$ is defined by $\hat\cun(\omega)=1\in\eR$ for all $\omega\in\Delta(C^*(A,\cun))$. Now, we know that there exists an homeomorphism between $\Delta(C^*(A,\cun))$ and $\sigma(A)$. Under this homeomorphism, $\cun(t)=t$ for all $t\in\sigma(A)$.

Since there is a bijection between $\sigma(A)$ and $\Delta(C^*(A,\cun))$, it should make no sense to define $\hat\cun$ outside of $\sigma(A)$.

\begin{proposition} \label{prop:mBABineq}
If $-B\leq A\leq B$, then $\| A \|\leq\| B \|$.
\end{proposition}

\begin{proof}
The relations $-B\leq A\leq B$ and $-\| B \|\cun\leq B\leq\| B \|\cun$ give $-B\leq A\leq B\leq\| B \|\cun$. But $B\leq\| B \|\cun$ implies $-\| B \|\cun\leq-B$ from definition of a linear partial order. Then $-\| B \|\cun\leq A\leq\| B \|\cun$. All this make that $\sigma(A)\subseteq[-\| B \|,\| B \|]$ and then that $\| A \|\leq\| B \|$ because $r(A)=\| A \|$ when $A=A^*$ and $r(A)=\sup\{ | z |\tq z\in\sigma(A) \}$.
\end{proof}

\begin{proposition}
Let $A$, $B\in\cA^*$ such that $\| A+B \|\leq k$. Then $\| A \|\leq k$.
\end{proposition}

\begin{proof}
From assumptions, $A=A^*$, the $A\leq\| A \|\cun$. Taking this relation with $A+B$ instead of $A$, we get $A+B\leq\| A+B \|\cun\leq k\cun$. From linearity of the partial order, this implies $0\leq A\leq k\cun -B$. Since $k\geq 0$, $-k\cun\leq 0$ and $k\cun-B\leq k\cun$ because $0\leq B$. Finally, proposition \ref{prop:mBABineq}  makes that $\| A \|\leq k\| \cun \|$ implies $\| A \|\leq k$.
\end{proof}


\begin{lemma}
For all $A$ such that $A=A^*$, we have a decomposition
\[ 
  A=A_++A_-
\]
where $A_+,A_-\in\cA^+$ and $A_+A_-=0$. Furthermore
\[ 
  \| A_{\pm} \|\leq \| A \|.
\]
 \label{lem:AsAdecm}
\end{lemma}

\begin{proof}
We apply the continuous functional calculus with $f=\in_{\sigma(A)}=f_++f_-$ where
\begin{equation} \label{eq:rrdeux}
\begin{aligned}
  \id_{\sigma(A)(t)}&=\max{0,t}&&\textrm{because $\sigma(A)\subset\eR^+$}\\
  f_+(t)&=\max\{ 0,t \}\\
  f_-(t)&=\max\{ -t,0 \}.
\end{aligned}
\end{equation}
Recall that when $A=A^*$, the spectral radius is given by $r(A)=\| A \|$. Then $\| f_{\pm} \|_{\infty}\leq r(A)=\| A \|$.

Let us prove that $f_+(A)\in\cA^+$. From the continuous calculus and the fact that $f_+(A)^*=f_+(A)$, we find that $\sigma(f_+(A))\subset\eR^+$. Since $A\in\cA_{\eR}$, we know that $\sigma(A)\subset\eR$ and thus that $f_+(\sigma(A))\subset\eR^+$. From equation part \ref{enuji} of the continuous functional calculus, theorem \ref{prop:cont_calc}, we conclude that $\sigma(f_+(A))\subset\eR^+$ and then that $f_+(t)f_-(t)=0$.
\end{proof}

\begin{lemma} \label{lem:rtrois}
If $-C^*C\in\cA^+$ for $C\in\cA$, then $C=0$.
\end{lemma}

\begin{proof}
We can decompose $C=D+iE$ with $D$, $E\in\cA_{\eR}$; then 
\begin{equation}  \label{eq:rquare}
C^*C=2D^2+2E^2-CC^*.
\end{equation}
 If $z\neq 0$ and $AB-z$ is invertible, then $BA-z^{-1}\cun$ is invertible and $(BA-z)^{-1}=B(AB-z)^{-1}A-z^{-1}\cun$. Then $\sigma(AB)\cup\{ 0 \}=\sigma(BA)\cup\{ 0 \}$ and $\sigma(C^*C)\subset\eR^-$ imply $\sigma(-CC^*)\subset\eR^+$. Now all terms of the right hand side of \eqref{eq:rquare} are in $\cA^+$ and $C^*C\in\cA^+$. Since the assumption is $-C^*C\in\cA^+$, we conclude that $\sigma(C^*C)=0$ and $C=0$. 
\end{proof}


\begin{theorem}     \label{ThoElsPositifsBBstar}
The set of positive elements in $\cA$ is given by
\begin{equation}
\cA^+=\{ A^2\tq A\in\cA_{\eR} \}=\{ B^*B\tq B\in\cA \}
\end{equation}
when $\cA$ is an unital $C^*$-algebra.
\end{theorem}

\begin{proof}
If $A\in\cA^+$, one can define $\sqrt{A}\in\cA_{\eR}$ in the same way as in proposition \ref{prop:cont_calc} with $f=\sqrt{\cdot}$. With this definition we have $(\sqrt{A})^2=A$, so that $\cA^+\subset\{ A^2\tq A\in\cA_{\eR} \}$.

Using the linearity of the involution term by term in the formula $\sqrt{A}=\sum_k c_kA^k$ shows that $\sqrt{A}\in\cA_{\eR}$ when $A=A^*$. 

For the inverse inclusion, consider $A\in\cA_{\eR}$. Since $A=A^*$, we have $\sigma(A)\subset\eR$. Using formula $\sigma(f(A))=f(\sigma(A))$ with $f(t)=t^2$, we find $\sigma(A^2)=\sigma(A)^2\subset\eR^+$. The first equality is proved.
 
For the second equality, we begin by applying lemma \ref{lem:AsAdecm} to $B^*B$, let $B^*B=A_+-A_-$. From equations \eqref{eq:rrdeux} we see that $A_+-A_-=-A_-$. Then $(A_-)^3=-A_-(A_+-A_-)A_-=-A_-B^*BA_-=-(BA_-)^*BA_-$. Since $AA_-$ is positive, $\sigma(A_-)\subset\eR^+$. Using the continuous calculus with $f(t)=t^3$, it proves that $(A_-)^3\geq 0$ and thus that $-(BA_-)^*BA_-\geq 0$. Lemma \ref{lem:rtrois} shows that $BA_-=0$.

This proves that $(A_-)^3=0$. From the continuous functional calculus with $f(t)=t^{1/3}$, it proves that $A_-=0$ and then $B^*B=A_+\in\cA^+$.
\end{proof}
\begin{corollary}
When $A_1,A_2\in\cA_{\eR}$ and $B\in\cA$, if $A_1\leq A_2$, then $B^*A_1B\leq B^*A_2B$ .
\end{corollary}

\begin{proof}
The assumption is $A_2-A_1\geq 0$, but from theorem \ref{ThoElsPositifsBBstar}, there exists $A_3\in\cA$ such that $A_2-A_1=A^*_3A_3$. The same property shows that $(A_3B)^*A_3B\geq 0$. This gives the corollary.
\end{proof}


\begin{corollary}
For all $A$, $B\in\cA$, we have
\[ 
  B^*A^*AB\leq \| A \|^2B^*B.
\]
 \label{cor:BeAAeB}
\end{corollary}

\begin{proof}
The inequality $-\| A \|\cun\leq A\leq \| A \|\cun$ holds when $A=A^*$. Let us write it for $A^*A$ and recall that $\| A^*A \|=\| A \|^2$ in all $C^*$-algebra. Then $A^*A\leq \| A \|^2\cun$ and by applying the previous corollary, we find $B^*A^*AB\leq\| A \|^2B^*B$ 
\end{proof}



\section{Approximate unit}
%++++++++++++++++++++++++++++++++

\begin{definition}
 A \defe{approximate unit}{approximate unit} of a normed algebra $\cA$ is a family $(u_i)_{i\in I}$ (where $I$ is an increasing filtered set, in order to makes sense to $i\to\infty$) such 
that 
\begin{enumerate}
\item  $\forall i$, $\|u_i\|\leq 1$,
\item  $\forall A\in\cA$, $\|u_iA-A\|\to 0$ and $\|Au_i-A\|\to 0$. \label{enuoii}
\end{enumerate}
\label{def:app_unit}
\end{definition}
It is clear that if $\cA$ has an unit $\mtu$, the choice $u_i=\mtu\,\forall i$ is an approximate unit.
\dixref{B29}

\subsubsection*{Example}

Let $C_0(\eR)$, the $C^*$-algebra of continuous functions on $\eR$ such that for all $\varepsilon>0$, there exists a compact $K$ outside of which $| f(x) |<\varepsilon$. This has no unit, but we can build an approximate unit as following. Let $n\in\eN$ and define
\[ 
  \cun_n=
\begin{cases}
 1& \textrm{on $[-n,n]$}\\
 0&\textrm{when $| x |>n+1$}\\
\textrm{continuous}&\textrm{otherwhise}
\end{cases}
\]
It should be noted that the limit $\lim_{n\to\infty}\cun_n$ in the norm  $\| . \|_{\infty}$ is \emph{not} $1_{\eR}$. 

\begin{proposition}
All non unital $C^*$-algebra admits an approximate unit. If $\cA$ is separable, then $\Lambda$ can be chosen countable.
\end{proposition}

\begin{proof}
Let $\Lambda$ be the set of finite subsets of $\cA$ endowed with the partial ordering given by inclusion. To each $\lambda=\{ A_1,\ldots,A_n \}\in\Lambda$, we define $B_{\lambda}=\sum_{i=1}^nA_i^*A_i$. From construction, $B_{\lambda}^*=B_{\lambda}$ and $B_{\lambda}\in\cA^+$. The latter point gives $\sigma(B_{\lambda})\subset\eR^+$. Si for all $z\in\eR^-$, the element $A-z\cun$ is invertible (in $\cA_{\cun}$) and $n^{-1}\cun+B_{\lambda}$ is invertible when $n<0$. It allows us to define 
\[ 
  \cun_{\lambda}=B_{\lambda}(n^{-1}\cun+B_{\lambda})^{-1}.
\]
Since $B_{\lambda}=B_{\lambda}^*$ and $B_{\lambda}$ commutes with all functions of itself (as $(n^{-1}\cun+B_{\lambda})^{-1}$), it leads to $\cun_{\lambda}=\cun_{\lambda}^*$. The element $(n^{-1}\cun+B_{\lambda})^{-1}$ is computed in $\cA_{\cun}$, the it can be written as $C+\mu\cun$ for certain $C\in\cA$ and $\mu\in\eC$. However, $\cun_{\lambda}\in\cA$. Indeed
\[ 
  \cun_{\lambda}=B_{\lambda}(C+\mu\cun)=B_{\lambda}C+\mu B_{\lambda}\in\cA.
\]

Now we use the continuous functional calculus on $B_{\lambda}$ with $f(f)=\frac{t}{n^{-1}+t}$; more precisely, we will use formula $\sigma(f(B_{\lambda}))=f(\sigma(B_{\lambda}))$ where we know that $\sigma(B_{\lambda})\subset\eR^+$. We find $f(B_{\lambda})=B_{\lambda}(n^{-1}\cun+B_{\lambda})^{-1}=\cun_{\lambda}$. Then
\[ 
  f(\sigma(B_{\lambda}))\subset f(\eR^+)=[0,1].
\]
In particular, $\sigma(\cun_{\lambda})\subset[0,1]$. 

In order to prove the second condition of definition \ref{def:app_unit}, we pose $C_i=\cun_{\lambda} A_i-A_i$. One can prove that \quext{Je ne vois pas comment.}
\[ 
  \sum_{i=1}^n C_iC_i^*=n^{-2}B_{\lambda}(n^{-1}+t)^{-2}.
\]
 We consider the function $f(t)=n^{-2}t(n^{-1}+t)^{-2}$ on $\sigma(B_{\lambda})$, then $f\geq 0$ and in particular
\[ 
  \sup_{t\in\eR^+}| f(t) |=\frac{1}{4n}
\]
and $f(t)$ takes its maximum at $t=1/n$. Since $f$ is defined on $\sigma(B_{\lambda})$ which is a part of $\eR^+$, we have $\| f \|_{\infty}\leq 1/4n$. Then $\| n^{-2}B_{\lambda}(n^{-1}\cun+B_{\lambda})^{-2} \|\leq 1/4n$ and
\[ 
  \| \sum_i C_iC_i^* \|\leq\frac{1}{4n}.
\]
In particular, for each $i$, we have $\| C_iC_i^* \|\leq \frac{1}{4n}$.

Now consider $A\in\cA$. Then $A$ belongs to a $\lambda\in\Lambda$ and we can build a directed subset of $\Lambda$ in which $A$ is always present. For $n\to\infty$,
\begin{equation}
\begin{split}
  \lim_{\lambda\to\infty}\| \cun_{\lambda}-A \|^2&=\lim_{\lambda\to\infty}\| (\cun_{\lambda}A-A)^*(\cun_{\lambda}A-A) \|\\
                                                &=\lim_{\lambda\to\infty}\| C_iC_i^* \|\\
                                                &=0 
\end{split}
\end{equation}
because $\lambda\to\infty$ needs $n\to\infty$ and $\| C_iC_i^* \|\leq 1/4n$.

If $\cA$ is separable, then we can find a countable set of $A_i$ dense in $\cA$. We build $\Lambda$ from them and the proof works because when $\lambda\to\infty$, any $A\in\cA$ is reached from density.

\end{proof}


\begin{lemma} \label{lem:taulimA}
Let $\{ \cun_{\lambda} \}$ be an approximate unit in the ideal $\cI$ and $A\in\cA$. Then
\begin{equation}
\| \tau(A)\|=\lim_{\lambda\to\infty}\| A-A\cun_{\lambda} \|.
\end{equation}

\end{lemma}


\begin{proof}
First remark that property \ref{enuoii} of definition of an approximate unit don't imply that the above limit is zero because it holds when $\{ \cun_{\lambda} \}$ is an approximate unit in $\cA$. 

Definition $\| \tau(A) \|:=\inf_{J\in\cI}\| A+J \|$ immediately gives
\begin{equation} \label{eq:ir512_2}
  \|\tau(A)\|\leq \| A-A\cun_{\lambda} \|.
\end{equation}

In order to prove the inverse inequality, we add (if necessary) an unit to $\cA$ and we consider a $J$ in $\cI$. Then
\begin{equation} \label{eq:ri512}
\begin{split}
  \| A-A\cun_{\lambda} \|&=\| (A+J)(\cun-\cun_{\lambda})+J(\cun_{\lambda}-\cun) \|\\
                         &\leq \| A+J \|\| \cun-\cun_{\lambda} \|+\| J\cun_{\lambda}-J \|.
\end{split}
\end{equation}

Remember equation \eqref{eq:norcin}: for all $c\geq \| A \|$, we have $\| c\cun-A \|\leq c$. Let us write it  with $\cun_{\lambda}$ instead of $A$ and $c=1$:
\[ 
  \| \cun-\cun_{\lambda} \|\leq 1.
\]
On the other hand, equation \eqref{eq:ri512} with $\lambda\to\infty$ (and then with $\| J\cun_{\lambda}-J \|\to 0$) gives
\[ 
  \lim_{\lambda\to\infty}\| A-A\cun_{\lambda} \|\leq\| A+J \|.
\]
From definition of the norm on $\cA/\cI$ and of an infimum, for all $\varepsilon>0$, there exists a $J\in \cI$ such that $\| \tau(A) \|+\varepsilon\geq\| A+J \|$. Let us fix an $\varepsilon$ and such a $J$. Then, using equation \eqref{eq:ir512_2} , 
\[ 
  \| A+J \|-\varepsilon\leq\| \tau(A) \|\leq\| A-A\cun_{\lambda} \|,
\]
 and then
\[ 
  \lim_{\lambda\to\infty}\| A-A\cun_{\lambda} \|-\varepsilon \leq\| \tau(A) \|\leq\| A-A\cun_{\lambda} \|,
\]
letting $\varepsilon\to 0$, it gives the lemma:
\[ 
  \| \tau(A) \|\geq \lim_{\lambda\to\infty}\| A-A\cun_{\lambda} \|.
\]


\end{proof}

\begin{theorem}
Let $\cI$ be an ideal in the $C^*$-algebra $\cA$. Then

\begin{enumerate}
\item the ideal $\cI$ is selfadjoint; in other words, if $A\in\cI$, then $A^*\in\cI$,  \label{enuni}
\item the quotient $\cA/\cI$ is a $C^*$-algebra for the norm \label{enunii}
\[ 
  \| \tau(A)\|:=\inf_{J\in\cI}\| A+J \| ,
\]
the multiplication
\[ 
  \tau(A)\tau(B):=\tau(AB),
\]
and the convolution
\[ 
  \tau(A)^*=\tau(A^*)
\]
where $\dpt{\tau}{\cA}{\cA/\cI}$ is the canonical projection.

\end{enumerate}
\label{tho_idautadjquo}
\end{theorem}

\begin{proof}
It is necessary to prove \ref{enuni} before \ref{enunii} because the point \ref{enuni} proves the well definiteness of the involution. Let $\cI^*=\{ A^*\tq A\in\cI \}$ and $J\in\cI$. Then $J^*J\in\cI\cap\cI^*$ because an ideal is left and right ideal (it is easy to see that $\cI^*$ is an ideal).

We are going to prove that $\cI\cap\cI^*$ is a $C^*$-subalgebra of $\cA$. If $J\in \cI\cap\cI^*$,then $J^*\in \cI\cap\cI^*$ too. On the other hand, $\cI\cap\cI^*$ is closed linear subspaces of $\cA$ because $\cI$ and $\cI^*$ are such from definition of an ideal. Ideals are Banach space and it is clear that the conditions about norm and involutions are true in $\cI\cap\cI^*$.

So $\cI\cap\cI^*$ is a non unital $C^*$-algebra and posses an approximate unit $\{ \cun_{\lambda} \}$. Let $J\in\cI$; the following upper bound holds:
\begin{equation}
\begin{split}
\| J^*-J^*\cun_{\lambda} \|^2&=\| (J-\cun_{\lambda}J)(J^*-J^*\cun_{\lambda}) \|\\
                            &=\| (JJ^*-JJ^*\cun_{\lambda})-\cun_{\lambda}(JJ^*-JJ^*\cun_{\lambda}) \|\\
                            &\leq \| JJ^*-JJ^*\cun_{\lambda} \|+\| \cun_{\lambda}(JJ^*-JJ^*\cun_{\lambda}) \|\\
                            &\leq \| JJ^*-JJ^*\cun_{\lambda} \|+\| \cun_{\lambda} \|\| JJ^*-JJ^*\cun_{\lambda} \|.
\end{split}
\end{equation}
Since $JJ^*$ belongs to $\cI\cap\cI^*$ which is a $C^*$-algebra an\quext{Cette phase me semble avoir un problème.} which $\cun_{\lambda}$ is build, we recognize in the latter two terms something of the form $\| A\cun_{\lambda}-A \|$ whose limit is zero. Then $\lim_{\lambda\to\infty}\| J^*-J^*\cun_{\lambda} \|=0$.
 We have proved that $J^*\cun_{\lambda}\in\cI\cap\cI^*$ and in particular that $J^*\cun_{\lambda}\in\cI$ for all $\lambda$. Then $J^*$ is the limit of a sequence in $\cI$, but the latter is closed, then $J^*\in\cI$.

Now we prove \ref{enunii}.
From proposition \ref{prop:ideal_Banach}, the quotient space $\cA/\cI$ is a Banach algebra in the chosen product and  norm. We just have to prove that $\tau(A)^*=\tau(A^*)$ gives a well behaved involution on $\cA/\cI$, i.e. we have to show that $\| A^*A \|=\| A \|^2$ for $A\in\cA/\cI$, or $\| \tau(A)^*\tau(A) \|=\| \tau(A) \|^2$ for $A\in\cA$. Using lemma \ref{lem:taulimA}, we can compute
\begin{equation}
\begin{aligned}
\| \tau(A)^2 \|&=\lim_{\lambda\to\infty}\| A-A\cun_{\lambda} \|^2&\textrm{The lemma}\\
               &=\lim_{\lambda\to\infty}\| (A-A\cun_{\lambda})^*(A-A\cun_{\lambda})&\textrm{from $\| A \|^2=\| A^*A \|$
in $\cA_{\cun}$} \\
               &=\lim_{\lambda\to\infty}\| (\cun-\cun_{\lambda})A^*A(\cun-\cun_{\lambda}) \|\\
               &\leq\lim_{\lambda\to\infty}\| \cun-\cun_{\lambda} \|\| A^*A(\cun-\cun_{\lambda}) \|\\
               &\leq\lim{\lambda\to\infty}\| A^*A(\cun-\cun_{\lambda}) \|&\textrm{because $\| \cun-\cun_{\lambda} \|\leq 1$}\\
               &=\lim_{\lambda\to\infty}\| A^*A-A^*\cun_{\lambda} \|\\
               &=\| \tau(A^*A) \|&\textrm{definition of the norm}\\
               &=\| \tau(A)^*\tau(A) \|.
\end{aligned}
\end{equation}
Now lemma \ref{lem:STARAlC} makes the Banach $\cA/\cI$ a $C^*$-algebra.

\end{proof}

\begin{corollary}
Morphism of $C^*$-algebras have following properties related to ideals:

\begin{enumerate}
\item The kernel of a morphism between two $C^*$-algebras is an ideal.  \label{enupi}
\item Any ideal in a $C^*$-algebra is the kernel of a morphism.\label{enupii}
\item Then any morphism has norm $1$.\label{enupiii}
\end{enumerate}

\end{corollary}

\begin{proof}
\ref{enupi} If $A\in\ker\varphi$, then $AB\in\ker\varphi$ because $\varphi(AB)=\varphi(A)\varphi(B)=0$. From proposition \ref{prop:vp_leq}, a morphism is continuous. The kernel of a continuous map is always continuous by the ``intermediate value'' property.

\ref{enupii} Let $\cI$ be an ideal in the $C^*$-algebra $\cA$ and $\dpt{\tau}{\cA}{\cA/\cI}$, the canonical projection. We know that $\cA/\cI$ is a $C^*$-algebra and that $\tau$ is a morphism. Let us show that $\cI=\ker\tau$.

On the one hand, if $J\in\cI$, then $\tau(J)=\tau(0)$ which is the zero of $\cA/\cI$. Let, on the other hand, $J\in\ker\tau$. Then $\tau(J)=0=\tau(0)$. The fact that $\tau(J)=\tau(0)$ shows that the difference between $0$ and $J$ is an element of $\cI$. In other words: $J\in\cI$.

This and the fact that $\| \tau \|=1$ give \ref{enupiii}.\quext{Je ne vois cependant pas pourquoi.}

\end{proof}

\begin{lemma}  \label{lem:injmorpisom}
An injective $C^*$-algebra morphism is is always an isometry and in particular, the image is closed.
\end{lemma}

\begin{proof}
Let $\varphi$ be the morphism and suppose that there exists a $B\in\cA$ such that $\| \varphi(B) \|\neq \| B \|$. Then
\[ 
  \| \varphi(B^*B) \|=\| \varphi(B) \|^2\neq\| B \|^2=\| B^*B \|.
\]
We pose $A=B^*B$ and we know that $\| A \|=\sqrt{r(A^*A)}$ because $A^*=A$. By definition, $r(B)=\sup\{ | z |\tq z\in\sigma(B)) \}$. If we apply $\sigma(f(A))=f(\sigma(A))$ to $f(t)=t^2$, we conclude that $\sigma(A)\neq\sigma(\varphi(A))$. We saw, during proof of \ref{prop:vp_leq}, that $\sigma(\varphi(A))\subseteq\sigma(A)$. Then we have a strict inclusion 
\[ 
   \sigma(\varphi(A))\subset\sigma(A).
\]
Thus there exists a continuous function $f\neq 0$ on $\sigma(A)$ such that $f(x)=0$ when $x\in\sigma(\varphi(A))$. In particular $f(\varphi(A))=0$ and by \ref{lem:fvpvpf}, we see that $\varphi(f(A))=0$. This is in contradiction with the injectivity of $\varphi$. 

\end{proof}

\begin{corollary}
The image of a $C^*$-algebra morphism $\dpt{\varphi}{\cA}{\cB}$ is closed. In particular $\varphi(\cA)$ is a $C^*$-subalgebra of $\cB$.
\end{corollary}

\begin{proof}
We  define $\dpt{\psi}{\cA/\ker\varphi}{\cB}$ by $\psi([a])=\varphi(a)$. It is a (bijective) vector space isomorphism, and $\varphi=\psi\circ\tau$. Since $\varphi$ is a morphism, $\ket \varphi$ is an ideal and then $\cA/\ker\varphi$ is a $C^*$-algebra. Now $\varphi=\psi\circ\tau$ is an injective morphism of $C^*$-algebra and its image is closed: $\psi(\cA/\ket\varphi)$ is closed in $\cB$. Since $\varphi$ is a morphism, its range is a $*$-algebra in $\cB$ which is closed for the norm in $\cB$. With induced operation from $\cB$, $\varphi(\cA)$ becomes a $C^*$-algebra.
\end{proof}


\begin{proposition} 
Let $\cA$ be a $C^*$-algebra  and $M$ a two-sided ideal in $\cA$ everywhere dense in $\cA$. There exists an increasing filtering approximate unit of $\cA$ contained in $M$.

If $\cA$ is separable, we can ask the approximate unit to be indexed by $\eN$.

 \lref{2.7.2}\dixref{1.7.2}
\end{proposition}
\quext{Ici, y'a un point pas clair dans Landsman 2.7.2. De toutes facons, c'est une d\'emonstration \`a refaire}
\begin{proof}
Let $\cA_{\cun}$ be the algebra obtained by adding an unit to $\cA$, and $\Lambda$ the set of finite parts of $M$ ordered by inclusion. For $\lambda=\{ A_1,\ldots, A_n \}$, we pose
\[ 
  v_{\lambda}=A_1A_1^*+\ldots+A_nA_n^*\in M
\]
 and 
\[ 
  u_{\lambda}=v_{\lambda}\big( \frac{1}{ n }+v_{\lambda} \big)^{-1}
\]
with $A_i\in M$ and $\lambda\in\Lambda$. The fact that $A^*\in M$ when $A\in M$ comes from point \ref{enuni} of theorem \ref{tho_idautadjquo}.

\begin{probleme}
Cette démonstration n'est pas du tout finie.
\end{probleme}

\end{proof}

The same kind of proof gives the following

\begin{proposition}\dixref{1.7.3}
Let $\cA$ be a $C^{*}$-algebra and $\mI$ a right ideal in $\cA$. There exists, in $\mI\cap\cA^+$, a family $(u_{\lambda})$ with $\lambda$ in a filtered set such that
\begin{enumerate}
\item $\|u_{\lambda}\|\leq 1$,
\item $\lambda\leq\mu$ implies $u_{\lambda}\leq u_{\mu}$,
\item $\forall A\in\overline{\mI}$, $\|u_{\lambda} A-A\|\to 0$.
\end{enumerate}
\end{proposition}

The definition of ``$A\leq B$''\ for $A,B$ in a $C^{*}$-algebra comes from the positivity notion \lsref{2.6}.

\begin{proposition}
      \quext{Il faudra déplacer cette proposition pcq elle utilise des trucs qui sont écrits plus loin.}
Let $\cA$ be a $C^{*}$-algebra $\cB$ an involutive normed algebra and $\dpt{\varphi}{\cA}{\cB}$ an injective morphism. Then $\forall A\in\cA$,
\[
   \|\varphi(A)\|\geq\|A\|.
\]\label{prop:vp_geq}
\end{proposition}
\dixref{1.8.1}

\begin{proof}
Consider $A\in\cA$, and suppose that $\|\varphi(A^*A)\|\geq\|A^*A\|$. Then
\[
\|A^2\|=\|A^*A\|\leq\|\varphi(A^*A)\|=\|\varphi(A)^*\varphi(A)\|\leq\|\varphi(A)\|^2,
\]
so that we can only consider the case where $A$ is hermitian. One can also consider only the restriction of $\varphi$ to the sub$C^{*}$-algebra generated by $A$ and then suppose that $\cA$ is commutative. In the same way, we consider only $\varphi(\cA)$ instead of $\cB$ so that $\cB$ can also be considered as commutative. In other words, we consider $C^*(A,\cA)$ and $C^*(\varphi(A),\mtu)$ instead of $\cA$ and $\cB$. But from now to the end of the proof, we will often write it as $\cA$ and $\cB$.

Now, we consider the completion of $\cB$ and we add an unit to $\cA$ and $\cB$. Then the structure spaces $\mS:=\Delta(C^*(A,\mtu))$ and $\mT:=\Delta(C^*(\varphi(A),\mtu))$ are compacts in the weak topology \lref{2.3.4}.

If $\chi\in\mT$, it is clear that $\chi\circ\varphi$ is a character because these are both multiplicative. Then $\chi\circ\varphi\in\mS$, and we denotes it by $\varphi'(\chi)$. It is easy to prove that $\dpt{\varphi'}{\mT}{\mS}$ is a continuous map. Indeed, an open set in $\mS=\Delta(C^*(A,\mtu))$ is of the form
\[
  \hB^{-1}(\mO)=\{ \omega\in\mS:\omega(B)\in\mO  \}
\]
 for $B\in C^*(B,\mtu)$. But
 \begin{equation}
\begin{split}
   {\varphi'}^{-1}(\hB^{-1}(\mO))&=\{ \chi\in\Delta(\cB):\varphi'(\chi)\in\hB^{-1}(\mO)  \}\\
                         &=\{ \chi\in\Delta(\cB): (\chi\circ\varphi)(B)\in\mO  \} \\
             &=\widehat{\varphi(B)}^{-1}(\mO),
\end{split}
\end{equation}
which is an open subset of $\mT=\Delta(\cB)$.
 
 Thus $\varphi'(\mT)$ is a compact subset of $\mS$.

Suppose that $\varphi'(\mT)\neq\mS$, and consider two functions $f,g$ on $\mS$ such that $fg=0$, but $f\neq 0$ and $g=1$ on $\varphi'(\mT)$. The Urysohn lemma gives a function $f$ which works for it: $f=0$ on $\varphi'(\mT)$ and $f\neq 0$ out of $\varphi'(\mT)$. Now, $f$, $g\in C(\Delta(\cA))$ but theorem \lref{2.3.5.1} gives a homomorphism between $\cA$ and $C(\Delta(\cA))$. Then we have $A$, $B\in\cA$ such that $\hat{A}=f$,  $\hat{B}=g$ with $AB=0$, $x\neq 0$, $\chi(\varphi(y))=1$ for any $\chi\in\mT$. From this last equality one conclude that $\varphi(B)$ is invertible in $\cB$. But $\varphi(A)\varphi(B)=0$ and $\varphi(A)\neq 0$, which contradict.

Thus $\varphi'(\mT)=\mS$, so that $\forall A\in\cA$, $\forall\xi\in\mS$, there exists $\chi\in\mT$ such that $\xi(A)=\varphi'(\chi)(A)$. Finally, for any $x\in\cA$,
\begin{equation}
 \|A\|=\sup_{\xi\in\mS}|\xi(A)|
      =\sup_{\chi\in\mT}|\varphi'(\chi)(A)|
      =\sup_{\chi\in\mT}|\chi\circ\varphi(A)|
      \leq\|\varphi(A)\|.
\end{equation}

\end{proof}

\section{States}
%+++++++++++++++

\begin{theorem}[Riesz representation theorem]
Let $X$ be a locally compact Hausdorff space and $\dpt{\Lambda}{C_0(X)}{\eC}$ be a positive linear functional. Then there exists a $\sigma$-algebra $\mM$ in $X$, which contains all Borel sets, and an unique measure $\mu$ on $\mM$ such that

\begin{enumerate}
\item $\Lambda f=\int_Xfd\mu$,
\item $\mu(K)<\infty$ for all compact $K\subset X$,
\item If $E\in\mM$, then
\[ 
  \mu(E)=\inf\{ \mu(V)\tq E\subset V,\; V\textrm{ open} \}
\]
\item If $E$ is open or if $E\in\mM$ with $\mu(E)<\infty$, then
\[ 
  \mu(E)=\sup\{ \mu(K)\tq K\subset E,\; K\textrm{ compact} \},
\]
\item The space $(X,\mM,\mu)$ is a complete measure space. That is, if $E\in\mM$, $A\subset E$ and $\mu(E)=0$, then $A\in\mM$.

\end{enumerate}
\end{theorem}

\lref{2.8}

\subsection{States on unital \texorpdfstring{$C^*$}{C*}-algebras}
%------------------------------------------------------------------

%VNVNVNVN
% Ici se trouvait la définition \label{DefStateUnital}, ainsi que la proposition \ref{Propstaretattraces}.

\begin{theorem}
The space of states of $\cA=C(X)$ is the set of probability measures on $X$. We suppose that $X$ is compact and Hausdorff.
\end{theorem}
\lref{2.8.2}

\begin{proof}
If $X$ is compact and Hausdorff, we know from a long time that $C(X)$ is an unital $C^*$-algebra with norm
\[ 
  \| f \|_{\infty}:=\sup_{x\in X}| f(x) |.
\]
We can apply the Riesz representation theorem: we have on $X$ a $\sigma$-algebra $\mM$ which contains all Borel sets and an (unique) measure $\mu$ on $\mM$ such that, among other properties,
\begin{equation}
  \omega(f)=\int_Xfd\mu.
\end{equation}
Since $X$ is compact, it fulfills $\mu(X)<\infty$. We have $\omega(\cun)=\int_X\cun d\mu=\mu(X)=1$. Then $\mu(X)=1$ and it is a probability measure.

On the other hand, it is clear that a probability is a state.

\end{proof}

Let $\omega\in\etS$ be positive. Then the definition\label{PgStateInn}
\begin{equation} \label{eq:defprodetat}
(A,B)_{\omega}:=\omega(A^*B)
\end{equation}
gives a \defe{pre-inner product}{pre-inner product}\label{pgdef_preinned}, i.e. an inner product without the condition $v^2=0\Rightarrow v=0$. Indeed, $(A,A)_{\omega}=\omega(A^*A)$, but $A^*A\in\cA^+$, then $\omega(A^*A)\geq 0$ because $\omega$ is a state. In the case of a faithful state, we get an inner product.

From Cauchy-Schwarz inequality,
\begin{equation} \label{eq:omABleq}
  | \omega(A^*B) |^2\leq \omega(A^*A)\omega(B^*B).
\end{equation}
Moreover, $\omega(A^*)=(A,\cun)_{\omega}$, then
\begin{equation} \label{eq:omABleqs}
  \omega(A^*)=\overline{\omega(A)}.
\end{equation}

\begin{proposition}
Let $\dpt{\omega}{\cA}{\eC}$, a linear map on an unital $C^*$-algebra $\cA$. It is positive if and only if $\omega$ is bounded and $\| \omega \|=\omega(\cun)$. In particular

\begin{enumerate}
\item A state in an unital $C^*$-algebra is bounded and has norm $1$, \label{71125ai}
\item an element $\omega\in\cA^*$ such that $\| \omega \|=\omega(\cun)=1$ is a state on $\cA$. \label{7125aii}
\end{enumerate}
\label{prop:linposboun}
\end{proposition}


\begin{proof}
\subdem{Direct sense}
Let $\omega$ be positive and begin by $A=A^*$. Then inequality $-\| A \|\cun\leq A\leq\| A \|\cun$ gives $\omega(A)\leq\| A \|\omega(1)$. Indeed
\begin{equation}
\begin{split}
  A\leq\| A \|\cun&\Rightarrow A-\| A \|\cun\leq 0\\
                  &\Rightarrow\omega(\| A \|\cun-A)\geq0\\
                  &\Rightarrow \| A \|\omega(\cun)\geq\omega(A).
\end{split}
\end{equation}
The same can be done with $-\| A \|\cun\leq A$ as starting point, then for $A=A^*$, we have $\| \omega(A) \|\leq \omega(\cun)\| A \|$.

Let us now consider any $A$. We know that $| \omega(A^*B) |^2\leq\omega(A^*A)\omega(B^*B)$. Writing it for $A=\cun $ and using $\| A^*A \|=\| A \|^2$, we find that $| \omega(B) |^2\leq\omega(\cun)^2\| B \|^2$,
and from the definition of the functional norm, it leads to 
\begin{equation} \label{eq:irquatre}
\| \omega \|\leq\omega(\cun).
\end{equation}
A $C^*$-algebra is a normed vector space, then for all $A\in\cA$, there exists a $R\in\cA$ such that $A=\| A \|R$ and $\| R \|=1$. For this $R$, we have
\[ 
  \frac{| \omega(A) |}{\| A \|}=| \omega(R) |,
\]
and the definition of $\| \omega \|$ can be rewritten as
\[ 
  \| \omega \|=\sup\{ \frac{| \omega(A) |}{\| A \|}\tq A\in\cA \}.
\]
Now equation \eqref{eq:irquatre} gives $\| \omega \|\leq\omega(A)$. From definition of $\| \omega \|$, we also know that $\| \omega \|\geq \omega(\cun)$. Finally, $\| \omega \|=\omega(\cun)$.

\subdem{Inverse sense}

We know that $\omega$ is bounded and that $\| \omega \|=\omega(\cun)$ and we want to prove that $\omega$ is positive. Let $A\in\cA_{\eR}$ and let us decompose $\omega(A)=\alpha+i\beta$ where $\alpha,\beta\in\eR$. We can adapt the reasoning that around equation \eqref{eq:rcinq} to prove that $\beta=0$. Namely,
\begin{equation}
  | \omega(B+it\cun) |^1\leq \omega(\cun)^2\| B+it\cun \|
                        \leq \omega(\cun)^2(\| B \|^2+t^2),
\end{equation}
and for the self-adjoint element $B:=\omega(\cun)A-\alpha\cun$,
\[ 
  | \omega(B+it\cun) |^2=\beta^2+2\beta t\omega(\cun)+t^2\omega(\cun)^2.
\]
Combining the two equations, and taking into account the fact that $\omega(\cun)=\| \omega \|>0$,
\[ 
  \beta^2+\beta t\omega(\cun)\leq\omega(\cun)^2\| B \|^2.
\]
Since the right hand side don't depend on $t$, we conclude that $\beta=0$. This shows that $\omega$ is real on $\cA_{\eR}$. Now we are going to prove that $\omega(A)\geq 0$ when $A\geq 0$. Let $s>0$ be so small that $\| \cun-sA \|\leq 1$. Then
\[ 
  1\geq\| \cun-sA \|=\underbrace{\frac{\| \omega \|}{\omega(\cun)}}_{=1}\| \cun-sA \|\geq \frac{| \omega(\cun-sA) |}{\omega(\cun)}.
\]
We conclude that $| \omega(\cun)-s\omega(A) |\leq\omega(\cun)$, which is only possible if $\omega(A)\geq 0$. It proves that $\omega$ is positive.

Let us now check that points \ref{71125ai} and \ref{7125aii} are effectively obtained. We proved that for a state $\| \omega \|=\omega(\cun)$ and in the definition, we impose $\omega(\cun)=1$.

\end{proof}

\subsection{States on non unital \texorpdfstring{$C^*$}{C*}-algebras}
%----------------------------------------------

We now relax the unital hypothesis. 

\begin{definition}  \label{DefApplPositive}
    A linear map $\dpt{q}{\cA}{\cB}$ between two $C^*$-algebra is \defe{positive}{positive!map between $C^*$-algebra} when $q(A)\geq 0$ in $\cB$ whenever $A\geq 0$ in $\cA$.
\end{definition}

\begin{proposition}
A positive map is bounded.\label{prop:posborn}
\end{proposition}

Since for linear map, to be bounded is equivalent to continuous, all positive map is continuous.

\begin{proof}
We first prove that a bounded map on $\cA^+$ is bounded on $\cA$. Let us decompose $A\in\cA$ into $A=A'+iA''$ with $A',A''\in\cA_{\eR}$ and use lemma \ref{lem:AsAdecm} to write $A=A'_+-A'_-+iA''_+-iA''_-$.We have $\| A' \|\leq \| A \|$ and $\| A'' \|\leq \| A \|$. If $B$ is one of $A'_{\mp}$ or $A''_{\pm}$, lemma also says that $\| B \|\leq \| A \|$. Now let us suppose that $\| q(B) \| \leq C\| N \|$ for all $B\in\cA^+$ and a certain $C>0$. Then $\| q(A) \|\leq 4 C\| A \|$ and $q$ is then bounded.

We are now going to prove that a positive map $q$ is bounded. Suppose that $q$ is unbounded. In particular, it is not bounded in $\cA^+$ (if it were, it should be bounded everywhere) and there exists a sequence $(A_n)\in\cA^+_1$ such that $\| q(A_n) \|\geq n^3$ for all $n$. Here, the symbol $\cA_A^+$ denote the elements $A\in\cA^+$ for which $\| A \|\leq 1$.

Let us consider the series $\sum_{n=0}^{\infty}n^{-2}A_n$. Since $\| A_n \|\leq 1$, this converges to an element $A\in\cA^+$. If $q$ is positive, then $q(A)\geq n^{-2}q(A_n)$ and from property $-B\leq A\leq B\Rightarrow\| A \|\leq\| B \|$, we see that
\[ 
 \| q(A) \|\geq n^{-2}\| q(A_n) \|\geq n 
\]
for all $n\in\eN$. Indeed, $q$ positive implies $q(A_n)\geq 0$ because $A_n\in\cA^+$. Now,
\begin{equation}
\begin{split}
&-q(A)\leq n^{-2}q(A_n)\leq q(A)\\
&\Rightarrow \| n^{-2}q(A_n) \|\leq\| q(A) \|\\
&\Rightarrow \| q(A) \|\geq n^{-2}\| q(A_n) \|\geq n^3.
\end{split}
\end{equation}
This proves that $\| q(A) \|$ is greater than $n^3$ for all $n$; this is impossible (because this is a finite number). Then $q$ is bounded on $\cA^+$ and on the whole $\cA$.

\end{proof}

If we consider $\cB=\eC$, states are particular cases of positive maps and then states on unital $C^*$-algebra are bounded and have norm $1$. 

In order to define a state on a non-unital $C^*$-algebra, we can't take the normalization $\omega(\cun)=1$. But we just prove a condition to get an unit norm in the unital case. So we define


\begin{proposition}
Let $\cA$ be an involutive Banach algebra with unit $\cun$ such that $\| \cun \|=1$. If $f$ is a positive linear form, then it is continuous and $\| f \|=f(\cun)$. \label{prop_Dix214}
\end{proposition}
\dixref{2.1.4}
\begin{proof}
This statement is nothing else than propositions \ref{prop:linposboun} and \ref{prop:posborn}.
\end{proof}


\begin{proposition}
Let $\cA$ be an involutive Banach algebra with approximate unit and $\cA_{\cun}$ the involutive algebra deduced from $\cA$ by adding an unit. We consider $f$, a linear positive and continuous form on $\cA$. Then
\begin{enumerate}
\item \label{itemi_prop_invaddunit} $\forall A\in \cA$, $f(A^*)=\overline{ f(A) }$ and
\[ 
  | f(A) |^2\leq \| f \|f(A^*A),
\]
\item \label{itemii_prop_invaddunit} $| f(B^*AB) |\leq\| A \|f(B^*B)$,
\item \label{itemiii_prop_invaddunit} $\| f \|=\sup_{\substack{A\in\cA\\\| A \|\leq1} } f(A^*A)$,
\item \label{itemiv_prop_invaddunit} Let $(A_i)_{i\in I}$ be elements of $\cA$ indexed by the filtering set $I$ such that $\| A_i \|\leq 1$ and $f(A_i)\to\| f \|$. Then $f(A^*_iA_i)\to\| f \|$,
\item \label{itemv_prop_invaddunit} If $(u_j)_{j\in J}$ is an approximate unit of $\cA$, then $f(u_j)\to \| f \|$ and $f(u_j^*u_j)\to\| f \|$,
\item \label{itemvi_prop_invaddunit} The function $f$ can be extended to an unique positive form $\tilde f$ on $\cA_{\cun}$ in such a way that $\tilde f(\cun)=\| f \|$. If a positive form on $\cA_{\cun}$ extends $f$, it is everywhere bigger than $f$.
\item \label{itemvii_prop_invaddunit} With the same $(A_i)_{i\in I}$ that in \ref{itemiv_prop_invaddunit}, we have $A_i\to\cun$ for the prehilbert structure defined by $\tilde f$ on $\cA_{\cun}$. Then $\cA$ is everywhere dense in $\cA_{\cun}$ for this structure.
\end{enumerate}

\label{prop_invaddunit}
\end{proposition}
\dixref{2.1.5}

\begin{proof}
The definition $(A,B):=f(A^*B)$ gives a pre-inner product (see page \pageref{pgdef_preinned}) because 
\[
(\lambda A,B)=f(\overline{ \lambda }A^*B)=\overline{ \lambda }(A,B).
\]
 So we want to write
\[ 
  f(A^*)=f(A^*\cun)=(A,\cun)=\overline{ (\cun,A) }=\overline{ f(A) },
\]
but we do not have unit. We therefore have to use continuity of $f$ instead:
\[ 
f(A^*)=\lim(A^*u_j)
    =\lim\overline{ f(u_j^*A) }
    =\lim\overline{ f\big( (A^*u_j)^* \big) }
    =\overline{ f(A^{**}) }
    =\overline{ f(A) }.
\]
The second equality is not a particular case of $f(A^*)=\overline{ f(A) }$, but the fact that $f(B^*A)=\overline{ f(A^*B) }$ because $f$ defines a pre-inner product\quext{C'est un problème pcq sans ça, je ne vois pas comment démontrer que ce $f$ donne bien un pré produit scalaire.}.

Since we have a pre-inner product, we have Cauchy-Schwarz:
\[ 
  \lim| f(Au_j) |^2\leq\lim f(A^*A)f(u^*_ju_j),
\]
hence
\begin{equation}
| f(A) |^2=\lim| f(Au_j) |^2\leq f(A^*A)\lim f(u^*_ju_j).
\end{equation}
An involutive Banach algebra fulfils $\| AB \|\leq\| A \|\,\| B \|$ for all $A$, $B\in\cA$, so $\| u^*_ju_j \|\leq 1$ because definition of an approximate unit gives $\| u_j \|\leq 1$. We have
\[ 
  \| f \|=\sup_{\| A \|=1}| f(A) |
\]
Let us consider $\lambda\in\eR$ such that $\| \lambda u_j^*u_j \|=1$; we necessarily have $\lambda\geq 1$. We have
\[ 
  \| f \|\geq f(\lambda u^*_ju_j)=\lambda f(u^*_ju_j)\geq f(u^*_ju_j)
\]
because $f$ is positive and $u^*_ju_j\in\cA_{\eR}$. We conclude that 
\[ 
  f(A^*A)\lim f(u^*_ju_j)\leq \| f \|f(A^*A)
\]
and this finally gives the first point:
\begin{equation}
  | f(A) |^2\leq \| f \|f(A^*A).
\end{equation}

\ref{itemvii_prop_invaddunit}$\Rightarrow$\ref{itemvii_prop_invaddunit}. Indeed, \ref{itemi_prop_invaddunit} implies $\| f \|^2\leq\| f \|\lim f(A_i^*A_i)$ which in turn gives $\lim f(A^*_iA_i)\geq\| f \|$. But $\| A_i \|\leq 1$, so $\| f \|\geq\lim f(A_i^*A_i)$ and we conclude that $\| f \|=\lim f(A_i^*A_i)$.

\ref{itemiv_prop_invaddunit}$\Rightarrow$\ref{itemiii_prop_invaddunit}. On the one hand, for all $\| A \|\leq 1$, we have $f(A^*A)\leq\| f \|$; on the other hand we have a sequence $A_i$ such that $f(A_i^*A_i)\to\| f \|$. So $\| f \|$ is an upper bound for the set of $f(A^*A)$ and in the same time, it is in the adherence of this set. So $\| f \|$ is the supremum of this set. 

Proof of \ref{itemvi_prop_invaddunit}. Unicity is clear because the prescription $\tilde f(\cun)=\| f \|$ gives $\tilde f$ on a basis of $\cA_{\cun}$. For existence, we pick $(\lambda,A)=\lambda+A\in\cA_{\cun}$ with $\lambda\in\eC$ and $A\in\cA$ and we pose $\tilde f(\lambda+A)=\lambda\| f \|+f(A)$. It is linear, extends $f$ and $\tilde f(\cun)=\| f \|$. To get positivity,
\begin{equation}
\begin{split}
 \tilde f\big( (\lambda+A)^*(\lambda+A) \big)&=f(A^*A-\overline{ \lambda }A+\lambda A^*)+| \lambda |^2\| f \|\\
        &=f(A^*A)+2\Reel\overline{ \lambda }f(A)+| \lambda |^2\| f \|\\
        &\geq f(A^*A)-2| \lambda |\| f \|^{\frac{ 1 }{2}}f(A^*A)^{\frac{ 1 }{2}}+| \lambda |^2\| f \|\\
        &=\big[ f(A^*A)^{\frac{ 1 }{2}}-| \lambda |\,\| f \|^{\frac{ 1 }{2}} \big]^{\frac{ 1 }{2}}\\&\geq 0.
\end{split}
\end{equation}
Let $g$ be a positive form on $\cA_{\cun}$ which extends $f$. The involutive Banach algebra of $\cA$ is extended to $\cA_{\cun}$ by setting $\cun=1$. Proposition \ref{prop_Dix214} and the fact that $g$ extends $f$ make $\| f \|\leq\| g \|=g(\cun)$. Hence
\[ 
  g\big( (\lambda+A)^*(\lambda+A) \big)\geq f\big( (\lambda+A)^*(\lambda+A) \big),
\]
and finally $g\geq\tilde f$.

Now we take $B\in\cA$ and we look at the form $g(A)=\tilde f(B^*AB)$. It is a positive form because
\[ 
  g(A^*A)=\tilde f(B^*A^*AB)=\tilde f\big( (AB)^*(AB) \big)\geq 0.
\]
For the norm of $g$, we have $\| g \|=g(\cun)=\tilde f(B^*B)$. Since $\| A^*A \|\cun\geq A^*A$, we find $\| A^*A \|f(\cun)\geq f(A^*A)$ and therefore
\[ 
  \| f \|\,\| A^*A \|\geq f(A^*A).
\]
We find
\[ 
  | f(B^*AB) |^2\leq f(B^*B)\| g \|\,\| A^*A \|,
\]
but $\| g \|=f(B^*B)$ and $A^*A\leq\| A \|^2$, so
\begin{equation}
| f(B^*AB) |\leq f(B^*B)\| A \|.
\end{equation}

Verification of \ref{itemvii_prop_invaddunit}. The pre-Hilbert structure defined by $\tilde f$ is the convergence notion $A_i\to A$ when $f(A_i)\to f(A)$. The first part of statement \ref{itemvii_prop_invaddunit} is thus just the fact that
\[ 
  \tilde f\big( (A_i-\cun)^*(A_i-\cun) \big)\to 0.
\]
\begin{probleme}
Pas fait le reste de la preuve. C'est dans \cite{Dixmier}.
\end{probleme}

\end{proof}


We consider an involutive Banach algebra $\cA$; $f$, a positive linear continuous form on $\cA$ and $\tilde f$ its extension to $\cA_{\cun}$ by proposition \ref{prop_invaddunit}. We claim that 
\[ 
  \{ A\in\cA\tq \tilde f(A^*A)=0 \}
\]
is a left ideal. Indeed let us show that $BA$ belongs to this set when $A$ does. We use \ref{itemii_prop_invaddunit} of proposition \ref{prop_invaddunit} to compute the norm on both side of the identity
\[ 
  \tilde f\big( (BA)^*(BA) \big)=\tilde f(A^*B^*BA).
\]
We find
\[ 
  | \tilde f\big( A^*(B^*B)A \big) |\leq \| B^*B \|f(A^*A)=0.
\]
Thus $\tilde f\big( A^*(B^*B)A \big)=0$.


\begin{definition}      \label{DefStateCSA}
A \defe{state}{state!on non-unital $C^*$-algebra} on a $C^*$-algebra is a linear map $\dpt{\omega}{\cA}{\eC}$ which is positive and has norm $1$.\label{def:etatnon}
\end{definition}
Proposition \ref{prop:posborn} shows that a state is bounded; then one can impose the condition $\| \omega \|=1$. From proposition \ref{prop:linposboun}, a state in the sense of definition \ref{def:etatnon} on an unital  $C^*$-algebra is a state in the sense of definition  \ref{DefStateUnital} because $1=\| \omega \|=\omega(\cun)$. 

\begin{proposition}
A state on a non unital $C^*$-algebra has an unique extension on the unitized algebra. \label{prop_st_unit_ext}
\end{proposition}
\lref{2.8.7}

\begin{proof}
Let $\omega$ be a state on the $C^*$-algebra $\cA$ and let us define the extension
\[ 
  \omega_{\cun}(A+\lambda\cun):=\omega(A)+\lambda
\]
 on $\cA_{cun}$. It is clear that $\omega_{cun}=1$. We have now to show that $\omega_{cun}(A)\geq 0$ whenever $A\in\cA^+$. On $\cA$, the form $\omega$ is a state and the is bounded; let $(\cun_{\lambda})$ be an approximate unit in $\cA$. Since $\lim_{\lambda\to\infty}\| A-\cun_{\lambda}A \|=0$, we have $| \omega(A-A\cun_{\lambda}) |\to 0$. Let us use \eqref{eq:omABleq} and \eqref{eq:omABleqs} in the particular case $A=B$:
\[ 
  | \overline{\omega(A)}\omega(A) |^2\leq\omega(A^*A)^2,
\]
but since $\omega$ is positive, the right hand side is a positive number and we can remove the square in both sides; using the fact that for any complex number, $| z\overline{z} |=| z |^2$, we find
\[ 
  | \omega(A)^2 |\leq \omega(A^*A).
\]
Then, with $\omega_{\cun}$, we find
\[ 
  \omega_{\cun}\big( (A+\lambda\cun)^*(A+\lambda\cun) \big)\geq | \omega(A+\lambda\cun) |^2=| \omega(A)+\lambda |^2,
\]
but $(A+\lambda\cun)^*(A+\lambda\cun)$ is a general element of $\cA_{\cun}$. It proves that $\omega$ is positive.
 
\end{proof}


The following result gives a lot of examples of states.

\begin{lemma}
 For all $A\in \cA$ and $a\in\sigma(A)$, there exists a state $\omega_a$ on $\cA$ such that $\omega_a(A)=a$. When $A=A^*$, we can find a state $\omega$ such that $| \omega(A) |=\| A \|$. \label{lem:omAenomA}
\end{lemma}
\lref{2.8.2}


\begin{proof}
If $\cA$ has no unit, we add one. We define $\dpt{\tilde\omega}{\eC A+\eC\cun}{\eC}$ by $\tilde\omega(\lambda A+\mu\cun):=\lambda a+\mu$. Since $a\in\sigma(A)$, we know that $\lambda a+\mu\in\sigma(\lambda A+\mu\cun)$ and then that $\lambda A+\mu\cun-(\lambda  a+\mu)\cun$ has no inverse. 

When $\cA$ has an unit, we know that $\omega\in\Delta(\cA)\Rightarrow | \omega(A) |\leq \| A \|$. We want to get the same for our $\tilde\omega$. Looking at the proof of \eqref{eq:omAleqnA}, we see that we have to prove that $\tilde\omega(x)\neq 0$ whenever $x$ is invertible. Be careful on a point: the question is posed in the $C^*$-algebra $\eC A+\eC\cun$, not in $\cA$ or $\cA_{\cun}$. So we take an invertible element $\lambda A+\mu\cun$ . Then $-\mu/\lambda\neq a$ because it is not in $\sigma(A)$. So $\tilde\omega_a(\lambda A+\mu\cun)=\lambda a+\mu\neq 0$ and we can affirm that
\begin{equation}
| \tilde\omega_a(\lambda A+\mu\cun) |\leq \| (\lambda A+\mu\cun) \|.
\end{equation}
Since $\tilde\omega_a(\cun)=1$, we know that $\| \tilde\omega_a \|\geq 1$, but the equation above shows that $\| \tilde\omega_a \|\leq 1$. We conclude that $\| \tilde\omega_a \|=1$.

Hahn-Banach theorem \ref{tho:hahnBanach} gives an extension $\omega_a$ of $\tilde\omega_a$ to the whole $\cA$ with norm $1$. 
Point \ref{7125aii} of proposition \ref{prop:linposboun} shows that $\omega_a$ is a state on $\cA_{\cun}$ with $\omega_a(A)=\tilde\omega_a(A)=a$. 

For the second assertion, we know that $\sigma(A)$ is a closed set, then there exists a $a\in\sigma(A)$ such that $r(A)=| A |$. For this $a$, $| \omega(A) |=| a |=r(a)=\| A \|$ because $\| A \|=r(A)$ when $A=A^*$. 

Now if $\cA$ has no unit, we consider as $\omega_a$, the restriction to $\cA$ of the $\omega_a$ that we build on $\cA_{\cun}$.

\end{proof}

%+++++++++++++++++++++++++++++++++++++++++++++++++++++++++++++++++++++++++++++++++++++++++++++++++++++++++++++++++++++++++++
\section{Uniqueness of the norm}
%+++++++++++++++++++++++++++++++++++++++++++++++++++++++++++++++++++++++++++++++++++++++++++++++++++++++++++++++++++++++++++

\begin{proposition}         \label{prop:unicitenormcsa}
The norm on a $C^*$-algebra is unique in the sense that if $\cA$ is a $C^*$-algebra, one cannot find an other norm on $\cA$ (as Banach algebra) for which $\cA$ is a $C^*$-algebra.

\end{proposition}

\begin{proof}
    Let us consider an element such that $A=A^*$. Point \ref{ItemSpecffSpecThoSpectral} of theorem \ref{ThoSpectralTho} applied to $f=\id_{\sigma(A)}$, and the fact that $\| f \|_{\infty}=r$ imply that $\| A \|=r(A)$. 

    Considering this argument with a general $A^*A$ instead of a particular $A$, we find that
    \begin{equation}
    \| A \|=\sqrt{r(A^*A)}.
    \end{equation}
    Since the spectral radius is determined by the Banach algebra structure only, this formula shows that the norm on a general element $A$ is unique.

\end{proof}

\begin{remark}
The way to complete a $*$-algebra is not unique. Thus one can construct several $C^*$-algebra from a given $*$-algebra.
\end{remark}

%+++++++++++++++++++++++++++++++++++++++++++++++++++++++++++++++++++++++++++++++++++++++++++++++++++++++++++++++++++++++++++
\section{State on unital $C^*$-algebra}
%+++++++++++++++++++++++++++++++++++++++++++++++++++++++++++++++++++++++++++++++++++++++++++++++++++++++++++++++++++++++++++

\begin{definition}      \label{DefStateUnital}
A \defe{state}{state} on an \emph{unital} $C^*$-algebra $\cA$ is a linear map $\dpt{\omega}{\cA}{\eC}$ which is

\begin{itemize}
\item positive: $\omega(A)\geq 0$ for every $A\in\cA^+$, 
\item normalised: $\omega(\cun)=1$.
\end{itemize}
We say that the state $\omega$ is \defe{faithful}{faithful!state} if $\omega(A^*A)=0$ implies $A=0$. We denote by $\etS(\cA)$ the space of all states on $\cA$.
\end{definition}
As an example, consider $\cA\subset\oB(\hH)$, a $C^*$-subalgebra of the bounded operators on the Hilbert space $\hH$. Then any $\psi\in\hH$ such that $\| \psi \|=1$ defines a state on $\cA$ by $\psi(A)=\scal{\psi}{A\psi}$. Indeed if $A\in\cA^+$, there exists a $B$ such that $A=B^*B$ and from definition of the involution on $\oB(\hH)$, we have $\psi(B^*B)=\| B\psi \|^2\geq 0$.

When a state is not faithful, the set on which $\omega(a^*a)=0$ is a vector space, and one usually takes the quotient of $\cA$ by that subspace.

As an example, let us prove the following.
\begin{proposition}                 \label{Propstaretattraces}
Every state in the $*$-algebra $\eM_n(\eC)$ of $n\times n$ matrices with complex coefficients read
\begin{equation}
\varphi(a)=\tr(as)
\end{equation}
for a certain positive matrix $s\in\eM_{n}(\eC)$ with $\tr(s)=1$. Moreover, every functional of that form is a state.
\end{proposition}

\begin{proof}
    Let us first prove the second claim. Let $s$ be a positive matrix with trace equal to $1$, we have
    \[ 
      \varphi(a^*a)=\tr(a^*as)=\tr(a^*as^{1/2}s^{1/2})=\tr(s^{1/2}a^*as^{1/2})
    \]
    which is positive. Here we used the fact that every positive element reads $a^*a$ and positivity of $s$ to define $s^{1/2}$.

    By linearity, a state on $\eM_n(\eC)$ must read $\varphi(a)=\sum_i\sum_j\varphi_{ij}a_{ij}$ for some coefficients $\varphi_{ij}$, so that $\varphi(a)=\tr(sa)$ where $s$ is the matrix of $(\varphi_{ij})$. The condition $\varphi(\mtu)=1$ immediately imposes $\tr(s)=1$. Now we study $\varphi(a^*a)=\tr(sa^*a)$. In the basis which diagonalises $a^*a$, we have $(a^*a)_{ij}=\delta_{ij}n_i$ (no sum) with $n_i>0$.  Thus 
    \[ 
      \varphi(a)=\sum_k(sa^*a)_{kk}=\sum_ks_{kk}n_k
    \]
    which must be positive for every choice of positive $n_k$. That imposes for all $s_{kk}$ to be positive, so that $s$ is a positive matrix.
\end{proof}

Notice that $s$ being positive, it is diagonalisable, so that it is nothing else than a list of positive numbers.

\subsection{Convexity}
%---------------------

A \defe{convex}{convex!subset} subspace $C$ of a vector space $V$ is a subset of $V$ such that for all $v$, $w\in C$ and $\lambda\in[0,1]$, we have $\lambda v+(1-\lambda)w\in C$.

If $p_i>0$, $\sum_ip_i=1$ and $v_i\in C$, then $\sum_i v_i\in C$.

\begin{lemma}
If $\cA$ is an unital $C^*$-algebra, then the space $\etS(\cA)$ of all states on $\cA$ is convex.
\end{lemma}

\begin{proof}
If $\omega$ and $\eta$ are states, then they are positive and $\omega(\cun)=\eta(\cun)=1$. Then $\lambda\omega+(1-\lambda)\eta$ is also positive and $\lambda\omega(\cun)+(1-\lambda)\eta(1)=1$.
\end{proof}

\begin{proposition}
In the non unital case, $\etS$ is convex too.
\end{proposition}

\begin{proof}
Let $\omega$ and $\eta$ be two states on a non unital $C^*$-algebra $\cA$ and $\omega_{cun}$, $\eta_{cun}$ their extensions to $\cA_{\cun}$. From lemma, $\xi_{cun}=\lambda\omega_{cun}+(1-\lambda)\eta_{cun}$ is a state on $\cA_{\cun}$. We want the restriction $\xi=\lambda\omega+(1-\lambda)\eta$ to be a state on $\cA$.

Since $\xi_{cun}(A+\mu\cun)=\lambda\omega(A)+(1-\lambda)\eta(A)+\mu$, the restriction reads
\[ 
  \xi(A)=\lambda\omega(A)+(1-\lambda)\eta(A).
\]
Then the unique extension of $\xi$ with same norm is precisely $\xi_{cun}$. It proves that $\| \xi \|=1$.
\end{proof}

The dual $\cB^*$ of a Banach space $\cB$ is the space of functional $\dpt{\rho}{\cB}{\eC}$ and a functional is by definition linear and continuous. If $\cA$ is unital, proposition \ref{prop:linposboun} ensures that $\etS(\sA)\subset\cA^*$. Let us recall the $w^*$-limit\index{$w^*$-limit} in $\cA^*$. We say that $\omega_n\to \omega$ when for all $v\in\cA$, $\omega_n(A)\to \omega(A)$. It is clear that the $w^*$-limit preserves the positivity: if $\omega_n\geq0$ and $\omega_n\to \omega$, then $\omega\geq 0$. This proves that $\etS(\cA)$ is closed in $\cA^*$ for the $w^*$-topology. From normalization $\| \omega \|=1$, we conclude  that $\etS(\cA)$ is a closed set in the unit ball of $\cA^*$. From Banach-Alaoglu theorem, $\etS(\cA)$ is compact. What is proved is

\begin{proposition}
The space of states of an unital $C^*$-algebra is convex and compact.
\end{proposition}

The simplest example is $\cA=\eC$. From $\omega(1)=1$ and linearity, we deduce $\omega(i)=i\omega(1)=i$. Then $\omega(a+bi)=a+bi$ is the only state on $\eC$. An isolated point is compact and convex. Let us now consider a less trivial example.

We now consider $\cA=\eC^2$. We see $\eC^2$ as $\eC\oplus\eC$ and we write $z\dot + z'$ the element $(z,z')\in\eC^2$. We define the product by
\[ 
  (\lambda_1\dot +\mu_1)(\lambda_2\dot +\mu_2)=\lambda_1\lambda_2\dot +\mu_1\mu_2.
\]
From characterization $\cA^+=\{ B^*B\tq B\in\cA \}$, a generic element in $\cA^+$ reads
\[ 
  (\bar\lambda\dot +\bar\mu)(\lambda\dot +\mu)=\bar\lambda\lambda\dot +\bar\mu\mu.
\]
So positive elements in $\eC^2$ are $(\lambda,\mu)$ with $\lambda,\mu\geq 0$. In order for $\omega$ to be positive, we need $\omega(\lambda,\mu)=c\lambda+d\nu\geq 0$ for all $\lambda,\mu\geq0$. For normalization $\omega(1,1)=1$, we also need $c+d=1$. Then $\etS(\eC^2)$ can be identified with $[0,1]$ which is convex and compact.

\section{Representation}
%+++++++++++++++++++++++

\subsection{Representation of involutive algebra}
%-------------------------------------------------


Let $\cA$ be an involutive algebra and $H$ a Hilbert space. A \defe{representation}{representation!of involutive algebra} of $\cA$ in $H$ is a map $\pi\colon \cA\to \mL(H)$ such that
\begin{equation}
\begin{aligned}
  \pi(A+B)&=\pi(A)+\pi(B),&\pi(\lambda A)&=\lambda\pi(A),\\
\pi(AB)&=\pi(A)\circ\pi(B),&\pi(A^*)&=\pi(A)^*.
\end{aligned}
\end{equation}

A linear form $f\colon \cA\to \eR$ on the involutive algebra $\cA$ is \defe{positive}{positive!form on involutive algebra} if $f(A)\geq 0$ whenever $A>0$.

\begin{lemma}
If $\rho\colon \eM_n(\eC)\to \End(V)$ is a representation of the matrix algebra $\eM_n(\eC)$ on the finite dimensional space $V$, then there exists an isomorphism $V\to \eC^n\oplus\ldots\oplus\eC^n$ which intertwines $\rho$ to a multiple of the standard representation of the matrices on $\eC^n$.
\end{lemma}


\begin{proposition}
Let $\cA$ be an involutive algebra. We have
\begin{enumerate}
\item \label{itemi_prop_invalgrepr} If $\pi$ is a representation of $\cA$ in $H$ and if $\xi\in H$, then $A\mapsto\scal{ \pi(A)\xi }{ \xi }$ is a positive form on $\cA$.
\item  \label{itemii_prop_invalgrepr} Let $\pi$ and $\pi'$ be two representations of $\cA$ in $H$ and $H'$ respectively, and $\xi,\xi'$, two corresponding totalizing vectors. If $\scal{ \pi(A)\xi }{ \xi }=\scal{ \pi'(A)\xi' }{ \xi' }$ for all $A\in\cA$, then there exists an unique isometry $H\to H'$ which transforms $\pi$ into $\pi'$ and $\xi$ into $\xi'$, i.e.
\[ 
 \begin{split}
U\pi(A)U^{-1}&=\pi'(A)\\
U\xi&=\xi'
\end{split} 
\]
\end{enumerate}
 \label{prop_invalgrepr}
\dixref{2.4.1}
\end{proposition}


\begin{proof}
 For the first point, it is easy:
\[ 
  \scal{ \pi(A^*A)\xi }{ \xi }=\scal{ \pi(A^*)\pi(A)\xi }{ \xi }=\scal{ \pi(A)\xi }{ \pi(A)\xi }=\| \pi(A)\xi \|^2\geq 0.
\]
We used property $\pi(A^*)=\pi(A)^*$.

For the second point, $\overline{ \pi(\cA)\xi }=H$ because $\xi$ is totalizing. We have
\begin{equation} \label{eq_piAxipreis}
\scal{ \pi(A)\xi }{ \pi(B)\xi }=\scal{ \pi(B^*A)\xi }{ \xi }
        =\scal{ \pi'(B^*A)\xi' }{ \xi' }
        =\scal{ \pi'(A)\xi }{ \pi'(B)\xi' },
\end{equation}
but vectors of the form $\pi(A)\xi$ are everywhere dense in $H$ (and $\pi'(A)\xi'$ in $H'$), so we have an isomorphism
\begin{equation}
\begin{aligned}
 U\colon H&\to H' \\ 
\pi(A)\xi&\mapsto \pi'(A)\xi' 
\end{aligned}
\end{equation}
Equation \eqref{eq_piAxipreis} shows that $U$ is an isometry; it is linear because of linearity of $\pi$. The map $U$ is surjective because $\xi'$ is totalizing and injective because if $\pi'(A)\xi'=\pi'(B)\xi'$, $\pi'(A-B)=0$ which proves that $A=B$ because $\pi'$ is linear.

Now we check that this $U$ is the searched map. For all $A$, $B\in\cA$, 
\begin{equation}
\begin{split}
  \big[ U\circ\pi(A) \big](\pi(B)\xi)&=U\pi(AB)\xi=\pi'(AB)\xi'\\
        &=\pi'(A)\pi'(B)\xi'=\big[ \pi'(A)\circ U \big](\pi(B)\xi),
\end{split}
\end{equation}
so $U$ transforms $\pi$ into $\pi'$. In order to prove that $U\xi=\xi'$, we will use the fact that $U$ is an isometry:
\[ 
 \scal{ \xi' }{ \pi'(A)\xi' }=\scal{ \xi }{ \pi(A)\xi }
        =\scal{ U\xi }{ U\pi(A)\xi }
        =\scal{ U\xi }{ \pi'(A)\xi' }.
\]

For unicity, notice that equation $U\big( \pi(A)\xi \big)=\pi'(A)\xi'$ defines $U$ on a dense subspace of $H$. Continuity finishes to fix $U$.

\end{proof}

With the same notations,  the map $A\to\scal{ \pi(A) }{ A }$ is the form \defe{associated}{associated form with a representation} with representation $\pi$ and the vector $\xi$. Let $\cB$ be an involutive subalgebra of $\mL(H)$ and $\xi$, any element in $H$. We denote by $\omega_{\xi}$ the form on $\cB$ defined by
\begin{equation}
\omega_{\xi}(A)=\scal{ A\xi }{ \xi }
\end{equation}
for all $A\in\cB$. A positive form $\eta$ on $\cB$ is a \defe{vector}{vector!form} form if there exists a $\xi\in H$ such that $\eta=\omega_{\xi}$.

\begin{proposition}
Let $\cA$ be an involutive Banach algebra with approximate unit $(u_i)$,$\pi$ a nondegenerate representation of $\cA$ in $H$, $\xi\in H$ and $f$ the positive form defined from $\pi$ and $\xi$. We have
\begin{equation}
\| f \|=\scal{ \xi }{ \xi }.
\end{equation}
\dixref{2.4.3}
\end{proposition}

\begin{proof}
Point \ref{itemv_prop_invaddunit} of proposition  \ref{prop_invaddunit} states that if $(u_i)_{i\in J}$ is an approximate unit in $\cA$, then 
\[ 
  f(u_j)\to\| f \|\quad\text{and}\quad f(u_j^*u_j)\to \| f \|.
\]
Hence $\| f \|=\lim f(u_j)=\lim\scal{ \pi(u_j)\xi }{ \xi }$.

\end{proof}

\begin{proposition}
Let $\cA_{\cun}$ be the involutive algebra deduced from $\cA$ by adding an unit and $\pi$, a representation of $\cA$ in $H$. There exists one and only one way to extend $\pi$ to a representation $\pi_{(\cun)}$ of $\cA_{\cun}$ in such a way that $\pi_{(\cun)}(\cun)=\id$.
\end{proposition}
The representation $\pi_{(\cun)}$ is the \defe{canonical extension}{canonic!extension of a representation} of $\pi$. When $\pi$ is a representation of $\cA$ in $H$, the set
\begin{equation}
 K=\{ \pi(A)\xi\tq A\in\cA,\xi\in H \}
\end{equation}
is a closed vector subspace of $H$. We say that $K$ is the \defe{essential subspace}{essential subspace of a representation} of $\pi$. The representation is \defe{nondegenerate}{nondegenerate!representation}\index{representation!nondegenerate} if $K=H$.

\begin{proposition}
Let $\cA$ be an involutive Banach algebra and $\pi$, a nondegenerate representation of $\cA$ on $H$. If $(u_i)$ is an approximate unit in $\cA$, then $\pi(u_i)$ strongly converges to $\id$ (see subsection \ref{subsec_topomL}). 
\end{proposition}

\begin{proof}
We have to prove that $\| \pi(u_i)\xi-\xi \|\to 0$ for any $\xi\in H$. From non degeneracy, the set $\{ \pi(B)\xi\, , B\in \cA \}$ is total in $H$, so it is sufficient to prove the convergence for $\xi$ of the form $\pi(B)\xi$. We have
\[ 
  \| \pi(u_iB)-\pi(B) \|\leq\| u_iB-B \|\to 0
\]
because $u_i$ is an approximate unit and relation \eqref{eq_morleqpi}. 

The fact that $\pi$ is nondegenerate makes
\[ 
  \{ \pi(A)\xi\tq A\in\cA,\,\xi\in H \}=H.
\]
If $\mU$ is an open set in $\mL(H)$ around $\id$, we have to prove that $\pi(u_i)\in\mU$ for all $i\geq i_0$. Open sets are taken in the sense of seminorms $s_{\xi}=\| A\xi \|$. The balls ---which are not open--- are of the form
\[ 
 \begin{split}
B(A;(\xi_j),(r_j))&=\{ X\in\mL(H)\tq s_{\xi_i}(X-A)<r_j\,\forall j \}\\
        &=\{ X\in\mL(H)\tq \| X\xi_j-A\xi_j \|<r_j\,\forall j \}.
\end{split} 
\]
If a sequence fall into the balls $B(A;(\xi_j),(r_j))$ for a fixed $A$ and if we consider an open set around this $A$, the latter open set will contain at least one of the $B(A;(\xi_j),(r_j))$, hence the sequence will fall into this open set too. So we have to prove that for all \emph{finite} sequence $(\xi_j)$ and $r_j$ ($\xi_j\in H$ and $r_j\in\eR^+$),
\[ 
  \pi(u_i)\in B(\id;(\xi_j),(r_j))
\]
when $i$ is large enough.

Let $B\in\cA$ and $\xi\in H$, we have already proved that $\| \pi(u_iB)-\pi(B) \|\leq\| u_iB-B \|\to 0$. We have to prove that $\| \pi(u_i)\xi_j-\xi_j \|< r_j$. Since $\pi(C)\zeta$ is a total set, by redefinition of $\xi_j$, we can write $\xi_j$ under the form $\pi(B_j)\xi_j$. So
\[ 
  \| \pi(u_i)\pi(B_j)\xi_j-\pi(B_j)\xi_j \|=\| \pi(u_iB_j)\xi_j-\pi(B_j)\xi_j \|=\| [\pi(u_iB_j)- \pi(B_j)]\xi_j \|,
\]
but we know that when $A$ is a bounded operator on a Hilbert space, $\| Av \|\leq \| A \|\,\| v \|$. Thus we have
\[ 
\| \pi(u_i)\pi(B_j)\xi_j-\pi(B_i)\xi_j \|=\| [\pi(u_iB_j)-\pi(B_i)]\xi_j \|
        =\| \pi(u_iB_j)-\pi(B_i) \|\,\| \xi_j \|.
\]
Since the sequence $(\xi_j)$ is finite, one can bound $\| \xi_j \|$ by a certain $M$, hence
\[ 
  \| \pi(u_i)\pi(B_j)\xi_j-\pi(B_i)\xi_j \|\leq M\| \pi(u_iB_j)-\pi(B_i) \|.
\]
When $i$ is large, the latter is as small as we want and in particular it can become smaller than all the $r_j$ of the sequence. This proves that $\pi(u_i)$ strongly converges to $\id$ in $\mL(H)$.
\end{proof}

% VNVNVNVN
% La proposition [Another version of GNS construction]      \label{PropGNSanother}
% était ici et a été déplacée pour les besoins du cours d'alg de VN.

\subsection{Cyclic representations of \texorpdfstring{$C^*$}{C*}-algebra }
%-----------------------------------------------------------------------

\begin{definition}
A \defe{representation}{representation!of a $C^*$-algebra} of the $C^*$-algebra $\cA$ on a Hilbert space $\hH$ is a linear map $\dpt{\pi}{\cA}{\oB(\hH)}$ such that

\begin{enumerate}
\item $\pi(AB)=\pi(A)\circ\pi(B)$,
\item $\pi(A^*)=\pi(A)^*$
\end{enumerate}
for all $A$, $B\in\cA$. 
\end{definition}
Most of time, we will denote a representation by the pair $(\pi,\hH)$. When the represented $C^*$-algebra is ambiguous, we write $(\cA,\pi,\hH)$.

\begin{lemma}       \label{Lemrepresnormpresou}
A representation $\pi$ is continuous and fulfills
\begin{equation}
\| \pi(A) \|\leq \| A \|,
\end{equation}
moreover when the representation is faithful, we have $\| \pi(A) \|=\| A \|$.
\end{lemma}

\begin{proof}
The spaces $\cA$ and $\oB(\hH)$ are $C^*$-algebra and $\pi$ is a morphism proposition \ref{prop:vp_leq} concludes. The second claims follows from lemma \ref{lem:injmorpisom}.
\end{proof}

Let $(\pi_1,\hH_1)$ and $(\pi_2,\hH_2)$ be two representations. They are \defe{equivalent}{equivalence!of representation of $C^*$-algebra } when there exists an unitary isomorphism $\dpt{U}{\hH_1}{\hH_2}$ such that
\begin{equation}
  U\pi_1(A)U^*=\pi_2(A)
\end{equation}
for all $A\in\cA$.

A representation $(\cA,\pi,\hH)$ is \defe{nondegenerate}{degenerated representation of $C^*$-algebra} if $0$ is the only vector to be cancelled by all $\pi(A)$. It is \defe{cyclic}{cyclic!representation} if there exists a \defe{cyclic vector}{cyclic!vector} $\Omega\in\hH$, i.e. the closure of $\pi(\cA)\Omega=\hH$.


\begin{lemma}
If $\pi$ is an irreducible representation on $\hH$, then any non zero vector is cyclic.
\end{lemma}

\begin{proof}
Let $v\neq 0$; if it were not cyclic, then $\overline{ \pi(\cA)v}$ should be a proper invariant  subspace of $\hH$.
\end{proof}

\subsection{Primitive spectrum}
%------------------------------
For this short note about primitive spectrum, we follow \cite{Landi}.

When $\cA$ is any (not specially commutative) $BC^*$-algebra, the \defe{primitive spectrum}{primitive spectrum}\index{spectrum!primitive} of $\cA$ is the set $\Prim\cA$\nomenclature{$\Prim$}{Primitive spectrum} of kernels of $*$ irreducible representations. An element in $\Prim\cA$ is a two-sided ideal.

There exists a suitable topology on this space, the \defe{hull-kernel topology}{hull-kernel topology} or \defe{Jacobson topology}{Jacobson topology}\index{topology!hull-kernel}\index{topology!Jacobson}. This is given by means of closure. If $W\subset\Prim\cA$, then we define the closure of $W$ by
\begin{equation}
  \overline{ W }=\{ \mI\in\Prim\cA\tq \cap W\subseteq \mI \}.
\end{equation}
where $\cap W=\cap_{\mJ\in W}\mJ$. The inclusion $\cap W\subseteq\mI$ is an inclusion of subsets of $\cA$.

\begin{proposition}
This definition defines a topology. Namely, it fulfils the Kuratowsky axioms:

\begin{enumerate}
\item\label{enu802i} $\overline{ \emptyset }=\emptyset$,
\item \label{enu802ii} $W\subseteq\overline{ W }$,
\item \label{enu802iii} $\overline{ \overline{ W } }=\overline{ W }$,
\item \label{enu802iv} $\overline{ W_1\cup W_2 }=\overline{ W_1 }\cup\overline{ W_2 }$.
\end{enumerate}
\end{proposition}

\begin{proof}
Points \ref{enu802i}  and \ref{enu802ii} are trivial. For \ref{enu802iii}, remark that $\cap W=\cap\overline{ W }$ (equality as subsets of $\cA$). Indeed, consider $A\in\cap W$. Then any $\mJ\in\overline{ W }$ contains $A$ and then $\cap\overline{ W }$ contains $A$. Now if $A\in\cap\overline{ W }$, then any $\mI$ such that $\cap W\subseteq\mI$ contains $A$ and then $A\in\cap W$.
 
The proof of \ref{enu802iv} is more complicated. If $V\subset W$, then $(\cap W)\subseteq(\cap V)$ and then $\overline{ V }\subseteq\overline{ W }$. Then $\overline{ W_i }\subseteq\overline{ W_1\cup W_2 }$ for each of $i=1,2$.

The inverse inclusion is as follows. Let $\mI$ be an ideal kernel of the irreducible representation $\pi$ of $\cA$ on the Hilbert space $\hH$. Suppose that $\mI\notin \overline{ W_1 } \cup \overline{ W_2 }$. Then we will prove that $\mI\notin\overline{ W_1\cup W_2 }$. There exists $A\in W_1$ and $B\in W_2$ such that $\pi(A)\neq 0$ and $\pi(B)\neq0$. Let $\xi\in\hH$ such that $\pi(A)\xi\neq0$. Since $\pi$ is irreducible, then $\pi(A)\xi$ is cyclic. Since $\pi(B)\neq 0$, then there exists a $\psi\in\hH$ such that $\pi(B)\psi\neq0$. Cyclicity of $\pi(A)\xi$ shows that there exists a $C$ such that $\pi(C)\pi(A)\xi$ is sufficiently close to $\psi$ to satisfy
\[ 
  \pi(B)\big( \pi(C)\pi(A)\xi \big)\neq0.
\]
 Then $BCA\notin\ker\pi=\mI$. Since $W_i$ are ideals, 
\[ 
  BCA\in(\cap W_1)\cap(\cap W_2)=\cap(W_1\cup W_2). 
\]
Then $BCA\in W_1$ and $\cap(W_1\cup W_2)\nsubseteq\mI$. Consequently, $\mI\notin\overline{ W_1\cap W_2 }$.


\end{proof}


\subsection{GNS construction}
%----------------------------

\begin{lemma}
Let $\mfM$ be a $*$-algebra in $\oB(\hH)$, $\psi\in\hH$ and $p$, the projection into the closure of $\mfM\psi$. Then $p\in\mfM'$, i.e. $[p,A]=0$ for all $A\in\mfM$.\label{lem_preGNS}
\end{lemma}

\begin{proof}
Let $A\in\mfM$; by definition of $p$, we have $Ap\hH\subseteq p\hH$. For a $A\in\mfM$, $Ap\hH=\{ AB\psi\tq B\in\mfM \}$, but $\mfM$ is an algebra, then $AB\in\mfM$ and $Ap\hH\subseteq\mfM\psi\subseteq p\hH$. If we define $p^{\perp}=\mtu-p$, we find $p^{\perp}Ap=0$.

 Indeed $(\mtu-p)Ap=Ap-pAp$ and $Apx-pApx$ can be computed by setting $x=B\psi$ for a certain $B$ in the closure of $\mfM$ (the part of $x$ ``outside'' $\mfM$ has no importance). Then
\begin{equation}
\begin{aligned}
   ApB \psi-pAB\psi&=AB\psi-pAB\psi  &&\textrm{because $B\psi\in\mfM$}\\
        &=AB\psi-AB\psi  &&\textrm{because $AB\psi\in\mfM\psi$}\\
        &=0.
\end{aligned}
\end{equation}
It shows that $ApB\psi-pAB\psi=0$ and then that $p^{\perp}Ap=0$. 

From this, we see that $Ap=pAp$. Let us now consider $A=A^*$ (for a general element in $\cA$, use the decomposition). We have $(Ap)^*=p^*A^*=pA$, but $(Ap)^*=(pAp)^*=pAp=Ap$. Then $[A,p]=0$.


\end{proof}


\begin{proposition}
Any nondegenerate representation is direct sum of cyclic representations.
\end{proposition}

\begin{proof}
We apply the lemma with $\mfM=\pi(\cA)$. The non degeneracy of $\pi$ makes $p$ non zero: $\mfM\psi$ is never zero. Now we consider the map $\dpt{\rho}{\cA}{\oB(\hH)}$, $\rho(A)=p \pi(A)$. This is a representations because
\begin{equation}
    \rho(AB)=p\pi(A)\pi(B)
        =p\pi(A)p\pi(B)
        =\rho(A)\rho(B),
\end{equation}
and
\begin{equation}
\begin{split}
\rho(A^*)&=p\pi(A^*)=\pi(A^*)p\\
        &=\pi(A)^*p=(p\pi(A))^*=\rho(A)^*.
\end{split}
\end{equation}
More precisely, $\rho$ is a representation on $p\hH$ and when $\pi(A)\in p\hH$, we have $\rho(A)=\pi(A)$. The representation $\rho$ is constructed in such a way that $\psi$ is a cyclic vector:
\[ 
  \rho(A)\psi=p\hH.
\]
The same construction with $\psi_2\in p^{\perp}\hH$ gives and going on gives the thesis.
\end{proof}


Let $(\cA,\pi,\hH)$ be a nondegenerate representation and $\Psi\in\hH$ a vector with norm $1$. The state $\psi$ given by formula
\[ 
  \psi(A)=\scal{\Psi}{\pi(A)\Psi}
\]
is the \defe{vector state}{vector!state}\index{state!vector} of $\Psi$ relative to $\pi$.

\begin{theorem}[GNS construction]       \label{ThoGNScontruction}
Let $\cA$ be an unital $C^*$-algebra  and $\dpt{\omega}{\cA}{\eC}$, a state

\begin{enumerate}
\item\label{GNSi} There exists a Hilbert space $\hH$, a representation $\dpt{\pi_{\omega}}{\cA}{\oB(\hH_{\omega})}$ and a cyclic unit vector $\Omega_{\omega}$ such that
\begin{equation}
  \omega(A)=\scal{\Omega_{\omega}}{\pi_{\omega}(A)\Omega_{\omega}}
\end{equation}
for all $A\in\cA$.
\item\label{GNSii} The triple $(\hH_{\omega},\pi_{\omega},\Omega_{\omega})$ is unique up to isomorphism in the following sense. Let $\hH$ be a Hilbert space, $\dpt{\pi}{\cA}{\oB(\cA)}$ a representation and $\Omega\in\hH$ an unit cyclic vector such that $\omega(A)=\scal{\Omega}{\pi(A)\Omega}$ for all $A\in\cA$; then there exists an unitary isomorphism $\dpt{u}{\hH_{\omega}}{\hH}$ such that $u(\Omega_{\omega})=\omega$ and
\[ 
  \pi(A)=u\pi_{\omega}(A)u^*
\]
for all $A\in \cA$.

\end{enumerate}
\label{tho:GNS}
\end{theorem}

\begin{proof}
From \eqref{eq:omABleq} we know that, if $A\in\mN_{\omega}$ and $B\in\cA$, then $\omega(B^*A)=0$. So we can define $\mN$ in the two equivalent ways:
\begin{equation}
    \begin{aligned}[]
        \mN_{\omega}&=\{ A\in\cA\tq \omega(A^*A)=0 \}\\
        &=\{ A\in\cA\tq \omega(B^*A)=0\textrm{ for all $B\in\cA$} \}.
    \end{aligned}
\end{equation}
From the second line, we see that $\mN_{\omega}$ is an ideal in $\cA$. The set $\mN_{\omega}$ is closed from continuity of $\omega$. We use the product defined by equation \eqref{eq:defprodetat}: $(A,B):=\omega(A^*B)$. It defines a sesquilinear form on the quotient $\cA/\mN_{\omega}$
\begin{equation}
\scal{ VA }{VB}=\omega(A^*B)
\end{equation}
where $\dpt{V}{\cA}{\cA/\mN_{\omega}}$ is the canonical projection $VA=A+\mN_{\omega}$. So $\cA/\mN_{\omega}$ is a pre-Hilbert space from which we build $\hH_{\omega}$ by completion. We define the cyclic vector $\Omega_{\omega}=V\cun\in\hH_{\omega}$.

For each $A\in\cA$, we define $\dpt{L_A}{\cA/\mN_{\omega}}{\cA/\mN_{\omega}}$ by
\begin{equation}  \label{eq:defpiomega}
L_AVB=V(AB). 
\end{equation}
We have
\begin{equation}
\begin{aligned}
\| VAB \|^2&=\scal{VAB}{VAB}^2\\
        &=\omega(B^*A^*AB)^2\\
        &\leq \| A \|^4\omega(B^*B)^2&&\textrm{corollary \ref{cor:BeAAeB}}\\
        &=\| A \|^4\| VB \|^2,
\end{aligned}
\end{equation}
then for all $\psi\in\cA/\mN_{\omega}$, $\| L_A\psi \|\leq \| A \|^2\| \psi \|$. We conclude that
\begin{equation}
  \| L_A \|\leq \| A \|^2.
\end{equation}
Then $L_A$ can be extended to a bounded operator $\pi_{\omega}(A)$ on the whole $\hH_{\omega}$. The map $\dpt{\pi_{\omega}}{\cA}{\oB(\hH_{\omega})}$ is a representation such that for all $A\in\cA$,
\begin{subequations}
\begin{align}
\omega(A)&=\scal{\Omega_{\omega}}{\pi_{\omega}(A)\Omega_{\omega}},\\
\overline{ \pi_{\omega}(A)\Omega_{\omega} }&=\overline{ \cA/\mN_{\omega} }=\hH_{\omega}.
\end{align}
\end{subequations}
It proves point \ref{GNSi}; we now turn our attention to \ref{GNSii}. For all $A$, $B\in\cA$, we have
\begin{equation}  \label{eq_r1903r4}
\scal{\pi_{\omega}(B)\Omega_{\omega}}{\pi_{\omega}(A)\Omega_{\omega}}=\scal{\Omega_{\omega}}{\pi_{\omega}(B^*A)\Omega_{\omega}}
        =\scal{V\cun}{V(B^*A)}
        =\omega(B^*A),
\end{equation}
but from assumptions, $\omega(A)=\scal{\Omega}{\pi(A)\Omega}$; then 
\begin{equation} \label{eq:BesAOmBeA}
  \omega(B^*A)=\scal{\Omega}{\pi(B^*A)\Omega}
        =\scal{\pi(B)\Omega}{\pi(A)\Omega}.
\end{equation}
We conclude that 
\begin{equation}
  \scal{\pi_{\omega}(B)\Omega_{\omega}}{\pi_{\omega}(A)\Omega_{\omega}}=
    \scal{\pi(B)\Omega}{\pi(A)\Omega}.
\end{equation}
From definition of a cyclic vector, $\pi_{\omega}(\cA)\Omega_{\omega}$ is dense in $\hH_{\omega}$ and $\pi(\cA)$ in $\hH$. This allows us to define $\dpt{u}{\hH_{\omega}}{\hH}$ by the condition
\[ 
  u\pi_{\omega}(A)\Omega_{\omega}=\pi(A)\Omega.
\]
It is a well defined Hilbert space isomorphism because if $\pi_{\omega}(A)\Omega_{\omega}=\pi_{\omega}(B)\Omega_{\omega}$, then equation (true for all $D\in\cA$) $\scal{\pi_{\omega}(B)\Omega_{\omega}}{\pi_{\omega}(D)\Omega_{\omega}}=\scal{\pi(B)\Omega}{\pi(D)\Omega}$ gives an equation of the form $\scal{x}{d}=\scal{y}{d}$ for all $d$ in a dense subset. This equation implies that $x=y$, or $\pi(A)\Omega=\pi(B)\Omega$. We know from general Hilbert space theory that a surjective isometry is unitary; this is the case of $u$.

\end{proof}

For later use, we mention that equation \eqref{eq:BesAOmBeA} gives
\begin{equation}  \label{eq:piomomaesm}
\| \pi_{\omega}(A)\Omega_{\omega} \|=\omega(A^*A)
\end{equation}
when $A=B$.

\begin{corollary}
Let $(\pi_i,\hH_i)$ $(i=1,2)$ be two cyclic representations with cyclic vectors $\Omega_i$. If for all $A\in\cA$, 
\[ 
  \omega_1(A):=\scal{\Omega_1}{\pi_1(A)\Omega_1}=\scal{\Omega_2}{\pi_2(A)\Omega_2}=:\omega_2(A),
\]
then they are equivalent representations.

\end{corollary}

\begin{proof}
The representation $\pi_i$ in $\hH_i$ is cyclic and induces a GNS representation $\pi_{\omega_i}$. Since $\omega_1=\omega_2$, these two GNS representations are the same and $\pi_1$ and $\pi_2$ are thus both equivalent to the same representation.

\end{proof}

The following is just a restatement of \ref{GNSii} of theorem \ref{tho:GNS}.
\begin{proposition}
If $(\cA,\pi,\hH)$ is a cyclic representation, then all the GNS constructions build from a vector state are unitary equivalent to $\pi$. \label{prop:cyclequivGNS}
\end{proposition}


\subsection{Universal representation}
%-----------------------------------

\begin{theorem}
Any $C^*$-algebra accepts an isometric representation on a Hilbert space.
\dixref{2.6.1}
\end{theorem}

\begin{proof}
Let $\cA_h$ be the real Banach space build from hermitian elements of $\cA$ and choose a non zero $A\in\cA$. The element $-A^*A$ does not belong to $\cA^+$. Since $\cA^+$ is a convex closed cone, there exists a linear continuous form $f_A$ on $\cA_h$ such that $f_A(B)\geq 0$ for all $B\in\cA^+$ and $f_A(-A^*A)<0$. One can identify $f_A$ to an hermitian form on $\cA$ because $B$ is decomposed as $B=A_1+iA_2$ with $A_1,A_2\in\cA_h$; we define $f_A(B)=f_A(B_1)+f_A(B_2)$. The function $f_A$ is also a positive form on $\cA$ because on any positive element $B^*B$, we have $f_A(B^*B)>0$.

Now we look at $\pi_A$, the representation defined by $f_A$. We have $f_A(B)=\scal{ \pi_A(B)\xi }{ \xi }$, thus
\[ 
  f_A(A^*A)=\scal{ \pi_A(A^*)\pi_A(A)\xi }{ \xi },
\]
which is zero if $\pi_A(A)=0$. Then $\pi_A(A)\xi\neq 0$ and we conclude that $\pi_A(A)\neq 0$.

We consider $\pi$, the direct sum of all the representations $\pi_A$ for all $A\in\cA$, $A\neq 0$. First we prove that $\pi$ is injective. Indeed if $\pi(B)=0$, we have $\pi_A(B)=0$ for all $A$; in particular 
\begin{equation}
  0=\scal{ \pi_A(B)\xi }{ \pi_A(B) }
        =\scal{ \pi_A(B^*B)\xi }{ \xi }
        =f_A(B^*B)
\end{equation}
by \ref{itemiv_prop_DixGNS} of proposition \ref{prop_DixGNS}. So if $\pi$ is not invertible, there exists a $B\neq 0$ such that for all $A$, $f_A(B^*B)=0$; this implies $\| B^*B \|=0$ and $B=0$. Contradiction.

The map $\pi$ is isometric as injective morphism: $\| \pi(A) \|=\| A \|$.
\end{proof}


The \defe{universal representation}{universal!representation} $\pi_u$ of a $C^*$-algebra $\cA$ is the direct sum of all the GNS representations $\pi_{\omega}$ with $\omega\in\etS(\cA)$. The representation space is 
\[ 
  \hH_u=\bigoplus_{\omega\in\etS(\cA)}\hH_{\omega}.
\]


\begin{theorem}[Gelfand-Neumark]
A $C^*$-algebra is isomorphic to a subalgebra of $\oB(\hH)$ for a certain Hilbert space $\hH$.
\end{theorem}

\begin{proof}
Let's show that $\hH=\hH_u$ and the isomorphism $\pi_u$ answer the question.

\subdem{Injective}
Let $A\in\cA$ such that $\pi_u(A)=0$; from definition of the direct sum and of universal representation,  $\pi_{\omega}(A)=0$ for all $\omega\in\etS(\cA)$. Using equation \eqref{eq:piomomaesm} we find that for such a $A$, we have $0=\| \pi_{\omega}(A)\Omega_{\omega} \|^2=\omega(A^*A)^2$. It is true for all $\omega\in\etS(\cA)$. Lemma \ref{lem:omAenomA} then shows that $\| A^*A \|=0$ and then that $A=0$ because of definition of a $C^*$-algebra. 

\subdem{Surjective}
The representation $\pi_u$ is not specially surjective on $\oB(\hH_u)$, but it is surjective on the subalgebra $\pi_u(\cA)$ which is enough for the present purpose.

\subdem{Morphism}
The map $\dpt{\pi_u}{\cA}{\oB(\hH_u)}$ is a morphism because it is a representation.


\subdem{Isometry}
Lemma \ref{lem:injmorpisom} says that an injective morphism of $C^*$-algebra is isometric.

\end{proof}

The universal representation is trivially faithful, but it is very huge. For example the smallest faithful representation of $\oB(\hH)$ is the definition representation on $\hH$.

\begin{corollary}
An operator $A$ is positive if and only if $\pi(A)\geq 0$ for all cyclic representation $\pi$.
\end{corollary}

\begin{proof}
Since $A$ is positive, then $\sigma(A)\subset\eR^+$ and $A^*=A$. In order to prove that $\pi_u(A)$ is positive, we have to show that $\sigma(\pi_u(A))\subset\sigma(A)$. Let $z\in\sigma(\pi_u(A))$: the operator $\pi_u(A)-z\mtu_u$ where $\mtu_u$ is the unit operator on $\hH_u$ is not invertible. From equation  \eqref{eq:defpiomega},  we have $\mtu_u=\pi_u(\mtu)=\sum_{\omega\in\etS(\cA)}\pi_{\omega}(\cun)$. 
 
Let $z\notin\sigma(A)$ and let us see that $z\notin\sigma(\pi_u(A))$. From assumption on $z$, there exists a $B\in\cA$ such that $B(A-z\cun)=(A-z\cun)=\cun$. Then $\pi_u(B)$ is the inverse of $\pi_u(A)-z\pi_u(\cun)$. It proves that $z\notin\sigma(\pi_u(A))$ and so that $\pi_u(A)$ is positive.

We now prove that $\pi_{\omega}(A)$ is positive for all GNS representation $\pi_{\omega}$. Since $\pi_u$ acts separately on each space $\hH_{\omega}$, all what we said about the invertibility about $\pi_u$ can be said for each $\pi_{\omega}$.

But we know that all cyclic representation is equivalent to a GNS representation from proposition \ref{prop:cyclequivGNS}. We just have to prove that positivity is conserved by equivalence. Suppose that it is not the case. Let $z\in\sigma(\pi_{\omega}(A))$ and $B\in\cA$ such that $B\big( \pi_{\omega}(A)-z\mtu_{\omega} \big)=\mtu_{\omega}$. Then 
\[ 
  UBU^*\big( \pi(A)-z\mtu_{\omega} \big)=\mtu_{\omega}
\]
and then $UBU^*$ is the inverse of $\pi(A)-z\mtu_{\omega}$ and $z\notin\sigma(\pi(A))$. Thus
\[ 
  \sigma(\pi(A))\subset\sigma(\pi_{\omega}(A))\subset\eR^+
\]
which proves that $\pi(A)$ is positive.

We now prove the second sense of the corollary. All GNS representation is cyclic, then $\pi_u(A)$ is positive as sum of positive representation. We want to deduce that $A\geq0$. Since $\pi_u(A)$ is positive, $\pi_u(A)=\pi_u(A)^*=\pi_u(A^*)$. This implies that $A=A^*$ because $\pi_u$ is injective. The positivity of $\pi_u(A)$ gives the existence of a $B$ such that  $\pi_u(A)=\pi_u(B)^*\pi_u(B)=\pi_u(B^*B)$. The injectivity then shows that $A$ is positive.

\end{proof}

The GNS construction is done for unital algebras with states. Since there exists a notion of state on non unital algebras, ones raises the question to generalization of the GNS construction to non unital algebras.

%%%%%%%%%%%%%%%%%%%%%%%%%%
%
   \section{Spaces of matrices}
%
%%%%%%%%%%%%%%%%%%%%%%%%

Let $\cA$ be a $C^*$-algebra and $n\in\eN$. The $C^*$-algebra $\mfM^n(\cA)$\nomenclature{$\mfM^n(\cA)$}{$C^*$-algebra of matrices} is the space on $n\times n$ matrices with entries in $\cA$.The multiplication is defined by
\begin{equation}
  (MN)_{ij}=\sum_kM_{ik}N_{kj}
\end{equation}
where the product in the right hand side is the (in general noncommutative) one in $\cA$. The involution is naturally given by
\begin{equation}
  (M^*)_{ij}=M_{ij}^*.
\end{equation}
One can identify $\mfM^n(\cA)$ to $\cA\otimes\mfM^n(\eC)$ by identifying\footnote{The matrix $E_{ij}$ is the matrix full of zero except a $1$ at position $ij$.} $E_{ij}\in\mfM^n(\cA)$ to $A\otimes E_{ij}\in\cA\otimes\mfM^n(\eC)$.

The Gelfand-Neumark gives the existence of a faithful representation $\pi$ of $\cA$ on $\hH$. If one sees elements of $\hH\otimes\eC^n$ as $n$-uples $(v_1,\ldots,v_n)$ where each $v_i\in\hH$, we can define $\pi_n$ on $\hH\otimes\eC^n$ by linear extension of 
\begin{equation} \label{eq:defreprezmfM}
  \big[\pi_n(M)v\big]_i:=\pi(M_{ij})v_j.
\end{equation}
The norm $\| M \|$ is defined as the norm of $\pi_n(M)$. Since $n<\infty$, $\pi_n(\mfM^n(\cA))$ is a closed $*$-algebra in $\oB(\hH\otimes\eC^n)$ and then $\mfM^n(\cA)$ is a $C^*$-algebra for this norm. Proposition \ref{prop:unicitenormcsa} states that the norm is unique and then that the norm $\| M \|=\pi_n(M)$ is independent of the choice of $\pi$.

\begin{definition}      \label{DefComplPositive}
    A linear map $\dpt{ q }{ \cA }{ \cB }$ is \defe{completely positive}{positive!completely!map between $C^*$-algebra}\index{completely!positive} if for all $n\in\eN$, the map $\dpt{ q_n }{ \mfM^n(\cA) }{ \mfM^n(\cB) }$ defined by
    \[ 
      (q_n(M))_{ij}=q(M_{ij})
    \]
    is positive. 
\end{definition}
As an example, a morphism $\varphi$ is always completely positive because if $a=b^*b$ in $\mfM^n(\cA)$, then $\varphi(a)=\varphi(b)^*\varphi(b)$ which is positive in $\mfM^n(\cB)$.

%%%%%%%%%%%%%%%%%%%%%%%%%%
%
   \section{Stinespring theorem}
%
%%%%%%%%%%%%%%%%%%%%%%%%

Let us state a classical result about Hilbert space

\begin{lemma}
If $K$ is closed in a Hilbert space $\hH$ and if the linear operator $\dpt{ T }{ \hH }{ \hH }$ fulfils

\begin{itemize}
\item $\scal{ Tx }{ y }=\scal{ x }{ Ty }$ for all $x$, $y\in\hH$,
\item $Ty=y$ for all $y\in K$
\item $Tz=0$ for all $z\in K^{\perp}$,
\end{itemize}
then $T$ is the projection on $K$.

\end{lemma}
This lemma allows us to check that $WW^*$ is the projection to the image of $W$. Let us prove that $W^*W$ is the projection on $K_1$. The first condition is clear. The third is satisfied by definition of $W$: if $z\in K_1^{\perp}$, then $W^*Wz=0$. For the second one, remark that if $y\in K_1$, $\scal{ W^*Wx }{ y }=\scal{ x }{ y }$ and if $y\in K_1^{\perp}$, then $\scal{ W^*Wx}{ y }=0=\scal{ x }{ y }$. In both cases $y\in K_1$ and $y\in K_1^{\perp}$, we have $\scal{ W^*Wx }{ y }=\scal{ x }{ y }$. It is sufficient to conclude that $W^*Wx=x$ because $\hH=K_1\oplus K_1^{\perp}$.

\begin{theorem} \index{Stinespring theorem}\index{theorem!Stinespring}
   Let $\dpt{ q }{ \cA }{ \cB }$ be a completely positive map between unital  $C^*$-algebra such that $q(\cun)=\cun$. We suppose that $\cB$ is given with a faithful representation $\cB\simeq\pi_{\chi}(\cB)\subseteq\oB(\hH_{\chi})$ for a certain Hilbert space $\hH_{\chi}$. Then there exists a Hilbert space $\hH^{\chi}$, a representation $\pi^{\chi}$ of $\cA$ on $\hH^{\chi}$ and a partial isometry $\dpt{ W }{ \hH_{\chi} }{ \hH^{\chi} }$ with $W^*W=\cun$ and
\begin{equation}  \label{eq:stinun}
  \pi_{\chi}(q(A))=W^*\pi^{\chi}(A)W
\end{equation}
for all $A\in\cA$.

Stated in an equivalent way, if we define $P=WW^*$ and $\tilde\hH_{\chi}=P\hH^{\chi}\subset\hH^{\chi}$, and $\dpt{ U }{ \hH_{\chi} }{ \tilde\hH_{\chi} }$ as the restriction of $W$ (in such a way that $U$ is unitary because $W$ is a partial isometry), then we have
\begin{equation} \label{eq:stindeux}
  U\pi_{\chi}(q(A))U^{-1}=P\pi^{\chi}(A)P.
\end{equation}
\label{tho:stinespring} 
\end{theorem}


\begin{proof}
On $\hH_{\chi}$ we have the scalar product $\scal{.}{.}_{\chi}$ and we define the sesquilinear form $\scal{.}{.}_0^{\chi}$ on $\cA\otimes\hH_{\chi}$ by sesquilinear extension of 
\[ 
  (A\otimes v,B\otimes w)_0^{\chi}=\scal{v}{\pi_{\chi}(q(A^*B))w}_{\chi}.
\]
 \subdem{This form is semi positive definite}
Let us compute
\begin{equation}
\sum_{ij}(A_i\otimes v_i,A_j\otimes v_j)_0^{\chi}=\sum_{ij}\scal{v_i}{ \pi(q(A_j^*A_j))v_j }_{\chi}.
\end{equation}
For this, we consider $a\in\mfM^n(\cA)$ with elements $a_{ij}=A_i^*A_j$. Let $\pi_n$ be the faithful representation \eqref{eq:defreprezmfM}  of $\mfM^n(\cA)$ on $\hH\otimes \eC^n$ defined by
\[ 
  [\pi_n(M)v]_i=\sum_j\pi(M_{ij})v_j\in\hH.
\]
where $\pi$ is a faithful representation of $\cA$. If $z\in\eC^n\otimes\hH$, we can define $(az)\in\eC^n\otimes\hH$ by 
\[ 
  (az)_i=\big( \pi_n(a)z \big)_i.
\]
So we have
\begin{equation}
 \scal{ z }{ az }=\sum_{ij}\scal{ z_i }{ \pi(a_{ij})z_j }
        =\sum_{ij}\scal{ \pi(A_i)z_i }{ \pi(A_j)z_j }
        =\| Az \|^2\geq0
\end{equation}
where we use the notation $Az=\sum_i\pi(A_i)z_i\in\hH$. The conclusion is that $a\geq0$. Since $q$ is completely positive, $b$ is positive if we define $b_{ij}=q(A_i^*A_j)$. Positivity of $b$ gives rise to an element $c\in\mfM^n(\cB)$ such that $n=c^*c$ where $(c^*)_{ij}=(c_{ji})^*$. From the representation $\pi_{\chi}$ of $\cB$, we can build the faithful representation $\pi'_n$ of $\mfM^n(\cB)$ on $\eC^n\otimes\hH_{\chi}$: $[\pi'_n(b)v]_i=\sum_j\pi_{\chi}(b_{ij})v_j$ where each $v_i$ now belongs to $\hH_{\chi}$.

We are now able to prove that $\scal{ . }{ . }_0^{\chi}$ is a positive form. Indeed 
\begin{equation}
\begin{split}
\sum_{ij}  ( A_i\otimes v_i , A_j \otimes v_j )_0^{\chi}
        &=\sum_{ij} \scal{ v_i }{ \pi_{\chi}\big( q(A_i^*A_j) \big)v_j }_{\chi}
        =\sum_{ij} \scal{ v_i }{ \pi_{\chi}(b_{ij})v_j }_{\chi}\\
        &=\sum_{ijk}\scal{ \pi_{\chi}(c_{ki})v_i }{ \pi_{\chi}(c_{kj})v_j }_{\chi}\geq0
\end{split}
\end{equation}
where the last equality comes from the fact that $(c^*c)_{ij}=\sum_k(c^*)_{ik}c_{kj}=\sum_k(c_{ki})^*c_{kj}$.  This proves that $( . , . )_0^{\chi}$ is positive semi definite.

\subdem{Definition of $\pi^{\chi}$.}

Now we denote by $\mN_{\chi}$ the null space of $( . , . )_0^{\chi}$. Let $\dpt{ V_{\chi} }{ \cA\otimes\hH_{\chi} }{ \cA\otimes\hH_{\chi}/\mN_{\chi} }$ be the canonical projection. We define
\begin{equation}
\scal{ V_{\chi}(A\otimes v) }{ V_{\chi}(B\otimes w) }^{\chi}:=( A\otimes v , B\otimes w )_0^{\chi}
\end{equation}
and we denote by $\hH^{\chi}$ the closure of $\cA\otimes\hH_{\chi}/\mN_{\chi}$ with respect to this scalar product. We can now define $\pi^{\chi}$, a representation of $\cA$ on $\cA\otimes\hH_{\chi}/\mN_{\chi}$ by linear extension of 
\begin{equation}
  \pi^{\chi}(A)V_{\chi}(B\otimes w)=V_{\chi}(AB\otimes w)
\end{equation}
which is well defined because $\pi^{\chi}(A)\mN_{\chi}\subseteq\mN_{\chi}$.


Let us now prove that $\| \pi^{\chi}(A) \|\leq \| A \|$. From equation \eqref{cor:BeAAeB} used in $\mfM^n(\cA)$ with $B=\cun_n$,we know that
\begin{equation} \label{eq:r502061}
  0\leq A^*A\cun_n\leq \| A \|^2\cun_n.
\end{equation}
Now we consider any $B_1,\cdots,B_n\in\cA$ and we build the matrix
\[ 
  b=
\begin{pmatrix}
B_1&\cdots&B_n\\
0&\cdots&0\\
\vdots&\ddots&\vdots\\
0&\cdots&0
\end{pmatrix},\quad
  b^*=
\begin{pmatrix}
B_1^*  & 0      & \ldots&0\\
\vdots &\vdots  & \ddots&\vdots\\
B^*_n  &0&\ldots & 0
\end{pmatrix}.
\]
We conjugate \eqref{eq:r502061} with $b$: 
\[ 
  0\leq b^*A^*Ab\leq \| A \|^2b^*b,
\]
but $q$ is completely positive, then it respects the inequality:
\[ 
  q_n(b^*A^*Ab)\leq \| A \|^2q_n(b^*b)
\]
where $\dpt{ q_n }{ \mfM^n(\cA) }{ \mfM^n(\cB) }$ is defined by $\big( q_n(M) \big)_{ij}=q(M_ij)$. The definition of complete positivity of $q$  is precisely positivity of $q_n$. We consider now the representation $\pi_{\chi}$ of $\cB$ on $\hH_{\chi}$. Since $a\geq0$, we can find a $e\in\mfM^n(\cB)$ such that $q_n(a)=e^*e$; then
\begin{equation}
\begin{split}
\sum_{ij}\scal{ v_i }{ \pi_{\chi}\big( q(a_{ij})v_j \big) }&=\sum_{ij}\scal{ v_i }{ \pi_{\chi}(e^*e)_{ij}v_j }\\
        &=\sum_{ijk}\scal{ v_i }{ \pi_{\chi}(e^*_{ki})\pi_{\chi}(e_{kj} }\\
        &=\sum_{ijk}\scal{ \pi_{\chi}(e)_{ki}v_i }{ \pi_{\chi}(e_{ij})v_j }\geq0.
\end{split}
\end{equation}

Let us consider $\Psi=\sum_iV_{\chi}B_i\otimes v_i$ and compute
\begin{equation}
\begin{split}
\| \pi^{\chi}(A)\Psi \|^2&=\sum_{ij}( AB_i\otimes v_i , AB_j\otimes v_j )_0^{\chi}\\
        &=\sum_{ij}\scal{ v_i }{ \pi_{\chi}\big( q(B_i^*A^*AB_j) \big)v_j }_{\chi}\\
        &\leq \| A \|^2\sum_{ij}\scal{ v_i }{ \pi_{\chi}\big( q(B^*_iB_j) \big)v_j }_{\chi}\\
        &=\| A \|^2\sum_{ij}\scal{ B_i\otimes v_i }{ B_j\otimes v_j }_{\chi}\\
        &=\| A \|^2( V_{\chi}\sum_i B_i\otimes v_i , V_{\chi}\sum_jB_j\otimes v_j )^{\chi}\\
        &=\| A \|^2\| \Psi \|^2.
\end{split}
\end{equation}
Then $\| \pi^{\chi}(A)\Psi \|\leq \| A \|^2\| \Psi \|$ which proves that
\begin{equation}
\| \pi^{\chi}(A) \|\leq\| A \|^2.
\end{equation}
The formula
\begin{equation}
  \pi^{\chi}(A)V_{\chi}(B\otimes w)=V_{\chi}(AB\otimes w)
\end{equation}
defines a continuous representation $\pi^{\chi}$ on $\cA\otimes\hH_{\chi}/\mN_{\chi}$ which can be extended to a continuous representation on the whole $\hH^{\chi}$. This extension fulfils $\pi^{\chi}(A^*)=\pi^{\chi}(A)^*$.

We define $\dpt{ W }{ \hH_{\chi} }{ \hH^{\chi} }$ by
\begin{equation}
  Wv=V_{\chi}\cun\otimes v.
\end{equation}

\subdem{The map $W$ is a partial isometry}

In order to prove that $W$ is a partial isometry, just compute
\begin{equation}
\begin{split}
( Wv , Ww )^{\chi}&=( V_{\chi}\cun\otimes v , V_{\chi}\cun\otimes w )^{\chi}\\
        &=( \cun\otimes v , \cun\otimes w )_0^{\chi}\\
        &=\scal{ v }{ w }_{\chi}.
\end{split}
\end{equation}

\subdem{Adjoint of $W$}

We claim that $\dpt{ W^* }{ \hH^{\chi} }{ \hH_{\chi} }$, $W^*V_{\chi} A\otimes v=\pi_{\chi}(q(A))v$ is the adjoint of $W$. Recall that the definition of the adjoint requires that 
\[ 
  \scal{ w }{ W^*\psi }_{\chi}=( Ww , \psi )^{\chi}
\]
for all $w\in\hH_{\chi}$ and all $\psi\in\hH^{\chi}=\overline{ A\otimes\hH_{\chi}/\mB_{\chi} }$. A $\psi\in\hH^{\chi}$ can be written under the form $V_{\chi}A\otimes v$ with $A\in\cA$ and $v\in\hH_{\chi}$; then
\begin{equation}
\begin{split}
\scal{ w }{ W^*V_{\chi}A\otimes v }_{\chi}&=\scal{ w }{ \pi_{\chi}(q(A))v }_{\chi}\\
        &=( \cun\otimes w , A\otimes v )_0^{\chi}\\
        &=( V_{\chi}\cun\otimes w , V_{\chi}A\otimes v )^{\chi}\\
        &=\scal{ Ww }{ \psi }
\end{split}
\end{equation}
as expected. One can check that $W^*W=\mtu$ and that $W^*\pi^{\chi}(A)W=\pi_{\chi}(q(A))$ because
\begin{equation}
  W^*\pi^{\chi}(A)Wv=W^*\pi^{\chi}(A)V_{\chi}(\cun\otimes v)
        =W^*V_{\chi}(A\otimes v)
        =\pi_{\chi}(q(A)).
\end{equation}

\subdem{Last point: \eqref{eq:stinun}$\Rightarrow$\eqref{eq:stindeux}}

Since $W$ is a partial isometry, $P=WW^*$ is the projection on the image of $W$ and $W^*W$ ($=\cun$) is the projector on the subspace of $\hH_{\chi}$ on which $W$ is isometric; this space is $\hH_{\chi}$ itself. The $\tilde\hH_{\chi}=P\hH^{\chi}=\hH^{\chi}$ and $U=W$. Then
\begin{equation}
  U\pi_{\chi}\big( q(A) \big)U^{-1}=W\pi_{\chi}\big( q(A) \big)W^*
        =WW^*\pi^{\chi}(A)WW^*
        =P\pi^{\chi}(A)P
\end{equation}
This concludes the proof of theorem \ref{tho:stinespring}.

\end{proof}


\begin{remark}
If on the one hand $q(\cun)$ is not $\cun$, then the construction works, but $W$ is no more a partial isometry and we have
\[ 
  \| W \|^2=\| q(\cun) \|,
\]
so $\hH_{\chi}$ can not be seen as a subspace of $\hH^{\chi}$ by the map $\dpt{W}{ \hH_{\chi} }{ \hH^{\chi} }$.

If on the other hand $\cA$ or $\cB$ is not unital, then works if $q$ can be extended (keeping positive) to the unitization of $\cA$ in such a way that it conserves the identity in (the unitization of ) $\cB$.

\end{remark}

\begin{definition}
Let $\{ O_1,\cdots,O_l \}$ be an open covering of the compact set topological space $X$. A \defe{partition of unity}{partition of unity}\index{unit!partition} subordinate to this covering is a set of functions $\phi_i\in C(X)$, $i=1,\cdots,l$ such that $\Supp\phi_i\subset O_i$ and
\[ 
  \sum_{i=1}^l\phi_i(x)=1
\]
for all $x\in X$.

\end{definition}
There exist some theorem which give the existence of such a partition for all compact topological spaces.


\begin{proposition}
Any positive map $q\colon \cA\to \cB$ from a commutative unital $C^*$-algebra $\cA$ is completely positive.
\end{proposition}

The proof will be decomposed into several propositions. Let us begin by a remark: from theorem \ref{thoGelfand}, we can write $\cA=C(X)$ for a certain locally compact Hausdorff space $X$. So we can identify $\mfM^n(C(X))$ with $C(X,\mfM^n(\eC))$ because to each $a\in\mfM^n(C(X))$, ($a_{ij}$ is a map $\dpt{ a_{ij} }{ X }{ \eC }$) we make correspond the map $\dpt{ \eta }{ X }{ \mfM^n(\eC) }$ defined by $\eta(x)_{ij}=a_{ij}(x)$.


\begin{proposition}
The set of finite linear combinations of elements $F$ of the form
\[ 
  F(x)=\sum_i f_i(x)M_i
\]
with $f_i\in C(X)$ and $M_i\in\mfM^n(\eC)$ is dense in $C(X,\mfM^n(\eC))$.
 \label{prop:lencombpart}
\end{proposition}


\begin{proof}
Let $G\in C(X,\mfM^n(\eC))$ and $\varepsilon>0$. Continuity of $G$ makes the set
\[ 
  \mO_x^{\varepsilon}=\{ y\in X\tq \| G(x)-G(y) \|\leq\varepsilon \}
\]
open for all $x\in X$. These set give an open covering of the compact space $X$, then we can extract a finite subcovering and build an unity partition $\varphi_i$. We define $F_l\in C\big( X,\mfM^n(\eC) \big)$ by
\begin{equation} \label{eq:Fllim}
  F_l(x)=\sum_{i=1}^l\varphi_i(x)G(x_i).
\end{equation}
We have
\begin{equation}
\| F_l(x)-G(x) \|=\| \sum_{i=1}^l\varphi(x)( G(x_i)-G(x) ) \|
        \leq \sum \varphi_i(x)\| G(x_i)-G(x) \|
        \leq \sum\varphi_i\varepsilon
    =\varepsilon
\end{equation}
Then $\| F_j-G \|=\sum_{x\in X}\| F_l(x)-G(x) \|\leq\varepsilon$ and the sequence $F_l$ converges to $G$. It proves the density.
\end{proof}


\begin{proposition}
When $\{ M_i \}$ is a basis of $\mfM^n(\eC)$ composed with positive elements, an element $F\in C(X,\mfM^n(\eC))$ of the form $F(x)=\sum_if_i(x)M_i$ is positive if and only if it each of $f_i$ is positive.
\end{proposition}

\begin{proof}
We know that $F\in C(X,\mfM^n(\eC))$ is positive when $F(x)$ is positive in $\mfM^n(\eC)$ for all $x\in X$. We have $F(x)=\sum_if_i(x)M_i$, but positivity of $F(x)$ requires $F(x)=F(x)^*$ and then $f_i(x)=f_i(x)^*$ because $M_i$ is positive.

\end{proof}

\begin{probleme}
  \cite{Landsman} states a stronger result that seems wrong to me because \( -3+7\) is positive.
\end{probleme}

\begin{proposition}
If $G\in C(X,\mfM^n(\eC))$ is positive, then there exists a sequence $F_k\geq 0$ such that $\lim_{k\to\infty}F_k=G$
\end{proposition}

\begin{proof}
Each element $F_l$ \eqref{eq:Fllim} is positive because  $G(x_i)$ is positive for all $x_i$.
\end{proof}

\begin{probleme}
    There are too much unclear thinks in my mind; I do not finish the proof.
\end{probleme}

The main result is the following proposition.

\begin{proposition}
If $\cA$ is a commutative unital $C^*$-algebra, then any positive map $\dpt{ q }{ \cA }{ \cB }$ is completely positive.
\end{proposition}


\section{Representations}
%+++++++++++++++++++++++++

As notational convention, when $H$ is a Hilbert space, we denote by $\mL(H)$\nomenclature{$\mL(H)$}{Space of continuous endomorphisms of $H$} the set of the continuous endomorphism of $H$. Topology on $\mL(H)$ is discussed in subsection \ref{subsec_topomL}.

When $\pi$ is a representation of $\cA$ in $H$ and $\xi\in H$, the space $\overline{ \pi(\cA)\xi }$ is a closed subspace of $H$ stable under $\pi(\cA)$.

\begin{definition}
A vector $\xi\in H$ is said \defe{totalizing}{totalizing vector} for a
representation $\pi$ of $\cA$ if $\overline{ \pi(\cA)\xi }=H$.
\end{definition}

\begin{proposition}
Let $\cA$ be an involutive algebra, $H$ an hermitian space and $\pi$ a representation of $\cA$ in $H$. Then the following facts are equivalent 
:
\begin{enumerate}
\item The only closed subspace in $H$ which are stable for $\pi(\cA)$ 
are $\{o\}$ and $H$.
\item The subset of $\mL(H)$ which commutes with $\pi(\cA)$ is reduced 
to $\mC$.
\item Any non zero vector in $H$ is totalizing for $\pi$, or $\pi$ have 
dimension $1$.
\end{enumerate}
 \label{prop:reprez_topo}\dixref{2.3.1}
\end{proposition}

\begin{proof}
We begin proving that (ii) implies (iii). Let $\xi\in H$, $\xi\neq 0$. 
If $\pi(\cA)\xi$ is not everywhere dense in $H$, (i) makes
$\pi(\cA)\xi=0$. Then $\eC\xi$ is stable under $\pi(\cA)$. But 
$\eC\xi\neq\{o\}$, then $\eC\xi=H$. Thus $H$ has dimension $1$ and $\pi$ 
is the null representation.

Now, we prove that (iii) implies (i). Let $K\neq\{o\}$ be a closed 
vector  
subspace of $H$ stable under $\pi(\cA)$. We have to show that $K=H$. If 
$\dim H=1$, it is obvious. Let us consider a non zero  $\xi\in K$. Since 
$K$ is stable, $\pi(\cA)\xi\subset K$, but (iii) implies
 $\overline{ \pi(\cA)\xi }=H$. Thus $K=H$.

We turn our attention to the equivalence between (i) and (ii). First 
(ii) implies (i). We consider $K$, a vector subspace of $H$ stable under 
$\pi(\cA)$. We want $K=\{o\}$ or $K=H$. Let us consider 
$\dpt{p_K}{H}{K}$, the orthogonal projection. Since $K$ is a vector 
subspace, it makes sense to write $H\ominus K$.

Let us consider $\xi\in K$ and $\eta\in H\ominus K$. For any $A\in\cA$, 
$\pi(A^*)\xi\in K$ because $\pi(A)\xi\in K$. This yields
\[
   \langle \pi(A)\eta|\xi\rangle =\langle\eta|\pi(A^*)\xi\rangle=0,
\]
then $\pi(A)\eta\in H\ominus K$. We can conclude that 
\[
   [p_K,\pi(\cA)]=0
\]
because $\pi(\cA)p_K\xi=\pi(\cA)\xi=p_K\pi(\cA)\xi$, $\pi(\cA)p_K\eta=0$  and $p_K\pi(\cA)\eta=0$ from $\langle \pi(\cA)\eta|\xi\rangle =0$.

From (ii), $p_K$ is then a scalar operator: $p_K=0$ or $p_K=id$, \emph{i.e.} $K=\{o\}$ or $K=H$.

Finally, we show the implication from (i) to (ii). Let $T$ be an element of $\mL(H)$ which commute with the whole $\pi(\cA)$; we have to show that $T$ is scalar. Since it is clear that $T+T^*$ and $T-T^*$ also commute with $\pi(\cA)$, we can suppose $T=T^*$.

We had shown that a projector $p_K$ commutes with $\pi(\cA)$ if $K$ is stable under $\pi(\cA)$, closed and a sub vector space of $H$. This is the case of the eigenspaces because $(T-\lambda\mtu)v=0$, then $(T-\lambda\mtu)\pi(\cA)v=0$ because $[T,\pi(\cA)]=0$.

The spectral projector of $T$ commute with $\pi(\cA)$. Thus, the eigenspaces $H_{\lambda}$ are stable under $\pi(\cA)$, thus (by (i)) these are only $\{o\}$ and $H$. In other words $(T-\lambda\mtu)v=0$ has solutions which are on two subspaces whose projectors are $0$ and $1$. On the space on which $p_{\lambda}=1$, $T=\lambda\mtu$ and the one where $p_{\lambda}=0$ is $\{0\}$ then $T=0$.

\begin{probleme}
    We have to show that every $v$ belong to one of these two spaces. Is it because $T$ is hermitian, or do we need the compact assumption ?
\end{probleme}

\end{proof}

\begin{definition}
Let $\cA$ be an involutive algebra, $H$ a hermitian space and $\pi$ a representation of $\cA$ in $H$. We say that $\pi$ is \defe{topologically irreducible}{irreducible!topologically} if it fulfils proposition \ref{prop:reprez_topo}. The representation $\pi$ is \defe{algebraically irreducible}{irreducible!algebraically} if the only stable vector subspaces of $H$ under $\pi(\cA)$ are $\{0\}$ and $H$.
\end{definition}

When $\dim H=\infty$, the second notion is stronger because it excludes the case where one has an \emph{open} stable subspace. Point 2.8.4 in \cite{Dixmier} shows that in the case of $C^*$-algebras, a topologically irreducible representation is automatically algebraically irreducible, so one can simply speak about irreducible representations.

%%%%%%%%%%%%%%%%%%%%%%%%%
%
   \section{Pure states}
%
%%%%%%%%%%%%%%%%%%%%%%%%


A subset $C$ of a vector space is \defe{convex}{convex} if for all $\lambda\in[0,1]$ and $v$, $w\in C$, the element $\lambda v+(1-\lambda)w$ belongs to C.

An \defe{extreme point}{extreme point} of a convex set $K$ is an element $\omega\in K$ which can de written under the form $\omega=\lambda\omega_1+(1-\lambda)\omega_2$ with $\lambda\in[0,1]$ only for $\omega_1=\omega_2=\omega$.

\begin{definition}
An extreme point of the state space $\etS(\cA)$ is a \defe{pure state}{state!pure}\index{pure state} and states that are not pure are \defe{mixed states}{mixed!state}\index{state!mixed}.
\end{definition}

The set of extreme points of the convex set $K$ is denoted by $\partial_cK$ and is called the \defe{extreme boundary}{boundary!extreme} of $K$.  An extreme point in the state space $\etS(\cA)$ of a $C^*$-algebra is a \emph{pure state} and other states are \defe{mixed states}{mixed!state}\index{state!mixed}. As notation, $\partial_c(\etS(\cA))$ is denoted by $\etP(\cA)$ or simply $\etP$ when there are no ambiguity.

\subsection*{Example \texorpdfstring{$\cA=\eC\oplus\eC$}{A=C+C}}
 Points are given by $(\lambda,\mu)$. Let's consider the convex set of states $r\in[0,1]$ defined by $r(\lambda,\mu)=(1-r)\lambda-r\mu$.  Extreme points are given by $0$ and $1$ : $0(\lambda, \mu)=\lambda$ and $1(\lambda,\mu)=\mu$.

\subsection*{Example: \texorpdfstring{$\cA=\mfM^2(\eC)$}{A=M2C}} 
We can identify $\mfM^2(\eC)$ with its dual in the following way. A linear form $\omega$ on $\cA$ can always be written as
\[ 
  \omega
\begin{pmatrix}
 A_{11}&A_{12}\\
A_{21}&A_{22}
\end{pmatrix}
=\omega_{11}A_{11} +\omega_{12}A_{21}+ \omega_{21}A_{12}+ \omega_{22}A_{22} 
\]
So to each form $\omega$, one can associate the matrix of $\omega_{ij}$ and the following holds :
\[ 
  \omega(A)=\tr(\omega A)
\]
The identification between $\mfM^2(\eC)$ and its dual is then well given by $\omega\simeq(\omega_{ij})$. Let us see in terms of this identification the set $\etS(\cA)$. The condition $\omega(A)\geq 0$ imposes to the matrix $(\omega_{ij})$ to be positive and the condition $\omega(\cun)=1$ imposes $\tr\omega=1$. Then $\etS(\cA)$ is parametrized by 
\[ 
  \rho=\frac{ 1 }{2}
\begin{pmatrix}
1+x&y+iz\\
y-iz&1-x
\end{pmatrix}
\]
where $x$, $y$, $z\in\eR$. The pure states are such matrices $\rho$ with $x^2+y^2+z^2=1$.

If $M$ is a $*$-algebra in $\oB(\hH)$, the \defe{commutant}{commutant} $M'$ is 
\[ 
  M'=\{ A\in\oB(\hH)\tq [A,m]=0    \forall\, m\in M \}.
\]

\begin{proposition} 
The following properties are equivalent :

\begin{enumerate}
 \item \label{enumgz} The representation $\pi(\cA)$ is irreducible in $\hH$.
\item\label{enumgi} The commutant of $\pi(\cA)$ in $\oB(\hH)$ is
\[ 
  \pi(\cA)'=\{ \lambda\cun\tq\lambda\in\eC \}.
\]
\item \label{enumgii} $\pi(\cA)''=\oB(\hH)$.

\item \label{enumgiii} Each vector $\Omega\in\hH$ is cyclic for $\pi(\cA)$.

\end{enumerate}
 \label{prop_equiv_rep_irred}
\end{proposition}

\begin{proof}
\ref{enumgii}$\Rightarrow$\ref{enumgi}
We have to see that an operator which commutes with the whole $\oB(\hH)$, then it is a multiple of identity. If $[a,A]=0$ for all $A\in\oB(\hH)$, then it commutes in particular with an operator $A$ which leaves the basis vector $e_{\beta}$ (and only this basis vector). In this case, $ae_{\beta}=\lambda_{\beta}e_{\beta}$. We conclude that $a$ must be diagonal. Since $a$ must also commute with an operator which leaves $e_{\alpha}+e_{\beta}$ unchanged, we conclude that $\lambda_{\alpha}=\lambda_{\beta}$, so that $a=\lambda\mtu$.

\ref{enumgz}$\Rightarrow$\ref{enumgi} Will be done later.
\ref{enumgi}$\Rightarrow$\ref{enumgz} Let us suppose that $\pi(\cA)'=\eC\cun$ and that $\pi$ is irreducible; we will find out a contradiction. We have a non trivial subspace of $\sH$ stable under $\pi(\cA)$. The projection operator on this space commutes with the whole $\pi(\cA)$ although it is not a multiple of identity.

\ref{enumgz}$\Rightarrow$\ref{enumgiii} We proceed by contradiction once again.  Let $\psi\in\hH$ such that $\pi(\cA)\psi$ is not dense in $\hH$ and $P$ be the projection on the closure of $\pi(\cA)\psi$. Lemma \ref{lem_preGNS} assures that $P\in\pi(\cA)'$. Consequently $\pi(\cA)'\neq \eC\cun$ and $\pi(\cA)$ is not irreducible by \ref{enumgi}.

\ref{enumgiii}$\Rightarrow$\ref{enumgz} If $\psi$ belongs to a (non trivial) invariant subspace, then $\pi(\cA)\psi$ cannot be dense because it is a proper subspace of $\hH$.

\end{proof}

\begin{probleme}
    There are still unfinished points in that proof.
\end{probleme}


Let us point out the following part of the proposition :

\begin{lemma}[Schur's lemma]
The representation $\pi$ on $\cA$ is irreducible if and only if the commutant of $\pi(\cA)$ in $\oB(\hH)$ is
 \[ 
  \pi(\cA)'=\{ \lambda\cun \}_{\lambda\in\eC}.
\]

\end{lemma}


\begin{lemma} 
Let $\hat Q$ be a bounded quadratic form on a Hilbert space $\hH$. There exists a bounded operator $Q$ on $\hH$ such that for each $\psi,\phi\in\hH$ we have 
\[
\hat Q(\psi,\phi)=\scald{ \psi}{Q\phi }
\]
 and $\| Q \|\leq C$ where $C$ is the ``bounding constant'' : $| \hat Q(\psi,\phi) |\leq \| \psi \|\| \phi \|$.
Moreover if 
%
\begin{equation}  \label{eq_r19032}
\hat Q(\phi,\psi)=\overline{ \hat Q(\psi,\phi) }
\end{equation}
 the operator $Q$ will be selfadjoint.
\label{lem_r19031}
\end{lemma}

\begin{proof}
Let us fix a $\psi\in\hH$ and look at the map $\phi\mapsto\hat Q(\psi,\phi)$. It is a bounded form, so Riesz theorem gives the existence of a $\Omega\in\hH$ such that $\hat Q(\psi,\phi)=\scald{ \Omega }{ \phi }$. We can define $Q$ by $Q\psi=\Omega$. It is clear that is is self-adjoint if equation \eqref{eq_r19032} is satisfied. 

From equalities $\hat Q(\psi,\phi)=\scald{ \Omega }{ \phi }=\scald{ Q\psi }{ \phi }$, we find
 \begin{equation}
\begin{split}
\| Q\psi \|^1&=| \scald{ Q\psi }{ Q\psi } |\\
        &=| \hat Q(Q\psi,\psi) |\\
        \leq \| Q\psi \|\| \psi \|\\
        &=\| Q \|\| \psi \|^2.
\end{split}
\end{equation}
Taking supremum on $\| \psi \|=1$, we find $\| Q \|^2\leq C\| Q \|$ and then
\[ 
  \| Q \|\leq C.
\]

\end{proof}

\begin{theorem} 
   The GNS representation $\pi_{\omega}$ of a state $\omega\in\etS(\cA)$ is irreducible if and only if $\omega$ is pure.
\label{tho_GNS_irred_pure}
\end{theorem}

\begin{proof}
Let us begin to suppose that $\omega$ is a pure state and that $\pi_{\omega}$ is reducible. Then the projection $P$ onto the invariant subspace $K$ of $\pi(\cA)$ belongs to $\pi(\cA)'$ (see the proof of Schur's lemma). Let $\Omega_{\omega}$ be the cyclic vector of $\pi_{\omega}$. 

If $P\Omega_{\omega}=0$, then for each $A\in\cA$ we have
\[ 
  0=\pi_{\omega}P\Omega_{\omega}=P\pi_{\omega}(A)\Omega_{\omega}.
\]
Since $\Omega_{\omega}$ is cyclic, it proves that $P=0$ which is impossible if $\pi_{\omega}$ is reducible. For the same reason, $P^{\perp}\Omega_{\omega}$ is neither not possible because it should implies that $P=\cun$. 

We define the two following states on $\cA$ :
\begin{subequations}
\begin{align}
  \psi(A)&=k\scald{ P\Omega_{\omega} }{ \pi(A)P\Omega_{\omega} }\\
  \psi^{\perp}(A)&=l\scald{ P^{\perp}\Omega_{\omega} }{ \pi_{\omega}(A)P^{\perp}\Omega_{\omega} }
\end{align}
\end{subequations}
From definition of a projection, we have $\scald{ Px }{ Py }=\scald{ P^*Px }{ y }=\scald{ Px }{ y }$. Taking any $k$, $\lambda=1/k$ and $l=1/(1-1/k)$, using the relation $\omega(A)=\scald{ \Omega_{\omega} }{ \pi_{\omega}(A)\Omega_{\omega} }$ we find
\[ 
  \omega=\lambda\psi+(1-\lambda)\psi^{\perp},
\]
which is in contradiction with the fact that $\omega$ is pure.

We now suppose that $\pi_{\omega}$ is irreducible, and that $\omega$ reads
\[ 
  \omega=\lambda\omega_1+(1-\lambda)\omega_2
\]
with $\lambda\in[0,1]$ and $\omega_1,\omega_2\in\etS(\cA)$. We will prove that $\omega_1$ is proportional to $\omega$, so that $\omega$ is pure. Since elements of $\etS(\cA)$ are positive and $(1-\lambda)\geq0$, the form $\omega-\lambda\omega_1=(1-\lambda)\omega_2$ is positive. Therefore for all $A\in\cA$, we have $\lambda\omega_1(A^*A)\leq\omega(A^*A)$. From equation \eqref{eq:omABleq}, we find
%
\begin{equation} \label{eq_r19031}
\begin{split}
| \lambda\omega_1(A^*B) |&\leq\lambda^2\omega_1(A^*A)\omega_1(B^*B)\\
        &\leq\omega(A^*A)\omega(B^*B).
\end{split}
\end{equation}
It allows us to define a quadratic form $\hat Q$ on $\pi_{\omega}(\cA)\Omega_{\omega}$ by
\begin{equation}
\hat Q\big( \pi_{\omega}(A)\Omega_{\omega},\pi_{\omega}(B)\Omega_{\omega} \big):=\lambda\omega_1(A^*B).
\end{equation}

\subdem{$\hat Q$ is well defined}

We have to prove that $\pi_{\omega}(A_1)=\pi_{\omega}(A_2)\Omega_{\omega}$ implies
\[ 
 \hat Q(\pi_{\omega}(A_1)\Omega_{\omega},\cdot)=\hat Q(\pi_{\omega}(A_2)\Omega_{\omega},\cdot).
\]
 Equation \eqref{eq:piomomaesm} gives
\[
  \| \pi_{\omega}(A)\Omega_{\omega} \|^2=\omega(A^*A)
\]
and makes that $\omega\big( (A_1-A_2)^*(A_1-A_2) \big)=0$. Thus, using \eqref{eq_r19031}, we have
\begin{equation}
\begin{split}
\Big| & \hat Q(\pi_{\omega}(A_1)\Omega_{\omega},\pi_{\omega}(B)\Omega_{\omega})-\hat Q(\pi_{\omega}(A_2)\Omega_{\omega},\pi_{\omega}(B)\Omega_{\omega})      \Big|^2\\
        &=\Big|   \lambda\omega_1(A_1^*B)-\lambda\omega_1(A_2^*B)    \Big|^2\\
        &=\Big|  \lambda\omega_1\big( (A_1-A_2)^*B \big)  \Big|^2
        \\&\leq0.
\end{split}
\end{equation}
The whole is finally zero.
 

\subdem{$\hat Q$ is bounded}

Equation \eqref{eq:piomomaesm} together with the equality $| \lambda\omega_1(A^*B) |^2\leq\omega(A^*A)\omega(B^*B)$ give
\begin{equation}
\begin{split}
  \Big|  \hat Q(\pi_{\omega}(A)\Omega_{\omega},\pi_{\omega}(B)\Omega_{\omega})   \Big|^2
    &=| \lambda\omega_1(A^*B) |^2\\
    &\leq\omega(A^*A)\omega(B^*B)\\
    &=\| \pi_{\omega}(A)\Omega_{\omega} \|^2\| \pi_{\omega}(B)\Omega_{\omega} \|^2.
\end{split}
\end{equation}
Therefore $| \hat Q(\psi,\phi) |^2\leq\| \psi \|\| \phi \|$ and $\hat Q$ is bounded.

The quadratic form $\hat Q$ can be continuously extended to the whole $\hH_{\omega}$. From general property $\omega(A^*)=\overline{ \omega(A) }$ (when $\omega$ is a state),
\[ 
  \hat Q(\phi,\psi)=\overline{ \hat Q(\psi,\phi) }.
\]
Lemma \ref{lem_r19031} immediately applies to our $\hat Q$, so we have an operator $Q$ such that $\scald{ \psi }{ Q\phi }=\hat Q(\psi,\phi)$. Therefore
%
\begin{equation} \label{eq_1903r3}
  \scald{ \pi_{\omega}(A)\Omega_{\omega}}{Q\pi_{\omega}(B)\Omega_{\omega}}=\lambda\omega_1(A^*B)
\end{equation}
Since $\pi$ is a representation, we have for all $A$, $B\in\cA$ :
 \begin{equation}
\begin{split}
  \scald{ \pi_{\omega}(A)\Omega_{\omega} }{ Q\pi_{\omega}(B)\Omega_{\omega} }
     &=\hat Q\big( \pi_{\omega}(B^*A)\Omega_{\omega},\Omega_{\omega} \big)\\
        &=\scald{ \pi_{\omega}(B^*A)\Omega_{\omega} }{ Q\Omega_{\omega} }\\
        &=\scald{ \pi_{\omega}(A)\Omega_{\omega} }{ \pi_{\omega}(B)Q\Omega_{\omega} }.
\end{split}
\end{equation}
This proves that $[Q,\pi_{\omega}(C)]=0$ for each $C\in\cA$. Therefore $Q\in\pi_{\omega}(\cA)'$ and there exists a $t\in\eR$ such that $Q=t\cun$. Using equation \eqref{eq_1903r3} and  \eqref{eq_r1903r4}, we find 
%
\begin{equation}
\begin{split}
t\omega(A^*B)&=t\scald{ \pi_{\omega}(A)\Omega_{\omega} }{ \pi_{\omega}(B)\Omega_{\omega} }\\
&=\lambda\omega_1(A^*B)\\
&=\hat Q\scald{ \pi_{\omega}(A)\Omega_{\omega} }{ \pi_{\omega}(B)\Omega_{\omega} }\\
\end{split}
\end{equation}
So $t\omega(A^*B)=\lambda\omega_1(A^*B)$; thus $\omega$ and $\omega_1$ are proportional and the decomposition
%
\[ 
  \omega=\lambda\omega_1+(1-\lambda)\omega_2
\]
is only possible for $\omega_1=\omega_2=\omega$. This proves that $\omega$ is a pure stare.

\end{proof}


\begin{corollary}
If the representation $\big( \pi(\cA),\hH \big)$ is irreducible, then the GNS representation $\big( \pi_{\omega}(\cA),\hH_{\omega} \big)$ build from any vector state (corresponding to $\Psi\in\hH$ such that $\| \Psi \|=1$) is unitary equivalent to $(\pi(\cA),\hH)$.
 \label{cor_GNSirredst}
\end{corollary}

\begin{proof}
    Since $\pi$ is irreducible, any vector in $\hH$ is cyclic from proposition \ref{prop_equiv_rep_irred}. So $\pi$ is cyclic and proposition \ref{prop:cyclequivGNS} concludes.

\end{proof}

\begin{corollary}
All irreducible representation of a $C^*$-algebra is (up to an equivalence) the GNS construction from a pure state.
\end{corollary}


\begin{proof}
We know that an irreducible representation is unitary equivalent to a GNS representation, but the GNS representation will only be irreducible when $\omega$ is pure state.
\end{proof}


\begin{proposition} 
A state $\omega$ is pure if and only if for each positive functional $\rho$ such that $0\leq \rho\leq\omega$, there exists a $t\in\eR^+$ such that $\rho=t\omega$.
\label{prop_pureiff}
\end{proposition}
\lref{2.12.8}

\begin{proof}
From corollary \ref{cor_csa_unit} and  proposition \ref{prop_st_unit_ext} we can suppose that $\cA$ is unital because if not, the notion of positivity is defined from unitization. When $\rho=0$ or $\rho=\omega$, the result is trivial. Now we suppose that $0\neq\rho\neq\omega$.

\subdem{Direct sense}

We know that $\omega$ is pure and $0\leq\rho\leq\omega$, $0<\rho(\cun)<1$ because $\omega-\rho$ is positive. Therefore
\[ 
  \| \omega-\rho \|=\omega(\cun)-\rho(\cun)=1-\rho(\cun),
\]
thus $\rho(\cun)=1$ should implies $\| \omega-\rho \|=0$ and then $\omega=\rho$. The possibility $\rho(\cun)=1$ is also not possible. Thus $\rho(\cun)$ is between $0$ and $1$.

This allows us to consider the states
\[ 
  \frac{ \omega-\rho }{ 1-\rho(\cun) },\quad\text{and}\quad\frac{ \rho }{ \rho(\cun) }.
\]
For $\lambda=1-\rho(\cun)$,
\[ 
  \omega=\lambda\frac{ \omega-\rho }{ 1-\rho(\cun) }-(1-\lambda)\frac{ \rho }{ \rho(\cun) }.
\]
Since $\omega$ is pure, it implies $\frac{ \omega-\rho }{ 1-\rho(\cun) }=\frac{ \rho }{ \rho(\cun) }$. So, $\rho=\rho(\cun)\omega$.

\subdem{Inverse sense}

Let us consider a decomposition $\omega=\lambda\omega_1+(1-\lambda)\omega_2$ of $\omega$. In the proof of theorem \ref{tho_GNS_irred_pure}, we find that $0\leq\lambda\omega_1<\omega$. Then the assumption says that $\lambda\omega_1=\omega=\omega_2$ and the normalization makes automatically $\omega_1=\omega=\omega_2$ which proves that $\omega$ is pure.

\end{proof}


\begin{lemma}
Let $\cA$ be a $C^*$-algebra and $\rho$, a positive form on $\cA$. If we pose $M=\ker(\rho)$, $N=\{ A\in\cA\tq \rho(A^*A)=0 \}$, then
\[ 
  N+N^*\subseteq M
\]
and if $\rho$ is pure, $N+N^*=M$.
\end{lemma}
\dixref{2.9.1}

\begin{proof}
The functional $\rho$ being positive, equation  \eqref{eq:defprodetat} holds. With $A=\cun$ we find 
\begin{equation}  \label{eq_pos_Bdtho}
| \rho(B)^2 |\leq\rho(\cun)\rho(B^*B)
\end{equation}
 and thus $\rho(A^*A)=0$ implies $\rho(A)=0$.

No proof for the second part.
\end{proof}


\begin{theorem}
The space of pure states of the (commutative) $C^*$-algebra $C_0(X)$ (the space of functions which are decreasing to zero at infinity) endowed with the relative $w^*$-topology is homeomorphic to $X$.
\end{theorem}
\lref{2.12.9}

\begin{probleme}
    The following proof is buggy and very unsure.
\end{probleme}

\begin{proof}
The $C^*$-algebra  $\cA=C_0(X)$ is not specially unital. If it is not, Gelfand theorem \ref{thoGelfand} says that there exists a locally compact and Hausdorff space $Y$ such that $\cA$ is isomorphic to $C_0(Y)$. If $\cA=C_0(X)$ with a non compact $X$, we begin to prove that the unitization is $\cA_{\cun}=C(\tilde X)$ where $\tilde X$ is the one point compactification of $X$. From proposition \ref{prop_unitariz_csa}, we have an unique unital $C^*$-algebra $\cA_{\cun}$  with an isometric morphism $\cA\to\cA_{\cun}$ such that $\cA_{\cun}/\cA\simeq \eC$.

 Therefore, we have to prove that $C(\tilde X)$ is a suitable $\cA_{\cun}$ and so it will be the unique unitization of $C_0(X)$. The identity map $\id\colon C_0(X)\to C(\tilde X)$ works. We have to check that $C(\tilde X)/C_0(X)\simeq\eC$. By identifying all functions of $C(\tilde X)$ which only differ by a function of $C_0(X)$, there are in fact only one function for each complex number $z$ : the which is constant (or another which is $z$ at $\infty$).

We have proved that if $\cA=C_0(X)$, then $\cA_{\cun}=C(\tilde X)$. From proposition \ref{prop_st_unit_ext} pure states on $C_0(X)$ uniquely extend to a pure state on $C(\tilde X)$. So if $X$ is non compact, we do not loss anything by considering $C(\tilde X)$ instead of $C_0(X)$. We still have to prove that taking $C(\tilde X)$ instead of $C(X)$ does not \emph{gain} anything: we must have $\etS\big( C_0(X) \big)=\etS\big( C(\tilde X) \big)$. In the case of unital $C^*$-algebras, pure states are linear functionals. The way to extend linear functionals from $\cA$ to $\cA_{\cun}$ is the same as the one to extend states.

If $X$ is compact, then $C_0(X)=C(X)$. Hence we are in both case ($X$ compact or not) reduced to prove the theorem for $C(X)$ with compact $X$.
 
Following proposition \ref{prop:comHauffhomeo}, $\Delta(C(X))$ is homeomorphic to $X$ because the latter is compact and Hausdorff. We have now to prove that there exists an homeomorphism between $\Delta(C(X))$ and set of pure states on $C(X)$. We are going to prove a bijection. Let on the one hand $\omega_x\in\Delta(C(X))$ be defined by
\begin{equation}
\begin{aligned}
 \omega_x\,:\,C(X)&\to \eC \\ 
\omega_x(f)&= f(x),
\end{aligned}
\end{equation}
and on the other hand a functional $\rho$ such that $0\leq\rho\leq\omega_x$. We have $\ker(\omega_x)\subset\ker(\rho)$. When $f$ is positive, $0\leq\rho(f)\leq\omega_x(f)$, so for any function,
\[ 
  o\leq\rho(f^*f)\leq\omega_x(f^*f),
\]
but $\omega_x$ is multiplicative, therefore $\rho(f^*f)\leq\omega_x(f^*)\omega_x(f)$. If $f\in\ker(\omega_x)$, we have $\rho(f^*f)=0$. Lemma concludes $\rho(f)=0$. So if $f\in\ker\omega_x$, we have $f(x)=0$. From theorem \ref{tho:ideal_kernel}, $\ker\omega_x$ is a maximal ideal while $\ker(\rho)$ is an ideal. So $\ker(\omega_x)$ is a maximal ideal contained in an ideal, therefore if $\rho\neq0$, $\ker(\omega_x)=\ker(\rho)$ which implies that $\rho=\lambda\omega_x$. Proposition \ref{prop_pureiff} concludes that $\omega_x$ is pure. 

\begin{probleme}
    Why is $\ker(\rho)$ an ideal ?
\end{probleme}


Now we suppose that $\omega$ is pure and we take  $g\in C(X)$ such that $0\leq g\leq 1_X$ on $C(X)$, we define
\begin{equation}
\begin{aligned}
 \omega_g\,:\,C(X)&\to \eC \\ 
f&\mapsto \omega(fg) 
\end{aligned}
\end{equation}
 So  $\omega(f)-\omega_g(f)=\omega\big( f(1-g) \big)$ and if $f$ is positive, $\omega(f)-\omega_g(f)\geq O$. So for a certain $t\in\eR^+$, we have 
\[ 
  \omega_g=t\omega
\]
 because $0\leq\omega_g\leq\omega$ and proposition \ref{prop_pureiff}. In particular, $\ker (\omega_g)=\ker(\omega)$, hence when $f\in \ker(\omega)$, for any $g\in C(X)$, $fg \in \ker(\omega)$ because any function in $C(X)$ is a linear combination of functions $g$ with $0\leq g\leq 1_X$. This proves that $\ker (\omega)$ is an ideal. On the other hand, $\ker (\omega)$ is a maximal ideal because the kernel of any functional on a vector space has codimension 1, see page \pageref{pg_codimun}. Theorem \ref{tho:ideal_kernel} shows that $\omega$ is multiplicative. So $\omega\in\Delta\big( C(X) \big)$.

\end{proof}

\subsection{Existence of pure states}
%------------------------------------

It is possible for $\etS$ to do not contain pure states. It is the case when $\etS$ is a convex cone. Such a $C^*$-algebra  has no irreducible representations. We are going to prove that this case is not possible. We define the convex hull of the part $A$ of a vector space by
\[ 
  co(A)=\{ \lambda+(1-\lambda)w\text{ with }w\in A,\lambda\in[0,1] \}.
\]



\begin{theorem}
 A compact connected set $K$ in a locally convex vector space is the closure of the convex hull of its extreme points. In other words :
    \[ 
  K=\overline{ co(\partial_eK) }.
\]

\end{theorem}
\begin{proof}
No proof
\end{proof}


\begin{lemma}

 Let $\cA$ be an unital $C^*$-algebra  and $\cB$ an self-adjoint vector subspace of $\cA$ with $\cun\in\cB$. Let $F$ be the set of linear forms $g$ on $B$ such that


\begin{itemize}
\item  $g(A^*) = \overline{g(A)}$ for all $A\in\cB$,
 \item $g(A) \geq 0$ for all $A\in \cB\cap \cA^+$,
\item $g(\cun)=1$.
\end{itemize}
 Any element of $F$ can be extended into a state on $\cA$.
\label{lem_DixcBprol}
 \end{lemma}
\dixref{1.10.1}


\begin{theorem}  
For all $A\in\cA$ and a $a\in\sigma(A)$, we have a pure state $\omega_a$ on $\cA$ such that $\omega_a(A)=a$. There also exists a pure state $\omega$ such that $| \omega(A) |=\| A \|$.
\label{tho_existsetat}
\end{theorem}
\lref{2.12.11}

\begin{proof}
Let us take an intermediary result in proof of lemma \ref{prop_st_unit_ext} :
\begin{equation}
\begin{aligned}
 \tilde\omega,:\,\eC A+\eC\cun&\to \eC \\ 
(\lambda A+\mu\cun)&\mapsto \lambda a+\mu 
\end{aligned}
\end{equation}
We extend this state by continuity and multiplicatively to $C^*(A,\cun)$ with formulas as $\tilde\omega_a(A^n)=a^n$. We have to check that this extension is pure: $\tilde\omega_a$ is positive and belongs to $\Delta(C^*(A,\cun))$. 

Positivity comes from assumption that $A\in\cA_{\eR}$. Indeed, $a\in\sigma(A)\subset\eR^+$ . So positives elements in $C^*(A,\cun)$ are even power (and completion) of $A$. So the images are even powers of  $a\in\eR$ and are therefore positives. 
The fact that $\tilde\omega_a$ is multiplicative on $C^*(A,\cun)$ comes from the fact that it is the same, in expressions as 
$\tilde\omega_a(xy)$, to distribute inside the $\tilde\omega_a$ and push out terms $a^n$ by linearity, or write $\tilde\omega_a(x)\tilde\omega_a(y)$ and distribute outside. So $\tilde\omega$ is a pure state in $C^*(A,\cun)$.

 We consider the set $K_a$ of extensions of $\tilde\omega_a$ which are states on $\cA$. The fact that $K_a$ is non empty comes from the lemma \ref{lem_DixcBprol}. 

Let us now prove that $K_a$ is convex. For, we take $\omega_1$ and $\omega_2$, two extensions of $\tilde\omega_a$ and we will prove that $\omega=\lambda\omega_1+(1-\lambda)\omega_2$ is an extension too. By linearity, $\omega$ is a state. It is an extension of $\tilde\omega_a$ because
\[ 
\begin{split}
\omega(\cun+A^2)&=\lambda\omega_1(\cun+A^2)+(A-\lambda)\omega_1(\cun+A^2)\\
        &=\lambda(1+a^2)+(1-\lambda)(1+a^2)\\
        &=1+a^2.
\end{split}  
\]

 In order to prove that $K_a$ is closed, we prove that its complement is open. Let $\omega\notin K_a$. We will prove that there exists $\varepsilon$ such that for all $\eta$ with $\| \omega-\eta \|\leq\varepsilon$. Let $\lambda$ be such that $\| \lambda A \|=1$ in such a manner that $\tilde\omega_a(\lambda A)=\lambda A$. Let $\omega(A)=s$ ($\omega\notin K_a$). We have
\[ 
\begin{split}
\| \omega-\eta \|&=\sup\{ | \omega-\eta |\text{ st } \| B \|=1 \}\\
        &\geq | (\omega-\eta)(\lambda A) |\\
        &=| \omega(\lambda A)-\eta(\lambda A) |,
\end{split}  
\]
but
\[ 
  | \omega(\lambda A)-\eta(\lambda A) |\leq \| \omega-\eta \|\leq\varepsilon.
\]
So $| s-\eta(\lambda A) |\leq \varepsilon$, and when $\varepsilon$ is small (for example when $\varepsilon<| s-\lambda a |$), $\eta(\lambda A)$ is close to $s$ which is different of $\lambda A$. This proves that $K_a$ is closed.

We know that $K_a$ is convex and closed. So it has at least one extreme point. Let $\omega_a$ be one of them. We are going to prove that it is also extreme in $\etS$. If not it can be decomposed as $\omega=\lambda\omega_1+(1-\lambda)\omega_2$. Taking the latter at $A$ shows that, on $C^*(A,\cun)$, the functionals $\omega$, $\omega_1$ and $\omega_2$ are equals. So $\omega_a$ cannot be an extreme point in $K_a$. This concludes the first part of the proof. 


Theorem \ref{tho:prop_sigma} says us that in a Banach algebra, $\sigma(A)$ is compact for all $A$. So there exists a $a\in\sigma(A)$ such that $r(A)=| a |$. With this $a$,
 \[ 
  | \omega_a(A) |=| A |=r(A)=\| A \|.
\]
because $A=A^*$.

\end{proof}

The Gelfand Neumark theorem was proved using lemma \ref{lem:omAenomA}. Now we have at hand a refining of this lemma. Hence we can use
\[ 
  \pi_r=\bigoplus_{\omega\in\etP(\cA)}\pi_{\omega}
\]
instead of the universal representation. The justification of this claim is that when $A$ is such that $\omega(A^*A)=0$ for all pure states, $\| A^*A \|=0$. Indeed $A^*A$ is positive, so there exists a pure state $\omega_a$ such that $\omega_a(A^*A)=\| A^*A \|$. Then $\| A^*A \|=0$.
 
Now we say that two states are \defe{equivalent}{equivalence!of states}\index{state!equivalent} if their GNS representations are equivalent. Gelfand Neumark theorem  says that
\[ 
  \cA\simeq\pi_r(\cA):=\bigoplus_{\omega\in[\etP(\cA)]}\pi_{\omega}(\cA).
\]
where $[\etP(\cA)]$ stands for the set of equivalences classes in $\etP(\cA)$.

\begin{proposition}
Any finite dimensional $C^*$-algebra is isomorphic to a direct sum of matricial algebras.
\end{proposition}
\begin{proof}
    No proof.
\end{proof}

\section{Generalization of matrix \texorpdfstring{$C^*$}{C}-algebra}
%+++++++++++++++++++++++++++++++++++++++++++++++

If we want to generalize the $C^*$-algebra $\mfM^n(\eC)$ to infinite dimensional Hilbert spaces, we first try to use the $C^*$-algebra of bounded operators. It does not work because such a $C^*$-algebra possesses many non equivalent representations on non separable Hilbert spaces.

\subsection{Example}
%------------------

Let us consider the $C^*$-algebra $\oB(\hH)$ and its definition representation which is obviously irreducible. From GNS construction and corollary \ref{cor_GNSirredst}, we know that all the GNS representations build from a vector state are equivalent to the definition one.

On the other hand, there exists some self-adjoint bounded operators with non empty continuous spectrum. The take $A\in\oB(\hH)$ and $a\in\sigma(A)$ such that there are no $\psi_a\in\hH$ such that $A\psi_a=a\psi_a$. Theorem \ref{tho_existsetat} gives a pure state $\omega_a$ such that $\omega_a(A)=a$, and hence a GNS representation $\pi_a$ on $\hH$. This representation is irreducible because $\omega_a$ is pure. In $\pi_a$, we have  a cyclic vector $\Omega_a$ such that
\[ 
  \scal{\Omega_a}{\pi_a(A)\Omega_a}=\omega_a(A)=a.
\]

%+++++++++++++++++++++++++++++++++++++++++++++++++++++++++++++++++++++++++++++++++++++++++++++++++++++++++++++++++++++++++++
\section{Tensor product}            \label{SecTensProdCSA}
%+++++++++++++++++++++++++++++++++++++++++++++++++++++++++++++++++++++++++++++++++++++++++++++++++++++++++++++++++++++++++++

If $\cA$ and $\cB$ are $C^*$-algebra, the \defe{tensor product}{tensor product!of $C^*$-algebra} is the completion of the space generated by the finite sums of the form $\sum_{i=1}^n A_i\otimes B_i$ with $A_i\in\cA$ and $B_i\in\cB$.

As an example of the importance of the completion, consider a compact group $G$ and $\cA=C(G)$ the $C^*$-algebra of continuous functions on $G$. We can build the map $\Delta\colon \cA\to \cA\otimes \cA$ by
\begin{equation}
    \Delta(f)(x,y)=f(xy)
\end{equation}
for every $x$ and $y$ in $G$ and $f\in C(G)$. The well-definiteness of $\Delta$ is due to the fact that $C(G)\otimes C(G)\simeq C(G\times G)$ by completion. This trick is used whenever we define a coproduct on a space of functions on a group, see for example subsection \ref{SubSecHoptUnivecvgp} and section \ref{SecExtenLemK} around equation \eqref{EqCABsimeqCACB}.

If $A$ and $B$ are manifolds, we have
\begin{equation}
    C^{\infty}(A)\otimes C^{\infty}(B)\simeq C^{\infty}(A\times B)
\end{equation}
by the map
\begin{equation}        \label{EqIsoCABCACBCstar}
    \begin{aligned}
        \varphi\colon  C^{\infty}(A)\otimes C^{\infty}(B)&\to  C^{\infty}(A\times B) \\
        \sum_ia_i\otimes b_i&\mapsto \Big[ (x,y)\mapsto\sum_ia_i(x)b_i(y) \Big]. 
    \end{aligned}
\end{equation}
The image by $\varphi$ of the \emph{algebraic} tensor product $ C^{\infty}(A)\otimes C^{\infty}(B)$ is dense in $ C^{\infty}(A\times B)$ as for example the polynomials are contained in the image. Indeed let $f(x,y)=\sum_{ij}f_{ij}x^iy^j$. The function $f$ is the image by $\varphi$ of
\begin{equation}        \label{EqDecompffklCABCACB}
    \sum_{kl} f_{kl} a_k\otimes b_l
\end{equation}
where $a_k(x)=x^k$ and $b_k(y)=y^k$. Thus, if we consider the $C^*$-algebraic tensor product, we have the equality.

See also \cite{Delaroche} for the sequel about tensor products. Let $\cA_1$ and $\cA_2$ be $C^*$-algebra and $\cA_1\odot\cA_2$ be their algebraic tensor product. There are at least two ways to define a $C^*$-norm on the $*$-algebra $\cA_1\odot\cA_2$.

\begin{enumerate}
    \item
        The \defe{maximal norm}{norm!maximal}\index{maximal!norm} of $A\in\cA_1\odot$ is defined by
        \begin{equation}
            \| A \|_{max}=\sup_{\pi}\| \pi(A) \|
        \end{equation}
        where the supremum is taken over all the representations\footnote{i.e. all the homomorphisms $\pi\colon \cA_1\odot\cA_2\to \opB(\hH)$.} of $\cA_1\odot\cA_2$ over some Hilbert space $\hH$. The \defe{maximal tensor product}{maximal!tensor product} is the completion of $\cA_1\odot\cA_2$ for that norm.

    \item
        The \defe{minimal norm}{minimal!norm} is obtained by taking the supremum only over the representations of the for $\pi_1\otimes \pi_2$ :
        \begin{equation}
            \| x \|_{min}=\sup_{\pi_1,\pi_2}\| (\pi_1\otimes\pi_2)(x) \|
        \end{equation}
        where $\pi_i$ is a representation of $\cA_i$ on an Hilbert space $\hH_i$.
\end{enumerate}

\begin{lemma}
    If $\cA_1$ and $\cA_2$ are sub-$C^*$-algebra of $\opB(\hH_1)$ and $\opB(\hH_2)$, then $\cA_1\otimes_{min}\cA_2$ is the closure of $\cA_1\odot\cA_2$ seen as subalgebra of $\opB(\hH_1\otimes\hH_2)$.
\end{lemma}


%+++++++++++++++++++++++++++++++++++++++++++++++++++++++++++++++++++++++++++++++++++++++++++++++++++++++++++++++++++++++++++
\section{Traces over $C^*$-algebra }
%+++++++++++++++++++++++++++++++++++++++++++++++++++++++++++++++++++++++++++++++++++++++++++++++++++++++++++++++++++++++++++
\label{SecTraceCstar}

Source : \cite{DixmierTrace}. For tracial functionals on von Neumann algebras, see section \ref{SecTracevonNeuman}.

\begin{definition}
    Let $\cA$ be a $C^*$-algebra. A \defe{trace}{trace!over $C^*$-algebra} on $\cA^+$ is a function $\tau\colon A^+\to \mathopen[ 0 , \infty \mathclose]$ such that
    \begin{enumerate}
        \item
            if $A$ and $B$ are in $\cA^+$, $\tau(A+B)=\tau(A)+\tau(B)$,
        \item
            If $A\in\cA^+$ and $\lambda\in\eR^+$, then $\tau(\lambda A)=\lambda\tau(A)$. Here if $\lambda=0$, we pose $0\times\infty=0$;
        \item
            If $Z\in\cA$, we have $\tau(ZZ^*)=\tau(Z^*Z)$.
    \end{enumerate}
    We say that $\tau$ is semifinite if for every $A\in\cA^+$,
    \begin{equation}
        \tau(A)=\sup\{ \tau(B)\tq B\in\cA^+,B\leq A,\tau(B)<\infty \}.
    \end{equation}
\end{definition}
We recall that for every $Z\in\cA$, we have $ZZ^*\in\cA^+$.

The following lemma is the lemma $3$ in \cite{DixmierTrace}.
\begin{lemma}       \label{LemTraceAplusextmlmn}
    Let $\cA$ be a $C^*$-algebra and $\tau$ a trace on $\cA^+$. Then the following hold.
    \begin{enumerate}
        \item
            The set
            \begin{equation}
                \mL=\{ A\in\cA\tq\tau(AA^*)<\infty \}
            \end{equation}
            is a bilateral ideal in $\cA$.
        \item
            The set $\mN=\langle \mL^2\rangle$ is the set of complex linear combinations of $\mN^+$: $\mN=\langle \mN^+\rangle.$
        \item
            We have
            \begin{equation}
                \mN^+=\{ A\in\cA^+\tq\tau(A)<\infty \}.
            \end{equation}
        \item
            There exists one and only one linear form $f$ on $\mN$ which coincides with $\tau$ on $\mN^+$.
        \item
            The linear form $f$ satisfies
            \begin{enumerate}
                \item
                    $f(A^*)=\overline{ f(A) }$;
                \item
                    $f(AB)=f(BA)$ for every $u$ and $v$ in $\mL$;
                \item
                    $f(ZA)=f(AZ)$ for every $A\in\mN$ and $Z\in \cA$.
            \end{enumerate}
            
    \end{enumerate}
    
\end{lemma}

\begin{proof}
    Let us begin by pointing out the fact that if $A\leq B$, then $\tau(A)\leq\tau(B)$ because of linearity: $\tau(B)=\tau(A)+\tau(B-A)\geq\tau(A)$ since $B-A\in\cA^+$.

    \begin{enumerate}
        \item
            If $A\in\mL$, then $A^*\in\mL$ because $\tau(A^*A)=\tau(AA^*)$ (by definition of a trace). If $A,B\in\mL$, we have
            \begin{equation}
                (A+B)(A+B)^*\leq 2(AA^*+BB^*),
            \end{equation}
            so that
            \begin{equation}
                \tau\big( (A+B)(A+B)^* \big)<2\tau(AA^*)+2\tau(BB^*)<\infty.
            \end{equation}
            This proves that $A+B\in\mL$.

            Let now $A\in\mL$ and $Z\in\cA$. Since $AZZ^*X^*\leq\| ZZ^* \|AA^*$, we have 
            \begin{equation}
                \tau\big( AZ(AZ)^* \big)<\infty.
            \end{equation}
            So $\mL$ is a right ideal in $\cA$. This is also a left ideal because $ZA=(A^*Z^*)^*$, but the fact that $A^*\in\mL$ implies $A^*Z^*\in\mL$, so that $ZA\in\mL$.
        \item
            An element $X$ in $\mN$ reads
            \begin{equation}
                X=\sum_{j=1}^{n}A_jB_j^*
            \end{equation}
            with $A_j,B_j\in\mL$. The \defe{polarization}{polarization} relation reads
            \begin{equation}        \label{EqPolaXAB}
                \begin{aligned}[]
                    4X&=\sum_j(A_j+B_j)(A_j+B_j)^*\\
                    &\quad+\sum_j(A_j-B_j)(A_j-B_j)^*\\
                    &\quad+\sum_j(A_j+iB_j)(A_j+iB_j)^*\\
                    &\quad+\sum_j(A_j-iB_j)(A_j-iB_j)^*.
                \end{aligned}
            \end{equation}
            So $X$ is a linear combination of elements in $\mN^+$ (that is elements of the form $AA^*$ with $A\in\mL$). Notice that $A\in\mL$ implies $iA\in\mL$ because
            \begin{equation}
                \tau\big( iA(iA)^* \big)=-\tau(iAiA^*)=\tau(AA^*)<\infty.
            \end{equation}
            This proves that $\mN\subset\langle \mN^+\rangle$. The fact that $\langle \mN^+\rangle$ is a subset of $\mN$ is by construction.
        \item
            An element $X$ in $\langle \mN^+\rangle$  reads $X=\sum_jA_jB^*_j$ where, for each $j$, we have $A_jB_j^*\in\mN^+$, in particular $A_jB^*_j=(A_jB^*_j)^*=B_jA_j^*$. We can still write down the polarization identity, but now the last two terms of \eqref{EqPolaXAB} give $2i(B_jA_j^*-A_jB_j^*)=0$. 

            Thus we have
            \begin{equation}
                \begin{aligned}[]
                    4X&=\sum_j(A_j+B_j)(A_j+B_j)^*-\sum_j(A_j-B_j)(A_j-B_j)^*\\
                    &\leq\sum_j(A_j+B_j)(A_j+B_j)^*,
                \end{aligned}
            \end{equation}
            but we already know that $\tau\big( \sum_j(A_j+B_j)(A_j+B_j)^* \big)<\infty$. Thus we have $\tau(X)<\infty$. So we proved that
            \begin{equation}
                \mN^+\subset\{ X\in\cA^+\tq\tau(X)<\infty \}.
            \end{equation}
            
            Let now $X\in\cA^+$ be such that $\tau(X)<\infty$. In order to prove that $X\in\mN^+$, it is sufficient to prove that $X\in\mN$. Since $X=X^*$, we can use the continuous functional calculus (see theorem \ref{ThoContFuncCalculus} and remark \ref{RemExpansionSqrtConCal}) in order to define $X^{1/2}$. We have
            \begin{equation}
                \tau\big( X^{1/2}(X^{1/2})^* \big)=\tau\big( X^{1/2}X^{1/2} \big)<\infty,
            \end{equation}
            so that $X^{1/2}\in\mL$.
        \item
            Since $\mN$ is generated by $\mN^{+}$, the functional $\tau$ on $\mN^+$ there is one and only one extension of $\tau$ to a linear functional $f$ on $\mN$.
        \item
            \begin{enumerate}
                \item
                    An element of $\mN$ is a linear combination of elements of $\mN^+$: $X=\sum_j\lambda_iA_i$ with $A_i\in\mN^+$. Thus using the linearity of $f$ and the properties of $\tau$, we have
                    \begin{equation}
                        \begin{aligned}[]
                            f(X^*)&=f\big( \sum_j\overline{ \lambda_j }A_j^* \big)\\
                            &=\sum_j\overline{ \lambda_j }f(A_j)\\
                            &=\sum_j\overline{ \lambda_j }\underbrace{\tau(A_j)}_{\in\eR^{+}}\\
                            &=\sum_j\overline{ \lambda_j \tau(A_j)}\\
                            &=\overline{ f(X) }.
                        \end{aligned}
                    \end{equation}
                    This is the first property we had to check.
                \item
                    If $A\in\mL$, we have $AA^*\in\mN$ and by definition of a trace,
                    \begin{equation}
                        f(AA^*)=\tau(AA^*)=\tau(A^*A)=f(A^*A).
                    \end{equation}
                    If $A$ and $B$ belong to $\mL$, we use the polarization identity:
                    \begin{equation}
                        4AB^*=(A+B)(A+B)^*-(A-B)(A-B)^*+i(A+iB)(A+iB)^*-i(A-iB)(A-iB)^*,
                    \end{equation}
                    and we do the same computation in order to get $f(AB^*)=f(B^*A)$ whenever $A$ and $B$ belong to $\mL$.
                \item
                    Let $Z\in\cA$. An element in $\mN$ reads $X=\sum_jA_jB_j$ with $A_j$ and $B_j$ in $\mL$. Since $\mL$ is an bilateral ideal in $\cA$ we have $ZA_j\in\mL$ and $B_jZ\in\mL$, thus we can make the following computation:
                    \begin{equation}
                        \begin{aligned}[]
                            f\big( Z\sum_jA_jB_j \big)&=\sum_jf\big( (ZA_j)B_j \big)\\
                            &=\sum_jf\big( B_j(ZA_j) \big)\\
                            &=\sum_jf\big( (B_jZ)A_j \big)\\
                            &=\sum_jf\big( A_j(B_jZ) \big)\\
                            &=f\big( (\sum_jA_jB_j)Z \big)\\
                            &=f(XZ).
                        \end{aligned}
                    \end{equation}              
            \end{enumerate}
            This concludes the proof of the lemma.
    \end{enumerate}
    
    
\end{proof}
