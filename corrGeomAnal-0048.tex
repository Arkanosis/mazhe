\begin{corrige}{GeomAnal-0048}

    \begin{enumerate}
        \item
            La variable \( x\) va de \( 0\) à \( 2\) tandis que pour chaque \( x\), la variable \( y\) va de \( 0\) à \( x\). Nous devons donc effectuer l'intégrale
            \begin{equation}
                \int_0^2\int_0^x(x+y)^3dy\,dx.
            \end{equation}
            Cela est un peu de calcul et la réponse est \( 24\) :
            \begin{verbatim}
----------------------------------------------------------------------
| Sage Version 4.7.1, Release Date: 2011-08-11                       |
| Type notebook() for the GUI, and license() for information.        |
----------------------------------------------------------------------
sage: f(x,y)=(x+y)**3
sage: f.integrate(y,0,x).integrate(x,0,2)
(x, y) |--> 24
            \end{verbatim}

        \item
            Les bornes d'intégration sont \( \theta\colon 0\to 2\pi\) et \( r\colon 0\to 1+\cos(\theta)\). L'intégrale en coordonnées polaires est donc
            \begin{equation}
                \int_0^{2\pi}d\theta\int_{0}^{1+\cos(\theta)}r\,dr\,d\theta.
            \end{equation}
            Ne pas oublier le jacobien qui vaut \( r\) en coordonnées polaires. Une fois encore c'est un peu de calcul et la réponse est \( \frac{ 3 }{2}\pi\) :
            \begin{verbatim}
sage: f(r,t)=r
sage: f.integrate(r,0,1+cos(t)).integrate(t,0,2*pi)
(r, t) |--> 3/2*pi
            \end{verbatim}
        \item
            L'équation \( (x+1)^2+y^2=1\) décrit un cercle centré en \( (-1,0)\). Le volume étant invariant par translation nous allons considérer le cylindre de même hauteur, mais dont la base est centrée en \( (0,0)\). En coordonnées cylindriques,
            \begin{subequations}
                \begin{numcases}{}
                    x=r\cos(\theta)\\
                    y=r\sin(\theta)\\
                    z=z,
                \end{numcases}
            \end{subequations}
            les équations qui définissent le cylindre sont \( r\leq 1\), \( 0\leq z\leq 3\). Nous avons donc l'intégrale
            \begin{equation}
                V=\int_0^1dr\int_0^3dz\int_0^{2\pi}d\theta\,r.
            \end{equation}
            Le résultat est celui attendu :
            \begin{verbatim}
sage: f(r,theta,z)=r
sage: f.integrate(theta,0,2*pi).integrate(z,0,3).integrate(r,0,1)
(r, theta, z) |--> 3*pi
            \end{verbatim}

    \end{enumerate}

\end{corrige}
