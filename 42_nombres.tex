% This is part of Mes notes de mathématique
% Copyright (c) 2011-2015
%   Laurent Claessens
% See the file fdl-1.3.txt for copying conditions.


%+++++++++++++++++++++++++++++++++++++++++++++++++++++++++++++++++++++++++++++++++++++++++++++++++++++++++++++++++++++++++++ 
\section{Quelque éléments sur les ensembles}
%+++++++++++++++++++++++++++++++++++++++++++++++++++++++++++++++++++++++++++++++++++++++++++++++++++++++++++++++++++++++++++

\begin{definition}      \label{DefEOZLooUMCzZR}
    Un ensemble est \defe{infini}{ensemble!infini} si il peut être mis en bijection avec un de ses sous-ensembles propres.
\end{definition}

%--------------------------------------------------------------------------------------------------------------------------- 
\subsection{Lemme de Zorn}
%---------------------------------------------------------------------------------------------------------------------------

\begin{definition}  \label{DefGHDfyyz}
    Un ensemble est \defe{totalement ordonné}{ordre!total} si deux éléments sont toujours comparables : si \( x,y\in E\) alors nous avons soit \( x<y\) soit \( y<x\).

    Un ensemble est \defe{inductif}{inductif} si tout sous-ensemble ordonné admet un majorant.
\end{definition}

Si \( E\) est un ensemble, l'inclusion est un ordre sur l'ensemble des partie de \( E\), mais pas un ordre total parce que si \( X,Y\) sont des parties de \( E\), alors nous n'avons pas automatiquement soit \( X\subset Y\) ou \( Y\subset X\).

\begin{lemma}[Lemme de Zorn]    \label{LemUEGjJBc}
    Tout ensemble ordonné inductif non vide admet un maximum.
\end{lemma}
\index{lemme!de Zorn}
%TODO : une preuve.
Le point intéressant de ce lemme est que le majorant soit un maximum, c'est à dire qu'il appartienne à l'ensemble.

\begin{proposition}[\cite{KZIoofzFLV}, lemme 8.1] \label{PropVCSooMzmIX}
    Si \( S\) est un ensemble infini alors il existe une bijection \( \varphi\colon \{ 1,2 \}\times S\to S\).
\end{proposition}
%TODO : la preuve

%--------------------------------------------------------------------------------------------------------------------------- 
\subsection{Complémentaire}
%---------------------------------------------------------------------------------------------------------------------------
\label{AppComplement}

Soit $E$, un ensemble et $A$, une partie de $E$ (c'est à dire un sous-ensemble de $E$). Nous désignons par $\complement A$\nomenclature[T]{$\complement A$}{Le complémentaire de l'ensemble $A$} désigne le \defe{complémentaire}{complémentaire} de l'ensemble $A$ dans $E$. Il s'agit de l'ensemble des points de $E$ qui ne font pas partie de $A$ :
\begin{equation}
	\complement A=E\setminus A=\{ x\in E\tq x\notin A \}.
\end{equation}
Nous allons aussi régulièrement noter le complémentaire de \( A\) par \( A^c\)\nomenclature[T]{\( A^c\)}{complémentaire de \( A\)}.

\begin{lemma}		\label{LemPropsComplement}
	Quelque propriétés à propos des complémentaires. Si $E$ est un ensemble et si $A$ et $B$ sont des sous-ensembles de $E$, nous avons
	\begin{enumerate}
		\item
			$\complement \complement A =A $, en d'autres termes, $E\setminus(E\setminus A)=A$,
		\item
			$\complement(A\cap B)=\complement A\cup\complement B$,
		\item
			$\complement(A\cup B)=\complement A\cap\complement B$,
		\item	\label{ItemLemPropComplementiii}
			$A\setminus B=A\cap\complement B$.
	\end{enumerate}
\end{lemma}

\begin{definition}[différence symétrique]    \label{DefBMLooVjlSG}
    Si \( A\) et \( B\) sont des ensembles, l'ensemble \( A\Delta B\)\nomenclature[T]{\( A\Delta B\)}{différence symétrique} est la \defe{différence symétrique}{ensemble!différence symétrique} d'ensembles : 
    \begin{equation}
        A\Delta B=(A\cup B)\setminus(A\cap B).
    \end{equation}
\end{definition}
C'est l'ensemble des éléments étant soit dans \( A\) soit dans \( B\) mais pas dans les deux.

\begin{lemma}   \label{LemCUVoohKpWB}
    Si \( A\) et \( B\) sont des ensembles nous avons
    \begin{enumerate}
        \item\label{ItemVUCooHAztC}
            \( A^c\Delta B^c=A\Delta B\).
        \item\label{ItemVUCooHAztCii}
            \( (A\Delta B)\Delta B=A\).
    \end{enumerate}
\end{lemma}

\begin{proof}
    La première assertion provient de l'égalité \( X^c\setminus Y^c=Y\setminus B\) (qu'on montre de façon classique, éventuellement en séparant les cas suivant que \( B\) est inclus à \( A\) ou non) :
    \begin{equation}
        A^c\Delta B^c=(A^c\cap B^x)\setminus(A^c\cap B^c)=(A\cap B)^c\setminus(A\cup B)^c=(A\cup B)\setminus (A\cap B)=A\Delta B.
    \end{equation}

    Pour la seconde assertion, il faut remarquer que \( (A\Delta B)\cup B=A\cup B\) et que \( (A\Delta B)\cap B=B\setminus A\), donc
    \begin{equation}
        (A\Delta B)\Delta B=(A\cup B)\setminus (B\setminus A)=A.
    \end{equation}
\end{proof}

%--------------------------------------------------------------------------------------------------------------------------- 
\subsection{Relations d'équivalence}
%---------------------------------------------------------------------------------------------------------------------------
\label{appEquivalence}

\begin{definition}  \label{DefHoJzMp}
Si $E$ est un ensemble, une \defe{relation d'équivalence}{equivalence@équivalence!relation} sur $E$ est une relation $\sim$ telle qui est à la fois
\begin{description}
	\item[réflexive] $x\sim x$ pour tout $x\in E$,
	\item[symétrique] $x\sim y$ si et seulement si $y\sim x$;
	\item[transitive] si $x\sim y$ et $y\sim z$, alors $x\sim z$.
\end{description}
\end{definition}

Par exemple, sur l'ensemble de tous les polygones du plan, la relation «a le même nombre de côté» est une relation d'équivalence. Plus précisément, si $P$ et $Q$ sont deux polygones, nous disons que $P\sim Q$ si et seulement si $P$ et $Q$ ont le même nombre de côté. Cela est une relation d'équivalence :
\begin{itemize}
	\item 
		un polygone $P$ a toujours le même nombre de côtés que lui-même : $P\sim P$;
	\item
		si $P$ a le même nombre de côtés que $Q$ ($P\sim Q$), alors $Q$ a le même nombre de côtés que $P$ ($Q\sim P$);
	\item
		si $P$ a le même nombre de côtés que $Q$ ($P\sim Q$) et que $Q$ a le même nombre de côtés que $R$ ($Q\sim R$), alors $P$ a le même nombre de côtés que $R$ ($P\sim R$).
\end{itemize}

Soit \( f\) une application entre deux ensembles \( E\) et \( F\). Nous définissons une relation d'équivalence sur \( E\) par
\begin{equation}
    x\sim y\Leftrightarrow f(x)=f(y).
\end{equation}
Nous notons par \( \pi\colon E\to E/\sim\) la projection canonique. L'application
\begin{equation}
    \begin{aligned}
        g\colon E/\sim&\to F \\
        [x]&\mapsto f(x) 
    \end{aligned}
\end{equation}
est bien définie et injective. Elle n'est pas surjective tant que \( f\) ne l'est pas. La \defe{décomposition canonique}{canonique!décomposition}\index{décomposition!canonique} de \( f\) est 
\begin{equation}
    f=g\circ\pi.
\end{equation}

%+++++++++++++++++++++++++++++++++++++++++++++++++++++++++++++++++++++++++++++++++++++++++++++++++++++++++++++++++++++++++++ 
\section{Les naturels}
%+++++++++++++++++++++++++++++++++++++++++++++++++++++++++++++++++++++++++++++++++++++++++++++++++++++++++++++++++++++++++++

\nomenclature{$\eN_0$}{les naturels non nuls : $\eN_0=\eN\setminus\{ 0 \}$}
\begin{definition}
    Un ensemble est \defe{dénombrable}{dénombrable} si il peut être mis en bijection avec \( \eN\). Il est \defe{non dénombrable}{non dénombrable} si il est infini et ne peut pas être mis en bijection avec \( \eN\).
\end{definition}
Mettant cela en rapport avec la définition \ref{DefEOZLooUMCzZR}, ce qui est bien c'est qu'un ensemble fini n'est ni dénombrable ni non dénombrable.

\begin{proposition} \label{PropQEPoozLqOQ}
    Toute partie d'un ensemble fini est finie, et toute partie d'un ensemble dénombrable est finie ou dénombrable.
\end{proposition}
%TODO : la preuve

%+++++++++++++++++++++++++++++++++++++++++++++++++++++++++++++++++++++++++++++++++++++++++++++++++++++++++++++++++++++++++++ 
\section{Les entiers}
%+++++++++++++++++++++++++++++++++++++++++++++++++++++++++++++++++++++++++++++++++++++++++++++++++++++++++++++++++++++++++++

Toutes les constructions sont faites dans \cite{RWWJooJdjxEK}.

%+++++++++++++++++++++++++++++++++++++++++++++++++++++++++++++++++++++++++++++++++++++++++++++++++++++++++++++++++++++++++++ 
\section{Les rationnels}
%+++++++++++++++++++++++++++++++++++++++++++++++++++++++++++++++++++++++++++++++++++++++++++++++++++++++++++++++++++++++++++

%--------------------------------------------------------------------------------------------------------------------------- 
\subsection{Suites de Cauchy dans les rationnels}
%---------------------------------------------------------------------------------------------------------------------------

\begin{definition}
    La suite \( (x_n)\) dans \( \eQ\) est \defe{de Cauchy}{suite!de Cauchy!dans $\eQ$} si pour tout \( \epsilon\in \eQ^+\), il existe \( n\in \eN\) tel que si \( p,q\geq N\) alors \( | x_p-x_q |\leq \epsilon\).
\end{definition}

\begin{definition}
    La suite \( (x_n)\) dans \( \eQ\) est \defe{convergente}{convergence!suite!dans $ \eQ$} si il existe \( q\in \eQ\) tel que pour tout \( \epsilon\in \eQ^+\), il existe \( N\) tel que si \( k\geq N\) alors \( | x_k-q |\leq \epsilon\).
\end{definition}

\begin{proposition}[\cite{RWWJooJdjxEK}]
    Principales propriétés des suites de Cauchy dans \( \eQ\).
    \begin{enumerate}
        \item
            Toute suite convergente est de Cauchy\footnote{Et non la réciproque, qui sera justement la grande innovation des nombres réels.}.
        \item
            Toute suite de Cauchy est bornée.
    \end{enumerate}
\end{proposition}

\begin{proof}
    Point par point.
    \begin{enumerate}
        \item
            Soit \( (x_n)\) une suite dans \( \eQ\) qui converge vers \( x\in \eQ\). Soit aussi\footnote{Le \( \epsilon\) que nous prenons maintenant est dans \( \eQ\), et ce sera toujours ainsi dans les prochaines pages; nous ne le répéterons pas à chaque fois.} \( \epsilon>0\). Il existe \( N_{\epsilon}\in \eN\) tel que \( n>N_{\epsilon}\) implique \( | x_n-x |\leq \epsilon\). Si \( p,q\geq N_{\epsilon/2}\) alors
            \begin{equation}
                | x_p-x_q |\leq | x_p-x |+| x-x_q |\leq \epsilon.
            \end{equation}
            Donc la suite \( (x_n)\) est de Cauchy.
        \item
            Soit \( (x_n)\) une suite de Cauchy dans \( \eQ\). Avec \( \epsilon=1\) dans la définition, si \( q>N_1\), nous avons
            \begin{equation}
                | x_q-x_{N_1} |\leq 1.
            \end{equation}
            <++>
    \end{enumerate}
    <++>
\end{proof}
<++>


%--------------------------------------------------------------------------------------------------------------------------- 
\subsection{Insuffisance des rationnels}
%---------------------------------------------------------------------------------------------------------------------------

Nous allons voir qu'il n'existe pas de nombres rationnels \( x\) tels que \( x^2=2\), mais que pourtant il existe une infinité de suites de rationnels \( (x_n)\) tels que \(  x_n^2\to 2  \).

\begin{lemma}       \label{LemJPIUooWFHaFM}
    Un entier \( x\) est pair si et seulement si l'entier \( x^2\) est pair\footnote{Le carré d'un nombre a en fait les mêmes diviseurs premiers que le nombre, mais nous n'en avons pas besoin ici.}.
\end{lemma}

\begin{proof}
    Si \( x\) est un nombre pair, alors il existe un entier \( a\) tel que \( x=2a\) alors \( x^2=4a^2\) est pair.

    Inversement, si \( x\) est impair alors il existe un entier \( a\) tel que \( x=2a+1\) et alors \( x^2=4a^2+4a+1=2(2a^2+2a)+1\) est impair.
\end{proof}

\begin{proposition}[Irrationalité de \( \sqrt{2}\)]
    Il n'existe pas de fractions d'entiers dont le carré soit égal à \( 2\).
\end{proposition}
\index{irrationalité!\( \sqrt{2}\)}

\begin{proof}
    Nous supposons que la fraction d'entiers \( a/b\) est telle que \( a^2/b^2=2\), et nous allons construire une suite d'entiers strictement décroissante et strictement positive, ce qui est impossible.

    Grâce au lemme \ref{LemJPIUooWFHaFM} nous avons successivement les affirmations suivantes :
    \begin{itemize}
        \item 
        $\frac{ a^2 }{ b^2 }=2$
    \item
        \( a^2=2b^2\), donc \( a^2\) est pair.
    \item
        \( a\) est alors pair et \( a^2\) est divisible en \( 4\). Soit \( a^2=4k\).
    \item
        \( 4k/b^2=2\), donc \( 4k=2b^2\), donc \( b^2=2k\) et \( b^2\) est pair.
    \item
        Nous déduisons que \( b\) est pair.
    \end{itemize}
    La fraction \( \frac{ a/2 }{ b/2 }\) est alors une nouvelle fraction d'entiers dont le carré vaut $2$. En procédant comme à nouveau, la fraction d'entiers \( \frac{ a/4 }{ b/4 }\) a la même propriété.

    En particulier, tous les nombres de la forme \( a/2^n\) sont des entiers.  Ils forment une suite strictement décroissante d'entiers positifs.  Impossible, me diriez vous ?!? Et vous auriez bien raison : il n'y a pas de fractions d'entiers dont le carré vaut \( 2\).
\end{proof}

\begin{proposition}
    Soit le suite de rationnels \( (x_n)\) définie par \( x_0\in \eQ^+\) et 
    \begin{equation}
        x_{n+1}=x_n+\frac{ x_n^2-2 }{ 2x_n }.
    \end{equation}
    Alors en posant \( y_n=x_n^2\) nous avons \( y_n\to 2\).
\end{proposition}

\begin{proof}
    Tout d'abord un petit calcul montre que
    \begin{equation}
        x_{n+1}^2=2+\frac{ (y_n-2)^2 }{ 4y_n },
    \end{equation}
    c'est à dire que quelle que soit la valeur de \( x_0\), dès le \( x_1\), les valeurs de \( y_n\) sont plus grandes que \( 2\). Notons qu'il n'est pas possible d'avoir \( y_n=2\). Nous pouvons donc supposer \( y_n>2\) pour tout \( n\). Alors en posant \( y_n=2+s\) nous avons
    \begin{equation}
        y_{n+1}=2+\frac{ s^2 }{ 8+4s }.
    \end{equation}
    Si \( s<1\) alors \( \frac{ s^2 }{ 8+4s }<s^2\) et en réalité le processus \( s\mapsto s^2/(8+4s)\) tend très vite vers zéro. Nous devons donc montrer qu'il existe un \( n\) tel que \( y_n=2+s\) avec \( s<1\).

    Montrons pour cela que si \( n<s<2n\) alors\footnote{Voir la remarque \ref{RemUZCAooWNogzI} pour comprendre d'où vient l'idée de cette majoration.}
    \begin{equation}\label{EqYNKQooUBfhgz}
        \frac{ s^2 }{ 8+4s }<n
    \end{equation}
    En effet si \( n<s<2n\) nous pouvons majorer le numérateur par \( 4n^2\) et minorer le dénominateur par \( 4n\).

    Prouvons à présent le résultat. Pour un \( n>1\) nous avons
    \begin{equation}
        y_{n+1}=2+\frac{ s }{ 8+4s }
    \end{equation}
    avec \( s>0\). Nous considérons \( n_0\in \eN\) tel que \( n_0<s<2n_0\). Alors
    \begin{equation}
        y_{n+2}=2+s_1
    \end{equation}
    avec \( s_1<n\). Il existe donc \( n_1\) tel que \( n_1<s_1<2n_1\) et \( n_1<n_0\). Nous avons alors \( y_{n+3}=2+s_2\) avec \( s_2<n_1<n_0\).

    Nous construisons ainsi des suites \( (s_i)\) et \( (n_i)\) telles que 
    \begin{equation}
        y_{n+k}=2+s_k
    \end{equation}
    avec \( s_k<n_{k-1}<n_{k-2}<\cdots<n_0\). En procédant ainsi au maximum \( n-1\) fois nous avons \( s_k<1\). À partir du moment où \( y_{n+k}=2+s_k\) avec \( s_k<1\), nous avons déjà vu qu'il est certain que \( y_n\to 2\).
\end{proof}

\begin{remark}\label{RemUZCAooWNogzI}
    Vu que nous n'avons pas encore défini les réels, ce qui suit n'est que informel. Ce qui a motivé la majoration \eqref{EqYNKQooUBfhgz} est la résolution de l'inéquation
    \begin{equation}
        \frac{ s^2 }{ 8+4s }<n
    \end{equation}
    qui donne les bornes \( 2n\pm\sqrt{n^2+2n}\) dont une est toujours négative, et l'autre plus grande que \( 2n\).
\end{remark}    

%+++++++++++++++++++++++++++++++++++++++++++++++++++++++++++++++++++++++++++++++++++++++++++++++++++++++++++++++++++++++++++ 
\section{Les réels}
%+++++++++++++++++++++++++++++++++++++++++++++++++++++++++++++++++++++++++++++++++++++++++++++++++++++++++++++++++++++++++++

Une construction des réels via les coupures de Dedekind est donnée dans \cite{PaulinTopGmVegN}.

\begin{normaltext}
La construction des réels va nécessiter un petit «\wikipedia{fr}{bootstrap}{bootstrap}» au niveau de la topologie. En effet la notion de suite de Cauchy est une notion topologique (définition \ref{DefZSnlbPc}) alors que la topologie métrique (celle entre autres de \( \eQ\)) ne sera définie que par le théorème \ref{ThoORdLYUu}. Nous allons donc définir \emph{ex nihilo} la notion de suite de Cauchy dans \( \eQ\), construire \( \eR\) comme ensemble de classes d'équivalence de suites de Cauchy dans \( \eQ\). Ce ne sera que plus tard, après avoir définit la topologie métrique que nous allons voir que \( \eR\) est complet. Et que nous verrons avec satisfaction que la notion de suite de Cauchy définie dans \( \eQ\) coïncide avec celle de sa topologie métrique. Il en va de même avec la notion de convergence de suite.
% position 11144-30436
\end{normaltext}

Un des buts de cette section est de prouver le résultat suivant :
\begin{proposition} \label{PropSLCUooUFgiSR}
    Quel que soit le réel \( r\), il existe une suite croissante de rationnels convergente vers \( r\).
\end{proposition}
\index{densité!de \( \eQ\) dans \( \eR\)}

%+++++++++++++++++++++++++++++++++++++++++++++++++++++++++++++++++++++++++++++++++++++++++++++++++++++++++++++++++++++++++++ 
\section{Les complexes}
%+++++++++++++++++++++++++++++++++++++++++++++++++++++++++++++++++++++++++++++++++++++++++++++++++++++++++++++++++++++++++++

 \subsection{Définitions}
 Un nombre complexe s'écrit sous la forme $z = a + b i$, où $a$ et $b$
 sont des nombres réels appelés (et notés) respectivement partie réelle
 ($a = \Re(z)$) et partie imaginaire ($b = \Im(z)$) de $z$. L'ensemble
 des nombres de cette forme s'appelle l'ensemble des nombres complexes
 ; cet ensemble porte une structure de corps et est noté $\eC$. Le
 nombre complexe $i = 0 + 1 i$ est un nombre imaginaire qui a la
 particularité que $i^2 = -1$.

 Deux nombres complexes $a + bi$ et $c + di$ sont égaux si et seulement
 si $a = c$ et $b = d$, c'est-à-dire leurs parties réelles sont égales,
 et leurs parties imaginaires sont égales.

 Un nombre complexe étant représenté par deux nombres, on peut le
 représenter dans un plan appelé « plan de Gauss ». La plupart des
 opérations sur les nombres complexes ont leur interprétation
 géométrique dans ce plan.

 Pour $z = a + bi$ un nombre complexe, on note $\bar z = a - bi$ le
 \Defn{complexe conjugué} de $z$. Dans le plan de Gauss, il s'agit du
 symétrique de $z$ par rapport à la droite réelle (généralement
 dessinée horizontalement).

 On définit le module du complexe $z$ par $\module z = \sqrt{z\bar z} =
 \sqrt{a^2 + b^2}$. Dans le plan de Gauss, il s'agit de la distance
 entre $0$ et $z$.

 \begin{proposition}
Pour tout $z = a+bi$ et $z^\prime$ nombres complexes, on a
   \begin{enumerate}
   \item $z \bar z = a^2 + b^2$;
   \item $\bar{\bar{z}} = z$;
   \item $\module z = \module {\bar z}$;
   \item $\module{zz^\prime} = \module z \module{z^\prime}$;
   \item $\module{z+z^\prime} \leq \module z + \module{z^\prime}$.
   \end{enumerate}
 \end{proposition}
