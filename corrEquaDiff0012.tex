% This is part of the Exercices et corrigés de mathématique générale.
% Copyright (C) 2009
%   Laurent Claessens
% See the file fdl-1.3.txt for copying conditions.
\begin{corrige}{EquaDiff0012}

	\begin{enumerate}

		\item
			La solution à l'équation sans second membre est $y_H(x)=Ae^x+B e^{4x}$. Nous devons maintenant trouver une solution particulière à l'équation avec un second membre. Comme le second membre est un polynôme, nous essayons un polynôme, soit $y_P(x)=ax+b$. Nous calculons
			\begin{equation}
				\begin{aligned}[]
					y(x)&=ax+b\\
					y'(x)&=a\\
					y''(x)&=0,
				\end{aligned}
			\end{equation}
			et nous remettons dans l'équation de départ :
			\begin{equation}
				0-5a+4(ax+b)=3-2x.
			\end{equation}
			Cela est une équation pour $a$ et $b$ dont la solution est $a=-1/2$ et $b=1/8$. La solution particulière que nous avons trouvée est
			\begin{equation}
				y_P(x)=-\frac{ x }{ 2 }+\frac{1}{ 8 },
			\end{equation}
			d'où nous déduisons que la solution générale à l'équation donnée est
			\begin{equation}
				y(x)=A e^{x}+B e^{4x}-\frac{ x }{ 2 }+\frac{1}{ 8 }.
			\end{equation}
			
		\item
			Le polynôme caractéristique de l'équation sans second membre est $\lambda^2-6\lambda+13=0$, et ses racines sont $3\pm 2i$. La solution \emph{réelle} générale de l'équation sans second membre est 
			\begin{equation}
				y_H(x)= e^{3x}\big( A\cos(2x)+B\sin(2x) \big).
			\end{equation}
			Reste à trouver une solution particulière. Étant donné que le second membre est une constante, nous essayons une constante. Il est facile de voir que la fonction constante $y_P(x)=3$ fonctionne.
		\item
			La solution générale de l'équation sans second membre est
			\begin{equation}
				y_H(x)=Ae^x+Bxe^x
			\end{equation}
			Le second membre étant une exponentielle, nous sommes incités à essayer $ae^{x}$ comme solution particulière. Hélas, cela ne fonctionne pas parce que cela est une solution de l'équation sans second membre. Alors nous devons trouver quelque chose de plus complexe. L'essai que nous faisons est
			\begin{equation}
				y_P(x)=(ax^2+bx+c) e^{x}.
			\end{equation}
			Étant donné que $ e^{x}$ et $x e^{x}$ sont des solutions de l'équation sans second membre, il est couru d'avance que les termes $(bx+c) e^{x}$ ne vont pas jouer. Nous pouvons donc déjà poser $b=c=0$. Nous avons donc
			\begin{equation}
				\begin{aligned}[]
					y_P(x)&=ax^2 e^{x}\\
					y'_P(x)&=2ax e^{x}+ax^2e^x\\
					y'_P(x)&=(2ax+2a+ax^2+2ax)e^x.
				\end{aligned}
			\end{equation}
			En remettant tout ça dans l'équation de départ, nous trouvons une équation pour $a$ que nous résolvons. La réponse est que une solution particulière est donnée par $y_P(x)=3x^2 e^{x}$.

		\item
			L'équation sans second membre est $y''+4y=0$, dont l'équation caractéristique est $\lambda^2+4=0$. Les solutions sont $\pm 2i$, et donc les solutions réelles sont
			\begin{equation}
				y_H(x)=A\cos(2x)+B\sin(2x).
			\end{equation}
			
			Il s'agit maintenant de trouver une solution particulière. Comme le second membre est $\cos(2x)$, nous voudrions essayer
			\begin{equation}
				y_P(x)=a\cos(2x)+b\sin(2x),
			\end{equation}
			mais cela ne va pas fonctionner parce que $\sin(2x)$ et $\cos(2x)$ sont déjà des solutions de l'équation sans second membre. Nous essayons alors
			\begin{equation}
				y_P(x)=ax\cos(2x)+bx\sin(2x).
			\end{equation}
			Afin de fixer les constantes $a$ et $b$, nous injectons cette fonction dans l'équation. Pour cela nous commençons par en calculer les dérivées :
			\begin{equation}
				y'_P(x)=\sin(2x)(-2ax+b)+\cos(2x)(2bx+a),
			\end{equation}
			et
			\begin{equation}
				y''_P(x)=\sin(2x)(-4bx+-4a)+\cos(2x)(-4ax+4b).
			\end{equation}
			L'équation devient donc
			\begin{equation}
				y''_P(x)+4y'_P(x)=-4(a\sin(2x)-b\cos(2x))=\cos(2x).
			\end{equation}
			Il faut donc choisir $a=0$ et $b=1/4$. La solution particulière que nous venons de construire est donc
			\begin{equation}
				y_P(x)=\frac{ x }{ 4 }\cos(2x),
			\end{equation}
			et la solution générale à l'équation qui nous intéresse est
			\begin{equation}
				y(x)=A\cos(2x)+B\sin(2x)+\frac{ x }{ 4 }\cos(2x).
			\end{equation}
		
	\end{enumerate}

\end{corrige}
