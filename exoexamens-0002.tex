% This is part of Exercices et corrections de MAT1151
% Copyright (C) 2010
%   Laurent Claessens
% See the file LICENCE.txt for copying conditions.

\begin{exercice}\label{exoexamens-0002}

    Soit la fonction 
    \begin{equation}
        f\colon (x,y,z)\in \Omega\to yz\ln(1-xyz)+2yz.
    \end{equation}
    \begin{enumerate}
        \item
            Quel est le domaine de définition de \( f\) ?

            Pour la suite on regardera la fonction \( f\) sur l'ensemble
            \begin{equation}
                \Omega-\{ (x,y,z)\in \eR^3\tq x>0,y>0,z>0 \}.
            \end{equation}
            
        \item
            Quel est le gradient de \( f\) au point \( (1,1,2)\) ?
        \item
            Quelle est la différentielle de \( f\) en ce point ?
    \end{enumerate}

\corrref{examens-0002}
\end{exercice}
