% This is part of Mes notes de mathématique
% Copyright (c) 2011-2014
%   Laurent Claessens
% See the file fdl-1.3.txt for copying conditions.

\thispagestyle{empty}
\begin{center}
  \begin{minipage}{15cm}
    \hrule\par
    \vspace{2mm}
    \begin{center}
    \Huge \bfseries  Pour les étudiants \par
    \end{center}
    \hrule\par
  \end{minipage}\\
\end{center}

\vspace{2cm}

\begin{center}
    Laurent \textsc{Claessens}\\
    \today
\end{center}

\vfill

\LogoEtLicence

\clearpage

\thispagestyle{empty}

Ce document regroupe les différents textes que j'ai tapé à l'université.
\begin{itemize}
    \item La partie «Outils mathématique» correspond au cours d'outils mathématiques que j'ai donné à l'université de Franche-Comté. Il contient la théorie des fonctions de deux ou trois variables ainsi que l'analyse vectorielle principalement destinée au étudiants du second semestre en physique et chimie.
    \item La partie «Matlab» contient des notes pour un cours d'introduction à matlab que j'ai donné à divers groupes d'étudiants allant de la première année en physique à la deuxième année en agronomie.

    \item La partie «Exercices» regroupe l'ensemble des exercices que j'ai donné. 

        \begin{itemize}
            \item Calcul différentiel et intégral pour les premières et seconde année en physique et mathématique (Bruxelles)
            \item Mathématique générale pour les géographes (Bruxelles)
            \item Mathématique générale pour SVT (Besançon)
            \item Mathématique générale pour la première année d'ingénieur en gestion (Louvain-la-Neuve, Belgique)
            \item Géométrie analytique pour la deuxième année en mathématique (Besançon)
            \item Analyse numérique pour la première année en physique (Louvain-la-Neuve, Belgique)
        \end{itemize}
        La plupart de ces exercices sont corrigés. 

        Hélas certaines corrections font référence à des exercices qui étaient distribués sur des feuilles ou à des cours dont je n'ai pas pu me procurer les sources ou que je n'ai pas eu le temps de retaper. Cela explique pourquoi de nombreuses références vers des équations ou des théorèmes sont manquantes.
\end{itemize}

\vfill

Pour avoir vraiment \emph{tout} ce que j'ai tapé en mathématique, y compris des parties de niveau recherche, il faut télécharger ceci :
\begin{center}
    \url{http://laurent.claessens-donadello.eu/pdf/mazhe.pdf}
\end{center}



\clearpage
