% This is part of Mes notes de mathématique
% Copyright (c) 2011-2012
%   Laurent Claessens
% See the file fdl-1.3.txt for copying conditions.

%+++++++++++++++++++++++++++++++++++++++++++++++++++++++++++++++++++++++++++++++++++++++++++++++++++++++++++++++++++++++++++
\section{Avertissement}
%+++++++++++++++++++++++++++++++++++++++++++++++++++++++++++++++++++++++++++++++++++++++++++++++++++++++++++++++++++++++++++

Ceci sont des notes «prises au vol». Aucune garantie. Merci de me signaler toute faute ou remarque. En particulier je serais content que quelqu'un me donne un avis sur le points suivants :
\begin{enumerate}
    \item
        Est-ce que la solidité de RSA tient au fait que résoudre Bezout
        \begin{equation}
            kp+lq=1
        \end{equation}
        lorsque \( p\) et \( q\) sont premiers est facile alors que c'est difficile lorsque \( p\) et \( q\) sont des nombres simplement premiers entre eux ? Voir la sous-section \ref{SecEVaFYi}.
    \item
        Est-ce que l'énoncé et la démonstration de la proposition \ref{PropyMTEbH} sont corrects ? Si \( a\) et \( b\) sont des racines de \( P\), alors \( \mu_a\mu_b\) divise \( P\) (si \( \mu_a\neq \mu_b\)). Cette proposition est utilisée dans la démonstration de l'irréductibilité des polynômes cyclotomiques (proposition \ref{PropoIeOVh}).
    \item
        Préciser l'énoncé et donner une démonstration de la proposition \ref{PropMpBStL} qui traite de sommes dénombrables.
    \item
        À quoi sert l'hypothèse «autre que \( \eF_2\)» dans le lemme \ref{LemcDOTzM} ?
    \item
        L'inversibilité de la somme de Gauss (proposition \ref{PropciRUov}) est-elle bien démontrée ?
    \item
        Trouver une preuve que l'anneau des polynômes est factoriel (proposition \ref{PropqGZXvr})
    \item
        Des commentaires sur l'exemple \ref{ExfUqQXQ} qui montre que \( X^p-X+1\) est irréductible sur \( \eF_p\).
    \item
        La justification que \( f_n\) est borélienne dans la proposition \ref{PropfqvLOl} mérite plus de détails.
    \item
        Une preuve du théorème \ref{ThoRWEoqY} qui donne la densité de \( C^{\infty}_c\) dans \( L^1\).
    \item
        L'énoncé et la démonstration de la proposition \ref{PropNsLqWb}.
    \item
        Où trouver une preuve de la proposition \ref{PropKZDqTR} sur le supplémentaire topologique ?
    \item
        La preuve du théorème \ref{ThoJsBKir}.
    \item
        La preuve du lemme \ref{LemjXywjH}.
    \item
        La preuve du théorème de Fredholm \ref{ThoagJPZJ}.
    \item
        La preuve du lemme \ref{LemooynkH}.
    \item
        La partie «unicité» du théorème \ref{ThokUUlgU}.
    \item
        La preuve de la proposition \ref{PropRZCKeO}.
    \item
        L'inversion entre la somme et l'intégrale de l'équation \eqref{EqXSgZGw}.
    \item
        Les idéaux de \( A/I\) sont en bijection avec les idéaux de \( A\) contenant \( I\). Justification de l'équation \eqref{EqKbrizu}.
    \item
        Si vous connaissez des exemples ou des contre-exemples intéressants à propos de quoi que ce soit, je suis toujours preneur.
\end{enumerate}

%+++++++++++++++++++++++++++++++++++++++++++++++++++++++++++++++++++++++++++++++++++++++++++++++++++++++++++++++++++++++++++
\section*{Originalité}
%+++++++++++++++++++++++++++++++++++++++++++++++++++++++++++++++++++++++++++++++++++++++++++++++++++++++++++++++++++++++++++

Ces notes ne sont pas originales par leur contenu : ce sont toutes des choses qu'on trouve facilement sur internet; je crois que la bibliographie est éloquente à ce sujet.

Ce cours se distingue des autres sur trois points.
\begin{enumerate}
    \item
        La longueur. J'ai décidé de ne pas me soucier de la taille du fichier. Il fera cinq mille pages si il le faut, mais il restera en un bloc. Étant donné qu'il n'existe qu'une seule mathématique, il ne m'a pas semblé intéressant de produire une division artificielle entre l'analyse, la géométrie ou l'algèbre. Tous le résultats d'une branche peuvent être utilisés dans toutes les autres branches.

    \item
        La licence. Ce document est publié sous une licence libre. Elle vous donne explicitement le droit de copier, modifier et redistribuer. Je me doute bien que la majorité des professeurs qui mettent leurs notes en ligne ne seront pas fâchés de voir qu'elles soient utilisées. Il n'empèche que par défaut la loi dit que l'auteur conserve tous ses droits. De plus vous ne savez pas si l'auteur accepte de voir le pdf sur votre site, de voir du copier-coller vers un autre document, \ldots bref un document laissé sans licence c'est moralement pas clair; et c'est légalement parfaitement clair : vous ne pouvez rien n'en faire. 

        Ici pas de problèmes : la licence vous dit tout ce que vous pouvez faire et pas faire, sous quelles conditions. En respectant les termes de la licence, vous avez à la fois la sécurité juridique et l'assurance morale de ne pas abuser.
    \item
        L'ISNB. Une fois par an, une version de ce document sera affublée d'un ISBN. Pourquoi ? Parce qu'en avoir un est le sésame qui permet d'entrer dans la bibliothèque de l'agrégation. J'espère que la version de septembre 2012 y sera. Si vous préparez votre agrégation pour passer en 2013, vérifiez.       

\end{enumerate}

%+++++++++++++++++++++++++++++++++++++++++++++++++++++++++++++++++++++++++++++++++++++++++++++++++++++++++++++++++++++++++++
					\section*{Ces notes sont les vôtres !}
%+++++++++++++++++++++++++++++++++++++++++++++++++++++++++++++++++++++++++++++++++++++++++++++++++++++++++++++++++++++++++


Il y a encore certainement des erreurs, des fautes de frappe et des choses pas claires. Je compte sur vous (oui : toi !) pour me signaler toute imperfection (y compris d'orthographe).

Plus vous signalez de fautes, meilleure sera la qualité du texte, et plus les étudiants de l'année prochaine vous seront reconnaissants.

%Tiens, tant que j'y suis je vous demanderais de penser à la quantité d'argent que vous auriez dû dépenser pour obtenir un texte tel que celui-ci chez un éditeur «classique» qui vous interdirait la photocopie et la redistribution. Maintenant que vous y avez pensé, je vous donnerais bien mon numéro de compte, mais non. J'ai tapé ce texte sur mes heures de travail à l'université; j'ai donc déjà été payé par les contribuables belges et français.

%+++++++++++++++++++++++++++++++++++++++++++++++++++++++++++++++++++++++++++++++++++++++++++++++++++++++++++++++++++++++++++
\section{Contre Moodle, Icampus, Claroline, et autres «plateformes de travail collaboratif»}
%+++++++++++++++++++++++++++++++++++++++++++++++++++++++++++++++++++++++++++++++++++++++++++++++++++++++++++++++++++++++++++

Ces notes ne sont pas destinées à être publiées sur des plateformes telles que Moodle, Icampus ou autres Clarolines. Pourquoi ? parce que la licence FDL l'interdit implicitement en demandant de publier sur des sites \emph{ouverts}.

L'internet est un système décentralisé et ouvert : tout le monde peut s'y connecter, y publier et y lire. C'est pour l'instant la meilleure solution technique inventée par l'humanité pour la diffusion d'information. Des sites comme \href{http://gitorious.org}{gitorious} ou \href{http://wikipedia.org}{wikipedia} sont de \emph{vrais} système de travail collaboratif.

Les plateformes soi-disant collaboratives comme Moodle en sont la négation. L'essentiel de ce qu'apporte Moodle par rapport à un vrai site internet n'est absolument pas la possibilité de partager des information (ça on peut le faire via internet depuis des décennies), mais bien de \emph{restreindre} l'accès à l'information via un système de mot de passe.

Lorsqu'un moine copiste du onzième siècle mettait un manuscrit dans sa bibliothèque, le document était immédiatement consultable par une centaine de moines, et (quitte à faire le déplacement) par des milliers d'érudits. Un document posté sur Moodle touche une dizaine de personnes. Utiliser Moodle pour partager ses documents est donc une régression non pas par rapport à l'internet d'il y a vingt ans, non pas par rapport à l'imprimerie d'il y a cinq siècles, mais bien par rapport aux bibliothèques d'abbayes d'il y a mille ans !

Lorsqu'on parle de science, qu'on veut y apporter un document, une question ou une réponse, un minimum d'honnête intellectuelle, d'éthique du partage de savoir (sans laquelle la science n'existe pas) et peut être aussi de courage, est de parler publiquement. Se cacher derrière un mot de passe et ne permettre l'accès au savoir qu'à ses seuls amis triés sur le volet est une négation de l'esprit scientifique; Moodle est une version dégénérée, une maladie de l'internet.

La faute fondamentale qui fait utiliser Moodle pour partager des documents de mathématique est la perte de notion entre le privé et le public ainsi que la paresse qui consiste à vouloir intégrer tous les outils dans une même interface, voire utiliser les mêmes outils pour effectuer des tâches différentes. Lorsqu'on pose une question de math, c'est essentiellement public; lorsqu'on pose une question d'organisation d'un cours, c'est privé. Ce sont deux activités totalement différentes qui nécessitent deux types d'outils différents. Dans le premier cas, l'outil adapté est internet, dans le deuxième cas, l'outil adapté est Moodle. Vouloir utiliser la même interface pour les deux est n'avoir fondamentalement pas compris le sens de l'internet et son utilité en tant que «outil de l'information».


%Une autre faute d'utilisation usuelle d'utilisation des plates-formes de «travail collaboratif» est d'oublier de mettre une licence libre sur les documents postés. En effet, le droit nous indique que si l'auteur ne précise rien il conserve tout ses droits. Les autres utilisateurs de Moodle n'ont donc légalement ni le droit de modifier ni le droit de redistribuer les textes postés sans licences particulières.

{\tiny Cela dit c'est pratique pour discuter des horaires des cours ou s'échanger des informations pratiques qui n'ont pas à être publiques.}

%+++++++++++++++++++++++++++++++++++++++++++++++++++++++++++++++++++++++++++++++++++++++++++++++++++++++++++++++++++++++++++
\section{Instructions pour les examens et interrogations}
%+++++++++++++++++++++++++++++++++++++++++++++++++++++++++++++++++++++++++++++++++++++++++++++++++++++++++++++++++++++++++++

Ceci sont des conseils généraux que nous vous conseillons de suivre dans toutes les matières.
\begin{description}
    \item[numéroter] Numérotez clairement toutes les questions. Si votre réponse prend plus d'une page, écrivez «suite au verso», «suite à l'intercalaire \( n\)» etc. À l'endroit où la réponse continue, écrivez «question \( n\), suite».

    \item[vérifiez] Certaines erreurs sont faciles à détecter. Par exemple
        \begin{enumerate}
            \item
                les aires et volumes sont positifs;
            \item
                une intégrale \emph{définie} qui contient «\( dx\)» ne peut pas contenir de \( x\) dans la réponse;

            \item
                en physique et en chimie, les unités doivent être cohérentes : si la réponse est une énergie, vous devez avoir des joules (\unit{\square\metre\kilo\gram\per\square\second}).

        \end{enumerate}
    \item[votre nom] Écrivez votre nom et votre numéro de carte d'étudiant.

    \item[les faciles d'abord] Lisez d'abord toutes les questions avant de répondre. Commencez par les questions faciles.

    \item[justifier] Justifiez vos réponses. N'hésitez pas à écrire des phrases complètes : sujet, verbe, complément. N'abusez pas des symboles dont vous ignorez la signification :
        \begin{enumerate}
            \item
                «\( \Leftrightarrow\)» signifie «si et seulement si», et non «la suite de mon calcul»;
            \item

                «\( \nexists\)» signifie «il n'existe pas», et non «n'existe pas» ou «n'est pas défini».
        \end{enumerate}

    \item[ne pas passer en force] Si vous savez que votre réponse est fausse, mais vous ne savez pas la corriger, écrivez sur votre feuille «cette réponse est fausse pour telle raison, mais je ne sais pas comment corriger». Ne comptez pas sur une inattention du correcteur. En science, affirmer un fait que vous savez être faux s'appelle de la falsification; c'est déontologiquement inacceptable. De la même façon, si vous copiez sur votre voisin\footnote{Indépendamment que c'est sans doute interdit par le règlement; vérifiez avant.}, vous êtes priés de le citer : on ne s'approprie pas le travail d'autrui.

    \item[approximations numériques] Lorsque vous voulez écrire une approximation numérique, réfléchissez au sens de ce que vous allez écrire. En mathématique, ça n'a presque jamais de sens d'écrire une approximation parce que vous ne savez pas dans quel contexte votre calcul pourra être utilisé. Si vous laissez deux décimales à \( \pi\) pour calculer le volume d'eau dans votre piscine gonflable, ça fera l'affaire; si c'est pour calculer la masse du Higgs ou pour mettre un satellite autour de Mars, vous perdez plusieurs millions d'euros.

        En sciences naturelles (physique, chimie ou autres), vous pouvez donner des approximations numériques de façon circonstanciée. Demandez à votre prof de labo.

    \item[orthographe] Sans être obligatoire, ça ne fait jamais de mal. Surtout si le français est votre langue maternelle.
    \item[santé] Mangez des fruits et des légumes de saisons. Choisissez des producteurs locaux qui n'utilisent pas d'engrais synthétisés à base de pétrole. De toutes façons \href{http://www.energybulletin.net/node/51306}{vous n'avez pas le choix}.

\end{description}

%+++++++++++++++++++++++++++++++++++++++++++++++++++++++++++++++++++++++++++++++++++++++++++++++++++++++++++++++++++++++++++
\section{Propagande : utilisez un ordinateur !}
%+++++++++++++++++++++++++++++++++++++++++++++++++++++++++++++++++++++++++++++++++++++++++++++++++++++++++++++++++++++++++++

Si vous faites des exercices supplémentaires et que vous voulez des corrections, n'oubliez pas que vous avez un ordinateur à disposition. De nos jours, les ordinateurs sont capables de calculer à peu près tout ce qui se trouve dans vos cours de math\footnote{Avec une notable exception pour les limites à deux variables.}.

D'ailleurs, je te rappelle que nous sommes est déjà largement dans le vingt et unième siècle et que tu te destines à une carrière professionnelle dans laquelle tu auras des calculs à faire; si tu n'es pas encore capable d'utiliser un ordinateur pour faire ces calculs, il est temps de combler cette lacune.

%---------------------------------------------------------------------------------------------------------------------------
\subsection{Lancez-vous dans Sage}
%---------------------------------------------------------------------------------------------------------------------------

Le logiciel que je vous propose est \href{http://www.sagemath.org}{Sage}. C'est depuis 2012 un logiciel disponible pour l'épreuve de modélisation de l'agrégation en mathématique. Pour l'utiliser, il n'est même pas nécessaire de l'installer sur votre ordinateur~: il tourne en ligne, directement dans votre navigateur.

\begin{enumerate}

	\item
		Aller sur \href{http://www.sagenb.org}{http://www.sagenb.org}
	\item
		Créer un compte
	\item
		Créer des feuilles de calcul et amusez-vous !!

\end{enumerate}

Il y a beaucoup de \href{http://lmgtfy.com/?q=sage+documentation}{documentation} sur le \href{http://www.sagemath.org}{site officiel}\footnote{\href{http://www.sagemath.org}{http://www.sagemath.org}}.

Si vous comptez utiliser régulièrement ce logiciel, je vous recommande \emph{chaudement} de \href{http://mirror.switch.ch/mirror/sagemath/index.html}{l'installer} sur votre ordinateur.

%---------------------------------------------------------------------------------------------------------------------------
\subsection{Exemples de ce que Sage peut faire pour vous}
%---------------------------------------------------------------------------------------------------------------------------

Voici une liste absolument pas exhaustive de ce que Sage peut faire pour vous, avec des exemples. 
\begin{enumerate}

	\item
		Calculer des limites de fonctions, voir l'exercice \ref{exoINGE11140028},

	\item
		D'autres limites et tracer des fonctions, voir l'exercice \ref{exoINGE11140031}.
	\item
		Calculer des dérivées, voir exercice \ref{exo0013}.
	\item
		Calculer des dérivées partielles de fonctions à plusieurs variables, voir exercice \ref{exoFoncDeuxVar0002}.
	\item
		Calculer des primitives, voir certains exercices \ref{exo0017}
	\item

		Résoudre des systèmes d'équations linéaires. Lire \href{http://www.sagemath.org/doc/constructions/linear_algebra.html#solving-systems-of-linear-equations}{la documentation} est ce qui fait la différence entre l'être humain et le non scientifique. Voir les exercices  \ref{exoINGE1121La0016} et \ref{exoINGE1121La0010}.

	\item
		Tout savoir d'une forme quadratique, voir exercice \ref{exoINGE1121La0018}.
	\item
		Calculer la matrice Hessienne de fonctions à deux variables, déterminer les points critiques, déterminer le genre de la matrice Hessienne aux points critiques et écrire extrema de la fonctions (sous réserve d'être capable de résoudre certaines équations), voir les exercices \ref{exoFoncDeuxVar0029} et \ref{exoFoncDeuxVar0028}.
	\item
		Lorsqu'il y a une infinité de solutions, Sage vous l'indique avec des paramètres, voir l'exercice \ref{exoDerrivePartielle-0007}. Pour les fonctions trigonométriques, 
        \begin{verbatim}
sage: solve(sin(x)/cos(x)==1,x,to_poly_solve=True)                                                         
[x == 1/4*pi + pi*z1]
sage: solve(sin(x)**2==cos(x)**2,x,to_poly_solve=True)
[sin(x) == cos(x), x == -1/4*pi + 2*pi*z86, x == 3/4*pi + 2*pi*z84]
        \end{verbatim}

	\item
		Calculer des dérivées symboliquement, voir exercice \ref{exoDerive-0002}.
	\item
		Calculer des approximations numériques comme dans l'exercice \ref{exoOutilsMath-0028}.
    \item
        Calculer dans un corps de polynômes modulo comme \( \eF_p[X]/P\) où \( P\) est un polynôme à coefficients dans \( \eF_p\). Voir l'exemple \ref{ExemWUdrcs}.
	\item
        Tracer des courbes en trois dimensions, voir exemple \ref{ExempleTroisDxxyy}. Notez que pour cela vous devez installer aussi le logiciel Jmol. Pour Ubuntu\footnote{Pour les autres, je ne sais pas, mais je laisserai jouer l' adage «Windows c'est facile». Quant aux utilisateurs d'OS «alternatifs» comme Hurd ou BSD ben heu \ldots} c'est dans le paquet \info{icedtea6-plugin}.
\end{enumerate}

Sage peut toutefois vous induire en erreur si vous n'y prenez pas garde parce qu'il sait des choses en mathématique que vous ne savez pas. Par conséquent il peut parfois vous donner des réponses (mathématiquement exactes) auxquelles vous ne vous attendez pas. Voir page \ref{PgpXBuBh}.

%+++++++++++++++++++++++++++++++++++++++++++++++++++++++++++++++++++++++++++++++++++++++++++++++++++++++++++++++++++++++++++
%\section{Propagande : n'utilisez pas votre calculatrice}
%+++++++++++++++++++++++++++++++++++++++++++++++++++++++++++++++++++++++++++++++++++++++++++++++++++++++++++++++++++++++++++

%D'abord, l'expérience montre que la majorité des fois qu'un étudiant sort sa calculatrice, c'est pour faire un calcul inutile, et le plus souvent la calculatrice fournit un résultat faux. Étant en 2012, vous ne devriez pas vous contenter de vos calculatrices qui coûtent un os, qui n'ont pas de puissance de calcul, qui ont une définition d'écran ridicule et en noir et blanc. Remarquez que votre GSM (et a forciori vos minis trucs qui se connectent a internet) sont considérablement plus puissants que ces vieilleries; ils ont un meilleur écran.
