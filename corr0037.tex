% This is part of Exercices et corrigés de CdI-1
% Copyright (c) 2011
%   Laurent Claessens
% See the file fdl-1.3.txt for copying conditions.

\begin{corrige}{0037}

Pour un même réel $a$, on peut trouver une suite $(q_i)_{i\in\eN}$ de rationnels et une suite $(r_i)_{i\in\eN}$ d'irrationnels qui tendent toutes deux vers $a$. Mais alors $f(q_i) = 1$ et $f(r_i) = 0$ pour tout $i$. Donc $f(q_i) \to 1$ et $f(r_i) \to 0$, ce qui montre que la limite
\begin{equation*}
	\lim_{x\to a} f(x)
\end{equation*}
n'existe pas.

Par contre, restreinte à $\QQ$ comme proposé, la fonction devient
constante... il devient alors clair que, pour tout $a \in \QQ$,
\begin{equation*}
  \forall \epsilon, \exists \delta : \forall x \in \QQ : \abs{x-a} <
  \delta \Rightarrow \abs{f(x) - f(a)} < \epsilon
\end{equation*}
puisque $\abs{f(x) - f(a)} = 0$ (donc la condition est satisfaite pour
n'importe quel $\delta$).


\end{corrige}
