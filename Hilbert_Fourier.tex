% This is part of Mes notes de mathématique
% Copyright (c) 2011-2012
%   Laurent Claessens
% See the file fdl-1.3.txt for copying conditions.

%+++++++++++++++++++++++++++++++++++++++++++++++++++++++++++++++++++++++++++++++++++++++++++++++++++++++++++++++++++++++++++
\section{Série de Fourier}
%+++++++++++++++++++++++++++++++++++++++++++++++++++++++++++++++++++++++++++++++++++++++++++++++++++++++++++++++++++++++++++

Source : \cite{MaureyHilbertFourier}

Nous utilisons ici des résultats de bases hilbertiennes de la sous-section \ref{SubsecDxkjut}.

Nous considérons la base trigonométrique de \( L^2\mathopen[ 0 , 2\pi \mathclose]\) donnée par
\begin{equation}
    e_k(x)= e^{2i\pi k x},
\end{equation}
et pour une fonction donnée \( f\in L^2\), nous définissons\nomenclature[Y]{\( S_nf\)}{somme partielle de série de Fourier} 
\begin{equation}
    S_nf=\sum_{k=-n}^n\langle f, e_k\rangle e_k.
\end{equation}
Nous avons alors la convergence
\begin{equation}
    S_nf\to f
\end{equation}
au sens \( L^2\). Si nous voulons une vraie convergence ponctuelle voir uniforme \( (S_nf)(x)\to f(x)\), alors il faut ajouter des hypothèses sur la continuité de \( f\) et le comportement des coefficients
\begin{equation}
    c_n=\langle f, e_n\rangle .
\end{equation}
Cela sera l'objet de la proposition \ref{PropSgvPab} et du théorème \ref{ThozHXraQ}.



\begin{proposition}     \label{PropSgvPab}
    Soit \( f\) une fonction \( 2\pi\)-périodique. Si \( \sum_{n\in \eZ}| c_n(f) |<\infty\), alors pour tout \( x\in \eR\) nous avons
    \begin{equation}
        f(x)=\sum_{n\in \eZ}c_n(f) e^{inx}.
    \end{equation}
    De plus, la suite \( (S_nf)\) converge uniformément vers \( f\).
\end{proposition}

\begin{proof}
    Nous posons 
    \begin{equation}
        g(x)=\sum_{n\in \eZ}c_n(f) e^{inx}.
    \end{equation}
    Étant donné les hypothèses, la série de droite converge absolument, la fonction \( g\) est continue sur \( \eR\). Nous avons
    \begin{equation}
        \big| g(x)-(S_nf)(x) \big|\leq \sum_{| k |> n}| c_k(f) |,
    \end{equation}
    mais le terme de droite tend vers zéro lorsque \( n\to \infty\) parce que c'est le reste d'une série convergente. Cela signifie que \( S_nf\) converge uniformément vers \( g\).

    Par ailleurs nous savons que dans \( L^2\) nous avons la convergence \( S_nf\to f\), ce qui signifie que \( g=f\) presque partout au sens \( L^2\). Ces deux fonctions étant continues, elles sont égales partout.
\end{proof}

\begin{theorem}     \label{ThozHXraQ}
    Soit \( f\), une fonction \( C^1\) et \( 2\pi\)-périodique. Nous notons \( (c_n)_{n\in \eZ}\) la suite de ses coefficients de Fourier. Alors \( (c_n)\in \ell^1(\eZ)\) et pour tout \( x\in \eR\) nous avons
    \begin{equation}
        f(x)=\sum_{n\in \eZ}c_n(f) e^{inx}.
    \end{equation}
\end{theorem}

\begin{proof}
    Soit \( n\in \eZ\). Nous posons \( g(t)=f(t) e^{-int}\). Nous avons
    \begin{equation}
        0=g(2\pi)-g(0)=\int_0^{2\pi}g'(t)dt=\int_0^{2\pi}\big[ f'(t) e^{-int}-inf(t) e^{-int} \big].
    \end{equation}
    Du coup, \( c_n(f')=inc_n(f)\). La fonction \( f'\) étant bornée (parce que continue sur \( \mathopen[ 0 , 2\pi \mathclose]\)), elle est de carré intégrable sur \( \mathopen[ 0 , 2\pi \mathclose]\) et par les inégalités de Parseval (théorème \ref{ThoyAjoqP}) nous avons
    \begin{equation}
        \sum_{n\in \eZ}| c_n(f') |^2<\infty.
    \end{equation}
    Par conséquent \( (c_n)\in \ell^2(\eZ)\) et a forciori \( (c_n)_{n\in \eN}\in \ell^2(\eN)\). L'inégalité de Cauchy-Schwartz nous indique alors
    \begin{equation}
        \sum_{n\in \eN}| c_n(f) |=\sum_{n\in \eN}\frac{1}{ n }| c_n(f') |\leq \left( \sum_n\frac{1}{ n^2 } \right)^{1/2}\left( \sum_{n}| c_n(f') |^2 \right)^{1/2}<\infty.
    \end{equation}
    Nous procédons de même pour \( n<0\). Cela prouve que 
    \begin{equation}
        \sum_{n\in \eZ}| c_n(f) |<\infty.
    \end{equation}
\end{proof}

%+++++++++++++++++++++++++++++++++++++++++++++++++++++++++++++++++++++++++++++++++++++++++++++++++++++++++++++++++++++++++++
\section{Transformée de Fourier}
%+++++++++++++++++++++++++++++++++++++++++++++++++++++++++++++++++++++++++++++++++++++++++++++++++++++++++++++++++++++++++++

Ici nous utilisons la convention de la transformée de Fourier de \wikipedia{fr}{Transformée_de_Fourier}{wikipedia}, c'est à dire
\begin{subequations}
    \begin{align}
        \hat f(\xi)&=\int_{\eR} e^{-i\xi x}f(x)dx\\
        f(x)&=2\pi\int_{\eR} e^{i\xi x}\hat f(\xi)d\xi.
    \end{align}
\end{subequations}

L'\defe{espace de Schwartz}{Schwartz!espace}\index{espace!de Schwartz} \( \swS(\eR^n,\eC)\)\nomenclature[Y]{\( \swS(\eR^n,\eC)\)}{fonctions Schwartz} est l'ensemble des fonctions dont toutes les dérivées décroissent plus vite que l'inverse de tout polynôme, c'est à dire
\begin{equation}
    \swS(\eR^n,\eC)=\{ f\in C^{\infty}(\eR^n,\eC)\tq \forall \alpha,\beta\in \eN^n,\sup_{x\in \eR^n}\big| (x)^{\alpha}D^{\beta}f(x) \big|<\infty \}
\end{equation}
où nous utilisons les notations \( x^{\alpha}=(x_1)^{\alpha_1}\ldots (x_n)^{\alpha_n}\) et \( D^{\beta}=\frac{ \partial^n  }{ \partial \beta_1\ldots\partial \beta_n }\).

\begin{proposition}[\cite{MesIntProbb}]
    La transformée de Fourier est une bijection de \( \swS(\eR^n,\eC)\).    
\end{proposition}

\begin{theorem}[\cite{MesIntProbb}]      \label{ThoRWEoqY}
    Soit \( \mu\) une mesure sur les boréliens de \( \eR^n\) finie sur les compacts. Alors \( C^{\infty}_c(\eR^n,\eR)\) est dense dans \( L^1(\eR^n,\Borelien(\eR^n),\mu)\).
\end{theorem}

\begin{proposition}     \label{PropfqvLOl}
    La transformée de Fourier est un morphisme vis-à-vis de la convolution\index{produit!convolution!et Fourier} sur \( L^1(\eR^n)\) :
    \begin{equation}
        \widehat{f*g}=\hat f\hat g.
    \end{equation}
\end{proposition}

\begin{proof}
    Nous devons étudier l'intégrale
    \begin{equation}
        \widehat{f*g}(\xi)=\int_{\eR}\left[ \int_{\eR} f(y)g(t-y)\right] e^{-it\xi} dt.
    \end{equation}
    Ici nous avons choisit des représentants \( f\) et \( g\) dans les classes de \( L^1\). Montrons que \( f\) est borélienne. D'abord \( f(x)=f_+(x)-f_-(x)\) où \( f_+\) et \( f_-\) sont des fonctions positives. Affin d'alléger les notations nous supposons un instant que \( f\) est positive et nous posons
    \begin{equation}
        f_n(x)=\sum_{k=1}^{2^n} \frac{ k }{ n }\mtu_{f(x)\in\mathopen[ \frac{ k }{ n } , \frac{ k+1 }{ n } [}.
    \end{equation}
    Le fait que \( f\) soit dans \( L^1\) implique que chacune des fonctions \( f_n\) est borélienne et donc que \( f\) l'est aussi en tant que limite ponctuelle de fonctions boréliennes\footnote{Le fait que \( f\) soit borélienne est une conséquence du théorème \ref{ThoRWEoqY}.}.
    
    Nous allons appliquer le théorème de Fubini \ref{ThoTKZKwP} à la fonction
    \begin{equation}
        \phi(x,y)=f(x)g(y) e^{-i\xi(x+y)}
    \end{equation}
    qui est borélienne en tant que produit et composé de fonctions boréliennes. Nous avons
    \begin{subequations}
        \begin{align}
            \int_{\eR}\left( \int_{\eR}| f(x) e^{-i\xi x} | |g(y) e^{-i\xi y} |dy \right)dx&=\int_{\eR}\left( | f(x) |\int_{\eR}| g(y) |dy \right)dx\\
            &=\int_{\eR}| f(x) |\| g \|_1\\
            &=\| f \|_1\| g \|_1<\infty.
        \end{align}
    \end{subequations}
    Le théorème est donc applicable. D'abord nous avons :
    \begin{subequations}
        \begin{align}
            \hat f(\xi)\hat g(\xi)&=\left(\int_{\eR}f(x) e^{-i\xi x}dx\right)\left(\int_{\eR}g(y) e^{-i\xi y}dy\right)\\
            &=\int_{\eR}\left( \int_{\eR}f(x)g(y) e^{-i\xi(x+y)}dy \right)dx\\
            &=\int_{\eR}\left( \int_{\eR}f(x)g(t-x) e^{-i\xi t} \right)dx.
        \end{align}
    \end{subequations}
    Jusqu'ici nous n'avons pas utilisé Fubini. Nous avons seulement introduit le nombre \( \int_{\eR}g(y) e^{-i\xi y}dy\) dans l'intégrale par rapport à \( x\) et effectué le changement de variables \( y\mapsto t=x+y\). Maintenant nous appliquons le théorème de Fubini pour inverser l'ordre des intégrales :
    \begin{subequations}
        \begin{align}
            \hat f(\xi)\hat g(\xi)&=\int_{\eR}\left( \int_{\eR}f(x)g(t-x) e^{-it\xi}dx \right)dy\\
            &=\int_{\eR} e^{-it\xi}\left( \int_{\eR}f(x)g(t-x)dx \right)dt\\
            &=\int_{\eR} e^{-it\xi}(f*g)(t)dt\\
            &=\widehat{f*g}(\xi).
        \end{align}
    \end{subequations}
\end{proof}

\begin{proposition}       \label{PropJvNfj}
    Soit une fonction \( f\in L^1(\eR^d)\). Alors sa transformée de Fourier est continue\index{transformée!Fourier!continuité}.
\end{proposition}

\begin{proof}
    Nous considérons une fonction \( f\) définie sur \( \eR^d\) et à valeurs dans \( \eR\) ou \( \eC\). Sa transformée de Fourier est donnée par
    \begin{equation}
        \hat f(\xi)=\int_{\eR^d} e^{-i\xi x}f(x)dx.
    \end{equation}
    Pour montrer que cette fonction \( \hat f\) est continue en \( \xi_0\) nous considérons une suite \( (\xi_n)\to \xi_0\) et nous voulons montrer que \( \hat f(\xi_n)\to\hat f(\xi_0)\). Pour cela nous considérons les fonctions
\begin{equation}
    g_n(x)= e^{-i\xi_nx}f(x)
\end{equation}
qui convergent simplement vers \( g(x)= e^{-i\xi x}f(x)\). Étant donné que
\begin{equation}
    | g_n(x) |<| f(x) |,
\end{equation}
le théorème de la convergence dominée donne alors
\begin{equation}
    \lim_{n\to \infty} \int g_n(x)=\int\lim_{n\to \infty } g_n(x),
\end{equation}
c'est à dire \( \lim_{n\to \infty} \hat f(\xi_n)=\hat f(\xi)\). La fonction \( \hat f\) est donc continue.
\end{proof}

\begin{lemma}
    Soit \( f\in L^1(\eR)\). Alors \( \| \hat f \|_{\infty}\leq \| f \|_1\).
\end{lemma}

\begin{proof}
    Cela est un simple calcul : étant donné que
    \begin{equation}
        \hat f(\xi)=\int_{\eR}f(x) e^{-ix\xi}dx,
    \end{equation}
    nous avons, pour tout \( \xi\),
    \begin{equation}
        | \hat f(\xi) |\leq\int_{\eR}| f(x) |dx,
    \end{equation}
    ce qui signifie exactement \( \| \hat f \|_{\infty}\leq \| f \|_1\).
\end{proof}

\begin{lemma}[Lemme de Riemann-Lebesgue\cite{MaureyHilbertFourier}]     \label{LesmRLaxXkQV}
    Si \( f\) est une fonction \( L^1(\eR)\) alors \( \lim_{\xi\to\pm\infty} \hat f(\xi)=0\).
\end{lemma}

\begin{proof}
    Nous commençons par prouver le résultat dans le cas d'une fonction \( g\) en escalier, et plus précisément par une fonction caractéristique d'un compact \( K=\mathopen[ a , b \mathclose]\). Au niveau de la transformée de Fourier nous avons
    \begin{equation}
        \hat\mtu_{K}(\xi)=\int_a^b e^{-i\xi x}dx=-\frac{1}{ i\xi }( e^{-ib\xi}- e^{-ia\xi}).
    \end{equation}
    Par conséquent
    \begin{equation}
        | \hat\mtu_K(\xi) |\leq \frac{ 2 }{ | \xi | }.
    \end{equation}
    Plus généralement si \( g=\sum_{i=1}^Nc_i\mtu_{K_i}\), alors
    \begin{equation}
        | \hat g(\xi) |\leq \frac{ 2 }{ | \xi | }\sum_{i=1}^N| c_i |,
    \end{equation}
    et donc nous avons effectivement \( \lim_{\xi\to\pm\infty}| \hat g(\xi) |=0\).

    Nous passons maintenant au cas général \( f\in L^1(\eR)\). Étant donné que les fonctions \( L^1\) en escalier sont denses dans \( L^1\), nous considérons une fonction \( g\in L^1(\eR)\) en escalier telle que \( \| f-g \|_1<\epsilon\). Nous avons donc
    \begin{equation}
        \| \hat f-\hat g \|_{\infty}\leq \| f-g \|_1<\epsilon.
    \end{equation}
    Donc
    \begin{equation}
        \| \hat f(\xi) \|\leq \| \hat f(\xi)-\hat g(\xi) \|_| \hat g(\xi) |.
    \end{equation}
    Le premier terme est plus petit que \( \epsilon\). Il nous reste à voir que 
    \begin{equation}
        \lim_{\xi\to \infty} | \hat g(\xi) |=0,
    \end{equation}
    mais cela est le résultat de la première partie de la preuve.    
\end{proof}

\begin{corollary}
    La transformée de Fourier d'une fonction \( L^1(\eR)\) est bornée.
\end{corollary}

\begin{proof}
    Par le corollaire \ref{PropJvNfj}, la transformée de Fourier d'une fonction \( L^1\) est continue. Le lemme de Riemann-Lebesgue \ref{LesmRLaxXkQV} impliquant qu'elle tend vers zéro en \( \pm\infty\), elle doit être bornée.    
\end{proof}

