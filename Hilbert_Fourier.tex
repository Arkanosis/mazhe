% This is part of Mes notes de mathématique
% Copyright (c) 2011-2012
%   Laurent Claessens
% See the file fdl-1.3.txt for copying conditions.

%+++++++++++++++++++++++++++++++++++++++++++++++++++++++++++++++++++++++++++++++++++++++++++++++++++++++++++++++++++++++++++
\section{Fourier}
%+++++++++++++++++++++++++++++++++++++++++++++++++++++++++++++++++++++++++++++++++++++++++++++++++++++++++++++++++++++++++++

Ici nous utilisons la convention de la transformée de Fourier de \wikipedia{fr}{Transformée_de_Fourier}{wikipedia}, c'est à dire
\begin{subequations}
    \begin{align}
        \hat f(\xi)&=\int_{\eR} e^{-i\xi x}f(x)dx\\
        f(x)&=2\pi\int_{\eR} e^{i\xi x}\hat f(\xi)d\xi.
    \end{align}
\end{subequations}

L'\defe{espace de Schwartz}{Schwartz!espace}\index{espace!de Schwartz} \( \swS(\eR^n,\eC)\)\nomenclature[Y]{\( \swS(\eR^n,\eC)\)}{fonctions Schwartz} est l'ensemble des fonctions dont toutes les dérivées décroissent plus vite que l'inverse de tout polynôme, c'est à dire
\begin{equation}
    \swS(\eR^n,\eC)=\{ f\in C^{\infty}(\eR^n,\eC)\tq \forall \alpha,\beta\in \eN^n,\sup_{x\in \eR^n}\big| (x)^{\alpha}D^{\beta}f(x) \big|<\infty \}
\end{equation}
où nous utilisons les notations \( x^{\alpha}=(x_1)^{\alpha_1}\ldots (x_n)^{\alpha_n}\) et \( D^{\beta}=\frac{ \partial^n  }{ \partial \beta_1\ldots\partial \beta_n }\).

\begin{proposition}[\cite{MesIntProbb}]
    La transformée de Fourier est une bijection de \( \swS(\eR^n,\eC)\).    
\end{proposition}

\begin{theorem}[\cite{MesIntProbb}]      \label{ThoRWEoqY}
    Soit \( \mu\) une mesure sur les boréliens de \( \eR^n\) finie sur les compacts. Alors \( C^{\infty}_c(\eR^n,\eR)\) est dense dans \( L^1(\eR^n,\Borelien(\eR^n),\mu)\).
\end{theorem}

\begin{proposition}     \label{PropfqvLOl}
    La transformée de Fourier est un morphisme vis-à-vis de la convolution\index{produit!convolution!et Fourier} sur \( L^1(\eR^n)\) :
    \begin{equation}
        \widehat{f*g}=\hat f\hat g.
    \end{equation}
\end{proposition}

\begin{proof}
    Nous devons étudier l'intégrale
    \begin{equation}
        \widehat{f*g}(\xi)=\int_{\eR}\left[ \int_{\eR} f(y)g(t-y)\right] e^{-it\xi} dt.
    \end{equation}
    Ici nous avons choisit des représentants \( f\) et \( g\) dans les classes de \( L^1\). Montrons que \( f\) est borélienne. D'abord \( f(x)=f_+(x)-f_-(x)\) où \( f_+\) et \( f_-\) sont des fonctions positives. Affin d'alléger les notations nous supposons un instant que \( f\) est positive et nous posons
    \begin{equation}
        f_n(x)=\sum_{k=1}^{2^n} \frac{ k }{ n }\mtu_{f(x)\in\mathopen[ \frac{ k }{ n } , \frac{ k+1 }{ n } [}.
    \end{equation}
    Le fait que \( f\) soit dans \( L^1\) implique que chacune des fonctions \( f_n\) est borélienne et donc que \( f\) l'est aussi en tant que limite ponctuelle de fonctions boréliennes\footnote{Le fait que \( f\) soit borélienne est une conséquence du théorème \ref{ThoRWEoqY}.}.
    
    Nous allons appliquer le théorème de Fubini \ref{ThoTKZKwP} à la fonction
    \begin{equation}
        \phi(x,y)=f(x)g(y) e^{-i\xi(x+y)}
    \end{equation}
    qui est borélienne en tant que produit et composé de fonctions boréliennes. Nous avons
    \begin{subequations}
        \begin{align}
            \int_{\eR}\left( \int_{\eR}| f(x) e^{-i\xi x} | |g(y) e^{-i\xi y} |dy \right)dx&=\int_{\eR}\left( | f(x) |\int_{\eR}| g(y) |dy \right)dx\\
            &=\int_{\eR}| f(x) |\| g \|_1\\
            &=\| f \|_1\| g \|_1<\infty.
        \end{align}
    \end{subequations}
    Le théorème est donc applicable. D'abord nous avons :
    \begin{subequations}
        \begin{align}
            \hat f(\xi)\hat g(\xi)&=\left(\int_{\eR}f(x) e^{-i\xi x}dx\right)\left(\int_{\eR}g(y) e^{-i\xi y}dy\right)\\
            &=\int_{\eR}\left( \int_{\eR}f(x)g(y) e^{-i\xi(x+y)}dy \right)dx\\
            &=\int_{\eR}\left( \int_{\eR}f(x)g(t-x) e^{-i\xi t} \right)dx.
        \end{align}
    \end{subequations}
    Jusqu'ici nous n'avons pas utilisé Fubini. Nous avons seulement introduit le nombre \( \int_{\eR}g(y) e^{-i\xi y}dy\) dans l'intégrale par rapport à \( x\) et effectué le changement de variables \( y\mapsto t=x+y\). Maintenant nous appliquons le théorème de Fubini pour inverser l'ordre des intégrales :
    \begin{subequations}
        \begin{align}
            \hat f(\xi)\hat g(\xi)&=\int_{\eR}\left( \int_{\eR}f(x)g(t-x) e^{-it\xi}dx \right)dy\\
            &=\int_{\eR} e^{-it\xi}\left( \int_{\eR}f(x)g(t-x)dx \right)dt\\
            &=\int_{\eR} e^{-it\xi}(f*g)(t)dt\\
            &=\widehat{f*g}(\xi).
        \end{align}
    \end{subequations}
\end{proof}

\begin{lemma}[Lemme de Riemann-Lebesgue\cite{MaureyHilbertFourier}]
    Si \( f\) est une fonction \( L^1(\eR)\) alors \( \lim_{\xi\to\pm\infty} \hat f(\xi)=0\).
\end{lemma}
<++>
