% This is part of Exercices de mathématique pour SVT
% Copyright (c) 2010
%   Laurent Claessens et Carlotta Donadello
% See the file fdl-1.3.txt for copying conditions.

\begin{corrige}{TD6b-0001}

    En injectant la forme de \( y\) proposée dans l'équation, le membre de gauche devient
    \begin{equation}
        y'-ay=4x^3 e^{2x}+2x^4 e^{2x}-ax^4 e^{2x},
    \end{equation}
    tandis que le membre de droite devient \( bx^3 e^{2t}\).

    Nous devons donc voir pour quelles valeurs de \( a\) et \( b\) ces deux grandeurs sont égales pour tout \( x\). En simplifiant par \(  e^{2x}\) (qui ne s'annule jamais),
    \begin{equation}
        4x^3+2x^4-ax^4=bx^3 e^{2x},
    \end{equation}
    et donc \( a=2\), \( b=4\).

\end{corrige}
