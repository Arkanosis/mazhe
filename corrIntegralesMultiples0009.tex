\begin{corrige}{IntegralesMultiples0009}

	En coordonnées polaires, le domaine d'intégration est $r^2<\sin(\theta)\cos(\theta)$. L'angle $\theta$ ne peut donc prendre que les valeurs telles que $\sin(\theta)\cos(\theta)\geq 0$, c'est à dire de $0$ à $\pi/2$ et puis de $\pi$ à $3\pi/2$.

    En ce qui concerne la surface, nous avons
    \begin{equation}
        S=2\int_0^{\pi/2}\int_0^{\sqrt{\sin(\theta)\cos(\theta)}}r\,drd\theta=2\int_0^{\pi/2}\frac{ 1 }{2}\sin(\theta)\cos(\theta)d\theta=\frac{ 1 }{2}.
    \end{equation}
    
	En ce qui concerne l'intégrale, la fonction que nous devons intégrer étant la même en $(x,y)$ qu'en $(-x,-y)$, nous pouvons simplement intégrer une des deux parties, et multiplier par deux :
	\begin{equation}
        I=2\int_0^{\pi/2}\int_0^{\sqrt{\sin(\theta)\cos(\theta)}}r^2\sqrt{\cos(\theta)\sin(\theta)}\,drd\theta.
	\end{equation}
    Ne pas oublier qu'il y a un $r$ qui provient du jacobien des coordonnées polaires. Nous avons
    \begin{equation}
        \begin{aligned}[]
            I&=2\int_0^{\pi}\sqrt{\sin(\theta)\cos(\theta)}\left[ \frac{ r^3 }{ 3 } \right]_0^{\sqrt{\cos(\theta)\sin(\theta)}}\\
            &=\frac{ 2 }{ 3 }\int_0^{\pi}\cos^2(\theta)\sin^2(\theta)d\theta.
        \end{aligned}
    \end{equation}
    Cette dernière intégrale se fait comme indiqué par la formule \eqref{EqTrucIntcossqsinsq}. Nous trouvons
    \begin{equation}
        I=\frac{ 2 }{ 3 }\left[ \frac{ \theta }{ 8 }-\frac{ 4\theta }{ 32 } \right]_0^{\pi}=\frac{ \pi }{ 12 }.
    \end{equation}

\end{corrige}
