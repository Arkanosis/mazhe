% This is part of Mes notes de mathématique
% Copyright (c) 2011-2013,2015-2016
%   Laurent Claessens
% See the file fdl-1.3.txt for copying conditions.


Ici nous utilisons la convention de la transformée de Fourier de \wikipedia{fr}{Transformée_de_Fourier}{wikipedia}, c'est à dire
\begin{subequations}
    \begin{align}
        \hat f(\xi)&=\int_{\eR} e^{-i\xi x}f(x)dx  \label{EQooWWAYooYvVlrF}\\
        f(x)&=2\pi\int_{\eR} e^{i\xi x}\hat f(\xi)d\xi.
    \end{align}
\end{subequations}
Nous allons par ailleurs utiliser indifféremment les notations \( \TF(f)\) ou \( \hat f\) pour la transformée de Fourier de \( f\). La notation \( \TF\) est pratique pour les transformées de loooooongues expressions ainsi que pour parler de l'application «transformée de Fourier» d'un espace de fonction vers un autre.

\begin{normaltext}
    Nous verrons dans le théorème \ref{THOooJLCDooAjTvJf} que la Transformée de Fourier n'est pas une isométrie de \( L^2\). Pour avoir une isométrie, il aurait fallu choisir des coefficients moins simples dans \eqref{EQooWWAYooYvVlrF}.
\end{normaltext}

%+++++++++++++++++++++++++++++++++++++++++++++++++++++++++++++++++++++++++++++++++++++++++++++++++++++++++++++++++++++++++++ 
\section{Transformée de Fourier dans \( L^1(\eR^d)\)}
%+++++++++++++++++++++++++++++++++++++++++++++++++++++++++++++++++++++++++++++++++++++++++++++++++++++++++++++++++++++++++++

\begin{definition}      \label{DEFooJAIUooFbaRkR}
    Si \( f\in L^1(\eR^d)\) alors nous définissons sa \defe{transformée de Fourier}{transformée!de Fourier} est la fonction donnée par
    \begin{equation}
        \TF(f)(\xi)=\hat f(\xi)=\int_{\eR} e^{-i\xi x}f(x)dx
    \end{equation}
\end{definition}

\begin{lemma}       \label{LEMooKGDKooVXSMCn}
    Si \( f\in L^1(\eR^d)\) et si \( g(x)=f(\lambda x)\) alors
    \begin{equation}
        \hat g(\xi)=\lambda^{-d}\hat f(\xi/\lambda).
    \end{equation}
\end{lemma}

\begin{proof}
    Il s'agit de faire le changement de variable \( y=\lambda x\) dans l'intégrale
    \begin{equation}
        \hat g(\xi)=\int_{\eR^d}f(\lambda x) e^{-i\xi x}dx.
    \end{equation}
    Dans le changement de variables, vient le coefficient \( dx=\lambda^{-d}dy\).
\end{proof}

\begin{proposition}     \label{PropfqvLOl}
    La transformée de Fourier est un morphisme vis-à-vis de la convolution\index{produit!de convolution!et Fourier} sur \( L^1(\eR^n)\) :
    \begin{equation}
        \widehat{f*g}=\hat f\hat g.
    \end{equation}
\end{proposition}

\begin{proof}
    Nous devons étudier l'intégrale
    \begin{equation}
        \widehat{f*g}(\xi)=\int_{\eR}\left[ \int_{\eR} f(y)g(t-y)\right] e^{-it\xi} dt.
    \end{equation}
    Ici nous avons choisit des représentants \( f\) et \( g\) dans les classes de \( L^1\). Montrons que \( f\) est borélienne. D'abord \( f(x)=f_+(x)-f_-(x)\) où \( f_+\) et \( f_-\) sont des fonctions positives. Afin d'alléger les notations nous supposons un instant que \( f\) est positive et nous posons
    \begin{equation}
        f_n(x)=\sum_{k=1}^{2^n} \frac{ k }{ n }\mtu_{f(x)\in\mathopen[ \frac{ k }{ n } , \frac{ k+1 }{ n } [}.
    \end{equation}
    Le fait que \( f\) soit dans \( L^1\) implique que chacune des fonctions \( f_n\) est borélienne et donc que \( f\) l'est aussi en tant que limite ponctuelle de fonctions boréliennes\footnote{Le fait que \( f\) soit borélienne est une conséquence du théorème \ref{ThoRWEoqY}.}.
    
    Nous allons appliquer le théorème de Fubini \ref{CorTKZKwP} à la fonction
    \begin{equation}
        \phi(x,y)=f(x)g(y) e^{-i\xi(x+y)}
    \end{equation}
    qui est borélienne en tant que produit et composé de fonctions boréliennes. Nous avons
    \begin{subequations}
        \begin{align}
            \int_{\eR}\left( \int_{\eR}| f(x) e^{-i\xi x} | |g(y) e^{-i\xi y} |dy \right)dx&=\int_{\eR}\left( | f(x) |\int_{\eR}| g(y) |dy \right)dx\\
            &=\int_{\eR}| f(x) |\| g \|_1\\
            &=\| f \|_1\| g \|_1<\infty.
        \end{align}
    \end{subequations}
    Le théorème est donc applicable. D'abord nous avons :
    \begin{subequations}
        \begin{align}
            \hat f(\xi)\hat g(\xi)&=\left(\int_{\eR}f(x) e^{-i\xi x}dx\right)\left(\int_{\eR}g(y) e^{-i\xi y}dy\right)\\
            &=\int_{\eR}\left( \int_{\eR}f(x)g(y) e^{-i\xi(x+y)}dy \right)dx\\
            &=\int_{\eR}\left( \int_{\eR}f(x)g(t-x) e^{-i\xi t} \right)dx.
        \end{align}
    \end{subequations}
    Jusqu'ici nous n'avons pas utilisé Fubini. Nous avons seulement introduit le nombre \( \int_{\eR}g(y) e^{-i\xi y}dy\) dans l'intégrale par rapport à \( x\) et effectué le changement de variables \( y\mapsto t=x+y\). Maintenant nous appliquons le théorème de Fubini pour inverser l'ordre des intégrales :
    \begin{subequations}
        \begin{align}
            \hat f(\xi)\hat g(\xi)&=\int_{\eR}\left( \int_{\eR}f(x)g(t-x) e^{-it\xi}dx \right)dy\\
            &=\int_{\eR} e^{-it\xi}\left( \int_{\eR}f(x)g(t-x)dx \right)dt\\
            &=\int_{\eR} e^{-it\xi}(f*g)(t)dt\\
            &=\widehat{f*g}(\xi).
        \end{align}
    \end{subequations}
\end{proof}

\begin{proposition}       \label{PropJvNfj}
    Soit une fonction \( f\in L^1(\eR^d)\). Alors sa transformée de Fourier est continue\index{transformée!de Fourier!continuité}.
\end{proposition}

\begin{proof}
    Nous considérons une fonction \( f\) définie sur \( \eR^d\) et à valeurs dans \( \eR\) ou \( \eC\). Sa transformée de Fourier est donnée par
    \begin{equation}
        \hat f(\xi)=\int_{\eR^d} e^{-i\xi x}f(x)dx.
    \end{equation}
    Pour montrer que cette fonction \( \hat f\) est continue en \( \xi_0\) nous considérons une suite \( (\xi_n)\to \xi_0\) et nous voulons montrer que \( \hat f(\xi_n)\to\hat f(\xi_0)\). Pour cela nous considérons les fonctions
\begin{equation}
    g_n(x)= e^{-i\xi_nx}f(x)
\end{equation}
qui convergent simplement vers \( g(x)= e^{-i\xi x}f(x)\). Étant donné que
\begin{equation}
    | g_n(x) |<| f(x) |,
\end{equation}
le théorème de la convergence dominée donne alors
\begin{equation}
    \lim_{n\to \infty} \int g_n(x)=\int\lim_{n\to \infty } g_n(x),
\end{equation}
c'est à dire \( \lim_{n\to \infty} \hat f(\xi_n)=\hat f(\xi)\). La fonction \( \hat f\) est donc continue.
\end{proof}

\begin{lemma}       \label{LEMooCBPTooYlcbrR}
    Pour tout \( f\in L^1(\eR^n)\) nous avons \( \| \hat f \|_{\infty}\leq \| f \|_1\).
\end{lemma}

\begin{proof}
    Cela est une simple vérification :
    \begin{equation}
        \hat f(\xi)=\int_{\eR^n}f(x) e^{-ix\xi}dx,
    \end{equation}
    nous avons, pour tout \( \xi\),
    \begin{equation}
        | \hat f(\xi) |\leq\int_{\eR}| f(x) |dx,
    \end{equation}
    ce qui signifie exactement \( \| \hat f \|_{\infty}\leq \| f \|_1\).
\end{proof}

\begin{lemma}[Lemme de Riemann-Lebesgue\cite{MaureyHilbertFourier}]     \label{LesmRLaxXkQV}
    Si \( f\) est une fonction \( L^1(\eR)\) alors \( \lim_{\xi\to\pm\infty} \hat f(\xi)=0\).
\end{lemma}

\begin{proof}
    Nous commençons par prouver le résultat dans le cas d'une fonction \( g\) en escalier, et plus précisément par une fonction caractéristique d'un compact \( K=\mathopen[ a , b \mathclose]\). Au niveau de la transformée de Fourier nous avons
    \begin{equation}
        \hat\mtu_{K}(\xi)=\int_a^b e^{-i\xi x}dx=-\frac{1}{ i\xi }( e^{-ib\xi}- e^{-ia\xi}).
    \end{equation}
    Par conséquent
    \begin{equation}
        | \hat\mtu_K(\xi) |\leq \frac{ 2 }{ | \xi | }.
    \end{equation}
    Plus généralement si \( g=\sum_{i=1}^Nc_i\mtu_{K_i}\), alors
    \begin{equation}
        | \hat g(\xi) |\leq \frac{ 2 }{ | \xi | }\sum_{i=1}^N| c_i |,
    \end{equation}
    et donc nous avons effectivement \( \lim_{\xi\to\pm\infty}| \hat g(\xi) |=0\).

    Nous passons maintenant au cas général \( f\in L^1(\eR)\). Étant donné que les fonctions \( L^1\) en escalier sont denses dans \( L^1\), nous considérons une fonction \( g\in L^1(\eR)\) en escalier telle que \( \| f-g \|_1<\epsilon\). Nous avons donc
    \begin{equation}
        \| \hat f-\hat g \|_{\infty}\leq \| f-g \|_1<\epsilon.
    \end{equation}
    Donc
    \begin{equation}
        \| \hat f(\xi) \|\leq \| \hat f(\xi)-\hat g(\xi) \|_| \hat g(\xi) |.
    \end{equation}
    Le premier terme est plus petit que \( \epsilon\). Il nous reste à voir que 
    \begin{equation}
        \lim_{\xi\to \infty} | \hat g(\xi) |=0,
    \end{equation}
    mais cela est le résultat de la première partie de la preuve.    
\end{proof}

\begin{corollary}
    La transformée de Fourier d'une fonction \( L^1(\eR)\) est bornée.
\end{corollary}

\begin{proof}
    Par le corollaire \ref{PropJvNfj}, la transformée de Fourier d'une fonction \( L^1\) est continue. Le lemme de Riemann-Lebesgue \ref{LesmRLaxXkQV} impliquant qu'elle tend vers zéro en \( \pm\infty\), elle doit être bornée.    
\end{proof}

%--------------------------------------------------------------------------------------------------------------------------- 
\subsection{Formule sommatoire de Poisson}
%---------------------------------------------------------------------------------------------------------------------------

\begin{proposition}[Formule sommatoire de Poisson]   \label{ProprPbkoQ}
    Soit \( f\colon \eR\to \eC\) une fonction continue et \( L^1(\eR)\). Nous supposons que
    \begin{enumerate}
        \item
    il existe \( M>0\) et \( \alpha>1\) tels que
    \begin{equation}
        | f(x) |\leq\frac{ M }{ (1+| x |)^{\alpha} },
    \end{equation}
        \item
            \( \sum_{n=-\infty}^{\infty}| \hat f(2\pi n) |<\infty\).

    \end{enumerate}
    Alors nous avons
    \begin{equation}
        \sum_{n=-\infty}^{\infty}f(n)=\sum_{n=-\infty}^{\infty}\hat f(2\pi n).
    \end{equation}
\end{proposition}
\index{convergence!rapidité}
\index{série!fonctions}
\index{transformation!Fourier}
\index{Fourier}
\index{série!entière}
\index{série!de Fourier}
\index{Poisson!formule sommatoire}
\index{formule!sommatoire de Poisson}

%TODO : Exprimer ce théorème comme truc sur les distributions et sur les machins tempérées, espace de Schwartz.

\begin{proof}
    \begin{subproof}
        \item[Convergence normale]
    
    Nous commençons par montrer qu'il y a convergence normale sur tout compact séparément des séries sur les \( n\geq 0\) et sur les \( n<0\).
    
    Soit \( K\) un compact de \( \eR\) contenu dans \( \mathopen[ -A , A \mathclose]\) et \( n\in \eZ\) tel que \( | n |\geq 2A\). Pour \( x\in K\) nous avons
    \begin{equation}
        | x+n |\geq | n |-| x |\geq | n |-A\geq \frac{ | n | }{ 2 }.
    \end{equation}
    Du coup nous avons un \( \alpha>1\) tel que
    \begin{equation}
        | f(x+n) |\leq \frac{ M }{ \big( 1+| x+n | \big)^{\alpha} }\leq \frac{ M }{ \left( 1+\frac{ | n | }{2} \right)^{\alpha} }.
    \end{equation}
    Lorsque \( n\) est grand, cela a le comportement de \( M/| n |^{\alpha}\) et donc la série
    \begin{equation}
        \sum_{n=0}^{\infty}f(x+n)
    \end{equation}
    est une série convergent normalement. Les deux séries (usuelles) 
    \begin{subequations}
        \begin{align}
            a_-=\sum_{n\leq 0}f(x+n)\\
            a_-=\sum_{n> 0}f(x+n)
        \end{align}
    \end{subequations}
    convergent normalement.
    
\item[Convergence commutative]
    Au sens de la définition \ref{DefIkoheE} nous avons
    \begin{equation}
        \sum_{n\in \eZ}f(x+n)=a_++a_-.
    \end{equation}
    En effet si nous prenons \( J'_0\subset\eN\) fini tel que \( |\sum_{\eN\setminus J_0}f(x+n)-a_+|\leq \epsilon\) et \( J'_1\in -\eN\) tel que \( |\sum_{n\in -\eN\setminus J'_1}f(x+n)|-a_-<\epsilon\), et si nous posons \( J_0=J'_0\cup J'_1\) alors si \( K\) est un ensemble fini de \( \eZ\) contenant \( J_0\) nous avons
    \begin{equation}
        | \sum_{n\in K}f(n+x)-(a_++a_-) |\leq | \sum_{n\in K^+}f(n+x)-a_+ |+| \sum_{n\in K^-}f(n+x)-a_- |\leq 2\epsilon
    \end{equation}
    où $K^+$ sont les éléments positifs de \(K\) et \( K^-\) sont les \emph{strictement} négatifs. Maintenant que la famille \( \{ f(n+x) \}_{n\in \eZ}\) est une famille sommable, nous savons qu'elle est commutativement sommable et que la proposition \ref{PropoWHdjw} nous permet de sommer dans l'ordre que l'on veut. Nous pouvons donc écrire sans ambigüité l'expression \( \sum_{n\in \eZ}f(x+n)\) ou \( \sum_{n=-\infty}^{\infty}f(x+n)\).
    
    \item[re-convergence normale]

        Nous posons donc sans complexes la série
        \begin{equation}
            F(x)=\sum_{n\in \eZ}f(x+n)
        \end{equation}
        qui converge tant commutativement que normalement. Notons que nous pouvons maintenant dire que la série sur \( \eZ\) converge normalement; pas seulement les deux séries séparément.

    \item[Continuité, périodicité]
        Étant donné que chacune des fonctions \( f(x+n)\) est continue, la convergence normale nous assure que \( F\) est continue.

        De plus \( F\) est périodique parce que
        \begin{equation}
            F(x+1)=\sum_{n=-\infty}^{\infty}f(x+1+n)=\sum_{p=-\infty}^{\infty}f(x+p)
        \end{equation}
        où nous avons posé \( p=1+n\).
        
    \item[Coefficients de Fourier]

        En vertu de la définition \eqref{EqhIPoPH} et de la périodicité de \( F\),
        \begin{subequations}
            \begin{align}
                c_n(F)&=\int_{-1/2}^{1/2}F(t) e^{-2\pi int}dt\\
                &=\int_0^1F(t) e^{-2\pi int}dt\\
                &=\int_0^1\sum_{n\in \eZ}f(t+n) e^{-2 i\pi nt}dt\\
                &=\sum_{n\in \eZ}\int_n^{n+1}f(u) e^{-2\pi i (u-n)t}du\\
                &=\int_{-\infty}^{\infty}f(u) e^{-2\pi inu}du\\
                &=\hat f(2\pi n).
            \end{align}
        \end{subequations}
        où nous avons effectué le changement de variables \( u=t+n\), et permuté l'intégrale et la somme en vertu du fait que la somme converge normalement.

    \item[Conclusion]

        Étant donné l'hypothèse \( \sum_{n\in \eZ}| \hat f(n) |<\infty\) la proposition \ref{PropSgvPab} nous dit que
        \begin{equation}
            F(x)=\sum_{n\in \eZ}c_n(F) e^{2\pi inx},
        \end{equation}
        c'est à dire que
        \begin{equation}
            \sum_{n-\infty}^{\infty}f(x+n)=\sum_{n=-\infty}^{\infty}\hat f(2\pi n) e^{2\pi i nx}.
        \end{equation}
        En écrivant cette égalité en \( x=0\) nous trouvons le résultat :
        \begin{equation}
            \sum_{n\in \eZ}f(n)=\sum_{n\in \eZ}\hat f(2\pi n).
        \end{equation}
    \end{subproof}
\end{proof}

\begin{example}\label{ExDLjesf}
\index{convergence!rapidité}
    La formule sommatoire de Poisson peut être utilisée pour calculer des sommes dans l'espace de Fourier plutôt que dans l'espace direct. Nous allons montrer dans cet exemple l'égalité
    \begin{equation}
        \sum_{n=-\infty}^{\infty} e^{-\alpha n^2}=\sum_{n=-\infty}^{\infty}\sqrt{\frac{ \pi }{ \alpha }} e^{-\pi^2 n^2/\alpha}.
    \end{equation}
    Si \( \alpha\) est grand, alors la somme de gauche est plus rapide, tandis que si \( \alpha\) est petit, c'est le contraire.

    Nous appliquons la formule sommatoire de Poisson à la fonction
    \begin{equation}
        f(x)= e^{-\alpha x^2}.
    \end{equation}
    Nous avons
    \begin{subequations}        \label{EqCDeLht}
        \begin{align}
            \hat f(x)&=\int_{\eR} e^{-\alpha t^2-ixt}dt\\
            &= e^{-x^2/4\alpha}\int_{\eR}e^{ -(\sqrt{\alpha}t+\frac{ ix }{ 2\sqrt{\alpha} })^2 }\\
            &= e^{-x^2/4\alpha}\frac{1}{ \sqrt{\alpha} }\int_{\eR+\frac{ ix }{ 2\sqrt{\alpha} }} e^{-u^2}du.
        \end{align}
    \end{subequations}
    Pour traiter cette intégrale nous utilisons la proposition \ref{PrpopwQSbJg} en considérant le chemin rectangulaire fermé qui joint les points \( -R\), \( R\), \( R+ai\), \( -R+ai\) et \( f(z)= e^{-z^2}\). Calculons l'intégrale sur les deux côtés verticaux. Nous posons
    \begin{equation}
        \gamma_R(t)=R+tia
    \end{equation}
    avec \( t\colon 0\to 1\). Nous avons
    \begin{subequations}
        \begin{align}
            \int_{\gamma_R}f&=\int_0^1f\big( \gamma_R(t) \big)\| \gamma_R'(t) \|dt\\
            &=a e^{-R^2}\int_0^1 e^{-2tRia+at^2}dt,
        \end{align}
    \end{subequations}
    donc en module nous avons
    \begin{equation}
        | \int_{\gamma_R}f |\leq a e^{-R^2}\int_0^1 e^{at^2}dt\leq M e^{-R^2},
    \end{equation}
    où \( M\) est une constante ne dépendant pas de \( R\). Lorsque \( R\to \infty\), la contribution des chemins verticaux s'annule et nous trouvons que
    \begin{equation}    \label{EqjrNxLr}
        \int_{\eR+ai} e^{-u^2}du=\int_{\eR} e^{-u^2}du,
    \end{equation}
    que nous pouvons utiliser pour continuer le calcul \eqref{EqCDeLht}. Nous avons
    \begin{equation}
        \hat f(x)= \frac{ e^{-x^2/4\alpha}}{\sqrt{\alpha}}\int_{R} e^{-u^2}du\\
            =\sqrt{\frac{ \pi }{ \alpha }} e^{-x^2/4\alpha}
    \end{equation}
    où nous avons utilisé la formule \eqref{EqFDvHTg}. Par conséquent ce qui rentre dans la formule sommatoire de Poisson est
    \begin{equation}
        \hat f(2\pi n)=\sqrt{\frac{ \pi }{ \alpha }} e^{-\pi^2 n^2/\alpha}.
    \end{equation}
\end{example}

%+++++++++++++++++++++++++++++++++++++++++++++++++++++++++++++++++++++++++++++++++++++++++++++++++++++++++++++++++++++++++++ 
\section{Transformée de Fourier dans l'espace de Schwartz}
%+++++++++++++++++++++++++++++++++++++++++++++++++++++++++++++++++++++++++++++++++++++++++++++++++++++++++++++++++++++++++++

La définition de la transformée de Fourier de \( \varphi\in\swS(\eR^d)\) est 
\begin{equation}
    \hat  \varphi(\xi)=\int_{\eR^n}\varphi(x) e^{-ix\cdot \xi}dx.
\end{equation}

Si \( \alpha\) est un multiindice de taille \( m\), nous notons 
\begin{equation}
    (M_{\alpha}f)(x)=x_{\alpha_1}\ldots x_{\alpha_m}f(x).
\end{equation}

\begin{lemma}[Lemme de transfert]   \label{LemQPVQjCx}
    Si \( \varphi\in\swS(\eR^d)\) et si \( \alpha\) est un multiindice, alors
    \begin{equation}
        \partial^{\alpha}\hat\varphi=(-i)^{| \alpha |}\widehat{M_{\alpha}\varphi}.
    \end{equation}
    et
    \begin{equation}
        \widehat{\partial^{\alpha}\varphi}(\xi)=(-i)^{| \alpha |}\xi^{\alpha}\hat\varphi(\xi).
    \end{equation}
\end{lemma}

\begin{proof}
    Nous considérons la fonction \( h(x,\xi)=\varphi(x) e^{-ix\cdot \xi}\) dont la dérivée par rapport à \( \xi_i\) est donnée par \( -i(M_{i}\varphi)(x) e^{x\cdot \xi}\). Cette fonction est majorée en norme par
    \begin{equation}
        G(x)=M_i\varphi(x),
    \end{equation}
    qui est encore une fonction à décroissance rapide et donc parfaitement intégrable sur \( \eR^d\). Le théorème \ref{ThoMWpRKYp} nous dit donc que la dérivée de \( \hat \varphi\) par rapport à \( \xi_i\) existe et vaut
    \begin{equation}
        \frac{ \partial \hat\varphi }{ \partial \xi_i }(\xi)=-i\int_{\eR^n}x_i\varphi(x) e^{-i\xi\cdot x}=-i\widehat{M_i\varphi}(\xi).
    \end{equation}
    En appliquant ce résultat en chaîne, nous trouvons la première formule annoncée.

    Nous passons à la seconde formule annoncée. Étant donné que \( \varphi\in\swS\), ses dérivées le sont aussi et par conséquent, il n'y a pas de problèmes pour écrire
    \begin{equation}    \label{EqTYizlnia}
        \widehat{\partial_{x_k}\varphi}(\xi)=\int_{\eR^d}\frac{ \partial \varphi }{ \partial x_k }(x) e^{-ix\cdot \xi}dx.
    \end{equation}
    Étant donné que
    \begin{equation}    \label{EqZAeYaCB}
        \frac{ \partial  }{ \partial x_k }\left( \varphi(x) e^{-ix\cdot\xi} \right)=\frac{ \partial \varphi }{ \partial x_k }(x) e^{-ix\cdot\xi}-i\xi_k\varphi(x) e^{-ix\cdot \xi},
    \end{equation}
    notre tâche sera de prouver que
    \begin{equation}    \label{EqVGvYBNK}
        \int_{\eR^d}\frac{ \partial  }{ \partial x_k }\left( \varphi(x) e^{-ix\cdot \xi} \right)dx=0.
    \end{equation}
    Autrement dit, nous voulons montrer que le terme au bord d'une intégration par partie s'annule. D'abord le fait que \( \varphi\) soit à décroissance rapide nous assure que l'intégrale \eqref{EqVGvYBNK} converge. Pour chaque \( \xi\), la fonction
    \begin{equation}
        f(x,\xi)=\frac{ \partial}{\partial x_k }\left( \varphi(x) e^{-ix\cdot \xi} \right)
    \end{equation}
    est intégrable par rapport à \( x\). De plus, \( f\) est dans \( \swS(\eR)\) pour chacune de ses variables (les autres étant fixées). Le théorème de Fubini \ref{ThoFubinioYLtPI} nous permet alors de décomposer l'intégrale en
    \begin{equation}
        \int_{\eR^d}f(x,\xi)dx=\int_{\eR}\ldots\int_{\eR} f(x_1,\ldots, x_d)dx_1\ldots dx_d.
    \end{equation}
    De plus nous pouvons intégrer dans l'ordre de notre choix et nous choisissons évidemment d'intégrer d'abord par rapport à \( x_k\).  Étudions donc l'intégrale
    \begin{equation}
        \int_{\eR}\frac{ \partial  }{ \partial x }\left( \varphi(x) e^{-ix\xi} \right)dx=\lim_{A\to\infty}\int_{-A}^A\frac{ \partial  }{ \partial x }\left( \varphi(x) e^{-ix\xi} \right)dx
    \end{equation}
    dans laquelle nous avons un peu allégé les notations. Une primitive de ce qui est intégré est toute trouvée : c'est \( \varphi(x) e^{-ix\xi}\), et nous pouvons utiliser le théorème fondamental du calcul intégral pour écrire que
    \begin{equation}
        \int_{-A}^A\left( \varphi(x) e^{-ix\xi} \right)'dx=\left[ \varphi(x) e^{-ix\xi} \right]_{x=-A}^{x=A}.
    \end{equation}
    Vu que \( \varphi\) est dans \( \swS\), la limite \( A\to\infty\) donne zéro.

    En substituant maintenant \eqref{EqZAeYaCB} dans \eqref{EqTYizlnia} et en tenant compte du terme que nous venons de montrer s'annuler, nous avons
    \begin{equation}
        \widehat{\partial_k\varphi}(\xi)=-i\xi_k\int_{\eR^d}\varphi(x) e^{-ix\cdot \xi}=-i\xi_k\hat\varphi(\xi).
    \end{equation}
    En recommençant la procédure \( | \alpha |\) fois nous trouvons la seconde formule annoncée.
\end{proof}

\begin{proposition}[\cite{MesIntProbb}] \label{PropKPsjyzT}
    L'espace de Schwartz est stable par transformée de Fourier. De plus l'application
    \begin{equation}
        \TF\colon \swS(\eR^d)\to \swS(\eR^d)
    \end{equation}
    est une bijection linéaire et continue.
\end{proposition}

\begin{proof}
    La linéarité découle de celle de l'intégrale. La difficulté est de prouver que pour \( \varphi\in\swS(\eR^d)\) nous avons bien que \( \hat\varphi\in\swS(\eR^d)\) et que cette association est continue\footnote{Pour rappel, en dimension infinie, il n'est pas garanti qu'une application linéaire soit continue.}.
    \begin{subproof}
        \item[Stabilité]
            Nous devons prouver que pour tout multiindices \( \alpha\) et \( \beta\), nous avons \( p_{\alpha,\beta}(\hat\varphi)<\infty\). Nous avons
            \begin{equation}
                \xi^{\beta}\partial^{\alpha}\hat\varphi(\xi)=\xi^{\beta}(-i)^{| \alpha |}\widehat{M_{\alpha}\varphi}(\xi)=(-i)^{| \alpha |+| \beta |}\widehat{\partial^{\beta}M_{\alpha}\varphi}(\xi).
            \end{equation}
            Ensuite nous nous souvenons que \( \| \hat f \|_{\infty}\leq \| f \|_1\) parce que
            \begin{equation}
                | \hat f(\xi) |\leq\int_{\eR^d}\big| f(x) e^{-ix\cdot \xi} \big|=\int_{\eR^d}| f(x) |dx=\| f \|_1.
            \end{equation}
            Donc 
            \begin{equation}
                p_{\alpha,\beta}(\hat\varphi)=\| \widehat{\partial^{\beta}M_{\alpha}\varphi} \|_{\infty}\leq \| \partial^{\beta}M_{\alpha}\varphi \|_1.
            \end{equation}
            Du fait que \( \varphi\) soit dans \( \swS\), la dernière expression est finie. Cela prouve déjà que
            \begin{equation}
                \TF\big( \swS(\eR^d) \big)\subset\swS(\eR^d).
            \end{equation}
            
        \item[Continuité]

            Nous supposons avoir une suite \( \varphi_n\stackrel{\swS}{\to}\varphi\), et nous devons prouver que \( \hat\varphi_n\stackrel{\swS}{\to}\hat\varphi\). Pour alléger les notations, nous posons \( f_n=\varphi_n-\varphi\). Nous avons
            \begin{subequations}    \label{subEqsSGsGGih}
                \begin{align}
                    \| \hat f \|_{\alpha,\beta}&=\| \xi^{\beta}\partial^{\alpha}\hat f \|_{\infty}\\
                    &=\| \widehat{  \partial^{\beta}M_{\alpha}f  } \|_{\infty}\,\text{lemme \ref{LemQPVQjCx}.}\\
                    &\leq \| \partial^{\beta}M_{\alpha}f \|_1
                \end{align}
            \end{subequations}
            La convergence \(f_n\stackrel{\swS}{\to}0\) nous dit ente autres que \( \partial^{\beta}M_{\alpha}f_n\stackrel{\swS}{\to}0\); en particulier la proposition \ref{PropGNXBeME} nous dit que \( \partial^{\beta}M_{\alpha}f_n\stackrel{L^1}{\to}0\), ce qui signifie, par les majorations \eqref{subEqsSGsGGih} que
            \begin{equation}
                \| \hat f_n \|_{\alpha,\beta}\leq \| \partial^{\beta}M_{\alpha}f_n \|_1\to0,
            \end{equation}
            ce qui prouve la continuité de transformée de Fourier dans \( \swS(\eR^d)\).
        \item[Bijection]
            Une preuve peut être trouvée dans \cite{BMoNzTY}.
    \end{subproof}
    % TODO : Faire le dernier morceau de cette preuve.
\end{proof}

\begin{proposition}[\cite{MonCerveau}]     \label{PROPooMVQMooGYAzSX}
    Soit \( \varphi\in\swS(\eR^n\times \eR^m)\) et la transformée de Fourier partielle
    \begin{equation}
        \tilde \varphi(x,k)=\int_{\eR^m}  e^{-iky}\varphi(x,y)dy.
    \end{equation}
    Alors \( \tilde \varphi\in\swS(\eR^n\times \eR^m  )\).
\end{proposition}

\begin{proof}
    Il s'agit de reprendre les étapes de la partie correspondante de la preuve de la proposition \ref{PropKPsjyzT}. Soient des multiindices \( \alpha\), \( \alpha'\), \( \beta\) et \( \beta'\) où \( \alpha\) et \( \beta\) se réfèrent à la variable \( x\) tandis que \( \alpha'\) et \( \beta'\) se réfèrent à la variable \( k\).

    Vu que la multiplication par \( k^{\beta'}\) commute avec \( \partial^{\alpha}\) nous avons
    \begin{equation}
        x^{\beta}k^{\beta'}\partial^{\alpha}\partial^{\alpha'}\tilde \varphi(x,k)=x^{\beta}k^{\beta'}\partial^{\alpha}(-i)^{| \alpha' |}\widetilde{M_{\alpha'}\varphi}(x,k)=(-i)^{| \alpha' |+| \beta' |}x^{\beta}\partial^{\alpha}\widetilde{    \partial^{\beta'}M_{\alpha'}\varphi  }(x,k).
    \end{equation}
    D'autre part nous avons \( \partial^{\alpha}\tilde \varphi=\widetilde{\partial^{\alpha}\varphi}\) parce que la fonction \( \partial_x\varphi\) étant Schwartz, la fonction
    \begin{equation}
        G(y)=\sup_{x\in \eR^n}|(\partial_x\varphi)(x,y)|
    \end{equation}
    est dans \( L^1(\eR^m)\) par le corollaire \ref{CORooZFPSooHCFUSH}. Par conséquent le théorème \ref{ThoMWpRKYp} permet de permuter la dérivée et l'intégrale dans 
    \begin{equation}
        \frac{ \partial  }{ \partial x }\tilde \varphi(x,k)=\frac{ \partial  }{ \partial x }\int_{\eR^m} e^{-iky}\varphi(x,y)dy.
    \end{equation}
    Dans le même ordre d'esprit mais dans difficultés de permutation de limites nous avons \( M_{\beta}\tilde \varphi=\widetilde{M_{\beta}\varphi}\).

    D'autre part nous avons encore \( \| \tilde \varphi \|_{\alpha}<\infty\) parce que
    \begin{equation}
        | \tilde \varphi(x,k) |\leq \int_{\eR^m}| \varphi(x,y) |dy\leq \sup_x\int_{\eR^m}| \varphi(x,y) |dy\leq \int_{\eR^m}| \sup_x\varphi(x,y) |dy<\infty
    \end{equation}
    parce que \( \varphi\) est Schwartz et le corollaire \ref{CORooZFPSooHCFUSH} donne l'intégrabilité.

    Donc nous avons
    \begin{equation}
        p_{(\alpha\alpha'),(\beta\beta')(\tilde \varphi)}=\|  \widetilde{   \partial^{\beta'}M_{\alpha'}M_{\beta}\partial^{\alpha}\varphi      }    \|_{\infty}<\infty.
    \end{equation}
    Cela prouve que \( \tilde \varphi\) est Schwartz.
\end{proof}

%--------------------------------------------------------------------------------------------------------------------------- 
\subsection{Quelque transformées de Fourier}
%---------------------------------------------------------------------------------------------------------------------------

\begin{example}[\cite{KXjFWKA}] \label{EXooLMXKooFcAZGR}
    Soit la fonction \( g_{\epsilon}(x)= e^{-\epsilon x^2}\). Sa transformée de Fourier sera déduite dans le lemma \ref{LEMooPAAJooCsoyAJ} en utilisant le lemme de transfert \ref{LemQPVQjCx}. Nous nous proposons ici de déduire de façon directe l'équation différentielle vérifiée par la transformée de Fourier de \( g_{\epsilon}\).

    Nous posons
    \begin{equation}
        I(k)=\int_{\eR} e^{-ikx} e^{-\epsilon x^2}dx.
    \end{equation}
    et nous considérons la fonction
    \begin{equation}
        f(k,x)= e^{-ikx} e^{-\epsilon x^2}.
    \end{equation}
    Elle est de classe \( C^1\) par rapport à \( k\), et intégrable en \( x\) pour chaque \( k\). De plus sa dérivée
    \begin{equation}
        (\partial_k f)(k,x)=-ix e^{-ikx} e^{-\epsilon x^2}
    \end{equation}
    vérifie \( | \partial_kf |\leq x e^{-\epsilon x^2}\). La dérivée est donc majorée (uniformément en \( k\)) par une fonction intégrable. Le théorème \ref{ThoMWpRKYp} permet de permuter la dérivée et l'intégrale :
    \begin{subequations}
        \begin{align}
            I'(k)&=\int_{\eR}-ix e^{-ikx} e^{-\epsilon x^2}dx\\
            &=i\int_{\eR} e^{-ikx}\frac{1}{ 2\epsilon } \frac{ d  }{ dx }\left(  e^{-\epsilon x^2} \right)dx\\
            &=\frac{ -i }{ 2\epsilon }\int_{\eR}\frac{ d }{ dx }\left(  e^{-ikx} \right) e^{-\epsilon x^2}dx     &\text{par partie}\\
            &=\frac{ -k }{ 2\epsilon }\int_{\eR} e^{-ikx} e^{-\epsilon x^2}dx\\
            &=\frac{ -k }{ 2\epsilon }I(k).
        \end{align}
    \end{subequations}
    D'où l'équation différentielle \( I'(k)=-\frac{ k }{ 2\epsilon }I(k)\).
\end{example}

\begin{lemma}[Transformée de Fourier de la Gausienne \cite{ooKDRBooDFsyfV}]       \label{LEMooPAAJooCsoyAJ}
    La transformée de Fourier de
    \begin{equation}
        \begin{aligned}
            g_{\epsilon}\colon \eR^d&\to \eR \\
            x&\mapsto  e^{-\epsilon\| x \|^2}
        \end{aligned}
    \end{equation}
    est donnée par
    \begin{equation}
        \hat g_{\epsilon}(\xi)=\left( \frac{ \pi }{ \epsilon } \right)^{d/2} e^{-\| \xi \|^2/4\epsilon}
    \end{equation}
\end{lemma}

\begin{proof}
    Nous commençons par la fonction $ g(x) = e^{-\| x \|^2/2}$ et nous prouvons que sa transformée de Fourier est $\hat g(\xi)=(2\pi)^{d/2}g(\xi)$. 
    \begin{subproof}
        \item[Réduction à la dimension \( 1\)]
        La fonction \( g\) est dans l'espace de Schwartz. Par le théorème de Fubini,
        \begin{equation}
                \hat g(\xi)=\int_{\eR^d}\prod_{k=1}^d e^{-x_k^2} e^{-i\xi_kx_k}dx
                =\prod_{k=1}^d\int_{\eR} e^{-t^2/2} e^{-\xi_kx}dt
                =\prod_{k=1}^d\hat f(\xi_k) \label{EQooXRLIooRCfIOd}
        \end{equation}
        où \( f\) est la fonction d'une variable
        \begin{equation}        \label{EQooFKSPooRBdgnk}
            f(x)= e^{-x^2/2}.
        \end{equation}
        Notons que \( f\in\swD(\eR)\).

    \item[Une équation différentielle]

        Voyons l'équation différentielle satisfaite par la transformée de Fourier \( \hat f\) de la fonction \eqref{EQooFKSPooRBdgnk}. Grâce au lemme \ref{LemQPVQjCx} nous trouvons l'équation différentielle\footnote{Une façon directe de déduire cette équation différentielle est donnée dans l'exemple \ref{EXooLMXKooFcAZGR}.}
        \begin{equation}
            \xi \hat f(\xi)+(\hat f)'(\xi)=0.
        \end{equation}
        C'est le moment d'utiliser le théorème de Cauchy-Lipschitz \eqref{ThokUUlgU}, appliqué à la fonction \( f(t,y)=-ty\) qui est Lipschitz et continue au au problème 
        \begin{subequations}        \label{SUBEQSooWZZKooNEKnME}
            \begin{numcases}{}
                y'+ty=0\\
                y(0)=y_0
            \end{numcases}
        \end{subequations}
        possède une unique solution maximale, en l'occurrence \( y(x)= y_0  e^{-x^2/2}  \). En ce qui concerne la condition initiale nous avons
        \begin{equation}
            \hat f(0)=\int_{\eR} e^{-x^2/2}dx=\sqrt{ 2\pi }.
        \end{equation}
        par l'exemple \ref{ExrgMIni}. Donc
        \begin{equation}
            \hat f(\xi)=\sqrt{ 2\pi } e^{-\xi^2/2}.
        \end{equation}
        En reformant le produit \eqref{EQooXRLIooRCfIOd} nous concluons.
   \end{subproof}

    Nous passons maintenant à la fonction \( g_{\epsilon}\). Nous pouvons écrire \( g_{\epsilon}\) sous la forme
    \begin{equation}
        g_{\epsilon}(x)=g(\sqrt{ 2\epsilon }x).
    \end{equation}
        Utilisant successivement la transformée de Fourier de \( g\) que nous venons de calculer et \ref{LEMooKGDKooVXSMCn} (facteur d'échelle) nous trouvons
        \begin{subequations}
            \begin{align}
                \hat g(\xi)&=(2\pi)^{d/2}g(\xi)\\
                \hat g_{\epsilon}(\xi)&=(2\epsilon)^{-d/2}\hat g\big( \xi/\sqrt{ 2\epsilon } \big)\\
                &=\left( \frac{ \pi }{ \epsilon } \right)^{d/2} e^{-| \xi |^2/4\epsilon} \label{SUBEQooFWIKooGMpFbo}..
            \end{align}
        \end{subequations}
        Nous voyons que \( \hat g_{\epsilon}\in\swS(\eR^d)\)  (c'était gagné d'avance par la proposition \ref{PropKPsjyzT}).
\end{proof}

%+++++++++++++++++++++++++++++++++++++++++++++++++++++++++++++++++++++++++++++++++++++++++++++++++++++++++++++++++++++++++++ 
\section{Suite régularisante}
%+++++++++++++++++++++++++++++++++++++++++++++++++++++++++++++++++++++++++++++++++++++++++++++++++++++++++++++++++++++++++++

\begin{definition}      \label{DEFooRIFYooUUUoha}
    Une \defe{suite régularisante}{suite!régularisante} est une suite \( (\rho_n)\) dans \( L^1(\eR^d)\) telle que
    \begin{enumerate}
        \item       \label{ITEMooEYXYooAkKeXX}
            pour tout \( n\), \( \rho_n\geq 0\) et \( \int_{\eR^d}\rho_n=1\);
        \item
            pour tout \( \alpha>0\),
            \begin{equation}
                \lim_{n\to \infty} \int_{| t |>\alpha}\rho_n=0.
            \end{equation}
    \end{enumerate}
\end{definition}
Une telle suite est régularisante parce que souvent \( \rho_n\in\swD(\eR^d)\), ce qui donne \( f*\rho_n\in C^{\infty}\) par le corollaire \ref{CORooBSPNooFwYQrc}.

\begin{proposition}[\cite{ooYTNMooJsvznx,ooINHXooZWALhj}]       \label{PROPooYUVUooMiOktf}
    Soit une suite régularisante \( \rho_n\in L^1(\eR^d)\). Alors :
    \begin{enumerate}
        \item       \label{ITEMooLWMIooFFamdf}
            Si \( f\) est continue à support compact, nous avons la convergence uniforme sur \( \eR^d\) :
            \begin{equation}
                f*\rho_n\stackrel{unif}{\longrightarrow} f.
            \end{equation}
        \item       \label{ITEMooEJKKooChcgyM}
            Si \( g\in L^p\) (\( 1\leq p<\infty\)) alors 
            \begin{equation}
                g*\rho_n\stackrel{L^p}{\longrightarrow}g.
            \end{equation}
    \end{enumerate}
\end{proposition}

\begin{proof}
    Si \( f\) est continue à support compact, elle est uniformément continue\footnote{Théorème de Heine \ref{ThoHeineContinueCompact}.}, et elle est bornée. Soit \( \epsilon>0\) et \( \alpha>0\) tel que pour tout \( x,y\) tels que \( \| x-y \|<\alpha\) nous ayons \( | f(x)-f(y) |<\epsilon\). Nous prenons de plus \(n\) suffisamment grand pour avoir \( \int_{B(0,\alpha)^c}\rho_n<\epsilon\). Nous avons alors
    \begin{subequations}
        \begin{align}
            | f(x)-(f*\rho_n)(x) |&=| \int_{\eR^d}\big( f(x)-f(y) \big)\rho_n(x-y)dy |\\
            &\leq \int_{B(x,\alpha)}\underbrace{| f(x)-f(y) |}_{\leq \epsilon}\rho_n(x-y)dy+\int_{B(x,\alpha)^c}\underbrace{| f(x)-f(y) |}_{\leq 2\| f \|_{\infty}}\rho_n(x-y)dy\\
            &\leq \epsilon(1+2\| f \|_{\infty}).
        \end{align}
    \end{subequations}
    Nous avons prouvé que pour tout \( \epsilon>0\), il existe \( N\) tel que \( n>N\) implique \( \big| f(x)-(f*\rho_n)(x) \big|\leq \epsilon\). Cela prouve l'uniforme convergence sur \( \eR^d\) de \( f*\rho_n\) vers \( f\).

    Pour le point \ref{ITEMooEJKKooChcgyM} nous considérons \( g\in L^1(\eR^d)\) et \( \phi\in \swD(\eR^d)\). Nous avons la majoration
    \begin{equation}
        \| g*\rho_n-g \|_p\leq \| g*\rho_n-\phi*\rho_n \|_p+\| \phi*\rho_n-\phi \|_p+\| \phi-g \|_p
    \end{equation}
    En ce qui concerne le premier terme;
    \begin{equation}
        \| (g-\phi)*\rho_n \|_p\leq \| g-\phi \|_p
    \end{equation}
    par la proposition \ref{PROPooDMMCooPTuQuS}. Donc
    \begin{equation}
        \| g*\rho_n-g \|_p\leq 2\| g-\phi \|_p+\| \phi*\rho_n-\phi \|_p.
    \end{equation}
    Par la densité de \( \swD\) dans \( L^p\) (théorème \ref{ThoILGYXhX}\ref{ItemYVFVrOIv}) nous pouvons considérer une suite \( \phi_i\stackrel{L^p}{\longrightarrow}g\) dans \( \swD(\eR^d)\). Pour tout \( i\) nous avons
    \begin{equation}
        \| g*\rho_n-g \|_p\leq 2\| g-\phi_i \|_p+\| \phi_i*\rho_n-\phi \|_p.
    \end{equation}
    Nous effectuons la limite sur \( n\to \infty\) :
    \begin{equation}
        \lim_{n\to \infty} \| g*\rho_n-g \|_p\leq 2\| g-\phi_i \|+\underbrace{\lim_{n\to \infty} \| \phi_i*\rho_n-\phi_i \|_p}_{=0}
    \end{equation}
    parce que le point \ref{ITEMooLWMIooFFamdf} s'applique à \( \phi_i\). Nous effectuons ensuite la limite sur \( i\to \infty\) dans
    \begin{equation}
        \lim_{n\to \infty} \| g*\rho_n-g \|\leq 2\| g-\phi_i \|\to 0.
    \end{equation}
\end{proof}

\begin{lemma}       \label{LEMooTDWSooSBJXdv}
    Si \( g_{\epsilon}(x)= e^{-\epsilon\| x \|^2}\) alors la suite
    \begin{equation}        \label{EQooWQWZooZIYGpq}
        \rho_n=\frac{1}{ (2\pi)^d }\hat g_{1/n}
    \end{equation}
    est une suite régularisante (définition \ref{DEFooRIFYooUUUoha}).
\end{lemma}

\begin{proof}
    Nous savons déjà la transformée de Fourier de \( g_{\epsilon}\) par le lemme \ref{LEMooPAAJooCsoyAJ}. Nous montrons que la suite \( \rho_n\) est régularisante. Nous avons \( \hat g_{\epsilon}\in L^1(\eR^d)\) et \( \hat g_{\epsilon}\geq 0\) ainsi que \( \lim_{\epsilon\to 0}\int_{B(0,\alpha)}\hat  g_{\epsilon}=0\) pour tout \( \alpha\). Il y a seulement un couac avec la norme. Nous calculons \( \int_{\eR^d}\hat g_{\epsilon}(\xi)d\xi\) avec la forme \eqref{SUBEQooFWIKooGMpFbo}. En utilisant sauvagement Fubini\footnote{Le pauvre !} pour séparer les intégrales et en effectuant le changement de variable \( u=t/(2\sqrt{ \epsilon })\) nous calculons :
        \begin{subequations}
            \begin{align}
                \int_{\eR^d} e^{-| \xi |^2/4\epsilon}d\xi&=\prod_{k=1}^d\int_{\eR} e^{-t^2/4\epsilon}dt\\
                &=2\sqrt{ \epsilon }\prod_{k=1}^d\int_{\eR} e^{-u^2}du\\
                &=\prod_{k=1}^d2\sqrt{ \epsilon }\sqrt{ \pi }\\
                &=2^d(\pi\epsilon)^{d/2}.
            \end{align}
        \end{subequations}
        Nous avons utilisé l'exemple \ref{EXooLUFAooGcxFUW} ou \ref{ExrgMIni} (au choix). Avec tout cela nous avons
        \begin{equation}
            \int_{\eR^d}\hat g_{\epsilon}=(2\pi)^d.
        \end{equation}
        Donc \( \frac{1}{ (2\pi)^d }\hat g_{1/n}\) est une suite régularisante.
\end{proof}

Le corollaire suivant regroupe les résultats à propos des suites régularisantes, leur utilité et leur existence.
\begin{corollary}
    Si la suite régularisante \( \rho_n\) est dans \( L^1(\eR^d)\cap  C^{\infty}(\eR^d)\) alors pour \( f\in L^p(\eR^d)\) en posant \( f_n=\rho_n*f\) nous avons
    \begin{enumerate}
        \item
            \( f_n\in C^{\infty}(\eR^d)\cap L^p(\eR^d)\)
        \item
            \( f_n\stackrel{L^p}{\longrightarrow}L^p(\eR^d)\)
    \end{enumerate}
    De plus, de telles suites existent.
\end{corollary}

\begin{proof}
    Le fait que \( f_n\) soit de classe \(  C^{\infty}\) est le corollaire \ref{CORooBSPNooFwYQrc}, et la convergence est la proposition \ref{PROPooYUVUooMiOktf}\ref{ITEMooEJKKooChcgyM}.

    De telles suites existent, par exemple celle donnée par le lemme \ref{LEMooTDWSooSBJXdv}.
\end{proof}

%--------------------------------------------------------------------------------------------------------------------------- 
\subsection{Formule d'inversion}
%---------------------------------------------------------------------------------------------------------------------------

\begin{proposition}[Formule d'inversion de Fourier\cite{KXjFWKA}]       \label{PROPooLWTJooReGlaN}
    Si \( f\in\swS(\eR)\), alors nous avons  la formule d'inversion
    \begin{equation}        \label{EQooHIDAooHARdNZ}
        f(x)=\frac{1}{ 2\pi }\int_{\eR} e^{ikx}\hat f(k)dk.
    \end{equation}
    Cette formule peut d'écrire de plusieurs autres façons :
    \begin{subequations}
        \begin{align}
        \TF\big( \TF(f) \big)(x)=2\pi f(-x),\\
        \TF^{-1}(f)(x)=\frac{1}{ 2\pi }\hat f(-x),\\
        f(x)=\frac{1}{ 2\pi }\TF(\hat f)(-x).     \label{EQooWBZTooPeBNeh}
        \end{align}
    \end{subequations}
\end{proposition}

\begin{proof}
    Pour \( \epsilon>0\) nous posons
    \begin{equation}
        f_{\epsilon}(k)= e^{-\epsilon k^2} e^{ikx}\hat f(k).
    \end{equation}
    Nous allons calculer
    \begin{equation}
        \lim_{\epsilon\to 0}\int_{\eR} e^{-\epsilon k^2} e^{ikx}\hat f(k)dk
    \end{equation}
    de deux façons.

    D'abord en utilisant directement le théorème de la convergence dominée \ref{ThoConvDomLebVdhsTf}. La fonction \( \hat f\) est dans \( \swS(\eR)\) (théorème \ref{PropKPsjyzT}) et par conséquent \( f_{\epsilon}\in L^1(\eR)\) parce que le facteur \(  e^{-\epsilon k^2}\) ne va certainement pas empêcher de converger. De plus \( | f_{\epsilon} |\leq | \hat f |\) et \( \hat f\in L^1\). Le théorème est de la convergence dominée est applicable et
    \begin{equation}        \label{EQooYIYGooXYubbW}
        \lim_{\epsilon\to 0}\int_{\eR} e^{-\epsilon k^2} e^{ikx}\hat f(k)dk=\int_{\eR} e^{ikx}\hat f(k)dk.
    \end{equation}
    
    Pour le deuxième calcul nous allons faire appel à Fubini\footnote{Parce qu'il est toujours plus simple de refiler le boulot aux autres que de le faire soi-même\ldots pauvre Fubini !} pour la fonction 
    \begin{equation}
        \begin{aligned}
            u\colon \eR\times \eR&\to \eR \\
            (k,y)&\mapsto  e^{ik(x-y)} e^{-\epsilon k^2}f(y). 
        \end{aligned}
    \end{equation}
    D'abord nous nous assurons que \( u\in L^1(\eR\times \eR)\) par le corollaire \ref{CorTKZKwP}, et ensuite nous utilisons le théorème de Fubini \ref{ThoFubinioYLtPI} pour manipuler les intégrales (et en particulier les inverser). Dans un premier temps nous avons :
    \begin{equation}
        \int_{\eR}\int_{\eR}|  e^{ik(x-y)} e^{-\epsilon k^2} f(y) |dy\,dk\leq \int_{\eR} e^{-\epsilon k^2} \big[  \int_{\eR}| f(y) |   dy \big] dk<\infty
    \end{equation}
    parce que $f$ étant dans \( \swS(\eR)\), l'intégrale intérieure se réduit à un nombre. Nous savons maintenant que \( u\in L^1(\eR\times \eR)\). Nous pouvons alors calculer un peu \ldots
    \begin{subequations}
        \begin{align}
            \int_{\eR} e^{ikx} e^{-\epsilon k^2}\hat f(k)dk&=\int_{\eR}\int_{\eR} e^{ikx} e^{-\epsilon k^2} e^{-iky}f(y)dy\,dk\\
            &=\int_{\eR}\big[ \int_{\eR} e^{ik(x-y)} e^{-\epsilon k^2}f(y)dk \big]dy\\
            &=\int_{\eR}f(y)\big[   \int_{\eR} e^{ik(x-y)} e^{-\epsilon k^2}dk  \big]dy\\
            &=\int_{\eR} f(y)\hat g_{\epsilon}(y-x)dy\\
            &=\sqrt{ \frac{ \pi }{ \epsilon } }\int_{\eR}f(y) e^{-(y-x)^2/4\epsilon}dy\\
            &=2\sqrt{ \epsilon }\sqrt{ \frac{ \pi }{ \epsilon } }\int_{\eR}f(x+2\sqrt{ \epsilon }t) e^{-t^2}dt\\
            &=2\sqrt{ \pi }\int_{\eR}f(x+2\sqrt{ \epsilon }t) e^{-t^2}dt\\
        \end{align}
    \end{subequations}
    Justifications  :
    \begin{itemize}
        \item 
           La fonction \( g_{\epsilon}\) est la gaussienne dont la transformée de Fourier est calculée dans le lemme \ref{LEMooPAAJooCsoyAJ}.
       \item
           Nous avons effectué le changement de variables \( t=(y-x)/(2\sqrt{ \epsilon })\) qui donne \( dt=dy/2\sqrt{ \epsilon }\).
    \end{itemize}
    La fonction \( f\) étant Schwartz (en particulier bornée), dans la dernière intégrale, nous pouvons effectuer la majoration
    \begin{equation}
        f(x+2\sqrt{ \epsilon }t) e^{-t^2}\leq \| f \|_{\infty} e^{-t^2},
    \end{equation}
    qui est une fonction intégrable. Nous pouvons donc permuter la limite et l'intégrale. Dans l'égalité
    \begin{equation}
        \lim_{\epsilon\to 0}\int_{\eR} e^{ikx} e^{-\epsilon k^2}\hat f(k)dk =\lim_{\epsilon\to 0} 2\sqrt{ \pi }\int_{\eR}f(x+2\sqrt{ \epsilon }t) e^{-t^2}dt
    \end{equation}
    À gauche nous avons déjà la limite depuis \eqref{EQooYIYGooXYubbW}, et à droite nous obtenons
    \begin{equation}
        \lim_{\epsilon\to 0} 2\sqrt{ \pi }\int_{\eR}f(x+2\sqrt{ \epsilon }t) e^{-t^2}dt=\int_{\eR}f(x) e^{-t^2}dt=2\sqrt{ \pi }f(x)\sqrt{ \pi }=2\pi f(x)
    \end{equation}
    où nous avons utilisé l'intégrale gaussienne \ref{ExrgMIni}. 
    
    En remettant tout ensemble,
    \begin{equation}
        2\pi f(x)=\lim_{\epsilon\to 0}\int_{\eR} e^{-\epsilon k^2} e^{ikx}\hat f(k)dk=\int_{\eR} e^{ikx}\hat f(k)dk,
    \end{equation}
    ce qu'il fallait prouver.
\end{proof}

\begin{corollary}       \label{CORooAZLZooSviTej}
    Nous avons la formule
    \begin{equation}        \label{EQooRJXRooElEMAa}
        \int_{\eR}\int_{\eR} e^{-ikx}f(x)dx\,dk=2\pi f(0).
    \end{equation}
\end{corollary}

\begin{proof}
    Poser \( x=0\) dans l'équation \eqref{EQooHIDAooHARdNZ}.
\end{proof}

\begin{normaltext}
    Les physiciens qui n'ont que rarement peur écrivent souvent la formule \eqref{EQooRJXRooElEMAa} sous la forme
    \begin{equation}
        \int_{\eR} e^{-ikx}dk=\delta(x)
    \end{equation}
    où \( \delta\) serait la fonction de Dirac qui vaut zéro partout sauf en \( x=0\) où elle vaudrait l'infini, mais pas n'importe quel infini; juste celui qu'il faut pour que sont intégrale valle \( 1\).
\end{normaltext}

\begin{lemma}   \label{LemYYjFZSa}
    Si \( \phi\in\swS(\eR\times \eR^n)\), alors
    \begin{equation}
        \partial_t\hat\phi=\widehat{\partial_t\phi}
    \end{equation}
    où le chapeau dénote la transformée de Fourier par rapport à la variable dans \( \eR^n\) et non par rapport à celle dans \( \eR\). Le \( t\) par contre est la variable dans \( \eR\).
\end{lemma}

\begin{proof}
    Par définition de la transformée de Fourier nous avons
    \begin{equation}
        (\partial_t\hat\phi)(t,\xi)=\frac{ \partial  }{ \partial t }\int_{\eR^n}\phi(t,x) e^{-i x\xi}dx.
    \end{equation}
    Notre but est de permuter l'intégrale et la dérivée en utilisant le théorème \ref{ThoMWpRKYp}. Il nous faut une fonction \( G\colon \eR^n\to \eR\) qui soit intégrable sur \( \eR^n\) et telle que
    \begin{equation}
        \left| \frac{ \partial \phi }{ \partial t }\phi(t,x) \right| \leq G(x)
    \end{equation}
    pour tout \( t\in B(t_0,\delta)\). Étant donné que la fonction \( \partial_t\phi\) est tout autant Schwartz que \( \phi\) elle-même nous pouvons alléger les notations et chercher une fonction \( G\) qui convient pour toute fonction \( \varphi\in\swS(\eR\times \eR^n)\). Soit la fonction
    \begin{equation}
        G(x)=\sup_{t\in B(t_0,\delta)}| \varphi(t,x) |.
    \end{equation}
    Pour tout multiindice \( \alpha\) nous avons alors
    \begin{equation}
        \sup_{x\in \eR^n}\big| x^{\alpha}G(x) \big|\leq \sup_{(t,x)\in \eR\times \eR^n}\big| x^{\alpha}\varphi(t,x) \big|\leq M_{\alpha}\in \eR.
    \end{equation}
    Grâce à la proposition \ref{PropCSmzwGv}, cela signifie que \( \varphi\) décroît plus vite que n'importe quel polynôme; \( G\) est donc intégrable sur \( \eR^n\).
\end{proof}

%+++++++++++++++++++++++++++++++++++++++++++++++++++++++++++++++++++++++++++++++++++++++++++++++++++++++++++++++++++++++++++ 
\section{Transformée de Fourier sur \( L^2(\eR^d)\)}
%+++++++++++++++++++++++++++++++++++++++++++++++++++++++++++++++++++++++++++++++++++++++++++++++++++++++++++++++++++++++++++

La théorie des transformées de Fourier est intéressante sur \( L^2(\eR^d)\) parce qu'elle y donne une isométrie. Nous allons la donner avec des fonctions à valeurs dans \( \eC\).

\begin{remark}
    Une remarque qui vaut ce qu'elle vaut, mais si \( u\) est une classe de fonction pour la relation \( u\sim v\) si et seulement si \( u (x)=v(x)\) pour presque tout \( v\) alors l'intégrale
    \begin{equation}
        \hat u(\xi)=\int_{\eR^d}u(x) e^{ix\xi}dx
    \end{equation}
    ne dépend pas du choix du représentant. Nous pouvons donc parfaitement parler de transformée de Fourier d'une classe de fonctions.
\end{remark}

%--------------------------------------------------------------------------------------------------------------------------- 
\subsection{Extension de \( L^1\cap L^2\) vers \( L^2\)}
%---------------------------------------------------------------------------------------------------------------------------

\begin{theorem}[Extention de la transformée de Fourier vers \( L^2(\eR^d)\)\cite{ooKDRBooDFsyfV}]       \label{THOooJLCDooAjTvJf}
    Soit \( f\in L^1(\eR^d)\cap L^2(\eR^d)\). Alors
    \begin{enumerate}
        \item
            Nous avons \( \TF(f)\in L^2\) et \( \| \hat f\|_{L^2}= (2\pi)^d  \| f \|_{L^2}\).
        \item
            L'application \( \TF\colon L^1\cap L^2\to L^2\) peut être étendue en une application \( \TF\colon L^2(\eR^d)\to L^2(\eR^d)\) vérifiant
            \begin{equation}
                \| \hat f \|_{L^2}=(2\pi)^d\| f \|_{L^2}
            \end{equation}
            pour tout \( f\in L^2(\eR^d)\).
    \end{enumerate}
\end{theorem}

\begin{proof}
    Le fait que \( f\in L^1\) implique \( \| \TF(f) \|_{\infty}\leq \| f \|_1\) (c'est le lemme \ref{LEMooCBPTooYlcbrR}). En particulier, \( | \TF(f)(\xi) |^2\) est majoré et l'intégrale
    \begin{equation}
        \clubsuit=\int_{\eR^d}| \hat f |^2 e^{-\epsilon\xi^2}d\xi
    \end{equation}
    existe et est finie. 
    \begin{subproof}
        \item[Découper l'intégrale]
    Dans un premier temps nous développons les intégrales. Dans les égalités suivantes, \( x\xi\) est le produit scalaire \( x\cdot \xi\) dans \( \eR^d\).
    \begin{subequations}
        \begin{align}
            \clubsuit&=\int_{\eR^d}\left( \int_{\eR^d} \overline{ f(x) } e^{ix\xi}dx \right)\left( \int_{\eR^d} f(y)  e^{-y\xi} \right) e^{- \epsilon | \xi |^2}d\xi\\
            &=\int_{\eR^d}\left[ \int_{\eR^d\times \eR^d}  \overline{ f(x) }  f(y) e^{i\xi(x-y)}dxdy \right] e^{-\epsilon| \xi |^2}d\xi.
        \end{align}
    \end{subequations}
    Nous avons utilisé le théorème de Fubini pour regrouper les intégrales\footnote{Dans la suite nous allons encore utiliser Fubini quelque fois pour regrouper et dégrouper des intégrales.}. Vu que \( f\in L^1(\eR^d)\), la fonction \( (x,y,\xi)\mapsto f(x)f(y) e^{-\epsilon| \xi |^2}\) est dans \( L^1(\eR^d\times \eR^d\times \eR^d)\) et le théorème de Fubini \ref{ThoFubinioYLtPI} avec \( \Omega_1=\eR^d\times \eR^d\) et \( \Omega_2=\eR^d\)  nous permet de permuter les intégrales pour avoir
    \begin{equation}        \label{EQooSUYWooCmtFeF}
        \clubsuit=\int_{\eR^d\times \eR^d} \overline{ f(x) }f(y)  \left[ \int_{\eR^d} e^{i\xi(x-y)} e^{-\epsilon| \xi |^2}d\xi \right]dxdy.
    \end{equation}
    \item[Discuter de cette gaussienne]
        En posant
        \begin{subequations}
            \begin{align}
                g(x)= e^{-| x |^2/2}\\
                g_{\epsilon}(x)=g(\sqrt{ 2\epsilon }x)= e^{-\epsilon| x |^2} 
            \end{align}
        \end{subequations} 
        nous avons \( g_{\epsilon}\in \swS(\eR^d)\) et le lemme \ref{LEMooTDWSooSBJXdv} nous autorise à écrire
        \begin{subequations}
            \begin{align}
                \hat g(\xi)&=(2\pi)^{d/2}g(\xi)\\
                \hat g_{\epsilon}(\xi)&=\left( \frac{ \pi }{ \epsilon } \right)^{d/2} e^{-| \xi |^2/4\epsilon} 
            \end{align}
        \end{subequations}
        Nous voyons que \( \hat g_{\epsilon}\in\swS(\eR^d)\) (c'était gagné d'avance par la proposition \ref{PropKPsjyzT}) et que \( \hat g_{\epsilon}\) est une fonction paire (encore une fois, c'était gagné d'avance parce que la transformée de Fourier d'une fonction paire est paire).

        Tout cela pour dire que l'intégrale entre crochet dans \eqref{EQooSUYWooCmtFeF} est \( \hat g_{\epsilon}(y-x)=\hat g_{\epsilon}(x-y)\), et donc
        \begin{equation}
            \clubsuit=\int_{\eR^d\times \eR^d} \overline{ f(x) }f(y)  \hat g_{\epsilon}(x-y)  dxdy.
        \end{equation}
        Encore une fois le théorème de Fubini permet de séparer les intégrales et de calculer l'intégrale sur \( y\) en premier. Vu que \( f\in L^1\) et que \( \hat g_{\epsilon}\in \swS(\eR^d)\), le produit de convolution \( f*\hat g_{\epsilon}\) est un élément de \( \swS(\eR^d)\) par la proposition \ref{PROPooUNFYooYdbSbJ}.
        Nous avons donc
        \begin{equation}
            \clubsuit=\int_{\eR^d}\overline{ f(x) }(f* \hat g_{\epsilon})(x)dx.
        \end{equation}
        Là, nous reconnaissons un produit scalaire dans \( L^2(\eR^d)\), et donc
        \begin{equation}        \label{EQooWIHNooHutHlS}
            \int_{\eR^d}| \hat f |^2 e^{-\epsilon\xi^2}d\xi=\langle f, f*\hat g_{\epsilon}\rangle_{L^2(\eR^d)}.
        \end{equation}
        Notons que tout a un sens : \( f\in L^2(\eR^d)\) et \( f*\hat g_{\epsilon}\in\swS(\eR^d)\subset L^2(\eR^d)\).

    \item[Suite régularisante]

        Nous prenons la suite régularisante du lemme \ref{LEMooTDWSooSBJXdv} donnée par
        \begin{equation}
            \rho_n=\frac{1}{ (2\pi)^d }\hat g_{1/n}.
        \end{equation}

    \item[Première conclusion]
        Nous reprenons \eqref{EQooWIHNooHutHlS}
        \begin{equation}
            \int_{\eR^d}| \hat f |^2 e^{-| \xi^2 |/n}d\xi=\langle f, f*\hat g_{1/n}\rangle_{L^2(\eR^d)}=(2\pi)^d\langle f, f* \rho_n \rangle .
        \end{equation}
        En prenant la limite \( n\to \infty\) nous trouvons
        \begin{equation}
            \lim_{n\to \infty}\int_{\eR^d}| \hat f |^2 e^{-\epsilon\xi^2}d\xi=(2\pi)^d\| f \|^2.
        \end{equation}
        Pour effectuer la limite du membre de gauche nous devons remarquer qu'en posant
        \begin{equation}
            g_n(\xi)=| \hat f(\xi) | e^{-| \xi |^2/n},
        \end{equation}
        nous avons une suite décroissante de fonction (c'est à dire que à \( \xi\) fixé, c'est décroissant en \(n\)). Par ailleurs ces fonctions sont toujours à valeurs dans \( \mathopen[ 0 , \infty \mathclose]\) et nous pouvons utiliser le théorème de la convergence monotone \ref{ThoRRDooFUvEAN} pour permuter la limite et l'intégrale. Au final :
        \begin{equation}
            \| \hat f \|_{L^2}=(2\pi)^d\| f \|_{L^2}.
        \end{equation}
    \end{subproof}

    En ce qui concerne l'extension, soit \( f\in L^2(\eR^d)\) et une suite \( (f_n)\) dans \( L^1\cap L^2\) telle que \( f_n\stackrel{L^2}{\longrightarrow}f\).
    \begin{subproof}
        \item[Existence d'une telle suite]
            Si \( f\in L^2(\eR^d)\), alors nous pouvons poser 
            \begin{equation}    \label{EQooHGJYooJsmxoX}
                f_n(x)=f(x) e^{-|x|^2/n^2}.
            \end{equation}
            Par l'inégalité de Hölder \eqref{EqLPKooPBCQYN} nous avons \( f_n\in L^1(\eR^d)\); de plus \( f_n\in L^2(\eR^d)\) parce que pour tout \( x\) nous avons \( | f_n(x) |\leq | f(x) |\). Montrons que \( f_n\stackrel{L^2}{\longrightarrow}f\). Nous avons
            \begin{equation}
                \| f_n-f \|_{L^2}^2=\int_{\eR^d}| f(x)(1- e^{-| x^2 |/n^2}) |^2dx.
            \end{equation}
            Nous voulons prendre la limite \( n\to \infty\). Pour ce faire à à droite nous remarquons que \(  e^{-| x |^2/n^2}\) est majoré par \( 1\); ce qui se trouve dans l'intégrale est donc majoré (uniformément en \( n\)) par \( | f(x) |^2\), qui est une fonction \( L^1\) parce que \( f\) est \( L^2\). Le théorème de la convergence dominée \ref{ThoConvDomLebVdhsTf} nous permet alors de permuter la limite et l'intégrale, ce qui donne
            \begin{equation}
                \lim_{n\to \infty} \| f_n-f \|_2^2=\int_{\eR^d}\lim_{n\to \infty} | f(x)(1- e^{-| x |^2/n^2}) |^2dx=0.
            \end{equation}

        \item[Définition de \( \TF\colon L^2\to L^2\)]

            La suite \( (f_n)\) est une suite convergence dans \( L^2\), et elle est donc de Cauchy. De plus pour chaque \( n,m\) nous avons
            \begin{equation}
                \| \hat f_n-\hat f_m \|=(2\pi)^d\| f_n-f_m \|.
            \end{equation}
            La suite \( (\hat f_n)\) est donc elle aussi de Cauchy, dans l'espace \( L^2(\eR^d)\) qui est complet (lemme \ref{LemIVWooZyWodb}). Nous posons
            \begin{equation}
                \hat f=\lim_{n\to \infty} \hat f_n.
            \end{equation}

        \item[Indépendance aux choix]
            Nous devons montrer que la définition de \( \hat f\) ne dépend pas de la suite approximant \( f\) dans \( L^1\cap L^2\). Soient dans deux suites \( f_n\stackrel{L^2}{\longrightarrow}f\) et \( g_n\stackrel{L^2}{\longrightarrow}f\) telles que \( \hat f_n\stackrel{L^2}{\longrightarrow}F\) et \( \hat g_n\stackrel{L^2}{\longrightarrow}G\). Alors
            \begin{equation}
                \| \hat f_n-\hat g_n \|=(2\pi)^d\| f_n-g_n \|\leq (2\pi)^d\| f_n-f \|+(2\pi)^d\| g_n-f \|\to 0.
            \end{equation}
            Par conséquent \( (\hat f_n-\hat g_n)_n\) est une suite qui converge vers zéro. Par unicité de la limite, \( F=G\).
    \end{subproof}
\end{proof}

\begin{remark}
    Une autre suite possible, à la place de \eqref{EQooHGJYooJsmxoX}, est 
    \begin{equation}
        f_n(x)=f(x)\mtu_{| x |<n}.
    \end{equation}
    C'est à dire la fonction \( f\) limitée à une boule de rayon \( n\) autour de \( 0\).
\end{remark}
