% This is part of Mes notes de mathématique
% Copyright (c) 2011-2016
%   Laurent Claessens
% See the file fdl-1.3.txt for copying conditions.

%+++++++++++++++++++++++++++++++++++++++++++++++++++++++++++++++++++++++++++++++++++++++++++++++++++++++++++++++++++++++++++ 
\section{Espaces de Sobolev}
%+++++++++++++++++++++++++++++++++++++++++++++++++++++++++++++++++++++++++++++++++++++++++++++++++++++++++++++++++++++++++++

%--------------------------------------------------------------------------------------------------------------------------- 
\subsection{Sur un intervalle de \( \eR\)}
%---------------------------------------------------------------------------------------------------------------------------

Sauf mention du contraire dans cette section \( I\) est un intervalle borné ouvert \( I=\mathopen] a , b \mathclose[\) de \( \eR\).

\begin{definition}
Soit \( f\in L^p(I)\) où \( I\) est l'intervalle ouvert \( \mathopen] a , b \mathclose[\). Sa \defe{dérivée au sens des distributions}{dérivée!au sens de distributions} est une fonction\footnote{En réalité, c'est une classe au sens de l'égalité presque partout.} \( g\) telle que
        \begin{equation}
            \int_If\varphi'=-\int_Ig\varphi
        \end{equation}
        pour tout \( \varphi\in C^{\infty}_c(I)\).
\end{definition}

\begin{lemma}
    Lorsqu'une fonction admet une dérivée au sens des distributions, cette dernière est unique (et justifie le singulier dans la définition).
\end{lemma}

\begin{proof}
    Soient \( g,h\in L^2\) tels que 
    \begin{equation}
        \int_Iu\varphi'=-\int_Ig\varphi=-\int_Ih\varphi
    \end{equation}
    pour tout \( \varphi\in C^{\infty}_c(I)\). Nous avons alors
    \begin{equation}
        \int_I(g-h)\varphi=0.
    \end{equation}
    Cela implique que \( g-h=0\) presque partout par la proposition \ref{PropUKLZZZh}\footnote{Ou alors par le lemme \ref{LemDQEKNNf} qui est moins général mais tout aussi bien pour ici.}.
\end{proof}

\begin{definition}
    Soit \( I=\mathopen] a , b \mathclose[\) un ouvert borné de \( \eR\). L'\defe{espace de Sobolev}{espace!de Sobolev} \( H^1(I)\)\nomenclature[Y]{\( H^1(I)\)}{espace de Sobolev} est l'ensemble
    \begin{equation}
        H^1(I)=\Big\{   u\in L^2(I)\tq\exists g\in L^2(I)\tq\forall \varphi\in  C^{\infty}_c(I),\int_Iu\varphi'=-\int_Ig\varphi   \Big\}.
    \end{equation}
\end{definition}
 
L'unique élément \( g\) de \( L^2(I)\) vérifiant \( \int_Iu\varphi'=-\int_Ig\varphi\) est noté \( u'\) est est nommé \defe{dérivée}{dérivée!dans Sobolev $ H^1(I)$}; nous verrons dans les prochaines pages pourquoi.

L'espace \( H^1\) accepte le produit scalaire suivant :
\begin{equation}
    \langle u, v\rangle =\int_Iuv+\int_Iu'v',
\end{equation}
et nous notons \( \| . \|_{H^1}\) la norme correspondante qui n'est autre que
\begin{equation}
    \| u \|_{H^1}=\langle u, u\rangle =\| u \|^2_{L^2}+\| u' \|_{L^2}.
\end{equation}

Nous introduisons l'espace \( L^1_{loc}(I)\)\nomenclature[Y]{\( L^1_{loc}(I)\)}{fonctions intégrables sur les compacts de \( I\)} des fonctions étant \( L^1\) sur tout compact de \( I\). 

\begin{proposition} \label{PropLGoLtcS}
    Si \( f\in L^1_{loc}(I)\) est telle que
    \begin{equation}
        \int_If\varphi'=0
    \end{equation}
    pour tout \( \varphi\in  C^{\infty}_c(I)\), alors il existe une constante \( C\) telle que \( f=C\) presque partout.
\end{proposition}

\begin{proof}
    Soit \( \psi\in C^{\infty}_c(I)\) une fonction d'intégrale \( 1\) sur \( I\). Si \( w\in C^{\infty}_c(I)\) alors nous considérons la fonction
    \begin{equation}
        h=w-\psi\int_Iw,
    \end{equation}
    qui est dans \(  C^{\infty}_c(I)\) et dont l'intégrale sur \( I\) est nulle. Par la proposition \ref{PropHFWNpRb}, la fonction \( h\) admet une primitive dans \(  C^{\infty}_c(I)\); et nous notons \( \varphi\) cette primitive. L'hypothèse appliquée à \( \varphi\) donne
    \begin{equation}
        0=\int_If\varphi'=\int_If\left( w-\psi\int_Iw \right)=\int_Ifw-\underbrace{\left( \int_If(x)\psi(x)dx \right)}_C\left( \int_Iw(y)dy \right)=\int_Iw(f-C).
    \end{equation}
    L'annulation de la dernière intégrale implique par la proposition \ref{PropUKLZZZh} que \( f-C=0\) dans \( L^2\), c'est à dire \( f=C\) presque partout.
\end{proof}

\begin{corollary}   \label{CorEVJYihj}
    Si \( u\in H^1(I)\) et si \( u'=0\) alors il existe une constant \( C\) telle que \( u=C\) presque partout.
\end{corollary}

\begin{proof}
    L'hypothèse \( u'=0\) signifie que pour tout fonction \( \varphi\in C^{\infty}_c(I)\),
    \begin{equation}
        \int_Iu\varphi'=\int_Iu'\varphi=0.
    \end{equation}
    La proposition \ref{PropLGoLtcS} nous dit alors qu'il existe une constante \( C\) telle que \( u=C\) presque partout.
\end{proof}

\begin{lemma}   \label{LemMPkbZxX}
    Tout élément de \( H^1(I)\) admet un unique représentant continu.
\end{lemma}
Nous verrons dans le corollaire \ref{CorCEPJGAu} que ce représentant pourra être prolongé par continuité sur \( \bar I\).

\begin{proof}
    Soit \( y_0\in I\) et \( u\in H^1(I)\). Nous considérons la fonction
    \begin{equation}
        \bar u(x)=\int_{y_0}^xu'(t)dt.
    \end{equation}
    Notons que par définition, \( u'\in L^2\) donc l'intégrale ne pose pas de problèmes. Montrons que \( \bar u\) est continue sur \( \bar I\). Pour cela nous considérons \( x\in\bar I\) et \( h\) tel que \( x+h\in \bar I\). Alors
    \begin{equation}
        \big| \bar u(x+h)-\bar u(x) \big|=\big| \int_x^{x+h}u' \big|\leq \int_x^{x+h}| u' |.
    \end{equation}
    Mais la fonction \( | u' |\) est dans \( L^1_{loc}(I)\) par le lemme \ref{LemTLHwYzD}; elle est en particulier intégrable sur un ouvert contenant \( x\) et par conséquent la dernière intégrale tend vers zéro lorsque \( h\) tend vers \( 0\).

    Nous prouvons à présent que \( \bar u\) est dans \( H^1(I)\) et que sa dérivée est égale à \( u'\); pour cela nous allons montrer que pour tout \( \varphi\in  C^{\infty}_c(I)\),
    \begin{equation}
        \int_I\bar u\varphi'=-\int_Iu'\varphi.
    \end{equation}
    Nous avons
    \begin{equation}
            \int_I\bar u\varphi'=\int_I\left( \int_{y_0}^xu'(t)dt\right)\varphi'(x)dx
            =\int_{a}^{y_0}\left( \int_{y_0}^xu'(t)dt\right)\varphi'(x)dx+\int_{y_0}^b\left( \int_{y_0}^xu'(t)dt\right)\varphi'(x)dx.
    \end{equation}
    Pour faire plus court, nous notons \( f(t,x)=u'(t)\varphi'(x)\). La première intégrale vaut
    \begin{subequations}
        \begin{align}
            \int_a^{y_0}\left( \int_{y_0}^x u'(t)\varphi'(x) \right)&=\int_a^{y_0}\left(\int_{y_0}^af(t,x)\mtu_{t<x}(t,x)dt\right)dx\\
            &=\int_{y_0}^a\int_a^{y_0}f(t,x)\mtu_{t>x}dxdt  \label{subeqBVyBPLp}\\
            &=\int_{y_0}^a\int_a^tf(t,x)dxdt\\
            &=-\int_a^{y_0}\int_a^tu'(t)\varphi'(x)dx\,dt
        \end{align}
    \end{subequations}
    La permutation d'intégrales pour obtenir \eqref{subeqBVyBPLp} est due au théorème de Fubini \ref{ThoFubinioYLtPI}\ref{ItemQMWiolgiii}. Par le même petit jeu, la seconde intégrale vaut
    \begin{equation}
        \int_{y_0}^b\int_t^b u'(t)\varphi'(x)dx\,dt.
    \end{equation}
    En refaisant la somme,
    \begin{subequations}
        \begin{align}
            \int_I\bar u\varphi'
            &=-\int_a^{y_0}u'(t)\left( \int_a^t\varphi'(x)dx \right)dt+\int_{y_0}^bu'(t)\left( \int_t^b\varphi'(x)dx \right)dt\\
            &=-\int_a^{y_0}u'(t)\big( \varphi(t)-\varphi(a) \big)dt+\int_{y_0}^bu'(t)\big( \varphi(b)-\varphi(t) \big)\\
            &=-\int_a^bu'\varphi\\
            &=-\int_Iu'\varphi.
        \end{align}
    \end{subequations}
    Notons que \( \varphi(a)=\varphi(b)=0\) parce que \( \varphi\) est à support compact dans \( \mathopen] a , b \mathclose[\). Nous avons donc prouvé que \( \bar u\) est dans \( H^1(I)\) et que \( \bar u'=u'\). Par le corollaire \ref{CorEVJYihj}, nous avons une constante \( C\) telle que \( \bar u=u+C\) presque partout, c'est à dire \( u=\bar u +C\) dans \( H^1(I)\). 

        En résumé, \( \tilde u\tilde u=\bar u+C\) est un représentant continu de \( u\) dans \( L^2(I)\).

        L'unicité du représentant continu est simplement le fait que deux fonctions continues égales presque partout sont égales (proposition  \ref{PropNCMToWI}).
    
\end{proof}

\begin{proposition}     \label{PropGWOIoDg}
    Si \( u\in H^1(I)\), alors
    \begin{equation}
        u(x)-u(y)=\int_y^xu'
    \end{equation}
    pour tout \( x,y\in I\).
\end{proposition}

\begin{proof}
    Pour fixer les idées, nous supposons \( x<y\). Nous considérons une suite \( \varphi_n\in C^{\infty}_c(I)\) convergeant uniformément sur \( I\) vers \( \mtu_{\mathopen[ x , y \mathclose]}\). Nous exigeons de plus que 
    \begin{itemize}
        \item 
        \( \varphi_n'\) est positive sur \( \mathopen[ a , x+\frac{1}{ n } \mathclose]\)
    \item
        \( \varphi_n'\) est négative sur \( \mathopen[ y-\frac{1}{ n } , b \mathclose]\) 
    \item
        \( \varphi_n=1\) sur \( \mathopen[ x+\frac{1}{ n } , y-\frac{1}{ n } \mathclose]\).
    \item
        \( \varphi_n=0\) sur \( \mathopen[ a , x-1/n \mathclose]\) et sur \( \mathopen[ y+1/n , b \mathclose]\).
    \end{itemize}
    Pour chaque \( n\), nous découpons l'intégrale comme
    \begin{equation}        \label{EqRPwqpve}
        -\int_Iu'\varphi_n=\int_Iu\varphi'_n=\int_a^{a-1/n}u\varphi'_n+\int_{x-1/n}^{x+1/n}u\varphi'_n+\int_{x+1/n}^{y-1/n}u\varphi'_n+\int_{y-1/n}^{y+1/n}u\varphi'_n+\int_{y+1/n}^{b}u\varphi'_n.
    \end{equation}
    Par construction de \( \varphi_n\), de ces \( 5\) morceaux, il n'en reste que deux de non nulles :
    \begin{equation}
        \int_Iu\varphi'=\underbrace{\int_{x-1/n}^{x+1/n}u(t)\varphi'_n(t)dt}_A+\underbrace{\int_{y-1/n}^{y+1/n}u(t)\varphi'_n(t)dt}_B
    \end{equation}

    Soit \( \epsilon>0\) et \( n\) suffisamment grand pour avoir \( u(t)\in B\big( u(x),\epsilon \big)\) pour tout \( t\in B(x,\frac{1}{ n })\) et (en même temps) \( u(t)\in B\big( u(y),\epsilon \big)\) pour tout \( t\in B(y,\frac{1}{ n })\). C'est la continuité de \( u\) qui permet de trouver un tel \( n\). Pour cette valeur de \( n\), en tenant compte des hypothèses sur la positivité de \( \varphi_n'\) nous avons
    \begin{equation}
        \int_{x-1/n}^{x+1/n}\big( u(x)-\epsilon \big)\varphi'_n(t)dt\leq\int_{x-1/n}^{x+1/n}u(t)\varphi'_n(t)dt\leq\int_{x-1/n}^{x+1/n}\big( u(x)+\epsilon \big)\varphi'_n(t)dt,
    \end{equation}
    mais par hypothèse sur \( \varphi_n\) nous trouvons
    \begin{equation}
        \int_{x-1/n}^{x+1/n}\varphi'_n(t)dt=\varphi_n(x+\frac{1}{ n })-\varphi(x+\frac{1}{ n })=1.
    \end{equation}
    donc
    \begin{equation}    \label{EqLYrpEdb}
        u(x)-\epsilon\leq\int_{x-1/n}^{x+1/n}u(t)\varphi'_n(t)dt\leq u(x)+\epsilon.
    \end{equation}
    Pour encadrer la seconde, il faut être plus prudent avec les signes parce que \( \varphi'_n\) y est négative. En posant \( \psi_n=-\varphi_n\) nous avons
    \begin{equation}
        -B=\int_{y-1/n}^{y+1/n}u(t)\psi_n(t)dt,
    \end{equation}
    et donc
    \begin{equation}
        u(y)-\epsilon\leq -B\leq u(y)+\epsilon
    \end{equation}
    ou encore
    \begin{equation}
        -\epsilon-u(y)\leq B\leq \epsilon-u(y).
    \end{equation}
    En additionnant avec \eqref{EqLYrpEdb} nous voyons que pour tout \( \epsilon>0\) il existe un \( N(\epsilon)\) tel que nous ayons
    \begin{equation}    \label{EqEBwWUxm}
        u(x)-u(y)-2\epsilon\leq\int_Iu'\varphi_{n}\leq u(x)-u(y)+2\epsilon
    \end{equation}
    pour tout \( n\geq N\). Nous voulons évidemment prendre la limite \( \epsilon\to 0\), c'est à dire \( n\to \infty\). Étant donné que \( \varphi_n(t)<1\) pour tout \( t\) et pour tout \( n\), la fonction \( t\mapsto u'(t)\varphi_n(t)\) est dominée par \( u'\), qui est dans \( L^1(I)\) par le lemme \ref{LemTLHwYzD}. Le théorème de la convergence dominée nous permet donc d'affirmer que
    \begin{equation}
        \lim_{n\to \infty} \int_Iu'\varphi_n=\int_Iu'\mtu_{[x,y]}=\int_x^yu',
    \end{equation}
    et donc les inégalités \eqref{EqEBwWUxm} donnent le résultat, grâce au signe dans \eqref{EqRPwqpve}.
\end{proof}

\begin{corollary}   \label{CorCEPJGAu}
    Si \( [u]\in H^1(I)\), le représentant continu \( u\in C^0(I)\) peut être prolongé par continuité en \( u\in C^0(\bar I)\).
\end{corollary}

\begin{proof}
    Soit \( (x_n)\) une suite strictement croissante dans \( \mathopen] a , b \mathclose[\) convergeant vers \( b\). Nous voulons montrer que la suite \( \big( u(x_n) \big)\) est de Cauchy dans \( \eR\), ce qui nous permettra de définir
        \begin{equation}
            u(b)=\lim_{n\to \infty} u(x_n).
        \end{equation}
        qui sera évidemment continue. Cette construction ne dépendra pas du choix de la suite \( (x_n)\) parce que deux fonctions continues sur \( \bar I\) et égales sur \( I\) sont égales sur \( \bar I\).

        En notant \( u'\) la dérivée de \( u\) dans \( H^1\), nous avons par construction du représentant continu : \( u(x)=\int_{y_0}^xu'(t)dt\). Et donc
        \begin{equation}
            \big| u(x_n)-u(x_{n+p}) \big|=\left| \int_{y_0}^{x_n}u'-\int_{y_0}^{x_{n+p}}u' \right| =\left| \int_{x_n}^{x_{n+p}}u' \right| .
        \end{equation}
        Vu que la suite \( (x_n)\) est de Cauchy et que \( u'\) est intégrable (même sur \( \bar I\)), la limite \( n\to\infty\) de cela est zéro, quelle que soit la valeur de \( p\). Donc \( \big( u(x_n) \big)\) est ce Cauchy dans \( \eR\) et est donc convergente.
\end{proof}
\index{prolongement!par continuité!dans \( H^1(I)\)}

\begin{proposition}[\cite{KXjFWKA}]     \label{ThoESIyxfU}
    Quelques propriétés de l'espace de Sobolev \( H^1(I)\) où \( I=\mathopen] a , b \mathclose[\) est un ouvert borné de \( \eR\).
    \begin{enumerate}
        \item
            \( H^1(I)\) est un espace de Hilbert.
        \item
            \( H^1(I)\) s'injecte de façon compacte dans \( C^0(\bar I)\).
        \item
            \( H^1(I)\) s'injecte de façon continue dans \( L^2(I)\).
    \end{enumerate}
\end{proposition}
\index{espace!de fonctions!Sobolev \( H^1\)}
\index{espace!de Hilbert!espace de Sobolev \( H^1\)}
\index{espace!\( L^2\)!Sobolev}
\index{dérivation!au sens des distribution!Sobolev}


\begin{proof}
    Nous prouvons point par point.
    \begin{enumerate}
        \item
            Le seul critère à vérifier est la complétude. Pour cela nous considérons une suite de Cauchy \( (u_n)\) dans \( H^1(I)\). Si \( \epsilon>0\), alors il existe \( N>0\) tel que pour tout \( p\geq 0\) nous ayons \( \| u_{n+p}-u_n \|_{H^1}^2\leq \epsilon\), c'est à dire
            \begin{equation}
                \| u_{n+p}-u_n \|^2_{L^2}+\| u'_{n+p}-u'_n \|^2_{L^2}+
            \end{equation}
            En particulier les suites \( (u_n)\) et \( (u'_n)\) sont de Cauchy dans \( L^2\) qui est complet par le théorème de Fischer-Riesz \ref{ThoGVmqOro}. Nous notons donc
            \begin{subequations}
                \begin{align}
                    u_n\stackrel{L^2}{\to}u\\
                    u'_n\stackrel{L^2}{\to}v.
                \end{align}
            \end{subequations}
            Nous allons démontrer les points suivants\quext{C'est le moment de lire l'énoncé du problème \ref{ProbTOElufz} et de m'écrire si vous avez une réponse.}
            \begin{itemize}
                \item \( u\in H^1(I)\) avec \( u'=v\).
                \item \( u_n\stackrel{H^1}{\to}u\).
            \end{itemize}
            Pour cela nous introduisons la dérivée faible de \( u\) dans \( L^2\), c'est à dire la forme linéaire continue \( \partial u\) sur \(  C^{\infty}_c(I)\) :
            \begin{equation}
                \begin{aligned}
                    \partial u\colon  C^{\infty}_c(I)&\to \eR \\
                    \varphi&\mapsto \langle \partial u, \varphi\rangle =-\int_Iu\varphi'. 
                \end{aligned}
            \end{equation}
            Pour tout \( \varphi\in C^{\infty}_c(I)\) nous avons
            \begin{subequations}
                \begin{align}
                \big| \langle \partial u, \varphi\rangle -\langle u_n', \varphi\rangle  \big|&=\left| -\int_Iu\varphi'-\int_Iu'_n\varphi \right| \\
                &=\left| -\int_Iu\varphi'-\int_Iu_n\varphi' \right| \\
            &\leq \int_I| u-u_n | |\varphi' |\\
            &\leq\| u-u_n \|_{L^2}\| \varphi' \|_{L^2}\,\text{Cauchy-Schwartz dans \( L^2\)}\\
            &\to 0.
                \end{align}
            \end{subequations}
            À la première ligne, la première intégrale est la définition de l'action de la forme \( \partial u\) sur \( \varphi\) alors que la seconde est seulement un produit scalaire dans \( L^2\). Tout deux sont notés avec les crochets. En tant que limite dans \( \eR\) nous avons
            \begin{equation}
                \lim_{n\to \infty} \langle u'_n, \varphi\rangle =\langle \partial u, \varphi\rangle .
            \end{equation}
            Dans le calcul suivant, les deux crochets sont des produits scalaires dans \( L^2\) :
            \begin{subequations}
                \begin{align}
                \big| \langle u_n', \varphi\rangle -\langle v, \varphi\rangle  \big|&=\left| -\int_Iu'_n\varphi-\int_Iv\varphi \right| \\
            &\leq \int_I| u'_n-v| |\varphi |\\
            &\leq\| u'_n-v \|_{L^2}\| \varphi \|_{L^2}\\
            &\to 0.
                \end{align}
            \end{subequations}
            Donc en tant que limite dans \( \eR\),
            \begin{equation}
                \lim_{n\to \infty} \langle u'_n, \varphi\rangle =\langle v, \varphi\rangle .
            \end{equation}
            Par unicité de la limite nous en déduisons que pour tout \( \varphi\in C^{\infty}_c(I)\),
            \begin{equation}
                \langle \partial u, \varphi\rangle =\langle v, \varphi\rangle .
            \end{equation}
            Encore une fois nous répétons qu'à gauche le crochet est l'application de la forme \( \partial u\) sur \( \varphi\) tandis qu'à droite c'est le produit scalaire dans \( L^2\). 

            Nous sommes maintenant à même de prouver que \( u\in H^1(I)\) et que sa dérivée (au sens de \( H^1\)) est \( v\). En effet
            \begin{equation}
                \int_Iu\varphi'=-\langle \partial u, \varphi\rangle =-\langle v, \varphi\rangle =-\int_Iv\varphi.
            \end{equation}
            Par conséquent nous avons \( u'=v\) dans \( H^1\) et aussi \( u'=v\) presque partout au sens des fonctions.

            Nous pouvons alors prouver que \( u_n\to u\) dans \( H^1(I)\) :
            \begin{equation}
                \| u_n-u \|^2_{H^1(I)}=\| u_n-u \|^2_{L^2}+\| u'_n-u' \|_{L^2}^2.
            \end{equation}
            Mais nous savons déjà que \( u_n\to u\) dans \( L^2\) (d'ailleurs c'est la définition de \( u\)) et que \( u'=v\) alors que par définition de \( v\), nous avons \( u'_n\to v\) dans \( L^2\). Tout cela donne que \( u_n\to u\) dans \( H^1(I)\) et donc que \( H^1(I)\) est un espace complet.

        \item

            L'application que nous allons prouver être compacte entre \( H^1(I)\) et \( C^0(\bar I)\) est
            \begin{equation}
                \begin{aligned}
                    \psi\colon H^1(I)&\to C^0(\bar I) \\
                    [u]&\mapsto \tilde u 
                \end{aligned}
            \end{equation}
            où \( [u]\) désigne une classe de fonction dans \( H^1(I)\) et \( \tilde u\) est son représentant continu prolongé par continuité à \( \bar I\)\footnote{Encore que par soucis d'économie d'encre nous n'allons pas écrire toujours les tildes et noter \( u\) le représentant continu prolongé à \( \bar I\) par le corollaire \ref{CorCEPJGAu}.}, qui existe par le lemme \ref{LemMPkbZxX} et le corollaire \ref{CorCEPJGAu}. Cette application est une injection par l'unicité du représentant continu. Nous allons prouver que c'est une application compacte en utilisant le critère \ref{ItemJIkpUbLii} de la proposition \ref{PropDGsPtpU}. Pour cela nous allons commencer par utiliser le théorème d'Ascoli sur l'ensemble \( \tilde \mB\) des représentants continus des éléments de \( \mB\), prolongés par continuité sur \( \bar I\); c'est à dire \( \tilde B\subset C^0(\bar I)\).

            Soit \( u\in \tilde \mB\); par la proposition \ref{PropGWOIoDg}, nous avons
            \begin{subequations}
                \begin{align}
                    \big| u(x)-u(y) \big|&=\big| \int_y^xu'(t)dt \big|\\
                    &=\left| \int_I\mtu_{[x,y]}(t)u'(t)dt \right| \\
                    &\leq\| \mtu_{\mathopen[ x , y \mathclose]} \|_{L^2}\| u' \|_{L^2}\\
                    &\leq\sqrt{| x-y |}\| u' \|_{H^1}\\
                    &\leq\sqrt{| x-y |}.
                \end{align}
            \end{subequations}
            où nous insistons sur le fait que la continuité n'impliquant pas la dérivabilité, le \( u'\) ici est la dérivé au sens de \( H^1\), et non la dérivée usuelle. Quoi qu'il en soit, l'ensemble \(\tilde  \mB\) est équicontinu\footnote{Définition \ref{DefUWmVBcZ}}. Nous montrons à présent qu'il est également borné pour la norme uniforme. Soit \( u\in\tilde \mB\); vu la construction du représentant continu au lemme \ref{LemMPkbZxX}, nous avons
            \begin{subequations}
                \begin{align}
                \big| u(x) \big|&=\left| \frac{1}{ b-a }\int_a^bu(x)dy \right| \\
                &=\left| \frac{1}{ b-a }\int_a^b\left( \int_y^xu'(t)dt-u(y) \right)dy \right| \\
                &=\left| \frac{1}{ b-a }\int_a\int_y^xu'(t)dtdy-\frac{1}{ b-a }\int_a^b u(y)dy \right| \\
                &\leq\frac{1}{ b-a }\int_a^b\int_a^b| u'(t) |dt\,dy+\frac{1}{ b-a }\int_a^b| u(y) |dy \label{EqCFwSOxh}.
                \end{align}
            \end{subequations}
            À ce niveau, il faut remarquer que dans la première intégrale, le passage de la valeur absolue à l'intérieur de l'intégrale en même temps que l'élargissement des bornes n'a rien d'innocent. Si \( x<y\), les bornes ne sont pas «dans le bon ordre» et nous ne pouvons pas faire la majoration usuelle en entrant simplement la valeur absolue. Ici nous tenons compte de cela en élargissant les bornes, et en les mettant dans le bon ordre. Le passage exact est le suivant : si \( x,y\in\mathopen] a , b \mathclose[\), nous avons
                \begin{equation}
                \left| \int_y^xf(t)dt \right| \leq\left| \int_y^x| f(t) |dt \right| \leq\left| \int_a^b| f(t) |dt \right| =\int_a^b| f(t) |dt.
                \end{equation}
                Notons en particulier que dans le cas du passage vers l'équation \eqref{EqCFwSOxh}, le nombre \( x\) est fixé alors que \( y\) est une variable d'intégration. Donc l'ordre des deux est certainement de temps en temps le «mauvais».

                Quoi qu'il en soit, la première intégrale se réduit à une multiplication par \( b-a\) et le calcul continue :
                \begin{subequations}
                    \begin{align}
                        \big| u(x) \big|&\leq \int_I| u'(t) |dt+\frac{1}{ b-a }\int_I| u |\\
                        &\leq \sqrt{b-a}\| u' \|_{L^2}+\frac{1}{ \sqrt{b-a} }\| u \|_{L^2}\\
                        &\leq\left( \sqrt{b-a}+\frac{1}{ \sqrt{b-a} } \right)\big( \| u' \|_{L^2}+\| u \|_{L^2} \big)\\
                        &\leq\left( \sqrt{b-a}+\frac{1}{ \sqrt{b-a} } \right) \| u \|_{H^1}\\
                        &= \sqrt{b-a}+\frac{1}{ \sqrt{b-a} }.
                    \end{align}
                \end{subequations}
                Donc \( \tilde \mB\) est borné pour la norme \( L^{\infty}\). Et c'est même borné par un nombre facilement calculable connaissant \( I\). En particulier l'ensemble
                \begin{equation}
                    \{ u(x)\tq u\in H^1 \}
                \end{equation}
                est pour, tout \( x\), contenu dans la boule de rayon \( \sqrt{a-b}+\frac{1}{ \sqrt{a-b} }\) et donc est relativement compact dans \( \eR\). Par conséquent le théorème d'Ascoli \ref{ThoKRbtpah} nous dit que l'ensemble \( \tilde B\) est relativement compact dans \( C^0(I)\).

                Par conséquent nous avons montré que l'image par \( \psi\) de la boule unité fermée \( \mB\) de \( H^1(I)\) est relativement compacte dans \( C^0(\bar I)\), ce qui signifie que \( \psi\) est une application compacte.


            \item

                Les éléments de \( H^1(I)\) sont des éléments de \( L^2(I)\); donc l'identité est une injection. Nous devons seulement étudier la continuité. Si \( (u_n)\) est une suite dans \( H^1\) convergeant dans \( H^1\) vers \( u\), alors
                \begin{equation}
                    \| u_n-u \|_{L^2}\leq\| u_n-u \|_{L^2}+\| u'_n-u' \|_{L^2}=\| u_n-u \|_{H^1}\to 0.
                \end{equation}
                Donc la suite des images (par l'identité) converge dans \( L^2\). L'identité est donc continue.

    \end{enumerate}
    
\end{proof}

\begin{probleme}    \label{ProbTOElufz}
    Au point de la preuve auquel vous devriez être si vous lisez ceci, vous pourriez avoir envie de démontrer \( u'=v\) de la façon suivante :
    \begin{equation}
        \int_I u\varphi'=\lim_{n\to \infty} \int_Iu_n\varphi=-\lim_{n\to \infty} \int_Iu'_n\varphi=-\int_Iv\varphi.
    \end{equation}
    J'avoue ne pas trouver d'exemples pour lesquels ça ne marche pas. Est-ce qu'on peut inverser la limite et l'intégrale dans \( L^2\) ?

    Ceci n'invalide pas la preuve donnée, mais ça suggère un sacré raccourcis.
\end{probleme}

%--------------------------------------------------------------------------------------------------------------------------- 
\subsection{Sur un ouvert de \( \eR^n\)}
%---------------------------------------------------------------------------------------------------------------------------

%///////////////////////////////////////////////////////////////////////////////////////////////////////////////////////////
\subsubsection{Dérivée partielle au sens faible}
%///////////////////////////////////////////////////////////////////////////////////////////////////////////////////////////

Soit \( \Omega\), un ouvert bornée de \( \eR^n\) et \( v\in L^2(\Omega)\) (définition \ref{DEFooSVCHooIwwuIx}). 

\begin{definition}
    Si \( i=1,\ldots, n\), la \defe{dérivée faible}{dérivée!faible} de \( v\) dans la direction \( e_i\) est l'application\footnote{En fait c'est une classe au sens de l'égalité presque partout.} notée \( \partial_iv\) définie par
    \begin{equation}
        \langle \partial_iv, \phi\rangle =-\langle v, \partial_i\phi\rangle 
    \end{equation}
    pour tout \( \phi\in  C^{\infty}_c(\Omega)\).
\end{definition}

\begin{lemma}
    Si \( v\in L^2\) admet une dérivée faible, alors cette dernière est unique.
\end{lemma}

\begin{proof}
    Supposons \( f,g\) telles que \( \langle g, \phi\rangle \) et \( \langle f, \phi\rangle \) soient tous deux égaux à \( -\langle v, \partial_i\phi\rangle \). En particulier pour tout \( \phi\in  \swD(\Omega)\) nous avons \( \langle (f-g), \phi\rangle =0\). 

    Cela donne \( f-g=0\) par la proposition \ref{PropUKLZZZh}.
\end{proof}

%///////////////////////////////////////////////////////////////////////////////////////////////////////////////////////////
\subsubsection{Définition}
%///////////////////////////////////////////////////////////////////////////////////////////////////////////////////////////

\begin{definition}
    Soit \( \Omega\) un ouvert borné de \( \eR^n\). L'espace de \defe{Sobolev}{Espace!de Sobolev} \( H^1(\Omega)\)\nomenclature[Y]{\( H^1(\Omega)\)}{espace de Sobolev sur \( \Omega\)} est:
    \begin{equation}
        H^1(\Omega)=\{ v\in L^2(\Omega)\tq \forall i=1,\ldots, n, \partial_iv\in L^2(\Omega) \}<++>
    \end{equation}
\end{definition}


%+++++++++++++++++++++++++++++++++++++++++++++++++++++++++++++++++++++++++++++++++++++++++++++++++++++++++++++++++++++++++++ 
\section{Théorèmes de Hahn-Banach}
%+++++++++++++++++++++++++++++++++++++++++++++++++++++++++++++++++++++++++++++++++++++++++++++++++++++++++++++++++++++++++++

\begin{theorem}[Hahn-Banach\cite{brezis,TQSWRiz}]
    Soit \( E\), un espace vectoriel réel et une application \( p\colon E\to \eR\) satisfaisant
    \begin{enumerate}
        \item
            \( p(\lambda x)=\lambda p(x)\) pour tout \( x\in E\) et pour tout \( \lambda>0\),
        \item
            \( p(x+y)\leq p(x)+p(y)\) pour tout \( x,y\in E\).
    \end{enumerate}
    Soit de plus \( G\subset E\) un sous-espace vectoriel muni d'une application \( g\colon G\to \eR\) vérifiant \( g(x)\leq p(x)\) pour tout \( x\in G\). Alors il existe \( f\in\aL(E,\eR)\) telle que \( f(x)=g(x)\) pour tout \( x\in G\) et \( f(x)\leq p(x)\) pour tout \( x\in E\).
\end{theorem}
\index{théorème!Hahn-Banach}

\begin{proof}
    Si \( h\) une application linéaire définie sur un sous-espace de \( E\), nous notons \( D_h\) ledit sous-espace. 
    
    \begin{subproof}
    \item[Un ensemble inductif]

        Nous considérons \( P\), l'ensemble des fonctions linéaires suivant 
        \begin{equation}
            P=\Big\{  h\colon D_h\to \eR\tq
            \begin{cases}
                G\subset D_h\\
                h(x)=g(x)&\forall x\in G\\
                h(x)\leq p(x)&\forall x\in D_h
            \end{cases}
        \Big\}
        \end{equation}
        Cet ensemble est non vide parce que \( g\) est dedans. Nous le munissons de la relation d'ordre \( h_1\leq h_2\) si et seulement si \( D_{h_1}\subset D_{h_2}\) et \( h_2\) prolonge \( h_1\). Nous montrons à présent que \( P\) est un ensemble inductif. Soit un sous-ensemble totalement ordonné \( Q\subset P\); nous définissons une fonction \( h\) de la façon suivante. D'abord \( D_h=\sup_{l\in Q}D_l\) et ensuite
        \begin{equation}
            \begin{aligned}
                h\colon D_h&\to \eR \\
                x&\mapsto l(x)&\text{si \( x\in D_l\)}
            \end{aligned}
        \end{equation}
        Cela est bien définit parce que si \( x\in D_l\cap D_{l'}\) alors, vu que \( Q\) on a obligatoirement \( D_l\subset D_{l'}\) et \( l'\) qui prolonge \( l\) (ou le contraire) parce que \( Q\) est totalement ordonné (i.e. \( l\leq l'\) ou \( l'\leq l\)). Donc \( h\) est un majorant de \( Q\) dans \( P\) parce que \( h\geq l\) pour tout \( l\in Q\). Cela montre que \( P\) est inductif (définition \ref{DefGHDfyyz}). Le lemme de Zorn \ref{LemUEGjJBc} nous dit alors que \( P\) possède un maximum \( f\) qui va être la réponse à notre théorème.

    \item[Le support de \( f\)]

        La fonction \( f\) est dans \( P\); donc \( f(x)\leq p(x)\) pour tout \( x\in D_h\) et \( f(x)=g(x)\) pour tout \( x\in G\). Pour terminer nous devons montrer que \( D_f=E\). Supposons donc que \( D_f\neq E\) et prenons \( x_0\notin D_f\). Nous allons contredire la maximalité de \( f\) en considérant la fonction \( h\) donnée par \( D_h=D_f+\eR x_0 \) et
        \begin{equation}
            h(x+tx_0)=f(x)+t\alpha
        \end{equation}
        où \( \alpha\) est une constante que nous allons fixer plus tard.

        Nous commençons par prouver que \( f\) est dans \( P\). Nous devons prouver que
        \begin{equation}    \label{EqOIXrlFe}
            h(x+tx_0)=f(x)+t\alpha\leq p(x+tx_0)
        \end{equation}
        Pour cela nous allons commencer par fixer \( \alpha\) pour avoir les relations suivantes :
        \begin{subequations}    \label{EqMDNkcQk}
            \begin{numcases}{}
                f(x)+\alpha\leq p(x+x_0)    \label{EqDYmRWEY}\\
                f(x)-\alpha\leq p(x-x_0)
            \end{numcases}
        \end{subequations}
        pour tout \( x\in D_f\). Ces relations sont équivalentes à demander \( \alpha \) tel que
        \begin{subequations}
            \begin{numcases}{}
                \alpha\leq p(x+x_0)-f(x)\\
                \alpha\geq f(x)-p(x-x_0)
            \end{numcases}
        \end{subequations}
        Nous nous demandons donc si il existe un \( \alpha\) qui satisfasse
        \begin{equation}
            \sup_{y\in D_f}\big( f(y)-p(y-x_0) \big)\leq \alpha\leq \inf_{z\in D_f}\big( p(z+x_0)-f(z) \big).
        \end{equation}
        Ou encore nous devons prouver que pour tout \( y,z\in D_f\),
        \begin{equation}
            p(z+x_0)-f(x)\geq f(y)-p(y-x_0)\geq 0.
        \end{equation}
        Par les propriétés de \( p\) et de \( f\),
        \begin{equation}
        p(z+x_0)+p(y-x_0)-f(z)-f(y)\geq p(z+y)-f(z+y)\geq 0.
        \end{equation}
        La dernière inégalité est le fait que \( f\in P\). Un choix de \( \alpha\) donnant les inéquations \eqref{EqMDNkcQk} est donc possible.
        
        À partir des inéquations \eqref{EqMDNkcQk} nous obtenons la relation \eqref{EqOIXrlFe} de la façon suivante. Si \( t>0\) nous multiplions l'équation \eqref{EqDYmRWEY} par \( t\) :
        \begin{equation}
            tf(x)+t\alpha\leq tp(x+x_0).
        \end{equation}
        Et nous écrivons cette relation avec \( x/t\) au lieu de \( x \) en tenant compte de la linéarité de \( f\) :
        \begin{equation}
            f(x)+t\alpha\leq  tp\big( \frac{ x }{ t }+x_0 \big)=p(x+tx_0).
        \end{equation}
        Avec \( t<0\), c'est similaire, en faisant attention au sens des inégalités.
        
        Nous avons donc construit \( h\colon D_h\to \eR\) avec \( h\in P\), \( D_f\subset D_h\) et \( h(x)=f(x)\) pour tout \( x\in D_f\). Cela pour dire que \( h>f\), ce qui contredit la maximalité de \( f\). Le domaine de \( f\) est donc \( E\) tout entier.

        La fonction \( f\) est donc une fonction qui remplit les conditions.

    \end{subproof}
\end{proof}

\begin{definition}  \label{DefPJokvAa}
    Un espace topologique est \defe{localement convexe}{convexité!locale} si tout point possède un système fondamental de voisinages formé de convexes.
\end{definition}
%TODO : il faudrait parler de système fondamental de voisinages.

\begin{definition}[Hyperplan qui sépare]
    Soit \( E\) un espace vectoriel topologique ainsi que \( A\), \( B\) des sous-ensembles de \( E\). Nous disons que l'hyperplan d'équation \( f=\alpha\) \defe{sépare au sens large}{hyperplan!séparer!au sens large} les parties \( A\) et \( B\) si \( f(x)\leq \alpha\) pour tout \( x\in A\) et \( f(x)\geq \alpha\) pour tout \( x\in B\).

    La séparation est \defe{au sens strict}{hyperplan!sépare!au sens strict} si il existe \( \epsilon>0\) tel que 
    \begin{subequations}
        \begin{align}
            f(x)\leq \alpha-\epsilon&&\text{pour tout \( x\in A\)}\\
            f(x)\geq \alpha+\epsilon&&\text{pour tout \( x\in B\)}.
        \end{align}
    \end{subequations}
\end{definition}

\begin{theorem}[Haha-Banach, première forme géométrique\cite{TQSWRiz}]  \label{ThoSAJjdZc}
    Soit \( E\) un espace vectoriel topologique et \( A\), \( B\) deux convexes non vides disjoints de \( E\). Si \( A\) est ouvert, il existe un hyperplan fermé qui sépare \( A\) et \( B\) au sens large.
\end{theorem}

\begin{theorem}[Hahn-Banach, seconde forme géométrique] \label{ThoACuKgtW}
    Soit un espace vectoriel topologique localement convexe\footnote{Définition \ref{DefPJokvAa}.} ainsi que des convexes non vides disjoints \( A\) et \( B\) tels que \( A\) soit compact et \( B\) soit fermé. Alors il existe un hyperplan fermé qui sépare strictement \( A\) et \( B\).
\end{theorem}

\begin{proof}
    Vu que \( B\) est fermé, \( A\) est dans l'ouvert \( E\setminus B\). Donc si \( a\in A\), il existe un voisinage ouvert convexe de \( a\) inclus à \( A\). Soit \( U_a\) un voisinage ouvert et convexe de \( 0\) tel que \( (a+U_a)\cap B=\emptyset\).

    Vu que la fonction \( (x,y)\mapsto x+y\) est continue, nous pouvons trouver un ouvert convexe \( V_a\) tel que \( V_a+V_a\subset U_a\). L'ensemble \( a+V_a\) est alors un voisinage ouvert de \( a\) et bien entendu \( \bigcup_a(a+V_a)\) recouvre \( A\) qui est compact. Nous en extrayons un sous-recouvrement fini, c'est à dire que nous considérons \( a_1,\ldots, a_n\in A\) tels que
    \begin{equation}
        A\subset \bigcup_{i=1}^n(a_i+V_{a_i}).
    \end{equation}
    Nous posons alors 
    \begin{equation}
        V=\bigcap_{i=1}^nV_{a_i}.
    \end{equation}
    Cet ensemble est non vide parce et il contient un voisinage de zéro parce que c'est une intersection finie de voisinages de zéro. Soit \( x\in A+V\). Il existe \( i\) tel que 
    \begin{equation}
        x\in a_i+U_{a_i}+V\subset a_i+V_{a_i}+V_{a_i}\subset a_i+U_{a_i}\subset E\setminus B.
    \end{equation}
    Donc \( (A+V)\cap B=\emptyset\). L'ensemble \( A+V\) est alors un ouvert convexe disjoint de \( B\). Par la première forme géométrique du théorème de Hahn-Banach \ref{ThoSAJjdZc} nous avons un hyperplan qui sépare \( A+V\) de \( B\) au sens large : il existe \( f\in E'\setminus\{ 0 \}\) tel que \( f(a)+f(v)\leq f(b)\) pour tout \( a\in A\), \( v\in V\) et \( b\in B\). 
    
    Il suffit donc de trouver un \( v\in V\) tel que \( f(v)\neq 0\) pour avoir la séparation au sens strict. Cela est facile : \( V\) étant un voisinage de zéro et \( f\) étant linéaire, si elle était nulle sur \( V\), elle serait nulle sur \( E\).
\end{proof}

%+++++++++++++++++++++++++++++++++++++++++++++++++++++++++++++++++++++++++++++++++++++++++++++++++++++++++++++++++++++++++++ 
\section{Théorème de Tietze}
%+++++++++++++++++++++++++++++++++++++++++++++++++++++++++++++++++++++++++++++++++++++++++++++++++++++++++++++++++++++++++++

\begin{definition}
Si \( E\) et \( F\) sont des espaces normés, une application \( f\colon E\to F\) est \defe{presque surjective}{presque!surjective} si il existe \( \alpha\in\mathopen] 0 , 1 \mathclose[\) et \( C>0\) tels que pour tout \( y\in \overline{ B_F(0,1) }\), il existe \( x\in\overline{ B_E(0,C) }\) tel que \( \| y-f(x) \|\leq \alpha\).
\end{definition}

\begin{lemma}[\cite{KXjFWKA}]   \label{LemBQLooRXhJzK}
    Soient \( E\) et \( F\) des espaces de Banach et \( f\in\cL(E,F)\)\footnote{L'ensemble des applications linéaires continues}. Si \( f\) est presque surjective, alors
    \begin{enumerate}
        \item   \label{ItemTSOooYkxvBui}
            \( f\) est surjective
        \item\label{ItemTSOooYkxvBuii}
            pour tout \( y\in \overline{ B_F(0,1) }\), il existe \( x\in\overline{ B_E(0,\frac{ C }{ 1-\alpha }) }\) tel que \( y=f(x)\).
    \end{enumerate}
\end{lemma}
Le point \ref{ItemTSOooYkxvBuii} est une précision du point \ref{ItemTSOooYkxvBui} : il dit quelle est la taille de la boule de \( E\) nécessaire à obtenir la boule unité dans \( F\).

\begin{proof}
    Soit \( y\in \overline{ B_F(0,1) }\). Nous allons construire \( x\in B\big( 0,\frac{ C }{ 1-\alpha } \big)\) qui donne \( f(x)=y\). Ce \( x\) sera la limite d'une série que nous allons construire par récurrence. Pour \( n=1\) nous utilisons la presque surjectivité pour considérer \( x_1\in\overline{ B_E(0,C) } \) tel que \( \| y-f(x_1) \|\leq \alpha\). Ensuite nous considérons la récurrence
    \begin{equation}
        x_n\in \overline{ B_E(0,C) }
    \end{equation}
    tel que
    \begin{equation}
        \big\| y-\sum_{i=1}^n\alpha^{i-1}f(x_i) \big\|\leq \alpha^n
    \end{equation}
    Pour montrer que cela existe nous supposons que la série est déjà construire jusqu'à \( n>1\) :
    \begin{equation}
        \frac{1}{ \alpha^n }\Big( y-\sum_{i=1}^n\alpha^{i-1}f(x_i) \Big)\in \overline{ B_F(0,1) }
    \end{equation}
    À partir de là, par presque surjectivité il existe un \( x_{n+1}\in \overline{ B_E(0,C) }\) tel que
    \begin{equation}
        \big\| \frac{ y-\sum_{i=1}^n\alpha^{i-1}f(x_i) }{ \alpha^n }-f(x_{n+1}) \big\|\leq \alpha.
    \end{equation}
    En multipliant par \( \alpha^{n}\), le terme \( \alpha^nf(x_{n+1})\) s'intègre bien dans la somme :
    \begin{equation}
        \big\| y=\sum_{i=1}^{n+1}\alpha^{i-1}f(x_i) \big\|\leq \alpha^{n+1}.
    \end{equation}
    Nous nous intéressons à une éventuelle limite à la somme des \( \alpha^{n-1}x_n\). D'abord nous avons la majoration \( \| \alpha^{n-1}x_n \|\leq \alpha^{n-1}C\), et vu que par la définition de la presque surjectivité \( 0<\alpha<1\), la série
    \begin{equation}
        \sum_{n=1}^{\infty}\alpha^{n-1}x_n
    \end{equation}
    converge absolument\footnote{Définition \ref{DefVFUIXwU}.} parce que la suite des normes est une suite géométrique de raison \( \alpha\). Vu que \( E\) est de Banach, la convergence absolue implique la convergence simple (la suite des sommes partielles est de Cauchy et Banach est complet). Nous posons
    \begin{equation}
        x=\sum_{n=1}^{\infty}\alpha^{n-1}x_n\in E,
    \end{equation}
    et en termes de normes, ça vérifie
    \begin{equation}
        \| x \|\leq\sum_{n=1}^{\infty}\alpha^{n-1}\| x_n \|\leq C\sum_{n=1}^{\infty}\alpha^{n-1}=\frac{ C }{ 1-\alpha }.
    \end{equation}
    Donc c'est bon pour avoir \( x\in B\big( 0,\frac{ C }{ 1-\alpha } \big)\). Nous devons encore vérifier que \( y=f(x)\). Pour cela nous remarquons que
    \begin{equation}
        \| y-f\Big( \sum_{n=1}^N\alpha^{n-1}x_n \Big) \|\leq \alpha^N.
    \end{equation}
    Nous pouvons prendre la limite \( N\to \infty\) et permuter \( f\) avec la limite (par continuité de \( f\)). Vu que \( 0<\alpha<1\) nous avons
    \begin{equation}
        \| y-f(x) \|=0.
    \end{equation}
\end{proof}

\begin{theorem}[Tietze\cite{KXjFWKA,ytMOpe}]   \label{ThoFFQooGvcLzJ}
    Soit un espace métrique \( (X,d)\) et un fermé \( Y\subset X\). Soit \( g_0\in C^0(Y,\eR)\). Alors \( g_0\) admet un prolongement continu sur \( X\).
\end{theorem}

\begin{proof}
    Soit l'opération de restriction
    \begin{equation}
        \begin{aligned}
            T\colon (C^0_b(X,\eR),\| . \|_{\infty})&\to (C^0_b(Y,\eR),\| . \|_{\infty}) \\
            f&\mapsto f|_Y. 
        \end{aligned}
    \end{equation}
    L'application \( T\) est évidemment linéaire. Elle est de plus borné pour la norme opérateur usuelle donnée par la proposition \ref{PropNormeAppLineaire} parce que \( \| T(f) \|\leq \| f \|<\infty\). L'application \( T\) est alors continue par la proposition \ref{PropmEJjLE}.

    \begin{subproof}
    \item[Presque surjection]

    Soit \( g\in C^0_b(Y,\eR)\) avec \( \| g \|_{\infty}\leq 1\). Nous posons
    \begin{subequations}
        \begin{align}
            Y^+=\{ x\in Y\tq \frac{1}{ 3 }\leq g(x)\leq 1 \}\\
            Y^-=\{ x\in Y\tq -1\leq g(x)\leq -\frac{1}{ 3 } \}.
        \end{align}
    \end{subequations}
    Nous considérons alors
    \begin{equation}
        \begin{aligned}
            f\colon X&\to \eR \\
            x&\mapsto \frac{1}{ 3 }\frac{ d(x,Y^-)-d(x,Y^+) }{ d(x,Y^-)+d(x,Y^+) } 
        \end{aligned}
    \end{equation}
    Vu qu'en valeur absolue le dénominateur est plus grand que le numérateur nous avons \( \| f \|_{\infty}\leq \frac{1}{ 3 }\). Notons que
    \begin{itemize}
        \item Si \( x\in Y^+\) alors \( f(x)=\frac{1}{ 3 }\) et \( g(x)\in\mathopen[ \frac{1}{ 3 } , 1 \mathclose]\);
        \item Si \( x\in Y^-\) alors \( f(x)=-\frac{1}{ 3 }\) et \( g(x)\in\mathopen[-1,-\frac{1}{ 3 } \mathclose]\);
        \item Si \( x\) n'est ni dans \( Y^+\) ni dans \( Y^-\) alors nous avons\footnote{Nous rappelons que \( \| g \|=1\), donc \( g(x)\) est forcément ente \( -1\) et \( 1\).} \( g(x)\in\mathopen[ -\frac{1}{ 3 } , \frac{1}{ 3 } \mathclose]\) et donc \( \big| f(x)-g(x) \big|\leq \big| f(x) \big|+\big| g(x) \big|\leq \frac{ 2 }{ 3 }\).
    \end{itemize}
    Dans les deux cas nous avons \( \big| f(x)-g(x) \big|\in\mathopen[ 0 , \frac{ 2 }{ 3 } \mathclose]\) pour tout \( x\in X\). Cela prouve que
    \begin{equation}
        \| T(f)-g \|_{Y,\infty}\leq \frac{ 2 }{ 3 }.
    \end{equation}
    En résumé nous avons pris \( g\) dans la boule \( \overline{ B(0,1) }\) de \( \big( C^0_b(Y,\eR), \| . \|_{\infty} \big)\) et nous avons construit une fonction \( f\) dans la boule \( \overline{ B(0,\frac{1}{ 3 }) }\) de \( \big( C^0_b(X,\eR),\| . \|_{\infty} \big)\) telle que \( \| T(f)-g \|_{\infty}\leq \frac{ 2 }{ 3 }\). L'application \( T\) est donc une presque surjection avec \( \alpha=\frac{1}{ 3 }\) et \( C=\frac{ 2 }{ 3 }\).

\item[Prolongement dans les boules unité fermées]

    La proposition \ref{PropSYMEZGU} nous assure que les espaces \( C^0_b(X,\eR)\) et \( C_b^0(Y,\eR)\) sont de Banach (complets), et le lemme \ref{LemBQLooRXhJzK} nous dit alors que \( T\) est surjective et que pour tout \( g\in\overline{ B(0,1) }\), il existe 
    \begin{equation}
        f\in\overline{ B\left( 0,\frac{ 1/3 }{ 1-\frac{ 2 }{ 3 } } \right) }=\overline{ B(0,1) }.
    \end{equation}
    telle que \( g=T(f)\).


\item[Prolongement pour les boules ouvertes]

    Jusqu'à présent nous avons montré qu'une fonction \( g\in\overline{ B(0,1) }\) admet une prolongement continu dans \( \overline{ B(0,1) }\). Nous allons montrer que si \( g\) est dans la boule ouverte \( B(0,1)\) de \( \big( C^0_b(Y,\eR),\| . \|_{\infty} \big)\) alors \( g\) admet un prolongement dans la boule ouverte \( B(0,1)\) de \( \big( C_b^0(X,\eR),\| . \|_{\infty} \big)\).

    Soit \( g\in B_{C^0_b(Y)}(0,1) \) et son prolongement \( h\in \overline{ B_{C_b^0(X)}(0,1) }\). Si \( \| h \|_{\infty}<1\) alors le résultat est vrai. Sinon nous considérons l'ensemble
    \begin{equation}
        Z=\{ x\in X\tq | h(x) |=1 \}.
    \end{equation}
    Nous avons \( Y\cap Z=\emptyset\) parce que nous avons \( h=g\) sur \( Y\) et nous avons choisi \( \| g \|_{\infty}<\infty\). Par ailleurs \( Y\) est fermé par hypothèse et \( Z\) est fermé parce que \( h\) est continue; par conséquent \( Y\cap Z\) est fermé, donc\footnote{Si vous avez l'intention de dire que \( \overline{ Y\cap Z }=\bar Y\cap\bar Z=Y\cap Z=\emptyset\), allez d'abord voir l'exemple \ref{ExBFLooUNyvbw}. Ici c'est correct parce que \( Y\) et \( Z\) sont fermés.}
    \begin{equation}
        \bar Y\cap\bar Z=Y\cap Z=\emptyset.
    \end{equation}
    Nous posons
    \begin{equation}
        \begin{aligned}
            u\colon X&\to \eR^+ \\
            x&\mapsto \frac{ d(x,Z) }{ d(x,Y)+d(x,Z) } 
        \end{aligned}
    \end{equation}
    Le dénominateur n'est pas nul parce qu'il faudrait \( d(x,Y)=d(x,Z)=0\), ce qui demanderait \( x\in \bar Y\cap\bar Z\), ce qui n'est pas possible. Nous posons \( f=uh\). Si \( x\in Y\) alors \( u(x)=1\), donc \( f\) est encore un prolongement de \( g\). De plus \( f\) est encore continue, et donc encore un bon candidat. Enfin si \( x\) est hors de \( Y\) alors \( d(x,Y)>0\) (strictement parce que \( Y\) est fermé) et donc \( 0<u(x)<1\), ce qui donne \( | f(x) |<| h(x) |\leq 1\). Donc \( \| f \|_{\infty}<1\).

    Nous avons donc trouvé qu'une fonction dans la boule ouverte \( B_{C^0_b(Y)}(0,1)\) se prolonge en une fonction dans la boule ouverte \( B_{C^0_b(X)}(0,1)\).

\item[Le cas non borné]

Soit enfin \( g_0\in C^0(Y,\eR)\). Nous allons nous ramener au cas de la boule unité ouverte en utilisant un homéomorphisme \( \phi\colon \eR\to \mathopen] -1 , 1 \mathclose[\). L'application \( g=\phi\circ g_0\) est dans la boule unité ouvert de \( C^0(Y,\eR)\) et donc admet un prolongement \( f\) dans la boule unité ouverte de \( C^0(X)\). L'application \( f_0=\phi^{-1}\circ f\) est un prolongement continu de \( g_0\).

    \end{subproof}
\end{proof}

Un homéomorphisme \( \phi\colon \eR\to \mathopen] -1 , 1 \mathclose[\) est donné par exemple par la fonction \( \phi(t)=\frac{ 2 }{ \pi }\arctan(t)\) dont le graphique est donné ci-dessous :
\begin{center}
    \input{Fig_FXVooJYAfif.pstricks}
\end{center}
