% This is part of Exercices et corrigés de CdI-1
% Copyright (c) 2011,2013
%   Laurent Claessens
% See the file fdl-1.3.txt for copying conditions.

\setcounter{CountExercice}{1}

%+++++++++++page de garde+++++++++++++++++++++++++++++++



\section{Intégrales de surface, Stokes et Green}




\setcounter{CountExercice}{0}


\noindent{\bf Exercice 6}\\

{\bf $(a)$ La suite $[k\rightarrow \f{1}{k}]$ est convergente.}\\

\noindent Nous allons montrer que cette suite converge vers $0$. Il faut donc prouver la chose suivante: 
   \begin{equation}\label{eqn1}\forall \epsilon >0 \hspace{0,3cm} \exists K_\eps \in \N \hspace{0,3cm} {\rm tq}  \hspace{0,3cm}  \forall k\geq K_\eps, \hspace{0,3cm}  |x_k-x|<\eps\end{equation}
{Remarque}: On pourrait également montrer que cette suite est {\it de Cauchy} pour prouver qu'elle est convergente sans devoir déterminer sa limite.\\

\noindent Pour prouver que (\ref{eqn1}) s'applique bien à la suite des $\f{1}{k}$ il nous faut montrer que

   \begin{equation}\label{eqn2}\forall \epsilon >0 \hspace{0,3cm} \exists K_\eps \in \N \hspace{0,3cm} {\rm tq}  \hspace{0,3cm}  \forall k\geq K_\eps,  \hspace{0,3cm} \f{1}{k}<\eps\end{equation}

\noindent Ceci est une conséquence immédiate de l'exercice précédent. On peut également le montrer de la manière suivante: à $\epsilon$ positif donné, si nous arrivons à déterminer l'indice $K_\eps$ de (\ref{eqn2}) tel que $\forall k\geq K_\eps,  \hspace{0,3cm} \f{1}{k}<\eps$, il est clair que la suite satisfait à la définition. Or, $\f{1}{k} < \eps \leftrightarrow \f{1}{\eps} <k$. Donc si nous prenons $K_\eps := \ceil\f{1}{\eps})+1$, on a bien que $\forall k\geq K_\eps$, $\f{1}{k}<\eps$, ce qui est ce qu'il fallait démontrer.


\vspace{1cm}
{\bf $(b)$ La suite $(1, \f{1}{2}, -\f{1}{3},  \f{1}{4}, -\f{1}{5}, \ldots )$ est convergente.}\\

\noindent On remarque que cette suite tend vers zéro. (Il suffit de voir que le numérateur est borné et que le dénominateur  tend vers l'infini). Si on l'écrit  sous la forme standard, on obtient:                 
              \[x_1 = 1, x_k = \f{(-1)^k}{k} \hspace{0.3cm} \forall k\geq 2\] 
Donc, ce que nous voulons voir est que $x_k \lra_{k\rightarrow  \infty} 0$, i.e.: 
   \begin{equation}\label{eqn3}\forall \epsilon >0 \hspace{0,3cm} \exists K_\eps \in \N \hspace{0,3cm} {\rm tq}                       
       \hspace{0,3cm}  \forall k\geq K_\eps,  \hspace{0,3cm} |\f{(-1)^k}{k}|<\eps\end{equation}
       
\noindent Étant donné que $|(-1)^k| = 1 \, \forall k$, il est clair que l'équation (\ref{eqn3}) est la même que l'équation (\ref{eqn2}), et donc que l'on peut affirmer que pour tout $\epsilon > 0$, il suffit de prendre $K\geq \f{1}{\epsilon}$ et la condition est satisfaite.

\noindent{\bf Exercice 7}\\

Ici il est demandé de prouver de nouvelles règles de calcul en repartant de la définition de la convergence vers l'infini:
\begin{equation}
 \label{eqnconvinfGene} x_k \lra \infty \hspace{0.3cm} {\rm si} \hspace{0.3cm}  \forall M > 0 \hspace{0.3cm} \exists K_M \in \N \hspace{0.3cm} {\rm tq} \hspace{0.3cm} \forall k \geq K_M, x_k \geq M \end{equation}
{\bf (a) $ \lim(x_k+y_k) = +\infty$.}\\

\noindent On veut voir  la chose suivante:
\begin{equation}
 \label{eqnconvinfCasA}  \forall M > 0 \hspace{0.3cm} \exists K_M \in \N \hspace{0.3cm} {\rm tq} \hspace{0.3cm} \forall k \geq K_M, x_k + y_k \geq M \end{equation}

\noindent Soit $M> 0$. Comme $x_k$ et $y_k$ convergent à l'infini, on sait que 
\[\left\{\begin{array}{c}   
         \exists K^x_M \in \N \hspace{0.3cm} {\rm tq} \hspace{0.3cm} \forall k \geq K^x_M, x_k \geq \f{M}{2}\\																		 
        \exists K^y_M \in \N \hspace{0.3cm} {\rm tq} \hspace{0.3cm} \forall k \geq K^y_M, y_k \geq \f{M}{2},																		
\end{array}\right.\]
et donc il suffit de prendre $K_M = \max(K_M^x, K_M^y)$ dans (\ref{eqnconvinfCasA}) pour s'assurer que la définition est satisfaite.


\vspace{0.5cm}
\noindent{\bf (b) $ \lim(x_ky_k) = +\infty$.}\\

\noindent On veut voir la chose suivante:
\begin{equation}
 \label{eqnconvinfprod}  \forall M > 0 \hspace{0.3cm} \exists K_M \in \N \hspace{0.3cm} {\rm tq} \hspace{0.3cm} \forall k \geq K_M, x_k  y_k \geq M \end{equation}

\noindent Soit $M> 0$. Comme $x_k$ et $y_k$ convergent à l'infini, on sait que 
\[\left\{\begin{array}{c}   
         \exists K^x_M \in \N \hspace{0.3cm} {\rm tq} \hspace{0.3cm} \forall k \geq K^x_M, x_k \geq \sqrt M\\																		 
        \exists K^y_M \in \N \hspace{0.3cm} {\rm tq} \hspace{0.3cm} \forall k \geq K^y_M, y_k \geq \sqrt M,																		
\end{array}\right.\]

\noindent et donc il suffit  de prendre  $K_M = \max(K_M^x, K_M^y)$ dans (\ref{eqnconvinfprod}) pour s'assurer que la définition est satisfaite.

\vspace{0.5cm}
\noindent{\bf (d) Soit $z_k$ une suite tendant vers un réel $a$ strictement positif. Prouvez que $\lim(x_k  z_k) = +\infty$.}\\

Le but de l'exercice est toujours le même, c'est à dire de prouver que 
\begin{equation}		\label{eqnconvinfz}
  \forall M > 0 \hspace{0.3cm} \exists K_M \in \N \hspace{0.3cm} {\rm tq} \hspace{0.3cm} \forall k \geq K_M, \;x_k  z_k \geq M 
\end{equation}

\noindent Soit $M>0$. On sait  que:

\begin{equation}
\label{eqn12}\left\{\begin{array}{l}   
        \forall \tilde{M} >0 \;\exists K^x_{\tilde{M}} \in \N \hspace{0.3cm} {\rm tq} \hspace{0.3cm} \forall k \geq K^x_{\tilde{M}},\; x_k \geq  \tilde{M} \\																		 
       \forall \epsilon >0\;\exists K^z_\eps \in \N \hspace{0.3cm} {\rm tq} \hspace{0.3cm} \forall k \geq K^z_\eps,\; |z_k-a| <\epsilon,																		
\end{array}\right.\end{equation}

\noindent Prenons un $\epsilon$ tel que $a-\epsilon>0$. Par la deuxième partie de (\ref{eqn12}) on voit qu'il existe un indice $ K^z_\eps$ tel que $ \forall k \geq K^z_\eps,\; z_k > a-\epsilon >0$.

\noindent Prenons un $\tilde{M}$ tel que $M= \tilde{M}(a-\epsilon)$. Par la première partie de (\ref{eqn12}) on voit qu'il existe un indice $ K^x_{\tilde{M}} $ tel que $\forall k \geq K^x_{\tilde{M}},\; x_k \geq  \tilde{M} $.

												
\noindent et donc il suffit  de prendre  $K_M = \max(K_{\tilde{M}}^x, K^z_\eps)$ dans (\ref{eqnconvinfz}) pour avoir que 
\[ \forall k \geq K_M, \;x_k  z_k \geq \tilde{M}(a-\epsilon)=M.\]


\noindent{\bf Exercice 8}\\

\noindent Une suite $x_k$ est bornée si $\exists N>0$ tel que $\forall k$, $|x_k| < N$.

\noindent On veut voir que $\f{x_k}{y_k}\lra 0$, i.e.

\begin{equation} 
\label{eqnconvborne}  \forall  \epsilon > 0 \hspace{0.3cm} \exists K_\epsilon \in \N \hspace{0.3cm} {\rm tq} \hspace{0.3cm} \forall k \geq K_\epsilon, \; |\f{x_k}{y_k}| < \epsilon \end{equation}

\noindent Soit $\epsilon >0$. Comme la suite $x_k$ est bornée, on a que  $|\f{x_k}{y_k}|<\f{N}{|y_k|}\; \forall k$. On utilise maintenant le fait que $y_k \lra \infty$. Prenons $M=\f{N}{\epsilon}$. On peut écrire que $\exists K_M$ tel que $\forall k \geq K_M, \; y_k \geq M=\f{N}{\epsilon}$, et donc si dans (\ref{eqnconvborne}) on prend $K_\epsilon= K_M$ on a:\[\forall k \geq K_\eps,\; \; |\f{x_k}{y_k}|<\f{N}{|y_k|}<\f{N}{N/\epsilon}=\epsilon.\]



\noindent{\bf Exercice 9}\\

\noindent Pour cet exercice, on peut utiliser les règles de calcul. Il faut faire attention que ces règles ne s'appliquent que si toutes les limites existent!

\vspace{0.5cm}
\noindent{ (a)} $x_k = \f{k+2}{k}\cos(k\pi)$\\

\noindent On voit que cette suite va dans deux directions différentes, $+1$ et $-1$ à cause du facteur $\cos(k\pi)=(-1)^k$. Elle ne converge donc pas. Pour le prouver, on peut prendre deux suites partielles de la suite $x_k$ qui convergent vers des limites différentes. 

\noindent Choisissons \[\left\{ \begin{array}{rcl} y_k &= x_{2k}&= \f{(2k)+2}{2k}(-1)^{2k}\\
 							  z_k &= x_{2k+1} &= \f{(2k+1)+2}{2k+1} (-1)^{2k+1}\end{array}\right.\]

\noindent Comme $x_k =\f{k+1}{k}= 1+\f{1}{k}$	et que $\f{1}{k} \rightarrow  0$, nous pouvons appliquer les règles de calcul et en déduire que $x_k \rightarrow  1$. On fait la même chose pour $y_k$.				  


\vspace{0.5cm}
\noindent{ (c)} $x_k = \f{k^3+k+1}{5k^3+2}$\\

\noindent Nous avons que \[\forall k, \;\;\;\;x_k =\; (\f{k^3}{k^3})\f{1+\f{1}{k} +\f{1}{k^3}}{5+\f{2}{k^3}}=\;\f{1+\f{1}{k} +\f{1}{k^3}}{5+\f{2}{k^3}} \]
Comme \[\forall k \geq 1\;\; \f{1}{k^3} \; \leq \;\f{1}{k^2}\; \leq \; \f{1}{k}\] et comme $\f{1}{k}\rightarrow 0$, nous pouvons appliquer la règle de l'étau pour voir que \[\f{1}{k^3} \rightarrow 0 \; \; \; {\rm et } \;\; \;\f{1}{k^2} \rightarrow 0.\]
En appliquant les règles de calcul à la suite $x_k$ transformée, on voit donc que $x_k \rightarrow  \f{1}{5}$.

\vspace{0.5cm}
\noindent{ (d)} $x_k = \f{k+(-1)^k}{k-(-1)^k}$\\

\noindent On peut le voir par exemple par la règle de l'étau:
\[\forall k \geq 0, \;\;\; \f{k-1}{k+1} \leq \f{k+(-1)^k}{k-(-1)^k} \leq \f{k+1}{k-1}. \]
Or, comme les deux suites qui bornent la suite $x_k$ convergent toutes les deux vers $1$, il est clair que $x_k$ converge aussi vers $1$.


\vspace{0.5cm}
\noindent{ (d)} $x_k = x_{k-1}^2\;+\;1,\, x_1=1$\\

\noindent Suite définie par récurrence. Ses premiers éléments sont \[1, \; 2, \; 5, \;  26, \; 677, \; \ldots\]
Toute  limite admissible réelle finie $l$  de cette suite doit satisfaire à \[l=l^2+1\] ce qui implique qu'elle ne peut avoir de limite réelle finie. En regardant ses premiers éléments, on remarque immédiatement qu'elle semble converger à l'infini. Nous allons le prouver en utilisant la définition.

\noindent Soit $M> 0$. On a que \[x_k \geq k \, \forall k.\] En effet (par récurrence sur $k$): il est clair que $x_1 \geq 1$. Supposons que $x_k \geq k$. Ceci implique t-il que $x_{k+1}\geq k+1$? Par définition des $x_k$, $x_{k+1} = x_k^2+1$. Par l'hypothèse de récurrence, on a donc $x_{k+1}\geq (k)^2 +1\geq k+1$ ce qui prouve le résultat. Comme la suite $y_k=k$ converge à l'infini, il en est de même pour la suite $x_k$.



\section{Continuité de fonctions réelles}


\begin{center}
\LARGE \bf
Travaux Personnels 
\end{center}

\begin{bf}
\begin{center}
BAC2 en sciences mathématiques et physiques
\end{center}
\end{bf}

{\bf Exercice 1.} Calculer les limites suivantes

\b
a) $\displaystyle \lim_{n \to \infty} \left( 1+ \frac{2}{n-4} \right)^n$

\medskip
b) 
$\displaystyle \lim_{n \to \infty} 
         \left( 1+ \frac 1n \right)^{\sqrt{n}}$

\medskip
c) $\displaystyle \lim_{x \to \infty} 
    \left( 1+ \frac \ga x \right)^x$

\medskip
d) 
$\displaystyle \lim_{x \to 0} \frac{\log \left( 1+ \ga x \right)}{x}$


\medskip
e) 
$\displaystyle \lim_{x \to \infty} 
\frac{a_0+a_1x + \dots +a_nx^n}{b_0+b_1x + \dots +b_mx^m}$
\quad où\, $a_j, b_j \in \eC$ \,et\, $n,m \ge 0$

\medskip
f) 
$\displaystyle \lim_{x \to 0} \frac{\sqrt{1-\cos x}}{x}$  




{\bf Exercice 2.} Prouver que

\medskip
a)
$\displaystyle \lim_{x \to \infty} x^{\frac 1x} = \lim_{x \to 0^+} x^x = 1$

\medskip
b)
$\displaystyle \lim_{x \to \infty} \frac{x^{\ln x}}{{\mathrm e}^x} =0$
\quad
càd ${\mathrm e}^x$ croit plus vite que $x^{\ln x}$


{\bf Exercice 3.} Prouver que
$$
\cosh 2x \,=\, \cosh^2 x + \sinh^2 x,
\qquad
\sinh 2x \,=\, 2 \sinh x \cosh x
$$


{\bf Exercice 4.} Prouver que

a)
$1 + \cos z + \cos 2z + \dots + \cos nz = \displaystyle \cos \frac{nz}{2} \cdot \frac{\sin (n+1)z/2}{\sin z/2}$

b)
$1 + \sin z + \sin 2z + \dots + \sin nz = \displaystyle \sin \frac{nz}{2} \cdot \frac{\sin (n+1)z/2}{\sin z/2}$

{\it Aide:}\;
$\displaystyle \sum_{k=0}^n 
\euler^{\sii kz} 
= 
\frac{1-\euler^{\sii (n+1)z}}{1-\euler^{\sii z}}
= \euler^{\sii nz/2} \cdot \frac{\euler^{\sii (n+1)z/2} - \euler^{-\sii (n+1)z/2}}{
\euler^{\sii z/2}-\euler^{-\sii z/2}}
$

Rappelons qu'une fonction $f \colon \mathbb{C} \supset D \to \eC$ est {\bf uniformément continue} si pour tout $\eps >0$ il existe un $\gd >0$ tel que 
$$
|x-y| < \gd \,\Longrightarrow\, |f(x)-f(y)| < \eps 
\quad \text{ pour tout }\, x,y \in D.
$$
Prouver que la fonction $f \colon \eR \to \eR$, $x \mapsto x^2$ est continue, mais n'est pas uniformément continue.


\section{Intégrales, longueur de courbes, EDO's linéaires}


\exerNico 
Soient $n,m \in \NN \cup \{0\}$.
Calculer
$$
\int_0^1 x^n (1-x)^m \,dx
\quad \text{ et } \quad
\int_{-1}^1 (1+x)^n (1-x)^m \,dx
$$

{\bf Solution:}
Posons $I_{n,m} := \int_0^1 x^n (1-x)^m \,dx$.
Intégration par partie donne
la formule récursive
$$
I_{n,m} \,=\, \frac {m}{n+1} I_{n+1,m-1}.
$$
Avec $I_{n+m,0} = \frac{1}{n+m+1}$ nous obtenons
$$
I_{n,m} \,=\, \frac{n!\,m!}{(n+m+1)!}
$$
La substitution $x := 2t-1$ fournit
$$
\int_{-1}^1 (1+x)^n (1-x)^m \,dx
\,=\, 2^{n+m+1} \int_0^1 t^n (1-t)^m \,dt \,=\,  2^{n+m+1} 
\cdot I_{n,m}. 
$$




\exerNico 
Soient $a,b >0$. 
Calculer
$$
\int_0^{\pi /2} \displaystyle \frac{d \gf}{a^2 \sin^2 \gf + b^2 \cos^2 \gf}
$$

{\bf Solution:}
$$
\,=\, \int_0^{\pi /2} \frac{1 / \cos^2 \gf}{a^2 \tan^2 \gf+b^2} d\gf \,=\, \int_0^\infty \frac {dt}{a^2t^2 + b^2} \,=\, \frac{\pi}{2ab}  
$$


\exerNico  
Calculer la longueur de l'arc de la parabole $y = x^2,\;x \in [0,b]$.

\medskip
{\bf Solution:}
$$
s \,=\, \int_0^b \sqrt{1+4x^2} \,dx \,=\, \frac b 2 \sqrt{1+4b^2}+ \frac 14 \ln \left(2b+ \sqrt{1+4b^2} \right)
$$


\exerNico  
La {\bf parabole de Neil} $\nu$ est la courbe définie par $\nu (t) = (t^2,t^3)$, pour  $t \in \eR$.

\medskip
a)
Esquisser la parabole de Neil.

\medskip
b)
Quelle est la signification du paramètre $t$?

\medskip
{\bf Solution:} $t = \tan \ga$

\medskip
c)
Calculer la longueur de l'arc 
$\left\{ \nu (t) \mid t \in [0,\tau] \right\}$.


\medskip
{\bf Solution:}
$$
s \,=\, \int_0^\tau \sqrt{4 t^2+9t^4} \,d\tau \,=\, \frac{8}{27} \left( \left(1+ \frac 94 \tau^2\right)^{3/2}-1 \right)
$$



\exerNico  
La {\bf hélice} $\gamma$ de pas $2 \pi h$ est la courbe dans $\RR^3$ définie par
$$
\gamma(t) \,=\, \left( r \cos t , r \sin t , h t \right)  .
$$


\medskip
a)
Esquisser la hélice.

\medskip
b)
Expliquer le mot ``pas''.


\medskip
c)
Calculer la longueur de l'arc sur la hélice si on fait un tour.

\medskip
{\bf Solution:} 
$\int_0^{2\pi} \sqrt{r^2+h^2} \,dt \,=\, 2 \pi \sqrt{r^2+h^2}$


\bigskip
\exerNico 
Calculer un système fondamental réel pour

\medskip
a) $y^{(4)}-y = 0$,

\medskip
b) $y^{(4)} +4y'' +4y = 0$,

\medskip
c) $y^{(4)} -2y^{(3)} +5y'' = 0$.


\bigskip
{\bf Solution:}

\medskip
a) ${\rm e}^x, {\rm e}^{-x}, \cos x, \sin x$

\medskip
b) $\cos \sqrt{2} x, x \cos \sqrt{2}x, \sin \sqrt{2}x, x \sin \sqrt{2}x$

\medskip
c)
$1, x, {\rm e}^x \cos 2x, {\rm e}^x \sin 2x$



\bigskip
\exerNico 
Déterminer une solution particulière de l'équation
$y''+y=q$ pour

\medskip
a) $q = x^3$,

\medskip
b) $q = \sinh x$,

\medskip
c) $q = 1/\sin x$.
 

\bigskip
{\bf Solution:}

\medskip
a) $x^3 - 6 x$

\medskip
b) $\frac 12 \sinh x$

\medskip
c) $\sin x \cdot \ln |\sin x| - x \cos x$


\bigskip
\exerNico  
L'équation différentielle $m \ddot y = mg - k\dot y$ 
décrit la chute d'un corps soumit
à la gravitation si la friction est proportionnelle à la vitesse (``un homme tombant de l'avion'').

\medskip
Calculer la solution avec $y(0) =0, \dot y(0) = 0$.
Calculer la ``vitesse finale'' $v_\infty = \displaystyle \lim_{t \to \infty} \dot y (t)$.



\bigskip
{\bf Solution:}

\medskip
L'équation homogène $\ddot y + k/m \cdot y = 0$
possède les solutions $c_1+c_2 {\rm e}^{-k/m \cdot t}$,
où $c_1, c_2 \in \RR$.
 
L'équation inhomogène $\ddot y + k/m \cdot y = g$
possède comme solution particulière une fonction lineaire, càd 
$y_p = (mg/k)t)$.
En tenant compte des conditions initiales nous obtenons
$$
y(t) \,=\, \frac{mg}{k} \left( t-\frac mk (1-{\rm e}^{-k/m \cdot t})\right).
$$
En particulier, $v_\infty = mg/k$. 

 




\bigskip
\exerNico  
Regardons l'ensemble des solutions de l'équation différentielle $P({\rm D})y =0$.

Montrer l'équivalence entre les propositions suivantes :
\begin{enumerate}

\item
Pour toute solution $y$ on a $\displaystyle \lim_{t \to \infty} y(t) = 0$

\item
Pour toute racine $z$ du polynôme caractéristique on a ${\rm Re}\, z <0$.

\end{enumerate}
Dans ce cas, l'équation différentielle est appelé  \defe{asymptotiquement stable}{asymptotiquement stable}.

\bigskip
{\bf Solution:}
On a 
$\displaystyle \lim_{t \to \infty} y(t) = 0$ pour toute solution $y$ ssi c'est vrai pour tout élément d'un système fondamental.
On a $\displaystyle \lim_{t \to \infty} t^k {\rm e}^{\gl t}=0$ ssi ${\rm Re }\,\gl <0$,
d'où l'affirmation suit.






\section{Calcul de limites}

\exerNico Déterminez si les limites suivantes existent et dans
l'affirmative calculez les en utilisant, s'il y a lieu, la règle de
l'Hospital ou la règle de l'étau.
\begin{enumerate}
\item $  \lim_{x \rightarrow  +\infty} \frac{x+1}{x^2+2} $
\item $  \lim_{x \rightarrow  +\infty} \frac{\sin(x)}{x} $
\item $  \lim_{x \rightarrow  0} \frac{\sin(x)}{x} $
\item $  \lim_{x \rightarrow  +\infty}  \frac{x ^n}{e ^x} $
\item $  \lim_{x \rightarrow  +\infty} (1 + \frac{a}{x})^x $
\item $  \lim_{x \rightarrow  0} (\frac{1}{\sin(x)} - \frac{1}{x} )$
\item $  \lim_{x \rightarrow  +\infty} \cos( 2 \pi x) $
\item $  \lim_{x \rightarrow  +\infty} \frac{1}{\sin(x)+2}(x) +\ln(x)\cos(x) $
\item $  \lim_{x \rightarrow  +\infty} \frac{ \ln(x)(\sin(x) +2)}{x} $
\item $  \lim_{x \rightarrow  +\infty} x ^\frac{1}{x} $
\end{enumerate}

\exerNico Déterminez si les limites suivantes existent et dans
l'affirmative calculez-les.
\begin{enumerate}
\item $  \lim_{x \rightarrow  0} x \sin(\frac{1}{x}) $
\item $  \lim_{x \rightarrow  0} \frac{\sin(\sin(x))}{x} $
\item $  \lim_{x \rightarrow  +\infty} (\ln(x))^\frac{1}{1 - \ln(x)}$
\end{enumerate}

\exerNico Calculez les limites suivantes:
\begin{enumerate}
\item $  \lim_{x \rightarrow  +\infty} \frac{\ln(x)}{x ^a} $
\item $  \lim_{x \rightarrow  +\infty} \frac{\ln(x)^a}{x ^b} $
\item $  \lim_{x \rightarrow  +\infty} a ^x $
\item $  \lim_{x \rightarrow  +\infty} a ^\frac{1}{x} $
\end{enumerate}
où $a$ et $b$ sont des réels positifs.
%

%

\exerNico Déterminez, pour chacune des suites suivantes, si elle converge
et dans l'affirmative calculez sa limite.
\begin{enumerate}
\item $  k \rightarrow  \cos( 2 \pi k) $
\item $  k \rightarrow  \cos(\frac{\pi}{3} k) $
\item $  k \rightarrow  k(a ^\frac{1}{k} -1 ) $
\end{enumerate}
où $a$ est une réel.\\



\exerNico Calculez  les limites suivantes si elles existent.
\begin{enumerate}
\item $  \lim_{x \rightarrow  +\infty} \cos x $
\item $  \lim_{x \rightarrow  \pm \infty }\sqrt{2x^4+3}-x^2 $

\end{enumerate}

\exerNico Déterminez si la limite de chacune des suites suivantes
existe et dans l'affirmative calculez la.
\begin{enumerate}
\item $  \lim_{k \rightarrow  +\infty }(\frac{a k +1}{k})^k $
\item $  \lim_{k \rightarrow  +\infty}\frac{1}{\sin(\frac{\pi}{6}k)+1}(k) + \ln(k)\cos(\frac{\pi}{5}k)$
\item $  \lim_{k \rightarrow  +\infty} \frac{\ln(k)(\sin(\frac{\pi}{3}k) +1)}{k} $
\item $  \lim_{k \rightarrow  +\infty } \sqrt[3k]{k} (1 +
\frac{1}{3k})^{3k} $
\end{enumerate}
où $a$ est un réel. 

\section{Dérivabilité}



\exerNico Déterminez l'ensemble des points où les fonctions suivantes
sont continues et celui où elles sont dérivables. Prouvez soigneusement
vos résultats.
\begin{enumerate}
\item $ x \rightarrow x]$
\item $ x \rightarrow |x| $
\item $ x \rightarrow
	\left\{ \begin{array}{ll}
	\frac{1}{x} & \mbox{si } x \not= 0 \\
	0 & \mbox{sinon}
	\end{array} \right. $
\item $ x \rightarrow x^2  $
\end{enumerate}




\exerNico Étudiez la dérivabilité et la continuité
de la dérivée de chacune des fonctions suivantes:
\begin{enumerate}
\item $ x \rightarrow
\left\{ \begin{array}{ll}
0 & \mbox{si } x \not= 0 \\
1 & \mbox{sinon}
\end{array} \right.$
%
\item $ x \rightarrow
\left\{ \begin{array}{ll}
\sin(\frac{1}{x}) & \mbox{si } x \not= 0 \\
0 & \mbox{sinon}
\end{array} \right.$
%
\item $ x \rightarrow
\left\{ \begin{array}{ll}
x \sin(\frac{1}{x}) & \mbox{si } x \not= 0 \\
0 & \mbox{sinon}
\end{array} \right.$
%
\item $ x \rightarrow
\left\{ \begin{array}{ll}
x^2 \sin(\frac{1}{x}) & \mbox{si } x \not= 0 \\
0 & \mbox{sinon}
\end{array} \right.$
\end{enumerate}

Le but de cet exercice est aussi d'exhiber des exemples illustrant les
différents types de comportements possibles, relativement à la
continuité et la dérivabilité, d'une fonction en un point.

\exerNico Étudiez la dérivabilité et la continuité
de la dérivée de chacune des fonctions suivantes:
\begin{enumerate}
\item $ x \rightarrow
\left\{ \begin{array}{ll}
\frac{2x+a}{1+e^{\frac{1}{x}}} & \mbox{si } x \not= 0 \\
0 & \mbox{sinon}
\end{array} \right.$
%
\item $ x \rightarrow
\left\{ \begin{array}{ll}
\frac{\sin(x)}{x} & \mbox{si } x \not= 0 \\
1 & \mbox{sinon}
\end{array} \right.$
%
\item $ x \rightarrow
\left\{ \begin{array}{ll}
e^{\frac{-1}{x}} & \mbox{si } x > 0 \\
0 & \mbox{sinon}
\end{array} \right.$
%
\item $ [-\frac{1}{2}, \frac{1}{2}] \rightarrow \R: x \rightarrow
\left\{ \begin{array}{ll}
(\frac{\sin(2x)}{x})^{x+1} & \mbox{si } x \not= 0 \\
1 & \mbox{sinon}
\end{array} \right.$
\end{enumerate}
où $a$ et $b$ sont des réels.


 \exerNico Considérons la fonction
$$f:\mathbb{R}\rightarrow\mathbb{R}:x\mapsto f(x)=\left\{
\begin{array}{ll}
x&\text{si }x\text{ est rationnel}\\
0&\text{si }x\text{ est irrationnel}
\end{array}
\right.$$

Vérifiez que $f$ est continue en $0$ mais n'est ni dérivable à  gauche ni dérivable à droite en
$0$.

\exerNico 
\begin{enumerate}
\item Soit $(X,d)$ un espace métrique et $f \colon (X,d) \to \RR$ une application continue.
Montrer que l'ensemble $$\left\{ x \mid f(x) = 0 \right\}$$ est fermé.

\item Soit $f \colon \RR \to \RR$ une application continue.
Montrer que l'ensemble 
$$
\{ x \in \RR \mid f(x) = x\}
$$
des points fixes de $f$ est fermé.

\end{enumerate}

\exerNico  Soit $A$ un sous ensemble de l'espace métrique $(X,d)$.
Montrer que la fonction
$$
\dist_A \colon X \to \RR,
\quad x \mapsto \inf_{a \in A} d(a,x)
$$ 
est continue.


\exerNico  Soient $(X,d_X)$, $(Y,d_Y)$ deux espaces métriques.
Une application $f \colon X \to Y$ est {\bf Lipschitzienne}
s'il existe une constante $L \ge 0$ telle que
$$
d_Y \bigl( f(x), f(x') \bigr) \,\le\, L \,d_X (x,x') 
\quad \text{ pour tout } x,x' \in X.
$$
Dans ce cas, on dit que $f$ est {\bf $L$-Lipschitzienne}.


\begin{enumerate}
\item
Montrer qu'une application Lipschitzienne est continue.
\item Montrer qu'une application $f \colon \RR \to \RR$, $x \mapsto ax+b$
est Lipschitzienne.
Quelle est la plus petite constante $L$ qui convienne?

\item Montrer que les fonctions $z \mapsto |z|$, 
$z \mapsto \overline z$,
$z \mapsto {\rm Re\,} z$ et $z \mapsto {\rm Im\,} z$ 
de $\eC$ dans $\eR$ sont Lipschitziennes.
Quelle sont les plus petites constantes $L$ qui conviennent?
\item Montrer que la fonction $\dist_A \colon X \to \RR$ de l'Exercice~13 est Lipschitzienne.

\end{enumerate}

