% This is part of the Exercices et corrigés de CdI-2.
% Copyright (C) 2008, 2009
%   Laurent Claessens
% See the file fdl-1.3.txt for copying conditions.


\begin{corrige}{_I-3-9}

Utilisons le critère d'Abel sur tout compact. Lorsque $x$ est restreint à parcourir un compact, nous pouvons borner
\begin{equation}
	\int_0^T\big[ \sin(xt)+\cos(xt) \big]dt
\end{equation}
de façon indépendante de $T$ et $x$. La fonction $\psi(x,t)=1/(1+t^2)$ vérifie ce qu'il faut pour Abel. La convergence uniforme sur tout compact implique la continuité de $F$ sur $\eR$.

Pour la dérivabilité, nous regardons l'intégrale de la dérivée :
\begin{equation}
	G(x)=-\int_0^{\infty}t\frac{ \sin(xt)-\cos(xt) }{ t^2+1 }dt.
\end{equation}
Sur chaque compact de $]0,\infty[$, le critère d'Abel donne la convergence uniforme. Donc $F$ est dérivable sur cet ensemble, et sa dérivée y vaut $G$.

\end{corrige}
