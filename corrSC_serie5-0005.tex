\begin{corrige}{SC_serie5-0005}

	Quelques éléments de technique
	\begin{itemize}
		\item 
			Pour définir la fonction, vu que le test d'égalité ne semble pas existe dans Matlab, il ne faut pas dire que la fonction vaut $x^2-4x$ si $\ln(x)-x+2<0$ (strict) et $\big( \ln(x)+2 \big)^2$ sinon.
		\item
			Pour tracer, la procédure habituelle serait de faire $Y=f(X)$ après avoir définit un vecteur d'abscisses $X$. Hélas, $f$ ne s'applique pas bien à un vecteur (à cause du fait que $x$ arrive dans un \verb+if+). Il faut donc faire à la main le passage de composante à composante. C'est à cela que sert la boucle \verb+for+.
	\end{itemize}
	
\lstinputlisting{SC_exo_5-5.m}

\end{corrige}
