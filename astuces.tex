% This is part of the Exercices et corrigés de CdI-2.
% Copyright (C) 2008, 2009, 2012
%   Laurent Claessens
% See the file fdl-1.3.txt for copying conditions.

Le but de ce chapitre est de répondre de façon plus détaillé aux questions qui ont été souvent posées durant les séances d'exercices.

%+++++++++++++++++++++++++++++++++++++++++++++++++++++++++++++++++++++++++++++++++++++++++++++++++++++++++++++++++++++++++++
                    \section{Le coup du compact}
%+++++++++++++++++++++++++++++++++++++++++++++++++++++++++++++++++++++++++++++++++++++++++++++++++++++++++++++++++++++++++++

Dans les sections qui suivent, nous verrons des fonctions définies par toute une série de processus de limite (suites, séries, intégrales). Une des questions centrales est de savoir si la fonction limite est continue, dérivable, intégrale, etc. étant donné que les fonctions sont continues.

Pour cela, nous inventons le concept de \emph{convergence uniforme}. Si la limite (série, intégrale) est uniforme, alors la fonction limite sera continue. Il arrive qu'une limite ne soit pas uniforme sur un intervalle ouvert $]0,1]$, et que nous voulions quand même prouver la continuité sur cet intervalle. C'est à cela que sert la notion de convergence uniforme \emph{sur tout compact}. En effet, la notion de continuité est une notion locale : savoir ce qu'il se passe dans un petit voisinage autour de $x$ est suffisant pour savoir la continuité en $x$ (idem pour sa dérivée).

Si nous avons uniforme convergence sur tout compact de $]0,1]$, mais pas uniforme convergence sur cet intervalle, la limite sera quand même continue sur $\mathopen] 0 , 1 \mathclose]$. En effet, si $x\in]0,1]$, il existe un ouvert autour de $x$ contenu dans un compact contenu dans $]0,1]$. L'uniforme convergence sur ce compact suffit à prouver la continuité en $x$.

Déduire la continuité sur un ouvert à partir de l'uniforme convergence sur tout compact de l'ouvert est appelé faire le \defe{coup du compact}{Compact!le coup du}.

%+++++++++++++++++++++++++++++++++++++++++++++++++++++++++++++++++++++++++++++++++++++++++++++++++++++++++++++++++++++++++++
\section{Vitesses de $x^{\alpha}$, de l'exponentielle et du logarithme}
%+++++++++++++++++++++++++++++++++++++++++++++++++++++++++++++++++++++++++++++++++++++++++++++++++++++++++++++++++++++++++++

Une astuce\label{PgAstuceLnPoly} usuelle qu'il faut savoir est que $\forall \alpha>0$, $\exists N$ tel que $n>N\Rightarrow \ln(n)\leq n^{\alpha}$. En effet, nous avons
\begin{equation}
    \lim_{x\to\infty} \frac{ x^{\alpha} }{ \ln(x) }=\lim_{x\to\infty} \frac{ \alpha x^{\alpha-1} }{ 1/x }=\lim_{x\to\infty} \alpha x^{\alpha}=\infty
\end{equation}
quand $\alpha>0$. Cela tient également lorsque nous considérons $\ln(x)^p$ au lieu de $\ln(x)$. De cela, nous disons que le logarithme croit moins vite que n'importe quel polynôme. Cela est également démontré à la page 135 du cours de première année.

Dans le même ordre d'idées, l'exponentielle croit plus vite que tout polynôme, et plus vite que que logarithme :
\begin{equation}        \label{EqExpDecrtPlusVite}
    \lim_{t\to\infty} e^{-t}(\ln t)^{n}t^{\alpha}=0
\end{equation}
pour tout $n$ et pour tout $\alpha$.

%+++++++++++++++++++++++++++++++++++++++++++++++++++++++++++++++++++++++++++++++++++++++++++++++++++++++++++++++++++++++++++
\section{Remarque : Abel et \texorpdfstring{$\sin(xt)$}{sin(xt)}}
%+++++++++++++++++++++++++++++++++++++++++++++++++++++++++++++++++++++++++++++++++++++++++++++++++++++++++++++++++++++++++++

Étant donné que la fonction sinus est bornée, il est tentant de l'utiliser comme $\varphi$ dans le critère d'Abel (théorème \ref{ThoAbelIntUnif}). Hélas,
\begin{equation}
    \int_0^T\sin(xt)=-\frac{ 1 }{ x }\big( \cos(xT)-\cos(x) \big),
\end{equation}
qui n'est pas bornée pour tout $x$ ! Poser $\varphi(x,t)=\sin(xt)$ \emph{ne fonctionne pas} pour assurer la convergence uniforme sur un intervalle qui contient des $x$ arbitrairement proches de $0$. Le critère d'Abel avec $\varphi(x,t)=\sin(xt)$ ne permet que de conclure à l'uniforme convergence \emph{sur tout compact} ne contenant pas $0$. Cela est toutefois souvent suffisant pour étudier la continuité ou la dérivabilité en se servant du fameux coup du compact.

%+++++++++++++++++++++++++++++++++++++++++++++++++++++++++++++++++++++++++++++++++++++++++++++++++++++++++++++++++++++++++++
                    \section{Formes différentielles}
%+++++++++++++++++++++++++++++++++++++++++++++++++++++++++++++++++++++++++++++++++++++++++++++++++++++++++++++++++++++++++++
\label{SecFormDiffRappel}

Petits rappels affin de résoudre et comprendre l'exercice \ref{exo_I-3-3}. Une \defe{forme}{forme} sur un espace vectoriel $V$ est une application linéaire $\omega\colon V\to \eR$. Si $D$ est un ouvert dans $\eR^n$, une \defe{forme différentielle}{forme!différentielle} sur $D$ est une application $\omega\colon D\to (\eR^n)^*$, c'est à dire, en chaque point $x\in D$, une forme $\omega_x\colon \eR^n\to \eR$.

Un exemple type de forme différentielle est la différentielle d'une fonction $f\colon D\to \eR$. En effet, la différentielle d'une telle fonction est l'application linéaire
\begin{equation}
    \begin{aligned}
        df\colon \eR^n&\to \eR \\
        v&\mapsto \frac{ \partial f }{ \partial x }v_x+\frac{ \partial f }{ \partial y }v_y. 
    \end{aligned}
\end{equation}
La forme différentielle $\omega$ est \defe{exacte}{forme!différentielle!exacte} si il existe une fonction $f$ telle que $\omega=df$; elle est dite \defe{fermée}{forme!différentielle!fermée} si $d\omega=0$.

Soit $D\subset\eR^n$. Par définition de la différentielle d'une $1$-forme, nous avons une formule de Leibnitz
\begin{equation}
    d(f\omega)=df\wedge\omega+fd\omega.
\end{equation}
En particulier,
\begin{equation}
    d(fdx)=df\wedge dx+f\underbrace{d(dx)}_{=0}=\frac{ \partial f }{ \partial x }dx\underbrace{dx\wedge dx}_{=0}+\frac{ \partial f }{ \partial y }dy\wedge dx. S
\end{equation}
Une forme différentielle $\omega$ continue de degré $1$ sur $D$ est \defe{exacte}{forme!différentielle!exacte} si il existe $F\colon D\to \eR$ telle que $\omega=dF$. On dit que la forme est \defe{fermée}{forme!différentielle!fermée} si $d\omega=0$. Dire que la forme différentielle $\omega=fdx+gdy$ est fermée, c'est dire que
\begin{equation}
    \frac{ \partial g }{ \partial x }=\frac{ \partial f }{ \partial y }.
\end{equation}

Si $F\colon \eR^2\to \eR$ est une fonction $C^2$, sa différentielle est la forme
\begin{equation}
    dF=\frac{ \partial F }{ \partial x }dx+\frac{ \partial F }{ \partial y }dy.
\end{equation}
Si nous nommons $f$ et $g$ les fonctions $\partial_xF$ et $\partial_yF$, nous avons donc
\begin{equation}
    Df=fdx+gdy,
\end{equation}
qui vérifie
\begin{equation}
    \partial_yf=\partial_xg,
\end{equation}
parce que $\frac{ \partial f }{ \partial y }=\frac{ \partial^2F  }{ \partial x\partial y }=\frac{ \partial^2F  }{ \partial y\partial x }=\frac{ \partial g }{ \partial x }$. Ce que nous avons donc prouvé, c'est que 
\begin{lemma}
Si $fdx+gdy$ est la différentielle d'une fonction de classe $C^2$, alors $\partial_yf=\partial_xg$.
\end{lemma}

\begin{theorem}[Théorème, page 574 du cours de première]        \label{ThoFermeExactFormRappel}
Supposons que $D\subset\eR^n$ soit un ouvert simplement connexe. Alors toute forme différentielle de degré $1$ et de classe $C^1$ sur $D$ qui est fermée est exacte.
\end{theorem}

%+++++++++++++++++++++++++++++++++++++++++++++++++++++++++++++++++++++++++++++++++++++++++++++++++++++++++++++++++++++++++++
                    \section{Que faire avec \texorpdfstring{$f(z)dz=g(t)dt$}{fzdz} ?}
%+++++++++++++++++++++++++++++++++++++++++++++++++++++++++++++++++++++++++++++++++++++++++++++++++++++++++++++++++++++++++++
\label{SecFairedzdt}

Dans de nombreux exercices d'équations différentielles, nous tombons sur $u'=f(t)$, et nous faisons formellement
\begin{equation}
    \begin{aligned}[]
        \frac{ du }{ dt }&=f(t) &\Rightarrow    &&du=f(t)dt.,
    \end{aligned}
\end{equation}
et ensuite, il y a la formule un peu magique
\begin{equation}
    u-u_0=\int_{t_0}^tf(t)dt.
\end{equation}
Voyons ce qu'il en est. Tout d'abord, il faut comprendre ce que signifie la formule
\begin{equation}        \label{EqDiffAstufzdz}
    f(z)dz=g(t)dt.
\end{equation}
Il s'agit d'une égalité entre deux formes différentielles sur $\eR$ où $z$ est une fonction de $t$.  Étant donné que $z$ est une fonction de $t$, il faut voir $dz$ comme la différentielle de cette fonction. La différentielle d'une fonction à une variable est donné par la dérivée :
\begin{equation}
    dz_t=z'(t)dt
\end{equation}
Écrire l'équation \eqref{EqDiffAstufzdz} pour chaque $t$ revient donc à écrire
\begin{equation}
    f\big( z(t) \big)z'(t)dt=g(t)dt
\end{equation}
Cela est une égalité entre deux formes différentielles. Nous avons donc égalité entre les intégrales des formes sur un chemin. Prenons un chemin tout simple de $t_0$ vers $t$ :
\begin{equation}
    \int_{t_0}^tf\big( z(t) \big)z'(t)dt=\int_{t_0}^tg(t)dt.
\end{equation}
Dans le premier membre, nous faisons un changement de variable $\xi=z(t)$, $d\xi=z'(t)dt$, et nous obtenons
\begin{equation}        \label{EqIntDiffAstuztz}
    \int_{z_0}^{z(t)}f(\xi)d\xi=\int_{t_0}^tg(t)dt.
\end{equation}
où nous avons remplacé la constante $z(t_0)$ par $z_0$ dans la borne d'intégration.  Si $F$ est une primitive de $f$ et $G$ une primitive de $g$, nous avons
\begin{equation}
    F(z)-F(z_0)=G(t)-G(t_0).
\end{equation}
Si aucun problème de Cauchy n'est donné, les constantes $F(z_0)$ et $G(t_0)$ sont mises en une seule et nous écrivons la solution
\begin{equation}
    F\big( z(t) \big)=G(t)+C,
\end{equation}
qui est une équation implicite pour $z(t)$. 

Nous trouvons assez souvent le cas simple
\begin{equation}    \label{EqAstfzdzdt}
    f(z)dz=dt.
\end{equation}
En remplaçant $g(t)=1$ dans \eqref{EqIntDiffAstuztz}, nous trouvons la fameuse
\begin{equation}        \label{Eqttzint}
    t-t_0=\int_{z_0}^zf(z)dz,
\end{equation}
dans laquelle il y a un abus de notation terrible entre le $z$ de la borne (que les étudiant(e)s oublient souvent) et la variable d'intégration $z$ !!

Le passage de \eqref{EqAstfzdzdt} à \eqref{Eqttzint} sera très souvent utilisé dans le cours de mécanique par exemple.


%+++++++++++++++++++++++++++++++++++++++++++++++++++++++++++++++++++++++++++++++++++++++++++++++++++++++++++++++++++++++++++
                    \section{Pourquoi la variation des constantes fonctionne toujours ?}
%+++++++++++++++++++++++++++++++++++++++++++++++++++++++++++++++++++++++++++++++++++++++++++++++++++++++++++++++++++++++++++

Prenons une équation non homogène 
\begin{equation}        \label{EqAstNNHomo}
    z'(t)=f(t)z(t)+g(t),
\end{equation}
et supposons avoir une solution de l'homogène associée sous la forme $z_H(t)=Ch(t)$. Le coup de la variation des constates consiste à essayer une solution pour l'équation non homogène sous la forme\footnote{Je ne sais plus qui a eu l'idée de changer le nom de la constante de $C$ vers $K$ au moment de la transformer en fonction, mais c'est une bonne idée.}
\begin{equation}
    z(t)=K(t)h(t).
\end{equation}
Nous injectons cette solution dans l'équation de départ en utilisant le fait que $z'(t)=K'(t)h(t)+K(t)h'(t)$ :
\begin{equation}
    K'(t)h(t)+K(t)h'(t)=f(t)K(t)h(t)+g(t).
\end{equation}
Le terme $K(t)h'(t)$ se récrit en utilisant la propriété de définition de $h$, c'est à dire que $h'(t)=f(t)h(t)$. Nous voyons que les termes ne contenant pas de $K'$ se simplifient; il reste
\begin{equation}
    K'h=g.
\end{equation}
Cette équation a comme solution
\begin{equation}
    K=\int \frac{ f }{ h }+C.
\end{equation}
J'insiste sur la constante d'intégration ! En réalité, celles et ceux qui auront compris l'équation \eqref{Eqttzint} sauront que $K$ est donné par
\begin{equation}
    K(t)=\int_{\xi_0}^{t}\frac{ f(\xi) }{ g(\xi) }d\xi
\end{equation}
où $\xi_0$ joue le rôle de la constante d'intégration.

Quoi qu'il en soit, la solution générale de l'équation non homogène est
\begin{equation}        \label{EqSolVarCosntCool}
    z(t)=K(t)h(t)=\left( \int\frac{ g }{ h }+C \right)h.
\end{equation}
Cette solution comprend deux termes : $Ch$ qui est solution de l'homogène, et $\left( \int \frac{ g }{ h } \right)h$ qui est une particulière de l'équation non homogène.

Quelque conclusions :

\begin{enumerate}
\item
Si vous avez encore du $K$ (et pas que du $K'$) dans votre équation qui donne $K$, c'est que vous n'être pas dans le cadre d'une équation de type \eqref{EqAstNNHomo}. Le plus souvent, c'est que vous avez fait une faute de calcul quelque part.

\item
La méthode des variations des constantes n'est pas en contradiction avec le principe de \og SGEH+SPENH\fg. En effet, la SGEP et la SPENH sont toutes deux dans la solution \eqref{EqSolVarCosntCool}.

\item
La variation des constantes peut être vue comme une façon cool de trouver une solution particulière de l'équation non homogène.

\item
La simplification ne se fait que après avoir remplacé $Kh'$ par $Kfh$, c'est à dire après avoir utilisé le fait que $z_H$ est solution de l'homogène. Sinon, la simplification n'est pas du tout évidente a priori. Il se peut même que, visuellement, les termes $Kh'$ et $Kfh$ ne se ressemblent pas du tout. C'est ce qui arrive par exemple à l'exercice \ref{exo_II-1-07}\ref{ItemCII107}, pour arriver à l'équation \eqref{EqFracII107exoVVprb}.

\end{enumerate}

