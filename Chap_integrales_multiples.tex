%+++++++++++++++++++++++++++++++++++++++++++++++++++++++++++++++++++++++++++++++++++++++++++++++++++++++++++++++++++++++++++
\section{Mesure de Lebesgue}
%+++++++++++++++++++++++++++++++++++++++++++++++++++++++++++++++++++++++++++++++++++++++++++++++++++++++++++++++++++++++++++

Nous construisons à présent la mesure de Lebesgue sur \( \eR^n\). Un \defe{pavé}{pavé} dans \( \eR^n\) est un ensemble de la forme 
\begin{equation}
    B=\prod_{i=1}^n\mathopen[ a_i , b_i \mathclose];
\end{equation}
le volume d'un tel pavé est défini par \( \Vol(B)=\prod_i(b_i-a_i)\). Soit maintenant \( A\subset \eR^n\). La \defe{mesure externe}{mesure!externe} de \( A\) est le nombre
\begin{equation}
    m^*(A)=\inf\{ \sum_{B\in\mF}\Vol(B)\text{ où \( \mF\) est un ensemble dénombrable de pavés dont l'union recouvre \( A\).} \}
\end{equation}
Nous disons que \( A\) est \defe{mesurable}{mesurable!Lebesgue} au sens de Lebesgue si pour tout ensemble \( S\subset \eR^n\) nous avons l'égalité
\begin{equation}
    m^*(S)=m^*(A\cap S)+m^*(S\setminus A).
\end{equation}
Dans ce cas nous disons que la mesure de Lebesgue de \( A\) est \( m(A)=m^*(A)\).

\begin{lemma}\label{LemTHBSEs}
    Si \( f\) est une fonction sur \( \mathopen[ a , \infty [\), alors nous avons la formule
    \begin{equation}
        \lim_{b\to \infty}\int_a^bf(x)dx=\int_a^{\infty}f(x)dx
    \end{equation}
    au sens où si un des deux membres existe, alors l'autre existe et est égal.
\end{lemma}

\begin{proof}
    Supposons que le membre de gauche existe. Cela signifie que la fonction
    \begin{equation}
        \psi(x)=\int_a^xf
    \end{equation}
    est bornée. Soit \( M\), un majorant. Pour toute fonction simple \( \varphi\) dominant \( f\), on a que \( \int\varphi\leq M\), donc l'ensemble sur lequel on prend le supremum pour calculer \( \int_a^{\infty}f\) est majoré par \( M\) et possède donc un supremum. Nous avons donc
    \begin{equation}
        \int_a^{\infty}f\leq\lim_{b\to\infty}\int_a^bf.
    \end{equation}
\end{proof}


%++++++++++++++++++++++++++++++++++++++++++++++++++++++++++++++++++++++++++++++++++++++++++++++++++++++++++++++++++++++++++++++
\section{Fonctions en escalier}
%++++++++++++++++++++++++++++++++++++++++++++++++++++++++++++++++++++++++++++++++++++++++++++++++++++++++++++++++++++++++++++++

%---------------------------------------------------------------------------------------------------------------------------
\subsection{Pavés et subdivisions}
%---------------------------------------------------------------------------------------------------------------------------

\begin{definition}
 Nous appelons \defe{pavé}{pavé} de $\eR^p$ toute partie de $\eR^p$ obtenue comme produit de $p$ intervalles de $\eR$. Plus explicitement, une partie $R$ est un pavé de $\eR^p$ si il s'écrit sous la forme
\[
R=\left\{(x_1,\ldots, x_p)\in\eR^p \,\big\vert\,x_i\in \mathcal{I}_i,  i=1,\ldots, p  \right\},
\]
où $\mathcal{I}_i$ est un intervalle de $\eR$ pour tout $i=1,\ldots, p$. 
\end{definition}
On appelle pavé fermé de $\eR^p$ le produit de $p$ intervalles fermés 
\[
R=\prod_{i=1}^{p}[a_i,b_i].
\]
On définit de même le pavé ouvert 
\[
S=\prod_{i=1}^{p}]a_i,b_i[.
\]
Un pavé $ R=\prod_{i=1}^{p}\mathcal{I}_i$ est dit borné si tous les intervalles $\mathcal{I}_i$ sont bornés dans $\eR$. Les pavés non bornés sont des produits d'intervalles où un (ou plusieurs) des intervalles n'est pas borné. Par exemple,
\[
N=]-\infty, 5]\times [0,13].
\]
L'espace $\eR^p$, lui-même, est un pavé de $\eR^p$. 
\begin{definition}
  Une partie $A$ de $\eR^p$ est dite  \defe{pavable}{pavable} s'il existe une famille finie de pavés bornés $R_j$, $j=1,\ldots, n$, et deux à deux disjoints tels que 
\[
A=\cup_{j=1}^{n}R_j.
\] 
\end{definition}
Un exemple de ensemble pavable dans $\eR^2$ est donné à la figure \ref{LabelFigPolirettangolo}. Il existe beaucoup d'ensembles dans $\eR^2$ qui ne sont pas pavables, par exemple les ellipses.
\newcommand{\CaptionFigPolirettangolo}{Un ensemble pavable.}
\input{Fig_Polirettangolo.pstricks}

Le complémentaire d'un pavé est  un ensemble pavable et, en particulier, tout complémentaire d'un pavé borné est une réunion de  pavés non bornés. Toute union finie et toute intersection d'ensemble pavables est pavable.    
\begin{definition}
	Soit $R$ un pavé borné de $\eR^p$, pour fixer les idées on peut penser $R=\prod_{i=1}^{p}[a_i,b_i]$. On appelle \defe{longueur}{longueur!d'une arrête} de l'$i$-ème arrête de $R$ le nombre $b_i-a_i$. La \defe{mesure $p$-dimensionnelle de $R$}{}, $m(R)$, est le produit des longueurs 
\[
m(R)=\prod_{i=1}^{p}(b_i-a_i).
\] 
\end{definition}
\begin{example}
  Dans $\eR^3$, l'ensemble $R=[-1,1]\times[3,4]\times[0,2]$ est un pavé fermé de mesure 
\[
m(R)= (1+1)\cdot(4-3)\cdot(2-0)=4.
\] 
Il s'agit du volume usuel du parallélépipède rectangle.
\end{example}
\begin{example}
 L'ensemble $R=\mathopen] -1 , 1 \mathclose[\times[3,4]\times[0,2]$ est un pavé de $\eR^3$. Il n'est ni fermé ni ouvert, sa mesure est encore $4$.  
\end{example}
Si $R$ est un pavé non borné on peut encore définir sa mesure. La notion de mesure se généralise en deux étapes. D'abord on dit que la longueur d'une arête non bornée est $\infty$. Ensuite, on adopte la convention $0\cdot \infty=0$. Il faut remarquer que avec cette généralisation tout point et toute droite dans $\eR^2$ ont mesure nulle.  


Afin de définir les intégrales, nous allons intensivement faire appel à la notion de subdivision d'intervalles, voir définition \ref{DefSubdivisionIntervalle} et la discussion qui suit.

Lorsqu'on considère un pavé borné $R=\prod_{i=1}^p\mI_i$ de $\eR^p$, on note $\sdS_i$ l'ensemble des subdivisions de l'intervalle $\mI_i$. La notion de subdivision de généralise au cas des pavés.
\begin{definition}
	Soir $R$ un pavé fermé borné de $\eR^p$, pour fixer les idées on peut penser à $R=\prod_{i=1}^p\mathopen[ a_i , b_i \mathclose]$. On appelle \defe{subdivision}{subdivision} finie de $R$ les éléments de l'ensemble $\mathcal{S}=\prod_{i=1}^{p}\mathcal{S}_i$, 
\[
\mathcal{S}=\left\{ (Y_{1},\ldots, Y_{p})\,\big\vert\, Y_{i}=(y_{i,j})_{j=1}^{n_i}\in\mathcal{S}_i,\, i=1,\ldots,p\right\}.
\]
On peut définir de même l'ensemble des subdivisions d'un pavé non borné. 
 \end{definition}
 Souvent, une subdivision d'un pavé $R=\prod_{i=1}^p\mI_i$ sera noté $\sigma=(y_{i,j})_{j=1}^{n_i}$. Dans cette notation, on sous-entend que pour chaque $i$ fixé, les nombres $y_{i,j}$ (il y en a $n_i$) forment une subdivision de l'intervalle $\mI_i$. Afin de vous familiariser avec ces notations, repérez bien tous les éléments de la figure \ref{LabelFigUneCellule}.
\newcommand{\CaptionFigUneCellule}{Une cellule d'une subdivision d'un pavé de $\eR^2$. La cellule grisée est $R_{(4,2)}$.}
\input{Fig_UneCellule.pstricks}

%On désigne par
%\[
%\delta(Y_i)=\max_{0\leq j\leq n}| y_{i,j}- y_{i,j-1}|,
%\] 
%le pas de la subdivision $Y_i$ dans $\mathcal{S}_i$ et par 
%\[
%\delta(\sigma)=\max_{0\leq i\leq p}\delta(Y_i),
%\]  
%le pas de la subdivision $\sigma$ dans $\mathcal{S}$.

\begin{definition}
	Si $\sigma$ est une subdivision d'un pavé $R$, un \defe{raffinement}{raffinement!subdivision d'un pavé} de $\sigma$ est une subdivision de $R$ obtenue en fixant plus de points dans chaque intervalle.
\end{definition}

La subdivision $\sigma$ de $R$ détermine $n_1\cdot n_2\cdot\cdots\cdot n_p$ pavés fermés de la forme 
\[
R_{(k_1,\ldots,k_p)}=\{(x_1,\ldots, x_p)\in\eR^p\,\big\vert\, y_{i,k_{i-1}}\leq x_i\leq y_{i,k_i}\},
\]
où $k_i$ est dans $\{1,\ldots, n_i\}$ et $i$ dans $\{1,\ldots, p\}$. On les appelles \defe{cellules}{cellule d'un pavage} de $\sigma$. On remarque que les cellules de $\sigma$ sont toujours deux à deux disjointes (sauf au plus sur leurs bords). 
\begin{lemma}\label{meas_sous}
	Soit $R$ un pavé borné de $\eR^p$ et soit $\sigma=(y_{i,j})_{j=1}^{n_i}$ une subdivision de $R$. 
On a 
\[
m(R)=\sum_{(k_1,\ldots,k_p)\in K} m(R_{(k_1,\ldots,k_p)}),
\] 
où $K=\{1,\ldots,n_1\}\times\{1,\ldots,n_2\}\times\ldots \times\{1,\ldots,n_p\}$.
\end{lemma}
Le lemme \ref{meas_sous} suggère de définir la mesure d'un ensemble borné pavable $P=\cup_{j=1}^{n}R_j$ comme la somme des mesures des pavés disjoints $R_j$, $j=1,\ldots, n$.
\begin{definition}
Une application $f:\eR^p\to\eR$ est dite \defe{application en escalier}{application!en escalier} sur $\eR^m$ si
  \begin{itemize}
  \item $f$ est une application bornée,
\item il existe une subdivision $\sigma$ de $\eR^p$ telle que la restriction de $f$  est une application constante sur toute cellule $R_k$ de $\sigma$
\[
f_{\vert_{R_k}}=C_k, \qquad C_k\in\eR,
\]
%Pour tout $k=(k_1,\ldots,k_p)$ dans $ K=\{1,\ldots,n_1\}\times\{1,\ldots,n_2\}\times\ldots \times\{1,\ldots,n_p\}$.
 
Une telle subdivision $\sigma$ est dite \defe{associée}{associée!subdivision}\index{subdivision!associée à une fonction} à $f$. 
  \end{itemize}
\end{definition} 
\begin{example}
  La fonction $f$ de $\eR^2$ dans  $\eR$ définie par 
  \begin{equation}
    f(x,y)=\left\{
    \begin{array}{ll}
      1&\qquad \textrm{si } (x,y) \in [0,3]\times[-1,2],\\
2 &\textrm{sinon.} 
    \end{array}\right.
  \end{equation}
est une application en escalier. Exercice : donner une subdivision de $\eR^2$ associée à cette fonction.
\end{example}
\begin{example}
  La fonction $f$ de $\eR^2$ dans  $\eR$ définie par 
  \begin{equation}
    f(x,y)=\left\{
    \begin{array}{ll}
      \frac{1}{m^2+n^2},&\qquad \textrm{si } (x,y) \in [m,m+1]\times[n,n+1], \quad m,\,n\in\eN_0,\\
0, &\textrm{sinon} 
    \end{array}\right.
  \end{equation}
est une application en escalier.  Observez que, dans ce cas, il n'existe pas une subdivision finie de $\eR^2$ associée à $f$. 
\end{example}
\begin{remark}
 Si la subdivision $\sigma$ est associée à $f$ alors tout raffinement de $\sigma$ (c'est à dire, toute subdivision obtenue en fixant plus de points dans chaque intervalle) a la même propriété. 

Si $f$ et $g$ sont deux application en escalier sur $R$ et $\sigma_f$ et $\sigma_g$ sont des subdivisions de $R$ associées respectivement à $f$ et $g$, alors on peut construire une troisième subdivision de $R$ qui est associée à $f$ et à $g$ en même temps. Soient $\sigma_f=(Y_{1},\ldots, Y_{p})$ et $\sigma_g=(Z_{1},\ldots, Z_{p})$, où  $Y_{i}=(y_{i,j})_{j=1}^{m_i}$ et $Z_{i}=(z_{i,j})_{j=1}^{n_i}$ sont des subdivision de l'intervalle $[a_i, b_i]$, pour $i=1,\ldots, p$. La subdivision de $[a_i, b_i]$ obtenue par l'union de $Y_i$ et $Z_i$ est encore une subdivision finie, qu'on appellera $\bar Y_i$. La subdivision $\bar \sigma = (\bar Y_{1},\ldots,\bar Y_{p})$ de $R$ est un raffinement de $\sigma_f $ et de $\sigma_g$, donc elle est associée à la fois à $f$ et à $g$. 

Cela nous permet de prouver que si $f$ et $g$ sont des application en escalier, alors $f+g$, $fg$, $\min\{f,g\}$, $\max\{f,g\}$ et $|f|$ sont des applications en escalier. 
\end{remark}

%---------------------------------------------------------------------------------------------------------------------------
\subsection{Intégrale d'une fonction en escalier}
%---------------------------------------------------------------------------------------------------------------------------

\begin{definition}
  Soit $f$ une fonction de $\eR^m$ dans $\eR^n$. Le \defe{support}{support} de $f$ est la fermeture de l'ensemble des points $x$ tels que $f(x)\neq 0$. 
\end{definition}
\begin{definition}
Une application en escalier $f$ est dite \defe{intégrable}{fonction en escalier intégrable} si son support est compact. 
\end{definition} 
Soit $f$ une application en escalier sur $\eR^p$. Soit $\sigma$ une subdivision de  $\eR^p$ associée à $f$ et appelons $R_k$ les cellules de $\sigma$, avec $k=(k_1,\ldots,k_p)$ dans $ K=\{1,\ldots,n_1\}\times\{1,\ldots,n_2\}\times\ldots \times\{1,\ldots,n_p\}$. Alors  
\[
f_{\vert_{R_k}}=C_k, \qquad C_k\in\eR.
\]

\begin{definition} 
On définit l'\defe{intégrale}{intégrale!fonction en escalier} de $f$ sur $\eR^p$ par
\[
\int_{\eR^p}f\,dV=\sum_{k\in K}C_km(R_k).
\] 
\end{definition}
L'intégrale ainsi définie est un nombre réel. La proposition suivante nous dit que l'intégrale est «bien définie», au sens que sa valeur ne dépend pas de la subdivision associée à $f$ qu'on utilise dans le calcul. 
\begin{proposition}
Soit $f$ une application en escalier intégrable sur $\eR^p$. Soient $\sigma_1$ et $\sigma_2$ deux subdivisions de $\eR^p$ associées à  $f$. L'intégrale de $f$ ne dépend pas de la subdivision choisie.
\end{proposition}
On ne donne pas une preuve complète de cette proposition. En fait elle est une conséquence de la formule de réduction introduite dans la suite de ce chapitre.  


%%%%%%%%%%%%%%%%%%%%%%%%%%%%%%%%%%%%%%%%%%%%%%%%%%%%%%%%%%%%%%%%%%%%%%%%%%%%%%%%
\subsection{Intégrales partielles}
%%%%%%%%%%%%%%%%%%%%%%%%%%%%%%%%%%%%%%%%%%%%%%%%%%%%%%%%%%%%%%%%%%%%%%%%%%%%%%%%
Soit $f$ de $\eR^p$ dans $\eR$ une fonction continue, nulle hors du pavé borné $R$. Posons  $R=\prod_{i=1}^{p}[a_i,b_i]$, pour fixer les idées. Pour chaque $i$ dans $\{1,\ldots, p\}$ fixé, on peut associer à $f$ la fonction $F_i$ de $p-1$ variables définie par
\[
F_i(x_1,\ldots, x_{i-1}, x_{i+1}, \ldots, x_p)=\int_{a_i}^{b_i}f(x_1,\ldots, x_{i-1},y, x_{i+1}, \ldots, x_p)\, dy.
\]  
La fonction $F_i$ est l'intégrale partielle de $f$ par rapport à la $i$-ème variable. 
En particulier, si $f(x_1,\ldots, x_p)=g(x_i)h(x_1,\ldots, x_{i-1}, x_{i+1}, \ldots, x_p)$ on obtient 
\[
F_i=\int_{a_i}^{b_i}g(y)h(x_1,\ldots, x_{i-1}, x_{i+1}, \ldots, x_p)\, dy= h\cdot\int_{a_i}^{b_i}g \, dy.
\]  
La fonction d'une seule variable qu'on obtient à partir de $f$ en fixant $x_1,\ldots, x_{i-1}, x_{i+1}, \ldots, x_p$ et qui associe à $x_i$ la valeur $f(x_1,\ldots, x_{i-1}, x_i, x_{i+1}, \ldots, x_p)$, est appelée $x_i$-ème section de $f$ en $x_1,\ldots, x_{i-1}, x_{i+1}, \ldots, x_p$. 
\begin{example}
  Soit $f$ la fonction de $\eR^2$ dans $\eR$ définie par 
  \begin{equation}
	  f(x,y)=\begin{cases}
		  x+3y	&	\text{si $(x,y)\in\mathopen[ 9 , 10 \mathclose]\times\mathopen] \pi , 5 \mathclose]$}\\
		  0	&	 \text{sinon}.
	  \end{cases}
  \end{equation}
 Les intégrales partielles de $f$ sont 
\[
F_1(y)=\int_{9}^{10}x+3y\,dx=\left[\frac{x^2}{2}+3xy\right]_{x=9}^{x=10}=\frac{19}{2}+3y,
\]
\[
F_2(x)=\int_{\pi}^{5}x+3y\,dy=\left[xy+\frac{3y^2}{2}\right]_{y=\pi}^{y=5}=x(5-\pi)+\frac{3}{2}(25-\pi^2).
\]
\end{example}
%%%%%%%%%%%%%%%%%%%%%%%%%%%%%%%%%%%%%%%%%%%%%%%%%%%%%%%%%%%%%%%%%%%%%%%%%%%%%%%%
\subsection{Réduction d'une intégrale multiple}
%%%%%%%%%%%%%%%%%%%%%%%%%%%%%%%%%%%%%%%%%%%%%%%%%%%%%%%%%%%%%%%%%%%%%%%%%%%%%%%%
 
Soit $R=[a,b]\times[c,d]$ un pavé fermé et borné de $\eR^2$ et soit $f$ une application en escalier intégrable sur $\eR^2$ telle que le support de $f$ soit contenu dans $R$. On considère la subdivision $\sigma$ de $R$ définie par les subdivisions 
\[
a=x_0\leq x_1\leq\ldots\leq x_m=b,
\]  
 \[
c=y_0\leq y_1\leq\ldots\leq y_n=d.
\]  
Les cellules de $\sigma$ sont 
\[
R_{i,j}=[x_{i},x_{i+1}]\times[y_{j},y_{j+1}], \quad\qquad i=0,\ldots,m-1, \quad j=0,\ldots,n-1.
\]
La mesure de $R$ est la somme des mesures des $R_{i,j}$
\begin{equation}
  \begin{aligned}
    m(R)=&\sum_{(i,j)\in \{0,\ldots, m-1\}\times\{0,\ldots, n-1\}} m(R_{i,j})=\\
&=\sum_{j=0}^{n-1}\sum_{i=0}^{m-1}(x_{i+1}-x_{i})\cdot(y_{i+1}-y_{i})=\\
&=\sum_{i=0}^{m-1}(x_{i+1}-x_{i})\cdot\sum_{j=0}^{n-1}(y_{i+1}-y_{i})=\\
&= (b-a)\cdot(d-c).
  \end{aligned}
\end{equation}
Si $f$ est constante sur chaque cellule de $\sigma$ on peut écrire $f$ de la forme suivante
\[
f(x,y)=\sum_{j=0}^{n-1}\sum_{i=0}^{m-1}C_{i,j}\,\chi_{R_{i,j}}
\]
où les $C_{i,j}$ sont des constantes réelles et $\chi_{R_{i,j}}$ est la \defe{fonction caractéristique}{fonction!caractéristique} de $R_{i,j}$
\begin{equation}
  \chi_{R_{i,j}}(x,y)=\left\{
      \begin{array}{ll}
      1,\qquad &\textrm{si } (x,y)\in R_{i,j} ,\\
0, & \textrm{sinon}.
      \end{array}\right.
\end{equation}
Comme $(x,y)$ est dans $R_{i,j}$ si et seulement si $x\in[x_{i},x_{i+1}]$ et $ y\in[y_{j},y_{j+1}]$, on vérifie que la fonction $\chi_{R_{i,j}}$ est égal au produit des fonctions caractéristiques des intervalles $[x_{i},x_{i+1}]$ et $[y_{j},y_{j+1}]$ 
\[
 \chi_{R_{i,j}}(x,y)=\chi_{[x_{i},x_{i+1}]}(x)\cdot\chi_{[y_{j},y_{j+1}]}(y).
\] 
On peut donc écrire la fonction $f$ de la façon suivante
\[
f(x,y)=\sum_{j=0}^{n-1}\sum_{i=0}^{m-1}C_{i,j}\,\chi_{[x_{i},x_{i+1}]}(x)\cdot\chi_{[y_{j},y_{j+1}]}(y).
\] 
Comme on suppose que le support de $f$ est une partie de $R$, l'intégrale de $f$ sur $\eR^2$ est
\begin{equation}
  \begin{aligned}
\int_{\eR^2}f \,dV = \sum_{j=0}^{n-1}\sum_{i=0}^{m-1}C_{i,j}\,m(R_{i,j})=\sum_{j=0}^{n-1}\sum_{i=0}^{m-1}C_{i,j}\,(x_{i+1}-x_i)\cdot(y_{j+1}-y_j).
 \end{aligned}
\end{equation} 
Cette intégrale peut être réduite à la composition de deux intégrales partielles. Il suffit de remarquer que la valeur de l'intégrale de la fonction caractéristique d'un intervalle est la longueur de l'intervalle, 
\begin{equation}
  \begin{aligned}
    C_{i,j}(x_{i+1}-x_i)&\cdot(y_{j+1}-y_j)=\\
&=C_{i,j}\left(\int_{x_i}^{x_{i+1}}\chi_{[x_{i},x_{i+1}]}(x)\, dx \right)\cdot \left(\int_{y_j}^{y_{j+1}}\chi_{[y_{ j},y_{ j+1}]}(y)\, dy \right)=\\
&=C_{i,j}\left(\int_{a}^{b}\chi_{[x_{i},x_{i+1}]}(x)\, dx \right)\cdot \left(\int_{c}^{d}\chi_{[y_{ j},y_{ j+1}]}(y)\, dy \right),
  \end{aligned}
\end{equation}
et utiliser les propriétés de linéarité de l'intégrale
\begin{equation}
  \begin{aligned}
   \int_{\eR^2}f \,dV =& \sum_{j=0}^{n-1}\sum_{i=0}^{m-1}C_{i,j}\,\left(\int_{a}^{b}\chi_{[x_{i},x_{i+1}]}(x)\, dx \right)\cdot \left(\int_{c}^{d}\chi_{[y_{ j},y_{ j+1}]}(y)\, dy \right)=\\
&=\int_{c}^{d}\int_{a}^{b}\sum_{j=0}^{n-1}\sum_{i=0}^{m-1}C_{i,j}\,\chi_{[x_{i},x_{i+1}]}(x)\cdot \chi_{[y_{ j},y_{ j+1}]}(y)\, dx dy=\\
&=\int_{c}^{d}\int_{a}^{b} f\, dx dy.  
  \end{aligned}
\end{equation}
De même on obtient
\begin{equation}
  \begin{aligned}
   \int_{\eR^2}f \,dV =&\int_{a}^{b}\int_{c}^{d}\sum_{j=0}^{n-1}\sum_{i=0}^{m-1}C_{i,j}\,\chi_{[x_{i},x_{i+1}]}(x)\cdot \chi_{[y_{ j},y_{ j+1}]}(y)\, dx dy=\\
&=\int_{a}^{b}\int_{c}^{d} f\, dx dy.  
  \end{aligned}
\end{equation}
En général, on preuve la proposition suivante
\begin{proposition}
 Soit $f$ une application en escalier intégrable sur $\eR^p$ et soit $R$ un pavé borné dans $\eR^p$ qui contient le support de $f$. Comme d'habitude, pour fixer les idées nous écrivons $=\prod_{i=1}^p[a_i,b_i]$. Alors
 \begin{equation}
   \begin{aligned}
     \int_{\eR^p}f(x_1,\ldots, x_p) \, dV =& \int_{a_p}^{b_p}\int_{a_{p-1}}^{b_{p-1}}\cdots\int_{a_1}^{b_1} f(x_1,\ldots, x_p) \, dx_1\cdots dx_p=\\
&=\int_{a_{s_p}}^{b_{s_p}}\int_{a_{s_{p-1}}}^{b_{s_{p-1}}}\cdots\int_{a_{s_1}}^{b_{s_1}} f(x_1,\ldots, x_p) \, dx_1\cdots dx_p,
   \end{aligned}
 \end{equation}
pour toute permutation $(s_1,\ldots,s_p)$ de l'ensemble $\{1,\ldots p\}$.
\end{proposition}
%%%%%%%%%%%%%%%%%%%%%%%%%%%%%%%%%%%%%%%%%%%%%%%%%%%%%%%%%%%%%%%%%%%%%%%%%%%%%%%%
\subsection{Propriétés de l'intégrale}
%%%%%%%%%%%%%%%%%%%%%%%%%%%%%%%%%%%%%%%%%%%%%%%%%%%%%%%%%%%%%%%%%%%%%%%%%%%%%%%%
Soient $f$ et $g$ deux fonctions en escalier intégrables de $\eR^p$ dans $\eR$, et soient $a$ et $b$ dans $\eR$. 
\begin{description}
\item[Linéarité de l'intégrale] : 
  \begin{itemize}
  \item Additivité : $f+g$ est intégrable et 
\[
\int_{\eR^p} (f+g)\, dV = \int_{\eR^p} f\, dV+ \int_{\eR^p} g\, dV,
\]
\item Homogénéité : $\lambda f$ est intégrable pour tout réel $\lambda$ 
\[
\int_{\eR^p} \lambda  f\, dV = \lambda\int_{\eR^p} f\, dV,
\]
  \end{itemize}
\item[Monotonie] Si $f\leq g$ alors 
\[
 \int_{\eR^p} f\, dV\leq \int_{\eR^p} g\, dV,
\]
\item[Inégalité fondamentale]
  \[
\lvert \int_{\eR^p}f\,dV\rvert \leq\int_{\eR^p}\lvert f\rvert\,dV.
\] 
Cette dernière inégalité s'obtient de la façon suivante :
\[
\lvert\int_{\eR^p}f\,dV\rvert =\lvert \sum_{k\in K} C_k m(R_k)\rvert \leq\sum_{k\in K}\lvert C_k\rvert m(R_k)=\int_{\eR^p}|f|\,dV.
\] 
\item[Inégalité de Čebičeff]  Si $f$ est une application en escalier alors pour tout $a>0$ dans $\eR$ l'ensemble $\{x\in\eR^p\,:\, |f(x)|\geq a\}$ est pavable et borné, et l'inégalité suivante est satisfaite
\[
m\left(\{x\in\eR^p\,:\, |f(x)|\geq a\}\right)\leq \frac{1}{a} \int_{\eR^p}\lvert f\rvert\,dV.
\]
\end{description}

%%%%%%%%%%%%%%%%%%%%%%%%%%%%%%%%%%%%%%%%%%%%%%%%%%%%%%%%%%%%%%%%%%%%%%%%%%%%%%%%
\section{Intégrales multiples, cas général}
%%%%%%%%%%%%%%%%%%%%%%%%%%%%%%%%%%%%%%%%%%%%%%%%%%%%%%%%%%%%%%%%%%%%%%%%%%%%%%%%

Dans cette section on veut généraliser la définition d'intégrale multiple au cas des domaines non pavables et de fonctions qui ne sont pas en escalier. Il y a plusieurs méthodes de le faire et ici on ne considère qu'une seule, introduite par Riemann.  
\begin{definition} Soit $f: \eR^p\to \eR$ une fonction.
  \begin{itemize}
	  \item Pour toute application en escalier intégrable $f_*$ telle que $f_*\leq f$, l'intégrale de $f_*$ est dit une \defe{somme inférieure}{somme!inférieure} de $f$. 
	  \item Pour toute application en escalier intégrable $f^*$ telle que $f_*\geq f$, l'intégrale de $f^*$ est dit une \defe{somme supérieure}{somme!supérieure} de $f$. 
  \end{itemize}
\end{definition}
Soient $\sum_* f$ et  $\sum^* f$ les ensembles des sommes inférieures et supérieures de $f$. Grâce à la propriété de  monotonie de l'intégrale on sait que si $a$ est dans $\sum_* f$ et  $b$ est dans $\sum^* f$ alors $a\leq b$. 
\begin{definition}
  La fonction $f$ est intégrable (au sens de Riemann) si $\sum_* f$ et  $\sum^* f$ ne sont pas vides et 
\[
\inf \Sigma^* f=I =\sup \Sigma_* f.
\] 
Dans ce cas, la valeur $I$ est appelée intégrale de $f$ sur $\eR^p$. 
\end{definition}
\begin{remark}
  Toute fonction intégrable est bornée et à support compact. En effet, si le support de la  fonction n'est pas compact alors soit $\sum_* f$ soit $\sum^* f$ doit être vide ! 
\end{remark}
L'intégrale qu'on vient de définir possède toutes les propriétés de l'intégrale pour les fonctions en escalier. Le produit de deux fonctions intégrables est intégrable. 

Il y a des cas où l'intégrabilité d'une fonction n'est pas évidente. Cependant, dans la plupart des exercices et des exemples de ce cours, nous nous aidons avec le critère suivant 
\begin{proposition}
  Toute fonction continue à support compact est intégrable. 
\end{proposition}
Cette proposition n'est a priori pas étonnante, vu qu'une fonction continue sur un support compact est bornée (théorème de Weierstrass \ref{ThoWeirstrassRn}).

%%%%%%%%%%%%%%%%%%%%%%%%%%%%%%%%%%%%%%%%%%%%%%%%%%%%%%%%%%%%%%%%%%%%%%%%%%%%%%%%
\subsection{Réduction d'une intégrale multiple}
%%%%%%%%%%%%%%%%%%%%%%%%%%%%%%%%%%%%%%%%%%%%%%%%%%%%%%%%%%%%%%%%%%%%%%%%%%%%%%%%
On n'utilise jamais la définition pour calculer la valeur d'une intégrale multiple. La méthode plus efficace, en pratique, est de réduire l'intégrale à la composition de plusieurs intégrales d'une variable.  
\begin{theorem}[de Fubini]\label{fub}
 Soit $f$ une fonction intégrable de $\eR^2$ dans $\eR$. Si pour tout $x$ dans $\eR$ la section $f(x,\cdot)$ est intégrable par rapport à $y$, alors
\[
\int_{\eR^2}f(x,y)\,dV=\int_{\eR}\left(\int_{\eR}f(x,y)\,dx\right)\,dy.
\]
De même, si pour tout $y$ dans $\eR$ la section $f(\cdot, y)$ est intégrable par rapport à $x$, alors
\[
\int_{\eR^2}f(x,y)\,dV=\int_{\eR}\left(\int_{\eR}f(x,y)\,dy\right)\,dx.
\] 
\end{theorem}		\label{ThoSectionINte}
En général, on ne peut pas dire que les sections d'une fonction intégrable sont intégrables, donc il faut vraiment se souvenir des hypothèses du théorème \ref{fub}. En dimension plus haute, on a le même résultat
\begin{theorem}
 Soit $f$ une fonction intégrable de $\eR^p$ dans $\eR$. Si pour tout $(p-1)$-uple $(x_1,\ldots, x_{i-1},x_{i+1}, \ldots, x_p)$ dans $\eR^{p-1}$ la section $f(x_1,\ldots, x_{i-1},\cdot,x_{i+1}, \ldots, x_p)$ est intégrable par rapport à $x_i$, alors
\[
\int_{\eR^p}f \,dV=\int_{\eR}\left(\int_{\eR^{p-1}}f \,dV\right)\,dx_i.
\]
\end{theorem}

 Si $f$ est une fonction positive et intégrable de $\eR^2$ dans $\eR$ on peut interpréter l'intégrale de $f$ comme le volume du solide au-dessous du graphe de $f$.  Avec cette interprétation,  l'intégrale partielle par rapport à $x$ pour $y=y_0$ fixé est l'aire de la tranche qu'on obtient en coupant le solide par le plan $y=y_0$.
 \begin{example}
   Le premier exemple à faire est celui d'une fonction en escalier intégrable et positive. Soit $f\colon \eR^2\to \eR$ la fonction
\begin{equation}
	f(x,y)=\begin{cases}
		1	&	\text{si $(x,y)\in R_1=\mathopen] -1 , 3 \mathclose]\times\mathopen[ 4 , 5 \mathclose]$}\\
		3	&	 \text{si $(x,y)\in R_2=\mathopen] 13 , 15 \mathclose[\times\mathopen[ 0 , 2 \mathclose[$}\\
		0	&	 \text{dans les autes cas.}
	\end{cases}
\end{equation}
L'intégrale de $f$ sur $\eR^2$ est $1\cdot m(R_1)+ 3\cdot m(R_2)= 16$. On voit tout de suite qu'il s'agit de la somme du volume des deux parallélépipèdes de hauteurs respectives $1$ et $3$ et bases $R_1$ et $R_2$. 
 \end{example}
\begin{example} 
On veut calculer le volume du solide $S$, borné par le paraboloïde elliptique $x^2+2y^2+z=16$ et le plans $x=2$, $x=0$, $y=2$ $y=0$, $z=0$. On observe que la portion de  paraboloïde elliptique qui nous intéresse est le graphe de la fonction $f(x,y)=16-x^2-2y^2$ pour $(x,y)$ dans $R=[0,2]\times[0,2]$. La fonction $f$ est continue ainsi que ses sections, donc on peut appliquer le théorème \ref{fub} et décomposer l'intégrale double en deux intégrales simples :
\begin{equation}
  \begin{aligned}
   & \int_R 16-x^2-2y^2 \,dV= \int_{0}^2\int_{0}^2f(x,y)\,dx dy= \\
&=\int_0^2 \left[(16-2y^2)x-\frac{x^3}{3}\right]_{x=0}^{x=2}\, dy =\\
& = \left[ \left(32-\frac{8}{3}\right) y -\frac{4y^3}{3}\right]_{x=0}^{x=2}= 64- \frac{16+32}{3}=48.
  \end{aligned}
\end{equation}
Vérifiez, comme exercice, qu'on obtient le même résultat en intégrant d'abord par rapport à $y$ et puis par rapport à $x$.  
\end{example}
\begin{example}
  Dans les hypothèses du théorème \ref{fub}  l'ordre des intégrations partielles ne change pas la valeur de l'intégrale. En fait, si les calculs sont faites par des êtres humains l'ordre d'intégration peut faire une certaine différence comme dans cet exemple. On veut évaluer la valeur de l'intégrale 
\[
\int_{\eR^2}f(x,y)\, dV
\]
où 
\begin{equation}
	f(x,y)=\begin{cases}
		y\sin(x,y)	&	\text{si $(x,y)\in\mathopen[ 1,2 ,  \mathclose]\times\mathopen[ 0 , \pi \mathclose]$,}\\
		0	&	 \text{sinon.}
	\end{cases}
\end{equation}
Les deux section de $f(x,y)=y\sin(xy)$ sont continues. Si on intègre d'abord par rapport à $y$ on obtient 
\[
-\int_1^2\frac{ \pi\cos(\pi x) }{ x }dx+\int_1^2\frac{ \sin(\pi x) }{ x^2 }dx,
\] 
qui n'est pas du tout immédiat, alors que, si on intègre d'abord par rapport à $x$ on obtient 
\[
\int_0^\pi \cos y - \cos(2y)\,dy.
\] 
\end{example}

%%%%%%%%%%%%%%%%%%%%%%%%%%%%%%%%%%%%%%%%%%%%%%%%%%%%%%%%%%%%%%%%%%%%%%%%%%%%%%%%
\subsection{Intégrales sur des parties de $\eR^2$ }
%%%%%%%%%%%%%%%%%%%%%%%%%%%%%%%%%%%%%%%%%%%%%%%%%%%%%%%%%%%%%%%%%%%%%%%%%%%%%%%%

On veut évaluer l'intégrale de la fonction $f(x,y)=\sqrt{1-x^2}$ sur son domaine, la boule unité $B((0,0),1)$. La théorie introduite jusqu'ici n'est pas suffisante pour résoudre  ce problème, parce que $B((0,0),1)$ n'est pas pavable. Les parties bornées de $\eR^p$ sur lesquelles on peut intégrer des fonction sont dites mesurables (au sens de Riemann) parce que, comme on verra dans la suite, la mesure d'une partie de $\eR^p$ est l'intégrale (s'il existe) de sa fonction caractéristique. 

On peut dire que une partie de $\eR^p$  est mesurable si son bord est <<assez régulier>>. Dans $\eR^2$ il est suffisant que le bord de $A$ soit une réunion finie de courbes paramétrées continues. En particulier, on est très souvent dans un des deux cas suivantes
\begin{description}
\item[Régions du premier type] $A$ est borné et contenu entre les graphes de deux fonctions continues de $x$
\[
A=\{(x,y)\in\eR^2 \,:\, a\leq x\leq b, \, g_1(x)\leq y\leq g_2(x)\}, 
\]
avec $g_1$ et $g_2$ continues. 
\item[Régions du deuxième type] $A$ est borné et contenu entre les graphes de deux fonctions continues de $y$
\[
A=\{(x,y)\in\eR^2 \,:\, c\leq y\leq d, \, h_1(y)\leq x\leq h_2(y)\}, 
\]
avec $h_1$ et $h_2$ continues.
\end{description}
%\ref{LabelFigRegioniPrimoeSecondoTipo}
\newcommand{\CaptionFigRegioniPrimoeSecondoTipo}{Régions du premier et du deuxième type}
\input{Fig_RegioniPrimoeSecondoTipo.pstricks}

\begin{example}
 Il y a des régions qui sont des deux types au même temps, comme les boules centrées à l'origine, le triangle de sommets  $(0,0)$, $(0,a)$ et $(b,0)$, ou la région $C$ délimité par les courbes $y=2x$ et $y=x^2$. Cette dernière admets les représentations suivantes
\[
C= \{(x,y)\in\eR^2 \,:\, 0\leq x\leq 1, \, x^2\leq y\leq 2x\},
\] 
et  
\[
C= \{(x,y)\in\eR^2 \,:\, 0\leq y\leq 1, \, y/2\leq x\leq \sqrt{y}\}.
\]  
\end{example}
\begin{definition}
  Soit $f$ une fonction de $\eR^2$ dans $\eR$ dont le support  $A$ est une région du premier ou du deuxième type. On définit la fonction $\bar f$ comme
 \begin{equation}
 \bar f(x,y) = \left\{ \begin{array}{ll}
     f(x,y), \qquad & \textrm{si } (x,y)\in A,\\
  0 , & \textrm{sinon.} 
    \end{array}\right.
  \end{equation}
  La fonction $f$ est dite \defe{intégrable}{intégrable!fonction non en escalier} si $\bar f$ est intégrable, et la valeur de son intégrale est 
\[
\int_A f\, dV=\int_{\eR^2} \bar f\, dV.
\] 
\end{definition}
Une fonction continue définie sur une région du premier ou du deuxième type est toujours intégrable. 

Pour fixer les idées on suppose ici que $A$ est du premier type et contenue dans le pavé borné $R=[a,b]\times [c,d]$. En suivant la définition on obtient
\begin{equation}
  \begin{aligned}
    \int_A f\, dV&=\int_{\eR^2} \bar f\, dV=\\
    &= \int_a^b\int_c^d \bar f\, dy dx=\\
&= \int_a^b\left(\int_c^{g_1(x)} \bar f\, dy+\int_{g_1(x)}^{g_2(x)} \bar f\, dy+\int_{g_2(x)}^d \bar f\, dy\right)\, dx= \\
&= \int_a^b\int_{g_1(x)}^{g_2(x)}  f\, dy dx.
  \end{aligned}
\end{equation}
De même, si $A$ est du deuxième type on obtient 
\begin{equation}
     \int_A f\, dV=\int_c^d\int_{h_1(y)}^{h_2(y)}  f\, dx dy.
\end{equation}
\begin{example}
	On peut maintenant résoudre notre problème de départ, évaluer l'intégrale de la fonction $f(x,y)=\sqrt{1-x^2}$ sur $B((0,0),1)$. Nous choisissons de décrire la boule unité de $\eR^2$ comme une région du premier type : $B((0,0),1)=\{(x,y)\, :\, x\in[-1,1], \, -\sqrt{1-x^2}\leq y\leq \sqrt{1-x^2} \}$. 
	\begin{equation}
		I=\int_{B}\sqrt{1-x^2}\, dV=\int_{-1}^1\int_{-\sqrt{1-x^2}}^{\sqrt{1-x^2}}\sqrt{1-x^2}dydx
	\end{equation}
	La première intégrale à effectuer, par rapport à $y$, est l'intégrale d'une fonction constante. Ne pas oublier que l'on intègre $\sqrt{1-x^2}$ par rapport à $y$; c'est bien une constante et l'intégrale consiste seulement à multiplier par $y$ :
	\begin{equation}
		I=\int_{-1}^1\left[ y\sqrt{1-x^2} \right]_{y=-\sqrt{1-x^2}}^{y=\sqrt{1-x^2}}dx=2\int_{-1}^1(1-x^2)dx.
	\end{equation}
	Cela est à nouveau une intégrale simple à effectuer. Le résultat est
	\begin{equation}
		2\int_{-1}^1(1-x^2)dx=2\left[ x-\frac{ x^3 }{ 3 } \right]_{x=-1}^{x=1}=\frac{ 8 }{ 3 }.
	\end{equation}
\end{example}
\begin{remark}
	Toutes les techniques d'intégration à une variable restent valables. Par exemple, lorsqu'une des intégrales est l'intégrale d'une fonction impaire sur un intervalle symétrique par rapport à zéro, l'intégrale vaut zéro.
\end{remark}

\begin{definition}		\label{DefMesureInt}
	On appelle \defe{mesure}{mesure dans $\eR^2$} d'une région borné de  $\eR^2$  l'intégrale de sa fonction caractéristique, si elle existe.  
\end{definition}
La mesure d'une région bornée de $\eR^2$ est dite son \defe{aire}{aire}, et celle d'une région bornée de $\eR^3$ est son \defe{volume}{volume!région bornée dans $\eR^3$}. Voir aussi la remarque \ref{RemLongIntUn}.


\begin{example}\label{exint}
  On veut calculer l'aire de la région de la figure \ref{LabelFigExampleIntegration} définie par 
\[
A=\{(x,y)\in\eR^2\,\vert\, 0\leq x\leq 1, x^3-1\leq y\leq x \}.
\]
On considère l'intégrale 
\[
\int_{\eR^2} \chi_{A}\, dV= \int_0^1\int^{x}_{x^3+1} 1 \, dy\, dx= \int_0^1 -x^3+x+1\, dx= -\frac{1}{4}+\frac{1}{2}+1=\frac{5}{4}.
\]
\end{example}
\newcommand{\CaptionFigExampleIntegration}{La région $A$ de l'exemple \ref{exint}}
\input{Fig_ExampleIntegration.pstricks}

\begin{exercice}

	% C'est moche, mais il faut laisser une ligne vide ici, sinon il n'y a pas de saut de ligne
	% entre le titre «exercice» et le texte.
  Parfois la région sur laquelle on veut intégrer peut être décrite indifféremment en deux façons, mais la fonction à intégrer nous force a choisir un ordre particulier. Vérifiez que la fonction $f(x,y)=\sin(y^2)$ sur la région triangulaire de sommets $(0,0)$, $(0, 2)$, $(2,2)$ doit être intégrée d'abord par rapport à $x$.     
\end{exercice}

Si une région bornée n'est pas de premier ou de deuxième type on peut normalement la découper en morceaux plus faciles à décrire. On utilise alors la propriété suivante. 
\begin{lemma}
  Soit $A$ un sous-ensemble borné de $\eR^2$ et soient $B_1$ et $B_2$ deux parties de $A$ telles que $B_1\cap B_2=\emptyset$ et $B_1\cup B_2= A$. Alors, pour toute fonction $f$ intégrable sur $A$ (et en particulier pour sa fonction caractéristique) on a
\[
\int_{A}f \, dV= \int_{B_1}f \, dV+\int_{B_2}f \, dV.
\] 
\end{lemma}
\begin{example}\label{exint2}
  La région $D$ que on voit sur la figure \ref{LabelFigExampleIntegrationdeux} est  bornée par la parabole $y^2=2x+6$ et la droite $y=x-1$. La région $D$ est une région du deuxième type. On peut la décrire aussi comme l'union de deux régions du premier type $D_1$ et $D_2$,
\[
D_1=\{(x,y)\,:\, -3\leq x \leq -1,\, -\sqrt{2x+6}\leq y \leq \sqrt{2x+6}\},
\]
 et 
\[
D_2=\{(x,y)\,:\, -3\leq x \leq -1, \, x-1\leq y \leq \sqrt{2x+6}\}.
\]
\newcommand{\CaptionFigExampleIntegrationdeux}{La région $D$ de l'exemple \ref{exint2}}
\input{Fig_ExampleIntegrationdeux.pstricks}
\end{example}
%%%%%%%%%%%%%%%%%%%%%%%%%%%%%%%%%%%%%%%%%%%%%%%%%%%%%%%%%%%%%%%%%%%%%%%%%%%%%%%%
\subsection{Intégrales sur des parties de $\eR^3$}
%%%%%%%%%%%%%%%%%%%%%%%%%%%%%%%%%%%%%%%%%%%%%%%%%%%%%%%%%%%%%%%%%%%%%%%%%%%%%%%%
Dans ces notes nous n'avons pas l'ambition de traiter d'une façon rigoureuse l'étude des ensemble mesurables de $\eR^3$. Comme dans la section précédente on se limitera à considérer des cas particuliers. 
\begin{definition}\label{primotipo_solida}
	Soit $E$ une région de  $\eR^3$. On dit que $E$ est une \defe{région solide de premier type}{premier type!région solide} si $E$ est contenue entre les graphes de deux fonctions continues de $x$ et $y$.
\[
E=\{(x,y,z)\in\eR^3\, \vert \, (x,y)\in A\subset \eR^2, u_1(x,y)\leq z\leq u_2(x,y) \}. 
\]   
\end{definition}
Le sous-ensemble de $A$  de $\eR^2$ qui apparaît dans la définition \ref{primotipo_solida} est la projection (ou l'ombre) de $E$ sur le plan $x$-$y$. 
\begin{example}\label{cornet}
 La région $E$ donnée par une portion de sphère collée à un cône est une région solide de premier type
\[
E=\{(x,y,z)\in\eR^3\, \vert \, (x,y)\in \bar B((0,0),1), \sqrt{x^2+y^2}\leq z\leq \sqrt{1-x^2-y^2} \}. 
\]
L'ombre de $E$ est la boule unité de $\eR^2$. L'ensemble $\sqrt{x^2+y^2}\leq z$ est un cône posé sur sa pointe tandis que l'ensemble $z\leq\sqrt{ 1-x^2-y^2 }$ est la demi-sphère. L'ensemble $E$ contient les points entre les deux, voir la figure \ref{LabelFigCornetGlace}.
\newcommand{\CaptionFigCornetGlace}{Il faut voir ça en trois dimensions.}
\input{Fig_CornetGlace.pstricks}

\end{example}
Si la fonction $f$, à intégrer sur $E$, et ses sections sont intégrables  alors on peut réduire l'intégrale 
\begin{equation}
  \begin{aligned}
     \int_E  f(x,y,z)\, dV&=\int_A\left(\int_{u_1(x,y)}^{u_2(x,y)}f(x,y,z)\, dz \right) \, dV=\\
&=\int_A\left(F(x,y,u_2(x,y))-F(x,y,u_1(x,y))\right)\, dV,
  \end{aligned}
\end{equation}
où $F$ est une primitive de $f$ par rapport à la variable $z$, c'est à dire en considérant $x$ et $y$ comme des constantes. Il faut ensuite évaluer la partie qui reste comme dans la section précédente. Comme le calcul des aires  dans $\eR^2$, le calcul des volumes dans $\eR^3$ est fait par des intégrales. En fait le \defe{volume}{volume!d'une région solide} d'une région solide dans $\eR^3$ est sa mesure. 
\begin{definition}
   La mesure d'une région de  $\eR^3$ est l'intégrale de sa fonction caractéristique. 
\end{definition}
Soit $E$ une région solide du premier type, nous pouvons évaluer son volume par l'intégrale
\[
\int_A\left(u_2(x,y)-u_1(x,y)\right)\, dV.
\]  
Parfois c'est plus intéressant de calculer le volume avec la formule de réduction contraire : l'intégrale double d'abord et puis l'intégrale simple par rapport à $z$. On parle alors de calcul de volume «par tranche».
\begin{example}
On veut calculer le volume de la boule de rayon $a$, centrée à l'origine $B=\{(x,y,z)\in\eR^3\,\vert\, x^2+y^2+z^2\leq a^2 \}$. On peut décrire $B$ par
\[
  B=\left\{(x,y,z)\in\eR^3\,\vert\, (x,y)\in D_a, -\sqrt{a^2-x^2-y^2}\leq z\leq \sqrt{a^2-x^2-y^2}  \right\},
\]
où $D_a$ est le disque de rayon $a$ centré en $(0,0)$, donc le volume $B$ sera
\[
2 \int_{D_a}\sqrt{a^2-x^2-y^2} dV.
\] 
Cet intégrale est un peu ennuyeuse à calculer. On peut simplifier le calcul en observant que pour $\bar z$ fixé dans l'intervalle $[-a,a]$ la section de la boule au niveau $\bar z$ est un disque de rayon $\sqrt{a^2-z^2}$. L'aire d'un tel disque est  $\pi (a^2+z^2)$. Si on réduit l'intégrale de volume de la façon
\[
\int_{B} 1\, dV=\int_{-a}^{a}  \sqrt{a^2-z^2}\, dz,
\] 
on obtient tout de suite la valeur cherchée : le volume de $B$ est $4/3 \pi a^3$.   
\end{example}
\begin{example}
	On calcul l'intégrale de $f(x,y,z)=z$ sur la pyramide $P$ bornée par le plans $x=0$, $y=0$, $x+y+z=1$, $x+y+z/2=1$. On remarque tout de suite que le plans $x+y+z=1$, $x+y+z/2=1$ se coupent en la droite $x+y=1$, $z=0$ (on se souvient qu'\emph{une} droite dans $\eR^3$, c'est \emph{deux} équations). Cela veut dire que la projection de $P$ sur le plan $x$-$y$ est le  triangle $T$ borné par les droites $x=z=0$, $y=z=0$ et $x+y=1$, $z=0$.  
On  décrit donc $P$ par
\[
P=\{(x,y,z)\in\eR^3\,\vert\, (x,y)\in T, \, 1-2x-2y\leq z\leq 1-x-y\}
\] 
et $T$ par 
\[
T=\{(x,y)\in\eR^2\,\vert\, 0\leq x\leq 1,\,  0\leq y\leq 1-x\},
\]
donc l'intégrale de $f$ sur $P$ est 
\[
\int_pf(x,y,z)\, dV= \int_{0}^{1}\int_{0}^{1-x}\int_{1-2x-2y}^{1-x-y}z \,dz\,dy\,dx=-\frac{1}{ 24 }.
\]
Notez que lorsque $x$ et $y$ sont entre $0$ et $1$, nous avons bien $1-2x-2y<1-x-y$, d'où le fait que nous mettons $1-2x-2y$ dans la borne inférieure de l'intégrale.
\end{example}
De façon analogue on définit les régions solides du deuxième et du troisième type.  
%+++++++++++++++++++++++++++++++++++++++++++++++++++++++++++++++++++++++++++++++++++++++++++++++++++++++++++++++++++++++++++++++++++++++++++++++++++++++++++++++++++++++++++++++++++++++++++++++++++++++++++++++++++++++++++++++++++++++++++++++++++++++++++++++++++++++++++++++++++
\section{Coordonnées polaires, cylindriques et sphériques}\label{sec_coord}
%+++++++++++++++++++++++++++++++++++++++++++++++++++++++++++++++++++++++++++++++++++++++++++++++++++++++++++++++++++++++++++++++++++++++++++++++++++++++++++++++++++++++++++++++++++++++++++++++++++++++++++++++++++++++++++++++++++++++++++++++++++++++++++++++++++++++++++++++++++
\subsection{Coordonnées polaires}
Soit $T$ la fonction de $]0, +\infty[\times \eR$ dans $\eR^2\setminus\{(0,0)\}$ définie par
\begin{equation}
  \begin{array}{lccc}
    T: &]0, +\infty[\times \eR & \to & \eR^2\setminus\{(0,0)\}\\
 & (r, \theta)&\mapsto& (r\cos \theta, r \sin \theta),
  \end{array}
\end{equation}
Cette fonction est surjective. Elle est bijective sur chaque bande de la forme  $]0, +\infty[\times [a-\pi,a+\pi[$. Si $a=0$ l'inverse de $T$  est la fonction $T^{-1}(x,y)= (\sqrt{x^2+y^2}, \arctg (y/x))$. Soit $P=(x,y)$ un élément dans $\eR^2$, on dit que $r=\sqrt{x^2+y^2}$ est le rayon de $P$ et que $\theta=\arctg (y/x) $ est son argument principal. L'origine ne peut pas être décrite en coordonnées polaires parce que si son rayon est manifestement zéro, on ne peut pas lui associer une valeur univoque de l'angle $\theta$. 
\begin{example}
L'équation du cercle de rayon $a$ et centre $(0, 0)$ en coordonnées polaires est $r=a$. 
\end{example}

\begin{example}
	Une équation possible pour la demi-droite $x=y$, $x>0$,  est $\theta=\pi/4$.         
\end{example}

%++++++++++++++++++++++++++++++++++++++++++++++++++++++++++++++++++++++++   
\subsection{Coordonnées cylindriques}
%++++++++++++++++++++++++++++++++++++++++++++++++++++++++++++++++++++++++
Soit $T$ la fonction de $]0, +\infty[\times \eR^2$ dans $\eR^3\setminus\{(0,0,0)\}$ définie par
\begin{equation}
  \begin{array}{lccc}
    T: &]0, +\infty[\times \eR\times \eR & \to & \eR^3\setminus\{(0,0,0)\}\\
 & (r, \theta, z)&\mapsto& (r\cos \theta, r \sin \theta, z),
  \end{array}
\end{equation}
Cette fonction est surjective. Elle est bijective sur chaque bande de la forme  $]0, +\infty[\times [a-\pi,a+\pi[\times \eR$, $a$ dans $\eR$. Il n'y a presque rien de nouveau par rapport aux coordonnées polaires. Les coordonnées  cylindriques sont intéressantes si on décrit un objet invariant par rapport aux rotations autour de l'axe des $z$. 

\begin{example}
Il faut savoir ce que décrivent les équations les plus simples en coordonnées cylindriques, 
\begin{itemize}
\item $r\leq a$, pour $a$ constant dans  $]0, +\infty[$, est le cylindre de hauteur infinie qui a pour axe l'axe des $z$ et pour base le disque de rayon $a$ centré à l'origine, 
\item $r= a$ est  la surface du cylindre,
\item $\theta = b$ est un demi-plan ouvert et sa fermeture contient l'axe des $z$,
\item $z=c$ est un plan parallèle au plan $x$-$y$. 
\end{itemize}
\end{example}

\begin{example}
  Un demi-cône qui a  son sommet en l'origine et  pour axe l'axe des $z$ est décrit par $z=d r$.  Si $d$ est positif  il s'agit  de la moitié supérieure du cône, si $d<0$ de la moitié inférieure.
\end{example}

\begin{example}
 De même,  la sphère de rayon $a$ et centrée à l'origine est l'assemblage des calottes $z=\sqrt{a^2-r^2}$ et $z=-\sqrt{a^2-r^2}$. 
\end{example}
%++++++++++++++++++++++++++++++++++++++++++++++++++++++++++++++++++++++++   
\subsection{Coordonnées sphériques}
%++++++++++++++++++++++++++++++++++++++++++++++++++++++++++++++++++++++++
Soit $T$ la fonction de $]0, +\infty[\times \eR^2$ dans $\eR^3\setminus\{(0,0,0)\}$ définie par
\begin{equation}
  \begin{array}{lccc}
    T: &]0, +\infty[\times \eR\times \eR & \to & \eR^3\setminus\{(0,0,0)\}\\
 & (\rho, \theta, \phi)&\mapsto& (\rho\cos \theta\sin \phi, \rho \sin \theta\sin \phi, \rho\cos \phi),
  \end{array}
\end{equation}
Cette fonction est surjective. Elle est bijective sur chaque bande de la forme  $]0, +\infty[\times [a-\pi,a+\pi[\times [b-\pi/2, b+\pi/2[$, $a$ et $b$ dans $\eR$.  Si $a =0$ et $b=-\pi/2$ la fonction inverse $T^{-1}$ est donnée donnée
\begin{equation}
  \begin{array}{lccc}
    T: &\eR^3\setminus\{(0,0,0)\} & \to & ]0, +\infty[\times [-\pi,\pi[\times [0, \pi[\\
 & (x,y,z)&\mapsto& \left(\sqrt{x^2+y^2+z^2}, \arctg \frac{y}{x}, \arccos \left(\frac{z}{\sqrt{x^2+y^2+z^2}}\right)\right). 
  \end{array}
\end{equation}
Soit $ P$ un point dans $\eR^3$. L'angle $\phi$ est l'angle entre le demi-axe positif des $z$ et le vecteur $\overrightarrow{OP}$, $\rho$ est la norme de $\overrightarrow{OP}$ et $\theta$ est l'argument en coordonnées polaires de la projection de $\overrightarrow{OP}$ sur le plan $x$-$y$.  

\begin{remark}
	Dans la littérature, les angles $\theta$ et $\phi$ sont parfois inversés (voire, changent de nom, par exemple $\varphi$ au lieu de $\phi$). Il faut donc être très prudent lorsqu'on veut utiliser dans un cours des formules données dans un autre cours.
\end{remark}

\begin{example}
Il faut connaître le sens des équations plus simples, 
\begin{itemize}
\item $\rho\leq a$, pour $a$ constant dans  $]0, +\infty[$, est la boule fermée de rayon $a$ centrée à l'origine, 
\item $\rho= a$ est  la sphère de rayon $a$ centrée à l'origine,
\item $\theta = b$ est un demi-plan ouvert et sa fermeture contient l'axe des $z$,
\item $\phi= c$ est un demi-cône qui a  son sommet à l'origine et  pour axe l'axe des $z$.  Si $c$ est positif  il s'agit  de la moitié supérieure du cône, si $d<0$ de la moitié inférieure. 
\end{itemize}
 \end{example}

%++++++++++++++++++++++++++++++++++++++++++++++++++++++++++++++++++++++++++++++++++++++++++++++++++++++++++++++++++++++++++++++++++++++++++++++++++++++++++++++++++++++++++++++++++
\section{Changement de variables}
%++++++++++++++++++++++++++++++++++++++++++++++++++++++++++++++++++++++++++++++++++++++++++++++++++++++++++++++++++++++++++++++++++++++++++++++++++++++++++++++++++++++++++++++++++
\begin{theorem}		\label{ThoChmVarInt}
  Soient $U$ et $V$ deux ouverts bornés de $\eR^p$, $\phi$ un difféomorphisme de classe $\mathcal{C}^1$ de $U$ sur $V$ et $f$ une fonction intégrable de $V$ sur $\eR$. Alors nous avons la formule de changement de variables 
  \begin{equation}
    \int_{V}f(y)\, dy= \int_{U} f(\phi(x))\, \left| J_{\phi}(x)\right|\, dx,
  \end{equation}
  où $J_{\phi}$ est le déterminant de la matrice jacobienne\index{jacobienne} de $\phi$. 
\end{theorem}
Si $\phi$ est linéaire  alors le facteur $|J_{\phi}|$ est la mesure de l'image par $\phi$ d'une portion de $\eR^p$ de mesure $1$, sinon  $|J_{\phi}|$ est le rapport entre la mesure de l'image d'un élément infinitésimale de volume de $\eR^p$ et sa mesure originale. 
Soit $\phi(u,v)=g(u,v)e_1+h(u,v)e_2$ un difféomorphisme dans $\eR^2$. Soit $(x_0, y_0)$ l'image par $\phi$ de $(u_0,v_0)$. On considère le petit rectangle $R$ de sommets $(u_0,v_0)$, $(u_0+\Delta u,v_0)$, $(u_0+\Delta u,v_0+\Delta v)$ et $(u_0,v_0+\Delta v)$. L'image de $R$ n'est pas un rectangle en général, mais peut être bien approximée par le rectangle de sommets $(x_0,y_0)$, $(x_0 ,y_0)+ \phi_{u}\Delta u$, $(x_0 ,y_0)+\phi_{u}\Delta u +\phi_{v}\Delta v$ et  $(x_0 ,y_0)+ \phi_{v}\Delta v$ et son aire est $\| \phi_{u}\times \phi_{v}\| \Delta u\Delta v$. La valeur $|\phi_{u}\times \phi_{v}|$ est exactement $|J_{\phi}|$ 

\begin{example}
Soit $V$ la région trapézoïdale de sommets $(0,-1)$, $(1,0)$, $(2,0)$, $(0,-2)$, comme à la figure \ref{LabelFigexamplechangementvariablesssvecchiaregione}. Calculons ensemble l'intégrale double  
\[
\int_{V}e^{\frac{x+y}{x-y}}\,dV,
\] 
avec le changement de variable $\psi(x,y)=(x+y,x-y)$. C'est à dire que nous considérons les nouvelles variables
\begin{subequations}
	\begin{numcases}{}
		u=x+y\\
		v=x-y.
	\end{numcases}
\end{subequations}
Il faut remarquer d'abord que le changement de variable proposé est dans le mauvais sens. On écrit alors $\phi(u,v)=\psi^{-1}(u,v)=\big((u+v)/2, (u-v)/2\big)$, c'est à dire
\begin{subequations}
	\begin{numcases}{}
		x=\frac{ u+v }{ 2 }\\
		y=\frac{ u-v }{2}.
	\end{numcases}
\end{subequations}
La région qui correspond à $V$ est $U$, le trapèze de sommets  $(-1,1)$, $(1,1)$, $(2,2)$ et $(-2,2)$, qu'on voit sur la figure \ref{LabelFigexamplechangementvariablesssnuovaregione} et qu'on décrit par
\[
U=\{ (u,v)\in\eR^2\,\vert\, 1\leq v\leq 2, \, -v\leq u\leq v\}.
\] 
%\ref{LabelFigexamplechangementvariables}
\newcommand{\CaptionFigexamplechangementvariables}{Avant et après le changement de variables}
\input{Fig_examplechangementvariables.pstricks}

On observe que $U$ est une région du premier type tandis que $V$ n'est pas du premier ou du deuxième type. Le déterminant de la  matrice  jacobienne de $\psi^{-1}$ est  $J_{\psi^{-1}}$,
\begin{equation}
 J_{\psi^{-1}}(u,v)= \left\vert\begin{array}{cc}
\frac{1}{2} & \frac{1}{2} \\
\frac{1}{2}  & -\frac{1}{2}
\end{array}\right\vert= -\frac{1}{2}.
\end{equation}
On a alors 
\[
\int_{V}e^{\frac{x+y}{x-y}}\,dV=\int_{U}e^{\frac{u}{v}}\,\frac{1}{2}\,dV=\int_1^2\int_{-v}^{v}e^{\frac{u}{v}}\,\frac{1}{2}\, du\,dv= \frac{3}{4}(e-e^{-1}).
\] 
\end{example}

\begin{example} 
\textbf{Coordonnées polaires : }On veut évaluer l'intégrale de la fonction $f(x,y)= x^2+y^2$ sur la région $V$ suivante :
\[
V=\{(x,y) \in \eR^2\,\vert\, x^2+y^2\leq 1,\, x>0,\, y>0\}.
\]
On peut faire le calcul directement,
\[
\int_{V}f(x,y)\, dV=\int_0^1\int_0^{\sqrt{1-x^2}}x^2+y^2\, dy\,dx=\int_0^1x^2\sqrt{1-x^2} + \frac{(1-x^2)^{3/2}}{3}\, dx  
\] 
mais c'est un peu ennuyeux. On peut simplifier beaucoup les calculs avec un changement de variables vers les coordonnées polaires. Dans ce cas, on sait bien que le difféomorphisme à utiliser est $\phi(r,\theta)=(r\cos \theta, r\sin\theta)$. Le jacobien  $J_{\phi}$ est
\begin{equation}
 J_{\phi}(r, \theta)= \left\vert\begin{array}{cc}
\cos \theta & \sin \theta \\
-r\sin \theta  & r\cos \theta
\end{array}\right\vert= r,
\end{equation}
qui est toujours positif. La fonction $f$ peut s'écrire comme $f(\phi(r,\theta))=r^2$ et $\phi^{-1}(V)=]0,1]\times]0, \pi/2[$.  
La formule du changement de variables nous donne
\[
\int_{V}f(x,y)\, dV=\int_0^{\pi/2}\int_0^{1}r^3 dr\,d\theta=\int_0^{\pi/2}\frac{1}{4}\,d\theta=\frac{\pi}{8}.  
\] 
\end{example}
\begin{example}
\textbf{Coordonnées cylindriques : }On veut calculer le volume de la région $A$ définie par  l'intersection entre la boule unité et le cylindre qui a pour base un disque de rayon $1/2$ centré en $(0, 1/2)$
\[
A=\{(x,y,z) \in\eR^3 \,\vert\, x^2+y^2+z^1\leq 1\}\cap\{(x,y,z) \in \eR^3\,\vert\, x^2+(y-1/2)^2\leq 1/4\}.
\]
On peut décrire $A$ en coordonnées cylindriques
\begin{equation}
  \begin{aligned}
    A=\Big\{(r,\theta,z) &\in ]0, +\infty[\times [-\pi,\pi[\times \eR\,\vert\,\\
& -\pi/2<\theta<\pi, \, 0<r\leq \sin\theta, \, -\sqrt{1-r^2}\leq z\leq\sqrt{1-r^2} \Big\}.
  \end{aligned}
\end{equation}
Le jacobien de ce changement de variables,  $J_{cyl}$, est
\begin{equation}
 J_{cyl}(r, \theta), z= \left\vert\begin{array}{ccc}
\cos \theta & \sin \theta & 0\\
-r\sin \theta  & r\cos \theta &0 \\
0&0&
\end{array}\right\vert= r,
\end{equation}
qui est toujours positif. Le volume de $A$ est donc
\[
\int_{\eR^3}\chi_{A}(x,y,z)\, dV=\int_{-\pi/2}^{\pi/2}\int_0^{\sin\theta}\int_{-\sqrt{1-r^2}}^{\sqrt{1-r^2}} r dz\,dr\,d\theta=\frac{2\pi}{8}+\frac{8}{9}.  
\] 
\end{example}
\begin{example}
\textbf{Volume d'un solide de révolution : }Soit $g:[a,b]\to\eR_+$ une fonction continue et positive. On dit que le solide $A$ décrit par
\[
A=\left\{(x,y,z)\in\eR^3\, \vert \, z\in[a,b], \,\sqrt{x^2+y^2}\leq g^2(z) \right\}
\]
est un solide de révolution. Afin de calculer son volume, on peut décrire $A$ en coordonnées cylindriques, 
\[
A=\left\{(r,\theta,z) \in ]0, +\infty[\times [-\pi,\pi[\times \eR\,\vert\, a\leq z\leq b, \, 0<r^2\leq g^2(z) \right\}.
\]
Le jacobien de ce changement de variables est  $J_{cyl}=r$, comme dans l'exemple précédent. Le volume de $A$ est donc
\[
\int_{\eR^3}\chi_{A}(x,y,z)\, dV=\int_a^{b}\int_{-\pi}^{\pi}\int_{0}^{g(z)} r  \,dr\,d\theta\, dz=\int_a^{b} \pi g^2(z) \, dz.
\] 
Cette formule peut être utilisée pour tout solide de révolution. 
\end{example}

\begin{example}
\textbf{Coordonnées sphériques : }On veut calculer le volume du cornet de glace  $A$ 
\[
A=\left\{(x,y,z)\in\eR^3\, \vert \, (x,y)\in \mathbb{S}^2, \,\sqrt{x^2+y^2}\leq z\leq \sqrt{1-x^2-y^2} \right\}. 
\]
On peut décrire $A$ en coordonnées sphériques. 
\[
A=\{(\rho,\theta,\phi) \in ]0, +\infty[\times [-\pi,\pi[\times [0,\pi[\,\vert\, 0<\phi\leq\pi/4, \, 0<\rho\leq 1 \}.
\]
Le jacobien de ce changement de variables  $J_{sph}$ est
\begin{equation}
 J_{sph}(\rho, \theta, \phi)= \left\vert\begin{array}{ccc}
\cos \theta \sin\phi & \sin \theta\sin\phi & \cos\phi\\
-\rho\sin \theta\sin\phi  & \rho\cos \theta\sin\phi & 0 \\
\rho\cos\theta\cos\phi&\rho\sin\theta\cos\phi& -\rho\sin\phi
\end{array}\right\vert= \rho^2\sin\phi,
\end{equation}
Le volume de $A$ est donc
\[
\int_{\eR^3}\chi_{A}(x,y,z)\, dV=\int_{-\pi}^{\pi}\int_0^{\pi/4}\int_{0}^{1}\rho^2\sin\phi \,d\rho\,d\phi\,d\theta=\frac{2\pi}{3}\left(1-\frac{1}{\sqrt{2}}\right).  
\] 
\end{example}

%---------------------------------------------------------------------------------------------------------------------------
\subsection{Récapitulatif}
%---------------------------------------------------------------------------------------------------------------------------

En pratique, nous retiendrons les formules suivantes:
%///////////////////////////////////////////////////////////////////////////////////////////////////////////////////////////
\subsubsection{Coordonnées polaires}
%///////////////////////////////////////////////////////////////////////////////////////////////////////////////////////////

\begin{subequations}
    \begin{numcases}{}
        x=r\cos\theta\\
        y=r\sin\theta
    \end{numcases}
\end{subequations}
avec \( r\in\mathopen] 0 , \infty \mathclose[\) et \( \theta\in\mathopen[ 0 , 2\pi [\). Le jacobien vaut \( r\).

%///////////////////////////////////////////////////////////////////////////////////////////////////////////////////////////
\subsubsection{Coordonnées cylindriques}
%///////////////////////////////////////////////////////////////////////////////////////////////////////////////////////////

\begin{subequations}
    \begin{numcases}{}
        x=r\cos\theta\\
        y=r\sin\theta\\
        z=z
    \end{numcases}
\end{subequations}
avec \( r\in\mathopen] 0 , \infty \mathclose[\), \( \theta\in\mathopen[ 0 , 2\pi [\) et \( z\in\eR\). Le jacobien vaut \( r\).

%///////////////////////////////////////////////////////////////////////////////////////////////////////////////////////////
\subsubsection{Coordonnées sphériques}
%///////////////////////////////////////////////////////////////////////////////////////////////////////////////////////////

\begin{subequations}
    \begin{numcases}{}
        x=\rho\cos\theta\sin\phi\\
        y=\rho\sin\theta\sin\phi\\
        z=\rho\cos\phi
    \end{numcases}
\end{subequations}
avec \( \rho\in\mathopen] 0 , \infty \mathclose[\), \( \theta\in\mathopen[ 0 , 2\pi [\) et \( \phi\in\mathopen[ 0 , \pi [\). Le jacobien vaut \( -\rho^2\sin\phi\). 

N'oubliez pas que lorsqu'on effectue un changement de variables dans une intégrale, la \emph{valeur absolue} du jacobien apparaît.

%+++++++++++++++++++++++++++++++++++++++++++++++++++++++++++++++++++++++++++++++++++++++++++++++++++++++++++++++++++++++++++
					\section{Intégrales multiples}
%+++++++++++++++++++++++++++++++++++++++++++++++++++++++++++++++++++++++++++++++++++++++++++++++++++++++++++++++++++++++++++

%%%%%%%%%%%%%%%%%%%%%%%%%%%%%%%%%%%
%  NOTE : toute cette partie a été reprise dans OutilsMath le 3 avril 2011.

Il est expliqué par-ci par-là comment on définit le nombre
\begin{equation}
	\int_Ef
\end{equation}
lorsque $E\subset\eR^n$ et $f\colon \eR^n\to \eR$. Nous allons maintenant montrer comment calculer des intégrales en pratique.

%---------------------------------------------------------------------------------------------------------------------------
					\subsection{Changement de variables}
%---------------------------------------------------------------------------------------------------------------------------

Le domaine $E=\{ (x,y)\in\eR^2\tq x^2+y^2<1 \}$ s'écrit plus facilement $E=\{ (r,\theta)\tq r<1 \}$ en coordonnées polaires. Le passage aux coordonnées polaire permet de transformer une intégration sur un domaine rond à une intégration sur le domaine rectangulaire $\mathopen]0,2\pi\mathclose[\times\mathopen]0,1\mathclose[$. La question est évidement de savoir si nous pouvons écrire
\begin{equation}
	\int_Ef=\int_{0}^{2\pi}\int_0^1f(r\cos\theta,r\sin\theta)drd\theta.
\end{equation}
Hélas, non; la vie n'est pas aussi simple.

\begin{theorem}
Soit $g\colon A\to B$ un difféomorphisme. Soient $F\subset B$ un ensemble mesurable et borné et $f\colon F\to \eR$ une fonction bornée et intégrable. Supposons que $g^{-1}(F)$ soit borné et que $Jg$ soit borné sur $g^{-1}(F)$. Alors
\begin{equation}
	\int_Ff(x)dy=\int_{g^{-1}(F)f\big( g(x) \big)}| Jg(x) |dx
\end{equation}
\end{theorem}
Pour rappel, $Jg$ est le déterminant de la matrice \href{http://fr.wikipedia.org/wiki/Matrice_jacobienne}{jacobienne} (aucun lien de \href{http://fr.wikipedia.org/wiki/Jacob}{parenté}) donnée par
\begin{equation}
	Jg=\det\begin{pmatrix}
	\partial_xg_1	&	\partial_yg_1	\\ 
	\partial_xg_2	&	\partial_tg_2	
\end{pmatrix}.
\end{equation}
Un \defe{difféomorphisme}{difféomorphisme} est une application $g\colon A\to B$ telle que $g$ et $g^{-1}\colon B\to A$ soient de classe $C^1$.

%///////////////////////////////////////////////////////////////////////////////////////////////////////////////////////////
					\subsubsection{Coordonnées polaires}
%///////////////////////////////////////////////////////////////////////////////////////////////////////////////////////////

Les coordonnées polaires sont données par le difféomorphisme
\begin{equation}
	\begin{aligned}
		g\colon \mathopen]0,\infty\mathclose[\times\mathopen]0,2\pi\mathclose[ &\to\eR^2\setminus D\\
		(r,\theta)&\mapsto \big( r\cos(\theta),r\sin(\theta) \big)
	\end{aligned}
\end{equation}
où $D$ est la demi droite $y=0$, $x\geq 0$. Le fait que les coordonnées polaires ne soient pas un difféomorphisme sur tout $\eR^2$ n'est pas un problème pour l'intégration parce que le manque de difféomorphisme est de mesure nulle dans $\eR^2$. Le jacobien est donné par
\begin{equation}
	Jg=\det\begin{pmatrix}
	\partial_rx	&	\partial_{\theta}x	\\ 
	\partial_ry	&	\partial_{\theta}y
\end{pmatrix}=\det\begin{pmatrix}
	\cos(\theta)	&	-r\sin(\theta)	\\ 
	\sin(\theta)	&	r\cos(\theta)	
\end{pmatrix}=r.
\end{equation}

Voir l'exemple \ref{ExpmfDtAtV}.

%///////////////////////////////////////////////////////////////////////////////////////////////////////////////////////////
\subsubsection{Coordonnées sphériques}
%///////////////////////////////////////////////////////////////////////////////////////////////////////////////////////////

Voir le point \ref{SubSubCoordSpJxhMwm}

%+++++++++++++++++++++++++++++++++++++++++++++++++++++++++++++++++++++++++++++++++++++++++++++++++++++++++++++++++++++++++++
\section{Formes différentielles et son intégrale sur un chemin}
%+++++++++++++++++++++++++++++++++++++++++++++++++++++++++++++++++++++++++++++++++++++++++++++++++++++++++++++++++++++++++++

%---------------------------------------------------------------------------------------------------------------------------
\subsection{Forme différentielle}
%---------------------------------------------------------------------------------------------------------------------------

La formule d'intégration d'un champ de vecteur,
\begin{equation}
	\int_{\gamma}G=\int_{[a,b]}\langle G (\gamma(t)), \gamma'(t)\rangle dt,
\end{equation}
contient quelque chose d'intéressant : la combinaison $\langle G( \gamma(t) ), \gamma'(t)\rangle$. Cette combinaison sert à transformer le vecteur tangent $\gamma'(t)$ en un nombre en utilisant le produit scalaire avec le vecteur $G( \gamma(t) )$.

Si $G$ est un champ de vecteur sur $\eR^n$, et si $x\in\eR^n$, nous pouvons considérer, de façon un peu plus abstraite, l'application
\begin{equation}		\label{EqDefBemol}
	\begin{aligned}[]
		G^{\flat}_x\colon \eR^n&\to \eR \\
			v&\mapsto \langle G(x), v\rangle . 
	\end{aligned}
\end{equation}
Cela permet de compactifier la notation et écrire
\begin{equation}
	\int_{\gamma}G=\int_{[a,b]} G^{\flat}_{\gamma(t)}\big( \gamma'(t)\big) dt.
\end{equation}

Nous nous proposons maintenant d'étudier plus en détail ce qu'est l'objet $G^{\flat}$. La règle \eqref{EqDefBemol} dit que pour chaque $x$, l'application $G_x^{\flat}$ est une forme sur $\eR^n$, c'est à dire une application linéaire de $\eR^n$ vers $\eR$. Nous écrivons que
\begin{equation}
	G_x^{\flat}\in\big( \eR^n \big)^*.
\end{equation}
Nous connaissons la \defe{base duale}{base!duale} de $(\eR^n)^*$, ce sont les formes $e^*_i$ définies par $e^*_i(e_j)=\delta_{ij}$. Dans le cadre du cours d'analyse, nous allons noter ces formes\footnote{Parce que ce sont les différentielles des fonctions (projections)
\begin{equation}
	\begin{aligned}
			x_i\colon \eR^n&\to \eR \\
			x&\mapsto x_i 
		\end{aligned}
	\end{equation}
}
par $dx_i$ :
\begin{equation}
	\begin{aligned}[]
		e^*_1&=dx_1\colon v\mapsto v_1	\\
			&\vdots			\\
		e^*_n&=dx_n\colon v\mapsto v_n
	\end{aligned}
\end{equation}
Étant donné que ces $dx_i$ forment une base de l'espace vectoriel $(\eR^n)^*$, toute application linéaire $L\colon \eR^n\to \eR$ s'écrit
\begin{equation}
	\begin{aligned}[]
		Lv&=a_1v_1+\ldots+a_nv_n\\
			&=a_1dx_1(v)+\ldots+a_ndx_n(v).
	\end{aligned}
\end{equation}
Plus abstraitement, nous notons
\begin{equation}
	\begin{aligned}[]
		L&=a_1dx_1+\ldots+a_ndx_n\\
		&=\sum_{i=1}^na_idx_i.
	\end{aligned}
\end{equation}
L'application $L$ est une combinaison linéaire des $dx_i$ au sens de l'espace vectoriel $(\eR^n)^*$.

L'objet $G^{\flat}$ est la donnée, en chaque point de $D$, d'une telle forme sur $\eR^n$. Nous donnons alors la définition suivante.
\begin{definition}
	Soit $D$, un domaine dans $\eR^n$. Une $1$-\defe{forme différentielle}{forme!différentielle} $\omega$ sur $D$ est une application
	\begin{equation}
		\begin{aligned}
				\omega\colon D&\to (\eR^n)^* \\
				x&\mapsto \omega_x. 
			\end{aligned}
		\end{equation}
\end{definition}
Étant donné que $\{ dx_i \}$ est une base de $(\eR^n)^*$, pour chaque $x\in D$, il existe des uniques réels $a_i(x)$ tels que
\begin{equation}
	\omega_x=a_1(x)dx_1+\ldots+a_n(x)dx_n.
\end{equation}
Nous disons qu'une $1$-forme différentielle est \defe{continue}{continue!forme différentielle} si les fonctions $a_i$ sont continues. La forme sera $C^k$ quand les $a_i$ seront $C^k$.

\begin{remark}
	L'ensemble des $1$-formes différentielles forment un espace vectoriel avec les définitions
	\begin{equation}
		\begin{aligned}[]
			(\lambda\omega)_x(v)&=\lambda\omega_x(v)\\
			(\omega+\mu)_x(v)&=\omega_x(v)+\mu_x(v).
		\end{aligned}
	\end{equation}
\end{remark}

Lorsque une $1$-forme différentielle s'écrit toujours sous la forme
\begin{equation}
	\omega=\sum_i a_idx_i
\end{equation}
pour certaines fonctions $a_i$. Évidemment, ces fonctions $a_i$ peuvent être trouvées en appliquant $\omega$ aux éléments de la base canonique de $\eR^n$ :
\begin{equation}
	a_j(x)=\omega_x(e_j)
\end{equation}
parce que $\omega_x(e_j)=\sum_ia_i(x)dx_i(e_i)=\sum_ia_i(x)\delta_{ij}=a_j(x)$.

Nous pouvons ainsi déterminer le développement de $G^{\flat}$ dans la base des $dx_i$ en faisant le calcul
\begin{equation}
	G_x^{\flat}(e_i)=\langle G(x), e_i\rangle =G_i(x),
\end{equation}
donc les composantes de $G^{\flat}$ dans la base $dx_i$ sont exactement les composantes de $G$ dans la base $e_i$ :
\begin{equation}
	G^{\flat}_x=G_1(x)dx_1+\ldots+G_n(x)dx_n.
\end{equation}


%///////////////////////////////////////////////////////////////////////////////////////////////////////////////////////////
\subsubsection{Une petite note pour titiller monsieur Jean Doyen}
%///////////////////////////////////////////////////////////////////////////////////////////////////////////////////////////

Pensons pendant quelque minutes aux fonctions de $\eR$ dans $\eR$. Monsieur Jean Doyen dit toujours que quand le sage demande la fonction $f$, le simple dit \og $f(x)$\fg. Or $f(x)$ n'est pas une fonction; c'est $f$, la fonction. En poussant un petit peu nous pouvons prétendre que $x$ désigne la fonction identité qui à chaque $x$ fait correspondre $x$ lui-même. Il est d'ailleurs un peu normal de désigner comme ça cette fonction. Dans ce cas, $f(x)$ désigne la fonction composée de la fonction $f$ avec la fonction $x$, et tout le monde est content\footnote{J'offre un pot à qui ose écrire que $f(x)$ est bien la \emph{fonction} composée de $f$ avec $x$ sur sa feuille d'examen du cours d'algèbre linéaire.}. .

Avouons que cela est un petit peu de mauvaise foi\footnote{De toutes façons, qui a la foi à l'ULB ?? ;)}. Vraiment ?

La fonction $x$ est une fonction de $\eR$ vers $\eR$. Sa différentielle en un point est donc une application de $\eR$ vers $\eR$. Devinez ce qu'elle vaut ? Ben oui : la différentielle de la fonction $x$ est \emph{vraiment} le $dx$ qu'on écrit tout le temps, la forme différentielle, la base de l'espace dual !

%///////////////////////////////////////////////////////////////////////////////////////////////////////////////////////////
\subsubsection{L'isomorphisme musical}
%///////////////////////////////////////////////////////////////////////////////////////////////////////////////////////////

Nous savons qu'un champ de vecteur $G$ produit la forme différentielle $G^{\flat}$. La construction inverse existe également. Si $\omega$ est une $1$-forme différentielle, nous pouvons définir le champ de vecteur $\omega^{\sharp}$ par la formule (implicite)
\begin{equation}
	\omega_x(v)=\langle \omega^{\sharp}(x), v\rangle 
\end{equation}
pour tout $v\in\eR^n$. Par définition, $(\omega^{\sharp})^{\flat}=\omega$. 

\begin{exercice}
	Prouver que, en composantes, 
	\begin{equation}
		\omega^{\sharp}(x)=\big( a_1(x),\ldots,a_n(x) \big),
	\end{equation}
	et vérifier que si $G$ est un champ de vecteurs, alors $(G^{\flat})^{\sharp}=G$.
\end{exercice}

%///////////////////////////////////////////////////////////////////////////////////////////////////////////////////////////
\subsubsection{Formes différentielles exactes et fermées}
%///////////////////////////////////////////////////////////////////////////////////////////////////////////////////////////

Considérons une fonction différentiable $f\colon D\to \eR$. Pour chaque $x\in D$, nous avons l'application différentielle
\begin{equation}
	\begin{aligned}
		df(x)\colon \eR^n&\to \eR \\
		v&\mapsto \sum_{i=1}^n\frac{ \partial f }{ \partial x_i }(x)v_i, 
	\end{aligned}
\end{equation}
c'est à dire que $df$ est une $1$-forme différentielle dont les composantes sont
\begin{equation}
	df(x)=\frac{ \partial f }{ \partial x_1 }(x)dx_1+\ldots+\frac{ \partial f }{ \partial x_n }(x)dx_n.
\end{equation}

Il est naturel de se demander si toutes les formes différentielles sont des différentielles de fonctions. Une réponse complète est délicate à établir, mais a d'innombrables conséquences en physique, notamment en ce qui concerne l'existence d'un potentiel vecteur pour le champ magnétique dans les équations de Maxwell.
\begin{definition}
	Deux classes importantes de formes différentielles sont à mettre en évidence
	\begin{enumerate}
		\item
			Une forme différentielle $\omega$ sur un ouvert $D\subset\eR^n$ est \defe{exacte}{forme!différentielle!exacte} si il existe une fonction différentiable $f\colon D\to \eR$ telle que
			\begin{equation}
				 \omega_x=df(x)
			\end{equation}
			pour tout $x\in D$.
		\item
			Une $1$-forme de classe $C^1$ sur l'ouvert $D$ est \defe{fermée}{forme!différentielle!fermée} si pour tout $i,j=1,\ldots n$, nous avons
			\begin{equation}
				\frac{ \partial a_i }{ \partial x_j }=\frac{ \partial a_j }{ \partial x_i }.
			\end{equation}
	\end{enumerate}
\end{definition}

\begin{proposition}
	Si $\omega$ est une $1$-forme exacte de classe $C^1$, alors $\omega$ est fermée.
\end{proposition}

\begin{proof}
	Le fait que $\omega$ soit exacte implique l'existence d'une fonction $f$ telle que $\omega=df$, c'est à dire
	\begin{equation}
		\omega_x=\sum_i a_i(x)dx_i=\sum_i\frac{ \partial f }{ \partial x_i }(x)dx_i,
	\end{equation}
	c'est à dire que $a_i(x)=\frac{ \partial f }{ \partial x_i }(x)$. L'hypothèse que $\omega$ est $C^1$ implique que $f$ est $C^2$, et donc que nous pouvons inverser l'ordre de dérivation pour les dérivées secondes $\partial^2_{ij}f=\partial^2_{ji}f$. Nous pouvons donc faire le calcul suivant :
	\begin{equation}
		\frac{ \partial a_i }{ \partial x_j }=\frac{ \partial  }{ \partial x_j }\frac{ \partial f }{ \partial x_i }=\frac{ \partial  }{ \partial x_i }\frac{ \partial f }{ \partial x_j }=\frac{ \partial a_j }{ \partial x_i },
	\end{equation}
	ce qu'il fallait démontrer.
\end{proof}

Ceci est assez pour les formes différentielles pour cette année.

%---------------------------------------------------------------------------------------------------------------------------
\subsection{Intégration d'une forme différentielle sur un chemin}
%---------------------------------------------------------------------------------------------------------------------------

Les formes intégrales que nous avons déjà vues sont celles de fonctions et de champs de vecteur sur des chemins. Si $\gamma\colon [a,b]\to \eR^n$ est le chemin, les formules sont
\begin{equation}
	\begin{aligned}[]
		\int_{\gamma}f&=\int_{[a,b]}f\big( \gamma(t) \big)\| \gamma'(t) \|dt\\
		\int_{\gamma}G&=\int_{[a,b]}\langle G\big( \gamma(t) \big), \gamma'(t)\rangle dt.
	\end{aligned}
\end{equation}
Dans les deux cas, le principe est que nous disposons de quelque chose (la fonction $f$ ou le vecteur $G$), et du vecteur tangent $\gamma'(t)$, et nous essayons d'en tirer un nombre que nous intégrons. Lorsque nous avons une $1$-forme, la façon de l'utiliser pour produire un nombre avec le vecteur tangent est évidement d'appliquer la $1$-forme au vecteur tangent. La définition suivante est donc naturelle.

\begin{definition}
	Soit $\gamma\colon [a,b]\to \eR^n$, un chemin de classe $C^1$ tel que son image est contenue dans le domaine $D$. Si $\omega$ es une $1$-forme différentielle sur $D$, nous définissons l'\defe{intégrale de $\omega$ le long de $\gamma$}{intégrale!d'une forme différentielle} le nombre
	\begin{equation}
		\begin{aligned}[]
			\int_{\gamma}\omega&=\int_a^b\omega_{\gamma(t)}\big( \gamma'(t) \big)dt\\
				&=\int_a^b\Big[ a_1\big( \gamma(t) \big)\gamma'_1(t)+\ldots +  a_n\big( \gamma(t) \big)\gamma'_n(t) \Big]dt.
		\end{aligned}
	\end{equation}
\end{definition}

Cette définition est une bonne définition parce que si on change la paramétrisation du chemin, on ne change pas la valeur de l'intégrale, c'est la proposition suivante.
\begin{proposition}
	Si $\gamma$ et $\beta$ sont des chemins équivalents, alors
	\begin{equation}
		\int_{\gamma}\omega=\int_{\beta}\omega,
	\end{equation}
	c'est à dire que l'intégrale est invariante sous les reparamétrisation du chemin.
\end{proposition}
\begin{proof}
	Deux chemins sont équivalents quand il existe un difféomorphisme $C^1$ $h\colon [a,b]\to [c,d]$ tel que $\gamma(t)=(\beta\circ h)(t)$. En remplaçant $\gamma$ par $(\beta\circ h)$ dans la définition de $\int_{\gamma}\omega$, nous trouvons
	\begin{equation}
		\int_a^b\omega_{\gamma(t)}\big( \gamma'(t) \big)dt=\int_a^b\omega_{(\beta\circ h)(t)}\big( (\beta\circ h)'(t) \big)dt.
	\end{equation}
	Un changement de variable $u=h(t)$ transforme cette dernière intégrale en $\int_{\beta}\omega$, ce qui prouve la proposition.
\end{proof}

\begin{remark}
	Si $\gamma$ est une somme de chemins, $\gamma=\gamma^{(1)}+\ldots+\gamma^{(n)}$, où chacun des $\gamma^{(i)}$ est un chemin, alors
	\begin{equation}
		\int_{\gamma}\omega=\sum_{i=1}^n\int_{\gamma_i}\omega
	\end{equation}
	parce que $\omega$ est linéaire.
\end{remark}

\begin{remark}
	Si $-\gamma$ est le chemin
	\begin{equation}
		\begin{aligned}
			- \gamma\colon [a,b]&\to \eR^n \\
			t&\mapsto \gamma\big( b-(t-a) \big),
		\end{aligned}
	\end{equation}
	alors
	\begin{equation}
		\int_{-\gamma}\omega=-\int_{\gamma}\omega,
	\end{equation}
	c'est à dire que si l'on parcours le chemin en sens inverse, alors on change le signe de l'intégrale.
\end{remark}

L'intégrale d'une forme différentielle sur un chemin est compatible avec l'intégrale déjà connue d'un champ de vecteur sur le chemin parce que si $G$ est un champ de vecteurs,
\begin{equation}
	\int_{\gamma}G^{\flat}=\int_{\gamma}G.
\end{equation}
En effet,
\begin{equation}
	\begin{aligned}[]
		\int_{\gamma G^{\flat}}	&=\int_a^b G_{\gamma(t)}^{\flat}(\gamma'(t))\\
					&=\int_a^b\big[ G_1( \gamma(t) )dx_1+\ldots G_n(\gamma(t))dx_n \big]\big( \gamma'_1(t),\ldots,\gamma'_n(t) \big)\\
					&=\int_{a}^b\langle G(\gamma(t)), \gamma'(t)\rangle \\
					&=\int_{\gamma}G.
	\end{aligned}
\end{equation}


\begin{proposition}
	Soit $\omega=df$, une $1$-forme exacte et continue sur le domaine $D$. Alors la valeur de $\int_{\gamma}df$ ne dépend que des valeurs de $f$ aux extrémités de $\gamma$.
\end{proposition}

\begin{proof}
	Nous avons
	\begin{equation}
		\begin{aligned}[]
			\int_{\gamma}\omega=\int_{\gamma}df&=\int_{a}^b\sum_{i=1}n\frac{ \partial f }{ \partial x_i }\big( \gamma(t) \big)\gamma'_i(t)dt\\
				&=\int_a^b\frac{ d }{ dt }\Big( (f\circ\gamma)(t) \Big)dt\\
				&=(f\circ\gamma)(b)-(f\circ\gamma(a)).
		\end{aligned}
	\end{equation}
\end{proof}

%---------------------------------------------------------------------------------------------------------------------------
\subsection{Interprétation physique : travail}
%---------------------------------------------------------------------------------------------------------------------------

\begin{definition}
	Une force $F\colon D\subset\eR^n\to \eR^n$ est \defe{\href{http://fr.wikipedia.org/wiki/Force_conservative}{conservative}}{Conservative} si elle dérive d'un potentiel, c'est à dire si il existe une fonction $V\in C^1(D,\eR)$ telle que 
	\begin{equation}
		F(x)=(\nabla V)(x).
	\end{equation}
\end{definition}
Étant donné que $F$ est un champ de vecteurs, nous avons une forme différentielle associée $F^{\flat}$,
\begin{equation}
	F^{\flat}_x\colon x\mapsto \langle F(x), v\rangle .
\end{equation}

\begin{lemma}
	Le champ $F$ est conservatif si et seulement si la $1$-forme différentielle $F^{\flat}$ est exacte.
\end{lemma}

\begin{proof}
	Supposons que la force $F$ soit conservative, c'est à dire qu'il existe une fonction $V$ telle que $F=\nabla V$. Dans ce cas, il est facile de prouver que $F^{\flat}$ est exacte et est donnée par $F_x^{\flat}=dV(x)$. En effet,
	\begin{equation}
		\begin{aligned}[]
			F_x^{\flat}(v)	&=\langle F(x), v\rangle \\
					&=F_1(x)v_1+\ldots+F_n(x)v_n\\
					&=\frac{ \partial V }{ \partial x_1 }(x)v_1+\ldots\frac{ \partial V }{ \partial x_n }(x)v_n\\
					&=dV(x)v.
		\end{aligned}
	\end{equation}
	
	Pour le sens inverse, supposons que $F^{\flat}$ soit exacte. Dans ce cas, nous avons une fonction $V$ telle que $F^{\flat}=dV$. Il est facile de prouver qu'alors, $F=\nabla V$.
\end{proof}
En résumé, nous avons deux façons équivalentes d'exprimer que la force $F$ dérive du potentiel $V$ :  soit nous disons $F=\nabla V$, soit nous disons $F^{\flat}=dV$.

\begin{proposition}
	Si $F$ est une force conservative, alors le \href{http://fr.wikipedia.org/wiki/Travail_d'une_force}{travail} de $F$ lors d'un déplacement ne dépend pas du chemin suivit.
\end{proposition}

\begin{proof}
	Le travail d'une force le long d'un chemin n'est autre que l'intégrale de la force le long du chemin, et le calcul est facile :
	\begin{equation}
		W_{\gamma}(F)=\int_{\gamma}F=\int_{\gamma}dV=V\big( \gamma(b) \big)-V\big( \gamma(a) \big).
	\end{equation}
	Donc si $\beta$ est un autre chemin tel que $\beta(a)=\gamma(a)$ et $\beta(b)=\gamma(b)$, nous avons $W_{\beta}(F)=W_{\gamma}(F)$.
\end{proof}

%+++++++++++++++++++++++++++++++++++++++++++++++++++++++++++++++++++++++++++++++++++++++++++++++++++++++++++++++++++++++++++
\section{Intégrale sur une variété}
%+++++++++++++++++++++++++++++++++++++++++++++++++++++++++++++++++++++++++++++++++++++++++++++++++++++++++++++++++++++++++++

%---------------------------------------------------------------------------------------------------------------------------
\subsection{Mesure sur une carte}
%---------------------------------------------------------------------------------------------------------------------------

Nous considérons dans cette section uniquement des variétés $M$ de dimension $2$ dans $\eR^3$.  Une particularité de $\eR^3$ (par rapport aux autres $\eR^n$) est qu'il existe le produit vectoriel. 

Si $v$, $w\in\eR^3$, alors le vecteur $v\times w$ est une vecteur normal au plan décrit par $v$ et $w$ qui jouit de l'importante propriété suivante :
\begin{equation}
	\text{aire du parallélogramme}=\| v\times w \|.
\end{equation}
L'aire du parallélogramme construit sur $v$ et $w$ est donnée par la norme du produit vectoriel. Afin de donner une mesure infinitésimale en un point $p\in M$, nous voudrions prendre deux vecteurs tangents à $M$ en $p$, et puis considérer la norme de leur produit vectoriel. Cette idée se heurte à la question du choix des vecteurs tangents à considérer.

Dans $\eR^2$, le choix est évident : nous choisissons $e_x$ et $e_y$, et nous avons $\|e_x\times e_y\|=1$. L'idée est donc de choisir une carte $F\colon W\to F(w)$ autour du point $p=F(w)$, et de choisir les vecteurs tangents qui correspondent à $e_x$ et $e_y$ via la carte, c'est à dire les vecteurs
\begin{equation}
	\begin{aligned}[]
		\frac{ \partial F }{ \partial x }(w),&&\text{et}&&\frac{ \partial F }{ \partial y }(w).
	\end{aligned}
\end{equation}
L'\defe{élément infinitésimal de surface}{Élément de surface} sur $M$ au point $p=F(w)$ est alors défini par
\begin{equation}
	d\sigma_F=\|  \frac{ \partial F }{ \partial x }(w)\times\frac{ \partial F }{ \partial y }(w) \|dw,
\end{equation}
et si la partie $A\subset M$ est entièrement contenue dans $F(W)$, nous définissons la \defe{mesure}{mesure!dans une carte} de $A$ par
\begin{equation}		\label{EqDefMuDeuxDF}
	\mu_2(A)=\int_{F^{-1}(A)}d\sigma_F=\int_{F^{-1}(A)}\| \frac{ \partial F }{ \partial x }(w)\times\frac{ \partial F }{ \partial y }(w) \|dw.
\end{equation}
\begin{remark}
	Afin que cette définition ait un sens, nous devons prouver qu'elle ne dépend pas du choix de la carte $F$. En effet, les vecteurs $\partial_xF$ et $\partial_yF$ dépendent de la carte $F$, donc leur produit vectoriel (et sa norme) dépendent également de la carte $F$ choisie. Il faudrait donc un petit miracle pour que le nombre $\mu_2(A)$ donné par \eqref{EqDefMuDeuxDF} soit indépendant du choix de $F$.  Nous allons bientôt voir comme cas particulier du théorème \ref{ThoIntIndepF} que c'est en fait le cas. C'est à dire que si $F$ et $\tilde F$ sont deux cartes qui contiennent $A$, alors
	\begin{equation}
		\int_{F^{-1}(A)}d\sigma_F=\int_{\tilde F^{-1}(A)}d\sigma_{\tilde F}.
	\end{equation}
\end{remark}

%///////////////////////////////////////////////////////////////////////////////////////////////////////////////////////////
\subsubsection{Exemple : la mesure de la sphère}
%///////////////////////////////////////////////////////////////////////////////////////////////////////////////////////////

Nous nous proposons maintenant de calculer la surface de la sphère $S^2=x^2+y^2+z^2=R^2$. L'application $F\colon B( (0,0),R)\to R^3$ donnée par
\begin{equation}
	F(x,y)=\begin{pmatrix}
		x	\\ 
		y	\\ 
		\sqrt{R^2-x^2-y^2}	
	\end{pmatrix}
\end{equation}
est une carte pour une demi sphère. Ses dérivées partielles sont
\begin{equation}
	\begin{aligned}[]
		\frac{ \partial F }{ \partial x }&=\begin{pmatrix}
			1	\\ 
			0	\\ 
			-\frac{ x }{ \sqrt{R^2-x^2-y^2} }	
		\end{pmatrix},
		&\frac{ \partial F }{ \partial y }&=\begin{pmatrix}
			0	\\ 
			1	\\ 
			-\frac{ y }{ \sqrt{R^2-x^2-y^2} }	
		\end{pmatrix}.
	\end{aligned}
\end{equation}
Le produit vectoriel de ces deux vecteurs tangents donne
\begin{equation}
	\frac{ \partial F }{ \partial x }(x,y)\times\frac{ \partial F }{ \partial y }(x,y)=\frac{ x }{ \alpha }e_1+\frac{ y }{ \alpha }e_2+e_3
\end{equation}
où $\alpha=\sqrt{R^2-x^2-y^2}$. En calculant la norme, nous trouvons
\begin{equation}
	\| \frac{ \partial F }{ \partial x }(x,y)\times\frac{ \partial F }{ \partial y }(x,y)\| =\sqrt{  \frac{ R^2 }{ R^2-x^2-y^2 } },
\end{equation}
et en passant aux coordonnées polaires, nous écrivons l'intégrale \eqref{EqDefMuDeuxDF} sous la forme
\begin{equation}
	\int_B\| \partial_xF\times\partial_yF \|=\int_0^{2\pi}d\theta\int_0^R r\sqrt{  \frac{ R^2 }{ R^2-x^2-y^2 } }dr=2\pi R^2,
\end{equation}
qui est bien la mesure de la demi sphère.

%---------------------------------------------------------------------------------------------------------------------------
\subsection{Intégrale sur une carte}
%---------------------------------------------------------------------------------------------------------------------------

Nous pouvons maintenant définir l'intégrale d'une fonction sur une carte de la variété $M$.
\begin{definition}
	Soit $F\colon W\subset \eR^2\to \eR^3$, une carte pour une variété $M$. Soit $A$, une partie de $F(W)$ telle que $A=F(B)$ où $B\subset W$ est mesurable.  Soit encore $f\colon A\to \eR$, une fonction continue. L'\defe{intégrale}{intégrale!d'une fonction sur une carte} de $f$ sur $A$ est le nombre
	\begin{equation}	\label{EqDefIntDeuxDF}
		\int_Af=\int_Afd\sigma_F=\int_{F^{-1}(A)}(f\circ F)(w)\|  \frac{ \partial F }{ \partial x }(w)\times\frac{ \partial F }{ \partial y }(w) \| dw
	\end{equation}
\end{definition}

\begin{remark}
	L'intégrale \eqref{EqDefIntDeuxDF} n'est pas toujours bien définie. Étant donné que $F$ est $C^1$ et que $f$ est continue, l'intégrante est continue. L'intégrale sera donc bien définie par exemple lorsque $B$ est borné et si la fermeture $\bar A$ est un compact contenu dans $F(w)$.
\end{remark}

Le théorème suivant montre que le travail que nous avons fait jusqu'à présent ne dépend en fait pas du choix de carte $F$ effectué.

\begin{theorem}\label{ThoIntIndepF}
	Soient $F\colon W\to F(w)$ et $\tilde F\colon \tilde W\to \tilde F(\tilde W)$, deux cartes de la variété $M$. Soit une partie $A\subset F(W)\cap\tilde F(\tilde W)$ telle que $A=F(B)$ avec $B\subset W$ mesurable.  Alors $A=\tilde F(\tilde B)$ avec $\tilde B\subset\tilde W$ mesurable.

	Si $f$ est une fonction continue, et si $\int_Afd\sigma_F$ existe, alors $\int_Afd\sigma_{\tilde F}$ existe et
	\begin{equation}
		\int_Afd\sigma_F=\int_Afd\sigma_{\tilde F}.
	\end{equation}
\end{theorem}


%---------------------------------------------------------------------------------------------------------------------------
\subsection{Exemples}
%---------------------------------------------------------------------------------------------------------------------------

Intégrons la fonction $f(x,y,z)$ sur le carré $K=\mathopen] 0 , 1 \mathclose[\times \mathopen] 0 , 2 \mathclose[\times\{ 1 \}$. La première carte que nous pouvons utiliser est
\begin{equation}
	\begin{aligned}
		F\colon \mathopen] 0 , 1 \mathclose[\times\mathopen] 0 , 2 \mathclose[&\to K \\
		(x,y)&\mapsto (x,y,1). 
	\end{aligned}
\end{equation}
Nous trouvons aisément les vecteurs tangents qui forment l'élément de surface:
\begin{equation}
	\begin{aligned}[]
		\frac{ \partial F }{ \partial x }&=\begin{pmatrix}
			1	\\ 
			0	\\ 
			0	
		\end{pmatrix},
		&\frac{ \partial F }{ \partial y }&=\begin{pmatrix}
			0	\\ 
			1	\\ 
			0	
		\end{pmatrix},
	\end{aligned}
\end{equation}
donc $d\sigma_F=1\cdot dxdy$, et
\begin{equation}		\label{IntKSurcarrUn}
	\int_Kfd\sigma_F=\int_{\mathopen] 0 , 1 \mathclose[\times\mathopen] 0 , 2 \mathclose[}f(x,y,1)\cdot 1\cdot dxdy.
\end{equation}

Nous pouvons également utiliser la carte
\begin{equation}
	\begin{aligned}
		\tilde F\colon \mathopen] 0 , \frac{ 1 }{2} \mathclose[\times\mathopen] 0 , 6 \mathclose[&\to K \\
		(\tilde x,\tilde y)&\mapsto (2\tilde x,\frac{ \tilde y }{ 3 },1). 
	\end{aligned}
\end{equation}
Les vecteurs tangents sont maintenant
\begin{equation}
	\begin{aligned}[]
		\frac{ \partial \tilde F }{ \partial \tilde x }&=\begin{pmatrix}
			2	\\ 
			0	\\ 
			0	
		\end{pmatrix},
		&\frac{ \partial \tilde F }{ \partial \tilde y }&=\begin{pmatrix}
			0	\\ 
			1/3	\\ 
			0	
		\end{pmatrix},
	\end{aligned}
\end{equation}
de telle façon à ce que $d\sigma_{\tilde F}=\| \frac{ 2 }{ 3 }e_3 \|=\frac{ 2 }{ 3 }$. Cette fois, l'intégrale de $f$ sur $K$ s'écrit
\begin{equation}
	\int_Kfd\sigma_{\tilde F}=\int_{\mathopen] 0 , \frac{ 1 }{2} \mathclose[\times\mathopen] 0 , 6 \mathclose[}f\big( 2\tilde x,\frac{ \tilde y }{ 3 },1 \big)\cdot\frac{ 2 }{ 3 }\cdot d\tilde xs\tilde y.
\end{equation}
Conformément au théorème \ref{ThoIntIndepF}, cette dernière intégrale est égale à l'intégrale \eqref{IntKSurcarrUn} parce qu'il s'agit juste d'un changement de variable.


%---------------------------------------------------------------------------------------------------------------------------
\subsection{Orientation}
%---------------------------------------------------------------------------------------------------------------------------


Soient $F\colon W\to F(w)$ et $\tilde F\colon \tilde W\to \tilde F(\tilde W)$, deux cartes de la variété $M$. Nous pouvons considérer la fonction $h=\tilde F^{-1}\circ F$, définie uniquement sur l'intersection des cartes :
\begin{equation}
	h\colon F^{-1}\big( F(W)\cap\tilde F(\tilde W) \big)\to \tilde F^{-1}\big( F(W)\cap\tilde F(\tilde W) \big).
\end{equation}
Nous disons que $F$ et $\tilde F$ ont même \defe{orientation}{orientation} si
\begin{equation}
	J_h(w)>0
\end{equation}
pour tout $w\in  F^{-1}\big( F(W)\cap\tilde F(\tilde W) \big)$.

Considérons les deux carte suivantes pour le même carré:
\begin{equation}
	\begin{aligned}
		F\colon\mathopen] 0 , 1 \mathclose[\times \mathopen] 0 , 1 \mathclose[ &\to \eR^3 \\
		(x,y)&\mapsto (x,y,0) 
	\end{aligned}
\end{equation}
et
\begin{equation}
	\begin{aligned}
		\tilde F\colon\mathopen] 0 , \frac{ 1 }{2} \mathclose[\times\mathopen] 0 , \frac{1}{ 3 } \mathclose[ &\to \eR^3 \\
		(x,y)&\mapsto (2x,3y,0) 
	\end{aligned}
\end{equation}
Ici, $h(x,y)=\left( \frac{ x }{ 2 },\frac{ y }{ 3 } \right)$ et nous avons $J_h=\frac{1}{ 6 }>0$. Ces deux cartes ont même orientation. Notez que
\begin{equation}
	\frac{ \partial F }{ \partial x }\times\frac{ \partial F }{ \partial y }=e_3,
\end{equation}
tandis que
\begin{equation}
	\frac{ \partial \tilde F }{ \partial x }\times\frac{ \partial \tilde F }{ \partial y }=6e_3.
\end{equation}
Les vecteurs normaux à la paramétrisation pointent dans le même sens.

Si par contre nous prenons la paramétrisation
\begin{equation}
	\begin{aligned}
		G\colon \mathopen] 0,1 \mathclose[\times\mathopen] 0,1 ,  \mathclose[&\to \eR^2 \\
		(x,y)&\mapsto (x,(1-y),0), 
	\end{aligned}
\end{equation}
nous avons
\begin{equation}
	\frac{ \partial G }{ \partial x }\times\frac{ \partial G }{ \partial y }=-e_3,
\end{equation}
et si $g=G^{-1}\circ F$, alors $J_g=-1$.

L'orientation d'une carte montre donc si le vecteur normal à la surface pointe d'un côté ou de l'autre de la surface.

\begin{definition}
	Une variété $M$ est \defe{orientable}{orientable!variété} si il existe un atlas de $M$ tel que deux cartes quelconques ont toujours même orientation. Une variété est \defe{orientée}{variété !orientée} lorsque qu'un tel choix d'atlas est fait.
\end{definition}

\begin{proposition}
	Soit $M$, une variété orientable et un atlas orienté $\{ F_i\colon W_i\to \eR^3 \}$. Alors le vecteur unitaire
	\begin{equation}
		\frac{   \frac{ \partial F }{ \partial x }(x,y)\times\frac{ \partial F }{ \partial y }(x,y)   }{ \| \frac{ \partial F }{ \partial x }(x,y)\times\frac{ \partial F }{ \partial y }(x,y)\| }
	\end{equation}
	ne dépend pas du choix de $F$ parmi les $F_i$.
\end{proposition}


\begin{proof}
	Considérons deux cartes $F_1$ et $F_2$, ainsi que l'application $h=F_2^{-1}\circ F_1$. Écrivons le vecteur $\partial_x F_1\times\partial_yF_1$ en utilisant $F_1=F_2\circ h$. D'abord, par la règle de dérivation de fonctions composées,
	\begin{equation}
		\frac{ \partial (F_2\circ h) }{ \partial x }=\frac{ \partial F_2 }{ \partial x }\frac{ \partial h_1 }{ \partial x }+\frac{ \partial F_2 }{ \partial y }\frac{ \partial h_2 }{ \partial x }.
	\end{equation}
	Après avoir fait le même calcul pour $\frac{ \partial (F_2\circ h) }{ \partial y }$, nous pouvons écrire
	\begin{equation}
		\partial_x(F_2\circ h)\times\partial_y(F_2\circ h)=(\partial_xh_1\partial_xF_2+\partial_xh_2\partial_yF_2)\times(\partial_yh_1\partial_xF_2+\partial_yh_2\partial_yF_2).
	\end{equation}
	Dans cette expression, les facteurs $\partial_ih_j$ sont des nombres, donc ils se factorisent dans les produits vectoriels. En tenant compte du fait que $\partial_xF_2\times\partial_xF_2=0$ et $\partial_yF_2\times\partial_yF_2=0$, ainsi que de l'antisymétrie du produit vectoriel, l'expression se réduit à
	\begin{equation}
		\left( \frac{ \partial F_2 }{ \partial x }\times\frac{ \partial F_2 }{ \partial y } \right)(\partial_xh_1\partial_yh_2-\partial_xh_2\partial_yh_2).
	\end{equation}
	Par conséquent,
	\begin{equation}
		\frac{ \partial F_1 }{ \partial x }\times\frac{ \partial F_1 }{ \partial y } =\frac{ \partial (F_2\circ h) }{ \partial x }\times\frac{ \partial (F_2\circ h) }{ \partial y } =\left( \frac{ \partial F_2 }{ \partial x }\times\frac{ \partial F_2 }{ \partial y } \right)\det J_h.
	\end{equation}
	Donc, tant que $J_h$ est positif, les vecteurs unitaires correspondants au membre de gauche et de droite sont égaux.
\end{proof}

\begin{corollary}
	Si nous avons choisit un atlas orienté pour la variété $M$, nous avons une fonction continue $G\colon M\to \eR^3$ telle que $\| G(p) \|=1$ pour tout $p\in M$. Cette fonction est donnée par
	\begin{equation}		\label{DefCarteGOritn}
		G(F(x,y))=\frac{   \frac{ \partial F }{ \partial x }(x,y)\times\frac{ \partial F }{ \partial y }(x,y)   }{ \| \frac{ \partial F }{ \partial x }(x,y)\times\frac{ \partial F }{ \partial y }(x,y)\| }
	\end{equation}
	sur l'image de la carte $F$.
\end{corollary}

\begin{proof}
	La fonction $G$ est construite indépendamment sur chaque carte $F(W)$ en utilisant la formule \eqref{DefCarteGOritn}. Cette fonction est une fonction bien définie sur tout $M$ parce que nous venons de démontrer que sur $F_1(W_1)\cap F_2(W_2)$, les fonctions construites à partir de $F_1$ et à partir de $F_2$ sont égales.
\end{proof}

Il est possible que prouver, bien que cela soit plus compliqué, que la réciproque est également vraie.
\begin{proposition}
	Une variété $M$ de dimension $2$ dans $\eR^3$ est orientable si et seulement si il existe une fonction continue $G\colon M\to \eR^3$ telle que pour tout $p\in M$, le vecteur $G(p)$ soit de norme $1$ et normal à $M$ au point $p$.
\end{proposition}

%---------------------------------------------------------------------------------------------------------------------------
\subsection{Intégrale d'une fonction sur une variété}
%---------------------------------------------------------------------------------------------------------------------------

Nous supposons à présent que $M$ est une variété compacte de dimension $2$ dans $\eR^3$. La compacité fait que $M$ possède un atlas contenant un nombre fini de cartes $F_i\colon W_i\to F_i(W_i)$. 

Si $A\subset M$ est tel que pour chaque $i$, $A\cap F_i(W_i)=F_i(V_i)$ pour une ensemble $V_i$ mesurable dans $\eR^2$, alors nous considérons
\begin{equation}
	A_1=A\cap F_1(W_2)=F_1(V_1).
\end{equation}
Ensuite, nous construisons $A_2$ en considérant $F_A(W_2)$ et en lui retranchant $A_1$ :
\begin{equation}
	A_2=\big( A\cap F_2(W_2) \big)\cap F_1(V_1).
\end{equation}
En continuant de la sorte, nous construisons la décomposition
\begin{equation}
	A=A_1\cup\ldots\cup A_p
\end{equation}
de $A$ en ouverts disjoints, chacun de ouverts $A_p$ étant compris dans une carte.

Il est possible de prouver que dans ce cas, la définition suivante a un sens et ne dépend pas du choix de l'atlas effectué.
\begin{definition}
	Si $f\colon A\to \eR$ est une fonction continue, alors l'\defe{intégrale}{intégrale!d'une fonction sur une variété} est le nombre
	\begin{equation}
		\int_Af=\sum_{i=1}^p\int_{A_i}fd\sigma_{F_i}.
	\end{equation}
\end{definition}

\section{Intégrales curvilignes}
\label{secintcurvi}

\subsection{Chemins de classe \texorpdfstring{$C^1$}{C1}}

Soit $p, q\in \eR^n$. Un \defe{chemin}{chemin} $C^1$ par morceaux joignant $p$ à $q$ est une application continue
\begin{equation}
  \gamma : [a,b] \to \eR^n \qquad \gamma(a) = p, \gamma(b) = q
\end{equation}
pour laquelle il existe une subdivision $a = t_0 < t_1 < \ldots < t_{r-1} < t_r = b$ telle que :
\begin{enumerate}
\item la restriction de $\gamma$ sur chaque ouvert $\mathopen]t_i,
  t_{i+1}\mathclose[$ est de classe $C^1$~;
\item pour tout $0 \leq i \leq r$, $\gamma^\prime$ possède une limite
  à gauche (sauf pour $i = 0$) et une limite à droite (sauf pour $i =
  r$) en $t_i$.
\end{enumerate}
Le \defe{chemin $\gamma$ est (globalement) de classe $C^1$}{Chemin!classe $C^2$} si la
subdivision peut être choisie de \og longueur\fg{} $r = 1$.

\begin{remark}
	Si $a$ et $b$ sont des points de
  $\eR^n$, on peut créer le chemin particulier
  \begin{equation*}
    \gamma : [0,1] \to \eR^n : t \mapsto (1-t)a + tb
  \end{equation*}
  qui relie ces points par un segment de droite.
\end{remark}

\subsection{Intégrer une fonction}

Soit $f : D \subset \eR^n \to \eR$ une fonction continue, et $\gamma
: [a,b] \to D$ un chemin $C^1$. On définit \Defn{l'intégrale de $f$
  sur $\gamma$} par
  \begin{equation}    \label{EqhJGRcb}
  \int_\gamma f d s = \int_\gamma f = \int_a^b f(\gamma(t)) \norme{\gamma^\prime(t)} d t.
\end{equation}

\begin{remark}
  Cette définition ne dépend pas de la paramétrisation choisie, ni du
  sens du chemin (échange entre point de départ et point d'arrivée).
\end{remark}

\begin{remark}      \label{RemiqswPd}
    Attention : les intégrales sur des chemins dans \( \eC\) ne sont la même chose. En effet \( \eC\) doit être souvent plutôt traité comme \( \eR\) que comme \( \eR^2\). Si \( \gamma\) est un chemin dans \( \eC\), l'intégrale
    \begin{equation}
        \int_{\gamma}f
    \end{equation}
    doit être comprise comme une généralisation de \( \int_a^bf(x)dx\) et non comme l'intégrale sur un chemin. La différence est qu'en retournant les bornes d'une intégrale usuelle sur \( \eR\) on change le signe, alors qu'en retournant un chemin dans \( \eR^2\), on ne change pas. Bref, la définition est que si \( \gamma\colon \mathopen[ a , b \mathclose]\to \eC\) est un chemin, alors
    \begin{equation}
        \int_{\gamma}f=\int_{\gamma}f(z)dz=\int_a^bf\big( \gamma(t) \big)\gamma'(t)dt.
    \end{equation}
\end{remark}


La formule qui donne la longueur d'un chemin est évidement l'intégrale de la fonction $1$ sur le chemin, c'est à dire
\begin{equation}
	L=\int_a^b\| \gamma'(t) \|dt.
\end{equation}
Si on veut savoir la longueur d'une courbe donnée sous la forme d'une fonction $y=y(x)$, un chemin qui trace la courbe est évidement donné par
\begin{equation}
	\gamma(t)=(t,y(t)),
\end{equation}
et le vecteur tangent au chemin est $\gamma'(t)=(1,y'(t))$. Donc
\begin{equation}
	\| \gamma'(t) \|=\sqrt{1+y'(t)^2},
\end{equation}
et 
\begin{equation}			\label{EqLongFonction}
	L=\int_a^b\sqrt{1+y'(t)^2}.
\end{equation}

\subsection{Intégrer un champ de vecteurs}
Un \Defn{champ de vecteur} est une application $G : \eR^n \to
\eR^n$. On définit l'intégrale de $G$ sur un chemin $\gamma : [a,b]
\to \eR^n$ par
\begin{equation*}
  \int_\gamma G \pardef \int_a^b \scalprod {G(\gamma(t))}{\gamma^\prime(t)} d t.
\end{equation*}

\begin{remark}
  Cette définition ne dépend pas de la paramétrisation choisie, mais
  le signe change selon le sens du chemin.
\end{remark}

%---------------------------------------------------------------------------------------------------------------------------
\subsection{Intégrer une forme différentielle sur un chemin}
%---------------------------------------------------------------------------------------------------------------------------

Une \defe{forme différentielle}{forme!différentielle} sur $\eR^n$ est une application
\begin{equation}
	\begin{aligned}
		\omega\colon \eR^n&\to (\eR^n)^* \\
		x&\mapsto \omega_x 
	\end{aligned}
\end{equation}
qui à chaque point $x$ de $\eR^n$ associe une forme linéaire $\omega_x: \eR^n \to \eR$.

On sait que $\{ d x_i \}_{1\leq i\leq n}$ est une base de
${(\eR^{n})}^{*}$, donc toute forme différentielle s'écrit
\begin{equation*}
  \omega_x = \sum_{i=0}^n a_i(x) d x_i
\end{equation*}
où $a_1,\ldots,a_n$ sont les \Defn{composantes de $\omega$} dans la
base usuelle, et sont des fonctions à valeurs réelles. Pour un vecteur
$v = (v_1,\ldots,v_n)$ on a donc par définition de $d x_i$
\begin{equation*}
  \omega_x v = \sum_{i=0}^n a_i(x) v_i.
\end{equation*}

L'intégrale d'une forme différentielle sur un chemin est définie par
\begin{equation*}
  \int_\gamma \omega = \int_a^b \omega_{\gamma(t)}\gamma^\prime(t) d t
\end{equation*}

\begin{remark}
  Cette définition ne dépend pas de la paramétrisation choisie, mais
  le signe change selon le sens du chemin.
\end{remark}

\subsection{Lien entre forme différentielle et champ vectoriel}
Si $G$ est un champ de vecteurs, on peut définir la forme différentielle
\begin{equation*}
  \omega^G : \eR^n \to {(\eR^n)}^\ast : x \mapsto \left\lbrack \omega^G_x :
  \eR^n \to \eR : v \mapsto \omega^G_x v = \scalprod {G(x)}v \right\rbrack
\end{equation*}
et réciproquement, si $\omega_x = \sum_i a_i(x)d x_i$ est une forme
différentielle on définit le champ de vecteurs
\begin{equation*}
  G^\omega(x) = (a_1(x),\ldots,a_n(x)).
\end{equation*}

Avec ces définitions, pour un chemin $\gamma$ donné on a
\begin{equation*}
  \int_\gamma \omega^G = \int_\gamma G^\omega
\end{equation*}

%---------------------------------------------------------------------------------------------------------------------------
\subsection{Intégrer un champs de vecteurs sur un bord en $2D$}
%---------------------------------------------------------------------------------------------------------------------------

Si $D\subset\eR^2$ est tel que $\partial D$ est une variété de dimension $1$ et tel que $D$ accepte un champ de vecteur normal extérieur unitaire $\nu$. Si nous voulons définir 
\begin{equation}
	\int_{\partial D}G,
\end{equation}
le mieux est de prendre une paramétrisation $\gamma\colon \mathopen[ 0 , 1 \mathclose]\to \eR^2$ et de calculer
\begin{equation}
	\int_0^1 \langle G_{\gamma(t)}, \frac{ \dot\gamma(t) }{ \| \dot\gamma(t) \| }\rangle dt.
\end{equation}
Hélas, cette définition ne fonctionne pas parce que son signe dépend du sens de la paramétrisation $\gamma$. Si la paramétrisation tourne dans l'autre sens, il y a un signe de différence.

Nous allons définir
\begin{equation}		\label{EqIntVectbordDeux}
	\int_{\partial D}G=\int_0^1\langle G_{\gamma(t)}, T(t)\rangle dt
\end{equation}
où $T(t)=\dot\gamma(t)/\| \dot\gamma(t) \|$ et où $\gamma$ est choisit de telle façon à ce que la rotation d'angle $\frac{ \pi }{ 2 }$ amène $\nu$ sur $T$. Cela fixe le choix de sens.

Ce choix de sens aura des répercussions dans l'application de la formule de Green et du théorème de Stokes.

%---------------------------------------------------------------------------------------------------------------------------
\subsection{Intégrer une forme différentielle sur un bord en $2D$}
%---------------------------------------------------------------------------------------------------------------------------

Nous n'allons pas chercher très loin :
\begin{equation}
	\int_{\partial D}\omega=\int_{\partial D}\omega^{\sharp},
\end{equation}
c'est à dire que l'intégrale de la forme différentielle est celle du champ de vecteur associé. Le membre de droite est définit par \eqref{EqIntVectbordDeux}, avec le choix d'orientation qui va avec.

%---------------------------------------------------------------------------------------------------------------------------
\subsection{Intégrer une forme différentielle sur un bord en $3D$}
%---------------------------------------------------------------------------------------------------------------------------

Nous allons maintenant intégrer une forme différentielle sur certains chemins fermés dans $\eR^3$. Soit $F(D)\subset\eR^3$, une variété de dimension $2$ dans $\eR^3$ où $F\colon D\subset\eR^2\to \eR^3$ est la carte. Nous supposons que $D$ vérifie les hypothèses de la formule de Green. Alors nous définissons
\begin{equation}		\label{EqDefIntTroisForBord}
	\int_{F(\partial D)}\omega = \int_{\partial D} F^*\omega
\end{equation}
où $F^*\omega$ est la forme différentielle définie sur $\partial D$ par $(F^*\omega)(v)=\omega\big( dF(v) \big)$.

Cette définition est très abstraite, mais nous n'allons, en pratique, jamais l'utiliser, grâce au théorème de Stokes.

%---------------------------------------------------------------------------------------------------------------------------
\subsection{Intégrer d'un champ de vecteurs sur un bord en $3D$}
%---------------------------------------------------------------------------------------------------------------------------

Encore une fois, nous n'allons pas chercher bien loin :
\begin{equation}
	\int_{F(\partial D)G}=\int_{F(\partial D)}G^{\flat}
\end{equation}
où $G^{\flat}$ est la forme différentielle associée au champ de vecteur. Le membre de droite est définit par l'équation \eqref{EqDefIntTroisForBord}.

%---------------------------------------------------------------------------------------------------------------------------
\subsection{Dérivées croisées et forme différentielle exacte}
%---------------------------------------------------------------------------------------------------------------------------

Nous considérons le problème suivant : trouver une fonction \( f\colon \eR^2\to \eR\) telle que
\begin{subequations}        \label{EqskfgfNr}
    \begin{numcases}{}
        \frac{ \partial f }{ \partial x }=a(x,y)\\
        \frac{ \partial f }{ \partial y }=b(x,y)
    \end{numcases}
\end{subequations}
où \( a\) et \( b\) sont des fonctions supposées suffisamment régulières. Nous savons que ce problème n'a pas de solutions lorsque
\begin{equation}
    \frac{ \partial a }{ \partial y }\neq\frac{ \partial b }{ \partial x }
\end{equation}
parce que cela impliquerait \( \partial^2_{xy}f\neq \partial^2_{yx}f\). Nous sommes en droit de nous demander si la condition
\begin{equation}
    \frac{ \partial a }{ \partial y }=\frac{ \partial b }{ \partial x }
\end{equation}
impliquerait qu'il existe une solution au problème \eqref{EqskfgfNr}. La réponse est oui, et nous allons brièvement la justifier. Pour plus de détails nous vous demandons de chercher un peu \href{http://www.bing.com/search?q=forme+diff\%C3\%A9rentielle+exacte+filetype\%3Apdf&form=QBRE&fit=all}{sur internet} les mots-clefs \emph{forme différentielles exacte}. Vous consulterez également avec profit \cite{DiffExact}.

\begin{proposition}
    Si \( a\) et \( b\) sont des fonctions qui satisfont à la condition
    \begin{equation}
        \frac{ \partial a }{ \partial y }=\frac{ \partial b }{ \partial x },
    \end{equation}
    alors la fonction
    \begin{equation}        \label{EqllhTaT}
        f(x,y)=\int_0^x a(t,0)dt+\int_0^yb(x,t)dt
    \end{equation}
    répond au problème
    \begin{subequations}     
        \begin{numcases}{}
            \frac{ \partial f }{ \partial x }=a(x,y)\\
            \frac{ \partial f }{ \partial y }=b(x,y)
        \end{numcases}
    \end{subequations}
\end{proposition}

La preuve qui suit n'en est pas complètement une parce qu'il manque des justification, notamment au moment de permuter la dérivée et l'intégrale.
\begin{proof}
    La clef de la preuve est le théorème fondamental de l'analyse :
    \begin{equation}
        \int_0^x \frac{ \partial f }{ \partial x }(t,y)dt=f(x,y)
    \end{equation}
    et son pendant par rapport à \( y\) :
    \begin{equation}
        \int_0^y \frac{ \partial f }{ \partial y }(x,t)dt=f(x,y).
    \end{equation}
    En appliquant ces version du théorème fondamental, nous obtenons immédiatement.
    \begin{equation}
        \frac{ \partial f }{ \partial y }=b(x,y).
    \end{equation}
    En ce qui concerne la dérivée par rapport à \( y\),
    \begin{subequations}
        \begin{align}
            \frac{ \partial f }{ \partial x }&=a(x,0)+\int_0^y\frac{ \partial b }{ \partial x }(x,t)dt\\
            &=a(x,0)+\int_0^y\frac{ \partial a }{ \partial y }(x,t)dt\\
            &=a(x,0)+[a(x,t)]_{t=0}^{t=y}\\
            &=a(x,y).
        \end{align}
    \end{subequations}
\end{proof}

En ce qui concerne l'unicité, supposons que \( f\) et \( g\) soient deux solutions au problème. L'équation
\begin{equation}
    \frac{ \partial f }{ \partial x }=a(x,y)=\frac{ \partial g }{ \partial x }
\end{equation}
implique que 
\begin{equation}
    f(x,y)=g(x,y)+C(y)
\end{equation}
où \( C\) est une fonction seulement de \( y\). L'autre équation implique
\begin{equation}
    f(x,y)=g(x,y)+D(x)
\end{equation}
où \( D\) est seulement une fonction de \( x\). En égalisant nous voyons que les fonctions \( C\) et \( D\) doivent être des constantes.

Par conséquent la fonction \( f\) est donnée à une constante près et en réalité la fonction \eqref{EqllhTaT} est suffisante pour répondre au problème de trouver toutes les fonctions dont les dérivées partielles sont données par les fonctions \( a\) et \( b\).

La fonction \( f\) ainsi créée est un \defe{potentiel}{potentiel} pour le champ de force
\begin{equation}
    F(x,y)=\begin{pmatrix}
        a(x,y)    \\ 
        b(x,y)  
    \end{pmatrix}.
\end{equation}
Notez que ce champ de vecteurs est le gradient de \( f\). La question initiale aurait donc pu être posée en les termes suivants : trouver une fonction \( f\) dont le gradient est donné par
\begin{equation}
    \nabla f=\begin{pmatrix}
        a(x,y)    \\ 
        b(x,y)    
    \end{pmatrix}.
\end{equation}


%+++++++++++++++++++++++++++++++++++++++++++++++++++++++++++++++++++++++++++++++++++++++++++++++++++++++++++++++++++++++++++
\section{Intégrales de surface}
%+++++++++++++++++++++++++++++++++++++++++++++++++++++++++++++++++++++++++++++++++++++++++++++++++++++++++++++++++++++++++++

%---------------------------------------------------------------------------------------------------------------------------
\subsection{Intégrale d'une fonction}
%---------------------------------------------------------------------------------------------------------------------------
\label{secintsurfaciques}
Soit $M$ une variété de dimension $n$ dans $\eR^m$. Soit $F : W \subset \eR^n \to M$ une paramétrisation d'un ouvert relatif de $M$.  

Si $f$ est une fonction définie sur un sous-ensemble $A \subset F(W)$ tel que $F^{-1}(A)$ est mesurable, l'\Defn{intégrale de $f$ sur $A$} est définie par
\begin{equation*}
  \int_A f = \int_{F^{-1}(A)} f(F(w)) \sqrt{\det(\transpose{J_F(w)} {J_F(w)})} dw
\end{equation*}
où l'intégrale est l'intégration usuelle (de Lebesgue) sur $F^{-1}(A) \subset \eR^n$. On écrit parfois cette intégrale $\int_{F^{-1}(A)} f(F(w)) d\sigma$ où
\begin{equation*}
  d\sigma = \sqrt{\det(\transpose{J_F(w)} {J_F(w)})} dw
\end{equation*}
est l'\Defn{élément infinitésimal de volume} de la variété. 

Si $m = 3$ et $n = 2$, l'élément infinitésimal de volume vaut
\begin{equation*}
  d \sigma = \norme{\pder F {w_1} \wedge \pder F {w_2}} dw
\end{equation*}
où $\wedge$ représente le produit vectoriel dans $\eR^3$, et $(w_1,w_2)$ sont les coordonnées sur $W \subset \eR^2$. Dans la suite, nous ne regarderons plus que ce cas.

\subsection{Intégrale d'un champ de vecteurs}
Dans l'intégration curviligne, on a noté que si l'intégrale d'une fonction ne dépendait pas de l'orientation du chemin, l'intégrale d'un champ de vecteurs ou d'une forme différentielle en dépendait. Ce problème d'orientation apparait également dans l'intégration sur des surfaces de l'espace.

%% Page 530, exemple 4
Une \Defn{orientation} sur une surface $S \subset \eR^3$ est le choix
d'un champ de vecteurs continu $\nu : S \to \eR^3$ dont la norme en
tout point de $S$ vaut $1$. On remarque qu'ayant fait un tel choix
d'orientation $\nu(x)$ en un point $x$, le seul autre choix possible
en $x$ est $-\nu(x)$.
%% Page 
Si $S$ est le bord d'un ouvert $D \subset \eR^3$, l'\Defn{orientation
  induite par $D$ sur $S$} est, si elle existe, l'orientation qui
pointe hors de $D$ en tout point de $S$. Plus précisément, il faut que
pour tout $x \in D$ il existe $\epsilon > 0$ vérifiant, pour tout $0 <
t < \epsilon$, la relation $t \nu(x) \notin D$. Dans ce cas, le champ
de vecteurs $\nu$ est appelé le \Defn{vecteur normal unitaire
  extérieur} à $D$ et il est forcément unique.

Soit $G$ un champ de vecteurs défini sur une surface orientée par un
champ $\nu$. L'intégrale de $G$ sur $S$, aussi appelée le \Defn{flux
de $G$ à travers $S$}, est
\begin{equation}\label{eqflux-star}
  \iint_S G \cdot d S \pardef \iint_S \scalprod{G}{\nu} d \sigma.
\end{equation}
Si on suppose que la surface est paramétrisée par une application
\begin{equation*}
  F : W \subset \eR^2 \to \eR^3 : (u,v) \mapsto (F_1(u,v),F_2(u,v),F_3(u,v))
\end{equation*}
alors un vecteur unitaire $\nu$ peut s'écrire sous la forme
\begin{equation*}
  \nu = \frac{\pder F u \wedge \pder F v}{\norme{\pder F u \wedge \pder F v}}
\end{equation*}
et grâce à cette paramétrisation l'intégrale \eqref{eqflux-star}
devient
\begin{equation*}
  \iint_S G \cdot d S = \iint_W \scalprod{G(F(u,v))}{\pder F u \wedge \pder F v} d u
  d v.
\end{equation*}
où on utilise l'expression de $d \sigma$ obtenue précédemment dans le
cas qui nous intéresse (surface dans l'espace).

%+++++++++++++++++++++++++++++++++++++++++++++++++++++++++++++++++++++++++++++++++++++++++++++++++++++++++++++++++++++++++++
\section{Divergence, Green, Stokes}
%+++++++++++++++++++++++++++++++++++++++++++++++++++++++++++++++++++++++++++++++++++++++++++++++++++++++++++++++++++++++++++


Le théorème de Stokes (et ses variations) peut se voir comme une généralisation du théorème fondamental du calcul différentiel et intégral qui stipule que
\begin{equation*}
	\int_a^b f^\prime(x) d x = f(b) - f(a)
\end{equation*}
c'est-à-dire qui relie l'intégrale de $f^\prime$ sur $I = [a,b]$ aux valeurs de $f$ sur le bord $\partial I = \{a,b\}$. Le signe $-$ qui apparait vient de l'orientation ; celle-ci requiert de la prudence dans l'utilisation des théorèmes.

Voici, pour votre culture générale, un énoncé général :
\begin{theorem}
	Si $M$ est une variété orientable de dimension $n$ avec un bord noté $\partial  M$, alors pour toute forme différentielle $\omega$ de degré $n-1$ on a 
	\begin{equation*}
		\int_{ M} d \omega = \int_{\partial  M} \omega.
	\end{equation*}
	où $d \omega$ désigne la différentielle extérieure de $\omega$.
\end{theorem}
Nous allons maintenant voir quelque cas particuliers. 


\subsection{Théorème de la divergence}

Si nous considérons une surface dans $\eR^n$ et un champ de vecteurs, il est bon de se demander quelle \og quantité de vecteurs\fg{} traverse la surface. Soit $D$, un ouvert borné de $\eR^n$ telle que $\partial D$ soit une variété de dimension $n-1$, et $G$, un champ de vecteurs défini sur $\bar D$. Afin de compter combien de $G$ traverse $\partial D$, il faudra faire en sorte de ne considérer que la composante de $G$ normale à $\partial D$ : pas question d'intégrer par exemple la norme de $G$ sur $\partial D$.

Comme nous le savons, la composante du vecteur $v$ dans la direction $w$ est le produit scalaire $v\cdot 1_w$ où $1_w$ est le vecteur de norme $1$ dans la direction $w$. Nous allons donc introduire le concept de vecteur normal extérieur. Soit $x\in\partial D$ et $\nu\in\eR^n$, nous disons que $\nu$ est un \defe{vecteur normal extérieur}{Normal extérieur!vecteur} de $\partial D$ si
\begin{enumerate}

	\item
		$\langle \nu, v\rangle =0$ pour tout vecteur tangent $v$ à $\partial D$ au point $x$. Pour rappel, $\partial D$ étant une variété de dimension $n-1$, il y a $n-1$ tels vecteurs $v$ linéairement indépendants.
	
	\item
		Il existe un $\delta>0$ tel que $\forall t\in\mathopen] 0 , \delta \mathclose[$, nous avons $c+t\nu\notin \bar D$ et $x-t\nu\in D$.
 
\end{enumerate}

Nous pouvons maintenant définir le concept de flux. Soit $D\subset \eR^n$ tel que $\partial D$ soit une variété de dimension $n-1$ qui admette un vecteur normal extérieur $\nu(x)$ en chaque point. Soit aussi $G\colon \bar D\to \eR^n$, un champ de vecteur de classe $C^1$. Le \defe{flux}{flux!d'un champ de vecteur} de $G$ au travers de $\partial D$ est le nombre
\begin{equation}
	\int_{\partial D}\langle G(x), \nu(x)\rangle d\sigma(x).
\end{equation}

Cette intégrale est en général très compliquée à calculer parce qu'il faut trouver le champ de vecteur normal, puis une paramétrisation de la surface $\partial D$ et ensuite appliquer la méthode décrite au point \ref{secintsurfaciques}. 

Heureusement, il y a un théorème qui nous permet de calculer plus facilement : sans devoir trouver de vecteurs normaux.


Il n'est pas plus contraignant d'énoncer ce théorème dans le cadre d'une hypersurface de $\eR^n$, ce que nous faisons donc~:
\begin{theorem}[Formule de la divergence]
	Soit $D$ un ouvert borné de $\eR^n$ dont le bord est \og assez régulier par morceaux\fg{}, c'est-à-dire~:
	\begin{equation}
		\partial D = A_1 \cup \ldots A_p \cup N
	\end{equation} 
	où
	\begin{enumerate}
		\item $A_1, \ldots, A_p, N$ sont deux à deux disjoints,
		\item pour tout $i \leq p$, $A_i$ est un ouvert relatif d'une certaine variété $M_i$ de dimension $(n-1)$
		\item $\bar A_i \subset M_i$
		\item $N$ est un compact contenu dans une réunion finie de variétés de dimensions $(n-2)$.
	\end{enumerate}
	Supposons également qu'en chaque point de $A_1 \cup \ldots \cup A_p$ il existe un vecteur normal extérieur $\nu$.
	
	Si $G$ est un champ de vecteurs de classe $C^1$ sur $\bar D$ alors
	\begin{equation}
		\int_D \nabla\cdot G = \sum_{i=1}^p \int_{A_i} \scalprod{G}{\nu}.
	\end{equation}
	L'intégrale du membre de gauche est l'intégrale sur un ouvert d'une simple fonction.
\end{theorem}

\subsection{Formule de Green}
La formule de Green est un cas particulier du théorème de la divergence dans
le cas $n = 2$, légèrement reformulé~:
\begin{theorem}
	Soit $D \subset \eR^2$ ouvert borné tel que son bord est est la réunion finie d'un certain nombre de chemins de classe $C^1$ de Jordan réguliers.  Supposons qu'en chaque point de son bord, $D$ possède un vecteur normal unitaire extérieur $\nu$. Soient $P$ et $Q$ deux fonctions réelles de classe $C^1$ sur $\bar D$. Alors
  \begin{equation*}
    \iint_D (\partial_xQ - \partial_yP)dx\,dy = \oint_{\partial D}
    Pd x + Q d y
  \end{equation*}
  où chaque chemin $\gamma$ formant le bord de $D$ est orienté de
  sorte que $T \nu = \frac{\dot\gamma}{\norme{\dot\gamma}}$ où $T$
  représente la rotation d'angle $+\frac\pi2$.
\end{theorem}

Pour rappel, une chemin $\gamma\colon \mathopen[ 0 , 1 \mathclose]\to \eR^n$ est \defe{régulier}{régulier!chemin} si il est $C^1$ et si $\gamma(t)\neq 0$ pour tout $r$. Le chemin est de \defe{Jordan}{Jordan!chemin} si $\gamma(1)=\gamma(0)$ et si $\gamma\colon \mathopen[ a , b [\to \eR^n$ est injective.

\subsection{Formule de Stokes}
\label{secstokesusuel}
La formule de Stokes est la version classique, qui permet d'exprimer la circulation d'un champ de vecteur le long d'une courbe de $\eR^3$ comme le flux de son rotationnel à travers n'importe quel surface dont le bord est la courbe. La version présentée ici suppose que la surface peut se paramétrer en un seul morceau~:
\begin{theorem}
  Soit $F : W\subset \eR^2 \to \eR^3$ une paramétrisation (carte) d'une surface dans $\eR^3$, supposée de classe $C^2$. Soit $D$ un ouvert de $\eR^2$ vérifiant les hypothèses de la formule de Green, et tel que $\bar D \subset W$. Soit $G$ un champ de vecteurs de classe $C^1$ défini sur $F(\bar D)$, et soit $N$ le champ normal unitaire donné par la paramétrisation
  \begin{equation}		
	N = \frac{\pder F u \wedge \pder F v}{\norme{\pder F u \wedge \pder F v}}
\end{equation}
  alors
  \begin{equation}\label{EqStokesTho}
    \iint_{F(D)} \scalprod{\rot G}{N} d\sigma_F = \int_{F(\partial D)} G
  \end{equation}
  où les chemins formant le bord $\partial D$ sont orientés comme dans le théorème de Green.
\end{theorem}
Notons, juste pour avoir une bonne nouvelle de temps en temps, que 
\begin{equation}
	d\sigma_F=\left\| \frac{ \partial F }{ \partial u }\times\frac{ \partial F }{ \partial v }  \right\|dudv,
\end{equation}
mais cette norme apparaît exactement au dénominateur de $N$. Il ne faut donc pas la calculer parce qu'elle se simplifie.

Sous forme un peu plus physicienne\footnote{et surtout plus explicite.}, la formule \eqref{EqStokesTho} s'écrit
\begin{equation}
	\int_{F(D)}\langle \nabla\times G, N(x)\rangle\, d\sigma_F(x)=\int_{F(\gamma)}\langle G, T\rangle\, ds
\end{equation}
où $T$ est le vecteur unitaire tangent à $F(\gamma)$.

%///////////////////////////////////////////////////////////////////////////////////////////////////////////////////////////
\subsubsection{Quelle est la bonne orientation ?}
%///////////////////////////////////////////////////////////////////////////////////////////////////////////////////////////

Le signe du vecteur normal $N$ dépend du choix de l'ordre des coordonnées dans la carte. Supposons que je veuille paramétrer la surface $x^2+y^2=1$, $z=1$. Nous prenons naturellement comme carte le cercle $C$ de rayon $1$ dans $\eR^2$ et la carte
\begin{equation}
	F(r,\theta)=\begin{pmatrix}
		r\cos\theta	\\ 
		r\sin\theta	\\ 
		1	
	\end{pmatrix}.
\end{equation}
Mais nous aurions aussi pu mettre les coordonnées $r$ et $\theta$ dans l'autre ordre :
\begin{equation}
	\tilde F(\theta,r)=\begin{pmatrix}
		r\cos\theta	\\ 
		r\sin\theta	\\ 
		1	
	\end{pmatrix}.
\end{equation}
Les vecteurs normaux ne sont pas les même : la carte $F$ donnera $\partial_rF\times\partial_{\theta}F$, tandis que l'autre donnera $\partial_{\theta}\tilde F\times\partial_r\tilde F$. Le signe change !

Il faut savoir laquelle choisir. Le cercle $C\subset \eR^2$ a une orientation donnée par le théorème de Green. Nous choisissons l'ordre des coordonnées pour que $1_{\theta}$ et $1_{r}$ soient dans la même orientation que les vecteurs $\nu$ et $T$ tels que donnés par le théorème de Green, et tels que dessinés sur la figure \ref{LabelFigCercleTnu}.
\newcommand{\CaptionFigCercleTnu}{L'orientation sur le cercle. Si nous les prenons dans l'ordre, les vecteurs $(1_r,1_{\theta})$ ont la même orientation que celle donnée par les vecteurs $(\nu,T)$ donnés par la convention de Green.}
\input{Fig_CercleTnu.pstricks}

%\ref{LabelFigCercleTnu}.
%\newcommand{\CaptionFigCercleTnu}{L'orientation sur le cercle. Si nous les prenons dans l'ordre, les vecteurs $(1_r,1_{\theta})$ ont la même orientation que celle donnée par les vecteurs $(\nu,T)$ donnés par la convention de Green.}
%\input{Fig_CercleTnu.pstricks}

Plus généralement, nous choisissons l'ordre des coordonnées $u$ et $v$ pour que la base $(1_u,1_v)$ ait la même orientation que $(\nu,T)$ où $T$ a le sens convenu dans le théorème de Green.

%+++++++++++++++++++++++++++++++++++++++++++++++++++++++++++++++++++++++++++++++++++++++++++++++++++++++++++++++++++++++++++
					\section[Intégrales de fonctions et domaines non bornées]{Intégrales de fonctions non bornées sur des domaines non bornés}
%+++++++++++++++++++++++++++++++++++++++++++++++++++++++++++++++++++++++++++++++++++++++++++++++++++++++++++++++++++++++++++

%---------------------------------------------------------------------------------------------------------------------------
					\subsection[Fonctions et ensembles non bornés]{Intégrales de fonctions non bornées sur des ensembles non bornés}
%---------------------------------------------------------------------------------------------------------------------------

Soit $f\colon \eR^n\to \overline{ \eR }$, une fonction positive. On dit qu'elle est \defe{intégrable}{intégrable!fonction positive} sur $E\subset\eR^n$ si
\begin{enumerate}
    \item $\forall r>0$, la fonction $f_r(x)=f(x)\mtu_{f<r}$ est intégrable sur $E_r$;
\item la limite $\lim_{r\to\infty}\int_{E_r}f_r$ est finie.
\end{enumerate}
Dans ce cas, on pose 
\begin{equation}
	\int_Ef=\lim_{r\to\infty}\int_{E_r}f_r.
\end{equation}

\begin{theorem}[Page I.38]		\label{ThoFnTestIntnnBorn}
Soit $E$ mesurable dans $\eR^n$ et $f\colon E\to \overline{ \eR }$. Si $f$ est mesurable et si il existe $g\colon E\to \overline{ \eR }$ intégrable sur $E$ telle que $| f(x) |\leq g(x)$ pour tout $x\in E$, alors $f$ est intégrable sur~$E$.

Réciproquement, si $f$ est intégrable sur $E$, alors $f$ est mesurable.
\end{theorem}

%---------------------------------------------------------------------------------------------------------------------------
					\subsection{Passage à la limite sous le signe intégral}
%---------------------------------------------------------------------------------------------------------------------------

Un autre résultat très important pour l'étude de l'intégrabilité est le théorème de la \defe{convergence dominée de Lebesgue}{}:
\begin{theorem}
	Soit $E\subset \eR^n$ un ensemble mesurable et $\{ f_k \}$, une suite de fonctions intégrables sur $E$ qui converge simplement vers une fonction $f\colon E\to \overline{ \eR }$. Supposons qu'il existe une fonction $g$ intégrable sur $E$ telle que pour tout $k$,
\begin{equation}
	| f(x) |\leq g(x)
\end{equation}
pour tout $x\in E$. Alors $f$ est intégrable sur $E$ et 
\begin{equation}
	\int_Ef=\lim_{k\to\infty}\int_Ef_k.
\end{equation}
\end{theorem}

%---------------------------------------------------------------------------------------------------------------------------
					\subsection{Théorème de Fubini et changement de variables}
%---------------------------------------------------------------------------------------------------------------------------

\begin{theorem}[Fubini]\index{théorème!Fubini}\index{Fubini!théorème}		\label{ThoFubini}
Soit $(x,t)\mapsto f(x,y)\in\bar \eR$ une fonction intégrable sur $B_n\times B_m\subset\eR^{n+m}$ où $B_n$ et $B_m$ sont des ensembles mesurables de $\eR^n$ et $\eR^m$. Alors :
\begin{enumerate}
\item pour tout $x\in B_n$, sauf éventuellement en les points d'un ensemble $G\subset B_n$ de mesure nulle, la fonction $y\in B_m\mapsto f(x,y)\in\bar\eR$ est intégrable sur $B_m$
\item
la fonction
\begin{equation}
	x\in B_n\setminus G\mapsto\int_{B_m}f(x,y)dy\in\eR
\end{equation}
est intégrable sur $B_n\setminus G$

\item 
On a
\begin{equation}
	\int_{B_n\times B_m}f(x,y)dxdy=\int_{B_n}\left( \int_{B_m}f(x,y)dy \right)dx.
\end{equation}

\end{enumerate}
\end{theorem}

Notons en particulier que si $f(x,y)=\varphi(x)\phi(y)$, alors $\int_{B_m}\varphi(y)dy$ est une constante qui peut sortir de l'intégrale sur $B_n$, et donc
\begin{equation}		\label{EqFubiniFactori}
	\int_{B_n\times B_m}\varphi(x)\phi(y)dxdy=\int_{B_n}\varphi(x)dx\int_{B_m}\phi(y)dy.
\end{equation}

%---------------------------------------------------------------------------------------------------------------------------
					\subsection{Intégrale en dimension un}
%---------------------------------------------------------------------------------------------------------------------------

\begin{proposition}[Critère de comparaison]
Soit $f$ mesurable sur $]a,\infty[$ et bornée sur tout $]a,b]$, et supposons qu'il existe un $X_0\geq a$, tel que sur $]X_0,\infty[$,
\begin{equation}
	| f(x) |\leq g(x)
\end{equation}
où $g(x)$ est intégrable. Alors $f(x)$ est intégrable sur $]a,\infty[$.
\end{proposition}

\begin{corollary}[Critère d'équivalence]
Soient $f$ et $g$ des fonctions mesurables et positives ou nulles sur $]a,\infty[$, bornées sur tout $]a,b]$, telles que 
\begin{equation}
	\lim_{x\to\infty}\frac{ f(x) }{ g(x) }=L
\end{equation}
existe dans $\bar\eR$.
\begin{enumerate}
\item Si $L\neq\infty$ et $\int_{a}^{\infty}g(x)$ existe, alors $\int_a^{\infty}f(x)dx$ existe,
\item Si $L\neq 0$ et si $\int_a^{\infty}f(x)dx$ existe, alors $\int_a^{\infty}g(x)dx$ existe,
\end{enumerate}
\end{corollary}

\begin{corollary}[Critère des fonctions test]			\label{CorCritFonsTest}
Soit $f(x)$ une fonction mesurable et positive ou nulle sur $]a,\infty[$ et bornée pour tout $]a,b]$. Nous posons
\begin{equation}
	L(\alpha)=\lim_{x\to\infty}x^{\alpha}f(x),
\end{equation}
et nous supposons qu'elle existe.
\begin{enumerate}
\item Si il existe $\alpha>1$ tel que $L(\alpha)\neq\infty$, alors $\int_a^{\infty}f(x)dx$ existe,
\item Si il existe $\alpha\leq1$ et $L(\alpha)\neq 0$, alors $\int_a^{\infty}f(x)dx$ n'existe pas.
\end{enumerate}
\end{corollary}

\begin{corollary}		\label{CorAlphaLCasInteabf}
	Soit $f\colon ]a,b]\to \eR$ une fonction mesurable, positive ou nulle, et bornée sur $[a+\epsilon,b]$ $\forall\epsilon>0$. Si $\lim_{x\to a}(x-a)^{\alpha}f(x)=L$ existe, alors
	\begin{enumerate}
		\item Si $\alpha<1$ et $L\neq\infty$, alors $\int_a^bf(x)dx$ existe,
		\item Si $\alpha\geq 1$ et $L\neq 0$, alors $\int_a^bf(x)dx$ n'existe pas.
	\end{enumerate}
\end{corollary}

%---------------------------------------------------------------------------------------------------------------------------
					\subsection{Intégrales convergentes}
%---------------------------------------------------------------------------------------------------------------------------

\begin{definition}

Soit $f$, une fonction mesurable sur $[a,\infty[$, bornée sur tout intervalle $[a,b]$. On dit que l'intégrale
\begin{equation}
	\int_a^{\infty}f(x)dx
\end{equation}
\defe{converge}{intégrale!convergente} si la limite
\begin{equation}		\label{EqDEfConvergeZeroInftX}
	\lim_{X\to\infty}\int_a^{X}f
\end{equation}
existe et est finie.
\end{definition}
