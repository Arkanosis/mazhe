\section{Dirac operator on \texorpdfstring{$AdS_2$}{AdS2}}
%-------------------------------------

Why to compute Dirac operator on anti de Sitter spaces ? Let $M=AdS_2$ and $R=AN$ acts on $M$. Let $\mO$ be an open orbit of $R\times M\to M$. In the specific case of $AdS_2$, we have $R=\mO=R\cdot\mfo$. In larger dimensions, there is a $\SO(1,n)$ which causes that the orbit is not exactly the acting group. It is
\[ 
  \mO=\frac{ R }{ R\cap \SO(1,n) }.
\]

\subsection{Clifford algebra and spin group}
%/////////////////////////////////////


As definition, we retain
\begin{equation}
\begin{split}
  AdS_2&\equiv t^2+u^2-x^2=1\\
	&=\frac{ \SO(2,1) }{ \SO(1,1) }.
\end{split}
\end{equation}
Let $V=\eR^{1,1}$ and $e_0$, $e_1$ an orthonormal basis. We pose 
\begin{align*}
  f_0=\frac{ 1 }{2}(e_0+e_1)\quad g_0=\frac{ 1 }{2}(e_0-e_1)
\end{align*}
and we define $\tilde\rho$ by
\begin{subequations}
\begin{align}
  \tilde\rho(f_0)\alpha&=f_0\wedge\alpha\\
	\tilde\rho(g_0)\alpha&=-i(g_0)\alpha
\end{align}
\end{subequations}
where $\alpha\in\Lambda W$, $W$ being the space spanned by $f_0$. More explicitly we have :
\begin{subequations}
\begin{align}
  \tilde\rho (f_0)1&=f_0&\tilde\rho (f_0)f_0&=0\\
\tilde\rho (g_0)1&=0&\tilde\rho (g_0)f_0&=-\eta(f_0,g_0).
\end{align}
\end{subequations}
As element of $\Lambda W$, $f_0$ stands for $\eta(f_0,.)$. If we choose the basis
\[ 
  1=
\begin{pmatrix}
1\\0
\end{pmatrix},
\quad
f_0=
\begin{pmatrix}
0\\1
\end{pmatrix},
\]
 the matrices of $\tilde\rho$ are given by
\[ 
  \tilde\rho(e_0)=
\begin{pmatrix}
0&-1/2\\
1&0
\end{pmatrix},
\quad
\tilde\rho(e_1)=
\begin{pmatrix}
0&1/2\\
1&0
\end{pmatrix}.
\]
Up to a change of basis,
\[ 
  \gamma_0=
\begin{pmatrix}
0&1\\1&0
\end{pmatrix},
\quad
\gamma_1=
\begin{pmatrix}
0&-1\\1&0
\end{pmatrix}
\quad
\gamma_0\gamma_1=
\begin{pmatrix}
1&0\\0&-1
\end{pmatrix},
\]
and a general element of $Cl_{(1,1)}$ reads
\[ 
  x\gamma_0+y\gamma_1+u\eR+v\gamma_0\gamma_1=
\begin{pmatrix}
u+v&x-y\\
x+y&u-v
\end{pmatrix}.
\]
With the change of basis $e_1\to ie_1$, we write it under a more simple form :
\begin{equation}
Cl_{(1,1)}\leadsto 
\begin{pmatrix}
\alpha&\beta\\
\bar\beta&\bar\alpha
\end{pmatrix}
\end{equation}
with $\alpha,\beta\in\eC$. In particular, an element of $V$, i.e. a combination of $\gamma_0$ and $\gamma_1$ is
\begin{equation}
V\leadsto
\begin{pmatrix}
0&\xi\\
\overline{\xi}&0
\end{pmatrix}.
\end{equation}
Let
\[ 
  1=
\begin{pmatrix}
1&0\\0&1
\end{pmatrix},
\quad
a=
\begin{pmatrix}
i&0\\0&-i
\end{pmatrix},
\quad b=
\begin{pmatrix}
0&1\\1&0
\end{pmatrix},
\quad
c=
\begin{pmatrix}
0&i\\-i&0
\end{pmatrix}
\]
Let us now determine $\alpha$, the extension of $-\id|_V$ into an automorphism and $\tau$, the extension of $\id|_V$ into an anti-automorphism. We have $\alpha(b)=-b$, $\alpha(c)=-c$, $\tau(b)=b$ and $\tau(c)=c$. We find the others by virtue of relations $bc=-a$ and $b^2=1$. Finally
\begin{equation}
\begin{aligned}
  \alpha(1)&=1&\alpha(a)&=a\\
\alpha(b)&=-b&\alpha(c)&=-c
\end{aligned}
\end{equation}
and
\begin{equation}
\begin{aligned}
  \tau(1)&=1&\tau(a)&=-a\\
\tau(b)&=b&\tau(c)&=c.
\end{aligned}
\end{equation}

The condition for $s\in Cl_{(1,1)}$ to belongs to $\Gamma_{(1,1)}$ is that $\alpha(s)v s^{-1}\in V$ for all $v\in V$. If we consider $s=
\begin{pmatrix}
\alpha&\beta\\\bar\beta&\bar\alpha
\end{pmatrix}$,
we have
\[ 
  \alpha(s)=
\begin{pmatrix}
\alpha&-\beta\\-\bar\beta&\bar\alpha
\end{pmatrix},
\text{ and }
s^{-1}=\frac{1}{ | \alpha |^2-| \beta |^2 }
\begin{pmatrix}
\bar\alpha&-\beta\\-\bar\beta&\alpha
\end{pmatrix}.
\]
If we impose $\alpha(s)v s^{-1}$ to be of the form $\begin{pmatrix}
0&\eta\\\bar\eta&0
\end{pmatrix}$ for all $v$ of the form $\begin{pmatrix}
0&\xi\\\bar\xi&0
\end{pmatrix}$, we find $\Reel(\bar\alpha\beta\bar\xi)=0$ and the $\bar\alpha\beta=0$. So generators of $\Gamma_{(1,1)}$ are
\begin{equation}
\Gamma_{(1,1)}\leadsto
\begin{pmatrix}
\alpha&0\\0&\bar\alpha
\end{pmatrix},\quad
\begin{pmatrix}
0&\beta\\\bar\beta&0
\end{pmatrix}.
\end{equation}
Elements of $Spin_{(1,1)}$ are elements of $\Gamma_{(1,1)}^+$ such that $\tau(s)=s^{-1}$. So
\begin{equation}
Spin_{(1,1)}\leadsto
\begin{pmatrix}
\alpha&0\\0&\bar\alpha
\end{pmatrix},\text{ such that }| \alpha |^2=1.
\end{equation}
We recognize $Spin_{(1,1)}=U(1)$.

\begin{probleme}
This is wrong: in fact $\Spin(1,1)\neq U(1)$.
\end{probleme}

\subsection{Relation between \texorpdfstring{$SU(1,1)$}{SU(1,1)} and \texorpdfstring{$\SO(2,1)$}{SO(2,1)}}
%//////////////////////////////////

A general matrix of $SU(1,1)$ is
\[ 
  U=\begin{pmatrix}
\alpha&\beta\\\bar\beta&\bar\alpha
\end{pmatrix},
\quad
U^{-1}=\begin{pmatrix}
\bar\alpha&-\beta\\-\bar\beta&\alpha
\end{pmatrix}
\]
with $| \alpha |^2 - | \beta |^2=1$. They are matrices which fulfil $\det U=1$ and $U^+g=gU^{-1}$.  If we denote by $V$ the space of matrices of the form $(r,z)=\begin{pmatrix}
r&\bar z\\z&r
\end{pmatrix}$ with $r\in\eR$ and $z\in\eC$, we have a bijection $\psi\colon \eR^{1,1}\to V$ given by
\[ 
  \begin{pmatrix}
u\\t\\x
\end{pmatrix}\mapsto
\begin{pmatrix}
x&t-iu\\ t+iu&x
\end{pmatrix}.  
\]
It becomes an isometry if we pose $\| (r,z) \|=z\bar z-r^2=-\det(r,z)$. The group $SU(1,1)$ has an isometric action on $V$ given by
\[ 
  Uv=UvU^{\dag}.
\]
We immediately remark that $Uv=(-U)v$. We define
\begin{equation}
\begin{aligned}
 T\colon SU(1,1)&\to \SO(2,1) \\ 
T(U)\begin{pmatrix}
u\\t\\x
\end{pmatrix}&= \psi^{-1}\big( U\psi\begin{pmatrix}
u\\t\\x
\end{pmatrix}U^{\dag} \big). 
\end{aligned}
\end{equation}

Now we want to know when $T(U)=T(\tilde U)$. Using the fact that $U^{-1}=gU^{\dag}g$ in the condition $UvU^{\dag}=\tilde Uv\tilde U^{\dag}$, we find
\[ 
  VvV^{\dag}=v
\]
with $V=\tilde U^{-1}U$. Then imposing
\[ 
  \begin{pmatrix}
r&\bar z\\z&r
\end{pmatrix}=
\begin{pmatrix}
\alpha&\beta\\\bar\beta&\bar\alpha
\end{pmatrix}
\begin{pmatrix}
r&\bar z\\z&r
\end{pmatrix}
\begin{pmatrix}
\bar \alpha&\beta\\\bar\beta&\alpha
\end{pmatrix},
\]
we find $T(U)=T(\tilde U)\Leftrightarrow \tilde U=\pm U$. We have
\[ 
  T\begin{pmatrix}
i\\&-i
\end{pmatrix}
=
\begin{pmatrix}
-1\\&-1\\&&1
\end{pmatrix}.
\]
The map $T\colon SU(1,1)\to \SO(2,1)$ is a double covering.

We are now going to explicitly compute the map $T$. First :
\[ 
 \begin{split}
\begin{pmatrix}
\alpha&\beta\\
 \bar\beta&\bar \alpha
\end{pmatrix}
\begin{pmatrix}
r&\bar z\\
z&r
\end{pmatrix}
&
\begin{pmatrix}
\bar\alpha&\beta\\
\bar\beta&\alpha
\end{pmatrix}=\\
&\begin{pmatrix}
\bar\alpha(\alpha r+\beta z)+\bar\beta(\alpha\bar z+\beta r)& \beta(\alpha r+\beta z)+\alpha(\alpha\bar z+\beta r)\\
\bar\alpha(\bar\beta+\bar\alpha z)+\bar\beta(\bar\beta\bar z+\bar\alpha r)&\beta(\bar\beta r+\bar\alpha z)+\alpha(\bar\beta\bar z+\bar\alpha r)
\end{pmatrix}.
\end{split} 
\]
When we pose $z=0$ and $r=1$, i.e., when we look at $\begin{pmatrix}
0\\0\\1
\end{pmatrix}$, we find
\[ 
  \begin{pmatrix}
\bar\alpha\alpha+\bar\beta\beta&2\beta\alpha\\
2\bar\alpha\bar\beta& \beta\bar\beta+\alpha\bar\alpha
\end{pmatrix}
\]
which corresponds to $x=\bar\alpha\alpha$, $t-iu=2\alpha\beta$ and $t+iu=2\bar\alpha\bar\beta$. We conclude that
\[ 
  T\begin{pmatrix}
\alpha&\beta\\\bar\beta&\bar\alpha
\end{pmatrix}
=
\begin{pmatrix}
.&.&i(\alpha\beta-\bar\alpha\bar\beta)\\
.&.&\alpha\beta+\bar\alpha\bar\beta\\
.&.&\alpha\bar\alpha+\beta\bar\beta
\end{pmatrix}.
\]
Similar computations lead to
\begin{equation}
T\begin{pmatrix}
\alpha&\beta\\
\bar\beta&\bar\alpha
\end{pmatrix}
=
\begin{pmatrix}
\frac{ \bar\alpha^2+\alpha^2-\beta^2-\bar\beta^2 }{2} &		\frac{ i }{2}(\alpha^2-\bar\alpha^2+\beta^2-\bar\beta^2) &	i(\alpha\beta-\bar\alpha\bar\beta)\\
\frac{ i }{2}(\beta^2-\bar\beta^2+\bar\alpha^2-\alpha^2) &	\frac{ 1 }{2}(\alpha^2+\bar\alpha^2+\beta^2+\bar\beta^2) &	\alpha\beta+\bar\alpha\bar\beta\\
i(\bar\alpha\beta-\bar\beta\alpha)			&	\bar\alpha\beta+\bar\beta\alpha 			&	\alpha\bar\alpha+\beta\bar\beta
\end{pmatrix}
\end{equation}

\subsection{Spin structure on \texorpdfstring{$AdS_2$}{AdS2}}
%/////////////////////////////////////////

We are going to build elements of the following spin structure:
\[
\xymatrix{ \Spin(1,1) \ar@{~>}[r]& SU{(1,1)} \ar[rr]^{\displaystyle\varphi} \ar[rd]_{\displaystyle\pi} && \SO(AdS_2) \ar[ld]^{\displaystyle p}&\SO(2,1) \ar@{~>}[l]  \\& & AdS_2 }
\]
First let $\{ e_t,e_u,e_x \}$ be a basis of $\eR^{1,1}$ with $e_t\in AdS_2$, $e_t\cdot e_t=e_u\cdot e_u=-e_x\cdot e_x=-1$. We suppose that $e_u$ and $e_x$ span tangent space at $e_t$. Let $T$ be a representation of $\SO(1,1)$ on $\eR^{2,1}$  which leaves $e_t$ unchanged: $T(A)e_t=e_t$ for all $A\in \SO(1,1)$. To each element $B\in \SO(AdS_2)$, one can associate an element of $B'\in \SO(AdS_2)$ such that $B$ has the form
\begin{equation} \label{eq_blaseAdS}
  B=\{ B'e_u,B'e_x \}_{B'e_t}.
\end{equation}
We define
\[ 
  p(B)=B'e_t.
\]
Now the action of $A\in \SO(1,1)$ on $B\in \SO(AdS_2)$ is defined, if $B$ has the form \eqref{eq_blaseAdS}, by
\begin{equation}
  B\cdot A=\{ T(A)B'e_u,T(A)B'e_x \}_{B'e_t}.
\end{equation}
The map $\varphi\colon SU(1,1)\to \SO(AdS_2)$ is given by
\begin{equation}
\big( \varphi(U) \big)'=(T\circ S)(U),
\end{equation}
and the projection $\pi\colon SU(1,1)\to AdS_2$, 
  $\pi=p\circ\varphi$.

The group $\Spin(1,1)$ must act on $SU(1,1)$; we define
\begin{equation}
U\cdot s=\varphi^{-1}\big( \varphi(U)\cdot \chi(s) \big).
\end{equation}
We have $\pi(U\cdot s)=\pi(U)$ because
\[ 
  \pi(U\cdot s)=p\big( \varphi(U)\cdot\chi(s) \big),
		=\big[ \varphi(U)\cdot \chi(s) \big]'e_t
		=\varphi(U)\circ T(\chi(s))e_t
		=\varphi(U)'e_t
		=\pi(U).
\]
We have used the fact that  $\chi(s)\in \SO(1,1)$ and that, therefore, $(T\circ\chi)(s)e_t=e_t$.


\subsection{Spinor bundle and connection}
%///////////////////////////////////////

We define $S=\Lambda W$ where $W$ is the (one dimensional) space spanned by $f_0$ and we define 
\begin{equation} \label{eq_mSSUrho}
  \mS= SU(1,1)\times_{\rho}S
\end{equation}
where $\rho\colon Spin_{(1,1)}\times\Lambda W\to \Lambda W$ is the representation of $Spin_{(1,1)}$ on $SU(1,1)$ given by
\begin{equation}
  \rho(s,\alpha)=\tilde\rho(s)\alpha.
\end{equation}
Recall that $\alpha$ is either a scalar either a multiple of $f_0$. The equivalence relation which arises in equation \eqref{eq_mSSUrho} is
\begin{equation}
  (U,\alpha)\sim(U\cdot s,\rho(s^{-1})\alpha).
\end{equation}
The projection is
\[ 
  \pi_{\mS}[(U,\alpha)]=\pi(U).
\]

For the connection on $\SO(AdS_2)$, we want that horizontal vector are tangent vectors to curves formed by parallel transport. In other word, a path 
\[
B(s)=\{ B'(s)e_u,B'(s)e_x \}_{B'(s)e_t}
\]
 has horizontal tangent vector if $B'(s)e_i$ ($i=u,x$) is a parallel transport of $B'(0)e_i$ along the curve $B'(s)e_e$ on $AdS_2$. Here, $B'(s)$ denotes the matrix of $\SO(2,1)$ associated with the basis $B(s)$ : the prime doesn't denotes a derivation. Let us define the $so(1,1)$ valued connection $1$-form which corresponds to this intuition. We consider $b_i(s)$ the parallel transported along the curve $B'(s)e_t$ of $B'(0)e_i$, and $A(s)$, the matrix of $\SO(1,1)$ such that $A(s)B'(s)e_i=b_i(s)$ ($i=u,x$). The definition is
\[ 
  \omega(\dot B)=\Dsdd{ A(s) }{s}{0}.
\]


\begin{proposition}
It is a connection $1$-form.
\end{proposition}

\begin{proof}
First we consider a fundamental vector field
\[ 
  X^*_B=\Dsdd{ B\cdot e^{-tX} }{t}{0}=\Dsdd{ \{ T( e^{-tX})B'e_u,T( e^{-tX})B'e_x \}_{B'e_t} }{t}{0}.
\]
The path in $AdS_2$ on which this path in $\SO(AdS_2)$ is build is constant: it is $B'e_t$. So the parallel transport is constant and the path $A(s)$ is given by
\[ 
  A(s)T( e^{-tX})B'e_u=B'e_u
\]
and $\omega(X^*_B)=X$.

It remains to be proved that for all $B\in \SO(AdS_2)$, $g\in \SO(1,1)$ and $X\in T_B\SO(AdS_2)$,
\begin{equation}
\omega\big( (dR_g)_BX \big)=\Ad(g^{-1})\omega_B(X).
\end{equation}
We give $X$ by the path 
\[ 
  X(s)=\{ B'(s)e_u,B'(s)e_x \}_{B'(s)e_t}.
\]
 The differential $dR_g$ gives rise to the new path
\[ 
  (dR_gX)(s)=\{ gB'(s)e_u,gB'(s)e_x \}_{B'(s)e_t}.
\]
Let $b_i(s)$ be the parallel transport of $B'(0)e_i$ ($i=u,x$) along the path $B'(s)e_t$. We have to compute $\omega_B(X)$ with $A(s)$ defined by $A(s)B'(s)e_i=b_i$. The parallel transport of $gB'(0)e_i$ is given by $gb_i$. Therefore $\omega(dR_gX)$ is given by the path $A^g(s)$ which satisfies $A^g(s)gB'(s)e_i=gA(s)B'(s)e_i$. So
\[ 
  A^g(s)=gA(s)g^{-1}
\]
and
\[ 
  \Dsdd{ A^g(s) }{s}{0}=\Ad(g)\omega(X).
\]

\end{proof}

\subsection{Clifford algebra \texorpdfstring{$(1,1)$}{(1,1)}}
%-------------------------------------

We consider the left invariant vector fields
\begin{subequations}
\begin{align}
  \tilde{e}_J(r_0)&=\Dsdd{ r_0 e^{-sJ} }{s}{0}=-r_0J\\
\tilde{e}_L(r_0)&=\Dsdd{ r_0 e^{-sL} }{s}{0}=-r_0L.
\end{align}
\end{subequations}
More precisely, we consider the vectors given by action of these matrices on the ``base point'' $\begin{pmatrix}
0\\1\\0
\end{pmatrix}$. Hence
\begin{align}
\tilde{e}_J(r_0)=-r_0\begin{pmatrix}
0\\0\\1
\end{pmatrix},\quad
\tilde{e}_L(r_0)=-r_0\begin{pmatrix}
1\\0\\0
\end{pmatrix}
\end{align}
and 
\[ 
  g=\begin{pmatrix}
1\\&-1
\end{pmatrix}.
\]
Remark that this metric is constant (it does not depend on $r_0$) because $r_0$ is an isometry. For this reason, we now turn our attention to Clifford algebra and spin group for $V=\eR^{1,1}$. Following matrices fulfill relation \eqref{3101r3}
\[ 
  \gamma_J=\begin{pmatrix}
0&1\\1&0
\end{pmatrix},\quad
\gamma_L=\begin{pmatrix}
0&-1\\1&0
\end{pmatrix}. 
\]
The complete Clifford algebra has the following matrices too :
\[ 
  1=\begin{pmatrix}
1&0\\0&1
\end{pmatrix},\quad
\gamma_{JL}=\begin{pmatrix}
-1\\&1
\end{pmatrix}.
\]
The Clifford algebra is nothing else than $GL(2,\eR)$, the set of all real $2\times 2$ matrices. From definitions, one can check that
\[ 
 \begin{aligned} 
\alpha(J)&=-J	&\tau(J)&=J\\
\alpha(L)&=-L	&\tau(L)&=L\\
\alpha(JL)&=JL	&\tau(JL)&=-JL\\
\alpha(1)&=1	&\tau(1)&=1
\end{aligned}
\]
Inverse and $\alpha$ of general element in $\Cl(1,1)$ are given by
\[ 
  \begin{pmatrix}
p&q\\r&s
\end{pmatrix}^{-1}=
\frac{1}{ ps-qr }\begin{pmatrix}
s&-q\\-r&p
\end{pmatrix},\quad
\alpha\begin{pmatrix}
p&q\\r&s
\end{pmatrix}=
\begin{pmatrix}
p&-q\\-r&s
\end{pmatrix}.
\]
A general element in $\eR^{1,1}$ is $\begin{pmatrix}
0&\alpha\\\beta&0
\end{pmatrix}$ with $\alpha,\beta\in\eR$, so the condition to belongs to $\Gamma(1,1)$ is that 
\[ 
  \frac{1}{ ps-qr }\begin{pmatrix}
p&-q\\-r&s
\end{pmatrix}
\begin{pmatrix}
0&\alpha\\\beta&0
\end{pmatrix}
\begin{pmatrix}
s&-q\\-r&p
\end{pmatrix}
\]
belongs to $\eR^{1,1}$ for all $\alpha$ and $\beta$. It requires, among others, that $qs\beta-rp\alpha=0$ for all $\alpha$ and $\beta$. Hence $qs=rp=0$, but the alternatives $p=r=0$ and $q=s=0$ are ruled out because we want the determinant $ps-qr$ to be non zero. Therefore, $\Gamma(1,1)$ is generated by
\[ 
  \Gamma(1,1)\leadsto
\begin{pmatrix}
p&0\\0&s
\end{pmatrix},\begin{pmatrix}
0&q\\r&0
\end{pmatrix}.
\]
The latter belongs to $\eR^{1,1}$, so
\[ 
  \Gamma^+(1,1)\leadsto\begin{pmatrix}
z+c&0\\0&z-c
\end{pmatrix}=z\mtu+c\gamma_{JL}.
\]
From 
\[ 
  \tau(z\mtu+c\gamma_{JL})=z\mtu-c\gamma_{JL},
\]
elements in $\Spin(1,1)$ are subject to the relation
\[ 
  \begin{pmatrix}
s&0\\0&p
\end{pmatrix}=
\frac{1}{ ps }\begin{pmatrix}
s&0\\0&p
\end{pmatrix}.
\]
As consequence, we find
\begin{equation}
\Spin(1,1)=\eR_0\leadsto
\begin{pmatrix}
1/p\\&p
\end{pmatrix}.
\end{equation}
 If we put (see decomposition \eqref{eq:expo_ANK})
\[ 
  A=\begin{pmatrix}
 e^{a}\\& e^{-a}
\end{pmatrix},
\]
we have $\Spin(1,1)=A\times\eZ_2$. Let us check that $\Spin(1,1)$ is a double covering of $\SO_0(1,1)$. We know that 
\[ 
  \SO(1,1)=\begin{pmatrix}
\cosh\xi&\sinh\xi\\\sinh\xi&\cosh\xi
\end{pmatrix}\times\eZ_2
\]
while $\SO_0(1,1)$ is 
\[ 
  \SO_0(1,1)=\begin{pmatrix}
\cosh\xi&\sinh\xi\\\sinh\xi&\cosh\xi
\end{pmatrix}=\eR.
\]
This structure of $\SO(1,1)$ comes from the fact (true for all $\SO(1,n)$) that $| \Lambda^0_0 |\geq1$ when $\Lambda$ is a Lorentz transformation. So $\mtu$ and $-\mtu$ cannot belong to the same connected component. Note that $\cosh\xi\geq 1$.
We see intuitively how to cover two times $\eR$ with $\eR_0$. Let us see how the map $\chi$ does that. From definition, $\chi(x)y=\alpha(x)yx^{-1}$, so it is easy to see that 
\[ 
  \chi(1)=\chi(-1)=\id|_{\eR^{1,1}}
\]

\subsection{Parallel transport}
%------------------------------

We have a connection on the frame bundle of $AdS_2$ and we wan to lift the vectors $\tilde{e}_J$ and $\tilde{e}_L$, i.e. we consider a point 
\[ 
  \xi_0=(r_0,v_1,v_2)\in \SO(AdS_2)
\]
where $v_1$ and $v_2$ form an orthonormal (in the sense of $g$) basis of $T_{r_0}AdS_2$. Then we have to find a path $s\to\xi(s)$ in $\SO(AdS_2)$ such that $\xi(0)=\xi_0$, $\omega(\xi'(0))=0$ and $dp\xi'(0)=\tilde{e}_a$. The latter condition allows us to compute $r(s)$ in the expression
\[ 
  \xi(s)=(r(s),v_1(s),v_2(s)),
\]
namely, $r(s)$ is the path of $\tilde{e}_a$. The condition to be horizontal imposes that vectors $v_i(s)$ are parallel transport of $v_i$ along $\tilde{e}_a$. So we have to compute the different $T_a(\tilde{e}_b)(s)$ which is the parallel transported of $\tilde{e}_b$ along the path of $\tilde{e}_a$ at a distance $s$; this is an element of $T_{\tilde{e}_a(s)}AdS_2$. It will be decomposed in the basis 
\[ 
 \begin{split}
  \tilde{e}_J\big( \tilde{e}_a(s) \big)&=-r_0 e^{-s a}J\\
\tilde{e}_L\big( \tilde{e}_a(s) \big)&=-r_0 e^{-sJ}L.
\end{split} 
\]
where we imply the action on the base point $\begin{pmatrix}
0\\1\\0
\end{pmatrix}$. For notational simplicity, from now we write $a(s)$ instead of $\tilde{e}_a(s)$. Various products are easy to compute; for example
\[ 
  \tilde{e}_J(J(s))\cdot\tilde{e}_L(J(s))
		=r_0 e^{-sJ}J\cdot r_0 e^{-sJ}L\\
		=J\cdot L\\
		=\begin{pmatrix}
0\\0\\1
\end{pmatrix}\cdot\begin{pmatrix}
1\\0\\0
\end{pmatrix}\\
		=0
\]
because $r_0 e^{-sJ}$ is an isometry.
In general :
\[ 
  \tilde{e}_a\big( c(s) \big)\cdot \tilde{e}_b\big( c(s) \big)=a\cdot b
\]
Now we pose in general
\[ 
  T_a(\tilde{e}_b)(s)=\alpha(s)\tilde{e}_b\big( a(s) \big)+\beta(s)\tilde{e}_L\big( a(s) \big),
\]
and we want to find the (real valued) functions $\alpha$ and $\beta$. Parallel transport fulfils two conditions: the norm and the angle with the path are constant. This leads us to two conditions :
\begin{subequations}
\begin{align}
  T_a\big( \tilde{e}_b(s) \big)\cdot T_a\big( \tilde{e}_b(s) \big)&=b\cdot b\\
 T_a\big( \tilde{e}_b(s) \big)\cdot \tilde{e}_a\big( a(s) \big)&=b\cdot a.
\end{align}
\end{subequations}
These equations extends to
\begin{subequations}
\begin{align}
  \beta(s)^2-\alpha(s)^2&=b\cdot b\\
\alpha(s)J\cdot a+\beta(s)L\cdot a&=b\cdot a.
\end{align}
\end{subequations}
There are four cases to be considered following that $a=J,L$ and $b=J,L$. The result is that
\begin{equation}
T_a\big(\tilde{e}_b \big)=\tilde{e}_b,
\end{equation}
in other terms, the vectors $\tilde{e}_J$ and $\tilde{e}_L$ are not only parallel vector fields, but each is parallel along the path of the other.

\subsection{Covariant derivative}
%--------------------------------

We will give the horizontal lift of $\tilde{e}_a$ at point
\[ 
  \xi(0)=\{ B_1^b\tilde{e}_b,B_2^c\tilde{e}_c \}_{r_0e_t}
\]
under the form of the path
\[ 
  \xi(s)=\{ B_1^b\tilde{e}_b\big( a(s) \big),B_2^c\tilde{e}_c\big( a(s) \big) \}_{\tilde{e}_a(s)}.
\]
We create a connection on the spinor bundle from the connexion via the formula
\[ 
  \widehat{\nabla_X^E\psi}(\xi)=\overline{ X }_{\xi}(\hat \psi).
\]
In our case, we take $\psi\colon M\to \mS$, or $\hat{\psi}\colon SU(1,1)\to \Lambda W$ such that
\[ 
  \hat{\psi}(U\cdot g)=\rho(g^{-1})\hat{\psi}(U).
\]
Since $\tilde\omega=\varphi^*\omega$, we have $\tilde\omega(X)=\omega(d\varphi X)$ and
\[ 
  \overline{ e }_a{}_{\xi_0}=\varphi^{-1}\big( \tilde{e}_a(s),\ldots \big).
\]
Therefore
\begin{equation}  \label{eq_whidpsinabla}
\widehat{\nabla_a\psi}(\xi_0)=\Dsdd{ (\hat{\psi}\circ\varphi^{-1})\{ B_i^c\tilde{e}_c\big( a(s) \big) \}_{\tilde{e}_a(s)} }{s}{0}
\end{equation}
where $\varphi$ is defined by
\[ 
  \varphi(U)=\{ U\tilde{e}_J,U\tilde{e}_L \}_{Ur_0e_t}.
\]
We have to find
\begin{equation}  \label{eq_varpBic}
  \varphi^{-1}\{ B_i^c\tilde{e}_c\big( a(s) \big) \}_{\tilde{e}_a}.
\end{equation}
Before to write down the inverse of $\varphi$, let us perform some computations.
\[ 
  J=\begin{pmatrix}
&0\\
0&0&1\\
&1
\end{pmatrix},\quad
L=\begin{pmatrix}
0&1&1\\
-1\\
1
\end{pmatrix},
\]
and as far as we only wants to compute derivatives, we can write the exponentials as
\begin{align} 
 e^{sJ}&=\mtu+sJ=\begin{pmatrix}
1\\
&1&s\\
&s&1
\end{pmatrix}\\
 e^{sL}&=\mtu+sL=\begin{pmatrix}
1&s&s\\
-s&1&0\\
s&0&1
\end{pmatrix}.
\end{align}
The path are given by
\begin{equation}
\tilde{e}_a(s)=r_0 e^{-sa}\begin{pmatrix}
0\\1\\0
\end{pmatrix},
\end{equation}
in particular
 \begin{align}
\tilde{e}_J(s)&=r_0\begin{pmatrix}
0\\1\\-s
\end{pmatrix},
&\tilde{e}_L(s)&=r_0\begin{pmatrix}
-s\\1\\0
\end{pmatrix}.
\end{align} 
For the various $\tilde{e}_b\big( a(s) \big)$, we have
\begin{equation}
\tilde{e}_b\big( a(s) \big)=\Dsdd{ a(s) e^{-tb} }{t}{0}\begin{pmatrix}
0\\1\\0
\end{pmatrix}\\
	=\Dsdd{ r_0 e^{-sa} e^{-tb} }{t}{0}\begin{pmatrix}
0\\1\\0
\end{pmatrix}\\
	=-r_0 e^{-sa}b\begin{pmatrix}
0\\1\\0
\end{pmatrix}.
\end{equation}
Results are
\begin{subequations}
\begin{align}
  \tilde{e}_J\big( J(s) \big)&=-r_0\begin{pmatrix}
0\\-s\\1
\end{pmatrix}
&
\tilde{e}_J\big( L(s) \big)&=-r_0\begin{pmatrix}
-s\\0\\1
\end{pmatrix}\\
\tilde{e}_L\big( J(s) \big)&=-r_0\begin{pmatrix}
1\\0\\0
\end{pmatrix}
&
\tilde{e}_L\big( L(s) \big)&=-r_0\begin{pmatrix}
1\\s\\-s
\end{pmatrix}.
\end{align}
\end{subequations}
We finally have to know that
\[ 
  B^c\tilde{e}_c\big( J(s) \big)=-r_0\begin{pmatrix}
B^L\\-sB^J\\B^J
\end{pmatrix},
\quad
B^c\tilde{e}_c\big( L(s) \big)=-r_0\begin{pmatrix}
-sB^J+B^L\\sB^L\\B^J-sB^L
\end{pmatrix}.
\]

Following equation \eqref{eq_varpBic}, in order to write down $\widehat{\nabla_a\psi}$, we have to find $U(s)\in SU(1,1)$ such that
\begin{enumerate}
\item $Ur_0e_t=\tilde{e}_a(s)$,
\item $U\tilde{e}_J=B^c_1\tilde{e}_c\big( a(s) \big)$,
\item $U\tilde{e}_L=B^c_2\tilde{e}_c\big( a(s) \big)$.
\end{enumerate}
If $\overline{ f }$ and $\overline{ g }$ are vectors, solutions in $U$ of equation $Ur_0\overline{ f }=r_0\overline{ g }$ are $U=\AD(r_0)B$ where $B$ fulfils $B\overline{ f }=\overline{ g }$. In the case of $a=J$, the three conditions successively give
\begin{subequations}
\begin{align}
U&=\AD(r_0)\begin{pmatrix}
.&0&.\\
.&1&.\\
.&-s&.
\end{pmatrix}\\
U&=\AD(r_0)\begin{pmatrix}
.&.&B_1^L\\
.&.&-sB_1^J\\
.&.&B_1^J
\end{pmatrix}\\
U&=\AD(r_0)\begin{pmatrix}
-B^L_2&.&.\\
sB_2^J&.&.\\
B_2^J&.&.
\end{pmatrix}.
\end{align}
\end{subequations}
Putting all together in equation \eqref{eq_whidpsinabla} we find
\begin{equation} \label{eq_nabJmoi}
\begin{split}
  \widehat{\nabla_J\psi}(\xi_0)&=\frac{ d }{ ds }\hat{\psi}\AD(r_0)
\begin{pmatrix}
-B_2^L	&	0	&	B_1^L\\
sB_2^J	&	1	&	-sB_1^J\\
B_2^J	&	-s	&	B_1^J
\end{pmatrix}\\
	&=d\hat{\psi}\AD(r_0)\begin{pmatrix}
0&0&0\\
B_2^J&0&-B_1^J\\
0&-1&0
\end{pmatrix}.
\end{split}
\end{equation}
The same with $L$ instead of $J$ leads to
\begin{equation} \label{eq_nabLmoi}
\widehat{\nabla_L \psi}(\xi_0)=d\hat{\psi}\AD(r_0)\begin{pmatrix}
-B_2^J&-1&B_1^J\\
B_2^L&0&B_1^L\\
-B_2^L&0&-B_1^L
\end{pmatrix}.
\end{equation}

However it should be shocking to get $3\times 3$ matrices in $SU(1,1)$ : we had abused between $\SO(2,1)$ and $SU(1,1)$.

\subsection{Another way to write a section (wrong way to do)}
%------------------------------------------------------------

The equivariant function $\hat{\psi}\colon SU(1,1)\to \Lambda W$ fulfills 
\[ 
  \hat{\psi}(U\cdot g)=\rho(g^{-1})\hat{\psi}(U)
\]
for all $g\in\Spin(1,1)$; in particular with $g=-\mtu$,
\begin{equation}
 \hat{\psi}(-U)=-\hat{\psi}(U).
\end{equation}
This gives the idea that it is not impossible to define $\hat{\psi}$ from its projection on $\SO(2,1)$ : we want to get $\tilde{\psi}\colon \SO(2,1)\to \Lambda W$ and define 
\[ 
  \hat{\psi}(U)=\tilde{\psi}\big( T(U) \big).
\]
More precisely, we parametrize $SU(1,1)$ by $\alpha$ and $\beta$ such that $| \alpha |^2-| \beta |^2=1$. Then we divide $SU(1,1)$ into two parts: $\alpha=x+iy$ is green when $x>0$ and when $x=0$, $y<0$; $\alpha$ is red when $x<0$ and when $x=0$, $y>0$. When $\alpha=0$, we classify following $\beta$ in the same way. The result is that $U$ is green if and only if $-U$ is red. For a map $\tilde{\psi}\colon \SO(2,1)\to \Lambda W$, we define
\begin{equation}
\hat{\psi}(U)=
\begin{cases}
\tilde{\psi}\big( T(U) \big)&\text{if $U$ is green}\\
-\tilde{\psi}\big(T(U)\big)&\text{if $U$ is red}
\end{cases}
\end{equation}
We define $T^{-1}\colon \SO(2,1)\to SU(1,1)$ as follows: $T^{-1}(A)$ is the green element of $SU(1,1)$ whose image by $T$ is $A$. In any cases we have
\[ 
\hat{\psi}\circ T^{-1}=\tilde{\psi}.  
\]
The meaning of equations \eqref{eq_nabJmoi} and \eqref{eq_nabLmoi} is that $\AD(r_0)$ is a matrix whose inverse image by $T$ should be given to $\hat{\psi}$; the difficulty is to know which of the two. When $U_0$ is green,
\[ 
 \begin{split}
\widehat{\nabla_a\psi}(U_0)&=\Dsdd{ (\hat{\psi}\circ T^{-1})  \AD(r_0)\Big( \cdots \Big)   }{s}{0} \\
		&=\Dsdd{ \tilde{\psi} \AD(r_0)\Big( \cdots \Big)}{s}{0},
\end{split} 
\]
while when $U_0$ is red,
\[ 
  \widehat{\nabla_a\psi}(U_0)=-\Dsdd{ \tilde{\psi}\AD(r_0)\Big( \cdots \Big) }{s}{0}.
\]
These two show that
\begin{equation}
\widetilde{\nabla_a\psi}\big( T(U_0) \big)=\Dsdd{   \tilde{\psi}\AD(r_0)\Big( \cdots \Big)    }{s}{0}
\end{equation}
All this is only proved in the interior of the green and red regions so that the path $U(s)$ keeps on only one region.

\subsection{Once again}  \label{pg_DiracADsdeux}
%-------------------------

We see $AdS_2$ as\footnote{Here, $G=SL(2,\eR)$} $\mO=\Ad(G)H$ and we consider a base point $o=\Ad(k_0)H$ with $G=SL(2,\eR)=ANK$. Let the principal bundle
\[ 
\xymatrix{%
   A \ar@{~>}[r]^{R}		&	G\ar[d]^{\pi}\\
    &	\mO
}
\]
with $A$ acting on $G$ by $(a,g)\mapsto ga$ and the projection
\begin{equation}
\pi(rk_0a)=\Ad(rk_0a)H.
\end{equation}
where $r\in R$ and $a\in A$. More precisely, the principal bundle we look at is
\begin{equation}
\xymatrix{%
   A \ar@{~>}[r]^{R}		&	\mU_G\ar[d]^{\pi}\\
   	&	\mU
}
\end{equation}
where $\mU_G=Rk_0A$ and $\mU=\pi(\mU_G)=\Ad(Rk_0A)H=\Ad(Rk_0)H=\Ad(R)o$. The $\mU_G$ is so defined in order to be the $\pi^{-1}$ of an orbit $\mU=\Ad(R)o$.

We have a manifold isomorphism $R\simeq\mU$ given by
\[ 
  \phi\colon r\to \Ad(r)o.
\]
How to see a left invariant vector field on $R$ \emph{via} this identification ? 
\[ 
  d\phi\tilde X_r=d\phi\Dsdd{ r e^{tX} }{t}{0}
		=\Dsdd{ \Ad(r)\Ad( e^{tX})o }{t}{0}.
\]
This leads us to consider the following field for $X\in\sR$. We define $\xi_X(rk_0a)\in T_{rk_0a}\mU_G$,
\begin{equation}
  \xi_X(rk_0a)=\Dsdd{ r e^{tX}k_0a }{t}{0}.
\end{equation}
Let's see the projection :
\[ 
\begin{split}
d\pi_{rk_0a}\xi_X(rk_0a)&=\Dsdd{ \pi(r e^{tX}k_0a) }{t}{0}\\
		&=\Dsdd{ \Ad(r e^{tX}k_0a)H }{t}{0}\\
		&=\Dsdd{ \Ad(r e^{tX})o }{t}{0}.
\end{split} 
\]
This gives us the idea to define $X^{\sharp}\in T_{\Ad(rk_0a)H}\mU=T_{\pi(rk_0a)}\mU$ by
\begin{equation}
X^{\sharp}_{rk_0a}=\Dsdd{ \Ad(re^{tX})o }{t}{0},
\end{equation}
which is a good definition because $\pi(rk_0a)=\pi(r'k_0a')$ only when $r=r'$. We put the following connection on $\mU_G$ :
\begin{equation}
\alpha_{rk_0a}(\Sigma)=\left[     \big( dL_{rk_0a}^{-1} \big)_{rk_0a}\Sigma    \right]_{\sA}.
\end{equation}
We hope $\xi_X$ to be the horizontal lift\footnote{We will see in proposition \ref{prop_horliftXdiz} that it is not the case, but for the moment, we hope it.} of $X^{\sharp}$; by construction $d\pi\xi_X=X^{\sharp}$. We have
\[ 
 \begin{split}
\alpha_{rk_0a}(\xi_X)&=\left[ dL_{(rk_0a)^{-1}}\xi_X(rk_0a) \right]_{\sA}\\
		&=\Dsdd{ a^{-1}k_0^{-1}r^{-1}r e^{tX}k_0a }{t}{0}^{\sA}\\
		&=\Dsdd{ a^{-1}\AD(k_0^{-1}) e^{tX}a}{t}{0}^{\sA}\\
		&=\left[ \Ad(a^{-1}k_0^{-1})X \right]_{\sA}.
\end{split} 
\]
One can, by brute force computation\footnote{Or by remarking that $\sA$ is abelian.}, show that the difference $\Ad(ak_0)X-\Ad(k_0)X$ is skew-diagonal when
\[ 
  X=\begin{pmatrix}
a'&n\\0&-a'
\end{pmatrix},
\quad k_0=\begin{pmatrix}
\cos a&\sin a\\\sin a&\cos a
\end{pmatrix},
\quad
a=\begin{pmatrix}
a&0\\0&1/p
\end{pmatrix}.
\]
So $\Ad(a)$ does not change the $\sA$-component of $\Ad(k_0)X$. We conclude that
\begin{equation}
  \alpha(\xi_X)=\left[ \Ad(k_0^{-1})X \right]_{\sA}.
\end{equation}
When $X\in\sR$, we consider $\tilde X_g=(dL_g)_eX$;
\begin{equation} 						 \label{eq_defXtilde}
  \tilde X_{rk_0a}=dL_{rk_0a}X,
\end{equation}
in particular, $\tilde X_r=\Dsdd{ r e^{tX} }{t}{0}$.
We denote by $\tau$ the action
\begin{equation}
\begin{aligned}
 \tau_g\colon\mO&\to \mO \\ 
\tau_g\Ad(r)H&= \Ad(gr)H 
\end{aligned}
\end{equation}
In particular
\[ 
 \begin{split}
d\pi_gdL_g Y&=\Dsdd{ \pi(g e^{tY}) }{t}{0}\\
		&=\Dsdd{ \ad(g e^{tY}H) }{t}{0}\\
		&=\Dsdd{ \tau_g\Ad( e^{tY})H }{t}{0}\\
		&=(d\tau_g)_Hd\pi_e Y,
\end{split} 
\]
thus
\begin{equation}
d\pi\circ dL = d\tau\circ d\pi.
\end{equation}
With definition \eqref{eq_defXtilde}, we have $\alpha(\tilde X)=X_{\sA}$ because
 \[ 
\alpha\big( dL_{rk_0a}X \big)=\left[ dL_{(rk_0a)^{-1}}dL_{rk_0a}X \right]_{\sA}
		=X_{\sA}.
\]
We are now able to find some horizontal lift. 
\begin{proposition}  \label{prop_horliftXdiz}
The horizontal lift of $X^{\sharp}$ is
\[
  \overline{ X^{\sharp} }=\xi_X-\widetilde{  [\Ad(k_0^{-1})X]_{\sA}  }.
\]

\end{proposition}

\begin{proof}
First, we have
\[ 
 \begin{split}
d\pi\overline{ X^{\sharp} }|_{rk_0a}&=d\pi\xi_X-d\pi(dL_{rk_0a})_e[\Ad(k_0^{-1})X]_{\sA}\\
				&=\Dsdd{ \Ad(r e^{tX}o) }{t}{0}-d\tau\,d\pi[\Ad(k_0^{-1})X]_{\sA}.
\end{split} 
\]
The first term is $X^{\sharp}$ while the second is zero because if $A\in\sA$, 
\[ 
 \begin{split}
d\pi A&=\Dsdd{ \pi( e^{tA}) }{t}{0}\\
		&=\Dsdd{ \Ad( e^{tA})H }{t}{0}\\
		&=0.
\end{split} 
\]
On the other hand,
\[ 
\alpha(\overline{ X^{\sharp} })=\alpha(\xi_X)-[\Ad(k_0)^{-1}X]_{\sA}=0.
\]
\end{proof}

Now we prove that the function $\overline{ X^{\sharp} }\cdot\hat{\psi}$ is equivariant, and therefore that the definition
\[ 
  \widehat{\nabla_{X^{\sharp}}\psi}=\overline{ X^{\sharp} }\cdot\hat{\psi}
\]
works. Using equivariance of $\hat{\psi}$, 
\begin{equation}
\begin{aligned}
  \overline{ X^{\sharp} }\cdot \hat{\psi}(ga_1)&=\Dsdd{ \hat{\psi}(\xi_X(t) }{t}{0}-\hat{\psi}\big( dL_{ga_1}[\Ad(k_0^{-1})X]_{\sA} \big)\\
		&=\Dsdd{ \hat{\psi}\big( r e^{tX}k_0aa_1 \big) }{t}{0}-\Dsdd{ \hat{\psi}\big( ga_1 e^{t[\Ad(k_0^{-1})X]_{\sA}} \big) }{t}{0}\\
		&=\Dsdd{ \rho(a_1)\hat{\psi}\big( r e^{tX}k_0a \big) }{t}{0}-\Dsdd{ \rho(a_1)\hat{\psi}\big( g e^{t[\Ad(k_0^{-1})X]_{\sA}} \big) }{t}{0},\\
		&=\rho(a_1)\hat{\psi}(\xi_X)-\Dsdd{ \rho(a_1)\hat{\psi}\big( \widetilde{[\Ad(k_0^{-1})X]_{\sA} }|_g \big) }{t}{0}\\
		&=\rho(a_1)\hat{\psi}(\xi_X)-\rho(a_1)\widetilde{ [\Ad(k_0^{-1})X]_{\sA}  }\hat{\psi}(g)\\
		&=\rho(a_1)(\overline{ X^{\sharp} }\cdot\hat{\psi})(g).
\end{aligned}
\end{equation}
for the third line, we used the fact that $\sA$ is abelian
\subsubsection{Clifford algebra for \texorpdfstring{$AdS_2$}{AdS2}}
%----------------------------------------------------------------

Our basis of $\sA\oplus\sN$ is
\[ 
  H=\begin{pmatrix}
1&0\\0&-1
\end{pmatrix}\quad\text{and}\quad
E=\begin{pmatrix}
0&1\\0&0
\end{pmatrix}
\]
and we choose
\[ 
  o=\Ad(k_0)H=\cos(2k_0)H+\sin(2k_0)(E+F).
\]
Since (at first order in $t$) $\Ad( e^{tH})o=\cos(2k_0)\mtu+\sin(2k_0)\big( E+2tE+F-2tF \big)$,
\[ 
  H^{\sharp}_{rk_0a}=2\sin(2k_0)\Ad(r)(E-F),
\]
and
\[ 
  E^{\sharp}_{rk_0a}=\Ad(r)\big( -2\cos(2k_0)E+\sin(2k_0)H \big).
\]
We have to compute the metric matrix for this basis; we know from equation \eqref{eq_KillAdinvariant}, the Killing form is $\Ad$-invariant and $(\mathfrak{sl}(2,\eR),B)\simeq(\eR^3,\eta_{21})$. So the $\Ad(r)$ disappears in the computation of $B(X^{\sharp},Y^{\sharp})$. We get
\[
\begin{split}
B(H^{\sharp},H^{\sharp})&=4\sin^2(2k_0)B(E-F,E-F)\\
		&=-32\sin^2(2k_0)\\
B(E^{\sharp},E^{\sharp})&=\sin^2(2k_0)B(H,H)\\
		&=8\sin^2(2k_0)\\
B(E^{\sharp},H^{\sharp})&=-4\sin(2k_0)\cos(2k_0)B(E,E-F)+2\sin^2(2k_0)B(H,E-F)\\
		&=16\sin^2(2k_0)\cos(2k_0).
\end{split}
\]
So the metric is in the basis $\{ H^{\sharp},E^{\sharp} \}$
\begin{equation}
g=
\begin{pmatrix}
-32\sin^2(2k_0) & 16\sin(2k_0)\cos(2k_0)\\
16\sin(2k_0)\cos(2k_0) & 8\sin^2(2k_0)
\end{pmatrix}.
\end{equation}
When we consider the orbit of $E+F$, we choose $o=E+F$, i.e. $\cos(2k_0)=0$, $\sin(2k_0)=1$ so that
\begin{equation}
H^{\sharp}_{rk_0a}=2\Ad(r)(E-F),\quad E^{\sharp}_{rk_0a}=\Ad(r)H,
\end{equation}
and
\[ 
  g=\begin{pmatrix}
-32&0\\0&8
\end{pmatrix};
\]
in the case of the orbit of $-(E+F)$, we get the same. The negative vector is $H^{\sharp}$ and the positive one is $E^{\sharp}$.

\subsubsection{Identification \texorpdfstring{$\sQ\leftrightarrow\Lambda W$}{QW}}
%////////////////////////////////////////////////////////////////////////////////

We want a linear bijection $\phi\colon \sQ\to \Lambda W$ such that
\[ 
  \rho(s)\phi(X)=\phi\big( \rho(s)X \big)
\]
where the left hand side action of $\Spin$ is the usual on $\Lambda W$ while the right hand side one remains to be defined. The implementation of this is easy: we can take any bijection between $\sQ$ and $\Lambda W$ and define
\begin{equation}
  \rho(s)X=\phi^{-1}\big( \rho(s)\phi(X) \big).
\end{equation}

Spinors on $AdS_2$ are given by equivariant functions $\hat{\psi}\colon \mU_G\to \Lambda W$ which are now replaced by $\tilde{\psi}\colon R\to \sQ\simeq\Lambda W$ by
\[ 
  \hat{\psi}(rk_0a)=\rho(a^{-1})\tilde{\psi}(r).
\]
So the set of sections of the spinor bundle over $\mU$ is
\[ 
  \Gamma_{\mU}\simeq  C^{\infty}(R,\Lambda W).
\]

\subsubsection{Covariant derivative}
%////////////////////////////////

The aim is now to compute 
\[ 
\begin{split}
   \widetilde{\nabla_{X^{\sharp}}\psi}(r)&=\widehat{\nabla_{X^{\sharp}}\psi}(rk_0)\\
		&=\overline{ X^{\sharp} }\cdot \hat{\psi}|_{rk_0}\\
		&=\big( \xi_X-\widetilde{ [\Ad(k_0^{-1})X]_{\sA}  } \big)\cdot\hat{\psi}|_{rk_0}\\
		&=\Dsdd{ \tilde{\psi}(r e^{tX}) }{t}{0}-\Dsdd{ \rho\big(  e^{t[\Ad(k_0^{-1})X]_{\sA}} \big) }{t}{0}\tilde{\psi}(r)\\
		&=\tilde X_r\tilde{\psi}(r)-d\rho_e\big( [\Ad(k_0^{-1})X]_{\sA} \big)\tilde{\psi}(r).
\end{split}  
\]
Our final formula for the covariant derivative is
\begin{equation}
  \widetilde{\nabla_{X^{\sharp}}\psi}(r)=\tilde X_r\tilde{\psi}-d\rho\big( [\Ad(k_0^{-1})X]_{\sA} \big)\tilde{\psi}.
\end{equation}
The Dirac operator will be a linear combination of vectors of the form
\[ 
  \tilde X+d\rho\big( [\Ad(k_0^{-1})X]_{\sA}  \big).  
\]
Notice that $\tilde X$ is left invariant and the second term is even independent of the point, so the whole is left invariant.

\section{Dirac operator on \texorpdfstring{$AdS_{3}$}{AdS3}}  \label{PgDiracAdSTrois}
%--------------------------------------------------------------

The definition is 
\[ 
  AdS_{3}=\frac{ \SO(2,2) }{ \SO(1,2) },
\]
and the group which acts is the $AN$ of $\SO(2,2)$. The Lie algebra is given by
\[ 
\begin{split}
  \sA&=\{ J_{1},J_{2} \}\\
  \sN&=\{ M,L \}
\end{split}  
\]
which has dimension $4$. So there is a stabiliser. One can prove that for the open orbit of $u=\begin{pmatrix}
0&1\\-1&0
\end{pmatrix}$, the stabiliser is $\{  e^{aJ_{2}} \}$, i.e.
\begin{equation}
  [ e^{aJ_{2}}u]=[u].
\end{equation}
For the spin group, we find 
\[ 
  \Spin(2,1)\simeq SL_2^*(\eR),
\]
the group of $2\times 2$ matrices with determinant equals to $\pm 1$ (cf \cite{Michelson}). Let us recall that the isomorphism $AdS_{3}\simeq SL(2,\eR)$ is given by
\[ 
  SL(2,\eR)=\begin{pmatrix}
t+x&y-u\\
y+u&t-x
\end{pmatrix}
\]
with $u^{2}+t^{2}-x^{2}-y^{2}=1$. For sake of simplicity, we denote $SL(2,\eR)$ by $G$. It is explained in \cite{Clement} that the map
\begin{equation}
\begin{aligned}
 \psi\colon (G\times G)\times AdS_{3}&\to AdS_3 \\ 
(g_{1},g_{2})x&= g_{1}xg_{2}^{-1} 
\end{aligned}
\end{equation}
provides a local isomorphism $G\times G\simeq O(2,2)$. Moreover we have locally :
\[ 
  \frac{ G\times G }{ \eZ_{2} }\simeq \SO(2,2).
\]
At each point $x\in AdS_3$, we have an isomorphism
\[ 
  \SO(2,2)_{x}\simeq \SO(2,1)
\]
where $\SO(2,2)_{x}$ is the stabiliser of $x$ in $\SO(2,2)$. So we define the isomorphism
\[ 
  \chi_{x}\colon \Spin(2,1)\to \SO(2,2)_{x}
\]
which is a double covering. If $d\psi\colon \mG\oplus\mG\to \mathfrak{so}(2,2)$ is the isomorphism of \cite{Clement}, we define $\psi\colon G\times G\to \SO(2,2)$ by
\[ 
  \psi( e^{X})= e^{d\psi X},
\]
which is a good definition because the exponential is surjective on $G\times G$. For each $x\in AdS_3$, we consider the isomorphism
\[ 
  \phi_{x}\colon \SO(2,1)\to \SO(2,2)_{x}
\]
such that $\phi_{x}\big( \SO(2,1) \big)=\SO(2,2)_{x}$.

We define $\chi(s)_i\colon \Spin(2,1)\to \SO(2,2)$ by
\[ 
  \chi(s)=\chi(s)_1v\chi(s)_2.
\]
The choice of $\chi(s)_i$ is not unique. So we define the action of $\Spin(2,1)$ on $G\times G$ by
\begin{equation}
(g,h)\cdot s=\big( \chi(s)_1g,\chi(s)_2^{-1}h \big).
\end{equation}
Therefore we have
\[ 
\begin{split}
  \psi\big( (g,h)\cdot s \big)x&=\chi(s)_1gxh^{-1}\chi(s)_2\\
		&=\chi(s)\big( gxh^{-1} \big)\\
		&=\chi(s)\big(\psi(g,h)x\big).
\end{split}  
\]
\subsection{Spin structure on \texorpdfstring{$AdS_3$}{AdS3}}
%+++++++++++++++++++++++++++++++++++

From previous considerations, the first choice should be
\[ 
  P=\frac{ AN }{ S }\times\Spin(2,1),
\]
but it is easy to remark that $\sR'=\{ J_{1},M,L \}$ is a Lie algebra. So we use the corresponding Lie group $R$ instead of the homogeneous space $AN/S$ (these two are isomorphic). Thus the choice is
\begin{equation}
P=R'\times\Spin(2,1),
\end{equation}
with the projection $\pi\colon P\to AdS_3$,
\[ 
  \pi\big( r',s \big)=\left[ r'\begin{pmatrix}
0&1\\-1&0
\end{pmatrix} \right]
\]
 We consider
\begin{equation}
\begin{aligned}
 \theta\colon R'&\to \mU=Ro \\ 
r'&\mapsto ro=\left[ r'\begin{pmatrix}
0&1\\-1&0
\end{pmatrix} \right].
\end{aligned}
\end{equation}
The projection $\pi\colon P\to \mU$ reads $\pi=\theta\circ\pr_{1}$,
\[ 
  \pi\big( r',s \big)=[ro].
\]
This definition works because for all $a$, there exists a $h\in H$ such that
\[ 
   e^{aJ_{2}}\begin{pmatrix}
0&1\\-1&0
\end{pmatrix}=
\begin{pmatrix}
0&1\\-1&0
\end{pmatrix}h,
\]
from construction of $S$. Then we look at the following :
\[ 
\xymatrix{%
   \Spin(2,1) \ar@{~>}[r]&R'\times\Spin(2,1)\ar[rr]^{\displaystyle\varphi}\ar[rd]_{\displaystyle\pi}&&\SO\big( \mU \big)\ar[ld]&\SO(2,1)\ar@{~>}[l]	  \\
  &&\mU
}
\]
The action of $\Spin(2,1)$ on $P$ is
\[ 
  \big( r',s' \big)\cdot s=\big( r',s's \big).
\]
In order to define $\varphi$, we consider the isomorphism
\[ 
  \phi_{x}\colon \SO(2,1)\to \SO(2,2)_{x}
\]
between $\SO(2,1)$ and the stabiliser of $x$ in $\SO(2,2)$. This extends to an automorphism
\[ 
  \phi_{x}\colon \SO(2,2)\to \SO(2,2),
\]

\begin{probleme}
	I'm not sure of that extension, but we do not use it here.
\end{probleme}


and we define the action of $\SO(2,1)$ on $\SO\big( AdS_3 \big)$ by
\begin{equation}
\{ b_{i} \}_{x}\cdot g=\{ \phi_{x}(g)b_{i} \}_{x}.
\end{equation}
Then we define
\begin{equation}
  \varphi\big( r',s \big)=\{ \phi_{\pi r'}\big( \chi(s) \big)b_{i} \}_{\pi r'}
\end{equation}
 if $\{ b_{i} \}$ is a reference basis at $\pi[r]$. So this construction implies the choice of a section of $\SO\big( AdS_3 \big)$. Now, using the fact that both $\phi_{x}$ and $\chi$ are morphisms, we find 
\begin{equation}
\begin{split}
\varphi\big( (r',s)\cdot s' \big)&=\left\{ \phi_{\pi r'}\big( \chi(ss') \big) \right\}_{\pi r'}\\
			&=\left\{ \phi_{\pi r'}\big( \chi(s) \big)b_{i} \right\}\cdot\chi(s)\\
			&=\varphi(r',s)\cdot\chi(s').
\end{split}
\end{equation}
This proves that the construction gives a spin structure.

\subsection{Connection on the spinor bundle}
%-----------------------------------------------

A left invariant vector on $\mU$ is of the form
\[ 
  \tilde X_{xo}=\Dsdd{ x e^{tX}o }{t}{0}
		=\Dsdd{ \pi\big( x e^{tX},s \big) }{t}{0}
\]
for any $s\in\Spin(2,1)$. On $AdS_3$ (in fact on $\mU$) we consider the left invariant vector field
\begin{equation}
X^{\sharp}_{[x]}=\Dsdd{ [x e^{tX}] }{t}{0}
\end{equation}
which leads us to consider the following field on $P$ :
\begin{equation}
\xi_{X}\big( r',s \big)=\Dsdd{ r' e^{tX},s }{t}{0}\in T_{( r',s )}P.
\end{equation}
This defines a field which projects to the left invariant field on $\mU$ :
\begin{equation}   \label{eq_xiXprojXsharp}
d\pi\xi_{X}(r',s)=X^{\sharp}_{r'}.
\end{equation}  


\begin{lemma}
On the general vector 
\begin{equation}   \label{eq_gebevectSig}
  \Sigma=\Dsdd{ r'(t),s(t) }{t}{0},
\end{equation}
the formula
\begin{equation}
\alpha_{(r',s_{0})}\Sigma=-\Dsdd{ s_{0}^{-1}s(t) }{t}{0}\in\spin(2,1)
\end{equation}
where $s_{0}=s(0)$ defines a connection form.

\end{lemma}

\begin{proof}
First let $A\in\spin(2,1)$ and
\[ 
  A^*_{\xi}=\Dsdd{ \xi\cdot e^{-tA} }{t}{0}.
\]
We have
\[ 
   \alpha\big( A^*_{(r'),s_{0}} \big)=\alpha \Dsdd{ (r',s_{0})\cdot e^{-tA}  }{t}{0}
		=\alpha\Dsdd{ (r',s_{0} e^{-tA}) }{t}{0}
		=-\Dsdd{ s_{0}^{-1}s_{0} e^{-tA}} {t}{0}
		=A.
\]
Now we take back the vector $\Sigma$ of equation \eqref{eq_gebevectSig}, an element $a\in\Spin(2,1)$ and we compute
\[ 
\begin{split}
  (dR_{a}\alpha)\Sigma&=\alpha\Dsdd{ \big( r'(t),s(t) \big)\cdot a }{t}{0}\\
		&=\alpha\Dsdd{ \big( r'(t),s(t)a \big) }{t}{0}\\
		&=-\Dsdd{ a^{-1}s(0)^{-1}s(t)a }{t}{0}\\
		&=-\Ad(a^{-1})\Dsdd{ s(0)^{-1}s(t) }{t}{0}\\
		&=\Ad(a^{-1})\alpha(\Sigma).
\end{split}  
\]

\end{proof}
Thus that is a connection. This is however not the spin connection. Let $\beta$ be the Levi-Civita connection on the frame bundle $\SO(AdS_3)$. If 
\[ 
  \Sigma=\Dsdd{ r'(t),s(t) }{t}{0},
\]
we have
\begin{equation} \label{eQbetadphiSigma}
  \beta d\phi\Sigma=\left. \phi_{r'}\big( \chi(s_{0}) \big)^{-1}\Dsdd{ \phi_{r'(t)}\big( \chi(s_{t}) \big) }{t}{0}\right|_{\sH}.
\end{equation}
If we note $\phi_{r'(t)}\big( \chi(s_{t}) \big)=\phi\big( r'(t),\chi(s_{f}) \big)$, the derivative in \eqref{eQbetadphiSigma} with respect to $t$ reads
\begin{equation}
\Dsdd{ \phi\big( r'(t),\chi(s_{0}) \big) }{t}{0}+\Dsdd{ \phi\big( r',\chi(s_{t}) \big) }{t}{0}.
\end{equation}
The second term of $\beta d\varphi\Sigma$ is
\begin{align*}
\left. \Dsdd{ \phi_{r'}(\chi(s_{0}))^{-1}\phi_{r'}(\chi(s_{t})) }{t}{0}\right|_{\sH}
		&=\left. \Dsdd{ \phi_{r'}\big( \chi(s_{0}^{-1}s_{t}) \big) }{t}{0}\right|_{\sH}\\
		&=\left. d\phi_{r'}d\chi(s_{0}^{-1}s'(0))\right|_{\sH}.
\end{align*}
From all that we want to define
\begin{equation}
   \alpha^{S}_{(r',s_{0})}\Sigma=\left.d\phi d\chi\big(s_{0}^{-1}s'(0)\big)\right|_{\sH}+\left.\phi_{r'}\big( \chi(s_{0}) \big)^{-1}\Dsdd{ \phi_{r'(t)}\chi(s_{0}) }{t}{0}\right|_{\sH},
\end{equation}
and we would not have $\alpha^{S}(\xi_{X})=0$.


\subsection{Horizontal lift}
%------------------------------

Since the spin component of the path of $\xi_{X}$ is constant, we have $\alpha(\xi_{X})=0$, so equation  \eqref{eq_xiXprojXsharp} says that
\begin{equation}
\overline{ X^{\sharp} }=\xi_{X}.
\end{equation}
Let us recall that an equivariant function (which defined a section of an associated bundle) is
\begin{equation}
\begin{aligned}
 \hat{\psi}\colon P&\to V \\ 
\hat{\psi}(\xi\cdot g)&= \rho(g^{-1})\hat{\psi}(\xi). 
\end{aligned}
\end{equation}
General definition of an equivariant derivative (theorem \ref{tho_dercovassoequiv}) leads to
\[ 
  \widehat{    \nabla_{X^{\sharp}}\psi    }=\overline{ X^{\sharp} }\cdot\hat{\psi}=\xi_{X} \cdot \hat{\psi}.
\]
In our setting, the equivariance of $\hat{\psi}$ reads, for all $a\in\Spin(2,1)$, 
\[ 
  \hat{\psi}\big( ([r],s)\cdot a \big)=\hat{\psi}\big( [r],sa \big)\stackrel{!}{=}\rho(a^{-1})\hat{\psi}\big( [r],s \big).
\]
We check the equivariance of $\widehat{\nabla_{X^{\sharp}}\psi}$ by the following computation :
\[ 
\begin{split}
\widehat{\nabla_{X^{\sharp}}\psi  }\big( ([r],s)\cdot a \big)&=\widehat{\nabla_{X^{\sharp}}\psi}( [r],sa )\\
		&=(\xi_{X}\cdot \hat{\psi})([r],sa)\\
		&=\Dsdd{ \hat{\psi}\big( [r e^{tX}],sa \big) }{t}{0}\\
		&=\Dsdd{ \rho(a^{-1})\hat{\psi}\big( [r e^{tX}],s \big) }{t}{0}\\
		&=\rho(a^{-1})(\xi_{X}\cdot \hat{\psi})\big( [r],s \big)\\
		&=\rho(a^{-1})\widehat{  \nabla_{X^{\sharp}}\psi  }\big( [r],s \big).
\end{split}  
\]

We define $\tilde{\psi}\colon AN/S\to \Lambda W$ by 
\[ 
  \tilde{\psi}([r])=\hat{\psi}( [r],e ),
\]
so that
\begin{equation}
\hat{\psi}([r],s)=\rho(s^{-1})\tilde{\psi}([r]).
\end{equation}
We can conclude
\[ 
\begin{split}
\widetilde{ \nabla_{X^{\sharp}}\psi  }([r])&=\widehat{\nabla_{X^{\sharp}}\psi}([r],e)\\
		&=\xi_{X}\hat{\psi}([r],e)\\
		&=\Dsdd{ \hat{\psi}\big( [r e^{tX}],e \big) }{t}{0}\\
		&=\Dsdd{ \tilde{\psi}\big( [r e^{tX}] \big) }{t}{0}\\
		&=\tilde X_{[r]}\tilde{\psi}([r]).		
\end{split}  
\]
So
\begin{equation}
\widetilde{\nabla_{X^{\sharp}}\psi}=\tilde X_{[r]}\tilde{\psi}.
\end{equation}

\subsection{Spin structure on \texorpdfstring{$AdS_3$}{AdS3} }
%+++++++++++++++++++++++++++++++++++++++++++++++++++++++++

\subsubsection{Spin structure on the whole \texorpdfstring{$AdS_3$}{AdS3} }

\begin{probleme}
	The following seems to contradict what I find in Michelson-Donaldson
\end{probleme}
The central fact is that
\[ 
  \Spin(2,1)\simeq\Delta\simeq\SL(2,\eR)
\]
where $\Delta=\{ (g,g)\tq g\in\SL(2,\eR) \}\subset G_0$. We take as notations: $G_{0}=\SL(2,\eR)$ and $\overline{G}=G_0\times G_{0}$. 

\begin{lemma}
We have the following homogeneous space isomorphism :
\[ 
  \overline{G}/\Delta\simeq\SL(2,\eR).
\]
\end{lemma}

\begin{proof} 

We have an action $\overline{G}\times AdS_3\to AdS_3$,
\begin{equation} \label{EqActghgxh}
  (g,h)x=gxh^{-1}
\end{equation}
where $x\in\SL(2,\eR)$ is seen as in $AdS_3$ by the usual isomorphism. Moreover we consider the isomorphism
\begin{align}
\overline{G}/\Delta&\simeq\SL(2,\eR)\\
[x_1,x_2]&\mapsto x_1x_2^{-1}
\end{align}
which is well defined because $[x_1g,x_2g]\mapsto x_1gg^{-1}x_2^{-1}=x_1x_2^{-1}$. In particular, $[g,g]\mapsto e\in\SL(2,\eR)$. So $\overline{G}$ acts on $\SL(2,\eR)$ and the elements which fix $e$ are the one of $\Delta$. It proves the lemma.
\end{proof}

We are going to take the following structure :
\begin{equation}  \label{EqScSpinAdS}
  \xymatrix{%
   \overline{G} \ar[rr]^{\displaystyle\varphi}\ar[dr]_{\displaystyle\pi}	&	&	\overline{G}/\eZ_{2}\ar[ld]\\
   						& M	
}
\end{equation}
where $M$ is $G_0$ seen as $M=\overline{G}/\Delta\simeq \SL(2,\eR)\simeq AdS_3$, and the projection $\pi\colon \overline{G}\to M$ is given by $\pi(g,h)=gh^{-1}$. The action of $\Delta\simeq \Spin(2,1)$ on $\overline{G}$ is given by formula $(xg,g)\cdot (a,a)=(xga,ga)$. First, let us prove the following.
\begin{proposition}
The frame bundle over $AdS_3$ can be seen as 
\[ 
  \SO(AdS_3)\simeq \overline{G}/\eZ_{2}
\]
where $\overline{G}=\SL(2,\eR)\times\SL(2,\eR)$.
\end{proposition}

\begin{proof}
In the fiber bundle $\pi\colon \overline{G}\to M$, the fibre over $x\in\SL(2,\eR)$ is the set of $(g,h)$ such that $gh^{-1}=x$, or
\[ 
  \overline{G}_{x}=\{ (xg,g) \}\subset \overline{G}.
\]
We will give a surjective map $\overline{G}_{x}\to\SO(M)_{x}$, the fibre of the frame bundle over $x\in AdS_3$. For this, we see a basis of $AdS_3$ as an isometric map $b\colon \sG_0\to T_{x}M$ where $\sG_{0}=\mathfrak{sl}(2,\eR)$, and we define
\begin{equation}
\begin{aligned}
 \psi_{x}\colon \overline{G}_{x}&\to \SO(M)_{x} \\ 
\psi_{x}(xg,g)(X)&=(dL_{x})_{e}\big( \Ad(g^{-1})X \big)
\end{aligned}
\end{equation}
for all $X\in\sG_0$. Let us study the kernel of this map, i.e. elements such that $\psi(xg_1,g_1)=\psi(xg_2,g_2)$. It needs, for all $X\in\mathfrak{sl}(2,\eR)$,
\[ 
  \Ad(g_1^{-1})X=\Ad(g_2^{-1})X,
\]
but we know that the requirement $\Ad(g)X=X$ is the fact the $g$ is in the center of the group. In our case, it results that $g_2^{-1}g_1=\pm\id$, so
\[ 
  \psi(xg_1,g_1)=\psi(\pm xg_1,\pm g_1)
\]
where the same $\pm$ has to be taken in both appearances of the right hand side. Now we put all the $\psi_{x}$ together to get $\psi\colon \overline{G}\to \SO(M)$. Once again we look in which cases $\psi(g_1,h_1)=\psi(g_2,h_2)$. We put this condition under the form
\[ 
  \psi(g_1h_1^{-1}h_1,h_1)=\psi(g_2h_2^{-1}h_2,h_2)
\]
which immediately gives $h_1=\pm h_2$. But on the other hand the base point of $\psi(g_{i}h_{i}^{-1},h_{i})$ is $g_{i}h_{i}^{-1}$, so that the condition also ask $g_1h_1^{-1}=g_2h_2^{-1}$ which in turn gives $g_1=\pm g_2$ with the same $\pm$ as in $h_1=\pm h_2$. We conclude that $\eZ_{2}$ is the problem for the inverse of $\psi$. This proves the proposition.
\end{proof}
We will usually use the same notation, $\psi$, to denote the map from $\overline{G}$ and the one from $\overline{G}/\eZ_{2}$. The following lemma will prove useful to study the actions of the structure groups in the picture \eqref{EqScSpinAdS}.

\begin{lemma}
The map
\begin{equation}
\begin{aligned}
 \SL(2,\eR)&\to \SO_{0}(1,2)\\
   g&\mapsto\Ad(g).
\end{aligned}
\end{equation}
is a double covering.
\end{lemma}
\begin{proof}
No proof.
\end{proof}
The action of $a\in\SO_{0}(1,2)$ on $(xg,g)\in\overline{G}/\eZ_{2}$ is defined by
\begin{equation}
\psi\big( (xg,g)\cdot a \big)=(dL_{x})_{e}\Ad(a^{-1}g^{-1}).
\end{equation}
On the other hand, let us see how does $(a,a)\in\Delta\simeq \Spin(2,1)$ acts on $\overline{G}$ and how does it reflects on the $\psi$ level. Since $(xg,g)\cdot (a,a)=(xga,ga)$, we have
\[ 
  \psi\big( [xg,g]\cdot a \big)=\psi\big( (xg,g)\cdot(a,a) \big),
\]
and then
\[ 
  \varphi\big( (xg,g)\cdot a \big)=(xg,g)\cdot(a,a).
\]
This proves that our structure is a spin structure.

\subsubsection{Reduction to one open orbit}
%----------------------------------------

We will use this isomorphism between $AdS_3$ and $\SL(2,\eR)$ :
\[ 
  \begin{pmatrix}
u\\t\\x\\y
\end{pmatrix}\mapsto
\begin{pmatrix}
u+x&y-t\\y+t&u-x
\end{pmatrix}.
\]
Then the famous point $[u]=\begin{pmatrix}
0&1\\-1&0
\end{pmatrix}\in AdS_3$ corresponds to the element $J:=\begin{pmatrix}
0&1\\-1&0
\end{pmatrix}\in \SL(2,\eR)$. This is our base point of the open orbit. We could also take
\[ 
  k_{0}=\frac{ \sqrt 2 }{ 2 }\begin{pmatrix}
1&1\\-1&1
\end{pmatrix}\in K_{0}
\]
where $K_{0}$ is the ``$K$'' of $\SL(2,\eR)$. 
\begin{probleme}
	I think that $J$ is also a complex structure. To be checked.
\end{probleme}
We have $J=k_{0}^{2}$ and following the action \eqref{EqActghgxh}, we have $J=(k_{0},k_{0}^{-1})e$. The subgroup $\overline{R}\subset\overline{G}$ acts on $AdS_3$, and we want to know the stabilizer of $J$. The condition is $(r,r')\cdot J=J$, or
\[ 
  r=\AD(J)r',
\]
but $\AD(J)=\theta$ (the Cartan involution). So an element $(r,r')\in\overline{R}$ stabilises $J$ if it is of the form $(r,\theta r)$, thus
\[ 
  \mfs=\text{Lie algebra of the stabiliser of $J$}=\{ (X,\theta X)\tq X\in\sR_{0} \}\cap\sR,
\]
where the intersection with $\sR$ is important because $\theta$ can send out of $\sR_{0}$. Note that when $X$ has a $\sN$ component, then $\theta X$ has a $\overline{ \sN }$ component, so $(X,\theta X)\in(\sA\oplus\sN,-\sA\oplus\overline{ \sN })$ where the minus sign comes from the fact that $\theta(\sA)=-\sA$. Then $X$ cannot have a $\sN$ component and finally,
\[ 
  \mfs=\eR (H,-H)\in\sQ.
\]
The group $R'$ is
\begin{equation}
R'= e^{\sR'}=\{ (an,an')\tq n,n'\in N_{0} \}
\end{equation}
because $\sR'$ is $\sR$ minus the stabiliser, i.e. $\sR'=\eR(H,H)\oplus\sN$. We have the identification $r'\mapsto r'\cdot J$ between $R'$ and the open orbit $\mU$. As usual, the action is $(g,h)\cdot x=gxh^{-1}$ if $r'=(g,h)$. Notice in particular that $R'\neq R_{0}'\times R_{0}'$.

Up to now we studied the fiber $\overline{G}\to M$; we are now able to restrict it to $\overline{G}|_{\mU}\to\mU$ and to establish an isomorphism with the trivial bundle $R'\times G_0\to R'$. The fiber over $x\in\mU$ is
\[ 
  \overline{G}_{x}=\{ (xg,g) \}.
\]
We define the isomorphism as follows:
\begin{equation}
\begin{aligned}
 \tau\colon R'\times G_0&\to \overline{G}|_{\mU} \\ 
(r',g)&\mapsto (r'\cdot Jg,g) 
\end{aligned}
\end{equation}
and we have the following picture: 
\[ 
  \xymatrix{%
   R'\times G_0 \ar[r]^-{\displaystyle\tau}\ar[d]	&	\overline{G}|_{\mU}\ar[d]^{\displaystyle\pi}\\
   R' \ar@{.>}[r]^{\displaystyle\tau}		&	\mU
}
\]
in which are defines by
\[ 
  \xymatrix{%
   (r',g) \ar[r]^-{\displaystyle\tau}\ar[d]	&	(r'\cdot Jg,g)\ar[d]^{\displaystyle\pi}\\
   r' \ar@{.>}[r]^{\displaystyle\tau}		&	r'\cdot J
}
\]
where the dotted line denotes the induced map from $\tau$, which is denoted by the same symbol. The map $\tau\colon R'\to \mU$ is just the restriction of the original $\tau$ to $g=e$. Notice that this $\tau$ provides a diffeomorphism of the basis spaces $R'$ and $\mU$.

\subsubsection{Spin connection}
%---------------------------

The spin connection on $\overline{G}|_{\mU}$ is given by
\begin{equation}    \label{EqDefConnAdS3}
  \alpha^{S}_{(g,h)}\Sigma=\left[ dL_{(g,h)^{-1}}\Sigma \right]_{\sH},
\end{equation}
or
\begin{equation}
\alpha^{S}_{(g,h)}=\pr_{\sH}\circ\big( dL_{(g,h)^{-1}} \big)_{(g,h)}.
\end{equation}
Notice that when we write $\sH$, we think about $\Delta$ : the group by which quotient  $\overline{G}$ in order to get $\SL(2,\eR)\simeq AdS_3$.
Our task now is to transfer this connection to $R'\times G_0$ by defining $\alpha'=\tau^*\alpha^{S}$. If $\Sigma\in T_{(r',g)}(R'\times G_0)$, we define
\begin{equation}
\alpha'_{(r',g)}\Sigma=\alpha^{S}(d\tau\Sigma).
\end{equation}
Let us take $X\in\sG_0$ and $0\in\sR'$ and let us compute $d\tau(0,X)$. More precisely, we consider
\begin{align*}
d\tau(0\oplus \tilde X_{g})_{(r',g)}&=d\tau\Dsdd{ r',g e^{tX} }{t}{0}\\
		&=\Dsdd{ r'\cdot Jg e^{tX},g e^{tX} }{t}{0}\\
		&=\big( \tilde X_{(r'\cdot Jg)},\tilde X_{g} \big).
\end{align*}
The next step is to compute $d\tau\Sigma$ in the case where $\Sigma=(\utilde Y\oplus -1)_{(r',g)}$ with $Y\in\sR'\subset\sR_{0}\oplus\sR_{0}$. We have
\begin{align}
d\tau\Sigma&=\Dsdd{ \tau\big(  e^{tY}r',g \big) }{t}{0}\\
	&=\Dsdd{ ( e^{tY}r'\cdot J)g,g }{t}{0}
\end{align}
where, if $r'=(r_{1},r_{2})$, we consider $Y=\big((\utilde Y_{1})_{r_{1}},(\utilde Y_{2})_{r_{2}}\big)  $. This appears to be difficult to be computed. This reflects the fact that the connection should be complicated in the trivial bundle $R'\times G_0$.

But there are no fate. We remember that $\tau$ furnish a diffeomorphism between the basis spaces, so one can consider the bundle 
\[ 
  \xymatrix{%
   \overline{G}|_{\mU} \ar[d]^{\displaystyle\tau^{-1}\circ\pi}\\		
   R'
}
\]
Vectors of $\sH$ are of the form $(X,X)$ with $X\in\mathfrak{sl}(2,\eR)$, thus $A\in T_{(xg,g)}\overline{G}|_{\mU}$ fulfils $\alpha^{S}(A)=0$ if and only if
\[ 
  dL_{(xg,g)^{-1}}(A)=(X,-X)
\]
for a certain $X\in \mathfrak{sl}(2,\eR)$. All this makes that the horizontal space over $(xg,g)$ is given by
\begin{equation}
\horsp(xg,g)=\big\{ (\tilde X_{xg},-\tilde X_{g})\tq X\in \sG_0=\mathfrak{sl}(2,\eR) \big\}.
\end{equation}
The strategy now is to project that on $R'$ and express Dirac operator in terms of the result. Let us make this simple computation :
\begin{align*}
d\pi(\tilde X_{xg},\tilde X_{g})&=\Dsdd{ \pi\big( xg e^{tX},g e^{-tX} \big) }{t}{0}\\
		&=\Dsdd{ xg e^{tX} e^{tX}g^{-1} }{t}{0}\\
		&=\Dsdd{ x e^{2t\Ad(g)X} }{t}{0}\\
		&=2(dL_{x})_{e}\Ad(g)X.
\end{align*}
This result has to be brought from $\mU$ to $R'$ by $\tau^{-1}$. Now we take a $\tilde Y\in\cvec(R')$ and we want to know which is the corresponding $X$, i.e. the $X\in\mathfrak{sl}(2,\eR)$ such that
\[ 
  d\tau^{-1}d\pi(\tilde X_{xg},-\tilde X_{g})=\tilde Y.
\]
From the previous computation, $\tilde Y=2d\tau^{-1}dL_{x}\Ad(g)X$, so
\begin{equation}  \label{EqXfracAdY}
  X=\frac{ 1 }{2}\Ad(g^{-1})dL_{x^{-1}}d\tau\tilde Y. 
\end{equation}
We now precise our idea: 
\begin{equation}   \label{EqtildeYrunrdeux}
  \tilde Y_{(r_{1},r_2)}=\big(    (\tilde Y_{1})_{r_1},(\tilde Y_{2})_{r_2}   \big)=\Dsdd{ r_1 e^{tY_{1}},r_2 e^{tY_{2}} }{t}{0}
\end{equation}
for $Y_{i}\in\sR'_{0}$ and $r_1$, $r_2\in R_{0}$. In this case, the ``$x$'' in equation \eqref{EqXfracAdY} is $(r'\cdot J)^{-1}$. Let us begin by taking $s'\in R'$ and compute $L_{(r'\cdot J)^{-1}}\tau(s')$. Remember that $r'\cdot J=r_1Jr_2^{-1}$ from the general action \eqref{EqActghgxh}, so if $r'=(r_1,r_2)$,
\begin{align*}
  dL_{(r'\cdot J)^{-1}}\tau(s')&=(r'\cdot J)^{-1}s_1 Js_2^{-1}\\
		&=(r_1Jr_2^{-1})^{-1}s_1Js_2^{-1}\\
		&=-r_2Jr_2^{-1}s_1Js_2^{-1}.
\end{align*}
Now, we apply that result on computation of \eqref{EqXfracAdY} with \eqref{EqtildeYrunrdeux} :
\begin{align*}
dL_{(r'\cdot J)^{-1}}d\tau\tilde Y&=\Dsdd{ -r_2Jr_1^{-1}r_1 e^{tY_{1}}J e^{-tY_{2}}r_2^{-1} }{t}{0}\\
		&=\Dsdd{ \AD(r_2)\big( -J e^{tY_{1}}J e^{-tY_{2}} \big) }{t}{0}\\
		&=\Dsdd{ \AD(r_2) e^{-tY_{2}} }{t}{0}+\Ad(r_2)\Ad(J)Y_{1}\\
		&=-\Ad(r_2)Y_{2}+\Ad(r_2)\theta(Y_{1}),
\end{align*}
and finally,
\begin{equation}  \label{EqValeurXAdtheta}
\begin{aligned}
X&=\frac{ 1 }{2}\Ad(g^{-1})dL_{(r'\cdot J)^{-1}}d\tau\tilde Y\\
		&=\frac{ 1 }{2}\Ad(g^{-1})\big[ \Ad(r_2)\theta(Y_{1})-\Ad(r_2)Y_{2} \big].
\end{aligned}
\end{equation}
For this $X$, the horizontal lift of $\tilde Y\in\cvec(R')$ is $(X,-X)\in T\overline{G}|_{\mU}$.

\subsection{Left invariance of Dirac}
%------------------------------------

Sections of the spin bundle over the open orbit $\mU$ are given by equivariant functions $\hat{\psi}\colon \overline{G}|_{\mU}\to \eR^{2}$. The action of $\Delta\simeq\Spin(2,1)$ on $\overline{G}$ is 
\[ 
  (g,h)\cdot(a,a)=(ga,ha).
\]
We define $\tilde{\psi}$ by
\begin{equation}
\tilde{\psi}(g)=\hat{\psi}(g,e)
\end{equation}
for $g\in\mU$. We get back the original $\hat{\psi}$ by formula
\begin{equation}
\hat{\psi}(g,h)=\rho(h,h)^{-1}\tilde{\psi}(gh^{-1}).
\end{equation}
Our intention is now to compute $\widehat{\nabla_{Z}\psi}(\xi)=\overline{ Z }_{\xi}(\hat{\psi})$ with $\xi=(xg,g)\in\overline{G}|_{\mU}$ (hence $x\in\mU$) and $Z\in\cvec(R')$. For instance we choose a left invariant $Z=\tilde Y=(\tilde Y_{1},\tilde Y_{2})$ for $Y_{1}$, $Y_2\in\sR_{0}'$. Recall that $\tilde Y$ is given by equation \eqref{EqtildeYrunrdeux}. From definition of the covariant derivative associated with the connection,
\[ 
  \widehat{\nabla_{\tilde Y}\psi}(\xi)=\overline{ \tilde Y }_{\xi}(\hat{\psi})=\overline{ \tilde Y }_{(xg,g)}(\hat{\psi})
\]
where $\overline{ \tilde Y }_{(xg,g)}$ is an horizontal vector at $(xg,g)$ whose projection is $\tilde Y$. From our previous work,
\[ 
  \overline{ \tilde Y }_{xg,g}=(\tilde X_{xg},-\tilde X_{g})
\]
with $X=\frac{ 1 }{2}\Ad(g^{-1})\big( \Ad(r_2)\theta Y_{1}-\Ad(r_2)Y_{2} \big)$. Let us understand the link between $(r_{1},r_2)$ and $g,x$. The vector $(\tilde X_{xg},\tilde X_{g})$ actually projects to a vector at $\tau^{-1}\circ\pi(xg,g)=\tau^{-1}(x)$. The fact that $x\in\mU$ guarantees existence and uniqueness of $(r_1,r_2)\in R'$ such that $r_1 Jr_2^{-1}=x$. We have
\[ 
\begin{split}
\protect\widetilde{\nabla_{\protect\tilde Y}\psi}(x)&=\widehat{\nabla_{\tilde{Y}}\psi}(x,e)\\
		&=(\tilde X_{x},-\tilde X_{e})\hat{\psi}\\
		&=\dsdd{ \hat{\psi}\Big( x e^{tX}, e^{-tX} \Big) }{t}{0}\\
		&=\dsdd{ \rho( e^{tX}, e^{tX})\tilde{\psi}(x e^{2tX}) }{t}{0}.
\end{split}  
\]
The first term of the derivation (the one with $t=0$ in the $\rho$) gives $2\tilde X_{x}\tilde{\psi}$. This is left invariant.
The second is 
\[
     \dsdd{ \rho( e^{tX}, e^{tX})\tilde{\psi}(x) }{t}{0}.
\]
We want to test the condition \eqref{EqDefLxinvarop} on this term. Let us pose
\[ 
  (E\tilde{\psi})(x)=(\tilde X_{x},\tilde X_{e})\hat{\psi}=\dsdd{ \rho( e^{tX}, e^{tX})\tilde{\psi}(x) }{t}{0}
\]
with $X$ given by equation \eqref{EqValeurXAdtheta}. On the one hand,
\begin{subequations}
\begin{equation}
L_{y}(E\tilde{\psi})(x)=(E\tilde{\psi})(yx)
		=\dsdd{ \rho( e^{tX_{a}}, e^{tX_{a}})\tilde{\psi}(yx) }{t}{0}
\end{equation}
with 
\begin{equation}
X_{a}=\frac{ 1 }{2}\Big( \Ad(r_2)\theta Y_{1}-\Ad(r_2) \Big)Y_{2}
\end{equation}
\end{subequations}
where $(r_1,r_2)$ is given by $yx$. On the other hand,
\begin{subequations}
\begin{equation}
E(L_{y}\tilde{\psi})(x)=\dsdd{ \rho( e^{tX_{b}}, e^{tX_{b}})\tilde{\psi}(yx) }{t}{0}
\end{equation}
with 
\begin{equation}
X_{b}=\frac{ 1 }{2}\big( \Ad(s_2)\theta Y_{1}-\Ad(s_2)Y_{2} \big)
\end{equation}
\end{subequations}
where $(s_1,s_2)$ is given by $x$.

The problem is that the choice of $y$ is arbitrary, so that $X_{a}$ and $X_{b}$ could be too different. Ok. That's the proof that Dirac is not invariant. Here is the proof that Dirac is invariant.

Following equation \eqref{EqDefConnAdS3}, the spin connection form is
\[ 
  \alpha^{S}_{(g,h)}\Sigma=\left( dL_{(g,h)^{-1}}\Sigma \right)_{\sH}.
\]
If $L_{(x,y)}$ is the left translation by $(x,y)$ we have
\[ 
  \big( L^{*}_{(x,y)}\alpha \big)_{(g,h)}\Sigma=\alpha_{(xg,yh)}\big( dL_{(x,y)}\Sigma \big)
		=\left( dL_{(g,h)^{-1}}\Sigma \right)_{\sH}.
\]
Thus we have $L^*_{(x,y)}\alpha^{S}=\alpha^{S}$. Now we consider the formula $\widehat{\nabla_{\tilde Y}\psi}(\xi)=(\tilde X_{xg},-\tilde X_{g})\hat{\psi}$, and we will check that 
\begin{equation}
 \big( L_{\eta}\widehat{\nabla_{Z}\psi} \big)(\xi)=\widehat{  \nabla_{Z}(L_{\eta}\psi)     }(\xi).
\end{equation}
with $\xi=(xg,g)$ and $\eta=(a,b)$. On the one hand,
\begin{align*}
  \big( L_{(a,b)}\widehat{\nabla_{Z}\psi} \big)(xg,g)&=\widehat{\nabla_{Z}\psi}(axg,bg)\\
		&=\widehat{\nabla_{Z}\psi}(axgg^{-1}b^{-1}bg,bg)\\
		&=\big( \tilde X_{(axb^{-1})bg},-\tilde X_{bg} \big)\hat{\psi}.
\end{align*}
On the other hand,
\begin{align*}
  \widehat{  \nabla_{Z}(L_{(a,b)}\psi)   }(xg,g)&=(\tilde X_{xg},-\tilde X_{g})\widehat{L_{(a,b)}\psi}\\
		&=\dsdd{ \widehat{L_{(a,b)}\psi}\big( xg e^{tX},g e^{-tX} \big) }{t}{0}\\
		&=\dsdd{ \hat{\psi}\big( axg e^{tX},bg e^{-tX} \big) }{t}{0}\\
		&= (\tilde X_{axg},-\tilde X_{bg}) \hat{\psi}\\
		&=( \tilde X_{(axb^{-1})bg},-\tilde X_{bg} )\hat{\psi}.	
\end{align*}



\section{Dirac operator on \texorpdfstring{$AdS_{4}$}{AdS4}}
%-----------------------------------------------------------

% Ok, il ne reste pas grand chose dans ce fichier, mais c'est appellé à grandir avec le temps.

\section{Dirac operator on \texorpdfstring{$AdS_{l}$}{AdSl}}
%+++++++++++++++++++++++++++++++++++++++++++++++++++++++++++

\subsection{Frame bundle}\index{frame!bundle!on $AdS_{l}$}
%------------------------

Construction of the frame bundle and the spin structure is a straightforward adaptation of theorem 2.2 (chapter ???) in \cite{AnnikFranc}, while Dirac operator and connection issues are adapted from proposition 1.3 (chapter III)

A \defe{basis}{basis} of a $m$ dimensional vector space $V$ is a free and generating part; it only has the structure of a set. A frame of the vector space $V$ is a nondegenerate map $b\colon \eR^{m}\to V$. Let us give an example in three dimensions the difference. If $\{ v_{1},v_{2},v_{3} \}$ is a basis of $V$, of course $\{ v_{2},v_{1},v_{3} \}$ is the same basis. Order has no importance. But if $\{ e_{1},e_{2},e_{3} \}$ is the canonical basis of $\eR^{3}$, the \emph{frames} $b(e_{1})=v_1$, $b(e_{2})=v_2$, $b(e_{3})=v_{3}$ and $c(e_{1})=v_2$, $c(e_{2})=v_1$, $c(e_{3})=v_{3}$ are not the same.

Now we consider $AdS_l=G/H=\SO(2,l-1)/\SO(1,l-1)$, the Lie algebra $\sG$ has a reductive homogeneous space decomposition $\sG=\sQ\oplus\sH$ and we consider the canonical projection $\pi\colon G\to AdS_l$.

Let the map (see relation \eqref{EqInclAdHSOq})
\begin{equation}
\begin{aligned}
 \alpha\colon H&\to \SO(\sQ) \\ 
h&\mapsto \Ad(h)|_{\sQ}.
\end{aligned}
\end{equation}
We consider, on $G\times\SO(\sQ)$, the equivalence relation $(g,A)\sim(g',A')$ if and only if there exists $h\in H$ such that $g'=gh$ and $A'=\alpha(h^{-1})A$. We denote by $G\times_{\alpha}\SO(\sQ)$ the set of equivalence classes. Now we have a principal bundle
\begin{equation}   \label{EqPrincPreB}
\xymatrix{%
   \SO(\sQ) \ar@{~>}[r]		&	G\times_{\alpha}\SO(\sQ)\ar[d]^{p}\\
   				&	   G/H
}
\end{equation}
where $p[g,A]=[g]$ and the action is given by $[g,A]\cdot B=[g,A]$. The fact that the projection fulfils $p\big( [g,A]\cdot B \big)=p[g,A]$ is evident, and the fact that the action is well defined is a simple computation :
if $[g',A']=[g,A]$, we have a $h\in H$ such that
\[ 
  [g',A']\cdot B=[g',A'B]
		=[gh,\alpha(h^{-1})AB]
		=[g,AB]
		=[g,A]\cdot B.
\]

\begin{proposition}
Let $\tau(g)\colon AdS_l\to AdS_l$ be the action of $g\in G$ on $AdS_l$ : $\tau(g)[g']=[gg']$, and $B$ be the frame bundle. We also consider the map $\sigma\colon \eR^{1,l-1}\to \sQ$ the isometry which sends the canonical basis of $\eR^{1,l-1}$ to the usual basis $\{ q_0,q_{1},\ldots,q_{l-1} \}$ of $\sQ$. The map
\begin{equation}
\begin{aligned}
 \beta\colon G\times_{\alpha} \SO(\sQ)&\to B \\ 
[g,A]&\mapsto d\tau(g)_{\mfo}A\circ\sigma 
\end{aligned}
\end{equation}
provides a principal bundle isomorphism between the principal bundle  \eqref{EqPrincPreB} and the frame bundle over $AdS_l$. 

\end{proposition}

By abuse of notation, we will not always write the $\sigma$.
\begin{proof}
We have to prove first that the map $\beta\colon G\times\SO(\sQ)\to B$ respects the classes. For that, consider $(g,A)\sim(g',A')$ and remark that
\[ 
\begin{split}
\beta(gh,\alpha(h^{-1}))&=d\tau(gh)_{\mfo}\alpha(h^{-1})A
		=d\tau(g)d\tau(h)d\pi \Ad(h^{-1})d\pi^{-1}A\\
		&=d\tau(g)d\tau(h)d\pi\Ad(h^{-1})d\pi^{-1} A
		=d\tau(g)d\pi dR_{h}d\pi^{-1}A\\
		&=d\tau(g)_{\mfo} A
		=\beta(g,A).
\end{split}  
\]
where we used equation \eqref{EqdpiAdpi} and the fact that $\pi\circ L_{g}=\tau(g)\circ\pi$. The frame bundle is
\begin{equation}   \label{EqPrincB}
\xymatrix{%
   \SO(1,l-1) \ar@{~>}[r]	&	B \ar[d]^{p}\\
   				&	   G/H
}
\end{equation}
where the fibre $B_{[g]}$ in $B$ over $[g]$ is the set of isometric maps $\eR^{1,l-1}\to T_{[g]}(AdS_l)$. So an element of $B$ is of the form $\big( [g],\tilde f\circ\sigma \big)$ where $g\in G$ and $\tilde f\colon \sQ\to T_{[g]}(AdS_l)$ contains the main information while $\sigma$ is the previously explained isometry. The action of $h\in\SO(1,l-1)$ on $\big( [g],\tilde f\circ\sigma \big)$ is defined by means of any fixed isomorphism $\varphi_{0}\colon \SO(1,l-1)\to \SO(\sQ)$ by
\begin{equation}
  \big( [g],\tilde f\circ\sigma \big)\cdot h=\big( [g],\tilde f\circ\varphi_{0}(h)\circ\sigma \big).
\end{equation}
The map $\beta$ is a morphism of principal bundle because
\[ 
\beta[g,A]\cdot\varphi_{0}^{-1}(B)	=\big( [g],d\tau(g)A\circ\sigma \big)\cdot \varphi_{0}^{-1}(B)
					=\big( [g],d\tau(g)A\circ B\circ \sigma \big)
					=\beta\big( [g,A]\cdot B \big).
\]
It remains to be proved that $\beta$ is a bijection. Surjectivity is natural: since $d\tau(g)$ is an isometry, $d\tau(g)A$ runs over the whole $\SO\big(T_{[g]}(AdS_l)\big)$ when $A$ runs over $\SO(\sQ)$. Injectivity is as follows; let's suppose $\beta[g,A]=\beta[g',A']$. It is immediate that in this case, $\exists h\in H$ such that $g'=gh$. Using the fact that $d\pi\circ dR_{h^{-1}}\circ d\pi^{-1}=\id$ and $d\tau(h)d\pi=d\pi dL_{h}$, we have
\[ 
d\tau(g)_{\mfo}A=d\tau(g)d\tau(h)A'
		=d\tau(g)d\tau(h)d\pi dR_{h^{-1}}d\pi^{-1}A'
		=d\pi\Ad(h)d\pi^{-1}A'
		=\alpha(h)A'.
\]

\end{proof}
From now on, we identify $G\times_{\alpha}\SO(\sQ)$ with the frame bundle over $AdS_l$. 

\subsection{Spin structure}
%--------------------------

We consider the principal bundle
\begin{equation}   
\xymatrix{%
   \Spin(1,l-1) \ar@{~>}[r]		&	G\times_{\tilde\alpha}\Spin(1,l-1)\ar[d]^{p}\\
   				&	   G/H
}
\end{equation}
where $\times_{\tilde\alpha}$ is the following equivalence relation on $G\times\Spin(1,l-1)$. We say that $(g,s)\sim(g',s')$ if and only if there exists a $h\in H$ such that
\begin{enumerate}
\item $g'=gh$,
\item $\chi(s')=\Ad(h^{-1})\chi(s)$.
\end{enumerate}
Notice that the second condition implies that $\Ad(h)\in \SO_0(\sQ)$. It is easy to prove that the given structure is well defined and is a principal bundle.  Now we consider the spin structure as follows:
\begin{equation}   
\xymatrix{%
   \Spin(1,l-1) \ar@{~>}[r]&G\times_{\tilde\alpha}\Spin(1,l-1)\ar[dr]\ar[rr]^-{\displaystyle\varphi}&&G\times_{\alpha} \SO(\sQ)\ar[dl]&\SO(\sQ)\ar@{~>}[l]  \\
   				&	   &G/H
}
\end{equation}
where $\varphi[g,s]=[g,\chi(s)]$. It is well defined since when $[g,s]=[g',s']$, there exists a $h\in H$ with $\chi(s')=\Ad(h^{-1})\chi(s)$ such that  $\varphi[g',s']=\varphi[gh,s']=[gh,\chi(s')]=[gh,\Ad(h^{-1})\chi(s)]=[g,\chi(s)]=\varphi[g,s]$.

\subsection{Reduction of the structural group}
%---------------------------------------------

The case of $AdS_l$ can be seen in the setting of subsection \ref{subsecCanConCovDer}. Let us show now that the bundle
\begin{equation}   \label{EqPrincHzGM}
\xymatrix{%
   H_0 \ar@{~>}[r]		&	G\ar[d]^{\pi}\\
   				&	G/H
 }
\end{equation}
is a reduction to $H_0$ (the identity component of $\SO(\sQ)$) of
\begin{equation}
\xymatrix{%
   G \ar@{~>}[r]		&	r(G)\ar[d]^{\pi}\\
   				&	G/H.
 }
\end{equation}
 Indeed, $u\colon G\to r(G)$ given by $u(g)=r(g)$ provides the reduction homomorphism: $r(gh)X=d\pi dL_{gh}X$ while $\big( r(g)\cdot h \big)X$ is the same.
 
\begin{lemma}
The tangent space $T(G/H)$ is an associated bundle of $r(G)$ trough the identification
\begin{equation}
\begin{aligned}
 \beta'\colon r(G)\times_{\rho}\sQ&\to T(G/H) \\ 
  [r(g),X]&\mapsto r(g)X 
\end{aligned}
\end{equation}
where $\rho(h)X=\Ad(h)X$, so that the quotient is given by $[g,X]=[gh,\Ad(h^{-1})X]$.
\end{lemma}
\begin{proof}
The proof is entirely similar to the one of lemma \ref{LemBazHGGH}.
\end{proof}
