% This is part of the Exercices et corrigés de CdI-2.
% Copyright (C) 2008, 2009,2013
%   Laurent Claessens
% See the file fdl-1.3.txt for copying conditions.


\begin{exercice}\label{exo_II-1-12}

Soit l'équation différentielle
\begin{equation}
	P(t,y)+Q(t,y)y'=0.
\end{equation}
\begin{enumerate}
\item
Montrer que si les fonctions $P$ et $Q$ sont homogènes de degré $n$, il existe un facteur intégrant $M$ homogène de degré $m=-(1+n)$. Indication : utiliser la formule d'Euler
\begin{equation}
	t\frac{ \partial P }{ \partial t }+t\frac{ \partial P }{ \partial y }=nP.
\end{equation}

\item
Montrer que si $P=yp(ty)$ et $Q=yq(ty)$, il existe un facteur intégrant $M=M(ty)$.

\item
Vérifier que dans chacun des cas précédents, il existe une autre façon de résoudre l'équation différentielle.

\end{enumerate}

\corrref{_II-1-12}
\end{exercice}

Juste pour le plaisir, prouvons la formule d'Euler. Le fait que $P$ et $Q$ sont homogènes de degré $n$ signifie que
\begin{equation}
	\begin{aligned}[]
		P(\lambda t,\lambda y)	&=\lambda^nP(t,y)\\
		Q(\lambda t,\lambda y)	&=\lambda^nQ(t,y).
	\end{aligned}
\end{equation}
Nous considérons $f(\lambda,t,y)=P(\lambda t,\lambda y)$, et nous calculons $(df/d\lambda)(1,t,y)$.
\begin{equation}
	\frac{ df }{ d\lambda }(1,t,y)=\frac{ \partial P }{ \partial t }(t,y)\frac{ d(\lambda t) }{ d\lambda }(1)+\frac{ \partial P }{ \partial y }(t,y)\frac{ d(\lambda y) }{ d\lambda }=\frac{ \partial P }{ \partial t }(t,y)t+y\frac{ \partial  }{ \partial y }(t,y).
\end{equation}
D'autre part, par hypothèse, $f(\lambda,t,y)=\lambda^nP(t,y)$, donc
\begin{equation}
	\frac{ df }{ d\lambda }(1,t,y)=\big( n\lambda^{n-1}P(t,y) \big)_{\lambda=1}=nP(t,y).
\end{equation}
Cela conclut la démonstration de la formule d'Euler.



