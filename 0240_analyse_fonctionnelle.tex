% This is part of Mes notes de mathématique
% Copyright (c) 2011-2013
%   Laurent Claessens
% See the file fdl-1.3.txt for copying conditions.

\begin{theorem}[Théorème d'isomorphisme de Banach]  \label{ThofQShsw}
    Une application linéaire continue et bijective entre deux espaces de Banach est un homéomorphisme.
\end{theorem}
\index{théorème!isomorphisme de Banach}
% TODO : une preuve.

%+++++++++++++++++++++++++++++++++++++++++++++++++++++++++++++++++++++++++++++++++++++++++++++++++++++++++++++++++++++++++++ 
\section{Théorème d'Ascoli}
%+++++++++++++++++++++++++++++++++++++++++++++++++++++++++++++++++++++++++++++++++++++++++++++++++++++++++++++++++++++++++++

\begin{definition}
    Une partie \( A\) d'un espace topologique \( X\) est \defe{relativement compacte}{compact!relatif}\index{relativement!compact} dans \( X\) si sa fermeture est compacte.
\end{definition}

\begin{proposition}[\cite{JIFGuct}] \label{PropDGsPtpU}
    Soient \( E\) et \( F\) deux espaces vectoriels normés sur \( \eR\) ou \( \eC\) et une application \( f\in\aL(E,F)\). Les propriétés suivantes sont équivalentes.
    \begin{enumerate}
        \item
            L'image d'un borné de \( E\) par \( f\) est relativement compact dans \( F\).
        \item   \label{ItemJIkpUbLii}
            L'image par \( f\) de la boule unité fermée est relativement compacte dans \( F\).
        \item
            Si \( (x_n)\) est une suite bornée dans \( E\), alors nous pouvons en extraire une sous-suite \( (x_{\varphi(n)})\) telle que \( \big( fx_{\varphi(n)} \big)\) converge dans \( F\).
    \end{enumerate}
\end{proposition}

\begin{definition}
    Une application vérifiant les conditions équivalentes de la proposition \ref{PropDGsPtpU} est dite \defe{compacte}{compact!opérateur}.
\end{definition}

\begin{definition}  \label{DefUWmVBcZ}
    Soit \( (f_i)_{i\in I}\) une famille de fonctions \( f_i\colon X\to Y\) entre espaces métriques. Cette famille est \defe{équicontinue}{équicontinuité} si pour tout \( \epsilon>0\) et pour tout \( x\in X\), il existe un \( \delta(x,\epsilon)>0\) tel que 
    \begin{equation}
        \| x-y \|_X\leq \delta\,\Rightarrow\,\| f_i(x)-f_i(y) \|_Y\leq \epsilon
    \end{equation}
    pour tout \( i\in I\).
\end{definition}

\begin{theorem}[Théorème d'Ascoli\cite{LBLADXV}]        \label{ThoKRbtpah}
    Soit \( K\) un espace topologique compact et un espace métrique \( (E,d)\). Nous considérons la topologie uniforme sur \( C(K,E)\). Une partie \( A\) de \( C(K,E)\) est relativement compacte si et seulement si les deux conditions suivantes sont remplies :
    \begin{enumerate}
        \item
            \( A\) est équicontinu,
        \item
            \( \forall x\in K\), l'ensemble \( \{ f(x)\tq f\in A \}\) est relativement compact dans \( E\).
    \end{enumerate}
\end{theorem}
\index{théorème!Ascoli}
%TODO : une preuve est sur Wikipédia.

%+++++++++++++++++++++++++++++++++++++++++++++++++++++++++++++++++++++++++++++++++++++++++++++++++++++++++++++++++++++++++++ 
\section{Théorème de Banach-Steinhaus}
%+++++++++++++++++++++++++++++++++++++++++++++++++++++++++++++++++++++++++++++++++++++++++++++++++++++++++++++++++++++++++++

\begin{theorem}[Théorème de Banach-Steinhaus\cite{KXjFWKA}] \label{ThoPFBMHBN}
    Soit \( E\) un espace de Banach\footnote{Définition \ref{DefVKuyYpQ}.} et \( F\) un espace vectoriel normé. Nous considérons un espace \( H\subset \aL_c(E,F)\) (espace des fonctions linéaires continues). Alors soit \( \{ \| f \| \}_{f\in H}\) est borné, soit il existe \( x\in E\) tel que \( \sup_{f\in H}\| f(x) \|=\infty\).

    Ici \( \| f \|\) désigne la norme de \( f\) en tant qu'application linéaire.
\end{theorem}
\index{théorème!Banach-Steinhaus}
\index{application!linéaire!théorème de Banach-Steinhaus}

\begin{proof}
    Pour chaque \( k\in \eN\) nous considérons l'ensemble
    \begin{equation}
        \Omega_k=\{ x\in E\tq \sup_{f\in H}\| f(x) \|>k \}.
    \end{equation}
    
    \begin{subproof}
        \item[Les \( \Omega_k\) sont ouverts]
            
            Soit \( x_0\in \Omega_k\); nous avons alors une fonction \( f\in H\) telle que \(  \| f(x_0) \|>k \), et par continuité de \( f\) il existe \( \rho>0\) tel que \( \| f(x) \|>k\) pour tout \( x\in B(x_0,\rho)\). Par conséquent \( B(x_0,\rho)\subset \Omega_k\) et \( \Omega_k\) est ouvert par le théorème \ref{ThoPartieOUvpartouv}.

        \item[Si les \( \Omega_k\) sont tous dense dans \( E\)]

            Nous supposons que les ensembles \( \Omega_k\) sont tous dense dans \( E\). Le théorème de Baire \ref{ThoBBIljNM} nous indique que \( E\) est un espace de Baire (parce que de Banach) et donc que 
            \begin{equation}
                \overline{ \bigcap_{k\in \eN}\Omega_k }=E.
            \end{equation}
            En particulier l'intersection des \( \Omega_k\) n'est pas vide. Soit \( x_0\in \bigcup_{k\in \eN}\Omega_k\). Nous avons alors
            \begin{equation}
                \sup_{f\in H}\| f(x) \|=\infty.
            \end{equation}
            
        \item[Si les \( \Omega_k\) ne sont pas tous denses dans \( E\)]

            Nous supposons à présent qu'il existe \( k\geq 0\) tel que \( \Omega_k\) ne soit pas dense dans \( E\), et nous voulons prouver que \( \{ \| f \|\tq f\in H \}\) est un ensemble borné. Soit donc \( k\geq 0\) tel que \( \Omega_k\) ne soit pas dense dans \( E\); il existe un \( x_0\in E\) et \( \rho>0\) tels que 
            \begin{equation}
                B(x_0,\rho)\cap \Omega_k=\emptyset.
            \end{equation}
            Si \( x\in B(x_0,\rho)\) alors \( x\) n'est pas dans \( \Omega_k\) et donc
            \begin{equation}
                \sup_{f\in H}\| f(x) \|\leq k.
            \end{equation}
            Afin d'évaluer \( \| f \|\) nous devons savoir ce qu'il se passe avec les vecteurs sur une boule autour de \( 0\). Pour tout \( x\in B(0,\rho)\) et pour tout \( f\in H\), la linéarité de \( f\) donne
            \begin{equation}
                \| f(x) \|=\| f(x+x_0)-f(x_0) \|\leq \| f(x+x_0)+f(x_0) \|\leq 2k.
            \end{equation}
            Par continuité nous avons alors \( \| f(x) \|\leq 2k\) pour tout \( x\in \overline{ B(0,\rho) }\). Si maintenant \( x\in F\) vérifie \( \| x \|=1\) nous avons
            \begin{equation}
                \| f(x) \|=\frac{1}{ \rho }\| f(\rho x) \|\leq \frac{ 2k }{ \rho },
            \end{equation}
            et donc \( \| f \|\leq \frac{ 2k }{ \rho }\), ce qui montre que \( 2k/\rho\) est un majorant de l'ensemble \( \{ \| f \|\tq f\in H \}\).

    \end{subproof}

\end{proof}
Une application du théorème de Banach-Steinhaus est l'existence de fonctions continues et périodiques dont la série de Fourier ne converge pas. Ce sera l'objet de la proposition \ref{PropREkHdol}.

%+++++++++++++++++++++++++++++++++++++++++++++++++++++++++++++++++++++++++++++++++++++++++++++++++++++++++++++++++++++++++++
\section{Espaces \texorpdfstring{$L^p$}{Lp}}
%+++++++++++++++++++++++++++++++++++++++++++++++++++++++++++++++++++++++++++++++++++++++++++++++++++++++++++++++++++++++++++
\label{SecVKiVIQK}

Soit \( (\Omega,\tribF,\mu)\) un espace mesuré. Deux fonctions \( f\) et \( g\) sur cet espaces sont dites \defe{équivalentes}{équivalence!classe de fonctions} et nous notons \( f\sim g\) si elles sont \( \mu\)-presque partout égales. Nous notons \( [f]\) la classe de \( f\) pour cette relation.

Nous introduisons l'opération
\begin{equation}
    \| f \|_p=\left( \int_{\Omega}| f(x) |^pd\mu(x) \right)^{1/p}
\end{equation}
et nous notons \( \mL^p(\Omega,\mu)\)\nomenclature[Y]{\( \mL^p\)}{espace de Lebesgue, sans les classes} l'ensemble des fonctions mesurables sur \( \Omega\) telles que \( \| f \|_p<\infty\).

\begin{lemma}
    L'ensemble \( \mL^p\) est un espace vectoriel.
\end{lemma}
    
\begin{proof}
    Le fait que si \( f\in L^p\), alors \( \lambda f\in L^p\) est évident. Ce qui est moins immédiat, c'est le fait que \( f+g\in L^p\) lorsque \( f\) et \( g\) sont dans \( L^p\). Cela découle du fait que la fonction \( \varphi\colon x\mapsto x^p\) est convexe, de telle sorte que
    \begin{equation}
        \varphi\left( \frac{ a+b }{2} \right)\leq\frac{ \varphi(a)+\varphi(b) }{2},
    \end{equation}
    ou encore
    \begin{equation}    \label{EqZFSduFa}
        (a+b)^p\leq 2^{p-1}(a^p+b^p)
    \end{equation}
\end{proof}

    L'opération \( f\mapsto \| f \|_p\) n'est pas une norme sur \( \mL^p\) parce que pour \( f\) presque partout nulle, nous avons \( | f |_p=0\). Il y a donc des fonctions non nulles sur lesquelles \( \| . \|_p\) s'annule.

\begin{lemma}       \label{LemKZVHVAR}
    Si \( f\in \mL^p(\Omega)\) et \( f\sim g\), alors \( g\in \mL^p(\Omega)\) et \( \| f \|_p=\| g \|_p\).
\end{lemma}

\begin{proof}
    Soit \( h(x)=| g(x) |^p-| f(x) |^p\); c'est une fonction par hypothèse presque partout nulle et donc intégrable sur \( \Omega\); son intégrale y vaut zéro. Nous avons
    \begin{equation}
        \int_{\Omega}| f(x) |^pd\mu(x)=\int_{\Omega}\Big( | f(x) |^p+h(x)\big)d\mu(x)=\int_{\omega}| g(x) |^pd\mu(x).
    \end{equation}
    Cela prouve que la dernière intégrale existe et vaut la même chose que la première.
\end{proof}

Nous pouvons donc considérer la norme \( | . |_p\) comme une norme sur l'ensemble des classes plutôt que sur l'ensemble des fonctions. Nous notons \( L^p\)\nomenclature[Y]{\( L^p\)}{espace de Lebesgue avec les classes} l'ensemble des classes des fonctions de \(\mL^p\). Cet espace est muni de la norme
\begin{equation}
    \| [f] \|_p=\| f \|_p,
\end{equation}
formule qui ne dépend pas du représentant par le lemme \ref{LemKZVHVAR}.

Maintenant la formule
\begin{equation}
    \| [f] \|_p=\left( \int_{\Omega}| f(x) |^pd\mu(x) \right)^{1/p}
\end{equation}
défini une norme sur \( L^p(\Omega,\mu)\). En effet si \( \| [f] \|_p=0\), nous avons
\begin{equation}
    \int_{\Omega}| f(x) |^pd\mu(x)=0,
\end{equation}
ce qui par le lemme \ref{Lemfobnwt} implique que \( | f(x) |^p=0\) pour presque tout \( x\). Ou encore \( f\sim 0\), c'est à dire \( [f]=[0]\) au niveau des classes. À partir de maintenant \( \big( L^p(\Omega,\mu),\| . \|_p \big)\) est un espace métrique avec toute la topologie qui va avec.

Dans la suite nous n'allons pas toujours écrire \( [f]\) pour la classe de \( f\). Par abus de notations nous allons souvent parler de \( f\in L^p\) comme si c'était une fonction.

\begin{proposition}[\cite{bJOSNQ}]  \label{PropWoywYG}
    Soit \( 1\leq p\leq \infty\) et supposons que la suite \( [f_n]\) dans \( L^p(\Omega,\tribF,\mu)\) converge vers \( [f]\) au sens \( L^p\). Alors il existe une sous-suite \( (h_n)\) qui converge ponctuellement \( \mu\)-presque partout vers \( f\).
\end{proposition}
\index{espace!\( L^p\)}
\index{suite!de fonctions}
\index{limite!inversion}

\begin{proof}
    Si \( p=\infty\) nous sommes en train de parler de la convergence uniforme et il ne faut même pas prendre ni de sous-suite ni de «presque partout».

    Supposons que \( 1\leq p<\infty\). Nous considérons une sous-suite \( [h_n]\) de \( [f_n]\) telle que
    \begin{equation}
        \| [h_j]-[f] \|_p<2^{-j},
    \end{equation}
    puis nous posons \( u_k(x)=| h_k(x)-f(x) |^p\). Notons que ce \( u_k\) est une vraie fonction, pas une classe. Et en plus c'est une fonction positive. Nous avons
    \begin{equation}
        \int_{\Omega}u_kd\mu=\int_{\omega}| h_k(x)-f(x) |^pd\mu(x)=\| h_k-f \|_p^p\leq 2^{-kp}.
    \end{equation}
    Vu que \( u_k\) est une fonction positive la suite des sommes partielles de \( \sum_ku_k\) est croissante et vérifie donc le théorème de la convergence monotone \ref{ThoConvMonFtBoVh} :
    \begin{equation}
            \int_{\Omega}\left( \sum_{k=0}^{\infty}u_k(x) \right)d\mu(x)=\sum_{k=0}^{\infty}\int_{\Omega}u_k(x)d\mu(x)
            \leq\sum_{k=0}^{\infty}2^{-kp}<\infty.
    \end{equation}
    Le fait que l'intégrale de la fonction \( \sum_ku_k\) est finie implique que cette fonction est finie \( \mu\)-presque partout. Donc le terme général tend vers zéro presque partout, c'est à dire
    \begin{equation}
        | h_k(x)-f(x) |^p\to 0.
    \end{equation}
    Cela signifie que \( h_k\to f\) presque partout ponctuellement.
\end{proof}

Est-ce qu'on peut faire mieux que la convergence ponctuelle presque partout d'une sous-suite ? En tout cas on ne peut pas espérer grand chose comme convergence pour la suite elle-même, comme le montre l'exemple suivant.

\begin{example} \label{ExPOmxICc}
    Nous allons montrer une suite de fonctions qui converge vers zéro dans \( L^p[0,1]\) (avec \( p<\infty\)) mais qui ne converge ponctuellement pour \emph{aucun} point. Cet exemple provient de \href{http://www.bibmath.net/dico/index.php?action=affiche&quoi=./b/bosseglissante.html}{bibmath.net}. 

%TODO : revoir tous les \href et mettre ceux qui doivent être en bibliographie. Par exemple celui ci-dessus.

    Nous construisons la suite de fonctions par paquets. Le premier paquet est formé de la fonction constante \( 1\).

    Le second paquet est formé de deux fonctions. La première est \( \mtu_{\mathopen[0 , 1/2 \mathclose]}\) et la seconde \( \mtu_{\mathopen[ 1/2 , 1 \mathclose]}\).

    Plus généralement le paquet numéro \( k\) est constitué des \( k\) fonctions \( \mtu_{\mathopen[ i/k , (i+1)/k \mathclose]}\) avec \( i=0,\ldots, k-1\).

    Vu que les fonctions du paquet numéro \( k\) ont pour norme \( \| f \|_p=\frac{1}{ k }\), nous avons évidemment \( f_n\to 0\) dans \( L^p\). Il est par contre visible que chaque paquet passe en revue tous les points de \( \mathopen[ 0 , 1 \mathclose]\). Donc pour tout \( x\) et pour tout \( N\), il existe (même une infinité) \( n>N\) tel que \( f_n(x)=1\). Il n'y a donc convergence ponctuelle nulle part.
\end{example}

La proposition suivante est une espèce de convergence dominée de Lebesgue pour \( L^p\).
\begin{proposition} \label{PropBVHXycL}
    Soit \( f\in L^p(\Omega)\) avec \( 1\leq p<\infty\) et \( (f_n)\) une suite de fonctions convergeant ponctuellement vers \( f\) et telle que \( | f_n |\leq | f |\). Alors \( f_n\) converge vers \( f\) au sens de \( L^p\).
\end{proposition}

\begin{proof}
    Nous avons immédiatement \( | f_n(x) |^p\leq | f(x) |^p\), de telle sorte que le théorème de la convergence dominée implique que \( f_n\in L^p\). La convergence dominée donne aussi que \( \| f_n \|_p\to\| f \|_p\), mais cela ne nous intéresse pas ici.

    Nous posons \( h_n(x)= | f_n(x)-f(x) | \). En reprenant la formule de majoration \eqref{EqZFSduFa} et en tenant compte du fait que \( | f_n(x) |\leq | f(x) |\), nous avons
   \begin{equation}
       h_n(x)\leq 2^{p-1}\big( | f_n(x) |^p+| f(x) |^p \big)\leq 2^p| f(x) |^p,
   \end{equation}
   ce qui prouve que \( | h_n |\) est uniformément (en \( n\)) majorée par une fonction intégrable, donc \( h_n\) est intégrable et on peut permuter la limite et l'intégrale (théorème de la convergence dominée \ref{ThoConvDomLebVdhsTf}) :
   \begin{equation}
       \lim_{n\to \infty} \| f_n-f \|^p_p=\lim_{n\to \infty} \int_{\eR^d}| f_n(x)-f(x) |^pdx=\int_{\eR^d}\lim_{n\to \infty} h_n(x)dx=0.
   \end{equation}
\end{proof}

%--------------------------------------------------------------------------------------------------------------------------- 
\subsection{Inégalité de Hölder et de Minkowski}
%---------------------------------------------------------------------------------------------------------------------------

\begin{proposition}[Inégalité de Hölder]       \label{ProptYqspT}
    Soit \( \Omega\) un espace mesuré et \( 1\leq p\), \( q\leq\infty\) satisfaisant \( \frac{1}{ p }+\frac{1}{ q }=1\). Soient \( f\in L^p(\Omega)\), \( g\in L^q(\Omega)\). Alors le produit \( fg\) est dans \( L^1(\Omega)\) et nous avons
    \begin{equation}
        \| fg \|_1\leq \| f \|_p\| g \|_q.
    \end{equation}
\end{proposition}
\index{inégalité!Hölder}
%TODO : une preuve.

\begin{remark}      \label{RemNormuptNird}
    Dans le cas d'un espace de probabilité, la fonction constante \( g=1\) appartient à \( L^p(\Omega)\). En prenant \( p=q=2\) nous obtenons
    \begin{equation}
        \| f \|_1\leq\| f \|_2.
    \end{equation}
\end{remark}

\begin{lemma}   \label{LemTLHwYzD}
    Lorsque \( I\) est borné nous avons \( L^2(I)\subset L^1(I)\). Si \( I\) n'est pas borné alors \( L^2(I)\subset L^1_{loc}(I)\).
\end{lemma}

\begin{proof}
    En effet si \( I\) est borné, alors la fonction constante \( 1\) est dans \( L^2(I)\) et l'inégalité de Hölder \ref{ProptYqspT} nous dit que le produit \( 1u\) est dans \( L^1(I)\).

    Si \( I\) n'est pas borné, nous refaisons le même raisonnement sur un compact \( K\) de \( I\).
\end{proof}

\begin{proposition}[Inégalité de Minkowski\cite{TUEWwUN}]     \label{PropInegMinkKUpRHg}
    Si \( 1\leq p<\infty\) et si \( f,g\in L^p(\Omega,\tribA,\mu)\) alors
    \begin{enumerate}
        \item   \label{ItemDHukLJi}
            \( \| f+g \|_p\leq \| f \|_p+\| g \|_p\)
        \item
            Il y a égalité si et seulement si les vecteurs \( f(x)\) et \( g(x)\) sont presque partout colinéaires : il existe \( \alpha,\beta\) tels que \( \alpha f+\beta g=0\) presque partout.

        \item   \label{ItemDHukLJiii}

            Si \( f(x,y)\) est mesurable sur l'espace produit \( \big( X\times Y,\mu\otimes\nu \big)\) et si \( p\geq 1\), alors
            \begin{equation}
                \left\|   x\mapsto\int_Y f(x,y)d\nu(y)   \right\|_p\leq
                \int_Y  \| f_y \|_pd\nu(y)
            \end{equation}
            où \( f_y(x)=f(x,y)\).

    \end{enumerate}
\end{proposition}
\index{inégalité!Minkowski}
%TODO : une preuve.

La partie \ref{ItemDHukLJiii} est une généralisation de l'inégalité triangulaire (c'est à dire du point \ref{ItemDHukLJi}) dans le cas où nous n'avons pas une somme de deux fonctions mais d'une infinité paramétrée par \( y\in Y\). Elle sera le plus souvent utilisée sous la forme déballée :
\begin{equation}    \label{EqZSiTZrH}
    \left[ \int_X\Big( \int_Y| f(x,y) |d\nu(y) \Big)^pd\mu(x) \right]^{1/p}\leq \int_Y\Big( \int_X| f(x,y) |^pd\mu(x) \Big)^{1/p}d\nu(y).
\end{equation}

%--------------------------------------------------------------------------------------------------------------------------- 
\subsection{Complétude}
%---------------------------------------------------------------------------------------------------------------------------

\begin{theorem}[\cite{SuquetFourierProba,UQSGIUo}]  \label{ThoUYBDWQX}
    Pour \( 1\leq p<\infty\), l'espace \( L^p(\Omega,\tribA,\mu)\) est complet.
\end{theorem}
\index{complétude}

\begin{proof}
    Soit \( (f_n)_{n\in\eN}\) une suite de Cauchy dans \( L^p\). Pour tout \( i\), il existe \( N_i\in\eN\) tel que $\| f_p-f_q \|_p\leq 2^{-i}$ pour tout \( p,q\geq N_i\). Nous considérons la sous suite \( g_i=f_{N_i}\), de telle sorte qu'en particulier
    \begin{equation}    \label{EqJLoDID}
        \|g_i-g_{i-1}\|_p\leq 2^{-i}.
    \end{equation}
    Pour chaque \( j\) nous considérons la somme télescopique
    \begin{equation}
        g_j=g_0+\sum_{i=1}^j(g_i-g_{i-1})
    \end{equation}
    et l'inégalité
    \begin{equation}
        | g_j |\leq | g_0 |+\sum_{i=1}^j| g_i-g_{i-1} |.
    \end{equation}
    Nous allons noter
    \begin{equation}        \label{EqSomPaFPQOWC}
        h_j=| g_0 |+\sum_{i=1}^j| g_i-g_{i-1} |.
    \end{equation}
    La suite de fonctions \( (h_j)\) ainsi définie est une suite croissante de fonctions positive qui converge donc (ponctuellement) vers une fonction \( h\) qui peut éventuellement valoir l'infini en certains points. Par continuité de la fonction \( x\mapsto x^p\) nous avons
    \begin{equation}
        \lim_{j\to \infty} h_j^p=h^p,
    \end{equation}
    puis par le théorème de la convergence monotone (théorème \ref{ThoConvMonFtBoVh}) nous avons
    \begin{equation}
        \lim_{j\to \infty} \int_{\Omega}h_j^pd\mu=\int_{\Omega}h^pd\mu.
    \end{equation}
    Utilisant à présent la continuité de la fonction \( x\mapsto x^{1/p}\) nous trouvons
    \begin{equation}
        \lim_{j\to \infty} \left( \int h_j^p \right)^{1/p}=\left( \int | h |^p \right)^{1/p}.
    \end{equation}
    Nous avons donc déjà montré que
    \begin{equation}
        \lim_{j\to \infty} \| h_j \|_p=\left( \int | h |^p \right)^{1/p}
    \end{equation}
    où, encore une fois, rien ne garantit à ce stade que l'intégrale à droite soit un nombre fini. En utilisant l'inégalité de Minkowski (proposition \ref{PropInegMinkKUpRHg}) et l'inégalité \eqref{EqJLoDID} nous trouvons
    \begin{equation}
        \|h_j\|_p\leq \|g_0\|_p+\sum_{i=1}^j\|g_i-g_{i-1}\|_p\leq \|g_0\|_p+1.
    \end{equation}
    En passant à la limite,
    \begin{equation}
        \left( \int| h |^p \right)^{1/p}=\lim_{j\to \infty}\|h_j\|_p \leq \|g_0\|_p+1<\infty.
    \end{equation}
    Par conséquent \( \int| h |^p\) est finie et
    \begin{equation}    \label{EqgLpdUPOBP}
        h\in L^p(\Omega,\tribA,\mu).
    \end{equation}
    En particulier, l'intégrale \( \int h\) est finie (parce que \( p\geq 1\)) et donc que \( h(x)<\infty\) pour presque tout \( x\in\Omega\).

    Nous savons que \( h(x)\) est la limite des sommes partielles \eqref{EqSomPaFPQOWC}, en particulier la série
    \begin{equation}
        \sum_{j=1}^{\infty}| g_i-g_{i-1} |
    \end{equation}
    converge ponctuellement. En vertu du corollaire \ref{CorCvAbsNormwEZdRc}, la série de terme général \( g_i-g_{i-1}\) converge ponctuellement. La suite \( g_i\) converge donc vers une fonction que nous notons \( g\). Par ailleurs la suite \( g_i\) est dominée par \( h\in L^p\), le théorème de la convergence dominée (théorème \ref{ThoConvDomLebVdhsTf}) implique que
    \begin{equation}
        \lim_{j\to \infty} \|g_j-g\|_p=0.
    \end{equation}
    Nous allons maintenant prouver que \( \lim_{n\to \infty\|f_n-g\|_p} =0\). Soit \( \epsilon>0\). Pour tout \( n\) et \( i\) nous avons
    \begin{equation}
        \|f_n-g\|_p=\|f_n-f_{N_i}+f_{N_i}-g\|_p\leq\|f_n-f_{N_i}\|_p+\|f_{N_i}-g\|_p.
    \end{equation}
    Pour rappel, \( f_{N_i}=g_i\). Si \(i\) et \( n\) sont suffisamment grands nous pouvons obtenir que chacun des deux termes est plus petit que \( \epsilon/2\).

    Il nous reste à prouver que \( g\in L^p(\Omega,\tribA,\mu)\). Nous avons déjà vu (équation \eqref{EqgLpdUPOBP}) que \( h\in L^p\), mais \( | g_i |\leq h^p\), par conséquent  \( g\in L^p\).

    Nous avons donc montré que la suite de Cauchy \( (f_n)\) converge vers une fonction de \( L^p\), ce qui signifie que \( L^p\) est complet.
\end{proof}

\begin{theorem}[Fischer-Riesz\cite{KXjFWKA}] \label{ThoGVmqOro}
    Soit un ouvert \( \Omega\) de \( \eR^n\) et \( p\in\mathopen[ 1 , \infty \mathclose]\). Alors
    \begin{enumerate}
        \item\label{ItemPDnjOJzi}
            Toute suite convergente dans \( L^p(\Omega)\) admet une sous-suite convergente presque partout sur \( \Omega\).
        \item\label{ItemPDnjOJzii}
            La sous-suite donnée en \ref{ItemPDnjOJzi} est dominée par un élément de \( L^p(\Omega)\).
        \item\label{ItemPDnjOJziii}
            L'espace \( L^p(\Omega)\) est de Banach.
    \end{enumerate}
\end{theorem}
\index{espace!de fonctions!$L^p$}
\index{complétude!espaces $ L^p$}

\begin{proof}
    Le cas \( p=\infty\) est à séparer des autres valeurs de \( p\) parce qu'on y parle de norme uniforme, et aucune sous-suite n'est à considérer.
    \begin{subproof}
    \item[Cas \( p=\infty\).]
    Nous commençons par prouver dans le cas \( p=\infty\). Soit \( (f_n)\) une suite de Cauchy dans \( L^{\infty}(\Omega)\), ou plus précisément une suite de représentants d'éléments de \( L^p\). Pour tout \( k\geq 1\), il existe \( N_k\geq 0\) tel que si \( m,n\geq N_k\), on a
    \begin{equation}
        \| f_m-f_n \|_{\infty}\leq \frac{1}{ k }.
    \end{equation}
    En particulier, il existe un ensemble de mesure nulle \( E_k\) sur lequel
    \begin{equation}
        | f_m(x)-f_n(x) |\leq\frac{1}{ k },
    \end{equation}
    et si nous posons \( E=\bigcup_{k\in \eN}E_k\), nous avons encore un ensemble de mesure nulle (lemme \ref{LemIDITgAy}). En  résumé, nous avons un \( N_k\) tel que si \( m,n\geq N_k\), alors 
    \begin{equation}    \label{EqKAWSmtG}
        | f_n(x)-f_m(x) |\leq \frac{1}{ k }
    \end{equation}
    pour tout \( x\) hors de \( E\). Donc pour chaque \( x\in\Omega\setminus E\), la suite \( n\mapsto f_n(x)\) est de Cauchy dans \( \eR\) et converge donc. Cela défini donc une fonction
    \begin{equation}
        \begin{aligned}
            f\colon \Omega\setminus E&\to \eR \\
            x&\mapsto \lim_{n\to \infty} f_n(x). 
        \end{aligned}
    \end{equation}
    Cela prouve le point \ref{ItemPDnjOJzi} : la convergence ponctuelle.

    En passant à la limite \( n\to \infty\) dans l'équation \ref{EqKAWSmtG} et tenant compte que cette majoration tient pour presque tout \( x\) dans \( \Omega\), nous trouvons
    \begin{equation}
        \| f-f_n \|_{\infty}\leq \frac{1}{ k }.
    \end{equation}
    Donc non seulement \( f\) est dans \( L^{\infty}\), mais en plus la suite \( (f_n)\) converge vers \( f\) au sens \( L^{\infty}\), c'est à dire uniformément. Cela prouve le point \ref{ItemPDnjOJziii}. En ce qui concerne le point \ref{ItemPDnjOJzii}, la suite \( f_n\) est entièrement (à partir d'un certain point) dominée par la fonction \( 1+| f |\) qui est dans \( L^p\).


    \item[Cas \( p=\infty\).]

        Toute suite convergente étant de Cauchy, nous considérons une suite de Cauchy \( (f_n)\) dans \( L^p(\Omega)\) et ce sera suffisant pour travailler sur le premier point. Pour montrer qu'une suite de Cauchy converge, il est suffisant de montrer qu'une sous-suite converge. Soit \( \varphi\colon \eN\to \eN\) une fonction strictement croissante telle que pour tout \( n\geq 1\) nous ayons
        \begin{equation}
            \| f_{\varphi(n+1)}-f_{\varphi(n)} \|_p\leq \frac{1}{ 2^{n} }.
        \end{equation}
        Pour créer la fonction \( \varphi\), il est suffisant de prendre le \( N_k\) donné par la condition de Cauchy pour \( \epsilon=1/2^k\) et de considérer la fonction définie par récurrence par \( \varphi(1)=N_1\) et \( \varphi(n+1)>\max\{ N_n,\varphi(n-1) \}\). Ensuite nous considérons la fonction
        \begin{equation}
            g_n(x)=\sum_{k=1}^n| f_{\varphi(k+1)}(x)-f_{\varphi(k)}(x) |.
        \end{equation}
        Notons que pour écrire cela nous avons considéré des représentants \( f_k\) qui sont alors des fonctions à l'ancienne. Étant donné que \( g_n\) est une somme de fonctions dans \( L^p\), c'est une fonction \( L^p\), comme nous pouvons le constater en calculant sa norme :
        \begin{equation}
            \| g_n \|_p\leq \sum_{k=1}^n\| f_{\varphi(k+1)}-f_{\varphi(k)} \|_p\leq\sum_{k=1}^n\frac{1}{ 2^k }\leq\sum_{k=1}^{\infty}\frac{1}{ 2^k }=1.
        \end{equation}
        Étant donné que tous les termes de la somme définissant \( g_n\) sont positifs, la suite \( (g_n)\) est croissante. Mais elle est bornée en norme \( L^p\) et donc sujette à obéir au théorème de Beppo-Levi \ref{ThoConvMonFtBoVh} sur la convergence monotone. Il existe donc une fonction \( g\in L^p(\Omega)\) telle que \( g_n\to g\) presque partout.

        Soit un \( x\in \Omega\) pour lequel \( g_n(x)\to g(x)\); alors pour tout \( n\geq 2\) et \( \forall q\geq 0\),
        \begin{subequations}    \label{EqWTHojCq}
            \begin{align}
                | f_{\varphi(n+q)}(x)-f_{\varphi(n)}(x) |&=\left| f_{\varphi(n+q)}(x)-\sum_{k=1}^{q-1}f_{\varphi(n+k)}(x) -\sum_{k=1}^{q-1}f_{\varphi(n+k)}(x)-f_{\varphi(n)}(x) \right| \\
                &=\left| \sum_{k=1}^qf_{\varphi(n+k)}-\sum_{k=1}^qf_{\varphi(n+k-1)}(x) \right|\\
                &\leq \sum_{k=1}^q\Big| f_{\varphi(n+k)}(x)-f_{\varphi(n+k-1)}(x) \Big|\\
                &=g_{n+q+1}(x)-g_{n+1}(x)\\
                &\leq g(x)-g_{n-1}(x).
            \end{align}
        \end{subequations}
        Nous prenons la limite \( n\to \infty\); la dernière expression tend vers zéro et donc
        \begin{equation}
            | f_{\varphi(n+q)}(x)-f_{\varphi(n)}(x) |\to 0
        \end{equation}
        pour tout \( q\). Donc pour presque tout \( x\in \Omega\), la suite \( n\mapsto f_{\varphi(n)}(x)\) est de Cauchy dans \( \eR\) et donc y converge vers un nombre que nous nommons \( f(x)\). Cela définit une fonction
        \begin{equation}
            \begin{aligned}
                f\colon \Omega\setminus E&\to \eR \\
                x&\mapsto \lim_{n\to \infty} f_{\varphi(n)}(x) 
            \end{aligned}
        \end{equation}
        où \( E\) est de mesure nulle. Montrons que \( f\) est bien dans \( L^p(\Omega)\); pour cela nous complétons la série d'inégalités \eqref{EqWTHojCq} en
        \begin{equation}
            \big| f_{\varphi(n+q)}(x)-f_{\varphi(n)}(x) \big|\leq g(x)-g_{n-1}(x)\leq g(x).
        \end{equation}
        En prenant la limite \( q\to \infty\) nous avons l'inégalité
        \begin{equation}    \label{EqMQbDRac}
            | f(x)-f_{\varphi(n)}(x) |\leq g(x)
        \end{equation}
        pour presque tout \( x\in\Omega\), c'est à dire pour tout \( x\in\Omega\setminus E\). Cette inégalité implique deux choses valables pour presque tout \( x\) dans \( \Omega\) :
        \begin{subequations}
            \begin{align}
                f(x)&\in B\big( g(x),f_{\varphi(n)}(x) \big)\\
                f_{\varphi(n)}(x)&\leq | f(x) |+| g(x) |.
            \end{align}
        \end{subequations}
        
        La première inégalité assure que \( | f |^p\) est intégrable sur \( \Omega\setminus E\) parce que \( | f |\) est majorée par \( | g |+| f_{\varphi(n)} |\). Elle prouve par conséquent le point \ref{ItemPDnjOJzi} parce que \(n\mapsto f_{\varphi(n)}\) est une sous-suite convergente presque partout. La seconde montre le point \ref{ItemPDnjOJzii}. 

        Attention : à ce point nous avons prouvé que \( n\mapsto f_{\varphi(n)}\) est une suite de fonctions qui converge \emph{ponctuellement presque partout} vers une fonction \( f\) qui s'avère être dans \( L^p\). Nous n'avons pas montré que cette suite convergeait au sens de \( L^p\) vers \( f\). Ce que nous devons montrer est que
        \begin{equation}    \label{EqJLfnEvj}
            \| f-f_{\varphi(n)} \|_p\to 0.
        \end{equation}
        L'inégalité \eqref{EqMQbDRac} nous donne aussi, toujours pour presque tout \( x\in \Omega\) :
        \begin{equation}
            \big| f(x)-f_{\varphi(n)}(x) \big|^p\leq g(x)^p
        \end{equation}
        ce qui signifie que la suite\quext{À ce point, \cite{KXjFWKA} se contente de majorer \( | f_{\varphi(n)}(x) |\) par \( | f(x) |+|g(x)\), mais je ne comprends pas comment cette majoration nous permet d'utiliser la convergence dominée de Lebesgue pour montrer \eqref{EqJLfnEvj}.} \(    | f-f_{\varphi(n)} |^p    \) est dominée par la fonction \( | g |^p\) qui est intégrable sur \( \Omega\setminus E\) et tout autant sur \( \Omega\) parce que \( E\) est négligeable; cela prouve au passage le point \ref{ItemPDnjOJzii}, et le théorème de la convergence dominée de Lebesgue (\ref{ThoConvDomLebVdhsTf}) nous dit que
        \begin{equation}
            \lim_{n\to \infty} \int_{\Omega} \big| f(x)-f_{\varphi(n)}(x) \big|^pdx=\int_{\Omega}\lim_{n\to \infty} \big| f(x)-f_{\varphi(n)}(x) \big|dx=0.
        \end{equation}
        Cette dernière suite d'inégalités se lit de la façon suivante :
        \begin{equation}
            \lim_{n\to \infty} \| f-f_{\varphi(n)} \|_p=\big\| \lim_{n\to \infty} | f-f_{\varphi(n)} | \big\|_p=0.
        \end{equation}
        Nous en déduisons que la suite \( n\mapsto f_{\varphi(n)}\) est convergente vers \( f\) au sens de la norme \( L^p(\Omega)\). Or la suite de départ \( (f_n)\) était de Cauchy (pour la norme \( L^p\)); donc l'existence d'une sous-suite convergente implique la convergence de la suite entière vers \( f\), ce qu'il fallait démontrer.
    \end{subproof}
\end{proof}

%--------------------------------------------------------------------------------------------------------------------------- 
\subsection{Théorème de représentation de Riesz}
%---------------------------------------------------------------------------------------------------------------------------

Dans le théorème suivant, \( E'\) est le dual topologique, c'est à dire l'espace des formes linéaires et continues.
\begin{definition}
    Un espace \( V\) est \defe{réflexif}{réflexif} si l'injection naturelle \( V\to V'\) est surjective.
\end{definition}

\begin{theorem}[\cite{LRBWftc}] \label{ThoSCiPRpq}
    Soit \( 1<p<\infty\) et une mesure quelconque. Alors
    \begin{enumerate}
        \item
            L'espace \( L^p\) est réflexif.
        \item
            Nous avons l'identification \( (L^p)'=L^q\) pour le nombre \( q\) tel que \( \frac{1}{ p }+\frac{1}{ q }=1\). Cette identification est donné de la façon suivante : l'application
            \begin{equation}
                \begin{aligned}
                    \Phi\colon L^q&\to (L^p)' \\
                    u&\mapsto \Big( \Phi_u\colon f\to \int_{\Omega}fu \Big) 
                \end{aligned}
            \end{equation}
            est une bijection isométrique.
    \end{enumerate}
    Si la mesure est \( \sigma\)-finie, alors
    \begin{enumerate}
        \item
            \( (L^1)'=L^{\infty}\)
        \item
            \( L^1\subset (L^{\infty})' \) avec une inclusion stricte sauf dans les cas triviaux.
    \end{enumerate}
\end{theorem}
% TODO : la preuve est sur Wikipédia.

\begin{proposition} \label{PropUKLZZZh}
    Soit \( f\in L^p(\Omega)\) telle que
    \begin{equation}
        \int_{\Omega}f\varphi=0
    \end{equation}
    pour tout \( \varphi\in C^{\infty}_c(\Omega)\). Alors \( f=0\) presque partout.
\end{proposition}

\begin{proof}
    Nous considérons la forme linéaire \( \Phi_f\in (L^q)'\) donnée par
    \begin{equation}
        \begin{aligned}
            \Phi_f\colon L^p&\to \eC \\
            u&\mapsto \int_{\Omega}fu
        \end{aligned}
    \end{equation}
    Par hypothèse cette forme est nulle sur la partie dense \(  C^{\infty}_c(\Omega)\). Si \( (\varphi_n)\) est une suite dans \(  C^{\infty}_c(\Omega)\) convergente vers \( u\) dans \( L^p\), nous avons pour tout \( n\) que
    \begin{equation}
        0=\Phi_f(\varphi_n)
    \end{equation}
    En passant à la limite, nous voyons que \( \Phi_f\) est la forme nulle. Elle est donc égale à \( \Phi_0\). La partie «unicité» du théorème de représentation de Riesz \ref{ThoSCiPRpq} nous indique alors que \( f=0\) dans \( L^p\) et donc \( f=0\) presque partout.
\end{proof}

%--------------------------------------------------------------------------------------------------------------------------- 
\subsection{Densité des fonctions infiniment dérivables à support compact}
%---------------------------------------------------------------------------------------------------------------------------

\begin{definition}
    Une fonction est \defe{étagée par rapport à \( L^p\)}{fonction!étagée} si elle est de la forme
    \begin{equation}
        f=\sum_{k=1}^Nc_k\mtu_{B_k}
    \end{equation}
    où les \( B_k\) sont des mesurables disjoints et \( \mtu_{B_k}\in L^p\) pour tout \( k\).
\end{definition}

\begin{lemma}   \label{LemWHIRdaX}
    Si \( f\) est une fonction étagée en même temps qu'être dans \( L^p\), alors elle est étagée par rapport à \( L^p\).
\end{lemma}

\begin{proof}
    Nous pouvons écrire
    \begin{equation}
        f=\sum_{k=1}^Nc_k\mtu_{B_k}
    \end{equation}
    où les \( B_k\) sont disjoints. Par hypothèse \( \| f \|_p\) existe. Donc chacune des intégrales \( \int_{\Omega}| \mtu_{B_k} |^p\) doit exister parce que les \( B_k\) étant disjoints, nous pouvons inverser la norme et la somme ainsi que la somme et l'intégrale :
    \begin{equation}
        \int_{\Omega}|f|^p=\int_{\Omega}\sum_{k=1}^N| c_k\mtu_{B_k}(x) |^pdx=\sum_{k=1}^N\int| c_k\mtu_{B_k}(x) |^pdx=\sum_{k=1}^N| c_k |^p\int_{\Omega}| \mtu_{B_k}(x) |^pdx.
    \end{equation}
\end{proof}
Le contraire n'est pas vrai : la fonction étagée sur \( \eR\) qui vaut \( n\) sur \( B(n,\frac{1}{ 4 })\) est étagée par rapport à \( L^p\), mais n'est pas dans \( L^p\).

\begin{theorem}[\cite{TUEWwUN}] \label{ThoILGYXhX}
    Nous avons des densités emboitées. Ici \( D\) est un borélien borné de \( \eR^d\) contenu dans \( B(0,r)\) et \( K\) est un compact contenant \( B(0,r+2)\).
    \begin{enumerate}
        \item
            Les fonctions étagées par rapport à \( L^p\) sur \( \eR^d\) sont denses dans \( L^p(\eR^d)\). A fortiori les fonctions étagées sont denses dans \( L^p\), mais nous n'en aurons pas besoin ici.
        \item\label{ItemYVFVrOIii}
            Il existe une suite \( f_n\) dans \(  C(K,\eC)\) telle que 
            \begin{equation}
                f_n\stackrel{L^p}{\to}\mtu_{D}.
            \end{equation}
        \item\label{ItemYVFVrOIiii}
            Si \( A\) est un borélien tel que \( \mtu_A\in L^p(\eR^d)\)\quext{Je pense que cette hypothèse manque dans \cite{TUEWwUN}. En tout cas je vois mal comment je pourrais justifier les différentes étapes de la preuve en prenant par exemple \( A=\eR^d\).} et si \( \epsilon>0\), alors il existe une suite de boréliens bornée \( (D_n)_{n\in \eN}\) tels que
            \begin{equation}
                \mtu_{D_n}\stackrel{L^p}{\to}\mtu_A.
            \end{equation}
        \item\label{ItemYVFVrOIiv}
            Il existe une suite \( \varphi_n\) dans \(  C^{\infty}_c(\eR^d)\) telle que 
            \begin{equation}
                \varphi_n\stackrel{L^p}{\to}\mtu_{D}.
            \end{equation}

        \item
            L'ensemble \( C^{\infty}_c(\eR^d)\) est dense dans \( L^p(\eR^d)\) pour tout \( 1\leq p<\infty\).
    \end{enumerate}
\end{theorem}
\index{densité!de \( C^{\infty}_c(\eR^d)\) dans \( L^p(\eR^d)\)}

\begin{proof}
   Nous allons montrer les choses point par point.
   \begin{enumerate}
       \item
           Si \( f\in L^1(\eR^d)\), nous savons par la proposition \ref{PropUXjnwLf} qu'il existe une suite \( f_n\) de fonctions étagées convergeant ponctuellement vers \( f\) telle que \( | f_n |\leq | f |\). La proposition \ref{PropBVHXycL} nous dit qu'alors \( f_n\stackrel{L^p}{\to}f\).

           La fonction \( f_n\) étant étagée et dans \( L^p\) en même temps, elle est automatiquement étagée par rapport à \( L^p\) par le lemme \ref{LemWHIRdaX}.

       \item\label{ItemYVFVrOIi}

           C'est le théorème d'approximation \ref{ThoAFXXcVa} appliqué au borélien \( D\) contenu dans l'espace mesuré \( K\).

       \item
            
           En vertu du point \ref{ItemYVFVrOIii}, il existe \( f\in C^0(K,\eR)\) telle que 
           \begin{equation}
             \| f-\mtu_D \|_p\leq \epsilon. 
           \end{equation}
           Ensuite, par le théorème de Weierstrass, il existe \( \varphi\in C^{\infty}(K,\eR)\) telle que \( \| f-\varphi \|_{\infty}\leq \epsilon\). Nous avons aussi
           \begin{equation}
               \| \varphi-f \|^p_p=\int_K| \varphi(x)-f(x) |^pdx\leq\mu(X)\| \varphi-f \|_{\infty}^p\leq \epsilon^p\mu(K).
           \end{equation}
           Quitte à prendre un \( \varphi\) correspondant à un \( \epsilon\) plus petit, nous avons
           \begin{equation}
               \| \varphi-f \|\leq \epsilon.
           \end{equation}
           En combinant et en passant à \( \epsilon/2\) nous avons trouvé une fonction \( \varphi\in  C^{\infty}(K,\eR)\) telle que
           \begin{equation}
               \| \varphi-\mtu_D \|\leq \epsilon.
           \end{equation}

       \item

           Nous considérons les boréliens fermés \( D_n=A\cap B(0,n)\). Alors \( \mtu_{D_n}\in L^p\) et nous avons pour \( n\) assez grand :
           \begin{equation}
               \int_{\eR^d}| \mtu_{D_n}(x)-\mtu_{A}(x) |^pdx=\int_{\eR^d\setminus B(0,n)}| \mtu_A(x) |^p<\epsilon,
           \end{equation}
           c'est à dire que \( \mtu_{D_n}\stackrel{L^p}{\to}\mtu_A\).

       \item

           Il suffit de remettre tout ensemble. Si \( f\in L^p(\eR^d)\), par le point \ref{ItemYVFVrOIi} nous commençons par prendre \( \sigma\) étagée par rapport à \( L^p\) telle que
           \begin{equation}
               \| \sigma-f \|_p\leq\epsilon.
           \end{equation}
           Ensuite nous écrivons \( \sigma\) sous la forme
           \begin{equation}
               \sigma=\sum_{k=1}^Nc_k\mtu_{B_k}
           \end{equation}
           et nous appliquons le point \ref{ItemYVFVrOIiii} à chacune des \( \mtu_{B_k}\) pour trouver des boréliens bornés \( D_k\) tels que
           \begin{equation}
               \| \mtu_{D_k}-\mtu_{B_k} \|_p\leq \epsilon.
           \end{equation}
           Enfin nous appliquons le point \ref{ItemYVFVrOIiv} pour trouver des fonctions \( \varphi_k\in C^{\infty}_c(\eR^d)\) telles que
           \begin{equation}
               \| \varphi_k-\mtu_{D_k} \|_p\leq \epsilon.
           \end{equation}
           
           Il n'est pas compliqué de calculer que
           \begin{equation}
               \big\| \sum_{k=1}^Nc_k\varphi_k-f \big\|_p\leq 2\epsilon\sum_kc_k+\epsilon.
           \end{equation}
        
   \end{enumerate}
\end{proof}

\begin{lemma}[\cite{TUEWwUN}]   \label{LemCUlJzkA}
    Soit \( 1\leq p<\infty\) et \( f\in L^p(\Omega)\). Nous notons \( \tau_v\) l'opérateur de translation par \( v\) :
    \begin{equation}
        \tau_vf\colon x\mapsto f(x-v).
    \end{equation}
    Alors sans surprises,
    \begin{equation}
        \lim_{v\to 0}\| \tau_v-f \|_p=0.
    \end{equation}
\end{lemma}

%--------------------------------------------------------------------------------------------------------------------------- 
\subsection{Approximation de l'unité}
%---------------------------------------------------------------------------------------------------------------------------

Nous considérons \( \Omega=\eR^d\) ou \( (S^1)^d\).

\begin{definition}
    Une \defe{approximation de l'unité}{approximation!de l'unité} sur \( \Omega\) est une suite \( (\varphi_n)\) dans \( L^1(\Omega)\) telle que
    \begin{subequations}
        \begin{align}
            \sup_k \| \varphi_k \|_1 <\infty\\
            \int_{\Omega}\varphi_n=1        \label{subeqAQcisBt}\\
            \lim_{k\to \infty} \int_{\Omega\setminus B(0,\alpha)}| \varphi_k |=0
        \end{align}
    \end{subequations}
    pour tout \( n\) et pour tout \( \alpha>0\).
\end{definition}
%TODO : voir si ça n'approxime pas un delta de Dirac d'une façon ou d'une autre.
Ce sont des fonctions dont la masse vient s'accumuler autour de zéro. En effet quel que soit le voisinage \( B(0,\alpha)\), si \( k\) est assez grand, il n'y a presque plus rien en dehors.

Pour le point \ref{subeqAQcisBt}, si \( \Omega\) est \( S^1\), la mesure que nous considérons est \( \frac{ dx }{ 2\pi }\).


\begin{example}
    Une façon de construire une approximation de l'unité sur \( \eR\) est de considérer une fonction \( \varphi\in L^1(\Omega)\) telle que \( \int\varphi=1\) puis de poser
    \begin{equation}
        \varphi_k(x)=k^d\varphi(kx).
    \end{equation}
    Ici, \( \Omega\) peut être \( \eR\) ou \( S^1\).
\end{example}

Le lemme suivant permet de construire des approximations de l'unité intéressantes.
\begin{lemma}[\cite{TUEWwUN}]   \label{LemCNjIYhv}
    Si nous posons
    \begin{equation}
        \varphi_n(x)=\left( \int\varphi(y)^n \right)^{-1}\varphi(x)^n,
    \end{equation}
    alors nous obtenons une approximation de l'unité dans les deux cas suivants :
    \begin{enumerate}
        \item
            Soit \( \varphi\) une fonction continue et positive sur \( S^1\) telle que \( \varphi(x)<\varphi(0)\) pour tout \( x\notin 2\pi \eZ\). Dans ce cas la mesure à prendre pour l'intégrale est \( \frac{ dy }{ 2\pi }\).
        \item
            Soit \( \varphi\) est une fonction continue et positive à support compact sur \( \eR^d\) telle que \( \varphi(x)>\varphi(0)\) pour tout \( x\neq 0\).
            
    \end{enumerate}
\end{lemma}

\begin{theorem}[\cite{TUEWwUN}] \label{ThoYQbqEez}
    Soit \( (\varphi_k)\) une approximation de l'unité sur \( \Omega=\eR^d\) ou \( (S^1)^d\).
    \begin{enumerate}
        \item
            Si \( g\) est mesurable et bornée sur \( \Omega\) et si \( g\) est continue en \( x_0\) alors 
            \begin{equation}
                (\varphi_k*g)(x_0)\to g(x_0).
            \end{equation}
        \item
            Si \( g\in L^p(\Omega)\) (\( 1\leq p<\infty\)) alors
            \begin{equation}
                \varphi_k*g\stackrel{L^p}{\to}g.
            \end{equation}
        \item
            Si \( g\) est uniformément continue et bornée, alors
            \begin{equation}
                \varphi_k*g\stackrel{L^{\infty}}{\to}g
            \end{equation}
    \end{enumerate}
\end{theorem}

\begin{proof}
    Les trois points vont se ressembler.
    \begin{enumerate}
        \item
            Nous notons \( d_k=(\varphi_k*g)(x_0)-g(x_0)\) et nous devons prouver que \( d_k\to 0\). Vu que \( \varphi_k\) est d'intégrale \( 1\) sur \( \Omega\) nous pouvons écrire
            \begin{equation}
                |d_k|=\big| \int_{\Omega}\big( g(x_0-y)-g(x_0) \big)\varphi_k(y)dy\big|\leq\int_{\Omega}\big| g(x_0-y)-g(x_0) \big| |\varphi_k(y) |dy. 
            \end{equation}
            Nous notons \( M=\sup_k\| \varphi_k \|_1\), et nous considérons \( \alpha>0\) tel que
            \begin{equation}
                \big| g(x_0-y)-g(x_0) \big|\leq \epsilon.
            \end{equation}
            Nous nous restreignons maintenant aux \( k\) suffisamment grands pour que \( \int_{\complement B(0,\alpha)}| \varphi_k(y) |dy\leq \epsilon\). Alors en découpant l'intégrale en \( B(0,\alpha)\) et son complémentaire dans \( \Omega\),
            \begin{equation}
                | d_k |\leq \epsilon M+\int_{\complement B(0,\alpha)} 2\| g \|_{\infty}| \varphi_k(y) |dy  \leq \epsilon M+2\| g \|_{\infty}\epsilon\leq \epsilon C.
            \end{equation}
            Donc oui, nous avons \( | d_k |\to 0\), et donc le premier point du théorème.

        \item

            Nous posons \( d_k(x)=(\varphi_k*g)(x)-g(x)\) et nous voulons prouver que \( \| d_k \|_{\infty}\to 0\), c'est à dire que \( d_k(x)\) converge vers zéro uniformément en \( x\). Nous posons aussi 
            \begin{equation}
                \tau_y(g)\colon x\mapsto g(x-y).
            \end{equation}
            En récrivant le produit de convolution, une petite majoration donne
            \begin{equation}
                | d_k(x) |\leq \int_{\Omega}\| \tau_y(g)-g \|_{\infty}| \varphi_k(y) |dy.
            \end{equation}
            L'uniforme continuité de \( g\) signifie que pour tout \( \epsilon\), il existe un \( \alpha\) tel que pour tout \( y\in B(0,\alpha)\),
            \begin{equation}
                \| \tau_y(g)-g \|_{\infty}\leq \epsilon.
            \end{equation}
            Encore une fois nous découpons le domaine d'intégration en \( B=B(0,\alpha)\) et son complémentaire :
            \begin{subequations}
                \begin{align}
                    \| d_k \|_{\infty}&\leq\int_B\underbrace{\| \tau_y(g)-g \|_{\infty}}_{\leq \epsilon}| \varphi_k(y) |dy+\int_{\complement B}\underbrace{\| \tau_y(g)-g \|_{\infty}}_{\leq 2\| g \|_{\infty}}| \varphi_k(y) |\\
                    &\leq \epsilon M+2\| g \|_{\infty}\epsilon
                \end{align}
            \end{subequations}
            où la seconde ligne est justifiée par le choix d'un \( k\) assez grand pour que \( \int_{\complement B}| \varphi_k(y) |dy\leq \epsilon\).

            Nous avons donc bien \( \| d_k \|_{\infty}\to 0\).

        \item

            Cette fois \( g\in L^p(\Omega)\) et nous cherchons à montrer que \( \| d_k \|_p\to 0\). Encore qu'ici \( d_k\) soit défini à partir d'un représentant dans la classe de \( g\) et que d'ailleurs, nous allons travailler avec ce représentant.

            D'abord nous développons un peu ce \( d_k\) :
            \begin{subequations}
                \begin{align}
                \| d_k \|_p&=\left[ \int_{\Omega}\left|     \int_{\Omega}\big( g(x-y)-g(x) \big)\varphi_k(y)dy  \right|^pdx \right]^{1/p}\\
                &\leq\left[    \int_{\Omega}\Big( \int_{\Omega}| g(x-y)-g(x) |\cdot |\varphi_k(y) |dy \Big)^pdx \right]^{1/p}.
                \end{align}
            \end{subequations}
            À cette dernière expression nous appliquons l'inégalité de Minkowski (théorème \ref{PropInegMinkKUpRHg}) sous la forme \eqref{EqZSiTZrH} pour la norme \( d\nu(y)=| \varphi_k(y) |dy\) et \( f(x,y)=g(x-y)-g(x)\) :
            \begin{equation}
                \| d_k \|_p\leq \int_{\Omega}\Big( \int_{\Omega}\big| g(x-y)-g(x) \big|^pdx \Big)^{1/p}| \varphi_k(y) |dy=\int_{\Omega}\| \tau_yg-g \|_p| \varphi_k(y) |dy.
            \end{equation}
            Par le lemme \ref{LemCUlJzkA} nous pouvons trouver \( \alpha>0\) tel que \( \| \tau_yg-g \|_p\leq \epsilon\) pour tout \( y\in B(0,\alpha)\). Avec cela nous découpons encore le domaine d'intégration :
            \begin{equation}
                \| d_k \|_p\leq \int_{B(0,\alpha)}\underbrace{\| \tau_yg-g \|_p}_{\leq \epsilon}| \varphi_k(y) |dy+\int_{\complement B(0,\alpha)}  \underbrace{\| \tau_yg-g \|_p}_{\leq 2\| g \|_p}| \varphi_k(y) |dy\leq \epsilon M+2\epsilon\| g \|_p.
            \end{equation}
    \end{enumerate}
\end{proof}

Une petite remarque en passant : aussi triste que cela en ait l'air, la convergence uniforme n'implique pas la convergence \( L^p(\Omega)\) si \( \Omega\) n'est pas borné. En effet si \( f\in L^p\), la suite donnée par
\begin{equation}
    f_n(x)=f(x)+\frac{1}{ n }
\end{equation}
converge uniformément vers \( f\), mais 
\begin{equation}
    \| f_n-f \|_p=\int_{\Omega}\frac{1}{ n }
\end{equation}
n'existe même pas si le domaine \( \Omega\) n'est pas borné.

%---------------------------------------------------------------------------------------------------------------------------
\subsection{Densité des polynôme trigonométrique}
%---------------------------------------------------------------------------------------------------------------------------

Nous allons beaucoup travailler avec le \defe{système trigonométrique}{système!trigonométrique} donné par \( \{ e_n \}_{n\in \eZ}\) et
\begin{equation}
    e_n(t)= e^{int}.
\end{equation}
Une bonne partie de la douleur qu'évoque mot «densité» consiste à montrer que ce système est total dans \( L^2(S^1)=L^2(\mathopen[ 0 , 2\pi \mathclose])\), et donc en est une base hilbertienne. Un \defe{polynôme trigonométrique}{polynôme!trigonométrique} est une fonction de la forme \( P=\sum_{k=-N}^NP_ke_k\) pour des constantes \( P_k\).

Un \defe{polynôme trigonométrique}{polynôme!trigonométrique} est une fonction de la forme
\begin{equation}
    P(t)=\sum_{n=-N}^Nc_n e^{int}.
\end{equation}

\begin{lemma}   \label{LemZVfZlms}
    Deux petits résultats simples mais utiles à propos des polynômes trigonométriques.
    \begin{enumerate}
        \item
    Si \( f\in L^1(S^1)\), alors nous avons la formule
    \begin{equation}
        f*e_n=c_n(f)e_n.
    \end{equation}
\item
            
    Si \( P\) est un polynôme trigonométrique et si \( f\in L^1(S^1)\) alors \( f*P\) est encore un polynôme trigonométrique.
    \end{enumerate}
\end{lemma}

\begin{proof}
    Le premier point est un simple calcul :
    \begin{subequations}
        \begin{align}
            (f*e_n)(x)&=\frac{1}{ 2\pi }\int_0^{2\pi}f(x-t)e_n(t)dt\\
            &= e^{inx}\int_0^{2\pi}f(x-t) e^{-in(x-t)}dt\\
            &=c_n(f)e_n(x).
        \end{align}
    \end{subequations}

    En ce qui concerne le second point, nous notons \( P=\sum_{k=-N}^NP_ke_k\), et par linéarité de la convolution,
    \begin{equation}
        f*P=\sum_{k=-N}^NP_kf*e_k=\sum_{k=-N}^nc_k(f)e_k,
    \end{equation}
    qui est encore un polynôme trigonométrique.
\end{proof}

\begin{example} \label{ExDMnVSWF}
    Sur \( S^1\) nous construisons alors l'approximation de l'unité basée sur la fonction \( 1+\cos(x)\) et le lemme \ref{LemCNjIYhv}. Cette fonction est évidemment un polynôme trigonométrique parce que 
    \begin{equation}
        \cos(x)=\frac{  e^{ix}+ e^{-ix} }{2}.
    \end{equation}
    Ensuite les puissances le sont aussi à cause de la formule du binôme :
    \begin{equation}
        \big( 1+\cos(x) \big)^n=\sum_{k=0}^n\binom{ n }{ k }\cos^n(x),
    \end{equation}
    dans laquelle nous pouvons remettre \( \cos(x)\) comme un polynôme trigonométrique et développer à nouveau la puissance avec (encore) la formule du binôme. La chose importante est qu'il existe une approximation de l'unité \( (\varphi_n)\) formée de polynômes trigonométrique.

    Ce qui fait la spécificité des polynômes trigonométriques est qu'ils sont à la fois stables par convolution (lemme \ref{LemZVfZlms}) et qu'ils permettent de créer une approximation de l'unité sur \( \mathopen[ 0 , 2\pi \mathclose]\). Ce sont ces deux choses qui permettent de prouver l'important théorème suivant.
\end{example}

\begin{theorem} \label{ThoQGPSSJq}
    Les polynôme trigonométriques sont dense dans \( L^p(S^1)\) pour \( 1\leq p<\infty\).
\end{theorem}
\index{densité!des polynômes trigonométriques dans \( L^p(S^1)\)}

\begin{proof}
    Soit \( f\in L^p(S^1)=L^p\big( \mathopen[ 0 , 2\pi \mathclose] \big)\), et \( (\varphi_n)\) une approximation de l'unité sur \( S^1\) formée de polynômes trigonométriques, par exemple ceux de l'exemple \ref{ExDMnVSWF}. D'abord \( \varphi_k*f\) est un polynôme trigonométrique par le lemme \ref{LemZVfZlms}, et ensuite nous avons convergence
    \begin{equation}
        \varphi_k*f\stackrel{L^p}{\to}f
    \end{equation}
    par le théorème \ref{ThoYQbqEez}. Nous avons donc convergence \( L^p\) d'une suite de polynômes trigonométrique, ce qui prouve que l'espace de polynômes trigonométriques est dense dans \( L^p(S^1)\).
\end{proof}

\begin{remark}
    Deux remarques.
    \begin{itemize}
        \item 
            Il n'est pas possible que les polynômes trigonométriques soient dense dans \( L^{\infty}\) parce qu'une limite uniforme de fonctions continues est continue (c'est le théorème \ref{ThoUnigCvCont}).
        \item 
            Nous donnerons au théorème \ref{ThoDPTwimI} une démonstration indépendante de la densité des polynômes trigonométriques dans \( L^p(S^1)\).
    \end{itemize}
\end{remark}

%+++++++++++++++++++++++++++++++++++++++++++++++++++++++++++++++++++++++++++++++++++++++++++++++++++++++++++++++++++++++++++
\section{L'espace \texorpdfstring{$L^2$}{$L^2$}}
%+++++++++++++++++++++++++++++++++++++++++++++++++++++++++++++++++++++++++++++++++++++++++++++++++++++++++++++++++++++++++++
\label{SecCKZSrZK}

L'espace \( L^2\) est l'espace \( L^p\) définit à la section \ref{SecVKiVIQK} avec \( p=2\). Cependant il possède une propriété extraordinaire\footnote{Tant et si bien que certains n'hésitent pas à définir le nombre \( 2\) comme étant l'unique \( p\) tel que \( L^p\) est un Hilbert.} par rapport aux autres \( L^p\), c'est que la norme \( | . |_2\) dérive d'un produit scalaire. Il sera donc un espace de Hilbert.

Soit \( (\Omega,\tribA,\mu)\) un espace mesuré. Nous considérons l'opération
\begin{equation}    \label{DefProdScalLubrgTj}
    \langle f, g\rangle =\int_{\Omega}f(\omega)\overline{ g(\omega)}d\mu(\omega)
\end{equation}
et la norme associée
\begin{equation}
    \| f \|_2=\sqrt{\langle f, f\rangle }.
\end{equation}
Nous considérons l'ensemble
\begin{equation}
    \mL^2(\Omega,\mu)=\{ f\colon \Omega\to \eR\tq \| f \|_2<\infty \}
\end{equation}
et la relation d'équivalence \( f\sim g\) si et seulement si \( f(x)=g(x)\) pour \( \mu\)-presque tout \( x\). L'espace que nous considérons est
\begin{equation}
    L^2=\mL^2/\sim.
\end{equation}

\begin{lemma}
    La formule \eqref{DefProdScalLubrgTj} définit un produit scalaire sur \( L^2\), et ce dernier est un espace de Hilbert.
\end{lemma}

\begin{proof}
    D'abord si \( f\) et \( g\) sont dans \( L^2\), alors l'inégalité de Hölder (proposition \ref{ProptYqspT}) nous indique que le produit \( fg\) est un élément de \( L^1\). Par conséquent la formule a un sens.

    Ensuite nous montrons que la formule passe au quotient. Pour cela, nous considérons des fonctions \( \alpha\) et \( \beta\) nulles presque partout et nous regardons le produit de \( f_1=f+\alpha\) par \( g_1=g+\beta\) :
    \begin{equation}
        \langle f_1, g_1\rangle =\int fg+\beta f+\alpha g+ \alpha\beta.
    \end{equation}
    Les fonction \( \beta f\), \( \alpha g\) et \( \alpha\beta\) étant nulles presque partout, leur intégrale est nulle et nous avons bien \( \langle f_1, g_1\rangle =\langle f,g \rangle \). Nous pouvons donc considérer le produit sur l'ensemble des classes.

    Pour vérifier que la formule est un produit scalaire, le seul point non évidement est de prouver que \( \langle f, f\rangle =0\) implique \( f=0\). Cela découle du fait que
    \begin{equation}
        \langle f, f\rangle =\int_{\Omega}| f |^2.
    \end{equation}
    La fonction \( x\mapsto | f(x) |^2\) vérifie les hypothèses du lemme \ref{Lemfobnwt}. Par conséquent \( | f(x) |^2\) est presque partout nulle.

    En ce qui concerne le fait que \( L^2(\Omega)\) soit un espace de Hilbert, il s'agit simplement de se remémorer que c'est un espace complet (théorème  \ref{ThoUYBDWQX}) et dont la norme dérive d'un produit scalaire. Nous sommes donc bien dans la définition \ref{DefORuBdBN}.
\end{proof}

Nous postposons à la section \ref{SecNCSkgUK}, lorsque nous parlerons de séries de Fourier, l'étude plus avancée de l'espace \( L^2(\mathopen] 0 , 2\pi \mathclose[)\). À ce moment, le produit scalaire sera défini avec un coefficient \( 1/2\pi\). Voir l'équation \eqref{EqQBcpyyJ}.

Nous notons ici une conséquence du théorème \ref{ThoGVmqOro} dans le cas de l'espace \( L^2\). La proposition suivante est une petite partie du corollaire \ref{CorQETwUdF}, qui sera d'ailleurs démontré de façon indépendante.

\begin{proposition}
    Si nous avons une suite de réels \( (a_k)\) telle que \( \sum_{k=0}^{\infty}| a_k |^2<\infty\) alors la suite
    \begin{equation}
        f_n(x)=\sum_{k=0}^na_k e^{ikx}
    \end{equation}
    converge dans \( L^2\big( \mathopen] 0 , 2\pi \mathclose[ \big)\).
\end{proposition}

\begin{proof}
    Quitte à séparer les parties réelles et imaginaires, nous pouvons faire abstraction du fait que nous parlons d'une série de fonctions à valeurs dans \( \eC\) au lieu de \( \eR\).

    Un simple calcul est :
    \begin{equation}    \label{EqHVdJxZT}
        \| f_n-f_m \|^2\leq\int_0^{2\pi}\sum_{k=n}^m| a_k |^2dx\leq 2\pi\sum_{k=n}^m| a_k |^2.
    \end{equation}
    Par hypothèse le membre de droite est \( | s_m-s_n |\) où \( s_k\) dénote la suite des somme partielle de la série des \( | a_k |^2\). Cette dernière est de Cauchy (parce que convergente dans \( \eR\)) et donc la limite \( n\to\infty\) (en gardant \( m>n\)) est zéro. Donc la suite des \( f_n\) est de Cauchy dans \( L^2\) et donc converge dans \( L^2\).
\end{proof}


%+++++++++++++++++++++++++++++++++++++++++++++++++++++++++++++++++++++++++++++++++++++++++++++++++++++++++++++++++++++++++++ 
\section{Espaces de Sobolev}
%+++++++++++++++++++++++++++++++++++++++++++++++++++++++++++++++++++++++++++++++++++++++++++++++++++++++++++++++++++++++++++

Sauf mention du contraire dans cette section \( I\) est un intervalle borné ouvert \( I=\mathopen] a , b \mathclose[\) de \( \eR\).

\begin{proposition} \label{PropHFWNpRb}
    Une fonction \( h\in C^{\infty}_c(I)\) admet une primitive dans \(  C^{\infty}_c(I)\) si et seulement si \( \int_Ih=0\).
\end{proposition}

\begin{proof}
    Si une primitive \( H\) de \( h\) est à support compact, alors
    \begin{equation}
        \int_Ih=H(b)-H(a)=0-0=0.
    \end{equation}
    Pas de problèmes dans ce sens.

    Supposons maintenant que \( \int_Ih=0\). Le fait que \( h\) admette une primitive dans \(  C^{\infty}(I)\) est évident : toute fonction continue admet une primitive. Soit \( H\) une telle primitive et \( \tilde H=H-H(b)\). Alors \( \tilde H(b)=0\) et 
    \begin{equation}
        \tilde H(a)=H(a)-H(b)=-\int_Ih=0.
    \end{equation}
    Nous rappelons que le support d'une fonction est \emph{la fermeture} de l'ensemble des points de non-annulation.

    Supposons que le support de \( h\) soit inclus dans \( \mathopen[ m , M \mathclose]\subset\mathopen] a , b \mathclose[\). En prenant des nombres \( m'\) et \( M'\) tels que \( a<m'<m\) et \( M<M'<b\) (nous insistons sur le caractère strict de ces inégalités), la fonction \( h\) est nulle sur \( \mathopen[ a , m' \mathclose]\) et sur \( \mathopen[ M' , b \mathclose]\); la fonction \( \tilde H\) doit donc y être constante. Mais nous avons déjà vu que \( \tilde H(a)=\tilde H(b)=0\). Donc l'ensemble des points sur lesquels \( \tilde H\) n'est pas nul est inclus à \( \mathopen] m' , M' \mathclose[\) et donc est strictement (des deux côtés) inclus à \( I\).
\end{proof}


\begin{definition}
    Soit \( f\in L^p(\Omega)\) où \( I\) est l'intervalle ouvert \( \mathopen] a , b \mathclose[\). Sa \defe{dérivée au sens des distributions}{dérivée!au sens de distributions} est une fonction \( g\) telle que
        \begin{equation}
            \int_If\varphi'=-\int_Ig\varphi
        \end{equation}
        pour tout \( \varphi\in C^{\infty}_c(I)\).
\end{definition}

\begin{lemma}
    Lorsqu'une fonction admet une dérivée au sens des distributions, cette dernière est unique (et justifie le singulier dans la définition).
\end{lemma}

\begin{proof}
    Soient \( g,h\in L^2\) tels que 
    \begin{equation}
        \int_Iu\varphi'=-\int_Ig\varphi=-\int_Ih\varphi
    \end{equation}
    pour tout \( \varphi\in C^{\infty}_c(I)\). Nous avons alors
    \begin{equation}
        \int_I(g-h)\varphi=0.
    \end{equation}
    Cela implique que \( g-h=0\) presque partout par la proposition \ref{PropUKLZZZh}\footnote{Ou alors par le lemme \ref{LemDQEKNNf} qui est moins général mais tout aussi bien pour ici.}.
\end{proof}

\begin{definition}
    Soit \( I=\mathopen] a , b \mathclose[\) un ouvert borné de \( \eR\). L'\defe{espace de Sobolev}{espace!de Sobolev} \( H^1(I)\)\nomenclature[Y]{\( H^1(I)\)}{espace de Sobolev} est l'ensemble
    \begin{equation}
        H^1(I)=\Big\{   u\in L^2(I)\tq\exists g\in L^2(I)\tq\forall \varphi\in  C^{\infty}_c(I),\int_Iu\varphi'=-\int_Ig\varphi   \Big\}.
    \end{equation}
\end{definition}
 
L'unique élément \( g\) de \( L^2(I)\) vérifiant \( \int_Iu\varphi'=-\int_Ig\varphi\) est noté \( u'\) est est nommé \defe{dérivée}{dérivée!dans Sobolev $ H^1(I)$}; nous verrons dans les prochaines pages pourquoi.

L'espace \( H^1\) accepte le produit scalaire suivant :
\begin{equation}
    \langle u, v\rangle =\int_Iuv+\int_Iu'v',
\end{equation}
et nous notons \( \| . \|_{H^1}\) la norme correspondante qui n'est autre que
\begin{equation}
    \| u \|_{H^1}=\langle u, u\rangle =\| u \|^2_{L^2}+\| u' \|_{L^2}.
\end{equation}

Nous introduisons l'espace \( L^1_{loc}(I)\)\nomenclature[Y]{\( L^1_{loc}(I)\)}{fonctions intégrables sur les compacts de \( I\)} des fonctions étant \( L^1\) sur tout compact de \( I\). 

\begin{proposition} \label{PropLGoLtcS}
    Si \( f\in L^1_{loc}(I)\) est telle que
    \begin{equation}
        \int_If\varphi'=0
    \end{equation}
    pour tout \( \varphi\in  C^{\infty}_c(I)\), alors il existe une constante \( C\) telle que \( f=C\) presque partout.
\end{proposition}

\begin{proof}
    Soit \( \psi\in C^{\infty}_c(I)\) une fonction d'intégrale \( 1\) sur \( I\). Si \( w\in C^{\infty}_c(I)\) alors nous considérons la fonction
    \begin{equation}
        h=w-\psi\int_Iw,
    \end{equation}
    qui est dans \(  C^{\infty}_c(I)\) et dont l'intégrale sur \( I\) est nulle. Par la proposition \ref{PropHFWNpRb}, la fonction \( h\) admet une primitive dans \(  C^{\infty}_c(I)\); et nous notons \( \varphi\) cette primitive. L'hypothèse appliquée à \( \varphi\) donne
    \begin{equation}
        0=\int_If\varphi'=\int_If\left( w-\psi\int_Iw \right)=\int_Ifw-\underbrace{\left( \int_If(x)\psi(x)dx \right)}_C\left( \int_Iw(y)dy \right)=\int_Iw(f-C).
    \end{equation}
    L'annulation de la dernière intégrale implique par la proposition \ref{PropUKLZZZh} que \( f-C=0\) dans \( L^2\), c'est à dire \( f=C\) presque partout.
\end{proof}

\begin{corollary}   \label{CorEVJYihj}
    Si \( u\in H^1(I)\) et si \( u'=0\) alors il existe une constant \( C\) telle que \( u=C\) presque partout.
\end{corollary}

\begin{proof}
    L'hypothèse \( u'=0\) signifie que pour tout fonction \( \varphi\in C^{\infty}_c(I)\),
    \begin{equation}
        \int_Iu\varphi'=\int_Iu'\varphi=0.
    \end{equation}
    La proposition \ref{PropLGoLtcS} nous dit alors qu'il existe une constante \( C\) telle que \( u=C\) presque partout.
\end{proof}

\begin{lemma}   \label{LemMPkbZxX}
    Tout élément de \( H^1(I)\) admet un unique représentant continu.
\end{lemma}
Nous verrons dans le corollaire \ref{CorCEPJGAu} que ce représentant pourra être prolongé par continuité sur \( \bar I\).

\begin{proof}
    Soit \( y_0\in I\) et \( u\in H^1(I)\). Nous considérons la fonction
    \begin{equation}
        \bar u(x)=\int_{y_0}^xu'(t)dt.
    \end{equation}
    Notons que par définition, \( u'\in L^2\) donc l'intégrale ne pose pas de problèmes. Montrons que \( \bar u\) est continue sur \( \bar I\). Pour cela nous considérons \( x\in\bar I\) et \( h\) tel que \( x+h\in \bar I\). Alors
    \begin{equation}
        \big| \bar u(x+h)-\bar u(x) \big|=\big| \int_x^{x+h}u' \big|\leq \int_x^{x+h}| u' |.
    \end{equation}
    Mais la fonction \( | u' |\) est dans \( L^1_{loc}(I)\) par le lemme \ref{LemTLHwYzD}; elle est en particulier intégrable sur un ouvert contenant \( x\) et par conséquent la dernière intégrale tend vers zéro lorsque \( h\) tend vers \( 0\).

    Nous prouvons à présent que \( \bar u\) est dans \( H^1(I)\) et que sa dérivée est égale à \( u'\); pour cela nous allons montrer que pour tout \( \varphi\in  C^{\infty}_c(I)\),
    \begin{equation}
        \int_I\bar u\varphi'=-\int_Iu'\varphi.
    \end{equation}
    Nous avons
    \begin{equation}
            \int_I\bar u\varphi'=\int_I\left( \int_{y_0}^xu'(t)dt\right)\varphi'(x)dx
            =\int_{a}^{y_0}\left( \int_{y_0}^xu'(t)dt\right)\varphi'(x)dx+\int_{y_0}^b\left( \int_{y_0}^xu'(t)dt\right)\varphi'(x)dx.
    \end{equation}
    Pour faire plus court, nous notons \( f(t,x)=u'(t)\varphi'(x)\). La première intégrale vaut
    \begin{subequations}
        \begin{align}
            \int_a^{y_0}\left( \int_{y_0}^x u'(t)\varphi'(x) \right)&=\int_a^{y_0}\left(\int_{y_0}^af(t,x)\mtu_{t<x}(t,x)dt\right)dx\\
            &=\int_{y_0}^a\int_a^{y_0}f(t,x)\mtu_{t>x}dxdt  \label{subeqBVyBPLp}\\
            &=\int_{y_0}^a\int_a^tf(t,x)dxdt\\
            &=-\int_a^{y_0}\int_a^tu'(t)\varphi'(x)dx\,dt
        \end{align}
    \end{subequations}
    La permutation d'intégrales pour obtenir \eqref{subeqBVyBPLp} est due au théorème de Fubini \ref{ThoFubinioYLtPI}\ref{ItemQMWiolgiii}. Par le même petit jeu, la seconde intégrale vaut
    \begin{equation}
        \int_{y_0}^b\int_t^b u'(t)\varphi'(x)dx\,dt.
    \end{equation}
    En refaisant la somme,
    \begin{subequations}
        \begin{align}
            \int_I\bar u\varphi'
            &=-\int_a^{y_0}u'(t)\left( \int_a^t\varphi'(x)dx \right)dt+\int_{y_0}^bu'(t)\left( \int_t^b\varphi'(x)dx \right)dt\\
            &=-\int_a^{y_0}u'(t)\big( \varphi(t)-\varphi(a) \big)dt+\int_{y_0}^bu'(t)\big( \varphi(b)-\varphi(t) \big)\\
            &=-\int_a^bu'\varphi\\
            &=-\int_Iu'\varphi.
        \end{align}
    \end{subequations}
    Notons que \( \varphi(a)=\varphi(b)=0\) parce que \( \varphi\) est à support compact dans \( \mathopen] a , b \mathclose[\). Nous avons donc prouvé que \( \bar u\) est dans \( H^1(I)\) et que \( \bar u'=u'\). Par le corollaire \ref{CorEVJYihj}, nous avons une constante \( C\) telle que \( \bar u=u+C\) presque partout, c'est à dire \( u=\bar u +C\) dans \( H^1(I)\). 

        En résumé, \( \tilde u\tilde u==\bar u+C\) est un représentant continu de \( u\) dans \( L^2(I)\).

        L'unicité du représentant continu est simplement le fait que deux fonctions continues égales presque partout sont égales (proposition  \ref{PropNCMToWI}).
    
\end{proof}

\begin{proposition}     \label{PropGWOIoDg}
    Si \( u\in H^1(I)\), alors
    \begin{equation}
        u(x)-u(y)=\int_y^xu'
    \end{equation}
    pour tout \( x,y\in I\).
\end{proposition}

\begin{proof}
    Pour fixer les idées, nous supposons \( x<y\). Nous considérons une suite \( \varphi_n\in C^{\infty}_c(I)\) convergeant uniformément sur \( I\) vers \( \mtu_{\mathopen[ x , y \mathclose]}\). Nous exigeons de plus que 
    \begin{itemize}
        \item 
        \( \varphi_n'\) est positive sur \( \mathopen[ a , x+\frac{1}{ n } \mathclose]\)
    \item
        \( \varphi_n'\) est négative sur \( \mathopen[ y-\frac{1}{ n } , b \mathclose]\) 
    \item
        \( \varphi_n=1\) sur \( \mathopen[ x+\frac{1}{ n } , y-\frac{1}{ n } \mathclose]\).
    \item
        \( \varphi_n=0\) sur \( \mathopen[ a , x-1/n \mathclose]\) et sur \( \mathopen[ y+1/n , b \mathclose]\).
    \end{itemize}
    Pour chaque \( n\), nous découpons l'intégrale comme
    \begin{equation}        \label{EqRPwqpve}
        -\int_Iu'\varphi_n=\int_Iu\varphi'_n=\int_a^{a-1/n}u\varphi'_n+\int_{x-1/n}^{x+1/n}u\varphi'_n+\int_{x+1/n}^{y-1/n}u\varphi'_n+\int_{y-1/n}^{y+1/n}u\varphi'_n+\int_{y+1/n}^{b}u\varphi'_n.
    \end{equation}
    Par construction de \( \varphi_n\), de ces \( 5\) morceaux, il n'en reste que deux de non nulles :
    \begin{equation}
        \int_Iu\varphi'=\underbrace{\int_{x-1/n}^{x+1/n}u(t)\varphi'_n(t)dt}_A+\underbrace{\int_{y-1/n}^{y+1/n}u(t)\varphi'_n(t)dt}_B
    \end{equation}

    Soit \( \epsilon>0\) et \( n\) suffisamment grand pour avoir \( u(t)\in B\big( u(x),\epsilon \big)\) pour tout \( t\in B(x,\frac{1}{ n })\) et (en même temps) \( u(t)\in B\big( u(y),\epsilon \big)\) pour tout \( t\in B(y,\frac{1}{ n })\). C'est la continuité de \( u\) qui permet de trouver un tel \( n\). Pour cette valeur de \( n\), en tenant compte des hypothèses sur la positivité de \( \varphi_n'\) nous avons
    \begin{equation}
        \int_{x-1/n}^{x+1/n}\big( u(x)-\epsilon \big)\varphi'_n(t)dt\leq\int_{x-1/n}^{x+1/n}u(t)\varphi'_n(t)dt\leq\int_{x-1/n}^{x+1/n}\big( u(x)+\epsilon \big)\varphi'_n(t)dt,
    \end{equation}
    mais par hypothèse sur \( \varphi_n\) nous trouvons
    \begin{equation}
        \int_{x-1/n}^{x+1/n}\varphi'_n(t)dt=\varphi_n(x+\frac{1}{ n })-\varphi(x+\frac{1}{ n })=1.
    \end{equation}
    donc
    \begin{equation}    \label{EqLYrpEdb}
        u(x)-\epsilon\leq\int_{x-1/n}^{x+1/n}u(t)\varphi'_n(t)dt\leq u(x)+\epsilon.
    \end{equation}
    Pour encadrer la seconde, il faut être plus prudent avec les signes parce que \( \varphi'_n\) y est négative. En posant \( \psi_n=-\varphi_n\) nous avons
    \begin{equation}
        -B=\int_{y-1/n}^{y+1/n}u(t)\psi_n(t)dt,
    \end{equation}
    et donc
    \begin{equation}
        u(y)-\epsilon\leq -B\leq u(y)+\epsilon
    \end{equation}
    ou encore
    \begin{equation}
        -\epsilon-u(y)\leq B\leq \epsilon-u(y).
    \end{equation}
    En additionnant avec \eqref{EqLYrpEdb} nous voyons que pour tout \( \epsilon>0\) il existe un \( N(\epsilon)\) tel que nous ayons
    \begin{equation}    \label{EqEBwWUxm}
        u(x)-u(y)-2\epsilon\leq\int_Iu'\varphi_{n}\leq u(x)-u(y)+2\epsilon
    \end{equation}
    pour tout \( n\geq N\). Nous voulons évidemment prendre la limite \( \epsilon\to 0\), c'est à dire \( n\to \infty\). Étant donné que \( \varphi_n(t)<1\) pour tout \( t\) et pour tout \( n\), la fonction \( t\mapsto u'(t)\varphi_n(t)\) est dominée par \( u'\), qui est dans \( L^1(I)\) par le lemme \ref{LemTLHwYzD}. Le théorème de la convergence dominée nous permet donc d'affirmer que
    \begin{equation}
        \lim_{n\to \infty} \int_Iu'\varphi_n=\int_Iu'\mtu_{[x,y]}=\int_x^yu',
    \end{equation}
    et donc les inégalités \eqref{EqEBwWUxm} donnent le résultat, grâce au signe dans \eqref{EqRPwqpve}.
\end{proof}

\begin{corollary}   \label{CorCEPJGAu}
    Si \( [u]\in H^1(I)\), le représentant continu \( u\in C^0(I)\) peut être prolongé par continuité en \( u\in C^0(\bar I)\).
\end{corollary}

\begin{proof}
    Soit \( (x_n)\) une suite strictement croissante dans \( \mathopen] a , b \mathclose[\) convergeant vers \( b\). Nous voulons montrer que la suite \( \big( u(x_n) \big)\) est de Cauchy dans \( \eR\), ce qui nous permettra de définir
        \begin{equation}
            u(b)=\lim_{n\to \infty} u(x_n).
        \end{equation}
        qui sera évidemment continue. Cette construction ne dépendra pas du choix de la suite \( (x_n)\) parce que deux fonctions continues sur \( \bar I\) et égales sur \( I\) sont égales sur \( \bar I\).

        En notant \( u'\) la dérivée de \( u\) dans \( H^1\), nous avons par construction du représentant continu : \( u(x)=\int_{y_0}^xu'(t)dt\). Et donc
        \begin{equation}
            \big| u(x_n)-u(x_{n+p}) \big|=\left| \int_{y_0}^{x_n}u'-\int_{y_0}^{x_{n+p}}u' \right| =\left| \int_{x_n}^{x_{n+p}}u' \right| .
        \end{equation}
        Vu que la suite \( (x_n)\) est de Cauchy et que \( u'\) est intégrable (même sur \( \bar I\)), la limite \( n\to\infty\) de cela est zéro, quelle que soit la valeur de \( p\). Donc \( \big( u(x_n) \big)\) est ce Cauchy dans \( \eR\) et est donc convergente.
\end{proof}
\index{prolongement!par continuité!dans \( H^1(I)\)}

\begin{proposition}[\cite{KXjFWKA}]     \label{ThoESIyxfU}
    Quelque propriétés de l'espace de Sobolev \( H^1(I)\) où \( I=\mathopen] a , b \mathclose[\) est un ouvert borné de \( \eR\).
    \begin{enumerate}
        \item
            \( H^1(I)\) est un espace de Hilbert.
        \item
            \( H^1(I)\) s'injecte de façon compacte dans \( C^0(\bar I)\).
        \item
            \( H^1(I)\) s'injecte de façon continue dans \( L^2(I)\).
    \end{enumerate}
\end{proposition}
\index{espace!de fonctions!Sobolev \( H^1\)}
\index{espace!de Hilbert!espace de Sobolev \( H^1\)}
\index{espace!\( L^2\)!Sobolev}
\index{dérivation!au sens des distribution!Sobolev}


\begin{proof}
    Nous prouvons point par point.
    \begin{enumerate}
        \item
            Le seul critère à vérifier est la complétude. Pour cela nous considérons une suite de Cauchy \( (u_n)\) dans \( H^1(I)\). Si \( \epsilon>0\), alors il existe \( N>0\) tel que pour tout \( p\geq 0\) nous ayons \( \| u_{n+p}-u_n \|_{H^1}^2\leq \epsilon\), c'est à dire
            \begin{equation}
                \| u_{n+p}-u_n \|^2_{L^2}+\| u'_{n+p}-u'_n \|^2_{L^2}+
            \end{equation}
            En particulier les suites \( (u_n)\) et \( (u'_n)\) sont de Cauchy dans \( L^2\) qui est complet par le théorème de Fischer-Riesz \ref{ThoGVmqOro}. Nous notons donc
            \begin{subequations}
                \begin{align}
                    u_n\stackrel{L^2}{\to}u\\
                    u'_n\stackrel{L^2}{\to}v.
                \end{align}
            \end{subequations}
            Nous allons démontrer les points suivants\quext{C'est le moment de lire l'énoncé du problème \ref{ProbTOElufz} et de m'écrire si vous avez une réponse.}
            \begin{itemize}
                \item \( u\in H^1(I)\) avec \( u'=v\).
                \item \( u_n\stackrel{H^1}{\to}u\).
            \end{itemize}
            Pour cela nous introduisons la dérivée faible de \( u\) dans \( L^2\), c'est à dire la forme linéaire continue \( \partial u\) sur \(  C^{\infty}_c(I)\) :
            \begin{equation}
                \begin{aligned}
                    \partial u\colon  C^{\infty}_c(I)&\to \eR \\
                    \varphi&\mapsto \langle \partial u, \varphi\rangle =-\int_Iu\varphi'. 
                \end{aligned}
            \end{equation}
            Pour tout \( \varphi\in C^{\infty}_c(I)\) nous avons
            \begin{subequations}
                \begin{align}
                \big| \langle \partial u, \varphi\rangle -\langle u_n', \varphi\rangle  \big|&=\left| -\int_Iu\varphi'-\int_Iu'_n\varphi \right| \\
                &=\left| -\int_Iu\varphi'-\int_Iu_n\varphi' \right| \\
            &\leq \int_I| u-u_n | |\varphi' |\\
            &\leq\| u-u_n \|_{L^2}\| \varphi' \|_{L^2}\,\text{Cauchy-Schwartz dans \( L^2\)}\\
            &\to 0.
                \end{align}
            \end{subequations}
            À la première ligne, la première intégrale est la définition de l'action de la forme \( \partial u\) sur \( \varphi\) alors que la seconde est seulement un produit scalaire dans \( L^2\). Tout deux sont notés avec les crochets. En tant que limite dans \( \eR\) nous avons
            \begin{equation}
                \lim_{n\to \infty} \langle u'_n, \varphi\rangle =\langle \partial u, \varphi\rangle .
            \end{equation}
            Dans le calcul suivant, les deux crochets sont des produits scalaires dans \( L^2\) :
            \begin{subequations}
                \begin{align}
                \big| \langle u_n', \varphi\rangle -\langle v, \varphi\rangle  \big|&=\left| -\int_Iu'_n\varphi-\int_Iv\varphi \right| \\
            &\leq \int_I| u'_n-v| |\varphi |\\
            &\leq\| u'_n-v \|_{L^2}\| \varphi \|_{L^2}\\
            &\to 0.
                \end{align}
            \end{subequations}
            Donc en tant que limite dans \( \eR\),
            \begin{equation}
                \lim_{n\to \infty} \langle u'_n, \varphi\rangle =\langle v, \varphi\rangle .
            \end{equation}
            Par unicité de la limite nous en déduisons que pour tout \( \varphi\in C^{\infty}_c(I)\),
            \begin{equation}
                \langle \partial u, \varphi\rangle =\langle v, \varphi\rangle .
            \end{equation}
            Encore une fois nous répétons qu'à gauche le crochet est l'application de la forme \( \partial u\) sur \( \varphi\) tandis qu'à droite c'est le produit scalaire dans \( L^2\). 

            Nous sommes maintenant à même de prouver que \( u\in H^1(I)\) et que sa dérivée (au sens de \( H^1\)) est \( v\). En effet
            \begin{equation}
                \int_Iu\varphi'=-\langle \partial u, \varphi\rangle =-\langle v, \varphi\rangle =-\int_Iv\varphi.
            \end{equation}
            Par conséquent nous avons \( u'=v\) dans \( H^1\) et aussi \( u'=v\) presque partout au sens des fonctions.

            Nous pouvons alors prouver que \( u_n\to u\) dans \( H^1(I)\) :
            \begin{equation}
                \| u_n-u \|^2_{H^1(I)}=\| u_n-u \|^2_{L^2}+\| u'_n-u' \|_{L^2}^2.
            \end{equation}
            Mais nous savons déjà que \( u_n\to u\) dans \( L^2\) (d'ailleurs c'est la définition de \( u\)) et que \( u'=v\) alors que par définition de \( v\), nous avons \( u'_n\to v\) dans \( L^2\). Tout cela donne que \( u_n\to u\) dans \( H^1(I)\) et donc que \( H^1(I)\) est un espace complet.

        \item

            L'application que nous allons prouver être compacte entre \( H^1(I)\) et \( C^0(\bar I)\) est
            \begin{equation}
                \begin{aligned}
                    \psi\colon H^1(I)&\to C^0(\bar I) \\
                    [u]&\mapsto \tilde u 
                \end{aligned}
            \end{equation}
            où \( [u]\) désigne une classe de fonction dans \( H^1(I)\) et \( \tilde u\) est son représentant continu prolongé par continuité à \( \bar I\)\footnote{Encore que par soucis d'économie d'encre nous n'allons pas écrire toujours les tildes et noter \( u\) le représentant continu prolongé à \( \bar I\) par le corollaire \ref{CorCEPJGAu}.}, qui existe par le lemme \ref{LemMPkbZxX} et le corollaire \ref{CorCEPJGAu}. Cette application est une injection par l'unicité du représentant continu. Nous allons prouver que c'est une application compacte en utilisant le critère \ref{ItemJIkpUbLii} de la proposition \ref{PropDGsPtpU}. Pour cela nous allons commencer par utiliser le théorème d'Ascoli sur l'ensemble \( \tilde \mB\) des représentants continus des éléments de \( \mB\), prolongés par continuité sur \( \bar I\); c'est à dire \( \tilde B\subset C^0(\bar I)\).

            Soit \( u\in \tilde \mB\); par la proposition \ref{PropGWOIoDg}, nous avons
            \begin{subequations}
                \begin{align}
                    \big| u(x)-u(y) \big|&=\big| \int_y^xu'(t)dt \big|\\
                    &=\left| \int_I\mtu_{[x,y]}(t)u'(t)dt \right| \\
                    &\leq\| \mtu_{\mathopen[ x , y \mathclose]} \|_{L^2}\| u' \|_{L^2}\\
                    &\leq\sqrt{| x-y |}\| u' \|_{H^1}\\
                    &\leq\sqrt{| x-y |}.
                \end{align}
            \end{subequations}
            où nous insistons sur le fait que la continuité n'impliquant pas la dérivabilité, le \( u'\) ici est la dérivé au sens de \( H^1\), et non la dérivée usuelle. Quoi qu'il en soit, l'ensemble \(\tilde  \mB\) est équicontinu\footnote{Définition \ref{DefUWmVBcZ}}. Nous montrons à présent qu'il est également borné pour la norme uniforme. Soit \( u\in\tilde \mB\); vu la construction du représentant continu au lemme \ref{LemMPkbZxX}, nous avons
            \begin{subequations}
                \begin{align}
                \big| u(x) \big|&=\left| \frac{1}{ b-a }\int_a^bu(x)dy \right| \\
                &=\left| \frac{1}{ b-a }\int_a^b\left( \int_y^xu'(t)dt-u(y) \right)dy \right| \\
                &=\left| \frac{1}{ b-a }\int_a\int_y^xu'(t)dtdy-\frac{1}{ b-a }\int_a^b u(y)dy \right| \\
                &\leq\frac{1}{ b-a }\int_a^b\int_a^b| u'(t) |dt\,dy+\frac{1}{ b-a }\int_a^b| u(y) |dy \label{EqCFwSOxh}.
                \end{align}
            \end{subequations}
            À ce niveau, il faut remarquer que dans la première intégrale, le passage de la valeur absolue à l'intérieur de l'intégrale en même temps que l'élargissement des bornes n'a rien d'innocent. Si \( x<y\), les bornes ne sont pas «dans le bon ordre» et nous ne pouvons pas faire la majoration usuelle en entrant simplement la valeur absolue. Ici nous tenons compte de cela en élargissant les bornes, et en les mettant dans le bon ordre. Le passage exact est le suivant : si \( x,y\in\mathopen] a , b \mathclose[\), nous avons
                \begin{equation}
                \left| \int_y^xf(t)dt \right| \leq\left| \int_y^x| f(t) |dt \right| \leq\left| \int_a^b| f(t) |dt \right| =\int_a^b| f(t) |dt.
                \end{equation}
                Notons en particulier que dans le cas du passage vers l'équation \eqref{EqCFwSOxh}, le nombre \( x\) est fixé alors que \( y\) est une variable d'intégration. Donc l'ordre des deux est certainement de temps en temps le «mauvais».

                Quoi qu'il en soit, la première intégrale se réduit à une multiplication par \( b-a\) et le calcul continue :
                \begin{subequations}
                    \begin{align}
                        \big| u(x) \big|&\leq \int_I| u'(t) |dt+\frac{1}{ b-a }\int_I| u |\\
                        &\leq \sqrt{b-a}\| u' \|_{L^2}+\frac{1}{ \sqrt{b-a} }\| u \|_{L^2}\\
                        &\leq\left( \sqrt{b-a}+\frac{1}{ \sqrt{b-a} } \right)\big( \| u' \|_{L^2}+\| u \|_{L^2} \big)\\
                        &\leq\left( \sqrt{b-a}+\frac{1}{ \sqrt{b-a} } \right) \| u \|_{H^1}\\
                        &= \sqrt{b-a}+\frac{1}{ \sqrt{b-a} }.
                    \end{align}
                \end{subequations}
                Donc \( \tilde \mB\) est borné pour la norme \( L^{\infty}\). Et c'est même borné par un nombre facilement calculable connaissant \( I\). En particulier l'ensemble
                \begin{equation}
                    \{ u(x)\tq u\in H^1 \}
                \end{equation}
                est pour, tout \( x\), contenu dans la boule de rayon \( \sqrt{a-b}+\frac{1}{ \sqrt{a-b} }\) et donc est relativement compact dans \( \eR\). Par conséquent le théorème d'Ascoli \ref{ThoKRbtpah} nous dit que l'ensemble \( \tilde B\) est relativement compact dans \( C^0(I)\).

                Par conséquent nous avons montré que l'image par \( \psi\) de la boule unité fermée \( \mB\) de \( H^1(I)\) est relativement compacte dans \( C^0(\bar I)\), ce qui signifie que \( \psi\) est une application compacte.


            \item

                Les éléments de \( H^1(I)\) sont des éléments de \( L^2(I)\); donc l'identité est une injection. Nous devons seulement étudier la continuité. Si \( (u_n)\) est une suite dans \( H^1\) convergeant dans \( H^1\) vers \( u\), alors
                \begin{equation}
                    \| u_n-u \|_{L^2}\leq\| u_n-u \|_{L^2}+\| u'_n-u' \|_{L^2}=\| u_n-u \|_{H^1}\to 0.
                \end{equation}
                Donc la suite des images (par l'identité) converge dans \( L^2\). L'identité est donc continue.

    \end{enumerate}
    
\end{proof}


\begin{probleme}    \label{ProbTOElufz}
    Au point de la preuve auquel vous devriez être si vous lisez ceci, vous pourriez avoir envie de démontrer \( u'=v\) de la façon suivante :
    \begin{equation}
        \int_I u\varphi'=\lim_{n\to \infty} \int_Iu_n\varphi=-\lim_{n\to \infty} \int_Iu'_n\varphi=-\int_Iv\varphi.
    \end{equation}
    J'avoue ne pas trouver d'exemples pour lesquels ça ne marche pas. Est-ce qu'on peut inverser la limite et l'intégrale dans \( L^2\) ?

    Ceci n'invalide pas la preuve donnée, mais ça suggère un sacré raccourcis.
\end{probleme}

%+++++++++++++++++++++++++++++++++++++++++++++++++++++++++++++++++++++++++++++++++++++++++++++++++++++++++++++++++++++++++++ 
\section{Espaces de Schwartz}
%+++++++++++++++++++++++++++++++++++++++++++++++++++++++++++++++++++++++++++++++++++++++++++++++++++++++++++++++++++++++++++

Pour un multiindice \( \alpha=(\alpha_1,\ldots, \alpha_d)\in \eN^d\), nous notons
\begin{equation}
    \partial^{\alpha}\varphi=\partial_{x_1}^{\alpha_1}\ldots\partial_{x_d}^{\alpha_d}\varphi
\end{equation}
pour peu que la fonction \( \varphi\) soit \( | \alpha |=\alpha_1+\ldots +\alpha_d\) fois dérivable.

\begin{definition}
    L'\defe{espace de Schwartz}{espace!de Schwartz} est l'ensemble des fonctions infiniment dérivables dont toutes les dérivées décroissent plus vite que tout polynôme :
    \begin{equation}
        \swS(\eR^d)=\big\{   \varphi\in C^{\infty}(\eR^d)\tq\forall \alpha,\beta\in \eN^d, p_{\alpha,\beta}(\varphi)<\infty   \big\}
    \end{equation}
    où nous avons considéré
    \begin{equation}    \label{EqOWdChCu}
        p_{\alpha,\beta}(\varphi)=\sup_{x\in \eR^d}| x^{\beta}(\partial^{\alpha}\varphi)(x) |=\| x^{\beta}\partial^{\alpha}\varphi \|_{\infty}.
    \end{equation}
    Une fonction \( \varphi\in\swS(\eR^d)\) est dite à \defe{décroissance rapide}{fonction!à décroissance rapide}.
\end{definition}

Pour simplifier les notations (surtout du côté de Fourier), nous allons parfois écrire \( M_i\varphi\)\nomenclature[Y]{\( M_i\varphi\)}{La fonction \( x\mapsto x_i\varphi(x)\)} pour la fonction \( x\mapsto x_i\varphi(x)\).

\begin{example}
    La fonction \(  e^{-x^2}\) est une fonction à décroissance rapide sur \( \eR\).
\end{example}

\begin{proposition} \label{PropCSmzwGv}
    Une fonction à décroissance rapide décrois plus vite que n'importe quel polynôme\footnote{D'où le nom des fonctions à décroissance rapide.}. Plus précisément, si \( \varphi\in\swS(\eR^d)\), pour tout polynôme \( Q\), il existe un \( r>0\) tel que \(  | \varphi(x) |<\frac{1}{ | Q(x) | } \) pour tout \( \| x \|\geq r\).
\end{proposition}

\begin{proof}
    Nous commençons par considérer un polynôme \( P\) donné par
    \begin{equation}
        P(x)=\sum_kc_kx^{\beta_k}
    \end{equation}
    où les \( \beta_k\) sont des multiindices, les \( c_k\) sont des constantes et la somme est finie. Nous avons la majoration
    \begin{equation}
        \sup_{x\in \eR^d}| \varphi(x)P(x) |\leq\sum_k\sup_x\big| c_k\varphi(x) x^{\beta_k} \big|\leq\sum_k| c_k |p_{0,\beta_k}(\varphi)<\infty.
    \end{equation}
    Nous allons noter \( M_P\) la constante \( \sum_k| c_k |p_{0,\beta_k}(\varphi)\), de sorte que pour tout \( x\in \eR^d\) nous ayons \( | \varphi(x)P(x) |\leq M_P\) et donc
    \begin{equation}
        | \varphi(x) |\leq \frac{ M_P }{ | P(x) | }=\frac{1}{ | \frac{1}{ M_P }P(x) | }.
    \end{equation}
    Notons que cette inégalité est a fortiori correcte pour les \( x\) sur lesquels \( P\) s'annule.

    Soit maintenant un polynôme \( Q\). Nous considérons le polynôme \( P(x)=\| x \|Q(x)\). Étant de plus haut degré, pour toute constante \( C\) il existe un rayon \( r_C\) tel que \( | P(x) |\geq C| Q(x) |\) pour tout \( | x |\geq r_C\). En particulier pour \( | x |\geq r_{M_P}\) nous avons
    \begin{equation}
        | P(x) |\geq M_P| Q(x) |
    \end{equation}
    et donc, pour ces \( x\), 
    \begin{equation}
        | \varphi(x) |\leq \frac{1}{ | \frac{1}{ M_P }P(x) | }\leq \frac{1}{ | Q(x) | }.
    \end{equation}
    La première inégalité est valable pour tout \( x\), et la seconde pour \( \| x \|\geq r_{M_P}\).
\end{proof}

%--------------------------------------------------------------------------------------------------------------------------- 
\subsection{Topologie}
%---------------------------------------------------------------------------------------------------------------------------

\begin{definition}
    Si \( E\) est un espace vectoriel, une \defe{semi-norme}{semi-norme} sur \( E\) est une application \( p\colon E\to \eR\) telle que
    \begin{enumerate}
        \item
            \( p(x)\geq 0\),
        \item
            \( p(\lambda x)=| \lambda |p(x)\)
        \item
            \( p(x+y)\leq p(x)+p(y)\).
    \end{enumerate}
\end{definition}

\begin{lemma}
    Les \( p_{\alpha,\beta}\) donnés par l'équation \eqref{EqOWdChCu} ci-dessus sont des semi-normes.
\end{lemma}
%TODO : une preuve pour égayer la galerie.

Soit une famille \( (p_i)_{i\in I}\) de semi-normes sur \( E\). Nous en déduisons une topologie sur \( E\) de la façon suivante\cite{SOdaAdx}. Pour tout \( J\) fini dans \( I\) nous définissons les boules ouvertes
\begin{equation}
    B_J(x,r)=\{ y\in E\tq p_j(y-x)<r\,\forall j\in J \}.
\end{equation}

La topologie sur \( E\) donnée par la famille de semi-norme est celle engendrée par les boules ainsi définies.

\begin{proposition}
    Une suite \( (x_n)\) dans \( E\) converge vers \( x\) au sens de la topologie des semi-normes si et seulement si pour tout \( i\in I\),
    \begin{equation}
        p_i(x-x_n)\to 0.
    \end{equation}
\end{proposition}

\begin{proof}
    Si la suite \( (x_n)\) converge vers \( x\), alors pour tout ouvert \( \mO\) autour de \( x\), il existe un \( N\) tel que si \( n\geq N\), alors \( x_n\in\mO\). En particulier pour tout \( j\) et pour tout \( \epsilon>0\), il doit exister un \( n\geq N_j\) tel que \( x_n\in B_j(x,\epsilon)\).

    Voyons l'implication inverse. Soit \( \epsilon>0\). Pour tout \( i\in I\), il existe un \( N_i\) tel que \( n\geq N_i\) implique \( p_i(x-x_n)\leq \epsilon\). Si \( \mO\) est un ouvert, il doit contenir une boule du type \( B_J(x,r)\) pour un certain ensemble fini \( J\subset I\).

    En prenant \( N=\max\{ N_j\tq j\in J \}\), nous avons \( p_j(x-x_n)\leq \epsilon\) pour tout \( j\) et donc \( x_n\in B_J(x,r)\).
\end{proof}

\begin{lemma}[\cite{OEVAuEz}]   \label{LemRJhCbkO}
    La topologie sur \( \swS(\eR^d)\) est donnée aussi par les semi-normes
    \begin{equation}
        q_{n,m}=\max_{| \alpha |\leq n}\sup_{x\in \eR^d}\big( 1+\| x \| \big)^m\big| \partial^{\alpha}\varphi(x) \big|.
    \end{equation}
    Autrement dit, une suite \( \varphi_n\stackrel{\swS(\eR^d)}{\to}0\) si et seulement si \( q_{n,m(\varphi)}\to 0\) pour tout \( n\) et \( m\).
\end{lemma}
Le fait que les \( q_{n,m}(\varphi)\) restent bornés est la proposition \ref{PropCSmzwGv}. Cependant ce lemme est plus précis parce qu'en disant seulement que \( \varphi\) est majoré par des polynôme, nous ne disons pas que les polynômes correspondants aux \( \varphi_n\) tendent vers zéro si \( \varphi_n\stackrel{\swS}{\to}0\). Et d'ailleurs on ne sait pas très bien ce que signifierait \( P_n\to 0\) pour une suite de polynômes.

\begin{proposition}     \label{PropGNXBeME}
    Pour \( p\in\mathopen[ 1 , \infty \mathclose]\), l'espace \( \swS(\eR^d)\) s'injecte continument dans \( L^p(\eR^d)\). 
\end{proposition}

\begin{proof}
    L'injection dont nous parlons est l'identité ou plus précisément l'identité suivie de la prise de classe. Il faut vérifier que cela est correct et continu, c'est à dire d'abord qu'une fonction à décroissance rapide est bien dans \( L^p\) et ensuite que si \( f_n\stackrel{\swS}{\to}0\), alors \( f_n\stackrel{L^p}{\to}0\).
    
    Commençons par \( p=\infty\). Alors \( \| f_n \|_{\infty}=p_{0,0}(f_n)\to 0\) parce que si \( f_n\stackrel{\swS}{\to}0\), alors en particulier \( p_{0,0}(f_n)\to 0\).

    Au tour de \( p<\infty\) maintenant. Nous savons qu'en dimension \( d\), la fonction
    \begin{equation}
        x\mapsto \frac{1}{ (1+\| x \|)^s }
    \end{equation}
    est intégrable dès que \( s>d\).
    %TODO : il faudrait une petite preuve de ça.
    Pour toute valeur de \( m\) nous avons
    \begin{equation}
        \| \varphi \|_p^p=\int_{\eR^d}| \varphi(x) |^pdx=\int_{\eR^d}\frac{ \big|    (1+\| x \|)^m\varphi(x)   \big|^p }{ \big( 1+\| x \| \big)^{mp} }\leq\int_{\eR^d}\frac{q_{0,m}(\varphi)^p}{ \big( 1+\| x \| \big)^{mp} }.
    \end{equation}
    En choisissant \( m\) de telle sorte que \( mp>d\), nous avons convergence de l'intégrale et donc \( \| \varphi \|_p<\infty\). Nous retenons que
    \begin{equation}    \label{EqVWfEFMk}
        \| \varphi \|_p^p\leq Cq_{0,m}(\varphi)^p
    \end{equation}
    pour une certaine constance \( C\) et un bon choix de \( m\).

    Ceci prouve que \( \swS(\eR^d)\subset L^p(\eR^d)\). Nous devons encore vérifier que l'inclusion est continue. Si \( \varphi_n\stackrel{\swS}{\to}0\), alors en particulier nous avons \( q_{0,m}(\varphi_n)\to 0\) par le lemme \ref{LemRJhCbkO}. Par conséquent la majoration \eqref{EqVWfEFMk} nous dit que \( \| \varphi_n \|_p\to 0\) également.

\end{proof}
En résumé, si \( \varphi_n\stackrel{\swS(\eR^d)}{\to}\varphi\) alors \( \varphi_n\stackrel{L^p}{\to}\varphi\).

%--------------------------------------------------------------------------------------------------------------------------- 
\subsection{Produit de convolution}
%---------------------------------------------------------------------------------------------------------------------------

\begin{proposition}[\cite{CXCQJIt}]
    Si \( \varphi\in L^1(\eR)\) et \( \psi\in\swS(\eR)\), alors \( \varphi * \psi\in \swS(\eR)\).
\end{proposition}

\begin{proof}
    Nous devons prouver que
    \begin{equation}
        p_{\alpha,\beta}(\varphi*\psi)=\sup_{x\in \eR^d}| x^{\beta}(\partial^{\alpha}(\varphi*\psi))(x) |
    \end{equation}
    est borné pour tout multiindices \( \alpha\) et \( \beta\). En appliquant \( | \alpha |\) fois la proposition \ref{PropHNbdMQe}, nous mettons toutes les dérivées sur \( \psi\) : \( \partial^{\alpha}(\varphi*\psi)=(\varphi*\partial^{\alpha}\psi)\). Cela étant fait, nous majorons
    \begin{subequations}
        \begin{align}
            \big| x^{\beta}(\varphi*\partial^{\alpha}\psi)(x) \big|&\leq| x^{\beta} |\int_{\eR^d} |\varphi(y)|\underbrace{\big| (\partial^{\alpha}\psi)(x-y)\big|}_{\leq\| \partial^{\alpha}\psi \|_{\infty}} dy \big|\\
            &\leq | x^{\beta} |  \| \partial^{\alpha}\psi \|_{\infty}\int_{\eR^d}| \varphi(y) |dy\\
            &\leq p_{\alpha,\beta}(\psi)\| \varphi \|_{_{L^1}}.
        \end{align}
    \end{subequations}
    Par conséquent, \( p_{\alpha,\beta}(\varphi*\psi)\leq \| \varphi \|_{L^1}p_{\alpha,\beta}(\psi)<\infty\).
\end{proof}

%--------------------------------------------------------------------------------------------------------------------------- 
\subsection{Transformée de Fourier}
%---------------------------------------------------------------------------------------------------------------------------

La définition de la transformée de Fourier de \( \varphi\in\swS(\eR^d)\) est 
\begin{equation}
    \hat  \varphi(\xi)=\int_{\eR^n}\varphi(x) e^{-ix\cdot \xi}.
\end{equation}

\begin{lemma}   \label{LemQPVQjCx}
    Si \( \varphi\in\swS(\eR^d)\) et si \( \alpha\) est un multiindice, alors
    \begin{equation}
        \partial^{\alpha}\hat\varphi=(-i)^{| \alpha |}\widehat{M_{\alpha}\varphi}.
    \end{equation}
    et
    \begin{equation}
        \widehat{\partial^{\alpha}\varphi}(\xi)=(-i)^{| \alpha |}\xi^{\alpha}\hat\varphi(\xi).
    \end{equation}
\end{lemma}

\begin{proof}
    Nous considérons la fonction \( h(x,\xi)=\varphi(x) e^{-ix\cdot \xi}\) dont la dérivée par rapport à \( \xi_i\) est donnée par \( -i(M_{i}\varphi)(x) e^{x\cdot \xi}\). Cette fonction est majorée en norme par
    \begin{equation}
        G(x)=M_i\varphi(x),
    \end{equation}
    qui est encore une fonction à décroissance rapide et donc parfaitement intégrable sur \( \eR^d\). Le théorème \ref{ThoMWpRKYp} nous dit donc que la dérivée de \( \hat \varphi\) par rapport à \( \xi_i\) existe et vaut
    \begin{equation}
        \frac{ \partial \hat\varphi }{ \partial \xi_i }(\xi)=-i\int_{\eR^n}x_i\varphi(x) e^{-i\xi\cdot x}=-i\widehat{M_i\varphi}(\xi).
    \end{equation}
    En appliquant ce résultat en chaîne, nous trouvons la première formule annoncée.

    Nous passons à la seconde formule annoncée. Étant donné que \( \varphi\in\swS\), ses dérivées le sont aussi et par conséquent, il n'y a pas de problèmes pour écrire
    \begin{equation}    \label{EqTYizlnia}
        \widehat{\partial_{x_k}\varphi}(\xi)=\int_{\eR^d}\frac{ \partial \varphi }{ \partial x_k }(x) e^{-ix\cdot \xi}dx.
    \end{equation}
    Étant donné que
    \begin{equation}    \label{EqZAeYaCB}
        \frac{ \partial  }{ \partial x_k }\left( \varphi(x) e^{-ix\cdot\xi} \right)=\frac{ \partial \varphi }{ \partial x_k }(x) e^{-ix\cdot\xi}-i\xi_k\varphi(x) e^{-ix\cdot \xi},
    \end{equation}
    notre tâche sera de prouver que
    \begin{equation}    \label{EqVGvYBNK}
        \int_{\eR^d}\frac{ \partial  }{ \partial x_k }\left( \varphi(x) e^{-ix\cdot \xi} \right)dx=0.
    \end{equation}
    Autrement dit, nous voulons montrer que le terme au bord d'une intégration par partie s'annule. D'abord le fait que \( \varphi\) soit à décroissance rapide nous assure que l'intégrale \eqref{EqVGvYBNK} converge. Pour chaque \( \xi\), la fonction
    \begin{equation}
        f(x,\xi)=\frac{ \partial x_k }{ \partial  }\left( \varphi(x) e^{-ix\cdot \xi} \right)
    \end{equation}
    est intégrable par rapport à \( x\). De plus, \( f\) est dans \( \swS(\eR)\) pour chacune de ses variables (les autres étant fixées). Le théorème de Fubini \ref{ThoFubinioYLtPI} nous permet alors de décomposer l'intégrale en
    \begin{equation}
        \int_{\eR^d}f(x,\xi)dx=\int_{\eR}\ldots\int_{\eR} f(x_1,\ldots, x_d)dx_1\ldots dx_d.
    \end{equation}
    De plus nous pouvons intégrer dans l'ordre de notre choix et nous choisissons évidemment d'intégrer d'abord par rapport à \( x_k\).  Étudions donc l'intégrale
    \begin{equation}
        \int_{\eR}\frac{ \partial  }{ \partial x }\left( \varphi(x) e^{-ix\xi} \right)dx=\lim_{A\to\infty}\int_{-A}^A\frac{ \partial  }{ \partial x }\left( \varphi(x) e^{-ix\xi} \right)dx
    \end{equation}
    dans laquelle nous avons un peu allégé les notations. Une primitive de ce qui est intégré est toute trouvée : c'est \( \varphi(x) e^{-ix\xi}\), et nous pouvons utiliser le théorème fondamental du calcul intégral pour écrire que
    \begin{equation}
        \int_{-A}^A\left( \varphi(x) e^{-ix\xi} \right)'dx=\left[ \varphi(x) e^{-ix\xi} \right]_{x=-A}^{x=A}.
    \end{equation}
    Vu que \( \varphi\) est dans \( \swS\), la limite \( A\to\infty\) donne zéro.

    En substituant maintenant \eqref{EqZAeYaCB} dans \eqref{EqTYizlnia} et en tenant compte du terme que nous venons de montrer s'annuler, nous avons
    \begin{equation}
        \widehat{\partial_k\varphi}(\xi)=-i\xi_k\int_{\eR^d}\varphi(x) e^{-ix\cdot \xi}=-i\xi_k\hat\varphi(\xi).
    \end{equation}
    En recommençant la procédure \( | \alpha |\) fois nous trouvons la seconde formule annoncée.
\end{proof}


\begin{proposition}
    L'espace de Schwartz est stable par transformée de Fourier. De plus l'application
    \begin{equation}
        \TF\colon \swS(\eR^d)\to \swS(\eR^d)
    \end{equation}
    est linéaire et continue.
\end{proposition}

\begin{proof}
    La linéarité découle de celle de l'intégrale. La difficulté est de prouver que pour \( \varphi\in\swS(\eR^d)\) nous avons bien que \( \hat\varphi\in\swS(\eR^d)\) et que cette association est continue\footnote{Pour rappel, en dimension infinie, il n'est pas garanti qu'une application linéaire soit continue.}.
    \begin{subproof}
        \item[Stabilité]
            Nous devons prouver que pour tout multiindices \( \alpha\) et \( \beta\), nous avons \( p_{\alpha,\beta}(\hat\varphi)<\infty\). Nous avons
            \begin{equation}
                \xi^{\beta}\partial^{\alpha}\hat\varphi(\xi)=\xi^{\beta}(-i)^{| \alpha |}\widehat{M_{\alpha}\varphi}(\xi)=(-i)^{| \alpha |+| \beta |}\widehat{\partial^{\beta}M_{\alpha}\varphi}(\xi).
            \end{equation}
            Ensuite nous nous souvenons que \( \| \hat f \|_{\infty}\leq \| f \|_1\) parce que
            \begin{equation}
                | \hat f(\xi) |\leq\int_{\eR^d}\big| f(x) e^{-ix\cdot \xi} \big|=\int_{\eR^d}| f(x) |dx=\| f \|_1.
            \end{equation}
            Donc 
            \begin{equation}
                p_{\alpha,\beta}(\hat\varphi)=\| \widehat{\partial^{\beta}M_{\alpha}\varphi} \|_{\infty}\leq \| \partial^{\beta}M_{\alpha}\varphi \|_1.
            \end{equation}
            Du fait que \( \varphi\) soit dans \( \swS\), la dernière expression est finie. Cela prouve déjà que
            \begin{equation}
                \TF\big( \swS(\eR^d) \big)\subset\swS(\eR^d).
            \end{equation}
            
        \item[Continuité]

            Nous supposons avoir une suite \( \varphi_n\stackrel{\swS}{\to}\varphi\), et nous devons prouver que \( \hat\varphi_n\stackrel{\swS}{\to}\hat\varphi\). Pour alléger les notations, nous posons \( f_n=\varphi_n-\varphi\). Nous avons
            \begin{subequations}    \label{subEqsSGsGGih}
                \begin{align}
                    \| \hat f \|_{\alpha,\beta}&=\| \xi^{\beta}\partial^{\alpha}\hat f \|_{\infty}\\
                    &=\| \widehat{  \partial^{\beta}M_{\alpha}f  } \|_{\infty}\,\text{lemme \ref{LemQPVQjCx}.}\\
                    &\leq \| \partial^{\beta}M_{\alpha}f \|_1
                \end{align}
            \end{subequations}
            La convergence \(f_n\stackrel{\swS}{\to}0\) nous dit ente autres que \( \partial^{\beta}M_{\alpha}f_n\stackrel{\swS}{\to}0\); en particulier la proposition \ref{PropGNXBeME} nous dit que \( \partial^{\beta}M_{\alpha}f_n\stackrel{L^1}{\to}0\), ce qui signifie, par les majorations \eqref{subEqsSGsGGih} que
            \begin{equation}
                \| \hat f_n \|_{\alpha,\beta}\leq \| \partial^{\beta}M_{\alpha}f_n \|_1\to0,
            \end{equation}
            ce qui prouve la continuité de transformée de Fourier dans \( \swS(\eR^d)\).
    \end{subproof}
    
\end{proof}

\begin{theorem}
    Nous avons la formule d'inversion
    \begin{equation}
        f(x)=\frac{1}{ (2\pi)^n }\int_{\eR^n}\hat f(\xi) e^{ix\cdot \xi}d\xi.
    \end{equation}
\end{theorem}
%TODO : à préciser

%+++++++++++++++++++++++++++++++++++++++++++++++++++++++++++++++++++++++++++++++++++++++++++++++++++++++++++++++++++++++++++ 
\section{Distributions tempérées}
%+++++++++++++++++++++++++++++++++++++++++++++++++++++++++++++++++++++++++++++++++++++++++++++++++++++++++++++++++++++++++++

\begin{definition}
    Une \defe{distribution tempérée}{distribution!tempérée} est une forme linéaire continue sur \( \swS(\eR^d)\). L'ensemble des distributions tempérées est noté \( \swS'(\eR^d)\)\nomenclature[Y]{\( \swS'(\eR^d)\)}{espace des distributions tempérées}. Si \( T\) est une telle distribution, nous notons $\langle T, \varphi\rangle$ l'image de \( \varphi\) par \( T\).
\end{definition}

Si \( f\) est une fonction sur \( \eR^d\) telle que \( f\varphi\in L^1(\eR^d)\) pour tout \( \varphi\in \swS(\eR^d)\), alors nous définissons la distribution \( T_f\in\swS'(\eR^d)\) par
\begin{equation}
    \langle T_f, \varphi\rangle =\int_{\eR^d}f(x)\varphi(x)dx.
\end{equation}
Cette définition ne fonctionne pas pour toute les fonctions. Par exemple pour \( f(x)= e^{x^2}\), et \( \varphi(x)= e^{-x^2}\in\swS(\eR)\) nous avons \( f\varphi=1\) qui n'est pas du tout intégrable sur \( \eR\).

\begin{example}
    La \defe{distribution de Dirac}{distribution!de Dirac} \( \delta\) est donnée par
    \begin{equation}
        \langle \delta, \varphi\rangle =\varphi(0).
    \end{equation}
    Montrons qu'elle est continue. Soit une suite \( \varphi_n\stackrel{\swS}{\to}0\). En particulier, \( p_{0,0}(\varphi_n)=\sup_x| \varphi_n(x) |\to 0\). Donc \( \varphi_n(0)\to 0\) comme il le faut.
\end{example}

\begin{example}
    La \defe{valeur principale}{valeur!principale (distribution)} de la fonction \( x\mapsto \frac{1}{ x }\) est la distribution
    \begin{equation}
        \begin{aligned}
            T\colon \swS(\eR)&\to \eR \\
            \varphi&\mapsto \lim_{\substack{\epsilon\to 0\\\epsilon>0}}\int_{| x |>\epsilon}\frac{ \varphi(x) }{ x }.
        \end{aligned}
    \end{equation}
    Montrons que cela définit bien une distribution tempérée.

    D'abord l'intégrale existe pour tout \( \epsilon\), par exemple parce que pour les grands \( | x |\) nous avons par exemple \( | \varphi(x)\leq x^3 |\) et donc \( \varphi(x)/x\leq 1/x^2\) dont l'intégrale converge. Nous devons maintenant regarder la limite.

    Nous considérons une suite \( \epsilon_n\to 0\) et la suite
    \begin{equation}
        a_n=\int_{| x |\geq \epsilon_n}\frac{ \varphi(x) }{ x }dx.
    \end{equation}
    Nous montrons que cette suite converge dans \( \eR\) en montrant qu'elle est de Cauchy. Pour cela nous travaillons un peu la forme de \( \varphi\) :
    \begin{equation}
        \varphi(x)=\varphi(0)+\int_0^x\varphi'(t)dt=\varphi(0)+\int_0^1x\varphi'(x\theta)d\theta.
    \end{equation}
    Ce qui est dans l'intégrale est borné par \( K=\| M_x\varphi' \|_{\infty}\) qui est parfaitement fini parce que \( \varphi\) est à décroissance rapide. Lorsque nous calculons \( | a_m-a_n |\), le terme \( \varphi(0)/x\) donne une intégrale nulle parce que le domaine d'intégration \( \epsilon_n\leq | x |\leq \epsilon_n\) est symétrique alors que la fonction \( 1/x\) est impaire.
    \begin{equation}
        | a_m-a_n |\leq \big| \int_{\epsilon_m<| x |<\epsilon_n}K \big|=2| \epsilon_n-\epsilon_m |K
    \end{equation}
    Tout cela nous dit que \( T\) est bien définie. Nous devons encore étudier sa continuité.

    Soit \( \chi\) une fonction dans \(  C^{\infty}_c(\eR)\) telle valant \( 1\) sur \( \mathopen[ -1 , 1 \mathclose]\), paire et à valeurs dans \( \mathopen[ 0 , 1 \mathclose]\).
    %TODO : il faudrait montrer qu'il existe des fonctions C infini à support compact qui ne sont pas nulles partout. C'est fait autour du lemme de Borel.
    Pour tout \( \epsilon>0\) nous avons \( \int_{| x |>\epsilon}\frac{ \chi(x) }{ x }dx=0\). 

    Nous avons aussi \( \varphi=\chi\varphi+(1-\chi)\varphi\), et donc
    \begin{subequations}
        \begin{align}
            \int_{| x |>\epsilon}\frac{ \varphi(x) }{ x }dx&=\int_{| \epsilon |>0}\chi(x)\frac{ \varphi(x)-\varphi(0) }{ x }dx+\int_{| \epsilon |>0}\big( 1-\chi(x) \big)\frac{ \varphi(x) }{ x }dx\\
            &=\int_{| \epsilon |>0}\chi(x)\int_0^1\underbrace{\varphi'(\theta x)}_{\leq \| \varphi' \|_{\infty}}d\theta+\int_{| x |\geq 1}\big( 1-\chi(x) \big)\frac{ \varphi(x) }{ x }dx\\
            &\leq\| \varphi' \|_{\infty}\int_{| x |\geq \epsilon}\chi(x)dx+\| \varphi \|_{L^1}\\
            &=C\| \varphi' \|_{\infty}+\| \varphi \|_{1}.
        \end{align}
    \end{subequations}
    Cela est valable pour toute fonction \( \varphi\in\swS(\eR)\). Mais nous savons que si \( \varphi_n\stackrel{\swS(\eR)}{\to}0\), alors \( \| \varphi_n \|_{\infty}\to 0\), \( \| \varphi'_n \|_{\infty}\to 0\) et \( \| \varphi_n \|_1\to 0\); donc si \( \varphi_n\stackrel{\swS(\eR)}{\to}0\), alors
    \begin{equation}
        T(\varphi_n)=\lim_{\substack{\epsilon\to 0\\\epsilon>0}}\int_{| x |>\epsilon}\frac{ \varphi(x) }{ x }\leq C\| \varphi_n' \|_{\infty}+\| \varphi_n \|_1\to 0.
    \end{equation}
\end{example}
