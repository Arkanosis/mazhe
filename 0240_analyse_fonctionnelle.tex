% This is part of Mes notes de mathématique
% Copyright (c) 2011-2014
%   Laurent Claessens
% See the file fdl-1.3.txt for copying conditions.

Bonjour. Si vous êtes ici, c'est sans doute que vous cherchez le résultat qui dit que si \( \int f=0\) c'est que \( f=0\) dans un sens ou dans un autre. 
\begin{enumerate}
    \item
        Il y a le lemme \ref{Lemfobnwt} qui dit ça.
    \item
        Un lemme du genre dans \( L^2\) existe aussi pour \( \int f\varphi=0\) pour tout \( \varphi\). C'est le lemme \ref{LemDQEKNNf}. 
    \item
        Et encore un pour \( L^p\) dans la proposition \ref{PropUKLZZZh}.
    \item
        Si \( \int f\chi=0\) pour tout \( \chi\) à support compact alors \( f=0\) presque partout, proposition \ref{PropAAjSURG}.
\end{enumerate}

\begin{proposition} \label{PropAAjSURG}
    Soit \( f\in L^1(\eR)\) une fonction telle que
    \begin{equation}
        \int_{\eR}f(t)\chi(t)dt=0
    \end{equation}
    pour toute fonction \( \chi\in\swD(\eR)\). Alors \( f=0\) presque partout.
\end{proposition}
% TODO : un énoncé plus précis et une preuve. Notons que c'est utilisé à la marque 13107277

\begin{theorem}[Théorème d'isomorphisme de Banach]  \label{ThofQShsw}
    Une application linéaire continue et bijective entre deux espaces de Banach est un homéomorphisme.
\end{theorem}
\index{théorème!isomorphisme de Banach}
% TODO : une preuve.

%+++++++++++++++++++++++++++++++++++++++++++++++++++++++++++++++++++++++++++++++++++++++++++++++++++++++++++++++++++++++++++ 
\section{Théorème d'Ascoli}
%+++++++++++++++++++++++++++++++++++++++++++++++++++++++++++++++++++++++++++++++++++++++++++++++++++++++++++++++++++++++++++

\begin{definition}
    Une partie \( A\) d'un espace topologique \( X\) est \defe{relativement compacte}{compact!relatif}\index{relativement!compact} dans \( X\) si sa fermeture est compacte.
\end{definition}

\begin{proposition}[\cite{JIFGuct}] \label{PropDGsPtpU}
    Soient \( E\) et \( F\) deux espaces vectoriels normés sur \( \eR\) ou \( \eC\) et une application \( f\in\aL(E,F)\). Les propriétés suivantes sont équivalentes.
    \begin{enumerate}
        \item
            L'image d'un borné de \( E\) par \( f\) est relativement compact dans \( F\).
        \item   \label{ItemJIkpUbLii}
            L'image par \( f\) de la boule unité fermée est relativement compacte dans \( F\).
        \item
            Si \( (x_n)\) est une suite bornée dans \( E\), alors nous pouvons en extraire une sous-suite \( (x_{\varphi(n)})\) telle que \( \big( fx_{\varphi(n)} \big)\) converge dans \( F\).
    \end{enumerate}
\end{proposition}

\begin{definition}
    Une application vérifiant les conditions équivalentes de la proposition \ref{PropDGsPtpU} est dite \defe{compacte}{compact!opérateur}.
\end{definition}

\begin{definition}  \label{DefUWmVBcZ}
    Soit \( (f_i)_{i\in I}\) une famille de fonctions \( f_i\colon X\to Y\) entre espaces métriques. Cette famille est \defe{équicontinue}{equicontinuite@équicontinuité} si pour tout \( \epsilon>0\) et pour tout \( x\in X\), il existe un \( \delta(x,\epsilon)>0\) tel que 
    \begin{equation}
        \| x-y \|_X\leq \delta\,\Rightarrow\,\| f_i(x)-f_i(y) \|_Y\leq \epsilon
    \end{equation}
    pour tout \( i\in I\).
\end{definition}

\begin{theorem}[Théorème d'Ascoli\cite{LBLADXV}]        \label{ThoKRbtpah}
    Soit \( K\) un espace topologique compact et un espace métrique \( (E,d)\). Nous considérons la topologie uniforme sur \( C(K,E)\). Une partie \( A\) de \( C(K,E)\) est relativement compacte si et seulement si les deux conditions suivantes sont remplies :
    \begin{enumerate}
        \item
            \( A\) est équicontinu,
        \item
            \( \forall x\in K\), l'ensemble \( \{ f(x)\tq f\in A \}\) est relativement compact dans \( E\).
    \end{enumerate}
\end{theorem}
\index{théorème!Ascoli}
%TODO : une preuve est sur Wikipédia.

%+++++++++++++++++++++++++++++++++++++++++++++++++++++++++++++++++++++++++++++++++++++++++++++++++++++++++++++++++++++++++++ 
\section{Théorème de Banach-Steinhaus}
%+++++++++++++++++++++++++++++++++++++++++++++++++++++++++++++++++++++++++++++++++++++++++++++++++++++++++++++++++++++++++++

\begin{theorem}[Théorème de Banach-Steinhaus\cite{KXjFWKA,VPvwAaQ}] \label{ThoPFBMHBN}
    Soit \( E\) un espace de Banach\footnote{Définition \ref{DefVKuyYpQ}.} et \( F\) un espace vectoriel normé. Nous considérons une partie \( H\subset \aL_c(E,F)\) (espace des fonctions linéaires continues). Alors \( H\) est uniformément borné si et seulement si il est simplement borné.
\end{theorem}
\index{théorème!Banach-Steinhaus}
\index{application!linéaire!théorème de Banach-Steinhaus}

\begin{proof}
    
    Si \( H\) est uniformément borné, il est borné; pas besoin de rester longtemps sur ce sens de l'équivalence. Supposons donc que \( H\) soit borné. Pour chaque \( k\in \eN^*\) nous considérons l'ensemble
    \begin{equation}
        \Omega_k=\{ x\in E\tq \sup_{f\in H}\| f(x) \|>k \}.
    \end{equation}
    
    \begin{subproof}
        \item[Les \( \Omega_k\) sont ouverts]
            
            Soit \( x_0\in \Omega_k\); nous avons alors une fonction \( f\in H\) telle que \(  \| f(x_0) \|>k \), et par continuité de \( f\) il existe \( \rho>0\) tel que \( \| f(x) \|>k\) pour tout \( x\in B(x_0,\rho)\). Par conséquent \( B(x_0,\rho)\subset \Omega_k\) et \( \Omega_k\) est ouvert par le théorème \ref{ThoPartieOUvpartouv}.

        \item[Les \( \Omega_k\) ne sont pas tous denses dans \( E\)]

            Nous supposons que les ensembles \( \Omega_k\) soient tous dense dans \( E\). Le théorème de Baire \ref{ThoBBIljNM} nous indique que \( E\) est un espace de Baire (parce que de Banach) et donc que 
            \begin{equation}
                \overline{ \bigcap_{k\in \eN}\Omega_k }=E.
            \end{equation}
            En particulier l'intersection des \( \Omega_k\) n'est pas vide. Soit \( x_0\in \bigcap_{k\in \eN}\Omega_k\). Nous avons alors
            \begin{equation}
                \sup_{f\in H}\| f(x) \|=\infty,
            \end{equation}
            ce qui est contraire à l'hypothèse. Donc les ouverts \( \Omega_k\) ne sont pas tous denses dans \(E\).
            
        \item[La majoration]

            Il existe \( k\geq 0\) tel que \( \Omega_k\) ne soit pas dense dans \( E\), et nous voulons prouver que \( \{ \| f \|\tq f\in H \}\) est un ensemble borné. Soit donc \( k\geq 0\) tel que \( \Omega_k\) ne soit pas dense dans \( E\); il existe un \( x_0\in E\) et \( \rho>0\) tels que 
            \begin{equation}
                B(x_0,\rho)\cap \Omega_k=\emptyset.
            \end{equation}
            Si \( x\in B(x_0,\rho)\) alors \( x\) n'est pas dans \( \Omega_k\) et donc
            \begin{equation}
                \sup_{f\in H}\| f(x) \|\leq k.
            \end{equation}
            Afin d'évaluer \( \| f \|\) nous devons savoir ce qu'il se passe avec les vecteurs sur une boule autour de \( 0\). Pour tout \( x\in B(0,\rho)\) et pour tout \( f\in H\), la linéarité de \( f\) donne
            \begin{equation}
                \| f(x) \|=\| f(x+x_0)-f(x_0) \|\leq \| f(x+x_0)+f(x_0) \|\leq 2k.
            \end{equation}
            Par continuité nous avons alors \( \| f(x) \|\leq 2k\) pour tout \( x\in \overline{ B(0,\rho) }\). Si maintenant \( x\in F\) vérifie \( \| x \|=1\) nous avons
            \begin{equation}
                \| f(x) \|=\frac{1}{ \rho }\| f(\rho x) \|\leq \frac{ 2k }{ \rho },
            \end{equation}
            et donc \( \| f \|\leq \frac{ 2k }{ \rho }\), ce qui montre que \( 2k/\rho\) est un majorant de l'ensemble \( \{ \| f \|\tq f\in H \}\).

    \end{subproof}

\end{proof}
Une application du théorème de Banach-Steinhaus est l'existence de fonctions continues et périodiques dont la série de Fourier ne converge pas. Ce sera l'objet de la proposition \ref{PropREkHdol}.

\begin{corollary}   \label{CorPGwLluz}
    Soient \( \big( E,(p_l) \big)\) et \( \big( F,(q_k) \big)\) deux espaces localement compacts munis de semi-normes. Nous supposons que \( E\) est métrisable et complet. Soit \( (T_j)_{j\in \eN}\) une suite d'application linéaires \( E\to F\) telles que pour tout \( x\in E\) il existe \( \alpha_x\in F\) tel que
    \begin{equation}
        T_jx\stackrel{F}{\longrightarrow}\alpha_x.
    \end{equation}
    Si nous posons \( Tx=\alpha_x\) alors
    \begin{enumerate}
        \item   \label{ItemAEOtOMLi}
            l'application \( T\) est linéaire et continue,
        \item
            pour tout compact \( K\) dans \( E\) et pour tout \( k\) nous avons
            \begin{equation}
                \lim_{j\to \infty} \sup_{x\in K}q_k(T_jx-Tx)=0,
            \end{equation}
        \item   \label{ItemAEOtOMLiii}
            si \( x_j\to x\) dans \( E\) alors
            \begin{equation}
                T_jx_j\to Tx
            \end{equation}
            dans \( F\).
    \end{enumerate}
\end{corollary}
%TODO : voir en quoi c'est un corollaire de Banach-Steinhaus.

La version suivante du théorème de Banach-Steinhaus est énoncée de façon \emph{ad hoc} pour fonctionner avec l'espace \( \swD(K)\) des fonctions de classe \(  C^{\infty}\) à support dans le compact \( K\). Un énoncé un peu plus fort est donné dans le cadre des espaces de Fréchet dans \cite{TQSWRiz}.
\begin{theorem}[Banach-Steinhaus avec des semi-normes]  \label{ThoNBrmGIg}
    Soit \( (E,d)\) un espace vectoriel métrique complet dont la topologie est également\footnote{Au sens où les ouverts sont les mêmes.} donnée par une famille \( \mP\) de semi-normes. Soit \( \{ T_{\alpha} \}_{\alpha\in A}\) une famille d'applications linéaires continues \( T_{\alpha}\colon E\to \eR\) telles que pour tout \( x\in E\) nous ayons
    \begin{equation}
        \sup_{\alpha\in A}\big| T_{\alpha}(x) \big|<\infty.
    \end{equation}
    Alors il existe une constante \( C>0\) et un sous-ensemble fini \( J\subset \mP\) tels que pour tout \( x\in E\) nous ayons
    \begin{equation}    \label{EqIFNGhtr}
        \big| T_{\alpha} (x)\big|\leq C\max_{j\in J}p_j(x).
    \end{equation}
\end{theorem}
\index{théorème!Banach-Steinhaus!avec semi-normes}

\begin{proof}
    Pour chaque \( k\in \eN^*\) nous posons
    \begin{equation}
        \Omega_k=\{ x\in E\tq \sup_{\alpha\in A}\big| T_{\alpha}(x) \big|>k \}.
    \end{equation}
    Ces ensembles sont des ouverts (pour la même raison que dans la preuve du théorème \ref{ThoPFBMHBN}) et leur union est \( E\) en entier parce que par hypothèse \( \sup_{\alpha\in A}\big| T_{\alpha}(x) \big|<\infty\).

    Si les \( \Omega_k\) étaient tous dense, le théorème de Baire \ref{ThoBBIljNM} nous dit que leur intersection est dense également; elle est donc non vide et si \( x_0\in\bigcap_{k\in \eN}\Omega_k\) nous avons
    \begin{equation}
        \sup_{\alpha\in A}\big| T_{\alpha}(x_0) \big|=\infty,
    \end{equation}
    ce qui contredirait l'hypothèse. Donc les \( \Omega_k\) ne sont pas tous denses. Soit \( k_0\in \eN^*\) tel que \( \Omega_{k_0}\) n'est pas dense dans \( E\). Il existe donc \( x_0\in E\) et un ouvert autour de \( x_0\) n'intersectant pas \( \Omega_{k_0}\).

    Nous jouons à présent sur la topologie de \( E\). L'ouvert dont il est question est un \( d\)-ouvert et donc un \( \mP\)-ouvert, lequel contient une \( \mP\)-boule ouverte. Cette dernière boule n'est pas spécialement une \( d\)-boule, mais c'est un \( d\)-ouvert.

    Il existe dont \( J\) fini dans \( \mP\) et \( \rho>0\) tels que \( B_J(x_0,\rho)\cap\Omega_{k_0}=\emptyset\). Donc pour tout \( y\in B_J(x_0,\rho)\) nous avons
    \begin{equation}
        \sup_{\alpha\in A}\big| T_{\alpha}(y) \big|\leq k_0.
    \end{equation}
    Si maintenant \( y\in B(0,\rho)\), nous avons \( y=(y+x_0)-x_0\) et donc
    \begin{subequations}
        \begin{align}
            \big| T_{\alpha}(y) \big|&=\big| T_{\alpha}(y+x_0)-T_{\alpha}(x_0) \big|\\
            &\leq \big| T_{\alpha}(y+x_0) \big|+\big| T_{\alpha}(x_0) \big|\\
            &\leq k_0+C
        \end{align}
    \end{subequations}
    où nous avons posé \( C=\sup_{\alpha\in A}\big| T_{\alpha}(x_0) \big|\). En normalisant, sur la boule \( B_J(0,1)\) nous avons
    \begin{equation}
        \big| T_{\alpha}(y) \big|\leq \rho^{-1}(k_0+C).
    \end{equation}
    Enfin su \( x\in E\) nous avons
    \begin{equation}
        \frac{ x }{ \max_{j\in J}p_j(x) }\in B_J(0,1)
    \end{equation}
    et donc
    \begin{equation}
        \left| T_{\alpha}\left( \frac{ x }{ \max_{j\in J} }p_j(x) \right) \right| \leq \rho^{-1}(k_0+C).
    \end{equation}
    Utilisant encore la linéarité de \( T_{\alpha}\) nous trouvons ce que nous devions trouver :
    \begin{equation}
        \big| T_{\alpha}(x) \big|\leq \max_{j\in J}p_j(x)\rho^{-1}(k_0+C)
    \end{equation}
    à redéfinition près de \( \rho^{-1}(k_0+C)\) en \( C\).
\end{proof}

%+++++++++++++++++++++++++++++++++++++++++++++++++++++++++++++++++++++++++++++++++++++++++++++++++++++++++++++++++++++++++++
\section{Espaces \texorpdfstring{$L^p$}{Lp}}
%+++++++++++++++++++++++++++++++++++++++++++++++++++++++++++++++++++++++++++++++++++++++++++++++++++++++++++++++++++++++++++
\label{SecVKiVIQK}

%--------------------------------------------------------------------------------------------------------------------------- 
\subsection{Généralités}
%---------------------------------------------------------------------------------------------------------------------------

Soit \( (\Omega,\tribF,\mu)\) un espace mesuré. Deux fonctions à valeurs complexes \( f\) et \( g\) sur cet espaces sont dites \defe{équivalentes}{equivalence@équivalence!classe de fonctions} et nous notons \( f\sim g\) si elles sont \( \mu\)-presque partout égales. Nous notons \( [f]\) la classe de \( f\) pour cette relation.

Nous introduisons l'opération
\begin{equation}
    \| f \|_p=\left( \int_{\Omega}| f(x) |^pd\mu(x) \right)^{1/p}
\end{equation}
et nous notons \( \mL^p(\Omega,\mu)\)\nomenclature[Y]{\( \mL^p\)}{espace de Lebesgue, sans les classes} l'ensemble des fonctions mesurables sur \( \Omega\) telles que \( \| f \|_p<\infty\).

\begin{lemma}
    L'ensemble \( \mL^p\) est un espace vectoriel.
\end{lemma}
    
\begin{proof}
    Le fait que si \( f\in L^p\), alors \( \lambda f\in L^p\) est évident. Ce qui est moins immédiat, c'est le fait que \( f+g\in L^p\) lorsque \( f\) et \( g\) sont dans \( L^p\). Cela découle du fait que la fonction \( \varphi\colon x\mapsto x^p\) est convexe, de telle sorte que
    \begin{equation}
        \varphi\left( \frac{ a+b }{2} \right)\leq\frac{ \varphi(a)+\varphi(b) }{2},
    \end{equation}
    ou encore
    \begin{equation}    \label{EqZFSduFa}
        (a+b)^p\leq 2^{p-1}(a^p+b^p)
    \end{equation}
\end{proof}

    L'opération \( f\mapsto \| f \|_p\) n'est pas une norme sur \( \mL^p\) parce que pour \( f\) presque partout nulle, nous avons \( | f |_p=0\). Il y a donc des fonctions non nulles sur lesquelles \( \| . \|_p\) s'annule.

\begin{lemma}       \label{LemKZVHVAR}
    Si \( f\in \mL^p(\Omega)\) et \( f\sim g\), alors \( g\in \mL^p(\Omega)\) et \( \| f \|_p=\| g \|_p\).
\end{lemma}

\begin{proof}
    Soit \( h(x)=| g(x) |^p-| f(x) |^p\); c'est une fonction par hypothèse presque partout nulle et donc intégrable sur \( \Omega\); son intégrale y vaut zéro. Nous avons
    \begin{equation}
        \int_{\Omega}| f(x) |^pd\mu(x)=\int_{\Omega}\Big( | f(x) |^p+h(x)\big)d\mu(x)=\int_{\omega}| g(x) |^pd\mu(x).
    \end{equation}
    Cela prouve que la dernière intégrale existe et vaut la même chose que la première.
\end{proof}

Nous pouvons donc considérer la norme \( | . |_p\) comme une norme sur l'ensemble des classes plutôt que sur l'ensemble des fonctions. Nous notons \( L^p\)\nomenclature[Y]{\( L^p\)}{espace de Lebesgue avec les classes} l'ensemble des classes des fonctions de \(\mL^p\). Cet espace est muni de la norme
\begin{equation}
    \| [f] \|_p=\| f \|_p,
\end{equation}
formule qui ne dépend pas du représentant par le lemme \ref{LemKZVHVAR}.

Maintenant la formule
\begin{equation}
    \| [f] \|_p=\left( \int_{\Omega}| f(x) |^pd\mu(x) \right)^{1/p}
\end{equation}
défini une norme sur \( L^p(\Omega,\mu)\). En effet si \( \| [f] \|_p=0\), nous avons
\begin{equation}
    \int_{\Omega}| f(x) |^pd\mu(x)=0,
\end{equation}
ce qui par le lemme \ref{Lemfobnwt} implique que \( | f(x) |^p=0\) pour presque tout \( x\). Ou encore \( f\sim 0\), c'est à dire \( [f]=[0]\) au niveau des classes. À partir de maintenant \( \big( L^p(\Omega,\mu),\| . \|_p \big)\) est un espace métrique avec toute la topologie qui va avec.

Dans la suite nous n'allons pas toujours écrire \( [f]\) pour la classe de \( f\). Par abus de notations nous allons souvent parler de \( f\in L^p\) comme si c'était une fonction.

\begin{proposition}[\cite{bJOSNQ}]  \label{PropWoywYG}
    Soit \( 1\leq p\leq \infty\) et supposons que la suite \( [f_n]\) dans \( L^p(\Omega,\tribF,\mu)\) converge vers \( [f]\) au sens \( L^p\). Alors il existe une sous-suite \( (h_n)\) qui converge ponctuellement \( \mu\)-presque partout vers \( f\).
\end{proposition}
\index{espace!\( L^p\)}
\index{suite!de fonctions}
\index{limite!inversion}

\begin{proof}
    Si \( p=\infty\) nous sommes en train de parler de la convergence uniforme et il ne faut même pas prendre ni de sous-suite ni de «presque partout».

    Supposons que \( 1\leq p<\infty\). Nous considérons une sous-suite \( [h_n]\) de \( [f_n]\) telle que
    \begin{equation}
        \| [h_j]-[f] \|_p<2^{-j},
    \end{equation}
    puis nous posons \( u_k(x)=| h_k(x)-f(x) |^p\). Notons que ce \( u_k\) est une vraie fonction, pas une classe. Et en plus c'est une fonction positive. Nous avons
    \begin{equation}
        \int_{\Omega}u_kd\mu=\int_{\omega}| h_k(x)-f(x) |^pd\mu(x)=\| h_k-f \|_p^p\leq 2^{-kp}.
    \end{equation}
    Vu que \( u_k\) est une fonction positive la suite des sommes partielles de \( \sum_ku_k\) est croissante et vérifie donc le théorème de la convergence monotone \ref{ThoRRDooFUvEAN} :
    \begin{equation}
            \int_{\Omega}\left( \sum_{k=0}^{\infty}u_k(x) \right)d\mu(x)=\sum_{k=0}^{\infty}\int_{\Omega}u_k(x)d\mu(x)
            \leq\sum_{k=0}^{\infty}2^{-kp}<\infty.
    \end{equation}
    Le fait que l'intégrale de la fonction \( \sum_ku_k\) est finie implique que cette fonction est finie \( \mu\)-presque partout. Donc le terme général tend vers zéro presque partout, c'est à dire
    \begin{equation}
        | h_k(x)-f(x) |^p\to 0.
    \end{equation}
    Cela signifie que \( h_k\to f\) presque partout ponctuellement.
\end{proof}

Est-ce qu'on peut faire mieux que la convergence ponctuelle presque partout d'une sous-suite ? En tout cas on ne peut pas espérer grand chose comme convergence pour la suite elle-même, comme le montre l'exemple suivant.

\begin{example} \label{ExPOmxICc}
    Nous allons montrer une suite de fonctions qui converge vers zéro dans \( L^p[0,1]\) (avec \( p<\infty\)) mais qui ne converge ponctuellement pour \emph{aucun} point. Cet exemple provient de \href{http://www.bibmath.net/dico/index.php?action=affiche&quoi=./b/bosseglissante.html}{bibmath.net}. 

%TODO : revoir tous les \href et mettre ceux qui doivent être en bibliographie. Par exemple celui ci-dessus.

    Nous construisons la suite de fonctions par paquets. Le premier paquet est formé de la fonction constante \( 1\).

    Le second paquet est formé de deux fonctions. La première est \( \mtu_{\mathopen[0 , 1/2 \mathclose]}\) et la seconde \( \mtu_{\mathopen[ 1/2 , 1 \mathclose]}\).

    Plus généralement le paquet numéro \( k\) est constitué des \( k\) fonctions \( \mtu_{\mathopen[ i/k , (i+1)/k \mathclose]}\) avec \( i=0,\ldots, k-1\).

    Vu que les fonctions du paquet numéro \( k\) ont pour norme \( \| f \|_p=\frac{1}{ k }\), nous avons évidemment \( f_n\to 0\) dans \( L^p\). Il est par contre visible que chaque paquet passe en revue tous les points de \( \mathopen[ 0 , 1 \mathclose]\). Donc pour tout \( x\) et pour tout \( N\), il existe (même une infinité) \( n>N\) tel que \( f_n(x)=1\). Il n'y a donc convergence ponctuelle nulle part.
\end{example}

La proposition suivante est une espèce de convergence dominée de Lebesgue pour \( L^p\).
\begin{proposition} \label{PropBVHXycL}
    Soit \( f\in L^p(\Omega)\) avec \( 1\leq p<\infty\) et \( (f_n)\) une suite de fonctions convergeant ponctuellement vers \( f\) et telle que \( | f_n |\leq | f |\). Alors \( f_n\) converge vers \( f\) au sens de \( L^p\).
\end{proposition}

\begin{proof}
    Nous avons immédiatement \( | f_n(x) |^p\leq | f(x) |^p\), de telle sorte que le théorème de la convergence dominée implique que \( f_n\in L^p\). La convergence dominée donne aussi que \( \| f_n \|_p\to\| f \|_p\), mais cela ne nous intéresse pas ici.

    Nous posons \( h_n(x)= | f_n(x)-f(x) | \). En reprenant la formule de majoration \eqref{EqZFSduFa} et en tenant compte du fait que \( | f_n(x) |\leq | f(x) |\), nous avons
   \begin{equation}
       h_n(x)\leq 2^{p-1}\big( | f_n(x) |^p+| f(x) |^p \big)\leq 2^p| f(x) |^p,
   \end{equation}
   ce qui prouve que \( | h_n |\) est uniformément (en \( n\)) majorée par une fonction intégrable, donc \( h_n\) est intégrable et on peut permuter la limite et l'intégrale (théorème de la convergence dominée \ref{ThoConvDomLebVdhsTf}) :
   \begin{equation}
       \lim_{n\to \infty} \| f_n-f \|^p_p=\lim_{n\to \infty} \int_{\eR^d}| f_n(x)-f(x) |^pdx=\int_{\eR^d}\lim_{n\to \infty} h_n(x)dx=0.
   \end{equation}
\end{proof}

%--------------------------------------------------------------------------------------------------------------------------- 
\subsection{Inégalité de Jensen, Hölder et de Minkowski}
%---------------------------------------------------------------------------------------------------------------------------

\begin{proposition}[Inégalité de Jensen\cite{MesIntProbb}] \label{PropXISooBxdaLk}
    Soit un espace mesuré de probabilité\footnote{C'est à dire que \( \int_{\Omega}d\mu=1\).} \( (\Omega,\tribA,\mu)\) ainsi qu'une fonction convexe \( f\colon \eR\to \eR\) et une application \( \alpha\colon \Omega\to \eR\) tels que \( \alpha\) et \( f\circ \alpha\) soient intégrables sur \( \Omega\). Alors
    \begin{equation}
        f\Big( \int_{\Omega}\alpha\,d\mu \Big)\leq \int_{\Omega}(f\circ\alpha) d\mu.
    \end{equation}
\end{proposition}
\index{inégalité!Jensen!version intégrale}

\begin{proof}
    Soit \( a\in \eR\) et le nombre \( c_a\) donné par la proposition \ref{PropNIBooSbXIKO} : pour tout \( \omega\in \Omega\) nous avons
    \begin{equation}    \label{EqOUMooIwknIP}
        f\big( \alpha(\omega) \big)-f(a)\geq c_a\big( \alpha(\omega)-a \big).
    \end{equation}
    Cela est en particulier vrai pour \( a=\int_{\Omega}\alpha\,d\mu\). Nous intégrons l'inégalité \eqref{EqOUMooIwknIP} sur \( \Omega\) en nous souvenant que \( \int d\mu=1\) :
    \begin{subequations}
        \begin{align}
            \int_{\Omega}(f\circ \alpha)d\mu-\int_{\Omega}f(a)d\mu&\geq c_a\big( \int_{\Omega}\alpha-\int_{\Omega}a \big)\\
            \int_{\Omega}(f\circ \alpha)d\mu-f(a)&\geq c_a(a-a)\\
            f(a)&\leq \int_{\Omega}(f\circ\alpha)d\mu.
        \end{align}
    \end{subequations}
    Cette dernière inégalité est celle que nous devions prouver.
\end{proof}

\begin{proposition}[Inégalité de Hölder]       \label{ProptYqspT}
    Soit \(  (\Omega,\tribA,\mu) \) un espace mesuré et \( 1\leq p\), \( q\leq\infty\) satisfaisant \( \frac{1}{ p }+\frac{1}{ q }=1\). Si \( f\in L^p(\Omega)\), \( g\in L^q(\Omega)\), alors le produit \( fg\) est dans \( L^1(\Omega)\) et nous avons
    \begin{equation}    \label{EqLPKooPBCQYN}
        \| fg \|_1\leq \| f \|_p\| g \|_q.
    \end{equation}
    Plus généralement si \( \frac{1}{ p }+\frac{1}{ q }=\frac{1}{ r }\) alors
    \begin{equation}    \label{EqAVZooFNyzmT}
        \| fg \|_r\leq \| f \|_p\| g \|_q
    \end{equation}
\end{proposition}
\index{inégalité!Hölder}
Ce qui est appelé «inégalité de Hölder» est généralement l'inéquation \eqref{EqLPKooPBCQYN}.

\begin{proof}
    Nous allons prouver l'inégalité \eqref{EqAVZooFNyzmT}. D'abord nous supposons \( \| g \|_q=1\) et nous posons
    \begin{equation}
        A=\{ x\in\Omega\tq | g(x) |>0 \}.
    \end{equation}
    Hors de \( A\), les intégrales que nous allons écrire sont nulles. Nous avons
    \begin{equation}
        \| fg \|_r^p=\Big|  \int_A| f |^r| g |^{r-q}| g |^q  \Big|^{p/r},
    \end{equation}
    et le coup tordu est de considérer cette intégrale comme étant une intégrale par rapport à la mesure \( \nu=| g |^qd\mu\) qui a la propriété d'être une mesure de probabilité par hypothèse sur \( g\). Nous pouvons alors utiliser l'inégalité de Jensen\footnote{Proposition \ref{PropXISooBxdaLk}.} parce que \( p/r>1\), ce qui fait de \( x\mapsto | x |^{p/r}\) une fonction convexe. Nous avons alors
    \begin{subequations}
        \begin{align}
            \| fg \|_r^p&\leq\int_A\big( | f |^r| g |^{r-q} \big)^{p/r}| g |^qd\mu\\
            &=\int_A| f |^{p}| g |^{p(r-q)/r}| g |^qd\mu
        \end{align}
    \end{subequations}
    La puissance de \( | g |\) dans cette expression est : \( q+\frac{ p(r-q) }{ r }=0\) parce que \( p(q-r)=rq\). Nous avons alors montré que
    \begin{equation}
        \| fg \|_r^p\leq \int_A| f |^pd\mu\leq \| f \|_p^p.
    \end{equation}
    La dernière inégalité est le fait que le domaine \( A\) n'est pas tout le domaine \( \Omega\).

    Si maintenant \( \| g \|_q\neq 1\) alors nous calculons
    \begin{equation}
        \| fg \|_r=\| g \|_q\| f\frac{ g }{ \| g \|_q } \|_r\leq \| g \|_q\| f \|_p
    \end{equation}
    en appliquant la première partie à la fonction \( \frac{ g }{ \| g \|_q }\) qui est de norme \( 1\).
\end{proof}


\begin{remark}      \label{RemNormuptNird}
    Dans le cas d'un espace de probabilité, la fonction constante \( g=1\) appartient à \( L^p(\Omega)\). En prenant \( p=q=2\) nous obtenons
    \begin{equation}
        \| f \|_1\leq\| f \|_2.
    \end{equation}
\end{remark}

\begin{lemma}   \label{LemTLHwYzD}
    Lorsque \( I\) est borné nous avons \( L^2(I)\subset L^1(I)\). Si \( I\) n'est pas borné alors \( L^2(I)\subset L^1_{loc}(I)\).
\end{lemma}

\begin{proof}
    En effet si \( I\) est borné, alors la fonction constante \( 1\) est dans \( L^2(I)\) et l'inégalité de Hölder \ref{ProptYqspT} nous dit que le produit \( 1u\) est dans \( L^1(I)\).

    Si \( I\) n'est pas borné, nous refaisons le même raisonnement sur un compact \( K\) de \( I\).
\end{proof}

\begin{proposition}[Inégalité de Minkowski\cite{TUEWwUN}]     \label{PropInegMinkKUpRHg}
    Si \( 1\leq p<\infty\) et si \( f,g\in L^p(\Omega,\tribA,\mu)\) alors
    \begin{enumerate}
        \item   \label{ItemDHukLJi}
            \( \| f+g \|_p\leq \| f \|_p+\| g \|_p\)
        \item
            Il y a égalité si et seulement si les vecteurs \( f(x)\) et \( g(x)\) sont presque partout colinéaires : il existe \( \alpha,\beta\) tels que \( \alpha f+\beta g=0\) presque partout.

        \item   \label{ItemDHukLJiii}

            Si \( f(x,y)\) est mesurable sur l'espace produit \( \big( X\times Y,\mu\otimes\nu \big)\) et si \( p\geq 1\), alors
            \begin{equation}
                \left\|   x\mapsto\int_Y f(x,y)d\nu(y)   \right\|_p\leq
                \int_Y  \| f_y \|_pd\nu(y)
            \end{equation}
            où \( f_y(x)=f(x,y)\).

    \end{enumerate}
\end{proposition}
\index{inégalité!Minkowski}
%TODO : une preuve.

La partie \ref{ItemDHukLJiii} est une généralisation de l'inégalité triangulaire (c'est à dire du point \ref{ItemDHukLJi}) dans le cas où nous n'avons pas une somme de deux fonctions mais d'une infinité paramétrée par \( y\in Y\). Elle sera le plus souvent utilisée sous la forme déballée :
\begin{equation}    \label{EqZSiTZrH}
    \left[ \int_X\Big( \int_Y| f(x,y) |d\nu(y) \Big)^pd\mu(x) \right]^{1/p}\leq \int_Y\Big( \int_X| f(x,y) |^pd\mu(x) \Big)^{1/p}d\nu(y).
\end{equation}

%--------------------------------------------------------------------------------------------------------------------------- 
\subsection{Ni inclusions ni inégalités}
%---------------------------------------------------------------------------------------------------------------------------

Aucun espace \( L^p(\eR)\) n'est inclus dans aucun autre ni aucune norme n'est plus grande qu'une autre (sur les intersections). Nous verrons cependant en la proposition \ref{PropIRDooFSWORl} que de telles inclusions et inégalités sont possibles pour \( L^p\big( \mathopen[ 0 , 1 \mathclose] \big)\).

Nous allons donner des exemples de tout ça en supposant \( p<q\) et en nous appuyant lourdement sur les intégrales de \( \frac{1}{ x^{\alpha} }\) étudiées par la proposition \ref{PropBKNooPDIPUc}.

\begin{subproof}
    \item[\( L^p\nsubseteq L^q\)]

        La fonction
        \begin{equation}    \label{EqXIEooZpxObV}
            f(x)=\begin{cases}
                \frac{1}{ x^{1/q} }    &   \text{si \( 0<x<1\)}\\
                0    &    \text{sinon}
            \end{cases}
        \end{equation}
        est dans \( L^p\) mais pas dans \( L^q\). En effet
        \begin{equation}
            \| f \|_p^p=\int_0^1\frac{1}{ x^{p/q} }dx<\infty
        \end{equation}
        parce que \( p<q\) et \( p/q<1\). Par contre
        \begin{equation}
            \| f \|_q^q=\int_0^1\frac{1}{ x }dx=\infty.
        \end{equation}

    \item[\( L^q\nsubseteq L^p\)]

        La fonction
        \begin{equation}
            f(x)=\begin{cases}
                \frac{1}{ x^{1/p} }    &   \text{si \( x>1\)}\\
                0    &    \text{sinon}
            \end{cases}
        \end{equation}
        est dans \( L^q\) mais pas dans \( L^p\). En effet
        \begin{equation}
            \| f \|_p^p=\int_1^{\infty}\frac{1}{ x }=\infty
        \end{equation}
        alors que
        \begin{equation}
            \| f \|_q^q=\int_1^{\infty}\frac{1}{ x^{q/p} }dx<\infty.
        \end{equation}

    \item[Exempe de \( \| f \|_p>\| f \|_q\)]

        La fonction
        \begin{equation}
            f(x)=\begin{cases}
                1    &   \text{si \( x\in\mathopen[ 0 , 2 \mathclose]\)}\\
                0    &    \text{sinon.}
            \end{cases}
        \end{equation}
        Nous avons
        \begin{subequations}
            \begin{align}
                \| f \|_p&=2^{1/p}\\
                \| f \|_q&=2^{1/q}.
            \end{align}
        \end{subequations}
        Mais comme \( p<q\) donc \( \| f \|_p>\| f \|_q\).


    \item[Exempe de \( \| f \|_p<\| f \|_q\)]
        
        La fonction
        \begin{equation}
            f(x)=\begin{cases}
                1    &   \text{si \( x\in\mathopen[ 0 , \frac{ 1 }{2} \mathclose]\)}\\
                0    &    \text{sinon.}
            \end{cases}
        \end{equation}
        Alors
        \begin{subequations}
            \begin{align}
                \| f \|_p&=\frac{1}{ 2^{1/p} }\\
                \| f \|_q&=\frac{1}{ 2^{1/q} }
            \end{align}
        \end{subequations}
        et donc \( \| f \|_p<\| f \|_q\).
\end{subproof}

Ces exemples donnent un exemple de fonction \( f\) telle que \( \| f \|_p<\| f \|_q\) pour tout espace \( L^p(I)\) et \( L^q(I)\) avec \( I\subset \eR\). Par contre l'exemple \( \| f \|_p>\| f \|_q\) ne fonctionne que si la taille de \( I\) est plus grande que \( 1\). Et pour cause : il y a des inclusions si \( I\) est borné.

\begin{proposition} \label{PropIRDooFSWORl}
    Soit \( 1<p<2\). Alors nous avons
    \begin{enumerate}
        \item
            $L^2\big( \mathopen[ 0 , 1 \mathclose] \big)\subset L^p\big( \mathopen[ 0 , 1 \mathclose] \big)$.
        \item   \label{ItemWSTooLcpOvXii}
            Si \( f\in L^2\big( \mathopen[ 0 , 1 \mathclose] \big)\) alors \( \| f \|_p\leq \| f \|_2\).
    \end{enumerate}
\end{proposition}
\begin{proof}
    Pour la simplicité des notations nous allons noter \( L^2\) pour \( L^2\big( \mathopen[ 0 , 1 \mathclose] \big)\), et pareillement pour \( L^p\). Soit \( f\in L^2\). Nous posons
    \begin{equation}
        A=\{ x\in\mathopen[ 0 , 1 \mathclose]\tq | f(x) |\geq 1 \}.
    \end{equation}
    Étant donné que \( p<2\) nous avons \( | f |^p\leq | f |^2\) sur \( A\); par conséquent \( \int_A| f |^p\) converge parce que \( \int_A| f |^2\) converge.

    L'ensemble \( A^c\) est évidemment borné (complémentaire dans \( \mathopen[ 0 , 1 \mathclose]\)) et sur \( A^c\) nous avons \( | f(x) |\leq 1\) et donc \( | f |^p\leq 1\). L'intégrale \( \int_{A^c}| f |^p\) converge donc également.

    Au final \( \int_{\mathopen[ 0 , 1 \mathclose]}| f |^p\) converge et \( f\in L^p\).


    Soit à présent \( f\in L^2\cap L^p\) et \( r\in \eR\) tel que \( \frac{1}{ 2/p }+\frac{1}{ r }=1\). Nous avons \( | f |^p\in L^{2/p}\), et vu que nous sommes sur un domaine borné, \( 1\in L^r\). Nous écrivons l'inégalité de Hölder \eqref{EqLPKooPBCQYN} avec ces fonctions. D'une part
    \begin{equation}
        \| f \|_1=\| | f |^p \|_1=\| f \|_p^p.
    \end{equation}
    D'autre part
    \begin{equation}
        \| | f |^p \|_{2/p}=\left( \int| f |^2 \right)^{p/2}=\| f \|_2^p.
    \end{equation}
    Donc \( \| f \|_p^p\leq \| f \|_2^p\), ce qui prouve l'assertion \ref{ItemWSTooLcpOvXii} parce que \( p>1\).
\end{proof}

\begin{remark}
    Nous n'avons cependant pas \( L^2\big( \mathopen[ 0 , 1 \mathclose] \big)=L^p\big( \mathopen[ 0 , 1 \mathclose] \big)\) parce que l'exemple \eqref{EqXIEooZpxObV} fonctionne encore :
    \begin{equation}
        f(x)=\frac{1}{ \sqrt{x} }
    \end{equation}
    pour \( x\in\mathopen[ 0 , 1 \mathclose]\) donne bien
    \begin{equation}
        \| f \|_2=\int_0^1\frac{1}{ x }=\infty
    \end{equation}
    et \( \| f \|_p=\int_0^1\frac{1}{ x^{p/2} }<\infty\) parce que \( 1<p<2\).
\end{remark}

%--------------------------------------------------------------------------------------------------------------------------- 
\subsection{Complétude}
%---------------------------------------------------------------------------------------------------------------------------

\begin{theorem}[\cite{SuquetFourierProba,UQSGIUo}]  \label{ThoUYBDWQX}
    Pour \( 1\leq p<\infty\), l'espace \( L^p(\Omega,\tribA,\mu)\) est complet.
\end{theorem}
\index{complétude}

\begin{proof}
    Soit \( (f_n)_{n\in\eN}\) une suite de Cauchy dans \( L^p\). Pour tout \( i\), il existe \( N_i\in\eN\) tel que $\| f_p-f_q \|_p\leq 2^{-i}$ pour tout \( p,q\geq N_i\). Nous considérons la sous suite \( g_i=f_{N_i}\), de telle sorte qu'en particulier
    \begin{equation}    \label{EqJLoDID}
        \|g_i-g_{i-1}\|_p\leq 2^{-i}.
    \end{equation}
    Pour chaque \( j\) nous considérons la somme télescopique
    \begin{equation}
        g_j=g_0+\sum_{i=1}^j(g_i-g_{i-1})
    \end{equation}
    et l'inégalité
    \begin{equation}
        | g_j |\leq | g_0 |+\sum_{i=1}^j| g_i-g_{i-1} |.
    \end{equation}
    Nous allons noter
    \begin{equation}        \label{EqSomPaFPQOWC}
        h_j=| g_0 |+\sum_{i=1}^j| g_i-g_{i-1} |.
    \end{equation}
    La suite de fonctions \( (h_j)\) ainsi définie est une suite croissante de fonctions positive qui converge donc (ponctuellement) vers une fonction \( h\) qui peut éventuellement valoir l'infini en certains points. Par continuité de la fonction \( x\mapsto x^p\) nous avons
    \begin{equation}
        \lim_{j\to \infty} h_j^p=h^p,
    \end{equation}
    puis par le théorème de la convergence monotone (théorème \ref{ThoRRDooFUvEAN}) nous avons
    \begin{equation}
        \lim_{j\to \infty} \int_{\Omega}h_j^pd\mu=\int_{\Omega}h^pd\mu.
    \end{equation}
    Utilisant à présent la continuité de la fonction \( x\mapsto x^{1/p}\) nous trouvons
    \begin{equation}
        \lim_{j\to \infty} \left( \int h_j^p \right)^{1/p}=\left( \int | h |^p \right)^{1/p}.
    \end{equation}
    Nous avons donc déjà montré que
    \begin{equation}
        \lim_{j\to \infty} \| h_j \|_p=\left( \int | h |^p \right)^{1/p}
    \end{equation}
    où, encore une fois, rien ne garantit à ce stade que l'intégrale à droite soit un nombre fini. En utilisant l'inégalité de Minkowski (proposition \ref{PropInegMinkKUpRHg}) et l'inégalité \eqref{EqJLoDID} nous trouvons
    \begin{equation}
        \|h_j\|_p\leq \|g_0\|_p+\sum_{i=1}^j\|g_i-g_{i-1}\|_p\leq \|g_0\|_p+1.
    \end{equation}
    En passant à la limite,
    \begin{equation}
        \left( \int| h |^p \right)^{1/p}=\lim_{j\to \infty}\|h_j\|_p \leq \|g_0\|_p+1<\infty.
    \end{equation}
    Par conséquent \( \int| h |^p\) est finie et
    \begin{equation}    \label{EqgLpdUPOBP}
        h\in L^p(\Omega,\tribA,\mu).
    \end{equation}
    En particulier, l'intégrale \( \int h\) est finie (parce que \( p\geq 1\)) et donc que \( h(x)<\infty\) pour presque tout \( x\in\Omega\).

    Nous savons que \( h(x)\) est la limite des sommes partielles \eqref{EqSomPaFPQOWC}, en particulier la série
    \begin{equation}
        \sum_{j=1}^{\infty}| g_i-g_{i-1} |
    \end{equation}
    converge ponctuellement. En vertu du corollaire \ref{CorCvAbsNormwEZdRc}, la série de terme général \( g_i-g_{i-1}\) converge ponctuellement. La suite \( g_i\) converge donc vers une fonction que nous notons \( g\). Par ailleurs la suite \( g_i\) est dominée par \( h\in L^p\), le théorème de la convergence dominée (théorème \ref{ThoConvDomLebVdhsTf}) implique que
    \begin{equation}
        \lim_{j\to \infty} \|g_j-g\|_p=0.
    \end{equation}
    Nous allons maintenant prouver que \( \lim_{n\to \infty\|f_n-g\|_p} =0\). Soit \( \epsilon>0\). Pour tout \( n\) et \( i\) nous avons
    \begin{equation}
        \|f_n-g\|_p=\|f_n-f_{N_i}+f_{N_i}-g\|_p\leq\|f_n-f_{N_i}\|_p+\|f_{N_i}-g\|_p.
    \end{equation}
    Pour rappel, \( f_{N_i}=g_i\). Si \(i\) et \( n\) sont suffisamment grands nous pouvons obtenir que chacun des deux termes est plus petit que \( \epsilon/2\).

    Il nous reste à prouver que \( g\in L^p(\Omega,\tribA,\mu)\). Nous avons déjà vu (équation \eqref{EqgLpdUPOBP}) que \( h\in L^p\), mais \( | g_i |\leq h^p\), par conséquent  \( g\in L^p\).

    Nous avons donc montré que la suite de Cauchy \( (f_n)\) converge vers une fonction de \( L^p\), ce qui signifie que \( L^p\) est complet.
\end{proof}

\begin{theorem}[Fischer-Riesz\cite{KXjFWKA}] \label{ThoGVmqOro}
    Soit un ouvert \( \Omega\) de \( \eR^n\) et \( p\in\mathopen[ 1 , \infty \mathclose]\). Alors
    \begin{enumerate}
        \item\label{ItemPDnjOJzi}
            Toute suite convergente dans \( L^p(\Omega)\) admet une sous-suite convergente presque partout sur \( \Omega\).
        \item\label{ItemPDnjOJzii}
            La sous-suite donnée en \ref{ItemPDnjOJzi} est dominée par un élément de \( L^p(\Omega)\).
        \item\label{ItemPDnjOJziii}
            L'espace \( L^p(\Omega)\) est de Banach.
    \end{enumerate}
\end{theorem}
\index{espace!de fonctions!$L^p$}
\index{complétude!espaces $ L^p$}

\begin{proof}
    Le cas \( p=\infty\) est à séparer des autres valeurs de \( p\) parce qu'on y parle de norme uniforme, et aucune sous-suite n'est à considérer.
    \begin{subproof}
    \item[Cas \( p=\infty\).]
    Nous commençons par prouver dans le cas \( p=\infty\). Soit \( (f_n)\) une suite de Cauchy dans \( L^{\infty}(\Omega)\), ou plus précisément une suite de représentants d'éléments de \( L^p\). Pour tout \( k\geq 1\), il existe \( N_k\geq 0\) tel que si \( m,n\geq N_k\), on a
    \begin{equation}
        \| f_m-f_n \|_{\infty}\leq \frac{1}{ k }.
    \end{equation}
    En particulier, il existe un ensemble de mesure nulle \( E_k\) sur lequel
    \begin{equation}
        | f_m(x)-f_n(x) |\leq\frac{1}{ k },
    \end{equation}
    et si nous posons \( E=\bigcup_{k\in \eN}E_k\), nous avons encore un ensemble de mesure nulle (lemme \ref{LemIDITgAy}). En  résumé, nous avons un \( N_k\) tel que si \( m,n\geq N_k\), alors 
    \begin{equation}    \label{EqKAWSmtG}
        | f_n(x)-f_m(x) |\leq \frac{1}{ k }
    \end{equation}
    pour tout \( x\) hors de \( E\). Donc pour chaque \( x\in\Omega\setminus E\), la suite \( n\mapsto f_n(x)\) est de Cauchy dans \( \eR\) et converge donc. Cela défini donc une fonction
    \begin{equation}
        \begin{aligned}
            f\colon \Omega\setminus E&\to \eR \\
            x&\mapsto \lim_{n\to \infty} f_n(x). 
        \end{aligned}
    \end{equation}
    Cela prouve le point \ref{ItemPDnjOJzi} : la convergence ponctuelle.

    En passant à la limite \( n\to \infty\) dans l'équation \ref{EqKAWSmtG} et tenant compte que cette majoration tient pour presque tout \( x\) dans \( \Omega\), nous trouvons
    \begin{equation}
        \| f-f_n \|_{\infty}\leq \frac{1}{ k }.
    \end{equation}
    Donc non seulement \( f\) est dans \( L^{\infty}\), mais en plus la suite \( (f_n)\) converge vers \( f\) au sens \( L^{\infty}\), c'est à dire uniformément. Cela prouve le point \ref{ItemPDnjOJziii}. En ce qui concerne le point \ref{ItemPDnjOJzii}, la suite \( f_n\) est entièrement (à partir d'un certain point) dominée par la fonction \( 1+| f |\) qui est dans \( L^p\).

\item[Cas \( p<\infty\).]

        Toute suite convergente étant de Cauchy, nous considérons une suite de Cauchy \( (f_n)\) dans \( L^p(\Omega)\) et ce sera suffisant pour travailler sur le premier point. Pour montrer qu'une suite de Cauchy converge, il est suffisant de montrer qu'une sous-suite converge. Soit \( \varphi\colon \eN\to \eN\) une fonction strictement croissante telle que pour tout \( n\geq 1\) nous ayons
        \begin{equation}
            \| f_{\varphi(n+1)}-f_{\varphi(n)} \|_p\leq \frac{1}{ 2^{n} }.
        \end{equation}
        Pour créer la fonction \( \varphi\), il est suffisant de prendre le \( N_k\) donné par la condition de Cauchy pour \( \epsilon=1/2^k\) et de considérer la fonction définie par récurrence par \( \varphi(1)=N_1\) et \( \varphi(n+1)>\max\{ N_n,\varphi(n-1) \}\). Ensuite nous considérons la fonction
        \begin{equation}
            g_n(x)=\sum_{k=1}^n| f_{\varphi(k+1)}(x)-f_{\varphi(k)}(x) |.
        \end{equation}
        Notons que pour écrire cela nous avons considéré des représentants \( f_k\) qui sont alors des fonctions à l'ancienne. Étant donné que \( g_n\) est une somme de fonctions dans \( L^p\), c'est une fonction \( L^p\), comme nous pouvons le constater en calculant sa norme :
        \begin{equation}
            \| g_n \|_p\leq \sum_{k=1}^n\| f_{\varphi(k+1)}-f_{\varphi(k)} \|_p\leq\sum_{k=1}^n\frac{1}{ 2^k }\leq\sum_{k=1}^{\infty}\frac{1}{ 2^k }=1.
        \end{equation}
        Étant donné que tous les termes de la somme définissant \( g_n\) sont positifs, la suite \( (g_n)\) est croissante. Mais elle est bornée en norme \( L^p\) et donc sujette à obéir au théorème de Beppo-Levi \ref{ThoRRDooFUvEAN} sur la convergence monotone. Il existe donc une fonction \( g\in L^p(\Omega)\) telle que \( g_n\to g\) presque partout.

        Soit un \( x\in \Omega\) pour lequel \( g_n(x)\to g(x)\); alors pour tout \( n\geq 2\) et \( \forall q\geq 0\),
        \begin{subequations}    \label{EqWTHojCq}
            \begin{align}
                | f_{\varphi(n+q)}(x)-f_{\varphi(n)}(x) |&=\left| f_{\varphi(n+q)}(x)-\sum_{k=1}^{q-1}f_{\varphi(n+k)}(x) -\sum_{k=1}^{q-1}f_{\varphi(n+k)}(x)-f_{\varphi(n)}(x) \right| \\
                &=\left| \sum_{k=1}^qf_{\varphi(n+k)}-\sum_{k=1}^qf_{\varphi(n+k-1)}(x) \right|\\
                &\leq \sum_{k=1}^q\Big| f_{\varphi(n+k)}(x)-f_{\varphi(n+k-1)}(x) \Big|\\
                &=g_{n+q+1}(x)-g_{n+1}(x)\\
                &\leq g(x)-g_{n-1}(x).
            \end{align}
        \end{subequations}
        Nous prenons la limite \( n\to \infty\); la dernière expression tend vers zéro et donc
        \begin{equation}
            | f_{\varphi(n+q)}(x)-f_{\varphi(n)}(x) |\to 0
        \end{equation}
        pour tout \( q\). Donc pour presque tout \( x\in \Omega\), la suite \( n\mapsto f_{\varphi(n)}(x)\) est de Cauchy dans \( \eR\) et donc y converge vers un nombre que nous nommons \( f(x)\). Cela définit une fonction
        \begin{equation}
            \begin{aligned}
                f\colon \Omega\setminus E&\to \eR \\
                x&\mapsto \lim_{n\to \infty} f_{\varphi(n)}(x) 
            \end{aligned}
        \end{equation}
        où \( E\) est de mesure nulle. Montrons que \( f\) est bien dans \( L^p(\Omega)\); pour cela nous complétons la série d'inégalités \eqref{EqWTHojCq} en
        \begin{equation}
            \big| f_{\varphi(n+q)}(x)-f_{\varphi(n)}(x) \big|\leq g(x)-g_{n-1}(x)\leq g(x).
        \end{equation}
        En prenant la limite \( q\to \infty\) nous avons l'inégalité
        \begin{equation}    \label{EqMQbDRac}
            | f(x)-f_{\varphi(n)}(x) |\leq g(x)
        \end{equation}
        pour presque tout \( x\in\Omega\), c'est à dire pour tout \( x\in\Omega\setminus E\). Cette inégalité implique deux choses valables pour presque tout \( x\) dans \( \Omega\) :
        \begin{subequations}
            \begin{align}
                f(x)&\in B\big( g(x),f_{\varphi(n)}(x) \big)\\
                f_{\varphi(n)}(x)&\leq | f(x) |+| g(x) |.
            \end{align}
        \end{subequations}
        
        La première inégalité assure que \( | f |^p\) est intégrable sur \( \Omega\setminus E\) parce que \( | f |\) est majorée par \( | g |+| f_{\varphi(n)} |\). Elle prouve par conséquent le point \ref{ItemPDnjOJzi} parce que \(n\mapsto f_{\varphi(n)}\) est une sous-suite convergente presque partout. La seconde montre le point \ref{ItemPDnjOJzii}. 

        Attention : à ce point nous avons prouvé que \( n\mapsto f_{\varphi(n)}\) est une suite de fonctions qui converge \emph{ponctuellement presque partout} vers une fonction \( f\) qui s'avère être dans \( L^p\). Nous n'avons pas montré que cette suite convergeait au sens de \( L^p\) vers \( f\). Ce que nous devons montrer est que
        \begin{equation}    \label{EqJLfnEvj}
            \| f-f_{\varphi(n)} \|_p\to 0.
        \end{equation}
        L'inégalité \eqref{EqMQbDRac} nous donne aussi, toujours pour presque tout \( x\in \Omega\) :
        \begin{equation}
            \big| f(x)-f_{\varphi(n)}(x) \big|^p\leq g(x)^p
        \end{equation}
        ce qui signifie que la suite\quext{À ce point, \cite{KXjFWKA} se contente de majorer \( | f_{\varphi(n)}(x) |\) par \( | f(x) |+|g(x)\), mais je ne comprends pas comment cette majoration nous permet d'utiliser la convergence dominée de Lebesgue pour montrer \eqref{EqJLfnEvj}.} \(    | f-f_{\varphi(n)} |^p    \) est dominée par la fonction \( | g |^p\) qui est intégrable sur \( \Omega\setminus E\) et tout autant sur \( \Omega\) parce que \( E\) est négligeable; cela prouve au passage le point \ref{ItemPDnjOJzii}, et le théorème de la convergence dominée de Lebesgue (\ref{ThoConvDomLebVdhsTf}) nous dit que
        \begin{equation}
            \lim_{n\to \infty} \int_{\Omega} \big| f(x)-f_{\varphi(n)}(x) \big|^pdx=\int_{\Omega}\lim_{n\to \infty} \big| f(x)-f_{\varphi(n)}(x) \big|dx=0.
        \end{equation}
        Cette dernière suite d'égalités se lit de la façon suivante :
        \begin{equation}
            \lim_{n\to \infty} \| f-f_{\varphi(n)} \|_p=\big\| \lim_{n\to \infty} | f-f_{\varphi(n)} | \big\|_p=0.
        \end{equation}
        Nous en déduisons que la suite \( n\mapsto f_{\varphi(n)}\) est convergente vers \( f\) au sens de la norme \( L^p(\Omega)\). Or la suite de départ \( (f_n)\) était de Cauchy (pour la norme \( L^p\)); donc l'existence d'une sous-suite convergente implique la convergence de la suite entière vers \( f\), ce qu'il fallait démontrer.
    \end{subproof}
\end{proof}

%--------------------------------------------------------------------------------------------------------------------------- 
\subsection{L'espace \texorpdfstring{$L^2$}{$L^2$}}
%---------------------------------------------------------------------------------------------------------------------------
\label{subSecCKZSrZK}

L'espace \( L^2\) est l'espace \( L^p\) définit à la section \ref{SecVKiVIQK} avec \( p=2\). Cependant il possède une propriété extraordinaire\footnote{Tant et si bien que certains n'hésitent pas à définir le nombre \( 2\) comme étant l'unique \( p\) tel que \( L^p\) est un Hilbert.} par rapport aux autres \( L^p\), c'est que la norme \( | . |_2\) dérive d'un produit scalaire. Il sera donc un espace de Hilbert.

Soit \( (\Omega,\tribA,\mu)\) un espace mesuré. Nous considérons l'opération
\begin{equation}    \label{DefProdScalLubrgTj}
    \langle f, g\rangle =\int_{\Omega}f(\omega)\overline{ g(\omega)}d\mu(\omega)
\end{equation}
et la norme associée
\begin{equation}
    \| f \|_2=\sqrt{\langle f, f\rangle }.
\end{equation}
Nous considérons l'ensemble
\begin{equation}
    \mL^2(\Omega,\mu)=\{ f\colon \Omega\to \eR\tq \| f \|_2<\infty \}
\end{equation}
et la relation d'équivalence \( f\sim g\) si et seulement si \( f(x)=g(x)\) pour \( \mu\)-presque tout \( x\). L'espace que nous considérons est
\begin{equation}
    L^2=\mL^2/\sim.
\end{equation}

\begin{lemma}   \label{LemIVWooZyWodb}
    La formule \eqref{DefProdScalLubrgTj} définit un produit scalaire sur \( L^2\), et ce dernier est un espace de Hilbert.
\end{lemma}

\begin{proof}
    D'abord si \( f\) et \( g\) sont dans \( L^2\), alors l'inégalité de Hölder (proposition \ref{ProptYqspT}) nous indique que le produit \( fg\) est un élément de \( L^1\). Par conséquent la formule a un sens.

    Ensuite nous montrons que la formule passe au quotient. Pour cela, nous considérons des fonctions \( \alpha\) et \( \beta\) nulles presque partout et nous regardons le produit de \( f_1=f+\alpha\) par \( g_1=g+\beta\) :
    \begin{equation}
        \langle f_1, g_1\rangle =\int fg+\beta f+\alpha g+ \alpha\beta.
    \end{equation}
    Les fonction \( \beta f\), \( \alpha g\) et \( \alpha\beta\) étant nulles presque partout, leur intégrale est nulle et nous avons bien \( \langle f_1, g_1\rangle =\langle f,g \rangle \). Nous pouvons donc considérer le produit sur l'ensemble des classes.

    Pour vérifier que la formule est un produit scalaire, le seul point non évidement est de prouver que \( \langle f, f\rangle =0\) implique \( f=0\). Cela découle du fait que
    \begin{equation}
        \langle f, f\rangle =\int_{\Omega}| f |^2.
    \end{equation}
    La fonction \( x\mapsto | f(x) |^2\) vérifie les hypothèses du lemme \ref{Lemfobnwt}. Par conséquent \( | f(x) |^2\) est presque partout nulle.

    En ce qui concerne le fait que \( L^2(\Omega)\) soit un espace de Hilbert, il s'agit simplement de se remémorer que c'est un espace complet (théorème  \ref{ThoUYBDWQX}) et dont la norme dérive d'un produit scalaire. Nous sommes donc bien dans la définition \ref{DefORuBdBN}.
\end{proof}

Attention : dans \ref{subSecCKZSrZK}, le produit scalaire a été défini
avec un coefficient \( 1/2\pi\). Voir l'équation \eqref{EqQBcpyyJ}.

Nous notons ici une conséquence du théorème \ref{ThoGVmqOro} dans le cas de l'espace \( L^2\). La proposition suivante est une petite partie du corollaire \ref{CorQETwUdF}, qui sera d'ailleurs démontré de façon indépendante.

\begin{proposition}
    Si nous avons une suite de réels \( (a_k)\) telle que \( \sum_{k=0}^{\infty}| a_k |^2<\infty\) alors la suite
    \begin{equation}
        f_n(x)=\sum_{k=0}^na_k e^{ikx}
    \end{equation}
    converge dans \( L^2\big( \mathopen] 0 , 2\pi \mathclose[ \big)\).
\end{proposition}

\begin{proof}
    Quitte à séparer les parties réelles et imaginaires, nous pouvons faire abstraction du fait que nous parlons d'une série de fonctions à valeurs dans \( \eC\) au lieu de \( \eR\).

    Un simple calcul est :
    \begin{equation}    \label{EqHVdJxZT}
        \| f_n-f_m \|^2\leq\int_0^{2\pi}\sum_{k=n}^m| a_k |^2dx\leq 2\pi\sum_{k=n}^m| a_k |^2.
    \end{equation}
    Par hypothèse le membre de droite est \( | s_m-s_n |\) où \( s_k\) dénote la suite des somme partielle de la série des \( | a_k |^2\). Cette dernière est de Cauchy (parce que convergente dans \( \eR\)) et donc la limite \( n\to\infty\) (en gardant \( m>n\)) est zéro. Donc la suite des \( f_n\) est de Cauchy dans \( L^2\) et donc converge dans \( L^2\).
\end{proof}

%--------------------------------------------------------------------------------------------------------------------------- 
\subsection{Coefficients et série de Fourier}
%---------------------------------------------------------------------------------------------------------------------------
\label{subSecXAYasNI}

Nous utilisons ici des résultats de bases hilbertiennes de la sous-section \ref{SubsecDxkjut}. Nous considérons l'espace de Hilbert \( L^2\mathopen[ -T/2 , T/2 \mathclose]\) muni du produit scalaire\footnote{Attention à la convention : nous mettons un coefficient qui n'est pas dans la définition générale \eqref{DefProdScalLubrgTj}.}
\begin{equation}    \label{EqQBcpyyJ}
    \langle f, g\rangle =\frac{1}{ T }\int_{-T/2}^{T/2}f(t)\overline{ g(t) }dt.
\end{equation}


Pour toute fonction pour laquelle ça a un sens (que ce soit des fonctions \( L^2\) ou non), nous posons
\begin{equation}\label{EqhIPoPH}
    c_n(f)=\frac{1}{ T }\int_{-T/2}^{T/2}f(t) e^{-2i\pi nt/T}dt.
\end{equation}
Ces nombres sont les \defe{coefficients de Fourier}{coefficients!de Fourier} de \( f\). Leur importance dans le cadre de \( L^2\) provient du fait que la famille de fonctions
\begin{equation}
    e_k(t)=  e^{2i\pi kt/T}
\end{equation}
est une base hilbertienne de \( L^2\mathopen[ -T/2 , T/2 \mathclose]\) et que
\begin{equation}
    c_n(f)=\langle f, e_n\rangle .
\end{equation}

Nous pouvons être plus précis. Pour un élément donné \( f\in L^2\big( \mathopen[ 0 , 2\pi \mathclose] \big)\), nous définissons\nomenclature[Y]{\( S_nf\)}{somme partielle de série de Fourier} 
\begin{equation}
    S_nf=\sum_{k=-n}^n\langle f, e_k\rangle e_k
\end{equation}
et nous avons le théorème suivant, qui récompense les efforts consentis à propos de la densité des polynômes trigonométriques dans \( L^2\).

\begin{theorem} \label{ThoYDKZLyv}
    Soit \( f\in L^2\big( \mathopen[ 0 , 2\pi \mathclose] \big)\). Nous avons égalité\footnote{Notons que la somme sur \( \eZ\) dans \eqref{EqXMMRpSN} est commutative; il n'est donc pas besoin d'être plus précis.}
    \begin{equation}    \label{EqXMMRpSN}
        f=\sum_{n\in \eZ}c_n(f)e_n
    \end{equation}
    dans \( L^2\).

    Nous avons aussi la convergence
\begin{equation}    \label{EqRBWKsYP}
    S_nf\stackrel{L^2}{\to} f.
\end{equation}
\end{theorem}

\begin{proof}
    Le système trigonométrique \( \{ e_n \}_{n\in \eZ}\) est total pour l'espace de Hilbert \( L^2\big( \mathopen[ 0 , 2\pi \mathclose] \big)\) (sans périodicité particulière). Donc le point \ref{ItemQGwoIxi} du théorème \ref{ThoyAjoqP} nous donne l'égalité demandée.

    La convergence \eqref{EqRBWKsYP} est une reformulation de l'égalité \eqref{EqXMMRpSN}.
\end{proof}

Si nous voulons une vraie convergence ponctuelle voir uniforme \( (S_nf)(x)\to f(x)\), alors il faut ajouter des hypothèses sur la continuité de \( f\), sa périodicité ou le comportement des coefficients \( c_n\).

\begin{remark}
    Obtenir la convergence \( L^2\) ne demande pas d'hypothèse de périodicité : la convergence \eqref{EqRBWKsYP} est automatique du fait que le système trigonométrique soit total. Ce n'est cependant pas plus qu'une convergence \( L^2\) et elle ne demande pas \( f(0)=f(2\pi)\), même si pour chacun des \( e_k\) nous avons \( e_k(0)=e_k(2\pi)\).

    Si \( f(2\pi)\neq f(0)\), alors il existe une suite \( (f_n)\) convergente vers \( f\) au sens \( L^2\) telle que \( f_n(0)=f_n(2\pi)\).
\end{remark}

%--------------------------------------------------------------------------------------------------------------------------- 
\subsection{Densité des fonctions infiniment dérivables à support compact}
%---------------------------------------------------------------------------------------------------------------------------

\begin{definition}
    Une fonction est \defe{étagée par rapport à \( L^p\)}{fonction!étagée} si elle est de la forme
    \begin{equation}
        f=\sum_{k=1}^Nc_k\mtu_{B_k}
    \end{equation}
    où les \( B_k\) sont des mesurables disjoints et \( \mtu_{B_k}\in L^p\) pour tout \( k\).
\end{definition}

\begin{lemma}   \label{LemWHIRdaX}
    Si \( f\) est une fonction étagée en même temps qu'être dans \( L^p\), alors elle est étagée par rapport à \( L^p\).
\end{lemma}

\begin{proof}
    Nous pouvons écrire
    \begin{equation}
        f=\sum_{k=1}^Nc_k\mtu_{B_k}
    \end{equation}
    où les \( B_k\) sont disjoints. Par hypothèse \( \| f \|_p\) existe. Donc chacune des intégrales \( \int_{\Omega}| \mtu_{B_k} |^p\) doit exister parce que les \( B_k\) étant disjoints, nous pouvons inverser la norme et la somme ainsi que la somme et l'intégrale :
    \begin{equation}
        \int_{\Omega}|f|^p=\int_{\Omega}\sum_{k=1}^N| c_k\mtu_{B_k}(x) |^pdx=\sum_{k=1}^N\int| c_k\mtu_{B_k}(x) |^pdx=\sum_{k=1}^N| c_k |^p\int_{\Omega}| \mtu_{B_k}(x) |^pdx.
    \end{equation}
\end{proof}
Le contraire n'est pas vrai : la fonction étagée sur \( \eR\) qui vaut \( n\) sur \( B(n,\frac{1}{ 4 })\) est étagée par rapport à \( L^p\), mais n'est pas dans \( L^p\).

\begin{theorem}[\cite{TUEWwUN}] \label{ThoILGYXhX}
    Nous avons des densités emboitées. Ici \( D\) est un borélien borné de \( \eR^d\) contenu dans \( B(0,r)\) et \( K\) est un compact contenant \( B(0,r+2)\).
    \begin{enumerate}
        \item
            Les fonctions étagées par rapport à \( L^p\) sur \( \eR^d\) sont denses dans \( L^p(\eR^d)\). A fortiori les fonctions étagées sont denses dans \( L^p\), mais nous n'en aurons pas besoin ici.
        \item\label{ItemYVFVrOIii}
            Il existe une suite \( f_n\) dans \(  C(K,\eC)\) telle que 
            \begin{equation}
                f_n\stackrel{L^p}{\to}\mtu_{D}.
            \end{equation}
        \item\label{ItemYVFVrOIiii}
            Si \( A\) est un borélien tel que \( \mtu_A\in L^p(\eR^d)\)\quext{Je pense que cette hypothèse manque dans \cite{TUEWwUN}. En tout cas je vois mal comment je pourrais justifier les différentes étapes de la preuve en prenant par exemple \( A=\eR^d\).} et si \( \epsilon>0\), alors il existe une suite de boréliens bornée \( (D_n)_{n\in \eN}\) tels que
            \begin{equation}
                \mtu_{D_n}\stackrel{L^p}{\to}\mtu_A.
            \end{equation}
        \item\label{ItemYVFVrOIiv}
            Il existe une suite \( \varphi_n\) dans \(  C^{\infty}_c(\eR^d)\) telle que 
            \begin{equation}
                \varphi_n\stackrel{L^p}{\to}\mtu_{D}.
            \end{equation}

        \item\label{ItemYVFVrOIv}
            L'ensemble \( C^{\infty}_c(\eR^d)\) est dense dans \( L^p(\eR^d)\) pour tout \( 1\leq p<\infty\).
    \end{enumerate}
\end{theorem}
\index{densité!de \( C^{\infty}_c(\eR^d)\) dans \( L^p(\eR^d)\)}

\begin{proof}
   Nous allons montrer les choses point par point.
   \begin{enumerate}
       \item
           Si \( f\in L^1(\eR^d)\), nous savons par la proposition \ref{PropUXjnwLf} qu'il existe une suite \( f_n\) de fonctions étagées convergeant ponctuellement vers \( f\) telle que \( | f_n |\leq | f |\). La proposition \ref{PropBVHXycL} nous dit qu'alors \( f_n\stackrel{L^p}{\to}f\).

           La fonction \( f_n\) étant étagée et dans \( L^p\) en même temps, elle est automatiquement étagée par rapport à \( L^p\) par le lemme \ref{LemWHIRdaX}.

       \item\label{ItemYVFVrOIi}

           C'est le théorème d'approximation \ref{ThoAFXXcVa} appliqué au borélien \( D\) contenu dans l'espace mesuré \( K\).

       \item
            
           En vertu du point \ref{ItemYVFVrOIii}, il existe \( f\in C^0(K,\eR)\) telle que 
           \begin{equation}
             \| f-\mtu_D \|_p\leq \epsilon. 
           \end{equation}
           Ensuite, par le théorème de Weierstrass, il existe \( \varphi\in C^{\infty}(K,\eR)\) telle que \( \| f-\varphi \|_{\infty}\leq \epsilon\). Nous avons aussi
           \begin{equation}
               \| \varphi-f \|^p_p=\int_K| \varphi(x)-f(x) |^pdx\leq\mu(X)\| \varphi-f \|_{\infty}^p\leq \epsilon^p\mu(K).
           \end{equation}
           Quitte à prendre un \( \varphi\) correspondant à un \( \epsilon\) plus petit, nous avons
           \begin{equation}
               \| \varphi-f \|\leq \epsilon.
           \end{equation}
           En combinant et en passant à \( \epsilon/2\) nous avons trouvé une fonction \( \varphi\in  C^{\infty}(K,\eR)\) telle que
           \begin{equation}
               \| \varphi-\mtu_D \|\leq \epsilon.
           \end{equation}

       \item

           Nous considérons les boréliens fermés \( D_n=A\cap B(0,n)\). Alors \( \mtu_{D_n}\in L^p\) et nous avons pour \( n\) assez grand :
           \begin{equation}
               \int_{\eR^d}| \mtu_{D_n}(x)-\mtu_{A}(x) |^pdx=\int_{\eR^d\setminus B(0,n)}| \mtu_A(x) |^p<\epsilon,
           \end{equation}
           c'est à dire que \( \mtu_{D_n}\stackrel{L^p}{\to}\mtu_A\).

       \item

           Il suffit de remettre tout ensemble. Si \( f\in L^p(\eR^d)\), par le point \ref{ItemYVFVrOIi} nous commençons par prendre \( \sigma\) étagée par rapport à \( L^p\) telle que
           \begin{equation}
               \| \sigma-f \|_p\leq\epsilon.
           \end{equation}
           Ensuite nous écrivons \( \sigma\) sous la forme
           \begin{equation}
               \sigma=\sum_{k=1}^Nc_k\mtu_{B_k}
           \end{equation}
           et nous appliquons le point \ref{ItemYVFVrOIiii} à chacune des \( \mtu_{B_k}\) pour trouver des boréliens bornés \( D_k\) tels que
           \begin{equation}
               \| \mtu_{D_k}-\mtu_{B_k} \|_p\leq \epsilon.
           \end{equation}
           Enfin nous appliquons le point \ref{ItemYVFVrOIiv} pour trouver des fonctions \( \varphi_k\in C^{\infty}_c(\eR^d)\) telles que
           \begin{equation}
               \| \varphi_k-\mtu_{D_k} \|_p\leq \epsilon.
           \end{equation}
           
           Il n'est pas compliqué de calculer que
           \begin{equation}
               \big\| \sum_{k=1}^Nc_k\varphi_k-f \big\|_p\leq 2\epsilon\sum_kc_k+\epsilon.
           \end{equation}
        
   \end{enumerate}
\end{proof}

\begin{corollary}   \label{CorFZWooYNbtPz}
    Si \( 1<p<\infty\) alors l'ensemble\footnote{Nous parlons bien ici de l'\emph{ensemble} \( L^2\) parce que nous le considérons sans norme ou topologie particulière. La densité dont nous parlons ici est celle pour la topologique de \( L^p\).} \( L^2\big( \mathopen[ 0 , 1 \mathclose] \big)\cap L^p\big( \mathopen[ 0 , 1 \mathclose] \big)\) est dense dans \( L^p\big( \mathopen[ 0 , 1 \mathclose] \big)\).
\end{corollary}
\index{densité!de \( L^2\big( \mathopen[ 0 , 1 \mathclose] \big)\) dans \( L^p\big( \mathopen[ 0 , 1 \mathclose] \big)\)}

\begin{proof}
    Nous savons du théorème \ref{ThoILGYXhX}\ref{ItemYVFVrOIv} que \(  C^{\infty}_c\big( \mathopen[ 0 , 1 \mathclose] \big)\) est dense dans \( L^p\). Mais nous avons évidemment \(  C^{\infty}_c\subset L^2\cap L^p\), donc \( L^2\cap L^p\) est dense dans \( L^p\).
\end{proof}

\begin{lemma}       \label{LemDQEKNNf}
    Soit \( f\in L^2(I)\) telle que 
    \begin{equation}
        \int_If\varphi=0
    \end{equation}
    pour toute fonction \( \varphi\in C^{\infty}_c(I)\). Alors \( f=0\) presque partout sur \( I\).
\end{lemma}

\begin{proof}
    Nous considérons la forme linéaire
    \begin{equation}
        \begin{aligned}
            \phi\colon L^2(I)&\to \eC \\
            g&\mapsto \langle f, g\rangle=\int_If\bar g . 
        \end{aligned}
    \end{equation}
    Par densité\footnote{Théorème \ref{ThoILGYXhX}\ref{ItemYVFVrOIv}.} nous pouvons aussi considérer une suite \( (\varphi_n)\) dans \(  C^{\infty}_c(I)\) convergent vers \( f\) dans \( L^2\). Alors nous avons pour tout \( n\) :
    \begin{equation}
        \langle f, \varphi_n\rangle =0.
    \end{equation}
    En passant à la limite, \( \langle f, f\rangle =0\), ce qui implique \( f=0\) dans \( L^2\) et donc \( f=0\) presque partout en tant que bonne fonction.
\end{proof}
Ce résultat est encore valable dans les espaces \( L^p\) (proposition \ref{PropUKLZZZh}), mais il demande le théorème de représentation de Riesz\footnote{Théorème \ref{ThoSCiPRpq}.}.

\begin{lemma}[\cite{TUEWwUN}]   \label{LemCUlJzkA}
    Soit \( 1\leq p<\infty\) et \( f\in L^p(\Omega)\). Nous notons \( \tau_v\) l'opérateur de translation par \( v\) :
    \begin{equation}
        \tau_vf\colon x\mapsto f(x-v).
    \end{equation}
    Alors sans surprises,
    \begin{equation}
        \lim_{v\to 0}\| \tau_v-f \|_p=0.
    \end{equation}
\end{lemma}

%--------------------------------------------------------------------------------------------------------------------------- 
\subsection{Dualité et théorème de représentation de Riesz}
%---------------------------------------------------------------------------------------------------------------------------

Parmi les théorèmes de dualité nous avons
\begin{enumerate}
    \item
        Le théorème de représentation de Riesz \ref{ThoQgTovL} pour les espaces de Hilbert.
    \item
        La proposition \ref{PropOAVooYZSodR} pour les espaces \( L^p\big( \mathopen[ 0 , 1 \mathclose] \big)\) avec \( 1<p<2\).
    \item
        Le théorème de représentation de Riesz \ref{ThoSCiPRpq} pour les espaces \( L^p\) en général.
\end{enumerate}
\index{dual!espaces \( L^p\)}
Tous ces théorèmes donnent la dualité par l'application \( \Phi_x=\langle x, .\rangle \).

Dans la suite \( E'\) est le dual topologique, c'est à dire l'espace des formes linéaires et continues sur \( E\).

\begin{proposition}[\cite{PAXrsMn}] \label{PropOAVooYZSodR}
    Soit \( 1<p<2\) et \( q\) tel que \( \frac{1}{ p }+\frac{1}{ q }=1\). L'application 
    \begin{equation}
        \begin{aligned}
            \Phi\colon L^q\big( \mathopen[ 0 , 1 \mathclose] \big)&\to  L^p\big( \mathopen[ 0 , 1 \mathclose] \big)'  \\
            \Phi_g(f)&= \int_{\mathopen[ 0 , 1 \mathclose]}f\bar g.
        \end{aligned}
    \end{equation}
    est une isométrie linéaire surjective.
\end{proposition}
\index{dual!de \( L^p\big( \mathopen[ 0 , 1 \mathclose] \big)\) avec \( 1<p<2\)}

\begin{proof}
    Pour la simplicité des notations nous allons noter \( L^2\) pour \( L^2\big( \mathopen[ 0 , 1 \mathclose] \big)\), et pareillement pour \( L^p\).
    \begin{subproof}
        \item[\( \Phi_g\) est un élément de \( (L^p)'\)]

            Si \( f\in L^p\) et \( g\in L^q\) nous devons prouver que \( \Phi_q(f)\) est bien définie. Pour cela nous utilisons l'inégalité de Hölder\footnote{Proposition \ref{ProptYqspT}.} qui dit que \( fg\in L^1\); par conséquent la fonction \( f\bar g\) est également dans \( L^1\) et nous avons
            \begin{equation}
                | \Phi_g(f) |\leq\int_{\mathopen[ 0 , 1 \mathclose]}| f\bar g |=\| fg \|_1\leq \| f \|_p\| g \|_q.
            \end{equation}
            En ce qui concerne la norme de l'application \( \Phi_g\) nous avons tout de suite
            \begin{equation}
                \| \Phi_g \|=\sup_{\| f\|_p=1}\big| \Phi_g(f) \big|\leq \| g \|_q.
            \end{equation}
            Cela signifie que l'application \( \Phi_g\) est bornée et donc continue par la proposition \ref{PropmEJjLE}. Nous avons donc bien \( \Phi_g\in (L^p)'\).

        \item[Isométrie]

            Afin de prouver que \( \| \Phi_g \|=\| g \|_q\) nous allons trouver une fonction \( f\in L^p\) telle que \( \frac{ | \Phi_g(f) | }{ \| f \|_p }=\| g \|_q\).  De cette façon nous aurons prouvé que \( | \Phi_g |\geq \| g \|_q\), ce qui conclurait que \( | \Phi_g |=\| g \|_q\).

            Nous posons \( f=g| g |^{q-2}\), de telle sorte que \( | f |=| g |^{q-1}\) et
            \begin{equation}
                \| f \|_p=\left( \int| g |^{p(q-1)} \right)^{1/p}=\left( \int | g |^q \right)^{1/p}=\| g \|_q^{q/p}
            \end{equation}
            où nous avons utilisé le fait que \( p(q-1)=q\). La fonction \( f\) est donc bien dans \( L^p\). D'autre part,
            \begin{equation}
                \Phi_g(f)=\int f\bar g=\int g| g |^{q-2}\bar g=\int | g |^q=\| g \|_q^q.
            \end{equation}
            Donc
            \begin{equation}
                \frac{ | \Phi_g(f) | }{ \| f \|_p }=\| g \|_q^{q-\frac{ q }{ p }}=\| g \|_q
            \end{equation}
            où nous avons encore utilisé le fait que \( q-\frac{ q }{ p }=\frac{ q(p-1) }{ p }=1\).

        \item[Surjectif]

            Soit \( \ell\in (L^p)'\); c'est une application \( \ell\colon L^p\to \eC\) sont nous pouvons prendre la restriction à \( L^2\) parce que la proposition \ref{PropIRDooFSWORl} nous indique que \( L^2\subset L^p\). Nous nommons \( \phi\colon L^2\to \eC\) cette restriction.

            \begin{subproof}

                \item[\( \phi\in (L^2)'\)]
                
                    Nous devons montrer que \( \phi\) est continue pour la norme sur \( L^2\). Pour cela nous montrons que sa norme opérateur (subordonnée à la norme de \( L^2\) et non de \( L^p\)) est finie :
                    \begin{equation}
                        \sup_{f\in L^2}\frac{ | \phi(f) | }{ \| f \|_{2} }\leq \sup_{f\in L^2}\frac{ | \ell(f) | }{ \| f \|_p }<\infty.
                    \end{equation}
                    Nous avons utilisé l'inégalité de norme \( \| f \|_p\leq \| f \|_2\) de la proposition \ref{PropIRDooFSWORl}\ref{ItemWSTooLcpOvXii}.

                \item[Utilisation du dual de \( L^2\)]

                    Étant donné que \( L^2\) est un espace de Hilbert (lemme \ref{LemIVWooZyWodb}) et que \( \phi\in (L^2)'\), le théorème \ref{ThoQgTovL} nous donne un élément \( g\in L^2\) tel que \( \phi(f)=\Phi_g(f)\) pour tout \( f\in L^2\).

                    Nous devons prouver que \( g\in L^q\) et que pour tout \( f\in L^p\) nous avons \( \ell(f)=\Phi_g(f)\).

                \item[\( g\in L^q\)]

                    Nous posons \( f_n=g| g |^{q-2}\mtu_{| g |<n}\). Nous avons d'une part
                    \begin{equation}    \label{EqEBUooOnlRHj}
                        \Phi_g(f_n)=\int_0^1f_n\bar g=\int_{| g |<n}| g |^q.
                    \end{equation}
                    Et d'autre part comme \( f_n\in L^2\) nous avons aussi \( \phi(f_n)=\Phi_g(f_n)\) et donc
                    \begin{subequations}
                        \begin{align}
                            0\leq \Phi(f_n)= \phi(f_n)&\leq \| \ell \|\| f_n \|_p\\
                            &=\| \ell \|\left( \int_{| g |<n}| g |^{(q-1)p} \right)^{1/p}\\
                            &=\| \ell \|\left( \int_{| g |<n}| g |^q \right)^{1/p}.
                        \end{align}
                    \end{subequations}
                    où nous avons à nouveau tenu compte du fait que \( p(q-1)=q\). En combinant avec \eqref{EqEBUooOnlRHj} nous trouvons
                    \begin{equation}
                        \int_{| g |<n}| g |^q\leq \| \ell \|\left( \int_{| g |<n}| g |^q \right)^{1/p},
                    \end{equation}
                    et donc
                    \begin{equation}
                        \left( \int_{| g |<n}| g |^{q} \right)^{1-\frac{1}{ p }}\leq \| \ell \|, 
                    \end{equation}
                    c'est à dire
                    \begin{equation}
                        \Big( \int_{| g |<n}| g |^q \Big)^{1/q}\leq \| \ell \|.
                    \end{equation}
                    
                    Si ce n'était pas encore fait nous nous fixons un représentant de la classe \( g\) (qui est dans \( L^2\)), et nous nommons également \( g\) ce représentant. Nous posons alors
                    \begin{equation}
                        g_n=| g |^q\mtu_{| g |<n}
                    \end{equation}
                    qui est une suite croissante de fonctions convergeant ponctuellement vers \( | g |^q\). Le théorème de Beppo-Levi \ref{ThoRRDooFUvEAN} nous permet alors d'écrire
                    \begin{equation}
                        \lim_{n\to \infty} \int_{| q |<n}| g |^q=\int_{0}^1| g |^q.
                    \end{equation}
                    Mais comme pour chaque \( n\) nous avons \( \int_{| g |<n}| q |^q\leq \| \ell \|^q\), nous conservons l'inégalité à la limite et
                    \begin{equation}
                        \int_0^1| g |^q\leq \| \ell \|^q.
                    \end{equation}
                    Cela prouve que \( g\in L^p\).

                \item[\( \ell(f)=\Phi_g(f)\)]

                    Soit \( f\in L^p\). En vertu de la densité de \( L^2\) dans \( L^p\) prouvée dans le corollaire \ref{CorFZWooYNbtPz} nous pouvons considérer une suite \( (f_n)\) dans \( L^2\) telle que \( f_n\stackrel{L^p}{\longrightarrow}f\). Pour tout \( n\) nous avons
                    \begin{equation}
                        \ell(f_n)=\Phi_g(f_n).
                    \end{equation}
                    Mais \( \ell\) et \( \Phi_g\) étant continues sur \( L^p\) nous pouvons prendre la limite et obtenir
                    \begin{equation}
                        \ell(f)=\Phi_g(f).
                    \end{equation}
            \end{subproof}
        \end{subproof}
\end{proof}

\begin{definition}
    Un espace \( V\) est \defe{réflexif}{réflexif} si l'injection naturelle \( V\to V'\) est surjective.
\end{definition}

\begin{theorem}[Théorème de représentation de Riesz\cite{LRBWftc,OYRmzAa}] \label{ThoSCiPRpq}
    Nous considérons pour ce théorème des fonctions à valeurs réelles.

    Soit espace mesuré \( (\Omega,\tribA,\mu)\) et \( 1<p<\infty\).
    \begin{enumerate}
        \item
            L'espace \( L^p(\Omega)\) est réflexif.
        \item
            Si \( q\) est le nombre tel que \( \frac{1}{ p }+\frac{1}{ q }=1\) alors l'application
            \begin{equation}
                \begin{aligned}
                    \Phi\colon L^q&\to (L^p)' \\
                    u&\mapsto \Big( \Phi_u\colon f\to \int_{\Omega}fu\,d\mu \Big) 
                \end{aligned}
            \end{equation}
            est une bijection isométrique.
    \end{enumerate}
    Si \( \mu\) est une mesure \( \sigma\)-finie, alors
    \begin{enumerate}
        \item
            \( (L^1)'=L^{\infty}\)
        \item
            \( L^1\subset (L^{\infty})' \) avec une inclusion stricte sauf dans les cas triviaux.
    \end{enumerate}
\end{theorem}
% TODO : la preuve est sur Wikipédia.
\index{théorème!représentation de Riesz}
\index{dual!de \( L^p(\Omega)\)}

\begin{proposition} \label{PropUKLZZZh}
    Soit \( f\in L^p(\Omega)\) telle que
    \begin{equation}
        \int_{\Omega}f\varphi=0
    \end{equation}
    pour tout \( \varphi\in C^{\infty}_c(\Omega)\). Alors \( f=0\) presque partout.
\end{proposition}

\begin{proof}
    Nous considérons la forme linéaire \( \Phi_f\in (L^q)'\) donnée par
    \begin{equation}
        \begin{aligned}
            \Phi_f\colon L^p&\to \eC \\
            u&\mapsto \int_{\Omega}fu
        \end{aligned}
    \end{equation}
    Par hypothèse cette forme est nulle sur la partie dense \(  C^{\infty}_c(\Omega)\). Si \( (\varphi_n)\) est une suite dans \(  C^{\infty}_c(\Omega)\) convergente vers \( u\) dans \( L^p\), nous avons pour tout \( n\) que
    \begin{equation}
        0=\Phi_f(\varphi_n)
    \end{equation}
    En passant à la limite, nous voyons que \( \Phi_f\) est la forme nulle. Elle est donc égale à \( \Phi_0\). La partie «unicité» du théorème de représentation de Riesz \ref{ThoSCiPRpq} nous indique alors que \( f=0\) dans \( L^p\) et donc \( f=0\) presque partout.
\end{proof}

%--------------------------------------------------------------------------------------------------------------------------- 
\subsection{Approximation de l'unité}
%---------------------------------------------------------------------------------------------------------------------------

Nous considérons \( \Omega=\eR^d\) ou \( (S^1)^d\).

\begin{definition}
    Une \defe{approximation de l'unité}{approximation!de l'unité} sur \( \Omega\) est une suite \( (\varphi_n)\) dans \( L^1(\Omega)\) telle que
    \begin{subequations}
        \begin{align}
            \sup_k \| \varphi_k \|_1 <\infty\\
            \int_{\Omega}\varphi_n=1        \label{subeqAQcisBt}\\
            \lim_{k\to \infty} \int_{\Omega\setminus B(0,\alpha)}| \varphi_k |=0
        \end{align}
    \end{subequations}
    pour tout \( n\) et pour tout \( \alpha>0\).
\end{definition}
%TODO : voir si ça n'approxime pas un delta de Dirac d'une façon ou d'une autre.
Ce sont des fonctions dont la masse vient s'accumuler autour de zéro. En effet quel que soit le voisinage \( B(0,\alpha)\), si \( k\) est assez grand, il n'y a presque plus rien en dehors.

Pour le point \ref{subeqAQcisBt}, si \( \Omega\) est \( S^1\), la mesure que nous considérons est \( \frac{ dx }{ 2\pi }\).


\begin{example}
    Une façon de construire une approximation de l'unité sur \( \eR\) est de considérer une fonction \( \varphi\in L^1(\Omega)\) telle que \( \int\varphi=1\) puis de poser
    \begin{equation}
        \varphi_k(x)=k^d\varphi(kx).
    \end{equation}
    Ici, \( \Omega\) peut être \( \eR\) ou \( S^1\).
\end{example}

Le lemme suivant permet de construire des approximations de l'unité intéressantes.
\begin{lemma}[\cite{TUEWwUN}]   \label{LemCNjIYhv}
    Si nous posons
    \begin{equation}
        \varphi_n(x)=\left( \int\varphi(y)^n \right)^{-1}\varphi(x)^n,
    \end{equation}
    alors nous obtenons une approximation de l'unité dans les deux cas suivants :
    \begin{enumerate}
        \item
            Soit \( \varphi\) une fonction continue et positive sur \( S^1\) telle que \( \varphi(x)<\varphi(0)\) pour tout \( x\notin 2\pi \eZ\). Dans ce cas la mesure à prendre pour l'intégrale est \( \frac{ dy }{ 2\pi }\).
        \item
            Soit \( \varphi\) est une fonction continue et positive à support compact sur \( \eR^d\) telle que \( \varphi(x)>\varphi(0)\) pour tout \( x\neq 0\).
            
    \end{enumerate}
\end{lemma}

\begin{theorem}[\cite{TUEWwUN}] \label{ThoYQbqEez}
    Soit \( (\varphi_k)\) une approximation de l'unité sur \( \Omega=\eR^d\) ou \( (S^1)^d\).
    \begin{enumerate}
        \item
            Si \( g\) est mesurable et bornée sur \( \Omega\) et si \( g\) est continue en \( x_0\) alors 
            \begin{equation}
                (\varphi_k*g)(x_0)\to g(x_0).
            \end{equation}
        \item
            Si \( g\in L^p(\Omega)\) (\( 1\leq p<\infty\)) alors
            \begin{equation}
                \varphi_k*g\stackrel{L^p}{\to}g.
            \end{equation}
        \item
            Si \( g\) est uniformément continue et bornée, alors
            \begin{equation}
                \varphi_k*g\stackrel{L^{\infty}}{\to}g
            \end{equation}
    \end{enumerate}
\end{theorem}

\begin{proof}
    Les trois points vont se ressembler.
    \begin{enumerate}
        \item
            Nous notons \( d_k=(\varphi_k*g)(x_0)-g(x_0)\) et nous devons prouver que \( d_k\to 0\). Vu que \( \varphi_k\) est d'intégrale \( 1\) sur \( \Omega\) nous pouvons écrire
            \begin{equation}
                |d_k|=\big| \int_{\Omega}\big( g(x_0-y)-g(x_0) \big)\varphi_k(y)dy\big|\leq\int_{\Omega}\big| g(x_0-y)-g(x_0) \big| |\varphi_k(y) |dy. 
            \end{equation}
            Nous notons \( M=\sup_k\| \varphi_k \|_1\), et nous considérons \( \alpha>0\) tel que
            \begin{equation}
                \big| g(x_0-y)-g(x_0) \big|\leq \epsilon.
            \end{equation}
            Nous nous restreignons maintenant aux \( k\) suffisamment grands pour que \( \int_{\complement B(0,\alpha)}| \varphi_k(y) |dy\leq \epsilon\). Alors en découpant l'intégrale en \( B(0,\alpha)\) et son complémentaire dans \( \Omega\),
            \begin{equation}
                | d_k |\leq \epsilon M+\int_{\complement B(0,\alpha)} 2\| g \|_{\infty}| \varphi_k(y) |dy  \leq \epsilon M+2\| g \|_{\infty}\epsilon\leq \epsilon C.
            \end{equation}
            Donc oui, nous avons \( | d_k |\to 0\), et donc le premier point du théorème.

        \item

            Nous posons \( d_k(x)=(\varphi_k*g)(x)-g(x)\) et nous voulons prouver que \( \| d_k \|_{\infty}\to 0\), c'est à dire que \( d_k(x)\) converge vers zéro uniformément en \( x\). Nous posons aussi 
            \begin{equation}
                \tau_y(g)\colon x\mapsto g(x-y).
            \end{equation}
            En récrivant le produit de convolution, une petite majoration donne
            \begin{equation}
                | d_k(x) |\leq \int_{\Omega}\| \tau_y(g)-g \|_{\infty}| \varphi_k(y) |dy.
            \end{equation}
            L'uniforme continuité de \( g\) signifie que pour tout \( \epsilon\), il existe un \( \alpha\) tel que pour tout \( y\in B(0,\alpha)\),
            \begin{equation}
                \| \tau_y(g)-g \|_{\infty}\leq \epsilon.
            \end{equation}
            Encore une fois nous découpons le domaine d'intégration en \( B=B(0,\alpha)\) et son complémentaire :
            \begin{subequations}
                \begin{align}
                    \| d_k \|_{\infty}&\leq\int_B\underbrace{\| \tau_y(g)-g \|_{\infty}}_{\leq \epsilon}| \varphi_k(y) |dy+\int_{\complement B}\underbrace{\| \tau_y(g)-g \|_{\infty}}_{\leq 2\| g \|_{\infty}}| \varphi_k(y) |\\
                    &\leq \epsilon M+2\| g \|_{\infty}\epsilon
                \end{align}
            \end{subequations}
            où la seconde ligne est justifiée par le choix d'un \( k\) assez grand pour que \( \int_{\complement B}| \varphi_k(y) |dy\leq \epsilon\).

            Nous avons donc bien \( \| d_k \|_{\infty}\to 0\).

        \item

            Cette fois \( g\in L^p(\Omega)\) et nous cherchons à montrer que \( \| d_k \|_p\to 0\). Encore qu'ici \( d_k\) soit défini à partir d'un représentant dans la classe de \( g\) et que d'ailleurs, nous allons travailler avec ce représentant.

            D'abord nous développons un peu ce \( d_k\) :
            \begin{subequations}
                \begin{align}
                \| d_k \|_p&=\left[ \int_{\Omega}\left|     \int_{\Omega}\big( g(x-y)-g(x) \big)\varphi_k(y)dy  \right|^pdx \right]^{1/p}\\
                &\leq\left[    \int_{\Omega}\Big( \int_{\Omega}| g(x-y)-g(x) |\cdot |\varphi_k(y) |dy \Big)^pdx \right]^{1/p}.
                \end{align}
            \end{subequations}
            À cette dernière expression nous appliquons l'inégalité de Minkowski (théorème \ref{PropInegMinkKUpRHg}) sous la forme \eqref{EqZSiTZrH} pour la norme \( d\nu(y)=| \varphi_k(y) |dy\) et \( f(x,y)=g(x-y)-g(x)\) :
            \begin{equation}
                \| d_k \|_p\leq \int_{\Omega}\Big( \int_{\Omega}\big| g(x-y)-g(x) \big|^pdx \Big)^{1/p}| \varphi_k(y) |dy=\int_{\Omega}\| \tau_yg-g \|_p| \varphi_k(y) |dy.
            \end{equation}
            Par le lemme \ref{LemCUlJzkA} nous pouvons trouver \( \alpha>0\) tel que \( \| \tau_yg-g \|_p\leq \epsilon\) pour tout \( y\in B(0,\alpha)\). Avec cela nous découpons encore le domaine d'intégration :
            \begin{equation}
                \| d_k \|_p\leq \int_{B(0,\alpha)}\underbrace{\| \tau_yg-g \|_p}_{\leq \epsilon}| \varphi_k(y) |dy+\int_{\complement B(0,\alpha)}  \underbrace{\| \tau_yg-g \|_p}_{\leq 2\| g \|_p}| \varphi_k(y) |dy\leq \epsilon M+2\epsilon\| g \|_p.
            \end{equation}
    \end{enumerate}
\end{proof}

Une petite remarque en passant : aussi triste que cela en ait l'air, la convergence uniforme n'implique pas la convergence \( L^p(\Omega)\) si \( \Omega\) n'est pas borné. En effet si \( f\in L^p\), la suite donnée par
\begin{equation}
    f_n(x)=f(x)+\frac{1}{ n }
\end{equation}
converge uniformément vers \( f\), mais 
\begin{equation}
    \| f_n-f \|_p=\int_{\Omega}\frac{1}{ n }
\end{equation}
n'existe même pas si le domaine \( \Omega\) n'est pas borné.

%---------------------------------------------------------------------------------------------------------------------------
\subsection{Densité des polynôme trigonométrique}
%---------------------------------------------------------------------------------------------------------------------------

Nous allons beaucoup travailler avec le \defe{système trigonométrique}{système!trigonométrique} donné par \( \{ e_n \}_{n\in \eZ}\) et
\begin{equation}
    e_n(t)= e^{int}.
\end{equation}
Une bonne partie de la douleur qu'évoque mot «densité» consiste à montrer que ce système est total dans \( L^2(S^1)=L^2(\mathopen[ 0 , 2\pi \mathclose])\), et donc en est une base hilbertienne. Un \defe{polynôme trigonométrique}{polynôme!trigonométrique} est une fonction de la forme \( P=\sum_{k=-N}^NP_ke_k\) pour des constantes \( P_k\).

Un \defe{polynôme trigonométrique}{polynôme!trigonométrique} est une fonction de la forme
\begin{equation}
    P(t)=\sum_{n=-N}^Nc_n e^{int}.
\end{equation}

\begin{lemma}   \label{LemZVfZlms}
    Deux petits résultats simples mais utiles à propos des polynômes trigonométriques.
    \begin{enumerate}
        \item
    Si \( f\in L^1(S^1)\), alors nous avons la formule
    \begin{equation}
        f*e_n=c_n(f)e_n.
    \end{equation}
\item
            
    Si \( P\) est un polynôme trigonométrique et si \( f\in L^1(S^1)\) alors \( f*P\) est encore un polynôme trigonométrique.
    \end{enumerate}
\end{lemma}

\begin{proof}
    Le premier point est un simple calcul :
    \begin{subequations}
        \begin{align}
            (f*e_n)(x)&=\frac{1}{ 2\pi }\int_0^{2\pi}f(x-t)e_n(t)dt\\
            &= e^{inx}\int_0^{2\pi}f(x-t) e^{-in(x-t)}dt\\
            &=c_n(f)e_n(x).
        \end{align}
    \end{subequations}

    En ce qui concerne le second point, nous notons \( P=\sum_{k=-N}^NP_ke_k\), et par linéarité de la convolution,
    \begin{equation}
        f*P=\sum_{k=-N}^NP_kf*e_k=\sum_{k=-N}^nc_k(f)e_k,
    \end{equation}
    qui est encore un polynôme trigonométrique.
\end{proof}

\begin{example} \label{ExDMnVSWF}
    Sur \( S^1\) nous construisons alors l'approximation de l'unité basée sur la fonction \( 1+\cos(x)\) et le lemme \ref{LemCNjIYhv}. Cette fonction est évidemment un polynôme trigonométrique parce que 
    \begin{equation}
        \cos(x)=\frac{  e^{ix}+ e^{-ix} }{2}.
    \end{equation}
    Ensuite les puissances le sont aussi à cause de la formule du binôme :
    \begin{equation}
        \big( 1+\cos(x) \big)^n=\sum_{k=0}^n\binom{ n }{ k }\cos^n(x),
    \end{equation}
    dans laquelle nous pouvons remettre \( \cos(x)\) comme un polynôme trigonométrique et développer à nouveau la puissance avec (encore) la formule du binôme. La chose importante est qu'il existe une approximation de l'unité \( (\varphi_n)\) formée de polynômes trigonométrique.

    Ce qui fait la spécificité des polynômes trigonométriques est qu'ils sont à la fois stables par convolution (lemme \ref{LemZVfZlms}) et qu'ils permettent de créer une approximation de l'unité sur \( \mathopen[ 0 , 2\pi \mathclose]\). Ce sont ces deux choses qui permettent de prouver l'important théorème suivant.
\end{example}

\begin{theorem} \label{ThoQGPSSJq}
    Les polynôme trigonométriques sont dense dans \( L^p(S^1)\) pour \( 1\leq p<\infty\).
\end{theorem}
\index{densité!des polynômes trigonométriques dans \( L^p(S^1)\)}

\begin{proof}
    Soit \( f\in L^p(S^1)=L^p\big( \mathopen[ 0 , 2\pi \mathclose] \big)\), et \( (\varphi_n)\) une approximation de l'unité sur \( S^1\) formée de polynômes trigonométriques, par exemple ceux de l'exemple \ref{ExDMnVSWF}. D'abord \( \varphi_k*f\) est un polynôme trigonométrique par le lemme \ref{LemZVfZlms}, et ensuite nous avons convergence
    \begin{equation}
        \varphi_k*f\stackrel{L^p}{\to}f
    \end{equation}
    par le théorème \ref{ThoYQbqEez}. Nous avons donc convergence \( L^p\) d'une suite de polynômes trigonométrique, ce qui prouve que l'espace de polynômes trigonométriques est dense dans \( L^p(S^1)\).
\end{proof}

\begin{remark}
    Deux remarques.
    \begin{itemize}
        \item 
            Il n'est pas possible que les polynômes trigonométriques soient dense dans \( L^{\infty}\) parce qu'une limite uniforme de fonctions continues est continue (c'est le théorème \ref{ThoUnigCvCont}).
        \item 
            Nous donnerons au théorème \ref{ThoDPTwimI} une démonstration indépendante de la densité des polynômes trigonométriques dans \( L^p(S^1)\).
    \end{itemize}
\end{remark}
