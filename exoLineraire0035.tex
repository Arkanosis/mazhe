% This is part of the Exercices et corrigés de mathématique générale.
% Copyright (C) 2009
%   Laurent Claessens
% See the file fdl-1.3.txt for copying conditions.
\begin{exercice}\label{exoLineraire0035}

	Exercice 9, page 90. On donne\footnote{si si, on vous la donne, vous pouvez la garder.} la matrice
	\begin{equation}
		M=\begin{pmatrix}
			a	&	0	&	c\\ 
			2	&	a	&	3(b-1)\\ 
			-3b	&	c	&	a	  
		\end{pmatrix}.
	\end{equation}
	\begin{enumerate}

		\item
			Déterminer les valeurs $c$ pour que $\det M=0$ quand $a=1$.

		\item
			Pour ces valeurs de $c$, déterminer les valeurs propres et les vecteurs propres de $M$. Dans quel(s) cas existe-t-il une base formée de vecteurs propres ?

	\end{enumerate}
	

\corrref{Lineraire0035}
\end{exercice}
