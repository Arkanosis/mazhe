% This is part of Mes notes de mathématique
% Copyright (c) 2011-2013,2016
%   Laurent Claessens
% See the file fdl-1.3.txt for copying conditions.

%+++++++++++++++++++++++++++++++++++++++++++++++++++++++++++++++++++++++++++++++++++++++++++++++++++++++++++++++++++++++++++
\section{Originalité}
%+++++++++++++++++++++++++++++++++++++++++++++++++++++++++++++++++++++++++++++++++++++++++++++++++++++++++++++++++++++++++++

Ces notes ne sont pas originales par leur contenu : ce sont toutes des choses qu'on trouve facilement sur internet; je crois que la bibliographie est éloquente à ce sujet. Ce cours se distingue des autres sur les points suivants.
\begin{description}
    \item[La longueur] J'ai décidé de ne pas me soucier de la taille du fichier. Il fera cinq mille pages s'il le faut, mais il restera en un bloc. Étant donné qu'il n'existe qu'une seule mathématique, il ne m'a pas semblé intéressant de produire une division artificielle entre l'analyse, la géométrie ou l'algèbre. Tous le résultats d'une branche peuvent (et sont) être utilisés dans toutes les autres branches.

        Dans cette optique, je me suis évertué à ne créer que des références «vers le haut». À moins d'oubli de ma part\footnote{Par exemple pour les théorèmes pour lesquels je n'ai pas lu ni a fortiori écrit de preuves.}, il n'y a aucun endroit du texte qui dépend d'un lemme démontré plus bas. Le fait qu'un théorème \( B\) soit plus bas qu'un théorème \( A\) signifie qu'on peut démontrer \( A\) sans savoir \( B\).

    \item[La licence] Ce document est publié sous une licence libre. Elle vous donne explicitement le droit de copier, modifier et redistribuer. 

    \item[Les mises à jour] Ce document est régulièrement mis à jour. Des fautes d'orthographe sont corrigées (presque) chaque jour. Si vous me signalez une faute de mathématique, elle sera corrigée.
    \item[Transparence] Je ne fais pas semblant que ces notes soient parfaites. Les points sur lesquels je ne suis pas sûr, les preuves que j'ai inventées moi-même sont clairement indiqués pour inciter le lecteur à redoubler de prudence. Une liste de questions à résoudre est inclue en la section \ref{SecooCKWWooBFgnea}. De plus de nombreuses notes en bas de page en fonte \info{texttt}\quext{Comme celle-ci} indiquent des points sur lesquels je doute ou des étapes intermédiaires de calculs que je ne parviens pas à reproduire en suivant mes sources. Lorsque vous voyez une telle note, redoublez de prudence, et allez voir la source.
        
\end{description}

%+++++++++++++++++++++++++++++++++++++++++++++++++++++++++++++++++++++++++++++++++++++++++++++++++++++++++++++++++++++++++++
\section{Propagande : utilisez un ordinateur !}
%+++++++++++++++++++++++++++++++++++++++++++++++++++++++++++++++++++++++++++++++++++++++++++++++++++++++++++++++++++++++++++

Si vous faites des exercices supplémentaires et que vous voulez des corrections, n'oubliez pas que vous avez un ordinateur à disposition. De nos jours, les ordinateurs sont capables de calculer à peu près tout ce qui se trouve dans vos cours de math\footnote{Avec une notable exception pour les limites à deux variables.}.

%Par ailleurs, le vingt et unième siècle est déjà largement entamé; si vous vous lancez dans une carrière scientifique\footnote{Si vous lisez ces notes, c'est le cas, même si vous voulez entrer dans l'éducation nationale.}, il vous faudra maitriser l'informatique un peu plus solidement qu'être virtuose es trouver le trajet le plus court sur le téléphone.

%---------------------------------------------------------------------------------------------------------------------------
\subsection{Lancez-vous dans Sage}
%---------------------------------------------------------------------------------------------------------------------------

Le logiciel que je vous propose est \href{http://www.sagemath.org}{Sage}. C'est depuis 2012 un logiciel disponible pour l'épreuve de modélisation de l'agrégation en mathématique. Pour l'utiliser, il n'est même pas nécessaire de l'installer sur votre ordinateur~: il tourne en ligne, directement dans votre navigateur.

\begin{enumerate}

	\item
        Aller sur \url{http://www.sagenmath.org},
	\item
		créer un compte,
	\item
		créer des feuilles de calcul et amusez-vous !!

\end{enumerate}

Il y a beaucoup de \href{http://lmgtfy.com/?q=sage+documentation}{documentation} sur le \href{http://www.sagemath.org}{site officiel}\footnote{\href{http://www.sagemath.org}{http://www.sagemath.org}}.

Si vous comptez utiliser régulièrement ce logiciel, je vous recommande \emph{chaudement} de \href{http://mirror.switch.ch/mirror/sagemath/index.html}{l'installer} sur votre ordinateur.

%---------------------------------------------------------------------------------------------------------------------------
\subsection{Exemples de ce que Sage peut faire pour vous}
%---------------------------------------------------------------------------------------------------------------------------

Voici une liste (non exhaustive) de ce que Sage peut faire pour vous.
\begin{enumerate}

	\item
        Calculer des limites de fonctions, exemples \ref{ExBCRXooDVUdcf} et \ref{ExCWDRooKxnjGL}.
	\item
        Tracer des graphes de fonctions, exemple \ref{ExCWDRooKxnjGL}.
	\item
        Tracer des courbes en trois dimensions, voir exemple \ref{ExempleTroisDxxyy}. 
	\item
		Calculer des dérivées partielles de fonctions à plusieurs variables, voir exemple \ref{exJMGTooZcZYNy}.
	\item
        Résoudre des systèmes d'équations linéaires. Voir les exemples \ref{exKGDIooVefujD} et \ref{ExBGCEooPIQgGW}. Lire aussi \href{http://www.sagemath.org/doc/constructions/linear_algebra.html#solving-systems-of-linear-equations}{la documentation}.
	\item
        Tout savoir d'une forme quadratique, voir exemple \ref{exBNGVooIvKfTT}.
	\item
        Calculer la matrice Hessienne de fonctions de deux variables, déterminer les points critiques, déterminer le genre de la matrice Hessienne aux points critiques et écrire extrema de la fonctions (sous réserve d'être capable de résoudre certaines équations), voir les exemples \ref{exZHGRooTQpVpq} et \ref{exHWIHooOAvaDQ}.
	\item
        Indiquer une infinité de solutions à une équation en utilisant des paramètres, voir l'exemple \ref{exEEHPooKDxLTJ}. Pour les fonctions trigonométriques, 
        \begin{verbatim}
sage: solve(sin(x)/cos(x)==1,x,to_poly_solve=True)                                                         
[x == 1/4*pi + pi*z1]
sage: solve(sin(x)**2==cos(x)**2,x,to_poly_solve=True)
[sin(x) == cos(x), x == -1/4*pi + 2*pi*z86, x == 3/4*pi + 2*pi*z84]
        \end{verbatim}

        Notez l'option \info{to\_poly\_solve=true} dans \info{solve}.

	\item
        Calculer des dérivées symboliquement, voir exemple \ref{exRNZKooUIOfPU}.
	\item
        Calculer des approximations numériques comme dans l'exemple \ref{exLFYFooNCXCJz}.
    \item
        Calculer dans un corps de polynômes modulo comme \( \eF_p[X]/P\) où \( P\) est un polynôme à coefficients dans \( \eF_p\). Voir l'exemple \ref{ExemWUdrcs}.
\end{enumerate}

Sage peut toutefois vous induire en erreur si vous n'y prenez pas garde parce qu'il sait des choses en mathématique que vous ne savez pas. Par conséquent il peut parfois vous donner des réponses (mathématiquement exactes) auxquelles vous ne vous attendez pas. Voir \ref{ooOPWYooDDSZWx}.
