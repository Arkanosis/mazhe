
\InternalLinks{suites et séries}
De façon un peu contre-intuitive, les suites et séries sont surtout traitées dans le chapitre sur les espaces vectoriels. Section \ref{SECooLLUGooOwZRyI} pour les suites et \ref{SECooYCQBooSZNXhd} pour les séries.

\begin{enumerate}
    \item
        La définition de la somme d'une infinité de termes est donnée par la définition \ref{DefGFHAaOL}.
    \item
        La définition de la convergence absolue est la définition \ref{DefVFUIXwU}.
    \item
        Les propriétés générales de la proposition \ref{propnseries_propdebase}.
    \item
        Quelque séries usuelles dans \( \eR\) dans la section \ref{SUBSECooDTYHooZjXXJf}.
\end{enumerate}

    Pour les sommes infinies l'associativité et la commutativité dans une série sont perdues. Néanmoins, il subsiste que
%TODO : donner des exemples de cette perte.
  \begin{enumerate}
  \item 
      si la série converge, on peut regrouper ses termes sans modifier la convergence ni la somme (associativité);
  \item
      si la série converge absolument, on peut modifier l'ordre des termes sans modifier la convergence ni la somme (commutativité, proposition \ref{PopriXWvIY}).
  \end{enumerate}

Une somme indexée par un ensemble quelconque est la définition \ref{DefHYgkkA}.

