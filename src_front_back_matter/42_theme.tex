
\InternalLinks{densité}
\begin{enumerate}
    \item 
        Densité des polynômes dans \( C^0\big( \mathopen[ 0 , 1 \mathclose] \big)\), théorème de Bernstein \ref{ThoDJIvrty}.
    \item
        Densité de \( \swD(\eR^d)\) dans \( L^p(\eR^d)\) pour \( 1\leq p<\infty\), théorème \ref{ThoILGYXhX}.
    \item
        Densité de \( \swS(\eR^d)\) dans l'espace de Sobolev \( H^s(\eR^d)\), proposition \ref{PROPooMKAFooKDNTbO}. 

    \item
        Densité de \( \swD(\eR^d)\) dans l'espace de Sobolev \( H^s(\eR^d)\), proposition \ref{PROPooLIQJooKpWtnV}. 

        Cela est utilisé pour le théorème de trace \ref{THOooXEJZooBKtXBW}.
    \item
        Les applications étagées dans les applications mesurables (qui plus est avec limite croissante), théorème fondamental d'approximation \ref{LempTBaUw}.
    \item
        Les fonctions continues à support compact dans \( L^2(I)\), théorème \ref{ThoJsBKir}.
\end{enumerate}
Les densités sont bien entendu utilisées pour prouver des formules sur un espace en sachant qu'elles sont vraies sur une partie dense. Mais également pour étendre une application définie seulement sur une partie dense. C'est par exemple ce qui est fait pour définir la trace \( \gamma_0\) sur les espaces de Sobolev \( H^s(\eR^d)\) en utilisant le théorème d'extension \ref{PropTTiRgAq}.

