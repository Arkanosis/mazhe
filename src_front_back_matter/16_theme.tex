
\InternalLinks{connexité}
    \begin{enumerate}
        \item
            Définition \ref{DefIRKNooJJlmiD}
        \item
            Le groupe \( \SL(n,\eK)\) est connexe par arcs : proposition \ref{PROPooALQCooLZCKrH}.
        \item
            Le groupe \( \GL(n,\eC)\) est connexe par arcs : proposition \ref{PROPooVJNIooMByUJQ}.
        \item
            Le groupe \( \GL(n,\eC)\) est connexe par arcs, proposition \ref{PROPooVJNIooMByUJQ}.
        \item
            Le groupe \( \GL(n,\eR)\) a exactement deux composantes connexes par arcs, proposition \ref{PROPooBIYQooWLndSW}.
        \item
            Le groupe \( \gO(n,\eR)\) n'est pas connexe, lemme \ref{LEMooIPOVooZJyNoH}.
        \item
            Les groupe \( \gU(n)\) et \( \SU(n)\) sont connexes par arcs, lemme \ref{LEMooQMXHooZQozMK}.
        \item
            Le groupe \( \SO(n)\) est connexe mais ce n'est pas encore démontré, proposition \ref{PROPooYKMAooCuLtyh}.
        \item 
            Connexité des formes quadratiques de signature donnée, proposition \ref{PropNPbnsMd}.
        \end{enumerate}

