\InternalLinks{suite de Cauchy, espace complet}     \label{THMooOCXTooWenIJE}
\begin{enumerate}
    \item
        La définition \ref{DEFooXOYSooSPTRTn} donne la notion de suite de Cauchy dans un espace métrique.
    \item
        La définition \ref{DefZSnlbPc} donne la notion de suite de Cauchy dans un espace vectoriel topologique.
    \item
        Deux espaces métriques (avec une distance) peuvent être isomorphes en tant qu'espaces topologiques, mais ne pas avoir les mêmes suites de Cauchy, exemple \ref{EXooNMNVooXyJSDm}.
    \item
        La proposition \ref{PropooUEEOooLeIImr} donne l'équivalence entre la définition «topologique» et la définition usuelle dans le cas des espaces vectoriels topologiques \emph{normés}.
    \item
        L'exemple \ref{EXooNMNVooXyJSDm} est un exemple pire que simplement une suite de Cauchy qui ne converge pas. Le problème de convergence de cette suite n'est pas simplement que la limite n'est pas dans l'espace; c'est que la suite de Cauchy donnée ne convergerait même pas dans \( \eR\).
\end{enumerate}
