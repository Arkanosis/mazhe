% This is part of Outils mathématiques
% Copyright (c) 2011,2013,2015
%   Laurent Claessens
% See the file fdl-1.3.txt for copying conditions.

%+++++++++++++++++++++++++++++++++++++++++++++++++++++++++++++++++++++++++++++++++++++++++++++++++++++++++++++++++++++++++++
\section{Exemples introductifs}
%+++++++++++++++++++++++++++++++++++++++++++++++++++++++++++++++++++++++++++++++++++++++++++++++++++++++++++++++++++++++++++

%---------------------------------------------------------------------------------------------------------------------------
\subsection{La vitesse}
%---------------------------------------------------------------------------------------------------------------------------

Si tu lis ces lignes, c'est que tu as plus que probablement déjà entendu le baratin des physiciens à propos de la nuance entre les vitesses instantanées et vitesses moyennes. Imprègne toi bien de ces idées.

Lorsqu'un mobile se déplace à une vitesse variable, nous obtenons la \emph{vitesse instantanée} en calculant une vitesse moyenne sur des intervalles de plus en plus petits. Si le mobile a un mouvement donné par $x(t)$, la vitesse moyenne entre $t=2$ et $t=5$ sera
\[ 
  v_{\text{moy}}(2\to 5)=\frac{ x(5)-x(2) }{ 5-2 }.
\]
Plus généralement, la vitesse moyenne entre $2$ et $2+\Delta t$ est donnée par
\[ 
  v_{\text{moy}}(2\to 2+\Delta t)=\frac{ x(2+\Delta t)-x(2) }{ \Delta t }.
\]
Cela est une fonction de $\Delta t$. Oui, mais je te rappelle qu'on a dans l'idée de calculer une vitesse instantanée, c'est à dire de voir ce que vaut la vitesse moyenne sur un intervalle très {\small très} {\footnotesize très} {\scriptsize très} {\tiny petit}. La notion de limite semble toute indiquée pour décrire mathématiquement l'idée physique de vitesse instantanée.

Nous allons dire que la vitesse instantanée d'un mobile est la limite quand $\Delta t$ tends vers zéro de sa vitesse moyenne sur l'intervalle de temps $\Delta t$, ou en formule :
\begin{equation}		\label{EqvinstlimiteOM}
	v(t_0)=\lim_{\Delta t\to 0}\frac{ x(t_0)-x(t_0+\Delta t) }{ \Delta t }.
\end{equation}

%---------------------------------------------------------------------------------------------------------------------------
\subsection{La tangente à une courbe}
%---------------------------------------------------------------------------------------------------------------------------

Passons maintenant à tout autre chose, mais toujours dans l'utilisation de la notion de limite pour résoudre des problèmes intéressants. Comment trouver l'équation de la tangente à la courbe $y=f(x)$ au point $(x_0,f(x_0))$ ?

Essayons de trouver la tangente au point $P$ donné de la courbe donnée à la figure \ref{LabelFigTangenteQuestionOM}.

\newcommand{\CaptionFigTangenteQuestionOM}{Comment trouver la tangente à la courbe au point $P$ ?}
\input{pictures_tex/Fig_TangenteQuestionOM.pstricks}

La tangente est la droite qui touche la courbe en un seul point sans la traverser. Afin de la construire, nous allons dessiner des droites qui touchent la courbe en $P$ et un autre point $Q$, et nous allons voir ce qu'il se passe quand $Q$ est très proche de $P$. Cela donnera une droite qui, certes, touchera la courbe en deux points, mais en deux point \emph{tellement proche que c'est comme si c'étaient les mêmes}. Tu sens que la notion de limite va encore venir.


%Pour rappel cette figure TangenteDetail est générée par phystricksRechercheTangente.py
\newcommand{\CaptionFigTangenteDetailOM}{Traçons d'abord une corde entre le point $P$ et un point $Q$ un peu plus loin.}
\input{pictures_tex/Fig_TangenteDetailOM.pstricks}


Nous avons placé le point, sur la figure \ref{LabelFigTangenteDetailOM}, le point $P$ en $a$ et le point $Q$ un peu plus loin $x$. En d'autres termes leurs coordonnées sont
\begin{align}
	P=\big(a,f(a)\big)&& Q=\big(x,f(x)\big).
\end{align}
Comme tu devrais le savoir sans même regarder la figure \ref{LabelFigTangenteDetailOM}, le coefficient directeur de la droite qui passe par ces deux points est donné par
\begin{equation}
	\frac{ f(x)-f(a) }{ x-a },
\end{equation}
et bang ! Encore le même rapport que celui qu'on avait trouvé à l'équation \eqref{EqvinstlimiteOM} en parlant de vitesses. Si tu regardes la figure \ref{LabelFigLesSubFiguresOM}, tu verras que réellement en faisant tendre $x$ vers $a$ on obtient la tangente.

\newcommand{\CaptionFigLesSubFiguresOM}{Recherche de la tangente par approximations successives.}
\input{pictures_tex/Fig_LesSubFiguresOM.pstricks}
%See also the subfigure \ref{LabelFigLesSubFiguressssubZ}
%See also the subfigure \ref{LabelFigLesSubFiguressssubO}
%See also the subfigure \ref{LabelFigLesSubFiguressssubT}
%See also the subfigure \ref{LabelFigLesSubFiguressssubTh}
%See also the subfigure \ref{LabelFigLesSubFiguressssubF}
%See also the subfigure \ref{LabelFigLesSubFiguressssubFi}

%---------------------------------------------------------------------------------------------------------------------------
\subsection{Définition de la dérivée}
%---------------------------------------------------------------------------------------------------------------------------

Soit $I\subset\eR$ un intervalle et une fonction
\begin{equation}
	\begin{aligned}
		f\colon I&\to \eR \\
		x&\mapsto f(x). 
	\end{aligned}
\end{equation}
On dit que $f$ est \defe{dérivable}{dérivable!fonction} en $a\in I$ si la limite
\begin{equation}	\label{EqLimDeirveOM}
	\lim_{x\to a} \frac{ f(x)-f(a) }{ x-a }
\end{equation}
existe. Formellement nous disons que cette limite existe et vaut $\ell$ lorsque pour tout $\epsilon>0$, il existe un $\delta>0$ tel que dès que $| x-a |<\delta$ on ait
\begin{equation}
	\left| \frac{ f(x)-f(a) }{ x-a } -\ell \right| <\epsilon.
\end{equation}

Lorsque la limite \eqref{EqLimDeirveOM} existe nous l'appelons $f'(a)$ et nous disons que la fonction $f$ est dérivable en $a$. Si la fonction est dérivable en tout point de $I$, nous disons qu'elle est dérivable sur $I$. Cela fournit un nombre $f'(x)$ en chaque point $x\in I$, c'est à dire une nouvelle fonction
\begin{equation}
	\begin{aligned}
		f'\colon I&\to \eR \\
		x&\mapsto f'(x)
	\end{aligned}
\end{equation}
qui sera nommée \defe{fonction dérivée}{dérivée} de $f$.

Il arrive que la fonction $f'$ soit elle-même dérivable. Dans ce cas nous nommons $f''$ la dérivée de $f$; cela est la \defe{dérivée seconde}{dérivée!seconde} de $f$.

%---------------------------------------------------------------------------------------------------------------------------
\subsection{L'aire en dessous d'une courbe}		\label{SubSecAirePrimIntoOM}
%---------------------------------------------------------------------------------------------------------------------------

Encore un exemple. Nous voudrions bien pouvoir calculer l'aire en dessous d'une courbe. Nous notons $S_f(x)$ l'aire en dessous de la fonction $f$ entre l'abscisse $0$ et $x$, c'est à dire l'aire bleue de la figure \ref{LabelFigBQXKooPqSEMN}.

\newcommand{\CaptionFigBQXKooPqSEMN}{L'aire en dessous d'une courbe. Le rectangle rouge d'aire $f(x)\Delta x$ approxime de combien la surface augmente lorsqu'on passe de $x$ à $x+\Delta x$.}
\input{pictures_tex/Fig_BQXKooPqSEMN.pstricks}

Si la fonction $f$ est continue et que $\Delta x$ est assez petit, la fonction ne varie pas beaucoup entre $x$ et $x+\Delta x$. L'augmentation de surface entre $x$ et $x+\Delta x$ peut donc être approximée par le rectangle de surface $f(x)\Delta x$. Ce que nous avons donc, c'est que quand $\Delta x$ est très petit,
\begin{equation}
	S_f(x+\Delta x)-S_f(x)=f(x)\Delta x,
\end{equation}
c'est à dire
\begin{equation}
	f(x)=\lim_{\Delta x\to 0}\frac{  S_f(x+\Delta x)-S_f(x)}{ \Delta x }.
\end{equation}
Donc, la fonction $f$ est la dérivée de la fonction qui représente l'aire en dessous de $f$. Calculer des surfaces revient donc au travail inverse de calculer des dérivées.

Nous avons déjà vu que calculer la dérivée d'une fonction n'est pas très compliqué. Aussi étonnant que cela puisse paraître, il se fait que le processus inverse est très compliqué : il est en général extrêmement difficile (et même souvent impossible) de trouver une fonction dont la dérivée est une fonction donnée.

Une fonction dont la dérivée est la fonction $f$ s'appelle une \defe{primitive}{primitive} de $f$, et la fonction qui donne l'aire en dessous de la fonction $f$ entre l'abscisse $0$ et $x$ est notée
\begin{equation}
	S_f(x)=\int_0^xf(t)dt.
\end{equation}
Nous pouvons nous demander si, pour une fonction $f$ donnée, il existe une ou plusieurs primitives, c'est à dire s'il existe une ou plusieurs fonctions $F$ telles que $F'=f$. La réponse viendra\ldots
%TODO : faire la référence


%+++++++++++++++++++++++++++++++++++++++++++++++++++++++++++++++++++++++++++++++++++++++++++++++++++++++++++++++++++++++++++
\section{Dérivation de fonctions d'une variable réelle}
%+++++++++++++++++++++++++++++++++++++++++++++++++++++++++++++++++++++++++++++++++++++++++++++++++++++++++++++++++++++++++++
%---------------------------------------------------------------------------------------------------------------------------
\subsection{Exemples}
%---------------------------------------------------------------------------------------------------------------------------

%///////////////////////////////////////////////////////////////////////////////////////////////////////////////////////////
\subsubsection{La fonction $f(x)=x$}
%///////////////////////////////////////////////////////////////////////////////////////////////////////////////////////////

Commençons par la fonction $f(x)=x$. Dans ce cas nous avons
\begin{equation}
	\frac{ f(x)-f(a) }{ x-a }=\frac{ x-a }{ x-a }=1.
\end{equation}
La dérivée est donc $1$.


%///////////////////////////////////////////////////////////////////////////////////////////////////////////////////////////
\subsubsection{La fonction $f(x)=x^2$}
%///////////////////////////////////////////////////////////////////////////////////////////////////////////////////////////

Prenons ensuite $f(x)=x^2$. En utilisant le produit remarquable $(x^2-a^2)=(x-a)(x+a)$ nous trouvons
\begin{equation}
	\frac{ f(x)-f(a) }{ x-a }=x+a.
\end{equation}
Lorsque $x\to a$, cela devient $2a$. Nous avons par conséquent
\begin{equation}
	f'(x)=2x.
\end{equation}

%///////////////////////////////////////////////////////////////////////////////////////////////////////////////////////////
\subsubsection{La fonction $f(x)=\sqrt{x}$}
%///////////////////////////////////////////////////////////////////////////////////////////////////////////////////////////

Considérons maintenant la fonction $f(x)=\sqrt{x}$. Nous avons
\begin{equation}
	\begin{aligned}[]
		\frac{ f(x)-f(a) }{ x-a }&=\frac{ \sqrt{x}-\sqrt{a} }{ x-a }\\
		&=\frac{ (\sqrt{x}-\sqrt{a})(\sqrt{x}+\sqrt{x}) }{ (x-a)(\sqrt{x}+\sqrt{x}) }\\
		&=\frac{1}{ \sqrt{x}+\sqrt{x} }.
	\end{aligned}
\end{equation}
Lorsque $x\to 0$, nous obtenons
\begin{equation}
	f'(a)=\frac{1}{ 2\sqrt{a} }.
\end{equation}
Notons que la dérivée de $f(x)=\sqrt{x}$ n'existe pas en $x=0$. En effet elle serait donnée par le quotient
\begin{equation}
	f'(0)=\lim_{x\to 0} \frac{ \sqrt{x}-\sqrt{0} }{ x }=\lim_{x\to 0} \frac{ \sqrt{x} }{ x }=\lim_{x\to 0} \frac{1}{ \sqrt{x} }.
\end{equation}
Mais si $x$ devient très petit, la dernière fraction tend vers l'infini.

%---------------------------------------------------------------------------------------------------------------------------
\subsection{Calcul de la dérivée}
%---------------------------------------------------------------------------------------------------------------------------

Soit $f,g\colon I\subset\eR\to\eR $ deux fonctions dérivables. Alors nous admettons les propriétés suivantes.
\begin{enumerate}
	\item
		la fonction $h=f+g$ est dérivable et $h'(x)=f'(x)+g'(x)$.
	\item
		la fonction $h=fg$ est dérivable et 
		\begin{equation}
			(fg)'(x)=f'(x)g(x)+f(x)g'(x).
		\end{equation}
		Cette formule est appelée \defe{règle de Leibnitz}{Leibnitz}.
	\item
		la fonction $h=\frac{ f }{ g }$ est dérivable en tout point $x$ tel que $g(x)\neq 0$ et 
		\begin{equation}
			\left( \frac{ f }{ g } \right)'(x)=\frac{ f'(x)g(x)-f(x)g'(x) }{ g(x)^2 }.
		\end{equation}
	\item
		la fonction $h=f\circ g$ est dérivable et 
		\begin{equation}
			(f\circ g)'(x)=f'\big( g(x) \big)g'(x).
		\end{equation}
		
\end{enumerate}

\begin{proposition}
	Si $f(x)=x^n$ avec $n\in\eN$, alors $f'(x)=nx^{n-1}$.
\end{proposition}
\begin{proof}
	Nous avons déjà vu que la proposition était vraie avec $n=1$ et $n=2$. Supposons qu'elles soit vraie avec $n=k$, et prouvons qu'elle est vraie pour $n=k+1$. Nous avons
	\begin{equation}
		x^{k+1}=xx^k.
	\end{equation}
	En utilisant la règle de Leibnitz et l'hypothèse de récurrence,
	\begin{equation}
		\begin{aligned}[]
			\big( x^{k+1} \big)'&=(x)'x^k+x\big( x^k \big)'\\
			&=x^k+x\big( kx^{k-1} \big)\\
			&=x^k+kx^k\\
			&=(k+1)x^k,
		\end{aligned}
	\end{equation}
	ce qu'il fallait démontrer.
\end{proof}

%+++++++++++++++++++++++++++++++++++++++++++++++++++++++++++++++++++++++++++++++++++++++++++++++++++++++++++++++++++++++++++
\section[Interprétation géométrique : tangente]{Interprétation géométrique de la dérivée : tangente}
%+++++++++++++++++++++++++++++++++++++++++++++++++++++++++++++++++++++++++++++++++++++++++++++++++++++++++++++++++++++++++++

Considérons le \defe{graphe}{graphe} de la fonction $f$ sur $I$, c'est à dire l'ensemble
\begin{equation}
	\big\{ \big( x,f(x) \big)\tq x\in I \big\}.
\end{equation}
Le nombre 
\begin{equation}
	\frac{ f(x)-f(a) }{ x-a }
\end{equation}
est la pente de la droite qui joint les points $\big( x,f(x) \big)$ et $\big( a,f(a) \big)$, voir la figure \ref{LabelFigDerivTangenteOM}.
\newcommand{\CaptionFigDerivTangenteOM}{Le coefficient directeur de la corde entre $a$ et $x$.}
\input{pictures_tex/Fig_DerivTangenteOM.pstricks}

Étant donné que $f'(a)$ est le coefficient directeur de la tangente au point $\big( a,f(a) \big)$, l'équation de la tangente est
\begin{equation}		\label{EqTgfaenOM}
	y-f(a)=f'(a)(x-a).
\end{equation}

%+++++++++++++++++++++++++++++++++++++++++++++++++++++++++++++++++++++++++++++++++++++++++++++++++++++++++++++++++++++++++++
\section[Interprétation géométrique : approximation affine]{Interprétation géométrique de la dérivée : approximation affine}
%+++++++++++++++++++++++++++++++++++++++++++++++++++++++++++++++++++++++++++++++++++++++++++++++++++++++++++++++++++++++++++

Le fait que la fonction $f$ soit dérivable au point $a\in I$ signifie que
\begin{equation}
	\lim_{x\to a} \frac{ f(x)-f(a) }{ x-a }=\ell
\end{equation}
pour un certain nombre $\ell$. Cela peut être récrit sous la forme
\begin{equation}
	\lim_{x\to a} \frac{ f(x)-f(a) }{ x-a }-\ell=0,
\end{equation}
ou encore
\begin{equation}
	\lim_{x\to a} \frac{ f(x)-f(a)-\ell(x-a) }{ x-a }=0.
\end{equation}
Introduisons la fonction
\begin{equation}
	\alpha(t)=\frac{ f(a+t)-f(a)-t\ell }{ t }.
\end{equation}
Cette fonction est faite exprès pour que
\begin{equation}		\label{EqIntermsaxaamaOM}
	\alpha(x-a)=\frac{ f(x)-f(a)-\ell(x-a) }{ x-a };
\end{equation}
par conséquent $\lim_{x\to a} \alpha(x-a)=0$. Nous récrivons l'équation \eqref{EqIntermsaxaamaOM} sous la forme
\begin{equation}        \label{EqCodeDerviffxamOM}
	f(x)-f(a)-\ell(x-a)=(x-a)\alpha(x-a).
\end{equation}
Le second membre tend vers zéro lorsque $x$ tend vers $a$ avec une «vitesse au carré» : c'est le produit de deux facteurs tous deux tendant vers zéro. Si $x$ n'est pas très loin de $a$, il n'est donc pas une mauvaise approximation de dire
\begin{equation}
	f(x)-f(a)-\ell(x-a)\simeq 0,
\end{equation}
c'est à dire
\begin{equation}		\label{EqfxsimesfaOM}
	f(x)\simeq f(a)+f'(a)(x-a).
\end{equation}
Nous avons retrouvé l'équation \eqref{EqTgfaenOM}. La manipulation que nous venons de faire revient donc à dire que la fonction $f$, au voisinage de $a$, est bien approximée par sa tangente.

L'équation \eqref{EqfxsimesfaOM} peut être aussi écrite sous la forme
\begin{equation}		\label{EqfxdxSimeqfxfpxOM}
	f(x+\Delta x)\simeq f(x)+f'(x)\Delta x
\end{equation}
qui est une approximation d'autan meilleure que $\Delta x$ est petit.

%+++++++++++++++++++++++++++++++++++++++++++++++++++++++++++++++++++++++++++++++++++++++++++++++++++++++++++++++++++++++++++
\section{Recherche d'extrema}
%+++++++++++++++++++++++++++++++++++++++++++++++++++++++++++++++++++++++++++++++++++++++++++++++++++++++++++++++++++++++++++

Soit une fonction $f\colon I\to \eR$, et soit $a\in I$. Si $f'(a)>0$, alors la tangente au graphe de $f$ au point $\big( a,f(a) \big)$ sera une droite croissante (coefficient directeur positif). Cela ne veut pas spécialement dire que la fonction elle-même sera croissante, mais en tout cas cela est un bon indice.

\begin{example}
	Si $f(x)=x^2$, il est connu que $f'(x)=2x$. Nous avons donc que $f'$ est positive si $x\geq 0$ et $f'>$ est négative si $x<0$. Cela correspond bien au fait que $x^2$ est décroissante sur $\mathopen] -\infty , 0 \mathclose[$ et croissante sur $\mathopen] 0 , \infty \mathclose[$.
\end{example}
 
Sur la figure \ref{LabelFigFonctionEtDeriveOM}, nous avons dessiné la fonction $f(x)=x\cos(x)$ et sa dérivée. Nous voyons que partout où la dérivée est négative, la fonction est décroissante tandis que, inversement, partout où la dérivée est positive, la fonction est croissante.
\newcommand{\CaptionFigFonctionEtDeriveOM}{La fonction $f(x)=x\cos(x)$ en bleu et sa dérivée en rouge.}
\input{pictures_tex/Fig_FonctionEtDeriveOM.pstricks}

Les extrema de la fonction $f$ sont donc placés là où $f'$ change de signe. En effet si $f'(x)<0$ pour $x<a$ et $f'(x)>0$ pour $x>a$, la fonction est décroissante jusqu'à $a$ et est ensuite croissante. Cela signifie que la fonction connait un creux en $a$. Le point $a$ est donc un minimum de la fonction.

Attention cependant. Le fait que $f'(a)=0$ ne signifie pas automatiquement que $f$ a un maximum ou un minimum en $a$. Nous avons par exemple tracé sur la figure \ref{LabelFigFonctionXtroisOM} les fonctions $x^3$ et sa dérivée. Il est à noter que, conformément à ce que l'on pense, certes la dérivée s'annule en $x=0$, mais elle ne change pas de signe.
\newcommand{\CaptionFigFonctionXtroisOM}{La dérivée de $x^3$ s'annule en $x=0$, mais ce n'est ni un minimum ni un maximum.}
\input{pictures_tex/Fig_FonctionXtroisOM.pstricks}

