% This is part of Exercices et corrigés de CdI-1
% Copyright (c) 2011
%   Laurent Claessens
% See the file fdl-1.3.txt for copying conditions.

\begin{corrige}{Implicite0001}

	Nous considérons la fonction
	\begin{equation}
		\begin{aligned}
			F\colon \eR^2\times \eR_0^+&\to \eR \\
			(x,y,z)&\mapsto z+\ln(z)-xy, 
		\end{aligned}
	\end{equation}
	et nous nous intéressons à la fonction $Z(x,y)$ définie au voisinage de $(x,y)=(1,1)$ par la condition
	\begin{equation}
		F\big( x,y,Z(x,y) \big)=0.
	\end{equation}
	Nous pouvons facilement trouver $z_0=Z(1,1)$, étant donné que sa définition est que
	\begin{equation}
		z_0+\ln(z_0)-1=0.
	\end{equation}
	Facile de voir que $z_0=1$ remplit la condition. Dans les notations du théorème de la fonction implicite, nous avons donc $\tilde x=(1,1)$ et $\tilde y=z_0$. Le déterminant à vérifier pour le théorème est une matrice $1\times 1$ parce qu'il n'y a que une seule variable $z$ que l'on veut exprimer en termes des autres ($x$ et $y$):
	\begin{equation}
		\frac{ \partial F }{ \partial z }(1,1,z_0)=1+\frac{1}{ z_0 }.
	\end{equation}
	Cela n'est pas nul parce que $z_0=1$. Il existe donc un voisinage $\mU$ de $(1,1)$ dans $\eR^2$ et un voisinage $V$ de $z_0$ dans $\eR$ tels que $\forall (x,y)\in\mU$, il existe un et un seul $Z(x,y)$ solution de $F\big( (x,y),Z(x,y) \big)=0$.

	Pour calculer les dérivées de $Z$, il faut bien se rendre compte que, pour chaque $x$ et $y$, nous avons l'égalité
	\begin{equation}
		Z(x,y)+\ln\big( Z(x,y) \big)-xy=0.
	\end{equation}
	Nous pouvons dériver cette égalité partiellement par rapport à $x$ pour obtenir\footnote{Nous n'allons plus, dans l'avenir, toujours écrire explicitement les dépendances en $x$ et $y$}
	\begin{equation}		\label{EqPartialXZexoI}
		(\partial_xZ)(x,y)+\frac{ (\partial_xZ)(x,y) }{ Z(x,y) }-y=0.
	\end{equation}
	Nous pouvons aisément isoler $(\partial_xZ)(x,y)$ dans cette équation :
	\begin{equation}
		\frac{ \partial Z }{ \partial x }=\frac{ yZ }{ Z+1 }.
	\end{equation}
	Notez que la fonction \og inconnue\fg{} apparaît dans l'expression de la dérivée. C'est la vie, et c'est pourquoi ce chapitre parle de fonctions \emph{implicite}.

	De la même manière, nous trouvons
	\begin{equation}
		\frac{ \partial Z }{ \partial y }=\frac{ xZ }{ Z+1 }.
	\end{equation}
	Pour la dérivée seconde $\partial_{xy}Z$, d'abord nous savons que $Z$ est une fonction $C^2$\footnote{parce que $F$ est $C^2$, voir l'énoncé du théorème.}, donc l'ordre des dérivées n'a pas d'importance. Dérivons donc par rapport à $x$ l'expression de $\partial_yZ$ :
	\begin{equation}
		\frac{ \partial^2 Z }{ \partial x\partial y }=\frac{ Z }{ Z+1 }+x\left( \frac{ \partial_xZ }{ Z+1 }-Z\frac{ \partial_xZ }{ (Z+1)^2 } \right).
	\end{equation}
	Nous devons encore substituer $\partial_xZ$ par sa valeur \eqref{EqPartialXZexoI} pour trouver la réponse finale
	\begin{equation}
		\frac{ \partial^2 Z }{ \partial x\partial y  }(x,y)=\frac{ Z }{ z+1 }+x\left( \frac{ yZ }{ (Z+1)^2 }-\frac{ yZ^2 }{ (Z+1)^2 } \right).
	\end{equation}

\end{corrige}
