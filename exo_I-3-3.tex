% This is part of the Exercices et corrigés de CdI-2.
% Copyright (C) 2008, 2009
%   Laurent Claessens
% See the file fdl-1.3.txt for copying conditions.


\begin{exercice}\label{exo_I-3-3}

Cet exercice consiste à prouver le résultat suivant :
\begin{theorem}
Soit $D\subset\eR^2$, une ouvert étoilé, et $\omega$, une $1$-forme fermée de classe $C^1$. Alors $\omega$ est exacte.
\end{theorem}
Voir la section \ref{SecFormDiffRappel} pour des détails; ceci est un cas particulier du théorème \ref{ThoFermeExactFormRappel}. En pratique, voici ce que vous devez faire.

Soit $D\subset\eR^2$, un ouvert étoilé par rapport à l'origine (cf. cours, première partie, page 575). Soient $f\colon D\to \eR$, $g\colon D\to \eR$, des fonctions de classe $C^1$ telles que
\begin{equation}
	\frac{ \partial f }{ \partial y }=\frac{ \partial g }{ \partial x }
\end{equation}
sur $D$, et
\begin{equation}		\label{EqDefFformI33}
	F(x,y)=\int_0^1\big[  f(tx,ty)x+g(tx,ty)y  \big]dt
\end{equation}
pour tout $(x,y)\in D$. Montrer que
\begin{equation}		\label{EqFormI33Fffdd}
	\begin{aligned}[]
		\frac{ \partial F }{ \partial x }&=f,  &\text{et}&& \frac{ \partial F }{ \partial y }=g.
	\end{aligned}
\end{equation}
(cf. Corollaire page 579 du cours première partie). En ayant prouvé cela, nous aurons prouvé que si $\omega=fdx+gdy$ avec $\partial_yf=\partial_xg$, alors $\omega=dF$ où $F$ est définie par \eqref{EqDefFformI33}.

\corrref{_I-3-3}
\end{exercice}
