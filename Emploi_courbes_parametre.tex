Lorsque nous faisons de la géométrie ou bien lorsque nous étudions des mouvements en physique, nous devons souvent considérer des \wikipedia{fr}{Courbe_plane}{courbes} plus générales que ce qui est possible décrire comme le graphe d'une fonction, par exemple celles qui sont dessinées à la figure \ref{LabelFigExempleArcParam}. Ce chapitre a pour objet la déscription paramétrique des courbes, qui nous permettra de travailler avec à peu près n'importe quel type de courbe dans l'espace.

La section \ref{SecDeExCPar} donne la définition de base et quelque exemples remarquables. Il faut comprendre ces exemples et conna\^{i}tre la paramétrisation des principales courbes connues.
 
Le calcul de la longueur d'un arc (section \ref{SecLongArc}) est essentielle pour calculer la distance entre deux points sur une courbe <<par rapport à la courbe>>. Bien sûr si on regarde les deux points comme des points d'un espace métrique leur distance est donnée par la distance définie sur tout l'espace. Cependant, cette notion de distance n'est pas satisfaisante si la courbe répresente le chemin que nous sommes obligés à suivre pour aller du premier point au deuxième : cela est toujour le cas quand nous voyageons en train ou en tram. En fait, dans le langage commun on dit souvent \emph{distance aérienne (ou à vol d'oiseau)} pour la distance définie sur tout l'espace et on appelle simplement \emph{distance} la longeur du chemin il nous faut parcourir. 

Le calcul de la longueur d'un arc permettra aussi de donner un changement de coordonnées qui simplifie de nombreuses expressions, il s'agit des abscisses curvilignes \ref{SubSecAbsCurv}.

Nous verrons ensuite la notion de vecteur tangent à une courbe. Il s'agira bien entendu d'une variation sur le thème de la dérivation. Dans la suites de vos études vous allez utiliser cet outil pour intégrer des fonctions sur une courbe. Des exemples très classiques sont : la masse totale d'un câble (intégrale de la densité) et  le travail d'une force sur un objet se mouvant sur la courbe.

%Dans votre vie de tous les jours, l'utilité principale du vecteur tangent à une courbe sera de calculer des intégrales de champs de vecteurs le long de courbes, et donc, en physique, de calculer le travail d'une force sur un objet se mouvant dans un champ de force.

Enfin la section \ref{SecFrenet} étudiera en détail ce que l'on peut faire avec les dérivées d'ordre supérieure. La dérivée première donner la tangente. Nous verrons que la dérivée seconde donne la courbure, et la dérivée troisième nous permettra d'exprimer la torsion de la courbe. Nous allons construire, en chaque point de la courbe une base orthonormée de $\eR^3$ adaptée à l'étude de la courbe.
