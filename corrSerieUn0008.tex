% This is part of Exercices et corrections de MAT1151
% Copyright (C) 2010
%   Laurent Claessens
% See the file LICENCE.txt for copying conditions.

\begin{corrige}{SerieUn0008}

	Nous voulons utiliser le corollaire \ref{CorConditionnementNormeNabla}. Pour cela, nous devons d'abord montrer que le problème est stable. De la même manière que, avant, nous prouvions la stabilité en montrant que la dérivée de $x$ est bornée, ici nous allons argumenter que la norme du gradient est bornée pour dire que le problème est stable\footnote{Nous n'allons pas entrer dans les détails de la preuve que la stabilité est impliquée par la borne sur le gradient, mais c'est un bon exercice.}.

	Supposons que $q > -p^2$. La plus grande des deux racines de $x^2 + 2px - q = 0$ est donnée par
	\begin{equation}
		x(p,q) = -p + \sqrt{p^2+q}.
	\end{equation}
	Nous avons donc
	\begin{equation}
		\frac{\partial x}{\partial p} = -1 + \frac{p}{\sqrt{p^2+q}} = \frac{-x(p,q)}{\sqrt{p^2+q}}
	\end{equation}
	et
	\begin{equation}
		\frac{\partial x}{\partial q} = \frac{1}{2\sqrt{p^2+q}},
	\end{equation}
	et la norme du gradient est donc
	\begin{equation}
		\| \nabla x(p_0,q_0) \| = \frac{\sqrt{4 x^2(p_0,q_0)+1}}{2\sqrt{p_0^2+q_0}}.
	\end{equation}
	Cette quantité reste bornée au voisinage de tout point vérifiant $q>-p^2$. Nous pouvons donc utiliser le corollaire \ref{CorConditionnementNormeNabla}. En fait, en calculant la norme de $\nabla x$, nous avons fait coup double : d'une part nous avons validé les hypothèses du corollaire, et d'autre part, nous en avons calculé la conclusion.

	Pour $\eta$ suffisamment petit nous avons donc
	\begin{equation}
		K_{\text{abs}}\big((p_0,q_0),\eta\big) \sim \| \nabla x(p_0,q_0) \| = \frac{\sqrt{4 x^2(p_0,q_0)+1}}{2\sqrt{p_0^2+q_0}}.
	\end{equation}
	Le conditionnement relatif vaut quant à lui
	\begin{equation}
		K_{\text{rel}}(\big(p_0,q_0),\eta\big) = K_{\text{abs}}\big((p_0,q_0),\eta\big) \frac{\sqrt{p_0^2+q_0^2}}{|x(p_0,q_0)|} \sim \frac{\sqrt{4 x^2(p_0,q_0)+1}}{|x(p_0,q_0)|}\frac{\sqrt{p_0^2+q_0^2}}{2\sqrt{p_0^2+q_0}}.
	\end{equation}
	On en déduit que le problème est mal conditionné quand $q_0$ est proche de $-p_0^2$, c.-à-d. quand les deux racines sont proches l'une de l'autre, ou quand $x(p_0,q_0) = 0$ (pourquoi ??).

\end{corrige}
