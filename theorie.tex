% This is part of Exercices et corrections de MAT1151
% Copyright (C) 2010
%   Laurent Claessens
% See the file LICENCE.txt for copying conditions.

D'autres lectures agréables dans \cite{GianlucaB,cmcsNum}

%+++++++++++++++++++++++++++++++++++++++++++++++++++++++++++++++++++++++++++++++++++++++++++++++++++++++++++++++++++++++++++
\section{Conditionnement et stabilité}
%+++++++++++++++++++++++++++++++++++++++++++++++++++++++++++++++++++++++++++++++++++++++++++++++++++++++++++++++++++++++++++

\begin{definition}

	Soit $F$ une fonction à valeurs réelles définie sur ${\bf X}\times{\bf D}$ o\`u ${\bf X}$ et ${\bf D}$ sont des espaces vectoriels réels normés. Le problème de la recherche des solutions de 
	\begin{equation}
		F(x,d)=0
	\end{equation}
	est dit \defe{stable}{stable} si 
	\begin{enumerate}
		\item 
			la solution $x=x(d)$ existe et est unique pour tout $d$;
		\item \label{ItemProbStableB}
			Pour tout $\eta>0$, et pour tout $d_0$, il existe un nombre $K>0$ tel que $|d-d_0|<\eta$ entraine $|x(d)-x(d_0)|\;\leq\;K\;|d-d_0|$.
	\end{enumerate}
\end{definition}
Le nombre 
\begin{equation}		\label{EqDefAABSOLU}
	K_{abs}(d_0,\eta):=\sup_{d\text{ tel que $|d_0-d|<\eta$}}\frac{|x(d)-x(d_0)|}{|d-d_0|}
\end{equation}
est appelé le \defe{conditionnement absolu}{Conditionnement!absolu} du problème autour de $d_0$.

\begin{definition}	
	Soit $F(x,d)=0$ un problème stable de conditionnement absolu $K_{\text{abs}}(d,\eta)$.  Le conditionnement relatif est défini par
	\begin{equation}
		K_{\text{rel}}(d,\eta):=K_{\text{abs}}(d,\eta)\frac{|d|}{|x(d)|}.
	\end{equation}
	Le problème est dit \defe{bien conditionné}{Bien conditionné} près de $d$ si $K_{\text{rel}}(d,\eta)$ est petit.
\end{definition}

Un résultat pratique pour étudier le conditionnement d'un problème est le suivant.
\begin{corollary}		\label{CorConditionnementNormeNabla}
	Soit $x=x(d)$ un problème stable. Supposons $\eD$ de dimension finie, supposons que $U$ est ouvert dans $\eD$. Supposons encore $x\colon U\to \eR$ différentiable en $d_0$. Alors quand $\eta$ est petit, on a
	\begin{equation}
		K_{\text{abs}}^{\eta}(d_0)\sim \| \nabla x(d_0) \|.
	\end{equation}
\end{corollary}

%---------------------------------------------------------------------------------------------------------------------------
\subsection{Comment choisir et penser le $K$?}
%---------------------------------------------------------------------------------------------------------------------------

La formule \eqref{EqDefAABSOLU} contient une formule qui ressemble étrangement à la dérivée. La stabilité d'un problème est très liée à la dérivée de $F$. La stabilité et la dérivée ne sont pas les mêmes choses, mais il n'est pas mauvais de penser au $K$ de la stabilité comme la dérivée. Ou plus précisément : le supremum de la dérivée.

Un fil conducteur des exercices \ref{exoSerieUn0002}, \ref{exoSerieUn0003} et \ref{exoSerieUn0001} est que l'on a un $K$ qui fonctionne lorsque la dérivée est bornée sur l'intervalle $\mathopen] d_0-\eta , d_0+\eta \mathclose[$. Dans le cas où ce supremum existe, le prendre en guise de $K$ fonctionne souvent.

Il faut cependant parfois faire acte d'imagination. La fonction $x\mapsto| x |$ n'est pas dérivable en $0$. Il n'empêche que $K=1$ fait fonctionner la définition de la stabilité. Remarquez que $K=1$ est le supremum de la dérivée là où elle existe.

À partir du moment où c'est clair que le $K$ est le supremum de la dérivée, on comprend pourquoi c'est le gradient qui arrive dans le corollaire \ref{CorConditionnementNormeNabla}. En effet, le gradient indique la direction de plus grande pente. C'est donc bien dans cette direction qu'il faut chercher la «plus grande dérivée».

%+++++++++++++++++++++++++++++++++++++++++++++++++++++++++++++++++++++++++++++++++++++++++++++++++++++++++++++++++++++++++++
\section{Norme opérateur}
%+++++++++++++++++++++++++++++++++++++++++++++++++++++++++++++++++++++++++++++++++++++++++++++++++++++++++++++++++++++++++++

Une \defe{norme}{norme} sur un espace vectoriel réel $V$ est une application $\| . \|\colon V\to \eR$ telle que
\begin{enumerate}
		\label{PgDefNorme}

	\item
		$\| v \|=0$ seulement si $A=0$,
	\item
		$\| \lambda v \|=| \lambda |\cdot\| v \|$,
	\item
		$\| v+w \|\leq\| v \|+\| w \|$

\end{enumerate}
pour tout $v,w\in V$ et pour tout $\lambda\in\eR$.

\begin{definition}
	Soit $\alpha$ une application linéaire entre espaces vectoriels réels normés. On définit sa \defe{\wikipedia{fr}{Norme_d'opérateur}{norme opérateur}}{norme!opérateur} comme le nombre
	\begin{equation}
		|\alpha|_{\mbox{op}}:=\sup_{|x|=1}\{|\alpha(x)|\}.
	\end{equation}
\end{definition}

\begin{proposition}	
	Pour le problème stable $x=x(d)$ avec $x\in C^1(\eR^n,\eR)$, on a
	\begin{equation}
		K_{abs}(d)\sim|x_{\star d}|_{\mbox{op}}
	\end{equation}
	où $x_{\star d}$ désigne la différentielle de $x$ en $d$.
\end{proposition}

%+++++++++++++++++++++++++++++++++++++++++++++++++++++++++++++++++++++++++++++++++++++++++++++++++++++++++++++++++++++++++++
\section{Algorithmes, stabilité, convergence et conditionnement}
%+++++++++++++++++++++++++++++++++++++++++++++++++++++++++++++++++++++++++++++++++++++++++++++++++++++++++++++++++++++++++++

%---------------------------------------------------------------------------------------------------------------------------
\subsection{Méthode de Newton pour trouver une racine d'une fonction}
%---------------------------------------------------------------------------------------------------------------------------
\label{SubSecMethodeNewton}

La méthode de Newton consiste a exprimer la solution $x$ de $f(x)=0$ avec $f\in C^1(\eR)$ comme limite d'une suite $\{x_n\}_{n\in\eN}$ définie par récurrence par la formule
\begin{equation}
	x_{n+1}=x_n-\frac{f(x_n)}{f'(x_n)}.
\end{equation}
où $x_0$ est arbitraire.

Si on veut exprimer cela en termes d'algorithmes, nous disons que l'algorithme de Newton est donné par la suite de problèmes
\begin{equation}		\label{EqFPourNewtonUn}
	F_n(x_{n+1},x_n,f)=x_{n+1}-x_n+\frac{ f(x_n) }{ f'(x_n) }.
\end{equation}
La donnée du problème est la fonction $f$, et rien que elle.

Plus précisément, une fois que la fonction $f$ est donnée, il existe une infinité de problèmes : pour chaque $a\in \eR$ nous avons le problème
\begin{equation}
	G_a(x_n,f)=x-a+\frac{ f(a) }{ f'(a) }.
\end{equation}
La méthode de Newton consiste à sélectionner une partie de ces problèmes de la façon suivante :
\begin{subequations}
	\begin{numcases}{}
		F_0=G_{x_0}\\
		F_n=G_{x_n}.
	\end{numcases}
\end{subequations}
Le problème $F_0$ fournit un nombre $x_1$ qui nous permet de sélectionner le problème $G_{x_1}$ qui va fournir le nombre $x_2$, etc.

Au moment de calculer le conditionnement de $F_n$, nous ne devons pas voir $x_{n-1}$ comme fonction de $x_0$ et de la donnée $f$. Il ne faut donc pas dériver à travers les $x_n$.

L'algorithme de Newton a les caractéristiques suivantes :
\begin{enumerate}

	\item
		Pour résoudre le problème numéro $n$, il faut avoir résolu le problème numéro $n-1$.
	\item
		Aucune des solutions $x_n$ aux problèmes intermédiaires n'est une solution au problème de départ (à moins d'un coup de chance).
	\item
		Étant donné que la donnée du problème $F_n$ est la fonction $f$ de départ, nous avons $d_m=d_n=d$ pour tout $m$ et $n$.

\end{enumerate}

%---------------------------------------------------------------------------------------------------------------------------
\subsection{Résoudre un système linéaire}
%---------------------------------------------------------------------------------------------------------------------------

Pour résoudre un système linéaire d'équations, nous échelonnons la matrice du système. Soit à résoudre le système $Ax=b$ où
\begin{equation}
	\begin{aligned}[]
		A&=\begin{pmatrix}
			2	&	4	&	-6	\\
			1	&	5	&	3	\\
			1	&	3	&	2
		\end{pmatrix}, &\text{et}&&b=\begin{pmatrix}
			-4	\\ 
			10	\\ 
			5	
		\end{pmatrix}.
	\end{aligned}
\end{equation}
En termes de problèmes, on écrit $F\big( x,(A,b) \big)=Ax-b$. La donnée de ce problème est le couple $(A,b)$.

En ce qui concerne l'algorithme, on pose comme premier problème
\begin{equation}
	F_1\big(x_1,(A_1,b_1)\big)=A_1x_1-b_1=0
\end{equation}
avec $A_1=A$ et $b_1=b$.

Ensuite, on commence à échelonner et le second problème est
\begin{equation}
	F_2\big(x_2,(A_2,b_2)\big)=A_2x_2-b_2=0
\end{equation}
avec 
\begin{equation}
	\begin{aligned}[]
		A&=\begin{pmatrix}
			2	&	4	&	-6	\\
			0	&	3	&	6	\\
			0	&	1	&	5
		\end{pmatrix}, &\text{et}&&b=\begin{pmatrix}
			-4	\\ 
			12	\\ 
			13	
		\end{pmatrix}.
	\end{aligned}
\end{equation}
Le troisième problème sera
\begin{equation}
	F_3\big(x_3,(A_3,b_3)\big)=A_3x_3-b_3=0
\end{equation}
avec 
\begin{equation}
	\begin{aligned}[]
		A&=\begin{pmatrix}
			2	&	4	&	-6	\\
			0	&	3	&	6	\\
			0	&	0	&	3
		\end{pmatrix}, &\text{et}&&b=\begin{pmatrix}
			-4	\\ 
			12	\\ 
			3	
		\end{pmatrix}.
	\end{aligned}
\end{equation}
Ce problème est facile à résoudre «à la main». Nous nous arrêtons donc ici avec l'algorithme, et nous trouvons le $x_3$ qui résous le problème $F_3$.

L'algorithme de résolution de systèmes linéaires d'équations a les propriétés suivantes, à mettre en contraste avec celles de Newton :
\begin{enumerate}

	\item
		Pour résoudre le problème numéro $n$, il n'a pas fallu résoudre le problème numéro $n-1$.
	\item
		Toutes les solutions $x_n$ des problèmes intermédiaires sont solutions du problème de départ. Nous avons $F_n(x,d_n)=0$ pour tout $n$ (ici, $d_n=(A_n,b_n)$).
	\item
		D'un problème à l'autre, les données changent énormément : la matrice échelonnée peut être très différente de la matrice de départ.

\end{enumerate}


%---------------------------------------------------------------------------------------------------------------------------
\subsection{Définitions}
%---------------------------------------------------------------------------------------------------------------------------

	Nous allons maintenant formaliser en donnant quelque définitions pour nommer les propriétés que nous avons vues. D'abord, un algorithme est une suite de problèmes. Un \defe{algorithme}{algorithme} pour résoudre un problème $F(x,d)=0$ est une suite de problèmes $\{F_n(x_n,d_n)=0\}_{n\in\eN}$.  

\begin{definition}
	Un tel algorithme est dit  \defe{fortement consistant}{algorithme!fortement consistant} si pour toutes données admissibles $d_n$, on a
	\begin{equation}
		F_n(x,d_n)=0\quad\forall \;n,
	\end{equation}
	où $x$ est la solution de $F(x,d)=0$.
\end{definition}
L'algorithme des matrices est fortement consistant, mais pas l'algorithme de Newton.

\begin{definition}
	Un algorithme est \defe{consistant}{algorithme!consistant} si $\lim_{n\to\infty}F_n(x,d_n)=0$.
\end{definition}
Dans le cas de l'algorithme de Newton, c'est plutôt une telle consistance qu'on attend.

L'algorithme est dit \defe{stable}{algorithme!stable} si pour tout $n$ le problème correspondant est stable.  Dans ce cas, on note $K^{\mbox{num}}$ le  \defe{conditionnement relatif asymptotique}{Conditionnement!relatif asymptotique} défini par
\begin{equation}
	K^{\mbox{num}}=\limsup_nK_n
\end{equation}
où $K_n$ est le conditionnement relatif du problème $F_n(x_n,d_n)=0$.

\begin{definition}		\label{DefAlgoConverge}
	Un algorithme est dit \defe{convergent}{algorithme!convergent} (en $d$) si pour tout $\epsilon>0$, il existe $N=N(\epsilon)$ et $\delta=\delta(N,\epsilon)$ tels que pour $n\geq0$ et $|d-d_n|<\delta$, on ait $|x(d)-x_n(d_n)|<\epsilon$.
\end{definition}

\begin{remark}		\label{RemConvAlgoNewton}
Dans le cas de l'algorithme de Newton, nous avons vu que la donnée $d_n$ du problème $F_n$ était en fait la même que la donnée initiale $d$, donc nous avons $d_n=d$, et par conséquent nous avons toujours $| d-d_n |<\delta$. Dans ce cas, la définition de la convergence revient à demander que la suite numérique des $x_n$ converge vers la solution $x$.
\end{remark}

\begin{remark}
Dans le cas des matrices par contre, les données sont très différentes les unes des autres, nous avons donc en général que $| d-d_n |>\delta$. Mais en revanche nous savons que tous les problèmes intermédiaires $F_n$ acceptent une solution unique\footnote{Nous n'envisageons que le cas où le déterminant est non nul.} $x_n(d_n)=x(d)$. Par conséquent, $| x_n(d_n)-x(d) |$ est toujours plus petit que $\epsilon$. L'algorithme des matrice est donc toujours un algorithme convergent.
\end{remark}

%+++++++++++++++++++++++++++++++++++++++++++++++++++++++++++++++++++++++++++++++++++++++++++++++++++++++++++++++++++++++++++
\section{Représentations numériques, erreurs}
%+++++++++++++++++++++++++++++++++++++++++++++++++++++++++++++++++++++++++++++++++++++++++++++++++++++++++++++++++++++++++++

\begin{definition}
	Soit $x$ un réel. On définit sa \defe{représentation en virgule fixe}{Représentation!virgule fixe} par 
	\begin{equation}
		x=\{[x_nx_{n-1}...x_0,x_{-1}...x_{-m}], b, s\}
	\end{equation}
	avec  $b\in\eN, b\geq2$, $s\in\{0,1\}$ et $x_j\in\eN,x_j<b$ suivant la formule
	\begin{equation}
		x=(-1)^{s}\sum_{j=-m}^nx_j.b^j.
	\end{equation}
	On définit sa \defe{représentation en \href{http://docs.python.org/tutorial/floatingpoint.html}{virgule flottante} normalisée}{Représentation!virgule flottante normalisée} par 
	\begin{equation}
		\{[a_1...a_t],b,e,s\}
	\end{equation}
	où $e\in\eZ,e\in[L,U]$ et $a_j\in\eN;0\leq a_j<b; a_1\geq1$ suivant la formule
	\begin{equation}		\label{EqRepreFlotNOrm}
		x=(-1)^sb^e\sum_{j=1}^ta_jb^{-j}.
	\end{equation}
\end{definition}

\begin{definition}
	L'\defe{erreur relative}{Erreur relative} commise en remplaçant un nombre réel $x$ par une valeur approchée $\hat{x}$ est définie par 
	\begin{equation}
		\epsilon_x:=\left|\frac{x-\hat{x}}{x}\right|.
	\end{equation}
\end{definition}

%+++++++++++++++++++++++++++++++++++++++++++++++++++++++++++++++++++++++++++++++++++++++++++++++++++++++++++++++++++++++++++
\section{Permutations}
%+++++++++++++++++++++++++++++++++++++++++++++++++++++++++++++++++++++++++++++++++++++++++++++++++++++++++++++++++++++++++++

On rappelle que toute permutation de $N$ objets s'écrit de manière unique comme composée de cycles disjoints. De plus, tout cycle de longueur $k$ s'écrit comme composée de $k-1$ transpositions. En particulier, pour chaque permutation $\sigma$,  il existe un nombre bien défini $t(\sigma)$ de transpositions dont la composée est $\sigma$. Le signe $\epsilon(\sigma):=(-1)^{t(\sigma)}$ est appelé la \defe{signature}{signature} de la permutation $\sigma$.

