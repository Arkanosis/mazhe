% This is part of the Exercices et corrigés de mathématique générale.
% Copyright (C) 2009
%   Laurent Claessens
% See the file fdl-1.3.txt for copying conditions.
\begin{corrige}{TP50002}

	Étant donné que nous avons seulement $3$ équations pour $4$ inconnues, nous savons que nous allons pouvoir résoudre en laissant (au moins) une variable dans la solution. Étant donné qu'il n'y a pas de zéros dans la colonne du $t$, je propose de choisir celle-là : c'est la colonne la moins appétissante, donc autant la choisir.
	Au départ, le système s'écrit sous la forme
	\begin{equation}
		\left(
		\begin{array}{cccc|c}
			1	&	2	&	1	&	-1	&	3	\\
			2	&	-1	&	0	&	1	&	2	\\
			0	&	5	&	2	&	-3	&	4
		\end{array}
		\right).
	\end{equation}
	Nous allons essayer de ne pas casser les zéros déjà présents. La première chose à faire est d'annuler la première case de la première ligne en utilisant la seconde : $L_1\to L_1-L_2/2$. Nous trouvons
	\begin{equation}
		\left(
		\begin{array}{cccc|c}
			0	&	5/2	&	1	&	-3/2	&	2	\\
			2	&	-1	&	0	&	1	&	2	\\
			0	&	5	&	2	&	-3	&	4
		\end{array}
		\right).
	\end{equation}
	Ensuite, nous pouvons annuler la case du $z$ dans la première ligne en utilisant la troisième : $L_1\to L_1-L_3/2$. Nous avons alors
	\begin{equation}
		\left(
		\begin{array}{cccc|c}
			0	&	0	&	0	&	0	&	0	\\
			2	&	-1	&	0	&	1	&	2	\\
			0	&	5	&	2	&	-3	&	4
		\end{array}
		\right).
	\end{equation}
	La première ligne est maintenant complètement nulle, on peut la barrer. Il ne reste donc que deux équations pour quatre inconnues, et nous pouvons donc choisir une seconde variable par rapport à laquelle nous ne voulons pas résoudre le système (une variable qui va être reléguée au rang de simple paramètre dans la solution). Étant donné que la colonne des $y$ est sans zéros, nous choisissons celle-là. Le système est donc
		\begin{subequations}
			\begin{numcases}{}
				2x-y+t=2\\
				5y+2z-3t=4.
			\end{numcases}
		\end{subequations}		
	Comme nous avons décidé de laisser $y$ et $t$, ce système est résolu. La première équation donne directement
	\begin{equation}
		x=\frac{ 1 }{2}(y-t+2)
	\end{equation}
	et la seconde donne
	\begin{equation}
		z=\frac{ 1 }{2}(3t-5y+4).
	\end{equation}
	
	

\end{corrige}
