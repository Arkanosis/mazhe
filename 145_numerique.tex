% This is part of Mes notes de mathématique
% Copyright (C) 2010-2013,2016
%   Laurent Claessens
% See the file LICENCE.txt for copying conditions.

D'autres lectures agréables dans \cite{GianlucaB}

%+++++++++++++++++++++++++++++++++++++++++++++++++++++++++++++++++++++++++++++++++++++++++++++++++++++++++++++++++++++++++++
\section{Conditionnement et stabilité}
%+++++++++++++++++++++++++++++++++++++++++++++++++++++++++++++++++++++++++++++++++++++++++++++++++++++++++++++++++++++++++++

\begin{definition}      \label{DEFooYIFAooSJbMkC}
	Soit $F$ une fonction à valeurs réelles définie sur $X\times D$ où $X$ et $D$ sont des espaces vectoriels réels normés. Le problème de la recherche des solutions de
	\begin{equation}
		F(x,d)=0
	\end{equation}
	est dit \defe{stable}{stable} autour de \( d_0\in D\) si
	\begin{enumerate}
		\item
			la solution $x=x(d)$ existe et est unique pour tout $d$;
		\item \label{ItemProbStableB}
			Pour tout $\eta>0$, et pour tout $d_0$, il existe un nombre $K>0$ tel que $\| d-d_0\|<\eta$ entraine $\|x(d)-x(d_0)\|\;\leq\;K\;\|d-d_0\|$.
	\end{enumerate}
    La seconde condition est le fait que \( x\) soit Lipschitz\footnote{Définition \ref{DEFooQHVEooDbYKmz}.} sur un voisinage de \( d_0\).
\end{definition}

\begin{example}[Stabilité de la différence]    \label{ExooXJONooTAYZVc}
    Prenons le problème qui consiste à calculer la différence entre deux nombres : \( x=a-b\). Cela se traduit par
    \begin{equation}
        \begin{aligned}
            F\colon \eR\times \eR^2&\to \eR \\
            x&\mapsto x-a+b. 
        \end{aligned}
    \end{equation}
    Nous avons : 
    \begin{subequations}
        \begin{align}
            \big| x(a,b)-x(a',b') \big|&=| a-b-a'+b' |\\
            &\leq| a-a' |+| b-b' |\\
            &=\|  (a,b)-(a',b')  \|_1
        \end{align}
    \end{subequations}
    où nous avons utilisé la norme \( \| . \|_1\) sur \( \eR^2\). Par la proposition \ref{PropLJEJooMOWPNi} sur les équivalences de normes, le nombre \( K=\sqrt{2}\) fonctionne pour toute valeurs de \( \eta\).

    La problème de la différence est donc un problème stable.
\end{example}

\begin{example}[Stabilité de la multiplication]
    Si \( a\) est fixé, le problème de calculer \( ab\) (\( b\) est la donnée) est stable. En effet ce problème est donné par la fonction \( F(x,b)=x-ab\), dont la solution est \( x(b)=ab\). Nous avons donc
    \begin{equation}
        \big| x(b)-x(b') \big|=| ab-ab' |=| a | |b-b' |.
    \end{equation}
    La constante de Lipschitz de ce problème est donc \( | a |\).
\end{example}

\begin{definition}
    Le nombre
    \begin{equation}        \label{EqDefAABSOLU}
	    K_{abs}(d_0,\eta):=\sup_{d\text{ tel que $|d_0-d|<\eta$}}\frac{\| x(d)-x(d_0)\|_X}{\|d-d_0\|_D}
    \end{equation}
    est appelé le \defe{conditionnement absolu}{conditionnement!absolu} du problème autour de $d_0$.

	Soit $F(x,d)=0$ un problème stable de conditionnement absolu $K_{\text{abs}}(d,\eta)$.  Le conditionnement relatif est défini par
    \begin{equation}        \label{DEFEQooSXDBooYbvGrC}
		K_{\text{rel}}(d,\eta):=K_{\text{abs}}(d,\eta)\frac{\| d \|_D}{\|x(d)\|_X}.
	\end{equation}
	Le problème est dit \defe{bien conditionné}{bien!conditionné} près de $d$ si $K_{\text{rel}}(d,\eta)$ est petit.
\end{definition}

\begin{example}[Mauvais conditionnement de la différence]
    Reprenons le problème de la différence, mais en fixant \( a\). Nous avons donc \( x(b)=a-b\) et le conditionnement absolu est
    \begin{equation}
        \sup\frac{ | x(b)-x(b_0) | }{ | b-b_0 | }=1
    \end{equation}
    Le conditionnement relatif est : 
    \begin{equation}
        K_{rel}(b_0,\eta)=\frac{ | b | }{ | a-b | }.
    \end{equation}
    Et donc le problème est mal conditionné autour de \( a\).

    Autrement dit, si \( a'\) est un nombre proche de \( a\), calculer la différence \( a-a'\) est un problème mal conditionné.
\end{example}

\begin{example}[Bon conditionnement de la multiplication]
    Pour le problème \( F(x,b)=x-ab\) nous avons
    \begin{equation}
        K_{abs}=\sup_{b'}\frac{ | ab-ab' | }{ | b-b' | }=| a |.
    \end{equation}
    Et aussi
    \begin{equation}
        K_{rel}=a\frac{ | b | }{ | ab | }=1.
    \end{equation}
    Le conditionnement relatif du problème de la multiplication est donc toujours \( 1\). Il est donc un toujours un problème bien conditionné.
\end{example}

Ne pas confondre :
\begin{description}
	\item[Le conditionnement] provient du problème lui-même.
	\item[La stabilité] provient de l'algorithme de résolution.
\end{description}

\begin{example}[Un problème mal conditionné]
	Le système
	\begin{subequations}
		\begin{numcases}{}
			2.1x +  3.5y = 8 \\
			4.19x + 7.0y = 15
		\end{numcases}
	\end{subequations}
	Solution : \( x=100\), \( y=  -57.714285\ldots \) (périodique)

	Perturbons : nous remplaçons \( 4.19\) par \( 4.192\). L'erreur relative est : \( 4.77\times 10^{-4}\).

	Solution : \( \bar x=125\), \( \bar y=-72.714285\ldots\), avec donc erreur relative de \( 0.26\). Autrement dit : l'erreur relative sur la solution est grande même avec une petite erreur relative sur la donnée.

	C'est un problème mal conditionné.

	Le fait est que c'est une intersection de deux droites presque parallèles. Donc effectivement une petite perturbation d'une des deux droites donne une grande perturbation du point d'intersection.

	Le fait est qu'un ordinateur effectue \emph{toujours} une perturbation, au moins de l'ordre \( 10^{-16}\) pour ne fut-ce que représenter les nombres. C'est à dire une perturbation sur les six nombres définissant le système. Il n'y a donc pas d'espoir d'obtenir un algorithme donnant une bonne réponse.
\end{example}

Un résultat pratique pour étudier le conditionnement d'un problème est le suivant.
\begin{corollary}       \label{CorConditionnementNormeNabla}
	Soit $x=x(d)$ un problème stable. Supposons $\eD$ de dimension finie, supposons que $U$ est ouvert dans $\eD$. Supposons encore $x\colon U\to \eR$ différentiable en $d_0$. Alors quand $\eta$ est petit, on a
	\begin{equation}
		K_{\text{abs}}^{\eta}(d_0)\sim \| \nabla x(d_0) \|.
	\end{equation}
\end{corollary}

\begin{lemma}   \label{LemITCxqyS}
	 Tout  problème de la forme $x=x(d)$ avec $d\in\eR$ et $x \in C^1(\eR)$ est stable.
\end{lemma}

\begin{proof}
	Il faut démontrer qu'une fonction $C^1$ sur $\eR$ vérifie automatiquement la condition \ref{ItemProbStableB} de la définition de la stabilité. Pour cela, remarquons qu'une fonction $C^1$ possède une dérivée continue, et donc bornée sur tout compact\footnote{Un compact est un ensemble fermé et borné, typiquement un intervalle du type $[a,b]$.}

	Prenons $\eta>0$ et $d_0\in\eR$ et puis un $d$ tel que $| d-d_0 |<\eta$. Par le théorème des bornes atteintes, la fonction $x'$ est bornée sur l'intervalle $[d_0-\eta,d_0+\eta]$. Appelons $K$ un majorant de $x'$ sur cet intervalle. La fonction
	\begin{equation}
		f(d)=x(d_0)+K| d-d_0 |
	\end{equation}
	majore $x(d)$, et donc on a
	\begin{equation}
		\big| x(d)-x(d_0) \big|\leq K| d-d_0 |.
	\end{equation}

	Attention : vérifier si ce raisonnement est correct avec $d_0>d$, et adapter au besoin.
\end{proof}

\begin{example} \label{ExRZrOeoi}
	Un exemple de problème stable de la forme  $x=x(d)$ avec $d\in\eR$ et $x \in C^0(\eR)\setminus C^1(\eR)$.

	La fonction
	\begin{equation}
		x(d)=\begin{cases}
			0   &   \text{si $x\geq 0$}\\
			x   &   \text{si $x>0$}
		\end{cases}
	\end{equation}
	est continue, mais pas $C^1$ (non dérivable en $x=0$). La dérivée est partout bornée par $1$, et donc le problème est stable.

	Un autre exemple très classique serait de prendre $x(d)=| d |$. Dans ce cas, on peut prendre n'importe que $\eta$ et $K=1$. Le calcul est que
	\begin{subequations}
		\begin{align}
			| x(d)-x(d_0) |&<K| d-d_0 |\\
			\big| | d |-| d_0 | \big|&<| d-d_0 |.
		\end{align}
	\end{subequations}
	Cette dernière inéquation est correcte, comme on peut le voir en mettant au carré les deux membres.

\end{example}

\begin{example} \label{PIluknK}
	Un exemple de problème instable de la forme $x=x(d)$ avec $d\in\eR$ et $x \in C^0(\eR)$.

	Un exemple assez classique de fonction dont la dérivée n'est pas bornée sans pour autant que la fonction aie un comportement immoral\footnote{Penser à $x\mapsto x\sin(1/x)$.} est $x\mapsto\sqrt{x}$. Afin d'avoir une fonction définie sur $\eR$ tout entier, nous regardons la fonction
	\begin{equation}
		x(d)=\sqrt{|d|}.
	\end{equation}
	Si nous considérons maintenant $d_0=0$ et n'importe quel $\eta$, nous avons
	\begin{equation}
		\frac{ | x(d)-x(d_0) | }{ | d-d_0 | }=\frac{ \sqrt{d} }{ d }=\frac{1}{ \sqrt{d} }.
	\end{equation}
	Il n'est pas possible de trouver un $K$ qui majore ce rapport. Le problème est donc mal conditionné.

	Attention : dans ce calcul nous avons supposé $d>0$. Pensez à adapter au cas $d<0$.
\end{example}

\begin{example}[Problème bien conditionné avec algorithme instable]
	Soit à calculer
	\begin{equation}
		I_n=\frac{1}{ e }\int_0^1x^ne^xdx
	\end{equation}
	avec \( n\geq 0\). Par partie, nous obtenons :
	\begin{equation}
		I_n=1-nI_{n-1}.
	\end{equation}
	D'autre part, \( I_0=\frac{ e-1 }{ e }\), \( I_1=\frac{1}{ e }\). Puis par récurrence, c'est tout en main.

	Du côté de l'ordinateur, nous lui donnons forcément une approximation de \( I_1\), parce que nous lui donnons une approximation de \( e\). Soit l'erreur \( \epsilon_1\) sur \( I_1\).

	Sans démonstration :
	\begin{lemma}
		Nous avons \( \lim_{n\to \infty} I_n=0\).
	\end{lemma}
	Mais numériquement, il n'est pas possible de rester longtemps sous \( \epsilon_1\) parce que nous n'espérons pas avoir une erreur plus petite que ça. Donc à partir du moment où \( I_n<\epsilon_1\), les valeurs sont toutes complètement fausses. Cela est le mieux que l'on puisse espérer. Mais la réalité est pire.

	En réalité, en lançant le calcul sur un ordinateur, les valeurs sont même croissantes avec \( n\) à partir d'un certain moment.

	On peut étudier l'erreur et montrer que l'erreur est donnée par :
	\begin{equation}
		\epsilon_n=(-1)^{n-1}n!\epsilon_1.
	\end{equation}
	Mais comme la factorielle est tellement forte que c'est sans espoir d'aller loin en essayant très fort de donner une petite erreur sur \( \epsilon_1\).

\end{example}

Il existe heureusement un algorithme stable pour cette intégrale. La formule est :
\begin{equation}
	I_{n-1}=\frac{1}{ n }(1-I_n).
\end{equation}
Si nous savons un \( I_N\) avec \( N\) grand, cette formule donne les \( I_i\) avec \( i=N,N-1,\ldots, 2\). Posons donc \( I_N=a\in \eR\) n'importe comment. Donc \( \epsilon_N\) est grand. Mais il se trouve que l'erreur sur \( \epsilon_1\) est donnée par
\begin{equation}
	\epsilon_1=\frac{ (-1)^{N-1} }{ N! }\epsilon_N.
\end{equation}
Donc même en prenant vraiment n'importe quoi pour \( I_N\), nous obtenons de bonnes approximations pour \( I_i\) avec les petits \( i\). Même avec \( I_{20}=1000\) (qui est complètement faux), nous trouvons énormément de chiffres significatifs corrects pour \( I_1\).

%---------------------------------------------------------------------------------------------------------------------------
\subsection{Comment choisir et penser le $K$?}
%---------------------------------------------------------------------------------------------------------------------------

La formule \eqref{EqDefAABSOLU} contient une formule qui ressemble étrangement à la dérivée. La stabilité d'un problème est très liée à la dérivée de $F$. La stabilité et la dérivée ne sont pas les mêmes choses, mais il n'est pas mauvais de penser au $K$ de la stabilité comme la dérivée. Ou plus précisément : le supremum de la dérivée.

Un fil conducteur du lemme \ref{LemITCxqyS} et des exemples \ref{ExRZrOeoi}, \ref{PIluknK} est que l'on a un $K$ qui fonctionne lorsque la dérivée est bornée sur l'intervalle $\mathopen] d_0-\eta , d_0+\eta \mathclose[$. Dans le cas où ce supremum existe, le prendre en guise de $K$ fonctionne souvent.

Il faut cependant parfois faire acte d'imagination. La fonction $x\mapsto| x |$ n'est pas dérivable en $0$. Il n'empêche que $K=1$ fait fonctionner la définition de la stabilité. Remarquez que $K=1$ est le supremum de la dérivée là où elle existe.

À partir du moment où c'est clair que le $K$ est le supremum de la dérivée, on comprend pourquoi c'est le gradient qui arrive dans le corollaire \ref{CorConditionnementNormeNabla}. En effet, le gradient indique la direction de plus grande pente. C'est donc bien dans cette direction qu'il faut chercher la «plus grande dérivée».

\begin{proposition}
	Pour le problème stable $x=x(d)$ avec $x\in C^1(\eR^n,\eR)$, on a
	\begin{equation}
		K_{abs}(d)\sim|dx_d|_{\mbox{op}}
	\end{equation}
	où \( dx_d\) désigne la différentielle de $x$ en $d$.
\end{proposition}

%+++++++++++++++++++++++++++++++++++++++++++++++++++++++++++++++++++++++++++++++++++++++++++++++++++++++++++++++++++++++++++ 
\section{Conditionnement d'une matrice}
%+++++++++++++++++++++++++++++++++++++++++++++++++++++++++++++++++++++++++++++++++++++++++++++++++++++++++++++++++++++++++++

Soit le système d'équations linéaires \( Au=b\) avec la matrice inversible \( A\) ainsi que le système perturbé \( (A+\Delta A)u'=(b+\Delta b)\). Nous notons \( \Delta u=u'-u\) et nous voudrions pouvoir dire des choses de l'erreur relative \( \frac{ \| \Delta u \| }{ \| u \| }\).

\begin{example}[\cite{ooLMMRooUXhOdx}]
    Soit la matrice
    \begin{equation}
        A=\begin{pmatrix}
            10    &   7    \\ 
            7    &   5    
        \end{pmatrix}
    \end{equation}
    et \( b=\begin{pmatrix}
        32    \\ 
        23    
    \end{pmatrix}\). La solution de \( Au=b\) est \( u=\begin{pmatrix}
        -1    \\ 
        6    
    \end{pmatrix}\). Si nous conservons la même matrice mais nous considérons \( b=\begin{pmatrix}
        32.1    \\ 
        22.9    
    \end{pmatrix}\). La solution devient \( u'=\begin{pmatrix}
        0.2    \\ 
        4.3    
    \end{pmatrix}\)

    En norme \( \| . \|_{\infty}\) nous avons\footnote{La proposition \ref{PropLJEJooMOWPNi}\ref{ItemABSGooQODmLNiii} montre que si nous voulions des estimations en norme \( \| . \|_2\), il y aurait au maximum un facteur \( \sqrt{2}\) par-ci par là.}
    \begin{equation}
        \frac{ \| \Delta b \| }{ \| b \| }=\frac{ 0.1 }{ 32 }=0.003125
    \end{equation}
    et
    \begin{equation}
        \frac{ \| \Delta u \| }{ \| u \| }=\frac{ 1.7 }{ 6 }=0.28.
    \end{equation}
    Cela montre environ amplification d'un facteur \( 100\) entre l'erreur sur \( b\) et l'erreur sur la solution.
\end{example}

\begin{definition}
    Le \defe{conditionnement}{conditionnement!d'une matrice inversible} de la matrice inversible \( A\in \GL(n,\eC)\) est le nombre positif
    \begin{equation}
        \Cond(A)=\| A \|\| A^{-1} \|.
    \end{equation}
\end{definition}

Cette dénomination sera justifié par le corollaire \ref{CORooXKPWooJVHVvh} parce qu'il est évident que le conditionnement d'une matrice est lié au conditionnement du problème de résolution d'un système linéaire.

\begin{remark}
    Le conditionnement dépend de la norme choisie, mais cette dependence est contrôlée par la proposition \ref{PropLJEJooMOWPNi} qui nous indique que si le conditionnement d'une matrice est grand dans une norme, il sera grand dans une autre norme.

    D'autre part, lorsque nous écrirons \( \| A \|\) nous supposerons toujours que \( \| . \|\) est une norme d'algèbre\footnote{Définition \ref{DefJWRWQue}.} et donc que nous avons toujours
    \begin{equation}
        \| AB \|\leq \| A \|\| B \|.
    \end{equation}
    De plus nous supposerons toujours avoir une norme subordonnée à une norme sur l'espace \( \eC^n\), de telle sorte à avoir
    \begin{equation}
        \| Au \|\leq \| A \|\| u \|
    \end{equation}
    pour tout \( u\in\eC^n\). Voir aussi la proposition \ref{PropEDvSQsA}.
\end{remark}

\begin{proposition}[\cite{ooLMMRooUXhOdx}]
    Si \( A\) est une matrice inversible et si \( \alpha\in \eC\) nous avons :
    \begin{enumerate}
        \item
            \( \Cond(A)\geq 1\)
        \item
            \( \Cond(A)=\Cond(A^{-1})\)
        \item
            \( \Cond(\alpha A)=\Cond(A)\).
    \end{enumerate}
    Si \( Q\in\gO(n)\) alors
    \begin{enumerate}
        \item
            Nous avons \( \Cond_2(Q)=1\) où \( \Cond_2\) est le conditionnement pour la norme \( \| . \|_2\).
        \item
            Nous avons aussi
            \begin{equation}
                \Cond_2(A)=\Cond_2(AQ)=\Cond_2(QA).
            \end{equation}
    \end{enumerate}
\end{proposition}

\begin{proof}
    Nous savons que \( \Cond(\mtu)=1\) et donc
    \begin{equation}
        1=\| \mtu \|\leq \| A \|\| A^{-1} \|
    \end{equation}
    parce que la norme utilisée est une norme matricielle.

    Les deux autres formules sont évidentes à partit du fait que la définition du conditionnement de \( A\) est symétrique entre \( A\) et \( A^{-1}\).

    En ce qui concerne les formules relatives à la matrice orthogonale \( Q\) nous savons par la proposition \ref{PropKBCXooOuEZcS}\ref{ITEMooOWMBooHUatNb} qu'une matrice orthogonale est une bijection de l'ensemble \(  \{ x\in \eR^n\tq \| x \|=1 \}  \). Par conséquent
    \begin{equation}
        \| AQ \|=\sup_{x\tq \| x \|=1}\| AQx \|=\sup_{ Q^{-1}x\tq \| x \|=1  }\| AQQ^{-1}x \|=\| A \|.
    \end{equation}
    Donc \( \| AQ \|=\| A \|\). Les assertions s'ensuivent immédiatement en remarquant que \( Q^{-1}\) est également orthogonale.
\end{proof}

\begin{lemma}[\cite{ooLMMRooUXhOdx}]   \label{LEMooHUGEooVYhZdZ}
    Si \( A\) est une matrice carré et inversible,
    \begin{equation}
        \Spec(A^*A)=\Spec(AA^*)
    \end{equation}
\end{lemma}

\begin{proof}
    Nous allons montrer l'égalité des polynômes caractéristiques. D'abord une simple multiplication montre que
    \begin{equation}
        (A^*A-\lambda\mtu)A^{-1}=A^{-1}(AA^*-\lambda\mtu).
    \end{equation}
    Nous prenons le déterminant de cette égalité en utilisant les propriétés \ref{PropYQNMooZjlYlA}\ref{ItemUPLNooYZMRJy} et \ref{ITEMooZMVXooLGjvCy} :
    \begin{equation}
        \det(A^*A-\lambda\mtu)\det(A^{-1})=\det(A^{-1})\det(AA^*-\lambda\mtu).
    \end{equation}
    En simplifiant par \( \det(A^{-1})\) (qui est non nul parce que \( A\) est inversible) nous obtenons l'égalité des polynômes caractéristiques et donc l'égalité des spectres.
\end{proof}

Soit une matrice inversible \( A\in \GL(n,\eC)\). La matrice \( A^*A\) est hermitienne\footnote{Définition \ref{DEFooKEBHooWwCKRK}.} et le théorème \ref{LEMooVCEOooIXnTpp} nous assure que ses valeurs propres sont réelles. Par la remarque \ref{REMooMLBCooTuKFmz}, ses valeurs propres sont même positives.
\begin{proposition}[\cite{ooLMMRooUXhOdx}]      \label{PROPooNUAUooIbVgcN}
    Soit une matrice inversible \( A\in\GL(n,\eC)\), et \( \mu_1\leq\ldots\leq \mu_n\) les valeurs propres de \( A^*A\). Alors nous avons la formule
    \begin{equation}
        \Cond_2(A)=\sqrt{ \frac{ \mu_n }{ \mu_1 }}.
    \end{equation}
\end{proposition}

\begin{proof}
    Par le théorème \ref{THOooNDQSooOUWQrK}, la norme de \( A\) est liée au au rayon spectral de \( A^*A\) par
    \begin{equation}
        \| A \|_2=\sqrt{ \rho(A^*A) }=\sqrt{ \mu_n }.
    \end{equation}
    Vu que le spectre de \( AA^*\) est le même que celui de \( A^*A\) (lemme \ref{LEMooHUGEooVYhZdZ}) nous avons aussi
    \begin{equation}
        \| A^{-1} \|_2=\sqrt{ \rho\big( (A^{-1})^*A^{-1} \big) }=\sqrt{ \rho\big( (A^*A)^{-1} \big) }=\frac{1}{ \sqrt{ \mu_1 } }
    \end{equation}
    parce que la plus grande valeur propre de \( (A^*A)^{-1}\) est l'inverse de la plus petite de \( A^*A\).

    Ces deux calculs étant,
    \begin{equation}
        \Cond_2(A)=\| A \|_2\| A^{-1} \|_2=\sqrt{ \frac{ \mu_n }{ \mu_1 } }.
    \end{equation}
\end{proof}

%--------------------------------------------------------------------------------------------------------------------------- 
\subsection{Perturbation du vecteur}
%---------------------------------------------------------------------------------------------------------------------------

\begin{proposition}[Système linéaire : perturbation du vecteur\cite{ooLMMRooUXhOdx}]        \label{PROPooGIXFooAhJkIs}
    Soit une matrice inversible \( A\) et les systèmes d'équations linéaires
    \begin{subequations}        \label{EQooYQIGooPXqWoX}
        \begin{align}
            Au=b\\
            Au'=b'.
        \end{align}
    \end{subequations}
    En notant \( \Delta u=u'-u\) et \( \Delta b=b'-b\) nous avons
    \begin{equation}        \label{EQooESXRooMYuvRa}
        \frac{ \| \Delta u \| }{ \| u \| }\leq \Cond(A)\frac{ \| \Delta b \| }{ \| b \| }.
    \end{equation}
\end{proposition}

\begin{proof}
    En soustrayant les équations \eqref{EQooYQIGooPXqWoX} nous avons \( \Delta b=A\Delta u\), et donc \( \Delta u=A^{-1} \Delta b\). D'une part nous avons alors
    \begin{equation}
        \| \Delta u \|\leq \| A^{-1} \|\| \Delta b \|.
    \end{equation}
    Et d'autre part, \( \| b \|\leq \| A \|\| u \|\), ce qui donne
    \begin{equation}
        \frac{ \| b \| }{ \| A \| }\leq \| u \|.
    \end{equation}
    En mettant les deux ensemble,
    \begin{equation}
        \frac{ \| \Delta u \| }{ \| u \| }\leq \frac{ \| A^{-1} \|\| \Delta b \| }{ \| b \| }\| A \|=\Cond(A)\frac{ \| \Delta b \| }{ \| b \| }.
    \end{equation}
\end{proof}

Le corollaire suivant justifie le nom «conditionnement» au conditionnement d'une matrice.
\begin{corollary}       \label{CORooXKPWooJVHVvh}
    Soit \( A\in \GL(n,\eC)\) fixée et le problème de résoudre \( Au=b\), c'est à dire la fonction
    \begin{equation}
        F(u,b)=Au-b.
    \end{equation}
    \begin{enumerate}
        \item
            Ce problème est stable pour toute valeur de \( b\).
        \item
            Nous avons une majoration pour le conditionnement relatif\footnote{Si vous doutez de la norme à prendre, lisez la remarque \ref{REMooAIKIooJEBEqi}} :
            \begin{equation}        \label{EQooZHQJooTMKYfr}
                K_{rel}(\eta,b_0)\leq \Cond(A).
            \end{equation}
    \end{enumerate}
\end{corollary}

\begin{proof}
    \begin{subproof}
    \item[Stabilité]
        Vu que \( A\) est inversible, il existe une solution unique à tout système de la forme \( Au=b'\). De plus \( u(b)=A^{-1} b\), donc
        \begin{equation}
            \| u(b)-u(b_0) \|= \| A^{-1}(b-b_0) \|\leq \| A^{-1} \|\| b-b_0 \|,
        \end{equation}
        de telle sorte que la condition \ref{DEFooYIFAooSJbMkC}\ref{ItemProbStableB} fonctionne avec \( K=\| A^{-1} \|\).
    \item[Conditionnement]
        En partant de la définition \ref{DEFEQooSXDBooYbvGrC}, et en utilisant la majoration de la proposition \ref{PROPooGIXFooAhJkIs} sous la forme
        \begin{equation}
            \| u(b)-u(b_0) \|\leq \Cond(A)\| u(b_0) \|\frac{ \| \Delta b \| }{ \| b_0 \| },
        \end{equation}
        nous obtenons :
        \begin{subequations}
            \begin{align}
                K_{rel}(b_0,\eta)&=K_{abs}(b_0,\eta)\frac{ \| b_0 \| }{ \| u(b_0) \| }\\
                &=\sup_{\| b-b_0 \|\leq \eta}\frac{ \| u(b)-u(b_0) \| }{ \| b-b_0 \| }\frac{ \| b_0 \| }{ \| u(b_0) \| }
                &\leq \sup_b\Cond(A)\frac{ \| u(b_0) \| }{ \| b_0 \| }\| \Delta b \|\frac{1}{ \| b-b_0 \| }\frac{ \| b_0 \| }{ \| u(b_0) \| }\\
                &=\Cond(A).
            \end{align}
        \end{subequations}
    \end{subproof}
\end{proof}


\begin{remark}      \label{REMooAIKIooJEBEqi}
    La notion de conditionnement relatif dépend aussi de la norme choisie. Dans la formule \eqref{EQooZHQJooTMKYfr} il faut prendre le conditionnement \( \Cond(A)\) pour la norme dans laquelle le \( K_{rel}\) est écrit. Encore une fois, toutes les normes étant équivalentes,  cette majoration est à constante près bonne pour toutes les normes. Si la dimension est très grande, cette constante peut par contre être grande.
\end{remark}

%--------------------------------------------------------------------------------------------------------------------------- 
\subsection{Perturbation de la matrice}
%---------------------------------------------------------------------------------------------------------------------------

\begin{proposition}[Système linéaire : perturbation de la matrice\cite{ooLMMRooUXhOdx}]
    Soient les systèmes linéaires
    \begin{subequations}
        \begin{align}
            Au=b\\
            A'u'=b
        \end{align}
    \end{subequations}
    avec \( A\) et \( A'\) inversibles. Nous notons \( \Delta A=A'-A\). Alors
    \begin{enumerate}
        \item       \label{ITEMooJMTKooSEBavB}
            \begin{equation}
                \frac{ \| \Delta u \| }{ \| u' \| }\leq \Cond(A)\frac{ \| \Delta A \| }{ \| A \| }
            \end{equation}
        \item
            \begin{equation}
                \frac{ \| \Delta u \| }{ \| u \| }\leq \Cond(A)\frac{ \| \Delta A \| }{ \| A \| }\big( 1+\alpha(\| \Delta A \|) \big)
            \end{equation}
            où \( \lim_{x\to 0} \alpha(x)=0\).
    \end{enumerate}
\end{proposition}

\begin{proof}
    D'abord nous avons
    \begin{subequations}
        \begin{align}
            0&=Au'-Au\\
            &=(A'-A)u'-Au'-Au\\
            &=\Delta Au'+A\Delta u.
        \end{align}
    \end{subequations}
    Par conséquent, \( \Delta u=-A^{-1}(\Delta A)u'\) et
    \begin{equation}        \label{EQooYYITooSSczEj}
        \| \Delta u \|\leq \| A^{-1} \|\| \Delta A \|\| u' \|.
    \end{equation}
    Donc
    \begin{equation}
        \frac{ \| \Delta u \| }{ \| u' \| }\leq   \| A^{-1} \|\| A \|\frac{ \| \Delta A \| }{ \| A \| }   =\Cond(A)\frac{ \| \Delta A \| }{ \| A \| }.
    \end{equation}
    Cela est \ref{ITEMooJMTKooSEBavB}.

    Pour l'autre inégalité, nous avons \( A'=A+\Delta A\) et donc
    \begin{equation}
        \| A'^{-1} \|=\| (A+\Delta A)^{-1} \|
    \end{equation}
    Nous repartons alors de \eqref{EQooYYITooSSczEj} en changeant le rôle de \( A\) et \( A'\) (et donc aussi de \( u\) et \( u'\)). Ce changement étant, \( \| \Delta u \|\) et \( \| \Delta A \|\) ne changent pas. Nous avons :
    \begin{subequations}
        \begin{align}
            \frac{ \| \Delta u \| }{ \| u \| }&\leq \| A'^{-1} \|\| \Delta A \|\\
            &=\| (A+\Delta A)^{-1} \|\| \Delta A \|\frac{ \Cond(A) }{ \| A \|\| A^{-1} \| }\\
            &=\frac{ \| (A+\Delta A)^{-1} \| }{ \| A^{-1} \| }\frac{ \| \Delta A \| }{ \| A \| }\Cond(A).
        \end{align}
    \end{subequations}
    Il reste à voir que
    \begin{equation}
        \lim_{\| \Delta A \|\to 0} \frac{ \| (A+\Delta A)^{-1} \| }{ \| A^{-1} \| }=1,
    \end{equation}
    ou autrement dit que
    \begin{equation}        \label{EQooJURGooFvYiAs}
        \lim_{A\to A'} \frac{ \| A'^{-1} \| }{ \| A^{-1} \| }=1
    \end{equation}
    où la limite est celle dans \( \GL(n,\eC)\). Par définition de la topologie, la norme est continue (quelle qu'elle soit par l'équivalence de norme \ref{ThoNormesEquiv}). Par le théorème \ref{ThoCINVBTJ}, l'application \( A\mapsto A^{-1}\) est également continue et commute donc avec la limite. Nous avons donc
    \begin{equation}
        \lim_{A'\to A}\| A'^{-1} \|=\| (\lim_{A'\to A} A')^{-1} \|=\| A^{-1} \|.
    \end{equation}
    Donc la limite du quotient \eqref{EQooJURGooFvYiAs} est bien \( 1\).
\end{proof}

%+++++++++++++++++++++++++++++++++++++++++++++++++++++++++++++++++++++++++++++++++++++++++++++++++++++++++++++++++++++++++++ 
\section{Un peu de points fixes}
%+++++++++++++++++++++++++++++++++++++++++++++++++++++++++++++++++++++++++++++++++++++++++++++++++++++++++++++++++++++++++++

Pour l'équation \( f(x)=0\), il existe une infinité de fonctions \( g\) pour lesquelles l'équation est équivalente à \( x=g(x)\).

Exemple : \( f(x)=x^2-2-\ln(x)\), nous pouvons faire
\begin{enumerate}
    \item
        \( x=x^2-2-\ln(x)+x\)
    \item
        Poser \( x^2=2+\ln(x)\) et donc
        \begin{subequations}
            \begin{align}
                x=-\sqrt{2+\ln(x)}\\
                x=\sqrt{2+\ln(x)}.
            \end{align}
        \end{subequations}
    \item
        Ou encore 
        \begin{equation}
            x=\frac{ 2+\ln(x) }{ x }
        \end{equation}
        où nous savons déjà que \( x\neq 0\) parce que \( x=0\) n'est pas dans le domaine de \( f\).
    \item
        Ou par l'exponentielle :
        \begin{equation}
            x= e^{x^2-2}.
        \end{equation}
\end{enumerate}
Dans tous ces cas nous pouvons construire une suite \( (x_n)\) en posant un nombre arbitraire pour \( x_0\) et ensuite la récurrence 
\begin{equation}
    x_{n+1}=g(x_n).
\end{equation}

Graphiquement, la solution de l'équation est l'intersection entre les courbes \( y=x\) et \( y=g(x)\). Un petit dessin pour montrer la convergence :

\begin{center}
   \input{Fig_UEGEooHEDIJVPn.pstricks}
\end{center}

Attention : cette méthode ne converge pas toujours. Parfois elle converge de façon monotone, et parfois pas. Le choix de la fonction \( g\) qui fait \( x=g(x)\) peut énormément changer la vitesse de convergence.

\begin{theorem}[Condition suffisante pour existence d'un point fixe]
    Une fonction continue \( f\colon \mathopen[ a , b \mathclose]\to \mathopen[ a , b \mathclose]\) admet au moins un point fixe dans \( \mathopen[ a , b \mathclose]\).
\end{theorem}

\begin{theorem}[Condition suffisante pour l'unicité]   
    Soit \( f\) continue sur \( \mathopen[ a , b \mathclose]\) avec \( g(x)\in\mathopen[ a , b \mathclose]\) pour tout \( x\in\mathopen[ a , b \mathclose]\). Supposons qu'il existe \( 0<k<1\) tel que pour tout \( x\in\mathopen[ a , b \mathclose]\) nous ayons \( | g'(x) |\leq k\) alors 
    \begin{enumerate}
        \item
            La fonction \( g\) possède un unique point fixe dans \( \mathopen[ a , b \mathclose]\).
        \item
            Pour tout \( x_0\in\mathopen[ a , b \mathclose]\), tous les termes de la suite \( x_{n+1}=g(x_n)\) sont dans \( \mathopen[ a , b \mathclose]\).
        \item
            Ladite suite \( (x_n)\) converge vers le point fixe.
    \end{enumerate}
\end{theorem}

\begin{theorem}
    Soit \( f\) continue sur \( \mathopen[ a , b \mathclose]\) avec \( g(x)\in\mathopen[ a , b \mathclose]\) pour tout \( x\in\mathopen[ a , b \mathclose]\). Supposons
    \begin{enumerate}
        \item
    qu'il existe \( 0<k<1\) tel que pour tout \( x\in\mathopen[ a , b \mathclose]\) nous ayons \( | g'(x) |\leq k\) et
\item
    \( g\) est \( p\) fois dérivable sur \( \mathopen[ a , b \mathclose]\).
\item
    \( g'(\alpha)=g''(\alpha)=\ldots=g^{(p-1)}(\alpha)\) et \( g^{(p)}(\alpha)\neq 0\) où \( \alpha\) est l'unique point fixe.
    \end{enumerate}
    Alors la suite \( (x_n)\) converge avec un ordre \( p\).
\end{theorem}

\begin{proposition}     \label{PROPooICWMooWxNcsr}
    Soit \( \alpha\) tel que \( x=g(x)\), avec \( g\) continue sur un voisinage de \( \alpha\) et dérivable dans l'intérieur. Nous supposons que
    \begin{equation}
        | g'(\alpha) |<1.
    \end{equation}
    Alors il existe un rayon \( \delta\) tel que si \( x_0\in B(\alpha,\delta)\), la suite \( (x_n)\) converge vers \( \alpha\).
\end{proposition}
 
Certes cette proposition demande moins d'hypothèses, mais en réalité, il ne donne pas de vrais moyens de choisir un point de départ \( x_0\). Avec les deux théorèmes précédents, nous pouvions prendre \( x_0\) n'importe où dans \( \mathopen[ a , b \mathclose]\). Le fait est que pour choisir \( x_0\) nous pouvons tracer et donner à la main un \( x_0\) proche de ce qui semble être \( \alpha\). Si ça ne converge pas, il faut donner un \(x_0\) plus proche. La proposition nous assure que si nous jouons bien à choisir \( x_0\) très proche, la suite finira par converger.

Notons que la proposition \ref{PROPooICWMooWxNcsr} a encore l'inconvénient de demander de calculer \( g'(\alpha)\) alors que \( \alpha\) est inconnu. La résolution de l'inéquation \( | g'(x) |<1\) nous donne un certain nombre d'intervalles dans \( \eR\). Soient \( I_n\) les intervalles solutions de l'inéquation.

Si \( \alpha\in I_n\) alors la méthode converge. Sinon, c'est pas garantit. En tout cas nous ne devons pas savoir réellement \( \alpha\) pour appliquer le théorème. Il suffit de savoir que \( \alpha\) est dans un des \( I_n\).

\begin{example}
    Nous reprenons 
    \begin{equation}
        f(x)=x^2-2-\ln(x).
    \end{equation}
    Et nous voulons résoudre \( f(x)=0\). Graphiquement c'est l'intersection entre \( y=x^2-2\) et \( y=\ln(x)\). Il est vite tracé de savoir qu'il y a deux solutions  : \( \alpha_1\in\mathopen[ 0 , 1 \mathclose]\) et \( \alpha_2\in\mathopen[ \sqrt{2} , 2 \mathclose]\).

    Déjà un petit problème : l'intervalle \( \mathopen[ 0 , 1 \mathclose]\) ne va pas parce que \( f\) n'y est pas continue. Un petit raffinement d'analyse nous fournit \( \alpha_1\in\mathopen[ e^{-2} , 1 \mathclose]\).

    Nous avons au moins les fonctions de points fixes suivantes :
    \begin{subequations}
        \begin{align}
            g_1(x)=\sqrt{ 2+\ln(x) }\\
            g_2(x)=e^{x^2-2}.
        \end{align}
    \end{subequations}
    Pour la première, il y avait un \( \pm\) qui a été négligé parce que nous savons que les deux solutions cherchées sont positives.
    Travaillons avec la première. D'abord
    \begin{equation}
        g'_1(x)=\frac{ 1 }{ 2x\sqrt{ 2-\ln(x) } }.
    \end{equation}
    Nous avons \( \lim_{x\to e^{-2}} g'_2(x)=+\infty\). Il ne sera donc pas possible de trouver \( 0<k<1\) tel que \( | g'(x) |\leq k\). Tentons quand même la méthode :
    \begin{equation}
        x_0=0.5
    \end{equation}
    Il se fait que cela est plus proche de \( \alpha_1\) que de \( \alpha_2\). Mais en réalité la suite converge vers \( \alpha_2\).

    Passons à la seconde méthode. 
    \begin{equation}
        g'_2(x)=2xe^{x^2-2}.
    \end{equation}
    Sur l'intervalle \( \mathopen[ e^{-2} , 1 \mathclose]\), \( g'_2\) est croissante et prend toutes ses valeurs dans \( \mathopen[ e^{-2} , 1 \mathclose]\). Nous pouvons prouver que 
    \begin{equation}
        | g'_2(x) |\leq 2e^{-1}<1.
    \end{equation}
    Donc poser \( k=2e^{-1}\) fait fonctionner la proposition. Donc quel que soit le \( x_0\) pris dans cet intervalle, nous aurons une suite convergente vers un point fixe à l'intérieur de l'intervalle. C'est à dire convergente vers \( \alpha_1\).

    Cela est un exemple de problème pour lequel changer de fonction \( g\) change réellement la vie.
\end{example}

\begin{proposition}[\cite{ooGYJXooIWExXK}]      \label{PROPooRPHKooLnPCVJ}
    Soit \( g\colon \eR\to \eR\) de classe \( C^1\) et \( \alpha\) un point fixe attractif\footnote{Définition \ref{DEFooTMZUooMoBDGC}.} de \( g\). Alors il existe \( \delta>0\) et \( k\) tel que \( 0\leq k<1\) pour lesquels
    \begin{enumerate}
        \item       \label{ITEMooOQKMooTRSvUo}
            La fonction \( g\) est \( k\)-contractante\footnote{Définition \ref{DEFooRSLCooAsWisu}} sur \( B(\alpha,\delta)\).
        \item       \label{ITEMooFTAQooPBsBcR}
            Nous avons \( g\big( B(\alpha,\delta) \big)\subset B(\alpha,\delta)\).
        \item       \label{ITEMooFSOAooKlcxih}
            Pour tout \( x_0\in B(\alpha,\delta)\) la suite \( x_{n+1}=g(x_n)\) converge vers \( \alpha\) et
            \begin{equation}
                | x_n-\alpha |\leq | x_0-\alpha |k^n.
            \end{equation}
    \end{enumerate}
    Si de plus \( g'(\alpha)=0\) et \( g\) est de classe \( C^2\) alors nous avons convergence quadratique (définition \ref{DEFooSUTRooAcXXjj}).
\end{proposition}

\begin{proof}
    Vu que \( \alpha\) est un point fixe attractif de \( g\) nous pouvons considérer un \( k\) tel que \( | g'(\alpha) |<k<1\). Et comme \( g\) est de classe \( C^1\), la fonction \( g'\) est continue et donc bornée sur toute boule du type \( \overline{ B(\alpha,\delta) }\). Soit \( \delta\) le plus grand nombre tel que \( \| g' \|_{\overline{ B(\alpha,\delta) }}\leq k\). Nous notons \( I=\overline{ B(\alpha,\delta) } \) pour cette valeur de \( \delta\).

    Pour tout \( x\in I\) nous avons, en utilisant le théorème des accroissements finis \ref{ThoAccFinis}\ref{ITEMooXRQKooDBFpdQ} :
    \begin{subequations}        \label{SUBEQooYXLHooSCnnRA}
        \begin{align}
            | g(x)-\alpha |&=| g(x)-g(\alpha) |\\
            &\leq\sup_{t\in I}| g'(t) | |x-\alpha |\\
            &\leq k| x-\alpha |\\
            &<\delta
        \end{align}
    \end{subequations}
    parce que \( k<1\) et \(| x-\alpha |\leq \delta\). Par conséquent \( g(x)\in B(\alpha,\delta)\). Cela prouve le point \ref{ITEMooFTAQooPBsBcR}. Pour le point \ref{ITEMooOQKMooTRSvUo}, soient \( x,y\in B(\alpha,\delta)\) et
    \begin{equation}
        | g(x)-g(y) |\leq \sup_{a\in I}| g'(a) | |x-y |\leq k| x-y |.
    \end{equation}
    Pout le point \ref{ITEMooFSOAooKlcxih} nous avons \( | g(x_n)-\alpha |\leq k| x_n-\alpha |\), c'est à dire
    \begin{equation}
        | x_{n+1}-\alpha |\leq k| x_n-\alpha |.
    \end{equation}
    Le résultat annoncé s'obtient par récurrence sur \( n\).

    En ce qui concerne la convergence quadratique, c'est du Taylor. Développons \( g(x_n)\) autour de \( g(\alpha)\) :
    \begin{equation}
        g(x_n)=g(\alpha)+g'(\alpha)(x_n-\alpha)+\frac{ 1 }{2}(x_n-\alpha)^2\epsilon(x_n-\alpha)
    \end{equation}
    avec \( \lim_{t\to 0} \epsilon(t)=0\). En posant \( C=\frac{ 1 }{2}\sup_{t<\delta}| \epsilon(t) | \) nous avons $| g(x_n)-g(\alpha) |\leq C|x_n-\alpha  |^2$, c'est à dire
    \begin{equation}
        | x_{n+1}-\alpha |\leq C| x_n-\alpha |^2.
    \end{equation}
\end{proof}

%+++++++++++++++++++++++++++++++++++++++++++++++++++++++++++++++++++++++++++++++++++++++++++++++++++++++++++++++++++++++++++ 
\section{Méthode de Newton}
%+++++++++++++++++++++++++++++++++++++++++++++++++++++++++++++++++++++++++++++++++++++++++++++++++++++++++++++++++++++++++++
\label{SECooIKXNooACLljs}

\begin{definition}      \label{DEFooXSOQooAnWqKM}
    Le nombre \( \alpha\) est une \defe{racine simple}{racine!simple} de l'équation \( f(x)=0\) si \( f(\alpha)=0\) et \( f'(\alpha)\neq 0\). Le nombre \( \alpha\) est une \defe{racine multiple}{racine!multiple} d'ordre \( r\) de \( f(x)=0\) si\index{multiplicité!racine de \( f(x)=0\)}
    \begin{equation}
        f(\alpha)=f'(\alpha)=\ldots=f^{(r-1)(\alpha)}=0
    \end{equation}
    et \( f^{(r)}(\alpha)\neq 0\).
\end{definition}

\begin{example}
    La fonction \( x\mapsto x^3\) en \( x=0\) est un racine d'ordre \( 3\).
\end{example}

%--------------------------------------------------------------------------------------------------------------------------- 
\subsection{«Justification» de la formule par Taylor}
%---------------------------------------------------------------------------------------------------------------------------

    Soit une fonction \( f\) continue et dérivable sur \( \mathopen[ a , b \mathclose]\). Soit \( \alpha\) une racine de \( f\) et \( x_n\) une de ses approximation.  Nous notons l'erreur \( \theta\) et nous avons \( \alpha=x_n+\theta\). Du coup nous avons \( f(x_n+\theta)=f(\alpha)=0\). 

    Écrivons la série de Taylor du théorème \ref{ThoTaylor} autour de \( x_n\) : il existe une fonction \( \epsilon\colon \eR\to \eR\) telle que \( \lim_{t\to 0} \epsilon(t)=0\) telle que
    \begin{equation}        \label{EQooOPUBooYaznay}
        f(\alpha)=f(x_n+\theta)=f(x_n)+\theta f'(x_n)+\frac{ \theta^2 }{ 2 }\epsilon(\theta).
    \end{equation}
    Nous isolons le \( \theta\) du terme d'ordre \( 1\) en nous souvenant que le membre de gauche est nul :
    \begin{equation}
        \theta=-\frac{ f(x_n)-\theta^2\epsilon(\theta) }{ f'(x_n) }
    \end{equation}
    Vu que \( \alpha=x_n+\theta\), nous pouvons écrire
    \begin{equation}
        \alpha=x_n-\frac{ f(x_n)+\theta^2\epsilon(\theta) }{ f'(x_n) }.
    \end{equation}
    Il est donc raisonnable de poser
    \begin{equation}
        x_{n+1}=x_n-\frac{ f(x_n) }{ f'(x_n) }
    \end{equation}
    en espérant que cela soit une meilleur approximation de \( \alpha\) que \( x_n\).

    En tout cas l'erreur sur \( x_{n+1}\) est 
    \begin{equation}
        \alpha-x_{n+1}=x_n+\theta-x_n+\frac{ f(x_n)+\theta^2\epsilon(\theta) }{ f'(x_n) }=\theta+\frac{ f(x_n)+\theta^2\epsilon(\theta) }{ f'(n_n) },
    \end{equation}
    qui ne doit pas être fondamentalement plus grand que \( \theta\) dès que \( \theta\) est petit, surtout que si \( x_n\) est une approximation de \( \alpha\), nous pouvons espérer que \( f(x_n)\) soit également petit. Là où les choses peuvent déraper en grand, c'est si \( f'(x_n)\) est petit.

Cette méthode de Newton ne converge pas toujours. Le pire est lorsque par malheur il y a une bosse pas loin de la racine. Alors il y a un risque de tomber sur \( f'(x_{n+1})=0\) ou en tout cas très proche de zéro. Dans ce cas le point \( x_{n+2}\) est envoyé très loin.

%--------------------------------------------------------------------------------------------------------------------------- 
\subsection{«Justification» par points fixes}
%---------------------------------------------------------------------------------------------------------------------------

Nous savons que pour résoudre \( f(x)=0\) par une méthode de point fixe, il y a de nombreux choix possibles de fonctions \( g\) telles que \( g(x)=x\) donne la même solution que \( f(x)=0\). Soit \( \alpha\) une solution de \( f(x)=0\) et cherchons une fonction \( g\) de la forme
\begin{equation}        \label{EQooYVFIooJXnJXa}
    g(x)=x-kf(x).
\end{equation}
Nous savons par la proposition \ref{PROPooRPHKooLnPCVJ} que la fonction \( g\) donne une convergence quadratique lorsque \( g'(\alpha)=0\). Pour la forme \eqref{EQooYVFIooJXnJXa} nous avons \( g'(\alpha)=1-kf'(\alpha)\), ce qui nous donne l'idée de poser \( k=\frac{1}{ f'(\alpha) }\).

Le fait est que \( f'(\alpha)\) n'est pas connu, mais nous pouvons l'approximer par \( f'(x)\) lorsque \( x\) est proche de \( \alpha\). D'où l'idée de considérer la fonction
\begin{equation}
    g(x)=x-\frac{ f(x) }{ f'(x) },
\end{equation}
et donc la suite \( x_{n+1}=g(x_n)\) c'est à dire
\begin{equation}
    x_{n+1}=x-\frac{ f(x_n) }{ f'(x_n) }.
\end{equation}
Dès que \( x_n\) est proche de \( \alpha\), sous l'hypothèse (raisonnable par continuité) que \( f'(x_n)\) soit proche de \( f'(\alpha)\), la méthode devrait donner une convergence quadratique.

\begin{remark}
    Cette justification par points fixes n'est pas vraiment différente de celle par Taylor parce que Taylor est utlilisé dans la preuve de la proposition \ref{PROPooRPHKooLnPCVJ}.
\end{remark}

%--------------------------------------------------------------------------------------------------------------------------- 
\subsection{Formalisation de l'algorithme}
%---------------------------------------------------------------------------------------------------------------------------


La méthode de Newton consiste a exprimer la solution $x$ de $f(x)=0$ avec $f\in C^1(\eR)$ comme limite d'une suite $\{x_n\}_{n\in\eN}$ définie par récurrence par la formule
\begin{equation}
	x_{n+1}=x_n-\frac{f(x_n)}{f'(x_n)}.
\end{equation}
où $x_0$ est arbitraire.

Si on veut exprimer cela en terme d'algorithme, nous disons que l'algorithme de Newton est donné par la suite de problèmes
\begin{equation}        \label{EqFPourNewtonUn}
	F_n(x_{n+1},x_n,f)=x_{n+1}-x_n+\frac{ f(x_n) }{ f'(x_n) }.
\end{equation}
La donnée du problème est la fonction $f$, et rien que elle.

Plus précisément, une fois que la fonction $f$ est donnée, il existe une infinité de problèmes : pour chaque $a\in \eR$ nous avons le problème
\begin{equation}
	G_a(x_n,f)=x-a+\frac{ f(a) }{ f'(a) }.
\end{equation}
La méthode de Newton consiste à sélectionner une partie de ces problèmes de la façon suivante :
\begin{subequations}
	\begin{numcases}{}
		F_0 = G_{x_0}\\
		F_n = G_{x_n}.
	\end{numcases}
\end{subequations}
Le problème $F_0$ fournit un nombre $x_1$ qui nous permet de sélectionner le problème $G_{x_1}$ qui va fournir le nombre $x_2$, etc.

Au moment de calculer le conditionnement de $F_n$, nous ne devons pas voir $x_{n-1}$ comme fonction de $x_0$ et de la donnée $f$. Il ne faut donc pas dériver à travers les $x_n$.

\begin{proposition}
    Si une racine est multiple, alors l'ordre de convergence de la méthode de Newton est \( 1\).
\end{proposition}

Voici un algorithme possible :

\lstinputlisting{codeSnip_2.py}

Commentaires :
\begin{enumerate}
    \item
        Notons que dans un langage vraiment numérique comme Matlab, il faut passer \( f'\) en argument.
    \item
        Dans le \info{while} il faudrait mettre \( x_{n+1}-x_n\) (en valeur absolue), mais cette différence est aussi utilisée pour calculer \( x_{n+1}\) donc on la calcule une seule fois.
    \item
        Il faudrait faire une vérification sur \( f(x_n)\neq 0\). Il n'y a pas tellement de choix que de changer le point initial.
\end{enumerate}

%--------------------------------------------------------------------------------------------------------------------------- 
\subsection{Caractéristiques}
%---------------------------------------------------------------------------------------------------------------------------

L'algorithme de Newton a les caractéristiques suivantes :
\begin{enumerate}

	\item
		Pour résoudre le problème numéro $n$, il faut avoir résolu le problème numéro $n-1$.
	\item
		Aucune des solutions $x_n$ aux problèmes intermédiaires n'est une solution au problème de départ (à moins d'un coup de chance).
	\item
		Étant donné que la donnée du problème $F_n$ est la fonction $f$ de départ, nous avons $d_m=d_n=d$ pour tout $m$ et $n$.
\end{enumerate}

\begin{theorem}     \label{THOooMACHooLofCVu}
    Soit \( f\) continue sur un voisinage de \( \alpha\), racine simple. Alors il existe un voisinage de \( \alpha\) de rayon \( \sigma\) tel que pour tout \( x_0\) dans ce voisinage, la méthode converge vers \( \alpha\) avec ordre de convergence \( p=2\).
\end{theorem}

Donc dès qu'on a continuité autour de la solution recherchée, il suffit de prendre \( x_0\) assez proche pour que tout se passe bien. Cela se fait par localisation des racines, par exemples en traçant la fonction avec un bon niveau de zoom. Le fait est qu'on cherche disons \( 3\) décimales à la main (travail sur ordinateur et graphique) et Newton donne les \( 20\) décimales suivantes à la vitesse de la lumière.

%--------------------------------------------------------------------------------------------------------------------------- 
\subsection{Si multiplicité}
%---------------------------------------------------------------------------------------------------------------------------

Supposons que \( \alpha\) soit de multiplicité \( r\) (définition \ref{DEFooXSOQooAnWqKM}).

Cela se remarque en voyant que la méthode de Newton demande plutôt \( 20\) itérations que \( 5\). Le problème que cela pose est que chaque itération, les valutations provoquent des erreurs. Donc moins d'itérations, c'est mieux.

Nous pouvons modifier la formule avec
\begin{equation}
    x_{n+1}=x_n-r\frac{ f(x_n) }{ f'(x_n) }.
\end{equation}
Il est possible de prouver que cette suite est à nouveau quadratique.

Ou alors on pose \( F(x)=f^{(r-1)}(x)\) et \( \alpha\) est une racine simple pour \( F\). Donc faire Newton pour \( F\) est à nouveau quadratique, tout en donnant la même solution parce que \( F(\alpha)=0\) et \( F'(\alpha)\neq 0\).

La seconde façon est bien parce que le théorème de localisation fonctionne \ref{THOooMACHooLofCVu}

Et si \( r\) n'est pas connu ?

Il est toujours possible de faire \( r=2\) puis \( r=3\) et caetera jusqu'au moment où l'on remarque que le nombre d'itérations baisse un grand coup.

Mais ça demande beaucoup de calculs.  Le mieux est de changer de méthode.

%--------------------------------------------------------------------------------------------------------------------------- 
\subsection{Et la dérivée ?}
%---------------------------------------------------------------------------------------------------------------------------

Un des problèmes de la méthode de Newton est que l'on doit pouvoir calculer la dérivée. Typiquement, il faut savoir \( f\) de façon analytique. Si cela n'est pas possible, nous pouvons changer de méthode et utiliser la méthode des sécantes décrite en \ref{SECooIUEUooVcHAoc}.

%--------------------------------------------------------------------------------------------------------------------------- 
\subsection{Estimation de l'ordre de convergence}
%---------------------------------------------------------------------------------------------------------------------------

Comment estimer numériquement l'ordre \( p\) de convergence de la méthode ? Soit une suite \( (x_n)\) convergente vers \( \alpha\). Considérons les \( 4\) termes \( x_{n-3}\), \( x_{n-2}\), \( x_{n-1}\), \( x_n\). Alors nous pouvons écrire l'approximation
\begin{equation}
    \frac{ | x_n -x_{n-1}| }{ | x_{n-1}-x_{n-2} | }\simeq \left( \frac{ | x_{n-1}-x_{n-2} | }{ | x_{n-1}-x_{n-3} | } \right)^p.
\end{equation}
Cette approximation ne serait pas trop mauvaise tant que \( n\) est assez grand pour que la convergence soit bien engagée. Passons au logarithme :
\begin{equation}
    \ln \frac{ | x_n -x_{n-1}| }{ | x_{n-1}-x_{n-2} | }\simeq p\ln \left( \frac{ | x_{n-1}-x_{n-2} | }{ | x_{n-1}-x_{n-3} | } \right).
\end{equation}
et donc
\begin{equation}
    p\simeq \frac{ \ln\left( \frac{ | x_n -x_{n-1}| }{ | x_{n-1}-x_{n-2} } \right) }{ \ln \left(\frac{ | x_{n-1}-x_{n-2} | }{ | x_{n-1}-x_{n-3} | } \right)}.
\end{equation}
Avec cette approximation, en réalité nous calculons une suite \( (p_i)\) qui sont les approximation de \( p\) à partir des termes \( i\) à \(i+3 \) de la suite \( (x_n)\). Il s'agit d'une suite d'estimations de \( p\).

\begin{enumerate}
    \item
Dans le cas de la bisection, nous obtenons toujours \( p_i=1\).
\item
    Dans le cas de la méthode de Newton (\ref{SECooIKXNooACLljs}) nous avons \( p=2\). Mais les premières valeurs de \( p_i\) peuvent être aussi bien \( 0\) que \( 7\). Après quelque itérations pourtant les \( p_i\) se regroupent autour de \( 2\).
\end{enumerate}
En tout cas, le plus important est de savoir si \( p>1\) ou non. Rappel : nous voulons la superlinéarité parce que nous voulons utiliser le test d'arrêt de la différence entre deux termes. % position 3224-20669

%+++++++++++++++++++++++++++++++++++++++++++++++++++++++++++++++++++++++++++++++++++++++++++++++++++++++++++++++++++++++++++ 
\section{Autres méthodes}
%+++++++++++++++++++++++++++++++++++++++++++++++++++++++++++++++++++++++++++++++++++++++++++++++++++++++++++++++++++++++++++

%--------------------------------------------------------------------------------------------------------------------------- 
\subsection{Méthode de Schröder}
%---------------------------------------------------------------------------------------------------------------------------

La formule est
\begin{equation}
    x_{n+1}=x_n-\frac{ f(x_n)f'(x_n) }{ f'(x_n)^2-f(x_n)f''(x_n) }
\end{equation}
Cette méthode est d'ordre \( 2\) pour toute racine et toute valeur de multiplicité. Le problème de cette méthode est qu'elle demande \( 3\) valutations. Son efficacité :
\begin{equation}
    E=\sqrt[3]{ 2 }\simeq 1.25
\end{equation}
Cela est donc moins efficace que Newton.

%--------------------------------------------------------------------------------------------------------------------------- 
\subsection{Halley}
%---------------------------------------------------------------------------------------------------------------------------

Il a \( p=3\) lorsque \( \alpha\) est racine simple. Mais encore \( p=1\) pour les racines multiples. Plus efficace que Newton pour les racines simples, mais même problème pour les racines multiples.

\begin{equation}
    x_{n+1}=x_n-\frac{ 2f(x_n)f'(x_n) }{ 2f'(x_n)^2-f(x_n)f''(x_n) }
\end{equation}

%+++++++++++++++++++++++++++++++++++++++++++++++++++++++++++++++++++++++++++++++++++++++++++++++++++++++++++++++++++++++++++ 
\section{Méthode des sécantes variables}
%+++++++++++++++++++++++++++++++++++++++++++++++++++++++++++++++++++++++++++++++++++++++++++++++++++++++++++++++++++++++++++
\label{SECooIUEUooVcHAoc}

Supposons de ne pas avoir \( f\) analytique, mais seulement la possibilité de calculer \( f(x)\) pour tout \( x\). Newton ne fonctionne pas, mais la bisection fonctionne.

Nous pouvons approximer
\begin{equation}
    f'(x_n)=\frac{ f(x_n)-f(x_{n-1}) }{ x_n-x_{n-1} }.
\end{equation}
En substituant dans la formule de Newton, nous obtenons
\begin{equation}
    x_{n+1}=x_n-\frac{ f(x_n)(x_n-x_{n-1}) }{ f(x_n)-f(x_{n-1}) }.
\end{equation}

Il s'agit de prendre la droite qui passe par \( (x_{n-1},f(x_{n-1}))\) et par \( (x_n,f(x_n))\) et de prendre l'intersection de cette droite avec l'axe \( y=0\). Cela donne le \( x_{n+1}\).

Pour cette méthode, il ne faut pas seulement \( x_0\) mais également \( x_1\).

L'ordre de convergence est le nombre d'or
\begin{equation}    \label{EQooQEFCooUsGVjP}
    p=\frac{ 1-\sqrt{ 5 } }{ 2 }\simeq 1.618.
\end{equation}
Cela est donc superlinéaire.

La nombre de valutations est \( s=1\) (il y a deux apparitions de \( f\) dans la formule, mais l'une des deux est récupérée dans l'itération suivante). Donc l'efficacité est
\begin{equation}
    E=p.
\end{equation}
Donc bien efficace.

\begin{proposition}
    Si \( \alpha\) est racine simple, il existe un voisinage de \( \alpha\) tel que pour tout choix de \( x_0\), \( x_1\) dans ce voisinage, la méthode converge.
\end{proposition}

Psychologiquement, on est tenté de prendre \( x_0\) et \( x_1\) de part et d'autre de \( \alpha\) (pensant à la bisection), mais en réalité ce n'est pas obligatoire du tout et n'a aucune influence. Il faut seulement les prendre très proches de \( \alpha\).

\begin{remark}
    La méthode de la sécante est souvent écrite sous la forme
    \begin{equation}        \label{EQooYVKLooKTFjwv}
        x_{n+1}=\frac{ x_{n-1}f(x_n)-x_nf(x_{n-1}) }{ f(x_n)-f(x_{n-1}) }.
    \end{equation}
    C'est évidemment algébriquement équivalent. 

    Les formules \eqref{EQooQEFCooUsGVjP} et \eqref{EQooYVKLooKTFjwv} ont toutes deux des erreurs de cancellation. Laquelle est la plus grave ?

    Dans la première, si la fraction est mal calculée, elle ne fait que modifier \( x_n\). C'est à dire qu'on peut espérer qu'à la prochaine itération, ça aille mieux. En tout cas, dans ce cas si la fraction est mal calculée, ça ne détruit pas tout.

    Dans la seconde, c'est la valeur elle-même qui risque d'être mal calculée. Et si la fraction est mal calculée, alors on casse complètement l'éventuel bonne approximation que nous avions déjà.
\end{remark}

%--------------------------------------------------------------------------------------------------------------------------- 
\subsection{Aitken}
%---------------------------------------------------------------------------------------------------------------------------

La méthode du \( \Delta^2\) de Aitken est une méthode d'accélération de la convergence.

Soit \( (x_n)\) une suite qui converge. Nous voudrions une nouvelle suite \( (y_n)\) telle que
\begin{equation}
    \lim_{n\to \infty} \frac{ y_n-\alpha }{ x_n-\alpha }
\end{equation}
C'est la définition d'une convergence accélérée.

La façon de faire est :
\begin{equation}
    y_n=\frac{ x_{n+2}x_n-x_{n+1}^2 }{ x_{n+2}-2x_{n+1}+x_n }=x_n-\frac{ (x_{n+1}-x_n) }{ x_{n+2}-2x_{x+1}+x_n }.
\end{equation}
La première expressions a deux cancellations (la seconde une seule) et de plus la première est $y_n$ elle-même alors que la seconde est une correction.

Donc la seconde expression est numériquement meilleure.

L'opérateur \( \Delta\) appliqué à une suite est :
\begin{equation}
    (\Delta x)_n=x_{n+1}-x_n
\end{equation}
Donc
\begin{equation}
    (\Delta^2x)_n= (\Delta x)_{n+1}-(\Delta x)_n=x_{n+2}-x_{n+1}-x_{n+1}+x_n=x_{n+2}-2x_{n+1}+x_n.
\end{equation}
L'accélération a alors la formule
\begin{equation}
    y_n=\frac{ (\Delta x)_n^2 }{ (\Delta^2x)_n }.
\end{equation}

Le problème est que ça accélère tellement que l'on arrive vite à des erreurs de cancellations, et donc à une précision en pics oscillants.

%+++++++++++++++++++++++++++++++++++++++++++++++++++++++++++++++++++++++++++++++++++++++++++++++++++++++++++++++++++++++++++ 
\section{Équations algébrique}
%+++++++++++++++++++++++++++++++++++++++++++++++++++++++++++++++++++++++++++++++++++++++++++++++++++++++++++++++++++++++++++

C'est une équation du type \( P(x)=0\) où \( P\) est un polynôme. Soit un polynôme de degré \( n\). Nous en savons des choses.

\begin{enumerate}
    \item
        L'équation a exactement \( n\) solutions dans \( \eC\) en comptant les multiplicités.
    \item
        Les racines complexes arrivent par paire complexes conjuguée. Elles sont donc toujours en nombre pair.
\end{enumerate}

Si donc nous avons \( n=3\), nous ne pouvons pas avoir \( 2\) racine réelles. Il y en a donc \( 1\) ou \( 3\) réelles. Pas zéro ni deux.

Quelque méthodes : Müller, matrice compagnon, Laguerre.

%---------------------------------------------------------------------------------------------------------------------------
\subsection{Résoudre un système linéaire}
%---------------------------------------------------------------------------------------------------------------------------

Pour résoudre un système linéaire d'équations, nous échelonnons la matrice du système. Soit à résoudre le système $Ax=b$ où
\begin{equation}
	\begin{aligned}[]
		A&=\begin{pmatrix}
			2   &   4   &   -6  \\
			1   &   5   &   3   \\
			1   &   3   &   2
		\end{pmatrix}, &\text{et}&&b=\begin{pmatrix}
			-4  \\
			10  \\
			5
		\end{pmatrix}.
	\end{aligned}
\end{equation}
En termes de problèmes, on écrit $F\big( x,(A,b) \big)=Ax-b$. La donnée de ce problème est le couple $(A,b)$.

En ce qui concerne l'algorithme, on pose comme premier problème
\begin{equation}
	F_1\big(x_1,(A_1,b_1)\big)=A_1x_1-b_1=0
\end{equation}
avec $A_1=A$ et $b_1=b$.

Ensuite, on commence à échelonner et le second problème est
\begin{equation}
	F_2\big(x_2,(A_2,b_2)\big)=A_2x_2-b_2=0
\end{equation}
avec
\begin{equation}
	\begin{aligned}[]
		A&=\begin{pmatrix}
			2   &   4   &   -6  \\
			0   &   3   &   6   \\
			0   &   1   &   5
		\end{pmatrix}, &\text{et}&&b=\begin{pmatrix}
			-4  \\
			12  \\
			13
		\end{pmatrix}.
	\end{aligned}
\end{equation}
Le troisième problème sera
\begin{equation}
	F_3\big(x_3,(A_3,b_3)\big)=A_3x_3-b_3=0
\end{equation}
avec
\begin{equation}
	\begin{aligned}[]
		A&=\begin{pmatrix}
			2   &   4   &   -6  \\
			0   &   3   &   6   \\
			0   &   0   &   3
		\end{pmatrix}, &\text{et}&&b=\begin{pmatrix}
			-4  \\
			12  \\
			3
		\end{pmatrix}.
	\end{aligned}
\end{equation}
Ce problème est facile à résoudre «à la main». Nous nous arrêtons donc ici avec l'algorithme, et nous trouvons le $x_3$ qui résous le problème $F_3$.

%--------------------------------------------------------------------------------------------------------------------------- 
\subsection{Caractéristiques}
%---------------------------------------------------------------------------------------------------------------------------

L'algorithme de résolution de systèmes linéaires d'équations a les propriétés suivantes, à mettre en contraste avec celles de Newton :
\begin{enumerate}

	\item
		Pour résoudre le problème numéro $n$, il n'a pas fallu résoudre le problème numéro $n-1$.
	\item
		Toutes les solutions $x_n$ des problèmes intermédiaires sont solutions du problème de départ. Nous avons $F_n(x,d_n)=0$ pour tout $n$ (ici, $d_n=(A_n,b_n)$).
	\item
		D'un problème à l'autre, les données changent énormément : la matrice échelonnée peut être très différente de la matrice de départ.

\end{enumerate}

%---------------------------------------------------------------------------------------------------------------------------
\subsection{Définitions}
%---------------------------------------------------------------------------------------------------------------------------

	Nous allons maintenant formaliser en donnant quelques définitions pour nommer les propriétés que nous avons vues. D'abord, un algorithme est une suite de problèmes. Un \defe{algorithme}{algorithme} pour résoudre un problème $F(x,d)=0$ est une suite de problèmes $\{F_n(x_n,d_n)=0\}_{n\in\eN}$.

\begin{definition}
	Un tel algorithme est dit  \defe{fortement consistant}{algorithme!fortement consistant} si pour toutes données admissibles $d_n$, on a
	\begin{equation}
		F_n(x,d_n)=0\quad\forall \;n,
	\end{equation}
	où $x$ est la solution de $F(x,d)=0$.
\end{definition}
L'algorithme des matrices est fortement consistant, mais pas l'algorithme de Newton.

\begin{definition}
	Un algorithme est \defe{consistant}{algorithme!consistant} si $\lim_{n\to\infty}F_n(x,d_n)=0$.
\end{definition}
Dans le cas de l'algorithme de Newton, c'est plutôt une telle consistance qu'on attend.

L'algorithme est dit \defe{stable}{algorithme!stable} si pour tout $n$ le problème correspondant est stable.  Dans ce cas, on note $K^{\mbox{num}}$ le  \defe{conditionnement relatif asymptotique}{conditionnement!relatif asymptotique} défini par
\begin{equation}
	K^{\mbox{num}}=\limsup_nK_n
\end{equation}
où $K_n$ est le conditionnement relatif du problème $F_n(x_n,d_n)=0$.

\begin{definition}      \label{DefAlgoConverge}
	Un algorithme est dit \defe{convergent}{algorithme!convergent} (en $d$) si pour tout $\epsilon>0$, il existe $N=N(\epsilon)$ et $\delta=\delta(N,\epsilon)$ tels que pour $n\geq0$ et $|d-d_n|<\delta$, on ait $|x(d)-x_n(d_n)|<\epsilon$.
\end{definition}

\begin{remark}      \label{RemConvAlgoNewton}
Dans le cas de l'algorithme de Newton, nous avons vu que la donnée $d_n$ du problème $F_n$ était en fait la même que la donnée initiale $d$, donc nous avons $d_n=d$, et par conséquent nous avons toujours $| d-d_n |<\delta$. Dans ce cas, la définition de la convergence revient à demander que la suite numérique des $x_n$ converge vers la solution $x$.
\end{remark}

\begin{remark}
Dans le cas des matrices par contre, les données sont très différentes les unes des autres, nous avons donc en général que $| d-d_n |>\delta$. Mais en revanche nous savons que tous les problèmes intermédiaires $F_n$ acceptent une solution unique\footnote{Nous n'envisageons que le cas où le déterminant est non nul.} $x_n(d_n)=x(d)$. Par conséquent, $| x_n(d_n)-x(d) |$ est toujours plus petit que $\epsilon$. L'algorithme des matrice est donc toujours un algorithme convergent.
\end{remark}

%+++++++++++++++++++++++++++++++++++++++++++++++++++++++++++++++++++++++++++++++++++++++++++++++++++++++++++++++++++++++++++
\section{Représentations numériques, erreurs}
%+++++++++++++++++++++++++++++++++++++++++++++++++++++++++++++++++++++++++++++++++++++++++++++++++++++++++++++++++++++++++++

\begin{definition}
	Soit $x$ un réel. On définit sa \defe{représentation en virgule fixe}{représentation!virgule fixe} par
	\begin{equation}
		x=\{[x_nx_{n-1}...x_0,x_{-1}...x_{-m}], b, s\}
	\end{equation}
	avec  $b\in\eN, b\geq2$, $s\in\{0,1\}$ et $x_j\in\eN,x_j<b$ suivant la formule
	\begin{equation}
		x=(-1)^{s}\sum_{j=-m}^nx_j.b^j.
	\end{equation}
	On définit sa \defe{représentation en \href{https://docs.python.org/tutorial/floatingpoint.html}{virgule flottante} normalisée}{Représentation!virgule flottante normalisée} par
	\begin{equation}
		\{[a_1...a_t],b,e,s\}
	\end{equation}
	où $e\in\eZ,e\in[L,U]$ et $a_j\in\eN;0\leq a_j<b; a_1\geq1$ suivant la formule
	\begin{equation}        \label{EqRepreFlotNOrm}
		x=(-1)^sb^e\sum_{j=1}^ta_jb^{-j}.
	\end{equation}
\end{definition}

\begin{definition}
	L'\defe{erreur relative}{erreur relative} commise en remplaçant un nombre réel $x$ par une valeur approchée $\hat{x}$ est définie par
	\begin{equation}
		\epsilon_x:=\left|\frac{x-\hat{x}}{x}\right|.
	\end{equation}
\end{definition}

L'erreur relative n'est pas influencée par l'ordre de grandeur de \( x\). En effet, l'ordre de grandeur de \( \hat x\) est certainement la même que celle de \( x\), dans la majorité des cas sans problèmes. Du coup si \( x'=200x\) alors \( \hat{x'}\simeq 200\hat{x}\) et le \( 200\) se simplifie.

Le nombre de chiffres significatifs correct dans l'approximation est donné par \( -\log_{10}(\epsilon_x)\). La partie entière de ce nombre est le nombre de chiffres tout à fait exacts et la partie décimale donne une idée sur le fait que le chiffre suivant est plus ou moins bien.

%+++++++++++++++++++++++++++++++++++++++++++++++++++++++++++++++++++++++++++++++++++++++++++++++++++++++++++++++++++++++++++
\section{Problèmes pour écrire des nombres}
%+++++++++++++++++++++++++++++++++++++++++++++++++++++++++++++++++++++++++++++++++++++++++++++++++++++++++++++++++++++++++++

\begin{remark}
	Si nous voulons donner \( x\in \eR\) à un ordinateur, nous sommes soumis à deux erreurs :
	\begin{enumerate}
		\item
			D'abord, vu que nous ne pouvons pas taper sur le clavier toutes les décimales de \( x\), nous faisons une \defe{erreur de troncature}{erreur!troncature}.
		\item
			L'ordinateur devant convertir cela en base deux, il commet une seconde erreur, dite \defe{erreur d'assignation}{erreur!assignation}.
	\end{enumerate}
\end{remark}

%--------------------------------------------------------------------------------------------------------------------------- 
\subsection{Troncature : la base}
%---------------------------------------------------------------------------------------------------------------------------

Supposons que nous voulions écrire le nombre (écrit ici en base \( 10\))
\begin{equation}
	0.4567894251
\end{equation}
de façon plus facile à lire, on peut demander de ne laisser que \( t\) chiffres significatifs. Disons \( t=3\).

\begin{description}
	\item[Technique de troncature] On garde \( 3\) chiffres significatifs : \( 0.456\). Facile.
	\item[Technique d'arrondi] Vu que le premier qu'on supprime est un \( 7\), le dernier qu'on garde est majoré de \( 1\) : on écrit \( 0.457\).
\end{description}

Que faire si le premier chiffre rejeté est un \( 5\) ? En première approximation, nous pouvons prendre la règle suivante : si le premier chiffre rejeté est un \( 5\), il faut augmenter de \( 1\) de dernier chiffre gardé parce qu'il y a presque certainement encore un chiffre non nul derrière.

\begin{remark}
	Les ordinateurs travaillent tous en mode d'arrondi.
\end{remark}

\begin{example}
    Si on doit entrer le nombre \( 0.38358546\) dans un ordinateur qui ne garde que \( 3\) chiffres significatifs, il faut taper \( 0.384\) au clavier (erreur classique dans les exercices).
\end{example}

%--------------------------------------------------------------------------------------------------------------------------- 
\subsection{Troncature : le drift}
%---------------------------------------------------------------------------------------------------------------------------

Soit une machine ne pouvant retenir que \( 3\) chiffres significatifs et effectuant les arrondis vers le haut lorsque le chiffre à éliminer est un \( 5\). Nous notons \( \oplus\) et \( \ominus\) les opérations d'addition et soustraction avec arrondis\cite{ooAGVZooTIcZZb}. Les égalités comprenant plus de trois chiffres significatifs sont des égalités au sens de la machine. Nous écrirons donc sans états d'âme :
\begin{equation}
    1\oplus0.555=1.555=1.56.
\end{equation}

Considérons la suite numérique
\begin{subequations}
    \begin{numcases}{}
        x_0=1.00\\
        x_n=(x_{n-1}\ominus y)\oplus y
    \end{numcases}
\end{subequations}
avec \( y=-0.555\).

Nous avons
\begin{equation}
    x_1=(1\oplus 0.555)\ominus 0.555=1.56\ominus 0.555=1.005=1.01
\end{equation}
et ensuite
\begin{equation}
    x_2=(1.01\oplus 0.555)\ominus 0.555=1.565\ominus 0.555=1.57\ominus 0.555=1.015=1.02.
\end{equation}
Et ainsi de suite. La suite est donc croissante alors que la définition nous donnerait envie d'avoir \( x_n=x_0\) pour tout \( n\).

\begin{remark}
    En réalité, cette suite se stabilise à \( x_n=10\) pour tout \( n\) à partir de \( n=845\). En effet,
    \begin{equation}
        (10\oplus 0.555)\ominus 0.555=10.555\ominus 0.555=10.6\ominus 0.555=10.045=10.
    \end{equation}
    Le fait est qu'à ce moment, l'erreur de troncature est assez loin dans les décimales pour que le premier chiffre négligé soit un ``0'' au lieu d'un ``5''.
    
    Notons toutefois que cette stabilité n'est pas là pour nous rassurer parce qu'elle n'en est pas moins complètement fausse.
\end{remark}

La règle de troncature adoptée dans Sage est d'arrondir au nombre pair le plus proche lorsque le premier nombre à négliger est un \( 5\). Donc \( 12.5\) s'arrondit à \( 12\) plutôt que \( 13\).

\begin{example}
	Soient les expressions (algébriquement égales) :
	\begin{enumerate}
		\item
			\(A= x(x+1)\)
		\item
			\(B= x^2+x\)
	\end{enumerate}
	Nous savons que
	\begin{equation}
		x=\fl(x)=10^{-30}
	\end{equation}
	et
	\begin{equation}
		1=\fl(1)
	\end{equation}
	parce que pour \( 1\) et \( 10^{-30}\), il n'y a pas d'erreurs d'assignation.

	En précision simple, \( 10^{-30}+1=1\) parce qu'en précision simple, il n'y a que \( 7\) ou \( 8\) chiffres significatifs\footnote{Erreur de « relation normale».}.

	Nous avons $A=10^{-30}$, mais \( x^2\) donne un \info{underflow} parce que \( 10^{-60}\) ne peut pas être représenté en précision simple. En pratique, beaucoup de logiciels en font \( 0\). Dans ce cas, en réalité \( B\) donne effectivement \( 10^{-30}\) après avoir fait \( x^2+x=0+x=10^{-30}\).
\end{example}

%---------------------------------------------------------------------------------------------------------------------------
\subsection{Quelque bonnes règles}
%---------------------------------------------------------------------------------------------------------------------------

\begin{enumerate}
	\item
		Si on a plusieurs nombres à additionner ou soustraire, il vaut mieux commencer par sommer ou soustraire ceux dont on sait qu'ils ont le même ordre de grandeur. Il n'y a donc pas tout à fait «associativité» des erreurs.
	\item
		Les opérations délicates sont l'addition et la soustraction. La multiplication et la division sont sans dangers, à part l'erreur de dépassement du maximum. Dans une multiplication, on perd au pire quelque chiffres significatifs, mais certainement les derniers, pas les premiers.
\end{enumerate}

%+++++++++++++++++++++++++++++++++++++++++++++++++++++++++++++++++++++++++++++++++++++++++++++++++++++++++++++++++++++++++++
\section{Erreur de ``cancellation''}
%+++++++++++++++++++++++++++++++++++++++++++++++++++++++++++++++++++++++++++++++++++++++++++++++++++++++++++++++++++++++++++

Lorsque deux nombres sont de même ordre de grandeur, avec plusieurs nombres significatifs identiques. La cancellation est le fait que, suite à la soustraction, tous les chiffres significatifs ou presque se sont simplifiés et qu'il ne reste plus que des chiffres non significatifs. 

\begin{example}[\cite{ooIZQWooYJmQmW}]
    Sur une machine ne gardant que \( 4\) chiffres significatifs, faire
    \begin{equation}
        0.5678\times 10^6-0.5677\times 10^6 = 0.0001\times 10^6=0.1000\times 10^3.
    \end{equation}
    Le fait est que les trois derniers zéros ne sont pas significatifs, mais maintenant la machine nous fait croire qu'ils le sont.

    Une autre façon de voir ce problème est d'imaginer qu'il faille faire la différence entre \( 0.5678\,289798\times 10^6\) et \( 0.5677\,3136907\) sur cette machine. Certes la machine nous autorise à avoir \( 4\) chiffres significatifs, donc au moment d'entrer les nombres nous perdons un beau paquet de chiffres. Mais au moment de faire la différence, nous perdons (presque) tout le reste. Donc là où nous pouvions espérer avoir \( 4\) chiffres significatifs de la différence, nous n'en avons que \( 1\). Les trois derniers zéros de la réponse (\( 0.1000\times 10^3\)) sont faux.
\end{example}

\begin{example}
	Soit à résoudre l'équation \( ax^2+bx+c=0\) avec \( a,b,c\neq 0\) et \( b^2-4ac>0\). Solution :
	\begin{equation}
		x_{1,2}=\frac{ -b\pm\sqrt{b^2-4ac} }{ 2a }.
	\end{equation}

	Supposons que \( | 4ac |\ll b^2\) avec tout de même pas tellement petit qu'on se perd dans la précision. Bref, on suppose que seules quelque dernières décimales de \( b^2-4ac\) sont différentes de zéro.

	On a :
	\begin{subequations}
		\begin{align}
			\sqrt{b^2-4ac}&=\sqrt{\tilde b}= | \tilde b | \\
			x_1&=\frac{ -b-\sqrt{b^2-4ac} }{ 2a }\\
			x_2&=\frac{ -b+\sqrt{b^2-4ac} }{ 2a }
		\end{align}
	\end{subequations}
	Si \( b>0\), nous avons une erreur de cancellation dans \( x_2\) parce qu'on fait la différence entre deux nombres presque égaux. Donc \( x_2\) mal calculé. Par contre \( x_1\) est bien calculé.


	Si par contre \( b<0\), c'est le contraire.


	Avec \( a=10^{-3}\), \( b=0.8\), \( c=-1.2\times 10^{-5}\). À la main nous obtenons : \( x_1=-800\), \( x_2=1.5\times 10^{-5}\), et un ordinateur se tromperait \ldots


\lstinputlisting{sageSnip001.sage}

	Donc Sage ne tombe pas dans le piège.
\end{example}

Comment résoudre ce problème ? Ou, autre façon de poser la question : comment Sage a fait pour résoudre le problème ?

Utilisons les relations coefficients-racines :
\begin{subequations}
	\begin{align}
		x_1+x_2&=-b/a\\
		x_1x_2&=c/a
	\end{align}
\end{subequations}
La première lie les deux racines par des opérations de addition et soustractions, et donc n'est pas intéressantes. La seconde est bien. Si nous connaissons \( x_1\), nous calculons
\begin{equation}
	x_2=\frac{ c }{ ax_1 }.
\end{equation}

Quitte à redéfinir \( x_1\) et \( x_2\), la solution bien calculée est :
\begin{equation}
	x_1=\frac{ -b-\signe(b)\sqrt{b^2-4ac} }{ 2a }.
\end{equation}

\begin{example}
	Nous considérons :
	\begin{equation}
		f(x)=cos(x+\delta)-\cos(x).
	\end{equation}
	Cela a une erreur de cancellation lorsque \( | \delta |\ll | x |\). On élimine l'erreur de cancellation par
	\begin{equation}
		f(x)=-2\sin(\delta/2)\sin\left( x+\frac{ \delta }{ 2 } \right).
	\end{equation}

	\begin{probleme}
		Pourquoi la condition pour avoir l'erreur est \( \delta\ll x\) et non simplement \( \delta\ll 1\) ?
	\end{probleme}

\end{example}

\begin{example}
	Pour
	\begin{equation}
		f(x)=\sqrt{x+\delta}-\sqrt{x}.
	\end{equation}
	On fait la coup du binôme conjugué :
	\begin{equation}
		f(x)=\frac{ \delta }{ \sqrt{x+\delta}+\sqrt{x} }.
	\end{equation}
	Plus d'erreur de cancellation, vu qu'au dénominateur nous avons une somme de deux positifs.
\end{example}

Les erreurs de cancellation ne se résolvent pas en augmentant la précision des nombres donnés.

%--------------------------------------------------------------------------------------------------------------------------- 
\subsection{Erreur d'absorption}
%---------------------------------------------------------------------------------------------------------------------------

L'addition d'un nombre avec un nombre très différent peut faire perdre de l'information sur le plus petit. Par exemple avec \( 4\) chiffres significatifs,
\begin{equation}
    0.5678\oplus 0.0001237=0.5679
\end{equation}
où nous avons perdu presque toute l'information du petit nombre.

Une situation particulièrement ennuyeuse est celle où justement c'est le petit nombre qui nous intéresse parce que le grand est censé se simplifier : 
\begin{equation}
    (0.0001327\oplus 0.5678)\ominus 0.5678=0.5679\ominus 0.5678=0.0001
\end{equation}
qui ne possède qu'un seul chiffre significatif correct alors que voyant le calcul, la réponse aurait pu être trouvée.

Moralité : si certains manipulations algébrique peuvent faire apparaître des simplifications avant de passer le calcul à la machine, il est bon de les effectuer.

%--------------------------------------------------------------------------------------------------------------------------- 
\subsection{Calcul d'une dérivée}
%---------------------------------------------------------------------------------------------------------------------------

Pour calculer la dérivée de \( f\) en \( a\), il est loisible d'utiliser la formule
\begin{equation}
    f'(a)=\lim_{h\to 0} \frac{ f(a+h)-f(a) }{ h }.
\end{equation}
Le numérateur est alors sujet à une erreur d'absorption dans le calcul de \( a+h\) et ensuite une erreur de cancellation dans le calcul de la différence.

En utilisant la formule
\begin{equation}
    f'(a)=\lim_{h\to 0} \frac{ f(a+h)-f(a-h) }{ 2h }
\end{equation}
nous pouvons espérer avoir une erreur de cancellation plus petite.

%+++++++++++++++++++++++++++++++++++++++++++++++++++++++++++++++++++++++++++++++++++++++++++++++++++++++++++++++++++++++++++
\section{Équations non linéaire}
%+++++++++++++++++++++++++++++++++++++++++++++++++++++++++++++++++++++++++++++++++++++++++++++++++++++++++++++++++++++++++++

Certains équations non linéaires sont résoluble explicitement, par exemples les polynômes de degré jusqu'à \( 4\) ou des choses comme
\begin{equation}
	\sin^2(x)+3\sin(x)+5=0.
\end{equation}
Mais ces exemples sont très rares.

Nous allons étudier des équations du type \( f(x)=0\), dans \( \eR\).

\begin{enumerate}
	\item
Un problème écrit sous la forme \( x=g(x)\) peut utiliser des théorèmes de points fixes.
\item
	Un problème sous la forme \( f(x)=0\) peut utiliser des méthodes de bisection, Newton ou autres.
\end{enumerate}
Il y a évidemment beaucoup de façons de transformer un problème pour passer d'une forme à l'autre.

\begin{example}
	Soit \( f(x)=x^2-a=0\) avec \( a>0\). Nous pouvons l'écrire
	\begin{equation}
		x^2+x-a=x
	\end{equation}
	qui donne une forme \( g(x)=x\) pour \( g(x)=x^2+x-a\).

	Ou encore \( x=\frac{ a }{ x }\) et donc \( g(x)=a/x\) (si par ailleurs on sait que \( x\neq 0\)). Notons que \( x\neq 0\) n'est pas une hypothèse très forte parce qu'on la vérifie directement sur \( a\).
\end{example}

\begin{example}
	Soit l'équation à résoudre
	\begin{equation}
		f(x)=x^2-2-\ln(x)=0
	\end{equation}
	Les solutions de cette équations peuvent être vues comme les intersections avec l'axe \( X\) du graphe \( y=x^2-2-\ln(x)\). Tracer peut donc aider. Par ailleurs, il faut noter que
	\begin{equation}
		\lim_{x\to \pm\infty} f(x)=\infty,
	\end{equation}
	donc les solutions sont certainement contenues dans un compact de \( \eR\).

	À part tracer nous pouvons écrire
	\begin{equation}
		x^2-2=\ln(x).
	\end{equation}
	Et là, ce sont deux fonctions dont nous pouvons tracer le graphe pour trouver graphiquement les points d'intersection. Une étude de fonction montre vite qu'il y a exactement deux solutions, qu'elles sont strictement positives. Pour trouver des bornes, il faut calculer par exemple pour \( x=2\) les valeurs de \( \ln(x)\) et \( x^2-2\) pour voir si le graphe de \( x^2-2\) est déjà plus haut.
\end{example}

La majorité des méthodes numériques de résolution d'équation du type \( f(x)=0\) ou \( x=g(x)\) seront sous la forme de suites. Avec questions à la clefs :
\begin{enumerate}
	\item
		Quel point de départ choisir ?
	\item
		Convergence ?
	\item
		Est-ce que la limite est bien une solution ?
	\item
		Vu que la limite est unique, comment faire si l'équation a plusieurs solutions ? (souvent c'est le choix du point initial qui va jouer sur ce point)
\end{enumerate}

\begin{normaltext}
	Si la fonction est très plate, il est possible d'avoir
	\begin{equation}
		| f(\tilde \alpha) |\leq \epsilon
	\end{equation}
	sans que \( \tilde \alpha\) ne soit une bonne approximation.

	Lorsqu'on fait tourner une méthode itérative résolvant \( f(x)=0\), il n'est pas suffisant de s'arrêter lorsque
	\begin{equation}
		f(x_n)\leq \epsilon_1.
	\end{equation}
	Il faut aussi s'assurer que, si \( \bar x\) est la solution exacte, \( | x_n-\bar x |\leq \epsilon_2\). Ici \( \epsilon_1\) et \( \epsilon_2\) sont deux «précisions» que nous nous fixons au départ.

	Évidemment, vérifier la condition \( | x_n-\bar x |\leq \epsilon_2\), il faudrait savoir \( \bar x\). Et savoir \( \bar x\) c'est justement le problème. Nous sommes donc amenés à faire des estimation de \( | x_n-\bar x |\).
\end{normaltext}

\begin{normaltext}
    Lorsque nous effectuons une méthode itérative, il faut donc contrôler deux grandeurs :
    \begin{subequations}
        \begin{align}
            | \bar x-x_n |\leq \epsilon_1\\
            | x_{n+1}-x_n |\leq \epsilon_2.
        \end{align}
    \end{subequations}
\end{normaltext}

\begin{proposition}
Soit \( p\) l'ordre de convergence de la suite \( (x_n)\) vers \( \bar x\). Si \( p>1\) et \( | x_{n+1}-x_n |\leq \epsilon_2\) alors \( | \bar x-x_n |\leq \epsilon_2\).
\end{proposition}

%---------------------------------------------------------------------------------------------------------------------------
\subsection{Méthode de bisection}
%---------------------------------------------------------------------------------------------------------------------------

Il y a ce théorème des valeurs intermédiaires.
\begin{theorem}
	Soit \( f\) continue sur \( \mathopen[ a , b \mathclose]\) telle que \( f(a)f(b)<0\). Alors il existe au moins une solution à l'équation \( f(x)=0\) sur l'intervalle \( \mathopen] a , b \mathclose[\).
\end{theorem}

Pour démarrer une bisection, il est toujours bon de prendre l'intervalle \( \mathopen[ a , b \mathclose]\) de façon à ne contenir qu'une seule solution.

Soit donc un premier intervalle \( \mathopen[ a_0 , b_O \mathclose]\) tel que \( f(a_0)f(b_0)<0\) et ne contenant qu'une seule solution. À chaque itération nous considérons la moitié de l'intervalle précédent, mais la moitié contenant la solution.

Le test d'arrêt de la méthode de bisection se base uniquement sur la taille de l'intervalle qui reste. En effet si nous avons
\begin{equation}
	| b_n-a_n |\leq \epsilon
\end{equation}
nous avons certainement
\begin{equation}
	| \bar x-x_n |\leq \frac{ \epsilon }{2}
\end{equation}
où \( x_n\) est le point du milieu de \( \mathopen[ a_n , b_n \mathclose]\).

\begin{normaltext}
    La fonction \( f\) n'intervient dans la méthode que via son signe, pas via ses valeurs exactes.
\end{normaltext}

\begin{normaltext}
    Notons que le théorème des valeurs intermédiaires n'est pas très puissant pour choisir l'intervalle de départ; penser à la fonction
    \begin{equation}
        f(x)=x^2-5
    \end{equation}
    sur l'intervalle \( \mathopen[ -10 , 10 \mathclose]\). Il y a bien deux solutions dans l'intervalle, mais elles sont invisibles du théorème des valeurs intermédiaires. La fonction \( x\mapsto x^2\) a sa solution en \( x=0\), mais elle aussi n'est pas visible.
\end{normaltext}

\begin{normaltext}
    Certes la méthodes de bisection assure la convergence vers une solution, mais elle n'assure pas la convergence monotone. Il peut arriver que \( | \bar x-x_n |<| \bar x-x_{n+1} |\). C'est le cas lorsque la solution est très proche du milieu de l'intervalle choisit. Le \( x_0\) est alors proche de \( \bar x\) alors que \( x_1\) sera à une distance de \( \bar x\) d'environ un quart de l'intervalle de départ.
\end{normaltext}

Supposons déjà avoir trouvé un intervalle \( \mathopen[ a , b \mathclose]\) dans lequel se trouve une unique solution à \( f(x)=0\). Voici un algorithme possible.

\lstinputlisting{codeSnip_1.py}

Plusieurs remarques :
\begin{enumerate}
    \item
Le fait de retourner le nombre d'itérations effectuées permet à l'utilisateur de savoir la précision et si le nombre maximum d'itérations est dépassé. Si ce \info{n} retourné est égal à \info{nmax}, l'utilisateur sait que le \info{x} retourné n'est pas fiable.
\item
    La ligne \info{from \_\_future\_\_ import division} fait en sorte que l'opération \info{/} est bien la division usuelle. Sinon, le défaut en python 2 est que \info{/} soit la division \emph{entière}, c'est à dire que \( 1/2=0\) en python 2.o
    En python 3, le symbole \info{/} désigne bien la division usuelle, mais Sage utilise Python 2.
\item
    Même si l'intervalle \( \mathopen[ a , b \mathclose]\) contient plus d'une solution, la méthode fonctionne et donne une solution. Il est simplement éventuellement très compliqué de savoir laquelle.
\item
    Nous faisons \info{amp=toll+1} parce que nous voulons absolument lancer le cycle au moins une fois. Sinon, le \info{x} à retourner ne serait pas définit au moment de sortir du cycle (si le cycle n'est pas exécuté).
\item
    Calculer le point milieu d'un intervalle \( \mathopen[ a , b \mathclose]\) est par la formule \( (a+b)/2\) sauf que cette opération est numériquement dangereuse parce qu'à cause de l'arithmétique en précision finie, il est possible que cela tombe \emph{exactement} sur \( a\) ou \( b\). D'où le fait de calculer le point milieu par
    \begin{equation}
        x=a+\frac{ amp }{2}.
    \end{equation}
\item
    Dans le cas \info{Problème ZERO} nous déduisons \( f(x)=0\). Attention que c'est pas que \( f(x)=0\) mais simplement que en mettant \( x\) dans \( f\), la \emph{machine} retourne son zéro.

    Il peut cependant avoir une fonction telle que \( f(1)=10^{-50}\) et \( f(2)=0\). L'algorithme de bisection risque de s'arrêter si \( x_n=1\). Parce que la machine risque de calculer \( f(x_n)=0\).

    Quoi qu'il en soit, nous y mettons \info{amp=0} pour être sûr de sortir de la boucle dès la prochaine vérification.
\item
    Il y a moyen de sauver les valeurs de \( f(a)\) et \( f(x)\) pour ne pas les recalculer, et en particulier au moment de faire \info{b=x} nous pouvons poser \(\info{fa=fx}\).
\end{enumerate}

Si \( \tau\) est la précision de la solution voulue, nous pouvons fixer a priori le nombre d'itérations à faire grâce à la formule
\begin{equation}
    n\geq\left\lceil  \log_2\big( \frac{ b-a }{ \tau } \big)  \right\rceil.
\end{equation}
Il y a un ``\( \geq\)'' et non une égalité parce qu'en arithmétique numérique, le nombre obtenu à droite pourrait ne pas être le bon à \( 1\) près.

Ici pour \( \nu\in \eR\) le nombre \( \lceil\nu\rceil\) est le plus petit entier à être plus grand ou égal à \( \nu\).

\begin{normaltext}
    Notons l'importance de la continuité de \( f\). Par exemple que ferait la bisection sur la fonction \( f(x)=1/x\) pour l'intervalle $\mathopen[ -3 , 1 \mathclose]$ ? 

    Il y a changement de signe sans avoir de racine.
\end{normaltext} 

Vu que \( 2^{10}\) est déjà \( 1024\). Donc si on veut de la précision de l'ordre de \( 1/1000\), dix itération suffisent. Si donc nous avons besoin de \( 200\) itérations pour atteindre la précision voulue, c'est l'occasion de trouver un intervalle plus petit. Par exemple en traçant la fonction, en faisant un zoom et en trouvant des valeurs de \( a\) et \( b\) qui sont déjà proches.


\begin{normaltext}
    Dans le monde réel, il arrive souvent d'utiliser une méthode de bisection pour se donner un point de départ pour une autre méthode.
\end{normaltext}

%+++++++++++++++++++++++++++++++++++++++++++++++++++++++++++++++++++++++++++++++++++++++++++++++++++++++++++++++++++++++++++ 
\section{Efficacité}
%+++++++++++++++++++++++++++++++++++++++++++++++++++++++++++++++++++++++++++++++++++++++++++++++++++++++++++++++++++++++++++

\begin{definition}
    L'\defe{efficacité}{efficacité!d'une méthode itérative} est le nombre
    \begin{equation}
        E=\sqrt[s]{ p }
    \end{equation}
    où \( p\) est l'ordre de convergence de la méthode et \( s\) est le nombre de fois qu'il faut calculer une valeur de la fonction à chaque itération (nous ne comptons pas l'initialisation).
\end{definition}
Que le nombre de valutations intervienne est logique parce que chaque valutation provoque une erreur possible.

\begin{example}[Bisection]
    Pour la méthode de bisection, nous avons \( s=1\) parce que chercher \( x_{n+1}\), il faut seulement calculer \( f(x_n)\).
\end{example}

\begin{example}[Newton]
    Pour l'algorithme de Newton nous avons \( p=2\) et il y a deux valutations à chaque itérations (une fois \( f\) et une fois \( f'\)), donc \( s=2\) et \( E=\sqrt{ 2 }\).
\end{example}
