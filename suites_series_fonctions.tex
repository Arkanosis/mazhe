% This is part of Mes notes de mathématique
% Copyright (c) 2011-2012
%   Laurent Claessens
% See the file fdl-1.3.txt for copying conditions.


%+++++++++++++++++++++++++++++++++++++++++++++++++++++++++++++++++++++++++++++++++++++++++++++++++++++++++++++++++++++++++++
\section{Suites et séries de fonctions}
%+++++++++++++++++++++++++++++++++++++++++++++++++++++++++++++++++++++++++++++++++++++++++++++++++++++++++++++++++++++++++++
Source : \cite{TrenchRealAnalisys}.

%---------------------------------------------------------------------------------------------------------------------------
\subsection{Suites de fonctions}
%---------------------------------------------------------------------------------------------------------------------------

\begin{definition}
    Nous disons qu'une suite de fonctions \( (f_n)\) définies sur un ensemble \( A\) \defe{converge uniformément}{convergence!uniforme} vers une fonction \( f\) si
    \begin{equation}
        \lim_{n\to \infty} \| f_n-f \|_A=0
    \end{equation}
    où \( \| g \|_A=\sup_{x\in A}\| g(x) \|\).
\end{definition}

\begin{theorem}			\label{ThoUnigCvCont}
    Soit \( A\), un ensemble mesuré et \( f_n\colon A\to \eR^n\), une suite de fonctions continues convergeant uniformément vers \( f\). Si les fonctions \( f_n\) sont toutes continues en \( x_0\in A\), alors \( f\) est continue en \( x_0\).
\end{theorem}

\begin{proof}
    Soit \( \epsilon>0\). Si \( x\in A\) nous avons, pour tout \( n\), la majoration
    \begin{subequations}
        \begin{align}
            \| f(x)-f(x_0) \|&\leq \| f(x)-f_n(x) \|+\| f_n(x)-f_n(x_0) \|+\| f_n(x_0)-f(x_0) \|\\
            &\leq\| f_n(x)-f_n(x_0) \|+2\| f_n-f \|_{\infty}.
        \end{align}
    \end{subequations}
    Grâce à l'uniforme convergence, nous considérons \(N\in \eN\) tel que \( \| f_n-f \|\leq \epsilon\) pour tout \( n\geq N\). Pour de tels \( n\), nous avons
    \begin{equation}
        \| f(x)-f(x_0) \|\leq 2\epsilon\| f_n-f \|+\| f_n(x)-f_n(x_0) \|.
    \end{equation}
    La continuité de \( f_n\) nous fournit un \( \delta>0\) tel que \( \| f_n(x_0)-f_n(x) \|<\epsilon\) dès que \( \| x-x_0 \|<\delta\). Pour ce \( \delta\), nous avons alors \( \| f(x)-f(x_0) \|<\epsilon\).
\end{proof}

\begin{proposition}[Permuter limite et intégrale]       \label{PropbhKnth}
    Soit \( f_n\to f\) uniformément sur un ensemble mesuré \( A\) de mesure finie. Alors si les fonctions \( f_n\) et \( f\) sont intégrables sur \( A\), nous avons
    \begin{equation}
        \lim_{n\to \infty} \int_A f_n=\int_A \lim_{n\to \infty} f_n.
    \end{equation}
\end{proposition}

\begin{proof}
    Notons \( f\) la limite de la suite \( (f_n)\). Pour tout \( n\) nous avons les majorations
    \begin{subequations}
        \begin{align}
            \left| \int_A f_n d\mu-\int_A fd\mu \right| &\leq \int_A| f_n-f |d\mu\\
            &\leq \int_A \| f_n-f \|_{\infty}d\mu\\
            &=\mu(A)\| f_n-f \|_{\infty}
        \end{align}
    \end{subequations}
    où \( \mu(A)\) est la mesure de \( A\). Le résultat découle maintenant du fait que \( \| f_n-f \|_{\infty}\to 0\).
\end{proof}
Il existe un résultat considérablement plus intéressant que cette proposition. En effet, l'intégrabilité de \( f\) n'est pas nécessaire. Cette hypothèse peut être remplacée soit par l'uniforme convergence de la suite (théorème \ref{ThoUnifCvIntRiem}), soit par le fait que les normes des \( f_n\) sont uniformément bornées (théorème de la convergence dominée de Lebesgue \ref{ThoConvDomLebVdhsTf}).

\begin{theorem}			\label{ThoUnifCvIntRiem}
    La limite uniforme d'une suite de fonctions intégrables sur un borné est intégrable, et on peut permuter la limite et l'intégrale. 
    
    Plus précisément, soit \( A\) un ensemble de \( \mu\)-mesure finie et \( f_n\colon A\to \eR\) des fonctions intégrables sur \( A\). Si la limite \( f_n\to f\) est uniforme, alors \( f\) est intégrable sur \( A\) et nous pouvons inverser la limite et l'intégrale :
    \begin{equation}
        \lim_{n\to \infty} \int_A f_n=\int_A\lim_{n\to \infty} f_n.
    \end{equation}
\end{theorem}
La preuve suivante est inspirée de celle fournie par \href{http://dubois.gilles.pagesperso-orange.fr/analyse_reelle/intlimites.html}{Gilles Dubois} dans le cas de l'intégrale de Riemann sur un intervalle compact.

\begin{proof}
    Soit \( \epsilon>0\) et \( n\) tel que \( \| f_n-f \|_{\infty}\leq \epsilon\) (ici la norme uniforme est prise sur \( A\)). Étant donné que \( f_n\) est intégrable sur \( A\), il existe une fonction simple \( \varphi_n\) qui minore \( f_n\) telle que
    \begin{equation}
        \left| \int_{A}\varphi_n-\int_A f_n \right| <\epsilon.
    \end{equation}
    La fonction \( \varphi_n+\epsilon\) est une fonction simple qui majore la fonction \( f\). Si \( \psi\) est une fonction simple qui minore \( f\), alors
    \begin{equation}
        \int_A\psi\leq\int_A\varphi_n+\epsilon\leq\int_A f_n+\epsilon\mu(A).
    \end{equation}
    Par conséquent le supremum qui définit \( \int_A f\) existe, ce qui montre que \( f\) est intégrable. Le fait qu'on puisse inverser la limite et l'intégrale est maintenant une conséquence de la proposition \ref{PropbhKnth}.
\end{proof}

\begin{remark}
    L'hypothèse sur le fait que \( A\) est de mesure finie est importante. Il n'est pas vrai qu'une suite uniformément convergente de fonctions intégrables est intégrables. En effet nous avons par exemple la suite
    \begin{equation}
        f_n(x)=\begin{cases}
            1/x    &   \text{si \( x<n\)}\\
            0    &    \text{sinon}
        \end{cases}
    \end{equation}
    qui converge uniformément vers \( f(x)=1/x\) sur \( A=\mathopen[ 1 , \infty [\). Le limite n'est cependant pas intégrable sur \( A\).
\end{remark}

\begin{theorem}[Page I.12]		\label{ThoSerUnifDerr}
	Soit $U\subset\eR^n$ ouvert, $f_k\colon U\to \eR$ et $f_k$ de classe $C^1$. Supposons que $f_k$ converge simplement vers $f$ et que $\partial_if_k$ converge uniformément sur tout compact  vers une fonction $g_i$ pour $i=1,\ldots,n$. Alors $f$ est de classe $C^1$ et $\partial_if=g_i$. De plus, $f_k$ converge vers $f$ uniformément.
\end{theorem}

\begin{theorem}				\label{ThoSerCritAbel}
	Soit $\sum_{k=1}^{\infty}g_k(x)$, une série de fonctions complexes où $g_k(x)=\varphi_k(x)\psi_k(x)$. Supposons que
	\begin{enumerate}

		\item
			$\varphi_k\colon A\to \eC$ et $| \sum_{k=1}^K\varphi_k(x) |\leq M$ où $M$ est indépendant de $x$ et $K$,
		\item
			$\psi_k\colon A\to \eR$ avec $\psi_k(x)\geq 0$ et pour tout $x$ dans $A$, $\psi_{k+1}(x)\leq \psi_k(x)$, et enfin supposons que $\psi_k(x)$ converge uniformément vers $0$.

	\end{enumerate}
	Alors $\sum_{k=1}^{\infty}g_k$ est uniformément convergente.
\end{theorem}

\begin{theorem}		\label{ThoAbelSeriePuiss}
	Si la série de puissances (réelle) converge en $x=x_0+R$, alors elle converge uniformément sur $\mathopen[ x_0-R+\epsilon , x_0+R \mathclose]$ ($\epsilon>0$) vers une fonction continue.
\end{theorem}

%---------------------------------------------------------------------------------------------------------------------------
\subsection{Convergence de suites de fonctions}
%---------------------------------------------------------------------------------------------------------------------------

Nous considérons un espace normé \( (\Omega,\| . \|)\). Nous disons qu'une suite de fonctions \( f_n\) \defe{converge}{convergence!en norme} vers \( f\) pour la norme \( \| . \|\) si \( \forall \epsilon>0\), \( \exists N\) tel que \( n\geq N\) implique \( \| f_n-f \|<\epsilon\).

Dans le cas particulier de la norme 
\begin{equation}
    \| f \|_{\infty}=\sup_{x\in\Omega}| f(x) |,
\end{equation}
nous parlons que \defe{convergence uniforme}{convergence!uniforme!suite de fonctions}.

\begin{theorem}[Critère de Cauchy]  \label{ThoCauchyZelUF}
    Une suite de fonctions  \( (f_n)_{n\in\eN}\) sur \( \Omega\) converge en norme sur \( \Omega\) si et seulement si \( \forall\epsilon>0\), \( \exists N\) tel que
    \begin{equation}
        \| f_n-f_m \|<\epsilon
    \end{equation}
    pour \( n,m>N\).
\end{theorem}

\begin{corollary}       \label{CorCauchyCkXnvY}
    La série \( \sum f_n\) converge en norme sur \( \Omega\) si et seulement si \( \exists N\) tel que
    \begin{equation}
        \| f_n+\ldots+f_m \|\leq \epsilon
    \end{equation}
    pour tout \( n,m>N\).
\end{corollary}

\begin{proof}
    L'hypothèse montre que la suite des sommes partielles de la série \( \sum f_n\) vérifie le critère de Cauchy du théorème \ref{ThoCauchyZelUF}.
\end{proof}

\begin{definition}
    Nous disons qu'un sous ensemble \( A\) de \( \Omega\) est \defe{complet}{complet} si toute suite de Cauchy d'éléments de \( A\) converge vers un élément de \( A\).
\end{definition}

%---------------------------------------------------------------------------------------------------------------------------
\subsection{Convergence monotone}
%---------------------------------------------------------------------------------------------------------------------------

Source : \cite{mathmecaChoi}.

\begin{theorem}[Théorème de la convergence monotone ou de Beppo-Levi] \label{ThoConvMonFtBoVh}\index{théorème!convergence monotone}\index{théorème!Beppo-Levi}
    Soit un espace mesuré \( (\Omega,\tribA,\mu)\) et \( (f_n)\) une suite croissante de fonctions mesurables à valeurs dans \( \mathopen[ 0 , \infty \mathclose]\). Alors la limite ponctuelle \( \lim_{n\to \infty} f_n\) existe, est mesurable et
    \begin{equation}
        \lim_{n\to \infty} \int_{\Omega}f_nd\mu= \int_{\Omega}\lim_{n\to \infty} f_nd\mu.
    \end{equation}
\end{theorem}

\begin{proof}
    La limite ponctuelle de la suite est la fonction à valeurs dans \( \mathopen[ 0 , \infty \mathclose]\) donnée par
    \begin{equation}
        f(x)=\lim_{n\to \infty} f_n(x).
    \end{equation}
    Ces limites existent parce que pour chaque \( x\) la suite \( f_n(x)\) est une suite numérique croissante. Nous notons
    \begin{equation}
        I_0=\int_{\Omega}fd\mu.
    \end{equation}
    Nous posons par ailleurs
    \begin{equation}
        I_n=\int_{\Omega}f_n.
    \end{equation}
    Cela est une suite numérique croissante qui a par conséquent une limite que nous notons \( I=\lim_{n\to \infty} I_n\). Notre objectif est de montrer que \( I=I_0\). D'abord par croissance de la suite, pour tous $n$ nous avons \( I_n\leq I_0\), par conséquent \( I\leq I_0\).

    Nous prouvons maintenant l'inégalité dans l'autre sens en nous servant de la définition \eqref{EqDefintYfdmu}. Soit une fonction simple \( h\) telle que \( h\leq f\), et une constante \( 0<C<1\). Nous considérons les ensembles
    \begin{equation}
        E_n=\{ x\in\Omega\tq f_n(x)\geq Ch(x) \}.
    \end{equation}
    Ces ensembles vérifient les propriétés \( E_n\subset E_{n+1}\) et \( \bigcup_{n=1}^{\infty}E_n=\Omega\). Pour chaque \( n\) nous avons les inégalités
    \begin{equation}
        \int_{\Omega}f_n\geq\int_{E_n}f_n\geq C\int_{E_n}h.
    \end{equation}
    Si nous prenons la limite \( n\to\infty\) dans ces inégalités,
    \begin{equation}
        \lim_{n\to \infty} \int_{\Omega}f_n\geq C\lim_{n\to \infty} \int_{E_n}h=C\int_{\Omega}h.
    \end{equation}
    Par conséquent \( \lim_{n\to \infty} \int f_n\geq C\int_{\Omega}h\). Mais étant donné que cette inégalité est valable pour tout \( C\) entre \( 0\) et \( 1\), nous pouvons l'écrire sans le \( C\) :
    \begin{equation}        \label{EqzAKEaU}
        \lim_{n\to \infty} \int_{\Omega}f_n\geq \int_{\Omega}h.
    \end{equation}
    Par définition, l'intégrale de \( f\) est donné par le supremum des intégrales de \( h\) où \( h\) est une fonction simple dominée par \( f\). En prenant le supremum sur \( h\) dans l'équation \eqref{EqzAKEaU} nous avons
    \begin{equation}
        \lim_{n\to \infty} \int_{\Omega}f_n\geq\int_{\Omega}f,
    \end{equation}
    ce qu'il nous fallait.
\end{proof}

\begin{corollary}[Inversion de somme et intégrales]
    Si \( (u_n)\) est une suite de fonctions mesurables positives ou nulles, alors
    \begin{equation}
        \sum_{i=0}^{\infty}\int u_i=\int\sum_{i=0}^{\infty}u_i.
    \end{equation}
\end{corollary}

\begin{proof}
    Nous considérons la suite des sommes partielles de \( (u_n)\) : \( f_n(x)=\sum_{i=0}^nu_n(x)\). Le théorème de la convergence monotone (théorème \ref{ThoConvMonFtBoVh}) implique que
    \begin{equation}
        \lim_{n\to \infty} \int f_n=\int\lim_{n\to \infty} f_n.
    \end{equation}
    Nous remplaçons maintenant \( f_n\) par sa valeur en termes des \( u_i\) et dans le membre de gauche nous permutons l'intégrale avec la somme finie :
    \begin{equation}
        \lim_{n\to \infty} \sum_{i=0}^{\infty}\int u_n=\int\sum_{i=0}^{\infty}u_n,
    \end{equation}
    ce qu'il fallait démontrer.
\end{proof}

\begin{lemma}[Lemme de Fatou]\index{lemme!Fatou}\index{Fatou}   \label{LemFatouUOQqyk}
    Soit \( (\Omega,\tribA,\mu)\) un espace mesuré et \( f_n\colon \Omega\to \mathopen[ 0 , \infty \mathclose]  \) une suite de fonctions mesurables. Alors la fonction \( f(x)=\liminf f_n(x)\) est mesurable et
    \begin{equation}
        \int_{\Omega}\liminf f_nd\mu\leq\liminf\int_{\Omega}fd\mu.
    \end{equation}
\end{lemma}

\begin{proof}
    Nous posons 
    \begin{equation}
        g_n(x)=\inf_{i\geq n}f_i(x).
    \end{equation}
    Cela est une suite croissance de fonctions positives mesurables telles que, par définition, 
    \begin{equation}
        \lim_{n\to \infty}g_n(x)=\liminf f_n(x).
    \end{equation}
    Nous pouvons y appliquer le théorème de la convergence monotone,
    \begin{equation}
        \lim_{n\to \infty} \int g_n(x)=\int\liminf f_n(x).
    \end{equation}
    Par ailleurs, pour chaque \( i\geq n\) nous avons
    \begin{equation}
        \int g_n\leq \int f_i,
    \end{equation}
    en passant à l'infimum nous avons
    \begin{equation}
        \int g_n\leq \inf_{i\geq n}\int f_i,
    \end{equation}
    et en passant à la limite nous avons
    \begin{equation}
        \int\liminf f_n=\lim_{n\to \infty} \int g_n\leq \lim_{n\to \infty} \inf_{i\geq n}\int f_i=\liminf_{i\to\infty}\inf f_i.
    \end{equation}
\end{proof}

L'inégalité donnée dans ce lemme n'est en général pas une égalité, comme le montre l'exemple suivant :
\begin{equation}
    f_i=\begin{cases}
        \mtu_{\mathopen[ 0 , 1 \mathclose]}    &   \text{si \( i\) est pair}\\
        \mtu_{\mathopen[ 1 , 2 \mathclose]}    &    \text{si \( i\) est impair}.
    \end{cases}
\end{equation}
Nous avons évidemment \( g_n(x)=0\) tandis que \( \int_{\mathopen[ 0 , 2 \mathclose]}f_i=1\) pour tout \( i\).

%---------------------------------------------------------------------------------------------------------------------------
\subsection{Convergence dominée de Lebesgue}
%---------------------------------------------------------------------------------------------------------------------------

\begin{theorem}[Convergence dominée de Lebesgue]\index{théorème!convergence dominée de Lebesgue}        \label{ThoConvDomLebVdhsTf}
    Soit \( (f_n)_{n\in\eN}\) une suite de fonctions intégrables sur \( (\Omega,\tribA,\mu)\) à valeurs dans \( \eC\) ou \( \eR\). Nous supposons que  \( f_n\to f\) simplement sur \( \Omega\) presque partout et qu'il existe une fonction intégrable \( g\) telle que
    \begin{equation}
        | f_n(x) |< g(x) 
    \end{equation}
    pour presque\footnote{Si il n'y avait pas le «presque» ici, ce théorème serait à peu près inutilisable en probabilité.} tout \( x\in\Omega\) et pour tout \( n\in \eN\). Alors
    \begin{enumerate}
        \item
            \( f\) est intégrable,
        \item
           $\lim_{n\to \infty} \int_{\Omega}f_n=\int_\Omega f$,
        \item
            $\lim_{n\to \infty} \int_{\Omega}| f_n-f |=0$.
    \end{enumerate}
\end{theorem}

\begin{proof}

    La fonction limite \( f\) est intégrable parce que \( | f |\leq g\) et \( g\) est intégrable (lemme \ref{LemPfHgal}). Par hypothèse nous avons
    \begin{equation}
        -g(x)\leq f_n(x)\leq g(x).
    \end{equation}
    En particulier la fonction \( g_n=f_n+g\) est positive et mesurable si bien que le lemme de Fatou (lemme \ref{LemFatouUOQqyk}) implique
    \begin{equation}
        \int_{\Omega}\liminf g_n\leq\liminf\int_{\Omega}g_n.
    \end{equation}
    Évidement nous avons \( \liminf g_n=f+g\), de telle sorte que
    \begin{equation}
        \int f+\int g\leq \liminf\int g_n=\liminf\int f_n+\int g,
    \end{equation}
    et le nombre \( \int g\) étant fini, nous pouvons le retrancher des deux côtés de l'inégalité :
    \begin{equation}
        \int f\leq\liminf\int f_n.
    \end{equation}
    Afin d'obtenir une minoration de \( \int f\) nous refaisons exactement le même raisonnement en utilisant la suite de fonctions \( k_n=-f_n\to k=-f\). Nous obtenons que
    \begin{equation}
        \int k\geq\liminf\int k_n=-\limsup\int f_n,
    \end{equation}
    et par conséquent
    \begin{equation}    \label{IneqsndMYTO}
        \liminf\int f_n\leq\int f\leq\limsup\int f_n.
    \end{equation}
    La limite supérieure étant plus grande ou égale à la limite inférieure, les trois quantités dans les inégalités \eqref{IneqsndMYTO} sont égales.
\end{proof}

\begin{remark}
    En probabilités, la fonction \( g\) peut être une constante parce que les constantes sont intégrables sur un espace de probabilité.
\end{remark}

\begin{corollary}       \label{CorCvAbsNormwEZdRc}
    Soit \( (a_i)_{i\in \eN}\) une suite numérique absolument convergente. Alors elle est convergente. Il en est de même pour les séries de fonctions si on considère la convergence ponctuelle.
\end{corollary}

\begin{proof}
    L'hypothèse est la convergence de l'intégrale \( \int_{\eN}| a_i |dm(i)\) où \( dm\) est la mesure de comptage. Étant donné que \( | a_i |\leq | a_i |\), la fonction \( a_i\) (fonction de \( i\)) peut jouer le rôle de \( g\) dans le théorème de la convergence dominée de Lebesgue (théorème \ref{ThoConvDomLebVdhsTf}).
\end{proof}
Nous utiliseront ce résultat pour montrer que la transformée de Fourier d'une fonction \( L^1(\eR^d)\) est continue (proposition \ref{PropJvNfj}).

%---------------------------------------------------------------------------------------------------------------------------
\subsection{Séries de fonctions}
%---------------------------------------------------------------------------------------------------------------------------

Les séries de fonctions sont des cas particuliers de suites, étant donné que, par définition,
\begin{equation}
    \sum_{n=1}^{\infty}f_n=\lim_{N\to \infty} \sum_{n=1}^{N}f_n.
\end{equation}

Nous rappelons qu'une série de nombres \( \sum_{n=0}^{\infty}a_n\) converge \defe{absolument}{convergence!absolue} si la série
\begin{equation}
    \sum_{n=0}^{\infty}| a_n |
\end{equation}
converge. Cette définition s'étend immédiatement aux séries dans n'importe quel espace normé.

Une série de fonctions \( \sum_{n\in \eN}u_n \) converge \defe{normalement}{convergence!normale} si la série de nombre \( \sum_n\| u_n \|_{\infty}\) converge.

La convergence normale est à ne pas confondre avec la convergence uniforme. La somme \( \sum_nf_n\) \defe{converge uniformément}{convergence!uniforme!série de fonctions} vers la fonction \( F\) si la suite des sommes partielles converge uniformément, c'est à dire si 
\begin{equation}
    \lim_{N\to \infty} \| \sum_{n=1}^Nf_n-F \|_{\infty}=0.
\end{equation}

\begin{lemma}
    Soient des fonctions \( u_n\colon \Omega\to \eC\). Si il existe une suite réelle positive \( (a_n)_{n\in \eN}\) telle que
    \begin{enumerate}
        \item
            pour tout \( z\in \Omega\) et pour tout \( n\in \eN\) nous avons \( | u_n(z) |\leq a_n\) (c'est à dire \( a_n\geq \| u_n \|_{\infty}\)),
        \item
            la somme \( \sum_{n}a_n\) converge,
    \end{enumerate}
    alors la série de fonctions \( \sum_{n=0}^{\infty}u_n\) converge normalement.
\end{lemma}

\begin{proof}
    Découle du lemme de comparaison \ref{LemgHWyfG}.
\end{proof}

\begin{proposition}     \label{PropUEMoNF}
    Soit \( (u_n)\) une suite de fonctions continues \( u_n\colon \Omega\subset\eC\to \eC\). Si la série \( \sum_nu_n\) converge normalement alors la somme est continue.
\end{proposition}

\begin{proof}
    Nous posons \( u(z)=\lim_{n\to \infty} u_n(z)\), et nous vérifions que la fonction ainsi définie sur \( \Omega\) est continue. Soit \( z\in \Omega\) et prouvons la continuité de \( u\) au point \( z\). Pour tout \( z'\) dans un voisinage de \( z\) nous avons 
    \begin{subequations}
        \begin{align}
            \big| u(z)-u(z') \big|&=\left| \sum_{n=0}^{N}u_n(z)-\sum_{n=0}^{N}u_n(z')+\sum_{n=N+1}^{\infty}u_n(z)-\sum_{n=N+1}^{\infty}u_n(z') \right| \\
            &\leq \left| \sum_{n=0}^N u_n(z)-\sum_{n=0}^Nu_n(z') \right| +\sum_{n=N+1}^{\infty}| u_n(z) |+\sum_{n=N+1}^{\infty}| u_n(z') |.
        \end{align}
    \end{subequations}
    Étant donné que les sommes partielles sont continues, en prenant \( N\) suffisamment grand, le premier terme peut être rendu arbitrairement petit. Si \( N\) est suffisamment grand, le second terme est également petit. Par contre, cet argument ne tient pas pour le troisième terme parce que nous souhaitons une majoration pour tout \( z'\) dans une boule autour de \( z\). Nous devons donc écrire
    \begin{equation}
        \sum_{n=N}^{\infty}| u_(z) |\leq \sum_{n=N+1}^{\infty}\| u_n \|_{\infty}.
    \end{equation}
    Ce dernier est arbitrairement petit lorsque \( N\) est grand. Notons que nous avons utilisé l'hypothèse de convergence normale.
\end{proof}

\begin{theorem}			\label{ThoSerUnifCont}
	Si les $g_k$ sont continues et si $\sum g_k$ converge uniformément, alors $\sum g_k$ est continue.
\end{theorem}

\begin{theorem}[Critère de Weierstrass]\index{critère!Weierstrass!série de fonctions}		\label{ThoCritWeierstrass}
	Soit une suite de fonctions $f_k\colon A\to \eC$ telles que $| f_k(x) |\leq M_k\in\eR$, $\forall x\in A$. Si $\sum_{k=1}^{\infty}M_k$ converge, alors $\sum_{k=1}^{\infty}f_k$ converge absolument et uniformément.
\end{theorem}

\begin{proof}
    La convergence normale est facile : l'hypothèse dit que \( \| f_k \|_{\infty}\leq M_k\), et donc que
    \begin{equation}
        \sum_{k=1}^{\infty}\| f_k \|_{\infty}\leq \sum_kM_k<\infty.
    \end{equation}
    
    La convergence uniforme est à peine plus subtile. Nous nommons \( F\) la fonction somme. Pour tout \( x\) et pour tout \( N\), nous avons
    \begin{subequations}
        \begin{align}
            \left\| \sum_{n=1}^Nf_n(x)-F(x) \right\|&=\| \sum_{n=N}^{\infty}f_n(x) \|\\
            &\leq\sum_{n=N}^{\infty}\| f_k(x) \|\\
            &\leq \sum_{n=N}^{\infty}\| f_n \|_{\infty}.
        \end{align}
    \end{subequations}
    La convergence normale étant assurée, la série \( \sum_{n_1}^{\infty}\| f_n \|_{\infty}\) est finie, ce qui implique que la queue de somme \( \sum_{n=N}^{\infty}\| f_n \|_{\infty}\) tend vers zéro lorsque \( N\to \infty\). Pour tout \( \epsilon\), il existe donc un \( N\) (non dépendant de \( x\)) tel que
    \begin{equation}
        \| \sum_{n=1}^Nf_n(x)-F(x) \|\leq \epsilon.
    \end{equation}
    En prenant le supremum sur \( x\in A\) nous trouvons la convergence uniforme.
\end{proof}

\begin{remark}
    Il n'y a pas de critère correspondant pour les suites. Il n'est pas vrai que si \( \lim_{n\to \infty}\| f_n \| \) existe, alors \( \lim_{n\to \infty} f_n\) existe, comme le montre l'exemple
    \begin{equation}
        f_n(x)=\begin{cases}
            1    &   \text{si \( x\in\mathopen[ 0 , 1 \mathclose]\) et \( n\) est pair}\\
            1    &    \text{si \( x\in\mathopen[ 1 , 2 \mathclose]\) et \( n\) est impair}\\
             0   &    \text{sinon.}
        \end{cases}
    \end{equation}
\end{remark}

\begin{theorem}      \label{ThoCciOlZ}
    La somme uniforme de fonctions intégrables sur un ensemble de mesure fini est intégrable et on peut permuter la somme et l'intégrale.

    En d'autres termes, supposons que \( \sum_{n=0}^{\infty}f_n\) converge uniformément vers \( F\) sur \( A\) avec \( \mu(A)<\infty\). Si \( F\) et \( f_n\) sont des fonctions intégrables sur \( A\) alors
    \begin{equation}
        \int_AF(x)d\mu(x)=\sum_{n=0}^{\infty}\int_Af_n(x)d\mu(x).
    \end{equation}
\end{theorem}

\begin{proof}
    Ce théorème est une conséquence du théorème \ref{ThoUnifCvIntRiem}. En effet nous définissons la suite des sommes partielles
    \begin{equation}
        F_N=\sum_{n=0}^Nf_n.
    \end{equation}
    La limite \( \lim_{N\to \infty} F_N=F\) est uniforme. Par conséquent la fonction \( F\) est intégrable et
    \begin{equation}
        \int_A F=\lim_{N\to \infty} \int_AF_N=\lim_{N\to \infty} \int_A\sum_{n=0}^Nf_n=\lim_{N\to \infty} \sum_{n=0}^N\int_Af_n=\sum_{n=0}^{\infty}\int_Af_n.
    \end{equation}
    La première égalité est le théorème \ref{ThoUnifCvIntRiem}, les autres sont de simples manipulations rhétoriques.
\end{proof}


Le théorème suivant est une paraphrase du théorème de la convergence dominée de Lebesgue (\ref{ThoConvDomLebVdhsTf}).
\begin{theorem}     \label{ThoockMHn}
    Soient des fonctions \( (f_n)_{n\in \eN}\) telles que \( \sum_{n=0}^Nf_n\) soit intégrable sur \( (\Omega,\tribA,\mu)\) pour chaque \( N\). Nous supposons que la somme converge simplement vers
    \begin{equation}
        f(x)=\sum_{n=0}^{\infty}f_n(x)
    \end{equation}
    et qu'il existe une fonction \( g\) telle que
    \begin{equation}
        \left| \sum_{n=0}^Nf_n \right| <g
    \end{equation}
    pour tout \( N\in \eN\). Alors
    \begin{enumerate}
        \item
            \( \sum_{n=O}^{\infty}f_n\) est intégrable,
        \item
            on peut permuter somme et intégrale :
            \begin{equation}
                \lim_{N\to \infty} \int_{\Omega}\sum_{n=0}^Nf_nd\mu=\int_{\Omega}\sum_{n=0}^{\infty}f_n,
            \end{equation}
        \item
            \begin{equation}
                \lim_{N\to \infty} \int_{\Omega}\left| \sum_{n=0}^Nf_n-\sum_{n=0}^{\infty}f_n \right| =\lim_{N\to \infty} \int_{\Omega}\left| \sum_{n=N}^{\infty}f_n \right| =0.
            \end{equation}
    \end{enumerate}
\end{theorem}


\begin{theorem} \label{ThoCSGaPY}
    Soit \( f_n\) des fonctions \( C^1\mathopen[ a , b \mathclose]\) telles que
    \begin{enumerate}
        \item
            la série \( \sum_n f_n(x_0)\) converge pour un certain \( x_0\in\mathopen[ a , b \mathclose]\),
        \item
            la série des dérivées \( \sum_n f'_n\) converge uniformément sur \( \mathopen[ a , b \mathclose]\).
    \end{enumerate}
    Alors la série \( \sum_n f_n\) converge vers une fonction \( F\) et
    \begin{enumerate}
        \item
            La convergence est uniforme sur \( \mathopen[ a , b \mathclose]\).
        \item
            La fonction \( F\) est dérivable
        \item
            \( F'(x)=\sum_nf'_n(x)\).
    \end{enumerate}
\end{theorem}

%---------------------------------------------------------------------------------------------------------------------------
\subsection{Suite équirépartie}
%---------------------------------------------------------------------------------------------------------------------------

\begin{definition}
    Une suite \( (x_n)\) dans \( \mathopen[ 0 , 1 [\) est \defe{équirépartie}{suite!équirépartie} si pour tout \( 0\leq a<b<1\), on a
    \begin{equation}
        \frac{1}{ n }N(n,a,b)\stackrel{n\to\infty}{\longrightarrow}b-a
    \end{equation}
    où
    \begin{equation}
        N(n,a,b)=\Card\{ m<n\tq a\leq x_m\leq b \}.
    \end{equation}
\end{definition}

\begin{proposition}[\cite{ytMOpe}]
    Soit \( (x_n)\) une suite dans \( \mathopen[ 0 , 1 [\). On a équivalence entre les deux points suivants.
    \begin{enumerate}
        \item
            La suite \( (x_n)\) est équirépartie.
        \item
            Pour toute fonction \( 1\)-périodique et continue,
            \begin{equation}
                \int_0^1f(x)dx=\lim_{n\to \infty} \frac{1}{ n }\sum_{k=1}^nf(x_k).
            \end{equation}
    \end{enumerate}
\end{proposition}

\begin{proof}
    On pose 
    \begin{equation}
        S_n(f)=\frac{1}{ n }\sum_{k=1}^nf(x_k).
    \end{equation}
    Nous supposons que la suite est équirépartie et nous commençons par montrer le résultat pour les fonctions en escalier. Soit donc la fonction en escalier \( \eta(x)=c_j\) sur \( a_{j-1}< x<a_j\). Sur le point \( a_j\) lui-même, la fonction \( \eta\) vaut soit \( c_j\) soit \( c_{j+1}\). Nous avons
    \begin{equation}    \label{EqohMuel}
        \frac{1}{ n }\sum_{k=1}^n\eta(x_k)=\frac{1}{ n }\left[  \sum_{j=1}^mc_jN(n,a_j,a_{j+1})-\sum_{j=1}^mc_jN(n,a_j,a_j)+\sum_{j=1}^m\eta(a_j)N(n,a_j,a_j) \right].
    \end{equation}
    À la limite \( n\to\infty\), les deux derniers termes tombent\footnote{J'en profite pour mentionner que mon équation \eqref{EqohMuel} n'est pas la même que celle de \cite{ytMOpe} dans laquelle il me semble voir une faute; quoi qu'il en soit, les termes litigieux tombent.} et il reste
    %TODO : savoir si c'est moi ou l'autre qui a raison.
    \begin{equation}
        \lim_{n\to \infty} \frac{1}{ n }\sum_{k=1}^n\eta(x_k)=\sum_{j=1}^mc_j(a_{j-1}-a_j).
    \end{equation}
    Or par construction, pour une fonction en escalier,
    \begin{equation}
        \sum_{j=1}^mc_j(a_{j-1}-a_j)=\int_0^1\eta.
    \end{equation}
    
    Soit maintenant une fonction continue \( f\). Par continuité elle est intégrable au sens de Riemann et nous considérons deux fonctions en escaliers \( \varphi\leq f\leq \psi\) telles que \( \int \psi-\varphi\leq \epsilon\). Nous avons aussi
    \begin{equation}    \label{EqHerTgZ}
        \int_0^1\varphi\leq\int_0^1f\leq \int_0^1\psi,
    \end{equation}
    et
    \begin{equation}
        S_n(\varphi)\leq S_n(f)\leq S_n(\psi).
    \end{equation}
    En soustrayant terme à terme les deux chaînes d'inégalités (en retournant la deuxième),
    \begin{equation}    \label{EqSnAZTw}
        \int\varphi-S_n(\psi)\leq \int f-S_n(f)\leq \int\psi-S_n(\varphi).
    \end{equation}
    En même temps par le cas des fonctions en escalier déjà prouvé, \( S_n(\varphi)\to\int_0^1\varphi\) et \( S_n(\psi)\to\int_0^1\psi\). En prenant alors la limite \( n\to \infty\) dans \eqref{EqSnAZTw},
    \begin{equation}
        -\epsilon\leq \int\varphi-\int \psi\leq \int f-\lim S_n(f)\leq \int\psi-\int\varphi\leq \epsilon.    
    \end{equation}
    Cela étant valable pour tout \( \epsilon\), nous avons bien
    \begin{equation}
        \int_0^1f=\lim_{n\to \infty} S_n(f).
    \end{equation}
    
    Nous prouvons maintenant le sens inverse. C'est à dire que pour toute fonction continue et \( 1\)-périodique, nous avons
    \begin{equation}
        \int_0^1f(x)dx=\lim_{n\to \infty} \frac{1}{ n }\sum_{k=1}^nf(x_k).
    \end{equation}
    Nous devons en déduire que \( (x_n)\) est équirépartie. Pour ce faire, soit \( x\in \mathopen[ 0 , 1 [\) et \( \epsilon>0\) tel que \( x+\epsilon<1\). Nous considérons \( \varphi=\mtu_{\mathopen[ x , 1 [}\) et
    \begin{equation}
        \varphi_{\epsilon(t)}=\begin{cases}
            0    &   \text{si \( t\in\mathopen[ 0 , x [\)}\\
            \frac{ t-x }{ \epsilon }    &   \text{si \( t\in \mathopen[ x , x+\epsilon [\)}\\
            1    &    \text{si \( t\geq x+\epsilon\)}.
        \end{cases}
    \end{equation}
    Cela est une fonction continue, donc
    \begin{equation}
        \lim_{n\to \infty} S_n\big( \varphi_{\epsilon}(t) \big)=\int_0^1\varphi_{\epsilon}(t)dt=\int_{x}^{x+\epsilon}\frac{ t-x }{ \epsilon }dt+\int_{x+\epsilon}^11dt=1-x-\frac{ \epsilon }{2}.
    \end{equation}
    Mais \( \varphi_{\epsilon}\leq \varphi\), donc \( S_n(\varphi_{\epsilon})\leq S_n(\varphi)\) et donc
    \begin{equation}
        \liminf_{n\to \infty}S_n(\varphi)\geq 1-x.
    \end{equation}
    Notons que nous ne savons pas si la \emph{vraie} limite de gauche existe; c'est pourquoi nous prenons la limite inférieure, qui existe toujours.

    Nous définissons aussi
    \begin{equation}
        \psi_{\epsilon}(t)=\begin{cases}
            0    &   \text{si \( t\in \mathopen[ 0 , x-\epsilon [\)}\\
            \frac{ t-x+\epsilon }{ \epsilon }    &   \text{si \( t\in\mathopen[ x-\epsilon , x [\)}\\
            1    &    \text{si \( t>x\)}.
        \end{cases}
    \end{equation}
    C'est encore une fonction continue et nous trouvons\footnote{Je recommande chaudement de dessiner les fonctions \( \varphi_{\epsilon}\) et \( \psi_{\epsilon}\) pour avoir une idée de la situation.}
    \begin{equation}
        \int_0^1\psi_{\epsilon}(t)dt=1-x+\frac{ \epsilon }{2}.
    \end{equation}
    Vu que \( \psi_{\epsilon}\geq\varphi\), nous avons \( S_n(\psi_{\epsilon})\geq S_n(\varphi)\) et donc
    \begin{equation}
        \limsup_{n}S_n(\varphi)\leq 1-x.
    \end{equation}
    Nous avons déjà obtenu que
    \begin{equation}
        1-x\leq\liminf S_n(\varphi)\leq \limsup S_n(\varphi)\leq 1-x,
    \end{equation}
    donc la limite existe et vaut
    \begin{equation}
        \lim_{n\to \infty} S_n(\varphi)=1-x.
    \end{equation}
    Cela est pour la fonction caractéristique \( \varphi=\mtu_{\mathopen[ x , 1 [}\). Si nous prenons une fonction caractéristique \( \mtu_{\mathopen[ a , b \mathclose]}\), nous avons la même chose parce que \( \mtu_{\mathopen[ a , b [}\) est une combinaisons linéaire de fonctions du type \( \mtu_{\mathopen[ x , 1 [}\).

    Nous avons donc
    \begin{equation}
        \lim_{n\to \infty} S_n\big( \mtu_{\mathopen[ a , b \mathclose]} \big)=b-a,
    \end{equation}
    alors que le membre de gauche n'est autre que
    \begin{equation}
        S_n\big( \mtu_{\mathopen[ a , b \mathclose]} \big)=\frac{1}{ n }\sum_{k=1}^n\mtu_{\mathopen[ a , b \mathclose]}(x_k)=\frac{1}{ n }N(n,a,b).
    \end{equation}
\end{proof}
