% This is part of Mes notes de mathématique
% Copyright (c) 2011-2012
%   Laurent Claessens
% See the file fdl-1.3.txt for copying conditions.

%+++++++++++++++++++++++++++++++++++++++++++++++++++++++++++++++++++++++++++++++++++++++++++++++++++++++++++++++++++++++++++
\section{Suites et séries de fonctions}
%+++++++++++++++++++++++++++++++++++++++++++++++++++++++++++++++++++++++++++++++++++++++++++++++++++++++++++++++++++++++++++
Source : \cite{TrenchRealAnalisys}.

\begin{theorem}			\label{ThoUnigCvCont}
	Soit $f_k\colon A\subset\eR^n\to \eR^m$ une suite de fonctions continues, convergeant uniformément vers $f$. Alors $f$ est continue.
\end{theorem}

\begin{theorem}			\label{ThoUnifCvIntRiem}
	Soit $A$ un ensemble mesurable et borné de $\eR^n$ et $f_k\colon A\to \eR$ des fonctions bornées et intégrables au sens de Lebesgue. Si la suite $f_k$ converge uniformément vers $f$, alors $f$ est bornée et intégrable au sens de Lebesgue et 
	\begin{equation}
		\int_A f=\lim_{k\to\infty} f_k.
	\end{equation}
\end{theorem}

\begin{theorem}			\label{ThoSerUnifCont}
	Si les $g_k$ sont continues et si $\sum g_k$ converge uniformément, alors $\sum g_k$ est continue.
\end{theorem}

\begin{theorem}		\label{ThoCritWeierstrass}
	Soient $g_k\colon A\to \eC$ et $\sum_{k=1}^{\infty}g_k$. Si $| g_k(x) |\leq M_k\in\eR$, $\forall x\in A$ et si $\sum_{k=1}^{\infty}M_k$ converge, alors $\sum_{k=1}^{\infty}g_k$ converge absolument et uniformément.
\end{theorem}

\begin{theorem}[Page I.12]		\label{ThoSerUnifDerr}
	Soit $U\subset\eR^n$ ouvert, $f_k\colon U\to \eR$ et $f_k$ de classe $C^1$. Supposons que $f_k$ converge simplement vers $f$ et que $\partial_if_k$ converge uniformément sur tout compact  vers une fonction $g_i$ pour $i=1,\ldots,n$. Alors $f$ est de classe $C^1$ et $\partial_if=g_i$. De plus, $f_k$ converge vers $f$ uniformément.
\end{theorem}

\begin{theorem}				\label{ThoSerCritAbel}
	Soit $\sum_{k=1}^{\infty}g_k(x)$, une série de fonctions complexes où $g_k(x)=\varphi_k(x)\psi_k(x)$. Supposons que
	\begin{enumerate}

		\item
			$\varphi_k\colon A\to \eC$ et $| \sum_{k=1}^K\varphi_k(x) |\leq M$ où $M$ est indépendant de $x$ et $K$,
		\item
			$\psi_k\colon A\to \eR$ avec $\psi_k(x)\geq 0$ et pour tout $x$ dans $A$, $\psi_{k+1}(x)\leq \psi_k(x)$, et enfin supposons que $\psi_k(x)$ converge uniformément vers $0$.

	\end{enumerate}
	Alors $\sum_{k=1}^{\infty}g_k$ est uniformément convergente.
\end{theorem}

\begin{theorem}		\label{ThoAbelSeriePuiss}
	Si la série de puissances (réelle) converge en $x=x_0+R$, alors elle converge uniformément sur $\mathopen[ x_0-R+\epsilon , x_0+R \mathclose]$ ($\epsilon>0$) vers une fonction continue.
\end{theorem}

%---------------------------------------------------------------------------------------------------------------------------
\subsection{Convergence de suites de fonctions}
%---------------------------------------------------------------------------------------------------------------------------

Nous considérons un espace normé \( (\Omega,\| . \|)\). Nous disons qu'une suite de fonctions \( f_n\) \defe{converge}{convergence!en norme} vers \( f\) pour la norme \( \| . \|\) si \( \forall \epsilon>0\), \( \exists N\) tel que \( n\geq N\) implique \( \| f_n-f \|<\epsilon\).

Dans le cas particulier de la norme 
\begin{equation}
    \| f \|_{\infty}=\sup_{x\in\Omega}| f(x) |,
\end{equation}
nous parlons que \defe{convergence uniforme}{convergence!uniforme!suite de fonctions}.

\begin{theorem}[Critère de Cauchy]  \label{ThoCauchyZelUF}
    Une suite de fonctions  \( (f_n)_{n\in\eN}\) sur \( \Omega\) converge en norme sur \( \Omega\) si et seulement si \( \forall\epsilon>0\), \( \exists N\) tel que
    \begin{equation}
        \| f_n-f_m \|<\epsilon
    \end{equation}
    pour \( n,m>N\).
\end{theorem}

\begin{corollary}       \label{CorCauchyCkXnvY}
    La série \( \sum f_n\) converge en norme sur \( \Omega\) si et seulement si \( \exists N\) tel que
    \begin{equation}
        \| f_n+\ldots+f_m \|\leq \epsilon
    \end{equation}
    pour tout \( n,m>N\).
\end{corollary}

\begin{proof}
    L'hypothèse montre que la suite des sommes partielles de la série \( \sum f_n\) vérifie le critère de Cauchy du théorème \ref{ThoCauchyZelUF}.
\end{proof}

\begin{definition}
    Nous disons qu'un sous ensemble \( A\) de \( \Omega\) est \defe{complet}{complet} si toute suite de Cauchy d'éléments de \( A\) converge vers un élément de \( A\).
\end{definition}

\begin{theorem}[Weierstrass]
    La série de fonctions \( \sum_{n=1}^{\infty}f_n\) converge en norme si \( \| f_n \|\leq M_n\) avec \( \sum_nM_n<\infty\).
\end{theorem}

\begin{proof}
    Si \( (S_n)_{n\in\eN}\) est la suite des sommes partielles de \( \sum M_n\), alors le critère de Cauchy pour la convergence de suites numériques dit que si \( m\) et \( n\) sont assez grands, \( S_m-S_{n-1}<\epsilon\), c'est à dire
    \begin{equation}
        M_n+\ldots+M_n<\epsilon.
    \end{equation}
    Par conséquent nous avons
    \begin{equation}
        \| f_n+\ldots+f_m \|\leq\| f_n \|+\ldots\| f_m \|\leq \epsilon.
    \end{equation}
    Le corollaire \ref{CorCauchyCkXnvY} montre alors la convergence de la série.
\end{proof}

En corollaire, si \( \sum_n\| f_n \|\) converge, alors \( \sum_nf_n\) converge.

\begin{remark}
    Il n'y a pas de critère correspondant pour les suites. Il n'est pas vrai que si \( \lim_{n\to \infty}\| f_n \| \) existe, alors \( \lim_{n\to \infty} f_n\) existe, comme le montre l'exemple
    \begin{equation}
        f_n(x)=\begin{cases}
            1    &   \text{si \( x\in\mathopen[ 0 , 1 \mathclose]\) et \( n\) est pair}\\
            1    &    \text{si \( x\in\mathopen[ 1 , 2 \mathclose]\) et \( n\) est impair}\\
             0   &    \text{sinon.}
        \end{cases}
    \end{equation}
\end{remark}

%---------------------------------------------------------------------------------------------------------------------------
\subsection{Convergence monotone}
%---------------------------------------------------------------------------------------------------------------------------

Source : \cite{mathmecaChoi}.

\begin{theorem}[Théorème de la convergence monotone ou de Beppo-Levi] \label{ThoConvMonFtBoVh}\index{théorème!convergence monotone}\index{théorème!Beppo-Levi}
    Soit un espace mesuré \( (\Omega,\tribA,\mu)\) et \( (f_n)\) une suite croissante de fonctions mesurables à valeurs dans \( \mathopen[ 0 , \infty \mathclose]\). Alors la limite ponctuelle \( \lim_{n\to \infty} f_n\) existe, est mesurable et
    \begin{equation}
        \lim_{n\to \infty} \int_{\Omega}f_nd\mu= \int_{\Omega}\lim_{n\to \infty} f_nd\mu.
    \end{equation}
\end{theorem}

\begin{proof}
    La limite ponctuelle de la suite est la fonction à valeurs dans \( \mathopen[ 0 , \infty \mathclose]\) donnée par
    \begin{equation}
        f(x)=\lim_{n\to \infty} f_n(x).
    \end{equation}
    Ces limites existent parce que pour chaque \( x\) la suite \( f_n(x)\) est une suite numérique croissante. Nous notons
    \begin{equation}
        I_0=\int_{\Omega}fd\mu.
    \end{equation}
    Nous posons par ailleurs
    \begin{equation}
        I_n=\int_{\Omega}f_n.
    \end{equation}
    Cela est une suite numérique croissante qui a par conséquent une limite que nous notons \( I=\lim_{n\to \infty} I_n\). Notre objectif est de montrer que \( I=I_0\). D'abord par croissance de la suite, pour tous $n$ nous avons \( I_n\leq I_0\), par conséquent \( I\leq I_0\).

    Nous prouvons maintenant l'inégalité dans l'autre sens en nous servant de la définition \eqref{EqDefintYfdmu}. Soit une fonction simple \( h\) telle que \( h\leq f\), et une constante \( 0<C<1\). Nous considérons les ensembles
    \begin{equation}
        E_n=\{ x\in\Omega\tq f_n(x)\geq Ch(x) \}.
    \end{equation}
    Ces ensembles vérifient les propriétés \( E_n\subset E_{n+1}\) et \( \bigcup_{n=1}^{\infty}E_n=\Omega\). Pour chaque \( n\) nous avons les inégalités
    \begin{equation}
        \int_{\Omega}f_n\geq\int_{E_n}f_n\geq C\int_{E_n}h.
    \end{equation}
    Si nous prenons la limite \( n\to\infty\) dans ces inégalités,
    \begin{equation}
        \lim_{n\to \infty} \int_{\Omega}f_n\geq C\lim_{n\to \infty} \int_{E_n}h=C\int_{\Omega}h.
    \end{equation}
    Par conséquent \( \lim_{n\to \infty} \int f_n\geq C\int_{\Omega}h\). Mais étant donné que cette inégalité est valable pour tout \( C\) entre \( 0\) et \( 1\), nous pouvons l'écrire sans le \( C\) :
    \begin{equation}        \label{EqzAKEaU}
        \lim_{n\to \infty} \int_{\Omega}f_n\geq \int_{\Omega}h.
    \end{equation}
    Par définition, l'intégrale de \( f\) est donné par le supremum des intégrales de \( h\) où \( h\) est une fonction simple dominée par \( f\). En prenant le supremum sur \( h\) dans l'équation \eqref{EqzAKEaU} nous avons
    \begin{equation}
        \lim_{n\to \infty} \int_{\Omega}f_n\geq\int_{\Omega}f,
    \end{equation}
    ce qu'il nous fallait.
\end{proof}

\begin{corollary}[Inversion de somme et intégrales]
    Si \( (u_n)\) est une suite de fonctions mesurables positives ou nulles, alors
    \begin{equation}
        \sum_{i=0}^{\infty}\int u_i=\int\sum_{i=0}^{\infty}u_i.
    \end{equation}
\end{corollary}

\begin{proof}
    Nous considérons la suite des sommes partielles de \( (u_n)\) : \( f_n(x)=\sum_{i=0}^nu_n(x)\). Le théorème de la convergence monotone (théorème \ref{ThoConvMonFtBoVh}) implique que
    \begin{equation}
        \lim_{n\to \infty} \int f_n=\int\lim_{n\to \infty} f_n.
    \end{equation}
    Nous remplaçons maintenant \( f_n\) par sa valeur en termes des \( u_i\) et dans le membre de gauche nous permutons l'intégrale avec la somme finie :
    \begin{equation}
        \lim_{n\to \infty} \sum_{i=0}^{\infty}\int u_n=\int\sum_{i=0}^{\infty}u_n,
    \end{equation}
    ce qu'il fallait démontrer.
\end{proof}

\begin{lemma}[Lemme de Fatou]\index{lemme!Fatou}\index{Fatou}   \label{LemFatouUOQqyk}
    Soit \( (\Omega,\tribA,\mu)\) un espace mesuré et \( f_n\colon \Omega\to \mathopen[ 0 , \infty \mathclose]  \) une suite de fonctions mesurables. Alors la fonction \( f(x)=\liminf f_n(x)\) est mesurable et
    \begin{equation}
        \int_{\Omega}\liminf f_nd\mu\leq\liminf\int_{\Omega}fd\mu.
    \end{equation}
\end{lemma}

\begin{proof}
    Nous posons 
    \begin{equation}
        g_n(x)=\inf_{i\geq n}f_i(x).
    \end{equation}
    Cela est une suite croissance de fonctions positives mesurables telles que, par définition, 
    \begin{equation}
        \lim_{n\to \infty}g_n(x)=\liminf f_n(x).
    \end{equation}
    Nous pouvons y appliquer le théorème de la convergence monotone,
    \begin{equation}
        \lim_{n\to \infty} \int g_n(x)=\int\liminf f_n(x).
    \end{equation}
    Par ailleurs, pour chaque \( i\geq n\) nous avons
    \begin{equation}
        \int g_n\leq \int f_i,
    \end{equation}
    en passant à l'infimum nous avons
    \begin{equation}
        \int g_n\leq \inf_{i\geq n}\int f_i,
    \end{equation}
    et en passant à la limite nous avons
    \begin{equation}
        \int\liminf f_n=\lim_{n\to \infty} \int g_n\leq \lim_{n\to \infty} \inf_{i\geq n}\int f_i=\liminf_{i\to\infty}\inf f_i.
    \end{equation}
\end{proof}

L'inégalité donnée dans ce lemme n'est en général pas une égalité, comme le montre l'exemple suivant :
\begin{equation}
    f_i=\begin{cases}
        \mtu_{\mathopen[ 0 , 1 \mathclose]}    &   \text{si \( i\) est pair}\\
        \mtu_{\mathopen[ 1 , 2 \mathclose]}    &    \text{si \( i\) est impair}.
    \end{cases}
\end{equation}
Nous avons évidemment \( g_n(x)=0\) tandis que \( \int_{\mathopen[ 0 , 2 \mathclose]}f_i=1\) pour tout \( i\).

%---------------------------------------------------------------------------------------------------------------------------
\subsection{Convergence dominée de Lebesgue}
%---------------------------------------------------------------------------------------------------------------------------

\begin{theorem}[Convergence dominée de Lebesgue]\index{théorème!convergence dominée de Lebesgue}        \label{ThoConvDomLebVdhsTf}
    Soit \( (f_n)_{n\in\eN}\) une suite de fonctions intégrables sur \( (\Omega,\tribA,\mu)\) à valeurs dans \( \eC\) ou \( \eR\). Nous supposons que  \( f_n\to f\) simplement sur \( \Omega\) presque partout et qu'il existe une fonction intégrable \( g\) telle que
    \begin{equation}
        | f_n(x) |< g(x) 
    \end{equation}
    pour tout \( x\in\Omega\) et pour tout \( n\in \eN\). Alors
    \begin{enumerate}
        \item
            \( f\) est intégrable,
        \item
           $\lim_{n\to \infty} \int_{\Omega}f_n=\int_\Omega f$,
        \item
            $\lim_{n\to \infty} \int_{\Omega}| f_n-f |=0$.
    \end{enumerate}
\end{theorem}

\begin{proof}

    La fonction limite \( f\) est intégrable parce que \( | f |\leq g\) et \( g\) est intégrable (lemme \ref{LemPfHgal}). Par hypothèse nous avons
    \begin{equation}
        -g(x)\leq f_n(x)\leq g(x).
    \end{equation}
    En particulier la fonction \( g_n=f_n+g\) est positive et mesurable si bien que le lemme de Fatou (lemme \ref{LemFatouUOQqyk}) implique
    \begin{equation}
        \int_{\Omega}\liminf g_n\leq\liminf\int_{\Omega}g_n.
    \end{equation}
    Évidement nous avons \( \liminf g_n=f+g\), de telle sorte que
    \begin{equation}
        \int f+\int g\leq \liminf\int g_n=\liminf\int f_n+\int g,
    \end{equation}
    et le nombre \( \int g\) étant fini, nous pouvons le retrancher des deux côtés de l'inégalité :
    \begin{equation}
        \int f\leq\liminf\int f_n.
    \end{equation}
    Afin d'obtenir une minoration de \( \int f\) nous refaisons exactement le même raisonnement en utilisant la suite de fonctions \( k_n=-f_n\to k=-f\). Nous obtenons que
    \begin{equation}
        \int k\geq\liminf\int k_n=-\limsup\int f_n,
    \end{equation}
    et par conséquent
    \begin{equation}    \label{IneqsndMYTO}
        \liminf\int f_n\leq\int f\leq\limsup\int f_n.
    \end{equation}
    La limite supérieure étant plus grande ou égale à la limite inférieure, les trois quantités dans les inégalités \eqref{IneqsndMYTO} sont égales.
\end{proof}

\begin{corollary}       \label{CorCvAbsNormwEZdRc}
    Soit \( (a_i)_{i\in \eN}\) une suite numérique absolument convergente. Alors elle est convergente. Il en est de même pour les séries de fonctions si on considère la convergence ponctuelle.
\end{corollary}

\begin{proof}
    L'hypothèse est la convergence de l'intégrale \( \int_{\eN}| a_i |dm(i)\) où \( dm\) est la mesure de comptage. Étant donné que \( | a_i |\leq | a_i |\), la fonction \( a_i\) (fonction de \( i\)) peut jouer le rôle de \( g\) dans le théorème de la convergence dominée de Lebesgue (théorème \ref{ThoConvDomLebVdhsTf}).
\end{proof}

\begin{corollary}
    Soit une fonction \( f\in L^1(\eR^d)\). Alors sa transformée de Fourier est continue\index{transformée!Fourier!continuité}.
\end{corollary}

\begin{proof}
    Nous considérons une fonction \( f\) définie sur \( \eR^d\) et à valeurs dans \( \eR\) ou \( \eC\). Sa transformée de Fourier est donnée par
    \begin{equation}
        \hat f(\xi)=\int_{\eR^d} e^{-i\xi x}f(x)dx.
    \end{equation}
    Pour montrer que cette fonction \( \hat f\) est continue en \( \xi_0\) nous considérons une suite \( (\xi_n)\to \xi_0\) et nous voulons montrer que \( \hat f(\xi_n)\to\hat f(\xi_0)\). Pour cela nous considérons les fonctions
\begin{equation}
    g_n(x)= e^{-i\xi_nx}f(x)
\end{equation}
qui convergent simplement vers \( g(x)= e^{-i\xi x}f(x)\). Étant donné que
\begin{equation}
    | g_n(x) |<| f(x) |,
\end{equation}
le théorème de la convergence dominée donne alors
\begin{equation}
    \lim_{n\to \infty} \int g_n(x)=\int\lim_{n\to \infty } g_n(x),
\end{equation}
c'est à dire \( \lim_{n\to \infty} \hat f(\xi_n)=\hat f(\xi)\). La fonction \( \hat f\) est donc continue.
\end{proof}

\begin{theorem} \label{ThoKnuSNd}
    Soit \( (\Omega,\mu)\) est un espace mesuré, soit \( x_0\in \eR\) et \( f\colon \eR\times \Omega\to \eR\). Nous supposons que
    \begin{enumerate}
        \item
            La fonction \( f(x,.)\) est dans \( L^1(\Omega,\mu)\) pour tout \( x \in \eR\).
        \item
            La fonction \( f(.,\omega)\) est continue en \( x_0\) pour tout \( \omega\in\Omega\).
        \item       \label{ItemNAuYNG}
            Il existe une fonction \( G\in L^1(\Omega)\) telle que
            \begin{equation}
                | f(x,\omega) |\leq G(\omega)
            \end{equation}
            pour tout \( x\in \eR\).
    \end{enumerate}
    Alors la fonction 
    \begin{equation}
        \begin{aligned}
            F\colon \eR&\to \eR \\
            x&\mapsto \int_{\Omega}f(x,\omega)d\mu(\omega) 
        \end{aligned}
    \end{equation}
    est continue en \( x_0\).
\end{theorem}

\begin{proof}
    Soit \( (x_n)\) une suite convergente vers \( x_0\). Nous considérons la suite de fonctions \( f_n(\omega)=f(x_n,\omega)\) pour qui nous pouvons utiliser le théorème de la convergence dominée (théorème \ref{ThoConvDomLebVdhsTf}) pour obtenir
    \begin{subequations}
        \begin{align}
            \lim_{n\to \infty} F(x_n)&=\lim_{n\to \infty} \int_{\Omega}f(x_n,\omega)d\mu(\omega)\\
            &=\int_{\Omega}\lim_{n\to \infty} f(x_n,\omega)d\mu(\omega)\\
            &=\int_{\Omega}f(x,\omega)d\mu(\omega)\\
            &=F(x).
        \end{align}
    \end{subequations}
    Nous avons utilisé la continuité de \( f(.,\omega)\).
\end{proof}

Soit \( (\Omega,\mu)\) un espace mesuré. Nous disons que l'intégrale
\begin{equation}
    \int_{\Omega}f(x,\omega)d\mu(\omega)
\end{equation}
\defe{converge uniformément}{convergence!uniforme!intégrale} si pour tout \( \epsilon>0\), il existe un compact \( K_0\) tel que pour tout compact \( K\) tel que \( K_0\subset K\) nous avons
\begin{equation}
    \left| \int_{\Omega\setminus K}f(x,\omega)d\mu(\omega) \right| \leq \epsilon.
\end{equation}
Le point important est que le choix de \( K_0\) ne dépend pas de \( x\).

\begin{lemma}       \label{LemOgQdpJ}
    Soit
    \begin{equation}
        F(x)=\int_{\Omega}f(x,\omega)d\mu(\omega),
    \end{equation}
    une intégrale uniformément convergente. Pour chaque \( k\in \eN\) nous considérons un compact \( K_k\) tel que
    \begin{equation}
        \left| \int_{\Omega\setminus K_k}f(x,\omega)d\mu(\omega) \right| \leq\frac{1}{ k }.
    \end{equation}
    Alors la suite de fonctions \( F_k\) définie par
    \begin{equation}
        F_k(x)=\int_{K_k}f(x,\omega)d\mu(\omega)
    \end{equation}
    converge uniformément vers \( F\).
\end{lemma}

\begin{proof}
    Nous avons
    \begin{subequations}
        \begin{align}
            \big| F_k(x)-F(x) \big|&=\left| \int_{K_k}f(x,\omega)d\mu(\omega)-\int_{\Omega}f(x,\omega)d\mu(\omega) \right| \\
            &=| \int_{\Omega\setminus K_k}f(x,\omega)d\mu(\omega) |\\
            &\leq \frac{1}{ k }.
        \end{align}
    \end{subequations}
\end{proof}

Si nous avons un peu de compatibilité entre la topologie et la mesure, alors nous pouvons utiliser l'uniforme convergence d'une intégrale pour obtenir la continuité d'une fonction définie par une intégrale.

\begin{theorem}
    Soit \( (\Omega,\mu)\) un espace topologique mesuré tel que tout compact est de mesure finie. Soit une fonction \( f\colon \eR\times \Omega\to \eR\) telle que
    \begin{enumerate}
        \item
            Pour chaque \( x\in \eR\), la fonction \( f(x,.)\) est \( L^1(\Omega,\mu)\).
        \item
            Pour chaque \( \omega\in \Omega\), la fonction \( f(.,\omega)\) est continue en \( x_0\).
        \item
            L'intégrale
            \begin{equation}
                F(x)=\int_{\Omega}f(x,\omega)d\mu(\omega)
            \end{equation}
            est uniformément convergente.
    \end{enumerate}
    Alors la fonction \( F\) est continue en \( x_0\).
\end{theorem}

\begin{proof}
    Nous reprenons les notations du lemme \ref{LemOgQdpJ}. Les fonctions
    \begin{equation}
        F_k(x)=\int_{K_k}f(x,\omega)d\mu(\omega)
    \end{equation}
    existent parce que les fonctions \( f(x,.)\) sont dans \( L^1(\Omega)\). Montrons que les fonctions \( F_k\) sont continues. Soit une suite \( x_k\to x_0\) nous avons
    \begin{equation}
        \lim_{n\to \infty} F_k(x_n)=\lim_{n\to \infty} \int_{K_k}f(x_n,\omega)d\mu(\omega).
    \end{equation}
    Nous pouvons inverser la limite et l'intégrale en utilisant le théorème de la convergence dominée. Pour cela, la fonction \( f(x_n,\omega)\) étant continue sur le compact \( K_k\), elle y est majorée par une constante. Le fait que les compacts soient de mesure finie (hypothèse) implique que les constantes soient intégrales sur \( K_k\). Le théorème de la convergence dominée implique alors que
    \begin{equation}
        \lim_{n\to \infty} F_k(x_n)=\int_{K_k}\lim_{n\to \infty} f(x_n,\omega)d\mu(\omega)=\int_{K_k}f(x_0,\omega)d\mu(\omega)=F_k(x_0).
    \end{equation}
    Nous avons utilisé le fait que \( f(.,\omega)\) était continue en \( x_0\).


    
    Le lemme \ref{LemOgQdpJ} nous indique alors que la convergence \( F_k\to F\) est uniforme. Les fonctions \( F_k\) étant continues, la fonction \( F\) est continue.
\end{proof}


%---------------------------------------------------------------------------------------------------------------------------
\subsection{Dérivation, intégration}
%---------------------------------------------------------------------------------------------------------------------------

\begin{theorem}[\cite{TrenchRealAnalisys}]      \label{ThoCciOlZ}
    Supposons que \( \sum_{n=0}^{\infty}f_n\) converge uniformément vers \( F\) sur \( \mathopen[ a , b \mathclose]\). Si \( F\) et \( f_n\) sont des fonctions intégrables sur \( \mathopen[ a , b \mathclose]\) alors
    \begin{equation}
        \int_a^bF(x)dx=\sum_{n=0}^{\infty}\int_a^bf_n(x)dx.
    \end{equation}
\end{theorem}

\begin{theorem} \label{ThoCSGaPY}
    Soit \( f_n\) des fonctions \( C^1\mathopen[ a , b \mathclose]\) telles que
    \begin{enumerate}
        \item
            la série \( \sum_n f_n(x_0)\) converge pour un certain \( x_0\in\mathopen[ a , b \mathclose]\),
        \item
            la série des dérivées \( \sum_n f'_n\) converge uniformément sur \( \mathopen[ a , b \mathclose]\).
    \end{enumerate}
    Alors la série \( \sum_n f_n\) converge vers une fonction \( F\) et
    \begin{enumerate}
        \item
            La convergence est uniforme sur \( \mathopen[ a , b \mathclose]\).
        \item
            La fonction \( F\) est dérivable
        \item
            \( F'(x)=\sum_nf'_n(x)\).
    \end{enumerate}
\end{theorem}


