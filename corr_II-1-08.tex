% This is part of the Exercices et corrigés de CdI-2.
% Copyright (C) 2008, 2009
%   Laurent Claessens
% See the file fdl-1.3.txt for copying conditions.


\begin{corrige}{_II-1-08}

Afin de mieux suivre les notations de la théorie (Bernoulli, page \pageref{SubSecBernh}) nous allons écrire $\beta$ au lieu de $\alpha$. Nous pouvons directement régler son compte au cas $\beta=0$. En effet, nous trouvons
\begin{equation}
	\frac{ dr }{ r }=\left( \tan(\theta)+\frac{ k }{ \cos(\theta) } \right)d\theta,
\end{equation}
d'où nous tirons $r(\theta)$ moyennant une simple primitive.

Si $\beta\neq 0$, nous trouvons l'équation
\begin{equation}
	r'=r\tan(\theta)+\frac{ k }{ \cos(\theta) }r^{\beta+1}
\end{equation}
qui est de la forme \eqref{EqBerNDiffalp} avec
\begin{equation}
	\begin{aligned}[]
		a(\theta)	&=\tan(\theta)\\
		b(\theta)	&=k/\cos(\theta)\\
		\alpha		&=\beta+1.
	\end{aligned}
\end{equation}
En suivant la méthode générale, poser $z=r^{-\beta}$ fournit l'équation linéaire
\begin{equation}		\label{EqII108LinPourz}
	z'=-\beta\tan(\theta) z+\frac{ k }{ \cos(\theta) }.
\end{equation}
L'équation homogène associée, $z'_H+\beta\tan(\theta)z_H=0$ (qui est à variable séparées), a pour solution
\begin{equation}
	z_H=K\cos^{\beta}(\theta).
\end{equation}
Nous utilisons maintenant la méthode de variations des constantes, c'est à dire que nous posons $z(\theta)=K(\theta)z_H(\theta)$. En remettant dans l'équation \ref{EqII108LinPourz}, et en effectuant la simplification qui se présente, nous trouvons l'équation suivante pour $K$ :
\begin{equation}
	K'=-\frac{ \beta k }{ \cos^{\beta+1}(\theta) }.
\end{equation}
Nous avons donc
\begin{equation}
	K(\theta)=-\beta k\int_0^{\theta}\cos^{-(\beta+1)}(t)dt,
\end{equation}
qui n'est pas une intégrale facile à calculer.

Lorsque $\beta=1$, les choses sont plus simples parce que nous savons que 
\begin{equation}
	\int\frac{dx}{ \cos^2(x) }=\tan(x).
\end{equation}
Donc $K(\theta)=-\beta k\tan(\theta)+K$, et en refaisant le changement de variable vers $r$, nous trouvons la solution
\begin{equation}
	r(\theta)=\frac{1}{ K\cos(\theta)-k\sin(\theta) }.
\end{equation}
Ceci est une équation polaire pour une courbe que nous devons identifier. Nous la récrivons sous la forme
\begin{equation}
	1=Kr\cos(\theta)-kr\sin(\theta),
\end{equation}
et nous identifions les coordonnés cartésiennes $x=r\cos(\theta)$ et $y=r\sin(\theta)$, donc
\begin{equation}
	1=Kx-ky,
\end{equation}
qui est l'équation d'une droite.

\end{corrige}
