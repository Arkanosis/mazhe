% This is part of Mes notes de mathématique
% Copyright (c) 2011-2013
%   Laurent Claessens
% See the file fdl-1.3.txt for copying conditions.

%+++++++++++++++++++++++++++++++++++++++++++++++++++++++++++++++++++++++++++++++++++++++++++++++++++++++++++++++++++++++++++
\section*{Originalité}
%+++++++++++++++++++++++++++++++++++++++++++++++++++++++++++++++++++++++++++++++++++++++++++++++++++++++++++++++++++++++++++

Ces notes ne sont pas originales par leur contenu : ce sont toutes des choses qu'on trouve facilement sur internet; je crois que la bibliographie est éloquente à ce sujet. Ce cours se distingue des autres sur trois points.
\begin{description}
    \item[La longueur] J'ai décidé de ne pas me soucier de la taille du fichier. Il fera cinq mille pages si il le faut, mais il restera en un bloc. Étant donné qu'il n'existe qu'une seule mathématique, il ne m'a pas semblé intéressant de produire une division artificielle entre l'analyse, la géométrie ou l'algèbre. Tous le résultats d'une branche peuvent (et sont) être utilisés dans toutes les autres branches.

        Dans cette optique, je me suis évertué à ne créer que des références «vers le haut». À moins d'oubli de ma part\footnote{Par exemple pour les théorèmes pour lesquels je n'ai pas lu ni a fortiori écrit de preuves.}, il n'y a aucun endroit du texte qui dépend d'un lemme démontré plus bas. Le fait qu'un théorème \( B\) soit plus bas qu'un théorème \( A\) signifie qu'on peut démontrer \( A\) sans savoir \( B\).

    \item[La licence] Ce document est publié sous une licence libre. Elle vous donne explicitement le droit de copier, modifier et redistribuer. Je me doute bien que la majorité des professeurs qui mettent leurs notes en ligne ne seront pas fâchés de les voir utilisées. Il n'empèche que, par défaut, la loi dit que l'auteur conserve tous ses droits. De plus vous ne savez pas si l'auteur accepterait de voir le pdf sur votre site, de voir du copier-coller vers un autre document, \ldots bref un document laissé sans licence c'est moralement pas clair; et c'est légalement parfaitement clair : vous ne pouvez rien n'en faire. 

        Ici pas de problèmes : la licence vous dit tout ce que vous pouvez faire et pas faire, sous quelles conditions. En respectant les termes de la licence, vous avez à la fois la sécurité juridique et l'assurance morale de ne pas abuser.
    \item[L'ISBN] Une fois par an, une version de ce document sera affublée d'un ISBN. Pourquoi ? Parce qu'en avoir un est le sésame qui permet d'entrer dans la bibliothèque de l'agrégation\footnote{\url{http://agreg.org/Pratique/bibliotheque.html}}. Quatre exemplaires de la version 2012\footnote{\url{http://student.ulb.ac.be/~lclaesse/mes_notes-2012.pdf}} sont disponibles dans la bibliothèque de l'agrégation à Paris.

        Si vous passez l'agrégation en 2015 ou plus tard, vérifiez si une nouvelle version ne serait pas disponible.
        
\end{description}

%+++++++++++++++++++++++++++++++++++++++++++++++++++++++++++++++++++++++++++++++++++++++++++++++++++++++++++++++++++++++++++
%\section{Contre Moodle, Icampus, Claroline, et autres «plateformes de travail collaboratif»}
%+++++++++++++++++++++++++++++++++++++++++++++++++++++++++++++++++++++++++++++++++++++++++++++++++++++++++++++++++++++++++++

%Ces notes ne sont pas destinées à être publiées sur des plateformes telles que Moodle, Icampus ou autres Clarolines. Pourquoi ? parce que la licence FDL l'interdit implicitement en demandant de publier sur des sites \emph{ouverts}.

%L'internet est un système décentralisé et ouvert : tout le monde peut s'y connecter, y publier et y lire. C'est pour l'instant la meilleure solution technique inventée par l'humanité pour la diffusion d'information. Des sites comme \href{http://gitorious.org}{gitorious} ou \href{http://wikipedia.org}{wikipedia} sont de \emph{vrais} système de travail collaboratif.

%Les plateformes soi-disant collaboratives comme Moodle en sont la négation. L'essentiel de ce qu'apporte Moodle par rapport à un vrai site internet n'est absolument pas la possibilité de partager des information (ça on peut le faire via internet depuis des décennies), mais bien de \emph{restreindre} l'accès à l'information via un système de mot de passe.

%Lorsqu'un moine copiste du onzième siècle mettait un manuscrit dans sa bibliothèque, le document était immédiatement consultable par une centaine de moines, et (quitte à faire le déplacement) par des milliers d'érudits. Un document posté sur Moodle touche une dizaine de personnes. Utiliser Moodle pour partager ses documents est donc une régression non pas par rapport à l'internet d'il y a vingt ans, non pas par rapport à l'imprimerie d'il y a cinq siècles, mais bien par rapport aux bibliothèques d'abbayes d'il y a mille ans !

%Lorsqu'on parle de science, qu'on veut y apporter un document, une question ou une réponse, un minimum d'honnête intellectuelle, d'éthique du partage de savoir (sans laquelle la science n'existe pas) et peut être aussi de courage, est de parler publiquement. Se cacher derrière un mot de passe et ne permettre l'accès au savoir qu'à ses seuls amis triés sur le volet est une négation de l'esprit scientifique; Moodle est une version dégénérée, une maladie de l'internet.

%La faute fondamentale qui fait utiliser Moodle pour partager des documents de mathématique est la perte de notion entre le privé et le public ainsi que la paresse qui consiste à vouloir intégrer tous les outils dans une même interface, voire utiliser les mêmes outils pour effectuer des tâches différentes. Lorsqu'on pose une question de math, c'est essentiellement public; lorsqu'on pose une question d'organisation d'un cours, c'est privé. Ce sont deux activités totalement différentes qui nécessitent deux types d'outils différents. Dans le premier cas, l'outil adapté est internet, dans le deuxième cas, l'outil adapté est Moodle. Vouloir utiliser la même interface pour les deux est n'avoir fondamentalement pas compris le sens de l'internet et son utilité en tant que «outil de l'information».

%Une autre faute d'utilisation usuelle d'utilisation des plates-formes de «travail collaboratif» est d'oublier de mettre une licence libre sur les documents postés. En effet, le droit nous indique que si l'auteur ne précise rien il conserve tout ses droits. Les autres utilisateurs de Moodle n'ont donc légalement ni le droit de modifier ni le droit de redistribuer les textes postés sans licences particulières.

%{\tiny Cela dit c'est pratique pour discuter des horaires des cours ou s'échanger des informations pratiques qui n'ont pas à être publiques.}

%Et enfin, un quiproquo usuel est de croire que ces plates-formes aident à avoir «mes documents où je veux, quand je veux, toujours disponibles parce qu'ils sont en ligne». Or il n'en est rien. En effet pour y accéder vous devez donner votre mot de passe, mais vous ne devriez pas donner votre mot de passe à un ordinateur en qui vous n'avez pas confiance (je parle bien de l'ordinateur, et non de son propriétaire). Vous n'avez donc certainement pas accès à vos documents sur les ordinateurs publics, et très modérément sur ceux de vos amis. Bref, vous n'avez de toutes façons le «droit» de vous connecter à votre compte en ligne que sur votre propre ordinateur. En gros, le nuage n'est pas plus mobile que votre propre ordinateur et ne présente donc pas d'intérêt quant à la disponibilité des documents «partout».

