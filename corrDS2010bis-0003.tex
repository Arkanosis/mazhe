% This is part of Exercices de mathématique pour SVT
% Copyright (C) 2010
%   Laurent Claessens et Carlotta Donadello
% See the file fdl-1.3.txt for copying conditions.

\begin{corrige}{DS2010bis-0003}

	Pour les limites des fonction rationnelles (c'est à dire: des fractions de polynômes), il y a deux techniques à ne pas confondre. La première est la factorisation et simplification; la seconde est la mise en évidence du plus haut degré au numérateur et au dénominateur.

	La mise en évidence se fait lorsqu'on a des limites en $\pm\infty$. La factorisation se fait pour les limites en des nombres.
	\begin{enumerate}
		\item
			Ici il faut factoriser le dénominateur et simplifier.
			\begin{equation}
			\lim_{x\to 3} \frac{(x-3)(x+15)}{x^2-9}	= \lim_{x\to 3} \frac{(x-3)(x+15)}{(x-3)(x+3)}=\lim_{x\to 3} \frac{x+15}{x+3}=\frac{18}{6}=3.
			\end{equation}
		\item
			Ici, il faut mettre en évidence la plus haute puissance de $x$ au numérateur et au dénominateur.
			\begin{equation}
				\begin{aligned}[]
				\lim_{x\to -\infty}\frac{x^{50}-3200x^{5}+1}{5x^{17}-3x+4}&= \\
                                &=\lim_{x\to -\infty}\frac{x^{50}\left(1-\frac{3200}{x^{45}}+\frac{1}{x^{50}}\right)}{x^{17}\left(5-\frac{3}{x^{16}}+\frac{4}{x^{17}}\right)}=\\
                                  &=\lim_{x\to -\infty}\frac{x^{33}\left(1-\frac{3200}{x^{45}}+\frac{1}{x^{50}}\right)}{\left(5-\frac{3}{x^{16}}+\frac{4}{x^{17}}\right)}=\\
                                    &=-\infty.
				\end{aligned}
			\end{equation}
		\item
			On connaît la limite remarquable $\lim_{x\to 0} \frac{ \sin(x) }{ x }=1$. 
		\item
			Après le changement de variable on trouve la même limite que au point précédent.
			\begin{equation}
                            \lim_{x\to +\infty} x \sin (\frac{1}{x})=\lim_{y\to 0} \frac{ \sin(y) }{ y }=1.
			\end{equation}
		\item
			Ici l'astuce est de multiplier et diviser par le «binôme conjugué» du numérateur, à savoir
			\begin{equation}
                          \sqrt{1-x^2}+\sqrt{1+x^2}.
			\end{equation}
                        par le produit remarquable $(\spadesuit-\clubsuit)(\spadesuit+\clubsuit)=\spadesuit^2-\clubsuit^2$, cela fait apparaître au numérateur le produit
			\begin{equation}
                          (\sqrt{1-x^2}-\sqrt{1+x^2})(\sqrt{1-x^2}+\sqrt{1+x^2})=1-x^2-1-x^2=-2x^2.
			\end{equation}
			 La limite à calculer est donc
			 \begin{equation}
				 \begin{aligned}[]
					\lim_{x\to 0}\frac{\sqrt{1-x^2}-\sqrt{1+x^2}}{5x}&=\lim_{x\to 0}\frac{-2x^2}{5x(\sqrt{1-x^2}+\sqrt{1+x^2})}\\
					&=\lim_{x\to 0}\frac{-2x}{5(\sqrt{1-x^2}+\sqrt{1+x^2})}\\
					&=\frac{0}{10}\\
					&=0.
				 \end{aligned}
			 \end{equation}

		\item
                  La fonction dont on veut calculer la limite peut s'écrire comme 
                  \begin{equation}
                    f(x)=x^{-3}\ln(x)=\frac{\ln(x)}{x^3}.
                  \end{equation}
                  Nous utilisons le fait que «la croissance du logarithme est lente». Donc le logarithme du numérateur tend moins vite vers l'infini que la puissance de $x$ qui est au dénominateur. Nous avons donc
			\begin{equation}
				\lim_{x\to \infty} \frac{ \ln(x) }{ x^{3} }=0.
			\end{equation}
	\end{enumerate}

\end{corrige}
