% This is part of Exercices et corrigés de CdI-1
% Copyright (c) 2011
%   Laurent Claessens
% See the file fdl-1.3.txt for copying conditions.

\section{Séries}
\label{secseries}

\subsection{Rappels et définitions}
La notion de série formalise le concept de somme infinie. L'absence de
certaines propriétés de ces objets (problèmes de commutativité et même
d'associativité) incitent à la prudence et montrent à quel point une
définition précise est importante.

Soit ${(a_k)}_{k \geq k_0}$ une suite réelle. La \Defn{série de terme
  général $(a_k)$}, notée
\begin{math}
  \sum_{i=k_0}^\infty a_i,
\end{math}
est la suite ${(s_n)}_{n \geq k_0}$ dont les termes sont donnés par
\begin{equation*}
  s_n \pardef \sum_{i=k_0}^k a_i
\end{equation*}
et sont appelés les \Defn{sommes partielles} de la série.

La série $\sum_{i=k_0}^\infty a_i$ \Defn{converge} si la suite $(s_n)$
converge vers un réel $s$. Sa limite est appelée la \Defn{somme de la
  série} et on note
\begin{equation}		\label{EqDefSommeLim}
  \sum_{i=k_0}^\infty a_i = \lim_{n\to\infty}\sum_{i=k_0}^na_i.
\end{equation}
Si la série ne converge pas, elle \Defn{diverge} et peut alors avoir
une limite infinie (uniquement si le terme général est réel, on note
alors $+\infty$ ou $-\infty$ sa limite) ou pas de limite.  La série
$\sum_{i=k_0}^\infty a_i$ \Defn{converge absolument} si la série
$\sum_{i=k_0}^\infty \abs{a_i}$ converge.

\begin{proposition}\label{propnseries_propdebase}
Les principales propriétés de la somme définie par la limite \eqref{EqDefSommeLim} sont
  \begin{enumerate}
  \item Si une série converge absolument, alors elle converge
    (simplement).
  \item Si la série est à termes positifs --c'est-à-dire pour tout
    indice $k$, $a_k \in \eR$ et $a_k \geq 0$-- il n'y a aucune
    différence entre convergence absolue et convergence simple.
  \item\label{point3-seriepropdebase} Si une série converge, son terme général doit tendre vers $0$.
\item 
Si la série converge alors la somme est associative
\item
Si la série converge absolument, alors la somme est commutative.
  \end{enumerate}
\end{proposition}

\begin{remark}Vue comme somme infinie, l'associativité et la
  commutativité dans une série sont perdues. Néanmoins, il subsiste
  que
  \begin{enumerate}
  \item si la série converge, on peut regrouper ses termes sans
    modifier la convergence ni la somme (associativité),
  \item si la série converge absolument, on peut modifier l'ordre des
    termes sans modifier la convergence ni la somme (commutativité).
  \end{enumerate}
\end{remark}

\begin{example}\label{exemplesseries}
\begin{enumerate}

\item
La \Defn{série harmonique} est
\begin{equation*}
\sum_{i=1}^\infty \frac1i
\end{equation*}
et diverge (possède une limite $+\infty$).

\item
La \Defn{série géométrique de raison $q \in \CC$} est
\begin{equation*}
\sum_{i=0}^\infty q^i
\end{equation*}
et converge absolument si $\abs{q} < 1$ et diverge si $\abs{q}
\geq 1$. De plus, pour tout naturel $n$,
\begin{equation*}
\sum_{i=0}^n q^i =  \frac{1-q^{n+1}}{1-q} \quad\text{et}\quad \sum_{i=0}^\infty q^i = \frac{1}{1-q}.
\end{equation*}
La seconde égalité est vraie si et seulement si $| q |<1$.

\item
Pour $\alpha \in \RR$, la série de Riemann (ou Dirichlet)
\begin{equation}		\label{EqSerRiem}
\sum_{i=1}^\infty \frac1{i^\alpha}
\end{equation}
converge (absolument, puisque réelle et positive) si et seulement
si $\alpha > 1$, et diverge sinon.
\end{enumerate}
\end{example}

\subsection{Critères de convergence}
\begin{remark}
  Étant donné le terme général d'une série, il est souvent --dans les cas qui nous intéressent-- difficile de déterminer la somme de la série. L'exemple de la série géométrique est particulier, puisqu'on connait une formule pour chaque somme partielle, mais pour l'exemple des séries de Riemann il n'y a aucune formule simple pour un $\alpha$ général. D'où l'intérêt d'avoir des critères de convergence ne nécessitant aucune connaissance de l'éventuelle limite de la série.
\end{remark}

\subsubsection{Critère de comparaison} 

Soient $\sum_i a_i$ et $\sum_j
b_j$ deux séries à termes positifs vérifiant
\begin{equation*}
  0 \leq a_i \leq b_i
\end{equation*}
alors
\begin{enumerate}
\item si $\sum_i a_i$ diverge, alors $\sum_j b_j$ diverge,
\item si $\sum_j b_j$ converge, alors $\sum_i a_i$ converge
  (absolument).
\end{enumerate}

\subsubsection{Critère d'équivalence}
\label{PgCritEquiv}

 Soient $\sum_i a_i$ et $\sum_j
b_j$ deux séries à termes positifs. Supposons l'existence de la limite
(éventuellement infinie) suivante
\begin{equation*}
  \limite i \infty \frac{a_i}{b_i} = \alpha \in \RR \text{ ou $\alpha =
    \infty$.}
\end{equation*}
Dans ce cas, nous avons
\begin{enumerate}
\item si $\alpha \neq 0$ et $\alpha\neq \infty$, alors
  \begin{equation*}
    \sum_i a_i \text{~converge} \ssi \sum_j b_j\text{~converge,}
  \end{equation*}
\item si $\alpha = 0$ et $\sum_j b_j$ converge, alors $\sum_i a_i$
  converge (absolument),
\item si $\alpha = +\infty$ et $\sum_j b_j$ diverge, alors $\sum_i
  a_i$ diverge.
\end{enumerate}

Une preuve est donnée à la page \pageref{PgPreuveCritEquiv}.

\subsubsection{Critère du quotient} Soit $\sum_i a_i$ une
série. Supposons l'existence de la limite (éventuellement infinie)
suivante
\begin{equation*}
  \limite i \infty \abs{\frac{a_{i+1}}{a_i}} = L\in \RR \text{ ou $L =
    \infty$.}
\end{equation*}
Alors
\begin{enumerate}
\item si $L < 1$, la série converge absolument,
\item si $L > 1$, la série diverge,
\item si $L = 1$ le critère échoue : il existe des exemple de convergence et des exemples de divergence.
\end{enumerate}

Une preuve est donnée à la page \pageref{PgPreuveCritQuotient}.

\subsubsection{Critère de la racine} Soit $\sum_i a_i$ une série, et
considérons
\begin{equation*}
  \limsup_{i \rightarrow \infty} \sqrt[i]{\abs{a_i}} = L \in \RR
  \text{ ou $L =
    \infty$.}
\end{equation*}
Alors
\begin{enumerate}
\item si $L < 1$, la série converge absolument,
\item si $L> 1$, la série diverge,
\item si $L = 1$ le critère échoue.
\end{enumerate}

Une preuve est donnée à la page \pageref{PgPreuveCritRacine}.

Les critères de comparaison, d'équivalence, du quotient et de la racine sont des critères de convergence absolue. Pour conclure à une convergence simple qui n'est pas une convergence absolue, le critère d'Abel sera notre outil principal.  

\subsubsection{Critère d'Abel}


\begin{proposition}[Proposition 5, page 122]

Soit la série $\sum_i c_iz_i$ avec
\begin{enumerate}
\item $(c_i)$ est une suite réelle décroissante qui tend vers zéro,
\item $(z_i)$ est une suite dans $\eC$ dont la suite des sommes partielles est bornée dans $\eC$, c'est à dire qu'il existe un $M>0$ tel que pour tout $n$,
\begin{equation}
	\left| \sum_{i=1}^nz_i \right| \leq M.
\end{equation}
Alors la série $\sum_ic_iz_i$ est convergente.

\end{enumerate}
\end{proposition}
Remarquons que ce critère ne donne pas de convergence absolue.


\begin{example}
  Le \Defn{critère des séries alternées} est un cas particulier de ce
  critère : si $c_i$ est une suite décroissante à limite nulle, alors
  la série
  \begin{equation*}
    \sum_{i=0}^\infty {(-1)}^i c_i
  \end{equation*}
  converge simplement.
\end{example}

%---------------------------------------------------------------------------------------------------------------------------
					\subsection{Série de puissances}
%---------------------------------------------------------------------------------------------------------------------------

Une \defe{série de puissance}{Série!de puissance} est une série de la forme
\begin{equation}		\label{eqseriepuissance}
	\sum_{k=0}^{\infty}c_k(z-z_0)^k
\end{equation}
où $z_0\in \eC$ est fixé, $(c_k)$ est une suite complexe fixée, et $z$ est un paramètre complexe. Nous disons que cette série est \emph{centrée} en $z_0$.

\begin{theorem}		\label{ThoSerPuissRap}
Considérons la série de puissances donnée par le terme général $c_k(z-z_0)^k$. Si nous posons
\begin{equation}		\label{EqRayCOnvSer}
	\alpha=\frac{1}{ R } =\limsup\sqrt[k]{| c_k |}
\end{equation}
alors la série converge absolument si $| z-z_0 |<R$ et diverge si $| z-z_0 |>R$.

De plus, ce rayon de convergence peut être calculé par la formule alternative 
\begin{equation}		\label{EqAlphaSerPuissAtern}
	\alpha=\frac{1}{ R }=\limite k \infty \abs{\frac{c_{k+1}}{c_k}}
\end{equation}
lorsque $c_k$ est non nul à partir d'un certain $k$.
\end{theorem}
Le disque $| z-z_0 |\leq R$ est le \defe{disque de convergence}{Disque de convergence} de la série \eqref{eqseriepuissance}. Notez que ce théorème ne dit rien pour les points tels que $| z-z_0 |=R$. Il faut traiter ces points au cas par cas. Et le pire, c'est qu'une série donnée peut converger pour certain des points sur le bord du disque, et diverger en d'autres.  Il y a un dessin à la figure \ref{LabelFigDisqueConv}.
\newcommand{\CaptionFigDisqueConv}{À l'intérieur du disque de convergence, la convergence est absolue. En dehors, la série diverge. Sur le cercle proprement dit, tout peut arriver.}
\input{Fig_DisqueConv.pstricks}

Le disque de centre $z_0$ et de rayon $R$ est appelé \Defn{disque de convergence}. Pour un complexe $z$ sur le bord de ce disque, c'est-à-dire tel que $\abs{z-z_0} = R$, le comportement peut-être très varié (convergence absolue, convergence simple ou divergence) et n'est éventuellement pas le même sur tout le bord.

L'étude de ce qu'il se passe sur le bord du disque de convergence commence par y étudier la convergence absolue, c'est à dire étudier la série
\begin{equation}
	\sum_k| c_k(z-z_0)^k |=\sum_k| c_k |R^k
\end{equation}
parce que sur le bord, $| z-z_0 |=R$. L'étude du terme général $| c_k |R^k$ a deux utilités :
\begin{enumerate}
\item Si la somme $\sum_{k}| c_k |R^k$ converge, alors la série converge uniformément sur le bord,
\item si la suite $| c_k |R^k$ ne tend pas vers zéro, alors la série ne converge même pas simplement sur le bord.
\end{enumerate}


\section{Continuité et dérivabilité}

\label{seccontetderiv}
\subsection{Rappels et outils}
On considère dans la suite une fonction $f : A \to \eR$, où $a \in A \subset \eR$ ; cependant, les notions de continuité et de dérivabilité se généralisent immédiatement au cas de fonctions à valeurs vectorielles ; la notion de continuité se généralise au cas des fonctions à plusieurs variables (la notion de dérivabilité est remplacée par celle de différentiabilité dans ce cadre).

\begin{defn}La fonction $f$ est \Defn{continue en $a$} si
  \begin{equation*}
    \limite x a f(x)
  \end{equation*}
  existe (et vaut alors $f(a)$).
\end{defn}

\begin{defn}La fonction $f$ est \Defn{dérivable en $a$} si $a \in
  \operatorname{int} A$ et si
  \begin{equation*}
    \lim_{\substack{x\rightarrow a\\x\neq a}} \frac{f(x)-f(a)}{x-a}
  \end{equation*}
  existe. On note alors cette quantité $f^\prime(a)$, c'est le nombre
  dérivé de $f$ en $a$. La \Defn{fonction dérivée} de $f$ est
  \begin{equation*}
    f^\prime : A^\prime \to \eR : a \mapsto f^\prime(a)
  \end{equation*}
  définie sur l'ensemble noté $A^\prime$ des points $a$ où $f$ est
  dérivable.
\end{defn}

\begin{defn}
  Une fonction est \Defn{continue} (resp. \Defn{dérivable}) si elle
  est continue (resp. dérivable) en tout point $a \in A$ de son
  domaine.
\end{defn}

\begin{example}
  Montrons que la fonction $f : \eR \to \eR : x\mapsto x$ est
  continue et dérivable.
  \begin{description}
  \item[Continuité] Soit $a \in \eR$, et montrons
    $\limite x a x = a$. En effet, soit $\epsilon > 0$, et choisissons
    $\delta = \epsilon$. Si $\abs{x-a} < \delta$ alors
    \begin{equation*}
      \abs{f(x) - f(a)} = \abs{x-a} < \delta = \epsilon
    \end{equation*}
    ce qui prouve la continuité.

  \item[Dérivabilité] Soit $a \in \eR$. Calculons la
    limite du quotient différentiel
    \begin{equation*}
      \limite[x\neq a]{x}{a} \frac{x-a}{x-a} = \limite[x\neq a]x a 1 = 1
    \end{equation*}
    ce qui prouve que $f$ est dérivable et que sa dérivée vaut $1$ en
    tout point $a$ de $\eR$.
  \end{description}

% Exceptionellement (bien qu'on sache que la dérivabilité implique la
% continuité), montrons ces deux assertions séparément.

% Il faut d'abord montrer que pour tout $a \in \eR$,
% \begin{equation*}
%   \limite x a x = a
% \end{equation*}
% c'est-à-dire
% \begin{equation*}
%   \forall \epsilon > 0, \exists \delta > 0 :  \forall x \in \eR \abs{x-a} <
%   \delta \Rightarrow \abs{x-a} < \epsilon
% \end{equation*}
% ce qui est clair en prenant $\delta = \epsilon$.

% Il faut ensuite montrer que pour tout $a \in \eR$,
% \begin{equation*}
%   \limite[x\neq a] x a \frac{x-a}{x-a} \qquad\text{ existe.}
% \end{equation*}
% ce qui, là encore, est clair, puisque la fonction $\frac{x-a}{x-a} =
% 1$ pour tout $x \neq a$, et il s'agit donc de la limite de la
% fonction constante $1$, qui vaut évidemment $1$.

% On a donc montré que la fonction $x \mapsto x$ est continue,
% dérivable, et que sa dérivée vaut $1$ en tout point $a$ de son
% domaine.

\end{example}


%+++++++++++++++++++++++++++++++++++++++++++++++++++++++++++++++++++++++++++++++++++++++++++++++++++++++++++++++++++++++++++
					\section{Développements de Taylor et Maclaurin}
%+++++++++++++++++++++++++++++++++++++++++++++++++++++++++++++++++++++++++++++++++++++++++++++++++++++++++++++++++++++++++++

Nous voulons formaliser l'idée d'une fonction qui tend vers zéro \og plus vite\fg{} qu'une autre. Nous disons que $f\in o\big(\varphi(x)\big)$ si
\begin{equation}
	\lim_{x\to 0} \frac{ f(x) }{ \varphi(x) }=0.
\end{equation}
En particulier, nous disons que $f\in o(x)$ lorsque $\lim_{x\to 0} f(x)/x=0$.

La proposition suivante se trouve à la page 216.
\begin{proposition}		\label{PropDevTaylorPol}
Soient $f\colon I\subset\eR\to \eR$ et $a\in\Int(I)$. Soit un entier $k\geq 1$. Si $f$ est $k$ fois dérivable en $a$, alors il existe un et un seul polynôme $P$ de degré $\leq k$ tel que
\begin{equation}
	f(x)-P(x-a)\in o\big( | x-a |^k \big)
\end{equation}
lorsque $x\to a$, $x\neq a$. Ce polynôme  est donné par
\begin{equation}
	P(h)=f(a)+f'(a)h+\frac{ f''(a) }{ 2! }h^2+\ldots+\frac{ f^{(k)}(a) }{ k! }h^k.
\end{equation}
\end{proposition}

La proposition suivant, de la page 218, donne une intéressante façon de trouver le reste d'un développement de Taylor.
\begin{proposition}		\label{PropResteTaylorc}
Soient $I$, un intervalle dans $\eR$ et $f\colon I\to \eR$ une fonction de classe $C^k$ sur $I$ telle que $f^{(k+1)}$ existe sur $I$. Soient $a\in\Int(I)$ et $x\in I$. Alors il existe $c$ strictement compris entre $x$ et $a$ tel que 
\begin{equation}
	R_{f,a,k}(x)=\frac{ f^{(k+1)}(c) }{ (k+1)! }(x-a)^{k+1}.
\end{equation}
\end{proposition}

%---------------------------------------------------------------------------------------------------------------------------
					\subsection{Exemple : un calcul heuristique de limite}
%---------------------------------------------------------------------------------------------------------------------------
\label{SubSecCalcLimHeuris}

Au cours de la résolution de l'exercice \ref{exoEqsDiff0002}\ref{ItemfEqsDiff00002}, nous devons calculer la limite suivante :
\begin{equation}
	\lim_{x\to 0} \frac{  e^{-2\cos(x)+2}\sin(x) }{ \sqrt{ e^{2\cos(x)+2}}-1 }.
\end{equation}
La stratégie que nous allons suivre pour calculer cette limite est de développer certaines parties de l'expression en série de Taylor, afin de simplifier l'expression. La première chose à faire est de remplacer $ e^{y(x)}$ par $1+y(x)$ lorsque $y(x)\to 0$. La limite devient
\begin{equation}
	\lim_{x\to 0} \frac{ \big( -2\cos(x)+3 \big)\sin(x) }{ \sqrt{-2\cos(x)+2} }.
\end{equation}
Nous allons maintenant remplacer $\cos(x)$ par $1$ au numérateur et par $1-x^2/2$ au dénominateur. Pourquoi ? Parce que le cosinus du dénominateur est dans une racine, donc nous nous attendons à ce que le terme de degré deux du cosinus donne un degré un en dehors de la racine, alors que du degré un est exactement ce que nous avons au numérateur : le développement du sinus commence par $x$.

Nous calculons donc
\begin{equation}
	\begin{aligned}[]
		\lim_{x\to 0} \frac{ \sin(x) }{ \sqrt{-2\left( 1-\frac{ x^2 }{ 2 } \right)+2} }=\lim_{x\to 0} \frac{ \sin(x) }{ x }=1.
	\end{aligned}
\end{equation}
Tout ceci n'est évidement pas très rigoureux, mais en principe vous avez tous les éléments en main pour justifier les étapes.
%+++++++++++++++++++++++++++++++++++++++++++++++++++++++++++++++++++++++++++++++++++++++++++++++++++++++++++++++++++++++++++
					\section{Équations différentielles}
%+++++++++++++++++++++++++++++++++++++++++++++++++++++++++++++++++++++++++++++++++++++++++++++++++++++++++++++++++++++++++++

%---------------------------------------------------------------------------------------------------------------------------
					\subsection{Équations à variables séparées}
%---------------------------------------------------------------------------------------------------------------------------

Une \defe{équation à variables séparées}{Équation!différentielle!à variables séparées} est une équation  de la forme
\begin{equation}		\label{EqDiffSeparee}
	y'=u(t)f(y).
\end{equation}
Les deux résultats qui résolvent ce cas sont donnés aux pages 311 et 313 du cours de première année.
\begin{proposition}
Nous considérons l'équation \eqref{EqDiffSeparee} avec $u(t)$ continue sur $I$ et $f$ continue sur $J$ avec $f(\eta)\neq 0$ pour tout $\eta\in J$. Soit $U$, une primitive de $u$ sur $I$, et $G$, une primitive de $1/f$ sur $J$.

Si $y\colon Y'\to J$ est une fonction sur un intervalle $I'\subset I$, alors $y$ est solution de l'équation \eqref{EqDiffSeparee} si et seulement si il existe $C\in\eR$ tel que
\begin{equation}		\label{EqSoluceEqDiffSep}
	G\big( y(t) \big)=U(t)+C.
\end{equation}
\end{proposition}
Cette proposition dit que toutes les solutions qui ne s'annulent jamais sur un intervalle ont la forme $G\big( y(t) \big)=U(t)+C$ et peuvent donc être trouvées en calculant des primitives.

La formule \eqref{EqSoluceEqDiffSep} peut être obtenue de la façon heuristique suivante, en écrivant $y'=dy/dt$, et en passant le $dt$ à droite. Nous trouvons successivement
\begin{equation}
	\begin{aligned}[]
		y'&=u(t)f(y)\\
		dy&=u(t)f(y)dt\\
		\frac{ dy }{ f(y) }&=u(t)dt\\
		\int\frac{ dy }{ f(y) }&=\int u(t)dt\\
		G(y)&=U(t)+C.
	\end{aligned}
\end{equation}

\begin{proposition}
Soient $u$ continue sur $I$ et $f$ continue sur $J$, et $f(\eta)\neq 0$ sur $J$. Soient $t_0\in I$ et $y_0\in J$. Alors il existe $I'\subset I$ avec $t_0\in I'$ et $f\in C^1(I'\to J)$ tels que
\begin{enumerate}

\item
$y$ est solution de \eqref{EqDiffSeparee} sur $I'$ et vérifie $y(t_0)=y_0$,
\item
si $z$ est une solution de \eqref{EqDiffSeparee} sur $I''\subset I'$ avec $t_0\in I''$ et $z(t_0)=y_0$, alors $I''\subset I'$ et $z(t)=y(t)$ pour tout $t\in I''$.

\end{enumerate}
\end{proposition}

%---------------------------------------------------------------------------------------------------------------------------
					\subsection{Équations linéaires}
%---------------------------------------------------------------------------------------------------------------------------

Une \defe{équation différentielle linéaire}{Équation différentielle!linéaire} est une équation de la forme
\begin{equation}
	y'+u(t)y=v(t).
\end{equation}
La technique pour résoudre cette équation est de commencer par résoudre l'équation homogène associée. Si $U(t)$ est une primitive de $u(t)$, nous avons
\begin{equation}
	\begin{aligned}[]
		y'_H(t)+u(t)y_H(t)&=0\\
		\frac{ y'_H }{ y_H }&=-u(t)\\
		\ln(y_H)&=-U(t)+C\\
		y_H(t)&= e^{-U(t)+C}=K e^{-U(t)}
	\end{aligned}
\end{equation}
où $K= e^{C}$.

Cela fournit la solution générale de l'équation homogène. Il existe un truc génial qui permet d'en tirer la solution générale du système non homogène. Lorsque nous avons trouvé $y_H(t)=K e^{-U(t)}$, le symbole $K$ désigne une constante. La méthode de \defe{variation des constantes}{variation des constantes} consiste à essayer la solution
\begin{equation}		\label{EqEssayVarSctr}
	y(t)=K(t) e^{-U(t)},
\end{equation}
c'est à dire à dire que la constante est en réalité une fonction. Afin de trouver quelle fonction $K(t)$ fait en sorte que l'essai \eqref{EqEssayVarSctr} soit une solution, nous la remplaçons dans l'équation de départ $y'+uy=v$. Maintenant,
\begin{equation}
	y'(t)=K'(t) e^{-U(t)}-K(t)u(t) e^{-U(t)}.
\end{equation}
En remettant dans l'équation,
\begin{equation}
	y'+uy=K' e^{-U}-Ku e^{-U}+uK e^{-U}=K' e^{-U}=v.
\end{equation}
Notez que les termes en $K$ se sont miraculeusement simplifiés. Cela est directement dû au fait que $ e^{-U}$ est solution de l'équation homogène. Nous restons avec l'équation
\begin{equation}
	K'=\frac{ v }{  e^{-U} }
\end{equation}
pour $K(t)$. La solution générale du problème non homogène est donc finalement donnée par
\begin{equation}
	y(t)=\big( W(t)+C \big) e^{-U(t)}
\end{equation}
si $W(t)$ est une primitive de $v(t)e^{U(t)}$.

Tout ceci est un peu heuristique. La proposition suivante (page 308) dit dans quels cas ça fonctionne.
\begin{proposition}
Soient $u$ et $v$ continues sur $I$ et $U$, une primitive de $u$ sur $I$ et $W$ une primitive de $v e^{-U}$ sur $I$. Une fonction $y\colon I\to \eR$ est solution de $y'+u(t)y=v(t)$ si et seulement si il existe une constante $C\in \eR$ telle que
\begin{equation}
	y(t)=\big( W(t)+C \big) e^{U(t)}
\end{equation}
pour tout $t\in I$.
\end{proposition}
Le corollaire de la page $310$ lui règle son compte au problème de Cauchy. Si $t_0\in I$ et si $y_0\in\eR$, alors il existe une unique solution $y$ sur $I$ à l'équation linéaire telle que $y(t_0)=y_0$.

%+++++++++++++++++++++++++++++++++++++++++++++++++++++++++++++++++++++++++++++++++++++++++++++++++++++++++++++++++++++++++++
					\section{Topologie en général}
%+++++++++++++++++++++++++++++++++++++++++++++++++++++++++++++++++++++++++++++++++++++++++++++++++++++++++++++++++++++++++++

Cette section sur la topologie en général peut être sautée dans un premier temps. Celles et ceux qui ont dans l'idée de faire de la mathématique plus avancée dans les années à venir devraient toutefois y jeter un œil.

\begin{definition}		\label{DefTopologieGene}

Soit $E$, un ensemble et $\mT$, une partie de l'ensemble de ses parties qui vérifie les propriétés suivantes
\begin{enumerate}

\item
les ensembles $\emptyset$ et $E$ sont dans $\mT$,

\item
Si $I$ est n'importe quel ensemble et si pour tout $i\in I$, nous avons un élément $\mO_i\in\mT$, alors $\cup_{i\in I}\mO_i\in\mT$,

\item
Si $J$ est un ensemble fini et si pour tout $j\in J$, nous avons un élément $\mO_j\in\mT$, alors $\cap_{j\in J}\mO_j\in\mT$.

\end{enumerate}
Les deux dernières propriétés s'énoncent en disant que toute réunions d'éléments de $\mT$ est un élément de $\mT$ et que toute intersection \emph{finie} d'éléments de $\mT$ est un élément de $\mT$.

Un tel choix $\mT$ de sous-ensembles de $E$ est une  \defe{\href{http://fr.wikipedia.org/wiki/Espace_topologique}{topologie}}{Topologie} sur $E$, et les éléments de $\mT$ sont appelés des \defe{ouverts}{Ouvert}. Nous disons que un sous ensemble $A$ de $E$ est \defe{fermé}{Fermé} si son complémentaire, $A^c$ est ouvert.
\end{definition}

Dès que nous avons une topologie, nous avons une notion de convergence de suite : nous disons qu'une suite $x_n$ d'éléments de $E$ \defe{converge}{Convergence!en topologie} vers l'élément $x$ de $E$ si pour tout ouvert $\mO$ contenant $x$, il existe un $K$ tel que $k>K$ implique $x_k\in\mO$. Cette définition est exactement celle donnée pour la convergence de suites dans $\eR^n$, à part que nous avons remplacé le mot \og boule\fg{} par \og ouvert\fg.

Dans un espace topologique, nous avons une caractérisation très importante des ouverts.
\begin{theorem}		\label{ThoPartieOUvpartouv}
Une partie $A$ de $E$ est ouverte si et seulement si pour tout $x\in A$, il existe un ouvert autour de $x$ contenu dans $A$.
\end{theorem}

\begin{proof}
Le sens direct est évident : $A$ lui-même est un ouvert autour de $x\in A$, qui est inclus à $A$.

Pour le sens inverse, pour chaque $x\in A$, nous considérons l'ensemble $\mO_x\subset A$, un ouvert autour de $x$. Nous avons que
\begin{equation}	\label{EqAUniondesOx}
	A=\bigcup_{x\in A}\mO_x.
\end{equation}
En effet $A\subset\cup_{x\in A}\mO_x$ parce que tous les éléments de $A$ sont dans un des $\mO_x$, par construction. D'autre part, $\cup_{x\in A}\mO_x\subset A$ parce que chacun des $\mO_x$ est compris dans $A$.

L'union du membre de droite de \eqref{EqAUniondesOx} est une union d'ouverts et est donc un ouvert. Cela prouve que $A$ est un ouvert.

\end{proof}

Une utilisation typique de ce théorème est faite à l'exercice \ref{exo0083}.

%%%%%%%%%%%%%%%%%%%%%%%%%%
%
   \section{Topologie dans \texorpdfstring{$\eR^n$}{Rn}}
%
%%%%%%%%%%%%%%%%%%%%%%%%



Dans cette section, nous travaillons dans l'espace $\eR^n$ pour un certain naturel $n$. Nous y définissons la notion d'ouvert et de fermé, qui sont la base de la topologie générale. Notons que ces définitions n'ont de sens que relativement à l'espace ambiant, aussi un ouvert de $\eR$ ne sera en général pas un ouvert de $\eR^2$~: d'une part, il n'y a pas d'inclusion canonique de $\eR$ dans $\eR^2$ (les ouverts du second ne sont même pas des sous-ensembles du premier) et, d'autre part, les définitions se basent sur la notion de boule de $\eR^n$ qui dépend évidemment de la valeur de $n$ (une boule dans $\eR$ est un intervalle, dans $\eR^2$ c'est un disque, etc.)

%---------------------------------------------------------------------------------------------------------------------------
					\subsection{Ouverts et fermés}
%---------------------------------------------------------------------------------------------------------------------------

\begin{definition}
	La \defe{boule ouverte}{Boule!ouverte} de centre $x_0 \in \eR^n$ et de rayon $r \in
	\eR^+$ est définie par
	\begin{equation}
		B(x_0,r) = \{ x \in \eR^n \tq \norme{x - x_0} < r \},
	\end{equation}
	tandis que la \defe{boule fermée}{Boule!fermée} de centre $x_0$ et de rayon $r$ est
	\begin{equation}
		\bar B(x_0,r) = \{ x \in \eR^n \tq \norme{x - x_0} \leq r \};
	\end{equation}
	la différence est que l'inégalité dans la première est stricte.
\end{definition}

%---------------------------------------------------------------------------------------------------------------------------
					\subsection{Intérieur, adhérence et frontière}
%---------------------------------------------------------------------------------------------------------------------------

\begin{definition}
  Soit $A \subset \eR^n$ et $x \in \eR^n$. Le point $x$ est \defe{intérieur}{Intérieur} à $A$ si il existe une boule autour de $x$ complètement contenue dans $A$. L'ensemble des points intérieurs à $A$ est noté $\interieur A$ ou $\mathring A$, de sorte qu'on a précisément
  \begin{equation*}
    x \in \interieur A \iffdefn  \exists \epsilon > 0 \tq
    B(x,\epsilon) \subset A.
  \end{equation*}
\end{definition}


\begin{definition}
Le point $x$ est dans l'\defe{adhérence}{Adhérence} de $A$ si toute boule autour de $x$ intersecte $A$. L'ensemble de ces points est noté $\adh A$ ou $\bar A$, et on a donc de manière plus précise
\begin{equation}
	x \in \adh A \iffdefn \forall \epsilon > 0, B(x,\epsilon) \cap A \neq \emptyset
\end{equation}
\end{definition}

\begin{proposition}
Pour $A \subset \eR^n$, nous avons
\begin{equation*}
	\interieur A \subseteq A  \subseteq \adh A
\end{equation*}
\end{proposition}

\begin{definition}
  La \defe{frontière}{Frontière} ou le \defe{bord}{Bord} de $A$ est défini par $\partial A = \adh A \setminus \interieur A$. L'ensemble $A$ est un \defe{ouvert}{Ouvert} si $A = \interieur A$, et c'est un \defe{fermé}{Fermé} si $A = \adh A$.
\end{definition}

On vérifiera que les notations et les dénominations sont cohérentes en
prouvant la proposition suivante.
\begin{proposition}Pour $\epsilon > 0$,
  \begin{enumerate}
  \item l'adhérence de $B(x,\epsilon)$ est $\bar B(x,\epsilon)$,
  \item l'intérieur de $\bar B(x,\epsilon)$ est $B(x,\epsilon)$,
  \item la boule ouverte $B(x,\epsilon)$ est un ouvert,
  \item la boule fermée $\bar B(x,\epsilon)$ est un fermé.
  \end{enumerate}
\end{proposition}

Nous avons également les liens suivants entre intérieur, adhérence,
ouvert, fermé et passage au complémentaire (noté ${}^c$)~:
\begin{proposition}
Si $A \subset \eR^n$ et $A^c = \eR^n\setminus A$, nous
  avons
  \begin{enumerate}
  \item $(\interieur A)^c = \adh (A^c)$ et $(\adh A)^c = \interieur
    (A^c)$,
  \item $A$ est ouvert si et seulement si $A^c$ est fermé,
  \item $\interieur A$ est le plus grand ouvert contenu dans $A$,
  \item $\adh A$ est le plus petit fermé contenant $A$,
    % \item
  \end{enumerate}
\end{proposition}

%---------------------------------------------------------------------------------------------------------------------------
					\subsection{Bornés et compacts}
%---------------------------------------------------------------------------------------------------------------------------


\begin{definition}
  Un sous ensemble $A \subset \eR^n$ est \defe{borné}{Borné} si il existe une boule de $\eR^n$ contenant $A$.
\end{definition}

\begin{proposition}
  Toute réunion finie d'ensembles bornés est un ensemble borné. Toute
  partie d'un ensemble borné est un ensemble borné.
\end{proposition}

\begin{definition}
  La partie $A \subset \eR^n$ est \defe{compacte}{Compact} si et seulement si, pour tout
  recouvrement de $A$ par des ouverts (c'est-à-dire une collection
  d'ouverts dont la réunion contient $A$) on peut tirer un
  recouvrement fini.
\end{definition}

% En particulier, si on recouvre $A$ par l'ensemble des boules
% $B(x,1)$ où $x$ parcourt $A$ (de sorte que tout point de $A$ est
% dans \og sa\fg{} boule, et donc la réunion des boules recouvre bien
% $A$), on doit pouvoir en tirer un recouvrement fini, c'est-à-dire
% des boules $B(x_1,1), B(x_2,1), \ldots, B(x_k,1)$ (avec $k$ un
% naturel) dont la réunion contient $A$.

\begin{proposition}
Une partie de $\eR^n$ est compacte si et seulement si elle est fermée et bornée.
\end{proposition}

%---------------------------------------------------------------------------------------------------------------------------
					\subsection{Connexité}
%---------------------------------------------------------------------------------------------------------------------------

\begin{definition}
  Le sous ensemble $A \subset \eR^n$ est \defe{connexe par arcs}{Connexe!par arc} si pour tout $x, y \in
  A$, il existe un chemin\footnote{Attention : ici quand on dit \emph{chemin}, on demande que l'application soit continue. Dans de nombreux cours de géométrie différentielle, on demande $ C^{\infty}$. Il faut s'adapter au contexte.} contenu dans $A$ les reliant, c'est-à-dire
  une application continue
  \begin{equation*}
    \gamma : [0,1] \to \eR^n \tq \gamma(0) = x~\text{et}~\gamma(1) = y
  \end{equation*}
  avec $\gamma(t) \in A$ pour tout $t\in [0,1]$.
\end{definition}

%+++++++++++++++++++++++++++++++++++++++++++++++++++++++++++++++++++++++++++++++++++++++++++++++++++++++++++++++++++++++++++
					\section{Topologie des espaces métriques}
%+++++++++++++++++++++++++++++++++++++++++++++++++++++++++++++++++++++++++++++++++++++++++++++++++++++++++++++++++++++++++++

Si $E$ est un ensemble, une \defe{distance}{Distance} sur $E$ est une application $d\colon E\times E\to \eR$ telle que pour tout $x,y\in E$,
\begin{enumerate}

\item
$d(x,y)\geq 0$

\item
$d(x,y)=0$ si et seulement si $x=y$,

\item
$d(x,y)=d(y,x)$

\item
$d(x,y)\leq d(x,z)+d(z,y)$.

\end{enumerate}
La dernière condition est l'\defe{inégalité triangulaire}{Inégalité!triangulaire}. Le couple $(E,d)$ d'un ensemble et d'une métrique est un \defe{espace métrique}{Espace!métrique}.

Dès que l'ensemble $E$ est muni d'une distance, nous définissons une topologie en disant que les boules
\begin{equation}
	B(x,r)=\{ y\in E\tq d(x,y)<r \}
\end{equation}
sont ouvertes.


%+++++++++++++++++++++++++++++++++++++++++++++++++++++++++++++++++++++++++++++++++++++++++++++++++++++++++++++++++++++++++++
					\section{Uniforme continuité}
%+++++++++++++++++++++++++++++++++++++++++++++++++++++++++++++++++++++++++++++++++++++++++++++++++++++++++++++++++++++++++++

Nous avons la proposition de la page $181$.
\begin{proposition}	\label{PropoInvCompCont}
Soit $f\colon A\subset\eR^n\to B\subset\eR^m$ une bijection continue. Si $A$ est compact, alors $f^{-1}\colon B\to A$ est continue.
\end{proposition}

La proposition suivante est la proposition 3, à la page $170$.
\begin{proposition}		\label{PropIntContMOnIvCont}
Soient $I$ un intervalle dans $\eR$ et $f\colon I\to \eR$ une fonction continue strictement monotone. Alors la fonction réciproque $f^{-1}\colon f(I)\to \eR$ est continue sur l'intervalle $f(I)$.
\end{proposition}



%+++++++++++++++++++++++++++++++++++++++++++++++++++++++++++++++++++++++++++++++++++++++++++++++++++++++++++++++++++++++++++
					\section{Maximisation sans contraintes}
%+++++++++++++++++++++++++++++++++++++++++++++++++++++++++++++++++++++++++++++++++++++++++++++++++++++++++++++++++++++++++++

%---------------------------------------------------------------------------------------------------------------------------
					\subsection{Fonctions à une seule variable}
%---------------------------------------------------------------------------------------------------------------------------

\begin{definition}
Soit $f\colon A\subset \eR\to \eR$ et $a\in A$. Le point $a$ est un \defe{maximum local}{Maximum!local} de $f$ si il existe un voisinage $\mU$ de $a$ tel que $f(a)\geq f(x)$ pour tout $x\in\mU\cap A$. Le point $a$ est un \defe{maximum global}{Maximum!global} si $f(a)\geq g(x)$ pour tout $x\in A$.
\end{definition}

La proposition basique à utiliser lors de la recherche d'extrema est la suivante :
\begin{proposition}
Soit $f\colon A\subset\eR\to \eR$ et $a\in\Int(A)$. Supposons que $f$ est dérivable en $a$. Si $a$ est un \href{http://fr.wikipedia.org/wiki/Extremum}{extremum} local, alors $f'(a)=0$.
\end{proposition}

La réciproque n'est pas vraie, comme le montre l'exemple de la fonction $x\mapsto x^3$ en $x=0$ : sa dérivée est nulle et pourtant $x=0$ n'est ni un maximum ni un minimum local. 

Cette proposition ne sert donc qu'à sélectionner des \emph{candidats} extremum. Afin de savoir si ces candidats sont des extrema, il y a la proposition suivante (proposition 1, page 227).

\begin{proposition}
Soit $f\colon I\subset \eR\to \eR$, une fonction de classe $C^k$ au voisinage d'un point $a\in\Int I$. Supposons que
\begin{equation}
	f'(a)=f''(a)=\ldots=f^{(k-1)}(a)=0,
\end{equation}
et que
\begin{equation}
	f^{(k)}(a)\neq 0.
\end{equation}
Dans ce cas,
\begin{enumerate}

\item
Si $k$ est pair, alors $a$ est un point d'extremum local de $f$, c'est un minimum si $f^{(k)}(a)>0$, et un maximum si $f^{(k)}(a)<0$,
\item
Si $k$ est impair, alors $a$ n'est pas un extremum local de $f$.

\end{enumerate}
\end{proposition}

Note : jusqu'à présent nous n'avons rien dit des extrema \emph{globaux} de $f$. Il n'y a pas grand chose à en dire. Si un point d'extremum global est situé dans l'intérieur du domaine de $f$, alors il sera extremum local (a fortiori). Ou alors, le maximum global peut être sur le bord du domaine. C'est ce qui arrive à des fonctions strictement croissantes sur un domaine compact.

Une seule certitude : si une fonction est continue sur un compact, elle possède une minimum et un maximum global.

%---------------------------------------------------------------------------------------------------------------------------
					\subsection{Fonctions à plusieurs variables}
%---------------------------------------------------------------------------------------------------------------------------


%---------------------------------------------------------------------------------------------------------------------------
					\subsection{Quelque mots à propos de matrices}
%---------------------------------------------------------------------------------------------------------------------------

Les notions qui suivent seront vues au cours de géométrie lorsqu'il sera temps. 

Si $g$ est une application bilinéaire sur $\eR^2$, nous disons qu'elle est
\begin{enumerate}

\item
\defe{Définie positive}{Application!définie positive} si $g(u,u)\geq 0$ pour tout $u\in\eR^2$ et $g(u,u)=0$ si et seulement si $u=0$.

\item
\defe{semi-définie positive}{Application!semi-définie positive} si $g(u,u)\geq 0$ pour tout $u\in\eR^2$. 

\end{enumerate}

Une matrice $M$ est définie positive si $v^tMv>0$ pour tout $v\neq 0$, en particulier si $v$ est un vecteur propre de valeur propre $\lambda$ (c'est à dire si $Mv=\lambda v$), alors $\lambda v^tv>0$, et donc $\lambda>0$. Donc $M$ sera définie positive si toutes ses valeurs propres sont positives.

\begin{proposition}
	Soit $M$, une matrice $2\times 2$ symétrique\footnote{la matrice $d^2f(a)$ est toujours symétrique quand $f$ est de classe $C^2$.}. Nous avons
	\begin{enumerate}
		\item
		$\det M>0$ et $\tr(M)>0$ implique $M$ définie positive,
		\item
		$\det M>0$ et $\tr(M)<0$ implique $M$ définie négative,
		\item
		$\det M<0$ implique ni semi définie positive, ni définie négative (et donc pas un extrema dans le cas où $M=d^2f(a)$ par le point \ref{ItemPropoExtreRn} de la proposition \ref{PropoExtreRn}),
		\item
		$\det M=0$ implique $M$ semi-définie positive ou semi-définie négative.
	\end{enumerate}
\end{proposition}
 

%---------------------------------------------------------------------------------------------------------------------------
					\subsection{Les théorèmes}
%---------------------------------------------------------------------------------------------------------------------------



Un point $a$ à l'intérieur du domaine d'une fonction $f\colon A\subset\eR^n\to \eR$ est une \defe{point critique}{Critique!point} de $f$ lorsque $df(a)=0$. Ces points sont analogues aux points où la dérivée d'une fonction sur $\eR$ s'annule. Les points critiques de $f$ sont dons les candidats à être des points d'extremum.

\begin{proposition}		\label{PropoExtreRn}
Soit $f\colon A\subset\eR^n\to \eR$ une fonction de classe $C^2$ au voisinage de $a\in\Int(A)$.
\begin{enumerate}

\item
Si $a$ est un point critique de $f$, et si $d^2f(a)$ est \href{http://fr.wikipedia.org/wiki/Matrice_définie_positive}{définie positive}, alors $a$ est un minimum local strict de $f$,
\item		\label{ItemPropoExtreRn}
Si $a$ est un minimum local, alors $a$ est un point critique et $d^2f(a)$ est semi-définie positive.

\end{enumerate}
\end{proposition}

La seconde partie de l'énoncé est tout à fait comparable au fait bien connu que, pour une fonction $f\colon \eR\to \eR$, si le point $a$ est minimum local, alors $f'(a)=0$ et $f''(a)\geq 0$. Notez le fait que l'inégalité n'est pas stricte, ce qui correspond à $d^2f(a)$ \emph{semi}-definie positive.

Pour rappel, dans le cas d'une fonction à deux variables, $d^2f(a)$ est la matrice (et donc l'application linéaire)
\begin{equation}
	d^2f(a)=\begin{pmatrix}
	\frac{ d^2f  }{ dx^2 }(a)	&	\frac{ d^2f  }{ dx\,dy }(a)	\\ 
	\frac{ d^2f  }{ dy\,dx }(a) 	&	\frac{ d^2f  }{ dy^2 }(a)
\end{pmatrix}.
\end{equation}
Dans le cas d'une fonction $C2$, cette matrice est symétrique.

La méthode pour chercher les extrema de $f$ est donc de suivre le points suivants :
\begin{enumerate}

\item
Trouver les candidats extrema en résolvant $\nabla f=(0,0)$,
\item
écrire $d^2f(a)$ pour chacun des candidats
\item
calculer les valeurs propres de $d^2f(a)$, déterminer si la matrice est définie positive ou négative,
\item
conclure.

\end{enumerate}


%+++++++++++++++++++++++++++++++++++++++++++++++++++++++++++++++++++++++++++++++++++++++++++++++++++++++++++++++++++++++++++
					\section{Limites à plusieurs variables}
%+++++++++++++++++++++++++++++++++++++++++++++++++++++++++++++++++++++++++++++++++++++++++++++++++++++++++++++++++++++++++++
		\label{SecLimVarsPlus}

Prenons une fonction $f\colon \eR^n\to \eR$. Nous disons que
\begin{equation}
	\lim_{x\to x_0}f(x)=l\in\eR
\end{equation}
lorsque $\forall \epsilon>0$, $\exists\delta$ tel que $\| x-x_0 \|\leq\delta$ implique $| f(x)-l |\leq \epsilon$. 

Remarquez qu'ici, $x\in\eR^n$, et sachez distinguer $\| . \|$, la norme dans $\eR^n$ de $| . |$ qui est la valeur absolue dans $\eR$. Une autre façon d'exprimer cette définition est que l'ensemble des valeurs atteintes par $f$ dans une boule de rayon $\delta$ autour de $x_0$ n'est pas très loin de $l$. Nous définissons donc
\begin{equation}
	E_{\delta}=\{ f(x)\tq x\in B(x_0,\delta) \}.
\end{equation}
Notez que si $f$ n'est pas définie en $x_0$, il n'y a pas de valeurs correspondantes au centre de la boule dans $E_{\delta}$. Ceci est évidement la situation générique lorsqu'il y a une indétermination à lever dans le calcul de la limite. Nous avons alors que
\begin{equation}
	\lim_{x\to x_0}f(x)=l
\end{equation}
lorsque $\forall\epsilon>0$, $\exists\delta$ tel que 
\begin{equation}		\label{Eqvmoinsrapplimdeux}
	\sup\{ | v-l |\tq v\in E_{\delta} \}\leq\epsilon.
\end{equation}
Une façon classique de montrer qu'une limite n'existe pas, est de prouver que, pour tout $\delta$, l'ensemble $E_{\delta}$ contient deux valeurs constantes. Si par exemple $0\in E_{\delta}$ et $1\in E_{\delta}$ pour tout $\delta$, alors aucune valeur de $l$ (même pas $l=\pm\infty$) ne peut satisfaire à la condition \eqref{Eqvmoinsrapplimdeux} pour toute valeur de $\epsilon$.

Nous laissons à la sagacité de l'étudiant le soin d'adapter tout ceci pour le cas $\lim_{x\to x_0}f(x)=\pm\infty$.

La proposition suivante semble évidente, mais nous sera tellement
utile qu'il est préférable de l'expliciter~:
\begin{proposition}
Soit $f : D \to \eR$ une fonction dont le domaine
  s'écrit comme une réunion \emph{finie}
  \begin{equation*}
    D = \bigcup_{i=1}^k A_i
  \end{equation*}  
  où $k$ est un entier. Soit $a \in \adh D$ tel que $a \in \adh A_i$
  pour tout $i \leq k$, et soit $b \in \eR$. Alors, la limite
  \begin{equation*}
    \limite x a f(x)
  \end{equation*}
  existe et vaut $b$ si et seulement si chacune des limites
  \begin{equation*}
    \limite[x \in A_i] x a f(x)
  \end{equation*}
  existe et vaut $b$.
\end{proposition}

\begin{proof}On sait déjà que si la limite de $f : D \to \eR$
  existe, alors toute restriction à $A_i$ admet la même limite. Il
  suffit donc de prouver la réciproque.

  Par hypothèse, pour tout $i = 1 \ldots k$, nous savons que
  \begin{equation*}
    \forall \epsilon > 0\, \exists \delta_i > 0 \tq (x \in A_i)
    \text{ et }
    (\norme{x-a} < \delta_i) \Rightarrow \norme{f(x) - b} < \epsilon
  \end{equation*}

  Si $\epsilon$ est fixé, posons $\delta = \min_i\{\delta_i\}$. Nous
  savons alors que
  \begin{enumerate}
  \item pour tout $x \in D$, il existe $i$ tel que $x \in A_i$, et
  \item si $x$ vérifie $\norme{x-a} < \delta$, alors pour tout $i$,
    $\norme{x-a} < \delta_i$ par définition de $\delta$.
  \end{enumerate}
  
  On en déduit que si $x \in D$ et $\norme{x-a} < \delta$, alors il
  existe $i$ tel que $x \in A_i$ et $\norme{x-a} < \delta_i$, ce qui
  implique $\abs{f(x) - b} < \epsilon$ et prouve la continuité.
\end{proof}

\begin{example}
  \begin{enumerate}
  \item Pour qu'une fonction $f : \eR \to \eR$ admette une limite en
    $a \in \eR$, il faut et il suffit qu'elle y admette une limite à
    droite et une limite à gauche qui soient égales.

  \item Une suite $(x_k)$ admet une limite si et seulement si les
    sous suites $(x_{2k})$ et $(x_{2k+1})$ convergent vers la même
    limite. Ceci n'est pas une application directe de la proposition,
    mais la teneur est la même.
  \end{enumerate}
\end{example}
%+++++++++++++++++++++++++++++++++++++++++++++++++++++++++++++++++++++++++++++++++++++++++++++++++++++++++++++++++++++++++++
					\section{Différentiabilité}
%+++++++++++++++++++++++++++++++++++++++++++++++++++++++++++++++++++++++++++++++++++++++++++++++++++++++++++++++++++++++++++


%---------------------------------------------------------------------------------------------------------------------------
					\subsection{Le pourquoi et le comment de la dérivée}
%---------------------------------------------------------------------------------------------------------------------------

La notion de dérivée est associée à la recherche de la droite tangente à une courbe. Reprenons rapidement le cheminement. La dérivée de $f\colon \eR\to \eR$ au point $a$ est un nombre $f'(a)$, qui définit donc une application linéaire dont le coefficients angulaire est $f'(a)$, et que nous notons $df_a$ :
\begin{equation}
	\begin{aligned}
		df_a\colon \eR&\to \eR \\
		u&\mapsto f'(a)u. 
	\end{aligned}
\end{equation}
La droite donnée par l'équation
\begin{equation}
	y(a+u)=f'(a)u
\end{equation}
est parallèle à la tangente en $a$. Pour trouver la tangente, il suffit de la décaler de la hauteur qu'il faut. L'équation de la droite tangente au graphe de $f$ au point $\big( a,f(a) \big)$ devient
\begin{equation}		\label{EqDiffRapTgDer}
	y(x)=f(a)+f'(a)(x-a)=f(a)+df_a(x-a).
\end{equation}
Nous nous proposons de généraliser cette formule au cas de la recherche du plan tangent à une surface.
 
%---------------------------------------------------------------------------------------------------------------------------
					\subsection{Dérivée partielle et directionnelles}
%---------------------------------------------------------------------------------------------------------------------------

Soit une fonction $f:A\subset \mathbb{R}^n \rightarrow \mathbb{R}^m$. Si $n\neq 1$, la notion de \emph{dérivée} de la fonction $f$ n'a plus de sens puisqu'on ne peut plus parler de pente de \emph{la} tangente au graphe de $f$ en un point. On introduit alors quelque notions qui feront, en dimension quelconque, le même travail que la dérivée en dimension un : les dérivées directionnelles et la différentielle. Nous allons voir qu'en dimension un, la différentielle coïncide avec la dérivée.


\begin{definition} 
	Soit un point $a \in int\,A$ et un vecteur $u \in \mathbb{R}^n$ avec $\| u \| =1$. La dérivée de $f$ au point $a$ dans la direction $u$ est donnée par la limite suivante, si elle existe 
	\begin{equation}
		\frac{\partial f}{\partial u}(a) = \lim_{t\rightarrow 0}\frac{f(a+tu) - f(a)}{t}
	\end{equation}
\end{definition}

Géométriquement, il s'agit du taux de variation instantané de $f$ en $a$ dans la direction du vecteur $u$, c'est-à-dire de la pente de la tangente dans la direction du vecteur $u$ au graphe de $f$ au point $(a, f(a))$.

\begin{remark}
On peut reformuler la définition en écrivant $x = a + u$, on obtient~:
\begin{equation}
	\limite[u\neq 0]{u}{0} \frac{f(a+u)-f(a)-T(u)}{\norme{u}} = 0.
\end{equation}
\end{remark}

\begin{remark}
Pourquoi avons-nous posé la condition $\| u \|=1$ ? Le but de la dérivée directionnelle dans la direction $u$ est de savoir à quelle vitesse la fonction monte lorsque l'on se déplace en suivant la direction $u$. Cette information n'aura un caractère \og objectif\fg{} que si l'on avance à une vitesse donnée. En effet, si on se déplace deux fois plus vite, la fonction montera deux fois plus vite. Par convention, nous demandons donc d'avancer à vitesse $1$.
\end{remark}

\subsubsection*{Cas particulier où $n=2$:} $a = (a_1, a_2)$, $u =
(u_1,u_2)$ et
$$\frac{\partial f}{\partial u}(a_1, a_2) = \lim_{t\rightarrow
0}\frac{f(a_1+tu_1,a_2+tu_2) - f(a_1, a_2)}{t}$$

Un cas particulier des dérivées directionnelles est la dérivée partielle. Si nous considérons la base canonique $e_i$ de $\eR^n$, nous notons
\begin{equation}
	\frac{ \partial f }{ \partial x_i }=\frac{ \partial f }{ \partial e_i }.
\end{equation}
Dans le cas d'une fonction à deux variables, nous avons donc les deux dérivées partielles
\begin{equation}
	\begin{aligned}[]
		\frac{ \partial f }{ \partial x }(a)&&\text{et}&&\frac{ \partial f }{ \partial y }(a)
	\end{aligned}
\end{equation}
qui correspondent aux dérivées directionnelles dans les directions des axes. Ces deux nombres représentent de combien la fonction $f$ monte lorsqu'on part de $a$ en se déplaçant dans le sens des axes $X$ et $Y$.

%///////////////////////////////////////////////////////////////////////////////////////////////////////////////////////////
					\subsubsection{Quelque propriétés et notations}
%///////////////////////////////////////////////////////////////////////////////////////////////////////////////////////////

\begin{enumerate}
\item
 $\forall \alpha \in \mathbb{R}$,
si $v = \alpha\,u$, alors $\frac{\partial f}{\partial v}(a) =
\alpha\,\frac{\partial f}{\partial u}(a)$.
\item Si on prend $u=e_j$ le $j$ème vecteur de la base canonique de
$\mathbb{R}^n$, alors
$$\frac{\partial f}{\partial e_j}(a) = \frac{\partial f}{\partial
x_j}(a)$$ c'est-à-dire que la dérivée de $f$ au point $a$ dans la
direction $e_j$ est la dérivée partielle de $f$ par rapport à sa
$j$ème variable.

\item 
Une fonction peut être dérivable dans certaines directions
mais pas dans d'autres (rappelez vous que si la limite à droite est
différente de la limite à gauche, la limite n'existe pas). 

\item
Même si une fonction est dérivable en un point dans toutes les
directions, on n'est pas sûr qu'elle soit continue en ce point. La
dérivabilité directionnelle n'est donc pas une notion suffisante
pour assurer la continuité. C'est pourquoi on introduit le concept
de \emph{différentiabilité}. 
\end{enumerate}

%---------------------------------------------------------------------------------------------------------------------------
					\subsection{Différentielle}
%---------------------------------------------------------------------------------------------------------------------------

\begin{definition}		\label{DefDifferentiablFnRn}
Soit un point $a \in int\,A$. La fonction $f$ est \defe{différentiable}{Différentiable} au point $a$ si il existe une application linéaire $df_a\colon \eR^n\to \eR^m$ telle que 
\begin{equation}		\label{EqDefDiffableT}
	\lim_{x\to a} \frac{f(x) - f(a) - df_a (x-a)}{\|x-a\|}=0.
\end{equation}
\end{definition}

Si $f$ est différentiable en $a$, l'application $df_a$ est appelée la différentielle de $f$ en $a$. Voyons comment cette application linéaire agit sur les vecteurs de $\mathbb{R}^n$.

Le théorème suivant reprend pas principales propriétés d'une fonction différentiable.
\begin{theorem}		\label{ThoRapPropDiffSi}
Si $f$ est différentiable en $a\in\eR^n$, alors
\begin{enumerate}
\item $f$ est continue en $a$.

\item  Toute les dérivées directionnelles $\partial_uf(a)$ existent et nous avons l'égalité
\begin{equation}		\label{EqDiffPartRap}
	\begin{aligned}
		df_a\colon \eR^n&\to \eR^m \\
		u&\mapsto df_a(u)=\frac{ \partial f }{ \partial u }(a)=\sum_i \frac{ \partial f }{ \partial x_i }u^i,
	\end{aligned}
\end{equation}
si les $u^i$ sont les composantes de $u$ dans la base canonique de $\eR^n$.

La différentielle de $f$ en $a$ envoie donc un vecteur $u$ sur la dérivée directionnelle de $f$ au point $a$ dans la direction $u$. 

\item\label{ItemThoDiffSiLin} L'application $df_a$ est une application linéaire.
\end{enumerate}
\end{theorem}
Le point \ref{ItemThoDiffSiLin} est évidement contenu dans la définition de la différentielle, mais c'est bien de la remettre en toute lettre. En regard avec la formule \eqref{EqDiffPartRap}, elle dit que $\partial_uf(a)$ est linéaire par rapport à $u$.

\subsubsection*{Cas particuliers} \begin{description} \item $n=1$:
$f: \mathbb{R}\rightarrow \mathbb{R}$ est dérivable en $a$ si et
seulement si $f$ est différentiable en $a$ et
$$df_a:\mathbb{R}\rightarrow \mathbb{R}: x \mapsto df_a(x) =
f'(a)\,.\,x$$ \item $n=2$: $f$ est différentiable en $a =(a_1, a_2)$
si et seulement si
$$\lim_{(v_1,v_2)\rightarrow (0,0)} \frac{f(a_1+v_1, a_2+v_2) - f(a_1,a_2) - [ \frac{\partial f}{\partial x}(a)\,v_1+
\frac{\partial f}{\partial y}(a)\,v_2]}{\sqrt{v_1^2+v_2^2}} = 0
$$\end{description}\vspace{0.3cm}


Parmi les vecteurs $u \in \mathbb{R}^n$, un vecteur d'origine $(a,
f(a))$ se distingue des autres: le vecteur gradient de $f$ en $a$
donnant la direction de plus grande pente de $f$ en
$a$.\vspace{0.3cm}


\begin{definition}
La courbe de niveau de $f$ associée à a est donnée par
$$ S_a = f^{-1}\,(f(a)) = \{(x_1, \ldots, x_n)\in \mathbb{R}^n : f(x_1, \ldots,
x_n)=f(a) \}$$
\end{definition}

Nous avons maintenant en main les concepts utiles pour trouver l'équation du plan tangent à une surface.

%---------------------------------------------------------------------------------------------------------------------------
					\subsection{Gradient et recherche du plan tangent}
%---------------------------------------------------------------------------------------------------------------------------

De la même manière que la tangente à une courbe était la droite de coefficient angulaire donné par la dérivée, maintenant, le plan tangent à une surface est le plan dont les vecteurs directeurs sont les dérivées partielles :

La généralisation de l'équation \eqref{EqDiffRapTgDer} est 
\begin{equation}		\label{EqDefPlanTag}
	T_a(x)=f(a)+\sum_i\frac{ \partial f }{ \partial x_i }(a)(x-a)^i
\end{equation}

Nous introduisons aussi souvent l'opérateur différentiel abstrait \defe{nabla}{Nabla}, noté $\nabla$ et qui est donné par le vecteur
\begin{equation}
	\nabla=\left( \frac{ \partial  }{ \partial x_1 },\ldots,\frac{ \partial  }{ \partial x_n } \right).
\end{equation}
Les égalités suivantes sont juste des notations, sommes toutes logiques, liées à $\nabla$ :
\begin{equation}
	\nabla f=\left( \frac{ \partial f }{ \partial x_1 },\ldots,\frac{ \partial f }{ \partial x_n } \right),
\end{equation}
et
\begin{equation}		\label{EqDefGradient}
	\nabla f(a) = \left(\frac{\partial f}{\partial x_1}(a), \frac{\partial f}{\partial x_2}(a), \ldots, \frac{\partial f}{\partial x_n}(a)\right).
\end{equation}
Ce dernier est un élément de $\eR^n$ : chaque entrée est un nombre réel.

\begin{definition} 
Le vecteur gradient de $f$ au point $a$ est le vecteur donné par la formule \eqref{EqDefGradient}.
\end{definition}
La notation $\nabla$ permet d'écrire la différentielle sous forme un peu plus compacte. En effet, la formule \eqref{EqDiffPartRap} peut être notée
\begin{equation}
	df_a(u)=\scal{\nabla f(a)}{u}.
\end{equation}

En utilisant ce produit scalaire, l'équation \eqref{EqDefPlanTag} peut se récrire
\begin{equation}
	T_a(x)=f(a)+\sum_i\frac{ \partial f }{ \partial x_i }(a)(x-a)^i=f(a)+\scal{\nabla f(a)}{x-a}.
\end{equation}

Affin d'éviter les confusions, il est parfois souhaitable de bien mettre les parenthèses et noter $(\nabla f)(a)$ au lieu de $\nabla f(a)$.

\begin{proposition}
$\nabla f(a)\,\bot \,S_a$
\end{proposition}


\begin{equation}		\label{EqPlanTgSansNabla}
	z=f(a)+\sum_i\frac{ \partial f }{ \partial f }(a)(x-a)^i.
\end{equation}

\subsubsection*{Cas particulier où $n=2$:} 
Le plan $T_a$ avec $a=(a_1,a_2)$ a pour équation dans $\eR^3$:
\begin{equation}		\label{EqPlanTgEnDimDeux}
	z = f(a_1,a_2) + \frac{\partial f}{\partial x}(a_1,a_2)\,(x-a_1)+ \frac{\partial f}{\partial y}(a_1,a_2)\,(y-a_2).
\end{equation}




%---------------------------------------------------------------------------------------------------------------------------
					\subsection{Différentielle comme élément de l'espace dual}
%---------------------------------------------------------------------------------------------------------------------------


Si nous considérons la base canonique $\{ e_i \}_{i=1,\ldots,n}$ de $\eR^n$. À partir d'elle, nous considérons la \defe{base duale}{Base duale}. En termes pratiques, nous définissons $dx_i$ comme la forme sur $\eR^n$ qui à un vecteur $u$ fait correspondre sa composante $i$ :
\begin{equation}
	dx_i\begin{pmatrix}
	u^1	\\ 
	\vdots	\\ 
	u^n	
\end{pmatrix}=u^i.
\end{equation}
En termes savants, $dx_i$ est le dual de $e_i$. Si tu ne l'as pas encore compris, Jean Doyen va te le faire comprendre !


Maintenant, dans la formule \eqref{EqDiffPartRap}, nous pouvons remplacer $u^i$ par $dx_i(u)$, et écrire
\begin{equation}
	df_a(u)=\sum_i\frac{ \partial f }{ \partial x_i }(a)u^i=\sum_i\frac{ \partial f }{ \partial x_i }(a)dx_i(u).
\end{equation}
Ce qui arrive tout à droite est explicitement vu comme une forme sur $\eR$, dont les composantes dans la base duale sont les dérivées partielles de $f$ au point $a$, agissant sur $u$. En faisant un pas en arrière, nous omettons le $u$, et nous écrivons
\begin{equation}
	df_a=\sum_{i=1}^n\frac{ \partial f }{ \partial x_i }(a)dx^i
\end{equation}

Cette notation $dx_i$ pour la forme duale de $e_i$ est en réalité parfaitement logique parce que $dx^i$ est la différentielle de la projection
\begin{equation}
	\begin{aligned}
		x^i\colon \eR^n&\to \eR \\
		(x^1,\ldots,x^n)&\mapsto x^i. 
	\end{aligned}
\end{equation}
Je te laisse un peu méditer sur cette différentielle de la projection. L'important est que tu aies compris cela d'ici la fin de ta deuxième année.


%---------------------------------------------------------------------------------------------------------------------------
					\subsection{Prouver qu'un fonction n'est pas différentiable}
%---------------------------------------------------------------------------------------------------------------------------

Chacun des point du théorème \ref{ThoRapPropDiffSi} est en soi un critère pour montrer qu'une fonction n'est pas différentiable en un point.

%///////////////////////////////////////////////////////////////////////////////////////////////////////////////////////////
					\subsubsection{Continuité}
%///////////////////////////////////////////////////////////////////////////////////////////////////////////////////////////


Le premier critère à vérifier est donc la continuité. Si une fonction n'est pas continue en un point, alors elle n'y sera pas différentiable. Pour rappel, la continuité en $a$ se teste en vérifiant si $\lim_{x\to a}f(x)=f(a)$.

%///////////////////////////////////////////////////////////////////////////////////////////////////////////////////////////
					\subsubsection{Linéarité}
%///////////////////////////////////////////////////////////////////////////////////////////////////////////////////////////

Un second test est la linéarité de la dérivée directionnelle par rapport à la direction : l'application $u\mapsto\frac{ \partial f }{ \partial u }(a)$ doit être linéaire, sinon $df_a$ n'existe pas.
\begin{example}		\label{Exemple0046Diff}
Examinons la fonction
\begin{equation}
	\begin{aligned}
		f\colon \eR^2&\to \eR \\
		(x,y)&\mapsto \begin{cases}
	\frac{ xy^2 }{ x^2+y^4 }	&	\text{si $(x,y)\neq (0,0)$}\\
	0	&	 \text{sinon}.
\end{cases}
	\end{aligned}
\end{equation}
Prenons $u=(u_1,u_2)$ et calculons la dérivée de $f$ dans la direction de $u$ au point~$(0,0)$ :
\begin{equation}
	\begin{aligned}[]
		\frac{ \partial f }{ \partial u }(0,0)	
			&=\lim_{t\to 0}\frac{ f(tu_1,tu_2)-f(0,0) }{ t }\\
			&=\lim_{t\to 0}\frac{1}{ t }\left( \frac{ tu_1t^2u_2 }{ t^2u_1^2+t^4u_2^4 } \right)\\
			&=\lim_{t\to 0}\left( \frac{ u_1u_2^2 }{ u_1^2+t^2u_2^4 } \right)\\
			&=\begin{cases}
	\frac{ u_2^2 }{ u_1 }	&	\text{si $u_1\neq 0$}\\
	0	&	 \text{si $u_1=0$}.
\end{cases}
	\end{aligned}
\end{equation}
Cette application n'est pas linéaire par rapport à $u$. En effet, notons
\begin{equation}
	\begin{aligned}
		A\colon \eR^n&\to \eR \\
		u&\mapsto \frac{ \partial f }{ \partial u }(0,0), 
	\end{aligned}
\end{equation}
et vérifions que pour tout $u$ et $v$ dans $\eR^n$ et $\lambda\in\eR$, nous ayons $A(\lambda u)=\lambda A(u)$ et $A(u+v)=A(u)+A(v)$. Le premier fonctionne parce que
\begin{equation}
	A(\lambda u)=A(\lambda u_1,\lambda u_2)=\frac{ \lambda^2 u_2^2 }{ \lambda u_1 }=\lambda\frac{ u_2^2 }{ u_1 }=\lambda A(u).
\end{equation}
Mais nous avons par exemple
\begin{equation}
	A\big( (0,1)+(2,3) \big)=A(2,4)=\frac{ 16 }{ 2 }=8,
\end{equation}
tandis que
\begin{equation}
	A(0,1)+A(2,3)=0+\frac{ 9 }{ 2 }\neq 8.
\end{equation}
La fonction $f$ n'est donc pas différentiable en $(0,0)$, parce que la candidate différentielle, $df_{(0,0)}(u)=\frac{ \partial f }{ \partial u }(0,0)$, n'est même pas linéaire.

Nous verrons, dans l'exercice \ref{exo0046}, une autre manière de traiter cette fonction.
\end{example}

%///////////////////////////////////////////////////////////////////////////////////////////////////////////////////////////
					\subsubsection{Cohérence des dérivées partielles et directionnelle}
%///////////////////////////////////////////////////////////////////////////////////////////////////////////////////////////

Dans la pratique, nous pouvons calculer $\partial_uf(a)$ pour une direction $u$ générale, et puis en déduire $\partial_xf$ et $\partial_yf$ comme cas particuliers en posant $u=(1,0)$ et $u=(0,1)$. Une chose incroyable, mais pourtant possible est qu'il peut arriver que
\begin{equation}
	\frac{ \partial f }{ \partial u }(a)\neq \sum_i\frac{ \partial f }{ \partial x_i }(a)u^i.
\end{equation}
Ceci se produit lorsque $f$ n'est pas différentiable en $a$. En voici un exemple.

%///////////////////////////////////////////////////////////////////////////////////////////////////////////////////////////
					\subsubsection{Un candidat dans la définition (marche toujours)}
%///////////////////////////////////////////////////////////////////////////////////////////////////////////////////////////

Lorsqu'une fonction est donné, un candidat différentielle au point $(a_1,a_2)$ est souvent assez simple à trouver en un point :
\begin{equation}
	T(u_1,u_2)=\frac{ \partial f }{ \partial x }(a_1,a_2)u_1+\frac{ \partial f }{ \partial y }(a_1,a_2)u_2.
\end{equation}
L'application $T$ est la candidate différentielle en ce sens que si la différentielle existe, alors elle est égale à $T$. Ensuite, il faut vérifier si
\begin{equation}		\label{EqLimDefDiff}
	\lim_{(x,y)\to (a_1,a_2)} \frac{f(x,y) - f(a_1,a_2) - T\big( (x,y)-(a_1,a_2) \big)}{\| (x,y)-(a_1,a_2) \|}=0
\end{equation}
ou non. Si oui, alors la différentielle existe et $df_{(a,b)}(u)=T(u)$, sinon\footnote{y compris si la limite \eqref{EqLimDefDiff} n'existe même pas.}, la différentielle n'existe pas.

Attention : dans la ZAP, les dérivées partielles $\partial_xf$ et $\partial_yf$ ne peuvent en général pas être calculées en utilisant les règles de calcul (c'est bien pour ça que la ZAP est une zone à problèmes). Il faut d'office utiliser la définition
\begin{equation}
	\frac{ \partial f }{ \partial x }(a_1,a_2)=\lim_{t\to 0}\frac{ f(a_1+t,a_2)-f(a_1,a_2) }{ t },
\end{equation}
et la définition correspondante pour $\partial_yf$.


\subsubsection*{Conclusion}
Soient $f:A\subset \eR^n \rightarrow \eR^m$, et $a\in int\,A$. Si $f$ est différentiable en $a$, $$ (df_a (e_j))_i = d(f_i)_a(e_j) =\frac{\partial f_i}{\partial x_j}(a)= [Jac(f)_{|a}]_{ij}$$ et la matrice de l'application linéaire $df_a$ est la matrice jacobienne $m\times n$ de $f$ en $a$ notée $Jac(f)_{|a}$.


%---------------------------------------------------------------------------------------------------------------------------
					\subsection{Calcul de différentielles}
%---------------------------------------------------------------------------------------------------------------------------


\begin{remark}		\label{deriveepartielles}
  En pratique, ayant une formule pour la fonction $f$, on dérive --grâce aux règles usuelles de dérivation-- par rapport à la variable $x_i$ en considérant que les autres ($x_j$ avec $j \neq i$) sont des constantes.
\end{remark}

\begin{example}Pour $f(x,y) = xy + x^2$, les dérivées partielles
  s'écrivent
  \begin{equation*}
    \frac{\partial f}{\partial x} = y + 2x \quad\text{et}\quad \frac{\partial f}{\partial y} = x
  \end{equation*}
\end{example}


Des \emph{règles de calcul} sont d'application. En particulier, quand
ces opérations existent, les sommes, différences, produits, quotients
et compositions d'applications différentiables sont différentiables.

Toute application linéaire est différentiable, et sa différentielle en
tout point est égale à l'application elle-même. En particulier, les
\Defn{projections canoniques}, c'est-à-dire les applications du type
$(x,y,z) \mapsto y$, sont linéaires donc différentiables.

\begin{example}
Les cas suivants sont faciles :
  \begin{enumerate}
  \item En combinant les projections canoniques avec les règles de
    calculs, on obtient que toute fonction polynômiale à $n$ variables
    est différentiable comme application de $\eR^n$ dans $\eR$.

  \item Toute fonction rationnelle, du type $f(x) \pardef
    \frac{P(x)}{Q(x)}$ où $P$ et $Q$ sont des polynômes, est
    différentiable en tout point $a$ tel que $Q(a) \neq 0$.

  \item Pour une fonction d'une variable $f : D \subset \eR \to
    \eR$, le caractère différentiable et le caractère dérivable
    coïncident. De plus, on a
    \begin{equation*}
      d f_a(u) = f'(a) u.
    \end{equation*}
  \end{enumerate}
\end{example}

%---------------------------------------------------------------------------------------------------------------------------
					\subsection{Notes idéologiques sur le concept de plan tangent}
%---------------------------------------------------------------------------------------------------------------------------
\label{ssecConceptPlanTag}

La page 94 du syllabus dit des choses très intéressantes que tu n'auras sans doute pas comprises \ldots et que tu ne comprendra sans doute pas complètement avant d'avoir un vrai cours de géométrie différentielle. Notons $G$, le graphe d'une fonction $f$, c'est à dire
\begin{equation}
	G=\{ (x,y,z)\in\eR^3\tq z=f(x,y) \}.
\end{equation}
Première affirmation : si $\gamma\colon \eR\to G$ est une courbe telle que $\gamma(0)=\big( a,f(a) \big)$, alors $\gamma'(0)\in\eR^n$ est dans le plan tangent à $G$ au point $\big( a,f(a) \big)$.

Plus fort : tous les éléments du plan tangent sont de cette forme.

Le plan tangent à $G$ en un point $x\in G$ est donc constitué des vecteurs vitesse de tous les chemins qui passent par $x$.

Prenons maintenant $S$, une courbe de niveau de $G$, c'est à dire
\begin{equation}
	S=\{ (x,y)\in\eR^2\tq f(x,y)=C \}.
\end{equation}
Si nous prenons un chemin dans $G$ qui est, de plus, contraint à $S$, c'est à dire tel que $\gamma(t)\in S$, alors $\gamma'(0)$ sera tangent à $G$ (ça, on le savait déjà), mais en plus, $\gamma'(0)$ sera tangent à $S$, ce qui est logique.

La morale est que si vous prenez un chemin qui se ballade dans n'importe quoi, alors la dérivée du chemin sera un vecteur tangent à ce n'importe quoi.

En outre, si $\gamma(t)\in S$ et $\gamma(0)=a$, alors
\begin{equation}
	\scal{\nabla f(a)}{\gamma'(0)}=0,
\end{equation}
c'est à dire que le vecteur tangent à la courbe de niveau est perpendiculaire au gradient. Cela est intuitivement logique parce que la tangente à la courbe de niveau correspond à la direction de \emph{moins} grande pente.

%+++++++++++++++++++++++++++++++++++++++++++++++++++++++++++++++++++++++++++++++++++++++++++++++++++++++++++++++++++++++++++
					\section{Jacobienne et calcul de différentielles}
%+++++++++++++++++++++++++++++++++++++++++++++++++++++++++++++++++++++++++++++++++++++++++++++++++++++++++++++++++++++++++++

\subsection{Rappels et définitions}

Dans cette section nous considérons des fonctions $f : D \to \eR^m$
où $D \subset \eR^n$, et un point $a \in \interieur D$ où $f$ est
différentiable.
\begin{remark}
  La définition de continuité (resp. différentiabilité) pour une
  fonction à valeurs vectorielles est celle introduite précédemment,
  et on remarque que pour avoir la continuité
  (resp. différentiabilité) de $f$ en un point, il faut et il suffit
  de chacune des composantes de $f = (f_1,\ldots, f_m)$, vues
  séparément comme fonctions à $n$ variables et à valeurs réelles,
  soit continue (resp. différentiable) en ce point.
\end{remark}

\begin{defn}La \Defn{jacobienne} de $f$ en $a$ est la matrice
  \begin{equation*}
    (\Jac f)_a \begin{pmatrix}
      \pder {f_1} {x_1}(a) & \ldots &\pder {f_1} {x_n}(a)\\
      \vdots& & \vdots\\
      \pder {f_m} {x_1}(a) & \ldots &\pder {f_m} {x_n}(a)
    \end{pmatrix}
  \end{equation*}
  composée de l'ensemble des dérivées partielles de $f$.

  Si $m = 1$, cette matrice ne contient qu'une ligne ; c'est donc un
  vecteur appelé le \Defn{gradient de $f$ en $a$} et noté $\nabla f(a)$.
\end{defn}

\begin{remark}
  \begin{enumerate}
  \item Si la fonction est supposée différentiable, calculer la
    jacobienne revient à connaître la différentielle. En effet, par
    linéarité de la différentielle et par définition des dérivées
    partielles, nous avons
    \begin{equation*}
      d f_a (u) =%
      \begin{pmatrix}
        \pder {f_1} {x_1}(a) & \ldots &\pder {f_1} {x_n}(a)\\
        \vdots& & \vdots\\
        \pder {f_m} {x_1}(a) & \ldots &\pder {f_m} {x_n}(a)
      \end{pmatrix}
      \begin{pmatrix}u_1\\\vdots\\u_n\end{pmatrix}
    \end{equation*}
    où $u = (u_1, \ldots, u_n)$ et où le membre de droite est un
    produit matriciel

  \item Remarquons que la jacobienne peut exister en un point donné
    sans que la fonction soit différentiable en ce point !
  \end{enumerate}
\end{remark}

\begin{proposition}[Règles de calculs] Soient $f$ et $g$ des fonctions
  différentiables en $g(a)$ et $a$ respectivement, alors la composée
  $f\circ g$ est différentiable en $a$ et
  \begin{equation*}
    d (f\circ g)_a = d f_{g(a)} \circ d g_a
  \end{equation*}
  et de plus les jacobiennes correspondantes vérifient
  \begin{equation*}
    \Jac (f\circ g)_a = \Jac f_{g(a)} \Jac g_a
  \end{equation*}
  où le membre de droite est le produit (non-commutatif !) des deux matrices.
\end{proposition}

\begin{corollary}[Chain rule] Si $f : \eR^p \to \eR$ et $g : \eR \to
  \eR^p$, alors
  \begin{equation*}
    (f\circ g)^\prime(t) = \sum_{i=1}^p \pder f {x_i}(g(t)) g_i^\prime(t).
  \end{equation*}
\end{corollary}

\begin{remark}
  \begin{enumerate}
  \item Si $p = 1$, on retrouve la règle usuelle de dérivation de
    fonctions composées.

  \item 
	  Si $g$ est à plusieurs variables, cette règle permet de déterminer les dérivées partielles de $f \circ g$, puisqu'une dérivée partielle peut être vue comme dérivée usuelle par rapport à une seule variable (voir remarque page \pageref{deriveepartielles}).

  \item Si $f$ est à valeurs vectorielles, cette formule permet de
    retrouver la jacobienne de $f \circ g$ puisqu'il suffit de traiter
    chaque composante de $f$ séparément.
  \end{enumerate}
\end{remark}

\begin{defn}
  Soit $f : \eR^n \to\eR$ une fonction différentiable en un point
  $a$. Le \emph{plan tangent} au graphe de $f$ en $(a,f(a))$ est
  l'ensemble des points
  \begin{equation*}
    \begin{split}
      T_af &= \{ (x,z) \in \eR^n \times \eR \tq z = f(a) + d f_a (x-a)\}\\
      &= \{ (x,z) \in \eR^n \times \eR \tq z = f(a) + \scalprod{\nabla f(a)}{x-a}\}
    \end{split}
  \end{equation*}
\end{defn}

%+++++++++++++++++++++++++++++++++++++++++++++++++++++++++++++++++++++++++++++++++++++++++++++++++++++++++++++++++++++++++++
					\section{Intégrales multiples}
%+++++++++++++++++++++++++++++++++++++++++++++++++++++++++++++++++++++++++++++++++++++++++++++++++++++++++++++++++++++++++++

%%%%%%%%%%%%%%%%%%%%%%%%%%%%%%%%%%%
%
%  NOTE : toute cette partie a été reprise dans OutilsMath le 3 avril 2011.
%       

Il est expliqué dans le cours théorique comment on définit le nombre
\begin{equation}
	\int_Ef
\end{equation}
lorsque $E\subset\eR^n$ et $f\colon \eR^n\to \eR$. Nous allons maintenant montrer comment calculer des intégrales en pratique.

%\label{PgRapIntMultFubiniRect}
% ceci était le label d'une sous section ``Intégrale sur un rectangle''.


%---------------------------------------------------------------------------------------------------------------------------
					\subsection{Intégrales sur d'autres domaines}
%---------------------------------------------------------------------------------------------------------------------------
Voir la sous section \ref{PgRapIntMultFubiniTri}.


%---------------------------------------------------------------------------------------------------------------------------
					\subsection{Changement de variables}
%---------------------------------------------------------------------------------------------------------------------------


Le domaine $E=\{ (x,y)\in\eR^2\tq x^2+y^2<1 \}$ s'écrit plus facilement $E=\{ (r,\theta)\tq r<1 \}$ en coordonnées polaires. Le passage aux coordonnées polaire permet de transformer une intégration sur un domaine rond à une intégration sur le domaine rectangulaire $\mathopen]0,2\pi\mathclose[\times\mathopen]0,1\mathclose[$. La question est évidement de savoir si nous pouvons écrire
\begin{equation}
	\int_Ef=\int_{0}^{2\pi}\int_0^1f(r\cos\theta,r\sin\theta)drd\theta.
\end{equation}
Hélas, non; la vie n'est pas aussi simple. Le théorème est celui de la page 480.

\begin{theorem}
Soit $g\colon A\to B$ un difféomorphisme. Soient $F\subset B$ un ensemble mesurable et borné et $f\colon F\to \eR$ une fonction bornée et intégrable. Supposons que $g^{-1}(F)$ soit borné et que $Jg$ soit borné sur $g^{-1}(F)$. Alors
\begin{equation}
	\int_Ff(x)dy=\int_{g^{-1}(F)f\big( g(x) \big)}| Jg(x) |dx
\end{equation}
\end{theorem}
Pour rappel, $Jg$ est le déterminant de la matrice \href{http://fr.wikipedia.org/wiki/Matrice_jacobienne}{jacobienne} (aucun lien de \href{http://fr.wikipedia.org/wiki/Jacob}{parenté}) donnée par
\begin{equation}
	Jg=\det\begin{pmatrix}
	\partial_xg_1	&	\partial_yg_1	\\ 
	\partial_xg_2	&	\partial_tg_2	
\end{pmatrix}.
\end{equation}
Un \defe{difféomorphisme}{Difféomorphisme} est une application $g\colon A\to B$ telle que $g$ et $g^{-1}\colon B\to A$ soient de classe $C^1$.

%///////////////////////////////////////////////////////////////////////////////////////////////////////////////////////////
					\subsubsection{Coordonnées polaires}
%///////////////////////////////////////////////////////////////////////////////////////////////////////////////////////////

Les coordonnées polaires sont données par le difféomorphisme
\begin{equation}
	\begin{aligned}
		g\colon \mathopen]0,\infty\mathclose[\times\mathopen]0,2\pi\mathclose[ &\to\eR^2\setminus D\\
		(r,\theta)&\mapsto \big( r\cos(\theta),r\sin(\theta) \big)
	\end{aligned}
\end{equation}
où $D$ est la demi droite $y=0$, $x\geq 0$. Le fait que les coordonnées polaires ne soient pas un difféomorphisme sur tout $\eR^2$ n'est pas un problème pour l'intégration parce que le manque de difféomorphisme est de mesure nulle dans $\eR^2$. Le jacobien est donné par
\begin{equation}
	Jg=\det\begin{pmatrix}
	\partial_rx	&	\partial_{\theta}x	\\ 
	\partial_ry	&	\partial_{\theta}y
\end{pmatrix}=\det\begin{pmatrix}
	\cos(\theta)	&	-r\sin(\theta)	\\ 
	\sin(\theta)	&	r\cos(\theta)	
\end{pmatrix}=r.
\end{equation}

Voir l'exemple \ref{ExpmfDtAtV}.

%///////////////////////////////////////////////////////////////////////////////////////////////////////////////////////////
\subsubsection{Coordonnées sphériques}
%///////////////////////////////////////////////////////////////////////////////////////////////////////////////////////////

Voir le point \ref{SubSubCoordSpJxhMwm}
 
%+++++++++++++++++++++++++++++++++++++++++++++++++++++++++++++++++++++++++++++++++++++++++++++++++++++++++++++++++++++++++++
					\section{Théorème de la fonction implicite}
%+++++++++++++++++++++++++++++++++++++++++++++++++++++++++++++++++++++++++++++++++++++++++++++++++++++++++++++++++++++++++++

%---------------------------------------------------------------------------------------------------------------------------
\subsection{Mise en situation}
%---------------------------------------------------------------------------------------------------------------------------

Dans un certain nombre de situation, il n'est pas possible de trouver des solutions explicites aux équations qui apparaissent. Néanmoins, l'existence \og théorique\fg{} d'une telle solution est déjà suffisante pour certaines situations, et c'est l'objet du théorème de la fonction implicite.

Prenons par exemple la fonction sur $\eR^2$ donnée par 
\begin{equation}
	F(x,y)=x^2+y^2-1.
\end{equation}
Nous pouvons bien entendu regarder l'ensemble des points donnés par $F(x,y)=0$. C'est le cercle dessiné à la figure \ref{LabelFigCercleImplicite}.
\newcommand{\CaptionFigCercleImplicite}{Un cercle pour montrer l'intérêt de la fonction implicite.}
\input{Fig_CercleImplicite.pstricks}
Nous ne pouvons pas donner le cercle sous la forme $y=y(x)$ à cause du $\pm$ qui arrive quand on prend la racine carrée. Mais si on se donne le point $P$, nous pouvons dire que \emph{autour de $P$}, le cercle est la fonction
\begin{equation}
	y(x)=\sqrt{1-x^2}.
\end{equation}
Tandis que autour du point $P'$, le cercle est la fonction
\begin{equation}
	y(x)=-\sqrt{1-x^2}.
\end{equation}
Autour de ces deux point, donc, le cercle est donné par une fonction. Il n'est par contre pas possible de donner le cercle autour du point $Q$ sous la forme d'une fonction.

Ce que nous voulons faire, en général, est de voir si l'ensemble des points tels que
\begin{equation}
	F(x_1,\ldots,x_n,y)=0
\end{equation}
peut être donné par une fonction $y=y(x_1,\ldots,x_n)$. En d'autre termes, est-ce qu'il existe une fonction $y(x_1,\ldots,x_n)$ telle que
\begin{equation}
	F\big( x_1,\ldots,x_n,y(x_1,\ldots,x_n)\big)=0.
\end{equation}



\subsection{Définitions et rappels}
Soit
\begin{equation*}
  F : D \subset (\RR^n \times \RR^m) \to \RR^m : (x,y) \mapsto F(x,y) =
  (F_1(x,y),\ldots,F_m(x,y))
\end{equation*}
avec $x = (x_1,\ldots, x_n)$ et $y = (y_1,\ldots,y_m)$.
% Pour chaque $x$ fixé, on s'intéresse aux solutions du système de $m$
% équations $F(x,y) = 0$ pour les inconnues $y$ ; en particulier, on
% voudrait pouvoir écrire $y = \varphi(x)$ vérifiant $F(x,\varphi(x))
% = 0$.

Pour $(x,y) \in \interieur D$, la matrice
\begin{equation*}
\begin{pmatrix}
\pder {F_1}{y_1}(x,y)& \ldots& \pder {F_1}{y_m}(x,y)\\
\vdots& \ddots & \vdots\\
\pder {F_m}{y_1}(x,y)& \ldots& \pder {F_m}{y_m}(x,y)\\
\end{pmatrix}
\end{equation*}
est la \defe{matrice jacobienne}{Jacobienne!matrice} de $F$ par rapport à $y$ (au point
$(x,y)$; son déterminant est appelé le \defe{jacobien}{Jacobien} de F par
    rapport à y et se note
  $\pder{(F_1,\ldots,F_m)}{(y_1,\ldots,y_m)}(x,y)$.


\begin{theorem}
	Soit $(\bar x,\bar y)$ tel que $F(\bar x,\bar y) = 0$ et
  $\pder{(F_1,\ldots,F_m)}{(y_1,\ldots,y_m)}(\bar x,\bar y) \neq
  0$. Alors il existe un voisinage $U$ de $x$ dans $\RR^n$, un
  voisinage $V$ de $y$ dans $\RR^m$ et une unique application $\varphi
  : U \to V$ tels que
  \begin{enumerate}
  \item $\varphi(\bar x) = \bar y$ ; 
  \item $F(x,\varphi(x)) = 0$ pour tout $x \in U$.
  \end{enumerate}
\end{theorem}
	
Le théorème de la fonction implicite a pour objet de donner l'existence de la fonction $\varphi$. Maintenant nous pouvons dire beaucoup de choses sur les dérivées de $\varphi$ en considérant la fonction
\begin{equation}
	x\mapsto F\big( x,\varphi(x) \big).
\end{equation}
Par définition de $\varphi$, cette fonction est toujours nulle. En particulier, nous pouvons dériver l'équation
\begin{equation}
	F\big( x,\varphi(x) \big)=0,
\end{equation}
et nous trouvons plein de choses.

%---------------------------------------------------------------------------------------------------------------------------
\subsection{Exemple}
%---------------------------------------------------------------------------------------------------------------------------

Prenons par exemple la fonction
\begin{equation}
	F\big( (x,y),z \big)=ze^z-x-y,
\end{equation}
et demandons nous ce que nous pouvons dire sur la fonction $z(x,y)$ telle que
\begin{equation}
	F\big( x,y,z(x,y) \big)=0,
\end{equation}
c'est à dire telle que
\begin{equation}		\label{EqDefZImplExemple}
	z(x,y) e^{z(x,y)}-x-y=0.
\end{equation}
pour tout $x$ et $y\in\eR$. Nous pouvons facilement trouver $z(0,0)$ parce que
\begin{equation}
	z(0,0) e^{z(0,0)}=0,
\end{equation}
donc $z(0,0)=0$.

Nous pouvons dire des choses sur les dérivées de $z(x,y)$. Voyons par exemple $(\partial_xz)(x,y)$. Pour trouver cette dérivée, nous dérivons la relation \eqref{EqDefZImplExemple} par rapport à $x$. Ce que nous trouvons est
\begin{equation}
	(\partial_xz)e^z+ze^z(\partial_xz)-1=0.
\end{equation}
Cette équation peut être résolue par rapport à $\partial_xz$~:
\begin{equation}
	\frac{ \partial z }{ \partial x }(x,y)=\frac{1}{ e^z(1+z) }.
\end{equation}
Remarquez que cette équation ne donne pas tout à fait la dérivée de $z$ en fonction de $x$ et $y$, parce que $z$ apparaît dans l'expression, alors que $z$ est justement la fonction inconnue. En général, c'est la vie, nous ne pouvons pas faire mieux.

Dans certains cas, on peut aller plus loin. Par exemple, nous pouvons calculer cette dérivée au point $(x,y)=(0,0)$ parce que $z(0,0)$ est connu :
\begin{equation}
	\frac{ \partial z }{ \partial x }(0,0)=1.
\end{equation}
Cela est pratique pour calculer, par exemple, le développement en Taylor de $z$ autour de $(0,0)$.


%+++++++++++++++++++++++++++++++++++++++++++++++++++++++++++++++++++++++++++++++++++++++++++++++++++++++++++++++++++++++++++
\section{Formes différentielles et son intégrale sur un chemin}
%+++++++++++++++++++++++++++++++++++++++++++++++++++++++++++++++++++++++++++++++++++++++++++++++++++++++++++++++++++++++++++

%---------------------------------------------------------------------------------------------------------------------------
\subsection{Forme différentielle}
%---------------------------------------------------------------------------------------------------------------------------

La formule d'intégration d'un champ de vecteur,
\begin{equation}
	\int_{\gamma}G=\int_{[a,b]}\langle G (\gamma(t)), \gamma'(t)\rangle dt,
\end{equation}
contient quelque chose d'intéressant : la combinaison $\langle G( \gamma(t) ), \gamma'(t)\rangle$. Cette combinaison sert à transformer le vecteur tangent $\gamma'(t)$ en un nombre en utilisant le produit scalaire avec le vecteur $G( \gamma(t) )$.

Si $G$ est un champ de vecteur sur $\eR^n$, et si $x\in\eR^n$, nous pouvons considérer, de façon un peu plus abstraite, l'application
\begin{equation}		\label{EqDefBemol}
	\begin{aligned}[]
		G^{\flat}_x\colon \eR^n&\to \eR \\
			v&\mapsto \langle G(x), v\rangle . 
	\end{aligned}
\end{equation}
Cela permet de compactifier la notation et écrire
\begin{equation}
	\int_{\gamma}G=\int_{[a,b]} G^{\flat}_{\gamma(t)}\big( \gamma'(t)\big) dt.
\end{equation}

Nous nous proposons maintenant d'étudier plus en détail ce qu'est l'objet $G^{\flat}$. La règle \eqref{EqDefBemol} dit que pour chaque $x$, l'application $G_x^{\flat}$ est une forme sur $\eR^n$, c'est à dire une application linéaire de $\eR^n$ vers $\eR$. Nous écrivons que
\begin{equation}
	G_x^{\flat}\in\big( \eR^n \big)^*.
\end{equation}
Nous connaissons la \defe{base duale}{Base duale} de $(\eR^n)^*$, ce sont les formes $e^*_i$ définies par $e^*_i(e_j)=\delta_{ij}$. Dans le cadre du cours d'analyse, nous allons noter ces formes\footnote{Parce que ce sont les différentielles des fonctions (projections)
\begin{equation}
	\begin{aligned}
			x_i\colon \eR^n&\to \eR \\
			x&\mapsto x_i 
		\end{aligned}
	\end{equation}
}
par $dx_i$ :
\begin{equation}
	\begin{aligned}[]
		e^*_1&=dx_1\colon v\mapsto v_1	\\
			&\vdots			\\
		e^*_n&=dx_n\colon v\mapsto v_n
	\end{aligned}
\end{equation}
Étant donné que ces $dx_i$ forment une base de l'espace vectoriel $(\eR^n)^*$, toute application linéaire $L\colon \eR^n\to \eR$ s'écrit
\begin{equation}
	\begin{aligned}[]
		Lv&=a_1v_1+\ldots+a_nv_n\\
			&=a_1dx_1(v)+\ldots+a_ndx_n(v).
	\end{aligned}
\end{equation}
Plus abstraitement, nous notons
\begin{equation}
	\begin{aligned}[]
		L&=a_1dx_1+\ldots+a_ndx_n\\
		&=\sum_{i=1}^na_idx_i.
	\end{aligned}
\end{equation}
L'application $L$ est une combinaison linéaire des $dx_i$ au sens de l'espace vectoriel $(\eR^n)^*$.

L'objet $G^{\flat}$ est la donnée, en chaque point de $D$, d'une telle forme sur $\eR^n$. Nous donnons alors la définition suivante.
\begin{definition}
	Soit $D$, un domaine dans $\eR^n$. Une $1$-\defe{forme différentielle}{Forme différentielle} $\omega$ sur $D$ est une application
	\begin{equation}
		\begin{aligned}
				\omega\colon D&\to (\eR^n)^* \\
				x&\mapsto \omega_x. 
			\end{aligned}
		\end{equation}
\end{definition}
Étant donné que $\{ dx_i \}$ est une base de $(\eR^n)^*$, pour chaque $x\in D$, il existe des uniques réels $a_i(x)$ tels que
\begin{equation}
	\omega_x=a_1(x)dx_1+\ldots+a_n(x)dx_n.
\end{equation}
Nous disons qu'une $1$-forme différentielle est \defe{continue}{Continue!forme différentielle} si les fonctions $a_i$ sont continues. La forme sera $C^k$ quand les $a_i$ seront $C^k$.

\begin{remark}
	L'ensemble des $1$-formes différentielles forment un espace vectoriel avec les définitions
	\begin{equation}
		\begin{aligned}[]
			(\lambda\omega)_x(v)&=\lambda\omega_x(v)\\
			(\omega+\mu)_x(v)&=\omega_x(v)+\mu_x(v).
		\end{aligned}
	\end{equation}
\end{remark}

Lorsque une $1$-forme différentielle s'écrit toujours sous la forme
\begin{equation}
	\omega=\sum_i a_idx_i
\end{equation}
pour certaines fonctions $a_i$. Évidemment, ces fonctions $a_i$ peuvent être trouvées en appliquant $\omega$ aux éléments de la base canonique de $\eR^n$ :
\begin{equation}
	a_j(x)=\omega_x(e_j)
\end{equation}
parce que $\omega_x(e_j)=\sum_ia_i(x)dx_i(e_i)=\sum_ia_i(x)\delta_{ij}=a_j(x)$.

Nous pouvons ainsi déterminer le développement de $G^{\flat}$ dans la base des $dx_i$ en faisant le calcul
\begin{equation}
	G_x^{\flat}(e_i)=\langle G(x), e_i\rangle =G_i(x),
\end{equation}
donc les composantes de $G^{\flat}$ dans la base $dx_i$ sont exactement les composantes de $G$ dans la base $e_i$ :
\begin{equation}
	G^{\flat}_x=G_1(x)dx_1+\ldots+G_n(x)dx_n.
\end{equation}


%///////////////////////////////////////////////////////////////////////////////////////////////////////////////////////////
\subsubsection{Une petite note pour titiller monsieur Jean Doyen}
%///////////////////////////////////////////////////////////////////////////////////////////////////////////////////////////

Pensons pendant quelque minutes aux fonctions de $\eR$ dans $\eR$. Monsieur Jean Doyen dit toujours que quand le sage demande la fonction $f$, le simple dit \og $f(x)$\fg. Or $f(x)$ n'est pas une fonction; c'est $f$, la fonction. Avec un tout petit peu de mauvaise foi, nous pouvons prétendre que $x$ désigne la fonction identité qui à chaque $x$ fait correspondre $x$ lui-même. Il est d'ailleurs un peu normal de désigner comme ça cette fonction. Dans ce cas, $f(x)$ désigne la fonction composée de la fonction $f$ avec la fonction $x$, et tout le monde est content.

Avouons que cela est un petit peu de mauvaise foi\footnote{J'offre un pot à qui ose écrire que $f(x)$ est bien la \emph{fonction} composée de $f$ avec $x$ sur sa feuille d'examen du cours d'algèbre linéaire.}. Vraiment ?

La fonction $x$ est une fonction de $\eR$ vers $\eR$. Sa différentielle en un point est donc une application de $\eR$ vers $\eR$. Devinez ce qu'elle vaut ? Ben oui : la différentielle de la fonction $x$ est \emph{vraiment} le $dx$ qu'on écrit tout le temps, la forme différentielle, la base de l'espace dual !

%///////////////////////////////////////////////////////////////////////////////////////////////////////////////////////////
\subsubsection{L'isomorphisme musical}
%///////////////////////////////////////////////////////////////////////////////////////////////////////////////////////////

Nous savons qu'un champ de vecteur $G$ produit la forme différentielle $G^{\flat}$. La construction inverse existe également. Si $\omega$ est une $1$-forme différentielle, nous pouvons définir le champ de vecteur $\omega^{\sharp}$ par la formule (implicite)
\begin{equation}
	\omega_x(v)=\langle \omega^{\sharp}(x), v\rangle 
\end{equation}
pour tout $v\in\eR^n$. Par définition, $(\omega^{\sharp})^{\flat}=\omega$. 

\begin{exercice}
	Prouver que, en composantes, 
	\begin{equation}
		\omega^{\sharp}(x)=\big( a_1(x),\ldots,a_n(x) \big),
	\end{equation}
	et vérifier que si $G$ est un champ de vecteurs, alors $(G^{\flat})^{\sharp}=G$.
\end{exercice}


%///////////////////////////////////////////////////////////////////////////////////////////////////////////////////////////
\subsubsection{Formes différentielles exactes et fermées}
%///////////////////////////////////////////////////////////////////////////////////////////////////////////////////////////

Considérons une fonction différentiable $f\colon D\to \eR$. Pour chaque $x\in D$, nous avons l'application différentielle
\begin{equation}
	\begin{aligned}
		df(x)\colon \eR^n&\to \eR \\
		v&\mapsto \sum_{i=1}^n\frac{ \partial f }{ \partial x_i }(x)v_i, 
	\end{aligned}
\end{equation}
c'est à dire que $df$ est une $1$-forme différentielle dont les composantes sont
\begin{equation}
	df(x)=\frac{ \partial f }{ \partial x_1 }(x)dx_1+\ldots+\frac{ \partial f }{ \partial x_n }(x)dx_n.
\end{equation}

Il est naturel de se demander si toutes les formes différentielles sont des différentielles de fonctions. Une réponse complète est délicate à établir, mais a d'innombrables conséquences en physique, notamment en ce qui concerne l'existence d'un potentiel vecteur pour le champ magnétique dans les équations de Maxwell.
\begin{definition}
	Deux classes importantes de formes différentielles sont à mettre en évidence
	\begin{enumerate}
		\item
			Une forme différentielle $\omega$ sur un ouvert $D\subset\eR^n$ est \defe{exacte}{Forme différentielle!exacte} si il existe une fonction différentiable $f\colon D\to \eR$ telle que
			\begin{equation}
				 \omega_x=df(x)
			\end{equation}
			pour tout $x\in D$.
		\item
			Une $1$-forme de classe $C^1$ sur l'ouvert $D$ est \defe{fermée}{Forme différentielle!fermée} si pour tout $i,j=1,\ldots n$, nous avons
			\begin{equation}
				\frac{ \partial a_i }{ \partial x_j }=\frac{ \partial a_j }{ \partial x_i }.
			\end{equation}
	\end{enumerate}
\end{definition}

\begin{proposition}
	Si $\omega$ est une $1$-forme exacte de classe $C^1$, alors $\omega$ est fermée.
\end{proposition}

\begin{proof}
	Le fait que $\omega$ soit exacte implique l'existence d'une fonction $f$ telle que $\omega=df$, c'est à dire
	\begin{equation}
		\omega_x=\sum_i a_i(x)dx_i=\sum_i\frac{ \partial f }{ \partial x_i }(x)dx_i,
	\end{equation}
	c'est à dire que $a_i(x)=\frac{ \partial f }{ \partial x_i }(x)$. L'hypothèse que $\omega$ est $C^1$ implique que $f$ est $C^2$, et donc que nous pouvons inverser l'ordre de dérivation pour les dérivées secondes $\partial^2_{ij}f=\partial^2_{ji}f$. Nous pouvons donc faire le calcul suivant :
	\begin{equation}
		\frac{ \partial a_i }{ \partial x_j }=\frac{ \partial  }{ \partial x_j }\frac{ \partial f }{ \partial x_i }=\frac{ \partial  }{ \partial x_i }\frac{ \partial f }{ \partial x_j }=\frac{ \partial a_j }{ \partial x_i },
	\end{equation}
	ce qu'il fallait démontrer.
\end{proof}

Ceci est assez pour les formes différentielles pour cette année.

%---------------------------------------------------------------------------------------------------------------------------
\subsection{Intégration d'une forme différentielle sur un chemin}
%---------------------------------------------------------------------------------------------------------------------------

Les formes intégrales que nous avons déjà vues sont celles de fonctions et de champs de vecteur sur des chemins. Si $\gamma\colon [a,b]\to \eR^n$ est le chemin, les formules sont
\begin{equation}
	\begin{aligned}[]
		\int_{\gamma}f&=\int_{[a,b]}f\big( \gamma(t) \big)\| \gamma'(t) \|dt\\
		\int_{\gamma}G&=\int_{[a,b]}\langle G\big( \gamma(t) \big), \gamma'(t)\rangle dt.
	\end{aligned}
\end{equation}
Dans les deux cas, le principe est que nous disposons de quelque chose (la fonction $f$ ou le vecteur $G$), et du vecteur tangent $\gamma'(t)$, et nous essayons d'en tirer un nombre que nous intégrons. Lorsque nous avons une $1$-forme, la façon de l'utiliser pour produire un nombre avec le vecteur tangent est évidement d'appliquer la $1$-forme au vecteur tangent. La définition suivante est donc naturelle.

\begin{definition}
	Soit $\gamma\colon [a,b]\to \eR^n$, un chemin de classe $C^1$ tel que son image est contenue dans le domaine $D$. Si $\omega$ es une $1$-forme différentielle sur $D$, nous définissons l'\defe{intégrale de $\omega$ le long de $\gamma$}{Intégrale d'une forme différentielle} le nombre
	\begin{equation}
		\begin{aligned}[]
			\int_{\gamma}\omega&=\int_a^b\omega_{\gamma(t)}\big( \gamma'(t) \big)dt\\
				&=\int_a^b\Big[ a_1\big( \gamma(t) \big)\gamma'_1(t)+\ldots +  a_n\big( \gamma(t) \big)\gamma'_n(t) \Big]dt.
		\end{aligned}
	\end{equation}
\end{definition}

Cette définition est une bonne définition parce que si on change la paramétrisation du chemin, on ne change pas la valeur de l'intégrale, c'est la proposition suivante.
\begin{proposition}
	Si $\gamma$ et $\beta$ sont des chemins équivalents, alors
	\begin{equation}
		\int_{\gamma}\omega=\int_{\beta}\omega,
	\end{equation}
	c'est à dire que l'intégrale est invariante sous les reparamétrisation du chemin.
\end{proposition}
\begin{proof}
	Deux chemins sont équivalents quand il existe un difféomorphisme $C^1$ $h\colon [a,b]\to [c,d]$ tel que $\gamma(t)=(\beta\circ h)(t)$. En remplaçant $\gamma$ par $(\beta\circ h)$ dans la définition de $\int_{\gamma}\omega$, nous trouvons
	\begin{equation}
		\int_a^b\omega_{\gamma(t)}\big( \gamma'(t) \big)dt=\int_a^b\omega_{(\beta\circ h)(t)}\big( (\beta\circ h)'(t) \big)dt.
	\end{equation}
	Un changement de variable $u=h(t)$ transforme cette dernière intégrale en $\int_{\beta}\omega$, ce qui prouve la proposition.
\end{proof}

\begin{remark}
	Si $\gamma$ est une somme de chemins, $\gamma=\gamma^{(1)}+\ldots+\gamma^{(n)}$, où chacun des $\gamma^{(i)}$ est un chemin, alors
	\begin{equation}
		\int_{\gamma}\omega=\sum_{i=1}^n\int_{\gamma_i}\omega
	\end{equation}
	parce que $\omega$ est linéaire.
\end{remark}

\begin{remark}
	Si $-\gamma$ est le chemin
	\begin{equation}
		\begin{aligned}
			- \gamma\colon [a,b]&\to \eR^n \\
			t&\mapsto \gamma\big( b-(t-a) \big),
		\end{aligned}
	\end{equation}
	alors
	\begin{equation}
		\int_{-\gamma}\omega=-\int_{\gamma}\omega,
	\end{equation}
	c'est à dire que si l'on parcours le chemin en sens inverse, alors on change le signe de l'intégrale.
\end{remark}

L'intégrale d'une forme différentielle sur un chemin est compatible avec l'intégrale déjà connue d'un champ de vecteur sur le chemin parce que si $G$ est un champ de vecteurs,
\begin{equation}
	\int_{\gamma}G^{\flat}=\int_{\gamma}G.
\end{equation}
En effet,
\begin{equation}
	\begin{aligned}[]
		\int_{\gamma G^{\flat}}	&=\int_a^b G_{\gamma(t)}^{\flat}(\gamma'(t))\\
					&=\int_a^b\big[ G_1( \gamma(t) )dx_1+\ldots G_n(\gamma(t))dx_n \big]\big( \gamma'_1(t),\ldots,\gamma'_n(t) \big)\\
					&=\int_{a}^b\langle G(\gamma(t)), \gamma'(t)\rangle \\
					&=\int_{\gamma}G.
	\end{aligned}
\end{equation}


\begin{proposition}
	Soit $\omega=df$, une $1$-forme exacte et continue sur le domaine $D$. Alors la valeur de $\int_{\gamma}df$ ne dépend que des valeurs de $f$ aux extrémités de $\gamma$.
\end{proposition}

\begin{proof}
	Nous avons
	\begin{equation}
		\begin{aligned}[]
			\int_{\gamma}\omega=\int_{\gamma}df&=\int_{a}^b\sum_{i=1}n\frac{ \partial f }{ \partial x_i }\big( \gamma(t) \big)\gamma'_i(t)dt\\
				&=\int_a^b\frac{ d }{ dt }\Big( (f\circ\gamma)(t) \Big)dt\\
				&=(f\circ\gamma)(b)-(f\circ\gamma(a)).
		\end{aligned}
	\end{equation}
\end{proof}

%---------------------------------------------------------------------------------------------------------------------------
\subsection{Interprétation physique : travail}
%---------------------------------------------------------------------------------------------------------------------------

\begin{definition}
	Une force $F\colon D\subset\eR^n\to \eR^n$ est \defe{\href{http://fr.wikipedia.org/wiki/Force_conservative}{conservative}}{Conservative} si elle dérive d'un potentiel, c'est à dire si il existe une fonction $V\in C^1(D,\eR)$ telle que 
	\begin{equation}
		F(x)=(\nabla V)(x).
	\end{equation}
\end{definition}
Étant donné que $F$ est un champ de vecteurs, nous avons une forme différentielle associée $F^{\flat}$,
\begin{equation}
	F^{\flat}_x\colon x\mapsto \langle F(x), v\rangle .
\end{equation}

\begin{lemma}
	Le champ $F$ est conservatif si et seulement si la $1$-forme différentielle $F^{\flat}$ est exacte.
\end{lemma}

\begin{proof}
	Supposons que la force $F$ soit conservative, c'est à dire qu'il existe une fonction $V$ telle que $F=\nabla V$. Dans ce cas, il est facile de prouver que $F^{\flat}$ est exacte et est donnée par $F_x^{\flat}=dV(x)$. En effet,
	\begin{equation}
		\begin{aligned}[]
			F_x^{\flat}(v)	&=\langle F(x), v\rangle \\
					&=F_1(x)v_1+\ldots+F_n(x)v_n\\
					&=\frac{ \partial V }{ \partial x_1 }(x)v_1+\ldots\frac{ \partial V }{ \partial x_n }(x)v_n\\
					&=dV(x)v.
		\end{aligned}
	\end{equation}
	
	Pour le sens inverse, supposons que $F^{\flat}$ soit exacte. Dans ce cas, nous avons une fonction $V$ telle que $F^{\flat}=dV$. Il est facile de prouver qu'alors, $F=\nabla V$.
\end{proof}
En résumé, nous avons deux façons équivalentes d'exprimer que la force $F$ dérive du potentiel $V$ :  soit nous disons $F=\nabla V$, soit nous disons $F^{\flat}=dV$.

\begin{proposition}
	Si $F$ est une force conservative, alors le \href{http://fr.wikipedia.org/wiki/Travail_d'une_force}{travail} de $F$ lors d'un déplacement ne dépend pas du chemin suivit.
\end{proposition}

\begin{proof}
	Le travail d'une force le long d'un chemin n'est autre que l'intégrale de la force le long du chemin, et le calcul est facile :
	\begin{equation}
		W_{\gamma}(F)=\int_{\gamma}F=\int_{\gamma}dV=V\big( \gamma(b) \big)-V\big( \gamma(a) \big).
	\end{equation}
	Donc si $\beta$ est un autre chemin tel que $\beta(a)=\gamma(a)$ et $\beta(b)=\gamma(b)$, nous avons $W_{\beta}(F)=W_{\gamma}(F)$.
\end{proof}

%+++++++++++++++++++++++++++++++++++++++++++++++++++++++++++++++++++++++++++++++++++++++++++++++++++++++++++++++++++++++++++
\section{Intégrale sur une variété}
%+++++++++++++++++++++++++++++++++++++++++++++++++++++++++++++++++++++++++++++++++++++++++++++++++++++++++++++++++++++++++++

%---------------------------------------------------------------------------------------------------------------------------
\subsection{Mesure sur une carte}
%---------------------------------------------------------------------------------------------------------------------------

Nous considérons dans cette section uniquement des variétés $M$ de dimension $2$ dans $\eR^3$.  Une particularité de $\eR^3$ (par rapport aux autres $\eR^n$) est qu'il existe le produit vectoriel. 

Si $v$, $w\in\eR^3$, alors le vecteur $v\times w$ est une vecteur normal au plan décrit par $v$ et $w$ qui jouit de l'importante propriété suivante :
\begin{equation}
	\text{aire du parallélogramme}=\| v\times w \|.
\end{equation}
L'aire du parallélogramme construit sur $v$ et $w$ est donnée par la norme du produit vectoriel. Afin de donner une mesure infinitésimale en un point $p\in M$, nous voudrions prendre deux vecteurs tangents à $M$ en $p$, et puis considérer la norme de leur produit vectoriel. Cette idée se heurte à la question du choix des vecteurs tangents à considérer.

Dans $\eR^2$, le choix est évident : nous choisissons $e_x$ et $e_y$, et nous avons $\|e_x\times e_y\|=1$. L'idée est donc de choisir une carte $F\colon W\to F(w)$ autour du point $p=F(w)$, et de choisir les vecteurs tangents qui correspondent à $e_x$ et $e_y$ via la carte, c'est à dire les vecteurs
\begin{equation}
	\begin{aligned}[]
		\frac{ \partial F }{ \partial x }(w),&&\text{et}&&\frac{ \partial F }{ \partial y }(w).
	\end{aligned}
\end{equation}
L'\defe{élément infinitésimal de surface}{Élément de surface} sur $M$ au point $p=F(w)$ est alors défini par
\begin{equation}
	d\sigma_F=\|  \frac{ \partial F }{ \partial x }(w)\times\frac{ \partial F }{ \partial y }(w) \|dw,
\end{equation}
et si la partie $A\subset M$ est entièrement contenue dans $F(W)$, nous définissons la \defe{mesure}{Mesure!dans une carte} de $A$ par
\begin{equation}		\label{EqDefMuDeuxDF}
	\mu_2(A)=\int_{F^{-1}(A)}d\sigma_F=\int_{F^{-1}(A)}\| \frac{ \partial F }{ \partial x }(w)\times\frac{ \partial F }{ \partial y }(w) \|dw.
\end{equation}
\begin{remark}
	Afin que cette définition ait un sens, nous devons prouver qu'elle ne dépend pas du choix de la carte $F$. En effet, les vecteurs $\partial_xF$ et $\partial_yF$ dépendent de la carte $F$, donc leur produit vectoriel (et sa norme) dépendent également de la carte $F$ choisie. Il faudrait donc un petit miracle pour que le nombre $\mu_2(A)$ donné par \eqref{EqDefMuDeuxDF} soit indépendant du choix de $F$.  Nous allons bientôt voir comme cas particulier du théorème \ref{ThoIntIndepF} que c'est en fait le cas. C'est à dire que si $F$ et $\tilde F$ sont deux cartes qui contiennent $A$, alors
	\begin{equation}
		\int_{F^{-1}(A)}d\sigma_F=\int_{\tilde F^{-1}(A)}d\sigma_{\tilde F}.
	\end{equation}
\end{remark}

%///////////////////////////////////////////////////////////////////////////////////////////////////////////////////////////
\subsubsection{Exemple : la mesure de la sphère}
%///////////////////////////////////////////////////////////////////////////////////////////////////////////////////////////

Nous nous proposons maintenant de calculer la surface de la sphère $S^2=x^2+y^2+z^2=R^2$. L'application $F\colon B( (0,0),R)\to R^3$ donnée par
\begin{equation}
	F(x,y)=\begin{pmatrix}
		x	\\ 
		y	\\ 
		\sqrt{R^2-x^2-y^2}	
	\end{pmatrix}
\end{equation}
est une carte pour une demi sphère. Ses dérivées partielles sont
\begin{equation}
	\begin{aligned}[]
		\frac{ \partial F }{ \partial x }&=\begin{pmatrix}
			1	\\ 
			0	\\ 
			-\frac{ x }{ \sqrt{R^2-x^2-y^2} }	
		\end{pmatrix},
		&\frac{ \partial F }{ \partial y }&=\begin{pmatrix}
			0	\\ 
			1	\\ 
			-\frac{ y }{ \sqrt{R^2-x^2-y^2} }	
		\end{pmatrix}.
	\end{aligned}
\end{equation}
Le produit vectoriel de ces deux vecteurs tangents donne
\begin{equation}
	\frac{ \partial F }{ \partial x }(x,y)\times\frac{ \partial F }{ \partial y }(x,y)=\frac{ x }{ \alpha }e_1+\frac{ y }{ \alpha }e_2+e_3
\end{equation}
où $\alpha=\sqrt{R^2-x^2-y^2}$. En calculant la norme, nous trouvons
\begin{equation}
	\| \frac{ \partial F }{ \partial x }(x,y)\times\frac{ \partial F }{ \partial y }(x,y)\| =\sqrt{  \frac{ R^2 }{ R^2-x^2-y^2 } },
\end{equation}
et en passant aux coordonnées polaires, nous écrivons l'intégrale \eqref{EqDefMuDeuxDF} sous la forme
\begin{equation}
	\int_B\| \partial_xF\times\partial_yF \|=\int_0^{2\pi}d\theta\int_0^R r\sqrt{  \frac{ R^2 }{ R^2-x^2-y^2 } }dr=2\pi R^2,
\end{equation}
qui est bien la mesure de la demi sphère.

%---------------------------------------------------------------------------------------------------------------------------
\subsection{Intégrale sur une carte}
%---------------------------------------------------------------------------------------------------------------------------

Nous pouvons maintenant définir l'intégrale d'une fonction sur une carte de la variété $M$.
\begin{definition}
	Soit $F\colon W\subset \eR^2\to \eR^3$, une carte pour une variété $M$. Soit $A$, une partie de $F(W)$ telle que $A=F(B)$ où $B\subset W$ est mesurable.  Soit encore $f\colon A\to \eR$, une fonction continue. L'\defe{intégrale}{Intégrala!d'une fonction sur une carte} de $f$ sur $A$ est le nombre
	\begin{equation}	\label{EqDefIntDeuxDF}
		\int_Af=\int_Afd\sigma_F=\int_{F^{-1}(A)}(f\circ F)(w)\|  \frac{ \partial F }{ \partial x }(w)\times\frac{ \partial F }{ \partial y }(w) \| dw
	\end{equation}
\end{definition}

\begin{remark}
	L'intégrale \eqref{EqDefIntDeuxDF} n'est pas toujours bien définie. Étant donné que $F$ est $C^1$ et que $f$ est continue, l'intégrante est continue. L'intégrale sera donc bien définie par exemple lorsque $B$ est borné et si la fermeture $\bar A$ est un compact contenu dans $F(w)$.
\end{remark}

Le théorème suivant montre que le travail que nous avons fait jusqu'à présent ne dépend en fait pas du choix de carte $F$ effectué.

\begin{theorem}\label{ThoIntIndepF}
	Soient $F\colon W\to F(w)$ et $\tilde F\colon \tilde W\to \tilde F(\tilde W)$, deux cartes de la variété $M$. Soit une partie $A\subset F(W)\cap\tilde F(\tilde W)$ telle que $A=F(B)$ avec $B\subset W$ mesurable.  Alors $A=\tilde F(\tilde B)$ avec $\tilde B\subset\tilde W$ mesurable.

	Si $f$ est une fonction continue, et si $\int_Afd\sigma_F$ existe, alors $\int_Afd\sigma_{\tilde F}$ existe et
	\begin{equation}
		\int_Afd\sigma_F=\int_Afd\sigma_{\tilde F}.
	\end{equation}
\end{theorem}


%---------------------------------------------------------------------------------------------------------------------------
\subsection{Exemples}
%---------------------------------------------------------------------------------------------------------------------------

Intégrons la fonction $f(x,y,z)$ sur le carré $K=\mathopen] 0 , 1 \mathclose[\times \mathopen] 0 , 2 \mathclose[\times\{ 1 \}$. La première carte que nous pouvons utiliser est
\begin{equation}
	\begin{aligned}
		F\colon \mathopen] 0 , 1 \mathclose[\times\mathopen] 0 , 2 \mathclose[&\to K \\
		(x,y)&\mapsto (x,y,1). 
	\end{aligned}
\end{equation}
Nous trouvons aisément les vecteurs tangents qui forment l'élément de surface:
\begin{equation}
	\begin{aligned}[]
		\frac{ \partial F }{ \partial x }&=\begin{pmatrix}
			1	\\ 
			0	\\ 
			0	
		\end{pmatrix},
		&\frac{ \partial F }{ \partial y }&=\begin{pmatrix}
			0	\\ 
			1	\\ 
			0	
		\end{pmatrix},
	\end{aligned}
\end{equation}
donc $d\sigma_F=1\cdot dxdy$, et
\begin{equation}		\label{IntKSurcarrUn}
	\int_Kfd\sigma_F=\int_{\mathopen] 0 , 1 \mathclose[\times\mathopen] 0 , 2 \mathclose[}f(x,y,1)\cdot 1\cdot dxdy.
\end{equation}

Nous pouvons également utiliser la carte
\begin{equation}
	\begin{aligned}
		\tilde F\colon \mathopen] 0 , \frac{ 1 }{2} \mathclose[\times\mathopen] 0 , 6 \mathclose[&\to K \\
		(\tilde x,\tilde y)&\mapsto (2\tilde x,\frac{ \tilde y }{ 3 },1). 
	\end{aligned}
\end{equation}
Les vecteurs tangents sont maintenant
\begin{equation}
	\begin{aligned}[]
		\frac{ \partial \tilde F }{ \partial \tilde x }&=\begin{pmatrix}
			2	\\ 
			0	\\ 
			0	
		\end{pmatrix},
		&\frac{ \partial \tilde F }{ \partial \tilde y }&=\begin{pmatrix}
			0	\\ 
			1/3	\\ 
			0	
		\end{pmatrix},
	\end{aligned}
\end{equation}
de telle façon à ce que $d\sigma_{\tilde F}=\| \frac{ 2 }{ 3 }e_3 \|=\frac{ 2 }{ 3 }$. Cette fois, l'intégrale de $f$ sur $K$ s'écrit
\begin{equation}
	\int_Kfd\sigma_{\tilde F}=\int_{\mathopen] 0 , \frac{ 1 }{2} \mathclose[\times\mathopen] 0 , 6 \mathclose[}f\big( 2\tilde x,\frac{ \tilde y }{ 3 },1 \big)\cdot\frac{ 2 }{ 3 }\cdot d\tilde xs\tilde y.
\end{equation}
Conformément au théorème \ref{ThoIntIndepF}, cette dernière intégrale est égale à l'intégrale \eqref{IntKSurcarrUn} parce qu'il s'agit juste d'un changement de variable.


%---------------------------------------------------------------------------------------------------------------------------
\subsection{Orientation}
%---------------------------------------------------------------------------------------------------------------------------


Soient $F\colon W\to F(w)$ et $\tilde F\colon \tilde W\to \tilde F(\tilde W)$, deux cartes de la variété $M$. Nous pouvons considérer la fonction $h=\tilde F^{-1}\circ F$, définie uniquement sur l'intersection des cartes :
\begin{equation}
	h\colon F^{-1}\big( F(W)\cap\tilde F(\tilde W) \big)\to \tilde F^{-1}\big( F(W)\cap\tilde F(\tilde W) \big).
\end{equation}
Nous disons que $F$ et $\tilde F$ ont même \defe{orientation}{Orientation} si
\begin{equation}
	J_h(w)>0
\end{equation}
pour tout $w\in  F^{-1}\big( F(W)\cap\tilde F(\tilde W) \big)$.

Considérons les deux carte suivantes pour le même carré:
\begin{equation}
	\begin{aligned}
		F\colon\mathopen] 0 , 1 \mathclose[\times \mathopen] 0 , 1 \mathclose[ &\to \eR^3 \\
		(x,y)&\mapsto (x,y,0) 
	\end{aligned}
\end{equation}
et
\begin{equation}
	\begin{aligned}
		\tilde F\colon\mathopen] 0 , \frac{ 1 }{2} \mathclose[\times\mathopen] 0 , \frac{1}{ 3 } \mathclose[ &\to \eR^3 \\
		(x,y)&\mapsto (2x,3y,0) 
	\end{aligned}
\end{equation}
Ici, $h(x,y)=\left( \frac{ x }{ 2 },\frac{ y }{ 3 } \right)$ et nous avons $J_h=\frac{1}{ 6 }>0$. Ces deux cartes ont même orientation. Notez que
\begin{equation}
	\frac{ \partial F }{ \partial x }\times\frac{ \partial F }{ \partial y }=e_3,
\end{equation}
tandis que
\begin{equation}
	\frac{ \partial \tilde F }{ \partial x }\times\frac{ \partial \tilde F }{ \partial y }=6e_3.
\end{equation}
Les vecteurs normaux à la paramétrisation pointent dans le même sens.

Si par contre nous prenons la paramétrisation
\begin{equation}
	\begin{aligned}
		G\colon \mathopen] 0,1 \mathclose[\times\mathopen] 0,1 ,  \mathclose[&\to \eR^2 \\
		(x,y)&\mapsto (x,(1-y),0), 
	\end{aligned}
\end{equation}
nous avons
\begin{equation}
	\frac{ \partial G }{ \partial x }\times\frac{ \partial G }{ \partial y }=-e_3,
\end{equation}
et si $g=G^{-1}\circ F$, alors $J_g=-1$.

L'orientation d'une carte montre donc si le vecteur normal à la surface pointe d'un côté ou de l'autre de la surface.

\begin{definition}
	Une variété $M$ est \defe{orientable}{Orientable!variété} si il existe un atlas de $M$ tel que deux cartes quelconques ont toujours même orientation. Une variété est \defe{orientée}{Variété !orientée} lorsque qu'un tel choix d'atlas est fait.
\end{definition}

\begin{proposition}
	Soit $M$, une variété orientable et un atlas orienté $\{ F_i\colon W_i\to \eR^3 \}$. Alors le vecteur unitaire
	\begin{equation}
		\frac{   \frac{ \partial F }{ \partial x }(x,y)\times\frac{ \partial F }{ \partial y }(x,y)   }{ \| \frac{ \partial F }{ \partial x }(x,y)\times\frac{ \partial F }{ \partial y }(x,y)\| }
	\end{equation}
	ne dépend pas du choix de $F$ parmi les $F_i$.
\end{proposition}


\begin{proof}
	Considérons deux cartes $F_1$ et $F_2$, ainsi que l'application $h=F_2^{-1}\circ F_1$. Écrivons le vecteur $\partial_x F_1\times\partial_yF_1$ en utilisant $F_1=F_2\circ h$. D'abord, par la règle de dérivation de fonctions composées,
	\begin{equation}
		\frac{ \partial (F_2\circ h) }{ \partial x }=\frac{ \partial F_2 }{ \partial x }\frac{ \partial h_1 }{ \partial x }+\frac{ \partial F_2 }{ \partial y }\frac{ \partial h_2 }{ \partial x }.
	\end{equation}
	Après avoir fait le même calcul pour $\frac{ \partial (F_2\circ h) }{ \partial y }$, nous pouvons écrire
	\begin{equation}
		\partial_x(F_2\circ h)\times\partial_y(F_2\circ h)=(\partial_xh_1\partial_xF_2+\partial_xh_2\partial_yF_2)\times(\partial_yh_1\partial_xF_2+\partial_yh_2\partial_yF_2).
	\end{equation}
	Dans cette expression, les facteurs $\partial_ih_j$ sont des nombres, donc ils se factorisent dans les produits vectoriels. En tenant compte du fait que $\partial_xF_2\times\partial_xF_2=0$ et $\partial_yF_2\times\partial_yF_2=0$, ainsi que de l'antisymétrie du produit vectoriel, l'expression se réduit à
	\begin{equation}
		\left( \frac{ \partial F_2 }{ \partial x }\times\frac{ \partial F_2 }{ \partial y } \right)(\partial_xh_1\partial_yh_2-\partial_xh_2\partial_yh_2).
	\end{equation}
	Par conséquent,
	\begin{equation}
		\frac{ \partial F_1 }{ \partial x }\times\frac{ \partial F_1 }{ \partial y } =\frac{ \partial (F_2\circ h) }{ \partial x }\times\frac{ \partial (F_2\circ h) }{ \partial y } =\left( \frac{ \partial F_2 }{ \partial x }\times\frac{ \partial F_2 }{ \partial y } \right)\det J_h.
	\end{equation}
	Donc, tant que $J_h$ est positif, les vecteurs unitaires correspondants au membre de gauche et de droite sont égaux.
\end{proof}

\begin{corollary}
	Si nous avons choisit un atlas orienté pour la variété $M$, nous avons une fonction continue $G\colon M\to \eR^3$ telle que $\| G(p) \|=1$ pour tout $p\in M$. Cette fonction est donnée par
	\begin{equation}		\label{DefCarteGOritn}
		G(F(x,y))=\frac{   \frac{ \partial F }{ \partial x }(x,y)\times\frac{ \partial F }{ \partial y }(x,y)   }{ \| \frac{ \partial F }{ \partial x }(x,y)\times\frac{ \partial F }{ \partial y }(x,y)\| }
	\end{equation}
	sur l'image de la carte $F$.
\end{corollary}

\begin{proof}
	La fonction $G$ est construite indépendamment sur chaque carte $F(W)$ en utilisant la formule \eqref{DefCarteGOritn}. Cette fonction est une fonction bien définie sur tout $M$ parce que nous venons de démontrer que sur $F_1(W_1)\cap F_2(W_2)$, les fonctions construites à partir de $F_1$ et à partir de $F_2$ sont égales.
\end{proof}

Il est possible que prouver, bien que cela soit plus compliqué, que la réciproque est également vraie.
\begin{proposition}
	Une variété $M$ de dimension $2$ dans $\eR^3$ est orientable si et seulement si il existe une fonction continue $G\colon M\to \eR^3$ telle que pour tout $p\in M$, le vecteur $G(p)$ soit de norme $1$ et normal à $M$ au point $p$.
\end{proposition}

%---------------------------------------------------------------------------------------------------------------------------
\subsection{Intégrale d'une fonction sur une variété}
%---------------------------------------------------------------------------------------------------------------------------

Nous supposons à présent que $M$ est une variété compacte de dimension $2$ dans $\eR^3$. La compacité fait que $M$ possède un atlas contenant un nombre fini de cartes $F_i\colon W_i\to F_i(W_i)$. 

Si $A\subset M$ est tel que pour chaque $i$, $A\cap F_i(W_i)=F_i(V_i)$ pour une ensemble $V_i$ mesurable dans $\eR^2$, alors nous considérons
\begin{equation}
	A_1=A\cap F_1(W_2)=F_1(V_1).
\end{equation}
Ensuite, nous construisons $A_2$ en considérant $F_A(W_2)$ et en lui retranchant $A_1$ :
\begin{equation}
	A_2=\big( A\cap F_2(W_2) \big)\cap F_1(V_1).
\end{equation}
En continuant de la sorte, nous construisons la décomposition
\begin{equation}
	A=A_1\cup\ldots\cup A_p
\end{equation}
de $A$ en ouverts disjoints, chacun de ouverts $A_p$ étant compris dans une carte.

Il est possible de prouver que dans ce cas, la définition suivante a un sens et ne dépend pas du choix de l'atlas effectué.
\begin{definition}
	Si $f\colon A\to \eR$ est une fonction continue, alors l'\defe{intégrale}{Intégrale!d'une fonction sur une variété} est le nombre
	\begin{equation}
		\int_Af=\sum_{i=1}^p\int_{A_i}fd\sigma_{F_i}.
	\end{equation}
\end{definition}

%+++++++++++++++++++++++++++++++++++++++++++++++++++++++++++++++++++++++++++++++++++++++++++++++++++++++++++++++++++++++++++
\section{Variétés et extrema liés}
%+++++++++++++++++++++++++++++++++++++++++++++++++++++++++++++++++++++++++++++++++++++++++++++++++++++++++++++++++++++++++++

\subsection{Introduction}
Soit $f : S^2 \to \eR$ une fonction définie sur la sphère usuelle
$S^2 \subset \eR^3$. Une question naturelle est d'estimer la
régularité de $f$ ; est-elle continue, dérivable, différentiable ? Il
n'existe pas de dérivée directionnelle étant donné que le quotient
différentiel
\begin{equation*}
  \frac{f(x + \epsilon u_1 ,y + \epsilon u_2) - f(x,y)}{\epsilon}
\end{equation*}
n'a pas de sens pour un point $(x + \epsilon u_1 ,y + \epsilon u_2)$
qui n'est pas --sauf valeurs particulières-- dans la surface. Pour la
même raison il n'est pas possible de parler de différentiabilité de
cette manière. Comment faire, sans devoir étendre le domaine de
définition de $f$ à un voisinage de la sphère ? Une solution possible
est de parler de la notion de variété.

Une variété est un objet qui ressemble, vu de près, à $\eR^m$ pour un
certain $m$. En d'autres termes, on imagine une variété comme un
recollement de morceaux de $\eR^m$ vivant dans un espace plus grand
$\eR^n$. Ces morceaux sont appelés des ouverts de carte, et
l'application qui exprime la ressemblance à $\eR^m$ est l'application
de carte.

%---------------------------------------------------------------------------------------------------------------------------
\subsection{Définition et propriétés}
%---------------------------------------------------------------------------------------------------------------------------

\begin{definition}
  Soit $\emptyset \neq M \subset \eR^n$, $1 \leq m < n$ et $k \geq
  1$. $M$ est une \Defn{variété de classe $C^k$ de dimension $m$} si
  pour tout $a \in \var M$, il existe un voisinage ouvert $U$ de $a$
  dans $\eR^n$, et un ouvert $V$ de $\eR^m$ tel que $U \cap \var M$
  soit le graphe d'une fonction $f : V \subset \eR^m \to \eR^{n-m}$
  de classe $C^1$, c'est-à-dire qu'il existe un réagencement des
  coordonnées $(x_{i_1}, \ldots, x_{i_m}, x_{i_{m+1}}, \ldots,
  x_{i_n})$ avec
  \begin{equation*}
    M \cap U = \left\{ (x_1, \ldots, x_n) \in \eR^n \tq
%      \begin{array}{l} % deux conditions
      (x_{i_1}, \ldots, x_{i_m}) \in V \quad \left\{\begin{array}{c!{=}l} % 1: equations
        x_{i_{m+1}} & f_1(x_{i_1}, \ldots, x_{i_m})\\
        \vdots & \vdots \\
        x_{i_n} & f_{n-m}(x_{i_1}, \ldots, x_{i_m})
      \end{array}\right.
%    \end{array}
    \right\}
  \end{equation*}
  où $V$ est un voisinage ouvert de $(a_{i_1}, \ldots, a_{i_m}) \in \eR^m$.
\end{definition}


Le cours regorge de théorèmes qui proposent des conditions équivalentes à la définition d'une variété. Celle que nous allons le plus utiliser est la suivante, de la page 268.

\begin{proposition}
	Soit $M\subset\eR^n$ et $1\leq m\leq n-1$. L'ensemble $M$ est une variété si et seulement si $\forall a\in M$, il existe un voisinage ouvert $\mU$ de $a$ dans $\eR^n$ et une application $F\colon W\subset\eR^m\to \eR^n$ où $W$ est un ouvert tels que
	\begin{enumerate}
		\item
			$F$ est un homéomorphisme de $W$ vers $M\cap\mU$,
		\item
			$F\in C^1(W,\eR^n)$,
		\item
			Le rang de $dF(w)\in L(\eR^m,\eR^n)$ est de rang maximum (c'est à dire $m$) en tout point $w\in W$.
	\end{enumerate}
\end{proposition}
Pour rappel, si $T\colon \eR^m\to \eR^n$ est une application linéaire, son \defe{rang}{Rang} est la dimension de son image. On peut prouver que si $A$ est la matrice d'une application linéaire, alors le rang de cette application linéaire est égal à la taille de la plus grande matrice carré de déterminant non nul contenue dans $A$.

La condition de rang maximum sert à éviter le genre de cas de la figure \ref{LabelFigExempleNonRang} qui représente l'image de l'ouvert $\mathopen] -1 , 1 \mathclose[$ par l'application $F(t)=(t^2,t^3)$.
\newcommand{\CaptionFigExempleNonRang}{Quelque chose qui n'est pas de rang maximum et qui n'est pas une variété.}
\input{Fig_ExempleNonRang.pstricks}
La différentielle a pour matrice
\begin{equation}
	dF(t)=(2t,3t^2).
\end{equation}
Le rang maximum est $1$, mais en $t=0$, la matrice vaut $(0,0)$ et son rang est zéro. Pour toute autre valeur de $t$, c'est bon.

Une autre caractérisation des variétés est donnée par la proposition suivante (proposition 3, page 274).
\begin{proposition}		\label{PropCarVarZerFonc}
	Soit $M\in \eR^n$ et $1\leq m\leq n-1$. L'ensemble $M$ est une variété si et seulement si $\forall a\in M$, il existe un voisinage ouvert $\mU$ de $a$ dans $\eR^n$ tel et une application $G\in C^1(\mU,\eR^{n-m})$ tel que
	\begin{enumerate}

		\item
			le rang de $dG(a)\in L(\eR^n,\eR^{n-m})$ soit maximum (c'est à dire $n-m$) en tout $a\in M$,
		\item
			$M\cap\mU=\{ x\in\mU\tq G(x)=0 \}$.

	\end{enumerate}
\end{proposition}


%---------------------------------------------------------------------------------------------------------------------------
\subsection{Espace tangent}
%---------------------------------------------------------------------------------------------------------------------------

Soit $M$, une variété dans $\eR^n$, et considérons un chemin $\gamma\colon I\to \eR^n$ tel que $\gamma(t)\in M$ pour tout $t\in I$ et tel que $\gamma(0)=a$ et que $\gamma$ est dérivable en $0$. La \defe{tangente}{Tangente à un chemin} au chemin $\gamma$ au point $a\in M$ est la droite
\begin{equation}
	s\mapsto a+s\gamma'(0).
\end{equation}
L'\defe{espace tangent}{Espace tangent} de $M$ au point $a$ est l'ensemble décrit par toutes les tangentes en $a$ pour tous les chemins $\gamma$ possibles.

\begin{proposition}			\label{PropDimEspTanVarConst}
	Une variété de dimension $m$ dans $\eR^n$ a un espace tangent de dimension $m$ en chacun de ses points.
\end{proposition}

%---------------------------------------------------------------------------------------------------------------------------
\subsection{Extrema liés}
%---------------------------------------------------------------------------------------------------------------------------
 
Soit $f$, une fonction sur $\eR^n$, et $M\subset \eR^n$ une variété de dimension $m$. Nous voulons savoir quelle sont les plus grandes et plus petites valeurs atteintes par $f$ sur $M$.

Pour ce faire, nous avons un théorème qui permet de trouver des extrema \emph{locaux} de $f$ sur la variété. Pour rappel, $a\in M$ est une \defe{extrema local de $f$ relativement}{Extrema local!relatif} à l'ensemble $M$ si il existe une boule $B(a,\epsilon)$ telle que $f(a)\leq f(x)$ pour tout $x\in B(a,\epsilon)\cap M$.

\begin{theorem}
	Soit $A\subset\eR^n$ et
	\begin{enumerate}

		\item
			une fonction (celle à minimiser) $f\in C^1(A,\eR)$,
		\item 
			des fonctions (les contraintes) $G_1,\ldots,G_k\in C^1(A,\eR)$,
		\item
			$M=\{ x\in A\tq G_i(x)=0\,\forall i\}$,
		\item
			un extrema local $a\in M$ de $f$ relativement à $M$.
	\end{enumerate}
	Supposons que les gradients $\nabla G_1(a)$, \ldots,$\nabla G_k(a)$ soient linéairement indépendants.

	Alors $a=(x_1,\ldots,x_n)$ est dans les solutions de $\nabla L=0$ où
	\begin{equation}
		L(x_1,\ldots,x_n,\lambda_1,\ldots,\lambda_k)=f(x_1,\ldots,x_n)+\sum_{i=1}^k\lambda_iG_i(x_1,\ldots,x_n)
	\end{equation}
\end{theorem}
La fonction $L$ est le \defe{lagrangien}{Lagrangien} du problème.

\href{http://www.sagenb.org/home/pub/353/}{En pratique}, les candidats extrema locaux sont tous les points où les gradients ne sont pas linéairement indépendants, plus tous les points donnés par l'équation $\nabla L=0$. Parmi ces candidats, il faut trouver lesquels sont maxima ou minima, locaux ou globaux.

L'existence d'extrema locaux se prouve généralement en invoquant de la compacité, et en invoquant le lemme suivant qui permet de réduire le problème à un compact.

\begin{lemma}		\label{LemmeMinSCimpliqueS}
	Soit $S$, un ensemble dans $\eR^n$ et $C$, un ouvert de $\eR^n$. Si $a\in\Int S$ est un minimum local relatif à $S\cap C$, alors il est un minimum local par rapport à $S$.
\end{lemma}

\begin{proof}
	Nous avons que $\forall x\in B(a,\epsilon_1)\cap S\cap C$, $f(x)\geq f(x)$. Mais étant donné que $C$ est ouvert, et que $a\in C$, il existe un $\epsilon_2$ tel que $B(a,\epsilon_2)\subset C$. En prenant $\epsilon=\min\{ \epsilon_1,\epsilon_2 \}$, nous trouvons que $f(x)\geq f(a)$ pour tout $x\in B(a,\epsilon)\cap(S\cap C)=B(a,\epsilon)\cap S$.
\end{proof}

\section{Intégrales curvilignes}
\label{secintcurvi}

\subsection{Chemins de classe \texorpdfstring{$C^1$}{C1}}

Soit $p, q\in \eR^n$. Un \defe{chemin}{Chemin} $C^1$ par morceaux joignant $p$ à $q$ est une application continue
\begin{equation}
  \gamma : [a,b] \to \eR^n \qquad \gamma(a) = p, \gamma(b) = q
\end{equation}
pour laquelle il existe une subdivision $a = t_0 < t_1 < \ldots < t_{r-1} < t_r = b$ telle que :
\begin{enumerate}
\item la restriction de $\gamma$ sur chaque ouvert $\mathopen]t_i,
  t_{i+1}\mathclose[$ est de classe $C^1$~;
\item pour tout $0 \leq i \leq r$, $\gamma^\prime$ possède une limite
  à gauche (sauf pour $i = 0$) et une limite à droite (sauf pour $i =
  r$) en $t_i$.
\end{enumerate}
Le \defe{chemin $\gamma$ est (globalement) de classe $C^1$}{Chemin!classe $C^2$} si la
subdivision peut être choisie de \og longueur\fg{} $r = 1$.

\begin{remark}
	Si $a$ et $b$ sont des points de
  $\eR^n$, on peut créer le chemin particulier
  \begin{equation*}
    \gamma : [0,1] \to \eR^n : t \mapsto (1-t)a + tb
  \end{equation*}
  qui relie ces points par un segment de droite.
\end{remark}

\subsection{Intégrer une fonction}
Soit $f : D \subset \eR^n \to \eR$ une fonction continue, et $\gamma
: [a,b] \to D$ un chemin $C^1$. On définit \Defn{l'intégrale de $f$
  sur $\gamma$} par
\begin{equation*}
  \int_\gamma f d s = \int_\gamma f = \int_a^b f(\gamma(t)) \norme{\gamma^\prime(t)} d t.
\end{equation*}

\begin{remark}
  Cette définition ne dépend pas de la paramétrisation choisie, ni du
  sens du chemin (échange entre point de départ et point d'arrivée).
\end{remark}

La formule qui donne la longueur d'un chemin est évidement l'intégrale de la fonction $1$ sur le chemin, c'est à dire
\begin{equation}
	L=\int_a^b\| \gamma'(t) \|dt.
\end{equation}
Si on veut savoir la longueur d'une courbe donnée sous la forme d'une fonction $y=y(x)$, un chemin qui trace la courbe est évidement donné par
\begin{equation}
	\gamma(t)=(t,y(t)),
\end{equation}
et le vecteur tangent au chemin est $\gamma'(t)=(1,y'(t))$. Donc
\begin{equation}
	\| \gamma'(t) \|=\sqrt{1+y'(t)^2},
\end{equation}
et 
\begin{equation}			\label{EqLongFonction}
	L=\int_a^b\sqrt{1+y'(t)^2}.
\end{equation}

\subsection{Intégrer un champ de vecteurs}
Un \Defn{champ de vecteur} est une application $G : \eR^n \to
\eR^n$. On définit l'intégrale de $G$ sur un chemin $\gamma : [a,b]
\to \eR^n$ par
\begin{equation*}
  \int_\gamma G \pardef \int_a^b \scalprod {G(\gamma(t))}{\gamma^\prime(t)} d t.
\end{equation*}

\begin{remark}
  Cette définition ne dépend pas de la paramétrisation choisie, mais
  le signe change selon le sens du chemin.
\end{remark}

%---------------------------------------------------------------------------------------------------------------------------
\subsection{Intégrer une forme différentielle sur un chemin}
%---------------------------------------------------------------------------------------------------------------------------

Une \defe{forme différentielle}{Forme!différentielle} sur $\eR^n$ est une application
\begin{equation}
	\begin{aligned}
		\omega\colon \eR^n&\to (\eR^n)^* \\
		x&\mapsto \omega_x 
	\end{aligned}
\end{equation}
qui à chaque point $x$ de $\eR^n$ associe une forme linéaire $\omega_x: \eR^n \to \eR$.

On sait que $\{ d x_i \}_{1\leq i\leq n}$ est une base de
${(\eR^{n})}^{*}$, donc toute forme différentielle s'écrit
\begin{equation*}
  \omega_x = \sum_{i=0}^n a_i(x) d x_i
\end{equation*}
où $a_1,\ldots,a_n$ sont les \Defn{composantes de $\omega$} dans la
base usuelle, et sont des fonctions à valeurs réelles. Pour un vecteur
$v = (v_1,\ldots,v_n)$ on a donc par définition de $d x_i$
\begin{equation*}
  \omega_x v = \sum_{i=0}^n a_i(x) v_i.
\end{equation*}

L'intégrale d'une forme différentielle sur un chemin est définie par
\begin{equation*}
  \int_\gamma \omega = \int_a^b \omega_{\gamma(t)}\gamma^\prime(t) d t
\end{equation*}

\begin{remark}
  Cette définition ne dépend pas de la paramétrisation choisie, mais
  le signe change selon le sens du chemin.
\end{remark}

\subsection{Lien entre forme différentielle et champ vectoriel}
Si $G$ est un champ de vecteurs, on peut définir la forme différentielle
\begin{equation*}
  \omega^G : \eR^n \to {(\eR^n)}^\ast : x \mapsto \left\lbrack \omega^G_x :
  \eR^n \to \eR : v \mapsto \omega^G_x v = \scalprod {G(x)}v \right\rbrack
\end{equation*}
et réciproquement, si $\omega_x = \sum_i a_i(x)d x_i$ est une forme
différentielle on définit le champ de vecteurs
\begin{equation*}
  G^\omega(x) = (a_1(x),\ldots,a_n(x)).
\end{equation*}

Avec ces définitions, pour un chemin $\gamma$ donné on a
\begin{equation*}
  \int_\gamma \omega^G = \int_\gamma G^\omega
\end{equation*}

%---------------------------------------------------------------------------------------------------------------------------
\subsection{Intégrer un champs de vecteurs sur un bord en $2D$}
%---------------------------------------------------------------------------------------------------------------------------

Si $D\subset\eR^2$ est tel que $\partial D$ est une variété de dimension $1$ et tel que $D$ accepte un champ de vecteur normal extérieur unitaire $\nu$. Si nous voulons définir 
\begin{equation}
	\int_{\partial D}G,
\end{equation}
le mieux est de prendre une paramétrisation $\gamma\colon \mathopen[ 0 , 1 \mathclose]\to \eR^2$ et de calculer
\begin{equation}
	\int_0^1 \langle G_{\gamma(t)}, \frac{ \dot\gamma(t) }{ \| \dot\gamma(t) \| }\rangle dt.
\end{equation}
Hélas, cette définition ne fonctionne pas parce que son signe dépend du sens de la paramétrisation $\gamma$. Si la paramétrisation tourne dans l'autre sens, il y a un signe de différence.

Nous allons définir
\begin{equation}		\label{EqIntVectbordDeux}
	\int_{\partial D}G=\int_0^1\langle G_{\gamma(t)}, T(t)\rangle dt
\end{equation}
où $T(t)=\dot\gamma(t)/\| \dot\gamma(t) \|$ et où $\gamma$ est choisit de telle façon à ce que la rotation d'angle $\frac{ \pi }{ 2 }$ amène $\nu$ sur $T$. Cela fixe le choix de sens.

Ce choix de sens aura des répercussions dans l'application de la formule de Green et du théorème de Stokes.

%---------------------------------------------------------------------------------------------------------------------------
\subsection{Intégrer une forme différentielle sur un bord en $2D$}
%---------------------------------------------------------------------------------------------------------------------------

Nous n'allons pas chercher très loin :
\begin{equation}
	\int_{\partial D}\omega=\int_{\partial D}\omega^{\sharp},
\end{equation}
c'est à dire que l'intégrale de la forme différentielle est celle du champ de vecteur associé. Le membre de droite est définit par \eqref{EqIntVectbordDeux}, avec le choix d'orientation qui va avec.

%---------------------------------------------------------------------------------------------------------------------------
\subsection{Intégrer une forme différentielle sur un bord en $3D$}
%---------------------------------------------------------------------------------------------------------------------------

Nous allons maintenant intégrer une forme différentielle sur certains chemins fermés dans $\eR^3$. Soit $F(D)\subset\eR^3$, une variété de dimension $2$ dans $\eR^3$ où $F\colon D\subset\eR^2\to \eR^3$ est la carte. Nous supposons que $D$ vérifie les hypothèses de la formule de Green. Alors nous définissons
\begin{equation}		\label{EqDefIntTroisForBord}
	\int_{F(\partial D)}\omega = \int_{\partial D} F^*\omega
\end{equation}
où $F^*\omega$ est la forme différentielle définie sur $\partial D$ par $(F^*\omega)(v)=\omega\big( dF(v) \big)$.

Cette définition est très abstraite, mais nous n'allons, en pratique, jamais l'utiliser, grâce au théorème de Stokes.

%---------------------------------------------------------------------------------------------------------------------------
\subsection{Intégrer d'un champ de vecteurs sur un bord en $3D$}
%---------------------------------------------------------------------------------------------------------------------------

Encore une fois, nous n'allons pas chercher bien loin :
\begin{equation}
	\int_{F(\partial D)G}=\int_{F(\partial D)}G^{\flat}
\end{equation}
où $G^{\flat}$ est la forme différentielle associée au champ de vecteur. Le membre de droite est définit par l'équation \eqref{EqDefIntTroisForBord}.

%+++++++++++++++++++++++++++++++++++++++++++++++++++++++++++++++++++++++++++++++++++++++++++++++++++++++++++++++++++++++++++
\section{Intégrales de surface}
%+++++++++++++++++++++++++++++++++++++++++++++++++++++++++++++++++++++++++++++++++++++++++++++++++++++++++++++++++++++++++++

%---------------------------------------------------------------------------------------------------------------------------
\subsection{Intégrale d'une fonction}
%---------------------------------------------------------------------------------------------------------------------------
\label{secintsurfaciques}
Soit $M$ une variété de dimension $n$ dans $\eR^m$. Soit $F : W \subset \eR^n \to M$ une paramétrisation d'un ouvert relatif de $M$.  

Si $f$ est une fonction définie sur un sous-ensemble $A \subset F(W)$ tel que $F^{-1}(A)$ est mesurable, l'\Defn{intégrale de $f$ sur $A$} est définie par
\begin{equation*}
  \int_A f = \int_{F^{-1}(A)} f(F(w)) \sqrt{\det(\transpose{J_F(w)} {J_F(w)})} dw
\end{equation*}
où l'intégrale est l'intégration usuelle (de Lebesgue) sur $F^{-1}(A) \subset \eR^n$. On écrit parfois cette intégrale $\int_{F^{-1}(A)} f(F(w)) d\sigma$ où
\begin{equation*}
  d\sigma = \sqrt{\det(\transpose{J_F(w)} {J_F(w)})} dw
\end{equation*}
est l'\Defn{élément infinitésimal de volume} de la variété. 

Si $m = 3$ et $n = 2$, l'élément infinitésimal de volume vaut
\begin{equation*}
  d \sigma = \norme{\pder F {w_1} \wedge \pder F {w_2}} dw
\end{equation*}
où $\wedge$ représente le produit vectoriel dans $\eR^3$, et $(w_1,w_2)$ sont les coordonnées sur $W \subset \eR^2$. Dans la suite, nous ne regarderons plus que ce cas.

\subsection{Intégrale d'un champ de vecteurs}
Dans l'intégration curviligne, on a noté que si l'intégrale d'une fonction ne dépendait pas de l'orientation du chemin, l'intégrale d'un champ de vecteurs ou d'une forme différentielle en dépendait. Ce problème d'orientation apparait également dans l'intégration sur des surfaces de l'espace.

%% Page 530, exemple 4
Une \Defn{orientation} sur une surface $S \subset \eR^3$ est le choix
d'un champ de vecteurs continu $\nu : S \to \eR^3$ dont la norme en
tout point de $S$ vaut $1$. On remarque qu'ayant fait un tel choix
d'orientation $\nu(x)$ en un point $x$, le seul autre choix possible
en $x$ est $-\nu(x)$.
%% Page 
Si $S$ est le bord d'un ouvert $D \subset \eR^3$, l'\Defn{orientation
  induite par $D$ sur $S$} est, si elle existe, l'orientation qui
pointe hors de $D$ en tout point de $S$. Plus précisément, il faut que
pour tout $x \in D$ il existe $\epsilon > 0$ vérifiant, pour tout $0 <
t < \epsilon$, la relation $t \nu(x) \notin D$. Dans ce cas, le champ
de vecteurs $\nu$ est appelé le \Defn{vecteur normal unitaire
  extérieur} à $D$ et il est forcément unique.

Soit $G$ un champ de vecteurs défini sur une surface orientée par un
champ $\nu$. L'intégrale de $G$ sur $S$, aussi appelée le \Defn{flux
de $G$ à travers $S$}, est
\begin{equation}\label{eqflux-star}
  \iint_S G \cdot d S \pardef \iint_S \scalprod{G}{\nu} d \sigma.
\end{equation}
Si on suppose que la surface est paramétrisée par une application
\begin{equation*}
  F : W \subset \eR^2 \to \eR^3 : (u,v) \mapsto (F_1(u,v),F_2(u,v),F_3(u,v))
\end{equation*}
alors un vecteur unitaire $\nu$ peut s'écrire sous la forme
\begin{equation*}
  \nu = \frac{\pder F u \wedge \pder F v}{\norme{\pder F u \wedge \pder F v}}
\end{equation*}
et grâce à cette paramétrisation l'intégrale \eqref{eqflux-star}
devient
\begin{equation*}
  \iint_S G \cdot d S = \iint_W \scalprod{G(F(u,v))}{\pder F u \wedge \pder F v} d u
  d v.
\end{equation*}
où on utilise l'expression de $d \sigma$ obtenue précédemment dans le
cas qui nous intéresse (surface dans l'espace).

%+++++++++++++++++++++++++++++++++++++++++++++++++++++++++++++++++++++++++++++++++++++++++++++++++++++++++++++++++++++++++++
\section{Divergence, Green, Stokes}
%+++++++++++++++++++++++++++++++++++++++++++++++++++++++++++++++++++++++++++++++++++++++++++++++++++++++++++++++++++++++++++


Le théorème de Stokes (et ses variations) peut se voir comme une généralisation du théorème fondamental du calcul différentiel et intégral qui stipule que
\begin{equation*}
	\int_a^b f^\prime(x) d x = f(b) - f(a)
\end{equation*}
c'est-à-dire qui relie l'intégrale de $f^\prime$ sur $I = [a,b]$ aux valeurs de $f$ sur le bord $\partial I = \{a,b\}$. Le signe $-$ qui apparait vient de l'orientation ; celle-ci requiert de la prudence dans l'utilisation des théorèmes.

Voici, pour votre culture générale, un énoncé général :
\begin{theorem}
	Si $M$ est une variété orientable de dimension $n$ avec un bord noté $\partial  M$, alors pour toute forme différentielle $\omega$ de degré $n-1$ on a 
	\begin{equation*}
		\int_{ M} d \omega = \int_{\partial  M} \omega.
	\end{equation*}
	où $d \omega$ désigne la différentielle extérieure de $\omega$.
\end{theorem}
Nous allons maintenant voir quelque cas particuliers. 


\subsection{Théorème de la divergence}

Si nous considérons une surface dans $\eR^n$ et un champ de vecteurs, il est bon de se demander quelle \og quantité de vecteurs\fg{} traverse la surface. Soit $D$, un ouvert borné de $\eR^n$ telle que $\partial D$ soit une variété de dimension $n-1$, et $G$, un champ de vecteurs défini sur $\bar D$. Afin de compter combien de $G$ traverse $\partial D$, il faudra faire en sorte de ne considérer que la composante de $G$ normale à $\partial D$ : pas question d'intégrer par exemple la norme de $G$ sur $\partial D$.

Comme nous le savons, la composante du vecteur $v$ dans la direction $w$ est le produit scalaire $v\cdot 1_w$ où $1_w$ est le vecteur de norme $1$ dans la direction $w$. Nous allons donc introduire le concept de vecteur normal extérieur. Soit $x\in\partial D$ et $\nu\in\eR^n$, nous disons que $\nu$ est un \defe{vecteur normal extérieur}{Normal extérieur!vecteur} de $\partial D$ si
\begin{enumerate}

	\item
		$\langle \nu, v\rangle =0$ pour tout vecteur tangent $v$ à $\partial D$ au point $x$. Pour rappel, $\partial D$ étant une variété de dimension $n-1$, il y a $n-1$ tels vecteurs $v$ linéairement indépendants.
	
	\item
		Il existe un $\delta>0$ tel que $\forall t\in\mathopen] 0 , \delta \mathclose[$, nous avons $c+t\nu\notin \bar D$ et $x-t\nu\in D$.
 
\end{enumerate}

Nous pouvons maintenant définir le concept de flux. Soit $D\subset \eR^n$ tel que $\partial D$ soit une variété de dimension $n-1$ qui admette un vecteur normal extérieur $\nu(x)$ en chaque point. Soit aussi $G\colon \bar D\to \eR^n$, un champ de vecteur de classe $C^1$. Le \defe{flux}{Flux!d'un champ de vecteur} de $G$ au travers de $\partial D$ est le nombre
\begin{equation}
	\int_{\partial D}\langle G(x), \nu(x)\rangle d\sigma(x).
\end{equation}

Cette intégrale est en général très compliquée à calculer parce qu'il faut trouver le champ de vecteur normal, puis une paramétrisation de la surface $\partial D$ et ensuite appliquer la méthode décrite au point \ref{secintsurfaciques}. 

Heureusement, il y a un théorème qui nous permet de calculer plus facilement : sans devoir trouver de vecteurs normaux.


Il n'est pas plus contraignant d'énoncer ce théorème dans le cadre d'une hypersurface de $\eR^n$, ce que nous faisons donc~:
\begin{theorem}[Formule de la divergence]
	Soit $D$ un ouvert borné de $\eR^n$ dont le bord est \og assez régulier par morceaux\fg{}, c'est-à-dire~:
	\begin{equation}
		\partial D = A_1 \cup \ldots A_p \cup N
	\end{equation} 
	où
	\begin{enumerate}
		\item $A_1, \ldots, A_p, N$ sont deux à deux disjoints,
		\item pour tout $i \leq p$, $A_i$ est un ouvert relatif d'une certaine variété $M_i$ de dimension $(n-1)$
		\item $\bar A_i \subset M_i$
		\item $N$ est un compact contenu dans une réunion finie de variétés de dimensions $(n-2)$.
	\end{enumerate}
	Supposons également qu'en chaque point de $A_1 \cup \ldots \cup A_p$ il existe un vecteur normal extérieur $\nu$.
	
	Si $G$ est un champ de vecteurs de classe $C^1$ sur $\bar D$ alors
	\begin{equation}
		\int_D \nabla\cdot G = \sum_{i=1}^p \int_{A_i} \scalprod{G}{\nu}.
	\end{equation}
	L'intégrale du membre de gauche est l'intégrale sur un ouvert d'une simple fonction.
\end{theorem}

\subsection{Formule de Green}
La formule de Green est un cas particulier du théorème de la divergence dans
le cas $n = 2$, légèrement reformulé~:
\begin{theorem}
	Soit $D \subset \eR^2$ ouvert borné tel que son bord est est la réunion finie d'un certain nombre de chemins de classe $C^1$ de Jordan réguliers.  Supposons qu'en chaque point de son bord, $D$ possède un vecteur normal unitaire extérieur $\nu$. Soient $P$ et $Q$ deux fonctions réelles de classe $C^1$ sur $\bar D$. Alors
  \begin{equation*}
    \iint_D (\partial_xQ - \partial_yP)dx\,dy = \oint_{\partial D}
    Pd x + Q d y
  \end{equation*}
  où chaque chemin $\gamma$ formant le bord de $D$ est orienté de
  sorte que $T \nu = \frac{\dot\gamma}{\norme{\dot\gamma}}$ où $T$
  représente la rotation d'angle $+\frac\pi2$.
\end{theorem}

Pour rappel, une chemin $\gamma\colon \mathopen[ 0 , 1 \mathclose]\to \eR^n$ est \defe{régulier}{Régulier!chemin} si il est $C^1$ et si $\gamma(t)\neq 0$ pour tout $r$. Le chemin est de \defe{Jordan}{Jordan!chemin} si $\gamma(1)=\gamma(0)$ et si $\gamma\colon \mathopen[ a , b [\to \eR^n$ est injective.

\subsection{Formule de Stokes}
\label{secstokesusuel}
La formule de Stokes est la version classique, qui permet d'exprimer la circulation d'un champ de vecteur le long d'une courbe de $\eR^3$ comme le flux de son rotationnel à travers n'importe quel surface dont le bord est la courbe. La version présentée ici suppose que la surface peut se paramétrer en un seul morceau~:
\begin{theorem}
  Soit $F : W\subset \eR^2 \to \eR^3$ une paramétrisation (carte) d'une surface dans $\eR^3$, supposée de classe $C^2$. Soit $D$ un ouvert de $\eR^2$ vérifiant les hypothèses de la formule de Green, et tel que $\bar D \subset W$. Soit $G$ un champ de vecteurs de classe $C^1$ défini sur $F(\bar D)$, et soit $N$ le champ normal unitaire donné par la paramétrisation
  \begin{equation}		
	N = \frac{\pder F u \wedge \pder F v}{\norme{\pder F u \wedge \pder F v}}
\end{equation}
  alors
  \begin{equation}\label{EqStokesTho}
    \iint_{F(D)} \scalprod{\rot G}{N} d\sigma_F = \int_{F(\partial D)} G
  \end{equation}
  où les chemins formant le bord $\partial D$ sont orientés comme dans le théorème de Green.
\end{theorem}
Notons, juste pour avoir une bonne nouvelle de temps en temps, que 
\begin{equation}
	d\sigma_F=\left\| \frac{ \partial F }{ \partial u }\times\frac{ \partial F }{ \partial v }  \right\|dudv,
\end{equation}
mais cette norme apparaît exactement au dénominateur de $N$. Il ne faut donc pas la calculer parce qu'elle se simplifie.

Sous forme un peu plus physicienne\footnote{et surtout plus explicite.}, la formule \eqref{EqStokesTho} s'écrit
\begin{equation}
	\int_{F(D)}\langle \nabla\times G, N(x)\rangle\, d\sigma_F(x)=\int_{F(\gamma)}\langle G, T\rangle\, ds
\end{equation}
où $T$ est le vecteur unitaire tangent à $F(\gamma)$.

%///////////////////////////////////////////////////////////////////////////////////////////////////////////////////////////
\subsubsection{Quelle est la bonne orientation ?}
%///////////////////////////////////////////////////////////////////////////////////////////////////////////////////////////

Le signe du vecteur normal $N$ dépend du choix de l'ordre des coordonnées dans la carte. Supposons que je veuille paramétrer la surface $x^2+y^2=1$, $z=1$. Nous prenons naturellement comme carte le cercle $C$ de rayon $1$ dans $\eR^2$ et la carte
\begin{equation}
	F(r,\theta)=\begin{pmatrix}
		r\cos\theta	\\ 
		r\sin\theta	\\ 
		1	
	\end{pmatrix}.
\end{equation}
Mais nous aurions aussi pu mettre les coordonnées $r$ et $\theta$ dans l'autre ordre :
\begin{equation}
	\tilde F(\theta,r)=\begin{pmatrix}
		r\cos\theta	\\ 
		r\sin\theta	\\ 
		1	
	\end{pmatrix}.
\end{equation}
Les vecteurs normaux ne sont pas les même : la carte $F$ donnera $\partial_rF\times\partial_{\theta}F$, tandis que l'autre donnera $\partial_{\theta}\tilde F\times\partial_r\tilde F$. Le signe change !

Il faut savoir laquelle choisir. Le cercle $C\subset \eR^2$ a une orientation donnée par le théorème de Green. Nous choisissons l'ordre des coordonnées pour que $1_{\theta}$ et $1_{r}$ soient dans la même orientation que les vecteurs $\nu$ et $T$ tels que donnés par le théorème de Green, et tels que dessinés sur la figure \ref{LabelFigCercleTnu}.
\newcommand{\CaptionFigCercleTnu}{L'orientation sur le cercle. Si nous les prenons dans l'ordre, les vecteurs $(1_r,1_{\theta})$ ont la même orientation que celle donnée par les vecteurs $(\nu,T)$ donnés par la convention de Green.}
\input{Fig_CercleTnu.pstricks}

Plus généralement, nous choisissons l'ordre des coordonnées $u$ et $v$ pour que la base $(1_u,1_v)$ ait la même orientation que $(\nu,T)$ où $T$ a le sens convenu dans le théorème de Green.

%+++++++++++++++++++++++++++++++++++++++++++++++++++++++++++++++++++++++++++++++++++++++++++++++++++++++++++++++++++++++++++
					\section{Un petit extra}
%+++++++++++++++++++++++++++++++++++++++++++++++++++++++++++++++++++++++++++++++++++++++++++++++++++++++++++++++++++++++++++

Soit $f$ une fonction de $\eR$ dans $\eR$. Supposons que 
\begin{enumerate}

\item		\label{ItemExtrai}
$f(1)=1$,

\item		\label{ItemExtraii}
$f(x+y)=f(x)+f(y)$ pour tout réels $x$ et $y$.

\end{enumerate}
Nous pouvons montrer\footnote{et toi, tu le peux ?} que la seule fonction {\it continue} qui possède ces propriétés est la fonction identité $f(x)=x$ pour tout $x\in\eR$.

De la même manière, il est aisé de voir que les seules applications linéaires de $\Rn$ dans $\Rn$ sont de la forme 
\begin{equation}
	f(x)=Ax
\end{equation}
pour une constante réelle $A$. Une question naturelle qu'on peut alors se poser est la suivante: 
\begin{quote} 
	Est-il possible de définir une fonction non continue ayant les propriétés \ref{ItemExtrai} et \ref{ItemExtraii} ?
\end{quote}
En fait, il est possible de démontrer que si $E$ est un espace vectoriel de dimension finie, alors toute application linéaire $f:E\rightarrow  F$ (où $F$ est un espace vectoriel) sera continue. Ceci ne reste plus vrai si l'espace vectoriel $E$ est de dimension infinie. Donc une manière de trouver une réponse positive à la question posée plus haut, serait de voir $\Rn$ comme espace vectoriel de dimension infinie. Après un peu de réflexion, la réponse est venue à nous (merci à Nicolas et à Samuel). 



Si nous admettons l'\href{http://fr.wikipedia.org/wiki/Axiome_du_choix}{axiome du choix}, alors nous pouvons appliquer le théorème de Zorn et nous savons que tout espace vectoriel admet une base. En particulier, l'ensemble des réels vu comme espace vectoriel sur $\Qn$ admet une base, i.e. $\exists (e_i)_{i\in I}$  des éléments de $\Rn$ tels que tout réel s'écrit comme combinaison linéaire à coefficients rationnels  de ces $e_i$, i.e.
\begin{equation}
	\forall r \in \Rn, \exists (\lambda_i)_{i\in I} \text{ des éléments de } \Qn \text{ tels que  } r = \sum_{i\in I} \lambda_i e_i.
\end{equation}
Utilisons cette base pour définir une fonction $h$ de la manière suivante.
\begin{equation}
\forall i \in I, \mbox{ on définit } h(e_i) = \alpha_i
\end{equation}
 où les $\alpha_i$ doivent être bien choisis dans $\Rn$. Pour satisfaire la propriété \ref{ItemExtrai}, choisissons sans perte de généralité $e_1 = 1$ et $h(e_1) = 1$.  Ajoutons à cette propriété la linéarité en imposant que 
\begin{equation}
h(\sum \lambda_i e_i) = \sum \lambda_i \alpha_i.
\end{equation}
Les équations (1) et (2) nous permettent de voir que, moyennant le choix des $\alpha_i$, la fonction $h$ est bien définie sur $\Rn$ et linéaire. Il est clair que si nous prenons par exemple
$$\alpha_i=e_i\;\forall i \in I$$ 
nous obtenons que la fonction $h$ est en fait la fonction identité sur $\Rn$. Par contre, si nous définissons la fonction $h$ comme satisfaisant la propriété (2) et si nous choisissons les $\alpha_i$ dans (1) de la manière suivante 
\begin{equation}
	\begin{aligned}[]
		h(e_1)	&= e_2\\
		h(e_2)	&= e_1\\
		h(e_i)	&= e_i	&&\forall i\in I\setminus\{ 1,2 \}
	\end{aligned}
\end{equation}
alors la fonction ainsi obtenue est  linéaire et bien définie mais n'est plus l'identité. Donc nous avons trouvé une application linéaire de $\Rn$ dans $\Rn$ qui n'est pas continue.  

\begin{exercice}
 Trouver d'autres exemples d'applications linéaires non continues (pas nécessairement des transformations de $\Rn$). 		
\end{exercice}

%+++++++++++++++++++++++++++++++++++++++++++++++++++++++++++++++++++++++++++++++++++++++++++++++++++++++++++++++++++++++++++
\section{Preuves de certains résultats cités}
%+++++++++++++++++++++++++++++++++++++++++++++++++++++++++++++++++++++++++++++++++++++++++++++++++++++++++++++++++++++++++++

%---------------------------------------------------------------------------------------------------------------------------
\subsection{Preuve du critère d'équivalence}
%---------------------------------------------------------------------------------------------------------------------------
\label{PgPreuveCritEquiv}

Prouvons le critère d'équivalence mentionné à la page \pageref{PgCritEquiv}. La preuve est tirée de celle donnée dans \cite{TrenchRealAnalisys}.

\begin{enumerate}
	\item
		Le fait que la suite $a_n/b_n$ converge vers $\alpha$ signifie que tant sa limite supérieure que sa limite inférieure convergent vers $\alpha$. En particulier la suite $\frac{ a_n }{ b_n }$ est bornée vers le haut et vers le bas. À partir d'un certain rang $N$, il existe $M$ tel que 
		\begin{equation}
			\frac{ a_n }{ b_n }<M
		\end{equation}
		et il existe $m$ tel que
		\begin{equation}
			\frac{ a_n }{ b_n }>m.
		\end{equation}
		Nous avons donc $a_n<Mb_n$ et $a_n>mb_n$. La série de $(a_n)$ converge donc si et seulement si la série de $(b_n)$ converge.
	\item
		Si $\alpha=0$, cela signifie que pour tout $\epsilon$, il existe un rang tel que $\frac{ a_n }{ b_n }<\epsilon$, et donc tel que $a_n<\epsilon b_k$. La suite de $(a_i)$ converge donc dès que la suite de $(b_i)$ converge.
	\item
		Pour tout $M$, il existe un rang dans la suite à partir duquel on a $\frac{ a_i }{ b_i }>M$, et donc $a_k>Mb_k$. Si la série de $(b_k)$ diverge, la série de $(a_k)$ doit également diverger.
\end{enumerate}

%---------------------------------------------------------------------------------------------------------------------------
\subsection{Preuve du critère du quotient}
%---------------------------------------------------------------------------------------------------------------------------
\label{PgPreuveCritQuotient}

Cette preuve est tirée de \cite{KeislerElemCalculus}.

\begin{enumerate}
	\item
		Soit $b$ tel que $L<b<1$. À partir d'un certain rang $K$, on a $\left| \frac{ a_{i+1} }{ a_i } \right| <b$. En particulier,
		\begin{equation}
			| a_{K+1} |<b| a_K |,
		\end{equation}
		et pour $a_{K+2}$ nous avons
		\begin{equation}
			| a_{K+2} |<b| a_{K+1} |<b^2| a_K |.
		\end{equation}
		Au final,
		\begin{equation}
			| a_{K+n} |<b^n| a_K |.
		\end{equation}
		Étant donné que la série $\sum_{n\geq K}b^n$ converge (parce que $b<1$), la queue de suite $\sum_{i\geq K}a_i$ converge, et par conséquent la suite au complet converge.
	\item
		Si $L>1$, on a
		\begin{equation}
			| a_K |<| a_{K+1} |<| a_{K+2} |<\ldots
		\end{equation}
		Il est donc impossible que la suite $(a_i)$ converge vers zéro. La série ne peut donc pas converger.
	\item
		Par exemple la suite harmonique $a_n=\frac{1}{ n }$ vérifie $L=1$, mais la série ne converge pas. Par contre, la suite $a_n=\frac{ 1 }{ n^2 }$ vérifie aussi le critère avec $L=1$ tandis que la série $\sum_n\frac{1}{ n^2 }$ converge.
\end{enumerate}

%---------------------------------------------------------------------------------------------------------------------------
\subsection{Preuve du critère de la racine}
%---------------------------------------------------------------------------------------------------------------------------
\label{PgPreuveCritRacine}
La preuve est tirée de \cite{TrenchRealAnalisys}.

\begin{enumerate}
	\item
		Si $L<1$, il existe un $r\in \mathopen] 0 , 1 \mathclose[$ tel que $| a_n |^{1/n}<r$ pour les grands $n$. Dans ce cas, $| a_n |<r^{n}$, et la série converge absolument parce que la série $\sum_nr^n$ converge du fait que $r<1$.
	\item
		Si $L>1$, il existe un $r>1$ tel que $| a_n |^{1/n}>r>1$. Cela fait que $| a_n |$ prend des valeurs plus grandes que $n$ pour une infinité de termes. Le terme général $a_n$ ne peut donc pas être une suite convergente. Par conséquent la suite diverge au sens où elle ne converge pas.

\end{enumerate}
