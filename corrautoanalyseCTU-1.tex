% This is part of Analyse Starter CTU
% Copyright (c) 2014
%   Laurent Claessens,Carlotta Donadello
% See the file fdl-1.3.txt for copying conditions.

\begin{corrige}{autoanalyseCTU-1}

\begin{center}
   \input{Fig_HLJooGDZnqF.pstricks}
\end{center}

Le tableau de variations est ceci :
\begin{equation*}
    \begin{array}[]{c|ccccccccc}
        x&-\infty&&0^-&0&&1&1^+&&\infty\\
        \hline
        &&&&&&&&&+\infty\\
        &&&&&&&&\nearrow&\\
        &&&&3&&&3&&\\
        f(x)&&&&&\searrow&&&&\\
        &&&0&&&0&&&\\
        &&\nearrow&&&&&&&\\
        &-\infty&&&&&&&&\\
    \end{array}
\end{equation*}
les colonnes \( 0^-\) et \( 1^+\) ne correspondent pas à des valeurs effectivement atteintes par la fonction, mais seulement les limites.

En ce qui concerne les antécédents, nous avons de la chance : la fonction est injective. Aucun nombre ne possède plusieurs antécédents. Il est donc possible d'écrire la fonction réciproque \( f^{-1}\) de la façon suivante :
\begin{equation}
    f^{-1}(y)=\begin{cases}
    2y   &   \text{si \( y\in\mathopen] -\infty , 0 \mathclose[\)}\\
        1-\frac{ y }{ 3 }    &    \text{si \( y\in\mathopen[ 0 , 3 \mathclose]\)}\\
    +\frac{ y }{ 3 }    &    \text{si \( y\in\mathopen] 3 , \infty \mathclose[\)}.
    \end{cases}
\end{equation}

\end{corrige}   
