%+++++++++++++++++++++++++++++++++++++++++++++++++++++++++++++++++++++++++++++++++++++++++++++++++++++++++++++++++++++++++++
\section{Suites numériques}
%+++++++++++++++++++++++++++++++++++++++++++++++++++++++++++++++++++++++++++++++++++++++++++++++++++++++++++++++++++++++++++

Une \defe{suite numérique}{suite numérique} est une application $x\colon \eN\to \eR$. Une telle application sera notée $(x_n)$. L'élément numéro $k$ de la suite sera noté $x_k$.

\begin{definition}[Limite d'une suite numérique]	\label{DefLimiteSuiteNum}
	Nous disons que la suite $(x_n)$ est une suite \defe{convergente}{convergence!suite numérique} si il existe un réel $\ell$ tel que
	\begin{equation}		\label{EqDefLimSuite}
		\forall \varepsilon>0,\,\exists N\in\eN\tq\forall n\geq N,\,| x_n-\ell |<\varepsilon.
	\end{equation}
	Dans ce cas, le nombre $\ell$ est nommé \defe{limite}{limite!suite numérique} de la suite $(x_n)$. Nous dirons aussi souvent que la suite \defe{converge}{convergence de suite} vers le nombre $\ell$.
\end{definition}
	Une façon équivalente d'exprimer le critère \eqref{EqDefLimSuite} est de dire que pour tout $\varepsilon$ positif, il existe un rang $N\in\eR$ tel que l'intervalle $\mathopen[ \ell-\varepsilon , \ell+\varepsilon \mathclose]$ contient tous les termes $x_n$ au-delà de $N$.

Il est à noter que le rang $N$ dont il est question dans la définition de suite convergente dépend de $\varepsilon$.

\begin{example}
	Quelque suites usuelles.
	\begin{enumerate}
		\item
			La suite $x_n=\frac{1}{ n }$ converge vers $0$.
		\item
			La suite $x_n=(-1)^n$ ne converge pas.
	\end{enumerate}
\end{example}

Une suite est dite \defe{contenue}{} dans un ensemble $A$ si $x_n\in A$ pour tout $n$. Une suite est \defe{bornée supérieurement}{bornée!suite} si il existe un $M$ tel que $x_n\leq M$ pour tout $n$. De la même manière, la suite est bornée inférieurement si il existe un $m$ tel que $x_n\geq m$ pour tout $n$.

Le lemme suivant est souvent utilisé pour prouver qu'une suite est convergente.
\begin{lemma}		\label{LemSuiteCrBorncv}
	Une suite croissante et bornée supérieurement converge. Une suite décroissante bornée inférieurement est convergente.
\end{lemma}

%+++++++++++++++++++++++++++++++++++++++++++++++++++++++++++++++++++++++++++++++++++++++++++++++++++++++++++++++++++++++++++
\section{Maximum, majorant, supremum et compagnie}
%+++++++++++++++++++++++++++++++++++++++++++++++++++++++++++++++++++++++++++++++++++++++++++++++++++++++++++++++++++++++++++

Lorsque vous lisez que la charge maximale d'un camion est de \unit{2.5}{\ton}, est-ce que cela veut dire que vous pouvez y mettre \unit{2.5}{\ton}, mais qui si un oiseau se pose dessus, le camion s'effondre ? Ou bien est-ce que cela signifie qu'à \unit{2.5}{\ton} le camion s'écroule, mais que toute charge inférieure est valable ?

C'est à cette rude question que nous allons nous attaquer maintenant.

\begin{definition}
Soit une partie $A$ de $\eR$. Nous disons qu'un nombre $M$ est un \defe{majorant}{majorant} de $A$ si $M$ est plus grand ou égal que tous les éléments de $A$, c'est à dire si
\begin{equation}
	\forall a\in A,\, M\geq a.
\end{equation}
Un \defe{minorant}{minorant} de $A$ est un nombre $m$ tel que 
\begin{equation}
	\forall a\in A,\, m\leq a.
\end{equation}
\end{definition}

\begin{definition}		\label{DefSupeA}
Soit $A$ une partie majorée de $\eR$. Le \defe{supremum}{supremum} de $A$ est le plus petit des majorants, c'est à dire le nombre $M$ tel que
\begin{enumerate}
	\item
		$M\geq x$ pour tout $x\in A$,
	\item
		pour tout $\varepsilon$, le nombre $M-\varepsilon$ n'est pas un majorant de $a$, c'est à dire qu'il existe un élément $x\in A$ tel que $x>M-\varepsilon$.
\end{enumerate}
Nous notons $\sup A$ le supremum de $A$.

De la même façon, \defe{l'infimum}{infimum} de $A$, noté $\inf A$, est le plus grand de ses minorants. 
\end{definition}
Par convention, si la partie n'est pas bornée vers le haut, nous dirons que son supremum n'existe pas, ou bien qu'il est égal à $+\infty$, suivant les contextes. Pour votre culture générale, sachez toutefois que $\infty\notin\eR$.

La définition est justifiée par le lemme \ref{LemInfUnique} et la proposition \ref{PropBorneSupInf}. Le premier montre que si $A$ possède un infimum, alors il est unique, tandis que le second montre que toute partie majorée de $\eR$ accepter un supremum, et que toute partie minorée accepte un infimum.
\begin{lemma}		\label{LemInfUnique}
	Soit $A$ une partie de $\eR$. Supposons que $m_1$ et $m_2$ soient deux nombres qui vérifient les propriétés de l'infimum de $A$. Alors $m_1=m_2$.
\end{lemma}

\begin{proof}
	Si $_1\neq m_2$, nous pouvons supposer $m_2>m_1$. Dans ce cas, étant donné que $m_1$ est un infimum, $m_2$ ne peut pas minorer $A$, et donc ne peut pas être un infimum.
\end{proof}

\begin{proposition}		\label{PropBorneSupInf}
	Tout sous-ensemble de $\eR$ borné vers le bas possède un infimum; tout sous-ensemble de $\eR$ borné vers le haut possède un supremum.
\end{proposition}

La preuve qui suit est proche de celle donnée par Wikipédia  dans l'article \wikipedia{en}{http://en.wikipedia.org/wiki/Least_upper_bound_principle}{Least uppert bound principle}.

\begin{proof}
	Soit $A$, une partie de $\eR$. Nous allons trouver son infimum en suivant une méthode de dichotomie. Pour cela nous allons construire trois suites en même temps de la façon suivante. D'abord nous choisissons un point $x_0$ de $A$ et un point $x_1$ qui minore $A$ (qui existe par hypothèse) :
	\begin{equation}
		\begin{aligned}[]
			x_0&\text{ est un élément de $A$},\\
			x_1&\text{ est un minorant de $A$},\\
			a_0&=x_0\\
			b_0&=x_1\\
			b_1&=x_1.
		\end{aligned}
	\end{equation}
	Ensuite, nous faisons la récurrence suivante :
	\begin{equation}
		\begin{aligned}[]
			x_{n+1}&=\frac{ a_n+b_n }{2},\\
			a_{n+1}&=\begin{cases}
				a_{n}	&	\text{si $x_{n+1}$ minore $A$}\\
				x_{n+1}	&	 \text{sinon},
			\end{cases}\\
			b_{n+1}&=\begin{cases}
				x_{n+1}	&	\text{si $x_{n+1}$ minore $A$}\\
				b_n	&	 \text{sinon}.
			\end{cases}
		\end{aligned}
	\end{equation}
    Nous allons montrer que \( a_n\) et \( (b_n)\) sont des suites convergentes de même limite et que cette limite est l'infimum de \( A\).

	Soit $n\in\eN$; il y a deux possibilités. Soit $a_n=a_{n-1}$ et $b_n=x_n$, soit $a_n=x_n$ et $b_n=b_{n-1}$. Supposons que nous soyons dans le premier cas (le second se traite de façon similaire). Alors nous avons
	\begin{equation}
		\begin{aligned}[]
			| a_n-b_n |&=| a_{n-1}-x_n |\\
			&=\left| a_{n-1}-\frac{ a_{n-1}+b_{n-1} }{2} \right| \\
			&=\frac{ 1 }{2}| a_{n-1}-b_{n-1} |,
		\end{aligned}
	\end{equation}
	ce qui prouve que $| a_n-b_n |\to 0$. Nous montrons maintenant que la suite \( (a_n)\) est de Cauchy. En effet nous avons
    \begin{equation}
        | a_n-a_{n-1} |=\begin{cases}
          0\\
          \left| \frac{ a_n -b_n}{ 2} \right|   
      \end{cases}\leq \frac{1}{ 2n }.
    \end{equation}
    Il en est de même pour la suite \( (b_n)\). Ce sont deux suites de Cauchy (donc convergentes) qui convergent vers la même limite. Soit \( \ell\) cette limite.
    
	Le nombre $\ell$ minore $A$. En effet si $a\in A$ est plus petit que $\ell$, les éléments $b_n$ tels que $| b_n-\ell |<| a-\ell |$ ne peuvent pas minorer $A$. D'autre part, pour tout $\epsilon$, le nombre $\ell+\epsilon$ ne peut pas minorer $A$. En effet, $\ell$ est la limite de la suite décroissante $(a_n)$, donc il existe $a_n$ entre $\ell$ et $\ell+\epsilon$. Mais $a_n$ ne minore pas $A$, donc $\ell+\epsilon$ ne minore pas non plus $A$.

	Nous avons prouvé que toute partie minorée de $\eR$ possède un infimum. La preuve que toute partie majorée possède un supremum se fait de la même façon.
	
\end{proof}


\begin{definition}
	Si le supremum d'un ensemble appartient à l'ensemble, nous l'appelons \defe{maximum}{maximum}. De la même façon si l'infimum d'un ensemble appartient à l'ensemble, nous disons que c'est le \defe{minimum}{minimum}.
\end{definition}

\begin{example}
	Pour les intervalles, ces notions sont simples : les bornes de l'intervalle sont les supremum et infimum, et ce sont des minima et maxima si l'intervalle est fermé. Le nombre $53$ est un majorant.
	\begin{enumerate}
		\item
			$A=\mathopen[ 1 , 2 \mathclose]$. Tous les nombres plus petits ou égaux à $1$ sont minorants, $1$ est infimum et minimum. Le nombre $2$ est un majorant, le maximum et le supremum.
		\item
			$B=\mathopen] 3 , \pi \mathclose[$. Le nombre $\pi$ est le supremum et est un majorant, mais n'est pas le maximum (parce que $\pi\notin B$). L'ensemble $B$ n'a pas de maximum. Bien entendu, $-1000$ est un minorant.
	\end{enumerate}
\end{example}

Il existe évidement de nombreux exemples plus vicieux.
\begin{example}
	Prenons $E=\{ \frac{1}{ n }\tq n\in\eN_0 \}$, dont les premiers points sont indiqués sur la figure \ref{LabelFigSuiteUnSurn}. Cet ensemble est constitué des nombres $1$, $\frac{ 1 }{2}$, $\frac{1}{ 3 }$, \ldots Le plus grand d'entre eux est $1$ parce que tous les nombres de la forme $\frac{1}{ n }$ avec $n\geq 1$ sont plus petits ou égaux à $1$. Le nombre $1$ est donc maximum de $E$.

	L'ensemble $E$ n'a par contre pas de minimum parce que tout élément de $E$ s'écrit $\frac{1}{ n }$ pour un certain $n$ et est plus grand que $\frac{1}{ n+1 }$ qui est également dans $E$.

	Prouvons que zéro est l'infimum de $E$. D'abord, tous les éléments de $E$ sont strictement positifs, donc zéro est certainement un minorant de $E$. Ensuite, nous savons que pour tout $\varepsilon>0$, il existe un $n$ tel que $\frac{1}{ n }$ est plus petit que $\varepsilon$. L'ensemble $E$ possède donc un élément plus petit que $0+\varepsilon$, et zéro est bien l'infimum.
\end{example}

\newcommand{\CaptionFigSuiteUnSurn}{Les premiers points du type $x_n=1/n$.}
\input{Fig_SuiteUnSurn.pstricks}

L'exemple suivant est une source classique d'erreurs en ce qui concerne l'infimum. Il sera à relire après avoir vu la définition de limite (définition \ref{DefLimiteSuiteNum}).
\begin{example}
	Les premiers points de l'ensemble $F=\{ \frac{ (-1)^n }{ n }\tq n\in\eN_0 \}$ sont représentés à la figure \ref{LabelFigSuiteInverseAlterne}. Bien que (comme nous le verrons plus tard) la limite de la suite $x_n=(-1)^n/n$ soit zéro, il n'est pas correct de dire que zéro est l'infimum de l'ensemble $F$. Le dessin, au contraire, montre bien que $-1$ est le minium (aucun point est plus bas que $-1$), tandis que le maximum est $1/2$.

	Nous reviendrons avec cet exemple dans la suite. Pour l'instant, ayez bien en tête que zéro n'est rien de spécial pour l'ensemble $F$ en ce qui concerne les notions de maximum, minimum et compagnie.
\end{example}
\newcommand{\CaptionFigSuiteInverseAlterne}{Les quelque premiers points du type $(-1)^n/n$.}
\input{Fig_SuiteInverseAlterne.pstricks}

%+++++++++++++++++++++++++++++++++++++++++++++++++++++++++++++++++++++++++++++++++++++++++++++++++++++++++++++++++++++++++++
\section{Point d'accumulation, point isolé}
%+++++++++++++++++++++++++++++++++++++++++++++++++++++++++++++++++++++++++++++++++++++++++++++++++++++++++++++++++++++++++++

Soit $D\subset\eR$. Un point $a\in D$ est \defe{isolé}{isolé!élément de \( \eR\)} dans $D$ (relativement à $\eR$) si il existe $\varepsilon>0$ tel que 
\begin{equation}
	\mathopen[ a-\varepsilon , a+\varepsilon \mathclose]\cap D=\{ a \}.
\end{equation}
Autrement dit, il existe un intervalle autour de $a$ dans lequel $a$ est le seul élément de $D$.

Un point $a\in \eR$ est un \defe{point d'accumulation}{accumulation!dans $\eR$} de $D$ si pour tout $\varepsilon>0$, 
\begin{equation}
	\Big( \mathopen[ a-\varepsilon , a+\varepsilon \mathclose]\setminus\{ a \} \Big)\cap D\neq\emptyset.
\end{equation}
Autrement dit, quel que soit l'intervalle autour de  $a$ que l'on considère, le point $a$ n'est pas tout seul dans $D$.

\begin{example}
	Prenons $D=\mathopen[ 0 , 1 [\cup\mathopen] 2 , 3 \mathclose]$. Cet ensemble n'a pas de points isolés, et l'ensemble de ses points d'accumulation est $\mathopen[ 0 , 1 \mathclose]\cup\mathopen[ 2,3  \mathclose]$.

	Notez que les points $1$ et $2$ sont des points d'accumulation de $D$ qui ne font pas partie de $D$. Il est possible d'être un point d'accumulation de $D$ sans être dans $D$, mais pour être un point isolé dans $D$, il faut être dans $D$.
\end{example}

\begin{example}
	Soit $D=\{ \frac{1}{ n }\}_{n\in\eN}$. Tous les points de cet ensemble sont des points isolés (vérifier !).  Aucun point de $D$ n'est point d'accumulation. Cependant $0$ est un point d'accumulation.
\end{example}

%+++++++++++++++++++++++++++++++++++++++++++++++++++++++++++++++++++++++++++++++++++++++++++++++++++++++++++++++++++++++++++
\section{Limite de fonction}
%+++++++++++++++++++++++++++++++++++++++++++++++++++++++++++++++++++++++++++++++++++++++++++++++++++++++++++++++++++++++++++
\label{SecLimiteFontion}

%---------------------------------------------------------------------------------------------------------------------------
\subsection{Définition}
%---------------------------------------------------------------------------------------------------------------------------

\begin{definition}[Limite d'une fonction]	\label{DefLimiteFonction}
	Soit une fonction $f\colon D\subset\eR\to \eR$ et $a$ un point d'accumulation de $D$. On dit que $f$ admet une \defe{limite}{limite!fonction} en $a$ si il existe un réel $\ell$ tel que 
	\begin{equation}\label{EqDefLimiteFonction}
		\forall\varepsilon>0,\,\exists\delta>0\tq \forall x\in D,\, 0<| x-a |<\delta\Rightarrow| f(x)-\ell |<\varepsilon.
	\end{equation}
\end{definition}

Si aucun nombre $\ell$ ne vérifie la condition de la définition, alors on dit que la fonction n'admet pas de limite en $a$. Lorsque $f$ possède la limite $\ell$ en $a$, nous notons
\begin{equation}
	\lim_{x\to a} f(x)=\ell.
\end{equation}

\begin{proposition}
	Soit une fonction $f\colon D\to \eR$. Si $a$ est un point d'accumulation de $D$ et si il existe une limite de $f$ en $a$, alors il en existe une seule. 
\end{proposition}

De façon équivalente, il ne peut pas exister deux nombres $\ell\neq\ell'$ vérifiant tout les deux la condition \eqref{EqDefLimiteFonction}.

\begin{proof}
	Soient $\ell$ et $\ell'$ deux limites de $f$ au point $a$. Par définition, pour tout $\varepsilon$ nous avons des nombres $\delta$ et $\delta'$ tels que
	\begin{equation}	\label{EqsContf2307Right}
		\begin{aligned}[]
			| x-a |<\delta&\Rightarrow \big| f(x)-\ell \big|<\varepsilon\\
			| x-a |<\delta'&\Rightarrow \big| f(x)-\ell' \big|<\varepsilon
		\end{aligned}
	\end{equation}
	Pour fixer les idées, supposons que $\delta<\delta'$ (le cas $\delta\geq\delta'$ se traite de la même manière).

	Étant donné que $a$ est un point d'accumulation du domaine $D$ de $f$, il existe un $x\in D$ tel que $| x-a |<\delta$. Évidemment, nous avons aussi $| x-a |<\delta'$. Les conditions \eqref{EqsContf2307Right} signifient alors que ce $x$ vérifie en même temps
	\begin{equation}
		| f(x)-\ell |<\varepsilon,
	\end{equation}
	et
	\begin{equation}
		| f(x)-\ell' |<\varepsilon.
	\end{equation}
	Afin de prouver que $\ell=\ell'$, nous allons maintenant calculer $| \ell-\ell' |$ et montrer que cette distance est plus petite que tout nombre. Nous avons (voir remarque \ref{RemTechniqueIneqs})
	\begin{equation}	\label{EqInesq2307ellellepr}
		| \ell-\ell' |=| \ell-f(x)+f(x)-\ell' |\leq | \ell-f(x) |+| f(x)-\ell' |<\varepsilon+\varepsilon.
	\end{equation}
	En résumé, pour tout $\varepsilon>0$ nous avons
	\begin{equation}
		| \ell-\ell' |<2\varepsilon,
	\end{equation}
	et donc $| \ell-\ell' |=0$, ce qui signifie que $\ell=\ell'$.
\end{proof}

\begin{remark}		\label{RemTechniqueIneqs}
	Les inégalités \eqref{EqInesq2307ellellepr} utilisent deux techniques très classiques en analyse qu'il convient d'avoir bien compris. La première est de faire
	\begin{equation}
		| A-B |=| A-C+C-B |.
	\end{equation}
	Il s'agit d'ajouter $-C+C$ dans la norme. Évidemment, cela ne change rien.

	La seconde technique est l'inégalité
	\begin{equation}
		| A+B |\leq| A |+| B |.
	\end{equation}
\end{remark}

\begin{example}
	Considérons la fonction $f(x)=2x$, et calculons la limite $\lim_{x\to 3} f(x)$. Vu que $f(3)=6$, nous nous attendons à avoir $\ell=6$. C'est ce que nous allons prouver maintenant. Pour chaque $\varepsilon>0$ nous devons trouver un $\delta>0$ tel que $| x-3 |<\delta$ implique $| f(x)-6 |<\varepsilon$. En remplaçant $f(x)$ par sa valeur en fonction de $x$ et avec quelque manipulations nous trouvons :
	\begin{equation}
		\begin{aligned}[]
			| f(x)-6 |&<\varepsilon\\
			| 2x-6 |&<\varepsilon\\
			2| x-3 |&<\varepsilon\\
			| x-3 |&<\frac{ \varepsilon }{2}
		\end{aligned}
	\end{equation}
	Donc dès que $| x-3 |<\frac{ \varepsilon }{2}$, nous avons $| f(x)-6 |<\varepsilon$. Nous posons donc $\delta=\frac{ \varepsilon }{2}$.

	Plus généralement, nous avons $\lim_{x\to a} f(x)=2a$, et cela se prouve en étudiant $| f(x)-2a |$ exactement de la même manière.
\end{example}

%---------------------------------------------------------------------------------------------------------------------------
\subsection{Propriétés de base}
%---------------------------------------------------------------------------------------------------------------------------

\begin{proposition}	\label{PropLimEstLineraure}
	La limite est une opération linéaire, c'est à dire que si $f$ et $g$ sont des fonctions qui admettent des limites en $a$ et si $\lambda$ est un nombre réel,
	\begin{enumerate}

		\item
			$\lim_{x\to a} (\lambda f)(x)=\lambda\lim_{x\to a} f(x)$,
		\item
			$\lim_{x\to a} (f+g)(x)=\lim_{x\to a} f(x)+\lim_{x\to a} g(x)$.
	\end{enumerate}
\end{proposition}
En combinant les deux propriétés de la proposition \ref{PropLimEstLineraure}, nous pouvons écrire
\begin{equation}
	\lim_{x\to a} (\lambda f+\mu g)(x)=\lambda\lim_{x\to a} f(x)+\mu\lim_{x\to a} g(x).
\end{equation}
pour toutes fonctions $f$ et $g$ admettant une limite en $a$ et pour tout réels $\lambda$ et $\mu$.

En plus d'être linéaire, la limite possède les deux propriétés suivantes.
\begin{proposition}
	Si $f$ et $g$ sont deux fonctions qui admettent une limite en $a$, alors
	\begin{equation}
		\lim_{x\to a} (fg)(x)=\lim_{x\to a} f(x)\cdot\lim_{x\to a} g(x).
	\end{equation}
	Si de plus $\lim_{x\to a} g(x)\neq 0$, alors
	\begin{equation}
		\lim_{x\to a} \frac{ f(x) }{ g(x) }=\frac{ \lim_{x\to a} f(x) }{ \lim_{x\to a} g(x) }.
	\end{equation}
\end{proposition}

%+++++++++++++++++++++++++++++++++++++++++++++++++++++++++++++++++++++++++++++++++++++++++++++++++++++++++++++++++++++++++++
\section{Continuité}
%+++++++++++++++++++++++++++++++++++++++++++++++++++++++++++++++++++++++++++++++++++++++++++++++++++++++++++++++++++++++++++
\label{SecContinue}

\begin{definition}		\label{DefFonctContinueRR}
	Soit une fonction $f\colon D\to \eR$ et un point $a$ dans $D$. Nous disons que $f$ est \defe{continue}{continue!fonction réelle} lorsque $f$ possède une limite en $a$ et $\lim_{x\to a} f(x)=f(a)$.
\end{definition}
En remplaçant $\ell$ par $f(a)$ dans la définition de la limite, nous exprimons la continuité de $f$ en $a$ par la façon suivante. Pour tout $\varepsilon>0$, il existe un $\delta>0$ tel que $\forall x\in D$,
\begin{equation}
	| x-a |<\delta\Rightarrow \big| f(x)-f(a) \big|<\varepsilon.
\end{equation}

\begin{theorem}[Accroissement finis]		\label{ThoAccFinisUneVariable}
	Soit $f$ une fonction continue sur $\mathopen[ a , b \mathclose]$ et dérivable sur $\mathopen] a , b \mathclose[$. Alors il existe au moins un réel $c\in\mathopen] a , b \mathclose[$ tel que $f(b)-f(a)=f'(c)(b-a)$.
\end{theorem}

\begin{theorem}[Valeurs intermédiaires]		\label{ThoValsIntern}
	Soit $I$ un intervalle de $\eR$ et $f$ une fonction continue sur $I$ avec $f(a)<f(b)$. Alors $\mathopen[ f(a) , f(b) \mathclose]\subset f(I)$
\end{theorem}

Cette propriété signifie que si une fonction passe par la hauteur $y_1$ et puis par la hauteur $y_2$, alors elle passe par toutes les hauteurs intermédiaires.

\begin{corollary}
	Soit $f\colon I\to \eR$ une fonction continue et deux points $a<b$ dans $I$ tels que $f(a)<0$ et $f(b)>0$. Alors il existe $c\in\mathopen] a , b \mathclose[$ tel que $f(c)=0$.
\end{corollary}


%+++++++++++++++++++++++++++++++++++++++++++++++++++++++++++++++++++++++++++++++++++++++++++++++++++++++++++++++++++++++++++
\section{Développement asymptotique, théorème de Taylor}
%+++++++++++++++++++++++++++++++++++++++++++++++++++++++++++++++++++++++++++++++++++++++++++++++++++++++++++++++++++++++++++
\label{AppSecTaylorR}

\begin{theorem}[Théorème de Taylor]		\label{ThoTaylor}
Soit $I\subset$ un intervalle non vide et non réduit à un point de $\eR$ ainsi que $a\in I$. Soit une fonction $f\colon I\to \eR$ telle que $f^{(n)}(a)$ existe. Alors il existe une fonction $\epsilon$ définie sur $I$ et à valeurs dans $\eR$ vérifiant les deux conditions suivantes :
\begin{subequations}		\label{SubEqsDevTauil}
	\begin{align}
		\lim_{x\to a}\epsilon(x)&=0,\\
		f(x)&=T^a_{f,n}(x)+\epsilon(x)(x-a)^{n}	&&\forall x\in I		\label{subeqfTepseqb}
	\end{align}
\end{subequations}
où $T^a_{f,n}(x)=\sum_{k=0}^n\frac{ f^{(k)}(a) }{ k! }(x-a)^k$ et $f^{(k)}$ dénote la $k$-ième dérivée de $f$ (en particulier, $f^{(0)}=f$, $f^{(1)}=f'$).\nomenclature{$f^{(n)}$}{La $n$-ième dérivée de la fonction $f$}
\end{theorem}

Nous insistons sur le fait que la formule \eqref{subeqfTepseqb} est une égalité, et non une approximation. Ce qui serait une approximation serait de récrire la formule dans le terme contenant $\epsilon$.

Le polynôme $T^a_{f,n}$ est le \defe{polynôme de Taylor}{Taylor} de $f$ au point $a$ à l'ordre $n$.  Une preuve du théorème peut être trouvée dans \cite{TrenchRealAnalisys}, théorème 2.5.1 à la page 99. La version donnée ici est inspirée de l'article sur \wikipedia{fr}{Développement_de_Taylor}{Wikipédia}\footnote{http://fr.wikipedia.org/wiki/Développement\_de\_Taylor}, qui donne également une preuve du résultat.

En termes de notations, nous définissons l'ensemble $o(x)$\nomenclature{$o(x)$}{fonction tendant rapidement vers zéro} l'ensemble des fonctions $f$ telles que
\begin{equation}
	\lim_{x\to 0} \frac{ f(x) }{ x }=0.
\end{equation}
Plus généralement si $g$ est une fonction telle que $\lim_{x\to 0} g(x)=0$, nous disons $f\in o(g)$ si
\begin{equation}
	\lim_{x\to 0} \frac{ f(x) }{ g(x) }=0.
\end{equation}
De façon intuitive, l'ensemble $o(g)$ est l'ensemble des fonctions qui tendent vers zéro «plus vite» que $g$.


Nous pouvons donner un énoncé alternatif au théorème \ref{ThoTaylor} en définissant $h(x)=\epsilon(x+a)x^n$. Cette fonction est définie exprès pour avoir
\begin{equation}
	h(x-a)=\epsilon(x)(x-a)^n,
\end{equation}
et donc
\begin{equation}
	\lim_{x\to 0} \frac{ h(x) }{ x^n }=\lim_{x\to 0} \epsilon(x-a)=\lim_{x\to a}\epsilon(x)=0. 
\end{equation}
Donc $h\in o(x^n)$.

Le théorème dit donc qu'il existe une fonction $\alpha\in o(x^n)$ telle que
\begin{equation}
	f(x)=T^a_{f,n}(x)+\alpha(x-a).
\end{equation}
pour tout $x\in I$. Pour plus de détails et quelque exemples, voir \cite{ExoCdI1}.

\begin{example}
	Le développement du cosinus est donné par
	\begin{equation}
		\cos(x)=1-\frac{ x^2 }{ 2 }+\frac{ x^4 }{ 4! }-\frac{ x^6 }{ 6! }\cdots
	\end{equation}
	Nous avons donc l'existence d'une fonction $h_1\in o(x^2)$ telle que $\cos(x)=1-\frac{ x^2 }{ 2 }+h_1(x)$. Il existe aussi une autre fonction $h_2\in o(x^4)$ telle que $\cos(x)=1-\frac{ x^2 }{ 2 }+\frac{ x^4 }{ 4! }+h_2(x)$.
\end{example}

\begin{example}		\label{ExempleUtlDev}
	Une des façons les plus courantes d'utiliser les formules \eqref{SubEqsDevTauil} est de développer $f(a+t)$ pour des petits $t$ en posant $x=a+t$ dans la formule :
	\begin{equation}	\label{EqDevfautouraeps}
		f(a+t)=f(a)+f'(a)t+f''(a)\frac{ t^2 }{ 2 }+\epsilon(a+t)t^2
	\end{equation}
	avec $\lim_{t\to 0} \epsilon(a+t)=0$. Ici, la fonction $T$ dont on parle dans le théorème est $T_{f,2}^a(a+t)=f(a)+f'(a)t+f''(a)\frac{ t^2 }{2}$.

	Lorsque $x$ et $y$ sont deux nombres «proches\footnote{par exemple dans une limite $(x,y)\to(h,h)$.}», nous pouvons développer $f(y)$ autour de $f(x)$ :
	\begin{equation}		\label{Eqfydevfx}
		f(y)=f(x)+f'(x)(y-x)+f''(x)\frac{ (y-x)^2 }{ 2 }+\epsilon(y-x)(y-x)^2,
	\end{equation}
	et donc écrire
	\begin{equation}
		f(x)-f(y)=-f'(x)(y-x)-f''(x)\frac{ (y-x)^2 }{ 2 }-\epsilon(y-x)(y-x)^2.
	\end{equation}
	De cette manière nous obtenons une formule qui ne contient plus que $y$ dans la différence $y-x$.
\end{example}

%+++++++++++++++++++++++++++++++++++++++++++++++++++++++++++++++++++++++++++++++++++++++++++++++++++++++++++++++++++++++++++
\section{Fonctions à valeurs dans $\eR^n$}
%+++++++++++++++++++++++++++++++++++++++++++++++++++++++++++++++++++++++++++++++++++++++++++++++++++++++++++++++++++++++++++

À peu près toutes les notions que vous connaissez à propos de fonctions de $\eR$ dans $\eR$ se généralises immédiatement au cas de fonctions de $\eR$ dans $\eR^n$.

Nous disons que la fonction $f\colon \eR\to \eR^n$ est \defe{de classe $C^1$}{classe $C^1$} si chacune de ses composantes $f_i$ est de classe $C^1$ en tant que fonctions de $\eR$ dans $\eR$.

La dérivée de $f$ est donnée par la dérivée composante par composante. Pour l'intégrale de $f$, il en va de même : composante par composante. 
\begin{equation}
	\int f(x)dx=\big(  \int f_1(x)dx,\,\int f_2(x)dx,\ldots,\int f_n(x)dx   \big).
\end{equation}

Par exemple si nous considérons le mouvent d'une particule sur une hélice, la position est donnée par
\begin{equation}
	f(t)=\big( R\sin(t),R\cos(t),t \big).
\end{equation}
La vitesse est donnée par
\begin{equation}
	f'(t)=\big( R\cos(t),-R\sin(t),1 \big),
\end{equation}
et l'intégrale sera donnée par
\begin{equation}
	\int f(t)dt=\big( -R\cos(t)+C_1,R\sin(t)+C_2,\frac{ t^2 }{ 2 }+C_3 \big).
\end{equation}

Si nous considérons une pierre lancée horizontalement du sommet d'une falaise avec une vitesse initiale $v_0$, la vitesse de la pierre sera donnée par
\begin{equation}
	v(t)=(v_0,gt).
\end{equation}
Pour trouver la position, nous intégrons la vitesse par rapport au temps :
\begin{equation}
	f(t)=\int v(t)dt=\big( v_0t+C_1,\frac{ gt^2 }{ 2 }+C_2 \big).
\end{equation}
Notez qu'il faut une constante d'intégration différente pour chaque composantes.

\begin{lemma}			\label{LemIneqnormeintintnorm}
	Pour toute fonction $u\colon \mathopen[ a , b \mathclose]\to \eR^n$, nous avons
	\begin{equation}
		\| \int_a^bu(t)dt\|\leq\int_a^b\| u(t) \|dt
	\end{equation}
	pourvu que le membre de gauche ait un sens.
\end{lemma}

\begin{proof}
	Étant donné que $\int_a^bu(t)dt$ est un élément de $\eR^n$, par la proposition \ref{LemSclNormeXi}, il existe un $\xi\in\eR^n$ de norme $1$ tel que
	\begin{equation}
		\| \int_a^bu(t)dt \|=\xi\cdot\int_a^b u(t)dt=\int_a^b u(t)\cdot\xi dt\leq\int_a^b\| u(t) \|   \| \xi \|=\int_a^b\| u(t) \|dt.
	\end{equation}
\end{proof}
