%+++++++++++++++++++++++++++++++++++++++++++++++++++++++++++++++++++++++++++++++++++++++++++++++++++++++++++++++++++++++++++
\section{Suites numériques}
%+++++++++++++++++++++++++++++++++++++++++++++++++++++++++++++++++++++++++++++++++++++++++++++++++++++++++++++++++++++++++++

Une \defe{suite numérique}{suite numérique} est une application $x\colon \eN\to \eR$. Une telle application sera notée $(x_n)$. L'élément numéro $k$ de la suite sera noté $x_k$.

\begin{definition}[Limite d'une suite numérique]	\label{DefLimiteSuiteNum}
	Nous disons que la suite $(x_n)$ est une suite \defe{convergente}{convergence!suite numérique} si il existe un réel $\ell$ tel que
	\begin{equation}		\label{EqDefLimSuite}
		\forall \varepsilon>0,\,\exists N\in\eN\tq\forall n\geq N,\,| x_n-\ell |<\varepsilon.
	\end{equation}
	Dans ce cas, le nombre $\ell$ est nommé \defe{limite}{limite!suite numérique} de la suite $(x_n)$. Nous dirons aussi souvent que la suite \defe{converge}{convergence de suite} vers le nombre $\ell$.
\end{definition}
	Une façon équivalente d'exprimer le critère \eqref{EqDefLimSuite} est de dire que pour tout $\varepsilon$ positif, il existe un rang $N\in\eR$ tel que l'intervalle $\mathopen[ \ell-\varepsilon , \ell+\varepsilon \mathclose]$ contient tous les termes $x_n$ au-delà de $N$.

Il est à noter que le rang $N$ dont il est question dans la définition de suite convergente dépend de $\varepsilon$.

\begin{example}
	Quelque suites usuelles.
	\begin{enumerate}
		\item
			La suite $x_n=\frac{1}{ n }$ converge vers $0$.
		\item
			La suite $x_n=(-1)^n$ ne converge pas.
	\end{enumerate}
\end{example}

Une suite est dite \defe{contenue}{} dans un ensemble $A$ si $x_n\in A$ pour tout $n$. Une suite est \defe{bornée supérieurement}{bornée!suite} si il existe un $M$ tel que $x_n\leq M$ pour tout $n$. De la même manière, la suite est bornée inférieurement si il existe un $m$ tel que $x_n\geq m$ pour tout $n$.

Le lemme suivant est souvent utilisé pour prouver qu'une suite est convergente.
\begin{lemma}		\label{LemSuiteCrBorncv}
	Une suite croissante et bornée supérieurement converge. Une suite décroissante bornée inférieurement est convergente.
\end{lemma}

%+++++++++++++++++++++++++++++++++++++++++++++++++++++++++++++++++++++++++++++++++++++++++++++++++++++++++++++++++++++++++++
\section{Maximum, majorant, supremum et compagnie}
%+++++++++++++++++++++++++++++++++++++++++++++++++++++++++++++++++++++++++++++++++++++++++++++++++++++++++++++++++++++++++++

Lorsque vous lisez que la charge maximale d'un camion est de \unit{2.5}{\ton}, est-ce que cela veut dire que vous pouvez y mettre \unit{2.5}{\ton}, mais qui si un oiseau se pose dessus, le camion s'effondre ? Ou bien est-ce que cela signifie qu'à \unit{2.5}{\ton} le camion s'écroule, mais que toute charge inférieure est valable ?

C'est à cette rude question que nous allons nous attaquer maintenant.

\begin{definition}
Soit une partie $A$ de $\eR$. Nous disons qu'un nombre $M$ est un \defe{majorant}{majorant} de $A$ si $M$ est plus grand ou égal que tous les éléments de $A$, c'est à dire si
\begin{equation}
	\forall a\in A,\, M\geq a.
\end{equation}
Un \defe{minorant}{minorant} de $A$ est un nombre $m$ tel que 
\begin{equation}
	\forall a\in A,\, m\leq a.
\end{equation}
\end{definition}

\begin{definition}		\label{DefSupeA}
Soit $A$ une partie majorée de $\eR$. Le \defe{supremum}{supremum} de $A$ est le plus petit des majorants, c'est à dire le nombre $M$ tel que
\begin{enumerate}
	\item
		$M\geq x$ pour tout $x\in A$,
	\item
		pour tout $\varepsilon$, le nombre $M-\varepsilon$ n'est pas un majorant de $a$, c'est à dire qu'il existe un élément $x\in A$ tel que $x>M-\varepsilon$.
\end{enumerate}
Nous notons $\sup A$ le supremum de $A$.

De la même façon, \defe{l'infimum}{infimum} de $A$, noté $\inf A$, est le plus grand de ses minorants. 
\end{definition}
Par convention, si la partie n'est pas bornée vers le haut, nous dirons que son supremum n'existe pas, ou bien qu'il est égal à $+\infty$, suivant les contextes. Pour votre culture générale, sachez toutefois que $\infty\notin\eR$.

La définition est justifiée par le lemme \ref{LemInfUnique} et la proposition \ref{PropBorneSupInf}. Le premier montre que si $A$ possède un infimum, alors il est unique, tandis que le second montre que toute partie majorée de $\eR$ accepter un supremum, et que toute partie minorée accepte un infimum.
\begin{lemma}		\label{LemInfUnique}
	Soit $A$ une partie de $\eR$. Supposons que $m_1$ et $m_2$ soient deux nombres qui vérifient les propriétés de l'infimum de $A$. Alors $m_1=m_2$.
\end{lemma}

\begin{proof}
	Si $_1\neq m_2$, nous pouvons supposer $m_2>m_1$. Dans ce cas, étant donné que $m_1$ est un infimum, $m_2$ ne peut pas minorer $A$, et donc ne peut pas être un infimum.
\end{proof}

\begin{proposition}		\label{PropBorneSupInf}
	Tout sous-ensemble de $\eR$ borné vers le bas possède un infimum; tout sous-ensemble de $\eR$ borné vers le haut possède un supremum.
\end{proposition}

La preuve qui suit est proche de celle donnée par Wikipédia  dans l'article \wikipedia{en}{http://en.wikipedia.org/wiki/Least_upper_bound_principle}{Least uppert bound principle}.

\begin{proof}
	Soit $A$, une partie de $\eR$. Nous allons trouver son infimum en suivant une méthode de dichotomie. Pour cela nous allons construire trois suites en même temps de la façon suivante. D'abord nous choisissons un point $x_0$ de $A$ et un point $x_1$ qui minore $A$ (qui existe par hypothèse) :
	\begin{equation}
		\begin{aligned}[]
			x_0&\text{ est un élément de $A$},\\
			x_1&\text{ est un minorant de $A$},\\
			a_0&=x_0\\
			b_0&=x_1\\
			b_1&=x_1.
		\end{aligned}
	\end{equation}
	Ensuite, nous faisons la récurrence suivante :
	\begin{equation}
		\begin{aligned}[]
			x_{n+1}&=\frac{ a_n+b_n }{2},\\
			a_{n+1}&=\begin{cases}
				a_{n}	&	\text{si $x_{n+1}$ minore $A$}\\
				x_{n+1}	&	 \text{sinon},
			\end{cases}\\
			b_{n+1}&=\begin{cases}
				x_{n+1}	&	\text{si $x_{n+1}$ minore $A$}\\
				b_n	&	 \text{sinon}.
			\end{cases}
		\end{aligned}
	\end{equation}
    Nous allons montrer que \( a_n\) et \( (b_n)\) sont des suites convergentes de même limite et que cette limite est l'infimum de \( A\).

	Soit $n\in\eN$; il y a deux possibilités. Soit $a_n=a_{n-1}$ et $b_n=x_n$, soit $a_n=x_n$ et $b_n=b_{n-1}$. Supposons que nous soyons dans le premier cas (le second se traite de façon similaire). Alors nous avons
	\begin{equation}
		\begin{aligned}[]
			| a_n-b_n |&=| a_{n-1}-x_n |\\
			&=\left| a_{n-1}-\frac{ a_{n-1}+b_{n-1} }{2} \right| \\
			&=\frac{ 1 }{2}| a_{n-1}-b_{n-1} |,
		\end{aligned}
	\end{equation}
	ce qui prouve que $| a_n-b_n |\to 0$. Nous montrons maintenant que la suite \( (a_n)\) est de Cauchy. En effet nous avons
    \begin{equation}
        | a_n-a_{n-1} |=\begin{cases}
          0\\
          \left| \frac{ a_n -b_n}{ 2} \right|   
      \end{cases}\leq \frac{1}{ 2n }.
    \end{equation}
    Il en est de même pour la suite \( (b_n)\). Ce sont deux suites de Cauchy (donc convergentes) qui convergent vers la même limite. Soit \( \ell\) cette limite.
    
	Le nombre $\ell$ minore $A$. En effet si $a\in A$ est plus petit que $\ell$, les éléments $b_n$ tels que $| b_n-\ell |<| a-\ell |$ ne peuvent pas minorer $A$. D'autre part, pour tout $\epsilon$, le nombre $\ell+\epsilon$ ne peut pas minorer $A$. En effet, $\ell$ est la limite de la suite décroissante $(a_n)$, donc il existe $a_n$ entre $\ell$ et $\ell+\epsilon$. Mais $a_n$ ne minore pas $A$, donc $\ell+\epsilon$ ne minore pas non plus $A$.

	Nous avons prouvé que toute partie minorée de $\eR$ possède un infimum. La preuve que toute partie majorée possède un supremum se fait de la même façon.
	
\end{proof}


\begin{definition}
	Si le supremum d'un ensemble appartient à l'ensemble, nous l'appelons \defe{maximum}{maximum}. De la même façon si l'infimum d'un ensemble appartient à l'ensemble, nous disons que c'est le \defe{minimum}{minimum}.
\end{definition}

\begin{example}
	Pour les intervalles, ces notions sont simples : les bornes de l'intervalle sont les supremum et infimum, et ce sont des minima et maxima si l'intervalle est fermé. Le nombre $53$ est un majorant.
	\begin{enumerate}
		\item
			$A=\mathopen[ 1 , 2 \mathclose]$. Tous les nombres plus petits ou égaux à $1$ sont minorants, $1$ est infimum et minimum. Le nombre $2$ est un majorant, le maximum et le supremum.
		\item
			$B=\mathopen] 3 , \pi \mathclose[$. Le nombre $\pi$ est le supremum et est un majorant, mais n'est pas le maximum (parce que $\pi\notin B$). L'ensemble $B$ n'a pas de maximum. Bien entendu, $-1000$ est un minorant.
	\end{enumerate}
\end{example}

Il existe évidement de nombreux exemples plus vicieux.
\begin{example}
	Prenons $E=\{ \frac{1}{ n }\tq n\in\eN_0 \}$, dont les premiers points sont indiqués sur la figure \ref{LabelFigSuiteUnSurn}. Cet ensemble est constitué des nombres $1$, $\frac{ 1 }{2}$, $\frac{1}{ 3 }$, \ldots Le plus grand d'entre eux est $1$ parce que tous les nombres de la forme $\frac{1}{ n }$ avec $n\geq 1$ sont plus petits ou égaux à $1$. Le nombre $1$ est donc maximum de $E$.

	L'ensemble $E$ n'a par contre pas de minimum parce que tout élément de $E$ s'écrit $\frac{1}{ n }$ pour un certain $n$ et est plus grand que $\frac{1}{ n+1 }$ qui est également dans $E$.

	Prouvons que zéro est l'infimum de $E$. D'abord, tous les éléments de $E$ sont strictement positifs, donc zéro est certainement un minorant de $E$. Ensuite, nous savons que pour tout $\varepsilon>0$, il existe un $n$ tel que $\frac{1}{ n }$ est plus petit que $\varepsilon$. L'ensemble $E$ possède donc un élément plus petit que $0+\varepsilon$, et zéro est bien l'infimum.
\end{example}

\newcommand{\CaptionFigSuiteUnSurn}{Les premiers points du type $x_n=1/n$.}
\input{Fig_SuiteUnSurn.pstricks}

L'exemple suivant est une source classique d'erreurs en ce qui concerne l'infimum. Il sera à relire après avoir vu la définition de limite (définition \ref{DefLimiteSuiteNum}).
\begin{example}
	Les premiers points de l'ensemble $F=\{ \frac{ (-1)^n }{ n }\tq n\in\eN_0 \}$ sont représentés à la figure \ref{LabelFigSuiteInverseAlterne}. Bien que (comme nous le verrons plus tard) la limite de la suite $x_n=(-1)^n/n$ soit zéro, il n'est pas correct de dire que zéro est l'infimum de l'ensemble $F$. Le dessin, au contraire, montre bien que $-1$ est le minium (aucun point est plus bas que $-1$), tandis que le maximum est $1/2$.

	Nous reviendrons avec cet exemple dans la suite. Pour l'instant, ayez bien en tête que zéro n'est rien de spécial pour l'ensemble $F$ en ce qui concerne les notions de maximum, minimum et compagnie.
\end{example}
\newcommand{\CaptionFigSuiteInverseAlterne}{Les quelque premiers points du type $(-1)^n/n$.}
\input{Fig_SuiteInverseAlterne.pstricks}


%+++++++++++++++++++++++++++++++++++++++++++++++++++++++++++++++++++++++++++++++++++++++++++++++++++++++++++++++++++++++++++
\section{Limites de suites}
%+++++++++++++++++++++++++++++++++++++++++++++++++++++++++++++++++++++++++++++++++++++++++++++++++++++++++++++++++++++++++++

L'étude des limites et de la continuité dans $\eR$ est déjà connue. Un rappel est donné en appendice dans les sections \ref{SecLimiteFontion} et \ref{SecContinue}. Lors de cette étude, nous avons remarqué que la valeur absolue jouait un rôle fondamental en mesurant la distance entre les points de $\eR$. Maintenant que nous avons étudié en détail la notion de norme et de distance sur un espace vectoriel normé, nous pouvons facilement définir et étudier les notions de limites et de continuité pour les fonctions $f\colon \eR^m\to \eR^n$.

Bien que nous allons parler de l'espace $\eR^n$ muni de la norme $\| . \|_2$, toute la partie limite et continuité dans $\eR^m$ sera recopier presque mot à mot ce qu'on a fait dans la partie sur les espaces vectoriels normés en général.

La définition suivante copie la définition \ref{DefCvSuiteEGVN}.
\begin{definition}[Limite d'une suite dans $\eR^m$]
	Une suite de points $(x_n)$ dans $\eR^m$ est dite \defe{convergente}{convergence!suite dans $\eR^m$} si il existe un élément $\ell\in\eR^m$ tel que
	\begin{equation}	\label{EqCondLimSuite}
		\forall\varepsilon>0,\,\exists N\in \eN\tq\,\forall n\geq N,\,\| x_n-\ell \|<\varepsilon.
	\end{equation}
	Dans ce cas, nous disons que $\ell$ est la \defe{limite}{limite!suite dans $\eR^m$} de la suite $(x_n)$ et nous écrivons $\lim x_n=\ell$ ou plus simplement $x_n\to \ell$.
\end{definition}
Notez aussi la similarité avec la définition \ref{DefLimiteSuiteNum}.

\begin{remark}
	Nous n'écrivons pas «$\lim_{n\to\infty}x_n$» parce que, lorsqu'on parle de suites, la limite est \emph{toujours} lorsque $n$ tend vers l'infini. Il n'y a aucun intérêt à chercher par exemple $\lim_{n\to 4}x_n$ parce que cela vaudrait $x_4$ et rien d'autre.

	Ceci est une différence importante avec les limites de fonctions.
\end{remark}

\begin{lemma}[Unicité de la limite]
	Il ne peut pas y avoir deux nombres différents qui satisfont à la condition \eqref{EqCondLimSuite}. En d'autres termes, si $\ell$ et $\ell'$ sont deux limites de la suite $(x_n)$, alors $\ell=\ell'$.
\end{lemma}

\begin{proof}
	Soit $\varepsilon>0$. Nous considérons $N$ tel que
	\begin{equation}
		\| x_n-\ell \|<\varepsilon
	\end{equation}
	pour tout $n\geq N$, et $N'>0$ tel que 
	\begin{equation}
		\| x_n-\ell' \|<\epsilon
	\end{equation}
	pour tout $n>N'$. Maintenant, nous prenons $n$ plus grand que $N$ et $N'$ de telle façon à ce que $x_n$ vérifie les deux inéquations en même temps. Alors
	\begin{equation}
		\| \ell-\ell' \|=\| \ell-x_n+x_n-\ell' \|\leq\| \ell-x_n \|+\| x_n-\ell' \|<2\varepsilon.
	\end{equation}
	Cela prouve que $\| \ell-\ell' \|=0$.
\end{proof}

\begin{proposition}		\label{PropCvRpComposante}
	Une suite $(x_n)$ dans $\eR^m$ est convergente dans $\eR^m$ si et seulement si les suites de chaque composantes sont convergentes dans $\eR$. Dans ce cas nous avons
	 \begin{equation}
		 \lim x_n=\Big( \lim(x_n)_1,\lim (x_n)_2,\ldots,\lim (x_n)_m \Big)
	 \end{equation}
	 où $(x_n)_k$ dénote la $k$-ième composante de $(x_n)$.
\end{proposition}

\begin{example}
	La suite $x_n=\big( \frac{1}{ n },1-\frac{1}{ n } \big)$ converge vers $(0,1)$ dans $\eR^2$. En effet, en utilisant la proposition \ref{PropCvRpComposante}, nous devons calculer séparément les limites
	\begin{equation}
		\begin{aligned}[]
			\lim\frac{1}{ n }&=0\\
			\lim\big( 1-\frac{1}{ n } \big)&=1.
		\end{aligned}
	\end{equation}
\end{example}

\begin{example}
	Étant donné que la suite $(-1)^n$ n'est pas convergente, la suite $x_n=\big( (-1)^n,\frac{1}{ n } \big)$ n'est pas convergente dans $\eR^2$.
\end{example}

%+++++++++++++++++++++++++++++++++++++++++++++++++++++++++++++++++++++++++++++++++++++++++++++++++++++++++++++++++++++++++++
\section{Limite de fonction}
%+++++++++++++++++++++++++++++++++++++++++++++++++++++++++++++++++++++++++++++++++++++++++++++++++++++++++++++++++++++++++++
\label{SecLimiteFontion}

%---------------------------------------------------------------------------------------------------------------------------
\subsection{Définition}
%---------------------------------------------------------------------------------------------------------------------------

\begin{definition}[Limite d'une fonction]	\label{DefLimiteFonction}
	Soit une fonction $f\colon D\subset\eR\to \eR$ et $a$ un point d'accumulation de $D$. On dit que $f$ admet une \defe{limite}{limite!fonction} en $a$ si il existe un réel $\ell$ tel que 
	\begin{equation}\label{EqDefLimiteFonction}
		\forall\varepsilon>0,\,\exists\delta>0\tq \forall x\in D,\, 0<| x-a |<\delta\Rightarrow| f(x)-\ell |<\varepsilon.
	\end{equation}
\end{definition}

Si aucun nombre $\ell$ ne vérifie la condition de la définition, alors on dit que la fonction n'admet pas de limite en $a$. Lorsque $f$ possède la limite $\ell$ en $a$, nous notons
\begin{equation}
	\lim_{x\to a} f(x)=\ell.
\end{equation}

\begin{proposition}
	Soit une fonction $f\colon D\to \eR$. Si $a$ est un point d'accumulation de $D$ et si il existe une limite de $f$ en $a$, alors il en existe une seule. 
\end{proposition}

De façon équivalente, il ne peut pas exister deux nombres $\ell\neq\ell'$ vérifiant tout les deux la condition \eqref{EqDefLimiteFonction}.

\begin{proof}
	Soient $\ell$ et $\ell'$ deux limites de $f$ au point $a$. Par définition, pour tout $\varepsilon$ nous avons des nombres $\delta$ et $\delta'$ tels que
	\begin{equation}	\label{EqsContf2307Right}
		\begin{aligned}[]
			| x-a |<\delta&\Rightarrow \big| f(x)-\ell \big|<\varepsilon\\
			| x-a |<\delta'&\Rightarrow \big| f(x)-\ell' \big|<\varepsilon
		\end{aligned}
	\end{equation}
	Pour fixer les idées, supposons que $\delta<\delta'$ (le cas $\delta\geq\delta'$ se traite de la même manière).

	Étant donné que $a$ est un point d'accumulation du domaine $D$ de $f$, il existe un $x\in D$ tel que $| x-a |<\delta$. Évidemment, nous avons aussi $| x-a |<\delta'$. Les conditions \eqref{EqsContf2307Right} signifient alors que ce $x$ vérifie en même temps
	\begin{equation}
		| f(x)-\ell |<\varepsilon,
	\end{equation}
	et
	\begin{equation}
		| f(x)-\ell' |<\varepsilon.
	\end{equation}
	Afin de prouver que $\ell=\ell'$, nous allons maintenant calculer $| \ell-\ell' |$ et montrer que cette distance est plus petite que tout nombre. Nous avons (voir remarque \ref{RemTechniqueIneqs})
	\begin{equation}	\label{EqInesq2307ellellepr}
		| \ell-\ell' |=| \ell-f(x)+f(x)-\ell' |\leq | \ell-f(x) |+| f(x)-\ell' |<\varepsilon+\varepsilon.
	\end{equation}
	En résumé, pour tout $\varepsilon>0$ nous avons
	\begin{equation}
		| \ell-\ell' |<2\varepsilon,
	\end{equation}
	et donc $| \ell-\ell' |=0$, ce qui signifie que $\ell=\ell'$.
\end{proof}

\begin{remark}		\label{RemTechniqueIneqs}
	Les inégalités \eqref{EqInesq2307ellellepr} utilisent deux techniques très classiques en analyse qu'il convient d'avoir bien compris. La première est de faire
	\begin{equation}
		| A-B |=| A-C+C-B |.
	\end{equation}
	Il s'agit d'ajouter $-C+C$ dans la norme. Évidemment, cela ne change rien.

	La seconde technique est l'inégalité
	\begin{equation}
		| A+B |\leq| A |+| B |.
	\end{equation}
\end{remark}

\begin{example}
	Considérons la fonction $f(x)=2x$, et calculons la limite $\lim_{x\to 3} f(x)$. Vu que $f(3)=6$, nous nous attendons à avoir $\ell=6$. C'est ce que nous allons prouver maintenant. Pour chaque $\varepsilon>0$ nous devons trouver un $\delta>0$ tel que $| x-3 |<\delta$ implique $| f(x)-6 |<\varepsilon$. En remplaçant $f(x)$ par sa valeur en fonction de $x$ et avec quelque manipulations nous trouvons :
	\begin{equation}
		\begin{aligned}[]
			| f(x)-6 |&<\varepsilon\\
			| 2x-6 |&<\varepsilon\\
			2| x-3 |&<\varepsilon\\
			| x-3 |&<\frac{ \varepsilon }{2}
		\end{aligned}
	\end{equation}
	Donc dès que $| x-3 |<\frac{ \varepsilon }{2}$, nous avons $| f(x)-6 |<\varepsilon$. Nous posons donc $\delta=\frac{ \varepsilon }{2}$.

	Plus généralement, nous avons $\lim_{x\to a} f(x)=2a$, et cela se prouve en étudiant $| f(x)-2a |$ exactement de la même manière.
\end{example}

%---------------------------------------------------------------------------------------------------------------------------
\subsection{Propriétés de base}
%---------------------------------------------------------------------------------------------------------------------------

\begin{proposition}	\label{PropLimEstLineraure}
	La limite est une opération linéaire, c'est à dire que si $f$ et $g$ sont des fonctions qui admettent des limites en $a$ et si $\lambda$ est un nombre réel,
	\begin{enumerate}

		\item
			$\lim_{x\to a} (\lambda f)(x)=\lambda\lim_{x\to a} f(x)$,
		\item
			$\lim_{x\to a} (f+g)(x)=\lim_{x\to a} f(x)+\lim_{x\to a} g(x)$.
	\end{enumerate}
\end{proposition}
En combinant les deux propriétés de la proposition \ref{PropLimEstLineraure}, nous pouvons écrire
\begin{equation}
	\lim_{x\to a} (\lambda f+\mu g)(x)=\lambda\lim_{x\to a} f(x)+\mu\lim_{x\to a} g(x).
\end{equation}
pour toutes fonctions $f$ et $g$ admettant une limite en $a$ et pour tout réels $\lambda$ et $\mu$.

En plus d'être linéaire, la limite possède les deux propriétés suivantes.
\begin{proposition}
	Si $f$ et $g$ sont deux fonctions qui admettent une limite en $a$, alors
	\begin{equation}
		\lim_{x\to a} (fg)(x)=\lim_{x\to a} f(x)\cdot\lim_{x\to a} g(x).
	\end{equation}
	Si de plus $\lim_{x\to a} g(x)\neq 0$, alors
	\begin{equation}
		\lim_{x\to a} \frac{ f(x) }{ g(x) }=\frac{ \lim_{x\to a} f(x) }{ \lim_{x\to a} g(x) }.
	\end{equation}
\end{proposition}

\label{SecLimitesRn}    % Pour la section ``Limites''

%---------------------------------------------------------------------------------------------------------------------------
\subsection{Limites de fonctions}
%---------------------------------------------------------------------------------------------------------------------------

\begin{definition}\label{def_limite}
	Soit $f\colon D\subset\eR^m\to \eR$ une fonction et $a$ un point d'accumulation de $D$. On dit que $f$ possède une \defe{limite}{limite!fonction de plusieurs variables} si il existe un élément $\ell\in\eR$ tel que
	\begin{equation}		\label{Eq2807CondiionLimifnm}
		\forall\varepsilon>0,\,\exists\delta>0\tq 0<\| x-a \|<\delta\Rightarrow | f(x)-\ell |<\varepsilon.
	\end{equation}
	
	Pour une fonction $f\colon D\subset\eR^m\to \eR^n$, la définition est la même, sauf que nous remplaçons la valeur absolue par la norme dans $\eR^n$. Nous disons donc que $\ell$ est la limite de $f$ lorsque $x$ tend vers $a$, et nous notons $\lim_{x\to a} f(x)=\ell$ lorsque pour tout $\varepsilon>0$, il existe un $\delta>0$ tel que
	\begin{equation}		\label{EqDefLimRpRn}
		0<\| x-a \|_{\eR^m}<\delta\Rightarrow\,\| f(x)-\ell \|_{\eR^n}<\varepsilon.
	\end{equation}
\end{definition}

\begin{remark}
	Dans l'équation \eqref{EqDefLimRpRn}, nous avons explicitement écrit les normes $\| . \|_{\eR^m}$ et $\| . \|_{\eR^n}$. Dans la suite nous allons le plus souvent noter $\| . \|$ sans plus de précision. Il est important de faire l'exercice de bien comprendre à chaque fois de quelle norme nous parlons.
\end{remark}

\begin{remark}
	Il est important de remarquer à quel point les définitions \ref{def_limite}, \ref{LimiteDansEVN} et \ref{DefLimiteFonction} sont analogues. En réalité, la définition fondamentale est la définition de la limite dans les espaces vectoriels normés; les deux autres sont des cas particuliers, adaptés à $\eR$ et $\eR^m$. Il en sera de même pour les définitions de fonctions continues : il y aura une définition pour la continuité de fonctions entre espaces vectoriels normés, et ensuite une définition pour les fonctions de $\eR^m$ dans $\eR^n$ qui en sera un cas particulier.
\end{remark}

Tentons de comprendre ce que signifie qu'un nombre $\ell$ \emph{ne soit pas} la limite de $f$ lorsque $x\to a$. Il s'agit d'inverser la condition \eqref{Eq2807CondiionLimifnm}. Le nombre $\ell$ n'est pas une limite de $f$ pour $x\to a$ lorsque
\begin{equation}		\label{EqCaractNonLim}
	\exists\varepsilon>0\tq\,\forall\delta>0,\,\exists x\tq 0<\| x-a \|<\delta\text{ et }\| f(x)-\ell \|>\varepsilon,
\end{equation}
c'est à dire qu'il existe un certain seuil $\varepsilon$ tel qu'on a beau s'approcher aussi proche qu'on veut de $a$ (distance $\delta$), on trouvera toujours un $x$ tel que $f(x)$ n'est pas $\varepsilon$-proche de $\ell$.

\begin{lemma}[Unicité de la limite]
	Si $\ell$ et $\ell'$ sont deux limites de $f(x)$ lorsque $x$ tend vers $a$, alors $\ell=\ell'$.
\end{lemma}

\begin{proof}
	Soit $\varepsilon>0$. Nous considérons $\delta$ tel que $\| f(x)-\ell \|<\varepsilon$ pour tout $x$ tel que $\| x-a \|<\delta$. De la même manière, nous prenons $\delta'$ tel que $\| x-a \|<\delta'$ implique $\| f(x)-\ell' \|<\varepsilon$. Pour les $x$ tels que $\| x-a \|$ est plus petit que $\delta$ et $\delta'$ en même temps, nous avons
	\begin{equation}
		\| \ell-\ell' \|=\| \ell-f(x)+f(x)-\ell' \|\leq\| \ell-f(x) \|+\| f(x)-\ell' \|<2\varepsilon,
	\end{equation}
	et donc $\| \ell-\ell' \|=0$ parce que c'est plus petit que $2\varepsilon$ pour tout $\varepsilon$.
\end{proof}

Le concept de limite appelle immédiatement celui de continuité.
\begin{definition}
	Soit $f\colon D\subset\eR^m\to \eR^n $ et $a\in D$. On dit que $f$ est \defe{continue}{continuité} en $a$ lorsque la limite $\lim_{x\to a} f(x)$ existe et est égale à $f(a)$.

	On dit que $f$ est continue sur une partie $A\subset D$ si elle est continue en tous les points de $a$.
\end{definition}

La continuité peut évidement être récrite avec une formule du même type que celle de la limite.
\begin{proposition}
	La fonction $f\colon D\subset\eR^m\to \eR^n$ est continue en $a\in D$ si et seulement si
	\begin{equation}
		\forall\varepsilon,\,\exists\delta>0\tq x\in D\cap B(a,\delta)\Rightarrow \| f(x)-f(a) \|<\varepsilon.
	\end{equation}
\end{proposition}


\begin{theorem}
	Une fonction $f$ de $\eR^m$ vers $\eR^n$ est continue si et seulement si pour tout ouvert $\mO$ de $\eR^n$, l'image inverse $f^{-1}(\mO)$ est ouverte dans $\eR^m$.
\end{theorem}

\begin{proof}
	Ce théorème est un cas particulier du théorème \ref{ThoContiueImageInvOUvert}. Il suffit de remplacer $V$ par $\eR^m$ et $W$ par $\eR^n$.
\end{proof}

Quasiment toutes les propriétés des limites ont un équivalent concernant la continuité.
\begin{proposition}	\label{PropLimParcompos}
	Soit $f\colon D\subset\eR^m\to \eR^n$. Nous avons 
	\begin{equation}
		\lim_{x\to a} f(x)=\ell
	\end{equation}
	si et seulement si 
	\begin{equation}
		\lim_{x\to a} f_i(x)=\ell_i
	\end{equation}
	pour tout $i\in\{ 1,\ldots,n \}$ où $f_i(x)$ dénote la $i$-ème composante de $f(x)$ et $\ell_i$ la $i$-ème composante de $\ell\in\eR^n$.
\end{proposition}
Cette proposition revient à dire que la convergence d'une fonction est équivalente à la convergence de chacune de ses composantes.

\begin{proof}
	L'élément clef de la preuve est le fait que pour tout vecteur $u\in\eR^p$, nous ayons l'inégalité
	\begin{equation}	\label{Equilequnorme}
		| u_i |\leq\sqrt{\sum_{k=1}^p| u_k |^2}=\| u \|.
	\end{equation}
	La norme (dans $\eR^p$) d'un vecteur est plus grande ou égale à la valeur absolue de chacune de ses composantes.

	Supposons que nous ayons une fonction dont chacune des composantes a une limite en $a$ : $\lim_{x\to a} f_i(x)=\ell_i$. Montrons que dans ce cas la fonction $f$ tend vers $\ell$. Si nous considérons $\varepsilon>0$, par définition de la limite de chacune des fonctions $f_i$, il  existent des $\delta_i$ tels que
	\begin{equation}
		\| x-a \|_{\eR^m}<\delta_i\Rightarrow | f_i(x)-\ell_i |<\varepsilon.
	\end{equation}
	Notez que la norme à gauche est une norme dans $\eR^m$ et que celle à droite est une simple valeur absolue dans $\eR$. Considérons $\delta=\min\{ \delta_i \}_{i=1,\ldots n}$. Si $\| x-a \|<\delta$, alors
	\begin{equation}
		\| f(x)-\ell \|=\sqrt{\sum_{i=1}^n| f_i(x)-\ell_i |^2}<\sqrt{\sum_{i=1}^n\varepsilon^2}=\sqrt{n\varepsilon^2}=\sqrt{n}\varepsilon.
	\end{equation}
	Nous voyons qu'en choisissant les $\delta_i$ tels que $| f_i(x)-\ell_i |<\varepsilon$, nous trouvons $\| f(x)-\ell \|<\sqrt{n}\varepsilon$. Afin d'obtenir $\| f(x)-\ell \|<\varepsilon$, nous choisissons donc les $\delta_i$ de telle manière a avoir $| f_i(x)-\ell_i |<\varepsilon/\sqrt{n}$.

	Nous avons donc prouvé que la limite composante par composante impliquait la limite de la fonction. Nous devons encore prouver le sens inverse.

	Supposons donc que $\lim_{x\to a} f(x)=\ell$, et prouvons que nous ayons $\lim_{x\to a} f_i(x)=\ell_i$ pour chaque $i$. Soit $\varepsilon>0$ et $\delta>0$ tel que $\| x-a \|<\delta$ implique $\| f(x)-\ell \|<\varepsilon$. Avec ces choix, nous avons
	\begin{equation}
		| f_i(x)-\ell_i |\leq\| f(x)-\ell \|<\varepsilon
	\end{equation}
	où nous avons utilisé la majoration \eqref{Equilequnorme} avec $f(x)-\ell$ en guise de $u$.
\end{proof}

De même, pour la continuité nous avons la proposition suivante :
\begin{proposition}
	Soit une fonction $f\colon D\subset\eR^m\to \eR^n$ et $a\in D$. La fonction $f$ est continue en $a$ si et seulement si chacune de ses composantes l'est, c'est à dire si et seulement si chacune des fonctions $f_i\colon D\to \eR$ est continue en $a$.
\end{proposition}
Essayez de prouver cette proposition directement par la définition de la continuité, en suivant pas à pas la démonstration de la proposition \ref{PropLimParcompos}.

\begin{proposition}		\label{Propfaposfxposcont}
	Soit $f\colon \eR^m\to \eR$ et $a$, un point du domaine de $f$ telle que $f(a)>0$. Alors il existe un rayon $r$ tel que $f(x)>0$ pour tout $x$ dans $B(a,r)$.
\end{proposition}
Cette proposition signifie que si la fonction est strictement positive en un point, alors elle restera strictement positive en tous les points «pas trop loin».

\begin{proof}
	Prenons $\varepsilon=f(a)/2$ dans la définition de la continuité. Il existe donc un rayon $\delta$ tel que pour tout $x$ dans $B(a,\delta)$,
	\begin{equation}
		| f(x)-f(a) |\leq \frac{ f(a) }{2},
	\end{equation}
	en d'autres termes, $f(x)\in B\big( f(a),\frac{ f(a) }{ 2 } \big)$. Évidement aucun nombre négatif ne fait partie de cette dernière boule lorsque $f(a)$ est strictement positif.
\end{proof}

\begin{corollary}		\label{CorfneqzOuvert}
	Si $f\colon \eR^m\to \eR$ est une fonction continue, alors l'ensemble
	\begin{equation}
		A=\{ x\in\eR^m\tqs f(x)\neq 0 \}
	\end{equation}
	est ouvert.
\end{corollary}

\begin{proof}
	Soit $x\in A$. Si $x>0$ (le cas $x<0$ est laissé en exercice), alors il existe une boule autour de $x$ sur laquelle $f$ reste strictement positive (proposition \ref{Propfaposfxposcont}). Cette boule est donc contenue dans $A$. Étant donné qu'autour de chaque point de $A$ nous pouvons trouver une boule contenue dans $A$, ce dernier est ouvert.
\end{proof}

La proposition suivante montre que la limite peut «passer à travers» les fonctions continues.
\begin{proposition}[limite de fonction composée]		\label{PropLimCompose}
	Soit $f\colon \eR^n\to \eR^q$ et $g\colon \eR^m\to \eR^n$ telles que
	\begin{subequations}
		\begin{align}
			\lim_{x\to a} g(x)&= p		\label{EqLimCompHypa}\\
			\lim_{y\to p} f(y)&= q		\label{EqLimCompHypb}
		\end{align}
	\end{subequations}
	Alors nous avons $\lim_{x\to a} (f\circ g)(x)=q$. 
\end{proposition}

\begin{proof}
	Comme presque toute preuve à propos de limite ou de continuité, nous commençons par choisir $\varepsilon>0$. Nous devons montrer qu'il existe un $\delta$ tel que $\| x-a \|\leq \delta$ implique $\| f\big( g(x) \big)-q \|\leq \varepsilon$.

	La limite \eqref{EqLimCompHypb} impose l'existence d'un $\tilde\delta$ tel que $\| y-p \|\leq\tilde\delta$ implique $\| f(y)-q \|\leq\varepsilon$, tandis que la limite \eqref{EqLimCompHypa} donne un $\delta$ tel que $\| x-a \|\leq\delta$ implique $\| g(x)-p \|\leq\tilde\delta$ (nous avons pris $\tilde\delta$ en guise de $\varepsilon$ dans la définition de la limite pour $g$).

	Avec ces choix, si $\| x-a \|\leq \delta$, alors $\| g(x)-p \|\leq\tilde\delta$, et par conséquent,
	\begin{equation}
		\| f\big( g(x) \big)-q \|\leq\varepsilon,
	\end{equation}
	ce que nous voulions.
\end{proof}

De façon pragmatique, la proposition \ref{PropLimCompose} nous fournit une formule pour les limites de fonctions composée :
\begin{equation}		\label{Eqlimfgvomp}
	\lim_{x\to a} (f\circ g)(x)=\lim_{y\to \lim_{x\to a} g(x)}f(y)
\end{equation}
lorsque $f$ est continue.

\begin{remark}
	La formule \eqref{Eqlimfgvomp} ne peut pas être utilisée à l'envers. Il existe des cas où $\lim_{x\to a} (g\circ f)(x)=q$, et $\lim_{x\to a} f(x)=p$ sans pour autant avoir $\lim_{y\to q} g(y)=q$. Par exemple
	\begin{subequations}
		\begin{align}
			g(x)&=\begin{cases}
				2	&	\text{si $x\geq0$,}\\
				0	&	 \text{si $x<0$}\\
			\end{cases}\\
			f(x)&=| x |.
		\end{align}
	\end{subequations}
	Nous avons $(g\circ f)(x)=2$ pour tout $x$, ainsi que $\lim_{x\to 0} f(x)=0$, mais la limite $\lim_{y\to 0} g(y)$ n'existe pas.
\end{remark}


\begin{theorem}[Caractérisation de la limite par les suites]		\label{ThoLimSuite}
	Une fonction $f\colon D\subset\eR^m\to \eR^n$ admet une limite $\ell$ en un point d'accumulation $a$ de $D$ si et seulement si pour toute suite $(x_n)$ dans $D\setminus\{ a \}$ convergente vers $a$, la suite $\big( f(x_n) \big)$ dans $\eR^n$ converge vers $\ell$.
\end{theorem}

\begin{proof}
	Supposons d'abord que la fonction ait une limite $\ell$ lorsque $x\to a$, et considérons une suite $(x_n)$ dans $D\setminus\{ a \}$ convergente vers $a$. Nous devons montrer que la suite $y_n=f(x_n)$ converge vers $\ell$, c'est à dire que si nous choisissons $\varepsilon>0$ nous devons montrer qu'il existe un $N$ tel que $n>N$ implique $\| y_n-\ell  \|=\| f(x_n)-\ell \|<\varepsilon$. 
	
	Nous avons deux hypothèses. La première est la convergence de la fonction et la seconde est la convergence de la suite $(x_n)$. L'hypothèse de convergence de la fonction nous dit que (le $\varepsilon$ a déjà été choisit dans le paragraphe précédent)
	\begin{equation}
		\exists\delta\tq\,0<\| x-a \|<\delta\Rightarrow\| f(x)-\ell \|<\varepsilon.
	\end{equation}
	Une fois choisit ce $\delta$ qui «va avec» le $\varepsilon$ qui a été choisit précédemment, la définition de la convergence de la suite nous enseigne que
	\begin{equation}
		\exists N\tq n>N\Rightarrow\| x_n-a \|<\delta.
	\end{equation}
	Récapitulons ce que nous avons fait. Nous avons choisit un $\varepsilon$, et puis nous avons construit un $N$. Lorsque $n>N$, nous avons $\| x_n-a \|<\delta$. Mais alors, par construction de ce $\delta$, nous avons $\| f(x_n)-\ell \|<\varepsilon$. Au final, $n>N$ implique bien $\| y_n-\ell \|<\varepsilon$, ce qu'il nous fallait.

	Nous supposons maintenant que la fonction $f$ \emph{ne} converge \emph{pas} vers $\ell$, et nous allons construire une suite d'éléments $x_n$ qui converge vers $a$ sans que $(y_n)=f(x_n)$ ne converge vers $\ell$. La fonction $f$ vérifie la condition \eqref{EqCaractNonLim}. Nous prenons donc un $\varepsilon$ tel que $\forall \delta$, il existe un $x$ qui vérifie \emph{en même temps} les deux conditions
	\begin{subequations}
		\begin{numcases}{}
			0<\| x-a \|<\delta\\
			\| f(x)-\ell \|>\varepsilon.
		\end{numcases}
	\end{subequations}
	Un tel $x$ existe pour tout choix de $\delta$. Choisissons un $n$ arbitraire et $\delta=\frac{1}{ n }$. Nous nommons $x_n$ le $x$ correspondant à ce choix de $n$. La suite $(x_n)$ ainsi construite converge vers $a$ parce que 
	\begin{equation}
		\| x_n-a \|<\delta_n=\frac{1}{ n },
	\end{equation}
	donc dès que $n$ est grand, $\| x_n-a \|$ est petit. Mais la suite $y_n=f(x_n)$ ne converge pas vers $\ell$ parce que
	\begin{equation}
		\| f(x_n)-\ell \|>\varepsilon
	\end{equation}
	pour tout $n$. La suite $y_n$ ne s'approche donc jamais à moins d'une distance $\varepsilon$ de $\ell$.
\end{proof}

Nous avons le même type de résultats pour la continuité.
\begin{proposition}		\label{PropFnContParSuite}
	Soit une fonction $f\colon D\subset\eR^m\to \eR^n$ et $a\in D$. La fonction $f$ est continue en $a$ si et seulement si pour toute suite $(x_n)$ dans $D\setminus\{ a \}$ convergente vers $a$, nous avons $\lim f(x_n)=f(a)$.
\end{proposition}
%+++++++++++++++++++++++++++++++++++++++++++++++++++++++++++++++++++++++++++++++++++++++++++++++++++++++++++++++++++++++++++
\section{Calcul de limites}		\label{SecCalculLimite}
%+++++++++++++++++++++++++++++++++++++++++++++++++++++++++++++++++++++++++++++++++++++++++++++++++++++++++++++++++++++++++++

Dans cette section, nous allons voir un certain nombre de techniques qui permettent de calculer des limites en plusieurs variables, ou bien de prouver qu'une limite n'existe pas.

Le calcul de limite en une variable est supposé connu.

%---------------------------------------------------------------------------------------------------------------------------
\subsection{Règles simples de calcul}
%---------------------------------------------------------------------------------------------------------------------------

Les opérations simples passent à la limite, sauf la division pour laquelle il faut faire attention au dénominateur.
\begin{proposition}     \label{PropOpsSimplesLimites}
    Soient \( f\) et \( g\) deux fonctions telles que \( \lim_{x\to a} f(x)=\alpha\) et \( \lim_{x\to a} g(x)=\beta\). Alors
    \begin{enumerate}
        \item
            \( \lim_{x\to a} f(x)+g(x)=\alpha+\beta\),
        \item
            \( \lim_{x\to a} f(x)g(x)=\alpha\beta\),
        \item
            si il existe un voisinage de \( a\) sur lequel \( g\) ne s'annule pas, alors \( \lim_{x\to a} \frac{ f(x) }{ g(x) }=\frac{ \alpha }{ \beta }\).
    \end{enumerate}
\end{proposition}

%---------------------------------------------------------------------------------------------------------------------------
\subsection{Règle de l'étau}
%---------------------------------------------------------------------------------------------------------------------------

Une première façon de calculer la limite d'une fonction est de la «\wikipedia{en}{Squeeze_theorem}{coincer}» entre deux fonctions dont nous connaissons la limite. Le théorème, que nous acceptons sans démonstration, est le suivant :
\begin{theorem}		\label{ThoRegleEtau}
	Soit $\mO$, un ouvert de $\eR^m$ contenant le point $a$. Soient $f$, $g$ et $h$, trois fonctions définies sur $\mO$ (éventuellement pas en $a$ lui-même). Supposons que pour tout $x\in\mO$ (à part éventuellement $a$), nous ayons les inégalités
	\begin{equation}
		g(x)\leq f(x)\leq h(x).
	\end{equation}
	Supposons de plus que
	\begin{equation}
		\lim_{x\to a} g(x)=\lim_{x\to a} h(x)=\ell.
	\end{equation}
	Alors la limite $\lim_{x\to a} f(x)$ existe et vaut $\ell$.
\end{theorem}

Nous insistons sur le fait que les deux fonctions entre lesquelles nous coinçons $f$ doivent tendre vers \emph{la même} valeur.

Cette méthode est très pratique lorsqu'on a des fonctions trigonométriques qui se factorisent parce qu'elles sont toujours majorables par $1$.
\begin{example}
	Prouvons que la fonction $f(x)=x\sin(x)$ tend vers zéro lorsque $x$ tend vers $0$. D'abord, nous coinçons la fonction entre deux fonctions connues :
	\begin{equation}
		0\leq| x\sin(x) |=| x | |\sin(x) |\leq | x |.
	\end{equation}
	Donc $| x\sin(x) |$ est coincé entre $g(x)=0$ et $h(x)=| x |$. Ces deux fonctions tendent vers $0$ lorsque $x\to 0$, et donc $f(x)$ tend vers zéro.
\end{example}


\begin{example}
	Prouver la continuité en $(0,0)$ de la fonction
	\begin{equation}
		f(x,y)=\begin{cases}
			\frac{ x | y | }{ \sqrt{x^2+y^2} }	&	\text{si $(x,y)\neq (0,0)$}\\
			0	&	 \text{sinon.}
		\end{cases}
	\end{equation}
	Considérons une suite $(x_n,y_n)\in\eR^2$ qui tend vers $(0,0)$. Étant donné que $\frac{ | y | }{ \sqrt{x^2+y^2} }<1$ pour tout $x$ et $y$, nous avons
	\begin{equation}
		0\leq | f(x_n,y_n) |=\left| \frac{ x_n | y_n | }{ \sqrt{x_n^2+y_n^2} } \right| \leq | x_n |\to 0.
	\end{equation}
	Donc nous avons
	\begin{equation}
		\lim_{(x,y)\to(0,0)}f(x,y)=0=f(0,0),
	\end{equation}
	ce qui prouve que la fonction est continue en $(0,0)$ par la proposition \ref{PropFnContParSuite}. Nous avons utilisé la règle de l'étau (théorème \ref{ThoRegleEtau}).
\end{example}

%---------------------------------------------------------------------------------------------------------------------------
\subsection{Méthode des chemins}
%---------------------------------------------------------------------------------------------------------------------------

Lorsque la limite n'existe pas, il y a une façon en général assez simple de le savoir, c'est la \defe{méthode des chemin}{méthode!des chemins}.

\newcommand{\CaptionFigMethodeChemin}{Sur toute la droite $y=-x$, la fonction vaut $-1/2$, tandis que sur toute la droite $y=x/2$, elle vaut $\frac{2}{ 5 }$. Il est donc impossible que la fonction ait une limite en $(0,0)$, parce que dans toute boule autour de zéro, il y aura toujours un point de chacune de ces deux droites.}
	\input{Fig_MethodeChemin.pstricks}

\begin{example}		\label{ExFNExempleMethodeTrigigi}
	Considérons la fonction
	\begin{equation}
		f(x,y)=\frac{ xy }{ x^2+y^2 },
	\end{equation}
	et remarquons que, quelle que soit la valeur de $y$, cette fonction est nulle lorsque $x=0$. De la même manière, nous voyons que si $x=y$, alors la fonction vaut\footnote{En fait ce que nous sommes en train de faire est de poser $\theta=\pi/2$ et $\theta=\pi/4$ dans \eqref{Eq2807fpolairerhodeuxcossin}.} $\frac{ 1 }{2}$. 

	Il est impossible que la fonction ait une limite en $(0,0)$ parce qu'on ne peut pas trouver un $\ell$ dont on s'approche à la fois en suivant la ligne $x=0$ et la ligne $x=y$.

	Deux autres chemins avec encore deux autres valeurs sont dessinés sur la figure \ref{LabelFigMethodeChemin}.

\end{example}

Nous pouvons formaliser cet exemple en utilisant le théorème \ref{ThoLimSuite}. Considérons les deux suites $x_n=(0,\frac{1}{ n })$ et $y_n=(\frac{1}{ n },\frac{1}{ n })$. Ce sont deux suites dans $\eR^2$ qui tendent vers $(0,0)$. Si la fonction $f$ convergeait vers $\ell$, alors nous aurions au moins
\begin{subequations}\label{Eq3007Lixxyyell}
	\begin{align}
		\lim f(x_n)&=\ell\\
		\lim f(y_n)&=\ell,
	\end{align}
\end{subequations}
mais nous savons que pour tout $n$, $f(x_n)=f(0,\frac{1}{ n })=0$ et $f(y_n)=f(\frac{1}{ n },\frac{1}{ n })=\frac{1}{ 2 }$. Il n'y a donc aucun nombre $\ell$ qui vérifie les deux équations \eqref{Eq3007Lixxyyell} parce que $\lim f(x_n)=0$ et $\lim f(y_n)=\frac{ 1 }{2}$.

Tout ceci est formalisé et généralisé dans la proposition suivante.
\begin{proposition}
	Soit $f\colon D\subset\eR^m\to \eR^n$ et $a$ un point d'adhérence de $D$. Alors nous avons
	\begin{equation}
		\lim_{x\to a} f(x)=\ell
	\end{equation}
	si et seulement si pour toute fonction $\gamma\colon \eR\to \eR^m$ telle que $\lim_{t\to 0} \gamma(t)=a$, nous avons
	\begin{equation}
		\lim_{t\to 0} (f\circ\gamma)(t)=\ell.
	\end{equation}	
\end{proposition}

\begin{corollary}	\label{CorMethodeChemin}
	Soient $f\colon D\subset\eR^m\to \eR^n$ et $a$ un point d'accumulation de $D$. Si nous avons deux fonctions $\gamma_1,\gamma_2\colon \eR\to \eR^m$ telles que
	\begin{equation}
		\lim_{t\to 0} \gamma_1(t)=\lim_{t\to 0} \gamma_2(t)=a
	\end{equation}
	tandis que
	\begin{equation}
		\lim_{t\to 0} (f\circ \gamma_1)(t)\neq\lim_{t\to 0} (f\circ \gamma_2)(t),
	\end{equation}
	ou bien que l'une des deux limites n'existe pas, alors la limite de $f(x)$ lorsque $x\to a$ n'existe pas.
\end{corollary}

\begin{corollary}	\label{CorMethodeChemoinNegatif}
	Soient $f\colon D\subset\eR^m\to \eR^n$ et $a$ un point d'accumulation de $D$. Si il existe une fonction $\gamma\colon \eR\to \eR^m$ avec $\gamma(0)=a$ telle que la limite $\lim_{t\to 0} (f\circ\gamma)(t)$ n'existe pas, alors la limite $\lim_{x\to a} f(x)$ n'existe pas.
\end{corollary}

En ce qui concerne le calcul de limites, la méthode des chemins peut être utilisé de trois façons :
\begin{enumerate}
	\item
		Dès que l'on trouve une fonction $\gamma\colon \eR\to \eR^m$ telle que $\lim_{t\to 0} (f\circ \gamma)(t)=\ell$, alors nous savons que \emph{si la limite $\lim_{x\to a} f(x)$ existe}, alors cette limite vaut $\ell$.
	\item
		Dès que l'on a trouvé deux fonctions $\gamma_i$ qui tendent vers $a$, mais dont les limites de $\lim_{t\to 0} (f\circ\gamma_i)(t)$ sont différentes, alors la limite $\lim_{x\to a} f(x)$ n'existe pas.
	\item
		Dès qu'on trouve une chemin le long duquel il n'y a pas de limite, alors la limite n'existe pas (corollaire \ref{CorMethodeChemoinNegatif}).
\end{enumerate}
La méthode des chemins ne permet donc pas de de calculer une limite quand elle existe. Elle permet uniquement de la «deviner», ou bien de prouver que la limite n'existe pas.

\begin{example}
	Soit à calculer
	\begin{equation}	\label{Eq3007ExempleLimiche}
		\lim_{(x,y)\to(0,0)}\frac{ x-y }{ x+y }.
	\end{equation}
	Si nous prenons le chemin $\gamma_1(t)=(t,t)$, nous avons bien $\lim_{t\to 0} \gamma_1(t)=(0,0)$, et nous avons
	\begin{equation}
		\lim_{t\to 0} (f\circ\gamma_1)(t)=\lim_{t\to 0} \frac{ t-t }{ t+t }=0.
	\end{equation}
	Donc si la limite \eqref{Eq3007ExempleLimiche} existait, elle vaudrait obligatoirement $0$. Mais si nous considérons $\gamma_2(t)=(0,t)$, nous avons
	\begin{equation}
		(f\circ\gamma_2)(t)=\frac{ -t }{ t }=-1,
	\end{equation}
	donc si la limite existe, elle doit obligatoirement valoir $-1$. Ne pouvant être égale à $0$ et à $-1$ en même temps, la limite \eqref{Eq3007ExempleLimiche} n'existe pas.
\end{example}

%---------------------------------------------------------------------------------------------------------------------------
\subsection{Méthode des coordonnées polaires}
%---------------------------------------------------------------------------------------------------------------------------

La proposition suivante exprime la définition de la limite en d'autres termes, et va être pratique dans le calcul de certaines limites.
\begin{proposition}		\label{PropMethodePolaire}
	Soit $f\colon D\subset\eR^m\to \eR^n$, $a$ un point d'accumulation de $D$ et $\ell\in \eR^n$. Nous définissons
	\begin{equation}
		E_r=\{ f(x)\tq x\in B(a,r)\cap D \},
	\end{equation}
	et
	\begin{equation}
		s_r=\sup\{ \| v-\ell \|\tq v\in E_r \}.
	\end{equation}
	Alors nous avons $\lim_{x\to a} f(x)=\ell$ si et seulement si $\lim_{r\to 0} s_r=0$.
\end{proposition}

Dans cette proposition, $E_r$ représente l'ensemble des valeurs atteintes par $f$ dans un rayon $r$ autour de $a$. Le nombre $s_r$ sélectionne, parmi toutes ces valeurs, celle qui est la plus éloignée de $\ell$ et donne la distance. En d'autres termes, $s_r$ est la distance maximale entre $f(x)$ et $\ell$ lorsque $x$ est à une distance au maximum $r$ de $a$.

Lorsque nous avons affaire à une fonction $f\colon \eR^2\to \eR$, cette proposition nous permet de calculer facilement les limites en passant aux coordonnées polaires.

\begin{example}		\label{ExempleMethodeTrigigi}
	Reprenons la fonction de l'exemple \ref{ExFNExempleMethodeTrigigi}:
	\begin{equation}
		f(x,y)=\frac{ xy }{ x^2+y^2 }.
	\end{equation}
	Son domaine est $\eR^2\setminus\{ (0,0) \}$. Nous voulons calculer $\lim_{(x,y)\to(0,0)}f(x,y)$. Écrivons la définition de $E_r$~:
	\begin{equation}
		E_r=\{ f(x,y)\tq (x,y)\in B\big( (0,0),r \big) \}.
	\end{equation}
	Les points de la boule sont, en coordonnées polaires, les points de la forme $(\rho,\theta)$ avec $\rho<r$. La chose intéressante est que $f(\rho,\theta)$ est relativement simple (plus simple que la fonction départ). En effet en remplaçant tous les $x$ par $\rho\cos(\theta)$ et tous les $y$ par $\rho\sin(\theta)$, et en utilisant le fait que $\cos^2(\theta)+\sin^2(\theta)=1$, nous trouvons
	\begin{equation}		\label{Eq2807fpolairerhodeuxcossin}
		f(\rho,\theta)=\frac{ \rho^2\cos(\theta)\sin(\theta) }{ \rho^2 }=\cos(\theta)\sin(\theta).
	\end{equation}
	Cela signifie que
	\begin{equation}
		E_r=\{ \cos(\theta)\sin(\theta)\tq\theta\in\mathopen[ 0 , 2\pi [ \}.
	\end{equation}
	Prenons $\ell$ quelconque. Le nombre $s_r$ est le supremum des
	\begin{equation}
		\| \ell-\cos(\theta)\sin(\theta) \|
	\end{equation}
	lorsque $\theta$ parcours $\mathopen[ 0 , 2\pi \mathclose]$. Nous ne sommes pas obligés calculer la valeur exacte de $s_r$. Ce qui compte ici est que $s_r$ ne vaut certainement pas zéro, et ne dépend pas de $r$. Donc il est impossible d'avoir $\lim_{r\to 0} s_r=0$, et la fonction donnée n'a pas de limite en $(0,0)$.
\end{example}

Nous pouvons retenir cette règle pour calculer les limites lorsque $(x,y)\to(0,0)$ de fonctions $f\colon \eR^2\to \eR$ :
\begin{enumerate}
	\item
		passer en coordonnées polaires, c'est à dire remplacer $x$ par $\rho\cos(\theta)$ et $y$ par $\rho\sin(\theta)$;
	\item
		nous obtenons une fonction $g$ de $\rho$ et $\theta$. Si la limite $\lim_{r\to 0} g(r,\theta)$ n'existe pas ou dépend de $\theta$, alors la fonction n'a pas de limite. Si on peut majorer $g$ par une fonction ne dépendant pas de $\theta$, et que cette fonction a une limite lorsque $r\to 0$, alors cette limite est la limite de la fonction.
\end{enumerate}

La vraie difficulté de la technique des coordonnées polaire est de trouver le supremum de $E_r$, ou tout au moins de montrer qu'il est borné par une fonction qui a une limite qui ne dépend pas de $\theta$. Une de situations classiques dans laquelle c'est facile est lorsque la fonction se présente comme une fonction de $r$ multiplié par une fonction de $\theta$. 

\begin{example}		\label{Exemplexyxsqysq}
	Soit à calculer la limite
	\begin{equation}
		\lim_{(x,y)\to(0,0)}xy\left( \frac{ x^2-y^2 }{ x^2+y^2 }\right).
	\end{equation}
	Le passage aux coordonnées polaires donne
	\begin{equation}
		f(r,\theta)=r^2\sin\theta\cos\theta(\cos^2\theta-\sin^2\theta).
	\end{equation}
	Déterminer le supremum de cela est relativement difficile. Mais nous savons que de toutes façons, la quantité $\sin\theta\cos\theta(\cos^2\theta-\sin^2\theta)$ est bornée par $1$. Donc
	\begin{equation}
		\| f(r,\theta) \|\leq r^2.
	\end{equation}
	Maintenant la règle de l'étau montre que $\lim_{(x,y)\to(0,0)}f(x,y)$ est zéro.

	La situation vraiment gênante serait celle avec une fonction de $\theta$ qui risque de s'annuler dans un dénominateur.
\end{example}

\begin{example}
	Soit à calculer
	\begin{equation}
		\lim_{(x,y)\to(0,0)}\frac{ x^2+y^2 }{ x-y }.
	\end{equation}
	Le passage en polaires donne
	\begin{equation}
		f(r,\theta)=\frac{ r^2 }{ r\big( \cos(\theta)-\sin(\theta) \big) }=\frac{ r }{ \cos(\theta)-\sin(\theta) }.
	\end{equation}
	Certes \emph{pour chaque $\theta$} nous avons $\lim_{r\to 0} f(r,\theta)=0$, mais il ne faut pas en déduire trop vite que la limite $\lim_{(x,y)\to(0,0)}f(x,y)$ vaut zéro parce que prendre la limite $r\to 0$ avec $\theta$ fixé revient à prendre la limite le long de la droite d'angle $\theta$.

	Il n'est pas possible de majorer $f(r,\theta)$ par une fonction ne dépendant pas de $\theta$ parce que cette fonction tend vers l'infini lorsque $\theta\to\pi/4$. Est-ce que cela veut dire que la limite n'existe pas ? Cela veut en tout cas dire que la méthode des coordonnées polaires ne parvient pas à résoudre l'exercice. Pour conclure, il faudra encore un peu travailler.

    Nous pouvons essayer de calculer le long d'un chemin plus général \( (r(t),\theta(t))\). Choisissons \( r(t)=t\) puis cherchons \( \theta(t)\) de telle sorte à avoir 
    \begin{equation}        \label{EqICrDSe}
        \cos\theta(t)-\sin\theta(t)=t^2.
    \end{equation}
    Le mieux serait de résoudre cette équation pour trouver \( \theta(t)\). Mais en réalité il n'est pas nécessaire de résoudre : montrer qu'il existe une solution suffit. Nous pouvons supposer que \( t^2<1\). Pour \( \theta=\pi/4\) nous avons \( \cos(\theta)-\sin(\theta)=0\) et pour \( \theta=0\) nous avons \( \cos(\theta)-\sin(\theta)=1\). Le théorème des valeurs intermédiaires nous enseigne alors qu'il existe une valeur de \( \theta\) qui résout l'équation \eqref{EqICrDSe}.

    %TODO : le citer lorsque le fork sera fait.
    Pour être rigoureux, nous devons aussi montrer que la fonction \( \theta(t)\) est continue. Pour cela il faudrait utiliser le \wikipedia{fr}{Théorème_des_fonctions_implicites}{théorème de la fonction implicite}. Nous verrons dans l'exemple \ref{ExmeASDLAf} comment s'en sortir sans théorème de la fonction implicite, au prix de plus de calculs.
\end{example}

\begin{example}
	Considérons encore la fonction 
	\begin{equation}
		f(x,y)=\frac{ x^2+y^2 }{ x-y }.
	\end{equation}
	Une mauvaise idée pour prouver que la limite n'existe pas pour $(x,y)\to(0,0)$ est de considérer le chemin $(t,t)$. En effet, la fonction n'existe pas sur ce chemin. Or la méthode des chemins parle uniquement de chemins contenus dans le domaine de la fonction.
\end{example}

\begin{example}     \label{ExmeASDLAf}
	Revenons encore et toujours sur la fonction 
	\begin{equation}
		(x^2+y^2)/(x-y).
	\end{equation}
	Nous prouvons que la limite n'existe pas en trouvant des chemins le long desquels les limites sont différentes. Si nous essayons le chemin \( (t,kt)\) avec \( k\) constant, nous trouvons
    \begin{equation}
        f(t,kt)=\frac{ t(1+k^2) }{ 1-k }.
    \end{equation}
    La limite \( t\to 0\) est hélas toujours \( 0\). Nous ne pouvons donc pas conclure.

    Nous allons maintenant utiliser la même technique que celle utilisée en coordonnées polaires. Vous noterez que dans ce cas, travailler en cartésiennes donne lieu à des calculs plus longs.  L'astuce consiste à prendre \( k\) non constant et à chercher par exemple \( k(t)\) de façon à avoir
    \begin{equation}
        \frac{ 1+k(t)^2 }{ 1-k(t) }=\frac{1}{ t }.
    \end{equation}
    Avec une telle fonction, la fonction \( t\mapsto f(t,tk(t))\) serait la constante \( 1\). L'équation à résoudre pour \( k\) est
    \begin{equation}
        tk^2+k+(t-1)=0,
    \end{equation}
    et les solutions sont
    \begin{equation}
        k(t)=\frac{ -1\pm\sqrt{1-4t(t-1)} }{ 2t }.
    \end{equation}
    Nous proposons donc les chemins
    \begin{equation}
        \begin{pmatrix}
            x    \\ 
            y    
        \end{pmatrix}=\begin{pmatrix}
            t    \\ 
            \frac{ -1\pm\sqrt{1-4t(t-1)}    }{2}
        \end{pmatrix}
    \end{equation}
    Nous devons vérifier deux points. D'abord que ce chemin est bien défini, et ensuite que \( tk(t)\) tend bien vers zéro lorsque \( t\to 0\) (sinon \( (t,k(t)t)\)) n'est pas un chemin passant par \( (0,0)\). Lorsque \( t\) est petit, ce qui se trouve sous la racine est proche de \( 1\) et ne pose pas de problèmes. Ensuite,
    \begin{equation}
        \lim_{t\to 0} tk(t)=\frac{ -1\pm 1 }{ 2 }.
    \end{equation}
    En choisissant le signe \( +\), nous trouvons un chemin qui nous convient. 

    Ce que nous avons prouvé est que
    \begin{equation}
        f\left( t,   \frac{ -1\pm\sqrt{1-4t(t-1)}    }{2}\right)=1
    \end{equation}
    pour tout \( t\). Le long de ce chemin, la limite de \( f\) est donc \( 1\). Cette limite est différente des limites obtenues le long de chemins avec \( k\) constant. La limite \( \lim_{(x,y)\to (0,0)} f(x,y)\) n'existe donc pas.
\end{example}

\begin{example}\label{seno}
	Considérons la fonction (figure \ref{LabelFigsenotopologo})
	\begin{equation}
		f(x,y)=\begin{cases}
			\sqrt{x^2+y^2}\sin\frac{1}{ x^2+y^2 }	&	\text{si $(x,y)\neq(0,0)$}\\
			0	&	 \text{si $(x,y)=(0,0)$},
		\end{cases}
	\end{equation}
	et cherchons la limite $(x,y)\to(0,0)$. Le passage en coordonnées polaires donne
	\begin{equation}		\label{EqFoncRho2907}
		f(\rho,\theta)=\rho\sin\frac{1}{ \rho }.
	\end{equation}
	Pour calculer la limite de cela lorsque $\rho\to 0$, nous remarquons que
	\begin{equation}
		0\leq|\rho\sin\frac{1}{ \rho }|\leq\rho
	\end{equation}
	parce que $\sin(\frac{1}{ \rho })\leq 1$ quel que soit $\rho$. Or évidement $\lim_{\rho\to 0} \rho=0$, donc la limite de la fonction \eqref{EqFoncRho2907} est zéro et ne dépend pas de $\theta$. Nous en concluons que $\lim_{(x,y)\to(0,0)}f(x,y)=0$.
\end{example}
\newcommand{\CaptionFigsenotopologo}{La fonction de l'exemple \ref{seno}.}
\input{Fig_senotopologo.pstricks}

%---------------------------------------------------------------------------------------------------------------------------
\subsection{Méthode du développement asymptotique}
%---------------------------------------------------------------------------------------------------------------------------

Nous savons que nous pouvons développer certaines fonctions en série grâce au développement de Taylor (théorème \ref{ThoTaylor}). Lorsque nous avons une limite à calculer, nous pouvons remplacer certaines parties de la fonction à traiter par la formule \ref{subeqfTepseqb}. Cela est très utile pour comparer des fonctions trigonométrique à des polynômes.

\begin{example}		\label{ExamLimSinxxa}
	La limite $\lim_{x\to 0} \frac{ \sin(x) }{ x }=1$ est bien connue. Une manière de la prouver des d'écrire
	\begin{equation}
		\sin(x)=x+h(x)
	\end{equation}
	avec $h\in o(x)$, c'est à dire $\lim_{x\to 0} h(x)/x=0$. Alors nous avons
	\begin{equation}
		\lim_{x\to 0} \frac{ \sin(x) }{ x }=\lim_{x\to 0} \frac{ x+h(x) }{ x }=\lim_{x\to 0} \frac{ x }{ x }+\lim_{x\to 0} \frac{ h(x) }{ x }=1.
	\end{equation}
\end{example}

L'utilisation de la proposition \ref{PropLimCompose} permet d'utiliser cette technique dans le cadre de limites à plusieurs variables. Reprenons l'exemple \ref{ExamLimSinxxa} un tout petit peu modifié :
\begin{example}
	Soit à calculer $\lim_{(x,y)\to(0,0)}f(x,y)$ où
	\begin{equation}
		f(x,y)=\frac{ \sin(xy) }{ xy }.
	\end{equation}
	La première chose à faire est de voir $f$ comme la composée de fonctions $f=f_1\circ f_2$ avec
	\begin{equation}
		\begin{aligned}
			f_1\colon \eR&\to \eR \\
			t&\mapsto \frac{ \sin(t) }{ t } 
		\end{aligned}
	\end{equation}
	et
	\begin{equation}
		\begin{aligned}
			f_2\colon \eR^2&\to \eR \\
			(x,y)&\mapsto xy. 
		\end{aligned}
	\end{equation}
	 Étant donné que $\lim_{(x,y)\to(0,0)}f_2(x,y)=0$, nous avons $\lim_{(x,y)\to(0,0)}f(x,y)=\lim_{t\to 0} f_1(t)=1$.
\end{example}




%+++++++++++++++++++++++++++++++++++++++++++++++++++++++++++++++++++++++++++++++++++++++++++++++++++++++++++++++++++++++++++
\section{Continuité}
%+++++++++++++++++++++++++++++++++++++++++++++++++++++++++++++++++++++++++++++++++++++++++++++++++++++++++++++++++++++++++++
\label{SecContinue}

\begin{definition}		\label{DefFonctContinueRR}
	Soit une fonction $f\colon D\to \eR$ et un point $a$ dans $D$. Nous disons que $f$ est \defe{continue}{continue!fonction réelle} lorsque $f$ possède une limite en $a$ et $\lim_{x\to a} f(x)=f(a)$.
\end{definition}
En remplaçant $\ell$ par $f(a)$ dans la définition de la limite, nous exprimons la continuité de $f$ en $a$ par la façon suivante. Pour tout $\varepsilon>0$, il existe un $\delta>0$ tel que $\forall x\in D$,
\begin{equation}
	| x-a |<\delta\Rightarrow \big| f(x)-f(a) \big|<\varepsilon.
\end{equation}

\begin{theorem}[Accroissement finis]		\label{ThoAccFinisUneVariable}
	Soit $f$ une fonction continue sur $\mathopen[ a , b \mathclose]$ et dérivable sur $\mathopen] a , b \mathclose[$. Alors il existe au moins un réel $c\in\mathopen] a , b \mathclose[$ tel que $f(b)-f(a)=f'(c)(b-a)$.
\end{theorem}

\begin{theorem}[Valeurs intermédiaires]		\label{ThoValsIntern}
	Soit $I$ un intervalle de $\eR$ et $f$ une fonction continue sur $I$ avec $f(a)<f(b)$. Alors $\mathopen[ f(a) , f(b) \mathclose]\subset f(I)$
\end{theorem}

Cette propriété signifie que si une fonction passe par la hauteur $y_1$ et puis par la hauteur $y_2$, alors elle passe par toutes les hauteurs intermédiaires.

\begin{corollary}
	Soit $f\colon I\to \eR$ une fonction continue et deux points $a<b$ dans $I$ tels que $f(a)<0$ et $f(b)>0$. Alors il existe $c\in\mathopen] a , b \mathclose[$ tel que $f(c)=0$.
\end{corollary}


%+++++++++++++++++++++++++++++++++++++++++++++++++++++++++++++++++++++++++++++++++++++++++++++++++++++++++++++++++++++++++++
\section{Développement asymptotique, théorème de Taylor}
%+++++++++++++++++++++++++++++++++++++++++++++++++++++++++++++++++++++++++++++++++++++++++++++++++++++++++++++++++++++++++++
\label{AppSecTaylorR}

\begin{theorem}[Théorème de Taylor]		\label{ThoTaylor}
Soit $I\subset$ un intervalle non vide et non réduit à un point de $\eR$ ainsi que $a\in I$. Soit une fonction $f\colon I\to \eR$ telle que $f^{(n)}(a)$ existe. Alors il existe une fonction $\epsilon$ définie sur $I$ et à valeurs dans $\eR$ vérifiant les deux conditions suivantes :
\begin{subequations}		\label{SubEqsDevTauil}
	\begin{align}
		\lim_{x\to a}\epsilon(x)&=0,\\
		f(x)&=T^a_{f,n}(x)+\epsilon(x)(x-a)^{n}	&&\forall x\in I		\label{subeqfTepseqb}
	\end{align}
\end{subequations}
où $T^a_{f,n}(x)=\sum_{k=0}^n\frac{ f^{(k)}(a) }{ k! }(x-a)^k$ et $f^{(k)}$ dénote la $k$-ième dérivée de $f$ (en particulier, $f^{(0)}=f$, $f^{(1)}=f'$).\nomenclature{$f^{(n)}$}{La $n$-ième dérivée de la fonction $f$}
\end{theorem}

Nous insistons sur le fait que la formule \eqref{subeqfTepseqb} est une égalité, et non une approximation. Ce qui serait une approximation serait de récrire la formule dans le terme contenant $\epsilon$.

Le polynôme $T^a_{f,n}$ est le \defe{polynôme de Taylor}{Taylor} de $f$ au point $a$ à l'ordre $n$.  Une preuve du théorème peut être trouvée dans \cite{TrenchRealAnalisys}, théorème 2.5.1 à la page 99. La version donnée ici est inspirée de l'article sur \wikipedia{fr}{Développement_de_Taylor}{Wikipédia}\footnote{http://fr.wikipedia.org/wiki/Développement\_de\_Taylor}, qui donne également une preuve du résultat.

En termes de notations, nous définissons l'ensemble $o(x)$\nomenclature{$o(x)$}{fonction tendant rapidement vers zéro} l'ensemble des fonctions $f$ telles que
\begin{equation}
	\lim_{x\to 0} \frac{ f(x) }{ x }=0.
\end{equation}
Plus généralement si $g$ est une fonction telle que $\lim_{x\to 0} g(x)=0$, nous disons $f\in o(g)$ si
\begin{equation}
	\lim_{x\to 0} \frac{ f(x) }{ g(x) }=0.
\end{equation}
De façon intuitive, l'ensemble $o(g)$ est l'ensemble des fonctions qui tendent vers zéro «plus vite» que $g$.


Nous pouvons donner un énoncé alternatif au théorème \ref{ThoTaylor} en définissant $h(x)=\epsilon(x+a)x^n$. Cette fonction est définie exprès pour avoir
\begin{equation}
	h(x-a)=\epsilon(x)(x-a)^n,
\end{equation}
et donc
\begin{equation}
	\lim_{x\to 0} \frac{ h(x) }{ x^n }=\lim_{x\to 0} \epsilon(x-a)=\lim_{x\to a}\epsilon(x)=0. 
\end{equation}
Donc $h\in o(x^n)$.

Le théorème dit donc qu'il existe une fonction $\alpha\in o(x^n)$ telle que
\begin{equation}
	f(x)=T^a_{f,n}(x)+\alpha(x-a).
\end{equation}
pour tout $x\in I$. Pour plus de détails et quelque exemples, voir \cite{ExoCdI1}.

\begin{example}
	Le développement du cosinus est donné par
	\begin{equation}
		\cos(x)=1-\frac{ x^2 }{ 2 }+\frac{ x^4 }{ 4! }-\frac{ x^6 }{ 6! }\cdots
	\end{equation}
	Nous avons donc l'existence d'une fonction $h_1\in o(x^2)$ telle que $\cos(x)=1-\frac{ x^2 }{ 2 }+h_1(x)$. Il existe aussi une autre fonction $h_2\in o(x^4)$ telle que $\cos(x)=1-\frac{ x^2 }{ 2 }+\frac{ x^4 }{ 4! }+h_2(x)$.
\end{example}

\begin{example}		\label{ExempleUtlDev}
	Une des façons les plus courantes d'utiliser les formules \eqref{SubEqsDevTauil} est de développer $f(a+t)$ pour des petits $t$ en posant $x=a+t$ dans la formule :
	\begin{equation}	\label{EqDevfautouraeps}
		f(a+t)=f(a)+f'(a)t+f''(a)\frac{ t^2 }{ 2 }+\epsilon(a+t)t^2
	\end{equation}
	avec $\lim_{t\to 0} \epsilon(a+t)=0$. Ici, la fonction $T$ dont on parle dans le théorème est $T_{f,2}^a(a+t)=f(a)+f'(a)t+f''(a)\frac{ t^2 }{2}$.

	Lorsque $x$ et $y$ sont deux nombres «proches\footnote{par exemple dans une limite $(x,y)\to(h,h)$.}», nous pouvons développer $f(y)$ autour de $f(x)$ :
	\begin{equation}		\label{Eqfydevfx}
		f(y)=f(x)+f'(x)(y-x)+f''(x)\frac{ (y-x)^2 }{ 2 }+\epsilon(y-x)(y-x)^2,
	\end{equation}
	et donc écrire
	\begin{equation}
		f(x)-f(y)=-f'(x)(y-x)-f''(x)\frac{ (y-x)^2 }{ 2 }-\epsilon(y-x)(y-x)^2.
	\end{equation}
	De cette manière nous obtenons une formule qui ne contient plus que $y$ dans la différence $y-x$.
\end{example}

%+++++++++++++++++++++++++++++++++++++++++++++++++++++++++++++++++++++++++++++++++++++++++++++++++++++++++++++++++++++++++++
\section{Fonctions à valeurs dans $\eR^n$}
%+++++++++++++++++++++++++++++++++++++++++++++++++++++++++++++++++++++++++++++++++++++++++++++++++++++++++++++++++++++++++++

À peu près toutes les notions que vous connaissez à propos de fonctions de $\eR$ dans $\eR$ se généralises immédiatement au cas de fonctions de $\eR$ dans $\eR^n$.

Nous disons que la fonction $f\colon \eR\to \eR^n$ est \defe{de classe $C^1$}{classe $C^1$} si chacune de ses composantes $f_i$ est de classe $C^1$ en tant que fonctions de $\eR$ dans $\eR$.

La dérivée de $f$ est donnée par la dérivée composante par composante. Pour l'intégrale de $f$, il en va de même : composante par composante. 
\begin{equation}
	\int f(x)dx=\big(  \int f_1(x)dx,\,\int f_2(x)dx,\ldots,\int f_n(x)dx   \big).
\end{equation}

Par exemple si nous considérons le mouvent d'une particule sur une hélice, la position est donnée par
\begin{equation}
	f(t)=\big( R\sin(t),R\cos(t),t \big).
\end{equation}
La vitesse est donnée par
\begin{equation}
	f'(t)=\big( R\cos(t),-R\sin(t),1 \big),
\end{equation}
et l'intégrale sera donnée par
\begin{equation}
	\int f(t)dt=\big( -R\cos(t)+C_1,R\sin(t)+C_2,\frac{ t^2 }{ 2 }+C_3 \big).
\end{equation}

Si nous considérons une pierre lancée horizontalement du sommet d'une falaise avec une vitesse initiale $v_0$, la vitesse de la pierre sera donnée par
\begin{equation}
	v(t)=(v_0,gt).
\end{equation}
Pour trouver la position, nous intégrons la vitesse par rapport au temps :
\begin{equation}
	f(t)=\int v(t)dt=\big( v_0t+C_1,\frac{ gt^2 }{ 2 }+C_2 \big).
\end{equation}
Notez qu'il faut une constante d'intégration différente pour chaque composantes.

\begin{lemma}			\label{LemIneqnormeintintnorm}
	Pour toute fonction $u\colon \mathopen[ a , b \mathclose]\to \eR^n$, nous avons
	\begin{equation}
		\| \int_a^bu(t)dt\|\leq\int_a^b\| u(t) \|dt
	\end{equation}
	pourvu que le membre de gauche ait un sens.
\end{lemma}

\begin{proof}
	Étant donné que $\int_a^bu(t)dt$ est un élément de $\eR^n$, par la proposition \ref{LemSclNormeXi}, il existe un $\xi\in\eR^n$ de norme $1$ tel que
	\begin{equation}
		\| \int_a^bu(t)dt \|=\xi\cdot\int_a^b u(t)dt=\int_a^b u(t)\cdot\xi dt\leq\int_a^b\| u(t) \|   \| \xi \|=\int_a^b\| u(t) \|dt.
	\end{equation}
\end{proof}
% This is part of Géométrie analytique
% Copyright (c) 2010-2011
%   Laurent Claessens
%   Carlotta Donadello
% See the file fdl-1.3.txt for copying conditions.

%+++++++++++++++++++++++++++++++++++++++++++++++++++++++++++++++++++++++++++++++++++++++++++++++++++++++++++++++++++++++++++
\section{Graphes de fonctions de plusieurs variables}		\label{SecGraphesFonc}
%+++++++++++++++++++++++++++++++++++++++++++++++++++++++++++++++++++++++++++++++++++++++++++++++++++++++++++++++++++++++++++

La plus grande partie de ce cours est consacrée à l'étude des fonction de plusieurs variables. Nous allons maintenant donner quelques indication sur comment <<dessiner>> une telle fonction. Vous connaissez déjà la définition de graphe pour une fonction $f$ d'une seule variable à valeurs dans $\eR$ : c'est l'ensemble des point du plan de la forme $(x, f(x))$. Vous voyez que cet ensemble n'est pas vraiment un gros morceau de $\eR^2$ parce que son intérieur est vide : il y a une seule valeur de $f$ qui correspond au point $x$, donc une boule de $\eR^2$ centrée en $(x, f(x))$ de n'importe quel rayon contient toujours des points qui ne font pas partie du graphe de $f$. 

%La première chose qu'on a envie de dire est que un tel graphe est une courbe dans $\eR^2$ mais cela n'est pas toujours vrai. Le graphe de la fonction cosinus est bien une courbe dans dans le plan, mais le graphe de la fonction tangente est une réunion infinie de courbes. Ce qui est vrai est que le graphe d'une fonction d'une variable est \emph{localement} une courbe si la fonction n'est pas trop mal choisie. % exemple? 

Le chapitre \ref{Chap_courbes} vous donnera plus d'informations sur les courbes dans l'espace.
Nous voulons donner une définition assez générale pour le graphe d'une fonction
\begin{definition}
  Soit $f$ une fonction de $\eR^m$ dans $\eR^n$. Le \defe{graphe}{graphe d'une fonction} de $f$ est la partie de $\eR^m\times \eR^n$ de la forme
  \begin{equation}
    \Graph f= \{ (x,y)\in \eR^m\times \eR^n \,|\, y=f(x)\}.
  \end{equation}
\end{definition}
Si $f$ est une fonction de deux variables indépendantes $x$ et $y$ à valeurs dans $\eR$, alors un point dans le graphe de $f$ est un point $(x,y,z)\in\eR^3$ tel que
\begin{equation}
	z=f(x,y),
\end{equation}
ou encore, un point de la forme
\begin{equation}
	\big( x,y,f(x,y) \big).
\end{equation}
%Si $g$ est une fonction d'une variable $x$ à valeurs dans $\eR^2$, alors un point dans le graphe de $g$ prend la forme $(x,g_1(x), g_2(x))$, où $g_1$ et $g_2$ sont les composantes de $g$.  Dans le deux cas le graphe est un sous-ensemble de $\eR^3$. 
Ici nous sommes intéressés par les fonctions de plusieurs variables à valeurs dans $\eR$. Donc, notre définition se spécialise 
\begin{definition}
  Soit $f$ une fonction de $\eR^m$ dans $\eR$. Le graphe de $f$ est la partie de $\eR^m\times \eR$ donné par
  \begin{equation}
    \Graph f= \{ (x,y)\in \eR^m\times \eR \,|\, y=f(x)\}.
  \end{equation}
\end{definition}  
Étant donné que nous ne donneront des exemples que de fonctions de $\eR^2$ dans $\eR$, la définition devient
\begin{equation}
	\Graph f= \{ (x,y,z)\in\eR^2\tq z=f(x,y) \}.
\end{equation}
C'est cette définition qu'il faut garder à l'esprit lorsqu'on travaille sur des dessins en trois dimensions.

%Nous considérons d'abord le cas d'une fonction $f$  de deux variables $x$ et $y$ à valeurs dans $\eR$. L'espace $\eR^3$ a trois dimensions, cela veut dire que il faut fixer trois paramètres indépendants pour désigner un point de manière unique (voir, au cours d'une deuxième lecture de ces notes, la section sur les coordonnées cylindriques et sphériques, \ref{sec_coord}). Le graphe d'une fonction comme $f$ est un sous-ensemble de $\eR^3$ où l'un des trois paramètres est d'office la valeur de $f$, donc il est décrit par seulement deux paramètres $x$ et $y$. Son intérieur est alors vide et, si $f$ est une fonction <<suffisamment gentille>>, $\Graph f$ est localement une surface dans $\eR^3$.    

Nous avons parfois besoin de donner des représentation graphiques d'une fonction. Nous pouvons, par exemple, penser à la fonction que associe à un point de la Terre son altitude. Lorsqu'on part pour une promenade en montagne on a envie de connaitre le graphe de cette fonction qui correspond en fait à la surface de la montagne. Bien sur nous ne voulons pas amener avec nous un modèle en 3D de la montagne donc il nous faut une méthode efficace pour projeter le graphe de $f$ sur le plan $x$-$y$ tout en gardant les informations fondamentales. Pour cela nous avons besoin de deux définitions (à ne pas confondre !)
\begin{definition}
	Soit $f$ une fonction de $\eR^2$ dans $\eR$ et soit $c$ dans $\eR$.  La \defe{$z$-section}{section!de graphe} de $\Graph f$ à la hauteur $c$ est donné par
\[
S^z_c=\{ (x,y,c)\in \eR^3\,|\, f(x,y)=c\}.
\]  
\end{definition}
\begin{definition}\label{def_niveau}
	Soit $f$ une fonction de $\eR^2$ dans $\eR$ et soit $c$ dans $\eR$. La \defe{courbe de niveau}{courbe de niveau} de $f$ à la hauteur $c$ est l'ensemble
\[
N_c=\{ (x,y)\in \eR^2\,|\, f(x,y)=c\}.
\]  
\end{definition}
On peut représenter la fonction $f$ d'une façon très précise en traçant quelques unes de ses courbes de niveau.  Dans la suite on pourra considérer aussi les $x$-sections et les $y$-sections du graphe d'une fonction de deux variables. La $x$-section de $\Graph f$ à la hauteur $a$ est     
\[
S^x_a=\{(a,y,z)\in\eR^3\,|\, f(a,y)=z\}.
\]
Comme vous avez peut être déjà compris, $S^x_a$ est le graphe de la fonction de $y$ qu'on obtient de $f$ en fixant $x=a$. Cette fonction est appelée $x$-section de $f$ pour $x=a$.


Certaines surfaces dans $\eR^3$ sont le graphe d'une fonction. 
\begin{example}
	Quelque graphes importants.
  \begin{description}
    \item[Un plan non vertical] Tout plan dans $\eR^3$ peut être décrit par une équation de la forme 
\[
a(x-x_0)+ b(y-y_0) + c(z-z_0) = r,
\] 
où, $(x_0, y_0, z_0)$ est vecteur dans $\eR^3$, et $a$, $b$, $c$ et $r$ sont des nombres réels. Si $c\neq 0$ alors le plan n'est pas vertical et on peut dire que il est le graphe de la fonction 
\[
P(x,y)= \frac{r+cz_0 -a(x-x_0)-b(y-y_0)}{c},
\]
quitte à choisir des nouvelles constantes $s$, $t$, $q$,
\[
P(x,y)=sx +ty +q.
\]
    \item[Un paraboloïde elliptique] Pour tous $\alpha$ et $\beta$ dans $\eR$ les  graphes des fonctions 
\[
PE_1(x,y)=\frac{x^2}{\alpha^2}+\frac{y^2}{\beta^2}
\]
ou de la fonction 
\[
PE_2(x,y)=-\frac{x^2}{\alpha^2}-\frac{y^2}{\beta^2}
\]
sont des paraboloïdes elliptiques. Le premier est contenu dans le demi-espace $z\geq 0$, l'autre dans $z\leq 0$. Le nom de cette surface vient de la forme de ses sections. En fait toutes  sections $S^z_c$ sont des ellipses, alors que les section $S^x_a$ et $S^y_b$ sont des paraboles.   
    \item[Un paraboloïde hyperbolique (selle)]  Pour tous $\alpha$ et $\beta$ dans $\eR$ les  graphes des fonctions 
\[
PH_1(x,y)=\frac{x^2}{\alpha^2}-\frac{y^2}{\beta^2}
\]
ou de la fonction 
\[
PH_2(x,y)=-\frac{x^2}{\alpha^2}+\frac{y^2}{\beta^2}
\]
sont des paraboloïdes hyperboliques. Remarquez que les  sections $S^z_c$ de ce graphe sont des hyperboles, alors que les section $S^x_a$ et $S^y_b$ sont des paraboles.   
    \item[Une demi-sphère] La fonction $S^+=\sqrt{R^2-x^2-y^2}$ a pour graphe la demi-sphère supérieure centrée en l'origine et de rayon $R$.  
Le dernier de ces exemples nous signale une chose très importante : une sphère entière n'est pas le graphe d'une fonction de $x$ et $y$. Par contre, une demi-sphère est bien le graphe de la fonction $f(x,y)=\sqrt{1-x^2-y^2}$.

L'équation que nous utilisons  pour d'écrire une sphère de rayon $R$ centrée en l'origine est 
\[
x^2+y^2+z^2=R^2
\] 
Donc, à  chaque point  $(x,y)$ dans le disque $x^2+y^2\leq R^2$ (notez que ce disque est contenu dans la section $S^z_0$), on peut associer deux valeurs de $z$ : $z_1=\sqrt{R^2-x^2-y^2}$ et  $z_2=-\sqrt{R^2-x^2-y^2}$. Par définition, une fonction n'associe qu'un seul valeur à chaque point de son domaine, d'où l'impossibilité de décrire cette sphère comme le graphe d'une fonction de $x$ et $y$.

  \end{description}
\end{example}

Considérons la fonction $Sp: \eR^3\to \eR$ qui associe à $(x,y,z)$ la valeur $x^2+y^2+z^2$. La sphère de rayon $R$ centrée en l'origine est l'ensemble de niveau $N_{R^2}$ de $Sp$. L'ensemble de niveau $N_{0}$ de $Sp$ est l'origine, et tous les ensemble de niveau de hauteur négative sont vides. La même chose est vraie pour les ellipsoïdes centrées en l'origine avec les axes $x$, $y$ et $z$ comme axes principaux et comme longueurs de demi-axes $a$, $b$ et $c$. Voici la fonction dont il sont les ensemble de niveau 
\[
El(x,y,z)= \frac{x^2}{a^2}+\frac{y^2}{b^2}+\frac{z^2}{c^2}.
\] 
\begin{example}
	Des ensembles de niveau importants.
  \begin{description}
    \item[Tout graphe] 
	    Le graphe de toute fonction $f$  de $\eR^2$ dans $\eR$ peut être considéré comme l'ensemble de niveau zéro de la fonction $F(x,y,z)=z-f(x,y)$.

    \item[Hyperboloïdes]
	    Les hyperboloïdes, comme les ellipsoïdes, sont une famille d'ensemble de niveau. En particulier, nous considérons des hyperboloïdes dont l'axe de symétrie est l'axe des $z$ et qui sont symétriques par rapport un plan $x$-$y$.  Une fois que les paramètres  $a$, $b$ et $c$ sont fixés la fonction que nous intéresse est 
\[
Hyp(x,y,z)= \frac{x^2}{a^2}+\frac{y^2}{b^2}-\frac{z^2}{c^2}.
\]
Les ensembles de niveau $N_d$ pour $d>0$ sont connexes, on les appelle \emph{hyperboloïdes à une feuille}. L'ensemble de niveau $N_0$ est \emph{cône (elliptique)}, le deux moitiés du cône se touchent en l'origine. Enfin, les ensembles de niveau $N_d$ pour $d<0$ ne sont  pas connexes et pour cette raison on les appelle \emph{hyperboloïdes à deux feuilles}.
  \end{description}
\end{example}


%+++++++++++++++++++++++++++++++++++++++++++++++++++++++++++++++++++++++++++++++++++++++++++++++++++++++++++++++++++++++++++
\section{Fonction sur des compacts}		\label{SecFonctionsSurCompacts}
%+++++++++++++++++++++++++++++++++++++++++++++++++++++++++++++++++++++++++++++++++++++++++++++++++++++++++++++++++++++++++++

\begin{definition}
	Une partie $A\subset\eR^m$ est dite \defe{bornée}{bornée!partie de $\eR^m$} si il existe un $M>0$ tel que $A\subset B(0,M)$. Le \defe{diamètre}{diamètre} de la partie $A$ est\nomenclature[T]{$\Diam(A)$}{Diamètre de la partie $A$} le nombre
	\begin{equation}
		\Diam(A)=\sup_{x,y\in A}\| x-y \|\in\mathopen[ 0 , \infty \mathclose].
	\end{equation}
\end{definition}
Lorsque $A$ est borné, il existe un $M$ tel que $\| x \|\leq M$ pour tout $x\in A$.

\begin{lemma}
	Si $A$ est une partie non vide de $\eR^m$, alors $\Diam(A)=\Diam(\bar A)$.
\end{lemma}
Nous n'allons pas donner de démonstrations de ce lemme.

\begin{definition}
	Une partie de $\eR^m$ qui est à la fois bornée et fermée est dite \defe{compacte}{compact!dans $\eR^m$}.
\end{definition}

Si $(x_n)$ est une suite et $I$ est un sous-ensemble infini de $\eN$, nous désignons par $x_I$ la suite des éléments $x_n$ tels que $n\in I$. Par exemple la suite $x_{\eN}$ est la suite elle-même, la suite $x_{2\eN}$ est la suite obtenue en ne prenant que les éléments d'indice pair.

Les suites $x_I$ ainsi construites sont dites des \defe{sous-suites}{sous-suite} de la suite $(x_n)$.


\begin{theorem}
	Toute suite réelle contenue dans un ensemble borné admet une sous-suite convergente.
\end{theorem}

\begin{proof}
	Soit $(x_n)$ une suite contenue dans la partie bornée $A\subset\eR$. Nous disons qu'un élément $x_k$ de la suite est \emph{maximal} si il est plus grand ou égal que tous les suivants : $x_k\geq x_{k'}$ dès que $k'\geq k$.

	Si la suite a un nombre infini d'éléments maximaux, alors la suite des éléments maximaux est décroissante. Si nous n'avons qu'un nombre fini de points maximaux, alors la suite de départ est croissante à partir du dernier point maximal.

	Dans les deux cas nous avons trouvé une sous-suite des $x_n$ qui est monotone (décroissante ou croissante selon le cas), et donc convergente parce que contenue dans un borné (lemme \ref{LemSuiteCrBorncv}).
\end{proof}

Ce théorème se généralise à $\eR^m$ de la façon suivante.
\begin{theorem}[Théorème de Bolzano-Weierstrass]		\label{ThoBolzanoWeierstrassRn}
	Toute suite contenue dans un ensemble borné de $\eR^m$ possède une sous-suite convergente.
\end{theorem}

\begin{proof}
	Soit $(x_n)$ une suite contenue dans une partie bornée de $\eR^m$. Considérons $(a_n)$, la suite réelle des premières composantes des éléments de $(x_n)$ : pour chaque $n\in\eN$, le nombre $a_n$ est la première composante de $x_n$. Étant donné que la suite $(x_n)$ est bornée, il existe un $M$ tel que $\| x_n \|<M$. En utilisant la relation \eqref{Equilequnorme}, nous avons
	\begin{equation}
		| a_n |\leq\| x_n \|\leq M.
	\end{equation}
	La suite $(a_n)$ est donc une suite réelle bornée et donc contient une sous-suite convergente. Soit $a_{I_1}$ une sous-suite convergente de $(a_n)$. Nous considérons maintenant $x_{I_1}$, c'est à dire la suite de départ dont on a enlevé tous les éléments qu'il faut pour qu'elle converge en ce qui concerne la première composante.

	Si nous considérons la suite $b_{I_1}$ des \emph{secondes} composantes de $x_{I_1}$, nous en extrayons, de la même façon que précédemment, une sous-suite convergente, c'est à dire que nous avons un $I_2\subset I_1$ tel que $b_{I_2}$ est convergent. Notons que $a_{I_2}$ est une sous-suite de la (sous) suite convergente $x_{I_1}$, et donc $a_{I_2}$ est encore convergente.

	En continuant ainsi, nous construisons une sous-sous-sous-suite $x_{I_3}$ telle que la suite des \emph{troisième} composantes est convergente. Lorsque nous avons effectué cette procédure $m$ fois, la suite $x_{I_m}$ est une suite dont toutes les composantes convergent, et donc est une suite convergente par la proposition \ref{PropCvRpComposante}.
	
	Le tableau suivant donne un petit schéma de la façon dont nous procédons. Les $\bullet$ sont les éléments de la suite que nous gardons, et les $\times$ sont ceux que nous «jetons».
	\begin{equation}
		\begin{array}{lccccccccccc}
			x_{\eN}	&	\bullet&\bullet&\bullet&\bullet&\bullet&\bullet&\bullet&\bullet&\bullet&\bullet&\ldots\\
			x_{I_1}	&	\times&\bullet&\bullet&\times&\bullet&\times&\times&\bullet&\bullet&\bullet&\ldots\\
			x_{I_2}	&	\times&\bullet&\times&\times&\bullet&\times&\times&\bullet&\bullet&\times&\ldots\\
			\vdots\\
			x_{I_m}	&	\times&\times&\times&\times&\bullet&\times&\times&\times&\bullet&\times&\ldots
		\end{array}
	\end{equation}
	La première ligne, $x_{\eN}$, est la suite de départ.
\end{proof}

Le théorème de Bolzano–Weierstrass a l'importante conséquence suivante.
\begin{theorem}[Weierstrass]		\label{ThoWeirstrassRn}
	Une fonction continue à valeurs réelles définie sur un compact de $\eR^m$ est bornée et atteint ses bornes.
\end{theorem}

\begin{proof}
	Soit $K$ une partie compacte de $\eR^m$ et $f\colon K\to \eR$ une fonction. Nous désignons par $A$ l'ensemble des valeurs prises par $f$ sur $K$ :
	\begin{equation}
		A=f(K)=\{ f(x)\tq x\in K \}.
	\end{equation}
	Nous considérons le supremum $M=\sup A=\sup_{x\in K}f(x)$ avec la convention comme quoi si $A$ n'est pas borné supérieurement, nous posons $M=\infty$ (voir définition \ref{DefSupeA}).

	Nous allons maintenant construire une suite $(x_n)$ de deux façons différentes suivant que $M=\infty$ ou non.
	\begin{enumerate}
		\item
			Si $M=\infty$, nous choisissons, pour chaque $n\in\eN$, un $x_n\in K$ tel que $f(x_n)>n$. Cela est certainement possible parce que si $A$ n'est pas borné, nous pouvons y trouver des nombres aussi grands que nous voulons.
		\item
			Si $M<\infty$, nous savons que pour tout $\varepsilon$, il existe un $y\in A$ tel que $y>M-\varepsilon$. Pour chaque $n$, nous choisissons donc $x_n\in K$ tel que $f(x_n)>M-\frac{1}{ n }$.
	\end{enumerate}
	Quel que soit le cas dans lequel nous sommes, la suite $(x_n)$ est une suite dans $K$ qui est un ensemble borné, et donc nous pouvons en extraire une sous-suite convergente. Afin d'alléger la notation, nous allons noter $(x_n)$ la sous-suite convergente. Nous avons donc 
	\begin{equation}
		x_n\to x\in \bar D=D
	\end{equation}
	par le lemme \ref{LemLimAbarA}. Par la proposition \ref{PropFnContParSuite}, nous avons que $f$ est définie en $x$ et prend la valeur
	\begin{equation}
		\lim_{n\to \infty} f(x_n)=f(x).
	\end{equation}
	Donc $f(x)<\infty$. Évidement, si nous avions été dans le cas où $M=\infty$, la suite $x_n$ aurait été choisie pour avoir $f(x_n)>n$ et donc il n'aurait pas été possible d'avoir $\lim_{n\to \infty} f(x_n)<\infty$. Nous en concluons que $M<\infty$, et donc que $f$ est bornée sur $K$.

	Afin de prouver que $f$ atteint sa borne, c'est à dire que $M\in A$, nous considérons les inégalités
	\begin{equation}
		M-\frac{1}{ n }<f(x_n)\leq M.
	\end{equation}
	En passant à la limite $n\to \infty$, ces inégalités deviennent
	\begin{equation}
		M\leq f(x)\leq M,
	\end{equation}
	et donc $f(x)=M$, ce qui prouve que $f$ atteint sa borne $M$ au point $x\in D$.
\end{proof}

%+++++++++++++++++++++++++++++++++++++++++++++++++++++++++++++++++++++++++++++++++++++++++++++++++++++++++++++++++++++++++++
\section{Uniforme continuité}		\label{SecUnifContinue}
%+++++++++++++++++++++++++++++++++++++++++++++++++++++++++++++++++++++++++++++++++++++++++++++++++++++++++++++++++++++++++++

Pour une fonction $f\colon D\subset\eR^m\to \eR$, la continuité au point $a$ signifie que pour tout $\varepsilon>0$,
\begin{equation}
	\exists\delta>0\tq 0<\| x-a \|<\delta\Rightarrow | f(x)-f(a) |<\varepsilon.
\end{equation}
Le $\delta$ qu'il faut choisir dépend évidement de $\varepsilon$, mais il dépend en général aussi du point $a$ où l'on veut tester la continuité. C'est à dire que, étant donné un $\varepsilon>0$, nous pouvons trouver un $\delta$ qui fonctionne pour certains points, mais qui ne fonctionne pas pour d'autres points.

Il peut cependant également arriver qu'un même $\delta$ fonctionne pour tous les points du domaine. Dans ce cas, nous disons que la fonction est uniformément continue sur le domaine.

\begin{definition}
	Une fonction $f\colon D\subset\eR^m\to \eR$ est dite \defe{uniformément continue}{continue!uniformément} sur $D$ si
	\begin{equation}	\label{EqConditionUnifCont}
		\forall\varepsilon>0,\,\exists\delta>0\tq\,\forall x,y\in D,\,\| x-y \|\leq\delta \Rightarrow| f(x)-f(a) |<\varepsilon.
	\end{equation}
\end{definition}

Il est intéressant de voir ce que signifie le fait de \emph{ne pas} être uniformément continue sur un domaine $D$. Il s'agit essentiellement de retourner tous les quantificateurs de la condition \eqref{EqConditionUnifCont} :
\begin{equation}	\label{EqConditionPasUnifCont}
	\exists\varepsilon>0\tq\forall\delta>0,\,\exists x,y\in D\tq \| x-y \|<\delta\text{ et }\big| f(x)-f(y) \big|>\varepsilon.
\end{equation}
Dans cette condition, les points $x$ et $y$ peuvent être fonction du $\delta$. L'important est que pour tout $\delta$, on puisse trouver deux points $\delta$-proches dont les images par $f$ ne soient pas $\varepsilon$-proches.

\begin{example}
	Prenons la fonction $f(x)=\frac{1}{ x }$, et demandons nous pour quel $\delta$ nous sommes sûr d'avoir
	\begin{equation}
		| f(a+\delta)-f(a) |=\left| \frac{1}{ a+\delta }-\frac{1}{ a } \right| <\varepsilon.
	\end{equation}
	Pour simplifier, nous supposons que $a>0$. Nous calculons
	\begin{equation}
		\begin{aligned}[]
			\frac{ 1 }{ a }-\frac{1}{ a+\delta }&<	\varepsilon\\
			\frac{ \delta }{ a(a+\delta) }&<\varepsilon\\
			\delta&<\varepsilon a^2+\varepsilon a\delta\\
			\delta(1-\varepsilon a)&<\varepsilon a^2\\
			\delta&<\frac{ \varepsilon a^2 }{ 1-\varepsilon a }.
		\end{aligned}
	\end{equation}
	Notons que, à $\varepsilon$ fixé, plus $a$ est petit, plus il faut choisir $\delta$ petit. La fonction $x\mapsto\frac{1}{ x }$ n'est donc pas uniformément continue. Cela correspond au fait que, proche de zéro, la fonction monte très vite. Une fonction uniformément continue sera une fonction qui ne montera jamais très vite.
\end{example}

\begin{proposition}
	Quelque propriétés des fonctions uniformément continues.
	\begin{enumerate}
		\item
			Toute application uniformément continue est continue;
		\item
			la composée de deux fonctions uniformément continues est uniformément continue;
		\item
			tout application lipschitzienne est uniformément continues.
	\end{enumerate}
\end{proposition}

Une fonction peut être uniformément continue sur un domaine et pas sur un autre. Le théorème suivant donne une importante indication à ce sujet.
\begin{theorem}[Heine]\index{théorème!Heine}\index{Heine (théorème)}		\label{ThoHeineContinueCompact}
	Une fonction continue sur un compact (fermé et borné) est uniformément continue.
\end{theorem}

\begin{proof}
	Nous allons prouver ce théorème par l'absurde. Nous commençons par écrire la condition \eqref{EqConditionPasUnifCont} qui exprime que $f$ n'est pas uniformément continue :
	\begin{equation}
		\exists\varepsilon>0\tq\forall\delta>0,\,\exists x,y\in D\tqs \| x-y \|<\delta\text{ et }\big| f(x)-f(y) \big|>\varepsilon.
	\end{equation}
	En particulier (en prenant $\delta=\frac{1}{ n }$ pour tout $n$), pour chaque $n$ nous pouvons trouver $x_n$ et $y_n$ dans $D$ qui vérifient simultanément les deux conditions suivantes :
	\begin{subequations}
		\begin{numcases}{}
			\| x_n-y_n \|<\frac{1}{ n }\\
			\big| f(x_n)-f(y_n) \big|>\varepsilon.	\label{EqCond3107fxfyepsppt}
		\end{numcases}
	\end{subequations}
	Nous insistons que c'est le même $\varepsilon$ pour chaque $n$. L'ensemble $D$ étant borné, les suites $(x_n)$ et $(y_n)$ possèdent une sous-suite convergente. Quitte à prendre une sous-suite, nous supposons que les suites $(x_n)$ et $(y_n)$ sont convergentes. Étant donné que pour chaque $n$ elles vérifient $\| x_n-y_n \|<\frac{1}{ n }$, les limites sont égales :
	\begin{equation}
		\lim x_n=\lim y_n=x.
	\end{equation}
	L'ensemble $D$ étant fermé, la limite $x$ est dans $D$. Par continuité de $f$, nous avons finalement
	\begin{equation}
		\lim f(x_n)=\lim f(y_n)=f(x),
	\end{equation}
	mais alors 
	\begin{equation}
		\lim_{n\to\infty}\big| f(x_n)-f(y_n) \big|=0,
	\end{equation}
	ce qui est en contradiction avec le choix \eqref{EqCond3107fxfyepsppt}.

	Tout ceci prouve que $f(K)$ est bornée supérieurement et que $f$ atteint son supremum (qui est donc un maximum). Le fait que $f(K)$ soit borné inférieurement se prouve en considérant la fonction $-f$ au lieu de $f$.

\end{proof}


