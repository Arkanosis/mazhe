% This is part of Mes notes de mathématique
% Copyright (c) 2011-2013
%   Laurent Claessens
% See the file fdl-1.3.txt for copying conditions.

%---------------------------------------------------------------------------------------------------------------------------
\subsection{Décomposition de Bruhat}
%---------------------------------------------------------------------------------------------------------------------------

Nous nommons \( E_{ij}\) la matrice remplie de zéros sauf à la case \( ij\) qui vaut \( 1\). Autrement dit
\begin{equation}
    (E_{ij})_{kl}=\delta_{ik}\delta_{jl}.
\end{equation}
\begin{definition}
    Une \defe{matrice de transvection}{transvection (matrice)}\index{matrice!de transvection} est une matrice de la forme
    \begin{equation}
        T_{ij}(\lambda)=\id+\lambda E_{ij}
    \end{equation}
    avec \( i\neq j\).

    Une \defe{matrice de dilatation}{matrice!de dilatation}\index{dilatation (matrice)} est une matrice de la forme
    \begin{equation}
        D_i(\lambda)=\id+(\lambda-1)E_{ii}.
    \end{equation}
    Ici le \( (\lambda-1)\) sert à avoir \( \lambda\) et non \( 1+\lambda\). C'est donc une matrice qui dilate d'un facteur \( \lambda\) la direction \( i\) tout en laissant le reste inchangé.

    Si \( \sigma\) est une permutation (un élément du groupe symétrique \( S_n\)) alors la \defe{matrice de permutations}{matrice!de permutation}\index{permutation!matrice} associée est la matrice d'entrées
    \begin{equation}
        (P_{\sigma})_{ij}=\delta_{i\sigma(j)}.
    \end{equation}
\end{definition}

\begin{lemma}   \label{LemyrAXQs}
    La matrice \( T_{ij}(\lambda)A=(\mtu+\lambda E_{ij})A\) est la matrice \( A\) à qui on a effectué la substitution
    \begin{equation}
        L_i\to L_i+\lambda L_j.
    \end{equation}
    La matrice \( AT_{ij}(\lambda)\) est la substitution 
    \begin{equation}
        C_j\to C_j+\lambda C_i.
    \end{equation}

    La matrice \( AP_{\sigma}\) est la matrice \( A\) dans laquelle nous avons permuté les colonnes avec \( \sigma\).

    La matrice \( P_{\sigma}A\) est la matrice \( A\) dans laquelle nous avons permuté les lignes avec \( \sigma^{-1}\).
\end{lemma}

\begin{proof}
    Calculons la composante \( kl\) de la matrice \( E_{ij}A\) :
    \begin{subequations}
        \begin{align}
            (E_{ij}A)_{kl}&=\sum_m(E_{ij})_kmA_{ml}\\
            &=\sum_m\delta_{ik}\delta_{jm}A_{ml}\\
            &=\delta_{ik}A_{jl}.
        \end{align}
    \end{subequations}
    C'est donc la matrice pleine de zéros, sauf la ligne \( i\) qui est donnée par la ligne \( j\) de \( A\). Donc effectivement la matrice
    \begin{equation}
        A+\lambda E_{ij}A
    \end{equation}
    est la matrice \( A\) à laquelle on a substitué la ligne \( i\) par la ligne \( i\) plus \( \lambda\) fois la ligne \( j\).

    En ce qui concerne l'autre assertion sur les transvections, le calcul est le même et nous obtenons
    \begin{equation}
        (AE_{ij})=A_{ki}\delta_{jl}.
    \end{equation}

    Pour les matrices de permutations, nous avons 
    \begin{equation}
        (AP_{\sigma})_{kl}=A_{k\sigma(l)}
    \end{equation}
    et
    \begin{equation}
        (P_{\sigma}A)_{kl}=\sum_m\delta_{k\sigma(m)}A_{ml}=\sum_m\delta_{\sigma^{-1}(k)m}A_{ml}=A_{\sigma^{-1}(k)l}.
    \end{equation}
\end{proof}

\begin{theorem}[Décomposition de Bruhat]\index{Bruhat (décomposition)}\index{décomposition!Bruhat}    \label{ThoizlYJO}
    Soit \( \eK\) un corps; un élément \( M\in\GL(n,\eR)\) s'écrit sous la forme
    \begin{equation}
        M=T_1P_{\sigma}T_2
    \end{equation}
    où \( T_1\) et \( T_2\) sont des matrices triangulaires supérieures inversibles et où \( P_{\sigma}\) est une matrice de permutation \( \sigma\in S_n\). De plus il y a unicité de \( \sigma\).
\end{theorem}
\index{groupe!permutation}
\index{groupe!linéaire}
\index{matrice}

\begin{proof}
    Afin de rendre les choses plus visuelles, nous nous permettons de donner des exemples au fur et à mesure de la preuve. Nous prenons l'exemple de la matrice
    \begin{equation}
        \begin{pmatrix}
            1    &   3    &   4    \\
            2    &   5    &   6    \\
            0    &   7    &   8
        \end{pmatrix}.
    \end{equation}
    \begin{subproof}
    \item[Existence]
        Soit \( M\in \GL(n,\eR)\); vu qu'elle est inversible, on a un indice \( i_1\) maximum tel que \( M_{i_1,1}\neq 0\). Nous changeons toutes les lignes jusque là, c'est à dire que nous faisons, pour \( 1\leq i< i_1\),
        \begin{equation}        \label{EqGHUbwR}
            L_i\to L_i-\frac{ M_{i1} }{ M_{i_11} }L_{i_1}.
        \end{equation}

        Nous avons donc obtenu une matrice dont la première colonne est nulle sauf la case numéro \( i_1\). L'opération \eqref{EqGHUbwR} revient à considérer la multiplication par la matrice de transvection
        \begin{equation}
            T_1^{(i)}=T_{ii_1}\left( -\frac{ M_{i1} }{ M_{i_11} } \right)
        \end{equation}
        pour tout \( i<i_1\). Pour rappel nous ne changeons que les lignes \emph{au-)dessus} de la \( i_1\). Du coup les matrices \( T^{(i)}_1\) sont triangulaires supérieures. Nous avons donc la nouvelle matrice \( M_1=\left( \prod_{i<i_1}T_1^{(i)} \right)M\) pour laquelle toute la première colonne est nulle sauf un élément.

        Dans le cas de l'exemple, le «pivot» sera la ligne \( (2,5,6)\) et la matrice se transforme à l'aide de la matrice \( T_1=T_{12}(-1/2)\) :
        \begin{equation}    \label{EqyjXIYf}
            \begin{pmatrix}
                1    &   -1/2    &   0    \\
                0    &   1    &   0    \\
                0    &   0    &   1
            \end{pmatrix}
            \begin{pmatrix}
                1    &   3    &   4    \\
                2    &   5    &   6    \\
                0    &   7    &   8
            \end{pmatrix}=
            \begin{pmatrix}
                0    &   1/2    &   1    \\
                2    &   5    &   6    \\
                0    &   7    &   8
            \end{pmatrix}.
        \end{equation}

    
    Maintenant nous faisons de même avec les colonnes (en renommant \( M\) la matrice obtenue à l'étape précédente) :
    \begin{equation}
        C_j\to C_j-\frac{ M_{i_1j} }{ M_{i_11} }C_1,
    \end{equation}
    qui revient à multiplier à droite par les matrices \( T_{1j}(\frac{ M_{i_1i} }{ M_{i_11} })\) avec \( j>1\). Encore une fois ce sont des matrices triangulaires supérieures.

    Dans l'exemple, pour traiter la seconde colonne, nous multiplions \eqref{EqyjXIYf} à droite par la matrice \( T_{12}(-5/2)\) :
    \begin{equation}
            \begin{pmatrix}
                0    &   1/2    &   1    \\
                2    &   5    &   6    \\
                0    &   7    &   8
            \end{pmatrix}
            \begin{pmatrix}
                1    &   -5/2    &   0    \\
                0    &   1    &   0    \\
                0    &   0    &   1
            \end{pmatrix}=
            \begin{pmatrix}
                0    &   1/2    &   1    \\
                2    &   0    &   6    \\
                0    &   7    &   8
            \end{pmatrix}.
    \end{equation}
    Appliquer encore la matrice \( T_{13}(-6/2)\) apporte la matrice
    \begin{equation}
        \begin{pmatrix}
            0    &   1/2    &   1    \\
            2    &   0    &   0    \\
            0    &   7    &   8
        \end{pmatrix}.
    \end{equation}
    Enfin nous multiplions la matrice obtenue par \( \frac{1}{ M_{i_11} }\mtu\) pour normaliser à \( 1\) l'élément «pivot» que nous avions choisit. Dans notre exemple nous multiplions par \( 1/2\) pour trouver
    \begin{equation}        \label{Eqduglwu}
        \begin{pmatrix}
            0    &   1/4    &   1/2    \\
            1    &   0    &   0    \\
            0    &   7/2    &   4
        \end{pmatrix}.
    \end{equation}

    La matrice obtenue jusqu'ici possède une ligne et une colonne de zéros avec un \( 1\) à leur intersection, et elle est de la forme
    \begin{equation}
        M'=T_1MT_2
    \end{equation}
    où \( T_1\) et \( T_2\) sont triangulaires supérieures et inversibles, produits de matrices de transvection (et d'une matrice scalaire pour la normalisation).

    Il reste à recommencer l'opération avec la seconde colonne (qui n'est pas toute nulle parce que le déterminant est encore non nul) puis la suivante etc. Dans notre exemple de l'équation \eqref{Eqduglwu}, nous éliminerions le \( 1/4\) et le \( 4\) en utilisant le \( 7/2\).

    Encore une fois tout cela se fait à l'aide de matrice supérieures parce qu'à chaque étape, les colonnes précédent le pivot sont déjà nulles (saut un \( 1\)) et ne doivent donc pas être touchées.

    À la fin de ce processus, ce qui reste est une matrice \( TMT'\) qui ne contient plus que un seul \( 1\) sur chaque ligne et chaque colonne, c'est à dire une matrice de permutation : \( P_{\sigma}=TMT'\) et donc
    \begin{equation}
        M=T^{-1}_{\sigma}(T')^{-1}.
    \end{equation}

        \item[Unicité]

            Soient \( \sigma,\sigma\in S_n'\) tels que \( T_1P_{\sigma}T_2=S_1P_{\tau}S_2\) avec \( T_i\) et \( S_i\) triangulaires supérieures et inversibles. En posant \( T=T_2S_2^{-1}\) et \( S=T_1^{-1}S_1\), nous avons
            \begin{equation}
                P_{\sigma}T=SP_{\tau}
            \end{equation}
            où \( S\) et \( T\) sont des matrices triangulaires supérieures et inversibles. Par les calculs de la preuve du lemme \ref{LemyrAXQs},
            \begin{subequations}
                \begin{numcases}{}
                    (P_{\sigma}T)_{kl}=T_{\sigma^{-1}(k)l}\\
                    (SP_{\tau})_{kl}=S_{k\tau(l)},
                \end{numcases}
            \end{subequations}
            et donc
            \begin{equation}    \label{EqKlmgOT}
                T_{\sigma^{-1}(k)l}=S_{k\tau(l)}.
            \end{equation}
            En écrivant cette équation avec \( k=\sigma(i)\) (nous rappelons que \( \sigma\) est bijective),
            \begin{equation}
                T_{il}=S_{\sigma(i)\tau(l)}.
            \end{equation}
            Nous savons que les termes diagonaux de \( T\) sont non nuls parce que \( T\) est triangulaire supérieure et inversible (donc pas de colonnes entières nulles). Nous avons donc, en prenant \( i=l=k\),
            \begin{equation}
                0\neq T_{kk}=S_{\sigma(k)\tau(k)}.
            \end{equation}
            La matrice étant triangulaire supérieure, cela implique 
            \begin{equation}    \label{EqEmiBTX}
                \sigma(k)\leq\tau(k).
            \end{equation}
            De la même manière en écrivant \eqref{EqKlmgOT} avec \( l=\tau^{-1}(i)\),
            \begin{equation}
                S_{ki}=T_{\sigma^{-1}(k)\tau^{-1}(i)}
            \end{equation}
            et donc
            \begin{equation}
                \sigma^{-1}(k)\leq \tau^{-1}(k).
            \end{equation}
            En écrivant cela avec \( k=\sigma(j)\), nous avons \( j\leq \tau^{-1}\sigma(j)\) et en appliquant enfin \( \tau\),
            \begin{equation}
                \tau(j)\leq \sigma(j).
            \end{equation}
            En comparant avec \eqref{EqEmiBTX}, nous avons \( \sigma=\tau\).
    \end{subproof}
\end{proof}

%++++++++++++++++++++++++++++++++++++++++++++++++++++++++++++++++++++++++++++++++++++++++++++++++++++++++++++++++++++++++++++++++++++++++
\section{Espaces de polynômes}		\label{SecEspacePolynomes}
%++++++++++++++++++++++++++++++++++++++++++++++++++++++++++++++++++++++++++++++++++++++++++++++++++++++++++++++++++++++++++++++++++++++++
 
Dans cette section nous abandonnons pour quelques minutes l'espace $\eR^m$ et considérons plus attentivement l'espace des fonctions polynômiales $\mathcal{P}_{\eR}$ et de ses sous-espaces $\mathcal{P}_{\eR}^k$, pour $k$ dans $\eN_0$. 

Pour chaque $k>0$ donné nous définissons
\begin{equation}
\mathcal{P}_\eR^k=\{p:\eR\to \eR\,|\, p : x\mapsto a_0+a_1 x +a_2 x^2 + \ldots+a_k x^k, \, a_i\in\eR,\,\forall i=0,\ldots,k\}.
\end{equation}   
Il est facile de se convaincre que la somme de deux polynômes de degré inférieur ou égal à $k$ est encore un polynôme de degré inférieur ou égal à $k$. En outre il est clair que la multiplication par un scalaire ne peut pas augmenter le degré d'un polynôme. L'ensemble $\mathcal{P}_\eR^k$ est donc un espace vectoriel muni des opérations héritées de $\mathcal{P}_{\eR}$. 

La base canonique de l'espace $\mathcal{P}_\eR^k$ est donné par les monômes $\mathcal{B}=\{x\mapsto x^j \,|\, j=0, \ldots, k\}$. Le fait que cela soit une base est vraiment facile à démontrer et est un exercice très utile si vous ne l'avez pas encore vu dans un cours précédent. 

Nous allons maintenant étudier trois application linéaires de $\mathcal{P}_\eR^k$ vers des autres espaces vectoriels
\begin{description}
  \item[L'isomorphisme canonique  $\phi:\mathcal{P}_\eR^k \to\eR^{k+1}$] Nous définissons $\phi$ par les relations suivantes
\[
\phi(x^j)=e_{j+1}, \qquad \forall j\in\{0,\dots, k\}. 
\]
Cela veut dire que pour tout $p$ dans $\mathcal{P}_\eR^k$, avec $p(x)=a_0+a_1 x +a_2 x^2 + \ldots+a_k x^K$, l'image de $p$ par $\phi$ est 
\[
\phi(p)=\phi\left(\sum_{j=0}^k a_j x^j\right)=\sum_{j=0}^k a_j e_{j+1}.
\]
\begin{example} Soit $k=5$ on a 
  \begin{equation}
    \phi(-8-7x-4x^2+4x^3+2x^5)=
  \begin{pmatrix}
    -8\\
    -7\\
    -4\\
    4\\
    0\\
    2
  \end{pmatrix}.
  \end{equation}
\end{example}

Cette application est clairement bijective et respecte les opérations d'espace vectoriel, donc elle est un isomorphisme d'espaces vectoriels. L'existence d'un isomorphisme entre $\mathcal{P}_\eR^k$  et $\eR^{k+1}$ est un cas particulier du théorème qui dit que  pour chaque $m$ dans $\eN_0$ fixée, tous les espaces vectoriels sur $\eR$ de dimension $m$ sont isomorphes à $\eR^m$. Vous connaissez peut être déjà ce théorème depuis votre cours d'algèbre linéaire.  
    \item[La dérivation $d: \mathcal{P}_\eR^k \to \mathcal{P}_\eR^{k-1}$] L'application de dérivation $d$ fait exactement ce qu'on s'attend d'elle 
\[
d(x^0)=d(1)=0, \qquad d(x^j)=j x^{j-1}, \quad \forall j\in\{1,\dots, k\}. 
\]
Cette application n'est pas injective, parce que l'image de $p$ ne dépend pas de la valeur de $a_0$, donc si deux polynômes sont les mêmes à une constante près ils auront la même image par $d$.

\begin{example} Soit $k=3$ on a 
  \begin{equation}
    d(-8-12x+4x^3)= -12 (1) + 4 (3x^2) = -12+12 x^2.
    \end{equation}

    Noter que $d(-30-12x+4x^3)=d(-8-12x+4x^3)$. Cela confirme, comme mentionné plus haut que la dérivée n'est pas injective.
\end{example}
      \item[L'intégration $I: \mathcal{P}_\eR^k \to \mathcal{P}_\eR^{k+1}$] Nous pouvons définir une application que est <<à une constante prés>> l'application inverse de la dérivation
        \begin{equation}
          I(p)= \int_0^x p(t) \,dt.
        \end{equation}
Il faut comprendre que dans l'intégral la variable $t$ est simplement la variable d'intégration. La <<vraie>> variable de la fonction image de $p$ sera $x$ !
 
Comme d'habitude nous écrivons explicitement l'action de $I$ sur les éléments de la base canonique
\begin{equation}
    I(x^j)=\int_0^x t^k \,dt= \frac{x^{j+1}}{j+1}.
\end{equation}

\begin{example} 
   Soit $k=4$ on a 
  \begin{equation}
    I(6+2x+x^2+x^4)= 6x+x^2+\frac{x^3}{3}+\frac{x^5}{5}.
    \end{equation}
\end{example}

Remarquez que, étant donné que dans la définition de $I$ nous avons décidé d'intégrer entre zéro et $x$, tous les polynômes dans $\mathcal{P}_\eR^{k+1}$ qui sont l'image par $I$ d'un polynôme de $\mathcal{P}_\eR^{k}$ ont $a_0=0$. Cela veut dire que nous pouvons générer toute l'image de $I$ en utilisant un sous-ensemble de la base canonique de $\mathcal{P}_\eR^{k+1}$,  en particulier $\mathcal{B}_1=\{x\mapsto x^j \,|\, j=1, \ldots, k\}\subset \mathcal{B}$ nous suffira. Cela n'est guère surprenant, parce que l'image par une application linéaire d'un espace vectoriel de dimension finie ne peut pas être un espace de dimension supérieure. 
\end{description}

Les applications de dérivation et intégration correspondent évidemment à des application linéaires de $\mathcal{P}_\eR$ dans lui-même. 

L'espace de tous les polynômes étant de dimension infinie, il peut servir de contre exemple assez simple. Dans la sous-section \ref{SubSecPOlynomesCE}, nous verrons que toutes les normes ne sont pas équivalentes sur l'espace des polynômes.



%---------------------------------------------------------------------------------------------------------------------------
\subsection{Polynômes symétriques, alternés ou semi-symétriques}
%---------------------------------------------------------------------------------------------------------------------------
\cite{fJhCTE}.

Soit \( \eK\) un corps de caractéristique différente\footnote{Le truc de la caractéristique deux est que \( a=-a\) n'implique pas \( a=0\).} de \(2\). Le groupe \( S_n\) agit sur l'anneau \( \eK[T_1,\ldots, T_n]\) par
\begin{equation}
    (\sigma\cdot f)(T_1,\ldots, T_n)=f\big( T_{\sigma(1)},\ldots, T_{\sigma(n)} \big).
\end{equation}
On peut vérifier que c'est un action.

\begin{definition}
    Un polynôme \( Q\) en \( n\) indéterminées est 
    \begin{enumerate}
        \item
            \defe{symétrique}{polynôme!symétrique}\index{symétrique!polynôme} si \( Q=\sigma\cdot Q\) pour tout \( \sigma\in S_n\);
        \item
            \defe{alterné}{polynôme!alterné}\index{alterné!polynôme} si \( \sigma\cdot Q=\epsilon(\sigma)Q\) pour tout \( \sigma\in S_n\);
        \item
            \defe{semi-symétrique}{semi-symétrique!polynôme}\index{polynôme!semi-symétrique} si \( \sigma\cdot Q=Q\) pour tout \( \sigma\in A_n\)
    \end{enumerate}
\end{definition}
Le polynôme \( T_1+T_2\) est symétrique; le polynôme \( T_1+T_2^2\) ne l'est pas. 

\begin{example}
    Le déterminant de Vandermonde (proposition \ref{PropnuUvtj}) est alterné, semi-symétrique et non symétrique. Le fait qu'il soit alterné est le fait qu'il soit un déterminant. Étant donné qu'il est alterné, il est semi-symétrique parce que sur \( A_n\), nous avons \( \epsilon=1\). Étant donné qu'il est alterné, il change de signe sous l'action des éléments impairs de \( S_n\) et n'est donc pas symétrique.
\end{example}

\begin{proposition}\index{action de groupe} \label{PropUDqXax}
    Un polynôme semi-symétrique \( f\in \eK[T_1,\ldots, T_n]\) se décompose de façon unique en
    \begin{equation}
        f=P+VQ
    \end{equation}
    où \( P\) et \( Q\) sont deux polynômes symétriques.
\end{proposition}
\index{groupe!permutation}
\index{polynôme!symétrique}

\begin{proof}

    Nous commençons par prouver l'unicité en montrant que si \( f=PVQ\) avec \( P\) et \( Q\) symétrique, alors \( P\) et \( Q\) sont donnés par des formules explicites en termes de \( f\).


    Si \( \sigma_1\) et \( \sigma_2\) sont deux permutations impaires de \( \{ 1,\ldots, n \}\), alors \( \sigma_1\cdot f=\sigma_2\cdot f\) parce que l'élément \( \sigma_2^{-1}\sigma_1\) est pair (proposition \ref{ProphIuJrC}), de telle sorte que \( \sigma_2^{-1}\sigma_1\cdot f=f\). Nous posons donc \( g=\tau\cdot f\) où \( \tau\) est une permutation impaire quelconque -- par exemple une transposition.

    Vu que \( V\) est alternée et que \( \tau\) est une transposition nous avons
    \begin{equation}
        g=\tau\cdot f=P-VQ.
    \end{equation}
    Donc \( f+g=2P\) et \( f-g=2VQ\). Cela donne \( P\) et \( Q\) en terme de \( f\) et \( g\), et donc l'unicité.

    Attention : cela ne donne pas un moyen de prouver l'existence parce que rien ne prouve pour l'instant que \( f-g\) peut effectivement être écrit sous la forme \( VQ\), c'est à dire que \( f-g\) soit divisible par \( V\). C'est cela que nous allons nous atteler à démontrer maintenant.

    Nous commençons par prouver que \( f+g\) est symétrique et \( f-g\) alterné. Si \( \sigma\) est une transposition,
    \begin{equation}
        \sigma\cdot(f+g)=\sigma\cdot f+\sigma\tau\cdot f=g+f
    \end{equation}
    parce que \( \sigma\tau\) est pair. De la même façon,
    \begin{equation}
        \sigma\cdot(f-g)=g-f=\epsilon(\sigma)(f-g).
    \end{equation}
    Dans les deux cas nous concluons en utilisant le fait que toute permutation est un produit de transpositions (proposition \ref{PropPWIJbu}) et que \( \epsilon\) est un homomorphisme.

    Soient maintenant deux entiers \( h<k\) dans \( \{ 1,\ldots, n \}\) et l'anneau
    \begin{equation}
        \big( \eK[T_1,\ldots, \hat T_k,\ldots, T_n] \big)[T_k].
    \end{equation}
    Cet anneau contient le polynôme \( T_k-T_h\) où \( T_k\) est la variable et \( T_h\) est un coefficient. Nous faisons la division euclidienne de \( f-g\) par  \( T_k-T_h\) parce que nous avons dans l'idée de faire arriver le déterminant de Vandermonde et donc le produit de toutes les différences \( T_k-T_h\) :
    \begin{equation}    \label{EqSHdgrG}
        f-g=(T_k-T_h)q+r
    \end{equation}
    où \( \deg_{T_k}r<1\), c'est à dire que \( r\) ne dépends pas de \( T_k\). Nous revoyons maintenant l'égalité \eqref{EqSHdgrG} dans \( \eK[T_1,\ldots, T_n]\) et nous y appliquons la transposition \( \tau_{kh}\). Nous savons que \( \tau_{kh}(f-g)=-(f-g)\) et \( \tau_{kh}(T_k-T_h)=-(T_k-T_h)\), et donc
    \begin{equation}    \label{EqVOhjKB}
        -(f-g)=-(T_k-T_h)\tau_{kh}\cdot   q+\tau_{kh}\cdot r
    \end{equation}
    où \(\tau_{kh}\cdot r\) ne dépend pas de \( T_h\). Nous appliquons à \eqref{EqVOhjKB} l'application
    \begin{equation}
        \begin{aligned}
            t\alpha\colon \eK[T_1,\ldots, T_n]&\to \eK[T_1,\ldots, \hat T_k,\ldots, T_n] \\
            \alpha(PT_1,\ldots, \hat T_k,\ldots, T_n)&=P(T_1,\ldots, T_h,\ldots, T_n). 
        \end{aligned}
    \end{equation}
    Cette application vérifie \( \alpha\big( \tau_{kh}\cdot r \big)=\alpha(r)\) et nous avons
    \begin{equation}
        -\alpha(f-g)=\alpha(r).
    \end{equation}
    Puis en appliquant \( \alpha\) à la relation \( f-g=(T_k-T_h)q+r\), nous trouvons
    \begin{equation}
        \alpha(f-g)=\alpha(r),
    \end{equation}
    et par conséquent \( \alpha(r)=0\). Ici nous utilisons l'hypothèse de caractéristique différente de deux. Dire que \( \alpha(r)=0\), c'est dire que \( r\) est divisible par \( T_k-T_h\), mais \( r\) étant de degré zéro en \( T_k\), nous avons \( r=0\). Par conséquent \( T_k-T_h\) divise \( f-g\) pour tout \( h<k\), et nous pouvons définir un polynôme \( Q\) par
    \begin{equation}    \label{EqrnbgdA}
        f-g=2Q\prod_{h<k}\prod_{k\leq n}(T_k-T_h)=2Q(T_1,\ldots, T_n)V(T_1,\ldots, T_n),
    \end{equation}
    où nous avons utilisé la formule du déterminant de Vandermonde de la proposition \ref{PropnuUvtj}.

    Étant donné que \( f+g\) est un polynôme symétrique, nous allons aussi poser \( f+g=2P\) avec \( P\) symétrique.

    Montrons à présent que \( Q\) est un polynôme symétrique. Soit \( \sigma\in S_n\); vu que nous savons déjà que \( f-g\) est alternée, nous avons
    \begin{equation}    \label{EqpSPEyq}
        \sigma\cdot (f-g)=\epsilon(\sigma)(f-g)=\epsilon(\sigma)2QV,
    \end{equation}
    Mais en appliquant \( \sigma\) à l'équation \eqref{EqrnbgdA},
    \begin{subequations}
        \begin{align}
            \sigma\cdot (f-g)&=2(\sigma\cdot V)(T_1,\ldots, ,T_n)(\sigma\cdot Q)(T_1,\ldots,T_n)\\
            &=2\epsilon(\sigma)V(T_1,\ldots, T_n)(\sigma\cdot Q)(T_1,\ldots, T_n).
        \end{align}
    \end{subequations}
    En égalisant avec \eqref{EqpSPEyq} et en se souvenant que l'anneau \( \eK[T_1,\ldots, T_n]\) était intègre (théorème \ref{ThoBUEDrJ}), nous simplifions par \( 2\epsilon(\sigma)V\) pour obtenir
    \begin{equation}
        Q=\sigma\cdot Q,
    \end{equation}
    c'est à dire que \( Q\) est symétrique.

    Au final nous avons \( f+q=2P\) et \( f-g=2VQ\) avec \( P\) et \( Q\) symétriques. En faisant la somme,
    \begin{equation}
        f=P+VQ.
    \end{equation}
\end{proof}

%---------------------------------------------------------------------------------------------------------------------------
\subsection{Polynôme symétrique élémentaire}
%---------------------------------------------------------------------------------------------------------------------------

Le \( k\)ième \defe{polynôme symétrique élémentaire}{élémentaire!polynôme symétrique}\index{polynôme!symétrique!élémentaire} à \( n\) inconnues est le polynôme est
\begin{equation}
    \sigma_k(T_1,\ldots, T_n)=\sum_{s\in F_k}\prod_{i=1}^kT_{s(i)}
\end{equation}
où \( F_k\) est l'ensemble des fonctions strictement croissantes \( \{ 1,2,\ldots, k \}\to\{ 1,2,\ldots, n \}\). Une autre façon de décrire ces polynômes élémentaires est
\begin{equation}
    \sigma_k=\sum_{1\leq i_1<\ldots<i_k\leq n}X_{i_1}\ldots X_{i_k}.
\end{equation}
Par exemple
\begin{subequations}
    \begin{align}
        \sigma_1(T_1,\ldots, T_n)&=T_1+T_2+\ldots +T_n\\
        \sigma_2(T_1,\ldots, T_n)&=T_1T_2+\ldots +T_1T_n+T_2T_3+\ldots +T_2T_n+\ldots +T_{n-1}T_n\\
        \sigma_n(T_1,\ldots, T_n)&=T_1\ldots T_n.
    \end{align}
\end{subequations}
En particulier, \( \sigma_2(x,y,z)=xy+yz+xz\).

\begin{theorem}[\cite{PoloPolSym}]  \label{TholReBiw}
    Si \( Q\) est un polynôme symétrique en \( T_1,\ldots, T_n\), alors il existe un et un seul polynôme \( P\) en \( n\) indéterminées tel que
    \begin{equation}
        Q(T_1,\ldots, T_n)=P\big( \sigma_1(T_1,\ldots, T_n),\ldots, \sigma_n(T_1,\ldots, T_n) \big).
    \end{equation}
\end{theorem}
%TODO : la preuve de ce théorème

\begin{example}
    Nous voulons décomposer \( P(x,y,z)=x^3+y^3+z^3\) en polynômes symétriques élémentaires, c'est à dire en
    \begin{subequations}
        \begin{numcases}{}
            \sigma_1=x+y+z\\
            \sigma_2=xy+xz+yz\\
            \sigma_3=xyz.
        \end{numcases}
    \end{subequations}
    Étant donné que \( P\) est de degré \( 3\), les seules combinaisons des \( \sigma_i\) qui peuvent intervenir sont \( \sigma_1^3\), \( \sigma_1\sigma_2\) et \( \sigma_3\). Étant donné que dans \( P\) le coefficient de \( x^3\) est un, il est obligatoire d'avoir un coefficient \( 1\) devant \( \sigma_1^3\). Nous le calculons :
    \begin{verbatim}
----------------------------------------------------------------------
| Sage Version 4.8, Release Date: 2012-01-20                         |
| Type notebook() for the GUI, and license() for information.        |
----------------------------------------------------------------------
sage: var('x,y,z')
(x, y, z)
sage: P=x**3+y**3+z**3  
sage: S1=x+y+z    
sage: S2=x*y+x*z+y*z
sage: S3=x*y*z
sage: (S1**3).expand()
x^3 + 3*x^2*y + 3*x^2*z + 3*x*y^2 + 6*x*y*z + 3*x*z^2 + y^3 + 3*y^2*z + 3*y*z^2 + z^3
sage: (S1**3-P).expand()
3*x^2*y + 3*x^2*z + 3*x*y^2 + 6*x*y*z + 3*x*z^2 + 3*y^2*z + 3*y*z^2
x^3 + 3*x^2*y + 3*x^2*z + 3*x*y^2 + 6*x*y*z + 3*x*z^2 + y^3 + 3*y^2*z + 3*y*z^2 + z^3
    \end{verbatim}
    Dans la différence \( \sigma_1^3-P\) nous voyons que le terme en \( xyz\) est \( 6xyz\); par conséquent nous savons que le coefficient de \( \sigma_3\) sera \( -6\). Il nous reste :
    \begin{verbatim}
sage: (S1**3+6*S3-P).expand()
3*x^2*y + 3*x^2*z + 3*x*y^2 + 12*x*y*z + 3*x*z^2 + 3*y^2*z + 3*y*z^2    
    \end{verbatim}
    que nous identifions facilement avec \( 3\sigma_1\sigma_2\). Nous avons donc
    \begin{equation}
        P=\sigma_1^3-3\sigma_1\sigma_2+3\sigma_3.
    \end{equation}
\end{example}


\begin{lemma}[\cite{fJhCTE}]    \label{LemSoXCQH}
    Soit \( \eK\) une extension de degré \( \delta\) de \( \eQ\) et \( P\in \eK[T_1,\ldots, T_m]\). Alors il existe \( \bar P\in \eQ[T_1,\ldots, T_m]\) tel que
    \begin{enumerate}
        \item
            $\deg\bar P=\delta\deg(P)$
        \item
            pour tout \( (z_1,\ldots, z_m)\in \eC^m\) tel que \( P(z_1,\ldots, z_m)=0\), on a \( \bar P(z_1,\ldots, z_m)=0\).
    \end{enumerate}
\end{lemma}
\index{polynôme!symétrique}
\index{polynôme!racines}
\index{extension!de corps}
\index{corps!extension}

\begin{proof}
    En vertu de la proposition \ref{PropUmxJVw} et de l'exemple \ref{ExvQTyBl}, \( \eK\) est une extension séparable de \( \eQ\), et donc vérifie le théorème de l'élément primitif (\ref{ThoORxgBC}). Il existe \( \theta\in \eK\) tel que \( \eK=\eQ(\theta)\). Soit \( P_{\theta}\in\eQ[X]\) le polynôme minimal de \( \theta\). L'extension \( \eK\) étant de degré \( \delta\), et \( \theta\) étant un générateur, une base de \( \eK\) comme espace vectoriel sur \( \eQ\) est 
    \begin{equation}
        \{ 1,\theta,\ldots, \theta^{\delta-1} \}.
    \end{equation}
    Mais par ailleurs la proposition \ref{PropdsRAsk} nous indique qu'une base est donnée par
    \begin{equation}
        \{ 1,\theta,\ldots, \theta^{n-1} \}
    \end{equation}
    où \( n\) est le degré de \( P_{\theta}\). Donc \( P_{\theta}\) est de degré \( \delta\). Nous nommons \( \theta_1,\ldots, \theta_{\delta}\) les racines de \( P_{\theta}\) dans un corps de décomposition. Ici nous notons \( \theta=\theta_1\) et nous ne prétendons pas que \( \theta_k\in \eK\). Notons que ces \( \theta_i\) sont toutes des racines simples de \( P_{\theta}\), sinon nous aurions un facteur irréductible \( (X-\theta_k)^2\), et \( P_{\theta}\) ne serait pas irréductible sur \( \eQ\).

    Soit \( \sigma_k\) le morphisme canonique
    \begin{equation}
        \begin{aligned}
            \sigma_k\colon \eQ(\theta)&\to \eQ(\theta_k) \\
            \sum_i q_i\theta^i&\mapsto \sum_iq_i\theta_k^i 
        \end{aligned}
    \end{equation}
    Nous avons \( \sigma_1\colon \eK\to \eK\) qui est l'identité.

    Notons \( N\) le degré du polynôme \( P\in \eK[T_1,\ldots, T_m]\) dont il est question dans l'énoncé. Nous le décomposons alors en
    \begin{equation}
        P=\sum_{l=0}^N\sum_{i=1}^mc_{il}T_i^l
    \end{equation}
    avec \( c_{il}\in \eK\). Nous voyons \( c_{i,.}\) comme un élément de \( \eK^m\) et donc nous écrivons\footnote{Il me semble qu'il manque la somme sur \( i\) dans \cite{fJhCTE}.}
    \begin{equation}
        P=\sum_{l=0}^N\sum_{i=1}^m c_l(\theta)_iT_i^l
    \end{equation}
    où \( c_l\in \eQ[X]^m\). Nous pouvons choisir \( \deg(c_l)<\delta\) parce que les puissances plus grandes de \( \theta\) ne génèrent rien de nouveau.

    Nous posons aussi
    \begin{equation}
        P^{\sigma_k}=\sum_{l,i} c_l(\theta_k)_iT_i^l\in \eQ(\theta_k)[T_1,\ldots, T_m],
    \end{equation}
    et \( \bar P=PP^{\sigma_2}\ldots P^{\sigma_k}\). Le coefficient de \( T_i^l\) dans \( \bar P\) est
    \begin{equation}
        \bar c_l(\theta_1,\ldots, \theta_{\delta})_i=\sum_{l_1+\ldots +l_{\delta}=l}c_{l_1}(\theta_1)_i\ldots c_{l_{\delta}}(\theta_{\delta})_i.
    \end{equation}
    Ce dernier est un polynôme en les \( \theta_k\) à coefficients dans \( \eQ\). Qui plus est, c'est un polynôme symétrique. En effet un terme contenant \( \theta_k^a\theta_l^b\) provenant de \( c_{l_i}(\theta_k)c_{l_j}(\theta_l)\) a un terme correspondant \( \theta_k^b\theta_l^a\) provenant de \( c_{l_j}(\theta_k)c_{l_i}(\theta_l)\).

    C'est donc le moment d'utiliser le théorème \ref{TholReBiw} à propos des polynômes symétriques élémentaires qui nous dit que les coefficients de \( \bar P\) sont en réalité des polynômes en ceux de \( P_{\theta}\) qui sont dans \( \eQ\). Donc \( \bar P\in \eQ[T_1,\ldots, T_m]\). Par ailleurs nous avons que
    \begin{equation}
        \deg(\bar P)=\delta \deg(P)
    \end{equation}
    parce que \( \bar P\) est le produit de \( \delta\) «copies»  de \( P\). De plus \( P=P^{\sigma_1}\) divise \( \bar P \) donc on a bien que si \( P(z)=0\) alors \( \bar P(z)=0\). Le polynôme \( \bar P\) est celui que nous cherchions. 
\end{proof}

%--------------------------------------------------------------------------------------------------------------------------- 
\subsection{Relations coefficients racines}
%---------------------------------------------------------------------------------------------------------------------------

\begin{theorem}[Relations coeffitients-racines] \label{ThoOQRgjpl}
    Soit le polynôme \( P=a_nX^n+\ldots +a_1X+a_0\) et \( r_i\) ses \( n\) racines. Alors nous avons pour chaque \( 1\leq k\leq n\) la relation
    \begin{equation}
        \sigma_k(r_1,\ldots, r_n)=(-1)^k\frac{ a_{n-k} }{ a_n }.
    \end{equation}
\end{theorem}
\index{relations!coefficient-racines}
\index{polynôme!symétrique!élémentaire}

%TODO : citer Wikipédia pour l'exemple suivant.
%TODO : ici aussi il faudra faire référence au théorème sur le fait qu'un polynôme ait toutes ses racines dans \eC.

\begin{example} \label{ExHIfHhBr}
    Soit le polynôme
    \begin{equation}
        P(x)=x^3+2x^2+3x+4
    \end{equation}
    et ses racines que nous nommons \( a,b,c\). Nous voudrions calculer \( a^2+b^2+c^2\). D'abord nous décomposons \( Q(a,b,c)=a^2+b^2+c^2\) en polynômes symétriques élémentaires : \( Q(a,b,c)=\sigma_1(a,b,c)^2-2\sigma_2(a,b,c)\).

    Mais les relations coefficients-racines (théorème \ref{ThoOQRgjpl}) nous donnent \( \sigma_1(a,b,c)=-2\) et \( \sigma_2(a,b,c)=3\), donc
    \begin{equation}
        a^2+b^2+c^2=(-2)^2-2\cdot 3=-2.
    \end{equation}

    Nous pouvons en avoir une vérification directe en calculant explicitement les racines (ce qui est possible pour le degré \( 3\)) :
    \lstinputlisting{VAYVmNRpolynomeSym.py}

    Notez qu'il faut un peu chipoter pour isoler les solutions depuis la réponse de la fonction \info{solve}.

\end{example}

En suivant le même cheminement que dans l'exemple, si \( P\) est un polynôme de degré \( n\) et si \( r_i\) sont ses racines, il est facile de calculer \( Q(r_1,\ldots, r_n)\) pour n'importe quel polynôme symétrique \( Q\)


%+++++++++++++++++++++++++++++++++++++++++++++++++++++++++++++++++++++++++++++++++++++++++++++++++++++++++++++++++++++++++++
\section{Polynômes cyclotomiques}
%+++++++++++++++++++++++++++++++++++++++++++++++++++++++++++++++++++++++++++++++++++++++++++++++++++++++++++++++++++++++++++

%---------------------------------------------------------------------------------------------------------------------------
\subsection{Définitions et propriétés}
%---------------------------------------------------------------------------------------------------------------------------

Pour \( n\in\eN^*\) nous considérons l'ensemble
\begin{equation}
    \Delta_n=\{  e^{2ki\pi/n}\tq 0\leq k\leq n-1,\pgcd(k,n)=1 \}.
\end{equation}
Voir \ref{SubSechZeTuL}. Le \defe{polynôme cyclotomique}{polynôme!cyclotomique} d'indice \( n\) est le polynôme
\begin{equation}    \label{EqLjGYKK}
    \phi_n(X)=\prod_{z\in\Delta_n}(X-z)
\end{equation}
où
\begin{equation}
    \Delta_n=\{  e^{2ik\pi/n}\tq 0\leq k\leq n-1\tq \pgcd(k,n)=1 \}.
\end{equation}
Le polynôme \( \phi_n\) est un polynôme unitaire de degré \( \varphi(n)\). Nous avons par exemple
\begin{subequations}
    \begin{align}
        \Delta_1&=\{ 1 \}\\
        \Delta_2&=\{ -1 \}\\
        \Delta_3&=\{  e^{2\pi i/3, e^{4\pi i/3}} \}
    \end{align}
\end{subequations}
et les premiers polynômes cyclotomiques sont donnés par
\begin{subequations}
    \begin{align}
        \phi_1(X)&=X-1\\
        \phi_2(X)&=X+1\\
        \phi_3(X)&=X^2+X+1.
    \end{align}
\end{subequations}
Pour le dernier nous avons utilisé le fait que \(  e^{6\pi i/3}=1\) et \(  e^{4\pi i/3+ e^{2\pi i/3}}=-1\).

\begin{proposition}     \label{PropUImYnL}
    Soient \( 1\leq m\leq n\) deux entiers et
    \begin{equation}
        T(X)=\frac{ X^n-1 }{ X^m-1 }\in \eZ(X).
    \end{equation}
    Soit \( \phi_n\) le \( n\)-ième polynôme cyclotomique. Alors
    \begin{enumerate}
        \item   \label{ItempnHhYk}
            \( X^n-1=\prod_{d\divides n}\phi_d(X)=\prod_{d\divides n}\prod_{z\in \Delta_d}(X-z)\),
        \item
            \( \phi_n\in \eZ[X]\),
        \item   \label{ItemhpDPKE}
            si \( m\divides n\) alors \( T\in \eZ[X]\),
        \item
            si \( m\divides n\) et si \( m<n\) alors \( \phi_n\) divise \( T\) dans \( \eZ[X]\).
    \end{enumerate}
\end{proposition}

\begin{proof}

    \begin{enumerate}
        \item
            La seconde égalité est seulement la définition \eqref{EqLjGYKK}. Nous ne devons que prouver la première. Notons juste pour le plaisir que dans le produit \( \prod_{d\divides n}\prod_{z\in\Delta_d}\), il y a bien \( n\) termes parce que \( \Card(\Delta_d)=\varphi(d)\) et \( \sum_{d\divides n}\varphi(d)=n\).

            Nous connaissons l'union disjointe \( \gU_n=\bigcup_{d\divides n}\Delta_d\) qui implique
            \begin{equation}
                \prod_{z\in \gU_n}(X-z)=\prod_{d\divides n}\prod_{z\in \Delta_d}(X-z)=\prod_{d\divides n}\phi_d(X),
            \end{equation}
            alors que par définition de \( \gU_n\) nous avons \( X^n-1=\prod_{z\in\gU_n}(X-z)\).

        \item

            Nous devons démontrer que les coefficients de \( \phi_n\) sont dans \( \eZ\) alors qu'ils sont a priori dans \( \eC\). Nous démontrons cela par récurrence. D'abord \( \phi_1(X)=X-1\), d'accord. Ensuite
            \begin{equation}
                X^{n+1}-1=\prod_{d\divides n+1}\phi_d(X)=\phi_{n+1}(X)\cdot\underbrace{\prod_{_{\substack{d\divides n+1\\d\leq n}}}\phi_d(X)}_{\in\eZ[X]\text{ par récurrence}}
            \end{equation}
            Le lemme \ref{LemzwkYdn} conclu que \( \phi_{n+1}\in \eZ[X]\). Nous avons vu \( \eZ\) comme sous anneau du corps \( \eC\).

        \item

            Si \( m\) divise \( n\) alors les diviseurs de \( n\) sont l'union des diviseurs de \( m\) et des diviseurs de \( n\) qui ne divisent pas \( m\). Soit
            \begin{equation}
                Q=\{\text{diviseurs de \( n\) ne divisant pas \( m\)} \}.
            \end{equation}
            Nous avons alors
            \begin{equation}
                X^n-1=\prod_{d\divides n}\phi_d(X)=\prod_{d\divides m}\phi_d(X)\cdot\prod_{q\in Q}\phi_q(X)=(X^m-1)\cdot\prod_{q\in Q}\phi_q(X).
            \end{equation}
            Nous avons donc
            \begin{equation}
                T(X)=\frac{ X^n-1 }{ X^m-1 }=\prod_{q\in Q}\phi_q(X)\in \eZ[X].
            \end{equation}
            
        \item

            Nous venons de montrer que
            \begin{equation}
                T=\prod_{q\in Q}\phi_q\in \eZ[X].
            \end{equation}
            Étant donné que \( m<n\) nous avons \( n\in Q\) et donc
            \begin{equation}
                T=\phi_n\cdot\prod_{q\in Q\setminus\{ n \}}\phi_q.
            \end{equation}
            Par conséquent \( \phi_n\) divise \( T\) dans \( \eZ[X]\).
        \end{enumerate}
\end{proof}

\begin{proposition}[\wikipedia{fr}{Polynôme_cyclotomique}{polynôme cyclotomique}] \label{PropoIeOVh}
    Les polynômes cyclotomiques sont irréductibles sur \( \eQ\).
\end{proposition}
\index{polynôme!cyclotomique!irréductibilité}
\index{Anneau!\( \eZ/n\eZ\)!polynôme cyclotomique}
\index{nombre premier!polynôme cyclotomique}
\index{racine!de l'unité}
\index{corps!de rupture!polynôme cyclotomique}


\begin{proof}
    Pour rappel, nous savons déjà que pour tout \( n\in\eN\), \( \phi_n\in \eZ[X]\). Vu que les racines de \( \phi_n\) sont les racines primitives de l'unité, nous devons montrer que toutes les racines primitives de l'unité ont même polynôme minimal (qui sera alors \( \phi_n\)); en effet vu que ces polynômes divisent \( \phi_n\), si ils sont distincts, la proposition \ref{PropyMTEbH} s'applique et le produit des polynômes minimaux diviserait \( \phi_n\). Dans le cas inverse, \( \phi_n\) est polynôme minimal des racines primitives de l'unité et est donc irréductible. Soit donc \( \xi\), une telle racine primitive. Une autre racine primitive est de la forme \( \xi^l\) où \( l\) est un nombre premier tel que \( \pgcd(l,n)=1\).

    Soient \( f\) et \( g\), les polynômes minimaux dans \( \eZ[X]\) de \( \xi\) et \( \xi^l\). Nous allons montrer que \( f=g\) et donc que \( f=g=\phi_n\). Supposons par l'absurde que \( f\neq g\). Dans ce cas ils seraient des facteurs irréductibles distincts de \( \phi_n\) et il existerait un polynôme \( h\) tel que \( \phi_n=fgh\). A priori, \( h\in \eQ[X]\) parce que nous sommes justement en train de prouver que \( \phi_n\) est irréductible dans \( \eQ[X]\). Quoi qu'il en soit, le lemme de Gauss \ref{LemEfdkZw} nous montre que \( h\in \eZ[X]\) parce que \( \phi_n\), \( f\) et \( g\) ont des coefficients entiers. Nous avons
    \begin{equation}
        f(\xi)=g(\xi^l)=0.
    \end{equation}
    Considérons le polynôme \( \psi(X)=g(X^l)\). Ce polynôme \( \psi\) est dans \( \eZ[X]\) et \( \psi\) est annulateur de \( \xi\), donc \( f\) divise \( \psi\) en tant que polynôme minimal de \( \xi\). Il y a un polynôme unitaire à coefficients entiers (lemme de Gauss forever) \( k\) tel que
    \begin{equation}
        \psi=fk
    \end{equation}
    Nous considérons maintenant les projections sur \( \eF_l[X]\) : étant donné que \( \phi_n=fgh\), nous savons que \( \bar f\bar g\) divise \( \bar\phi_n\). En même temps, \( \bar f\) divise \( \bar \psi\). En utilisant le morphisme de Frobenius (c'est ici que la projection sur \( \eF_l\) joue), nous avons aussi
    \begin{equation}
        \bar\psi(X)=\bar g(X^l)=\bar g(X)^l.
    \end{equation}
    Par conséquent dire que \( \bar f\) divise \( \bar\psi\) revient à dire que \( \bar f(X)\) divise \( \bar g(X)^l\). En particulier tous facteur irréductible de \( \bar f\) divise \( \bar g\). Un facteur irréductible de \( \bar f\) serait donc à la fois dans \( \bar f\) et dans \( \bar g\) et donc deux fois (au moins) dans \( \bar\phi_n\) parce que \( \bar f\bar g\) divise \( \phi_n\). Dans un corps de décomposition de ce facteur, \( \phi_n\) aurait une racine double, alors que ce n'est pas le cas. Contradiction. Nous concluons que \( f=g\).
\end{proof}

\begin{theorem} \label{ThojCJpFW}
    Soit \( P\in \eZ[X]\) un polynôme unitaire irréductible non constant tel que toutes les racines dans \( \eC\) soient de module \( \leq 1\). Alors \( P=X\) ou \( P\) est un polynôme cyclotomique.
\end{theorem}

\begin{proof}
    Nous supposons que \( X\neq 0\), et nous notons \( P=\sum_ia_iX^i\). Étant donné que \( P\) est irréductible et différent de \( X\), nous avons \( a_0\neq 0\) (sinon \( x=0\) serait une racine). Nous allons montrer que les racines de \( P\) sont toutes des racines \( N\)-ièmes de l'unité (avec le même \( N\) pour toutes).

    Soient \( \{ \xi_i \}_{i=1,\ldots, d}\) les racines de \( P\); on a
    \begin{equation}
        P=\prod_{i=1}^d(X-\xi_i)
    \end{equation}
    avec \( \prod_{i=1}^d\xi_i=a_0\). Par hypothèse, \( | \xi_i |\leq 1\) et donc \( 0<| a_0 |\leq 1\). Vu que \( P\in \eZ[X]\) nous avons donc \( a_0=1\) et donc \( | \xi_i |=1\) pour tout \( i\).

    Nous introduisons les polynômes
    \begin{equation}
        g_q(X)=\prod_{i=1}^d\big( X-(\xi_i)^q \big),
    \end{equation}
    et en particulier \( g_1=P\), et nous développons
    \begin{equation}
        g_q(X)=X^n+C_{1,q}X^{n-1}+\ldots +C_{n,q}
    \end{equation}
    où
    \begin{equation}
        C_{k,q}=(-1)^k\sum_{1\leq i_1<\ldots<i_k\leq d}(\xi_{i_1}\ldots \xi_{i_k})^q.
    \end{equation}
    Nous introduisons aussi les polynômes
    \begin{equation}
        F_{k,q}(X_1,\ldots, X_n)=(-1)^k\sum_{1\leq i_1<\ldots< i_k\leq d}(X_{i_1}\ldots X_{i_k})^q
    \end{equation}
    qui sont des polynômes symétriques. Ils vérifient deux propriétés. La première est que
    \begin{equation}
        C_{r,q}=F_{r,q}(\xi_1,\ldots, \xi_n),
    \end{equation}
    et la seconde est que les polynômes \( F_{r,1}\) sont les polynômes symétriques élémentaires à un coefficients près. Le théorème \ref{TholReBiw} nous donne alors des polynômes \( G_{k,q}\in \eZ[X_1,\ldots, X_n]\) tels que
    \begin{equation}
        F_{k,q}(X_1,\ldots, X_n)=G_{k,q}\big( F_{1,1}(X_1,\ldots, X_n),\ldots, F_{k,1}(X_1,\ldots, X_n) \big).
    \end{equation}
    Nous savons que
    \begin{equation}
        | C_{k,q} |\leq \sum_{1\leq i_1<\ldots<i_k<d}1={d\choose k}.
    \end{equation}
    Donc \( g_q\) fait partie de l'ensemble fini des polynômes dans \( \eZ[q]\) dont tous les coefficients sont bornée en valeur absolue par 
    \begin{equation}
        \max_{k=1,\ldots, d}{d\choose k}.
    \end{equation}
    Il existe un certain nombre d'ensembles \( \{ \xi_i \}\) qui sont racines de polynômes vérifiant les conditions du théorème. À chacun de ces ensembles est associé une suite de polynômes \( g_q\) et donc des coefficients \( C_{k,q}\). Ce que nous avons vu est que l'ensemble de tous les coefficients \( C_{k,q}\) possibles (pour un choix donné des \( \{ \xi_i \}\)) est fini, en particulier, vu que \( C_{1,q}=\sum_i\xi_i^q\), pour chaque \( k\), l'ensemble
    \begin{equation}
        \{ \xi_k^q\tq q\in \eN \}.
    \end{equation}
    Par le principe des tiroirs, il existe \( q_1\) et \( q_2\) tels que \( \xi_k^{q_1}=\xi_k^{q_2}\). Ici, \( q_1\) et \( q_2\) dépendent de \( k\) et nous notons \( N_k=q_1-q_2\); nous avons donc \( \xi_k^{N_k}=1\).

    En posant \( N=\ppcm(N_1,\ldots, N_d)\), nous avons
    \begin{equation}
        \xi_k^N=1
    \end{equation}
    pour tout \( k\).

    Mais \( P\) est irréductible dans \( \eZ[X]\); si il a \( \pm 1\) comme racines, alors c'est que \( P=X+1\) ou \( P=X-1\) et ce sont des polynômes cyclotomiques. Si \( P\) n'a pas \( \pm 1\) parmi ses racines, alors \( P\) n'a pas de racines dans \( \eQ\) parce que \( \pm 1\) sont les seules racines de \( X^N-1\) dans \( \eQ\).

    Par conséquent \( P\) est un facteur irréductible de \( X^N-1\) dans \( \eQ[X]\). Mais étant donné que
    \begin{equation}
        X^N-1=\prod_{d\divides N}\phi_d(X),
    \end{equation}
    les polynômes cyclotomiques sont les seuls facteurs irréductibles de \( X^N-1\). Donc \( P\) est un polynôme cyclotomique.
\end{proof}

%---------------------------------------------------------------------------------------------------------------------------
\subsection{Nombres premiers}
%---------------------------------------------------------------------------------------------------------------------------

\begin{lemma}[\cite{naKXuR}]    \label{LemiAqLEn}
    Soit \( n\geq 1\). Il existe un nombre premier \( p\) et un entier \( a\) tels que
    \begin{enumerate}
        \item
            \( p\) divise \( \phi_n(a)\),
        \item
            \( p\) ne divise aucun de \( \phi_d(a)\) avec \( d\divides n\) et \( d\neq n\).
    \end{enumerate}
    De tels \( p\) et \( a\) vérifient automatiquement
    \begin{enumerate}
        \item
            \( p\) divise \( a^n-1\),
        \item
            \( p\) ne divise aucun des \( a^d-1\) pour \( d\divides n\), \( d\neq n\).
    \end{enumerate}
\end{lemma}

\begin{proof}
    Nous posons
    \begin{equation}
        B(X)=\prod_{_{\substack{d\divides n\\d\neq n}}}\phi_d(X),
    \end{equation}
    et nous commençons par montrer que \( \phi_n\) est premier avec \( B\). Nous avons \( X^n-1=B\phi_n\), donc \( B\) et \( \phi_n\) n'ont pas de racines communes (même pas dans \( \eC\)) parce que ce serait une racine double de \( X^n-1\). Notons que par définition \ref{EqLjGYKK}, les polynômes cyclotomiques sont scindés (dans \( \eC\)), donc en particulier les polynômes \( \phi_n\) et \( B\) sont scindés et dons premiers entre eux, dans \( \eC\) et a fortiori dans \( \eQ\). Par Bézout (corollaire \ref{CorimHyXy}), il existe \( U,V\in\eQ[X]\) tels que
    \begin{equation}
        U\phi_n+VB=1.
    \end{equation}
    Si nous prenons \( a\in \eZ\) tel que \( U'=aU\) et \( V'=aV\) soient tous deux dans \( \eZ[X]\), alors nous avons
    \begin{equation}    \label{EqCpNMEi}
        U'\phi_n+V'B=a,
    \end{equation}
    égalité dans \( \eZ[X]\). Quitte à prendre un multiple assez grand de \( a\), nous pouvons choisir \( a\) de telle sorte que \( | \phi_n(a) |\geq 2\). Nous prenons alors un nombre premier \( p\) divisant \( \phi_n(a)\). 

    Montrons que le \( a\) et le \( p\) ainsi construis satisfont aux exigences.

    Vu que \( X^n-1=B\phi_n\), si \( p\) divise \( \phi_n(a)\), il divise automatiquement \( a^n-1\) et donc \( [a^n]_p=1\), ce qui signifie entre autres que \( a\) et \( p\) sont premiers entre eux. Évaluons l'équation \eqref{EqCpNMEi} en~\( a\) :
    \begin{equation}
        U'(a)\phi_n(a)+V'(a)B(a)=a.
    \end{equation}
    Le nombre \( p\) ne divisant pas \( a\), mais divisant \( \phi_n(a)\), il ne peux pas diviser \( B(a)\)\footnote{C'est pour pouvoir dire ça que l'on a choisit \( V'\in \eZ[X]\) de telle sorte que \( V'(a)\) soit dans \( \eZ\)}. Étant donné que \( p\) ne divise pas \( B(a)\), il ne divise aucun des \( \phi_d(a)\) avec \( d\divides n\) et \( d\neq n\).

    Nous passons maintenant à la seconde partie de la preuve. Nous supposons avoir \( a\) et \( p\) tels que \( p\) soit un nombre premier divisant \( \phi_n(a)\) et tels que \( p\) ne divise aucun des \( \phi_d(a)\) avec \( d\divides n\), \( d\neq n\). Le fait de diviser \( \phi_n(a)\) entraine le fait de diviser \( a^n-1\) parce que \( \phi_n\) est un des facteurs de \( X^n-1\). Soit maintenant \( d\neq n\) divisant \( n\); nous avons
    \begin{equation}    \label{EqwTWcCu}
        X^d-1=\prod_{d'\divides d}\phi_{d'},
    \end{equation}
    et cela est une partie du produit
    \begin{equation}
        \prod_{\substack{d\divides n\\d\neq n}}\phi_d.
    \end{equation}
    Vu que \( p\) ne divise aucun des \( \phi_d(a)\) de ce dernier produit, a fortiori, il ne divise pas le produit \ref{EqwTWcCu}, et donc pas \( a^d-1\).
\end{proof}

\begin{lemma}       \label{LemrZnmpG}
    Si \( n\geq 1\), alors il existe un nombre premier dans \( [1]_n\), c'est à dire un nombre premier de la forme \( 1+kn\) avec \( k\in \eN^*\). 
\end{lemma}

\begin{proof}
    Soit \( n\geq 1\) et les nombres \( p,a\) donnés par le lemme \ref{LemiAqLEn}. Vu que \( p\) divise \( \phi_n(a)\), \( p\) divise \( a^n-1\) et donc \( [a]_p\) a un ordre qui divise \( n\) dans \( (\eZ/p\eZ)^*\) parce que \( [a]_p^n=[1]_p\).

    Prenons \( d\neq n\) divisant \( n\). Nous savons que
    \begin{equation}
        a^d-1=\prod_{d'\divides d}\phi_{d'}(a).
    \end{equation}
    
    Par construction de \( a\) et \( p\), nous avons
    \begin{equation}
        [\phi_{d'}(a)]_p\neq 0
    \end{equation}
    Vu que \( \eZ/p\eZ\) est intègre, le produit est également non nul, c'est à dire
    \begin{equation}
        \big[ \prod_{d'\divides d}\phi_{d'}(a) \big]_p\neq 0,
    \end{equation}
    et donc \( [a]_p^a\neq 1\). Nous avons donc montré que si \( d\neq n\) divise \( n\), alors nous avons en même temps
    \begin{equation}
        [a]_p^n=1
    \end{equation}
    et
    \begin{equation}
        [a]_p^d\neq 1.
    \end{equation}
    Cela prouve que \( [a]_p\) est d'ordre exactement \( n\). Oui, mais l'ordre de \( [a]_p\) doit diviser l'ordre du groupe \( \eZ/p\eZ\) qui est \( p-1\), donc \( n\) divise \( p-1\) et nous écrivons \( p=kn+1\) avec \( k\) entier.
\end{proof}

\begin{theorem}[Forme faible du théorème de Dirichlet \cite{fJhCTE}]    \label{ThoxwTjcl}   
    Pour tout \( n\geq 1\), il existe une infinité de nombres premiers dans \( [1]_n\).
\end{theorem}
\index{nombre!premier}
\index{Dirichlet!théorème (sur les nombres premiers)}
\index{théorème!Dirichlet!forme faible}
\index{anneau!\( \eZ/n\eZ\)}
\index{racine de l'unité}

\begin{proof}
    Le lemme \ref{LemrZnmpG} nous donne déjà l'existence de nombres premiers dans \( [1]_n\). Il faut maintenant voir qu'il y en a une infinité. Nous supposons qu'il y en ait seulement un nombre fini : \( p_1,\ldots, p_r\), et nous notons 
    \begin{equation}
        N=np_1\ldots p_r.
    \end{equation}
    Nous utilisons maintenant le lemme \ref{LemrZnmpG} avec ce \( N\), c'est à dire qu'on a un nombre premier de la forme
    \begin{equation}
        p=1+kN=1+knp_1\ldots p_r.
    \end{equation}
    Cela est un nombre premier plus grand que tous les \( p_i\) et de la forme \( 1+\lambda n\). Cela contredit l'exhaustivité de la liste \( p_1,\ldots, p_r\).
\end{proof}

%---------------------------------------------------------------------------------------------------------------------------
\subsection{Le jeu de la roulette}
%---------------------------------------------------------------------------------------------------------------------------
\label{pTqJLY}
\index{groupe!fini}
\index{groupe!de permutations}
\index{groupe!et géométrie}
\index{combinatoire}
\index{dénombrement}

Source : \cite{HEBOFl}.

Soit une roulette à \( n\) secteurs que nous voulons colorier en \( q\) couleurs. Nous voulons savoir le nombre de possibilités à rotations près. Soit d'abord \( E\) l'ensemble des coloriages possibles sans contraintes; il y a naturellement \( q^n\) possibilités. Sur l'ensemble \( E\), le groupe cyclique \( G\) des rotations d'angle \( 2\pi/n\) agit. Deux coloriages étant identiques si ils sont reliés par une rotation, la réponse à notre problème est donné par le nombre d'orbites de l'action de \( G\) sur \( E\) qui sera donnée par la formule de Burnside \ref{EqTUsblv}. 

Nous devons calculer \( \Card\big( \Fix(g) \big)\) pour tout \( g\in G\). Soit \( g\), un élément d'ordre \( d\) dans \( G\). Si \( g\) agit sur la roulette, chaque secteur a une orbite contenant \( d\) éléments. Autrement dit, \( g\) divise la roulette en \( n/d\) secteurs. Un élément de \( E\) appartenant à \( \Fix(g)\) doit colorier ces \( n/d\) secteurs de façon uniforme; il y a \( q^{n/d}\) possibilités.

Il reste à déterminer le nombre d'éléments d'ordre \( d\) dans \( G\). Un élément de \( G\) est donné par un nombre complexe de la forme \(  e^{2ik\pi/n}\). Les éléments d'ordre \( d\) sont les racines primitives\footnote{Une racine non primitive \( 8\)ième de l'unité est par exemple \( i\). Certes \( i^8=1\), mais \( i^4=1\) aussi. Le nombre \( i\) est d'ordre \( 4\).} \( d\)ièmes de l'unité. Nous savons que --par définition-- il y a \( \varphi(d)\) telles racines primitives de l'unité. Bref il y a \( \varphi(d)\) éléments d'ordre \( d\) dans \( G\). 

La formule de Burnside nous donne maintenant le nombre d'orbites :
\begin{equation}
    \frac{1}{ n }\sum_{d|n}\varphi(d)q^{n/d}.
\end{equation}
Cela est le nombre de coloriage possibles de la roulette à \( n\) secteurs avec \( q\) couleurs.

%---------------------------------------------------------------------------------------------------------------------------
\subsection{L'affaire du collier}
%---------------------------------------------------------------------------------------------------------------------------
\label{siOQlG}

Nous avons maintenant des perles de \( q\) couleurs différentes et nous voulons en faire un collier à \( n\) perles. Cette fois non seulement les rotations donnent des colliers équivalents, mais en outre les symétries axiales (il est possible de retourner un collier, mais pas une roulette). Le groupe agissant sur \( E\) est maintenant le groupe diédral\index{diédral}\index{groupe!diédral} \( D_n\) conservant un polygone a \( n\) sommets.

Nous devons séparer le cas \( n\) impair du cas \( n\) pair.

Si \( n\) est impair, alors les axes de symétries passent par un sommet par le milieu du côté opposé. Le groupe \( D_n\) contient \( n\) symétries axiales. Nous avons donc maintenant
\begin{equation}
    | G |=2n.
\end{equation}
Nous écrivons la formule de Burnside
\begin{equation}
    \Card(\Omega)=\frac{1}{ 2n }\sum_{g\in G}\Card\big( \Fix(g) \big).
\end{equation}
Si \( g\) est une rotation, le travail est déjà fait. Si \( g\) est une symétrie, nous avons le choix de la couleur du sommet par lequel passe l'axe et le choix de la couleur des \( (n-1)/2\) paires de sommets. Cela fait
\begin{equation}
    qq^{(n-1)/2}=q^{\frac{ n+1 }{2}}
\end{equation}
possibilités. Nous avons donc
\begin{equation}
    \Card(\Omega)=\frac{1}{ 2n }\left( \sum_{d|n}q^{n/d}\varphi(d)+nq^{\frac{ n+1 }{2}} \right).
\end{equation}

Si \( n\) est pair, le choses se compliquent un tout petit peu. En plus de symétries axiales passant par un sommet et le milieu du côté opposé, il y a les axes passant par deux sommets opposés. Pour colorier un collier en tenant compte d'une telle symétrie, nous pouvons choisir la couleur des deux perles par lesquelles passe l'axe ainsi que la couleur des \( (n-2)/2\) paires de perles. Cela fait en tout
\begin{equation}
    q^2q^{\frac{ n-2 }{2}}=q^{\frac{ n+2 }{2}}.
\end{equation}
Le groupe \( G\) contient \( n/2\) tels axes.

Notons que cette fois \( G\) ne contient plus que \( n/2\) symétries passant par un sommet et un côté. L'ordre de $G$ est donc encore \( 2n\). La formule de Burnside donne
\begin{equation}
    \Card(\Omega)=\frac{1}{ 2n }\left( \sum_{d\divides n}\varphi(d)q^{n/d}+\frac{ n }{2}q^{(n+2)/2}+\frac{ n }{2}q^{n/2} \right).
\end{equation}

%---------------------------------------------------------------------------------------------------------------------------
\subsection{Théorème de Wedderburn}
%---------------------------------------------------------------------------------------------------------------------------

\begin{theorem}[Théorème de Wedderburn\cite{SQxrsoL}]    \label{ThoMncIWA}
    Tout corps fini est commutatif.
\end{theorem}
\index{groupe!fini}
\index{théorème!Wedderburn}
\index{action!de groupe!Wedderburn}
\index{nombre!complexe!norme \( 1\)}
\index{groupe!fini!Wedderburn}
\index{corps!fini!Wedderburn}

\begin{proof}
    Soit \( \eK\) un corps fini et \( Z\), le centre de \( \eK\). Ce dernier est un corps fini et un sous corps de \( \eK\). Si \( q=\Card(Z)\) alors par le lemme \ref{LemobATFP} nous avons
    \begin{equation}
        \Card(\eK)=q^n
    \end{equation}
    pour un certain \( n\).

    Nous supposons maintenant que \( \eK\) est non commutatif. Dans ce cas \( Z\neq \eK\) et nous avons \( n\geq 2\). Nous considérons aussi
    \begin{equation}
        Z_x=\{ a\in \eK\tq ax=xa \}.
    \end{equation}
    Le centre \( Z\) est un sous corps de \( Z_x\), donc il existe \( d(x)\) tel  que
    \begin{equation}
        \Card(Z_x)=q^{d(x)}.
    \end{equation}
    De la même manière, \( Z_x\) est un sous corps de \( \eK\), donc il existe \( m(x)\) tel que
    \begin{equation}
        \Card(\eK)=\Card(Z_x)^{m(x)}.
    \end{equation}
    En mettant bout à bout nous avons
    \begin{equation}
        q^n=\Card(Z_x)^{m(x)}=q^{d(x)m(x)},
    \end{equation}
    et par conséquent \( n=d(x)m(x)\). Le point important à retenir est que \( d(x)\) divise \( n\) pour tout \( x\in \eK\).

    Nous considérons maintenant l'action adjointe du groupe \( \eK^*\) sur lui-même :
    \begin{equation}
        \varphi(k)x=kxk^{-1}.
    \end{equation}
    Nous notons \( \mO_x\) l'orbite de \( x\in \eK^*\) pour cette action, et \( \Stab(x)\) son stabilisateur. Nous avons
    \begin{equation}
        Z_y=\Stab(y)\cup\{ 0 \}
    \end{equation}
    parce que \( Z_y\) et \( \Stab(y)\) ont les mêmes définitions, sauf que \( \Stab(y)\) est dans \( \eK^*\) alors que \( Z_y\) est dans \( \eK\). Nous avons donc
    \begin{equation}
        \Card\big( \Stab(y) \big)=\Card(Z_y)-1=q^{d(y)}-1.
    \end{equation}
    Nous avons \( \Card(\mO_x)=1\) si et seulement si \( \mO_x=\{ x \}\) si et seulement si \( \Stab(x)=\eK^*\) si et seulement si \( z\in Z^*\). Soient \( z_0,\ldots, z_{q-1}\) les éléments de \( Z\) avec \( z_0=0\). Ce sont les éléments qui auront une orbite réduite à un point. Les orbites qui coupent \( Z^*\) sont
    \begin{equation}
        \{ z_1 \},\ldots, \{ z_{q-1} \}
    \end{equation}
    et il y en a \( q-1\). Soient \( \mO_{y_1},\ldots, \mO_{y_r}\), les autres orbites. Nous utilisons l'équation des classes \eqref{EqkgGmoq} :
    \begin{equation}
        \Card(\eK^*)=\Card(Z^*)+\sum_{i=1}^{r}\frac{ \Card(\eK^*) }{ \Card(\Stab(y_i)) },
    \end{equation}
    mais \( \Card(Z^*)=q-1\), \( \Card(\eK^*)=q^n-1\) et \( \Card\big( \Stab(y_i) \big)=q^{d(y_i)}-1\), donc
    \begin{equation}        \label{EqBPBDzE}
        q^n-1=(q-1)+\sum_{i=1}^{r}\frac{ q^n-1 }{ q^{d(y_i)}-1 }.
    \end{equation}
    Nous considérons la fraction rationnelle
    \begin{equation}        \label{EqATGciu}
        F(X)=(X^n-1)-\sum_{i=1}^{r}\frac{ X^n-1 }{ X^{d(y_i)}-1 }.
    \end{equation}
    Étant donné que \( d(y_i)\) divise \( n\), nous avons, contrairement aux apparences, que \( F\in \eZ[X]\) par la proposition \ref{PropUImYnL}\ref{ItemhpDPKE}.

    Nous pouvons exploiter un peu mieux la proposition \ref{PropUImYnL} en remarquant que \( d(y_i)<n\) parce que sinon \( \Card(Z_{y_i})=\Card(\eK)\), ce qui signifierait que \( y_i\in Z\), ce qui nous avions exclu. Par conséquent le polynôme cyclotomique \( \phi_n\) divise 
    \begin{equation}
        \frac{ X^n-1 }{ X^{d(y_i)}-1 }
    \end{equation}
    dans \( \eZ[X]\). Le polynôme cyclotomique \( \phi_n\) divise également \( X^n-1\) et par conséquent \( \phi_n\) divise \( F\). Il existe donc \( Q\in \eZ[X]\) tel que \( F=Q\phi_n\). En particulier en évaluant en \( q\) :
    \begin{equation}    \label{eqmoLdJy}
        F(q)=Q(q)\phi_n(q)=q-1.
    \end{equation}
    En effet nous avons \( F(q)=q-1\) par construction : comparer \eqref{EqBPBDzE} avec \eqref{EqATGciu}. Évidemment \( q\neq 1\) parce que si \( q=1\) alors \( \Card(\eK)=1\) et le théorème est trivial. Par ailleurs \( Q(q)\) est un entier (parce que \( Q\in \eZ[X]\) et \( q\in \eN\)) et \( Q(q)\neq 0\), parce qu'à droite de \eqref{eqmoLdJy} nous avons \( q-1\neq 0\). Nous avons donc \( | Q(q) |\geq 1\) et donc
    \begin{equation}
        | \phi_n(q) |\leq q-1.
    \end{equation}
    Par définition du polynôme cyclotomique nous avons
    \begin{equation}
        | \phi_n(q) |=\prod_{z\in\Delta_n}| q-z |.
    \end{equation}
    Étant donné que ce produit doit être inférieur à \( q-1\), au moins un des termes doit l'être : il existe \( z_0\in \Delta_n\) tel que \( | z_0-q |\leq q-1\). Étant donné que \( n\geq 2\) nous avons \( z_0\neq 1\).

    Mais d'autre part, comme indiqué sur la figure \ref{LabelFigtrigoWedd}, la distance entre \( z_0\) et \( q\) doit être strictement plus grande que \( q-1\) parce que \( q-1\) est le minimum de la distance entre le cercle trigonométrique et \( q\), et n'est atteint qu'en \( z=1\).
    \newcommand{\CaptionFigtrigoWedd}{Nous devons avoir \( | z_0-q |>q-1\).}
    \input{Fig_trigoWedd.pstricks}

    Nous avons ainsi obtenu une contradiction, et nous concluons que le corps \( \eK\) est commutatif.
\end{proof}
