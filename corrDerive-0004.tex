% This is part of Outils mathématiques
% Copyright (c) 2011
%   Laurent Claessens
% See the file fdl-1.3.txt for copying conditions.

\begin{corrige}{Derive-0004}

	La partie du cercle dans le premier quadrant est donnée par la fonction
	\begin{equation}
		y(x)=\sqrt{R^2-x^2}
	\end{equation}
	dont la dérivée vaut
	\begin{equation}		\label{EqzqDery}
		y'(x)=\frac{ -x }{ \sqrt{R^2-x^2} }.
	\end{equation}
	Le point du cercle qui correspond à l'angle $\frac{ \pi }{ 4 }$ est le point qui correspond à $x=R\cos(\pi/4)=\frac{ R }{ \sqrt{2} }$. En mettant cette valeur de $x$ dans l'expression \eqref{EqzqDery} nous trouvons
	\begin{equation}
		y'\big( \frac{ R }{ \sqrt{2} } \big)=-1.
	\end{equation}
	Notez que le $R$ s'est simplifié.

	La tangente que nous cherchons est donc la droite de coefficient directeur $-1$ qui passe par $(x,y)=\big( R\frac{1}{ \sqrt{2} },R\frac{1}{ \sqrt{2} } \big)$. Si $y=ax+b$, nous avons immédiatement $a=-1$ Ensuite $b$ se trouve par l'équation
	\begin{equation}
		\frac{R}{ \sqrt{2} }=-\frac{R}{ \sqrt{2} }+b,
	\end{equation}
	d'où nous tirons $b=R\sqrt{2}$. En définitive,
	\begin{equation}
		y=-x+R\sqrt{2}.
	\end{equation}

\end{corrige}
