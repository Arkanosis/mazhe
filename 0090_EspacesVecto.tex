% This is part of Mes notes de mathématique
% Copyright (c) 2011-2013
%   Laurent Claessens, Carlotta Donadello
% See the file fdl-1.3.txt for copying conditions.

%+++++++++++++++++++++++++++++++++++++++++++++++++++++++++++++++++++++++++++++++++++++++++++++++++++++++++++++++++++++++++++
\section{Espaces de matrices}
%+++++++++++++++++++++++++++++++++++++++++++++++++++++++++++++++++++++++++++++++++++++++++++++++++++++++++++++++++++++++++++

L'ensemble des matrices est un espace vectoriel. Nous identifions $\eM(n,\eR)$ avec $ \eR^{n^2}$; plus précisément, nous identifions une matrice 
\begin{equation}
    A = (a_{i,j})_{1\leq i \leq n, 1 \leq j \leq n}
\end{equation}
avec le vecteur $x = (x_1, x_2, \dots, x_{n^2}) \in \eR^{n^2}$, où $ a_{i,j} = x_{(n-1)i + j}$. 

%---------------------------------------------------------------------------------------------------------------------------
\subsection{Connexité par arcs}
%---------------------------------------------------------------------------------------------------------------------------

\begin{lemma}
    Les groupes \( \gU(n)\) et \( \SU(n)\) sont connexes par arcs.
\end{lemma}

\begin{proof}
    Soit \( A\), une matrice unitaire et \( Q\) une matrice unitaire qui diagonalise \( A\). Étant donné que les valeurs propres arrivent par paires complexes conjuguées,
    \begin{equation}
        QAQ^{-1}=\begin{pmatrix}
            e^{i\theta_1}    &       &       &       &   \\  
            &    e^{-i\theta_1}    &       &       &   \\  
            &       &    \ddots    &       &   \\  
            &       &       &    e^{i\theta_r}    &   \\  
            &       &       &       &        e^{-i\theta_r}
        \end{pmatrix}.
    \end{equation}
    Le chemin \( U(t)\) obtenu en remplaçant \( \theta_i\) par \( t\theta_i\) avec \( t\in\mathopen[ 0 , 1 \mathclose]\) joint \( QAQ^{-1}\) à l'identité. Par conséquent \( Q^{-1}U(t)Q\) joint \( A\) à l'unité.
\end{proof}

\begin{theorem}
    Les matrices \wikipedia{fr}{Endomorphisme_normal}{normales} forment un espace connexe par arc.
\end{theorem}

\begin{proof}
    Soit \( A\) une matrice normale, et \( U\) une matrice unitaire qui diagonalise \( A\). Nous considérons \( U(t)\), un chemin qui joint \( \mtu\) à \( U\) dans \( \gU(n)\). Pour chaque \( t\), la matrice
    \begin{equation}
        A(t)=U(t)^{-1} AU(t)
    \end{equation}
    est normale. Nous avons donc trouvé un chemin dans les matrices normales qui joint \( A\) à une matrice diagonale. Il est à présent facile de la joindre à l'identité.

    Toutes les matrices normales étant connexes à l'identité, l'ensemble des matrices normales est connexe.
\end{proof}

%---------------------------------------------------------------------------------------------------------------------------
\subsection{Densité}
%---------------------------------------------------------------------------------------------------------------------------

\begin{proposition}     \label{PropDigDensVxzPuo}
    Les matrices diagonalisables sont denses dans \( \eM(n,\eC)\).
\end{proposition}

\begin{proof}
    D'après le lemme de Schur \ref{LemSchurComplHAftTq}, une matrice de \( \eM(n,\eC)\) est de la forme
    \begin{equation}
        A=Q\begin{pmatrix}
            \lambda_1    &   *    &   *    \\
              0  &   \ddots    &   *    \\
            0    &   0    &   \lambda_n
        \end{pmatrix}Q^{-1}.
    \end{equation}
    Les valeurs propres sont sur la diagonale. La matrice est diagonalisable si les éléments de la diagonales sont tous différents. Il suffit maintenant de considérer \( n\) suites \( (\epsilon^{(r)}_k)_{k\in\eN}\) convergentes vers zéro telles que pour chaque \( k\) les nombres \( \lambda_r+\epsilon^{(r)}_k\) soient tous différents. La suite de matrices
    \begin{equation}
        A_k=Q\begin{pmatrix}
            \lambda_1+\epsilon^{(1)}_k    &   *    &   *    \\
              0  &   \ddots    &   *    \\
              0    &   0    &   \lambda_n+\epsilon^{(n)}_k
        \end{pmatrix}Q^{-1}.
    \end{equation}
    est alors diagonalisable pour tout \( k\) et nous avons \( \lim_{k\to \infty} A_k=A\).
\end{proof}

\begin{proposition}
    Si \( A\in\eM(n,\eC)\) alors
    \begin{equation}
        e^{\tr(A)}=\det( e^{A}).
    \end{equation}
\end{proposition}

\begin{proof}
    Le résultat est un simple calcul pour les matrices diagonalisable. Si \( A\) n'est pas diagonalisable, nous considérons une suite de matrices diagonalisables \( A_k\) dont la limite est \( A\) (proposition \ref{PropDigDensVxzPuo}). La suite
    \begin{equation}
        a_k= e^{\tr(A_k)}
    \end{equation}
    converge vers \(  e^{\tr(A)}\) tandis que la suite 
    \begin{equation}
        b_k=\det( e^{A_k})
    \end{equation}
    converge vers \( \det( e^{A})\). Mais nous avons \( a_k=b_k\) pour tout \( k\); les limites sont donc égales.
\end{proof}

%---------------------------------------------------------------------------------------------------------------------------
\subsection{Racine carré d'une matrice hermitienne positive}
%---------------------------------------------------------------------------------------------------------------------------

\begin{proposition}     \label{PropVZvCWn}
    Si \( A\in \eM(n,\eC)\) est une matrice hermitienne positive, alors il existe une unique matrice hermitienne positive \( R\) telle que \( A=R^2\). De plus \( R\) est un polynôme (de \( \eR[X]\)) en \( A\).
\end{proposition}
\index{matrice!semblable}
\index{polynôme!d'endomorphisme}
\index{endomorphisme!diagonalisable}
\index{matrice!hermitienne!racine carré}
\index{racine!carré!de matrice hermitienne}

La matrice \( R\) ainsi définie est la \defe{racine carré de}{matrice!racine carré}\index{racine!carré de matrice!hermitienne positive} de \( A\), et est notée \( \sqrt{A}\)\nomenclature[A]{\( \sqrt{A}\)}{racine d'une matrice hermitienne positive}. Une des applications usuelles de cette proposition est la décomposition polaire.

\begin{proof}
    \begin{subproof}
    \item[Existence]
        Étant donné que \( A \) est hermitienne, elle est diagonalisable par une unitaire (proposition \ref{ThogammwA}), et ses valeurs propres sont réelles et positives (parce que \( A\) est positive). Soit donc \( P\) une matrice unitaire telle que
        \begin{equation}
            P^*AP=\begin{pmatrix}
                \alpha_1    &       &       \\
                    &   \ddots    &       \\
                    &       &   \alpha_n
            \end{pmatrix}
        \end{equation}
        avec \( \alpha_i>0\). Si on pose
        \begin{equation}
            R=P\begin{pmatrix}
                \sqrt{\alpha_1}    &       &       \\
                    &   \ddots    &       \\
                    &       &   \sqrt{\alpha_n}
            \end{pmatrix}P^*,
        \end{equation}
        alors \( R^2=A\) parce que \( P^*P=\mtu\).
    \item[Hermitienne positive]
        La matrice \( R\) est hermitienne parce que, avec un peu de notation raccourcie, \( R=P^*\sqrt{\alpha}P\) et \( R^*=P^*\sqrt{\alpha}P\). D'autre part, elle est positive parce que ses valeurs propres sont les \( \sqrt{\alpha_i}\) qui sont positives.
        
    \item[Polynôme]
        Nous montrons maintenant que la matrice \( R\) est un polynôme en \( A\). Pour cela nous considérons un polynôme \( Q\) tel que \( A(\alpha_i)=\sqrt{\alpha_i}\) pour tout \( i\). Soit \( \{ e_i \}\) une base de diagonalisation de \( A\) : \( Ae_i=\alpha_ie_i\). Alors c'est encore une base de diagonalisation de \( Q(A)\). En effet si \( Q=\sum_ka_kX^k\), alors
        \begin{equation}
            Q(A)e_i=(\sum_ka_kA^k)e_i=(\sum_ka_k\alpha_i^k)e_i=Q(\alpha_i)e_i=\sqrt{\alpha_i}e_i.
        \end{equation}
        Les valeurs propres de \( Q(A)\) sont donc \( \sqrt{\alpha_i}\). Nous savons maintenant que \( Q(A)\) a la même base de diagonalisation de \( A\) (et donc la même matrice unitaire \( P\) qui diagonalise), c'est à dire que
        \begin{equation}
            Q(A)=P^*\begin{pmatrix}
                \sqrt{\alpha_1}    &       &       \\
                    &   \ddots    &       \\
                    &       &   \sqrt{\alpha_n}
            \end{pmatrix}=R.
        \end{equation}
        Donc oui, \( R\) est un polynôme en \( A\).

        Notons que ce \( Q\) n'est pas du tout unique; il existe une infinité de polynômes qui envoient \( n\) nombres donnés sur \( n\) nombres donnés.

    \item[Unicité]
        Soit \( S\) une matrice hermitienne positive telle que \( R^2=S^2=A\). D'abord \( S\) commute avec \( A\) parce que
        \begin{equation}
            SA=S^3=S^2S=AS.
        \end{equation}
        Donc \( S\) commute aussi avec \( Q(A)=R\). Étant donné que \( S\) et \( R\) commutent et sont diagonalisables, ils sont simultanément diagonalisables par le corollaire \ref{CorQeVqsS}. Soient \( D_R=PRP^*\) et \( D_S=PSP^*\) les formes diagonales de \( R\) et \( S\) dans une base de simultanée diagonalisation. Les carrés des valeurs propres de \( R\) et \( S\) étant identiques (ce sont les valeurs propres de \( A\)) et les valeurs propres de \( R\) et \( S\) étant positives, nous déduisons que \( D_R=D_S\) et donc que \( R=P^*D_RP=P^*D_SP=S\).
    \end{subproof}
\end{proof}

%---------------------------------------------------------------------------------------------------------------------------
\subsection{Racine carré d'une matrice symétrique positive}
%---------------------------------------------------------------------------------------------------------------------------

\begin{lemma}[\cite{JJdQPyK}]   \label{LemTLlTAAf}
    Le groupe orthogonal \( O(n,\eR) \) est compact.
\end{lemma}

\begin{proof}
    Nous avons \( O(n)=f^{-1}\big( \{ \mtu_n \} \big)\) où \( f\) est l'application continue \( A\mapsto A^tA\). En tant qu'image inverse d'un fermé par une application continue, le groupe \( O(n)\) est fermé.

    De plus il est borné parce que tous les coefficients d'une matrice orthogonale sont \( \leq 1\), donc \( \| A \|_{\infty}\) pour tout \( A\in O(n)\).
\end{proof}

\begin{proposition} \label{PropPEMDqVT}
    Une matrice symétrique semi (ou pas) définie positive admet une unique racine carré symétrique. Le spectre de la racine carré est la racine carré du spectre de la matrice de départ.
\end{proposition}

\begin{proof}
    Ceci est une phrase pour que les titres se mettent bien.
    \begin{subproof}
        \item[Existence]
            Soit \( T\) une matrice symétrique et \( Q\) une matrice orthogonale qui diagonalise\footnote{Théorème \ref{ThoeTMXla}.} \( T\) : \( QTQ^{-1}=D\) avec \( D=\diag(\lambda_i)\) et \( \lambda_i\geq 0\). En posant \( R=Q^{-1}\sqrt{D}Q\), il est vite vérifié que \( R^2=T\) et que \( R\) est symétrique. En ce qui concerne le spectre, \( R\) a pour valeurs propres les \( \sqrt{\lambda_i}\).
        \item[Unicité]

            Soit \( R\) une matrice symétrique de \( T\) : \( R^2=T\). Du coup \( R\) et \( T\) commutent : \( RT=R^3=TR\). Par conséquent les espaces propres de \( T\) sont stables sous \( R\). Soit \( E_{\lambda} \) l'un d'eux de dimension \( d\), et \( T_F\), \( R_F\) les restrictions de \( T\) et \( R\) à \( E_{\lambda}\). L'application \( T_F\) est une homothétie et \( R_F^2=T_F=\lambda\mtu\). Mais \( R_F\) est encore une matrice symétrique définie positive, donc nous pouvons considérer une base \( \{ e_1,\ldots, e_d \}\) de \( E_{\lambda}\) qui diagonalise \( R_F\) avec les valeurs propres \( \mu_i\); nous avons donc en même temps
            \begin{subequations}
                \begin{align}
                    R_f^2(e_i)&=\mu_i^2 e_i\\
                    T_F(e_i)&=\lambda e_i,
                \end{align}
            \end{subequations}
            de telle sorte que \( \mu_i^2=\lambda\). Mais les valeurs propres de \( R_F\) sont positives, sont \( \mu_i=\sqrt{\lambda}\) pour tout \( i\). En conclusion \( R_F\) est univoquement déterminé par la donnée de \( T\). Vu que cela est valable pour tous les espaces propres de \( T\) et que ces espaces propres engendrent tout \( E\), l'opérateur \( R\) est déterminé de façon univoque par \( T\).
    \end{subproof}
\end{proof}
Notons que nous n'avons démontré l'unicité qu'au sein des matrices symétriques.

%--------------------------------------------------------------------------------------------------------------------------- 
\subsection{Décomposition polaires : cas réel}
%---------------------------------------------------------------------------------------------------------------------------

Nous nommons \( S^+(n,\eR)\) l'ensemble des matrices \( n\times n\) symétriques réelles définies positives et \( S^{++}(n,\eR)\) le sous-ensemble de \( S^+(n,\eR)\) des matrices strictement définies positives.
\nomenclature[B]{\( S^+(n,\eR)\)}{matrices symétriques définies positives}
\nomenclature[B]{\( S^{++}(n,\eR)\)}{matrices symétriques strictement définies positives}

\begin{lemma}   \label{LemZKJWqIP}
    La fermeture de l'ensemble des matrice symétriques strictement définies positives est l'ensemble des matrices définies positives : \( \overline{ S^{++}(n,\eR) }=S^+(n,\eR)\).
\end{lemma}

\begin{proof}
    Nous savons que \( S^+(n,\eR)\) est fermé et que \( S^{++}(n,\eR)\) y est inclus, donc il suffit de montrer que toute suite de \( S^{++}(n,\eR)\) convergente dans \( \eM(n,\eR)\) a une limite dans \( S^+(n,\eR)\).

    En effet si \( S_k\) est une suite de matrices symétriques convergeant dans \( \eM(n,\eR)\) vers la matrice \( A\), les suites \( (S_k)_{ij}\) et \( (S_k)_{ji}\) des composantes \( ij\) et \( ji\) sont des suites égales, et donc leurs limites sont égales\footnote{Ici nous utilisons le critère de convergence composante par composante et le fait que nous ne sommes pas trop inquiétés par la norme que nous choisissons parce que toutes les normes sont équivalentes par le théorème \ref{ThoNormesEquiv}.}. Donc la limite est symétrique.

    En ce qui concerne le spectre, nous diagonalisons : \( S_k=Q_kD_kQ_k^{-1}\) où les \( D_k\) sont des matrices diagonales remplies de nombres strictement positifs. Vu que \( O(n)\) est compact\footnote{Lemme \ref{LemTLlTAAf}.}, nous avons une sous-suite \( Q_{\varphi(k)}\) convergente : \( Q_{\varphi(k)}\to Q\). Pour chaque \( k\), nous avons
    \begin{equation}
        S_{\varphi(k)}=Q_{\varphi(k)}D_{\varphi(k)}Q^{-1}_{\varphi(k)},
    \end{equation}
    dont la limite existe et vaut \( A\). Étant donné que le membre de droite est le produit de trois facteurs dont deux sont convergents et dont le tout converge, le troisième facteur est également convergent : \( S_{\varphi(k)}\to S\). En passant à la limite :
    \begin{equation}
        A=\lim_{k\to \infty } S_{\varphi(k)}=QDQ^{-1},
    \end{equation}
    et donc le spectre de \( A\) est la limite de ceux des matrices \( D_{\varphi(k)}\). Chacun étant strictement positif, la limite est positive (mais pas spécialement strictement). Donc \( A\in S^+(n,\eR)\).
\end{proof}

\begin{theorem}[Décomposition polaire de matrices symétriques définies positives\cite{JJdQPyK,AABkVai}] \label{ThoLHebUAU}
   En ce qui concerne les matrices inversibles :
   \begin{equation}
       \begin{aligned}
           f\colon O(n,\eR)\times S^{++}(n,\eR)&\to \GL(n,\eR) \\
           (Q,S)&\mapsto SQ 
       \end{aligned}
   \end{equation}
   est un difféomorphisme.

   En ce qui concerne les matrices en général :
   \begin{equation}
       \begin{aligned}
           g\colon O(n,\eR)\times S^+(n,\eR)&\to \eM(n,\eR) \\
           (Q,S)&\mapsto SQ 
       \end{aligned}
   \end{equation}
   est une surjection mais pas une injection.

   De plus les mêmes conclusions tiennent si nous regardons \( (Q,S)\mapsto QS\) au lieu de \( SQ\).
\end{theorem}
\index{groupe!linéaire!décomposition polaire}
\index{endomorphisme!décomposition!polaire}
\index{décomposition!polaire}

%TODO : prouver le difféomorphisme.
%TODO : je crois qu'on doit pouvoir prouver que les éléments de la décomposition polaire sont des polynômes en M.

\begin{proof}
    La preuve qui suit ne démontre qu'un homéomorphisme dans le cas des matrices inversibles.
    \begin{subproof}
        \item[Existence et unicité]

            Si \( M=SQ\), alors \( MM^t=SQQ^tS^t=S^2\), donc \( S\) doit être une racine carré symétrique de la matrice définie positive \( MM^t\). La proposition \ref{PropPEMDqVT} nous dit que ça existe et que c'est unique. Donc \( S\) est univoquement déterminé par \( M\). Maintenant avoir \( Q=MS^{-1}\) est obligatoire (unicité) et fonctionne :
            \begin{equation}
                Q^tQ=(S^{-1})^tM^tMS^{-1}=S^{-1}S^2S^{-1}=\mtu,
            \end{equation}
            donc \( Q\) ainsi défini est orthogonale.

            Ceci prouve l'existence et l'unicité de la décomposition polaire d'une matrice inversible.
        
        \item[Homéomorphisme]

            Le fait que \( f\) soit continue n'est pas un problème : c'est un produit de matrice. Nous devons vérifier que \( f^{-1}\) est continue. Soit une suite convergente \( M_k\to M\) dans \( \GL(n,\eR)\). Si nous nommons \( (Q_k,S_k)\) la décomposition polaire de \( M_k\) et \( (Q,S)\) celle de \( M\), nous devons prouver que \( Q_k\to Q\) et \( S_k\to S\). En effet dans ce cas nous aurions
            \begin{equation}    \label{EqJIkoaJv}
                \lim_{k\to \infty} f^{-1}(M_k)=\lim_{k\to \infty} (Q_k,S_k)=(Q,S)=f^{-1}(M).
            \end{equation}
            
            Étant donné que \( O(n)\) est compact (lemme \ref{LemTLlTAAf}), la suite \( (Q_k)\) admet une sous-suite convergente (Bolzano-Weierstrass, théorème \ref{ThoBWFTXAZNH}) que nous nommons
            \begin{equation}
                Q_{\varphi(k)}\to F\in O(n).
            \end{equation}
            Vu que la suite \( (M_k)\) converge, sa sous-suite converge vers la même limite : \( M_{\varphi(k)}\to M\) et vu que pour tout \( k\) nous avons \( S_k=M_kQ_k^{-1}\),
            \begin{equation}
                S_{\varphi(k)}\to G=MF^{-1}.
            \end{equation}
            Vu que chacune des matrices \( S_{\varphi(k)}\) est symétrique définie positive, la limite est symétrique et semi-définie positive\footnote{Lemme \ref{LemZKJWqIP}}. Donc \( G\in S^+(n,\eR)\cap \GL(n,\eR)\) parce que de plus \( M\) et \( F\) étant inversibles, \( G\) est inversible. En ce qui concerne la sous-suite nous avons
            \begin{equation}
                M_{\varphi(k)}=S_{\varphi(k)}Q_{\varphi(k)}\to GF=M
            \end{equation}
            où \( F\in O(n)\) et \( G\in S^+(n,\eR)\). Par unicité de la décomposition polaire de \( M\) (partie déjà démontrée), nous avons \( G=S\) et \( F=Q\).

            Nous avons prouvé que toute sous-suite convergente de \( Q_k\) a \( Q\) pour limite. Donc la suite elle-même converge\footnote{Proposition \ref{PropHNylIAW}, pas difficile.} vers \( Q\). Donc \( Q_k\to Q\). Du coup vu que \( S_k=M_kQ_k^{-1}\) est un produit de suites convergentes, \( S_k\) converge également, vers \( S\) :  \( S_k\to S\).

            Au final l'application \( f^{-1}\) est bien continue parce que les égalités \eqref{EqJIkoaJv} ont bien lieu.

    \end{subproof}
\end{proof}

%---------------------------------------------------------------------------------------------------------------------------
\subsection{Enveloppe convexe du groupe orthogonal}
%---------------------------------------------------------------------------------------------------------------------------

Le théorème suivant n'est pas indispensablissime parce qu'il est le même que le théorème de la projection sur les espaces de Hilbert\footnote{Théorème \ref{ThoProjOrthuzcYkz}}. Cependant la partie existence est plus simple en se limitant au cas de dimension finie.
\begin{theorem}[Théorème de la projection]  \label{ThoWKwosrH}
    Soit \( E\) un espace vectoriel réel ou complexe de dimension finie, \( x\in E\), et \( C\) un sous ensemble fermé convexe de \(E\).
    \begin{enumerate}
        \item
            Les deux conditions suivantes sur \( y\in E\) sont équivalentes:
    \begin{enumerate}
        \item   \label{zzETsfYCSItemi}
            \( \| x-y \|=\inf\{ \| x-z \|\tq z\in C \}\),
        \item\label{zzETsfYCSItemii}
            pour tout \( z\in C\), \( \Reel\langle x-y, z-y\rangle \leq 0\).
    \end{enumerate}
\item
    Il existe un unique \( y\in E\), noté \( y=\pr_C(x)\) vérifiant ces conditions.
    \end{enumerate}
\end{theorem}
%TODO : il y a sûrement un endroit plus adapté pour mettre ce théorème.

\begin{proof}
    Nous commençons par prouver l'existence et l'unicité d'un élément dans \( C\) vérifiant la première condition. Ensuite nous verrons l'équivalence. 

    \begin{subproof}
        \item[Existence]
        
            Soit \( z_0\in C\) et \( r=\| x-z_0 \|\). La boule fermée \( \overline{ B(x,r) }\) est compacte\footnote{C'est ceci qui ne marche plus en dimension infinie.} et intersecte \( C\). Vu que \( C\) est fermé, l'ensemble \( C'=C\cap\overline{ B(x,r) }\) est compacte. Tous les points qui minimisent la distance entre \( x\) et \( C\) sont dans \( C'\); la fonction 
            \begin{equation}
                \begin{aligned}
                     C'&\to \eR \\
                    z&\mapsto d(x,z) 
                \end{aligned}
            \end{equation}
            est continue sur un compact et donc a un minimum qu'elle atteint. Un point \( P\) réalisant ce minimum prouve l'existence d'un point vérifiant la première condition.

        \item[Unicité]
            Soient \( y_1\) et \( y_2\), deux éléments de \( C\) minimisant la distance avec \( x\), et soit \( d\) ce minimum. Nous avons par l'identité du parallélogramme \eqref{EqYCLtWfJ} que
            \begin{equation}
                \| y_1-y_2 \|^2=-4\left\| \frac{ y_1+y_2-x }{2} \right\|^2+2\| y_1-x \|^2+2\| y_2-x \|^2\leq -4d+2d+2d=0.
            \end{equation}
            Par conséquent \( y_1=y_2\).

        \item[\ref{zzETsfYCSItemi}\( \Rightarrow\) \ref{zzETsfYCSItemii}]

            Soit \( z\in C\) et \( t\in \mathopen] 0 , 1 \mathclose[\); nous notons \( P=\pr_Cx\). Par convexité le point \( z=ty+(1-t)P\) est dans \( C\), et par conséquent,
                \begin{equation}
                    \| x-P \|^2\leq\| x-tz-(1-t)P \|^2=\| (x-P)-t(z-P) \|^2.
                \end{equation}
                Nous sommes dans un cas \( \| a \|^2\leq | a-b |^2\), qui implique \( 2\Reel\langle a, b\rangle \leq \| b \|^2\). Dans notre cas,
                \begin{equation}
                    2\Reel\langle x-P , t(z-P)\rangle \leq t^2\| z-P \|^2.
                \end{equation}
                En divisant par \( t\) et en faisant \( t\to 0\) nous trouvons l'inégalité demandée :
                \begin{equation}
                    2\Reel\langle x-P, z-P\rangle \leq 0.
                \end{equation}
                
        \item[\ref{zzETsfYCSItemii}\( \Rightarrow\) \ref{zzETsfYCSItemi}]

            Soit un point \( P\in C\) vérifiant 
            \begin{equation}
                \Reel\langle x-P, z-P\rangle \leq 0
            \end{equation}
            pour tout \( z\in C\). Alors en notant \( a=x-P\) et \( b=P-z\),
            \begin{equation}
                \begin{aligned}[]
                \| x-z \|^2=\| x-P+P-z \|^2&=\| a+b \|^2\\
                &=\| a \|^2+\| b \|^2+2\Reel\langle a, b\rangle \\
                &=\| a \|^2+\| b \|^2-2\Reel\langle x-P, z-P\rangle \\
                &\geq \| b \|^2,
                \end{aligned}
            \end{equation}
            ce qu'il fallait.

    \end{subproof}

\end{proof}

\begin{definition}
    Soit \( A\) une partie d'un espace vectoriel \( E\). L'\defe{enveloppe convexe}{enveloppe!convexe} de \( A\), notée \( \Conv(A)\)\nomenclature[G]{\( \Conv(A)\)}{enveloppe convexe} est l'intersection de tous les convexes contenant \( A\).
\end{definition}
L'enveloppe convexe est un convexe. En effet soit \( C\) un convexe contenant \( A\) et \( x,y\in\Conv(A)\); alors \( x\) et \( y \) sont dans \( C\) et par conséquent le segment \( [x,y]\) est inclus à \( C\). Ce segment étant inclus à tout convexe contenant \( A\), il est inclus à \( \Conv(A)\).

\begin{theorem}[Carathéodory\cite{KXjFWKA}] \label{ThoJLDjXLe}
    Dans un espace affine de dimension \( n\), l'enveloppe convexe de \( A\) est l'ensemble des barycentres à coefficients positifs ou nuls de familles de \( n+1\) points.
\end{theorem}
\index{barycentre!enveloppe convexe}
\index{espace!vectoriel!dimension}
\index{théorème!Carathéodory}

\begin{proof}
    Soit \( x\in\Conv(A)\); on sait par la proposition \ref{PropYHMTmZX} que \( x\) est barycentre de points de \( A\) avec des coefficients positifs :
    \begin{equation}    \label{EqWJDwOTH}
        x=\sum_{k=1}^p\lambda_kx_k
    \end{equation}
    avec \( \sum_k\lambda_k=1\). Nous supposons que \( p>n+1\) (sinon le théorème est réglé), et nous allons faire une récurrence à l'envers en montrant qu'on peut aussi écrire \( x\) sous forme d'un barycentre de strictement moins de \( p\) points.
    
    Étant donné que \( p-1>n\), la famille \( \{ x+i-x_1 \}_{i=2,\ldots, p}\) est liée et il existe donc \( \alpha_1,\ldots, \alpha_p\in \eR\) tels que \( \sum_{i=2}^p\alpha_i(x_i-x_1)=0\), c'est à dire telle que
    \begin{equation}
        \sum_{i=2}^p\alpha_ix_i=\sum_{i=2}^p\alpha_ix_1.
    \end{equation}
    Nous posons \( \alpha_1=-\sum_{i=2}^p\alpha_1\). Remarquons qu'alors \( \sum_{i=1}^p\alpha_ix_i=0\) parce que
    \begin{equation}
        \sum_{i=1}^p\alpha_ix_i=\alpha_1x_1+\sum_{i=2}^p\alpha_ix_i=\alpha_1x_1+\sum_{i=2}^p\alpha_ix_1=\sum_{i=1}^p\alpha_ix_1=0.
    \end{equation}
    Par conséquent ça ne coûte rien de récrire \eqref{EqWJDwOTH} sous la forme
    \begin{equation}
        x=\sum_{i=1}^p(\lambda_i+t\alpha_i)x_i.
    \end{equation}
    Les \( \alpha_i\) ne sont pas tous nuls, mais leur somme est nulle, donc il y en a au moins un négatif. Nous notons
    \begin{equation}
        \tau=\min\{ -\frac{ \lambda_i }{ \alpha_i }\tq \alpha_i<0 \},
    \end{equation}
    et \( J\) l'ensemble de \( i\) pour lesquels ce minimum est atteint. Nous considérons aussi le nombres \( \mu_i=\lambda_i+\tau\alpha_i\). Plusieurs remarques. 
    \begin{enumerate}
        \item
            Si \( j\in J\), alors \( \mu_j=0\)
        \item
            Si \( \alpha_i>0\) alors \( \mu_i\geq 0\), mais si \( \alpha_i<0\) alors
            \begin{equation}
                \lambda_i+\tau\alpha_i\geq \lambda_i+(-\frac{ \lambda_i }{ \alpha_i })\alpha_i=0m
            \end{equation}
            donc \( \mu_i\geq 0\) quand même.
        \item
            \( \sum_{i=1}^p\mu_i=1\), toujours parce que \( \sum_{i=1}^p\alpha_i=0\).
    \end{enumerate}
    Avec tout ça, nous avons
    \begin{equation}
        \sum_{i\notin J}\mu_ix_i=\sum_{i=1}^p\mu_ix_i=x.
    \end{equation}
    Et voila, nous avons écrit \( x\) comme un barycentre à coefficients positifs de moins de \( p\) éléments parce que \( J\) n'est pas vide.
\end{proof}

\begin{corollary}   \label{CorOFrXzIf}
    Dans un espace affine de dimension finie, l'enveloppe convexe d'un compact est compacte.
\end{corollary}

\begin{proof}
    Soit \( A\) une partie compacte de l'espace vectoriel \( E\), et \( \Conv(A)\) son enveloppe convexe. Nous allons montrer que toute suite dans \( \Conv(A)\) admet une sous-suite convergente en écrivant un point de \( \Conv(A)\) comme le théorème de Carathéodory \ref{ThoJLDjXLe} nous le suggère. Pour cela nous considérons le simplexe
    \begin{equation}
        \Lambda=\left\{  \lambda\in \eR^{n+1}\tq \sum_{k=1}^{n+1}\lambda_k=1\text{ et } \lambda_k\geq 0\forall k   \right\}.    
    \end{equation}
    Montrons en passant que \( \Lambda\) est compact. Si \( \lambda_k\in \Lambda\) est une suite, alors chacun des \( \lambda_k\) est un \( (n+1)\)-uple de nombres dans \( \mathopen[ 0 , 1 \mathclose]\) :
    \begin{equation}
        k\mapsto (\lambda_k)_i
    \end{equation}
    est une suite qui possède une sous-suite convergente. En passant \( n+1\) fois à une sous-suite, nous tombons sur une suite convergente vers \( \lambda\in\Lambda\), grâce à la convergence composante par composante. De plus pour chaque \( k\) nous avons \( \sum_{i=1}^{n+1}(\lambda_k)_i=1\), et en passant à la limite, la somme étant une application continue, \( \sum_{i}\lambda_i=1\).

    Considérons l'application
    \begin{equation}
        \begin{aligned}
            f\colon \Lambda\times A^{n+1}&\to \Conv(A) \\
            (\lambda,x)&\mapsto \sum_{k=1}^{n+1}\lambda_kx_k. 
        \end{aligned}
    \end{equation}
    C'est une application continue parce qu'elle est bilinéaire en dimension finie; son image est contenue dans \( \Conv(A)\) par la proposition \ref{PropSVvAQzi}, et elle est surjective par le théorème de Carathéodory. Bref, \( \Conv(A)=f(\Lambda\times A^{n+1})\) est donc l'image d'un compact par une application continue; elle est donc compacte par le théorème \ref{ThoImCompCotComp}.
\end{proof}
Notons que sans le théorème de Carathéodory, peut être que le nombre de points utiles pour décomposer les différents \( a_k\) n'était pas borné; dans ce cas nous aurions du prendre une infinité de sous-suites et rien n'aurait été sûr.

\begin{definition}
    Sur \( C\) est un ensemble convexe, un point \( x\in C\) est un \defe{point extrémal}{extrémal!point dans un convexe} si \( C\setminus\{ x \}\) est encore convexe.
\end{definition}

\begin{theorem}[\cite{KXjFWKA}] \label{ThoBALmoQw}
    Soit \( E\) un espace euclidien de dimension \( n\geq 1\) et \( \aL(E)\) l'espace des opérateurs linéaires sur \( E\) sur lequel nous considérons la norme subordonnée\footnote{Voir la définition donnée dans l'exemple \ref{ExemdefnormpMrt}.} à celle sur \( E\). L'ensemble des points extrémaux de la boule unité fermée de \( \aL(E)\) est le groupe orthogonal \( O(n,\eR)\).
\end{theorem}
\index{densité!points extrémaux dans \( \aL\)}

\begin{proof}
    Nous notons \( \mB\) la boule unité fermée de \( \aL(E)\). Montrons pour commencer que les éléments de \( O(n)\) sont extrémaux dans \( \mB\). D'abord si \( A\in O(E)\) alors \( \| A \|=1\) parce que \( \| Ax \|=\| x \|\). Supposons maintenant que \( A\) n'est pas extrémal, c'est à dire qu'il est le milieu d'un segment joignant deux points (distincts) de la boule unité de \( \aL(E)\). Soient donc \( T,U\in\mB\) tels que \( A=\frac{ 1 }{2}(T+U)\). Pour tout \( x\in E\) tel que \( \| x \|=1\) nous avons 
    \begin{equation}    \label{EqKTuAIIE}
        1=\| x \|=\| Ax \|=\frac{ 1 }{2}\| Tx+Ux \|\leq \frac{ 1 }{2}\big( \| Tx \|+\| Ux \| \big)\leq\frac{ 1 }{2}\big( \| T \|+| U | \big)\leq 1
    \end{equation}
    Toutes les inégalités sont en réalité des égalités. En particulier nous avons
    \begin{equation}
        \| Tx+Ux \|=\| Tx \|+\| Ux \|,
    \end{equation}
    mais alors nous sommes dans un cas d'égalité dans l'inégalité de Cauchy-Schwartz (théorème \ref{ThoAYfEHG}) et donc il existe \( \lambda\geq 0\) tel que \( Tx=\lambda Ux\). Mais de plus les \sout{inégalité} égalités \eqref{EqKTuAIIE} nous donnent
    \begin{equation}
        \frac{ 1 }{2}\big( \| Tx \|+\| Ux \| \big)=1
    \end{equation}
    alors que nous savons que \( \| Tx \|,\| Ux \|\leq 1\), donc \( \| Tx \|=\| Ux \|=1\). La seule possibilité est d'avoir \( \lambda=1\) et donc que \( U=T\) parce que nous avons \( Tx=Ux\) pour tout \( x\) de norme \( 1\). Au final \( A\) n'est pas le milieu d'un segment dans \( \mB\).

    Nous passons donc à l'inclusion inverse : nous prouvons que les points extrémaux de \( \mB\) sont dans \( O(E)\). Pour cela nous prenons \( U\in\mB\setminus O(E)\) et nous allons montrer que \( U\) n'est pas un point extrémal : nous allons l'écrire comme milieu d'un segment dans \( \mB\).

    Par la seconde partie du théorème de décomposition polaire \ref{ThoLHebUAU}, il existe \( Q\in O(n,\eR)\) et \( S\in S^+(n,\eR)\) tels que \( U=QS\). Nous diagonalisons \( S\) à l'aide de la matrice orthogonale \( P\) :
    \begin{equation}
        S=PDP^{-1}
    \end{equation}
    avec \( D=\diag(\lambda_i)\). En termes de normes, nous avons
    \begin{equation}
        \| U \|=\| S \|=\| S \|.
    \end{equation}
    En effet vu que \( Q\) est orthogonale, \( \| Ux \|=\| QSx \|=\| Sx \|\) pour tout \( x\), donc \( \| U \|=\| S \|\). De plus pour tout \( x\) nous avons
    \begin{equation}
        \| Sx \|=\| PDP^{-1} x \|=\| DP^{-1}x \|.
    \end{equation}
    Étant donné que \( P^{-1}\) est une bijection, le supremum des \( \| Sx \|\) sera le même que celui des \( \| Dx \|\) et donc \( \| S \|=\| D \|\). Étant donné que par définition \( \| U \|\leq 1\), nous avons aussi \( \| D \|\leq 1\) et donc \( 0\leq\lambda_i\leq 1\) (pour rappel, les valeurs propres de \( D\) sont positives ou nulles parce que \( S\) est ainsi). 

    Comme \( U\notin O(E)\), au moins une des valeurs propres n'est pas \( 1\), supposons que ce soit \( \lambda_1\). Alors nous avons \( \alpha,\beta\in\mathopen[ -1 , 1 \mathclose]\) avec \( -1\leq \alpha<\beta\leq 1\) et \( \lambda_1=\frac{ 1 }{2}(\alpha+\beta)\). Nous posons alors
    \begin{subequations}
        \begin{align}
            D_1=\diag(\alpha,\lambda_2,\ldots, \lambda_n)\\
            D_2=\diag(\beta,\lambda_2,\ldots, \lambda_n).
        \end{align}
    \end{subequations}
    Nous avons bien \( D_1\neq D2\) et \( D_1+ D_2=D\). Par conséquent
    \begin{equation}
        U=\frac{ 1 }{2}\big( QPD_1P^{-1}+QPD_2P^{-1} \big)
    \end{equation}
    avec \( QPD_1P^{-1}\neq QPD_2^{-1}\). La matrice \( U \) est donc le milieu d'un segment. Reste à montrer que ce segment est dans \( \mB\). Pour ce faire, prenons \( x\in E\) et calculons :
    \begin{equation}
        \| QPD_iP^{-1}x \|=\| D_iP^{-1}x \|\leq\| P^{-1}x \|=\| x \|
    \end{equation}
    parce que \( \| D_i \|\leq 1\) et \( P^{-1}\) est orthogonale. Au final la norme de \( QPD_iP\) est plus petite que \( 1\) et donc \( U\) est bien le milieu d'un segment dans \( \mB\), et donc non extrémal.
\end{proof}

\begin{theorem}[\cite{NHXUsTa}] \label{ThoVBzqUpy}
    L'enveloppe convexe de \( O(n)\) dans \( \eM_n(\eR)\) est la boule unité pour la norme induite de \( \| . \|_2\) sur \( \eR^n\).
\end{theorem}
\index{convexité!enveloppe de $O(n)$}
\index{groupe!linéaire!enveloppe convexe de $\Omega(n)$}

\begin{proof}
    Nous notons \( \mB\) la boule unité fermée de \( \eM(n,\eR)\) et \( \Conv\big( O(n,\eR) \big)\) l'enveloppe convexe de \( O(n,\eR)\). Vu que \( \mB\) est convexe nous avons \( \Conv\big( O(n) \big)\subset\mB\).


    Maintenant nous devons prouver l'inclusion inverse. Pour ce faire nous supposons avoir un élément \( A\in \mB\setminus\Conv\big( O(n) \big)\) et nous allons dériver une contradiction.
    
    Remarquons que \( O(n)\) est compact par le lemme \ref{LemTLlTAAf} et que par conséquent \( \Conv(O(n))\) est compacte par le corollaire \ref{CorOFrXzIf} et donc fermée. Nous considérons un produit scalaire \( (X,Y)\mapsto X\cdot Y\) sur \( \eM\). Vu que \( \Conv\big( O(n) \big)\) est un fermé convexe nous pouvons considérer la projection\footnote{Le théorème de projection : théorème \ref{ThoWKwosrH}.} sur \( \Conv(A)\) relativement au produit scalaire choisis.

    Nous notons \( P=\pr_{\Conv\big( O(n) \big)}(A)\). En vertu du théorème de projection, nous avons
    \begin{equation}    \label{EqYSisLTL}
        (A-P)\cdot (M-P)\leq 0
    \end{equation}
    pour tout \( M\in\Conv O(n)\). Notons \( B=A-P\) pour alléger les notations. L'équation \eqref{EqYSisLTL} s'écrit
    \begin{equation}    \label{EqQDLZqXQ}
        B\cdot M\leq B\cdot P.
    \end{equation}
    D'autre par vu que \( B \neq 0\) nous avons \( B\cdot B> 0\), c'est à dire \( B\cdot (A-P)>0\) et donc
    \begin{equation}
        B\cdot A>B\cdot P.
    \end{equation}
    En combinant avec \eqref{EqQDLZqXQ},
    \begin{equation}        \label{EqIQNlwql}
        B\cdot M\leq B\cdot P<B\cdot A.
    \end{equation}
    Nous utilisons maintenant la décomposition polaire, théorème \ref{ThoLHebUAU}, pour écrire \( B=QS\) avec \( Q\in O(n)\) et \( S\in S^+(n,\eR)\). Vu que l'inégalité \eqref{EqIQNlwql} tient pour tout \( M\in\Conv(O(n))\), elle tient en particulier pour \( Q\in O(n)\). Donc
    \begin{equation}
        B\cdot Q=B\cdot A.
    \end{equation}
    Nous nous particularisons à présent au produit scalaire \( (X,Y)\mapsto\tr(X^tY)\) de la proposition \ref{PropMAQoKAg}. D'abord
    \begin{equation}    \label{EaHVxWdau}
        B\cdot Q=\tr(B^tQ)=\tr(S^tQ^tQ)=\tr(S^t)=\tr(S),
    \end{equation}
    et ensuite l'inégalité \eqref{EaHVxWdau} devient
    \begin{equation}
        \tr(S)<B\cdot A=\tr(S^tQ^tA).
    \end{equation}
    Nous choisissons une basse \( \{ e_i \}\) diagonalisant \( S\) : \( Se_i=\lambda_ie_i\) vérifiant automatiquement \( \lambda_i>0\) parce que \( S\) est semi-définie positive. Alors
    \begin{subequations}
        \begin{align}
            \tr(S)&<\tr(S^tQ^tA)\\
            &=\sum_i\langle S^tQ^tAe_i, e_i\rangle \\
            &=\sum_i\langle Ae_i, QSe_i\rangle \\
            &\leq \sum_i \| Ae_i \| | \lambda_i | \underbrace{\| Qe_i \|}_{=1} \\
            &\leq \sum_i\lambda_i   & A\in\mB\Rightarrow\| Ae_i \|\leq 1\\
            &=\tr(S).
        \end{align}
    \end{subequations}
    Il faut noter que la première inégalité est stricte, et donc nous avons une contradiction.
\end{proof}

%+++++++++++++++++++++++++++++++++++++++++++++++++++++++++++++++++++++++++++++++++++++++++++++++++++++++++++++++++++++++++++
\section{Sous espaces caractéristiques}
%+++++++++++++++++++++++++++++++++++++++++++++++++++++++++++++++++++++++++++++++++++++++++++++++++++++++++++++++++++++++++++

% TODO : lire le blog de Pierre Bernard; en particulier celle-ci : http://allken-bernard.org/pierre/weblog/?p=2299

Sources : \cite{MneimneReduct} et \wikipedia{fr}{Décomposition_de_Dunford}{divers articles sur Wikipédia}.
%TODO : citer mieux Wikipédia.

Lorsqu'un opérateur n'est pas diagonalisable, les valeurs propres jouent quand même un rôle important.

Soit \( E\) un \( \eK\)-espace vectoriel et \( f\in\End(E)\). Pour \( \lambda\in \eK\) nous définissons
\begin{equation}
    F_{\lambda}(f)=\{ v\in E\tq (f-\lambda\mtu)^nv=0, n\in\eN \}.
\end{equation}
C'est l'ensemble de nilpotence de l'opérateur \( f-\lambda\mtu\) et c'est un \defe{sous-espace caractéristique}{sous-espace!caractéristique} de \( f\).

\begin{lemma}
    L'ensemble \( F_{\lambda}(f)\) est non vide si et seulement si \( \lambda\) est une valeur propre de \( f\). L'espace \( F_{\lambda}(f)\) est invariant sous \( f\).
\end{lemma}

\begin{proof}
    Si \( F_{\lambda}(f)\) est non vide, nous considérons \( v\in F_{\lambda}(f)\) et \( n\) le plus petit entier non nul tel que \( (f-\lambda)^nv=0\). Alors \( (f-\lambda)^{n-1}v\) est un vecteur propre de \( f\) pour la valeur propre \( \lambda\). Inversement si \( v\) est une valeur propre de \( f\) pour la valeur propre \( \lambda\), alors \( v\in F_{\lambda}(f)\).

    En ce qui concerne l'invariance, remarquons que \( f\) commute avec \( f-\lambda\mtu\). Si \( x\in F_{\lambda}(f)\) il existe \( n\) tel que \( (f-\lambda\mtu)^nx=0\). Nous avons aussi
    \begin{equation}
        (f-\lambda\mtu)^nf(x)=f\big( (f-\lambda\mtu)^nx \big)=0,
    \end{equation}
    par conséquent \( f(x)\in F_{\lambda}(f)\).
\end{proof}

\begin{remark}
    Toute matrice sur \( \eC\) n'est pas diagonalisable. Considérons en effet l'endomorphisme \( f\) donné par la matrice
    \begin{equation}
        \begin{pmatrix}
            a&    \alpha    &   \beta    \\
            0    &   a    &   \gamma    \\
            0    &   0    &   b
        \end{pmatrix}
    \end{equation}
    où \( a\neq b\), \( \alpha\neq 0\), \( \beta\) et \( \gamma\) sont des nombres complexes quelconques.
    Son polynôme caractéristique est 
    \begin{equation}
        \chi_f(\lambda)=(a-\lambda)^2(b-\lambda)
    \end{equation}
    de telle façon à ce que les valeurs propres soient \( a\) et \( b\). Nous trouvons les vecteurs propres pour la valeur \( a\) en résolvant
    \begin{equation}
        \begin{pmatrix}
            a    &   \alpha    &   \beta    \\
            0    &   a    &   \gamma    \\
            0    &   0    &   b
        \end{pmatrix}\begin{pmatrix}
            x    \\ 
            y    \\ 
            z    
        \end{pmatrix}=\begin{pmatrix}
            ax    \\ 
            ay    \\ 
            az    
        \end{pmatrix}.
    \end{equation}
    L'espace propre \( E_a(f)\) est réduit à une seule dimension générée par \( (1,0,0)\). De la même façon l'espace propre correspondant à la valeur propre \( b\) est donné par 
    \begin{equation}
        \begin{pmatrix}
            \frac{1}{ b-a }\left( \beta+\frac{ \alpha\gamma }{ b-a } \right)    \\ 
            \frac{ \gamma }{ b-a }    \\ 
            1    
        \end{pmatrix}.
    \end{equation}
    Il n'y a donc pas trois vecteurs propres linéairement indépendants, et l'opérateur \( f\) n'est pas diagonalisable.

    Par contre nous pouvons voir que
    \begin{equation}
        (f-\alpha\mtu)^2\begin{pmatrix}
             0   \\ 
            1    \\ 
            0    
        \end{pmatrix}=
        \begin{pmatrix}
            a    &   \alpha    &   \beta    \\
            0    &   a    &   \gamma    \\
            0    &   0    &   b
        \end{pmatrix}
        \begin{pmatrix}
            \alpha    \\ 
            0    \\ 
            0    
        \end{pmatrix}-\begin{pmatrix}
            a\alpha    \\ 
            0    \\ 
            0    
        \end{pmatrix}=\begin{pmatrix}
            0    \\ 
            0    \\ 
            0    
        \end{pmatrix},
    \end{equation}
    de telle sorte que le vecteur \( (0,1,0)\) soit également dans l'espace caractéristique \( F_a(f)\).

    Dans cet exemple, la multiplicité algébrique de la racine \( a\) du polynôme caractéristique vaut \( 2\) tandis que sa multiplicité géométrique vaut seulement \( 1\).
\end{remark}

%TODO : Je crois qu'on peut remplacer l'hypothèse de corps algébriquement clos par le polynôme caractéristique scindé.
\begin{theorem}[Théorème spectral, décomposition primaire]\index{théorème!spectral}     \label{ThoSpectraluRMLok}
    Soit \( E\) espace vectoriel de dimension finie sur le corps algébriquement clos \( \eK\) et \( f\in\End(E)\). Alors
    \begin{equation}
        E=F_{\lambda_1}(f)\oplus\ldots\oplus F_{\lambda_k}(f)
    \end{equation}
    où la somme est sur les valeurs propres distinctes de \( f\).

    Les projecteurs sur les espaces caractéristique forment un système complet et orthogonal.
\end{theorem}
\index{décomposition!primaire}
\index{décomposition!spectrale}
\index{décomposition!sous-espaces caractéristiques}

\begin{proof}
    Soit \( P\) le polynôme caractéristique de \( f\) et une décomposition
    \begin{equation}
        P=(f-\lambda_1)^{\alpha_1}\ldots(f-\lambda_r)^{\alpha_r}
    \end{equation}
    en facteurs irréductibles. La le théorème de noyaux (\ref{ThoDecompNoyayzzMWod}) nous avons
    \begin{equation}        \label{EqDeFVSaYv}
        E=\ker(f-\lambda_1)^{\alpha_1}\oplus\ldots\oplus\ker(f-\lambda_r)^{\alpha_r}.
    \end{equation}
    Les projecteurs sont des polynômes en \( f\) et forment un système orthogonal. Il nous reste à prouver que \( \ker(f-\lambda_i)^{\alpha_i}=F_{\lambda_i}(f)\). L'inclusion
    \begin{equation}    \label{EqzmNxPi}
        \ker(f-\lambda_i)^{\alpha_i}\subset F_{\lambda_i}(f)
    \end{equation}
    est évidente. Nous devons montrer l'inclusion inverse. Prouvons que la somme des \( F_{\lambda_i}(f)\) est directe. Si \( v\in F_{\lambda_i}(f)\cap F_{\lambda_j}(f)\), alors il existe \( v_1=(f-\lambda_i)^nv\neq 0\) avec \( (f-\lambda_i)v_1=0\). Étant donné que \( (f-\lambda_i)\) commute avec \( (f-\lambda_j)\), ce \( v_1\) est encore dans \( F_{\lambda_j}(f)\) et par conséquent il existe \( w=(f-\lambda_j)^mv_1\) non nul tel que 
    \begin{subequations}
        \begin{numcases}{}
            (f-\lambda_i)w=0\\
            (f-\lambda_j)w=0.
        \end{numcases}
    \end{subequations}
    Ce \( w\) serait donc un vecteur propre simultané pour les valeurs propres \( \lambda_i\) et \( \lambda_j\), ce qui est impossible parce que les espaces propres sont linéairement indépendants. Les espaces \( F_{\lambda_i}\) sont donc en somme directe et
    \begin{equation}
        \sum_i\dim F_{\lambda_i}(f)\leq \dim E.
    \end{equation}
    En tenant compte de l'inclusion \eqref{EqzmNxPi} nous avons même
    \begin{equation}
        \dim E=\sum_i\dim\ker(f-\lambda_i)^{\alpha_i}\leq\sum_i F_{\lambda_i}(f)\leq \dim E.
    \end{equation}
    Par conséquent nous avons \( \dim\ker(f-\lambda_i)^{\alpha_i}=\dim F_{\lambda_i}(f)\) et l'égalité des deux espaces.
\end{proof}

Le théorème suivant généralise le théorème de diagonalisabilité \ref{ThoDigLEQEXR} au cas où le polynôme minimum est seulement scindé.

\begin{probleme}
    \begin{enumerate}
\item 
    Dans le cas où le corps n'est pas algébriquement clos, il paraît qu'il faut remplacer «diagonalisable» par «semi-simple».
    \end{enumerate}
\end{probleme}
%TODO : peut-être qu'il y a la réponse dans http://www.math.jussieu.fr/~romagny/agreg/dvt/endom_semi_simples.pdf

\begin{definition}
    Un endomorphisme d'un espace vectoriel est \defe{semi-simple}{semi-simple!endomorphisme} si tout sous-espace stable par \( u\) possède un supplémentaire stable.
\end{definition}
Si l'espace vectoriel est sur un corps algébriquement clos, alors les endomorphismes semi-simples sont les endomorphismes diagonaux.


%TODO : Je crois qu'on peut remplacer l'hypothèse de corps algébriquement clos par le polynôme caractéristique scindé.
\begin{theorem}[Décomposition de Dunford]\index{décomposition!Dunford}\index{Dunford!décomposition} \label{ThoRURcpW}
    Soit \( E\) un espace vectoriel sur le corps algébriquement clos \( \eK\) et \( u\in\End(E)\) un endomorphisme de \( E\). 
    
    \begin{enumerate}
        \item
            
            L'endomorphisme \( u\) se décompose de façon unique sous la forme
            \begin{equation}
                u=s+n
            \end{equation}
            où \( s\) est diagonalisable, \( n\) est nilpotent et \( [s,n]=0\).
        \item
            Les endomorphismes \( s\) et \( n\) sont des polynômes en \( u\) et commutent avec \( u\).
    \end{enumerate}
\end{theorem}
\index{réduction!d'endomorphisme}
\index{endomorphisme!sous-espace stable}
\index{polynôme!d'endomorphisme!décomposition de Dunford}
\index{endomorphisme!diagonalisable!Dunford}
\index{endomorphisme!nilpotent!Dunford}
%TODO : comprendre comment on calcule des exponentielles de matrices avec Dunford.

\begin{proof}
    Le théorème spectral \ref{ThoSpectraluRMLok} nous indique que
    \begin{equation}
        E=\bigoplus_iF_{\lambda_i}(f).
    \end{equation}
    Nous considérons l'endomorphisme \( s\) de \( E\) qui consiste à dilater d'un facteur \( \lambda\) l'espace caractéristique \( F_{\lambda}(f)\) :
    \begin{equation}
        s=\sum_i\lambda_ip_i
    \end{equation}
    où \( p_i\colon E\to F_{\lambda_i}(u)\) est la projection de \( E\) sur \( F_{\lambda_i}(u)\).

    Nous allons prouver que \( [s,f]=0\) et \( n=f-s\) est nilpotent. Cela impliquera que \( [s,n]=0\).

    Si \( x\in F_{\lambda}(f)\), alors nous avons \( sf(x)=\lambda f(x)\) parce que \( f(x)\in F_{\lambda}(f)\) tandis que \( fs(x)=f(\lambda x)=\lambda f(x)\). Par conséquent \( f\) commute avec \( s\).

    Pour montrer que \( f-s\) est nilpotent, nous en considérons la restriction
    \begin{equation}
        f-s\colon F_{\lambda}(f)\to F_{\lambda}(f).
    \end{equation}
    Cet opérateur est égal à \( f-\lambda\mtu\) et est par conséquent nilpotent.

    Prouvons à présent l'unicité. Soit \( u=s'+n'\) une autre décomposition qui satisfait aux conditions : \( s'\) est diagonalisable, \( n'\) est nilpotent et \( [n',s']=0\). Commençons par prouver que \( s'\) et \( n'\) commutent avec \( u\). En multipliant \( u=s'+n'\) par \( s'\) nous avons
    \begin{equation}
        s'u=s'^2+s'n'=s'^2+n's'=(s'+n')s'=us',
    \end{equation}
    par conséquent \( [u,s']=0\). Nous faisons la même chose avec \( n'\) pour trouver \( [u,n']=0\). Notons que pour obtenir ce résultat nous avons utilisé le fait que \( n'\) et \( s'\) commutent, mais pas leur propriétés de nilpotence et de diagonalisibilité.
    
    
    Si \( s'+n'=s+n\) est une autre décomposition, \( s'\) et \( n'\) commutent avec \( u\), et par conséquent avec tous les polynômes en \( u\). Ils commutent en particulier avec \( n\) et \( s\). Les endomorphismes \( s\) et \( s'\) sont alors deux endomorphismes diagonalisables qui commutent. Par la proposition \ref{PropGqhAMei}, ils sont simultanément diagonalisables. Dans la base de simultanée diagonalisation, la matrice de l'opérateur \( s'-s=n-n'\) est donc diagonale. Mais \( n-n'\) est également nilpotent, en effet si \( A\) et \( B\) sont deux opérateurs nilpotents,
    \begin{equation}
        (A+B)^n=\sum_{k=0}^n\binom{k}{n}A^kB^{n-k}.
    \end{equation}
    Si \( n\) est assez grand, au moins un parmi \( A^k\) ou \( B^{n-k}\) est nul.

    Maintenant que \( n-n'\) est diagonal et nilpotent, il est nul et \( n=n'\). Nous avons alors immédiatement aussi \( s=s'\).

\end{proof}

%---------------------------------------------------------------------------------------------------------------------------
\subsection{Valeurs singulières}
%---------------------------------------------------------------------------------------------------------------------------

\begin{definition}
    Soit \( M\) une matrice \( m\times n\) sur \( \eK\) (\( \eK\) est \( \eR\) ou \( \eC\)). Un nombre réel \( \sigma\) est une \defe{valeur singulière}{valeur!singulière} de \( M\) si il existent des vecteurs unitaires \( u\in \eK^m\), \( v\in \eK^n\) tels que
    \begin{subequations}
        \begin{align}
            Mv&=\sigma u\\
            M^*u&=\sigma v.
        \end{align}
    \end{subequations}
\end{definition}

\begin{theorem}[Décomposition en valeurs singulières]
    Soit \( M\in \eM(m\times n,\eK)\) où \( \eK=\eR,\eC\). Alors \( M\) se décompose en
    \begin{equation}
        M=ADB
    \end{equation}
    où
    il existe deux matrices unitaires \( A\in \gU(m\times m)\), \( B\in \gU(n\times n)\) et une matrice (pseudo)diagonale \( D\in \eM(m\times n)\) tels que
    \begin{enumerate}
        \item 
            \( A\in\gU(m\times m)\), \( B\in\gU(n\times n)\) sont deux matrices unitaires;,
        \item
            \( D\) est (pseudo)diagonale,
        \item
            les éléments diagonaux de \( D\) sont les valeurs singulières de \( M\),
        \item
            le nombre d'éléments non nuls sur la diagonale de \( D\) est le rang de \( M\).
    \end{enumerate}
\end{theorem}

\begin{corollary}
    Soit \( M\in \eM(n,\eC)\). Il existe un isomorphisme \( f\colon \eC^n\to \eC^n\) tel que \( fM\) soit autoadjoint.
\end{corollary}

\begin{proof}
    Si \( M=ADB\) est la décomposition de \( M\) en valeurs singulières, alors nous pouvons prendre \( f=\overline{ B }^tA^{-1}\) qui est une matrice inversible. Pour la vérification que ce \( f\) répond bien à la question, ne pas oublier que \( D\) est réelle, même si \( M\) ne l'est pas.
\end{proof}
