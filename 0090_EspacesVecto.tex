% This is part of Mes notes de mathématique
% Copyright (c) 2011-2014
%   Laurent Claessens, Carlotta Donadello
% See the file fdl-1.3.txt for copying conditions.


%+++++++++++++++++++++++++++++++++++++++++++++++++++++++++++++++++++++++++++++++++++++++++++++++++++++++++++++++++++++++++++ 
\section{Calcul différentiel dans un espace vectoriel normé}
%+++++++++++++++++++++++++++++++++++++++++++++++++++++++++++++++++++++++++++++++++++++++++++++++++++++++++++++++++++++++++++
\label{SecLStKEmc}

Nous développons dans cette section le concept de différentielle de fonction de et vers des espaces vectoriels normés au lieu de \( \eR^n\).

%--------------------------------------------------------------------------------------------------------------------------- 
\subsection{Différentielle}
%---------------------------------------------------------------------------------------------------------------------------

\begin{definition}  \label{DefKZXtcIT}
    Soit une application \( f\colon E\to F\) entre deux espaces de Banach. Nous disons que \( f\) est \defe{différentiable}{différentiable!dans un Banach} en \( a\in E\) si il existe une application linéaire continue\footnote{Nous demandons bien que le candidat différentielle soit continue; en dimension infinie ce n'est pas le cas de toutes les fonctions linéaires, comme le montre l'exemple \ref{ExHKsIelG}.} \( T\colon E\to F\) telle que
    \begin{equation}\label{EqIQuRGmO}
        \lim_{h\to 0} \frac{ f(a+h)-f(a)-T(h) }{ \| h \| }=0.
    \end{equation}
\end{definition}

L'application \( a\mapsto T\) est la \defe{différentielle}{différentielle} de \( f\) au point \( a\) et est notée \( df_a\). L'application différentielle
\begin{equation}
    \begin{aligned}
        df\colon E&\to \aL(E,F) \\
        a&\mapsto df_a 
    \end{aligned}
\end{equation}
est également très importante. 

\begin{definition}      \label{DefJYBZooPTsfZx}
Une application \( f\colon E\to F\) est de \defe{classe \( C^1\)}{classe $C^1$} lorsque l'application différentielle \( df\colon E\to \aL(E,F)\) est continue. Voir aussi les définitions \ref{DefPNjMGqy} pour les applications de classe \( C^k\).
\end{definition}

On fixe maintenant une définition largement utilisée dans la suite. 
\begin{definition}      \label{DefAQIQooYqZdya}
	 Soient $U$ et $V$, deux ouverts d'un espace vectoriel normé. Une application $f$ de $U$ dans $V$ est un \defe{difféomorphisme}{difféomorphisme} si elle est bijective, différentiable et dont l'inverse $f^{-1}:V\to U $ est aussi différentiable. 
\end{definition}

\begin{remark}
	Il n'est pas possible d'avoir une application inversible d'un ouvert de $\eR^m$ vers un ouvert de $\eR^n$ si $m\neq n$. Il n'y a donc pas de notion de difféomorphismes entre ouverts de dimensions différentes.
\end{remark}

\begin{remark}      \label{RemATQVooDnZBbs}
    L'application norme étant continue, le critère du théorème \ref{ThoWeirstrassRn} est en réalité assez général. Par exemple à partir d'une application différentiable\footnote{Définition \ref{DefKZXtcIT}.} \( f\colon X\to Y\)  nous pouvons considérer la fonction réelle
    \begin{equation}
        a\mapsto \|  df_a   \|
    \end{equation}
    où la norme est la norme opérateur\footnote{Définition \ref{DefNFYUooBZCPTr}.}. Si \( f\) est de classe \( C^1\) alors cette application est continue et donc bornée sur un compact \( K\) de \( X\).
\end{remark}


%--------------------------------------------------------------------------------------------------------------------------- 
\subsection{(non ?) Différentiabilité des applications linéaires}
%---------------------------------------------------------------------------------------------------------------------------

Si \( E\) et \( F\) sont deux espaces vectoriels nous notons \( \aL(E,F)\)\nomenclature[Y]{\( \aL(E,F)\)}{Les applications linéaires de \( E\) vers \( F\)} l'ensemble des applications linéaires de \( E\) vers \( F\) et \( \cL(E,F)\)\nomenclature[Y]{\( \cL\)}{Les applications linéaires continues de \( E\) vers \( F\)} l'ensemble des applications linéaires continues de \( E\) vers \( F\). Ces espaces seront bien entendu, sauf mention du contraire, toujours munis de la norme opérateur de l'exemple \ref{ExemdefnormpMrt}. 

\begin{example}[Une application linéaire non continue]  \label{ExHKsIelG}
    Soit \( V\) l'espace vectoriel normé des suites \emph{finies} de réels muni de la norme usuelle $\| c \|=\sqrt{\sum_{i=0}^{\infty}| c_i |^2}$ où la somme est finie. Nous nommons \( \{ e_k \}_{k\in \eN}\) la base usuelle de cet espace, et nous considérons l'opérateur \( f\colon V\to V\) donnée par \( f(e_k)=ke_k\). C'est évidemment linéaire, mais ce n'est pas continu en zéro. En effet la suite \( u_k=e_k/k\) converge vers \( 0\) alors que \( f(u_k)=e_k\) ne converge pas.
\end{example}

Cet exemple aurait pu également être donnée dans un espace de Hilbert, mais il aurait fallu parler de domaine.
%TODO : le faire, et regarder si Hilbet n'est pas la complétion de cet espace. Référencer à l'endroit qui définit l'espace vectoriel librement engendré. Ici ce serait par N.

%TODO : dire qu'une application bilinéaire sur RxR n'est pas une application linéaire sur R^2

\begin{example}[Une autre application linéaire non continue\cite{GTkeGni}]
    En dimension infinie, une application linéaire n'est pas toujours continue. Soit \( E\) l'espace des polynômes à coefficients réels sur \( \mathopen[ 0 , 1 \mathclose]\) muni de la norme uniforme. L'application de dérivation \( \varphi\colon E\to E\), \( \varphi(P)=P'\) n'est pas continue.

    Pour la voir nous considérons la suite \( P_n=\frac{1}{ n }X^n\). D'une part nous avons \( P_n\to 0\) dans \( E\) parce que \( P_n(x)=\frac{ x^n }{ n }\) avec \( x\in \mathopen[ 0 , 1 \mathclose]\). Mais en même temps nous avons \( \varphi(P_n)=X^{n-1}\) et donc \( \| \varphi(P_n) \|=1\).

    Nous n'avons donc pas \( \lim_{n\to \infty} \varphi(P_n)=\varphi(\lim_{n\to \infty} P_n)\) et l'application \( \varphi\) n'est pas continue en \( 0\). Elle n'est donc continue nulle part par linéarité.

    Nous avons utilisé le critère séquentiel de la continuité, voir la définition \ref{DefENioICV} et la proposition \ref{PropFnContParSuite}.
\end{example}

Nous avons cependant le résultat suivant.
\begin{proposition}[\cite{GKPYTMb} Continue si et seulement si bornée] \label{PropmEJjLE}
    Soient \( E\) et \( F\) des espaces vectoriels normés, et \( u\colon E\to F\) une application linéaire. Alors \( u\) est bornée\footnote{Au sens où \( \| u \|<\infty\) pour la norme opérateur.} si et seulement si elle est continue.
\end{proposition}
\index{opérateur!linéaire!borné}
\index{application!linéaire!bornée}

\begin{proof}
    Nous commençons par supposer que \( u\) est bornée. Pour tout \( x,y\in E\) nous avons
    \begin{equation}
        \| u(x)-u(y) \|=\| u(x-y) \|\leq \| u \|\| x-y \|.
    \end{equation}
    En particulier si \( x_n\stackrel{E}{\longrightarrow}x\) alors
    \begin{equation}
        0\leq \| u(x_n)-u(x) \|\leq \| u \|\| x-x_n \|\to 0
    \end{equation}
    et \( u\) est continue en vertu de la caractérisation séquentielle de la continuité, proposition \ref{PropFnContParSuite}.

Supposons maintenant que \( \| u \|\) ne soit pas borné, c'est à dire que l'ensemble \( \{ \| u(x) \|\tq \| x \|=1 \}\) ne soit pas borné. Alors pour tout \( k\geq 1\) il existe \( x_k\in B(0,1)\) tel que \( \| u(x_k) \|>k\). La suite \( x_k/k\) tend vers zéro parce que \( \| x_k \|=1\), mais \( \| u(x_k) \|\geq 1\) pour tout \( k\). Cela montre que \( u\) n'est pas continue.
\end{proof}

\begin{remark}  \label{RemOAXNooSMTDuN}
Cette proposition permet de retrouver l'exemple \ref{ExHKsIelG} plus simplement. Si \( \{ e_k \}_{k\in \eN}\) est une base d'un espace vectoriel normé formée de vecteurs de norme \( 1\), alors l'opérateur linéaire donné par \( u(e_k)=ke_k\) n'est pas borné et donc pas continu.
\end{remark}

C'est également ce résultat qui montre que le produit scalaire est continu sur un espace de Hilbert par exemple.

\begin{lemma}
    Si \( f\) est linéaire et différentiable alors \( df_a(u)=f(u)\).
\end{lemma}

\begin{proof}
    En effet la linéarité de \( f\) donne
    \begin{equation}
        f(a+h)-f(a)-f(h)=0
    \end{equation}
    pour tout \( h\). Donc la limite \eqref{EqIQuRGmO} est nulle. Les applications linéaires non continues ne sont donc pas différentiables.
\end{proof}

\begin{lemma}   \label{LemLLvgPQW}
    Une application linéaire continue est de classe \(  C^{\infty}\).
\end{lemma}

\begin{proof}
    Soit \( a\in E\). Étant donné que \( f\) est linéaire et continue, elle est différentiable et
    \begin{equation}
        \begin{aligned}
            df\colon E&\to \cL(E,F) \\
            a&\mapsto f 
        \end{aligned}
    \end{equation}
    est une fonction constante et en particulier continue; nous avons donc \( f\in C^1\). Pour la différentielle seconde nous avons \( d(df)_a=0\) parce que \( df(a+h)-df(a)=f-f=0\). Toutes les différentielles suivantes sont nulles.
\end{proof}

%--------------------------------------------------------------------------------------------------------------------------- 
\subsection{Dérivation en chaine et formule de Leibnitz}
%---------------------------------------------------------------------------------------------------------------------------

\begin{proposition} \label{PropOYtgIua}
    Soient \( f_i\colon U\to F_i\), des fonctions de classe \( C^r\) où \( U\) est ouvert dans l'espace vectoriel normé \( E\) et les \( F_i\) sont des espaces vectoriels normés. Alors l'application
    \begin{equation}
        \begin{aligned}
        f=f_1\times \cdots\times f_n\colon U&\to F_1\times \cdots\times F_n \\
    x&\mapsto \big( f_1(x),\ldots, f_n(x) \big) 
        \end{aligned}
    \end{equation}
    est de classe \( C^r\) et
    \begin{equation}
    d^rf=d^rf_1\times\ldots d^rf_n.
    \end{equation}
\end{proposition}

\begin{proof}
    Soit \( x\in U\) et \( h\in E\). La différentiabilité des fonctions \( f_i\) donne
    \begin{equation}
        f_i(x+h)=f_i(x)+(df_i)_x(h)+\alpha_i(h)
    \end{equation}
    avec \( \lim_{h\to 0} \alpha_i(h)/\| h \|=0\). Par conséquent
    \begin{equation}
        f(x+h)=\big( \ldots, f_i(x)+(df_i)_x(h)+\alpha_i(h),\ldots \big)= \big( \ldots,f_i(x),\ldots \big)+ \big( \ldots,(df_i)_x(h),\ldots \big)+ \big( \ldots,\alpha_i(h),\ldots \big).
    \end{equation}
    Mais la définition \ref{DefFAJgTCE} de la norme dans un espace produit donne
    \begin{equation}
        \lim_{h\to 0} \frac{ \| \big( \alpha_1(h),\ldots, \alpha_n(h) \big) \| }{ \| h \| }=0,
    \end{equation}
    ce qui nous permet de noter \( \alpha(h)=\big( \alpha_1(h),\ldots, \alpha_n(h) \big)\) et avoir \( \lim_{h\to 0} \alpha(h)/\| h \|=0\). Avec tout ça nous avons bien
    \begin{equation}
        f(x+h)=f(x)+\big( (df_1)_x(h)+\ldots +(df_n)_x(h) \big)+\alpha(h),
    \end{equation}
    ce qui signifie que \( f\) est différentiable et
    \begin{equation}
        df_x=\big( df_1,\ldots, df_n \big).
    \end{equation}
\end{proof}

\begin{theorem}[Différentielle de fonctions composées\cite{SNPdukn}]    \label{ThoAGXGuEt}
    Soient \( E\), \( F\) et \( G\) des espaces vectoriels normés, \( U\) ouvert dans \( E\) et \( V\) ouvert dans \( F\). Soient des applications de classe \( C^r\) (\( r\geq 1\))
    \begin{subequations}
        \begin{align}
            f\colon U\to V\\
            g\colon V\to G.
        \end{align}
    \end{subequations}
    Alors l'application \( g\circ f\colon V\to G\) est de classe \( C^r\) et
    \begin{equation}\label{EqHFmezmr}
        d(g\circ f)_x=dg_{f(x)}\circ df_x.
    \end{equation}
\end{theorem}

\begin{proof}
    Nous nous fixons \( x\in U\). La fonction \( f\) est différentiable en \( x\in U\) et \( g\) en \( f(x)\), donc nous pouvons écrire
    \begin{equation}
        f(x+h)=f(x)+df_x(h)+\alpha(h)
    \end{equation}
    et
    \begin{equation}
        g\big( f(x)+u \big)=g\big( f(x) \big)+dg_{f(x)}(u)+\beta(u)
    \end{equation}
    où la fonction \( \alpha\) a la propriété que
    \begin{equation}
        \lim_{h\to 0} \frac{ \| \alpha(h) \| }{ \| h \| }=0;
    \end{equation}
    et la même chose pour \( \beta\). La fonction composée en \( x+h\) s'écrit donc
    \begin{equation}    \label{EqCXcfhfH}
        (g\circ f)(x+h)=g\big( f(x)+df_x(h)+\alpha(h) \big)=g\big( f(x) \big)+dg_{f(x)}\big( df_x(h)+\alpha(h) \big)+\beta\big( df_x(h)+\alpha(h) \big).
    \end{equation}
    Nous montrons que tous les «petits» termes de cette formule peuvent être groupés. D'abord si \( h\) est proche de \( 0\), nous avons
    \begin{equation}
        \frac{ \| df_x(h)+\alpha(h) \| }{ \| h \| }\leq\frac{ \| df_x \|\| h \| }{ \| h \| }+\frac{ \| \alpha(h) \| }{ \| h \| }.
    \end{equation}
    Si \( h\) est petit, le second terme est arbitrairement petit, donc en prenant n'importe que \( M>\| df_x \|\) nous avons
    \begin{equation}
        \frac{ \| df_x(h)+\alpha(h) \| }{ \| h \| }\leq M.
    \end{equation}
    Par ailleurs, nous avons
    \begin{equation}
        \frac{ \| \beta\big( df_x(h)+\alpha(h) \big) \| }{ \| h \| }=\frac{  \| \beta\big( df_x(h)+\alpha(h) \big) \|  }{ \| df_x(h)+\alpha(h) \| }\frac{  \| df_x(h)+\alpha(h) \|  }{ \| h \| }\leq M\frac{  \| \beta\big( df_x(h)+\alpha(h) \big) \|  }{   \| df_x(h)+\alpha(h) \| }.
    \end{equation}
    Vu que la fraction est du type \( \frac{ \beta( f(h)) }{ f(h) }\) avec \( \lim_{h\to 0} f(h)=0\), la fraction tend vers zéro lorsque \( h\to 0\). En posant
    \begin{equation}
        \gamma_1(h)=\beta\big( df_x(h)+\alpha(h) \big)
    \end{equation}
    nous avons \( \lim_{h\to 0} \gamma_1(h)/\| h \|=0\).

    L'autre candidat à être un petit terme dans \eqref{EqCXcfhfH} est traité en utilisant la proposition \ref{PropEDvSQsA} :
    \begin{equation}
        \| dg_{f(x)}\big( \alpha(h) \big) \|\leq \| dg_{f(x)} \|\| \alpha(h) \|.
    \end{equation}
    Donc
    \begin{equation}
        \frac{ \| dg_{f(x)}\big( \alpha(h) \big) \| }{ \| h \| }\leq \| dg_{f(x)} \|\frac{ \| \alpha(h) \| }{ \| h \| },
    \end{equation}
    ce qui nous permet de poser
    \begin{equation}
        \gamma_2(h)=dg_{f(x)}\big( \alpha(h) \big)
    \end{equation}
    avec \( \gamma_2\) qui a la même propriété que \( \gamma_1\). Avec tout cela, en posant \( \gamma=\gamma_1+\gamma_2\) nous récrivons
    \begin{equation}
        (g\circ f)(x+h)=g\big( f(x) \big)+dg_{f(x)}\big( df_x(h) \big)+\gamma(h)
    \end{equation}
    avec \( \lim_{h\to 0} \frac{ \gamma(h) }{ \| h \| }=0\). Tout cela pour dire que
    \begin{equation}
        \lim_{h\to 0} \frac{ (g\circ f)(x+h)-(g\circ f)(x)-\big( dg_{f(x)}\circ df_x \big)(h) }{ \| h \| }=0,
    \end{equation}
    ce qui signifie que 
    \begin{equation}
        d(g\circ f)_x=dg_{f(x)}\circ df_x.
    \end{equation}
    Nous avons donc montré que si \( f\) et \( g\) sont différentiables, alors \( g\circ f\) est différentiable avec différentielle donnée par \eqref{EqHFmezmr}.

    Nous passons à la régularité. Nous supposons maintenant que \( f\) et \( g\) sont de classe \( C^r\) et nous considérons l'application
    \begin{equation}
        \begin{aligned}
            \varphi\colon L(F,G)\times L(E,F)&\to L(E,G) \\
            (A,B)&\mapsto A\circ B. 
        \end{aligned}
    \end{equation}
    Montrons que l'application \( \varphi\) est continue en montrant qu'elle est bornée\footnote{Proposition \ref{PropmEJjLE}.}. Pour cela nous écrivons la norme opérateur
    \begin{equation}
        \| \varphi \|=\sup_{\| (A,B) \|=1}\| \varphi(A,B) \|=\sup_{\| (A,B) \|=1}\| A\circ B \|\leq\sup_{\| (A,B) \|=1}\| A \|\| B \|\leq 1.
    \end{equation}
    Pour ce calcul nous avons utilisé le fait que la norme opérateur soit une norme algébrique (proposition \ref{PropEDvSQsA}) ainsi que la définition \ref{DefFAJgTCE} de la norme sur un espace produit pour la dernière majoration. L'application \( \varphi\) est donc continue et donc \(  C^{\infty}\) par le lemme \ref{LemLLvgPQW}. Nous considérons également l'application
    \begin{equation}
        \begin{aligned}
        \psi\colon U&\to L(F,G)\times L(E,F) \\
        x&\mapsto \big( dg_{f(x)},df_x \big). 
        \end{aligned}
    \end{equation}
    Vu que \( f\) et \( g\) sont \( C^1\), l'application \( \psi\) est continue. Ces deux applications \( \varphi\) et \( \psi\) sont choisies pour avoir
    \begin{equation}
        (\varphi\circ\psi)(x)=\varphi\big( dg_{f(x)},df_x \big)=dg_{f(x)}\circ df_x,
    \end{equation}
    c'est à dire \( \varphi\circ\psi=d(g\circ f)\). Les applications \( \varphi\) et \( \psi\) étant continues, l'application \( d(g\circ f)\) est continue, ce qui prouve que \( g\circ f\) est \( C^1\).

    Si \( f\) et \( g\) sont \( C^r\) alors \( dg\in C^{r-1}\) et \( dg\circ f\in C^{r-1}\) où il ne faut pas se tromper : \( dg\colon F\to L(F,G)\) et \( f\colon U\to F\); la composée est \( dg\circ f\colon x\mapsto dg_{f(x)}\in L(F,G)\). 
    
    Pour la récurrence nous supposons que \( f,g\in C^{r-1}\) implique \( g\circ f\in C^{r-1}\) pour un certain \( r\geq 2\) (parce que nous venons de prouver cela avec \( r=1\) et \( r=2\)). Soient \( f,g\in C^r\) et montrons que \( g\circ f\in C^r\). Par la proposition \ref{PropOYtgIua} nous avons
    \begin{equation}
        \psi=dg\circ f\times df\in C^{r-1},
    \end{equation}
    et donc \( d(g\circ f)=\varphi\circ\psi\in C^{r-1}\), ce qui signifie que \( g\circ f\in C^r\).
\end{proof}

\begin{lemma}[Leibnitz pour les formes bilinéaires\cite{SNPdukn}]\label{LemFRdNDCd}
    Si \( B\colon E\times F\to G\) est bilinéaire et continue, elle est \(  C^{\infty}\) et
    \begin{equation}    \label{EqXYJgDBt}
        dB_{(x,y)}(u,v)=B(x,v)+B(u,y).
    \end{equation}
\end{lemma}

\begin{proof}
    D'abord le membre de droite de \eqref{EqXYJgDBt} est une application linéaire et continue, donc c'est un bon candidat à être différentielle. Nous allons prouver que ça l'est, ce qui prouvera la différentiabilité de \( B\). Avec ce candidat, le numérateur de la définition \eqref{EqIQuRGmO} s'écrit dans notre cas
    \begin{equation}
        B\big( (x,y)+(u,v) \big)-B(x,y)-B(x,v)-B(u,y)=B(u,v).
    \end{equation}
    Il reste à voir que 
    \begin{equation}
        \lim_{ (u,v)\to (0,0) } \frac{ B(u,v) }{ \| (u,v) \| }=0
    \end{equation}
    Par l'équation \eqref{EqYLnbRbC} nous avons
    \begin{equation}
        \frac{ \| B(u,v) \| }{ \| (u,v) \| }\leq \frac{ \| B \|\| u \|\| v \| }{ \| u \| }=\| B \|\| v \|
    \end{equation}
    parce que \( \| (u,v) \|\geq \| u \|\). À partir de là il est maintenant clair que
    \begin{equation}
        \lim_{(u,v)\to (0,0)}\frac{ \| B(u,v) \| }{ \| (u,v) \| }=0,
    \end{equation}
    ce qu'il fallait.
\end{proof}

\begin{proposition}[Règle de Leibnitz\cite{SNPdukn}]
    Soient \( E,F_1,F_2\) des espaces vectoriels normés, \( U\) ouvert dans \( E\) et des applications de classe \( C^r\) (\( r\geq 1\))
    \begin{subequations}
        \begin{align}
            f_1\colon U\to F_1\\
            f_2\colon U\to F_2\\
        \end{align}
    \end{subequations}
    et \( B\in\cL(F_1\times F_2,G)\). Alors l'application
    \begin{equation}
        \begin{aligned}
            \varphi\colon U&\to G \\
            x&\mapsto B\big( f_1(x),f_2(x) \big) 
        \end{aligned}
    \end{equation}
    est de classe \( C^r\) et
    \begin{equation}    \label{EqMNGBXWc}
        d\varphi_x(u)=\varphi\big( (df_1)_x(u),f_2(x) \big)+\varphi\big( f_1(x),(df_2)_x(u) \big).
    \end{equation}
\end{proposition}
\index{Leibnitz!applications entre espaces vectoriels normés}

\begin{proof}
    Par hypothèse \( B\) est continue (c'est la définition de l'espace \( \cL\)), et donc \(  C^{\infty}\) par le lemme \ref{LemFRdNDCd}. Par ailleurs la fonction \( f_1\times f_2\) est de classe \( C^r\) parce que \( f_1\) et \( f_2\) le sont et parce que la proposition \ref{PropOYtgIua} le dit. L'application composée \( B\circ(f_1\times f_2)\) est donc également de classe \( C^r\) par le théorème \ref{ThoAGXGuEt}.

    Il ne nous reste donc qu'à prouver la formule \ref{EqMNGBXWc}. En utilisant la différentielle du produit cartésien\footnote{Proposition \ref{PropOYtgIua}.} nous avons
    \begin{equation}
        f\big( B\circ(f_1\times f_2) \big)_x(h)=dB_{(f_1\times f_2)(x)}\big( (df_1)_x(h),(df_2)_x(h) \big).
    \end{equation}
    Nous développons cela en utilisant le lemme \ref{LemFRdNDCd} :
    \begin{subequations}
        \begin{align}
        d\big( B\circ(f_1\times f_2) \big)_x(h)&=dB_{\big( f_1(x),f_2(x) \big)}\big( (df_1)_x(h),(df_2)_x(h) \big)\\
        &=B\big( f_1(x),(df_2)_x(h) \big)+B\big( (df_1)_x(h),f_2(x) \big),
        \end{align}
    \end{subequations}
    comme souhaité.
\end{proof}

%--------------------------------------------------------------------------------------------------------------------------- 
\subsection{Différentielle partielle}
%---------------------------------------------------------------------------------------------------------------------------

\begin{definition}[Différentielle partielle]    \label{VJM_CtSKT}
    Soient \( E\), \( F\) et \( G\) des espaces vectoriels normés et une fonction \( f\colon E\times F\to G\). Nous définissons sa \defe{différentielle partielle}{différentielle!partielle} sur l'espace \( E\) par
    \begin{equation}
        \begin{aligned}
            d_1f_{(x_0,y_0)}\colon E&\to G \\
            u&\mapsto \Dsdd{ f(x_0+tu,y_0 }{t}{0} .
        \end{aligned}
    \end{equation}
    La différentielle \( d_2\) se définit de la même façon.
\end{definition}

\begin{proposition}[\cite{SNPdukn}] \label{PropLDN_nHWDF}
    Soient \( E_1\), \( E_2\) et \( F\) des espaces vectoriels normés, soit un ouvert \( U\subset E_1\times E_2\) et une fonction \( f\colon U\to F\).
    \begin{enumerate}
        \item   \label{ItemRDD_oPmXVi}
            Si \( f\) est différentiable alors les différentielles partielles existent et
            \begin{subequations}
                \begin{align}
                    d_1f_{(x_0,y_0)}(u)=df_{(x_0,y_0)}(u,0)\\
                    d_2f_{(x_0,y_0)}(v)=df_{(x_0,y_0)}(0,v)
                \end{align}
            \end{subequations}
            où \( u\in E_1\) et \( v\in E_2\).
        \item
            Si \( f\) est différentiable alors
            \begin{equation}
                df_{(x_0,y_0)}(u,v)=d_1f_{(x_,y_0)}(u)+d_2f_{(x_0,y_0)}(v).
            \end{equation}
    \end{enumerate}
\end{proposition}

\begin{proof}
    Nous posons \( \alpha=(x_0,y_0)\in U\) et
    \begin{equation}
        \begin{aligned}
            j_{\alpha}^{(1)}\colon E_1&\to E_1\times E_2 \\
            x&\mapsto (x,y_0). 
        \end{aligned}
    \end{equation}
    C'est une fonction de classe \(  C^{\infty}\) et 
    \begin{equation}
        (dj_{\alpha}^{(1)})_{x_0}(u)=\Dsdd{ j_{\alpha}^{(1)}(x_0+tu) }{t}{0}=\Dsdd{ (x_0+tu,y_0) }{t}{0}=(u,0).
    \end{equation}
    D'autre part 
    \begin{subequations}
        \begin{align}
            (d_1f)_{\alpha}(u)&=\Dsdd{ f(x_0+tu,y_0) }{t}{0}\\
            &=\Dsdd{ (f\circ j_{\alpha}^{(1)})(x_0+tu) }{t}{0}\\
            &=\big( d(f\circ j_{\alpha}^{(1)}) \big)_{x_0}(u).
        \end{align}
    \end{subequations}
    À ce moment nous utilisons la règle des différentielles composées \ref{ThoAGXGuEt} pour dire que
    \begin{equation}
        (d_1f)_{\alpha}(u)=df_{j_{\alpha}^{(1)}(x_0)}\circ (dj_{\alpha}^{(1)})_{x_0}(u)=df_{\alpha}(u,0).
    \end{equation}
    Voila qui prouve déjà le point \ref{ItemRDD_oPmXVi}.

    Pour la suite nous considérons les fonctions 
    \begin{equation}
        \begin{aligned}[]
            P_1(x,y)&=x,&&&J_1(u)&=(u,0),\\
            P_2(x,y)&=y,&&&J_2(v)&=(0,v)
        \end{aligned}
    \end{equation}
    et nous avons l'égalité évidente
    \begin{equation}
        J_1\circ P_1+J_2\circ P_2=\mtu
    \end{equation}
    sur \( E_1\times E_2\). En appliquant \( df_{\alpha}\) à cette dernière égalité, en appliquant à \( (u,v)\) et en utilisant la linéarité de \( df_{\alpha}\) nous trouvons
    \begin{subequations}
        \begin{align}
            df_{\alpha}(u,v)&=df_{\alpha}\big( (J_1\circ P_1)(u,v) \big)+df_{\alpha}\big( (J_2\circ P_2)(u,v) \big)\\
            &=df_{\alpha}(u,0)+df_{\alpha}(0,v)\\
            &=(d_1f)_{\alpha}(u)+(d_2f)_{\alpha}(v)
        \end{align}
    \end{subequations}
    où nous avons utilisé le point \ref{ItemRDD_oPmXVi} pour la dernière égalité.
\end{proof}

%--------------------------------------------------------------------------------------------------------------------------- 
\subsection{Formule des accroissements finis}
%---------------------------------------------------------------------------------------------------------------------------

\begin{proposition} \label{PropDQLhSoy}
    Soit \( E\) un espace vectoriel normé. Soient \( a<b\) dans \( \eR\) et deux fonctions
    \begin{subequations}
        \begin{align}
            f\colon \mathopen[ a , b \mathclose]\to E\\
            g\colon \mathopen[ a , b \mathclose]\to \eR
        \end{align}
    \end{subequations}
    continues sur \( \mathopen[ a , b \mathclose]\) et dérivables sur \( \mathopen] a , b \mathclose[\). Si pour tout \( t\in\mathopen] a , b \mathclose[\) nous avons \( \| f'(t) \|\leq g'(t)\) alors
        \begin{equation}
            \| f(b)-f(a) \|\leq g(b)-g(a).
        \end{equation}
\end{proposition}

\begin{proof}
    Soit \( \epsilon>0\) et la fonction
    \begin{equation}
        \begin{aligned}
            \varphi_{\epsilon}\colon \mathopen[ a , b \mathclose]&\to \eR \\
            t&\mapsto \| f(t)-f(a) \|-g(t)-\epsilon t. 
        \end{aligned}
    \end{equation}
    Cela est une fonction continue réelle à variable réelle. En particulier pour tout \( u\in\mathopen] a , b \mathclose[\) la fonction \( \varphi_{\epsilon}\) est continue sur le compact \( \mathopen[ u , b \mathclose]\) et donc y atteint son minimum en un certain point \( c\in\mathopen[ u , b \mathclose]\); c'est le bon vieux théorème de Weierstrass \ref{ThoWeirstrassRn}. Nous commençons par montrer que pour tout \( u\), ledit minimum ne peut être que \( b\). Pour cela nous allons montrer que si \( t\in\mathopen[ u , b [\), alors \( \varphi_{\epsilon}(s)<\varphi_{\epsilon}(t)\) pour un certain \( s>t\). Par continuité si \( s\) est proche de \( t\) nous avons
        \begin{equation}
            \left\|  \frac{ f(s)-f(t) }{ s-t }  \right\|-\frac{ \epsilon }{2}<\| f'(t) \|<g'(t)+\frac{ \epsilon }{2}=\frac{ g(s)-g(t) }{ s-t }+\frac{ \epsilon }{2}.
        \end{equation}
        Ces inégalités proviennent de la limite
        \begin{equation}
            \lim_{s\to t} \frac{ f(s)-f(t) }{ s-t }=f'(t),
        \end{equation}
        donc si \( s\) et \( t\) sont proches,
        \begin{equation}
            \left\| \frac{ f(s)-f(t) }{ s-t }-f'(t) \right\|
        \end{equation}
        est petit. Si \( s>t\) nous pouvons oublier des valeurs absolues et transformer l'inégalité en
        \begin{equation}
            \| f(s)-f(t) \|<g(s)-g(t)+\epsilon(s-t).
        \end{equation}
        Utilisant cela et l'inégalité triangulaire,
        \begin{subequations}
            \begin{align}
                \varphi_{\epsilon}(s)&\leq\| f(s)-f(t) \|+\| f(t)-f(a) \|-g(s)-\epsilon s\\
                &\leq g(s)-g(t)+\epsilon s-\epsilon t+\| f(t)-f(a) \|-g(s)-\epsilon s\\
                &=\varphi_{\epsilon}(t).
            \end{align}
        \end{subequations}
        Donc nous avons bien \( \varphi_{\epsilon}(s)<\varphi_{\epsilon}(t)\) avec l'inégalité stricte. Par conséquent pour tout \( u\in\mathopen] a , b \mathclose[\) nous avons \( \varphi_{\epsilon}(b)<\varphi_{\epsilon}(u)\) et en prenant la limite \( u\to a\) nous avons
        \begin{equation}
            \varphi_{\epsilon}(b)\leq \varphi_{\epsilon}(a).
        \end{equation}
        Cette inégalité donne immédiatement
        \begin{equation}
            \| f(b)-f(a) \|\leq g(b)-g(a)+\epsilon(b-a)
        \end{equation}
         pour tout \( \epsilon>0\) et donc
         \begin{equation}
            \| f(b)-f(a) \|\leq g(b)-g(a).
         \end{equation}
\end{proof}

\begin{theorem}[Théorème des accroissements finis]\label{ThoNAKKght}
    Soient \( E\) et \( F\) des espaces vectoriels normés, \( U \) ouvert dans \( E\) et une application différentiable \( f\colon U\to F\). Pour tout segment \( \mathopen[ a , b \mathclose]\subset U\) nous avons
    \begin{equation}
        \| f(b)-f(a) \|\leq\left( \sup_{x\in\mathopen[ a , b \mathclose]}\| df_x \| \right)\| b-a \|.
    \end{equation}
\end{theorem}
\index{théorème!accroissements finis}
Une version de ce théorème adaptée aux espaces de dimension finie est le théorème \ref{val_medio_2}.

\begin{proof}
    Nous prenons les applications
    \begin{equation}
        \begin{aligned}
            k\colon \mathopen[ 0 , 1 \mathclose]&\to E \\
            t&\mapsto f\big( (1-t)a+tb \big) 
        \end{aligned}
    \end{equation}
    et
    \begin{equation}
        \begin{aligned}
            g\colon \mathopen[ 0 , 1 \mathclose]&\to \eR \\
            t&\mapsto t\sup_{x\in\mathopen[ a , b \mathclose]}\| df_x \|\| b-a \|.
        \end{aligned}
    \end{equation}
    Pour tout \( t\) nous avons \( g'(t)=M\| b-a \|\) où il n'est besoin de dire ce qu'est \( M\). D'un autre côté nous avons aussi
    \begin{equation}
        \begin{aligned}[]
            k'(t)&=\lim_{\epsilon\to 0}\frac{ f\big( (1-t-\epsilon)a+(t+\epsilon)b \big)-f\big( (1-t)a+tb \big) }{ \epsilon }\\
            &=\Dsdd{ f\big( (1-t)a+tb+\epsilon(b-a) \big)  }{\epsilon}{0}\\
            &=df_{(1-t)a+tb}(b-a)
        \end{aligned}
    \end{equation}
    où nous avons utilisé l'hypothèse de différentiabilité de \( f\) sur \( \mathopen[ a , b \mathclose]\) et donc en \( (1-t)a+tb\). Nous avons donc
    \begin{equation}
        \| k'(t) \|\leq \| b-a \|\| df_{(1-t)a+tb} \|\leq M\| b-a \|=g'(t)
    \end{equation}
    La proposition \ref{PropDQLhSoy} est donc utilisable et
    \begin{equation}
        \| k(1)-k(0) \|=g(1)-g(0),
    \end{equation}
    c'est à dire
    \begin{equation}
        \| f(b)-f(a) \|=M\| b-a \|
    \end{equation}
    comme il se doit.
\end{proof}

\begin{proposition} \label{ProFSjmBAt}
    Soient \( E\) et \( F\) des espaces vectoriels normés, \( U \) ouvert dans \( E\) et une application \( f\colon U\to F\). Soient \( a,b\in U\) tels que \( \mathopen[ a , b \mathclose]\subset U\). Nous posons \( u=(b-a)/\| b-a \|\) et nous supposons que pour tout \( x\in\mathopen[ a , b \mathclose]\), la dérivée directionnelle
    \begin{equation}
        \frac{ \partial f }{ \partial u }(x)=\Dsdd{ f(x+tu) }{t}{0}
    \end{equation}
    existe. Nous supposons de plus que \( \frac{ \partial f }{ \partial u }(x)\) est continue en \( x=a\). Alors
    \begin{equation}
        \| f(b)-f(a) \|\leq\left( \sup_{x\in\mathopen[ a , b \mathclose]}\| \frac{ \partial f }{ \partial u }(x) \| \right)\| b-a \|.
    \end{equation}
\end{proposition}

\begin{proof}
    Nous posons évidemment 
    \begin{equation}
        M=\sup_{x\in\mathopen[ a , b \mathclose]}\| \frac{ \partial f }{ \partial u }(x) \| 
    \end{equation}
    et nous considérons les fonctions
    \begin{equation}
        k(t)=f\big( (1-t)a+tb \big)
    \end{equation}
    et
    \begin{equation}
        g(t)=tM\| b-a \|.
    \end{equation}
    Pour alléger les notations nous posons \( x=(1-t)a+tb\) et nous calculons avec un petit changement de variables dans la limite :
    \begin{equation}
        k'(t)=\Dsdd{  f\big( x+\epsilon(b-a) \big)  }{\epsilon}{0}=\| b-a \|\Dsdd{ f\big( x+\frac{ \epsilon }{ \| b-a \| }(b-a) \big) }{\epsilon}{0}=\| b-a \|\frac{ \partial f }{ \partial u }(x),
    \end{equation}
    et donc encore une fois nous avons
    \begin{equation}
        \| k'(t) \|\leq g'(t),
    \end{equation}
    ce qui donne
    \begin{equation}
        \| k(1)-k(0) \|=g(1)-g(0),
    \end{equation}
    c'est à dire
    \begin{equation}
        \| f(b)-f(a) \|\leq \sup_{x\in\mathopen[ a , b \mathclose]}\| \frac{ \partial f }{ \partial u }(x) \|\| b-a \|.
    \end{equation}
\end{proof}

\begin{theorem} \label{ThoOYwdeVt}
    Soient \( E,V\) deux espaces vectoriels normés, une application \( f\colon E\to V\), un point \( a\in E\) tel que pour tout \( u\in E\), la dérivée
    \begin{equation}
        \Dsdd{ f(x+tu) }{t}{0}
    \end{equation}
    existe pour tout \( x\in B(a,r)\) et est continue (par rapport à \( x\)) en \( x=a\). Nous supposons de plus que\quext{Je ne suis pas certain que cette hypothèse soit nécessaire, voir la question \ref{ItemLPrIWZhPg} de la page \pageref{ItemLPrIWZhPg}.}
    \begin{equation}
        \frac{ \partial f }{ \partial u }(a)=0
    \end{equation}
    pour tout \( u\in E\). Alors \( f\) est différentiable en \( a\) et
    \begin{equation}
        df_a=0
    \end{equation}
\end{theorem}

\begin{proof}
    Soit \( \epsilon>0\). Pourvu que \( \| h \|\) soit assez petit pour que \( a+h\in B(a,r)\), la proposition \ref{ProFSjmBAt} nous donne
    \begin{equation}
        \| f(a+h)-f(a) \|\leq \sup_{x\in\mathopen[ a , a+h \mathclose]}\| \frac{ \partial f }{ \partial u }(x) \|  |h |
    \end{equation}
    où \( u=h/\| h \|\). Par continuité de \( \partial_uf(x)\) en \( x=a\) et par le fait que cela vaut \( 0\) en \( x=a\), il existe un \( \delta>0\) tel que si \( \| h \|<\delta\) alors
    \begin{equation}
        \| \frac{ \partial f }{ \partial u }(a+h) \|\leq \epsilon.
    \end{equation}
    Pour de tels \( h\) nous avons
    \begin{equation}
        \| f(a+h)-f(a) \|\leq \epsilon\| h \|,
    \end{equation}
    ce qui prouve que l'application linéaire \( T(u)=0\) convient parfaitement pour faire fonctionner la définition \ref{DefKZXtcIT}.
%
%    Nous ne supposons plus que les dérivées directionnelles de \( f\) sont nulles en \( x=a\). Alors nous posons, pour \( x\in U\),
%    \begin{equation}    \label{EqCUgHXHy}
%        g(x)=f(x)-\Dsdd{ f(a+s(x-a)) }{s}{0}.
%    \end{equation}
%    Le fait que cette fonction soit bien définie est encore un coup de hypothèses sur les dérivées directionnelles de \( f\) qui sont bien définies autour de \( a\). Cette nouvelle fonction \( g\) satisfait à \( \frac{ \partial g }{ \partial v }(a)=0\) pour tout \( v\in E\) parce que
%    \begin{subequations}
%        \begin{align}
%            \frac{ \partial g }{ \partial v }(a)&=\Dsdd{ g(a+tv) }{t}{0}\\
%            &=\Dsdd{ f(a+tv)-\Dsdd{ f\big( a+s(tv) \big) }{s}{0} }{t}{0}\\
%            &=\frac{ \partial f }{ \partial v }(a)-\Dsdd{ t\frac{ \partial f }{ \partial v }(a) }{t}{0}\\
%            &=0.
%        \end{align}
%    \end{subequations}
%    Pour la dérivée par rapport à \( s\) nous avons effectué le changement de variables \( s\to ts\), ce qui explique la présence d'un \( t\) en facteur. La fonction \( g\) est donc différentiable en \( a\).
%
%
% Position 229262367
    % Attention : ce qui suit est faux. Mais il y a peut-être moyen d'adapter.
%\item[Dérivées non nulles]
%
%    Nous allons montrer que la fonction 
%    \begin{equation}
%        l(x)=\Dsdd{ f\big( a+s(x-a) \big) }{t}{0}
%    \end{equation}
%    est différentiable en \( x=a\), de différentielle \( T(u)=l(u+a)\). Cela fournira la différentiabilité de \( f\) parce que \eqref{EqCUgHXHy} donnerait alors \( f\) comme somme de deux fonctions différentiables.
%
%    En premier lieu nous devons montrer que \( T\) ainsi définie est linéaire.
%    
%    Notre but est donc de prouver que
%    \begin{equation}
%        \lim_{h \to 0}\frac{ \| l(x+h)-l(x)-l(h) \| }{ \| h \| }=0.
%    \end{equation}
%    Un premier pas est de calculer
%    \begin{subequations}
%        \begin{align}
%            l(x+h)-l(x)-l(h)&=\lim_{s\to 0}\frac{ f\big( s(x+h) \big)-f(0)-f(sx)+f(0)-f(sh)+f(0) }{ s }\\
%            &=\lim_{s\to 0}\frac{ f\big( s(x+h) \big)-f(sx)-f(sh)+f(0) }{ s }.
%        \end{align}
%    \end{subequations}
%    Ensuite nous étudions le numérateur en utilisant la proposition \ref{ProFSjmBAt}:
%    \begin{subequations}
%        \begin{align}
%            \| f\big( s(x+h) \big)-f(sx)-f(sh)+f(0) \|&\leq  \| f\big( s(x+h) \big)-f(sx)\| + \|f(sh)-f(0) \|  \\
%            &\leq \sup_{z\in\mathopen[ sx , sx+sh \mathclose]}\| \frac{ \partial f }{ \partial h }(z) \|\| sh \|\\
%            &\quad +\sup_{z\in\mathopen[ 0 , sh \mathclose]}\| \frac{ \partial f }{ \partial h }(z) \|\| sh \|.
%        \end{align}
%    \end{subequations}
%    La division par \( s\) se passe bien et nous avons
%    \begin{subequations}
%        \begin{align}
%            \| l(x+h)-l(x)-l(h) \|&\leq \lim_{s\to 0}  \sup_{z\in\mathopen[ sx , sx+sh \mathclose]}\| \frac{ \partial f }{ \partial h }(z) \|\| h \|+ \sup_{z\in\mathopen[ 0 , sh \mathclose]}\| \frac{ \partial f }{ \partial h }(z) \|\| h \|\\
%            &=2\| h \|\| \frac{ \partial f }{ \partial h }(0) \|        \label{SubeqVMMoSDH}\\
%            &=2\| h \|^2\| \frac{ \partial f }{ \partial u }(0) \|
%        \end{align}
%    \end{subequations}
%    où nous avons posé \( u=h/\| h \|\). Pour l'égalité \eqref{SubeqVMMoSDH} nous avons utilisé la continuité de \( \frac{ \partial f }{ \partial h }(z)\) en \( z=0\). Du coup
%    \begin{equation}
%        \lim_{y\to 0} \frac{ \| f(x+h)-f(x)-f(h) \| }{ \| h \| }=\lim_{h\to 0} 2\| h \|\| \frac{ \partial f }{ \partial u }(0) \|=0.
%    \end{equation}
%    Cela prouve que \( l\) est bien différentiable en \( x=0\).
%
%    \end{subproof}
%
\end{proof}

%--------------------------------------------------------------------------------------------------------------------------- 
\subsection{L'inverse, sa différentielle}
%---------------------------------------------------------------------------------------------------------------------------

Si \( E\) est un espace de Banach, nous sommes intéressé à l'espace \( \GL(E)\) des endomorphismes inversibles de \( E\) sur \( E\). Cet ensemble est métrique par la formule usuelle
\begin{equation}
    \| T \|=\sup_{\| x \|=1}\| T(x) \|_E.
\end{equation}

\begin{theorem}[Inverse dans \( \GL(E)\)\cite{laudenbach2000calcul,SNPdukn}]    \label{ThoCINVBTJ}
    Soient \( E\) et \( F\) des espaces vectoriels normés.
    \begin{enumerate}
        \item
        L'ensemble \( \GL(E)\) est ouvert dans \( \End(E)\).
    \item
        L'application inverse
    \begin{equation}
        \begin{aligned}
        i\colon \GL(E,F)&\to \GL(F,E) \\
        u&\mapsto u^{-1} 
        \end{aligned}
    \end{equation}
    est de classe \( C^{\infty}\) et
    \begin{equation}
        di_{u_0}(h)=-u_0^{-1}\circ h\circ u_0^{-1}
    \end{equation}
    pour tout \( h\in\End(E)\)
    \end{enumerate}
\end{theorem}
\index{différentielle!de $u\mapsto u^{-1}$}

\begin{proof}
Nous supposons que \( \GL(E,F)\) n'est pas vide, sinon ce n'est pas du jeu.
        \begin{subproof}

        \item[Cas de dimension finie]

            Si la dimension de \( E\) et \( F\) est finie, elles doivent être égales, sinon il n'y a pas de fonctions inversibles \( E\to F\). L'ensemble \( \GL(E,F)\) est donc naturellement \( \GL(n,\eR)\). Un élément de \( \eM(n,\eR)\) est dans \( \GL(n,\eR)\) si et seulement si son déterminant est non nul. Le déterminant étant une fonction continue (polynomiale) en les entrées de la matrice, l'ensemble \( \GL(n,\eR)\) est ouvert dans \( \eM(n,\eR)\).

            Même idée pour la régularité de la fonction \( i\colon \GL(n,\eR)\to \GL(n,\eR)\), \( X\mapsto X^{-1}\). Les entrées de \( X^{-1}\) sont les cofacteurs de \( X\) divisé par \( \det(X)\), et donc des polynômes en les entrées de \( X\) divisés par un polynôme qui ne s'annule pas sur \( \GL(n,\eR)\), et donc sur un ouvert autour de \( X\) et de \( X^{-1}\). Bref, tout est \(  C^{\infty}\).

            Le reste de la preuve parle de la dimension infinie.

        \item[Ouvert autour de l'identité]
            
        Nous commençons par prouver que \( B(\mtu,1)\subset \GL(E)\). Pour cela il suffit de remarquer que si \( \| u \|<1\) alors le lemme \ref{PropQAjqUNp} nous donne un inverse de \( (1+u)\) en la personne de \( \sum_{k=0}^{\infty}(-u)^k\).

    \item[Ouvert en général]

        Soit maintenant \( u_0\in\GL(E)\). Si \( \| u \|<\frac{1}{ \| u_0^{-1} \| }\) alors \( \| u_0^{-1}u \|<1\), ce qui signifie que
        \begin{equation}
            \mtu+u_0^{-1}u
        \end{equation}
    est inversible. Mais \( u_0+u=u_0(\mtu+u_0^{-1}u)\), donc \( u_0+u\in\GL(E)\) ce qui signifie que
    \begin{equation}
    B\left( u_0,\frac{1}{ \| u_0^{-1} \| } \right)\subset \GL(E).
    \end{equation}

    \item[Différentielle en l'identité]

    Nous commençons par prouver que \( di_{\mtu}(u)=-u\). Pour cela nous posons 
    \begin{equation}
        \alpha(h)=\sum_{k=2}^{\infty}(-1)^kh^k
    \end{equation}
    et nous calculons
    \begin{equation}
    di_{\mtu}(u)=\Dsdd{ i(\mtu+tu) }{t}{0}=\Dsdd{ \mtu-tu+\alpha(tu) }{t}{0}.
    \end{equation}
    Il suffit de prouver que \( \Dsdd{ \alpha(tu) }{t}{0}=0\) pour conclure que \( di_{\mtu}(u)=-u\). Pour cela, nous remarquons que \( \alpha(0)=0\) et donc que
    \begin{subequations}
        \begin{align}
        \Dsdd{ \alpha(tu) }{t}{0}&=\lim_{t\to 0} \frac{ \alpha(tu)-\alpha(0) }{ t }\\
        &=\lim_{t\to 0} \sum_{k=2}^{\infty}(-1)^k\frac{ (tu)^k }{ t }\\
        &=-\lim_{t\to 0} u\sum_{k=1}^{\infty}(-1)^kt^ku^k.
        \end{align}
    \end{subequations}
    La norme de ce qui est dans la limite est majorée par
    \begin{equation}
    \| u \|\sum_{k=1}^{\infty}\| tu \|^k=\| u \|\left( \frac{1}{ 1-\| tu \| }-1 \right),
    \end{equation}
    et cela tend vers zéro lorsque \( t\to\infty\). Nous avons utilisé la somme \ref{EqRGkBhrX} de la série géométrique. Nous avons bien prouvé que \( di_{\mtu}(u)=-u\).

    \item[Différentielle en général]
    Soit maintenant \( u_0\in\GL(E)\) et \( h\in\End(E)\) tel que \( u_0+h\in \GL(E)\); par le premier point, il suffit de prendre \( \| h \|\) suffisamment petit. Vu que \( u_0+h=u_0(\mtu+u_0^{-1}h)\) nous avons
    \begin{equation}
        (u_0+h)^{-1}=(\mtu+u_0^{-1}h)^{-1}u_0^{-1}.
    \end{equation}
    Nous pouvons donc calculer
    \begin{equation}
        (u_0+h)^{-1}=\big( \mtu-u_0^{-1}h+\alpha(u_0^{-1}h) \big)u_0^{-1}=u_0^{-1}-u_0^{-1}hu_0^{-1}+\alpha(u_0^{-1}h)u_0^{-1},
    \end{equation}
    et ensuite
    \begin{equation}
        di_{u_0}(h)=\Dsdd{ i(u_0+th) }{t}{0}=\Dsdd{ u_0^{-1}-tu_0^{-1}hu_0^{-1}+\alpha(tu_0^{-1}h)u_0^{-1} }{t}{0},
    \end{equation}
    mais nous avons déjà vu que
    \begin{equation}
        \Dsdd{ \alpha(th) }{t}{0}=0,
    \end{equation}
    donc
    \begin{equation}
        di_{u_0}(h)=-u_0^{-1}hu_0^{-1}
    \end{equation}
    Cela donne la différentielle de l'application inverse.

    \item[Continuité de l'inverse]

        L'application \( i\) est continue parce que différentiable.
    \item[L'inverse est \(  C^{\infty}\)]

        Nous allons écrire la fonction inverse comme une composée. Soient les applications
        \begin{equation}
            \begin{aligned}
                B\colon \cL(F,E)\times \cL(F,E)&\to \cL\big( \cL(E,F),\cL(F,E) \big) \\
                B(\psi_1,\psi_2)(A)&= -\psi_1\circ A\circ\psi_2
            \end{aligned}
        \end{equation}
        et
        \begin{equation}
            \begin{aligned}
                \Delta\colon \cL(F,E)&\to \cL(F,E)\times \cL(F,E) \\
                \varphi&\mapsto (\varphi,\varphi) 
            \end{aligned}
        \end{equation}
        Nous avons alors 
        \begin{equation}
            di=B\circ\Delta\circ i.
        \end{equation}
        L'application \( \Delta\) est de classe \(  C^{\infty}\). Nous devons voir que \( B\) l'est aussi. Pour le voir nous commençons par prouver qu'elle est bornée :
        \begin{equation}
            \begin{aligned}[]
                \| B \|&=\sup_{\| \psi_1 \|,\| \psi_2 \|=1}\| B(\psi_1,\psi_2) \|_{\aL\big( L(E,F),L(F,E) \big)}\\
                &=\sup_{  \| \psi_1 \|,\| \psi_2 \|=1 }\sup_{\| A \|=1}\| \psi_1\circ A\circ\psi_2 \|_{L(F,E)}\\
                &\leq \sup_{\| \psi_1 \|,\| \psi_2 \|=1}\sup_{\| A \|=1}\| \psi_1 \|\| A \|\| \psi_2 \|\\
                &\leq 1.
            \end{aligned}
        \end{equation}
        Donc \( B\) est bien bornée et par conséquent continue. Une application bilinéaire continue est \(  C^{\infty}\) par le lemme \ref{LemFRdNDCd}. La décomposition \( di=B\circ \Delta\circ i\) nous donne donc que \( i\in C^{\infty}\) dès que \( i\) est continue, ce que nous avions déjà montré.
        \end{subproof}
\end{proof}

%---------------------------------------------------------------------------------------------------------------------------
\subsection{Projection orthogonale}
%---------------------------------------------------------------------------------------------------------------------------

Le théorème suivant n'est pas indispensablissime parce qu'il est le même que le théorème de la projection sur les espaces de Hilbert\footnote{Théorème \ref{ThoProjOrthuzcYkz}}. Cependant la partie existence est plus simple en se limitant au cas de dimension finie.
\begin{theorem}[Théorème de la projection]  \label{ThoWKwosrH}
    Soit \( E\) un espace vectoriel réel ou complexe de dimension finie, \( x\in E\), et \( C\) un sous ensemble fermé convexe de \(E\).
    \begin{enumerate}
        \item
            Les deux conditions suivantes sur \( y\in E\) sont équivalentes:
    \begin{enumerate}
        \item   \label{zzETsfYCSItemi}
            \( \| x-y \|=\inf\{ \| x-z \|\tq z\in C \}\),
        \item\label{zzETsfYCSItemii}
            pour tout \( z\in C\), \( \real\langle x-y, z-y\rangle \leq 0\).
    \end{enumerate}
\item
    Il existe un unique \( y\in E\), noté \( y=\pr_C(x)\) vérifiant ces conditions.
    \end{enumerate}
\end{theorem}
%TODO : il y a sûrement un endroit plus adapté pour mettre ce théorème.

\begin{proof}
    Nous commençons par prouver l'existence et l'unicité d'un élément dans \( C\) vérifiant la première condition. Ensuite nous verrons l'équivalence. 

    \begin{subproof}
        \item[Existence]
        
            Soit \( z_0\in C\) et \( r=\| x-z_0 \|\). La boule fermée \( \overline{ B(x,r) }\) est compacte\footnote{C'est ceci qui ne marche plus en dimension infinie.} et intersecte \( C\). Vu que \( C\) est fermé, l'ensemble \( C'=C\cap\overline{ B(x,r) }\) est compacte. Tous les points qui minimisent la distance entre \( x\) et \( C\) sont dans \( C'\); la fonction 
            \begin{equation}
                \begin{aligned}
                     C'&\to \eR \\
                    z&\mapsto d(x,z) 
                \end{aligned}
            \end{equation}
            est continue sur un compact et donc a un minimum qu'elle atteint\footnote{Théorème \ref{ThoMKKooAbHaro}.}. Un point \( P\) réalisant ce minimum prouve l'existence d'un point vérifiant la première condition.

        \item[Unicité]
            Soient \( y_1\) et \( y_2\), deux éléments de \( C\) minimisant la distance avec \( x\), et soit \( d\) ce minimum. Nous avons par l'identité du parallélogramme \eqref{EqYCLtWfJ} que
            \begin{equation}
                \| y_1-y_2 \|^2=-4\left\| \frac{ y_1+y_2-x }{2} \right\|^2+2\| y_1-x \|^2+2\| y_2-x \|^2\leq -4d+2d+2d=0.
            \end{equation}
            Par conséquent \( y_1=y_2\).

        \item[\ref{zzETsfYCSItemi}\( \Rightarrow\) \ref{zzETsfYCSItemii}]

            Soit \( z\in C\) et \( t\in \mathopen] 0 , 1 \mathclose[\); nous notons \( P=\pr_Cx\). Par convexité le point \( z=ty+(1-t)P\) est dans \( C\), et par conséquent,
                \begin{equation}
                    \| x-P \|^2\leq\| x-tz-(1-t)P \|^2=\| (x-P)-t(z-P) \|^2.
                \end{equation}
                Nous sommes dans un cas \( \| a \|^2\leq | a-b |^2\), qui implique \( 2\real\langle a, b\rangle \leq \| b \|^2\). Dans notre cas,
                \begin{equation}
                    2\real\langle x-P , t(z-P)\rangle \leq t^2\| z-P \|^2.
                \end{equation}
                En divisant par \( t\) et en faisant \( t\to 0\) nous trouvons l'inégalité demandée :
                \begin{equation}
                    2\real\langle x-P, z-P\rangle \leq 0.
                \end{equation}
                
        \item[\ref{zzETsfYCSItemii}\( \Rightarrow\) \ref{zzETsfYCSItemi}]

            Soit un point \( P\in C\) vérifiant 
            \begin{equation}
                \real\langle x-P, z-P\rangle \leq 0
            \end{equation}
            pour tout \( z\in C\). Alors en notant \( a=x-P\) et \( b=P-z\),
            \begin{equation}
                \begin{aligned}[]
                \| x-z \|^2=\| x-P+P-z \|^2&=\| a+b \|^2\\
                &=\| a \|^2+\| b \|^2+2\real\langle a, b\rangle \\
                &=\| a \|^2+\| b \|^2-2\real\langle x-P, z-P\rangle \\
                &\geq \| b \|^2,
                \end{aligned}
            \end{equation}
            ce qu'il fallait.
    \end{subproof}
\end{proof}


%+++++++++++++++++++++++++++++++++++++++++++++++++++++++++++++++++++++++++++++++++++++++++++++++++++++++++++++++++++++++++++
\section{Sous espaces caractéristiques}
%+++++++++++++++++++++++++++++++++++++++++++++++++++++++++++++++++++++++++++++++++++++++++++++++++++++++++++++++++++++++++++

% TODO : lire le blog de Pierre Bernard; en particulier celle-ci : http://allken-bernard.org/pierre/weblog/?p=2299

Sources : \cite{MneimneReduct} et \wikipedia{fr}{Décomposition_de_Dunford}{divers articles sur Wikipédia}.
%TODO : citer mieux Wikipédia.

Lorsqu'un opérateur n'est pas diagonalisable, les valeurs propres jouent quand même un rôle important.

\begin{definition}  \label{DefFBNIooCGbIix}
    Soit \( E\) un \( \eK\)-espace vectoriel  \( f\in\End(E)\). Pour \( \lambda\in \eK\) nous définissons
    \begin{equation}
        F_{\lambda}(f)=\{ v\in E\tq (f-\lambda\mtu)^nv=0, n\in\eN \}
    \end{equation}
    et nous appelons ça un \defe{sous-espace caractéristique}{sous-espace!caractéristique} de \( f\).
\end{definition}
L'espace \( F_{\lambda}(f)\) est l'ensemble de nilpotence de l'opérateur \( f-\lambda\mtu\) et

\begin{lemma}   \label{LemBLPooHMAoyJ}
    L'ensemble \( F_{\lambda}(f)\) est non vide si et seulement si \( \lambda\) est une valeur propre de \( f\). L'espace \( F_{\lambda}(f)\) est invariant sous \( f\).
\end{lemma}

\begin{proof}
    Si \( F_{\lambda}(f)\) est non vide, nous considérons \( v\in F_{\lambda}(f)\) et \( n\) le plus petit entier non nul tel que \( (f-\lambda)^nv=0\). Alors \( (f-\lambda)^{n-1}v\) est un vecteur propre de \( f\) pour la valeur propre \( \lambda\). Inversement si \( v\) est une valeur propre de \( f\) pour la valeur propre \( \lambda\), alors \( v\in F_{\lambda}(f)\).

    En ce qui concerne l'invariance, remarquons que \( f\) commute avec \( f-\lambda\mtu\). Si \( x\in F_{\lambda}(f)\) il existe \( n\) tel que \( (f-\lambda\mtu)^nx=0\). Nous avons aussi
    \begin{equation}
        (f-\lambda\mtu)^nf(x)=f\big( (f-\lambda\mtu)^nx \big)=0,
    \end{equation}
    par conséquent \( f(x)\in F_{\lambda}(f)\).
\end{proof}

\begin{remark}  \label{RemBOGooCLMwyb}
    Toute matrice sur \( \eC\) n'est pas diagonalisable : nous en avons déjà donné une exemple simple en \ref{ExBRXUooIlUnSx}. Nous en voyons maintenant un moins simple. Considérons en effet l'endomorphisme \( f\) donné par la matrice
    \begin{equation}
        \begin{pmatrix}
            a&    \alpha    &   \beta    \\
            0    &   a    &   \gamma    \\
            0    &   0    &   b
        \end{pmatrix}
    \end{equation}
    où \( a\neq b\), \( \alpha\neq 0\), \( \beta\) et \( \gamma\) sont des nombres complexes quelconques.
    Son polynôme caractéristique est 
    \begin{equation}
        \chi_f(\lambda)=(a-\lambda)^2(b-\lambda)
    \end{equation}
    de telle façon à ce que les valeurs propres soient \( a\) et \( b\). Nous trouvons les vecteurs propres pour la valeur \( a\) en résolvant
    \begin{equation}
        \begin{pmatrix}
            a    &   \alpha    &   \beta    \\
            0    &   a    &   \gamma    \\
            0    &   0    &   b
        \end{pmatrix}\begin{pmatrix}
            x    \\ 
            y    \\ 
            z    
        \end{pmatrix}=\begin{pmatrix}
            ax    \\ 
            ay    \\ 
            az    
        \end{pmatrix}.
    \end{equation}
    L'espace propre \( E_a(f)\) est réduit à une seule dimension générée par \( (1,0,0)\). De la même façon l'espace propre correspondant à la valeur propre \( b\) est donné par 
    \begin{equation}
        \begin{pmatrix}
            \frac{1}{ b-a }\left( \beta+\frac{ \alpha\gamma }{ b-a } \right)    \\ 
            \frac{ \gamma }{ b-a }    \\ 
            1    
        \end{pmatrix}.
    \end{equation}
    Il n'y a donc pas trois vecteurs propres linéairement indépendants, et l'opérateur \( f\) n'est pas diagonalisable.

    Par contre nous pouvons voir que
    \begin{equation}
        (f-\alpha\mtu)^2\begin{pmatrix}
             0   \\ 
            1    \\ 
            0    
        \end{pmatrix}=
        \begin{pmatrix}
            a    &   \alpha    &   \beta    \\
            0    &   a    &   \gamma    \\
            0    &   0    &   b
        \end{pmatrix}
        \begin{pmatrix}
            \alpha    \\ 
            0    \\ 
            0    
        \end{pmatrix}-\begin{pmatrix}
            a\alpha    \\ 
            0    \\ 
            0    
        \end{pmatrix}=\begin{pmatrix}
            0    \\ 
            0    \\ 
            0    
        \end{pmatrix},
    \end{equation}
    de telle sorte que le vecteur \( (0,1,0)\) soit également dans l'espace caractéristique \( F_a(f)\).

    Dans cet exemple, la multiplicité algébrique de la racine \( a\) du polynôme caractéristique vaut \( 2\) tandis que sa multiplicité géométrique vaut seulement \( 1\).
\end{remark}

%--------------------------------------------------------------------------------------------------------------------------- 
\subsection{Théorèmes de décomposition}
%---------------------------------------------------------------------------------------------------------------------------

%TODO : Je crois qu'on peut remplacer l'hypothèse de corps algébriquement clos par le polynôme caractéristique scindé.
\begin{theorem}[Théorème spectral, décomposition primaire]\index{théorème!spectral}     \label{ThoSpectraluRMLok}
    Soit \( E\) espace vectoriel de dimension finie sur le corps algébriquement clos \( \eK\) et \( f\in\End(E)\). Alors
    \begin{equation}    \label{EqCTFHooBSGhYK}
        E=F_{\lambda_1}(f)\oplus\ldots\oplus F_{\lambda_k}(f)
    \end{equation}
    où la somme est sur les valeurs propres distinctes de \( f\).

    Les projecteurs sur les espaces caractéristique forment un système complet et orthogonal.
\end{theorem}
\index{décomposition!primaire}
\index{décomposition!spectrale}
\index{décomposition!sous-espaces caractéristiques}

\begin{proof}
    Soit \( P\) le polynôme caractéristique de \( f\) et une décomposition
    \begin{equation}
        P=(f-\lambda_1)^{\alpha_1}\ldots(f-\lambda_r)^{\alpha_r}
    \end{equation}
    en facteurs irréductibles. La le théorème de noyaux (\ref{ThoDecompNoyayzzMWod}) nous avons
    \begin{equation}        \label{EqDeFVSaYv}
        E=\ker(f-\lambda_1)^{\alpha_1}\oplus\ldots\oplus\ker(f-\lambda_r)^{\alpha_r}.
    \end{equation}
    Les projecteurs sont des polynômes en \( f\) et forment un système orthogonal. Il nous reste à prouver que \( \ker(f-\lambda_i)^{\alpha_i}=F_{\lambda_i}(f)\). L'inclusion
    \begin{equation}    \label{EqzmNxPi}
        \ker(f-\lambda_i)^{\alpha_i}\subset F_{\lambda_i}(f)
    \end{equation}
    est évidente. Nous devons montrer l'inclusion inverse. Prouvons que la somme des \( F_{\lambda_i}(f)\) est directe. Si \( v\in F_{\lambda_i}(f)\cap F_{\lambda_j}(f)\), alors il existe \( v_1=(f-\lambda_i)^nv\neq 0\) avec \( (f-\lambda_i)v_1=0\). Étant donné que \( (f-\lambda_i)\) commute avec \( (f-\lambda_j)\), ce \( v_1\) est encore dans \( F_{\lambda_j}(f)\) et par conséquent il existe \( w=(f-\lambda_j)^mv_1\) non nul tel que 
    \begin{subequations}
        \begin{numcases}{}
            (f-\lambda_i)w=0\\
            (f-\lambda_j)w=0.
        \end{numcases}
    \end{subequations}
    Ce \( w\) serait donc un vecteur propre simultané pour les valeurs propres \( \lambda_i\) et \( \lambda_j\), ce qui est impossible parce que les espaces propres sont linéairement indépendants. Les espaces \( F_{\lambda_i}\) sont donc en somme directe et
    \begin{equation}
        \sum_i\dim F_{\lambda_i}(f)\leq \dim E.
    \end{equation}
    En tenant compte de l'inclusion \eqref{EqzmNxPi} nous avons même
    \begin{equation}
        \dim E=\sum_i\dim\ker(f-\lambda_i)^{\alpha_i}\leq\sum_i F_{\lambda_i}(f)\leq \dim E.
    \end{equation}
    Par conséquent nous avons \( \dim\ker(f-\lambda_i)^{\alpha_i}=\dim F_{\lambda_i}(f)\) et l'égalité des deux espaces.
\end{proof}


\begin{probleme}
    Dans le cas où le corps n'est pas algébriquement clos, il paraît qu'il faut remplacer «diagonalisable» par «semi-simple».
\end{probleme}
%TODO : peut-être qu'il y a la réponse dans http://www.math.jussieu.fr/~romagny/agreg/dvt/endom_semi_simples.pdf

\begin{definition}
    Un endomorphisme d'un espace vectoriel est \defe{semi-simple}{semi-simple!endomorphisme} si tout sous-espace stable par \( u\) possède un supplémentaire stable.
\end{definition}
Si l'espace vectoriel est sur un corps algébriquement clos, alors les endomorphismes semi-simples sont les endomorphismes diagonaux.


%TODO : Je crois qu'on peut remplacer l'hypothèse de corps algébriquement clos par le polynôme caractéristique scindé.
\begin{theorem}[Décomposition de Dunford] \label{ThoRURcpW}
    Soit \( E\) un espace vectoriel sur le corps algébriquement clos \( \eK\) et \( u\in\End(E)\) un endomorphisme de \( E\). 
    
    \begin{enumerate}
        \item
            
            L'endomorphisme \( u\) se décompose de façon unique sous la forme
            \begin{equation}
                u=s+n
            \end{equation}
            où \( s\) est diagonalisable, \( n\) est nilpotent et \( [s,n]=0\).
        \item
            Les endomorphismes \( s\) et \( n\) sont des polynômes en \( u\) et commutent avec \( u\).
        \item   \label{ItemThoRURcpWiii}
            Les parties \( s\) et \( n\) sont données par
            \begin{subequations}
                \begin{align}
                    s&=\sum_i\lambda_ip_i\\
                    n&=\sum_i(s-\lambda_i\mtu)p_i
                \end{align}
            \end{subequations}
            où les sommes sont sur les valeurs propres distinctes\footnote{C'est à dire sur les sous-espaces caractéristiques.} de \( f\) et où \( p_i\colon E\to F_{\lambda_i}(u)\) est la projection de \( E\) sur \( F_{\lambda_i}(u)\).
    \end{enumerate}
\end{theorem}
\index{décomposition!Dunford}
\index{Dunford!décomposition}
\index{réduction!d'endomorphisme}
\index{endomorphisme!sous-espace stable}
\index{polynôme!d'endomorphisme!décomposition de Dunford}
\index{endomorphisme!diagonalisable!Dunford}
\index{endomorphisme!nilpotent!Dunford}
%TODO : comprendre comment on calcule des exponentielles de matrices avec Dunford.

\begin{proof}
    Le théorème spectral \ref{ThoSpectraluRMLok} nous indique que
    \begin{equation}
        E=\bigoplus_iF_{\lambda_i}(f).
    \end{equation}
    Nous considérons l'endomorphisme \( s\) de \( E\) qui consiste à dilater d'un facteur \( \lambda\) l'espace caractéristique \( F_{\lambda}(f)\) :
    \begin{equation}
        s=\sum_i\lambda_ip_i
    \end{equation}
    où \( p_i\colon E\to F_{\lambda_i}(u)\) est la projection de \( E\) sur \( F_{\lambda_i}(u)\).

    Nous allons prouver que \( [s,f]=0\) et \( n=f-s\) est nilpotent. Cela impliquera que \( [s,n]=0\).

    Si \( x\in F_{\lambda}(f)\), alors nous avons \( sf(x)=\lambda f(x)\) parce que \( f(x)\in F_{\lambda}(f)\) tandis que \( fs(x)=f(\lambda x)=\lambda f(x)\). Par conséquent \( f\) commute avec \( s\).

    Pour montrer que \( f-s\) est nilpotent, nous en considérons la restriction
    \begin{equation}
        f-s\colon F_{\lambda}(f)\to F_{\lambda}(f).
    \end{equation}
    Cet opérateur est égal à \( f-\lambda\mtu\) et est par conséquent nilpotent.

    Prouvons à présent l'unicité. Soit \( u=s'+n'\) une autre décomposition qui satisfait aux conditions : \( s'\) est diagonalisable, \( n'\) est nilpotent et \( [n',s']=0\). Commençons par prouver que \( s'\) et \( n'\) commutent avec \( u\). En multipliant \( u=s'+n'\) par \( s'\) nous avons
    \begin{equation}
        s'u=s'^2+s'n'=s'^2+n's'=(s'+n')s'=us',
    \end{equation}
    par conséquent \( [u,s']=0\). Nous faisons la même chose avec \( n'\) pour trouver \( [u,n']=0\). Notons que pour obtenir ce résultat nous avons utilisé le fait que \( n'\) et \( s'\) commutent, mais pas leur propriétés de nilpotence et de diagonalisibilité.
    
    
    Si \( s'+n'=s+n\) est une autre décomposition, \( s'\) et \( n'\) commutent avec \( u\), et par conséquent avec tous les polynômes en \( u\). Ils commutent en particulier avec \( n\) et \( s\). Les endomorphismes \( s\) et \( s'\) sont alors deux endomorphismes diagonalisables qui commutent. Par la proposition \ref{PropGqhAMei}, ils sont simultanément diagonalisables. Dans la base de simultanée diagonalisation, la matrice de l'opérateur \( s'-s=n-n'\) est donc diagonale. Mais \( n-n'\) est également nilpotent, en effet si \( A\) et \( B\) sont deux opérateurs nilpotents,
    \begin{equation}
        (A+B)^n=\sum_{k=0}^n\binom{k}{n}A^kB^{n-k}.
    \end{equation}
    Si \( n\) est assez grand, au moins un parmi \( A^k\) ou \( B^{n-k}\) est nul.

    Maintenant que \( n-n'\) est diagonal et nilpotent, il est nul et \( n=n'\). Nous avons alors immédiatement aussi \( s=s'\).

\end{proof}

%--------------------------------------------------------------------------------------------------------------------------- 
\subsection{Diverses conséquences}
%---------------------------------------------------------------------------------------------------------------------------

\begin{theorem}
    Soit une matrice \( A\in \eM(n,\eC)\). On a que la suite \( (A^kx)\) tends vers zéro pour tout \( x\) si et seulement si \( \rho(A)<1\) où \( \rho(A)\)\index{rayon!spectral} est le rayon spectral de $A$
\end{theorem}
\index{décomposition!Dunford!exponentielle de matrice}

\begin{proof}
    Dans le sens direct, il suffit de prendre comme \( x\), un vecteur propre de \( A\). Dans ce cas nous avons \( A^kx=\lambda^kx\). Mais \( \lambda^kx\) ne tend vers zéro que si \( \lambda<1\). Donc toute les valeurs propres de \( A\) doivent être plus petite que \( 1\) et \( \rho(A)<1\).

    Pour l'autre sens nous utilisons la décomposition de Dunford (théorème \ref{ThoRURcpW}) : il existe une matrice inversible \( P\) telle que
    \begin{equation}
        A=P^{-1}(D+N)P
    \end{equation}
    où \( D\) est diagonale, \( N\) est nilpotente et \( [D,N]=0\). Étant donné que \( D+N\) est triangulaire, son polynôme caractéristique que
    \begin{equation}
        \chi_{D+N}(\lambda)=\prod_i D_{ii}-\lambda.
    \end{equation}
    Par similitude, c'est le même polynôme caractéristique que celui de \( A\) et nous savons alors que la diagonale de \( D\) contient les valeurs propres de \( A\).

    Par ailleurs nous avons
    \begin{subequations}
        \begin{align}
            A^k&=P^{-1}(D+N)^kP\\
            &=P^{-1}\sum_{j=0}^k{j\choose k}D^{j-k}N^jP\\
            &=P^{-1}\sum_{j=0}^{n-1}{j\choose k}D^{j-k}N^jP
        \end{align}
    \end{subequations}
    où nous avons utilité le fait que \( D\) et \( N\) commutent ainsi que \( N^{n-1}=0\) parce que \( N\) est nilpotente. Nous utilisons la norme matricielle usuelle, pour laquelle \( \| D \|=\rho(D)=\rho(A)\). Nous avons alors
    \begin{equation}
        \| (D+N)^k \|\leq \sum_{j=0}^k{j\choose k}\rho(D)^{k-j}\| N \|^j.
    \end{equation}
    Du coup si \( \rho(D)<1\) alors \( \| (D+N)^k \|\to 0\) (et c'est même un si et seulement si).
\end{proof}

Une application de la décomposition de Jordan est l'existence d'un logarithme pour les matrices. La proposition suivant va d'une certaine manière donner un logarithme pour les matrices inversibles complexes. Dans le cas des matrices réelles \( m\) telles que \( \| m-\mtu \|<1\), nous donnerons au lemme \ref{LemQZIQxaB} une formule pour le logarithme sous forme d'une série; ce logarithme sera réel.
\begin{proposition} \label{PropKKdmnkD}
    Toute matrice inversible complexe est une exponentielle.
\end{proposition}
\index{exponentielle!de matrice}
\index{décomposition!Jordan!et exponentielle de matrice}

\begin{proof}
    Soit \( A\in \GL(n,\eC)\); nous allons donner une matrice \( B\in \eM(n,\eC)\) telle que \( A=\exp(B)\). D'abord remarquons qu'il suffit de prouver le résultat pour une matrice par classe de similitude. En effet si \( A=\exp(B)\) et si \( M\) est inversible alors 
    \begin{subequations}    \label{EqqACuGK}
        \begin{align}
            \exp(MBM^{-1})&=\sum_k\frac{1}{ k! }(MBM^{-1})^k\\
            &=\sum_k\frac{1}{ k! }MB^kM^{-1}\\
            &=M\exp(B)M^{-1}.
        \end{align}
    \end{subequations}
    Donc \( MAM^{-1}=\exp(MBM^{-1})\). Nous pouvons donc nous contenter de trouver un logarithme pour les blocs de Jordan. Nous supposons donc que \( A=(\mtu+N)\) avec \( N^m=0\). En nous inspirant de \eqref{EqweEZnV}, nous posons
    \begin{equation}
        D(t)=tN-\frac{ t^2 }{ 2 }N^2+\ldots +(-1)^m\frac{ t^{m-1} }{ m-1 }N^{m-1}
    \end{equation}
    et nous allons prouver que \(  e^{D(1)}=\mtu+N\). Notons que \( N\) étant nilpotente, cette somme ainsi que toutes celles qui viennent sont finies. Il n'y a donc pas de problèmes de convergences dans cette preuve (si ce n'est les passages des équations \eqref{EqqACuGK}).

    Nous posons \( S(t)= e^{D(t)}\) (la somme est finie), et nous avons
    \begin{equation}
        S'(t)=D'(t) e^{D(t)}
    \end{equation}
    Afin d'obtenir une expression qui donne \( S'\) en termes de \( S\), nous multiplions par \( (\mtu+tN)\) en remarquant que \( (\mtu+tN)D'(t)=N\) nous avons
    \begin{equation}
        (\mtu+tN)S'(t)=NS(t).
    \end{equation}
    En dérivant à nouveau,
    \begin{equation}    \label{EqKjccqP}
        (\mtu+tN)S''(t)=0.
    \end{equation}
    La matrice \( (\mtu+tN)\) est inversible parce que son noyau est réduit à \( \{ 0 \}\). En effet si \( (\mtu+tN)x=0\), alors \( Nx=-\frac{1}{ t }x\), ce qui est impossible parce que \( N\) est nilpotente. Ce que dit l'équation \eqref{EqKjccqP} est alors que \( S''(t)=0\). Si nous développons \( S(t)\) en puissances de \( t\) nous nous arrêtons au terme d'ordre \( 1\) et nous avons
    \begin{equation}
        S(t)=S(0)+tS'(0)=\mtu+tD'(0)=1+tN.
    \end{equation}
    En \( t=1\) nous trouvons \( S(1)=\mtu+N\). La matrice \( D(1)\) donnée est donc bien un logarithme de $\mtu+N$.
\end{proof}

\begin{proposition}[\cite{fJhCTE}]
    Si \( A\in \eM(n,\eC)\) est telle que \( \rho(A)<1\), alors \( A^n\to 0\).
\end{proposition}

\begin{proof}
    Nous nous plaçons dans une base des espaces caractéristiques de \( A\), c'est à dire que nous supposons que la matrice \( A\) a la forme
    \begin{equation}        \label{EqWMvkgLo}
        A=\begin{pmatrix}
            \lambda_1\mtu+N_1    &       &       \\
                &   \ddots    &       \\
                &       &   \lambda_s\mtu+N_s
        \end{pmatrix}
    \end{equation}
    où les \( \lambda_i\) sont les valeurs propres de \( A\) et les \( N_i\) sont nilpotentes. En effet nous savons que l'espace caractéristique \( F_{\lambda_i}\) est l'espace de nilpolence de \( A-\lambda_i\mtu\). Si nous notons \( A_i\) la restriction de \( A\) à cet espace, la matrice \( N_i=A_i-\lambda_i\mtu\) est nilpotente. Du coup \( A_i=\lambda_I\mtu+N_i\) et nous avons bien la décomposition \eqref{EqWMvkgLo}.

    Nous avons donc \( A^n\to 0\) si et seulement si \( (N_i+\lambda_i\mtu)^n\to 0\) pour tout \( i\). Soit donc \( N\) nilpotente et \( \lambda<1\) (parce que nous savons que toutes les valeurs propres de \( A\) sont inférieures à un). Nous avons
    \begin{equation}
            (\lambda\mtu+N)^n=\sum_{k=0}^n\binom{ n }{ k }\lambda^{n-k}N^{k}
            =\sum_{k=0}^{r-1}\binom{ n }{ k }\lambda^{n-k}N^{k}.
    \end{equation}
    Nous voyons que le nombre de termes dans la somme ne dépend pas de \( n\). De plus pour chacun de termes, la puissance de \( N\) ne dépend pas non plus de \( n\). Le terme
    \begin{equation}
        \binom{ n }{ k }\lambda^{n-k}\leq P(n)\lambda^{n-k}
    \end{equation}
    où \( P\) est un polynôme tend vers zéro lorsque \( n\) devient grand parce que c'est une cas polynôme fois exponentielle.
\end{proof}

%--------------------------------------------------------------------------------------------------------------------------- 
\subsection{Diagonalisabilité d'exponentielle}
%---------------------------------------------------------------------------------------------------------------------------

\begin{proposition}[\cite{fJhCTE}]      \label{PropCOMNooIErskN}
    Si \( A\in \eM(n,\eR)\) a un polynôme caractéristique scindé, alors \( A\) est diagonalisable si et seulement si \( e^A\) est diagonalisable.
\end{proposition}
\index{décomposition!Dunford!application}
\index{exponentielle!de matrice}
\index{diagonalisable!exponentielle}

\begin{proof}
    Si \( A\) est diagonalisable, alors il existe une matrice inversible \( M\) telle que \( D=M^{-1}AM\) soit diagonale (c'est la définition \ref{DefCNJqsmo}). Dans ce cas nous avons aussi \( (M^{-1}AM)^k=M^{-1}A^kM\) et donc \( M^{-1}e^AM=e^{M^{-1}AM}=e^D\) qui est diagonale.

    La partie difficile est donc le contraire. 
    
    \begin{subproof}
        \item[Qui est diagonalisable et comment ?]
            Nous supposons que \( e^A\) est diagonalisable et nous écrivons la décomposition de Dunford (théorème \ref{ThoRURcpW}) :
            \begin{equation}
                A=S+N
            \end{equation}
            où \( S\) est diagonalisable, \( N\) est nilpotente, \( [S,N]=0\). Nous avons besoin de prouver que \( N=0\).
    
            Les matrices \( A\) est \( S\) commutent; en passant au développement nous en déduisons que \( A\) et \( e^S\) commutent, puis encore en passant au développement que \( e^A\) et \( e^S\) commutent. Vu que \( S\) est diagonalisable, \( e^S\) l'est et par hypothèse \( e^A\) est également diagonalisable. Donc \( e^A\) et \( e^{-S}\) sont simultanément diagonalisables par la proposition \ref{PropGqhAMei}.

            Étant donné que \( A\) et \( S\) commutent, nous avons \( e^N=e^{A-S}=e^Ae^{-S}\), et nous en déduisons que \( e^N\) est diagonalisable vu que les deux facteurs \( e^A\) et \( e^{-S}\) sont simultanément diagonalisables.

        \item[Unipotence]

            Si \( r\) est le degré de nilpotence de \( N\), nous avons
            \begin{equation}    \label{EqQHjvLZQ}
                e^N-\mtu=N+\frac{ N^2 }{2}+\ldots +\frac{ N^{r-1} }{ (r-1)! }.
            \end{equation}
            Donc
            \begin{equation}
                (e^N-\mtu)^k=\left( N+\frac{ N^2 }{2}+\ldots +\frac{ N^{r-1} }{ (r-1)! } \right)^k
            \end{equation}
            où le membre de droite est un polynôme en \( N\) dont le terme de plus bas degré est de degré \( k\). Donc \( (e^N-\mtu)\) est nilpotente et \( e^N\) est unipotente.

            Si \( M\) est la matrice qui diagonalise \( e^N\), alors la matrice diagonale \( M^{-1}e^NM\) est tout autant unipotente que \( e^N\) elle-même. En effet,
            \begin{subequations}
                \begin{align}
                    (M^{-1}e^NM-\mtu)^r&=\sum_{k=0}^r\binom{ r }{ k }(-1)^{r-k}M^{-1}(e^N)^kM\\
                    &=M^{-1}\left( \sum_{k=0}^r\binom{ r }{ k }(-1)^{r-k}(e^N)^k \right)M\\
                    &=M^{-1}(e^N-\mtu)^rM\\
                    &=0.
                \end{align}
            \end{subequations}

            La matrice \( M^{-1}e^NM\) est donc une matrice diagonale et unipotente; donc \( M^{-1}e^NM=\mtu\), ce qui donne immédiatement que \( e^N=\mtu\).

        \item[Polynômes annulateurs]

            En reprenant le développement \eqref{EqQHjvLZQ} sachant que \( e^N=\mtu\), nous savons que
            \begin{equation}
                N+\frac{ N^2 }{2}+\ldots +\frac{ N^{r-1} }{ (r-1)! }=0.
            \end{equation}
            Dit en termes pompeux (mais non moins porteurs de sens), le polynôme
            \begin{equation}
                Q(X)=X+\frac{ X^2 }{2}+\ldots +\frac{ X^{r-1} }{ (r-1)! }
            \end{equation}
            est un polynôme annulateur de \( N\).
            
            La proposition \ref{PropAnnncEcCxj} stipule que le polynôme minimal d'un endomorphisme divise tous les polynômes annulateurs. Dans notre cas, \( X^r\) est un polynôme annulateur et donc le polynôme minimal de \( N\) est de la forme \( X^k\). Donc il est \( X^r\) lui-même.
            
            Nous avons donc \( X^r\divides Q\). Mais \( Q\) est un polynôme contenant le monôme \( X\) donc \( X^r\) ne peut diviser \( Q\) que si \( r=1\). Nous en concluons que \( X\) est un polynôme annulateur de \( N\). C'est à dire que \( N=0\).

        \item[Conclusion]

            Vu que Dunford\footnote{Théorème \ref{ThoRURcpW}.} dit que \( A=S+N\) et que nous venons de prouver que \( N=0\), nous concluons que \( A=S\) avec \( S\) diagonalisable.

    \end{subproof}
\end{proof}

%---------------------------------------------------------------------------------------------------------------------------
\subsection{Valeurs singulières}
%---------------------------------------------------------------------------------------------------------------------------

\begin{definition}
    Soit \( M\) une matrice \( m\times n\) sur \( \eK\) (\( \eK\) est \( \eR\) ou \( \eC\)). Un nombre réel \( \sigma\) est une \defe{valeur singulière}{valeur!singulière} de \( M\) si il existent des vecteurs unitaires \( u\in \eK^m\), \( v\in \eK^n\) tels que
    \begin{subequations}
        \begin{align}
            Mv&=\sigma u\\
            M^*u&=\sigma v.
        \end{align}
    \end{subequations}
\end{definition}

\begin{theorem}[Décomposition en valeurs singulières]
    Soit \( M\in \eM(m\times n,\eK)\) où \( \eK=\eR,\eC\). Alors \( M\) se décompose en
    \begin{equation}
        M=ADB
    \end{equation}
    où
    il existe deux matrices unitaires \( A\in \gU(m\times m)\), \( B\in \gU(n\times n)\) et une matrice (pseudo)diagonale \( D\in \eM(m\times n)\) tels que
    \begin{enumerate}
        \item 
            \( A\in\gU(m\times m)\), \( B\in\gU(n\times n)\) sont deux matrices unitaires;,
        \item
            \( D\) est (pseudo)diagonale,
        \item
            les éléments diagonaux de \( D\) sont les valeurs singulières de \( M\),
        \item
            le nombre d'éléments non nuls sur la diagonale de \( D\) est le rang de \( M\).
    \end{enumerate}
\end{theorem}

\begin{corollary}
    Soit \( M\in \eM(n,\eC)\). Il existe un isomorphisme \( f\colon \eC^n\to \eC^n\) tel que \( fM\) soit autoadjoint.
\end{corollary}

\begin{proof}
    Si \( M=ADB\) est la décomposition de \( M\) en valeurs singulières, alors nous pouvons prendre \( f=\overline{ B }^tA^{-1}\) qui est une matrice inversible. Pour la vérification que ce \( f\) répond bien à la question, ne pas oublier que \( D\) est réelle, même si \( M\) ne l'est pas.
\end{proof}

%+++++++++++++++++++++++++++++++++++++++++++++++++++++++++++++++++++++++++++++++++++++++++++++++++++++++++++++++++++++++++++ 
\section{Endomorphismes nilpotents et trigonalisables}
%+++++++++++++++++++++++++++++++++++++++++++++++++++++++++++++++++++++++++++++++++++++++++++++++++++++++++++++++++++++++++++

%---------------------------------------------------------------------------------------------------------------------------
\subsection{Endomorphismes nilpotents}
%---------------------------------------------------------------------------------------------------------------------------

La \defe{trace}{trace!matrice} d'une matrice \( A\in \eM(n,\eK)\) est la somme de ses éléments diagonaux :
\begin{equation}
    \tr(A)=\sum_{i=1}^nA_{ii}.
\end{equation}
Une propriété importante est son invariance cyclique.
\begin{lemma}   \label{LemhbZTay}
    Si \( A\) et \( B\) sont des matrices carré, alors \( \tr(AB)=\tr(BA)\).

    La trace est un invariant de similitude.
\end{lemma}

\begin{proof}
    C'est un simple calcul :
    \begin{equation}
            \tr(AB)=\sum_{ik}A_{ik}B_{ki}
            =\sum_{ik}A_{ki}B_{ik} 
            =\sum_{ik}B_{ik}A_{ki}
            =\sum_i(BA)_{ii}
            =\tr(BA)
    \end{equation}
    où nous avons simplement renommé les indices \( i\leftrightarrow k\).

    En particulier, la trace est un invariant de similitude parce que \( \tr(ABA^{-1})=\tr(A^{-1} AB)=\tr(B)\). 
\end{proof}
La trace étant un invariant de similitude, nous pouvons donc définir la \defe{trace}{trace!endomorphisme} comme étant la trace de sa matrice dans une base quelconque. Si la matrice est diagonalisable, alors la trace est la somme des valeurs propres.

\begin{lemma}[\cite{fJhCTE}]   \label{LemzgNOjY}
    L'endomorphisme \( u\in\End(\eC^n)\) est nilpotent si et seulement si \( \tr(u^p)=0\) pour tout \( p\).
\end{lemma}

\begin{proof}
    Supposons que \( u\) est nilpotent. Alors ses valeurs propres sont toutes nulles et celles de \( u^p\) le sont également. La trace étant la somme des valeurs propres, nous avons alors tout de suite \( \tr(u^p)=0\).

    Supposons maintenant que \( \tr(u^p)=0\) pour tout \( p\). Le polynôme caractéristique \eqref{Eqkxbdfu} est
    \begin{equation}    \label{EqfnCqWq}
        \chi_u=(-1)^nX^{\alpha}(X-\lambda_1)^{\alpha_1}\ldots (X-\alpha_r)^{\alpha_r}.
    \end{equation}
    où les \( \lambda_i\) (\( i=1,\ldots, r\)) sont les valeurs propres non nulles distinctes de \( u\).

    Il est vite vu que le coefficient de \( X^{n-1}\) dans \( \chi_u\) est \( -\tr(u)\) parce que le coefficient de \( X^{n-1}\) se calcule en prenant tous les $X$ sauf une fois \( -\lambda_i\). D'autre part le polynôme caractéristique de \( u^p \) est le même que celui de \( u\), en remplaçant \( \lambda_i\) par \( \lambda_i^p\); cela est dû au fait que si \( v\) est vecteur propre de valeur propre \( \lambda\), alors \( u^pv=\lambda^pv\).

    Par l'équation \eqref{EqfnCqWq}, nous voyons que le coefficient du terme \( X^{n-1}\) dans les polynôme caractéristique est 
    \begin{equation}        \label{eqSoDSKH}
        0=\tr(u^p)=\alpha_1\lambda_1^p+\ldots +\alpha_r\lambda_r^p.
    \end{equation}
    Donc les nombres \( (\alpha_1,\ldots, \alpha_r)\) est une solution non triviale\footnote{Si \( \alpha_1=\ldots=\alpha_r=0\), alors les valeurs propres sont toutes nulles et la matrice est en réalité nulle dès le départ.} du système
    \begin{subequations}    \label{EqDpvTnu}
        \begin{numcases}{}
            \alpha_1X_1+\ldots +\lambda_rX_r=0\\
            \qquad\vdots\\
            \lambda^r_1X_1+\ldots +\lambda_r^rX_r=0.
        \end{numcases}
    \end{subequations}
    Cela sont les équations \eqref{eqSoDSKH} écrites avec \( p=1,\ldots, r\). Le déterminant de ce système est
    \begin{equation}
        \lambda_1\ldots\lambda_r\det\begin{pmatrix}
             1   &   \ldots    &   1    \\
             \lambda_1   &   \ldots    &   \lambda_1    \\
             \vdots   &       &   \vdots    \\ 
             \lambda_1^{r-1}   &   \ldots    &   \lambda_r^{r-1}
         \end{pmatrix}\neq 0,
    \end{equation}
    qui est un déterminant de Vandermonde (proposition \ref{PropnuUvtj}) valant
    \begin{equation}
        0=\lambda_1\ldots\lambda_r\prod_{1\leq i\leq j\leq r}(\lambda_i-\lambda_j).
    \end{equation}
    Étant donné que les \( \lambda_i\) sont distincts et non nuls, nous avons une contradiction et nous devons conclure que \( (\alpha_1,\ldots, \alpha_r)\) était une solution triviale du système \eqref{EqDpvTnu}.
\end{proof}

\begin{proposition}[\cite{SVSFooIOYShq}]    \label{PropMWWJooVIXdJp}
    Soit un \( \eK\)-espace vectoriel \( E\). Un endomorphisme \( u\in\End(E)\) est nilpotent si et seulement si il existe une base de \( E\) dans laquelle la matrice de \( u\) est strictement triangulaire supérieure.
\end{proposition}

\begin{proof}
    \begin{subproof}
       \item[\( \Rightarrow\)]
           Nous faisons la démonstration par récurrence sur la dimension de \( E\). Lorsque \( n=1\) nous avons \( u=(a)\) avec \( a\in \eK\). Vu que \( a^k=0\) pour un certain \( k\) nous avons \( a=0\) parce qu'un corps est toujours un anneau intègre\footnote{Lemme \ref{LemAnnCorpsnonInterdivzer}.}. 

           Lorsque \( \dim(E)=n\) nous savons que \( u\) a un noyau non réduit au vecteur nul (parce qu'il est nilpotent). Soit donc un vecteur non nul \( x\in\ker(u)\) et une base
           \begin{equation}
               \{ x,e_2,\ldots, e_n \}
           \end{equation}
           donnée par le théorème de la base incomplète \ref{ThonmnWKs}. La matrice de \( u\) dans cette base s'écrit
           \begin{equation}
               \begin{pmatrix}
                       \begin{array}[]{c|c}
                           0&\begin{matrix} 
                               * &   *    &   *    
                           \end{matrix}\\
                           \hline
                           \begin{matrix}
                               0 \\ 
                               0 \\ 
                               0 
                           \end{matrix}&
                           \begin{pmatrix}
                                &       &       \\
                                &   A    &       \\
                                &       &   
                           \end{pmatrix}
                       \end{array}
               \end{pmatrix}.
           \end{equation}
           Un tout petit peu de calcul de produit de matrice montre que la matrice de \( u^k\) est de la forme
           \begin{equation}
               \begin{pmatrix}
                       \begin{array}[]{c|c}
                           0&\begin{matrix} 
                               * &   *    &   *    
                           \end{matrix}\\
                           \hline
                           \begin{matrix}
                               0 \\ 
                               0 \\ 
                               0 
                           \end{matrix}&
                           \begin{pmatrix}
                                &       &       \\
                                &   A^k    &       \\
                                &       &   
                           \end{pmatrix}
                       \end{array}
               \end{pmatrix}.
           \end{equation}
           Étant donné que \( u\) est nilpotente, la matrice \( A\) l'est aussi. L'hypothèse de récurrence dit alors que \( A\) est strictement triangulaire supérieure (ou en tout cas peut le devenir par un changement de base adéquat).

       \item[\( \Leftarrow\)]

            Lorsqu'une matrice est triangulaire supérieure stricte, elle applique
            \begin{equation}
                \Span\{ e_1,\ldots, e_k \}\to\Span\{ e_1,\ldots, e_{k-1} \}.
            \end{equation}
            Donc tout vecteur finit sur zéro si on lui applique \( u\) assez souvent.
    \end{subproof}
\end{proof}



\begin{proposition}
    Soit \( A\in\GL(n,\eC)\). La suite \( (A^k)_{k\in \eZ}\) est bornée si et seulement si \( A\) est diagonalisable et \( \Spec(A)\subset \gS^1\).
\end{proposition}

\begin{proof}
    Si \( A\) est diagonalisable avec les valeurs propres \( \lambda_i\) de norme \( 1\) dans \( \eC\), alors \( A^k\) est la matrice diagonale avec les \( \lambda_i^k\) sur la diagonale. Cela reste borné pour toute valeur entière de \( k\).

    En ce qui concerne l'autre sens, nous supposons encore que
    \begin{equation}
        A=\begin{pmatrix}
            \lambda_1\mtu+N_1    &       &       \\
                &   \ddots    &       \\
                &       &   \lambda_s\mtu+N_s
        \end{pmatrix},
    \end{equation}
    et nous regardons un des blocs. Nous voulons prouver que \( N=0\) et que \( | \lambda |=1\).
    
    Nous commençons par regarder ce qu'implique le fait que \( (\lambda \mtu+N)^n\) reste borné pour \( n>0\). En notant \( r\) l'ordre de nilpotence de \( N\), nous avons le développement
    \begin{equation}
        (\lambda\mtu+N)^n=\sum_{k=0}^{r-1}\binom{ n }{ k }N^k\lambda^{n-k}.
    \end{equation}
    Par la proposition \ref{PropMWWJooVIXdJp}, une matrice nilpotente s'écrit dans une base sous la forme
    \begin{equation}
        N=\begin{pmatrix}
             0   &   1    &       &       \\
                &   0    &   1    &       \\
                & &   \ddots   &   \ddots    &      \\ 
                &&       &   0    &   1     \\
                &&       &      &   0     
         \end{pmatrix}
    \end{equation}
    et effectuer \( A^k\) revient à décaler la diagonale de \( 1\). Donc la famille
    \begin{equation}
        \{ \mtu,N,\ldots, N^{r-1} \}
    \end{equation}
    est libre. Par conséquent la suite \( (\lambda\mtu+N)^n\) restera bornée si et seulement si chacun des termes 
    \begin{equation}    \label{EqXRDVDCM}
        \binom{ n }{ k }N^k\lambda^{n-k}
    \end{equation}
    reste borné. Le premier terme étant \( \lambda^n\mtu\), nous avons obligatoirement \( | \lambda |\leq 1\). Si \( | \lambda |<1\), alors le coefficient \( \binom{ n }{ k }\lambda^{n-k}\) tend vers zéro. Si \( | \lambda |=1\) par contre ce coefficient tend vers l'infini et la seule façon pour que \eqref{EqXRDVDCM} reste borné est que \( N=0\). Nous avons donc deux possibilités :
    \begin{itemize}
        \item \( | \lambda |<1\)
        \item \( | \lambda |=1\) et \( N=0\).
    \end{itemize}

    Nous nous tournons maintenant sur la contrainte que \( (\lambda\mtu+N)^n\) doive rester borné pour \( n<0\). Nous avons
    \begin{equation}
        \lambda\mtu+N=\lambda(\mtu+\lambda^{-1}N),
    \end{equation}
    et nous pouvons appliquer la proposition \ref{PropQAjqUNp} à l'opérateur nilpotent \( -\lambda^{-1} N\) pour avoir
    \begin{equation}
        (\mtu+\lambda^{-1}N)^{-1}=\mtu+\sum_{k=1}^{\infty}(-\lambda)^{-1}N^k.
    \end{equation}
    Ceci pour dire que \( (\lambda\mtu+N)^{-1}=\lambda^{-1}(\mtu+\lambda^{-1}N')\) pour une autre matrice nilpotente \( N'\). Le travail déjà fait, appliqué à \( \lambda^{-1}\) et \( N'\), nous donne deux possibilités :
    \begin{itemize}
        \item \( | \lambda^{-1} |<1\)
        \item \( | \lambda^{-1} |=1\) et \( N'=0\).
    \end{itemize}
    La possibilité \( | \lambda^{-1} |<1\) est exclue parce qu'elle impliquerait \( | \lambda |>1\) qui avait déjà été exclu. Il ne reste donc que la possibilité \( | \lambda |=1\) et \( N=N'=0\).
\end{proof}

%--------------------------------------------------------------------------------------------------------------------------- 
\subsection{Endomorphismes trigonalisables}
%---------------------------------------------------------------------------------------------------------------------------

\begin{definition}[\cite{MQMKooPBfnZN}]
    Une matrice dans \( \eM(n,\eK)\) est \defe{trigonalisable}{matrice!trigonalisable} lorsqu'elle est semblable\footnote{Définition \ref{DefCQNFooSDhDpB}.} à une matrice triangulaire supérieure.
\end{definition}

\begin{proposition} \label{PropKNVFooQflQsJ}
    Soit \( u\) un endomorphisme d'un espace vectoriel \( E\) sur le corps \( \eK\). Les faits suivants sont équivalents.
    \begin{enumerate}
        \item   \label{ItemZKDMooOrTHkwi}
            L'endomorphisme \( u\) est trigonalisable (auquel cas les valeurs propres sont sur la diagonale).
        \item   \label{ItemZKDMooOrTHkwii}
            Le polynôme caractéristique de \( u\) est scindé\footnote{Définition \ref{DefCPLSooQaHJKQ}.}.
    \end{enumerate}
\end{proposition}
\index{trigonalisation!et polynôme caractéristique}

\begin{proof}
    \begin{subproof}
        \item[\ref{ItemZKDMooOrTHkwii}\( \Rightarrow\)\ref{ItemZKDMooOrTHkwi}]
            Nous avons par hypothèse que
            \begin{equation}
                \chi_u(X)=\prod_{i=1}^r(X-\lambda_i)^{\alpha_i}
            \end{equation}
            où les \( \lambda_i\) sont les valeurs propres de \( u\). Le théorème de Cayley-Hamilton \ref{ThoCalYWLbJQ} dit que \( \chi_u(u)=0\), ce qui permet d'utiliser le théorème de décomposition des noyaux \ref{ThoDecompNoyayzzMWod} :
            \begin{equation}
                E=\ker(X-\lambda_1)^{\alpha_1}\oplus\ldots\oplus\ker(X-\lambda_r)^{\alpha_r}.
            \end{equation}
            Les espaces \( F_{\lambda_i}(u)=\ker(X-\lambda_i)^{\alpha_i}\) sont les espaces caractéristiques de \( u\), ce qui fait que \( u-\lambda_i\mtu\) est nilpotent sur \( F_{\lambda_i}(u)\). L'endomorphisme \( u-\lambda_i\mtu\) est donc strictement trigonalisable supérieur sur son bloc\footnote{Proposition \ref{PropMWWJooVIXdJp}.}. Cela signifie que \( u\) est triangulaire supérieure avec les valeurs propres sur la diagonale.

        \item[\ref{ItemZKDMooOrTHkwi}\( \Rightarrow\)\ref{ItemZKDMooOrTHkwii}]

            C'est immédiat parce que le déterminant d'une matrice triangulaire est le produit des éléments de sa diagonale.
    \end{subproof}
\end{proof}

\begin{remark}  \label{RemXFZTooXkGzQg}
    Si \( \eK\) est algébriquement clos (comme \( \eC\) par exemple), alors tous les polynômes sont scindés et toutes les matrices sont trigonalisables\footnote{Le lemme de Schur complexe \ref{LemSchurComplHAftTq} va un peu plus loin et précise que la trigonalisation peut être faite par une matrice unitaire.}. Un exemple un peu simple de cela est la matrice
    \begin{equation}
        u=\begin{pmatrix}
            0    &   -1    \\ 
            1    &   0    
        \end{pmatrix}.
    \end{equation}
    Le polynôme caractéristique est \( \chi_u(X)=X^2+1\) et les valeurs propres sont \( \pm i\). Il est vite vu que dans la base
    \begin{equation}
        \{ \begin{pmatrix}
        i    \\ 
    1    
\end{pmatrix}, \begin{pmatrix}
1    \\ 
i    
\end{pmatrix}\}
    \end{equation}
    de \( \eC^2\), la matrice \( u\) se note \( \begin{pmatrix}
        i    &   0    \\ 
        0    &   -i    
    \end{pmatrix}\).
\end{remark}

\begin{remark}  \label{RemREOSooGEDJWX}
    Cela nous donne une autre façon de prouver qu'une matrice nilpotente de \( \eM(n,\eC)\) ou \( \eM(n,\eR)\) est trigonalisable\cite{KDUFooVxwqlC}. D'abord dans \( \eM(n,\eC)\), toutes les matrices sont trigonalisables\footnote{Parce que le polynôme caractéristique est scindé, voir remarque \ref{RemXFZTooXkGzQg}.}, et les valeurs propres arrivent sur la diagonale. Mais comme les valeurs propres d'une matrice nilpotente sont zéro, elle est triangulaire stricte. Par ailleurs son polynôme caractéristique est alors \( X^n\).

    Ensuite si \( u\in \eM(n,\eR)\) nous pouvons voir \( u\) comme une matrice dans \( \eM(n,\eC)\) et y calculer son polynôme caractéristique qui sera tout de même \( X^n\). Ce polynôme étant scindé, la proposition \ref{PropKNVFooQflQsJ} nous assure que \( u\) est trigonalisable. Une fois de plus, les valeurs propres étant sur la diagonale, elle est triangulaire supérieure stricte.
\end{remark}

\begin{remark}
    La méthode des pivots de Gauss\footnote{Le lemme \ref{LemZMxxnfM}.} certes permet de trigonaliser n'importe quoi, mais elle ne correspond pas à un changement de base. Autrement dit, les pivots de Gauss ne sont pas de similitudes.

    C'est là qu'il faut bien avoir en tête la différence entre \emph{équivalence} et \emph{similarité} (définition \ref{DefBLELooTvlHoB}). Lorsqu'on parle de changement de base, de matrice trigonalisable ou diagonalisable, nous parlons de similarité et non d'équivalence.
\end{remark}

%--------------------------------------------------------------------------------------------------------------------------- 
\subsection{Théorème de Burnside}
%---------------------------------------------------------------------------------------------------------------------------

\begin{theorem}[Burnside\cite{fJhCTE}]\label{ThooJLTit}
    Un sous-groupe de \( \GL(n,\eC)\) est fini si et seulement si il est d'exposant fini.
\end{theorem}
\index{exposant}
\index{racine!de l'unité}
\index{endomorphisme!diagonalisable}

\begin{proof}
    Soit \( G\) un sous-groupe de \( \GL(n,\eC)\). Si \( G\) est fini, l'ordre de ses éléments divise \( | G |\) (corollaire \ref{CorpZItFX}) au théorème de Lagrange et l'exposant est le PPCM qui est donc fini également.

    Nous supposons maintenant que l'ordre de \( G\) est fini. Nous notons \( e\) l'exposant de \( G\). En particulier, tous les éléments de \( G\) sont des racines du polynôme \( X^e-1\).
    
    \begin{subproof}
        \item[Générateurs]

            Le groupe \( G\) est une partie de \( \eM(n,\eC)\) dont nous considérons l'algèbre engendrée (définition \ref{DefkAXaWY}) \( \mG\). Soit \( C_1,\ldots, C_r\) une famille génératrice de \( \mG\) constituée d'éléments de \( G\) et la fonction
            \begin{equation}
                \begin{aligned}
                    \tau\colon G&\to \eC^r \\
                    A&\mapsto \big( \tr(AC_1),\ldots, \tr(AC_r) \big). 
                \end{aligned}
            \end{equation}

        \item[\( \tau\) est injective] Supposons que \( \tau(A)=\tau(B)\). Alors pour tout générateur \( C_i\) nous avons \( \tr(AC_i)=\tr(BC_i)\) et par linéarité de la trace, nous avons
            \begin{equation}    \label{EqnCYmKW}
                \tr(AM)=\tr(BM)
            \end{equation}
            pour tout \( M\in G\). Notons par ailleurs
            \begin{equation}
                N=AB^{-1}-\mtu,
            \end{equation}
            qui est diagonalisable parce que \( AB^{-1}\in G\) et donc est annulé par le polynôme \( X^e-1\) qui est scindé à racines simples. Du coup \( AB^{-1}\) est diagonalisable; posons \( PAB^{-1}P^{-1}=D\), alors \( P\big( AB^{-1}-\mtu \big)P^{-1}=D-\mtu\) qui est encore diagonale. Donc \( N\) est diagonalisable.

            Par ailleurs nous avons
            \begin{subequations}
                \begin{align}
                    \tr\big( (AB^{-1})^p \big)&=\tr\big( AB^{-1}(AB^{-1})^{p-1} \big)\\
                    &=\tr\big( BB^{-1}(AB^{-1})^{p-1} \big) &\text{\eqref{EqnCYmKW}}\\
                    &=\tr\big( (AB^{-1})^{p-1} \big).
                \end{align}
            \end{subequations}
            En continuant nous obtenons
            \begin{equation}
                \tr\big(  (AB^{-1})^p \big)=\tr(\mtu)=n.
            \end{equation}
            
            D'autre part, 
            \begin{equation}
                N^k=(AB^{-1}-\mtu)^k=\sum_{p=0}^k{p\choose k}(-1)^{k-p}(AB^{-1})^p
            \end{equation}
            En prenant la trace, et en tenant compte du fait que \( \tr\big( (AB^{-1})^p \big)=n\),
            \begin{equation}
                \tr(N^k)=\sum_{p=0}^k{p\choose k}(-1)^{k-p}n=n(1-1)^k=0.
            \end{equation}
            Donc la trace de \( N^k\) est nulle et le lemme \ref{LemzgNOjY} nous enseigne que \( N\) est alors nilpotente. Étant donné qu'elle est aussi diagonalisable, elle est nulle. Nous en concluons que \( AB^{-1}=\mtu\) et donc que \( A=B\). La fonction \( \tau\) est donc injective.

        \item[Nombre fini de valeurs]

            Les éléments de \( G\) sont annulés par \( X^e-1\) qui est un polynôme scindé à racines simples. Dons le polynôme minimal d'un élément de \( G\) est (a fortiori) scindé à racines simples et le théorème \ref{ThoDigLEQEXR} nous assure alors que ces éléments sont diagonalisables. Du coup les valeur propres des matrices de \( G\) sont des racines \( e\)ièmes de l'unité. Par conséquent les traces des éléments de \( G\) ne peuvent prendre qu'un nombre fini de valeurs : toutes les sommes de \( n\) racines \( e\)ièmes de l'unité. Mais vu que les \( C_i\) sont dans \( G\), nous avons
            \begin{equation}
                \Image(\tau)=\{ \tr(A)\tq A\in G \}^r,
            \end{equation}
            qui est un ensemble fini. Par conséquent \( G\) est fini parce que \( \tau\) est injective.
    \end{subproof}
\end{proof}

%--------------------------------------------------------------------------------------------------------------------------- 
\subsection{Théorème de Lie-Kolchin}
%---------------------------------------------------------------------------------------------------------------------------

Contrairement à ce que l'on peut parfois croire, il n'est pas vrai que toute matrice à coefficient réel est diagonalisable, même pas sur \( \eC\). La raison est qu'une telle matrice peut très bien avoir des valeurs propres multiples.

\begin{example} \label{ExBRXUooIlUnSx}
    Le théorème \ref{ThoDigLEQEXR} nous donne une façon simple de trouver des matrices non diagonalisables sur \( \eC\) : il suffit que le polynôme minimal ne soit pas scindé à racines simples. Par exemple
    \begin{equation}
        A=\begin{pmatrix}
            1    &   1    \\ 
            0    &   1    
        \end{pmatrix},
    \end{equation}
    dont le polynôme caractéristique est \( \chi_A=(1-X)^2\). Ce polynôme n'a manifestement pas des racines simples. Nous pouvons faire le calcul explicite pour montrer que \( A\) n'est pas diagonalisable. D'abord l'unique valeur propre de \( A\) est \( 1\) et nous pouvons sans peine résoudre
    \begin{equation}
        \begin{pmatrix}
            1    &   1    \\ 
            0    &   1    
        \end{pmatrix}\begin{pmatrix}
            x    \\ 
            y    
        \end{pmatrix}=\begin{pmatrix}
            x    \\ 
            y    
        \end{pmatrix}
    \end{equation}
    qui revient au système
    \begin{subequations}
        \begin{numcases}{}
            x+y=x\\
            y=y.
        \end{numcases}
    \end{subequations}
    La première équation donne directement \( y=0\). Le seul espace propre est de dimension \( 1\) et est engendré par \( \begin{pmatrix}
        1    \\ 
        0    
    \end{pmatrix}\).
\end{example}

La remarque \ref{RemBOGooCLMwyb} donne un exemple un peu plus avancé, qui montre la multiplicité algébrique et géométrique d'une racine d'un polynôme caractéristique.

\begin{lemma}[Trigonalisation simultanée]   \label{LemSLGPooIghEPI}
    Une famille de matrices de \( \GL(n,\eC)\) commutant deux à deux est simultanément trigonalisable.
\end{lemma}
\index{trigonalisation!simultanée}

\begin{proof}
    Commençons par enfoncer une porte ouverte par la proposition \ref{PropKNVFooQflQsJ} : sur \( \GL(n,\eC)\) toutes les matrices sont trigonalisables parce que tous les polynômes sont scindés.

    Nous effectuons la démonstration par récurrence sur la dimension. Si \( n=1\) alors toues les matrice sont triangulaires et nous ne nous posons pas de questions. Nous supposons donc \( n>1\).

    Soit la famille \( (A_i)_{i\in I}\) dans \( \GL(n,\eC)\) et \( A_0\) un de ses éléments. Nous nommons \( \lambda_1,\ldots, \lambda_r\) les valeurs propres distinctes de \( A_0\). Le théorème de décomposition primaire \ref{ThoSpectraluRMLok} nous donne la somme directe d'espaces caractéristiques\footnote{Définition \ref{DefFBNIooCGbIix}.}
    \begin{equation}
        E=F_{\lambda_1}(A_0)\oplus\ldots\oplus F_{\lambda_r}(A_0).
    \end{equation}
    Nous pouvons supposer que cette somme n'est pas réduite à un seul terme. En effet si tel était le cas, \( A_0\) serait un multiple de l'identité parce que \( A_0\) n'aurait qu'une seule valeur propre et les sommes dans la décomposition de Dunford \ref{ThoRURcpW}\ref{ItemThoRURcpWiii} se réduisent à un seul terme (et \( p_i=\id\)). En particulier les dimensions des espaces \( F_{\lambda}(A_0)\) sont strictement plus petites que \( n\).

    Vu que tous les \( A_i\) commutent avec \( A_0\), les espaces \( F_{\lambda}(A_0)\) sont stables par les \( A_i\) et nous pouvons trigonaliser les \( A_i\) simultanément sur chacun des \( F_{\lambda}(A_0)\) en utilisant l'hypothèse de récurrence.
\end{proof}

\begin{theorem}[Lie-Kolchin\cite{PAXrsMn}]  \label{ThoUWQBooCvutTO}
    Tout sous-groupe connexe et résoluble de \( \GL(n,\eC)\) est conjugué à un groupe de matrices triangulaires.
\end{theorem}
\index{trigonalisation!simultanée}
\index{théorème!Lie-Kolchin}

\begin{proof}
    Soit \( G\) un sous-groupe connexe et résoluble de \( \GL(n,\eC)\).
    
    \begin{subproof}
        \item[Si sous-espace non trivial stable par \( G\)]

    Nous commençons par voir ce qu'il se passe si il existe un sous-espace vectoriel non trivial \( V\) de \( \eC^n\) stabilisé par \( G\). Pour cela nous considérons une base de \( \eC^n\) dont les premiers éléments forment une base de \( V\) (base incomplète, théorème \ref{ThonmnWKs}). Les éléments de \( G\) s'écrivent, dans cette base,
    \begin{equation}    \label{EqGOKTooEaGACG}
        \begin{pmatrix}
            g_1    &   *    \\ 
            0    &   g_2    
        \end{pmatrix}.
    \end{equation}
    Les matrices \( g_1\) et \( g_2\) sont carrés. Nous considérons alors l'application \( \psi\) définie par
    \begin{equation}
        \begin{aligned}
            \psi\colon G&\to \GL(V) \\
            g&\mapsto g_1.
        \end{aligned}
    \end{equation}
    Cela est un morphisme de groupes parce que
    \begin{equation}
        \begin{pmatrix}
            g_1    &   *    \\ 
            0    &   g_2    
        \end{pmatrix}\begin{pmatrix}
            h_1    &   *    \\ 
            0    &   h_2    
        \end{pmatrix}=
        \begin{pmatrix}
            g_1h_1    &   *    \\ 
            0    &   g_2h_2    
        \end{pmatrix},
    \end{equation}
    de telle sorte que \( \psi(gh)=\psi(g)\psi(h)\).

    Le groupe \( \psi(G)\) est connexe et résoluble. En effet \( \psi(G)\) est connexe en tant qu'image d'un connexe par une application continue (proposition \ref{PropGWMVzqb}). Et il est résoluble en tant qu'image d'un groupe résoluble par un homomorphisme par la proposition \ref{PropBNEZooJMDFIB}. Vu que \( \psi(G)\) est un sous-groupe résoluble et connexe de \( \GL(V)\) et que la dimension de \( V\) est strictement plis petite que celle de \( \eC^n\), une récurrence sur la dimension indique que \( \psi(G)\) est conjugué à un groupe de matrices triangulaires. C'est à dire qu'il existe une base de \( V\) dans laquelle toutes les matrices \( g_1\) (avec \( g\in G\)) sont triangulaires supérieures.

    On fait de même avec l'application \( g\mapsto g_2\), ce qui donne une base du supplémentaire de \( V\) dans laquelle les matrices \( g_2\) sont triangulaires. 

    En couplant ces deux bases, nous obtenons une base de \( \eC^n\) dans laquelle toutes les matrices \eqref{EqGOKTooEaGACG} (c'est à dire toutes les matrices de \( G\)) sont triangulaires supérieures.

    \item[Sinon]

    Nous supposons à présent que \( \eC^n\) n'a pas de sous-espaces non triviaux stables sous \( G\). Nous posons \( m=\min\{ k\tq D^k(G)=\{ e \} \}\), qui existe parce que \( G\) et résoluble et que sa suite dérivée termine sur \( {e}\) (proposition \ref{PropRWYZooTarnmm}).

\item[Si \( m=1\)]

    Si \( m=1\) alors \( G\) est abélien et il existe une base de \( G\) dans laquelle toutes les matrices de \( G\) sont triangulaires (lemme \ref{LemSLGPooIghEPI}). Le premier vecteur d'une telle base serait stable par \( G\), mais comme nous avons supposé qu'il n'y avait pas de sous-espaces non triviaux stabilisés par \( G\), il faut déduire que ce vecteur stable est à lui tout seul non trivial, c'est à dire que \( n=1\). Dans ce cas, le théorème est démontré.

\item[Si \( m>1\)]

    Nous devons maintenant traiter le cas où \( m>1\). Nous posons \( H=D^{m-1}(G)\); cela est un sous-groupe normal et abélien de \( G\). Encore une fois le résultat de trigonalisation simultanée \ref{LemSLGPooIghEPI} donne une base dans laquelle tous les éléments de \( H\) sont triangulaires. En particulier le premier élément de cette base est un vecteur propre commun à toutes les matrices de \( H\).

    Soit \( V\) le sous-espace engendré par tous les vecteurs propres communs de \( H\). Nous venons de voir que \( V\) n'est pas vide. Nous allons montrer que \( V\) est stable par \( G\). Soient \( h\in H\), \( v\in V\) et \( g\in G\) :
    \begin{equation}    \label{EqPMOBooVLIhrJ}
        h\big( g(v) \big)=g\underbrace{g^{-1}hg}_{\in H}(v)=g(\lambda v)=\lambda g(v)
    \end{equation}
    parce que \( v\) est vecteur propre de \( g^{-1} hg\). Ce que le calcul \eqref{EqPMOBooVLIhrJ} montre est que \( g(v)\) est vecteur propre de \( h\) pour la valeur propre \( \lambda\). Donc \( g(v)\in V\) et \( V\) est stabilisé par \( G\). Mais comme il n'existe pas d'espaces non triviaux stabilisés par \( G\), nous en déduisons que \( V=\eC^n\). Donc tous les vecteurs de \( \eC^n\) sont vecteurs propres communs de \( H\). Autrement dit on a une base de diagonalisation simultanée de \( H\).

\item[\( H\) est dans le centre de \( G\)]

    Montrons à présent que \( H\) est dans le centre de \( G\), c'est à dire que pour tout \( g\in G\) et \( h\in H\) il faut \( ghg^{-1}=h\). D'abord \( ghg^{-1}\) est une matrice diagonale (parce que elle est dans \( H\)) ayant les mêmes valeurs propres que \( h\). En effet si \( \lambda\) est valeur propre de \( ghg^{-1}\) pour le vecteur propre \( v\), alors
    \begin{subequations}
        \begin{align}
            (ghg^{-1})(v)&=\lambda v\\
            h\big( g^{-1} v \big)&=\lambda \big( g^{-1}v \big),
        \end{align}
    \end{subequations}
    c'est à dire que \( \lambda\) est également valeur propre de \( h\), pour le vecteur propre \( g^{-1} v\). Mais comme \( h\) a un nombre fini de valeurs propres, il n'y a qu'un nombre fini de matrices diagonales ayant les mêmes valeurs propres que \( h\). L'ensemble \( \AD(G)h\) est donc un ensemble fini. D'autre part, l'application \( g\mapsto g^{-1}hg\) est continue, et \( G\) est connexe, donc l'ensemble \( \AD(G)h\) est connexe. Un ensemble fini et connexe dans \( \GL(n,\eC)\) est nécessairement réduit à un seul point. Cela prouve que \( ghg^{-1}=h\) pour tout \( g\in G\) et \( h\in H\).

\item[Espaces propres stables pour tout \( G\)]

        Soit \( h\in H\) et \( W\) un espace propre de \( h\) (ça existe non vide parce que \( H\) est triangularisé, voir plus haut). Alors nous allons prouver que \( W\) est stable pour tous les éléments de \( G\). En effet si \( w\in W\) avec \( h(w)=\lambda w\) alors en permutant \( g\) et \( h\),
        \begin{equation}
            hg(w)=g(hw)=\lambda g(w),
        \end{equation}
        donc \( g(w)\) est aussi vecteur propre de \( h\) pour la valeurs propre \( \lambda\), c'est à dire que \( g(w)\in W\). Vu que nous supposons que \( \eC^n\) n'a pas d'espaces invariants non triviaux, nous devons conclure que \( W=\eC^n\), c'est à dire que \( H\) est composé d'homothéties. C'est à dire que pour tout \( h\in H\) nous avons \( h=\lambda_h\mtu\).

    \item[Contradiction sur la minimalité de \( m\)]

        Les éléments d'un groupe dérivé sont de déterminant \( 1\) parce que \( \det(g_1g_2g_1^{-1}g_2^{-1})=1\). Par conséquent pour tout \( h\), le nombre \( \lambda_h\) est une racine \( n\)\ieme de l'unité. Vu qu'il n'y a qu'une quantité finie de racines \( n\)\ieme de l'unité, le groupe\( H\) est fini et connexe et donc une fois de plus réduit à un élément, c'est à dire \( H=\{ e \}\). Cela contredit la minimalité de \( m\) et donc produit une contradiction. Nous devons donc avoir \( m=1\).

    \item[Conclusion]

        Nous avons vu que si \( \eC^n\) avait un sous-espace non trivial fixé par \( G\) alors le théorème était démontré. Par ailleurs si \( \eC^n\) n'a pas un tel sous-espace, soit \( m=1\) (et alors le théorème est également prouvé), soit \( m>1\) et alors on a une contradiction.

        Bref, le théorème est prouvé sous peine de contradiction.

    \end{subproof}

\end{proof}
