%+++++++++++++++++++++++++++++++++++++++++++++++++++++++++++++++++++++++++++++++++++++++++++++++++++++++++++++++++++++++++++
\section{Généralités}
%+++++++++++++++++++++++++++++++++++++++++++++++++++++++++++++++++++++++++++++++++++++++++++++++++++++++++++++++++++++++++++

Source : \cite{Tauvel}.

\begin{definition}
    Un \defe{anneau}{anneau} est un triple \( (A,+,\cdot)\) avec les conditions
    \begin{enumerate}
        \item
            \( (A,+)\) est un groupe abélien. Nous notons \( 0\) le neutre.
        \item
            La multiplication est associative et nous notons \( 1\) le neutre
        \item
            La multiplication est distributive par rapport à l'addition.
    \end{enumerate}
\end{definition}

\begin{remark}
    Un anneau est ce qu'on appelle «\emph{ring}» en anglais.
\end{remark}


Soit \( X\) un ensemble et $A$ un anneau. Nous considérons \( \Fun(X,A)\)\nomenclature[A]{\( \Fun(X,Y)\)}{les applications de \( X\) vers \( Y\)} l'ensemble des applications \( X\to A\). Cet ensemble devient un anneau avec les définitions
\begin{subequations}
    \begin{align}
        (f+g)(x)=f(x)+g(x)\\
        (fg)(x)=f(x)g(x).
    \end{align}
\end{subequations}
Cela est la \defe{structure canonique}{structure d'anneau canonique} d'anneau sur \( \Fun(X,A)\).

Le \defe{centralisateur}{centralisateur} de \( x\in A\) dans \( A\) est l'ensemble
\begin{equation}
    \{ y\in A\tq xy=yx \},
\end{equation}
le \defe{centre}{centre} de \( A\) est
\begin{equation}
    \{ y\in A\tq xy=yx\forall x\in A \}.
\end{equation}
Un élément \( a\neq 0\) est un \defe{diviseur de zéro à gauche}{diviseur!de zéro} si il existe \( x\neq 0\) tel que $xa=0$. L'élément \( a\) est un diviseur de zéro \defe{à droite}{diviseur!de zéro à droite} si il existe \( b\) tel que \( ab=0\). Un anneau est \defe{intègre}{intègre!anneau}\index{anneau!intègre} si il est non nul et ne possède pas de diviseurs de zéro.

\begin{example}
    L'ensemble \( \eZ\) avec les opérations usuelles est un anneau intègre.
\end{example}

Un élément \( a\in A\) est \defe{régulier à droite}{régulier à droite} \( ba=0\) implique \( b=0\). Il est régulier ) gauche si \( ab=0\) implique \( b=0\).

L'ensemble \( U(A)\)\nomenclature[A]{\( U(A)\)}{ensemble des inversibles} des éléments inversibles de \( A\) est un groupe pour la multiplication. Nous notons \( A^*=A\setminus\{ 0 \}\).

\begin{lemma}
    Si \( a\) et \( b\) commutent, nous avons la formule
    \begin{equation}        \label{Eqarpurmkbk}
        a^{r+1}-b^{r+1}=(a-b)(\sum_ka^{r-k}b^k).
    \end{equation}
\end{lemma}

\begin{proposition}
    Si \( a\) est un élément nilpotent de l'anneau \( A\), alors \( 1-a\) est inversible. Si \( a\) est nilpotent non nul, alors il est diviseur de zéro.
\end{proposition}

\begin{proof}
    Soit \( n\) le minimum tel que \( a^n=0\). En vertu de la formule \eqref{Eqarpurmkbk} nous avons
    \begin{equation}
        1=1-a^n=(1-a)(1+a+\ldots+a^{n-1})=(1+a+\ldots+a^{n-1})(1-a).
    \end{equation}
    La somme \( 1+a+\ldots+a^{n-1}\) est donc un inverse de \( (1-a)\).
\end{proof}

\begin{definition}
    Si \( A\) et \( B\) sont des anneaux, un \defe{morphisme}{morphisme!d'anneaux} est une application \( f\colon A\to B\) telle que pour tout \( x,y\in A\) nous ayons
    \begin{enumerate}
        \item
            \( f(x+y)=f(x)+f(y)\)
        \item
            \( f(xy)=f(x)f(y)\)
        \item
            \( f(1)=1\)
    \end{enumerate}
\end{definition}

Si \( f\) est un morphisme, nous avons \( f(0)=0\) et \( f(x)^{-1}=f(x^{-1})\).

%+++++++++++++++++++++++++++++++++++++++++++++++++++++++++++++++++++++++++++++++++++++++++++++++++++++++++++++++++++++++++++
\section{Idéaux dans des anneaux}
%+++++++++++++++++++++++++++++++++++++++++++++++++++++++++++++++++++++++++++++++++++++++++++++++++++++++++++++++++++++++++++

\begin{definition}
    Un sous ensemble \( B\subset A\) d'un anneau est un \defe{sous anneau}{sous anneau} si
    \begin{enumerate}
        \item
            \( 1\in B\)
        \item
            \( B\) est un sous groupe pour l'addition
        \item
            \( B\) est stable pour la multiplication.
    \end{enumerate}
    Un sous ensemble \( I\subset A\) est un \defe{idéal}{idéal!dans un anneau} à gauche si
    \begin{enumerate}
        \item
            \( I\) est un sous groupe pour l'addition
        \item
            si \( x\in I\) et \( a\in A\), alors \( ax\in I\).
    \end{enumerate}
\end{definition}

Lorsqu'un ensemble est idéal à gauche et à droite, nous disons que c'est un \defe{idéal bilatère}{idéal!bilatère}. Lorsque nous parlons d'idéal sans précisions, nous parlons d'idéal bilatère.

\begin{remark}
    Un idéal n'est pas toujours un anneau parce que l'identité pourrait manquer. Un idéal qui contient l'identité est l'anneau complet.
\end{remark}

\begin{example}
    L'ensemble \( 2\eZ\) est un idéal de \( \eZ\). Tous les idéaux de \( \eZ\) sont de la forme \( n\eZ\). En effet en vertu de la proposition \ref{PropSsgpZestnZ}, les seule sous groupes de \( \eZ\) (en tant que groupe additif) sont les \( n\eZ\).
\end{example}

Soit \( A\), un anneau, \( I\) un idéal bilatère de \( A\). Nous considérons la relation d'équivalence \( x\sim y\) si et seulement si \( x-y\in I\). Dans ce cas, le quotient
\begin{equation}
    A/\sim=A/I
\end{equation}
est un anneau appelé \defe{anneau quotient}{anneau!quotient par un idéal}. La surjection \( A\to A/I\) est un morphisme.

\begin{proposition}
    Soient \( A\) et \( B\) des anneaux et un homomorphisme \( f\colon A\to B\). Nous considérons l'injection canonique \( j\colon f(A)\to B\) et la surjection canonique \( \phi\colon A\to A/\ker f\). Alors il existe un unique isomorphisme
    \begin{equation}
        \tilde f \colon A/\ker f\to f(A)
    \end{equation}
    tel que \( f=j\circ\tilde f\circ\phi\).

    \begin{equation}
        \xymatrix{%
        A \ar[r]^{f}\ar[d]_{\phi}        &   B\ar[d]^{j}\\
           A/\ker f \ar[r]_{\tilde f}   &   f(A)\subset B
           }
    \end{equation}
\end{proposition}

\begin{proposition}
    Soit \( I\), un idéal de \( A\) et \( \phi\colon A\to A/I\) la surjection canonique. Les idéaux de \( A/I\) sont les \( \phi(J)\) où \( J\) est un idéal de \( A\) contenant \( I\)
\end{proposition}

\begin{proof}
    Si \( I\subset J\) et si \( J \) est un idéal de \( A\), alors \( \phi(J)\) est un idéal dans \( A/I\). En effet un élément de \( \phi(J)\) est de la forme \( \phi(j)\) et un élément de \( A/I\) est de la forme \( \phi(i)\). Leur produit vaut
    \begin{equation}
        \phi(i)\phi(j)=\phi(ij)\in\phi(J).
    \end{equation}
    
    Soit maintenant \( K\), un idéal dans \( A/I\). Soit \( J=\phi^{-1}(K)\). Étant donné qu'un idéal doit contenir \( 0\) (parce qu'un idéal est un groupe pour l'addition), \( [0]\in K\) et par conséquent \( I\subset\phi^{-1}(K)\).
\end{proof}

\begin{corollary}
    Les quotients de \( \eZ\) sont \( \eZ_n\)
\end{corollary}

\begin{proof}
    Nous avons déjà vu que les seuls idéaux de \( \eZ\) sont les \( n\eZ\).
\end{proof}

\begin{proposition}     \label{PropZpintssiprempUzn}
    Soit \( n\geq 2\) un entier et \( \phi\colon \eZ\to \eZ_n\), la surjection canonique. Nous noterons \( \tilde a=\phi(a)\). Alors
    \begin{equation}
        U(\eZ_n)=\phi(P_n)=\{ \tilde x\tq 0\leq x\leq n\tq\pgcd(x,n)=1 \}.
    \end{equation}
    où \( P_n\) est l'ensemble décrit par l'équation \eqref{EqDefPnEntierldeost}. En particulier, \( \Card\big( U(\eZ_n) \big)=\varphi(n)\).

    L'anneau \( \eZ_n\) est intègre si et seulement si \( n\) est premier.
\end{proposition}

\begin{proof}
    Soit \( 0\leq x\leq n\) tel que \( \pgcd(x,n)=1\). Il existe donc \( p,q\in\eZ\) tels que \( px+qn=1\). En passant aux classes,
    \begin{equation}
        \tilde p\tilde x=\tilde 1,
    \end{equation}
    donc \( \tilde p\) est l'inverse de \( \tilde x\). Cela prouve que \( \phi(P_n)\subset U(\eZ_n)\).

    Nous prouvons maintenant l'inclusion inverse. Soit \( \tilde x\) et \( \tilde y\) inverses l'un de l'autre : $\tilde x\tilde y=\tilde 1$. Il existe donc \( q\in\eZ\) tel que \( xy-qn=1\), ce qui prouve que \( \pgcd(x,n)=1\).

    Si \( n\) est premier, tous les éléments de \( \eZ_n\) sont inversibles parce que tous les éléments rentrent dans \( \phi(P_n)\). Donc \( \eZ_n\) est intègre.

    Si \( n\) n'est pas premier, il existe \( p,q\in\eN^*\) tels que \( pq=n\). Dans ce cas au niveau des classes nous avons \( \tilde p\tilde q=0\) avec \( \tilde p\neq 0\neq\tilde q\), ce qui montre que \( \eZ_n\) a des diviseurs de zéro et n'est pas intègre.
\end{proof}

%---------------------------------------------------------------------------------------------------------------------------
\subsection{Caractéristique}
%---------------------------------------------------------------------------------------------------------------------------

L'application 
\begin{equation}
    \begin{aligned}
        \mu\colon \eZ&\to A \\
        n&\mapsto n\cdot 1_A 
    \end{aligned}
\end{equation}
est un morphisme d'anneaux. Le noyau de \( \mu\) étant un sous groupe de \( \eZ\), il existe un et un seul \( p\in\eZ\) tel que \( \ker\mu=p\eZ\). Ce \( p\) est la \defe{caractéristique}{caractéristique!d'un anneau} de \( A\).

\begin{lemma}
    Si \( A\) est de caractéristique nulle, alors \( A\) est infini.
\end{lemma}

\begin{proof}
    En effet, \( \ker\mu=0\) implique que \( n1_A\neq  m1_A\) et par conséquent \( A\) est infini.
\end{proof}

\begin{lemma}
    Si \( p\) est la caractéristique de l'anneau \( A\), alors nous avons l'isomorphisme d'anneaux
    \begin{equation}
         \eZ 1_A\simeq\eZ/p\eZ.
    \end{equation}
\end{lemma}

\begin{proof}
    L'isomorphisme est donné par l'application \( n1_A\mapsto \phi(n)\) si \( \phi\) est la projection canonique \( \eZ\to \eZ_p\).
\end{proof}

\begin{lemma}
    La caractéristique d'un anneau intègre est zéro ou un nombre premier.
\end{lemma}

\begin{proof}
    Si \( A\) est intègre, alors \( \eZ 1_A\) est intègre (a fortiori), et \( \eZ_p\) est intègre parce qu'il est isomorphe à \( \eZ A_A\). Mais nous savons que \( \eZ_p\) est intègre si et seulement si \( p\) est premier (proposition \ref{PropZpintssiprempUzn}).
\end{proof}

\begin{theorem}[Théorème chinois]\index{théorème!chinois}
    Soit \( A\) un anneau commutatif, \( n\geq 2\), des éléments \( x_1,\ldots,x_n\) dans \( A\) et des idéaux \( I_1,\ldots,I_n\) tels que \( I_i+I_j=A\) pour tout \( i\neq j\).

    Alors il existe un \( x\in A\) tel que \( x-x_i\in I_i\) pour tout \( 1\les i\leq n\).
\end{theorem}

\begin{proof}
    Pour \( i\in\{ 1,\ldots,n \}\) nous notons \( J_i\) le produit \( J_i=\prod_{k\neq i}I_k\). Étant donné que chaque \( I_i\) est un idéal, nous avons \( I_k\in J_i\) lorsque \( i\neq k\).

    Soit \( i\) fixé, et considérons \( j\neq i\). Nous pouvons trouver \( a_j\in I_i\) et \( b_j\in I_j\) tel que \( a_j+b_j=1\). Nous avons alors
    \begin{equation}
        1=\prod_{j\neq i}(a_j+b_j).
    \end{equation}
    Par ailleurs \( I_i+J_i=A\) parce que \( J_i\) contient \( I_k\) avec \( k\neq i\) et \( I_i+I_k=A\). Nous pouvons donc prendre \( \alpha_i\in I_i\) et \( \beta_i\in J_i\) tels que
    \begin{equation}
        \prod_{j\neq i}(a_j+b_j)=\alpha_i+\beta_i.
    \end{equation}
    Nous considérons alors l'élément \( x=\beta_1x_1+\ldots+\beta_nx_n\) et nous avons
    \begin{subequations}
        \begin{align}
            x-x_&=(\beta_1-1)x_1+\beta_2x_n+\ldots+\beta_nx_n\\
            &=-\alpha_1x_1+\beta_2x_2+\ldots+\beta_nx_n.
        \end{align}
    \end{subequations}
    Mais \( \alpha_1\in I_1\) et tous les autres termes sont dans les \( J_i\) avec \( i\neq 1\). Par conséquent le tout est dans \( I_1\). Ici nous utilisons par exemple le fait que \( \beta_2\in J_2\subset I_1\) parce que les éléments de \( J_2\) sont des produits d'éléments dont un facteur est dans \( I_1\).
\end{proof}
<++>

