%+++++++++++++++++++++++++++++++++++++++++++++++++++++++++++++++++++++++++++++++++++++++++++++++++++++++++++++++++++++++++++
\section{Généralités}
%+++++++++++++++++++++++++++++++++++++++++++++++++++++++++++++++++++++++++++++++++++++++++++++++++++++++++++++++++++++++++++

Source : \cite{Tauvel}.

\begin{definition}
    Un \defe{anneau}{anneau} est un triple \( (A,+,\cdot)\) avec les conditions
    \begin{enumerate}
        \item
            \( (A,+)\) est un groupe abélien. Nous notons \( 0\) le neutre.
        \item
            La multiplication est associative et nous notons \( 1\) le neutre
        \item
            La multiplication est distributive par rapport à l'addition.
    \end{enumerate}
\end{definition}

\begin{remark}
    Un anneau est ce qu'on appelle «\emph{ring}» en anglais.
\end{remark}

Soit \( X\) un ensemble et $A$ un anneau. Nous considérons \( \Fun(X,A)\)\nomenclature[A]{\( \Fun(X,Y)\)}{les applications de \( X\) vers \( Y\)} l'ensemble des applications \( X\to A\). Cet ensemble devient un anneau avec les définitions
\begin{subequations}
    \begin{align}
        (f+g)(x)=f(x)+g(x)\\
        (fg)(x)=f(x)g(x).
    \end{align}
\end{subequations}
Cela est la \defe{structure canonique}{structure d'anneau canonique} d'anneau sur \( \Fun(X,A)\).

Le \defe{centralisateur}{centralisateur} de \( x\in A\) dans \( A\) est l'ensemble
\begin{equation}
    \{ y\in A\tq xy=yx \},
\end{equation}
le \defe{centre}{centre} de \( A\) est
\begin{equation}
    \{ y\in A\tq xy=yx\forall x\in A \}.
\end{equation}
Un élément \( a\neq 0\) est un \defe{diviseur de zéro à gauche}{diviseur!de zéro} si il existe \( x\neq 0\) tel que $xa=0$. L'élément \( a\) est un diviseur de zéro \defe{à droite}{diviseur!de zéro à droite} si il existe \( b\) tel que \( ab=0\). Un anneau est \defe{intègre}{intègre!anneau}\index{anneau!intègre} si il est non nul et ne possède pas de diviseurs de zéro.

\begin{example}
    L'ensemble \( \eZ\) avec les opérations usuelles est un anneau intègre.
\end{example}

Un élément \( a\in A\) est \defe{régulier à droite}{régulier à droite} \( ba=0\) implique \( b=0\). Il est régulier ) gauche si \( ab=0\) implique \( b=0\).

L'ensemble \( U(A)\)\nomenclature[A]{\( U(A)\)}{ensemble des inversibles} des éléments inversibles de \( A\) est un groupe pour la multiplication. Nous notons \( A^*=A\setminus\{ 0 \}\).

\begin{lemma}
    Si \( a\) et \( b\) commutent, nous avons la formule
    \begin{equation}        \label{Eqarpurmkbk}
        a^{r+1}-b^{r+1}=(a-b)(\sum_{k=0}^ra^{r-k}b^k).
    \end{equation}
\end{lemma}

\begin{proposition}
    Si \( a\) est un élément nilpotent de l'anneau \( A\), alors \( 1-a\) est inversible. Si \( a\) est nilpotent non nul, alors il est diviseur de zéro.
\end{proposition}

\begin{proof}
    Soit \( n\) le minimum tel que \( a^n=0\). En vertu de la formule \eqref{Eqarpurmkbk} nous avons
    \begin{equation}
        1=1-a^n=(1-a)(1+a+\ldots+a^{n-1})=(1+a+\ldots+a^{n-1})(1-a).
    \end{equation}
    La somme \( 1+a+\ldots+a^{n-1}\) est donc un inverse de \( (1-a)\).
\end{proof}

\begin{definition}
    Si \( A\) et \( B\) sont des anneaux, un \defe{morphisme}{morphisme!d'anneaux} est une application \( f\colon A\to B\) telle que pour tout \( x,y\in A\) nous ayons
    \begin{enumerate}
        \item
            \( f(x+y)=f(x)+f(y)\)
        \item
            \( f(xy)=f(x)f(y)\)
        \item
            \( f(1)=1\)
    \end{enumerate}
\end{definition}

Si \( f\) est un morphisme, nous avons \( f(0)=0\) et \( f(x)^{-1}=f(x^{-1})\).

%+++++++++++++++++++++++++++++++++++++++++++++++++++++++++++++++++++++++++++++++++++++++++++++++++++++++++++++++++++++++++++
\section{Idéaux dans des anneaux}
%+++++++++++++++++++++++++++++++++++++++++++++++++++++++++++++++++++++++++++++++++++++++++++++++++++++++++++++++++++++++++++

\begin{definition}
    Un sous ensemble \( B\subset A\) d'un anneau est un \defe{sous anneau}{sous anneau} si
    \begin{enumerate}
        \item
            \( 1\in B\)
        \item
            \( B\) est un sous groupe pour l'addition
        \item
            \( B\) est stable pour la multiplication.
    \end{enumerate}
    Un sous ensemble \( I\subset A\) est un \defe{idéal}{idéal!dans un anneau} à gauche si
    \begin{enumerate}
        \item
            \( I\) est un sous groupe pour l'addition
        \item
            si \( x\in I\) et \( a\in A\), alors \( ax\in I\).
    \end{enumerate}
\end{definition}

Lorsqu'un ensemble est idéal à gauche et à droite, nous disons que c'est un \defe{idéal bilatère}{idéal!bilatère}. Lorsque nous parlons d'idéal sans précisions, nous parlons d'idéal bilatère.

\begin{remark}
    Un idéal n'est pas toujours un anneau parce que l'identité pourrait manquer. Un idéal qui contient l'identité est l'anneau complet.
\end{remark}

\begin{example}
    L'ensemble \( 2\eZ\) est un idéal de \( \eZ\). Tous les idéaux de \( \eZ\) sont de la forme \( n\eZ\). En effet en vertu de la proposition \ref{PropSsgpZestnZ}, les seule sous groupes de \( \eZ\) (en tant que groupe additif) sont les \( n\eZ\).
\end{example}

Soit \( A\), un anneau, \( I\) un idéal bilatère de \( A\). Nous considérons la relation d'équivalence \( x\sim y\) si et seulement si \( x-y\in I\). Dans ce cas, le quotient
\begin{equation}
    A/\sim=A/I
\end{equation}
est un anneau appelé \defe{anneau quotient}{anneau!quotient par un idéal}. La surjection \( A\to A/I\) est un morphisme.

\begin{proposition}
    Soient \( A\) et \( B\) des anneaux et un homomorphisme \( f\colon A\to B\). Nous considérons l'injection canonique \( j\colon f(A)\to B\) et la surjection canonique \( \phi\colon A\to A/\ker f\). Alors il existe un unique isomorphisme
    \begin{equation}
        \tilde f \colon A/\ker f\to f(A)
    \end{equation}
    tel que \( f=j\circ\tilde f\circ\phi\).

    \begin{equation}
        \xymatrix{%
        A \ar[r]^{f}\ar[d]_{\phi}        &   B\ar[d]^{j}\\
           A/\ker f \ar[r]_{\tilde f}   &   f(A)\subset B
           }
    \end{equation}
\end{proposition}

\begin{proposition}     \label{PropIJJIdsousphi}
    Soit \( I\), un idéal de \( A\) et \( \phi\colon A\to A/I\) la surjection canonique. Les idéaux de \( A/I\) sont les \( \phi(J)\) où \( J\) est un idéal de \( A\) contenant \( I\)
\end{proposition}

\begin{proof}
    Si \( I\subset J\) et si \( J \) est un idéal de \( A\), alors \( \phi(J)\) est un idéal dans \( A/I\). En effet un élément de \( \phi(J)\) est de la forme \( \phi(j)\) et un élément de \( A/I\) est de la forme \( \phi(i)\). Leur produit vaut
    \begin{equation}
        \phi(i)\phi(j)=\phi(ij)\in\phi(J).
    \end{equation}
    
    Soit maintenant \( K\), un idéal dans \( A/I\). Soit \( J=\phi^{-1}(K)\). Étant donné qu'un idéal doit contenir \( 0\) (parce qu'un idéal est un groupe pour l'addition), \( [0]\in K\) et par conséquent \( I\subset\phi^{-1}(K)\).
\end{proof}

\begin{corollary}
    Les quotients de \( \eZ\) sont \( \eZ_n\)
\end{corollary}

\begin{proof}
    Nous avons déjà vu que les seuls idéaux de \( \eZ\) sont les \( n\eZ\).
\end{proof}

\begin{proposition}     \label{PropZpintssiprempUzn}
    Soit \( n\geq 2\) un entier et \( \phi\colon \eZ\to \eZ_n\), la surjection canonique. Nous noterons \( \tilde a=\phi(a)\). Alors
    \begin{equation}
        U(\eZ_n)=\phi(P_n)=\{ \tilde x\tq 0\leq x\leq n\tq\pgcd(x,n)=1 \}.
    \end{equation}
    où \( P_n\) est l'ensemble décrit par l'équation \eqref{EqDefPnEntierldeost}. En particulier, \( \Card\big( U(\eZ_n) \big)=\varphi(n)\).

\end{proposition}

\begin{proof}
    Soit \( 0\leq x\leq n\) tel que \( \pgcd(x,n)=1\). Il existe donc \( p,q\in\eZ\) tels que \( px+qn=1\). En passant aux classes,
    \begin{equation}
        \tilde p\tilde x=\tilde 1,
    \end{equation}
    donc \( \tilde p\) est l'inverse de \( \tilde x\). Cela prouve que \( \phi(P_n)\subset U(\eZ_n)\).

    Nous prouvons maintenant l'inclusion inverse. Soit \( \tilde x\) et \( \tilde y\) inverses l'un de l'autre : $\tilde x\tilde y=\tilde 1$. Il existe donc \( q\in\eZ\) tel que \( xy-qn=1\), ce qui prouve que \( \pgcd(x,n)=1\).

\end{proof}

\begin{corollary}   \label{CorZnInternprem}
    L'anneau \( \eZ_n\) est intègre si et seulement si \( n\) est premier.
\end{corollary}

\begin{proof}
    Si \( n\) est premier, tous les éléments de \( \eZ_n\) sont inversibles parce que tous les éléments rentrent dans \( \phi(P_n)\). Donc \( \eZ_n\) est intègre.

    Si \( n\) n'est pas premier, il existe \( p,q\in\eN^*\) tels que \( pq=n\). Dans ce cas au niveau des classes nous avons \( \tilde p\tilde q=0\) avec \( \tilde p\neq 0\neq\tilde q\), ce qui montre que \( \eZ_n\) a des diviseurs de zéro et n'est pas intègre.
\end{proof}

Un idéal \( I\) dans \(\eA\) est \defe{principal à gauche}{idéal!principal!à gauche} si il existe \( a\in I\) tel que \( I=\eA a\). Il est \defe{principal à droite}{idéal!principal!à droite} si il existe \( a\in I\) tel que \( I=a\eA\). Nous disons qu'il est \defe{principal}{principal!idéal} si il est principal à gauche et à droite.

%---------------------------------------------------------------------------------------------------------------------------
\subsection{Caractéristique}
%---------------------------------------------------------------------------------------------------------------------------

L'application 
\begin{equation}
    \begin{aligned}
        \mu\colon \eZ&\to A \\
        n&\mapsto n\cdot 1_A 
    \end{aligned}
\end{equation}
est un morphisme d'anneaux. Le noyau de \( \mu\) étant un sous groupe de \( \eZ\), il existe un et un seul \( p\in\eZ\) tel que \( \ker\mu=p\eZ\). Ce \( p\) est la \defe{caractéristique}{caractéristique!d'un anneau} de \( A\).

\begin{lemma}
    Si \( A\) est de caractéristique nulle, alors \( A\) est infini.
\end{lemma}

\begin{proof}
    En effet, \( \ker\mu=0\) implique que \( n1_A\neq  m1_A\) et par conséquent \( A\) est infini.
\end{proof}

\begin{lemma}
    Si \( p\) est la caractéristique de l'anneau \( A\), alors nous avons l'isomorphisme d'anneaux
    \begin{equation}
         \eZ 1_A\simeq\eZ/p\eZ.
    \end{equation}
\end{lemma}

\begin{proof}
    L'isomorphisme est donné par l'application \( n1_A\mapsto \phi(n)\) si \( \phi\) est la projection canonique \( \eZ\to \eZ_p\).
\end{proof}

\begin{lemma}       \label{LemCaractIntergernbrcartpre}
    La caractéristique d'un anneau intègre est zéro ou un nombre premier.
\end{lemma}

\begin{proof}
    Si \( A\) est intègre, alors \( \eZ 1_A\) est intègre (a fortiori), et \( \eZ_p\) est intègre parce qu'il est isomorphe à \( \eZ A_A\). Mais nous savons que \( \eZ_p\) est intègre si et seulement si \( p\) est premier (proposition \ref{CorZnInternprem}).
\end{proof}

\begin{theorem}[Théorème chinois]\index{théorème!chinois}
    Soit \( A\) un anneau commutatif, \( n\geq 2\), des éléments \( x_1,\ldots,x_n\) dans \( A\) et des idéaux \( I_1,\ldots,I_n\) tels que \( I_i+I_j=A\) pour tout \( i\neq j\).

    Alors il existe un \( x\in A\) tel que \( x-x_i\in I_i\) pour tout \( 1\leq i\leq n\).
\end{theorem}

\begin{proof}
    Pour \( i\in\{ 1,\ldots,n \}\) nous notons \( J_i\) le produit \( J_i=\prod_{k\neq i}I_k\). Étant donné que chaque \( I_i\) est un idéal, nous avons \( I_k\in J_i\) lorsque \( i\neq k\).

    Soit \( i\) fixé, et considérons \( j\neq i\). Nous pouvons trouver \( a_j\in I_i\) et \( b_j\in I_j\) tel que \( a_j+b_j=1\). Nous avons alors
    \begin{equation}
        1=\prod_{j\neq i}(a_j+b_j).
    \end{equation}
    Par ailleurs \( I_i+J_i=A\) parce que \( J_i\) contient \( I_k\) avec \( k\neq i\) et \( I_i+I_k=A\). Nous pouvons donc prendre \( \alpha_i\in I_i\) et \( \beta_i\in J_i\) tels que
    \begin{equation}
        \prod_{j\neq i}(a_j+b_j)=\alpha_i+\beta_i.
    \end{equation}
    Nous considérons alors l'élément \( x=\beta_1x_1+\ldots+\beta_nx_n\) et nous avons
    \begin{subequations}
        \begin{align}
            x-x_1&=(\beta_1-1)x_1+\beta_2x_2+\ldots+\beta_nx_n\\
            &=-\alpha_1x_1+\beta_2x_2+\ldots+\beta_nx_n
        \end{align}
    \end{subequations}
    Mais \( \alpha_1\in I_1\) et tous les autres termes sont dans les \( J_i\) avec \( i\neq 1\). Par conséquent le tout est dans \( I_1\). Ici nous utilisons par exemple le fait que \( \beta_2\in J_2\subset I_1\) parce que les éléments de \( J_2\) sont des produits d'éléments dont un facteur est dans \( I_1\).
\end{proof}

%+++++++++++++++++++++++++++++++++++++++++++++++++++++++++++++++++++++++++++++++++++++++++++++++++++++++++++++++++++++++++++
\section{Corps}
%+++++++++++++++++++++++++++++++++++++++++++++++++++++++++++++++++++++++++++++++++++++++++++++++++++++++++++++++++++++++++++

\begin{definition}
    Un \defe{corps}{corps} est un anneau non nul dans lequel tout élément non nul est inversible.
\end{definition}

\begin{lemma}       \label{LemAnnCorpsnonInterdivzer}
    En tant que anneau, un corps n'a pas de diviseurs zéro.
\end{lemma}

\begin{proof}
    En effet si \( a\) est un diviseur de zéro, alors \( ax=0\) pour un certain \( x\neq 0\). Si \( a\) était inversible, nous aurions \( x=a^{-1}ax=0\), ce qui est impossible.
\end{proof}

Par le lemme \ref{LemCaractIntergernbrcartpre}, un anneau sans diviseur de zéro est de caractéristique zéro ou première. Le fait d'être intègre pour un anneau n'assure cependant pas le fait d'être un corps. Nous avons cependant ce résultat pour les anneaux finis.

\begin{proposition}     \label{PropanfinintimpCorp}
    Un anneau fini intègre est un corps.
\end{proposition}

\begin{proof}
    Soit \( A\) un tel anneau. Soit \( a\neq 0\). Les applications 
    \begin{subequations}
        \begin{align}
            l_a\colon x\to ax\\
            r_a\colon x\to xa
        \end{align}
    \end{subequations}
    sont injectives. En tant que applications injectives entre ensembles finis, elles sont surjectives. Il existe donc \( b\) et \( c\) tels que \( 1=ba=ac\). Il se fait que \( b\) et \( c\) sont égaux parce que
    \begin{equation}
        b=b(ac)=(ba)c=c.
    \end{equation}
    Par conséquent \( b\) est un inverse de \( a\).
\end{proof}

\begin{proposition}
    Soit \( n\in\eN^*\). Les conditions suivantes sont équivalentes :
    \begin{enumerate}
        \item
            \( n\) est premier.
        \item
            \( \eZ_n\) est un anneau intègre.
        \item
            \( \eZ_n\) est un corps.
    \end{enumerate}
\end{proposition}

\begin{proof}
    L'équivalence entre les deux premiers points est le contenu du corollaire \ref{CorZnInternprem}. Le fait que \( \eZ_n\) soit un corps lorsque \( \eZ_n\) est intègre est la proposition \ref{PropanfinintimpCorp}. Le fait que \( \eZ_n\) soit intègre lorsque \( \eZ_n\) est un corps est une propriété générale des corps : ce sont en particulier des anneaux intègres (lemme \ref{LemAnnCorpsnonInterdivzer}).
\end{proof}

\begin{proposition}     \label{AnnCorpsIdeal}
    Si \( A\) est un anneau, nous avons les équivalences
    \begin{enumerate}
        \item
            \( A\) est un corps.
        \item
            \( A\) est non nul et ses seuls idéaux à gauche sont \( \{ 0 \}\) et \( A\).
        \item
            \( A\) est non nul et ses seuls idéaux à droite sont \( \{ 0 \}\) et \( A\).
    \end{enumerate}
\end{proposition}

\begin{proposition}
    Soit \( A\), un anneau commutatif non nul et \( I\), un idéal dans \( A\). L'ensemble \( I\) est un idéal maximum de \( A\) si et seulement si \( A/I\) est un corps.
\end{proposition}

\begin{proof}
    Par la proposition \ref{AnnCorpsIdeal}, le fait pour \( A/I\) d'être un corps signifie qu'il n'a pas d'idéaux non triviaux. Si \( \phi\) est la projection canonique \( A\mapsto A/I\), nous savons que les idéaux de \( A/I\) sont les \( \phi(J)\) où \( J\) est un idéal de \( A\) contenant \( I\) (proposition \ref{PropIJJIdsousphi}). Si \( I\) est un idéal maximum de \( A\), un tel \( J\) n'existe pas. Inversement si \( A/I\) n'a pas d'autres idéaux que \( A/I\) et \( \phi(0)\), c'est que \( I\) est un idéal maximum.
\end{proof}

%---------------------------------------------------------------------------------------------------------------------------
\subsection{Corps des fractions}
%---------------------------------------------------------------------------------------------------------------------------

\begin{theorem}
    Soit \( \eA\) un anneau commutatif intègre. Il existe un corps commutatif \( \eK\) et un morphisme injectif (d'anneaux) \( \epsilon\colon \eA\to \eK\) tels que pour tout \( \lambda\in\eK\), il existe \( (a,b)\in \eA\times \eA^*\) tels que
    \begin{equation}
        \lambda=\epsilon(a)\big( \epsilon(b) \big)^{-1}
    \end{equation}

    De plus si \( (\eK',\epsilon')\) est un autre couple qui vérifie la propriété, les corps \( \eK\) et \( \eK'\) sont isomorphes.
\end{theorem}
Le corps \( \eK\) associé à l'anneau \( \eA\) est le \defe{corps des fractions}{corps!des fractions}\index{fractions (corps)} de \( \eA\).

Notons que l'application \( \eA\times \eA^*\to \eK\) donnée par \( (a,b)\mapsto \epsilon(a)\big( \epsilon(b) \big)^{-1}\) envoie \( (xa,xb)\) sur le même que \( (a,b)\).

L'ensemble \( \eQ\)\nomenclature[A]{\( \eQ\)}{corps des fractions sur \( \eZ\)} est le corps des fractions de \( \eZ\).

%---------------------------------------------------------------------------------------------------------------------------
\subsection{Corps premier}
%---------------------------------------------------------------------------------------------------------------------------

\begin{definition}
    Un corps est \defe{premier}{corps!premier}\index{premier!corps} si il est son seul sous corps. Le \defe{sous corps premier}{premier!sous corps} d'un corps est l'intersection de tous ses sous corps.
\end{definition}

\begin{lemma}
    Un corps premier est commutatif
\end{lemma}

\begin{proof}
    Le centre d'un corps est certainement un sous corps. Par conséquent un corps premier doit être contenu dans son propre centre, c'est à dire être commutatif.
\end{proof}

Soit \( p\) un nombre premier. Nous notons \( \eF_p=\eZ_p=\eZ/p\eZ\). Nous avons par exemple 
\begin{equation}
    \eF_2=\eZ/2\eZ=\{ 0,1 \}\)
\end{equation}
avec la loi \( 2=0\).

\begin{proposition}
    Soit \( \eK\) un corps de caractéristique \( p\) et \( \eP\) son sous corps premier. Si \( p=0\) alors \( \eP=\eQ\). Si \( p>0\), alors \( \eP=\eF_p\).
\end{proposition}

\begin{proof}
    Notons d'abord que 
\end{proof}
<++>

%+++++++++++++++++++++++++++++++++++++++++++++++++++++++++++++++++++++++++++++++++++++++++++++++++++++++++++++++++++++++++++
\section{Modules}
%+++++++++++++++++++++++++++++++++++++++++++++++++++++++++++++++++++++++++++++++++++++++++++++++++++++++++++++++++++++++++++

Soit \( \modE\) un \( A\)-module et \( x=(x_i)_{i\in I}\) une famille d'éléments de \( \modE\), paramétrée par l'ensemble \( I\). Nous considérons l'application
\begin{equation}
    \begin{aligned}
        \mu_x\colon A^{(I)}&\to \modE \\
        (a_i)_{i\in I}&\mapsto \sum_{i\in I}a_ix_i.
    \end{aligned}
\end{equation}
Ici \( A^{(I)}\) désigne l'ensemble de toutes les applications \( I\to A\) de support fini.  

\begin{definition}      \label{DefBasePouyKj}
    À l'instar des espaces vectoriels, les modules ont une notion de partie libre, génératrice et de bases :
    \begin{enumerate}
        \item
            Si \( \mu_x\) est surjective, nous disons que \( x\) est une partie \defe{génératrice}{génératrice!partir d'un module}.
        \item
            Si \( \mu_x\) est injective, nous disons que la partie \( x\) est \defe{libre}{libre!partie d'un module}.
        \item
            Si \( \mu_x\) est bijective, nous disons que la partie \( x\) est une \defe{base}{base!d'un module}.
    \end{enumerate}
\end{definition}

\begin{definition}
    Soit \( \modE\) un module sur un anneau commutatif \( A\). Un \defe{projecteur}{projecteur!dans un module} est une application linéaire \( p\colon \modE\to \modE\) telle que \( p^2=p\).

    Une famille \( (p_i)_{i\in I}\) sur \( \modE\) est \defe{orthogonale}{orthogonal!famille de projecteurs} si \( p_i\circ p_j=0\) pour tout \( i\neq j\). La famille est \defe{complète}{complète!famille de projecteurs} si \( \sum_{i\in I}p_i=\mtu\).
\end{definition}

\begin{theorem}     \label{ThoProjModpAlsUR}
    Soient des sous modules \( \modE_1,\ldots,\modE_n\) du module \( \modE\) tels que \( \modE=\modE_1\oplus\ldots\oplus\modE_n\). Les applications \( p_i\) définies par
    \begin{equation}
        p_i(x_1+x_n)=x_i
    \end{equation}
    forment une famille orthogonale de projecteurs et \( p_1+\ldots +p_n=\id\).

    Inversement, si \( (p_1,\ldots, p_n)\) est une famille orthogonale de projecteurs dans un module \( \modE\) tel que \( \sum_{i=1}^np_i=\id\), alors
    \begin{equation}
        \modE=\bigoplus_{i=1}^np_i(\modE).
    \end{equation}
\end{theorem}


%+++++++++++++++++++++++++++++++++++++++++++++++++++++++++++++++++++++++++++++++++++++++++++++++++++++++++++++++++++++++++++
\section{Polynômes}
%+++++++++++++++++++++++++++++++++++++++++++++++++++++++++++++++++++++++++++++++++++++++++++++++++++++++++++++++++++++++++++

Soit \( A\) un anneau commutatif. Nous considérons \( \polyP\) l'ensemble des suites presque nulles d'éléments de \( A\), ce sont les suites \( (a_n)_{n\in\eN}\) telles que il existe \( N\) tel que \( a_i=0\) pour tout \( i>N\).

Cela est un \( A\)-module libre de base (définition \ref{DefBasePouyKj})
\begin{equation}
    (e_n)_k=\delta_{nk}.
\end{equation}
Si \( (a_n)_{n\in \eN}\) et \( (b_n)_{n\in\eN}\) sont des éléments de \( \polyP\), nous définissons le produit \( ab\) par
\begin{equation}
    (ab)_n=\sum_{p+q=n}a_pb_q.
\end{equation}
Cela est bien un élément de \( \polyP\) parce qu'il existe \( N\in\eN\) tel que \( a_n=b_n=0\) pour tout \( n\geq N\). Avec la somme et le produit par un scalaire, le module \( \polyP\) devient une \( A\)-algèbre commutative unitaire. L'unité est 
\begin{equation}
    e_0=(1,0,\ldots).
\end{equation}

\begin{definition}
    En tant que \( A\)-algèbre, l'ensemble \( \polyP\) est l'\defe{algèbre des polynômes en une indéterminée}{algèbre!polynômes} à coefficients dans \( A\).
\end{definition}

Si nous posons que \( X=e_1\), et que nous prenons la convention \( X^0=1\), alors nous avons \( e_k=X^k\) et nous notons \( A[X]\) l'anneau \( \polyP\) exprimé avec \( X\). Les éléments de la forme \( \lambda X^k\) avec \( \lambda\in A\) et \( k\in\eN\) sont des \defe{monômes}{monôme}. Nous allons aussi considérer
\begin{equation}\nomenclature[A]{\( A_n[X]\)}{les polynômes à coefficients dans \( A\) et de degré inférieur à \( n\)}
    A_n[X]=\{ P\in A[X]\tq \deg(P)\leq n \}.
\end{equation}
Cela est un sous module libre.

\begin{theorem}
    L'anneau \( A\) est intègre si et seulement si \( A[X]\) est intègre.
\end{theorem}

\begin{proof}
    Soient \( P\) et \( Q\) des éléments non nuls de \( A[X]\). Vu que l'anneau \( A\) est intègre, nous avons
    \begin{equation}
        \deg(PQ)=\deg(P)+\deg(Q)
    \end{equation}
    et le produit ne peut pas être nul. L'anneau \( A[X]\) est donc intègre.

    Si \( A[X]\) est intègre, \( A\) est intègre parce qu'il peut être vu comme sous anneau.
\end{proof}

\begin{remark}
    Si \( A\) n'est pas intègre, soit \( \alpha\beta=0\), alors \( (\alpha X)(\beta x)=0\) et le degré du produit n'est pas la somme des degrés.
\end{remark}

\begin{corollary}
    Si \( A\) est intègre, les inversibles de \( A[X]\) sont les éléments de \( U(A)\).
\end{corollary}

\begin{proof}
    Pour que \( Q\) soit inversible, il faut un \( P\) tel que \( PQ=1\). Mais l'anneau \( A\) étant intègre, les degrés s'additionnent. Par conséquent ils doivent être de degrés zéro et il faut que \( P,Q\in A\). Enfin pour qu'ils soient inversibles, ils doivent être dans \( U(A)\).
\end{proof}

La \defe{valuation}{valuation} de \( P\) du polynôme \( P=\sum_n a_nX^n\), notée \( \val(P)\), est 
\begin{equation}
    \val(P)=\min\{ n\tq a_n\neq 0 \}.
\end{equation}
Nous avons \( \val(P)\leq \deg(P)\) et \( \val(P)=\deg(P)\) si et seulement si \( P\) est un monôme. Si \( P=0\), nous convenons que \( \val(0)=\infty\) et \( \deg(0)=-\infty\).

%---------------------------------------------------------------------------------------------------------------------------
\subsection{Irréductibilité}
%---------------------------------------------------------------------------------------------------------------------------

\begin{theorem}[d'Alembert-Gauss]\index{théorème!d'Alembert-Gauss}      \label{ThovgyUuA}
    Tout polynôme non constant à coefficients complexes possède au moins une racine complexe.
\end{theorem}


Un polynôme est \defe{irréductible}{irréductible!polynôme} lorsqu'il ne peut pas être écrit sous la forme de produits de polynômes de degré supérieurs à \( 1\).

\begin{proposition}
    Un polynôme irréductible à coefficients réels est soit de degré un soit de degré \( 2\) avec un discriminant négatif.
\end{proposition}

\begin{proof}
    Soit un polynôme \( P\) à coefficients réels de degré plus grand que \( 1\). Alors le théorème de d'Alembert-Gauss (théorème \ref{ThovgyUuA}) implique l'existence d'une racine \( \alpha\). Il est facile de montrer que le conjugué complexe \( \bar \alpha\) est également racine. Par conséquent les polynômes \( (X-\alpha)\) et \( (X-\bar \alpha)\) divisent \( P\).

    Ces deux polynômes sont premiers entre eux parce que
    \begin{equation}
        a(X-\alpha)+b(X-\bar \alpha)=0
    \end{equation}
    implique \( a=b=0\). Par conséquent le produit 
    \begin{equation}
        X^2-(\alpha+\bar \alpha)X+\alpha\bar\alpha
    \end{equation}
    divise également \( P\). Ce dernier est un polynôme à coefficients réels de degré \( 2\). Donc tout polynôme de degré \( 3\) ou plus est réductible.
\end{proof}

Nous disons que \( P\in\eK[X]\setminus\eK\) est \defe{scindé}{polynôme!scindé} sur \(\eK\) si il est produit dans \(\eK[X]\) de polynômes de degré \( 1\).

%---------------------------------------------------------------------------------------------------------------------------
\subsection{Division euclidienne}
%---------------------------------------------------------------------------------------------------------------------------

Le théorème suivant établit la \defe{division euclidienne}{division!euclidienne} dans \( \eA[X]\) du polynôme \( A\) par \( B\).
\begin{theorem}     \label{ThodivEuclPsFexf}
    Soit \( B\neq 0\) dans \( \eA[X]\) de coefficient dominant inversible dans \( \eA\). Pour tout \( A\in\eA[X]\), il existe \( P,Q\in \eA[X]\) tels que
    \begin{equation}
        A=BQ+R
    \end{equation}
    avec \( \deg(R)<\deg(B)\).

    Les polynômes \( Q\) et \( R\) sont déterminés de façon univoque par cette condition. Le polynôme \( Q\) est le \defe{quotient}{quotient} et \( R\) est le \defe{reste}{reste} de la division euclidienne de \( A\) par \( B\).
\end{theorem}

Deux polynômes \( P\) et \( Q\) sont dits \defe{étrangers}{étrangers!polynômes} entre eux si \( 1\) est un \( \pgcd\) de \( P\) et \( Q\). Un ensemble de polynômes \( (P_i)_{i\in I}\) est étranger \defe{dans leur ensemble}{étranger!dans leur ensemble} si \( 1\) est un \( \pgcd\) des \( P_i\).

\begin{theorem}[Bezout] \label{ThoBezoutOuGmLB}
    Les polynômes \( P_1,\ldots,P_n\) dans \( \eK[X]\) sont étrangers entre eux si et seulement si il existe des polynômes \( Q_1,\ldots,Q_n\in\eK[X]\) tels que
    \begin{equation}
        P_1Q_1+\ldots+P_nQ_n=1.
    \end{equation}
\end{theorem}

\begin{lemma}       \label{LemuALZHn}
    Soient \( (P_i)_{i=1,\ldots,n}\in \eK[X]\) des polynômes étrangers deux à deux. Alors les polynômes
    \begin{equation}
        Q_i=\prod_{j\neq i}P_j
    \end{equation}
    sont étrangers entre eux\footnote{Et non juste deux à deux.}.
\end{lemma}

%---------------------------------------------------------------------------------------------------------------------------
\subsection{Idéaux}
%---------------------------------------------------------------------------------------------------------------------------

Soit \( P\in \eK[X]\) un polynôme. Nous notons \( (P)\) l'idéal engendré par \( P\) :
\begin{equation}
    (P)=\{ PR\tq R\in\eK[X] \}.
\end{equation}

\begin{lemma}
    Nous avons
    \begin{enumerate}
        \item
            \( (P)\subset (Q)\) si et seulement si \( Q\) divise \( P\),
        \item
            \( (P)=(Q)\) si et seulement si \( P\) et \( Q\) sont multiples (non nuls) l'un de l'autre.
    \end{enumerate}
\end{lemma}

\begin{proof}
    Si \( (P)\subset (Q)\), en particulier \( P\in(Q)\) et il existe \( R\in\eK[X]\) tel que \( P=QR\), ce qui signifie que \( Q\) divise \( P\).

    Si les idéaux de \( P\) et de \( Q\) sont identiques, l'un divise l'autre et l'autre divise l'un. Ils sont donc multiples l'un de l'autre.
\end{proof}

\begin{theorem}     \label{ThoCCHkoU}
    Soit \( I\) un idéal dans \( \eK[X]\). Alors il existe un polynôme \( P\) tel que \( I=(P)\). Plus précisément, si \( P\) est de degré minimal, alors \( (P)=I\).

    De plus si \( I\neq \{  0\}\), il existe un unique polynôme unitaire \( U\) tel que \( I=(U)\).
\end{theorem}

\begin{proof}
    Si \( I=\{ 0 \}\), le résultat est évident. Nous supposons donc \( I\) non nul. Soit \( P\) de degré minimum parmi les éléments de \( I\). Évidemment \( (P)\subset I\). Nous allons démontrer qu'en réalité \( (P)=I\).

    Soit \( A\in I\). Par le théorème \ref{ThodivEuclPsFexf} de la division euclidienne\footnote{Ici \( \eK\) est un corps et donc l'hypothèse d'inversibilité est automatiquement vérifiée.}, il existe \( Q\) et \( R\) dans \( \eK[X]\) tels que \( A=PQ+R\) avec \( \deg(R)<\deg(P)\). Étant donné que \( R=A-PQ\) nous avons \( R\in I\) et par conséquent \( R=0\) parce que \( P\) a été choisit de degré minimum dans \( I\). Nous avons donc \( A=PQ\) et \( I\subset (P)\).

    L'existence d'un polynôme unitaire qui génère \( I\) est obtenu en choisissant \( U=P/a_n\) où \( a_n\) est le coefficient du terme de plus haut degré.
\end{proof}
Nous voyons que n'importe quel polynôme de degré minimum dans un idéal génère l'idéal. Une importante conséquence du théorème \ref{ThoCCHkoU} que nous verrons plus bas est que tout polynôme annulateur d'un endomorphisme est divisé par le polynôme minimal (proposition \ref{PropAnnncEcCxj}).

%+++++++++++++++++++++++++++++++++++++++++++++++++++++++++++++++++++++++++++++++++++++++++++++++++++++++++++++++++++++++++++
\section{Racines de polynômes}
%+++++++++++++++++++++++++++++++++++++++++++++++++++++++++++++++++++++++++++++++++++++++++++++++++++++++++++++++++++++++++++

Soit \( \eA\) un anneau et \( P\in \eA[X]\) un polynôme et \( \alpha\in \eA\). Le \defe{degré}{degré!d'une racine} ou la \defe{multiplicité}{multiplicité!d'une racine} de \( \alpha\) par rapport à \( P\) est l'entier \( h\) tel que \( P\) est divisible par \( (X-\alpha)^h\) mais pas divisible par \( (X-\alpha)^{h+1}\).

Nous noterons \( \theta_{\alpha}(P)\)\nomenclature[A]{\( \theta_{\alpha}(P)\)}{l'ordre de \( \alpha\) par rapport à \( P\)} l'ordre de \( \alpha\) par rapport à \( P\).

\begin{proposition}     \label{PropahQQpA}
    L'élément \( \alpha\in \eA\) est d'ordre \( h\) par rapport à \( \) si et seulement si il existe \( Q\in\eA[X]\) tel que \( P(X-\alpha)^hQ\) avec \( Q(\alpha)\neq 0\).
\end{proposition}

\begin{lemma}       \label{LemIeLhpc}
    Soient \( P\) et \( Q\) des polynômes non nuls de \( \eA[X]\) et \( \alpha\in \eA\) d'ordre \( p\) pour \( P\) et d'ordre \( q\) pour \( Q\). Alors
    \begin{enumerate}
        \item
            \( \theta_{\alpha}(P+Q)\geq\ln\{ \theta_{\alpha}(P),\theta_{\alpha}(Q) \}\)
        \item
            si \( \theta_{\alpha}(P)\neq \theta_{\alpha}(Q)\), alors \( \theta_{\alpha}(P+Q)=\min\{ \theta_{\alpha}(P),\theta_{\alpha}(Q) \}\)
        \item
            \( \theta_{\alpha}(PQ)\geq \theta_{\alpha(P)}+\theta_{\alpha}(Q)\);
        \item       \label{ItemIeLhpciv}
            si \(\eA \) est intègre alors \( \theta_{\alpha}(PQ)= \theta_{\alpha}(P)+\theta_{\alpha}(Q)\);
    \end{enumerate}
\end{lemma}

\begin{theorem}
    Soit \( \eA\) un anneau intègre et \( P\in \eA[X]\setminus\{ 0 \}\), un polynôme de degré \( n\). Si \( \alpha_1,\ldots, \alpha_p\in\eA\) sont des racines deux à deux distinctes d'ordres \( k_1,\ldots, k_p\), alors il existe \( Q\in \eA[X]\) tel que
    \begin{enumerate}
        \item
            \( Q(\alpha_i)\neq 0\);
        \item
            \( P=Q\prod_{i=1}^p(X-\alpha_i)\);
    \end{enumerate}
    De plus la sommes des ordres des racines de \( P\) est au plus \( \deg(P)\).
\end{theorem}

\begin{proof}
    Si \( p=1\), alors le résultat est la proposition \ref{PropahQQpA}. Nous supposons que \( p\geq 2\) et nous effectuons une récurrence sur \( P\). Nous considérons donc pas \( p-1\) premières racines \( \alpha_1,\ldots, \alpha_{p-1}\) et un polynôme \( R\in\eA[X]\) tel que \( R(\alpha_i)\neq 0\) pour \( i=1,\ldots, p-1\) et
    \begin{equation}
        P=\underbrace{(X-\alpha_1)^{k_1}\ldots (X-\alpha_{p-1})^{k_{p-1}}}_SR.
    \end{equation}
    Par hypothèse \( P(\alpha_p)=S(\alpha_p)R(\alpha_p)=0\). L'anneau \( \eA\) étant intègre, \( S(\alpha_p)\neq 0\) parce que \( \alpha_i\neq \alpha_p\) pour \( i\neq p\). Par conséquent, \( R(\alpha_p)=0\).
    
    Nous devons encore vérifier que l'ordre de \( \alpha_p\) est \( k_p\) par rapport à \( R\). Pour cela nous utilisons le point \ref{ItemIeLhpciv} du lemme \ref{LemIeLhpc} affin de dire que le degré de \( \alpha_p\) pour \( P=SR\) est \( k_p\). Par conséquent
    \begin{equation}
        R=(X-\alpha_p)^{k_p}T
    \end{equation}
    avec \( T(\alpha_p)\neq 0\) et enfin
    \begin{equation}
        P=\prod_{i=1}^p(X-\alpha_i)T.
    \end{equation}
    De plus \( T(\alpha_i)\neq 0\), sinon \( R(\alpha_i)\) serait nul.
\end{proof}


%+++++++++++++++++++++++++++++++++++++++++++++++++++++++++++++++++++++++++++++++++++++++++++++++++++++++++++++++++++++++++++
\section{Polynômes cyclotomiques}
%+++++++++++++++++++++++++++++++++++++++++++++++++++++++++++++++++++++++++++++++++++++++++++++++++++++++++++++++++++++++++++

Soit \( n\in \eN^*\). Nous considérons le polynôme \( (X^n-1)\in\eC[X]\). Les racines de ce polynôme forment le groupe
\begin{equation}
    \gU_n=\{ \xi^k\tq k=0,\ldots, n-1 \}
\end{equation}
avec \( \xi= e^{2i\pi/n}\). Ce groupe est un groupe cyclique généré par \( \xi\). Les autres générateurs sont les \( \xi^p\) avec \( \pgcd(p,n)=1\). Nous notons \( \Delta_n\) l'ensemble des générateurs de \( \gU_n\) :
\begin{equation}
    \Delta_n=\{  e^{2ki\pi/n}\tq 0\leq k\leq n-1,\pgcd(k,n)=1 \}.
\end{equation}
Ces éléments sont les \defe{racines primitives}{racine!primitive de l'unité} de l'unité dans \( \eC\). Nous avons 
\begin{equation}
    \Card(\Delta_n)=\varphi(n)
\end{equation}
où \( \varphi\) est la fonction d'Euler définie par \eqref{EqEulerGqPsvi}.

Nous avons par exemple
\begin{subequations}
    \begin{align}
        \Delta_1&=\{ 1 \}\\
        \Delta_2&=\{  e^{\pi i} \}\\
        \Delta_4&=\{  e^{\pi i/2}, e^{3\pi i/2} \}.
    \end{align}
\end{subequations}

\begin{lemma}
    Nous avons
    \begin{equation}        \label{EqpZuIyL}
        \gU_n=\bigcup_{d\divides n}\Delta_d
    \end{equation}
    et l'union est disjointe.

    Nous avons aussi la formule
    \begin{equation}        \label{EqTPHqgJ}
        n=\sum_{d\divides n}\varphi(d).
    \end{equation}
\end{lemma}

\begin{proof}
    À l'application \( x\mapsto  e^{2i\pi x}\) près, nous pouvons considérer
    \begin{equation}
        \Delta_d=\{ \frac{ k }{ d }\tq k=0,\ldots, d-1, \pgcd(k,d)=1 \},
    \end{equation}
    c'est à dire l'ensemble des fractions irréductibles dont le dénominateur est \( d\). L'union des \( \Delta_d\) sera donc disjointe.
    
    Toujours à l'application \( x\mapsto  e^{2i\pi x}\) près, le groupe \( \gU_n\) est donné par
    \begin{equation}
        \gU_n=\{ \frac{ k }{ n }\tq k=0,\ldots, n-1 \}.
    \end{equation}
    L'égalité \eqref{EqpZuIyL} revient maintenant à dire que toute fraction de la forme \( \frac{ k }{ n }\) s'écrit de façon irréductible avec un dénominateur qui divise \( n\).

    La relation \eqref{EqTPHqgJ} consiste à prendre le cardinal des deux côtés de \eqref{EqpZuIyL}. Nous avons \( \Card(\gU_n)=n\) et l'union étant disjointe, à droite nous avons la somme des cardinaux.
\end{proof}

%---------------------------------------------------------------------------------------------------------------------------
\subsection{Suite}
%---------------------------------------------------------------------------------------------------------------------------

Soit \( n\in\eN^*\) et \( \eK\), un corps. On définit 
\begin{equation}
    \mu_n(\eK)=\{ \xi\in\eK\tq \xi^n=1\}.
\end{equation}
Les éléments de \( \mu_n(\eK)\) sont les \defe{racines de l'unité}{racine!de l'unité} dans \( \eK\).

\begin{lemma}
    L'ensemble \( \mu_n(\eK)\) est un groupe cyclique d'ordre au plus \( n\) dans \( \eK\).
\end{lemma}


