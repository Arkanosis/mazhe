% This is part of Mes notes de mathématique
% Copyright (c) 2011-2012
%   Laurent Claessens
% See the file fdl-1.3.txt for copying conditions.

%+++++++++++++++++++++++++++++++++++++++++++++++++++++++++++++++++++++++++++++++++++++++++++++++++++++++++++++++++++++++++++
\section{Généralités}
%+++++++++++++++++++++++++++++++++++++++++++++++++++++++++++++++++++++++++++++++++++++++++++++++++++++++++++++++++++++++++++

Source : \cite{Tauvel}.

\begin{definition}
    Un \defe{anneau}{anneau} est un triple \( (A,+,\cdot)\) avec les conditions
    \begin{enumerate}
        \item
            \( (A,+)\) est un groupe abélien. Nous notons \( 0\) le neutre.
        \item
            La multiplication est associative et nous notons \( 1\) le neutre
        \item
            La multiplication est distributive par rapport à l'addition.
    \end{enumerate}
\end{definition}

\begin{remark}
    Un anneau est ce qu'on appelle «\emph{ring}» en anglais. Un corps est en anglais «\emph{field}».
\end{remark}

Soit \( X\) un ensemble et $A$ un anneau. Nous considérons \( \Fun(X,A)\)\nomenclature[A]{\( \Fun(X,Y)\)}{les applications de \( X\) vers \( Y\)} l'ensemble des applications \( X\to A\). Cet ensemble devient un anneau avec les définitions
\begin{subequations}
    \begin{align}
        (f+g)(x)=f(x)+g(x)\\
        (fg)(x)=f(x)g(x).
    \end{align}
\end{subequations}
Cela est la \defe{structure canonique}{structure d'anneau canonique} d'anneau sur \( \Fun(X,A)\).

Le \defe{centralisateur}{centralisateur} de \( x\in A\) dans \( A\) est l'ensemble
\begin{equation}
    \{ y\in A\tq xy=yx \},
\end{equation}
le \defe{centre}{centre} de \( A\) est
\begin{equation}
    \{ y\in A\tq xy=yx\forall x\in A \}.
\end{equation}
Un élément \( a\neq 0\) est un \defe{diviseur de zéro à gauche}{diviseur!de zéro} si il existe \( x\neq 0\) tel que $xa=0$. L'élément \( a\) est un diviseur de zéro \defe{à droite}{diviseur!de zéro à droite} si il existe \( b\) tel que \( ab=0\). Un anneau est \defe{intègre}{intègre!anneau}\index{anneau!intègre} si il est non nul et ne possède pas de diviseurs de zéro.

\begin{example}
    L'ensemble \( \eZ\) avec les opérations usuelles est un anneau intègre.
\end{example}

Un élément \( a\in A\) est \defe{régulier à droite}{régulier à droite} \( ba=0\) implique \( b=0\). Il est régulier ) gauche si \( ab=0\) implique \( b=0\).

L'ensemble \( U(A)\)\nomenclature[A]{\( U(A)\)}{ensemble des inversibles} des éléments inversibles de \( A\) est un groupe pour la multiplication. Nous notons \( A^*=A\setminus\{ 0 \}\).

\begin{lemma}
    Si \( a\) et \( b\) commutent, nous avons la formule
    \begin{equation}        \label{Eqarpurmkbk}
        a^{r+1}-b^{r+1}=(a-b)(\sum_{k=0}^ra^{r-k}b^k).
    \end{equation}
\end{lemma}

\begin{proposition}
    Si \( a\) est un élément nilpotent de l'anneau \( A\), alors \( 1-a\) est inversible. Si \( a\) est nilpotent non nul, alors il est diviseur de zéro.
\end{proposition}

\begin{proof}
    Soit \( n\) le minimum tel que \( a^n=0\). En vertu de la formule \eqref{Eqarpurmkbk} nous avons
    \begin{equation}
        1=1-a^n=(1-a)(1+a+\ldots+a^{n-1})=(1+a+\ldots+a^{n-1})(1-a).
    \end{equation}
    La somme \( 1+a+\ldots+a^{n-1}\) est donc un inverse de \( (1-a)\).
\end{proof}

\begin{definition}
    Si \( A\) et \( B\) sont des anneaux, un \defe{morphisme}{morphisme!d'anneaux} est une application \( f\colon A\to B\) telle que pour tout \( x,y\in A\) nous ayons
    \begin{enumerate}
        \item
            \( f(x+y)=f(x)+f(y)\)
        \item
            \( f(xy)=f(x)f(y)\)
        \item
            \( f(1)=1\)
    \end{enumerate}
\end{definition}

Si \( f\) est un morphisme, nous avons \( f(0)=0\) et \( f(x)^{-1}=f(x^{-1})\).

%+++++++++++++++++++++++++++++++++++++++++++++++++++++++++++++++++++++++++++++++++++++++++++++++++++++++++++++++++++++++++++
\section{Idéaux dans des anneaux}
%+++++++++++++++++++++++++++++++++++++++++++++++++++++++++++++++++++++++++++++++++++++++++++++++++++++++++++++++++++++++++++

\begin{definition}
    Un sous ensemble \( B\subset A\) d'un anneau est un \defe{sous anneau}{sous anneau} si
    \begin{enumerate}
        \item
            \( 1\in B\)
        \item
            \( B\) est un sous groupe pour l'addition
        \item
            \( B\) est stable pour la multiplication.
    \end{enumerate}
    Un sous ensemble \( I\subset A\) est un \defe{idéal}{idéal!dans un anneau} à gauche si
    \begin{enumerate}
        \item
            \( I\) est un sous groupe pour l'addition
        \item
            si \( x\in I\) et \( a\in A\), alors \( ax\in I\).
    \end{enumerate}
\end{definition}

Lorsqu'un ensemble est idéal à gauche et à droite, nous disons que c'est un \defe{idéal bilatère}{idéal!bilatère}. Lorsque nous parlons d'idéal sans précisions, nous parlons d'idéal bilatère.

\begin{remark}
    Un idéal n'est pas toujours un anneau parce que l'identité pourrait manquer. Un idéal qui contient l'identité est l'anneau complet.
\end{remark}

\begin{example}
    L'ensemble \( 2\eZ\) est un idéal de \( \eZ\). Tous les idéaux de \( \eZ\) sont de la forme \( n\eZ\). En effet en vertu de la proposition \ref{PropSsgpZestnZ}, les seule sous groupes de \( \eZ\) (en tant que groupe additif) sont les \( n\eZ\).
\end{example}

Soit \( A\), un anneau, \( I\) un idéal bilatère de \( A\). Nous considérons la relation d'équivalence \( x\sim y\) si et seulement si \( x-y\in I\). Dans ce cas, le quotient
\begin{equation}
    A/\sim=A/I
\end{equation}
est un anneau appelé \defe{anneau quotient}{anneau!quotient par un idéal}. La surjection \( A\to A/I\) est un morphisme.

\begin{proposition}
    Soient \( A\) et \( B\) des anneaux et un homomorphisme \( f\colon A\to B\). Nous considérons l'injection canonique \( j\colon f(A)\to B\) et la surjection canonique \( \phi\colon A\to A/\ker f\). Alors il existe un unique isomorphisme
    \begin{equation}
        \tilde f \colon A/\ker f\to f(A)
    \end{equation}
    tel que \( f=j\circ\tilde f\circ\phi\).

    \begin{equation}
        \xymatrix{%
        A \ar[r]^{f}\ar[d]_{\phi}        &   B\ar[d]^{j}\\
           A/\ker f \ar[r]_{\tilde f}   &   f(A)\subset B
           }
    \end{equation}
\end{proposition}

\begin{proposition}     \label{PropIJJIdsousphi}
    Soit \( I\), un idéal de \( A\) et \( \phi\colon A\to A/I\) la surjection canonique. Les idéaux de \( A/I\) sont les \( \phi(J)\) où \( J\) est un idéal de \( A\) contenant \( I\). De plus cette relation est bijective :
    \begin{equation}        \label{EqKbrizu}
        \{ \text{idéaux de \( A\) contenant \( I\)}\}\simeq\{ \text{idéaux de \( R/I\)} \}.
    \end{equation}
\end{proposition}

\begin{proof}
    Si \( I\subset J\) et si \( J \) est un idéal de \( A\), alors \( \phi(J)\) est un idéal dans \( A/I\). En effet un élément de \( \phi(J)\) est de la forme \( \phi(j)\) et un élément de \( A/I\) est de la forme \( \phi(i)\). Leur produit vaut
    \begin{equation}
        \phi(i)\phi(j)=\phi(ij)\in\phi(J).
    \end{equation}
    
    Soit maintenant \( K\), un idéal dans \( A/I\). Soit \( J=\phi^{-1}(K)\). Étant donné qu'un idéal doit contenir \( 0\) (parce qu'un idéal est un groupe pour l'addition), \( [0]\in K\) et par conséquent \( I\subset\phi^{-1}(K)\).

\end{proof}

\begin{corollary}
    Les quotients de \( \eZ\) sont \( \eZ_n\)
\end{corollary}

\begin{proof}
    Nous avons déjà vu que les seuls idéaux de \( \eZ\) sont les \( n\eZ\).
\end{proof}

\begin{proposition}     \label{PropZpintssiprempUzn}
    Soit \( n\geq 2\) un entier et \( \phi\colon \eZ\to \eZ_n\), la surjection canonique. Nous noterons \( \tilde a=\phi(a)\). Alors
    \begin{equation}
        U(\eZ_n)=\phi(P_n)=\{ \tilde x\tq 0\leq x\leq n\tq\pgcd(x,n)=1 \}.
    \end{equation}
    où \( P_n\) est l'ensemble décrit par l'équation \eqref{EqDefPnEntierldeost}. En particulier, \( \Card\big( U(\eZ_n) \big)=\varphi(n)\).

\end{proposition}

\begin{proof}
    Soit \( 0\leq x\leq n\) tel que \( \pgcd(x,n)=1\). Il existe donc \( p,q\in\eZ\) tels que \( px+qn=1\). En passant aux classes,
    \begin{equation}
        \tilde p\tilde x=\tilde 1,
    \end{equation}
    donc \( \tilde p\) est l'inverse de \( \tilde x\). Cela prouve que \( \phi(P_n)\subset U(\eZ_n)\).

    Nous prouvons maintenant l'inclusion inverse. Soit \( \tilde x\) et \( \tilde y\) inverses l'un de l'autre : $\tilde x\tilde y=\tilde 1$. Il existe donc \( q\in\eZ\) tel que \( xy-qn=1\), ce qui prouve que \( \pgcd(x,n)=1\).

\end{proof}

\begin{corollary}   \label{CorZnInternprem}
    L'anneau \( \eZ_n\) est intègre si et seulement si \( n\) est premier.
\end{corollary}

\begin{proof}
    Si \( n\) est premier, tous les éléments de \( \eZ_n\) sont inversibles parce que tous les éléments rentrent dans \( \phi(P_n)\). Donc \( \eZ_n\) est intègre.

    Si \( n\) n'est pas premier, il existe \( p,q\in\eN^*\) tels que \( pq=n\). Dans ce cas au niveau des classes nous avons \( \tilde p\tilde q=0\) avec \( \tilde p\neq 0\neq\tilde q\), ce qui montre que \( \eZ_n\) a des diviseurs de zéro et n'est pas intègre.
\end{proof}

%---------------------------------------------------------------------------------------------------------------------------
\subsection{Caractéristique}
%---------------------------------------------------------------------------------------------------------------------------

L'application 
\begin{equation}
    \begin{aligned}
        \mu\colon \eZ&\to A \\
        n&\mapsto n\cdot 1_A 
    \end{aligned}
\end{equation}
est un morphisme d'anneaux. Le noyau de \( \mu\) étant un sous groupe de \( \eZ\), il existe un et un seul \( p\in\eZ\) tel que \( \ker\mu=p\eZ\). Ce \( p\) est la \defe{caractéristique}{caractéristique!d'un anneau} de \( A\).

Par exemple la caractéristique que \( \eQ\) est zéro parce qu'aucun multiple de l'unité n'est nul.

\begin{lemma}
    Si \( A\) est de caractéristique nulle, alors \( A\) est infini.
\end{lemma}

\begin{proof}
    En effet, \( \ker\mu=0\) implique que \( n1_A\neq  m1_A\) et par conséquent \( A\) est infini.
\end{proof}

\begin{lemma}       \label{LemHmDaYH}
    Si \( p\) est la caractéristique de l'anneau \( A\), alors nous avons l'isomorphisme d'anneaux
    \begin{equation}
         \eZ 1_A\simeq\eZ/p\eZ.
    \end{equation}
\end{lemma}

\begin{proof}
    L'isomorphisme est donné par l'application \( n1_A\mapsto \phi(n)\) si \( \phi\) est la projection canonique \( \eZ\to \eZ_p\).
\end{proof}

\begin{lemma}       \label{LemCaractIntergernbrcartpre}
    La caractéristique d'un anneau intègre est zéro ou un nombre premier.
\end{lemma}

\begin{proof}
    Si \( A\) est intègre, alors \( \eZ 1_A\) est intègre (a fortiori), et \( \eZ_p\) est intègre parce qu'il est isomorphe à \( \eZ A_A\). Mais nous savons que \( \eZ_p\) est intègre si et seulement si \( p\) est premier (proposition \ref{CorZnInternprem}).
\end{proof}

\begin{example}
    Il existe des corps dont la caractéristique n'est pas égale au cardinal (contrairement à ce que laisserait penser l'exemple des \( \eZ/p\eZ\)). En effet les matrices \( n\times n\) inversibles sur \( \eF_{3}\) forment un corps qui n'est pas de cardinal trois alors que la caractéristique est \( 3\) :
    \begin{equation}
        \begin{pmatrix}
            1    &       \\ 
                &   1    
            \end{pmatrix}+\begin{pmatrix}
                1    &       \\ 
                    &   1    
                \end{pmatrix}+\begin{pmatrix}
                    1    &       \\ 
                        &   1    
                \end{pmatrix}=0.
    \end{equation}
\end{example}

\begin{proposition}     \label{PropGExaUK}
    La caractéristique d'un anneau fini divise son cardinal.
\end{proposition}

\begin{proof}
    Si \( \eA\) est un anneau, le groupe \( \eZ\) agit sur \( \eA\) par
    \begin{equation}
        n\cdot a=a+n1_A.
    \end{equation}
    Chaque orbite de cette action est de la forme
    \begin{equation}
        \mO_a=\{ a+n1_A\tq n=0,\ldots, p-1 \}
    \end{equation}
    où \( p\) est la caractéristique de \( \eA\). Les orbites ont \( p\) éléments et forment une partition de \( \eA\), donc le cardinal de \( \eA\) est un multiple de \( p\).
\end{proof}

%---------------------------------------------------------------------------------------------------------------------------
\subsection{Anneaux factoriels}
%---------------------------------------------------------------------------------------------------------------------------

On dit que les éléments \( a\) et \( b\) d'un anneau sont \defe{associés}{associé}\index{éléments!associés} si il existe un élément \( u\) inversible dans \( \eA\) tel que \( a=ub\).

\begin{definition}
    Un anneau commutatif unitaire \( \eA\) est \defe{factoriel}{factoriel!anneau}\index{anneau!factoriel} si il vérifie les propriétés suivantes.
    \begin{enumerate}
        \item
            L'anneau \( \eA\) est intègre.
        \item
            Si \( a\in \eA\) est non nul et non inversible alors il admet une décomposition \( a=p_1\ldots p_k\) où les \( p_i\) sont irréductibles.
        \item
            Si \( a=q_1\ldots q_m\) est une autre décomposition de \( a\) en irréductibles, alors \( m=k\) et il existe une permutation \( \sigma\in S_k\) telle que \( p_i\) et \( q_{\sigma(i)}\) soient associés.
    \end{enumerate}
\end{definition}

%+++++++++++++++++++++++++++++++++++++++++++++++++++++++++++++++++++++++++++++++++++++++++++++++++++++++++++++++++++++++++++
\section{Corps}
%+++++++++++++++++++++++++++++++++++++++++++++++++++++++++++++++++++++++++++++++++++++++++++++++++++++++++++++++++++++++++++

\begin{definition}
    Un \defe{corps}{corps} est un anneau non nul dans lequel tout élément non nul est inversible.
\end{definition}

\begin{lemma}       \label{LemAnnCorpsnonInterdivzer}
    En tant que anneau, un corps n'a pas de diviseurs zéro.
\end{lemma}

\begin{proof}
    En effet si \( a\) est un diviseur de zéro, alors \( ax=0\) pour un certain \( x\neq 0\). Si \( a\) était inversible, nous aurions \( x=a^{-1}ax=0\), ce qui est impossible.
\end{proof}

Par le lemme \ref{LemCaractIntergernbrcartpre}, un anneau sans diviseur de zéro est de caractéristique zéro ou première. Le fait d'être intègre pour un anneau n'assure cependant pas le fait d'être un corps. Nous avons cependant ce résultat pour les anneaux finis.

\begin{proposition}     \label{PropanfinintimpCorp}
    Un anneau fini intègre est un corps.
\end{proposition}

\begin{proof}
    Soit \( A\) un tel anneau. Soit \( a\neq 0\). Les applications 
    \begin{subequations}
        \begin{align}
            l_a\colon x\to ax\\
            r_a\colon x\to xa
        \end{align}
    \end{subequations}
    sont injectives. En tant que applications injectives entre ensembles finis, elles sont surjectives. Il existe donc \( b\) et \( c\) tels que \( 1=ba=ac\). Il se fait que \( b\) et \( c\) sont égaux parce que
    \begin{equation}
        b=b(ac)=(ba)c=c.
    \end{equation}
    Par conséquent \( b\) est un inverse de \( a\).
\end{proof}

\begin{proposition}     \label{PropzhFgNJ}
    Soit \( n\in\eN^*\). Les conditions suivantes sont équivalentes :
    \begin{enumerate}
        \item
            \( n\) est premier.
        \item
            \( \eZ_n\) est un anneau intègre.
        \item
            \( \eZ_n\) est un corps.
    \end{enumerate}
\end{proposition}

\begin{proof}
    L'équivalence entre les deux premiers points est le contenu du corollaire \ref{CorZnInternprem}. Le fait que \( \eZ_n\) soit un corps lorsque \( \eZ_n\) est intègre est la proposition \ref{PropanfinintimpCorp}. Le fait que \( \eZ_n\) soit intègre lorsque \( \eZ_n\) est un corps est une propriété générale des corps : ce sont en particulier des anneaux intègres (lemme \ref{LemAnnCorpsnonInterdivzer}).
\end{proof}

\begin{proposition}     \label{AnnCorpsIdeal}
    Si \( A\) est un anneau, nous avons les équivalences
    \begin{enumerate}
        \item
            \( A\) est un corps.
        \item
            \( A\) est non nul et ses seuls idéaux à gauche sont \( \{ 0 \}\) et \( A\).
        \item
            \( A\) est non nul et ses seuls idéaux à droite sont \( \{ 0 \}\) et \( A\).
    \end{enumerate}
\end{proposition}

\begin{proposition}
    Soit \( A\), un anneau commutatif non nul et \( I\), un idéal dans \( A\). L'ensemble \( I\) est un idéal maximum de \( A\) si et seulement si \( A/I\) est un corps.
\end{proposition}

\begin{proof}
    Par la proposition \ref{AnnCorpsIdeal}, le fait pour \( A/I\) d'être un corps signifie qu'il n'a pas d'idéaux non triviaux. Si \( \phi\) est la projection canonique \( A\mapsto A/I\), nous savons que les idéaux de \( A/I\) sont les \( \phi(J)\) où \( J\) est un idéal de \( A\) contenant \( I\) (proposition \ref{PropIJJIdsousphi}). Si \( I\) est un idéal maximum de \( A\), un tel \( J\) n'existe pas. Inversement si \( A/I\) n'a pas d'autres idéaux que \( A/I\) et \( \phi(0)\), c'est que \( I\) est un idéal maximum.
\end{proof}

%---------------------------------------------------------------------------------------------------------------------------
\subsection{Morphisme de Frobenius}
%---------------------------------------------------------------------------------------------------------------------------

\begin{proposition}     \label{Propqrrdem}
    Soit \( \eA\) un anneau commutatif de caractéristique première \( p\). Alors \( \sigma(x)=x^p\) est un automorphisme de l'anneau \( \eA\). Nous avons la formule
    \begin{equation}
        (a+b)^p=a^p+b^p
    \end{equation}
    pour tout \( a,b\in \eA\).
\end{proposition}

\begin{proof}
    Nous utilisons la formule du binôme de la proposition \ref{PropBinomFExOiL} et le fait que les coefficients binomiaux non extrêmes sont divisibles par \( p\) et donc nuls.
\end{proof}

\begin{proposition} \label{PropFrobHAMkTY}
    Soit \( \eA\) un anneau commutatif unitaire de caractéristique \( p\). L'application
    \begin{equation}
        \begin{aligned}
            \Frob_\eA\colon \eA&\to \eA \\
            x&\mapsto x^p 
        \end{aligned}
    \end{equation}
    est un automorphisme d'anneau unitaire.
\end{proposition}
Nous le nommons le \defe{morphisme de Frobenius}{morphisme!Frobenius}\index{Frobenius!morphisme}. Nous utiliserons aussi les itérés du morphisme de Frobenius : \( \Frob^k\colon x\mapsto x^{p^k}\).

%---------------------------------------------------------------------------------------------------------------------------
\subsection{Corps des fractions}
%---------------------------------------------------------------------------------------------------------------------------

\begin{theorem}     \label{ThogbhWgo}
    Soit \( \eA\) un anneau commutatif intègre. Il existe un corps commutatif \( \eK\) et un morphisme injectif (d'anneaux) \( \epsilon\colon \eA\to \eK\) tels que pour tout \( \lambda\in\eK\), il existe \( (a,b)\in \eA\times \eA^*\) tels que
    \begin{equation}
        \lambda=\epsilon(a)\big( \epsilon(b) \big)^{-1}
    \end{equation}

    De plus si \( (\eK',\epsilon')\) est un autre couple qui vérifie la propriété, les corps \( \eK\) et \( \eK'\) sont isomorphes.
\end{theorem}
Le corps \( \eK\) associé à l'anneau \( \eA\) est le \defe{corps des fractions}{corps!des fractions}\index{fractions (corps)} de \( \eA\).

Notons que l'application \( \eA\times \eA^*\to \eK\) donnée par \( (a,b)\mapsto \epsilon(a)\big( \epsilon(b) \big)^{-1}\) envoie \( (xa,xb)\) sur le même que \( (a,b)\).

L'ensemble \( \eQ\)\nomenclature[A]{\( \eQ\)}{corps des fractions sur \( \eZ\)} est le corps des fractions de \( \eZ\).

%---------------------------------------------------------------------------------------------------------------------------
\subsection{Corps premier}
%---------------------------------------------------------------------------------------------------------------------------
\label{subseccorpspremhBlYIv}

\begin{definition}
    Un corps est \defe{premier}{corps!premier}\index{premier!corps} si il est son seul sous corps. Le \defe{sous corps premier}{premier!sous corps} d'un corps est l'intersection de tous ses sous corps.
\end{definition}

\begin{lemma}
    Un corps premier est commutatif
\end{lemma}

\begin{proof}
    Le centre d'un corps est certainement un sous corps. Par conséquent un corps premier doit être contenu dans son propre centre, c'est à dire être commutatif.
\end{proof}

Soit \( p\) un nombre premier. Nous notons \( \eF_p=\eZ_p=\eZ/p\eZ\)\nomenclature[A]{\( \eF_p\)}{lorsque \( p\) est premier}. Nous verrons plus loi (section \ref{SecCorpsFinizkAcbS}) comment nous pouvons définir \( \eF_{p^n}\) lorsque \( p\) est premier.

Nous avons par exemple 
\begin{equation}
    \eF_2=\eZ/2\eZ=\{ 0,1 \}
\end{equation}
avec la loi \( 2=0\).

Notons que \( \eF_p\) est un corps possédant \( p\) éléments. L'ensemble \( \eF_p^*\) est un groupe d'ordre \( p-1\).

\begin{lemma}
    Les corps \( \eQ\) et \( \eF_p\) (avec \( p\) premier) sont premiers.
\end{lemma}

\begin{proof}
    Tout sous corps de \( \eQ\) doit contenir \( 1\), et par conséquent \( \eZ\). Devant également contenir tous les inverses, il contient \( \eQ\).

    Tout sous corps de \(\eF_p \) doit contenir \( 1\) et donc \( \eF_p\) en entier. Par ailleurs nous savons de la proposition \ref{PropzhFgNJ} que \( \eF_p\) est un corps lorsque \( p\) est premier.
\end{proof}

\begin{proposition}
    Soit \( \eK\) un corps de caractéristique \( p\) et \( \eP\) son sous corps premier. Si \( p=0\) alors \( \eP=\eQ\). Si \( p>0\), alors \( \eP=\eF_p\).
\end{proposition}

\begin{proof}
    Notons d'abord que la caractéristique d'un corps est toujours un nombre premier parce qu'un corps est en particulier un anneau intègre (proposition \ref{LemCaractIntergernbrcartpre}).

    Étant donné que \( 1\) est dans tout sous corps, nous devons avoir \( \eZ 1\subseteq \eP\). Si \( p=0\), alors \( \eZ 1\simeq \eZ\), et nous avons
    \begin{equation}
        \eZ 1_{\eA}\subset \eP\subset \eK.
    \end{equation}
    Pour chaque \( (n,m)\in \eZ 1_{\eA}\times (\eZ 1_{\eA})^*\) l'élément \( nm^{-1}\in \eK\) est dans \( \eP\) parce que \( \eP\) est un corps. Nous en déduisons que le corps des fractions de \( \eZ\) est contenu dans \( \eP\) par conséquent \( \eP=\eQ\) (théorème \ref{ThogbhWgo}). 

    Si par contre la caractéristique de \( \eK\) est \( p\neq 0\), nous avons \( \eZ 1_{\eA}\simeq\eZ/p\eZ=\eF_p\) par le lemme \ref{LemHmDaYH}. L'ensemble \( \eF_p\) étant un corps, c'est le corps premier de \( \eK\).
\end{proof}

\begin{proposition}     \label{PropqPPrgJ}
    Soit \( \eK\) un corps et \( \eP\) sont sous corps premier. Si \( \sigma\in\Aut(\eK)\) alors \( \sigma|_{\eP}=\id\), c'est à dire que
    \begin{equation}
        \sigma(x)=x
    \end{equation}
    pour tout \( x\in \eP\).
\end{proposition}

\begin{theorem}[Petit théorème de Fermat]       \label{ThoOPQOiO}
    Soit \( p\) un nombre premier. Si \( x\in \eF_p\) alors \( x^p=x\). Si \( x\in(\eF_p)^*\), alors \( x^{p-1}=1\).

    En particulier si \( x\in \eF_p^*\) alors \( x^{-1}=x^{p-2}\).
\end{theorem}

\begin{proof}
    Étant donné que \( \eF_p\) est un corps commutatif et que \( p\) est premier, la proposition \ref{Propqrrdem} nous indique que \( \sigma(x)=x^p\) est un automorphisme. La proposition \ref{PropqPPrgJ} nous indique alors que
    \begin{equation}
        a^p=a.
    \end{equation}
    Si \( a\) est inversible alors \( a^{p-1}=a^pa^{-1}=1\).
\end{proof}

\begin{example}
    Soit \( \eK=\eF_{29}^*\). Le nombre \( 29\) étant premier, \( \eK\) est un corps premier. C'est le corps des entier modulo \( 29\). Nous avons donc
    \begin{equation}
            -142=-113=-84=-55=-26=3=32=61=90=119.
    \end{equation}
    Le petit théorème de Fermat nous permet aussi de calculer des exposants et des inverses. Nous avons
    \begin{equation}
        x^{-a}=(x^a)^{27}=x^{27a}.
    \end{equation}
    Le nombre \( 27 a\) peut être grand par rapport à \( 29\). Étant donné que \( 1=x^{28}\) nous avons
    \begin{equation}
        x^{-a}=x^{27 a\mod 28}.
    \end{equation}
    Dans les exposants nous calculons modulo \( 28\).
\end{example}

%+++++++++++++++++++++++++++++++++++++++++++++++++++++++++++++++++++++++++++++++++++++++++++++++++++++++++++++++++++++++++++
\section{Résultats chinois}
%+++++++++++++++++++++++++++++++++++++++++++++++++++++++++++++++++++++++++++++++++++++++++++++++++++++++++++++++++++++++++++

Nous allons maintenant parler du système d'équations
\begin{subequations}
    \begin{numcases}{}
        x=a_1\mod p\\
        x=a_2\mod q
    \end{numcases}
\end{subequations}
avec \( a_1\), \( a_2\) donnés dans \( \eZ\) et \( p,q\) des entiers premiers entre eux. Le lemme chinois nous donne la liste des solutions ainsi qu'une manière de les construire. Le théorème chinois en sera une espèce de corollaire qui établira l'isomorphisme d'anneaux \( \eZ/pq\eZ\simeq \eZ/p\eZ\times \eZ/q\eZ\). Voir \href{http://www.les-mathematiques.net/b/a/d/node10.php}{les-mathematiques.net}.

\begin{lemma}[Lemme chinois \cite{CongrDuchSyl}]        \label{LemCtUeGA}
    Soient \( n_1,n_2\) deux entiers premiers entre eux. Soient \( a_1,a_2\in \eZ\). Les solutions du système
    \begin{subequations}        \label{SysVwvLKv}
        \begin{numcases}{}
            x=a_1\mod n_1\\
            x=a_2\mod n_2
        \end{numcases}
    \end{subequations}
    pour \( x\in \eZ/n_1n_2\eZ\) sont données de la façon suivante. Soient \( u_1,u_2\) deux entiers qui satisfont la relation de Bezout\footnote{voir le théorème \ref{ThoBuNjam}}
    \begin{equation}        \label{EqWcucUG}
        u_1n_1+u_2n_2=1,
    \end{equation}
    et 
    \begin{equation}        \label{EqHGchlQ}
        a=\big( a_1u_2n_2+a_2 u_1n_1 \big)\mod(n_1).
    \end{equation}
    Alors \( x=a\mod(n_1n_2)\).
\end{lemma}

\begin{proof}
    Vérifions que le \( x\) donné par \(x=a\mod(n_1n_2)\) est bien une solution. D'abord
    \begin{subequations}
        \begin{align}
            a\mod n_2&=a_1u_2n_2\mod n_1\\
            &=a_1(1-u_1n_1)\mod n_1\\
            &=a_1\mod n_1
        \end{align}
    \end{subequations}
    où nous avons utilisé l'identité de Bezout \eqref{EqWcucUG}. La vérification de \( a\mod n_2=a_2\mod n_2\) est la même.

    Soit maintenant \( x\in \eZ/n_1n_2\eZ\) une solution du système \eqref{SysVwvLKv} et \( a\) donné par la formule \eqref{EqHGchlQ}. Alors
    \begin{subequations}
        \begin{align}
            (x-a)\mod n_1&=\Big( a_1-(a_1n_2u_2+a_2u_1n_1) \Big)\mod n_1\\
            &=a_1-a_1u_2n_2\mod n_1\\
            &=0,
        \end{align}
    \end{subequations}
    donc \( (x-a)\mod n_1=0\), ce qui signifie que \( x-a\) est divisible par \( n_1\). De la même façon, \( (x-a)\mod n_2=0\) et \( x-a\) est divisible par \( n_2\). Nous savons maintenant que \( x-a\) est divisible par \( n_1\) et \( n_2\). Étant donné que \( n_1\) et \( n_2\) sont premiers entre eux, nous en déduisons que \( x-a\) est divisible par \( n_1n_2\), ou encore que \( x=a\mod n_1n_2\).
\end{proof}

\begin{theorem}[Théorème chinois]
    Soient \( p,q\) deux naturels premiers entre eux. Si \( p,q\geq 2\) alors l'application
    \begin{equation}
        \begin{aligned}
            \phi\colon \eZ/pq\eZ&\to \eZ/p\eZ\times \eZ/q\eZ \\
            x\mod(pq)&\mapsto \big( x\mod p,x\mod q \big) 
        \end{aligned}
    \end{equation}
    est un isomorphisme d'anneaux.
\end{theorem}

\begin{proof}
    Afin d'alléger les notations, nous allons noter \( [x]_p\) au lieu de \( x\mod p\). Nous devons prouver que l'application \( \phi\) respecte la somme, le produit et qu'elle est bijective. En ce qui concerne la somme,
    \begin{subequations}
        \begin{align}
            \phi([q]_{pq}+[y]_{pq})&=            \phi\big( (x+y)\mod pq \big)\\
            &=\big( [x+y]_{p},[x+y]_q \big)\\
            &=\big( [x]_p+[y]_p,[x]_q+[y]_q \big)\\
            &=\big( [x]_p,[x]_q \big)+\big( [y]_p,[y]_q \big)\\
            &=\phi(x)+\phi(y).
        \end{align}
    \end{subequations}
    En ce qui concerne le produit, c'est le même jeu : nous obtenons
    \begin{equation}
        \phi\big( [xy]_{pq} \big)=\phi([x]_{pq}])\phi([y]_{pq})
    \end{equation}
    en utilisant le fait que \( [xy]_{p}=[x]_p[y]_p\).

    Montrons maintenant que \( \phi\) est surjective. Soient \( y_1,y_2\in \eZ\) et \( x\in \eZ\). Demander
    \begin{equation}
        \phi([x]_{pq})=\big( [y_1]_p,[y_2]_q \big)
    \end{equation}
    revient à demander que \( [x]_p=[y_1]_p\) et \( [x]_q=[y_2]_q\), c'est à dire que \( x\) résolve le système
    \begin{subequations}
        \begin{numcases}{}
            x=y_1\mod p\\
            x=y_2\mod q.
        \end{numcases}
    \end{subequations}
    Le lemme chinois \ref{LemCtUeGA} nous assure qu'une solution existe.

    En ce qui concerne l'injectivité, nous supposons que \( x\) et \( y\) soient deux entiers tels que
    \begin{equation}
        \phi([x]_{pq})=\phi([y]_{pq}).
    \end{equation}
    Nous en déduisons le système
    \begin{subequations}
        \begin{numcases}{}
            x\mod p=y\mod p\\
            x\mod q=y\mod q
        \end{numcases}
    \end{subequations}
    c'est à dire qu'il existe des entiers \( k\) et \( l\) tels que \( x=y+kp\) et \( x=y+lq\) ou encore tels que
    \begin{equation}
        kp+lq=0.
    \end{equation}
    Étant donné que \( p\) et \( q\) sont premiers entre eux, la seule possibilité est \( k=l=0\), c'est à dire \( x=y\).
\end{proof}

Le résultat suivant est une généralisation au cas des anneaux commutatifs.
\begin{theorem}[Théorème chinois]\index{théorème!chinois}
    Soit \( A\) un anneau commutatif, \( n\geq 2\), des éléments \( x_1,\ldots,x_n\) dans \( A\) et des idéaux \( I_1,\ldots,I_n\) tels que \( I_i+I_j=A\) pour tout \( i\neq j\).

    Alors il existe un \( x\in A\) tel que \( x-x_i\in I_i\) pour tout \( 1\leq i\leq n\).
\end{theorem}

\begin{proof}
    Pour \( i\in\{ 1,\ldots,n \}\) nous notons \( J_i\) le produit \( J_i=\prod_{k\neq i}I_k\). Étant donné que chaque \( I_i\) est un idéal, nous avons \( I_k\in J_i\) lorsque \( i\neq k\).

    Soit \( i\) fixé, et considérons \( j\neq i\). Nous pouvons trouver \( a_j\in I_i\) et \( b_j\in I_j\) tel que \( a_j+b_j=1\). Nous avons alors
    \begin{equation}
        1=\prod_{j\neq i}(a_j+b_j).
    \end{equation}
    Par ailleurs \( I_i+J_i=A\) parce que \( J_i\) contient \( I_k\) avec \( k\neq i\) et \( I_i+I_k=A\). Nous pouvons donc prendre \( \alpha_i\in I_i\) et \( \beta_i\in J_i\) tels que
    \begin{equation}
        \prod_{j\neq i}(a_j+b_j)=\alpha_i+\beta_i.
    \end{equation}
    Nous considérons alors l'élément \( x=\beta_1x_1+\ldots+\beta_nx_n\) et nous avons
    \begin{subequations}
        \begin{align}
            x-x_1&=(\beta_1-1)x_1+\beta_2x_2+\ldots+\beta_nx_n\\
            &=-\alpha_1x_1+\beta_2x_2+\ldots+\beta_nx_n
        \end{align}
    \end{subequations}
    Mais \( \alpha_1\in I_1\) et tous les autres termes sont dans les \( J_i\) avec \( i\neq 1\). Par conséquent le tout est dans \( I_1\). Ici nous utilisons par exemple le fait que \( \beta_2\in J_2\subset I_1\) parce que les éléments de \( J_2\) sont des produits d'éléments dont un facteur est dans \( I_1\).
\end{proof}

\begin{remark}
    Ce théorème chinois est bien une généralisation du lemme chinois \ref{LemCtUeGA}. En effet, l'élément \( x\) dont il est question est solution du problème \( x=x_i\mod I_i\). L'hypothèse \( I_i+I_j=A\) n'est pas nouvelle non plus étant donné que si \( p\) et \( q\) sont des entiers premiers entre eux nous avons \( p\eZ+q\eZ=\eZ\) par le corollaire \ref{CorgEMtLj}.
\end{remark}

%---------------------------------------------------------------------------------------------------------------------------
\subsection{Codage RSA}
%---------------------------------------------------------------------------------------------------------------------------
\label{SecEVaFYi}

\begin{probleme}
    Je ne suis pas très sûr de ce que j'ai écrit ici. Il vaut mieux lire \wikipedia{fr}{Rivest_Shamir_Adleman}{la page Wikipédia}. Merci de me contacter si vous trouvez des fautes.
\end{probleme}

Alice veut envoyer un message à Bob. Pour ce faire, Bob choisit deux nombres premiers \( p_1,p_2\), calcule \( n=p_1p_2\) et choisit un nombre \( e\in \eF_n\), c'est à dire \( e\) premier avec \( n\). Bob publie le couple \( (n,e)\).

Alice convertit son message en un élément \( X\in \eF_n\), et envoie \( Y=X^e\). Notons qu'Alice a utilisé \( n\) et \( e\). Pour décoder, Bob doit simplement calculer \( d=e^{-1}\) dans \( \eF_n\) et faire l'opération
\begin{equation}
    Y^d=(X^e)^d=X.
\end{equation}
Deux questions : d'abord comment Bob calcule \( d\) ? Et ensuite comment une tierce personne connaissance \( n\) et \( e\) pourrait trouver \( d\) ?

Connaissant \( p_1\) et \( p_2\), Bob n'a pas trop de difficultés. Il doit d'abord trouver les inverses de \( e\) modulo \( p_1\) et \( p_2\) :
\begin{subequations}
    \begin{numcases}{}
        d_1e=1\mod p_1\\
        d_2e=1\mod p_2.
    \end{numcases}
\end{subequations}
Pour ce faire, il doit résoudre les relations de Bezout
\begin{equation}
    d_1e-kp_1=1,
\end{equation}
(et idem avec \( p_2\)), ce qui est possible parce que \( e\) et \( p_1\) sont premiers entre eux. Ensuite Bob trouve \( d\) sous la forme
\begin{equation}
    d=d_1+kp_1=d_2+lp_2,
\end{equation}
ce qui revient encore à résoudre une relation de Bezout :
\begin{equation}
    d_1-d_2=lp_2-kp_1.
\end{equation}




%+++++++++++++++++++++++++++++++++++++++++++++++++++++++++++++++++++++++++++++++++++++++++++++++++++++++++++++++++++++++++++
\section{Modules}
%+++++++++++++++++++++++++++++++++++++++++++++++++++++++++++++++++++++++++++++++++++++++++++++++++++++++++++++++++++++++++++

Soit \( \modE\) un \( A\)-module et \( x=(x_i)_{i\in I}\) une famille d'éléments de \( \modE\), paramétrée par l'ensemble \( I\). Nous considérons l'application
\begin{equation}
    \begin{aligned}
        \mu_x\colon A^{(I)}&\to \modE \\
        (a_i)_{i\in I}&\mapsto \sum_{i\in I}a_ix_i.
    \end{aligned}
\end{equation}
Ici \( A^{(I)}\) désigne l'ensemble de toutes les applications \( I\to A\) de support fini.  

\begin{definition}      \label{DefBasePouyKj}
    À l'instar des espaces vectoriels, les modules ont une notion de partie libre, génératrice et de bases :
    \begin{enumerate}
        \item
            Si \( \mu_x\) est surjective, nous disons que \( x\) est une partie \defe{génératrice}{génératrice!partir d'un module}.
        \item
            Si \( \mu_x\) est injective, nous disons que la partie \( x\) est \defe{libre}{libre!partie d'un module}.
        \item
            Si \( \mu_x\) est bijective, nous disons que la partie \( x\) est une \defe{base}{base!d'un module}.
    \end{enumerate}
\end{definition}

\begin{definition}
    Soit \( \modE\) un module sur un anneau commutatif \( A\). Un \defe{projecteur}{projecteur!dans un module} est une application linéaire \( p\colon \modE\to \modE\) telle que \( p^2=p\).

    Une famille \( (p_i)_{i\in I}\) sur \( \modE\) est \defe{orthogonale}{orthogonal!famille de projecteurs} si \( p_i\circ p_j=0\) pour tout \( i\neq j\). La famille est \defe{complète}{complète!famille de projecteurs} si \( \sum_{i\in I}p_i=\mtu\).
\end{definition}

\begin{theorem}     \label{ThoProjModpAlsUR}
    Soient des sous modules \( \modE_1,\ldots,\modE_n\) du module \( \modE\) tels que \( \modE=\modE_1\oplus\ldots\oplus\modE_n\). Les applications \( p_i\) définies par
    \begin{equation}
        p_i(x_1+x_n)=x_i
    \end{equation}
    forment une famille orthogonale de projecteurs et \( p_1+\ldots +p_n=\id\).

    Inversement, si \( (p_1,\ldots, p_n)\) est une famille orthogonale de projecteurs dans un module \( \modE\) tel que \( \sum_{i=1}^np_i=\id\), alors
    \begin{equation}
        \modE=\bigoplus_{i=1}^np_i(\modE).
    \end{equation}
\end{theorem}

%+++++++++++++++++++++++++++++++++++++++++++++++++++++++++++++++++++++++++++++++++++++++++++++++++++++++++++++++++++++++++++
\section{Polynômes}
%+++++++++++++++++++++++++++++++++++++++++++++++++++++++++++++++++++++++++++++++++++++++++++++++++++++++++++++++++++++++++++

Soit \( A\) un anneau commutatif. Nous considérons \( \polyP\) l'ensemble des suites presque nulles d'éléments de \( A\), ce sont les suites \( (a_n)_{n\in\eN}\) telles que il existe \( N\) tel que \( a_i=0\) pour tout \( i>N\).

Cela est un \( A\)-module libre de base (définition \ref{DefBasePouyKj})
\begin{equation}
    (e_n)_k=\delta_{nk}.
\end{equation}
Si \( (a_n)_{n\in \eN}\) et \( (b_n)_{n\in\eN}\) sont des éléments de \( \polyP\), nous définissons le produit \( ab\) par
\begin{equation}
    (ab)_n=\sum_{p+q=n}a_pb_q.
\end{equation}
Cela est bien un élément de \( \polyP\) parce qu'il existe \( N\in\eN\) tel que \( a_n=b_n=0\) pour tout \( n\geq N\). Avec la somme et le produit par un scalaire, le module \( \polyP\) devient une \( A\)-algèbre commutative unitaire. L'unité est 
\begin{equation}
    e_0=(1,0,\ldots).
\end{equation}

\begin{definition}
    En tant que \( A\)-algèbre, l'ensemble \( \polyP\) est l'\defe{algèbre des polynômes en une indéterminée}{algèbre!polynômes} à coefficients dans \( A\).
\end{definition}

Si nous posons que \( X=e_1\), et que nous prenons la convention \( X^0=1\), alors nous avons \( e_k=X^k\) et nous notons \( A[X]\) l'anneau \( \polyP\) exprimé avec \( X\). Les éléments de la forme \( \lambda X^k\) avec \( \lambda\in A\) et \( k\in\eN\) sont des \defe{monômes}{monôme}. Nous allons aussi considérer
\begin{equation}\nomenclature[A]{\( A_n[X]\)}{les polynômes à coefficients dans \( A\) et de degré inférieur à \( n\)}
    A_n[X]=\{ P\in A[X]\tq \deg(P)\leq n \}.
\end{equation}
Cela est un sous module libre.

\begin{theorem}
    L'anneau \( A\) est intègre si et seulement si \( A[X]\) est intègre.
\end{theorem}

\begin{proof}
    Soient \( P\) et \( Q\) des éléments non nuls de \( A[X]\). Vu que l'anneau \( A\) est intègre, nous avons
    \begin{equation}
        \deg(PQ)=\deg(P)+\deg(Q)
    \end{equation}
    et le produit ne peut pas être nul. L'anneau \( A[X]\) est donc intègre.

    Si \( A[X]\) est intègre, \( A\) est intègre parce qu'il peut être vu comme sous anneau.
\end{proof}

\begin{remark}
    Si \( A\) n'est pas intègre, soit \( \alpha\beta=0\), alors \( (\alpha X)(\beta x)=0\) et le degré du produit n'est pas la somme des degrés.
\end{remark}

\begin{corollary}
    Si \( A\) est intègre, les inversibles de \( A[X]\) sont les éléments de \( U(A)\).
\end{corollary}

\begin{proof}
    Pour que \( Q\) soit inversible, il faut un \( P\) tel que \( PQ=1\). Mais l'anneau \( A\) étant intègre, les degrés s'additionnent. Par conséquent ils doivent être de degrés zéro et il faut que \( P,Q\in A\). Enfin pour qu'ils soient inversibles, ils doivent être dans \( U(A)\).
\end{proof}

La \defe{valuation}{valuation} de \( P\) du polynôme \( P=\sum_n a_nX^n\), notée \( \val(P)\), est 
\begin{equation}
    \val(P)=\min\{ n\tq a_n\neq 0 \}.
\end{equation}
Nous avons \( \val(P)\leq \deg(P)\) et \( \val(P)=\deg(P)\) si et seulement si \( P\) est un monôme. Si \( P=0\), nous convenons que \( \val(0)=\infty\) et \( \deg(0)=-\infty\).

%---------------------------------------------------------------------------------------------------------------------------
\subsection{Irréductibilité}
%---------------------------------------------------------------------------------------------------------------------------

\begin{theorem}[d'Alembert-Gauss]\index{théorème!d'Alembert-Gauss}      \label{ThovgyUuA}
    Tout polynôme non constant à coefficients complexes possède au moins une racine complexe.
\end{theorem}


\begin{definition}      \label{DefIrredfIqydS}
    Un polynôme est \defe{irréductible}{irréductible!polynôme} lorsqu'il ne peut pas être écrit sous la forme de produits de polynômes de degré supérieurs à \( 1\).
\end{definition}

\begin{proposition}
    Un polynôme irréductible à coefficients réels est soit de degré un soit de degré \( 2\) avec un discriminant négatif.
\end{proposition}

\begin{proof}
    Soit un polynôme \( P\) à coefficients réels de degré plus grand que \( 1\). Alors le théorème de d'Alembert-Gauss (théorème \ref{ThovgyUuA}) implique l'existence d'une racine \( \alpha\). Il est facile de montrer que le conjugué complexe \( \bar \alpha\) est également racine. Par conséquent les polynômes \( (X-\alpha)\) et \( (X-\bar \alpha)\) divisent \( P\).

    Ces deux polynômes sont premiers entre eux parce que
    \begin{equation}
        a(X-\alpha)+b(X-\bar \alpha)=0
    \end{equation}
    implique \( a=b=0\). Par conséquent le produit 
    \begin{equation}
        X^2-(\alpha+\bar \alpha)X+\alpha\bar\alpha
    \end{equation}
    divise également \( P\). Ce dernier est un polynôme à coefficients réels de degré \( 2\). Donc tout polynôme de degré \( 3\) ou plus est réductible.
\end{proof}

Nous disons que \( P\in\eK[X]\setminus\eK\) est \defe{scindé}{polynôme!scindé} sur \(\eK\) si il est produit dans \(\eK[X]\) de polynômes de degré \( 1\).

%---------------------------------------------------------------------------------------------------------------------------
\subsection{Division euclidienne}
%---------------------------------------------------------------------------------------------------------------------------

Le théorème suivant établit la \defe{division euclidienne}{division!euclidienne} dans \( \eA[X]\) du polynôme \( A\) par \( B\).
\begin{theorem}     \label{ThodivEuclPsFexf}
    Soit \( B\neq 0\) dans \( \eA[X]\) de coefficient dominant inversible dans \( \eA\). Pour tout \( A\in\eA[X]\), il existe \( Q,R\in \eA[X]\) tels que
    \begin{equation}
        A=BQ+R
    \end{equation}
    avec \( \deg(R)<\deg(B)\).

    Les polynômes \( Q\) et \( R\) sont déterminés de façon univoque par cette condition. Le polynôme \( Q\) est le \defe{quotient}{quotient} et \( R\) est le \defe{reste}{reste} de la division euclidienne de \( A\) par \( B\).
\end{theorem}

Deux polynômes \( P\) et \( Q\) sont dits \defe{étrangers}{étrangers!polynômes} entre eux si \( 1\) est un \( \pgcd\) de \( P\) et \( Q\). Un ensemble de polynômes \( (P_i)_{i\in I}\) est étranger \defe{dans leur ensemble}{étranger!dans leur ensemble} si \( 1\) est un \( \pgcd\) des \( P_i\).

\begin{theorem}[Bezout] \label{ThoBezoutOuGmLB}
    Les polynômes \( P_1,\ldots,P_n\) dans \( \eK[X]\) sont étrangers entre eux si et seulement si il existe des polynômes \( Q_1,\ldots,Q_n\in\eK[X]\) tels que
    \begin{equation}
        P_1Q_1+\ldots+P_nQ_n=1.
    \end{equation}
\end{theorem}

\begin{lemma}       \label{LemuALZHn}
    Soient \( (P_i)_{i=1,\ldots,n}\in \eK[X]\) des polynômes étrangers deux à deux. Alors les polynômes \begin{equation} Q_i=\prod_{j\neq i}P_j \end{equation}
    sont étrangers entre eux\footnote{Et non juste deux à deux.}.
\end{lemma}

\begin{lemma}   \label{LemzwkYdn}
    Soit \( \eK\) un corps commutatif et \( \eA\subset \eK\) un sous anneau de \( \eK\). Soit \( \phi\in \eK[X]\). Si il existe \( Q\in \eA[X]\) unitaire tel que \( \phi Q\in \eA[X]\), alors \( \phi\in \eA[X]\).
\end{lemma}
Une preuve peut être trouvée dans la page des lemmes pour le théorème de Wedderburn sur \href{http://www.les-mathematiques.net/d/a/w/node5.php}{les-mathematiques.net}.

%---------------------------------------------------------------------------------------------------------------------------
\subsection{Idéaux}
%---------------------------------------------------------------------------------------------------------------------------

Soit \( P\in \eK[X]\) un polynôme. Nous notons \( (P)\) l'idéal engendré par \( P\) :
\begin{equation}        \label{EqDefxMkDtW}
    (P)=\{ PR\tq R\in\eK[X] \}.
\end{equation}

\begin{lemma}
    Nous avons
    \begin{enumerate}
        \item
            \( (P)\subset (Q)\) si et seulement si \( Q\) divise \( P\),
        \item
            \( (P)=(Q)\) si et seulement si \( P\) et \( Q\) sont multiples (non nuls) l'un de l'autre.
    \end{enumerate}
\end{lemma}

\begin{proof}
    Si \( (P)\subset (Q)\), en particulier \( P\in(Q)\) et il existe \( R\in\eK[X]\) tel que \( P=QR\), ce qui signifie que \( Q\) divise \( P\).

    Si les idéaux de \( P\) et de \( Q\) sont identiques, l'un divise l'autre et l'autre divise l'un. Ils sont donc multiples l'un de l'autre.
\end{proof}

\begin{theorem}     \label{ThoCCHkoU}
    Soit \( I\) un idéal dans \( \eK[X]\). Alors il existe un polynôme \( P\) tel que \( I=(P)\). Plus précisément, si \( P\) est de degré minimal, alors \( (P)=I\).

    De plus si \( I\neq \{  0\}\), il existe un unique polynôme unitaire \( U\) tel que \( I=(U)\).
\end{theorem}

\begin{proof}
    Si \( I=\{ 0 \}\), le résultat est évident. Nous supposons donc \( I\) non nul. Soit \( P\) de degré minimum parmi les éléments de \( I\). Évidemment \( (P)\subset I\). Nous allons démontrer qu'en réalité \( (P)=I\).

    Soit \( A\in I\). Par le théorème \ref{ThodivEuclPsFexf} de la division euclidienne\footnote{Ici \( \eK\) est un corps et donc l'hypothèse d'inversibilité est automatiquement vérifiée.}, il existe \( Q\) et \( R\) dans \( \eK[X]\) tels que \( A=PQ+R\) avec \( \deg(R)<\deg(P)\). Étant donné que \( R=A-PQ\) nous avons \( R\in I\) et par conséquent \( R=0\) parce que \( P\) a été choisit de degré minimum dans \( I\). Nous avons donc \( A=PQ\) et \( I\subset (P)\).

    L'existence d'un polynôme unitaire qui génère \( I\) est obtenu en choisissant \( U=P/a_n\) où \( a_n\) est le coefficient du terme de plus haut degré.
\end{proof}
Nous voyons que n'importe quel polynôme de degré minimum dans un idéal génère l'idéal. Une importante conséquence du théorème \ref{ThoCCHkoU} que nous verrons plus bas est que tout polynôme annulateur d'un endomorphisme est divisé par le polynôme minimal (proposition \ref{PropAnnncEcCxj}).

\begin{corollary}
    Soit \( P\in \eK[X]\) et \( a\in \eK\), une racine de \( P\). Alors le polynôme minimal de \( a\) dans \( \eK[X]\) divise \( P\). En d'autre termes, le polynôme minimal d'un élément divise tout polynôme annulateur.
\end{corollary}

\begin{proof}
    Nous considérons l'idéal
    \begin{equation}
        I=\{ Q\in \eK[X]\tq Q(a)=0 \}.
    \end{equation}
    Le fait que cela soit un idéal est simplement dû à la définition du produit : \( (PQ)(a)=P(a)Q(a)\). Par le théorème \ref{ThoCCHkoU}, le polynôme minimal \( \mu_a\) de \( a\) est dans \( I\) et qui plus est le génère : \( I=(\mu_a)\). Par conséquent tout polynôme annulateur de \( a\) est divisé par \( \mu_a\).
\end{proof}

\begin{corollary}       \label{CorvlvOWA}
    Soit \( \eK\) un corps commutatif. Alors l'anneau des polynômes \( \eK[X]\) est principal.
\end{corollary}

\begin{proof}
    En effet \( \eK[X]\) est un anneau commutatif intègre (pas de diviseurs de zéro), et d'après le théorème \ref{ThoCCHkoU}, tous les idéaux sont principaux en tant que pour tout idéal \( I\), il existe \( P\in I\) tel que \( I=\eK[X]P=(P)\).
\end{proof}

\begin{definition}  \label{Decyyumy}
    Soit \( E\), un espace vectoriel et \( f\colon E\to E\) un endomorphisme de \( E\). Pour chaque \( x\in E\) nous considérons l'idéal
    \begin{equation}
        I_{f,x}=\{ P\in \eK[X]\tq P(f)x=0 \}.
    \end{equation}
    C'est l'ensemble des polynômes qui annulent \( f\) en \( x\). Le générateur unitaire de \( I_{f,x}\) est le \defe{polynôme minimal ponctuel}{polynôme!minimal!ponctuel}\index{polynôme!minimal!relativement à un point} de \( f\) en \( x\). Il sera noté \( \mu_{f,x}\).
\end{definition}
Ces définitions sont légitimées par les faits suivants. L'idéal \( I_{f,x}\) n'est pas réduit à \( \{ 0 \}\) parce que le polynôme minimal de \( f\) fait partie de \( I_{f,x}\). C'est le théorème \ref{ThoCCHkoU} qui nous assure l'existence d'un unique générateur unitaire dans~\( I_{f,x}\). 

\begin{lemma}\label{LemSYsJJj}
    Soit \( f\colon E\to E\) un endomorphisme de l'espace vectoriel \( E\). Il existe un élément \( x\in E\) tel que \( \mu_{f,x}=\mu_f\).
\end{lemma}

\begin{proof}
    Nous savons que pour tout \( x\in E\), \( \mu_f\in I_{f,x}\), donc le polynôme \( \mu_{f,x}\) divise \( \mu_f\) pour tous les \( x\). Nous en déduisons que l'ensemble
    \begin{equation}
        \{ \mu_{f,x}\tq x\in E \}
    \end{equation}
    est en réalité un ensemble fini, sinon \( \mu_f\) ne serait pas un polynôme. Soient donc les points \( x_1,\ldots, x_l\) tels que
    \begin{equation}
        \{ \mu_{f,x}\tq x\in E \}=\{ \mu_{f,x_1},\ldots, \mu_{f,x_l} \}.
    \end{equation}
    Étant donné que \( x\in \ker\mu_{f,x}\) nous avons \( \mu_{f,x}\in I_{f,x}\) et donc \( \mu_{f,x}(f)x=0\). Par conséquent
    \begin{equation}
        E=\bigcup_{1\leq i\leq l}\ker\mu_{f,x_i(f)}.
    \end{equation}
    En vertu de la proposition \ref{PropTVKbxU}, un des termes de l'union doit être l'espace \( E\) entier. Il existe donc un \( x_i\) tel que
    \begin{equation}
        E=\ker\big( \mu_{f,x_i}(f) \big).
    \end{equation}
    Le polynôme \( \mu_{f,x_i}\) annule \( f\) et est donc divisé par le polynôme minimal de \( f\). Nous avons donc montré que \( \mu_{f,_{x_i}}\) divise et est divisé par \( \mu_f\). Par conséquent \( \mu_f=\mu_{f,x_i}\).
\end{proof}

\Exo{reserve0004}

%---------------------------------------------------------------------------------------------------------------------------
\subsection{Racines de polynômes}
%---------------------------------------------------------------------------------------------------------------------------

Soit \( \eA\) un anneau et \( P\in \eA[X]\) un polynôme et \( \alpha\in \eA\). Le \defe{degré}{degré!d'une racine} ou la \defe{multiplicité}{multiplicité!d'une racine} de \( \alpha\) par rapport à \( P\) est l'entier \( h\) tel que \( P\) est divisible par \( (X-\alpha)^h\) mais pas divisible par \( (X-\alpha)^{h+1}\).

Nous noterons \( \theta_{\alpha}(P)\)\nomenclature[A]{\( \theta_{\alpha}(P)\)}{l'ordre de \( \alpha\) par rapport à \( P\)} l'ordre de \( \alpha\) par rapport à \( P\).

\begin{proposition}     \label{PropahQQpA}
    L'élément \( \alpha\in \eA\) est d'ordre \( h\) par rapport à \( \) si et seulement si il existe \( Q\in\eA[X]\) tel que \( P(X-\alpha)^hQ\) avec \( Q(\alpha)\neq 0\).
\end{proposition}

\begin{lemma}       \label{LemIeLhpc}
    Soient \( P\) et \( Q\) des polynômes non nuls de \( \eA[X]\) et \( \alpha\in \eA\) d'ordre \( p\) pour \( P\) et d'ordre \( q\) pour \( Q\). Alors
    \begin{enumerate}
        \item
            \( \theta_{\alpha}(P+Q)\geq\ln\{ \theta_{\alpha}(P),\theta_{\alpha}(Q) \}\)
        \item
            si \( \theta_{\alpha}(P)\neq \theta_{\alpha}(Q)\), alors \( \theta_{\alpha}(P+Q)=\min\{ \theta_{\alpha}(P),\theta_{\alpha}(Q) \}\)
        \item
            \( \theta_{\alpha}(PQ)\geq \theta_{\alpha(P)}+\theta_{\alpha}(Q)\);
        \item       \label{ItemIeLhpciv}
            si \(\eA \) est intègre alors \( \theta_{\alpha}(PQ)= \theta_{\alpha}(P)+\theta_{\alpha}(Q)\);
    \end{enumerate}
\end{lemma}

\begin{theorem}
    Soit \( \eA\) un anneau intègre et \( P\in \eA[X]\setminus\{ 0 \}\), un polynôme de degré \( n\). Si \( \alpha_1,\ldots, \alpha_p\in\eA\) sont des racines deux à deux distinctes d'ordres \( k_1,\ldots, k_p\), alors il existe \( Q\in \eA[X]\) tel que
    \begin{enumerate}
        \item
            \( Q(\alpha_i)\neq 0\);
        \item
            \( P=Q\prod_{i=1}^p(X-\alpha_i)\);
    \end{enumerate}
    De plus la sommes des ordres des racines de \( P\) est au plus \( \deg(P)\).
\end{theorem}

\begin{proof}
    Si \( p=1\), alors le résultat est la proposition \ref{PropahQQpA}. Nous supposons que \( p\geq 2\) et nous effectuons une récurrence sur \( P\). Nous considérons donc pas \( p-1\) premières racines \( \alpha_1,\ldots, \alpha_{p-1}\) et un polynôme \( R\in\eA[X]\) tel que \( R(\alpha_i)\neq 0\) pour \( i=1,\ldots, p-1\) et
    \begin{equation}
        P=\underbrace{(X-\alpha_1)^{k_1}\ldots (X-\alpha_{p-1})^{k_{p-1}}}_SR.
    \end{equation}
    Par hypothèse \( P(\alpha_p)=S(\alpha_p)R(\alpha_p)=0\). L'anneau \( \eA\) étant intègre, \( S(\alpha_p)\neq 0\) parce que \( \alpha_i\neq \alpha_p\) pour \( i\neq p\). Par conséquent, \( R(\alpha_p)=0\).
    
    Nous devons encore vérifier que l'ordre de \( \alpha_p\) est \( k_p\) par rapport à \( R\). Pour cela nous utilisons le point \ref{ItemIeLhpciv} du lemme \ref{LemIeLhpc} affin de dire que le degré de \( \alpha_p\) pour \( P=SR\) est \( k_p\). Par conséquent
    \begin{equation}
        R=(X-\alpha_p)^{k_p}T
    \end{equation}
    avec \( T(\alpha_p)\neq 0\) et enfin
    \begin{equation}
        P=\prod_{i=1}^p(X-\alpha_i)T.
    \end{equation}
    De plus \( T(\alpha_i)\neq 0\), sinon \( R(\alpha_i)\) serait nul.
\end{proof}

%+++++++++++++++++++++++++++++++++++++++++++++++++++++++++++++++++++++++++++++++++++++++++++++++++++++++++++++++++++++++++++
\section{Anneaux principaux}
%+++++++++++++++++++++++++++++++++++++++++++++++++++++++++++++++++++++++++++++++++++++++++++++++++++++++++++++++++++++++++++

Nous avons parlé de l'idéal des polynôme annulateurs dans le théorème \ref{ThoCCHkoU}.

\begin{definition}      \label{DefIdPrinpuMrbOq}
    Un idéal \( I\) dans \(\eA\) est \defe{principal à gauche}{idéal!principal!à gauche} si il existe \( a\in I\) tel que \( I=\eA a\). Il est \defe{principal à droite}{idéal!principal!à droite} si il existe \( a\in I\) tel que \( I=a\eA\). Nous disons qu'il est \defe{principal}{principal!idéal} si il est principal à gauche et à droite.

    Un anneau commutatif intègre est \defe{principal}{principal!anneau} si tous ses idéaux sont principaux.
\end{definition}

Un idéal \( I\) dans l'anneau \( \eA\) est \defe{maximal}{maximal!idéal}\index{idéal!maximal} si les seuls idéaux de \( \eA\) contenants \( I\) sont \( I\) et \( \eA\).

\begin{definition}  \label{DeirredBDhQfA}
    Soit \( \eA\) un anneau commutatif intègre. Un élément \( a\in\eA\) est \defe{irréductible}{irréductible!dans un anneau} si \( a\) n'est pas inversible, mais si \( a=xy\), alors soit \( x\) soit \( y\) est inversible. Nous notons \( U(\eA)\) l'ensemble des éléments inversibles de \( \eA\).
\end{definition}

\begin{remark}
    Un corps n'a pas d'éléments irréductibles parce qu'à part zéro tous les éléments sont inversibles alors que \( 0\) n'est certainement pas irréductible vu que \( 0=0\cdot 0\).
\end{remark}


\begin{definition}
    Nous disons qu'un idéal \( I\) dans \( \eA\) est \defe{premier}{premier!idéal} si \( \eA\) est un anneau commutatif intègre et si \( A/I\) est intègre.
\end{definition}

\begin{proposition} \label{PropomqcGe}
    Soit \( \eA\) un anneau principal qui n'est pas un corps. Pour un idéal \( I\subset \eA\), les conditions suivantes sont équivalentes :
    \begin{enumerate}
        \item
            \( I\) est un idéal maximum;
        \item
            \( I\) est un idéal premier non nul;
        \item
            il existe \( p\) irréductible dans \( \eA\) tel que \( I=(p)\).
    \end{enumerate}
\end{proposition}
Ici, \( (p)\) est l'idéal dans \( \eA\) engendré par \( p\), c'est à dire \( p\eA\)\nomenclature[A]{\( (p)\)}{idéal engendré par \( p\)}.

\begin{proposition}
    Un idéal \( I\) dans \( \eA\) est premier si et seulement si \( I\) est strictement inclus dans \( \eA\) et si pour tout \( a,b\in\eA\) tels que \( ab\in I\) nous avons \( a\in I\) ou \( b\in I\).
\end{proposition}

\begin{proposition}     \label{PropoTMMXCx}
    Si \( \eA\) est un anneau principal et si \( p\) est irréductible, alors \( \eA/p\) est un corps.
\end{proposition}

\begin{proposition}     \label{PropqGZXvr}
    L'anneau \( \eK[X]\) des polynômes sur un corps commutatif \( \eK\) est factoriel.
\end{proposition}

%---------------------------------------------------------------------------------------------------------------------------
\subsection{Anneaux principaux et polynômes}
%---------------------------------------------------------------------------------------------------------------------------

Nous supposons que \( \eK\) est une corps commutatif, et nous étudions l'anneau \( \eK[X]\). Étant donné que \( \eK\) est commutatif pour tout polynôme que l'idéal engendré par \( P\) est \( (P)=\eK[X]P\), voir la notation \eqref{EqDefxMkDtW}.

\begin{remark}
    Un polynôme est irréductible dans \( \eK[X]\) au sens de la définition \ref{DeirredBDhQfA} si et seulement si il est irréductible au sens de la définition \ref{DefIrredfIqydS} parce que seules les constantes (non nulles) sont inversibles dans \( \eK[X]\).
\end{remark}

\begin{corollary}       \label{CorsLGiEN}
    Si \( \eK\) est un corps et si \( P\) est un polynôme irréductible de degré \( n\), alors l'ensemble \( \eL=\eK[X]/(P)\) est un corps. De plus \( \eL\) est un espace vectoriel de dimension \( n\).
\end{corollary}

\begin{proof}
    En effet \( \eK[X]\) est un anneau principal par le corollaire \ref{CorvlvOWA}, par conséquent la proposition \ref{PropoTMMXCx} déduit que \( \eK[X]/(P)\) est un corps.

    Une base de \( \eL\) est donnée par les projections de \( 1,X,X^2,\ldots, X^n\).
\end{proof}

%---------------------------------------------------------------------------------------------------------------------------
\subsection{Anneaux euclidiens}
%---------------------------------------------------------------------------------------------------------------------------

\begin{definition}[\wikipedia{fr}{Anneau_euclidien}{Wikipédia}]
    Soit \( \eA\) un anneau intègre. Un \defe{stathme euclidien}{stathme euclidien} sur \( \eA\) est une application \( \alpha\colon \eA\setminus\{ 0 \}\to \eN\) tel que
    \begin{enumerate}
        \item
            \( \forall a,b\in \eA\setminus\{ 0 \}\), il existe \( q,r\in \eA\) tel que
            \begin{equation}
                a=bq+r
            \end{equation}
            et \( \alpha(r)<\alpha(b)\).
        \item
            Pour tout \( a,b\in \eA\setminus\{ 0 \}\), \( \alpha(b)\leq \alpha(ab)\).
    \end{enumerate}
    Un anneau est \defe{euclidien}{euclidien!anneau} si il accepte un stathme euclidien.
\end{definition}
Le stathme est la fonction qui donne le «degré» à utiliser dans la division euclidienne. La contrainte est que le degré du reste soit plus petit que le degré du dividende.

\begin{example} \label{ExwqlCwvV}
    Le stathme de \( \eN\) pour la division euclidienne usuelle est \( \alpha(n)=n\). Si \( a,b\in \eN\) nous écrivons
    \begin{equation}
        a=bq+r
    \end{equation}
    où \( q\) est l'entier le plus proche \emph{inférieur} à \( a/b\) (on veut que le reste soit positif) et \( r=a-bq\). Nous avons donc
    \begin{equation}
        r-b=a-b(q+1)<a-b\frac{ a }{ b }=0,
    \end{equation}
    ce qui montre que \( r<b\).
\end{example}

\begin{proposition}[\wikipedia{fr}{Anneau_euclidien}{Wikipédia}]\label{Propkllxnv}
    Un anneau euclidien est principal.
\end{proposition}

\begin{proof}
    Soit \( \eA\) un anneau principal et \( \alpha\) un stathme sur \( \eA\). Nous considérons un idéal \( I\) non nul de \( \eA\). Nous devons montrer que \( I\) est généré par un élément. En l'occurrence nous allons montrer que l'élément \( a\in I\setminus\{ 0 \}\) qui minimise \( \alpha(a)\) va générer. Soit \( x\in I\). Par construction, il existe \( q,r\in \eA\) tels que \( a=aq+r\) avec \( r=0\) ou \( \alpha(r)<\alpha(a)\). Étant donné que \( x,a\in I\), \( r\in I\). Si \( r\neq 0\), alors \( r\) contredirait la minimalité de \( \alpha(a)\). Donc \( r=0\) et \( x=aq\), ce qui signifie que \( I\) est principal.
\end{proof}


%+++++++++++++++++++++++++++++++++++++++++++++++++++++++++++++++++++++++++++++++++++++++++++++++++++++++++++++++++++++++++++
\section{Polynômes cyclotomiques}
%+++++++++++++++++++++++++++++++++++++++++++++++++++++++++++++++++++++++++++++++++++++++++++++++++++++++++++++++++++++++++++

Soit \( n\in \eN^*\). Nous considérons le polynôme \( (X^n-1)\in\eC[X]\). Les racines de ce polynôme forment le groupe
\begin{equation}
    \gU_n=\{ \xi^k\tq k=0,\ldots, n-1 \}
\end{equation}
avec \( \xi= e^{2i\pi/n}\). Ce groupe est un groupe cyclique généré par \( \xi\). Les autres générateurs sont les \( \xi^p\) avec \( \pgcd(p,n)=1\). Nous notons \( \Delta_n\) l'ensemble des générateurs de \( \gU_n\) :
\begin{equation}
    \Delta_n=\{  e^{2ki\pi/n}\tq 0\leq k\leq n-1,\pgcd(k,n)=1 \}.
\end{equation}
Ces éléments sont les \defe{racines primitives}{racine!primitive de l'unité} de l'unité dans \( \eC\). Nous avons 
\begin{equation}
    \Card(\Delta_n)=\varphi(n)
\end{equation}
où \( \varphi\) est la fonction d'Euler définie par \eqref{EqEulerGqPsvi}.

Nous avons par exemple
\begin{subequations}
    \begin{align}
        \Delta_1&=\{ 1 \}\\
        \Delta_2&=\{  e^{\pi i} \}\\
        \Delta_4&=\{  e^{\pi i/2}, e^{3\pi i/2} \}.
    \end{align}
\end{subequations}
Notons que \( 1\in \Delta_d\) seulement avec \( d=1\).

\begin{lemma}       \label{LemKcpjee}
    Nous avons
    \begin{equation}        \label{EqpZuIyL}
        \gU_n=\bigcup_{d\divides n}\Delta_d
    \end{equation}
    et l'union est disjointe. Nous avons aussi la formule
    \begin{equation}        \label{EqTPHqgJ}
        n=\sum_{d\divides n}\varphi(d).
    \end{equation}
\end{lemma}

\begin{proof}
    À l'application \( x\mapsto  e^{2i\pi x}\) près, nous pouvons considérer
    \begin{equation}
        \Delta_d=\{ \frac{ k }{ d }\tq k=0,\ldots, d-1, \pgcd(k,d)=1 \},
    \end{equation}
    c'est à dire l'ensemble des fractions irréductibles dont le dénominateur est \( d\). L'union des \( \Delta_d\) sera donc disjointe.
    
    Toujours à l'application \( x\mapsto  e^{2i\pi x}\) près, le groupe \( \gU_n\) est donné par
    \begin{equation}
        \gU_n=\{ \frac{ k }{ n }\tq k=0,\ldots, n-1 \}.
    \end{equation}
    L'égalité \eqref{EqpZuIyL} revient maintenant à dire que toute fraction de la forme \( \frac{ k }{ n }\) s'écrit de façon irréductible avec un dénominateur qui divise \( n\).

    La relation \eqref{EqTPHqgJ} consiste à prendre le cardinal des deux côtés de \eqref{EqpZuIyL}. Nous avons \( \Card(\gU_n)=n\) et l'union étant disjointe, à droite nous avons la somme des cardinaux.
\end{proof}

Le \defe{polynôme cyclotomique}{polynôme!cyclotomique} d'indice \( n\) est le polynôme
\begin{equation}
    \phi_n(X)=\prod_{z\in\Delta_n}(X-z).
\end{equation}
C'est un polynôme unitaire de degré \( \varphi(n)\). Pour former \( \Delta_n\), nous posons \( \xi= e^{2\pi i/n}\) et puis
\begin{equation}
    \Delta_n=\{ \xi^k\, 0\leq k\leq n-1\tq \pgcd(k,n)=1 \}.
\end{equation}
Nous avons par exemple
\begin{subequations}
    \begin{align}
        \Delta_1&=\{ 1 \}\\
        \Delta_2&=\{ -1 \}\\
        \Delta_3&=\{  e^{2\pi i/3, e^{4\pi i/3}} \}
    \end{align}
\end{subequations}
et les premiers polynômes cyclotomiques sont donnés par
\begin{subequations}
    \begin{align}
        \phi_1(X)&=X-1\\
        \phi_2(X)&=X+1\\
        \phi_3(X)&=X^2+X+1.
    \end{align}
\end{subequations}
Pour le dernier nous avons utilisé le fait que \(  e^{6\pi i/3}=1\) et \(  e^{4\pi i/3+ e^{2\pi i/3}}=-1\).

\begin{proposition}     \label{PropUImYnL}
    Soient \( 1\leq m\leq n\) deux entiers et
    \begin{equation}
        T(X)=\frac{ X^n-1 }{ X^m-1 }\in \eZ(X).
    \end{equation}
    Soit \( \phi_n\) le \( n\)-ième polynôme cyclotomique. Alors
    \begin{enumerate}
        \item
            \( X^n-1=\prod_{d\divides n}\phi_d(X)\),
        \item
            \( \phi_n\in \eZ[X]\),
        \item   \label{ItemhpDPKE}
            si \( m\divides n\) alors \( T\in \eZ[X]\),
        \item
            si \( m\divides n\) et si \( m<n\) alors \( \phi_n\) divise \( T\) dans \( \eZ[X]\).
    \end{enumerate}
\end{proposition}

\begin{proof}

    \begin{enumerate}
        \item

            Nous connaissons l'union disjointe \( \gU_n=\bigcup_{d\divides n}\Delta_d\) qui implique
            \begin{equation}
                \prod_{z\in \gU_n}(X-z)=\prod_{d\divides n}\prod_{z\in \Delta_d}(X-z)=\prod_{d\divides n}\phi_d(X),
            \end{equation}
            alors que par définition de \( \gU_n\) nous avons \( X^n-1=\prod_{z\in\gU_n}(X-z)\).

        \item

            Nous devons démontrer que les coefficients de \( \phi_n\) sont dans \( \eZ\) alors qu'ils sont a priori dans \( \eC\). Nous démontrons cela par récurrence. D'abord \( \phi_1(X)=X-1\), d'accord. Ensuite
            \begin{equation}
                X^{n+1}-1=\prod_{d\divides n+1}\phi_d(X)=\phi_{n+1}(X)\cdot\underbrace{\prod_{_{\substack{d\divides n+1\\d\leq n}}}\phi_d(X)}_{\in\eZ[X]\text{ par récurrence}}
            \end{equation}
            Le lemme \ref{LemzwkYdn} conclu que \( \phi_{n+1}\in \eZ[X]\). Nous avons vu \( \eZ\) comme sous anneau du corps \( \eC\).

        \item

            Si \( m\) divise \( n\) alors les diviseurs de \( n\) sont l'union des diviseurs de \( m\) et des diviseurs de \( n\) qui ne divisent pas \( m\). Soit
            \begin{equation}
                Q=\{\text{diviseurs de \( n\) ne divisant pas \( m\)} \}.
            \end{equation}
            Nous avons alors
            \begin{equation}
                X^n-1=\prod_{d\divides n}\phi_d(X)=\prod_{d\divides m}\phi_d(X)\cdot\prod_{q\in Q}\phi_q(X)=(X^m-1)\cdot\prod_{q\in Q}\phi_q(X).
            \end{equation}
            Nous avons donc
            \begin{equation}
                T(X)=\frac{ X^n-1 }{ X^m-1 }=\prod_{q\in Q}\phi_q(X)\in \eZ[X].
            \end{equation}
            
        \item

            Nous venons de montrer que
            \begin{equation}
                T=\prod_{q\in Q}\phi_q\in \eZ[X].
            \end{equation}
            Étant donné que \( m<n\) nous avons \( n\in Q\) et donc
            \begin{equation}
                T=\phi_n\cdot\prod_{q\in Q\setminus\{ n \}}\phi_q.
            \end{equation}
            Par conséquent \( \phi_n\) divise \( T\) dans \( \eZ[X]\).

        \end{enumerate}

\end{proof}

%+++++++++++++++++++++++++++++++++++++++++++++++++++++++++++++++++++++++++++++++++++++++++++++++++++++++++++++++++++++++++++
\section{Extensions de corps}
%+++++++++++++++++++++++++++++++++++++++++++++++++++++++++++++++++++++++++++++++++++++++++++++++++++++++++++++++++++++++++++

\begin{lemma}       \label{LemobATFP}
    Soit \( \eL\) un corps fini et \( \eK\) un sous corps de \( \eK\). Alors il existe \( s\in \eN\) tel que
    \begin{equation}        \label{EqUgqlJQ}
        \Card(\eK)=\Card(\eL)^n.
    \end{equation}
\end{lemma}

\begin{proof}
    Le corps \( \eL\) est un \( \eK\)-espace vectoriel de dimension finie. Si \( s\) est la dimension alors nous avons la formule \eqref{EqUgqlJQ} parce que chaque élément de \( \eL\) est un \( s\)-uple d'éléments de \( \eK\).
\end{proof}

\begin{definition}
    Soit \( \eK\) un corps commutatif. Une \defe{extension}{extension!de corps} de \( \eK\) est un corps \( \eL\) muni d'un morphisme \( i\colon \eL\to \eK\). Nous identifions le plus souvent \( \eK\) avec \( i(\eK)\subset \eL\).
\end{definition}

Notons que \( \eL\) est un espace vectoriel sur \( \eK\) parce que si \( \lambda\in \eK\) et \( x\in \eL\) nous pouvons considérer la multiplication scalaire
\begin{equation}
    \lambda\cdot x=i(\lambda)x
\end{equation}
où la multiplication du membre de droite est celle du corps \( \eL\). Le \defe{degré}{degré!extension de corps} de \( \eL\) est la dimension de cet espace vectoriel. Il est noté \( [\eL:\eK]\)\nomenclature[A]{\( [\eL:\eK]\)}{degré d'une extension de corps}; notons qu'il peut être infini.

Soit \( \eL\) une extension de \( \eK\) et \( A\subset \eL\). Nous notons \( \eK(A)\)\nomenclature[A]{$\eK(A)$}{corps contenant \( \eK\) et \( A\)} le plus petit sous corps de \( \eL\) qui contient \( \eK\) et \( A\). Nous notons \( \eK[A]\)\nomenclature[A]{$\eK[A]$}{anneau contenant \( \eK\) et \( A\)} le plus petit sous anneau de \( \eL\) qui contienne \( \eK\) et \( A\).

Nous disons que l'extension \( \eL\) de \( \eK\) est \defe{monogène}{monogène!extension de corps} ou \defe{\wikipedia{fr}{Extension_simple}{simple}}{extension!simple}\index{simple!extension de corps} si il existe \( x\in\eL\) tel que \( \eL=\eK(x)\).

Une extension \( \eL\) de \( \eK\) est \defe{séparable}{séparable!extension de corps} si le polynôme minimum de tout élément de \( \eL\) n'admet que des racines simples.

\begin{lemma}
    Soit \( P\in\eK[X]\) un polynôme unitaire irréductible de degré \( n\). Il existe une extension \( \eL\) de \( \eK\) et \( a\in \eL\) telle que \( \eL=\eK(a)\) et \( P\) est le polynôme minimal de \( a\) dans \( \eL\).
\end{lemma}

\begin{proof}
    Nous prenons \( \eL=\eK[X]/(P)\) où \( (P)\) est l'idéal dans \( \eK[X]\) généré par \( P\). Cela est un corps par le corollaire \ref{CorsLGiEN}. Nous identifions \( \eK\) avec \( \phi(\eK)\) où
    \begin{equation}
        \phi\colon \eK[X]\to \eL 
    \end{equation}
    est la projection canonique. Nous considérons également \( a=\phi(X)\).

    Nous avons alors \( P(a)=0\) dans \( \eL\). En effet \( P(a)=P\big( \phi(X) \big)\) est à voir comme l'application du polynôme \( P\) au polynôme \( X\), le résultat étant encore un élément de \( \eL\). En l'occurrence le résultat est \( P\) qui vaut \( 0\) dans \( \eL\).

    Le polynôme \( P\) étant unitaire et irréductible, il est minimum dans \( \eL\).

    Nous devons encore montrer que \( \eL=\eK(a)\). Le fait que \( \eK(a)\subset \eL\) est une tautologie parce qu'on calcule \( \eK(a)\) dans \( \eL\). Pour l'inclusion inverse soit \( Q(X)=\sum_iQ_iX^i\) dans \( \eK[X]\). Dans \( \eL\) nous avons évidemment \( Q=\sum_iQ_ia^i\).
\end{proof}

%---------------------------------------------------------------------------------------------------------------------------
\subsection{Théorème de Wedderburn}
%---------------------------------------------------------------------------------------------------------------------------

\begin{theorem}[\href{http://www.les-mathematiques.net/d/a/w/node5.php}{Théorème de Wedderburn}]
    Tout corps fini est commutatif.
\end{theorem}

\begin{proof}
    Soit \( \eK\) un corps fini et \( Z\), le centre de \( \eK\). Ce dernier est un corps fini et un sous corps de \( \eK\). Si \( q=\Card(Z)\) alors par le lemme \ref{LemobATFP} nous avons
    \begin{equation}
        \Card(\eK)=q^n
    \end{equation}
    pour un certain \( n\).

    Nous supposons maintenant que \( \eK\) est non commutatif. Dans ce cas \( Z\neq \eK\) et nous avons \( n\geq 2\). Nous considérons aussi
    \begin{equation}
        Z_x=\{ a\in \eK\tq ax=xa \}.
    \end{equation}
    Le centre \( Z\) est un sous corps de \( Z_x\), donc il existe \( d(x)\) tel  que
    \begin{equation}
        \Card(Z_x)=q^{d(x)}.
    \end{equation}
    De la même manière, \( Z_x\) est un sous corps de \( \eK\), donc il existe \( m(x)\) tel que
    \begin{equation}
        \Card(\eK)=\Card(Z_x)^{m(x)}.
    \end{equation}
    En mettant bout à bout nous avons
    \begin{equation}
        q^n=\Card(Z_x)^{m(x)}=q^{d(x)m(x)},
    \end{equation}
    et par conséquent \( n=d(x)m(x)\). Le point important à retenir est que \( d(x)\) divise \( n\) pour tout \( x\in \eK\).

    Nous considérons maintenant l'action adjointe du groupe \( \eK^*\) sur lui-même :
    \begin{equation}
        \varphi(k)x=kxk^{-1}.
    \end{equation}
    Nous notons \( \mO_x\) l'orbite de \( x\in \eK^*\) pour cette action, et \( \Stab(x)\) son stabilisateur. Nous avons
    \begin{equation}
        Z_y=\Stab(y)\cup\{ 0 \}
    \end{equation}
    parce que \( Z_y\) et \( \Stab(y)\) ont les mêmes définitions, sauf que \( \Stab(y)\) est dans \( \eK^*\) alors que \( Z_y\) est dans \( \eK\). Nous avons donc
    \begin{equation}
        \Card\big( \Stab(y) \big)=\Card(Z_y)-1=q^{d(y)}-1.
    \end{equation}
    Nous avons \( \Card(\mO_x)=1\) si et seulement si \( \mO_x=\{ x \}\) si et seulement si \( \Stab(x)=\eK^*\) si et seulement si \( z\in Z^*\). Soient \( z_0,\ldots, z_{q-1}\) les éléments de \( Z\) avec \( z_0=0\). Ce sont les éléments qui auront une orbite réduite à un point. Les orbites qui coupent \( Z^*\) sont
    \begin{equation}
        \{ z_1 \},\ldots, \{ z_{q-1} \}
    \end{equation}
    et il y en a \( q-1\). Soient \( \mO_{y_1},\ldots, \mO_{y_r}\), les autres orbites. Nous utilisons l'équation des classes \eqref{EqkgGmoq} :
    \begin{equation}
        \Card(\eK^*)=\Card(Z^*)+\sum_{i=1}^{r}\frac{ \Card(\eK^*) }{ \Card(\Stab(y_i)) },
    \end{equation}
    mais \( \Card(Z^*)=q-1\), \( \Card(\eK^*)=q^n-1\) et \( \Card\big( \Stab(y_i) \big)=q^{d(y_i)}-1\), donc
    \begin{equation}        \label{EqBPBDzE}
        q^n-1=(q-1)+\sum_{i=1}^{r}\frac{ q^n-1 }{ q^{d(y_i)}-1 }.
    \end{equation}
    Nous considérons la fraction rationnelle
    \begin{equation}        \label{EqATGciu}
        F(X)=(X^n-1)-\sum_{i=1}^{r}\frac{ X^n-1 }{ X^{d(y_i)}-1 }.
    \end{equation}
    Étant donné que \( d(y_i)\) divise \( n\), nous avons, contrairement aux apparences, que \( F\in \eZ[X]\) par la proposition \ref{PropUImYnL}\ref{ItemhpDPKE}.

    Nous pouvons exploiter un peu mieux la proposition \ref{PropUImYnL} en remarquant que \( d(y_i)<n\) parce que sinon \( \Card(Z_{y_i})=\Card(\eK)\), ce qui signifierait que \( y_i\in Z\), ce qui nous avions exclu. Par conséquent le polynôme cyclotomique \( \phi_n\) divise 
    \begin{equation}
        \frac{ X^n-1 }{ X^{d(y_i)}-1 }
    \end{equation}
    dans \( \eZ[X]\). Le polynôme cyclotomique \( \phi_n\) divise également \( X^n-1\) et par conséquent \( \phi_n\) divise \( F\). Il existe donc \( Q\in \eZ[X]\) tel que \( F=Q\phi_n\). En particulier en évaluant en \( q\) :
    \begin{equation}    \label{eqmoLdJy}
        F(q)=Q(q)\phi_n(q)=q-1.
    \end{equation}
    En effet nous avons \( F(q)=q-1\) par construction : comparer \eqref{EqBPBDzE} avec \eqref{EqATGciu}. Évidemment \( q\neq 1\) parce que si \( q=1\) alors \( \Card(\eK)=1\) et le théorème est trivial. Par ailleurs \( Q(q)\) est un entier (parce que \( Q\in \eZ[X]\) et \( q\in \eN\)) et \( Q(q)\neq 0\), parce qu'à droite de \eqref{eqmoLdJy} nous avons \( q-1\neq 0\). Nous avons donc \( | Q(q) |\geq 1\) et donc
    \begin{equation}
        | \phi_n(q) |\leq q-1.
    \end{equation}
    Par définition du polynôme cyclotomique nous avons
    \begin{equation}
        | \phi_n(q) |=\prod_{z\in\Delta_n}| q-z |.
    \end{equation}
    Étant donné que ce produit doit être inférieur à \( q-1\), au moins un des termes doit l'être : il existe \( z_0\in \Delta_n\) tel que \( | z_0-q |\leq q-1\). Étant donné que \( n\geq 2\) nous avons \( z_0\neq 1\).

    Mais d'autre part, comme indiqué sur la figure \ref{LabelFigtrigoWedd}, la distance entre \( z_0\) et \( q\) doit être strictement plus grande que \( q-1\) parce que \( q-1\) est le minimum de la distance entre le cercle trigonométrique et \( q\), et n'est atteint qu'en \( z=1\).
    \newcommand{\CaptionFigtrigoWedd}{Nous devons avoir \( | z_0-q |>q-1\).}
    \input{Fig_trigoWedd.pstricks}

    Nous avons ainsi obtenu une contradiction, et nous concluons que le corps \( \eK\) est commutatif.
\end{proof}

%---------------------------------------------------------------------------------------------------------------------------
\subsection{Corps de rupture}
%---------------------------------------------------------------------------------------------------------------------------

\begin{definition}
    Soit \( P\in\eK[X]\) un polynôme irréductible. Une extension \( \eL\) de \( \eK\) est un \defe{corps de rupture}{corps!de rupture}\index{rupture!corps} pour \( P\) si il existe \( a\in \eL\) tel que \( P(a)=0\) et \( \eL=\eK(a)\).
\end{definition}

\begin{example}     \label{ExemGVxJUC}
    Soit \( \eK=\eQ\) et \( P(X)=X^2-2\). On pose \( a=\sqrt{2}\) et \( \eL=\eQ(\sqrt{2})\subset\eR\). De cette façon \( P\) est scindé :
    \begin{equation}
        P=(X-\sqrt{2})(X+\sqrt{2}).
    \end{equation}
    Le corps \( \eQ(\sqrt{2})\) est donc un corps de rupture pour \( P\).
\end{example}

\begin{example}
    Dans l'exemple \ref{ExemGVxJUC}, le polynôme \( P\) était scindé dans son corps de rupture. Il n'en est pas toujours ainsi. Prenons 
    \begin{equation}
        P(X)=X^3-2
    \end{equation}
    et \( a=\sqrt[3]{2}\). Nous avons, certes, \( P(a)=0\) dans \( \eQ(\sqrt[3]{2})\), mais \( P\) n'est pas scindé parce qu'il y a deux racines complexes.
\end{example}

\begin{example}
    Nous considérons le corps \( \eZ/p\eZ\) où \( p\) est un nombre premier. Si \( s\in \eZ/p\eZ\) n'est pas un carré, alors le polynôme \(P= X^2+s\) est irréductible et un corps de rupture de \( P\) sur \( \eZ/p\eZ\) est donné par \( (\eZ/p\eZ)[X]/(X^2+s)\), c'est à dire l'ensemble des polynômes de degrés \( 1\) en \( \sqrt{s}\). Le cardinal en est \( p^2\).
\end{example}

%---------------------------------------------------------------------------------------------------------------------------
\subsection{Corps de décomposition}
%---------------------------------------------------------------------------------------------------------------------------

\begin{definition}
    Soit \( \eK\) un corps commutatif et \( F=(P_i)_{i\in I}\) une famille d'éléments non constants de \( \eK[X]\). Un \defe{corps de décomposition}{corps!de décomposition}\index{décomposition!corps} de \( F\) est une extension \( \eL\) de \( \eK\) telle que
    \begin{enumerate}
        \item
            les \( P_i\) sont scindés sur \( \eL\),
        \item
            \( \eL=\eK(R)\) où \( R=\bigcup_{i\in I}\{ x\in\eL\tq P_i(x)=0 \}\).
    \end{enumerate}
    C'est à dire que \( \eL\) étends \( \eK\) par toutes les racines de tous les polynômes de \( F\).
\end{definition}

L'unicité est due à la proposition suivante.
\begin{proposition}     \label{PropTMkfyM}
    Soit \( \eK\) un corps et \( P\in\eK[X]\). Soient \( \eL\) et \( \eF\) deux corps de décomposition de \( P\). Alors il existe un isomorphisme \( f\colon \eL\to \eF\) tel que \( f|_{\eK}=\id\).
\end{proposition}
Nous pouvons donc parler du corps de décomposition d'un polynôme.

Soit \( \eK\), un corps et \( \eL\), une extension de \( \eK\). Un élément \( a\in \eL\) est \defe{algébrique}{algébrique!nombre} sur \( \eK\) si il existe un polynôme \( P\in \eK[X]\) tel que \( P(a)=0\).

Une \defe{clôture algébrique}{clôture algébrique} du corps \( \eK\) est une extension algébriquement close de \( \eK\) dont tous les éléments sont algébriques sur \( \eK\).

\begin{remark}
    L'ensemble \( \eC\) n'est pas une clôture algébrique de \( \eQ\) parce qu'il existe des éléments de \( \eC\) qui ne sont pas des racines de polynômes à coefficients rationnels.
\end{remark}
L'existence d'une clôture algébrique pour tout corps est le théorème de Steinitz.
%TODO : à faire, le théorème de Steinitz.

\begin{example}     \label{ExfUqQXQ}
    Soit \( p\) un nombre premier. Montrons que le polynôme 
    \begin{equation}
        Q(X)=X^p-X+1
    \end{equation}
    est irréductible dans \( \eF_p\). 

    Nous supposons qu'il n'est pas irréductible, c'est à dire que
    \begin{equation}
        Q(X)=R(X)S(X)
    \end{equation}
    avec \( R\) et \( S\), des polynômes de degrés \( \geq 1\) dans \( \eF_p[X]\)

    Soit \( \bar\eF_p\) une clôture algébrique de \( \eF_p\) et \( \alpha\in \bar \eF_p\) tel que \( R(\alpha)=0\). Pour tout \( a\in \eF_p\), nous avons
    \begin{subequations}
        \begin{align}
            Q(\alpha+a)&=(\alpha+a)^p-(\alpha+a)+1\\
            &=\alpha^p+a^p-\alpha-a+1\\
            &=\alpha^p-\alpha+1\\
            &=Q(\alpha)\\
            &=0
        \end{align}
    \end{subequations}
    où nous avons utilisé le fait que \( a^p=a\) et que \( \alpha\) était une racine de \( Q\). Ce que nous venons de prouver est que l'ensemble des racines de \( Q\) dans \( \bar\eF_p\) est donné par \( \{ \alpha+a\tq a\in \eF_p \}\).

    Les polynômes \( R\) et \( S\) sont donc formés de produits de termes \( X-(\alpha+a)\) avec \( a\in \eF_p\). L'un des deux --disons \( R\) pour fixer les idées-- doit bien en avoir plus que \( 1\). Nous avons alors
    \begin{equation}
        R(X)=\prod_{i=1}^{k}\big( X-(\alpha+a_i) \big)
    \end{equation}
    où les \( a_i\) sont les éléments de \( \eF_p\). En développant un peu,
    \begin{equation}
        R(X)=X^k-\sum_{i=1}^k(\alpha+a_i^{k-1})+\text{termes de degré plus bas en \( X\)}.
    \end{equation}
    Le coefficient devant \( X^{k-1}\) n'est autre que \( k\alpha+\sum_ia_i\). Étant donné que \( k\neq 0\) et que \( R\in \eF_p[X]\), nous devons avoir \( \alpha\in \eF_p\). Par conséquent nous avons \( \alpha^p=\alpha\) et une contradiction :
    \begin{equation}
        Q(\alpha)=\alpha^p-\alpha+1=1\neq 0.
    \end{equation}

    Le polynôme \( X^p-X+1\) est donc irréductible sur \( \eF_p\).
\end{example}

%+++++++++++++++++++++++++++++++++++++++++++++++++++++++++++++++++++++++++++++++++++++++++++++++++++++++++++++++++++++++++++
\section{Théorie de Galois}
%+++++++++++++++++++++++++++++++++++++++++++++++++++++++++++++++++++++++++++++++++++++++++++++++++++++++++++++++++++++++++++

Vous trouverez des détails et des preuves dans \cite{GalIEl}.

\begin{definition}
    Soit $\eK$, un corps.
    
    Le \defe{groupe de Galois}{groupe!de Galois} d'une extension \( \eL\) de \( \eK\) est le groupe des automorphismes de \( \eL\) laissant \( \eK\) invariant. 

    Le groupe de Galois d'un polynôme sur \( \eK\) est le groupe de Galois de son corps de décomposition sur \( \eK\).
\end{definition}

\begin{definition}
    Des éléments \( b_1,\ldots, b_n\) d'une extension de \( \eK\) sont \defe{algébriquement indépendants}{algébriquement!indépendant}\index{indépendance!algébrique} si ils ne satisfont à aucune relation du type
    \begin{equation}
        \sum \alpha_{i_1\ldots i_n}b_1^{i_1}\ldots b_n^{i_n}=0
    \end{equation}
    avec \( \alpha_{i_1\ldots i_n}\in \eK\).
\end{definition}

Nous disons que l'équation
\begin{equation}
    x^n+a_{n-1}x^{n-1}+\ldots+a_1x+a_0=0
\end{equation}
est l'\defe{équation générale}{équation!générale de degré \( n\)} de degré \( n\) si les coefficients \( a_i\) sont algébriquement indépendants sur \( \eK\).

\begin{theorem}
    Le groupe de Galois d'un polynôme de degré \( n\) est isomorphe au groupe symétrique \( S_n\).
\end{theorem}

\begin{corollary}
    L'équation générale de degré \( n\) est résoluble par radicaux si et seulement si \( n\geq 5\).
\end{corollary}

%+++++++++++++++++++++++++++++++++++++++++++++++++++++++++++++++++++++++++++++++++++++++++++++++++++++++++++++++++++++++++++
\section{Corps finis}
%+++++++++++++++++++++++++++++++++++++++++++++++++++++++++++++++++++++++++++++++++++++++++++++++++++++++++++++++++++++++++++
\label{SecCorpsFinizkAcbS}

%---------------------------------------------------------------------------------------------------------------------------
\subsection{Existence, unicité}
%---------------------------------------------------------------------------------------------------------------------------

Nous avons déjà défini le corps fini \( \eF_p\) lorsque \( p\) est un nombre premier dans la section \ref{subseccorpspremhBlYIv}. Le théorème suivant sert à définir \( \eF_{p^n}\)\nomenclature[A]{\( \eF_{p^n}\)}{corps fini à \( p^n\) éléments} lorsque \( p\) est premier.
\begin{theorem}     \label{ThoOzgSfy}
    Soit \( p\) un nombre premier, soit \( n\in \eN^*\) et \( q=p^n\). Alors il existe un unique corps \( \eK\) de cardinal \( q\). Ce corps est le corps de décomposition du polynôme \( X^q-X\) sur \( \eF_p\).
\end{theorem}

\begin{proof}
    Montrons l'unicité. Soit \( \eK\) un corps fini de cardinal \( q=p^n\). Le groupe multiplicatif \( \eK^*\) est de cardinal \( q-1\), et par le corollaire \ref{CorpZItFX} tous les éléments de \( \eK^*\) vérifient \( g^{q-1}=e\), c'est à dire que dans \( \eK[X]\), les éléments de \( \eK^*\) sont des racines du polynôme
    \begin{equation}
        X^{q-1}-1
    \end{equation}
    Par conséquent \( \eK\) est un corps de décomposition pour le polynôme \( Q(X)=X^q-X=X(X^{q-1}-1)\) parce que \( Q(X)=0\) dans \( \eK\). Il est unique par la proposition \ref{PropTMkfyM}.

    Montrons maintenant que le corps de décomposition de \( P=X^q-X\) sur \( \eF_p\) est un corps de cardinal \( q\). Pour ce faire nous considérons \( \eK\) ce corps de décomposition et \(\eE\), l'ensemble des racines de \( P\) dans \( \eK\). Nous allons montrer que \( \eE=\eK\) et que \( \eE\) est un corps contenant \( q\) éléments.

    Montrons que \( \eE\) est un corps. Pour \( \alpha,\beta\in \eE\) nous avons
    \begin{equation}
        (\alpha\beta)^q=\alpha^q\beta^q=\alpha\beta
    \end{equation}
    parce que \( \alpha^q=\alpha\). Le produit \( \alpha\beta\) est donc encore dans \( \eE\). Pour la somme,
    \begin{equation}
        (\alpha+\beta)^q=(\alpha+\beta)^{p^n}=\Big( (\alpha+\beta)^p \Big)^{p^{n-1}}=(\alpha^p+\beta^p)^{p^{n-1}}=\ldots=\alpha^{p^n}+\beta^{p^n}=\alpha+\beta.
    \end{equation}
    En ce qui concerne l'inverse,
    \begin{equation}
        (\alpha^{-1})^q=(\alpha^q)^{-1}=\alpha^{-1}.
    \end{equation}
    Donc \( \eE\) est un corps. Évidemment \( \eE\) est un corps de décomposition de \( P\) au sens où \( \eE\) est une extension de \( \eF_p\) sur lequel \( P\) est scindé (parce qu'il est scindé sur \( \eK\) et \( \eE\) est le sous corps de \( \eK\) contenant les racines de \( P\)) et tel que \( \eE=\eF_p(\{ \alpha_i \})\) où les \( \alpha_i\) sont les racines de \( P\). Notons que \( \eF_p\subset \eE\) parce que dans \( \eF_p\) on a \( x^q=x\).

    Par unicité, nous avons \( \eK=\eE\). Nous devons montrer que \( P\) possède exactement \( q\) racines distinctes, affin d'avoir \( \Card(\eE)=q\). Pour cela remarquons que 
    \begin{equation}
        P'(X)=qX^{q-1}-1=-1
    \end{equation}
    dans \( \eF_p\). En effet \( P\in\eF_p\) et \( q=0\) dans \( \eF_p\). Par conséquent \( P'\) ne s'annule pas et \( P\) n'a pas de racines doubles. Toutes les racines étant simples, il y en a exactement \( q\).

\end{proof}

Le théorème \ref{ThoOzgSfy} ne permet pas de \emph{construire} le corps à \( q=p^n\) éléments. Nous allons maintenant voir un certain nombre de résultats donnant des façons de construire. Ces résultats proviennent de \cite{MichelMerlecorpsfinis,GabrielPeyre,RodierCorpsFinis} et de \wikipedia{fr}{Théorème_de_l'élément_primitif}{wikipedia} 

\begin{proposition}     \label{PropnfebjI}
    Soit \( \eK\) un corps fini. Alors le groupe multiplicatif \( \eK^*\) est cyclique.
\end{proposition}

\begin{proof}
    Soit \( \eK\) un corps ayant \( q\) éléments. Le groupe \( \eK^*\) en a \( q-1\), de telle façon à ce que l'ordre des éléments de \( \eK^*\) soient des diviseurs de \( q-1\); c'est le corollaire \ref{CorpZItFX}. Soit \( d\) un diviseur de \( q-1\) et
    \begin{subequations}
        \begin{align}
            H^*_d&=\{ x\,\text{d'ordre \( d\) dans \( \eK^*\)} \}\\
            H_d&=\{ \text{racines de \( X^d-1\) dans \( \eK\)} \}.
        \end{align}
    \end{subequations}
    Ici le polynôme \( X^d-1\) est vu dans \( \eK[X]\). Notons que nous avons automatiquement \( H^*_d\subset H_d\), mais l'inclusion inverse n'est pas assurée parce que les éléments d'ordre \( d/2\) par exemple sont aussi dans \( H_d\). Supposons \( H^*_d\neq \emptyset\) et considérons \( a\in H^*_d\). Alors l'application
    \begin{equation}
        \begin{aligned}
            \phi\colon \eZ/d\eZ&\to H_d \\
            n&\mapsto a^n 
        \end{aligned}
    \end{equation}
    est un isomorphisme d'anneaux. En effet étant donné que \( a\in H^*_d\subset H_d\), l'ensemble \( H_d\) contient le groupe cyclique engendré par \( a\). Ce dernier contient, par construction, \( d\) éléments. Mais \( \Card(H_d)\leq d\) parce que \( H_d\) est l'ensemble des racines d'un polynôme de degré \( d\). Par conséquent \( \Card(H_d)=d\) et l'ensemble \( H_d\) est bien engendré par \( a\) et \( \phi\) est bien un isomorphisme. Par conséquent tous les éléments de \( H^*_d\) sont des générateurs de \( H_d\).

    Inversement soit \( x\) un générateur de \( H_d\). L'ordre de \( H_d\) étant \( d\), l'ordre de \( x\) doit être un diviseur de \( d\). Supposons donc que \( x\) soit d'ordre \( d/k\). Dans ce cas nous devrions avoir \( \Card(H_d)=d/k\), ce qui contredit l'isomorphisme \( \phi\).

    En conclusion, \( H^*_d\) est l'ensemble des générateurs du groupe \( H_d\). Le nombre de générateurs de \( \eZ/d\eZ\) étant \( \varphi(d)\) par la proposition \ref{PropZnmuphiGensn}, et \( H_d\) étant isomorphe à \( \eZ/d\eZ\) nous avons
    \begin{equation}
        \Card(H^*_d)=\varphi(d).
    \end{equation}
    
    Par conséquent si \( H^*_d\) n'est pas vide, son cardinal est \( \varphi(d)\). Nous avons 
    \begin{subequations}
        \begin{align}
            q-1&=\Card(\eK^*)\\
            &=\Card\big( \bigcup_{d\divides q-1}H^*_d \big)\\
            &=\sum_{d\divides q-1}\Card(H^*_d)\\
            &\leq \sum_{d\divides q-1}\varphi(d)\\
            &=q-1
        \end{align}
    \end{subequations}
    où nous avons utilisé le lemme \ref{LemKcpjee}. Par conséquent pour tout \( d\) divisant \( q-1\) nous avons \( \Card(H^*_d)=\varphi(d)\) et il y a au moins un élément d'ordre \( q-1\) dans \( \eK\). Cet élément engendre \( \eK^*\) parce que \( \eK^*\) contient exactement \( q-1\) éléments. Par conséquent \( \eK\) est cyclique.
\end{proof}

Lorsque \( \eK\) est un corps les éléments du groupe \( \eK^*\) sont les \defe{éléments primitifs}{primitif!élément d'un corps} de \( \eK\).

\begin{proposition}     \label{propQRcUlq}
    Soit \( \eK\) un corps contenant \( q\) éléments. Alors
    \begin{enumerate}
        \item
            \( x^q=x\) pour tout \( x\in \eK\),
        \item
            \( X^q-X=\prod_{a\in \eK}(X-a)\).
    \end{enumerate}
\end{proposition}

\begin{proof}
    Le groupe \( \eK^*\) ayant \( q-1\) éléments, ses éléments vérifient \( a^{q-1}=1\) par le corollaire \ref{CorpZItFX} et par conséquent \( a^q=aa^{q-1}=a \).

    Soit \( a\in \eK\). Étant donné que \( a^q-a=0\), le polynôme \( (X-a)\) divise \( X^q-X\) dans \( \eK[X]\). Par conséquent 
    \begin{equation}
        \prod_{a\in \eK}(X-a)
    \end{equation}
    divise également \( X^q-X\). Les polynômes \( X^q-X\) et \( \prod_{a\in \eK}(X-a)\) étant deux polynômes unitaires de même degré, le fait que l'un divise l'autre montre qu'ils sont égaux.
\end{proof}

\begin{example}
    Soit \( \eK=\eQ\) et \( \eL=\eQ(\sqrt{2},\sqrt{3})\). Afin de montrer que \( \eL=\eQ(\alpha)\) avec \( \alpha=\sqrt{2}+\sqrt{3}\) nous devons montrer que \( \sqrt{2}\) et \( \sqrt{3}\) sont des polynômes en \( \alpha\).
\end{example}

%---------------------------------------------------------------------------------------------------------------------------
\subsection{Symboles de Legendre et carrés}
%---------------------------------------------------------------------------------------------------------------------------

Source : \cite{RecQuadVento}.

Nous disons que \( a\in \eF_p\) est un \defe{carré}{carré!dans un corps fini} si il existe \( b\in \eF_p\) tel que \( a=b^2\).

\begin{definition}
    Soit \( n\in \eN\) et \( p>2\) un nombre premier. Le \defe{symbole de Legendre}{symbole!de Legendre}\index{Legendre!symbole} par
    \begin{equation}
        \left( \frac{ n }{ p } \right)=\begin{cases}
            0    &   \text{si \( p\) divise \( n\)}\\
            1    &    \text{si \( n\) est un carré dans \( \eF_p\)}\\
            -1    &    \text{sinon}.
        \end{cases}
    \end{equation}
\end{definition}

\begin{proposition} \label{PropcGsJjk}
    Soit un nombre premier \( p>2\). Le corps \( \eF_p^*\) contient autant de carrés que de non carrés. De plus pour tout \( n\in \eN\) nous avons
    \begin{equation}    \label{Eqbcugos}
        \left(\frac{n}{p}\right)=n^{(p-1)/n}\mod p.
    \end{equation}
\end{proposition}

\begin{proof}
    Nous considérons l'application 
    \begin{equation}
        \begin{aligned}
            \psi\colon \eF^*_p&\to \eF^*_p \\
            x&\mapsto x^2. 
        \end{aligned}
    \end{equation}
    C'est un morphisme de groupes multiplicatifs et \( \ker\psi=\{ -1,1 \}\). Étant donné que \( p>2\), nous avons alors
    \begin{equation}
        \Card(\ker\psi)=2
    \end{equation}
    parce que \( 1\neq -1\). Évidemment l'ensemble des carrés dans \( \eF^*_p\) est l'image de \( \psi\). Le premier théorème d'isomorphisme \ref{ThoPremierthoisomo}\ref{ItemWLCLdk} nous permet alors de conclure que
    \begin{equation}
        \Card(\Image(\psi))=\frac{ \Card(\eF^*_p) }{2}.
    \end{equation}
    Ceci prouve la première assertion.

    Par le petit théorème de Fermat (théorème \ref{ThoOPQOiO}), nous avons \( x^{p-1}=1\) pour tout \( x\in \eF^*_p\). Les \( (p-1)\) éléments de \( \eF^*_p\) sont donc tous racines d'un des deux polynômes
    \begin{equation}
        X^{(p-1)/2}=\pm 1.
    \end{equation}
    Mais chacun des deux ne peut avoir, au maximum, que \( (p-1)/2\) solutions. Ils ont donc chacun exactement \( (p-1)/2\) racines.

    Nous pouvons maintenant prouver la formule \eqref{Eqbcugos}. D'abord si \( n=0\), elle est évidente. Si \( n\) est un carré dans \( \eF_p\), nous posons \( n=x^2\) et nous avons
    \begin{equation}
        n^{(p-1)/2}=n^{p-1}=1=\left(\frac{n}{p}\right).
    \end{equation}
    Si \( n\) n'est pas un carré, c'est que \( n\) n'est pas une racine de \( X^{(p-1)/2}=1\). Le nombre \( n\) est alors une racine de \( X^{(p-1)/2}=-1\). Nous avons alors
    \begin{equation}
        n^{(p-1)/2}=-1=\left(\frac{n}{p}\right).
    \end{equation}
\end{proof}

\begin{corollary}   \label{CoruJosNz}
    Si \( a,b\in \eN\) et si \( p>2\) est un nombre premier, alors
    \begin{equation}
        \left(\frac{ab}{p}\right)=\left(\frac{a}{p}\right)\left(\frac{b}{p}\right).
    \end{equation}
\end{corollary}

\begin{proof}
    Par la formule \eqref{Eqbcugos},
    \begin{equation}
        \left(\frac{ab}{p}\right)=(ab)^{(p-1)/2}=a^{(p-1)/2}b^{(p-1)/2}=\left(\frac{a}{p}\right)\left(\frac{b}{p}\right).
    \end{equation}
\end{proof}

Soit un nombre premier \( q>2\) et \( \eA\), un anneau de caractéristique \( p\). Si \( \alpha\in \eA\) vérifie
\begin{equation}
    1+\alpha+\ldots+\alpha^{q-1}=0,
\end{equation}
nous définissons la \defe{somme de Gauss}{Gauss!somme de} par
\begin{equation}
    \tau=\sum_{x\in \eF_q}\left(\frac{i}{q}\right)\alpha^i=\sum_{x=1}^{q-1}\left(\frac{x}{q}\right)\alpha^i.
\end{equation}

\begin{proposition}
    Les sommes de gauss vérifient les propriétés suivantes.
    \begin{enumerate}
        \item
                $\tau^2=\epsilon(q)q$
            où \( \epsilon(q)=\left(\frac{-1}{q}\right)\).

        \item
            Si \( \eA\) est de caractéristique \( p\geq 3\) et si \( p\neq q\) alors
            \begin{equation}
                \tau^p=\left(\frac{p}{q}\right)\tau.
            \end{equation}
    \end{enumerate}
\end{proposition}

\begin{proof}
    D'abord nous notons que
    \begin{equation}
        \alpha^q-1=(\alpha-1)(1+\alpha+\ldots+\alpha^{q-1})=0
    \end{equation}
    par définition de \( \alpha\). Nous calculons
    \begin{subequations}
        \begin{align}
            \epsilon(q)\tau^2&=\epsilon(q)\sum_{x,y\in \eF_q}\left(\frac{x}{q}\right)\left(\frac{y}{q}\right)\alpha^{x+y}\\
            &=\sum_{x,y\in \eF_q}\left(\frac{-xy}{q}\right)\alpha^{x+y}. \label{EqlObFeo}\\
            &=\sum_{z\in \eF_q}\sum_{y\in \eF_q}\left(\frac{-(z-y)y}{q}\right)\alpha^{z}    \label{EqWyIhhk}\\
            &=\sum_{z\in \eF_q}s_z\alpha^z  \label{EqWoIszS}
        \end{align}
    \end{subequations}
    Justifications :
    \begin{itemize}
        \item 
            Pour obtenir \eqref{EqlObFeo} nous avons utilisé le corollaire \ref{CoruJosNz}. 
        \item
            \eqref{EqWyIhhk} est un changement de variable \( z=x+y\) dans la somme sur \( x\).
        \item
            Pour \eqref{EqWoIszS} nous avons posé
            \begin{equation}
                s_z=\sum_{y\in \eF_q}\left(\frac{-(z-y)y}{q}\right).
            \end{equation}
            
    \end{itemize}
    Nous avons 
    \begin{equation}
        s_0=\sum_{y\in \eF_q}\left(\frac{y^2}{q}\right).
    \end{equation}
    Dans cette somme, tous les termes sont \( 1\) sauf celui avec \( y=0\) qui vaut zéro. Nous avons donc \( s_0=q-1\). Voyons maintenant \( s_y\) avec \( y\neq 0\). L'application
    \begin{equation}
        \begin{aligned}
            \eF^*_q&\to \eF_q\setminus\{ 1 \} \\
            k&\mapsto 1-zy^{-1} 
        \end{aligned}
    \end{equation}
    étant une bijection nous pouvons effectuer le changement de variables \( t=y^{-1}z-1\) pour la somme sur \( y\) en notant \( y^{-1}\) l'inverse de \( y\) dans \( \eF^*_q\), nous trouvons alors
    \begin{subequations}
        \begin{align}
            \sum_{y\in \eF_q}\left(\frac{y(z-y)}{q}\right)&=\sum_{y\in \eF_q}\left(\frac{y^2(y^{-1}z-1)}{q}\right)\\
            &=\sum_{y\in \eF_q}\left(\frac{y^{-1}z-1}{q}\right)\\
            &=\sum_{t\in \eF_q\setminus\{ 1 \}}\left(\frac{t}{q}\right)\\
            &=\underbrace{\sum_{t\in \eF_q}\left(\frac{y}{q}\right)}_{=0}-\left(\frac{1}{1}\right)\\
            &=-1
        \end{align}
    \end{subequations}
    parce qu'il  y a autant de carrés que de non carrés dans \( \eF_q^*\) (proposition \ref{PropcGsJjk}). En résumé nous avons
    \begin{equation}
        \epsilon(q)\tau^2=\sum_{z\in \eF_q}s_z\alpha^z
    \end{equation}
    où
    \begin{equation}
        s_z=\begin{cases}
            q-1    &   \text{si \( z=0\)}\\
            -1    &    \text{sinon}.
        \end{cases}
    \end{equation}
    Cela donne
    \begin{equation}
        \epsilon(q)\tau^2=(q-1)-\underbrace{(\alpha+\ldots +\alpha^{q-1})}_{=-1}=q
    \end{equation}
    où nous avons utilisé l'hypothèse sur \( \alpha\). Donc \( \epsilon(q)\tau^2=q\), et étant donné que \( \epsilon(q)=\pm 1\) nous concluons
    \begin{equation}
        \tau^2=\epsilon(q)q.
    \end{equation}
    
    Nous prouvons maintenant la seconde partie. Vu que \( \eA\) est de caractéristique \( p\) en utilisant le fait que le morphisme de Frobenius est un morphisme,
    \begin{equation}
        \tau^p=\left( \sum_{x\in \eF_q}\left(\frac{x}{q}\right)\alpha^x \right)^p=\sum_{x\in \eF_q}\left(\frac{x}{q}\right)^p\alpha^{px}.
    \end{equation}
    Étant donné que \( \left(\frac{x}{q}\right)=\pm 1\) et que \( p\) est impair, nous avons
    \begin{equation}
        \left(\frac{x}{q}\right)^p=\left(\frac{x}{q}\right).
    \end{equation}
    Du coup nous avons
    \begin{equation}
        \left(\frac{p}{q}\right)\tau^p=\sum_{x\in \eF_p}\left(\frac{xp}{q}\right)\alpha^{px}.
    \end{equation}
    Mais \( p\) étant inversible dans \( \eF_q\), l'application \( x\mapsto px\) est une bijection et nous pouvons sommer sur \( px\) au lieu de \( x\) :
    \begin{equation}
        \left(\frac{p}{q}\right)\tau^p=\sum_{x\in \eF_p}\left(\frac{x}{q}\right)\alpha^x=\tau.
    \end{equation}
    Nous trouvons alors que
    \begin{equation}
        \tau^p=\left(\frac{p}{q}\right)\tau.
    \end{equation}
\end{proof}

\begin{lemma}\label{Lemoabzrn}
    Si \( p\) est un nombre premier \( p\geq 3\), alors le symbole de Legendre \( x\mapsto\left(\frac{x}{p}\right)\) est l'unique morphisme non trivial de \( \eF^*_p\) dans \( \{ -1,1 \}\).
\end{lemma}

\begin{proof}
    Le fait que le symbole de Legendre soit non trivial est simplement le fait qu'il y ait des carrés et des non carrés dans \( \eF_p^*\); voir la proposition \ref{PropcGsJjk}. Pour l'unicité, soit \( \alpha\colon \eF^*_p\to \{ -1,1 \}\) un morphisme surjectif (c'est à dire non trivial). Étant donné que 
    \begin{equation}
        \eF^*_p=\ker(\alpha)\cup-\ker(\alpha),
    \end{equation}
    le groupe \( \eF_p^*/\ker(\alpha)\) ne contient que deux éléments : \( [1]\) et \( [-1]\). Autrement dit, \( \ker(\alpha)\) est d'indice \( 2\) dans \( \eF_p^*\). 
    
    Or \( \eF_p^*\) ne possède qu'un seul sous-groupe d'indice \( 2\). En effet soit \( S\) un tel sous groupe et \( a\), un générateur de \( \eF_p^*\) (qui est cyclique par la proposition \ref{PropnfebjI}), alors \( a^2\in S)\) par le lemme \ref{PropubeiGX}. Par conséquent \( S\) contient le groupe des puissances paires de \( a\). Le groupe $S$ ne peut rien contenir de plus parce qu'il est d'indice \( 2\) et que l'ordre de \( \eF_p^*\) est pair.

    Bref, le sous-groupe \( \ker(\alpha)\) est l'unique sous-groupe d'indice \( 2\) dans \( \eF_p^*\). Mais la proposition \ref{PropcGsJjk} nous indique que \( | (\eF_p^*)^2 |=\frac{ p-1 }{2}\), c'est à dire que le groupe des carrés est d'indice \( 2\). Nous avons donc, par l'unicité,
    \begin{equation}
        \ker(\alpha)=(\eF_p^*)^2.
    \end{equation}
    Au final, pour \( y\in \eF_p^*\),
    \begin{equation}
        \alpha(y)=\begin{cases}
           1 &   \text{si \( y\) est un carré}\\
            -1    &    \text{sinon.}
        \end{cases}
    \end{equation}
    Cela est bien la définition des symboles de Legendre.
\end{proof}
 

%---------------------------------------------------------------------------------------------------------------------------
\subsection{Théorème de Chevalley-Warning}
%---------------------------------------------------------------------------------------------------------------------------

\begin{lemma}
    Soit \( \eK\) un corps de caractéristique \( p\) et de cardinal \( q\). Pour \( m\in \eN\) nous définissons
    \begin{equation}
        S_m=\sum_{x\in \eK}x^m.
    \end{equation}
    Alors nous avons
    \begin{equation}
        S_m\mod p=\begin{cases}
            -1     &   \text{si \( m\geq 1\) et \( m\) divisible par \( q-1\)}\\
            0    &    \text{sinon}.
        \end{cases}
    \end{equation}
\end{lemma}

\begin{proof}
    Si \( m=0\), alors \( x^0=1\) et \( S_m=q\). Par conséquent \( S_m\mod p=0\) parce que la caractéristique d'un corps divise son ordre (proposition \ref{PropGExaUK}). 

    Nous prenons maintenant \( m\geq 1\) et nous voyons séparément les cas où \( q-1\) divise \( m\) ou non. Si \( q-1\) divise \( m\), alors pour tout \( x\neq 0\) nous avons
    \begin{equation}
        x^m=x^{k(q-1)}=1
    \end{equation}
    parce que \( \eK^*\) est cyclique et \( x^{q-1}=1\) par le petit théorème de Fermat (théorème \ref{ThoOPQOiO}). Par conséquent nous avons
    \begin{equation}
        \sum_{x\in \eK}x^m=\sum_{x\in \eK^*}1=q-1.
    \end{equation}
    
    Si le nombre \( m\geq 1\) n'est pas divisible par \( q-1\) alors nous prenons un générateur \( y\) du groupe \( \eK^*\). Un tel élément vérifie \( y^m\neq 1\). En effet, si \( y\) vérifiait \( y^m=1\) alors cela signifierait que l'ordre de \( \eK^*\) est un diviseur de \( m\), ce qui n'est pas le cas ici parce que l'ordre de \( \eK^*\) est \( q-1\). Pour un tel \( y\), l'application
    \begin{equation}
        \begin{aligned}
            \varphi\colon \eK^*&\to \eK^* \\
            x&\mapsto yx 
        \end{aligned}
    \end{equation}
    est une bijection\footnote{Notons que nous n'avons pas réellement besoin que \( y\) soit un générateur. Nous n'utilisons seulement le fait que \( y^m\neq 1\) et \( y\neq 0\).}. En ce qui concerne l'injectivité, \( ya=yb\) implique \( a=b\). En ce qui concerne la surjectivité, si \( a\) est un générateur, si \( z=a^l\) et si \( y=a^k\), alors
    \begin{equation}
        z=\varphi(a^{l-k}).
    \end{equation}
    Nous pouvons maintenant faire le calcul.
    \begin{equation}
        S_m=\sum_{x\in \eK^*}x^n=\sum_{x\in \eK^*}(yx)^m=y^m\sum_{x\in \eK^*}x^m=y^mS_m.
    \end{equation}
    Étant donné que \( y^m\neq 1\), la seule solution est \( S_m=0\).
\end{proof}

\begin{theorem}[\wikipedia{en}{Chevalley–Warning_theorem}{Chevalley-Warning}]\index{théorème!Chevalley-Warning}        \label{ThoLTcYKk}
    Soit \( \eK\) un corps fini de cardinal \( q\) et de caractéristique \( p\). Soient \( P_1,\ldots, P_r\) des éléments de \( \eK[X_1,\ldots, X_n]\) tels que \( \sum_{i=1}^r\deg(P_i)<n\). Nous considérons l'ensemble des zéros communs à tous les polynômes :
    \begin{equation}
        V=\{ x\in \eK^n\tq P_1(x)=\ldots=P_r(x)=0 \}.
    \end{equation}
    Alors \( \Card(V)=0\mod p\).
\end{theorem}

\begin{proof}
    Nous considérons le polynôme
    \begin{equation}
        P=\prod_{i=1}^r(1-P_i^{q-1}).
    \end{equation}
    Montrons que
    \begin{equation}
        P(x)=\begin{cases}
            1    &   \text{si \( x\in V\)}\\
            0    &    \text{sinon}.
        \end{cases}
    \end{equation}
    La première ligne est facile : étant donné que tous les \( P_i(x)\) sont nuls pour \( x\in V\), nous avons \( P(x)=1\). Si \( x\) n'est pas dans \( V\), alors nous avons un \( i\) tel que \( P_i(x)\in \eK^*\). Mais dans ce cas (toujours la cyclicité de \( \eK^*\)) nous avons \( P_i(x)^{q-1}=1\) et donc le produit est nul.

    En utilisant l'hypothèse sur le degré des \( P_i\), nous trouvons que
    \begin{equation}
        \deg(P)=\sum_{i=1}^r(q-1)\deg(P_i)<n(q-1).
    \end{equation}

    Pour un polynôme \( Q\in \eK[X_1,\ldots, X_n]\), nous définissons 
    \begin{equation}
        \int Q=\sum_{x\in \eK^n}Q(x).
    \end{equation}
    Nous avons immédiatement
    \begin{equation}
        \int P=\sum_{x\in \eK^n}P(x)=\sum_{x\in V}1=\Card(V)\mod p.
    \end{equation}
    Nous insistons sur le «modulo \( p\)» parce que dans la formule \( P(x)=1\), le membre de droite est le \( 1\) de \( \eK\); il est donc automatiquement modulo la caractéristique de \( \eK\).

    Il nous reste à prouver que \( \int P=0\). Pour cela nous décomposons 
    \begin{equation}        \label{EqHnUVlM}
        P=\sum_m c_mX_1^{m_1}\ldots X_n^{m_n}
    \end{equation}
    où la somme s'étend sur les \( m\in \eN^n\) tels que \( c_m\neq 0\). Nous avons
    \begin{subequations}
        \begin{align}
            \int P&=\sum_{\in \eK^n}\sum_mc_m x_1^{m_1}\ldots x_n^{m_n}\\
            &=\sum_{m}c_m\left( \sum_{x\in \eK^n}x_1^{m_1}\ldots x_n^{m_n} \right)\\
            &=\sum_m c_m S_{m_1}\ldots S_{m_n}.
        \end{align}
    \end{subequations}
    Le terme de plus haut degré dans la décomposition \eqref{EqHnUVlM} est celui du \( m\) tel que \( \sum_im_i\) est le plus grand, mais étant donné que nous savons que ce degré est plus petit que \( n(q-1)\), nous avons pour tous les \( m\) entrant dans la somme que
    \begin{equation}
        \sum_{i=1}^nm_i<n(q-1).
    \end{equation}
    En particulier pour tout \( m\in \eN^n\), il existe \( i\) tel que \( m_i<q-1\), et dans ce cas \( S_{m_i}=0\). Donc tous les termes de la somme
    \begin{equation}
        \sum_{m\in \eN^n}c_mS_{m_1}\ldots S_{m_n}
    \end{equation}
    ont un facteur nul.
\end{proof}

\begin{corollary}       \label{CorfuHNKz}
    Soit \( P_i\) des polynômes à \( n\) variables avec \( \sum_{i=1}^r\deg(P_i)<n\). Si les \( P_i\) n'ont pas de termes constants, alors ils ont un zéro commun non trivial.
\end{corollary}

\begin{proof}
    Nous reprenons les notations du théorème \ref{ThoLTcYKk}. Étant donné que les \( P_i\) n'ont pas de termes constants, \( 0\in V\), mais \( \Card(V)=0\mod p\). Par conséquent nous devons avoir \( \Card(V)>p\).
\end{proof}

\begin{example}
    Nous considérons les polynômes
    \begin{subequations}
        \begin{align}
            P_1(x,y,t,u)=xy+x+ux\\
            P_2(x,y,t,u)=x+y-3t.
        \end{align}
    \end{subequations}
    La somme de leurs degrés est \( 3\) et ce sont des polynômes à \( 4\) variables. Nous devons donc avoir, en vertu du corollaire \ref{CorfuHNKz}, des autres racines que la racine triviale \( (x,y,t,u)=(0,0,0,0)\).

    Le corollaire nous donne aussi une borne inférieure nombre de racines à chercher : plus que la caractéristique du corps sur lequel nous travaillons. Nous pouvons dire cela sans avoir la moindre idée de la façon dont on pourrait résoudre le système \( P_1=P_2=0\).
\end{example}
