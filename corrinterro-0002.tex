% This is part of Exercices de mathématique pour SVT
% Copyright (c) 2010-2011
%   Laurent Claessens et Carlotta Donadello
% See the file fdl-1.3.txt for copying conditions.

\begin{corrige}{interro-0002}

\newcommand{\CaptionFigSubfiguresTracerUn}{Les fonctions de la question \ref{exointerro-0002}.}
\input{Fig_SubfiguresTracerUn.pstricks}

Les graphes sont tracés sur la figure \ref{LabelFigSubfiguresTracerUn}.
%See also the subfigure \ref{Fnex}
%See also the subfigure \ref{ValAbs}
%See also the subfigure \ref{Compo}
%See also the subfigure \ref{Logxmu}

\begin{enumerate}
	\item
		\begin{enumerate}
			\item
				La constante $e$ est juste un nombre (qui vaut approximativement $2.718$). La fonction est donc une simple droite de coefficient angulaire $e$.
			\item
				Remarquez que à droite, elle est plus pentue qu'à gauche.
			\item
				La fonction valeur absolue est une fonction de base.
			\item
				Nous avons toujours $\log_a(a^b)=b$. Donc ici il s'agit seulement de la fonction $y=x-1$ qui est une droite.
		\end{enumerate}
	\item
		De nombreux exemples sont possibles. Citons les plus courantes.
		\begin{description}
			\item[Fonctions paires] $\cos(x)$, $x^2$, $x^4$, toutes les fonctions constantes.
			\item[Fonctions impaires] $\sin(x)$, $x^3$, $x$, $x^5$, la fonction constante zéro.
		\end{description}
		Notez que la fonction constante $y=0$ est une fonction paire et impaire en même temps. La fonction $x^3+1$ n'est pas une fonction impaire.

\end{enumerate}


\end{corrige}
