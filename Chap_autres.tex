Nous regroupons dans cet appendice quelque définitions et résultats utilisés dans le cours, mais qui alourdiraient inutilement le texte principal.

\nomenclature{$\eN_0$}{les naturels non nuls : $\eN_0=\eN\setminus\{ 0 \}$}
%+++++++++++++++++++++++++++++++++++++++++++++++++++++++++++++++++++++++++++++++++++++++++++++++++++++++++++++++++++++++++++
\section{Complémentaire}
%+++++++++++++++++++++++++++++++++++++++++++++++++++++++++++++++++++++++++++++++++++++++++++++++++++++++++++++++++++++++++++
\label{AppComplement}

Soit $E$, un ensemble et $A$, une partie de $E$ (c'est à dire un sous-ensemble de $E$). Nous désignons par $\complement A$\nomenclature[T]{$\complement A$}{Le complémentaire de l'ensemble $A$} désigne le \defe{complémentaire}{complémentaire} de l'ensemble $A$ dans $E$. Il s'agit de l'ensemble des points de $E$ qui ne font pas partie de $A$ :
\begin{equation}
	\complement A=E\setminus A=\{ x\in E\tq x\notin A \}.
\end{equation}

\begin{lemma}		\label{LemPropsComplement}
	Quelque propriétés à propos des complémentaires. Si $E$ est un ensemble et si $A$ et $B$ sont des sous-ensembles de $E$, nous avons
	\begin{enumerate}
		\item
			$\complement \complement A =A $, en d'autres termes, $E\setminus(E\setminus A)=A$,
		\item
			$\complement(A\cap B)=\complement A\cup\complement B$,
		\item
			$\complement(A\cup B)=\complement A\cap\complement B$,
		\item	\label{ItemLemPropComplementiii}
			$A\setminus B=A\cap\complement B$.
	\end{enumerate}
\end{lemma}

%+++++++++++++++++++++++++++++++++++++++++++++++++++++++++++++++++++++++++++++++++++++++++++++++++++++++++++++++++++++++++++
\section{Relations d'équivalence}
%+++++++++++++++++++++++++++++++++++++++++++++++++++++++++++++++++++++++++++++++++++++++++++++++++++++++++++++++++++++++++++
\label{appEquivalence}

Si $E$ est un ensemble, une \defe{relation d'équivalence}{equivalence@équivalence!relation} sur $E$ est une relation $\sim$ telle qui est à la fois
\begin{description}
	\item[réflexive] $x\sim x$ pour tout $x\in E$,
	\item[symétrique] $x\sim y$ si et seulement si $y\sim x$;
	\item[transitive] si $x\sim y$ et $y\sim z$, alors $x\sim z$.
\end{description}
Par exemple, sur l'ensemble de tous les polygones du plan, la relation «a le même nombre de côté» est une relation d'équivalence. Plus précisément, si $P$ et $Q$ sont deux polygones, nous disons que $P\sim Q$ si et seulement si $P$ et $Q$ ont le même nombre de côté. Cela est une relation d'équivalence :
\begin{itemize}
	\item 
		un polygone $P$ a toujours le même nombre de côtés que lui-même : $P\sim P$;
	\item
		si $P$ a le même nombre de côtés que $Q$ ($P\sim Q$), alors $Q$ a le même nombre de côtés que $P$ ($Q\sim P$);
	\item
		si $P$ a le même nombre de côtés que $Q$ ($P\sim Q$) et que $Q$ a le même nombre de côtés que $R$ ($Q\sim R$), alors $P$ a le même nombre de côtés que $R$ ($P\sim R$).
\end{itemize}

