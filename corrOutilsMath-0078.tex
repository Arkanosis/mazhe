% This is part of Exercices et corrigés de CdI-1
% Copyright (c) 2011
%   Laurent Claessens
% See the file fdl-1.3.txt for copying conditions.

\begin{corrige}{OutilsMath-0078}

    Les coordonnées cylindriques sont données par le changement de variables
    \begin{equation}
        M(r,\theta,z)=\begin{pmatrix}
            r\cos(\theta)    \\ 
            r\sin(\theta)    \\ 
            z    
        \end{pmatrix}.
    \end{equation}
    Les dérivées de $M$ par rapport aux coordonnées cylindriques sont
    \begin{equation}
        \begin{aligned}[]
            \frac{ \partial M }{ \partial r }&=\frac{ \partial x }{ \partial r }e_x+\frac{ \partial y }{ \partial r }e_y+\frac{ \partial z }{ \partial r }e_z\\
            &=\cos(\theta)e_x+\sin(\theta)e_y\\
            \frac{ \partial M }{ \partial \theta }&=\frac{ \partial x }{ \partial \theta }e_x+\frac{ \partial \theta }{ \partial r }e_y+\frac{ \partial z }{ \partial \theta }e_z\\
            &=-r\sin(\theta)e_x+r\cos(\theta)e_y\\
            \frac{ \partial M }{ \partial z }&=\frac{ \partial x }{ \partial z }e_x+\frac{ \partial y }{ \partial z }e_y+\frac{ \partial z }{ \partial z }e_z\\
            &\quad=e_z.
        \end{aligned}
    \end{equation}
    Les vecteurs de la base locale des coordonnées cylindriques sont donc donnés en normalisant ces trois vecteurs :
    \begin{equation}
        \begin{aligned}[]
            e_r&=\cos(\theta)e_x+\sin(\theta)e_y\\          
            e_{\theta}&=-\sin(\theta)e_x+\cos(\theta)e_y\\          
            e_z&=e_z
        \end{aligned}
    \end{equation}
    Le fait que cette base locale soit orthogonale signifie que ces trois derniers vecteurs sont orthogonaux. Vérification :
    \begin{equation}
        \begin{aligned}[]
            e_r\cdot e_{\theta}&=\begin{pmatrix}
                \cos\theta    \\ 
                \sin\theta    \\ 
                0    
            \end{pmatrix}\cdot\begin{pmatrix}
                -\sin\theta    \\ 
                \cos\theta    \\ 
                0    
            \end{pmatrix}=0\\
            e_r\cdot e_z&=\begin{pmatrix}
                \cos\theta    \\ 
                \sin\theta    \\ 
                0    
            \end{pmatrix}\cdot\begin{pmatrix}
                0    \\ 
                0    \\ 
                1    
            \end{pmatrix}=0\\
            e_{\theta}\cdot e_z&=\begin{pmatrix}
                -\sin\theta    \\ 
                \cos\theta    \\ 
                0    
            \end{pmatrix}\cdot\begin{pmatrix}
                0    \\ 
                0    \\ 
                1    
            \end{pmatrix}=0.
        \end{aligned}
    \end{equation}

\end{corrige}
