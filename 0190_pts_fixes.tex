% This is part of Mes notes de mathématique
% Copyright (c) 2011-2014
%   Laurent Claessens
% See the file fdl-1.3.txt for copying conditions.

%+++++++++++++++++++++++++++++++++++++++++++++++++++++++++++++++++++++++++++++++++++++++++++++++++++++++++++++++++++++++++++
\section{Convolution}
%+++++++++++++++++++++++++++++++++++++++++++++++++++++++++++++++++++++++++++++++++++++++++++++++++++++++++++++++++++++++++++

Le théorème qui permet de définir le produit de convolution est la suivant.

\begin{theorem}[\cite{MesIntProbb}]
    Soient \( f,g\in L^1(\eR^n)\). 
    \begin{enumerate}
        \item
            Pour presque tout \( x\in \eR^n\), la fonction
            \begin{equation}
                y\mapsto g(x-y)f(y)
            \end{equation}
            est dans \( L^1(\eR^n)\), et nous définissons le \defe{produit de convolution}{produit!de convolution} de \( f\) et \( g\) par
            \begin{equation}
                (f*g)(x)=\int_{\eR^n} f(y)g(x-y)dy.
            \end{equation}
        \item
            \( f*g\in L^1(\eR^n)\).
        \item
            \( \| f*g \|_1\leq \| f \|_1\| g \|_1\).
    \end{enumerate}
\end{theorem}

L'ensemble \( L^1(\eR^n)\) devient alors une algèbre de Banach.

\begin{lemma}
    Le produit de convolution est commutatif : \( f*g=g*f\).
\end{lemma}

\begin{proof}
    Le théorème de Fubini (théorème \ref{ThoFubinioYLtPI}) permet d'écrire
    \begin{equation}
        (f*g)(x)=\int_{\eR^n}f(y)g(x-y)dy=\int_{-\infty}^{\infty}dy_1\ldots \int_{-\infty}^{\infty}dy_nf(y)g(x-y).
    \end{equation}
    En effectuant le changement de variable \( z_i=x_i-y_i\) dans chacune des intégrales nous obtenons
    \begin{equation}
        (f*g)(x)=\int_{\eR^n}g(z)f(x-z)dz=(g*f)(x).
    \end{equation}
\end{proof}

\begin{proposition}[\cite{CXCQJIt}] \label{PropHNbdMQe}
    Si \( f\in L^1(\eR)\) et si \( g\) est dérivable avec \( g'\in L^{\infty}\), alors \( f*g\) est dérivable et \( (f*g)'=f*g'\).
\end{proposition}

\begin{proof}
    La fonction qu'il faut intégrer pour obtenir \( f*g\) est $f(t)g(x-t)$, dont la dérivée par rapport à \( x\) est \( f(t)g'(x-t)\). La norme de cette dernière est majorée (uniformément en \( x\)) par \( G(t)=| f(t) | \| g' \|_{\infty}\). La fonction \( f\) étant dans \( L^1(\eR)\), la fonction \( G\) est intégrable et le théorème de dérivation sous l'intégrale (théorème \ref{ThoMWpRKYp}) nous dit que \( f*g\) est dérivable et
    \begin{equation}
        (f*g)'(x)=\frac{ d }{ dx }\int_{\eR}f(t)g(x-t)dt=\int_{\eR}f(t)g'(x-t)dt=(f*g')(x).
    \end{equation}
\end{proof}

%+++++++++++++++++++++++++++++++++++++++++++++++++++++++++++++++++++++++++++++++++++++++++++++++++++++++++++++++++++++++++++
\section{Théorème du point fixe de Picard}
%+++++++++++++++++++++++++++++++++++++++++++++++++++++++++++++++++++++++++++++++++++++++++++++++++++++++++++++++++++++++++++

\begin{definition}
    Une application \( f\colon (X,\| . \|_X)\to (Y,\| . \|_Y)\) entre deux espaces métriques est une \defe{contraction}{contraction} si elle est \( k\)-\defe{Lipschitz}{Lipschitz} pour un certain \( 0\leq k<1\), c'est à dire si pour tout \( x,y\in X\) nous avons
    \begin{equation}
        \| f(x)-f(y) \|_Y\leq k\| x-y \|_{X}.
    \end{equation}
\end{definition}

\begin{theorem}[Picard \cite{ClemKetl,NourdinAnal}\footnote{Il me semble qu'à la page 100 de \cite{NourdinAnal}, l'hypothèse H1 qui est prouvée ne prouve pas Hn dans le cas \( n=1\). Merci de m'écrire si vous pouvez confirmer ou infirmer. La preuve donnée ici ne contient pas cette «erreur».}.]     \label{ThoEPVkCL}
    Soit \( X\) un espace métrique complet et \( f\colon X\to X\) une application contractante, de constante de Lipschitz \( k\). Alors \( f\) admet un unique point fixe, nommé \( \xi\). Ce dernier est donné par la limite de la suite définie par récurrence 
    \begin{subequations}
        \begin{numcases}{}
            x_0\in X\\
            x_{n+1}=f(x_n).
        \end{numcases}
    \end{subequations}
    De plus nous pouvons majorer l'erreur par
    \begin{equation}    \label{EqKErdim}
        \| x_n-x \|\leq \frac{ k^n }{ 1-k }\| x_n-x_{n-1} \|\leq \frac{ k^n }{ 1-k }\| x_1-x_0 \|.
    \end{equation}

    Soit \( r>0\), \( a\in X\) tels que la fonction \( f\) laisse la boule \( K=\overline{ B(a,r) }\) invariante (c'est à dire que \( f\) se restreint à \( f\colon K\to K\)). Nous considérons les suites \( (u_n)\) et \( (v_n)\) définies par
    \begin{subequations}
        \begin{numcases}{}
            u_0=v_0\in K\\
            u_{n+1}=f(v_n), v_{n+1}\in B(u_n,\epsilon).
        \end{numcases}
    \end{subequations}
    Alors le point fixe \( \xi\) de \( f\) est dans \( K\) et la suite \( (v_n)\) satisfait l'estimation
    \begin{equation}
        \| v_n-\xi \|\leq \frac{ k^n }{ 1-k }\| u_1-u_0 \|+\frac{ \epsilon }{ 1-k }.
    \end{equation}
\end{theorem}
\index{théorème!Picard}
\index{point fixe!Picard}

La première inégalité \eqref{EqKErdim} donne une estimation de l'erreur calculable en cours de processus; la seconde donne une estimation de l'erreur calculable avant de commencer.

\begin{proof}
    
    Nous commençons par l'unicité du point fixe. Si \( a\) et \( b\) sont des points fixes, alors \( f(a)=a\) et \( f(b)=b\). Par conséquent
    \begin{equation}
        \| f(a)-f(b) \|=\| a-b \|,
    \end{equation}
    ce qui contredit le fait que \( f\) soit une contraction.

    En ce qui concerne l'existence, notons que si la suite des \( x_n\) converge dans \( X\), alors la limite est un point fixe. En effet en prenant la limite des deux côtés de l'équation \( x_{n+1}=f(x_n)\), nous obtenons \( \xi=f(\xi)\), c'est à dire que \( \xi\) est un point fixe de \( f\). Notons que nous avons utilisé ici la continuité de \( f\), laquelle est une conséquence du fait qu'elle soit Lipschitz. Nous allons donc porter nos efforts à prouver que la suite est de Cauchy (et donc convergente parce que \( X\) est complet). Nous commençons par prouver que \( \| x_{n+1}-x_n \|\leq k^n\| x_0-x_1 \|\). En effet pour tout \( n\) nous avons
    \begin{equation}
        \| x_{n+1}-x_n \|=\| f(x_n)-f(x_{n-1}) \|\leq k\| x_n-x_{n-1} \|.
    \end{equation}
    La relation cherchée s'obtient alors par récurrence. Soient \( q>p\). En utilisant une somme télescopique,
    \begin{subequations}
        \begin{align}
            \| x_q-x_p \|&\leq \sum_{l=p}^{q-1}\| x_{l+1}-x_l \|\\
            &\leq\left( \sum_{l=p}^{q-1}k^l \right)\| x_1-x_0 \|\\
            &\leq\left(\sum_{l=p}^{\infty}k^l\right)\| x_1-x_0 \|.
        \end{align}
    \end{subequations}
    Étant donné que \( k<1\), la parenthèse est la queue d'une série qui converge, et donc tend vers zéro lorsque \( p\) tend vers l'infini.

    En ce qui concerne les inégalités \eqref{EqKErdim}, nous refaisons une somme télescopique :
    \begin{subequations}
        \begin{align}
            \| x_{n+p}-x_n \|&\leq \| x_{n+p}-x_{n+p-1} \|+\ldots +\| x_{n+1}-x_n \|\\
            &\leq k^p\| x_n-x_{n-1} \|+k^{p-1}\| x_n-x_{n-1} \|+\ldots +k\| x_n-x_{n-1} \|\\
            &=k(1+\ldots +k^{p-1})\| x_n-x_{n-1}\|  \\
            &\leq \frac{ k }{ 1-k }\| x_n-x_{n-1} \|.
        \end{align}
    \end{subequations}
    En prenant la limite \( p\to \infty\) nous trouvons
    \begin{equation}        \label{EqlUMVGW}
        \| \xi-x_n \|\leq \frac{ k }{ 1-k }\| x_n-x_{n-1} \|\leq \frac{ k }{ 1-k }\| x_1-x_0 \|.
    \end{equation}

    Nous passons maintenant à la seconde partie du théorème en supposant que \( f\) se restreigne en une fonction \( f\colon K\to K\). D'abord \( K\) est encore un espace métrique complet, donc la première partie du théorème s'y applique et \( f\) y a un unique point fixe.
    
    Nous allons montrer la relation par récurrence. Tout d'abord pour \( n=1\) nous avons
    \begin{equation}
        \| v_1-\xi \|\leq\| v_1-u_1 \|+\| u_1-\xi \|\leq \epsilon+\frac{ k }{ 1-k }\| u_1-u_0 \|
    \end{equation}
    où nous avons utilisé l'estimation \eqref{EqlUMVGW}, qui reste valable en remplaçant \( x_1\) par \( u_1\)\footnote{Elle n'est cependant pas spécialement valable si on remplace \( x_n\) par \( u_n\).}. Nous pouvons maintenant faire la récurrence :
    \begin{subequations}
        \begin{align}
            \| v_{n+1}-\xi \|&\leq \| v_{n+1}-u_{n+1} \|+\| u_{n+1}-\xi \|\\
            &\leq \epsilon+k\| v_n-\xi \|\\
            &\leq \epsilon+k\left( \frac{ k^n }{ 1-k }\| u_1-u_0 \|+\frac{ \epsilon }{ 1-k } \right)\\
            &=\frac{ \epsilon }{ 1-k }+\frac{ k^{n+1} }{ 1-k }\| u_1-u_0 \|.
        \end{align}
    \end{subequations}
\end{proof}

\begin{remark}
    Ce théorème comporte deux parties d'intérêts différents. La première partie est un théorème de point fixe usuel, qui sera utilisé pour prouver l'existence de certaines équations différentielles.

    La seconde partie est intéressante d'un point de vie numérique. En effet, ce qu'elle nous enseigne est que si à chaque pas de calcul de la récurrence \( x_{n+1}=f(x_n)\) nous commettons une erreur d'ordre de grandeur \( \epsilon\), alors le procédé (la suite \( (v_n)\)) ne converge plus spécialement vers le point fixe, mais tend vers le point fixe avec une erreur majorée par \( \epsilon/(k-1)\).
\end{remark}

\begin{remark}
Au final l'erreur minimale qu'on peut atteindre est de l'ordre de \( \epsilon\). Évidemment si on commet une faute de calcul de l'ordre de \( \epsilon\) à chaque pas, on ne peut pas espérer mieux.
\end{remark}

\begin{remark}  \label{remIOHUJm}
    Si \( f\) elle-même n'est pas contractante, mais si \( f^p\) est contractante pour un certain \( p\in \eN\) alors la conclusion du théorème de Picard reste valide et \( f\) a le même unique point fixe que \( f^p\). En effet nommons \( x\) le point fixe de \( f\) : \( f^p(x)=x\). Nous avons alors
    \begin{equation}
        f^p\big( f(x) \big)=f\big( f^p(x) \big)=f(x),
    \end{equation}
    ce qui prouve que \( f(x)\) est un point fixe de \( f^p\). Par unicité nous avons alors \( f(x)=x\), c'est à dire que \( x\) est également un point fixe de \( f\).
\end{remark}

Si la fonction n'est pas Lipschitz mais presque, nous avons une variante.
\begin{proposition}
    Soit \( E\) un ensemble compact\footnote{Notez cette hypothèse plus forte} et si \( f\colon E\to E\) est une fonction telle que
    \begin{equation}        \label{EqLJRVvN}
        \| f(x)-f(y) \|< \| x-y \|
    \end{equation}
    pour tout \( x\neq y\) dans \( E\) alors \( f\) possède un unique point fixe.
\end{proposition}

\begin{proof}
    La suite \( x_{n+1}=f(x_n)\) possède une sous suite convergente. La limite de cette sous suite est un point fixe de \( f\) parce que \( f\) est continue. L'unicité est due à l'aspect strict de l'inégalité \eqref{EqLJRVvN}.
\end{proof}

%---------------------------------------------------------------------------------------------------------------------------
\subsection{Théorème de Cauchy-Lipschitz}
%---------------------------------------------------------------------------------------------------------------------------

\begin{definition}
    Une fonction 
    \begin{equation}
        \begin{aligned}
            f\colon \eR^n\times R^m&\to \eR^p \\
            (t,y)&\mapsto f(t,y) 
        \end{aligned}
    \end{equation}
    est \defe{localement Lipschitz}{Lipschitz!localement} en \( y\) au point \( (t_0,y_0)\) si il existe des voisinages \( V\) de \( t_0\) et \( W\) de \( y_0\) et un nombre \( k>0\) tels que pour tout \( (t,y)\in V\times W\) on ait
    \begin{equation}
        \big\| f(t_0,y_0)-f(t,y) \big\|\leq k\| y-y_0 \|.
    \end{equation}
    La fonction est localement Lipschitz sur un ouvert \( U\) de \( \eR^n\times \eR^m\) si elle est localement Lipschitz en chaque point de \( U\).
\end{definition}

\begin{proposition}[Limite uniforme de fonctions continues]\label{PropCZslHBx}
    Soit \( X\) un espace topologique et \( (Y,d)\) un espace métrique. Si une suite de fonctions \( f_n\colon X\to Y\) continues converge uniformément, alors la limite est séquentiellement continue\footnote{Si \( X\) est métrique, alors c'est la continuité usuelle par la proposition \ref{PropFnContParSuite}.}.
\end{proposition}

\begin{proof}
    Soit \( a\in X\) et prouvons que \( f\) est séquentiellement continue en \( a\). Pour cela nous considérons une suite \( x_n\to a\) dans \( X\). Nous savons que \( f(x_n)\stackrel{Y}{\longrightarrow}f(x)\). Pour tout \(k\in \eN\), tout \( n\in \eN\) et tout \( x\in X\) nous avons la majoration
    \begin{equation}
        \big\| f(x_n)-f(x) \big\|\leq \big\| f(x_n)-f_k(x_n) \big\|+\big\| f_k(x_n)-f_k(x) \big\|+\big\| f_k(x)-f(x) \big\|\leq 2\| f-f_k \|_{\infty}+\big\| f_k(x_n)-f_k(x) \big\|.    
    \end{equation}
    Soit \( \epsilon>0\). Si nous choisissons \( k\) suffisamment grand la premier terme est plus petit que \( \epsilon\). Et par continuité de \( f_k\), en prenant \( n\) assez grand, le dernier terme est également plus petit que \( \epsilon\).
\end{proof}

\begin{proposition} \label{PropSYMEZGU}
    Soit \( X\) un espace topologique métrique \( (Y,d)\) un espace espace métrique complet. Alors l'espace des fonctions continues et bornées \( X\to Y\) muni de la norme uniforme \( \big( C^0_b(X,Y),\| . \|_{\infty} \big)\) est complet.
\end{proposition}
\index{espace!complet!\( C^0_b(X,Y)\),norme uniforme}

\begin{proof}
    Soit \( (f_n)\) une suite de Cauchy dans \( C(X,Y)\), c'est à dire que pour tout \( \epsilon>0\) il existe \( N\in \eN\) tel que si \( k,l>N\) nous avons \( \| f_k-f_l \|_{\infty}\leq \epsilon\). Cette suite vérifie le critère de Cauchy uniforme \ref{PropNTEynwq} et donc converge uniformément vers une fonction \( f\colon X\to Y\). La continuité de la fonction \( f\) découle de la convergence uniforme et de la proposition \ref{PropCZslHBx} (c'est pour avoir l'équivalence entre la continuité séquentielle et la continuité normale que nous avons pris l'hypothèse d'espace métrique).
\end{proof}
    Notons que si \( X\) est compact, les fonctions continues sont bornées par le théorème \ref{ThoImCompCotComp} et nous pouvons simplement dire que \( C^0(X,Y)\) est complet, sans préciser que nous parlons des fonctions bornées.


\begin{lemma}       \label{LemdLKKnd}
    Soient \( A\) et \( B\) deux espaces compact. L'ensemble des fonctions continues de \( A\) vers \( B\) muni de la norme uniforme est complet.
\end{lemma}
\index{espace!complet!\(\big( C(A,B),\| . \|_{\infty} \big)\)}
% TODO : revoir cette preuve à la lumière du critère de Cauchy uniforme \ref{PropNTEynwq}.

\begin{proof}
    Soit \( (f_k)\) une suite de Cauchy de fonctions dans \( C(A,B)\). Pour chaque \( x\in A \) nous avons
    \begin{equation}
        \| f_k(x)-f_l(x) \|_B\leq \| f_k-f_l \|_{\infty},
    \end{equation}
    de telle sorte que la suite \( (f_k(x))\) est de Cauchy dans \( B\) et converge donc vers un élément de \( B\). La suite de Cauchy \( (f_k)\) converge donc ponctuellement vers une fonction \( f\colon A\to B\). Nous devons encore voir que cette fonction est continue; ce sera l'uniformité de la norme qui donnera la continuité. En effet soit \( x_n\to x\) une suite dans \( A\) convergent vers \( x\in A\). Pour chaque \( k\in \eN\) nous avons
    \begin{equation}
        \| f(x_n)-f(x) \|\leq \| f(x_n)-f_k(x_n) \|  +\| f_k(x_n)-f_k(x) \|+\| f_k(x)-f(x) \|.
    \end{equation}
    En prenant \( k\) et \( n\) assez grands, cette expression peut être rendue aussi petite que l'on veut; le premier et le troisième terme par convergence ponctuelle \( f_k\to f\), le second terme par continuité de \( f_k\). La suite \( f(x_n)\) est donc convergente vers \( f(x)\) et la fonction \( f\) est continue.
\end{proof}

\begin{theorem}[Cauchy-Lipschitz\cite{SandrineCL}]\index{théorème!Cauchy-Lipschitz}\label{ThokUUlgU}
    Nous considérons l'équation différentielle
    \begin{subequations}        \label{XtiXON}
        \begin{numcases}{}
            y'=f(t,y)\\
            y(t_0)=y_0
        \end{numcases}
    \end{subequations}
    avec \( f\colon U\to \eR^n\) où \( U\) est un ouvert de \( \eR\times \eR^n\). Nous supposons que \( f\) est continue sur \( U\) et localement Lipschitz\footnote{Nous ne supposons pas que \( f\) soit une contraction.} par rapport à \( y\). Alors le système \eqref{XtiXON} admet une unique solution maximale. Cette solution est \( C^1\). 
\end{theorem}

\begin{remark}
    L'écriture «\( y'=f(t,y)\)» est un abus de notation pour demander que pour chaque \( t\) nous ayons \( y'(t)=f\big(t,y(t)\big)\).
\end{remark}

\begin{proof}
    Si \( y\) est une solution de l'équation différentielle considérée, elle vérifie
    \begin{equation}        \label{EqPGLwcL}
        y(t)=y_0+\int_{t_0}^tf\big( u,y(u) \big)du.
    \end{equation}
    Ceci nous incite à considérer l'opérateur \( \Phi\colon \mF\to \mF\) défini par
    \begin{equation}
        \Phi(y)(t)=y_0+\int_{t_0}^tf\big( u,y(u) \big)du.
    \end{equation}

    \begin{subproof}
    \item[Cylindre de sécurité et espace fonctionnel]

    Précisons l'espace fonctionnel \( \mF\) adéquat. Soient \( V\) et \( W\) les voisinages de \( t_0\) et \( y_0\) sur lesquels \( f\) est localement Lipschitz. Nous considérons les quantités suivantes :
    \begin{enumerate}
        \item
            \( M=\sup_{V\times W}f\) ;
        \item
            \( r>0\) tel que \( \overline{ B(y_0,r) }\subset V\)
        \item
            \( T>0\) tel que \( \overline{ B(t_0,T) }\subset W\) et \( T<r/M\).
    \end{enumerate}
    Nous considérons alors \( \mF\), l'ensemble des fonctions continues \( \overline{ B(t_0,T) }\to \overline{ B(y_0,r) }\) muni de la norme uniforme. Par le lemme \ref{LemdLKKnd} l'espace \( \mF\) est complet.

    Le fait que \( \Phi(y)\) soit continue lorsque \( y\) est continue est une propriété de l'intégration et du fait que \( f\) soit continue en ses deux variables. Prouvons que \( \Phi(y)(t)\in\overline{ B(y_0,r) }\). Pour cela, notons que
    \begin{equation}
        | \Phi(y)(t)-y_0 |\leq \int_{t_0}^t |f\big( u,y(u) \big)|du\leq | t-t_0 |\| f \|_{\infty}.
    \end{equation}
    Étant donné que \( t\in\overline{ B(t_0,T) }\) nous avons \( | t-t_0 |\leq r/M\) et donc \( | \Phi(y)(t)-y_0 |\leq r\).

    L'équation \eqref{EqPGLwcL} signifie que \( y\) est un point fixe de \( \Phi\). L'espace \( \mF\) étant complet le théorème de point fixe de Picard (théorème \ref{ThoEPVkCL}) s'applique. Nous allons montrer qu'il existe un \( p\in\eN\) tel que \( \Phi^p\) soit contractante. Par conséquent \( \Phi^p\) aura un unique point fixe qui sera également unique point fixe de \( \Phi\) par la remarque \ref{remIOHUJm}.
    
\item[Une contraction]

    Prouvons donc que \( \Phi^p\) est contractante pour un certain \( p\). Pour cela nous commençons par montrer la formule suivante par récurrence :
    \begin{equation}        \label{EqRAdKxT}
        \big\| \Phi^p(x)(t)-\Phi^p(y)(t) \big\|\leq \frac{ k^p| t-t_0 |^p }{ p! }\| x-y \|_{\infty}
    \end{equation}
    pour tout \( x,y\in\mF\), et pour tout \( t\in\overline{ B(t_0,T) }\). Pour \( p=0\) la formule \eqref{EqRAdKxT} est vérifiée parce que \( \| x-y \|_{\infty}\) est le supremum de \( \| x(t)-y(t) \|\) pour \( t\in\overline{ B(t_0,T) }\). Supposons que la formule soit vraie pour \( p\) et calculons pour \( p+1\). Pour tout \( t\in\overline{ B(t_0,T) }\) nous avons
    \begin{subequations}
        \begin{align}
            \big\| \Phi^{p+1}(x)(t)-\Phi^{p+1}(y)(t) \big\|&\leq \left| \int_{t_0}^t\big\| f\big( u,\Phi^p(x)(u) \big)-f\big( u,\Phi^p(y)(u) \big) \big\|du \right| \\
            &\leq \left| \int_{t_0}^tk\| \Phi^p(x)(u)-\Phi^p(y)(u) \|du \right|    \label{subIKYixF}\\
            &\leq \left| \int_{t_0}^tk\frac{ k^p| t-t_0 | }{ p! }\| x-y \|_{\infty} \right| \label{subxkNjiV} \\
            &=\frac{ k^{p+1}| t-t_0 |^{p+1} }{ (p+1)! }\| x-y \|_{\infty}.
        \end{align}
    \end{subequations}
    Justifications :
    \begin{itemize}
        \item \eqref{subIKYixF} parce que \( f\) est Lipschitz.
        \item \eqref{subxkNjiV} par hypothèse de récurrence.
    \end{itemize}
    La formule \eqref{EqRAdKxT} est maintenant établie. Nous pouvons maintenant montrer que \( \Phi^p\) est une contraction pour un certain \( p\). Pour tout \( t\in \overline{ B(t_0,T) }\) nous avons
    \begin{equation}
         \| \Phi^p(x)(t)-\Phi^p(y)(t) \|\leq \frac{ k^p }{ t! }| t-t_0 |^p\| x-y \|_{\infty}     \leq \frac{ k^pT^p }{ p! }\| x-y \|_{\infty}
    \end{equation}
    où nous avons utilisé le fait que \( | t-t_0 |^p<T^p\). En prenant le supremum sur \( t\) des deux côtés il vient
    \begin{equation}
        \| \Phi^p(x)-\Phi^p(y) \|_{\infty}\leq\frac{ k^pT^p }{ p! }\| x-y \|_{\infty}.
    \end{equation}
    Le membre de droite tend vers zéro lorsque \( p\to\infty\) parce que \( k<1\) et \( T^p/p!\to 0\)\footnote{C'est le terme général du développement de \(  e^{T}\) qui est une série convergente.}. Nous concluons donc que \( \Phi^p\) est une contraction pour un certain \( p\).

\item[Conclusion]

    L'unique point fixe de \( \Phi\) est alors l'unique solution continue de l'équation différentielle \eqref{XtiXON}. Par ailleurs l'équation elle-même \( y'=f(t,y)\) demande implicitement que \( y\) soit dérivable et donc continue. Nous concluons que l'unique point fixe de \( \Phi\) est l'unique solution de l'équation différentielle donnée. Cette dernière est automatiquement \( C^1\) parce que si \( y\) est continue alors \( u\mapsto f(u,y(u))\) est continue, c'est à dire que \( y'\) est continue.

\item[Unicité]

    Nous passons maintenant à la partie «prolongement maximum» du théorème. Soient \( x_1\) et \( x_2\) deux solutions maximales du problème \eqref{XtiXON} sur des intervalles \( I_1\) et \( I_2\) respectivement. Les intervalles \( I_1\) et \( I_2\) contiennent \( \overline{ B(t_0,r) }\) sur lequel \( x_1=x_2\) par unicité.
    
    
    Nous allons maintenant montrer que pour tout \( t\geq t_0\) pour lequel \( x_1\) ou \( x_2\) est défini, \( x_1(t)\) et \( x_2(t)\) sont définis et sont égaux. Le raisonnement sur \( t\leq t_0\) est similaire.
    
    Supposons que l'ensemble des \( t\geq t_0\) tels que \( x_1=x_2\) soit ouvert à droite, c'est à dire soit de la forme \( \mathopen[ t_0 ,b [\). Dans ce cas, soit \( x_1\) soit \( x_2\) (soit les deux) cesse d'exister en \( b\). En effet si nous avions les fonctions \( x_i\) sur \(\mathopen[ t_0 , b+\epsilon [\) alors l'équation \( x_1=x_2\) définirait un fermé dans \( \mathopen[ t_0 , b+\epsilon [\). Supposons pour fixer les idées que \( x_1\) cesse d'exister : le domaine de \( x_1\) (parmi les \( t\geq 0\)) est \( \mathopen[ t_0 , b [\) et sur ce domaine nous avons \( x_1=x_2\). Dans ce cas \( x_1\) pourrait être prolongé en \( x_2\) au-delà de \( b\). Si \( x_1\) et \( x_2\) s'arrêtent d'exister en même temps en \( b\), alors nous avons bien \( x_1=x_2\).

    Nous devons donc traiter le cas où \( x_1=x_2\) sur \( \mathopen[ t_0 , b \mathclose]\) alors que \( x_1\) et \( x_2\) existent sur \( \mathopen[ t_0 , b+\epsilon [\) pour un certain \( \epsilon\).

    Nous pouvons appliquer le théorème d'existence locale au problème
    \begin{subequations}
        \begin{numcases}{}
            y'=f(t,y)\\
            y(b)=x_1(b).
        \end{numcases}
    \end{subequations}
    Il existe un voisinage de \( b\) sur lequel la solution est unique. Sur ce voisinage nous devons donc avoir \( x_1=x_2\), ce qui contredit le fait que \( x_1\neq x_2\) en dehors de \( \mathopen[ t_0 , b \mathclose]\).
    \end{subproof}
\end{proof}

%---------------------------------------------------------------------------------------------------------------------------
\subsection{Équation de Fredholm}
%---------------------------------------------------------------------------------------------------------------------------

\begin{theorem}[Équation de Fredholm]\index{Fredholm!équation}\index{équation!Fredholm}     \label{ThoagJPZJ}
    Soit \( K\colon \mathopen[ a , b \mathclose]\times \mathopen[ a , b \mathclose]\to \eR\) et \( \varphi\colon \mathopen[ a , b \mathclose]\to \eR\), deux fonctions continues. Alors si \( \lambda\) est suffisamment petit, l'équation
    \begin{equation}
        f(x)=\lambda\int_a^bK(x,y)f(y)dy+\varphi(x)
    \end{equation}
    admet une unique solution qui sera de plus continue sur \( \mathopen[ a , b \mathclose]\).
\end{theorem}

\begin{proof}
    Nous considérons l'ensemble \( \mF\) des fonctions continues \( \mathopen[ a , b \mathclose]\to\mathopen[ a , b \mathclose]\) muni de la norme uniforme. Le lemme \ref{LemdLKKnd} implique que \( \mF\) est complet. Nous considérons l'application \( \Phi\colon \mF\to \mF\) donnée par
    \begin{equation}
        \Phi(f)(x)=\lambda\int_a^bK(x,y)f(y)dy+\varphi(x). 
    \end{equation}
    Nous montrons que \( \Phi^p\) est une application contractante pour un certain \( p\). Pour tout \( x\in \mathopen[ a , b \mathclose]\) nous avons
    \begin{subequations}
        \begin{align}
            \| \Phi(f)-\Phi(g) \|_{\infty}&\leq \| \Phi(f)(x)-\Phi(g)(x) \|\\
            &=| \lambda |\Big\| \int_a^bK(x,y)\big( f(y)-g(y) \big)dy  \Big\|\\
            &\leq | \lambda |\| K \|_{\infty}| b-a |\| f-g \|_{\infty}
        \end{align}
    \end{subequations}
    Si \( \lambda\) est assez petit, et si \( p\) est assez grand, l'application \( \Phi^p\) est donc une contraction. Elle possède donc un unique point fixe par le théorème de Picard \ref{ThoEPVkCL}.
\end{proof}
