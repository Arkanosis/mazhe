% This is part of Mes notes de mathématique
% Copyright (c) 2011-2014
%   Laurent Claessens
% See the file fdl-1.3.txt for copying conditions.

%+++++++++++++++++++++++++++++++++++++++++++++++++++++++++++++++++++++++++++++++++++++++++++++++++++++++++++++++++++++++++++ 
\section{Calcul différentiel dans un espace de Banach}
%+++++++++++++++++++++++++++++++++++++++++++++++++++++++++++++++++++++++++++++++++++++++++++++++++++++++++++++++++++++++++++
\label{SecLStKEmc}

Nous développons dans cette section le concept de différentielle de fonction de et vers des espaces de Banach (ou plus généralement des espaces vectoriels normés) au lieu de \( \eR^n\).

%--------------------------------------------------------------------------------------------------------------------------- 
\subsection{Différentielle}
%---------------------------------------------------------------------------------------------------------------------------

\begin{definition}  \label{DefKZXtcIT}
    Soit une application \( f\colon E\to F\) entre deux espaces de Banach. Nous disons que \( f\) est \defe{différentiable}{différentiable!dans un Banach} en \( a\in E\) si il existe une application linéaire continue\footnote{Nous demandons bien que le candidat différentielle soit continue; en dimension infinie ce n'est pas le cas de toutes les fonctions linéaires, comme le montre l'exemple \ref{ExHKsIelG}.} \( T\colon E\to F\) telle que
    \begin{equation}\label{EqIQuRGmO}
        \lim_{h\to 0} \frac{ f(a+h)-f(a)-T(h) }{ \| h \| }=0.
    \end{equation}
\end{definition}

%--------------------------------------------------------------------------------------------------------------------------- 
\subsection{(non) différentiabilité des applications linéaires}
%---------------------------------------------------------------------------------------------------------------------------

Si \( E\) et \( F\) sont deux espaces vectoriels nous notons \( \aL(E,F)\)\nomenclature[Y]{\( \aL(E,F)\)}{Les applications linéaires de \( E\) vers \( F\)} l'ensemble des applications linéaires de \( E\) vers \( F\) et \( \cL\)\nomenclature[Y]{\( \cL\)}{Les applications linéaires continues de \( E\) vers \( F\)} l'ensemble des applications linéaires continues de \( E\) vers \( F\).

\begin{example}[Une application linéaire non continue]  \label{ExHKsIelG}
    Si \( \{ e_k \}_{k\in \eN}\) est une base d'un espace vectoriel normé \( V\) alors l'application linéaire \( f\colon V\to v\) donnée par \( f(e_k)=ke_k\) n'est pas continue. En effet soit \( r>0\); dès que \( rk>1\) nous avons \( f(re_k)\notin B(0,1)\) et donc \( B(0,r)\) n'est pas inclue dans \( f^{-1}\big( B(0,1) \big)\). Par conséquent il n'existe pas de voisinages de \( 0\) à être inclu à \( f^{-1}\big( B(0,1) \big)\) ce qui prouve que ce dernier n'est pas ouvert alors qu'il est image d'un ouvert par \( f^{-1}\).
\end{example}

Il est immédiat de voir que si \( f\) est linéaire et différentiable alors \( df_a(u)=f(u)\). En effet la linéarité de \( f\) donne
\begin{equation}
    f(a+h)-f(a)-f(h)=0
\end{equation}
pour tout \( h\). Donc la limite \eqref{EqIQuRGmO} est nulle. Les applications linéaires non continues ne sont donc pas différentiables.

\begin{lemma}
    Une application linéaire continue est de classe \(  C^{\infty}\).
\end{lemma}

\begin{proof}
    Soit \( a\in E\). Étant donné que \( f\) est linéaire et continue, elle est différentiable et \( df_a=f\). Vu que \( f\) est continue, \( df\) est continue et nous avons \( f\in C^1\). Pour la différentielle seconde,
    \begin{equation}
        \begin{aligned}
            df\colon E&\to \cL(E,F) \\
            a&\mapsto f 
        \end{aligned}
    \end{equation}
    est une fonction constante. Donc \( f\) est de classe \( C^2\) et \( f(df)_a=0\) parce que \( df(a+h)-df(a)=f-f=0\). Toutes les différentielles suivantes sont nulles.
\end{proof}

%--------------------------------------------------------------------------------------------------------------------------- 
\subsection{Dérivation en chaine et Leibnitz}
%---------------------------------------------------------------------------------------------------------------------------

\begin{proposition} \label{PropDQLhSoy}
    Soit \( E\) un espace vectoriel normé. Soient \( a<b\) dans \( \eR\) et deux fonctions
    \begin{subequations}
        \begin{align}
            f\colon \mathopen[ a , b \mathclose]\to E\\
            g\colon \mathopen[ a , b \mathclose]\to \eR
        \end{align}
    \end{subequations}
    continues sur \( \mathopen[ a , b \mathclose]\) et dérivables sur \( \mathopen] a , b \mathclose[\). Si pour tout \( t\in\mathopen] a , b \mathclose[\) nous avons \( \| f'(t) \|\leq g'(t)\) alors
        \begin{equation}
            \| f(b)-f(a) \|\leq g(b)-g(a).
        \end{equation}
\end{proposition}

\begin{proof}
    Soit \( \epsilon>0\) et la fonction
    \begin{equation}
        \begin{aligned}
            \varphi_{\epsilon}\colon \mathopen[ a , b \mathclose]&\to \eR \\
            t&\mapsto \| f(t)-f(a) \|-g(t)-\epsilon t. 
        \end{aligned}
    \end{equation}
    Cela est une fonction continue réelle à variable réelle. En particulier pour tout \( u\in\mathopen] a , b \mathclose[\) la fonction \( \varphi_{\epsilon}\) est continue sur le compact \( \mathopen[ u , b \mathclose]\) et donc y atteint son minimum en un certain point \( c\in\mathopen[ u , b \mathclose]\); c'est le bon vieux théorème de Weierstrass \ref{ThoWeirstrassRn}. Nous commençons par montrer que pour tout \( u\), ledit minimum ne peut être que \( b\). Pour cela nous allons montrer que si \( t\in\mathopen[ u , b [\), alors \( \varphi_{\epsilon}(s)<\varphi_{\epsilon}(t)\) pour un certain \( s>t\). Par continuité si \( s\) est proche de \( t\) nous avons
        \begin{equation}
            \left\|  \frac{ f(s)-f(t) }{ s-t }  \right\|-\frac{ \epsilon }{2}<\| f'(t) \|<g'(t)+\frac{ \epsilon }{2}=\frac{ g(s)-g(t) }{ s-t }+\frac{ \epsilon }{2}.
        \end{equation}
        Ces inégalités proviennent de la limite
        \begin{equation}
            \lim_{s\to t} \frac{ f(s)-f(t) }{ s-t }=f'(t),
        \end{equation}
        donc si \( s\) et \( t\) sont proches,
        \begin{equation}
            \left\| \frac{ f(s)-f(t) }{ s-t }-f'(t) \right\|
        \end{equation}
        est petit. Si \( s>t\) nous pouvons oublier des valeurs absolues et transformer l'inégalité en
        \begin{equation}
            \| f(s)-f(t) \|<g(s)-g(t)+\epsilon(s-t).
        \end{equation}
        Utilisant cela et l'inégalité triangulaire,
        \begin{subequations}
            \begin{align}
                \varphi_{\epsilon}(s)&\leq\| f(s)-f(t) \|+\| f(t)-f(a) \|-g(s)-\epsilon s\\
                &\leq g(s)-g(t)+\epsilon s-\epsilon t+\| f(t)-f(a) \|-g(s)-\epsilon s\\
                &=\varphi_{\epsilon}(t).
            \end{align}
        \end{subequations}
        Donc nous avons bien \( \varphi_{\epsilon}(s)<\varphi_{\epsilon}(t)\) avec l'inégalité stricte. Par conséquent pour tout \( u\in\mathopen] a , b \mathclose[\) nous avons \( \varphi_{\epsilon}(b)<\varphi_{\epsilon}(u)\) et en prenant la limite \( u\to a\) nous avons
        \begin{equation}
            \varphi_{\epsilon}(b)\leq \varphi_{\epsilon}(a).
        \end{equation}
        Cette inégalité donne immédiatement
        \begin{equation}
            \| f(b)-f(a) \|\leq g(b)-g(a)+\epsilon(b-a)
        \end{equation}
         pour tout \( \epsilon>0\) et donc
         \begin{equation}
            \| f(b)-f(a) \|\leq g(b)-g(a).
         \end{equation}
\end{proof}

\begin{proposition}
    Soient \( E\) et \( F\) des espaces vectoriels normés, \( U \) ouvert dans \( E\) et une application différentiable \( f\colon U\to F\). Pour tout segment \( \mathopen[ a , b \mathclose]\subset U\) nous avons
    \begin{equation}
        \| f(b)-f(a) \|\leq\left( \sup_{x\in\mathopen[ a , b \mathclose]}\| df_x \| \right)\| b-a \|.
    \end{equation}
\end{proposition}

\begin{proof}
    Nous prenons les applications
    \begin{equation}
        \begin{aligned}
            k\colon \mathopen[ 0 , 1 \mathclose]&\to E \\
            t&\mapsto f\big( (1-t)a+tb \big) 
        \end{aligned}
    \end{equation}
    et
    \begin{equation}
        \begin{aligned}
            g\colon \mathopen[ 0 , 1 \mathclose]&\to \eR \\
            t&\mapsto t\sup_{x\in\mathopen[ a , b \mathclose]}\| df_x \|\| b-a \|.
        \end{aligned}
    \end{equation}
    Pour tout \( t\) nous avons \( g'(t)=M\| b-a \|\) où il n'est besoin de dire ce qu'est \( M\). D'un autre côté nous avons aussi
    \begin{equation}
        \begin{aligned}[]
            k'(t)&=\lim_{\epsilon\to 0}\frac{ f\big( (1-t-\epsilon)a+(t+\epsilon)b \big)-f\big( (1-t)a+tb \big) }{ \epsilon }\\
            &=\Dsdd{ f\big( (1-t)a+tb+\epsilon(b-a) \big)  }{\epsilon}{0}\\
            &=df_{(1-t)a+tb}(b-a)
        \end{aligned}
    \end{equation}
    où nous avons utilisé l'hypothèse de différentiabilité de \( f\) sur \( \mathopen[ a , b \mathclose]\) et donc en \( (1-t)a+tb\). Nous avons donc
    \begin{equation}
        \| k'(t) \|\leq \| b-a \|\| df_{(1-t)a+tb} \|\leq M\| b-a \|=g'(t)
    \end{equation}
    La proposition \ref{PropDQLhSoy} est donc utilisable et
    \begin{equation}
        \| k(1)-k(0) \|=g(1)-g(0),
    \end{equation}
    c'est à dire
    \begin{equation}
        \| f(b)-f(a) \|=M\| b-a \|
    \end{equation}
    comme il se doit.
\end{proof}

\begin{proposition} \label{ProFSjmBAt}
    Soient \( E\) et \( F\) des espaces vectoriels normés, \( U \) ouvert dans \( E\) et une application \( f\colon U\to F\). Soient \( a,b\in U\) tels que \( \mathopen[ a , b \mathclose]\subset U\). Nous posons \( u=(b-a)/\| b-a \|\) et nous supposons que pour tout \( x\in\mathopen[ a , b \mathclose]\), la dérivée directionnelle
    \begin{equation}
        \frac{ \partial f }{ \partial u }(x)=\Dsdd{ f(x+tu) }{t}{0}
    \end{equation}
    existe. Nous supposons de plus que \( \frac{ \partial f }{ \partial u }(x)\) est continue en \( x=a\). Alors
    \begin{equation}
        \| f(b)-f(a) \|\leq\left( \sup_{x\in\mathopen[ a , b \mathclose]}\| \frac{ \partial f }{ \partial u }(x) \| \right)\| b-a \|.
    \end{equation}
\end{proposition}

\begin{proof}
    Nous posons évidemment 
    \begin{equation}
        M=\sup_{x\in\mathopen[ a , b \mathclose]}\| \frac{ \partial f }{ \partial u }(x) \| 
    \end{equation}
    et nous considérons les fonctions
    \begin{equation}
        k(t)=f\big( (1-t)a+tb \big)
    \end{equation}
    et
    \begin{equation}
        g(t)=tM\| b-a \|.
    \end{equation}
    Pour alléger les notations nous posons \( x=(1-t)a+tb\) et nous calculons avec un petit changement de variables dans la limite :
    \begin{equation}
        k'(t)=\Dsdd{  f\big( x+\epsilon(b-a) \big)  }{\epsilon}{0}=\| b-a \|\Dsdd{ f\big( x+\frac{ \epsilon }{ \| b-a \| }(b-a) \big) }{\epsilon}{0}=\| b-a \|\frac{ \partial f }{ \partial u }(x),
    \end{equation}
    et donc encore une fois nous avons
    \begin{equation}
        \| k'(t) \|\leq g'(t),
    \end{equation}
    ce qui donne
    \begin{equation}
        \| k(1)-k(0) \|=g(1)-g(0),
    \end{equation}
    c'est à dire
    \begin{equation}
        \| f(b)-f(a) \|\leq \sup_{x\in\mathopen[ a , b \mathclose]}\| \frac{ \partial f }{ \partial u }(x) \|\| b-a \|.
    \end{equation}
\end{proof}

\begin{theorem} \label{ThoOYwdeVt}
    Soient \( E,V\) deux espaces vectoriels normés, une application \( f\colon E\to V\), un point \( a\in E\) tel que pour tout \( u\in E\), la dérivée
    \begin{equation}
        \Dsdd{ f(x+tu) }{t}{0}
    \end{equation}
    existe pour tout \( x\in B(a,r)\) et est continue (par rapport à \( x\)) en \( x=a\). Nous supposons de plus que\quext{Je ne suis pas certain que cette hypothèse soit nécessaire, voir la question \ref{ItemLPrIWZhPg} de la page \pageref{ItemLPrIWZhPg}.}
    \begin{equation}
        \frac{ \partial f }{ \partial u }(a)=0
    \end{equation}
    pour tout \( u\in E\). Alors \( f\) est différentiable en \( a\) et
    \begin{equation}
        df_a=0
    \end{equation}
\end{theorem}

\begin{proof}
    Soit \( \epsilon>0\). Pourvu que \( \| h \|\) soit assez petit pour que \( a+h\in B(a,r)\), la proposition \ref{ProFSjmBAt} nous donne
    \begin{equation}
        \| f(a+h)-f(a) \|\leq \sup_{x\in\mathopen[ a , a+h \mathclose]}\| \frac{ \partial f }{ \partial u }(x) \|  |h |
    \end{equation}
    où \( u=h/\| h \|\). Par continuité de \( \partial_uf(x)\) en \( x=a\) et par le fait que cela vaut \( 0\) en \( x=a\), il existe un \( \delta>0\) tel que si \( \| h \|<\delta\) alors
    \begin{equation}
        \| \frac{ \partial f }{ \partial u }(a+h) \|\leq \epsilon.
    \end{equation}
    Pour de tels \( h\) nous avons
    \begin{equation}
        \| f(a+h)-f(a) \|\leq \epsilon\| h \|,
    \end{equation}
    ce qui prouve que l'application linéaire \( T(u)=0\) convient parfaitement pour faire fonctionner la définition \ref{DefKZXtcIT}.
%
%    Nous ne supposons plus que les dérivées directionnelles de \( f\) sont nulles en \( x=a\). Alors nous posons, pour \( x\in U\),
%    \begin{equation}    \label{EqCUgHXHy}
%        g(x)=f(x)-\Dsdd{ f(a+s(x-a)) }{s}{0}.
%    \end{equation}
%    Le fait que cette fonction soit bien définie est encore un coup de hypothèses sur les dérivées directionnelles de \( f\) qui sont bien définies autour de \( a\). Cette nouvelle fonction \( g\) satisfait à \( \frac{ \partial g }{ \partial v }(a)=0\) pour tout \( v\in E\) parce que
%    \begin{subequations}
%        \begin{align}
%            \frac{ \partial g }{ \partial v }(a)&=\Dsdd{ g(a+tv) }{t}{0}\\
%            &=\Dsdd{ f(a+tv)-\Dsdd{ f\big( a+s(tv) \big) }{s}{0} }{t}{0}\\
%            &=\frac{ \partial f }{ \partial v }(a)-\Dsdd{ t\frac{ \partial f }{ \partial v }(a) }{t}{0}\\
%            &=0.
%        \end{align}
%    \end{subequations}
%    Pour la dérivée par rapport à \( s\) nous avons effectué le changement de variables \( s\to ts\), ce qui explique la présence d'un \( t\) en facteur. La fonction \( g\) est donc différentiable en \( a\).
%
%
% Position 229262367
    % Attention : ce qui suit est faux. Mais il y a peut-être moyen d'adapter.
%\item[Dérivées non nulles]
%
%    Nous allons montrer que la fonction 
%    \begin{equation}
%        l(x)=\Dsdd{ f\big( a+s(x-a) \big) }{t}{0}
%    \end{equation}
%    est différentiable en \( x=a\), de différentielle \( T(u)=l(u+a)\). Cela fournira la différentiabilité de \( f\) parce que \eqref{EqCUgHXHy} donnerait alors \( f\) comme somme de deux fonctions différentiables.
%
%    En premier lieu nous devons montrer que \( T\) ainsi définie est linéaire.
%    
%    Notre but est donc de prouver que
%    \begin{equation}
%        \lim_{h \to 0}\frac{ \| l(x+h)-l(x)-l(h) \| }{ \| h \| }=0.
%    \end{equation}
%    Un premier pas est de calculer
%    \begin{subequations}
%        \begin{align}
%            l(x+h)-l(x)-l(h)&=\lim_{s\to 0}\frac{ f\big( s(x+h) \big)-f(0)-f(sx)+f(0)-f(sh)+f(0) }{ s }\\
%            &=\lim_{s\to 0}\frac{ f\big( s(x+h) \big)-f(sx)-f(sh)+f(0) }{ s }.
%        \end{align}
%    \end{subequations}
%    Ensuite nous étudions le numérateur en utilisant la proposition \ref{ProFSjmBAt}:
%    \begin{subequations}
%        \begin{align}
%            \| f\big( s(x+h) \big)-f(sx)-f(sh)+f(0) \|&\leq  \| f\big( s(x+h) \big)-f(sx)\| + \|f(sh)-f(0) \|  \\
%            &\leq \sup_{z\in\mathopen[ sx , sx+sh \mathclose]}\| \frac{ \partial f }{ \partial h }(z) \|\| sh \|\\
%            &\quad +\sup_{z\in\mathopen[ 0 , sh \mathclose]}\| \frac{ \partial f }{ \partial h }(z) \|\| sh \|.
%        \end{align}
%    \end{subequations}
%    La division par \( s\) se passe bien et nous avons
%    \begin{subequations}
%        \begin{align}
%            \| l(x+h)-l(x)-l(h) \|&\leq \lim_{s\to 0}  \sup_{z\in\mathopen[ sx , sx+sh \mathclose]}\| \frac{ \partial f }{ \partial h }(z) \|\| h \|+ \sup_{z\in\mathopen[ 0 , sh \mathclose]}\| \frac{ \partial f }{ \partial h }(z) \|\| h \|\\
%            &=2\| h \|\| \frac{ \partial f }{ \partial h }(0) \|        \label{SubeqVMMoSDH}\\
%            &=2\| h \|^2\| \frac{ \partial f }{ \partial u }(0) \|
%        \end{align}
%    \end{subequations}
%    où nous avons posé \( u=h/\| h \|\). Pour l'égalité \eqref{SubeqVMMoSDH} nous avons utilisé la continuité de \( \frac{ \partial f }{ \partial h }(z)\) en \( z=0\). Du coup
%    \begin{equation}
%        \lim_{y\to 0} \frac{ \| f(x+h)-f(x)-f(h) \| }{ \| h \| }=\lim_{h\to 0} 2\| h \|\| \frac{ \partial f }{ \partial u }(0) \|=0.
%    \end{equation}
%    Cela prouve que \( l\) est bien différentiable en \( x=0\).
%
%    \end{subproof}
%
\end{proof}


Si \( E\) est un espace de Banach, nous sommes intéressé à l'espace \( \GL(E)\) des endomorphismes inversibles de \( E\) sur \( E\). Cet ensemble est métrique par la formule usuelle
\begin{equation}
    \| T \|=\sup_{\| x \|=1}\| T(x) \|_E.
\end{equation}

\begin{lemma}   \label{LemWVNnKNo}
Si \( \| h \|<1\) alors nous avons la formule
\begin{equation}
    (\mtu+h)^{-1}=\sum_{k=0}^{\infty}(-1)^kh^k.
\end{equation}
\end{lemma}
Ce lemme est aussi la proposition \ref{PropQAjqUNp}.
%TODO : à fusionner.

\begin{proof}
    D'abord la série converge normalement\footnote{Définition \ref{DefQDrDqek}} parce que \( \| h^k \|\leq \| h \|^k\) et que la série des \( \| h \|^k\) est la série géométrique qui converge.

    Montrons ensuite que la limite est bien un inverse de \( (\mtu+h)\) :
    \begin{subequations}
        \begin{align}
            (\mtu+h)\sum_{k=0}^{\infty}(-1)^kh^k&=\sum_{k=0}^{\infty}(-1)^kh^k+\sum_{k=0}^{\infty}(-1)^kh^{k+1}\\
            &=\sum_{k=0}^{\infty}(-1)^kh^k+\sum_{k=1}^{\infty}(-1)^{k-1}h^h\\
            &=\mtu+\sum_{k=1}^{\infty}\Big[ (-1)^kh^k+(-1)^{k-1}h^k \Big]\\
            &=\mtu.
        \end{align}
    \end{subequations}
    Nous avons utilisé l'associativité de la somme, proposition \ref{propnseries_propdebase}.
\end{proof}

\begin{lemma}   \label{LemWWXVSae}
Soit \( F\) un espace de Banach et deux suites \( A_k\to A\) et \( B_k\to B\) dans \( \aL(F,F)\). Alors \( A_k\circ B_k\to A\circ B\) dans \( \aL(F,F)\).
\end{lemma}

\begin{proof}
    Il suffit d'écrire
    \begin{equation}
        \| A_kB_k-AB \|\leq \| A_kB_k-A_kB \|+\| A_kB-AB \|.
    \end{equation}
    Le premier terme tend vers zéro pour \( k\to\infty\) parce que 
    \begin{subequations}
        \begin{align}
            \| A_kB_k-A_kB \|=\| A_k(B_k-B) \|\leq \| A_k \|\| B_k-B \|\to \| A \|\cdot 0=0
        \end{align}
    \end{subequations}
    où nous avons utilisé la propriété fondamentale de la norme opérateur : la proposition \ref{PropEDvSQsA}. Le second terme tend également vers zéro pour la même raison.
\end{proof}

\begin{theorem}[Différentielle de fonctions composées\cite{SNPdukn}]
    Soient \( E\), \( F\) et \( G\) des espaces vectoriels normés, \( U\) ouvert dans \( E\) et \( V\) ouvert dans \( F\). Soient des applications de classe \( C^r\) (\( r\geq 1\))
    \begin{subequations}
        \begin{align}
            f\colon U\to V\\
            g\colon V\to G.
        \end{align}
    \end{subequations}
    Alors l'application \( g\circ f\colon V\to G\) est de classe \( C^r\) et
    \begin{equation}
        f(g\circ f)(u)=dg_{f(u)}\circ df_u.
    \end{equation}
\end{theorem}
%AFAIRE : la preuve.

\begin{proposition}[Règle de Leibnitz\cite{SNPdukn}]
    Soient \( E,F_1,F_2\) des espaces vectoriels normés, \( U\) ouvert dans \( E\) et des applications de classe \( C^r\) (\( r\geq 1\))
    \begin{subequations}
        \begin{align}
            f_1\colon U\to F_1\\
            f_1\colon U\to F_1\\
        \end{align}
    \end{subequations}
    et \( B\in\cL(F_1\times F_2,G)\). Alors l'application
    \begin{equation}
        \begin{aligned}
            \varphi\colon U&\to G \\
            x&\mapsto B\big( f_1(x),f_2(x) \big) 
        \end{aligned}
    \end{equation}
    est de classe \( C^r\) et
    \begin{equation}
        d\varphi_x(u)=\varphi\big( (df_1)_x(u),f_2(x) \big)+\varphi\big( f_1(x),(df_2)_x(u) \big).
    \end{equation}
\end{proposition}

\begin{proof}
    L'application \( \varphi\) est une composée
    \begin{equation}
        \varphi=B\circ(f_1\times f_2)\circ \Delta
    \end{equation}
    avec \( \Delta\colon U\to U\times U\), \( \Delta(x)=(x,x)\).
    %AFAIRE : teminer
\end{proof}
<++>

\begin{proposition}[Inverse dans \( \GL(E)\)\cite{laudenbach2000calcul,SNPdukn}]
    Soient \( E\) et \( F\) des espaces vectoriels normés.
    \begin{enumerate}
        \item
        L'ensemble \( \GL(E)\) est ouvert dans \( \End(E)\).
    \item
        L'application inverse
    \begin{equation}
        \begin{aligned}
        i\colon \GL(E,F)&\to \GL(F,E) \\
        u&\mapsto u^{-1} 
        \end{aligned}
    \end{equation}
    est de classe \( C^{\infty}\) et
    \begin{equation}
        di_{u_0}(h)=-u_0^{-1}\circ h\circ u_0^{-1}
    \end{equation}
    pour tout \( h\in\End(E)\)
    \end{enumerate}
\end{proposition}
\index{différentielle!de $u\mapsto u^{-1}$}

\begin{proof}
Nous supposons que \( \GL(E,F)\) n'est pas vide, sinon ce n'est pas du jeu.
        \begin{subproof}
        \item[Ouvert autour de l'identité]
            
        Nous commençons par prouver que \( B(\mtu,1)\subset \GL(E)\). Pour cela il suffit de remarquer que si \( \| u \|<1\) alors le lemme \ref{LemWVNnKNo} nous donne un inverse de \( (1+u)\) en la personne de \( \sum_{k=0}^{\infty}(-u)^k\).

    \item[Ouvert en général]

        Soit maintenant \( u_0\in\GL(E)\). Si \( \| u \|<\frac{1}{ \| u_0^{-1} \| }\) alors \( \| u_0^{-1}u \|<1\), ce qui signifie que
        \begin{equation}
            \mtu+u_0^{-1}u
        \end{equation}
    est inversible. Mais \( u_0+u=u_0(\mtu+u_0^{-1}u)\), donc \( u_0+u\in\GL(E)\) ce qui signifie que
    \begin{equation}
    B\left( u_0,\frac{1}{ \| u_0^{-1} \| } \right)\subset \GL(E).
    \end{equation}

    \item[Différentielle en l'identité]

    Nous commençons par prouver que \( di_{\mtu}(u)=-u\). Pour cela nous posons 
    \begin{equation}
        \alpha(h)=\sum_{k=2}^{\infty}(-1)^kh^k
    \end{equation}
    et nous calculons
    \begin{equation}
    di_{\mtu}(u)=\Dsdd{ i(\mtu+tu) }{t}{0}=\Dsdd{ \mtu-tu+\alpha(tu) }{t}{0}.
    \end{equation}
    Il suffit de prouver que \( \Dsdd{ \alpha(tu) }{t}{0}=0\) pour conclure que \( di_{\mtu}(u)=-u\). Pour cela, nous remarquons que \( \alpha(0)=0\) et donc que
    \begin{subequations}
        \begin{align}
        \Dsdd{ \alpha(tu) }{t}{0}&=\lim_{t\to 0} \frac{ \alpha(tu)-\alpha(0) }{ t }\\
        &=\lim_{t\to 0} \sum_{k=2}^{\infty}(-1)^k\frac{ (tu)^k }{ t }\\
        &=-\lim_{t\to 0} u\sum_{k=1}^{\infty}(-1)^kt^ku^k.
        \end{align}
    \end{subequations}
    La norme de ce qui est dans la limite est majorée par
    \begin{equation}
    \| u \|\sum_{k=1}^{\infty}\| tu \|^k=\| u \|\left( \frac{1}{ 1-\| tu \| }-1 \right),
    \end{equation}
    et cela tend vers zéro lorsque \( t\to\infty\). Nous avons utilisé la somme \ref{EqRGkBhrX} de la série géométrique. Nous avons bien prouvé que \( di_{\mtu}(u)=-u\).

    \item[Différentielle en général]
    Soit maintenant \( u_0\in\GL(E)\) et \( h\in\End(E)\) tel que \( u_0+h\in \GL(E)\); par le premier point, il suffit de prendre \( \| h \|\) suffisamment petit. Vu que \( u_0+h=u_0(\mtu+u_0^{-1}h)\) nous avons
    \begin{equation}
        (u_0+h)^{-1}=(\mtu+u_0^{-1}h)^{-1}u_0^{-1}.
    \end{equation}
    Nous pouvons donc calculer
    \begin{equation}
        (u_0+h)^{-1}=\big( \mtu-u_0^{-1}h+\alpha(u_0^{-1}h) \big)u_0^{-1}=u_0^{-1}-u_0^{-1}hu_0^{-1}+\alpha(u_0^{-1}h)u_0^{-1},
    \end{equation}
    et ensuite
    \begin{equation}
        di_{u_0}(h)=\Dsdd{ i(u_0+th) }{t}{0}=\Dsdd{ u_0^{-1}-tu_0^{-1}hu_0^{-1}+\alpha(tu_0^{-1}h)u_0^{-1} }{t}{0},
    \end{equation}
    mais nous avons déjà vu que
    \begin{equation}
        \Dsdd{ \alpha(th) }{t}{0}=0,
    \end{equation}
    donc
    \begin{equation}
        di_{u_0}(h)=-u_0^{-1}hu_0^{-1}
    \end{equation}
    Cela donne la différentielle de l'application inverse.

    \item[Continuité de l'inverse]

        L'application \( i\) est continue parce que différentiable.

%    \item[Continuité de la différentielle de l'inverse]

%
%    Nous devons montrer que l'application \( di\colon \GL(E)\to \aL\big( \GL(E),\GL(E) \big)\) est continue. Pour cela nous allons l'écrire comme composée de fonctions continues. Si
%    \begin{equation}
%        \begin{aligned}
%        g\colon \GL(E)&\to \aL\big( \GL(E),\GL(E) \big) \\
%            g(v)h&= -vhv 
%        \end{aligned}
%    \end{equation}
%    alors \( di=g\circ i\). Nous avons déjà mentionné le fait que \( i\) était continue. Il reste à voir \( g\). Nous pouvons écrire
%    \begin{equation}
%    g(v)=-L_v\circ R_v
%    \end{equation}
%    où \( L_v(h)=vh\) et \( R_v(h)=hv\). Montrons que \( L\colon \GL(E)\to \aL\big( \GL(E),\GL(E) \big)\) est continue en considérant \( v_k\to v\) dans \( \GL(E)\) et la caractérisation séquentielle de la continuité, proposition \ref{PropFnContParSuite}. Alors
%    \begin{equation}
%        \| L_{v_k}-L_v \|_{\aL\big( \GL(E),\GL(E) \big)}=\sup_{\| h \|=1}\| (v_k-v)h \|\leq \sup_{\| h \|=1}\| v_k-v \|\| h \|\to 0.
%    \end{equation}
%L'application \( R\) est également continue, avec le même calcul. Nous avons donc \( L_{v_k}\to L_v\) et \( R_{v_k}\to R_v\) dans \( \aL\big( \GL(E),\GL(E) \big)\), donc le lemme \ref{LemWWXVSae} avec \( F=\aL\big( \GL(E),\GL(E) \big)\) nous dit que
%    \begin{equation}
%    L_{v_k}\circ R_{v_k}\to L_v\circ R_v
%    \end{equation}
%    dans \( \aL\big( \GL(E),\GL(E) \big)\), ce qui signifie que \( g\) est continue. 
%
%    Par conséquent \( di=i\circ g\) est également continue. 
%
    \item[L'inverse est \(  C^{\infty}\)]

        Pour cette partie de la preuve, voir \cite{SNPdukn}. Nous allons écrire la fonction inverse comme une composée. Soient les applications
        %AFAIRE : teminer
        \begin{equation}
            \begin{aligned}
                B\colon \cL(F,E)\times \cL(F,E)&\to \cL\big( \cL(E,F),\cL(F,E) \big) \\
                B(\psi_1,\psi_2)(A)&\mapsto -\psi_1\circ A\circ\psi_2
            \end{aligned}
        \end{equation}
        et
        \begin{equation}
            \begin{aligned}
                \Delta\colon \cL(F,E)&\to \cL(F,E)\times \cL(F,E) \\
                \varphi&\mapsto (\varphi,\varphi) 
            \end{aligned}
        \end{equation}
        Nous avons alors 
        \begin{equation}
            di=B\circ\Delta\circ i.
        \end{equation}
%
%        Nous commençons par prouver que \( g\) est de classe \( C^{\infty}\). Pour cela nous commençons par remarquer que
%        \begin{equation}
%            \Dsdd{ L_{v+tu}R_{v+tu} }{t}{0}=\lim_{t\to 0} \frac{ L_{v+tu}R_{v+tu}-L_vR_v }{ t }=L_vR_u+L_uR_v.
%        \end{equation}
%        En effet, un mini-calcul montre que
%        \begin{equation}
%        \sup_{\| h \|=1}\| \frac{ (v+tu)h(v+tu) }{ t }-\frac{ vhv }{ t }-vhu-uhv \|=\sup_{\| h \|=1}\| tuhu \|\to 0.
%        \end{equation}
%    Ensuite nous nous attaquons à la différentielle \( k\)\ieme de \( g\) :
%    \begin{subequations}
%        \begin{align}
%        (d^kg)_{v_1,\ldots, v_k}(u)&=\frac{ d }{ dt_1 }\cdots\frac{ d }{ dt_k }\Big( L_{u+t_1v_1+\ldots +t_kv_k}R_{u+tv_1+\ldots +t_kv_k} \Big)\\
%        &=\frac{ d }{ dt_1 }\cdots\frac{ d }{ dt_{k-1} }\Big( L_{u+t_1v_1+\ldots +t_{k-1}v_{k-1}}R_{v_k}+L_{v_k}R_{u+t_1v_1+\ldots +t_{k-1}v_{k-1}} \Big).
%        \end{align}
%    \end{subequations}
%    En continuant ainsi nous obtenons une belle somme de termes de la forme \( L_uR_{v_i}\), \( L_{v_i}R_u\) et \( L_{v_i}R_{v_j}\) qui sont tous continues en les \( v_i\), donc \( g\) est de classe \( C^k\) et donc de classe \( C^{\infty}\).
%
%L'application \( di=g\circ i\) est alors \( C^1\) en tant que composée de fonctions de classes \( C^1\), ce qui entraine que \( i\) est \( C^2\) etc. C'est l'argument en lacet de chaussure qui itère l'égalité \( di=i\circ g\).
%
        \end{subproof}
\end{proof}

%+++++++++++++++++++++++++++++++++++++++++++++++++++++++++++++++++++++++++++++++++++++++++++++++++++++++++++++++++++++++++++
\section{Convolution}
%+++++++++++++++++++++++++++++++++++++++++++++++++++++++++++++++++++++++++++++++++++++++++++++++++++++++++++++++++++++++++++

Le théorème qui permet de définir le produit de convolution est la suivant.

\begin{theorem}[\cite{MesIntProbb}]
    Soient \( f,g\in L^1(\eR^n)\). 
    \begin{enumerate}
        \item
            Pour presque tout \( x\in \eR^n\), la fonction
            \begin{equation}
                y\mapsto g(x-y)f(y)
            \end{equation}
            est dans \( L^1(\eR^n)\), et nous définissons le \defe{produit de convolution}{produit!de convolution} de \( f\) et \( g\) par
            \begin{equation}
                (f*g)(x)=\int_{\eR^n} f(y)g(x-y)dy.
            \end{equation}
        \item
            \( f*g\in L^1(\eR^n)\).
        \item
            \( \| f*g \|_1\leq \| f \|_1\| g \|_1\).
    \end{enumerate}
\end{theorem}

L'ensemble \( L^1(\eR^n)\) devient alors une algèbre de Banach.

\begin{lemma}
    Le produit de convolution est commutatif : \( f*g=g*f\).
\end{lemma}

\begin{proof}
    Le théorème de Fubini (théorème \ref{ThoFubinioYLtPI}) permet d'écrire
    \begin{equation}
        (f*g)(x)=\int_{\eR^n}f(y)g(x-y)dy=\int_{-\infty}^{\infty}dy_1\ldots \int_{-\infty}^{\infty}dy_nf(y)g(x-y).
    \end{equation}
    En effectuant le changement de variable \( z_i=x_i-y_i\) dans chacune des intégrales nous obtenons
    \begin{equation}
        (f*g)(x)=\int_{\eR^n}g(z)f(x-z)dz=(g*f)(x).
    \end{equation}
\end{proof}

\begin{proposition}[\cite{CXCQJIt}] \label{PropHNbdMQe}
    Si \( f\in L^1(\eR)\) et si \( g\) est dérivable avec \( g'\in L^{\infty}\), alors \( f*g\) est dérivable et \( (f*g)'=f*g'\).
\end{proposition}

\begin{proof}
    La fonction qu'il faut intégrer pour obtenir \( f*g\) est $f(t)g(x-t)$, dont la dérivée par rapport à \( x\) est \( f(t)g'(x-t)\). La norme de cette dernière est majorée (uniformément en \( x\)) par \( G(t)=| f(t) | \| g' \|_{\infty}\). La fonction \( f\) étant dans \( L^1(\eR)\), la fonction \( G\) est intégrable et le théorème de dérivation sous l'intégrale (théorème \ref{ThoMWpRKYp}) nous dit que \( f*g\) est dérivable et
    \begin{equation}
        (f*g)'(x)=\frac{ d }{ dx }\int_{\eR}f(t)g(x-t)dt=\int_{\eR}f(t)g'(x-t)dt=(f*g')(x).
    \end{equation}
\end{proof}

%+++++++++++++++++++++++++++++++++++++++++++++++++++++++++++++++++++++++++++++++++++++++++++++++++++++++++++++++++++++++++++
\section{Théorème du point fixe de Picard}
%+++++++++++++++++++++++++++++++++++++++++++++++++++++++++++++++++++++++++++++++++++++++++++++++++++++++++++++++++++++++++++

\begin{definition}
    Une application \( f\colon (X,\| . \|_X)\to (Y,\| . \|_Y)\) entre deux espaces métriques est une \defe{contraction}{contraction} si elle est \( k\)-\defe{Lipschitz}{Lipschitz} pour un certain \( 0\leq k<1\), c'est à dire si pour tout \( x,y\in X\) nous avons
    \begin{equation}
        \| f(x)-f(y) \|_Y\leq k\| x-y \|_{X}.
    \end{equation}
\end{definition}

\begin{theorem}[Picard \cite{ClemKetl,NourdinAnal}\footnote{Il me semble qu'à la page 100 de \cite{NourdinAnal}, l'hypothèse H1 qui est prouvée ne prouve pas Hn dans le cas \( n=1\). Merci de m'écrire si vous pouvez confirmer ou infirmer. La preuve donnée ici ne contient pas cette «erreur».}.]     \label{ThoEPVkCL}
    Soit \( X\) un espace métrique complet et \( f\colon X\to X\) une application contractante, de constante de Lipschitz \( k\). Alors \( f\) admet un unique point fixe, nommé \( \xi\). Ce dernier est donné par la limite de la suite définie par récurrence 
    \begin{subequations}
        \begin{numcases}{}
            x_0\in X\\
            x_{n+1}=f(x_n).
        \end{numcases}
    \end{subequations}
    De plus nous pouvons majorer l'erreur par
    \begin{equation}    \label{EqKErdim}
        \| x_n-x \|\leq \frac{ k^n }{ 1-k }\| x_n-x_{n-1} \|\leq \frac{ k^n }{ 1-k }\| x_1-x_0 \|.
    \end{equation}

    Soit \( r>0\), \( a\in X\) tels que la fonction \( f\) laisse la boule \( K=\overline{ B(a,r) }\) invariante (c'est à dire que \( f\) se restreint à \( f\colon K\to K\)). Nous considérons les suites \( (u_n)\) et \( (v_n)\) définies par
    \begin{subequations}
        \begin{numcases}{}
            u_0=v_0\in K\\
            u_{n+1}=f(v_n), v_{n+1}\in B(u_n,\epsilon).
        \end{numcases}
    \end{subequations}
    Alors le point fixe \( \xi\) de \( f\) est dans \( K\) et la suite \( (v_n)\) satisfait l'estimation
    \begin{equation}
        \| v_n-\xi \|\leq \frac{ k^n }{ 1-k }\| u_1-u_0 \|+\frac{ \epsilon }{ 1-k }.
    \end{equation}
\end{theorem}
\index{théorème!Picard}
\index{point fixe!Picard}

La première inégalité \eqref{EqKErdim} donne une estimation de l'erreur calculable en cours de processus; la seconde donne une estimation de l'erreur calculable avant de commencer.

\begin{proof}
    
    Nous commençons par l'unicité du point fixe. Si \( a\) et \( b\) sont des points fixes, alors \( f(a)=a\) et \( f(b)=b\). Par conséquent
    \begin{equation}
        \| f(a)-f(b) \|=\| a-b \|,
    \end{equation}
    ce qui contredit le fait que \( f\) soit une contraction.

    En ce qui concerne l'existence, notons que si la suite des \( x_n\) converge dans \( X\), alors la limite est un point fixe. En effet en prenant la limite des deux côtés de l'équation \( x_{n+1}=f(x_n)\), nous obtenons \( \xi=f(\xi)\), c'est à dire que \( \xi\) est un point fixe de \( f\). Notons que nous avons utilisé ici la continuité de \( f\), laquelle est une conséquence du fait qu'elle soit Lipschitz. Nous allons donc porter nos efforts à prouver que la suite est de Cauchy (et donc convergente parce que \( X\) est complet). Nous commençons par prouver que \( \| x_{n+1}-x_n \|\leq k^n\| x_0-x_1 \|\). En effet pour tout \( n\) nous avons
    \begin{equation}
        \| x_{n+1}-x_n \|=\| f(x_n)-f(x_{n-1}) \|\leq k\| x_n-x_{n-1} \|.
    \end{equation}
    La relation cherchée s'obtient alors par récurrence. Soient \( q>p\). En utilisant une somme télescopique,
    \begin{subequations}
        \begin{align}
            \| x_q-x_p \|&\leq \sum_{l=p}^{q-1}\| x_{l+1}-x_l \|\\
            &\leq\left( \sum_{l=p}^{q-1}k^l \right)\| x_1-x_0 \|\\
            &\leq\left(\sum_{l=p}^{\infty}k^l\right)\| x_1-x_0 \|.
        \end{align}
    \end{subequations}
    Étant donné que \( k<1\), la parenthèse est la queue d'une série qui converge, et donc tend vers zéro lorsque \( p\) tend vers l'infini.

    En ce qui concerne les inégalités \eqref{EqKErdim}, nous refaisons une somme télescopique :
    \begin{subequations}
        \begin{align}
            \| x_{n+p}-x_n \|&\leq \| x_{n+p}-x_{n+p-1} \|+\ldots +\| x_{n+1}-x_n \|\\
            &\leq k^p\| x_n-x_{n-1} \|+k^{p-1}\| x_n-x_{n-1} \|+\ldots +k\| x_n-x_{n-1} \|\\
            &=k(1+\ldots +k^{p-1})\| x_n-x_{n-1}\|  \\
            &\leq \frac{ k }{ 1-k }\| x_n-x_{n-1} \|.
        \end{align}
    \end{subequations}
    En prenant la limite \( p\to \infty\) nous trouvons
    \begin{equation}        \label{EqlUMVGW}
        \| \xi-x_n \|\leq \frac{ k }{ 1-k }\| x_n-x_{n-1} \|\leq \frac{ k }{ 1-k }\| x_1-x_0 \|.
    \end{equation}

    Nous passons maintenant à la seconde partie du théorème en supposant que \( f\) se restreigne en une fonction \( f\colon K\to K\). D'abord \( K\) est encore un espace métrique complet, donc la première partie du théorème s'y applique et \( f\) y a un unique point fixe.
    
    Nous allons montrer la relation par récurrence. Tout d'abord pour \( n=1\) nous avons
    \begin{equation}
        \| v_1-\xi \|\leq\| v_1-u_1 \|+\| u_1-\xi \|\leq \epsilon+\frac{ k }{ 1-k }\| u_1-u_0 \|
    \end{equation}
    où nous avons utilisé l'estimation \eqref{EqlUMVGW}, qui reste valable en remplaçant \( x_1\) par \( u_1\)\footnote{Elle n'est cependant pas spécialement valable si on remplace \( x_n\) par \( u_n\).}. Nous pouvons maintenant faire la récurrence :
    \begin{subequations}
        \begin{align}
            \| v_{n+1}-\xi \|&\leq \| v_{n+1}-u_{n+1} \|+\| u_{n+1}-\xi \|\\
            &\leq \epsilon+k\| v_n-\xi \|\\
            &\leq \epsilon+k\left( \frac{ k^n }{ 1-k }\| u_1-u_0 \|+\frac{ \epsilon }{ 1-k } \right)\\
            &=\frac{ \epsilon }{ 1-k }+\frac{ k^{n+1} }{ 1-k }\| u_1-u_0 \|.
        \end{align}
    \end{subequations}
\end{proof}

\begin{remark}
    Ce théorème comporte deux parties d'intérêts différents. La première partie est un théorème de point fixe usuel, qui sera utilisé pour prouver l'existence de certaines équations différentielles.

    La seconde partie est intéressante d'un point de vie numérique. En effet, ce qu'elle nous enseigne est que si à chaque pas de calcul de la récurrence \( x_{n+1}=f(x_n)\) nous commettons une erreur d'ordre de grandeur \( \epsilon\), alors le procédé (la suite \( (v_n)\)) ne converge plus spécialement vers le point fixe, mais tend vers le point fixe avec une erreur majorée par \( \epsilon/(k-1)\).
\end{remark}

\begin{remark}
Au final l'erreur minimale qu'on peut atteindre est de l'ordre de \( \epsilon\). Évidemment si on commet une faute de calcul de l'ordre de \( \epsilon\) à chaque pas, on ne peut pas espérer mieux.
\end{remark}

\begin{remark}  \label{remIOHUJm}
    Si \( f\) elle-même n'est pas contractante, mais si \( f^p\) est contractante pour un certain \( p\in \eN\) alors la conclusion du théorème de Picard reste valide et \( f\) a le même unique point fixe que \( f^p\). En effet nommons \( x\) le point fixe de \( f\) : \( f^p(x)=x\). Nous avons alors
    \begin{equation}
        f^p\big( f(x) \big)=f\big( f^p(x) \big)=f(x),
    \end{equation}
    ce qui prouve que \( f(x)\) est un point fixe de \( f^p\). Par unicité nous avons alors \( f(x)=x\), c'est à dire que \( x\) est également un point fixe de \( f\).
\end{remark}

Si la fonction n'est pas Lipschitz mais presque, nous avons une variante.
\begin{proposition}
    Soit \( E\) un ensemble compact\footnote{Notez cette hypothèse plus forte} et si \( f\colon E\to E\) est une fonction telle que
    \begin{equation}        \label{EqLJRVvN}
        \| f(x)-f(y) \|< \| x-y \|
    \end{equation}
    pour tout \( x\neq y\) dans \( E\) alors \( f\) possède un unique point fixe.
\end{proposition}

\begin{proof}
    La suite \( x_{n+1}=f(x_n)\) possède une sous suite convergente. La limite de cette sous suite est un point fixe de \( f\) parce que \( f\) est continue. L'unicité est due à l'aspect strict de l'inégalité \eqref{EqLJRVvN}.
\end{proof}

%---------------------------------------------------------------------------------------------------------------------------
\subsection{Théorème de Cauchy-Lipschitz}
%---------------------------------------------------------------------------------------------------------------------------

\begin{definition}
    Une fonction 
    \begin{equation}
        \begin{aligned}
            f\colon \eR^n\times R^m&\to \eR^p \\
            (t,y)&\mapsto f(t,y) 
        \end{aligned}
    \end{equation}
    est \defe{localement Lipschitz}{Lipschitz!localement} en \( y\) au point \( (t_0,y_0)\) si il existe des voisinages \( V\) de \( t_0\) et \( W\) de \( y_0\) et un nombre \( k>0\) tels que pour tout \( (t,y)\in V\times W\) on ait
    \begin{equation}
        \big\| f(t_0,y_0)-f(t,y) \big\|\leq k\| y-y_0 \|.
    \end{equation}
    La fonction est localement Lipschitz sur un ouvert \( U\) de \( \eR^n\times \eR^m\) si elle est localement Lipschitz en chaque point de \( U\).
\end{definition}

\begin{proposition}[Limite uniforme de fonctions continues]\label{PropCZslHBx}
    Soit \( X\) un espace topologique et \( (Y,d)\) un espace métrique. Si une suite de fonctions \( f_n\colon X\to Y\) continues converge uniformément, alors la limite est séquentiellement continue\footnote{Si \( X\) est métrique, alors c'est la continuité usuelle par la proposition \ref{PropFnContParSuite}.}.
\end{proposition}

\begin{proof}
    Soit \( a\in X\) et prouvons que \( f\) est séquentiellement continue en \( a\). Pour cela nous considérons une suite \( x_n\to a\) dans \( X\). Nous savons que \( f(x_n)\stackrel{Y}{\longrightarrow}f(x)\). Pour tout \(k\in \eN\), tout \( n\in \eN\) et tout \( x\in X\) nous avons la majoration
    \begin{equation}
        \big\| f(x_n)-f(x) \big\|\leq \big\| f(x_n)-f_k(x_n) \big\|+\big\| f_k(x_n)-f_k(x) \big\|+\big\| f_k(x)-f(x) \big\|\leq 2\| f-f_k \|_{\infty}+\big\| f_k(x_n)-f_k(x) \big\|.    
    \end{equation}
    Soit \( \epsilon>0\). Si nous choisissons \( k\) suffisamment grand la premier terme est plus petit que \( \epsilon\). Et par continuité de \( f_k\), en prenant \( n\) assez grand, le dernier terme est également plus petit que \( \epsilon\).
\end{proof}

\begin{proposition} \label{PropSYMEZGU}
    Soit \( X\) un espace topologique métrique compact et \( (Y,d)\) un espace espace métrique complet. Alors l'espace des fonctions continues \( X\to Y\) muni de la norme uniforme \( \big( C(X,Y),\| . \|_{\infty} \big)\) est complet.
\end{proposition}
\index{espace!complet!\( C(X,Y)\),norme uniforme}

\begin{proof}
    Notons que l'hypothèse de compacité de \( X\) sert à donner un sens à la norme uniforme : vu que \( X\) est compact et que les fonctions sont continues, elles sont bornées par le théorème \ref{ThoImCompCotComp}. 

    Soit \( (f_n)\) une suite de Cauchy dans \( C(X,Y)\), c'est à dire que pour tout \( \epsilon>0\) il existe \( N\in \eN\) tel que si \( k,l>N\) nous avons \( \| f_k-f_l \|_{\infty}\leq \epsilon\). Cette suite vérifie le critère de Cauchy uniforme \ref{PropNTEynwq} et donc converge uniformément vers une fonction \( f\colon X\to Y\). La continuité de la fonction \( f\) découle de la convergence uniforme et de la proposition \ref{PropCZslHBx} (c'est pour avoir l'équivalence entre la continuité séquentielle et la continuité normale que nous avons pris l'hypothèse d'espace métrique).
\end{proof}


\begin{lemma}       \label{LemdLKKnd}
    Soient \( A\) et \( B\) deux espaces compact. L'ensemble des fonctions continues de \( A\) vers \( B\) muni de la norme uniforme est complet.
\end{lemma}
\index{espace!complet!\(\big( C(A,B),\| . \|_{\infty} \big)\)}
% TODO : revoir cette preuve à la lumière du critère de Cauchy uniforme \ref{PropNTEynwq}.

\begin{proof}
    Soit \( (f_k)\) une suite de Cauchy de fonctions dans \( C(A,B)\). Pour chaque \( x\in A \) nous avons
    \begin{equation}
        \| f_k(x)-f_l(x) \|_B\leq \| f_k-f_l \|_{\infty},
    \end{equation}
    de telle sorte que la suite \( (f_k(x))\) est de Cauchy dans \( B\) et converge donc vers un élément de \( B\). La suite de Cauchy \( (f_k)\) converge donc ponctuellement vers une fonction \( f\colon A\to B\). Nous devons encore voir que cette fonction est continue; ce sera l'uniformité de la norme qui donnera la continuité. En effet soit \( x_n\to x\) une suite dans \( A\) convergent vers \( x\in A\). Pour chaque \( k\in \eN\) nous avons
    \begin{equation}
        \| f(x_n)-f(x) \|\leq \| f(x_n)-f_k(x_n) \|  +\| f_k(x_n)-f_k(x) \|+\| f_k(x)-f(x) \|.
    \end{equation}
    En prenant \( k\) et \( n\) assez grands, cette expression peut être rendue aussi petite que l'on veut; le premier et le troisième terme par convergence ponctuelle \( f_k\to f\), le second terme par continuité de \( f_k\). La suite \( f(x_n)\) est donc convergente vers \( f(x)\) et la fonction \( f\) est continue.
\end{proof}

\begin{theorem}[Cauchy-Lipschitz\cite{SandrineCL}]\index{théorème!Cauchy-Lipschitz}\label{ThokUUlgU}
    Nous considérons l'équation différentielle
    \begin{subequations}        \label{XtiXON}
        \begin{numcases}{}
            y'=f(t,y)\\
            y(t_0)=y_0
        \end{numcases}
    \end{subequations}
    avec \( f\colon U\to \eR^n\) où \( U\) est un ouvert de \( \eR\times \eR^n\). Nous supposons que \( f\) est continue sur \( U\) et localement Lipschitz\footnote{Nous ne supposons pas que \( f\) soit une contraction.} par rapport à \( y\). Alors le système \eqref{XtiXON} admet une unique solution maximale. Cette solution est \( C^1\). 
\end{theorem}

\begin{remark}
    L'écriture «\( y'=f(t,y)\)» est un abus de notation pour demander que pour chaque \( t\) nous ayons \( y'(t)=f\big(t,y(t)\big)\).
\end{remark}

\begin{proof}
    Si \( y\) est une solution de l'équation différentielle considérée, elle vérifie
    \begin{equation}        \label{EqPGLwcL}
        y(t)=y_0+\int_{t_0}^tf\big( u,y(u) \big)du.
    \end{equation}
    Ceci nous incite à considérer l'opérateur \( \Phi\colon \mF\to \mF\) défini par
    \begin{equation}
        \Phi(y)(t)=y_0+\int_{t_0}^tf\big( u,y(u) \big)du.
    \end{equation}

    \begin{subproof}
    \item[Cylindre de sécurité et espace fonctionnel]

    Précisons l'espace fonctionnel \( \mF\) adéquat. Soient \( V\) et \( W\) les voisinages de \( t_0\) et \( y_0\) sur lesquels \( f\) est localement Lipschitz. Nous considérons les quantités suivantes :
    \begin{enumerate}
        \item
            \( M=\sup_{V\times W}f\) ;
        \item
            \( r>0\) tel que \( \overline{ B(y_0,r) }\subset V\)
        \item
            \( T>0\) tel que \( \overline{ B(t_0,T) }\subset W\) et \( T<r/M\).
    \end{enumerate}
    Nous considérons alors \( \mF\), l'ensemble des fonctions continues \( \overline{ B(t_0,T) }\to \overline{ B(y_0,r) }\) muni de la norme uniforme. Par le lemme \ref{LemdLKKnd} l'espace \( \mF\) est complet.

    Le fait que \( \Phi(y)\) soit continue lorsque \( y\) est continue est une propriété de l'intégration et du fait que \( f\) soit continue en ses deux variables. Prouvons que \( \Phi(y)(t)\in\overline{ B(y_0,r) }\). Pour cela, notons que
    \begin{equation}
        | \Phi(y)(t)-y_0 |\leq \int_{t_0}^t |f\big( u,y(u) \big)|du\leq | t-t_0 |\| f \|_{\infty}.
    \end{equation}
    Étant donné que \( t\in\overline{ B(t_0,T) }\) nous avons \( | t-t_0 |\leq r/M\) et donc \( | \Phi(y)(t)-y_0 |\leq r\).

    L'équation \eqref{EqPGLwcL} signifie que \( y\) est un point fixe de \( \Phi\). L'espace \( \mF\) étant complet le théorème de point fixe de Picard (théorème \ref{ThoEPVkCL}) s'applique. Nous allons montrer qu'il existe un \( p\in\eN\) tel que \( \Phi^p\) soit contractante. Par conséquent \( \Phi^p\) aura un unique point fixe qui sera également unique point fixe de \( \Phi\) par la remarque \ref{remIOHUJm}.
    
\item[Une contraction]

    Prouvons donc que \( \Phi^p\) est contractante pour un certain \( p\). Pour cela nous commençons par montrer la formule suivante par récurrence :
    \begin{equation}        \label{EqRAdKxT}
        \big\| \Phi^p(x)(t)-\Phi^p(y)(t) \big\|\leq \frac{ k^p| t-t_0 |^p }{ p! }\| x-y \|_{\infty}
    \end{equation}
    pour tout \( x,y\in\mF\), et pour tout \( t\in\overline{ B(t_0,T) }\). Pour \( p=0\) la formule \eqref{EqRAdKxT} est vérifiée parce que \( \| x-y \|_{\infty}\) est le supremum de \( \| x(t)-y(t) \|\) pour \( t\in\overline{ B(t_0,T) }\). Supposons que la formule soit vraie pour \( p\) et calculons pour \( p+1\). Pour tout \( t\in\overline{ B(t_0,T) }\) nous avons
    \begin{subequations}
        \begin{align}
            \big\| \Phi^{p+1}(x)(t)-\Phi^{p+1}(y)(t) \big\|&\leq \left| \int_{t_0}^t\big\| f\big( u,\Phi^p(x)(u) \big)-f\big( u,\Phi^p(y)(u) \big) \big\|du \right| \\
            &\leq \left| \int_{t_0}^tk\| \Phi^p(x)(u)-\Phi^p(y)(u) \|du \right|    \label{subIKYixF}\\
            &\leq \left| \int_{t_0}^tk\frac{ k^p| t-t_0 | }{ p! }\| x-y \|_{\infty} \right| \label{subxkNjiV} \\
            &=\frac{ k^{p+1}| t-t_0 |^{p+1} }{ (p+1)! }\| x-y \|_{\infty}.
        \end{align}
    \end{subequations}
    Justifications :
    \begin{itemize}
        \item \eqref{subIKYixF} parce que \( f\) est Lipschitz.
        \item \eqref{subxkNjiV} par hypothèse de récurrence.
    \end{itemize}
    La formule \eqref{EqRAdKxT} est maintenant établie. Nous pouvons maintenant montrer que \( \Phi^p\) est une contraction pour un certain \( p\). Pour tout \( t\in \overline{ B(t_0,T) }\) nous avons
    \begin{equation}
         \| \Phi^p(x)(t)-\Phi^p(y)(t) \|\leq \frac{ k^p }{ t! }| t-t_0 |^p\| x-y \|_{\infty}     \leq \frac{ k^pT^p }{ p! }\| x-y \|_{\infty}
    \end{equation}
    où nous avons utilisé le fait que \( | t-t_0 |^p<T^p\). En prenant le supremum sur \( t\) des deux côtés il vient
    \begin{equation}
        \| \Phi^p(x)-\Phi^p(y) \|_{\infty}\leq\frac{ k^pT^p }{ p! }\| x-y \|_{\infty}.
    \end{equation}
    Le membre de droite tend vers zéro lorsque \( p\to\infty\) parce que \( k<1\) et \( T^p/p!\to 0\)\footnote{C'est le terme général du développement de \(  e^{T}\) qui est une série convergente.}. Nous concluons donc que \( \Phi^p\) est une contraction pour un certain \( p\).

\item[Conclusion]

    L'unique point fixe de \( \Phi\) est alors l'unique solution continue de l'équation différentielle \eqref{XtiXON}. Par ailleurs l'équation elle-même \( y'=f(t,y)\) demande implicitement que \( y\) soit dérivable et donc continue. Nous concluons que l'unique point fixe de \( \Phi\) est l'unique solution de l'équation différentielle donnée. Cette dernière est automatiquement \( C^1\) parce que si \( y\) est continue alors \( u\mapsto f(u,y(u))\) est continue, c'est à dire que \( y'\) est continue.

\item[Unicité]

    Nous passons maintenant à la partie «prolongement maximum» du théorème. Soient \( x_1\) et \( x_2\) deux solutions maximales du problème \eqref{XtiXON} sur des intervalles \( I_1\) et \( I_2\) respectivement. Les intervalles \( I_1\) et \( I_2\) contiennent \( \overline{ B(t_0,r) }\) sur lequel \( x_1=x_2\) par unicité.
    
    
    Nous allons maintenant montrer que pour tout \( t\geq t_0\) pour lequel \( x_1\) ou \( x_2\) est défini, \( x_1(t)\) et \( x_2(t)\) sont définis et sont égaux. Le raisonnement sur \( t\leq t_0\) est similaire.
    
    Supposons que l'ensemble des \( t\geq t_0\) tels que \( x_1=x_2\) soit ouvert à droite, c'est à dire soit de la forme \( \mathopen[ t_0 ,b [\). Dans ce cas, soit \( x_1\) soit \( x_2\) (soit les deux) cesse d'exister en \( b\). En effet si nous avions les fonctions \( x_i\) sur \(\mathopen[ t_0 , b+\epsilon [\) alors l'équation \( x_1=x_2\) définirait un fermé dans \( \mathopen[ t_0 , b+\epsilon [\). Supposons pour fixer les idées que \( x_1\) cesse d'exister : le domaine de \( x_1\) (parmi les \( t\geq 0\)) est \( \mathopen[ t_0 , b [\) et sur ce domaine nous avons \( x_1=x_2\). Dans ce cas \( x_1\) pourrait être prolongé en \( x_2\) au-delà de \( b\). Si \( x_1\) et \( x_2\) s'arrêtent d'exister en même temps en \( b\), alors nous avons bien \( x_1=x_2\).

    Nous devons donc traiter le cas où \( x_1=x_2\) sur \( \mathopen[ t_0 , b \mathclose]\) alors que \( x_1\) et \( x_2\) existent sur \( \mathopen[ t_0 , b+\epsilon [\) pour un certain \( \epsilon\).

    Nous pouvons appliquer le théorème d'existence locale au problème
    \begin{subequations}
        \begin{numcases}{}
            y'=f(t,y)\\
            y(b)=x_1(b).
        \end{numcases}
    \end{subequations}
    Il existe un voisinage de \( b\) sur lequel la solution est unique. Sur ce voisinage nous devons donc avoir \( x_1=x_2\), ce qui contredit le fait que \( x_1\neq x_2\) en dehors de \( \mathopen[ t_0 , b \mathclose]\).
    \end{subproof}
\end{proof}

%---------------------------------------------------------------------------------------------------------------------------
\subsection{Équation de Fredholm}
%---------------------------------------------------------------------------------------------------------------------------

\begin{theorem}[Équation de Fredholm]\index{Fredholm!équation}\index{équation!Fredholm}     \label{ThoagJPZJ}
    Soit \( K\colon \mathopen[ a , b \mathclose]\times \mathopen[ a , b \mathclose]\to \eR\) et \( \varphi\colon \mathopen[ a , b \mathclose]\to \eR\), deux fonctions continues. Alors si \( \lambda\) est suffisamment petit, l'équation
    \begin{equation}
        f(x)=\lambda\int_a^bK(x,y)f(y)dy+\varphi(x)
    \end{equation}
    admet une unique solution qui sera de plus continue sur \( \mathopen[ a , b \mathclose]\).
\end{theorem}

\begin{proof}
    Nous considérons l'ensemble \( \mF\) des fonctions continues \( \mathopen[ a , b \mathclose]\to\mathopen[ a , b \mathclose]\) muni de la norme uniforme. Le lemme \ref{LemdLKKnd} implique que \( \mF\) est complet. Nous considérons l'application \( \Phi\colon \mF\to \mF\) donnée par
    \begin{equation}
        \Phi(f)(x)=\lambda\int_a^bK(x,y)f(y)dy+\varphi(x). 
    \end{equation}
    Nous montrons que \( \Phi^p\) est une application contractante pour un certain \( p\). Pour tout \( x\in \mathopen[ a , b \mathclose]\) nous avons
    \begin{subequations}
        \begin{align}
            \| \Phi(f)-\Phi(g) \|_{\infty}&\leq \| \Phi(f)(x)-\Phi(g)(x) \|\\
            &=| \lambda |\Big\| \int_a^bK(x,y)\big( f(y)-g(y) \big)dy  \Big\|\\
            &\leq | \lambda |\| K \|_{\infty}| b-a |\| f-g \|_{\infty}
        \end{align}
    \end{subequations}
    Si \( \lambda\) est assez petit, et si \( p\) est assez grand, l'application \( \Phi^p\) est donc une contraction. Elle possède donc un unique point fixe par le théorème de Picard \ref{ThoEPVkCL}.
\end{proof}
