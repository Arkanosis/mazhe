% This is part of Mes notes de mathématique
% Copyright (c) 2011-2014
%   Laurent Claessens
% See the file fdl-1.3.txt for copying conditions.

%+++++++++++++++++++++++++++++++++++++++++++++++++++++++++++++++++++++++++++++++++++++++++++++++++++++++++++++++++++++++++++
\section{Convolution}
%+++++++++++++++++++++++++++++++++++++++++++++++++++++++++++++++++++++++++++++++++++++++++++++++++++++++++++++++++++++++++++

Le théorème qui permet de définir le produit de convolution est la suivant.

\begin{theorem}[\cite{MesIntProbb}]
    Soient \( f,g\in L^1(\eR^n)\). 
    \begin{enumerate}
        \item
            Pour presque tout \( x\in \eR^n\), la fonction
            \begin{equation}
                y\mapsto g(x-y)f(y)
            \end{equation}
            est dans \( L^1(\eR^n)\), et nous définissons le \defe{produit de convolution}{produit!de convolution} de \( f\) et \( g\) par
            \begin{equation}
                (f*g)(x)=\int_{\eR^n} f(y)g(x-y)dy.
            \end{equation}
        \item
            \( f*g\in L^1(\eR^n)\).
        \item
            \( \| f*g \|_1\leq \| f \|_1\| g \|_1\).
    \end{enumerate}
\end{theorem}

L'ensemble \( L^1(\eR^n)\) devient alors une algèbre de Banach.

\begin{lemma}
    Le produit de convolution est commutatif : \( f*g=g*f\).
\end{lemma}

\begin{proof}
    Le théorème de Fubini (théorème \ref{ThoFubinioYLtPI}) permet d'écrire
    \begin{equation}
        (f*g)(x)=\int_{\eR^n}f(y)g(x-y)dy=\int_{-\infty}^{\infty}dy_1\ldots \int_{-\infty}^{\infty}dy_nf(y)g(x-y).
    \end{equation}
    En effectuant le changement de variable \( z_i=x_i-y_i\) dans chacune des intégrales nous obtenons
    \begin{equation}
        (f*g)(x)=\int_{\eR^n}g(z)f(x-z)dz=(g*f)(x).
    \end{equation}
\end{proof}

\begin{proposition}[\cite{CXCQJIt}] \label{PropHNbdMQe}
    Si \( f\in L^1(\eR)\) et si \( g\) est dérivable avec \( g'\in L^{\infty}\), alors \( f*g\) est dérivable et \( (f*g)'=f*g'\).
\end{proposition}

\begin{proof}
    La fonction qu'il faut intégrer pour obtenir \( f*g\) est $f(t)g(x-t)$, dont la dérivée par rapport à \( x\) est \( f(t)g'(x-t)\). La norme de cette dernière est majorée (uniformément en \( x\)) par \( G(t)=| f(t) | \| g' \|_{\infty}\). La fonction \( f\) étant dans \( L^1(\eR)\), la fonction \( G\) est intégrable et le théorème de dérivation sous l'intégrale (théorème \ref{ThoMWpRKYp}) nous dit que \( f*g\) est dérivable et
    \begin{equation}
        (f*g)'(x)=\frac{ d }{ dx }\int_{\eR}f(t)g(x-t)dt=\int_{\eR}f(t)g'(x-t)dt=(f*g')(x).
    \end{equation}
\end{proof}

%+++++++++++++++++++++++++++++++++++++++++++++++++++++++++++++++++++++++++++++++++++++++++++++++++++++++++++++++++++++++++++ 
\section{Théorèmes de de point fixe}
%+++++++++++++++++++++++++++++++++++++++++++++++++++++++++++++++++++++++++++++++++++++++++++++++++++++++++++++++++++++++++++

Nous allons voir les résultats suivants.
\begin{description}
    \item[Théorème de Picard] \ref{ThoEPVkCL} donne un point fixe comme limite d'itéré d'une fonction Lipschitz. Il aura pour conséquence le théorème de Cauchy-Lipschitz \ref{ThokUUlgU} et l'équation de Fredholm, théorème \ref{ThoagJPZJ}.
\item[Théorème de Brouwer] qui donne un point fixe pour une application d'une boule vers elle-même. Nous allons donner plusieurs versions et preuves.
        \begin{enumerate}
            \item
                Dans \( \eR^n\) en version \( C^{\infty}\) via le théorème de Stokes, proposition \ref{PropDRpYwv}.
            \item
                Dans \( \eR^n\) en version continue, en s'appuyant sur le cas \( C^{\infty}\) et en faisant un passage à la limite, théorème \ref{ThoRGjGdO}.
            \item
                Dans \( \eR^2\) via l'homotopie, théorème \ref{ThoLVViheK}. Oui, c'est très loin. Et c'est normal parce que ça va utiliser la formule de l'indice qui est de l'analyse complexe\footnote{On aime bien parce que ça ne demande pas Stokes, mais quand même hein, c'est pas gratos non plus.}.
        \end{enumerate}
    \item[Théorème de Markov-Kakutani]\ref{ThoeJCdMP} qui donne un point fixe à une application continue d'un convexe fermé borné dans lui-même. Ce théorème donnera la mesure de Haar \ref{ThoBZBooOTxqcI} sur les groupes compacts.
    \item[Théorème de Schauder] \ref{ThovHJXIU} qui est une version valable en dimension infinie du théorème de Brouwer. Il a pour conséquence le théorème de Cauchy-Arzela \ref{ThoHNBooUipgPX} pour les équations différentielles.
\end{description}

Le théorème de Schauder \ref{ThovHJXIU} permet de démontrer une version du théorème de Cauchy-Lipschitz (théorème \ref{ThokUUlgU}) sans la condition Lipschitz, mais alors sans unicité de la solution. Notons que de ce point de vue nous sommes dans la même situation que la différence entre le théorème de Brouwer et celui de Picard : hors hypothèse de type «contraction», point d'unicité.

%--------------------------------------------------------------------------------------------------------------------------- 
\subsection{Picard}
%---------------------------------------------------------------------------------------------------------------------------

\begin{definition}
    Une application \( f\colon (X,\| . \|_X)\to (Y,\| . \|_Y)\) entre deux espaces métriques est une \defe{contraction}{contraction} si elle est \( k\)-\defe{Lipschitz}{Lipschitz} pour un certain \( 0\leq k<1\), c'est à dire si pour tout \( x,y\in X\) nous avons
    \begin{equation}
        \| f(x)-f(y) \|_Y\leq k\| x-y \|_{X}.
    \end{equation}
\end{definition}

\begin{theorem}[Picard \cite{ClemKetl,NourdinAnal}\footnote{Il me semble qu'à la page 100 de \cite{NourdinAnal}, l'hypothèse H1 qui est prouvée ne prouve pas Hn dans le cas \( n=1\). Merci de m'écrire si vous pouvez confirmer ou infirmer. La preuve donnée ici ne contient pas cette «erreur».}.]     \label{ThoEPVkCL}
    Soit \( X\) un espace métrique complet et \( f\colon X\to X\) une application contractante, de constante de Lipschitz \( k\). Alors \( f\) admet un unique point fixe, nommé \( \xi\). Ce dernier est donné par la limite de la suite définie par récurrence 
    \begin{subequations}
        \begin{numcases}{}
            x_0\in X\\
            x_{n+1}=f(x_n).
        \end{numcases}
    \end{subequations}
    De plus nous pouvons majorer l'erreur par
    \begin{equation}    \label{EqKErdim}
        \| x_n-x \|\leq \frac{ k^n }{ 1-k }\| x_n-x_{n-1} \|\leq \frac{ k^n }{ 1-k }\| x_1-x_0 \|.
    \end{equation}

    Soit \( r>0\), \( a\in X\) tels que la fonction \( f\) laisse la boule \( K=\overline{ B(a,r) }\) invariante (c'est à dire que \( f\) se restreint à \( f\colon K\to K\)). Nous considérons les suites \( (u_n)\) et \( (v_n)\) définies par
    \begin{subequations}
        \begin{numcases}{}
            u_0=v_0\in K\\
            u_{n+1}=f(v_n), v_{n+1}\in B(u_n,\epsilon).
        \end{numcases}
    \end{subequations}
    Alors le point fixe \( \xi\) de \( f\) est dans \( K\) et la suite \( (v_n)\) satisfait l'estimation
    \begin{equation}
        \| v_n-\xi \|\leq \frac{ k^n }{ 1-k }\| u_1-u_0 \|+\frac{ \epsilon }{ 1-k }.
    \end{equation}
\end{theorem}
\index{théorème!Picard}
\index{point fixe!Picard}

La première inégalité \eqref{EqKErdim} donne une estimation de l'erreur calculable en cours de processus; la seconde donne une estimation de l'erreur calculable avant de commencer.

\begin{proof}
    
    Nous commençons par l'unicité du point fixe. Si \( a\) et \( b\) sont des points fixes, alors \( f(a)=a\) et \( f(b)=b\). Par conséquent
    \begin{equation}
        \| f(a)-f(b) \|=\| a-b \|,
    \end{equation}
    ce qui contredit le fait que \( f\) soit une contraction.

    En ce qui concerne l'existence, notons que si la suite des \( x_n\) converge dans \( X\), alors la limite est un point fixe. En effet en prenant la limite des deux côtés de l'équation \( x_{n+1}=f(x_n)\), nous obtenons \( \xi=f(\xi)\), c'est à dire que \( \xi\) est un point fixe de \( f\). Notons que nous avons utilisé ici la continuité de \( f\), laquelle est une conséquence du fait qu'elle soit Lipschitz. Nous allons donc porter nos efforts à prouver que la suite est de Cauchy (et donc convergente parce que \( X\) est complet). Nous commençons par prouver que \( \| x_{n+1}-x_n \|\leq k^n\| x_0-x_1 \|\). En effet pour tout \( n\) nous avons
    \begin{equation}
        \| x_{n+1}-x_n \|=\| f(x_n)-f(x_{n-1}) \|\leq k\| x_n-x_{n-1} \|.
    \end{equation}
    La relation cherchée s'obtient alors par récurrence. Soient \( q>p\). En utilisant une somme télescopique,
    \begin{subequations}
        \begin{align}
            \| x_q-x_p \|&\leq \sum_{l=p}^{q-1}\| x_{l+1}-x_l \|\\
            &\leq\left( \sum_{l=p}^{q-1}k^l \right)\| x_1-x_0 \|\\
            &\leq\left(\sum_{l=p}^{\infty}k^l\right)\| x_1-x_0 \|.
        \end{align}
    \end{subequations}
    Étant donné que \( k<1\), la parenthèse est la queue d'une série qui converge, et donc tend vers zéro lorsque \( p\) tend vers l'infini.

    En ce qui concerne les inégalités \eqref{EqKErdim}, nous refaisons une somme télescopique :
    \begin{subequations}
        \begin{align}
            \| x_{n+p}-x_n \|&\leq \| x_{n+p}-x_{n+p-1} \|+\ldots +\| x_{n+1}-x_n \|\\
            &\leq k^p\| x_n-x_{n-1} \|+k^{p-1}\| x_n-x_{n-1} \|+\ldots +k\| x_n-x_{n-1} \|\\
            &=k(1+\ldots +k^{p-1})\| x_n-x_{n-1}\|  \\
            &\leq \frac{ k }{ 1-k }\| x_n-x_{n-1} \|.
        \end{align}
    \end{subequations}
    En prenant la limite \( p\to \infty\) nous trouvons
    \begin{equation}        \label{EqlUMVGW}
        \| \xi-x_n \|\leq \frac{ k }{ 1-k }\| x_n-x_{n-1} \|\leq \frac{ k }{ 1-k }\| x_1-x_0 \|.
    \end{equation}

    Nous passons maintenant à la seconde partie du théorème en supposant que \( f\) se restreigne en une fonction \( f\colon K\to K\). D'abord \( K\) est encore un espace métrique complet, donc la première partie du théorème s'y applique et \( f\) y a un unique point fixe.
    
    Nous allons montrer la relation par récurrence. Tout d'abord pour \( n=1\) nous avons
    \begin{equation}
        \| v_1-\xi \|\leq\| v_1-u_1 \|+\| u_1-\xi \|\leq \epsilon+\frac{ k }{ 1-k }\| u_1-u_0 \|
    \end{equation}
    où nous avons utilisé l'estimation \eqref{EqlUMVGW}, qui reste valable en remplaçant \( x_1\) par \( u_1\)\footnote{Elle n'est cependant pas spécialement valable si on remplace \( x_n\) par \( u_n\).}. Nous pouvons maintenant faire la récurrence :
    \begin{subequations}
        \begin{align}
            \| v_{n+1}-\xi \|&\leq \| v_{n+1}-u_{n+1} \|+\| u_{n+1}-\xi \|\\
            &\leq \epsilon+k\| v_n-\xi \|\\
            &\leq \epsilon+k\left( \frac{ k^n }{ 1-k }\| u_1-u_0 \|+\frac{ \epsilon }{ 1-k } \right)\\
            &=\frac{ \epsilon }{ 1-k }+\frac{ k^{n+1} }{ 1-k }\| u_1-u_0 \|.
        \end{align}
    \end{subequations}
\end{proof}

\begin{remark}
    Ce théorème comporte deux parties d'intérêts différents. La première partie est un théorème de point fixe usuel, qui sera utilisé pour prouver l'existence de certaines équations différentielles.

    La seconde partie est intéressante d'un point de vie numérique. En effet, ce qu'elle nous enseigne est que si à chaque pas de calcul de la récurrence \( x_{n+1}=f(x_n)\) nous commettons une erreur d'ordre de grandeur \( \epsilon\), alors le procédé (la suite \( (v_n)\)) ne converge plus spécialement vers le point fixe, mais tend vers le point fixe avec une erreur majorée par \( \epsilon/(k-1)\).
\end{remark}

\begin{remark}
Au final l'erreur minimale qu'on peut atteindre est de l'ordre de \( \epsilon\). Évidemment si on commet une faute de calcul de l'ordre de \( \epsilon\) à chaque pas, on ne peut pas espérer mieux.
\end{remark}

\begin{remark}  \label{remIOHUJm}
    Si \( f\) elle-même n'est pas contractante, mais si \( f^p\) est contractante pour un certain \( p\in \eN\) alors la conclusion du théorème de Picard reste valide et \( f\) a le même unique point fixe que \( f^p\). En effet nommons \( x\) le point fixe de \( f\) : \( f^p(x)=x\). Nous avons alors
    \begin{equation}
        f^p\big( f(x) \big)=f\big( f^p(x) \big)=f(x),
    \end{equation}
    ce qui prouve que \( f(x)\) est un point fixe de \( f^p\). Par unicité nous avons alors \( f(x)=x\), c'est à dire que \( x\) est également un point fixe de \( f\).
\end{remark}

Si la fonction n'est pas Lipschitz mais presque, nous avons une variante.
\begin{proposition}
    Soit \( E\) un ensemble compact\footnote{Notez cette hypothèse plus forte} et si \( f\colon E\to E\) est une fonction telle que
    \begin{equation}        \label{EqLJRVvN}
        \| f(x)-f(y) \|< \| x-y \|
    \end{equation}
    pour tout \( x\neq y\) dans \( E\) alors \( f\) possède un unique point fixe.
\end{proposition}

\begin{proof}
    La suite \( x_{n+1}=f(x_n)\) possède une sous suite convergente. La limite de cette sous suite est un point fixe de \( f\) parce que \( f\) est continue. L'unicité est due à l'aspect strict de l'inégalité \eqref{EqLJRVvN}.
\end{proof}

\begin{theorem}[Équation de Fredholm]\index{Fredholm!équation}\index{équation!Fredholm}     \label{ThoagJPZJ}
    Soit \( K\colon \mathopen[ a , b \mathclose]\times \mathopen[ a , b \mathclose]\to \eR\) et \( \varphi\colon \mathopen[ a , b \mathclose]\to \eR\), deux fonctions continues. Alors si \( \lambda\) est suffisamment petit, l'équation
    \begin{equation}
        f(x)=\lambda\int_a^bK(x,y)f(y)dy+\varphi(x)
    \end{equation}
    admet une unique solution qui sera de plus continue sur \( \mathopen[ a , b \mathclose]\).
\end{theorem}

\begin{proof}
    Nous considérons l'ensemble \( \mF\) des fonctions continues \( \mathopen[ a , b \mathclose]\to\mathopen[ a , b \mathclose]\) muni de la norme uniforme. Le lemme \ref{LemdLKKnd} implique que \( \mF\) est complet. Nous considérons l'application \( \Phi\colon \mF\to \mF\) donnée par
    \begin{equation}
        \Phi(f)(x)=\lambda\int_a^bK(x,y)f(y)dy+\varphi(x). 
    \end{equation}
    Nous montrons que \( \Phi^p\) est une application contractante pour un certain \( p\). Pour tout \( x\in \mathopen[ a , b \mathclose]\) nous avons
    \begin{subequations}
        \begin{align}
            \| \Phi(f)-\Phi(g) \|_{\infty}&\leq \| \Phi(f)(x)-\Phi(g)(x) \|\\
            &=| \lambda |\Big\| \int_a^bK(x,y)\big( f(y)-g(y) \big)dy  \Big\|\\
            &\leq | \lambda |\| K \|_{\infty}| b-a |\| f-g \|_{\infty}
        \end{align}
    \end{subequations}
    Si \( \lambda\) est assez petit, et si \( p\) est assez grand, l'application \( \Phi^p\) est donc une contraction. Elle possède donc un unique point fixe par le théorème de Picard \ref{ThoEPVkCL}.
\end{proof}

%--------------------------------------------------------------------------------------------------------------------------- 
\subsection{Brouwer}
%---------------------------------------------------------------------------------------------------------------------------
\label{subSecZCCmMnQ}

\begin{proposition}
    Soit \( f\colon \mathopen[ a , b \mathclose]\to \mathopen[ a , b \mathclose]\) une fonction continue. Alors \( f\) accepte un point fixe.
\end{proposition}

\begin{proof}
    En effet si nous considérons \( g(x)=f(x)-x\) alors nous avons \( g(a)=f(a)-a\geq 0\) et \( g(b)=f(b)-b\leq 0\). Si \( g(a)\) ou \( g(b)\) est nul, la proposition est démontrée; nous supposons donc que \( g(a)>0\) et \( g(b)<0\). La proposition découle à présent du théorème des valeurs intermédiaires \ref{ThoValInter}.
\end{proof}

\begin{example}
    La fonction \( x\mapsto\cos(x)\) est continue entre \( \mathopen[ -1 , 1 \mathclose]\) et \( \mathopen[ -1 , 1 \mathclose]\). Elle admet donc un point fixe. Par conséquent il existe (au moins) une solution à l'équation \( \cos(x)=x\).
\end{example}

\begin{proposition}[Brouwer dans \( \eR^n\) version \(  C^{\infty}\) via Stokes]     \label{PropDRpYwv}
    Soit \( B\) la boule fermée de centre \( 0\) et de rayon \( 1\) de \( \eR^n\) et \( f\colon B\to B\) une fonction \(  C^{\infty}\). Alors \( f\) admet un point fixe.
\end{proposition}
\index{point fixe!Brouwer}

\begin{proof}
    Supposons que \( f\) ne possède pas de points fixes. Alors pour tout \( x\in B\) nous considérons la ligne droite partant de \( x\) dans la direction de \( f(x)\) (cette droite existe parce que \( x\) et \( f(x)\) sont supposés distincts). Cette ligne intersecte \( \partial B\) en un point que nous appelons \( g(x)\). Prouvons que cette fonction est \( C^k\) dès que \( f\) est \( C^k\) (y compris avec \( k=\infty\)).

   Le point \( g(x) \) est la solution du système
    \begin{subequations}
        \begin{numcases}{}
        g(x)-f(x)=\lambda\big( x-f(x) \big)\\
        \| g(x) \|^2=1\\
        \lambda\geq 0.
        \end{numcases}
    \end{subequations}
    En substituant nous obtenons l'équation
    \begin{equation}
        P_x(\lambda)=\| \lambda\big( x-f(x) \big)+f(x) \|^2-1=0,
    \end{equation}
    ou encore
    \begin{equation}
        \lambda^2\| x-f(x) \|^2+2\lambda\big( x-f(x) \big)\cdot f(x)+\| f(x) \|^2-1=0.
    \end{equation}
    En tenant compte du fait que \( \| f(x)<1 \|\) (pare que les images de \( f\) sont dans \( \mB\)), nous trouvons que \( P_x(0)\leq 0\) et \( P_x(1)\leq 0\). De même \( \lim_{\lambda\to\infty} P_x(\lambda)=+\infty\). Par conséquent le polynôme de second degré \( P_x\) a exactement deux racines distinctes \( \lambda_1\leq 0\) et \( \lambda_2\geq 1\). La racine que nous cherchons est la seconde. Le discriminant est strictement positif, donc pas besoin d'avoir peur de la racine dans
    \begin{equation}
        \lambda(x)=\frac{ -\big( x-f(x) \big)\cdot f(x)+\sqrt{   \Delta_x  } }{ \| x-f(x) \|^2 }
    \end{equation}
    où 
    \begin{equation}
        \Delta_x=4\Big( \big( x-f(x) \big)\cdot f(x) \Big)^2-4\| x-f(x) \|^2\big( \| f(x) \|^2-1 \big).
    \end{equation}
    Notons que la fonction \( \lambda(x)\) est \( C^k\) dès que \( f\) est \( C^k\); et en particulier elle est \( C^{\infty}\) si \( f\) l'est.

    En résumé la fonction \( g\) ainsi définie vérifie deux propriétés :
    \begin{enumerate}
        \item
            elle est \(  C^{\infty}\);
        \item
            elle est l'identité sur \( \partial B\).
    \end{enumerate}
    La suite de la preuve consiste à montrer qu'une telle rétraction sur \( B\) ne peut pas exister\footnote{Notons qu'il n'existe pas non plus de rétractions continues sur \( B\), mais pour le montrer il faut utiliser d'autres méthodes que Stokes, ou alors présenter les choses dans un autre ordre.}.

    Nous considérons une forme de volume \( \omega\) sur \( \partial B\) : l'intégrale de \( \omega\) sur \( \partial B\) est la surface de \( \partial B\) qui est non nulle. Nous avons alors
    \begin{equation}
        0<\int_{\partial B}\omega
        =\int_{\partial B}g^*\omega
        =\int_Bd(g^*\omega)
        =\int_Bg^*(d\omega)
        =0
    \end{equation}
    Justifications :
    \begin{itemize}
        \item 
            L'intégrale \( \int_{\partial B}\omega\) est la surface de \( \partial B\) et est donc non nulle.
        \item
            La fonction \( g\) est l'identité sur \( \partial B\). Nous avons donc \( \omega=g^*\omega\).
        \item
            Le lemme \ref{LemdwLGFG}.
        \item
            La forme \( \omega\) est de volume, par conséquent de degré maximum et \( d\omega=0\).
    \end{itemize}
\end{proof}

Un des points délicats est de se ramener au cas de fonctions \( C^{\infty}\). Pour la régularisation par convolution, voir \cite{AllardBrouwer}; pour celle utilisant le théorème de Weierstrass, voir \cite{KuttlerTopInAl}.
\begin{theorem}[Brouwer dans \( \eR^n\) version continue]\label{ThoRGjGdO}
    Soit \( B\) la boule fermée de centre \( 0\) et de rayon \( 1\) de \( \eR^n\) et \( f\colon B\to B\) une fonction continue. Alors \( f\) admet un point fixe.
\end{theorem}
\index{théorème!Brouwer}

\begin{proof}
    Nous commençons par définir une suite de fonctions
    \begin{equation}
        f_k(x)=\frac{ f(x) }{ 1+\frac{1}{ k } }.
    \end{equation}
    Nous avons \( \| f_k-f \|_{\infty}\leq \frac{1}{ 1+k }\) où la norme est la norme uniforme sur \( B\). Par le théorème de Weierstrass \ref{ThoWmAzSMF} il existe une suite de fonctions \(  C^{\infty}\) \( g_k\) telles que
    \begin{equation}
        \|  g_k-f_k\|_{\infty}\leq\frac{1}{ 1+k }.
    \end{equation}
    Vérifions que cette fonction \( g_k\) soit bien une fonction qui prend ses valeurs dans \( B\) :
    \begin{subequations}
        \begin{align}
            \| g_k(x) \|&\leq \| g_k(x)-f_k(x) \|+\| f_k(x) \|\\
            &\leq \frac{1}{ 1+k }+\frac{ \| f(x) \| }{ 1+\frac{1}{ k } }\\
            &\leq \frac{1}{ 1+k}+\frac{1}{ 1+\frac{1}{ k } }\\
            &=1.
        \end{align}
    \end{subequations}
    Par la version \(  C^{\infty}\) du théorème (proposition \ref{PropDRpYwv}), \( g_k\) admet un point fixe que l'on nomme \( x_k\).

    Étant donné que \( x_k\) est dans le compact \( B\), quitte à prendre une sous suite nous supposons que la suite \( (x_k)\) converge vers un élément \( x\in B\). Nous montrons maintenant que \( x\) est un point fixe de \( f\) :
    \begin{subequations}
        \begin{align}
            \| f(x)-x \|&=\| f(x)-g_k(x)+g_k(x)-x_k+x_k-x \|\\
            &\leq \| f(x)-g_k(x) \| +\underbrace{\| g_k(x)-x_k \|}_{=0}+\| x_k-x \|\\
            &\leq \frac{1}{ 1+k }+\| x_k-x \|.
        \end{align}
    \end{subequations}
    En prenant le limite \( k\to\infty\) le membre de droite tend vers zéro et nous obtenons \( f(x)=x\).
\end{proof}

%---------------------------------------------------------------------------------------------------------------------------
\subsection{Théorème de Schauder}
%---------------------------------------------------------------------------------------------------------------------------

Une conséquence du théorème de Brouwer est le théorème de Schauder qui est valide en dimension infinie.

\begin{theorem}[Théorème de Schauder\cite{LeDretSc}]\index{théorème!Schauder}       \label{ThovHJXIU}
    Soit \( E\), un espace vectoriel normé, \( K\) un convexe compact de \( E\) et \( f\colon K\to K\) une fonction continue. Alors \( f\) admet un point fixe.
\end{theorem}
\index{théorème!Schauder}
\index{point fixe!Schauder}

\begin{proof}
    Étant donné que \( f\colon K\to K\) est continue, elle y est uniformément continue. Si nous choisissons \( \epsilon\) alors il existe \( \delta>0\) tel que 
    \begin{equation}
        \| f(x)-f(y) \|\leq \epsilon
    \end{equation}
    dès que \( \| x-y \|\leq \delta\). La compacité de \( K\) permet de choisir un recouvrement fini par des ouverts de la forme
    \begin{equation}    \label{EqKNPUVR}
        K\subset \bigcup_{1\leq i\leq p}B(x_j,\delta)
    \end{equation}
    où \( \{ x_1,\ldots, x_p \}\subset K\). Nous considérons maintenant \( L=\Span\{ f(x_j)\tq 1\leq j\leq p \}\) et
    \begin{equation}
        K^*=K\cap L.
    \end{equation}
    Le fait que \( K\) et \( L\) soient convexes implique que \( K^*\) est convexe. L'ensemble \( K^*\) est également compact parce qu'il s'agit d'une partie fermée de \( K\) qui est compact (lemme \ref{LemnAeACf}). Notons en particulier que \( K^*\) est contenu dans un espace vectoriel de dimension finie, ce qui n'est pas le cas de \( K\).

    Nous allons à présent construire une sorte de partition de l'unité subordonnée au recouvrement \eqref{EqKNPUVR} sur \( K\) (voir le lemme \ref{LemGPmRGZ}). Nous commençons par définir
    \begin{equation}
        \psi_j(x)=\begin{cases}
            0    &   \text{si \( \| x-x_j \|\geq \delta\)}\\
            1-\frac{ \| x-x_j \| }{ \delta }    &    \text{sinon}.
        \end{cases}
    \end{equation}
    pour chaque \( 1\leq j\leq p\). Notons que \( \psi_j\) est une fonction positive, nulle en-dehors de \( B(x_j,\delta)\). En particulier la fonction suivante est bien définie :
    \begin{equation}
        \varphi_j(x)=\frac{ \psi_j(x) }{ \sum_{k=1}^p\psi_k(x) }
    \end{equation}
    et nous avons \( \sum_{j=1}^p\varphi_j(x)=1\). Les fonctions \( \varphi_j\) sont continues sur \( K\) et nous définissons finalement
    \begin{equation}
        g(x)=\sum_{j=1}^p\varphi_j(x)f(x_j).
    \end{equation}
    Pour chaque \( x\in K\), l'élément \( g(x)\) est une combinaison des éléments \( f(x_j)\in K^*\). Étant donné que \( K^*\) est convexe et que la somme des coefficients \( \varphi_j(x)\) vaut un, nous avons que \( g\) prend ses valeurs dans \( K^*\) par la proposition \ref{PropPoNpPz}.

    Nous considérons seulement la restriction \( g\colon K^*\to K^*\) qui est continue sur un compact contenu dans un espace vectoriel de dimension finie. Le théorème de Brouwer nous enseigne alors que \( g\) a un point fixe (proposition \ref{ThoRGjGdO}). Nous nommons \( y\) ce point fixe. Notons que \( y\) est fonction du \( \epsilon\) choisit au début de la construction, via le \( \delta\) qui avait conditionné la partition de l'unité.

    Nous avons
    \begin{subequations}        \label{EqoXuTzE}
        \begin{align}
            f(y)-y&=f(y)-g(y)\\
            &=\sum_{j=1}^p\varphi_j(y)f(y)-\sum_{j=1}^p\varphi_j(y)f(x_j)\\
            &=\sum_{j=1}^p\varphi(j)(y)\big( f(y)-f(x_j) \big).
        \end{align}
    \end{subequations}
    Par construction, \( \varphi_j(y)\neq 0\) seulement si \( \| y-x_j \|\leq \delta\) et par conséquent seulement si \( \| f(y)-f(x_j) \|\leq \epsilon\). D'autre par nous avons \( \varphi_j(y)\geq 0\); en prenant la norme de \eqref{EqoXuTzE} nous trouvons
    \begin{equation}
        \| f(y)-y \|\leq \sum_{j=1}^p\| \varphi_j(y)\big( f(y)-f(x_j) \big) \|\leq \sum_{j=1}^p\varphi_j(y)\epsilon=\epsilon.
    \end{equation}
    Nous nous souvenons maintenant que \( y\) était fonction de \( \epsilon\). Soit \( y_m\) le \( y\) qui correspond à \( \epsilon=2^{-m}\). Nous avons alors
    \begin{equation}
        \| f(y_m)-y_m \|\leq 2^{-m}.
    \end{equation}
    L'élément \( y_m\) est dans \( K^*\) qui est compact, donc quitte à choisir une sous suite nous pouvons supposer que \( y_m\) est une suite qui converge vers \( y^*\in K\)\footnote{Notons que même dans la sous suite nous avons \( \| f(y_m)-y_m \|\leq 2^{-m}\), avec le même «\( m\)» des deux côtés de l'inégalité.}. Nous avons les majorations
    \begin{equation}
        \| f(y^*)-y^* \|\leq \| f(y^*)-f(y_m) \|+\| f(y_m)-y_m \|+\| y_m-y^* \|.
    \end{equation}
    Si \( m\) est assez grand, les trois termes du membre de droite peuvent être rendus arbitrairement petits, d'où nous concluons que
    \begin{equation}
        f(y^*)=y^*
    \end{equation}
    et donc que \( f\) possède un point fixe.
\end{proof}


%--------------------------------------------------------------------------------------------------------------------------- 
\subsection{Théorème de Markov-Kakutani et mesure de Haar}
%---------------------------------------------------------------------------------------------------------------------------

\begin{definition}
    Soit \( G\) un groupe topologique. Une \defe{mesure de Haar}{mesure!de Haar} sur \( G\) est une mesure \( \mu\) telle que 
    \begin{enumerate}
        \item
            \( \mu(gA)=\mu(A)\) pour tout mesurable \( A\) et tout \( g\in G\),
        \item
            \( \mu(K)<\infty\) pour tout compact \( K\subset G\).
    \end{enumerate}
    Si de plus le groupe \( G\) lui-même est compact nous demandons que la mesure soit normalisée : \( \mu(G)=1\).
\end{definition}

Le théorème suivant nous donne l'existence d'une mesure de Haar sur un groupe compact.
\begin{theorem}[Markov-Katutani\cite{BeaakPtFix}]\index{théorème!Markov-Takutani}   \label{ThoeJCdMP}
    Soit \( E\) un espace vectoriel normé et \( L\), une partie non vide, convexe, fermée et bornée de \( E'\). Soit \( T\colon L\to L\) une application continue. Alors \( T\) a un point fixe.
\end{theorem}

\begin{proof}
    Nous considérons un point \( x_0\in L\) et la suite
    \begin{equation}
        x_n=\frac{1}{ n+1 }\sum_{i=0}^n T^ix_0.
    \end{equation}
    La somme des coefficients devant les \( T^i(x_0)\) étant \( 1\), la convexité de \( L\) montre que \( x_n\in L\). Nous considérons l'ensemble
    \begin{equation}
        C=\bigcap_{n\in \eN}\overline{ \{ x_m\tq m\geq n \} }.
    \end{equation}
    Le lemme \ref{LemooynkH} indique que \( C\) n'est pas vide, et de plus il existe une sous suite de \( (x_n)\) qui converge vers un élément \( x\in C\). Nous avons
    \begin{equation}
        \lim_{n\to \infty} x_{\sigma(n)}(v)=x(v)
    \end{equation}
    pour tout \( v\in E\). Montrons que \( x\) est un point fixe de \( T\). Nous avons
    \begin{subequations}
        \begin{align}
            \| (Tx_{\sigma(k)}-x_{\sigma(k)})v \|&=\Big\| T\frac{1}{ 1+\sigma(k) }\sum_{i=0}^{\sigma(k)}T^ix_0(v)-\frac{1}{ 1+\sigma(k) }\sum_{i=0}^{\sigma(k)}T^ix_0(v) \Big\|\\
            &=\Big\| \frac{1}{ 1+\sigma(k) }\sum_{i=0}^{\sigma(k)}T^{i+1}x_0(v)-T^ix_0(v) \Big\|\\
            &=\frac{1}{ 1+\sigma(k) }\big\| T^{\sigma(k)+1}x_0(v)-x_0(v) \big\|\\
            &\leq\frac{ 2M }{ \sigma(k)+1 }
        \end{align}
    \end{subequations}
    où \( M=\sum_{y\in L}\| y(v) \|<\infty\) parce que \( L\) est borné. En prenant \( k\to\infty\) nous trouvons
    \begin{equation}
        \lim_{k\to \infty} \big( Tx_{\sigma(k)}-x_{\sigma(k)} \big)v=0,
    \end{equation}
    ce qui signifie que \( Tx=x\) parce que \( T\) est continue.
\end{proof}

Le théorème suivant est une conséquence du théorème de Markov-Katutani.
\begin{theorem} \label{ThoBZBooOTxqcI}
    Si \( G\) est un groupe topologique compact possédant une base dénombrable de topologie alors \( G\) accepte une unique mesure de Haar normalisée. De plus elle est unimodulaire :
    \begin{equation}
        \mu(Ag)=\mu(gA)=\mu(A)
    \end{equation}
    pour tout mesurables \( A\subset G\) et tout élément \( g\in G\).
\end{theorem}
\index{mesure!de Haar}

