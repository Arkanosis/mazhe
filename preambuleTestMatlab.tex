\documentclass[a4paper]{article}
\usepackage[utf8]{inputenc}
\usepackage[french]{babel}

\usepackage[compat2]{geometry}
\usepackage{amsmath,amssymb,amsthm}
\usepackage{cases}


%%%%%%%%%%%%%%%%%%%%%%%%%%%%%%%%%%%%%%%%%%%%%%%%%%%%%%%%%%%%%%%%%%%%%
% Choix draft / pas draft
%%%%%%%%%%%%%%%%%%%%%%%%%%%%%%%%%%%%%%%%%%%%%%%%%%%%%%%%%%%%%%%%%%%%%

\usepackage{ifthen}

\newboolean{draft}
\setboolean{draft}{false}


%%%%%%%%%%%%%%%%%%%%%%%%%%%%%%%%%%%%%%%%%%%%%%%%%%%%%%%%%%%%%%%%%%%%%
% En-tête
%%%%%%%%%%%%%%%%%%%%%%%%%%%%%%%%%%%%%%%%%%%%%%%%%%%%%%%%%%%%%%%%%%%%%

\newcommand{\dateexam}{Novembre 2009}

\newcommand{\entete}{%
  \noindent
  \begin{center}
  \begin{tabular*}{\textwidth}{|c|l|}
    \hline
    \begin{minipage}{0.5999\textwidth}
      \medskip
      \centering
      \textbf{BIR1200 Mathématiques générales} \\
      Test Matlab \\
      \dateexam
      \medskip
    \end{minipage}
  & 
    \begin{minipage}{0.35\textwidth}
      \medskip
      \textsc{Nom :} \\
      \textsc{Prénom :} \\
      Année et groupe :
      \medskip
    \end{minipage}
  \\
  \hline
  \end{tabular*}
  \end{center}
  
  \bigskip
  
  \noindent
  \emph{Résolvez les problèmes dans Matlab et recopiez ensuite vos commandes en dessous de chaque énoncé. Si vos commandes se trouvent dans un fichier, notez clairement son nom \underline{en souligné} avant les lignes de code correspondantes.}
  
  \bigskip
}


%%%%%%%%%%%%%%%%%%%%%%%%%%%%%%%%%%%%%%%%%%%%%%%%%%%%%%%%%%%%%%%%%%%%%
% Pour commencer un nouveau questionnaire d'examen
%%%%%%%%%%%%%%%%%%%%%%%%%%%%%%%%%%%%%%%%%%%%%%%%%%%%%%%%%%%%%%%%%%%%%

\newcommand{\nouvelexamen}[1]{%
  \ifthenelse{\boolean{draft}}{%
    \bigskip\hrule\hrule\bigskip \noindent #1 \bigskip \hrule \bigskip%
  }{%
    \setcounter{page}{1}%
  }
  \setcounter{numexo}{0}%
  \renewcommand{\dateexam}{#1}%
}


%%%%%%%%%%%%%%%%%%%%%%%%%%%%%%%%%%%%%%%%%%%%%%%%%%%%%%%%%%%%%%%%%%%%%
% Pour compatibilité avec l'incroyable système de Laurent
%%%%%%%%%%%%%%%%%%%%%%%%%%%%%%%%%%%%%%%%%%%%%%%%%%%%%%%%%%%%%%%%%%%%%

\newcommand{\Exo}[1]{%
  \ifthenelse{\boolean{draft}}{%
    \noindent (Fichier \emph{exo#1.tex}) \input{exo#1}%
  }{%
    \input{exo#1}%
  }%
}
\newcommand{\corrref}[1]{}

\newcounter{numexo}
\setcounter{numexo}{0}
\newenvironment{exercice}{
    \stepcounter{numexo}
    \ifthenelse{\boolean{draft}}{}{\entete}
    \noindent\textbf{Question~\the\value{numexo}}
    \bigskip
  }{
    \ifthenelse{\boolean{draft}}{\bigskip}{\newpage}
  }

\newcommand{\eR}{\mathbb{R}}
\newcommand{\eZ}{\mathbb{Z}}
\newcommand{\eQ}{\mathbb{Q}}
\newcommand{\eC}{\mathbb{C}}
\newcommand{\eN}{\mathbb{N}}

\newcommand{\per}{/}
\newcommand{\meter}{m}
\newcommand{\second}{s}
\newcommand{\unit}[2]{\ensuremath #1 \, \text{#2}}

