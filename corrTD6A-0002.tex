\begin{corrige}{TD6A-0002}

    \begin{enumerate}
        \item
            Si \( x_1(t)=\sin(t)\) nous avons
            \begin{equation}
                \frac{ dx_1 }{ dt }=\cos(t),
            \end{equation}
            et 
            \begin{equation}
                \sqrt{1-x_1^2(t)}=\sqrt{1-\sin^2(t)}=\sqrt{\cos^2(t)}=\cos(t).
            \end{equation}
            Attention: en général \( \sqrt{\cos^2(t)}=| \cos(t) |\). Ici nous pouvons enlever les valeurs absolues parce que \( \cos(t)>0\) lorsque \( t\in\mathopen] -\pi/2 , \pi/2 \mathclose[\).

        \item

            La vérification est immédiate parce que la dérivée des solutions constante est nulle.

        \item

            Voir la figure figure \ref{LabelFigSolsEqDiffSin}.
            \newcommand{\CaptionFigSolsEqDiffSin}{Trois de solutions à l'équation différentielle de l'exercice \ref{exoTD6A-0002}.}
            \input{Fig_SolsEqDiffSin.pstricks}

        \item
            
            Non parce que l'équation différentielle \( x'(t)=\sqrt{1-x(t)^2}\) dit que \( x'\) doit toujours être positive. Mais si nous prolongeons \( x_1(t)=\sin(t)\) en dehors de l'intervalle \( \mathopen] -\pi/2 , \pi/2 \mathclose[\), la dérivée devient négative.

            Il est cependant possible de la prolonger sur \( \mathopen[ -\pi/2 , \pi/2 \mathclose]\).

        \item

            Étant donné que l'équation que doit satisfaire \( x(t)\) (pour tout \( t\)) contient \( \sqrt{1-x(t)^2}\), le nombre \( x(t)\) doit rester entre \( -1\) et \( 1\) pour tout \( t\).

        \item

            La méthode de séparation des variables consiste à mettre tous les \( x\) d'un côté (y compris \( dx\)) et tous les \( t\) de l'autre, y compris \( dt\). Nous trouvons
            \begin{equation}
                \frac{ dx }{ \sqrt{1-x^2} }=dt.
            \end{equation}
            Nous prenons l'intégrale des deux côtés en n'oubliant pas la constante d'intégration :

            \begin{verbatim}
----------------------------------------------------------------------
| Sage Version 4.7.1, Release Date: 2011-08-11                       |
| Type notebook() for the GUI, and license() for information.        |
----------------------------------------------------------------------
sage: f(x)=1/sqrt(1-x**2)
sage: f.integrate(x)
x |--> arcsin(x)
            \end{verbatim}
            Notez que \href{http://sagemath.org}{Sage} ne vous donne constante d'intégration\footnote{Vous pouvez voir votre cours de math comme un cours d'anticipation des fautes que vous commettriez si vous vous fiiez trop facilement à votre calculatrice ou votre ordinateur.} ni domaines\footnote{Il ne peut pas savoir quel est le contexte dans lequel vous êtes en train de calculer !} (nous y reviendrons). Nous avons :
            \begin{equation}
                t=\arcsin(x)+C.
            \end{equation}
            Par conséquent,
            \begin{equation}
                x(t)=\sin(t+C).
            \end{equation}
            Ici nous avons changé \( C\) en \( -C\) pour la commodité d'écriture.

            Nous demandons d'avoir \( x(0)=\alpha\), c'est à dire
            \begin{equation}
                x(0)=\sin(C)=\alpha,
            \end{equation}
            ou encore \( C=\arcsin(\alpha)\). La solution de l'équation que nous cherchons est donc
            \begin{equation}
                x(t)=\sin\big(t+\arcsin(\alpha)\big).
            \end{equation}
            
            Maintenant nous devons réfléchir sur les domaines\ldots Regardons donc encore une fois la condition d'existence \( x'(t)\geq 0\) :
            \begin{equation}
                x'(t)=\cos(t+C)\geq 0
            \end{equation}
            implique \( t+C\in\mathopen[ -\frac{ \pi }{2} , \frac{ \pi }{2} \mathclose]\) et par conséquent
            \begin{equation}
                t\in\mathopen[ -\frac{ \pi }{2}-\arcsin(\alpha) , \frac{ \pi }{ 2 }-\arcsin(\alpha) \mathclose].
            \end{equation}
            Notons que le domaine dépend de \( \alpha\) ! Nous prenons aussi la convention que \( \arcsin(\alpha)\) est le nombre \( z\) entre \( -\frac{ \pi }{2}\) et \( \frac{ \pi }{2}\) tel que \( \sin(z)=\alpha\).
            

        \item

            Quelque solutions sont dessinées à la figure \ref{LabelFigSolsSinpA}.
            \newcommand{\CaptionFigSolsSinpA}{Des solutions pour l'équation différentielle de l'exercice \ref{exoTD6A-0002}.}
            \input{Fig_SolsSinpA.pstricks}

    \end{enumerate}
  
\end{corrige}
